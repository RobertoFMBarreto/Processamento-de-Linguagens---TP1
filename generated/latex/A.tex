\documentclass{article}
\usepackage[portuguese]{babel}
\title{A}
\begin{document}
\section{Achafundar}
\begin{itemize}
\item {Grp. gram.:v. t.}
\end{itemize}
\begin{itemize}
\item {Utilização:Pop.}
\end{itemize}
Enterrar no lodo; meter no fundo da água.
\section{Abacamartado}
\begin{itemize}
\item {Grp. gram.:adj.}
\end{itemize}
Parecido com um bacamarte:«\textunderscore uma cravina abacamartada.\textunderscore »(De um testamento de 1693)
\section{Abafo}
\begin{itemize}
\item {Grp. gram.:m.}
\end{itemize}
\begin{itemize}
\item {Proveniência:(De \textunderscore abafar\textunderscore )}
\end{itemize}
Roupa ou pelles, que servem de agasalho.
\section{Abaloso}
\begin{itemize}
\item {Grp. gram.:adj.}
\end{itemize}
\begin{itemize}
\item {Utilização:Bras. de Minas}
\end{itemize}
\begin{itemize}
\item {Proveniência:(De \textunderscore abalo\textunderscore )}
\end{itemize}
Pouco cômmodo, (falando-se do andar do cavallo).
\section{Abarregamento}
\begin{itemize}
\item {Grp. gram.:m.}
\end{itemize}
Acto ou effeito de abarregar.
\section{Abarticular}
\begin{itemize}
\item {Grp. gram.:adj.}
\end{itemize}
\begin{itemize}
\item {Utilização:Med.}
\end{itemize}
\begin{itemize}
\item {Proveniência:(Do lat. \textunderscore ab\textunderscore  + \textunderscore articulus\textunderscore )}
\end{itemize}
Diz-se do rheumatismo, que ataca os órgãos, os músculos, os nervos, sem atacar as articulações.
\section{Abegão}
\begin{itemize}
\item {Grp. gram.:m.}
\end{itemize}
\begin{itemize}
\item {Utilização:Prov.}
\end{itemize}
O mesmo que \textunderscore besoiro\textunderscore .
(Cast. \textunderscore abejón\textunderscore )
\section{Abelhina}
\begin{itemize}
\item {Grp. gram.:f.}
\end{itemize}
Espécie de orchídeas, o mesmo que \textunderscore abelha-flôr\textunderscore .
\section{Abelhuar-se}
\begin{itemize}
\item {Grp. gram.:v. p.}
\end{itemize}
\begin{itemize}
\item {Utilização:Des.}
\end{itemize}
Apressar-se. Cf. Costa e Sá, \textunderscore Diccion.\textunderscore  (Talvez escrita incorrecta, por \textunderscore abelhoar-se\textunderscore , de \textunderscore abelhão\textunderscore )
\section{Abesoiro}
\begin{itemize}
\item {Grp. gram.:m.}
\end{itemize}
O mesmo que \textunderscore besoiro\textunderscore .
\section{Abesouro}
\begin{itemize}
\item {Grp. gram.:m.}
\end{itemize}
O mesmo que \textunderscore besouro\textunderscore .
\section{Abestunto}
\begin{itemize}
\item {Grp. gram.:m.}
\end{itemize}
\begin{itemize}
\item {Utilização:Prov.}
\end{itemize}
\begin{itemize}
\item {Utilização:trasm.}
\end{itemize}
O mesmo que \textunderscore bestunto\textunderscore . (Colhido em Villa-Real)
\section{Abetumador}
\begin{itemize}
\item {Grp. gram.:m.}
\end{itemize}
O que abetuma.
\section{Abodegação}
\begin{itemize}
\item {Grp. gram.:f.}
\end{itemize}
Acto ou effeito de abodegar.
\section{Aboiadinho}
\begin{itemize}
\item {fónica:bói}
\end{itemize}
\begin{itemize}
\item {Grp. gram.:adj.}
\end{itemize}
\begin{itemize}
\item {Utilização:T. dos pescadores de Viana}
\end{itemize}
\begin{itemize}
\item {Proveniência:(De \textunderscore aboiado\textunderscore )}
\end{itemize}
Diz-se do peixe morto, que os vapores de pesca abandonaram.
\section{Aboiado}
\begin{itemize}
\item {fónica:bói}
\end{itemize}
\begin{itemize}
\item {Grp. gram.:adj.}
\end{itemize}
\begin{itemize}
\item {Utilização:Pesc.}
\end{itemize}
\begin{itemize}
\item {Proveniência:(De \textunderscore aboiar\textunderscore )}
\end{itemize}
Que anda á tona da água.
\section{Abomba}
\begin{itemize}
\item {Grp. gram.:f.}
\end{itemize}
\begin{itemize}
\item {Utilização:Prov.}
\end{itemize}
\begin{itemize}
\item {Utilização:alent.}
\end{itemize}
O mesmo que \textunderscore bomba\textunderscore ^1. Cf. \textunderscore Rev. Lus.\textunderscore , II, 29.
\section{Aboucar}
\begin{itemize}
\item {Grp. gram.:v. t.}
\end{itemize}
\begin{itemize}
\item {Utilização:Prov.}
\end{itemize}
\begin{itemize}
\item {Utilização:trasm.}
\end{itemize}
O mesmo que \textunderscore esmocar\textunderscore .
Matar. (Colhido em Villa-Real)
\section{Abrachiocephalia}
\begin{itemize}
\item {fónica:qui}
\end{itemize}
\begin{itemize}
\item {Grp. gram.:f.}
\end{itemize}
\begin{itemize}
\item {Utilização:Terat.}
\end{itemize}
Estado de abrachiocéphalo.
\section{Abrachiocéphalo}
\begin{itemize}
\item {fónica:qui}
\end{itemize}
\begin{itemize}
\item {Grp. gram.:m.}
\end{itemize}
\begin{itemize}
\item {Utilização:Terat.}
\end{itemize}
\begin{itemize}
\item {Proveniência:(Do gr. \textunderscore a\textunderscore  priv. + \textunderscore brakhion\textunderscore  + \textunderscore kephale\textunderscore )}
\end{itemize}
Monstro, sem braços nem cabeça.
\section{Abraquiocefalia}
\begin{itemize}
\item {Grp. gram.:f.}
\end{itemize}
\begin{itemize}
\item {Utilização:Terat.}
\end{itemize}
Estado de abraquiocéfalo.
\section{Abraquiocéfalo}
\begin{itemize}
\item {Grp. gram.:m.}
\end{itemize}
\begin{itemize}
\item {Utilização:Terat.}
\end{itemize}
\begin{itemize}
\item {Proveniência:(Do gr. \textunderscore a\textunderscore  priv. + \textunderscore brakhion\textunderscore  + \textunderscore kephale\textunderscore )}
\end{itemize}
Monstro, sem braços nem cabeça.
\section{Abrasoar}
\begin{itemize}
\item {Grp. gram.:v. t.}
\end{itemize}
O mesmo que \textunderscore abrasonar\textunderscore .
\section{Abronceiro}
\begin{itemize}
\item {Grp. gram.:m.}
\end{itemize}
\begin{itemize}
\item {Utilização:Prov.}
\end{itemize}
\begin{itemize}
\item {Utilização:trasm.}
\end{itemize}
O mesmo que \textunderscore espinheiro\textunderscore .
\section{Abrótiga}
\begin{itemize}
\item {Grp. gram.:f.}
\end{itemize}
O mesmo que \textunderscore abrótega\textunderscore . Cf. \textunderscore Rev. Lus.\textunderscore , XI, 287.
\section{Abrunhar}
\begin{itemize}
\item {Grp. gram.:v. t.}
\end{itemize}
\begin{itemize}
\item {Utilização:T. da Murça}
\end{itemize}
O mesmo que \textunderscore acabrunhar\textunderscore .
\section{Abundidade}
\begin{itemize}
\item {Grp. gram.:f.}
\end{itemize}
O mesmo que \textunderscore abundeza\textunderscore . Cf. Camillo, \textunderscore Curso de Liter.\textunderscore , 105.
\section{Abúlico}
\begin{itemize}
\item {Grp. gram.:adj.}
\end{itemize}
Relativo á abulia.
Que padece abulia.
\section{Abusamento}
\begin{itemize}
\item {Grp. gram.:m.}
\end{itemize}
\begin{itemize}
\item {Utilização:Prov.}
\end{itemize}
\begin{itemize}
\item {Utilização:alent.}
\end{itemize}
O mesmo que \textunderscore abuso\textunderscore .
\section{Acadar}
\begin{itemize}
\item {Grp. gram.:v. t.}
\end{itemize}
\begin{itemize}
\item {Utilização:T. de Pare -de-Coira}
\end{itemize}
\begin{itemize}
\item {Utilização:des.}
\end{itemize}
Receber nas mãos ou no regaço.
(Contr. de \textunderscore arrecadar\textunderscore )
\section{Acafanhar}
\begin{itemize}
\item {Grp. gram.:v. t.}
\end{itemize}
\begin{itemize}
\item {Utilização:Prov.}
\end{itemize}
\begin{itemize}
\item {Utilização:minh.}
\end{itemize}
O mesmo que \textunderscore estragar\textunderscore . (Colhido em Paredes-de-Coira)
\section{Acaparrar}
\begin{itemize}
\item {Grp. gram.:v. t.}
\end{itemize}
\begin{itemize}
\item {Utilização:Prov.}
\end{itemize}
\begin{itemize}
\item {Utilização:trasm.}
\end{itemize}
O mesmo que \textunderscore mascarar\textunderscore . (Colhido em Villa-Real)
\section{Acapelar}
\begin{itemize}
\item {Grp. gram.:v. t.}
\end{itemize}
\begin{itemize}
\item {Utilização:Des.}
\end{itemize}
Meter debaixo de água; submergir. Cf. Costa e Sá, \textunderscore Diccion.\textunderscore 
(Da mesma or. de \textunderscore acapellar\textunderscore ?)
\section{Acapnia}
\begin{itemize}
\item {Grp. gram.:f.}
\end{itemize}
\begin{itemize}
\item {Utilização:Med.}
\end{itemize}
\begin{itemize}
\item {Proveniência:(Do gr. \textunderscore a\textunderscore  priv. + \textunderscore kapnos\textunderscore )}
\end{itemize}
Deminuição do ácido carbónico, contido no sangue.
\section{Açarçalhar}
\begin{itemize}
\item {Grp. gram.:v. t.}
\end{itemize}
\begin{itemize}
\item {Utilização:Prov.}
\end{itemize}
\begin{itemize}
\item {Utilização:minh.}
\end{itemize}
O mesmo que \textunderscore gaguejar\textunderscore . (Colhido em Paredes-de-Coira)
\section{Acardíaco}
\begin{itemize}
\item {Grp. gram.:adj.}
\end{itemize}
\begin{itemize}
\item {Utilização:Terat.}
\end{itemize}
\begin{itemize}
\item {Proveniência:(Do gr. \textunderscore a\textunderscore  priv. + \textunderscore kardia\textunderscore )}
\end{itemize}
Diz-se do monstro, que não tem coração.
\section{Acarditar}
\begin{itemize}
\item {Grp. gram.:v. t.}
\end{itemize}
\begin{itemize}
\item {Utilização:Pop.}
\end{itemize}
O mesmo que \textunderscore acreditar\textunderscore .
\section{Acarlinga}
\begin{itemize}
\item {Grp. gram.:f.}
\end{itemize}
\begin{itemize}
\item {Utilização:Náut.}
\end{itemize}
\begin{itemize}
\item {Utilização:Des.}
\end{itemize}
O mesmo que \textunderscore carlinga\textunderscore .
\section{Abandonatário}
\begin{itemize}
\item {Grp. gram.:m.}
\end{itemize}
\begin{itemize}
\item {Utilização:Neol.}
\end{itemize}
\begin{itemize}
\item {Proveniência:(De \textunderscore abandonar\textunderscore )}
\end{itemize}
Indivíduo, que adquiriu direito ou coisa abandonada por outrem.
\section{Abotoado}
\begin{itemize}
\item {Grp. gram.:adj.}
\end{itemize}
Fechado com botões: \textunderscore casaco abotoado\textunderscore .
Fechado, cerrado: \textunderscore lábios abotoados\textunderscore .
Que está ainda em botão: \textunderscore flôr abotoada\textunderscore .
\section{Abricoqueiro}
\begin{itemize}
\item {Grp. gram.:m.}
\end{itemize}
O mesmo que \textunderscore albricoqueiro\textunderscore .
\section{Abutre-fusco}
\begin{itemize}
\item {Grp. gram.:m.}
\end{itemize}
Ave, o mesmo que \textunderscore pica-osso\textunderscore . Cf. P. Moraes, \textunderscore Zool. Elem.\textunderscore , 352.
\section{Acácia-bastarda}
\begin{itemize}
\item {Grp. gram.:f.}
\end{itemize}
Espécie de acácia, (\textunderscore robinia pseudo-acacia\textunderscore ).
\section{Acácia-dos-alemães}
\begin{itemize}
\item {Grp. gram.:f.}
\end{itemize}
Espécie de abrunheiro bravo, (\textunderscore prunus spinosa\textunderscore , Lin.).
\section{Acaramular}
\begin{itemize}
\item {Grp. gram.:v. t.}
\end{itemize}
\begin{itemize}
\item {Utilização:Prov.}
\end{itemize}
\begin{itemize}
\item {Grp. gram.:V. i.}
\end{itemize}
\begin{itemize}
\item {Proveniência:(De \textunderscore caramulo\textunderscore )}
\end{itemize}
Amontoar.
Accumular-se; amontoar-se. (Colhido na Bairrada)
\section{Acidificável}
\begin{itemize}
\item {Grp. gram.:adj.}
\end{itemize}
Que se póde acidificar.
\section{Acochichar}
\begin{itemize}
\item {Grp. gram.:v. t.}
\end{itemize}
\begin{itemize}
\item {Utilização:Prov.}
\end{itemize}
Agachar; acocorar, encolher: \textunderscore ficou para alli, acochichado a um canto\textunderscore . (Colhido na Bairrada)
\section{Açofaifa}
\begin{itemize}
\item {Grp. gram.:f.}
\end{itemize}
(Outra fórma de \textunderscore açofeifa\textunderscore ). Cf. B. Pereira, \textunderscore Prosodia\textunderscore , vb. \textunderscore jujuba\textunderscore .
\section{Acrolínio}
\begin{itemize}
\item {Grp. gram.:m.}
\end{itemize}
Gênero de plantas dos jardins.
\section{Actinólitho}
\begin{itemize}
\item {Grp. gram.:m.}
\end{itemize}
\begin{itemize}
\item {Utilização:Miner.}
\end{itemize}
\begin{itemize}
\item {Proveniência:(Do gr. \textunderscore aktin\textunderscore  + \textunderscore lithos\textunderscore )}
\end{itemize}
O mesmo que \textunderscore actinoto\textunderscore .
\section{Actinólito}
\begin{itemize}
\item {Grp. gram.:m.}
\end{itemize}
\begin{itemize}
\item {Utilização:Miner.}
\end{itemize}
\begin{itemize}
\item {Proveniência:(Do gr. \textunderscore aktin\textunderscore  + \textunderscore lithos\textunderscore )}
\end{itemize}
O mesmo que \textunderscore actinoto\textunderscore .
\section{Actinoto}
\begin{itemize}
\item {Grp. gram.:m.}
\end{itemize}
\begin{itemize}
\item {Utilização:Miner.}
\end{itemize}
\begin{itemize}
\item {Proveniência:(Gr. \textunderscore aktinotos\textunderscore )}
\end{itemize}
Silicato de cálcio, magnésio e ferro.
\section{Acarofobia}
\begin{itemize}
\item {Grp. gram.:f.}
\end{itemize}
\begin{itemize}
\item {Utilização:Med.}
\end{itemize}
\begin{itemize}
\item {Proveniência:(Do gr. \textunderscore akari\textunderscore  + \textunderscore phobein\textunderscore )}
\end{itemize}
Terror mórbido da sarna.
\section{Acarophobia}
\begin{itemize}
\item {Grp. gram.:f.}
\end{itemize}
\begin{itemize}
\item {Utilização:Med.}
\end{itemize}
\begin{itemize}
\item {Proveniência:(Do gr. \textunderscore akari\textunderscore  + \textunderscore phobein\textunderscore )}
\end{itemize}
Terror mórbido da sarna.
\section{Acarradoiro}
\begin{itemize}
\item {Grp. gram.:m.}
\end{itemize}
\begin{itemize}
\item {Utilização:Prov.}
\end{itemize}
Lugar, onde o gado passa as horas do calor, ordinariamente nas lapas. Cf. \textunderscore Rev. Lus.\textunderscore , XI, 146.
\section{Acarradouro}
\begin{itemize}
\item {Grp. gram.:m.}
\end{itemize}
\begin{itemize}
\item {Utilização:Prov.}
\end{itemize}
Lugar, onde o gado passa as horas do calor, ordinariamente nas lapas. Cf. \textunderscore Rev. Lus.\textunderscore , XI, 146.
\section{Acarreja}
\begin{itemize}
\item {Grp. gram.:f.}
\end{itemize}
\begin{itemize}
\item {Utilização:Prov.}
\end{itemize}
\begin{itemize}
\item {Utilização:trasm.}
\end{itemize}
Carrada (de cereaes).
(Cp. \textunderscore carrejar\textunderscore )
\section{Acarvalhar}
\begin{itemize}
\item {Grp. gram.:v. i.}
\end{itemize}
\begin{itemize}
\item {Utilização:T. de Pare -de-Coira}
\end{itemize}
\begin{itemize}
\item {Utilização:des.}
\end{itemize}
Cortar, pela primeira vez, os galhos do carvalho.
\section{Acatarrado}
\begin{itemize}
\item {Grp. gram.:adj.}
\end{itemize}
O mesmo que [[encatarroado|encatarroar-se]].
\section{Acautelamento}
\begin{itemize}
\item {Grp. gram.:m.}
\end{itemize}
Acto de acautelar.
Cautela.
\section{Accrecentar}
\textunderscore v. t.\textunderscore  (e der.)
O mesmo que \textunderscore acrescentar\textunderscore , etc.
\section{Accrecer}
\textunderscore v. i.\textunderscore  (e der.)
O mesmo que \textunderscore accrescer\textunderscore , etc.
\section{Accubitor}
\begin{itemize}
\item {Grp. gram.:m.}
\end{itemize}
\begin{itemize}
\item {Utilização:Ant.}
\end{itemize}
\begin{itemize}
\item {Proveniência:(Lat. \textunderscore accubitor\textunderscore )}
\end{itemize}
O mesmo que \textunderscore commensal\textunderscore .
Criado, que dormia perto do leito dos imperadores de Constantinopla. Cf. C. e Sá, \textunderscore Diccion.\textunderscore 
\section{Acefalomia}
\begin{itemize}
\item {Grp. gram.:f.}
\end{itemize}
\begin{itemize}
\item {Utilização:Terat.}
\end{itemize}
\begin{itemize}
\item {Proveniência:(Do gr. \textunderscore a\textunderscore  priv. + \textunderscore kephale\textunderscore  + \textunderscore alasthai\textunderscore )}
\end{itemize}
Estado de um feto, cuja cabeça é monstruosa.
\section{Acefalorraquia}
\begin{itemize}
\item {Grp. gram.:f.}
\end{itemize}
\begin{itemize}
\item {Proveniência:(Do gr. \textunderscore a\textunderscore  priv. + \textunderscore kephale\textunderscore  + \textunderscore rakhis\textunderscore )}
\end{itemize}
Monstruosidade, caracterizada pela ausência de cabeça e de coluna vertebral.
\section{Aceifar}
\begin{itemize}
\item {Grp. gram.:v. t.}
\end{itemize}
O mesmo que \textunderscore ceifar\textunderscore .
\section{Acenheiro}
\begin{itemize}
\item {Grp. gram.:m.}
\end{itemize}
\begin{itemize}
\item {Utilização:Des.}
\end{itemize}
Dono de acenha.
\section{Acephalomia}
\begin{itemize}
\item {Grp. gram.:f.}
\end{itemize}
\begin{itemize}
\item {Utilização:Terat.}
\end{itemize}
\begin{itemize}
\item {Proveniência:(Do gr. \textunderscore a\textunderscore  priv. + \textunderscore kephale\textunderscore  + \textunderscore alasthai\textunderscore )}
\end{itemize}
Estado de um feto, cuja cabeça é monstruosa.
\section{Acephalorachia}
\begin{itemize}
\item {fónica:raqui}
\end{itemize}
\begin{itemize}
\item {Grp. gram.:f.}
\end{itemize}
\begin{itemize}
\item {Proveniência:(Do gr. \textunderscore a\textunderscore  priv. + \textunderscore kephale\textunderscore  + \textunderscore rakhis\textunderscore )}
\end{itemize}
Monstruosidade, caracterizada pela ausência de cabeça e de columna vertebral.
\section{Acerário}
\textunderscore m. Neol.\textunderscore  (?)
Officina, em que se prepara aço.
\section{Aceto}
\begin{itemize}
\item {Grp. gram.:m.}
\end{itemize}
O mesmo que \textunderscore vinagre\textunderscore ^1, em Alquimia. Cf. C. e Sá, \textunderscore Diccion.\textunderscore 
\section{Acetosa}
\begin{itemize}
\item {Grp. gram.:f.}
\end{itemize}
\begin{itemize}
\item {Utilização:Des.}
\end{itemize}
Planta, o mesmo que \textunderscore labaça\textunderscore .
\section{Achegado}
\begin{itemize}
\item {Grp. gram.:m.}
\end{itemize}
\begin{itemize}
\item {Utilização:Ant.}
\end{itemize}
Amigo.
Partidário. Cf. \textunderscore Rev. Lus.\textunderscore , XVI, 1.
\section{Acheira}
\begin{itemize}
\item {Grp. gram.:f.}
\end{itemize}
\begin{itemize}
\item {Utilização:Prov.}
\end{itemize}
O mesmo que \textunderscore acha\textunderscore ^1. Cf. \textunderscore Rev. Lus.\textunderscore , XIII, 89.
\section{Achiria}
\begin{itemize}
\item {fónica:qui}
\end{itemize}
\begin{itemize}
\item {Grp. gram.:f.}
\end{itemize}
\begin{itemize}
\item {Utilização:Terat.}
\end{itemize}
\begin{itemize}
\item {Proveniência:(Do gr. \textunderscore a\textunderscore  priv. + \textunderscore kheir\textunderscore )}
\end{itemize}
Ausência congénita das mãos ou de uma só.
\section{Achumaço}
\begin{itemize}
\item {Grp. gram.:m.}
\end{itemize}
\begin{itemize}
\item {Utilização:Prov.}
\end{itemize}
\begin{itemize}
\item {Utilização:alent.}
\end{itemize}
O mesmo que \textunderscore chumaço\textunderscore .
\section{Acubitor}
\begin{itemize}
\item {Grp. gram.:m.}
\end{itemize}
\begin{itemize}
\item {Utilização:Ant.}
\end{itemize}
\begin{itemize}
\item {Proveniência:(Lat. \textunderscore accubitor\textunderscore )}
\end{itemize}
O mesmo que \textunderscore commensal\textunderscore .
Criado, que dormia perto do leito dos imperadores de Constantinopla. Cf. C. e Sá, \textunderscore Diccion.\textunderscore 
\section{Aquiria}
\begin{itemize}
\item {Grp. gram.:f.}
\end{itemize}
\begin{itemize}
\item {Utilização:Terat.}
\end{itemize}
\begin{itemize}
\item {Proveniência:(Do gr. \textunderscore a\textunderscore  priv. + \textunderscore kheir\textunderscore )}
\end{itemize}
Ausência congénita das mãos ou de uma só.
\section{Acidose}
\begin{itemize}
\item {Grp. gram.:f.}
\end{itemize}
\begin{itemize}
\item {Utilização:Med.}
\end{itemize}
\begin{itemize}
\item {Proveniência:(De \textunderscore ácido\textunderscore )}
\end{itemize}
Impregnação ácida dos tecidos, que se observa especialmente nos diabéticos em imminência de coma.
\section{Acianoblepsia}
\begin{itemize}
\item {Grp. gram.:f.}
\end{itemize}
\begin{itemize}
\item {Utilização:Med.}
\end{itemize}
\begin{itemize}
\item {Proveniência:(Do gr. \textunderscore a\textunderscore  priv. + \textunderscore kuanos\textunderscore  + \textunderscore blepsis\textunderscore )}
\end{itemize}
Insensibilidade visual para a côr azul.
\section{Acincho}
\begin{itemize}
\item {Grp. gram.:m.}
\end{itemize}
\begin{itemize}
\item {Utilização:Prov.}
\end{itemize}
\begin{itemize}
\item {Utilização:beir.}
\end{itemize}
O mesmo que \textunderscore cincho\textunderscore ^1.
\section{Acinésico}
\begin{itemize}
\item {Grp. gram.:adj.}
\end{itemize}
O mesmo que \textunderscore acinético\textunderscore .
\section{Acinético}
\begin{itemize}
\item {Grp. gram.:adj.}
\end{itemize}
\begin{itemize}
\item {Utilização:Med.}
\end{itemize}
\begin{itemize}
\item {Proveniência:(Do gr. \textunderscore a\textunderscore  priv. + \textunderscore kinesis\textunderscore )}
\end{itemize}
Que serve para combater a agitação; calmante.
\section{Acipreste}
\begin{itemize}
\item {Grp. gram.:m.}
\end{itemize}
(Corr. pop. de \textunderscore arcipreste\textunderscore )
\section{Achlamídeas}
\begin{itemize}
\item {Grp. gram.:adj. f. pl.}
\end{itemize}
\begin{itemize}
\item {Utilização:Bot.}
\end{itemize}
\begin{itemize}
\item {Proveniência:(Do gr. \textunderscore a\textunderscore  priv. + \textunderscore chlâmide\textunderscore )}
\end{itemize}
Diz-se das flôres, que não têm cálice nem corolla.
\section{Aclamídeas}
\begin{itemize}
\item {Grp. gram.:adj. f. pl.}
\end{itemize}
\begin{itemize}
\item {Utilização:Bot.}
\end{itemize}
\begin{itemize}
\item {Proveniência:(Do gr. \textunderscore a\textunderscore  priv. + \textunderscore chlâmide\textunderscore )}
\end{itemize}
Diz-se das flôres, que não têm cálice nem corola.
\section{Á-clara}
\begin{itemize}
\item {Grp. gram.:f. Loc. adv.}
\end{itemize}
\begin{itemize}
\item {Utilização:Des.}
\end{itemize}
O mesmo que \textunderscore ás-claras\textunderscore .
\section{Acme}
\begin{itemize}
\item {Grp. gram.:m.}
\end{itemize}
\begin{itemize}
\item {Utilização:Med.}
\end{itemize}
\begin{itemize}
\item {Proveniência:(Gr. \textunderscore akme\textunderscore )}
\end{itemize}
Diz-se \textunderscore período de acme\textunderscore  o período, em que uma doença ou um symptoma attinge a maior intensidade.
\section{Acnite}
\begin{itemize}
\item {Grp. gram.:f.}
\end{itemize}
\begin{itemize}
\item {Utilização:Med.}
\end{itemize}
\begin{itemize}
\item {Proveniência:(De \textunderscore acne\textunderscore )}
\end{itemize}
Pequena saliência subcutânea, que amollece e deixa sair uma serosidade sanguinolenta.
\section{Açoitador}
\begin{itemize}
\item {Grp. gram.:m.  e  adj.}
\end{itemize}
O que açoita.
\section{Açoitadura}
\begin{itemize}
\item {Grp. gram.:f.}
\end{itemize}
\begin{itemize}
\item {Utilização:P. us.}
\end{itemize}
O mesmo que \textunderscore açoite\textunderscore , acto de açoitar.
\section{Acoito}
\begin{itemize}
\item {Grp. gram.:m.}
\end{itemize}
Acto de acoitar; lugar onde alguém se acoita; coito. Cf. R. Jorge, \textunderscore El Greco\textunderscore , 21.
\section{Acologia}
\begin{itemize}
\item {Grp. gram.:f.}
\end{itemize}
\begin{itemize}
\item {Utilização:Med.}
\end{itemize}
\begin{itemize}
\item {Proveniência:(Do gr. \textunderscore akos\textunderscore  + \textunderscore logos\textunderscore )}
\end{itemize}
O mesmo que \textunderscore acognosia\textunderscore .
\section{Acompadrado}
\begin{itemize}
\item {Grp. gram.:m.}
\end{itemize}
\begin{itemize}
\item {Utilização:Des.}
\end{itemize}
\begin{itemize}
\item {Proveniência:(De \textunderscore acompadrar\textunderscore )}
\end{itemize}
O mesmo que \textunderscore compadrio\textunderscore .
\section{Açoutador}
\begin{itemize}
\item {Grp. gram.:m.  e  adj.}
\end{itemize}
O que açouta.
\section{Açoutadura}
\begin{itemize}
\item {Grp. gram.:f.}
\end{itemize}
\begin{itemize}
\item {Utilização:P. us.}
\end{itemize}
O mesmo que \textunderscore açoute\textunderscore , acto de açoutar.
\section{Acrasia}
\begin{itemize}
\item {Grp. gram.:f.}
\end{itemize}
\begin{itemize}
\item {Utilização:Med.}
\end{itemize}
\begin{itemize}
\item {Proveniência:(Gr. \textunderscore akrasia\textunderscore )}
\end{itemize}
Qualquer espécie de intemperança.
\section{Acrinia}
\begin{itemize}
\item {Grp. gram.:f.}
\end{itemize}
\begin{itemize}
\item {Utilização:Med.}
\end{itemize}
\begin{itemize}
\item {Proveniência:(Do gr. \textunderscore a\textunderscore  priv. + \textunderscore krinein\textunderscore )}
\end{itemize}
Ausência ou deminuição de secreção.
\section{Acroático}
\begin{itemize}
\item {Grp. gram.:adj.}
\end{itemize}
\begin{itemize}
\item {Proveniência:(Lat. \textunderscore acroaticus\textunderscore )}
\end{itemize}
Dizia-se dos livros reservados, que Aristóteles só deixava que fôssem lidos pelos seus discípulos.
\section{Acroneurose}
\begin{itemize}
\item {Grp. gram.:f.}
\end{itemize}
\begin{itemize}
\item {Utilização:Med.}
\end{itemize}
\begin{itemize}
\item {Proveniência:(Do gr. \textunderscore akros\textunderscore  + \textunderscore neuron\textunderscore )}
\end{itemize}
Perturbações nervosas das extremidades do corpo.
\section{Acropostite}
\begin{itemize}
\item {Grp. gram.:f.}
\end{itemize}
\begin{itemize}
\item {Utilização:Med.}
\end{itemize}
\begin{itemize}
\item {Proveniência:(Do gr. \textunderscore akros\textunderscore  + \textunderscore poste\textunderscore )}
\end{itemize}
Inflamação do prepúcio.
\section{Actinomórfico}
\begin{itemize}
\item {Grp. gram.:adj.}
\end{itemize}
\begin{itemize}
\item {Utilização:Bot.}
\end{itemize}
\begin{itemize}
\item {Proveniência:(Do gr. \textunderscore aktin\textunderscore  + \textunderscore morphe\textunderscore )}
\end{itemize}
Diz-se das flôres, de simetria radiada, regulares.
\section{Actinomórphico}
\begin{itemize}
\item {Grp. gram.:adj.}
\end{itemize}
\begin{itemize}
\item {Utilização:Bot.}
\end{itemize}
\begin{itemize}
\item {Proveniência:(Do gr. \textunderscore aktin\textunderscore  + \textunderscore morphe\textunderscore )}
\end{itemize}
Diz-se das flôres, de symetria radiada, regulares.
\section{Açuda}
\begin{itemize}
\item {Grp. gram.:f.}
\end{itemize}
\begin{itemize}
\item {Utilização:Prov.}
\end{itemize}
\begin{itemize}
\item {Utilização:trasm.}
\end{itemize}
O mesmo que \textunderscore açude\textunderscore .
\section{Acuo}
\begin{itemize}
\item {Grp. gram.:m.}
\end{itemize}
\begin{itemize}
\item {Utilização:Bras}
\end{itemize}
Acto de acuar.
\section{Acurvejar-se}
\begin{itemize}
\item {Grp. gram.:v. p.}
\end{itemize}
\begin{itemize}
\item {Utilização:T. da Guarda}
\end{itemize}
\begin{itemize}
\item {Proveniência:(De \textunderscore curvar\textunderscore )}
\end{itemize}
Curvar-se.
Humilhar-se.
\section{Acyanoblepsia}
\begin{itemize}
\item {Grp. gram.:f.}
\end{itemize}
\begin{itemize}
\item {Utilização:Med.}
\end{itemize}
\begin{itemize}
\item {Proveniência:(Do gr. \textunderscore a\textunderscore  priv. + \textunderscore kuanos\textunderscore  + \textunderscore blepsis\textunderscore )}
\end{itemize}
Insensibilidade visual para a côr azul.
\section{Adamastoriano}
\begin{itemize}
\item {Grp. gram.:adj.}
\end{itemize}
Relativo ao Adamastor, gigante imaginário dos \textunderscore Lusíadas\textunderscore .
\section{Adejo}
\begin{itemize}
\item {Grp. gram.:adj.}
\end{itemize}
\begin{itemize}
\item {Utilização:Bras. do N}
\end{itemize}
Diz-se do cavallo, que vagueia sem cavalleiro nem carga.
(Alter. de \textunderscore andejo\textunderscore , provavelmente)
\section{Adelfeira}
\begin{itemize}
\item {Grp. gram.:f.}
\end{itemize}
\begin{itemize}
\item {Utilização:Bot.}
\end{itemize}
O mesmo que \textunderscore adelfa\textunderscore . Cf. Coutinho, \textunderscore Flora\textunderscore , 462.
\section{Adenocancro}
\begin{itemize}
\item {Grp. gram.:m.}
\end{itemize}
O mesmo que \textunderscore adnocarcinoma\textunderscore .
\section{Adnocarcinoma}
\begin{itemize}
\item {Grp. gram.:m.}
\end{itemize}
\begin{itemize}
\item {Utilização:Med.}
\end{itemize}
\begin{itemize}
\item {Proveniência:(Do gr. \textunderscore aden\textunderscore  + \textunderscore karkinoma\textunderscore )}
\end{itemize}
Cancro, de origem glandular, e que se observa principalmente no fígado.
\section{Adefagia}
\begin{itemize}
\item {Grp. gram.:f.}
\end{itemize}
\begin{itemize}
\item {Utilização:Med.}
\end{itemize}
\begin{itemize}
\item {Proveniência:(Do gr. \textunderscore aden\textunderscore  + \textunderscore phagein\textunderscore )}
\end{itemize}
Voracidade; apetite insaciável.
\section{Adenografia}
\begin{itemize}
\item {Grp. gram.:f.}
\end{itemize}
\begin{itemize}
\item {Utilização:Anat.}
\end{itemize}
\begin{itemize}
\item {Proveniência:(Do gr. \textunderscore aden\textunderscore  + \textunderscore graphein\textunderscore )}
\end{itemize}
Descripção das glândulas.
\section{Adenographia}
\begin{itemize}
\item {Grp. gram.:f.}
\end{itemize}
\begin{itemize}
\item {Utilização:Anat.}
\end{itemize}
\begin{itemize}
\item {Proveniência:(Do gr. \textunderscore aden\textunderscore  + \textunderscore graphein\textunderscore )}
\end{itemize}
Descripção das glândulas.
\section{Adenomalacia}
\begin{itemize}
\item {Grp. gram.:f.}
\end{itemize}
\begin{itemize}
\item {Utilização:Med.}
\end{itemize}
\begin{itemize}
\item {Proveniência:(Do gr. \textunderscore aden\textunderscore  + \textunderscore malakia\textunderscore )}
\end{itemize}
Amollecimento das glândulas.
\section{Adenomixoma}
\begin{itemize}
\item {fónica:csô}
\end{itemize}
\begin{itemize}
\item {Grp. gram.:m.}
\end{itemize}
\begin{itemize}
\item {Utilização:Med.}
\end{itemize}
\begin{itemize}
\item {Proveniência:(Do gr. \textunderscore aden\textunderscore  + \textunderscore muxa\textunderscore )}
\end{itemize}
Tumor, desenvolvido á custa dos elementos de uma glândula.
\section{Adenomyxoma}
\begin{itemize}
\item {fónica:csô}
\end{itemize}
\begin{itemize}
\item {Grp. gram.:m.}
\end{itemize}
\begin{itemize}
\item {Utilização:Med.}
\end{itemize}
\begin{itemize}
\item {Proveniência:(Do gr. \textunderscore aden\textunderscore  + \textunderscore muxa\textunderscore )}
\end{itemize}
Tumor, desenvolvido á custa dos elementos de uma glândula.
\section{Adenosclerose}
\begin{itemize}
\item {Grp. gram.:f.}
\end{itemize}
\begin{itemize}
\item {Utilização:Med.}
\end{itemize}
\begin{itemize}
\item {Proveniência:(Do gr. \textunderscore aden\textunderscore  + \textunderscore skleros\textunderscore )}
\end{itemize}
Endurecimento das glândulas.
(Antónymo de \textunderscore adenomalacia\textunderscore )
\section{Adephagia}
\begin{itemize}
\item {Grp. gram.:f.}
\end{itemize}
\begin{itemize}
\item {Utilização:Med.}
\end{itemize}
\begin{itemize}
\item {Proveniência:(Do gr. \textunderscore aden\textunderscore  + \textunderscore phagein\textunderscore )}
\end{itemize}
Voracidade; appetite insaciável.
\section{Adevão}
\begin{itemize}
\item {Grp. gram.:adj.}
\end{itemize}
\begin{itemize}
\item {Utilização:Bras. do N}
\end{itemize}
Valente; corajoso.
\section{Adipogenia}
\begin{itemize}
\item {Grp. gram.:f.}
\end{itemize}
\begin{itemize}
\item {Utilização:Med.}
\end{itemize}
\begin{itemize}
\item {Proveniência:(Do lat. \textunderscore adeps\textunderscore , \textunderscore adipos\textunderscore  + gr. \textunderscore genos\textunderscore )}
\end{itemize}
Formação da gordura no organismo.
\section{Admoestativo}
\begin{itemize}
\item {fónica:mo-es}
\end{itemize}
\begin{itemize}
\item {Grp. gram.:adj.}
\end{itemize}
O mesmo que \textunderscore admoestatório\textunderscore .
\section{Adôbe}
\begin{itemize}
\item {Grp. gram.:m.}
\end{itemize}
\begin{itemize}
\item {Utilização:Prov.}
\end{itemize}
\begin{itemize}
\item {Utilização:trasm.}
\end{itemize}
O mesmo que \textunderscore adube\textunderscore .
\section{Adraganta}
\begin{itemize}
\item {Grp. gram.:f.}
\end{itemize}
O mesmo que \textunderscore tragacantha\textunderscore .
\section{Adubabela}
\begin{itemize}
\item {Grp. gram.:f.}
\end{itemize}
\begin{itemize}
\item {Utilização:Prov.}
\end{itemize}
\begin{itemize}
\item {Utilização:minh.}
\end{itemize}
\begin{itemize}
\item {Proveniência:(De \textunderscore adubar\textunderscore )}
\end{itemize}
Grande sova ou tareia. (Colhido em Paredes-de-Coira)
\section{Adulativo}
\begin{itemize}
\item {Grp. gram.:adj.}
\end{itemize}
\begin{itemize}
\item {Utilização:P. us.}
\end{itemize}
Próprio para adular; que envolve adulação: \textunderscore carta adulativa\textunderscore .
\section{Adverbialidade}
\begin{itemize}
\item {Grp. gram.:f.}
\end{itemize}
Qualidade de adverbial.
\section{Advogacia}
\begin{itemize}
\item {Grp. gram.:f.}
\end{itemize}
(V.advocacia)
\section{Aeremia}
\begin{itemize}
\item {fónica:a-e}
\end{itemize}
\begin{itemize}
\item {Grp. gram.:f.}
\end{itemize}
\begin{itemize}
\item {Utilização:Med.}
\end{itemize}
\begin{itemize}
\item {Proveniência:(Do gr. \textunderscore aer\textunderscore  + \textunderscore haima\textunderscore )}
\end{itemize}
Conjunto de accidentes, que sobrevêm nos operários, a quem se comprime o ar, como nos mergulhadores: hemorragias, paralysias, etc.
\section{Aeroclube}
\begin{itemize}
\item {fónica:a-e}
\end{itemize}
\begin{itemize}
\item {Grp. gram.:m.}
\end{itemize}
\begin{itemize}
\item {Utilização:Neol.}
\end{itemize}
Clube recreativo, em que se cultiva o desporto das ascensões atmosphéricas.
\section{Aerocolia}
\begin{itemize}
\item {fónica:a-e}
\end{itemize}
\begin{itemize}
\item {Grp. gram.:f.}
\end{itemize}
\begin{itemize}
\item {Utilização:Med.}
\end{itemize}
\begin{itemize}
\item {Proveniência:(Do gr. \textunderscore aer\textunderscore  + \textunderscore kolon\textunderscore )}
\end{itemize}
Meteorismo intestinal.
\section{Aerotropismo}
\begin{itemize}
\item {fónica:a-e}
\end{itemize}
\begin{itemize}
\item {Grp. gram.:m.}
\end{itemize}
\begin{itemize}
\item {Proveniência:(Do gr. \textunderscore aer\textunderscore  + \textunderscore trophe\textunderscore )}
\end{itemize}
Propriedade, que o protoplasma tem, de reagir perante a acção do oxygênio.
\section{Afaquear}
\begin{itemize}
\item {Grp. gram.:v. t.}
\end{itemize}
\begin{itemize}
\item {Utilização:T. da Guarda}
\end{itemize}
O mesmo que \textunderscore esfaquear\textunderscore .
\section{Afeitear}
\begin{itemize}
\item {Grp. gram.:v. t.}
\end{itemize}
\begin{itemize}
\item {Utilização:Prov.}
\end{itemize}
\begin{itemize}
\item {Utilização:minh.}
\end{itemize}
O mesmo que \textunderscore afeitar\textunderscore ^1.
\section{Aferível}
\begin{itemize}
\item {Grp. gram.:adj.}
\end{itemize}
Que se póde aferir:«\textunderscore medidas aferíveis.\textunderscore »\textunderscore Bol. do Trab. Industr.\textunderscore , LXXXII, 17.
\section{Afermosear}
\begin{itemize}
\item {Grp. gram.:v. t.}
\end{itemize}
\begin{itemize}
\item {Utilização:Ant.}
\end{itemize}
\begin{itemize}
\item {Proveniência:(De \textunderscore fermoso\textunderscore )}
\end{itemize}
O mesmo que \textunderscore aformosear\textunderscore .
\section{Afermosentar}
\begin{itemize}
\item {Grp. gram.:v. t.}
\end{itemize}
\begin{itemize}
\item {Utilização:Ant.}
\end{itemize}
\begin{itemize}
\item {Proveniência:(De \textunderscore fermoso\textunderscore )}
\end{itemize}
O mesmo que \textunderscore aformosentar\textunderscore .
\section{Afestonar}
\begin{itemize}
\item {Grp. gram.:v. t.}
\end{itemize}
O mesmo que \textunderscore afestoar\textunderscore .
\section{Afilador}
\begin{itemize}
\item {Grp. gram.:m.}
\end{itemize}
\begin{itemize}
\item {Proveniência:(De \textunderscore afilar\textunderscore ^1)}
\end{itemize}
O mesmo que \textunderscore aferidor\textunderscore .
\section{Afiladura}
\begin{itemize}
\item {Grp. gram.:f.}
\end{itemize}
Acto de afilar^1; aferição.
\section{Afinhar}
\begin{itemize}
\item {Grp. gram.:v. t.}
\end{itemize}
\begin{itemize}
\item {Utilização:Prov.}
\end{itemize}
\begin{itemize}
\item {Utilização:trasm.}
\end{itemize}
\begin{itemize}
\item {Grp. gram.:V. i.}
\end{itemize}
Importunar; fatigar.
O mesmo que \textunderscore emmagrecer\textunderscore .
(Cp. \textunderscore afinar\textunderscore )
\section{Afinheiro}
\begin{itemize}
\item {Grp. gram.:adj.}
\end{itemize}
\begin{itemize}
\item {Utilização:Prov.}
\end{itemize}
\begin{itemize}
\item {Utilização:trasm.}
\end{itemize}
\begin{itemize}
\item {Proveniência:(De \textunderscore afinhar\textunderscore )}
\end{itemize}
Que afinha, que mortifica.
Importuno.
Teimoso.
\section{Afistulado}
\begin{itemize}
\item {Grp. gram.:adj.}
\end{itemize}
\begin{itemize}
\item {Proveniência:(De \textunderscore fistular\textunderscore )}
\end{itemize}
Que tem fístulas; fistuloso.
\section{Afloixar}
\begin{itemize}
\item {Grp. gram.:v. t.}
\end{itemize}
\begin{itemize}
\item {Utilização:Ant.}
\end{itemize}
O mesmo que \textunderscore afroixar\textunderscore .
\section{Aflouxar}
\begin{itemize}
\item {Grp. gram.:v. t.}
\end{itemize}
\begin{itemize}
\item {Utilização:Ant.}
\end{itemize}
O mesmo que \textunderscore afrouxar\textunderscore .
\section{Aforritar}
\begin{itemize}
\item {Grp. gram.:v. i.}
\end{itemize}
\begin{itemize}
\item {Utilização:Prov.}
\end{itemize}
Fugir, voar, (falando-se de ave, que escapou das mãos de alguém).
(Cp. \textunderscore fôrro\textunderscore ^2)
\section{Afragatado}
\begin{itemize}
\item {Grp. gram.:adj.}
\end{itemize}
Semelhante a um fragata, (embarcação). Cf. B. Pereira, \textunderscore Prosódia\textunderscore , vb. \textunderscore ratiariae\textunderscore .
\section{Afregulhado}
\begin{itemize}
\item {Grp. gram.:adj.}
\end{itemize}
Apressado; precipitado.
(Talvez por \textunderscore afagulhado\textunderscore , de \textunderscore fagulha\textunderscore . Cp. \textunderscore fagulha\textunderscore )
\section{Agaitado}
\begin{itemize}
\item {Grp. gram.:adj.}
\end{itemize}
\begin{itemize}
\item {Utilização:Prov.}
\end{itemize}
\begin{itemize}
\item {Utilização:minh.}
\end{itemize}
Diz-se da voz ou do som, semelhante ao de uma gaita. (Colhido em Paredes-de-Coira)
\section{Agalha}
\begin{itemize}
\item {Grp. gram.:f.}
\end{itemize}
\begin{itemize}
\item {Utilização:Des.}
\end{itemize}
O mesmo que \textunderscore galha\textunderscore ^2.
\section{Agasalhado}
\begin{itemize}
\item {Grp. gram.:m.}
\end{itemize}
\begin{itemize}
\item {Proveniência:(De \textunderscore agasalhar\textunderscore )}
\end{itemize}
O mesmo que \textunderscore agasalho\textunderscore ; hospedagem carinhosa. Cf. Pant. de Aveiro, \textunderscore Itiner.\textunderscore , 142, (2.^a ed.).
\section{Agglutinabilidade}
\begin{itemize}
\item {Grp. gram.:f.}
\end{itemize}
Qualidade de agglutinável:«\textunderscore a agglutinabilidade das bactérias\textunderscore ». \textunderscore Rev. da Univ. de Coímbra\textunderscore , II, 67.
\section{Agglutinável}
\begin{itemize}
\item {Grp. gram.:adj.}
\end{itemize}
Que se póde agglutinar.
\section{Aggressiva}
\begin{itemize}
\item {Grp. gram.:f.}
\end{itemize}
\begin{itemize}
\item {Utilização:Med.}
\end{itemize}
Substância, segregada por certas bactérias, e que é aggressiva para as céllulas do organismo.
(Cp. \textunderscore aggressivo\textunderscore )
\section{Aglutinabilidade}
\begin{itemize}
\item {Grp. gram.:f.}
\end{itemize}
Qualidade de aglutinável:«\textunderscore a aglutinabilidade das bactérias\textunderscore ». \textunderscore Rev. da Univ. de Coímbra\textunderscore , II, 67.
\section{Aglutinável}
\begin{itemize}
\item {Grp. gram.:adj.}
\end{itemize}
Que se póde aglutinar.
\section{Agminado}
\begin{itemize}
\item {Grp. gram.:adj.}
\end{itemize}
\begin{itemize}
\item {Utilização:Physiol.}
\end{itemize}
\begin{itemize}
\item {Proveniência:(Do lat. \textunderscore agminari\textunderscore )}
\end{itemize}
Diz-se de vários órgãos elementares da mesma espécie, quando reunidos ou aproximados.
\section{Agramatismo}
\begin{itemize}
\item {Grp. gram.:m.}
\end{itemize}
\begin{itemize}
\item {Utilização:Med.}
\end{itemize}
\begin{itemize}
\item {Proveniência:(Do gr. \textunderscore a\textunderscore  priv. + \textunderscore grammata\textunderscore )}
\end{itemize}
Vício de pronúncia, que consiste na omissão de um ou mais sons de uma palavra.
Impossibilidade de colocar as palavras, segundo a sintaxe.
\section{Agrammatismo}
\begin{itemize}
\item {Grp. gram.:m.}
\end{itemize}
\begin{itemize}
\item {Utilização:Med.}
\end{itemize}
\begin{itemize}
\item {Proveniência:(Do gr. \textunderscore a\textunderscore  priv. + \textunderscore grammata\textunderscore )}
\end{itemize}
Vício de pronúncia, que consiste na omissão de um ou mais sons de uma palavra.
Impossibilidade de collocar as palavras, segundo a syntaxe.
\section{Agressiva}
\begin{itemize}
\item {Grp. gram.:f.}
\end{itemize}
\begin{itemize}
\item {Utilização:Med.}
\end{itemize}
Substância, segregada por certas bactérias, e que é agressiva para as células do organismo.
(Cp. \textunderscore agressivo\textunderscore )
\section{Aguareira}
\begin{itemize}
\item {Grp. gram.:f.}
\end{itemize}
\begin{itemize}
\item {Utilização:Açor}
\end{itemize}
Espécie de gaivota.
\section{Águia-pesqueira}
\begin{itemize}
\item {Grp. gram.:f.}
\end{itemize}
\begin{itemize}
\item {Utilização:Zool.}
\end{itemize}
O mesmo que \textunderscore aurifrísio\textunderscore . Cf. P. Moraes, \textunderscore Zool. Elem.\textunderscore , 360.
\section{Ahume}
\begin{itemize}
\item {Grp. gram.:m.}
\end{itemize}
\begin{itemize}
\item {Utilização:Ant.}
\end{itemize}
O mesmo que \textunderscore alúmen\textunderscore .
\section{Aimara}
\begin{itemize}
\item {Grp. gram.:f.}
\end{itemize}
Árvore de Timor, de madeira avermelhada, pesada e dura. Cf. \textunderscore Século\textunderscore , de 30-VII-911.
\section{Aioro}
\begin{itemize}
\item {Grp. gram.:m.}
\end{itemize}
Árvore de Timor, de madeira branca.
\section{Airela}
\begin{itemize}
\item {Grp. gram.:f.}
\end{itemize}
Árvore, o mesmo que \textunderscore arando\textunderscore .
\section{Alabandite}
\begin{itemize}
\item {Grp. gram.:f.}
\end{itemize}
O mesmo que \textunderscore alabandina\textunderscore .
\section{Alabaça}
\begin{itemize}
\item {Grp. gram.:f.}
\end{itemize}
Planta, o mesmo que \textunderscore labaça\textunderscore .
\section{Alacremente}
\begin{itemize}
\item {Grp. gram.:adv.}
\end{itemize}
De modo álacre.
\section{Alampião}
\begin{itemize}
\item {Grp. gram.:m.}
\end{itemize}
\begin{itemize}
\item {Utilização:ant.}
\end{itemize}
\begin{itemize}
\item {Utilização:Pop.}
\end{itemize}
O mesmo que \textunderscore lampião\textunderscore . Cf. B. Pereira, \textunderscore Prosódia\textunderscore , vb. \textunderscore polymixus\textunderscore .
\section{Alantoídea}
\begin{itemize}
\item {Grp. gram.:f.}
\end{itemize}
\begin{itemize}
\item {Utilização:Veter.}
\end{itemize}
Vesícula, situada entre o chórion e o amnios. Cf. M. Pinto, \textunderscore Comp. de Veter.\textunderscore , II, 230.
\section{Alastrim}
\begin{itemize}
\item {Grp. gram.:m.}
\end{itemize}
\begin{itemize}
\item {Utilização:Bras}
\end{itemize}
O mesmo que \textunderscore varíola-mansa\textunderscore .
(Cp. \textunderscore alastrar\textunderscore )
\section{Albatoça}
\begin{itemize}
\item {Grp. gram.:f.}
\end{itemize}
O mesmo que \textunderscore albetoça\textunderscore .
\section{Albinia}
\begin{itemize}
\item {Grp. gram.:f.}
\end{itemize}
\begin{itemize}
\item {Utilização:Ant.}
\end{itemize}
O mesmo que \textunderscore albinismo\textunderscore .
\section{Albuminismo}
\begin{itemize}
\item {Grp. gram.:m.}
\end{itemize}
\begin{itemize}
\item {Proveniência:(De \textunderscore albumina\textunderscore )}
\end{itemize}
Excesso de substâncias albuminóides na alimentação.
Perturbações da nutrição, resultantes dêsse excesso.
\section{Alcalinimetria}
\begin{itemize}
\item {Grp. gram.:f.}
\end{itemize}
Dosagem da alcalinidade de um líquido orgânico, do sangue especialmente.
\section{Alcalinismo}
\begin{itemize}
\item {Grp. gram.:m.}
\end{itemize}
Uso immoderado de substâncias alcalinas.
Effeitos perniciosos dêsse uso.
\section{Alcaniça}
\begin{itemize}
\item {Grp. gram.:f.}
\end{itemize}
\begin{itemize}
\item {Utilização:Des.}
\end{itemize}
O mesmo que \textunderscore mesquita\textunderscore . Cf. Deusdado, \textunderscore Escorços\textunderscore .
(Cast. \textunderscore alcaniza\textunderscore )
\section{Alcaptonuria}
\begin{itemize}
\item {Grp. gram.:f.}
\end{itemize}
\begin{itemize}
\item {Utilização:Med.}
\end{itemize}
Presença da alcaptona na urina.
\section{Alcocerino}
\begin{itemize}
\item {Grp. gram.:adj.}
\end{itemize}
\begin{itemize}
\item {Grp. gram.:M.}
\end{itemize}
Relativo a Alcocer, no Egýpto.
Habitante de Alcocer. Cf. \textunderscore Rot. do Mar-Vermelho\textunderscore , 187.
\section{Alcoforado}
\begin{itemize}
\item {Grp. gram.:adj.}
\end{itemize}
\begin{itemize}
\item {Proveniência:(De \textunderscore alcofor\textunderscore  = \textunderscore antimónio\textunderscore )}
\end{itemize}
Diz-se dos olhos, orlados de escuro, natural ou artificialmente.
\section{Alcoolomania}
\begin{itemize}
\item {Grp. gram.:f.}
\end{itemize}
Período latente da entoxicação alcoólica chrónica, durante o qual a acção do álcool apenas se manifesta pelo hábito e necessidade de o usar.
\section{Alcovez}
\begin{itemize}
\item {Grp. gram.:f.}
\end{itemize}
\begin{itemize}
\item {Utilização:Des.}
\end{itemize}
O mesmo que \textunderscore alcovitagem\textunderscore .
\section{Alcovitagem}
\begin{itemize}
\item {Grp. gram.:f.}
\end{itemize}
Acto de alcovitar. Cf. Fialho, \textunderscore Gatos\textunderscore , II, 5.
\section{Alçufeifa}
\begin{itemize}
\item {Grp. gram.:f.}
\end{itemize}
\begin{itemize}
\item {Utilização:Prov.}
\end{itemize}
\begin{itemize}
\item {Utilização:alg.}
\end{itemize}
O mesmo que \textunderscore açofeifa\textunderscore . Cf. \textunderscore Port. au Point de Vue Agr.\textunderscore , 620.
\section{Alegra-campo}
\begin{itemize}
\item {Grp. gram.:m.}
\end{itemize}
\begin{itemize}
\item {Utilização:Bot.}
\end{itemize}
Planta liliácea, (\textunderscore semele androgina\textunderscore , Kunth.), cujos ramos os Madeirenses empregam na ornamentação de igrejas e de janelas, por occasião de festas religiosas. Cf. \textunderscore Bol. da Socied. de Geogr.\textunderscore , XXX, 620.
O mesmo que \textunderscore legação\textunderscore ^1.
\section{Alface-dos-montes}
\begin{itemize}
\item {Grp. gram.:f.}
\end{itemize}
\begin{itemize}
\item {Utilização:Bot.}
\end{itemize}
Planta, o mesmo que \textunderscore tripa-de-ovelha\textunderscore .
\section{Alfarge}
\begin{itemize}
\item {Grp. gram.:m.  e  adj.}
\end{itemize}
Diz-se de um estilo peninsular de artes decorativas, caracterizado por lavores multiformes.
\section{Alfarrabístico}
\begin{itemize}
\item {Grp. gram.:adj.}
\end{itemize}
Relativo a alfarrabista: \textunderscore aspecto alfarrabístico\textunderscore .
\section{Alforfa}
\begin{itemize}
\item {fónica:fôr}
\end{itemize}
\begin{itemize}
\item {Grp. gram.:f.}
\end{itemize}
\begin{itemize}
\item {Utilização:Prov.}
\end{itemize}
\begin{itemize}
\item {Utilização:alent.}
\end{itemize}
O mesmo que \textunderscore alfova\textunderscore .
\section{Alfova}
\begin{itemize}
\item {fónica:fô}
\end{itemize}
\begin{itemize}
\item {Grp. gram.:f.}
\end{itemize}
O mesmo que \textunderscore alforva\textunderscore . Cf. B. Pereira, \textunderscore Prosodia\textunderscore , vb. \textunderscore ceratis\textunderscore .
\section{Algarvão}
\begin{itemize}
\item {Grp. gram.:m.}
\end{itemize}
Ave, o mesmo que \textunderscore alcaravão\textunderscore .
\section{Algesímetro}
\begin{itemize}
\item {Grp. gram.:m.}
\end{itemize}
\begin{itemize}
\item {Utilização:Med.}
\end{itemize}
\begin{itemize}
\item {Proveniência:(Do gr. \textunderscore algesis\textunderscore  + \textunderscore metron\textunderscore )}
\end{itemize}
Apparelho, para medir a intensidade da excitação, necessária para produzir uma impressão dolorosa.
\section{Algia}
\begin{itemize}
\item {Grp. gram.:f.}
\end{itemize}
\begin{itemize}
\item {Utilização:Med.}
\end{itemize}
\begin{itemize}
\item {Proveniência:(Do gr. \textunderscore algos\textunderscore )}
\end{itemize}
Dôr num órgão ou numa região do corpo, sem corresponder a uma lesão anatómica apreciável.
\section{Algidez}
\begin{itemize}
\item {Grp. gram.:f.}
\end{itemize}
Qualidade ou estado de álgido; frialdade: \textunderscore algidez cadavérica\textunderscore .
\section{Algofilia}
\begin{itemize}
\item {Grp. gram.:f.}
\end{itemize}
\begin{itemize}
\item {Utilização:Med.}
\end{itemize}
Estado do algófilo.
\section{Algofobia}
\begin{itemize}
\item {Grp. gram.:f.}
\end{itemize}
\begin{itemize}
\item {Utilização:Med.}
\end{itemize}
\begin{itemize}
\item {Proveniência:(Do gr. \textunderscore algos\textunderscore  + \textunderscore phobein\textunderscore )}
\end{itemize}
Terror mórbido das dores.
\section{Algophilia}
\begin{itemize}
\item {Grp. gram.:f.}
\end{itemize}
\begin{itemize}
\item {Utilização:Med.}
\end{itemize}
Estado do algóphilo.
\section{Algophobia}
\begin{itemize}
\item {Grp. gram.:f.}
\end{itemize}
\begin{itemize}
\item {Utilização:Med.}
\end{itemize}
\begin{itemize}
\item {Proveniência:(Do gr. \textunderscore algos\textunderscore  + \textunderscore phobein\textunderscore )}
\end{itemize}
Terror mórbido das dores.
\section{Algóstase}
\begin{itemize}
\item {Grp. gram.:f.}
\end{itemize}
\begin{itemize}
\item {Utilização:Med.}
\end{itemize}
\begin{itemize}
\item {Proveniência:(Do gr. \textunderscore algos\textunderscore  + \textunderscore stasis\textunderscore )}
\end{itemize}
Deminuição ou extincção da sensibilidade á dôr, em casos de grande traumatismo.
\section{Alhendros}
\begin{itemize}
\item {Grp. gram.:m.}
\end{itemize}
\begin{itemize}
\item {Utilização:Mad}
\end{itemize}
O mesmo que \textunderscore estramónio\textunderscore .
\section{Alindres}
\begin{itemize}
\item {Grp. gram.:m.}
\end{itemize}
\begin{itemize}
\item {Utilização:Mad}
\end{itemize}
O mesmo que \textunderscore estramónio\textunderscore .
\section{Alfitomancia}
\begin{itemize}
\item {Grp. gram.:f.}
\end{itemize}
\begin{itemize}
\item {Proveniência:(Do gr. \textunderscore alphiton\textunderscore  + \textunderscore manteia\textunderscore )}
\end{itemize}
Supposta adivinhação, por meio de farinha.
\section{Alinfia}
\begin{itemize}
\item {Grp. gram.:f.}
\end{itemize}
\begin{itemize}
\item {Utilização:Med.}
\end{itemize}
\begin{itemize}
\item {Proveniência:(Do gr. \textunderscore a\textunderscore  priv. + \textunderscore lumphe\textunderscore )}
\end{itemize}
Falta de linfa.
\section{Aliteratado}
\begin{itemize}
\item {Grp. gram.:adj.}
\end{itemize}
Um tanto literato. Cf. Camillo, \textunderscore Serões\textunderscore , I, 8.
\section{Allopsychose}
\begin{itemize}
\item {fónica:có}
\end{itemize}
\begin{itemize}
\item {Grp. gram.:f.}
\end{itemize}
\begin{itemize}
\item {Utilização:Med.}
\end{itemize}
\begin{itemize}
\item {Proveniência:(Do gr. \textunderscore allos\textunderscore  + \textunderscore psukhe\textunderscore )}
\end{itemize}
Psychose, caracterizada por perturbação na percepção dos phenómenos externos.
\section{Allophthalmia}
\begin{itemize}
\item {Grp. gram.:f.}
\end{itemize}
\begin{itemize}
\item {Utilização:Med.}
\end{itemize}
\begin{itemize}
\item {Proveniência:(Do gr. \textunderscore allos\textunderscore  + \textunderscore phthalmos\textunderscore )}
\end{itemize}
Differença de coloração da íris, nos dois olhos do mesmo indivíduo.
\section{Allotriotecnia}
\begin{itemize}
\item {Grp. gram.:f.}
\end{itemize}
\begin{itemize}
\item {Utilização:Med.}
\end{itemize}
\begin{itemize}
\item {Proveniência:(Do gr. \textunderscore allotrios\textunderscore  + \textunderscore teknon\textunderscore )}
\end{itemize}
Expulsão de um feto monstruoso.
\section{Aloftalmia}
\begin{itemize}
\item {Grp. gram.:f.}
\end{itemize}
\begin{itemize}
\item {Utilização:Med.}
\end{itemize}
\begin{itemize}
\item {Proveniência:(Do gr. \textunderscore allos\textunderscore  + \textunderscore phthalmos\textunderscore )}
\end{itemize}
Diferença de coloração da íris, nos dois olhos do mesmo indivíduo.
\section{Alopsicose}
\begin{itemize}
\item {Grp. gram.:f.}
\end{itemize}
\begin{itemize}
\item {Utilização:Med.}
\end{itemize}
\begin{itemize}
\item {Proveniência:(Do gr. \textunderscore allos\textunderscore  + \textunderscore psukhe\textunderscore )}
\end{itemize}
Psicose, caracterizada por perturbação na percepção dos fenómenos externos.
\section{Alotriotecnia}
\begin{itemize}
\item {Grp. gram.:f.}
\end{itemize}
\begin{itemize}
\item {Utilização:Med.}
\end{itemize}
\begin{itemize}
\item {Proveniência:(Do gr. \textunderscore allotrios\textunderscore  + \textunderscore teknon\textunderscore )}
\end{itemize}
Expulsão de um feto monstruoso.
\section{Almandite}
\begin{itemize}
\item {Grp. gram.:f.}
\end{itemize}
\begin{itemize}
\item {Utilização:Miner.}
\end{itemize}
O mesmo que \textunderscore almandina\textunderscore .
\section{Almas}
\begin{itemize}
\item {Grp. gram.:f. pl.}
\end{itemize}
\begin{itemize}
\item {Utilização:Prov.}
\end{itemize}
O mesmo que \textunderscore alminhas\textunderscore .
\section{Almeice}
\begin{itemize}
\item {Grp. gram.:m.}
\end{itemize}
O mesmo que \textunderscore almece\textunderscore . Cf. B. Pereira, \textunderscore Prosodia\textunderscore , vb. \textunderscore ichor\textunderscore .
\section{Alpe}
\begin{itemize}
\item {Grp. gram.:m.}
\end{itemize}
\begin{itemize}
\item {Utilização:Ant.}
\end{itemize}
O mesmo que \textunderscore monte\textunderscore  ou \textunderscore serra\textunderscore . Cf. \textunderscore Port. Mon. Hist.\textunderscore ; \textunderscore Elucidario\textunderscore  de S. R. Viterbo, etc.
\section{Alperceiro-do-japão}
\begin{itemize}
\item {Grp. gram.:m.}
\end{itemize}
Arvoreta, o mesmo que \textunderscore cáqui\textunderscore .
\section{Alphitomancia}
\begin{itemize}
\item {Grp. gram.:f.}
\end{itemize}
\begin{itemize}
\item {Proveniência:(Do gr. \textunderscore alphiton\textunderscore  + \textunderscore manteia\textunderscore )}
\end{itemize}
Supposta adivinhação, por meio de farinha.
\section{Alpivre}
\begin{itemize}
\item {Grp. gram.:m.}
\end{itemize}
\begin{itemize}
\item {Utilização:Bot.}
\end{itemize}
O mesmo que \textunderscore erva-abelha\textunderscore . Cf. P. Coutinho, \textunderscore Flora\textunderscore , 151.
\section{Alporcador}
\begin{itemize}
\item {Grp. gram.:m.}
\end{itemize}
Aquelle que alporca.
\section{Alporcamento}
\begin{itemize}
\item {Grp. gram.:m.}
\end{itemize}
O mesmo que \textunderscore alporque\textunderscore . Cf. B. Pereira, \textunderscore Prosodia\textunderscore , vb. \textunderscore imporcitor\textunderscore .
\section{Altér}
\begin{itemize}
\item {Grp. gram.:m.}
\end{itemize}
\begin{itemize}
\item {Proveniência:(De \textunderscore Altér\textunderscore , n. p.)}
\end{itemize}
Raça fina de cavallos portugueses.
\section{Altibaixo}
\begin{itemize}
\item {Grp. gram.:m.}
\end{itemize}
Desigualdade de terreno; terreno accidentado: \textunderscore a região é cheia de altibaixos\textunderscore .
\section{Altitúdico}
\begin{itemize}
\item {Grp. gram.:adj.}
\end{itemize}
Relativo a altitude: \textunderscore influência altitúdica\textunderscore .
\section{Aluar-se}
\begin{itemize}
\item {Grp. gram.:v. p.}
\end{itemize}
\begin{itemize}
\item {Proveniência:(De \textunderscore lua\textunderscore )}
\end{itemize}
Diz-se dos animais, que andam com cio.
\section{Alumío}
\begin{itemize}
\item {Grp. gram.:m.}
\end{itemize}
\begin{itemize}
\item {Utilização:Prov.}
\end{itemize}
\begin{itemize}
\item {Utilização:trasm.}
\end{itemize}
\begin{itemize}
\item {Proveniência:(De \textunderscore alumiar\textunderscore )}
\end{itemize}
O mesmo que \textunderscore relâmpago\textunderscore .
\section{Alvéloa-ribeirinha}
\begin{itemize}
\item {Grp. gram.:f.}
\end{itemize}
Espécie de alvéloa marítima. Cf. \textunderscore Rot. do Mar-Vermelho\textunderscore , 8.
\section{Alvéola}
\begin{itemize}
\item {Grp. gram.:f.}
\end{itemize}
O mesmo que \textunderscore alvéloa\textunderscore .
\section{Alveolite}
\begin{itemize}
\item {Grp. gram.:f.}
\end{itemize}
\begin{itemize}
\item {Utilização:Med.}
\end{itemize}
\begin{itemize}
\item {Proveniência:(De \textunderscore alvéolo\textunderscore )}
\end{itemize}
Periostite nos alvéolos dentários.
Inflammação dos alvéolos pulmonares.
\section{Alvorar}
\begin{itemize}
\item {Grp. gram.:v. i.}
\end{itemize}
\begin{itemize}
\item {Proveniência:(De \textunderscore alvor\textunderscore )}
\end{itemize}
O mesmo que \textunderscore alvorejar\textunderscore :«\textunderscore alvorava uma nova Medicina...\textunderscore »R. Jorge.
\section{Alymphia}
\begin{itemize}
\item {Grp. gram.:f.}
\end{itemize}
\begin{itemize}
\item {Utilização:Med.}
\end{itemize}
\begin{itemize}
\item {Proveniência:(Do gr. \textunderscore a\textunderscore  priv. + \textunderscore lumphe\textunderscore )}
\end{itemize}
Falta de lympha.
\section{Amassadeiro}
\begin{itemize}
\item {Grp. gram.:m.}
\end{itemize}
Aquelle que amassa; amassador.
\section{Amastia}
\begin{itemize}
\item {Grp. gram.:f.}
\end{itemize}
O mesmo que \textunderscore amazia\textunderscore .
\section{Ambaubeira}
\begin{itemize}
\item {fónica:ba-u}
\end{itemize}
\begin{itemize}
\item {Grp. gram.:f.}
\end{itemize}
O mesmo que \textunderscore ambaúba\textunderscore .
\section{Ambrósia-das-pharmácias}
\begin{itemize}
\item {Grp. gram.:f.}
\end{itemize}
Planta aromática, (\textunderscore chenopodium botrys\textunderscore )
\section{Ambrósia-do-méxico}
\begin{itemize}
\item {Grp. gram.:f.}
\end{itemize}
Espécie de erva-formigueira.
\section{Amebíase}
\begin{itemize}
\item {Grp. gram.:f.}
\end{itemize}
\begin{itemize}
\item {Utilização:Med.}
\end{itemize}
Doença, causada por amebas.
\section{Amelía}
\begin{itemize}
\item {Grp. gram.:f.}
\end{itemize}
\begin{itemize}
\item {Utilização:Terat.}
\end{itemize}
\begin{itemize}
\item {Proveniência:(Do gr. \textunderscore a\textunderscore  priv. + \textunderscore melos\textunderscore )}
\end{itemize}
Ausência congênita dos quatros membros.
\section{Ametria}
\begin{itemize}
\item {Grp. gram.:f.}
\end{itemize}
\begin{itemize}
\item {Utilização:Med.}
\end{itemize}
\begin{itemize}
\item {Proveniência:(Do gr. \textunderscore a\textunderscore  priv. + \textunderscore metra\textunderscore )}
\end{itemize}
Ausência de útero.
\section{Amieiro-negro}
\begin{itemize}
\item {Grp. gram.:m.}
\end{itemize}
Arbusto rhamnáceo, espécie de frângula, (\textunderscore frangula vulgaris\textunderscore , Reichenbach).
\section{Amieiros}
\begin{itemize}
\item {Grp. gram.:m. pl.}
\end{itemize}
\begin{itemize}
\item {Utilização:Prov.}
\end{itemize}
\begin{itemize}
\item {Utilização:trasm.}
\end{itemize}
\begin{itemize}
\item {Proveniência:(De \textunderscore amieiro\textunderscore , madeira, de que se fazem tamancos)}
\end{itemize}
Tamancos, socos.
\section{Amielia}
\begin{itemize}
\item {Grp. gram.:f.}
\end{itemize}
\begin{itemize}
\item {Utilização:Med.}
\end{itemize}
\begin{itemize}
\item {Proveniência:(Do gr. \textunderscore a\textunderscore  priv. + \textunderscore muelos\textunderscore )}
\end{itemize}
Monstruosidade, caracterizada pela ausência de medulla espinhal.
\section{Amigdalotomia}
\begin{itemize}
\item {Grp. gram.:f.}
\end{itemize}
Emprêgo cirúrgico do amigdalótomo.
\section{Aministrar}
\begin{itemize}
\item {Grp. gram.:v. t.}
\end{itemize}
\begin{itemize}
\item {Utilização:Ant.}
\end{itemize}
O mesmo que \textunderscore ministrar\textunderscore . Cf. \textunderscore Rev. Lus.\textunderscore , XVI, 2.
\section{Amiotaxia}
\begin{itemize}
\item {fónica:csi}
\end{itemize}
\begin{itemize}
\item {Grp. gram.:f.}
\end{itemize}
\begin{itemize}
\item {Utilização:Med.}
\end{itemize}
\begin{itemize}
\item {Proveniência:(Do gr. \textunderscore a\textunderscore  priv. + \textunderscore mus\textunderscore  + \textunderscore taxis\textunderscore )}
\end{itemize}
Convulsões involuntárias, de origem reflexa, determinadas muitas vezes por vários neurites.
\section{Amiotrofia}
\begin{itemize}
\item {Grp. gram.:f.}
\end{itemize}
\begin{itemize}
\item {Utilização:Med.}
\end{itemize}
\begin{itemize}
\item {Proveniência:(Do gr. \textunderscore a\textunderscore  priv. + \textunderscore mus\textunderscore  + \textunderscore trophe\textunderscore )}
\end{itemize}
Atrofia dos músculos.
\section{Amisurado}
\begin{itemize}
\item {Grp. gram.:adj.}
\end{itemize}
\begin{itemize}
\item {Utilização:Náut.}
\end{itemize}
Meio colhido, para não apanhar tanto vento:«\textunderscore com traquetes amisurados sem outra alguma vela.\textunderscore »\textunderscore Rot. do Mar-Vermelho\textunderscore , 45.
(Por \textunderscore amesurado\textunderscore , de \textunderscore mesurar\textunderscore )
\section{Amixia}
\begin{itemize}
\item {fónica:csi}
\end{itemize}
\begin{itemize}
\item {Grp. gram.:f.}
\end{itemize}
\begin{itemize}
\item {Utilização:Med.}
\end{itemize}
\begin{itemize}
\item {Proveniência:(Do gr. \textunderscore a\textunderscore  priv. + \textunderscore muxa\textunderscore )}
\end{itemize}
Ausência de secreção do muco normal.
\section{Amolancar}
\begin{itemize}
\item {Grp. gram.:v. t.}
\end{itemize}
\begin{itemize}
\item {Utilização:Prov.}
\end{itemize}
\begin{itemize}
\item {Utilização:trasm.}
\end{itemize}
O mesmo que \textunderscore amolgar\textunderscore .
\section{Amoral}
\begin{itemize}
\item {Grp. gram.:adj.}
\end{itemize}
\begin{itemize}
\item {Utilização:Neol.}
\end{itemize}
\begin{itemize}
\item {Proveniência:(Do gr. \textunderscore a\textunderscore  priv. + \textunderscore moral\textunderscore )}
\end{itemize}
O mesmo que \textunderscore immoral\textunderscore .
Que não reconhece lei moral.
\section{Ampério-hora}
\begin{itemize}
\item {Grp. gram.:m.}
\end{itemize}
\begin{itemize}
\item {Utilização:Phýs.}
\end{itemize}
Corrente eléctrica de um ampério numa hora.
\section{Amphiarthrose}
\begin{itemize}
\item {Grp. gram.:f.}
\end{itemize}
\begin{itemize}
\item {Utilização:Anat.}
\end{itemize}
\begin{itemize}
\item {Proveniência:(Do gr. \textunderscore amphi\textunderscore  + \textunderscore arthrosis\textunderscore )}
\end{itemize}
União íntima de duas superfícies articulares, por meio de um corpo fibro-cartilaginoso, simples e elástico.
\section{Amphiboloxisto}
\begin{itemize}
\item {Grp. gram.:m.}
\end{itemize}
\begin{itemize}
\item {Utilização:Miner.}
\end{itemize}
Rocha primitiva, em que predomina o quartzo e a horneblanda.
\section{Amyelia}
\begin{itemize}
\item {Grp. gram.:f.}
\end{itemize}
\begin{itemize}
\item {Utilização:Med.}
\end{itemize}
\begin{itemize}
\item {Proveniência:(Do gr. \textunderscore a\textunderscore  priv. + \textunderscore muelos\textunderscore )}
\end{itemize}
Monstruosidade, caracterizada pela ausência de medulla espinhal.
\section{Amygdalotomia}
\begin{itemize}
\item {Grp. gram.:f.}
\end{itemize}
Emprêgo cirúrgico do amygdalótomo.
\section{Amyotaxia}
\begin{itemize}
\item {fónica:csi}
\end{itemize}
\begin{itemize}
\item {Grp. gram.:f.}
\end{itemize}
\begin{itemize}
\item {Utilização:Med.}
\end{itemize}
\begin{itemize}
\item {Proveniência:(Do gr. \textunderscore a\textunderscore  priv. + \textunderscore mus\textunderscore  + \textunderscore taxis\textunderscore )}
\end{itemize}
Convulsões involuntárias, de origem reflexa, determinadas muitas vezes por vários neurites.
\section{Amyotrophia}
\begin{itemize}
\item {Grp. gram.:f.}
\end{itemize}
\begin{itemize}
\item {Utilização:Med.}
\end{itemize}
\begin{itemize}
\item {Proveniência:(Do gr. \textunderscore a\textunderscore  priv. + \textunderscore mus\textunderscore  + \textunderscore trophe\textunderscore )}
\end{itemize}
Atrophia dos músculos.
\section{Amyxia}
\begin{itemize}
\item {fónica:csi}
\end{itemize}
\begin{itemize}
\item {Grp. gram.:f.}
\end{itemize}
\begin{itemize}
\item {Utilização:Med.}
\end{itemize}
\begin{itemize}
\item {Proveniência:(Do gr. \textunderscore a\textunderscore  priv. + \textunderscore muxa\textunderscore )}
\end{itemize}
Ausência de secreção do muco normal.
\section{Anagênese}
\begin{itemize}
\item {Grp. gram.:f.}
\end{itemize}
\begin{itemize}
\item {Utilização:Med.}
\end{itemize}
\begin{itemize}
\item {Proveniência:(Do gr. \textunderscore ana\textunderscore  + \textunderscore genesis\textunderscore )}
\end{itemize}
Regeneração de partes destruídas.
\section{Analcite}
\begin{itemize}
\item {Grp. gram.:f.}
\end{itemize}
Espécie de xeólitho sódico-cálcico que apparece nos tufos melaphýricos.
\section{Anarthria}
\begin{itemize}
\item {Grp. gram.:f.}
\end{itemize}
\begin{itemize}
\item {Utilização:Med.}
\end{itemize}
\begin{itemize}
\item {Proveniência:(Do gr. \textunderscore an\textunderscore  priv. + \textunderscore arthron\textunderscore )}
\end{itemize}
Impossibilidade de articular palavras, em consequência da paralysia de certos músculos.
\section{Anartria}
\begin{itemize}
\item {Grp. gram.:f.}
\end{itemize}
\begin{itemize}
\item {Utilização:Med.}
\end{itemize}
\begin{itemize}
\item {Proveniência:(Do gr. \textunderscore an\textunderscore  priv. + \textunderscore arthron\textunderscore )}
\end{itemize}
Impossibilidade de articular palavras, em consequência da paralisia de certos músculos.
\section{Ancilotia}
\begin{itemize}
\item {Grp. gram.:f.}
\end{itemize}
\begin{itemize}
\item {Utilização:Med.}
\end{itemize}
\begin{itemize}
\item {Proveniência:(Do gr. \textunderscore ankule\textunderscore  + \textunderscore ous\textunderscore , \textunderscore otos\textunderscore )}
\end{itemize}
Aderência das paredes do conducto auditivo.
\section{Ancylotia}
\begin{itemize}
\item {Grp. gram.:f.}
\end{itemize}
\begin{itemize}
\item {Utilização:Med.}
\end{itemize}
\begin{itemize}
\item {Proveniência:(Do gr. \textunderscore ankule\textunderscore  + \textunderscore ous\textunderscore , \textunderscore otos\textunderscore )}
\end{itemize}
Adherência das paredes do conducto auditivo.
\section{Andesite}
\begin{itemize}
\item {Grp. gram.:f.}
\end{itemize}
O mesmo que \textunderscore andesina\textunderscore .
\section{Andesito}
\begin{itemize}
\item {Grp. gram.:m.}
\end{itemize}
O mesmo que \textunderscore andesina\textunderscore .
\section{Andradite}
\begin{itemize}
\item {Grp. gram.:f.}
\end{itemize}
Variedade de granada (pedra fina).
\section{Andradito}
\begin{itemize}
\item {Grp. gram.:m.}
\end{itemize}
Variedade de granada (pedra fina).
\section{Andranatomia}
\begin{itemize}
\item {Grp. gram.:f.}
\end{itemize}
\begin{itemize}
\item {Proveniência:(Do gr. \textunderscore aner\textunderscore  + \textunderscore anatome\textunderscore )}
\end{itemize}
Anatomia do homem.
\section{Anêspera}
\begin{itemize}
\item {Grp. gram.:f.}
\end{itemize}
\begin{itemize}
\item {Utilização:ant.}
\end{itemize}
\begin{itemize}
\item {Utilização:Pop.}
\end{itemize}
O mesmo que \textunderscore nêspera\textunderscore . Cf. B. Pereira, \textunderscore Prosodia\textunderscore , vb. \textunderscore pytmena\textunderscore .
\section{Anfiartrose}
\begin{itemize}
\item {Grp. gram.:f.}
\end{itemize}
\begin{itemize}
\item {Utilização:Anat.}
\end{itemize}
\begin{itemize}
\item {Proveniência:(Do gr. \textunderscore amphi\textunderscore  + \textunderscore arthrosis\textunderscore )}
\end{itemize}
União íntima de duas superfícies articulares, por meio de um corpo fibro-cartilaginoso, simples e elástico.
\section{Anfiboloxisto}
\begin{itemize}
\item {Grp. gram.:m.}
\end{itemize}
\begin{itemize}
\item {Utilização:Miner.}
\end{itemize}
Rocha primitiva, em que predomina o quartzo e a horneblanda.
\section{Angialgia}
\begin{itemize}
\item {Grp. gram.:f.}
\end{itemize}
\begin{itemize}
\item {Utilização:Med.}
\end{itemize}
\begin{itemize}
\item {Proveniência:(Do gr. \textunderscore angeion\textunderscore  + \textunderscore algos\textunderscore )}
\end{itemize}
Dôr no trajecto de um vaso, sem lesão apreciável dêste.
\section{Angiectopia}
\begin{itemize}
\item {Grp. gram.:f.}
\end{itemize}
\begin{itemize}
\item {Utilização:Med.}
\end{itemize}
\begin{itemize}
\item {Proveniência:(Do gr. \textunderscore angeion\textunderscore  + \textunderscore ek\textunderscore  + \textunderscore topos\textunderscore )}
\end{itemize}
Situação anómala de um váso sanguíneo.
\section{Angiite}
\begin{itemize}
\item {Grp. gram.:f.}
\end{itemize}
\begin{itemize}
\item {Utilização:Med.}
\end{itemize}
\begin{itemize}
\item {Proveniência:(Do gr. \textunderscore angeion\textunderscore )}
\end{itemize}
Inflammação de um vaso.
\section{Anexite}
\begin{itemize}
\item {fónica:csi}
\end{itemize}
\begin{itemize}
\item {Grp. gram.:f.}
\end{itemize}
\begin{itemize}
\item {Utilização:Med.}
\end{itemize}
Inflamação dos anexos do útero, (trompas e ovários).
\section{Angiomatose}
\begin{itemize}
\item {Grp. gram.:f.}
\end{itemize}
\begin{itemize}
\item {Utilização:Med.}
\end{itemize}
Apparecimento de angiomas cutâneos multiplos.
\section{Angiorrafia}
\begin{itemize}
\item {Grp. gram.:f.}
\end{itemize}
\begin{itemize}
\item {Utilização:Med.}
\end{itemize}
\begin{itemize}
\item {Proveniência:(Do gr. \textunderscore angeion\textunderscore  + \textunderscore raphe\textunderscore )}
\end{itemize}
Sutura ou anastomose dos vasos sanguíneos.
\section{Angiorraphia}
\begin{itemize}
\item {Grp. gram.:f.}
\end{itemize}
\begin{itemize}
\item {Utilização:Med.}
\end{itemize}
\begin{itemize}
\item {Proveniência:(Do gr. \textunderscore angeion\textunderscore  + \textunderscore raphe\textunderscore )}
\end{itemize}
Sutura ou anastomose dos vasos sanguíneos.
\section{Angiosclerose}
\begin{itemize}
\item {Grp. gram.:f.}
\end{itemize}
\begin{itemize}
\item {Utilização:Med.}
\end{itemize}
\begin{itemize}
\item {Proveniência:(Do gr. \textunderscore angeion\textunderscore  + \textunderscore skleros\textunderscore )}
\end{itemize}
Esclerose das paredes vasculares.
\section{Angiótribo}
\begin{itemize}
\item {Grp. gram.:m.}
\end{itemize}
\begin{itemize}
\item {Utilização:Med.}
\end{itemize}
\begin{itemize}
\item {Proveniência:(Do gr. \textunderscore angeion\textunderscore  + \textunderscore tribein\textunderscore )}
\end{itemize}
Apparelho, ou pinça forte, com que se faz o esmagamento dos vasos sanguíneos, para se obter a hemóstase.
\section{Angiotripsia}
\begin{itemize}
\item {Grp. gram.:f.}
\end{itemize}
\begin{itemize}
\item {Utilização:Med.}
\end{itemize}
Emprêgo do angiótribo.
\section{Angosto}
\begin{itemize}
\item {fónica:gôs}
\end{itemize}
\begin{itemize}
\item {Grp. gram.:adj.}
\end{itemize}
\begin{itemize}
\item {Utilização:Ant.}
\end{itemize}
O mesmo que \textunderscore angusto\textunderscore . Cf. \textunderscore Rot. do Mar-Vermelho\textunderscore , 40.
\section{Angular}
\begin{itemize}
\item {Grp. gram.:v. i.}
\end{itemize}
\begin{itemize}
\item {Utilização:Fam.}
\end{itemize}
Andar, formando ângulo com uma linha, uma rua, um objecto: \textunderscore ia seguindo a carruagem, mas depois angulei para a esquerda\textunderscore .
\section{Anhidrose}
\begin{itemize}
\item {fónica:ni}
\end{itemize}
\begin{itemize}
\item {Grp. gram.:f.}
\end{itemize}
\begin{itemize}
\item {Utilização:Med.}
\end{itemize}
\begin{itemize}
\item {Proveniência:(Do gr. \textunderscore an\textunderscore  priv. + \textunderscore hidrosis\textunderscore )}
\end{itemize}
Falta ou deminuição de suor.
\section{Anisochromático}
\begin{itemize}
\item {Grp. gram.:adj.}
\end{itemize}
\begin{itemize}
\item {Utilização:Med.}
\end{itemize}
Em que há anisochromia.
\section{Anisochromia}
\begin{itemize}
\item {Grp. gram.:f.}
\end{itemize}
\begin{itemize}
\item {Utilização:Med.}
\end{itemize}
\begin{itemize}
\item {Proveniência:(Do gr. \textunderscore an\textunderscore  priv. + \textunderscore isos\textunderscore  + \textunderscore khroma\textunderscore )}
\end{itemize}
Desigualdade de coloração nos glóbulos rubros do sangue.
\section{Anisocromático}
\begin{itemize}
\item {Grp. gram.:adj.}
\end{itemize}
\begin{itemize}
\item {Utilização:Med.}
\end{itemize}
Em que há anisocromia.
\section{Anisocromia}
\begin{itemize}
\item {Grp. gram.:f.}
\end{itemize}
\begin{itemize}
\item {Utilização:Med.}
\end{itemize}
\begin{itemize}
\item {Proveniência:(Do gr. \textunderscore an\textunderscore  priv. + \textunderscore isos\textunderscore  + \textunderscore khroma\textunderscore )}
\end{itemize}
Desigualdade de coloração nos glóbulos rubros do sangue.
\section{Anisopia}
\begin{itemize}
\item {Grp. gram.:f.}
\end{itemize}
\begin{itemize}
\item {Utilização:Med.}
\end{itemize}
\begin{itemize}
\item {Proveniência:(Do gr. \textunderscore an\textunderscore  priv. + \textunderscore isos\textunderscore  + \textunderscore ops\textunderscore )}
\end{itemize}
Desigualdade da acuidade visual dos dois olhos.
\section{Anisuria}
\begin{itemize}
\item {Grp. gram.:f.}
\end{itemize}
\begin{itemize}
\item {Utilização:Med.}
\end{itemize}
\begin{itemize}
\item {Proveniência:(Do gr. \textunderscore an\textunderscore  priv. + \textunderscore isos\textunderscore  + \textunderscore ouron\textunderscore )}
\end{itemize}
Desigualdade nos intervallos da emissão da urina.
\section{Anjão}
\begin{itemize}
\item {Grp. gram.:m.}
\end{itemize}
\begin{itemize}
\item {Utilização:Pop.}
\end{itemize}
Figura desmarcada, representando um anjo. Cf. R. Jorge, \textunderscore El Greco\textunderscore , 47.
\section{Annexite}
\begin{itemize}
\item {fónica:csi}
\end{itemize}
\begin{itemize}
\item {Grp. gram.:f.}
\end{itemize}
\begin{itemize}
\item {Utilização:Med.}
\end{itemize}
Inflammação dos annexos do útero, (trompas e ovários).
\section{Anomochromia}
\begin{itemize}
\item {Grp. gram.:f.}
\end{itemize}
\begin{itemize}
\item {Utilização:Med.}
\end{itemize}
\begin{itemize}
\item {Proveniência:(Do gr. \textunderscore anomos\textunderscore  + \textunderscore khromos\textunderscore )}
\end{itemize}
Desigualdade na coloração da pelle.
\section{Anomocromia}
\begin{itemize}
\item {Grp. gram.:f.}
\end{itemize}
\begin{itemize}
\item {Utilização:Med.}
\end{itemize}
\begin{itemize}
\item {Proveniência:(Do gr. \textunderscore anomos\textunderscore  + \textunderscore khromos\textunderscore )}
\end{itemize}
Desigualdade na coloração da pele.
\section{Anoneira}
\begin{itemize}
\item {Grp. gram.:f.}
\end{itemize}
O mesmo que \textunderscore anona\textunderscore ^1.
\section{Anoopsia}
\begin{itemize}
\item {Grp. gram.:f.}
\end{itemize}
\begin{itemize}
\item {Utilização:Med.}
\end{itemize}
\begin{itemize}
\item {Proveniência:(Do gr. \textunderscore ana\textunderscore  + \textunderscore opsis\textunderscore )}
\end{itemize}
Estrabismo, em que o ôlho se volta para cima.
\section{Anorchia}
\begin{itemize}
\item {fónica:qui}
\end{itemize}
\begin{itemize}
\item {Grp. gram.:f.}
\end{itemize}
\begin{itemize}
\item {Utilização:Med.}
\end{itemize}
\begin{itemize}
\item {Proveniência:(Do gr. \textunderscore an\textunderscore  priv. + \textunderscore orkhia\textunderscore )}
\end{itemize}
Ausência congênita dos testículos.
\section{Anorquia}
\begin{itemize}
\item {Grp. gram.:f.}
\end{itemize}
\begin{itemize}
\item {Utilização:Med.}
\end{itemize}
\begin{itemize}
\item {Proveniência:(Do gr. \textunderscore an\textunderscore  priv. + \textunderscore orkhia\textunderscore )}
\end{itemize}
Ausência congênita dos testículos.
\section{Anórthico}
\begin{itemize}
\item {Grp. gram.:adj.}
\end{itemize}
Relativo á anorthita.
\section{Anórtico}
\begin{itemize}
\item {Grp. gram.:adj.}
\end{itemize}
Relativo á anortita.
\section{Anoto}
\begin{itemize}
\item {Grp. gram.:adj.}
\end{itemize}
\begin{itemize}
\item {Utilização:Zool.}
\end{itemize}
\begin{itemize}
\item {Proveniência:(Do gr. \textunderscore an\textunderscore  priv. + \textunderscore otos\textunderscore )}
\end{itemize}
Que não tem orelhas.
\section{Anserino}
\begin{itemize}
\item {Grp. gram.:adj.}
\end{itemize}
\begin{itemize}
\item {Utilização:Med.}
\end{itemize}
\begin{itemize}
\item {Proveniência:(Lat. \textunderscore anserinus\textunderscore )}
\end{itemize}
Relativo a ganso.
Diz-se da pelle em estado de horripilação.
\section{Anteconjugal}
\begin{itemize}
\item {Grp. gram.:adj.}
\end{itemize}
O mesmo que \textunderscore antenupcial\textunderscore .
\section{Antemolar}
\begin{itemize}
\item {Grp. gram.:adj.}
\end{itemize}
Diz-se dos dentes molares anteriores.
\section{Antes-de-ontem}
\begin{itemize}
\item {Grp. gram.:adj.}
\end{itemize}
\begin{itemize}
\item {Utilização:Prov.}
\end{itemize}
O mesmo que \textunderscore anteontem\textunderscore .
\section{Antetónico}
\begin{itemize}
\item {Grp. gram.:adj.}
\end{itemize}
\begin{itemize}
\item {Utilização:Gram.}
\end{itemize}
O mesmo que \textunderscore pretónico\textunderscore .
\section{Antevém}
\begin{itemize}
\item {Grp. gram.:m.}
\end{itemize}
\begin{itemize}
\item {Utilização:T. de Amarante}
\end{itemize}
\begin{itemize}
\item {Proveniência:(De \textunderscore ante...\textunderscore  + \textunderscore vir\textunderscore )}
\end{itemize}
Refeição, antes do jantar; lanche; piqueta.
\section{Anthropomorphia}
\begin{itemize}
\item {Grp. gram.:f.}
\end{itemize}
Qualidade de anthropomorpho.
\section{Antibérico}
\begin{itemize}
\item {Grp. gram.:adj.}
\end{itemize}
Opposto ao iberismo.
\section{Antiibérico}
\begin{itemize}
\item {Grp. gram.:adj.}
\end{itemize}
Opposto ao iberismo.
\section{Antropomorfia}
\begin{itemize}
\item {Grp. gram.:f.}
\end{itemize}
Qualidade de antropomorfo.
\section{Afanítico}
\begin{itemize}
\item {Grp. gram.:adj.}
\end{itemize}
Relativo á afanita.
\section{Afrónitro}
\begin{itemize}
\item {Grp. gram.:m.}
\end{itemize}
\begin{itemize}
\item {Utilização:Des.}
\end{itemize}
\begin{itemize}
\item {Proveniência:(Lat. \textunderscore aphronitrum\textunderscore )}
\end{itemize}
Flôr ou escuma de nitro.
\section{Antiberismo}
\begin{itemize}
\item {Grp. gram.:m.}
\end{itemize}
Partido ou systema, contrário á união ibérica.
\section{Antiberista}
\begin{itemize}
\item {Grp. gram.:m.}
\end{itemize}
Partidário do antiberismo.
\section{Antidínico}
\begin{itemize}
\item {Grp. gram.:adj.}
\end{itemize}
\begin{itemize}
\item {Utilização:Med.}
\end{itemize}
\begin{itemize}
\item {Proveniência:(Do gr. \textunderscore anti\textunderscore  + \textunderscore dinos\textunderscore )}
\end{itemize}
Que é applicável contra as vertigens.
\section{Antiiberismo}
\begin{itemize}
\item {Grp. gram.:m.}
\end{itemize}
Partido ou systema, contrário á união ibérica.
\section{Antiiberista}
\begin{itemize}
\item {Grp. gram.:m.}
\end{itemize}
Partidário do antiiberismo.
\section{Antipodismo}
\begin{itemize}
\item {Grp. gram.:m.}
\end{itemize}
\begin{itemize}
\item {Utilização:Neol.}
\end{itemize}
Qualidade de antípoda.
\section{Antisezonático}
\begin{itemize}
\item {fónica:se}
\end{itemize}
\begin{itemize}
\item {Grp. gram.:adj.}
\end{itemize}
\begin{itemize}
\item {Proveniência:(De \textunderscore anti...\textunderscore  + \textunderscore sezonático\textunderscore )}
\end{itemize}
Opposto ao sezonismo:«\textunderscore commissão de defesa antisezonática...\textunderscore »\textunderscore Diário-do Gov.\textunderscore , 29-V-911.
\section{Antissezonático}
\begin{itemize}
\item {Grp. gram.:adj.}
\end{itemize}
\begin{itemize}
\item {Proveniência:(De \textunderscore anti...\textunderscore  + \textunderscore sezonático\textunderscore )}
\end{itemize}
Opposto ao sezonismo:«\textunderscore commissão de defesa antissezonática...\textunderscore »\textunderscore Diário-do Gov.\textunderscore , 29-V-911.
\section{Apanado}
\begin{itemize}
\item {Grp. gram.:adj.}
\end{itemize}
\begin{itemize}
\item {Utilização:Ant.}
\end{itemize}
\begin{itemize}
\item {Proveniência:(Do lat. \textunderscore panis\textunderscore )}
\end{itemize}
Dizia-se dos terrenos, applicados á cultura de cereaes.
\section{Aparvado}
\begin{itemize}
\item {Grp. gram.:adj.}
\end{itemize}
O mesmo que \textunderscore aparvalhado\textunderscore . Cf. Camillo, \textunderscore Ôlho de Vidro\textunderscore , 204.
\section{Apascento}
\begin{itemize}
\item {Grp. gram.:m.}
\end{itemize}
\begin{itemize}
\item {Utilização:P. us.}
\end{itemize}
O mesmo que \textunderscore apascentamento\textunderscore .
\section{Apatalado}
\begin{itemize}
\item {Grp. gram.:adj.}
\end{itemize}
\begin{itemize}
\item {Utilização:Prov.}
\end{itemize}
\begin{itemize}
\item {Proveniência:(De \textunderscore pata\textunderscore ?)}
\end{itemize}
Diz-se do feijão branco, largo, espalmado.
\section{Apeaça}
\begin{itemize}
\item {Grp. gram.:f.}
\end{itemize}
\begin{itemize}
\item {Utilização:Prov.}
\end{itemize}
O mesmo que \textunderscore peaça\textunderscore .
\section{Apendicalgia}
\begin{itemize}
\item {Grp. gram.:f.}
\end{itemize}
\begin{itemize}
\item {Utilização:Med.}
\end{itemize}
\begin{itemize}
\item {Proveniência:(De \textunderscore apêndice\textunderscore  + gr. \textunderscore algos\textunderscore )}
\end{itemize}
Dôr, na região ileocecal, não sintomática de inflamação.
\section{Apendicectomia}
\begin{itemize}
\item {Grp. gram.:f.}
\end{itemize}
\begin{itemize}
\item {Utilização:Med.}
\end{itemize}
\begin{itemize}
\item {Proveniência:(De \textunderscore apêndice\textunderscore  + gr. \textunderscore ektome\textunderscore )}
\end{itemize}
Ablação do apêndice ileocecal.
\section{Apendicismo}
\begin{itemize}
\item {Grp. gram.:m.}
\end{itemize}
\begin{itemize}
\item {Utilização:Med.}
\end{itemize}
Inflamação dos tecidos, vizinhos do apêndice ileocecal.
\section{Apendicocele}
\begin{itemize}
\item {Grp. gram.:m.}
\end{itemize}
\begin{itemize}
\item {Utilização:Med.}
\end{itemize}
\begin{itemize}
\item {Proveniência:(De \textunderscore apêndice\textunderscore  + gr. \textunderscore kele\textunderscore )}
\end{itemize}
Hérnia do apêndice ileocecal.
\section{Aphanítico}
\begin{itemize}
\item {Grp. gram.:adj.}
\end{itemize}
Relativo á aphanita.
\section{Aphrónitro}
\begin{itemize}
\item {Grp. gram.:m.}
\end{itemize}
\begin{itemize}
\item {Utilização:Des.}
\end{itemize}
\begin{itemize}
\item {Proveniência:(Lat. \textunderscore aphronitrum\textunderscore )}
\end{itemize}
Flôr ou escuma de nitro.
\section{Apisteiro}
\begin{itemize}
\item {Grp. gram.:m.}
\end{itemize}
\begin{itemize}
\item {Proveniência:(De \textunderscore apisto\textunderscore )}
\end{itemize}
Pequeno vaso, que termina em bico, e por onde bebem os doentes, deitados.
Pequeno vaso análogo, que termina superiormente numa têta artificial de borracha, e com que se aleitam artificialmente as crianças; mamadeira; biberão.
\section{Apneumia}
\begin{itemize}
\item {Grp. gram.:f.}
\end{itemize}
\begin{itemize}
\item {Utilização:Terat.}
\end{itemize}
\begin{itemize}
\item {Proveniência:(Do gr. \textunderscore a\textunderscore  priv. + \textunderscore pneumon\textunderscore )}
\end{itemize}
Ausência do pulmão.
\section{Apofilito}
\begin{itemize}
\item {Grp. gram.:f.}
\end{itemize}
\begin{itemize}
\item {Utilização:Miner.}
\end{itemize}
Zeólito cálcico-potássico.
\section{Apophylite}
\begin{itemize}
\item {Grp. gram.:f.}
\end{itemize}
Zeólitho cálcico-potássico.
\section{Apophylito}
\begin{itemize}
\item {Grp. gram.:f.}
\end{itemize}
\begin{itemize}
\item {Utilização:Miner.}
\end{itemize}
Zeólitho cálcico-potássico.
\section{Apotecar}
\begin{itemize}
\item {Grp. gram.:v. t.}
\end{itemize}
O mesmo que \textunderscore hipotecar\textunderscore . Cf. B. Pereira, \textunderscore Prosodia\textunderscore , vb. \textunderscore pressor\textunderscore .
\section{Apothecar}
\begin{itemize}
\item {Grp. gram.:v. t.}
\end{itemize}
O mesmo que \textunderscore hypothecar\textunderscore . Cf. B. Pereira, \textunderscore Prosodia\textunderscore , vb. \textunderscore pressor\textunderscore .
\section{Appendicalgia}
\begin{itemize}
\item {Grp. gram.:f.}
\end{itemize}
\begin{itemize}
\item {Utilização:Med.}
\end{itemize}
\begin{itemize}
\item {Proveniência:(De \textunderscore appêndice\textunderscore  + gr. \textunderscore algos\textunderscore )}
\end{itemize}
Dôr, na região ileocecal, não symptomática de inflammação.
\section{Appendicectomia}
\begin{itemize}
\item {Grp. gram.:f.}
\end{itemize}
\begin{itemize}
\item {Utilização:Med.}
\end{itemize}
\begin{itemize}
\item {Proveniência:(De \textunderscore appêndice\textunderscore  + gr. \textunderscore ektome\textunderscore )}
\end{itemize}
Ablação do appêndice ileocecal.
\section{Appendicismo}
\begin{itemize}
\item {Grp. gram.:m.}
\end{itemize}
\begin{itemize}
\item {Utilização:Med.}
\end{itemize}
Inflammação dos tecidos, vizinhos do appêndice ileocecal.
\section{Appendicocele}
\begin{itemize}
\item {Grp. gram.:m.}
\end{itemize}
\begin{itemize}
\item {Utilização:Med.}
\end{itemize}
\begin{itemize}
\item {Proveniência:(De \textunderscore appêndice\textunderscore  + gr. \textunderscore kele\textunderscore )}
\end{itemize}
Hérnia do appêndice ileocecal.
\section{Apremiar}
\begin{itemize}
\item {Grp. gram.:v. t.}
\end{itemize}
\begin{itemize}
\item {Utilização:Ant.}
\end{itemize}
O mesmo que \textunderscore apremar\textunderscore . Cf. B. Pereira, \textunderscore Prosodia\textunderscore , vb. \textunderscore munio\textunderscore , \textunderscore Torquatus\textunderscore , etc.
\section{Apriorismo}
\begin{itemize}
\item {Grp. gram.:m.}
\end{itemize}
Systema dos que argumentam \textunderscore a priori\textunderscore , por hypóthese, independentemente dos factos.
(Da loc. lat. \textunderscore a priori\textunderscore )
\section{Apsiquia}
\begin{itemize}
\item {Grp. gram.:f.}
\end{itemize}
\begin{itemize}
\item {Utilização:Med.}
\end{itemize}
\begin{itemize}
\item {Proveniência:(Do gr. \textunderscore a\textunderscore  priv. + \textunderscore psukhe\textunderscore )}
\end{itemize}
Perda dos sentidos.
\section{Apsychia}
\begin{itemize}
\item {fónica:qui}
\end{itemize}
\begin{itemize}
\item {Grp. gram.:f.}
\end{itemize}
\begin{itemize}
\item {Utilização:Med.}
\end{itemize}
\begin{itemize}
\item {Proveniência:(Do gr. \textunderscore a\textunderscore  priv. + \textunderscore psukhe\textunderscore )}
\end{itemize}
Perda dos sentidos.
\section{Arábio}
\begin{itemize}
\item {Grp. gram.:m.  e  adj.}
\end{itemize}
O mesmo que \textunderscore arábico\textunderscore . Cf. Góes, \textunderscore João III\textunderscore , 30, (ed. de Coímbra).
\section{Aracanguira}
\begin{itemize}
\item {Grp. gram.:f.}
\end{itemize}
\begin{itemize}
\item {Utilização:Bras. do N}
\end{itemize}
Nome vulgar de um peixe.
\section{Aramaçan}
\begin{itemize}
\item {Grp. gram.:m.}
\end{itemize}
\begin{itemize}
\item {Utilização:Bras. do N}
\end{itemize}
Nome vulgar de um peixe.
\section{Arauaná}
\begin{itemize}
\item {Grp. gram.:m.}
\end{itemize}
Espécie de peixe do Amazonas.
\section{Arboreto}
\begin{itemize}
\item {fónica:borê}
\end{itemize}
\begin{itemize}
\item {Grp. gram.:m.}
\end{itemize}
\begin{itemize}
\item {Utilização:Neol.}
\end{itemize}
\begin{itemize}
\item {Proveniência:(Lat. \textunderscore arboretum\textunderscore )}
\end{itemize}
Mata; floresta; agrupamento de vegetaes lenhosos: \textunderscore o arboreto da Pena, em Sintra\textunderscore .
\section{Archimoquenqueiro}
\begin{itemize}
\item {fónica:qui}
\end{itemize}
\begin{itemize}
\item {Grp. gram.:adj.}
\end{itemize}
Que é muitíssimo moquenco. Cf. Filinto, \textunderscore Fáb. de Lafont.\textunderscore , 160.
\section{Areeira}
\begin{itemize}
\item {Grp. gram.:f.}
\end{itemize}
\begin{itemize}
\item {Utilização:Des.}
\end{itemize}
O mesmo que \textunderscore areeiro\textunderscore . Cf. \textunderscore Rot. do Mar-Verm.\textunderscore , 164.
\section{Arenata}
\begin{itemize}
\item {Grp. gram.:f.}
\end{itemize}
\begin{itemize}
\item {Proveniência:(Lat. \textunderscore arenata\textunderscore )}
\end{itemize}
Terreno areento e estéril.
\section{Arenisco}
\begin{itemize}
\item {Grp. gram.:adj.}
\end{itemize}
\begin{itemize}
\item {Proveniência:(Do lat. \textunderscore arena\textunderscore )}
\end{itemize}
Arenoso, areento.
\section{Areocele}
\begin{itemize}
\item {Grp. gram.:m.}
\end{itemize}
\begin{itemize}
\item {Utilização:Med.}
\end{itemize}
\begin{itemize}
\item {Proveniência:(Do gr. \textunderscore araios\textunderscore  + \textunderscore kele\textunderscore )}
\end{itemize}
Tumor gasoso do pescoço.
\section{Argençana-dos-pastores}
\begin{itemize}
\item {Grp. gram.:f.}
\end{itemize}
\begin{itemize}
\item {Utilização:Bot.}
\end{itemize}
O mesmo que \textunderscore genciana\textunderscore . Cf. P. Coutinho, \textunderscore Flora\textunderscore , 484.
\section{Argirismo}
\begin{itemize}
\item {Grp. gram.:m.}
\end{itemize}
\begin{itemize}
\item {Utilização:Med.}
\end{itemize}
Conjunto dos fenómenos tóxicos, provocados pelo emprêgo prolongado dos saes de prata, como é a dispneia, a hidropsia, a argiria sobretudo.
(Cp. \textunderscore argiria\textunderscore )
\section{Argyrismo}
\begin{itemize}
\item {Grp. gram.:m.}
\end{itemize}
\begin{itemize}
\item {Utilização:Med.}
\end{itemize}
Conjunto dos phenómenos tóxicos, provocados pelo emprêgo prolongado dos saes de prata, como é a dispneia, a hydropsia, a argyria sobretudo.
(Cp. \textunderscore argyria\textunderscore )
\section{Arnelha}
\begin{itemize}
\item {fónica:nê}
\end{itemize}
\begin{itemize}
\item {Grp. gram.:f.}
\end{itemize}
O mesmo que \textunderscore ranilha\textunderscore . (Colhido em Alcanena)
\section{Arnilha}
\begin{itemize}
\item {Grp. gram.:f.}
\end{itemize}
\begin{itemize}
\item {Utilização:Prov.}
\end{itemize}
O mesmo que \textunderscore ranilha\textunderscore . (Colhido em Alcanena)
\section{Arquimoquenqueiro}
\begin{itemize}
\item {Grp. gram.:adj.}
\end{itemize}
Que é muitíssimo moquenco. Cf. Filinto, \textunderscore Fáb. de Lafont.\textunderscore , 160.
\section{Arrabaça}
\begin{itemize}
\item {Grp. gram.:f.}
\end{itemize}
\begin{itemize}
\item {Utilização:ant.}
\end{itemize}
\begin{itemize}
\item {Utilização:Pop.}
\end{itemize}
Planta, o mesmo que \textunderscore rabaça\textunderscore .
\section{Arranha}
\begin{itemize}
\item {Grp. gram.:f.}
\end{itemize}
O mesmo que \textunderscore ranha\textunderscore . Cf. \textunderscore Port. Ant. e Mod.\textunderscore , VII, 197.
\section{Arraposar}
\begin{itemize}
\item {Grp. gram.:v. i.}
\end{itemize}
\begin{itemize}
\item {Utilização:Prov.}
\end{itemize}
\begin{itemize}
\item {Utilização:trasm.}
\end{itemize}
Cabular; fazer parede (o estudante).
\section{Arrastadiço}
\begin{itemize}
\item {Grp. gram.:adj.}
\end{itemize}
Próprio para se arrastar. Cf. B. Pereira, \textunderscore Prosodia\textunderscore , vb. \textunderscore tracticius\textunderscore .
\section{Arreatada}
\begin{itemize}
\item {Grp. gram.:f.}
\end{itemize}
Pancada com arreata.
\section{Arrepicar}
\begin{itemize}
\item {Grp. gram.:v. t.}
\end{itemize}
O mesmo que \textunderscore repicar\textunderscore . Cf. Pant. de Aveiro, \textunderscore Itiner.\textunderscore , 295, (2.^a ed.).
\section{Arritmia}
\begin{itemize}
\item {Grp. gram.:f.}
\end{itemize}
\begin{itemize}
\item {Utilização:Med.}
\end{itemize}
\begin{itemize}
\item {Proveniência:(Do gr. \textunderscore a\textunderscore  priv. + \textunderscore ruthmos\textunderscore )}
\end{itemize}
Irregularidade e desigualdade das contracções do coração.
\section{Arroaz}
\begin{itemize}
\item {Grp. gram.:m.}
\end{itemize}
Peixe, o mesmo que \textunderscore roaz\textunderscore .
\section{Arroz-dos-telhados}
\begin{itemize}
\item {Grp. gram.:m.}
\end{itemize}
\begin{itemize}
\item {Utilização:Bot.}
\end{itemize}
Planta crassulácea, o mesmo que \textunderscore pinhões-de-rato\textunderscore .
\section{Arruda-gallega}
\begin{itemize}
\item {Grp. gram.:f.}
\end{itemize}
Planta, variedade de arruda. Cf. B. Pereira, \textunderscore Prosodia\textunderscore , vb. \textunderscore lopta\textunderscore .
\section{Arruir}
\begin{itemize}
\item {Grp. gram.:v. i.}
\end{itemize}
O mesmo que \textunderscore ruir\textunderscore ; desmoronar-se.
\section{Arsonvalização}
\begin{itemize}
\item {Grp. gram.:f.}
\end{itemize}
\begin{itemize}
\item {Utilização:Med.}
\end{itemize}
\begin{itemize}
\item {Proveniência:(De \textunderscore Arsonval\textunderscore , n. p.)}
\end{itemize}
Applicação terapêutica das correntes eléctricas de alta frequência.
\section{Artrografia}
\begin{itemize}
\item {Grp. gram.:f.}
\end{itemize}
\begin{itemize}
\item {Proveniência:(Do gr. \textunderscore arthron\textunderscore  + \textunderscore graphein\textunderscore )}
\end{itemize}
Descripção das articulações do organismo animal.
\section{Arthrographia}
\begin{itemize}
\item {Grp. gram.:f.}
\end{itemize}
\begin{itemize}
\item {Proveniência:(Do gr. \textunderscore arthron\textunderscore  + \textunderscore graphein\textunderscore )}
\end{itemize}
Descripção das articulações do organismo animal.
\section{Arvela}
\begin{itemize}
\item {Grp. gram.:f.}
\end{itemize}
\begin{itemize}
\item {Utilização:Prov.}
\end{itemize}
Pessôa fraca.
\section{Arvorinho}
\begin{itemize}
\item {Grp. gram.:m.}
\end{itemize}
\begin{itemize}
\item {Proveniência:(De \textunderscore árvore\textunderscore )}
\end{itemize}
O mesmo que \textunderscore arboreto\textunderscore . (Us. por Brotero)
\section{Arythmia}
\begin{itemize}
\item {fónica:ri}
\end{itemize}
\begin{itemize}
\item {Grp. gram.:f.}
\end{itemize}
\begin{itemize}
\item {Utilização:Med.}
\end{itemize}
\begin{itemize}
\item {Proveniência:(Do gr. \textunderscore a\textunderscore  priv. + \textunderscore ruthmos\textunderscore )}
\end{itemize}
Irregularidade e desigualdade das contracções do coração.
\section{Asarabácara}
\begin{itemize}
\item {Grp. gram.:f.}
\end{itemize}
\begin{itemize}
\item {Proveniência:(Do lat. \textunderscore asarum\textunderscore  + \textunderscore baccharis\textunderscore )}
\end{itemize}
O mesmo que \textunderscore ásaro\textunderscore .
\section{Ás-claras}
\begin{itemize}
\item {Grp. gram.:loc. adv.}
\end{itemize}
\begin{itemize}
\item {Proveniência:(De \textunderscore claro\textunderscore )}
\end{itemize}
Á vista de todos; publicamente.
Categoricamente; sem rodeios.
\section{Asnilho}
\begin{itemize}
\item {Grp. gram.:m.}
\end{itemize}
\begin{itemize}
\item {Utilização:Ant.}
\end{itemize}
\begin{itemize}
\item {Utilização:Fig.}
\end{itemize}
Asno pequeno; burrinho.
A substância material do homem, (por opposição á alma). Cf. \textunderscore Rev. Lus.\textunderscore , XVI, 2.
\section{Asnoga}
\begin{itemize}
\item {Grp. gram.:f.}
\end{itemize}
\begin{itemize}
\item {Utilização:Ant.}
\end{itemize}
(Outra fórma de \textunderscore esnoga\textunderscore )
\section{Assã}
\begin{itemize}
\item {Grp. gram.:f.}
\end{itemize}
\begin{itemize}
\item {Utilização:Prov.}
\end{itemize}
Verme, o mesmo que \textunderscore san\textunderscore ^3.
\section{Assan}
\begin{itemize}
\item {Grp. gram.:f.}
\end{itemize}
\begin{itemize}
\item {Utilização:Prov.}
\end{itemize}
Verme, o mesmo que \textunderscore san\textunderscore ^3.
\section{Assôa-queixos}
\begin{itemize}
\item {Grp. gram.:m.}
\end{itemize}
\begin{itemize}
\item {Utilização:Prov.}
\end{itemize}
O mesmo que \textunderscore sopapo\textunderscore . (Colhido na Bairrada)
\section{Assobear}
\begin{itemize}
\item {Grp. gram.:v. t.}
\end{itemize}
\begin{itemize}
\item {Utilização:Prov.}
\end{itemize}
\begin{itemize}
\item {Utilização:trasm.}
\end{itemize}
Prender com o sobeu a cabeçalha do (carro) ao jugo.
\section{Atelia}
\begin{itemize}
\item {Grp. gram.:f.}
\end{itemize}
\begin{itemize}
\item {Utilização:Terat.}
\end{itemize}
\begin{itemize}
\item {Proveniência:(Do gr. \textunderscore ateles\textunderscore )}
\end{itemize}
Ausência congênita de uma parte do corpo.
\section{Atentar}
\begin{itemize}
\item {Grp. gram.:v. t.}
\end{itemize}
\begin{itemize}
\item {Utilização:Pop.}
\end{itemize}
O mesmo que \textunderscore tentar\textunderscore : \textunderscore o demónio atenta a gente\textunderscore .
\section{Aterragem}
\begin{itemize}
\item {Grp. gram.:f.}
\end{itemize}
\begin{itemize}
\item {Utilização:Neol.}
\end{itemize}
Acto de aterrar^2, (descer á terra um aeroplano).
\section{Atiçonado}
\begin{itemize}
\item {Grp. gram.:adj.}
\end{itemize}
Diz-se do cavallo, que tem malhas escuras ou negras, que parecem feitas com um tição.
\section{Atmisatria}
\begin{itemize}
\item {Grp. gram.:f.}
\end{itemize}
\begin{itemize}
\item {Utilização:Med.}
\end{itemize}
\begin{itemize}
\item {Proveniência:(Do gr. \textunderscore atmos\textunderscore  + \textunderscore iatreia\textunderscore )}
\end{itemize}
Applicação de vapores medicamentosos em affecções respiratórias.
Aerotherapia. Cf. J. Saavedra &amp; A. Barradas, \textunderscore Dicion.\textunderscore 
\section{Atonado}
\begin{itemize}
\item {Grp. gram.:adj.}
\end{itemize}
\begin{itemize}
\item {Utilização:Prov.}
\end{itemize}
\begin{itemize}
\item {Utilização:dur.}
\end{itemize}
\begin{itemize}
\item {Utilização:Fig.}
\end{itemize}
\begin{itemize}
\item {Proveniência:(De \textunderscore atonar\textunderscore )}
\end{itemize}
Que vem á tona da água, (falando-se do peixe, atordoado por coca ou outra substância venenosa).
Afflicto.
\section{Atonar}
\begin{itemize}
\item {Grp. gram.:v. i.}
\end{itemize}
\begin{itemize}
\item {Utilização:Prov.}
\end{itemize}
\begin{itemize}
\item {Utilização:dur.}
\end{itemize}
Vir á tona da água (o peixe envenenado pelo pescador).
\section{Atormentar}
\begin{itemize}
\item {Grp. gram.:v. t.}
\end{itemize}
\begin{itemize}
\item {Utilização:Pop.}
\end{itemize}
O mesmo que \textunderscore adormentar\textunderscore , tornar insensível: \textunderscore deu-lhe tão violenta pancada, que lhe atormentou o braço\textunderscore .
\section{Atormentadiço}
\begin{itemize}
\item {Grp. gram.:adj.}
\end{itemize}
\begin{itemize}
\item {Proveniência:(De \textunderscore atormentar\textunderscore ^1)}
\end{itemize}
Susceptível de sêr atormentado; que se afflige facilmente. Cf. B. Pereira, \textunderscore Prosódia\textunderscore , vb. \textunderscore tortivus\textunderscore .
\section{Augite}
\begin{itemize}
\item {Grp. gram.:f.}
\end{itemize}
Espécie de pyroxênio verde-negro ou acastanhado, elemento predominante das diábases.
\section{Augito}
\begin{itemize}
\item {Grp. gram.:m.}
\end{itemize}
\begin{itemize}
\item {Utilização:Miner.}
\end{itemize}
Espécie de pyroxênio verde-negro ou acastanhado, elemento predominante das diábases.
\section{Auroreal}
\begin{itemize}
\item {Grp. gram.:adj.}
\end{itemize}
\begin{itemize}
\item {Utilização:Neol.}
\end{itemize}
O mesmo que \textunderscore auroral\textunderscore . Cf. Junqueiro, \textunderscore Lágrima\textunderscore .
\section{Autocratiz}
\begin{itemize}
\item {Grp. gram.:f.}
\end{itemize}
\begin{itemize}
\item {Utilização:Neol.}
\end{itemize}
Mulhér de autócrata.
Soberana absoluta. Cf. Latino, \textunderscore Hist. Pol. e Mil.\textunderscore , I, 394.
\section{Autólise}
\begin{itemize}
\item {Grp. gram.:f.}
\end{itemize}
\begin{itemize}
\item {Utilização:Med.}
\end{itemize}
\begin{itemize}
\item {Proveniência:(Do gr. \textunderscore autos\textunderscore  + \textunderscore lusis\textunderscore )}
\end{itemize}
Dissolução espontânea de tecidos orgânicos, ou independente de qualquer intervenção microbiana. Cf. \textunderscore Rev. da Univ. de Coímbra\textunderscore , I, 133.
\section{Autólyse}
\begin{itemize}
\item {Grp. gram.:f.}
\end{itemize}
\begin{itemize}
\item {Utilização:Med.}
\end{itemize}
\begin{itemize}
\item {Proveniência:(Do gr. \textunderscore autos\textunderscore  + \textunderscore lusis\textunderscore )}
\end{itemize}
Dissolução espontânea de tecidos orgânicos, ou independente de qualquer intervenção microbiana. Cf. \textunderscore Rev. da Univ. de Coímbra\textunderscore , I, 133.
\section{Autopsial}
\begin{itemize}
\item {Grp. gram.:adj.}
\end{itemize}
Relativo a autópsia.
\section{Autópsico}
\begin{itemize}
\item {Grp. gram.:adj.}
\end{itemize}
Relativo a autópsia. Cf. R. Jorge, \textunderscore Epid. de Lisbôa\textunderscore .
\section{Avairana}
\begin{itemize}
\item {Grp. gram.:f.}
\end{itemize}
Bello arbusto da região do Amazonas.
\section{Avècido}
\begin{itemize}
\item {Grp. gram.:adj.}
\end{itemize}
\begin{itemize}
\item {Utilização:Prov.}
\end{itemize}
\begin{itemize}
\item {Utilização:trasm.}
\end{itemize}
Sabedor; esperto.
\section{Ave-leal}
\begin{itemize}
\item {Grp. gram.:f.}
\end{itemize}
\begin{itemize}
\item {Utilização:Ant.}
\end{itemize}
O mesmo que \textunderscore rouxinol\textunderscore . Cf. \textunderscore Égloga de Crisfal\textunderscore .
\section{Avellutar}
\begin{itemize}
\item {Grp. gram.:v. t.}
\end{itemize}
\begin{itemize}
\item {Utilização:Des.}
\end{itemize}
O mesmo que \textunderscore avelludar\textunderscore . Cf. Pant. de Aveiro, \textunderscore Itiner.\textunderscore , 127, (2.^a ed.).
\section{Avelutar}
\begin{itemize}
\item {Grp. gram.:v. t.}
\end{itemize}
\begin{itemize}
\item {Utilização:Des.}
\end{itemize}
O mesmo que \textunderscore aveludar\textunderscore . Cf. Pant. de Aveiro, \textunderscore Itiner.\textunderscore , 127, (2.^a ed.).
\section{Avespão}
\begin{itemize}
\item {Grp. gram.:m.}
\end{itemize}
\begin{itemize}
\item {Utilização:Des.}
\end{itemize}
\begin{itemize}
\item {Utilização:Pop.}
\end{itemize}
O mesmo que \textunderscore vespão\textunderscore .
\section{Avesseiro}
\begin{itemize}
\item {Grp. gram.:adj.}
\end{itemize}
\begin{itemize}
\item {Utilização:Prov.}
\end{itemize}
\begin{itemize}
\item {Utilização:dur.}
\end{itemize}
Diz-se do terreno, húmido ou frio, em que não dá o sol.
(Cp. \textunderscore avessedo\textunderscore  e \textunderscore abicheiro\textunderscore )
\section{Avião}
\begin{itemize}
\item {Grp. gram.:m.}
\end{itemize}
\begin{itemize}
\item {Utilização:Neol.}
\end{itemize}
Apparelho de aviação, com asas horizontaes.
\section{Avincu}
\begin{itemize}
\item {Grp. gram.:m.}
\end{itemize}
\begin{itemize}
\item {Utilização:T. da Guarda}
\end{itemize}
O mesmo que \textunderscore pyrilampo\textunderscore .
\section{Avogar}
\begin{itemize}
\item {Grp. gram.:v. t.  e  i.}
\end{itemize}
\begin{itemize}
\item {Utilização:Ant.}
\end{itemize}
O mesmo que \textunderscore advogar\textunderscore . Cf. B. Pereira, \textunderscore Prosódia\textunderscore , vb. \textunderscore sophocles\textunderscore .
\section{Avol}
\begin{itemize}
\item {Grp. gram.:adj.}
\end{itemize}
\begin{itemize}
\item {Utilização:Ant.}
\end{itemize}
O mesmo que \textunderscore mau\textunderscore . Cf. \textunderscore Port. Mon. Hist.\textunderscore , \textunderscore Script.\textunderscore , 248 e 249.
\section{Avondança}
\begin{itemize}
\item {Grp. gram.:f.}
\end{itemize}
O mesmo que \textunderscore avondamento\textunderscore . Cf. R. Pina, \textunderscore D. Duarte\textunderscore , 100.
\section{Avòsar}
\begin{itemize}
\item {Grp. gram.:v. t.}
\end{itemize}
\begin{itemize}
\item {Utilização:bras}
\end{itemize}
\begin{itemize}
\item {Utilização:Neol.}
\end{itemize}
Dar o tratamento de \textunderscore vós\textunderscore  a. Cf. Pacheco da Silva, \textunderscore Gram. Port.\textunderscore , 593, (2.^a ed.)
\section{Azambrado}
\begin{itemize}
\item {Grp. gram.:adv.}
\end{itemize}
\begin{itemize}
\item {Utilização:Pop.}
\end{itemize}
\begin{itemize}
\item {Proveniência:(De \textunderscore zambro\textunderscore )}
\end{itemize}
Desajeitado; mal conformado. Cf. Júl. Moreira, \textunderscore Estudos\textunderscore , II, 131.
\section{Azarolado}
\begin{itemize}
\item {Grp. gram.:adj.}
\end{itemize}
\begin{itemize}
\item {Utilização:Fig.}
\end{itemize}
Aziumado; mal humorado:«\textunderscore cá o azarolado morgado de Fafe...\textunderscore »Camillo, \textunderscore Morg. de Fafe\textunderscore , 7, (ed. 1865).
\section{Azeche}
\begin{itemize}
\item {Grp. gram.:m.}
\end{itemize}
\begin{itemize}
\item {Proveniência:(Do ár. \textunderscore az-zej\textunderscore )}
\end{itemize}
Terra escura, também chamada \textunderscore terra de Sevilha\textunderscore .
Cp. \textunderscore terra\textunderscore .
\section{Azemeleiro}
\begin{itemize}
\item {Grp. gram.:m.}
\end{itemize}
Conductor de azeméis.
\section{Azereiro-dos-damnados}
\begin{itemize}
\item {Grp. gram.:m.}
\end{itemize}
O mesmo que \textunderscore azereiro-pado\textunderscore .
\section{Azereiro-pado}
\begin{itemize}
\item {Grp. gram.:m.}
\end{itemize}
Arbusto, o mesmo que \textunderscore pado\textunderscore .
\section{Azibó}
\begin{itemize}
\item {Grp. gram.:m.}
\end{itemize}
\begin{itemize}
\item {Utilização:Prov.}
\end{itemize}
\begin{itemize}
\item {Utilização:trasm.}
\end{itemize}
Espécie de cogumelo. (Colhido em Chaves)
\section{Azodol}
\begin{itemize}
\item {Grp. gram.:m.}
\end{itemize}
Preparação chímica, espécie de pasta, para limpar os dentes.
\section{Azuagos}
\begin{itemize}
\item {Grp. gram.:m. Pl.}
\end{itemize}
Povo antigo de Marrocos, que se bateu com os Portugueses em Alcácer-Quibir. Hoje, \textunderscore Zuavos\textunderscore .
Cp. \textunderscore zuavo\textunderscore .
\section{Azues-lóios}
\begin{itemize}
\item {Grp. gram.:m. pl.}
\end{itemize}
Frades, da Ordem do Hospital ou de Malta. Cf. P. Carvalho, \textunderscore Corogr. Port.\textunderscore , II, 10.
\section{Azuraque}
\begin{itemize}
\item {Grp. gram.:m.}
\end{itemize}
Planta, o mesmo que \textunderscore zuraque\textunderscore .
\section{A}
\begin{itemize}
\item {fónica:á}
\end{itemize}
\begin{itemize}
\item {Grp. gram.:m.}
\end{itemize}
\begin{itemize}
\item {Utilização:Mús.}
\end{itemize}
\begin{itemize}
\item {Grp. gram.:Adj.}
\end{itemize}
\begin{itemize}
\item {Grp. gram.:Adj.}
\end{itemize}
Primeira letra do alphabeto português.
Primeira nota da escala na denominação alphabética dos sons.
Primeiro, (falando-se de um número ou de um objecto que faz parte de uma série).
Que é de primeira classe, (falando-se de carruagens de caminho de ferro).
\section{A}
\begin{itemize}
\item {fónica:á}
\end{itemize}
\begin{itemize}
\item {Grp. gram.:art.}
\end{itemize}
\begin{itemize}
\item {Proveniência:(Do lat. \textunderscore illa\textunderscore )}
\end{itemize}
(fem. de o).
Ant. \textunderscore la\textunderscore .
\section{A}
\begin{itemize}
\item {fónica:â}
\end{itemize}
\begin{itemize}
\item {Grp. gram.:prep.}
\end{itemize}
Que indica várias relações.
(Lat. \textunderscore ad\textunderscore ).
\section{A}
\begin{itemize}
\item {fónica:â}
\end{itemize}
\begin{itemize}
\item {Grp. gram.:pron.}
\end{itemize}
Flexão fem. do pron. o.
\section{A...}
\begin{itemize}
\item {Grp. gram.:pref.}
\end{itemize}
\begin{itemize}
\item {Proveniência:(Do lat. \textunderscore a\textunderscore  e \textunderscore ab\textunderscore ; e do gr. \textunderscore a\textunderscore  e \textunderscore an\textunderscore , quando indica negação)}
\end{itemize}
(que designa \textunderscore intensidade\textunderscore , \textunderscore separação\textunderscore , \textunderscore agglomeração\textunderscore , \textunderscore imitação\textunderscore , \textunderscore prolongação\textunderscore , \textunderscore transformação\textunderscore , \textunderscore perseguição\textunderscore , \textunderscore collocação\textunderscore , \textunderscore negação\textunderscore , \textunderscore aproximação\textunderscore , \textunderscore uniformidade\textunderscore , \textunderscore juncção\textunderscore ; e que se emprega como \textunderscore expletivo\textunderscore , sem alterar a significação do radical).
\section{A.}
\begin{itemize}
\item {Grp. gram.:abrev.}
\end{itemize}
De \textunderscore autor\textunderscore .
\section{Á}
(contr. da prep. \textunderscore a\textunderscore  com o art. \textunderscore a\textunderscore ).
\section{Á}
\begin{itemize}
\item {Grp. gram.:interj.}
\end{itemize}
O mesmo que \textunderscore ah!\textunderscore .
\section{AA}
\begin{itemize}
\item {fónica:ás}
\end{itemize}
\begin{itemize}
\item {Grp. gram.:m. pl.}
\end{itemize}
Caracteres, que pluralizam a representação phonética da primeira letra do alphabeto.
\section{Aacima}
\begin{itemize}
\item {Grp. gram.:adv.}
\end{itemize}
\begin{itemize}
\item {Utilização:Ant.}
\end{itemize}
Finalmente; em conclusão.
\section{Aal}
\begin{itemize}
\item {Grp. gram.:m.}
\end{itemize}
Árvore terebinthácea, cuja casca aromatiza o vinho.
\section{Aalênio}
\begin{itemize}
\item {Grp. gram.:m.  e  adj.}
\end{itemize}
\begin{itemize}
\item {Utilização:Geol.}
\end{itemize}
Diz-se do terreno que predomina nas cercanias de Aalen, (Alemanha).
\section{Aata}
\begin{itemize}
\item {Grp. gram.:f.}
\end{itemize}
\begin{itemize}
\item {Utilização:Bras}
\end{itemize}
Canôa de casca de madeira, com as extremidades achatadas em fórma de bico de pato.
(Do tupi \textunderscore aa\textunderscore , mal e \textunderscore atá\textunderscore , andar).
\section{Aaz}
\begin{itemize}
\item {Grp. gram.:f.}
\end{itemize}
\begin{itemize}
\item {Utilização:Ant.}
\end{itemize}
\begin{itemize}
\item {Proveniência:(Do lat. \textunderscore acies\textunderscore )}
\end{itemize}
Ala, acampamento. Companhia, communidade.
\section{Ab...}
\begin{itemize}
\item {Grp. gram.:pref.}
\end{itemize}
\begin{itemize}
\item {Proveniência:(Do lat. \textunderscore ab\textunderscore )}
\end{itemize}
(que designa \textunderscore intensão\textunderscore , \textunderscore separação\textunderscore , \textunderscore opposição\textunderscore ).
\section{Aba}
\begin{itemize}
\item {Grp. gram.:f.}
\end{itemize}
Extremidade (de alguns vestidos).
Prolongamento dos lados de um corpo ou superfície: \textunderscore a aba do chapéu\textunderscore .
Peça saliente, em certas obras de carpintaria, alvenaria e serralharia.
Vizinhança, sopé: \textunderscore na aba da serra\textunderscore .
\section{Aba}
\begin{itemize}
\item {Grp. gram.:m.}
\end{itemize}
\begin{itemize}
\item {Proveniência:(Do syr. \textunderscore abba\textunderscore )}
\end{itemize}
Título do Bispo em algumas Igrejas orientaes.
\section{Abá}
\begin{itemize}
\item {Grp. gram.:m.}
\end{itemize}
Manto de beduinos.
\section{Ababás}
\begin{itemize}
\item {Grp. gram.:m. pl.}
\end{itemize}
Aborigenes do Brasil, que habitaram Mato-Grosso.
\section{Abacá}
\begin{itemize}
\item {Grp. gram.:m.}
\end{itemize}
Espécie de bananeira, também conhecida por côfo.
Fibra dessa planta.
\section{Abaçanar}
\begin{itemize}
\item {Grp. gram.:v. t.}
\end{itemize}
(V.abacinar).
\section{Abacial}
\begin{itemize}
\item {Grp. gram.:adj.}
\end{itemize}
\begin{itemize}
\item {Proveniência:(Do b. lat. \textunderscore abbatialis\textunderscore )}
\end{itemize}
Que diz respeito ao abbade ou á abbadia.
\section{Abacate}
\begin{itemize}
\item {Grp. gram.:m.}
\end{itemize}
Fruto do abacateiro.
O mesmo que \textunderscore abacateiro\textunderscore .
\section{Abacateiro}
\begin{itemize}
\item {Grp. gram.:m.}
\end{itemize}
Árvore tropical, de fruto comestível.
\section{Abacatina}
\begin{itemize}
\item {Grp. gram.:f.}
\end{itemize}
Peixe do Brasil.
\section{Abacatirana}
\begin{itemize}
\item {Grp. gram.:f.}
\end{itemize}
Arvore laurácea do Brasil.
\section{Abacaxi}
\begin{itemize}
\item {Grp. gram.:m.}
\end{itemize}
Espécie de ananás, (\textunderscore bromelia ananas\textunderscore , Lin.)
\section{Abacelado}
\begin{itemize}
\item {Grp. gram.:adj.}
\end{itemize}
Convertido em bacello.
Plantado de bacello: \textunderscore encosta abacellada\textunderscore .
\section{Abacelamento}
\begin{itemize}
\item {Grp. gram.:m.}
\end{itemize}
Acto ou effeito de abacellar.
\section{Abacelar}
\begin{itemize}
\item {Grp. gram.:v. t.}
\end{itemize}
Converter em bacello.
Plantar de bacello.
\section{Abacellado}
\begin{itemize}
\item {Grp. gram.:adj.}
\end{itemize}
Convertido em bacello.
Plantado de bacello: \textunderscore encosta abacellada\textunderscore .
\section{Abacellamento}
\begin{itemize}
\item {Grp. gram.:m.}
\end{itemize}
Acto ou effeito de abacellar.
\section{Abacellar}
\begin{itemize}
\item {Grp. gram.:v. t.}
\end{itemize}
Converter em bacello.
Plantar de bacello.
\section{Abaceto}
\begin{itemize}
\item {fónica:cê}
\end{itemize}
\begin{itemize}
\item {Grp. gram.:m.}
\end{itemize}
Gênero de coleópteros pentâmeros.
\section{Abacinar}
\begin{itemize}
\item {Grp. gram.:v. t.}
\end{itemize}
\begin{itemize}
\item {Proveniência:(Do b. lat. \textunderscore abacinare\textunderscore )}
\end{itemize}
Tornar escuro, privar da claridade.
Arroxear (a pelle).
\section{Abacista}
\begin{itemize}
\item {Grp. gram.:m.}
\end{itemize}
O que trabalhava ou estudava no ábaco.
(B. lat. \textunderscore abacista\textunderscore ).
\section{Ábaco}
\begin{itemize}
\item {Grp. gram.:m.}
\end{itemize}
\begin{itemize}
\item {Utilização:Archit.}
\end{itemize}
\begin{itemize}
\item {Utilização:Artilh.}
\end{itemize}
Parte superior do capitel, em que assenta a architrave.
Quadro antigo, em que se inscreviam os algarismos, para ensinar a calcular.
Antigo bufete ou aparador.
Desenho, para se fazerem certos cálculos, em problemas de tiro, não por números, mas por linhas.
(Lat. \textunderscore abacus\textunderscore ).
\section{Abactor}
\begin{itemize}
\item {Grp. gram.:m.}
\end{itemize}
Ladrão de gados.
(Lat. \textunderscore abactor\textunderscore ).
\section{Abáculo}
\begin{itemize}
\item {Grp. gram.:m.}
\end{itemize}
Pequeno ábaco, antiga mesa pequena.
Pedra variegada, usada pelos Romanos em certos jogos.
Cubo de tijolo ou vidro, pintado e embutido em pavimentos de mosaico.
\section{Abacutaia}
\begin{itemize}
\item {Grp. gram.:f.}
\end{itemize}
Peixe do Brasil.
\section{Abada}
\begin{itemize}
\item {Grp. gram.:f.}
\end{itemize}
Aba cheia; grande quantidade: \textunderscore trazia abadas de rosas\textunderscore .
(De \textunderscore aba\textunderscore ).
\section{Abada}
\begin{itemize}
\item {Grp. gram.:f.}
\end{itemize}
\begin{itemize}
\item {Proveniência:(Do mal. \textunderscore badaq\textunderscore , rhinoceronte)}
\end{itemize}
Fêmea do rhinoceronte; pachiderme, análogo àquelle.
\section{Abadado}
\begin{itemize}
\item {Grp. gram.:m.}
\end{itemize}
\begin{itemize}
\item {Utilização:Ant.}
\end{itemize}
O mesmo que \textunderscore abbadia\textunderscore .
\section{Abadágio}
\begin{itemize}
\item {Grp. gram.:m.}
\end{itemize}
Refeição, que os parochianos eram obrigados a dar ao abbade.
\section{Abadalassa}
\begin{itemize}
\item {Grp. gram.:m.}
\end{itemize}
Jôgo antigo.
Cf. \textunderscore Cancion\textunderscore . \textunderscore Ger.\textunderscore , I, 146.
\section{Abadar}
\begin{itemize}
\item {Grp. gram.:v. t.}
\end{itemize}
Prover de abbade.
\section{Abadavina}
\begin{itemize}
\item {Grp. gram.:f.}
\end{itemize}
Pássaro conirostro.
\section{Abade}
\begin{itemize}
\item {Grp. gram.:m.}
\end{itemize}
\begin{itemize}
\item {Proveniência:(Do b. lat. \textunderscore abbas\textunderscore )}
\end{itemize}
Prelado de ordem monástica.
Aquelle que governa uma abbadia.
Cura de almas, párocho.
\section{Abadecídio}
\begin{itemize}
\item {Grp. gram.:m.}
\end{itemize}
Assassinio de abbade. Cf. B. Pato, \textunderscore Ciprestes\textunderscore , 107.
\section{Abadejo}
\begin{itemize}
\item {Grp. gram.:m.}
\end{itemize}
\begin{itemize}
\item {Utilização:Prov.}
\end{itemize}
O mesmo que \textunderscore badejo\textunderscore .
O mesmo que \textunderscore vaca-loira\textunderscore .
\section{Abadengo}
\begin{itemize}
\item {Grp. gram.:m.}
\end{itemize}
\begin{itemize}
\item {Utilização:Ant.}
\end{itemize}
\begin{itemize}
\item {Grp. gram.:Adj.}
\end{itemize}
Legado pio ou esmola, que se dava em vida ou se deixava por morte ao confessor.
O antigo direito de sêr abbade de alguma igreja.
Relativo ao território ou á jurisdicção de abbade.
\section{Abadernar}
\begin{itemize}
\item {Grp. gram.:v.}
\end{itemize}
\begin{itemize}
\item {Utilização:t. Náut.}
\end{itemize}
Apertar com badernas.
\section{Abadesco}
\begin{itemize}
\item {Grp. gram.:adj.}
\end{itemize}
Próprio de abbade.
\section{Abadessa}
\begin{itemize}
\item {Grp. gram.:f.}
\end{itemize}
\begin{itemize}
\item {Proveniência:(Do b. lat. \textunderscore abbatissa\textunderscore )}
\end{itemize}
Prelada de communidade religiosa.
\section{Abadessado}
\begin{itemize}
\item {Grp. gram.:m.}
\end{itemize}
Cargo de abbadessa.
O tempo que dura esse cargo.
As festas, com que se elege a abbadessa.
\section{Abadessona}
\begin{itemize}
\item {Grp. gram.:f.}
\end{itemize}
\begin{itemize}
\item {Utilização:Fam.}
\end{itemize}
Abbadessa muito encorpada. Cf. Garrett, \textunderscore Flores\textunderscore , 95.
\section{Abadia}
\begin{itemize}
\item {Grp. gram.:f.}
\end{itemize}
Igreja, regida por abbade.
Paróchia.
Rendimento do abbade.
\section{Abadiado}
\begin{itemize}
\item {Grp. gram.:m.}
\end{itemize}
O mesmo que \textunderscore abbadia\textunderscore .
\section{Abadim}
\begin{itemize}
\item {Grp. gram.:m.}
\end{itemize}
\begin{itemize}
\item {Utilização:Ant.}
\end{itemize}
Lugar, habitado por observantes religiosos.
(De \textunderscore abbade\textunderscore ).
\section{Abadir}
\begin{itemize}
\item {Grp. gram.:m.}
\end{itemize}
Pedra, que, segundo a Mythologia, Saturno enguliu, enganado por sua mulher. Cf. Castilho, \textunderscore Fastos\textunderscore , II, 251.
O mesmo que \textunderscore bétylo\textunderscore .
(Lat. \textunderscore abadir\textunderscore ).
\section{Abado}
\begin{itemize}
\item {fónica:àbá}
\end{itemize}
\begin{itemize}
\item {Grp. gram.:adj.}
\end{itemize}
Que tem abas grandes: \textunderscore chapéu abado\textunderscore .
\section{Abaetado}
\begin{itemize}
\item {fónica:baê}
\end{itemize}
\begin{itemize}
\item {Grp. gram.:adj.}
\end{itemize}
Semelhante a baêta.
\section{Abaetar}
\begin{itemize}
\item {fónica:baê}
\end{itemize}
\begin{itemize}
\item {Grp. gram.:v. t.}
\end{itemize}
Vestir com baêta.
Fabricar (tecido que imite a baêta).
\section{Abafa!}
\begin{itemize}
\item {Grp. gram.:interj.}
\end{itemize}
Grito imperativo, para os marinheiros ferrarem as velas.
\section{Abafação}
\begin{itemize}
\item {Grp. gram.:f.}
\end{itemize}
Acto de \textunderscore abafar\textunderscore :«\textunderscore Tenho ataques de abafação\textunderscore ». Camillo, \textunderscore Mulher Fatal\textunderscore , p. 210.
\section{Abafadamente}
\begin{itemize}
\item {Grp. gram.:adv.}
\end{itemize}
De modo abafado.
\section{Abafadela}
\begin{itemize}
\item {Grp. gram.:f.}
\end{itemize}
O mesmo que \textunderscore abafação\textunderscore .
O mesmo que \textunderscore abafarete\textunderscore .
\section{Abafadiço}
\begin{itemize}
\item {Grp. gram.:adj.}
\end{itemize}
Susceptível de se suffocar.
Que suffoca: \textunderscore tempo abafadiço\textunderscore .
Irascível.
\section{Abafado}
\begin{itemize}
\item {Grp. gram.:adj.}
\end{itemize}
Que se respira com difficuldade: \textunderscore ar abafado\textunderscore .
Pouco ou mal ventilado: \textunderscore casa abafada\textunderscore .
Diz-se do vinho, de cujo mosto se impediu a fermentação.
(De \textunderscore abafar\textunderscore ).
\section{Abafador}
\begin{itemize}
\item {Grp. gram.:adj.}
\end{itemize}
\begin{itemize}
\item {Grp. gram.:M.}
\end{itemize}
\begin{itemize}
\item {Grp. gram.:M.}
\end{itemize}
\begin{itemize}
\item {Utilização:Prov.}
\end{itemize}
Que abafa, que suffoca: \textunderscore calor abafador\textunderscore .
Peça, que em certos instrumentos abafa o som, suspendendo a vibração das cordas.
Aquelle que, em certas seitas, apressava a morte ao moribundo, afogando-o com almofadas, para que não peccasse, depois de consolado pelo sacerdote.
Pano, com que se cobre o bule do chá, para que êste não arrefeça.
\section{Abafadura}
\begin{itemize}
\item {Grp. gram.:f.}
\end{itemize}
O mesmo que abafamento.
\section{Abafamento}
\begin{itemize}
\item {Grp. gram.:m.}
\end{itemize}
\begin{itemize}
\item {Grp. gram.:Adj.}
\end{itemize}
Acto de \textunderscore abafar\textunderscore .
Falta de ar.
Abafante.
O mesmo que \textunderscore abafador\textunderscore .
\section{Abafar}
\begin{itemize}
\item {Grp. gram.:v. t.}
\end{itemize}
\begin{itemize}
\item {Utilização:Vin.}
\end{itemize}
\begin{itemize}
\item {Grp. gram.:V. i.}
\end{itemize}
\begin{itemize}
\item {Grp. gram.:V. p.}
\end{itemize}
\begin{itemize}
\item {Utilização:Prov.}
\end{itemize}
Tirar o bafo a; suffocar: \textunderscore o calor abafava os espectadores\textunderscore .
Suffocar; asphyxiar.
Impedir a combustão de: \textunderscore abafar o lume\textunderscore .
Dissimular, conter: \textunderscore abafar um grito\textunderscore .
Impedir a continuação de: \textunderscore abafar a questão\textunderscore .
Esconder: \textunderscore abafar uma carta\textunderscore .
Agasalhar com roupas: \textunderscore abafar a criança\textunderscore .
Impedir a fermentação de (mosto).
Respirar a custo: \textunderscore eu abafo\textunderscore .
Agasalhar-se com roupas.
Cavar e preparar (o terreno), para a sementeira do milho. (Colhido em Alcanena)
\section{Abafar}
\begin{itemize}
\item {Grp. gram.:m.}
\end{itemize}
Peixe plagióstomo, cinzento por cima e esbranquiçado por baixo.
\section{Abafarete}
\begin{itemize}
\item {fónica:farê}
\end{itemize}
\begin{itemize}
\item {Grp. gram.:m.}
\end{itemize}
\begin{itemize}
\item {Utilização:Neol.}
\end{itemize}
Acto de \textunderscore abafar\textunderscore  (uma questão parlamentar).
No voltarete, acto de não mostrar na mesa a espadilha e o basto, quando há obrigação de os mostrar.
\section{Abafas}
\begin{itemize}
\item {Grp. gram.:f. pl.}
\end{itemize}
\begin{itemize}
\item {Utilização:Ant.}
\end{itemize}
Bravatas, ameaças arrogantes.
\section{Abafeira}
\begin{itemize}
\item {Grp. gram.:f.}
\end{itemize}
\begin{itemize}
\item {Grp. gram.:M.}
\end{itemize}
Paúl, charco.
Abafo.
Acto de \textunderscore abafar\textunderscore .
Aquillo que abafa ou resguarda.
Aconchego, carinho.
\section{Abagoar}
\begin{itemize}
\item {Grp. gram.:v. t.}
\end{itemize}
\begin{itemize}
\item {Utilização:Prov.}
\end{itemize}
\begin{itemize}
\item {Utilização:minh.}
\end{itemize}
Desengranzar; desenfiar.
(De \textunderscore bago\textunderscore ).
\section{Abagum}
\begin{itemize}
\item {Grp. gram.:m.}
\end{itemize}
Ave tropical.
\section{Abahuladamente}
\begin{itemize}
\item {fónica:ba-u}
\end{itemize}
\begin{itemize}
\item {Grp. gram.:adv.}
\end{itemize}
De modo \textunderscore abahulado\textunderscore , á maneira de bahu.
\section{Abahulado}
\begin{itemize}
\item {fónica:baú}
\end{itemize}
\begin{itemize}
\item {Grp. gram.:adj.}
\end{itemize}
Que tem fórma de bahu.
\section{Abahulador}
\begin{itemize}
\item {fónica:baú}
\end{itemize}
\begin{itemize}
\item {Grp. gram.:m.}
\end{itemize}
O que abahula.
\section{Abahulamento}
\begin{itemize}
\item {fónica:baú}
\end{itemize}
\begin{itemize}
\item {Grp. gram.:m.}
\end{itemize}
Acto ou effeito de \textunderscore abahular\textunderscore .
Arco abatido nas abóbadas.
Superfície curva das ruas calçadas, para se facilitar o escoamento das águas.
Convexidade.
\section{Abahular}
\begin{itemize}
\item {fónica:baú}
\end{itemize}
\begin{itemize}
\item {Grp. gram.:v. t.}
\end{itemize}
Dar fórma de bahu a; tornar convexo.
\section{Abainhar}
\begin{itemize}
\item {fónica:ba-i}
\end{itemize}
\begin{itemize}
\item {Grp. gram.:v. t.}
\end{itemize}
Fazer a baínha de.
Impropriamente, o mesmo que \textunderscore embainhar\textunderscore .
\section{Abaionetar}
\begin{itemize}
\item {Grp. gram.:v. t.}
\end{itemize}
Ferir ou trespassar com baioneta.
\section{Abairramento}
\begin{itemize}
\item {Grp. gram.:m.}
\end{itemize}
Acto ou effeito de abairrar.
\section{Abairrar}
\begin{itemize}
\item {Grp. gram.:v. t.}
\end{itemize}
Dividir em bairros.
\section{Abaixa}
\begin{itemize}
\item {Grp. gram.:f.}
\end{itemize}
Espécie de fisga, usada na pesca da lampreia, ao norte de Portugal.
\section{Abaixadela}
\begin{itemize}
\item {Grp. gram.:f.}
\end{itemize}
O mesmo que \textunderscore abaixadura\textunderscore .
\section{Abaixador}
\begin{itemize}
\item {Grp. gram.:m.}
\end{itemize}
O que abaixa.
\section{Abaixadura}
\begin{itemize}
\item {Grp. gram.:f.}
\end{itemize}
Acto ou effeito de \textunderscore abaixar\textunderscore .
\section{Abaixamento}
\begin{itemize}
\item {Grp. gram.:m.}
\end{itemize}
Acto ou effeito de \textunderscore abaixar\textunderscore .
\section{Abaixante}
\begin{itemize}
\item {Grp. gram.:m.}
\end{itemize}
O mesmo que \textunderscore abaixador\textunderscore .
\section{Abaixar}
\begin{itemize}
\item {Grp. gram.:v. t.}
\end{itemize}
\begin{itemize}
\item {Utilização:Fig.}
\end{itemize}
\begin{itemize}
\item {Grp. gram.:V. p.}
\end{itemize}
\begin{itemize}
\item {Utilização:Prov.}
\end{itemize}
\begin{itemize}
\item {Grp. gram.:Loc.}
\end{itemize}
\begin{itemize}
\item {Utilização:fam.}
\end{itemize}
Dirigir para baixo, abater: \textunderscore abaixou os olhos\textunderscore .
Fazer descer.
Conter, reprimir.
Aviltar.
Humilhar-se.
Defecar.
Deminuir (altura, preço, temperatura, etc.).
Traçar (linha perpendicular), partindo de um ponto para uma recta ou plano.
\textunderscore Abaixar a prôa\textunderscore , submeter-se, humilhar-se, sujeitar-se.
\textunderscore Abaixar a prôa a alguém\textunderscore , submetê-lo, reprimi-lo, humilhá-lo.
\section{Abaixo}
\begin{itemize}
\item {Grp. gram.:adv.}
\end{itemize}
\begin{itemize}
\item {Grp. gram.:M.}
\end{itemize}
Para a parte inferior; na parte inferior, inferiormente.
\textunderscore Um abaixo-assignado\textunderscore , petição, representação, ou documento subscrito por várias pessoas.
\section{Abajoujamento}
\begin{itemize}
\item {Grp. gram.:m.}
\end{itemize}
Acto de abajoujar-se.
\section{Abajoujar-se}
\begin{itemize}
\item {Grp. gram.:v. p.}
\end{itemize}
Fazer-se bajoujo.
\section{Abaju}
\begin{itemize}
\item {Grp. gram.:m.}
\end{itemize}
\begin{itemize}
\item {Proveniência:(Do fr. \textunderscore abat-jour\textunderscore )}
\end{itemize}
(V.Quebra-luz;pantalha).
\section{Abaju}
\begin{itemize}
\item {Grp. gram.:m.}
\end{itemize}
Raça mestiça do Brasil.
\section{Abajúrdio}
\begin{itemize}
\item {Grp. gram.:m.}
\end{itemize}
Voc. us. por Garrett, em vez do francesismo \textunderscore abaju\textunderscore ^1.
\section{Abalada}
\begin{itemize}
\item {Grp. gram.:f.}
\end{itemize}
Acto de \textunderscore abalar\textunderscore , partida.
Direcção tomada pela caça que se levanta.
\section{Abalado}
\begin{itemize}
\item {Grp. gram.:adj.}
\end{itemize}
Pouco firme, mal seguro: \textunderscore um dente abalado\textunderscore .
Commovido, impressionado: \textunderscore fiquei abalado com a noticia\textunderscore .
\section{Abaladura}
\begin{itemize}
\item {Grp. gram.:f.}
\end{itemize}
\begin{itemize}
\item {Utilização:Prov.}
\end{itemize}
\begin{itemize}
\item {Utilização:minh.}
\end{itemize}
O mesmo que \textunderscore abôrto\textunderscore .
\section{Abalamento}
\begin{itemize}
\item {Grp. gram.:m.}
\end{itemize}
O mesmo que \textunderscore abalo\textunderscore :«\textunderscore abalamento de doença\textunderscore ». Galvão, \textunderscore Chrón. de Aff. H.\textunderscore 
\section{Abalançamento}
\begin{itemize}
\item {Grp. gram.:m.}
\end{itemize}
Acto de abalançar.
\section{Abalançar}
\begin{itemize}
\item {Grp. gram.:v. t.}
\end{itemize}
Pesar com balança.
Dar movimento liberatório a.
Arrojar, impellir.
\section{Abalar}
\begin{itemize}
\item {Grp. gram.:v. t.}
\end{itemize}
\begin{itemize}
\item {Grp. gram.:V. i.}
\end{itemize}
Sacudir, tornando menos firme; fazer tremer.
Deminuir: \textunderscore abalar o crédito\textunderscore .
Impressionar, commover.
Partir ou fugir com pressa.
\section{Abalaustramento}
\begin{itemize}
\item {fónica:la-us}
\end{itemize}
\begin{itemize}
\item {Grp. gram.:m.}
\end{itemize}
Acto de abalaustrar.
\section{Abalaustrar}
\begin{itemize}
\item {fónica:la-us}
\end{itemize}
\begin{itemize}
\item {Grp. gram.:v. t.}
\end{itemize}
Adornar com balaústres.
\section{Abalável}
\begin{itemize}
\item {Grp. gram.:adj.}
\end{itemize}
Que póde sêr abalado.
\section{Abaldear}
\begin{itemize}
\item {Grp. gram.:v. t.}
\end{itemize}
O mesmo que \textunderscore baldear\textunderscore .
\section{Abalienação}
\begin{itemize}
\item {Grp. gram.:f.}
\end{itemize}
\begin{itemize}
\item {Proveniência:(Do lat. \textunderscore abalienatio\textunderscore )}
\end{itemize}
Transferência de gados, escravos ou terras a quem tinha o direito de os adquirir, entre os Romanos.
\section{Abalistar}
\begin{itemize}
\item {Grp. gram.:v. t.}
\end{itemize}
Atacar com tiros de balista.
\section{Abalizadamente}
\begin{itemize}
\item {Grp. gram.:adv.}
\end{itemize}
De modo \textunderscore abalizado\textunderscore , com distincção.
\section{Abalizado}
\begin{itemize}
\item {Grp. gram.:adj.}
\end{itemize}
Muito illustrado.
Distinto.
(De \textunderscore abalizar\textunderscore ).
\section{Abalienar}
\begin{itemize}
\item {Grp. gram.:v. t.}
\end{itemize}
\begin{itemize}
\item {Proveniência:(Do lat. \textunderscore abalienare\textunderscore )}
\end{itemize}
Transferir por abalienação.
\section{Abalizador}
\begin{itemize}
\item {Grp. gram.:m.}
\end{itemize}
\begin{itemize}
\item {Grp. gram.:m.}
\end{itemize}
Aquelle que abaliza.
Vara, com que se medem as terras.
\section{Abalizamento}
\begin{itemize}
\item {Grp. gram.:m.}
\end{itemize}
Acto ou effeito de abalizar.
\section{Abalizar}
\begin{itemize}
\item {Grp. gram.:v. t.}
\end{itemize}
Marcar com balizas, balizar.
\section{Abalo}
\begin{itemize}
\item {Grp. gram.:m.}
\end{itemize}
Acto ou effeito de \textunderscore abalar\textunderscore .
Terremoto.
Commoção.
Fuga, partida.
\section{Abaloar}
\begin{itemize}
\item {Grp. gram.:v. t.}
\end{itemize}
Dar fórma de balão a.
\section{Abalofar}
\begin{itemize}
\item {Grp. gram.:v. t.}
\end{itemize}
Tornar balofo.
\section{Abalonado}
\begin{itemize}
\item {Grp. gram.:adj.}
\end{itemize}
\begin{itemize}
\item {Utilização:Neol.}
\end{itemize}
Que tem fórma de balão. Cf. Val. Magalhães, \textunderscore Contos\textunderscore .
\section{Abalrôa}
\begin{itemize}
\item {Grp. gram.:f.}
\end{itemize}
O mesmo que \textunderscore balrôa\textunderscore .
\section{Abalroação}
\begin{itemize}
\item {Grp. gram.:f.}
\end{itemize}
O mesmo que \textunderscore abalroamento\textunderscore .
\section{Abalroada}
\begin{itemize}
\item {Grp. gram.:f.}
\end{itemize}
(V.abalroamento).
\section{Abalroadela}
\begin{itemize}
\item {Grp. gram.:f.}
\end{itemize}
(V.abalroamento).
\section{Abalroador}
\begin{itemize}
\item {Grp. gram.:m.  e  adj.}
\end{itemize}
O que abalrôa.
\section{Abalroamento}
\begin{itemize}
\item {Grp. gram.:m.}
\end{itemize}
Acto ou effeito de abalroar.
\section{Abalroar}
\begin{itemize}
\item {Grp. gram.:v. t.}
\end{itemize}
Atracar com balrôas.
Ir de encontro a.
\section{Abalsar}
\begin{itemize}
\item {Grp. gram.:v. t.}
\end{itemize}
Meter na balsa ou no balseiro.
\section{Abalseirar}
\begin{itemize}
\item {Grp. gram.:v. t.}
\end{itemize}
O mesmo que \textunderscore abalsar\textunderscore .
\section{Abaluartamento}
\begin{itemize}
\item {Grp. gram.:m.}
\end{itemize}
Acto de abaluartar.
\section{Abaluartar}
\begin{itemize}
\item {Grp. gram.:v. t.}
\end{itemize}
Guarnecer de baluartes.
Dar fórma de baluarte a.
\section{Abama}
\begin{itemize}
\item {Grp. gram.:f.}
\end{itemize}
Planta liliácea, (\textunderscore anthericum ossifragum\textunderscore , Lin.)
\section{Abâmeas}
\begin{itemize}
\item {Grp. gram.:f. pl.}
\end{itemize}
Grupo de plantas, que têm por typo a abama.
\section{Abanadela}
\begin{itemize}
\item {Grp. gram.:f.}
\end{itemize}
\begin{itemize}
\item {Utilização:Fam.}
\end{itemize}
Acto de \textunderscore abanar\textunderscore .
\section{Abanador}
\begin{itemize}
\item {Grp. gram.:m.}
\end{itemize}
O mesmo que \textunderscore abano\textunderscore .
\section{Abanadura}
\begin{itemize}
\item {Grp. gram.:f.}
\end{itemize}
O mesmo que \textunderscore abanadela\textunderscore .
\section{Abanamoscas}
\begin{itemize}
\item {Grp. gram.:m.}
\end{itemize}
\begin{itemize}
\item {Utilização:Fig.}
\end{itemize}
Enxotamoscas.
Pequeno valor, insignificância.
\section{Abananado}
\begin{itemize}
\item {Grp. gram.:adj.}
\end{itemize}
\begin{itemize}
\item {Utilização:Fam.}
\end{itemize}
\begin{itemize}
\item {Utilização:Prov.}
\end{itemize}
\begin{itemize}
\item {Utilização:alent.}
\end{itemize}
Aparvalhado, apalermado.
Aturdido.
O mesmo que \textunderscore adoentado\textunderscore .
(De \textunderscore abananar\textunderscore ).
\section{Abananar}
\begin{itemize}
\item {Grp. gram.:v. t.}
\end{itemize}
\begin{itemize}
\item {Utilização:Fam.}
\end{itemize}
Tornar palerma, aparvalhar.
Aturdir.
\section{Abanante}
\begin{itemize}
\item {Grp. gram.:adj.}
\end{itemize}
Que abana:«\textunderscore calção de abanante orelha\textunderscore », Garrett, \textunderscore Fáb.\textunderscore 
\section{Abanão}
\begin{itemize}
\item {Grp. gram.:m.}
\end{itemize}
\begin{itemize}
\item {Utilização:Pop.}
\end{itemize}
Acto de \textunderscore abanar\textunderscore  ou agitar com fôrça.
\section{Abanar}
\begin{itemize}
\item {Grp. gram.:v. t.}
\end{itemize}
\begin{itemize}
\item {Utilização:Fam.}
\end{itemize}
\begin{itemize}
\item {Grp. gram.:Loc.}
\end{itemize}
\begin{itemize}
\item {Utilização:fam.}
\end{itemize}
Ventilar com abano.
Agitar.
Sacudir.
Aproximar-se de (alguém), para que êste satisfaça um pedido, ou dê a conhecer intenções.
\textunderscore Abanar as orelhas\textunderscore , não annuir a um pedido ou proposta.
\section{Abancado}
\begin{itemize}
\item {Grp. gram.:m.}
\end{itemize}
\textunderscore Pedra de abancado\textunderscore , a pedra arrancada do último banco inferior da pedreira.
(De \textunderscore abancar\textunderscore ).
\section{Abancar}
\begin{itemize}
\item {Grp. gram.:v. t.}
\end{itemize}
\begin{itemize}
\item {Grp. gram.:V. i.}
\end{itemize}
Distribuir por lugares á roda da banca.
Sentar-se á banca.
\section{Abandalhado}
\begin{itemize}
\item {Grp. gram.:adj.}
\end{itemize}
Que se abandalhou.
(De \textunderscore abandalhar\textunderscore ).
\section{Abandalhar}
\begin{itemize}
\item {Grp. gram.:v. t.}
\end{itemize}
Tornar bandalho, reles.
\section{Abandar}
\begin{itemize}
\item {Grp. gram.:v. t.}
\end{itemize}
\begin{itemize}
\item {Grp. gram.:V. p.}
\end{itemize}
Reunir em bando.
Formar bando, bandear-se.
\section{Abandar}
\begin{itemize}
\item {Grp. gram.:v. t.}
\end{itemize}
Pôr de banda, separar, dar como quinhão: \textunderscore na partilha, o pai abandou-lhe duas belgas\textunderscore .
Pôr bandas em: \textunderscore abandar de velludo um capote\textunderscore . Cf. Camillo, \textunderscore Narcót.\textunderscore , II, 304.
\section{Abandear}
\begin{itemize}
\item {Grp. gram.:v. t.}
\end{itemize}
O mesmo que \textunderscore bandear\textunderscore ^1.
\section{Abandeirar}
\begin{itemize}
\item {Grp. gram.:v. t.}
\end{itemize}
O mesmo que \textunderscore embandeirar\textunderscore .
\section{Abandejar}
\begin{itemize}
\item {Grp. gram.:v. t.}
\end{itemize}
Dar fórma de bandeja a.
Limpar (cereaes), separando com bandeja o grão e a palha.
\section{Abandoar}
\begin{itemize}
\item {Grp. gram.:v. t.}
\end{itemize}
(V. \textunderscore bandear\textunderscore ^1).
\section{Abandonadamente}
\begin{itemize}
\item {Grp. gram.:adv.}
\end{itemize}
Com abandono, solitariamente.
Desamparadamente.
\section{Abandonado}
\begin{itemize}
\item {Grp. gram.:adj.}
\end{itemize}
Que ficou ao abandono.
Desamparado.
Solitário.
(De \textunderscore abandonar\textunderscore ).
\section{Abandonamento}
\begin{itemize}
\item {Grp. gram.:m.}
\end{itemize}
(V.abandono).
\section{Abandonar}
\begin{itemize}
\item {Grp. gram.:v. t.}
\end{itemize}
\begin{itemize}
\item {Grp. gram.:V. p.}
\end{itemize}
Deixar ao abandono.
Desamparar.
Largar.
Renunciar.
Entregar-se.
\section{Abandonável}
\begin{itemize}
\item {Grp. gram.:adj.}
\end{itemize}
Que deve sêr abandonado.
Que se póde abandonar.
\section{Abandono}
\begin{itemize}
\item {Grp. gram.:m.}
\end{itemize}
Desamparo, desabrigo.
Desleixo.
Desistência, renúncia.
\section{Abaneenga}
\begin{itemize}
\item {Grp. gram.:m.}
\end{itemize}
\begin{itemize}
\item {Utilização:Bras}
\end{itemize}
Guarani ou tupi do sul.
\section{Abanga}
\begin{itemize}
\item {Grp. gram.:f.}
\end{itemize}
Fruto de certa palmeira das Antilhas.
Designação genérica, que nalguns povos da América do Sul se dá á bananeira.
O mesmo que \textunderscore bango\textunderscore .
\section{Abanheenga}
\begin{itemize}
\item {Grp. gram.:m.}
\end{itemize}
\begin{itemize}
\item {Utilização:Bras}
\end{itemize}
Guarani ou tupi do sul.
\section{Abanicar}
\begin{itemize}
\item {Grp. gram.:v. i.}
\end{itemize}
Diz-se do toireiro, quando move o capote de um lado para o outro.
Abanar com leque.
(De \textunderscore abanico\textunderscore ).
\section{Abanico}
\begin{itemize}
\item {Grp. gram.:m.}
\end{itemize}
\begin{itemize}
\item {Utilização:Ant.}
\end{itemize}
Pequeno abano. Leque.
Gorjeira ou enfeite para o pescoço.
\section{Abaninho}
\begin{itemize}
\item {Grp. gram.:m.}
\end{itemize}
(V.abanico).
\section{Abano}
\begin{itemize}
\item {Grp. gram.:m.}
\end{itemize}
\begin{itemize}
\item {Grp. gram.:m.}
\end{itemize}
O mesmo que \textunderscore ventarola\textunderscore .
Utensílio, em fórma de leque, para activar a combustão, agitando o ar.
Modo de pesca, usado no Algarve, pondo-se no anzol um pedaço de pano branco, em vez de isca.
Guarnição de vestuário, espécie de folho pregueado.
\section{Abantesma}
\begin{itemize}
\item {fónica:tês}
\end{itemize}
\begin{itemize}
\item {Grp. gram.:f.}
\end{itemize}
\begin{itemize}
\item {Utilização:Pop.}
\end{itemize}
\begin{itemize}
\item {Utilização:Fig.}
\end{itemize}
O mesmo que \textunderscore fantasma\textunderscore .
Apparição medonha.
Objecto muito grande, espantoso.
\section{Abanto}
\begin{itemize}
\item {Grp. gram.:m.}
\end{itemize}
\begin{itemize}
\item {Grp. gram.:Adj.}
\end{itemize}
Espécie de abutre.
Cobarde, (falando-se do toiro).
\section{Abapo}
\begin{itemize}
\item {Grp. gram.:m.}
\end{itemize}
Planta exótica, amaryllidea.
\section{Abaquetar}
\begin{itemize}
\item {Grp. gram.:v. t.}
\end{itemize}
Dar fórma de baqueta a.
\section{Abar}
\begin{itemize}
\item {Grp. gram.:v. t.}
\end{itemize}
Formar as abas de (um chapéu).
\section{Abará}
\begin{itemize}
\item {Grp. gram.:m.}
\end{itemize}
\begin{itemize}
\item {Utilização:Bras}
\end{itemize}
Iguaria, feita de massa de feijão, pimenta e pijericum. (T. tupi).
\section{Abaratar}
\begin{itemize}
\item {Grp. gram.:v. t.}
\end{itemize}
Tornar barato, abaixar o preço de, baratear.
\section{Abarbado}
\begin{itemize}
\item {Grp. gram.:adj.}
\end{itemize}
Sobrecarregado, afrontado.
(De \textunderscore abarbar\textunderscore ).
\section{Abarbar}
\begin{itemize}
\item {Grp. gram.:v. t.}
\end{itemize}
\begin{itemize}
\item {Grp. gram.:V. i.}
\end{itemize}
Tocar com a barba, attingir.
Igualar.
Afrontar.
Igualar em altura.
Diz-se das abelhas, quando se reúnem em massa, fóra da colmeia, formando cacho.
\section{Abarbarado}
\begin{itemize}
\item {Grp. gram.:adj.}
\end{itemize}
\begin{itemize}
\item {Utilização:Bras}
\end{itemize}
Temerário.
Valente.
Terrivel.
(De \textunderscore bárbaro\textunderscore ).
\section{Abarbarizar}
\begin{itemize}
\item {Grp. gram.:v. t.}
\end{itemize}
(V.barbarizar).
\section{Abarbear}
\begin{itemize}
\item {Grp. gram.:v. t.}
\end{itemize}
Pôr barbilho em.
\section{Abarbelar}
\begin{itemize}
\item {Grp. gram.:v. t.}
\end{itemize}
Prender com barbella.
Abarbetar.
Levantar (a âncora) á altura da barbeta.
\section{Abarbellar}
\begin{itemize}
\item {Grp. gram.:v. t.}
\end{itemize}
Prender com barbella.
Abarbetar.
Levantar (a âncora) á altura da barbeta.
\section{Abarca}
\begin{itemize}
\item {Grp. gram.:f.}
\end{itemize}
Calçado, feito de uma sola, ligada ao pé por correias.
Tamanco.
Calçado largo e mal feito.
(Cp. \textunderscore abarcar\textunderscore ).
\section{Abarcador}
\begin{itemize}
\item {Grp. gram.:m.}
\end{itemize}
Aquelle que abarca.
\section{Abarcamento}
\begin{itemize}
\item {Grp. gram.:m.}
\end{itemize}
Acto ou effeito de \textunderscore abarcar\textunderscore .
\section{Abarcante}
\begin{itemize}
\item {Grp. gram.:adj.}
\end{itemize}
Que abarca.
\section{Abarcar}
\begin{itemize}
\item {Grp. gram.:v. t.}
\end{itemize}
\begin{itemize}
\item {Proveniência:(Do cast. \textunderscore abarcar\textunderscore ?)}
\end{itemize}
Abranger, cingir.
Conter.
Monopolizar.
\section{Abarcas}
\begin{itemize}
\item {Grp. gram.:f. pl.}
\end{itemize}
\begin{itemize}
\item {Utilização:Prov.}
\end{itemize}
\begin{itemize}
\item {Utilização:alg.}
\end{itemize}
Luta, braço a braço, para experiência de fôrças.
(De \textunderscore abarcar\textunderscore ).
\section{Abarém}
\begin{itemize}
\item {Grp. gram.:m.}
\end{itemize}
\begin{itemize}
\item {Utilização:Bras}
\end{itemize}
Espécie de bolo de milho ou de arroz moído.
\section{Abaremo-temo}
\begin{itemize}
\item {Grp. gram.:m.}
\end{itemize}
Árvore leguminosa do Brasil.
\section{Abarga}
\begin{itemize}
\item {Grp. gram.:f.}
\end{itemize}
\begin{itemize}
\item {Utilização:Ant.}
\end{itemize}
Espécie de nassa.
\section{Abaritonado}
\begin{itemize}
\item {Grp. gram.:adj.}
\end{itemize}
Semelhante a barítono: \textunderscore um tenor abaritonado\textunderscore .
\section{Abarracado}
\begin{itemize}
\item {Grp. gram.:adj.}
\end{itemize}
Que tem fórma de barraca: \textunderscore casa abarracada\textunderscore .
(De \textunderscore abarracar\textunderscore ).
\section{Abarracamento}
\begin{itemize}
\item {Grp. gram.:m.}
\end{itemize}
Acto ou effeito de \textunderscore abarracar\textunderscore .
Série de barracas.
Lugar onde há barracas.
\section{Abarracar}
\begin{itemize}
\item {Grp. gram.:v. t.}
\end{itemize}
Formar barracas em.
Meter em barracas.
Dar fórma de barraca a.
\section{Abarrancar}
\begin{itemize}
\item {Grp. gram.:v. i.}
\end{itemize}
\begin{itemize}
\item {Grp. gram.:V. p.}
\end{itemize}
Fazer barrancos em.
Meter-se em barrancos.
\section{Abarregado}
\begin{itemize}
\item {Grp. gram.:adj.}
\end{itemize}
\begin{itemize}
\item {Utilização:Topogr.}
\end{itemize}
Diz-se do casal ou herdade, em que o dono não reside e que por isso está exposto ao roubo.
\section{Abarregar}
\begin{itemize}
\item {Grp. gram.:v. t.}
\end{itemize}
\begin{itemize}
\item {Utilização:Ant.}
\end{itemize}
\begin{itemize}
\item {Grp. gram.:V. p.}
\end{itemize}
Amancebar.
Têr relações illicitas com pessôa de outro sexo.
(Cp. \textunderscore barregan\textunderscore ).
\section{Abarreirar}
\begin{itemize}
\item {Grp. gram.:v. t.}
\end{itemize}
Cercar de barreiras.
Entrincheirar.
\section{Abarretar}
\begin{itemize}
\item {Grp. gram.:v. t.}
\end{itemize}
Cobrir com barrete.
\section{Abarroado}
\begin{itemize}
\item {Grp. gram.:adj.}
\end{itemize}
\begin{itemize}
\item {Utilização:Ant.}
\end{itemize}
Malcriado.
Indecente.
(De \textunderscore barrão\textunderscore ).
\section{Abarrotado}
\begin{itemize}
\item {Grp. gram.:adj.}
\end{itemize}
\begin{itemize}
\item {Utilização:Ant.}
\end{itemize}
Muito cheio.
Empanzinado.
Contumaz, renitente, teimoso.
(De \textunderscore abarrotar\textunderscore ).
\section{Abarrotamento}
\begin{itemize}
\item {Grp. gram.:m.}
\end{itemize}
Acto ou effeito de abarrotar.
\section{Abarrotar}
\begin{itemize}
\item {Grp. gram.:v. t.}
\end{itemize}
Cobrir de barrotes.
Encher muito.
Empanturrar.
\section{Abarruntar}
\begin{itemize}
\item {Grp. gram.:v. t.}
\end{itemize}
\begin{itemize}
\item {Utilização:Prov.}
\end{itemize}
\begin{itemize}
\item {Utilização:trasm.}
\end{itemize}
Lobrigar, dar fé de.
(De \textunderscore barruntar\textunderscore ).
\section{Abaruna}
\begin{itemize}
\item {Grp. gram.:m.}
\end{itemize}
\begin{itemize}
\item {Utilização:Bras}
\end{itemize}
Homem, vestido de negro; padre. Cf. Alencar, \textunderscore Minas de Prata\textunderscore , 111.
\section{Abasia}
\begin{itemize}
\item {Grp. gram.:f.}
\end{itemize}
\begin{itemize}
\item {Utilização:Med.}
\end{itemize}
\begin{itemize}
\item {Proveniência:(Do gr. \textunderscore a\textunderscore  priv. + \textunderscore basis\textunderscore , o andar)}
\end{itemize}
Impossibilidade da marcha.
\section{Abasicarpo}
\begin{itemize}
\item {Grp. gram.:m.}
\end{itemize}
\begin{itemize}
\item {Proveniência:(Do gr. \textunderscore a\textunderscore  + \textunderscore basis\textunderscore  + \textunderscore karpos\textunderscore )}
\end{itemize}
Planta crucífera
\section{Abásico}
\begin{itemize}
\item {Grp. gram.:adj.}
\end{itemize}
\begin{itemize}
\item {Grp. gram.:M.}
\end{itemize}
Relativo a abasia.
Aquelle que soffre abasia.
\section{Abassis}
\begin{itemize}
\item {Grp. gram.:m. pl.}
\end{itemize}
\begin{itemize}
\item {Utilização:Ant.}
\end{itemize}
O mesmo que [[abexins|abexim]]. Cf. \textunderscore Lusiadas\textunderscore , X, 95.
\section{Abassor}
\begin{itemize}
\item {Grp. gram.:adj.}
\end{itemize}
\begin{itemize}
\item {Utilização:Anat.}
\end{itemize}
\begin{itemize}
\item {Proveniência:(Do gr. \textunderscore bassus\textunderscore )}
\end{itemize}
Diz-se de vários músculos, que abaixam.
\section{Abastadamente}
\begin{itemize}
\item {Grp. gram.:adv.}
\end{itemize}
De modo \textunderscore abastado\textunderscore , com abastança.
\section{Abastado}
\begin{itemize}
\item {Grp. gram.:adj.}
\end{itemize}
Que tem o bastante ou o que é necessário.
Rico.
(De \textunderscore abastar\textunderscore ).
\section{Abastamente}
\begin{itemize}
\item {Grp. gram.:adv.}
\end{itemize}
O mesmo que \textunderscore abastadamente\textunderscore .
\section{Abastamento}
\begin{itemize}
\item {Grp. gram.:m.}
\end{itemize}
Acto ou effeito de \textunderscore abastar.\textunderscore 
Grande fornecimento.
\section{Abastança}
\begin{itemize}
\item {Grp. gram.:f.}
\end{itemize}
Sufficiência.
Fartura; riqueza.
(De \textunderscore abastar\textunderscore ).
\section{Abastar}
\begin{itemize}
\item {Grp. gram.:v. t.}
\end{itemize}
\begin{itemize}
\item {Grp. gram.:V. i.}
\end{itemize}
Prover do que é bastante ou necessário.
Abastecer.
Fartar.
Sêr bastante:«\textunderscore um coxim abastará.\textunderscore »Camões. \textunderscore Seleuco\textunderscore .
\section{Abastardado}
\begin{itemize}
\item {Grp. gram.:adj.}
\end{itemize}
\begin{itemize}
\item {Utilização:Fig.}
\end{itemize}
\begin{itemize}
\item {Proveniência:(De \textunderscore abastardar\textunderscore )}
\end{itemize}
Degenerado.
Que não é puro ou correcto.
\section{Abastardamento}
\begin{itemize}
\item {Grp. gram.:m.}
\end{itemize}
Acto ou effeito de \textunderscore abastardar\textunderscore .
Degenerescência.
\section{Abastardar}
\begin{itemize}
\item {Grp. gram.:v. t.}
\end{itemize}
\begin{itemize}
\item {Proveniência:(De \textunderscore bastardo\textunderscore )}
\end{itemize}
Alterar.
Fazer degenerar.
\section{Abastardear}
\begin{itemize}
\item {Grp. gram.:v. t.}
\end{itemize}
O mesmo que \textunderscore abastardar\textunderscore . Cf. Garrett, \textunderscore Port. na Bal.\textunderscore , 33.
\section{Abastecedor}
\begin{itemize}
\item {Grp. gram.:m.}
\end{itemize}
Aquelle que abastece.
\section{Abastecer}
\begin{itemize}
\item {Grp. gram.:v. t.}
\end{itemize}
O mesmo que \textunderscore abastar\textunderscore .
\section{Abastecimento}
\begin{itemize}
\item {Grp. gram.:m.}
\end{itemize}
Acto de \textunderscore abastecer\textunderscore .
Abastança.
\section{Abastimento}
\begin{itemize}
\item {Grp. gram.:m.}
\end{itemize}
\begin{itemize}
\item {Utilização:Prov.}
\end{itemize}
\begin{itemize}
\item {Utilização:trasm.}
\end{itemize}
Aviamento, execução.
\section{Abasto}
\begin{itemize}
\item {Grp. gram.:m.}
\end{itemize}
\begin{itemize}
\item {Utilização:Ant.}
\end{itemize}
O mesmo que \textunderscore abastança\textunderscore .
\section{Abastosamente}
\begin{itemize}
\item {Grp. gram.:adv.}
\end{itemize}
De modo \textunderscore abastoso\textunderscore , com abundância.
\section{Abastoso}
\begin{itemize}
\item {Grp. gram.:adj.}
\end{itemize}
Rico.
Abundante.
(De \textunderscore abasto\textunderscore ).
\section{Abatatado}
\begin{itemize}
\item {Grp. gram.:adj.}
\end{itemize}
Que tem fórma de batata.
Grosso, largo: \textunderscore nariz abatatado\textunderscore .
(De \textunderscore abatatar\textunderscore ).
\section{Abatatar}
\begin{itemize}
\item {Grp. gram.:v. t.}
\end{itemize}
Dar fórma de batata a.
Tornar largo, disforme.
\section{Abate}
\begin{itemize}
\item {Grp. gram.:m.}
\end{itemize}
O mesmo que \textunderscore abatimento\textunderscore .
\section{Abatedor}
\begin{itemize}
\item {Grp. gram.:m.}
\end{itemize}
Aquelle que abate.
\section{Abater}
\begin{itemize}
\item {Grp. gram.:v. t.}
\end{itemize}
\begin{itemize}
\item {Grp. gram.:V. i.}
\end{itemize}
\begin{itemize}
\item {Proveniência:(Do lat. \textunderscore adbattere\textunderscore )}
\end{itemize}
Abaixar.
Prostrar.
Humilhar.
Deminuir (preço, altura, etc.).
Têr abatimento.
Desabar: \textunderscore o telhado abateu\textunderscore .
\section{Abátia}
\begin{itemize}
\item {Grp. gram.:f.}
\end{itemize}
Gênero de plantas arbustivas da América.
\section{Abatidamente}
\begin{itemize}
\item {Grp. gram.:adv.}
\end{itemize}
Do modo \textunderscore abatido\textunderscore .
\section{Abatido}
\begin{itemize}
\item {Grp. gram.:adj.}
\end{itemize}
Enfraquecido, definhado.
(De \textunderscore abater\textunderscore ).
\section{Abatimento}
\begin{itemize}
\item {Grp. gram.:m.}
\end{itemize}
\begin{itemize}
\item {Utilização:Náut.}
\end{itemize}
Acto ou effeito de \textunderscore abater\textunderscore .
Prostração, fraqueza.
Desconto, reducção.
Ângulo, feito pela quilha com a esteira, indicando quanto o navio se desvia para o través.
\section{Abatina}
\begin{itemize}
\item {Grp. gram.:f.}
\end{itemize}
O mesmo que \textunderscore batina\textunderscore .
\section{Abatinar}
\begin{itemize}
\item {Grp. gram.:v. t.}
\end{itemize}
Vestir com batina.
\section{Abatirás}
\begin{itemize}
\item {Grp. gram.:m. pl.}
\end{itemize}
Aborígenes brasileiros, que dominaram na antiga capitania de Porto-Seguro.
\section{Abatis}
\begin{itemize}
\item {Grp. gram.:m.}
\end{itemize}
Trincheira, formada de árvores cortadas.
\section{Abatis}
\begin{itemize}
\item {Grp. gram.:m.}
\end{itemize}
\begin{itemize}
\item {Utilização:Bras}
\end{itemize}
Espécie de cabidela, feita de entranhas de aves.
\section{Abatixi}
\begin{itemize}
\item {Grp. gram.:m.}
\end{itemize}
\begin{itemize}
\item {Utilização:Bras}
\end{itemize}
Planta aquática do Amazonas.
\section{Abatocar}
\begin{itemize}
\item {Grp. gram.:v. t.}
\end{itemize}
Fechar com batoque.
Arrolhar.
\section{Abatufado}
\begin{itemize}
\item {Grp. gram.:adj.}
\end{itemize}
\begin{itemize}
\item {Utilização:Prov.}
\end{itemize}
\begin{itemize}
\item {Utilização:trasm.}
\end{itemize}
Que tem gordura balofa, mormente na cara.
(Cp. [[empantufado|empantufar-se]]).
\section{Abauladamente}
\begin{itemize}
\item {fónica:ba-u}
\end{itemize}
\begin{itemize}
\item {Grp. gram.:adv.}
\end{itemize}
De modo \textunderscore abahulado\textunderscore , á maneira de bahu.
\section{Abaulado}
\begin{itemize}
\item {fónica:baú}
\end{itemize}
\begin{itemize}
\item {Grp. gram.:adj.}
\end{itemize}
Que tem fórma de bahu.
\section{Abaulador}
\begin{itemize}
\item {fónica:baú}
\end{itemize}
\begin{itemize}
\item {Grp. gram.:m.}
\end{itemize}
O que abahula.
\section{Abaulamento}
\begin{itemize}
\item {fónica:baú}
\end{itemize}
\begin{itemize}
\item {Grp. gram.:m.}
\end{itemize}
Acto ou effeito de \textunderscore abahular\textunderscore .
Arco abatido nas abóbadas.
Superfície curva das ruas calçadas, para se facilitar o escoamento das águas.
Convexidade.
\section{Abaular}
\begin{itemize}
\item {fónica:baú}
\end{itemize}
\begin{itemize}
\item {Grp. gram.:v. t.}
\end{itemize}
Dar fórma de bahu a; tornar convexo.
\section{Abaúna}
\begin{itemize}
\item {Grp. gram.:adj. f.}
\end{itemize}
Diz-se de uma raça autóchthone do Brasil.
\section{Abaxial}
\begin{itemize}
\item {Grp. gram.:adj.}
\end{itemize}
\begin{itemize}
\item {Utilização:Ópt.}
\end{itemize}
Que não está no eixo.
(De \textunderscore ab\textunderscore  priv. + lat. \textunderscore axis\textunderscore , eixo).
\section{Abba}
\begin{itemize}
\item {Grp. gram.:m.}
\end{itemize}
\begin{itemize}
\item {Proveniência:(Do syr. \textunderscore abba\textunderscore )}
\end{itemize}
Título do Bispo em algumas Igrejas orientaes.
\section{Abbacial}
\begin{itemize}
\item {Grp. gram.:adj.}
\end{itemize}
\begin{itemize}
\item {Proveniência:(Do b. lat. \textunderscore abbatialis\textunderscore )}
\end{itemize}
Que diz respeito ao abbade ou á abbadia.
\section{Abbadado}
\begin{itemize}
\item {Grp. gram.:m.}
\end{itemize}
\begin{itemize}
\item {Utilização:Ant.}
\end{itemize}
O mesmo que \textunderscore abbadia\textunderscore .
\section{Abbadágio}
\begin{itemize}
\item {Grp. gram.:m.}
\end{itemize}
Refeição, que os parochianos eram obrigados a dar ao abbade.
\section{Abbadar}
\begin{itemize}
\item {Grp. gram.:v. t.}
\end{itemize}
Prover de abbade.
\section{Abbade}
\begin{itemize}
\item {Grp. gram.:m.}
\end{itemize}
\begin{itemize}
\item {Proveniência:(Do b. lat. \textunderscore abbas\textunderscore )}
\end{itemize}
Prelado de ordem monástica.
Aquelle que governa uma abbadia.
Cura de almas, párocho.
\section{Abbadecídio}
\begin{itemize}
\item {Grp. gram.:m.}
\end{itemize}
Assassinio de abbade. Cf. B. Pato, \textunderscore Ciprestes\textunderscore , 107.
\section{Abbadengo}
\begin{itemize}
\item {Grp. gram.:m.}
\end{itemize}
\begin{itemize}
\item {Utilização:Ant.}
\end{itemize}
\begin{itemize}
\item {Grp. gram.:Adj.}
\end{itemize}
Legado pio ou esmola, que se dava em vida ou se deixava por morte ao confessor.
O antigo direito de sêr abbade de alguma igreja.
Relativo ao território ou á jurisdicção de abbade.
\section{Abbadesco}
\begin{itemize}
\item {Grp. gram.:adj.}
\end{itemize}
Próprio de abbade.
\section{Abbadessa}
\begin{itemize}
\item {Grp. gram.:f.}
\end{itemize}
\begin{itemize}
\item {Proveniência:(Do b. lat. \textunderscore abbatissa\textunderscore )}
\end{itemize}
Prelada de communidade religiosa.
\section{Abbadessado}
\begin{itemize}
\item {Grp. gram.:m.}
\end{itemize}
Cargo de abbadessa.
O tempo que dura esse cargo.
As festas, com que se elege a abbadessa.
\section{Abbadessona}
\begin{itemize}
\item {Grp. gram.:f.}
\end{itemize}
\begin{itemize}
\item {Utilização:Fam.}
\end{itemize}
Abbadessa muito encorpada. Cf. Garrett, \textunderscore Flores\textunderscore , 95.
\section{Abbadia}
\begin{itemize}
\item {Grp. gram.:f.}
\end{itemize}
Igreja, regida por abbade.
Paróchia.
Rendimento do abbade.
\section{Abbadiado}
\begin{itemize}
\item {Grp. gram.:m.}
\end{itemize}
O mesmo que \textunderscore abbadia\textunderscore .
\section{Abbadim}
\begin{itemize}
\item {Grp. gram.:m.}
\end{itemize}
\begin{itemize}
\item {Utilização:Ant.}
\end{itemize}
Lugar, habitado por observantes religiosos.
(De \textunderscore abbade\textunderscore ).
\section{Abbatina}
\begin{itemize}
\item {Grp. gram.:f.}
\end{itemize}
O mesmo que \textunderscore batina\textunderscore .
\section{Abc}
\begin{itemize}
\item {fónica:á-bê-cê}
\end{itemize}
\begin{itemize}
\item {Grp. gram.:m.}
\end{itemize}
Abecedário; cartinha para aprender a lêr.
Primeiras noções de qualquer coisa.
(Das primeiras letras do alphabeto).
\section{Abceder}
\begin{itemize}
\item {Grp. gram.:v. i.}
\end{itemize}
\begin{itemize}
\item {Proveniência:(Do lat. \textunderscore abscedere\textunderscore )}
\end{itemize}
Degenerar em abcesso.
Supurar.
\section{Abcesso}
\begin{itemize}
\item {Grp. gram.:m.}
\end{itemize}
\begin{itemize}
\item {Proveniência:(Do lat. \textunderscore abscessus\textunderscore )}
\end{itemize}
Tumor.
Inchação, produzida pela formação do pus.
Pus, acumulado no tumor.
\section{Abcisão}
\begin{itemize}
\item {Grp. gram.:f.}
\end{itemize}
\begin{itemize}
\item {Proveniência:(Do lat. \textunderscore abscisio\textunderscore )}
\end{itemize}
Córte na parte carnosa do corpo.
\section{Abcissa}
\begin{itemize}
\item {Grp. gram.:f.}
\end{itemize}
\begin{itemize}
\item {Utilização:Geom.}
\end{itemize}
\begin{itemize}
\item {Proveniência:(Do lat. \textunderscore abscissa\textunderscore )}
\end{itemize}
Uma das coordenadas que servem para fixar um ponto num plano.
\section{Abdalita}
\begin{itemize}
\item {Grp. gram.:m.}
\end{itemize}
\begin{itemize}
\item {Proveniência:(Do ar. \textunderscore abol\textunderscore , servo, e \textunderscore Allah\textunderscore , Deus)}
\end{itemize}
Membro de uma seita indiana.
\section{Abderiano}
\begin{itemize}
\item {Grp. gram.:m.  e  adj.}
\end{itemize}
De Abdera.
\section{Abderita}
\begin{itemize}
\item {Grp. gram.:m.  e  adj.}
\end{itemize}
De Abdera.
\section{Abdicação}
\begin{itemize}
\item {Grp. gram.:f.}
\end{itemize}
\begin{itemize}
\item {Proveniência:(Do lat. \textunderscore abdicatio\textunderscore )}
\end{itemize}
Acto ou effeito de abdicar.
\section{Abdicador}
\begin{itemize}
\item {Grp. gram.:m.}
\end{itemize}
Aquelle que abdica.
\section{Abdicante}
\begin{itemize}
\item {Grp. gram.:m.}
\end{itemize}
O mesmo que \textunderscore abdicador\textunderscore .
\section{Abdicar}
\begin{itemize}
\item {Grp. gram.:v. t.}
\end{itemize}
\begin{itemize}
\item {Proveniência:(Do lat. \textunderscore abdicare\textunderscore .)}
\end{itemize}
Renunciar.
Abandonar (um cargo).
Ceder.
\section{Abdicatário}
\begin{itemize}
\item {Grp. gram.:m.  e  adj.}
\end{itemize}
O que fez abdicação.
\section{Abdicativo}
\begin{itemize}
\item {Grp. gram.:adj.}
\end{itemize}
Relativo a abdicação.
Que envolve abdicação: \textunderscore declaração abdicativa\textunderscore .
\section{Abdicavel}
\begin{itemize}
\item {Grp. gram.:adj.}
\end{itemize}
Que se póde abdicar.
\section{Ábdito}
\begin{itemize}
\item {Grp. gram.:adj.}
\end{itemize}
\begin{itemize}
\item {Utilização:Des.}
\end{itemize}
\begin{itemize}
\item {Proveniência:(Do lat. \textunderscore abditus\textunderscore )}
\end{itemize}
Escondido.
Afastado.
\section{Abdome}
\begin{itemize}
\item {fónica:dô}
\end{itemize}
\begin{itemize}
\item {Grp. gram.:m.}
\end{itemize}
\begin{itemize}
\item {Proveniência:(Do lat. \textunderscore abdomen\textunderscore )}
\end{itemize}
Uma das cavidades do tronco do homem, limitada inferiormente pela bacia e superiormente pelo diaphragma.
Ventre, barriga.
\section{Abdômen}
\begin{itemize}
\item {Grp. gram.:m.}
\end{itemize}
(V.abdome).
\section{Abdominaes}
\begin{itemize}
\item {Grp. gram.:m. pl.}
\end{itemize}
\begin{itemize}
\item {Utilização:Ichthyol.}
\end{itemize}
Ordem de peixes malacopterygios, de barbatanas suspensas abaixo do abdome.
\section{Abdominais}
\begin{itemize}
\item {Grp. gram.:m. pl.}
\end{itemize}
\begin{itemize}
\item {Utilização:Ichthyol.}
\end{itemize}
Ordem de peixes malacopterygios, de barbatanas suspensas abaixo do abdome.
\section{Abdominal}
\begin{itemize}
\item {Grp. gram.:adj.}
\end{itemize}
\begin{itemize}
\item {Proveniência:(Do lat. \textunderscore abdominalis\textunderscore )}
\end{itemize}
Relativo ao abdome.
\section{Abdominoscopia}
\begin{itemize}
\item {Grp. gram.:f.}
\end{itemize}
\begin{itemize}
\item {Proveniência:(Do lat. \textunderscore abdomen\textunderscore  + gr. \textunderscore skopein\textunderscore , ver)}
\end{itemize}
Observação do abdome pelo tacto e pela percussão.
\section{Abdominoso}
\begin{itemize}
\item {Grp. gram.:adj.}
\end{itemize}
Que tem grande abdome.
Barrigudo.
\section{Abdominothorácico}
\begin{itemize}
\item {Grp. gram.:adj.}
\end{itemize}
\begin{itemize}
\item {Utilização:Anat.}
\end{itemize}
Relativo ao abdome e ao thórax.
\section{Abdominotorácico}
\begin{itemize}
\item {Grp. gram.:adj.}
\end{itemize}
\begin{itemize}
\item {Utilização:Anat.}
\end{itemize}
Relativo ao abdome e ao thórax.
\section{Abdução}
\begin{itemize}
\item {Grp. gram.:f.}
\end{itemize}
\begin{itemize}
\item {Utilização:Anat.}
\end{itemize}
\begin{itemize}
\item {Proveniência:(Do lat. \textunderscore abductio\textunderscore )}
\end{itemize}
Movimento, com que, do plano médio, que se suppõe dividir o corpo humano em duas partes iguaes, se afasta um membro ou qualquer parte.
\section{Abducção}
\begin{itemize}
\item {Grp. gram.:f.}
\end{itemize}
\begin{itemize}
\item {Utilização:Anat.}
\end{itemize}
\begin{itemize}
\item {Proveniência:(Do lat. \textunderscore abductio\textunderscore )}
\end{itemize}
Movimento, com que, do plano médio, que se suppõe dividir o corpo humano em duas partes iguaes, se afasta um membro ou qualquer parte.
\section{Abducente}
\begin{itemize}
\item {Grp. gram.:adj.}
\end{itemize}
\begin{itemize}
\item {Proveniência:(Do lat. \textunderscore abducens\textunderscore )}
\end{itemize}
Que produz a abducção.
\section{Abductivo}
\begin{itemize}
\item {Grp. gram.:adj.}
\end{itemize}
Que serve para abduzir; que abduz.
Que se abduz.
\section{Abductor}
\begin{itemize}
\item {Grp. gram.:m.}
\end{itemize}
\begin{itemize}
\item {Utilização:Anat.}
\end{itemize}
\begin{itemize}
\item {Proveniência:(Do lat. \textunderscore abductor\textunderscore )}
\end{itemize}
Músculo, que produz a abducção.
\section{Abdutivo}
\begin{itemize}
\item {Grp. gram.:adj.}
\end{itemize}
Que serve para abduzir; que abduz.
Que se abduz.
\section{Abdutor}
\begin{itemize}
\item {Grp. gram.:m.}
\end{itemize}
\begin{itemize}
\item {Utilização:Anat.}
\end{itemize}
\begin{itemize}
\item {Proveniência:(Do lat. \textunderscore abductor\textunderscore )}
\end{itemize}
Músculo, que produz a abducção.
\section{Abduzir}
\begin{itemize}
\item {Grp. gram.:v. t.}
\end{itemize}
\begin{itemize}
\item {Proveniência:(Do lat. \textunderscore abducere\textunderscore )}
\end{itemize}
Tirar com fôrça.
Desviar de um ponto, afastar.
\section{Abeatador}
\begin{itemize}
\item {Grp. gram.:adj.}
\end{itemize}
Que tem modos de beato.
(De \textunderscore abeatar\textunderscore ).
\section{Abeatar}
\begin{itemize}
\item {Grp. gram.:v. t.}
\end{itemize}
Tornar beato.
Dar modos de beato a.
\section{Abêbera}
\begin{itemize}
\item {Grp. gram.:f.}
\end{itemize}
O mesmo que \textunderscore bêbera\textunderscore .
\section{Abeberar}
\begin{itemize}
\item {Grp. gram.:v. t.}
\end{itemize}
Dar de beber a.
Regar.
Ensopar.
(Corresponde ao cast. \textunderscore abrevar\textunderscore , fr. \textunderscore abreuver\textunderscore , it. \textunderscore abbederare\textunderscore , do lat. \textunderscore ad\textunderscore  + \textunderscore bibere\textunderscore . Cf. Viana, \textunderscore Apost.\textunderscore , vb. \textunderscore baforeira\textunderscore ).
\section{Abebra}
\begin{itemize}
\item {fónica:bê}
\end{itemize}
\begin{itemize}
\item {Grp. gram.:f.}
\end{itemize}
(V.bêbera).
\section{Abecar}
\begin{itemize}
\item {Grp. gram.:v. t.}
\end{itemize}
\begin{itemize}
\item {Utilização:Bras}
\end{itemize}
Segurar pela golla, prender.
\section{Abecê}
\begin{itemize}
\item {Grp. gram.:m.}
\end{itemize}
O mesmo ou melhor que \textunderscore abc\textunderscore .
\section{Abecedário}
\begin{itemize}
\item {Grp. gram.:m.}
\end{itemize}
\begin{itemize}
\item {Proveniência:(Do lat. \textunderscore abecedarius\textunderscore )}
\end{itemize}
Alphabeto.
Livrinho para aprender a lêr.
\section{Abechedo}
\begin{itemize}
\item {Grp. gram.:m.}
\end{itemize}
\begin{itemize}
\item {Utilização:Prov.}
\end{itemize}
\begin{itemize}
\item {Utilização:trasm.}
\end{itemize}
O mesmo que \textunderscore abicheiro\textunderscore .
\section{Abechucho}
\begin{itemize}
\item {Grp. gram.:m.}
\end{itemize}
\begin{itemize}
\item {Utilização:Prov.}
\end{itemize}
\begin{itemize}
\item {Utilização:trasm.}
\end{itemize}
Pessoa muito encorpada e desageitada.
\section{Abecoínha}
\begin{itemize}
\item {Grp. gram.:f.}
\end{itemize}
\begin{itemize}
\item {Utilização:Prov.}
\end{itemize}
O mesmo que \textunderscore abibe\textunderscore .
\section{Abegão}
\begin{itemize}
\item {Grp. gram.:m.}
\end{itemize}
\begin{itemize}
\item {Utilização:Prov.}
\end{itemize}
\begin{itemize}
\item {Utilização:alent.}
\end{itemize}
\begin{itemize}
\item {Utilização:Prov.}
\end{itemize}
\begin{itemize}
\item {Utilização:minh.}
\end{itemize}
\begin{itemize}
\item {Proveniência:(Do lat. hyp. \textunderscore pecudonem\textunderscore , de \textunderscore pecus\textunderscore , gado)}
\end{itemize}
Aquelle que trata de abegoaria.
Feitor de herdade ou quinta.
Carpinteiro de carros.
O mesmo que \textunderscore abelhão\textunderscore .
\section{Abegôa}
(\textunderscore fem.\textunderscore  de \textunderscore abegão\textunderscore ).
\section{Abegoaria}
\begin{itemize}
\item {Grp. gram.:f.}
\end{itemize}
Lugar, em que se guarda gado e utensilios de lavoira ou carros.
(De \textunderscore abegão\textunderscore ).
\section{Abegoira}
\begin{itemize}
\item {Grp. gram.:f.}
\end{itemize}
\begin{itemize}
\item {Utilização:Ant.}
\end{itemize}
Sementeira.
Lavoira.
(De \textunderscore abegão\textunderscore ).
\section{Abegoura}
\begin{itemize}
\item {Grp. gram.:f.}
\end{itemize}
\begin{itemize}
\item {Utilização:Ant.}
\end{itemize}
Sementeira.
Lavoira.
(De \textunderscore abegão\textunderscore ).
\section{Abeirar}
\begin{itemize}
\item {Grp. gram.:v. t.}
\end{itemize}
Chegar á beira de, aproximar.
\section{Abeixamim}
\begin{itemize}
\item {Grp. gram.:m.}
\end{itemize}
\begin{itemize}
\item {Utilização:Ant.}
\end{itemize}
Espécie de tecido indiano.
\section{Abelha}
\begin{itemize}
\item {fónica:bê}
\end{itemize}
\begin{itemize}
\item {Grp. gram.:f.}
\end{itemize}
\begin{itemize}
\item {Utilização:Fig.}
\end{itemize}
\begin{itemize}
\item {Proveniência:(Do lat. \textunderscore apicula\textunderscore )}
\end{itemize}
Insecto hymenóptero, que produz o mel e a cera.
\textunderscore Abelha mestra\textunderscore , a que preside a cada colmeia e á qual compete a propagação da espécie.
Mulher astuta.
Planta, o mesmo que \textunderscore erva-abelha\textunderscore .
\section{Abelha-flôr}
\begin{itemize}
\item {Grp. gram.:f.}
\end{itemize}
Designação vulgar de uma espécie de orchídeas.
\section{Abelhal}
\begin{itemize}
\item {Grp. gram.:f.}
\end{itemize}
Casta de uva branca, nas regiões duriense e trasm.ntana.
\section{Abelhão}
\begin{itemize}
\item {Grp. gram.:m.}
\end{itemize}
O mesmo que \textunderscore zângão\textunderscore .
\section{Abelhar}
\begin{itemize}
\item {Grp. gram.:v. i.}
\end{itemize}
\begin{itemize}
\item {Utilização:Prov.}
\end{itemize}
Concorrer com uma quantia qualquer para um fim de interesse commum.
(De \textunderscore abelha\textunderscore ).
\section{Abelharuco}
\begin{itemize}
\item {Grp. gram.:m.}
\end{itemize}
O mesmo que \textunderscore abelheiro\textunderscore .
(Cp. cast. \textunderscore abejaruco\textunderscore ).
\section{Abelheira}
\begin{itemize}
\item {Grp. gram.:f.}
\end{itemize}
Ninho de abelhas.
Planta papilionácea.
Buraco, que apparece nas pedras e mármores, semelhante aos que as abelhas fazem no tronco das árvores.
\section{Abelheiro}
\begin{itemize}
\item {Grp. gram.:m.}
\end{itemize}
\begin{itemize}
\item {Proveniência:(De \textunderscore abelha\textunderscore )}
\end{itemize}
Homem, que trata de abelhas.
Ave, que se nutre de abelhas o que é também conhecida por \textunderscore abelharuco\textunderscore , (\textunderscore merops apiaster\textunderscore ).
\section{Abelhudamente}
\begin{itemize}
\item {Grp. gram.:adv.}
\end{itemize}
De modo \textunderscore abelhudo\textunderscore .
\section{Abelhudice}
\begin{itemize}
\item {Grp. gram.:f.}
\end{itemize}
Qualidade ou acto de \textunderscore abelhudo\textunderscore .
\section{Abelhudo}
\begin{itemize}
\item {Grp. gram.:adj.}
\end{itemize}
\begin{itemize}
\item {Proveniência:(De \textunderscore abelha\textunderscore )}
\end{itemize}
Atrevido, intrometido.
Apressado.
\section{Abélia}
\begin{itemize}
\item {Grp. gram.:f.}
\end{itemize}
\begin{itemize}
\item {Proveniência:(De \textunderscore Abel\textunderscore , n. p.)}
\end{itemize}
Gênero de árvores lonicéreas.
\section{Abeliceia}
\begin{itemize}
\item {Grp. gram.:f.}
\end{itemize}
Espécie de sândalo, de Creta.
\section{Abelidar-se}
\begin{itemize}
\item {Grp. gram.:v. p.}
\end{itemize}
Adquirir belidas.
\section{Abellota}
\begin{itemize}
\item {Grp. gram.:f.}
\end{itemize}
O mesmo que \textunderscore bellota\textunderscore .
\section{Abelmeluco}
\begin{itemize}
\item {Grp. gram.:m.}
\end{itemize}
Planta da Mauritânia, de sementes negras e oblongas.
\section{Abelmosco}
\begin{itemize}
\item {fónica:môs}
\end{itemize}
\begin{itemize}
\item {Grp. gram.:m.}
\end{itemize}
\begin{itemize}
\item {Proveniência:(Do ar. \textunderscore habb-el-misc\textunderscore . Cp. \textunderscore almíscar\textunderscore )}
\end{itemize}
Semente odorífera, de que se fabricam os chamados \textunderscore pós de Chypre\textunderscore .
\section{Abeloira}
\begin{itemize}
\item {Grp. gram.:f.}
\end{itemize}
Planta vivaz, de flôr raiada de preto e vermelho, em que as abelhas se demoram, deixando uma substância mellífera, que os rapazes procuram e apreciam.
(Por \textunderscore abelhoira\textunderscore ?)
\section{Abelota}
\begin{itemize}
\item {Grp. gram.:f.}
\end{itemize}
O mesmo que \textunderscore bellota\textunderscore .
\section{Abeloura}
\begin{itemize}
\item {Grp. gram.:f.}
\end{itemize}
Planta vivaz, de flôr raiada de preto e vermelho, em que as abelhas se demoram, deixando uma substância mellífera, que os rapazes procuram e apreciam.
(Por \textunderscore abelhoira\textunderscore ?)
\section{Abém!}
\begin{itemize}
\item {Grp. gram.:interj.}
\end{itemize}
\begin{itemize}
\item {Utilização:Ant.}
\end{itemize}
O mesmo que \textunderscore e-bem!\textunderscore .
\section{Abemolado}
\begin{itemize}
\item {Grp. gram.:adj.}
\end{itemize}
\begin{itemize}
\item {Proveniência:(De \textunderscore abemolar\textunderscore )}
\end{itemize}
Suave, brando: \textunderscore voz abemolada\textunderscore .
\section{Abemolar}
\begin{itemize}
\item {Grp. gram.:v.}
\end{itemize}
\begin{itemize}
\item {Utilização:t. Mús.}
\end{itemize}
\begin{itemize}
\item {Utilização:Fig.}
\end{itemize}
Marcar com bemol.
Suavizar, abrandar.
\section{Abencerragens}
\begin{itemize}
\item {Grp. gram.:m. pl.}
\end{itemize}
\begin{itemize}
\item {Proveniência:(De \textunderscore Ebn-Serrag\textunderscore , n. p.)}
\end{itemize}
Tribo árabe, que dominou em Granada, antes da conquista dos reis cathólicos, Fernando e Isabel.
\section{Abencerrajens}
\begin{itemize}
\item {Grp. gram.:m. pl.}
\end{itemize}
\begin{itemize}
\item {Proveniência:(De \textunderscore Ebn-Serrag\textunderscore , n. p.)}
\end{itemize}
Tribo árabe, que dominou em Granada, antes da conquista dos reis cathólicos, Fernando e Isabel.
\section{Abençoadeiro}
\begin{itemize}
\item {Grp. gram.:m.}
\end{itemize}
\begin{itemize}
\item {Proveniência:(De \textunderscore abençoar\textunderscore )}
\end{itemize}
Benzedeiro.
\section{Abençoado}
\begin{itemize}
\item {Grp. gram.:adj.}
\end{itemize}
\begin{itemize}
\item {Proveniência:(De \textunderscore abençoar\textunderscore )}
\end{itemize}
Feliz.
Próspero.
\section{Abençoador}
\begin{itemize}
\item {Grp. gram.:m.}
\end{itemize}
Aquelle que abençôa.
\section{Abençoamento}
\begin{itemize}
\item {Grp. gram.:m.}
\end{itemize}
Acto ou effeito de \textunderscore abençoar\textunderscore . Cf. A. Gama, \textunderscore Ult. Dona\textunderscore , 327.
\section{Abençoar}
\begin{itemize}
\item {Grp. gram.:v. t.}
\end{itemize}
\begin{itemize}
\item {Grp. gram.:V. p.}
\end{itemize}
Dar a bênção a.
Proteger.
Bem-dizer.
Benzer-se.
Receber a bênção. Cf. Costa e Sá, \textunderscore Diccion.\textunderscore 
\section{Abençoável}
\begin{itemize}
\item {Grp. gram.:adj.}
\end{itemize}
Digno de se abençoar. Cf. Castilho, \textunderscore Felic. pela Agr.\textunderscore 
\section{Abendiçoar}
\begin{itemize}
\item {Grp. gram.:v. t.}
\end{itemize}
O mesmo que \textunderscore abençoar\textunderscore .
\section{Aberém}
\begin{itemize}
\item {Grp. gram.:m.}
\end{itemize}
\begin{itemize}
\item {Utilização:Bras}
\end{itemize}
O mesmo que \textunderscore abarém\textunderscore .
\section{Aberema}
\begin{itemize}
\item {Grp. gram.:f.}
\end{itemize}
Gênero de plantas anonáceas.
(\textunderscore uvaria\textunderscore , Lin.)
\section{Abéria}
\begin{itemize}
\item {Grp. gram.:f.}
\end{itemize}
\begin{itemize}
\item {Utilização:Bot.}
\end{itemize}
Gênero de flacurtiáceas.
\section{Aberingelado}
\begin{itemize}
\item {Grp. gram.:adj.}
\end{itemize}
Que tem côr de beringela.
\section{Aberração}
\begin{itemize}
\item {Grp. gram.:f.}
\end{itemize}
\begin{itemize}
\item {Proveniência:(Do lat. \textunderscore aberratio\textunderscore )}
\end{itemize}
Acto ou effeito de aberrar.
Movimento apparente das estrêllas fixas.
Diffusão dos raios luminosos, que atravessam corpos diáphanos.
Desarranjo na situação ou no exercicio dos órgãos do corpo.
Êrro de raciocínio, extravagância de conceito.
\section{Aberrante}
\begin{itemize}
\item {Grp. gram.:adj.}
\end{itemize}
\begin{itemize}
\item {Proveniência:(Do lat. \textunderscore aberrans\textunderscore )}
\end{itemize}
Que aberra.
\section{Aberrar}
\begin{itemize}
\item {Grp. gram.:v. i.}
\end{itemize}
\begin{itemize}
\item {Proveniência:(Do lat. \textunderscore aberrare\textunderscore )}
\end{itemize}
Desviar-se do que é verdadeiro ou bom.
Fazer aberração.
\section{Aberrativo}
\begin{itemize}
\item {Grp. gram.:adj.}
\end{itemize}
\begin{itemize}
\item {Utilização:Neol.}
\end{itemize}
Em que há aberração.
\section{Aberta}
\begin{itemize}
\item {Grp. gram.:f.}
\end{itemize}
\begin{itemize}
\item {Grp. gram.:f.}
\end{itemize}
\begin{itemize}
\item {Proveniência:(De \textunderscore aberto\textunderscore )}
\end{itemize}
Abertura, fenda, intervallo.
O abrirem-se as nuvens, mostrando uma nesga do céu, em dias chuvosos.
\section{Abertamente}
\begin{itemize}
\item {Grp. gram.:adv.}
\end{itemize}
\begin{itemize}
\item {Proveniência:(De \textunderscore aberto\textunderscore )}
\end{itemize}
Com franqueza.
Claramente.
\section{Aberto}
\begin{itemize}
\item {Grp. gram.:adj.}
\end{itemize}
\begin{itemize}
\item {Utilização:T. da Bairrada}
\end{itemize}
\begin{itemize}
\item {Grp. gram.:M.}
\end{itemize}
\begin{itemize}
\item {Proveniência:(Do lat. \textunderscore apertus\textunderscore )}
\end{itemize}
Descerrado, patente: \textunderscore uma janela aberta\textunderscore .
Sincero, franco: \textunderscore carácter aberto\textunderscore .
Que se póde discutir livremente: \textunderscore uma questão aberta\textunderscore .
Diz-se do vinho que abre bem, isto é, que é palhete ou clarete.
Abertura: \textunderscore fazer abertos num bordado\textunderscore . Cf. Benalcanfor, \textunderscore Cartas de Viagem\textunderscore , XXVIII.
\section{Abertoiras}
\begin{itemize}
\item {Grp. gram.:f. pl.}
\end{itemize}
Extremos das redes de arrastar, no rio Minho, formando seio ou bolso.
(De \textunderscore aberto\textunderscore ).
\section{Abertona}
\begin{itemize}
\item {Grp. gram.:f.}
\end{itemize}
\begin{itemize}
\item {Utilização:Náut.}
\end{itemize}
Grande abertura no porão dos navios.
\section{Abertura}
\begin{itemize}
\item {Grp. gram.:f.}
\end{itemize}
\begin{itemize}
\item {Utilização:Mús.}
\end{itemize}
\begin{itemize}
\item {Utilização:Mús.}
\end{itemize}
\begin{itemize}
\item {Utilização:Fig.}
\end{itemize}
\begin{itemize}
\item {Proveniência:(De \textunderscore aberto\textunderscore )}
\end{itemize}
Acto ou effeito de abrir.
Buraco.
Fenda.
Symphonia ou peça de música orchestral, que precede uma ópera, uma oratória, um drama com música, ou outra composição de grande desenvolvimento.
Peça de música para orchestra, destinada a sêr executada em concertos ou a servir de introducção ou de intermédio em qualquer espectáculo ou solennidade, e no estilo de abertura propriamente dita.
Inauguração.
\section{Abesantar}
\begin{itemize}
\item {Grp. gram.:v. t.}
\end{itemize}
Ornar com besantes.
\section{Abesconinha}
\begin{itemize}
\item {Grp. gram.:f.}
\end{itemize}
\begin{itemize}
\item {Utilização:Prov.}
\end{itemize}
O mesmo que \textunderscore abibe\textunderscore .
\section{Abeses}
\begin{itemize}
\item {Grp. gram.:m. pl.}
\end{itemize}
Nome, que os antigos davam aos oásis do Sahará. Cf. Barros, \textunderscore Déc.\textunderscore  I, l. III, c. 8.
\section{Abesoirar}
\begin{itemize}
\item {Grp. gram.:v. t.}
\end{itemize}
\begin{itemize}
\item {Utilização:Pop.}
\end{itemize}
\begin{itemize}
\item {Proveniência:(De \textunderscore besoiro\textunderscore )}
\end{itemize}
Importunar com palavras monótonas ou desarrazoadas: \textunderscore não me abesoires os ouvidos\textunderscore .
\section{Abesourar}
\begin{itemize}
\item {Grp. gram.:v. t.}
\end{itemize}
\begin{itemize}
\item {Utilização:Pop.}
\end{itemize}
\begin{itemize}
\item {Proveniência:(De \textunderscore besoiro\textunderscore )}
\end{itemize}
Importunar com palavras monótonas ou desarrazoadas: \textunderscore não me abesoires os ouvidos\textunderscore .
\section{Abespa}
\begin{itemize}
\item {fónica:bês}
\end{itemize}
\begin{itemize}
\item {Grp. gram.:f.}
\end{itemize}
(V.bespa)
\section{Abespinhadamente}
\begin{itemize}
\item {Grp. gram.:adv.}
\end{itemize}
De modo \textunderscore abespinhado\textunderscore , com irritação.
\section{Abespinhado}
\begin{itemize}
\item {Grp. gram.:adj.}
\end{itemize}
\begin{itemize}
\item {Proveniência:(De \textunderscore abespinhar\textunderscore )}
\end{itemize}
Irritado.
\section{Abespinhar-se}
\begin{itemize}
\item {Grp. gram.:v. p.}
\end{itemize}
\begin{itemize}
\item {Proveniência:(De \textunderscore bespa\textunderscore )}
\end{itemize}
Irritar-se.
\section{Abesso}
\begin{itemize}
\item {Grp. gram.:m.}
\end{itemize}
\begin{itemize}
\item {Utilização:Ant.}
\end{itemize}
Injúria; injustiça.
Desordem.
\section{Abestruz}
\begin{itemize}
\item {Grp. gram.:m.  ou  f.}
\end{itemize}
O mesmo que \textunderscore avestruz\textunderscore .
\section{Abeta}
\begin{itemize}
\item {fónica:bê}
\end{itemize}
\begin{itemize}
\item {Grp. gram.:f.}
\end{itemize}
Pequena aba.
\section{Abetarda}
\begin{itemize}
\item {Grp. gram.:f.}
\end{itemize}
Grande ave gallinácea, de pelle duríssima,
(\textunderscore otis tarda\textunderscore , Lin.)
\section{Abetardado}
\begin{itemize}
\item {Grp. gram.:adj.}
\end{itemize}
Que tem a côr da betarda.
\section{Abete}
\begin{itemize}
\item {Grp. gram.:m.}
\end{itemize}
\begin{itemize}
\item {Proveniência:(Do lat. \textunderscore abies\textunderscore )}
\end{itemize}
Árvore abietínea.
Pinheiro alvar.
\section{Abeterno}
\begin{itemize}
\item {Grp. gram.:adv.}
\end{itemize}
Desde sempre, desde toda a eternidade. Cf. \textunderscore Luz e Calor\textunderscore , 441 e 446.
(Da loc. lat. \textunderscore ab aeterno\textunderscore )
\section{Abeto}
\begin{itemize}
\item {Grp. gram.:m.}
\end{itemize}
\begin{itemize}
\item {Proveniência:(Do lat. \textunderscore abies\textunderscore )}
\end{itemize}
Árvore abietínea.
Pinheiro alvar.
\section{Abetoiro}
\begin{itemize}
\item {Grp. gram.:m.}
\end{itemize}
Ave pernalta.
(\textunderscore botaurus stellaris\textunderscore . Lin.)
\section{Abetoninha}
\begin{itemize}
\item {Grp. gram.:f.}
\end{itemize}
O mesmo que \textunderscore abitoninha\textunderscore .
\section{Abetum}
\begin{itemize}
\item {Grp. gram.:m.}
\end{itemize}
\begin{itemize}
\item {Utilização:Ant.}
\end{itemize}
Apoio, auxilio.
\section{Abetumado}
\begin{itemize}
\item {Grp. gram.:adj.}
\end{itemize}
\begin{itemize}
\item {Utilização:Prov.}
\end{itemize}
\begin{itemize}
\item {Utilização:trasm.}
\end{itemize}
\begin{itemize}
\item {Utilização:Fig.}
\end{itemize}
Calafetado.
Diz-se do pão que, depois de cozido, fica muito compacto e pesado.
Macambúzio. Cf. \textunderscore Eufrosina\textunderscore , 15.
(De \textunderscore abetumar\textunderscore ).
\section{Abetumar}
\begin{itemize}
\item {Grp. gram.:v. t.}
\end{itemize}
Cobrir com betume.
Calafetar.
\section{Abexigar}
\begin{itemize}
\item {Grp. gram.:v. t.}
\end{itemize}
\begin{itemize}
\item {Utilização:Fam.}
\end{itemize}
Fazer escárneo de; ridiculizar.
\section{Abexim}
\begin{itemize}
\item {Grp. gram.:m.}
\end{itemize}
\begin{itemize}
\item {Grp. gram.:Adj.}
\end{itemize}
Aquelle que é natural da Abyssínia.
Língua da Abyssínia.
Relativo á Abyssínia.
(De \textunderscore Abexia\textunderscore , n. p., com que alguns clássicos nossos designaram a Abyssínia).
\section{Abezelgado}
\begin{itemize}
\item {Grp. gram.:adj.}
\end{itemize}
Desenvolto?:«\textunderscore dama abezelgada e frescalhona\textunderscore »: Camillo, \textunderscore Canc. Alegre\textunderscore , 402.
\section{Abezerrado}
\begin{itemize}
\item {Grp. gram.:adj.}
\end{itemize}
Semelhante a bezerro.
\section{Abia}
\begin{itemize}
\item {Grp. gram.:f.}
\end{itemize}
Insecto hymenóptero, especie de tenthredém.
\section{Abibe}
\begin{itemize}
\item {Grp. gram.:f.}
\end{itemize}
Ave pernalta de arribação.
(\textunderscore tringa vanellus\textunderscore , Lin.)
\section{Abibliotecar}
\begin{itemize}
\item {Grp. gram.:v. t.}
\end{itemize}
Conservar ou dispor em bibliotheca.
\section{Abibliothecar}
\begin{itemize}
\item {Grp. gram.:v. t.}
\end{itemize}
Conservar ou dispor em bibliotheca.
\section{Abicar}
\begin{itemize}
\item {Grp. gram.:v. t.}
\end{itemize}
\begin{itemize}
\item {Grp. gram.:V. i.}
\end{itemize}
Fazer o bico a, aguçar.
Fazer tocar, aproximar.
Chegar.
Deitar ferro, ancorar.
\section{Abichar}
\begin{itemize}
\item {Grp. gram.:v. t.}
\end{itemize}
\begin{itemize}
\item {Utilização:Fam.}
\end{itemize}
Obter (qualquer coisa vantajosa).
\section{Abicheiro}
\begin{itemize}
\item {Grp. gram.:m.}
\end{itemize}
\begin{itemize}
\item {Utilização:Prov.}
\end{itemize}
\begin{itemize}
\item {Utilização:trasm.}
\end{itemize}
\begin{itemize}
\item {Utilização:Prov.}
\end{itemize}
\begin{itemize}
\item {Utilização:beir.}
\end{itemize}
\begin{itemize}
\item {Proveniência:(Do lat. hyp. \textunderscore aversiarius\textunderscore )}
\end{itemize}
Lugar, onde não dá o sol.
O lado opposto ao sul.
\section{Abichornado}
\begin{itemize}
\item {Grp. gram.:adj.}
\end{itemize}
\begin{itemize}
\item {Utilização:Bras}
\end{itemize}
Desalentado, aborrecido.
Envergonhado, vexado.
(Cp. \textunderscore bochorno\textunderscore ).
\section{Abieiro}
\begin{itemize}
\item {Grp. gram.:m.}
\end{itemize}
Arvoreta sapotácea da América equatorial.
(De \textunderscore abio\textunderscore ).
\section{Abiético}
\begin{itemize}
\item {Grp. gram.:adj.}
\end{itemize}
Diz-se de um ácido, descoberto na resina do abeto.
(De \textunderscore abieto\textunderscore ).
\section{Abietina}
\begin{itemize}
\item {Grp. gram.:f.}
\end{itemize}
Substância crystallizável, que se encontra em certas terebentinas.
(Fem. de \textunderscore abietino\textunderscore ).
\section{Abietíneas}
\begin{itemize}
\item {Grp. gram.:f. pl.}
\end{itemize}
Família de plantas coníferas, que abrange o abeto, o pinheiro, a araucária, o cedro, etc.
(De \textunderscore abietíneo\textunderscore ).
\section{Abietíneo}
\begin{itemize}
\item {Grp. gram.:adj.}
\end{itemize}
\begin{itemize}
\item {Proveniência:(Do lat. \textunderscore abies\textunderscore , \textunderscore abietis\textunderscore )}
\end{itemize}
Relativo ou semelhante ao abeto.
\section{Abietino}
\begin{itemize}
\item {Grp. gram.:adj.}
\end{itemize}
O mesmo que \textunderscore abietíneo\textunderscore .
\section{Abieto}
\begin{itemize}
\item {Grp. gram.:m.}
\end{itemize}
O mesmo que \textunderscore abeto\textunderscore .
\section{Abiga}
\begin{itemize}
\item {Grp. gram.:f.}
\end{itemize}
Espécie de pinheiro.
(\textunderscore teucrium chamoepites\textunderscore , Lin.)
\section{Abigoiro}
\begin{itemize}
\item {Grp. gram.:m.}
\end{itemize}
\begin{itemize}
\item {Utilização:Prov.}
\end{itemize}
\begin{itemize}
\item {Utilização:beir.}
\end{itemize}
O mesmo que \textunderscore vespa\textunderscore .
(Cp. o lat. hyp. \textunderscore apicularius\textunderscore )
\section{Abilhamento}
\begin{itemize}
\item {Grp. gram.:m.}
\end{itemize}
\begin{itemize}
\item {Utilização:Ant.}
\end{itemize}
Acto de \textunderscore abilhar\textunderscore .
Ornato, enfeite.
\section{Abilhar}
\begin{itemize}
\item {Grp. gram.:v. t.}
\end{itemize}
\begin{itemize}
\item {Utilização:Ant.}
\end{itemize}
\begin{itemize}
\item {Proveniência:(Do fr. \textunderscore habiller\textunderscore )}
\end{itemize}
Ataviar, ornar.
\section{Abinício}
\begin{itemize}
\item {Grp. gram.:adv.}
\end{itemize}
\begin{itemize}
\item {Proveniência:(Do boc. lat. \textunderscore ab initio\textunderscore )}
\end{itemize}
Desde o princípio.
Desde que há mundo. Cf. Usque. \textunderscore Tribulações\textunderscore .
\section{Abio}
\begin{itemize}
\item {Grp. gram.:m.}
\end{itemize}
Fruto de abieiro.
\section{Abiorama}
\begin{itemize}
\item {Grp. gram.:f.}
\end{itemize}
\begin{itemize}
\item {Utilização:Bras}
\end{itemize}
Árvore sapotácea (\textunderscore lucuma lasiocarpa\textunderscore ).
Fruto dessa árvore.
\section{Abiótica}
\begin{itemize}
\item {Grp. gram.:f.}
\end{itemize}
\begin{itemize}
\item {Proveniência:(De \textunderscore abiótico\textunderscore )}
\end{itemize}
A sciência do mundo inorgânico.
\section{Abiótico}
\begin{itemize}
\item {Grp. gram.:adj.}
\end{itemize}
\begin{itemize}
\item {Proveniência:(Do gr. \textunderscore a\textunderscore  priv. e \textunderscore bios\textunderscore , vida)}
\end{itemize}
Contrário á vida.
\textunderscore Meio abiótico\textunderscore , \textunderscore zona abiótica\textunderscore , o meio ou a zona em que se não póde viver.
\section{Abipão}
\begin{itemize}
\item {Grp. gram.:m.}
\end{itemize}
Um dos principaes dialectos da lingua peruana.
\section{Abirritaçao}
\begin{itemize}
\item {Grp. gram.:f.}
\end{itemize}
\begin{itemize}
\item {Proveniência:(De \textunderscore ab...\textunderscore  + \textunderscore irritação\textunderscore )}
\end{itemize}
Fraqueza, atonia.
\section{Abiscoitar}
\begin{itemize}
\item {Grp. gram.:v. t.}
\end{itemize}
\begin{itemize}
\item {Utilização:Fam.}
\end{itemize}
\begin{itemize}
\item {Utilização:Bras}
\end{itemize}
Cozer como biscoito.
Abichar.
Furtar.
\section{Abismado}
\begin{itemize}
\item {Grp. gram.:adj.}
\end{itemize}
Espantado, admirado.
(De \textunderscore abismar\textunderscore ).
\section{Abismal}
\begin{itemize}
\item {Grp. gram.:adj.}
\end{itemize}
Relativo ao \textunderscore abismo\textunderscore .
\section{Abismar}
\begin{itemize}
\item {Grp. gram.:v. t.}
\end{itemize}
Lançar no abismo.
Arruinar.
Causar espanto ou admiração a.
\section{Abismo}
\begin{itemize}
\item {Grp. gram.:m.}
\end{itemize}
\begin{itemize}
\item {Utilização:Fig.}
\end{itemize}
\begin{itemize}
\item {Proveniência:(Do b. lat. \textunderscore abissimus\textunderscore , superl. de \textunderscore abyssus\textunderscore )}
\end{itemize}
Voragem, precipício profundo.
O último grau; \textunderscore um abismo de miséria\textunderscore .
Perdição.
Mystério.
Oceano.
O inferno.
\section{Abita}
\begin{itemize}
\item {Grp. gram.:f.}
\end{itemize}
\begin{itemize}
\item {Utilização:Náut.}
\end{itemize}
Peça de madeira ou de ferro, na prôa de navio, para fixar a amarra da âncora.
(Do germ.).
\section{Abitar}
\begin{itemize}
\item {Grp. gram.:v. t.}
\end{itemize}
Prender na abita.
\section{Abitílio}
\begin{itemize}
\item {Grp. gram.:m.}
\end{itemize}
Planta, de folhas semelhantes ás da malva.
(Corr. de \textunderscore abutilo\textunderscore )
\section{Abitolar}
\begin{itemize}
\item {Grp. gram.:v. t.}
\end{itemize}
Medir com bitola; conferir com ella.
\section{Abitoninha}
\begin{itemize}
\item {Grp. gram.:f.}
\end{itemize}
\begin{itemize}
\item {Utilização:Prov.}
\end{itemize}
\begin{itemize}
\item {Utilização:beir.}
\end{itemize}
O mesmo que \textunderscore abibe\textunderscore .
\section{Abivacar}
\begin{itemize}
\item {Grp. gram.:v. t.}
\end{itemize}
O mesmo que \textunderscore bivacar\textunderscore . Cf. Latino, \textunderscore Humboldt\textunderscore . 185, 213.
\section{Abjecção}
\begin{itemize}
\item {Grp. gram.:f.}
\end{itemize}
\begin{itemize}
\item {Proveniência:(Do lat. \textunderscore abjectio\textunderscore )}
\end{itemize}
Aviltamento; baixeza.
\section{Abjectamente}
\begin{itemize}
\item {Grp. gram.:adv.}
\end{itemize}
De modo abjecto.
Vilmente.
\section{Abjecto}
\begin{itemize}
\item {Grp. gram.:adj.}
\end{itemize}
\begin{itemize}
\item {Grp. gram.:M.}
\end{itemize}
\begin{itemize}
\item {Proveniência:(Do lat. \textunderscore abjectus\textunderscore )}
\end{itemize}
Vil.
Desprezível; desprezado.
Homem vil.
\section{Abjeição}
\begin{itemize}
\item {Grp. gram.:f.}
\end{itemize}
(V.abjecção)
\section{Abjudicação}
\begin{itemize}
\item {Grp. gram.:f.}
\end{itemize}
\begin{itemize}
\item {Proveniência:(Do lat. \textunderscore abjudicatio\textunderscore )}
\end{itemize}
Acto ou effeito de abjudicar.
\section{Abjudicar}
\begin{itemize}
\item {Grp. gram.:v. t.}
\end{itemize}
\begin{itemize}
\item {Proveniência:(Do lat. \textunderscore abjudicare\textunderscore )}
\end{itemize}
Tirar judicialmente ao possuidor illegítimo (o que pertence a outrem)
\section{Abjugar}
\begin{itemize}
\item {Grp. gram.:v. t.}
\end{itemize}
\begin{itemize}
\item {Proveniência:(De \textunderscore ab...\textunderscore  + \textunderscore jugo\textunderscore )}
\end{itemize}
Tirar do jugo.
Libertar.
\section{Abjunção}
\begin{itemize}
\item {Grp. gram.:f.}
\end{itemize}
\begin{itemize}
\item {Proveniência:(Do lat. \textunderscore ab\textunderscore  + \textunderscore junctio\textunderscore )}
\end{itemize}
Separação.
\section{Abjuncção}
\begin{itemize}
\item {Grp. gram.:f.}
\end{itemize}
\begin{itemize}
\item {Proveniência:(Do lat. \textunderscore ab\textunderscore  + \textunderscore junctio\textunderscore )}
\end{itemize}
Separação.
\section{Abjuração}
\begin{itemize}
\item {Grp. gram.:f.}
\end{itemize}
\begin{itemize}
\item {Proveniência:(Do lat. \textunderscore abjuratio\textunderscore )}
\end{itemize}
Acto ou effeito de abjurar.
\section{Abjurador}
\begin{itemize}
\item {Grp. gram.:m.}
\end{itemize}
Aquelle que abjura.
\section{Abjurante}
\textunderscore m\textunderscore . e \textunderscore adj\textunderscore .
Aquelle que abjura.
\section{Abjurar}
\begin{itemize}
\item {Grp. gram.:v. t.}
\end{itemize}
\begin{itemize}
\item {Proveniência:(Do lat. \textunderscore abjurare\textunderscore )}
\end{itemize}
Abandonar.
Renunciar (crenças, especialmente).
\section{Abjuratório}
\begin{itemize}
\item {Grp. gram.:adj.}
\end{itemize}
Relativo á abjuração.
\section{Abjurgar}
\begin{itemize}
\item {Grp. gram.:v. t.}
\end{itemize}
\begin{itemize}
\item {Proveniência:(Do lat. \textunderscore abjurgare\textunderscore )}
\end{itemize}
Tirar judicialmente.
\section{Ablação}
\begin{itemize}
\item {Grp. gram.:f.}
\end{itemize}
\begin{itemize}
\item {Utilização:Gram.}
\end{itemize}
\begin{itemize}
\item {Proveniência:(Do lat. \textunderscore ablatio\textunderscore )}
\end{itemize}
Acto de tirar.
Acto de cortar: \textunderscore ablação de um tumor\textunderscore .
Aphérese.
\section{Ablactação}
\begin{itemize}
\item {Grp. gram.:f.}
\end{itemize}
\begin{itemize}
\item {Proveniência:(Do lat. \textunderscore ablactatio\textunderscore )}
\end{itemize}
O desmamar das crianças.
\section{Ablactar}
\begin{itemize}
\item {Grp. gram.:v. t.}
\end{itemize}
\begin{itemize}
\item {Proveniência:(Do lat. \textunderscore ablactare\textunderscore )}
\end{itemize}
Desmamar.
\section{Ablamelares}
\begin{itemize}
\item {Grp. gram.:f. pl.}
\end{itemize}
\begin{itemize}
\item {Utilização:Bot.}
\end{itemize}
\begin{itemize}
\item {Proveniência:(De \textunderscore ab...\textunderscore  + \textunderscore lamella\textunderscore )}
\end{itemize}
Plantas, caracterizadas pelo afastamento das lamellas.
\section{Ablamellares}
\begin{itemize}
\item {Grp. gram.:f. pl.}
\end{itemize}
\begin{itemize}
\item {Utilização:Bot.}
\end{itemize}
\begin{itemize}
\item {Proveniência:(De \textunderscore ab...\textunderscore  + \textunderscore lamella\textunderscore )}
\end{itemize}
Plantas, caracterizadas pelo afastamento das lamellas.
\section{Ablaminares}
\begin{itemize}
\item {Grp. gram.:f. pl.}
\end{itemize}
O mesmo que \textunderscore ablamellares\textunderscore .
\section{Ablânia}
\begin{itemize}
\item {Grp. gram.:f.}
\end{itemize}
Árvore liliácea da Guiana.
\section{Ablaqueação}
\begin{itemize}
\item {Grp. gram.:f.}
\end{itemize}
\begin{itemize}
\item {Proveniência:(Do lat. \textunderscore ablaqueatio\textunderscore )}
\end{itemize}
Acto ou effeito de ablaquear.
\section{Ablaquear}
\begin{itemize}
\item {Grp. gram.:v. t.}
\end{itemize}
\begin{itemize}
\item {Proveniência:(Do lat. \textunderscore ablaqueare\textunderscore )}
\end{itemize}
Desprender, desenlaçar.
Escavar em roda (as árvores).
\section{Ablaquecer}
\begin{itemize}
\item {Grp. gram.:v. t.}
\end{itemize}
\begin{itemize}
\item {Utilização:Ant.}
\end{itemize}
O mesmo que \textunderscore ablaquear\textunderscore .
\section{Ablativo}
\begin{itemize}
\item {Grp. gram.:adj.}
\end{itemize}
\begin{itemize}
\item {Grp. gram.:M.}
\end{itemize}
\begin{itemize}
\item {Utilização:Gram.}
\end{itemize}
Que póde extrair.
Um dos casos da declinação latina.
\textunderscore Fazer ablativo de viagem\textunderscore , partir inesperadamente, desapparecer.
\section{Ablator}
\begin{itemize}
\item {Grp. gram.:m.}
\end{itemize}
Aquelle ou aquillo que extrái.
Instrumento de castração.
\section{Ablectos}
\begin{itemize}
\item {Grp. gram.:m. pl.}
\end{itemize}
\begin{itemize}
\item {Proveniência:(Do lat. \textunderscore ablecti\textunderscore )}
\end{itemize}
Soldados romanos, escolhidos, que formavam a guarda dos cônsules, em tempo de guerra.
\section{Ablefaria}
\begin{itemize}
\item {Grp. gram.:f.}
\end{itemize}
Qualidade ou estado de quem é ablépharo.
\section{Ablegação}
\begin{itemize}
\item {Grp. gram.:f.}
\end{itemize}
\begin{itemize}
\item {Proveniência:(Do lat. \textunderscore ablegatio\textunderscore )}
\end{itemize}
Acto ou effeito do ablegar.
\section{Ablegado}
\begin{itemize}
\item {Grp. gram.:m.}
\end{itemize}
\begin{itemize}
\item {Proveniência:(Do lat. \textunderscore ablegatus\textunderscore )}
\end{itemize}
Commissário encarregado, pela côrte de Roma, de levar o barrete a um Cardeal, recentemente promovido.
\section{Ablegar}
\begin{itemize}
\item {Grp. gram.:v. t.}
\end{itemize}
\begin{itemize}
\item {Proveniência:(Do lat. \textunderscore ablegare\textunderscore )}
\end{itemize}
Enviar para longe.
Desterrar.
\section{Ableitar}
\begin{itemize}
\item {Grp. gram.:v. t.}
\end{itemize}
(V.ablactar)
\section{Ablepharia}
\begin{itemize}
\item {Grp. gram.:f.}
\end{itemize}
Qualidade ou estado de quem é ablépharo.
\section{Abléfaro}
\begin{itemize}
\item {Grp. gram.:adj.}
\end{itemize}
\begin{itemize}
\item {Proveniência:(Do gr. \textunderscore a\textunderscore , priv. e \textunderscore blepharon\textunderscore , pálpebra)}
\end{itemize}
Que não tem pálpebras.
\section{Ablépharo}
\begin{itemize}
\item {Grp. gram.:adj.}
\end{itemize}
\begin{itemize}
\item {Proveniência:(Do gr. \textunderscore a\textunderscore , priv. e \textunderscore blepharon\textunderscore , pálpebra)}
\end{itemize}
Que não tem pálpebras.
\section{Ablução}
\begin{itemize}
\item {Grp. gram.:f.}
\end{itemize}
\begin{itemize}
\item {Proveniência:(Do lat. \textunderscore ablutio\textunderscore )}
\end{itemize}
Acto de abluir.
Banho, que se dá a todo o corpo ou a parte dêlle, com esponja embebida em água, ou com toalha molhada.
\section{Abluente}
\begin{itemize}
\item {Grp. gram.:adj.}
\end{itemize}
\begin{itemize}
\item {Grp. gram.:M.}
\end{itemize}
\begin{itemize}
\item {Proveniência:(Do lat. \textunderscore abluens\textunderscore )}
\end{itemize}
Próprio para abluir.
O que ablui.
\section{Abluir}
\begin{itemize}
\item {Grp. gram.:v. t.}
\end{itemize}
\begin{itemize}
\item {Proveniência:(Do lat. \textunderscore abluere\textunderscore )}
\end{itemize}
Purificar, lavando.
\section{Ablutor}
\begin{itemize}
\item {Grp. gram.:m.}
\end{itemize}
\begin{itemize}
\item {Proveniência:(Do lat. \textunderscore ablutor\textunderscore )}
\end{itemize}
O que lava.
O que purifica.
\section{Abnegação}
\begin{itemize}
\item {Grp. gram.:f.}
\end{itemize}
\begin{itemize}
\item {Proveniência:(Do lat. \textunderscore abnegatio\textunderscore )}
\end{itemize}
Renúncia.
Desprendimento do interesse próprio.
Desinteresse.
\section{Abnegador}
\begin{itemize}
\item {Grp. gram.:m.}
\end{itemize}
\begin{itemize}
\item {Proveniência:(Do lat. \textunderscore abnegator\textunderscore )}
\end{itemize}
Aquelle que abnega.
\section{Abnegar}
\begin{itemize}
\item {Grp. gram.:v. t.}
\end{itemize}
\begin{itemize}
\item {Proveniência:(Do lat. \textunderscore abnegare\textunderscore )}
\end{itemize}
Renunciar.
Abster-se de.
\section{Abneto}
\begin{itemize}
\item {Grp. gram.:m.}
\end{itemize}
(V.trineto)
\section{Aboamento}
\begin{itemize}
\item {Grp. gram.:m.}
\end{itemize}
\begin{itemize}
\item {Utilização:Carp.}
\end{itemize}
Inclinação, que se dá aos lados internos de porta ou janela, para que a portada não fique esquadriada, mas bem aberta.
\section{Abóbada}
\begin{itemize}
\item {Grp. gram.:f.}
\end{itemize}
\begin{itemize}
\item {Proveniência:(Do cast. \textunderscore bóveda\textunderscore )}
\end{itemize}
Construcção em arco. Tecto arqueado.
* \textunderscore Abóbada celeste\textunderscore , o céu, considerado sob a fórma com que elle se arqueia sôbre as nossas cabeças.
* \textunderscore Abóbada de aço\textunderscore , ceremónia, em que os mações, formando duas alas, estendem os braços e cruzam as espadas de uma ala com as da outra, para que passe por baixo dellas o recipiendário.
\section{Abobadado}
\begin{itemize}
\item {Grp. gram.:adj.}
\end{itemize}
\begin{itemize}
\item {Proveniência:(De \textunderscore abobadar\textunderscore )}
\end{itemize}
Que tem fórma de abóbada.
\section{Abobadar}
\begin{itemize}
\item {Grp. gram.:v. t.}
\end{itemize}
Cobrir com abóbada.
Dar fórma de abóbada a.
\section{Abobadilha}
\begin{itemize}
\item {Grp. gram.:f.}
\end{itemize}
\begin{itemize}
\item {Utilização:Prov.}
\end{itemize}
\begin{itemize}
\item {Utilização:alent.}
\end{itemize}
Abóbada de gesso.
Abóbada, formada de ladrilhos, que não são postos de cunha, mas de chapa.
\section{Abobadilheiro}
\begin{itemize}
\item {Grp. gram.:m.}
\end{itemize}
\begin{itemize}
\item {Utilização:Prov.}
\end{itemize}
\begin{itemize}
\item {Utilização:alent.}
\end{itemize}
Aquelle que faz abobadilhas de tijolo.
\section{Abobar-se}
\begin{itemize}
\item {Grp. gram.:v. p.}
\end{itemize}
Fazer-se bobo.
Fingir-se inepto.
\section{Abóbeda}
\begin{itemize}
\item {Grp. gram.:f.}
\end{itemize}
\begin{itemize}
\item {Utilização:Ant.}
\end{itemize}
(V.abóbada)
\section{Abóbora}
\begin{itemize}
\item {Grp. gram.:f.}
\end{itemize}
\begin{itemize}
\item {Utilização:Fig.}
\end{itemize}
Fruto da aboboreira.
Nome de uma armação de pesca de atum, em Tavira.
Homem fraco, indolente ou cobarde.
Mulher gorda.
\section{Aboborado}
\begin{itemize}
\item {Grp. gram.:adj.}
\end{itemize}
\begin{itemize}
\item {Utilização:Fig.}
\end{itemize}
\begin{itemize}
\item {Proveniência:(De \textunderscore aboborar\textunderscore )}
\end{itemize}
Amollecido; amadurecido.
\section{Aboboral}
\begin{itemize}
\item {Grp. gram.:m.}
\end{itemize}
Lugar, onde crescem abóboras.
\section{Aboborar}
\begin{itemize}
\item {Grp. gram.:v. t.}
\end{itemize}
(V.abeberar). Cf. Filinto, VII, 67.
\section{Aboboreira}
\begin{itemize}
\item {Grp. gram.:f.}
\end{itemize}
\begin{itemize}
\item {Proveniência:(De \textunderscore abóbora\textunderscore )}
\end{itemize}
Planta cucurbitácea.
\section{Aboborinha-do-mato}
\begin{itemize}
\item {Grp. gram.:f.}
\end{itemize}
\begin{itemize}
\item {Utilização:Bras}
\end{itemize}
O mesmo que \textunderscore taiuiá\textunderscore .
\section{Abobra}
\textunderscore f.\textunderscore  (e der.) \textunderscore Pop.\textunderscore 
O mesmo que \textunderscore abóbora\textunderscore , etc.
\section{Abocador}
\begin{itemize}
\item {Grp. gram.:m.}
\end{itemize}
O que abocanha.
\section{Abocadura}
\begin{itemize}
\item {Grp. gram.:f.}
\end{itemize}
\begin{itemize}
\item {Proveniência:(De \textunderscore abocar\textunderscore )}
\end{itemize}
O mesmo que \textunderscore seteira\textunderscore .
\section{Abocamento}
\begin{itemize}
\item {Grp. gram.:m.}
\end{itemize}
Acto de \textunderscore abocar\textunderscore .
Encontro de duas bocas.
Collóquio.
\section{Aboçamento}
\begin{itemize}
\item {Grp. gram.:m.}
\end{itemize}
Acto de \textunderscore aboçar\textunderscore .
\section{Abocanhar}
\begin{itemize}
\item {Grp. gram.:v. t.}
\end{itemize}
\begin{itemize}
\item {Proveniência:(De \textunderscore abocar\textunderscore )}
\end{itemize}
Cortar com os dentes; morder.
Difamar.
\section{Abocanhar}
\begin{itemize}
\item {Grp. gram.:v. i.}
\end{itemize}
\begin{itemize}
\item {Utilização:Prov.}
\end{itemize}
\begin{itemize}
\item {Proveniência:(De \textunderscore bocanho\textunderscore )}
\end{itemize}
Diz-se do tempo, que se allivia em dias de chuva; fazer bocanho.
\section{Abocar}
\begin{itemize}
\item {Grp. gram.:v. t.}
\end{itemize}
Tocar com a boca; apanhar com a boca. Cf. Castilho, \textunderscore Fausto\textunderscore , 84.
Chegar á entrada de.
Apontar ou voltar a boca de (espingarda)
\section{Aboçar}
\begin{itemize}
\item {Grp. gram.:v. t.}
\end{itemize}
Prender nas boças.
\section{Abocetar}
\begin{itemize}
\item {Grp. gram.:v. t.}
\end{itemize}
Guardar em boceta.
Dar fórma de boceta a; arredondar.
\section{Abochornado}
\begin{itemize}
\item {Grp. gram.:adj.}
\end{itemize}
\begin{itemize}
\item {Proveniência:(De \textunderscore bochorno\textunderscore )}
\end{itemize}
Quente; abafadiço.
\section{Abôço}
\begin{itemize}
\item {Grp. gram.:m.}
\end{itemize}
\begin{itemize}
\item {Utilização:Náut.}
\end{itemize}
\begin{itemize}
\item {Proveniência:(De \textunderscore aboçar\textunderscore )}
\end{itemize}
Parte do cabo virador, em que elle abraça ou amarra.
\section{Abodegado}
\begin{itemize}
\item {Grp. gram.:adj.}
\end{itemize}
\begin{itemize}
\item {Utilização:Bras. de Minas}
\end{itemize}
\begin{itemize}
\item {Proveniência:(De \textunderscore abodegar\textunderscore )}
\end{itemize}
Incommodado; maçado; aborrecido.
\section{Abodegar}
\begin{itemize}
\item {Grp. gram.:v. t.}
\end{itemize}
\begin{itemize}
\item {Utilização:Bras. de Minas}
\end{itemize}
\begin{itemize}
\item {Grp. gram.:v. t.}
\end{itemize}
\begin{itemize}
\item {Utilização:Bras}
\end{itemize}
\begin{itemize}
\item {Grp. gram.:v. t.}
\end{itemize}
\begin{itemize}
\item {Utilização:Bras. do N}
\end{itemize}
Incommodar.
Fazer zangar.
Causar raiva ou aborrecimento a.
Importunar, enfadar.
\section{Abofé}
\begin{itemize}
\item {Grp. gram.:adj.}
\end{itemize}
O mesmo que \textunderscore bofé\textunderscore . Cf. \textunderscore Eufrosina\textunderscore , IX.
\section{Aboiar}
\begin{itemize}
\item {fónica:bôi}
\end{itemize}
\begin{itemize}
\item {Grp. gram.:v. t.}
\end{itemize}
\begin{itemize}
\item {Grp. gram.:v. i.}
\end{itemize}
Prender á bóia.
O mesmo que \textunderscore boiar\textunderscore . Cf. Vieira, X, 84.
\section{Aboiar}
\begin{itemize}
\item {fónica:bôi}
\end{itemize}
\begin{itemize}
\item {Grp. gram.:v. t.}
\end{itemize}
\begin{itemize}
\item {Utilização:Açor}
\end{itemize}
Atirar para longe.
\section{Aboiar}
\begin{itemize}
\item {fónica:bôi}
\end{itemize}
\begin{itemize}
\item {Grp. gram.:v. t.}
\end{itemize}
\begin{itemize}
\item {Utilização:Prov.}
\end{itemize}
\begin{itemize}
\item {Utilização:bras}
\end{itemize}
\begin{itemize}
\item {Utilização:minh.}
\end{itemize}
\begin{itemize}
\item {Grp. gram.:V. i.}
\end{itemize}
Trabalhar com (bois).
Falar aos bois; cantar-lhes.
\section{Aboio}
\begin{itemize}
\item {Grp. gram.:m.}
\end{itemize}
\begin{itemize}
\item {Utilização:Bras. de Minas}
\end{itemize}
\begin{itemize}
\item {Proveniência:(De \textunderscore aboiar\textunderscore ^2)}
\end{itemize}
O canto do vaqueiro.
\section{Aboís}
\begin{itemize}
\item {Grp. gram.:f.}
\end{itemize}
O mesmo que \textunderscore boiz\textunderscore .
\section{Aboíz}
\begin{itemize}
\item {Grp. gram.:f.}
\end{itemize}
O mesmo que \textunderscore boiz\textunderscore .
\section{Abolar}
\begin{itemize}
\item {Grp. gram.:v. t.}
\end{itemize}
Dar fórma de bolo a.
Amachucar.
\section{Abolar}
\begin{itemize}
\item {Grp. gram.:v. t.}
\end{itemize}
\begin{itemize}
\item {Utilização:Ant.}
\end{itemize}
O mesmo que \textunderscore abolir\textunderscore .
\section{Aboldriar-se}
\begin{itemize}
\item {Grp. gram.:v. p.}
\end{itemize}
Cingir-se com boldrié.
\section{Aboleimado}
\begin{itemize}
\item {Grp. gram.:adj.}
\end{itemize}
\begin{itemize}
\item {Proveniência:(De \textunderscore aboleimar\textunderscore )}
\end{itemize}
Grosseiro.
Parvo.
\section{Aboleimar}
\begin{itemize}
\item {Grp. gram.:v. t.}
\end{itemize}
Dar fórma de boleima ou bolo grosseiro a.
Aparvalhar.
\section{Aboleirar}
\begin{itemize}
\item {Grp. gram.:v. t.}
\end{itemize}
\begin{itemize}
\item {Utilização:Prov.}
\end{itemize}
\begin{itemize}
\item {Utilização:minh.}
\end{itemize}
O mesmo que \textunderscore rebolar\textunderscore .
\section{Aboletamento}
\begin{itemize}
\item {Grp. gram.:m.}
\end{itemize}
Acto ou effeito de aboletar.
\section{Aboletar}
\begin{itemize}
\item {Grp. gram.:v. t.}
\end{itemize}
Dar boleto a; aquartelar em casa particular.
\section{Abolição}
\begin{itemize}
\item {Grp. gram.:f.}
\end{itemize}
\begin{itemize}
\item {Proveniência:(Do lat. \textunderscore abolitie\textunderscore )}
\end{itemize}
Acto ou effeito de abolir.
\section{Abolicionismo}
\begin{itemize}
\item {Grp. gram.:m.}
\end{itemize}
\begin{itemize}
\item {Proveniência:(Do lat. \textunderscore abolitio\textunderscore )}
\end{itemize}
Systema dos que defendem a abolição da escravatura.
\section{Abolicionista}
\begin{itemize}
\item {Grp. gram.:m.}
\end{itemize}
\begin{itemize}
\item {Proveniência:(Do lat. \textunderscore abolitio\textunderscore )}
\end{itemize}
Sectário do abolicionismo.
\section{Abolimento}
\begin{itemize}
\item {Grp. gram.:m.}
\end{itemize}
O mesmo que \textunderscore abolição\textunderscore .
\section{Abolinar}
\begin{itemize}
\item {Grp. gram.:v. i.}
\end{itemize}
Ir pela bolina.
\section{Abolir}
\begin{itemize}
\item {Grp. gram.:v. t.}
\end{itemize}
\begin{itemize}
\item {Proveniência:(Lat. \textunderscore abolere\textunderscore )}
\end{itemize}
Extinguir.
Revogar.
Pôr fóra do uso.
\section{Abolorecer}
\begin{itemize}
\item {Grp. gram.:v. i.}
\end{itemize}
Criar bolor.
\section{Abolorecido}
\begin{itemize}
\item {Grp. gram.:adj.}
\end{itemize}
Que criou bolor; que tem bolor.
\section{Abolório}
\begin{itemize}
\item {Grp. gram.:m.}
\end{itemize}
\begin{itemize}
\item {Utilização:Ant.}
\end{itemize}
Ascendência, avoengos.
(Cp. cast. \textunderscore abuelo\textunderscore )
\section{Abómaso}
\begin{itemize}
\item {Grp. gram.:m.}
\end{itemize}
\begin{itemize}
\item {Proveniência:(Do lat. \textunderscore ab\textunderscore  + \textunderscore omasum\textunderscore )}
\end{itemize}
Quarto estômago dos ruminantes.
\section{Abombado}
\begin{itemize}
\item {Grp. gram.:adj.}
\end{itemize}
\begin{itemize}
\item {Utilização:Bras}
\end{itemize}
\begin{itemize}
\item {Proveniência:(De \textunderscore abombar\textunderscore )}
\end{itemize}
Cansado; arquejante; esfalfado.
\section{Abombar}
\begin{itemize}
\item {Grp. gram.:v. i.}
\end{itemize}
\begin{itemize}
\item {Utilização:Bras}
\end{itemize}
Diz-se do cavallo que suspende a marcha, por effeito do calor.
\section{Abominábil}
\begin{itemize}
\item {Grp. gram.:adj.}
\end{itemize}
\begin{itemize}
\item {Utilização:Ant.}
\end{itemize}
(V.abominável)
\section{Abominação}
\begin{itemize}
\item {Grp. gram.:f.}
\end{itemize}
\begin{itemize}
\item {Proveniência:(Lat. \textunderscore abominatio\textunderscore )}
\end{itemize}
Acto ou effeito de abominar.
Repulsão.
Aquillo que é abominável.
\section{Abominadôr}
\begin{itemize}
\item {Grp. gram.:m.}
\end{itemize}
O que abomina.
\section{Abominando}
\begin{itemize}
\item {Grp. gram.:adj.}
\end{itemize}
\begin{itemize}
\item {Proveniência:(Lat. \textunderscore abominandus\textunderscore )}
\end{itemize}
Que deve ser abominado.
\section{Abominar}
\begin{itemize}
\item {Grp. gram.:v. t.}
\end{itemize}
\begin{itemize}
\item {Proveniência:(Lat. \textunderscore abominari\textunderscore )}
\end{itemize}
Detestar.
Repellir com horror.
\section{Abominário}
\begin{itemize}
\item {Grp. gram.:m.}
\end{itemize}
\begin{itemize}
\item {Utilização:Ant.}
\end{itemize}
\begin{itemize}
\item {Proveniência:(De \textunderscore abominar\textunderscore )}
\end{itemize}
Registo de anáthemas.
\section{Abominável}
\begin{itemize}
\item {Grp. gram.:adj.}
\end{itemize}
\begin{itemize}
\item {Proveniência:(Lat. \textunderscore abominabilis\textunderscore )}
\end{itemize}
Que merece abominação; detestável.
\section{Abominavelmente}
\begin{itemize}
\item {Grp. gram.:adv.}
\end{itemize}
De modo \textunderscore abominável\textunderscore .
\section{Abomínio}
\begin{itemize}
\item {Grp. gram.:m.}
\end{itemize}
O mesmo que \textunderscore abominação\textunderscore . Cf. Camillo, \textunderscore Insómnia\textunderscore , X, 61.
\section{Abominosamente}
\begin{itemize}
\item {Grp. gram.:adv.}
\end{itemize}
O mesmo que \textunderscore abominavelmente\textunderscore .
\section{Abominoso}
\begin{itemize}
\item {Grp. gram.:adj.}
\end{itemize}
O mesmo que \textunderscore abominável\textunderscore .
\section{Abonação}
\begin{itemize}
\item {Grp. gram.:f.}
\end{itemize}
Acto ou effeito de \textunderscore abonar\textunderscore .
\section{Abonadamente}
\begin{itemize}
\item {Grp. gram.:adv.}
\end{itemize}
\begin{itemize}
\item {Proveniência:(De \textunderscore abonado\textunderscore )}
\end{itemize}
Com abono.
\section{Abonado}
\begin{itemize}
\item {Grp. gram.:adj.}
\end{itemize}
\begin{itemize}
\item {Proveniência:(De \textunderscore abonar\textunderscore )}
\end{itemize}
Que dispõe de bons recursos; que tem crédito.
\section{Abonador}
\begin{itemize}
\item {Grp. gram.:m.  e  adj.}
\end{itemize}
O que abona; o que afiança.
\section{Abonamento}
\begin{itemize}
\item {Grp. gram.:m.}
\end{itemize}
O mesmo que \textunderscore abonação\textunderscore .
\section{Abonançar}
\begin{itemize}
\item {Grp. gram.:v. t.  e  i.}
\end{itemize}
Produzir bonança.
Serenar, sossegar.
\section{Abonar}
\begin{itemize}
\item {Grp. gram.:v. t.}
\end{itemize}
\begin{itemize}
\item {Utilização:Mús.}
\end{itemize}
\begin{itemize}
\item {Utilização:ant.}
\end{itemize}
Apresentar como bom.
Afiançar; ser fiador de.
Resolver (uma dissonância) sôbre a consonância.
\section{Abonatório}
\begin{itemize}
\item {Grp. gram.:adj.}
\end{itemize}
Proprio para abonar ou confirmar: \textunderscore documentos abonatórios da sua identidade\textunderscore .
\section{Abonaxi}
\begin{itemize}
\item {Grp. gram.:m.}
\end{itemize}
Animal, de que dizem que ladra como o cão.
\section{Abondança}
\begin{itemize}
\item {Grp. gram.:f.}
\end{itemize}
\begin{itemize}
\item {Utilização:Des.}
\end{itemize}
O mesmo que \textunderscore abundância\textunderscore . Cf. \textunderscore Lusiadas\textunderscore , V, 54.
\section{Abondar}
\begin{itemize}
\item {Grp. gram.:v. t.}
\end{itemize}
\begin{itemize}
\item {Utilização:Prov.}
\end{itemize}
\begin{itemize}
\item {Utilização:trasm.}
\end{itemize}
Aproximar, chegar a si ou ao alcance da mão.
\section{Abondo}
\begin{itemize}
\item {Grp. gram.:adj.}
\end{itemize}
\begin{itemize}
\item {Utilização:Ant.}
\end{itemize}
O mesmo que \textunderscore avonde\textunderscore .
\section{Abonecado}
\begin{itemize}
\item {Grp. gram.:adj.}
\end{itemize}
\begin{itemize}
\item {Proveniência:(De \textunderscore boneco\textunderscore )}
\end{itemize}
Que traja pretensiosamente; casquilho, peralta. Cf. \textunderscore Primo Basílio\textunderscore , 151.
\section{Abono}
\begin{itemize}
\item {Grp. gram.:m.}
\end{itemize}
\begin{itemize}
\item {Utilização:Mús.}
\end{itemize}
\begin{itemize}
\item {Utilização:ant.}
\end{itemize}
O mesmo que \textunderscore abonação\textunderscore .
Acto de abonar ou resolver uma dissonância.
O mesmo que \textunderscore resolução\textunderscore , ou terceira parte de uma prolongação musical.
\section{Aboquejar}
\begin{itemize}
\item {Grp. gram.:v. t.}
\end{itemize}
\begin{itemize}
\item {Utilização:Prov.}
\end{itemize}
\begin{itemize}
\item {Utilização:trasm.}
\end{itemize}
\begin{itemize}
\item {Proveniência:(De \textunderscore boca\textunderscore )}
\end{itemize}
O mesmo que \textunderscore abocanhar\textunderscore ^1.
Estar quási a dizêr (uma coisa), dá-la a perceber.
\section{Aboquejos}
\begin{itemize}
\item {Grp. gram.:m. pl.}
\end{itemize}
\begin{itemize}
\item {Utilização:Prov.}
\end{itemize}
\begin{itemize}
\item {Utilização:trasm.}
\end{itemize}
Vascas, últimos alentos vitaes, agonia de moribundo.
(Cp. \textunderscore aboquejar\textunderscore )
\section{Aborbitar}
\begin{itemize}
\item {Grp. gram.:v. i.}
\end{itemize}
\begin{itemize}
\item {Utilização:Ant.}
\end{itemize}
O mesmo que \textunderscore exorbitar\textunderscore .
\section{Aborbulhar}
\begin{itemize}
\item {Grp. gram.:v. i.}
\end{itemize}
Criar borbulhas.
\section{Aborcar}
\begin{itemize}
\item {Grp. gram.:v. t.}
\end{itemize}
O mesmo que \textunderscore emborcar\textunderscore .
\section{Abordada}
\begin{itemize}
\item {Grp. gram.:f.}
\end{itemize}
O mesmo que \textunderscore abordagem\textunderscore .
\section{Abordador}
\begin{itemize}
\item {Grp. gram.:m.  e  adj.}
\end{itemize}
O que aborda.
\section{Abordagem}
\begin{itemize}
\item {Grp. gram.:f.}
\end{itemize}
Acto ou effeito de \textunderscore abordar\textunderscore .
\section{Abordar}
\begin{itemize}
\item {Grp. gram.:v. t.}
\end{itemize}
Tocar com o bordo.
Abalroar (um navio) para o assaltar.
\section{Abordar}
\begin{itemize}
\item {Grp. gram.:v. t.}
\end{itemize}
\begin{itemize}
\item {Utilização:Gal}
\end{itemize}
\begin{itemize}
\item {Proveniência:(Fr. \textunderscore aborder\textunderscore )}
\end{itemize}
Abeirar-se de; chegar a; tocar.
\section{Abordável}
\begin{itemize}
\item {Grp. gram.:adj.}
\end{itemize}
Que se póde \textunderscore abordar\textunderscore .
\section{Abordo}
\begin{itemize}
\item {Grp. gram.:m.}
\end{itemize}
O mesmo que \textunderscore abordagem\textunderscore .
\section{Abordoar}
\begin{itemize}
\item {Grp. gram.:v. t.}
\end{itemize}
\begin{itemize}
\item {Grp. gram.:v. p.}
\end{itemize}
Firmar em bordão.
Apoiar-se, firmar-se.
\section{Aborígine}
\begin{itemize}
\item {Grp. gram.:adj.}
\end{itemize}
\begin{itemize}
\item {Grp. gram.:m. pl.}
\end{itemize}
\begin{itemize}
\item {Proveniência:(Lat. \textunderscore aborigines\textunderscore )}
\end{itemize}
Originário do país em que vive.
Primitivos habitantes.
\section{Abornalar}
\begin{itemize}
\item {Grp. gram.:v. t.}
\end{itemize}
(V.embornalar)
\section{Aborrascar-se}
\begin{itemize}
\item {Grp. gram.:v. p.}
\end{itemize}
\begin{itemize}
\item {Proveniência:(De \textunderscore borrasca\textunderscore )}
\end{itemize}
Tornar-se borrascoso.
\section{Aborrecedor}
\begin{itemize}
\item {Grp. gram.:m.  e  adj.}
\end{itemize}
O que aborrece.
\section{Aborrecer}
\begin{itemize}
\item {Grp. gram.:v. t.}
\end{itemize}
\begin{itemize}
\item {Proveniência:(De \textunderscore aborrir\textunderscore )}
\end{itemize}
Sentir horror por.
Causar horror a.
\section{Aborrecidamente}
\begin{itemize}
\item {Grp. gram.:adv.}
\end{itemize}
De modo \textunderscore aborrecido\textunderscore .
\section{Aborrecido}
\begin{itemize}
\item {Grp. gram.:adj.}
\end{itemize}
\begin{itemize}
\item {Proveniência:(De \textunderscore aborrecer\textunderscore )}
\end{itemize}
Que causa aborrecimento.
Enfadonho; \textunderscore conversa aborrecida\textunderscore .
\section{Aborrecimento}
\begin{itemize}
\item {Grp. gram.:m.}
\end{itemize}
Acto de \textunderscore aborrecer\textunderscore .
Tédio.
Repugnância.
\section{Aborrecível}
\begin{itemize}
\item {Grp. gram.:adj.}
\end{itemize}
\begin{itemize}
\item {Proveniência:(De \textunderscore aborrecer\textunderscore )}
\end{itemize}
Que causa aborrecimento.
Que merece sêr aborrecido.
\section{Aborregado}
\begin{itemize}
\item {Grp. gram.:adj.}
\end{itemize}
\begin{itemize}
\item {Utilização:Geol.}
\end{itemize}
Diz-se dos glaciares, quando a sua fronte se eleva, apresentando saliências lisas e arredondadas. Cf. G. Guimarães, \textunderscore Geol.\textunderscore , 170.
\section{Aborridamente}
\begin{itemize}
\item {Grp. gram.:adv.}
\end{itemize}
De modo \textunderscore aborrido\textunderscore .
\section{Aborrido}
\begin{itemize}
\item {Grp. gram.:adj.}
\end{itemize}
\begin{itemize}
\item {Proveniência:(De \textunderscore aborrir\textunderscore )}
\end{itemize}
Aborrecido.
Triste.
\section{Aborrimento}
\begin{itemize}
\item {Grp. gram.:m.}
\end{itemize}
Effeito de aborrir.
\section{Aborrir}
\begin{itemize}
\item {Grp. gram.:v. t.}
\end{itemize}
\begin{itemize}
\item {Proveniência:(Lat. \textunderscore abhorrere\textunderscore )}
\end{itemize}
O mesmo que \textunderscore aborrecer\textunderscore .
\section{Aborrível}
\begin{itemize}
\item {Grp. gram.:adj.}
\end{itemize}
O mesmo que \textunderscore aborrecível\textunderscore .
\section{Aborso}
\begin{itemize}
\item {Grp. gram.:m.}
\end{itemize}
\begin{itemize}
\item {Utilização:P. us.}
\end{itemize}
O mesmo que \textunderscore abôrto\textunderscore . Vieira, VI, 503; \textunderscore Luz e Calor\textunderscore , p. 278.
(B. lat. \textunderscore aborsus\textunderscore )
\section{Abortamento}
\begin{itemize}
\item {Grp. gram.:m.}
\end{itemize}
O mesmo que abôrto.
\section{Abortar}
\begin{itemize}
\item {Grp. gram.:v. t.}
\end{itemize}
\begin{itemize}
\item {Grp. gram.:V. i.}
\end{itemize}
Produzir antes de tempo.
Parir, antes do termo da gestação.
Mallograr-se.
(Ant. lat. \textunderscore abortare\textunderscore )
\section{Abortício}
\begin{itemize}
\item {Grp. gram.:adj.}
\end{itemize}
Que nasceu por abôrto.
\section{Abortífero}
\begin{itemize}
\item {Grp. gram.:adj.}
\end{itemize}
Que produz abôrto. Cf. Castilho, \textunderscore Factos\textunderscore , I, 507.
\section{Abortivo}
\begin{itemize}
\item {Grp. gram.:m.}
\end{itemize}
\begin{itemize}
\item {Grp. gram.:Adj.}
\end{itemize}
\begin{itemize}
\item {Proveniência:(Lat. \textunderscore abortivus\textunderscore )}
\end{itemize}
Substância, que faz abortar.
Que abortou.
Que fez abortar.
Que, é filho de abôrto. Cf. \textunderscore Luz e Calor\textunderscore , 448.
\section{Abôrto}
\begin{itemize}
\item {Grp. gram.:m.}
\end{itemize}
\begin{itemize}
\item {Proveniência:(Lat. \textunderscore abortus\textunderscore )}
\end{itemize}
Parto prematuro.
O que nasceu antes do tempo próprio.
Individuo, que nasceu disforme.
\section{Abossadura}
\begin{itemize}
\item {Grp. gram.:f.}
\end{itemize}
O mesmo que \textunderscore bossagem\textunderscore .
\section{Abostelado}
\begin{itemize}
\item {Grp. gram.:adj.}
\end{itemize}
Que tem bostellas.
\section{Abostelar}
\begin{itemize}
\item {Grp. gram.:v. i.}
\end{itemize}
Criar bostella.
\section{Abostellado}
\begin{itemize}
\item {Grp. gram.:adj.}
\end{itemize}
Que tem bostellas.
\section{Abostellar}
\begin{itemize}
\item {Grp. gram.:v. i.}
\end{itemize}
Criar bostella.
\section{Abotinado}
\begin{itemize}
\item {Grp. gram.:adj.}
\end{itemize}
Que tem fórma de botina.
\section{Abotinar}
\begin{itemize}
\item {Grp. gram.:v. t.}
\end{itemize}
Dar fórma de botina a.
\section{Abotoação}
\begin{itemize}
\item {Grp. gram.:m.}
\end{itemize}
\begin{itemize}
\item {Proveniência:(De \textunderscore abotoar\textunderscore )}
\end{itemize}
Formação de botões (na planta)
\section{Abotoadeira}
\begin{itemize}
\item {Grp. gram.:f.}
\end{itemize}
Instrumento para abotoar.
\section{Abotoador}
\begin{itemize}
\item {Grp. gram.:m.}
\end{itemize}
O que abotôa.
Abotoadeira.
\section{Abotoadura}
\begin{itemize}
\item {Grp. gram.:f.}
\end{itemize}
Jogo de botões para um vestuário.
Acto de \textunderscore abotoar\textunderscore .
\section{Abotoar}
\begin{itemize}
\item {Grp. gram.:v. t.}
\end{itemize}
\begin{itemize}
\item {Grp. gram.:V. i.}
\end{itemize}
\begin{itemize}
\item {Grp. gram.:V. p.}
\end{itemize}
\begin{itemize}
\item {Utilização:Bras}
\end{itemize}
\begin{itemize}
\item {Grp. gram.:V. p.}
\end{itemize}
\begin{itemize}
\item {Utilização:Fam.}
\end{itemize}
Fechar com botões.
Pregar botões em.
Lançar botões ou gomos (a planta)
Meter os botões nas respectivas casas, fechando o próprio vestuário.
Lançar gomos ou botões (uma planta)
Agarrar ou segurar pelos botões; segurar (alguém), deitando-lhe a mão ao peito. Cf. Macedo Soares, \textunderscore Dicc. Bras.\textunderscore 
Lucrar.
Locupletar-se.
\section{Abotocadura}
\begin{itemize}
\item {Grp. gram.:f.}
\end{itemize}
\begin{itemize}
\item {Utilização:Náut.}
\end{itemize}
\begin{itemize}
\item {Proveniência:(De \textunderscore abotocar\textunderscore )}
\end{itemize}
Designação genérica das cadeias, chapas e cavilhas, que seguram as mesas das enxárcias reaes, contra o costado do navio.
\section{Abotocar}
\begin{itemize}
\item {Grp. gram.:v.}
\end{itemize}
\begin{itemize}
\item {Utilização:t. Náut.}
\end{itemize}
O mesmo que \textunderscore abatocar\textunderscore .
\section{Abougar}
\begin{itemize}
\item {Grp. gram.:v. i.}
\end{itemize}
\begin{itemize}
\item {Utilização:T. da Bairrada}
\end{itemize}
Perder o tino, alucinar-se.
(Relaciona-se com \textunderscore apoucar\textunderscore ?)
\section{Abouvila}
\begin{itemize}
\item {Grp. gram.:f.}
\end{itemize}
\begin{itemize}
\item {Utilização:Ant.}
\end{itemize}
O mesmo que \textunderscore abovila\textunderscore .
\section{Abovila}
\begin{itemize}
\item {Grp. gram.:f.}
\end{itemize}
Tecido antigo, talvez fabricado em Abbeville, (França):«\textunderscore esta abovilla de quinze soldos a alna.\textunderscore »Herculano, \textunderscore Cister\textunderscore .
\section{Abovilla}
\begin{itemize}
\item {Grp. gram.:f.}
\end{itemize}
Tecido antigo, talvez fabricado em Abbeville, (França):«\textunderscore esta abovilla de quinze soldos a alna.\textunderscore »Herculano, \textunderscore Cister\textunderscore .
\section{Abra}
\begin{itemize}
\item {Grp. gram.:f.}
\end{itemize}
Ancoradoiro; baía.
(Cast. \textunderscore abra\textunderscore )
\section{Abracadabra}
\begin{itemize}
\item {Grp. gram.:m.}
\end{itemize}
Palavra mágica, a que os antigos attribuiam a virtude de curar moléstias várias, e cujas letras deviam ser escritas em triângulo, de modo que pudesse sêr lida de todos os lados.
\section{Abraçadeira}
\begin{itemize}
\item {Grp. gram.:f.}
\end{itemize}
\begin{itemize}
\item {Proveniência:(De \textunderscore abraçar\textunderscore )}
\end{itemize}
Chapa do ferro, para segurar paredes ou vigamentos.
Tira ou cordão, que abraça um cortinado, apanhando-o, e segurando-o ao lado.
\section{Abraçada}
\begin{itemize}
\item {Grp. gram.:f.}
\end{itemize}
\begin{itemize}
\item {Utilização:Prov.}
\end{itemize}
O mesmo que \textunderscore braçada\textunderscore .
\section{Abraçador}
\begin{itemize}
\item {Grp. gram.:m.}
\end{itemize}
Aquelle que abraça.
\section{Abraçamento}
\begin{itemize}
\item {Grp. gram.:m.}
\end{itemize}
Acto ou effeito de \textunderscore abraçar\textunderscore .
\section{Abraçar}
\begin{itemize}
\item {Grp. gram.:v. t.}
\end{itemize}
\begin{itemize}
\item {Utilização:Fig.}
\end{itemize}
\begin{itemize}
\item {Grp. gram.:V. p.}
\end{itemize}
Rodear ou cingir com os braços; apertar entre os braços.
Abranger: \textunderscore abraçar o horizonte com a vista\textunderscore .
Seguir, adoptar: \textunderscore abraçar o Protestantismo\textunderscore .
Cingir alguém com os braços:«\textunderscore abraçou-se nella e assim se ficaram\textunderscore ». Camillo, \textunderscore F. do Regicida\textunderscore .
\section{Abrachia}
\begin{itemize}
\item {fónica:qui}
\end{itemize}
\begin{itemize}
\item {Grp. gram.:f.}
\end{itemize}
Ausência congênita de braços.
(Cp. \textunderscore abráchio\textunderscore )
\section{Abráchio}
\begin{itemize}
\item {fónica:qui}
\end{itemize}
\begin{itemize}
\item {Grp. gram.:m.}
\end{itemize}
\begin{itemize}
\item {Proveniência:(Do gr. \textunderscore a\textunderscore  priv. e \textunderscore brakhion\textunderscore , braço)}
\end{itemize}
Monstro sem braços.
\section{Abraço}
\begin{itemize}
\item {Grp. gram.:m.}
\end{itemize}
\begin{itemize}
\item {Utilização:Archit.}
\end{itemize}
Acto de \textunderscore abraçar\textunderscore .
Gavinha.
Entrelaçamento de folhagens lavradas, á volta de uma columna.
\section{Abraêmo}
\begin{itemize}
\item {Grp. gram.:m.}
\end{itemize}
Antiga moéda de Gôa.
\section{Abrahâmico}
\begin{itemize}
\item {Grp. gram.:adj.}
\end{itemize}
Relativo ao patriarcha hebreu Abrahão.
\section{Abrancaçado}
\begin{itemize}
\item {Grp. gram.:adj.}
\end{itemize}
\begin{itemize}
\item {Utilização:Prov.}
\end{itemize}
\begin{itemize}
\item {Utilização:minh.}
\end{itemize}
O mesmo que \textunderscore esbranquiçado\textunderscore . Cf. \textunderscore Gaz. das Aldeias\textunderscore , 705.
\section{Abrandamento}
\begin{itemize}
\item {Grp. gram.:m.}
\end{itemize}
Acto de \textunderscore abrandar\textunderscore .
\section{Abrandar}
\begin{itemize}
\item {Grp. gram.:v. t.}
\end{itemize}
\begin{itemize}
\item {Utilização:Fig.}
\end{itemize}
\begin{itemize}
\item {Grp. gram.:V. i.}
\end{itemize}
Tornar brando.
Suavizar: \textunderscore abrandar mágoas\textunderscore .
Serenar.
Tornar-se brando, menos intenso: \textunderscore o calor abrandou\textunderscore .
\section{Abrandecer}
\begin{itemize}
\item {Grp. gram.:v. t.}
\end{itemize}
(V.embrandecer)
\section{Abranger}
\begin{itemize}
\item {Grp. gram.:v. t.}
\end{itemize}
\begin{itemize}
\item {Utilização:Prov.}
\end{itemize}
\begin{itemize}
\item {Utilização:beir.}
\end{itemize}
\begin{itemize}
\item {Proveniência:(Do lat. \textunderscore vergere\textunderscore , segundo Cornu)}
\end{itemize}
Abraçar, cingir: \textunderscore abranger um tronco\textunderscore .
Conter: \textunderscore Lisbôa abrange muitos monumentos\textunderscore .
Comprehender: \textunderscore abranger os grandes problemas\textunderscore .
Abarcar: \textunderscore abranger o horizonte\textunderscore .
Aproximar, segurando com a mão: \textunderscore abrange-me essa cadeira\textunderscore .
\section{Abraquia}
\begin{itemize}
\item {Grp. gram.:f.}
\end{itemize}
Ausência congênita de braços.
(Cp. \textunderscore abráchio\textunderscore )
\section{Abráquio}
\begin{itemize}
\item {Grp. gram.:m.}
\end{itemize}
\begin{itemize}
\item {Proveniência:(Do gr. \textunderscore a\textunderscore  priv. e \textunderscore brakhion\textunderscore , braço)}
\end{itemize}
Monstro sem braços.
\section{Abrasadamente}
\begin{itemize}
\item {Grp. gram.:adv.}
\end{itemize}
De modo \textunderscore abrasado\textunderscore .
Fogosamente.
\section{Abrasador}
\begin{itemize}
\item {Grp. gram.:adj.}
\end{itemize}
Que abrasa; ardente: \textunderscore sol abrasador\textunderscore .
\section{Abrasamento}
\begin{itemize}
\item {Grp. gram.:m.}
\end{itemize}
Acto ou effeito de \textunderscore abrasar\textunderscore .
\section{Abrasante}
\begin{itemize}
\item {Grp. gram.:adj.}
\end{itemize}
Que abrasa.
\section{Abrasão}
\begin{itemize}
\item {Grp. gram.:f.}
\end{itemize}
\begin{itemize}
\item {Utilização:Cir.}
\end{itemize}
\begin{itemize}
\item {Proveniência:(Lat. \textunderscore abrasio\textunderscore )}
\end{itemize}
Raspagem dos ossos cariados.
\section{Abrasar}
\begin{itemize}
\item {Grp. gram.:v. t.}
\end{itemize}
\begin{itemize}
\item {Utilização:Fig.}
\end{itemize}
Converter em brasas.
Queimar, incendiar.
Aquecer: \textunderscore o sol abrasa o areal\textunderscore .
Enthusiasmar.
\section{Abrasear}
\begin{itemize}
\item {Grp. gram.:v. t.}
\end{itemize}
O mesmo que \textunderscore esbrasear\textunderscore . Cf. Camillo, \textunderscore Brasileira\textunderscore , 348.
\section{Abrasileirado}
\begin{itemize}
\item {Grp. gram.:adj.}
\end{itemize}
Que tem modos de brasileiro.
\section{Abrasileirar}
\begin{itemize}
\item {Grp. gram.:v. t.}
\end{itemize}
Dar feição brasileira a.
\section{Abrasonar}
\begin{itemize}
\item {Grp. gram.:v. t.}
\end{itemize}
Dar brasão a; pôr brasão em. Cf. Camillo, \textunderscore Cancion. Al.\textunderscore , 208.
\section{Abrastol}
\begin{itemize}
\item {Grp. gram.:m.}
\end{itemize}
O mesmo que \textunderscore asaprol\textunderscore .
\section{Abraxas}
\begin{itemize}
\item {Grp. gram.:m. pl.}
\end{itemize}
Pedras gravadas, que contêm sýmbolos religiosos de certas seitas.
(Palavra, inventada pelo heresiarcha Basílides e formada de sete letras que, tomadas numericamente, formavam entre os Gregos o número 365, número dos dias do anno)
\section{Abrazite}
\begin{itemize}
\item {Grp. gram.:f.}
\end{itemize}
Substância branca, composta de sílica, alumìnio e cal.
\section{Abre-boca}
\begin{itemize}
\item {Grp. gram.:m.}
\end{itemize}
Instrumento, com que os alveitares abriam a boca dos animaes.
Instrumento, com que se abre a boca a um doente anesthesiado, ou a quem se não consegue abri-la, senão forçadamente.
\section{Abrecu}
\begin{itemize}
\item {Grp. gram.:m.}
\end{itemize}
\begin{itemize}
\item {Utilização:Prov.}
\end{itemize}
\begin{itemize}
\item {Utilização:dur.}
\end{itemize}
O mesmo que \textunderscore pyrilampo\textunderscore .
\section{Ábrego}
\begin{itemize}
\item {Grp. gram.:m.}
\end{itemize}
\begin{itemize}
\item {Utilização:Ant.}
\end{itemize}
\begin{itemize}
\item {Proveniência:(Lat. \textunderscore africus\textunderscore )}
\end{itemize}
Vento do sudoéste.
O lado do sul, nas demarcações antigas.
\section{Abre-ilhós}
\begin{itemize}
\item {Grp. gram.:m.}
\end{itemize}
Instrumento, com que se abrem buracos para ilhós; furador.
\section{Abrejeirado}
\begin{itemize}
\item {Grp. gram.:adj.}
\end{itemize}
Que tem modos de brejeiro.
Que tem aspecto de brejeirice.
\section{Abrenhar}
\begin{itemize}
\item {Grp. gram.:v. t.}
\end{itemize}
(V.embrenhar)
\section{Abrenunciação}
\begin{itemize}
\item {Grp. gram.:f.}
\end{itemize}
Acto de \textunderscore abrenunciar\textunderscore .
\section{Abrenunciar}
\begin{itemize}
\item {Grp. gram.:v. t.}
\end{itemize}
\begin{itemize}
\item {Proveniência:(De \textunderscore ab...\textunderscore  + \textunderscore renunciar\textunderscore )}
\end{itemize}
Renunciar, repellir.
\section{Abrenúncio!}
\begin{itemize}
\item {Grp. gram.:interj.}
\end{itemize}
\begin{itemize}
\item {Proveniência:(Do lat. \textunderscore ab\textunderscore  + \textunderscore renuntio\textunderscore )}
\end{itemize}
Ápage!
Deus me livre!
\section{Abreptício}
\begin{itemize}
\item {fónica:ré}
\end{itemize}
\begin{itemize}
\item {Grp. gram.:adj.}
\end{itemize}
\begin{itemize}
\item {Proveniência:(Do lat. \textunderscore abreptus\textunderscore )}
\end{itemize}
Exaltado; arrebatado.
\section{Abretanhado}
\begin{itemize}
\item {Grp. gram.:adj.}
\end{itemize}
Semelhante ao pano chamado bretanha.
\section{Abreu}
\begin{itemize}
\item {Grp. gram.:m.}
\end{itemize}
\begin{itemize}
\item {Utilização:Bras}
\end{itemize}
Espécie de abelha do Piauí.
\section{Abrevar}
\begin{itemize}
\item {Grp. gram.:v. t.}
\end{itemize}
\begin{itemize}
\item {Utilização:Ant.}
\end{itemize}
\begin{itemize}
\item {Grp. gram.:V. i.}
\end{itemize}
\begin{itemize}
\item {Proveniência:(Fr. \textunderscore abreuver\textunderscore )}
\end{itemize}
Matar a sêde a. Cf. Filinto, XIV, 139; Usque, \textunderscore Tribulações\textunderscore ; Castilho, \textunderscore Fastos\textunderscore , I, 139, e \textunderscore Geórgicas\textunderscore , 157.
Dessedentar-se. Cf. Usque, \textunderscore loc. cit.\textunderscore  f. 29, v.^o
\section{Abreviação}
\begin{itemize}
\item {Grp. gram.:f.}
\end{itemize}
Acto ou effeito de \textunderscore abreviar\textunderscore .
\section{Abreviadamente}
\begin{itemize}
\item {Grp. gram.:adj.}
\end{itemize}
De modo \textunderscore abreviado\textunderscore .
Em resumo.
\section{Abreviado}
\begin{itemize}
\item {Grp. gram.:adj.}
\end{itemize}
\begin{itemize}
\item {Proveniência:(De \textunderscore abreviar\textunderscore )}
\end{itemize}
Reduzido a epitome; resumido: \textunderscore história abreviada da Igreja\textunderscore .
\section{Abreviador}
\begin{itemize}
\item {Grp. gram.:m.}
\end{itemize}
Aquelle que abrevia.
\section{Abreviamento}
\begin{itemize}
\item {Grp. gram.:m.}
\end{itemize}
O mesmo que \textunderscore abreviação\textunderscore .
\section{Abreviar}
\begin{itemize}
\item {Grp. gram.:v. t.}
\end{itemize}
\begin{itemize}
\item {Proveniência:(Lat. \textunderscore abbreviare\textunderscore )}
\end{itemize}
Tornar breve, resumir.
\section{Abreviativo}
\begin{itemize}
\item {Grp. gram.:adj.}
\end{itemize}
Que serve para abreviar.
\section{Abreviatura}
\begin{itemize}
\item {Grp. gram.:f.}
\end{itemize}
Resumo.
Fracção de palavra, designando-a toda.
\section{Abricó}
\begin{itemize}
\item {Grp. gram.:m.}
\end{itemize}
\begin{itemize}
\item {Proveniência:(Fr. \textunderscore abricot\textunderscore )}
\end{itemize}
Fruto brasileiro, semelhante ao damasco, mas menor.
\section{Abricote}
\begin{itemize}
\item {Grp. gram.:m.}
\end{itemize}
\begin{itemize}
\item {Proveniência:(Fr. \textunderscore abricot\textunderscore )}
\end{itemize}
Fruto brasileiro, semelhante ao damasco, mas menor.
\section{Abricoteiro}
\begin{itemize}
\item {Grp. gram.:m.}
\end{itemize}
Árvore gutífera, que produz o abricote.
\section{Abrideira}
\begin{itemize}
\item {Grp. gram.:f.}
\end{itemize}
\begin{itemize}
\item {Utilização:Bras}
\end{itemize}
Pequena porção de bebida alcoólica, que de ordinário se toma antes da comida, para abrir o appetite.
\section{Abridor}
\begin{itemize}
\item {Grp. gram.:m.}
\end{itemize}
O que abre.
Gravador.
Burilador.
\section{Abriga}
\begin{itemize}
\item {Grp. gram.:f.}
\end{itemize}
O mesmo que \textunderscore abrigo\textunderscore . Cf. \textunderscore Hist. Trág. Marit.\textunderscore , 106.
\section{Abrigada}
\begin{itemize}
\item {Grp. gram.:f.}
\end{itemize}
Lugar que abriga, ou em que há abrigo.
\section{Abrigadoiro}
\begin{itemize}
\item {Grp. gram.:m.}
\end{itemize}
O mesmo que \textunderscore abrigada\textunderscore .
\section{Abrigador}
\begin{itemize}
\item {Grp. gram.:m.  e  adj.}
\end{itemize}
O que abriga.
\section{Abrigadouro}
\begin{itemize}
\item {Grp. gram.:m.}
\end{itemize}
O mesmo que \textunderscore abrigada\textunderscore .
\section{Abrigar}
\begin{itemize}
\item {Grp. gram.:v. t.}
\end{itemize}
\begin{itemize}
\item {Proveniência:(Lat. \textunderscore apricare\textunderscore )}
\end{itemize}
Dar abrigo a; agasalhar.
Proteger.
\section{Abrigo}
\begin{itemize}
\item {Grp. gram.:m.}
\end{itemize}
\begin{itemize}
\item {Proveniência:(Lat. \textunderscore apricum\textunderscore )}
\end{itemize}
Resguardo; cobertura.
Protecção.
\section{Abrigoír}
\begin{itemize}
\item {Grp. gram.:v. i.}
\end{itemize}
\begin{itemize}
\item {Utilização:Açor}
\end{itemize}
Contender, disputar.
\section{Abrigoso}
\begin{itemize}
\item {Grp. gram.:adj.}
\end{itemize}
Que fornece ou dá abrigo. Cf. Th. Ribeiro, \textunderscore Jornadas\textunderscore , I, 123.
\section{Abril}
\begin{itemize}
\item {Grp. gram.:m.}
\end{itemize}
\begin{itemize}
\item {Utilização:Fig.}
\end{itemize}
\begin{itemize}
\item {Proveniência:(Lat. \textunderscore aprilis\textunderscore )}
\end{itemize}
Quarto mês do anno gregoriano.
Idade da alegria e da innocência; juventude.
\section{Abrilada}
\begin{itemize}
\item {Grp. gram.:f.}
\end{itemize}
Acontecimento em Abril.
Revolta de Abril de 1824.
\section{Abrilhantar}
\begin{itemize}
\item {Grp. gram.:v. t.}
\end{itemize}
Tornar brilhante.
\section{Abrimento}
\begin{itemize}
\item {Grp. gram.:m.}
\end{itemize}
O mesmo que \textunderscore abertura\textunderscore .
\textunderscore Abrimento de boca\textunderscore , bocejo.
\section{Abrir}
\begin{itemize}
\item {Grp. gram.:v. t.}
\end{itemize}
\begin{itemize}
\item {Grp. gram.:V. i.}
\end{itemize}
\begin{itemize}
\item {Grp. gram.:V. p.}
\end{itemize}
\begin{itemize}
\item {Proveniência:(Do lat. \textunderscore aperire\textunderscore )}
\end{itemize}
Desunir, descerrar: \textunderscore abrir a janela\textunderscore .
Romper, cortar: \textunderscore abrir as veias\textunderscore .
Desimpedir, desobstruir: \textunderscore abrir caminho\textunderscore .
Começar.
Devassar.
Escavar: \textunderscore abrir um poço\textunderscore .
Gravar: \textunderscore abrir letras\textunderscore .
\textunderscore Abrir o sinal\textunderscore , registar o nome nos livros dos notários.
\textunderscore Abrir mão de\textunderscore , largar, pôr de lado.
\textunderscore Abrir crédito a alguém\textunderscore , autorizá-lo a dispor de certa quantia.
Descerrar as pétalas: \textunderscore a flor abriu\textunderscore .
Romper, começar: \textunderscore abriu o dia\textunderscore .
\textunderscore Abrir o tempo\textunderscore , aliviar-se o tempo.
Abrir a porta a alguém:«\textunderscore vindo de noite, sua esposa não lhe quis abrir\textunderscore ». \textunderscore Luz e Calor\textunderscore , 296.
Têr expansões, sêr franco: \textunderscore abriu-se comigo\textunderscore .
\section{Abro}
\begin{itemize}
\item {Grp. gram.:m.}
\end{itemize}
\begin{itemize}
\item {Proveniência:(Gr. \textunderscore abrios\textunderscore )}
\end{itemize}
Planta papilionácea.
\section{Abrocadado}
\begin{itemize}
\item {Grp. gram.:adj.}
\end{itemize}
Semelhante ao brocado.
\section{Abrochador}
\begin{itemize}
\item {Grp. gram.:m.}
\end{itemize}
Aquelle ou aquillo que abrocha.
\section{Abrochadura}
\begin{itemize}
\item {Grp. gram.:f.}
\end{itemize}
Acto ou effeito de \textunderscore abrochar\textunderscore .
\section{Abrochar}
\begin{itemize}
\item {Grp. gram.:v. t.}
\end{itemize}
Lidar com broche ou com brocha.
Abotoar.
Apertar.
\section{Abrogação}
\begin{itemize}
\item {fónica:ro}
\end{itemize}
\begin{itemize}
\item {Grp. gram.:f.}
\end{itemize}
\begin{itemize}
\item {Proveniência:(Lat. \textunderscore abrogatio\textunderscore )}
\end{itemize}
Acto de abrogar.
\section{Abrogador}
\begin{itemize}
\item {fónica:ro}
\end{itemize}
\begin{itemize}
\item {Grp. gram.:m.}
\end{itemize}
\begin{itemize}
\item {Proveniência:(Lat. \textunderscore abrogator\textunderscore )}
\end{itemize}
O que abroga.
\section{Abrogar}
\begin{itemize}
\item {fónica:ro}
\end{itemize}
\begin{itemize}
\item {Grp. gram.:v. t.}
\end{itemize}
\begin{itemize}
\item {Proveniência:(Lat. \textunderscore abrogare\textunderscore )}
\end{itemize}
Annullar.
Supprimír.
Pôr fóra do uso.
Derogar: \textunderscore abrogar uma lei\textunderscore .
\section{Abrogativo}
\begin{itemize}
\item {fónica:ro}
\end{itemize}
\begin{itemize}
\item {Grp. gram.:adj.}
\end{itemize}
Que abroga, ou que produz abrogação.
\section{Abrogatório}
\begin{itemize}
\item {fónica:ro}
\end{itemize}
\begin{itemize}
\item {Grp. gram.:adj.}
\end{itemize}
Que abroga, ou que produz abrogação.
\section{Abrolhador}
\begin{itemize}
\item {Grp. gram.:adj.}
\end{itemize}
Que abrolha.
\section{Abrolhal}
\begin{itemize}
\item {Grp. gram.:m.}
\end{itemize}
Lugar, onde crescem abrolhos.
\section{Abrolhamento}
\begin{itemize}
\item {Grp. gram.:m.}
\end{itemize}
Acto de \textunderscore abrolhar\textunderscore .
\section{Abrolhar}
\begin{itemize}
\item {Grp. gram.:v. t.}
\end{itemize}
\begin{itemize}
\item {Grp. gram.:V. i.}
\end{itemize}
Cobrir de abrolhos.
Produzir abrolhos.
Lançar gomos ou rebentos.
\section{Abrolho}
\begin{itemize}
\item {fónica:brô}
\end{itemize}
\begin{itemize}
\item {Grp. gram.:m.}
\end{itemize}
\begin{itemize}
\item {Utilização:Fig.}
\end{itemize}
\begin{itemize}
\item {Proveniência:(De \textunderscore abrir\textunderscore  + \textunderscore ôlho\textunderscore )}
\end{itemize}
Planta herbácea, de fruto espinhoso.
Espinho dêsse fruto.
Contrariedade.
Mortificação: \textunderscore os abrolhos da vida\textunderscore .
\section{Abrolhoso}
\begin{itemize}
\item {Grp. gram.:adj.}
\end{itemize}
\begin{itemize}
\item {Utilização:Fig.}
\end{itemize}
Coberto de abrolhos.
Espinhoso.
Cheio de obstáculos.
\section{Abroma}
\begin{itemize}
\item {Grp. gram.:m.}
\end{itemize}
Planta intertropical, de cuja entrecasca fazem cordas os Índios.
\section{Abrónia}
\begin{itemize}
\item {Grp. gram.:f.}
\end{itemize}
Planta nyctaginea.
\section{Abronzear}
\begin{itemize}
\item {Grp. gram.:v. t.}
\end{itemize}
(V.bronzear)
\section{Abroquelar}
\begin{itemize}
\item {Grp. gram.:v. t.}
\end{itemize}
Resguardar com broquel.
Proteger; defender.
\section{Abróstolo}
\begin{itemize}
\item {Grp. gram.:m.}
\end{itemize}
\begin{itemize}
\item {Proveniência:(Do gr. \textunderscore abros\textunderscore  + \textunderscore stole\textunderscore )}
\end{itemize}
Insecto lepidóptero, da fam. dos nocturnos.
\section{Abrotal}
\begin{itemize}
\item {Grp. gram.:adj.}
\end{itemize}
\begin{itemize}
\item {Utilização:Des.}
\end{itemize}
Lugar, onde crescem abróteas.
\section{Abrótano}
\begin{itemize}
\item {Grp. gram.:m.}
\end{itemize}
(Corr. de \textunderscore abrótono\textunderscore )
\section{Abrotar}
\begin{itemize}
\item {Grp. gram.:v. i.}
\end{itemize}
(V.brotar)
\section{Abrótea}
\begin{itemize}
\item {Grp. gram.:f.}
\end{itemize}
\begin{itemize}
\item {Proveniência:(Do gr. \textunderscore abrotos\textunderscore ?)}
\end{itemize}
Planta liliácea medicinal.
\section{Abroteal}
\begin{itemize}
\item {Grp. gram.:m.}
\end{itemize}
Lugar, onde crescem abróteas.
\section{Abrótega}
\begin{itemize}
\item {Grp. gram.:f.}
\end{itemize}
\begin{itemize}
\item {Utilização:Prov.}
\end{itemize}
\begin{itemize}
\item {Utilização:beir.}
\end{itemize}
O mesmo que \textunderscore abrótono\textunderscore .
O mesmo que \textunderscore abrótea\textunderscore .
\section{Abrótia}
\begin{itemize}
\item {Grp. gram.:f.}
\end{itemize}
(Melhor escrita que \textunderscore abrótea\textunderscore )
\section{Abrótica}
\begin{itemize}
\item {Grp. gram.:f.}
\end{itemize}
Peixe de Portugal.
\section{Abrotinas}
\begin{itemize}
\item {Grp. gram.:f. pl.}
\end{itemize}
\begin{itemize}
\item {Utilização:Prov.}
\end{itemize}
\begin{itemize}
\item {Utilização:alg.}
\end{itemize}
\begin{itemize}
\item {Proveniência:(De \textunderscore abrotar\textunderscore )}
\end{itemize}
O mesmo que \textunderscore varicella\textunderscore .
\section{Abrótono}
\begin{itemize}
\item {Grp. gram.:m.}
\end{itemize}
\begin{itemize}
\item {Proveniência:(Gr. \textunderscore abrotonon\textunderscore )}
\end{itemize}
Arbusto, da tribo das artemísias.
\section{Abrotonoide}
\begin{itemize}
\item {Grp. gram.:f.}
\end{itemize}
\begin{itemize}
\item {Proveniência:(Do gr. \textunderscore abrotonon\textunderscore  + \textunderscore eidos\textunderscore )}
\end{itemize}
Espécie de madrépora, que vive no fundo do mar, em rochas.
\section{Abrumar}
\begin{itemize}
\item {Grp. gram.:v. t.}
\end{itemize}
Cobrir de bruma.
Tornar escuro.
Tornar aprehensivo, triste. Cf. Latino, \textunderscore Camões\textunderscore , 25.
\section{Abrunhal}
\begin{itemize}
\item {Grp. gram.:m.}
\end{itemize}
Variedade de uva.
\section{Abrunheiro}
\begin{itemize}
\item {Grp. gram.:m.}
\end{itemize}
\begin{itemize}
\item {Proveniência:(De \textunderscore abrunho\textunderscore )}
\end{itemize}
Planta rosácea, da tribo das amygdáleas.
\section{Abrunho}
\begin{itemize}
\item {Grp. gram.:m.}
\end{itemize}
\begin{itemize}
\item {Proveniência:(Do lat. \textunderscore pruneum\textunderscore )}
\end{itemize}
Fruto do abrunheiro.
\section{Abrunho-do-duque}
\begin{itemize}
\item {Grp. gram.:m.}
\end{itemize}
Espécie de ameixa redonda, vermelho-escura, maculada de azul.
\section{Abrunho-do-rei}
\begin{itemize}
\item {Grp. gram.:m.}
\end{itemize}
Espécie de ameixa redonda, de côr acerejada, maculada de azul.
\section{Abrupção}
\begin{itemize}
\item {fónica:ru}
\end{itemize}
\begin{itemize}
\item {Grp. gram.:f.}
\end{itemize}
\begin{itemize}
\item {Proveniência:(Lat. \textunderscore abruptio\textunderscore )}
\end{itemize}
Fractura de osso.
\section{Abruptamente}
\begin{itemize}
\item {fónica:ru}
\end{itemize}
\begin{itemize}
\item {Grp. gram.:adv.}
\end{itemize}
\begin{itemize}
\item {Proveniência:(De \textunderscore abrupto\textunderscore )}
\end{itemize}
Em grande declive.
De repente.
\section{Abruptela}
\begin{itemize}
\item {fónica:ru}
\end{itemize}
\begin{itemize}
\item {Grp. gram.:f.}
\end{itemize}
\begin{itemize}
\item {Proveniência:(De \textunderscore abrupto\textunderscore )}
\end{itemize}
Terra desbravada.
\section{Abrupto}
\begin{itemize}
\item {fónica:ru}
\end{itemize}
\begin{itemize}
\item {Grp. gram.:adj.}
\end{itemize}
\begin{itemize}
\item {Proveniência:(Lat. \textunderscore abruptus\textunderscore )}
\end{itemize}
Íngreme; com grande inclinação.
Repentino.
\section{Abruso}
\begin{itemize}
\item {Grp. gram.:m.}
\end{itemize}
\begin{itemize}
\item {Proveniência:(Do gr. \textunderscore abros\textunderscore )}
\end{itemize}
Planta papilionácea, semelhante á acácia.
\section{Abrutadamente}
\begin{itemize}
\item {Grp. gram.:adv.}
\end{itemize}
De modo \textunderscore abrutado\textunderscore .
\section{Abrutado}
\begin{itemize}
\item {Grp. gram.:adj.}
\end{itemize}
\begin{itemize}
\item {Proveniência:(De \textunderscore abrutar\textunderscore )}
\end{itemize}
Que tem modos grosseiros.
Bruto.
Brutal.
\section{Abrutalhado}
\begin{itemize}
\item {Grp. gram.:adj.}
\end{itemize}
\begin{itemize}
\item {Proveniência:(De \textunderscore bruto\textunderscore )}
\end{itemize}
Grosseiro.
Brusco.
\section{Abrutalhar-se}
\begin{itemize}
\item {Grp. gram.:v. p.}
\end{itemize}
Tornar-se abrutalhado.
\section{Abrutamento}
\begin{itemize}
\item {Grp. gram.:m.}
\end{itemize}
O mesmo que \textunderscore brutalidade\textunderscore .
\section{Abrutar}
\begin{itemize}
\item {Grp. gram.:v. t.}
\end{itemize}
Tornar grosseiro, bruto.
\section{Abrutecer}
\begin{itemize}
\item {Grp. gram.:v. t.}
\end{itemize}
O mesmo que \textunderscore embrutecer\textunderscore .
\section{Abs...}
\begin{itemize}
\item {Grp. gram.:pref.}
\end{itemize}
O mesmo que \textunderscore ab...\textunderscore 
\section{Absceder}
\begin{itemize}
\item {Grp. gram.:v. i.}
\end{itemize}
\begin{itemize}
\item {Proveniência:(Do lat. \textunderscore abscedere\textunderscore )}
\end{itemize}
Degenerar em abcesso.
Supurar.
\section{Abscesso}
\begin{itemize}
\item {Grp. gram.:m.}
\end{itemize}
\begin{itemize}
\item {Proveniência:(Do lat. \textunderscore abscessus\textunderscore )}
\end{itemize}
Tumor.
Inchação, produzida pela formação do pus.
Pus, acumulado no tumor.
\section{Abscisão}
\begin{itemize}
\item {Grp. gram.:f.}
\end{itemize}
\begin{itemize}
\item {Proveniência:(Do lat. \textunderscore abscisio\textunderscore )}
\end{itemize}
Córte na parte carnosa do corpo.
\section{Abscissa}
\begin{itemize}
\item {Grp. gram.:f.}
\end{itemize}
\begin{itemize}
\item {Utilização:Geom.}
\end{itemize}
\begin{itemize}
\item {Proveniência:(Do lat. \textunderscore abscissa\textunderscore )}
\end{itemize}
Uma das coordenadas que servem para fixar um ponto num plano.
\section{Abscondado}
\begin{itemize}
\item {Grp. gram.:adv.}
\end{itemize}
\begin{itemize}
\item {Utilização:Ant.}
\end{itemize}
Ás escondidas, occultamente.
\section{Absconder}
\begin{itemize}
\item {Grp. gram.:v. t.}
\end{itemize}
(V.esconder)
\section{Abscônsia}
\begin{itemize}
\item {Grp. gram.:f.}
\end{itemize}
Lâmpada de dormitório, usada antigamente nalguns mosteiros.
(B. lat. \textunderscore absconsa\textunderscore )
\section{Absconso}
\begin{itemize}
\item {Grp. gram.:m.  e  adj.}
\end{itemize}
O mesmo que \textunderscore esconso\textunderscore . Cf. Filinto, X, 144; XIV, 119.
\section{Absência}
\begin{itemize}
\item {Grp. gram.:f.}
\end{itemize}
\begin{itemize}
\item {Utilização:Ant.}
\end{itemize}
(V.ausência)
\section{Absentar}
\begin{itemize}
\item {Grp. gram.:v. t.}
\end{itemize}
\begin{itemize}
\item {Utilização:Ant.}
\end{itemize}
(V. [[ausentar|ausentar-se]])
\section{Absente}
\begin{itemize}
\item {Grp. gram.:adj.}
\end{itemize}
\begin{itemize}
\item {Utilização:Ant.}
\end{itemize}
O mesmo que \textunderscore ausente\textunderscore .
\section{Absentismo}
\begin{itemize}
\item {Grp. gram.:m.}
\end{itemize}
\begin{itemize}
\item {Proveniência:(De \textunderscore absente\textunderscore )}
\end{itemize}
Systema de exploração agricola, em que há um gerente ou feitor intermediário ao cultivador e ao proprietário ausente.
\section{Abissal}
\begin{itemize}
\item {Grp. gram.:adj}
\end{itemize}
\begin{itemize}
\item {Proveniência:(De \textunderscore abysso\textunderscore )}
\end{itemize}
Relativo ao abysso.
Relativo ás profundidades marítimas.
Que vive na profundidade do mar.
\section{Abissínio}
\begin{itemize}
\item {Grp. gram.:m.}
\end{itemize}
O mesmo que \textunderscore abexim\textunderscore .
\section{Abisso}
\begin{itemize}
\item {Grp. gram.:m.}
\end{itemize}
\begin{itemize}
\item {Proveniência:(Gr. \textunderscore abussos\textunderscore )}
\end{itemize}
O mesmo que \textunderscore abismo\textunderscore .
Gênero de plantas de jardim.
\section{Absentista}
\begin{itemize}
\item {Grp. gram.:m.}
\end{itemize}
Proprietário de terras, exploradas pelo systema do absentismo.
\section{Ábsida}
\begin{itemize}
\item {Grp. gram.:f.}
\end{itemize}
O mesmo que \textunderscore ábside\textunderscore .
\section{Absidal}
\begin{itemize}
\item {Grp. gram.:adj.}
\end{itemize}
Que tem fórma de ábside.
\section{Ábside}
\begin{itemize}
\item {Grp. gram.:f.}
\end{itemize}
\begin{itemize}
\item {Proveniência:(Lat. \textunderscore absidem\textunderscore )}
\end{itemize}
Capella-mór.
Oratório reservado, atrás do altar-mór.
Relicário de ossos de santos, exposto no altar.
Curvatura, de abóbada.
Círculo, que os astros descrevem em seu movimento.
\section{Absintar}
\begin{itemize}
\item {Grp. gram.:v. t.}
\end{itemize}
Misturar com absintho.
Tornar amargo.
\section{Absintato}
\begin{itemize}
\item {Grp. gram.:m.}
\end{itemize}
\begin{itemize}
\item {Proveniência:(Lat. \textunderscore absinthiatus\textunderscore )}
\end{itemize}
Sal, produzido pela combinação do ácido absínthico com uma base salinável.
\section{Absinthar}
\begin{itemize}
\item {Grp. gram.:v. t.}
\end{itemize}
Misturar com absintho.
Tornar amargo.
\section{Absinthato}
\begin{itemize}
\item {Grp. gram.:m.}
\end{itemize}
\begin{itemize}
\item {Proveniência:(Lat. \textunderscore absinthiatus\textunderscore )}
\end{itemize}
Sal, produzido pela combinação do ácido absínthico com uma base salinável.
\section{Absínthico}
\begin{itemize}
\item {Grp. gram.:adj.}
\end{itemize}
Diz-se do ácido que se descobriu no absintho.
\section{Absinthina}
\begin{itemize}
\item {Grp. gram.:f.}
\end{itemize}
Principio amargo do absintho.
\section{Absínthio}
\begin{itemize}
\item {Grp. gram.:m.}
\end{itemize}
\begin{itemize}
\item {Proveniência:(Lat. \textunderscore absinthium\textunderscore )}
\end{itemize}
Planta vivaz, de sabor amargo e aromático.
Losna.
\section{Absinthismo}
\begin{itemize}
\item {Grp. gram.:m.}
\end{itemize}
Doença, causada pelo abuso do absintho.
\section{Absinthite}
\begin{itemize}
\item {Grp. gram.:m.}
\end{itemize}
\begin{itemize}
\item {Proveniência:(Lat. \textunderscore absinthites\textunderscore )}
\end{itemize}
Vinho absinthado.
Vinho de losna.
\section{Absintho}
\begin{itemize}
\item {Grp. gram.:m.}
\end{itemize}
Bebida alcoólica e amarga, preparada com folhas e botões de várias espécies de losna e de outras plantas.
(Fórma falsa, tirada do fr. \textunderscore absinthe\textunderscore . A fórma portuguesa é absínthio. V. \textunderscore absínthio\textunderscore )
\section{Absíntico}
\begin{itemize}
\item {Grp. gram.:adj.}
\end{itemize}
Diz-se do ácido que se descobriu no absintho.
\section{Absintina}
\begin{itemize}
\item {Grp. gram.:f.}
\end{itemize}
Principio amargo do absintho.
\section{Absíntio}
\begin{itemize}
\item {Grp. gram.:m.}
\end{itemize}
\begin{itemize}
\item {Proveniência:(Lat. \textunderscore absinthium\textunderscore )}
\end{itemize}
Planta vivaz, de sabor amargo e aromático.
Losna.
\section{Absintismo}
\begin{itemize}
\item {Grp. gram.:m.}
\end{itemize}
Doença, causada pelo abuso do absintho.
\section{Absintite}
\begin{itemize}
\item {Grp. gram.:m.}
\end{itemize}
\begin{itemize}
\item {Proveniência:(Lat. \textunderscore absinthites\textunderscore )}
\end{itemize}
Vinho absinthado.
Vinho de losna.
\section{Absinto}
\begin{itemize}
\item {Grp. gram.:m.}
\end{itemize}
Bebida alcoólica e amarga, preparada com folhas e botões de várias espécies de losna e de outras plantas.
(Fórma falsa, tirada do fr. \textunderscore absinthe\textunderscore . A fórma portuguesa é absínthio. V. \textunderscore absínthio\textunderscore )
\section{Absogra}
\begin{itemize}
\item {Grp. gram.:fem.}
\end{itemize}
De absogro.
\section{Absogro}
\begin{itemize}
\item {fónica:sô}
\end{itemize}
\begin{itemize}
\item {Grp. gram.:m.}
\end{itemize}
\begin{itemize}
\item {Proveniência:(Do lat. \textunderscore absocer\textunderscore )}
\end{itemize}
Bisavô do marido ou da mulher em relação ao outro cônjuge.
\section{Absoleto}
\begin{itemize}
\item {Grp. gram.:adj.}
\end{itemize}
\begin{itemize}
\item {Utilização:Ant.}
\end{itemize}
O mesmo que \textunderscore obsoleto\textunderscore . Cf. Garrett, \textunderscore Romanceiro\textunderscore , vol. II.
\section{Absolto}
\begin{itemize}
\item {Grp. gram.:adj.}
\end{itemize}
\begin{itemize}
\item {Proveniência:(Lat. \textunderscore absolutus\textunderscore )}
\end{itemize}
O mesmo que [[absolvido|absolver]]. Cf. \textunderscore Peregrinação\textunderscore , CIII; Camillo, \textunderscore Caveira\textunderscore , 22.
\section{Absolução}
\begin{itemize}
\item {Grp. gram.:f.}
\end{itemize}
\begin{itemize}
\item {Utilização:Ant.}
\end{itemize}
\begin{itemize}
\item {Proveniência:(Lat. \textunderscore absolutio\textunderscore )}
\end{itemize}
O mesmo que \textunderscore absolvição\textunderscore .
\section{Absolutamente}
\begin{itemize}
\item {Grp. gram.:adv.}
\end{itemize}
De modo \textunderscore absoluto\textunderscore .
\section{Absolutismo}
\begin{itemize}
\item {Grp. gram.:m.}
\end{itemize}
Systema de govêrno, em que o poder dos governantes é absoluto.
\section{Absolutista}
\begin{itemize}
\item {Grp. gram.:adj.}
\end{itemize}
\begin{itemize}
\item {Grp. gram.:M.}
\end{itemize}
Próprio do absolutismo.
O sectário do absolutismo.
\section{Absoluto}
\begin{itemize}
\item {Grp. gram.:adj.}
\end{itemize}
\begin{itemize}
\item {Proveniência:(Lat. \textunderscore absolutus\textunderscore )}
\end{itemize}
Independente, sem restricções.
Inteiro.
Incondicional.
Illimitado.
Imperioso.
Supremo.
Único.
Abstracto.
Cabal.
Irrecusável.
Autoritário.
Estreme.
Absolvido.
\section{Absolutório}
\begin{itemize}
\item {Grp. gram.:adj.}
\end{itemize}
\begin{itemize}
\item {Proveniência:(De \textunderscore absoluto\textunderscore )}
\end{itemize}
Relativo á absolvição.
\section{Absolver}
\begin{itemize}
\item {Grp. gram.:v. t.}
\end{itemize}
\begin{itemize}
\item {Utilização:Fig.}
\end{itemize}
\begin{itemize}
\item {Proveniência:(Lat. \textunderscore absolvere\textunderscore )}
\end{itemize}
Isentar do castigo correspondente a uma culpa.
Perdoar peccados a.
Desquitar; exonerar.
Resolver, decidir.
\section{Absolvição}
\begin{itemize}
\item {Grp. gram.:f.}
\end{itemize}
\begin{itemize}
\item {Proveniência:(Lat. \textunderscore absolutio\textunderscore )}
\end{itemize}
Acto ou effeito de absolver.
\section{Absolvimento}
\begin{itemize}
\item {Grp. gram.:m.}
\end{itemize}
O mesmo que \textunderscore absolvição\textunderscore .
\section{Ábsono}
\begin{itemize}
\item {Grp. gram.:adj.}
\end{itemize}
\begin{itemize}
\item {Proveniência:(Lat. \textunderscore absonus\textunderscore )}
\end{itemize}
Discordante; destoante.
\section{Absorpção}
\begin{itemize}
\item {Grp. gram.:f.}
\end{itemize}
\begin{itemize}
\item {Proveniência:(Lat. \textunderscore absorptio\textunderscore )}
\end{itemize}
Acto de absorver.
\section{Absorpciometria}
\begin{itemize}
\item {Grp. gram.:f.}
\end{itemize}
\begin{itemize}
\item {Proveniência:(Do lat. \textunderscore absorptio\textunderscore  + gr. \textunderscore metron\textunderscore )}
\end{itemize}
Determinação dos coefficientes de absorpção entre os líquidos e os gases.
\section{Absorpciométrico}
\begin{itemize}
\item {Grp. gram.:adj.}
\end{itemize}
Relativo á \textunderscore absorpciometria\textunderscore .
\section{Absorto}
\begin{itemize}
\item {Grp. gram.:adj.}
\end{itemize}
\begin{itemize}
\item {Proveniência:(Lat. \textunderscore absortus\textunderscore )}
\end{itemize}
Concentrado em seus pensamentos.
Extasiado.
\section{Absorvedor}
\begin{itemize}
\item {Grp. gram.:m.  e  adj.}
\end{itemize}
O mesmo que \textunderscore absorvente\textunderscore .
\section{Absorvedoiro}
\begin{itemize}
\item {Grp. gram.:m.}
\end{itemize}
O mesmo que \textunderscore sorvedoiro\textunderscore .
\section{Absorvedouro}
\begin{itemize}
\item {Grp. gram.:m.}
\end{itemize}
O mesmo que \textunderscore sorvedoiro\textunderscore .
\section{Absorvência}
\begin{itemize}
\item {Grp. gram.:f.}
\end{itemize}
Faculdade de absorver.
\section{Absorvente}
\begin{itemize}
\item {Grp. gram.:m.  e  adj.}
\end{itemize}
\begin{itemize}
\item {Proveniência:(Lat. \textunderscore absorbens\textunderscore )}
\end{itemize}
O que absorve.
\section{Absorver}
\begin{itemize}
\item {Grp. gram.:v. t.}
\end{itemize}
\begin{itemize}
\item {Grp. gram.:V. p.}
\end{itemize}
\begin{itemize}
\item {Proveniência:(Lat. \textunderscore absorbere\textunderscore )}
\end{itemize}
Recolher em si: \textunderscore a camisola absorve o suor\textunderscore .
Sorver.
Aspirar.
Engulir.
Enxugar.
Consumir: \textunderscore os cuidados absorvem o tempo\textunderscore .
Enthusiasmar.
Preocupar: \textunderscore absorve-o o vício do jôgo\textunderscore .
Extasiar-se.
Concentrar-se: \textunderscore absorver-se no estudo\textunderscore .
\section{Absorvimento}
\begin{itemize}
\item {Grp. gram.:m.}
\end{itemize}
O mesmo que \textunderscore absorpção\textunderscore .
\section{Absôrvo}
\begin{itemize}
\item {Grp. gram.:m.}
\end{itemize}
Parte das bombas de incêndio, que recolhe a água para a mangueira.
\section{Abstemia}
\begin{itemize}
\item {Grp. gram.:f.}
\end{itemize}
Qualidade de abstêmio.
\section{Abstêmico}
\begin{itemize}
\item {Grp. gram.:adj.}
\end{itemize}
O mesmo que \textunderscore abstêmio\textunderscore .
\section{Abstêmio}
\begin{itemize}
\item {Grp. gram.:adj.}
\end{itemize}
\begin{itemize}
\item {Grp. gram.:M.}
\end{itemize}
\begin{itemize}
\item {Proveniência:(Lat. \textunderscore abstemius\textunderscore )}
\end{itemize}
Que se abstém de vinho; frugal, moderado.
Aquelle que se abstém de vinho.
\section{Abstenção}
\begin{itemize}
\item {Grp. gram.:f.}
\end{itemize}
\begin{itemize}
\item {Proveniência:(Lat. \textunderscore abstentio\textunderscore )}
\end{itemize}
Acto ou effeito de abster.
\section{Abstencionista}
\begin{itemize}
\item {Grp. gram.:m.}
\end{itemize}
Aquelle que se abstém (de votar em eleições); partidário do principio da abstenção. Cf. A. Cândido, \textunderscore Phil. Pol.\textunderscore , 118.
\section{Abster}
\begin{itemize}
\item {Grp. gram.:v. t.}
\end{itemize}
\begin{itemize}
\item {Grp. gram.:V. p.}
\end{itemize}
\begin{itemize}
\item {Grp. gram.:V. i.}
\end{itemize}
\begin{itemize}
\item {Proveniência:(Do lat. \textunderscore abstinere\textunderscore )}
\end{itemize}
Privar de; desviar.
Conter-se, reprimir-se: \textunderscore abster-se de falar\textunderscore .
Ter moderação, sobriedade.
Não tomar parte numa deliberação.
Privar-se; não intervir. Cf. Rui Barbosa, \textunderscore Réplica\textunderscore , II, 160.
\section{Abstergência}
\begin{itemize}
\item {Grp. gram.:f.}
\end{itemize}
Qualidade de abstergente.
\section{Abstergente}
\begin{itemize}
\item {Grp. gram.:adj.}
\end{itemize}
\begin{itemize}
\item {Proveniência:(Lat. \textunderscore abstergens\textunderscore )}
\end{itemize}
Que absterge.
\section{Absterger}
\begin{itemize}
\item {Grp. gram.:v. t.}
\end{itemize}
\begin{itemize}
\item {Proveniência:(Lat. \textunderscore abstergere\textunderscore )}
\end{itemize}
Limpar, lavar.
\section{Abstero}
\begin{itemize}
\item {Grp. gram.:adj.}
\end{itemize}
\begin{itemize}
\item {Utilização:Ant.}
\end{itemize}
O mesmo que \textunderscore austero\textunderscore .
\section{Abstersão}
\begin{itemize}
\item {Grp. gram.:f.}
\end{itemize}
\begin{itemize}
\item {Proveniência:(Lat. \textunderscore abstersio\textunderscore )}
\end{itemize}
Acto ou effeito de absterger.
\section{Abstersivo}
\begin{itemize}
\item {Grp. gram.:adj.}
\end{itemize}
\begin{itemize}
\item {Grp. gram.:M.}
\end{itemize}
\begin{itemize}
\item {Proveniência:(De \textunderscore absterso\textunderscore )}
\end{itemize}
Próprio para absterger.
Substância que absterge.
\section{Absterso}
\begin{itemize}
\item {Grp. gram.:adj.}
\end{itemize}
\begin{itemize}
\item {Proveniência:(Lat. \textunderscore abstersus\textunderscore )}
\end{itemize}
Lavado, limpo.
\section{Abstinência}
\begin{itemize}
\item {Grp. gram.:f.}
\end{itemize}
\begin{itemize}
\item {Proveniência:(Lat. \textunderscore abstinentia\textunderscore )}
\end{itemize}
Qualidade de abstinente; dieta.
Jejum.
\section{Abstinente}
\begin{itemize}
\item {Grp. gram.:adj.}
\end{itemize}
\begin{itemize}
\item {Proveniência:(Lat. \textunderscore abstinens\textunderscore )}
\end{itemize}
Que se abstém.
\section{Abstracção}
\begin{itemize}
\item {Grp. gram.:f.}
\end{itemize}
\begin{itemize}
\item {Proveniência:(Lat. \textunderscore abstractio\textunderscore )}
\end{itemize}
Acto de abstrahir.
Estado, em que o espírito considera insuladamente coisas realmente unidas.
\section{Abstractamente}
\begin{itemize}
\item {Grp. gram.:adv.}
\end{itemize}
De modo \textunderscore abstracto\textunderscore .
\section{Abstractivo}
\begin{itemize}
\item {Grp. gram.:adj.}
\end{itemize}
\begin{itemize}
\item {Proveniência:(De \textunderscore abstracto\textunderscore )}
\end{itemize}
Que abstrai.
\section{Abstracto}
\begin{itemize}
\item {Grp. gram.:adj.}
\end{itemize}
\begin{itemize}
\item {Grp. gram.:M.}
\end{itemize}
\begin{itemize}
\item {Proveniência:(Lat. \textunderscore abstractus\textunderscore )}
\end{itemize}
Que designa propriedade ou qualidade, insulada da pessôa ou coisa a que pertence, como \textunderscore belleza\textunderscore , \textunderscore alegria\textunderscore , etc.
Que opera sôbre abstracções e não sôbre realidades.
\textunderscore Numero abstracto\textunderscore , aquelle de que se não indica a natureza da unidade.
Aquillo que é abstracto.
\section{Abstrahimento}
\begin{itemize}
\item {fónica:tra-i}
\end{itemize}
\begin{itemize}
\item {Grp. gram.:m.}
\end{itemize}
O mesmo que \textunderscore abstracção\textunderscore . Cf. Camillo, \textunderscore Olho de Vidro\textunderscore , 75.
\section{Abstrahir}
\begin{itemize}
\item {Grp. gram.:v. t.}
\end{itemize}
\begin{itemize}
\item {Grp. gram.:V. i.}
\end{itemize}
\begin{itemize}
\item {Grp. gram.:V. p.}
\end{itemize}
\begin{itemize}
\item {Proveniência:(Lat. \textunderscore abstrahere\textunderscore )}
\end{itemize}
Separar.
Considerar separadamente.
Concentrar-se.
Alhear-se.
\section{Abstraimento}
\begin{itemize}
\item {fónica:tra-i}
\end{itemize}
\begin{itemize}
\item {Grp. gram.:m.}
\end{itemize}
O mesmo que \textunderscore abstracção\textunderscore . Cf. Camillo, \textunderscore Olho de Vidro\textunderscore , 75.
\section{Abstrair}
\begin{itemize}
\item {Grp. gram.:v. t.}
\end{itemize}
\begin{itemize}
\item {Grp. gram.:V. i.}
\end{itemize}
\begin{itemize}
\item {Grp. gram.:V. p.}
\end{itemize}
\begin{itemize}
\item {Proveniência:(Lat. \textunderscore abstrahere\textunderscore )}
\end{itemize}
Separar.
Considerar separadamente.
Concentrar-se.
Alhear-se.
\section{Abstrusamente}
\begin{itemize}
\item {Grp. gram.:adv.}
\end{itemize}
De modo \textunderscore abstruso\textunderscore .
Obscuramente.
\section{Abstrusidade}
\begin{itemize}
\item {Grp. gram.:f.}
\end{itemize}
Qualidade de abstruso.
\section{Abstruso}
\begin{itemize}
\item {Grp. gram.:adj.}
\end{itemize}
\begin{itemize}
\item {Proveniência:(Lat. \textunderscore abstrusus\textunderscore )}
\end{itemize}
Desordenado, confuso.
Que não tem méthodo.
\section{Absumir}
\begin{itemize}
\item {Grp. gram.:v. t.}
\end{itemize}
\begin{itemize}
\item {Utilização:Des.}
\end{itemize}
\begin{itemize}
\item {Proveniência:(Lat. \textunderscore absumere\textunderscore )}
\end{itemize}
O mesmo que \textunderscore consumir\textunderscore .
\section{Absumpção}
\begin{itemize}
\item {Grp. gram.:f.}
\end{itemize}
\begin{itemize}
\item {Proveniência:(Lat. \textunderscore absumptio\textunderscore )}
\end{itemize}
O mesmo que \textunderscore consumpção\textunderscore .
\section{Absurdamente}
\begin{itemize}
\item {Grp. gram.:adv.}
\end{itemize}
De modo \textunderscore absurdo\textunderscore .
\section{Absurdeza}
\begin{itemize}
\item {Grp. gram.:f.}
\end{itemize}
Qualidade daquillo que é absurdo.
O mesmo que \textunderscore absurdo\textunderscore . Cf. Camillo, \textunderscore Esqueleto\textunderscore , 51.
\section{Absurdo}
\begin{itemize}
\item {Grp. gram.:adj.}
\end{itemize}
\begin{itemize}
\item {Grp. gram.:M.}
\end{itemize}
\begin{itemize}
\item {Proveniência:(Lat. \textunderscore absurdus\textunderscore )}
\end{itemize}
Disparatado, opposto á razão ou ao senso commum.
Logicamente contradictório.
Tolice, disparate, contra-senso.
\section{Abu}
\begin{itemize}
\item {Grp. gram.:m.}
\end{itemize}
Espécie de palmeira.
\section{Abuçar}
\begin{itemize}
\item {Grp. gram.:v. t.}
\end{itemize}
\begin{itemize}
\item {Utilização:Gír.}
\end{itemize}
O mesmo que \textunderscore cercar\textunderscore .
\section{Abugão}
\begin{itemize}
\item {Grp. gram.:m.}
\end{itemize}
\begin{itemize}
\item {Utilização:Prov.}
\end{itemize}
\begin{itemize}
\item {Utilização:minh.}
\end{itemize}
O mesmo que \textunderscore abigoiro\textunderscore .
\section{Abujão}
\begin{itemize}
\item {Grp. gram.:m.}
\end{itemize}
\begin{itemize}
\item {Proveniência:(Lat. \textunderscore abusio\textunderscore )}
\end{itemize}
O mesmo ou melhor que \textunderscore avejão\textunderscore .
\section{Abular}
\begin{itemize}
\item {Grp. gram.:v. t.}
\end{itemize}
\begin{itemize}
\item {Grp. gram.:V. p.}
\end{itemize}
Sellar com bulla ou com sêllo de chumbo.
Adquirir bulla.
\section{Abulia}
\begin{itemize}
\item {Grp. gram.:f.}
\end{itemize}
\begin{itemize}
\item {Proveniência:(Do gr. \textunderscore a\textunderscore  priv. + \textunderscore boule\textunderscore , vontade)}
\end{itemize}
Doença, caracterizada pelo afroixamento da volição.
\section{Abullar}
\begin{itemize}
\item {Grp. gram.:v. t.}
\end{itemize}
\begin{itemize}
\item {Grp. gram.:V. p.}
\end{itemize}
Sellar com bulla ou com sêllo de chumbo.
Adquirir bulla.
\section{Abuna}
\begin{itemize}
\item {Grp. gram.:m.}
\end{itemize}
\begin{itemize}
\item {Utilização:Bras}
\end{itemize}
Nome, que os Índios da América davam aos jesuitas e aos padres em geral.
\section{Abundamento}
\begin{itemize}
\item {Grp. gram.:m.}
\end{itemize}
\begin{itemize}
\item {Utilização:Ant.}
\end{itemize}
O mesmo que abundância.
\section{Abundância}
\begin{itemize}
\item {Grp. gram.:f.}
\end{itemize}
\begin{itemize}
\item {Proveniência:(Lat. \textunderscore abundantia\textunderscore )}
\end{itemize}
Qualidade de abundante.
Grande quantidade.
Riqueza: \textunderscore viver na abundância\textunderscore .
\section{Abundante}
\begin{itemize}
\item {Grp. gram.:adj.}
\end{itemize}
\begin{itemize}
\item {Utilização:Fig.}
\end{itemize}
Que abunda, que tem grande cópia.
Rico: \textunderscore país abundante\textunderscore .
Opulento em termos e phrases: \textunderscore estilo abundante\textunderscore .
\section{Abundantemente}
\begin{itemize}
\item {Grp. gram.:adv.}
\end{itemize}
De modo \textunderscore abundante\textunderscore , com abundância.
\section{Abundar}
\begin{itemize}
\item {Grp. gram.:v. i.}
\end{itemize}
\begin{itemize}
\item {Utilização:Fam.}
\end{itemize}
\begin{itemize}
\item {Proveniência:(Lat. \textunderscore abundare\textunderscore )}
\end{itemize}
Têr ou existir, em grande quantidade: \textunderscore a miséria abunda nas grandes cidades\textunderscore .
Trasbordar.
Estar cheio, rico.
Affluir.
\textunderscore Abundar na opinião de alguém\textunderscore , sêr da mesma opinião.
\section{Abundosamente}
\begin{itemize}
\item {Grp. gram.:adv.}
\end{itemize}
O mesmo que \textunderscore abundantemente\textunderscore .
\section{Abundoso}
\begin{itemize}
\item {Grp. gram.:adj.}
\end{itemize}
\begin{itemize}
\item {Utilização:Ant.}
\end{itemize}
O mesmo que \textunderscore abundante\textunderscore .
Que tem poderes bastantes para tratar negocios de outrem. Cf. \textunderscore Peregrinação\textunderscore , II, 117.
\section{Abunhadio}
\begin{itemize}
\item {Grp. gram.:m.}
\end{itemize}
Cargo de abunhado.
\section{Abunhado}
\begin{itemize}
\item {Grp. gram.:m.}
\end{itemize}
Trabalhador indiano, que, nascido em terra de um senhorio, é obrigado a viver e trabalhar nellas.
\section{Abunhar}
\begin{itemize}
\item {Grp. gram.:v. i.}
\end{itemize}
Viver como abunhado.
\section{Abur!}
\begin{itemize}
\item {Grp. gram.:interj.}
\end{itemize}
\begin{itemize}
\item {Utilização:Prov.}
\end{itemize}
\begin{itemize}
\item {Utilização:trasm.}
\end{itemize}
Usa-se maliciosamente, para designar despedida, que não deixa saudades.
(Relaciona-se com \textunderscore embora\textunderscore ?)
\section{Aburacar}
\begin{itemize}
\item {Grp. gram.:v. t.}
\end{itemize}
(V.esburacar)
\section{Aburelar}
\begin{itemize}
\item {Grp. gram.:v. t.}
\end{itemize}
Dar aspecto de burel a.
\section{Aburguesado}
\begin{itemize}
\item {Grp. gram.:adj.}
\end{itemize}
\begin{itemize}
\item {Proveniência:(De \textunderscore aburguesar\textunderscore )}
\end{itemize}
Próprio de burguês.
Semelhante a burguês.
\section{Aburguesar-se}
\begin{itemize}
\item {Grp. gram.:v. p.}
\end{itemize}
Adquirir hábitos ou modos de burguês.
\section{Aburrar}
\begin{itemize}
\item {Grp. gram.:v. t.}
\end{itemize}
\begin{itemize}
\item {Utilização:Des.}
\end{itemize}
\begin{itemize}
\item {Proveniência:(De \textunderscore burro\textunderscore )}
\end{itemize}
Mostrar-se triste.
\section{Aburrinhar}
\begin{itemize}
\item {Grp. gram.:v. i.}
\end{itemize}
\begin{itemize}
\item {Utilização:Prov.}
\end{itemize}
\begin{itemize}
\item {Proveniência:(De \textunderscore burrinho\textunderscore , dem. de \textunderscore burro\textunderscore , por allusão á maneira de andar em quatro pés)}
\end{itemize}
Diz-se da criança, que começa a andar de gatas, a engatinhar: \textunderscore ainda eu aburrinhava...\textunderscore 
\section{Abusão}
\begin{itemize}
\item {Grp. gram.:f.}
\end{itemize}
\begin{itemize}
\item {Proveniência:(Lat. \textunderscore abusio\textunderscore )}
\end{itemize}
Abuso.
Engano.
Superstição.
\section{Abusar}
\begin{itemize}
\item {Grp. gram.:v. i.}
\end{itemize}
\begin{itemize}
\item {Proveniência:(De \textunderscore abuso\textunderscore )}
\end{itemize}
Fazer mau uso.
Faltar á confiança.
Despropositar.
Causar prejuízo.
\section{Abusivamente}
\begin{itemize}
\item {Grp. gram.:adv.}
\end{itemize}
De modo \textunderscore abusivo\textunderscore .
\section{Abusivo}
\begin{itemize}
\item {Grp. gram.:adj.}
\end{itemize}
\begin{itemize}
\item {Proveniência:(Lat. \textunderscore abusivus\textunderscore )}
\end{itemize}
Feito com abuso.
\section{Abuso}
\begin{itemize}
\item {Grp. gram.:m.}
\end{itemize}
Uso excessivo, errado ou injusto: \textunderscore abuso do poder\textunderscore .
Êrro.
Prevaricação.
Práticas condemnáveis: \textunderscore os abusos do absolutismo\textunderscore .
\section{Abuta}
\begin{itemize}
\item {Grp. gram.:f.}
\end{itemize}
\begin{itemize}
\item {Utilização:Ant.}
\end{itemize}
Boceta.
Caixinha; caixa de rapé.
\section{Abutilão}
\begin{itemize}
\item {Grp. gram.:m.}
\end{itemize}
Planta malvácea, ornamental.
\section{Abutilo}
\begin{itemize}
\item {Grp. gram.:m.}
\end{itemize}
O mesmo que \textunderscore abutilão\textunderscore .
\section{Abutiloide}
\begin{itemize}
\item {Grp. gram.:f.}
\end{itemize}
\begin{itemize}
\item {Proveniência:(De \textunderscore abutilo\textunderscore )}
\end{itemize}
Planta exótica, da fam. das malváceas.
\section{Abutre}
\begin{itemize}
\item {Grp. gram.:m.}
\end{itemize}
\begin{itemize}
\item {Utilização:Fig.}
\end{itemize}
\begin{itemize}
\item {Proveniência:(Lat. \textunderscore vultur\textunderscore )}
\end{itemize}
Ave de rapina, da ordem das diurnas.
Homem usurário.
Homem sem escrúpulos.
\section{Abutreiro}
\begin{itemize}
\item {Grp. gram.:m.}
\end{itemize}
Caçador de abutres.
\section{Abútua}
\begin{itemize}
\item {Grp. gram.:f.}
\end{itemize}
O mesmo que \textunderscore bútua\textunderscore .
\section{Abutumado}
\begin{itemize}
\item {Grp. gram.:adj.}
\end{itemize}
O mesmo que \textunderscore abetumado\textunderscore , macambúzio. Cf. \textunderscore Eufrosina\textunderscore , 15.
\section{Abuzinado}
\begin{itemize}
\item {Grp. gram.:adj.}
\end{itemize}
\begin{itemize}
\item {Proveniência:(De \textunderscore abuzinar\textunderscore )}
\end{itemize}
Que tem fórma de buzina.
\section{Abuzinar}
\begin{itemize}
\item {Grp. gram.:v. i.}
\end{itemize}
Tocar buzina.
Fazer barulho incómmodo.
\section{Abysmo}
\textunderscore m.\textunderscore  (e der.)
(V. \textunderscore abismo\textunderscore , etc.)
\section{Abyssal}
\begin{itemize}
\item {Grp. gram.:adj}
\end{itemize}
\begin{itemize}
\item {Proveniência:(De \textunderscore abysso\textunderscore )}
\end{itemize}
Relativo ao abysso.
Relativo ás profundidades marítimas.
Que vive na profundidade do mar.
\section{Abyssínio}
\begin{itemize}
\item {Grp. gram.:m.}
\end{itemize}
O mesmo que \textunderscore abexim\textunderscore .
\section{Abysso}
\begin{itemize}
\item {Grp. gram.:m.}
\end{itemize}
\begin{itemize}
\item {Proveniência:(Gr. \textunderscore abussos\textunderscore )}
\end{itemize}
O mesmo que \textunderscore abismo\textunderscore .
Gênero de plantas de jardim.
\section{A. C.}
\begin{itemize}
\item {Grp. gram.:abrev.}
\end{itemize}
(que acompanha certas datas, significando: \textunderscore antes de Christo\textunderscore )
\section{...aça}
\begin{itemize}
\item {Grp. gram.:suf., fem.}
\end{itemize}
De... aço.
\section{Aca}
\begin{itemize}
\item {Grp. gram.:m.}
\end{itemize}
\begin{itemize}
\item {Utilização:Bras. de Minas}
\end{itemize}
Mau cheiro; fétido.
\section{Aça}
\begin{itemize}
\item {Grp. gram.:m., f.  e  adj.}
\end{itemize}
\begin{itemize}
\item {Utilização:Bras}
\end{itemize}
Homem ou animal albino.
\section{Acá}
\begin{itemize}
\item {Grp. gram.:m.}
\end{itemize}
Árvore silvestre do Brasil.
\section{Acabaçar}
\begin{itemize}
\item {Grp. gram.:v. t.}
\end{itemize}
Dar feitio de cabaça a.
\section{Acabadamente}
\begin{itemize}
\item {Grp. gram.:adv.}
\end{itemize}
De modo \textunderscore acabado\textunderscore , com perfeição.
Inteiramente.
\section{Acabadiço}
\begin{itemize}
\item {Grp. gram.:adj.}
\end{itemize}
Doentio, acanaveado. Cf. Cenáculo, \textunderscore Visita\textunderscore , 37.
\section{Acabado}
\begin{itemize}
\item {Grp. gram.:adj.}
\end{itemize}
\begin{itemize}
\item {Utilização:Fam.}
\end{itemize}
\begin{itemize}
\item {Grp. gram.:M.}
\end{itemize}
\begin{itemize}
\item {Proveniência:(De \textunderscore acabar\textunderscore )}
\end{itemize}
Abatido.
Muito magro, avelhentado.
O mesmo que \textunderscore acabamento\textunderscore : \textunderscore é trabalho de óptimo acabado\textunderscore .
\section{Acabador}
\begin{itemize}
\item {Grp. gram.:m.}
\end{itemize}
\begin{itemize}
\item {Grp. gram.:m.}
\end{itemize}
Aquelle que acaba.
Operário, encarregado do acabamento numa peça de tecido de lan.
\section{Acabadote}
\begin{itemize}
\item {Grp. gram.:adj.}
\end{itemize}
\begin{itemize}
\item {Proveniência:(De \textunderscore acabado\textunderscore )}
\end{itemize}
Um tanto acabado, avelhentado, (falando-se de alguém).
\section{Acabamento}
\begin{itemize}
\item {Grp. gram.:m.}
\end{itemize}
Acto ou effeito de \textunderscore acabar\textunderscore .
Perfeição.
\section{Acabanado}
\begin{itemize}
\item {Grp. gram.:adj.}
\end{itemize}
\begin{itemize}
\item {Utilização:Prov.}
\end{itemize}
Que tem fórma de cabana.
Virado para dentro ou para baixo, (falando-se dos chifres dos bois, do lóbulo da orelha, da aba do chapéu, etc.)
\section{Acabar}
\begin{itemize}
\item {Grp. gram.:v. t.}
\end{itemize}
\begin{itemize}
\item {Grp. gram.:V. i.}
\end{itemize}
\begin{itemize}
\item {Grp. gram.:Loc. conj.}
\end{itemize}
\begin{itemize}
\item {Proveniência:(De \textunderscore cabo\textunderscore )}
\end{itemize}
Pôr termo a; levar a cabo, concluir.
Aperfeiçoar.
Gastar.
Findar, concluir-se.
Morrer.
\textunderscore Acabar com alguém que\textunderscore , resolvê-lo, convencê-lo de que. Cf. Pant. de Aveiro, \textunderscore Itiner.\textunderscore , 22. (2.^a ed.).
\section{Acabellar}
\textunderscore v. i.\textunderscore  (e der.)
(V. \textunderscore encabellar\textunderscore , etc.)
\section{Acaboclado}
\begin{itemize}
\item {Grp. gram.:adj.}
\end{itemize}
Que tem origem, côr ou feições de caboclo.
\section{Acabramar}
\begin{itemize}
\item {Grp. gram.:v. t.}
\end{itemize}
\begin{itemize}
\item {Utilização:Des.}
\end{itemize}
Embaraçar os movimentos de (bois), ligando com corda o pé ao corno, para que não maltratem quem delles cura.
(Cp. \textunderscore cabre\textunderscore ^2)
\section{Acabramo}
\begin{itemize}
\item {Grp. gram.:m.}
\end{itemize}
Peia de acabramar.
\section{Acabrunhadamente}
\begin{itemize}
\item {Grp. gram.:adv.}
\end{itemize}
De modo \textunderscore acabrunhado\textunderscore .
\section{Acabrunhado}
\begin{itemize}
\item {Grp. gram.:adj.}
\end{itemize}
\begin{itemize}
\item {Proveniência:(De \textunderscore acabrunhar\textunderscore )}
\end{itemize}
Enfraquecido, adoentado.
Melancólico.
\section{Acabrunhar}
\begin{itemize}
\item {Grp. gram.:v. t.}
\end{itemize}
\begin{itemize}
\item {Grp. gram.:V. p.}
\end{itemize}
\begin{itemize}
\item {Utilização:T. de Murça}
\end{itemize}
Affligir, opprimir; humilhar.
Cair doente; ficar doente de cama.
\section{Acacá}
\begin{itemize}
\item {Grp. gram.:m.}
\end{itemize}
\begin{itemize}
\item {Utilização:Bras}
\end{itemize}
Espécie de bolo de arroz ou de milho moído.
\section{Açacal}
\begin{itemize}
\item {Grp. gram.:m.}
\end{itemize}
\begin{itemize}
\item {Utilização:Ant.}
\end{itemize}
O mesmo que \textunderscore aguadeiro\textunderscore .
(Do ár.)
\section{Açacaladamente}
\begin{itemize}
\item {Grp. gram.:adv.}
\end{itemize}
De modo \textunderscore açacalado\textunderscore .
\section{Açacalador}
\begin{itemize}
\item {Grp. gram.:m.}
\end{itemize}
O que açacala.
Brunidor de armas brancas.
\section{Açacaladura}
\begin{itemize}
\item {Grp. gram.:f.}
\end{itemize}
Acto de ou effeito de \textunderscore açacalar\textunderscore .
\section{Açacalar}
\begin{itemize}
\item {Grp. gram.:v. t.}
\end{itemize}
\begin{itemize}
\item {Utilização:Fig.}
\end{itemize}
\begin{itemize}
\item {Proveniência:(De \textunderscore açacal\textunderscore )}
\end{itemize}
Polir; (falando-se de armas brancas)
Illustrar.
Aperfeiçoar.
\section{Acaçapadamente}
\begin{itemize}
\item {Grp. gram.:adv.}
\end{itemize}
De modo \textunderscore acaçapado\textunderscore .
\section{Acaçapado}
\begin{itemize}
\item {Grp. gram.:adj.}
\end{itemize}
\begin{itemize}
\item {Proveniência:(De \textunderscore acaçapar\textunderscore )}
\end{itemize}
Que se abaixou, que se encolheu: \textunderscore um vulto acaçapado\textunderscore .
\section{Acaçapar}
\begin{itemize}
\item {Grp. gram.:v. t.}
\end{itemize}
\begin{itemize}
\item {Proveniência:(De \textunderscore caçapo\textunderscore )}
\end{itemize}
Abaixar, encolher.
\section{Acachafundar}
\begin{itemize}
\item {Grp. gram.:v. t.}
\end{itemize}
\begin{itemize}
\item {Utilização:Pop.}
\end{itemize}
Meter na água, de cabeça para baixo.
\section{Acachapar}
\begin{itemize}
\item {Grp. gram.:v. t.}
\end{itemize}
O mesmo que \textunderscore acaçapar\textunderscore :«\textunderscore eis se acachapa meio morto o pastor\textunderscore ». Filinto, XII, 209.
\section{Acachar}
\begin{itemize}
\item {Grp. gram.:v. t.}
\end{itemize}
(V.agachar)
\section{Acachoar}
\begin{itemize}
\item {Grp. gram.:v. t.}
\end{itemize}
\begin{itemize}
\item {Grp. gram.:V. i.}
\end{itemize}
Pôr em cachão.
Formar cachão.
\section{Acachuchar}
\begin{itemize}
\item {Grp. gram.:v. t.}
\end{itemize}
\begin{itemize}
\item {Utilização:Prov.}
\end{itemize}
\begin{itemize}
\item {Utilização:trasm.}
\end{itemize}
Matar (alguém), amarfanhando-o, sem o deixar piar.
\section{Acácia}
\begin{itemize}
\item {Grp. gram.:f.}
\end{itemize}
\begin{itemize}
\item {Proveniência:(Lat. \textunderscore acacia\textunderscore )}
\end{itemize}
Árvore ornamental, da fam. das leguminosas.
\section{Acacifar}
\begin{itemize}
\item {Grp. gram.:v. t.}
\end{itemize}
Meter em cacifo.
\section{Açacu}
\begin{itemize}
\item {Grp. gram.:m.}
\end{itemize}
Árvore euphorbiácea das margens do Amazonas, (\textunderscore ura brasiliensis\textunderscore ).
\section{Acadeirar-se}
\begin{itemize}
\item {Grp. gram.:v. p.}
\end{itemize}
Sentar-se em cadeira.
\section{Academia}
\begin{itemize}
\item {Grp. gram.:f.}
\end{itemize}
\begin{itemize}
\item {Proveniência:(Lat. \textunderscore academia\textunderscore )}
\end{itemize}
Lugar em que se ensina.
Escola de instrucção superior.
Sociedade de sábios, artistas ou literatos.
(A pronúncia exacta seria \textunderscore académia\textunderscore , mas não se usa)
\section{Acadêmia}
\begin{itemize}
\item {Grp. gram.:f.}
\end{itemize}
Figura de gesso, ou estampa, para estudo.
(Cp. \textunderscore academia\textunderscore )
\section{Academial}
\begin{itemize}
\item {Grp. gram.:adj.}
\end{itemize}
(V.acadêmico)
\section{Academialmente}
\begin{itemize}
\item {Grp. gram.:adj.}
\end{itemize}
(V.academicamente)
\section{Academiar}
\begin{itemize}
\item {Grp. gram.:v. i.}
\end{itemize}
\begin{itemize}
\item {Proveniência:(De \textunderscore academia\textunderscore )}
\end{itemize}
Falar ou proceder academicamente.
\section{Academicamente}
\begin{itemize}
\item {Grp. gram.:adv.}
\end{itemize}
De modo \textunderscore acadêmico\textunderscore .
\section{Acadêmico}
\begin{itemize}
\item {Grp. gram.:adj.}
\end{itemize}
\begin{itemize}
\item {Grp. gram.:M.}
\end{itemize}
\begin{itemize}
\item {Proveniência:(Lat. \textunderscore academicus\textunderscore )}
\end{itemize}
Relativo a \textunderscore academia\textunderscore .
Membro de academia.
Estudante.
\section{Academista}
\begin{itemize}
\item {Grp. gram.:m.}
\end{itemize}
Estudante, que frequenta uma academia.
Especialmente, quem frequenta uma academia recreativa.
\section{Acadianos}
\begin{itemize}
\item {Grp. gram.:m. pl.}
\end{itemize}
Povos, que habitaram a região de Babylónia, antes da dominação dos Assýrios.
\section{Acadimar-se}
\begin{itemize}
\item {Grp. gram.:v. p.}
\end{itemize}
\begin{itemize}
\item {Utilização:Prov.}
\end{itemize}
\begin{itemize}
\item {Proveniência:(De \textunderscore cadimo\textunderscore )}
\end{itemize}
Habituar-se a certo trabalho; tornar-se perito nelle.
\section{Acadrimar-se}
\begin{itemize}
\item {Grp. gram.:v. p.}
\end{itemize}
O mesmo que \textunderscore acadimar-se\textunderscore .
Affeiçoar-se:«\textunderscore não se acadrimou a nenhum\textunderscore ». Camillo, \textunderscore Myst. de Fafe\textunderscore , 10.
\section{Acaecer}
\begin{itemize}
\item {Grp. gram.:v. i.}
\end{itemize}
\begin{itemize}
\item {Utilização:Ant.}
\end{itemize}
Acontecêr.
(Cp. lat. \textunderscore cadere\textunderscore )
\section{Acaecente}
\begin{itemize}
\item {Grp. gram.:adj.}
\end{itemize}
\begin{itemize}
\item {Utilização:Ant.}
\end{itemize}
Que acaéce.
\section{Acaecimento}
\begin{itemize}
\item {Grp. gram.:m.}
\end{itemize}
Acto de \textunderscore acaecer\textunderscore .
\section{Açafaitar}
\begin{itemize}
\item {Grp. gram.:v. t.}
\end{itemize}
\begin{itemize}
\item {Utilização:T. da Bairrada}
\end{itemize}
Entrajar bem, revestir garridamente.
(Por \textunderscore açafatar\textunderscore , de \textunderscore açafata\textunderscore )
\section{Açafata}
\begin{itemize}
\item {Grp. gram.:f.}
\end{itemize}
Moça da raínha, (moça de açafate).
\section{Açafatar}
\begin{itemize}
\item {Grp. gram.:v. t.}
\end{itemize}
Meter ou acommodar em açafate.
\section{Açafate}
\begin{itemize}
\item {Grp. gram.:m.}
\end{itemize}
Cesto baixo, redondo ou oval, sem arco nem tampa.
(Ár. \textunderscore açafat\textunderscore )
\section{Açafate-de-oiro}
\begin{itemize}
\item {Grp. gram.:m.}
\end{itemize}
\begin{itemize}
\item {Utilização:Bras}
\end{itemize}
O mesmo que \textunderscore alysso\textunderscore .
\section{Acafelador}
\begin{itemize}
\item {Grp. gram.:m.}
\end{itemize}
O que acafela.
\section{Acafeladura}
\begin{itemize}
\item {Grp. gram.:f.}
\end{itemize}
O mesmo que \textunderscore acafelamento\textunderscore .
\section{Acafelamento}
\begin{itemize}
\item {Grp. gram.:m.}
\end{itemize}
Acto ou effeito de \textunderscore acafelar\textunderscore .
\section{Acafelar}
\begin{itemize}
\item {Grp. gram.:v. t.}
\end{itemize}
Rebocar.
Encobrir.
Tapar com asphalto.
(Cast. \textunderscore acafelar\textunderscore )
\section{Acafetar}
\begin{itemize}
\item {Grp. gram.:v. t.}
\end{itemize}
\begin{itemize}
\item {Proveniência:(De \textunderscore café\textunderscore . Cp. \textunderscore cafeteira\textunderscore )}
\end{itemize}
Dar côr de café a.
\section{Açaflôr}
\begin{itemize}
\item {Grp. gram.:m.}
\end{itemize}
O mesmo que \textunderscore açafrão\textunderscore .
\section{Açafões}
\begin{itemize}
\item {Grp. gram.:m. pl.}
\end{itemize}
O mesmo que \textunderscore çafões\textunderscore .
\section{Açafrão}
\begin{itemize}
\item {Grp. gram.:m.}
\end{itemize}
\begin{itemize}
\item {Proveniência:(Do ár. \textunderscore azzaferan\textunderscore )}
\end{itemize}
Planta bulbosa, da fam. das irídeas.
Flôr dessa planta.
\section{Açafrar}
\begin{itemize}
\item {Grp. gram.:v. i.}
\end{itemize}
\begin{itemize}
\item {Utilização:Ant.}
\end{itemize}
Tornar-se sáfaro? mostrar-se desdenhoso?
(Cp. \textunderscore safra\textunderscore ^1)
\section{Açafrôa}
\begin{itemize}
\item {Grp. gram.:f.}
\end{itemize}
Pequena planta, semelhante ao açafrão.
\section{Açafroado}
\begin{itemize}
\item {Grp. gram.:adj.}
\end{itemize}
\begin{itemize}
\item {Proveniência:(De \textunderscore açafroar\textunderscore )}
\end{itemize}
Que tem côr de açafrão.
Temperado com açafrão.
\section{Açafroador}
\begin{itemize}
\item {Grp. gram.:m.}
\end{itemize}
O que açafrôa.
\section{Açafroal}
\begin{itemize}
\item {Grp. gram.:m.}
\end{itemize}
Lugar, em que cresce o açafrão.
\section{Açafroamento}
\begin{itemize}
\item {Grp. gram.:m.}
\end{itemize}
Acto de açafroar.
\section{Açafroar}
\begin{itemize}
\item {Grp. gram.:v. t.}
\end{itemize}
Dar a côr do açafrão a.
Temperar com açafrão.
\section{Açafroeira}
\begin{itemize}
\item {Grp. gram.:f.}
\end{itemize}
O mesmo que \textunderscore açafrão\textunderscore , planta.
\section{Açafrôl}
\begin{itemize}
\item {Grp. gram.:m.}
\end{itemize}
\begin{itemize}
\item {Utilização:Prov.}
\end{itemize}
\begin{itemize}
\item {Utilização:alg.}
\end{itemize}
O mesmo que \textunderscore açafrôa\textunderscore .
\section{Açagador}
\begin{itemize}
\item {Grp. gram.:m.}
\end{itemize}
\begin{itemize}
\item {Utilização:Ant.}
\end{itemize}
Cutileiro, serralheiro, armeiro. Cf. Arn. Gama, \textunderscore Bailio\textunderscore , 38.
(Cp. \textunderscore açacalador\textunderscore )
\section{Acahi}
\begin{itemize}
\item {Grp. gram.:m.}
\end{itemize}
\begin{itemize}
\item {Utilização:Des.}
\end{itemize}
Pedra hume.
\section{Açahi}
\begin{itemize}
\item {Grp. gram.:m.}
\end{itemize}
Fruta do açahizeiro.
Calda escura e substanciosa, extrahida dessa fruta.
\section{Açahizeiro}
\begin{itemize}
\item {fónica:ça-i}
\end{itemize}
\begin{itemize}
\item {Grp. gram.:m.}
\end{itemize}
\begin{itemize}
\item {Proveniência:(De \textunderscore açahi\textunderscore )}
\end{itemize}
Espécie de palmeira do norte do Brasil, (\textunderscore euterpe oleracea\textunderscore )
\section{Acaí}
\begin{itemize}
\item {Grp. gram.:m.}
\end{itemize}
\begin{itemize}
\item {Utilização:Des.}
\end{itemize}
Pedra hume.
\section{Açaí}
\begin{itemize}
\item {Grp. gram.:m.}
\end{itemize}
Fruta do açahizeiro.
Calda escura e substanciosa, extrahida dessa fruta.
\section{Acaia}
\begin{itemize}
\item {Grp. gram.:f.}
\end{itemize}
\begin{itemize}
\item {Utilização:Bras}
\end{itemize}
Planta burserácea, medicinal.
\section{Açaimar}
\begin{itemize}
\item {Grp. gram.:v. t.}
\end{itemize}
\begin{itemize}
\item {Utilização:Fig.}
\end{itemize}
Prender com açamo.
Pôr açamo a.
Reprimir.
\section{Açaime}
\begin{itemize}
\item {Grp. gram.:m.}
\end{itemize}
(V. \textunderscore açamo\textunderscore , etc.)
\section{Açaimo}
\textunderscore m.\textunderscore  (e der.)
(V. \textunderscore açamo\textunderscore , etc.)
\section{Acaipirar-se}
\begin{itemize}
\item {Grp. gram.:v. p.}
\end{itemize}
\begin{itemize}
\item {Utilização:Bras}
\end{itemize}
Adquirir modos de caipira ou de roceiro.
Mostrar-se acanhado.
\section{Acairelador}
\begin{itemize}
\item {Grp. gram.:m.}
\end{itemize}
O que acairela.
\section{Acairelamento}
\begin{itemize}
\item {Grp. gram.:m.}
\end{itemize}
Acto de acairelar.
\section{Acairelar}
\begin{itemize}
\item {Grp. gram.:v. t.}
\end{itemize}
Pôr cairel em.
Guarnecer de cairel.
\section{Acais-a-cajo}
\begin{itemize}
\item {Grp. gram.:loc. adv.}
\end{itemize}
\begin{itemize}
\item {Utilização:Ant.}
\end{itemize}
Por um és-não-és, por um tris.
(Corr. do \textunderscore quási\textunderscore  + \textunderscore quási\textunderscore )
\section{Açaizeiro}
\begin{itemize}
\item {fónica:ça-i}
\end{itemize}
\begin{itemize}
\item {Grp. gram.:m.}
\end{itemize}
\begin{itemize}
\item {Proveniência:(De \textunderscore açahi\textunderscore )}
\end{itemize}
Espécie de palmeira do norte do Brasil, (\textunderscore euterpe oleracea\textunderscore ).
\section{Acajá}
\begin{itemize}
\item {Grp. gram.:m.}
\end{itemize}
O mesmo que \textunderscore cajá\textunderscore .
\section{Acajadar}
\begin{itemize}
\item {Grp. gram.:v. t.}
\end{itemize}
Bater com cajado.
Espancar.
\section{Acaju}
\begin{itemize}
\item {Grp. gram.:m.}
\end{itemize}
(V.caju)
\section{Acajuadiço}
\begin{itemize}
\item {Grp. gram.:adj.}
\end{itemize}
\begin{itemize}
\item {Utilização:Prov.}
\end{itemize}
\begin{itemize}
\item {Utilização:alg.}
\end{itemize}
Propenso, tendente.
Que occasiona, (falando-se de doenças)
(Por \textunderscore occasiadiço\textunderscore , de \textunderscore occasião\textunderscore ?)
\section{Acajueiro}
\begin{itemize}
\item {Grp. gram.:m.}
\end{itemize}
(V.cajueiro)
\section{Acalacas}
\begin{itemize}
\item {Grp. gram.:m.}
\end{itemize}
Grande formiga da América.
\section{Acalantho}
\begin{itemize}
\item {Grp. gram.:m.}
\end{itemize}
\begin{itemize}
\item {Proveniência:(Gr. \textunderscore akalanthos\textunderscore )}
\end{itemize}
Nôme scientífico do tentilhão.
\section{Acalanto}
\begin{itemize}
\item {Grp. gram.:m.}
\end{itemize}
\begin{itemize}
\item {Proveniência:(Gr. \textunderscore akalanthos\textunderscore )}
\end{itemize}
Nôme scientífico do tentilhão.
\section{Acalar}
\begin{itemize}
\item {Grp. gram.:v. i.}
\end{itemize}
\begin{itemize}
\item {Utilização:Prov.}
\end{itemize}
Submergir-se, (falando-se de barcos).
(Cp. \textunderscore calar\textunderscore )
\section{Acalcanhado}
\begin{itemize}
\item {Grp. gram.:adj.}
\end{itemize}
\begin{itemize}
\item {Utilização:Fig.}
\end{itemize}
\begin{itemize}
\item {Proveniência:(De \textunderscore acalcanhar\textunderscore )}
\end{itemize}
Torto, (falando-se de calçado maltratado ou com muito uso).
Abatido, decadente:«\textunderscore a freira estava completamente acalcanhada e desgraciada\textunderscore ». Camillo, \textunderscore Caveira\textunderscore , 276.
\section{Acalcanhamento}
\begin{itemize}
\item {Grp. gram.:m.}
\end{itemize}
Acto de acalcanhar.
\section{Acalcanhar}
\begin{itemize}
\item {Grp. gram.:v. t.}
\end{itemize}
\begin{itemize}
\item {Grp. gram.:V. i.}
\end{itemize}
Pisar com o calcanhar.
Entortar o calçado, andando.
Esmagar, aniquilar:«\textunderscore quando as injustiças dêste planeta o acalcanharem...\textunderscore »Camillo, \textunderscore Volcoens\textunderscore , 82.
\section{Acalcar}
\begin{itemize}
\item {Grp. gram.:v. t.}
\end{itemize}
O mesmo que \textunderscore calcar\textunderscore .
\section{Acalefologia}
\begin{itemize}
\item {Grp. gram.:f.}
\end{itemize}
Parte da Zoologia, que trata dos acalephos.
\section{Acalefos}
\begin{itemize}
\item {Grp. gram.:m. pl.}
\end{itemize}
\begin{itemize}
\item {Utilização:Zool.}
\end{itemize}
\begin{itemize}
\item {Proveniência:(Do gr. \textunderscore akalephe\textunderscore )}
\end{itemize}
Classe de zoóphytos, a que pertence a alforreca.
Designação antiga dos celenterados.
\section{Acalentador}
\begin{itemize}
\item {Grp. gram.:adj.}
\end{itemize}
\begin{itemize}
\item {Proveniência:(De \textunderscore acalentar\textunderscore )}
\end{itemize}
Que acalenta. Cf. J. Diniz, \textunderscore Morgadinha\textunderscore , 172.
\section{Acalentar}
\begin{itemize}
\item {Grp. gram.:v. t.}
\end{itemize}
\begin{itemize}
\item {Utilização:Fig.}
\end{itemize}
\begin{itemize}
\item {Proveniência:(Do lat. \textunderscore calens\textunderscore , \textunderscore calentis\textunderscore )}
\end{itemize}
Aquecer nos braços, aconchegando ao peito.
Embalar.
Amimar.
Tranquillizar.
Lisonjear.
\section{Acalento}
\begin{itemize}
\item {Grp. gram.:m.}
\end{itemize}
Acto de \textunderscore acalentar\textunderscore .
\section{Acalephologia}
\begin{itemize}
\item {Grp. gram.:f.}
\end{itemize}
Parte da Zoologia, que trata dos acalephos.
\section{Acalephos}
\begin{itemize}
\item {Grp. gram.:m. pl.}
\end{itemize}
\begin{itemize}
\item {Utilização:Zool.}
\end{itemize}
\begin{itemize}
\item {Proveniência:(Do gr. \textunderscore akalephe\textunderscore )}
\end{itemize}
Classe de zoóphytos, a que pertence a alforreca.
Designação antiga dos celenterados.
\section{Acalicino}
\begin{itemize}
\item {Grp. gram.:adj.}
\end{itemize}
\begin{itemize}
\item {Utilização:Bot.}
\end{itemize}
\begin{itemize}
\item {Proveniência:(De \textunderscore a priv.\textunderscore  + \textunderscore cálice\textunderscore )}
\end{itemize}
Que não tem cálice.
\section{Acalifa}
\begin{itemize}
\item {Grp. gram.:f.}
\end{itemize}
\begin{itemize}
\item {Proveniência:(Gr. \textunderscore akalupha\textunderscore )}
\end{itemize}
Planta medicinal de Malabar, (\textunderscore acalypha indica\textunderscore , Lin.)
\section{Acalipto}
\begin{itemize}
\item {Grp. gram.:m.}
\end{itemize}
\begin{itemize}
\item {Proveniência:(Do gr. \textunderscore kaluptos\textunderscore .)}
\end{itemize}
Gênero de serpentes venenosas que habitam nos pântanos.
\section{Acalmação}
\begin{itemize}
\item {Grp. gram.:f.}
\end{itemize}
Acto ou effeito de \textunderscore acalmar\textunderscore .
\section{Acalmamento}
\begin{itemize}
\item {Grp. gram.:m.}
\end{itemize}
O mesmo que \textunderscore acalmação\textunderscore .
\section{Acalmar}
\begin{itemize}
\item {Grp. gram.:v. t.}
\end{itemize}
\begin{itemize}
\item {Grp. gram.:V. i.}
\end{itemize}
Tornar calmo, tranquillizar.
Serenar: \textunderscore acalmar paixões\textunderscore .
Ficar em sossêgo.
Abrandar: \textunderscore o vento acalmou\textunderscore .
\section{Açalmar}
\begin{itemize}
\item {Grp. gram.:v. t.}
\end{itemize}
\begin{itemize}
\item {Utilização:Ant.}
\end{itemize}
Abastecer, prover. Cf. Fernão Lopes.
\section{Acalmia}
\begin{itemize}
\item {Grp. gram.:f.}
\end{itemize}
\begin{itemize}
\item {Utilização:Med.}
\end{itemize}
\begin{itemize}
\item {Proveniência:(De \textunderscore a\textunderscore  priv. + \textunderscore calmo\textunderscore )}
\end{itemize}
Período de acalmação, que succede ao período do calor e da vivacidade das ideias, no estado febril. Cf. Sousa Martins, \textunderscore Nosographia\textunderscore .
\section{Acalorar}
\begin{itemize}
\item {Grp. gram.:v. t.}
\end{itemize}
Communicar calor a, aquecer.
Excitar.
\section{Açalpão}
\begin{itemize}
\item {Grp. gram.:m.}
\end{itemize}
\begin{itemize}
\item {Utilização:Bras}
\end{itemize}
O mesmo que \textunderscore alçapão\textunderscore .
(Metáth. de \textunderscore alçapão\textunderscore )
\section{Acalypha}
\begin{itemize}
\item {Grp. gram.:f.}
\end{itemize}
\begin{itemize}
\item {Proveniência:(Gr. \textunderscore akalupha\textunderscore )}
\end{itemize}
Planta medicinal de Malabar, (\textunderscore acalypha indica\textunderscore , Lin.).
\section{Acalypto}
\begin{itemize}
\item {Grp. gram.:m.}
\end{itemize}
\begin{itemize}
\item {Proveniência:(Do gr. \textunderscore kaluptos\textunderscore .)}
\end{itemize}
Gênero de serpentes venenosas que habitam nos pântanos.
\section{Acamacu}
\begin{itemize}
\item {Grp. gram.:m.}
\end{itemize}
Pássaro dentirostro do Senegal e Madagáscar.
\section{Açamado}
\begin{itemize}
\item {Grp. gram.:adj.}
\end{itemize}
\begin{itemize}
\item {Utilização:Fig.}
\end{itemize}
Preso com açamo.
Reprimido.
\section{Acamar}
\begin{itemize}
\item {Grp. gram.:v. t.}
\end{itemize}
\begin{itemize}
\item {Grp. gram.:V. i.}
\end{itemize}
Dispor em camada: \textunderscore acamar a fruta\textunderscore .
Estender no chão ou noutra superficie.
Ficar doente de cama.
Definhar.
Diz-se das searas que, em certa direcção, se inclinam, deitando-se quási no solo, por effeito do vento ou do pêso das espigas: \textunderscore a seara acamou\textunderscore .
Deitar-se na cama; dormir.
\section{Açamar}
\begin{itemize}
\item {Grp. gram.:v. t.}
\end{itemize}
\begin{itemize}
\item {Utilização:Fig.}
\end{itemize}
Prender com açamo.
Pôr açamo a.
Reprimir.
\section{Acamaradar-se}
\begin{itemize}
\item {Grp. gram.:v. t.}
\end{itemize}
Tornar-se camarada ou companheiro.
Abandear-se.
\section{Açamarrado}
\begin{itemize}
\item {Grp. gram.:adj.}
\end{itemize}
Vestido de çamarra: \textunderscore um pastor açamarrado\textunderscore .
\section{Açamarrar-se}
\begin{itemize}
\item {Grp. gram.:v. p.}
\end{itemize}
Vestir-se de çamarra.
\section{Açambarcadeira}
\begin{itemize}
\item {Grp. gram.:f.}
\end{itemize}
\begin{itemize}
\item {Utilização:Prov.}
\end{itemize}
\begin{itemize}
\item {Utilização:minh.}
\end{itemize}
Mulher, que açambarca os gêneros trazidos ao mercado, para os revender, lucrando.
\section{Açambarcador}
\begin{itemize}
\item {Grp. gram.:m.  e  adj.}
\end{itemize}
O que açambarca.
\section{Açambarcagem}
\begin{itemize}
\item {Grp. gram.:f.}
\end{itemize}
O mesmo que \textunderscore açambarcamento\textunderscore .
\section{Açambarcamento}
\begin{itemize}
\item {Grp. gram.:m.}
\end{itemize}
Acto ou effeito de \textunderscore açambarcar\textunderscore .
\section{Açambarcar}
\begin{itemize}
\item {Grp. gram.:v. t.}
\end{itemize}
Chamar a si ou adquirir, privando outros da respectiva vantagem.
Monopolizar.
\section{Acamboar}
\begin{itemize}
\item {Grp. gram.:v. t.}
\end{itemize}
Meter (os bois) ao cambão.
\section{Acambolhado}
\begin{itemize}
\item {Grp. gram.:adj.}
\end{itemize}
Deitado de cambolhada.
\section{Açame}
\begin{itemize}
\item {Grp. gram.:m.}
\end{itemize}
Apparelho, que se applica ao focinho dos cães ou de outros animaes, para não morderem ou não comerem.
Focinheira.
Mordaça.
\section{Açamo}
\begin{itemize}
\item {Grp. gram.:m.}
\end{itemize}
Apparelho, que se applica ao focinho dos cães ou de outros animaes, para não morderem ou não comerem.
Focinheira.
Mordaça.
\section{Acamonia}
\begin{itemize}
\item {Grp. gram.:f.}
\end{itemize}
\begin{itemize}
\item {Utilização:Bras}
\end{itemize}
Espécie de doce, em que se emprega o gengibre.
\section{Açamoucado}
\begin{itemize}
\item {Grp. gram.:m.}
\end{itemize}
Mau emprêgo de materiaes de construcçâo, sem arte, sem gôsto e sem segurança.
\section{Acampainhar}
\begin{itemize}
\item {fónica:pa-i}
\end{itemize}
\begin{itemize}
\item {Grp. gram.:v. t.}
\end{itemize}
Dar fórma de campaínha a.
\section{Acampamento}
\begin{itemize}
\item {Grp. gram.:m.}
\end{itemize}
Acto de \textunderscore acampar\textunderscore .
Lugar, onde se acampou.
Arraial.
\section{Acampar}
\begin{itemize}
\item {Grp. gram.:v. t.}
\end{itemize}
\begin{itemize}
\item {Grp. gram.:V. i.  e  p.}
\end{itemize}
Estabelecer em campo.
Estabelecer-se no campo.
Assentar arraial.
Estacionar, (falando-se de muita gente).
\section{Acampsia}
\begin{itemize}
\item {Grp. gram.:f.}
\end{itemize}
\begin{itemize}
\item {Utilização:Med.}
\end{itemize}
Inflexibilidade das articulações.
Anquilose.
\section{Acampto}
\begin{itemize}
\item {Grp. gram.:m.}
\end{itemize}
\begin{itemize}
\item {Utilização:Phýs.}
\end{itemize}
\begin{itemize}
\item {Proveniência:(Do gr. \textunderscore a\textunderscore  priv. + \textunderscore kamptein\textunderscore , dobrar)}
\end{itemize}
Que não reflecte luz.
\section{Acamptosomos}
\begin{itemize}
\item {fónica:tossô}
\end{itemize}
\begin{itemize}
\item {Grp. gram.:m. pl.}
\end{itemize}
\begin{itemize}
\item {Utilização:Zool.}
\end{itemize}
\begin{itemize}
\item {Proveniência:(Do gr. \textunderscore a\textunderscore  priv. + \textunderscore kampto\textunderscore , dóbro + \textunderscore soma\textunderscore , corpo)}
\end{itemize}
Família dos animaes da classe dos cirrípedes.
\section{Acamurçado}
\begin{itemize}
\item {Grp. gram.:adj.}
\end{itemize}
\begin{itemize}
\item {Proveniência:(De \textunderscore acamurçar\textunderscore )}
\end{itemize}
Que tem aspecto de camurça.
\section{Acamurçar}
\begin{itemize}
\item {Grp. gram.:v. t.}
\end{itemize}
Preparar como camurça.
\section{Açan}
\begin{itemize}
\item {Grp. gram.:m.}
\end{itemize}
\begin{itemize}
\item {Utilização:Prov.}
\end{itemize}
\begin{itemize}
\item {Utilização:beir.}
\end{itemize}
\begin{itemize}
\item {Utilização:trasm.}
\end{itemize}
Bicho, que apparece no queijo e na carne de porco.
\section{Acanadas}
\begin{itemize}
\item {Grp. gram.:f. pl.}
\end{itemize}
\begin{itemize}
\item {Utilização:Bot.}
\end{itemize}
O mesmo que \textunderscore chicoriáceas\textunderscore .
\section{Acanalado}
\begin{itemize}
\item {Grp. gram.:adj.}
\end{itemize}
\begin{itemize}
\item {Proveniência:(De \textunderscore acanalar\textunderscore )}
\end{itemize}
Que tem estrias.
Cavado longitudinalmente.
\section{Acanalador}
\begin{itemize}
\item {Grp. gram.:adj.}
\end{itemize}
Que acanala.
\section{Acanaladura}
\begin{itemize}
\item {Grp. gram.:f.}
\end{itemize}
\begin{itemize}
\item {Proveniência:(De \textunderscore acanalar\textunderscore )}
\end{itemize}
Estria, cavidade ou rêgo longitudinal.
\section{Acanalar}
\begin{itemize}
\item {Grp. gram.:v. t.}
\end{itemize}
\begin{itemize}
\item {Proveniência:(De \textunderscore canal\textunderscore )}
\end{itemize}
Abrir estrias em.
Cavar longitudinalmente.
\section{Acanalhado}
\begin{itemize}
\item {Grp. gram.:adj.}
\end{itemize}
\begin{itemize}
\item {Proveniência:(De \textunderscore acanalhar\textunderscore )}
\end{itemize}
Que tem modos de canalha.
\section{Acanalhar}
\begin{itemize}
\item {Grp. gram.:v. t.}
\end{itemize}
\begin{itemize}
\item {Grp. gram.:V. p.}
\end{itemize}
Dar modos de canalha a.
Tornar canalha.
Fazer-se canalha, aviltar-se. Cf. Camillo, \textunderscore Narcót\textunderscore , II, 80.
\section{Acanaveado}
\begin{itemize}
\item {Grp. gram.:adj.}
\end{itemize}
\begin{itemize}
\item {Utilização:Fig.}
\end{itemize}
\begin{itemize}
\item {Proveniência:(De \textunderscore acanavear\textunderscore )}
\end{itemize}
Abatido, magro.
\section{Acanaveadura}
\begin{itemize}
\item {Grp. gram.:f.}
\end{itemize}
Acto ou effeito de \textunderscore acanavear\textunderscore .
\section{Acanavear}
\begin{itemize}
\item {Grp. gram.:v. t.}
\end{itemize}
\begin{itemize}
\item {Utilização:Fam.}
\end{itemize}
\begin{itemize}
\item {Proveniência:(De \textunderscore cana\textunderscore )}
\end{itemize}
Suppliciar, metendo puas de cana, entre as unhas e a carne.
Tornar abatido, magro.
\section{Açancalhar}
\begin{itemize}
\item {Grp. gram.:v. t.}
\end{itemize}
\begin{itemize}
\item {Utilização:Prov.}
\end{itemize}
\begin{itemize}
\item {Utilização:beir.}
\end{itemize}
\begin{itemize}
\item {Grp. gram.:V. i.}
\end{itemize}
\begin{itemize}
\item {Utilização:T. da Bairrada}
\end{itemize}
\begin{itemize}
\item {Proveniência:(De \textunderscore çanco\textunderscore )}
\end{itemize}
Traçar ou escrever mal.
Realizar mal ou atabalhoadamente.
Trabalhar á pressa e mal.
Pernear, escabujar.
\section{Açancanhar}
\begin{itemize}
\item {Grp. gram.:v. t.}
\end{itemize}
\begin{itemize}
\item {Utilização:Prov.}
\end{itemize}
\begin{itemize}
\item {Utilização:trasm.}
\end{itemize}
\begin{itemize}
\item {Grp. gram.:V. i.}
\end{itemize}
O mesmo que \textunderscore pisar\textunderscore .
Andar ligeiro.
\section{Acanelado}
\begin{itemize}
\item {Grp. gram.:adj.}
\end{itemize}
\begin{itemize}
\item {Proveniência:(De \textunderscore acanelar\textunderscore )}
\end{itemize}
Que tem côr de canela.
\section{Acaneladura}
\begin{itemize}
\item {Grp. gram.:f.}
\end{itemize}
(V.caneladura)
\section{Acanelar}
\begin{itemize}
\item {Grp. gram.:v. t.}
\end{itemize}
Dar côr de canela a.
Cobrir com o pó de canela.
O mesmo que \textunderscore acanalar\textunderscore .
\section{Acanga}
\begin{itemize}
\item {Grp. gram.:f.}
\end{itemize}
O mesmo que \textunderscore gallinha-da-Índia\textunderscore .
\section{Açanganhar}
\begin{itemize}
\item {Grp. gram.:v. t.}
\end{itemize}
O mesmo que \textunderscore açancanhar\textunderscore .
\section{Acanhadamente}
\begin{itemize}
\item {Grp. gram.:adv.}
\end{itemize}
De modo \textunderscore acanhado\textunderscore , com acanhamento.
\section{Acanhado}
\begin{itemize}
\item {Grp. gram.:adj.}
\end{itemize}
\begin{itemize}
\item {Utilização:Fig.}
\end{itemize}
\begin{itemize}
\item {Proveniência:(De \textunderscore acanhar\textunderscore )}
\end{itemize}
Pouco desenvolvido.
Encolhido.
Tímido.
Mesquinho.
\section{Acanhador}
\begin{itemize}
\item {Grp. gram.:m.  e  adj.}
\end{itemize}
O que causa \textunderscore acanhamento\textunderscore .
\section{Acanhamento}
\begin{itemize}
\item {Grp. gram.:m.}
\end{itemize}
Acto ou effeito de \textunderscore acanhar\textunderscore .
Qualidade de acanhado.
Timidez.
\section{Acanhar}
\begin{itemize}
\item {Grp. gram.:v. t.}
\end{itemize}
\begin{itemize}
\item {Utilização:Fig.}
\end{itemize}
\begin{itemize}
\item {Proveniência:(De \textunderscore canho\textunderscore )}
\end{itemize}
Impedir o desenvolvimento a.
Tornar tímido.
Deprimir.
\section{Acanho}
\begin{itemize}
\item {Grp. gram.:m.}
\end{itemize}
O mesmo que \textunderscore acanhamento\textunderscore . Cf. Garrett, \textunderscore Ret. de Vênus\textunderscore , 189.
\section{Acanhoar}
\begin{itemize}
\item {Grp. gram.:v. t.}
\end{itemize}
O mesmo que \textunderscore acanhonear\textunderscore .
\section{Acanhonear}
\begin{itemize}
\item {Grp. gram.:v. t.}
\end{itemize}
Disparar canhões contra.
Bombardear.
\section{Acanhotado}
\begin{itemize}
\item {Grp. gram.:adj.}
\end{itemize}
\begin{itemize}
\item {Utilização:Gír.}
\end{itemize}
\begin{itemize}
\item {Utilização:Prov.}
\end{itemize}
\begin{itemize}
\item {Utilização:minh.}
\end{itemize}
\begin{itemize}
\item {Proveniência:(De \textunderscore canhoto\textunderscore )}
\end{itemize}
O mesmo que \textunderscore triste\textunderscore .
Tôsco, grosseiro: \textunderscore bengala acanhotada\textunderscore .
Um tanto estúpido; apalermado: \textunderscore indivíduo acanhotado\textunderscore .
\section{Acanonicamente}
\begin{itemize}
\item {Grp. gram.:adv.}
\end{itemize}
De modo \textunderscore acanónico\textunderscore .
\section{Acanónico}
\begin{itemize}
\item {Grp. gram.:adj.}
\end{itemize}
Contrário aos cânones ou ao Direito canónico.
\section{Acanonista}
\begin{itemize}
\item {Grp. gram.:m.}
\end{itemize}
O transgressor dos cânones.
\section{Acantábalo}
\begin{itemize}
\item {Grp. gram.:m.}
\end{itemize}
\begin{itemize}
\item {Proveniência:(Do gr. \textunderscore akantha\textunderscore  + \textunderscore ballo\textunderscore )}
\end{itemize}
Instrumento cirúrgico, para tirar esquírolas de ossos ou corpos estranhos, introduzidos nos órgãos.
\section{Acantáceas}
\begin{itemize}
\item {Grp. gram.:f. pl.}
\end{itemize}
Familia de plantas, que tem por typo o acantho.
\section{A-cantaros}
\begin{itemize}
\item {Grp. gram.:loc. adv.}
\end{itemize}
Copiosamente, com abundância.
\section{Acanteirado}
\begin{itemize}
\item {Grp. gram.:adj.}
\end{itemize}
\begin{itemize}
\item {Proveniência:(De \textunderscore acanteirar\textunderscore )}
\end{itemize}
Dividido em canteiros.
\section{Acanteirar}
\begin{itemize}
\item {Grp. gram.:v. t.}
\end{itemize}
Dividir em canteiros, (falando-se de hortas ou jardins).
\section{Acanthábalo}
\begin{itemize}
\item {Grp. gram.:m.}
\end{itemize}
\begin{itemize}
\item {Proveniência:(Do gr. \textunderscore akantha\textunderscore  + \textunderscore ballo\textunderscore )}
\end{itemize}
Instrumento cirúrgico, para tirar esquírolas de ossos ou corpos estranhos, introduzidos nos órgãos.
\section{Acantháceas}
\begin{itemize}
\item {Grp. gram.:f. pl.}
\end{itemize}
Familia de plantas, que tem por typo o acantho.
\section{Acanthia}
\begin{itemize}
\item {Grp. gram.:f.}
\end{itemize}
\begin{itemize}
\item {Proveniência:(Do gr. \textunderscore akantha\textunderscore )}
\end{itemize}
Insecto hemíptero, de que o percevejo é uma espécie.
\section{Acanthião}
\begin{itemize}
\item {Grp. gram.:m.}
\end{itemize}
\begin{itemize}
\item {Proveniência:(Do gr. \textunderscore akantha\textunderscore )}
\end{itemize}
Gênero de mamíferos espinhosos, a que pertence o ouriço.
\section{Acantho}
\begin{itemize}
\item {Grp. gram.:m.}
\end{itemize}
\begin{itemize}
\item {Proveniência:(Gr. \textunderscore akanthos\textunderscore )}
\end{itemize}
Gênero de plantas, vulgarmente conhecido por \textunderscore erva gigante\textunderscore .
Ornato de architectura, que representa folhas daquella planta.
\section{Acanthocárpio}
\begin{itemize}
\item {Grp. gram.:adj.}
\end{itemize}
\begin{itemize}
\item {Proveniência:(Do gr. \textunderscore akanthos\textunderscore  + \textunderscore carpos\textunderscore )}
\end{itemize}
Diz-se da planta, cujos frutos são cobertos de espinhos.
\section{Acanthocéphalos}
\begin{itemize}
\item {Grp. gram.:m. pl.}
\end{itemize}
\begin{itemize}
\item {Proveniência:(Do gr. \textunderscore akantha\textunderscore  + \textunderscore kephale\textunderscore )}
\end{itemize}
Vermes intestinaes, cuja cabeça é armada de aguilhão.
\section{Acanthocládio}
\begin{itemize}
\item {Grp. gram.:adj.}
\end{itemize}
\begin{itemize}
\item {Proveniência:(Do gr. \textunderscore akantha\textunderscore  + \textunderscore klados\textunderscore )}
\end{itemize}
Diz-se das plantas que têm ramos espinhosos.
\section{Acanthodáctylos}
\begin{itemize}
\item {Grp. gram.:m. pl.}
\end{itemize}
\begin{itemize}
\item {Proveniência:(Do gr. \textunderscore akantha\textunderscore  + \textunderscore daktulos\textunderscore )}
\end{itemize}
Gênero de reptis, caracterizados por terem os dedos lateralmente dentados.
\section{Acanthómetro}
\begin{itemize}
\item {Grp. gram.:m.}
\end{itemize}
\begin{itemize}
\item {Proveniência:(Do gr. \textunderscore akanthos\textunderscore  + \textunderscore metron\textunderscore )}
\end{itemize}
Animal microscópico, da classe dos rhizópodes.
\section{Acanthóphago}
\begin{itemize}
\item {Grp. gram.:adj.}
\end{itemize}
\begin{itemize}
\item {Proveniência:(Do gr. \textunderscore akantha\textunderscore  + \textunderscore phagein\textunderscore )}
\end{itemize}
Diz-se dos animaes, que se nutrem de cardos.
\section{Acanthóphoro}
\begin{itemize}
\item {Grp. gram.:adj.}
\end{itemize}
\begin{itemize}
\item {Proveniência:(Do gr. \textunderscore akantha\textunderscore  + \textunderscore phoros\textunderscore )}
\end{itemize}
Erriçado de espinhos.
\section{Acanthópio}
\begin{itemize}
\item {Grp. gram.:adj.}
\end{itemize}
\begin{itemize}
\item {Proveniência:(Do gr. \textunderscore akantha\textunderscore  + \textunderscore ops\textunderscore )}
\end{itemize}
Diz-se dos animaes, que têm os olhos rodeados de espinhos.
\section{Acanthopomos}
\begin{itemize}
\item {Grp. gram.:m. pl.}
\end{itemize}
\begin{itemize}
\item {Proveniência:(Do gr. \textunderscore akantha\textunderscore  + \textunderscore poma\textunderscore )}
\end{itemize}
Família de peixes, caracterizados por opérculos cercados de espinhos.
\section{Acanthopterygiano}
\begin{itemize}
\item {Grp. gram.:adj.}
\end{itemize}
Relativo aos acanthopterygios.
\section{Acanthopterygio}
\begin{itemize}
\item {Grp. gram.:adj.}
\end{itemize}
\begin{itemize}
\item {Grp. gram.:Pl.}
\end{itemize}
\begin{itemize}
\item {Proveniência:(Do gr. \textunderscore akanthos\textunderscore  + \textunderscore pterux\textunderscore )}
\end{itemize}
Que tem barbatanas espinhosas.
Ordem de peixes acanthopterygios.
\section{Acanthorinas}
\begin{itemize}
\item {Grp. gram.:f. pl.}
\end{itemize}
\begin{itemize}
\item {Proveniência:(Do gr. \textunderscore akantha\textunderscore  + \textunderscore rhin\textunderscore )}
\end{itemize}
Família de peixes, que apresentam entre os olhos uma espécie de nariz, armado de aguilhão.
\section{Acanthospermo}
\begin{itemize}
\item {Grp. gram.:m.}
\end{itemize}
\begin{itemize}
\item {Proveniência:(Do gr. \textunderscore akantha\textunderscore  + \textunderscore sperma\textunderscore )}
\end{itemize}
Genero de plantas compostas.
\section{Acanthura}
\begin{itemize}
\item {Grp. gram.:m.}
\end{itemize}
\begin{itemize}
\item {Proveniência:(Do gr. \textunderscore akantha\textunderscore  + \textunderscore oura\textunderscore )}
\end{itemize}
Peixe que, de cada lado da cauda, tem um grande espinho movediço.
\section{Acantia}
\begin{itemize}
\item {Grp. gram.:f.}
\end{itemize}
\begin{itemize}
\item {Proveniência:(Do gr. \textunderscore akantha\textunderscore )}
\end{itemize}
Insecto hemíptero, de que o percevejo é uma espécie.
\section{Acantião}
\begin{itemize}
\item {Grp. gram.:m.}
\end{itemize}
\begin{itemize}
\item {Proveniência:(Do gr. \textunderscore akantha\textunderscore )}
\end{itemize}
Gênero de mamíferos espinhosos, a que pertence o ouriço.
\section{Acanto}
\begin{itemize}
\item {Grp. gram.:m.}
\end{itemize}
\begin{itemize}
\item {Proveniência:(Gr. \textunderscore akanthos\textunderscore )}
\end{itemize}
Gênero de plantas, vulgarmente conhecido por \textunderscore erva gigante\textunderscore .
Ornato de architectura, que representa folhas daquella planta.
\section{Acantoameato}
\begin{itemize}
\item {Grp. gram.:m.}
\end{itemize}
Acto ou effeito de acantoar.
\section{Acantoar}
\begin{itemize}
\item {Grp. gram.:v. t.}
\end{itemize}
Pôr a um canto.
Occultar.
Separar; apartar.
Desprezar.
\section{Acantocárpio}
\begin{itemize}
\item {Grp. gram.:adj.}
\end{itemize}
\begin{itemize}
\item {Proveniência:(Do gr. \textunderscore akanthos\textunderscore  + \textunderscore carpos\textunderscore )}
\end{itemize}
Diz-se da planta, cujos frutos são cobertos de espinhos.
\section{Acantocéfalos}
\begin{itemize}
\item {Grp. gram.:m. pl.}
\end{itemize}
\begin{itemize}
\item {Proveniência:(Do gr. \textunderscore akantha\textunderscore  + \textunderscore kephale\textunderscore )}
\end{itemize}
Vermes intestinaes, cuja cabeça é armada de aguilhão.
\section{Acantochanado}
\begin{itemize}
\item {Grp. gram.:adj.}
\end{itemize}
\begin{itemize}
\item {Utilização:Mús.}
\end{itemize}
Que tem o carácter de cantochão.
\section{Acantocládio}
\begin{itemize}
\item {Grp. gram.:adj.}
\end{itemize}
\begin{itemize}
\item {Proveniência:(Do gr. \textunderscore akantha\textunderscore  + \textunderscore klados\textunderscore )}
\end{itemize}
Diz-se das plantas que têm ramos espinhosos.
\section{Acantodáctilos}
\begin{itemize}
\item {Grp. gram.:m. pl.}
\end{itemize}
\begin{itemize}
\item {Proveniência:(Do gr. \textunderscore akantha\textunderscore  + \textunderscore daktulos\textunderscore )}
\end{itemize}
Gênero de reptis, caracterizados por terem os dedos lateralmente dentados.
\section{Acantófago}
\begin{itemize}
\item {Grp. gram.:adj.}
\end{itemize}
\begin{itemize}
\item {Proveniência:(Do gr. \textunderscore akantha\textunderscore  + \textunderscore phagein\textunderscore )}
\end{itemize}
Diz-se dos animaes, que se nutrem de cardos.
\section{Acantóforo}
\begin{itemize}
\item {Grp. gram.:adj.}
\end{itemize}
\begin{itemize}
\item {Proveniência:(Do gr. \textunderscore akantha\textunderscore  + \textunderscore phoros\textunderscore )}
\end{itemize}
Erriçado de espinhos.
\section{Acantómetro}
\begin{itemize}
\item {Grp. gram.:m.}
\end{itemize}
\begin{itemize}
\item {Proveniência:(Do gr. \textunderscore akanthos\textunderscore  + \textunderscore metron\textunderscore )}
\end{itemize}
Animal microscópico, da classe dos rhizópodes.
\section{Acantonamento}
\begin{itemize}
\item {Grp. gram.:m.}
\end{itemize}
Acto ou effeito de \textunderscore acantonar\textunderscore .
Lugar, onde se acantonam tropas.
\section{Acantonar}
\begin{itemize}
\item {Grp. gram.:v. t.}
\end{itemize}
\begin{itemize}
\item {Proveniência:(De \textunderscore cantão\textunderscore )}
\end{itemize}
Dispor ou distribuir (tropas) por cantões ou aldeias.
\section{Acantópio}
\begin{itemize}
\item {Grp. gram.:adj.}
\end{itemize}
\begin{itemize}
\item {Proveniência:(Do gr. \textunderscore akantha\textunderscore  + \textunderscore ops\textunderscore )}
\end{itemize}
Diz-se dos animaes, que têm os olhos rodeados de espinhos.
\section{Acantopomos}
\begin{itemize}
\item {Grp. gram.:m. pl.}
\end{itemize}
\begin{itemize}
\item {Proveniência:(Do gr. \textunderscore akantha\textunderscore  + \textunderscore poma\textunderscore )}
\end{itemize}
Família de peixes, caracterizados por opérculos cercados de espinhos.
\section{Acantopterigiano}
\begin{itemize}
\item {Grp. gram.:adj.}
\end{itemize}
Relativo aos acanthopterygios.
\section{Acantopterígio}
\begin{itemize}
\item {Grp. gram.:adj.}
\end{itemize}
\begin{itemize}
\item {Grp. gram.:Pl.}
\end{itemize}
\begin{itemize}
\item {Proveniência:(Do gr. \textunderscore akanthos\textunderscore  + \textunderscore pterux\textunderscore )}
\end{itemize}
Que tem barbatanas espinhosas.
Ordem de peixes acanthopterygios.
\section{Acantorinas}
\begin{itemize}
\item {Grp. gram.:f. pl.}
\end{itemize}
\begin{itemize}
\item {Proveniência:(Do gr. \textunderscore akantha\textunderscore  + \textunderscore rhin\textunderscore )}
\end{itemize}
Família de peixes, que apresentam entre os olhos uma espécie de nariz, armado de aguilhão.
\section{Acantospermo}
\begin{itemize}
\item {Grp. gram.:m.}
\end{itemize}
\begin{itemize}
\item {Proveniência:(Do gr. \textunderscore akantha\textunderscore  + \textunderscore sperma\textunderscore )}
\end{itemize}
Genero de plantas compostas.
\section{Acantura}
\begin{itemize}
\item {Grp. gram.:m.}
\end{itemize}
\begin{itemize}
\item {Proveniência:(Do gr. \textunderscore akantha\textunderscore  + \textunderscore oura\textunderscore )}
\end{itemize}
Peixe que, de cada lado da cauda, tem um grande espinho movediço.
\section{Acanular}
\begin{itemize}
\item {Grp. gram.:v. t.}
\end{itemize}
Dar fórma de cana ou de cânula:«\textunderscore mecha acanulada de chumbo\textunderscore ». Corvo, \textunderscore Anno na Côrte\textunderscore , II, 15.
\section{Açapar}
\begin{itemize}
\item {Grp. gram.:v. t.}
\end{itemize}
\begin{itemize}
\item {Utilização:Pop.}
\end{itemize}
O mesmo que \textunderscore acaçapar\textunderscore .
\section{Acapelado}
\begin{itemize}
\item {Grp. gram.:adj.}
\end{itemize}
\begin{itemize}
\item {Utilização:ant.}
\end{itemize}
\begin{itemize}
\item {Utilização:Pop.}
\end{itemize}
\begin{itemize}
\item {Utilização:Des.}
\end{itemize}
\begin{itemize}
\item {Proveniência:(De \textunderscore acapelar\textunderscore )}
\end{itemize}
Perseguido.
Metido debaixo de água, submergido. Cf. Costa e Sá, \textunderscore Diccion.\textunderscore 
\section{Acapelar}
\begin{itemize}
\item {Grp. gram.:v. t.}
\end{itemize}
Dar feitio de capello a.
Cobrir com capello.
Encapellar.
\section{Acapellar}
\begin{itemize}
\item {Grp. gram.:v. t.}
\end{itemize}
Dar feitio de capello a.
Cobrir com capello.
Encapellar.
\section{Acapitular}
\begin{itemize}
\item {Grp. gram.:v. t.}
\end{itemize}
Dividir em capitulos.
Admoestar ou censurar em capitulo.
\section{Acapna}
\begin{itemize}
\item {Grp. gram.:f.}
\end{itemize}
\begin{itemize}
\item {Proveniência:(Do gr. \textunderscore akapnon\textunderscore )}
\end{itemize}
Lenha sêca, que não deita fumo.
\section{Acapno}
\begin{itemize}
\item {Grp. gram.:adj.}
\end{itemize}
\begin{itemize}
\item {Proveniência:(Gr. \textunderscore akapnos\textunderscore )}
\end{itemize}
Qualificação do melhor mel, que se extrái da colmeia sem expulsar as abelhas por meio de fumo.
\section{Acapu}
\begin{itemize}
\item {Grp. gram.:m.}
\end{itemize}
Árvore leguminosa da América.
\section{Acapu-rana}
\begin{itemize}
\item {Grp. gram.:m.}
\end{itemize}
Árvore do norte do Brasil, bôa para construcções.
\section{...açar}
\begin{itemize}
\item {Grp. gram.:suf.}
\end{itemize}
(designativo de \textunderscore aumento\textunderscore  ou \textunderscore frequência\textunderscore )
\section{Acará}
\begin{itemize}
\item {Grp. gram.:m.}
\end{itemize}
\begin{itemize}
\item {Utilização:Bras}
\end{itemize}
O mesmo que \textunderscore acarajé\textunderscore .
\section{Acará}
\begin{itemize}
\item {Grp. gram.:m.}
\end{itemize}
\begin{itemize}
\item {Utilização:Bras}
\end{itemize}
Designação vulgar de várias espécies de peixes.
\section{Acarajé}
\begin{itemize}
\item {Grp. gram.:m.}
\end{itemize}
\begin{itemize}
\item {Utilização:Bras}
\end{itemize}
Iguaria de massa de feijão cozido.
\section{Acaramelado}
\begin{itemize}
\item {Grp. gram.:adj.}
\end{itemize}
Coberto de açúcar, em ponto de caramelo.
\section{Acarangado}
\begin{itemize}
\item {Grp. gram.:adj.}
\end{itemize}
O mesmo que \textunderscore encarangado\textunderscore .
\section{Acarão}
\begin{itemize}
\item {Grp. gram.:adv.}
\end{itemize}
\begin{itemize}
\item {Utilização:Ant.}
\end{itemize}
\begin{itemize}
\item {Proveniência:(De \textunderscore carão\textunderscore )}
\end{itemize}
Na frente, em frente, de frente.
\section{Acarapinhar}
\begin{itemize}
\item {Grp. gram.:v. t.}
\end{itemize}
(V.encarapinhar)
\section{Acarar}
\begin{itemize}
\item {Grp. gram.:v. t.}
\end{itemize}
(V.encarar)
\section{Acaraúba}
\begin{itemize}
\item {Grp. gram.:f.}
\end{itemize}
Árvore medicinal do Alto Amazonas.
\section{Acarda}
\begin{itemize}
\item {Grp. gram.:f.}
\end{itemize}
\begin{itemize}
\item {Proveniência:(Do lat. \textunderscore cardo\textunderscore )}
\end{itemize}
Mollusco gasterópode.
\section{Acardia}
\begin{itemize}
\item {Grp. gram.:f.}
\end{itemize}
\begin{itemize}
\item {Proveniência:(Do gr. \textunderscore a\textunderscore  priv. + \textunderscore kardia\textunderscore )}
\end{itemize}
Aberração orgânica, que consiste na falta de coração.
\section{Acardumar-se}
\begin{itemize}
\item {Grp. gram.:v. p.}
\end{itemize}
Reunir-se em cardume.
\section{Acareação}
\begin{itemize}
\item {Grp. gram.:f.}
\end{itemize}
Acto ou effeito de \textunderscore acarear\textunderscore .
\section{Acareador}
\begin{itemize}
\item {Grp. gram.:adj.}
\end{itemize}
Que acareia.
\section{Acareamento}
\begin{itemize}
\item {Grp. gram.:m.}
\end{itemize}
O mesmo que \textunderscore acareação\textunderscore .
\section{Acareante}
\begin{itemize}
\item {Grp. gram.:adj.}
\end{itemize}
Que attrai, que se torna sympáthico:«\textunderscore muito acareante pessôa\textunderscore ». Filinto, XIX, 243.
\section{Acarear}
\begin{itemize}
\item {Grp. gram.:v. t.}
\end{itemize}
\begin{itemize}
\item {Proveniência:(De \textunderscore cara\textunderscore )}
\end{itemize}
Pôr em frente.
Confrontar ou pôr em presença umas das outras (testemunhas ou outras pessôas, cujos depoimentos ou declarações não foram concordes), para se apurar a verdade, ouvindo-as de novo.
\section{Acarear}
\begin{itemize}
\item {Grp. gram.:v. t.}
\end{itemize}
\begin{itemize}
\item {Utilização:Prov.}
\end{itemize}
\begin{itemize}
\item {Utilização:minh.}
\end{itemize}
\begin{itemize}
\item {Proveniência:(De \textunderscore caro?\textunderscore  ou affim de \textunderscore carear^1?\textunderscore )}
\end{itemize}
Attrahir com afagos; chamar para si, tornar-se alvo de (sympathias, affeição).
O mesmo que \textunderscore acarrear\textunderscore . Cf. Filinto, \textunderscore V. de D. Manuel\textunderscore . I, 119, e II, 10; \textunderscore Id., Obras\textunderscore , XIX, 13; XX, 8; etc.
Reunir e conduzir para a córte (o gado, geralmente o vacum).
\section{Acarentar}
\begin{itemize}
\item {Grp. gram.:v. t.  e  i.}
\end{itemize}
\begin{itemize}
\item {Utilização:Ant.}
\end{itemize}
Tornar caro, encarecer.
\section{Acari}
\begin{itemize}
\item {Grp. gram.:m.}
\end{itemize}
Bichinho, que se cria no queijo, na farinha e na cera:«\textunderscore bichinho tamanino, qual é o chamado acari e se cria na cera corrupta\textunderscore ». \textunderscore Luz e Calor\textunderscore , 554.
\section{Acaríase}
\begin{itemize}
\item {Grp. gram.:f.}
\end{itemize}
\begin{itemize}
\item {Proveniência:(Do gr. \textunderscore akari\textunderscore )}
\end{itemize}
Doença, causada por ácaros.
\section{Acaricada}
\begin{itemize}
\item {Grp. gram.:f.}
\end{itemize}
\begin{itemize}
\item {Utilização:Bras}
\end{itemize}
Planta medicinal.
\section{Acariciadamente}
\begin{itemize}
\item {Grp. gram.:adv.}
\end{itemize}
\begin{itemize}
\item {Proveniência:(De \textunderscore acariciado\textunderscore )}
\end{itemize}
Com carícia.
\section{Acariciador}
\begin{itemize}
\item {Grp. gram.:adj.}
\end{itemize}
Que acaricia.
\section{Acariciante}
\begin{itemize}
\item {Grp. gram.:adj.}
\end{itemize}
O mesmo que \textunderscore acariciador\textunderscore .
\section{Acariciar}
\begin{itemize}
\item {Grp. gram.:v. t.}
\end{itemize}
Fazer caricias a.
Amimar; acarinhar.
Fazer festas a.
Lisonjear.
Roçar.
Tocar de leve.
\section{Acariciativo}
\begin{itemize}
\item {Grp. gram.:adj.}
\end{itemize}
Acariciador; em que há carícias.
\section{Acaricuara}
\begin{itemize}
\item {Grp. gram.:f.}
\end{itemize}
\begin{itemize}
\item {Utilização:Bras}
\end{itemize}
Árvore e madeira de construcção, na região do Purus.
\section{Acaridar-se}
\begin{itemize}
\item {Grp. gram.:v. p.}
\end{itemize}
Têr caridade, compadecer-se.
\section{Acarídeos}
\begin{itemize}
\item {Grp. gram.:m. pl.}
\end{itemize}
\begin{itemize}
\item {Proveniência:(Do gr. \textunderscore akari\textunderscore  + \textunderscore eidos\textunderscore )}
\end{itemize}
Ordem de arachnídeos, a que os ácaros dão o nome.
\section{Acarima}
\begin{itemize}
\item {Grp. gram.:m.}
\end{itemize}
Macaco da Guiana, semelhante ao saguim.
\section{Acarinhar}
\begin{itemize}
\item {Grp. gram.:v. t.}
\end{itemize}
Tratar com carinho, acariciar.
\section{Acário}
\begin{itemize}
\item {Grp. gram.:m.}
\end{itemize}
Planta resinosa do Cabo da Bôa-Esperança.
\section{Acarminado}
\begin{itemize}
\item {Grp. gram.:adj.}
\end{itemize}
Tirante a carmim.
\section{Acarna}
\begin{itemize}
\item {Grp. gram.:m.}
\end{itemize}
\begin{itemize}
\item {Proveniência:(Gr. \textunderscore akarna\textunderscore )}
\end{itemize}
Planta exótica, da fam. das compostas.
\section{Acarneirado}
\begin{itemize}
\item {Grp. gram.:adj.}
\end{itemize}
\begin{itemize}
\item {Proveniência:(De \textunderscore carneiro\textunderscore )}
\end{itemize}
Diz-se do cavallo, cujo joêlho mostra grande depressão na parte anterior.
\section{Ácaro}
\begin{itemize}
\item {Grp. gram.:m.}
\end{itemize}
\begin{itemize}
\item {Proveniência:(Do gr. \textunderscore akari\textunderscore )}
\end{itemize}
Animálculo arachnideo.
\section{Acárpio}
\begin{itemize}
\item {Grp. gram.:adj.}
\end{itemize}
\begin{itemize}
\item {Proveniência:(Do gr. \textunderscore a\textunderscore  priv. + \textunderscore karpos\textunderscore )}
\end{itemize}
Diz-se das plantas que não dão fruto.
\section{Acarraçado}
\begin{itemize}
\item {Grp. gram.:adj.}
\end{itemize}
Agarrado, apegado como a carraça.
\section{Acarrado}
\begin{itemize}
\item {Grp. gram.:adj.}
\end{itemize}
Que acarrou.
\section{Acarrapatado}
\begin{itemize}
\item {Grp. gram.:adj.}
\end{itemize}
Semelhante ao carrapato.
\section{Acarrar}
\begin{itemize}
\item {Grp. gram.:v. i.}
\end{itemize}
\begin{itemize}
\item {Utilização:Prov.}
\end{itemize}
\begin{itemize}
\item {Utilização:Fig.}
\end{itemize}
Deixar de mover-se.
Estar no chôco.
Estar no acarro.
Dormir a sesta, descansar.
Estar doente de cama.
\section{Acarrear}
\begin{itemize}
\item {Grp. gram.:v. t.}
\end{itemize}
O mesmo que \textunderscore carrear\textunderscore .
Occasionar, causar:«\textunderscore a angústia que a memória de tuas felicidades passadas te acarreia.\textunderscore »Usque, \textunderscore Tribulações\textunderscore , 53.
\section{Acarredar}
\begin{itemize}
\item {Grp. gram.:v. t.}
\end{itemize}
\begin{itemize}
\item {Utilização:Ant.}
\end{itemize}
O mesmo que \textunderscore arrecadar\textunderscore .
(por metáth.)
\section{Acarrejar}
\begin{itemize}
\item {Grp. gram.:v. t.  e  der.}
\end{itemize}
\begin{itemize}
\item {Grp. gram.:V. i.}
\end{itemize}
\begin{itemize}
\item {Utilização:Prov.}
\end{itemize}
O mesmo que \textunderscore carrejar\textunderscore , etc.
Fazer fretes.
\section{Acarretador}
\begin{itemize}
\item {Grp. gram.:m.}
\end{itemize}
\begin{itemize}
\item {Utilização:Prov.}
\end{itemize}
\begin{itemize}
\item {Utilização:alg.}
\end{itemize}
O que acarreta.
Aquelle que transporta cereaes para o moinho, em carro ou muares.
\section{Acarretadura}
\begin{itemize}
\item {Grp. gram.:f.}
\end{itemize}
Acto ou effeito de \textunderscore acarretar\textunderscore .
\section{Acarretamento}
\begin{itemize}
\item {Grp. gram.:m.}
\end{itemize}
Acto ou effeito de \textunderscore acarretar\textunderscore .
\section{Acarretar}
\begin{itemize}
\item {Grp. gram.:v. t.}
\end{itemize}
Transportar em carrêta.
Conduzir.
Acarrear.
Occasionar.
\section{Acarrêto}
\begin{itemize}
\item {Grp. gram.:m.}
\end{itemize}
(V.carrêto)
\section{Acarro}
\begin{itemize}
\item {Grp. gram.:m.}
\end{itemize}
\begin{itemize}
\item {Utilização:Prov.}
\end{itemize}
\begin{itemize}
\item {Utilização:alent.}
\end{itemize}
\begin{itemize}
\item {Proveniência:(De \textunderscore acarrar\textunderscore )}
\end{itemize}
Sítio, aonde no verão conduzem as ovelhas, para descansarem á sombra, durante o calor.
\section{Acartadeira}
\begin{itemize}
\item {Grp. gram.:f.}
\end{itemize}
\begin{itemize}
\item {Utilização:Prov.}
\end{itemize}
\begin{itemize}
\item {Utilização:beir.}
\end{itemize}
Mulher, que num tabuleiro leva o pão ao forno, trazendo-o, depois de cozido, no mesmo tabuleiro.
\section{Acartar}
\begin{itemize}
\item {Grp. gram.:v. t.}
\end{itemize}
(Fórma pop. de \textunderscore acarretar\textunderscore )
\section{Acartonado}
\begin{itemize}
\item {Grp. gram.:adj.}
\end{itemize}
\begin{itemize}
\item {Proveniência:(De \textunderscore acartonar\textunderscore )}
\end{itemize}
Que tem aspecto de cartão.
\section{Acartonar}
\begin{itemize}
\item {Grp. gram.:v. t.}
\end{itemize}
Tornar semelhante ao cartão.
\section{Acarvar}
\begin{itemize}
\item {Grp. gram.:v. t.}
\end{itemize}
\begin{itemize}
\item {Utilização:Ant.}
\end{itemize}
Affligir, angustiar.
(Metáth. de \textunderscore acravar\textunderscore ?)
\section{Acas}
\begin{itemize}
\item {Grp. gram.:m. pl.}
\end{itemize}
Povos da África central.
\section{Acasacado}
\begin{itemize}
\item {Grp. gram.:adj.}
\end{itemize}
Semelhante á casaca.
\section{Acasalar}
\begin{itemize}
\item {Grp. gram.:v. t.}
\end{itemize}
Reunir macho e fêmea em casal, para criação.
Reunir, emparelhar.
\section{Acascarrilhado}
\begin{itemize}
\item {Grp. gram.:adj.}
\end{itemize}
Diz-se do jôgo, em que se toma a cascarra ou algumas cartas della.
\section{Acasear}
\textunderscore v. t.\textunderscore  (e der.)
O mesmo que \textunderscore casear\textunderscore , etc.
\section{Acasmurrado}
\begin{itemize}
\item {Grp. gram.:adj.}
\end{itemize}
Que tem modos de casmurro. Cf. Filinto, III, 46.
\section{Acaso}
\begin{itemize}
\item {Grp. gram.:m.}
\end{itemize}
\begin{itemize}
\item {Grp. gram.:Adv.}
\end{itemize}
\begin{itemize}
\item {Proveniência:(De \textunderscore caso\textunderscore )}
\end{itemize}
Eventualidade, caso fortuito.
Casualmente, eventualmente.
Porventura.
\section{Acasquilhar}
\begin{itemize}
\item {Grp. gram.:v. t.}
\end{itemize}
\begin{itemize}
\item {Grp. gram.:V. p.}
\end{itemize}
Tornar casquilho, alindar.
Tornar-se casquilho, janota.
\section{Acastanhado}
\begin{itemize}
\item {Grp. gram.:adj.}
\end{itemize}
\begin{itemize}
\item {Proveniência:(De \textunderscore acastanhar\textunderscore )}
\end{itemize}
Que tem côr tirante a castanha.
\section{Acastanhar}
\begin{itemize}
\item {Grp. gram.:v. t.}
\end{itemize}
Dar côr semelhante á da castanha a.
\section{Acastelado}
\begin{itemize}
\item {Grp. gram.:v. t.}
\end{itemize}
Que tem aspecto de castello.
Fortificado como castello.
\section{Acastelamento}
\begin{itemize}
\item {Grp. gram.:m.}
\end{itemize}
Acto de acastellar.
\section{Acastelar}
\begin{itemize}
\item {Grp. gram.:v. t.}
\end{itemize}
Fortalecer com castello.
Construir á maneira de castello.
Fortificar.
\section{Acastelhanado}
\begin{itemize}
\item {Grp. gram.:adj.}
\end{itemize}
\begin{itemize}
\item {Proveniência:(De \textunderscore acastelhanar\textunderscore )}
\end{itemize}
Que tem modos de castelhano, no falar ou no vestir.
\section{Acastelhanar}
\begin{itemize}
\item {Grp. gram.:v. t.}
\end{itemize}
Dar feição de castelhano a.
\section{Acastellado}
\begin{itemize}
\item {Grp. gram.:v. t.}
\end{itemize}
Que tem aspecto de castello.
Fortificado como castello.
\section{Acastellamento}
\begin{itemize}
\item {Grp. gram.:m.}
\end{itemize}
Acto de acastellar.
\section{Acastellar}
\begin{itemize}
\item {Grp. gram.:v. t.}
\end{itemize}
Fortalecer com castello.
Construir á maneira de castello.
Fortificar.
\section{Acastos}
\begin{itemize}
\item {Grp. gram.:m. pl.}
\end{itemize}
Gênero de crustáceos cirrípedes.
\section{Acasulado}
\begin{itemize}
\item {Grp. gram.:adj.}
\end{itemize}
Que tem fórma de casulo.
\section{Acasuso}
\begin{itemize}
\item {Grp. gram.:m.  e  adj.}
\end{itemize}
\begin{itemize}
\item {Utilização:Des.}
\end{itemize}
O mesmo que \textunderscore acaso\textunderscore .
\section{Acatadamente}
\begin{itemize}
\item {Grp. gram.:adv.}
\end{itemize}
Com acatamento.
Respeitosamente.
\section{Acatador}
\begin{itemize}
\item {Grp. gram.:m.}
\end{itemize}
Aquelle que acata.
\section{Acatadura}
\begin{itemize}
\item {Grp. gram.:f.}
\end{itemize}
O mesmo que \textunderscore acatamento\textunderscore .
\section{Acatafasia}
\begin{itemize}
\item {Grp. gram.:f.}
\end{itemize}
\begin{itemize}
\item {Utilização:Med.}
\end{itemize}
\begin{itemize}
\item {Proveniência:(Do gr. \textunderscore a\textunderscore  priv. + \textunderscore kataphasis\textunderscore , affirmação)}
\end{itemize}
Impossibilidade de collocar em ordem syntáctica as palavras da phrase.
\section{Acataia}
\begin{itemize}
\item {Grp. gram.:f.}
\end{itemize}
Planta medicinal do Brasil.
\section{Acataléctico}
\begin{itemize}
\item {Grp. gram.:adj.}
\end{itemize}
\begin{itemize}
\item {Proveniência:(Gr. \textunderscore akatalektikos\textunderscore )}
\end{itemize}
Diz-se do verso grego ou latino, a que não falta nem sobeja sýllaba alguma.
\section{Acatalepsia}
\begin{itemize}
\item {Grp. gram.:f.}
\end{itemize}
\begin{itemize}
\item {Proveniência:(Gr. \textunderscore akatalepsia\textunderscore )}
\end{itemize}
Impossibilidade de comprehender.
Ausência da certeza nos conhecimentos humanos.
\section{Acataléptico}
\begin{itemize}
\item {Grp. gram.:adj.}
\end{itemize}
Relativo á \textunderscore acatalepsia\textunderscore .
\section{Acatamento}
\begin{itemize}
\item {Grp. gram.:m.}
\end{itemize}
Acto de \textunderscore acatar\textunderscore .
Respeito, veneração.
\section{Acataphasia}
\begin{itemize}
\item {Grp. gram.:f.}
\end{itemize}
\begin{itemize}
\item {Utilização:Med.}
\end{itemize}
\begin{itemize}
\item {Proveniência:(Do gr. \textunderscore a\textunderscore  priv. + \textunderscore kataphasis\textunderscore , affirmação)}
\end{itemize}
Impossibilidade de collocar em ordem syntáctica as palavras da phrase.
\section{Acatápose}
\begin{itemize}
\item {Grp. gram.:f.}
\end{itemize}
\begin{itemize}
\item {Utilização:Med.}
\end{itemize}
\begin{itemize}
\item {Proveniência:(Do gr. \textunderscore a\textunderscore  priv. + \textunderscore kataposis\textunderscore , acção de engulir)}
\end{itemize}
Dificuldade ou impossibilidade de engulir.
\section{Acatar}
\begin{itemize}
\item {Grp. gram.:v. t.}
\end{itemize}
\begin{itemize}
\item {Grp. gram.:V. i.}
\end{itemize}
\begin{itemize}
\item {Utilização:Ant.}
\end{itemize}
\begin{itemize}
\item {Proveniência:(Do lat. \textunderscore captare\textunderscore ?)}
\end{itemize}
Respeitar; venerar.
Observar, cumprir: \textunderscore acatar as leis\textunderscore .
Cuidar, vigiar.
\section{Acatarroado}
\begin{itemize}
\item {Grp. gram.:adj.}
\end{itemize}
(V. [[encatarroado|encatarroar-se]])
\section{Acatarsia}
\begin{itemize}
\item {Grp. gram.:f.}
\end{itemize}
\begin{itemize}
\item {Utilização:Med.}
\end{itemize}
\begin{itemize}
\item {Proveniência:(Gr. \textunderscore akatharsia\textunderscore )}
\end{itemize}
Impureza dos humores naturaes.
\section{Acatasolado}
\begin{itemize}
\item {fónica:tasso}
\end{itemize}
\begin{itemize}
\item {Grp. gram.:adj.}
\end{itemize}
Semelhante ao catasol.
\section{Acatástico}
\begin{itemize}
\item {Grp. gram.:adj.}
\end{itemize}
\begin{itemize}
\item {Utilização:Med.}
\end{itemize}
\begin{itemize}
\item {Proveniência:(Do gr. \textunderscore a\textunderscore  priv. + \textunderscore katasticos\textunderscore , estável)}
\end{itemize}
Diz-se das doenças, cujos phenómenos variam irregularmente.
\section{Acatável}
\begin{itemize}
\item {Grp. gram.:adj.}
\end{itemize}
Que é digno de acatamento.
Respeitável.
\section{Acatético}
\begin{itemize}
\item {Grp. gram.:adj.}
\end{itemize}
\begin{itemize}
\item {Utilização:Med.}
\end{itemize}
\begin{itemize}
\item {Proveniência:(Do gr. \textunderscore a\textunderscore  priv. + \textunderscore kathektikos\textunderscore , que retém)}
\end{itemize}
Diz-se da céllula hepática, quando incapaz de reter o pigmento biliar.
\section{Acatharsia}
\begin{itemize}
\item {Grp. gram.:f.}
\end{itemize}
\begin{itemize}
\item {Utilização:Med.}
\end{itemize}
\begin{itemize}
\item {Proveniência:(Gr. \textunderscore akatharsia\textunderscore )}
\end{itemize}
Impureza dos humores naturaes.
\section{Acathéctico}
\begin{itemize}
\item {Grp. gram.:adj.}
\end{itemize}
\begin{itemize}
\item {Utilização:Med.}
\end{itemize}
\begin{itemize}
\item {Proveniência:(Do gr. \textunderscore a\textunderscore  priv. + \textunderscore kathektikos\textunderscore , que retém)}
\end{itemize}
Diz-se da céllula hepática, quando incapaz de reter o pigmento biliar.
\section{Acathisia}
\begin{itemize}
\item {Grp. gram.:f.}
\end{itemize}
\begin{itemize}
\item {Utilização:Med.}
\end{itemize}
\begin{itemize}
\item {Proveniência:(Do gr. \textunderscore a\textunderscore  priv. + \textunderscore kathisis\textunderscore , acção de se sentar)}
\end{itemize}
Impossibilidade, que alguns doentes têm, de ficar sentados.
\section{Acathólico}
\begin{itemize}
\item {Grp. gram.:adj.}
\end{itemize}
\begin{itemize}
\item {Proveniência:(De \textunderscore a\textunderscore  priv. + \textunderscore cathólico\textunderscore )}
\end{itemize}
Diz-se do christão que não é cathólico.
\section{Acatia}
\begin{itemize}
\item {Grp. gram.:f.}
\end{itemize}
Lepidóptero nocturno.
\section{Acatingado}
\begin{itemize}
\item {Grp. gram.:adj.}
\end{itemize}
\begin{itemize}
\item {Utilização:Bras}
\end{itemize}
Que tem alguma catinga.
\section{Acatisia}
\begin{itemize}
\item {Grp. gram.:f.}
\end{itemize}
\begin{itemize}
\item {Utilização:Med.}
\end{itemize}
\begin{itemize}
\item {Proveniência:(Do gr. \textunderscore a\textunderscore  priv. + \textunderscore kathisis\textunderscore , acção de se sentar)}
\end{itemize}
Impossibilidade, que alguns doentes têm, de ficar sentados.
\section{Acatitar}
\begin{itemize}
\item {Grp. gram.:v. t.}
\end{itemize}
Tomar catita. Cf. Eça, \textunderscore P. Basílio\textunderscore , 546.
\section{Ácato}
\begin{itemize}
\item {Grp. gram.:m.}
\end{itemize}
Taça, em fórma de batel, destinada ás libações, entre os antigos.
\section{Acatólico}
\begin{itemize}
\item {Grp. gram.:adj.}
\end{itemize}
\begin{itemize}
\item {Proveniência:(De \textunderscore a\textunderscore  priv. + \textunderscore cathólico\textunderscore )}
\end{itemize}
Diz-se do christão que não é cathólico.
\section{Acauan}
\begin{itemize}
\item {fónica:ca-u}
\end{itemize}
\begin{itemize}
\item {Grp. gram.:m.}
\end{itemize}
\begin{itemize}
\item {Utilização:Bras}
\end{itemize}
\begin{itemize}
\item {Proveniência:(T. onom., der. do canto dessa ave)}
\end{itemize}
Espécie de ave de rapina, que ataca especialmente os ophídios.
\section{Acaudelar}
\begin{itemize}
\item {Grp. gram.:v. t.}
\end{itemize}
\begin{itemize}
\item {Utilização:Ant.}
\end{itemize}
O mesmo que \textunderscore acaudilhar\textunderscore .
\section{Acaudilhadamente}
\begin{itemize}
\item {Grp. gram.:adv.}
\end{itemize}
Disciplinadamente, com caudilho.
\section{Acaudilhar}
\begin{itemize}
\item {Grp. gram.:v. t.}
\end{itemize}
Sêr caudilho de, capitanear.
Dirigir.
\section{Acaule}
\begin{itemize}
\item {Grp. gram.:adj.}
\end{itemize}
\begin{itemize}
\item {Utilização:Bot.}
\end{itemize}
\begin{itemize}
\item {Proveniência:(Do gr. \textunderscore a\textunderscore  priv. + \textunderscore kaulos\textunderscore , haste)}
\end{itemize}
Diz-se das plantas, que não têm haste, ou cuja haste é tão curta, que não apparece.
\section{Acauteladamente}
\begin{itemize}
\item {Grp. gram.:adv.}
\end{itemize}
De modo acautelado, com cautela.
\section{Aca}
\begin{itemize}
\item {Grp. gram.:f.}
\end{itemize}
\begin{itemize}
\item {Utilização:T. da Índia Port}
\end{itemize}
Pensão pecuniária, vitalicia e hereditária, dada pela autoridade soberana, em compensação de serviços públicos.
\section{Acadar}
\begin{itemize}
\item {Grp. gram.:m.}
\end{itemize}
\begin{itemize}
\item {Utilização:T. da Índia Port}
\end{itemize}
Aquelle que recebeu ou frue uma acca.
\section{Acautelado}
\begin{itemize}
\item {Grp. gram.:adj.}
\end{itemize}
Que tem cautela.
Prudente.
\section{Acautelar}
\begin{itemize}
\item {Grp. gram.:v. t.}
\end{itemize}
Tratar com cautela.
Prevenir, precaver.
\section{Acava}
\begin{itemize}
\item {Grp. gram.:f.}
\end{itemize}
\begin{itemize}
\item {Utilização:Des.}
\end{itemize}
Certa porção ou determinado feixe de junco.
\section{Acavalado}
\begin{itemize}
\item {Grp. gram.:adj.}
\end{itemize}
\begin{itemize}
\item {Utilização:Bras. do N}
\end{itemize}
\begin{itemize}
\item {Proveniência:(De \textunderscore cavallo\textunderscore )}
\end{itemize}
Muito grande.
\section{Acavalar}
\begin{itemize}
\item {Grp. gram.:v. t.}
\end{itemize}
Pôr sôbre.
Amontoar.
Lançar (égua) a cavallo de cobrição.
\section{Acavaleirar}
\begin{itemize}
\item {Grp. gram.:v. t.}
\end{itemize}
Pôr a cavalleiro; amontoar.
\section{Acavaletado}
\begin{itemize}
\item {Grp. gram.:adj.}
\end{itemize}
\begin{itemize}
\item {Proveniência:(De \textunderscore cavallete\textunderscore )}
\end{itemize}
Diz-se do nariz aquilino ou arqueado.
\section{Acavallado}
\begin{itemize}
\item {Grp. gram.:adj.}
\end{itemize}
\begin{itemize}
\item {Utilização:Bras. do N}
\end{itemize}
\begin{itemize}
\item {Proveniência:(De \textunderscore cavallo\textunderscore )}
\end{itemize}
Muito grande.
\section{Acavallar}
\begin{itemize}
\item {Grp. gram.:v. t.}
\end{itemize}
Pôr sôbre.
Amontoar.
Lançar (égua) a cavallo de cobrição.
\section{Acavalleirar}
\begin{itemize}
\item {Grp. gram.:v. t.}
\end{itemize}
Pôr a cavalleiro; amontoar.
\section{Acavalletado}
\begin{itemize}
\item {Grp. gram.:adj.}
\end{itemize}
\begin{itemize}
\item {Proveniência:(De \textunderscore cavallete\textunderscore )}
\end{itemize}
Diz-se do nariz aquilino ou arqueado.
\section{Acc...}
O mesmo que \textunderscore ac...\textunderscore , excepto quando o primeiro \textunderscore c\textunderscore  determina a modulação aberta da vogal que o precede, como em \textunderscore acção\textunderscore , \textunderscore redacção\textunderscore , etc.
\section{Acca}
\begin{itemize}
\item {Grp. gram.:f.}
\end{itemize}
\begin{itemize}
\item {Utilização:T. da Índia Port}
\end{itemize}
Pensão pecuniária, vitalicia e hereditária, dada pela autoridade soberana, em compensação de serviços públicos.
\section{Accadar}
\begin{itemize}
\item {Grp. gram.:m.}
\end{itemize}
\begin{itemize}
\item {Utilização:T. da Índia Port}
\end{itemize}
Aquelle que recebeu ou frue uma acca.
\section{Acção}
\begin{itemize}
\item {Grp. gram.:f.}
\end{itemize}
\begin{itemize}
\item {Utilização:Gram.}
\end{itemize}
\begin{itemize}
\item {Proveniência:(Do lat. \textunderscore actio\textunderscore )}
\end{itemize}
Modo de actuar.
Resultado de uma fôrça phýsica ou moral: \textunderscore a acção da vontade\textunderscore .
Movimento.
Successo: \textunderscore uma acção memorável\textunderscore .
Energia traduzida em actos: \textunderscore homem de acção\textunderscore .
Combate: \textunderscore a acção do Bussaco\textunderscore .
Assumpto: \textunderscore a acção do poema\textunderscore .
Processo forense: \textunderscore uma acção commercial\textunderscore .
Título, que se passa aos que fazem parte de uma Companhia ou Sociedade commercial ou industrial.
O que um verbo exprime.
\section{Accedente}
\begin{itemize}
\item {Grp. gram.:adj.}
\end{itemize}
\begin{itemize}
\item {Proveniência:(Lat. \textunderscore accedens\textunderscore )}
\end{itemize}
Que accede.
\section{Acceder}
\begin{itemize}
\item {Grp. gram.:v. i.}
\end{itemize}
\begin{itemize}
\item {Proveniência:(Lat. \textunderscore accedere\textunderscore )}
\end{itemize}
Annuir, acquiescer.
Conformar-se.
\section{Acceitabilidade}
\begin{itemize}
\item {Grp. gram.:f.}
\end{itemize}
Qualidade daquillo que é acceitável.
\section{Acceitação}
\begin{itemize}
\item {Grp. gram.:f.}
\end{itemize}
Acto ou effeito de acceitar.
Applauso.
Bem-querença.
\section{Acceitador}
\begin{itemize}
\item {Grp. gram.:m.}
\end{itemize}
Aquelle que acceita.
\section{Acceitamento}
\begin{itemize}
\item {Grp. gram.:m.}
\end{itemize}
\begin{itemize}
\item {Utilização:Ant.}
\end{itemize}
O mesmo que \textunderscore acceitação\textunderscore .
Desafio acceitado.
Duello.
\section{Acceitante}
\begin{itemize}
\item {Grp. gram.:m.  e  adj.}
\end{itemize}
Aquelle que acceita: \textunderscore o acceitante de uma letra commercial\textunderscore .
\section{Acceitar}
\begin{itemize}
\item {Grp. gram.:v. t.}
\end{itemize}
\begin{itemize}
\item {Proveniência:(Lat. \textunderscore acceptare\textunderscore )}
\end{itemize}
Receber (o que se offerece): \textunderscore acceitar um presente\textunderscore .
Admittir: \textunderscore acceitar uma satisfação\textunderscore .
\textunderscore Acceitar uma letra de câmbio\textunderscore , obrigar-se, por escrito na mesma letra, a pagá-la no dia em que se vencer.
\section{Acceitável}
\begin{itemize}
\item {Grp. gram.:adj.}
\end{itemize}
Que se póde acceitar.
\section{Acceite}
\begin{itemize}
\item {Grp. gram.:m.}
\end{itemize}
Acto ou assinatura, com que se acceita uma letra de câmbio.
\section{Acceito}
\begin{itemize}
\item {Grp. gram.:adj.}
\end{itemize}
Bem recebido, bemquisto.
\section{Acceleração}
\begin{itemize}
\item {Grp. gram.:f.}
\end{itemize}
\begin{itemize}
\item {Proveniência:(Lat. \textunderscore acceleratio\textunderscore )}
\end{itemize}
Augmento de velocidade, pressa.
Execução rápida, diligência.
Acto de accelerar.
\section{Acceleradamente}
\begin{itemize}
\item {Grp. gram.:adv.}
\end{itemize}
De modo \textunderscore accelerado\textunderscore .
\section{Accelerado}
\begin{itemize}
\item {Grp. gram.:adj.}
\end{itemize}
\begin{itemize}
\item {Proveniência:(De \textunderscore accelerar\textunderscore )}
\end{itemize}
O mesmo que \textunderscore rápido\textunderscore .
\section{Accelerador}
\begin{itemize}
\item {Grp. gram.:adj.}
\end{itemize}
Que accelera.
\section{Acceleramento}
\begin{itemize}
\item {Grp. gram.:m.}
\end{itemize}
(V.acceleração)
\section{Accelerar}
\begin{itemize}
\item {Grp. gram.:v. t.}
\end{itemize}
\begin{itemize}
\item {Proveniência:(Lat. \textunderscore accelerare\textunderscore )}
\end{itemize}
Tornar célere.
Aumentar a velocidade de.
Appressar.
Instigar.
\section{Accelerativo}
\begin{itemize}
\item {Grp. gram.:adj.}
\end{itemize}
Que produz acceleração.
\section{Acceleratriz}
\begin{itemize}
\item {Grp. gram.:adj.}
\end{itemize}
(fem. de \textunderscore accelerador\textunderscore )
\section{Accendalha}
\begin{itemize}
\item {Grp. gram.:f.}
\end{itemize}
\begin{itemize}
\item {Proveniência:(De \textunderscore accender\textunderscore )}
\end{itemize}
Substância combustível, com que se ateia o lume, (aparas, cavacos, carqueja, etc.)
\section{Accendedalha}
\begin{itemize}
\item {Grp. gram.:f.}
\end{itemize}
O mesmo que \textunderscore acendalha\textunderscore . Cf. Amador Arráiz.
\section{Accendedor}
\begin{itemize}
\item {Grp. gram.:m.}
\end{itemize}
O que accende.
\section{Accender}
\begin{itemize}
\item {Grp. gram.:v. t.}
\end{itemize}
\begin{itemize}
\item {Utilização:Fig.}
\end{itemize}
\begin{itemize}
\item {Proveniência:(Lat. \textunderscore accendere\textunderscore )}
\end{itemize}
Fazer arder: \textunderscore accender uma vela\textunderscore .
Pôr fogo a: \textunderscore accender a lenha\textunderscore .
Atear.
Illuminar.
Enthusiasmar.
Estimular: \textunderscore accender os brios\textunderscore .
\section{Accendidamente}
\begin{itemize}
\item {Grp. gram.:adv.}
\end{itemize}
\begin{itemize}
\item {Utilização:Fig.}
\end{itemize}
Com enthusiasmo, com excitação.
\section{Accendimento}
\begin{itemize}
\item {Grp. gram.:m.}
\end{itemize}
Acto de \textunderscore accender\textunderscore .
\section{Accendível}
\begin{itemize}
\item {Grp. gram.:adj.}
\end{itemize}
Que se póde accender.
\section{Accenso}
\begin{itemize}
\item {Grp. gram.:m.}
\end{itemize}
\begin{itemize}
\item {Proveniência:(Lat. \textunderscore accensus\textunderscore )}
\end{itemize}
Antigo official subalterno, adjunto a qualquer alto funccionário romano.
\section{Accento}
\begin{itemize}
\item {Grp. gram.:m.}
\end{itemize}
\begin{itemize}
\item {Proveniência:(Lat. \textunderscore accentus\textunderscore )}
\end{itemize}
Inflexão da voz, na pronúncia das palavras.
Sinal, com que na escrita se mostra a figura das vogaes.
Tom de voz, timbre.
Há o accento \textunderscore expiratório\textunderscore , que é o que se faz com expiração enérgica, como no português e noutras línguas; e há o \textunderscore musical\textunderscore , que é independente do tónico, como no sueco, em que a sýllaba predominante de uma palavra é menos alta que outra.
\section{Accentuação}
\begin{itemize}
\item {Grp. gram.:f.}
\end{itemize}
Acto de \textunderscore accentuar\textunderscore .
Modo de accentuar, na pronúncia ou na escrita.
\section{Accentuar}
\begin{itemize}
\item {Grp. gram.:v. t.}
\end{itemize}
\begin{itemize}
\item {Utilização:Fig.}
\end{itemize}
\begin{itemize}
\item {Proveniência:(De \textunderscore accento\textunderscore )}
\end{itemize}
Empregar accentos ortográphicos em.
Dar (ás vogaes) determinada modulação phonética.
Pronunciar claramente.
Dar relêvo a, tornar saliente.
Exprimir com vigor: \textunderscore accentuar uma offensa\textunderscore .
\section{Accepção}
\begin{itemize}
\item {Grp. gram.:f.}
\end{itemize}
\begin{itemize}
\item {Proveniência:(Lat. \textunderscore acceptio\textunderscore )}
\end{itemize}
Interpretação.
Sentido, em que se toma uma palavra.
\section{Acceptar}
\textunderscore v. t.\textunderscore  (e der.)
Fórma ant. de \textunderscore acceitar\textunderscore , etc.
\section{Acceptilação}
\begin{itemize}
\item {Grp. gram.:f.}
\end{itemize}
\begin{itemize}
\item {Utilização:Des.}
\end{itemize}
\begin{itemize}
\item {Proveniência:(Lat. \textunderscore acceptilatio\textunderscore )}
\end{itemize}
Acto, em que um crèdor dá quitação a um devedor, sem que êste pague a dívida.
\section{Acceptuar}
\begin{itemize}
\item {Grp. gram.:v. i.}
\end{itemize}
\begin{itemize}
\item {Utilização:Prov.}
\end{itemize}
\begin{itemize}
\item {Utilização:trasm.}
\end{itemize}
Fazer combinação.
Combinar os meios ou o modo de fazer alguma coisa.
(Cp. \textunderscore acceptar\textunderscore )
\section{Accesamente}
\begin{itemize}
\item {Grp. gram.:adv.}
\end{itemize}
O mesmo que \textunderscore accendidamente\textunderscore .
\section{Acceso}
\begin{itemize}
\item {Grp. gram.:adj.}
\end{itemize}
\begin{itemize}
\item {Utilização:Fig.}
\end{itemize}
\begin{itemize}
\item {Proveniência:(Lat. \textunderscore accensus\textunderscore )}
\end{itemize}
Que se accendeu.
Inflammado.
Excitado.
\section{Accessão}
\begin{itemize}
\item {Grp. gram.:f.}
\end{itemize}
\begin{itemize}
\item {Proveniência:(Lat. \textunderscore accessio\textunderscore )}
\end{itemize}
Acto de \textunderscore acceder\textunderscore .
Accessório.
\section{Accessibilidade}
\begin{itemize}
\item {Grp. gram.:f.}
\end{itemize}
\begin{itemize}
\item {Proveniência:(Do b. lat. \textunderscore accessibilis\textunderscore )}
\end{itemize}
Facilidade na aproximação.
\section{Accessível}
\begin{itemize}
\item {Grp. gram.:adj.}
\end{itemize}
Que se póde possuir.
A que se póde chegar.
Tratável, lhano.
(B. lat. \textunderscore accessibilis\textunderscore )
\section{Accesso}
\begin{itemize}
\item {Grp. gram.:m.}
\end{itemize}
\begin{itemize}
\item {Proveniência:(Lat. \textunderscore accessus\textunderscore )}
\end{itemize}
Chegada.
Aproximação.
Trato.
Phenómeno pathológico que, a espaços, cessa e recrudesce.
\section{Accessoriamente}
\begin{itemize}
\item {Grp. gram.:adv.}
\end{itemize}
De modo accessório.
\section{Accessório}
\begin{itemize}
\item {Grp. gram.:adj.}
\end{itemize}
\begin{itemize}
\item {Proveniência:(De \textunderscore acceder\textunderscore )}
\end{itemize}
Que está junto a alguma coisa, sem della fazer parte integrante.
\section{Acciano}
\begin{itemize}
\item {Grp. gram.:adj.}
\end{itemize}
Diz-se dos jogos públicos, instituídos por César, em memória da batalha de Áccio. Cf. Castilho, \textunderscore Fastos\textunderscore , I, 606.
\section{Accidentação}
\begin{itemize}
\item {Grp. gram.:f.}
\end{itemize}
\begin{itemize}
\item {Proveniência:(De \textunderscore accidentar\textunderscore )}
\end{itemize}
Qualidade de um terreno accidentado.
\section{Accidentado}
\begin{itemize}
\item {Grp. gram.:adj.}
\end{itemize}
Que não é plano: \textunderscore região accidentada\textunderscore .
\section{Accidental}
\begin{itemize}
\item {Grp. gram.:adj.}
\end{itemize}
\begin{itemize}
\item {Utilização:Mús.}
\end{itemize}
\begin{itemize}
\item {Proveniência:(De \textunderscore accidente\textunderscore )}
\end{itemize}
Fortuito, casual, imprevisto.
Acessório.
Diz-se de cada uma das notas, que não fazem parte integrante dos acordes.
\section{Accidentalmente}
\begin{itemize}
\item {Grp. gram.:adv.}
\end{itemize}
De modo \textunderscore accidental\textunderscore .
\section{Accidentar}
\begin{itemize}
\item {Grp. gram.:v. t.}
\end{itemize}
Variar.
Produzir accidente em.
\section{Accidentariamente}
\begin{itemize}
\item {Grp. gram.:adv.}
\end{itemize}
De modo \textunderscore accidentário\textunderscore .
\section{Accidentário}
\begin{itemize}
\item {Grp. gram.:adj.}
\end{itemize}
O mesmo que \textunderscore accidental\textunderscore .
\section{Accidentável}
\begin{itemize}
\item {Grp. gram.:adj.}
\end{itemize}
Que se póde \textunderscore accidentar\textunderscore .
\section{Accidente}
\begin{itemize}
\item {Grp. gram.:m.}
\end{itemize}
\begin{itemize}
\item {Utilização:Pop.}
\end{itemize}
\begin{itemize}
\item {Proveniência:(Lat. \textunderscore accideno\textunderscore )}
\end{itemize}
O que é casual, fortuito.
Desgraça.
Disposição variada de terreno.
Variada distribuição de luz.
Ataque epiléptico; sýncope, desmaio.
\section{Accingir}
\begin{itemize}
\item {Grp. gram.:v. t.}
\end{itemize}
\begin{itemize}
\item {Utilização:Des.}
\end{itemize}
\begin{itemize}
\item {Proveniência:(Lat. \textunderscore accingere\textunderscore )}
\end{itemize}
O mesmo que \textunderscore cingir\textunderscore  (especialmente falando de armas). Cf. Fr. Fort., \textunderscore Inéd.\textunderscore , II, 62.
\section{Accionado}
\begin{itemize}
\item {Grp. gram.:m.}
\end{itemize}
\begin{itemize}
\item {Proveniência:(De \textunderscore accionar\textunderscore )}
\end{itemize}
Gestos.
\section{Accionador}
\begin{itemize}
\item {Grp. gram.:m.}
\end{itemize}
O que acciona.
\section{Accional}
\begin{itemize}
\item {Grp. gram.:adj.}
\end{itemize}
\begin{itemize}
\item {Utilização:Neol.}
\end{itemize}
\begin{itemize}
\item {Proveniência:(Do lat. \textunderscore actio, actionis\textunderscore )}
\end{itemize}
Relativo a acção.
\section{Accionar}
\begin{itemize}
\item {Grp. gram.:v. t.}
\end{itemize}
\begin{itemize}
\item {Grp. gram.:V. i.}
\end{itemize}
\begin{itemize}
\item {Proveniência:(Do lat. \textunderscore actio, actionis\textunderscore )}
\end{itemize}
Demandar em juizo.
Gesticular.
\section{Accionário}
\begin{itemize}
\item {Grp. gram.:m.}
\end{itemize}
(V.accionista)
\section{Accionista}
\begin{itemize}
\item {Grp. gram.:m.}
\end{itemize}
\begin{itemize}
\item {Proveniência:(Do lat. \textunderscore actio, actionis\textunderscore )}
\end{itemize}
O que tem acções de Companhia industrial ou commercial.
\section{Accipitrário}
\begin{itemize}
\item {Grp. gram.:m.}
\end{itemize}
\begin{itemize}
\item {Proveniência:(Do lat. \textunderscore accipiter\textunderscore )}
\end{itemize}
Armadilha para aves de rapina.
\section{Accípitres}
\begin{itemize}
\item {Grp. gram.:m. pl.}
\end{itemize}
\begin{itemize}
\item {Proveniência:(Lat. \textunderscore accipiter\textunderscore )}
\end{itemize}
Primeiro grupo das aves de rapina, segundo Cuvier.
\section{Accipitrianos}
\begin{itemize}
\item {Grp. gram.:m. pl.}
\end{itemize}
\begin{itemize}
\item {Proveniência:(Do lat. \textunderscore accipiter\textunderscore )}
\end{itemize}
Sub-família dos accipitrídeos.
\section{Accipitrídeos}
\begin{itemize}
\item {Grp. gram.:m. pl.}
\end{itemize}
\begin{itemize}
\item {Proveniência:(Do lat. \textunderscore accipiter\textunderscore  + gr. \textunderscore eidos\textunderscore )}
\end{itemize}
O mesmo que falconídeos.
\section{Accipitrino}
\begin{itemize}
\item {Grp. gram.:adj.}
\end{itemize}
\begin{itemize}
\item {Proveniência:(Do lat. \textunderscore accipiter\textunderscore )}
\end{itemize}
Relativo a aves de rapina.
\section{Acclamação}
\begin{itemize}
\item {Grp. gram.:f.}
\end{itemize}
\begin{itemize}
\item {Proveniência:(Lat. \textunderscore acclamatio\textunderscore )}
\end{itemize}
Acto de acclamar.
\section{Acclamador}
\begin{itemize}
\item {Grp. gram.:m.}
\end{itemize}
\begin{itemize}
\item {Proveniência:(De \textunderscore acclamar\textunderscore )}
\end{itemize}
O que acclama.
\section{Acclamar}
\begin{itemize}
\item {Grp. gram.:v. t.}
\end{itemize}
\begin{itemize}
\item {Proveniência:(Lat. \textunderscore acclamare\textunderscore )}
\end{itemize}
Applaudir ou approvar, bradando.
Saudar.
Proclamar, eleger por acclamação.
\section{Acclive}
\begin{itemize}
\item {Grp. gram.:m.}
\end{itemize}
\begin{itemize}
\item {Grp. gram.:Adj.}
\end{itemize}
\begin{itemize}
\item {Proveniência:(Lat. \textunderscore acclivis\textunderscore )}
\end{itemize}
Ladeira, declive.
Íngreme.
\section{Accommodação}
\begin{itemize}
\item {Grp. gram.:f.}
\end{itemize}
Acto ou effeito de \textunderscore accommodar\textunderscore .
\section{Accommodadamente}
\begin{itemize}
\item {Grp. gram.:adv.}
\end{itemize}
De modo \textunderscore accommodado\textunderscore .
\section{Accommodadiço}
\begin{itemize}
\item {Grp. gram.:adj.}
\end{itemize}
O mesmo que \textunderscore accommodatício\textunderscore .
\section{Accommodamento}
\begin{itemize}
\item {Grp. gram.:m.}
\end{itemize}
(V.accommodação)
\section{Accommodar}
\begin{itemize}
\item {Grp. gram.:v. t.}
\end{itemize}
\begin{itemize}
\item {Proveniência:(Lat. \textunderscore accommodare\textunderscore )}
\end{itemize}
Tornar cômmodo.
Adequar.
Arrumar, pôr em ordem: \textunderscore accommodar os livros\textunderscore .
Empregar, dar posição a.
Applicar.
Habituar.
Sossegar.
Hospedar.
\section{Accommodatício}
\begin{itemize}
\item {Grp. gram.:adj.}
\end{itemize}
Que se accommoda facilmente.
\section{Accommodável}
\begin{itemize}
\item {Grp. gram.:adj.}
\end{itemize}
Que se póde \textunderscore accommodar\textunderscore .
\section{Accômodo}
\begin{itemize}
\item {Grp. gram.:adj.}
\end{itemize}
\begin{itemize}
\item {Utilização:Ant.}
\end{itemize}
Opportuno, cômmodo.
\section{Accorredor}
\begin{itemize}
\item {Grp. gram.:adj.}
\end{itemize}
Que vem em auxílio. Cf. Filinto, \textunderscore D. Man.\textunderscore  II, 222.
\section{Accorrer}
\begin{itemize}
\item {Grp. gram.:v. i.}
\end{itemize}
\begin{itemize}
\item {Proveniência:(Lat. \textunderscore accurrere\textunderscore )}
\end{itemize}
Ir em auxílio; acudir.
\section{Accorrimento}
\begin{itemize}
\item {Grp. gram.:m.}
\end{itemize}
\begin{itemize}
\item {Utilização:Ant.}
\end{itemize}
\begin{itemize}
\item {Proveniência:(De \textunderscore accorrer\textunderscore )}
\end{itemize}
Soccorro, auxílio.
\section{Accorro}
\begin{itemize}
\item {Grp. gram.:m.}
\end{itemize}
\begin{itemize}
\item {Proveniência:(De \textunderscore accorrer\textunderscore )}
\end{itemize}
O mesmo que \textunderscore soccorro\textunderscore .
\section{Accrescentada}
\begin{itemize}
\item {Grp. gram.:adj. f.}
\end{itemize}
\begin{itemize}
\item {Utilização:Prov.}
\end{itemize}
\begin{itemize}
\item {Proveniência:(De \textunderscore accrescentar\textunderscore )}
\end{itemize}
Diz-se da mulher grávida. (Colhido em Turquel)
\section{Accrescentador}
\begin{itemize}
\item {Grp. gram.:m.  e  adj.}
\end{itemize}
O que accrescenta.
\section{Accrescentamento}
\begin{itemize}
\item {Grp. gram.:m.}
\end{itemize}
Acto ou effeito de \textunderscore accrescentar\textunderscore .
\section{Accrescentar}
\begin{itemize}
\item {Grp. gram.:v. t.}
\end{itemize}
\begin{itemize}
\item {Proveniência:(De \textunderscore accrescer\textunderscore )}
\end{itemize}
Tornar maior, aumentar.
Dar mais grandeza, fôrça ou número a.
\section{Accrescente}
\begin{itemize}
\item {Grp. gram.:m.}
\end{itemize}
\begin{itemize}
\item {Utilização:Pop.}
\end{itemize}
Acto de \textunderscore accrescentar\textunderscore .
Accrescentamento.
O mesmo que \textunderscore chinó\textunderscore .
\section{Accrescento}
\begin{itemize}
\item {Grp. gram.:m.}
\end{itemize}
O mesmo que \textunderscore accrescentamento\textunderscore . Cf. Castilho, \textunderscore Avarento\textunderscore , 144.
\section{Accrescer}
\begin{itemize}
\item {Grp. gram.:v. i.}
\end{itemize}
\begin{itemize}
\item {Grp. gram.:V. t.}
\end{itemize}
\begin{itemize}
\item {Proveniência:(Lat. \textunderscore accrescere\textunderscore )}
\end{itemize}
Ajuntar-se.
Sobrevir.
Juntar.
Aumentar.
\section{Accrescido}
\begin{itemize}
\item {Grp. gram.:m.}
\end{itemize}
Aquillo que accresceu.
Annexo, accessório.
Dependência:«\textunderscore uma lei de Affonso III sobre os accrescidos dos rios.\textunderscore »\textunderscore Port. Mon. Hist.\textunderscore , I, 149.
\section{Accrescimento}
\begin{itemize}
\item {Grp. gram.:m.}
\end{itemize}
Acto ou effeito de \textunderscore accrescer\textunderscore .
\section{Accréscimo}
\begin{itemize}
\item {Grp. gram.:m.}
\end{itemize}
O mesmo que \textunderscore accrescimento\textunderscore , febre intermittente.
\section{Accúbito}
\begin{itemize}
\item {Grp. gram.:m.}
\end{itemize}
\begin{itemize}
\item {Proveniência:(Lat. \textunderscore accubilum\textunderscore )}
\end{itemize}
Cadeira-leito, espécie de canapé, em que os Romanos se sentavam á mesa.
\section{Accumbente}
\begin{itemize}
\item {Grp. gram.:adj.}
\end{itemize}
\begin{itemize}
\item {Proveniência:(Lat. \textunderscore accumbens\textunderscore )}
\end{itemize}
Diz-se da radicula das plantas crucíferas, quando curvada na borda dos cotylédones.
\section{Accumulação}
\begin{itemize}
\item {Grp. gram.:f.}
\end{itemize}
\begin{itemize}
\item {Proveniência:(Lat. \textunderscore accumulatio\textunderscore )}
\end{itemize}
Acto ou effeito de accumular.
\section{Accumuladamente}
\begin{itemize}
\item {Grp. gram.:adv.}
\end{itemize}
Com accumulação.
\section{Accumulador}
\begin{itemize}
\item {Grp. gram.:m.  e  adj.}
\end{itemize}
\begin{itemize}
\item {Utilização:Phýs.}
\end{itemize}
O que accumula.
Máquina, que armazena a fôrça, para a restituir, quando necessária.
Máquina, que armazena o potencial eléctrico.
\section{Accumulamento}
\begin{itemize}
\item {Grp. gram.:m.}
\end{itemize}
(V.accumulação)
\section{Accumular}
\begin{itemize}
\item {Grp. gram.:v. t.}
\end{itemize}
\begin{itemize}
\item {Grp. gram.:V. p.}
\end{itemize}
\begin{itemize}
\item {Utilização:Fig.}
\end{itemize}
\begin{itemize}
\item {Proveniência:(Lat. \textunderscore accumulare\textunderscore )}
\end{itemize}
Amontoar, pôr em cúmulos.
Reunir em si (várias funcções ou occupações).
Succeder-se, sobrevir.
\section{Accumulativamente}
\begin{itemize}
\item {Grp. gram.:adv.}
\end{itemize}
De modo \textunderscore accumulativo\textunderscore ; conjuntamente.
\section{Accumulativo}
\begin{itemize}
\item {Grp. gram.:adj.}
\end{itemize}
Que se póde \textunderscore accumular\textunderscore .
\section{Accúmulo}
\begin{itemize}
\item {Grp. gram.:m.}
\end{itemize}
\begin{itemize}
\item {Utilização:Neol.}
\end{itemize}
O mesmo que \textunderscore accumulação\textunderscore .
\section{Accuradamente}
\begin{itemize}
\item {Grp. gram.:adv.}
\end{itemize}
Cuidadosamente.
\section{Accurar}
\begin{itemize}
\item {Grp. gram.:v. t.}
\end{itemize}
\begin{itemize}
\item {Proveniência:(Lat. \textunderscore accurare\textunderscore )}
\end{itemize}
Tratar com cuidado, com desvelo.
\section{Accusação}
\begin{itemize}
\item {Grp. gram.:f.}
\end{itemize}
\begin{itemize}
\item {Proveniência:(Lat. \textunderscore Accusatio\textunderscore )}
\end{itemize}
Acto ou effeito de accusar.
\section{Accusado}
\begin{itemize}
\item {Grp. gram.:m.}
\end{itemize}
\begin{itemize}
\item {Proveniência:(De \textunderscore accusar\textunderscore )}
\end{itemize}
Aquelle a quem se imputa delito ou crime.
\section{Accusador}
\begin{itemize}
\item {Grp. gram.:m.  e  adj.}
\end{itemize}
\begin{itemize}
\item {Proveniência:(Lat. \textunderscore accusator\textunderscore )}
\end{itemize}
O que accusa.
\section{Acedente}
\begin{itemize}
\item {Grp. gram.:adj.}
\end{itemize}
\begin{itemize}
\item {Proveniência:(Lat. \textunderscore accedens\textunderscore )}
\end{itemize}
Que accede.
\section{Aceder}
\begin{itemize}
\item {Grp. gram.:v. i.}
\end{itemize}
\begin{itemize}
\item {Proveniência:(Lat. \textunderscore accedere\textunderscore )}
\end{itemize}
Annuir, acquiescer.
Conformar-se.
\section{Aceitabilidade}
\begin{itemize}
\item {Grp. gram.:f.}
\end{itemize}
Qualidade daquillo que é acceitável.
\section{Aceitação}
\begin{itemize}
\item {Grp. gram.:f.}
\end{itemize}
Acto ou effeito de acceitar.
Applauso.
Bem-querença.
\section{Aceitador}
\begin{itemize}
\item {Grp. gram.:m.}
\end{itemize}
Aquelle que acceita.
\section{Aceitamento}
\begin{itemize}
\item {Grp. gram.:m.}
\end{itemize}
\begin{itemize}
\item {Utilização:Ant.}
\end{itemize}
O mesmo que \textunderscore acceitação\textunderscore .
Desafio acceitado.
Duello.
\section{Aceitante}
\begin{itemize}
\item {Grp. gram.:m.  e  adj.}
\end{itemize}
Aquelle que acceita: \textunderscore o acceitante de uma letra commercial\textunderscore .
\section{Aceitar}
\begin{itemize}
\item {Grp. gram.:v. t.}
\end{itemize}
\begin{itemize}
\item {Proveniência:(Lat. \textunderscore acceptare\textunderscore )}
\end{itemize}
Receber (o que se offerece): \textunderscore acceitar um presente\textunderscore .
Admittir: \textunderscore acceitar uma satisfação\textunderscore .
\textunderscore Acceitar uma letra de câmbio\textunderscore , obrigar-se, por escrito na mesma letra, a pagá-la no dia em que se vencer.
\section{Aceitável}
\begin{itemize}
\item {Grp. gram.:adj.}
\end{itemize}
Que se póde acceitar.
\section{Aceite}
\begin{itemize}
\item {Grp. gram.:m.}
\end{itemize}
Acto ou assinatura, com que se acceita uma letra de câmbio.
\section{Aceito}
\begin{itemize}
\item {Grp. gram.:adj.}
\end{itemize}
Bem recebido, bemquisto.
\section{Aceleração}
\begin{itemize}
\item {Grp. gram.:f.}
\end{itemize}
\begin{itemize}
\item {Proveniência:(Lat. \textunderscore acceleratio\textunderscore )}
\end{itemize}
Augmento de velocidade, pressa.
Execução rápida, diligência.
Acto de accelerar.
\section{Aceleradamente}
\begin{itemize}
\item {Grp. gram.:adv.}
\end{itemize}
De modo \textunderscore accelerado\textunderscore .
\section{Acelerado}
\begin{itemize}
\item {Grp. gram.:adj.}
\end{itemize}
\begin{itemize}
\item {Proveniência:(De \textunderscore accelerar\textunderscore )}
\end{itemize}
O mesmo que \textunderscore rápido\textunderscore .
\section{Acelerador}
\begin{itemize}
\item {Grp. gram.:adj.}
\end{itemize}
Que accelera.
\section{Aceleramento}
\begin{itemize}
\item {Grp. gram.:m.}
\end{itemize}
(V.acceleração)
\section{Acelerar}
\begin{itemize}
\item {Grp. gram.:v. t.}
\end{itemize}
\begin{itemize}
\item {Proveniência:(Lat. \textunderscore accelerare\textunderscore )}
\end{itemize}
Tornar célere.
Aumentar a velocidade de.
Appressar.
Instigar.
\section{Acelerativo}
\begin{itemize}
\item {Grp. gram.:adj.}
\end{itemize}
Que produz acceleração.
\section{Aceleratriz}
\begin{itemize}
\item {Grp. gram.:adj.}
\end{itemize}
(fem. de \textunderscore accelerador\textunderscore )
\section{Acendalha}
\begin{itemize}
\item {Grp. gram.:f.}
\end{itemize}
\begin{itemize}
\item {Proveniência:(De \textunderscore accender\textunderscore )}
\end{itemize}
Substância combustível, com que se ateia o lume, (aparas, cavacos, carqueja, etc.)
\section{Acendedalha}
\begin{itemize}
\item {Grp. gram.:f.}
\end{itemize}
O mesmo que \textunderscore acendalha\textunderscore . Cf. Amador Arráiz.
\section{Acendedor}
\begin{itemize}
\item {Grp. gram.:m.}
\end{itemize}
O que accende.
\section{Acender}
\begin{itemize}
\item {Grp. gram.:v. t.}
\end{itemize}
\begin{itemize}
\item {Utilização:Fig.}
\end{itemize}
\begin{itemize}
\item {Proveniência:(Lat. \textunderscore accendere\textunderscore )}
\end{itemize}
Fazer arder: \textunderscore accender uma vela\textunderscore .
Pôr fogo a: \textunderscore accender a lenha\textunderscore .
Atear.
Illuminar.
Enthusiasmar.
Estimular: \textunderscore accender os brios\textunderscore .
\section{Acendidamente}
\begin{itemize}
\item {Grp. gram.:adv.}
\end{itemize}
\begin{itemize}
\item {Utilização:Fig.}
\end{itemize}
Com enthusiasmo, com excitação.
\section{Acendimento}
\begin{itemize}
\item {Grp. gram.:m.}
\end{itemize}
Acto de \textunderscore accender\textunderscore .
\section{Acendível}
\begin{itemize}
\item {Grp. gram.:adj.}
\end{itemize}
Que se póde accender.
\section{Acenso}
\begin{itemize}
\item {Grp. gram.:m.}
\end{itemize}
\begin{itemize}
\item {Proveniência:(Lat. \textunderscore accensus\textunderscore )}
\end{itemize}
Antigo official subalterno, adjunto a qualquer alto funccionário romano.
\section{Acento}
\begin{itemize}
\item {Grp. gram.:m.}
\end{itemize}
\begin{itemize}
\item {Proveniência:(Lat. \textunderscore accentus\textunderscore )}
\end{itemize}
Inflexão da voz, na pronúncia das palavras.
Sinal, com que na escrita se mostra a figura das vogaes.
Tom de voz, timbre.
Há o accento \textunderscore expiratório\textunderscore , que é o que se faz com expiração enérgica, como no português e noutras línguas; e há o \textunderscore musical\textunderscore , que é independente do tónico, como no sueco, em que a sýllaba predominante de uma palavra é menos alta que outra.
\section{Acentuação}
\begin{itemize}
\item {Grp. gram.:f.}
\end{itemize}
Acto de \textunderscore accentuar\textunderscore .
Modo de accentuar, na pronúncia ou na escrita.
\section{Acentuar}
\begin{itemize}
\item {Grp. gram.:v. t.}
\end{itemize}
\begin{itemize}
\item {Utilização:Fig.}
\end{itemize}
\begin{itemize}
\item {Proveniência:(De \textunderscore accento\textunderscore )}
\end{itemize}
Empregar accentos ortográphicos em.
Dar (ás vogaes) determinada modulação phonética.
Pronunciar claramente.
Dar relêvo a, tornar saliente.
Exprimir com vigor: \textunderscore accentuar uma offensa\textunderscore .
\section{Acepção}
\begin{itemize}
\item {Grp. gram.:f.}
\end{itemize}
\begin{itemize}
\item {Proveniência:(Lat. \textunderscore acceptio\textunderscore )}
\end{itemize}
Interpretação.
Sentido, em que se toma uma palavra.
\section{Aceptilação}
\begin{itemize}
\item {Grp. gram.:f.}
\end{itemize}
\begin{itemize}
\item {Utilização:Des.}
\end{itemize}
\begin{itemize}
\item {Proveniência:(Lat. \textunderscore acceptilatio\textunderscore )}
\end{itemize}
Acto, em que um crèdor dá quitação a um devedor, sem que êste pague a dívida.
\section{Aceptuar}
\begin{itemize}
\item {Grp. gram.:v. i.}
\end{itemize}
\begin{itemize}
\item {Utilização:Prov.}
\end{itemize}
\begin{itemize}
\item {Utilização:trasm.}
\end{itemize}
Fazer combinação.
Combinar os meios ou o modo de fazer alguma coisa.
(Cp. \textunderscore acceptar\textunderscore )
\section{Acesamente}
\begin{itemize}
\item {Grp. gram.:adv.}
\end{itemize}
O mesmo que \textunderscore accendidamente\textunderscore .
\section{Aceso}
\begin{itemize}
\item {Grp. gram.:adj.}
\end{itemize}
\begin{itemize}
\item {Utilização:Fig.}
\end{itemize}
\begin{itemize}
\item {Proveniência:(Lat. \textunderscore accensus\textunderscore )}
\end{itemize}
Que se accendeu.
Inflammado.
Excitado.
\section{Acessão}
\begin{itemize}
\item {Grp. gram.:f.}
\end{itemize}
\begin{itemize}
\item {Proveniência:(Lat. \textunderscore accessio\textunderscore )}
\end{itemize}
Acto de \textunderscore acceder\textunderscore .
Accessório.
\section{Acessibilidade}
\begin{itemize}
\item {Grp. gram.:f.}
\end{itemize}
\begin{itemize}
\item {Proveniência:(Do b. lat. \textunderscore accessibilis\textunderscore )}
\end{itemize}
Facilidade na aproximação.
\section{Acessível}
\begin{itemize}
\item {Grp. gram.:adj.}
\end{itemize}
Que se póde possuir.
A que se póde chegar.
Tratável, lhano.
(B. lat. \textunderscore accessibilis\textunderscore )
\section{Acesso}
\begin{itemize}
\item {Grp. gram.:m.}
\end{itemize}
\begin{itemize}
\item {Proveniência:(Lat. \textunderscore accessus\textunderscore )}
\end{itemize}
Chegada.
Aproximação.
Trato.
Phenómeno pathológico que, a espaços, cessa e recrudesce.
\section{Acessoriamente}
\begin{itemize}
\item {Grp. gram.:adv.}
\end{itemize}
De modo accessório.
\section{Acessório}
\begin{itemize}
\item {Grp. gram.:adj.}
\end{itemize}
\begin{itemize}
\item {Proveniência:(De \textunderscore acceder\textunderscore )}
\end{itemize}
Que está junto a alguma coisa, sem della fazer parte integrante.
\section{Accusamento}
\begin{itemize}
\item {Grp. gram.:m.}
\end{itemize}
O mesmo que \textunderscore accusação\textunderscore .
\section{Accusante}
\begin{itemize}
\item {Grp. gram.:m.}
\end{itemize}
Aquelle que accusa.
\section{Accusar}
\begin{itemize}
\item {Grp. gram.:v. t.}
\end{itemize}
\begin{itemize}
\item {Proveniência:(Lat. \textunderscore accusare\textunderscore )}
\end{itemize}
Imputar falta ou crime a.
Notificar: \textunderscore accusar a recepção de uma carta\textunderscore .
Mostrar.
Confessar: \textunderscore accusar os próprios defeitos\textunderscore .
\section{Accusativo}
\begin{itemize}
\item {Grp. gram.:adj.}
\end{itemize}
\begin{itemize}
\item {Grp. gram.:M.}
\end{itemize}
\begin{itemize}
\item {Utilização:Gram.}
\end{itemize}
\begin{itemize}
\item {Proveniência:(Lat. \textunderscore accusativus\textunderscore )}
\end{itemize}
Que serve para accusar.
Caso que, na declinação dos nomes latinos e gregos, designa principalmente o regime directo.
\section{Accusatoriamente}
\begin{itemize}
\item {Grp. gram.:adv.}
\end{itemize}
Do modo \textunderscore accusatório\textunderscore .
\section{Accusatório}
\begin{itemize}
\item {Grp. gram.:adj.}
\end{itemize}
\begin{itemize}
\item {Proveniência:(Lat. \textunderscore accusatorius\textunderscore )}
\end{itemize}
Relativo á accusação.
\section{Accusável}
\begin{itemize}
\item {Grp. gram.:adj.}
\end{itemize}
\begin{itemize}
\item {Proveniência:(Lat. \textunderscore accusabilis\textunderscore )}
\end{itemize}
Que póde ou deve sêr accusado.
\section{Accuse}
\begin{itemize}
\item {Grp. gram.:m.}
\end{itemize}
(V.accuso)
\section{Accuso}
\begin{itemize}
\item {Grp. gram.:m.}
\end{itemize}
O mesmo que \textunderscore accusação\textunderscore .
Designa a declaração que, no jôgo da bisca, o parceiro faz, de têr reunido, entre as suas cartas, duas figuras do mesmo valor.
Também há \textunderscore accuso\textunderscore  no jôgo dos Três-setes.
\section{Acebar}
\begin{itemize}
\item {Grp. gram.:v. t.}
\end{itemize}
\begin{itemize}
\item {Utilização:Prov.}
\end{itemize}
\begin{itemize}
\item {Utilização:trasm.}
\end{itemize}
Açular (cães)
\section{Acedia}
\begin{itemize}
\item {Grp. gram.:f.}
\end{itemize}
\begin{itemize}
\item {Proveniência:(Do gr. \textunderscore a\textunderscore  Priv. + \textunderscore kedos\textunderscore )}
\end{itemize}
Froixidão.
Negligência.
\section{Acedrenchado}
\begin{itemize}
\item {Grp. gram.:adj.}
\end{itemize}
\begin{itemize}
\item {Utilização:Ant.}
\end{itemize}
\begin{itemize}
\item {Proveniência:(De \textunderscore acedrenche\textunderscore )}
\end{itemize}
Que tem aspecto de acedrenche; axadrezado, estampado aos quadradinhos.
\section{Acedrenche}
\begin{itemize}
\item {Grp. gram.:m.}
\end{itemize}
\begin{itemize}
\item {Utilização:Ant.}
\end{itemize}
Jôgo, que correspondia ao xadrez. Cf. \textunderscore Hist. Geneal. da Casa Real\textunderscore .
\section{Acefalia}
\begin{itemize}
\item {Grp. gram.:f.}
\end{itemize}
\begin{itemize}
\item {Proveniência:(De \textunderscore acéphalo\textunderscore )}
\end{itemize}
Monstruosidade, que consiste na falta de cabeça.
\section{Acefálico}
\begin{itemize}
\item {Grp. gram.:adj.}
\end{itemize}
O mesmo que \textunderscore acéphalo\textunderscore .
\section{Acefalismo}
\begin{itemize}
\item {Grp. gram.:m.}
\end{itemize}
Seita dos \textunderscore Acefalitas\textunderscore .
\section{Acefalitas}
\begin{itemize}
\item {Grp. gram.:m. pl.}
\end{itemize}
Herejes, que não admittiam o concílio de Chalcedónia.
\section{Acéfalo}
\begin{itemize}
\item {Grp. gram.:adj.}
\end{itemize}
\begin{itemize}
\item {Proveniência:(Do gr. \textunderscore a\textunderscore  priv. + \textunderscore kephale\textunderscore , cabeça)}
\end{itemize}
Que não tem cabeça.
Sem chefe.
Idiota.
E diz-se de uma classe de molluscos.
\section{Acefalobraquia}
\begin{itemize}
\item {Grp. gram.:f.}
\end{itemize}
\begin{itemize}
\item {Proveniência:(De \textunderscore acephalobráchio\textunderscore )}
\end{itemize}
Monstruosidade acéphala, sem braços.
\section{Acefalobráquio}
\begin{itemize}
\item {Grp. gram.:adj.}
\end{itemize}
\begin{itemize}
\item {Proveniência:(Do gr. \textunderscore a\textunderscore  priv. + \textunderscore kephale\textunderscore  + \textunderscore brakhion\textunderscore )}
\end{itemize}
Que não tem cabeça nem braços.
\section{Acefalocardia}
\begin{itemize}
\item {Grp. gram.:f.}
\end{itemize}
\begin{itemize}
\item {Proveniência:(De \textunderscore acephalocárdio\textunderscore )}
\end{itemize}
Monstruosidade sem cabeça nem coração.
\section{Acefalocárdio}
\begin{itemize}
\item {Grp. gram.:adj.}
\end{itemize}
\begin{itemize}
\item {Proveniência:(Do gr. \textunderscore a\textunderscore  priv. + \textunderscore kephale\textunderscore  + \textunderscore kardia\textunderscore )}
\end{itemize}
Que não tem cabeça nem coração.
\section{Acefalogastria}
\begin{itemize}
\item {Grp. gram.:f.}
\end{itemize}
\begin{itemize}
\item {Proveniência:(Do gr. \textunderscore a\textunderscore  priv. + \textunderscore kephale\textunderscore  + \textunderscore gaster\textunderscore )}
\end{itemize}
Monstruosidade sem cabeça nem a parte superior do ventre.
\section{Acefalogástrico}
\begin{itemize}
\item {Grp. gram.:adj.}
\end{itemize}
\begin{itemize}
\item {Proveniência:(De \textunderscore acephalogastria\textunderscore )}
\end{itemize}
Diz-se do feto sem cabeça nem a parte superior do ventre.
\section{Acefalopodia}
\begin{itemize}
\item {Grp. gram.:f.}
\end{itemize}
\begin{itemize}
\item {Grp. gram.:f.}
\end{itemize}
\begin{itemize}
\item {Utilização:Terat.}
\end{itemize}
\begin{itemize}
\item {Proveniência:(De \textunderscore acephalópode\textunderscore )}
\end{itemize}
Monstruosidade sem pés nem cabeça.
Estado de acefalópode.
\section{Acefalópode}
\begin{itemize}
\item {Grp. gram.:m. e adj.}
\end{itemize}
\begin{itemize}
\item {Proveniência:(Do gr. \textunderscore a\textunderscore  priv. + \textunderscore kephale\textunderscore  + \textunderscore podos\textunderscore )}
\end{itemize}
Monstro sem pés nem cabeça.
\section{Acefalostomia}
\begin{itemize}
\item {Grp. gram.:f.}
\end{itemize}
Monstruosidade do \textunderscore acefalóstomo\textunderscore .
\section{Acefalóstomo}
\begin{itemize}
\item {Grp. gram.:m.}
\end{itemize}
\begin{itemize}
\item {Proveniência:(Do gr. \textunderscore a\textunderscore  priv. + \textunderscore kephale\textunderscore  + \textunderscore stoma\textunderscore )}
\end{itemize}
Monstruosidade acéfala, que no lugar da cabeça tem uma espécie de boca.
\section{Acefalotoracia}
\begin{itemize}
\item {Grp. gram.:f.}
\end{itemize}
\begin{itemize}
\item {Proveniência:(Do gr. \textunderscore a\textunderscore  priv. + \textunderscore kephale\textunderscore  + \textunderscore thorax\textunderscore )}
\end{itemize}
Monstruosidade sem cabeça nem thórax.
\section{Acefalotorácico}
\begin{itemize}
\item {Grp. gram.:adj.}
\end{itemize}
Que tem \textunderscore acefalotoracia\textunderscore .
\section{Aceifa}
\begin{itemize}
\item {Grp. gram.:f.}
\end{itemize}
O mesmo que \textunderscore ceifa\textunderscore .
\section{Aceifão}
\begin{itemize}
\item {Grp. gram.:m.}
\end{itemize}
\begin{itemize}
\item {Utilização:Prov.}
\end{itemize}
\begin{itemize}
\item {Utilização:alent.}
\end{itemize}
O mesmo que \textunderscore ceifeiro\textunderscore .
\section{Aceiramento}
\begin{itemize}
\item {Grp. gram.:m.}
\end{itemize}
Acto de \textunderscore aceirar\textunderscore .
\section{Aceirar}
\begin{itemize}
\item {Grp. gram.:v. t.}
\end{itemize}
\begin{itemize}
\item {Proveniência:(De \textunderscore aceiro\textunderscore ^1)}
\end{itemize}
Temperar com aço.
\section{Aceirar}
\begin{itemize}
\item {Grp. gram.:v. t.}
\end{itemize}
\begin{itemize}
\item {Utilização:Bras}
\end{itemize}
Cortar (a vegetação) em volta da mata.
Cortar (o mato) nas estremas das herdades, para as demarcar e evitar communicação de incêndio; sesmar.
Vigiar, andando á roda; observar de lado; aproximar-se para vêr melhor. Cf. M. Soares, \textunderscore Dicc. Bras.\textunderscore 
\section{Aceiro}
\begin{itemize}
\item {Grp. gram.:m.}
\end{itemize}
\begin{itemize}
\item {Utilização:Ant.}
\end{itemize}
Aquelle que trabalha em aço.
Barra de aço.
\section{Aceiro}
\begin{itemize}
\item {Grp. gram.:m.}
\end{itemize}
Faixa de terra arroteada, dentro ou em volta das herdades, para evitar a communicação de fogo ou facilitar o trânsito de carros; sesmo.
\section{Aceitan}
\begin{itemize}
\item {Grp. gram.:f.}
\end{itemize}
Ave avermelhada, conhecida em Trás-os-Montes.
\section{Acelga}
\begin{itemize}
\item {Grp. gram.:f.}
\end{itemize}
O mesmo que \textunderscore celga\textunderscore .
\section{Acém}
\begin{itemize}
\item {Grp. gram.:m.}
\end{itemize}
\begin{itemize}
\item {Proveniência:(Do ár. Cp. cast. \textunderscore acen\textunderscore )}
\end{itemize}
Parte do lombo do boi ou da vaca, entre a pá e a extremidade do cachaço.
\section{Acena}
\begin{itemize}
\item {Grp. gram.:f.}
\end{itemize}
\begin{itemize}
\item {Proveniência:(Gr. \textunderscore akaina\textunderscore )}
\end{itemize}
Planta vivaz, da fam. das rosáceas.
\section{Acenamento}
\begin{itemize}
\item {Grp. gram.:m.}
\end{itemize}
O mesmo que \textunderscore aceno\textunderscore .
\section{Acenar}
\begin{itemize}
\item {Grp. gram.:v. i.}
\end{itemize}
Fazer acenos.
Chamar a attenção.
\section{Acendrar}
\begin{itemize}
\item {Grp. gram.:v. t.}
\end{itemize}
\begin{itemize}
\item {Utilização:Fig.}
\end{itemize}
\begin{itemize}
\item {Proveniência:(Do lat. \textunderscore cinerare\textunderscore )}
\end{itemize}
Limpar com cinza.
Purificar.
Acrisolar.
\section{Acenha}
\begin{itemize}
\item {Grp. gram.:f.}
\end{itemize}
O mesmo ou melhor que \textunderscore azenha\textunderscore . Cf. G. Vicente, \textunderscore M. Parda\textunderscore .
\section{Aceno}
\begin{itemize}
\item {Grp. gram.:m.}
\end{itemize}
Gesto com a mão ou com a cabeça.
\section{Acenoso}
\begin{itemize}
\item {Grp. gram.:adj.}
\end{itemize}
\begin{itemize}
\item {Utilização:Bot.}
\end{itemize}
\begin{itemize}
\item {Proveniência:(De \textunderscore aceno\textunderscore )}
\end{itemize}
Diz-se dos órgãos vegetaes, que se apresentam curvos na ponta.
\section{Acentróptero}
\begin{itemize}
\item {Grp. gram.:m.}
\end{itemize}
\begin{itemize}
\item {Proveniência:(Do gr. \textunderscore a\textunderscore  priv. + \textunderscore kentron\textunderscore  + \textunderscore pteron\textunderscore )}
\end{itemize}
Gênero de coleópteros pentâmeros.
\section{...áceo}
\begin{itemize}
\item {Grp. gram.:suf. adj.}
\end{itemize}
\begin{itemize}
\item {Proveniência:(Lat. \textunderscore ...aceus\textunderscore )}
\end{itemize}
(designativo das qualidades geraes de um grupo ou série de indivíduos, cujo typo é, geralmente, expresso no radical da palavra respectiva)
\section{Acephalia}
\begin{itemize}
\item {Grp. gram.:f.}
\end{itemize}
\begin{itemize}
\item {Proveniência:(De \textunderscore acéphalo\textunderscore )}
\end{itemize}
Monstruosidade, que consiste na falta de cabeça.
\section{Acephálico}
\begin{itemize}
\item {Grp. gram.:adj.}
\end{itemize}
O mesmo que \textunderscore acéphalo\textunderscore .
\section{Acephalismo}
\begin{itemize}
\item {Grp. gram.:m.}
\end{itemize}
Seita dos \textunderscore Acephalitas\textunderscore .
\section{Acephalitas}
\begin{itemize}
\item {Grp. gram.:m. pl.}
\end{itemize}
Herejes, que não admittiam o concílio de Chalcedónia.
\section{Acéphalo}
\begin{itemize}
\item {Grp. gram.:adj.}
\end{itemize}
\begin{itemize}
\item {Proveniência:(Do gr. \textunderscore a\textunderscore  priv. + \textunderscore kephale\textunderscore , cabeça)}
\end{itemize}
Que não tem cabeça.
Sem chefe.
Idiota.
E diz-se de uma classe de molluscos.
\section{Acephalobrachia}
\begin{itemize}
\item {fónica:qui}
\end{itemize}
\begin{itemize}
\item {Grp. gram.:f.}
\end{itemize}
\begin{itemize}
\item {Proveniência:(De \textunderscore acephalobráchio\textunderscore )}
\end{itemize}
Monstruosidade acéphala, sem braços.
\section{Acephalobráchio}
\begin{itemize}
\item {fónica:qui}
\end{itemize}
\begin{itemize}
\item {Grp. gram.:adj.}
\end{itemize}
\begin{itemize}
\item {Proveniência:(Do gr. \textunderscore a\textunderscore  priv. + \textunderscore kephale\textunderscore  + \textunderscore brakhion\textunderscore )}
\end{itemize}
Que não tem cabeça nem braços.
\section{Acephalocardia}
\begin{itemize}
\item {Grp. gram.:f.}
\end{itemize}
\begin{itemize}
\item {Proveniência:(De \textunderscore acephalocárdio\textunderscore )}
\end{itemize}
Monstruosidade sem cabeça nem coração.
\section{Acephalocárdio}
\begin{itemize}
\item {Grp. gram.:adj.}
\end{itemize}
\begin{itemize}
\item {Proveniência:(Do gr. \textunderscore a\textunderscore  priv. + \textunderscore kephale\textunderscore  + \textunderscore kardia\textunderscore )}
\end{itemize}
Que não tem cabeça nem coração.
\section{Acephalogastria}
\begin{itemize}
\item {Grp. gram.:f.}
\end{itemize}
\begin{itemize}
\item {Proveniência:(Do gr. \textunderscore a\textunderscore  priv. + \textunderscore kephale\textunderscore  + \textunderscore gaster\textunderscore )}
\end{itemize}
Monstruosidade sem cabeça nem a parte superior do ventre.
\section{Acephalogástrico}
\begin{itemize}
\item {Grp. gram.:adj.}
\end{itemize}
\begin{itemize}
\item {Proveniência:(De \textunderscore acephalogastria\textunderscore )}
\end{itemize}
Diz-se do feto sem cabeça nem a parte superior do ventre.
\section{Acephalopodia}
\begin{itemize}
\item {Grp. gram.:f.}
\end{itemize}
\begin{itemize}
\item {Grp. gram.:f.}
\end{itemize}
\begin{itemize}
\item {Utilização:Terat.}
\end{itemize}
\begin{itemize}
\item {Proveniência:(De \textunderscore acephalópode\textunderscore )}
\end{itemize}
Monstruosidade sem pés nem cabeça.
Estado de acephalópode.
\section{Acephalópode}
\begin{itemize}
\item {Grp. gram.:m. e adj.}
\end{itemize}
\begin{itemize}
\item {Proveniência:(Do gr. \textunderscore a\textunderscore  priv. + \textunderscore kephale\textunderscore  + \textunderscore podos\textunderscore )}
\end{itemize}
Monstro sem pés nem cabeça.
\section{Acephalostomia}
\begin{itemize}
\item {Grp. gram.:f.}
\end{itemize}
Monstruosidade do \textunderscore acephalóstomo\textunderscore .
\section{Acephalóstomo}
\begin{itemize}
\item {Grp. gram.:m.}
\end{itemize}
\begin{itemize}
\item {Proveniência:(Do gr. \textunderscore a\textunderscore  priv. + \textunderscore kephale\textunderscore  + \textunderscore stoma\textunderscore )}
\end{itemize}
Monstruosidade acéphala, que no lugar da cabeça tem uma espécie de boca.
\section{Acephalothoracia}
\begin{itemize}
\item {Grp. gram.:f.}
\end{itemize}
\begin{itemize}
\item {Proveniência:(Do gr. \textunderscore a\textunderscore  priv. + \textunderscore kephale\textunderscore  + \textunderscore thorax\textunderscore )}
\end{itemize}
Monstruosidade sem cabeça nem thórax.
\section{Acephalothorácico}
\begin{itemize}
\item {Grp. gram.:adj.}
\end{itemize}
Que tem \textunderscore acephalothoracia\textunderscore .
\section{Acepilhador}
\begin{itemize}
\item {Grp. gram.:m.}
\end{itemize}
O que acepilha.
\section{Acepilhadura}
\begin{itemize}
\item {Grp. gram.:f.}
\end{itemize}
Acto de \textunderscore acepilhar\textunderscore .
Apara, maravalha.
\section{Acepilhar}
\begin{itemize}
\item {Grp. gram.:v. t.}
\end{itemize}
\begin{itemize}
\item {Proveniência:(De \textunderscore cepilho\textunderscore )}
\end{itemize}
Alisar com cepilho.
Aperfeiçoar.
\section{Acepipar}
\begin{itemize}
\item {Grp. gram.:v. t.}
\end{itemize}
\begin{itemize}
\item {Proveniência:(De \textunderscore acepipe\textunderscore )}
\end{itemize}
Dar gôsto delicado a (um alimento); tornar saboroso.
\section{Acepipe}
\begin{itemize}
\item {Grp. gram.:m.}
\end{itemize}
Guloseima.
Pitéu.
(Cast. \textunderscore acebibe\textunderscore )
\section{Acepipeiro}
\begin{itemize}
\item {Grp. gram.:adj.}
\end{itemize}
Que gosta de acepipes, guloso.
\section{Acéqua}
\begin{itemize}
\item {Grp. gram.:f.}
\end{itemize}
\begin{itemize}
\item {Utilização:Ant.}
\end{itemize}
O mesmo que \textunderscore acequia\textunderscore .
\section{Acequia}
\begin{itemize}
\item {Grp. gram.:f.}
\end{itemize}
\begin{itemize}
\item {Utilização:Prov.}
\end{itemize}
\begin{itemize}
\item {Utilização:alent.}
\end{itemize}
\begin{itemize}
\item {Proveniência:(Do ár. \textunderscore acequia\textunderscore )}
\end{itemize}
Açude.
Azenha.
Aqueducto.
Cano collector das águas pluviaes.
\section{Ácer}
\begin{itemize}
\item {Grp. gram.:m.}
\end{itemize}
\begin{itemize}
\item {Proveniência:(Lat. \textunderscore acer\textunderscore )}
\end{itemize}
O mesmo que \textunderscore bordo\textunderscore ^2, árvore.
\section{Aceração}
\begin{itemize}
\item {Grp. gram.:f.}
\end{itemize}
Acto de \textunderscore acerar\textunderscore .
\section{Aceráceas}
\begin{itemize}
\item {Grp. gram.:f. pl.}
\end{itemize}
O mesmo que \textunderscore aceríneas\textunderscore .
\section{Aceradamente}
\begin{itemize}
\item {Grp. gram.:adv.}
\end{itemize}
De modo \textunderscore acerado\textunderscore .
\section{Aceradas}
\begin{itemize}
\item {Grp. gram.:f. pl.}
\end{itemize}
\begin{itemize}
\item {Proveniência:(De \textunderscore ácer\textunderscore )}
\end{itemize}
Classe de vegetaes dicotyledóneos, que comprehende o bórdo e outras árvores.
\section{Acerado}
\begin{itemize}
\item {Grp. gram.:adj.}
\end{itemize}
\begin{itemize}
\item {Utilização:Fig.}
\end{itemize}
Que corta.
Afiado.
Penetrante, cáustico: \textunderscore um dito acerado\textunderscore .
\section{Acerantho}
\begin{itemize}
\item {Grp. gram.:m.}
\end{itemize}
\begin{itemize}
\item {Proveniência:(Do gr. \textunderscore a\textunderscore  priv. + \textunderscore keras\textunderscore )}
\end{itemize}
Planta do Japão.
\section{Aceranto}
\begin{itemize}
\item {Grp. gram.:m.}
\end{itemize}
\begin{itemize}
\item {Proveniência:(Do gr. \textunderscore a\textunderscore  priv. + \textunderscore keras\textunderscore )}
\end{itemize}
Planta do Japão.
\section{Acerar}
\begin{itemize}
\item {Grp. gram.:v. t.}
\end{itemize}
Dar têmpera de aço a.
Afiar.
Estimular.
(Por \textunderscore aceirar\textunderscore , de \textunderscore aceiro\textunderscore ^1)
\section{Acerato}
\begin{itemize}
\item {Grp. gram.:m.}
\end{itemize}
Planta asclepiádea da América.
\section{Acerato}
\begin{itemize}
\item {Grp. gram.:m.}
\end{itemize}
\begin{itemize}
\item {Proveniência:(De \textunderscore ácer\textunderscore )}
\end{itemize}
Sal, resultante da combinação do ácido acérico com uma base.
\section{Acerbamente}
\begin{itemize}
\item {Grp. gram.:adv.}
\end{itemize}
\begin{itemize}
\item {Proveniência:(De \textunderscore acerbo\textunderscore )}
\end{itemize}
Asperamente.
Cruelmente.
\section{Acerbidade}
\begin{itemize}
\item {Grp. gram.:f.}
\end{itemize}
\begin{itemize}
\item {Utilização:Fig.}
\end{itemize}
Qualidade do que é \textunderscore acerbo\textunderscore .
Rigor.
Aspereza.
\section{Acerbar}
\begin{itemize}
\item {Grp. gram.:v. t.}
\end{itemize}
Tornar acerbo ou angustioso; angustiar. Cf. Rui Barb., \textunderscore Répl.\textunderscore , 157.
\section{Acerbo}
\begin{itemize}
\item {Grp. gram.:adj.}
\end{itemize}
\begin{itemize}
\item {Proveniência:(Lat. \textunderscore acerbus\textunderscore )}
\end{itemize}
Azêdo.
Severo; cruel.
\section{Acêrca}
\begin{itemize}
\item {Proveniência:(Do lat. \textunderscore circa\textunderscore )}
\end{itemize}
\textunderscore loc. prep.\textunderscore  (seg. da prep. \textunderscore de\textunderscore )
A respeito de; relativamente a; sôbre.
\section{Á cêrca}
\begin{itemize}
\item {Proveniência:(Do lat. \textunderscore circa\textunderscore )}
\end{itemize}
\textunderscore loc. prep.\textunderscore  (seg. da prep. \textunderscore de\textunderscore )
A respeito de; relativamente a; sôbre.
\section{Acêrca}
\begin{itemize}
\item {Grp. gram.:adv.}
\end{itemize}
Perto:«\textunderscore seu pai acêrca morava\textunderscore ». \textunderscore Menina e Moça\textunderscore . Cf. \textunderscore Ethiópia Or.\textunderscore  II, 38 e 42.
\section{Acercar-se}
\begin{itemize}
\item {Grp. gram.:v. p.}
\end{itemize}
\begin{itemize}
\item {Proveniência:(De \textunderscore cêrca\textunderscore )}
\end{itemize}
Aproximar-se; avizinhar-se.
\section{Acerdésia}
\begin{itemize}
\item {Grp. gram.:f.}
\end{itemize}
Óxydo de manganés hydratado.
\section{Acerejado}
\begin{itemize}
\item {Grp. gram.:adj.}
\end{itemize}
Que tem côr de cereja.
\section{Acerejar}
\begin{itemize}
\item {Grp. gram.:v. t.}
\end{itemize}
Dar côr de cereja a.
\section{Acéreo}
\begin{itemize}
\item {Grp. gram.:adj.}
\end{itemize}
\begin{itemize}
\item {Grp. gram.:M. pl.}
\end{itemize}
\begin{itemize}
\item {Proveniência:(De \textunderscore ácer\textunderscore )}
\end{itemize}
O mesmo que \textunderscore aceríneo\textunderscore .
Família de molluscos gasterópodes opistóbrânchios.
\section{Acérico}
\begin{itemize}
\item {Grp. gram.:adj.}
\end{itemize}
Diz-se do ácido que constitue a essência do ácer.
\section{Acerina}
\begin{itemize}
\item {Grp. gram.:f.}
\end{itemize}
Gênero de peixes acanthópteros.
\section{Aceríneas}
\begin{itemize}
\item {Grp. gram.:f. pl.}
\end{itemize}
Família de plantas, da classe das aceradas.
(Fem. pl. de \textunderscore acerineo\textunderscore )
\section{Aceríneo}
\begin{itemize}
\item {Grp. gram.:adj.}
\end{itemize}
Relativo ao ácer.
\section{Acero}
\begin{itemize}
\item {Grp. gram.:m.}
\end{itemize}
\begin{itemize}
\item {Utilização:Bras}
\end{itemize}
O mesmo que \textunderscore aceiro\textunderscore ^2.
\section{Aceroídeas}
\begin{itemize}
\item {Grp. gram.:f. pl.}
\end{itemize}
\begin{itemize}
\item {Proveniência:(Do lat. \textunderscore acer\textunderscore  + gr. \textunderscore eidos\textunderscore )}
\end{itemize}
Ordem de plantas, que abrange as aceríneas, sapindáceas e outras.
\section{Acerolo}
\begin{itemize}
\item {Grp. gram.:m.}
\end{itemize}
\begin{itemize}
\item {Proveniência:(De \textunderscore ácer\textunderscore )}
\end{itemize}
Fruto de Espanha, semelhante á cereja.
\section{Áceros}
\begin{itemize}
\item {Grp. gram.:m. pl.}
\end{itemize}
\begin{itemize}
\item {Proveniência:(Do gr. \textunderscore a\textunderscore  priv. + \textunderscore keras\textunderscore , corno)}
\end{itemize}
Insectos sem antennas.
\section{Aceroso}
\begin{itemize}
\item {Grp. gram.:adj.}
\end{itemize}
\begin{itemize}
\item {Proveniência:(De \textunderscore ácer\textunderscore )}
\end{itemize}
Diz-se das fôlhas lineares e persistentes, como a do pinheiro.
\section{Acerotosia}
\begin{itemize}
\item {Grp. gram.:f.}
\end{itemize}
\begin{itemize}
\item {Proveniência:(Do gr. \textunderscore a\textunderscore  priv. + \textunderscore keras\textunderscore , corno)}
\end{itemize}
Monstruosidade dos ruminantes, caracterizada pela ausência de cornos.
\section{Acerra}
\begin{itemize}
\item {Grp. gram.:f.}
\end{itemize}
\begin{itemize}
\item {Proveniência:(Lat. \textunderscore acerra\textunderscore )}
\end{itemize}
Naveta.
Vaso de perfumes.
\section{Acerrimamente}
\begin{itemize}
\item {Grp. gram.:adv.}
\end{itemize}
De modo \textunderscore acérrimo\textunderscore .
\section{Acérrimo}
\begin{itemize}
\item {Grp. gram.:adj.}
\end{itemize}
\begin{itemize}
\item {Proveniência:(Lat. \textunderscore accerrimus\textunderscore )}
\end{itemize}
Muito agre, picante.
Pertinaz, insistente.
\section{Acertadamente}
\begin{itemize}
\item {Grp. gram.:adv.}
\end{itemize}
De modo \textunderscore acertado\textunderscore .
\section{Acertador}
\begin{itemize}
\item {Grp. gram.:m.}
\end{itemize}
O que acerta.
\section{Acertamento}
\begin{itemize}
\item {Grp. gram.:m.}
\end{itemize}
Acto de \textunderscore acertar\textunderscore .
Acaso, coincidência:«\textunderscore uma quarta-feira, que per acertamento foy béspera de Corpo de Deus\textunderscore ». Rui Pina, \textunderscore Chrón. de Af. V\textunderscore , C. CXXXI.
\section{Acertar}
\begin{itemize}
\item {Grp. gram.:v. t.}
\end{itemize}
\begin{itemize}
\item {Grp. gram.:V. i.}
\end{itemize}
Descobrir, achar ao certo.
Pôr certo, igualar.
Harmonizar.
Dar no alvo; attingir.
Coincidir.
Acontecer.
\section{Acêrto}
\begin{itemize}
\item {Grp. gram.:m.}
\end{itemize}
\begin{itemize}
\item {Utilização:Prov.}
\end{itemize}
Acto de \textunderscore acertar\textunderscore .
Tino.
Casualidade.
\section{Acervação}
\begin{itemize}
\item {Grp. gram.:f.}
\end{itemize}
\begin{itemize}
\item {Proveniência:(Lat. \textunderscore acervatio\textunderscore )}
\end{itemize}
Acto ou effeito de amontoar; acervo.
\section{Acervar}
\begin{itemize}
\item {Grp. gram.:v. t.}
\end{itemize}
Amontoar. Cf. Gonçalves Dias, \textunderscore Poes.\textunderscore , II, 117.
\section{Acervejado}
\begin{itemize}
\item {Grp. gram.:adj.}
\end{itemize}
\begin{itemize}
\item {Utilização:Fig.}
\end{itemize}
Que tem côr ou sabor de cerveja.
Que reflecte a influência dos povos, onde se consome muita cerveja:«\textunderscore portugueses acervejados de germanismo\textunderscore ». Camillo, \textunderscore Cavar em Ruínas\textunderscore , 57.
\section{Acervo}
\begin{itemize}
\item {Grp. gram.:m.}
\end{itemize}
\begin{itemize}
\item {Proveniência:(Lat. \textunderscore acervus\textunderscore )}
\end{itemize}
Montão, cúmulo.
Abundância.
\section{Acérvulo}
\begin{itemize}
\item {Grp. gram.:m.}
\end{itemize}
Pequeno \textunderscore acervo\textunderscore .
\section{Acescência}
\begin{itemize}
\item {Grp. gram.:f.}
\end{itemize}
\begin{itemize}
\item {Proveniência:(De \textunderscore acescente\textunderscore )}
\end{itemize}
Disposição para se azedar.
\section{Acescente}
\begin{itemize}
\item {Grp. gram.:adj.}
\end{itemize}
\begin{itemize}
\item {Proveniência:(Lat. \textunderscore acescens\textunderscore )}
\end{itemize}
Que começa a azedar-se.
\section{Acetabulária}
\begin{itemize}
\item {Grp. gram.:f.}
\end{itemize}
\begin{itemize}
\item {Proveniência:(Do lat. \textunderscore acetabulum\textunderscore )}
\end{itemize}
Alga marinha unicellular.
\section{Acetabulário}
\begin{itemize}
\item {Grp. gram.:m.}
\end{itemize}
\begin{itemize}
\item {Proveniência:(Do lat. \textunderscore acetabulum\textunderscore )}
\end{itemize}
Alga marinha unicellular.
\section{Acetabulífero}
\begin{itemize}
\item {Grp. gram.:adj.}
\end{itemize}
\begin{itemize}
\item {Grp. gram.:M. pl.}
\end{itemize}
\begin{itemize}
\item {Proveniência:(Do lat. \textunderscore acetabulum\textunderscore  + \textunderscore ferre\textunderscore )}
\end{itemize}
Que tem ventosa nos tentáculos, (falando-se de certos molluscos).
Ordem de molluscos, que tem sugadoiro ou ventosa na extremidade dos tentáculos.
\section{Acetabuliforme}
\begin{itemize}
\item {Grp. gram.:adj.}
\end{itemize}
\begin{itemize}
\item {Proveniência:(De \textunderscore acetábulo\textunderscore  + \textunderscore fórma\textunderscore )}
\end{itemize}
Que tem a fórma de taça.
\section{Acetábulo}
\begin{itemize}
\item {Grp. gram.:m.}
\end{itemize}
\begin{itemize}
\item {Proveniência:(Lat. \textunderscore acetabulum\textunderscore )}
\end{itemize}
Antigo vaso para vinagre.
Cavidade cotyloídea.
\section{Acetacético}
\begin{itemize}
\item {Grp. gram.:adj.}
\end{itemize}
Diz-se de um ácido, o mesmo que \textunderscore diacético\textunderscore .
\section{Acetal}
\begin{itemize}
\item {Grp. gram.:m.}
\end{itemize}
\begin{itemize}
\item {Proveniência:(De \textunderscore acético\textunderscore )}
\end{itemize}
Producto da oxydação do álcool.
Líquido incolor, de cheiro ethéreo.
\section{Acetamido}
\begin{itemize}
\item {Grp. gram.:m.}
\end{itemize}
\begin{itemize}
\item {Proveniência:(De \textunderscore acetum\textunderscore  lat. + \textunderscore amido\textunderscore )}
\end{itemize}
Amido, derivado do ácido acético.
\section{Acetâmido}
\begin{itemize}
\item {Grp. gram.:m.}
\end{itemize}
\begin{itemize}
\item {Proveniência:(De \textunderscore acetum\textunderscore  lat. + \textunderscore amido\textunderscore )}
\end{itemize}
Amido, derivado do ácido acético.
\section{Acetanilido}
\begin{itemize}
\item {Grp. gram.:m.}
\end{itemize}
\begin{itemize}
\item {Utilização:Chím.}
\end{itemize}
Corpo branco, crystallizado em lâminas.
\section{Acetar}
\begin{itemize}
\item {Grp. gram.:v. t.}
\end{itemize}
\begin{itemize}
\item {Proveniência:(Do lat. \textunderscore acetum\textunderscore )}
\end{itemize}
Tornar azêdo.
\section{Acetário}
\begin{itemize}
\item {Grp. gram.:m.}
\end{itemize}
\begin{itemize}
\item {Proveniência:(Do lat. \textunderscore acetum\textunderscore )}
\end{itemize}
Medicamento, que tem por base o vinagre.
\section{Acetato}
\begin{itemize}
\item {Grp. gram.:m.}
\end{itemize}
\begin{itemize}
\item {Proveniência:(Do lat. \textunderscore acetum\textunderscore )}
\end{itemize}
Sal, resultante da combinação do ácido acético com uma base.
\section{Acéter}
\begin{itemize}
\item {Grp. gram.:m.}
\end{itemize}
Púcaro antigo.
(Cast. \textunderscore acetre\textunderscore )
\section{Acético}
\begin{itemize}
\item {Grp. gram.:adj.}
\end{itemize}
\begin{itemize}
\item {Proveniência:(Lat. \textunderscore aceticus\textunderscore )}
\end{itemize}
Relativo ao vinagre.
Ácido.
\section{Acetidina}
\begin{itemize}
\item {Grp. gram.:f.}
\end{itemize}
\begin{itemize}
\item {Proveniência:(Do lat. \textunderscore acetum\textunderscore )}
\end{itemize}
Líquido oleoso, de cheiro agradável, análogo ao do éther acético.
\section{Acetificação}
\begin{itemize}
\item {Grp. gram.:f.}
\end{itemize}
Acto de \textunderscore acetificar\textunderscore .
\section{Acetificar}
\begin{itemize}
\item {Grp. gram.:v. t.}
\end{itemize}
\begin{itemize}
\item {Proveniência:(Do lat. \textunderscore acetum\textunderscore  + \textunderscore facere\textunderscore )}
\end{itemize}
Converter em vinagre.
Azedar.
\section{Acetímetro}
\begin{itemize}
\item {Grp. gram.:m.}
\end{itemize}
O mesmo que \textunderscore acetómetro\textunderscore .
\section{Acetina}
\begin{itemize}
\item {Grp. gram.:f.}
\end{itemize}
\begin{itemize}
\item {Proveniência:(De \textunderscore acético\textunderscore )}
\end{itemize}
Líquido neutro, que se obtém pela reacção do ácido acético e da glycerina.
\section{Acetol}
\begin{itemize}
\item {Grp. gram.:m.}
\end{itemize}
\begin{itemize}
\item {Proveniência:(Do lat. \textunderscore acetum\textunderscore )}
\end{itemize}
Expressão, adoptada para designar o vinagre na sua maior pureza.
\section{Acetolado}
\begin{itemize}
\item {Grp. gram.:m.}
\end{itemize}
\begin{itemize}
\item {Proveniência:(De \textunderscore acetol\textunderscore )}
\end{itemize}
Vinagre medicinal, preparado por solução.
\section{Acetolato}
\begin{itemize}
\item {Grp. gram.:m.}
\end{itemize}
O mesmo que \textunderscore acetolado\textunderscore .
\section{Acetomel}
\begin{itemize}
\item {Grp. gram.:m.}
\end{itemize}
\begin{itemize}
\item {Proveniência:(Do lat. \textunderscore acetum\textunderscore  + \textunderscore mel\textunderscore )}
\end{itemize}
Xarope de vinagre melado.
\section{Acetómetro}
\begin{itemize}
\item {Grp. gram.:m.}
\end{itemize}
\begin{itemize}
\item {Proveniência:(Do lat. \textunderscore acetum\textunderscore  + gr. \textunderscore metron\textunderscore )}
\end{itemize}
Instrumento, para medir a graduação do vinagre.
\section{Acetona}
\begin{itemize}
\item {Grp. gram.:f.}
\end{itemize}
\begin{itemize}
\item {Proveniência:(De \textunderscore acetico\textunderscore )}
\end{itemize}
Líquido incolor, obtido pela destillação de acetatos alcalinos, depois de muito secos.
\section{Acetonato}
\begin{itemize}
\item {Grp. gram.:m.}
\end{itemize}
\begin{itemize}
\item {Proveniência:(De \textunderscore acetona\textunderscore )}
\end{itemize}
Qualquer sal, formado pela combinação do ácido acetónico com uma base.
\section{Acetonemia}
\begin{itemize}
\item {Grp. gram.:f.}
\end{itemize}
\begin{itemize}
\item {Utilização:Med.}
\end{itemize}
\begin{itemize}
\item {Proveniência:(De \textunderscore acetona\textunderscore  + gr. \textunderscore haima\textunderscore , sangue)}
\end{itemize}
Presença de acetona no sangue.
\section{Acetonia}
\begin{itemize}
\item {Grp. gram.:f.}
\end{itemize}
(V.acetona)
\section{Acetónico}
\begin{itemize}
\item {Grp. gram.:adj.}
\end{itemize}
\begin{itemize}
\item {Proveniência:(De \textunderscore acetonia\textunderscore )}
\end{itemize}
Diz-se de um ácido, derivado de acetatos alcalinos.
\section{Acetonina}
\begin{itemize}
\item {Grp. gram.:f.}
\end{itemize}
\begin{itemize}
\item {Proveniência:(De \textunderscore acetonia\textunderscore )}
\end{itemize}
Álcali orgânico, solúvel na água, no álcool e no éther.
\section{Acetonuria}
\begin{itemize}
\item {Grp. gram.:f.}
\end{itemize}
\begin{itemize}
\item {Utilização:Med.}
\end{itemize}
\begin{itemize}
\item {Proveniência:(De \textunderscore acetona\textunderscore  + gr. \textunderscore ouron\textunderscore )}
\end{itemize}
Eliminação de acetona, pela urina.
\section{Acaico}
\begin{itemize}
\item {Grp. gram.:adj.}
\end{itemize}
\begin{itemize}
\item {Proveniência:(Lat. \textunderscore achaiens\textunderscore )}
\end{itemize}
Relativo aos Acheus.
\section{Acaina}
\begin{itemize}
\item {Grp. gram.:f.}
\end{itemize}
\begin{itemize}
\item {Proveniência:(Do gr. \textunderscore a\textunderscore  priv. + \textunderscore khanein\textunderscore )}
\end{itemize}
Fruto monospérmico, indehiscente, cujo pericarpo adhere ao invólucro do grão e ao tubo do cálice, como se observa nas synanthéreas.
\section{Acaio}
\begin{itemize}
\item {Grp. gram.:adj.}
\end{itemize}
\begin{itemize}
\item {Proveniência:(Lat. \textunderscore achaius\textunderscore )}
\end{itemize}
O mesmo que \textunderscore achaico\textunderscore .
\section{Acalmópteros}
\begin{itemize}
\item {Grp. gram.:m. pl.}
\end{itemize}
Divisão dos lepidópteros, no systema de Blanchard, na qual se comprehendem aquelles, cujas asas, durante o repoiso, estão erguidas.
\section{Acária}
\begin{itemize}
\item {Grp. gram.:f.}
\end{itemize}
Gênero de plantas passiflóreas.
\section{Acetilanilina}
\begin{itemize}
\item {Grp. gram.:f.}
\end{itemize}
Substância alcalina, solúvel na água e no álcool, insolúvel no éther.
\section{Acetilena}
\begin{itemize}
\item {Grp. gram.:f.}
\end{itemize}
O mesmo que \textunderscore acetilene\textunderscore .
\section{Acetilene}
\begin{itemize}
\item {Grp. gram.:m.}
\end{itemize}
Gás, que se obtém pelo carboneto de cálcio e que começa agora a applicar-se á illuminação.
\section{Acetilênico}
\begin{itemize}
\item {Grp. gram.:adj.}
\end{itemize}
Diz-se de um grupo de carbonetos.
\section{Acetileno}
\begin{itemize}
\item {Grp. gram.:m.}
\end{itemize}
\begin{itemize}
\item {Utilização:Chím.}
\end{itemize}
O mesmo ou melhor que \textunderscore acetylene\textunderscore .
Um dos carbonetos do grupo acetylênico.
\section{Acetilogênio}
\begin{itemize}
\item {Grp. gram.:m.}
\end{itemize}
Lâmpada de acetylene.
\section{Acétilo}
\begin{itemize}
\item {Grp. gram.:m.}
\end{itemize}
\begin{itemize}
\item {Proveniência:(De \textunderscore acético\textunderscore )}
\end{itemize}
Rad. hypothético dos compostos acéticos, e cuja fórmula é C^4 H^3 O^2.
\section{Acetosamina}
\begin{itemize}
\item {Grp. gram.:f.}
\end{itemize}
\begin{itemize}
\item {Proveniência:(De \textunderscore acetoso\textunderscore )}
\end{itemize}
Substância alcalina, insolúvel no éther, solúvel no álcool e na água.
\section{Acetosidade}
\begin{itemize}
\item {Grp. gram.:f.}
\end{itemize}
Qualidade de \textunderscore acetoso\textunderscore .
\section{Acetoso}
\begin{itemize}
\item {Grp. gram.:adj.}
\end{itemize}
\begin{itemize}
\item {Proveniência:(Lat. \textunderscore acetosus\textunderscore )}
\end{itemize}
Que tem sabor de vinagre.
\section{Acetre}
\begin{itemize}
\item {Grp. gram.:m.}
\end{itemize}
\begin{itemize}
\item {Utilização:Ant.}
\end{itemize}
Lavatório portátil.
O mesmo que \textunderscore acéter\textunderscore .
\section{Acetulatura}
\begin{itemize}
\item {Grp. gram.:f.}
\end{itemize}
Vinagre, feito com sumo de plantas verdes.
\section{Acetylanilina}
\begin{itemize}
\item {Grp. gram.:f.}
\end{itemize}
Substância alcalina, solúvel na água e no álcool, insolúvel no éther.
\section{Acetylena}
\begin{itemize}
\item {Grp. gram.:f.}
\end{itemize}
O mesmo que \textunderscore acetylene\textunderscore .
\section{Acetylene}
\begin{itemize}
\item {Grp. gram.:m.}
\end{itemize}
Gás, que se obtém pelo carboneto de cálcio e que começa agora a applicar-se á illuminação.
\section{Acetylênico}
\begin{itemize}
\item {Grp. gram.:adj.}
\end{itemize}
Diz-se de um grupo de carbonetos.
\section{Acetyleno}
\begin{itemize}
\item {Grp. gram.:m.}
\end{itemize}
\begin{itemize}
\item {Utilização:Chím.}
\end{itemize}
O mesmo ou melhor que \textunderscore acetylene\textunderscore .
Um dos carbonetos do grupo acetylênico.
\section{Acetylogênio}
\begin{itemize}
\item {Grp. gram.:m.}
\end{itemize}
Lâmpada de acetylene.
\section{Acétylo}
\begin{itemize}
\item {Grp. gram.:m.}
\end{itemize}
\begin{itemize}
\item {Proveniência:(De \textunderscore acético\textunderscore )}
\end{itemize}
Rad. hypothético dos compostos acéticos, e cuja fórmula é C^4 H^3 O^2.
\section{Acevadado}
\begin{itemize}
\item {Grp. gram.:adj.}
\end{itemize}
Alimentado com cevada.
\section{Acevadar}
\begin{itemize}
\item {Grp. gram.:v. t.}
\end{itemize}
Alimentar com cevada.
\section{Acevar}
\begin{itemize}
\item {Grp. gram.:v. t.}
\end{itemize}
(V. \textunderscore cevar\textunderscore ^1)
\section{Acha}
\begin{itemize}
\item {Grp. gram.:f.}
\end{itemize}
\begin{itemize}
\item {Grp. gram.:Pl.}
\end{itemize}
\begin{itemize}
\item {Proveniência:(Do lat. \textunderscore astula\textunderscore )}
\end{itemize}
Cavaca, pedaço de madeira tôsca para o lume.
Lenha.
\section{Acha}
\begin{itemize}
\item {Grp. gram.:f.}
\end{itemize}
\begin{itemize}
\item {Proveniência:(Do \textunderscore ant. al.\textunderscore )}
\end{itemize}
Arma antiga, do feitio de machada.
\section{Achabaçar}
\begin{itemize}
\item {Grp. gram.:v. t.}
\end{itemize}
\begin{itemize}
\item {Utilização:Prov.}
\end{itemize}
\begin{itemize}
\item {Utilização:beir.}
\end{itemize}
Despedaçar, fazer em cacos.
\section{Achaboucado}
\begin{itemize}
\item {Grp. gram.:adj.}
\end{itemize}
\begin{itemize}
\item {Utilização:Prov.}
\end{itemize}
Tôsco, mal acabado.
Desajeitado.
(Colhido no concelho de Lamego)
\section{Achacadamente}
\begin{itemize}
\item {Grp. gram.:adv.}
\end{itemize}
De modo \textunderscore achacado\textunderscore .
\section{Achacadiço}
\begin{itemize}
\item {Grp. gram.:adj.}
\end{itemize}
\begin{itemize}
\item {Proveniência:(De \textunderscore achacado\textunderscore )}
\end{itemize}
Sujeito a achaques, enfermiço.
\section{Achacado}
\begin{itemize}
\item {Grp. gram.:adj.}
\end{itemize}
\begin{itemize}
\item {Utilização:T. da Bairrada}
\end{itemize}
Adoentado.
Enfermiço.
Doente do figado.
\section{Achacana}
\begin{itemize}
\item {Grp. gram.:f.}
\end{itemize}
Espécie de cacto do Peru.
\section{Achacar}
\begin{itemize}
\item {Grp. gram.:v. i.}
\end{itemize}
\begin{itemize}
\item {Utilização:Ant.}
\end{itemize}
\begin{itemize}
\item {Proveniência:(De \textunderscore achaque\textunderscore )}
\end{itemize}
Adoecer.
Accusar; denunciar.
Infamar.
\section{Achacoso}
\begin{itemize}
\item {Grp. gram.:adj.}
\end{itemize}
Que tem achaques.
\section{Achada}
\begin{itemize}
\item {Grp. gram.:f.}
\end{itemize}
\begin{itemize}
\item {Utilização:Ant.}
\end{itemize}
Acto ou effeito de \textunderscore achar\textunderscore .
Multa ou coima.
\section{Achada}
\begin{itemize}
\item {Grp. gram.:f.}
\end{itemize}
\begin{itemize}
\item {Utilização:Prov.}
\end{itemize}
\begin{itemize}
\item {Utilização:alg.}
\end{itemize}
\begin{itemize}
\item {Utilização:Ant.}
\end{itemize}
\begin{itemize}
\item {Proveniência:(De \textunderscore achaada\textunderscore , e \textunderscore achãada\textunderscore , contr. de \textunderscore achanada\textunderscore , de \textunderscore chan\textunderscore )}
\end{itemize}
Planície.
\section{Achádego}
\begin{itemize}
\item {Grp. gram.:m.}
\end{itemize}
\begin{itemize}
\item {Utilização:Ant.}
\end{itemize}
\begin{itemize}
\item {Proveniência:(De \textunderscore achar\textunderscore )}
\end{itemize}
Prémio, que, segundo as \textunderscore Ordenações do Reino\textunderscore , se dava a quem achasse alguma coisa.
\section{Achadiço}
\begin{itemize}
\item {Grp. gram.:adj.}
\end{itemize}
\begin{itemize}
\item {Grp. gram.:M.}
\end{itemize}
\begin{itemize}
\item {Utilização:Prov.}
\end{itemize}
Que facilmente se acha.
Indivíduo de pouca importância; vadio adventício. (Colhido na Bairrada)
\section{Achádigo}
\begin{itemize}
\item {Grp. gram.:m.}
\end{itemize}
O mesmo ou melhor que \textunderscore achádego\textunderscore .
\section{Achadilha}
\begin{itemize}
\item {Grp. gram.:f.}
\end{itemize}
\begin{itemize}
\item {Utilização:Prov.}
\end{itemize}
\begin{itemize}
\item {Utilização:trasm.}
\end{itemize}
\begin{itemize}
\item {Proveniência:(De \textunderscore achar\textunderscore )}
\end{itemize}
Lembrança súbitae extravagante.
Pretexto para faltar á palavra dada.
Escapatória.
\section{Achado}
\begin{itemize}
\item {Grp. gram.:m.}
\end{itemize}
\begin{itemize}
\item {Proveniência:(De \textunderscore achar\textunderscore )}
\end{itemize}
Aquillo que se achou.
Descobrimento, invento.
\section{Achadoiro}
\begin{itemize}
\item {Grp. gram.:m.}
\end{itemize}
Lugar, onde se achou alguma coisa.
\section{Achador}
\begin{itemize}
\item {Grp. gram.:m.}
\end{itemize}
O que acha.
\section{Achadouro}
\begin{itemize}
\item {Grp. gram.:m.}
\end{itemize}
Lugar, onde se achou alguma coisa.
\section{Achagual}
\begin{itemize}
\item {Grp. gram.:m.}
\end{itemize}
Peixe das costas da América do Sul.
\section{Achaico}
\begin{itemize}
\item {fónica:cai}
\end{itemize}
\begin{itemize}
\item {Grp. gram.:adj.}
\end{itemize}
\begin{itemize}
\item {Proveniência:(Lat. \textunderscore achaiens\textunderscore )}
\end{itemize}
Relativo aos Acheus.
\section{Achaina}
\begin{itemize}
\item {fónica:cai}
\end{itemize}
\begin{itemize}
\item {Grp. gram.:f.}
\end{itemize}
\begin{itemize}
\item {Proveniência:(Do gr. \textunderscore a\textunderscore  priv. + \textunderscore khanein\textunderscore )}
\end{itemize}
Fruto monospérmico, indehiscente, cujo pericarpo adhere ao invólucro do grão e ao tubo do cálice, como se observa nas synanthéreas.
\section{Achaio}
\begin{itemize}
\item {fónica:cai}
\end{itemize}
\begin{itemize}
\item {Grp. gram.:adj.}
\end{itemize}
\begin{itemize}
\item {Proveniência:(Lat. \textunderscore achaius\textunderscore )}
\end{itemize}
O mesmo que \textunderscore achaico\textunderscore .
\section{Achalmópteros}
\begin{itemize}
\item {Grp. gram.:m. pl.}
\end{itemize}
Divisão dos lepidópteros, no systema de Blanchard, na qual se comprehendem aquelles, cujas asas, durante o repoiso, estão erguidas.
\section{Achaloucado}
\begin{itemize}
\item {Grp. gram.:adj.}
\end{itemize}
\begin{itemize}
\item {Utilização:Prov.}
\end{itemize}
Desajeitado; precipitado.
(Colhido em Turquel)
\section{Achamalotado}
\begin{itemize}
\item {Grp. gram.:adj.}
\end{itemize}
Semelhante a chamalote.
\section{Achamalotar-se}
\begin{itemize}
\item {Grp. gram.:v. p.}
\end{itemize}
\begin{itemize}
\item {Utilização:Neol.}
\end{itemize}
Dar aspecto de chamalote:«\textunderscore o crystal da corrente achamalotou-se\textunderscore ». J. Alencar.
\section{Achamboar}
\begin{itemize}
\item {Grp. gram.:v. t.}
\end{itemize}
Tornar chambão, grosseiro.
\section{Achamboirado}
\begin{itemize}
\item {Grp. gram.:adj.}
\end{itemize}
\begin{itemize}
\item {Proveniência:(De \textunderscore achamboar\textunderscore )}
\end{itemize}
Tôsco, mal feito.
Desajeitado.
\section{Achamento}
\begin{itemize}
\item {Grp. gram.:m.}
\end{itemize}
O mesmo que \textunderscore achada\textunderscore ^1.
\section{Achamorrado}
\begin{itemize}
\item {Grp. gram.:adj.}
\end{itemize}
\begin{itemize}
\item {Utilização:Bras}
\end{itemize}
Achatado.
Rombo. Cp. \textunderscore chamorro\textunderscore .
\section{Achanadamente}
\begin{itemize}
\item {Grp. gram.:adv.}
\end{itemize}
De modo nivelado, \textunderscore achanado\textunderscore .
\section{Achanar}
\begin{itemize}
\item {Grp. gram.:v. t.}
\end{itemize}
\begin{itemize}
\item {Utilização:Ant.}
\end{itemize}
\begin{itemize}
\item {Grp. gram.:V. p.}
\end{itemize}
Tornar chão, plano.
Tornar-se lhano, tratável.
\section{Achanci}
\begin{itemize}
\item {Grp. gram.:m.}
\end{itemize}
Antigo magistrado da ilha de Ainão. (Cf. \textunderscore Peregrinação\textunderscore , XLV, LXXXV)
\section{Achancil}
\begin{itemize}
\item {Grp. gram.:m.}
\end{itemize}
Antigo magistrado da ilha de Ainão. (Cf. \textunderscore Peregrinação\textunderscore , XLV, LXXXV)
\section{Achancilado}
\begin{itemize}
\item {Grp. gram.:m.}
\end{itemize}
Jurisdição de achancil.
\section{Achânia}
\begin{itemize}
\item {Grp. gram.:f.}
\end{itemize}
Planta malvácea da America do Sul.
\section{Achantar}
\begin{itemize}
\item {Grp. gram.:v. t.}
\end{itemize}
\begin{itemize}
\item {Utilização:Ant.}
\end{itemize}
O mesmo que \textunderscore plantar\textunderscore .
\section{Achantis}
\begin{itemize}
\item {Grp. gram.:m. pl.}
\end{itemize}
Povos da Achantia, na África.
\section{Achaparrado}
\begin{itemize}
\item {Grp. gram.:adj.}
\end{itemize}
\begin{itemize}
\item {Proveniência:(De \textunderscore achaparrar\textunderscore )}
\end{itemize}
Semelhante a \textunderscore chaparro\textunderscore .
Grosso e baixo.
\section{Achaparrar}
\begin{itemize}
\item {Grp. gram.:v. i.}
\end{itemize}
\begin{itemize}
\item {Proveniência:(De \textunderscore chaparro\textunderscore )}
\end{itemize}
Engrossar, crescendo pouco em altura (a árvore).
\section{Achaque}
\begin{itemize}
\item {Grp. gram.:m.}
\end{itemize}
\begin{itemize}
\item {Utilização:Ant.}
\end{itemize}
Disposição mórbida.
Doença habitual.
Motivo de queixa, acto de queixar-se.
(Cast. \textunderscore achaque\textunderscore )
\section{Achaquilho}
\begin{itemize}
\item {Grp. gram.:m.}
\end{itemize}
Pequeno \textunderscore achaque\textunderscore .
\section{Achar}
\begin{itemize}
\item {Grp. gram.:v. t.}
\end{itemize}
\begin{itemize}
\item {Proveniência:(Do lat. \textunderscore afflare\textunderscore )}
\end{itemize}
Descobrir.
Inventar.
Julgar.
\section{Achar}
\begin{itemize}
\item {Grp. gram.:m.}
\end{itemize}
\begin{itemize}
\item {Proveniência:(T. mal)}
\end{itemize}
Especie de conserva indiana.
\section{Achária}
\begin{itemize}
\item {fónica:cá}
\end{itemize}
\begin{itemize}
\item {Grp. gram.:f.}
\end{itemize}
Gênero de plantas passiflóreas.
\section{Acharoado}
\begin{itemize}
\item {Grp. gram.:adj.}
\end{itemize}
\begin{itemize}
\item {Proveniência:(De \textunderscore acharoar\textunderscore )}
\end{itemize}
Envernizado como charão.
\section{Acharoar}
\begin{itemize}
\item {Grp. gram.:v. t.}
\end{itemize}
Envernizar como charão.
\section{Achatadela}
\begin{itemize}
\item {Grp. gram.:f.}
\end{itemize}
\begin{itemize}
\item {Utilização:Fam.}
\end{itemize}
Acto de \textunderscore achatar\textunderscore .
\section{Achatado}
\begin{itemize}
\item {Grp. gram.:adj.}
\end{itemize}
\begin{itemize}
\item {Utilização:Fig.}
\end{itemize}
\begin{itemize}
\item {Proveniência:(De \textunderscore achatar\textunderscore ^1)}
\end{itemize}
Humilhado.
\section{Achatadura}
\begin{itemize}
\item {Grp. gram.:f.}
\end{itemize}
O mesmo que \textunderscore achatadela\textunderscore .
\section{Achatamento}
\begin{itemize}
\item {Grp. gram.:m.}
\end{itemize}
Acto de \textunderscore achatar\textunderscore ^1.
\section{Achatar}
\begin{itemize}
\item {Grp. gram.:v. t.}
\end{itemize}
\begin{itemize}
\item {Utilização:Fam.}
\end{itemize}
Tornar chato.
Humilhar, abater.
Vencer em discussão.
\section{Achatar}
\begin{itemize}
\item {Grp. gram.:v. t.}
\end{itemize}
\begin{itemize}
\item {Utilização:Ant.}
\end{itemize}
Conseguir, obter.
(Relaciona-se com o fr. \textunderscore acheter\textunderscore ?)
\section{Achavascado}
\begin{itemize}
\item {Grp. gram.:adj.}
\end{itemize}
\begin{itemize}
\item {Proveniência:(De \textunderscore achavascar\textunderscore )}
\end{itemize}
Grosseiro.
\section{Achavascar}
\begin{itemize}
\item {Grp. gram.:v. t.}
\end{itemize}
Tornar grosseiro, tôsco.
\section{Achega}
\begin{itemize}
\item {fónica:chê}
\end{itemize}
\begin{itemize}
\item {Grp. gram.:f.}
\end{itemize}
\begin{itemize}
\item {Utilização:Fam.}
\end{itemize}
\begin{itemize}
\item {Grp. gram.:M.}
\end{itemize}
\begin{itemize}
\item {Utilização:Ant.}
\end{itemize}
\begin{itemize}
\item {Proveniência:(De \textunderscore achegar\textunderscore )}
\end{itemize}
Aditamento.
Subsídio, auxílio.
Pequeno lucro.
Rendimento accessório ou eventual.
Cada um dos partícipes de um casal, cuja pensão total era paga por um cabecel.
\section{Achegadamente}
\begin{itemize}
\item {Grp. gram.:adv.}
\end{itemize}
Muito de perto.
Proximamente.
\section{Achegador}
\begin{itemize}
\item {Grp. gram.:m.}
\end{itemize}
\begin{itemize}
\item {Utilização:Prov.}
\end{itemize}
\begin{itemize}
\item {Utilização:minh.}
\end{itemize}
\begin{itemize}
\item {Utilização:Ant.}
\end{itemize}
O que achega.
O mesmo que \textunderscore alcoviteiro\textunderscore .
Official de justiça.
\section{Achegamento}
\begin{itemize}
\item {Grp. gram.:m.}
\end{itemize}
Acto de \textunderscore achegar\textunderscore .
\section{Achegança}
\begin{itemize}
\item {Grp. gram.:f.}
\end{itemize}
\begin{itemize}
\item {Utilização:Ant.}
\end{itemize}
\begin{itemize}
\item {Proveniência:(De \textunderscore achegar\textunderscore )}
\end{itemize}
Pertença.
Foragem, pensão.
\section{Achegar}
\begin{itemize}
\item {Grp. gram.:v. t.}
\end{itemize}
\begin{itemize}
\item {Utilização:Ant.}
\end{itemize}
\begin{itemize}
\item {Proveniência:(De \textunderscore chegar\textunderscore )}
\end{itemize}
Aproximar.
Conchegar.
Ajuntar, adquirir. Cf. \textunderscore Rev. Lus.\textunderscore , XVI, 1.
\section{Acheguilho}
\begin{itemize}
\item {Grp. gram.:m.}
\end{itemize}
\begin{itemize}
\item {Utilização:Prov.}
\end{itemize}
\begin{itemize}
\item {Proveniência:(De \textunderscore achêga\textunderscore )}
\end{itemize}
O mesmo que \textunderscore accessório\textunderscore .
\section{Achênio}
\begin{itemize}
\item {fónica:quê}
\end{itemize}
\begin{itemize}
\item {Grp. gram.:adj.}
\end{itemize}
\begin{itemize}
\item {Grp. gram.:M.}
\end{itemize}
Diz-se de um período geológico, criado por Dumond.
Terreno na base da série infra-cretácea, constituído por um conjunto de areias brancas ou ferruginosas e de argila, que cobrem directamente as camadas carboníferas.
(Por \textunderscore aachênio\textunderscore , de \textunderscore Aachen\textunderscore , n. p. al. de Aix-la-Chapelle)
\section{Achens}
\begin{itemize}
\item {Grp. gram.:m. pl.}
\end{itemize}
Habitantes do antigo reino de Achém. Cf. \textunderscore Peregrinação\textunderscore .
\section{Acheronte}
\begin{itemize}
\item {fónica:que}
\end{itemize}
\begin{itemize}
\item {Grp. gram.:m.}
\end{itemize}
\begin{itemize}
\item {Proveniência:(Lat. \textunderscore acheron, acherontis\textunderscore )}
\end{itemize}
O inferno mythológico.
\section{Acheronteu}
\begin{itemize}
\item {fónica:que}
\end{itemize}
\begin{itemize}
\item {Grp. gram.:adj.}
\end{itemize}
O mesmo que \textunderscore acherôntico\textunderscore .
\section{Acherôntia}
\begin{itemize}
\item {fónica:que}
\end{itemize}
\begin{itemize}
\item {Grp. gram.:f.}
\end{itemize}
Insecto, cuja larva ataca e destrói as flôres do tabaco.
\section{Acherôntico}
\begin{itemize}
\item {fónica:que}
\end{itemize}
\begin{itemize}
\item {Grp. gram.:adj.}
\end{itemize}
Relativo ao acheronte.
\section{Acheu}
\begin{itemize}
\item {fónica:queu}
\end{itemize}
\begin{itemize}
\item {Grp. gram.:adj.}
\end{itemize}
\begin{itemize}
\item {Grp. gram.:M.}
\end{itemize}
\begin{itemize}
\item {Grp. gram.:Pl.}
\end{itemize}
\begin{itemize}
\item {Proveniência:(Lat. \textunderscore achaeus\textunderscore )}
\end{itemize}
Relativo á Achaia; achaico.
Habitante da Achaia.
Antigo povo da Grécia, o mesmo que \textunderscore achivos\textunderscore .
\section{Achibantado}
\begin{itemize}
\item {Grp. gram.:adj.}
\end{itemize}
Que tem modos de chibante.
\section{Achicar}
\begin{itemize}
\item {Grp. gram.:v. i.}
\end{itemize}
\begin{itemize}
\item {Grp. gram.:V. t.}
\end{itemize}
Enxugar.
Esgotar-se (a água das embarcações).
Enxugar, esgotar.
\section{Achicarar}
\begin{itemize}
\item {Grp. gram.:v. t.}
\end{itemize}
Dar feitio de chícara a.
\section{Achilária}
\begin{itemize}
\item {fónica:qui}
\end{itemize}
\begin{itemize}
\item {Grp. gram.:f.}
\end{itemize}
Monstruosidade vegetal, caracterizada pela ausência accidental dos lábios em corollas que normalmente os têm.
\section{Achilia}
\begin{itemize}
\item {fónica:qui}
\end{itemize}
\begin{itemize}
\item {Grp. gram.:f.}
\end{itemize}
\begin{itemize}
\item {Proveniência:(Do gr. \textunderscore a\textunderscore  priv. + \textunderscore kheilos\textunderscore , lábio)}
\end{itemize}
Monstruosidade, caracterizada pela falta de lábios.
\section{Achilleas}
\begin{itemize}
\item {fónica:qui}
\end{itemize}
\begin{itemize}
\item {Grp. gram.:f. pl.}
\end{itemize}
\begin{itemize}
\item {Proveniência:(De \textunderscore achilleia\textunderscore )}
\end{itemize}
Grupo de plantas corymbíferas, segundo Jussieu.
\section{Achilleia}
\begin{itemize}
\item {fónica:qui}
\end{itemize}
\begin{itemize}
\item {Grp. gram.:f.}
\end{itemize}
\begin{itemize}
\item {Proveniência:(Gr. \textunderscore akhilleia\textunderscore )}
\end{itemize}
Planta, de flôres radiadas, dispostas em corymbo.
\section{Achilleico}
\begin{itemize}
\item {fónica:qui}
\end{itemize}
\begin{itemize}
\item {Grp. gram.:adj.}
\end{itemize}
Diz-se do ácido que existe na achilleia.
\section{Achilleja}
\begin{itemize}
\item {fónica:qui}
\end{itemize}
\begin{itemize}
\item {Grp. gram.:f.}
\end{itemize}
O mesmo que \textunderscore achilleia\textunderscore .
\section{Achim}
\begin{itemize}
\item {Grp. gram.:m.}
\end{itemize}
Especie de pimentão indiano.
\section{Achim}
\begin{itemize}
\item {Grp. gram.:m.}
\end{itemize}
Lingua de Samatra.
\section{Achinar}
\begin{itemize}
\item {Grp. gram.:v. t.}
\end{itemize}
\begin{itemize}
\item {Proveniência:(De \textunderscore china\textunderscore )}
\end{itemize}
Dar forma chinesa a.
\section{Achinar}
\begin{itemize}
\item {Grp. gram.:v. t.}
\end{itemize}
\begin{itemize}
\item {Utilização:Prov.}
\end{itemize}
\begin{itemize}
\item {Utilização:trasm.}
\end{itemize}
\begin{itemize}
\item {Proveniência:(De \textunderscore chino\textunderscore ^2)}
\end{itemize}
Marcar com a pedra, chamada \textunderscore chino\textunderscore , (o lugar, onde o ferro bateu, no jôgo da barra).
\section{Achincalhação}
\begin{itemize}
\item {Grp. gram.:f.}
\end{itemize}
Acto ou effeito de \textunderscore achincalhar\textunderscore .
\section{Achincalhamento}
\begin{itemize}
\item {Grp. gram.:m.}
\end{itemize}
O mesmo que \textunderscore achincalhação\textunderscore .
\section{Achincalhar}
\begin{itemize}
\item {Grp. gram.:v. t.}
\end{itemize}
\begin{itemize}
\item {Proveniência:(De \textunderscore chinquilho\textunderscore ?)}
\end{itemize}
Tornar vil.
Ridiculizar, chacotear.
\section{Achincalhe}
\begin{itemize}
\item {Grp. gram.:m.}
\end{itemize}
O mesmo que \textunderscore achincalhação\textunderscore .
\section{Achincalho}
\begin{itemize}
\item {Grp. gram.:m.}
\end{itemize}
O mesmo que \textunderscore achincalhação\textunderscore .
\section{Achinelado}
\begin{itemize}
\item {Grp. gram.:adj.}
\end{itemize}
Que tem fórma de chinelo.
\section{Achinelar}
\begin{itemize}
\item {Grp. gram.:v. t.}
\end{itemize}
Dar fórma de chinela a.
\section{Achinesar}
\begin{itemize}
\item {Grp. gram.:v. t.}
\end{itemize}
O mesmo que \textunderscore achinar\textunderscore ^1.
\section{Achiota}
\begin{itemize}
\item {Grp. gram.:f.}
\end{itemize}
Fruto do achiote.
\section{Achiote}
\begin{itemize}
\item {Grp. gram.:m.}
\end{itemize}
Árvore americana, semelhante á laranjeira.
\section{Achiro}
\begin{itemize}
\item {fónica:qui}
\end{itemize}
\begin{itemize}
\item {Grp. gram.:m.}
\end{itemize}
\begin{itemize}
\item {Proveniência:(Do gr. \textunderscore a\textunderscore  priv. + \textunderscore ckeir\textunderscore , mão)}
\end{itemize}
Peixe pleuronecto, semelhante ao linguado.
\section{Achivo}
\begin{itemize}
\item {fónica:qui}
\end{itemize}
\begin{itemize}
\item {Grp. gram.:m. e adj.}
\end{itemize}
\begin{itemize}
\item {Proveniência:(Lat. \textunderscore achivus\textunderscore )}
\end{itemize}
O mesmo que \textunderscore grego\textunderscore  da Thessália ou do Peloponneso; grego.
\section{Achlâmydas}
\begin{itemize}
\item {Grp. gram.:adj. f. pl.}
\end{itemize}
\begin{itemize}
\item {Proveniência:(Do gr. \textunderscore a\textunderscore  priv. + \textunderscore khlamus\textunderscore )}
\end{itemize}
Diz-se das algas, cujos filamentos são desprovidos de segundo envoltório.
\section{Achnantho}
\begin{itemize}
\item {Grp. gram.:m.}
\end{itemize}
\begin{itemize}
\item {Proveniência:(Do gr. \textunderscore achne\textunderscore  + \textunderscore anthos\textunderscore )}
\end{itemize}
Alga microscópica, diatomácea.
\section{...acho}
\begin{itemize}
\item {Grp. gram.:suf.}
\end{itemize}
(designativo de deminuição ou depreciação)
\section{Achoar}
\begin{itemize}
\item {Grp. gram.:v.}
\end{itemize}
\begin{itemize}
\item {Utilização:t. Marn.}
\end{itemize}
\begin{itemize}
\item {Utilização:Prov.}
\end{itemize}
\begin{itemize}
\item {Proveniência:(De \textunderscore chão\textunderscore )}
\end{itemize}
Recalcar com os pés.
Deitar ao chão.
Moer com pancadas.
\section{Acholia}
\begin{itemize}
\item {fónica:co}
\end{itemize}
\begin{itemize}
\item {Grp. gram.:f.}
\end{itemize}
\begin{itemize}
\item {Utilização:Med.}
\end{itemize}
\begin{itemize}
\item {Proveniência:(Do gr. \textunderscore a\textunderscore  priv. + \textunderscore khole\textunderscore , bile)}
\end{itemize}
Suppressão da secreção biliar.
\section{Achonaris}
\begin{itemize}
\item {Grp. gram.:m. pl.}
\end{itemize}
Aborígenes brasileiros, que habitaram no Pará.
\section{Achores}
\begin{itemize}
\item {fónica:cô}
\end{itemize}
\begin{itemize}
\item {Grp. gram.:m. pl.}
\end{itemize}
\begin{itemize}
\item {Proveniência:(Gr. \textunderscore akhor\textunderscore )}
\end{itemize}
Tinha mucosa.
\section{Achromasia}
\begin{itemize}
\item {Grp. gram.:f.}
\end{itemize}
\begin{itemize}
\item {Proveniência:(Do gr. \textunderscore a\textunderscore  priv. + \textunderscore khroma\textunderscore , côr)}
\end{itemize}
Pallidez cachéctica.
\section{Achromático}
\begin{itemize}
\item {Grp. gram.:adj.}
\end{itemize}
\begin{itemize}
\item {Proveniência:(Do gr. \textunderscore a\textunderscore  priv. + \textunderscore khroma\textunderscore , côr)}
\end{itemize}
Que faz desapparecer as irisações produzidas por certas lentes.
\section{Achromatina}
\begin{itemize}
\item {Grp. gram.:f.}
\end{itemize}
\begin{itemize}
\item {Proveniência:(Gr. \textunderscore akhromatos\textunderscore , sem côr)}
\end{itemize}
Parte da substância do núcleo cellular, sôbre a qual não têm acção os reagentes còrantes.
\section{Achromatismo}
\begin{itemize}
\item {Grp. gram.:m.}
\end{itemize}
Qualidade do objecto \textunderscore achromático\textunderscore .
\section{Achromatização}
\begin{itemize}
\item {Grp. gram.:f.}
\end{itemize}
Acto de \textunderscore achromatizar\textunderscore .
\section{Achromatizar}
\begin{itemize}
\item {Grp. gram.:v. t.}
\end{itemize}
\begin{itemize}
\item {Proveniência:(De \textunderscore achromático\textunderscore )}
\end{itemize}
Fazer desapparecer (as côres irisadas) na imagem de um objecto.
\section{Achromatopsia}
\begin{itemize}
\item {Grp. gram.:f.}
\end{itemize}
\begin{itemize}
\item {Proveniência:(Do gr. \textunderscore a\textunderscore  priv. + \textunderscore khroma\textunderscore  + \textunderscore ops\textunderscore )}
\end{itemize}
Estado de quem não póde distinguir as côres.
\section{Achromatóptico}
\begin{itemize}
\item {Grp. gram.:adj.}
\end{itemize}
Que tem achromatopsia.
\section{Achromia}
\begin{itemize}
\item {Grp. gram.:f.}
\end{itemize}
\begin{itemize}
\item {Utilização:Med.}
\end{itemize}
\begin{itemize}
\item {Proveniência:(Do gr. \textunderscore a\textunderscore  priv. + \textunderscore khroma\textunderscore , côr)}
\end{itemize}
Descoramento parcial da pelle.
\section{Achromo}
\begin{itemize}
\item {Grp. gram.:adj.}
\end{itemize}
\begin{itemize}
\item {Proveniência:(Do gr. \textunderscore a\textunderscore  priv. + \textunderscore khroma\textunderscore , côr)}
\end{itemize}
Que não tem côr.
\section{Achromodermia}
\begin{itemize}
\item {Grp. gram.:f.}
\end{itemize}
O mesmo que \textunderscore achromasia\textunderscore .
\section{Achromolena}
\begin{itemize}
\item {Grp. gram.:f.}
\end{itemize}
\begin{itemize}
\item {Proveniência:(Do gr. \textunderscore chroma\textunderscore  + \textunderscore laina\textunderscore )}
\end{itemize}
Planta composta, originária da Nova Hollanda.
\section{Achtheographia}
\begin{itemize}
\item {Grp. gram.:f.}
\end{itemize}
\begin{itemize}
\item {Proveniência:(Do gr. \textunderscore achthos\textunderscore  + \textunderscore graphein\textunderscore )}
\end{itemize}
Descripção ou nomenclatura dos pesos.
\section{Achtheómetro}
\begin{itemize}
\item {Grp. gram.:m.}
\end{itemize}
\begin{itemize}
\item {Proveniência:(Do gr. \textunderscore akhtos\textunderscore  + \textunderscore metron\textunderscore )}
\end{itemize}
Instrumento, para medir o pêso dos carros sôbre as rodas.
\section{Achumbar}
\begin{itemize}
\item {Grp. gram.:v. t.}
\end{itemize}
Tornar semelhante ao chumbo.
\section{Achyrantho}
\begin{itemize}
\item {fónica:qui}
\end{itemize}
\begin{itemize}
\item {Grp. gram.:m.}
\end{itemize}
\begin{itemize}
\item {Proveniência:(Do gr. \textunderscore akhuron\textunderscore  + \textunderscore anthos\textunderscore )}
\end{itemize}
Gênero de plantas amarantáceas.
\section{Achyrastro}
\begin{itemize}
\item {fónica:qui}
\end{itemize}
\begin{itemize}
\item {Grp. gram.:m.}
\end{itemize}
\begin{itemize}
\item {Proveniência:(Do gr. \textunderscore akhuron\textunderscore  + \textunderscore astron\textunderscore )}
\end{itemize}
Planta do grupo das chicoriáceas, e cujo cálice tem a fórma de martinete.
\section{Achyróphoro}
\begin{itemize}
\item {fónica:qui}
\end{itemize}
\begin{itemize}
\item {Grp. gram.:m.}
\end{itemize}
\begin{itemize}
\item {Proveniência:(Do gr. \textunderscore akhuron\textunderscore  + \textunderscore phoros\textunderscore )}
\end{itemize}
Gênero de plantas compostas.
\section{Achyróphyto}
\begin{itemize}
\item {fónica:qui}
\end{itemize}
\begin{itemize}
\item {Grp. gram.:adj.}
\end{itemize}
\begin{itemize}
\item {Proveniência:(Do gr. \textunderscore akhuron\textunderscore  + \textunderscore phuton\textunderscore )}
\end{itemize}
Diz-se da planta, cuja flôr é composta de palhetas.
\section{Achyrospermo}
\begin{itemize}
\item {fónica:aki}
\end{itemize}
\begin{itemize}
\item {Proveniência:(Do gr. \textunderscore akhuron\textunderscore  + \textunderscore sperma\textunderscore )}
\end{itemize}
Gênero de plantas labiadas.
\section{Aciano}
\begin{itemize}
\item {Grp. gram.:m.}
\end{itemize}
Designação de uma flôr, (\textunderscore acianus maior\textunderscore ).
\section{Aciantho}
\begin{itemize}
\item {Grp. gram.:m.}
\end{itemize}
\begin{itemize}
\item {Proveniência:(Do gr. \textunderscore akis\textunderscore  + \textunderscore anthe\textunderscore )}
\end{itemize}
Planta, da família das orchídeas.
\section{Acianto}
\begin{itemize}
\item {Grp. gram.:m.}
\end{itemize}
\begin{itemize}
\item {Proveniência:(Do gr. \textunderscore akis\textunderscore  + \textunderscore anthe\textunderscore )}
\end{itemize}
Planta, da família das orchídeas.
\section{Acicalar}
\begin{itemize}
\item {Grp. gram.:v. t.}
\end{itemize}
(V.açacalar)
\section{Acicárfio}
\begin{itemize}
\item {Grp. gram.:m.}
\end{itemize}
\begin{itemize}
\item {Proveniência:(Do gr. \textunderscore akis\textunderscore  + \textunderscore karphos\textunderscore )}
\end{itemize}
Planta da América do Sul.
\section{Acicárphio}
\begin{itemize}
\item {Grp. gram.:m.}
\end{itemize}
\begin{itemize}
\item {Proveniência:(Do gr. \textunderscore akis\textunderscore  + \textunderscore karphos\textunderscore )}
\end{itemize}
Planta da América do Sul.
\section{Acicate}
\begin{itemize}
\item {Grp. gram.:m.}
\end{itemize}
\begin{itemize}
\item {Utilização:Fig.}
\end{itemize}
Espora antiga, de uma só ponta.
Espora.
Incentivo.
(Do ár.)
\section{Acicoca}
\begin{itemize}
\item {Grp. gram.:f.}
\end{itemize}
Erva medicinal do Peru.
\section{Acícula}
\begin{itemize}
\item {Grp. gram.:f.}
\end{itemize}
\begin{itemize}
\item {Proveniência:(Lat. \textunderscore acicula\textunderscore )}
\end{itemize}
Gênero de molluscos gaterópodes.
Gancho de osso, de metal ou madeira, com que as damas romanas seguravam os cabellos.
\section{Aciculado}
\begin{itemize}
\item {Grp. gram.:adj.}
\end{itemize}
\begin{itemize}
\item {Proveniência:(De \textunderscore acícula\textunderscore )}
\end{itemize}
Que tem fórma de agulha.
\section{Acicular}
\begin{itemize}
\item {Grp. gram.:adj.}
\end{itemize}
O mesmo que \textunderscore aciculado\textunderscore .
\section{Acidação}
\begin{itemize}
\item {Grp. gram.:f.}
\end{itemize}
\begin{itemize}
\item {Utilização:Chím.}
\end{itemize}
Acção de converter em ácido.
\section{Acidade}
\begin{itemize}
\item {Grp. gram.:f.}
\end{itemize}
(V.acidez)
\section{Acidália}
\begin{itemize}
\item {Grp. gram.:f.}
\end{itemize}
\begin{itemize}
\item {Proveniência:(Lat. \textunderscore Acidalia\textunderscore , n. p.)}
\end{itemize}
Gênero de lepidópteros nocturnos.
\section{Acidante}
\begin{itemize}
\item {Grp. gram.:adj.}
\end{itemize}
\begin{itemize}
\item {Utilização:Chím.}
\end{itemize}
Que faz mudar em ácido.
\section{Acidar}
\begin{itemize}
\item {Grp. gram.:v.}
\end{itemize}
\begin{itemize}
\item {Utilização:t. Chím.}
\end{itemize}
Mudar em ácido.
\section{Acidável}
\begin{itemize}
\item {Grp. gram.:adj.}
\end{itemize}
Que póde converter-se em ácido.
\section{Acidentação}
\begin{itemize}
\item {Grp. gram.:f.}
\end{itemize}
\begin{itemize}
\item {Proveniência:(De \textunderscore accidentar\textunderscore )}
\end{itemize}
Qualidade de um terreno acidentado.
\section{Acidentado}
\begin{itemize}
\item {Grp. gram.:adj.}
\end{itemize}
Que não é plano: \textunderscore região acidentada\textunderscore .
\section{Acidental}
\begin{itemize}
\item {Grp. gram.:adj.}
\end{itemize}
\begin{itemize}
\item {Utilização:Mús.}
\end{itemize}
\begin{itemize}
\item {Proveniência:(De \textunderscore accidente\textunderscore )}
\end{itemize}
Fortuito, casual, imprevisto.
Acessório.
Diz-se de cada uma das notas, que não fazem parte integrante dos acordes.
\section{Acidentalmente}
\begin{itemize}
\item {Grp. gram.:adv.}
\end{itemize}
De modo \textunderscore acidental\textunderscore .
\section{Acidentar}
\begin{itemize}
\item {Grp. gram.:v. t.}
\end{itemize}
\begin{itemize}
\item {Utilização:Prov.}
\end{itemize}
\begin{itemize}
\item {Utilização:trasm.}
\end{itemize}
\begin{itemize}
\item {Grp. gram.:v. t.}
\end{itemize}
O mesmo que \textunderscore acinzentar\textunderscore .
Variar.
Produzir acidente em.
\section{Acidentariamente}
\begin{itemize}
\item {Grp. gram.:adv.}
\end{itemize}
De modo \textunderscore acidentário\textunderscore .
\section{Acidentário}
\begin{itemize}
\item {Grp. gram.:adj.}
\end{itemize}
O mesmo que \textunderscore acidental\textunderscore .
\section{Acidentável}
\begin{itemize}
\item {Grp. gram.:adj.}
\end{itemize}
Que se póde \textunderscore acidentar\textunderscore .
\section{Acidente}
\begin{itemize}
\item {Grp. gram.:m.}
\end{itemize}
\begin{itemize}
\item {Utilização:Pop.}
\end{itemize}
\begin{itemize}
\item {Proveniência:(Lat. \textunderscore accideno\textunderscore )}
\end{itemize}
O que é casual, fortuito.
Desgraça.
Disposição variada de terreno.
Variada distribuição de luz.
Ataque epiléptico; sýncope, desmaio.
\section{Acidez}
\begin{itemize}
\item {Grp. gram.:f.}
\end{itemize}
Propriedade das coisas ácidas.
\section{Acidífero}
\begin{itemize}
\item {Grp. gram.:adj.}
\end{itemize}
\begin{itemize}
\item {Proveniência:(Do lat. \textunderscore acidus\textunderscore  + \textunderscore ferre\textunderscore )}
\end{itemize}
Que tem ou produz ácido.
\section{Acidificação}
\begin{itemize}
\item {Grp. gram.:f.}
\end{itemize}
Acto de \textunderscore acidificar\textunderscore .
\section{Acidificante}
\begin{itemize}
\item {Grp. gram.:adj.}
\end{itemize}
Que acidifica.
\section{Acidificar}
\begin{itemize}
\item {Grp. gram.:v. t.}
\end{itemize}
\begin{itemize}
\item {Proveniência:(Do lat. \textunderscore acidus\textunderscore  + \textunderscore facere\textunderscore )}
\end{itemize}
Converter em ácido.
\section{Acidimetria}
\begin{itemize}
\item {Grp. gram.:f.}
\end{itemize}
Applicação do \textunderscore acidímetro\textunderscore .
\section{Acidímetro}
\begin{itemize}
\item {Grp. gram.:m.}
\end{itemize}
\begin{itemize}
\item {Proveniência:(De \textunderscore ácido\textunderscore  + gr. \textunderscore metron\textunderscore )}
\end{itemize}
Apparelho para medir o ácido contido num líquido.
\section{Ácido}
\begin{itemize}
\item {Grp. gram.:m.}
\end{itemize}
\begin{itemize}
\item {Grp. gram.:Adj.}
\end{itemize}
\begin{itemize}
\item {Grp. gram.:M. pl.}
\end{itemize}
\begin{itemize}
\item {Proveniência:(Lat. \textunderscore acidus\textunderscore )}
\end{itemize}
Corpo, que se combina com uma base, para formar saes, em Chímica.
Substância azêda.
Azêdo.
Acre.
Corpos, derivados dos álcooes e dos aldehidos, por oxydação.
\section{Acidopirástica}
\begin{itemize}
\item {Grp. gram.:f.}
\end{itemize}
\begin{itemize}
\item {Utilização:Med.}
\end{itemize}
Exploração ou sondagem das partes profundas de um organismo.
\section{Acidotão}
\begin{itemize}
\item {Grp. gram.:m.}
\end{itemize}
\begin{itemize}
\item {Proveniência:(Do gr. \textunderscore akidotos\textunderscore )}
\end{itemize}
Planta euphorbiácea da Jamaica.
\section{Acidrado}
\begin{itemize}
\item {Grp. gram.:adj.}
\end{itemize}
Semelhante á cidra.
\section{Acidulacão}
\begin{itemize}
\item {Grp. gram.:f.}
\end{itemize}
Acto de \textunderscore acidular\textunderscore . Cf. F. Lapa, \textunderscore Techn. Rur.\textunderscore , 477.
\section{Acidulante}
\begin{itemize}
\item {Grp. gram.:adj.}
\end{itemize}
Que acidula.
\section{Acidular}
\begin{itemize}
\item {Grp. gram.:v. t.}
\end{itemize}
Tornar acídulo.
\section{Acídulo}
\begin{itemize}
\item {Grp. gram.:adj.}
\end{itemize}
\begin{itemize}
\item {Proveniência:(Lat. \textunderscore acidulus\textunderscore )}
\end{itemize}
Levemente ácido.
\section{Aciganar-se}
\begin{itemize}
\item {Grp. gram.:v. p.}
\end{itemize}
Tomar modos de cigano.
Tornar-se manhoso ou trapaceiro.
\section{Acima}
\begin{itemize}
\item {Grp. gram.:adj.}
\end{itemize}
Para a parte superior.
Em cima.
(Cp. \textunderscore cima\textunderscore ^1)
\section{Á-cima}
\begin{itemize}
\item {Grp. gram.:loc. adv.}
\end{itemize}
\begin{itemize}
\item {Utilização:Ant.}
\end{itemize}
Por fim, afinal.
\section{Acimar}
\begin{itemize}
\item {Grp. gram.:v. t.}
\end{itemize}
\begin{itemize}
\item {Utilização:Ant.}
\end{itemize}
Chegar ao cimo ou ao termo de.
Concluir, acabar.
\section{Acimento}
\begin{itemize}
\item {Grp. gram.:m.}
\end{itemize}
\begin{itemize}
\item {Utilização:Des.}
\end{itemize}
Cimo, elevação, cume.
(Cp. \textunderscore acima\textunderscore )
\section{Acinace}
\begin{itemize}
\item {Grp. gram.:m.}
\end{itemize}
\begin{itemize}
\item {Proveniência:(Lat. \textunderscore acinaces\textunderscore )}
\end{itemize}
Sabre curto e curvo, usado outrora por alguns povos orientaes.
\section{Acináceo}
\begin{itemize}
\item {Grp. gram.:m.}
\end{itemize}
\begin{itemize}
\item {Proveniência:(Do gr. \textunderscore akinakes\textunderscore )}
\end{itemize}
Peixe do Atlántico, em fórma de folha de sabre.
\section{Acinaciforme}
\begin{itemize}
\item {Grp. gram.:adj.}
\end{itemize}
\begin{itemize}
\item {Proveniência:(Lat. \textunderscore acinaciformis\textunderscore )}
\end{itemize}
Que tem fórma de sabre, (falando-se das folhas de certos vegetaes).
\section{Acinesia}
\begin{itemize}
\item {Grp. gram.:f.}
\end{itemize}
\begin{itemize}
\item {Proveniência:(Do gr. \textunderscore a.\textunderscore  priv. + \textunderscore kinein\textunderscore )}
\end{itemize}
Immobilidade.
\section{Acineta}
\begin{itemize}
\item {Grp. gram.:f.}
\end{itemize}
\begin{itemize}
\item {Proveniência:(Do gr. \textunderscore akinetos\textunderscore )}
\end{itemize}
Animálculo, da classe dos rhizópodes.
\section{Acinetinas}
\begin{itemize}
\item {Grp. gram.:f. pl.}
\end{itemize}
\begin{itemize}
\item {Proveniência:(De \textunderscore acineta\textunderscore )}
\end{itemize}
Infusórios, de uma só abertura e cílios alongados.
\section{Acingir}
\begin{itemize}
\item {Grp. gram.:v. t.}
\end{itemize}
\begin{itemize}
\item {Utilização:Des.}
\end{itemize}
O mesmo que \textunderscore cingir\textunderscore  (especialmente falando de armas). Cf. Fr. Fort., \textunderscore Inéd.\textunderscore , II, 62. Lat. \textunderscore accingere\textunderscore .
\section{Acinia}
\begin{itemize}
\item {Grp. gram.:f.}
\end{itemize}
\begin{itemize}
\item {Proveniência:(Do gr. \textunderscore akinos\textunderscore )}
\end{itemize}
Insecto diptero.
\section{Ácino}
\begin{itemize}
\item {Grp. gram.:m.}
\end{itemize}
\begin{itemize}
\item {Proveniência:(Lat. \textunderscore acinus\textunderscore )}
\end{itemize}
Grão de uva.
Nome de uma baga molle e transparente, de grãos duros.
\section{Acinóforo}
\begin{itemize}
\item {Grp. gram.:m.}
\end{itemize}
\begin{itemize}
\item {Proveniência:(Do gr. \textunderscore akinos\textunderscore  + \textunderscore phoros\textunderscore )}
\end{itemize}
Gênero de cogumêlos.
\section{Acinóphoro}
\begin{itemize}
\item {Grp. gram.:m.}
\end{itemize}
\begin{itemize}
\item {Proveniência:(Do gr. \textunderscore akinos\textunderscore  + \textunderscore phoros\textunderscore )}
\end{itemize}
Gênero de cogumêlos.
\section{Acinopo}
\begin{itemize}
\item {Grp. gram.:m.}
\end{itemize}
\begin{itemize}
\item {Proveniência:(Do gr. \textunderscore akinos\textunderscore  + \textunderscore pous\textunderscore )}
\end{itemize}
Insecto coleóptero pentâmero.
\section{Acinte}
\begin{itemize}
\item {Grp. gram.:m.}
\end{itemize}
\begin{itemize}
\item {Grp. gram.:Adv.}
\end{itemize}
Acção ou propósito de fazer alguma coisa, contra a vontade ou gôsto de alguém.
De propósito, de caso pensado.
\section{Acintemente}
\begin{itemize}
\item {Grp. gram.:adv.}
\end{itemize}
Por acinte.
De modo acintoso.
\section{Acíntli}
\begin{itemize}
\item {Grp. gram.:f.}
\end{itemize}
Especie de gallinha, de plumagem escura e branca, no México.
(\textunderscore gallina purpurea\textunderscore , Lin.)
\section{Acintosamente}
\begin{itemize}
\item {Grp. gram.:adv.}
\end{itemize}
Acintemente, de modo \textunderscore acintoso\textunderscore .
\section{Acintoso}
\begin{itemize}
\item {Grp. gram.:adj.}
\end{itemize}
Em que há acinte.
Em que há propósito ou mau propósito.
\section{Acinzado}
\begin{itemize}
\item {Grp. gram.:adj.}
\end{itemize}
\begin{itemize}
\item {Proveniência:(De \textunderscore acinzar\textunderscore )}
\end{itemize}
Que tem côr de cinza.
\section{Acinzador}
\begin{itemize}
\item {Grp. gram.:m.}
\end{itemize}
O que acinza.
\section{Acinzamento}
\begin{itemize}
\item {Grp. gram.:m.}
\end{itemize}
Acto de \textunderscore acinzar\textunderscore .
\section{Acinzar}
\begin{itemize}
\item {Grp. gram.:v. t.}
\end{itemize}
Dar côr de cinza a.
\section{Acinzeirado}
\begin{itemize}
\item {Grp. gram.:adj.}
\end{itemize}
O mesmo que \textunderscore acinzentado\textunderscore .
\section{Acinzelar}
\textunderscore v. t.\textunderscore  (e der.)
O mesmo que \textunderscore cinzelar\textunderscore , etc.
\section{Acinzentado}
\begin{itemize}
\item {Grp. gram.:adj.}
\end{itemize}
\begin{itemize}
\item {Proveniência:(De \textunderscore acinzentar\textunderscore )}
\end{itemize}
Cuja côr é tirante a cinzento. Cf. Herculano, \textunderscore Lend. e Narr.\textunderscore , I, 90.
\section{Acinzentar}
\begin{itemize}
\item {Grp. gram.:v. t.}
\end{itemize}
Tornar acinzentado.
\section{Aciôa}
\begin{itemize}
\item {Grp. gram.:f.}
\end{itemize}
Planta das Guianas.
\section{Acipitrário}
\begin{itemize}
\item {Grp. gram.:m.}
\end{itemize}
\begin{itemize}
\item {Proveniência:(Do lat. \textunderscore accipiter\textunderscore )}
\end{itemize}
Armadilha para aves de rapina.
\section{Acípitres}
\begin{itemize}
\item {Grp. gram.:m. pl.}
\end{itemize}
\begin{itemize}
\item {Proveniência:(Lat. \textunderscore accipiter\textunderscore )}
\end{itemize}
Primeiro grupo das aves de rapina, segundo Cuvier.
\section{Acipitrianos}
\begin{itemize}
\item {Grp. gram.:m. pl.}
\end{itemize}
\begin{itemize}
\item {Proveniência:(Do lat. \textunderscore accipiter\textunderscore )}
\end{itemize}
Sub-família dos acipitrídeos.
\section{Acipitrídeos}
\begin{itemize}
\item {Grp. gram.:m. pl.}
\end{itemize}
\begin{itemize}
\item {Proveniência:(Do lat. \textunderscore accipiter\textunderscore  + gr. \textunderscore eidos\textunderscore )}
\end{itemize}
O mesmo que falconídeos.
\section{Acipitrino}
\begin{itemize}
\item {Grp. gram.:adj.}
\end{itemize}
\begin{itemize}
\item {Proveniência:(Do lat. \textunderscore accipiter\textunderscore )}
\end{itemize}
Relativo a aves de rapina.
\section{Acioto}
\begin{itemize}
\item {fónica:ôto}
\end{itemize}
\begin{itemize}
\item {Grp. gram.:m.}
\end{itemize}
\begin{itemize}
\item {Proveniência:(Do gr. \textunderscore akis\textunderscore )}
\end{itemize}
Planta melastomácea das Antilhas.
\section{Acipipe}
\begin{itemize}
\item {Grp. gram.:m.}
\end{itemize}
(V.acepipe)
\section{Aciprestal}
\begin{itemize}
\item {Grp. gram.:m.}
\end{itemize}
Lugar, onde crescem aciprestes. Cf. Castilho, \textunderscore Fastos\textunderscore , III, 13.
\section{Acipreste}
\begin{itemize}
\item {Grp. gram.:m.}
\end{itemize}
O mesmo que \textunderscore cipreste\textunderscore .
\section{Acirandar}
\begin{itemize}
\item {Grp. gram.:v. t.}
\end{itemize}
Limpar com ciranda.
Cirandar.
\section{Acirrado}
\begin{itemize}
\item {Grp. gram.:adj.}
\end{itemize}
Contumaz, frenético.
\section{Acirrante}
\begin{itemize}
\item {Grp. gram.:adj.}
\end{itemize}
\begin{itemize}
\item {Grp. gram.:M.}
\end{itemize}
Que acirra.
Picante, (falando-se de comida).
Aperitivo, estimulante. Cf. Camillo, \textunderscore Narcót.\textunderscore , I, 169.
\section{Acirrar}
\begin{itemize}
\item {Grp. gram.:v. t.}
\end{itemize}
Açular.
Irritar.
\section{Acisantero}
\begin{itemize}
\item {Grp. gram.:m.}
\end{itemize}
\begin{itemize}
\item {Proveniência:(Do gr. \textunderscore akis\textunderscore  + \textunderscore antheros\textunderscore )}
\end{itemize}
Planta da Jamaica.
\section{Acisanthero}
\begin{itemize}
\item {Grp. gram.:m.}
\end{itemize}
\begin{itemize}
\item {Proveniência:(Do gr. \textunderscore akis\textunderscore  + \textunderscore antheros\textunderscore )}
\end{itemize}
Planta da Jamaica.
\section{Acisba}
\begin{itemize}
\item {Grp. gram.:m.}
\end{itemize}
Insecto coleóptero.
\section{Aciselar}
\begin{itemize}
\item {Grp. gram.:v. t.}
\end{itemize}
\begin{itemize}
\item {Utilização:Gal}
\end{itemize}
(perpetrado por Camillo, em vez de \textunderscore cinzelar\textunderscore , nas \textunderscore Scenas da Foz\textunderscore , 12.)
\section{Acitara}
\begin{itemize}
\item {Grp. gram.:f.}
\end{itemize}
\begin{itemize}
\item {Utilização:Ant.}
\end{itemize}
Espécie de cobertura.
\section{Acizentar}
\begin{itemize}
\item {Grp. gram.:v. t.}
\end{itemize}
\begin{itemize}
\item {Utilização:Prov.}
\end{itemize}
\begin{itemize}
\item {Utilização:trasm.}
\end{itemize}
Açular, incitar.
\section{Aclamação}
\begin{itemize}
\item {Grp. gram.:f.}
\end{itemize}
\begin{itemize}
\item {Proveniência:(Lat. \textunderscore acclamatio\textunderscore )}
\end{itemize}
Acto de aclamar.
\section{Aclamador}
\begin{itemize}
\item {Grp. gram.:m.}
\end{itemize}
\begin{itemize}
\item {Proveniência:(De \textunderscore acclamar\textunderscore )}
\end{itemize}
O que aclama.
\section{Aclamar}
\begin{itemize}
\item {Grp. gram.:v. t.}
\end{itemize}
\begin{itemize}
\item {Proveniência:(Lat. \textunderscore acclamare\textunderscore )}
\end{itemize}
Applaudir ou approvar, bradando.
Saudar.
Proclamar, eleger por aclamação.
\section{Aclâmidas}
\begin{itemize}
\item {Grp. gram.:adj. f. pl.}
\end{itemize}
\begin{itemize}
\item {Proveniência:(Do gr. \textunderscore a\textunderscore  priv. + \textunderscore khlamus\textunderscore )}
\end{itemize}
Diz-se das algas, cujos filamentos são desprovidos de segundo envoltório.
\section{Aclaração}
\begin{itemize}
\item {Grp. gram.:f.}
\end{itemize}
Acto ou effeito de \textunderscore aclarar\textunderscore .
\section{Aclaradamente}
\begin{itemize}
\item {Grp. gram.:adv.}
\end{itemize}
De modo \textunderscore aclarado\textunderscore .
\section{Aclaramento}
\begin{itemize}
\item {Grp. gram.:m.}
\end{itemize}
(V.aclaração)
\section{Aclarar}
\begin{itemize}
\item {Grp. gram.:v. t.}
\end{itemize}
\begin{itemize}
\item {Grp. gram.:V. i.}
\end{itemize}
\begin{itemize}
\item {Proveniência:(Lat. \textunderscore clarare\textunderscore )}
\end{itemize}
Tornar claro: \textunderscore a aurora aclarou o céu\textunderscore .
Deixar vêr.
Embranquecer.
Purificar: \textunderscore aclarar o vinho\textunderscore .
Manifestar.
Deslindar; explicar: \textunderscore aclarar um negócio\textunderscore .
Evidenciar.
Tornar-se claro: \textunderscore o dia aclarou\textunderscore .
\section{Aclavado}
\begin{itemize}
\item {Grp. gram.:adj.}
\end{itemize}
\begin{itemize}
\item {Utilização:Bot.}
\end{itemize}
\begin{itemize}
\item {Proveniência:(Lat. \textunderscore clavatus\textunderscore )}
\end{itemize}
Diz-se de certos órgãos vegetaes, que têm a fórma de clava. Cf. Benevides, \textunderscore Gloss. Bot.\textunderscore 
\section{Acleia}
\begin{itemize}
\item {Grp. gram.:f.}
\end{itemize}
Planta, semelhante ao cardo mosto.
(Grego \textunderscore akleia\textunderscore )
\section{Aclerizar-se}
\begin{itemize}
\item {Grp. gram.:v. p.}
\end{itemize}
\begin{itemize}
\item {Proveniência:(De \textunderscore clero\textunderscore )}
\end{itemize}
Tornar-se clérigo, fazer-se padre.
Adquirir costumes de clérigo.
\section{Aclésia}
\begin{itemize}
\item {Grp. gram.:f.}
\end{itemize}
Mollusco gasteropode opistobrânchio.
\section{Aclia}
\begin{itemize}
\item {Grp. gram.:f.}
\end{itemize}
Planta, semelhante ao cardo mosto.
(Grego \textunderscore akleia\textunderscore )
\section{Áclide}
\begin{itemize}
\item {Grp. gram.:f.}
\end{itemize}
\begin{itemize}
\item {Proveniência:(Lat. \textunderscore aclis\textunderscore )}
\end{itemize}
Frecha delgada e cylíndrica, que se arremessava com uma correia.
\section{Aclimação}
\begin{itemize}
\item {Grp. gram.:f.}
\end{itemize}
Acto de \textunderscore aclimar\textunderscore .
\section{Aclimamento}
\begin{itemize}
\item {Grp. gram.:m.}
\end{itemize}
Effeito da aclimação.
\section{Aclimar}
\begin{itemize}
\item {Grp. gram.:v. t.}
\end{itemize}
\begin{itemize}
\item {Utilização:Fig.}
\end{itemize}
Habituar a um clima.
Conformar, habituar.
\section{Aclimatação}
\begin{itemize}
\item {Grp. gram.:f.}
\end{itemize}
O mesmo que \textunderscore aclimação\textunderscore .
\section{Aclimatar}
\begin{itemize}
\item {Grp. gram.:v. t.}
\end{itemize}
O mesmo que \textunderscore aclimar\textunderscore . Cf. Castilho, \textunderscore Fastos\textunderscore , II, 391.
\section{Aclimatizar}
\begin{itemize}
\item {Grp. gram.:v. t.}
\end{itemize}
O mesmo que \textunderscore aclimar\textunderscore . Cf. Garrett, \textunderscore Romanceiro\textunderscore , II, p. XXIX.
\section{Aclive}
\begin{itemize}
\item {Grp. gram.:m.}
\end{itemize}
\begin{itemize}
\item {Grp. gram.:Adj.}
\end{itemize}
\begin{itemize}
\item {Proveniência:(Lat. \textunderscore acclivis\textunderscore )}
\end{itemize}
Ladeira, declive.
Íngreme.
\section{Aclopo}
\begin{itemize}
\item {Grp. gram.:m.}
\end{itemize}
\begin{itemize}
\item {Proveniência:(Do gr. \textunderscore aklees\textunderscore  + \textunderscore ops\textunderscore )}
\end{itemize}
Insecto coleóptero pentâmero.
\section{Acmadena}
\begin{itemize}
\item {Grp. gram.:f.}
\end{itemize}
\begin{itemize}
\item {Proveniência:(Do gr. \textunderscore akme\textunderscore  + \textunderscore aden\textunderscore )}
\end{itemize}
Arbusto do Cabo da Bôa-Esperança.
\section{Acmástica}
\begin{itemize}
\item {Grp. gram.:adj. f.}
\end{itemize}
\begin{itemize}
\item {Proveniência:(Do gr. \textunderscore akme\textunderscore  + \textunderscore stao\textunderscore )}
\end{itemize}
Diz-se da febre que mantém no seu curso intensidade igual.
\section{Acmela}
\begin{itemize}
\item {Grp. gram.:f.}
\end{itemize}
\begin{itemize}
\item {Proveniência:(Do gr. \textunderscore akme\textunderscore )}
\end{itemize}
Planta indiana e americana.
\section{Acmeodoro}
\begin{itemize}
\item {Grp. gram.:m.}
\end{itemize}
\begin{itemize}
\item {Proveniência:(Do gr. \textunderscore akmaios\textunderscore  + \textunderscore dere\textunderscore )}
\end{itemize}
Insecto coleóptero pentâmero.
\section{Acmócero}
\begin{itemize}
\item {Grp. gram.:m.}
\end{itemize}
\begin{itemize}
\item {Proveniência:(Do gr. \textunderscore akme\textunderscore  + \textunderscore keras\textunderscore )}
\end{itemize}
Insecto coleóptero.
\section{Acna}
\begin{itemize}
\item {Grp. gram.:f.}
\end{itemize}
\begin{itemize}
\item {Proveniência:(Lat. \textunderscore acna\textunderscore )}
\end{itemize}
Antiga medida agrária dos Romanos, usada nos campos da Bética.
\section{Acnanto}
\begin{itemize}
\item {Grp. gram.:m.}
\end{itemize}
\begin{itemize}
\item {Proveniência:(Do gr. \textunderscore achne\textunderscore  + \textunderscore anthos\textunderscore )}
\end{itemize}
Alga microscópica, diatomácea.
\section{Acne}
\begin{itemize}
\item {Grp. gram.:f.}
\end{itemize}
\begin{itemize}
\item {Utilização:Med.}
\end{itemize}
\begin{itemize}
\item {Proveniência:(Do gr. \textunderscore akhne\textunderscore )}
\end{itemize}
Moléstia dos follículos sebáceos da pelle.
\section{Acne-caparrosa}
\begin{itemize}
\item {Grp. gram.:f.}
\end{itemize}
O mesmo que \textunderscore acne-rosácea\textunderscore .
\section{Acnéfalo}
\begin{itemize}
\item {Grp. gram.:m.}
\end{itemize}
\begin{itemize}
\item {Proveniência:(Do gr. \textunderscore a\textunderscore  priv. + \textunderscore knephalon\textunderscore )}
\end{itemize}
Insecto diptero, originário da ilha de Paxo.
\section{Acne-mentagra}
\begin{itemize}
\item {Grp. gram.:f.}
\end{itemize}
O mesmo que \textunderscore mentagra\textunderscore .
\section{Acnéphalo}
\begin{itemize}
\item {Grp. gram.:m.}
\end{itemize}
\begin{itemize}
\item {Proveniência:(Do gr. \textunderscore a\textunderscore  priv. + \textunderscore knephalon\textunderscore .)}
\end{itemize}
Insecto diptero, originário da ilha de Paxo.
\section{Acne-rosácea}
\begin{itemize}
\item {Grp. gram.:f.}
\end{itemize}
Pequenos tubérculos duros, borbulhas pertinazes, que atacam o rosto.
\section{Acnisto}
\begin{itemize}
\item {Grp. gram.:m.}
\end{itemize}
\begin{itemize}
\item {Proveniência:(Do gr. \textunderscore aknestis\textunderscore )}
\end{itemize}
Arbusto americano, da fam. das solâneas.
\section{...aco}
\begin{itemize}
\item {Grp. gram.:suf.}
\end{itemize}
(designativo da qualidade do objecto expresso no rad. da respectiva palavra)
\section{Aço}
\begin{itemize}
\item {Grp. gram.:m.}
\end{itemize}
\begin{itemize}
\item {Proveniência:(Do lat. \textunderscore acies\textunderscore )}
\end{itemize}
Ferro, combinado com carbone e endurecido pela têmpera.
Arma branca.
Fôrça: \textunderscore músculos de aço\textunderscore .
Liga de estanho e mercúrio, applicada nos espelhos.
\section{...aço}
\begin{itemize}
\item {Grp. gram.:suf.}
\end{itemize}
(designativo de \textunderscore aumento\textunderscore )
\section{Acoanhar}
\begin{itemize}
\item {Grp. gram.:v. t.}
\end{itemize}
\begin{itemize}
\item {Utilização:Prov.}
\end{itemize}
\begin{itemize}
\item {Utilização:alg.}
\end{itemize}
O mesmo que \textunderscore coanhar\textunderscore .
\section{Açobar}
\begin{itemize}
\item {Grp. gram.:v. t.}
\end{itemize}
\begin{itemize}
\item {Utilização:Prov.}
\end{itemize}
\begin{itemize}
\item {Utilização:trasm.}
\end{itemize}
Açular (cães)
\section{Acobardadamente}
\begin{itemize}
\item {Grp. gram.:adv.}
\end{itemize}
De modo \textunderscore acobardado\textunderscore .
\section{Acobardado}
\begin{itemize}
\item {Grp. gram.:adj.}
\end{itemize}
Atemorizado.
Acanhado.
\section{Acobardamento}
\begin{itemize}
\item {Grp. gram.:m.}
\end{itemize}
Acto de \textunderscore acobardar\textunderscore .
\section{Acobardar}
\begin{itemize}
\item {Grp. gram.:v. t.}
\end{itemize}
Tornar cobarde.
Intimidar, amedrontar.
\section{Acobertar}
\begin{itemize}
\item {Grp. gram.:v. t.}
\end{itemize}
Tapar com coberta.
Defender.
Encobrir; dissimular.
\section{Acobilhar}
\begin{itemize}
\item {Grp. gram.:v. t.}
\end{itemize}
(V.acovilhar)
\section{Acobreação}
\begin{itemize}
\item {Grp. gram.:f.}
\end{itemize}
Acto de \textunderscore acobrear\textunderscore .
\section{Acobreado}
\begin{itemize}
\item {Grp. gram.:adj.}
\end{itemize}
Que tem aspecto ou côr de cobre.
\section{Acobrear}
\begin{itemize}
\item {Grp. gram.:v. t.}
\end{itemize}
Dar aspecto de cobre a.
\section{Acocéfalo}
\begin{itemize}
\item {Grp. gram.:m.}
\end{itemize}
\begin{itemize}
\item {Proveniência:(Do gr. \textunderscore akoe\textunderscore  + \textunderscore kephale\textunderscore )}
\end{itemize}
Insecto hemíptero.
\section{Acocéphalo}
\begin{itemize}
\item {Grp. gram.:m.}
\end{itemize}
\begin{itemize}
\item {Proveniência:(Do gr. \textunderscore akoe\textunderscore  + \textunderscore kephale\textunderscore )}
\end{itemize}
Insecto hemíptero.
\section{Acochar}
\begin{itemize}
\item {Grp. gram.:v. t.}
\end{itemize}
\begin{itemize}
\item {Utilização:Bras}
\end{itemize}
\begin{itemize}
\item {Proveniência:(De \textunderscore cochar\textunderscore )}
\end{itemize}
Conchegar, apertando ou calcando.
\section{Acóchlidos}
\begin{itemize}
\item {Grp. gram.:m. pl.}
\end{itemize}
\begin{itemize}
\item {Proveniência:(Do gr. \textunderscore a\textunderscore  priv. + \textunderscore cocklis\textunderscore )}
\end{itemize}
Família de molluscos, que não tem concha.
\section{Acocho}
\begin{itemize}
\item {fónica:cô}
\end{itemize}
\begin{itemize}
\item {Grp. gram.:m.}
\end{itemize}
\begin{itemize}
\item {Utilização:Bras}
\end{itemize}
Acto de \textunderscore acochar\textunderscore .
\section{Acóclidos}
\begin{itemize}
\item {Grp. gram.:m. pl.}
\end{itemize}
\begin{itemize}
\item {Proveniência:(Do gr. \textunderscore a\textunderscore  priv. + \textunderscore cocklis\textunderscore )}
\end{itemize}
Família de molluscos, que não tem concha.
\section{Acocoradamente}
\begin{itemize}
\item {Grp. gram.:adv.}
\end{itemize}
De cócoras.
\section{Acocoramento}
\begin{itemize}
\item {Grp. gram.:m.}
\end{itemize}
Acto do pôr-se de cócoras.
\section{Acocorar}
\begin{itemize}
\item {Grp. gram.:v. t.}
\end{itemize}
\begin{itemize}
\item {Grp. gram.:V. p.}
\end{itemize}
\begin{itemize}
\item {Utilização:Fig.}
\end{itemize}
Pôr de cócoras; abaixar (as nádegas)
Pôr-se de cócoras.
Humilhar-se.
\section{Açodadamente}
\begin{itemize}
\item {Grp. gram.:adv.}
\end{itemize}
De modo \textunderscore açodado\textunderscore .
\section{Açodado}
\begin{itemize}
\item {Grp. gram.:adj.}
\end{itemize}
\begin{itemize}
\item {Proveniência:(De \textunderscore açodar\textunderscore )}
\end{itemize}
Apressado.
\section{Açodamento}
\begin{itemize}
\item {Grp. gram.:m.}
\end{itemize}
Acto de \textunderscore açodar\textunderscore .
\section{Açodar}
\begin{itemize}
\item {Grp. gram.:v. t.}
\end{itemize}
Instigar.
Apressar.
\section{Açôfar}
\begin{itemize}
\item {Grp. gram.:m.}
\end{itemize}
\begin{itemize}
\item {Utilização:Ant.}
\end{itemize}
Metal fingido; pechisbeque; latão.
\section{Açofeifa}
\begin{itemize}
\item {Grp. gram.:f.}
\end{itemize}
\begin{itemize}
\item {Proveniência:(Do ár. \textunderscore azzofaizaif\textunderscore )}
\end{itemize}
Fruto da açofeifeira.
\section{Açofeifeira}
\begin{itemize}
\item {Grp. gram.:f.}
\end{itemize}
\begin{itemize}
\item {Proveniência:(De \textunderscore açofeifa\textunderscore )}
\end{itemize}
Árvore fructifera.
\section{Acognosia}
\begin{itemize}
\item {Grp. gram.:f.}
\end{itemize}
Conhecimento dos meios therapêuticos.
\section{Acogombrado}
\begin{itemize}
\item {Grp. gram.:adj.}
\end{itemize}
\begin{itemize}
\item {Utilização:Ant.}
\end{itemize}
\begin{itemize}
\item {Proveniência:(De \textunderscore cogombro\textunderscore )}
\end{itemize}
Que tem fórma de pepino ou o sabor de pepino.
\section{Acoguladamente}
\begin{itemize}
\item {Grp. gram.:adv.}
\end{itemize}
De modo \textunderscore acogulado\textunderscore .
\section{Acogulado}
\begin{itemize}
\item {Grp. gram.:adj.}
\end{itemize}
Muito cheio; que faz cogulo: \textunderscore medida acogulada\textunderscore .
\section{Acoguladura}
\begin{itemize}
\item {Grp. gram.:f.}
\end{itemize}
O mesmo que \textunderscore cogulo\textunderscore .
\section{Acogular}
\begin{itemize}
\item {Grp. gram.:v. t.}
\end{itemize}
Encher, fazendo cogulo.
\section{Açoiabá}
\begin{itemize}
\item {Grp. gram.:m.}
\end{itemize}
\begin{itemize}
\item {Utilização:Bras}
\end{itemize}
Tanga de pennas, usada por Índios do Brasil.
\section{Acoimador}
\begin{itemize}
\item {Grp. gram.:m.}
\end{itemize}
O que acoima.
\section{Acoimamento}
\begin{itemize}
\item {Grp. gram.:m.}
\end{itemize}
Acto de \textunderscore acoimar\textunderscore .
\section{Acoimar}
\begin{itemize}
\item {Grp. gram.:v. t.}
\end{itemize}
\begin{itemize}
\item {Proveniência:(De \textunderscore coima\textunderscore )}
\end{itemize}
Impor coima a.
Punir.
Accusar, classificar em sentido depreciativo.
\textunderscore Acoimar morte\textunderscore , tirar vingança do matador.
\section{Acoirelado}
\begin{itemize}
\item {Grp. gram.:adj.}
\end{itemize}
Dividido em coirelas.
\section{Acoirelamento}
\begin{itemize}
\item {Grp. gram.:m.}
\end{itemize}
\begin{itemize}
\item {Utilização:Ant.}
\end{itemize}
Divisão de um terreno em coirelas ou casaes.
Acto de \textunderscore acoirelar\textunderscore .
\section{Acoirelar}
\begin{itemize}
\item {Grp. gram.:v. t.}
\end{itemize}
Dividir em coirelas.
\section{Acoitadar}
\begin{itemize}
\item {Grp. gram.:v. t.}
\end{itemize}
\begin{itemize}
\item {Utilização:Prov.}
\end{itemize}
\begin{itemize}
\item {Utilização:trasm.}
\end{itemize}
\begin{itemize}
\item {Proveniência:(De \textunderscore coitado\textunderscore )}
\end{itemize}
Dizer palavras de compaixão a respeito de.
Lastimar.
\section{Açoitadiço}
\begin{itemize}
\item {Grp. gram.:adj.}
\end{itemize}
Que tem sido açoitado muitas vezes.
\section{Acoitador}
\begin{itemize}
\item {Grp. gram.:m.}
\end{itemize}
O que acoita.
\section{Acoitadura}
\begin{itemize}
\item {Grp. gram.:f.}
\end{itemize}
\begin{itemize}
\item {Utilização:Ant.}
\end{itemize}
O mesmo que \textunderscore acoitamento\textunderscore .
\section{Acoitamento}
\begin{itemize}
\item {Grp. gram.:m.}
\end{itemize}
Acto de \textunderscore acoitar\textunderscore .
\section{Acoitar}
\begin{itemize}
\item {Grp. gram.:v. t.}
\end{itemize}
Dar coito a; agasalhar, acolher.
\section{Açoitar}
\begin{itemize}
\item {Grp. gram.:v. t.}
\end{itemize}
Dar com açoite em.
Fustigar.
Bater.
Varejar.
Devastar.
Affligir.
\section{Açoite}
\begin{itemize}
\item {Grp. gram.:m.}
\end{itemize}
\begin{itemize}
\item {Utilização:Fam.}
\end{itemize}
Instrumento de tiras de coiro, para bater.
Azorrague.
Látego.
Pancada com a mão em nádegas de criança.
(Do ár.)
\section{Açoiteiras}
\begin{itemize}
\item {Grp. gram.:f. pl.}
\end{itemize}
\begin{itemize}
\item {Utilização:Bras. do S}
\end{itemize}
\begin{itemize}
\item {Proveniência:(De \textunderscore açoite\textunderscore )}
\end{itemize}
Ponta das rédeas, com que o cavalleiro açoita o cavallo.
\section{Acoiteza}
\begin{itemize}
\item {Grp. gram.:f.}
\end{itemize}
\begin{itemize}
\item {Utilização:Ant.}
\end{itemize}
O mesmo que \textunderscore côito\textunderscore , abrigo.
\section{Acola}
\begin{itemize}
\item {Grp. gram.:f.}
\end{itemize}
Iguaria, feita de chocolate e farinha de milho, que se usou no Brasil.
\section{Acolá}
\begin{itemize}
\item {Grp. gram.:adv.}
\end{itemize}
\begin{itemize}
\item {Proveniência:(Do lat. \textunderscore eccu'illac\textunderscore )}
\end{itemize}
Além, naquelle lugar.
\section{Acolasto}
\begin{itemize}
\item {Grp. gram.:m.}
\end{itemize}
\begin{itemize}
\item {Proveniência:(Do gr. \textunderscore akolastos\textunderscore )}
\end{itemize}
Insecto diptero, de órgãos geradores muito desenvolvidos.
\section{Acólcetra}
\begin{itemize}
\item {Grp. gram.:f.}
\end{itemize}
\begin{itemize}
\item {Utilização:Ant.}
\end{itemize}
\begin{itemize}
\item {Proveniência:(Do lat. \textunderscore culcita\textunderscore )}
\end{itemize}
O mesmo que \textunderscore cócedra\textunderscore .
\section{Acolchetador}
\begin{itemize}
\item {Grp. gram.:m.}
\end{itemize}
O que acolcheta.
\section{Acolchetamento}
\begin{itemize}
\item {Grp. gram.:m.}
\end{itemize}
Acto ou effeito de \textunderscore acolchetar\textunderscore .
\section{Acolchetar}
\begin{itemize}
\item {Grp. gram.:v. t.}
\end{itemize}
Apertar com colchete.
Engranzar.
\section{Acolchoadeira}
\begin{itemize}
\item {Grp. gram.:f.}
\end{itemize}
Mulher que acolchôa.
\section{Acolchoadinho}
\begin{itemize}
\item {Grp. gram.:m.}
\end{itemize}
\begin{itemize}
\item {Utilização:Des.}
\end{itemize}
Espécie de tecido branco ou de côres, que imita estofo acolchoado.
\section{Acolchoador}
\begin{itemize}
\item {Grp. gram.:m.}
\end{itemize}
O que acolchôa.
\section{Acolchoamento}
\begin{itemize}
\item {Grp. gram.:m.}
\end{itemize}
Acto ou effeito de \textunderscore acolchoar\textunderscore .
\section{Acolchoar}
\begin{itemize}
\item {Grp. gram.:v. t.}
\end{itemize}
Forrar ou encher, como a um colchão.
Lavrar ou tecer, á maneira de colcha.
\section{Acoleijos}
\begin{itemize}
\item {Grp. gram.:m.}
\end{itemize}
Designação pop. e ant. da \textunderscore aquilégia\textunderscore . Cf. \textunderscore Desengano da Med.\textunderscore , 82.
\section{Acolhedor}
\begin{itemize}
\item {Grp. gram.:m.  e  adj.}
\end{itemize}
O que acolhe.
Hospitaleiro.
\section{Acolheita}
\begin{itemize}
\item {Grp. gram.:f.}
\end{itemize}
\begin{itemize}
\item {Utilização:Prov.}
\end{itemize}
\begin{itemize}
\item {Utilização:trasm.}
\end{itemize}
O mesmo que \textunderscore colheita\textunderscore  e \textunderscore acolhimento\textunderscore .
Lugar, onde os peixes se acolhem, no rio, debaixo de fragas.
\section{Acolheitar}
\begin{itemize}
\item {Grp. gram.:v. t.}
\end{itemize}
Fazer colheita de. Cf. \textunderscore Bibl. da Gente do Campo\textunderscore , 276.
\section{Acolhença}
\begin{itemize}
\item {Grp. gram.:f.}
\end{itemize}
\begin{itemize}
\item {Utilização:Des.}
\end{itemize}
Acolhimento.
Affabilidade. Cf. \textunderscore Menina e Moça\textunderscore , 133 (ed. 1852); e Rui Barb., \textunderscore Répl.\textunderscore  157.
\section{Acolher}
\begin{itemize}
\item {Grp. gram.:v. t.}
\end{itemize}
\begin{itemize}
\item {Proveniência:(De \textunderscore colher\textunderscore )}
\end{itemize}
Recolher, agasalhar.
Receber.
\section{Acolherar}
\begin{itemize}
\item {Grp. gram.:v. t.}
\end{itemize}
\begin{itemize}
\item {Utilização:Bras. do S}
\end{itemize}
Atrelar ou ajoujar, por meio de colhera (cavallos).
\section{Acolhida}
\begin{itemize}
\item {Grp. gram.:f.}
\end{itemize}
O mesmo que \textunderscore acolhimento\textunderscore .
\section{Acolhido}
\begin{itemize}
\item {Grp. gram.:adj.}
\end{itemize}
\begin{itemize}
\item {Utilização:Ant.}
\end{itemize}
O mesmo que \textunderscore fugitivo\textunderscore .
\section{Acolhimento}
\begin{itemize}
\item {Grp. gram.:m.}
\end{itemize}
Acto e effeito de \textunderscore acolher\textunderscore .
\section{Acolho}
\begin{itemize}
\item {fónica:cô}
\end{itemize}
\begin{itemize}
\item {Grp. gram.:m.}
\end{itemize}
\begin{itemize}
\item {Utilização:Bras}
\end{itemize}
Acto de \textunderscore acolher\textunderscore .
\section{Acolia}
\begin{itemize}
\item {Grp. gram.:f.}
\end{itemize}
\begin{itemize}
\item {Utilização:Med.}
\end{itemize}
\begin{itemize}
\item {Proveniência:(Do gr. \textunderscore a\textunderscore  priv. + \textunderscore khole\textunderscore , bile)}
\end{itemize}
Suppressão da secreção biliar.
Paragem da secreção biliar.
\section{Acolim}
\begin{itemize}
\item {Grp. gram.:m.}
\end{itemize}
Codorniz do México.
\section{Acolitado}
\begin{itemize}
\item {Grp. gram.:m.}
\end{itemize}
Uma das quatro Ordens menores, na vida ecclesiástica.
\section{Acolitar}
\begin{itemize}
\item {Grp. gram.:v. t.}
\end{itemize}
Acompanhar como \textunderscore acólito\textunderscore .
Ajudar.
Acompanhar, seguir.
\section{Acólito}
\begin{itemize}
\item {Grp. gram.:m.}
\end{itemize}
\begin{itemize}
\item {Proveniência:(Lat. \textunderscore acolythus\textunderscore )}
\end{itemize}
Pessôa, que tem a graduação ecclesiástica dêste nome, ou o que exerce o acolitado.
O que acompanha; o que ajuda.
\section{Acolytado}
\begin{itemize}
\item {Grp. gram.:m.}
\end{itemize}
Uma das quatro Ordens menores, na vida ecclesiástica.
\section{Acolytar}
\begin{itemize}
\item {Grp. gram.:v. t.}
\end{itemize}
Acompanhar como \textunderscore acólyto\textunderscore .
Ajudar.
Acompanhar, seguir.
\section{Acólyto}
\begin{itemize}
\item {Grp. gram.:m.}
\end{itemize}
\begin{itemize}
\item {Proveniência:(Lat. \textunderscore acolythus\textunderscore )}
\end{itemize}
Pessôa, que tem a graduação ecclesiástica dêste nome, ou o que exerce o acolytado.
O que acompanha; o que ajuda.
\section{Acomadrar-se}
\begin{itemize}
\item {Grp. gram.:v. p.}
\end{itemize}
Tornar-se comadre; entrar na intimidade de outrem, (falando-se de mulheres).
\section{Acomás}
\begin{itemize}
\item {Grp. gram.:m.}
\end{itemize}
Árvore das Antilhas, cuja madeira é empregada em construcções.
\section{Acometedor}
\begin{itemize}
\item {Grp. gram.:m.  e  adj.}
\end{itemize}
O que acommete.
\section{Acometer}
\begin{itemize}
\item {Grp. gram.:v. t.}
\end{itemize}
\begin{itemize}
\item {Grp. gram.:V. i.}
\end{itemize}
\begin{itemize}
\item {Proveniência:(De \textunderscore commeter\textunderscore )}
\end{itemize}
Assaltar, atacar, investir.
Hostilizar.
Provocar.
Injuriar.
Aproximar-se de.
Encetar briga.
Abalançar-se.
Sentir ímpetos.
\section{Acometida}
\begin{itemize}
\item {Grp. gram.:f.}
\end{itemize}
O mesmo que \textunderscore acometimento\textunderscore .
\section{Acometimento}
\begin{itemize}
\item {Grp. gram.:m.}
\end{itemize}
Acto ou effeito de \textunderscore acometer\textunderscore .
\section{Acometível}
\begin{itemize}
\item {Grp. gram.:adj.}
\end{itemize}
Que se póde \textunderscore acometer\textunderscore .
\section{Acomia}
\begin{itemize}
\item {Grp. gram.:f.}
\end{itemize}
\begin{itemize}
\item {Proveniência:(Do gr. \textunderscore a\textunderscore  priv. e \textunderscore koma\textunderscore , cabello)}
\end{itemize}
O mesmo que \textunderscore calvície\textunderscore .
\section{Acommetedor}
\begin{itemize}
\item {Grp. gram.:m.  e  adj.}
\end{itemize}
O que acommete.
\section{Acommeter}
\begin{itemize}
\item {Grp. gram.:v. t.}
\end{itemize}
\begin{itemize}
\item {Grp. gram.:V. i.}
\end{itemize}
\begin{itemize}
\item {Proveniência:(De \textunderscore commeter\textunderscore )}
\end{itemize}
Assaltar, atacar, investir.
Hostilizar.
Provocar.
Injuriar.
Aproximar-se de.
Encetar briga.
Abalançar-se.
Sentir ímpetos.
\section{Acommetida}
\begin{itemize}
\item {Grp. gram.:f.}
\end{itemize}
O mesmo que \textunderscore acommetimento\textunderscore .
\section{Acommetimento}
\begin{itemize}
\item {Grp. gram.:m.}
\end{itemize}
Acto ou effeito de \textunderscore acommeter\textunderscore .
\section{Acommetível}
\begin{itemize}
\item {Grp. gram.:adj.}
\end{itemize}
Que se póde \textunderscore acommeter\textunderscore .
\section{Acomodação}
\begin{itemize}
\item {Grp. gram.:f.}
\end{itemize}
Acto ou effeito de \textunderscore acomodar\textunderscore .
\section{Acomodadamente}
\begin{itemize}
\item {Grp. gram.:adv.}
\end{itemize}
De modo \textunderscore acomodado\textunderscore .
\section{Acomodadiço}
\begin{itemize}
\item {Grp. gram.:adj.}
\end{itemize}
O mesmo que \textunderscore acomodatício\textunderscore .
\section{Acomodamento}
\begin{itemize}
\item {Grp. gram.:m.}
\end{itemize}
(V.acomodação)
\section{Acomodar}
\begin{itemize}
\item {Grp. gram.:v. t.}
\end{itemize}
\begin{itemize}
\item {Proveniência:(Lat. \textunderscore accommodare\textunderscore )}
\end{itemize}
Tornar cômmodo.
Adequar.
Arrumar, pôr em ordem: \textunderscore accommodar os livros\textunderscore .
Empregar, dar posição a.
Applicar.
Habituar.
Sossegar.
Hospedar.
\section{Acomodatício}
\begin{itemize}
\item {Grp. gram.:adj.}
\end{itemize}
Que se acomoda facilmente.
\section{Acomodável}
\begin{itemize}
\item {Grp. gram.:adj.}
\end{itemize}
Que se póde \textunderscore acomodar\textunderscore .
\section{Acômodo}
\begin{itemize}
\item {Grp. gram.:adj.}
\end{itemize}
\begin{itemize}
\item {Utilização:Ant.}
\end{itemize}
Opportuno, cômmodo.
\section{Acompadrar}
\begin{itemize}
\item {Grp. gram.:v. t.}
\end{itemize}
Tornar compadre, amigo.
Familiarizar.
\section{Acompanhadeira}
\begin{itemize}
\item {Grp. gram.:f.}
\end{itemize}
Mulher que acompanha.
\section{Acompanhador}
\begin{itemize}
\item {Grp. gram.:m.}
\end{itemize}
O que acompanha.
\section{Acompanhamento}
\begin{itemize}
\item {Grp. gram.:m.}
\end{itemize}
Acto de \textunderscore acompanhar\textunderscore .
Séquito, cortejo.
Música, que acompanha recitação ou canto.
\section{Acompanhante}
\begin{itemize}
\item {Grp. gram.:adj.}
\end{itemize}
Que acompanha.
\section{Acompanhar}
\begin{itemize}
\item {Grp. gram.:v. t.}
\end{itemize}
\begin{itemize}
\item {Proveniência:(De \textunderscore companha\textunderscore )}
\end{itemize}
Fazer companhia a.
Seguir.
Seguir com instrumento (a recitação, o canto ou a parte cantante da música).
\section{Acompleicionado}
\begin{itemize}
\item {Grp. gram.:adj.}
\end{itemize}
Que tem compleição (bôa ou má).
\section{Acompleiçoado}
\begin{itemize}
\item {Grp. gram.:adj.}
\end{itemize}
O mesmo que \textunderscore acompleicionado\textunderscore .
\section{Acomplexionado}
\begin{itemize}
\item {fónica:csi}
\end{itemize}
\begin{itemize}
\item {Grp. gram.:adj.}
\end{itemize}
O mesmo que \textunderscore acompleicionado\textunderscore .
\section{Acompridar}
\begin{itemize}
\item {Grp. gram.:v. t.}
\end{itemize}
Tornar comprido, alongar.
\section{Acômpsia}
\begin{itemize}
\item {Grp. gram.:f.}
\end{itemize}
\begin{itemize}
\item {Proveniência:(Do gr. \textunderscore akompsos\textunderscore )}
\end{itemize}
Insecto lepidóptero nocturno.
\section{Aconans}
\begin{itemize}
\item {Grp. gram.:m. pl.}
\end{itemize}
Selvagens do Brasil, que habitavam no interior da antiga província de Pernambuco.
\section{Aconapar}
\begin{itemize}
\item {Grp. gram.:v. t.}
\end{itemize}
\begin{itemize}
\item {Utilização:Prov.}
\end{itemize}
\begin{itemize}
\item {Utilização:beir.}
\end{itemize}
Serzir, remendar grosseiramente: \textunderscore nem sabe aconapar as calças do marido\textunderscore !
\section{Aconchado}
\begin{itemize}
\item {Grp. gram.:adj.}
\end{itemize}
\begin{itemize}
\item {Utilização:Archit.}
\end{itemize}
\begin{itemize}
\item {Proveniência:(De \textunderscore concha\textunderscore )}
\end{itemize}
Diz-se do tecto, estabelecido por fórma, que aproveita o vão do telhado.
\section{Aconchavar}
\begin{itemize}
\item {Grp. gram.:v. t.}
\end{itemize}
O mesmo que \textunderscore conchavar\textunderscore . Cf. Camillo, \textunderscore Serões\textunderscore .
\section{Aconchegado}
\begin{itemize}
\item {Grp. gram.:adj.}
\end{itemize}
Muito aproximado.
Agasalhado:«\textunderscore o pescoço muito aconchegado numa pelatina preta\textunderscore ». Camillo, \textunderscore Corja\textunderscore , 285.
\section{Aconchegar}
\textunderscore v. t.\textunderscore  (e der.)
(V. \textunderscore conchegar\textunderscore , etc.)
\section{Acondicionador}
\begin{itemize}
\item {Grp. gram.:m.}
\end{itemize}
O que acondiciona.
\section{Acondicionamento}
\begin{itemize}
\item {Grp. gram.:m.}
\end{itemize}
Acto de \textunderscore acondicionar\textunderscore .
\section{Acondicionar}
\begin{itemize}
\item {Grp. gram.:v. t.}
\end{itemize}
\begin{itemize}
\item {Grp. gram.:v. t.}
\end{itemize}
Pôr em (bôa) condição.
Guardar em sítio conveniente.
Preservar de deterioração.
Preparar, dispor:«\textunderscore mal acondicionava o surtir das empresas\textunderscore ». Filinto, \textunderscore D. Man. II\textunderscore , 202.
\section{Acondicionável}
\begin{itemize}
\item {Grp. gram.:adj.}
\end{itemize}
Que se póde ou se deve acondicionar.
\section{Acondiçoar}
\begin{itemize}
\item {Grp. gram.:v. t.}
\end{itemize}
O mesmo que \textunderscore acondicionar\textunderscore .
\section{Acôndilo}
\begin{itemize}
\item {Grp. gram.:m.}
\end{itemize}
\begin{itemize}
\item {Utilização:Anat.}
\end{itemize}
\begin{itemize}
\item {Proveniência:(Do gr. \textunderscore a\textunderscore  priv. + \textunderscore kondulos\textunderscore )}
\end{itemize}
Que não tem côndilo.
\section{Acondimentar}
\begin{itemize}
\item {Grp. gram.:v. t.}
\end{itemize}
O mesmo que \textunderscore condimentar\textunderscore . Cf. Camillo, \textunderscore Caveira\textunderscore , 352; Arn. Gama, \textunderscore Motim\textunderscore , 433.
\section{Acôndylo}
\begin{itemize}
\item {Grp. gram.:m.}
\end{itemize}
\begin{itemize}
\item {Utilização:Anat.}
\end{itemize}
\begin{itemize}
\item {Proveniência:(Do gr. \textunderscore a\textunderscore  priv. + \textunderscore kondulos\textunderscore )}
\end{itemize}
Que não tem côndylo.
\section{Aconeína}
\begin{itemize}
\item {Grp. gram.:f.}
\end{itemize}
Substância extrahida do \textunderscore acónito\textunderscore .
\section{Aconfeitar}
\begin{itemize}
\item {Grp. gram.:v. t.}
\end{itemize}
Dar fórma de confeito a.
\section{Aconfradar}
\begin{itemize}
\item {Grp. gram.:v. t.}
\end{itemize}
Tornar confrade.
Abandear.
\section{Aconhecer}
\begin{itemize}
\item {Grp. gram.:v. t.}
\end{itemize}
O mesmo que \textunderscore reconhecer\textunderscore .
\section{Aconina}
\begin{itemize}
\item {Grp. gram.:f.}
\end{itemize}
Um dos extractos do acónito.
\section{Aconitato}
\begin{itemize}
\item {Grp. gram.:m.}
\end{itemize}
\begin{itemize}
\item {Proveniência:(De \textunderscore acónito\textunderscore )}
\end{itemize}
Sal, produzido pela combinação do ácido aconítico com uma base.
\section{Aconitela}
\begin{itemize}
\item {Grp. gram.:f.}
\end{itemize}
Planta ranunculácea, semelhante ao acónito.
\section{Aconítico}
\begin{itemize}
\item {Grp. gram.:adj.}
\end{itemize}
Diz-se do ácido, que se acha no suco do acónito.
\section{Aconitina}
\begin{itemize}
\item {Grp. gram.:f.}
\end{itemize}
Alcaloide, extrahido da acónito.
\section{Acónito}
\begin{itemize}
\item {Grp. gram.:m.}
\end{itemize}
\begin{itemize}
\item {Proveniência:(Lat. \textunderscore aconitum\textunderscore )}
\end{itemize}
Planta venenosa e medicinal, da fam. das ranunculáceas.
\section{Aconselhadamente}
\begin{itemize}
\item {Grp. gram.:adj.}
\end{itemize}
Prudentemente; de modo (bem) \textunderscore aconselhado\textunderscore .
\section{Aconselhador}
\begin{itemize}
\item {Grp. gram.:m.}
\end{itemize}
\begin{itemize}
\item {Proveniência:(Lat. \textunderscore consiliator\textunderscore )}
\end{itemize}
O que aconselha.
\section{Aconselhar}
\begin{itemize}
\item {Grp. gram.:v. t.}
\end{itemize}
Dar conselho a.
Procurar convencer.
Persuadir.
\section{Aconsoantar}
\begin{itemize}
\item {Grp. gram.:v. t.}
\end{itemize}
Tornar consoante, rimar. Cf. Filinto, VIII, 9; X, 126; XVIII, 220.
\section{Acontecedeiro}
\begin{itemize}
\item {Grp. gram.:adj.}
\end{itemize}
Que acontece amiúde.
Freqüente.
\section{Acontecer}
\begin{itemize}
\item {Grp. gram.:v. i.}
\end{itemize}
\begin{itemize}
\item {Grp. gram.:V. p.}
\end{itemize}
\begin{itemize}
\item {Utilização:Ant.}
\end{itemize}
Realizar-se inesperadamente.
Succeder.
Passar a sêr realidade.
Sobrevir.
(A mesma sign.). Cf. \textunderscore Rev. Lus.\textunderscore , XVIII.
\section{Acontecimento}
\begin{itemize}
\item {Grp. gram.:m.}
\end{itemize}
\begin{itemize}
\item {Grp. gram.:m.}
\end{itemize}
Aquillo que acontece.
Facto, que produz sensação.
Acaso, eventualidade. Cf. Filinto, \textunderscore D. Man.\textunderscore , I, 31.
\section{Acôntia}
\begin{itemize}
\item {Grp. gram.:f.}
\end{itemize}
\begin{itemize}
\item {Proveniência:(Do gr. \textunderscore akontios\textunderscore )}
\end{itemize}
Insecto lepidóptero nocturno.
\section{Acontiado}
\begin{itemize}
\item {Grp. gram.:m.}
\end{itemize}
\begin{itemize}
\item {Utilização:Ant.}
\end{itemize}
\begin{itemize}
\item {Proveniência:(Do ant. \textunderscore contia\textunderscore )}
\end{itemize}
Vassalo, que, segundo a \textunderscore quantia\textunderscore  de seus haveres, devia estar prestes a servir o rei com armas, ou com armas e cavallos.
\section{Acôntias}
\begin{itemize}
\item {Grp. gram.:m. pl.}
\end{itemize}
\begin{itemize}
\item {Proveniência:(Do gr. \textunderscore akontias\textunderscore )}
\end{itemize}
Reptis ophídios.
\section{Acôntio}
\begin{itemize}
\item {Grp. gram.:m.}
\end{itemize}
\begin{itemize}
\item {Proveniência:(Do gr. \textunderscore akontion\textunderscore )}
\end{itemize}
Pequeno dardo.
Seta.
Frecha.
\section{Acontioso}
\begin{itemize}
\item {Grp. gram.:adj.}
\end{itemize}
\begin{itemize}
\item {Utilização:Ant.}
\end{itemize}
\begin{itemize}
\item {Proveniência:(Do ant. \textunderscore contia\textunderscore )}
\end{itemize}
Abonado, abastado.
\section{Acontista}
\begin{itemize}
\item {Grp. gram.:m.}
\end{itemize}
\begin{itemize}
\item {Proveniência:(Do gr. \textunderscore akontistes\textunderscore )}
\end{itemize}
Frecheiro.
\section{Acontraltado}
\begin{itemize}
\item {Grp. gram.:adj.}
\end{itemize}
\begin{itemize}
\item {Utilização:Mús.}
\end{itemize}
Diz-se da voz de soprano, que desce quási como um contralto:«\textunderscore tenor acontraltado\textunderscore ». \textunderscore Hyssope\textunderscore , nota A, ed. de 1871.
\section{Ácopo}
\begin{itemize}
\item {Grp. gram.:m.}
\end{itemize}
\begin{itemize}
\item {Proveniência:(Gr. \textunderscore akopos\textunderscore )}
\end{itemize}
Remédio contra o cansaço.
Designação antiga de uma pedra preciosa, esponjosa e raiada de oiro.
\section{Acoposo}
\begin{itemize}
\item {Grp. gram.:adj.}
\end{itemize}
\begin{itemize}
\item {Utilização:Ant.}
\end{itemize}
\begin{itemize}
\item {Proveniência:(De \textunderscore ácopo\textunderscore )}
\end{itemize}
Dizia-se dos unguentos e outros medicamentos, que curavam o cansaço, produzido pelo trabalho.
\section{Açor}
\begin{itemize}
\item {Grp. gram.:m.}
\end{itemize}
\begin{itemize}
\item {Proveniência:(Do lat. \textunderscore acceptor?\textunderscore )}
\end{itemize}
Ave de rapina, do género falcão.
\section{Açoramento}
\begin{itemize}
\item {Grp. gram.:m.}
\end{itemize}
Acto ou effeito de \textunderscore açorar\textunderscore .
\section{Acorão}
\begin{itemize}
\item {Grp. gram.:m.}
\end{itemize}
\begin{itemize}
\item {Proveniência:(Do gr. \textunderscore akoros\textunderscore )}
\end{itemize}
Nome scientifico da chamada \textunderscore pimenta das abelhas\textunderscore .
\section{Açorar}
\begin{itemize}
\item {Grp. gram.:v. t.}
\end{itemize}
\begin{itemize}
\item {Utilização:Ant.}
\end{itemize}
\begin{itemize}
\item {Proveniência:(De \textunderscore açor\textunderscore )}
\end{itemize}
Atear grande desejo em.
Provocar com tentações.
O mesmo que \textunderscore açodar\textunderscore , apressar.
\section{Acorçoar}
\begin{itemize}
\item {Grp. gram.:v. t.}
\end{itemize}
O mesmo que \textunderscore acoroçoar\textunderscore . Cf. Filinto, IV, 216; \textunderscore D. Man.\textunderscore , III, 31O.
\section{Acorcovar}
\begin{itemize}
\item {Grp. gram.:v. t.}
\end{itemize}
(V.corcovar)
\section{Acorcundado}
\begin{itemize}
\item {Grp. gram.:adj.}
\end{itemize}
Que é um tanto corcunda.
\section{Açorda}
\begin{itemize}
\item {Grp. gram.:f.}
\end{itemize}
\begin{itemize}
\item {Utilização:Fig.}
\end{itemize}
\begin{itemize}
\item {Utilização:Gír.}
\end{itemize}
\begin{itemize}
\item {Proveniência:(Do ár. \textunderscore ath-thorda\textunderscore )}
\end{itemize}
Sôpa de pão, alhos, azeite, etc.
Pessôa molle, negligente.
Bebedeira.
\section{Acordação}
\begin{itemize}
\item {Grp. gram.:f.}
\end{itemize}
\begin{itemize}
\item {Utilização:Ant.}
\end{itemize}
(V.acôrdo)
\section{Acordadamente}
\begin{itemize}
\item {Grp. gram.:adv.}
\end{itemize}
Com acôrdo.
\section{Acordam}
\begin{itemize}
\item {Grp. gram.:m.}
\end{itemize}
Sentença, resolução de recursos em tribunaes collectivos, administrativos ou judiciaes.
(Da 3.^a pess. do pl. do pres. do ind. do v. \textunderscore acordar\textunderscore ).
\section{Açordamento}
\begin{itemize}
\item {Grp. gram.:m.}
\end{itemize}
Acto de \textunderscore acordar\textunderscore .
\section{Acordança}
\begin{itemize}
\item {Grp. gram.:f.}
\end{itemize}
O mesmo que \textunderscore acôrdo\textunderscore .
Melodia; consonância.
\section{Acordante}
\begin{itemize}
\item {Grp. gram.:adj.}
\end{itemize}
\begin{itemize}
\item {Proveniência:(De \textunderscore acordar\textunderscore )}
\end{itemize}
Harmonioso, acorde.
\section{Acórdão}
\begin{itemize}
\item {Grp. gram.:m.}
\end{itemize}
Sentença, resolução de recursos em tribunaes collectivos, administrativos ou judiciaes.
(Da 3.^a pess. do pl. do pres. do ind. do v. \textunderscore acordar\textunderscore ).
\section{Acordar}
\begin{itemize}
\item {Grp. gram.:v. t.}
\end{itemize}
\begin{itemize}
\item {Grp. gram.:V. i.}
\end{itemize}
\begin{itemize}
\item {Grp. gram.:V. p.}
\end{itemize}
\begin{itemize}
\item {Utilização:Des.}
\end{itemize}
\begin{itemize}
\item {Proveniência:(Do lat. \textunderscore cor\textunderscore , \textunderscore cordis\textunderscore )}
\end{itemize}
Despertar.
Lembrar.
Conciliar.
Tirar-se do somno.
Fazer acôrdo.
Combinar-se, ajustar-se.
Recordar-se, lembrar-se.
\section{Acorde}
\begin{itemize}
\item {Grp. gram.:m.}
\end{itemize}
\begin{itemize}
\item {Grp. gram.:Adj.}
\end{itemize}
\begin{itemize}
\item {Proveniência:(De \textunderscore acordar\textunderscore )}
\end{itemize}
União.
Harmonia.
Harmónico.
Concorde.
\section{Acordeão}
\begin{itemize}
\item {Grp. gram.:m.}
\end{itemize}
Instrumento, composto de palhetas metállicas, que entram em vibração por meio de um folle.
É também conhecido por \textunderscore harmónica\textunderscore .
\section{Acordemente}
\begin{itemize}
\item {Grp. gram.:adv.}
\end{itemize}
\begin{itemize}
\item {Proveniência:(De \textunderscore acorde\textunderscore )}
\end{itemize}
Harmonicamente.
\section{Acordina}
\begin{itemize}
\item {Grp. gram.:f.}
\end{itemize}
Relógio, que marca as horas por meio das notas de um acorde perfeito.
Também se chama \textunderscore relógio musical\textunderscore .
\section{Acórdo}
\begin{itemize}
\item {Grp. gram.:m.}
\end{itemize}
Instrumento italiano, de quinze cordas.
\section{Acôrdo}
\begin{itemize}
\item {Grp. gram.:m.}
\end{itemize}
\begin{itemize}
\item {Proveniência:(De \textunderscore acordar\textunderscore )}
\end{itemize}
Conformidade.
Conciliação.
Convenção; ajuste.
Cautela.
Tino, juízo; conhecimento de si próprio: \textunderscore não dar acôrdo de si\textunderscore .
\section{Acordoar}
\begin{itemize}
\item {Grp. gram.:v. t.}
\end{itemize}
(V.encordoar)
\section{Açoreanismo}
\begin{itemize}
\item {Grp. gram.:m.}
\end{itemize}
Palavra ou locução privativa dos Açores.
\section{Açoreanista}
\begin{itemize}
\item {Grp. gram.:m.}
\end{itemize}
Aquelle que se dedica a estudos sôbre os Açores, ou que é dedicado aos interesses dos açoreanos.
\section{Açoreano}
\begin{itemize}
\item {Grp. gram.:adj.}
\end{itemize}
\begin{itemize}
\item {Grp. gram.:M.}
\end{itemize}
Relativo aos Açores.
O que é natural dos Açores.
\section{Açoreiro}
\begin{itemize}
\item {Grp. gram.:m.}
\end{itemize}
O que tinha a seu cargo o tratar dos açores, para a caça.
\section{Açorenha}
\begin{itemize}
\item {Grp. gram.:f.}
\end{itemize}
\begin{itemize}
\item {Proveniência:(De \textunderscore açor\textunderscore )}
\end{itemize}
Ave de rapina.
\section{Açorenho}
\begin{itemize}
\item {Grp. gram.:m.  e  adj.}
\end{itemize}
\begin{itemize}
\item {Utilização:P. us.}
\end{itemize}
O mesmo que \textunderscore açoreano\textunderscore .
\section{Acôres}
\begin{itemize}
\item {Grp. gram.:m. pl.}
\end{itemize}
\begin{itemize}
\item {Proveniência:(Gr. \textunderscore akhor\textunderscore )}
\end{itemize}
Tinha mucosa.
\section{Acori}
\begin{itemize}
\item {Grp. gram.:m.}
\end{itemize}
Coral azul.
\section{Acoria}
\begin{itemize}
\item {Grp. gram.:f.}
\end{itemize}
\begin{itemize}
\item {Proveniência:(Do gr. \textunderscore a\textunderscore  priv. + \textunderscore khorhe\textunderscore )}
\end{itemize}
Defeito orgânico, que consiste na falta de pupilla.
\section{Acória}
\begin{itemize}
\item {Grp. gram.:f.}
\end{itemize}
\begin{itemize}
\item {Proveniência:(Gr. \textunderscore akoria\textunderscore )}
\end{itemize}
Fome canina.
\section{Açorear}
\textunderscore v. t.\textunderscore  (e der.)
(V. \textunderscore assorear\textunderscore , etc.)
\section{Açórico}
\begin{itemize}
\item {Grp. gram.:adj.}
\end{itemize}
O mesmo que \textunderscore açoreano\textunderscore .
\section{Acoríneas}
\begin{itemize}
\item {Grp. gram.:f. pl.}
\end{itemize}
\begin{itemize}
\item {Proveniência:(De \textunderscore âcoro\textunderscore )}
\end{itemize}
Sub-tribo de plantas da fam. das aroídeas.
\section{Acório}
\begin{itemize}
\item {Grp. gram.:m.}
\end{itemize}
Gênero de insectos coleópteros pentâmeros.
\section{Acorite}
\begin{itemize}
\item {Grp. gram.:f.}
\end{itemize}
Vinho com ácoro e outras substâncias.
\section{Açorite}
\begin{itemize}
\item {Grp. gram.:f.}
\end{itemize}
\begin{itemize}
\item {Proveniência:(De \textunderscore Açores, n. p.\textunderscore )}
\end{itemize}
Substância mineral, amarelada ou esverdeada, talvez um tantalato de cal.
\section{Acormóseo}
\begin{itemize}
\item {Grp. gram.:adj.}
\end{itemize}
\begin{itemize}
\item {Proveniência:(Do gr. \textunderscore a\textunderscore  + \textunderscore kormos\textunderscore )}
\end{itemize}
Diz-se das plantas, cujas fôlhas nascem da raiz.
\section{Acornar}
\begin{itemize}
\item {Grp. gram.:v. t.}
\end{itemize}
Dar fórma de côrno a.
\section{Ácoro}
\begin{itemize}
\item {Grp. gram.:m.}
\end{itemize}
\begin{itemize}
\item {Proveniência:(Lat. \textunderscore acorum\textunderscore )}
\end{itemize}
Planta medicinal, (\textunderscore calamus aromaticus\textunderscore ).
\section{Acoroçoadamente}
\begin{itemize}
\item {Grp. gram.:adv.}
\end{itemize}
De modo \textunderscore acoroçoado\textunderscore .
\section{Acoroçoado}
\begin{itemize}
\item {Grp. gram.:adj.}
\end{itemize}
\begin{itemize}
\item {Proveniência:(De \textunderscore acoroçoar\textunderscore )}
\end{itemize}
Animado; incitado.
\section{Acoroçoador}
\begin{itemize}
\item {Grp. gram.:adj.}
\end{itemize}
Que acoroçôa.
\section{Acoroçoamento}
\begin{itemize}
\item {Grp. gram.:m.}
\end{itemize}
Acto de \textunderscore acoroçoar\textunderscore .
\section{Acoroçoar}
\begin{itemize}
\item {Grp. gram.:v. t.}
\end{itemize}
\begin{itemize}
\item {Proveniência:(De \textunderscore coração\textunderscore )}
\end{itemize}
Incitar; animar.
\section{Acorredor}
\begin{itemize}
\item {Grp. gram.:adj.}
\end{itemize}
Que vem em auxílio. Cf. Filinto, \textunderscore D. Man.\textunderscore , II, 222.
\section{Acorreitar}
\begin{itemize}
\item {Grp. gram.:v. i.}
\end{itemize}
\begin{itemize}
\item {Utilização:Prov.}
\end{itemize}
\begin{itemize}
\item {Utilização:trasm.}
\end{itemize}
Melhorar de uma doença.
\section{Acorrentamento}
\begin{itemize}
\item {Grp. gram.:m.}
\end{itemize}
Acto de \textunderscore acorrentar\textunderscore .
\section{Acorrentar}
\begin{itemize}
\item {Grp. gram.:v. t.}
\end{itemize}
\begin{itemize}
\item {Grp. gram.:V. i.}
\end{itemize}
\begin{itemize}
\item {Utilização:Prov.}
\end{itemize}
\begin{itemize}
\item {Utilização:trasm.}
\end{itemize}
Prender com corrente; encadear.
Curar-se de uma doença; restabelecer-se.
\section{Acorrer}
\begin{itemize}
\item {Grp. gram.:v. i.}
\end{itemize}
\begin{itemize}
\item {Proveniência:(Lat. \textunderscore accurrere\textunderscore )}
\end{itemize}
Ir em auxílio; acudir.
\section{Acorrilhar}
\begin{itemize}
\item {Grp. gram.:v. t.}
\end{itemize}
Meter em corro.
Acantoar.
\section{Acorrimento}
\begin{itemize}
\item {Grp. gram.:m.}
\end{itemize}
\begin{itemize}
\item {Utilização:Ant.}
\end{itemize}
\begin{itemize}
\item {Proveniência:(De \textunderscore accorrer\textunderscore )}
\end{itemize}
Soccorro, auxílio.
\section{Acorro}
\begin{itemize}
\item {Grp. gram.:m.}
\end{itemize}
\begin{itemize}
\item {Proveniência:(De \textunderscore accorrer\textunderscore )}
\end{itemize}
O mesmo que \textunderscore soccorro\textunderscore .
\section{Acortinamento}
\begin{itemize}
\item {Grp. gram.:m.}
\end{itemize}
Acto ou effeito de \textunderscore acortinar\textunderscore .
\section{Acortinar}
\begin{itemize}
\item {Grp. gram.:v. t.}
\end{itemize}
Ornar com cortinas.
\section{Acoruchado}
\begin{itemize}
\item {Grp. gram.:adj.}
\end{itemize}
Que tem feitio de \textunderscore coruchéu\textunderscore .
\section{Acosmia}
\begin{itemize}
\item {Grp. gram.:f.}
\end{itemize}
\begin{itemize}
\item {Proveniência:(Gr. \textunderscore akosmia\textunderscore )}
\end{itemize}
Irregularidade no período crítico de uma doença.
\section{Acosmo}
\begin{itemize}
\item {Grp. gram.:m.}
\end{itemize}
\begin{itemize}
\item {Proveniência:(Gr. \textunderscore akosmos\textunderscore )}
\end{itemize}
Insecto coleóptero heterómero, do Cabo da Bôa-Esperança.
\section{Acossa}
\begin{itemize}
\item {Grp. gram.:f.}
\end{itemize}
\begin{itemize}
\item {Utilização:Pop.}
\end{itemize}
Acto ou effeito de \textunderscore acossar\textunderscore .
Perseguição.
Estafa.
\section{Acossadamente}
\begin{itemize}
\item {Grp. gram.:adv.}
\end{itemize}
De modo \textunderscore acossado\textunderscore .
\section{Acossado}
\begin{itemize}
\item {Grp. gram.:adj.}
\end{itemize}
Em que há perseguição.
Perseguido.
\section{Acossamento}
\begin{itemize}
\item {Grp. gram.:m.}
\end{itemize}
Acto de \textunderscore acossar\textunderscore .
\section{Acossar}
\begin{itemize}
\item {Grp. gram.:v. t.}
\end{itemize}
\begin{itemize}
\item {Proveniência:(De \textunderscore côsso\textunderscore , por \textunderscore côrso?\textunderscore )}
\end{itemize}
Ir no encalço de; perseguir.
Dar caça a.
\section{Acostadamente}
\begin{itemize}
\item {Grp. gram.:adv.}
\end{itemize}
De modo \textunderscore acostado\textunderscore .
\section{Acostado}
\begin{itemize}
\item {Grp. gram.:adj.}
\end{itemize}
\begin{itemize}
\item {Utilização:Ant.}
\end{itemize}
\begin{itemize}
\item {Grp. gram.:M.}
\end{itemize}
\begin{itemize}
\item {Utilização:Ant.}
\end{itemize}
\begin{itemize}
\item {Utilização:Prov.}
\end{itemize}
\begin{itemize}
\item {Utilização:alg.}
\end{itemize}
\begin{itemize}
\item {Proveniência:(De \textunderscore acostar\textunderscore )}
\end{itemize}
Assoldadado por fidalgos antigos.
Aquelle que vivia ao lado de um príncipe ou fidalgo, ou ao seu serviço, com certo ordenado ou moradia. Cf. Herculano, \textunderscore Bobo\textunderscore , 254.
Embarcação, que acompanha e ajuda os galeões de pesca.
\section{Acostamento}
\begin{itemize}
\item {Grp. gram.:m.}
\end{itemize}
\begin{itemize}
\item {Proveniência:(De \textunderscore acostar\textunderscore )}
\end{itemize}
Moradia, que se dava aos fidalgos da côrte. Cf. Herculano, \textunderscore Abóbada\textunderscore .
\section{Acostar}
\begin{itemize}
\item {Grp. gram.:v. t.}
\end{itemize}
\begin{itemize}
\item {Grp. gram.:V. i.}
\end{itemize}
\begin{itemize}
\item {Utilização:Prov.}
\end{itemize}
\begin{itemize}
\item {Utilização:trasm.}
\end{itemize}
\begin{itemize}
\item {Utilização:Ant.}
\end{itemize}
\begin{itemize}
\item {Proveniência:(De \textunderscore costa\textunderscore )}
\end{itemize}
Encostar.
Juntar.
Estar de acôrdo, annuir.
Confinar, ser limítrophe ou contíguo, (falando-se de prédios).
\section{Acostável}
\begin{itemize}
\item {Grp. gram.:adj.}
\end{itemize}
Diz-se do caes, a que as embarcações se podem acostar.
\section{Acostumadamente}
\begin{itemize}
\item {Grp. gram.:adv.}
\end{itemize}
De modo \textunderscore acostumado\textunderscore .
\section{Acostumado}
\begin{itemize}
\item {Grp. gram.:adj.}
\end{itemize}
Em que há costume.
Habituado; habitual.
\section{Acostumar}
\begin{itemize}
\item {Grp. gram.:v. t.}
\end{itemize}
Fazer adquirir um costume.
Habituar.
\section{Açoteia}
\begin{itemize}
\item {Grp. gram.:f.}
\end{itemize}
\begin{itemize}
\item {Utilização:Prov.}
\end{itemize}
\begin{itemize}
\item {Utilização:alg.}
\end{itemize}
\begin{itemize}
\item {Grp. gram.:f.}
\end{itemize}
\begin{itemize}
\item {Utilização:Ant.}
\end{itemize}
\begin{itemize}
\item {Proveniência:(Do ár. \textunderscore açotheia\textunderscore )}
\end{itemize}
Eirado ou terrado, em substituição do telhado.
O mesmo que \textunderscore assoteia\textunderscore .
(Cp. \textunderscore sótão\textunderscore )
Terraço; mirante.
\section{Acothurnado}
\begin{itemize}
\item {Grp. gram.:adj.}
\end{itemize}
\begin{itemize}
\item {Proveniência:(De \textunderscore cothurno\textunderscore )}
\end{itemize}
Diz-se do calçado, que cobre inteiramente o pé, á maneira de cothurno ou peúga.
\section{Acotiar}
\begin{itemize}
\item {Grp. gram.:v. t.}
\end{itemize}
\begin{itemize}
\item {Proveniência:(De \textunderscore cotio\textunderscore )}
\end{itemize}
Usar a cote.
Frequentar.
Ter persistência ou assiduidade em.
\section{Acotibóia}
\begin{itemize}
\item {Grp. gram.:f.}
\end{itemize}
Espécie de serpente do Brasil.
\section{Acoticar}
\begin{itemize}
\item {Grp. gram.:v. t.}
\end{itemize}
Atravessar com coticas.
\section{Acotiledóneas}
\begin{itemize}
\item {Grp. gram.:f. pl.}
\end{itemize}
Classe das plantas acotilédonas.
\section{Acotiledóneo}
\begin{itemize}
\item {Grp. gram.:adj.}
\end{itemize}
Que não tem \textunderscore cotilédones\textunderscore .
\section{Acotilédono}
\begin{itemize}
\item {Grp. gram.:adj.}
\end{itemize}
Que não tem \textunderscore cotilédones\textunderscore .
\section{Acotoar}
\begin{itemize}
\item {Grp. gram.:v. t.}
\end{itemize}
Cobrir de cotão.
\section{Acotovelado}
\begin{itemize}
\item {Grp. gram.:adj.}
\end{itemize}
Que tem fórma de cotovelo. Cf. F. Lapa, \textunderscore Techn. Rur.\textunderscore , 245.
\section{Acotovelador}
\begin{itemize}
\item {Grp. gram.:m.}
\end{itemize}
\begin{itemize}
\item {Proveniência:(De \textunderscore acotovelar\textunderscore )}
\end{itemize}
O que acotovela.
\section{Acotovelamento}
\begin{itemize}
\item {Grp. gram.:m.}
\end{itemize}
Acto de \textunderscore acotovelar\textunderscore .
\section{Acotovelar}
\begin{itemize}
\item {Grp. gram.:v. t.}
\end{itemize}
Tocar com o cotovelo.
Provocar.
\section{Acoturnado}
\begin{itemize}
\item {Grp. gram.:adj.}
\end{itemize}
\begin{itemize}
\item {Proveniência:(De \textunderscore cothurno\textunderscore )}
\end{itemize}
Diz-se do calçado, que cobre inteiramente o pé, á maneira de cothurno ou peúga.
\section{Acotyledóneas}
\begin{itemize}
\item {Grp. gram.:f. pl.}
\end{itemize}
Classe das plantas acotilédonas.
\section{Acotyledóneo}
\begin{itemize}
\item {Grp. gram.:adj.}
\end{itemize}
Que não tem \textunderscore cotylédones\textunderscore .
\section{Acotylédono}
\begin{itemize}
\item {Grp. gram.:adj.}
\end{itemize}
Que não tem \textunderscore cotylédones\textunderscore .
\section{Açougada}
\begin{itemize}
\item {Grp. gram.:f.}
\end{itemize}
\begin{itemize}
\item {Proveniência:(De \textunderscore açougue\textunderscore )}
\end{itemize}
Barulho, vozearia.
\section{Açougagem}
\begin{itemize}
\item {Grp. gram.:f.}
\end{itemize}
\begin{itemize}
\item {Utilização:Des.}
\end{itemize}
\begin{itemize}
\item {Proveniência:(De \textunderscore açougue\textunderscore )}
\end{itemize}
Imposto, que se pagava por qualquer lugar ou praça, em que se vendia carne, e também por aquelles em que se vendia pão, fruta, peixe, loiça, hortaliça, etc. Cf. Herculano, \textunderscore Hist. de Port.\textunderscore  IV, 420, 423.
O mesmo que \textunderscore açougada\textunderscore .
\section{Açougaria}
\begin{itemize}
\item {Grp. gram.:f.}
\end{itemize}
O mesmo que \textunderscore açougada\textunderscore .
\section{Açougue}
\begin{itemize}
\item {Grp. gram.:m.}
\end{itemize}
\begin{itemize}
\item {Utilização:Ant.}
\end{itemize}
\begin{itemize}
\item {Proveniência:(Do ár. \textunderscore as-sougue\textunderscore )}
\end{itemize}
Matadoiro.
Talho.
Lugar, onde se matam reses para consumo, ou onde se vende a carne dellas.
Matança.
Lugar ou mercado, em que se vendiam gêneros alimentícios. Cf. \textunderscore Port. au Point de Vue Agr.\textunderscore , p. XXXIII.
\section{Açougueiro}
\begin{itemize}
\item {Grp. gram.:m.}
\end{itemize}
\begin{itemize}
\item {Utilização:Bras}
\end{itemize}
Proprietário de açougue.
Carniceiro.
\section{Acourôa}
\begin{itemize}
\item {Grp. gram.:f.}
\end{itemize}
Árvore medicinal das Guianas.
\section{Açoutar}
\begin{itemize}
\item {Grp. gram.:v. t.}
\end{itemize}
Dar com açoite em.
Fustigar.
Bater.
Varejar.
Devastar.
Affligir.
\section{Açoute}
\begin{itemize}
\item {Grp. gram.:m.}
\end{itemize}
\begin{itemize}
\item {Utilização:Fam.}
\end{itemize}
Instrumento de tiras de coiro, para bater.
Azorrague.
Látego.
Pancada com a mão em nádegas de criança.
(Do ár.)
\section{Acovar}
\begin{itemize}
\item {Grp. gram.:v. t.}
\end{itemize}
O mesmo que \textunderscore encovar\textunderscore .
\section{Acovilhar}
\begin{itemize}
\item {Grp. gram.:v. t.}
\end{itemize}
\begin{itemize}
\item {Proveniência:(De \textunderscore covil\textunderscore )}
\end{itemize}
Dar agasalho a.
Recolher em casa.
\section{Acquiescência}
\begin{itemize}
\item {Grp. gram.:f.}
\end{itemize}
Acto de \textunderscore acquiescêr\textunderscore .
\section{Acquiescêr}
\begin{itemize}
\item {Grp. gram.:v. t.}
\end{itemize}
\begin{itemize}
\item {Proveniência:(Lat. \textunderscore acquiescere\textunderscore )}
\end{itemize}
Annuir, transigir.
\section{Acquirente}
\begin{itemize}
\item {Grp. gram.:adj.}
\end{itemize}
Que acquire.
\section{Acquirição}
\begin{itemize}
\item {Grp. gram.:f.}
\end{itemize}
\begin{itemize}
\item {Utilização:Ant.}
\end{itemize}
O mesmo que \textunderscore acquisição\textunderscore .
\section{Acquiridor}
\begin{itemize}
\item {Grp. gram.:m.}
\end{itemize}
O que acquire.
\section{Acquirimento}
\begin{itemize}
\item {Grp. gram.:m.}
\end{itemize}
\begin{itemize}
\item {Utilização:Ant.}
\end{itemize}
O mesmo que \textunderscore acquisição\textunderscore .
\section{Acquirir}
\begin{itemize}
\item {Grp. gram.:v. t.}
\end{itemize}
\begin{itemize}
\item {Utilização:Ant.}
\end{itemize}
O mesmo que \textunderscore adquirir\textunderscore .
\section{Acquiritivo}
\begin{itemize}
\item {Grp. gram.:adj.}
\end{itemize}
Próprio para acquirir.
\section{Acquisição}
\begin{itemize}
\item {Grp. gram.:f.}
\end{itemize}
Acto ou effeito de \textunderscore acquirir\textunderscore .
\section{Acquísito}
\begin{itemize}
\item {Grp. gram.:adj.}
\end{itemize}
\begin{itemize}
\item {Proveniência:(Lat. \textunderscore acquisitus\textunderscore )}
\end{itemize}
Adquirido. Cf. Vieira, VI, 277.
\section{Acquistar}
\begin{itemize}
\item {Grp. gram.:v. t.}
\end{itemize}
\begin{itemize}
\item {Utilização:Ant.}
\end{itemize}
\begin{itemize}
\item {Proveniência:(De \textunderscore acquisto\textunderscore )}
\end{itemize}
O mesmo que \textunderscore adquirir\textunderscore .
\section{Acquisto}
\begin{itemize}
\item {Grp. gram.:m.}
\end{itemize}
\begin{itemize}
\item {Utilização:Des.}
\end{itemize}
Acquisicão.
Conquista.
(Contr. do lat. \textunderscore acquisitus\textunderscore )
\section{Acracia}
\begin{itemize}
\item {Grp. gram.:f.}
\end{itemize}
\begin{itemize}
\item {Utilização:Neol.}
\end{itemize}
\begin{itemize}
\item {Proveniência:(Do gr. \textunderscore a\textunderscore  priv. + \textunderscore kratein\textunderscore , governar)}
\end{itemize}
Falta de govêrno.
Desordem social, o mesmo que \textunderscore anarchia\textunderscore .
\section{Acradênia}
\begin{itemize}
\item {Grp. gram.:f.}
\end{itemize}
Gênero de plantas rutáceas.
\section{Acrania}
\begin{itemize}
\item {Grp. gram.:f.}
\end{itemize}
Falta de crânio.
(Cp. \textunderscore acrânio\textunderscore )
\section{Acranianos}
\begin{itemize}
\item {Grp. gram.:m. pl.}
\end{itemize}
\begin{itemize}
\item {Utilização:Zool.}
\end{itemize}
\begin{itemize}
\item {Proveniência:(De \textunderscore a\textunderscore  priv. + \textunderscore crânio\textunderscore )}
\end{itemize}
Vertebrados de ordem inferior, cujo esqueleto rudimentar é de tecido mucoso.
\section{Acrânio}
\begin{itemize}
\item {Grp. gram.:adj}
\end{itemize}
\begin{itemize}
\item {Proveniência:(De \textunderscore a\textunderscore  priv. + \textunderscore crânio\textunderscore )}
\end{itemize}
Que não tem crânio.
\section{Acraniota}
\begin{itemize}
\item {Grp. gram.:m.  e  adj.}
\end{itemize}
\begin{itemize}
\item {Proveniência:(De \textunderscore a\textunderscore  priv. e \textunderscore craniota\textunderscore )}
\end{itemize}
Diz-se dos animaes que não têm crânio.
\section{Acrantera}
\begin{itemize}
\item {Grp. gram.:f.}
\end{itemize}
\begin{itemize}
\item {Proveniência:(Do gr. \textunderscore akron\textunderscore  + \textunderscore antheros\textunderscore )}
\end{itemize}
Planta rubiácea de Ceilão.
\section{Acranthera}
\begin{itemize}
\item {Grp. gram.:f.}
\end{itemize}
\begin{itemize}
\item {Proveniência:(Do gr. \textunderscore akron\textunderscore  + \textunderscore antheros\textunderscore )}
\end{itemize}
Planta rubiácea de Ceilão.
\section{Acrantho}
\begin{itemize}
\item {Grp. gram.:m.}
\end{itemize}
\begin{itemize}
\item {Proveniência:(Gr. \textunderscore akrantos\textunderscore )}
\end{itemize}
Espécie de lagarto.
\section{Acranto}
\begin{itemize}
\item {Grp. gram.:m.}
\end{itemize}
\begin{itemize}
\item {Proveniência:(Gr. \textunderscore akrantos\textunderscore )}
\end{itemize}
Espécie de lagarto.
\section{Acrata}
\begin{itemize}
\item {Grp. gram.:m.}
\end{itemize}
\begin{itemize}
\item {Utilização:Neol.}
\end{itemize}
Partidário da acracia; anarchista.
\section{Acratera}
\begin{itemize}
\item {Grp. gram.:f}
\end{itemize}
\begin{itemize}
\item {Proveniência:(Do gr. \textunderscore akra\textunderscore  + \textunderscore ather\textunderscore )}
\end{itemize}
Planta gramínea do norte da Índia.
\section{Acrathera}
\begin{itemize}
\item {Grp. gram.:f}
\end{itemize}
\begin{itemize}
\item {Proveniência:(Do gr. \textunderscore akra\textunderscore  + \textunderscore ather\textunderscore )}
\end{itemize}
Planta gramínea do norte da Índia.
\section{Acrático}
\begin{itemize}
\item {Grp. gram.:adj.}
\end{itemize}
\begin{itemize}
\item {Utilização:Neol.}
\end{itemize}
Relativo á acracia ou aos acratas.
\section{Acratóforo}
\begin{itemize}
\item {Grp. gram.:m.}
\end{itemize}
\begin{itemize}
\item {Proveniência:(Gr. \textunderscore akratophoron\textunderscore )}
\end{itemize}
Taça para vinho, usada entre os Gregos e Romanos.
\section{Acratóphoro}
\begin{itemize}
\item {Grp. gram.:m.}
\end{itemize}
\begin{itemize}
\item {Proveniência:(Gr. \textunderscore akratophoron\textunderscore )}
\end{itemize}
Taça para vinho, usada entre os Gregos e Romanos.
\section{Acratópoto}
\begin{itemize}
\item {Grp. gram.:adj.}
\end{itemize}
\begin{itemize}
\item {Utilização:Des.}
\end{itemize}
\begin{itemize}
\item {Proveniência:(Do gr. \textunderscore akratos\textunderscore  + lat. \textunderscore potare\textunderscore )}
\end{itemize}
Que bebe vinho puro.
\section{Acravar}
\begin{itemize}
\item {Grp. gram.:v. t.}
\end{itemize}
\begin{itemize}
\item {Utilização:Fig.}
\end{itemize}
\begin{itemize}
\item {Proveniência:(De \textunderscore cravo\textunderscore )}
\end{itemize}
Atravessar com cravos.
Traspassar.
Attribular, affligir.
\section{Acre}
\begin{itemize}
\item {Grp. gram.:m.}
\end{itemize}
\begin{itemize}
\item {Proveniência:(Do al. \textunderscore aker\textunderscore )}
\end{itemize}
Medida agrária em alguns países.
\section{Acre}
\begin{itemize}
\item {Grp. gram.:adj.}
\end{itemize}
\begin{itemize}
\item {Proveniência:(Lat. \textunderscore acer\textunderscore , \textunderscore acris\textunderscore , \textunderscore acre\textunderscore )}
\end{itemize}
Que tem sabor picante.
Azêdo.
O mesmo que \textunderscore agre\textunderscore .
\section{Acreditador}
\begin{itemize}
\item {Grp. gram.:m.}
\end{itemize}
\begin{itemize}
\item {Proveniência:(De \textunderscore acreditar\textunderscore )}
\end{itemize}
O que acredita.
\section{Acreditar}
\begin{itemize}
\item {Grp. gram.:v. t.}
\end{itemize}
\begin{itemize}
\item {Grp. gram.:V. i.}
\end{itemize}
\begin{itemize}
\item {Proveniência:(De \textunderscore crédito\textunderscore )}
\end{itemize}
Têr fé em.
Dar crédito a.
Abonar, afiançar.
Têr fé, crêr.
\section{Acreditável}
\begin{itemize}
\item {Grp. gram.:adj.}
\end{itemize}
Que se póde ou se deve \textunderscore acreditar\textunderscore .
\section{Acre-doce}
\begin{itemize}
\item {Grp. gram.:m.  e  adj.}
\end{itemize}
O mesmo que \textunderscore agridoce\textunderscore :«\textunderscore o acre-doce das flores silvestres...\textunderscore »Camillo.
\section{Acredor}
\begin{itemize}
\item {fónica:cré}
\end{itemize}
\begin{itemize}
\item {Grp. gram.:m.}
\end{itemize}
O mesmo que \textunderscore crèdor\textunderscore .
\section{Acrejo}
\begin{itemize}
\item {Grp. gram.:m.}
\end{itemize}
\begin{itemize}
\item {Utilização:Ant.}
\end{itemize}
O mesmo que acredor.
\section{Acremente}
\begin{itemize}
\item {Grp. gram.:adv.}
\end{itemize}
De modo \textunderscore acre\textunderscore .
\section{Acrescentada}
\begin{itemize}
\item {Grp. gram.:adj. f.}
\end{itemize}
\begin{itemize}
\item {Utilização:Prov.}
\end{itemize}
\begin{itemize}
\item {Proveniência:(De \textunderscore accrescentar\textunderscore )}
\end{itemize}
Diz-se da mulher grávida. (Colhido em Turquel)
\section{Acrescentador}
\begin{itemize}
\item {Grp. gram.:m.  e  adj.}
\end{itemize}
O que acrescenta.
\section{Acrescentamento}
\begin{itemize}
\item {Grp. gram.:m.}
\end{itemize}
Acto ou effeito de \textunderscore acrescentar\textunderscore .
\section{Acrescentar}
\begin{itemize}
\item {Grp. gram.:v. t.}
\end{itemize}
\begin{itemize}
\item {Proveniência:(De \textunderscore accrescer\textunderscore )}
\end{itemize}
Tornar maior, aumentar.
Dar mais grandeza, fôrça ou número a.
\section{Acrescente}
\begin{itemize}
\item {Grp. gram.:m.}
\end{itemize}
\begin{itemize}
\item {Utilização:Pop.}
\end{itemize}
Acto de \textunderscore acrescentar\textunderscore .
Acrescentamento.
O mesmo que \textunderscore chinó\textunderscore .
\section{Acrescento}
\begin{itemize}
\item {Grp. gram.:m.}
\end{itemize}
O mesmo que \textunderscore acrescentamento\textunderscore . Cf. Castilho, \textunderscore Avarento\textunderscore , 144.
\section{Acrescer}
\begin{itemize}
\item {Grp. gram.:v. i.}
\end{itemize}
\begin{itemize}
\item {Grp. gram.:V. t.}
\end{itemize}
\begin{itemize}
\item {Proveniência:(Lat. \textunderscore accrescere\textunderscore )}
\end{itemize}
Ajuntar-se.
Sobrevir.
Juntar.
Aumentar.
\section{Acrescido}
\begin{itemize}
\item {Grp. gram.:m.}
\end{itemize}
Aquillo que acresceu.
Annexo, accessório.
Dependência:«\textunderscore uma lei de Affonso III sobre os accrescidos dos rios.\textunderscore »\textunderscore Port. Mon. Hist.\textunderscore , I, 149.
\section{Acrescimento}
\begin{itemize}
\item {Grp. gram.:m.}
\end{itemize}
Acto ou effeito de \textunderscore acrescer\textunderscore .
\section{Acréscimo}
\begin{itemize}
\item {Grp. gram.:m.}
\end{itemize}
O mesmo que \textunderscore acrescimento\textunderscore , febre intermittente.
\section{Acriançado}
\begin{itemize}
\item {Grp. gram.:adj.}
\end{itemize}
Que tem modos de criança.
Ingênuo.
Leviano.
\section{Acriançar-se}
\begin{itemize}
\item {Grp. gram.:v. p.}
\end{itemize}
Adquirir modos de criança; fazer-se criança.
\section{Acribologia}
\begin{itemize}
\item {Grp. gram.:f.}
\end{itemize}
\begin{itemize}
\item {Proveniência:(Gr. \textunderscore akribologia\textunderscore )}
\end{itemize}
Rigor e precisão no estilo.
\section{Acribólogo}
\begin{itemize}
\item {Grp. gram.:m.}
\end{itemize}
O que pratica a \textunderscore acribologia\textunderscore .
\section{Acribómetro}
\begin{itemize}
\item {Grp. gram.:m.}
\end{itemize}
\begin{itemize}
\item {Proveniência:(Do gr. \textunderscore akribes\textunderscore  + \textunderscore metron\textunderscore )}
\end{itemize}
Instrumento, para medir objectos muito pequenos.
\section{Acridão}
\begin{itemize}
\item {Grp. gram.:f.}
\end{itemize}
O mesmo que \textunderscore acridez\textunderscore .
\section{Acridez}
\begin{itemize}
\item {Grp. gram.:f.}
\end{itemize}
Qualidade do que é acre.
\section{Acrídia}
\begin{itemize}
\item {Grp. gram.:f.}
\end{itemize}
\begin{itemize}
\item {Proveniência:(Do gr. \textunderscore akris\textunderscore , \textunderscore akridos\textunderscore )}
\end{itemize}
O mesmo que \textunderscore gafanhoto\textunderscore .
\section{Acridiano}
\begin{itemize}
\item {Grp. gram.:adj.}
\end{itemize}
\begin{itemize}
\item {Grp. gram.:M. pl.}
\end{itemize}
\begin{itemize}
\item {Proveniência:(De \textunderscore acrídia\textunderscore )}
\end{itemize}
Relativo ou semelhante ao gafanhoto.
Família de insectos, que têm por typo o gafanhoto.
\section{Acrídio}
\begin{itemize}
\item {Grp. gram.:adj.}
\end{itemize}
\begin{itemize}
\item {Grp. gram.:M. pl.}
\end{itemize}
\begin{itemize}
\item {Proveniência:(De \textunderscore acrídia\textunderscore )}
\end{itemize}
Relativo ou semelhante ao gafanhoto.
Família de insectos, que têm por typo o gafanhoto.
\section{Acridocarpo}
\begin{itemize}
\item {Grp. gram.:m.}
\end{itemize}
\begin{itemize}
\item {Proveniência:(Do gr. \textunderscore akris\textunderscore  + \textunderscore karpos\textunderscore )}
\end{itemize}
Planta americana, cujos frutos têm semelhança com os gafanhotos.
\section{Acridófago}
\begin{itemize}
\item {Grp. gram.:m.}
\end{itemize}
\begin{itemize}
\item {Proveniência:(Do gr. \textunderscore akris\textunderscore  + \textunderscore phagein\textunderscore )}
\end{itemize}
O que se alimenta de gafanhotos.
\section{Acridogenose}
\begin{itemize}
\item {Grp. gram.:f.}
\end{itemize}
\begin{itemize}
\item {Proveniência:(Do gr. \textunderscore akris\textunderscore  + \textunderscore genos\textunderscore )}
\end{itemize}
Doença dos vegetaes, produzida pelos gafanhotos.
\section{Acridóphago}
\begin{itemize}
\item {Grp. gram.:m.}
\end{itemize}
\begin{itemize}
\item {Proveniência:(Do gr. \textunderscore akris\textunderscore  + \textunderscore phagein\textunderscore )}
\end{itemize}
O que se alimenta de gafanhotos.
\section{Acridótero}
\begin{itemize}
\item {Grp. gram.:m.}
\end{itemize}
\begin{itemize}
\item {Proveniência:(Do gr. \textunderscore akris\textunderscore  + \textunderscore therao\textunderscore )}
\end{itemize}
Ave que come gafanhotos; gaivão.
\section{Acrífico}
\begin{itemize}
\item {Grp. gram.:adj.}
\end{itemize}
\begin{itemize}
\item {Utilização:Des.}
\end{itemize}
Que se tornou acre.
Azêdo.
Que tem mau humor. Cf. \textunderscore Anat. Joc.\textunderscore , I, 358.
\section{Acrimancia}
\begin{itemize}
\item {Grp. gram.:f.}
\end{itemize}
Supposta arte de adivinhar, por meio do fogo. Cf. Castilho, \textunderscore Fastos\textunderscore , III, 311.
\section{Acriminar}
\begin{itemize}
\item {Grp. gram.:v. t.}
\end{itemize}
O mesmo que \textunderscore criminar\textunderscore .
\section{Acrimónia}
\begin{itemize}
\item {Grp. gram.:f.}
\end{itemize}
\begin{itemize}
\item {Proveniência:(Lat. \textunderscore acrimonia\textunderscore )}
\end{itemize}
O mesmo que \textunderscore acridez\textunderscore .
Azedume.
Aspereza.
\section{Acrimoniar}
\begin{itemize}
\item {Grp. gram.:v. t.}
\end{itemize}
Tornar acrimonioso:«\textunderscore vá,--disse Francisco, acrimoníando o monosýllabo.\textunderscore »Camillo,
\textunderscore Caveira\textunderscore , 166.
\section{Acrimonioso}
\begin{itemize}
\item {Grp. gram.:adj.}
\end{itemize}
Que tem \textunderscore acrimónia\textunderscore .
\section{Acrisolado}
\begin{itemize}
\item {Grp. gram.:adj.}
\end{itemize}
Purificado.
Intenso: \textunderscore amor acrisolado\textunderscore .
\section{Acrisolador}
\begin{itemize}
\item {Grp. gram.:m.}
\end{itemize}
O que acrisola.
\section{Acrisolamento}
\begin{itemize}
\item {Grp. gram.:m.}
\end{itemize}
Acto de \textunderscore acrisolar\textunderscore . Cf. F. Lapa, \textunderscore Techn. Rur.\textunderscore , 290.
\section{Acrisolar}
\begin{itemize}
\item {Grp. gram.:v. t.}
\end{itemize}
Apurar no crisol.
Purificar.
Acendrar.
\section{Ácritos}
\begin{itemize}
\item {Grp. gram.:m. pl.}
\end{itemize}
\begin{itemize}
\item {Proveniência:(Gr. \textunderscore akritos\textunderscore , indeciso)}
\end{itemize}
Divisão do reino animal, que comprehende infusórios, pólypos e parte dos intestinaes.
\section{Acritude}
\begin{itemize}
\item {Grp. gram.:f.}
\end{itemize}
O mesmo que \textunderscore acridez\textunderscore .
\section{Acrivar}
\begin{itemize}
\item {Grp. gram.:v. t.}
\end{itemize}
\begin{itemize}
\item {Utilização:Prov.}
\end{itemize}
\begin{itemize}
\item {Proveniência:(De \textunderscore crivo\textunderscore )}
\end{itemize}
O mesmo que \textunderscore joeirar\textunderscore .
\section{Acro}
\begin{itemize}
\item {Grp. gram.:adj.}
\end{itemize}
O mesmo que \textunderscore acre\textunderscore ^2.
\section{Acróama}
\begin{itemize}
\item {Grp. gram.:m.}
\end{itemize}
\begin{itemize}
\item {Proveniência:(Gr. \textunderscore akroama\textunderscore )}
\end{itemize}
Canto ou discurso harmonioso.
\section{Acroamático}
\begin{itemize}
\item {Grp. gram.:adj.}
\end{itemize}
\begin{itemize}
\item {Proveniência:(Gr. \textunderscore akroamatikos\textunderscore )}
\end{itemize}
Grato ao ouvido.
Sublime.
\section{Acroás}
\begin{itemize}
\item {Grp. gram.:m. pl.}
\end{itemize}
Selvagens do Brasil, que dominavam nas margens do Rio-Corrente, em Goiás.
\section{Acrobacia}
\begin{itemize}
\item {Grp. gram.:f.}
\end{itemize}
Arte de acrobata.
\section{Acrobata}
\begin{itemize}
\item {Grp. gram.:m.}
\end{itemize}
\begin{itemize}
\item {Proveniência:(Do gr. \textunderscore akros\textunderscore  + \textunderscore batein\textunderscore )}
\end{itemize}
O que dança em corda.
Saltimbanco.
Palhaço.
Equilibrista.
\section{Acróbata}
\begin{itemize}
\item {Grp. gram.:m.}
\end{itemize}
\begin{itemize}
\item {Proveniência:(Do gr. \textunderscore akros\textunderscore  + \textunderscore batein\textunderscore )}
\end{itemize}
O que dança em corda.
Saltimbanco.
Palhaço.
Equilibrista.
\section{Acrobaticão}
\begin{itemize}
\item {Grp. gram.:m.}
\end{itemize}
\begin{itemize}
\item {Proveniência:(Do gr. \textunderscore akrobatikon\textunderscore )}
\end{itemize}
Palanque, tablado, que os antigos construíam, para melhor observar o que se passava nas praças.
\section{Acrobático}
\begin{itemize}
\item {Grp. gram.:adj.}
\end{itemize}
Relativo a \textunderscore acrobata\textunderscore .
\section{Acrobatismo}
\begin{itemize}
\item {Grp. gram.:m.}
\end{itemize}
\begin{itemize}
\item {Grp. gram.:m.}
\end{itemize}
\begin{itemize}
\item {Utilização:Fig.}
\end{itemize}
Profissão ou exercicios de acrobata.
Difficuldade do equilibrio.
Instabilidade de opiniões. Cf. Castilho, \textunderscore Fastos\textunderscore , I, 137; e Camillo, \textunderscore Cavar em Ruin.\textunderscore , 110.
\section{Acrobustite}
\begin{itemize}
\item {Proveniência:(Do gr. \textunderscore akrobustia\textunderscore )}
\end{itemize}
\textunderscore f.\textunderscore 
Inflammação na pelle dos animaes.
\section{Acrocarpo}
\begin{itemize}
\item {Grp. gram.:m.}
\end{itemize}
\begin{itemize}
\item {Proveniência:(Do gr. \textunderscore akron\textunderscore  + \textunderscore karpos\textunderscore )}
\end{itemize}
Espécie de musgo, que frutifica na extremidade dos ramos.
\section{Acrocefalia}
\begin{itemize}
\item {Grp. gram.:f.}
\end{itemize}
Estado ou qualidade de \textunderscore acrocéfalo\textunderscore .
\section{Acrocéfalo}
\begin{itemize}
\item {Grp. gram.:adj.}
\end{itemize}
\begin{itemize}
\item {Proveniência:(Do gr. \textunderscore akros\textunderscore  + \textunderscore kephale\textunderscore )}
\end{itemize}
Que tem grande altura de crânio.
\section{Acrocephalia}
\begin{itemize}
\item {Grp. gram.:f.}
\end{itemize}
Estado ou qualidade de \textunderscore acrocéphalo\textunderscore .
\section{Acrocéphalo}
\begin{itemize}
\item {Grp. gram.:adj.}
\end{itemize}
\begin{itemize}
\item {Proveniência:(Do gr. \textunderscore akros\textunderscore  + \textunderscore kephale\textunderscore )}
\end{itemize}
Que tem grande altura de crânio.
\section{Acrochado}
\begin{itemize}
\item {Grp. gram.:adj.}
\end{itemize}
\begin{itemize}
\item {Utilização:Prov.}
\end{itemize}
\begin{itemize}
\item {Utilização:trasm.}
\end{itemize}
Muito embuçado.
Embiocado (Por \textunderscore acarochado\textunderscore , de \textunderscore carocha\textunderscore ).
\section{Acrochar-se}
\begin{itemize}
\item {Grp. gram.:v. p.}
\end{itemize}
\begin{itemize}
\item {Utilização:Prov.}
\end{itemize}
\begin{itemize}
\item {Utilização:trasm.}
\end{itemize}
Embiocar-se.
Tapar o rosto quási todo, puxando para a frente o lenço ou envolvendo a cabeça com o chale.
(Por \textunderscore acarochar-se\textunderscore , de \textunderscore carocha\textunderscore )
\section{Acrocheta}
\begin{itemize}
\item {fónica:quê}
\end{itemize}
\begin{itemize}
\item {Proveniência:(Do gr. \textunderscore akron\textunderscore  + \textunderscore khaita\textunderscore )}
\end{itemize}
Insecto díptero do Brasil.
\section{Acrochirismo}
\begin{itemize}
\item {fónica:qui}
\end{itemize}
\begin{itemize}
\item {Grp. gram.:m.}
\end{itemize}
\begin{itemize}
\item {Proveniência:(Do gr. \textunderscore akron\textunderscore  + \textunderscore kheir\textunderscore )}
\end{itemize}
Na gymnástica antiga, espécie de luta, em que os lutadores apenas se serviam da extremidade dos dedos.
\section{Acrocino}
\begin{itemize}
\item {Grp. gram.:m.}
\end{itemize}
\begin{itemize}
\item {Proveniência:(Do gr. \textunderscore akron\textunderscore  + \textunderscore kineios\textunderscore )}
\end{itemize}
Insecto coleóptero.
\section{Acroclado}
\begin{itemize}
\item {Grp. gram.:m.}
\end{itemize}
Gênero de algas.
\section{Acrocómia}
\begin{itemize}
\item {Grp. gram.:f.}
\end{itemize}
O mesmo que \textunderscore acrócomo\textunderscore .
\section{Acrócomo}
\begin{itemize}
\item {Grp. gram.:m.}
\end{itemize}
\begin{itemize}
\item {Proveniência:(Do gr. \textunderscore akron\textunderscore  + \textunderscore komé\textunderscore )}
\end{itemize}
Espécie de palmeira.
\section{Acrocórdio}
\begin{itemize}
\item {Grp. gram.:m.}
\end{itemize}
\begin{itemize}
\item {Proveniência:(Do gr. \textunderscore akrokhordon\textunderscore )}
\end{itemize}
Reptil ophidio, não venenôso.
\section{Acrodinia}
\begin{itemize}
\item {Grp. gram.:f.}
\end{itemize}
\begin{itemize}
\item {Proveniência:(Do gr. \textunderscore akron\textunderscore  + \textunderscore odune\textunderscore )}
\end{itemize}
Moléstia epidêmica, caracterizada por uma dolorosa comichão nos pés ou nas mãos.
\section{Acrodynia}
\begin{itemize}
\item {Grp. gram.:f.}
\end{itemize}
\begin{itemize}
\item {Proveniência:(Do gr. \textunderscore akron\textunderscore  + \textunderscore odune\textunderscore )}
\end{itemize}
Moléstia epidêmica, caracterizada por uma dolorosa comichão nos pés ou nas mãos.
\section{Acrofobia}
\begin{itemize}
\item {Grp. gram.:f.}
\end{itemize}
\begin{itemize}
\item {Proveniência:(Do gr. \textunderscore akron\textunderscore , cume, + \textunderscore phobos\textunderscore , medo)}
\end{itemize}
Receio mórbido de lugares muito altos.
\section{Acróforo}
\begin{itemize}
\item {Grp. gram.:m.}
\end{itemize}
\begin{itemize}
\item {Proveniência:(Do gr. \textunderscore akros\textunderscore  + \textunderscore pheros\textunderscore )}
\end{itemize}
Apparelho portátil, com reservatório do ar, para substituir a respiração de uma atmosphera viciada ou deletéria.
\section{Acrogênias}
\begin{itemize}
\item {Grp. gram.:f. pl.}
\end{itemize}
\begin{itemize}
\item {Proveniência:(Do gr. \textunderscore akros\textunderscore  + \textunderscore genos\textunderscore )}
\end{itemize}
Plantas acotyledóneas, cujo crescimento se manifesta só na parte superior.
\section{Acroleato}
\begin{itemize}
\item {Grp. gram.:m.}
\end{itemize}
Sal, formado pela combinação do ácido acroleico com uma base.
\section{Acroleico}
\begin{itemize}
\item {Grp. gram.:adj.}
\end{itemize}
Diz-se de um ácido, resultante da oxydação da acroleína.
\section{Acroleína}
\begin{itemize}
\item {Grp. gram.:f.}
\end{itemize}
\begin{itemize}
\item {Proveniência:(De \textunderscore aere\textunderscore  + \textunderscore oleina\textunderscore )}
\end{itemize}
Liquido incolor, que se obtém pela destillação de uma mistura de glycerina e ácido phosphórico anhydro.
\section{Acrólitho}
\begin{itemize}
\item {Grp. gram.:m.}
\end{itemize}
\begin{itemize}
\item {Proveniência:(Do gr. \textunderscore akros\textunderscore  + \textunderscore lithos\textunderscore )}
\end{itemize}
Estátua antiga, cuja extremidade superior era de pedra, e de outra substância o resto.
\section{Acrólito}
\begin{itemize}
\item {Grp. gram.:m.}
\end{itemize}
\begin{itemize}
\item {Proveniência:(Do gr. \textunderscore akros\textunderscore  + \textunderscore lithos\textunderscore )}
\end{itemize}
Estátua antiga, cuja extremidade superior era de pedra, e de outra substância o resto.
\section{Acrologia}
\begin{itemize}
\item {Grp. gram.:f.}
\end{itemize}
\begin{itemize}
\item {Proveniência:(Gr. \textunderscore akron\textunderscore  + \textunderscore logos\textunderscore )}
\end{itemize}
Investigação do absoluto, dos primeiros principios.
\section{Acrológico}
\begin{itemize}
\item {Grp. gram.:adj.}
\end{itemize}
Relativo á \textunderscore acrologia\textunderscore .
\section{Acrólofo}
\begin{itemize}
\item {Grp. gram.:m.}
\end{itemize}
\begin{itemize}
\item {Proveniência:(Do gr. \textunderscore akron\textunderscore  + \textunderscore lophos\textunderscore )}
\end{itemize}
Insecto lepidóptero nocturno.
\section{Acrólopho}
\begin{itemize}
\item {Grp. gram.:m.}
\end{itemize}
\begin{itemize}
\item {Proveniência:(Do gr. \textunderscore akron\textunderscore  + \textunderscore lophos\textunderscore )}
\end{itemize}
Insecto lepidóptero nocturno.
\section{Acromania}
\begin{itemize}
\item {Grp. gram.:f.}
\end{itemize}
\begin{itemize}
\item {Proveniência:(Do gr. \textunderscore akros\textunderscore  + \textunderscore mania\textunderscore )}
\end{itemize}
Loucura completa, incurável.
\section{Acromasia}
\begin{itemize}
\item {Grp. gram.:f.}
\end{itemize}
\begin{itemize}
\item {Proveniência:(Do gr. \textunderscore a\textunderscore  priv. + \textunderscore khroma\textunderscore , côr)}
\end{itemize}
Pallidez cachéctica.
\section{Acromático}
\begin{itemize}
\item {Grp. gram.:adj.}
\end{itemize}
\begin{itemize}
\item {Proveniência:(Do gr. \textunderscore a\textunderscore  priv. + \textunderscore khroma\textunderscore , côr)}
\end{itemize}
Que faz desapparecer as irisações produzidas por certas lentes.
\section{Acromatina}
\begin{itemize}
\item {Grp. gram.:f.}
\end{itemize}
\begin{itemize}
\item {Proveniência:(Gr. \textunderscore akhromatos\textunderscore , sem côr)}
\end{itemize}
Parte da substância do núcleo cellular, sôbre a qual não têm acção os reagentes còrantes.
\section{Acromatismo}
\begin{itemize}
\item {Grp. gram.:m.}
\end{itemize}
Qualidade do objecto \textunderscore acromático\textunderscore .
\section{Acromatizaçao}
\begin{itemize}
\item {Grp. gram.:f.}
\end{itemize}
Acto de \textunderscore acromatizar\textunderscore .
\section{Acromatizar}
\begin{itemize}
\item {Grp. gram.:v. t.}
\end{itemize}
\begin{itemize}
\item {Proveniência:(De \textunderscore achromático\textunderscore )}
\end{itemize}
Fazer desapparecer (as côres irisadas) na imagem de um objecto.
\section{Acromatopsía}
\begin{itemize}
\item {Grp. gram.:f.}
\end{itemize}
\begin{itemize}
\item {Proveniência:(Do gr. \textunderscore a\textunderscore  priv. + \textunderscore khroma\textunderscore  + \textunderscore ops\textunderscore )}
\end{itemize}
Estado de quem não póde distinguir as côres.
\section{Acromatóptico}
\begin{itemize}
\item {Grp. gram.:adj.}
\end{itemize}
Que tem achromatopsia.
\section{Acromegalia}
\begin{itemize}
\item {Grp. gram.:f.}
\end{itemize}
\begin{itemize}
\item {Utilização:Med.}
\end{itemize}
\begin{itemize}
\item {Proveniência:(Do gr. \textunderscore akron\textunderscore , extremidade, + \textunderscore megas\textunderscore , grande)}
\end{itemize}
Trophoneurose, caracterizada pelo crescimento notável das extremidades do corpo, especialmente mãos, pés e cabeça.
\section{Acromelalgia}
\begin{itemize}
\item {Grp. gram.:f.}
\end{itemize}
\begin{itemize}
\item {Utilização:Med.}
\end{itemize}
\begin{itemize}
\item {Proveniência:(Do gr. \textunderscore akron\textunderscore  + \textunderscore melos\textunderscore  + \textunderscore algos\textunderscore )}
\end{itemize}
Moléstia, caracterizada por dores na extremidade dos membros.
\section{Acromia}
\begin{itemize}
\item {Grp. gram.:f.}
\end{itemize}
\begin{itemize}
\item {Utilização:Med.}
\end{itemize}
\begin{itemize}
\item {Proveniência:(Do gr. \textunderscore a\textunderscore  priv. + \textunderscore khroma\textunderscore , côr)}
\end{itemize}
Descoramento parcial da pelle.
\section{Acromial}
\begin{itemize}
\item {Grp. gram.:adj.}
\end{itemize}
Relativo ao \textunderscore acrómio\textunderscore .
\section{Acrómio}
\begin{itemize}
\item {Grp. gram.:m.}
\end{itemize}
\begin{itemize}
\item {Proveniência:(Lat. \textunderscore acromium\textunderscore )}
\end{itemize}
Apóphyse, que termina a espinha da omoplata.
\section{Acromo}
\begin{itemize}
\item {Grp. gram.:adj.}
\end{itemize}
\begin{itemize}
\item {Proveniência:(Do gr. \textunderscore a\textunderscore  priv. + \textunderscore khroma\textunderscore , côr)}
\end{itemize}
Que não tem côr.
\section{Acromodermia}
\begin{itemize}
\item {Grp. gram.:f.}
\end{itemize}
O mesmo que \textunderscore acromasia\textunderscore .
\section{Acromolena}
\begin{itemize}
\item {Grp. gram.:f.}
\end{itemize}
\begin{itemize}
\item {Proveniência:(Do gr. \textunderscore chroma\textunderscore  + \textunderscore laina\textunderscore )}
\end{itemize}
Planta composta, originária da Nova Hollanda.
\section{Acromphálio}
\begin{itemize}
\item {Grp. gram.:m.}
\end{itemize}
\begin{itemize}
\item {Utilização:Anat.}
\end{itemize}
\begin{itemize}
\item {Proveniência:(Do gr. \textunderscore akron\textunderscore  + \textunderscore omphalos\textunderscore )}
\end{itemize}
Extremidade do cordão umbilical, que fica presa ao feto, depois do nascimento.
\section{Acronfálio}
\begin{itemize}
\item {Grp. gram.:m.}
\end{itemize}
\begin{itemize}
\item {Utilização:Anat.}
\end{itemize}
\begin{itemize}
\item {Proveniência:(Do gr. \textunderscore akron\textunderscore  + \textunderscore omphalos\textunderscore )}
\end{itemize}
Extremidade do cordão umbilical, que fica presa ao feto, depois do nascimento.
\section{Acrónico}
\begin{itemize}
\item {Grp. gram.:adj.}
\end{itemize}
\begin{itemize}
\item {Proveniência:(Gr. \textunderscore akronukhos\textunderscore )}
\end{itemize}
Diz-se de um astro, que apparece em lugar opposto ao do sol.
\section{Ácrono}
\begin{itemize}
\item {Grp. gram.:m.}
\end{itemize}
Ovário vegetal, que se alonga na base, formando uma espécie de disco carnudo.
\section{Acrónyco}
\begin{itemize}
\item {Grp. gram.:adj.}
\end{itemize}
\begin{itemize}
\item {Proveniência:(Gr. \textunderscore akronukhos\textunderscore )}
\end{itemize}
Diz-se de um astro, que apparece em lugar opposto ao do sol.
\section{Acropathia}
\begin{itemize}
\item {Grp. gram.:f.}
\end{itemize}
\begin{itemize}
\item {Proveniência:(Do gr. \textunderscore akros\textunderscore  + \textunderscore pathos\textunderscore )}
\end{itemize}
Doença na extremidade do corpo.
\section{Acropatia}
\begin{itemize}
\item {Grp. gram.:f.}
\end{itemize}
\begin{itemize}
\item {Proveniência:(Do gr. \textunderscore akros\textunderscore  + \textunderscore pathos\textunderscore )}
\end{itemize}
Doença na extremidade do corpo.
\section{Acrophobia}
\begin{itemize}
\item {Grp. gram.:f.}
\end{itemize}
\begin{itemize}
\item {Proveniência:(Do gr. \textunderscore akron\textunderscore , cume, + \textunderscore phobos\textunderscore , medo)}
\end{itemize}
Receio mórbido de lugares muito altos.
\section{Acróphoro}
\begin{itemize}
\item {Grp. gram.:m.}
\end{itemize}
\begin{itemize}
\item {Proveniência:(Do gr. \textunderscore akros\textunderscore  + \textunderscore pheros\textunderscore )}
\end{itemize}
Apparelho portátil, com reservatório do ar, para substituir a respiração de uma atmosphera viciada ou deletéria.
\section{Acropódio}
\begin{itemize}
\item {Grp. gram.:m.}
\end{itemize}
\begin{itemize}
\item {Proveniência:(Lat. \textunderscore acropodium\textunderscore )}
\end{itemize}
Plintho baixo e quadrado, em que se firma uma estátua.
\section{Acrópole}
\begin{itemize}
\item {Grp. gram.:f.}
\end{itemize}
\begin{itemize}
\item {Proveniência:(Gr. \textunderscore akrópolis\textunderscore )}
\end{itemize}
A parte mais alta nas cidades gregas.
\section{Acropólio}
\begin{itemize}
\item {Grp. gram.:m.}
\end{itemize}
\begin{itemize}
\item {Utilização:Ant.}
\end{itemize}
O mesmo que \textunderscore acrópole\textunderscore .
\section{Acroquetas}
\begin{itemize}
\item {Proveniência:(Do gr. \textunderscore akron\textunderscore  + \textunderscore khaita\textunderscore )}
\end{itemize}
Insecto díptero do Brasil.
\section{Acroquirismo}
\begin{itemize}
\item {Grp. gram.:m.}
\end{itemize}
\begin{itemize}
\item {Proveniência:(Do gr. \textunderscore akron\textunderscore  + \textunderscore kheir\textunderscore )}
\end{itemize}
Na gymnástica antiga, espécie de luta, em que os lutadores apenas se serviam da extremidade dos dedos.
\section{Acrosofia}
\begin{itemize}
\item {Grp. gram.:f.}
\end{itemize}
\begin{itemize}
\item {Proveniência:(Do gr. \textunderscore akros\textunderscore  + \textunderscore sophos\textunderscore )}
\end{itemize}
A sabedoria superior.
A sabedoria de Deus.
\section{Acrosophia}
\begin{itemize}
\item {Grp. gram.:f.}
\end{itemize}
\begin{itemize}
\item {Proveniência:(Do gr. \textunderscore akros\textunderscore  + \textunderscore sophos\textunderscore )}
\end{itemize}
A sabedoria superior.
A sabedoria de Deus.
\section{Acrospermo}
\begin{itemize}
\item {Grp. gram.:m.}
\end{itemize}
\begin{itemize}
\item {Proveniência:(Do gr. \textunderscore akros\textunderscore  + \textunderscore sperma\textunderscore )}
\end{itemize}
Espécie de cogumelo.
\section{Acróstico}
\begin{itemize}
\item {Grp. gram.:m.}
\end{itemize}
\begin{itemize}
\item {Proveniência:(Do gr. \textunderscore akros\textunderscore  + \textunderscore stikhos\textunderscore )}
\end{itemize}
Composição poética, cujo objecto é determinado por uma palavra, cada uma de cujas letras é a primeira de cada verso.
\section{Acrostólio}
\begin{itemize}
\item {Grp. gram.:m.}
\end{itemize}
\begin{itemize}
\item {Proveniência:(Gr. \textunderscore akrostolion\textunderscore )}
\end{itemize}
Ornato que, em fórma de capacete, escudo, pescoço de cysne, etc., os antigos collocavam na prôa dos navios.
\section{Acrotério}
\begin{itemize}
\item {Grp. gram.:m.}
\end{itemize}
\begin{itemize}
\item {Proveniência:(Lat. \textunderscore acroterium\textunderscore )}
\end{itemize}
Pedestal das figuras, sobrepostas na frontaria de edificios.
\section{Acroteriose}
\begin{itemize}
\item {Grp. gram.:f.}
\end{itemize}
\begin{itemize}
\item {Proveniência:(De \textunderscore acrotério\textunderscore )}
\end{itemize}
Gangrena senil das extremidades dos membros.
\section{Acrotismo}
\begin{itemize}
\item {Grp. gram.:m.}
\end{itemize}
Falta de pulsações.
Philosophia transcendente.
(Cp. lat. \textunderscore acroteria\textunderscore )
\section{Acrotóforo}
\begin{itemize}
\item {Grp. gram.:m.}
\end{itemize}
\begin{itemize}
\item {Proveniência:(Do gr. \textunderscore akroton\textunderscore  + \textunderscore phoros\textunderscore )}
\end{itemize}
Vaso, em que os Romanos tinham á mesa o vinho puro.
\section{Acrotomia}
\begin{itemize}
\item {Grp. gram.:f.}
\end{itemize}
\begin{itemize}
\item {Utilização:Cir.}
\end{itemize}
\begin{itemize}
\item {Proveniência:(Do gr. \textunderscore akron\textunderscore , extremidade, + \textunderscore temnein\textunderscore , cortar)}
\end{itemize}
Amputação das extremidades.
\section{Acrótomo}
\begin{itemize}
\item {Grp. gram.:adj.}
\end{itemize}
\begin{itemize}
\item {Utilização:Miner.}
\end{itemize}
\begin{itemize}
\item {Proveniência:(Gr. \textunderscore akrotomos\textunderscore )}
\end{itemize}
Que tem clivagem parallela á base.
\section{Acrotóphoro}
\begin{itemize}
\item {Grp. gram.:m.}
\end{itemize}
\begin{itemize}
\item {Proveniência:(Do gr. \textunderscore akroton\textunderscore  + \textunderscore phoros\textunderscore )}
\end{itemize}
Vaso, em que os Romanos tinham á mesa o vinho puro.
\section{Acta}
\begin{itemize}
\item {Grp. gram.:f.}
\end{itemize}
\begin{itemize}
\item {Proveniência:(Lat. \textunderscore acta\textunderscore )}
\end{itemize}
Registo de sessão de corporações.
\section{Acteáceas}
\begin{itemize}
\item {Grp. gram.:f. pl.}
\end{itemize}
\begin{itemize}
\item {Proveniência:(Do lat. \textunderscore acte\textunderscore )}
\end{itemize}
Familia de plantas, que tem por typo o ébulo ou erva-de-San-Christovam.
\section{Acteão}
\begin{itemize}
\item {Grp. gram.:m.}
\end{itemize}
Mollusco gasterópode.
Insecto lepidóptero nocturno.
\section{Actébia}
\begin{itemize}
\item {Grp. gram.:f.}
\end{itemize}
\begin{itemize}
\item {Proveniência:(Do gr. \textunderscore akte\textunderscore  + \textunderscore bio\textunderscore )}
\end{itemize}
Lepidóptero nocturno.
\section{Acíclico}
\begin{itemize}
\item {Grp. gram.:adj.}
\end{itemize}
\begin{itemize}
\item {Utilização:Bot.}
\end{itemize}
\begin{itemize}
\item {Proveniência:(Do gr. \textunderscore a\textunderscore  priv. + \textunderscore kuklos\textunderscore , circulo)}
\end{itemize}
Diz-se das flores em que, dispostas em espiral as partes appendiculares, o intervallo, que separa um grupo de appêndices do seguinte, não coincide com um número determinado de voltas da espira.
\section{Acirologia}
\begin{itemize}
\item {Grp. gram.:f.}
\end{itemize}
\begin{itemize}
\item {Proveniência:(Gr. \textunderscore akurologia\textunderscore )}
\end{itemize}
Impropriedade de expressão.
\section{Acirológico}
\begin{itemize}
\item {Grp. gram.:adj.}
\end{itemize}
Relativo á \textunderscore acirologia\textunderscore .
\section{Acisia}
\begin{itemize}
\item {Grp. gram.:f.}
\end{itemize}
\begin{itemize}
\item {Proveniência:(Do gr. \textunderscore a\textunderscore  priv. + \textunderscore kuein\textunderscore )}
\end{itemize}
Impotência; esterilidade.
\section{Acistia}
\begin{itemize}
\item {Grp. gram.:f.}
\end{itemize}
\begin{itemize}
\item {Proveniência:(Gr. \textunderscore a\textunderscore  + \textunderscore kustis\textunderscore )}
\end{itemize}
Ausência da bexiga urinária.
\section{Acistinervia}
\begin{itemize}
\item {Grp. gram.:f.}
\end{itemize}
\begin{itemize}
\item {Proveniência:(Do gr. \textunderscore a\textunderscore  + \textunderscore kustis\textunderscore  + \textunderscore neuron\textunderscore )}
\end{itemize}
Paralysia da bexiga.
\section{Acisturotrofia}
\begin{itemize}
\item {Grp. gram.:f.}
\end{itemize}
\begin{itemize}
\item {Proveniência:(Do gr. \textunderscore a\textunderscore  + \textunderscore kustis\textunderscore  + \textunderscore oura\textunderscore  + \textunderscore trophe\textunderscore )}
\end{itemize}
Atrophia da bexiga.
\section{Acitários}
\begin{itemize}
\item {Grp. gram.:m. pl.}
\end{itemize}
Grupo de animaes rhizóporos, cujo corpo é constituído por uma substância mucosa sem divisão de céllulas.
\section{Actéfila}
\begin{itemize}
\item {Grp. gram.:f.}
\end{itemize}
\begin{itemize}
\item {Proveniência:(Do gr. \textunderscore akte\textunderscore  + \textunderscore philein\textunderscore )}
\end{itemize}
Planta euphorbiácea, de flores monoicas.
\section{Acteia}
\begin{itemize}
\item {Grp. gram.:f.}
\end{itemize}
\begin{itemize}
\item {Proveniência:(Gr. \textunderscore aktaia\textunderscore )}
\end{itemize}
Planta venenosa, ranunculácea.
\section{Actenista}
\begin{itemize}
\item {Grp. gram.:f.}
\end{itemize}
\begin{itemize}
\item {Proveniência:(Gr. \textunderscore aktenistos\textunderscore )}
\end{itemize}
Insecto coleóptero pentâmere, do Brasil.
\section{Acteografia}
\begin{itemize}
\item {Grp. gram.:f.}
\end{itemize}
\begin{itemize}
\item {Proveniência:(Do gr. \textunderscore achthos\textunderscore  + \textunderscore graphein\textunderscore )}
\end{itemize}
Descripção ou nomenclatura dos pesos.
\section{Acteómetro}
\begin{itemize}
\item {Grp. gram.:m.}
\end{itemize}
\begin{itemize}
\item {Proveniência:(Do gr. \textunderscore akhtos\textunderscore  + \textunderscore metron\textunderscore )}
\end{itemize}
Instrumento, para medir o pêso dos carros sôbre as rodas.
\section{Actéphila}
\begin{itemize}
\item {Grp. gram.:f.}
\end{itemize}
\begin{itemize}
\item {Proveniência:(Do gr. \textunderscore akte\textunderscore  + \textunderscore philein\textunderscore )}
\end{itemize}
Planta euphorbiácea, de flores monoicas.
\section{Áctia}
\begin{itemize}
\item {Grp. gram.:f.}
\end{itemize}
\begin{itemize}
\item {Proveniência:(Gr. \textunderscore aktis\textunderscore )}
\end{itemize}
Insecto diptero.
\section{Actigeia}
\begin{itemize}
\item {Grp. gram.:f.}
\end{itemize}
\begin{itemize}
\item {Proveniência:(Do gr. \textunderscore aktis\textunderscore  + \textunderscore ge\textunderscore )}
\end{itemize}
Espécie de cogumelo.
\section{Actinanta}
\begin{itemize}
\item {Grp. gram.:f.}
\end{itemize}
\begin{itemize}
\item {Proveniência:(Do gr. \textunderscore aktis\textunderscore  + \textunderscore anthe\textunderscore )}
\end{itemize}
Planta umbellifera.
\section{Actinantha}
\begin{itemize}
\item {Grp. gram.:f.}
\end{itemize}
\begin{itemize}
\item {Proveniência:(Do gr. \textunderscore aktis\textunderscore  + \textunderscore anthe\textunderscore )}
\end{itemize}
Planta umbellifera.
\section{Actineia}
\begin{itemize}
\item {Grp. gram.:f.}
\end{itemize}
\begin{itemize}
\item {Proveniência:(Do rad. do gr. \textunderscore aktis\textunderscore )}
\end{itemize}
Planta, da fam. das synanthéreas.
\section{Actinéria}
\begin{itemize}
\item {Grp. gram.:f.}
\end{itemize}
\begin{itemize}
\item {Proveniência:(Do rad. do gr. \textunderscore aktis\textunderscore )}
\end{itemize}
Pólypo, da fam. dos actinianos.
\section{Actínia}
\begin{itemize}
\item {Grp. gram.:f.}
\end{itemize}
\begin{itemize}
\item {Proveniência:(Do gr. \textunderscore aktis\textunderscore )}
\end{itemize}
Pólypo.
Nome scientífico do pólypo, chamado vulgarmente \textunderscore anêmona-do-mar\textunderscore .
\section{Actinianos}
\begin{itemize}
\item {Grp. gram.:m. pl.}
\end{itemize}
\begin{itemize}
\item {Proveniência:(De \textunderscore actínia\textunderscore )}
\end{itemize}
Gênero de pólypos coraliários.
\section{Actínico}
\begin{itemize}
\item {Grp. gram.:adj.}
\end{itemize}
Diz-se da luz e dos raios luminosos, que exercem acção chímica em certas substâncias.
(Cp. \textunderscore actinismo\textunderscore )
\section{Actinimorfo}
\begin{itemize}
\item {Grp. gram.:adj.}
\end{itemize}
\begin{itemize}
\item {Proveniência:(Do gr. \textunderscore aktin\textunderscore  + \textunderscore morphe\textunderscore )}
\end{itemize}
Que tem fórma radiada.
\section{Actinimorpho}
\begin{itemize}
\item {Grp. gram.:adj.}
\end{itemize}
\begin{itemize}
\item {Proveniência:(Do gr. \textunderscore aktin\textunderscore  + \textunderscore morphe\textunderscore )}
\end{itemize}
Que tem fórma radiada.
\section{Actínio}
\begin{itemize}
\item {Grp. gram.:m.}
\end{itemize}
Outro elemento do pechurano, descoberto em 1890, mas pouco conhecido por ora.
\section{Actinismo}
\begin{itemize}
\item {Grp. gram.:m.}
\end{itemize}
\begin{itemize}
\item {Proveniência:(Do gr. \textunderscore aktin\textunderscore )}
\end{itemize}
Influência dos raios do sol em certas substâncias, resultando daí uma acção chímica, como succede na photographía.
\section{Actino}
\begin{itemize}
\item {Grp. gram.:m.}
\end{itemize}
\begin{itemize}
\item {Proveniência:(Do gr. \textunderscore aktin\textunderscore )}
\end{itemize}
Gênero de insectos dípteros.
\section{Actinobolismo}
\begin{itemize}
\item {Grp. gram.:m.}
\end{itemize}
Designação antiga do phenómeno physiológico, que hôje se chama \textunderscore hypnotismo\textunderscore .
\section{Actinóbolo}
\begin{itemize}
\item {Grp. gram.:m.}
\end{itemize}
Gênero de protozoários.
\section{Actinocarpo}
\begin{itemize}
\item {Grp. gram.:m.}
\end{itemize}
\begin{itemize}
\item {Proveniência:(Do gr. \textunderscore aktin\textunderscore  + \textunderscore karpos\textunderscore )}
\end{itemize}
Gênero de plantas alismácias.
\section{Actinocéfalo}
\begin{itemize}
\item {Grp. gram.:m.}
\end{itemize}
Gênero de algas.
\section{Actinocéphalo}
\begin{itemize}
\item {Grp. gram.:m.}
\end{itemize}
Gênero de algas.
\section{Actinodafno}
\begin{itemize}
\item {Grp. gram.:m.}
\end{itemize}
Gênero de plantas lauríneas.
\section{Actinodaphno}
\begin{itemize}
\item {Grp. gram.:m.}
\end{itemize}
Gênero de plantas lauríneas.
\section{Actinodendro}
\begin{itemize}
\item {Grp. gram.:m.}
\end{itemize}
Gênero de celenterados.
\section{Actinodermo}
\begin{itemize}
\item {Grp. gram.:m.}
\end{itemize}
\begin{itemize}
\item {Proveniência:(Do gr. \textunderscore aktin\textunderscore  + \textunderscore derma\textunderscore )}
\end{itemize}
Espécie de cogumelo.
\section{Actínodo}
\begin{itemize}
\item {Grp. gram.:m.}
\end{itemize}
Gênero de plantas myrtáceas.
\section{Actinóforo}
\begin{itemize}
\item {Grp. gram.:m.}
\end{itemize}
\begin{itemize}
\item {Proveniência:(Do gr. \textunderscore aktin\textunderscore  + \textunderscore phoros\textunderscore )}
\end{itemize}
Gênero de insectos coleópteros pentámeros.
\section{Actinogênico}
\begin{itemize}
\item {Grp. gram.:adj.}
\end{itemize}
\begin{itemize}
\item {Proveniência:(Do gr. \textunderscore aktin\textunderscore  + \textunderscore geneia\textunderscore )}
\end{itemize}
Que produz irradiação eléctrica, (falando-se dos tubos do apparelho radiográphico). Cf. Vergílio Machado, \textunderscore Raios X\textunderscore .
\section{Actinoide}
\begin{itemize}
\item {Grp. gram.:adj.}
\end{itemize}
Semelhante á actínia.
\section{Actinologia}
\begin{itemize}
\item {Grp. gram.:f.}
\end{itemize}
\begin{itemize}
\item {Proveniência:(Do gr. \textunderscore aktin\textunderscore  + \textunderscore logos\textunderscore )}
\end{itemize}
Descripção dos animaes radiados.
\section{Actinológico}
\begin{itemize}
\item {Grp. gram.:adj.}
\end{itemize}
Relativo á \textunderscore actinologia\textunderscore .
\section{Actinomancía}
\begin{itemize}
\item {Grp. gram.:f.}
\end{itemize}
\begin{itemize}
\item {Proveniência:(Do gr. \textunderscore aktin\textunderscore  + \textunderscore manteia\textunderscore )}
\end{itemize}
Arte de adivinhar por meio das estrêllas.
\section{Actinometria}
\begin{itemize}
\item {Grp. gram.:f.}
\end{itemize}
\begin{itemize}
\item {Proveniência:(De \textunderscore actinómetro\textunderscore )}
\end{itemize}
Avaliação do grau da irradiação solar.
\section{Actinómetro}
\begin{itemize}
\item {Grp. gram.:m.}
\end{itemize}
\begin{itemize}
\item {Proveniência:(Do gr. \textunderscore aktin\textunderscore  + \textunderscore metron\textunderscore )}
\end{itemize}
Apparelho, para medir a intensidade da luz pela da electricidade desenvolvida.
\section{Actinomiceto}
\begin{itemize}
\item {Grp. gram.:m.}
\end{itemize}
\begin{itemize}
\item {Proveniência:(Do gr. \textunderscore aktin\textunderscore , raio, + \textunderscore mukes\textunderscore , \textunderscore muketos\textunderscore , cogumelo)}
\end{itemize}
Cogumelo parasito, que produz a actinomycose.
\section{Actinomicose}
\begin{itemize}
\item {Grp. gram.:f.}
\end{itemize}
\begin{itemize}
\item {Utilização:Med.}
\end{itemize}
Moléstia parasitária, produzida por um cogumelo especial, e cujo principal symptoma são alterações nas gengivas.
\section{Actinomyceto}
\begin{itemize}
\item {Grp. gram.:m.}
\end{itemize}
\begin{itemize}
\item {Proveniência:(Do gr. \textunderscore aktin\textunderscore , raio, + \textunderscore mukes\textunderscore , \textunderscore muketos\textunderscore , cogumelo)}
\end{itemize}
Cogumelo parasito, que produz a actinomycose.
\section{Actinomycose}
\begin{itemize}
\item {Grp. gram.:f.}
\end{itemize}
\begin{itemize}
\item {Utilização:Med.}
\end{itemize}
Moléstia parasitária, produzida por um cogumelo especial, e cujo principal symptoma são alterações nas gengivas.
\section{Actinóphoro}
\begin{itemize}
\item {Grp. gram.:m.}
\end{itemize}
\begin{itemize}
\item {Proveniência:(Do gr. \textunderscore aktin\textunderscore  + \textunderscore phoros\textunderscore )}
\end{itemize}
Gênero de insectos coleópteros pentámeros.
\section{Actinospira}
\begin{itemize}
\item {Grp. gram.:f.}
\end{itemize}
Gênero de cogumelos, segundo Corda.
\section{Actinóstemo}
\begin{itemize}
\item {Grp. gram.:m.}
\end{itemize}
Gênero de plantas euphorbiáceas.
\section{Actinóstomo}
\begin{itemize}
\item {Grp. gram.:adj.}
\end{itemize}
\begin{itemize}
\item {Proveniência:(Do gr. \textunderscore aktin\textunderscore  + \textunderscore stoma\textunderscore )}
\end{itemize}
Diz-se dos animaes, cuja boca é rodeada de raios com tentáculos.
\section{Activa}
\begin{itemize}
\item {Grp. gram.:f.}
\end{itemize}
\begin{itemize}
\item {Proveniência:(De \textunderscore activo\textunderscore )}
\end{itemize}
Voz activa dos verbos.
Parte principal, exercida na realização de um acto.
\section{Activamente}
\begin{itemize}
\item {Grp. gram.:adv.}
\end{itemize}
De modo \textunderscore activo\textunderscore .
\section{Activar}
\begin{itemize}
\item {Grp. gram.:v. t.}
\end{itemize}
Dar actividade a.
Tornar activo.
Dar impulso a.
\section{Actividade}
\begin{itemize}
\item {Grp. gram.:f.}
\end{itemize}
\begin{itemize}
\item {Proveniência:(Lat. \textunderscore activitas\textunderscore )}
\end{itemize}
Diligência.
Faculdade de operar.
\section{Activo}
\begin{itemize}
\item {Grp. gram.:adj.}
\end{itemize}
\begin{itemize}
\item {Utilização:Gram.}
\end{itemize}
\begin{itemize}
\item {Proveniência:(Lat. \textunderscore activus\textunderscore )}
\end{itemize}
Diligente, expedito.
Que tem acção, que actua.
Que opéra com energia.
Ininterrupto.
Que denota a acção de um verbo.
\section{Acto}
\begin{itemize}
\item {Grp. gram.:m.}
\end{itemize}
\begin{itemize}
\item {Proveniência:(Lat. \textunderscore actus\textunderscore )}
\end{itemize}
O que se fez ou se póde fazer.
Acção.
Exame final em cada anno dos cursos de Universidade.
Divisão de obra dramática.
\section{Actol}
\begin{itemize}
\item {Grp. gram.:m.}
\end{itemize}
Medicamento antiséptico de lactato de prata.
\section{Actor}
\begin{itemize}
\item {Grp. gram.:m.}
\end{itemize}
\begin{itemize}
\item {Proveniência:(Lat. \textunderscore actor\textunderscore )}
\end{itemize}
Agente; o que pratica o acto.
Aquelle que representa em theatro.
\section{Actriz}
\begin{itemize}
\item {Grp. gram.:f.}
\end{itemize}
Mulher, que representa em theatro.
(Fem. de \textunderscore actor\textunderscore )
\section{Actuação}
\begin{itemize}
\item {Grp. gram.:f.}
\end{itemize}
Acção de \textunderscore actuar\textunderscore .
\section{Actual}
\begin{itemize}
\item {Grp. gram.:adj.}
\end{itemize}
\begin{itemize}
\item {Proveniência:(Lat. \textunderscore actualis\textunderscore )}
\end{itemize}
Effectivo.
Que existe presentemente.
\section{Actualidade}
\begin{itemize}
\item {Grp. gram.:f.}
\end{itemize}
\begin{itemize}
\item {Proveniência:(De \textunderscore actual\textunderscore )}
\end{itemize}
Effectividade.
Occasião presente.
\section{Actualizar}
\begin{itemize}
\item {Grp. gram.:v. t.}
\end{itemize}
Tornar actual; modernizar.
\section{Actualmente}
\begin{itemize}
\item {Grp. gram.:adv.}
\end{itemize}
Nesta occasião.
De modo \textunderscore actual\textunderscore .
\section{Actuante}
\begin{itemize}
\item {Grp. gram.:adj.}
\end{itemize}
Que actua.
\section{Actuar}
\begin{itemize}
\item {Grp. gram.:v. i.}
\end{itemize}
\begin{itemize}
\item {Proveniência:(De \textunderscore acto\textunderscore )}
\end{itemize}
Exercer acção.
\section{Actuário}
\begin{itemize}
\item {Grp. gram.:m.}
\end{itemize}
\begin{itemize}
\item {Proveniência:(Lat. \textunderscore actuarius\textunderscore )}
\end{itemize}
Escriba, encarregado de redigir os discursos pronunciados no senado romano e noutras reuniões públicas.
Secretário.
Modernamente, dá-se este nome ao indivíduo, encarregado por Companhia de Seguros de Vida de estabelecer as bases das suas operações e verificar-lhes os resultados, devendo reunir as qualidades de estatistico, financeiro e mathemático.
\section{Actuável}
\begin{itemize}
\item {Grp. gram.:adj.}
\end{itemize}
\begin{itemize}
\item {Proveniência:(De \textunderscore actuar\textunderscore )}
\end{itemize}
Dócil.
Dirigivel.
\section{Actuosamente}
\begin{itemize}
\item {Grp. gram.:adv.}
\end{itemize}
Com actividade.
De modo \textunderscore actuoso\textunderscore .
\section{Actuoso}
\begin{itemize}
\item {Grp. gram.:adj.}
\end{itemize}
\begin{itemize}
\item {Proveniência:(Lat. \textunderscore actuosus\textunderscore )}
\end{itemize}
Que opéra.
Activo, muito diligente.
\section{Açu}
\begin{itemize}
\item {Grp. gram.:m.}
\end{itemize}
\begin{itemize}
\item {Utilização:Bras}
\end{itemize}
Espécie de jacaré das margens do Amazonas.
\section{Acuação}
\begin{itemize}
\item {Grp. gram.:f.}
\end{itemize}
\begin{itemize}
\item {Utilização:Bras}
\end{itemize}
Acto de \textunderscore acuar\textunderscore .
Acto de perseguir a caça até a obrigar a meter-se na toca.
Acto de perseguir o inimigo até metê-lo em reducto, donde não possa sair.
\section{Acuador}
\begin{itemize}
\item {Grp. gram.:adj.}
\end{itemize}
\begin{itemize}
\item {Utilização:Bras. de Minas}
\end{itemize}
\begin{itemize}
\item {Proveniência:(De \textunderscore acuar\textunderscore )}
\end{itemize}
Diz-se da cavalgadura que recusa andar.
\section{Acuamênto}
\begin{itemize}
\item {Grp. gram.:m.}
\end{itemize}
Acto de \textunderscore acuar\textunderscore .
\section{Acuar}
\begin{itemize}
\item {Grp. gram.:v. i.}
\end{itemize}
\begin{itemize}
\item {Grp. gram.:V. p.}
\end{itemize}
\begin{itemize}
\item {Grp. gram.:V. t.}
\end{itemize}
\begin{itemize}
\item {Utilização:Bras}
\end{itemize}
\begin{itemize}
\item {Proveniência:(De \textunderscore cu\textunderscore )}
\end{itemize}
Abaixar as nádegas, para fazer salto.
Recuar, retroceder.
(a mesma sign.)
Perseguir (caça ou inimigo).
\section{Acubito}
\begin{itemize}
\item {Grp. gram.:m.}
\end{itemize}
\begin{itemize}
\item {Proveniência:(Lat. \textunderscore accubilum\textunderscore )}
\end{itemize}
Cadeira-leito, espécie de canapé, em que os Romanos se sentavam á mesa.
\section{Açúcar}
\begin{itemize}
\item {Grp. gram.:m.}
\end{itemize}
\begin{itemize}
\item {Utilização:Fig.}
\end{itemize}
\begin{itemize}
\item {Proveniência:(Do ár. \textunderscore çucar\textunderscore )}
\end{itemize}
Substância doce de certos vegetaes e de algumas secreções animaes.
Corpo, que, dissolvido em água e em contacto com o fermento, se converte em álcool e ácido carbónico.
Suavidade.
Prazer.
Lisonja.
\section{Açucaradamente}
\begin{itemize}
\item {Grp. gram.:adv.}
\end{itemize}
De modo \textunderscore açucarado\textunderscore .
Com modos mellífluos.
\section{Açucarado}
\begin{itemize}
\item {Grp. gram.:adj.}
\end{itemize}
\begin{itemize}
\item {Utilização:Fig.}
\end{itemize}
Que foi adoçado com açucar.
Lisonjeiro.
Mellifluo: \textunderscore palavras açucaradas\textunderscore .
\section{Açucarar}
\begin{itemize}
\item {Grp. gram.:v. t.}
\end{itemize}
\begin{itemize}
\item {Utilização:Fig.}
\end{itemize}
Temperar, cobrir, ou misturar com açúcar.
Adoçar.
Tornar agradável, suave, meigo.
\section{Açucareiro}
\begin{itemize}
\item {Grp. gram.:adj.}
\end{itemize}
\begin{itemize}
\item {Grp. gram.:M.}
\end{itemize}
Relativo a açúcar: \textunderscore indústria açucareira\textunderscore .
Que tem ou produz açúcar.
Pequeno vaso, em que se serve o açúcar.
Fabricante de açúcar.
\section{Açucena}
\begin{itemize}
\item {Grp. gram.:f.}
\end{itemize}
\begin{itemize}
\item {Utilização:Bras}
\end{itemize}
\begin{itemize}
\item {Proveniência:(Do ár. \textunderscore çucena\textunderscore )}
\end{itemize}
Espécie de lírio branco.
O mesmo que \textunderscore arandela\textunderscore .
\section{Açucenal}
\begin{itemize}
\item {Grp. gram.:m.}
\end{itemize}
Lugar, onde crescem açucenas.
\section{Açucre}
\begin{itemize}
\item {Grp. gram.:m.}
\end{itemize}
Fórma popular de \textunderscore açúcar\textunderscore .
\section{Açuçuapara}
\begin{itemize}
\item {Grp. gram.:f.}
\end{itemize}
\begin{itemize}
\item {Utilização:Bras}
\end{itemize}
Espécie de veado grande? Cf. J. J. Lisboa, \textunderscore Descripção Curiosa\textunderscore .
\section{Acucular}
\begin{itemize}
\item {Grp. gram.:v. t.}
\end{itemize}
(V.acogular)
\section{Açudada}
\begin{itemize}
\item {Grp. gram.:f.}
\end{itemize}
Porção de água, represada por açude.
\section{Açude}
\begin{itemize}
\item {Grp. gram.:m.}
\end{itemize}
\begin{itemize}
\item {Utilização:Prov.}
\end{itemize}
\begin{itemize}
\item {Utilização:beir.}
\end{itemize}
Construcção de pedra ou madeira, para represar águas de rio ou levada, destinadas a rega ou moagem.
Batoréu, arrêto.
(Ár. \textunderscore azud\textunderscore )
\section{Acudidor}
\begin{itemize}
\item {Grp. gram.:adj.}
\end{itemize}
Que acode. Cf. Filinto, XX, 208.
\section{Acudimento}
\begin{itemize}
\item {Grp. gram.:m.}
\end{itemize}
Acto de \textunderscore acudir\textunderscore . Cf. Filinto, \textunderscore D. Man.\textunderscore , 100.
\section{Acudir}
\begin{itemize}
\item {Grp. gram.:v. i.}
\end{itemize}
Ir em soccorro.
Concorrer.
Retorquir.
\section{Acuera}
\begin{itemize}
\item {Grp. gram.:m. ,  f.  e  adj.}
\end{itemize}
\begin{itemize}
\item {Utilização:Bras. do N}
\end{itemize}
Diz-se de coisas antigas, abandonadas ou extintas.
(Do tupi)
\section{Açugar}
\begin{itemize}
\item {Grp. gram.:v. t.}
\end{itemize}
\begin{itemize}
\item {Utilização:T. da Bairrada}
\end{itemize}
O mesmo que \textunderscore açular\textunderscore .
\section{Acuidade}
\begin{itemize}
\item {fónica:cu-i}
\end{itemize}
\begin{itemize}
\item {Grp. gram.:f.}
\end{itemize}
\begin{itemize}
\item {Proveniência:(Do rad. do lat. \textunderscore acus\textunderscore )}
\end{itemize}
Qualidade do que é agudo.
Penetração, perspicácia.
\section{Acuitar}
\begin{itemize}
\item {Grp. gram.:v. t.}
\end{itemize}
\begin{itemize}
\item {Utilização:Ant.}
\end{itemize}
O mesmo que \textunderscore coitar\textunderscore ^2.
\section{Acujar}
\begin{itemize}
\item {Grp. gram.:v. i.}
\end{itemize}
\begin{itemize}
\item {Utilização:Ant.}
\end{itemize}
\begin{itemize}
\item {Proveniência:(De \textunderscore cujo\textunderscore )}
\end{itemize}
Preguntar com insistência.
\section{Açulador}
\begin{itemize}
\item {Grp. gram.:m.}
\end{itemize}
O que açula.
\section{Açulamento}
\begin{itemize}
\item {Grp. gram.:m.}
\end{itemize}
Acto de \textunderscore açular\textunderscore .
\section{Açular}
\begin{itemize}
\item {Grp. gram.:v. t.}
\end{itemize}
\begin{itemize}
\item {Proveniência:(Do ár. \textunderscore çaula\textunderscore )}
\end{itemize}
Incitar (cães) a morder.
Provocar.
\section{Aculear}
\begin{itemize}
\item {Grp. gram.:v. t.}
\end{itemize}
\begin{itemize}
\item {Proveniência:(De \textunderscore acúleo\textunderscore )}
\end{itemize}
Armar de aguilhão.
\section{Aculeiforme}
\begin{itemize}
\item {Grp. gram.:adj.}
\end{itemize}
\begin{itemize}
\item {Proveniência:(Do lat. \textunderscore aculeus\textunderscore  + \textunderscore forma\textunderscore )}
\end{itemize}
Que tem fórma de acúleo.
\section{Acúleo}
\begin{itemize}
\item {Grp. gram.:m.}
\end{itemize}
\begin{itemize}
\item {Proveniência:(Lat. \textunderscore aculeus\textunderscore )}
\end{itemize}
Aguilhão.
Espinho de certas plantas, como as silvas, roseiras, etc.
\section{Acumbente}
\begin{itemize}
\item {Grp. gram.:adj.}
\end{itemize}
\begin{itemize}
\item {Proveniência:(Lat. \textunderscore accumbens\textunderscore )}
\end{itemize}
Diz-se da radicula das plantas cruciferas, quando curvada na borda dos cotylédones.
\section{Acume}
\begin{itemize}
\item {Grp. gram.:m.}
\end{itemize}
\begin{itemize}
\item {Proveniência:(Lat. \textunderscore acumen\textunderscore )}
\end{itemize}
O mesmo que \textunderscore cume\textunderscore .
\section{Acúmetro}
\begin{itemize}
\item {Grp. gram.:m.}
\end{itemize}
\begin{itemize}
\item {Proveniência:(Do gr. \textunderscore akouen\textunderscore  + \textunderscore metron\textunderscore )}
\end{itemize}
Instrumento, para medir a qualidade auditiva do homem.
\section{Acuminação}
\begin{itemize}
\item {Grp. gram.:f.}
\end{itemize}
\begin{itemize}
\item {Utilização:Geom.}
\end{itemize}
Angulo sólido. Cf. Eug. Pacheco, \textunderscore Phýs. Posit.\textunderscore , 95.
\section{Acuminar}
\begin{itemize}
\item {Grp. gram.:v. t.}
\end{itemize}
\begin{itemize}
\item {Proveniência:(Lat. \textunderscore acuminare\textunderscore )}
\end{itemize}
Aguçar.
\section{Acumulação}
\begin{itemize}
\item {Grp. gram.:f.}
\end{itemize}
\begin{itemize}
\item {Proveniência:(Lat. \textunderscore accumulatio\textunderscore )}
\end{itemize}
Acto ou effeito de \textunderscore acumular\textunderscore 
\section{Acumuladamente}
\begin{itemize}
\item {Grp. gram.:adv.}
\end{itemize}
Com acumulação.
\section{Acumulador}
\begin{itemize}
\item {Grp. gram.:m.  e  adj.}
\end{itemize}
\begin{itemize}
\item {Utilização:Phýs.}
\end{itemize}
O que acumula.
Máquina, que armazena a fôrça, para a restituir, quando necessária.
Máquina, que armazena o potencial eléctrico.
\section{Acumulamento}
\begin{itemize}
\item {Grp. gram.:m.}
\end{itemize}
(V. \textunderscore acumulação\textunderscore .)
\section{Acumular}
\begin{itemize}
\item {Grp. gram.:v. t.}
\end{itemize}
\begin{itemize}
\item {Grp. gram.:V. p.}
\end{itemize}
\begin{itemize}
\item {Utilização:Fig.}
\end{itemize}
\begin{itemize}
\item {Proveniência:(Lat. \textunderscore accumulare\textunderscore )}
\end{itemize}
Amontoar, pôr em cúmulos.
Reunir em si (várias funcções ou occupações).
Succeder-se, sobrevir.
\section{Acumulativamente}
\begin{itemize}
\item {Grp. gram.:adv.}
\end{itemize}
De modo \textunderscore acumulativo\textunderscore ; conjuntamente.
\section{Acumulativo}
\begin{itemize}
\item {Grp. gram.:adj.}
\end{itemize}
Que se póde \textunderscore acumular\textunderscore .
\section{Acúmulo}
\begin{itemize}
\item {Grp. gram.:m.}
\end{itemize}
\begin{itemize}
\item {Utilização:Neol.}
\end{itemize}
O mesmo que \textunderscore acumulação\textunderscore .
\section{Acunhar}
\begin{itemize}
\item {Grp. gram.:v. t.}
\end{itemize}
\begin{itemize}
\item {Utilização:Náut.}
\end{itemize}
\begin{itemize}
\item {Utilização:Fam.}
\end{itemize}
Apertar com cunha.
Pôr os mastaréus á cunha, isto é, içá-los por ante-avante dos mastros, até ficarem no seu lugar.
Proteger com empenhos.
\section{Acunhear}
\begin{itemize}
\item {Grp. gram.:v. t.}
\end{itemize}
Dar fórma de cunha a.
\section{Acupalpo}
\begin{itemize}
\item {Grp. gram.:m.}
\end{itemize}
\begin{itemize}
\item {Proveniência:(Do lat. \textunderscore acus\textunderscore  + \textunderscore palpum\textunderscore )}
\end{itemize}
Coleóptero pardo.
\section{Acupressura}
\begin{itemize}
\item {Grp. gram.:f.}
\end{itemize}
\begin{itemize}
\item {Proveniência:(Do lat. \textunderscore acus\textunderscore  + \textunderscore pressura\textunderscore )}
\end{itemize}
Operação cirúrgica, para suspender a hemorragia arterial ou venosa.
\section{Acupunctura}
\begin{itemize}
\item {Grp. gram.:f.}
\end{itemize}
\begin{itemize}
\item {Proveniência:(Lat. \textunderscore acupunctura\textunderscore )}
\end{itemize}
Operação cirúrgica, em que uma agulha metállica é introduzida no corpo humano.
\section{Acuradamente}
\begin{itemize}
\item {Grp. gram.:adv.}
\end{itemize}
Cuidadosamente.
\section{Acurar}
\begin{itemize}
\item {Grp. gram.:v. t.}
\end{itemize}
\begin{itemize}
\item {Proveniência:(Lat. \textunderscore accurare\textunderscore )}
\end{itemize}
Tratar com cuidado, com desvelo.
\section{Acursar}
\begin{itemize}
\item {Grp. gram.:v.}
\end{itemize}
\begin{itemize}
\item {Utilização:t. Ven.}
\end{itemize}
\begin{itemize}
\item {Proveniência:(De \textunderscore curso\textunderscore )}
\end{itemize}
Alcançar (certo espaço) uma espingarda com o tiro: \textunderscore a minha caçadeira acursa oitenta passos\textunderscore .
\section{Acurtar}
\begin{itemize}
\item {Grp. gram.:v. t.}
\end{itemize}
O mesmo que \textunderscore encurtar\textunderscore .
\section{Acurvar}
\begin{itemize}
\item {Grp. gram.:v. t.}
\end{itemize}
O mesmo que \textunderscore encurvar\textunderscore  e \textunderscore curvar\textunderscore .
\section{Acurvilhar}
\begin{itemize}
\item {Grp. gram.:v. t.}
\end{itemize}
(freq. de \textunderscore acurvar\textunderscore )
\section{Acusação}
\begin{itemize}
\item {Grp. gram.:f.}
\end{itemize}
\begin{itemize}
\item {Proveniência:(Lat. \textunderscore Accusatio\textunderscore )}
\end{itemize}
Acto ou effeito de acusar.
\section{Acusado}
\begin{itemize}
\item {Grp. gram.:m.}
\end{itemize}
\begin{itemize}
\item {Proveniência:(De \textunderscore accusar\textunderscore )}
\end{itemize}
Aquelle a quem se imputa delito ou crime.
\section{Acusador}
\begin{itemize}
\item {Grp. gram.:m.  e  adj.}
\end{itemize}
\begin{itemize}
\item {Proveniência:(Lat. \textunderscore accusator\textunderscore )}
\end{itemize}
O que acusa.
\section{Acusamento}
\begin{itemize}
\item {Grp. gram.:m.}
\end{itemize}
O mesmo que \textunderscore acusação\textunderscore .
\section{Acusante}
\begin{itemize}
\item {Grp. gram.:m.}
\end{itemize}
Aquelle que acusa.
\section{Acusar}
\begin{itemize}
\item {Grp. gram.:v. t.}
\end{itemize}
\begin{itemize}
\item {Proveniência:(Lat. \textunderscore accusare\textunderscore )}
\end{itemize}
Imputar falta ou crime a.
Notificar: \textunderscore acusar a recepção de uma carta\textunderscore .
Mostrar.
Confessar: \textunderscore acusar os próprios defeitos\textunderscore .
\section{Acusativo}
\begin{itemize}
\item {Grp. gram.:adj.}
\end{itemize}
\begin{itemize}
\item {Grp. gram.:M.}
\end{itemize}
\begin{itemize}
\item {Utilização:Gram.}
\end{itemize}
\begin{itemize}
\item {Proveniência:(Lat. \textunderscore accusativus\textunderscore )}
\end{itemize}
Que serve para acusar.
Caso que, na declinação dos nomes latinos e gregos, designa principalmente o regime directo.
\section{Acusatoriamente}
\begin{itemize}
\item {Grp. gram.:adv.}
\end{itemize}
Do modo \textunderscore acusatório\textunderscore .
\section{Acusatório}
\begin{itemize}
\item {Grp. gram.:adj.}
\end{itemize}
\begin{itemize}
\item {Proveniência:(Lat. \textunderscore accusatorius\textunderscore )}
\end{itemize}
Relativo á acusação.
\section{Acusável}
\begin{itemize}
\item {Grp. gram.:adj.}
\end{itemize}
\begin{itemize}
\item {Proveniência:(Lat. \textunderscore accusabilis\textunderscore )}
\end{itemize}
Que póde ou deve sêr acusado.
\section{Acuse}
\begin{itemize}
\item {Grp. gram.:m.}
\end{itemize}
(V.acuso)
\section{Acusma}
\begin{itemize}
\item {Grp. gram.:f.}
\end{itemize}
O mesmo que \textunderscore acúsmata\textunderscore .
\section{Acúsmata}
\begin{itemize}
\item {Grp. gram.:m.}
\end{itemize}
\begin{itemize}
\item {Proveniência:(Gr. \textunderscore akousma\textunderscore )}
\end{itemize}
Ruído imaginário.
Ruído, de que se não vê ou de que se não conhece a causa.
\section{Acusmático}
\begin{itemize}
\item {Grp. gram.:adj.}
\end{itemize}
Relativo a \textunderscore acusma\textunderscore .
\section{Acúsmato}
\begin{itemize}
\item {Grp. gram.:m.}
\end{itemize}
O mesmo que \textunderscore acúsmata\textunderscore .
\section{Acuso}
\begin{itemize}
\item {Grp. gram.:m.}
\end{itemize}
O mesmo que \textunderscore acusação\textunderscore .
Designa a declaração que, no jôgo da bisca, o parceiro faz, de têr reunido, entre as suas cartas, duas figuras do mesmo valor.
Também há \textunderscore acuso\textunderscore  no jôgo dos Três-setes.
\section{Acústica}
\begin{itemize}
\item {Grp. gram.:f.}
\end{itemize}
\begin{itemize}
\item {Proveniência:(De \textunderscore acústico\textunderscore )}
\end{itemize}
Parte da Phýsica, que se occupa das leis dos sons.
\section{Acústico}
\begin{itemize}
\item {Grp. gram.:adj.}
\end{itemize}
\begin{itemize}
\item {Proveniência:(Gr. \textunderscore akoustikos\textunderscore )}
\end{itemize}
Relativo aos sons.
\section{Acuta}
\begin{itemize}
\item {Grp. gram.:f.}
\end{itemize}
\begin{itemize}
\item {Proveniência:(Lat. \textunderscore acuta\textunderscore )}
\end{itemize}
Esquadria.
Instrumento para medir ângulos.
\section{Acutangulado}
\begin{itemize}
\item {Grp. gram.:adj.}
\end{itemize}
\begin{itemize}
\item {Proveniência:(De \textunderscore acutângulo\textunderscore )}
\end{itemize}
Que tem ângulos agudos.
\section{Acutangular}
\begin{itemize}
\item {Grp. gram.:adj.}
\end{itemize}
\begin{itemize}
\item {Proveniência:(De \textunderscore acutângulo\textunderscore )}
\end{itemize}
Que fórma ângulo agudo.
\section{Acutângulo}
\begin{itemize}
\item {Grp. gram.:adj.}
\end{itemize}
\begin{itemize}
\item {Proveniência:(Do lat. \textunderscore acutus\textunderscore  + \textunderscore angulus\textunderscore )}
\end{itemize}
Que tem ângulos agudos.
\section{Acutelado}
\begin{itemize}
\item {Grp. gram.:adj.}
\end{itemize}
Que tem fórma de cutelo.
\section{Acutenáculo}
\begin{itemize}
\item {Grp. gram.:m.}
\end{itemize}
\begin{itemize}
\item {Proveniência:(Do lat. \textunderscore acus\textunderscore  + \textunderscore tenaculum\textunderscore )}
\end{itemize}
Instrumento cirúrgico, para segurar as agulhas, quando se fazem suturas onde as mãos não podem funccionar.
\section{Acuticórneo}
\begin{itemize}
\item {Grp. gram.:adj.}
\end{itemize}
\begin{itemize}
\item {Proveniência:(Do lat. \textunderscore acutuscornu\textunderscore )}
\end{itemize}
Diz-se dos animaes, que têm as antennas terminadas em ponta.
\section{Acutifólio}
\begin{itemize}
\item {Grp. gram.:adj.}
\end{itemize}
\begin{itemize}
\item {Proveniência:(Do lat. \textunderscore acutus\textunderscore  + \textunderscore folium\textunderscore )}
\end{itemize}
Diz-se das plantas, que têm fôlhas acuminadas.
\section{Acutiladiço}
\begin{itemize}
\item {Grp. gram.:adj.}
\end{itemize}
\begin{itemize}
\item {Grp. gram.:M.}
\end{itemize}
\begin{itemize}
\item {Utilização:Ant.}
\end{itemize}
Acutilado com frequência:«\textunderscore contra o acutiladiço Don Lourenço da Chamusca\textunderscore ». Camillo, \textunderscore M. da Fonte\textunderscore , 178; \textunderscore Idem\textunderscore , \textunderscore Narcót.\textunderscore , I, 116.
O mesmo que acutilador.
\section{Acutilador}
\begin{itemize}
\item {Grp. gram.:m.}
\end{itemize}
Aquelle que acutila.
Indivíduo brigoso.
\section{Acutilamento}
\begin{itemize}
\item {Grp. gram.:m.}
\end{itemize}
Acto de \textunderscore acutilar\textunderscore .
\section{Acutilar}
\begin{itemize}
\item {Grp. gram.:v. t.}
\end{itemize}
Dar cutiladas em. Golpear.
(Corr. de \textunderscore acutelar\textunderscore , de \textunderscore cutelo\textunderscore )
\section{Acutipuru}
\begin{itemize}
\item {Grp. gram.:m.}
\end{itemize}
\begin{itemize}
\item {Utilização:Bras. do N}
\end{itemize}
Macaco pequenino, de pelle felpuda, lustrosa e preta.
\section{Acutirostro}
\begin{itemize}
\item {fónica:rós}
\end{itemize}
\begin{itemize}
\item {Grp. gram.:adj.}
\end{itemize}
\begin{itemize}
\item {Proveniência:(Do lat. \textunderscore acutus\textunderscore  + \textunderscore rostrum\textunderscore )}
\end{itemize}
Diz-se dos animaes, cuja cabeça se prolonga em bico.
\section{Acutirrostro}
\begin{itemize}
\item {Grp. gram.:adj.}
\end{itemize}
\begin{itemize}
\item {Proveniência:(Do lat. \textunderscore acutus\textunderscore  + \textunderscore rostrum\textunderscore )}
\end{itemize}
Diz-se dos animaes, cuja cabeça se prolonga em bico.
\section{Acýclico}
\begin{itemize}
\item {Grp. gram.:adj.}
\end{itemize}
\begin{itemize}
\item {Utilização:Bot.}
\end{itemize}
\begin{itemize}
\item {Proveniência:(Do gr. \textunderscore a\textunderscore  priv. + \textunderscore kuklos\textunderscore , circulo)}
\end{itemize}
Diz-se das flores em que, dispostas em espiral as partes appendiculares, o intervallo, que separa um grupo de appêndices do seguinte, não coincide com um número determinado de voltas da espira.
\section{Acyrologia}
\begin{itemize}
\item {Grp. gram.:f.}
\end{itemize}
\begin{itemize}
\item {Proveniência:(Gr. \textunderscore akurologia\textunderscore )}
\end{itemize}
Impropriedade de expressão.
\section{Acyrológico}
\begin{itemize}
\item {Grp. gram.:adj.}
\end{itemize}
Relativo á \textunderscore acyrologia\textunderscore .
\section{Acysia}
\begin{itemize}
\item {Grp. gram.:f.}
\end{itemize}
\begin{itemize}
\item {Proveniência:(Do gr. \textunderscore a\textunderscore  priv. + \textunderscore kuein\textunderscore )}
\end{itemize}
Impotência; esterilidade.
\section{Acystia}
\begin{itemize}
\item {Grp. gram.:f.}
\end{itemize}
\begin{itemize}
\item {Proveniência:(Gr. \textunderscore a\textunderscore  + \textunderscore kustis\textunderscore )}
\end{itemize}
Ausência da bexiga urinária.
\section{Acystinervia}
\begin{itemize}
\item {Grp. gram.:f.}
\end{itemize}
\begin{itemize}
\item {Proveniência:(Do gr. \textunderscore a\textunderscore  + \textunderscore kustis\textunderscore  + \textunderscore neuron\textunderscore )}
\end{itemize}
Paralysia da bexiga.
\section{Acysturotrophia}
\begin{itemize}
\item {Grp. gram.:f.}
\end{itemize}
\begin{itemize}
\item {Proveniência:(Do gr. \textunderscore a\textunderscore  + \textunderscore kustis\textunderscore  + \textunderscore oura\textunderscore  + \textunderscore trophe\textunderscore )}
\end{itemize}
Atrophia da bexiga.
\section{Acyttários}
\begin{itemize}
\item {Grp. gram.:m. pl.}
\end{itemize}
Grupo de animaes rhizóporos, cujo corpo é constituído por uma substância mucosa sem divisão de céllulas.
\section{Ad...}
\begin{itemize}
\item {Grp. gram.:pref.}
\end{itemize}
(designativo de \textunderscore direcção\textunderscore , \textunderscore juncção\textunderscore , \textunderscore fim\textunderscore , \textunderscore movimento\textunderscore , etc.)
\section{...ada}
\begin{itemize}
\item {Grp. gram.:suf.}
\end{itemize}
(designativo de \textunderscore collectividade\textunderscore , \textunderscore impulso\textunderscore , \textunderscore acção\textunderscore , etc.)
\section{Adaca}
\begin{itemize}
\item {Grp. gram.:f.}
\end{itemize}
Planta medicinal do Malabar.
\section{Adactilia}
\begin{itemize}
\item {Grp. gram.:f.}
\end{itemize}
Qualidade ou estado de \textunderscore adáctilo\textunderscore .
\section{Adáctilo}
\begin{itemize}
\item {Grp. gram.:adj.}
\end{itemize}
\begin{itemize}
\item {Utilização:Zool.}
\end{itemize}
\begin{itemize}
\item {Proveniência:(Do gr. \textunderscore a\textunderscore  priv. + \textunderscore daktulos\textunderscore , dedo)}
\end{itemize}
Que não tem dedos.
\section{Adactylia}
\begin{itemize}
\item {Grp. gram.:f.}
\end{itemize}
Qualidade ou estado de \textunderscore adáctylo\textunderscore .
\section{Adáctylo}
\begin{itemize}
\item {Grp. gram.:adj.}
\end{itemize}
\begin{itemize}
\item {Utilização:Zool.}
\end{itemize}
\begin{itemize}
\item {Proveniência:(Do gr. \textunderscore a\textunderscore  priv. + \textunderscore daktulos\textunderscore , dedo)}
\end{itemize}
Que não tem dedos.
\section{Adafina}
\begin{itemize}
\item {Grp. gram.:f.}
\end{itemize}
Espécie de guisado, que os Judeus usavam em Espanha.
\section{Adaga}
\begin{itemize}
\item {Grp. gram.:f.}
\end{itemize}
Arma branca, larga e curta.
(B. lat. \textunderscore daga\textunderscore )
\section{Adagada}
\begin{itemize}
\item {Grp. gram.:f.}
\end{itemize}
Golpe de adaga.
\section{Adagial}
\begin{itemize}
\item {Grp. gram.:adj.}
\end{itemize}
Relativo a adágios.
\section{Adagiar}
\begin{itemize}
\item {Grp. gram.:v. i.}
\end{itemize}
Citar adágios.
\section{Adagiário}
\begin{itemize}
\item {Grp. gram.:m.}
\end{itemize}
Collecção de adágios.
\section{Adagieiro}
\begin{itemize}
\item {Grp. gram.:adj.}
\end{itemize}
Que sabe e emprega muitos adágios. Cf. \textunderscore Anat. Joc.\textunderscore , I, 353.
\section{Adágio}
\begin{itemize}
\item {Grp. gram.:m.}
\end{itemize}
\begin{itemize}
\item {Proveniência:(Lat. \textunderscore adagium\textunderscore )}
\end{itemize}
Provérbio, sentença moral.
Trecho musical, de andamento vagaroso.
\section{Adague}
\begin{itemize}
\item {Grp. gram.:m.}
\end{itemize}
\begin{itemize}
\item {Utilização:Prov.}
\end{itemize}
\begin{itemize}
\item {Utilização:beir.}
\end{itemize}
Pilha de madeira.
Camada de telha, que se põe a cozer no respectivo forno.
(Colhido no Fundão)
\section{Adagueiro}
\begin{itemize}
\item {Grp. gram.:m.}
\end{itemize}
\begin{itemize}
\item {Proveniência:(De \textunderscore adaga\textunderscore )}
\end{itemize}
Veado novo, cujas pontas são agudas como adagas.
\section{Adai}
\begin{itemize}
\item {Grp. gram.:f.}
\end{itemize}
Planta, de que os Abexins fazem palitos para os dentes, (\textunderscore salvadora persica\textunderscore , Lin.).
\section{Adail}
\begin{itemize}
\item {Grp. gram.:m.}
\end{itemize}
\begin{itemize}
\item {Proveniência:(Do ár. \textunderscore ad-dalil\textunderscore )}
\end{itemize}
Antigo cabo de guerra.
Vedeta.
Caudilho.
\section{Adalide}
\begin{itemize}
\item {Grp. gram.:m.}
\end{itemize}
Chefe militar, entre os antigos Moiros. Cf. Garrett, \textunderscore D. Branca\textunderscore , 72.
\section{Adamado}
\begin{itemize}
\item {Grp. gram.:adj.}
\end{itemize}
\begin{itemize}
\item {Utilização:Prov.}
\end{itemize}
\begin{itemize}
\item {Utilização:Prov.}
\end{itemize}
Vestido com affectação ou com o cuidado próprio de dama.
Diz-se do vinho que não é verde.
Diz-se do vinho licoroso, que não excede a fôrça alcoólica de 15%.
\section{Adamanes}
\begin{itemize}
\item {Grp. gram.:m. pl.}
\end{itemize}
Atabales, usados na Índia.
\section{Adamante}
\begin{itemize}
\item {Grp. gram.:m.}
\end{itemize}
\begin{itemize}
\item {Proveniência:(Lat. \textunderscore adamas\textunderscore , \textunderscore adamantis\textunderscore )}
\end{itemize}
Planta medicinal, crucífera.
\section{Adamantino}
\begin{itemize}
\item {Grp. gram.:adj.}
\end{itemize}
\begin{itemize}
\item {Proveniência:(Lat. \textunderscore adamantinus\textunderscore )}
\end{itemize}
Semelhante ao diamante; diamantino.
\section{Adamar-se}
\begin{itemize}
\item {Grp. gram.:v. p.}
\end{itemize}
Enfeitar-se com o cuidado próprio de dama.
Effeminar-se.
\section{Adamascar}
\begin{itemize}
\item {Grp. gram.:v. t.}
\end{itemize}
Dar lavor ou côr de damasco a.
\section{Adamásia}
\begin{itemize}
\item {Grp. gram.:f.}
\end{itemize}
Planta liliácea.
\section{Adâmeo}
\begin{itemize}
\item {Grp. gram.:adj.}
\end{itemize}
O mesmo que \textunderscore adâmico\textunderscore . Cf. Filinto, XVI, 292.
\section{Adâmia}
\begin{itemize}
\item {Grp. gram.:f.}
\end{itemize}
\begin{itemize}
\item {Proveniência:(De \textunderscore Adam\textunderscore , n. p.)}
\end{itemize}
Planta saxífraga.
\section{Adamiano}
\begin{itemize}
\item {Grp. gram.:adj.}
\end{itemize}
O mesmo que \textunderscore adâmico\textunderscore .
\section{Adâmico}
\begin{itemize}
\item {Grp. gram.:adj.}
\end{itemize}
\begin{itemize}
\item {Proveniência:(De \textunderscore Adam\textunderscore , forma lat. de \textunderscore Adão\textunderscore , n. p.)}
\end{itemize}
Primitivo.
Relativo ao primeiro homem.
\section{Adamita}
\begin{itemize}
\item {Grp. gram.:m.}
\end{itemize}
Membro de uma seita que imitava a nudez de Adão.
(Cp. \textunderscore adâmico\textunderscore )
\section{Adamítico}
\begin{itemize}
\item {Grp. gram.:adj.}
\end{itemize}
\begin{itemize}
\item {Proveniência:(De \textunderscore adamita\textunderscore )}
\end{itemize}
Que diz respeito aos tempos primitivos. Cf. Garrett, \textunderscore D. Branca\textunderscore , 128.
\section{Adanar}
\begin{itemize}
\item {Grp. gram.:v. i.}
\end{itemize}
\begin{itemize}
\item {Utilização:açor}
\end{itemize}
\begin{itemize}
\item {Utilização:Pop.}
\end{itemize}
Nadar.
(Metáth. de \textunderscore nadar\textunderscore )
\section{Adansónia}
\begin{itemize}
\item {Grp. gram.:f.}
\end{itemize}
\begin{itemize}
\item {Proveniência:(De \textunderscore Adanson\textunderscore , n. p.)}
\end{itemize}
Robusta árvore africana, o mesmo que \textunderscore baobab\textunderscore .
\section{Adão}
\begin{itemize}
\item {Grp. gram.:m.}
\end{itemize}
Árvore da Índia portuguêsa.
\section{Adaptabilidade}
\begin{itemize}
\item {Grp. gram.:f.}
\end{itemize}
Qualidade do que é adaptável.
\section{Adaptação}
\begin{itemize}
\item {Grp. gram.:f.}
\end{itemize}
Acto de \textunderscore adaptar\textunderscore .
\section{Adaptadamente}
\begin{itemize}
\item {Grp. gram.:adv.}
\end{itemize}
De modo \textunderscore adaptado\textunderscore .
\section{Adaptado}
\begin{itemize}
\item {Grp. gram.:adj.}
\end{itemize}
Que se adaptou.
Apropriado.
\section{Adaptador}
\begin{itemize}
\item {Grp. gram.:m.}
\end{itemize}
O que adapta.
\section{Adaptar}
\begin{itemize}
\item {Grp. gram.:v. t.}
\end{itemize}
\begin{itemize}
\item {Proveniência:(Lat. \textunderscore adaptare\textunderscore )}
\end{itemize}
Tornar apto a.
Ajustar; apropriar.
\section{Adaptativo}
\begin{itemize}
\item {Grp. gram.:adj.}
\end{itemize}
Próprio para se adaptar.
\section{Adaptável}
\begin{itemize}
\item {Grp. gram.:adj.}
\end{itemize}
Que se póde adaptar.
\section{Adarce}
\begin{itemize}
\item {Grp. gram.:m.}
\end{itemize}
Salsugem, que no tempo da séca se péga ás plantas, junto das lagôas.
\section{Adarço}
\begin{itemize}
\item {Grp. gram.:m.}
\end{itemize}
\begin{itemize}
\item {Utilização:Ant.}
\end{itemize}
Escolho, baixio.
\section{Adarga}
\begin{itemize}
\item {Grp. gram.:f.}
\end{itemize}
Antigo escudo oval, de coiro.
(Ár. \textunderscore ad-daraca\textunderscore )
\section{Adargado}
\begin{itemize}
\item {Grp. gram.:adj.}
\end{itemize}
Que tem adarga.
Protegido.
\section{Adargar}
\begin{itemize}
\item {Grp. gram.:v. t.}
\end{itemize}
Defender com adarga.
Amparar; proteger. Cf. \textunderscore Eufrosina\textunderscore , 15.
\section{Adargueiro}
\begin{itemize}
\item {Grp. gram.:m.}
\end{itemize}
Militar, que usava adarga.
Fabricante de adargas.
\section{Adarme}
\begin{itemize}
\item {Grp. gram.:m.}
\end{itemize}
\begin{itemize}
\item {Proveniência:(T. cast.)}
\end{itemize}
Pêso antigo, meia oitava.
Calibre da bala.
\section{Adarvar}
\begin{itemize}
\item {Grp. gram.:v. t.}
\end{itemize}
Fortificar com adarves.
\section{Adarve}
\begin{itemize}
\item {Grp. gram.:m.}
\end{itemize}
Muro de fortaleza, com ameias.
Rua estreita, sôbre o muro da fortaleza.
(Ár. \textunderscore addarb\textunderscore )
\section{Adastra}
\begin{itemize}
\item {Grp. gram.:f.}
\end{itemize}
\begin{itemize}
\item {Proveniência:(De \textunderscore adastrar\textunderscore )}
\end{itemize}
Instrumento de ourives, para corrigir aros de anéis.
Bigorna de estender folha.
\section{Adastragem}
\begin{itemize}
\item {Grp. gram.:f.}
\end{itemize}
Acto de \textunderscore adastrar\textunderscore .
\section{Adastrar}
\begin{itemize}
\item {Grp. gram.:v. t.}
\end{itemize}
Endireitar na adastra.
(Alter. de \textunderscore adestrar\textunderscore )
\section{Adatis}
\begin{itemize}
\item {Grp. gram.:m.}
\end{itemize}
Musselina da Índia.
\section{Addenda}
\begin{itemize}
\item {Grp. gram.:f.}
\end{itemize}
Aquillo que se accrescenta ou se deve accrescentar no fim de um livro.
(Pl. de \textunderscore addendus\textunderscore , gerundivo do v. lat. \textunderscore addere\textunderscore , juntar)
\section{Addensa-nuvens}
\begin{itemize}
\item {Grp. gram.:adj.}
\end{itemize}
\begin{itemize}
\item {Utilização:Poét.}
\end{itemize}
Que accumula as nuvens.
\section{Addensar}
\begin{itemize}
\item {Grp. gram.:v. t.}
\end{itemize}
\begin{itemize}
\item {Proveniência:(Lat. \textunderscore addensare\textunderscore )}
\end{itemize}
Condensar.
\section{Addental}
\begin{itemize}
\item {Grp. gram.:m.  e  adj.}
\end{itemize}
Diz-se, em
Anatomia, de uma das peças elementares de uma das vértebras cephálicas.
\section{Addição}
\begin{itemize}
\item {Grp. gram.:f.}
\end{itemize}
\begin{itemize}
\item {Proveniência:(Lat. \textunderscore additio\textunderscore )}
\end{itemize}
Acto ou effeito de addir.
Somma.
Parcella.
Successo recente.
Appêndice de uma construcção.
\section{Addicionação}
\begin{itemize}
\item {Grp. gram.:f.}
\end{itemize}
Acto ou effeito de \textunderscore addicionar\textunderscore .
\section{Addicionador}
\begin{itemize}
\item {Grp. gram.:m.}
\end{itemize}
\begin{itemize}
\item {Proveniência:(De \textunderscore addicionar\textunderscore )}
\end{itemize}
O que addiciona.
\section{Addicional}
\begin{itemize}
\item {Grp. gram.:adj.}
\end{itemize}
\begin{itemize}
\item {Grp. gram.:M.}
\end{itemize}
\begin{itemize}
\item {Proveniência:(Lat. \textunderscore additio\textunderscore , \textunderscore additionis\textunderscore )}
\end{itemize}
Que se addiciona.
Que accresce.
Aquillo que accresce ou se addiciona.
\section{Addicionamento}
\begin{itemize}
\item {Grp. gram.:m.}
\end{itemize}
O mesmo que \textunderscore addicionação\textunderscore .
\section{Addicionar}
\begin{itemize}
\item {Grp. gram.:v. t.}
\end{itemize}
\begin{itemize}
\item {Proveniência:(Do lat. \textunderscore additio\textunderscore , \textunderscore additionis\textunderscore )}
\end{itemize}
Ajuntar.
Accrescentar em nota ou commentário.
\section{Addicionável}
\begin{itemize}
\item {Grp. gram.:adj.}
\end{itemize}
Que se póde addicionar.
\section{Addicto}
\begin{itemize}
\item {Grp. gram.:adj.}
\end{itemize}
\begin{itemize}
\item {Grp. gram.:adj.}
\end{itemize}
\begin{itemize}
\item {Proveniência:(Lat. \textunderscore addictus\textunderscore )}
\end{itemize}
Afeiçoado; dedicado.
Adjunto.
\section{Addido}
\begin{itemize}
\item {Grp. gram.:m.}
\end{itemize}
Funccionário, que está junto a um dignitário ou corporação, para auxiliar.
Funccionário de qualquer categoria, que está a mais do quadro respectivo, por exceder o número legal, ou por sêr officialmente dispensado de servir.
\section{Addir}
\begin{itemize}
\item {Grp. gram.:v. t.}
\end{itemize}
\begin{itemize}
\item {Grp. gram.:V. p.}
\end{itemize}
Accrescentar.
Ajuntar; aggregar.
Ajuntar-se, ligar-se. Cf. Camillo, \textunderscore O Senh. do Paço de Nin.\textunderscore , 152.
\section{Additamento}
\begin{itemize}
\item {Grp. gram.:m.}
\end{itemize}
Acto de \textunderscore addittar\textunderscore .
Aquillo que se additou.
\section{Additar}
\begin{itemize}
\item {Proveniência:(Do lat. \textunderscore additus\textunderscore , de \textunderscore addere\textunderscore )}
\end{itemize}
\textunderscore v. t.\textunderscore  (e der.)
O mesmo que \textunderscore addicionar\textunderscore .
\section{Adducção}
\begin{itemize}
\item {Grp. gram.:f.}
\end{itemize}
Acto ou effeito de adduzir.
\section{Adducente}
\begin{itemize}
\item {Grp. gram.:adj.}
\end{itemize}
\begin{itemize}
\item {Proveniência:(Lat. \textunderscore adducens\textunderscore )}
\end{itemize}
Que adduz.
\section{Adductivo}
\begin{itemize}
\item {Grp. gram.:adj.}
\end{itemize}
\begin{itemize}
\item {Proveniência:(Do lat. \textunderscore adductus\textunderscore )}
\end{itemize}
Que póde adduzir.
\section{Adductor}
\begin{itemize}
\item {Grp. gram.:m.}
\end{itemize}
\begin{itemize}
\item {Proveniência:(Lat. \textunderscore adductor\textunderscore )}
\end{itemize}
O que adduz.
\section{Adduzer}
\begin{itemize}
\item {Grp. gram.:v. t.}
\end{itemize}
\begin{itemize}
\item {Utilização:Ant.}
\end{itemize}
O mesmo que \textunderscore adduzir\textunderscore .
\section{Adduzir}
\begin{itemize}
\item {Grp. gram.:v. t.}
\end{itemize}
\begin{itemize}
\item {Proveniência:(Lat. \textunderscore adducere\textunderscore )}
\end{itemize}
Trazer.
Expor, apresentar.
\section{Ade}
\begin{itemize}
\item {Grp. gram.:m.}
\end{itemize}
\begin{itemize}
\item {Utilização:T. de Macau}
\end{itemize}
O mesmo que \textunderscore adem\textunderscore .
Qualquer ave palmípede.
\section{Adeantadamente}
\begin{itemize}
\item {Grp. gram.:adv.}
\end{itemize}
Com antecipação.
De modo \textunderscore adeantado\textunderscore .
\section{Adeantado}
\begin{itemize}
\item {Grp. gram.:m.}
\end{itemize}
\begin{itemize}
\item {Grp. gram.:Adj.}
\end{itemize}
\begin{itemize}
\item {Proveniência:(De \textunderscore adeantar\textunderscore )}
\end{itemize}
Antigo governador de província.
Pago antecipadamente: \textunderscore dinheiro adeantado\textunderscore .
Que mostra progressos: \textunderscore nação adeantada\textunderscore .
Que vai adeante de outros na carreira escolar: \textunderscore um estudante adeantado\textunderscore .
\section{Adeantamento}
\begin{itemize}
\item {Grp. gram.:m.}
\end{itemize}
Acto de \textunderscore adeantar\textunderscore , ou de se adeantar.
Progresso.
Abono de dinheiro, antes do tempo, a que corresponde o respectivo pagamento.
\section{Adeantar}
\begin{itemize}
\item {Grp. gram.:v. t.}
\end{itemize}
\begin{itemize}
\item {Grp. gram.:V. p.}
\end{itemize}
\begin{itemize}
\item {Utilização:Fam.}
\end{itemize}
Fazer com antecedência.
Pagar antecipadamente.
Fazer progredir.
Accelerar: \textunderscore adeantar o relógio\textunderscore .
Avançar.
Avantajar-se.
Atrever-se.
\section{Adeante}
\begin{itemize}
\item {Grp. gram.:adv.}
\end{itemize}
\begin{itemize}
\item {Proveniência:(De \textunderscore a\textunderscore  + \textunderscore deante\textunderscore )}
\end{itemize}
Na frente.
Em primeiro lugar.
No lugar immediato.
No futuro.
Successivamente.
Na página ou páginas seguintes.
\section{Adecto}
\begin{itemize}
\item {Grp. gram.:adj.}
\end{itemize}
\begin{itemize}
\item {Grp. gram.:M.}
\end{itemize}
\begin{itemize}
\item {Proveniência:(Gr. \textunderscore adektos\textunderscore )}
\end{itemize}
Diz-se do medicamento brando, que acalma o effeito de um medicamento enérgico.
Gênero de fêtos, proposto por Link.
\section{Adedentro}
\begin{itemize}
\item {Grp. gram.:adv.}
\end{itemize}
Interiormente. Cf. Rui Barbosa, \textunderscore Répl.\textunderscore , 157.
\section{Adefora}
\begin{itemize}
\item {Grp. gram.:adv.}
\end{itemize}
\begin{itemize}
\item {Utilização:Ant.}
\end{itemize}
\begin{itemize}
\item {Proveniência:(De \textunderscore fóra\textunderscore )}
\end{itemize}
Exteriormente.
Apparentemente.
\section{Adega}
\begin{itemize}
\item {Grp. gram.:f.}
\end{itemize}
\begin{itemize}
\item {Proveniência:(Do lat. \textunderscore apotheca\textunderscore )}
\end{itemize}
Casa térrea, em que se guarda vinho envasilhado, e outras bebidas alcoólicas.
\section{Adegar}
\begin{itemize}
\item {Grp. gram.:v. t.}
\end{itemize}
\begin{itemize}
\item {Grp. gram.:V. i.}
\end{itemize}
\begin{itemize}
\item {Utilização:Fig.}
\end{itemize}
Guardar em adega.
Beber demasiadamente.
\section{Adegueiro}
\begin{itemize}
\item {Grp. gram.:m.}
\end{itemize}
Homem que trata de adega.
\section{Adeito}
\begin{itemize}
\item {Grp. gram.:m.}
\end{itemize}
\begin{itemize}
\item {Utilização:Prov.}
\end{itemize}
\begin{itemize}
\item {Utilização:beir.}
\end{itemize}
Porção de linho, antes de assedado, e atado de fórma que dá ideia de uma boneca.
(Colhido no Fundão)
\section{Adejar}
\begin{itemize}
\item {Grp. gram.:v. i.}
\end{itemize}
Librar as asas.
Pairar.
Esvoaçar.
Voejar.
(Por \textunderscore alejar\textunderscore , do lat. \textunderscore ala\textunderscore , asa)
\section{Adejo}
\begin{itemize}
\item {Grp. gram.:m.}
\end{itemize}
Acto de \textunderscore adejar\textunderscore .
\section{Adela}
\begin{itemize}
\item {Grp. gram.:f.}
\end{itemize}
\begin{itemize}
\item {Proveniência:(De \textunderscore adelo\textunderscore )}
\end{itemize}
Mulher, que compra e vende fato feito, e outros objectos, usados.
\section{Adela}
\begin{itemize}
\item {Grp. gram.:f.}
\end{itemize}
\begin{itemize}
\item {Utilização:Ant.}
\end{itemize}
O mesmo que \textunderscore aduela\textunderscore .
\section{Adelaida}
\begin{itemize}
\item {Grp. gram.:f.}
\end{itemize}
Copada árvore americana.
\section{Adelaide}
\begin{itemize}
\item {Grp. gram.:f.}
\end{itemize}
\begin{itemize}
\item {Utilização:Pop.}
\end{itemize}
O mesmo que \textunderscore adelaidinha\textunderscore .
\section{Adelaidinha}
\begin{itemize}
\item {Grp. gram.:f.}
\end{itemize}
Designação pop. da planta, também conhecida por \textunderscore bons-dias\textunderscore .
\section{Adeleira}
\begin{itemize}
\item {Grp. gram.:f.}
\end{itemize}
\begin{itemize}
\item {Utilização:T. do Porto}
\end{itemize}
\begin{itemize}
\item {Proveniência:(De \textunderscore adeleiro\textunderscore )}
\end{itemize}
Mulher, que compra e vende fato e outros objectos, usados.
Mulher, que inculca criadas de servir.
\section{Adeleiro}
\begin{itemize}
\item {Grp. gram.:m.}
\end{itemize}
\begin{itemize}
\item {Utilização:Prov.}
\end{itemize}
\begin{itemize}
\item {Utilização:dur.}
\end{itemize}
\begin{itemize}
\item {Utilização:T. do Porto}
\end{itemize}
O mesmo que \textunderscore adelo\textunderscore .
Inculcador de criadas.
\section{Adelfa}
\begin{itemize}
\item {Grp. gram.:f.}
\end{itemize}
\begin{itemize}
\item {Proveniência:(Do ár. \textunderscore addifla\textunderscore )}
\end{itemize}
O mesmo que \textunderscore loendro\textunderscore .
\section{Adelfia}
\begin{itemize}
\item {Grp. gram.:f.}
\end{itemize}
\begin{itemize}
\item {Utilização:Bot.}
\end{itemize}
\begin{itemize}
\item {Proveniência:(De \textunderscore adelpho\textunderscore )}
\end{itemize}
União dos estames por meio dos seus filetes.
\section{Adelfo}
\begin{itemize}
\item {Grp. gram.:adj.}
\end{itemize}
\begin{itemize}
\item {Utilização:Bot.}
\end{itemize}
\begin{itemize}
\item {Grp. gram.:M.}
\end{itemize}
\begin{itemize}
\item {Proveniência:(Gr. \textunderscore adelphos\textunderscore )}
\end{itemize}
Que tem ligados entre si os filetes dos estames.
Gênero de insectos coleópteros heterómenos.
\section{Adelgaçadamente}
\begin{itemize}
\item {Grp. gram.:adv.}
\end{itemize}
De modo \textunderscore adelgaçado\textunderscore .
\section{Adelgaçado}
\begin{itemize}
\item {Grp. gram.:adj.}
\end{itemize}
\begin{itemize}
\item {Proveniência:(De \textunderscore adelgaçar\textunderscore )}
\end{itemize}
Que se tornou delgado.
Desgastado com o uso.
\section{Adelgaçador}
\begin{itemize}
\item {Grp. gram.:m.}
\end{itemize}
O que adelgaça.
\section{Adelgaçamento}
\begin{itemize}
\item {Grp. gram.:m.}
\end{itemize}
Acto ou efeito de \textunderscore adelgaçar\textunderscore .
\section{Adelgaçar}
\begin{itemize}
\item {Grp. gram.:v. t.}
\end{itemize}
\begin{itemize}
\item {Proveniência:(Do b. lat. \textunderscore addelicatiare\textunderscore )}
\end{itemize}
Tornar delgado, agudo.
Desgastar.
\section{Adelgadar}
\begin{itemize}
\item {Grp. gram.:v. t.}
\end{itemize}
\begin{itemize}
\item {Proveniência:(De \textunderscore delgado\textunderscore )}
\end{itemize}
O mesmo que \textunderscore adelgaçar\textunderscore .
\section{Adelgar}
\begin{itemize}
\item {Grp. gram.:v. t.}
\end{itemize}
O mesmo que \textunderscore adelgaçar\textunderscore .
\section{Adelha}
\begin{itemize}
\item {fónica:dê}
\end{itemize}
\begin{itemize}
\item {Grp. gram.:f.}
\end{itemize}
\begin{itemize}
\item {Utilização:Prov.}
\end{itemize}
\begin{itemize}
\item {Utilização:minh.}
\end{itemize}
Caixa de madeira, em fórma de pyrâmide invertida, e na qual se deita o cereal que vai cair no adelhão, para sêr moído.
Tremonha; canoira; moéga.
\section{Adelhão}
\begin{itemize}
\item {Grp. gram.:m.}
\end{itemize}
\begin{itemize}
\item {Utilização:Prov.}
\end{itemize}
\begin{itemize}
\item {Utilização:minh.}
\end{itemize}
\begin{itemize}
\item {Proveniência:(De \textunderscore adelha\textunderscore )}
\end{itemize}
Pequena caleira, suspensa da adelha, e cuja inclinação é regulada por um cordel, preso ao pau da varela, nas azenhas.
\section{Adélia}
\begin{itemize}
\item {Grp. gram.:f.}
\end{itemize}
\begin{itemize}
\item {Proveniência:(Do gr. \textunderscore a\textunderscore  priv. + \textunderscore delos\textunderscore )}
\end{itemize}
Planta euphorbiácea.
\section{Adelicadar-se}
\begin{itemize}
\item {Grp. gram.:v. p.}
\end{itemize}
\begin{itemize}
\item {Utilização:P. us.}
\end{itemize}
Tornar-se delicado.
\section{Adélido}
\begin{itemize}
\item {Grp. gram.:adj.}
\end{itemize}
\begin{itemize}
\item {Utilização:Pathol.}
\end{itemize}
\begin{itemize}
\item {Proveniência:(Do gr. \textunderscore adelos\textunderscore , pouco apparente, incerto)}
\end{itemize}
Pouco sensível; que mal se percebe.
\section{Adélio}
\begin{itemize}
\item {Grp. gram.:m.}
\end{itemize}
Gênero de insectos hymenópteros.
\section{Adelo}
\begin{itemize}
\item {fónica:dê}
\end{itemize}
\begin{itemize}
\item {Grp. gram.:m.}
\end{itemize}
\begin{itemize}
\item {Proveniência:(Do ár. \textunderscore ad-dellala\textunderscore )}
\end{itemize}
O que compra e vende fato e outros objectos, usados.
\section{Adelóbio}
\begin{itemize}
\item {Grp. gram.:m.}
\end{itemize}
\begin{itemize}
\item {Proveniência:(Do gr. \textunderscore adelos\textunderscore  + \textunderscore bios\textunderscore )}
\end{itemize}
Gênero de insectos coleópteros.
\section{Adelobrânchio}
\begin{itemize}
\item {fónica:qui}
\end{itemize}
\begin{itemize}
\item {Grp. gram.:adj.}
\end{itemize}
\begin{itemize}
\item {Grp. gram.:M. pl.}
\end{itemize}
\begin{itemize}
\item {Proveniência:(Do gr. \textunderscore adelos\textunderscore  + \textunderscore brankhia\textunderscore )}
\end{itemize}
Que tem as brânchias visíveis.
Grupo de molluscos gasterópodes.
\section{Adelobrânquio}
\begin{itemize}
\item {Grp. gram.:adj.}
\end{itemize}
\begin{itemize}
\item {Grp. gram.:M. pl.}
\end{itemize}
\begin{itemize}
\item {Proveniência:(Do gr. \textunderscore adelos\textunderscore  + \textunderscore brankhia\textunderscore )}
\end{itemize}
Que tem as brânchias visíveis.
Grupo de molluscos gasterópodes.
\section{Adelópode}
\begin{itemize}
\item {Grp. gram.:adj.}
\end{itemize}
\begin{itemize}
\item {Utilização:Zool.}
\end{itemize}
\begin{itemize}
\item {Proveniência:(Do gr. \textunderscore adelos\textunderscore  + \textunderscore pous\textunderscore , \textunderscore podos\textunderscore )}
\end{itemize}
Que não tem pés apparentes.
\section{Adelopódio}
\begin{itemize}
\item {Grp. gram.:adj.}
\end{itemize}
\begin{itemize}
\item {Utilização:Zool.}
\end{itemize}
\begin{itemize}
\item {Proveniência:(Do gr. \textunderscore adelos\textunderscore  + \textunderscore pous\textunderscore , \textunderscore podos\textunderscore )}
\end{itemize}
Que não tem pés apparentes.
\section{Adelosa}
\begin{itemize}
\item {Grp. gram.:f.}
\end{itemize}
\begin{itemize}
\item {Proveniência:(Do gr. \textunderscore adelos\textunderscore )}
\end{itemize}
Gênero de plantas verbenáceas.
\section{Adelphia}
\begin{itemize}
\item {Grp. gram.:f.}
\end{itemize}
\begin{itemize}
\item {Utilização:Bot.}
\end{itemize}
\begin{itemize}
\item {Proveniência:(De \textunderscore adelpho\textunderscore )}
\end{itemize}
União dos estames por meio dos seus filetes.
\section{Adelpho}
\begin{itemize}
\item {Grp. gram.:adj.}
\end{itemize}
\begin{itemize}
\item {Utilização:Bot.}
\end{itemize}
\begin{itemize}
\item {Grp. gram.:M.}
\end{itemize}
\begin{itemize}
\item {Proveniência:(Gr. \textunderscore adelphos\textunderscore )}
\end{itemize}
Que tem ligados entre si os filetes dos estames.
Gênero de insectos coleópteros heterómenos.
\section{A-d'el-rei}
\begin{itemize}
\item {Grp. gram.:m.}
\end{itemize}
Grito de soccorro, o mesmo que \textunderscore aqui-d'el-rei\textunderscore :«\textunderscore os a-d'el-reis rara noite se não ouviam\textunderscore ». Camillo, \textunderscore Cavar em Ruin.\textunderscore , 222.
\section{Adem}
\begin{itemize}
\item {Grp. gram.:f.}
\end{itemize}
\begin{itemize}
\item {Proveniência:(Do lat. \textunderscore anas\textunderscore , \textunderscore anatis\textunderscore )}
\end{itemize}
Ave palmipede, lamellirostra.
\section{Adema}
\begin{itemize}
\item {Grp. gram.:f.}
\end{itemize}
\begin{itemize}
\item {Utilização:Ant.}
\end{itemize}
Terreno cultivado.
\section{Ademado}
\textunderscore adj.\textunderscore  (?)
(«\textunderscore ...outro mar achou a nao tão ademada, que quasi a acabou de meter debaixo de agoa...\textunderscore »\textunderscore Hist. Trág.-Marít.\textunderscore , 50).
Provavelmente, é êrro typográphico, devendo lêr-se \textunderscore adernada\textunderscore .
\section{Ademães}
\begin{itemize}
\item {Grp. gram.:m. pl.}
\end{itemize}
O mesmo que \textunderscore ademanes\textunderscore . Cf. Castilho, \textunderscore Metam.\textunderscore , XXII.
\section{A-de-mais}
\begin{itemize}
\item {Grp. gram.:adv.  e  prep.}
\end{itemize}
Além.
Demais.
\section{Ademan}
\begin{itemize}
\item {Grp. gram.:m.}
\end{itemize}
\begin{itemize}
\item {Proveniência:(Do lat. \textunderscore ad\textunderscore  + \textunderscore de\textunderscore  + \textunderscore manus\textunderscore )}
\end{itemize}
Gestos, trejeitos.
Modos affectados.
\section{Ademanes}
\begin{itemize}
\item {Grp. gram.:m. pl.}
\end{itemize}
\begin{itemize}
\item {Proveniência:(Do lat. \textunderscore ad\textunderscore  + \textunderscore de\textunderscore  + \textunderscore manus\textunderscore )}
\end{itemize}
Gestos, trejeitos.
Modos affectados.
\section{Adêmea}
\begin{itemize}
\item {Grp. gram.:f.}
\end{itemize}
Terreno susceptível de cultura, entre monte e várzea.
(Cp. \textunderscore adema\textunderscore )
\section{Adempção}
\begin{itemize}
\item {Grp. gram.:f.}
\end{itemize}
\begin{itemize}
\item {Proveniência:(Lat. \textunderscore ademptio\textunderscore )}
\end{itemize}
Revogação (de legado).
\section{Áden}
\begin{itemize}
\item {Grp. gram.:f.}
\end{itemize}
\begin{itemize}
\item {Proveniência:(Gr. \textunderscore aden\textunderscore )}
\end{itemize}
Glândula; corpo glanduloso.
\section{Adenacantho}
\begin{itemize}
\item {Grp. gram.:m.}
\end{itemize}
Gênero de plantas acantháceas.
\section{Adenacanto}
\begin{itemize}
\item {Grp. gram.:m.}
\end{itemize}
Gênero de plantas acantháceas.
\section{Adenandro}
\begin{itemize}
\item {Grp. gram.:m.}
\end{itemize}
Gênero de plantas rutáceas.
\section{Adenalgia}
\begin{itemize}
\item {Grp. gram.:f.}
\end{itemize}
\begin{itemize}
\item {Proveniência:(Do gr. \textunderscore aden\textunderscore  + \textunderscore algos\textunderscore )}
\end{itemize}
Dôr numa glândula.
\section{Adenálgico}
\begin{itemize}
\item {Grp. gram.:adj.}
\end{itemize}
Relativo a adenalgia.
\section{Adenantero}
\begin{itemize}
\item {Grp. gram.:m.}
\end{itemize}
\begin{itemize}
\item {Proveniência:(Do gr. \textunderscore aden\textunderscore  + \textunderscore anthera\textunderscore )}
\end{itemize}
Gênero de plantas leguminosas.
\section{Adenanthero}
\begin{itemize}
\item {Grp. gram.:m.}
\end{itemize}
\begin{itemize}
\item {Proveniência:(Do gr. \textunderscore aden\textunderscore  + \textunderscore anthera\textunderscore )}
\end{itemize}
Gênero de plantas leguminosas.
\section{Adenantho}
\begin{itemize}
\item {Grp. gram.:m.}
\end{itemize}
\begin{itemize}
\item {Proveniência:(Do gr. \textunderscore aden\textunderscore  + \textunderscore anthos\textunderscore )}
\end{itemize}
Gênero de plantas protáceas.
\section{Adenanto}
\begin{itemize}
\item {Grp. gram.:m.}
\end{itemize}
\begin{itemize}
\item {Proveniência:(Do gr. \textunderscore aden\textunderscore  + \textunderscore anthos\textunderscore )}
\end{itemize}
Gênero de plantas protáceas.
\section{Adenção}
\begin{itemize}
\item {Grp. gram.:f.}
\end{itemize}
\begin{itemize}
\item {Proveniência:(Lat. \textunderscore ademptio\textunderscore )}
\end{itemize}
Revogação (de legado).
\section{Adenia}
\begin{itemize}
\item {Grp. gram.:f.}
\end{itemize}
\begin{itemize}
\item {Proveniência:(Do gr. \textunderscore aden\textunderscore )}
\end{itemize}
Doença das glândulas.
Planta trepadeira, venenosa, da Arábia.
\section{Adenite}
\begin{itemize}
\item {Grp. gram.:f.}
\end{itemize}
\begin{itemize}
\item {Proveniência:(Do gr. \textunderscore aden\textunderscore )}
\end{itemize}
Inflammação de glândulas.
\section{Adenocarpo}
\begin{itemize}
\item {Grp. gram.:m.}
\end{itemize}
\begin{itemize}
\item {Proveniência:(Do gr. \textunderscore aden\textunderscore  + \textunderscore karpos\textunderscore )}
\end{itemize}
Gênero de plantas papilionáceas.
\section{Adenograma}
\begin{itemize}
\item {Grp. gram.:m.}
\end{itemize}
Gênero de plantas, da fam. das figueiras, (\textunderscore stendelia\textunderscore , Presl.).
\section{Adenogramma}
\begin{itemize}
\item {Grp. gram.:m.}
\end{itemize}
Gênero de plantas, da fam. das figueiras, (\textunderscore stendelia\textunderscore , Presl.).
\section{Adenófora}
\begin{itemize}
\item {Grp. gram.:f.}
\end{itemize}
\begin{itemize}
\item {Proveniência:(Do gr. \textunderscore aden\textunderscore  + \textunderscore phoros\textunderscore )}
\end{itemize}
Gênero de plantas campanuláceas.
\section{Adenoide}
\begin{itemize}
\item {Grp. gram.:adj.}
\end{itemize}
\begin{itemize}
\item {Proveniência:(Do gr. \textunderscore aden\textunderscore  + \textunderscore eidos\textunderscore )}
\end{itemize}
Que tem fórma de glândula.
\section{Adenologia}
\begin{itemize}
\item {Grp. gram.:f.}
\end{itemize}
\begin{itemize}
\item {Proveniência:(Do gr. \textunderscore aden\textunderscore  + \textunderscore logos\textunderscore )}
\end{itemize}
Parte da Anatomia, que trata de glândulas.
\section{Adenoma}
\begin{itemize}
\item {Grp. gram.:m.}
\end{itemize}
\begin{itemize}
\item {Proveniência:(Do gr. \textunderscore aden\textunderscore )}
\end{itemize}
Tumor glandular.
\section{Adenopata}
\begin{itemize}
\item {Grp. gram.:m.}
\end{itemize}
O que soffre de \textunderscore adenopatia\textunderscore .
\section{Adenopatha}
\begin{itemize}
\item {Grp. gram.:m.}
\end{itemize}
O que soffre de \textunderscore adenopathia\textunderscore .
\section{Adenopathia}
\begin{itemize}
\item {Grp. gram.:f.}
\end{itemize}
\begin{itemize}
\item {Proveniência:(Do gr. \textunderscore aden\textunderscore  + \textunderscore pathos\textunderscore )}
\end{itemize}
Doença das glândulas em geral e dos gânglios lympháticos em particular.
\section{Adenopatia}
\begin{itemize}
\item {Grp. gram.:f.}
\end{itemize}
\begin{itemize}
\item {Proveniência:(Do gr. \textunderscore aden\textunderscore  + \textunderscore pathos\textunderscore )}
\end{itemize}
Doença das glândulas em geral e dos gânglios lympháticos em particular.
\section{Adenóphora}
\begin{itemize}
\item {Grp. gram.:f.}
\end{itemize}
\begin{itemize}
\item {Proveniência:(Do gr. \textunderscore aden\textunderscore  + \textunderscore phoros\textunderscore )}
\end{itemize}
Gênero de plantas campanuláceas.
\section{Adenos}
\begin{itemize}
\item {Grp. gram.:m.}
\end{itemize}
Designação antiga do algodão.
\section{Adenostêmono}
\begin{itemize}
\item {Grp. gram.:adj.}
\end{itemize}
\begin{itemize}
\item {Proveniência:(Do gr. \textunderscore aden\textunderscore  + \textunderscore stemon\textunderscore )}
\end{itemize}
Diz-se das plantas que têm glândulas nos filetes dos estames.
\section{Adenostíleas}
\begin{itemize}
\item {Grp. gram.:f. pl.}
\end{itemize}
Tribo de synanthéreas, segundo Cassini.
\section{Adenostýleas}
\begin{itemize}
\item {Grp. gram.:f. pl.}
\end{itemize}
Tribo de synanthéreas, segundo Cassini.
\section{Adenotomia}
\begin{itemize}
\item {Grp. gram.:f.}
\end{itemize}
\begin{itemize}
\item {Proveniência:(Do gr. \textunderscore aden\textunderscore  + \textunderscore tome\textunderscore )}
\end{itemize}
Dissecção das glândulas.
\section{Adenotómico}
\begin{itemize}
\item {Grp. gram.:adj.}
\end{itemize}
Relativo á \textunderscore adenòtomia\textunderscore .
\section{Adensa-nuvens}
\begin{itemize}
\item {Grp. gram.:adj.}
\end{itemize}
\begin{itemize}
\item {Utilização:Poét.}
\end{itemize}
Que accumula as nuvens.
\section{Adensar}
\begin{itemize}
\item {Grp. gram.:v. t.}
\end{itemize}
\begin{itemize}
\item {Proveniência:(Lat. \textunderscore addensare\textunderscore )}
\end{itemize}
Condensar.
\section{Adental}
\begin{itemize}
\item {Grp. gram.:m.  e  adj.}
\end{itemize}
Diz-se, em
Anatomia, de uma das peças elementares de uma das vértebras cephálicas.
\section{Adentar}
\begin{itemize}
\item {Grp. gram.:v. t.}
\end{itemize}
O mesmo que \textunderscore dentar\textunderscore .
\section{Adente}
\begin{itemize}
\item {Grp. gram.:adv.}
\end{itemize}
\begin{itemize}
\item {Utilização:Prov.}
\end{itemize}
\begin{itemize}
\item {Utilização:minh.}
\end{itemize}
O mesmo que \textunderscore adeante\textunderscore .
\section{Adentrar}
\begin{itemize}
\item {Grp. gram.:v. i.}
\end{itemize}
O mesmo que \textunderscore adentrar-se\textunderscore . Cf. Filinto, \textunderscore D. Man.\textunderscore , I, 67.
\section{Adentrar-se}
\begin{itemize}
\item {Grp. gram.:v. p.}
\end{itemize}
\begin{itemize}
\item {Proveniência:(De \textunderscore adentro\textunderscore )}
\end{itemize}
Entrar, concentrar-se.
\section{Adentro}
\begin{itemize}
\item {Grp. gram.:adv.}
\end{itemize}
\begin{itemize}
\item {Proveniência:(De \textunderscore dentro\textunderscore )}
\end{itemize}
Para dentro.
Interiormente.
\section{Adeos}
\begin{itemize}
\item {Grp. gram.:adv.}
\end{itemize}
\begin{itemize}
\item {Grp. gram.:M.}
\end{itemize}
\begin{itemize}
\item {Proveniência:(De \textunderscore a\textunderscore  + \textunderscore Deus\textunderscore , n. p., segundo a opinião corrente; mas a fórma antiga \textunderscore ay-Deos\textunderscore  contraria essa opinião)}
\end{itemize}
Fica com Deus.
Deus vá contigo.
Despedida.
Fim.
\section{Adepto}
\begin{itemize}
\item {Grp. gram.:m.}
\end{itemize}
\begin{itemize}
\item {Proveniência:(Lat. \textunderscore adeptus\textunderscore )}
\end{itemize}
Sectário; partidário.
\section{Adequação}
\begin{itemize}
\item {Grp. gram.:f.}
\end{itemize}
\begin{itemize}
\item {Proveniência:(Lat. \textunderscore adaequatio\textunderscore )}
\end{itemize}
Acto de adequar.
\section{Adequadamente}
\begin{itemize}
\item {Grp. gram.:adv.}
\end{itemize}
De modo \textunderscore adequado\textunderscore .
\section{Adequado}
\begin{itemize}
\item {Grp. gram.:adj.}
\end{itemize}
\begin{itemize}
\item {Proveniência:(De \textunderscore adequar\textunderscore )}
\end{itemize}
Apropriado.
Proporcionado.
\section{Adequar}
\begin{itemize}
\item {Grp. gram.:v. t.}
\end{itemize}
\begin{itemize}
\item {Proveniência:(Lat. \textunderscore adaequare\textunderscore )}
\end{itemize}
Accommodar.
Proporcionar.
\section{Ader}
\begin{itemize}
\item {Grp. gram.:v. t.}
\end{itemize}
\begin{itemize}
\item {Utilização:Ant.}
\end{itemize}
O mesmo que \textunderscore addir\textunderscore .
\section{Aderar}
\begin{itemize}
\item {Grp. gram.:v. t.}
\end{itemize}
\begin{itemize}
\item {Utilização:Ant.}
\end{itemize}
\begin{itemize}
\item {Proveniência:(Do lat. \textunderscore ad\textunderscore  + \textunderscore aes\textunderscore , \textunderscore aeris\textunderscore , moéda)}
\end{itemize}
Taxar a dinheiro.
Avaliar, apreçar.
\section{Adereçamento}
\begin{itemize}
\item {Grp. gram.:m.}
\end{itemize}
Acto ou effeito de \textunderscore adereçar\textunderscore .
\section{Adereçar}
\begin{itemize}
\item {Grp. gram.:v. t.}
\end{itemize}
\begin{itemize}
\item {Proveniência:(De \textunderscore aderêço\textunderscore )}
\end{itemize}
Adornar.
Dirigir, enviar.
\section{Aderecista}
\begin{itemize}
\item {Grp. gram.:m.}
\end{itemize}
O que adereça ou enfeita.
O encarregado de adereços de theatro.
\section{Aderêço}
\begin{itemize}
\item {Grp. gram.:m.}
\end{itemize}
\begin{itemize}
\item {Proveniência:(Do lat. hyp. \textunderscore directius\textunderscore )}
\end{itemize}
Enfeite, adôrno.
Indicação da residência ou estabelecimento de alguém.
\section{Aderençar}
\begin{itemize}
\item {Grp. gram.:v. t.}
\end{itemize}
\begin{itemize}
\item {Grp. gram.:V. i.}
\end{itemize}
\begin{itemize}
\item {Utilização:Ant.}
\end{itemize}
Adereçar.
Endereçar.
Dirigir-se, encaminhar-se.
(Cp. \textunderscore adereçar\textunderscore )
\section{Aderência}
\begin{itemize}
\item {Grp. gram.:f.}
\end{itemize}
Qualidade do que é \textunderscore adherente\textunderscore .
Acto de adherir.
\section{Aderente}
\begin{itemize}
\item {Grp. gram.:m.  e  adj.}
\end{itemize}
\begin{itemize}
\item {Grp. gram.:M.}
\end{itemize}
O que adhere.
Partidário, prosélyto.
\section{Adergar}
\begin{itemize}
\item {Grp. gram.:v. i.}
\end{itemize}
\begin{itemize}
\item {Grp. gram.:V. t.}
\end{itemize}
O mesmo que \textunderscore adregar\textunderscore .
Encontrar por acaso.
Descobrir.
Conseguir. Cf. Castilho, \textunderscore Fastos\textunderscore , II, 161.
\section{Aderir}
\begin{itemize}
\item {Grp. gram.:v. i.}
\end{itemize}
\begin{itemize}
\item {Grp. gram.:V. t.}
\end{itemize}
Estar unido.
Conformar-se, aprovando.
Unir, juntar:«\textunderscore veja-se póde adheri-los ao epicrânio.\textunderscore »Camillo, \textunderscore Sebenta\textunderscore , VII, 19.
\section{Adernar}
\begin{itemize}
\item {Grp. gram.:v. i.}
\end{itemize}
\begin{itemize}
\item {Utilização:Náut.}
\end{itemize}
Inclinar-se, ficando, de um lado, debaixo de água (o navio).
\section{Aderno}
\begin{itemize}
\item {Grp. gram.:m.}
\end{itemize}
\begin{itemize}
\item {Proveniência:(Do lat. \textunderscore alaternus\textunderscore )}
\end{itemize}
Arbusto, da fam. das rhamnáceas.
\section{Á-derradeira}
\begin{itemize}
\item {Grp. gram.:loc. adv.}
\end{itemize}
\begin{itemize}
\item {Utilização:Ant.}
\end{itemize}
Em-fim, a final.
\section{Adesão}
\begin{itemize}
\item {Grp. gram.:f.}
\end{itemize}
\begin{itemize}
\item {Proveniência:(Lat. \textunderscore adhaesio\textunderscore )}
\end{itemize}
Acto de \textunderscore adherir\textunderscore .
Ligação.
Acôrdo.
\section{A-deshoras}
\begin{itemize}
\item {Grp. gram.:loc. adv.}
\end{itemize}
Tarde.
Inopportunamente.
\section{Adesivado}
\begin{itemize}
\item {Grp. gram.:adj.}
\end{itemize}
Que tem adesivo: \textunderscore paus adesivados\textunderscore .
\section{Adesivamente}
\begin{itemize}
\item {Grp. gram.:adv.}
\end{itemize}
Com adesão.
De modo \textunderscore adhesivo\textunderscore .
\section{Adesivo}
\begin{itemize}
\item {Grp. gram.:m.}
\end{itemize}
\begin{itemize}
\item {Grp. gram.:Adj.}
\end{itemize}
\begin{itemize}
\item {Proveniência:(De \textunderscore adheso\textunderscore )}
\end{itemize}
Emplasto, que adere á pelle.
Que adere.
\section{Adeso}
\begin{itemize}
\item {Proveniência:(Lat. \textunderscore adhaesus\textunderscore )}
\end{itemize}
Part. irr. de \textunderscore aderir\textunderscore .
\section{Adesol}
\begin{itemize}
\item {Grp. gram.:m.}
\end{itemize}
Produto pharmacêutico, que tem o mesmo uso que o collódio.
\section{A-desoras}
\begin{itemize}
\item {Grp. gram.:loc. adv.}
\end{itemize}
Tarde.
Inopportunamente.
\section{Adesmia}
\begin{itemize}
\item {Grp. gram.:f.}
\end{itemize}
\begin{itemize}
\item {Utilização:Bot.}
\end{itemize}
Falta de soldadura ou de juncção em certos órgãos vegetaes.
\section{Adestra}
\begin{itemize}
\item {Grp. gram.:adv.}
\end{itemize}
\begin{itemize}
\item {Utilização:bras}
\end{itemize}
\begin{itemize}
\item {Utilização:Ant.}
\end{itemize}
\begin{itemize}
\item {Proveniência:(De \textunderscore adestro\textunderscore )}
\end{itemize}
Ao lado.
De refôrço.
\section{Adestradamente}
\begin{itemize}
\item {Grp. gram.:adv.}
\end{itemize}
De modo \textunderscore adestrado\textunderscore .
\section{Adestrado}
\begin{itemize}
\item {Grp. gram.:adj.}
\end{itemize}
Que se adestrou.
Hábil.
Perito.
\section{Adestrador}
\begin{itemize}
\item {Grp. gram.:m.}
\end{itemize}
O que adestra.
\section{Adestramento}
\begin{itemize}
\item {Grp. gram.:m.}
\end{itemize}
Acto ou effeito de \textunderscore adestrar\textunderscore .
\section{Adestrar}
\begin{itemize}
\item {Grp. gram.:v. t.}
\end{itemize}
Tornar destro.
Ensinar.
\section{Adestro}
\begin{itemize}
\item {Grp. gram.:adj.}
\end{itemize}
\begin{itemize}
\item {Proveniência:(Do lat. \textunderscore ad\textunderscore  + \textunderscore dextram\textunderscore )}
\end{itemize}
Que vai ao lado; que acompanha para refôrço ou por luxo:«\textunderscore mandou-lhe dar outro andor, que trazia adestro.\textunderscore »Barros, \textunderscore Déc.\textunderscore  I.
\section{Adeus}
\begin{itemize}
\item {Grp. gram.:adv.}
\end{itemize}
\begin{itemize}
\item {Grp. gram.:M.}
\end{itemize}
\begin{itemize}
\item {Proveniência:(De \textunderscore a\textunderscore  + \textunderscore Deus\textunderscore , n. p., segundo a opinião corrente; mas a fórma antiga \textunderscore ay-Deos\textunderscore  contraria essa opinião)}
\end{itemize}
Fica com Deus.
Deus vá contigo.
Despedida.
Fim.
\section{Adeusar}
\begin{itemize}
\item {Grp. gram.:v. t.}
\end{itemize}
O mesmo que \textunderscore endeusar\textunderscore .
\section{Adeveres}
\begin{itemize}
\item {fónica:vê}
\end{itemize}
\begin{itemize}
\item {Grp. gram.:m. pl.}
\end{itemize}
\begin{itemize}
\item {Utilização:Prov.}
\end{itemize}
\begin{itemize}
\item {Utilização:trasm.}
\end{itemize}
\begin{itemize}
\item {Utilização:Prov.}
\end{itemize}
\begin{itemize}
\item {Utilização:minh.}
\end{itemize}
\begin{itemize}
\item {Proveniência:(De \textunderscore dever\textunderscore )}
\end{itemize}
Attenções, honras, deferências.
Roupa e calçado, que o patrão dá annualmente ao criado de lavoira.
\section{Adherência}
\begin{itemize}
\item {Grp. gram.:f.}
\end{itemize}
Qualidade do que é \textunderscore adherente\textunderscore .
Acto de adherir.
\section{Adherente}
\begin{itemize}
\item {Grp. gram.:m.  e  adj.}
\end{itemize}
\begin{itemize}
\item {Grp. gram.:M.}
\end{itemize}
O que adhere.
Partidário, prosélyto.
\section{Adherir}
\begin{itemize}
\item {Grp. gram.:v. i.}
\end{itemize}
\begin{itemize}
\item {Grp. gram.:V. t.}
\end{itemize}
Estar unido.
Conformar-se, aprovando.
Unir, juntar:«\textunderscore veja-se póde adheri-los ao epicrânio.\textunderscore »Camillo, \textunderscore Sebenta\textunderscore , VII, 19.
\section{Adhesão}
\begin{itemize}
\item {Grp. gram.:f.}
\end{itemize}
\begin{itemize}
\item {Proveniência:(Lat. \textunderscore adhaesio\textunderscore )}
\end{itemize}
Acto de \textunderscore adherir\textunderscore .
Ligação.
Acôrdo.
\section{Adhesivado}
\begin{itemize}
\item {Grp. gram.:adj.}
\end{itemize}
Que tem adhesivo: \textunderscore paus adhesivados\textunderscore .
\section{Adhesivamente}
\begin{itemize}
\item {Grp. gram.:adv.}
\end{itemize}
Com adhesão.
De modo \textunderscore adhesivo\textunderscore .
\section{Adhesivo}
\begin{itemize}
\item {Grp. gram.:m.}
\end{itemize}
\begin{itemize}
\item {Grp. gram.:Adj.}
\end{itemize}
\begin{itemize}
\item {Proveniência:(De \textunderscore adheso\textunderscore )}
\end{itemize}
Emplasto, que adhere á pelle.
Que adhere.
\section{Adheso}
\begin{itemize}
\item {Proveniência:(Lat. \textunderscore adhaesus\textunderscore )}
\end{itemize}
Part. irr. de \textunderscore adherir\textunderscore .
\section{Adhesol}
\begin{itemize}
\item {Grp. gram.:m.}
\end{itemize}
Produto pharmacêutico, que tem o mesmo uso que o collódio.
\section{Adhortar}
\begin{itemize}
\item {Grp. gram.:v. t.}
\end{itemize}
Exhortar.
Excitar, estimular. Cf. Rui Barbosa, \textunderscore Répl.\textunderscore , 157.
\section{Adi}
\begin{itemize}
\item {Grp. gram.:f.}
\end{itemize}
Espécie de palmeira de San-Thomé.
\section{Adiado}
\begin{itemize}
\item {Grp. gram.:adj.}
\end{itemize}
Que se adiou.
Demorado, retardado.
Diz-se do estudante que, não ficando approvado em exame, se considera esperado para outra época de exames.
\section{Adiafa}
\begin{itemize}
\item {Grp. gram.:f.}
\end{itemize}
\begin{itemize}
\item {Utilização:Prov.}
\end{itemize}
\begin{itemize}
\item {Utilização:alg.}
\end{itemize}
\begin{itemize}
\item {Utilização:alent.}
\end{itemize}
Gorgeta, gratificação.
Refeição, que se dá aos trabalhadores, depois da conclusão de uma obra.
Volta dos trabadores da colheita da azeitona.
O mesmo que \textunderscore diafa\textunderscore .
\section{Adiáfano}
\begin{itemize}
\item {Grp. gram.:adj.}
\end{itemize}
\begin{itemize}
\item {Proveniência:(De \textunderscore a\textunderscore priv. + \textunderscore diáphano\textunderscore )}
\end{itemize}
Que não é transparente; opaco.
\section{Adiáforo}
\begin{itemize}
\item {Grp. gram.:adj.}
\end{itemize}
\begin{itemize}
\item {Proveniência:(Gr. \textunderscore adiaphoros\textunderscore )}
\end{itemize}
Accessório, não essencial.
\section{Adiamantado}
\begin{itemize}
\item {Grp. gram.:adj.}
\end{itemize}
Brilhante e duro, como o diamante.
\section{Adiamento}
\begin{itemize}
\item {Grp. gram.:m.}
\end{itemize}
Acto ou effeito de \textunderscore adiar\textunderscore .
\section{Adiantáceas}
\begin{itemize}
\item {Grp. gram.:f. pl.}
\end{itemize}
Grupo de fêtos, que têm por typo o adianto.
\section{Adiante}
\textunderscore adv.\textunderscore  (e der.)
(V. \textunderscore adeante\textunderscore , etc.)
\section{Adianto}
\begin{itemize}
\item {Grp. gram.:m.}
\end{itemize}
\begin{itemize}
\item {Proveniência:(Gr. \textunderscore adiantos\textunderscore )}
\end{itemize}
Planta medicinal, da fam. dos fêtos.
\section{Adiáphano}
\begin{itemize}
\item {Grp. gram.:adj.}
\end{itemize}
\begin{itemize}
\item {Proveniência:(De \textunderscore a\textunderscore priv. + \textunderscore diáphano\textunderscore )}
\end{itemize}
Que não é transparente; opaco.
\section{Adiáphoro}
\begin{itemize}
\item {Grp. gram.:adj.}
\end{itemize}
\begin{itemize}
\item {Proveniência:(Gr. \textunderscore adiaphoros\textunderscore )}
\end{itemize}
Accessório, não essencial.
\section{Adiar}
\begin{itemize}
\item {Grp. gram.:v. t.}
\end{itemize}
Deixar para outro dia; procrastinar.
Demorar.
\section{Adiatésico}
\begin{itemize}
\item {Grp. gram.:adj.}
\end{itemize}
\begin{itemize}
\item {Proveniência:(De \textunderscore a\textunderscore priv. + \textunderscore diáthese\textunderscore )}
\end{itemize}
Que não tem diátese.
\section{Adiathésico}
\begin{itemize}
\item {Grp. gram.:adj.}
\end{itemize}
\begin{itemize}
\item {Proveniência:(De \textunderscore a\textunderscore priv. + \textunderscore diáthese\textunderscore )}
\end{itemize}
Que não tem diáthese.
\section{Adiável}
\begin{itemize}
\item {Grp. gram.:adj.}
\end{itemize}
Que se póde ou se deve \textunderscore adiar\textunderscore .
\section{Adibe}
\begin{itemize}
\item {Grp. gram.:m.}
\end{itemize}
\begin{itemize}
\item {Proveniência:(Do ár. \textunderscore ad-dzib\textunderscore )}
\end{itemize}
Espécie de lobo ou chacal (\textunderscore canis vulpes\textunderscore ).
\section{Adibe}
\begin{itemize}
\item {Grp. gram.:m.}
\end{itemize}
Accrescentamento, addição:«\textunderscore se eu alargasse a esta nota as ensanchas com os adibes que lhe vêm ao justo...\textunderscore »Filinto, XII, 210.
\section{Á-dica}
\begin{itemize}
\item {Grp. gram.:loc. adv.}
\end{itemize}
\begin{itemize}
\item {Utilização:Gír.}
\end{itemize}
Perto.
Donde se vê bem.
(Cp. \textunderscore adicar\textunderscore )
\section{Adiça}
\begin{itemize}
\item {Grp. gram.:f.}
\end{itemize}
\begin{itemize}
\item {Utilização:Ant.}
\end{itemize}
Mina de oiro.
\section{Adição}
\begin{itemize}
\item {Grp. gram.:f.}
\end{itemize}
\begin{itemize}
\item {Proveniência:(Lat. \textunderscore additio\textunderscore )}
\end{itemize}
Acto ou effeito de \textunderscore adir\textunderscore .
Somma.
Parcella.
Successo recente.
Appêndice de uma construcção.
\section{Adicar}
\begin{itemize}
\item {Grp. gram.:v. ir.}
\end{itemize}
\begin{itemize}
\item {Utilização:Gír.}
\end{itemize}
Vêr.
(Or. ind.)
\section{Adicção}
\begin{itemize}
\item {Grp. gram.:v. t.}
\end{itemize}
(V.dicção)
\section{Adiceiro}
\begin{itemize}
\item {Grp. gram.:m.}
\end{itemize}
\begin{itemize}
\item {Utilização:Ant.}
\end{itemize}
O que trabalhava em adiças.
\section{Adicionação}
\begin{itemize}
\item {Grp. gram.:f.}
\end{itemize}
Acto ou effeito de \textunderscore ddicionar\textunderscore .
\section{Adicionador}
\begin{itemize}
\item {Grp. gram.:m.}
\end{itemize}
\begin{itemize}
\item {Proveniência:(De \textunderscore addicionar\textunderscore )}
\end{itemize}
O que adiciona.
\section{Adicional}
\begin{itemize}
\item {Grp. gram.:adj.}
\end{itemize}
\begin{itemize}
\item {Grp. gram.:M.}
\end{itemize}
\begin{itemize}
\item {Proveniência:(Lat. \textunderscore additio\textunderscore , \textunderscore additionis\textunderscore )}
\end{itemize}
Que se adiciona.
Que acresce.
Aquillo que acresce ou se adiciona.
\section{Adicionamento}
\begin{itemize}
\item {Grp. gram.:m.}
\end{itemize}
O mesmo que \textunderscore adicionação\textunderscore .
\section{Adicionar}
\begin{itemize}
\item {Grp. gram.:v. t.}
\end{itemize}
\begin{itemize}
\item {Proveniência:(Do lat. \textunderscore additio\textunderscore , \textunderscore additionis\textunderscore )}
\end{itemize}
Ajuntar.
Acrescentar em nota ou comentário.
\section{Adicionável}
\begin{itemize}
\item {Grp. gram.:adj.}
\end{itemize}
Que se póde adicionar.
\section{Adicto}
\begin{itemize}
\item {Grp. gram.:adj.}
\end{itemize}
\begin{itemize}
\item {Grp. gram.:adj.}
\end{itemize}
\begin{itemize}
\item {Proveniência:(Lat. \textunderscore addictus\textunderscore )}
\end{itemize}
Afeiçoado; dedicado.
Adjunto.
\section{Adido}
\begin{itemize}
\item {Grp. gram.:m.}
\end{itemize}
\begin{itemize}
\item {Grp. gram.:m.}
\end{itemize}
Funccionário, que está junto a um dignitário ou corporação, para auxiliar.
Funccionário de qualquer categoria, que está a mais do quadro respectivo, por exceder o número legal, ou por sêr officialmente dispensado de servir.
\section{Adietar}
\begin{itemize}
\item {Grp. gram.:v. t.}
\end{itemize}
Pôr em dieta.
\section{Adigar}
\begin{itemize}
\item {Grp. gram.:m.}
\end{itemize}
O mesmo que \textunderscore digar\textunderscore .
\section{Adil}
\begin{itemize}
\item {Grp. gram.:m.}
\end{itemize}
\begin{itemize}
\item {Utilização:Prov.}
\end{itemize}
\begin{itemize}
\item {Utilização:trasm.}
\end{itemize}
O mesmo que \textunderscore poisio\textunderscore : \textunderscore a terra ficou de adil\textunderscore .
Terreno de poisio.
\section{Adilado}
\begin{itemize}
\item {Grp. gram.:adj.}
\end{itemize}
\begin{itemize}
\item {Utilização:Prov.}
\end{itemize}
\begin{itemize}
\item {Utilização:trasm.}
\end{itemize}
Diz-se do terreno que ficou de adil.
\section{Adilar}
\begin{itemize}
\item {Grp. gram.:v. t.}
\end{itemize}
\begin{itemize}
\item {Utilização:Prov.}
\end{itemize}
\begin{itemize}
\item {Utilização:trasm.}
\end{itemize}
Deixar de adil (um terreno).
\section{Adimplemento}
\begin{itemize}
\item {Grp. gram.:m.}
\end{itemize}
\begin{itemize}
\item {Proveniência:(Do lat. \textunderscore adimplere\textunderscore )}
\end{itemize}
Acto de completar; complemento.
Preenchimento.
Realização.
\section{Adinheirado}
\begin{itemize}
\item {Grp. gram.:adj.}
\end{itemize}
O mesmo que \textunderscore endinheirado\textunderscore .
\section{Adinho}
\begin{itemize}
\item {Grp. gram.:m.}
\end{itemize}
Filho pequeno do adem. Cf. \textunderscore Peregrinação\textunderscore , c. XCVII.
\section{Adino}
\begin{itemize}
\item {Grp. gram.:m.}
\end{itemize}
Gênero de plantas rubiáceas das regiões quentes da América e da Ásia.
\section{Ádipe}
\begin{itemize}
\item {Grp. gram.:m.}
\end{itemize}
\begin{itemize}
\item {Proveniência:(Lat. \textunderscore adeps\textunderscore , \textunderscore adipis\textunderscore )}
\end{itemize}
Gordura.
\section{Ádipo}
\begin{itemize}
\item {Grp. gram.:m.}
\end{itemize}
O mesmo que \textunderscore ádipe\textunderscore . Cf. Camillo,
\textunderscore Narcót.\textunderscore , II, 210.
\section{Adipocera}
\begin{itemize}
\item {fónica:cé}
\end{itemize}
\begin{itemize}
\item {Grp. gram.:f.}
\end{itemize}
\begin{itemize}
\item {Proveniência:(De \textunderscore ádipe\textunderscore  + \textunderscore cera\textunderscore )}
\end{itemize}
Gordura de cadáver.
\section{Adipocira}
\begin{itemize}
\item {Grp. gram.:f.}
\end{itemize}
O mesmo que \textunderscore adipocera\textunderscore .
\section{Adipociriforme}
\begin{itemize}
\item {Grp. gram.:adj.}
\end{itemize}
Que tem a apparência de adipocira.
\section{Adipociro}
\begin{itemize}
\item {Grp. gram.:m.}
\end{itemize}
O mesmo que \textunderscore adipocera\textunderscore .
\section{Adipoma}
\begin{itemize}
\item {Grp. gram.:m.}
\end{itemize}
\begin{itemize}
\item {Proveniência:(Do lat. \textunderscore adeps\textunderscore )}
\end{itemize}
Tumor gorduroso.
Lipoma.
\section{Adipose}
\begin{itemize}
\item {Grp. gram.:f.}
\end{itemize}
\begin{itemize}
\item {Utilização:Med.}
\end{itemize}
\begin{itemize}
\item {Proveniência:(De \textunderscore ádipe\textunderscore )}
\end{itemize}
Estado mórbido, caracterizado por acumulação de gordura no tecido cellular subcutâneo.
\section{Adiposo}
\begin{itemize}
\item {Grp. gram.:adj.}
\end{itemize}
\begin{itemize}
\item {Proveniência:(De \textunderscore ádipe\textunderscore )}
\end{itemize}
Que tem gordura.
\section{Adipsia}
\begin{itemize}
\item {Grp. gram.:f.}
\end{itemize}
\begin{itemize}
\item {Proveniência:(Do gr. \textunderscore a\textunderscore priv. + \textunderscore dipsos\textunderscore )}
\end{itemize}
Privação de appetite de liquidos.
\section{Adir}
\begin{itemize}
\item {Grp. gram.:v. t.}
\end{itemize}
\begin{itemize}
\item {Proveniência:(Lat. \textunderscore adire\textunderscore )}
\end{itemize}
Entrar na posse de (herança).
\section{Adir}
\begin{itemize}
\item {Grp. gram.:v. t.}
\end{itemize}
\begin{itemize}
\item {Grp. gram.:V. p.}
\end{itemize}
Accrescentar.
Ajuntar; aggregar.
Ajuntar-se, ligar-se. Cf. Camillo, \textunderscore O Senh. do Paço de Nin.\textunderscore , 152.
\section{Adita}
\begin{itemize}
\item {Grp. gram.:f.}
\end{itemize}
\begin{itemize}
\item {Utilização:Prov.}
\end{itemize}
\begin{itemize}
\item {Utilização:minh.}
\end{itemize}
Agrado, sympathia.
\section{Aditamento}
\begin{itemize}
\item {Grp. gram.:m.}
\end{itemize}
Acto de \textunderscore adittar\textunderscore .
Aquillo que se aditou.
\section{Aditar}
\begin{itemize}
\item {Grp. gram.:v. t.}
\end{itemize}
Causar a dita de; tornar feliz.
\section{Aditar}
\begin{itemize}
\item {Grp. gram.:v. t.}
\end{itemize}
\begin{itemize}
\item {Proveniência:(De \textunderscore ádito\textunderscore ^1)}
\end{itemize}
Entrar.
\section{Aditar}
\begin{itemize}
\item {Proveniência:(Do lat. \textunderscore additus\textunderscore , de \textunderscore addere\textunderscore )}
\end{itemize}
\textunderscore v. t.\textunderscore  (e der.)
O mesmo que \textunderscore adicionar\textunderscore .
\section{Ádito}
\begin{itemize}
\item {Grp. gram.:m.}
\end{itemize}
\begin{itemize}
\item {Proveniência:(Lat. \textunderscore aditus\textunderscore )}
\end{itemize}
Entrada.
Accesso, aproximação.
\section{Adival}
\begin{itemize}
\item {Grp. gram.:m.}
\end{itemize}
\begin{itemize}
\item {Utilização:Prov.}
\end{itemize}
Antiga medida agrária, de 12 braças.
Corda de carro.
\section{Adivinha}
\begin{itemize}
\item {Grp. gram.:f.}
\end{itemize}
Coisa para se adivinhar; enigma.
\section{Adivinha}
\begin{itemize}
\item {Grp. gram.:f.}
\end{itemize}
Mulher, de quem se diz que adivinha o futuro e as coisas occultas.
(Fem. de \textunderscore adivinho\textunderscore )
\section{Adivinhação}
\begin{itemize}
\item {Grp. gram.:f.}
\end{itemize}
Acto de \textunderscore adivinhar\textunderscore .
Adivinha.
\section{Adivinhadeira}
\begin{itemize}
\item {Grp. gram.:f.}
\end{itemize}
(V. \textunderscore adivinha\textunderscore ^2)
\section{Adivinhadeiro}
\begin{itemize}
\item {Grp. gram.:m.}
\end{itemize}
(V.adivinho)
\section{Adivinhador}
\begin{itemize}
\item {Grp. gram.:m.}
\end{itemize}
O mesmo que \textunderscore adivinho\textunderscore .
\section{Adivinhamento}
\begin{itemize}
\item {Grp. gram.:m.}
\end{itemize}
(V.adivinhação)
\section{Adivinhança}
\begin{itemize}
\item {Grp. gram.:f.}
\end{itemize}
\begin{itemize}
\item {Utilização:Ant.}
\end{itemize}
O mesmo que \textunderscore adivinhação\textunderscore .
\section{Adivinhão}
\begin{itemize}
\item {Grp. gram.:m.}
\end{itemize}
(V.adivinho)
\section{Adivinhar}
\begin{itemize}
\item {Grp. gram.:v. t.}
\end{itemize}
\begin{itemize}
\item {Proveniência:(Lat. \textunderscore divinare\textunderscore )}
\end{itemize}
Prever (o futuro), por meio de sortilégio ou de falsas sciências.
Conhecer por meios sobrenaturaes.
Decifrar.
Conjecturar.
Interpretar.
\section{Adivinho}
\begin{itemize}
\item {Grp. gram.:m.}
\end{itemize}
\begin{itemize}
\item {Proveniência:(De \textunderscore adivinhar\textunderscore )}
\end{itemize}
Homem, de quem se diz que adivinha.
\section{Adjacência}
\begin{itemize}
\item {Grp. gram.:f.}
\end{itemize}
Estado, qualidade do que é \textunderscore adjacente\textunderscore .
\section{Adjacente}
\begin{itemize}
\item {Grp. gram.:adj.}
\end{itemize}
\begin{itemize}
\item {Utilização:Geom.}
\end{itemize}
\begin{itemize}
\item {Proveniência:(Lat. \textunderscore adjacens\textunderscore )}
\end{itemize}
Que está contíguo ou junto.
Diz-se dos dois ângulos contíguos, de entre os quatro, formados por duas rectas que se cortam.
\section{Adjazer}
\begin{itemize}
\item {Grp. gram.:v. i.}
\end{itemize}
\begin{itemize}
\item {Proveniência:(Lat. \textunderscore adjacere\textunderscore )}
\end{itemize}
Estar próximo ou junto. Cf. Castilho, \textunderscore Fastos\textunderscore , III, 113.
\section{Adjecção}
\begin{itemize}
\item {Grp. gram.:f.}
\end{itemize}
\begin{itemize}
\item {Proveniência:(Lat. \textunderscore adjectio\textunderscore )}
\end{itemize}
Addição.
\section{Adjectivação}
\begin{itemize}
\item {Grp. gram.:f.}
\end{itemize}
Acto ou effeito de \textunderscore adjectivar\textunderscore .
\section{Adjectivadamente}
\begin{itemize}
\item {Grp. gram.:adv.}
\end{itemize}
De modo \textunderscore adjectivado\textunderscore .
\section{Adjectivado}
\begin{itemize}
\item {Grp. gram.:adj.}
\end{itemize}
Diz-se do estilo ou da linguagem em que abundam os adjectivos ou termos empregados adjectivamente.
\section{Adjectival}
\begin{itemize}
\item {Grp. gram.:adj.}
\end{itemize}
\begin{itemize}
\item {Utilização:P. us.}
\end{itemize}
Relativo ao adjectivo.
\section{Adjectivamente}
\begin{itemize}
\item {Grp. gram.:adv.}
\end{itemize}
Á maneira de \textunderscore adjectivo\textunderscore .
\section{Adjectivar}
\begin{itemize}
\item {Grp. gram.:v. t.}
\end{itemize}
Acompanhar de adjectivo.
Qualificar.
Tomar como adjectivo.
Ornar (uma phrase, um periodo) com grande cópia de adjectivos.
\section{Adjectivo}
\begin{itemize}
\item {Grp. gram.:m.}
\end{itemize}
\begin{itemize}
\item {Utilização:Gram.}
\end{itemize}
\begin{itemize}
\item {Grp. gram.:Adj.}
\end{itemize}
\begin{itemize}
\item {Proveniência:(Lat. \textunderscore adjectivus\textunderscore )}
\end{itemize}
Palavra, que se junta ou se refere a outra, para a qualificar ou determinar, indicando louvor, vitupério, meio, ou accidente.
Que se junta.
Adjecto.
Relativo ao adjectivo.
\section{Adjecto}
\begin{itemize}
\item {Grp. gram.:adj.}
\end{itemize}
\begin{itemize}
\item {Proveniência:(Lat. \textunderscore adjectus\textunderscore )}
\end{itemize}
Accrescentado.
\section{Adjudicação}
\begin{itemize}
\item {Grp. gram.:f.}
\end{itemize}
Acto ou effeito de \textunderscore adjudicar\textunderscore .
\section{Adjudicador}
\begin{itemize}
\item {Grp. gram.:m.}
\end{itemize}
O que adjudica.
\section{Adjudicar}
\begin{itemize}
\item {Grp. gram.:v. t.}
\end{itemize}
\begin{itemize}
\item {Proveniência:(Lat. \textunderscore adjudicare\textunderscore )}
\end{itemize}
Dar, por sentença.
Declarar judicialmente que (uma coisa) pertence a alguém.
\section{Adjudicatário}
\begin{itemize}
\item {Grp. gram.:m.}
\end{itemize}
Aquelle, a quem alguma coisa se adjudica.
\section{Adjudicativo}
\begin{itemize}
\item {Grp. gram.:adj.}
\end{itemize}
O mesmo que \textunderscore adjudicatório\textunderscore .
\section{Adjudicatório}
\begin{itemize}
\item {Grp. gram.:adj.}
\end{itemize}
Diz-se do acto ou sentença que adjudica.
\section{Adjudoiro}
\begin{itemize}
\item {Grp. gram.:m.}
\end{itemize}
\begin{itemize}
\item {Utilização:Ant.}
\end{itemize}
O mesmo que \textunderscore adjutório\textunderscore .
\section{Adjunção}
\begin{itemize}
\item {Grp. gram.:f.}
\end{itemize}
\begin{itemize}
\item {Proveniência:(Lat. \textunderscore adjunctio\textunderscore )}
\end{itemize}
Acto ou effeito de unir ou de associar.
\section{Adjuncção}
\begin{itemize}
\item {Grp. gram.:f.}
\end{itemize}
\begin{itemize}
\item {Proveniência:(Lat. \textunderscore adjunctio\textunderscore )}
\end{itemize}
Acto ou effeito de unir ou de associar.
\section{Adjuncto}
\begin{itemize}
\item {Grp. gram.:adj.}
\end{itemize}
\begin{itemize}
\item {Grp. gram.:M.}
\end{itemize}
\begin{itemize}
\item {Utilização:Prov.}
\end{itemize}
\begin{itemize}
\item {Utilização:trasm.}
\end{itemize}
\begin{itemize}
\item {Utilização:Bras}
\end{itemize}
\begin{itemize}
\item {Proveniência:(Lat. \textunderscore adjunctus\textunderscore )}
\end{itemize}
Unido.
Associado.
Contíguo.
O que é aggregado, associado.
Auxiliar.
Ajuntamento, reunião.
O mesmo que \textunderscore muxirão\textunderscore .
\section{Adjungir}
\begin{itemize}
\item {Grp. gram.:v. t.}
\end{itemize}
\begin{itemize}
\item {Proveniência:(Lat. \textunderscore adjungere\textunderscore )}
\end{itemize}
Reunir, associar. Cf. Castilho, \textunderscore Fastos\textunderscore , II, 159.
\section{Adjunto}
\begin{itemize}
\item {Grp. gram.:adj.}
\end{itemize}
\begin{itemize}
\item {Grp. gram.:M.}
\end{itemize}
\begin{itemize}
\item {Utilização:Prov.}
\end{itemize}
\begin{itemize}
\item {Utilização:trasm.}
\end{itemize}
\begin{itemize}
\item {Utilização:Bras}
\end{itemize}
\begin{itemize}
\item {Proveniência:(Lat. \textunderscore adjunctus\textunderscore )}
\end{itemize}
Unido.
Associado.
Contíguo.
O que é aggregado, associado.
Auxiliar.
Ajuntamento, reunião.
O mesmo que \textunderscore muxirão\textunderscore .
\section{Adjuração}
\begin{itemize}
\item {Grp. gram.:f.}
\end{itemize}
Acto ou effeito de \textunderscore adjurar\textunderscore .
\section{Adjurador}
\begin{itemize}
\item {Grp. gram.:m.}
\end{itemize}
O que adjura.
\section{Adjurar}
\begin{itemize}
\item {Grp. gram.:v. t.}
\end{itemize}
\begin{itemize}
\item {Proveniência:(Lat. \textunderscore adjurare\textunderscore )}
\end{itemize}
Esconjurar.
Exorcismar.
Pedir com instância.
\section{Adjutor}
\begin{itemize}
\item {Grp. gram.:m.}
\end{itemize}
\begin{itemize}
\item {Proveniência:(Lat. \textunderscore adjutor\textunderscore )}
\end{itemize}
Ajudante.
\section{Adjutório}
\begin{itemize}
\item {Grp. gram.:m.}
\end{itemize}
\begin{itemize}
\item {Proveniência:(Lat. \textunderscore adjutorium\textunderscore )}
\end{itemize}
Ajuda.
\section{Adjuvante}
\begin{itemize}
\item {Grp. gram.:adj.}
\end{itemize}
\begin{itemize}
\item {Proveniência:(Lat. \textunderscore adjuvans\textunderscore )}
\end{itemize}
Que ajuda.
\section{Adlegação}
\begin{itemize}
\item {Grp. gram.:f.}
\end{itemize}
\begin{itemize}
\item {Proveniência:(Do lat. \textunderscore ad\textunderscore  + \textunderscore legatio\textunderscore )}
\end{itemize}
Direito, que os antigos Estados germânicos tinham, de delegar ministros que, juntamente com os do imperador, tratassem dos negócios de interesse commum.
\section{Adligado}
\begin{itemize}
\item {Grp. gram.:adj.}
\end{itemize}
\begin{itemize}
\item {Proveniência:(Do lat. \textunderscore ad\textunderscore  + \textunderscore ligatus\textunderscore )}
\end{itemize}
Diz-se, em Botânica, da planta fixada por appêndices.
\section{Adligante}
\begin{itemize}
\item {Grp. gram.:adj.}
\end{itemize}
\begin{itemize}
\item {Proveniência:(Do lat. \textunderscore ad\textunderscore  + \textunderscore ligans\textunderscore )}
\end{itemize}
Diz-se da raiz que fixa um vegetal parasito ao corpo em que vive.
\section{Adlúmia}
\begin{itemize}
\item {Grp. gram.:f.}
\end{itemize}
Espécie de planta juncácea.
\section{Adlúmio}
\begin{itemize}
\item {Grp. gram.:m.}
\end{itemize}
O mesmo que \textunderscore adlúmia\textunderscore .
\section{Admena}
\begin{itemize}
\item {Grp. gram.:f.}
\end{itemize}
O mesmo que admene.
\section{Admene}
\begin{itemize}
\item {Grp. gram.:m.}
\end{itemize}
\begin{itemize}
\item {Utilização:Ant.}
\end{itemize}
Rua, ladeada de árvores frondosas.
Alameda.
\section{Adminiculante}
\begin{itemize}
\item {Grp. gram.:adj.}
\end{itemize}
Que serve de adminículo.
\section{Adminicular}
\begin{itemize}
\item {Grp. gram.:adj.}
\end{itemize}
\begin{itemize}
\item {Grp. gram.:V. t.}
\end{itemize}
O mesmo que \textunderscore adminiculante\textunderscore .
Subsidiar.
\section{Adminiculativo}
\begin{itemize}
\item {Grp. gram.:adj.}
\end{itemize}
Que serve para adminicular.
\section{Adminículo}
\begin{itemize}
\item {Grp. gram.:m.}
\end{itemize}
\begin{itemize}
\item {Grp. gram.:M. pl.}
\end{itemize}
\begin{itemize}
\item {Proveniência:(Lat. \textunderscore adminiculum\textunderscore )}
\end{itemize}
Subsídio.
Auxílio, apoio.
Enfeites, que orlam uma medalha.
\section{Administração}
\begin{itemize}
\item {Grp. gram.:f.}
\end{itemize}
\begin{itemize}
\item {Proveniência:(Lat. \textunderscore administratio\textunderscore )}
\end{itemize}
Gerência de negócios.
Acção de administrar.
\section{Administracionalizar}
\begin{itemize}
\item {Grp. gram.:v. t.}
\end{itemize}
\begin{itemize}
\item {Utilização:Neol.}
\end{itemize}
\begin{itemize}
\item {Proveniência:(Fr. \textunderscore administrationaliser\textunderscore )}
\end{itemize}
Sujeitar aos regulamentos administrativos, ou a um systema de administração.
\section{Administradeira}
\begin{itemize}
\item {Grp. gram.:f.}
\end{itemize}
\begin{itemize}
\item {Utilização:Bras}
\end{itemize}
Mulher que administra.
\section{Administrador}
\begin{itemize}
\item {Grp. gram.:m.}
\end{itemize}
\begin{itemize}
\item {Proveniência:(Lat. \textunderscore administrator\textunderscore )}
\end{itemize}
O que administra.
\section{Administrante}
\begin{itemize}
\item {Grp. gram.:adj.}
\end{itemize}
\begin{itemize}
\item {Proveniência:(Lat. \textunderscore administrans\textunderscore )}
\end{itemize}
Que administra.
\section{Administrar}
\begin{itemize}
\item {Grp. gram.:v. t.}
\end{itemize}
\begin{itemize}
\item {Proveniência:(Lat. \textunderscore administrare\textunderscore )}
\end{itemize}
Gerir (negócios).
Dar a alguém (medicamentos).
Ministrar; applicar.
\section{Administrativamente}
\begin{itemize}
\item {Grp. gram.:adv.}
\end{itemize}
De modo \textunderscore administrativo\textunderscore .
\section{Administrativo}
\begin{itemize}
\item {Grp. gram.:adj.}
\end{itemize}
\begin{itemize}
\item {Proveniência:(Lat. \textunderscore administrativus\textunderscore )}
\end{itemize}
Relativo a administração: \textunderscore autoridades administrativas\textunderscore .
\section{Admirábil}
\begin{itemize}
\item {Grp. gram.:adj.}
\end{itemize}
(V.admirável)
\section{Admirabilidade}
\begin{itemize}
\item {Grp. gram.:f.}
\end{itemize}
\begin{itemize}
\item {Proveniência:(Lat. \textunderscore admirabilitas\textunderscore )}
\end{itemize}
Qualidade do que é admirável.
\section{Admiração}
\begin{itemize}
\item {Grp. gram.:f.}
\end{itemize}
\begin{itemize}
\item {Proveniência:(Lat. \textunderscore admiratio\textunderscore )}
\end{itemize}
Acto de admirar.
Espanto.
\section{Admiradamente}
\begin{itemize}
\item {Grp. gram.:adv.}
\end{itemize}
De modo \textunderscore admirado\textunderscore .
Com admiração.
\section{Admirado}
\begin{itemize}
\item {Grp. gram.:adj.}
\end{itemize}
Que sente admiração: \textunderscore fiquei admirado\textunderscore .
Que é objecto de admiração: \textunderscore um trabalho admirado\textunderscore .
\section{Admirador}
\begin{itemize}
\item {Grp. gram.:m.}
\end{itemize}
O que admira.
\section{Admirando}
\begin{itemize}
\item {Grp. gram.:adj.}
\end{itemize}
Que merece admiração:«\textunderscore um novo rio admirando\textunderscore ». Andrade Caminha.
\section{Admirante}
\begin{itemize}
\item {Grp. gram.:adj.}
\end{itemize}
\begin{itemize}
\item {Proveniência:(Lat. \textunderscore admirans\textunderscore )}
\end{itemize}
Que admira.
\section{Admirar}
\begin{itemize}
\item {Grp. gram.:v. t.}
\end{itemize}
\begin{itemize}
\item {Grp. gram.:v. p.}
\end{itemize}
\begin{itemize}
\item {Proveniência:(Lat. \textunderscore admirari\textunderscore )}
\end{itemize}
Vêr com espanto.
Causar espanto a.
Sentir espanto, admiração.
\section{Admirativamente}
\begin{itemize}
\item {Grp. gram.:adv.}
\end{itemize}
De modo \textunderscore admirativo\textunderscore .
\section{Admirativo}
\begin{itemize}
\item {Grp. gram.:adj.}
\end{itemize}
Que envolve admiração.
Cheio de admiração.
\section{Admirável}
\begin{itemize}
\item {Grp. gram.:adj.}
\end{itemize}
\begin{itemize}
\item {Proveniência:(Lat. \textunderscore admirabilis\textunderscore )}
\end{itemize}
Que é digno de admiração.
\section{Admiravelmente}
\begin{itemize}
\item {Grp. gram.:adv.}
\end{itemize}
De modo \textunderscore admirável\textunderscore .
\section{Admissão}
\begin{itemize}
\item {Grp. gram.:f.}
\end{itemize}
\begin{itemize}
\item {Proveniência:(Lat. \textunderscore admissio\textunderscore )}
\end{itemize}
Acto ou effeito de admittir.
\section{Admissibilidade}
\begin{itemize}
\item {Grp. gram.:f.}
\end{itemize}
\begin{itemize}
\item {Proveniência:(Do lat. \textunderscore admissibilis\textunderscore )}
\end{itemize}
Qualidade do que é admissivel.
\section{Admissível}
\begin{itemize}
\item {Grp. gram.:adj.}
\end{itemize}
\begin{itemize}
\item {Proveniência:(Lat. \textunderscore admissibilis\textunderscore )}
\end{itemize}
Que se póde admittir.
\section{Admitir}
\begin{itemize}
\item {Grp. gram.:v. t.}
\end{itemize}
\begin{itemize}
\item {Proveniência:(Lat. \textunderscore admittere\textunderscore )}
\end{itemize}
Receber; deixar entrar.
Concordar com; concordar em.
\section{Admittir}
\begin{itemize}
\item {Grp. gram.:v. t.}
\end{itemize}
\begin{itemize}
\item {Proveniência:(Lat. \textunderscore admittere\textunderscore )}
\end{itemize}
Receber; deixar entrar.
Concordar com; concordar em.
\section{Admoestação}
\begin{itemize}
\item {fónica:mo-es}
\end{itemize}
\begin{itemize}
\item {Grp. gram.:f.}
\end{itemize}
Acto ou effeito de \textunderscore admoestar\textunderscore .
\section{Admoestador}
\begin{itemize}
\item {fónica:mo-es}
\end{itemize}
\begin{itemize}
\item {Grp. gram.:m.  e  adj.}
\end{itemize}
O que admoésta.
\section{Admoestamento}
\begin{itemize}
\item {fónica:mo-es}
\end{itemize}
\begin{itemize}
\item {Grp. gram.:m.}
\end{itemize}
Acto ou effeito de \textunderscore admoestar\textunderscore ; admoestação.
\section{Admoestar}
\begin{itemize}
\item {fónica:mo-es}
\end{itemize}
\begin{itemize}
\item {Grp. gram.:v. t.}
\end{itemize}
\begin{itemize}
\item {Proveniência:(Do lat. \textunderscore ad\textunderscore  + \textunderscore molestare\textunderscore ?)}
\end{itemize}
Reprehender levemente.
Advertir de uma falta.
\section{Admoestatório}
\begin{itemize}
\item {fónica:mo-es}
\end{itemize}
\begin{itemize}
\item {Grp. gram.:adj.}
\end{itemize}
Que envolve admoestação.
Próprio para admoestar.
\section{Admonenda}
\begin{itemize}
\item {Grp. gram.:f.}
\end{itemize}
Admoestação, reprehensão leve. Cf. Cortesão, \textunderscore Subsídios\textunderscore .
\section{Admonição}
\begin{itemize}
\item {Grp. gram.:f.}
\end{itemize}
\begin{itemize}
\item {Proveniência:(Lat. \textunderscore admonitio\textunderscore )}
\end{itemize}
Admoestação.
\section{Admonitor}
\begin{itemize}
\item {Grp. gram.:m.  e  adj.}
\end{itemize}
\begin{itemize}
\item {Proveniência:(Lat. \textunderscore admonitor\textunderscore )}
\end{itemize}
Admoestador.
\section{Admonitório}
\begin{itemize}
\item {Grp. gram.:m.}
\end{itemize}
\begin{itemize}
\item {Grp. gram.:Adj.}
\end{itemize}
\begin{itemize}
\item {Proveniência:(Lat. \textunderscore admonitorium\textunderscore )}
\end{itemize}
Admoestação.
Que serve de admoestar.
\section{Adnasal}
\begin{itemize}
\item {Grp. gram.:adj.}
\end{itemize}
\begin{itemize}
\item {Utilização:Anat.}
\end{itemize}
\begin{itemize}
\item {Proveniência:(De \textunderscore ad...\textunderscore  + \textunderscore nasal\textunderscore )}
\end{itemize}
Diz-se de uma das peças elementares de uma das vértebras cephálicas.
\section{Adnascente}
\begin{itemize}
\item {Grp. gram.:adj.}
\end{itemize}
\begin{itemize}
\item {Utilização:Bot.}
\end{itemize}
\begin{itemize}
\item {Proveniência:(De \textunderscore ad...\textunderscore  + \textunderscore nascente\textunderscore )}
\end{itemize}
Diz-se do rebento, que se apresenta na axilla dos cascos periphéricos de um bolbo.
\section{Adnata}
\begin{itemize}
\item {Grp. gram.:f.}
\end{itemize}
\begin{itemize}
\item {Utilização:Anat.}
\end{itemize}
Túnica exterior do globo ocular.
(Fem. de \textunderscore adnato\textunderscore )
\section{Adnato}
\begin{itemize}
\item {Grp. gram.:adj.}
\end{itemize}
\begin{itemize}
\item {Proveniência:(Lat. \textunderscore adnatus\textunderscore )}
\end{itemize}
Ligado a alguma coisa, de que parece fazer parte.
\section{Adnex...}
O mesmo que \textunderscore annex...\textunderscore 
\section{Adnominação}
\begin{itemize}
\item {Grp. gram.:f.}
\end{itemize}
\begin{itemize}
\item {Proveniência:(De \textunderscore ad...\textunderscore  + \textunderscore nominação\textunderscore )}
\end{itemize}
(V.paranomásia)
\section{Adnotação}
\begin{itemize}
\item {Grp. gram.:f.}
\end{itemize}
Resposta do Pontífice a uma súpplica, resposta que consiste só numa assignatura.
(Cp. \textunderscore annotação\textunderscore )
\section{Adnotar}
\textunderscore v. t.\textunderscore  (e der.)
O mesmo que \textunderscore annotar\textunderscore , etc.
\section{Adnumerar}
\begin{itemize}
\item {Grp. gram.:v. t.}
\end{itemize}
\begin{itemize}
\item {Utilização:Des.}
\end{itemize}
(V. enumerar)
\section{Adoba}
\begin{itemize}
\item {fónica:dô}
\end{itemize}
\begin{itemize}
\item {Grp. gram.:f.}
\end{itemize}
O mesmo que \textunderscore adobe\textunderscore .
\section{Adoba}
\begin{itemize}
\item {fónica:dô}
\end{itemize}
\begin{itemize}
\item {Grp. gram.:f.}
\end{itemize}
\begin{itemize}
\item {Utilização:Ant.}
\end{itemize}
O mesmo que \textunderscore algema\textunderscore .
\section{Adobar}
\begin{itemize}
\item {Grp. gram.:v. t.}
\end{itemize}
\begin{itemize}
\item {Utilização:Ant.}
\end{itemize}
O mesmo que \textunderscore algemar\textunderscore .
(De \textunderscore adôba\textunderscore ^2).
\section{Adobar}
\textunderscore v. t.\textunderscore  (?)«\textunderscore ...logo que adobou as naus... mandou... levantar âncora\textunderscore ». Filinto, \textunderscore D. Man.\textunderscore  I, 74.
\section{Adobe}
\begin{itemize}
\item {fónica:dô}
\end{itemize}
\begin{itemize}
\item {Grp. gram.:m.}
\end{itemize}
\begin{itemize}
\item {Proveniência:(Do ár. \textunderscore at-tob\textunderscore )}
\end{itemize}
Tijolo cru.
Seixo arredondado do leito dos rios.
\section{Adobo}
\begin{itemize}
\item {fónica:dô}
\end{itemize}
\begin{itemize}
\item {Grp. gram.:m.}
\end{itemize}
O mesmo que \textunderscore adobe\textunderscore .
\section{Adoçamento}
\begin{itemize}
\item {Grp. gram.:m.}
\end{itemize}
Acto de \textunderscore adoçar\textunderscore .
Canelura, que liga uma parede á saliência de uma moldura.
Moldura côncava, que liga um plintho a uma cornija.
\section{Adoçante}
\begin{itemize}
\item {Grp. gram.:adj.}
\end{itemize}
\begin{itemize}
\item {Grp. gram.:M.}
\end{itemize}
Que adoça.
Medicamento adoçante.
\section{Adoçar}
\begin{itemize}
\item {Grp. gram.:v. t.}
\end{itemize}
Tornar doce.
Abrandar.
Aplanar.
Atenuar.
Polir.
Tornar dúctil (o metal), por meio do fogo.
\section{Adocicado}
\begin{itemize}
\item {Grp. gram.:adj.}
\end{itemize}
\begin{itemize}
\item {Proveniência:(De \textunderscore adocicar\textunderscore )}
\end{itemize}
Um tanto doce.
Suave, mas um pouco affectado: \textunderscore palavras adocicadas\textunderscore .
\section{Adocicamento}
\begin{itemize}
\item {Grp. gram.:m.}
\end{itemize}
Acto ou effeito de \textunderscore adocicar\textunderscore .
\section{Adocicar}
\begin{itemize}
\item {Grp. gram.:v. t.}
\end{itemize}
Adoçar um pouco.
\section{Adoecer}
\begin{itemize}
\item {fónica:do-e}
\end{itemize}
\begin{itemize}
\item {Grp. gram.:v. i.}
\end{itemize}
\begin{itemize}
\item {Grp. gram.:V. t.}
\end{itemize}
\begin{itemize}
\item {Proveniência:(Do lat. \textunderscore dolescere\textunderscore )}
\end{itemize}
Tornar-se doente; enfermar.
Tornar doente.
\section{Adoecimento}
\begin{itemize}
\item {fónica:do-e}
\end{itemize}
\begin{itemize}
\item {Grp. gram.:m.}
\end{itemize}
Acto ou effeito de \textunderscore adoecer\textunderscore .
\section{Adoentado}
\begin{itemize}
\item {Grp. gram.:adj.}
\end{itemize}
Um tanto doente.
Fraco, abatido, (falando-se de alguém ou de animaes).
\section{Adoentar}
\begin{itemize}
\item {Grp. gram.:v. t.}
\end{itemize}
Tornar doente, ou um pouco doente.
\section{Adoidado}
\begin{itemize}
\item {Grp. gram.:adj.}
\end{itemize}
\begin{itemize}
\item {Proveniência:(De \textunderscore adoidar\textunderscore )}
\end{itemize}
Um tanto doido.
Imprudente, leviano.
\section{Adoidar}
\begin{itemize}
\item {Grp. gram.:v. t.}
\end{itemize}
Tornar doido, ou um pouco doido.
\section{Adoito}
\begin{itemize}
\item {Grp. gram.:m.}
\end{itemize}
\begin{itemize}
\item {Utilização:Ant.}
\end{itemize}
Hábito, costume.
(Talvez do lat. \textunderscore adductus\textunderscore )
\section{Adolescência}
\begin{itemize}
\item {Grp. gram.:f.}
\end{itemize}
\begin{itemize}
\item {Proveniência:(Lat. \textunderscore adolescentia\textunderscore )}
\end{itemize}
Período na vida humana, entre a puberdade e a virilidade.
\section{Adolescente}
\begin{itemize}
\item {Grp. gram.:m.  e  adj.}
\end{itemize}
\begin{itemize}
\item {Proveniência:(Lat. \textunderscore adolescens\textunderscore )}
\end{itemize}
O que está na adolescência.
\section{Adolescêntulo}
\begin{itemize}
\item {Grp. gram.:m.}
\end{itemize}
\begin{itemize}
\item {Utilização:Des.}
\end{itemize}
\begin{itemize}
\item {Proveniência:(Lat. \textunderscore adolescentulus\textunderscore )}
\end{itemize}
Rapazinho.
\section{Adolescer}
\begin{itemize}
\item {Grp. gram.:v. i.}
\end{itemize}
\begin{itemize}
\item {Proveniência:(Lat. \textunderscore adolescere\textunderscore )}
\end{itemize}
Entrar na adolescência.
Desenvolver-se.
\section{Adoli}
\begin{itemize}
\item {Grp. gram.:m.}
\end{itemize}
Planta do Malabar.
\section{Adólia}
\begin{itemize}
\item {Grp. gram.:f.}
\end{itemize}
Planta rhamnácea; o mesmo que \textunderscore adoli\textunderscore ?
\section{Adólio}
\begin{itemize}
\item {Grp. gram.:m.}
\end{itemize}
Gênero de insectos lepidópteros diurnos.
\section{Adolorado}
\begin{itemize}
\item {Grp. gram.:adj.}
\end{itemize}
\begin{itemize}
\item {Utilização:Des.}
\end{itemize}
Que tem dores: \textunderscore a Virgem adolorada\textunderscore .
\section{Adomingado}
\begin{itemize}
\item {Grp. gram.:adj.}
\end{itemize}
Vestido de roupas domingueiras.
\section{Adomingar-se}
\begin{itemize}
\item {Grp. gram.:v. p.}
\end{itemize}
\begin{itemize}
\item {Proveniência:(De \textunderscore domingo\textunderscore )}
\end{itemize}
Vestir-se de roupas domingueiras.
\section{Adonai}
\begin{itemize}
\item {Grp. gram.:m.}
\end{itemize}
\begin{itemize}
\item {Proveniência:(Lat. \textunderscore Adonai\textunderscore )}
\end{itemize}
Um dos nomes que os Hebreus davam á Divindade.
Um dos nomes que os Judeus davam á divindade.
\section{Adonairar}
\begin{itemize}
\item {Grp. gram.:v. t.}
\end{itemize}
Dar donaire a. Cf. Arn. Gama, \textunderscore Ultima Dona\textunderscore , 39.; \textunderscore Idem\textunderscore , \textunderscore Segr. do Abbade\textunderscore , 6, 76, 96, 118.
\section{Adonde}
\begin{itemize}
\item {Grp. gram.:adv.}
\end{itemize}
\begin{itemize}
\item {Utilização:Ant.}
\end{itemize}
\begin{itemize}
\item {Utilização:Pop.}
\end{itemize}
(V.aonde)
O mesmo que \textunderscore onde\textunderscore .
\section{Adónico}
\begin{itemize}
\item {Grp. gram.:adj.}
\end{itemize}
O mesmo que \textunderscore adónio\textunderscore .
\section{Adónida}
\begin{itemize}
\item {Grp. gram.:f.}
\end{itemize}
Planta ranunculácea.
\section{Adonidina}
\begin{itemize}
\item {Grp. gram.:f.}
\end{itemize}
Princípio activo vegetal, empregado como tónico cardíaco.
\section{Adónio}
\begin{itemize}
\item {Grp. gram.:adj.}
\end{itemize}
\begin{itemize}
\item {Proveniência:(Lat. \textunderscore adonius\textunderscore )}
\end{itemize}
Diz-se do verso, em que entra um dáctylo e um espondeu.
\section{Adónis}
\begin{itemize}
\item {Grp. gram.:m.}
\end{itemize}
\begin{itemize}
\item {Proveniência:(Do gr. \textunderscore Adonis\textunderscore ^2, n. p.)}
\end{itemize}
Rapaz galante e presumido.
Planta ranunculácea, de aspecto elegante.
\section{Adonisar}
\begin{itemize}
\item {Grp. gram.:v. t.}
\end{itemize}
\begin{itemize}
\item {Proveniência:(De \textunderscore adónis\textunderscore )}
\end{itemize}
Tornar galante; embellezar. Cf. Camillo, \textunderscore Canc. al.\textunderscore , 342.
\section{Adónis-da-Itália}
\begin{itemize}
\item {Grp. gram.:m.}
\end{itemize}
Planta medicinal, ranunculácea, (\textunderscore adonis vernalis\textunderscore ).
\section{Adopção}
\begin{itemize}
\item {Grp. gram.:f.}
\end{itemize}
\begin{itemize}
\item {Proveniência:(Lat. \textunderscore adoptio\textunderscore )}
\end{itemize}
Acto ou effeito de adoptar.
\section{Adoperar}
\begin{itemize}
\item {Grp. gram.:v. t.}
\end{itemize}
\begin{itemize}
\item {Proveniência:(Do lat. \textunderscore ad\textunderscore  + \textunderscore operare\textunderscore )}
\end{itemize}
Empregar numa obra.
Empregar.
Manufacturar.
\section{Adoptação}
\begin{itemize}
\item {Grp. gram.:f.}
\end{itemize}
O mesmo que \textunderscore adopção\textunderscore .
\section{Adoptante}
\begin{itemize}
\item {Grp. gram.:adj.}
\end{itemize}
\begin{itemize}
\item {Proveniência:(Lat. \textunderscore adoptans\textunderscore )}
\end{itemize}
Que adopta.
\section{Adoptar}
\begin{itemize}
\item {Grp. gram.:v. t.}
\end{itemize}
\begin{itemize}
\item {Proveniência:(Lat. \textunderscore adoptare\textunderscore )}
\end{itemize}
Tomar; acceitar.
Perfilhar.
\section{Adoptivamente}
\begin{itemize}
\item {Grp. gram.:adv.}
\end{itemize}
De modo \textunderscore adoptivo\textunderscore .
\section{Adoptivo}
\begin{itemize}
\item {Grp. gram.:adj.}
\end{itemize}
\begin{itemize}
\item {Proveniência:(Lat. \textunderscore adoptivus\textunderscore )}
\end{itemize}
Relativo a adopção.
Que adoptou.
Que foi adoptado.
\section{Adorabundo}
\begin{itemize}
\item {Grp. gram.:adj.}
\end{itemize}
\begin{itemize}
\item {Utilização:Poét.}
\end{itemize}
Que está em adoração.
\section{Adoração}
\begin{itemize}
\item {Grp. gram.:f.}
\end{itemize}
\begin{itemize}
\item {Proveniência:(Lat. \textunderscore adoratio\textunderscore )}
\end{itemize}
Acto de adorar.
\section{Adôrádo}
\begin{itemize}
\item {Grp. gram.:adj.}
\end{itemize}
\begin{itemize}
\item {Utilização:Ant.}
\end{itemize}
\begin{itemize}
\item {Utilização:Prov.}
\end{itemize}
\begin{itemize}
\item {Utilização:trasm.}
\end{itemize}
\begin{itemize}
\item {Proveniência:(De \textunderscore dôr\textunderscore )}
\end{itemize}
Adoentado.
Dolorido.
Muito dado ou dedicado a alguma coisa.
\section{Adorador}
\begin{itemize}
\item {Grp. gram.:m.  e  adj.}
\end{itemize}
O que adora.
\section{Adoramento}
\begin{itemize}
\item {Grp. gram.:m.}
\end{itemize}
O mesmo que \textunderscore adoração\textunderscore .
\section{Adorando}
\begin{itemize}
\item {Grp. gram.:adj.}
\end{itemize}
\begin{itemize}
\item {Proveniência:(Lat. \textunderscore adorandus\textunderscore )}
\end{itemize}
Adorável.
\section{Adorante}
\begin{itemize}
\item {Grp. gram.:adj.}
\end{itemize}
\begin{itemize}
\item {Proveniência:(Lat. \textunderscore adorans\textunderscore )}
\end{itemize}
Que adora.
\section{Adorar}
\begin{itemize}
\item {Grp. gram.:v. t.}
\end{itemize}
\begin{itemize}
\item {Proveniência:(Lat. \textunderscore adorare\textunderscore )}
\end{itemize}
Prestar culto a.
Reverenciar.
Amar muito.
Venerar.
\section{Adorativo}
\begin{itemize}
\item {Grp. gram.:adj.}
\end{itemize}
Que tem o carácter de adoração: \textunderscore affeição adorativa\textunderscore .
\section{Adorável}
\begin{itemize}
\item {Grp. gram.:adj.}
\end{itemize}
\begin{itemize}
\item {Proveniência:(Lat. \textunderscore adorabilis\textunderscore )}
\end{itemize}
Digno de sêr adorado.
\section{Adoravelmente}
\begin{itemize}
\item {Grp. gram.:adv.}
\end{itemize}
De maneira \textunderscore adorável\textunderscore .
\section{Adorbital}
\begin{itemize}
\item {Grp. gram.:m.  e  adj.}
\end{itemize}
\begin{itemize}
\item {Utilização:Anat.}
\end{itemize}
\begin{itemize}
\item {Proveniência:(De \textunderscore ad...\textunderscore  + \textunderscore orbital\textunderscore )}
\end{itemize}
Diz-se do osso que fórma a órbita.
\section{Adoriás}
\begin{itemize}
\item {Grp. gram.:m. pl.}
\end{itemize}
Selvagens, que habitaram nos sertões do Pará.
\section{Adório}
\begin{itemize}
\item {Grp. gram.:m.}
\end{itemize}
Gênero de plantas umbellíferas.
\section{Adormecedor}
\begin{itemize}
\item {Grp. gram.:adj.}
\end{itemize}
Que adormece.
\section{Adormecente}
\begin{itemize}
\item {Grp. gram.:adj.}
\end{itemize}
Que adormece. Cf. Filinto, VI, 277.
\section{Adormecer}
\begin{itemize}
\item {Grp. gram.:v. t.}
\end{itemize}
\begin{itemize}
\item {Grp. gram.:V. i.  e  p.}
\end{itemize}
Fazer dormir.
Acalentar.
Entorpecer.
Acalmar.
Cair no somno.
Immobilizar-se.
\section{Adormecido}
\begin{itemize}
\item {Grp. gram.:adj.}
\end{itemize}
Que adormeceu; que está dormindo.
\section{Adormecimento}
\begin{itemize}
\item {Grp. gram.:m.}
\end{itemize}
Acto de \textunderscore adormecer\textunderscore .
Entorpecimento.
\section{Adormentado}
\begin{itemize}
\item {Grp. gram.:adj.}
\end{itemize}
\begin{itemize}
\item {Proveniência:(De \textunderscore adormentar\textunderscore )}
\end{itemize}
Dormente.
Entorpecido.
\section{Adormentador}
\begin{itemize}
\item {Grp. gram.:adj.}
\end{itemize}
\begin{itemize}
\item {Grp. gram.:M.}
\end{itemize}
Que adormenta.
Medicamento que adormenta.
\section{Adormentar}
\begin{itemize}
\item {Grp. gram.:v. t.}
\end{itemize}
Tornar dormente.
Causar somno a.
Suavizar.
\section{Adormido}
\begin{itemize}
\item {Grp. gram.:adj.}
\end{itemize}
\begin{itemize}
\item {Utilização:Ant.}
\end{itemize}
O mesmo que \textunderscore adormecido\textunderscore . Cf. G. Vicente, I, 157.
\section{Adormir}
\begin{itemize}
\item {Grp. gram.:v. i.}
\end{itemize}
O mesmo que \textunderscore adormecer\textunderscore . Cf. Pacheco, \textunderscore Promptuário\textunderscore .
\section{Adornadamente}
\begin{itemize}
\item {Grp. gram.:adv.}
\end{itemize}
De modo \textunderscore adornado\textunderscore .
Com adôrno.
\section{Adornado}
\begin{itemize}
\item {Grp. gram.:adj.}
\end{itemize}
\begin{itemize}
\item {Proveniência:(De \textunderscore adornar\textunderscore )}
\end{itemize}
Em que há adôrno.
Enfeitado.
\section{Adornamento}
\begin{itemize}
\item {Grp. gram.:m.}
\end{itemize}
Acto de adornar^1.
\section{Adornar}
\begin{itemize}
\item {Grp. gram.:v. t.}
\end{itemize}
\begin{itemize}
\item {Proveniência:(Lat. \textunderscore adornare\textunderscore )}
\end{itemize}
Pôr adôrno em.
Aformosear.
\section{Adornar}
\begin{itemize}
\item {Grp. gram.:v. i.}
\end{itemize}
(V.adernar)
\section{Adôrno}
\begin{itemize}
\item {Grp. gram.:m.}
\end{itemize}
\begin{itemize}
\item {Proveniência:(De \textunderscore adornar\textunderscore )}
\end{itemize}
Enfeite, ornato, atavio.
\section{Adortar}
\begin{itemize}
\item {Grp. gram.:v. t.}
\end{itemize}
Exhortar.
Excitar, estimular. Cf. Rui Barbosa, \textunderscore Répl.\textunderscore , 157.
\section{Adossado}
\begin{itemize}
\item {Grp. gram.:adj.}
\end{itemize}
\begin{itemize}
\item {Utilização:Heráld.}
\end{itemize}
\begin{itemize}
\item {Proveniência:(Fr. \textunderscore adossé\textunderscore )}
\end{itemize}
Que está costas com costas, (falando-se de certas peças do escudo).
\section{Adotar}
\begin{itemize}
\item {Grp. gram.:v. t.}
\end{itemize}
\begin{itemize}
\item {Utilização:Prov.}
\end{itemize}
\begin{itemize}
\item {Utilização:trasm.}
\end{itemize}
O mesmo que \textunderscore dotar\textunderscore .
\section{Adoutar}
\textunderscore v. t.\textunderscore  (e der.) \textunderscore Ant.\textunderscore 
O mesmo que \textunderscore adoptar\textunderscore , etc.
\section{Adova}
\begin{itemize}
\item {fónica:dô}
\end{itemize}
\begin{itemize}
\item {Grp. gram.:f.}
\end{itemize}
\begin{itemize}
\item {Utilização:Ant.}
\end{itemize}
Chamava-se \textunderscore casa da adova\textunderscore  a sala livre, annexa ás cadeias, na qual passeavam os réus de culpas mais leves.
(Cp. \textunderscore adoba\textunderscore ^2)
\section{Adoxa}
\begin{itemize}
\item {fónica:csa}
\end{itemize}
\begin{itemize}
\item {Grp. gram.:f.}
\end{itemize}
Gênero de plantas, de cheiro almiscarado.
\section{Adoxo}
\begin{itemize}
\item {fónica:cso}
\end{itemize}
\begin{itemize}
\item {Grp. gram.:m.}
\end{itemize}
Parasito das plantas, e cuja larva é perigosa para as videiras.
\section{Adquirente}
\begin{itemize}
\item {Grp. gram.:adj.}
\end{itemize}
Que adquire.
\section{Adquirição}
\begin{itemize}
\item {Grp. gram.:f.}
\end{itemize}
(V.acquisição)
\section{Adquiridôr}
\begin{itemize}
\item {Grp. gram.:m.  e  adj.}
\end{itemize}
O que adquire.
\section{Adquirimento}
\begin{itemize}
\item {Grp. gram.:m.}
\end{itemize}
O mesmo que \textunderscore acquisição\textunderscore .
\section{Adquirir}
\begin{itemize}
\item {Grp. gram.:v. t.}
\end{itemize}
\begin{itemize}
\item {Proveniência:(Do lat. \textunderscore ad\textunderscore  + \textunderscore quaerere\textunderscore )}
\end{itemize}
Obter; alcançar; conseguir.
Receber.
Ganhar.
\section{Adquirível}
\begin{itemize}
\item {Grp. gram.:adj.}
\end{itemize}
Que se póde \textunderscore adquirir\textunderscore .
\section{Adquisição}
\begin{itemize}
\item {Grp. gram.:f.}
\end{itemize}
(V.acquisição)
\section{Adquisitivo}
\begin{itemize}
\item {Grp. gram.:adj.}
\end{itemize}
Relativo a acquisicão. Cf. Assis, \textunderscore Águas\textunderscore , 324.
\section{Adrachne}
\begin{itemize}
\item {Grp. gram.:f.}
\end{itemize}
\begin{itemize}
\item {Proveniência:(Gr. \textunderscore adrakhne\textunderscore )}
\end{itemize}
Arbusto, semelhante ao medronheiro, e de que se faz papel na China.
\section{Adracne}
\begin{itemize}
\item {Grp. gram.:f.}
\end{itemize}
\begin{itemize}
\item {Proveniência:(Gr. \textunderscore adrakhne\textunderscore )}
\end{itemize}
Arbusto, semelhante ao medronheiro, e de que se faz papel na China.
\section{Adraganthina}
\begin{itemize}
\item {Grp. gram.:f.}
\end{itemize}
Princípio immediato do adragantho.
\section{Adragantho}
\begin{itemize}
\item {Grp. gram.:m.}
\end{itemize}
(Alter. de \textunderscore tragacantho\textunderscore )
\section{Adragantina}
\begin{itemize}
\item {Grp. gram.:f.}
\end{itemize}
Princípio immediato do adragantho.
\section{Adraganto}
\begin{itemize}
\item {Grp. gram.:m.}
\end{itemize}
(Alter. de \textunderscore tragacantho\textunderscore )
\section{Adraguncho}
\begin{itemize}
\item {Grp. gram.:m.}
\end{itemize}
\begin{itemize}
\item {Utilização:Des.}
\end{itemize}
Enfermidade nas pernas e peito dos cavallos, procedente de tumores ruíns.
\section{Adrasto}
\begin{itemize}
\item {Grp. gram.:m.}
\end{itemize}
Insecto coleóptero pentâmero.
\section{Adrede}
\begin{itemize}
\item {Grp. gram.:adv.}
\end{itemize}
De propósito; acintemente.
(Provn. \textunderscore adreit\textunderscore , do lat. \textunderscore directus\textunderscore )
\section{Adregar}
\begin{itemize}
\item {Grp. gram.:v. i.}
\end{itemize}
\begin{itemize}
\item {Utilização:Prov.}
\end{itemize}
\begin{itemize}
\item {Utilização:trasm.}
\end{itemize}
Acontecer por acaso.
Chegar a propósito.
Acertar casualmente.
Enganar; illudir: \textunderscore não me adregas, descansa\textunderscore . Cf. \textunderscore Rev. Lus.\textunderscore , XI, 288.
\section{Adrêgo}
\begin{itemize}
\item {Grp. gram.:m.}
\end{itemize}
\begin{itemize}
\item {Utilização:Prov.}
\end{itemize}
\begin{itemize}
\item {Utilização:alent.}
\end{itemize}
\begin{itemize}
\item {Proveniência:(De \textunderscore adregar\textunderscore )}
\end{itemize}
Acaso, casualidade.
\section{Adrenal}
\begin{itemize}
\item {fónica:re}
\end{itemize}
\begin{itemize}
\item {Grp. gram.:adj.}
\end{itemize}
\begin{itemize}
\item {Utilização:Anat.}
\end{itemize}
\begin{itemize}
\item {Proveniência:(Do lat. \textunderscore ad\textunderscore  + \textunderscore renalis\textunderscore )}
\end{itemize}
Diz-se da glande ou cápsula, que está sobre o rim.
\section{Adrenalina}
\begin{itemize}
\item {fónica:re}
\end{itemize}
\begin{itemize}
\item {Grp. gram.:f.}
\end{itemize}
\begin{itemize}
\item {Utilização:Pharm.}
\end{itemize}
Substância medicamentosa, crystallina, de propriedades hemostáticas, extrahida das cápsulas supra-renaes.
\section{Adriça}
\begin{itemize}
\item {Grp. gram.:f.}
\end{itemize}
\begin{itemize}
\item {Proveniência:(It. \textunderscore addrizza\textunderscore )}
\end{itemize}
Cabo, para içar velas ou bandeiras, nos navios.
\section{Adriçar}
\begin{itemize}
\item {Grp. gram.:v. t.}
\end{itemize}
Erguer, por meio de adriças.
\section{Adrípia}
\begin{itemize}
\item {Grp. gram.:f.}
\end{itemize}
Planta hortense, mencionada em documentos da Idade-Média e hoje desconhecida.
\section{Adro}
\begin{itemize}
\item {Grp. gram.:m.}
\end{itemize}
\begin{itemize}
\item {Proveniência:(Do lat. \textunderscore atrium\textunderscore )}
\end{itemize}
Terreiro, em frente, e ás vezes em volta, da igreja.
\section{Adrogação}
\begin{itemize}
\item {fónica:ro}
\end{itemize}
\begin{itemize}
\item {Grp. gram.:f.}
\end{itemize}
Acto de \textunderscore adrogar\textunderscore .
\section{Adrogar}
\begin{itemize}
\item {fónica:ro}
\end{itemize}
\begin{itemize}
\item {Grp. gram.:v. t.}
\end{itemize}
\begin{itemize}
\item {Proveniência:(Lat. \textunderscore adrogare\textunderscore )}
\end{itemize}
Adoptar, ou tomar por adopção, pessôa de maior idade.
\section{Adinamia}
\begin{itemize}
\item {Grp. gram.:f.}
\end{itemize}
\begin{itemize}
\item {Proveniência:(Gr. \textunderscore adunamia\textunderscore )}
\end{itemize}
Debilidade, prostração de fôrças.
\section{Adinâmico}
\begin{itemize}
\item {Grp. gram.:adj.}
\end{itemize}
Relativo á \textunderscore adinamia\textunderscore .
\section{Adinamizar}
\begin{itemize}
\item {Grp. gram.:v. t.}
\end{itemize}
Tornar adínamo.
\section{Adínamo}
\begin{itemize}
\item {Grp. gram.:adj.}
\end{itemize}
\begin{itemize}
\item {Proveniência:(De \textunderscore adynamia\textunderscore )}
\end{itemize}
Débil, enfraquecido.
\section{Ádito}
\begin{itemize}
\item {Grp. gram.:m.}
\end{itemize}
\begin{itemize}
\item {Proveniência:(Gr. \textunderscore adutos\textunderscore )}
\end{itemize}
Câmara secreta, nos templos antigos.
\section{Adscrever}
\begin{itemize}
\item {Grp. gram.:v. t.}
\end{itemize}
Addicionar ao que está escripto.
Inscrever, registar. Cf. Castilho, \textunderscore Fastos\textunderscore , II, 489; III, 55.
\section{Adscripção}
\begin{itemize}
\item {Grp. gram.:f.}
\end{itemize}
\begin{itemize}
\item {Proveniência:(Lat. \textunderscore adscriptio\textunderscore )}
\end{itemize}
Additamento ao que se escreveu.
\section{Adscriptício}
\begin{itemize}
\item {Grp. gram.:adj.}
\end{itemize}
\begin{itemize}
\item {Proveniência:(De \textunderscore adscripto\textunderscore )}
\end{itemize}
Dizia-se do colono, obrigado a viver e trabalhar em terra determinada.
\section{Adscripto}
\begin{itemize}
\item {Grp. gram.:adj.}
\end{itemize}
\begin{itemize}
\item {Proveniência:(Do lat. \textunderscore adscriptus\textunderscore )}
\end{itemize}
Aditado.
Arrolado.
\section{Adsperso}
\begin{itemize}
\item {Grp. gram.:adj.}
\end{itemize}
\begin{itemize}
\item {Utilização:Bot.}
\end{itemize}
\begin{itemize}
\item {Proveniência:(Lat. \textunderscore adspersus\textunderscore )}
\end{itemize}
O mesmo que \textunderscore rajado\textunderscore .
\section{Adstrição}
\begin{itemize}
\item {Grp. gram.:f.}
\end{itemize}
Emprêgo de substância adstringente.
\section{Adstricção}
\begin{itemize}
\item {Grp. gram.:f.}
\end{itemize}
Emprêgo de substância adstringente.
\section{Adstrictivo}
\begin{itemize}
\item {Grp. gram.:adj.}
\end{itemize}
Que adstringe.
\section{Adstricto}
\begin{itemize}
\item {Grp. gram.:adj.}
\end{itemize}
\begin{itemize}
\item {Utilização:Med.}
\end{itemize}
\begin{itemize}
\item {Proveniência:(Lat. \textunderscore adstrictus\textunderscore )}
\end{itemize}
Adjunto, ligado.
Dependente.
Submetido.
Unido, apertado.
\section{Adstrictório}
\begin{itemize}
\item {Grp. gram.:adj.}
\end{itemize}
O mesmo que \textunderscore adstrictivo\textunderscore .
\section{Adstringência}
\begin{itemize}
\item {Grp. gram.:f.}
\end{itemize}
Qualidade do que é \textunderscore adstringente\textunderscore .
\section{Adstringente}
\begin{itemize}
\item {Grp. gram.:m.  e  adj.}
\end{itemize}
O que adstringe.
\section{Adstringir}
\begin{itemize}
\item {Grp. gram.:v. t.}
\end{itemize}
\begin{itemize}
\item {Proveniência:(Do lat. \textunderscore ad\textunderscore  + \textunderscore stringere\textunderscore )}
\end{itemize}
Unir; apertar.
Obrigar.
\section{Adstringitivo}
\begin{itemize}
\item {Grp. gram.:adj.}
\end{itemize}
(V.adstringente)
\section{Adstringivo}
\begin{itemize}
\item {Grp. gram.:m.  e  adj.}
\end{itemize}
(V.adstringente)
\section{Adstritivo}
\begin{itemize}
\item {Grp. gram.:adj.}
\end{itemize}
Que adstringe.
\section{Adstrito}
\begin{itemize}
\item {Grp. gram.:adj.}
\end{itemize}
\begin{itemize}
\item {Utilização:Med.}
\end{itemize}
\begin{itemize}
\item {Proveniência:(Lat. \textunderscore adstrictus\textunderscore )}
\end{itemize}
Adjunto, ligado.
Dependente.
Submetido.
Unido, apertado.
\section{Adstritório}
\begin{itemize}
\item {Grp. gram.:adj.}
\end{itemize}
O mesmo que \textunderscore adstrictivo\textunderscore .
\section{Adu}
\begin{itemize}
\item {Grp. gram.:adv.}
\end{itemize}
\begin{itemize}
\item {Utilização:Ant.}
\end{itemize}
Para onde.
\section{Adua}
\begin{itemize}
\item {Grp. gram.:f.}
\end{itemize}
\begin{itemize}
\item {Utilização:Prov.}
\end{itemize}
\begin{itemize}
\item {Utilização:alent.}
\end{itemize}
\begin{itemize}
\item {Utilização:Prov.}
\end{itemize}
\begin{itemize}
\item {Utilização:Ant.}
\end{itemize}
\begin{itemize}
\item {Utilização:Prov.}
\end{itemize}
\begin{itemize}
\item {Utilização:beir.}
\end{itemize}
Matilha de cães em correria.
Quadrilha de carrêtas.
Chamamento á guerra.
Obrigação de alistamento.
Impôsto, para isenção de alistamento.
Correria.
Rebanho.
Água de partilhas entre camponeses.
Local, onde os porcos, pertencentes a diversos habitantes da mesma povoação, permanecem durante o dia.
\section{Adua}
\begin{itemize}
\item {Grp. gram.:f.}
\end{itemize}
\begin{itemize}
\item {Utilização:Ant.}
\end{itemize}
Imposto para isenção do recrutamento; anaduva.
\section{Aduada}
\begin{itemize}
\item {Grp. gram.:f.}
\end{itemize}
\begin{itemize}
\item {Utilização:Prov.}
\end{itemize}
\begin{itemize}
\item {Utilização:beir.}
\end{itemize}
\begin{itemize}
\item {Proveniência:(De \textunderscore adua\textunderscore ^1)}
\end{itemize}
Manada (de porcos).
\section{Aduagem}
\begin{itemize}
\item {Grp. gram.:f.}
\end{itemize}
Acto de \textunderscore aduar\textunderscore ^2.
\section{Aduana}
\begin{itemize}
\item {Grp. gram.:f.}
\end{itemize}
\begin{itemize}
\item {Proveniência:(Do ár. \textunderscore adainan\textunderscore )}
\end{itemize}
Alfândega.
Imposto alfandegário.
Bairro de christãos em terras de moiros.
\section{Aduanar}
\begin{itemize}
\item {Grp. gram.:v. t.}
\end{itemize}
Despachar, registar na aduana.
\section{Aduaneiro}
\begin{itemize}
\item {Grp. gram.:adj.}
\end{itemize}
Relativo a aduanas; alfandegário.
\section{Aduar}
\begin{itemize}
\item {Grp. gram.:m.}
\end{itemize}
\begin{itemize}
\item {Utilização:Ant.}
\end{itemize}
\begin{itemize}
\item {Proveniência:(Do ár. \textunderscore ad-duar\textunderscore )}
\end{itemize}
Acampamento moirisco.
\section{Aduar}
\begin{itemize}
\item {Grp. gram.:v. t.}
\end{itemize}
Dividir em aduas (quinhões) a água de rega.
\section{Aduba}
\begin{itemize}
\item {Grp. gram.:f.}
\end{itemize}
O mesmo que \textunderscore anaduva\textunderscore .
\section{Adubação}
\begin{itemize}
\item {Grp. gram.:f.}
\end{itemize}
Acto ou effeito de \textunderscore adubar\textunderscore .
\section{Adubador}
\begin{itemize}
\item {Grp. gram.:m.}
\end{itemize}
O que aduba.
\section{Adubagem}
\begin{itemize}
\item {Grp. gram.:f.}
\end{itemize}
Acto de \textunderscore adubar\textunderscore .
\section{Adubamento}
\begin{itemize}
\item {Grp. gram.:m.}
\end{itemize}
Acto de \textunderscore adubar\textunderscore .
\section{Adubar}
\begin{itemize}
\item {Grp. gram.:v. t.}
\end{itemize}
\begin{itemize}
\item {Utilização:Ant.}
\end{itemize}
\begin{itemize}
\item {Utilização:Ant.}
\end{itemize}
\begin{itemize}
\item {Proveniência:(Do germ. \textunderscore dubba\textunderscore )}
\end{itemize}
Temperar; condimentar.
Curtir.
Concertar.
Preparar; apromptar. Cf. \textunderscore Port. Mon. Hist.\textunderscore , \textunderscore Script.\textunderscore , 245.
\section{Adube}
\begin{itemize}
\item {Grp. gram.:m.}
\end{itemize}
\begin{itemize}
\item {Utilização:T. do Porto}
\end{itemize}
O mesmo que \textunderscore adubo\textunderscore .
Toicinho para tempêro.
\section{Adubiar}
\begin{itemize}
\item {Grp. gram.:v. t.}
\end{itemize}
\begin{itemize}
\item {Utilização:Ant.}
\end{itemize}
\begin{itemize}
\item {Proveniência:(De \textunderscore adúbio\textunderscore )}
\end{itemize}
O mesmo que \textunderscore adubar\textunderscore .
\section{Adubio}
\begin{itemize}
\item {Grp. gram.:m.}
\end{itemize}
\begin{itemize}
\item {Utilização:Ant.}
\end{itemize}
Amanho de terras.
Trabalho de reparação e concêrto.
(Alter. de \textunderscore adubo\textunderscore )
\section{Adubo}
\begin{itemize}
\item {Grp. gram.:m.}
\end{itemize}
Tempêro.
Aquillo com que se aduba.
Aquillo com que se estrumam os terrenos.
Producto industrial, destinado a misturar-se com a terra arável ou cultivável, para a tornar mais productiva.
\section{Aduboiro}
\begin{itemize}
\item {Grp. gram.:m.}
\end{itemize}
\begin{itemize}
\item {Utilização:Ant.}
\end{itemize}
O mesmo que \textunderscore adubo\textunderscore .
\section{Adução}
\begin{itemize}
\item {Grp. gram.:f.}
\end{itemize}
Acto ou effeito de aduzir.
\section{Aducente}
\begin{itemize}
\item {Grp. gram.:adj.}
\end{itemize}
\begin{itemize}
\item {Proveniência:(Lat. \textunderscore adducens\textunderscore )}
\end{itemize}
Que aduz.
\section{Aduchar}
\begin{itemize}
\item {Grp. gram.:v.}
\end{itemize}
\begin{itemize}
\item {Utilização:t. Náut.}
\end{itemize}
\begin{itemize}
\item {Proveniência:(De \textunderscore aducho\textunderscore )}
\end{itemize}
Colhêr e enrolar (cabo e amarra)
\section{Aduchas}
\begin{itemize}
\item {Grp. gram.:f. pl.}
\end{itemize}
\begin{itemize}
\item {Utilização:Náut.}
\end{itemize}
\begin{itemize}
\item {Proveniência:(De \textunderscore aduchar\textunderscore )}
\end{itemize}
Voltas dos cabos enrolados.
\section{Aducho}
\begin{itemize}
\item {Grp. gram.:adj.}
\end{itemize}
\begin{itemize}
\item {Utilização:Ant.}
\end{itemize}
\begin{itemize}
\item {Grp. gram.:M.}
\end{itemize}
\begin{itemize}
\item {Proveniência:(Do lat. \textunderscore adductus\textunderscore )}
\end{itemize}
Dizia-se da testemunha adduzida ou apresentada.
Conducção.
\section{Adueiro}
\begin{itemize}
\item {Grp. gram.:m.}
\end{itemize}
\begin{itemize}
\item {Proveniência:(De \textunderscore adua\textunderscore ^1)}
\end{itemize}
Guarda de rebanho; pastor.
\section{Aduela}
\begin{itemize}
\item {Grp. gram.:f.}
\end{itemize}
\begin{itemize}
\item {Grp. gram.:Loc.}
\end{itemize}
\begin{itemize}
\item {Utilização:pop.}
\end{itemize}
Cada uma das tábuas, que formam o corpo dos tonéis, pipas, dornas, selhas, etc.
Cada uma das pedras do arco da abóbada.
Madeira americana.
Abertura do fôrro dos sacatrapos.
\textunderscore Aduela de menos\textunderscore , mania, bolha, falta de tino.
(Cast. \textunderscore duela\textunderscore )
\section{Aduelagem}
\begin{itemize}
\item {Grp. gram.:f.}
\end{itemize}
Execução e collocação de aduelas.
\section{Adufa}
\begin{itemize}
\item {Grp. gram.:f.}
\end{itemize}
\begin{itemize}
\item {Proveniência:(Do ár. \textunderscore ad\textunderscore  + \textunderscore duffa\textunderscore )}
\end{itemize}
Resguardo exterior das janelas, feito de tábuas estreitas, mas não unidas. Comporta.
Roda de pedra, galga, que esmaga a azeitona no lagar de azeite.
\section{Adufar}
\begin{itemize}
\item {Grp. gram.:v. t.}
\end{itemize}
Tapar com adufas.
\section{Adufar}
\begin{itemize}
\item {Grp. gram.:v. i.}
\end{itemize}
Tocar adufe.
\section{Adufe}
\begin{itemize}
\item {Grp. gram.:m.}
\end{itemize}
\begin{itemize}
\item {Proveniência:(Do ár. \textunderscore addofe\textunderscore )}
\end{itemize}
Pandeiro quadrado.
\section{Adufeiro}
\begin{itemize}
\item {Grp. gram.:m.}
\end{itemize}
O que toca adufe ou faz adufes.
\section{Adufo}
\begin{itemize}
\item {Grp. gram.:m.}
\end{itemize}
Peça quadrilonga de barro, amassado e sêco ao sol.
\section{Adufo}
\begin{itemize}
\item {Grp. gram.:m.}
\end{itemize}
\begin{itemize}
\item {Utilização:Bras}
\end{itemize}
O mesmo que \textunderscore adufe\textunderscore .
\section{Adugar}
\begin{itemize}
\item {Grp. gram.:v. t.}
\end{itemize}
\begin{itemize}
\item {Utilização:Ant.}
\end{itemize}
Fazer apparecer.
Apresentar como testemunha.
\section{Adulação}
\begin{itemize}
\item {Grp. gram.:f.}
\end{itemize}
\begin{itemize}
\item {Proveniência:(Lat. \textunderscore adulatio\textunderscore )}
\end{itemize}
Lisonja; acto de adular.
\section{Aduladamente}
\begin{itemize}
\item {Grp. gram.:adv.}
\end{itemize}
Com adulação.
\section{Adulador}
\begin{itemize}
\item {Grp. gram.:m.}
\end{itemize}
\begin{itemize}
\item {Proveniência:(Lat. \textunderscore adulator\textunderscore )}
\end{itemize}
O que adula.
\section{Adulante}
\begin{itemize}
\item {Grp. gram.:adj.}
\end{itemize}
Que adula.
\section{Adulão}
\begin{itemize}
\item {Grp. gram.:m.  e  adj.}
\end{itemize}
O mesmo que \textunderscore adulador\textunderscore .
\section{Adular}
\begin{itemize}
\item {Grp. gram.:v. t.}
\end{itemize}
\begin{itemize}
\item {Proveniência:(Lat. \textunderscore adulari\textunderscore )}
\end{itemize}
Lisonjear servilmente; bajular.
Gabar, por interesse próprio.
\section{Adulária}
\begin{itemize}
\item {Grp. gram.:f.}
\end{itemize}
\begin{itemize}
\item {Proveniência:(De \textunderscore Adule\textunderscore , n. p.)}
\end{itemize}
O mesmo que \textunderscore orthosa\textunderscore .
\section{Adulatoriamente}
\begin{itemize}
\item {Grp. gram.:adv.}
\end{itemize}
De modo \textunderscore adulatório\textunderscore .
\section{Adulatório}
\begin{itemize}
\item {Grp. gram.:adj.}
\end{itemize}
\begin{itemize}
\item {Proveniência:(Lat. \textunderscore adulatorius\textunderscore )}
\end{itemize}
Que contém adulação.
\section{Adulçorar}
\begin{itemize}
\item {Grp. gram.:v. t.}
\end{itemize}
Adoçar; suavizar. Cf. Cortesão, \textunderscore Subsídios\textunderscore .
\section{Adulo}
\begin{itemize}
\item {Grp. gram.:adv.}
\end{itemize}
\begin{itemize}
\item {Utilização:Prov.}
\end{itemize}
\begin{itemize}
\item {Utilização:trasm.}
\end{itemize}
\begin{itemize}
\item {Utilização:Ant.}
\end{itemize}
O mesmo que \textunderscore onde\textunderscore .
\section{Aduloso}
\begin{itemize}
\item {Grp. gram.:adj.}
\end{itemize}
O mesmo que \textunderscore adulante\textunderscore .
\section{Adúltera}
(fem. de \textunderscore adúltero\textunderscore )
\section{Adulteração}
\begin{itemize}
\item {Grp. gram.:f.}
\end{itemize}
\begin{itemize}
\item {Proveniência:(Lat. \textunderscore adulteratio\textunderscore )}
\end{itemize}
Acto de adulterar.
Adultério.
Falsificação.
\section{Adulteradamente}
\begin{itemize}
\item {Grp. gram.:adv.}
\end{itemize}
De modo \textunderscore adulterado\textunderscore ; com adulteração.
Falsificadamente.
\section{Adulterado}
\begin{itemize}
\item {Grp. gram.:adj.}
\end{itemize}
\begin{itemize}
\item {Proveniência:(De \textunderscore adulterar\textunderscore )}
\end{itemize}
Falsificado.
Imitado dolosamente: \textunderscore vinho adulterado\textunderscore .
\section{Adulterador}
\begin{itemize}
\item {Grp. gram.:m.}
\end{itemize}
\begin{itemize}
\item {Proveniência:(Lat. \textunderscore adulterator\textunderscore )}
\end{itemize}
O que adultera.
\section{Adulteramente}
\begin{itemize}
\item {Grp. gram.:adv.}
\end{itemize}
\begin{itemize}
\item {Proveniência:(De \textunderscore adúltero\textunderscore )}
\end{itemize}
Com adultério.
\section{Adulterar}
\begin{itemize}
\item {Grp. gram.:v. t.}
\end{itemize}
\begin{itemize}
\item {Grp. gram.:V. i.}
\end{itemize}
\begin{itemize}
\item {Proveniência:(Lat. \textunderscore adulterare\textunderscore )}
\end{itemize}
Falsificar; contrafazer.
Corromper.
Commeter adultério.
\section{Adulterino}
\begin{itemize}
\item {Grp. gram.:adj.}
\end{itemize}
\begin{itemize}
\item {Proveniência:(Lat. \textunderscore adulterinus\textunderscore )}
\end{itemize}
Proveniente de adultério: \textunderscore filho adulterino\textunderscore .
Que soffreu adulteração.
\section{Adultério}
\begin{itemize}
\item {Grp. gram.:m.}
\end{itemize}
\begin{itemize}
\item {Proveniência:(Lat. \textunderscore adulterium\textunderscore )}
\end{itemize}
Infidelidade conjugal.
Falsificação; adulteração.
\section{Adulterioso}
\begin{itemize}
\item {Grp. gram.:adj.}
\end{itemize}
Adulteroso.
Que tem o carácter de adultério.
\section{Adulterismo}
\begin{itemize}
\item {Grp. gram.:m.}
\end{itemize}
Palavra ou nome adulterado.
\section{Adúltero}
\begin{itemize}
\item {Grp. gram.:m.}
\end{itemize}
\begin{itemize}
\item {Proveniência:(Lat. \textunderscore adulter\textunderscore )}
\end{itemize}
O que violou a fé conjugal.
\section{Adulteroso}
\begin{itemize}
\item {Grp. gram.:adj.}
\end{itemize}
\begin{itemize}
\item {Proveniência:(De \textunderscore adúltero\textunderscore )}
\end{itemize}
Em que há adultério.
Propenso ao adultério.
\section{Adulto}
\begin{itemize}
\item {Grp. gram.:m.}
\end{itemize}
\begin{itemize}
\item {Grp. gram.:Adj.}
\end{itemize}
\begin{itemize}
\item {Proveniência:(Lat. \textunderscore adultus\textunderscore )}
\end{itemize}
Homem, que passou a época da puberdade e está na idade da adolescência ou da virilidade.
Que chegou á idade madura e vigorosa.
\section{Adumar}
\begin{itemize}
\item {Grp. gram.:v. i.}
\end{itemize}
\begin{itemize}
\item {Utilização:Prov.}
\end{itemize}
\begin{itemize}
\item {Utilização:minh.}
\end{itemize}
Dormir em pé.
\section{Adomar}
\begin{itemize}
\item {Grp. gram.:v. i.}
\end{itemize}
\begin{itemize}
\item {Utilização:Prov.}
\end{itemize}
\begin{itemize}
\item {Utilização:minh.}
\end{itemize}
Dormir em pé.
\section{Adumbrar}
\begin{itemize}
\item {Grp. gram.:v. t.}
\end{itemize}
\begin{itemize}
\item {Proveniência:(Lat. \textunderscore adumbrare\textunderscore )}
\end{itemize}
Sombrear, assombrear.
\section{Adunação}
\begin{itemize}
\item {Grp. gram.:f.}
\end{itemize}
Acto de \textunderscore adunar\textunderscore .
\section{Adunar}
\begin{itemize}
\item {Grp. gram.:v. t.}
\end{itemize}
\begin{itemize}
\item {Proveniência:(Lat. \textunderscore adunare\textunderscore )}
\end{itemize}
Reunir em um.
Congregar.
Coadunar.
\section{Aduncado}
\begin{itemize}
\item {Grp. gram.:adj.}
\end{itemize}
O mesmo que adunco. Cf. Camillo, \textunderscore Caveira\textunderscore , 415.
\section{Aduncidade}
\begin{itemize}
\item {Grp. gram.:f.}
\end{itemize}
Qualidade de adunco.
\section{Aduncirostro}
\begin{itemize}
\item {fónica:rós}
\end{itemize}
\begin{itemize}
\item {Grp. gram.:adj.}
\end{itemize}
\begin{itemize}
\item {Proveniência:(Do lat. \textunderscore aduncus\textunderscore  + \textunderscore rostrum\textunderscore )}
\end{itemize}
Que tem bico adunco.
\section{Aduncirrostro}
\begin{itemize}
\item {Grp. gram.:adj.}
\end{itemize}
\begin{itemize}
\item {Proveniência:(Do lat. \textunderscore aduncus\textunderscore  + \textunderscore rostrum\textunderscore )}
\end{itemize}
Que tem bico adunco.
\section{Adunco}
\begin{itemize}
\item {Grp. gram.:adj.}
\end{itemize}
\begin{itemize}
\item {Proveniência:(Lat. \textunderscore aduncus\textunderscore )}
\end{itemize}
Curvo, em fórma de gancho.
\section{Adúnia}
\begin{itemize}
\item {Grp. gram.:adv.}
\end{itemize}
\begin{itemize}
\item {Utilização:Ant.}
\end{itemize}
De toda a parte.
Em abundância.
\section{Adur}
\begin{itemize}
\item {Grp. gram.:adv.}
\end{itemize}
\begin{itemize}
\item {Utilização:Ant.}
\end{itemize}
\begin{itemize}
\item {Grp. gram.:M.}
\end{itemize}
Difficilmente.
Apenas; mal.
Maldade.
Traição.
(Talvez do lat. \textunderscore dure\textunderscore )
\section{Adur}
\begin{itemize}
\item {Grp. gram.:m.}
\end{itemize}
\begin{itemize}
\item {Utilização:Ant.}
\end{itemize}
Burla; traição.
Velhacaria.
\section{Adurar}
\begin{itemize}
\item {Grp. gram.:v. t.}
\end{itemize}
\begin{itemize}
\item {Utilização:Ant.}
\end{itemize}
O mesmo que \textunderscore durar\textunderscore . Cf. C. de Barcellos, \textunderscore L. das Cantigas\textunderscore .
\section{Adurência}
\begin{itemize}
\item {Grp. gram.:f.}
\end{itemize}
Qualidade de adurente.
\section{Adurente}
\begin{itemize}
\item {Grp. gram.:m.}
\end{itemize}
\begin{itemize}
\item {Grp. gram.:Adj.}
\end{itemize}
\begin{itemize}
\item {Proveniência:(Lat. \textunderscore adurens\textunderscore )}
\end{itemize}
Medicamento cáustico.
Que queima.
\section{Adurir}
\begin{itemize}
\item {Grp. gram.:v. t.}
\end{itemize}
\begin{itemize}
\item {Utilização:Des.}
\end{itemize}
\begin{itemize}
\item {Proveniência:(Lat. \textunderscore adurere\textunderscore )}
\end{itemize}
Queimar.
\section{Aduro}
\begin{itemize}
\item {Grp. gram.:adv.}
\end{itemize}
\begin{itemize}
\item {Utilização:Ant.}
\end{itemize}
O mesmo que \textunderscore adur\textunderscore ^1.
\section{Adussia}
\begin{itemize}
\item {Grp. gram.:f.}
\end{itemize}
\begin{itemize}
\item {Utilização:Ant.}
\end{itemize}
\begin{itemize}
\item {Grp. gram.:Adj. f.}
\end{itemize}
Capella-mór.
Dizia-se da cadeira de braços ou de espaldar.
\section{Adustão}
\begin{itemize}
\item {Grp. gram.:f.}
\end{itemize}
\begin{itemize}
\item {Proveniência:(Lat. \textunderscore adustio\textunderscore )}
\end{itemize}
Cauterização com fogo.
\section{Adustível}
\begin{itemize}
\item {Grp. gram.:adj.}
\end{itemize}
O mesmo que \textunderscore combustível\textunderscore .
\section{Adustivo}
\begin{itemize}
\item {Grp. gram.:adj.}
\end{itemize}
O mesmo que \textunderscore adurente\textunderscore .
\section{Adusto}
\begin{itemize}
\item {Grp. gram.:adj.}
\end{itemize}
\begin{itemize}
\item {Proveniência:(Lat. \textunderscore adustus\textunderscore )}
\end{itemize}
Queimado.
Ardente, esbraseado.
\section{Adutivo}
\begin{itemize}
\item {Grp. gram.:adj.}
\end{itemize}
\begin{itemize}
\item {Proveniência:(Do lat. \textunderscore adductus\textunderscore )}
\end{itemize}
Que póde aduzir.
\section{Adutor}
\begin{itemize}
\item {Grp. gram.:m.}
\end{itemize}
\begin{itemize}
\item {Proveniência:(Lat. \textunderscore adductor\textunderscore )}
\end{itemize}
O que aduz.
\section{Adutra}
\begin{itemize}
\item {Grp. gram.:f.}
\end{itemize}
\begin{itemize}
\item {Utilização:Ant.}
\end{itemize}
Pano verde e vermelho, com pinturas que representavam aves, e usado antigamente na Índia portuguesa.
\section{Aduzer}
\begin{itemize}
\item {Grp. gram.:v. t.}
\end{itemize}
\begin{itemize}
\item {Utilização:Ant.}
\end{itemize}
O mesmo que \textunderscore aduzir\textunderscore .
\section{Aduzir}
\begin{itemize}
\item {Grp. gram.:v. t.}
\end{itemize}
\begin{itemize}
\item {Proveniência:(Lat. \textunderscore adducere\textunderscore )}
\end{itemize}
Trazer.
Expor, apresentar.
\section{Ádvena}
\begin{itemize}
\item {Grp. gram.:m.}
\end{itemize}
\begin{itemize}
\item {Proveniência:(Lat. \textunderscore advena\textunderscore )}
\end{itemize}
Estrangeiro.
Adventício.
\section{Advendiço}
\begin{itemize}
\item {Grp. gram.:adj.}
\end{itemize}
\begin{itemize}
\item {Utilização:Ant.}
\end{itemize}
O mesmo que \textunderscore adventicio\textunderscore .
\section{Advenidiço}
\begin{itemize}
\item {Grp. gram.:adj.}
\end{itemize}
O mesmo que \textunderscore adventicio\textunderscore . Cf. Latino, \textunderscore Humboldt\textunderscore , 375.
\section{Adveniente}
\begin{itemize}
\item {Grp. gram.:adj.}
\end{itemize}
Que advém. Cf. Camillo, \textunderscore Am. de Salvação\textunderscore , 20.
\section{Adventiciamente}
\begin{itemize}
\item {Grp. gram.:adv.}
\end{itemize}
De modo \textunderscore adventício\textunderscore .
\section{Adventicio}
\begin{itemize}
\item {Grp. gram.:adj.}
\end{itemize}
\begin{itemize}
\item {Proveniência:(Lat. \textunderscore adventicius\textunderscore )}
\end{itemize}
Que chega de fóra.
Estrangeiro.
Casual.
Aquelle que vem de fóra; que é estranho ou intruso.
\section{Advento}
\begin{itemize}
\item {Grp. gram.:m.}
\end{itemize}
\begin{itemize}
\item {Proveniência:(Lat. \textunderscore adventus\textunderscore )}
\end{itemize}
Chegada; vinda.
Princípio.
Período das quatro semanas anteriores ao Natal.
\section{Adverbial}
\begin{itemize}
\item {Grp. gram.:adj.}
\end{itemize}
\begin{itemize}
\item {Proveniência:(Lat. \textunderscore adverbialis\textunderscore )}
\end{itemize}
Relativo a advérbio.
Equivalente a um advérbio: \textunderscore locução adverbial\textunderscore .
\section{Adverbiar}
\begin{itemize}
\item {Grp. gram.:v. t.}
\end{itemize}
Empregar como advérbio, ou como terminação adverbial.
\section{Advérbio}
\begin{itemize}
\item {Grp. gram.:m.}
\end{itemize}
\begin{itemize}
\item {Proveniência:(Lat. \textunderscore adverbium\textunderscore )}
\end{itemize}
Palavra invariável, que representa um complemento circunstancial.
\section{Adversamente}
\begin{itemize}
\item {Grp. gram.:adv.}
\end{itemize}
De modo \textunderscore adverso\textunderscore .
\section{Adversão}
\begin{itemize}
\item {Grp. gram.:f.}
\end{itemize}
\begin{itemize}
\item {Proveniência:(Lat. \textunderscore adversio\textunderscore )}
\end{itemize}
Advertência.
Opposição.
Acto de adversar.
\section{Adversar}
\begin{itemize}
\item {Grp. gram.:v. t.}
\end{itemize}
\begin{itemize}
\item {Proveniência:(Lat. \textunderscore adversari\textunderscore )}
\end{itemize}
Contrariar; combater.
\section{Adversário}
\begin{itemize}
\item {Grp. gram.:m.}
\end{itemize}
\begin{itemize}
\item {Grp. gram.:Adj.}
\end{itemize}
\begin{itemize}
\item {Proveniência:(Lat. \textunderscore adversarius\textunderscore )}
\end{itemize}
Inimigo; o que se oppõe.
Que luta contra.
\section{Adversativamente}
\begin{itemize}
\item {Grp. gram.:adv.}
\end{itemize}
De modo \textunderscore adversativo\textunderscore .
\section{Adversativo}
\begin{itemize}
\item {Grp. gram.:adj.}
\end{itemize}
\begin{itemize}
\item {Utilização:Gram.}
\end{itemize}
\begin{itemize}
\item {Proveniência:(Lat. \textunderscore adversativus\textunderscore )}
\end{itemize}
Opposto.
Que indica differença entre o que precede e o que segue: \textunderscore preposicão adversativa\textunderscore .
\section{Adversia}
\begin{itemize}
\item {Grp. gram.:f.}
\end{itemize}
\begin{itemize}
\item {Utilização:Des.}
\end{itemize}
\begin{itemize}
\item {Proveniência:(De \textunderscore adverso\textunderscore )}
\end{itemize}
Inspiração diabólica.
\section{Adversidade}
\begin{itemize}
\item {Grp. gram.:f.}
\end{itemize}
\begin{itemize}
\item {Proveniência:(Lat. \textunderscore adversitas\textunderscore )}
\end{itemize}
Contrariedade.
Infelicidade; sorte adversa.
\section{Adversifólio}
\begin{itemize}
\item {Grp. gram.:adj.}
\end{itemize}
\begin{itemize}
\item {Proveniência:(Do lat. \textunderscore adversus\textunderscore  + \textunderscore folium\textunderscore )}
\end{itemize}
Diz-se das plantas que apresentam fôlhas oppostas, no mesmo tronco.
\section{Adverso}
\begin{itemize}
\item {Grp. gram.:adj.}
\end{itemize}
\begin{itemize}
\item {Proveniência:(Lat. \textunderscore adversus\textunderscore )}
\end{itemize}
Opposto; contrário.
\section{Advertência}
\begin{itemize}
\item {Grp. gram.:f.}
\end{itemize}
Acto ou effeito de \textunderscore advertir\textunderscore .
\section{Advertidamente}
\begin{itemize}
\item {Grp. gram.:adv.}
\end{itemize}
Com advertência.
\section{Advertimento}
\begin{itemize}
\item {Grp. gram.:m.}
\end{itemize}
O mesmo que \textunderscore advertência\textunderscore .
\section{Advertir}
\begin{itemize}
\item {Grp. gram.:v. t.}
\end{itemize}
\begin{itemize}
\item {Proveniência:(Lat. \textunderscore advertere\textunderscore )}
\end{itemize}
Chamar a attenção de; notar.
Reprehender levemente.
\section{Advindo}
\begin{itemize}
\item {Grp. gram.:adj.}
\end{itemize}
Que adveio, que sobreveio.
\section{Advir}
\begin{itemize}
\item {Grp. gram.:v. i.}
\end{itemize}
\begin{itemize}
\item {Proveniência:(Lat. \textunderscore advenire\textunderscore )}
\end{itemize}
Chegar.
Succeder.
Accrescer.
\section{Advocação}
\begin{itemize}
\item {Grp. gram.:f.}
\end{itemize}
\begin{itemize}
\item {Utilização:Des.}
\end{itemize}
\begin{itemize}
\item {Proveniência:(Lat. \textunderscore advocatio\textunderscore )}
\end{itemize}
Invocação.
\section{Advocacia}
\begin{itemize}
\item {Grp. gram.:f.}
\end{itemize}
\begin{itemize}
\item {Proveniência:(Do lat. \textunderscore advocatus\textunderscore )}
\end{itemize}
Profissão de advogado.
\section{Advocar}
\begin{itemize}
\item {Grp. gram.:v. t.}
\end{itemize}
\begin{itemize}
\item {Utilização:Des.}
\end{itemize}
O mesmo que \textunderscore chamar\textunderscore . Cf. Vieira, II, 212.
\section{Advocatura}
\begin{itemize}
\item {Grp. gram.:f.}
\end{itemize}
(V.advocacia)
\section{Advogado}
\begin{itemize}
\item {Grp. gram.:m.}
\end{itemize}
\begin{itemize}
\item {Proveniência:(Lat. \textunderscore advocatus\textunderscore )}
\end{itemize}
O que advoga em juizo.
Patrono.
Protector.
\section{Advogar}
\begin{itemize}
\item {Grp. gram.:v. t.}
\end{itemize}
\begin{itemize}
\item {Grp. gram.:V. i.}
\end{itemize}
\begin{itemize}
\item {Utilização:Des.}
\end{itemize}
\begin{itemize}
\item {Proveniência:(Lat. \textunderscore advocare\textunderscore )}
\end{itemize}
Defender; patrocinar.
Exercer a profissão de advogado, defendendo ou atacando.
O mesmo que \textunderscore avocar\textunderscore .
\section{Advogaria}
\begin{itemize}
\item {Grp. gram.:f.}
\end{itemize}
\begin{itemize}
\item {Utilização:Ant.}
\end{itemize}
O mesmo que \textunderscore advocacia\textunderscore .
\section{Ady}
\begin{itemize}
\item {Grp. gram.:f.}
\end{itemize}
Espécie de palmeira da ilha de San-Thomé.
\section{Adynamia}
\begin{itemize}
\item {Grp. gram.:f.}
\end{itemize}
\begin{itemize}
\item {Proveniência:(Gr. \textunderscore adunamia\textunderscore )}
\end{itemize}
Debilidade, prostração de fôrças.
\section{Adynâmico}
\begin{itemize}
\item {Grp. gram.:adj.}
\end{itemize}
Relativo á \textunderscore adynamia\textunderscore .
\section{Adynamizar}
\begin{itemize}
\item {Grp. gram.:v. t.}
\end{itemize}
Tornar adýnamo.
\section{Adýnamo}
\begin{itemize}
\item {Grp. gram.:adj.}
\end{itemize}
\begin{itemize}
\item {Proveniência:(De \textunderscore adynamia\textunderscore )}
\end{itemize}
Débil, enfraquecido.
\section{Ádyto}
\begin{itemize}
\item {Grp. gram.:m.}
\end{itemize}
\begin{itemize}
\item {Proveniência:(Gr. \textunderscore adutos\textunderscore )}
\end{itemize}
Câmara secreta, nos templos antigos.
\section{Aédo}
\begin{itemize}
\item {Grp. gram.:m.}
\end{itemize}
\begin{itemize}
\item {Proveniência:(Gr. \textunderscore aedon\textunderscore )}
\end{itemize}
Poeta, cantor, (entre os Gregos antigos).
\section{...aens}
\begin{itemize}
\item {Grp. gram.:suf. pl.}
\end{itemize}
O mesmo que ...\textunderscore ães\textunderscore .
\section{Aer...}
\begin{itemize}
\item {Grp. gram.:pref.}
\end{itemize}
\begin{itemize}
\item {Proveniência:(Gr. \textunderscore aer\textunderscore )}
\end{itemize}
(Designativo de ar.)
\section{Aeração}
\begin{itemize}
\item {fónica:a-e}
\end{itemize}
\begin{itemize}
\item {Grp. gram.:f.}
\end{itemize}
\begin{itemize}
\item {Proveniência:(Do gr. \textunderscore aer\textunderscore )}
\end{itemize}
Movimento do ar; ventilação:«\textunderscore apparelho de aeração\textunderscore ». Castilho, \textunderscore Fastos\textunderscore , III, 473.
\section{Aereamente}
\begin{itemize}
\item {fónica:a-e}
\end{itemize}
\begin{itemize}
\item {Grp. gram.:adv.}
\end{itemize}
De modo aéreo; á tôa; levianamente.
\section{Aeremotoxia}
\begin{itemize}
\item {fónica:a-e...csi}
\end{itemize}
\begin{itemize}
\item {Grp. gram.:f.}
\end{itemize}
\begin{itemize}
\item {Proveniência:(Do gr. \textunderscore aer\textunderscore  + \textunderscore kaima\textunderscore  + \textunderscore toxikon\textunderscore )}
\end{itemize}
Morte, produzida pela introducção de ar nas veias, durante certas operações cirúrgicas.
\section{Aéreo}
\begin{itemize}
\item {Grp. gram.:adj.}
\end{itemize}
\begin{itemize}
\item {Proveniência:(Lat. \textunderscore aereus\textunderscore )}
\end{itemize}
Que é do ar.
Que vive no ar.
Semelhante ao ar.
Que está suspenso no ar.
Vão, sem fundamento.
\section{Aéreo}
\begin{itemize}
\item {Grp. gram.:adj.}
\end{itemize}
\begin{itemize}
\item {Proveniência:(Lat. \textunderscore aereus\textunderscore )}
\end{itemize}
O mesmo que \textunderscore éreo\textunderscore .
\section{Aerethmia}
\begin{itemize}
\item {fónica:a-e}
\end{itemize}
\begin{itemize}
\item {Grp. gram.:f.}
\end{itemize}
Infiltração de ar no tecido cellular.
\section{Aeretmia}
\begin{itemize}
\item {fónica:a-e}
\end{itemize}
\begin{itemize}
\item {Grp. gram.:f.}
\end{itemize}
Infiltração de ar no tecido cellular.
\section{Aerícola}
\begin{itemize}
\item {fónica:a-e}
\end{itemize}
\begin{itemize}
\item {Grp. gram.:adj.}
\end{itemize}
\begin{itemize}
\item {Proveniência:(Do lat. \textunderscore aer\textunderscore  + \textunderscore colere\textunderscore )}
\end{itemize}
Que vive no ar.
\section{Aérida}
\begin{itemize}
\item {Grp. gram.:f.}
\end{itemize}
\begin{itemize}
\item {Proveniência:(Do lat. \textunderscore aer\textunderscore )}
\end{itemize}
Nome commum ás plantas que vivem só no ar.
\section{Aérido}
\begin{itemize}
\item {Grp. gram.:m.}
\end{itemize}
\begin{itemize}
\item {Grp. gram.:Adj.}
\end{itemize}
O mesmo que \textunderscore aérida\textunderscore .
O mesmo que \textunderscore aerícola\textunderscore .
\section{Aerifero}
\begin{itemize}
\item {fónica:a-e}
\end{itemize}
\begin{itemize}
\item {Grp. gram.:adj.}
\end{itemize}
\begin{itemize}
\item {Proveniência:(Do lat. \textunderscore aer\textunderscore  + \textunderscore ferre\textunderscore )}
\end{itemize}
Que conduz ar.
\section{Aerificação}
\begin{itemize}
\item {fónica:a-e}
\end{itemize}
\begin{itemize}
\item {Grp. gram.:f.}
\end{itemize}
Acto de \textunderscore aerificar\textunderscore .
\section{Aerificar}
\begin{itemize}
\item {fónica:a-e}
\end{itemize}
\begin{itemize}
\item {Grp. gram.:v. t.}
\end{itemize}
\begin{itemize}
\item {Proveniência:(Do lat. \textunderscore aer\textunderscore  + \textunderscore facere\textunderscore )}
\end{itemize}
Reduzir a estado gasoso.
\section{Aeriforme}
\begin{itemize}
\item {fónica:a-e}
\end{itemize}
\begin{itemize}
\item {Grp. gram.:adj.}
\end{itemize}
\begin{itemize}
\item {Proveniência:(Do lat. \textunderscore aer\textunderscore  + \textunderscore forma\textunderscore )}
\end{itemize}
Semelhante ao ar.
\section{Aerínea}
\begin{itemize}
\item {fónica:a-e}
\end{itemize}
\begin{itemize}
\item {Grp. gram.:f.}
\end{itemize}
\begin{itemize}
\item {Proveniência:(Do gr. \textunderscore aer\textunderscore )}
\end{itemize}
Vestido azul celeste, usado na antiga comédia grega pelas mulheres idosas.
\section{Aerívoro}
\begin{itemize}
\item {fónica:a-e}
\end{itemize}
\begin{itemize}
\item {Grp. gram.:adj.}
\end{itemize}
\begin{itemize}
\item {Proveniência:(Do lat. \textunderscore aer\textunderscore  + \textunderscore vorare\textunderscore )}
\end{itemize}
Que vive ou se alimenta do ar.
\section{Aerização}
\begin{itemize}
\item {fónica:a-e}
\end{itemize}
\begin{itemize}
\item {Grp. gram.:f.}
\end{itemize}
Acto de \textunderscore aerizar\textunderscore .
\section{Aerizar}
\begin{itemize}
\item {fónica:a-e}
\end{itemize}
\begin{itemize}
\item {Grp. gram.:v. t.}
\end{itemize}
O mesmo que \textunderscore aerificar\textunderscore .
\section{Aerobata}
\begin{itemize}
\item {fónica:a-e}
\end{itemize}
\begin{itemize}
\item {Grp. gram.:m.}
\end{itemize}
\begin{itemize}
\item {Proveniência:(Do gr. \textunderscore aer\textunderscore  + \textunderscore bates\textunderscore )}
\end{itemize}
O que anda pelo ar.
Nephelibata.
\section{Aeróbio}
\begin{itemize}
\item {fónica:a-e}
\end{itemize}
\begin{itemize}
\item {Grp. gram.:adj.}
\end{itemize}
\begin{itemize}
\item {Proveniência:(Do gr. \textunderscore aer\textunderscore  + \textunderscore bios\textunderscore )}
\end{itemize}
Que vive no ar.
\section{Aeroclavicórdio}
\begin{itemize}
\item {fónica:a-e}
\end{itemize}
\begin{itemize}
\item {Grp. gram.:m.}
\end{itemize}
Instrumento, hoje desusado, espécie de cravo, cujas cordas vibravam sob a acção do ar.
\section{Aerodinâmica}
\begin{itemize}
\item {fónica:a-e}
\end{itemize}
\begin{itemize}
\item {Grp. gram.:f.}
\end{itemize}
\begin{itemize}
\item {Proveniência:(De \textunderscore aer...\textunderscore  + \textunderscore dynâmica\textunderscore )}
\end{itemize}
Parte da Phýsica, que trata das leis reguladoras do movimento dos fluidos elásticos, ou das que regulam a pressão do ar exterior.
\section{Aeródromo}
\begin{itemize}
\item {fónica:a-e}
\end{itemize}
\begin{itemize}
\item {Grp. gram.:m.}
\end{itemize}
\begin{itemize}
\item {Utilização:Bras}
\end{itemize}
\begin{itemize}
\item {Proveniência:(Do gr. \textunderscore aer\textunderscore  + \textunderscore dromos\textunderscore )}
\end{itemize}
Barracão, destinado ao enchimento e preparação dos balões.
\section{Aerodynâmica}
\begin{itemize}
\item {fónica:a-e}
\end{itemize}
\begin{itemize}
\item {Grp. gram.:f.}
\end{itemize}
\begin{itemize}
\item {Proveniência:(De \textunderscore aer...\textunderscore  + \textunderscore dynâmica\textunderscore )}
\end{itemize}
Parte da Phýsica, que trata das leis reguladoras do movimento dos fluidos elásticos, ou das que regulam a pressão do ar exterior.
\section{Aerófago}
\begin{itemize}
\item {fónica:a-e}
\end{itemize}
\begin{itemize}
\item {Grp. gram.:m.}
\end{itemize}
\begin{itemize}
\item {Proveniência:(Do gr. \textunderscore aer\textunderscore  + \textunderscore phagein\textunderscore )}
\end{itemize}
O mesmo que \textunderscore aerívoro\textunderscore .
\section{Aerófita}
\begin{itemize}
\item {fónica:a-e}
\end{itemize}
\begin{itemize}
\item {Grp. gram.:f.}
\end{itemize}
\begin{itemize}
\item {Proveniência:(Do gr. \textunderscore aer\textunderscore  + \textunderscore phuton\textunderscore )}
\end{itemize}
O mesmo que \textunderscore aérida\textunderscore , segundo Lamouroux.
\section{Aerófito}
\begin{itemize}
\item {fónica:a-e}
\end{itemize}
\begin{itemize}
\item {Grp. gram.:adj.}
\end{itemize}
\begin{itemize}
\item {Proveniência:(Do gr. \textunderscore aer\textunderscore  + \textunderscore phuton\textunderscore )}
\end{itemize}
Diz-se da planta que vive no ar.
\section{Aerofobia}
\begin{itemize}
\item {fónica:a-e}
\end{itemize}
\begin{itemize}
\item {Grp. gram.:f.}
\end{itemize}
\begin{itemize}
\item {Proveniência:(De \textunderscore aeróphobo\textunderscore )}
\end{itemize}
Doença, caracterizada pêlo horror ao ar.
\section{Aerófobo}
\begin{itemize}
\item {fónica:a-e}
\end{itemize}
\begin{itemize}
\item {Grp. gram.:m.}
\end{itemize}
\begin{itemize}
\item {Proveniência:(Do gr. \textunderscore aer\textunderscore  + \textunderscore phobos\textunderscore )}
\end{itemize}
O que tem horror ao ar.
\section{Aerofónio}
\begin{itemize}
\item {fónica:a-e}
\end{itemize}
\begin{itemize}
\item {Grp. gram.:adj.}
\end{itemize}
\begin{itemize}
\item {Grp. gram.:M.}
\end{itemize}
\begin{itemize}
\item {Grp. gram.:Pl.}
\end{itemize}
\begin{itemize}
\item {Proveniência:(Do gr. \textunderscore aer\textunderscore  + \textunderscore phone\textunderscore )}
\end{itemize}
Que canta no ar, (falando-se de aves).
Grande instrumento, recentemente inventado, semelhante a um órgão, e que, movido a vapor, produz sons de extraordinária fôrça.
Tribo ou grupo de pássaros cultirostros.
\section{Aeróforo}
\begin{itemize}
\item {fónica:a-e}
\end{itemize}
\begin{itemize}
\item {Grp. gram.:adj.}
\end{itemize}
\begin{itemize}
\item {Proveniência:(Do gr. \textunderscore aer\textunderscore  + \textunderscore phoros\textunderscore )}
\end{itemize}
O mesmo que \textunderscore aerifero\textunderscore .
\section{Aerófugo}
\begin{itemize}
\item {fónica:a-e}
\end{itemize}
\begin{itemize}
\item {Grp. gram.:adj.}
\end{itemize}
\begin{itemize}
\item {Proveniência:(Do lat. \textunderscore aer\textunderscore  + \textunderscore fugere\textunderscore )}
\end{itemize}
Que se oppõe á introducção do ar.
Impermeável ao ar.
\section{Aerogastrectasia}
\begin{itemize}
\item {fónica:a-e}
\end{itemize}
\begin{itemize}
\item {Grp. gram.:f.}
\end{itemize}
\begin{itemize}
\item {Proveniência:(Do gr. \textunderscore aer\textunderscore  + \textunderscore gaster\textunderscore  + \textunderscore ektasis\textunderscore )}
\end{itemize}
Dilatação do estômago, produzida pelos gases.
\section{Aerogênio}
\begin{itemize}
\item {fónica:a-e}
\end{itemize}
\begin{itemize}
\item {Grp. gram.:m.}
\end{itemize}
\begin{itemize}
\item {Proveniência:(Do gr. \textunderscore aer\textunderscore  + \textunderscore genes\textunderscore )}
\end{itemize}
Espécie de gás de illuminação, exposto e premiado na Exposição de Paris, em 1900, e já adoptado em muitas povoações.
\section{Aerognosia}
\begin{itemize}
\item {fónica:a-e}
\end{itemize}
\begin{itemize}
\item {Grp. gram.:f.}
\end{itemize}
\begin{itemize}
\item {Proveniência:(Do gr. \textunderscore aer\textunderscore  + \textunderscore gnosis\textunderscore )}
\end{itemize}
Parte da Phýsica, que trata das propriedades do ar.
\section{Aerognosta}
\begin{itemize}
\item {fónica:a-e}
\end{itemize}
\begin{itemize}
\item {Grp. gram.:m.}
\end{itemize}
Indivíduo, sabedor de aerognosia.
\section{Aerognóstico}
\begin{itemize}
\item {Grp. gram.:adj.}
\end{itemize}
Relativo á aerognosia.
\section{Aerografia}
\begin{itemize}
\item {fónica:a-e}
\end{itemize}
\begin{itemize}
\item {Grp. gram.:f.}
\end{itemize}
\begin{itemize}
\item {Proveniência:(De \textunderscore aerographo\textunderscore )}
\end{itemize}
Descripção do ar.
\section{Aerógrafo}
\begin{itemize}
\item {fónica:a-e}
\end{itemize}
\begin{itemize}
\item {Grp. gram.:m.}
\end{itemize}
\begin{itemize}
\item {Proveniência:(Do gr. \textunderscore aer\textunderscore  + \textunderscore graphein\textunderscore )}
\end{itemize}
O que descreve o ar.
\section{Aerograma}
\begin{itemize}
\item {fónica:a-e}
\end{itemize}
\begin{itemize}
\item {Grp. gram.:m.}
\end{itemize}
Communicação, feita pelo ar ou pela telegraphia sem fios.
(Do grego \textunderscore aer\textunderscore  + \textunderscore gramma\textunderscore )
\section{Aerogramma}
\begin{itemize}
\item {fónica:a-e}
\end{itemize}
\begin{itemize}
\item {Grp. gram.:m.}
\end{itemize}
Communicação, feita pelo ar ou pela telegraphia sem fios.
(Do grego \textunderscore aer\textunderscore  + \textunderscore gramma\textunderscore )
\section{Aerographia}
\begin{itemize}
\item {fónica:a-e}
\end{itemize}
\begin{itemize}
\item {Grp. gram.:f.}
\end{itemize}
\begin{itemize}
\item {Proveniência:(De \textunderscore aerographo\textunderscore )}
\end{itemize}
Descripção do ar.
\section{Aerógrapho}
\begin{itemize}
\item {fónica:a-e}
\end{itemize}
\begin{itemize}
\item {Grp. gram.:m.}
\end{itemize}
\begin{itemize}
\item {Proveniência:(Do gr. \textunderscore aer\textunderscore  + \textunderscore graphein\textunderscore )}
\end{itemize}
O que descreve o ar.
\section{Aerohidro}
\begin{itemize}
\item {fónica:a-e}
\end{itemize}
\begin{itemize}
\item {Grp. gram.:adj.}
\end{itemize}
\begin{itemize}
\item {Proveniência:(Do gr. \textunderscore aer\textunderscore  + \textunderscore hudros\textunderscore )}
\end{itemize}
Diz-se de um corpo, que em cavidade tubulada contém um líquido e uma bolha de ar.
\section{Aerohidropatia}
\begin{itemize}
\item {fónica:a-e}
\end{itemize}
\begin{itemize}
\item {Grp. gram.:f.}
\end{itemize}
\begin{itemize}
\item {Proveniência:(Do gr. \textunderscore aer\textunderscore  + \textunderscore hudros\textunderscore  + \textunderscore pathos\textunderscore )}
\end{itemize}
Tratamento de doenças, em que o ar e a água são principaes elementos.
\section{Aerohydro}
\begin{itemize}
\item {Grp. gram.:adj.}
\end{itemize}
\begin{itemize}
\item {Proveniência:(Do gr. \textunderscore aer\textunderscore  + \textunderscore hudros\textunderscore )}
\end{itemize}
Diz-se de um corpo, que em cavidade tubulada contém um líquido e uma bolha de ar.
\section{Aerohydropathia}
\begin{itemize}
\item {fónica:a-e}
\end{itemize}
\begin{itemize}
\item {Grp. gram.:f.}
\end{itemize}
\begin{itemize}
\item {Proveniência:(Do gr. \textunderscore aer\textunderscore  + \textunderscore hudros\textunderscore  + \textunderscore pathos\textunderscore )}
\end{itemize}
Tratamento de doenças, em que o ar e a água são principaes elementos.
\section{Aeroide}
\begin{itemize}
\item {fónica:a-e}
\end{itemize}
\begin{itemize}
\item {Grp. gram.:adj.}
\end{itemize}
\begin{itemize}
\item {Proveniência:(Do gr. \textunderscore aer\textunderscore  + \textunderscore eidos\textunderscore )}
\end{itemize}
Que é da natureza do ar, ou semelhante ao ar.
\section{Aeroides}
\begin{itemize}
\item {fónica:a-e}
\end{itemize}
\begin{itemize}
\item {Grp. gram.:adj.}
\end{itemize}
\begin{itemize}
\item {Proveniência:(Do gr. \textunderscore aer\textunderscore  + \textunderscore eidos\textunderscore )}
\end{itemize}
Que é da natureza do ar, ou semelhante ao ar.
\section{Aérola}
\begin{itemize}
\item {Grp. gram.:f.}
\end{itemize}
\begin{itemize}
\item {Proveniência:(Do lat. \textunderscore aer\textunderscore )}
\end{itemize}
Pústula cheia de ar.
Redoma de vidro transparente.
\section{Aerólita}
\begin{itemize}
\item {fónica:a-e}
\end{itemize}
\begin{itemize}
\item {Grp. gram.:f.}
\end{itemize}
(V.aerólito)
\section{Aerólite}
\begin{itemize}
\item {fónica:a-e}
\end{itemize}
\begin{itemize}
\item {Grp. gram.:f.}
\end{itemize}
(V.aerólito)
\section{Aerólitha}
\begin{itemize}
\item {fónica:a-e}
\end{itemize}
\begin{itemize}
\item {Grp. gram.:f.}
\end{itemize}
(V.aerólitho)
\section{Aerólithe}
\begin{itemize}
\item {fónica:a-e}
\end{itemize}
\begin{itemize}
\item {Grp. gram.:f.}
\end{itemize}
(V.aerólitho)
\section{Aerolíthico}
\begin{itemize}
\item {fónica:a-e}
\end{itemize}
\begin{itemize}
\item {Grp. gram.:adj.}
\end{itemize}
Que pertence aos aerólithos, ou que participa da sua natureza.
\section{Aerólitho}
\begin{itemize}
\item {fónica:a-e}
\end{itemize}
\begin{itemize}
\item {Grp. gram.:m.}
\end{itemize}
\begin{itemize}
\item {Proveniência:(Do gr. \textunderscore aer\textunderscore  + \textunderscore lithos\textunderscore )}
\end{itemize}
Pedra caída do ar.
Massa mineral, que, atravessando a atmosphera, cai sôbre a terra.
\section{Aerolítico}
\begin{itemize}
\item {fónica:a-e}
\end{itemize}
\begin{itemize}
\item {Grp. gram.:adj.}
\end{itemize}
Que pertence aos aerólithos, ou que participa da sua natureza.
\section{Aerólito}
\begin{itemize}
\item {fónica:a-e}
\end{itemize}
\begin{itemize}
\item {Grp. gram.:m.}
\end{itemize}
\begin{itemize}
\item {Proveniência:(Do gr. \textunderscore aer\textunderscore  + \textunderscore lithos\textunderscore )}
\end{itemize}
Pedra caída do ar.
Massa mineral, que, atravessando a atmosphera, cai sôbre a terra.
\section{Aerologia}
\begin{itemize}
\item {fónica:a-e}
\end{itemize}
\begin{itemize}
\item {Grp. gram.:f.}
\end{itemize}
\begin{itemize}
\item {Proveniência:(De \textunderscore aerólogo\textunderscore )}
\end{itemize}
Tratado do ar.
\section{Aerólogo}
\begin{itemize}
\item {fónica:a-e}
\end{itemize}
\begin{itemize}
\item {Grp. gram.:m.}
\end{itemize}
\begin{itemize}
\item {Proveniência:(Do gr. \textunderscore aer\textunderscore  + \textunderscore logos\textunderscore )}
\end{itemize}
Aquelle que trata do ar; o que estuda os phenómenos atmosphéricos.
\section{Aeromancia}
\begin{itemize}
\item {fónica:a-e}
\end{itemize}
\begin{itemize}
\item {Grp. gram.:f.}
\end{itemize}
\begin{itemize}
\item {Proveniência:(Do gr. \textunderscore aer\textunderscore  + \textunderscore manteia\textunderscore )}
\end{itemize}
Supposta arte de adivinhar, por meio da observação do ar.
\section{Aeromante}
\begin{itemize}
\item {fónica:a-e}
\end{itemize}
\begin{itemize}
\item {Grp. gram.:m.}
\end{itemize}
O que pratica a aeromancia.
\section{Aerometria}
\begin{itemize}
\item {fónica:a-e}
\end{itemize}
\begin{itemize}
\item {Grp. gram.:f.}
\end{itemize}
\begin{itemize}
\item {Proveniência:(De \textunderscore aerómetro\textunderscore )}
\end{itemize}
Sciência de medir a densidade dos elementos do ar.
\section{Aerómetro}
\begin{itemize}
\item {fónica:a-e}
\end{itemize}
\begin{itemize}
\item {Grp. gram.:m.}
\end{itemize}
\begin{itemize}
\item {Proveniência:(Do gr. \textunderscore aer\textunderscore  + \textunderscore metron\textunderscore )}
\end{itemize}
Instrumento, que indica o grau de condensação ou rarefacção do ar.
\section{Aeronaugrafia}
\begin{itemize}
\item {fónica:a-e}
\end{itemize}
\begin{itemize}
\item {Grp. gram.:f.}
\end{itemize}
Tratado de aeronáutica.
\section{Aeronaugráfico}
\begin{itemize}
\item {fónica:a-e}
\end{itemize}
\begin{itemize}
\item {Grp. gram.:adj.}
\end{itemize}
Relativo á aeronaugraphia.
\section{Aeronaugraphia}
\begin{itemize}
\item {fónica:a-e}
\end{itemize}
\begin{itemize}
\item {Grp. gram.:f.}
\end{itemize}
Tratado de aeronáutica.
\section{Aeronaugráphico}
\begin{itemize}
\item {fónica:a-e}
\end{itemize}
\begin{itemize}
\item {Grp. gram.:adj.}
\end{itemize}
Relativo á aeronaugraphia.
\section{Aeronauta}
\begin{itemize}
\item {fónica:a-e}
\end{itemize}
\begin{itemize}
\item {Grp. gram.:m.}
\end{itemize}
\begin{itemize}
\item {Proveniência:(Do gr. \textunderscore aer\textunderscore  + \textunderscore nautes\textunderscore )}
\end{itemize}
O que navega no ar.
O que sobe e anda no ar, em aeróstato.
\section{Aeronáutica}
\begin{itemize}
\item {fónica:a-e}
\end{itemize}
\begin{itemize}
\item {Grp. gram.:f.}
\end{itemize}
Arte de aeronauta.
\section{Aeronáutico}
\begin{itemize}
\item {fónica:a-e}
\end{itemize}
\begin{itemize}
\item {Grp. gram.:adj.}
\end{itemize}
Relativo á \textunderscore aeronáutica\textunderscore .
\section{Aeronave}
\begin{itemize}
\item {fónica:a-e}
\end{itemize}
\begin{itemize}
\item {Grp. gram.:f.}
\end{itemize}
O mesmo que \textunderscore aeróstato\textunderscore .
\section{Aeronavegação}
\begin{itemize}
\item {fónica:a-e}
\end{itemize}
\begin{itemize}
\item {Grp. gram.:f.}
\end{itemize}
Navegação aérea, ou arte de viajar em aeróstatos.
\section{Aeróphago}
\begin{itemize}
\item {fónica:a-e}
\end{itemize}
\begin{itemize}
\item {Grp. gram.:m.}
\end{itemize}
\begin{itemize}
\item {Proveniência:(Do gr. \textunderscore aer\textunderscore  + \textunderscore phagein\textunderscore )}
\end{itemize}
O mesmo que \textunderscore aerívoro\textunderscore .
\section{Aerophobia}
\begin{itemize}
\item {fónica:a-e}
\end{itemize}
\begin{itemize}
\item {Grp. gram.:f.}
\end{itemize}
\begin{itemize}
\item {Proveniência:(De \textunderscore aeróphobo\textunderscore )}
\end{itemize}
Doença, caracterizada pêlo horror ao ar.
\section{Aeróphobo}
\begin{itemize}
\item {fónica:a-e}
\end{itemize}
\begin{itemize}
\item {Grp. gram.:m.}
\end{itemize}
\begin{itemize}
\item {Proveniência:(Do gr. \textunderscore aer\textunderscore  + \textunderscore phobos\textunderscore )}
\end{itemize}
O que tem horror ao ar.
\section{Aerophónio}
\begin{itemize}
\item {fónica:a-e}
\end{itemize}
\begin{itemize}
\item {Grp. gram.:adj.}
\end{itemize}
\begin{itemize}
\item {Grp. gram.:M.}
\end{itemize}
\begin{itemize}
\item {Grp. gram.:Pl.}
\end{itemize}
\begin{itemize}
\item {Proveniência:(Do gr. \textunderscore aer\textunderscore  + \textunderscore phone\textunderscore )}
\end{itemize}
Que canta no ar, (falando-se de aves).
Grande instrumento, recentemente inventado, semelhante a um órgão, e que, movido a vapor, produz sons de extraordinária fôrça.
Tribo ou grupo de pássaros cultirostros.
\section{Aeróphoro}
\begin{itemize}
\item {fónica:a-e}
\end{itemize}
\begin{itemize}
\item {Grp. gram.:adj.}
\end{itemize}
\begin{itemize}
\item {Proveniência:(Do gr. \textunderscore aer\textunderscore  + \textunderscore phoros\textunderscore )}
\end{itemize}
O mesmo que \textunderscore aerifero\textunderscore .
\section{Aeróphyta}
\begin{itemize}
\item {fónica:a-e}
\end{itemize}
\begin{itemize}
\item {Grp. gram.:f.}
\end{itemize}
\begin{itemize}
\item {Proveniência:(Do gr. \textunderscore aer\textunderscore  + \textunderscore phuton\textunderscore )}
\end{itemize}
O mesmo que \textunderscore aérida\textunderscore , segundo Lamouroux.
\section{Aeróphyto}
\begin{itemize}
\item {fónica:a-e}
\end{itemize}
\begin{itemize}
\item {Grp. gram.:adj.}
\end{itemize}
\begin{itemize}
\item {Proveniência:(Do gr. \textunderscore aer\textunderscore  + \textunderscore phuton\textunderscore )}
\end{itemize}
Diz-se da planta que vive no ar.
\section{Aeroplano}
\begin{itemize}
\item {fónica:a-e}
\end{itemize}
\begin{itemize}
\item {Grp. gram.:m.}
\end{itemize}
\begin{itemize}
\item {Proveniência:(De \textunderscore aéreo\textunderscore  + \textunderscore plano\textunderscore )}
\end{itemize}
Apparelho aerostático, movido a vapor, e sustentado sobre planos ou lâminas, postos em acção por um motor da fôrça de um cavallo.
Foi inventado recentemente, em 1896, por Langley.
\section{Aeroposta}
\begin{itemize}
\item {fónica:a-e}
\end{itemize}
\begin{itemize}
\item {Grp. gram.:f.}
\end{itemize}
\begin{itemize}
\item {Proveniência:(De \textunderscore aéreo\textunderscore ^1 + \textunderscore posta\textunderscore )}
\end{itemize}
Correio aéreo.
Apparelho, que consta de um tubo, em que se move uma caixa com cartas, e que faz serviço postal.
\section{Aeroscopia}
\begin{itemize}
\item {fónica:a-e}
\end{itemize}
\begin{itemize}
\item {Grp. gram.:f.}
\end{itemize}
Applicação do aeroscópio.
\section{Aeroscópio}
\begin{itemize}
\item {fónica:a-e}
\end{itemize}
\begin{itemize}
\item {Grp. gram.:m.}
\end{itemize}
\begin{itemize}
\item {Proveniência:(Do gr. \textunderscore aer\textunderscore  + \textunderscore skopein\textunderscore )}
\end{itemize}
Instrumento de Phýsica, para fazer observações no ar.
\section{Aerosfera}
\begin{itemize}
\item {fónica:a-e}
\end{itemize}
\begin{itemize}
\item {Grp. gram.:f.}
\end{itemize}
\begin{itemize}
\item {Proveniência:(Do gr. \textunderscore aer\textunderscore  + \textunderscore sphaira\textunderscore )}
\end{itemize}
O mesmo que \textunderscore atmosphera\textunderscore .
\section{Aerosístilo}
\begin{itemize}
\item {fónica:a-e}
\end{itemize}
\begin{itemize}
\item {Grp. gram.:m.}
\end{itemize}
\begin{itemize}
\item {Utilização:Archit.}
\end{itemize}
Systema de intercolúmnios, que consta de columnas duplas, separadas as de cada grupo pelo intervallo de um módulo, ficando cada grupo de duas columnas separado de outro pelo espaço de sete módulos.
\section{Aerosphera}
\begin{itemize}
\item {fónica:a-e}
\end{itemize}
\begin{itemize}
\item {Grp. gram.:f.}
\end{itemize}
\begin{itemize}
\item {Proveniência:(Do gr. \textunderscore aer\textunderscore  + \textunderscore sphaira\textunderscore )}
\end{itemize}
O mesmo que \textunderscore atmosphera\textunderscore .
\section{Aerostação}
\begin{itemize}
\item {fónica:a-e}
\end{itemize}
\begin{itemize}
\item {Grp. gram.:f.}
\end{itemize}
Arte de empregar o aeróstato.
\section{Aeróstata}
\begin{itemize}
\item {Grp. gram.:m.}
\end{itemize}
O mesmo que \textunderscore aeróstato\textunderscore . Cf. Castilho, \textunderscore Fastos\textunderscore , I, 560 e 562.
\section{Aerostática}
\begin{itemize}
\item {fónica:a-e}
\end{itemize}
\begin{itemize}
\item {Grp. gram.:f.}
\end{itemize}
Parte da Phýsica, em que se estudam as leis do equilíbrio do ar.
(Fem. de \textunderscore aerostático\textunderscore )
\section{Aerostático}
\begin{itemize}
\item {fónica:a-e}
\end{itemize}
\begin{itemize}
\item {Grp. gram.:adj.}
\end{itemize}
Relativo a aeróstatos, á aerostação.
\section{Aeróstato}
\begin{itemize}
\item {fónica:a-e}
\end{itemize}
\begin{itemize}
\item {Grp. gram.:m.}
\end{itemize}
\begin{itemize}
\item {Utilização:Ext.}
\end{itemize}
\begin{itemize}
\item {Proveniência:(Do gr. \textunderscore aer\textunderscore  + \textunderscore statos\textunderscore )}
\end{itemize}
Balão, que se enche de ar aquecido ou de gás mais leve que o ar, e que por isso se eleva e se sustém na atmosphera.
Qualquer vehículo ou apparelho, que, dirigido por alguém, se eleva na atmosphera e a percorre. Cp. \textunderscore aerostática\textunderscore .
\section{Aerosýstylo}
\begin{itemize}
\item {fónica:a-e}
\end{itemize}
\begin{itemize}
\item {Grp. gram.:m.}
\end{itemize}
\begin{itemize}
\item {Utilização:Archit.}
\end{itemize}
Systema de intercolúmnios, que consta de columnas duplas, separadas as de cada grupo pelo intervallo de um módulo, ficando cada grupo de duas columnas separado de outro pelo espaço de sete módulos.
\section{Aerotechnia}
\begin{itemize}
\item {fónica:a-e}
\end{itemize}
\begin{itemize}
\item {Grp. gram.:f.}
\end{itemize}
\begin{itemize}
\item {Proveniência:(Do gr. \textunderscore aer\textunderscore  + \textunderscore tekhne\textunderscore )}
\end{itemize}
Sciência, que trata do ar ou das suas applicações á indústria.
\section{Aerotéchnico}
\begin{itemize}
\item {fónica:a-e}
\end{itemize}
\begin{itemize}
\item {Grp. gram.:adj.}
\end{itemize}
Relativo á \textunderscore aerotechnia\textunderscore .
\section{Aerotecnia}
\begin{itemize}
\item {fónica:a-e}
\end{itemize}
\begin{itemize}
\item {Grp. gram.:f.}
\end{itemize}
\begin{itemize}
\item {Proveniência:(Do gr. \textunderscore aer\textunderscore  + \textunderscore tekhne\textunderscore )}
\end{itemize}
Sciência, que trata do ar ou das suas applicações á indústria.
\section{Aerotécnico}
\begin{itemize}
\item {fónica:a-e}
\end{itemize}
\begin{itemize}
\item {Grp. gram.:adj.}
\end{itemize}
Relativo á \textunderscore aerotechnia\textunderscore .
\section{Aeroterapeutica}
\begin{itemize}
\item {Grp. gram.:f.}
\end{itemize}
O mesmo que \textunderscore aeroterapia\textunderscore .
\section{Aeroterapia}
\begin{itemize}
\item {fónica:a-e}
\end{itemize}
\begin{itemize}
\item {Grp. gram.:f.}
\end{itemize}
\begin{itemize}
\item {Proveniência:(Do gr. \textunderscore aer\textunderscore  + \textunderscore therapia\textunderscore )}
\end{itemize}
Applicação do ar ao tratamento de doenças.
\section{Aeroterápico}
\begin{itemize}
\item {fónica:a-e}
\end{itemize}
\begin{itemize}
\item {Grp. gram.:adj.}
\end{itemize}
Relativo á \textunderscore aeroterapia\textunderscore .
\section{Aerotermo}
\begin{itemize}
\item {fónica:a-e}
\end{itemize}
\begin{itemize}
\item {Grp. gram.:m.}
\end{itemize}
\begin{itemize}
\item {Proveniência:(Do gr. \textunderscore aer\textunderscore  + \textunderscore thermos\textunderscore )}
\end{itemize}
Apparelho, em que se emprega o ar aquecido, para servir, por sua vez, de meio de aquecimento. Cf. Castilho, \textunderscore Fastos\textunderscore , III, 482.
\section{Aerotherapêutica}
\begin{itemize}
\item {Grp. gram.:f.}
\end{itemize}
O mesmo que \textunderscore aerotherapia\textunderscore .
\section{Aerotherapia}
\begin{itemize}
\item {Grp. gram.:f.}
\end{itemize}
\begin{itemize}
\item {Proveniência:(Do gr. \textunderscore aer\textunderscore  + \textunderscore therapia\textunderscore )}
\end{itemize}
Applicação do ar ao tratamento de doenças.
\section{Aerotherápico}
\begin{itemize}
\item {fónica:a-e}
\end{itemize}
\begin{itemize}
\item {Grp. gram.:adj.}
\end{itemize}
Relativo á \textunderscore aerotherapia\textunderscore .
\section{Aerothermo}
\begin{itemize}
\item {fónica:a-e}
\end{itemize}
\begin{itemize}
\item {Grp. gram.:m.}
\end{itemize}
\begin{itemize}
\item {Proveniência:(Do gr. \textunderscore aer\textunderscore  + \textunderscore thermos\textunderscore )}
\end{itemize}
Apparelho, em que se emprega o ar aquecido, para servir, por sua vez, de meio de aquecimento. Cf. Castilho, \textunderscore Fastos\textunderscore , III, 482.
\section{Aerozoários}
\begin{itemize}
\item {fónica:a-e}
\end{itemize}
\begin{itemize}
\item {Grp. gram.:m. pl.}
\end{itemize}
\begin{itemize}
\item {Proveniência:(Do gr. \textunderscore aer\textunderscore  + \textunderscore zoon\textunderscore )}
\end{itemize}
Animaes, a cuja existência é indispensável o ar.
Os vertebrados e os articulados.
\section{Aerva}
\begin{itemize}
\item {Grp. gram.:f.}
\end{itemize}
Gênero de plantas amarantáceas.
\section{...aes}
\begin{itemize}
\item {Grp. gram.:suf.}
\end{itemize}
\textunderscore pl.\textunderscore  de \textunderscore ...al\textunderscore .
\section{...ães}
\begin{itemize}
\item {Grp. gram.:suf.}
\end{itemize}
\textunderscore pl.\textunderscore  de vários subst. e adj. terminados em \textunderscore ão\textunderscore .
\section{Aethrioscópio}
\begin{itemize}
\item {fónica:a-e}
\end{itemize}
\begin{itemize}
\item {Grp. gram.:m.}
\end{itemize}
\begin{itemize}
\item {Proveniência:(Do gr. \textunderscore aithria\textunderscore  + \textunderscore skopein\textunderscore )}
\end{itemize}
Instrumento, para medir o calor que irradia da terra.
\section{Aetite}
\begin{itemize}
\item {fónica:a-e}
\end{itemize}
\begin{itemize}
\item {Grp. gram.:f.}
\end{itemize}
\begin{itemize}
\item {Proveniência:(Gr. \textunderscore aetites\textunderscore )}
\end{itemize}
Pedra ôca, conhecida também por pedra de águia, por se suppor que apparecia em o ninho das águias.
\section{Aetrioscópio}
\begin{itemize}
\item {fónica:a-e}
\end{itemize}
\begin{itemize}
\item {Grp. gram.:m.}
\end{itemize}
\begin{itemize}
\item {Proveniência:(Do gr. \textunderscore aithria\textunderscore  + \textunderscore skopein\textunderscore )}
\end{itemize}
Instrumento, para medir o calor que irradia da terra.
\section{Afadigadamente}
\begin{itemize}
\item {Grp. gram.:adv.}
\end{itemize}
De modo \textunderscore afadigado\textunderscore .
Com trabalho.
\section{Afadigado}
\begin{itemize}
\item {Grp. gram.:adj.}
\end{itemize}
\begin{itemize}
\item {Proveniência:(De \textunderscore afadigar\textunderscore )}
\end{itemize}
Que tem fadiga.
Que mostra fadiga ou cansaço.
\section{Afadigador}
\begin{itemize}
\item {Grp. gram.:m.}
\end{itemize}
O que afadiga.
\section{Afadigar}
\begin{itemize}
\item {Grp. gram.:v. t.}
\end{itemize}
Causar fadiga a; fatigar; cansar.
\section{Afadigoso}
\begin{itemize}
\item {Grp. gram.:adj.}
\end{itemize}
Que afadiga.
\section{Afadistado}
\begin{itemize}
\item {Grp. gram.:adj.}
\end{itemize}
Que tem modos ou hábitos de fadista.
\section{Afadistar-se}
\begin{itemize}
\item {Grp. gram.:v. p.}
\end{itemize}
Tornar-se fadista; adquirir hábitos de fadista.
\section{Afagadeiro}
\begin{itemize}
\item {Grp. gram.:adj.}
\end{itemize}
O mesmo que \textunderscore afagante\textunderscore .
\section{Afagador}
\begin{itemize}
\item {Grp. gram.:m.  e  adj.}
\end{itemize}
O que afaga.
\section{Afagamento}
\begin{itemize}
\item {Grp. gram.:m.}
\end{itemize}
\begin{itemize}
\item {Utilização:Des.}
\end{itemize}
Acto de \textunderscore afagar\textunderscore .
Afago.
\section{Afagante}
\begin{itemize}
\item {Grp. gram.:adj.}
\end{itemize}
Que afaga.
\section{Afagar}
\begin{itemize}
\item {Grp. gram.:v. t.}
\end{itemize}
\begin{itemize}
\item {Proveniência:(Do lat. \textunderscore ad faciem lecare\textunderscore , segundo Cornu. Cp. cast. \textunderscore halagar\textunderscore )}
\end{itemize}
Tratar com afago; acariciar.
Conservar com prazer: \textunderscore afagar esperanças\textunderscore .
Desbastar saliências de; alisar.
\section{Afago}
\begin{itemize}
\item {Grp. gram.:m.}
\end{itemize}
Acto de \textunderscore afagar\textunderscore .
Carícia.
Mimo.
\section{Afagoso}
\begin{itemize}
\item {Grp. gram.:adj.}
\end{itemize}
\begin{itemize}
\item {Utilização:Ant.}
\end{itemize}
Que afaga; afagador.
\section{Afagueiro}
\begin{itemize}
\item {Grp. gram.:adj.}
\end{itemize}
O mesmo que \textunderscore afagoso\textunderscore .
\section{Afalado}
\begin{itemize}
\item {Grp. gram.:adj.}
\end{itemize}
Diz-se do animal que entende as falas e se dirige por ellas.
\section{Afalcassar}
\begin{itemize}
\item {Grp. gram.:v.}
\end{itemize}
\begin{itemize}
\item {Utilização:t. Náut.}
\end{itemize}
Dar voltas com o fio em (chicotes dos cabos), para se não descocharem.
\section{Afalcoado}
\begin{itemize}
\item {Grp. gram.:adj.}
\end{itemize}
\begin{itemize}
\item {Utilização:Prov.}
\end{itemize}
\begin{itemize}
\item {Utilização:alent.}
\end{itemize}
\begin{itemize}
\item {Utilização:Prov.}
\end{itemize}
\begin{itemize}
\item {Proveniência:(De \textunderscore afalcoar\textunderscore )}
\end{itemize}
Fatigado.
Adoentado.
Desfalcado (Colhido em Turquel).
\section{Afalcoar}
\begin{itemize}
\item {Grp. gram.:v. i.}
\end{itemize}
\begin{itemize}
\item {Utilização:Prov.}
\end{itemize}
\begin{itemize}
\item {Utilização:alent.}
\end{itemize}
\begin{itemize}
\item {Proveniência:(De \textunderscore falcão\textunderscore , por allusão ás caçadas de altanaria?)}
\end{itemize}
Sentir-se fatigado.
Interromper a marcha, em consequência de cansaço.
\section{Afaluado}
\begin{itemize}
\item {Grp. gram.:adj.}
\end{itemize}
\begin{itemize}
\item {Utilização:Bras}
\end{itemize}
Azafamado; esbaforido.
\section{Afamadamente}
\begin{itemize}
\item {Grp. gram.:adv.}
\end{itemize}
De modo \textunderscore afamado\textunderscore .
\section{Afamado}
\begin{itemize}
\item {Grp. gram.:adj.}
\end{itemize}
Que tem fama; célebre.
\section{Afamador}
\begin{itemize}
\item {Grp. gram.:m.}
\end{itemize}
O que dá bôa fama, o que nobilita.
\section{Afamar}
\begin{itemize}
\item {Grp. gram.:v. t.}
\end{itemize}
Dar fama a.
Tornar célebre.
\section{Afamilhar-se}
\begin{itemize}
\item {Grp. gram.:v. p.}
\end{itemize}
\begin{itemize}
\item {Utilização:Bras}
\end{itemize}
\begin{itemize}
\item {Proveniência:(De \textunderscore familha\textunderscore , por \textunderscore família\textunderscore . Cp. fr. \textunderscore famille\textunderscore )}
\end{itemize}
Têr muitos filhos; encher-se de família.
\section{Afan}
\begin{itemize}
\item {Grp. gram.:m.}
\end{itemize}
Ânsia.
Cuidado.
Trabalho.
(Cp. \textunderscore afano\textunderscore )
\section{Afanar}
\begin{itemize}
\item {Grp. gram.:v. i.  e  p.}
\end{itemize}
\begin{itemize}
\item {Grp. gram.:V. t.}
\end{itemize}
Trabalhar com afan.
Afadigar-se.
Buscar, adquirir, com afan.
(B. lat. \textunderscore ahanare\textunderscore )
\section{Afanchonado}
\begin{itemize}
\item {Grp. gram.:adj.}
\end{itemize}
Inclinado aos vícios de fanchono.
\section{Afandangado}
\begin{itemize}
\item {Grp. gram.:adj.}
\end{itemize}
Parecido ao fandango.
\section{Afano}
\begin{itemize}
\item {Grp. gram.:m.}
\end{itemize}
\begin{itemize}
\item {Proveniência:(De \textunderscore afanar\textunderscore )}
\end{itemize}
O mesmo que \textunderscore afan\textunderscore . Cf. Filinto, VII, 202.
\section{Afanosamente}
\begin{itemize}
\item {Grp. gram.:adv.}
\end{itemize}
\begin{itemize}
\item {Proveniência:(De \textunderscore afanoso\textunderscore )}
\end{itemize}
Com afan.
\section{Afanoso}
\begin{itemize}
\item {Grp. gram.:adj.}
\end{itemize}
Cheio de afan.
Trabalhoso.
\section{Afão}
\begin{itemize}
\item {Grp. gram.:m.}
\end{itemize}
\begin{itemize}
\item {Utilização:P. us.}
\end{itemize}
O mesmo que \textunderscore afan\textunderscore .
\section{Afaragatar}
\begin{itemize}
\item {Grp. gram.:v. t.}
\end{itemize}
\begin{itemize}
\item {Utilização:Prov.}
\end{itemize}
\begin{itemize}
\item {Utilização:trasm.}
\end{itemize}
Attrahir a casa, afazer a ella (cães, gatos, rapazes, etc.).
(Cp. \textunderscore afragatar-se\textunderscore )
\section{Afarvar-se}
\begin{itemize}
\item {Grp. gram.:v. p.}
\end{itemize}
\begin{itemize}
\item {Utilização:Prov.}
\end{itemize}
\begin{itemize}
\item {Utilização:trasm.}
\end{itemize}
\begin{itemize}
\item {Utilização:Prov.}
\end{itemize}
Afanar-se.
Apressar-se.
Mostrar-se atarefado.
Apanhar calor demasiado.
(Por \textunderscore afervar-se\textunderscore , de \textunderscore fervor\textunderscore ?)
\section{Afasta!}
\begin{itemize}
\item {Grp. gram.:interj.}
\end{itemize}
(Imper. de \textunderscore afastar\textunderscore )
\section{Afastadamente}
\begin{itemize}
\item {Grp. gram.:adv.}
\end{itemize}
De modo \textunderscore afastado\textunderscore .
\section{Afastado}
\begin{itemize}
\item {Grp. gram.:adj.}
\end{itemize}
\begin{itemize}
\item {Proveniência:(De \textunderscore afastar\textunderscore )}
\end{itemize}
Distante: \textunderscore terras afastadas\textunderscore .
Que passou há muito: \textunderscore tempos afastados\textunderscore .
Diz-se do parente, que o não é nos primeiros graus: \textunderscore parente afastado\textunderscore .
\section{Afastador}
\begin{itemize}
\item {Grp. gram.:m.}
\end{itemize}
O que afasta.
\section{Afastamento}
\begin{itemize}
\item {Grp. gram.:m.}
\end{itemize}
Acto de \textunderscore afastar\textunderscore .
\section{Afastar}
\begin{itemize}
\item {Grp. gram.:v. t.}
\end{itemize}
\begin{itemize}
\item {Proveniência:(Do ant. cast. \textunderscore fasta\textunderscore )}
\end{itemize}
Desviar.
Apartar.
Tirar para longe.
\section{Afatiar}
\begin{itemize}
\item {Grp. gram.:v. t.}
\end{itemize}
Cortar em fatias.
Retalhar.
\section{Afazendar-se}
\begin{itemize}
\item {Grp. gram.:v. p.}
\end{itemize}
Adquirir ou têr fazendas.
Enriquecer.
\section{Afazer}
\begin{itemize}
\item {Grp. gram.:v. t.}
\end{itemize}
\begin{itemize}
\item {Grp. gram.:M.}
\end{itemize}
\begin{itemize}
\item {Utilização:Prov.}
\end{itemize}
\begin{itemize}
\item {Utilização:trasm.}
\end{itemize}
\begin{itemize}
\item {Proveniência:(De \textunderscore fazer\textunderscore )}
\end{itemize}
Acostumar; habituar.
Aclimar.
Trabalho agrícola.
\section{Afazeres}
\begin{itemize}
\item {Grp. gram.:m. pl.}
\end{itemize}
\begin{itemize}
\item {Utilização:Gal}
\end{itemize}
\begin{itemize}
\item {Proveniência:(Do fr. \textunderscore affaire\textunderscore )}
\end{itemize}
Negócios; occupações, faina.
\section{Afazimento}
\begin{itemize}
\item {Grp. gram.:m.}
\end{itemize}
\begin{itemize}
\item {Utilização:Ant.}
\end{itemize}
\begin{itemize}
\item {Proveniência:(De \textunderscore afazer\textunderscore )}
\end{itemize}
Communicação deshonesta; cóito.
Costume, hábito.
\section{Afeadamente}
\begin{itemize}
\item {Grp. gram.:adv.}
\end{itemize}
De modo \textunderscore afeado\textunderscore .
Com fealdade.
\section{Afeado}
\begin{itemize}
\item {Grp. gram.:adj.}
\end{itemize}
\begin{itemize}
\item {Proveniência:(De \textunderscore afear\textunderscore )}
\end{itemize}
Que se tornou feio.
Um tanto feio.
\section{Afeador}
\begin{itemize}
\item {Grp. gram.:m.}
\end{itemize}
O que afeia.
\section{Afeamento}
\begin{itemize}
\item {Grp. gram.:m.}
\end{itemize}
Acto de \textunderscore afear\textunderscore .
\section{Afear}
\begin{itemize}
\item {Grp. gram.:v. t.}
\end{itemize}
\begin{itemize}
\item {Proveniência:(Do ant. \textunderscore fêo\textunderscore ==\textunderscore feio\textunderscore )}
\end{itemize}
Tornar feio.
\section{Afábil}
\begin{itemize}
\item {Grp. gram.:adj.}
\end{itemize}
\begin{itemize}
\item {Utilização:Ant.}
\end{itemize}
(V.affável)
\section{Afabilidade}
\begin{itemize}
\item {Grp. gram.:f.}
\end{itemize}
\begin{itemize}
\item {Proveniência:(Lat. \textunderscore affabilitas\textunderscore )}
\end{itemize}
Qualidade do que é afável.
\section{Afável}
\begin{itemize}
\item {Grp. gram.:adj.}
\end{itemize}
\begin{itemize}
\item {Proveniência:(Lat. \textunderscore affabilis\textunderscore )}
\end{itemize}
Benévolo.
Cortês; delicado.
\section{Afavelmente}
\begin{itemize}
\item {Grp. gram.:adv.}
\end{itemize}
De modo \textunderscore afável\textunderscore , com afabilidade.
\section{Afecção}
\begin{itemize}
\item {Grp. gram.:f.}
\end{itemize}
\begin{itemize}
\item {Proveniência:(Lat. \textunderscore affectio\textunderscore )}
\end{itemize}
Phenómeno mórbido do organismo animal.
\section{Afectação}
\begin{itemize}
\item {Grp. gram.:f.}
\end{itemize}
\begin{itemize}
\item {Proveniência:(Lat. \textunderscore affectatio\textunderscore )}
\end{itemize}
Acto ou effeito de \textunderscore afectar\textunderscore .
Fingimento.
Presumpção, vaidade.
\section{Afectadamente}
\begin{itemize}
\item {Grp. gram.:adv.}
\end{itemize}
De modo \textunderscore afectado\textunderscore .
\section{Afectado}
\begin{itemize}
\item {Grp. gram.:adj.}
\end{itemize}
\begin{itemize}
\item {Utilização:Bras}
\end{itemize}
Que tem afectação.
Presumido.
Pretensioso.
Tísico, tuberculoso.
\section{Afectante}
\begin{itemize}
\item {Grp. gram.:adj.}
\end{itemize}
Que afecta ou finge o que não é.
\section{Afectar}
\begin{itemize}
\item {Grp. gram.:v. t.}
\end{itemize}
\begin{itemize}
\item {Grp. gram.:V. p.}
\end{itemize}
\begin{itemize}
\item {Proveniência:(Do lat. \textunderscore affectare\textunderscore )}
\end{itemize}
Fingir.
Têr disposição para offender.
Esmerar-se ridiculamente.
\section{Afectativo}
\begin{itemize}
\item {Grp. gram.:adj.}
\end{itemize}
\begin{itemize}
\item {Utilização:Des.}
\end{itemize}
Desejoso.
\section{Afectivamente}
\begin{itemize}
\item {Grp. gram.:adv.}
\end{itemize}
De modo \textunderscore afectivo\textunderscore , com afecto.
\section{Afectividade}
\begin{itemize}
\item {Grp. gram.:f.}
\end{itemize}
Faculdade, relativa aos sentimentos afectivos.
\section{Afectivo}
\begin{itemize}
\item {Grp. gram.:adj.}
\end{itemize}
\begin{itemize}
\item {Proveniência:(Lat. \textunderscore affectivus\textunderscore )}
\end{itemize}
Relativo a \textunderscore afecto\textunderscore .
Que mostra afecto; afectuoso.
\section{Afecto}
\begin{itemize}
\item {Grp. gram.:m.}
\end{itemize}
\begin{itemize}
\item {Grp. gram.:Adj.}
\end{itemize}
\begin{itemize}
\item {Proveniência:(Lat. \textunderscore affectus\textunderscore )}
\end{itemize}
Sentimento de inclinação para alguém.
Amizade.
Sympathia.
Paixão.
Dedicado.
Afeiçoado.
Incumbido.
Pendente ou dependente de resolução superior: \textunderscore o assumpto está affecto ao Ministro da Guerra\textunderscore .
\section{Afectuosamente}
\begin{itemize}
\item {Grp. gram.:adv.}
\end{itemize}
De modo \textunderscore afectuoso\textunderscore .
\section{Afectuoso}
\begin{itemize}
\item {Grp. gram.:adj.}
\end{itemize}
\begin{itemize}
\item {Proveniência:(Lat. \textunderscore affectuosus\textunderscore )}
\end{itemize}
Que tem afecto.
\section{Afegane}
\begin{itemize}
\item {Grp. gram.:adj.}
\end{itemize}
\begin{itemize}
\item {Grp. gram.:M.}
\end{itemize}
Relativo ao Afeganistão.
Habitante do Afeganistão.
Língua dêste país.
\section{Afeição}
\begin{itemize}
\item {Grp. gram.:f.}
\end{itemize}
\begin{itemize}
\item {Proveniência:(Lat. \textunderscore affectio\textunderscore )}
\end{itemize}
Affecto.
\section{Afeiçoadamente}
\begin{itemize}
\item {Grp. gram.:adv.}
\end{itemize}
De modo \textunderscore affeiçoado\textunderscore .
\section{Afeiçoado}
\begin{itemize}
\item {Grp. gram.:adj.}
\end{itemize}
\begin{itemize}
\item {Proveniência:(De \textunderscore afeiçoar\textunderscore )}
\end{itemize}
Que tem certa feição.
Apropriado.
\section{Afeiçoado}
\begin{itemize}
\item {Grp. gram.:adj.}
\end{itemize}
\begin{itemize}
\item {Proveniência:(De \textunderscore affeiçoar\textunderscore )}
\end{itemize}
Que tem afeição (a alguém ou a alguma coisa).
Inclinado (a certas ideias ou systema).
\section{Afeiçoador}
\begin{itemize}
\item {Grp. gram.:m.}
\end{itemize}
O que afeiçôa.
\section{Afeiçoamento}
\begin{itemize}
\item {Grp. gram.:m.}
\end{itemize}
Acto de \textunderscore afeiçoar\textunderscore .
\section{Afeiçoar}
\begin{itemize}
\item {Grp. gram.:v. t.}
\end{itemize}
Dar feição a.
Formar.
Adaptar, apropriar.
\section{Afeiçoar}
\begin{itemize}
\item {Grp. gram.:v. t.}
\end{itemize}
\begin{itemize}
\item {Grp. gram.:V. i.}
\end{itemize}
\begin{itemize}
\item {Grp. gram.:V. p.}
\end{itemize}
Tomar afeição a.
Adquirir afeição.
Tomar afeição.
\section{Afeitar}
\begin{itemize}
\item {Grp. gram.:v. t.}
\end{itemize}
\begin{itemize}
\item {Utilização:Ant.}
\end{itemize}
O mesmo que \textunderscore enfeitar\textunderscore .
\section{Afeitar}
\begin{itemize}
\item {Grp. gram.:v. t.}
\end{itemize}
\begin{itemize}
\item {Utilização:Ant.}
\end{itemize}
O mesmo que \textunderscore afectar\textunderscore .
\section{Afeite}
\begin{itemize}
\item {Grp. gram.:m.}
\end{itemize}
O mesmo que \textunderscore enfeite\textunderscore .
Arrebique pretensioso:«\textunderscore versos que são afeites e desdoiros do estilo\textunderscore ». Castilho, \textunderscore Primavera\textunderscore , (notas) Cf. \textunderscore Eufrosina\textunderscore , 294; Ferreira, \textunderscore Cioso\textunderscore , III; \textunderscore Peregrinação\textunderscore , Vieira, etc.
\section{Afeito}
\begin{itemize}
\item {Grp. gram.:adj.}
\end{itemize}
\begin{itemize}
\item {Proveniência:(De \textunderscore afazer\textunderscore )}
\end{itemize}
Acostumado, habituado.
\section{A-feito}
\begin{itemize}
\item {Grp. gram.:loc. adv.}
\end{itemize}
\begin{itemize}
\item {Utilização:Prov.}
\end{itemize}
\begin{itemize}
\item {Utilização:minh.}
\end{itemize}
A eito, a fio, seguidamente.
\section{Afelear}
\begin{itemize}
\item {Grp. gram.:v. t.}
\end{itemize}
\begin{itemize}
\item {Utilização:Fig.}
\end{itemize}
Misturar com fel.
Dar fel a.
Desgostar.
\section{Afemear}
\begin{itemize}
\item {Grp. gram.:v. t.}
\end{itemize}
\begin{itemize}
\item {Utilização:Des.}
\end{itemize}
O mesmo que \textunderscore effeminar\textunderscore .
\section{Afemençar}
\begin{itemize}
\item {Grp. gram.:v. t.}
\end{itemize}
\begin{itemize}
\item {Utilização:Ant.}
\end{itemize}
Avistar, vêr.
\section{Afeminar}
\textunderscore v. t.\textunderscore  (e der.)
(V. \textunderscore effeminar\textunderscore , etc.)
\section{Aferência}
\begin{itemize}
\item {Grp. gram.:f.}
\end{itemize}
Qualidade de aferente. Cf. Castilho, \textunderscore Fastos\textunderscore , I, 139.
\section{Aferente}
\begin{itemize}
\item {Grp. gram.:adj.}
\end{itemize}
\begin{itemize}
\item {Proveniência:(Lat. \textunderscore afferens\textunderscore )}
\end{itemize}
Que conduz, que leva.
\section{Afergulhar-se}
\begin{itemize}
\item {Grp. gram.:v. t.}
\end{itemize}
\begin{itemize}
\item {Utilização:Prov.}
\end{itemize}
\begin{itemize}
\item {Utilização:trasm.}
\end{itemize}
O mesmo que \textunderscore afarvar-se\textunderscore .
\section{Aferição}
\begin{itemize}
\item {Grp. gram.:f.}
\end{itemize}
Acto ou effeito de \textunderscore aferir\textunderscore .
\section{Aferido}
\begin{itemize}
\item {Grp. gram.:m.}
\end{itemize}
Cale do moinho, caneiro de água para mover a azenha.
\section{Aferidor}
\begin{itemize}
\item {Grp. gram.:m.}
\end{itemize}
O que afere.
\section{Aferimento}
\begin{itemize}
\item {Grp. gram.:m.}
\end{itemize}
O mesmo que \textunderscore aferição\textunderscore .
\section{Aferir}
\begin{itemize}
\item {Grp. gram.:v. t.}
\end{itemize}
Conferir; cotejar, comparar.
(Refl. de \textunderscore conferir\textunderscore ?)
\section{Aferradamente}
\begin{itemize}
\item {Grp. gram.:adv.}
\end{itemize}
De modo \textunderscore aferrado\textunderscore .
\section{Aferrado}
\begin{itemize}
\item {Grp. gram.:adj.}
\end{itemize}
\begin{itemize}
\item {Proveniência:(De \textunderscore aferrar\textunderscore )}
\end{itemize}
Teimoso; obstinado.
\section{Aferrador}
\begin{itemize}
\item {Grp. gram.:m.}
\end{itemize}
\begin{itemize}
\item {Utilização:Bras}
\end{itemize}
Ave, o mesmo que \textunderscore ferrador\textunderscore  ou \textunderscore araponga\textunderscore .
\section{Aferramento}
\begin{itemize}
\item {Grp. gram.:m.}
\end{itemize}
O mesmo que \textunderscore afêrro\textunderscore .
\section{Aferrar}
\begin{itemize}
\item {Grp. gram.:v. t.}
\end{itemize}
\begin{itemize}
\item {Grp. gram.:V. i.}
\end{itemize}
\begin{itemize}
\item {Utilização:Ant.}
\end{itemize}
\begin{itemize}
\item {Grp. gram.:V. p.}
\end{itemize}
Prender com ferro.
Segurar.
Lançar arpão.
Teimar, obstinar-se.
\section{Aferrenhadamente}
\begin{itemize}
\item {Grp. gram.:adv.}
\end{itemize}
Com afêrro; obstinadamente.
\section{Aferrenhar}
\begin{itemize}
\item {Grp. gram.:v. t.}
\end{itemize}
Endurecer como ferro.
Obstinar.
Emperrar.
\section{Aferretoar}
\begin{itemize}
\item {Grp. gram.:v. t.}
\end{itemize}
(V.aferroar)
\section{Afêrro}
\begin{itemize}
\item {Grp. gram.:m.}
\end{itemize}
Acto ou effeito de \textunderscore aferrar\textunderscore .
Obstinação.
Grande dedicação.
\section{Aferroador}
\begin{itemize}
\item {Grp. gram.:m.}
\end{itemize}
O que aferrôa.
\section{Aferroar}
\begin{itemize}
\item {Grp. gram.:v. t.}
\end{itemize}
\begin{itemize}
\item {Grp. gram.:V. i.}
\end{itemize}
\begin{itemize}
\item {Utilização:Prov.}
\end{itemize}
\begin{itemize}
\item {Utilização:dur.}
\end{itemize}
Picar com ferrão.
Espicaçar.
Amuar; embezerrar.
\section{Aferrolhador}
\begin{itemize}
\item {Grp. gram.:m.}
\end{itemize}
O que aferrolha.
\section{Aferrolhar}
\begin{itemize}
\item {Grp. gram.:v. t.}
\end{itemize}
Fechar com ferrolho.
Meter na prisão.
Guardar com cuidado: \textunderscore aferrolhar dinheiro\textunderscore .
\section{Aferventação}
\begin{itemize}
\item {Grp. gram.:f.}
\end{itemize}
Acto de \textunderscore aferventar\textunderscore .
\section{Aferventamento}
\begin{itemize}
\item {Grp. gram.:m.}
\end{itemize}
O mesmo que \textunderscore aferventação\textunderscore .
Calor.
\section{Aferventar}
\begin{itemize}
\item {Grp. gram.:v. t.}
\end{itemize}
\begin{itemize}
\item {Proveniência:(De \textunderscore fervente\textunderscore )}
\end{itemize}
Fazer ferver pouco.
Cozer com uma só fervura.
\section{Afervescido}
\begin{itemize}
\item {Grp. gram.:adj.}
\end{itemize}
Que se aferventou. Cf. Filinto, XVIII, 114.
\section{Afervoradamente}
\begin{itemize}
\item {Grp. gram.:adv.}
\end{itemize}
Com fervor.
Fervorosamente.
\section{Afervorar}
\begin{itemize}
\item {Grp. gram.:v. t.}
\end{itemize}
Pôr em fervura.
Communicar fervor, ardor, a.
Estimular: \textunderscore afervorar o zêlo de alguém\textunderscore .
\section{Afestoar}
\begin{itemize}
\item {Grp. gram.:v. t.}
\end{itemize}
O mesmo ou melhor que \textunderscore festoar\textunderscore . Cf. Camillo, \textunderscore Brasileira\textunderscore , 201; Filinto, V, 20.
\section{Afezoar}
\begin{itemize}
\item {fónica:fé}
\end{itemize}
\begin{itemize}
\item {Grp. gram.:v. t.}
\end{itemize}
\begin{itemize}
\item {Utilização:Ant.}
\end{itemize}
\begin{itemize}
\item {Proveniência:(De \textunderscore fé\textunderscore )}
\end{itemize}
Prometer casamento, jurando.
\section{Affábil}
\begin{itemize}
\item {Grp. gram.:adj.}
\end{itemize}
\begin{itemize}
\item {Utilização:Ant.}
\end{itemize}
(V.affável)
\section{Affabilidade}
\begin{itemize}
\item {Grp. gram.:f.}
\end{itemize}
\begin{itemize}
\item {Proveniência:(Lat. \textunderscore affabilitas\textunderscore )}
\end{itemize}
Qualidade do que é affável.
\section{Affabilmente}
\begin{itemize}
\item {Grp. gram.:adv.}
\end{itemize}
O mesmo que \textunderscore affavelmente\textunderscore . Cf. Filinto, \textunderscore D. Man.\textunderscore  I, 184.
\section{Affável}
\begin{itemize}
\item {Grp. gram.:adj.}
\end{itemize}
\begin{itemize}
\item {Proveniência:(Lat. \textunderscore affabilis\textunderscore )}
\end{itemize}
Benévolo.
Cortês; delicado.
\section{Affavelmente}
\begin{itemize}
\item {Grp. gram.:adv.}
\end{itemize}
De modo \textunderscore affável\textunderscore , com affabilidade.
\section{Affecção}
\begin{itemize}
\item {Grp. gram.:f.}
\end{itemize}
\begin{itemize}
\item {Proveniência:(Lat. \textunderscore affectio\textunderscore )}
\end{itemize}
Phenómeno mórbido do organismo animal.
\section{Affectação}
\begin{itemize}
\item {Grp. gram.:f.}
\end{itemize}
\begin{itemize}
\item {Proveniência:(Lat. \textunderscore affectatio\textunderscore )}
\end{itemize}
Acto ou effeito de \textunderscore affectar\textunderscore .
Fingimento.
Presumpção, vaidade.
\section{Affectadamente}
\begin{itemize}
\item {Grp. gram.:adv.}
\end{itemize}
De modo \textunderscore affectado\textunderscore .
\section{Affectado}
\begin{itemize}
\item {Grp. gram.:adj.}
\end{itemize}
\begin{itemize}
\item {Utilização:Bras}
\end{itemize}
Que tem affectação.
Presumido.
Pretensioso.
Tísico, tuberculoso.
\section{Affectante}
\begin{itemize}
\item {Grp. gram.:adj.}
\end{itemize}
Que affecta ou finge o que não é.
\section{Affectar}
\begin{itemize}
\item {Grp. gram.:v. t.}
\end{itemize}
\begin{itemize}
\item {Grp. gram.:V. p.}
\end{itemize}
\begin{itemize}
\item {Proveniência:(Do lat. \textunderscore affectare\textunderscore )}
\end{itemize}
Fingir.
Têr disposição para offender.
Esmerar-se ridiculamente.
\section{Affectativo}
\begin{itemize}
\item {Grp. gram.:adj.}
\end{itemize}
\begin{itemize}
\item {Utilização:Des.}
\end{itemize}
Desejoso.
\section{Affectivamente}
\begin{itemize}
\item {Grp. gram.:adv.}
\end{itemize}
De modo \textunderscore affectivo\textunderscore , com affecto.
\section{Affectividade}
\begin{itemize}
\item {Grp. gram.:f.}
\end{itemize}
Faculdade, relativa aos sentimentos affectivos.
\section{Affectivo}
\begin{itemize}
\item {Grp. gram.:adj.}
\end{itemize}
\begin{itemize}
\item {Proveniência:(Lat. \textunderscore affectivus\textunderscore )}
\end{itemize}
Relativo a \textunderscore affecto\textunderscore .
Que mostra affecto; affectuoso.
\section{Affecto}
\begin{itemize}
\item {Grp. gram.:m.}
\end{itemize}
\begin{itemize}
\item {Grp. gram.:Adj.}
\end{itemize}
\begin{itemize}
\item {Proveniência:(Lat. \textunderscore affectus\textunderscore )}
\end{itemize}
Sentimento de inclinação para alguém.
Amizade.
Sympathia.
Paixão.
Dedicado.
Afeiçoado.
Incumbido.
Pendente ou dependente de resolução superior: \textunderscore o assumpto está affecto ao Ministro da Guerra\textunderscore .
\section{Affectuosamente}
\begin{itemize}
\item {Grp. gram.:adv.}
\end{itemize}
De modo \textunderscore affectuoso\textunderscore .
\section{Affectuoso}
\begin{itemize}
\item {Grp. gram.:adj.}
\end{itemize}
\begin{itemize}
\item {Proveniência:(Lat. \textunderscore affectuosus\textunderscore )}
\end{itemize}
Que tem affecto.
\section{Affeição}
\begin{itemize}
\item {Grp. gram.:f.}
\end{itemize}
\begin{itemize}
\item {Proveniência:(Lat. \textunderscore affectio\textunderscore )}
\end{itemize}
Affecto.
\section{Affeiçoadamente}
\begin{itemize}
\item {Grp. gram.:adv.}
\end{itemize}
De modo \textunderscore affeiçoado\textunderscore .
\section{Affeiçoado}
\begin{itemize}
\item {Grp. gram.:adj.}
\end{itemize}
\begin{itemize}
\item {Proveniência:(De \textunderscore affeiçoar\textunderscore )}
\end{itemize}
Que tem affeição (a alguém ou a alguma coisa).
Inclinado (a certas ideias ou systema).
\section{Affeiçoamento}
\begin{itemize}
\item {Grp. gram.:m.}
\end{itemize}
Acto de \textunderscore affeiçoar\textunderscore .
\section{Affeiçoar}
\begin{itemize}
\item {Grp. gram.:v. t.}
\end{itemize}
\begin{itemize}
\item {Grp. gram.:V. i.}
\end{itemize}
\begin{itemize}
\item {Grp. gram.:V. p.}
\end{itemize}
Tomar affeição a.
Adquirir affeição.
Tomar affeição.
\section{Affeitar}
\begin{itemize}
\item {Grp. gram.:v. t.}
\end{itemize}
\begin{itemize}
\item {Utilização:Ant.}
\end{itemize}
O mesmo que \textunderscore affectar\textunderscore .
\section{Afferência}
\begin{itemize}
\item {Grp. gram.:f.}
\end{itemize}
Qualidade de afferente. Cf. Castilho, \textunderscore Fastos\textunderscore , I, 139.
\section{Afferente}
\begin{itemize}
\item {Grp. gram.:adj.}
\end{itemize}
\begin{itemize}
\item {Proveniência:(Lat. \textunderscore afferens\textunderscore )}
\end{itemize}
Que conduz, que leva.
\section{Affiliação}
\begin{itemize}
\item {Grp. gram.:f.}
\end{itemize}
Acto ou effeito de \textunderscore affiliar\textunderscore .
\section{Affiliar}
\begin{itemize}
\item {Grp. gram.:v. t.}
\end{itemize}
\begin{itemize}
\item {Proveniência:(Do lat. \textunderscore ad\textunderscore  + \textunderscore filium\textunderscore )}
\end{itemize}
Aggregar, juntar, a uma corporação ou sociedade.
\section{Affim}
\begin{itemize}
\item {Grp. gram.:m.  e  adj.}
\end{itemize}
\begin{itemize}
\item {Proveniência:(Lat. \textunderscore affinis\textunderscore )}
\end{itemize}
Parente por affinidade.
Igual, semelhante: \textunderscore linguas affins\textunderscore .
\section{Affimento}
\begin{itemize}
\item {Grp. gram.:m.}
\end{itemize}
\begin{itemize}
\item {Utilização:Des.}
\end{itemize}
Limite commum de herdades.
(Cp. \textunderscore affim\textunderscore )
\section{Affinidade}
\begin{itemize}
\item {Grp. gram.:f.}
\end{itemize}
\begin{itemize}
\item {Utilização:Chím.}
\end{itemize}
\begin{itemize}
\item {Proveniência:(Lat. \textunderscore affinitas\textunderscore )}
\end{itemize}
Parentesco, que um cônjuge contrái com a família do outro cônjuge.
Relação.
Semelhança; analogia.
Conformidade.
Attracção molecular.
Relações orgânicas, (entre os vegetaes e os animaes).
\section{Affirmação}
\begin{itemize}
\item {Grp. gram.:f.}
\end{itemize}
\begin{itemize}
\item {Proveniência:(Lat. \textunderscore affirmatio\textunderscore )}
\end{itemize}
Acto de affirmar.
\section{Affirmadamente}
\begin{itemize}
\item {Grp. gram.:adv.}
\end{itemize}
Com affirmação.
\section{Affirmador}
\begin{itemize}
\item {Grp. gram.:m.}
\end{itemize}
\begin{itemize}
\item {Proveniência:(Lat. \textunderscore affirmator\textunderscore )}
\end{itemize}
O que affirma.
\section{Affirmante}
\begin{itemize}
\item {Grp. gram.:adj.}
\end{itemize}
\begin{itemize}
\item {Proveniência:(Lat. \textunderscore affirmans\textunderscore )}
\end{itemize}
Que affirma.
\section{Affirmar}
\begin{itemize}
\item {Grp. gram.:v. t.}
\end{itemize}
\begin{itemize}
\item {Grp. gram.:V. p.}
\end{itemize}
\begin{itemize}
\item {Utilização:Ant.}
\end{itemize}
\begin{itemize}
\item {Proveniência:(Lat. \textunderscore affirmare\textunderscore )}
\end{itemize}
Asseverar; declarar com firmeza.
Confirmar.
Certificar-se, vendo.
Olhar bem, attentamente.
Tornar firme; segurar bem. Cf. \textunderscore Rev. Lus.\textunderscore , XVI, 1.
\section{Affirmativa}
\begin{itemize}
\item {Grp. gram.:f.}
\end{itemize}
\begin{itemize}
\item {Utilização:Náut.}
\end{itemize}
Declaração que affirma.
Affirmação.
Nos códigos de sinaes, a bandeira com que se affirma.
\section{Affirmativamente}
\begin{itemize}
\item {Grp. gram.:adv.}
\end{itemize}
De modo \textunderscore affirmativo\textunderscore .
\section{Affirmativo}
\begin{itemize}
\item {Grp. gram.:adj.}
\end{itemize}
\begin{itemize}
\item {Proveniência:(Lat. \textunderscore affirmativus\textunderscore )}
\end{itemize}
Que affirma.
Que confirma.
Que envolve affirmação.
\section{Affixação}
\begin{itemize}
\item {Grp. gram.:f.}
\end{itemize}
Acto de \textunderscore affixar\textunderscore .
\section{Affixar}
\begin{itemize}
\item {Grp. gram.:v. t.}
\end{itemize}
\begin{itemize}
\item {Proveniência:(Do lat. \textunderscore affixus\textunderscore )}
\end{itemize}
Tornar fixo; segurar.
Pregar em logar público: \textunderscore affixar annúncios\textunderscore .
\section{Affixivo}
\begin{itemize}
\item {Grp. gram.:adj.}
\end{itemize}
\begin{itemize}
\item {Utilização:Glot.}
\end{itemize}
Que tem affixo; que se caracteriza por affixos.
\section{Affixo}
\begin{itemize}
\item {Grp. gram.:m.}
\end{itemize}
\begin{itemize}
\item {Grp. gram.:Adj.}
\end{itemize}
\begin{itemize}
\item {Proveniência:(Lat. \textunderscore affixus\textunderscore )}
\end{itemize}
Designação commum dos prefixos e suffixos.
Fixado a; unido.
\section{Afflante}
\begin{itemize}
\item {Grp. gram.:adj.}
\end{itemize}
\begin{itemize}
\item {Proveniência:(Lat. \textunderscore afflans\textunderscore )}
\end{itemize}
Que respira.
Offegante. Cf. Camillo, \textunderscore Narcót\textunderscore , I, 156.
\section{Afflar}
\begin{itemize}
\item {Grp. gram.:v. t.}
\end{itemize}
\begin{itemize}
\item {Proveniência:(Lat. \textunderscore afflare\textunderscore )}
\end{itemize}
Bafejar.
Soprar.
Inspirar.
\section{Afflato}
\begin{itemize}
\item {Grp. gram.:m.}
\end{itemize}
\begin{itemize}
\item {Proveniência:(Lat. \textunderscore afflatus\textunderscore )}
\end{itemize}
Sopro; bafejo; hálito.
\section{Afflicção}
\begin{itemize}
\item {Grp. gram.:f.}
\end{itemize}
\begin{itemize}
\item {Proveniência:(Lat. \textunderscore afflictio\textunderscore )}
\end{itemize}
Grande soffrimento.
Atribulação.
Tormento.
\section{Afflictamente}
\begin{itemize}
\item {Grp. gram.:adv.}
\end{itemize}
De modo \textunderscore afflicto\textunderscore .
Com afflicção.
\section{Afflictivamente}
\begin{itemize}
\item {Grp. gram.:adv.}
\end{itemize}
De modo \textunderscore afflictivo\textunderscore : \textunderscore gritar afflictivamente\textunderscore .
\section{Afflictivo}
\begin{itemize}
\item {Grp. gram.:adj.}
\end{itemize}
\begin{itemize}
\item {Proveniência:(De \textunderscore afflicto\textunderscore )}
\end{itemize}
Que produz afflicção; que contém afflicção.
\section{Afflicto}
\begin{itemize}
\item {Grp. gram.:adj.}
\end{itemize}
\begin{itemize}
\item {Proveniência:(Lat. \textunderscore afflictus\textunderscore )}
\end{itemize}
Que tem ou mostra afflicção.
Angustiado.
\section{Affligente}
\begin{itemize}
\item {Grp. gram.:adj.}
\end{itemize}
O mesmo que \textunderscore afflictivo\textunderscore .
\section{Affligidamente}
\begin{itemize}
\item {Grp. gram.:adv.}
\end{itemize}
O mesmo que \textunderscore afflictivamente\textunderscore .
\section{Affligidor}
\begin{itemize}
\item {Grp. gram.:m.  e  adj.}
\end{itemize}
O que afflige.
\section{Affligimento}
\begin{itemize}
\item {Grp. gram.:m.}
\end{itemize}
O mesmo que \textunderscore afflicção\textunderscore .
\section{Affligir}
\begin{itemize}
\item {Grp. gram.:v. t.}
\end{itemize}
\begin{itemize}
\item {Proveniência:(Lat. \textunderscore affligere\textunderscore )}
\end{itemize}
Causar afflicção a.
Atormentar.
Angustiar.
\section{Affluência}
\begin{itemize}
\item {Grp. gram.:f.}
\end{itemize}
\begin{itemize}
\item {Proveniência:(Lat. \textunderscore affluentia\textunderscore )}
\end{itemize}
Corrente abundante.
Abundância.
Grande concorrência (de pessôas ou coisas)
\section{Affluente}
\begin{itemize}
\item {Grp. gram.:adj.}
\end{itemize}
\begin{itemize}
\item {Grp. gram.:M.}
\end{itemize}
\begin{itemize}
\item {Proveniência:(Lat. \textunderscore affluens\textunderscore )}
\end{itemize}
Que aflue; que corre.
Abundante.
Corrente de água, que se vai lançar noutra.
\section{Affluir}
\begin{itemize}
\item {Grp. gram.:v. t.}
\end{itemize}
\begin{itemize}
\item {Proveniência:(Lat. \textunderscore affluere\textunderscore )}
\end{itemize}
Correr para um lugar ou lado.
Derivar.
Concorrer.
\section{Affluxo}
\begin{itemize}
\item {Grp. gram.:m.}
\end{itemize}
\begin{itemize}
\item {Proveniência:(Lat. \textunderscore affluxus\textunderscore )}
\end{itemize}
Acto de affluir.
\section{Affonsim}
\begin{itemize}
\item {Grp. gram.:adj.}
\end{itemize}
\begin{itemize}
\item {Utilização:Ant.}
\end{itemize}
O mesmo que \textunderscore affonsino\textunderscore .
Antiga moéda portuguesa.
\section{Affonsinhos}
\begin{itemize}
\item {Grp. gram.:m. pl.}
\end{itemize}
Us. na loc. fam. \textunderscore era dos affonsinhos\textunderscore , (falando-se de coisa antiga ou fóra da moda).
(Cp. \textunderscore affonsino\textunderscore )
\section{Affonsino}
\begin{itemize}
\item {Grp. gram.:adj.}
\end{itemize}
Relativo á 1.^a dynastia dos reis portugueses.
Também se diz das \textunderscore Ordenações\textunderscore , publicadas por Affonso V.
Antiquado, obsoleto:«\textunderscore termos affonsinos\textunderscore ». Filinto, I, 5.
\section{Affonsista}
\begin{itemize}
\item {Grp. gram.:m.}
\end{itemize}
Partidário de Affonso XII, em Espanha.
\section{Affricata}
\begin{itemize}
\item {Grp. gram.:f.}
\end{itemize}
\begin{itemize}
\item {Utilização:Gram.}
\end{itemize}
Dithongo consonântico, cujo principal elemento é uma explosiva e o segundo uma fricativa do mesmo órgão. Cf. G. Viana, \textunderscore Pronúncia Norm.\textunderscore , 21.
\section{Affundir}
\begin{itemize}
\item {Grp. gram.:v. t.}
\end{itemize}
Banhar em jactos de água.
\section{Affusão}
\begin{itemize}
\item {Grp. gram.:f.}
\end{itemize}
\begin{itemize}
\item {Proveniência:(Do lat. \textunderscore ad\textunderscore  + \textunderscore fusio\textunderscore )}
\end{itemize}
Acto de derramar.
Banho.
Aspersão.
Jacto de água no corpo, para se obter resfriamento subito.
\section{Afiação}
\begin{itemize}
\item {Grp. gram.:f.}
\end{itemize}
Acto de \textunderscore afiar\textunderscore .
\section{Afiado}
\begin{itemize}
\item {Grp. gram.:adj.}
\end{itemize}
Que tem fio cortante.
Acicalado.
\section{Afiador}
\begin{itemize}
\item {Grp. gram.:m.}
\end{itemize}
O que afia.
\section{Afiambrado}
\begin{itemize}
\item {Grp. gram.:adj.}
\end{itemize}
\begin{itemize}
\item {Utilização:Fam.}
\end{itemize}
\begin{itemize}
\item {Proveniência:(De \textunderscore afiambrar-se\textunderscore )}
\end{itemize}
Diz-se do indivíduo demasiadamente apurado no trajar.
\section{Afiambrar-se}
\begin{itemize}
\item {Grp. gram.:v. p.}
\end{itemize}
\begin{itemize}
\item {Proveniência:(De \textunderscore fiambre\textunderscore )}
\end{itemize}
Apurar-se demasiadamente no trajar.
\section{Afiançador}
\begin{itemize}
\item {Grp. gram.:m.}
\end{itemize}
O que afiança.
\section{Afiançar}
\begin{itemize}
\item {Grp. gram.:v. t.}
\end{itemize}
\begin{itemize}
\item {Proveniência:(De \textunderscore fiança\textunderscore )}
\end{itemize}
Sêr fiador de; abonar; responsabilizar-se por.
\section{Afiar}
\begin{itemize}
\item {Grp. gram.:v. t.}
\end{itemize}
\begin{itemize}
\item {Utilização:Fig.}
\end{itemize}
Dar fio a.
Tornar cortante.
Irritar.
Aperfeiçoar.
\section{Aficadamente}
\begin{itemize}
\item {Grp. gram.:adv.}
\end{itemize}
\begin{itemize}
\item {Proveniência:(De \textunderscore aficado\textunderscore )}
\end{itemize}
Com insistência.
\section{Aficado}
\begin{itemize}
\item {Grp. gram.:adj.}
\end{itemize}
\begin{itemize}
\item {Proveniência:(De \textunderscore aficar\textunderscore )}
\end{itemize}
Assediado.
Apertado. Cf. Fernão Lopes.
\section{Aficamento}
\begin{itemize}
\item {Grp. gram.:m.}
\end{itemize}
\begin{itemize}
\item {Utilização:Ant.}
\end{itemize}
Acto de \textunderscore aficar\textunderscore .
\section{Aficar}
\begin{itemize}
\item {Grp. gram.:v. i.  e  p.}
\end{itemize}
\begin{itemize}
\item {Utilização:Ant.}
\end{itemize}
Insistir, teimar.
(Cp. \textunderscore fixar\textunderscore )
\section{Afidalgadamente}
\begin{itemize}
\item {Grp. gram.:adv.}
\end{itemize}
Á maneira de fidalgo.
\section{Afidalgado}
\begin{itemize}
\item {Grp. gram.:adj.}
\end{itemize}
\begin{itemize}
\item {Proveniência:(De \textunderscore afidalgar\textunderscore )}
\end{itemize}
Que tem ares ou maneiras de fidalgo.
\section{Afidalgamento}
\begin{itemize}
\item {Grp. gram.:m.}
\end{itemize}
Acto de \textunderscore afidalgar\textunderscore  ou de se afidalgar.
\section{Afidalgar}
\begin{itemize}
\item {Grp. gram.:v. t.}
\end{itemize}
Tornar fidalgo.
Dar semelhança de fidalgo a.
\section{Afifano}
\begin{itemize}
\item {Grp. gram.:m.}
\end{itemize}
\begin{itemize}
\item {Utilização:Fig.}
\end{itemize}
Homem natural de Afife, no termo do Porto.
Trolha, caiador.
\section{Afifar}
\begin{itemize}
\item {Grp. gram.:v. t.  e  i.}
\end{itemize}
\begin{itemize}
\item {Utilização:Prov.}
\end{itemize}
O mesmo que \textunderscore afinfar\textunderscore .
\section{Afiguração}
\begin{itemize}
\item {Grp. gram.:f.}
\end{itemize}
Acto ou effeito de \textunderscore afigurar\textunderscore .
\section{Afiguradamente}
\begin{itemize}
\item {Grp. gram.:adv.}
\end{itemize}
Em figura; na presença.
\section{Afigurar}
\begin{itemize}
\item {Grp. gram.:v. t.}
\end{itemize}
\begin{itemize}
\item {Grp. gram.:V. p.}
\end{itemize}
Representar.
Imaginar.
Dar figura, fórma a.
Dar ideia.
Parecer: \textunderscore afigura-se-me que é verdade\textunderscore .
\section{Afigurativo}
\begin{itemize}
\item {Grp. gram.:adj.}
\end{itemize}
Que encerra figura ou parábola.
\section{Afilado}
\begin{itemize}
\item {Grp. gram.:adj.}
\end{itemize}
\begin{itemize}
\item {Proveniência:(De \textunderscore afilar\textunderscore ^3)}
\end{itemize}
Adelgaçado.
Ponteagudo: \textunderscore nariz afilado\textunderscore .
\section{Afilamento}
\begin{itemize}
\item {Grp. gram.:m.}
\end{itemize}
Acto ou effeito de \textunderscore afilar\textunderscore ^1, aferição.
\section{Afilar}
\begin{itemize}
\item {Grp. gram.:v. t.}
\end{itemize}
Aferir, examinar ou cotejar com outro (pêso, balança ou medida).
(Cf. \textunderscore afiar\textunderscore )
\section{Afilar}
\begin{itemize}
\item {Grp. gram.:v. t.}
\end{itemize}
\begin{itemize}
\item {Proveniência:(De \textunderscore filar\textunderscore )}
\end{itemize}
Açular, instigar (um cão) para que file.
\section{Afilar}
\begin{itemize}
\item {Grp. gram.:v. t.}
\end{itemize}
\begin{itemize}
\item {Proveniência:(Do lat. \textunderscore filum\textunderscore , fio)}
\end{itemize}
Adelgaçar.
Tornar afiado.
\section{Afilhada}
\begin{itemize}
\item {Grp. gram.:f.}
\end{itemize}
\begin{itemize}
\item {Utilização:Prov.}
\end{itemize}
\begin{itemize}
\item {Utilização:alent.}
\end{itemize}
\begin{itemize}
\item {Grp. gram.:Adj.}
\end{itemize}
\begin{itemize}
\item {Utilização:Prov.}
\end{itemize}
\begin{itemize}
\item {Utilização:alent.}
\end{itemize}
(flexão \textunderscore fem.\textunderscore  de \textunderscore afilhado\textunderscore )
Porca, que já conhece os filhos e que, para isso, teve de estar fechada com êlles, por alguns dias.
Diz-se da fêmea que tem filhos.
\section{Afilhadagem}
\begin{itemize}
\item {Grp. gram.:f.}
\end{itemize}
Os afilhados; porção de afilhados.
Patronato.
Nepotismo.
\section{Afilhado}
\begin{itemize}
\item {Grp. gram.:m.}
\end{itemize}
\begin{itemize}
\item {Proveniência:(De \textunderscore afilhar\textunderscore )}
\end{itemize}
Diz-se, em relação aos padrinhos, o que recebe o baptismo ou confirmação; o que se casa; o que se bate em duello, etc.
O que é protegido, em relação ao protector.
\section{Afilhador}
\begin{itemize}
\item {Grp. gram.:m.}
\end{itemize}
\begin{itemize}
\item {Utilização:Prov.}
\end{itemize}
\begin{itemize}
\item {Utilização:alent.}
\end{itemize}
O cabreiro que afilha as cabras.
\section{A final}
\begin{itemize}
\item {Grp. gram.:adv.}
\end{itemize}
\begin{itemize}
\item {Proveniência:(De \textunderscore final\textunderscore )}
\end{itemize}
Em-fim, finalmente.
\section{Afilhar}
\begin{itemize}
\item {Grp. gram.:v. i.}
\end{itemize}
\begin{itemize}
\item {Grp. gram.:V. t.}
\end{itemize}
\begin{itemize}
\item {Utilização:Prov.}
\end{itemize}
\begin{itemize}
\item {Utilização:alent.}
\end{itemize}
Dar filhos, rebentos, (falando-se de plantas).
Distribuir a (cabras) os filhos destas.--Quando as mães voltam sem êlles do pasto ao corveiro, onde ficaram os filhos, o cabreiro distribue a cada uma o filho respectivo.
\section{Afilharar}
\begin{itemize}
\item {Grp. gram.:v. i.}
\end{itemize}
\begin{itemize}
\item {Utilização:Prov.}
\end{itemize}
Ter filharada, ter muitos filhos.
\section{Afilhastro}
\begin{itemize}
\item {Grp. gram.:m.}
\end{itemize}
\begin{itemize}
\item {Utilização:T. de Moncorvo}
\end{itemize}
Filho natural.
Enteado.
\section{Afiliação}
\begin{itemize}
\item {Grp. gram.:f.}
\end{itemize}
Acto ou effeito de \textunderscore afiliar\textunderscore .
\section{Afiliar}
\begin{itemize}
\item {Grp. gram.:v. t.}
\end{itemize}
\begin{itemize}
\item {Proveniência:(Do lat. \textunderscore ad\textunderscore  + \textunderscore filium\textunderscore )}
\end{itemize}
Aggregar, juntar, a uma corporação ou sociedade.
\section{Afim}
\begin{itemize}
\item {Grp. gram.:m.  e  adj.}
\end{itemize}
\begin{itemize}
\item {Proveniência:(Lat. \textunderscore affinis\textunderscore )}
\end{itemize}
Parente por afinidade.
Igual, semelhante: \textunderscore linguas afins\textunderscore .
\section{Afimento}
\begin{itemize}
\item {Grp. gram.:m.}
\end{itemize}
\begin{itemize}
\item {Utilização:Des.}
\end{itemize}
Limite commum de herdades.
(Cp. \textunderscore affim\textunderscore )
\section{Afinação}
\begin{itemize}
\item {Grp. gram.:f.}
\end{itemize}
Acto de \textunderscore afinar\textunderscore .
Qualidade de afinado.
\section{Afinadamente}
\begin{itemize}
\item {Grp. gram.:adv.}
\end{itemize}
Com afinação.
\section{Afinado}
\begin{itemize}
\item {Grp. gram.:adj.}
\end{itemize}
\begin{itemize}
\item {Utilização:Prov.}
\end{itemize}
\begin{itemize}
\item {Utilização:dur.}
\end{itemize}
Que se afinou; que está no devido tom, (falando-se de instrumentos músicos).
Finório, sagaz.
\section{Afinador}
\begin{itemize}
\item {Grp. gram.:m.}
\end{itemize}
O que afina.
\section{Afinagem}
\begin{itemize}
\item {Grp. gram.:f.}
\end{itemize}
Purificação de metaes.
\section{Afinal}
\begin{itemize}
\item {Grp. gram.:adv.}
\end{itemize}
\begin{itemize}
\item {Proveniência:(De \textunderscore final\textunderscore )}
\end{itemize}
Em-fim, finalmente.
\section{Afinamento}
\begin{itemize}
\item {Grp. gram.:m.}
\end{itemize}
Acto de \textunderscore afinar\textunderscore , (metaes, instrumentos, etc.).
\section{Afinar}
\begin{itemize}
\item {Grp. gram.:v. t.}
\end{itemize}
\begin{itemize}
\item {Utilização:Pop.}
\end{itemize}
\begin{itemize}
\item {Utilização:Prov.}
\end{itemize}
\begin{itemize}
\item {Utilização:dur.}
\end{itemize}
Tornar fino.
Apurar.
Pôr no devido tom.
Purificar, depurar (um metal).
Irritar. Cf. Moreno, \textunderscore Comédia\textunderscore , II, 153.
Escutar; espreitar.
\section{Afincadamente}
\begin{itemize}
\item {Grp. gram.:adv.}
\end{itemize}
Com afinco.
\section{Afincamento}
\begin{itemize}
\item {Grp. gram.:m.}
\end{itemize}
(V.afinco)
\section{Afincância}
\begin{itemize}
\item {Grp. gram.:f.}
\end{itemize}
\begin{itemize}
\item {Utilização:Prov.}
\end{itemize}
\begin{itemize}
\item {Utilização:alent.}
\end{itemize}
\begin{itemize}
\item {Proveniência:(De \textunderscore afincar\textunderscore )}
\end{itemize}
Persistência no trabalho, no esfôrço.
\section{Afincar}
\begin{itemize}
\item {Grp. gram.:v. t.}
\end{itemize}
\begin{itemize}
\item {Utilização:Pop.}
\end{itemize}
\begin{itemize}
\item {Grp. gram.:V. i.}
\end{itemize}
\begin{itemize}
\item {Proveniência:(De \textunderscore fincar\textunderscore )}
\end{itemize}
Plantar de estaca.
Afinfar.
Aferrar-se.
Insistir; persistir.
\section{Afinco}
\begin{itemize}
\item {Grp. gram.:m.}
\end{itemize}
Acto de \textunderscore afincar\textunderscore .
Afêrro, pertinácia.
\section{Afincoar}
\begin{itemize}
\item {Grp. gram.:v. t.  e  i.}
\end{itemize}
Pôr fincões.
\section{Afinfar}
\begin{itemize}
\item {Grp. gram.:v. t.}
\end{itemize}
\begin{itemize}
\item {Utilização:Gír.}
\end{itemize}
\begin{itemize}
\item {Grp. gram.:V. i.}
\end{itemize}
\begin{itemize}
\item {Utilização:Chul.}
\end{itemize}
Dar, despedir (pancada).
Bater.
Têr cópula (um homem).
\section{Afinidade}
\begin{itemize}
\item {Grp. gram.:f.}
\end{itemize}
\begin{itemize}
\item {Utilização:Chím.}
\end{itemize}
\begin{itemize}
\item {Proveniência:(Lat. \textunderscore affinitas\textunderscore )}
\end{itemize}
Parentesco, que um cônjuge contrái com a família do outro cônjuge.
Relação.
Semelhança; analogia.
Conformidade.
Attracção molecular.
Relações orgânicas, (entre os vegetaes e os animaes).
\section{Áfio}
\begin{itemize}
\item {Grp. gram.:adj.}
\end{itemize}
\begin{itemize}
\item {Utilização:Fam.}
\end{itemize}
Successivo, que não tem interrupção:«\textunderscore dois dias áfios\textunderscore ». Camillo, \textunderscore Volcoens\textunderscore , 108.
(Alter. da loc. \textunderscore a fio\textunderscore )
\section{Afirmação}
\begin{itemize}
\item {Grp. gram.:f.}
\end{itemize}
\begin{itemize}
\item {Proveniência:(Lat. \textunderscore affirmatio\textunderscore )}
\end{itemize}
Acto de afirmar.
\section{Afirmadamente}
\begin{itemize}
\item {Grp. gram.:adv.}
\end{itemize}
Com afirmação.
\section{Afirmador}
\begin{itemize}
\item {Grp. gram.:m.}
\end{itemize}
\begin{itemize}
\item {Proveniência:(Lat. \textunderscore affirmator\textunderscore )}
\end{itemize}
O que afirma.
\section{Afirmante}
\begin{itemize}
\item {Grp. gram.:adj.}
\end{itemize}
\begin{itemize}
\item {Proveniência:(Lat. \textunderscore affirmans\textunderscore )}
\end{itemize}
Que afirma.
\section{Afirmar}
\begin{itemize}
\item {Grp. gram.:v. t.}
\end{itemize}
\begin{itemize}
\item {Grp. gram.:V. p.}
\end{itemize}
\begin{itemize}
\item {Proveniência:(Lat. \textunderscore affirmare\textunderscore )}
\end{itemize}
Asseverar; declarar com firmeza.
Confirmar.
Certificar-se, vendo.
Olhar bem, attentamente.
\section{Afirmativa}
\begin{itemize}
\item {Grp. gram.:f.}
\end{itemize}
\begin{itemize}
\item {Utilização:Náut.}
\end{itemize}
Declaração que afirma.
Afirmação.
Nos códigos de sinaes, a bandeira com que se afirma.
\section{Afirmativamente}
\begin{itemize}
\item {Grp. gram.:adv.}
\end{itemize}
De modo \textunderscore afirmativo\textunderscore .
\section{Afirmativo}
\begin{itemize}
\item {Grp. gram.:adj.}
\end{itemize}
\begin{itemize}
\item {Proveniência:(Lat. \textunderscore affirmativus\textunderscore )}
\end{itemize}
Que afirma.
Que confirma.
Que envolve afirmação.
\section{Afistular}
\begin{itemize}
\item {Grp. gram.:v. t.}
\end{itemize}
Fazer fístula a.
Converter em fístula.
\section{Afitado}
\begin{itemize}
\item {Grp. gram.:adj.}
\end{itemize}
\begin{itemize}
\item {Utilização:Bot.}
\end{itemize}
\begin{itemize}
\item {Proveniência:(De \textunderscore afitar\textunderscore ^1)}
\end{itemize}
Diz-se das fôlhas lineares, muito compridas.
\section{Afitar}
\textunderscore v. t.\textunderscore  (e der.)
(V. \textunderscore fitar\textunderscore , etc.)
\section{Afitar}
\begin{itemize}
\item {Grp. gram.:v. t.}
\end{itemize}
\begin{itemize}
\item {Utilização:Ant.}
\end{itemize}
\begin{itemize}
\item {Proveniência:(De \textunderscore afito\textunderscore )}
\end{itemize}
Causar indigestão, diarreia.
\section{Afito}
\begin{itemize}
\item {Grp. gram.:m.}
\end{itemize}
\begin{itemize}
\item {Utilização:Ant.}
\end{itemize}
Indigestão, diarrheia.
\section{Afiusar}
\begin{itemize}
\item {fónica:fi-u}
\end{itemize}
\begin{itemize}
\item {Grp. gram.:v. t.}
\end{itemize}
\begin{itemize}
\item {Grp. gram.:V. p.}
\end{itemize}
\begin{itemize}
\item {Proveniência:(De \textunderscore fiúsa\textunderscore )}
\end{itemize}
Inspirar confiança a.
Têr confiança ou esperança em alguma coisa.
\section{Afivelar}
\begin{itemize}
\item {Grp. gram.:v. t.}
\end{itemize}
Apertar com fivela.
Segurar.
\section{Afixação}
\begin{itemize}
\item {Grp. gram.:f.}
\end{itemize}
Acto de \textunderscore afixar\textunderscore .
\section{Afixar}
\begin{itemize}
\item {Grp. gram.:v. t.}
\end{itemize}
\begin{itemize}
\item {Proveniência:(Do lat. \textunderscore affixus\textunderscore )}
\end{itemize}
Tornar fixo; segurar.
Pregar em logar público: \textunderscore afixar annúncios\textunderscore .
\section{Afixivo}
\begin{itemize}
\item {Grp. gram.:adj.}
\end{itemize}
\begin{itemize}
\item {Utilização:Glot.}
\end{itemize}
Que tem afixo; que se caracteriza por afixos.
\section{Afixo}
\begin{itemize}
\item {Grp. gram.:m.}
\end{itemize}
\begin{itemize}
\item {Grp. gram.:Adj.}
\end{itemize}
\begin{itemize}
\item {Proveniência:(Lat. \textunderscore affixus\textunderscore )}
\end{itemize}
Designação commum dos prefixos e suffixos.
Fixado a; unido.
\section{Aflamengado}
\begin{itemize}
\item {Grp. gram.:adj.}
\end{itemize}
Semelhante aos Flamengos.
\section{Aflante}
\begin{itemize}
\item {Grp. gram.:adj.}
\end{itemize}
\begin{itemize}
\item {Proveniência:(Lat. \textunderscore afflans\textunderscore )}
\end{itemize}
Que respira.
Offegante. Cf. Camillo, \textunderscore Narcót\textunderscore , I, 156.
\section{Aflar}
\begin{itemize}
\item {Grp. gram.:v. t.}
\end{itemize}
\begin{itemize}
\item {Utilização:Ant.}
\end{itemize}
O mesmo que \textunderscore achar\textunderscore ^1.
\section{Aflar}
\begin{itemize}
\item {Grp. gram.:v. t.}
\end{itemize}
\begin{itemize}
\item {Proveniência:(Lat. \textunderscore afflare\textunderscore )}
\end{itemize}
Bafejar.
Soprar.
Inspirar.
\section{Aflato}
\begin{itemize}
\item {Grp. gram.:m.}
\end{itemize}
\begin{itemize}
\item {Proveniência:(Lat. \textunderscore afflatus\textunderscore )}
\end{itemize}
Sopro; bafejo; hálito.
\section{Aflautar}
\begin{itemize}
\item {Grp. gram.:v. t.}
\end{itemize}
Tornar semelhante á flauta no feitio, no som.
Tornar esguio.
\section{Afleimar}
\begin{itemize}
\item {Grp. gram.:v. t.}
\end{itemize}
(V.afleumar)
\section{Afleimar-se}
\begin{itemize}
\item {Grp. gram.:v. p.}
\end{itemize}
\begin{itemize}
\item {Utilização:Pop.}
\end{itemize}
\begin{itemize}
\item {Proveniência:(Do lat. \textunderscore flamma\textunderscore )}
\end{itemize}
Irritar-se.
Impacientar-se.
\section{Afleumar}
\begin{itemize}
\item {Grp. gram.:v. t.}
\end{itemize}
\begin{itemize}
\item {Proveniência:(De \textunderscore fleuma\textunderscore )}
\end{itemize}
Tornar fleumático, pachorrento.
\section{Aflição}
\begin{itemize}
\item {Grp. gram.:f.}
\end{itemize}
\begin{itemize}
\item {Proveniência:(Lat. \textunderscore afflictio\textunderscore )}
\end{itemize}
Grande soffrimento.
Atribulação.
Tormento.
\section{Afligente}
\begin{itemize}
\item {Grp. gram.:adj.}
\end{itemize}
O mesmo que \textunderscore aflitivo\textunderscore .
\section{Afligidamente}
\begin{itemize}
\item {Grp. gram.:adv.}
\end{itemize}
O mesmo que \textunderscore aflitivamente\textunderscore .
\section{Afligidor}
\begin{itemize}
\item {Grp. gram.:m.  e  adj.}
\end{itemize}
O que aflige.
\section{Afligimento}
\begin{itemize}
\item {Grp. gram.:m.}
\end{itemize}
O mesmo que \textunderscore aflição\textunderscore .
\section{Afligir}
\begin{itemize}
\item {Grp. gram.:v. t.}
\end{itemize}
\begin{itemize}
\item {Proveniência:(Lat. \textunderscore affligere\textunderscore )}
\end{itemize}
Causar aflicção a.
Atormentar.
Angustiar.
\section{Aflitamente}
\begin{itemize}
\item {Grp. gram.:adv.}
\end{itemize}
De modo \textunderscore aflicto\textunderscore .
Com aflicção.
\section{Aflitivamente}
\begin{itemize}
\item {Grp. gram.:adv.}
\end{itemize}
De modo \textunderscore aflictivo\textunderscore : \textunderscore gritar aflictivamente\textunderscore .
\section{Aflitivo}
\begin{itemize}
\item {Grp. gram.:adj.}
\end{itemize}
\begin{itemize}
\item {Proveniência:(De \textunderscore afflicto\textunderscore )}
\end{itemize}
Que produz aflicção; que contém aflicção.
\section{Aflito}
\begin{itemize}
\item {Grp. gram.:adj.}
\end{itemize}
\begin{itemize}
\item {Proveniência:(Lat. \textunderscore afflictus\textunderscore )}
\end{itemize}
Que tem ou mostra aflicção.
Angustiado.
\section{Afloração}
\begin{itemize}
\item {Grp. gram.:f.}
\end{itemize}
Acto ou effeito de \textunderscore aflorar\textunderscore .
Nivelamento.
Emergencia de um filão á superficie da terra.
Extremidade dêsse filão.
\section{Afloramento}
\begin{itemize}
\item {Grp. gram.:m.}
\end{itemize}
Acto de \textunderscore aflorar\textunderscore . Cf. \textunderscore Techn. Rur.\textunderscore , 45.
\section{Aflorar}
\begin{itemize}
\item {Grp. gram.:v. t.}
\end{itemize}
\begin{itemize}
\item {Grp. gram.:V. i.}
\end{itemize}
\begin{itemize}
\item {Proveniência:(De \textunderscore flôr\textunderscore )}
\end{itemize}
Nivelar (uma superficie) com outra.
Nivelar.
Emergir á superficie. Cf. Castilho, \textunderscore Fastos\textunderscore , III, 421.
\section{Afloxar}
\begin{itemize}
\item {Grp. gram.:v. t.}
\end{itemize}
\begin{itemize}
\item {Utilização:Ant.}
\end{itemize}
O mesmo que \textunderscore afroixar\textunderscore . Cp. Usque, \textunderscore Tribulações\textunderscore , 24.
\section{Afluência}
\begin{itemize}
\item {Grp. gram.:f.}
\end{itemize}
\begin{itemize}
\item {Proveniência:(Lat. \textunderscore affluentia\textunderscore )}
\end{itemize}
Corrente abundante.
Abundância.
Grande concorrência (de pessôas ou coisas).
\section{Afluente}
\begin{itemize}
\item {Grp. gram.:adj.}
\end{itemize}
\begin{itemize}
\item {Grp. gram.:M.}
\end{itemize}
\begin{itemize}
\item {Proveniência:(Lat. \textunderscore affluens\textunderscore )}
\end{itemize}
Que aflue; que corre.
Abundante.
Corrente de água, que se vai lançar noutra.
\section{Afluir}
\begin{itemize}
\item {Grp. gram.:v. t.}
\end{itemize}
\begin{itemize}
\item {Proveniência:(Lat. \textunderscore affluere\textunderscore )}
\end{itemize}
Correr para um lugar ou lado.
Derivar.
Concorrer.
\section{A-flux}
\begin{itemize}
\item {fónica:aflus}
\end{itemize}
\begin{itemize}
\item {Grp. gram.:loc. adv.}
\end{itemize}
\begin{itemize}
\item {Proveniência:(De \textunderscore fluxo\textunderscore )}
\end{itemize}
Em abundância.
Totalmente.
\section{Afluxo}
\begin{itemize}
\item {Grp. gram.:m.}
\end{itemize}
\begin{itemize}
\item {Proveniência:(Lat. \textunderscore affluxus\textunderscore )}
\end{itemize}
Acto de afluir.
\section{Afobado}
\begin{itemize}
\item {Grp. gram.:adj.}
\end{itemize}
\begin{itemize}
\item {Utilização:Bras}
\end{itemize}
\begin{itemize}
\item {Utilização:pop.}
\end{itemize}
Atrapalhado, precipitado.
\section{Afobar-se}
\begin{itemize}
\item {Grp. gram.:v. p.}
\end{itemize}
\begin{itemize}
\item {Utilização:Bras}
\end{itemize}
\begin{itemize}
\item {Utilização:pop.}
\end{itemize}
Atrapalhar-se.
Fazer qualquer coisa com precipitação.
\section{Afocinhamento}
\begin{itemize}
\item {Grp. gram.:m.}
\end{itemize}
Acto de \textunderscore afocinhar\textunderscore .
\section{Afocinhar}
\begin{itemize}
\item {Grp. gram.:v. t.}
\end{itemize}
\begin{itemize}
\item {Grp. gram.:V. i.}
\end{itemize}
Escavar com o focinho, fossar.
Bater com o focinho.
Fazer cair de focinhos.
Ir ao chão.
\section{Afofadamente}
\begin{itemize}
\item {Grp. gram.:adv.}
\end{itemize}
Á maneira de objecto afofado.
\section{Afofado}
\begin{itemize}
\item {Grp. gram.:adj.}
\end{itemize}
Que se tornou fofo.
\section{Afofamento}
\begin{itemize}
\item {Grp. gram.:m.}
\end{itemize}
Effeito de \textunderscore afofar\textunderscore .
\section{Afofar}
\begin{itemize}
\item {Grp. gram.:v. t.}
\end{itemize}
\begin{itemize}
\item {Grp. gram.:V. i.}
\end{itemize}
\begin{itemize}
\item {Utilização:Prov.}
\end{itemize}
\begin{itemize}
\item {Utilização:alent.}
\end{itemize}
Tornar fofo, molle.
Antegostar qualquer coisa; preparar-se para gozar.
\section{Afogação}
\begin{itemize}
\item {Grp. gram.:f.}
\end{itemize}
\begin{itemize}
\item {Utilização:Ant.}
\end{itemize}
\begin{itemize}
\item {Proveniência:(De \textunderscore fogo\textunderscore )}
\end{itemize}
Imposto, que os emphyteutas pagavam pelo direito de moradia.
\section{Afogadamente}
\begin{itemize}
\item {Grp. gram.:adv.}
\end{itemize}
De afogadilho.
Á pressa.
\section{Afogadela}
\begin{itemize}
\item {Grp. gram.:f.}
\end{itemize}
(V.afogadilho)
\section{Afogadiço}
\begin{itemize}
\item {Grp. gram.:adj.}
\end{itemize}
Que facilmente se afoga; falto de ar.
\section{Afogadilho}
\begin{itemize}
\item {Grp. gram.:m.}
\end{itemize}
\begin{itemize}
\item {Grp. gram.:Loc. adv.}
\end{itemize}
\begin{itemize}
\item {Proveniência:(De \textunderscore afogar\textunderscore )}
\end{itemize}
Pressa.
\textunderscore De afogadilho\textunderscore , apressadamente.
Atabalhoadamente.
\section{Afogador}
\begin{itemize}
\item {Grp. gram.:m.  e  adj.}
\end{itemize}
\begin{itemize}
\item {Grp. gram.:M.}
\end{itemize}
O que afoga.
Abafador.
Collar de mulher, gargantilha.
O mesmo que o \textunderscore abafador\textunderscore , em certas seitas.
\section{Afogadura}
\begin{itemize}
\item {Grp. gram.:f.}
\end{itemize}
O mesmo que \textunderscore afogamento\textunderscore .
\section{Afogamento}
\begin{itemize}
\item {Grp. gram.:m.}
\end{itemize}
Acto de \textunderscore afogar\textunderscore .
\section{Afogar}
\begin{itemize}
\item {Grp. gram.:v. t.}
\end{itemize}
\begin{itemize}
\item {Proveniência:(Do lat. \textunderscore offocare\textunderscore )}
\end{itemize}
Asphyxiar, abafar.
Impedir: \textunderscore afogar expansões\textunderscore .
Ensopar.
Não deixar crescer: \textunderscore as ervas afogam as flores\textunderscore .
Acanhar.
Extinguir.
\section{Afôgo}
\begin{itemize}
\item {Grp. gram.:m.}
\end{itemize}
\begin{itemize}
\item {Proveniência:(De \textunderscore afogar\textunderscore )}
\end{itemize}
Suffocação.
Oppressão.
Afflicção.
Pressa.
\section{Afogueadamente}
\begin{itemize}
\item {Grp. gram.:adv.}
\end{itemize}
Com calor.
Enthusiasticamente.
\section{Afogueado}
\begin{itemize}
\item {Grp. gram.:adj.}
\end{itemize}
\begin{itemize}
\item {Proveniência:(De \textunderscore afoguear\textunderscore )}
\end{itemize}
Esbraseado.
Vermelho, còrado: \textunderscore faces afogueadas\textunderscore .
Que tem côr de fogo; avermelhado: \textunderscore na fimbria afogueada do poente\textunderscore .
\section{Afogueamento}
\begin{itemize}
\item {Grp. gram.:m.}
\end{itemize}
Acto ou effeito de \textunderscore afoguear\textunderscore .
Cf. Camillo, \textunderscore Estrêl. Fun.\textunderscore , 153.
\section{Afoguear}
\begin{itemize}
\item {Grp. gram.:v. t.}
\end{itemize}
\begin{itemize}
\item {Proveniência:(De \textunderscore fogo\textunderscore )}
\end{itemize}
Queimar.
Enrubescer; tornar còrado.
Dar côr de fogo a.
Enthusiasmar.
\section{Afoguentar}
\begin{itemize}
\item {Grp. gram.:v. t.}
\end{itemize}
\begin{itemize}
\item {Utilização:Prov.}
\end{itemize}
\begin{itemize}
\item {Utilização:alg.}
\end{itemize}
\begin{itemize}
\item {Proveniência:(De \textunderscore fogo\textunderscore )}
\end{itemize}
Apressar.
\section{Afoicinhado}
\begin{itemize}
\item {Grp. gram.:adj.}
\end{itemize}
\begin{itemize}
\item {Utilização:Ant.}
\end{itemize}
Dizia-se do capão, cujas pennas na cauda são grandes e reviradas como uma foice.
\section{Afoitadamente}
\begin{itemize}
\item {Grp. gram.:adv.}
\end{itemize}
(V.afoitamente)
\section{Afoitamente}
\begin{itemize}
\item {Grp. gram.:adv.}
\end{itemize}
\begin{itemize}
\item {Proveniência:(De \textunderscore afoito\textunderscore )}
\end{itemize}
Com afoiteza.
\section{Afoitar}
\begin{itemize}
\item {Grp. gram.:v. t.}
\end{itemize}
Tornar afoito.
Estimular.
\section{Afoiteza}
\begin{itemize}
\item {Grp. gram.:f.}
\end{itemize}
Qualidade do que é \textunderscore afoito\textunderscore .
Coragem.
Ousadia.
\section{Afoito}
\begin{itemize}
\item {Grp. gram.:adj.}
\end{itemize}
Ousado.
Corajoso.
Audaz.
\section{Afolar}
\begin{itemize}
\item {Grp. gram.:m.}
\end{itemize}
O mesmo que \textunderscore folar\textunderscore .
\section{Afolhamento}
\begin{itemize}
\item {Grp. gram.:m.}
\end{itemize}
Acto de \textunderscore afolhar\textunderscore .
\section{Afolhar}
\begin{itemize}
\item {Grp. gram.:v. t.}
\end{itemize}
Dividir (o campo) em fôlhas ou porções, para lhes alternar a cultura.
\section{Afomear}
\begin{itemize}
\item {Grp. gram.:v. t.}
\end{itemize}
Causar fome a. Cf. Filinto, XI, 123.
\section{Afonsim}
\begin{itemize}
\item {Grp. gram.:adj.}
\end{itemize}
\begin{itemize}
\item {Utilização:Ant.}
\end{itemize}
O mesmo que \textunderscore afonsino\textunderscore .
Antiga moéda portuguesa.
\section{Afonsinhos}
\begin{itemize}
\item {Grp. gram.:m. pl.}
\end{itemize}
Us. na loc. fam. \textunderscore era dos afonsinhos\textunderscore , (falando-se de coisa antiga ou fóra da moda).
(Cp. \textunderscore affonsino\textunderscore )
\section{Afonsino}
\begin{itemize}
\item {Grp. gram.:adj.}
\end{itemize}
Relativo á 1.^a dynastia dos reis portugueses.
Também se diz das \textunderscore Ordenações\textunderscore , publicadas por Affonso V.
Antiquado, obsoleto:«\textunderscore termos affonsinos\textunderscore ». Filinto, I, 5.
\section{Afonsista}
\begin{itemize}
\item {Grp. gram.:m.}
\end{itemize}
Partidário de Affonso XII, em Espanha.
\section{Afora}
\begin{itemize}
\item {Grp. gram.:prep.}
\end{itemize}
\begin{itemize}
\item {Proveniência:(De \textunderscore a\textunderscore  + \textunderscore fora\textunderscore )}
\end{itemize}
Á excepção de.
Além de.
\section{Aforação}
\begin{itemize}
\item {Grp. gram.:f.}
\end{itemize}
O mesmo que \textunderscore aforamento\textunderscore .
\section{Aforador}
\begin{itemize}
\item {Grp. gram.:m.}
\end{itemize}
O que afora ou dá alguma coisa de aforamento.
\section{Aforamento}
\begin{itemize}
\item {Grp. gram.:m.}
\end{itemize}
Acto ou effeito de \textunderscore aforar\textunderscore .
\section{Aforar}
\begin{itemize}
\item {Grp. gram.:v. t.}
\end{itemize}
Receber, com foro.
Dar, por meio de foro.
\section{Aforciar}
\begin{itemize}
\item {Grp. gram.:v. t.}
\end{itemize}
\begin{itemize}
\item {Utilização:Ant.}
\end{itemize}
O mesmo que \textunderscore forçar\textunderscore  ou violentar (uma mulher).
\section{Aforçurado}
\begin{itemize}
\item {Grp. gram.:adj.}
\end{itemize}
\begin{itemize}
\item {Proveniência:(De \textunderscore aforçurar\textunderscore )}
\end{itemize}
Apressado, afanoso.
\section{Aforçuramento}
\begin{itemize}
\item {Grp. gram.:m.}
\end{itemize}
Acto de \textunderscore aforçurar\textunderscore .
\section{Aforçurar}
\begin{itemize}
\item {Grp. gram.:v. t.}
\end{itemize}
\begin{itemize}
\item {Utilização:Pop.}
\end{itemize}
O mesmo que \textunderscore apressar\textunderscore .
(Talvez corr. de \textunderscore apressurar\textunderscore , sob a infl. de \textunderscore fôrça\textunderscore )
\section{Aformosar}
\begin{itemize}
\item {Grp. gram.:v. t.}
\end{itemize}
O mesmo que \textunderscore aformosear\textunderscore . Cf. Filinto, I, 28 e 199.
\section{Aformoseadamente}
\begin{itemize}
\item {Grp. gram.:adv.}
\end{itemize}
De modo \textunderscore aformoseado\textunderscore .
\section{Aformoseado}
\begin{itemize}
\item {Grp. gram.:adj.}
\end{itemize}
\begin{itemize}
\item {Proveniência:(De \textunderscore aformosear\textunderscore )}
\end{itemize}
Que se tornou formoso.
Enfeitado.
\section{Aformoseador}
\begin{itemize}
\item {Grp. gram.:m.}
\end{itemize}
O que aformoseia.
\section{Aformoseamento}
\begin{itemize}
\item {Grp. gram.:m.}
\end{itemize}
Acto ou effeito de \textunderscore aformosear\textunderscore .
\section{Aformosear}
\begin{itemize}
\item {Grp. gram.:v. t.}
\end{itemize}
Tornar formoso.
Enfeitar.
\section{Aformosentar}
\begin{itemize}
\item {Grp. gram.:v. t.}
\end{itemize}
O mesmo que \textunderscore aformosear\textunderscore .
\section{Aforquilhamento}
\begin{itemize}
\item {Grp. gram.:m.}
\end{itemize}
Acto de \textunderscore aforquilhar\textunderscore .
\section{Aforquilhar}
\begin{itemize}
\item {Grp. gram.:v. t.}
\end{itemize}
Prender com forquilha.
Dar fórma de forquilha a.
\section{Aforradamente}
\begin{itemize}
\item {Grp. gram.:adv.}
\end{itemize}
\begin{itemize}
\item {Proveniência:(De \textunderscore aforrar\textunderscore ^2)}
\end{itemize}
Livremente. Cf. Herculano, \textunderscore Lend. e Narr.\textunderscore , II, 67; e \textunderscore Bobo\textunderscore , 102.
\section{Aforrado}
\begin{itemize}
\item {Grp. gram.:adj.}
\end{itemize}
\begin{itemize}
\item {Proveniência:(De \textunderscore forrar\textunderscore ^1)}
\end{itemize}
Forrado, enchumaçado.
Disfarçado em traje que não é seu.
\section{Aforrado}
\begin{itemize}
\item {Grp. gram.:adj.}
\end{itemize}
\begin{itemize}
\item {Utilização:Ant.}
\end{itemize}
Apressado. Cf. Góes, \textunderscore D. Man.\textunderscore  II, 21.
\section{Aforramento}
\begin{itemize}
\item {Grp. gram.:m.}
\end{itemize}
Acto ou effeito de \textunderscore aforrar\textunderscore .
\section{Aforrar}
\begin{itemize}
\item {Grp. gram.:v. t.}
\end{itemize}
Pôr fôrro em, enchumaçar.
O mesmo que \textunderscore forrar\textunderscore ^1.
Arregaçar (a manga), dobrando o bocal para cima.
\section{Aforrar}
\begin{itemize}
\item {Grp. gram.:v. t.}
\end{itemize}
\begin{itemize}
\item {Grp. gram.:V. i.}
\end{itemize}
Tornar fôrro, libertar.
Economizar, juntando.
O mesmo que \textunderscore forrar\textunderscore ^2.
Viajar como escoteiro. Cf. Camillo, \textunderscore Caveira\textunderscore .
\section{Afortalezar}
\begin{itemize}
\item {Grp. gram.:v. t.}
\end{itemize}
\begin{itemize}
\item {Utilização:Ant.}
\end{itemize}
\begin{itemize}
\item {Proveniência:(De \textunderscore fortaleza\textunderscore )}
\end{itemize}
Fortificar.
\section{Afortelegar}
\begin{itemize}
\item {Grp. gram.:v. t.}
\end{itemize}
\begin{itemize}
\item {Utilização:Ant.}
\end{itemize}
O mesmo que \textunderscore afortalezar\textunderscore .
\section{Afortunadamente}
\begin{itemize}
\item {Grp. gram.:adv.}
\end{itemize}
De modo \textunderscore afortunado\textunderscore .
\section{Afortunado}
\begin{itemize}
\item {Grp. gram.:adj.}
\end{itemize}
\begin{itemize}
\item {Proveniência:(De \textunderscore afortunar\textunderscore )}
\end{itemize}
Favorecido pela fortuna.
Feliz.
\section{Afortunar}
\begin{itemize}
\item {Grp. gram.:v. t.}
\end{itemize}
\begin{itemize}
\item {Proveniência:(Lat. \textunderscore fortunare\textunderscore )}
\end{itemize}
Dar fortuna a; tornar feliz.
\section{Afortunoso}
\begin{itemize}
\item {Grp. gram.:adj.}
\end{itemize}
O mesmo que \textunderscore afortunado\textunderscore .
\section{Afosseirado}
\begin{itemize}
\item {Grp. gram.:adj.}
\end{itemize}
\begin{itemize}
\item {Utilização:Ant.}
\end{itemize}
Onerado com o imposto ou encargo de fossadeira. Cf. Herculano, \textunderscore Hist. de Port.\textunderscore , III, 323 e 327.
\section{Afouto}
\begin{itemize}
\item {Grp. gram.:adj.}
\end{itemize}
Ousado.
Corajoso.
Audaz.
\section{Afracado}
\begin{itemize}
\item {Grp. gram.:adj.}
\end{itemize}
Que está fraco.
\section{Afracamento}
\begin{itemize}
\item {Grp. gram.:m.}
\end{itemize}
Acto de \textunderscore afracar\textunderscore .
\section{Afracar}
\begin{itemize}
\item {Grp. gram.:v. t.}
\end{itemize}
O mesmo que \textunderscore enfraquecer\textunderscore .
\section{Afragar}
\begin{itemize}
\item {Grp. gram.:m.}
\end{itemize}
\begin{itemize}
\item {Utilização:Ant.}
\end{itemize}
O mesmo que \textunderscore verdete\textunderscore .
\section{Afragatar-se}
\begin{itemize}
\item {Grp. gram.:v. p.}
\end{itemize}
\begin{itemize}
\item {Utilização:T. de Lisbôa}
\end{itemize}
Galantear, requestar alguém.
Insinuar-se, para attingir fins libidinosos.
\section{Aframengado}
\begin{itemize}
\item {Grp. gram.:adj.}
\end{itemize}
Descomposto, desenvolto?«\textunderscore aframengado de gesto\textunderscore ». \textunderscore Anat. Joc.\textunderscore  I, 3.
Que parece framengo.
Alvo e loiro.
(Por \textunderscore aflamengado\textunderscore , de \textunderscore flamengo\textunderscore ?)
\section{Afrancesadamente}
\begin{itemize}
\item {Grp. gram.:adv.}
\end{itemize}
Á francesa; á maneira dos Franceses.
\section{Afrancesado}
\begin{itemize}
\item {Grp. gram.:adj.}
\end{itemize}
\begin{itemize}
\item {Proveniência:(De \textunderscore afrancesar\textunderscore )}
\end{itemize}
Que tem modos, aspecto ou feitio de francês.
\section{Afrancesar}
\begin{itemize}
\item {Grp. gram.:v. t.}
\end{itemize}
Tornar semelhante a francês; dar modos de francês a.
\section{Afraquentar}
\begin{itemize}
\item {Grp. gram.:v. t.}
\end{itemize}
\begin{itemize}
\item {Utilização:Des.}
\end{itemize}
Tornar fraco; enfraquecer.
\section{Afrechado}
\begin{itemize}
\item {fónica:fré}
\end{itemize}
\begin{itemize}
\item {Grp. gram.:adj.}
\end{itemize}
\begin{itemize}
\item {Proveniência:(De \textunderscore afrechar\textunderscore )}
\end{itemize}
Que tem fórma de frecha.
\section{Afrechar}
\begin{itemize}
\item {fónica:fré}
\end{itemize}
\begin{itemize}
\item {Grp. gram.:v. t.}
\end{itemize}
Dar fórma de frecha a.
Ferir com frecha.
\section{Afreguesado}
\begin{itemize}
\item {fónica:fré}
\end{itemize}
\begin{itemize}
\item {Grp. gram.:adj.}
\end{itemize}
\begin{itemize}
\item {Proveniência:(De \textunderscore afreguesar\textunderscore )}
\end{itemize}
Que se tornou freguês.
Cliente.
\section{Afreguesar}
\begin{itemize}
\item {fónica:fré}
\end{itemize}
\begin{itemize}
\item {Grp. gram.:v. t.}
\end{itemize}
Tornar freguês, cliente.
Adquirir fregueses, clientes, para: \textunderscore afreguesar um botequim\textunderscore .
\section{Afreimar}
\begin{itemize}
\item {Grp. gram.:v. t.}
\end{itemize}
\begin{itemize}
\item {Proveniência:(De \textunderscore freima\textunderscore )}
\end{itemize}
O mesmo que \textunderscore afleumar\textunderscore .
Tornar apressado, impaciente. Cf. Castilho, \textunderscore Fausto\textunderscore , 117 e 390.
\section{Afrentar}
\begin{itemize}
\item {Grp. gram.:v. i.}
\end{itemize}
\begin{itemize}
\item {Utilização:Ant.}
\end{itemize}
\begin{itemize}
\item {Proveniência:(De \textunderscore frente\textunderscore )}
\end{itemize}
Confinar, convizinhar, sêr contíguo.
\section{Afrescar}
\textunderscore v. t.\textunderscore  (e der.)
(V. \textunderscore refrescar\textunderscore , etc.)
\section{Afretar}
\textunderscore v. t.\textunderscore  (e der.)
(V. \textunderscore fretar\textunderscore , etc.)
\section{África}
\begin{itemize}
\item {Grp. gram.:f.}
\end{itemize}
\begin{itemize}
\item {Utilização:Pop.}
\end{itemize}
\begin{itemize}
\item {Proveniência:(Lat. \textunderscore África\textunderscore , n. p.)}
\end{itemize}
Proêza, façanha.
\section{Africanada}
\begin{itemize}
\item {Grp. gram.:f.}
\end{itemize}
\begin{itemize}
\item {Utilização:Açor}
\end{itemize}
\begin{itemize}
\item {Proveniência:(De \textunderscore africano\textunderscore )}
\end{itemize}
Fanfarronada.
\section{Africanamente}
\begin{itemize}
\item {Grp. gram.:adv.}
\end{itemize}
Á maneira de africano.
Como na África.
\section{Africanas}
\begin{itemize}
\item {Grp. gram.:f. pl.}
\end{itemize}
Argolas de oiro para as orelhas, á semelhança das que usam indígenas da África.
\section{Africanismo}
\begin{itemize}
\item {Grp. gram.:m.}
\end{itemize}
Vício, costume ou modo próprio da África.
\section{Africanista}
\begin{itemize}
\item {Grp. gram.:m.}
\end{itemize}
O que se dedica ao estudo das coisas da África.
\section{Africanizar}
\begin{itemize}
\item {Grp. gram.:v. t.}
\end{itemize}
Tornar africano; dar feição africana a.
\section{Africano}
\begin{itemize}
\item {Grp. gram.:adj.}
\end{itemize}
\begin{itemize}
\item {Grp. gram.:M.}
\end{itemize}
\begin{itemize}
\item {Proveniência:(Lat. \textunderscore africanus\textunderscore )}
\end{itemize}
Relativo á África.
Indivíduo natural da África.
\section{Africata}
\begin{itemize}
\item {Grp. gram.:f.}
\end{itemize}
\begin{itemize}
\item {Utilização:Gram.}
\end{itemize}
Dithongo consonântico, cujo principal elemento é uma explosiva e o segundo uma fricativa do mesmo órgão. Cf. G. Viana, \textunderscore Pronúncia Norm.\textunderscore , 21.
\section{Áfrico}
\begin{itemize}
\item {Grp. gram.:adj.}
\end{itemize}
\begin{itemize}
\item {Grp. gram.:M.}
\end{itemize}
Relativo á África, africano.
Indivíduo natural da África.
Vento de sudoéste.
\section{Afro}
\begin{itemize}
\item {Grp. gram.:m.  e  adj.}
\end{itemize}
\begin{itemize}
\item {Proveniência:(Lat. \textunderscore afer\textunderscore )}
\end{itemize}
O mesmo que \textunderscore africano\textunderscore .
\section{Afroixadamente}
\begin{itemize}
\item {Grp. gram.:adv.}
\end{itemize}
O mesmo que \textunderscore froixamente\textunderscore .
\section{Afroixamento}
\begin{itemize}
\item {Grp. gram.:m.}
\end{itemize}
Acto ou effeito de \textunderscore afroixar\textunderscore .
\section{Afroixar}
\begin{itemize}
\item {Grp. gram.:v. t.}
\end{itemize}
Tornar froixo.
Alargar.
Deminuir o movimento de.
\section{Afroixelar}
\begin{itemize}
\item {Grp. gram.:v. t.}
\end{itemize}
Tornar macio como froixel.
Cobrir de froixel.
\section{A-froixo}
\begin{itemize}
\item {Grp. gram.:adv.}
\end{itemize}
O mesmo que \textunderscore a-flux\textunderscore .
\section{Afronhado}
\begin{itemize}
\item {Grp. gram.:adj.}
\end{itemize}
Que tem fórma de fronha.
\section{Afronta}
\begin{itemize}
\item {Grp. gram.:f.}
\end{itemize}
\begin{itemize}
\item {Proveniência:(De \textunderscore afrontar\textunderscore )}
\end{itemize}
Desprêzo ou injúria lançada em rosto.
Assalto; violência.
\section{Afrontadamente}
\begin{itemize}
\item {Grp. gram.:adv.}
\end{itemize}
Com afronta.
\section{Afrontadiço}
\begin{itemize}
\item {Grp. gram.:adj.}
\end{itemize}
Que facilmente se afronta.
\section{Afrontador}
\begin{itemize}
\item {Grp. gram.:m.}
\end{itemize}
O que afronta.
\section{Afrontamento}
\begin{itemize}
\item {Grp. gram.:m.}
\end{itemize}
Acto de \textunderscore afrontar\textunderscore .
Estado do que se afronta.
Perturbação de cabeça.
\section{Afrontante}
\begin{itemize}
\item {Grp. gram.:adj.}
\end{itemize}
Que afronta. Cf. Ortigão, \textunderscore Praias\textunderscore , 118.
\section{Afrontar}
\begin{itemize}
\item {Grp. gram.:v. t.}
\end{itemize}
\begin{itemize}
\item {Utilização:Prov.}
\end{itemize}
\begin{itemize}
\item {Utilização:minh.}
\end{itemize}
\begin{itemize}
\item {Grp. gram.:V. p.}
\end{itemize}
Desprezar; injuriar pessoalmente, directamente.
Ultrajar de cara a cara.
Causar perturbação de cabeça a.
Segurar; tornar firme de um lado.
Encontrar-se de frente, de cara a cara:«\textunderscore até que se afrontou com dous mercadores de Tunes.\textunderscore »Filinto, \textunderscore D. Man.\textunderscore , I, 92.
(B. lat. \textunderscore frontare\textunderscore )
\section{Afrontosamente}
\begin{itemize}
\item {Grp. gram.:adv.}
\end{itemize}
De modo \textunderscore afrontoso\textunderscore .
\section{Afrontoso}
\begin{itemize}
\item {Grp. gram.:adj.}
\end{itemize}
\begin{itemize}
\item {Proveniência:(De \textunderscore afrontar\textunderscore )}
\end{itemize}
Que envolve afronta.
Que causa afronta.
\section{Afrouxar}
\begin{itemize}
\item {Grp. gram.:v. t.}
\end{itemize}
Tornar froixo.
Alargar.
Deminuir o movimento de.
\section{Afruitejugar}
\begin{itemize}
\item {Grp. gram.:v. i.}
\end{itemize}
\begin{itemize}
\item {Utilização:Ant.}
\end{itemize}
O mesmo que \textunderscore afruitenegar\textunderscore .
\section{Afruitenegar}
\begin{itemize}
\item {Grp. gram.:v. i.}
\end{itemize}
\begin{itemize}
\item {Utilização:Ant.}
\end{itemize}
O mesmo que \textunderscore afrutar\textunderscore ; tornar-se productivo, depois de sêr estéril, (falando-se de terrenos).
\section{Afrutado}
\begin{itemize}
\item {Grp. gram.:adj.}
\end{itemize}
Carregado de frutos.
Fecundo; prolífico.
\section{Afrutar}
\begin{itemize}
\item {Grp. gram.:v. i.}
\end{itemize}
\begin{itemize}
\item {Grp. gram.:V. t.}
\end{itemize}
\begin{itemize}
\item {Utilização:Prov.}
\end{itemize}
Carregar-se de frutos.
Frutificar.
Pôr a dar fruto; cultivar.
\section{Aftaguir}
\begin{itemize}
\item {Grp. gram.:m.}
\end{itemize}
Pendão, usado nas solennidades gentílicas da Índia Portuguesa. Cf. Th. Ribeiro, \textunderscore Jornadas\textunderscore , II, 113.
\section{Afugentador}
\begin{itemize}
\item {Grp. gram.:m.}
\end{itemize}
O que afugenta.
\section{Afugentamento}
\begin{itemize}
\item {Grp. gram.:m.}
\end{itemize}
Acto de \textunderscore afugentar\textunderscore .
\section{Afugentar}
\begin{itemize}
\item {Grp. gram.:v. t.}
\end{itemize}
\begin{itemize}
\item {Proveniência:(De \textunderscore fugente\textunderscore )}
\end{itemize}
Pôr em fuga.
Repellir: \textunderscore afugentar os inimigos\textunderscore .
\section{Afumados}
\begin{itemize}
\item {Grp. gram.:m. pl.}
\end{itemize}
\begin{itemize}
\item {Utilização:T. de Moncorvo}
\end{itemize}
Arredores, cercanias.
\section{Afumadura}
\begin{itemize}
\item {Grp. gram.:f.}
\end{itemize}
Acto ou effeito de \textunderscore afumar\textunderscore .
\section{Afumar}
\begin{itemize}
\item {Grp. gram.:v. t.}
\end{itemize}
\begin{itemize}
\item {Utilização:Ant.}
\end{itemize}
\begin{itemize}
\item {Proveniência:(De \textunderscore fumo\textunderscore )}
\end{itemize}
Tornar escuro; esfumar.
Tornar cultivado e habitado (um terreno). Cf. Garrett, \textunderscore Camões\textunderscore .
\section{Afumear}
\begin{itemize}
\item {Grp. gram.:v. t.}
\end{itemize}
O mesmo que \textunderscore afumar\textunderscore . Cf. Filinto, XVIII, 232.
\section{Afumegar}
\begin{itemize}
\item {Grp. gram.:v. i.}
\end{itemize}
O mesmo que \textunderscore fumegar\textunderscore . Cf. Herculano, \textunderscore Hist. de Port.\textunderscore , III, 369.
\section{Afundamento}
\begin{itemize}
\item {Grp. gram.:m.}
\end{itemize}
Repressão.
Acto ou effeito de \textunderscore afundar\textunderscore .
\section{Afundar}
\begin{itemize}
\item {Grp. gram.:v. t.}
\end{itemize}
\begin{itemize}
\item {Grp. gram.:V. i.}
\end{itemize}
Meter no fundo.
Escavar.
Examinar: \textunderscore afundar um problema\textunderscore .
Ir ao fundo. Cf. Filinto, IV, 101.
\section{Afundir}
\begin{itemize}
\item {Grp. gram.:v. t.}
\end{itemize}
O mesmo que \textunderscore afundar\textunderscore .
\section{Afundir}
\begin{itemize}
\item {Grp. gram.:v. t.}
\end{itemize}
Banhar em jactos de água.
\section{Afuniladamente}
\begin{itemize}
\item {Grp. gram.:adv.}
\end{itemize}
Á maneira de funil.
\section{Afunilado}
\begin{itemize}
\item {Grp. gram.:adj.}
\end{itemize}
Que tem fórma semelhante á de funil, ou fórma de funil.
\section{Afunilar}
\begin{itemize}
\item {Grp. gram.:v. t.}
\end{itemize}
Dar fórma de funil a.
\section{Afurá}
\begin{itemize}
\item {Grp. gram.:m.}
\end{itemize}
\begin{itemize}
\item {Utilização:Bras}
\end{itemize}
Bolo de arroz fermentado.
\section{Afuroador}
\begin{itemize}
\item {Grp. gram.:m.}
\end{itemize}
O que afurôa.
\section{Afuroar}
\begin{itemize}
\item {Grp. gram.:v. t.}
\end{itemize}
\begin{itemize}
\item {Grp. gram.:V. i.}
\end{itemize}
Lançar o furão a.
Investigar.
Descobrir. Cf. Castilho, \textunderscore Fausto\textunderscore , 44.
Investigar, fazer pesquisas. Cf. Filinto, III, 101.
\section{Afusado}
\begin{itemize}
\item {Grp. gram.:adj.}
\end{itemize}
\begin{itemize}
\item {Proveniência:(De \textunderscore afusar\textunderscore )}
\end{itemize}
Aguçado como um fuso.
\section{Afusal}
\begin{itemize}
\item {Grp. gram.:m.}
\end{itemize}
O mesmo que \textunderscore efusal\textunderscore .
\section{Afusão}
\begin{itemize}
\item {Grp. gram.:f.}
\end{itemize}
\begin{itemize}
\item {Proveniência:(Do lat. \textunderscore ad\textunderscore  + \textunderscore fusio\textunderscore )}
\end{itemize}
Acto de derramar.
Banho.
Aspersão.
Jacto de água no corpo, para se obter resfriamento subito.
\section{Afusar}
\begin{itemize}
\item {Grp. gram.:v. t.}
\end{itemize}
Aguçar como um fuso.
\section{Afustuado}
\begin{itemize}
\item {Grp. gram.:adj.}
\end{itemize}
Que tem fustes. Cf. Benalcanfor, \textunderscore Cartas de Viagem\textunderscore , LVI.
(Má derivação de \textunderscore fuste\textunderscore )
\section{Á-futelifate}
\begin{itemize}
\item {Grp. gram.:loc. adv.}
\end{itemize}
\begin{itemize}
\item {Utilização:Prov.}
\end{itemize}
\begin{itemize}
\item {Utilização:extrem.}
\end{itemize}
\begin{itemize}
\item {Utilização:Des.}
\end{itemize}
Furtivamente, subrepticiamente, ás escondidas.
(Dizem sêr talvez contr. e alter. de \textunderscore furta-lhe o fato\textunderscore )
\section{Afuzilar}
\textunderscore v. t.\textunderscore  e \textunderscore i.\textunderscore  (e der)
O mesmo que \textunderscore fuzilar\textunderscore , etc.
\section{Afzélia}
\begin{itemize}
\item {Proveniência:(De \textunderscore Afzélius\textunderscore , n. p.)}
\end{itemize}
Planta, africana, da fam. das leguminosas.
\section{Agá}
\begin{itemize}
\item {Grp. gram.:m.}
\end{itemize}
Nome da letra \textunderscore H\textunderscore .
\section{Agá}
\begin{itemize}
\item {Grp. gram.:m.}
\end{itemize}
Dignidade militar entre os Turcos.
\section{Agabar}
\textunderscore v. t.\textunderscore  (e der.) \textunderscore Ant.\textunderscore 
O mesmo que \textunderscore gabar\textunderscore , etc.
\section{Agachadamente}
\begin{itemize}
\item {Grp. gram.:adv.}
\end{itemize}
Ás escondidas.
\section{Agachadeira}
\begin{itemize}
\item {Grp. gram.:f.}
\end{itemize}
Pequena ave pernalta do Brasil.
\section{Agachados}
\begin{itemize}
\item {Grp. gram.:m. pl.}
\end{itemize}
\begin{itemize}
\item {Utilização:Bras. de Minas}
\end{itemize}
\begin{itemize}
\item {Proveniência:(De \textunderscore agachar\textunderscore )}
\end{itemize}
Mesuras; adulação.
\section{Agachamento}
\begin{itemize}
\item {Grp. gram.:m.}
\end{itemize}
Acto ou effeito de \textunderscore agachar\textunderscore .
\section{Agachar}
\begin{itemize}
\item {Grp. gram.:v. t.}
\end{itemize}
\begin{itemize}
\item {Grp. gram.:V. p.}
\end{itemize}
Esconder, encobrir.
Abaixar-se, encolher-se, para se esconder.
\section{Agachis}
\begin{itemize}
\item {Grp. gram.:m.}
\end{itemize}
\begin{itemize}
\item {Utilização:Prov.}
\end{itemize}
\begin{itemize}
\item {Utilização:beir.}
\end{itemize}
\begin{itemize}
\item {Proveniência:(De \textunderscore agachar\textunderscore )}
\end{itemize}
Cabana de mato, onde o caçador se agacha, esperando a caça.
\section{Agacho}
\begin{itemize}
\item {Grp. gram.:m.}
\end{itemize}
O mesmo que \textunderscore agachamento\textunderscore .
\section{Agadanhador}
\begin{itemize}
\item {Grp. gram.:m.}
\end{itemize}
O que agadanha.
\section{Agadanhar}
\begin{itemize}
\item {Grp. gram.:v. t.}
\end{itemize}
Agarrar com o gadanho.
Ferir com as unhas; agatanhar.
\section{Agafanhar}
\begin{itemize}
\item {Grp. gram.:v. t.}
\end{itemize}
Agarrar com a gafa.
Empolgar.
\section{Agafita}
\begin{itemize}
\item {Grp. gram.:f.}
\end{itemize}
Turquesa azul, vulgarmente \textunderscore turquesa oriental\textunderscore .
\section{Agaiar}
\begin{itemize}
\item {Grp. gram.:v. t.}
\end{itemize}
\begin{itemize}
\item {Utilização:Prov.}
\end{itemize}
\begin{itemize}
\item {Utilização:trasm.}
\end{itemize}
\begin{itemize}
\item {Grp. gram.:V. p.}
\end{itemize}
Esconder.
Fazer parede (o estudante).
\section{Agaiatado}
\begin{itemize}
\item {Grp. gram.:adj.}
\end{itemize}
Que tem modos de gaiato.
Malicioso.
\section{Agaiatar-se}
\begin{itemize}
\item {Grp. gram.:v. p.}
\end{itemize}
Tornar-se gaiato; adquirir modos de gaiato.
\section{Agajar}
\begin{itemize}
\item {Grp. gram.:v. t.}
\end{itemize}
\begin{itemize}
\item {Utilização:Ant.}
\end{itemize}
Seguir ou acompanhar por obrigação.
(Cp. \textunderscore engajar\textunderscore )
\section{Agalactação}
\begin{itemize}
\item {Grp. gram.:f.}
\end{itemize}
\begin{itemize}
\item {Proveniência:(Do gr. \textunderscore a\textunderscore priv. + \textunderscore gala\textunderscore , leite)}
\end{itemize}
Falta ou suppressão de leite no peito das mulheres, sôbre o parto.
\section{Agalactia}
\begin{itemize}
\item {Grp. gram.:f.}
\end{itemize}
\begin{itemize}
\item {Proveniência:(Do gr. \textunderscore a\textunderscore priv. + \textunderscore gala\textunderscore )}
\end{itemize}
Falta de leite para amamentar.
\section{Agaláctico}
\begin{itemize}
\item {Grp. gram.:adj.}
\end{itemize}
Relativo á agalactia.
\section{Agalanar}
\textunderscore v. t.\textunderscore  (e der.)
(V. \textunderscore engalanar\textunderscore , etc.)
\section{Agalardoar}
\textunderscore v. t.\textunderscore  (e der.) \textunderscore Ant.\textunderscore 
O mesmo que \textunderscore galardoar\textunderscore , etc.
\section{Agalaxia}
\begin{itemize}
\item {fónica:csi}
\end{itemize}
\begin{itemize}
\item {Grp. gram.:f.}
\end{itemize}
O mesmo que \textunderscore agalactia\textunderscore .
\section{Agalegadamente}
\begin{itemize}
\item {Grp. gram.:adv.}
\end{itemize}
Á maneira de galego.
\section{Agalegado}
\begin{itemize}
\item {Grp. gram.:adj.}
\end{itemize}
Relativo aos Galegos.
Que tem modos de galego.
\section{Agalegar}
\begin{itemize}
\item {Grp. gram.:v. t.}
\end{itemize}
Tornar semelhante a galego.
Tornar indelicado.
\section{Agalgado}
\begin{itemize}
\item {Grp. gram.:adj.}
\end{itemize}
Parecido com o galgo.
\section{Agalimar}
\begin{itemize}
\item {Grp. gram.:v. t.}
\end{itemize}
\begin{itemize}
\item {Utilização:Prov.}
\end{itemize}
\begin{itemize}
\item {Utilização:beir.}
\end{itemize}
Acarinhar; afagar.
\section{Agallegadamente}
\begin{itemize}
\item {Grp. gram.:adv.}
\end{itemize}
Á maneira de gallego.
\section{Agallegado}
\begin{itemize}
\item {Grp. gram.:adj.}
\end{itemize}
Relativo aos Gallegos.
Que tem modos de gallego.
\section{Agallegar}
\begin{itemize}
\item {Grp. gram.:v. t.}
\end{itemize}
Tornar semelhante a gallego.
Tornar indelicado.
\section{Agalmatólitho}
\begin{itemize}
\item {Grp. gram.:m.}
\end{itemize}
Talco compacto, de que se fazem na China figuras grutescas.
\section{Agalmatólito}
\begin{itemize}
\item {Grp. gram.:m.}
\end{itemize}
Talco compacto, de que se fazem na China figuras grutescas.
\section{Agaloado}
\begin{itemize}
\item {Grp. gram.:adj.}
\end{itemize}
Guarnecido de galões.
Que usa traje com galões.
\section{Agaloadura}
\begin{itemize}
\item {Grp. gram.:f.}
\end{itemize}
Guarnição de galões.
Acto de \textunderscore agaloar\textunderscore .
\section{Agaloar}
\begin{itemize}
\item {Grp. gram.:v. t.}
\end{itemize}
Guarnecer de galões.
\section{Agálloco}
\begin{itemize}
\item {Grp. gram.:m.}
\end{itemize}
Planta euphorbiácea, cuja madeira é o chamado \textunderscore calambuco\textunderscore .
\section{Agáloco}
\begin{itemize}
\item {Grp. gram.:m.}
\end{itemize}
Planta euphorbiácea, cuja madeira é o chamado \textunderscore calambuco\textunderscore .
\section{Agalostêmono}
\begin{itemize}
\item {Grp. gram.:adj.}
\end{itemize}
\begin{itemize}
\item {Proveniência:(Do gr. \textunderscore galos\textunderscore  + \textunderscore stemon\textunderscore )}
\end{itemize}
Diz-se das plantas, cujos estames existem alternadamente no cálice e na corolla.
\section{Agami}
\begin{itemize}
\item {Grp. gram.:f.}
\end{itemize}
Ave gallinácea da América do Sul.
\section{Agamia}
\begin{itemize}
\item {Grp. gram.:f.}
\end{itemize}
Qualidade das plantas ágamas.
\section{Agamianos}
\begin{itemize}
\item {Grp. gram.:m. pl.}
\end{itemize}
Familia de reptis sáurios.
\section{Ágamo}
\begin{itemize}
\item {Grp. gram.:adj.}
\end{itemize}
\begin{itemize}
\item {Proveniência:(Gr. \textunderscore agamos\textunderscore )}
\end{itemize}
Diz-se das plantas, em que se não conhecem órgãos sexuaes.
\section{Agamogênese}
\begin{itemize}
\item {Grp. gram.:m.}
\end{itemize}
\begin{itemize}
\item {Utilização:Physiol.}
\end{itemize}
\begin{itemize}
\item {Proveniência:(Do gr. \textunderscore agamos\textunderscore  + \textunderscore genesis\textunderscore )}
\end{itemize}
Geração asexual.
\section{Aganado}
\begin{itemize}
\item {Grp. gram.:adj.}
\end{itemize}
\begin{itemize}
\item {Proveniência:(De \textunderscore aganar\textunderscore )}
\end{itemize}
Entanguido, cansado. Cf. \textunderscore Techn. Rur.\textunderscore , 80; Camillo, \textunderscore Brasileira\textunderscore , 205.
\section{Aganar}
\begin{itemize}
\item {Grp. gram.:v. t.}
\end{itemize}
\begin{itemize}
\item {Grp. gram.:V. i.}
\end{itemize}
\begin{itemize}
\item {Utilização:Prov.}
\end{itemize}
\begin{itemize}
\item {Utilização:trasm.}
\end{itemize}
O mesmo que [[entanguir|estanguir-se]].
Offegar, deitando a língua de fóra, (falando-se dos cães).
\section{Aganísia}
\begin{itemize}
\item {Grp. gram.:f.}
\end{itemize}
\begin{itemize}
\item {Proveniência:(Do gr. \textunderscore aganos\textunderscore )}
\end{itemize}
Planta americana, da fam. das orchídeas.
\section{Aganistos}
\begin{itemize}
\item {Grp. gram.:m. pl.}
\end{itemize}
Gênero de insectos lepidópteros.
\section{Ágapa}
\begin{itemize}
\item {Grp. gram.:f.}
\end{itemize}
O mesmo que \textunderscore ágape\textunderscore . Cf. Herculano, \textunderscore Lendas\textunderscore , II, 197.
\section{Agapânthia}
\begin{itemize}
\item {Grp. gram.:f.}
\end{itemize}
\begin{itemize}
\item {Proveniência:(Do gr. \textunderscore agapao\textunderscore  + \textunderscore anthos\textunderscore )}
\end{itemize}
Insecto coleóptero tetrâmero.
\section{Agapantho}
\begin{itemize}
\item {Grp. gram.:m.}
\end{itemize}
\begin{itemize}
\item {Proveniência:(Do gr. \textunderscore agape\textunderscore  + \textunderscore anthos\textunderscore )}
\end{itemize}
Planta liliácea, de origem africana, raiz bulbosa, e flôres azues.
\section{Agapântia}
\begin{itemize}
\item {Grp. gram.:f.}
\end{itemize}
\begin{itemize}
\item {Proveniência:(Do gr. \textunderscore agapao\textunderscore  + \textunderscore anthos\textunderscore )}
\end{itemize}
Insecto coleóptero tetrâmero.
\section{Agapanto}
\begin{itemize}
\item {Grp. gram.:m.}
\end{itemize}
\begin{itemize}
\item {Proveniência:(Do gr. \textunderscore agape\textunderscore  + \textunderscore anthos\textunderscore )}
\end{itemize}
Planta liliácea, de origem africana, raiz bulbosa, e flôres azues.
\section{Ágape}
\begin{itemize}
\item {Grp. gram.:m.}
\end{itemize}
\begin{itemize}
\item {Proveniência:(Gr. \textunderscore agape\textunderscore )}
\end{itemize}
Refeição, que os primeiros Christãos faziam em commum.
\section{Agapetas}
\begin{itemize}
\item {fónica:pê}
\end{itemize}
\begin{itemize}
\item {Grp. gram.:f. pl.}
\end{itemize}
\begin{itemize}
\item {Proveniência:(Gr. \textunderscore agapete\textunderscore )}
\end{itemize}
Virgens ou viúvas, que, nos primeiros tempos do Christianismo, faziam vida commum.
\section{Agapetos}
\begin{itemize}
\item {fónica:pê}
\end{itemize}
\begin{itemize}
\item {Grp. gram.:m. pl.}
\end{itemize}
\begin{itemize}
\item {Proveniência:(Gr. \textunderscore agapetos\textunderscore )}
\end{itemize}
Clérigos, que, nos primeiros tempos do Christianismo, viviam em commum.
\section{Agaporni}
\begin{itemize}
\item {Grp. gram.:m.}
\end{itemize}
\begin{itemize}
\item {Proveniência:(Do gr. \textunderscore agapao\textunderscore  + \textunderscore ornis\textunderscore )}
\end{itemize}
Pequeno papagaio da América do Sul.
\section{Agar}
\begin{itemize}
\item {Grp. gram.:m.}
\end{itemize}
\begin{itemize}
\item {Utilização:Bacter.}
\end{itemize}
O mesmo que \textunderscore gelatina\textunderscore : \textunderscore cultura de bacillos em agar\textunderscore .
\section{Agara}
\begin{itemize}
\item {Grp. gram.:m.}
\end{itemize}
Madeira da China e do Japão, também conhecida por \textunderscore madeira de cheiro\textunderscore .
\section{Agar-agar}
\begin{itemize}
\item {Grp. gram.:m.}
\end{itemize}
Gêlo compacto, preparado pelos Malaios, e de que actualmente se fazem na Europa diversas applicações industriaes.
\section{Agaranis}
\begin{itemize}
\item {Grp. gram.:m. pl.}
\end{itemize}
Indígenas brasileiros das margens do Rio-Branco.
\section{Agaras}
\begin{itemize}
\item {Grp. gram.:f. pl.}
\end{itemize}
Gênero de esponjas, de fibras longitudinaes simples, separadas por membrana finíssima.
\section{Agárdias}
\begin{itemize}
\item {Grp. gram.:f. pl.}
\end{itemize}
Gênero de algas.
\section{Agaré}
\begin{itemize}
\item {Grp. gram.:m.}
\end{itemize}
\begin{itemize}
\item {Utilização:Bras}
\end{itemize}
Planta têxtil.
\section{Agarejo}
\begin{itemize}
\item {Grp. gram.:m.}
\end{itemize}
\begin{itemize}
\item {Utilização:Prov.}
\end{itemize}
\begin{itemize}
\item {Utilização:alent.}
\end{itemize}
O mesmo que \textunderscore algarejo\textunderscore .
\section{Agareno}
\begin{itemize}
\item {Grp. gram.:m.  e  adj.}
\end{itemize}
\begin{itemize}
\item {Proveniência:(De \textunderscore Agar\textunderscore , n. p.)}
\end{itemize}
Descendente de Agar; ismaelita; árabe; moiro.
\section{Agaricáceas}
\begin{itemize}
\item {Grp. gram.:f. pl.}
\end{itemize}
\begin{itemize}
\item {Proveniência:(De \textunderscore agariceo\textunderscore )}
\end{itemize}
Família de cogumelos, que têm por typo o agárico.
\section{Agaríceas}
\begin{itemize}
\item {Grp. gram.:f. pl.}
\end{itemize}
\begin{itemize}
\item {Proveniência:(De \textunderscore agariceo\textunderscore )}
\end{itemize}
Família de cogumelos, que têm por typo o agárico.
\section{Agariceo}
\begin{itemize}
\item {Grp. gram.:adj.}
\end{itemize}
Relativo ou semelhante ao agárico.
\section{Agarícia}
\begin{itemize}
\item {Grp. gram.:f.}
\end{itemize}
Pólypo dos países quentes, semelhante ao agárico.
\section{Agaricícola}
\begin{itemize}
\item {Grp. gram.:adj.}
\end{itemize}
\begin{itemize}
\item {Proveniência:(Do gr. \textunderscore ogaricon\textunderscore  + lat. \textunderscore colere\textunderscore )}
\end{itemize}
Diz-se dos insectos, que vivem nos agáricos.
\section{Agaricina}
\begin{itemize}
\item {Grp. gram.:f.}
\end{itemize}
\begin{itemize}
\item {Utilização:Chím.}
\end{itemize}
Princípio activo do agárico, usado em Medicina para combater os suores frios dos tísicos.
\section{Agaricíneas}
\begin{itemize}
\item {Grp. gram.:f. pl.}
\end{itemize}
O mesmo que \textunderscore agaríceas\textunderscore .
\section{Agárico}
\begin{itemize}
\item {Grp. gram.:m.}
\end{itemize}
\begin{itemize}
\item {Proveniência:(Lat. \textunderscore agaricum\textunderscore )}
\end{itemize}
Nome de vários cogumelos.
\section{Agarimar}
\begin{itemize}
\item {Grp. gram.:v. t.}
\end{itemize}
\begin{itemize}
\item {Utilização:Prov.}
\end{itemize}
\begin{itemize}
\item {Utilização:minh.}
\end{itemize}
O mesmo que \textunderscore agalimar\textunderscore .
\section{Agaristos}
\begin{itemize}
\item {Grp. gram.:m. pl.}
\end{itemize}
\begin{itemize}
\item {Proveniência:(Gr. \textunderscore akharistos\textunderscore )}
\end{itemize}
Gênero de insectos lepidópteros.
\section{Agarnachar}
\begin{itemize}
\item {Grp. gram.:v. t.}
\end{itemize}
Vestir de garnacha.
\section{Ágaro}
\begin{itemize}
\item {Grp. gram.:m.}
\end{itemize}
Gênero de algas dos mares boreaes.
\section{Agarotado}
\begin{itemize}
\item {Grp. gram.:adj.}
\end{itemize}
Semelhante a garoto; que tem modos de garoto.
\section{Agarotar}
\begin{itemize}
\item {Grp. gram.:v. t.}
\end{itemize}
Tornar garoto, travesso.
\section{Agarração}
\begin{itemize}
\item {Grp. gram.:f.}
\end{itemize}
Acto de \textunderscore agarrar\textunderscore :«\textunderscore seria inverosímil agarração\textunderscore ». Camillo, \textunderscore Caveira\textunderscore , 64. Cf. Filinto, I. 94.
\section{Agarradiço}
\begin{itemize}
\item {Grp. gram.:adj.}
\end{itemize}
Costumado a agarrar-se.
\section{Agarrado}
\begin{itemize}
\item {Grp. gram.:adj.}
\end{itemize}
\begin{itemize}
\item {Utilização:Fam.}
\end{itemize}
\begin{itemize}
\item {Proveniência:(De \textunderscore agarrar\textunderscore )}
\end{itemize}
Muito económico.
Sovina, avarento.
\section{Agarrador}
\begin{itemize}
\item {Grp. gram.:m.}
\end{itemize}
\begin{itemize}
\item {Grp. gram.:m.}
\end{itemize}
O que agarra.
Nome de um peixe, o mesmo que \textunderscore rêmora\textunderscore .
\section{Agarrante}
\begin{itemize}
\item {Grp. gram.:m.  e  adj.}
\end{itemize}
Aquelle que agarra. Cf. Herculano, \textunderscore M. de Cister\textunderscore , II, 90.
\section{Agarrar}
\begin{itemize}
\item {Grp. gram.:v. t.}
\end{itemize}
\begin{itemize}
\item {Proveniência:(De \textunderscore garra\textunderscore )}
\end{itemize}
Segurar com garra.
Prender.
Apanhar.
\section{Agarrochar}
\begin{itemize}
\item {Grp. gram.:v. t.}
\end{itemize}
Picar com garrocha.
Estimular, incitar.
\section{Agarrotar}
\textunderscore v. t.\textunderscore  (e der.)
(V. \textunderscore garrotar\textunderscore , etc.)
\section{Agarruchar}
\begin{itemize}
\item {Grp. gram.:v. t.}
\end{itemize}
\begin{itemize}
\item {Utilização:Ant.}
\end{itemize}
Atar com garruchas.
\section{Agarrunchar}
\begin{itemize}
\item {Grp. gram.:v. t.}
\end{itemize}
Ligar com garruncho.
\section{Agasalhadamente}
\begin{itemize}
\item {Grp. gram.:adv.}
\end{itemize}
Com agasalho.
\section{Agasalhadeiro}
\begin{itemize}
\item {Grp. gram.:adj.}
\end{itemize}
Que dá agasalho.
\section{Agasalhador}
\begin{itemize}
\item {Grp. gram.:m.}
\end{itemize}
\begin{itemize}
\item {Grp. gram.:Adj.}
\end{itemize}
Aquelle que agasalha.
Que agasalha.
\section{Agasalhar}
\begin{itemize}
\item {Grp. gram.:v. t.}
\end{itemize}
\begin{itemize}
\item {Grp. gram.:V. p.}
\end{itemize}
\begin{itemize}
\item {Proveniência:(De \textunderscore agasalho\textunderscore )}
\end{itemize}
Hospedar.
Abrigar.
Aquecer.
Dar agasalho a.
Resguardar-se do frio ou da chuva, com roupa de abafo.
\section{Agasalho}
\begin{itemize}
\item {Grp. gram.:m.}
\end{itemize}
Acção de agasalhar.
Hospedagem.
Roupa para aquecer o corpo.
(Cast. \textunderscore agasajo\textunderscore )
\section{Agássia}
\begin{itemize}
\item {Grp. gram.:f.}
\end{itemize}
O mesmo que \textunderscore pêga\textunderscore ^1.
(B. lat. \textunderscore agasia\textunderscore )
\section{Agastadamente}
\begin{itemize}
\item {Grp. gram.:adv.}
\end{itemize}
Com ira ou enfado.
\section{Agastadiço}
\begin{itemize}
\item {Grp. gram.:adj.}
\end{itemize}
Atreito a \textunderscore agastar-se\textunderscore .
\section{Agastado}
\begin{itemize}
\item {Grp. gram.:adj.}
\end{itemize}
\begin{itemize}
\item {Proveniência:(De \textunderscore agastar\textunderscore )}
\end{itemize}
Que se agastou.
Irritado.
Enfurecido.
\section{Agastadura}
\begin{itemize}
\item {Grp. gram.:f.}
\end{itemize}
O mesmo que \textunderscore agastamento\textunderscore .
\section{Agastamento}
\begin{itemize}
\item {Grp. gram.:m.}
\end{itemize}
Acto de \textunderscore agastar\textunderscore .
\section{Agastar}
\begin{itemize}
\item {Grp. gram.:v. t.}
\end{itemize}
\begin{itemize}
\item {Grp. gram.:V. i.}
\end{itemize}
\begin{itemize}
\item {Utilização:Pop.}
\end{itemize}
\begin{itemize}
\item {Proveniência:(De \textunderscore gastar\textunderscore )}
\end{itemize}
Irar, irritar; enfadar.
Têr debilidade, por falta de alimento.
\section{Agastria}
\begin{itemize}
\item {Grp. gram.:f.}
\end{itemize}
\begin{itemize}
\item {Proveniência:(Do gr. \textunderscore a\textunderscore  priv. + \textunderscore gaster\textunderscore )}
\end{itemize}
Qualidade do animal que é agástrico.
\section{Agástrico}
\begin{itemize}
\item {Grp. gram.:adj.}
\end{itemize}
\begin{itemize}
\item {Proveniência:(Do gr. \textunderscore a\textunderscore  priv. + \textunderscore gaster\textunderscore )}
\end{itemize}
Diz-se dos animaes acéphalos, que não têm indícios de canal intestinal.
\section{Agastronervia}
\begin{itemize}
\item {Grp. gram.:f.}
\end{itemize}
\begin{itemize}
\item {Proveniência:(Do gr. \textunderscore a\textunderscore  priv. + \textunderscore gaster\textunderscore  + \textunderscore neuron\textunderscore )}
\end{itemize}
Falta de acção nervosa do estômago.
\section{Agastronomia}
\begin{itemize}
\item {Grp. gram.:f.}
\end{itemize}
\begin{itemize}
\item {Utilização:Med.}
\end{itemize}
Falta de acção nervosa no estômago, o mesmo que \textunderscore agastronervia\textunderscore .
\section{Agastrozoários}
\begin{itemize}
\item {Grp. gram.:m. pl.}
\end{itemize}
\begin{itemize}
\item {Proveniência:(Do gr. \textunderscore a\textunderscore  priv. + \textunderscore gaster\textunderscore  + \textunderscore zoon\textunderscore )}
\end{itemize}
Infusórios, que não têm cavidade digestiva.
\section{Agasturas}
\begin{itemize}
\item {Grp. gram.:f. pl.}
\end{itemize}
\begin{itemize}
\item {Utilização:Prov.}
\end{itemize}
\begin{itemize}
\item {Utilização:alent.}
\end{itemize}
\begin{itemize}
\item {Proveniência:(De \textunderscore agastar\textunderscore )}
\end{itemize}
Debilidade, por falta de alimento.
Necessidade de comer.
\section{Ágata}
\begin{itemize}
\item {Grp. gram.:f.}
\end{itemize}
\begin{itemize}
\item {Proveniência:(Do gr. \textunderscore akhates\textunderscore )}
\end{itemize}
Quartzo translúcido, de côres variadas.
\section{Agatanhadura}
\begin{itemize}
\item {Grp. gram.:f.}
\end{itemize}
\begin{itemize}
\item {Proveniência:(De \textunderscore agatanhar\textunderscore )}
\end{itemize}
Arranhadura.
\section{Agatanhar}
\begin{itemize}
\item {Grp. gram.:v. t.}
\end{itemize}
\begin{itemize}
\item {Proveniência:(De \textunderscore gato\textunderscore ? Por \textunderscore agadanhar\textunderscore , de \textunderscore gadanho\textunderscore ?)}
\end{itemize}
Arranhar, ferir com as unhas.
\section{Agateado}
\begin{itemize}
\item {Grp. gram.:adj.}
\end{itemize}
\begin{itemize}
\item {Utilização:Bras. de Minas}
\end{itemize}
Diz-se dos olhos azulados ou semelhantes aos dos gatos.
\section{Agateia}
\begin{itemize}
\item {Grp. gram.:f.}
\end{itemize}
\begin{itemize}
\item {Proveniência:(Do gr. \textunderscore agatheos\textunderscore )}
\end{itemize}
Planta, da fam. das compostas, e semelhante á cinerária.
\section{Agáteo}
\begin{itemize}
\item {Grp. gram.:adj.}
\end{itemize}
\begin{itemize}
\item {Utilização:Miner.}
\end{itemize}
Que tem veios semelhantes aos da ágata.
\section{Agatheia}
\begin{itemize}
\item {Grp. gram.:f.}
\end{itemize}
\begin{itemize}
\item {Proveniência:(Do gr. \textunderscore agatheos\textunderscore )}
\end{itemize}
Planta, da fam. das compostas, e semelhante á cinerária.
\section{Agathídia}
\begin{itemize}
\item {Grp. gram.:f.}
\end{itemize}
\begin{itemize}
\item {Proveniência:(Do gr. \textunderscore agathis\textunderscore )}
\end{itemize}
Insecto coleóptero tetrâmero, que vive nos cogumelos.
\section{Agathino}
\begin{itemize}
\item {Grp. gram.:m.}
\end{itemize}
Gênero de molluscos gasterópodes.
\section{Agathoide}
\begin{itemize}
\item {Grp. gram.:adj.}
\end{itemize}
\begin{itemize}
\item {Proveniência:(Gr. \textunderscore agathoeides\textunderscore )}
\end{itemize}
Que possue a natureza do bem.
Benigno.
\section{Agati}
\begin{itemize}
\item {Grp. gram.:m.}
\end{itemize}
Planta leguminosa, de grãos comestíveis, originária da Índia.
\section{Agatídia}
\begin{itemize}
\item {Grp. gram.:f.}
\end{itemize}
\begin{itemize}
\item {Proveniência:(Do gr. \textunderscore agathis\textunderscore )}
\end{itemize}
Insecto coleóptero tetrâmero, que vive nos cogumelos.
\section{Agatífero}
\begin{itemize}
\item {Grp. gram.:adj.}
\end{itemize}
\begin{itemize}
\item {Utilização:Geol.}
\end{itemize}
\begin{itemize}
\item {Proveniência:(Do gr. \textunderscore akhates\textunderscore  + lat. \textunderscore ferre\textunderscore )}
\end{itemize}
Que contém ágata.
\section{Agatificar}
\begin{itemize}
\item {Grp. gram.:v. t.}
\end{itemize}
\begin{itemize}
\item {Proveniência:(Do gr. \textunderscore akhates\textunderscore  + lat. \textunderscore facere\textunderscore )}
\end{itemize}
Transformar em ágata.
\section{Agatinhar}
\begin{itemize}
\item {Grp. gram.:v. t.}
\end{itemize}
Subir, trepando com difficuldade:«\textunderscore Aníbal, agatinhando as agruras dos Alpes\textunderscore ». Camillo, \textunderscore Sc. da Foz\textunderscore , 4.^a ed., 37.
(Cp. \textunderscore gatinha\textunderscore )
\section{Agatino}
\begin{itemize}
\item {Grp. gram.:adj.}
\end{itemize}
O mesmo que \textunderscore agáteo\textunderscore .
\section{Agatino}
\begin{itemize}
\item {Grp. gram.:m.}
\end{itemize}
Gênero de molluscos gasterópodes.
\section{Agatizar}
\begin{itemize}
\item {Grp. gram.:v. t.}
\end{itemize}
Converter em ágata.
\section{Agatoide}
\begin{itemize}
\item {Grp. gram.:adj.}
\end{itemize}
\begin{itemize}
\item {Proveniência:(Do gr. \textunderscore akhates\textunderscore  + \textunderscore eidos\textunderscore )}
\end{itemize}
Semelhante a ágata.
\section{Agatoide}
\begin{itemize}
\item {Grp. gram.:adj.}
\end{itemize}
\begin{itemize}
\item {Proveniência:(Gr. \textunderscore agathoeides\textunderscore )}
\end{itemize}
Que possue a natureza do bem.
Benigno.
\section{Agatosmo}
\begin{itemize}
\item {Grp. gram.:m.}
\end{itemize}
Planta aromática, da fam. das diósmeas.
\section{Agatomérida}
\begin{itemize}
\item {Grp. gram.:f.}
\end{itemize}
Planta corymbífera.
\section{Agaturrar}
\begin{itemize}
\item {Grp. gram.:v. t.}
\end{itemize}
\begin{itemize}
\item {Utilização:Bras}
\end{itemize}
\begin{itemize}
\item {Proveniência:(De \textunderscore caturar\textunderscore , =\textunderscore capturar\textunderscore , sob a infl. de \textunderscore gato\textunderscore  e \textunderscore agarrar\textunderscore )}
\end{itemize}
Agarrar; prender.
\section{Agauchar-se}
\begin{itemize}
\item {fónica:ga-u}
\end{itemize}
\begin{itemize}
\item {Grp. gram.:v. p.}
\end{itemize}
\begin{itemize}
\item {Utilização:Bras}
\end{itemize}
Tomar hábitos de gaúcho.
\section{Agave}
\begin{itemize}
\item {Grp. gram.:f.}
\end{itemize}
\begin{itemize}
\item {Proveniência:(Fr. \textunderscore agave\textunderscore )}
\end{itemize}
Planta americana, da fam. das amaryllídeas.
\section{Agávea}
\begin{itemize}
\item {Grp. gram.:f.}
\end{itemize}
\begin{itemize}
\item {Utilização:Bras}
\end{itemize}
O mesmo que \textunderscore agave\textunderscore .
\section{Agavelar}
\begin{itemize}
\item {Grp. gram.:v. t.}
\end{itemize}
Juntar e atar em gavelas; engavelar.
\section{Agazar}
\begin{itemize}
\item {Grp. gram.:v. t.}
\end{itemize}
\begin{itemize}
\item {Utilização:Prov.}
\end{itemize}
\begin{itemize}
\item {Utilização:minh.}
\end{itemize}
Apupar.
(Colhido em Lanhoso)
\section{Agazela}
\begin{itemize}
\item {Grp. gram.:f.}
\end{itemize}
\begin{itemize}
\item {Utilização:Ant.}
\end{itemize}
(V.gazela)
\section{Agazuado}
\begin{itemize}
\item {Grp. gram.:adj.}
\end{itemize}
Semelhante a gazua.
\section{Agazular}
\begin{itemize}
\item {Grp. gram.:v. t.}
\end{itemize}
\begin{itemize}
\item {Utilização:Prov.}
\end{itemize}
\begin{itemize}
\item {Utilização:trasm.}
\end{itemize}
Catrafilar.
Agarrar pela golla da véstia.
\section{Agédula}
\begin{itemize}
\item {Grp. gram.:f.}
\end{itemize}
Cúpula de alguns cogumelos.
Urnário de certos musgos.
\section{Ageia}
\begin{itemize}
\item {Grp. gram.:f.}
\end{itemize}
\begin{itemize}
\item {Proveniência:(Lat. \textunderscore agea\textunderscore )}
\end{itemize}
Passagem ou corredor, nas embarcações romanas, por onde o mestre da equipagem communicava com os remadores.
\section{Agedoíte}
\begin{itemize}
\item {Grp. gram.:f.}
\end{itemize}
Substância crystallina, que se extrai do alcaçus.
\section{Agedra}
\begin{itemize}
\item {Grp. gram.:f.}
\end{itemize}
\begin{itemize}
\item {Utilização:Ant.}
\end{itemize}
O mesmo que \textunderscore mangerona\textunderscore .
\section{Agegelado}
\begin{itemize}
\item {Grp. gram.:adj.}
\end{itemize}
\begin{itemize}
\item {Utilização:Ant.}
\end{itemize}
Dizia-se do terreno que, sendo inclinado, se fez plano para sêr cultivado em leiras.
\section{Ageitar}
\textunderscore v. t.\textunderscore  (e der.)
(V. \textunderscore ajeitar\textunderscore , etc.)
\section{Agelaia}
\begin{itemize}
\item {Grp. gram.:f.}
\end{itemize}
\begin{itemize}
\item {Proveniência:(Gr. \textunderscore agelaios\textunderscore )}
\end{itemize}
Insecto hymenóptero.
\section{Agelástica}
\begin{itemize}
\item {Grp. gram.:f.}
\end{itemize}
\begin{itemize}
\item {Proveniência:(Do gr. \textunderscore agelastos\textunderscore )}
\end{itemize}
Gênero de insectos coleópteros.
\section{Ageleia}
\begin{itemize}
\item {Grp. gram.:f.}
\end{itemize}
\begin{itemize}
\item {Proveniência:(Gr. \textunderscore agelaios\textunderscore )}
\end{itemize}
Insecto hymenóptero.
\section{Agelena}
\begin{itemize}
\item {Grp. gram.:f.}
\end{itemize}
\begin{itemize}
\item {Proveniência:(Do gr. \textunderscore agele\textunderscore )}
\end{itemize}
Espécie de aranha.
\section{...agem}
\begin{itemize}
\item {Grp. gram.:suf.}
\end{itemize}
(design. de acção continuada)
\section{Agema}
\begin{itemize}
\item {Grp. gram.:f.}
\end{itemize}
\begin{itemize}
\item {Proveniência:(Gr. \textunderscore agema\textunderscore )}
\end{itemize}
Divisão militar, espécie de guarda real, no antigo exército macedónio.
\section{Agenceio}
\begin{itemize}
\item {Grp. gram.:m.}
\end{itemize}
\begin{itemize}
\item {Utilização:Prov.}
\end{itemize}
\begin{itemize}
\item {Utilização:trasm.}
\end{itemize}
\begin{itemize}
\item {Proveniência:(De \textunderscore agenciar\textunderscore )}
\end{itemize}
Ganho, provento.
\section{Agência}
\begin{itemize}
\item {Grp. gram.:f.}
\end{itemize}
\begin{itemize}
\item {Proveniência:(De \textunderscore agente\textunderscore )}
\end{itemize}
Actividade; indústria.
Funcções de agentes.
Retribuïção do agente.
Estabelecimento, onde se tratam negócios por conta alheia.
\section{Agenciador}
\begin{itemize}
\item {Grp. gram.:m.}
\end{itemize}
O que agencía.
\section{Agenciar}
\begin{itemize}
\item {Grp. gram.:v. t.}
\end{itemize}
\begin{itemize}
\item {Proveniência:(De \textunderscore agência\textunderscore )}
\end{itemize}
Negociar.
Promover.
\section{Agencioso}
\begin{itemize}
\item {Grp. gram.:adj.}
\end{itemize}
\begin{itemize}
\item {Proveniência:(De \textunderscore agenciar\textunderscore )}
\end{itemize}
Diligente; activo.
Que agencía.
\section{Agenda}
\begin{itemize}
\item {Grp. gram.:f.}
\end{itemize}
\begin{itemize}
\item {Proveniência:(Lat. \textunderscore agenda\textunderscore )}
\end{itemize}
Carteira, taboleta ou quadro, em que se nota o que se tem de fazer.
\section{Agênere}
\begin{itemize}
\item {Grp. gram.:f.}
\end{itemize}
\begin{itemize}
\item {Utilização:Espir.}
\end{itemize}
Apparição tangível, ou estado de certos espíritos, que podem momentaneamente assumir a fórma de pessôas vivas, produzindo perfeita illusão.
\section{Agenesia}
\begin{itemize}
\item {Grp. gram.:f.}
\end{itemize}
\begin{itemize}
\item {Proveniência:(Do gr. \textunderscore a\textunderscore  priv. + \textunderscore genesis\textunderscore )}
\end{itemize}
Impossibilidade de gerar.
\section{Agenésico}
\begin{itemize}
\item {Grp. gram.:adj.}
\end{itemize}
\begin{itemize}
\item {Proveniência:(De \textunderscore agenesia\textunderscore )}
\end{itemize}
Que não póde gerar.
\section{Ágeno}
\begin{itemize}
\item {Grp. gram.:adj.}
\end{itemize}
\begin{itemize}
\item {Proveniência:(Do gr. \textunderscore a\textunderscore  priv. + \textunderscore genea\textunderscore )}
\end{itemize}
Diz-se dos vegetaes cellulares, segundo alguns naturalistas.
\section{Agenória}
\begin{itemize}
\item {Grp. gram.:f.}
\end{itemize}
\begin{itemize}
\item {Proveniência:(De \textunderscore Agenor\textunderscore , n. p.)}
\end{itemize}
Gênero de plantas.
\section{Agenosômo}
\begin{itemize}
\item {Grp. gram.:m.}
\end{itemize}
\begin{itemize}
\item {Proveniência:(Do gr. \textunderscore agenos\textunderscore  + \textunderscore soma\textunderscore )}
\end{itemize}
Monstruosidade, com eventração lateral ou mediana, em a parte inferior do abdome.
\section{Agente}
\begin{itemize}
\item {Grp. gram.:m.}
\end{itemize}
\begin{itemize}
\item {Utilização:Gram.}
\end{itemize}
\begin{itemize}
\item {Grp. gram.:Adj.}
\end{itemize}
\begin{itemize}
\item {Proveniência:(Lat. \textunderscore agens\textunderscore )}
\end{itemize}
Tudo o que opéra.
Aquelle que trata de negócios alheios.
A causa, o autor.
Sujeito da oração, se o verbo respectivo é transitivo.
Aquelle ou aquillo, a que é devida a acção significada pelo verbo passivo.
Que opéra.
\section{Ageolhar}
\begin{itemize}
\item {Grp. gram.:v. i.}
\end{itemize}
\begin{itemize}
\item {Utilização:Ant.}
\end{itemize}
\begin{itemize}
\item {Proveniência:(De \textunderscore geôlho\textunderscore )}
\end{itemize}
O mesmo que \textunderscore ajoelhar\textunderscore .
\section{Ageometria}
\begin{itemize}
\item {Grp. gram.:f.}
\end{itemize}
O mesmo que \textunderscore ageometrosia\textunderscore .
\section{Ageometrosia}
\begin{itemize}
\item {Grp. gram.:f.}
\end{itemize}
Ignorância dos princípios da Geometria.
(Do grego \textunderscore a\textunderscore  priv. + \textunderscore geos\textunderscore  + \textunderscore metron\textunderscore )
\section{Agerasia}
\begin{itemize}
\item {Grp. gram.:f.}
\end{itemize}
\begin{itemize}
\item {Proveniência:(Do gr. \textunderscore a\textunderscore  priv. + \textunderscore geras\textunderscore )}
\end{itemize}
Qualidade de quem não envelhece.
Velhice robusta.
\section{Agerato}
\begin{itemize}
\item {Grp. gram.:m.}
\end{itemize}
Planta ornamental, de bellas flôres azues.
\section{Agermanar}
\begin{itemize}
\item {Grp. gram.:v. t.}
\end{itemize}
\begin{itemize}
\item {Proveniência:(Do lat. \textunderscore germanus\textunderscore )}
\end{itemize}
Tornar irmão; igualar.
\section{Agermolho}
\begin{itemize}
\item {fónica:mô}
\end{itemize}
\begin{itemize}
\item {Grp. gram.:m.}
\end{itemize}
\begin{itemize}
\item {Utilização:Prov.}
\end{itemize}
\begin{itemize}
\item {Utilização:alg.}
\end{itemize}
O mesmo que \textunderscore caspacho\textunderscore .
(Por \textunderscore agromolho\textunderscore , de \textunderscore agro\textunderscore  + \textunderscore môlho\textunderscore ?)
\section{Ageropigado}
\begin{itemize}
\item {Grp. gram.:adj.}
\end{itemize}
Que tem sabor de geropiga. Cf. \textunderscore Techn. Rur.\textunderscore , 38.
\section{Agestrata}
\begin{itemize}
\item {Grp. gram.:f.}
\end{itemize}
\begin{itemize}
\item {Proveniência:(Gr. \textunderscore agestratos\textunderscore )}
\end{itemize}
Insecto coleóptero pentâmero, originário de Java e da China.
\section{Ageustia}
\begin{itemize}
\item {Grp. gram.:f.}
\end{itemize}
\begin{itemize}
\item {Proveniência:(Do gr. \textunderscore a\textunderscore  priv. + \textunderscore geusis\textunderscore )}
\end{itemize}
Ausência de paladar.
Deminuíção do sentido do gôsto.
\section{Aggédula}
\begin{itemize}
\item {Grp. gram.:f.}
\end{itemize}
Cúpula de alguns cogumelos.
Urnário de certos musgos.
\section{Agglomeração}
\begin{itemize}
\item {Grp. gram.:f.}
\end{itemize}
Ajuntamento.
Acto de \textunderscore agglomerar\textunderscore .
\section{Agglomerado}
\begin{itemize}
\item {Grp. gram.:M.}
\end{itemize}
\begin{itemize}
\item {Grp. gram.:Pl.}
\end{itemize}
\begin{itemize}
\item {Utilização:Zool.}
\end{itemize}
Conjunto de cimento e pedras, que imita mármore.
Argamassa hydráulica de cimento e pedra britada.
Fragmentos, que constituem uma rocha elástica, sem que se lhes interponha substância alguma, estranha.
(Cp. \textunderscore conglomerados\textunderscore )
\section{Agglomerar}
\begin{itemize}
\item {Grp. gram.:v. t.}
\end{itemize}
\begin{itemize}
\item {Proveniência:(Lat. \textunderscore agglomerare\textunderscore )}
\end{itemize}
Ajuntar; acumular.
\section{Agglutinação}
\begin{itemize}
\item {Grp. gram.:f.}
\end{itemize}
\begin{itemize}
\item {Proveniência:(Lat. \textunderscore agglutinatio\textunderscore )}
\end{itemize}
Acto de agglutinar.
\section{Agglutinante}
\begin{itemize}
\item {Grp. gram.:adj.}
\end{itemize}
\begin{itemize}
\item {Proveniência:(Lat. \textunderscore agglutinans\textunderscore )}
\end{itemize}
Que agglutina.
\section{Agglutinar}
\begin{itemize}
\item {Grp. gram.:v. t.}
\end{itemize}
\begin{itemize}
\item {Proveniência:(Lat. \textunderscore agglutinare\textunderscore )}
\end{itemize}
Collar.
Unir.
Reunir (palavras) a outra, principal, para formar um todo.
\section{Agglutinativo}
\begin{itemize}
\item {Grp. gram.:adj.}
\end{itemize}
O mesmo que \textunderscore agglutinante\textunderscore .
\section{Aggravação}
\begin{itemize}
\item {Grp. gram.:f.}
\end{itemize}
O mesmo que \textunderscore aggravamento\textunderscore .
\section{Aggravadamente}
\begin{itemize}
\item {Grp. gram.:adv.}
\end{itemize}
Com aggravamento.
\section{Aggravador}
\begin{itemize}
\item {Grp. gram.:m.  e  adj.}
\end{itemize}
O que aggrava.
\section{Aggravamento}
\begin{itemize}
\item {Grp. gram.:m.}
\end{itemize}
Acto de \textunderscore aggravar\textunderscore .
\section{Aggravante}
\begin{itemize}
\item {Grp. gram.:adj.}
\end{itemize}
\begin{itemize}
\item {Proveniência:(Lat. \textunderscore aggravans\textunderscore )}
\end{itemize}
Que aggrava.
\section{Aggravar}
\begin{itemize}
\item {Grp. gram.:v. t.}
\end{itemize}
\begin{itemize}
\item {Proveniência:(Lat. \textunderscore aggravare\textunderscore )}
\end{itemize}
Tornar grave.
Augmentar.
Fazer aggravo a.
\section{Aggravista}
\begin{itemize}
\item {Grp. gram.:m.}
\end{itemize}
\begin{itemize}
\item {Utilização:Ant.}
\end{itemize}
Juiz, que, nos tribunaes superiores, tomava conhecimento dos aggravos e os julgava.
\section{Aggravo}
\begin{itemize}
\item {Grp. gram.:m.}
\end{itemize}
\begin{itemize}
\item {Proveniência:(De \textunderscore aggravar\textunderscore )}
\end{itemize}
Offensa; injúria.
Recurso judicial, contra uma presumida injustiça.
\section{Aggravoso}
\begin{itemize}
\item {Grp. gram.:adj.}
\end{itemize}
Que causa aggravo.
\section{Aggredir}
\begin{itemize}
\item {Grp. gram.:v. t.}
\end{itemize}
\begin{itemize}
\item {Proveniência:(Lat. \textunderscore aggredi\textunderscore )}
\end{itemize}
Atacar.
Assaltar.
Insultar.
Ir contra.
\section{Aggregação}
\begin{itemize}
\item {Grp. gram.:f.}
\end{itemize}
Acto de \textunderscore aggregar\textunderscore .
\section{Aggregado}
\begin{itemize}
\item {Grp. gram.:adj.}
\end{itemize}
\begin{itemize}
\item {Grp. gram.:M.}
\end{itemize}
\begin{itemize}
\item {Proveniência:(De \textunderscore aggregar\textunderscore )}
\end{itemize}
Annexo.
Adjunto.
Reunido.
Funccionário, que um diploma legal aggrega a uma classe ou corporação, além do número normal dos funccionários dessa corporação ou classe: \textunderscore na Relação de Lisbôa há quatro aggregados\textunderscore .
\section{Aggregar}
\begin{itemize}
\item {Grp. gram.:v. t.}
\end{itemize}
\begin{itemize}
\item {Proveniência:(Lat. \textunderscore aggregare\textunderscore )}
\end{itemize}
Ajuntar, anexar.
Associar.
\section{Aggregativo}
\begin{itemize}
\item {Grp. gram.:adj.}
\end{itemize}
Que aggrega.
\section{Aggregato}
\begin{itemize}
\item {Grp. gram.:adj.}
\end{itemize}
(V.aggregado)
\section{Aggressão}
\begin{itemize}
\item {Grp. gram.:f.}
\end{itemize}
\begin{itemize}
\item {Proveniência:(Lat. \textunderscore aggressio\textunderscore )}
\end{itemize}
Ataque.
Acto de aggredir.
\section{Aggressivamente}
\begin{itemize}
\item {Grp. gram.:adv.}
\end{itemize}
De modo \textunderscore aggressivo\textunderscore .
\section{Aggressivo}
\begin{itemize}
\item {Grp. gram.:adj.}
\end{itemize}
\begin{itemize}
\item {Proveniência:(Do lat. \textunderscore aggressus\textunderscore )}
\end{itemize}
Que envolve aggressão.
\section{Aggressor}
\begin{itemize}
\item {Grp. gram.:m.  e  adj.}
\end{itemize}
\begin{itemize}
\item {Proveniência:(Lat. \textunderscore aggressor\textunderscore )}
\end{itemize}
O que aggride.
Provocador.
\section{Aggressório}
\begin{itemize}
\item {Grp. gram.:adj.}
\end{itemize}
O mesmo que \textunderscore aggressivo\textunderscore .
\section{Aggriçar}
\begin{itemize}
\item {Grp. gram.:v. i.}
\end{itemize}
\begin{itemize}
\item {Utilização:Náut.}
\end{itemize}
\begin{itemize}
\item {Utilização:ant.}
\end{itemize}
Pôr a cordagem.
\section{Agicrânio}
\begin{itemize}
\item {Grp. gram.:m.}
\end{itemize}
Ornato de esculptura antiga, o qual representa uma cabeça de cabra ou de bode.
\section{Ágidas}
\begin{itemize}
\item {Grp. gram.:m. pl.}
\end{itemize}
Uma das dynastias dos reis espartanos. Cf. Camillo, \textunderscore Márt.\textunderscore , II, 52.
\section{Agigantadamente}
\begin{itemize}
\item {Grp. gram.:adv.}
\end{itemize}
De modo descommunal.
Á semelhança de gigante.
\section{Agigantado}
\begin{itemize}
\item {Grp. gram.:adj.}
\end{itemize}
Que tem aspecto de gigante.
Muito encorpado.
Descommunal.
\section{Agigantamento}
\begin{itemize}
\item {Grp. gram.:m.}
\end{itemize}
Acto de \textunderscore agigantar\textunderscore .
\section{Agigantar}
\begin{itemize}
\item {Grp. gram.:v. t.}
\end{itemize}
Tornar gigante.
Engrandecer.
Avolumar.
\section{Agigantear}
\begin{itemize}
\item {Grp. gram.:v. t.}
\end{itemize}
O mesmo que \textunderscore agigantar\textunderscore . Cf. Castilho, \textunderscore Montalverne\textunderscore .
\section{Ágil}
\begin{itemize}
\item {Grp. gram.:adj.}
\end{itemize}
\begin{itemize}
\item {Proveniência:(Lat. \textunderscore agilis\textunderscore )}
\end{itemize}
Ligeiro.
Leve.
Que tem facilidade em se mover.
\section{Agilidade}
\begin{itemize}
\item {Grp. gram.:f.}
\end{itemize}
\begin{itemize}
\item {Proveniência:(Lat. \textunderscore agilitas\textunderscore )}
\end{itemize}
Presteza.
Desembaraço.
\section{Agilitar}
\begin{itemize}
\item {Grp. gram.:v. t.}
\end{itemize}
Tornar ágil.
Desenvolver.
\section{Agilmente}
\begin{itemize}
\item {Grp. gram.:adv.}
\end{itemize}
\begin{itemize}
\item {Proveniência:(De \textunderscore ágil\textunderscore )}
\end{itemize}
Com agilidade.
\section{Aginha}
\begin{itemize}
\item {Grp. gram.:adv.}
\end{itemize}
\begin{itemize}
\item {Utilização:Ant.}
\end{itemize}
O mesmo que \textunderscore azinha\textunderscore ^1.
\section{Aginhado}
\begin{itemize}
\item {Grp. gram.:adv.}
\end{itemize}
\begin{itemize}
\item {Utilização:Ant.}
\end{itemize}
\begin{itemize}
\item {Proveniência:(De \textunderscore aginha\textunderscore )}
\end{itemize}
Promptamente.
\section{Ágio}
\begin{itemize}
\item {Grp. gram.:m.}
\end{itemize}
\begin{itemize}
\item {Proveniência:(It. \textunderscore aggio\textunderscore )}
\end{itemize}
Benefício, resultante de câmbio.
Usura.
Especulação e jôgo de fundos públicos.
\section{Agiografia}
\begin{itemize}
\item {Grp. gram.:f.}
\end{itemize}
\begin{itemize}
\item {Proveniência:(De \textunderscore agiógrapho\textunderscore )}
\end{itemize}
História de santos.
\section{Agiógrafo}
\begin{itemize}
\item {Grp. gram.:m.}
\end{itemize}
\begin{itemize}
\item {Proveniência:(Do gr. \textunderscore agios\textunderscore  + \textunderscore graphein\textunderscore )}
\end{itemize}
Aquelle que descreve vidas de santos.
\section{Agiographia}
\begin{itemize}
\item {Grp. gram.:f.}
\end{itemize}
\begin{itemize}
\item {Proveniência:(De \textunderscore agiógrapho\textunderscore )}
\end{itemize}
História de santos.
\section{Agiógrapho}
\begin{itemize}
\item {Grp. gram.:m.}
\end{itemize}
\begin{itemize}
\item {Proveniência:(Do gr. \textunderscore agios\textunderscore  + \textunderscore graphein\textunderscore )}
\end{itemize}
Aquelle que descreve vidas de santos.
\section{Agiólogico}
\begin{itemize}
\item {Grp. gram.:adj.}
\end{itemize}
Relativo ao \textunderscore agiológio\textunderscore .
\section{Agiológio}
\begin{itemize}
\item {Grp. gram.:m.}
\end{itemize}
\begin{itemize}
\item {Proveniência:(Do gr. \textunderscore agios\textunderscore  + \textunderscore logion\textunderscore )}
\end{itemize}
Tratado de santos.
\section{Agiólogo}
\begin{itemize}
\item {Grp. gram.:m.}
\end{itemize}
\begin{itemize}
\item {Proveniência:(Do gr. \textunderscore agios\textunderscore  + \textunderscore logos\textunderscore )}
\end{itemize}
Aquelle que escreve á cêrca de santos.
\section{Agiospermia}
\begin{itemize}
\item {Grp. gram.:f.}
\end{itemize}
Segunda ordem de uma das classes botânicas de Linneu.
\section{Agiospérmico}
\begin{itemize}
\item {Grp. gram.:adj.}
\end{itemize}
\begin{itemize}
\item {Proveniência:(De \textunderscore agiosperma\textunderscore )}
\end{itemize}
Diz-se dos vegetaes, cujos grãos estão revestidos por um pericarpo distinto.
\section{Agiostíride}
\begin{itemize}
\item {Grp. gram.:f.}
\end{itemize}
\begin{itemize}
\item {Utilização:Ant.}
\end{itemize}
Oratório com portas.
\section{Agiostýride}
\begin{itemize}
\item {Grp. gram.:f.}
\end{itemize}
\begin{itemize}
\item {Utilização:Ant.}
\end{itemize}
Oratório com portas.
\section{Agiota}
\begin{itemize}
\item {Grp. gram.:m.}
\end{itemize}
Aquelle que procura ágio.
Usurário.
Homem interesseiro.
\section{Agiotado}
\begin{itemize}
\item {Grp. gram.:adj.}
\end{itemize}
Contratado com agiotas: \textunderscore empréstimo agiotado\textunderscore .
\section{Agiotador}
\begin{itemize}
\item {Grp. gram.:m.  e  adj.}
\end{itemize}
\begin{itemize}
\item {Utilização:Ant.}
\end{itemize}
\begin{itemize}
\item {Proveniência:(De \textunderscore agiotar\textunderscore )}
\end{itemize}
O que exerce agiotagem.
Ladrão astucioso.
\section{Agiotagem}
\begin{itemize}
\item {Grp. gram.:f.}
\end{itemize}
\begin{itemize}
\item {Proveniência:(De \textunderscore agiotar\textunderscore )}
\end{itemize}
Usura.
Especulação exaggerada.
\section{Agiotar}
\begin{itemize}
\item {Grp. gram.:v. i.}
\end{itemize}
\begin{itemize}
\item {Proveniência:(De \textunderscore agiota\textunderscore )}
\end{itemize}
Exercer agiotagem.
\section{Agiotista}
\begin{itemize}
\item {Grp. gram.:m.}
\end{itemize}
\begin{itemize}
\item {Proveniência:(De \textunderscore agiota\textunderscore )}
\end{itemize}
Agiotador.
Aquelle que joga com fundos públicos.
\section{Agir}
\begin{itemize}
\item {Grp. gram.:v. i.}
\end{itemize}
\begin{itemize}
\item {Grp. gram.:V. t.}
\end{itemize}
\begin{itemize}
\item {Utilização:Des.}
\end{itemize}
\begin{itemize}
\item {Proveniência:(Lat. \textunderscore agere\textunderscore )}
\end{itemize}
Proceder.
Pôr em prática um plano, um intuito.
Obrar, realizar.
\section{Agirafado}
\begin{itemize}
\item {Grp. gram.:adj.}
\end{itemize}
\begin{itemize}
\item {Utilização:Burl.}
\end{itemize}
Esguio como a girafa: \textunderscore estas inglesas agirafadas...\textunderscore 
\section{Agironado}
\begin{itemize}
\item {Grp. gram.:adj.}
\end{itemize}
\begin{itemize}
\item {Utilização:Ant.}
\end{itemize}
\begin{itemize}
\item {Proveniência:(De \textunderscore girão\textunderscore )}
\end{itemize}
Debruado; guarnecido.
\section{Agitação}
\begin{itemize}
\item {Grp. gram.:f.}
\end{itemize}
Perturbação.
Acto de \textunderscore agitar\textunderscore .
\section{Agitadamente}
\begin{itemize}
\item {Grp. gram.:adv.}
\end{itemize}
Com agitação.
\section{Agitadiço}
\begin{itemize}
\item {Grp. gram.:adv.}
\end{itemize}
Que se agita facilmente; que se agita muitas vezes.
\section{Agitador}
\begin{itemize}
\item {Grp. gram.:m.}
\end{itemize}
Aquelle que agita.
\section{Agitamento}
\begin{itemize}
\item {Grp. gram.:m.}
\end{itemize}
O mesmo que \textunderscore agitação\textunderscore .
\section{Agitante}
\begin{itemize}
\item {Grp. gram.:adj.}
\end{itemize}
\begin{itemize}
\item {Proveniência:(Lat. \textunderscore agitans\textunderscore )}
\end{itemize}
Que agita.
\section{Agitar}
\begin{itemize}
\item {Grp. gram.:v. t.}
\end{itemize}
\begin{itemize}
\item {Proveniência:(Lat. \textunderscore agitare\textunderscore )}
\end{itemize}
Mover com frequência.
Abalar; commover.
Suscitar: \textunderscore agitar discussões\textunderscore .
\section{Agitato}
\begin{itemize}
\item {Grp. gram.:adv.}
\end{itemize}
\begin{itemize}
\item {Utilização:Mús.}
\end{itemize}
T. it., que indica que um trecho musical se deve executar com agitação.
\section{Agitável}
\begin{itemize}
\item {Grp. gram.:adj.}
\end{itemize}
Que se póde \textunderscore agitar\textunderscore .
\section{Agla}
\begin{itemize}
\item {Grp. gram.:f.}
\end{itemize}
O mesmo que \textunderscore águila\textunderscore .
\section{Aglaia}
\begin{itemize}
\item {Grp. gram.:f.}
\end{itemize}
\begin{itemize}
\item {Proveniência:(Gr. \textunderscore aglaia\textunderscore )}
\end{itemize}
Gênero de plantas meliáceas.
\section{Áglia}
\begin{itemize}
\item {Grp. gram.:f.}
\end{itemize}
Gênero de insectos lepidópteros nocturnos.
\section{Aglobulia}
\begin{itemize}
\item {Grp. gram.:f.}
\end{itemize}
\begin{itemize}
\item {Proveniência:(De \textunderscore a\textunderscore  priv. e \textunderscore glóbulo\textunderscore )}
\end{itemize}
Deminuição dos glóbulos vermelhos do sangue.
\section{Aglomeração}
\begin{itemize}
\item {Grp. gram.:f.}
\end{itemize}
Ajuntamento.
Acto de \textunderscore aglomerar\textunderscore .
\section{Aglomerar}
\begin{itemize}
\item {Grp. gram.:v. t.}
\end{itemize}
\begin{itemize}
\item {Proveniência:(Lat. \textunderscore agglomerare\textunderscore )}
\end{itemize}
Ajuntar; acumular.
\section{Aglossa}
\begin{itemize}
\item {Grp. gram.:f.}
\end{itemize}
Gênero de insectos lepidópteros, que se alimentam de manteiga e toicinho.
(Fem. de \textunderscore aglosso\textunderscore )
\section{Aglossia}
\begin{itemize}
\item {Grp. gram.:f.}
\end{itemize}
Estado ou qualidade de \textunderscore aglosso\textunderscore .
\section{Aglosso}
\begin{itemize}
\item {Grp. gram.:adj.}
\end{itemize}
\begin{itemize}
\item {Grp. gram.:M.}
\end{itemize}
\begin{itemize}
\item {Proveniência:(Gr. \textunderscore aglossos\textunderscore )}
\end{itemize}
Que não tem língua.
Que fala barbaramente.
O mesmo que \textunderscore aglossa\textunderscore .
\section{Aglutição}
\begin{itemize}
\item {Grp. gram.:f.}
\end{itemize}
\begin{itemize}
\item {Proveniência:(De \textunderscore a\textunderscore  priv. + lat. \textunderscore glutitio\textunderscore )}
\end{itemize}
Impossibilidade de engulir.
\section{Aglutinação}
\begin{itemize}
\item {Grp. gram.:f.}
\end{itemize}
\begin{itemize}
\item {Proveniência:(Lat. \textunderscore agglutinatio\textunderscore )}
\end{itemize}
Acto de aglutinar.
\section{Aglutinante}
\begin{itemize}
\item {Grp. gram.:adj.}
\end{itemize}
\begin{itemize}
\item {Proveniência:(Lat. \textunderscore agglutinans\textunderscore )}
\end{itemize}
Que aglutina.
\section{Aglutinar}
\begin{itemize}
\item {Grp. gram.:v. t.}
\end{itemize}
\begin{itemize}
\item {Proveniência:(Lat. \textunderscore agglutinare\textunderscore )}
\end{itemize}
Collar.
Unir.
Reunir (palavras) a outra, principal, para formar um todo.
\section{Aglutinativo}
\begin{itemize}
\item {Grp. gram.:adj.}
\end{itemize}
O mesmo que \textunderscore aglutinante\textunderscore .
\section{Agmatologia}
\begin{itemize}
\item {Grp. gram.:f.}
\end{itemize}
\begin{itemize}
\item {Utilização:Med.}
\end{itemize}
\begin{itemize}
\item {Proveniência:(Do gr. \textunderscore agma\textunderscore  + \textunderscore logos\textunderscore )}
\end{itemize}
Tratado das fracturas.
\section{Agnação}
\begin{itemize}
\item {Grp. gram.:f.}
\end{itemize}
Qualidade de \textunderscore agnado\textunderscore .
\section{Agnacato}
\begin{itemize}
\item {Grp. gram.:m.}
\end{itemize}
Árvore americana.
\section{Agnado}
\begin{itemize}
\item {Grp. gram.:m.}
\end{itemize}
\begin{itemize}
\item {Proveniência:(Lat. \textunderscore agnatus\textunderscore )}
\end{itemize}
Parente por varonia.
\section{Agnantho}
\begin{itemize}
\item {Grp. gram.:m.}
\end{itemize}
\begin{itemize}
\item {Proveniência:(Do gr. \textunderscore agnos\textunderscore  + \textunderscore anthos\textunderscore )}
\end{itemize}
Arbusto, da fam. das verbenáceas, originário das Antilhas.
\section{Agnanto}
\begin{itemize}
\item {Grp. gram.:m.}
\end{itemize}
\begin{itemize}
\item {Proveniência:(Do gr. \textunderscore agnos\textunderscore  + \textunderscore anthos\textunderscore )}
\end{itemize}
Arbusto, da fam. das verbenáceas, originário das Antilhas.
\section{Agnathos}
\begin{itemize}
\item {Grp. gram.:m. pl.}
\end{itemize}
\begin{itemize}
\item {Proveniência:(Do gr. \textunderscore a\textunderscore priv. + \textunderscore gnathos\textunderscore )}
\end{itemize}
Nome, que os naturalistas deram a uma família de insectos, com quatro asas reticuladas, boca pequena e sem mandíbulas.
\section{Agnatício}
\begin{itemize}
\item {Grp. gram.:adj.}
\end{itemize}
\begin{itemize}
\item {Proveniência:(De \textunderscore agnato\textunderscore )}
\end{itemize}
Relativo aos agnados.
\section{Agnático}
\begin{itemize}
\item {Grp. gram.:adj.}
\end{itemize}
O mesmo que \textunderscore agnatício\textunderscore .
\section{Agnato}
\begin{itemize}
\item {Grp. gram.:m.}
\end{itemize}
(V.agnado)
\section{Agnatos}
\begin{itemize}
\item {Grp. gram.:m. pl.}
\end{itemize}
\begin{itemize}
\item {Proveniência:(Do gr. \textunderscore a\textunderscore priv. + \textunderscore gnathos\textunderscore )}
\end{itemize}
Nome, que os naturalistas deram a uma família de insectos, com quatro asas reticuladas, boca pequena e sem mandíbulas.
\section{Agnelina}
\begin{itemize}
\item {Grp. gram.:f.}
\end{itemize}
\begin{itemize}
\item {Proveniência:(Fr. \textunderscore agneline\textunderscore , do lat. \textunderscore agnellus\textunderscore )}
\end{itemize}
Pelle de cordeiro com lan.
\section{Agnellina}
\begin{itemize}
\item {Grp. gram.:f.}
\end{itemize}
\begin{itemize}
\item {Proveniência:(Fr. \textunderscore agneline\textunderscore , do lat. \textunderscore agnellus\textunderscore )}
\end{itemize}
Pelle de cordeiro com lan.
\section{Agnição}
\begin{itemize}
\item {Grp. gram.:f.}
\end{itemize}
\begin{itemize}
\item {Utilização:Ant.}
\end{itemize}
\begin{itemize}
\item {Proveniência:(Lat. \textunderscore agnitio\textunderscore )}
\end{itemize}
Conhecimento.
Reconhecimento.
Acção de conhecer.
\section{Agno}
\begin{itemize}
\item {Grp. gram.:m.}
\end{itemize}
(V.anho)
\section{Agnocasto}
\begin{itemize}
\item {Grp. gram.:m.}
\end{itemize}
\begin{itemize}
\item {Proveniência:(Do lat. \textunderscore agnus\textunderscore  + \textunderscore castus\textunderscore )}
\end{itemize}
Arbusto aromático, da fam. das verbenáceas.
\section{Agnóia}
\begin{itemize}
\item {Grp. gram.:f.}
\end{itemize}
\begin{itemize}
\item {Utilização:Med.}
\end{itemize}
\begin{itemize}
\item {Proveniência:(Do gr. \textunderscore a\textunderscore priv. + \textunderscore gnoo\textunderscore )}
\end{itemize}
Estado do doente, que não conhece nada do que o cérca.
\section{Agnome}
\begin{itemize}
\item {Grp. gram.:m.}
\end{itemize}
\begin{itemize}
\item {Proveniência:(Lat. \textunderscore agnomen\textunderscore )}
\end{itemize}
Epítheto, ou apellido, que, entre os Romanos, se accrescentava ao cognome.
\section{Agnominação}
\begin{itemize}
\item {Grp. gram.:f.}
\end{itemize}
\begin{itemize}
\item {Proveniência:(Lat. \textunderscore agnominatio\textunderscore )}
\end{itemize}
Repetição de uma palavra, variando-se-lhe o sentido com a simples mudança de uma letra ou letras.
\section{Agnosia}
\begin{itemize}
\item {Grp. gram.:f.}
\end{itemize}
O mesmo ou melhor que \textunderscore agnoia\textunderscore .
\section{Agnosticismo}
\begin{itemize}
\item {Grp. gram.:m.}
\end{itemize}
\begin{itemize}
\item {Proveniência:(De \textunderscore a\textunderscore priv. + \textunderscore gnosticismo\textunderscore )}
\end{itemize}
Systema philosóphico, que exclue da competência da razão humana o conhecimento do absoluto.
\section{Agnóstico}
\begin{itemize}
\item {Grp. gram.:adj.}
\end{itemize}
Partidário do agnosticismo.
\section{Ago}
\begin{itemize}
\item {Grp. gram.:m.}
\end{itemize}
Planta medicinal da ilha de San-Thomé.
\section{Agoge}
\begin{itemize}
\item {Grp. gram.:f.}
\end{itemize}
\begin{itemize}
\item {Proveniência:(Gr. \textunderscore agoge\textunderscore )}
\end{itemize}
Subdivisão, na música antiga.
\section{Agógico}
\begin{itemize}
\item {Grp. gram.:adj.}
\end{itemize}
\begin{itemize}
\item {Utilização:Gram.}
\end{itemize}
Diz-se do sentido que se infere das palavras.
\section{Agoiral}
\begin{itemize}
\item {Grp. gram.:adj.}
\end{itemize}
Relativo a \textunderscore agoiro\textunderscore .
\section{Agoirar}
\begin{itemize}
\item {Grp. gram.:v. t.}
\end{itemize}
Fazer agoiro de.
Antever; predizer.
\section{Agoirar}
\begin{itemize}
\item {Grp. gram.:v. i.}
\end{itemize}
\begin{itemize}
\item {Utilização:Archit.}
\end{itemize}
Collocar fragmentos de telha na juntura das telhas de cobrir, para se fazerem as braceiras.
(Por \textunderscore agueirar\textunderscore , de \textunderscore agueiro\textunderscore ?)
\section{Agoireiro}
\begin{itemize}
\item {Grp. gram.:adj.}
\end{itemize}
Que agoira.
\section{Agoirentar}
\begin{itemize}
\item {Grp. gram.:v. t.}
\end{itemize}
Tornar agoirento.
Fazer mau agoiro sôbre.
Ameaçar com desgraça.
\section{Agoirento}
\begin{itemize}
\item {Grp. gram.:adj.}
\end{itemize}
Que envolve mau agoiro.
\section{Agoirice}
\begin{itemize}
\item {Grp. gram.:f.}
\end{itemize}
Mania ou hábito de agoirar. Cf. Filinto, VI, 238.
\section{Agoiro}
\begin{itemize}
\item {Grp. gram.:m.}
\end{itemize}
\begin{itemize}
\item {Proveniência:(Lat. \textunderscore augurium\textunderscore )}
\end{itemize}
Predicção.
Preságio; sinal que presagia.
\section{Agolfinhado}
\begin{itemize}
\item {Grp. gram.:adj.}
\end{itemize}
Que tem feitio ou ares de golfinho. Cf. Camillo, \textunderscore N. de Insómn.\textunderscore , X, 43.
\section{Agolpear}
\textunderscore v. t.\textunderscore  (e der.)
(V. \textunderscore golpear\textunderscore , etc.)
\section{Agomado}
\begin{itemize}
\item {Grp. gram.:adj.}
\end{itemize}
\begin{itemize}
\item {Utilização:Bot.}
\end{itemize}
\begin{itemize}
\item {Proveniência:(De \textunderscore agomar\textunderscore )}
\end{itemize}
Que deita gomos.
\section{Agoman}
\begin{itemize}
\item {Grp. gram.:m.}
\end{itemize}
O princípio do mal, na Mythologia brasílica.
\section{Agomar}
\begin{itemize}
\item {Grp. gram.:v. i.}
\end{itemize}
\begin{itemize}
\item {Utilização:Bot.}
\end{itemize}
Deitar gomos.
Germinar.
\section{Agomia}
\begin{itemize}
\item {Grp. gram.:f.}
\end{itemize}
Arma curva, usada no Malabar.
Faca, de ponta recurvada, que usam alguns trabalhadores do campo.
\section{Agomia}
\begin{itemize}
\item {Grp. gram.:f.}
\end{itemize}
O mesmo que \textunderscore agomil\textunderscore .
\section{Agomiada}
\begin{itemize}
\item {Grp. gram.:f.}
\end{itemize}
Golpe de agomia^1.
\section{Agomil}
\begin{itemize}
\item {Grp. gram.:m.}
\end{itemize}
\begin{itemize}
\item {Utilização:Ant.}
\end{itemize}
O mesmo que \textunderscore gomil\textunderscore .
\section{Agomilado}
\begin{itemize}
\item {Grp. gram.:adj.}
\end{itemize}
Que tem fórma de gomil.
\section{Agomphíase}
\begin{itemize}
\item {Grp. gram.:f.}
\end{itemize}
O mesmo que \textunderscore agomphose\textunderscore .
\section{Agomphose}
\begin{itemize}
\item {Grp. gram.:f.}
\end{itemize}
Estado dos dentes que, abalados, se movem nos alvéolos.
\section{Agomphosíaco}
\begin{itemize}
\item {Grp. gram.:adj.}
\end{itemize}
Diz-se dos dentes, que tem agomphose. Cf. Pacheco, \textunderscore Promptuário\textunderscore , 20.
\section{Agonaes}
\begin{itemize}
\item {Grp. gram.:f. pl.}
\end{itemize}
\begin{itemize}
\item {Proveniência:(Lat. \textunderscore agonalia\textunderscore )}
\end{itemize}
Festas em honra de Jano.
\section{Agonais}
\begin{itemize}
\item {Grp. gram.:f. pl.}
\end{itemize}
\begin{itemize}
\item {Proveniência:(Lat. \textunderscore agonalia\textunderscore )}
\end{itemize}
Festas em honra de Jano.
\section{Agonal}
\begin{itemize}
\item {Grp. gram.:adj.}
\end{itemize}
Relativo ás festas agonaes. Cf. Castilho, \textunderscore Fastos\textunderscore , I, 35.
\section{Agone}
\begin{itemize}
\item {Grp. gram.:m.}
\end{itemize}
Sacrificador romano, que, antes de ferir a victima, preguntava ao povo: \textunderscore agone?\textunderscore  Cf. Castilho, \textunderscore Fastos\textunderscore , I, 35.
\section{Agonfíase}
\begin{itemize}
\item {Grp. gram.:f.}
\end{itemize}
O mesmo que \textunderscore agonfose\textunderscore .
\section{Agonfose}
\begin{itemize}
\item {Grp. gram.:f.}
\end{itemize}
Estado dos dentes que, abalados, se movem nos alvéolos.
\section{Agonfosíaco}
\begin{itemize}
\item {Grp. gram.:adj.}
\end{itemize}
Diz-se dos dentes, que tem agonfose. Cf. Pacheco, \textunderscore Promptuário\textunderscore , 20.
\section{Agongorado}
\begin{itemize}
\item {Grp. gram.:adj.}
\end{itemize}
Diz-se do estilo obscuro e rebuscado, como o de Gôngora.
\section{Agonia}
\begin{itemize}
\item {Grp. gram.:f.}
\end{itemize}
\begin{itemize}
\item {Utilização:Prov.}
\end{itemize}
\begin{itemize}
\item {Utilização:minh.}
\end{itemize}
\begin{itemize}
\item {Proveniência:(Gr. \textunderscore agonia\textunderscore )}
\end{itemize}
Extincção gradual das fôrças vitaes.
Último gráu de decadência.
Afflicção; náuseas.
Ralho, discussão, zanga.
\section{Agoniadamente}
\begin{itemize}
\item {Grp. gram.:adv.}
\end{itemize}
Com agonia.
\section{Agoniado}
\begin{itemize}
\item {Grp. gram.:adv.}
\end{itemize}
Que sente agonias, ânsias.
Amargurado, afflicto.
\section{Agoniador}
\begin{itemize}
\item {Grp. gram.:adj.}
\end{itemize}
\begin{itemize}
\item {Proveniência:(De \textunderscore agoniar\textunderscore )}
\end{itemize}
Que produz agonias. Cf. Arn. Gama, \textunderscore Motim\textunderscore , 272.
\section{Agoniar}
\begin{itemize}
\item {Grp. gram.:v. t.}
\end{itemize}
Causar agonia, afflicção, náuseas a.
Affligir; desgostar.
Inquietar.
\section{Agónico}
\begin{itemize}
\item {Grp. gram.:adj.}
\end{itemize}
\begin{itemize}
\item {Utilização:Neol.}
\end{itemize}
Relativo a agonia.
\section{Agonística}
\begin{itemize}
\item {Grp. gram.:f.}
\end{itemize}
\begin{itemize}
\item {Proveniência:(Gr. \textunderscore agonistike\textunderscore )}
\end{itemize}
Parte da antiga gymnástica, relativa aos combates dos athletas.
\section{Agonístico}
\begin{itemize}
\item {Grp. gram.:adj.}
\end{itemize}
\begin{itemize}
\item {Utilização:Ant.}
\end{itemize}
\begin{itemize}
\item {Proveniência:(Do gr. \textunderscore agonizein\textunderscore )}
\end{itemize}
Relativo a combates.
\section{Agonizadamente}
\begin{itemize}
\item {Grp. gram.:adv.}
\end{itemize}
O mesmo que \textunderscore agoniadamente\textunderscore .
\section{Agonizante}
\begin{itemize}
\item {Grp. gram.:adj.}
\end{itemize}
Que está agonizando, moribundo.
Que causa agonia.
\section{Agonizar}
\begin{itemize}
\item {Grp. gram.:v. t.}
\end{itemize}
\begin{itemize}
\item {Grp. gram.:V. i.}
\end{itemize}
\begin{itemize}
\item {Proveniência:(Gr. \textunderscore agonizein\textunderscore )}
\end{itemize}
Causar agonia a.
Estar moribundo.
Ir acabando.
\section{Ágono}
\begin{itemize}
\item {Grp. gram.:adj.}
\end{itemize}
\begin{itemize}
\item {Proveniência:(Do gr. \textunderscore a\textunderscore priv. + \textunderscore gonia\textunderscore )}
\end{itemize}
Que não tem ângulo.
\section{Agonóstomo}
\begin{itemize}
\item {Grp. gram.:m.}
\end{itemize}
\begin{itemize}
\item {Proveniência:(Do gr. \textunderscore a\textunderscore priv. + \textunderscore gonia\textunderscore  + \textunderscore stoma\textunderscore )}
\end{itemize}
Gênero de insectos acanthopterýgios.
\section{Agonoteto}
\begin{itemize}
\item {Grp. gram.:m.}
\end{itemize}
\begin{itemize}
\item {Proveniência:(Gr. \textunderscore agonothete\textunderscore )}
\end{itemize}
Presidente dos jogos sagrados, na Grécia.
\section{Agonotheto}
\begin{itemize}
\item {Grp. gram.:m.}
\end{itemize}
\begin{itemize}
\item {Proveniência:(Gr. \textunderscore agonothete\textunderscore )}
\end{itemize}
Presidente dos jogos sagrados, na Grécia.
\section{Agora}
\begin{itemize}
\item {Grp. gram.:adv.}
\end{itemize}
\begin{itemize}
\item {Grp. gram.:Conj.}
\end{itemize}
\begin{itemize}
\item {Proveniência:(Do lat. \textunderscore hac\textunderscore  + \textunderscore hora\textunderscore )}
\end{itemize}
Nesta hora.
Presentemente.
Todavia, mas: \textunderscore poderás mentir-me; agora enganar-me, isso nunca\textunderscore !
\section{Ágora}
\begin{itemize}
\item {Grp. gram.:f.}
\end{itemize}
\begin{itemize}
\item {Proveniência:(Gr. \textunderscore agora\textunderscore )}
\end{itemize}
Praça pública, mercado, (entre os Gregos).
\section{Àgóra!}
\begin{itemize}
\item {Grp. gram.:interj.  e  interrog.}
\end{itemize}
\begin{itemize}
\item {Utilização:Prov.}
\end{itemize}
Quem sabe? deveras? Isso sim!:«\textunderscore àgóra pequei!\textunderscore »Camillo, \textunderscore Bruxa\textunderscore , 2.^a p., c. V.
(Por \textunderscore há\textunderscore  + \textunderscore agora\textunderscore )
\section{Agorafobia}
\begin{itemize}
\item {Grp. gram.:f.}
\end{itemize}
\begin{itemize}
\item {Proveniência:(Do gr. \textunderscore agora\textunderscore  + \textunderscore phobos\textunderscore )}
\end{itemize}
Estado mórbido, caracterizado pelo medo de atravessar largos ou praças.
\section{Agoráfobo}
\begin{itemize}
\item {Grp. gram.:m.}
\end{itemize}
Aquelle que soffre \textunderscore agorafobia\textunderscore .
\section{Agoranomia}
\begin{itemize}
\item {Grp. gram.:f.}
\end{itemize}
Cargo ou funcções de \textunderscore agorânomo\textunderscore .
\section{Agorânomo}
\begin{itemize}
\item {Grp. gram.:m.}
\end{itemize}
\begin{itemize}
\item {Proveniência:(Gr. \textunderscore agoranomos\textunderscore )}
\end{itemize}
Magistrado atheniense, que tinha a seu cargo a polícia dos mercados.
\section{Agòrantes}
\begin{itemize}
\item {Grp. gram.:adv.}
\end{itemize}
\begin{itemize}
\item {Utilização:Prov.}
\end{itemize}
Pouco antes. Cf. Rui Barbosa, \textunderscore Répl.\textunderscore , II, 157.
\section{Agoraphobia}
\begin{itemize}
\item {Grp. gram.:f.}
\end{itemize}
\begin{itemize}
\item {Proveniência:(Do gr. \textunderscore agora\textunderscore  + \textunderscore phobos\textunderscore )}
\end{itemize}
Estado mórbido, caracterizado pelo medo de atravessar largos ou praças.
\section{Agoráphobo}
\begin{itemize}
\item {Grp. gram.:m.}
\end{itemize}
Aquelle que soffre \textunderscore agoraphobia\textunderscore .
\section{Agorarca}
\begin{itemize}
\item {Grp. gram.:m.}
\end{itemize}
\begin{itemize}
\item {Proveniência:(Do gr. \textunderscore agora\textunderscore  + \textunderscore arkhe\textunderscore )}
\end{itemize}
Magistrado, que, em Esparta, desempenhava as mesmas funcções que o agorânomo em Athenas.
\section{Agorarcha}
\begin{itemize}
\item {fónica:ca}
\end{itemize}
\begin{itemize}
\item {Grp. gram.:m.}
\end{itemize}
\begin{itemize}
\item {Proveniência:(Do gr. \textunderscore agora\textunderscore  + \textunderscore arkhe\textunderscore )}
\end{itemize}
Magistrado, que, em Esparta, desempenhava as mesmas funcções que o agorânomo em Athenas.
\section{Agorentar}
\begin{itemize}
\item {Grp. gram.:v. t.}
\end{itemize}
Encurtar.
Deminuir.
Aparar em roda.
Aguarentar.
\section{Agorinha}
\begin{itemize}
\item {Grp. gram.:adv.}
\end{itemize}
\begin{itemize}
\item {Utilização:Bras}
\end{itemize}
Agora mesmo; neste instante.
(Dem. de \textunderscore agora\textunderscore )
\section{Agostadoiro}
\begin{itemize}
\item {Grp. gram.:m.}
\end{itemize}
\begin{itemize}
\item {Utilização:Prov.}
\end{itemize}
\begin{itemize}
\item {Utilização:alent.}
\end{itemize}
Acto de pastar o gado em Agosto.
\section{Agostadouro}
\begin{itemize}
\item {Grp. gram.:m.}
\end{itemize}
\begin{itemize}
\item {Utilização:Prov.}
\end{itemize}
\begin{itemize}
\item {Utilização:alent.}
\end{itemize}
Acto de pastar o gado em Agosto.
\section{Agostar-se}
\begin{itemize}
\item {Grp. gram.:v. p.}
\end{itemize}
\begin{itemize}
\item {Utilização:Prov.}
\end{itemize}
\begin{itemize}
\item {Utilização:trasm.}
\end{itemize}
\begin{itemize}
\item {Proveniência:(De \textunderscore Agosto\textunderscore )}
\end{itemize}
Murchar, por falta de frescura; estiolar-se.
\section{Agostenga}
\begin{itemize}
\item {Grp. gram.:f.}
\end{itemize}
Videira do Brasil.
\section{Agostinha}
\begin{itemize}
\item {Grp. gram.:f.}
\end{itemize}
Variedade de pera.
Variedade de maçan.
Variedade de cerejas de Agosto.
Variedade de batata.
\section{Agostinhas}
\begin{itemize}
\item {Grp. gram.:f. pl.}
\end{itemize}
Congregação de religiosas enfermeiras, fundada em França no século VII.
\section{Agostinho}
\begin{itemize}
\item {Grp. gram.:m.}
\end{itemize}
\begin{itemize}
\item {Grp. gram.:Adj.}
\end{itemize}
Frade, da Ordem de Santo Agostinho.
Relativo àquella ordem.
\section{Agosto}
\begin{itemize}
\item {fónica:gôs}
\end{itemize}
\begin{itemize}
\item {Grp. gram.:m.}
\end{itemize}
\begin{itemize}
\item {Proveniência:(Lat. \textunderscore augustus\textunderscore )}
\end{itemize}
Oitavo mês do anno romano.
\section{Agoural}
\begin{itemize}
\item {Grp. gram.:adj.}
\end{itemize}
Relativo a \textunderscore agouro\textunderscore .
\section{Agourar}
\begin{itemize}
\item {Grp. gram.:v. t.}
\end{itemize}
Fazer agouro de.
Antever; predizer.
\section{Agourar}
\begin{itemize}
\item {Grp. gram.:v. i.}
\end{itemize}
\begin{itemize}
\item {Utilização:Archit.}
\end{itemize}
Collocar fragmentos de telha na juntura das telhas de cobrir, para se fazerem as braceiras.
(Por \textunderscore agueirar\textunderscore , de \textunderscore agueiro\textunderscore ?)
\section{Agoureiro}
\begin{itemize}
\item {Grp. gram.:adj.}
\end{itemize}
Que agoura.
\section{Agourentar}
\begin{itemize}
\item {Grp. gram.:v. t.}
\end{itemize}
Tornar agourento.
Fazer mau agouro sôbre.
Ameaçar com desgraça.
\section{Agourento}
\begin{itemize}
\item {Grp. gram.:adj.}
\end{itemize}
Que envolve mau agouro.
\section{Agourice}
\begin{itemize}
\item {Grp. gram.:f.}
\end{itemize}
Mania ou hábito de agourar. Cf. Filinto, VI, 238.
\section{Agouro}
\begin{itemize}
\item {Grp. gram.:m.}
\end{itemize}
\begin{itemize}
\item {Proveniência:(Lat. \textunderscore augurium\textunderscore )}
\end{itemize}
Predicção.
Preságio; sinal que presagia.
\section{Agra}
\begin{itemize}
\item {Utilização:Prov.}
\end{itemize}
\begin{itemize}
\item {Utilização:ant.}
\end{itemize}
Campo.
Brejo, pântano.
(V. \textunderscore agro\textunderscore ^1)
\section{Agra}
\begin{itemize}
\item {Grp. gram.:m.}
\end{itemize}
Gênero de insectos coleópteros pentâmeros, da fam. dos carábicos.
\section{Agracarambo}
\begin{itemize}
\item {Grp. gram.:m.}
\end{itemize}
Madeira odorífera da China.
\section{Agraciação}
\begin{itemize}
\item {Grp. gram.:f.}
\end{itemize}
Acto de \textunderscore agraciar\textunderscore .
\section{Agraciadamente}
\begin{itemize}
\item {Grp. gram.:adv.}
\end{itemize}
De bom grado. Cf. Pacheco, \textunderscore Promptuário\textunderscore , 79.
\section{Agraciado}
\begin{itemize}
\item {Grp. gram.:adj.}
\end{itemize}
\begin{itemize}
\item {Proveniência:(De \textunderscore agraciar\textunderscore )}
\end{itemize}
Que recebeu graça ou título honorífico: \textunderscore indivíduo, agraciado com uma commenda\textunderscore .
Que tem graça:«\textunderscore linguagem agraciada de meninices\textunderscore ». Camillo, \textunderscore Caveira\textunderscore , 203.
\section{Agraciador}
\begin{itemize}
\item {Grp. gram.:m.}
\end{itemize}
Aquelle que agracia.
\section{Agraciar}
\begin{itemize}
\item {Grp. gram.:v. t.}
\end{itemize}
\begin{itemize}
\item {Proveniência:(Do lat. \textunderscore gratia\textunderscore )}
\end{itemize}
Conceder graça ou mercê, a.
\section{Agraço}
\begin{itemize}
\item {Grp. gram.:m.}
\end{itemize}
\begin{itemize}
\item {Proveniência:(De \textunderscore agro\textunderscore )}
\end{itemize}
Estado das uvas, antes de amadurecerem.
As uvas verdes.
Verdura.
\section{Agradábil}
\begin{itemize}
\item {Grp. gram.:adj.}
\end{itemize}
\begin{itemize}
\item {Utilização:Ant.}
\end{itemize}
(V.agradável)
\section{Agradabilissimo}
\begin{itemize}
\item {Grp. gram.:adj.}
\end{itemize}
(sup. de \textunderscore agradável\textunderscore )
\section{Agradar}
\begin{itemize}
\item {Grp. gram.:v. i.}
\end{itemize}
\begin{itemize}
\item {Grp. gram.:V. t.}
\end{itemize}
\begin{itemize}
\item {Utilização:Ant.}
\end{itemize}
\begin{itemize}
\item {Grp. gram.:V. p.}
\end{itemize}
\begin{itemize}
\item {Proveniência:(De \textunderscore grado\textunderscore )}
\end{itemize}
Sêr bem-quisto.
Parecer bem; aprazer.
Sêr agradável a:«\textunderscore quanto mais o ama, mais procura agradállo em todas as coisas\textunderscore ». \textunderscore Luz e Calor\textunderscore , 86 e 93.
Que sente prazer.
Que se compraz.
\section{Agradar}
\begin{itemize}
\item {Grp. gram.:v. t.}
\end{itemize}
\begin{itemize}
\item {Utilização:Prov.}
\end{itemize}
O mesmo que \textunderscore gradar\textunderscore ^1.
\section{Agradável}
\begin{itemize}
\item {Grp. gram.:adj.}
\end{itemize}
Que agrada.
Prazenteiro.
Suave, ameno.
\section{Agradavelmente}
\begin{itemize}
\item {Grp. gram.:adv.}
\end{itemize}
De modo agradável.
\section{Agrade}
\begin{itemize}
\item {Grp. gram.:f.}
\end{itemize}
O mesmo que \textunderscore grade\textunderscore .
\section{Agradecer}
\begin{itemize}
\item {Grp. gram.:v. t.}
\end{itemize}
\begin{itemize}
\item {Grp. gram.:V. i.}
\end{itemize}
\begin{itemize}
\item {Proveniência:(De \textunderscore grado\textunderscore )}
\end{itemize}
Mostrar gratidão por.
Dar agradecimentos.
Dizer-se grato.
\section{Agradecidamente}
\begin{itemize}
\item {Grp. gram.:adv.}
\end{itemize}
De modo \textunderscore agradecido\textunderscore .
\section{Agradecido}
\begin{itemize}
\item {Grp. gram.:adj.}
\end{itemize}
Grato; reconhecido.
Que se agradece: \textunderscore favor agradecido\textunderscore .
\section{Agradecimento}
\begin{itemize}
\item {Grp. gram.:m.}
\end{itemize}
Acto de \textunderscore agradecer\textunderscore .
Gratidão; reconhecimento.
\section{Agradecível}
\begin{itemize}
\item {Grp. gram.:adj.}
\end{itemize}
Que merece sêr agradecido.
\section{Agrado}
\begin{itemize}
\item {Grp. gram.:m.}
\end{itemize}
\begin{itemize}
\item {Proveniência:(De \textunderscore agradar\textunderscore )}
\end{itemize}
Satisfação; aprazimento.
Affabilidade.
\section{Agrafia}
\begin{itemize}
\item {Grp. gram.:f.}
\end{itemize}
\begin{itemize}
\item {Utilização:Med.}
\end{itemize}
\begin{itemize}
\item {Proveniência:(Do gr. \textunderscore a\textunderscore priv. + \textunderscore graphein\textunderscore )}
\end{itemize}
Abolição do movimento necessário para a escrita.
\section{Agráfico}
\begin{itemize}
\item {Grp. gram.:adj.}
\end{itemize}
\begin{itemize}
\item {Utilização:Med.}
\end{itemize}
\begin{itemize}
\item {Proveniência:(Do gr. \textunderscore a\textunderscore priv. + \textunderscore graphein\textunderscore )}
\end{itemize}
Que tem impossibilidade phýsica de escrever.
\section{Agramente}
\begin{itemize}
\item {fónica:á-gra}
\end{itemize}
\begin{itemize}
\item {Grp. gram.:m.}
\end{itemize}
O mesmo que \textunderscore acremente\textunderscore .
\section{Agranar}
\begin{itemize}
\item {Grp. gram.:v. i.}
\end{itemize}
\begin{itemize}
\item {Utilização:Prov.}
\end{itemize}
\begin{itemize}
\item {Proveniência:(Do lat. \textunderscore granum\textunderscore )}
\end{itemize}
Criar grão, (falando-se de cereaes).
\section{Agranizar}
\begin{itemize}
\item {Grp. gram.:v. t.}
\end{itemize}
Cobrir de granizo. Cf. Filinto, XIII, 218.
\section{Agrão}
\begin{itemize}
\item {Grp. gram.:m.}
\end{itemize}
\begin{itemize}
\item {Utilização:Ant.}
\end{itemize}
Grande agro^1? Cf. G. Vicente, II, 29.
\section{Agraphia}
\begin{itemize}
\item {Grp. gram.:f.}
\end{itemize}
\begin{itemize}
\item {Utilização:Med.}
\end{itemize}
\begin{itemize}
\item {Proveniência:(Do gr. \textunderscore a\textunderscore priv. + \textunderscore graphein\textunderscore )}
\end{itemize}
Abolição do movimento necessário para a escrita.
\section{Agráphico}
\begin{itemize}
\item {Grp. gram.:adj.}
\end{itemize}
\begin{itemize}
\item {Utilização:Med.}
\end{itemize}
\begin{itemize}
\item {Proveniência:(Do gr. \textunderscore a\textunderscore priv. + \textunderscore graphein\textunderscore )}
\end{itemize}
Que tem impossibilidade phýsica de escrever.
\section{Agrapim}
\begin{itemize}
\item {Grp. gram.:m.}
\end{itemize}
\begin{itemize}
\item {Utilização:Ant.}
\end{itemize}
\begin{itemize}
\item {Proveniência:(Fr. \textunderscore grappin\textunderscore )}
\end{itemize}
Alamar.
Espécie de colchete.
\section{Agrar}
\begin{itemize}
\item {Grp. gram.:v. t.}
\end{itemize}
Converter em agro; tornar plano (um terreno) para semear ou cultivar.
\section{Agrário}
\begin{itemize}
\item {Grp. gram.:adj.}
\end{itemize}
\begin{itemize}
\item {Proveniência:(Lat. \textunderscore agrarius\textunderscore )}
\end{itemize}
Relativo a campos.
\section{Agraudar}
\begin{itemize}
\item {fónica:gra-u}
\end{itemize}
\begin{itemize}
\item {Grp. gram.:v. i.}
\end{itemize}
Crescer, tornar-se graúdo.
\section{Agraula}
\begin{itemize}
\item {Grp. gram.:f.}
\end{itemize}
Gênero de plantas gramíneas.
\section{Agráulias}
\begin{itemize}
\item {Grp. gram.:f. pl.}
\end{itemize}
Antigas festas athenienses, em honra de Minerva.
\section{Agraulis}
\begin{itemize}
\item {Grp. gram.:m.}
\end{itemize}
Gênero de insectos lepidópteros.
\section{Agravação}
\begin{itemize}
\item {Grp. gram.:f.}
\end{itemize}
O mesmo que \textunderscore agravamento\textunderscore .
\section{Agravadamente}
\begin{itemize}
\item {Grp. gram.:adv.}
\end{itemize}
Com agravamento.
\section{Agravador}
\begin{itemize}
\item {Grp. gram.:m.  e  adj.}
\end{itemize}
O que agrava.
\section{Agravamento}
\begin{itemize}
\item {Grp. gram.:m.}
\end{itemize}
Acto de \textunderscore agravar\textunderscore .
\section{Agravante}
\begin{itemize}
\item {Grp. gram.:adj.}
\end{itemize}
\begin{itemize}
\item {Proveniência:(Lat. \textunderscore aggravans\textunderscore )}
\end{itemize}
Que agrava.
\section{Agravar}
\begin{itemize}
\item {Grp. gram.:v. t.}
\end{itemize}
\begin{itemize}
\item {Proveniência:(Lat. \textunderscore aggravare\textunderscore )}
\end{itemize}
Tornar grave.
Augmentar.
Fazer agravo a.
\section{Agravista}
\begin{itemize}
\item {Grp. gram.:m.}
\end{itemize}
\begin{itemize}
\item {Utilização:Ant.}
\end{itemize}
Juiz, que, nos tribunaes superiores, tomava conhecimento dos agravos e os julgava.
\section{Agravo}
\begin{itemize}
\item {Grp. gram.:m.}
\end{itemize}
\begin{itemize}
\item {Proveniência:(De \textunderscore aggravar\textunderscore )}
\end{itemize}
Offensa; injúria.
Recurso judicial, contra uma presumida injustiça.
\section{Agravoso}
\begin{itemize}
\item {Grp. gram.:adj.}
\end{itemize}
Que causa agravo.
\section{Agraz}
\begin{itemize}
\item {Grp. gram.:m.}
\end{itemize}
\begin{itemize}
\item {Utilização:Prov.}
\end{itemize}
\begin{itemize}
\item {Utilização:trasm.}
\end{itemize}
\begin{itemize}
\item {Grp. gram.:Pl.}
\end{itemize}
\begin{itemize}
\item {Utilização:Ext.}
\end{itemize}
\textunderscore Têr agraz no ôlho\textunderscore , sêr muito fino ou perspicaz.
Uvas verdes.
Fruta verde.
(Cast. \textunderscore agraz\textunderscore )
\section{Agre}
\begin{itemize}
\item {Grp. gram.:m.}
\end{itemize}
O mesmo que \textunderscore acre\textunderscore ^1.
\section{Agrear}
\begin{itemize}
\item {Grp. gram.:v. t.}
\end{itemize}
Tornar agro ou agre; azedar. Cf. Pato, \textunderscore Ciprestes\textunderscore , 52.
\section{Agredir}
\begin{itemize}
\item {Grp. gram.:v. t.}
\end{itemize}
\begin{itemize}
\item {Proveniência:(Lat. \textunderscore aggredi\textunderscore )}
\end{itemize}
Atacar.
Assaltar.
Insultar.
Ir contra.
\section{Agregação}
\begin{itemize}
\item {Grp. gram.:f.}
\end{itemize}
Acto de \textunderscore agregar\textunderscore .
\section{Agregado}
\begin{itemize}
\item {Grp. gram.:adj.}
\end{itemize}
\begin{itemize}
\item {Grp. gram.:M.}
\end{itemize}
\begin{itemize}
\item {Proveniência:(De \textunderscore aggregar\textunderscore )}
\end{itemize}
Annexo.
Adjunto.
Reunido.
Funccionário, que um diploma legal aggrega a uma classe ou corporação, além do número normal dos funccionários dessa corporação ou classe: \textunderscore na Relação de Lisbôa há quatro agregados\textunderscore .
\section{Agregar}
\begin{itemize}
\item {Grp. gram.:v. t.}
\end{itemize}
\begin{itemize}
\item {Proveniência:(Lat. \textunderscore aggregare\textunderscore )}
\end{itemize}
Ajuntar, anexar.
Associar.
\section{Agregativo}
\begin{itemize}
\item {Grp. gram.:adj.}
\end{itemize}
Que agrega.
\section{Agregato}
\begin{itemize}
\item {Grp. gram.:adj.}
\end{itemize}
(V.agregado)
\section{Agreira}
\begin{itemize}
\item {Grp. gram.:f.}
\end{itemize}
\begin{itemize}
\item {Proveniência:(De \textunderscore agre\textunderscore ?)}
\end{itemize}
O mesmo que \textunderscore lódão\textunderscore .
\section{Agreiro}
\begin{itemize}
\item {Grp. gram.:m.}
\end{itemize}
(Fórma pop. de \textunderscore argueiro\textunderscore )
\section{Agrela}
\begin{itemize}
\item {Grp. gram.:f.}
\end{itemize}
\begin{itemize}
\item {Utilização:Ant.}
\end{itemize}
\begin{itemize}
\item {Proveniência:(De \textunderscore agra\textunderscore )}
\end{itemize}
Pequeno campo.
\section{Agrelo}
\begin{itemize}
\item {fónica:grê}
\end{itemize}
\begin{itemize}
\item {Grp. gram.:m.}
\end{itemize}
\begin{itemize}
\item {Utilização:Prov.}
\end{itemize}
\begin{itemize}
\item {Utilização:minh.}
\end{itemize}
\begin{itemize}
\item {Utilização:Ant.}
\end{itemize}
Pequeno agro, agrela.
\section{Agrém}
\begin{itemize}
\item {Grp. gram.:m.}
\end{itemize}
Espécie de tribuna ou púlpito, em pagodes chineses. Cf. \textunderscore Peregrinação\textunderscore , CXXVII.
\section{Agremente}
\begin{itemize}
\item {Grp. gram.:adv.}
\end{itemize}
De modo agre; com azedume. Cf. Sousa, \textunderscore Vida do Arc.\textunderscore , II, 15.
\section{Agremiação}
\begin{itemize}
\item {Grp. gram.:f.}
\end{itemize}
Ajuntamento, reunião.
Associação.
Acto de \textunderscore agremiar\textunderscore .
\section{Agremiadamente}
\begin{itemize}
\item {Grp. gram.:adv.}
\end{itemize}
Em grêmio, em associação.
Em commum.
\section{Agremiado}
\begin{itemize}
\item {Grp. gram.:adj.}
\end{itemize}
\begin{itemize}
\item {Grp. gram.:M.}
\end{itemize}
\begin{itemize}
\item {Proveniência:(De \textunderscore agremiar\textunderscore )}
\end{itemize}
Que faz parte de um grêmio.
Membro de uma agremiação.
\section{Agremiar}
\begin{itemize}
\item {Grp. gram.:v. t.}
\end{itemize}
Reunir em grêmio, em assembleia, em associação.
\section{Agressão}
\begin{itemize}
\item {Grp. gram.:f.}
\end{itemize}
\begin{itemize}
\item {Proveniência:(Lat. \textunderscore aggressio\textunderscore )}
\end{itemize}
Ataque.
Acto de agredir.
\section{Agressivamente}
\begin{itemize}
\item {Grp. gram.:adv.}
\end{itemize}
De modo \textunderscore agressivo\textunderscore .
\section{Agressivo}
\begin{itemize}
\item {Grp. gram.:adj.}
\end{itemize}
\begin{itemize}
\item {Proveniência:(Do lat. \textunderscore aggressus\textunderscore )}
\end{itemize}
Que envolve agressão.
\section{Agressor}
\begin{itemize}
\item {Grp. gram.:m.  e  adj.}
\end{itemize}
\begin{itemize}
\item {Proveniência:(Lat. \textunderscore aggressor\textunderscore )}
\end{itemize}
O que agride.
Provocador.
\section{Agressório}
\begin{itemize}
\item {Grp. gram.:adj.}
\end{itemize}
O mesmo que \textunderscore agressivo\textunderscore .
\section{Agresta}
\begin{itemize}
\item {Grp. gram.:f.}
\end{itemize}
\begin{itemize}
\item {Proveniência:(De \textunderscore agre\textunderscore )}
\end{itemize}
Sumo de agraço.
\section{Agreste}
\begin{itemize}
\item {Grp. gram.:adj.}
\end{itemize}
\begin{itemize}
\item {Grp. gram.:M.}
\end{itemize}
\begin{itemize}
\item {Utilização:Bras. do N}
\end{itemize}
\begin{itemize}
\item {Proveniência:(Lat. \textunderscore agrestis\textunderscore )}
\end{itemize}
Relativo a agro, (campo).
Rústico; silvestre.
Áspero.
Indelicado.
O mesmo que \textunderscore litoral\textunderscore , por opposição a \textunderscore sertão\textunderscore .
\section{Agrestia}
\begin{itemize}
\item {Grp. gram.:f.}
\end{itemize}
\begin{itemize}
\item {Utilização:Bras}
\end{itemize}
Qualidade de \textunderscore agreste\textunderscore ; rudeza; desabrimento.
\section{Ágria}
\begin{itemize}
\item {Grp. gram.:f.}
\end{itemize}
\begin{itemize}
\item {Proveniência:(Gr. \textunderscore agrios\textunderscore )}
\end{itemize}
Herpes; impigem corrosiva.
Pústula maligna.
\section{Agrião}
\begin{itemize}
\item {Grp. gram.:m.}
\end{itemize}
\begin{itemize}
\item {Proveniência:(Gr. \textunderscore agrion\textunderscore )}
\end{itemize}
Planta herbácea, da fam. das crucíferas.
\section{Agrião}
\begin{itemize}
\item {Grp. gram.:m.}
\end{itemize}
Tumor duro e sem dôr, no curvilhão das bêstas.
\section{Agriçar}
\begin{itemize}
\item {Grp. gram.:v. i.}
\end{itemize}
\begin{itemize}
\item {Utilização:Náut.}
\end{itemize}
\begin{itemize}
\item {Utilização:ant.}
\end{itemize}
Pôr a cordagem.
\section{Agrico-industrial}
\begin{itemize}
\item {Grp. gram.:adj.}
\end{itemize}
\begin{itemize}
\item {Proveniência:(De \textunderscore agri*cola\textunderscore  + \textunderscore industrial\textunderscore )}
\end{itemize}
Relativo á agricultura e á industria simultaneamente.
\section{Agrícola}
\begin{itemize}
\item {Grp. gram.:m.}
\end{itemize}
\begin{itemize}
\item {Grp. gram.:Adj.}
\end{itemize}
\begin{itemize}
\item {Proveniência:(Lat. \textunderscore agricola\textunderscore )}
\end{itemize}
O mesmo que \textunderscore agricultor\textunderscore .
Relativo á agricultura: \textunderscore trabalhos agrícolas\textunderscore .
\section{Agricolar}
\begin{itemize}
\item {Grp. gram.:adj.}
\end{itemize}
(V. \textunderscore agrícola\textunderscore , adj.)
\section{Agricultado}
\begin{itemize}
\item {Grp. gram.:adj.}
\end{itemize}
\begin{itemize}
\item {Proveniência:(De \textunderscore agricultar\textunderscore )}
\end{itemize}
Cultivado, (falando-se de terrenos).
\section{Agricultar}
\begin{itemize}
\item {Grp. gram.:v. t.}
\end{itemize}
\begin{itemize}
\item {Grp. gram.:V. i.}
\end{itemize}
\begin{itemize}
\item {Proveniência:(Do lat. \textunderscore ager\textunderscore  + \textunderscore cultus\textunderscore )}
\end{itemize}
Cultivar (o campo).
Dedicar-se a trabalhos de agricultura.
\section{Agricultável}
\begin{itemize}
\item {Grp. gram.:adj.}
\end{itemize}
Que póde sêr agricultado.
\section{Agricultor}
\begin{itemize}
\item {Grp. gram.:m.}
\end{itemize}
\begin{itemize}
\item {Grp. gram.:Adj.}
\end{itemize}
\begin{itemize}
\item {Proveniência:(Lat. \textunderscore agricultor\textunderscore )}
\end{itemize}
O que agriculta.
Que agriculta.
\section{Agricultura}
\begin{itemize}
\item {Grp. gram.:f.}
\end{itemize}
\begin{itemize}
\item {Proveniência:(Lat. \textunderscore agricultura\textunderscore )}
\end{itemize}
Arte de cultivar os campos.
Cultivo da terra.
\section{Agridoce}
\begin{itemize}
\item {Grp. gram.:adj.}
\end{itemize}
Agro e doce.
\section{Agridoçura}
\begin{itemize}
\item {Grp. gram.:f.}
\end{itemize}
Qualidade de \textunderscore agridoce\textunderscore . Cf. Lapa, \textunderscore Processos de Vin.\textunderscore , 9.
\section{Agridulce}
\begin{itemize}
\item {Grp. gram.:adj.}
\end{itemize}
O mesmo que \textunderscore agridoce\textunderscore . Cf. Castilho, \textunderscore Fausto\textunderscore , 11.
\section{Agrigentino}
\begin{itemize}
\item {Grp. gram.:m.  e  adj.}
\end{itemize}
O que é natural de Agrigento. Cf. Latino, \textunderscore Or. da Corôa\textunderscore , CXCII.
\section{Agrilhoamento}
\begin{itemize}
\item {Grp. gram.:m.}
\end{itemize}
Acto de \textunderscore agrilhoar\textunderscore .
\section{Agrilhoar}
\begin{itemize}
\item {Grp. gram.:v. t.}
\end{itemize}
Prender com grilhões.
Prender.
Comprimir.
\section{Agrimar-se}
\begin{itemize}
\item {Grp. gram.:v. p.}
\end{itemize}
\begin{itemize}
\item {Utilização:Prov.}
\end{itemize}
\begin{itemize}
\item {Utilização:beir.}
\end{itemize}
Acolher-se á protecção de alguém.
(Cp. \textunderscore agalimar\textunderscore )
\section{Agrimensão}
\begin{itemize}
\item {Grp. gram.:f.}
\end{itemize}
(V.agrimensura)
\section{Agrimensor}
\begin{itemize}
\item {Grp. gram.:m.}
\end{itemize}
\begin{itemize}
\item {Proveniência:(Lat. \textunderscore agrimensor\textunderscore )}
\end{itemize}
O que mede as terras, os campos.
\section{Agrimensório}
\begin{itemize}
\item {Grp. gram.:adj.}
\end{itemize}
Relativo á \textunderscore agrimensura\textunderscore .
\section{Agrimensura}
\begin{itemize}
\item {Grp. gram.:f.}
\end{itemize}
\begin{itemize}
\item {Proveniência:(Lat. \textunderscore agrimensura\textunderscore )}
\end{itemize}
Arte de medir campos.
Medida das terras.
\section{Agrimónia}
\begin{itemize}
\item {Grp. gram.:f.}
\end{itemize}
\begin{itemize}
\item {Proveniência:(Lat. \textunderscore agrimonia\textunderscore )}
\end{itemize}
Planta herbácea, de caule hirsuto e fôlhas negras.
\section{Agrimónia}
\begin{itemize}
\item {Grp. gram.:f.}
\end{itemize}
(V.acrimónia)
\section{Agrinaldar}
\textunderscore v. t.\textunderscore  (e der.)
O mesmo que \textunderscore engrinaldar\textunderscore , etc. Cf. Camillo, \textunderscore Regicida\textunderscore , 31.
\section{Agriodafno}
\begin{itemize}
\item {Grp. gram.:m.}
\end{itemize}
Planta laurácea.
\section{Agriodaphno}
\begin{itemize}
\item {Grp. gram.:m.}
\end{itemize}
Planta laurácea.
\section{Agriodendro}
\begin{itemize}
\item {Grp. gram.:m.}
\end{itemize}
\begin{itemize}
\item {Utilização:Bot.}
\end{itemize}
Gênero de liliáceas.
\section{Agriógrafo}
\begin{itemize}
\item {Grp. gram.:m.}
\end{itemize}
\begin{itemize}
\item {Proveniência:(Do gr. \textunderscore agrios\textunderscore  + \textunderscore phagein\textunderscore )}
\end{itemize}
Aquelle que se alimenta de animaes selvagens.
\section{Agriógrapho}
\begin{itemize}
\item {Grp. gram.:m.}
\end{itemize}
\begin{itemize}
\item {Proveniência:(Do gr. \textunderscore agrios\textunderscore  + \textunderscore phagein\textunderscore )}
\end{itemize}
Aquelle que se alimenta de animaes selvagens.
\section{Agriónias}
\begin{itemize}
\item {Grp. gram.:f. pl.}
\end{itemize}
Antigas festas em honra de Baccho, na Beócia.
\section{Agriopo}
\begin{itemize}
\item {Grp. gram.:m.}
\end{itemize}
\begin{itemize}
\item {Proveniência:(Gr. \textunderscore agriopos\textunderscore )}
\end{itemize}
Peixe dos mares austraes.
Insecto lepidóptero nocturno.
\section{Agriota}
\begin{itemize}
\item {Grp. gram.:f.}
\end{itemize}
Cereja brava.
(Por \textunderscore agreota\textunderscore  de \textunderscore agre\textunderscore )
\section{Agriote}
\begin{itemize}
\item {Grp. gram.:m.}
\end{itemize}
Insecto coleóptero pentâmero.
\section{Agriothymia}
\begin{itemize}
\item {Grp. gram.:f.}
\end{itemize}
Tendência irresistível para actos de crueldade.
\section{Agriothýmico}
\begin{itemize}
\item {Grp. gram.:adj.}
\end{itemize}
Relativo á \textunderscore agriothymia\textunderscore .
\section{Agriotimia}
\begin{itemize}
\item {Grp. gram.:f.}
\end{itemize}
Tendência irresistível para actos de crueldade.
\section{Agriotímico}
\begin{itemize}
\item {Grp. gram.:adj.}
\end{itemize}
Relativo á \textunderscore agriotimia\textunderscore .
\section{Agripa}
\begin{itemize}
\item {Grp. gram.:adj.}
\end{itemize}
\begin{itemize}
\item {Utilização:Med.}
\end{itemize}
Diz-se da criança que, ao nascer, apresenta primeiro os pés.
\section{Agripalma}
\begin{itemize}
\item {Grp. gram.:f.}
\end{itemize}
Planta medicinal labiada.
\section{Agripene}
\begin{itemize}
\item {Grp. gram.:adj.}
\end{itemize}
\begin{itemize}
\item {Proveniência:(Do lat. \textunderscore acer\textunderscore  + \textunderscore penna\textunderscore )}
\end{itemize}
Diz-se das aves que têm a cauda em fórma aguçada.
\section{Agripenne}
\begin{itemize}
\item {Grp. gram.:adj.}
\end{itemize}
\begin{itemize}
\item {Proveniência:(Do lat. \textunderscore acer\textunderscore  + \textunderscore penna\textunderscore )}
\end{itemize}
Diz-se das aves que têm a cauda em fórma aguçada.
\section{Agripina}
\begin{itemize}
\item {Grp. gram.:f.}
\end{itemize}
Espécie de borboleta brasileira.
\section{Agrippa}
\begin{itemize}
\item {Grp. gram.:adj.}
\end{itemize}
\begin{itemize}
\item {Utilização:Med.}
\end{itemize}
Diz-se da criança que, ao nascer, apresenta primeiro os pés.
\section{Agrippina}
\begin{itemize}
\item {Grp. gram.:f.}
\end{itemize}
Espécie de borboleta brasileira.
\section{Agrisalhado}
\begin{itemize}
\item {Grp. gram.:adj.}
\end{itemize}
\begin{itemize}
\item {Proveniência:(De \textunderscore agrisalhar\textunderscore )}
\end{itemize}
Um tanto grisalho.
\section{Agrisalhar}
\begin{itemize}
\item {Grp. gram.:v. t.}
\end{itemize}
Tornar grisalho.
\section{Agro}
\begin{itemize}
\item {Grp. gram.:m.}
\end{itemize}
\begin{itemize}
\item {Utilização:Ant.}
\end{itemize}
\begin{itemize}
\item {Utilização:Prov.}
\end{itemize}
\begin{itemize}
\item {Utilização:alent.}
\end{itemize}
\begin{itemize}
\item {Proveniência:(Lat. \textunderscore ager\textunderscore )}
\end{itemize}
Campo.
Terra cultivada ou cultivável.
Rendimento de sementeiras e gados de uma herdade.
\section{Agro}
\begin{itemize}
\item {Grp. gram.:adj.}
\end{itemize}
\begin{itemize}
\item {Grp. gram.:M.}
\end{itemize}
\begin{itemize}
\item {Proveniência:(Lat. \textunderscore acrus\textunderscore )}
\end{itemize}
O mesmo que \textunderscore acre\textunderscore ^2.
Azêdo.
Escabroso.
Sabor ácido.
O mesmo que \textunderscore azedume\textunderscore . Cf. Garrett, \textunderscore Romanceiro\textunderscore , I, 27.
\section{Agrodoce}
\begin{itemize}
\item {Grp. gram.:adj.}
\end{itemize}
(V.agridoce)
\section{Agoa}
\begin{itemize}
\item {Grp. gram.:f.}
\end{itemize}
\begin{itemize}
\item {Grp. gram.:Pl.}
\end{itemize}
\begin{itemize}
\item {Proveniência:(Lat. \textunderscore aqua\textunderscore )}
\end{itemize}
Substância líquida, incolor e inodora, composta de hydrogênio e oxygênio.
Qualquer líquido, em que água é parte principal.
Chuva.
Lustre de diamantes e pérolas.
Infusão.
\textunderscore Mãe de água\textunderscore , o mesmo que \textunderscore fonte\textunderscore . Cf. Castilho, \textunderscore Escavações\textunderscore , 15.
Vertentes do telhado.
Marés.
Hemorragia, que precede o parto.
\section{Agripnia}
\begin{itemize}
\item {Grp. gram.:f.}
\end{itemize}
\begin{itemize}
\item {Utilização:Med.}
\end{itemize}
\begin{itemize}
\item {Proveniência:(Gr. \textunderscore agripnia\textunderscore )}
\end{itemize}
O mesmo que \textunderscore insómnia\textunderscore . Cf. Pacheco, \textunderscore Promptuário\textunderscore , 25.
Ausência de somno; insómnia.
\section{Agripnocoma}
\begin{itemize}
\item {Grp. gram.:f.}
\end{itemize}
\begin{itemize}
\item {Utilização:Med.}
\end{itemize}
Insómnia, com entorpecimento ou vontade de dormir. Cf. Pacheco, \textunderscore Promptuário\textunderscore , 25.
\section{Agripnodo}
\begin{itemize}
\item {Grp. gram.:adj.}
\end{itemize}
Que priva do somno. Cf. Pacheco, \textunderscore Promptuário\textunderscore , 25.
\section{Agrografia}
\begin{itemize}
\item {Grp. gram.:f.}
\end{itemize}
\begin{itemize}
\item {Proveniência:(Do gr. \textunderscore agros\textunderscore  + \textunderscore graphein\textunderscore )}
\end{itemize}
Descripção de coisas relativas á agricultura.
\section{Agrográfico}
\begin{itemize}
\item {Grp. gram.:adj.}
\end{itemize}
Relativo á \textunderscore agrografia\textunderscore .
\section{Agrógrafo}
\begin{itemize}
\item {Grp. gram.:m.}
\end{itemize}
Aquelle que trata de agrografia.
\section{Agrographia}
\begin{itemize}
\item {Grp. gram.:f.}
\end{itemize}
\begin{itemize}
\item {Proveniência:(Do gr. \textunderscore agros\textunderscore  + \textunderscore graphein\textunderscore )}
\end{itemize}
Descripção de coisas relativas á agricultura.
\section{Agrográphico}
\begin{itemize}
\item {Grp. gram.:adj.}
\end{itemize}
Relativo á \textunderscore agrographia\textunderscore .
\section{Agrógrapho}
\begin{itemize}
\item {Grp. gram.:m.}
\end{itemize}
Aquelle que trata de agrographia.
\section{Agrologia}
\begin{itemize}
\item {Grp. gram.:f.}
\end{itemize}
\begin{itemize}
\item {Proveniência:(Do gr. \textunderscore agros\textunderscore  + \textunderscore logos\textunderscore )}
\end{itemize}
Sciência, que tem por objecto o conhecimento dos terrenos, nas suas relações com a agricultura.
\section{Agrológico}
\begin{itemize}
\item {Grp. gram.:adj.}
\end{itemize}
Que diz respeito á \textunderscore agrologia\textunderscore .
\section{Agrólogo}
\begin{itemize}
\item {Grp. gram.:m.}
\end{itemize}
Aquelle que trata de agrologia.
\section{Agromancia}
\begin{itemize}
\item {Grp. gram.:f.}
\end{itemize}
\begin{itemize}
\item {Proveniência:(Do lat. \textunderscore ager\textunderscore  + gr. \textunderscore manteia\textunderscore )}
\end{itemize}
Supposta arte de adivinhar, por meio de terra.
\section{Agromante}
\begin{itemize}
\item {Grp. gram.:m.}
\end{itemize}
Aquelle que pratica a agromancia.
\section{Agromântico}
\begin{itemize}
\item {Grp. gram.:adj.}
\end{itemize}
Relativo á agromancia.
\section{Agromania}
\begin{itemize}
\item {Grp. gram.:f.}
\end{itemize}
\begin{itemize}
\item {Proveniência:(De \textunderscore agro\textunderscore ^1 + \textunderscore mania\textunderscore )}
\end{itemize}
Paixão, mania, pela agricultura.
\section{Agromaníaco}
\begin{itemize}
\item {Grp. gram.:m.  e  adj.}
\end{itemize}
O que é maníaco pela agricultura.
\section{Agromiza}
\begin{itemize}
\item {Grp. gram.:f.}
\end{itemize}
Gênero de insectos dípteros.
\section{Agromyza}
\begin{itemize}
\item {Grp. gram.:f.}
\end{itemize}
Gênero de insectos dípteros.
\section{Agronometria}
\begin{itemize}
\item {Grp. gram.:f.}
\end{itemize}
\begin{itemize}
\item {Proveniência:(Do gr. \textunderscore agronomia\textunderscore  + \textunderscore metron\textunderscore )}
\end{itemize}
Conhecimento do que póde produzir um terreno cultivado.
\section{Agronomia}
\begin{itemize}
\item {Grp. gram.:f.}
\end{itemize}
Complexo de preceitos e theorias sobre agricultura.
(Cp. \textunderscore agrónomo\textunderscore )
\section{Agronómico}
\begin{itemize}
\item {Grp. gram.:adj.}
\end{itemize}
Relativo á \textunderscore agronomia\textunderscore .
\section{Agrónomo}
\begin{itemize}
\item {Grp. gram.:m.}
\end{itemize}
\begin{itemize}
\item {Proveniência:(Gr. \textunderscore agronomos\textunderscore )}
\end{itemize}
Aquelle que tem diploma official de agronomia.
Aquelle que é versado em agronomia.
Aquelle que a professa.
\section{Agropila}
\begin{itemize}
\item {Grp. gram.:m.}
\end{itemize}
Bezoar das cabras e de outros animaes.
\section{Agropiro}
\begin{itemize}
\item {Grp. gram.:m.}
\end{itemize}
Gênero de planta gramíneas.
\section{Agropyla}
\begin{itemize}
\item {Grp. gram.:m.}
\end{itemize}
Bezoar das cabras e de outros animaes.
\section{Agropyro}
\begin{itemize}
\item {Grp. gram.:m.}
\end{itemize}
Gênero de planta gramíneas.
\section{Agror}
\begin{itemize}
\item {Grp. gram.:m.}
\end{itemize}
\begin{itemize}
\item {Proveniência:(Lat. \textunderscore acror\textunderscore )}
\end{itemize}
Azedume.
Amargura. Cf. Crespo, \textunderscore Miniaturas\textunderscore , 21.
\section{Agrósteas}
\begin{itemize}
\item {Grp. gram.:f. pl.}
\end{itemize}
O mesmo que \textunderscore agrostídeas\textunderscore .
\section{Agrostema}
\begin{itemize}
\item {Grp. gram.:f.}
\end{itemize}
Gênero de plantas, que crescem entre o trigo.
\section{Agróstero}
\begin{itemize}
\item {Grp. gram.:m.}
\end{itemize}
Secção de lepidópteros.
\section{Agrostícula}
\begin{itemize}
\item {Grp. gram.:f.}
\end{itemize}
Gênero de gramíneas.
\section{Agróstide}
\begin{itemize}
\item {Grp. gram.:f.}
\end{itemize}
\begin{itemize}
\item {Proveniência:(Gr. \textunderscore agrostis\textunderscore )}
\end{itemize}
Planta vivaz, da fam. das gramíneas.
\section{Agrostídeas}
\begin{itemize}
\item {Grp. gram.:f. pl.}
\end{itemize}
\begin{itemize}
\item {Proveniência:(De \textunderscore agróstide\textunderscore )}
\end{itemize}
Tribo de plantas gramíneas.
\section{Agróstido}
\begin{itemize}
\item {Grp. gram.:m.}
\end{itemize}
O mesmo que \textunderscore agróstide\textunderscore .
\section{Agróstis}
\begin{itemize}
\item {Grp. gram.:f.}
\end{itemize}
(V.agróstide)
\section{Agrostografia}
\begin{itemize}
\item {Grp. gram.:f.}
\end{itemize}
Parte da Botânica, que estuda as plantas gramíneas.
(Cp. \textunderscore agrostógrapho\textunderscore )
\section{Agrostográfico}
\begin{itemize}
\item {Grp. gram.:adj.}
\end{itemize}
Relativo á agrostografia.
\section{Agrostógrafo}
\begin{itemize}
\item {Grp. gram.:m.}
\end{itemize}
\begin{itemize}
\item {Proveniência:(Do gr. \textunderscore agrostis\textunderscore  + \textunderscore graphein\textunderscore )}
\end{itemize}
Aquelle que se dedica á agrostografia.
\section{Agrostographia}
\begin{itemize}
\item {Grp. gram.:f.}
\end{itemize}
Parte da Botânica, que estuda as plantas gramíneas.
(Cp. \textunderscore agrostógrapho\textunderscore )
\section{Agrostográphico}
\begin{itemize}
\item {Grp. gram.:adj.}
\end{itemize}
Relativo á agrostographia.
\section{Agrostógrapho}
\begin{itemize}
\item {Grp. gram.:m.}
\end{itemize}
\begin{itemize}
\item {Proveniência:(Do gr. \textunderscore agrostis\textunderscore  + \textunderscore graphein\textunderscore )}
\end{itemize}
Aquelle que se dedica á agrostographia.
\section{Agrostologia}
\begin{itemize}
\item {Grp. gram.:f.}
\end{itemize}
O mesmo que \textunderscore agrostographia\textunderscore .
\section{Agrostológico}
\begin{itemize}
\item {Grp. gram.:adj.}
\end{itemize}
O mesmo que \textunderscore agrostográphico\textunderscore .
\section{Agrostólogo}
\begin{itemize}
\item {Grp. gram.:adj.}
\end{itemize}
O mesmo que \textunderscore agrostógrapho\textunderscore .
\section{Agrótide}
\begin{itemize}
\item {Grp. gram.:f.}
\end{itemize}
\begin{itemize}
\item {Proveniência:(Gr. \textunderscore agrotis\textunderscore )}
\end{itemize}
Gênero de insectos da fam. dos agrotídeos.
\section{Agrotídeos}
\begin{itemize}
\item {Grp. gram.:m. pl.}
\end{itemize}
\begin{itemize}
\item {Proveniência:(Do gr. \textunderscore agrotis\textunderscore  + \textunderscore eidos\textunderscore )}
\end{itemize}
Família de insectos nocturnos, da ordem dos lepidópteros.
\section{Agrótido}
\begin{itemize}
\item {Grp. gram.:m.}
\end{itemize}
(V.agrótide)
\section{Agroujado}
\begin{itemize}
\item {Grp. gram.:adj.}
\end{itemize}
\begin{itemize}
\item {Utilização:T. de Avintes}
\end{itemize}
O mesmo que \textunderscore adoentado\textunderscore .
\section{Agrumelar}
\textunderscore v. t.\textunderscore  (e \textunderscore der.\textunderscore )
(V. \textunderscore agrumular\textunderscore , etc.)
\section{Agrumular}
\begin{itemize}
\item {Grp. gram.:v. t.}
\end{itemize}
Fazer coagular em grúmulos.
\section{Agrupação}
\begin{itemize}
\item {Grp. gram.:f.}
\end{itemize}
O mesmo que \textunderscore agrupamento\textunderscore . Cf. A. Serpa, \textunderscore Da Nacionalidade\textunderscore , 18.
\section{Agrupamento}
\begin{itemize}
\item {Grp. gram.:m.}
\end{itemize}
Reunião.
Acto ou effeito de agrupar.
\section{Agrupar}
\begin{itemize}
\item {Grp. gram.:v. t.}
\end{itemize}
Formar grupo de.
Dispor em grupo.
\section{Agrura}
\begin{itemize}
\item {Grp. gram.:f.}
\end{itemize}
Qualidade do que é agro.
Aspereza.
Dissabor.
\section{Agrypnia}
\begin{itemize}
\item {Grp. gram.:f.}
\end{itemize}
\begin{itemize}
\item {Utilização:Med.}
\end{itemize}
O mesmo que \textunderscore insómnia\textunderscore . Cf. Pacheco, \textunderscore Promptuário\textunderscore , 25.
\section{Agrypnocoma}
\begin{itemize}
\item {Grp. gram.:f.}
\end{itemize}
\begin{itemize}
\item {Utilização:Med.}
\end{itemize}
Insómnia, com entorpecimento ou vontade de dormir. Cf. Pacheco, \textunderscore Promptuário\textunderscore , 25.
\section{Agrypnodo}
\begin{itemize}
\item {Grp. gram.:adj.}
\end{itemize}
Que priva do somno. Cf. Pacheco, \textunderscore Promptuário\textunderscore , 25.
\section{Água}
\begin{itemize}
\item {Grp. gram.:f.}
\end{itemize}
\begin{itemize}
\item {Grp. gram.:Pl.}
\end{itemize}
\begin{itemize}
\item {Proveniência:(Lat. \textunderscore aqua\textunderscore )}
\end{itemize}
Substância líquida, incolor e inodora, composta de hydrogênio e oxygênio.
Qualquer líquido, em que água é parte principal.
Chuva.
Lustre de diamantes e pérolas.
Infusão.
\textunderscore Mãe de água\textunderscore , o mesmo que \textunderscore fonte\textunderscore . Cf. Castilho, \textunderscore Escavações\textunderscore , 15.
Vertentes do telhado.
Marés.
Hemorragia, que precede o parto.
\section{Água}
\begin{itemize}
\item {Grp. gram.:f.}
\end{itemize}
Arvore da ilha de San-Thomé.
\section{Aguá}
\begin{itemize}
\item {Grp. gram.:m.}
\end{itemize}
Grande reptil batrácio do Brasil.
\section{Água-aberta}
\begin{itemize}
\item {Grp. gram.:f.}
\end{itemize}
\begin{itemize}
\item {Utilização:Pop.}
\end{itemize}
O mesmo que \textunderscore gonorrheia\textunderscore .
\section{Agua-ardente}
\begin{itemize}
\item {Grp. gram.:f.}
\end{itemize}
(V.aguardente)
\section{Aguaça}
\begin{itemize}
\item {Grp. gram.:f.}
\end{itemize}
Porção de água, que corre pelo chão, depois de um aguaceiro; enxurrada.
(Cast. \textunderscore aguaza\textunderscore )
\section{Aguaçal}
\begin{itemize}
\item {fónica:á-gu}
\end{itemize}
\begin{itemize}
\item {Grp. gram.:m.}
\end{itemize}
\begin{itemize}
\item {Proveniência:(De \textunderscore aguaça\textunderscore )}
\end{itemize}
Charco.
\section{Aguacate}
\begin{itemize}
\item {Grp. gram.:m.}
\end{itemize}
Árvore laurácea.
Fruto da mesma árvore.
Espécie de esmeralda, cuja fórma se assemelha á do fruto do aguacate.
\section{Aguaceira}
\begin{itemize}
\item {fónica:á-gu}
\end{itemize}
\begin{itemize}
\item {Grp. gram.:f.}
\end{itemize}
Água ou saliva que, por indisposição de estômago, sai da boca.
\section{Aguaceiro}
\begin{itemize}
\item {fónica:á-gu}
\end{itemize}
\begin{itemize}
\item {Grp. gram.:m.}
\end{itemize}
\begin{itemize}
\item {Utilização:Fig.}
\end{itemize}
\begin{itemize}
\item {Proveniência:(De \textunderscore aguaça\textunderscore )}
\end{itemize}
Chuva forte e repentina.
Contratempo, infortúnio.
Ralhos, zanga.
\section{Aguacento}
\begin{itemize}
\item {fónica:á-gu}
\end{itemize}
\begin{itemize}
\item {Grp. gram.:adj.}
\end{itemize}
\begin{itemize}
\item {Proveniência:(De \textunderscore aguaça\textunderscore )}
\end{itemize}
Semelhante a água.
Impregnado de água.
Aquoso.
\section{Aguachado}
\begin{itemize}
\item {Grp. gram.:adj.}
\end{itemize}
\begin{itemize}
\item {Utilização:Bras}
\end{itemize}
\begin{itemize}
\item {Proveniência:(De \textunderscore guacho\textunderscore ^2?)}
\end{itemize}
Diz-se do cavallo gordo e barrigudo.
\section{Aguacuacuan}
\begin{itemize}
\item {Grp. gram.:m.}
\end{itemize}
Sapo grande do Brasil.
\section{Aguada}
\begin{itemize}
\item {fónica:á-gua}
\end{itemize}
\begin{itemize}
\item {Grp. gram.:f.}
\end{itemize}
\begin{itemize}
\item {Proveniência:(De \textunderscore aguado\textunderscore )}
\end{itemize}
Abastecimento de água doce, para viagens marítimas.
Lugar, onde se faz êsse abastecimento.
Mistura de água e claras de ovos, empregada pelos encadernadores.
Desenho a tinta ou a lápis, sôbre que se applicam tintas transparentes.
\section{Aguadeira}
\begin{itemize}
\item {fónica:á-gu}
\end{itemize}
\begin{itemize}
\item {Grp. gram.:f.}
\end{itemize}
\begin{itemize}
\item {Grp. gram.:Adj.}
\end{itemize}
\begin{itemize}
\item {Utilização:Prov.}
\end{itemize}
\begin{itemize}
\item {Utilização:ant.}
\end{itemize}
Mulher, que vende água ou a leva aos domicílios.
Dizia-se de uma capa, própria para resguardar da água da chuva. Cf. \textunderscore Cancion. da Vaticana\textunderscore .
(Cp. \textunderscore aguadeiro\textunderscore )
\section{Aguadeiras}
\begin{itemize}
\item {fónica:á-gu}
\end{itemize}
\begin{itemize}
\item {Grp. gram.:f. pl.}
\end{itemize}
As pennas maiores das asas dos falcões e de outras aves que se empregavam em altanaria.
(Corr. de \textunderscore aguiadeiras\textunderscore , de \textunderscore guiar\textunderscore )
\section{Aguadeiro}
\begin{itemize}
\item {fónica:á-gu}
\end{itemize}
\begin{itemize}
\item {Grp. gram.:m.}
\end{itemize}
Aquelle que anda pelas ruas vendendo água.
Aquelle que fornece, acarretando-a, água para consumo doméstico.
Feixe de linho em rama, para meter na água.
\section{Água-de-vegeto}
\begin{itemize}
\item {Grp. gram.:f.}
\end{itemize}
Água vegeto-mineral.
\section{Aguadilha}
\begin{itemize}
\item {fónica:á-gu}
\end{itemize}
\begin{itemize}
\item {Grp. gram.:f.}
\end{itemize}
\begin{itemize}
\item {Proveniência:(De \textunderscore aguada\textunderscore )}
\end{itemize}
Sorosidade.
Humor semelhante á água.
\section{Aguado}
\begin{itemize}
\item {fónica:á-gu}
\end{itemize}
\begin{itemize}
\item {Grp. gram.:adj.}
\end{itemize}
\begin{itemize}
\item {Utilização:Prov.}
\end{itemize}
\begin{itemize}
\item {Utilização:minh.}
\end{itemize}
\begin{itemize}
\item {Utilização:Ant.}
\end{itemize}
Misturado com água: \textunderscore Vinho aguado\textunderscore .
Que padece aguamento.
Diz-se do cabello fino e levantado.
O mesmo que \textunderscore guloso\textunderscore . Cf. Andrade Caminha.
\section{Aguadoiro}
\begin{itemize}
\item {fónica:á-gu}
\end{itemize}
\begin{itemize}
\item {Grp. gram.:m.}
\end{itemize}
\begin{itemize}
\item {Proveniência:(De \textunderscore aguar\textunderscore )}
\end{itemize}
Mólho de linho em rama, que se vai enriar.
\section{Aguador}
\begin{itemize}
\item {fónica:á-gu}
\end{itemize}
\begin{itemize}
\item {Grp. gram.:m.}
\end{itemize}
\begin{itemize}
\item {Proveniência:(De \textunderscore aguar\textunderscore )}
\end{itemize}
Regador, borrifador.
Vaso para regar.
\section{Águadouro}
\begin{itemize}
\item {fónica:á-gu}
\end{itemize}
\begin{itemize}
\item {Grp. gram.:m.}
\end{itemize}
\begin{itemize}
\item {Proveniência:(De \textunderscore aguar\textunderscore )}
\end{itemize}
Mólho de linho em rama, que se vai enriar.
\section{Aguaforte}
\begin{itemize}
\item {fónica:á-gu}
\end{itemize}
\begin{itemize}
\item {Grp. gram.:f.}
\end{itemize}
\begin{itemize}
\item {Proveniência:(It. \textunderscore acqua-forte\textunderscore )}
\end{itemize}
Ácido nítrico dissolvido.
\section{Aguafortista}
\begin{itemize}
\item {fónica:á-gu}
\end{itemize}
\begin{itemize}
\item {Grp. gram.:m.}
\end{itemize}
Gravador, que se serve de aguaforte.
\section{Água-furtada}
\begin{itemize}
\item {Grp. gram.:f.}
\end{itemize}
O último andar de uma casa, quando a janela ou as janelas dêsse andar deitam sôbre o telhado, interrompendo-se a vertente com a construcção da mesma janela ou janelas.
\section{Aguagem}
\begin{itemize}
\item {Grp. gram.:f.}
\end{itemize}
Acção de \textunderscore aguar\textunderscore .
Movimento de águas, que faz jogar o navio.
\section{Aguamá}
\begin{itemize}
\item {fónica:á-gu}
\end{itemize}
\begin{itemize}
\item {Grp. gram.:f.}
\end{itemize}
\begin{itemize}
\item {Utilização:Pop.}
\end{itemize}
Mollusco, que se desfaz em água e que não é comestível.
Nome que os pescadores dão ao celenterado \textunderscore physalia pelagica\textunderscore .
\section{Aguamãe}
\begin{itemize}
\item {fónica:á-gu}
\end{itemize}
\begin{itemize}
\item {Grp. gram.:f.}
\end{itemize}
\begin{itemize}
\item {Proveniência:(De \textunderscore água\textunderscore  + \textunderscore mãe\textunderscore )}
\end{itemize}
Água, que contém grande porção de saes.
\section{Aguamãi}
\begin{itemize}
\item {fónica:á-gu}
\end{itemize}
\begin{itemize}
\item {Grp. gram.:f.}
\end{itemize}
\begin{itemize}
\item {Proveniência:(De \textunderscore água\textunderscore  + \textunderscore mãe\textunderscore )}
\end{itemize}
Água, que contém grande porção de saes.
\section{Água-marinha-junquilha}
\begin{itemize}
\item {Grp. gram.:f.}
\end{itemize}
Espécie de pedra preciosa. Cf. Camillo, \textunderscore Caveira\textunderscore , 163.
\section{Água-mel}
\begin{itemize}
\item {Grp. gram.:f.}
\end{itemize}
\begin{itemize}
\item {Utilização:Prov.}
\end{itemize}
\begin{itemize}
\item {Utilização:alg.}
\end{itemize}
O mesmo que \textunderscore hydromel\textunderscore .
\section{Aguamento}
\begin{itemize}
\item {fónica:á-gu}
\end{itemize}
\begin{itemize}
\item {Grp. gram.:m.}
\end{itemize}
\begin{itemize}
\item {Proveniência:(De \textunderscore aguar\textunderscore )}
\end{itemize}
Doença de animaes domésticos, por excesso de trabalho ou resfriamento.
\section{Aguante}
\begin{itemize}
\item {fónica:á-gu}
\end{itemize}
\begin{itemize}
\item {Grp. gram.:m.}
\end{itemize}
\begin{itemize}
\item {Proveniência:(De \textunderscore aguantar\textunderscore , por \textunderscore aguentar\textunderscore )}
\end{itemize}
Velame, que o navio comporta.
\section{Aguapá}
\begin{itemize}
\item {fónica:á-gu}
\end{itemize}
\begin{itemize}
\item {Grp. gram.:f.}
\end{itemize}
Planta medicinal da América.
\section{Água-pé}
\begin{itemize}
\item {fónica:á-gu}
\end{itemize}
\begin{itemize}
\item {Grp. gram.:f.}
\end{itemize}
\begin{itemize}
\item {Proveniência:(De \textunderscore água\textunderscore  + \textunderscore pé\textunderscore )}
\end{itemize}
Bebida, que se prepara, deitando água no resíduo ou pé das uvas, depois de feito o vinho.
Vinho fraco.
\section{Água-pé}
\begin{itemize}
\item {fónica:á-gu}
\end{itemize}
\begin{itemize}
\item {Grp. gram.:m.}
\end{itemize}
\begin{itemize}
\item {Utilização:Bras}
\end{itemize}
Nome genérico das vegetações que se criam á superfície de águas estagnadas.
(Do tupi)
\section{Aguapeca}
\begin{itemize}
\item {fónica:pê}
\end{itemize}
\begin{itemize}
\item {Grp. gram.:f.}
\end{itemize}
Ave brasileira.
\section{Aguar}
\begin{itemize}
\item {fónica:á-gu}
\end{itemize}
\begin{itemize}
\item {Grp. gram.:v. t.}
\end{itemize}
\begin{itemize}
\item {Grp. gram.:V. i.}
\end{itemize}
\begin{itemize}
\item {Proveniência:(De \textunderscore água\textunderscore )}
\end{itemize}
Regar, borrifar.
Misturar com água.
Frustrar.
Adquirir o mal de aguamento.
Adoecer (uma criança), por apetecer um alimento que lhe não deram.
\section{Aguarapondá}
\begin{itemize}
\item {Grp. gram.:f.}
\end{itemize}
Planta do Brasil.
\section{Água-raz}
\begin{itemize}
\item {Grp. gram.:f.}
\end{itemize}
Essência de therebentina.
\section{Aguardadoiro}
\begin{itemize}
\item {Grp. gram.:adj.}
\end{itemize}
Digno de se aguardar.
Cortejado.
\section{Aguardador}
\begin{itemize}
\item {Grp. gram.:m.}
\end{itemize}
Aquelle que aguarda.
\section{Aguardadouro}
\begin{itemize}
\item {Grp. gram.:adj.}
\end{itemize}
Digno de se aguardar.
Cortejado.
\section{Aguardamento}
\begin{itemize}
\item {Grp. gram.:m.}
\end{itemize}
Acto de \textunderscore aguardar\textunderscore .
\section{Aguardante}
\begin{itemize}
\item {Grp. gram.:m.  e  adj.}
\end{itemize}
\begin{itemize}
\item {Utilização:Des.}
\end{itemize}
O que guarda ou observa as condições de um contrato.
\section{Aguardar}
\begin{itemize}
\item {Grp. gram.:v. t.}
\end{itemize}
\begin{itemize}
\item {Grp. gram.:V. i.}
\end{itemize}
\begin{itemize}
\item {Proveniência:(De \textunderscore guardar\textunderscore )}
\end{itemize}
Vigiar.
Esperar.
Acatar.
Cortejar.
Estar esperando.
\section{Aguardentação}
\begin{itemize}
\item {fónica:á-gu}
\end{itemize}
\begin{itemize}
\item {Grp. gram.:f.}
\end{itemize}
Acto ou effeito de \textunderscore aguardentar\textunderscore . Cf. F. Lapa, \textunderscore Alm. do Lavrador\textunderscore , 1869.
\section{Aguardentadamente}
\begin{itemize}
\item {fónica:á-gu}
\end{itemize}
\begin{itemize}
\item {Grp. gram.:adv.}
\end{itemize}
De modo \textunderscore aguardentado\textunderscore .
\section{Aguardentado}
\begin{itemize}
\item {fónica:á-gu}
\end{itemize}
\begin{itemize}
\item {Grp. gram.:adj.}
\end{itemize}
Misturado com aguardente.
Que tem sabor de aguardente.
Que denota o hábito de beber aguardente: \textunderscore voz aguardentada\textunderscore .
\section{Aguardentar}
\begin{itemize}
\item {fónica:á-gu}
\end{itemize}
\begin{itemize}
\item {Grp. gram.:v. t.}
\end{itemize}
Misturar com aguardente.
Fartar de aguardente.
\section{Aguardente}
\begin{itemize}
\item {fónica:á-gu}
\end{itemize}
\begin{itemize}
\item {Grp. gram.:f.}
\end{itemize}
\begin{itemize}
\item {Proveniência:(De \textunderscore água\textunderscore  + \textunderscore ardente\textunderscore )}
\end{itemize}
Bebida espirituosa e branca, resultante de destillação de substâncias susceptíveis de fermentação.
\section{Aguardenteiro}
\begin{itemize}
\item {fónica:á-gu}
\end{itemize}
\begin{itemize}
\item {Grp. gram.:m.}
\end{itemize}
Aquelle que faz aguardente; aquelle que a vende; aquelle que a bebe em abundância.
\section{Aguardentia}
\begin{itemize}
\item {fónica:á-gu}
\end{itemize}
\begin{itemize}
\item {Grp. gram.:f.}
\end{itemize}
\begin{itemize}
\item {Utilização:Chul.}
\end{itemize}
Estado de quem se embriagou com aguardente.
\section{Aguardentoso}
\begin{itemize}
\item {fónica:á-gu}
\end{itemize}
\begin{itemize}
\item {Grp. gram.:adj.}
\end{itemize}
Aguardentado.
Que tem o sabor ou o cheiro da aguardente.
\section{Aguardo}
\begin{itemize}
\item {Grp. gram.:m.}
\end{itemize}
\begin{itemize}
\item {Utilização:Prov.}
\end{itemize}
\begin{itemize}
\item {Utilização:alent.}
\end{itemize}
\begin{itemize}
\item {Proveniência:(De \textunderscore aguardar\textunderscore )}
\end{itemize}
Sítio, onde o caçador espera a caça.
\section{Água-régia}
\begin{itemize}
\item {Grp. gram.:f.}
\end{itemize}
Reagente muito enérgico, que dissolve o oiro e é produzido pela mistura de ácido nítrico e de ácido chlorhýdrico.
\section{Aguarela}
\begin{itemize}
\item {fónica:á-gu}
\end{itemize}
\begin{itemize}
\item {Grp. gram.:f.}
\end{itemize}
\begin{itemize}
\item {Proveniência:(It. \textunderscore acquarella\textunderscore )}
\end{itemize}
Tinta diluída em água.
Lavadura de gesso moído e colla de baldreu.
Pintura com tintas diluídas em água, sem sobreposição de umas a outras.
\section{Aguarelista}
\begin{itemize}
\item {fónica:á-gu}
\end{itemize}
\begin{itemize}
\item {Grp. gram.:m.}
\end{itemize}
Aquelle que pinta aguarelas.
\section{Aguarella}
\begin{itemize}
\item {fónica:á-gu}
\end{itemize}
\begin{itemize}
\item {Grp. gram.:f.}
\end{itemize}
\begin{itemize}
\item {Proveniência:(It. \textunderscore acquarella\textunderscore )}
\end{itemize}
Tinta diluída em água.
Lavadura de gesso moído e colla de baldreu.
Pintura com tintas diluídas em água, sem sobreposição de umas a outras.
\section{Aguarentador}
\begin{itemize}
\item {Grp. gram.:m.}
\end{itemize}
Aquelle que aguarenta.
Detractor.
\section{Aguarentar}
\begin{itemize}
\item {Grp. gram.:v. t.}
\end{itemize}
Cercear, aparar em roda.
Murmurar de; desacreditar.
Dissaborear. Cf. J. Dinis, \textunderscore Morgadinha\textunderscore , 114.
\section{Aguariço}
\begin{itemize}
\item {Grp. gram.:m.}
\end{itemize}
Planta, de fôlhas semelhantes ás do zimbro.
\section{Água-ruça}
\begin{itemize}
\item {Grp. gram.:f.}
\end{itemize}
Líquido escuro, que é o resíduo do fabrico do azeite.
\section{Águas-furtadas}
\begin{itemize}
\item {Grp. gram.:f. pl.}
\end{itemize}
O último andar de uma casa, quando a janela ou as janelas dêsse andar deitam sôbre o telhado, interrompendo-se a vertente com a construcção da mesma janela ou janelas.
\section{Aguastar}
\begin{itemize}
\item {Grp. gram.:v. i.}
\end{itemize}
\begin{itemize}
\item {Utilização:Ant.}
\end{itemize}
\begin{itemize}
\item {Proveniência:(De \textunderscore água\textunderscore  + \textunderscore estar\textunderscore )}
\end{itemize}
Ir com todas as velas (o navio).
\section{Água-tofana}
\begin{itemize}
\item {Grp. gram.:f.}
\end{itemize}
\begin{itemize}
\item {Proveniência:(It. \textunderscore acqua\textunderscore  + \textunderscore Toffana\textunderscore , n. p.)}
\end{itemize}
Veneno, célebre em Itália nos séculos XVI e XVII, inventado por uma mulher,
\textunderscore Toffana\textunderscore , e que era uma solução concentrada de arsênico.
\section{Água-viva}
\begin{itemize}
\item {Grp. gram.:f.}
\end{itemize}
\begin{itemize}
\item {Utilização:Açor}
\end{itemize}
O mesmo que \textunderscore alforreca\textunderscore .
\section{Aguazil}
\begin{itemize}
\item {fónica:á-gu}
\end{itemize}
\begin{itemize}
\item {Grp. gram.:m.}
\end{itemize}
\begin{itemize}
\item {Proveniência:(Do ár. \textunderscore al-uazir\textunderscore )}
\end{itemize}
Antigo empregado administrativo e judicial.
Designação genérica de qualquer empregado inferior de justiça, como o official de diligências.
\section{Aguça}
\begin{itemize}
\item {Grp. gram.:f.}
\end{itemize}
\begin{itemize}
\item {Utilização:Ant.}
\end{itemize}
\begin{itemize}
\item {Proveniência:(De \textunderscore aguçar\textunderscore )}
\end{itemize}
Pressa.
Diligência, zêlo.
\section{Aguçadamente}
\begin{itemize}
\item {Grp. gram.:adv.}
\end{itemize}
* Apressadamente; com zêlo.
De modo \textunderscore aguçado\textunderscore .
\section{Aguçadeira}
\begin{itemize}
\item {Grp. gram.:f.}
\end{itemize}
Pedra de aguçar ou de amolar.
\section{Aguçado}
\begin{itemize}
\item {Grp. gram.:adj.}
\end{itemize}
\begin{itemize}
\item {Utilização:Ant.}
\end{itemize}
Agudo.
Afiado.
Apressado; diligente.
\section{Aguçadura}
\begin{itemize}
\item {Grp. gram.:f.}
\end{itemize}
Acto de \textunderscore aguçar\textunderscore .
\section{Aguçamento}
\begin{itemize}
\item {Grp. gram.:m.}
\end{itemize}
Effeito de \textunderscore aguçar\textunderscore .
\section{Aguçar}
\begin{itemize}
\item {Grp. gram.:v. t.}
\end{itemize}
\begin{itemize}
\item {Utilização:Fig.}
\end{itemize}
\begin{itemize}
\item {Grp. gram.:V. i.}
\end{itemize}
\begin{itemize}
\item {Utilização:Bras}
\end{itemize}
Tornar agudo, adelgaçar na ponta.
Afiar.
Amolar.
Incitar; estimular.
Tornar perspicaz.
Tornar activo, ligeiro, zeloso.
Avançar para o poste do vencedor, em corridas de cavallos.
(B. lat. \textunderscore acutiare\textunderscore )
\section{Aguço}
\begin{itemize}
\item {Grp. gram.:m.}
\end{itemize}
Objecto aguçado; espêto; estoque.
Acto ou effeito de \textunderscore aguçar\textunderscore .
\section{Aguçosamente}
\begin{itemize}
\item {Grp. gram.:adv.}
\end{itemize}
De modo \textunderscore aguçoso\textunderscore .
\section{Aguçoso}
\begin{itemize}
\item {Grp. gram.:adj.}
\end{itemize}
\begin{itemize}
\item {Utilização:Ant.}
\end{itemize}
\begin{itemize}
\item {Proveniência:(De \textunderscore aguçar\textunderscore )}
\end{itemize}
Apressado.
Activo, diligente.
\section{Agudamente}
\begin{itemize}
\item {Grp. gram.:adv.}
\end{itemize}
De modo \textunderscore agudo\textunderscore .
Sagazmente; com agudeza.
\section{Agude}
\begin{itemize}
\item {Grp. gram.:f.}
\end{itemize}
Formiga grande, luzidia, e com asas, aproveitada pelos passarinheiros como isca, em caça de aves.
(Alter. de \textunderscore alúdea\textunderscore  = cast. \textunderscore aluda\textunderscore , fem. de \textunderscore aludo\textunderscore  = port. \textunderscore alado\textunderscore , que tem asas)
\section{Agúdea}
\begin{itemize}
\item {Grp. gram.:f.}
\end{itemize}
Formiga grande, luzidia, e com asas, aproveitada pelos passarinheiros como isca, em caça de aves.
(Alter. de \textunderscore alúdea\textunderscore  = cast. \textunderscore aluda\textunderscore , fem. de \textunderscore aludo\textunderscore  = port. \textunderscore alado\textunderscore , que tem asas)
\section{Agudelho}
\begin{itemize}
\item {fónica:dê}
\end{itemize}
\begin{itemize}
\item {Grp. gram.:m.}
\end{itemize}
O mesmo que \textunderscore agudenho\textunderscore .
\section{Agudenho}
\begin{itemize}
\item {Grp. gram.:m.}
\end{itemize}
\begin{itemize}
\item {Proveniência:(De \textunderscore agudo\textunderscore )}
\end{itemize}
Casta de uva do Doiro e do Alentejo.
\section{Agudento}
\begin{itemize}
\item {Grp. gram.:adj.}
\end{itemize}
\begin{itemize}
\item {Utilização:Ant.}
\end{itemize}
Ponteagudo:«\textunderscore o nariz afilado e agudento\textunderscore ». \textunderscore Chrón. dos Carmelitas\textunderscore , t. I, p. III.
\section{Agudez}
\begin{itemize}
\item {Grp. gram.:f.}
\end{itemize}
O mesmo que \textunderscore agudeza\textunderscore .
\section{Agudeza}
\begin{itemize}
\item {Grp. gram.:f.}
\end{itemize}
\begin{itemize}
\item {Utilização:Fig.}
\end{itemize}
Qualidade do que é cortante ou \textunderscore agudo\textunderscore .
Perspicácia; intensidade.
\section{Agudo}
\begin{itemize}
\item {Grp. gram.:adj.}
\end{itemize}
\begin{itemize}
\item {Utilização:Fig.}
\end{itemize}
\begin{itemize}
\item {Utilização:Ant.}
\end{itemize}
\begin{itemize}
\item {Proveniência:(Lat. \textunderscore acutus\textunderscore )}
\end{itemize}
Terminado em ponta ou gume.
Fino; penetrante.
Sagaz.
O mesmo que \textunderscore ácido\textunderscore . Cf. \textunderscore Leal Conselheiro\textunderscore , \textunderscore in\textunderscore  Cortesão, \textunderscore Subsíd.\textunderscore 
Irado; exasperado. Cf. \textunderscore Port. Mon. Hist.\textunderscore , \textunderscore Script.\textunderscore , 281.
\section{Á!}
\begin{itemize}
\item {Grp. gram.:interj.}
\end{itemize}
\begin{itemize}
\item {Proveniência:(T. commum a muitas línguas)}
\end{itemize}
(design. de admiração, dôr, alegria, etc.)
\section{Á! á! á!}
\begin{itemize}
\item {Grp. gram.:interj.}
\end{itemize}
(correspondente ao riso franco, á gargalhada)
\section{Aginário}
\begin{itemize}
\item {Grp. gram.:adj.}
\end{itemize}
\begin{itemize}
\item {Proveniência:(Do gr. \textunderscore a\textunderscore  priv. + \textunderscore gune\textunderscore )}
\end{itemize}
Diz-se, segundo De-Candolle, das flores, formadas pelos tegumentos floraes e estames transformados, e em que falta o pistillo.
\section{Aginiano}
\begin{itemize}
\item {Grp. gram.:m.}
\end{itemize}
\begin{itemize}
\item {Proveniência:(Do gr. \textunderscore a\textunderscore  priv. + \textunderscore gune\textunderscore )}
\end{itemize}
Membro de uma seita christan, que proscrevia o casamento.
\section{Agínico}
\begin{itemize}
\item {Grp. gram.:adj.}
\end{itemize}
\begin{itemize}
\item {Utilização:Bot.}
\end{itemize}
\begin{itemize}
\item {Proveniência:(Do gr. \textunderscore a\textunderscore  priv. + \textunderscore gune\textunderscore )}
\end{itemize}
Diz-se da inserção dos estames, quando estes não adherem ao ovário.
\section{Ágino}
\begin{itemize}
\item {Grp. gram.:adj.}
\end{itemize}
\begin{itemize}
\item {Proveniência:(Do gr. \textunderscore a\textunderscore  priv. + \textunderscore gune\textunderscore )}
\end{itemize}
Que não tem órgãos femininos, (falando-se de vegetaes).
\section{Agueira}
\begin{itemize}
\item {fónica:á-gu}
\end{itemize}
\begin{itemize}
\item {Grp. gram.:f.}
\end{itemize}
\begin{itemize}
\item {Utilização:Prov.}
\end{itemize}
\begin{itemize}
\item {Utilização:beir.}
\end{itemize}
O mesmo que \textunderscore goteira\textunderscore .
\section{Agueiro}
\begin{itemize}
\item {fónica:á-gu}
\end{itemize}
\begin{itemize}
\item {Grp. gram.:m.}
\end{itemize}
Rêgo, em que se juntam as águas da estrada.
Cano, em que se reúnem as águas do telhado.
Orifício, nos muros das propriedades rústicas, pelo qual entram as águas aproveitaveis na cultura.
\section{Aguentador}
\begin{itemize}
\item {fónica:á-gu-en}
\end{itemize}
\begin{itemize}
\item {Grp. gram.:m.}
\end{itemize}
\begin{itemize}
\item {Proveniência:(De \textunderscore aguentar\textunderscore )}
\end{itemize}
O que \textunderscore aguenta\textunderscore .
\section{Aguentar}
\begin{itemize}
\item {fónica:á-gu-en}
\end{itemize}
\begin{itemize}
\item {Grp. gram.:v. t.}
\end{itemize}
\begin{itemize}
\item {Proveniência:(It. \textunderscore agguantare\textunderscore )}
\end{itemize}
Conservar em equilíbrio.
Sustentar na mão.
Soffrer, supportar.
\section{Aguente}
\begin{itemize}
\item {fónica:gu-en}
\end{itemize}
\begin{itemize}
\item {Grp. gram.:m.}
\end{itemize}
O mesmo ou melhor que \textunderscore aguante\textunderscore .
\section{Aguerrear}
\begin{itemize}
\item {Grp. gram.:v. t.}
\end{itemize}
A*fazer á guerra; acostumar a ella.
\section{Aguerridamente}
\begin{itemize}
\item {Grp. gram.:adv.}
\end{itemize}
De modo \textunderscore aguerrido\textunderscore .
\section{Aguerrido}
\begin{itemize}
\item {Grp. gram.:adj.}
\end{itemize}
\begin{itemize}
\item {Proveniência:(De \textunderscore aguerrir\textunderscore )}
\end{itemize}
Inclinado á guerra.
Que tem modos bellicosos.
Valente.
\section{Aguerrilhar}
\begin{itemize}
\item {Grp. gram.:v. t.}
\end{itemize}
Converter em guerrilha; formar guerrilhas de.
\section{Aguerrimento}
\begin{itemize}
\item {Grp. gram.:m.}
\end{itemize}
Acto de \textunderscore aguerrir\textunderscore .
\section{Aguerrir}
\begin{itemize}
\item {Grp. gram.:v. t.}
\end{itemize}
\begin{itemize}
\item {Proveniência:(T. cast.)}
\end{itemize}
Acostumar á guerra.
Afazer ás lutas, aos trabalhos.
\section{Águia}
\begin{itemize}
\item {Grp. gram.:f.}
\end{itemize}
\begin{itemize}
\item {Proveniência:(Lat. \textunderscore aquila\textunderscore )}
\end{itemize}
Grande e vigorosa ave de rapina.
Insignia da bandeira de algumas nações.
Moéda de oiro nos Estados-Unidos.
Constellação do hemisphério boreal.
Antiga peça de artilharia.
\section{Aguiamento}
\begin{itemize}
\item {Grp. gram.:m.}
\end{itemize}
\begin{itemize}
\item {Utilização:Ant.}
\end{itemize}
\begin{itemize}
\item {Proveniência:(De \textunderscore aguiar\textunderscore  por \textunderscore guiar\textunderscore )}
\end{itemize}
Prudência, discrição.
Perspicácia.
\section{Aguian}
\begin{itemize}
\item {Grp. gram.:m.}
\end{itemize}
Gênio do mal que, segundo a crença dos Índios do Brasil, tem o poder de transformar os homens em demónios; o mesmo que \textunderscore agoman\textunderscore .
\section{Aguião}
\begin{itemize}
\item {Grp. gram.:m.}
\end{itemize}
\begin{itemize}
\item {Utilização:Ant.}
\end{itemize}
O mesmo que \textunderscore aquilão\textunderscore ^1.
\section{Aguiarado}
\begin{itemize}
\item {Grp. gram.:adj.}
\end{itemize}
\begin{itemize}
\item {Utilização:Ant.}
\end{itemize}
Esburacado; esfarrapado.
\section{Aguida}
\begin{itemize}
\item {Grp. gram.:f.}
\end{itemize}
\begin{itemize}
\item {Utilização:Prov.}
\end{itemize}
\begin{itemize}
\item {Utilização:alg.}
\end{itemize}
O mesmo que \textunderscore agude\textunderscore .
\section{Aguieiro}
\begin{itemize}
\item {Grp. gram.:m.}
\end{itemize}
Pau que vai do frechal ao pau de fileira.
(Corr. de \textunderscore guieiro\textunderscore ?)
\section{Aguieta}
\begin{itemize}
\item {fónica:êia}
\end{itemize}
\begin{itemize}
\item {Grp. gram.:f.}
\end{itemize}
Pequena águia.
\section{Aguiguiar}
\begin{itemize}
\item {Grp. gram.:v. t.}
\end{itemize}
\begin{itemize}
\item {Utilização:Prov.}
\end{itemize}
\begin{itemize}
\item {Utilização:beir.}
\end{itemize}
Soltar o grito de guigui.
\section{Aguiguros}
\begin{itemize}
\item {Grp. gram.:m. pl.}
\end{itemize}
Índios do Brasil, que habitavam a costa marítima, entre a Baía e Pernambuco.
\section{Águila}
\begin{itemize}
\item {Grp. gram.:f.}
\end{itemize}
Planta indiana, de madeira resinosa e aromática.
\section{Aguíla}
\begin{itemize}
\item {Grp. gram.:f.}
\end{itemize}
Tecido de algodão de Alepo.
\section{Aguilhada}
\begin{itemize}
\item {Grp. gram.:f.}
\end{itemize}
\begin{itemize}
\item {Proveniência:(De \textunderscore aguilhar\textunderscore )}
\end{itemize}
Pau delgado e comprido, ordinariamente com ferrão na ponta, para picar os bois na lavoira e na carretagem.
Antiga medida agrária, de 18 palmos de comprimento.
\section{Aguilhão}
\begin{itemize}
\item {Grp. gram.:m.}
\end{itemize}
\begin{itemize}
\item {Utilização:Fig.}
\end{itemize}
\begin{itemize}
\item {Utilização:Prov.}
\end{itemize}
\begin{itemize}
\item {Utilização:trasm.}
\end{itemize}
Ponta de ferro.
Ferrão.
Espécie de pequeno dardo retráctil na extremidade do abdome de alguns insectos, como a vespa.
Estímulo.
Sofrimento.
Pedra oval, no fundo do rodizio, o mesmo que \textunderscore gogo\textunderscore .
Pedra aguda, submersa no leito de um rio. Cf. \textunderscore Port. Ant. e Mod.\textunderscore , XII, 2128.
\section{Aguilhar}
\begin{itemize}
\item {Grp. gram.:v. t.}
\end{itemize}
\begin{itemize}
\item {Utilização:Ant.}
\end{itemize}
O mesmo que \textunderscore aguilhoar\textunderscore .
\section{Aguilhó}
\begin{itemize}
\item {Grp. gram.:m.}
\end{itemize}
Antigo toucado de mulher.
\section{Aguilhoada}
\begin{itemize}
\item {Grp. gram.:f.}
\end{itemize}
Picada com aguilhão.
\section{Aguilhoadamente}
\begin{itemize}
\item {Grp. gram.:adv.}
\end{itemize}
Com aguilhoadas.
\section{Aguilhoadela}
\begin{itemize}
\item {Grp. gram.:f.}
\end{itemize}
O mesmo que \textunderscore aguilhoamento\textunderscore .
\section{Aguilhoamento}
\begin{itemize}
\item {Grp. gram.:m.}
\end{itemize}
Acto de \textunderscore aguilhoar\textunderscore .
\section{Aguilhoar}
\begin{itemize}
\item {Grp. gram.:v. t.}
\end{itemize}
Picar com aguilhão ou aguilhada.
Ferir.
Magoar.
Estimular.
\section{Aguisadamente}
\begin{itemize}
\item {Grp. gram.:adv.}
\end{itemize}
\begin{itemize}
\item {Utilização:Ant.}
\end{itemize}
Apropriadamente; em bôa ordem, de modo \textunderscore aguisado\textunderscore .
\section{Aguisado}
\begin{itemize}
\item {Grp. gram.:adj.}
\end{itemize}
\begin{itemize}
\item {Utilização:Ant.}
\end{itemize}
Concertado; pôsto em ordem.
\section{Aguisar}
\begin{itemize}
\item {Grp. gram.:v. t.}
\end{itemize}
\begin{itemize}
\item {Utilização:Ant.}
\end{itemize}
\begin{itemize}
\item {Proveniência:(De \textunderscore guisa\textunderscore )}
\end{itemize}
Pôr em ordem; combinar, concertar.
\section{Aguista}
\begin{itemize}
\item {fónica:gu-is}
\end{itemize}
\begin{itemize}
\item {Grp. gram.:adj.}
\end{itemize}
\begin{itemize}
\item {Utilização:Neol.}
\end{itemize}
\begin{itemize}
\item {Grp. gram.:M.}
\end{itemize}
Relativo a águas thermaes.
Aquelle que toma águas thermaes.
\section{Aguitarrado}
\begin{itemize}
\item {Grp. gram.:adj.}
\end{itemize}
\begin{itemize}
\item {Proveniência:(De \textunderscore guitarra\textunderscore )}
\end{itemize}
Que se parece á guitarra, no feitio ou no som.
\section{Agulha}
\begin{itemize}
\item {Grp. gram.:f.}
\end{itemize}
\begin{itemize}
\item {Utilização:Prov.}
\end{itemize}
\begin{itemize}
\item {Utilização:dur.}
\end{itemize}
\begin{itemize}
\item {Utilização:Prov.}
\end{itemize}
\begin{itemize}
\item {Utilização:Prov.}
\end{itemize}
\begin{itemize}
\item {Utilização:trasm.}
\end{itemize}
\begin{itemize}
\item {Utilização:Fam.}
\end{itemize}
\begin{itemize}
\item {Grp. gram.:Pl.}
\end{itemize}
\begin{itemize}
\item {Utilização:Bras}
\end{itemize}
\begin{itemize}
\item {Utilização:T. da Guarda}
\end{itemize}
\begin{itemize}
\item {Proveniência:(It. \textunderscore aguglia\textunderscore , do lat. hyp. \textunderscore acucula\textunderscore , de \textunderscore acus\textunderscore )}
\end{itemize}
Pequena e fina haste de metal, aguçada de um lado, com um orifício no outro, por onde se enfia a linha ou cordão que serve na costura.
Offício de costureira.
Varinha curva, de metal, que tem, em vez de orifício, um pequeno gancho, e que serve para fazer meia.
Peça de máquina de costura, que tem o orifício próximo do lado aguçado. Obelisco.
Dente canino.
Fôlha de pinheiro.
Lâmina de aço magnetizado, que gira livremente sôbre um fulcro.
Extremidade do campanário.
Travessa horizontal, entre as orcellas, que serve, de eixo, á vara do lagar.
O pique de certas bebidas: \textunderscore êste vinho tem agulha\textunderscore .
Carris de ferro móveis, para facilitar a passagem dos carros, de uma para outra via.
Lâmina de aço, que percute o fulminante nas modernas armas de fogo.
Peixe de Portugal.
Ponto de juncção das espáduas.
Pique da uva, em resultado do acido carbónico.
Nome de uma armação de pesca da sardinha, na costa de Cesimbra.
\textunderscore Agulha ferrugenta\textunderscore , pessoa intrigante.
Pedaço de carne, unido ao osso do espinhaço do boi.
Cada uma das travessas, que unem as chedas do carro.
Cume da cernelha do cavallo.
\section{Agulhada}
\begin{itemize}
\item {Grp. gram.:f.}
\end{itemize}
Ferimento com agulha.
\section{Agulha-de-mato}
\begin{itemize}
\item {Grp. gram.:f.}
\end{itemize}
Planta brasileira, da fam. das leguminosas.
\section{Agulha-de-pastor}
\begin{itemize}
\item {Grp. gram.:f.}
\end{itemize}
O mesmo que \textunderscore erva-agulheira\textunderscore .
\section{Agulhadoiro}
\begin{itemize}
\item {Grp. gram.:m.}
\end{itemize}
Furo no coice da vara do lagar, e nas pedras chamadas orcellas, madres ou virgens, e pelo qual passa a agulha da vara.
\section{Agulhadouro}
\begin{itemize}
\item {Grp. gram.:m.}
\end{itemize}
Furo no coice da vara do lagar, e nas pedras chamadas orcellas, madres ou virgens, e pelo qual passa a agulha da vara.
\section{Agulhando}
\begin{itemize}
\item {Grp. gram.:m.}
\end{itemize}
\begin{itemize}
\item {Utilização:Ant.}
\end{itemize}
Alvíçaras ou presentes de anno novo. Cf. \textunderscore Cancion. C. Brancuti\textunderscore .
\section{Agulhão}
\begin{itemize}
\item {Grp. gram.:m.}
\end{itemize}
Nome, que os pescadores do Algarve dão ao peixe-agulha.
Bússola, usada a bordo, na qual se fazem as marcações da terra, e pela qual se guia o official de quarto.
\section{Agulhar}
\begin{itemize}
\item {Grp. gram.:v. t.}
\end{itemize}
\begin{itemize}
\item {Grp. gram.:V. i.}
\end{itemize}
Ferir com agulha.
Meter a agulha do lagar nos agulhadoiros.
\section{Agulheado}
\begin{itemize}
\item {Grp. gram.:m.}
\end{itemize}
Que tem fórma de agulha.
\section{Agulheira}
\begin{itemize}
\item {Grp. gram.:f.}
\end{itemize}
\begin{itemize}
\item {Proveniência:(De \textunderscore agulha\textunderscore )}
\end{itemize}
Planta, da fam. das corymbíferas.
\section{Agulheiro}
\begin{itemize}
\item {Grp. gram.:m.}
\end{itemize}
\begin{itemize}
\item {Utilização:Ant.}
\end{itemize}
\begin{itemize}
\item {Grp. gram.:Pl.}
\end{itemize}
\begin{itemize}
\item {Utilização:Prov.}
\end{itemize}
\begin{itemize}
\item {Utilização:dur.}
\end{itemize}
Pequeno estojo, em que se guardam agulhas.
Fabricante de agulhas.
Homem, que trabalha nas agulhas das linhas férreas.
Abertura estreita e profunda.
Tubo de grânulos ou lentículas de medicamentos dosimétricos.
O mesmo que \textunderscore balhesteira\textunderscore .
Orifícios, feitos expressamente entre o ensaio e as cavernas dos barcos rabelos, para escoamento da água que invade a embarcação.
\section{Agulheta}
\begin{itemize}
\item {fónica:lhê}
\end{itemize}
\begin{itemize}
\item {Grp. gram.:f.}
\end{itemize}
\begin{itemize}
\item {Grp. gram.:M.}
\end{itemize}
Agulha sem ponta, e de fundo largo, para enfiar cordões ou fitas, que hão de entrar em baínhas ou ilhós.
Extremidade metállica de cordões, com que se abotôam espartilhos, botas, etc.
Tubo metállico, que, adaptado a um tubo de borracha, em communicação com um depósito de água, serve para rega.
Uma das insígnias de general.
Bombeiro, que trabalha com as agulhetas de incêndio.
\section{Agulheteiro}
\begin{itemize}
\item {Grp. gram.:m.}
\end{itemize}
Fabricante de agulhas ou de agulhetas.
\section{Agurina}
\begin{itemize}
\item {Grp. gram.:f.}
\end{itemize}
Medicamento diurético.
\section{Agustina}
\begin{itemize}
\item {Grp. gram.:f.}
\end{itemize}
\begin{itemize}
\item {Proveniência:(De \textunderscore a\textunderscore  priv. e do lat. \textunderscore gustus\textunderscore )}
\end{itemize}
Certa terra da Saxónia, que se dizia sêr formada de saes insípidos.
\section{Aguti}
\begin{itemize}
\item {Grp. gram.:m.}
\end{itemize}
Mammífero americano, da ordem dos roedores.
\section{Agutiguepá}
\begin{itemize}
\item {Grp. gram.:f.}
\end{itemize}
Planta medicinal do Brasil.
\section{Agynário}
\begin{itemize}
\item {Grp. gram.:adj.}
\end{itemize}
\begin{itemize}
\item {Proveniência:(Do gr. \textunderscore a\textunderscore  priv. + \textunderscore gune\textunderscore )}
\end{itemize}
Diz-se, segundo De-Candolle, das flores, formadas pelos tegumentos floraes e estames transformados, e em que falta o pistillo.
\section{Agyniano}
\begin{itemize}
\item {Grp. gram.:m.}
\end{itemize}
\begin{itemize}
\item {Proveniência:(Do gr. \textunderscore a\textunderscore  priv. + \textunderscore gune\textunderscore )}
\end{itemize}
Membro de uma seita christan, que proscrevia o casamento.
\section{Agynico}
\begin{itemize}
\item {Grp. gram.:adj.}
\end{itemize}
\begin{itemize}
\item {Utilização:Bot.}
\end{itemize}
\begin{itemize}
\item {Proveniência:(Do gr. \textunderscore a\textunderscore  priv. + \textunderscore gune\textunderscore )}
\end{itemize}
Diz-se da inserção dos estames, quando estes não adherem ao ovário.
\section{Agyno}
\begin{itemize}
\item {Grp. gram.:adj.}
\end{itemize}
\begin{itemize}
\item {Proveniência:(Do gr. \textunderscore a\textunderscore priv. + \textunderscore gune\textunderscore )}
\end{itemize}
Que não tem órgãos femininos, (falando-se de vegetaes).
\section{Ah!}
\begin{itemize}
\item {Grp. gram.:interj.}
\end{itemize}
\begin{itemize}
\item {Proveniência:(T. commum a muitas línguas)}
\end{itemize}
(design. de admiração, dôr, alegria, etc.)
\section{Ah! ah! ah!}
\begin{itemize}
\item {Grp. gram.:interj.}
\end{itemize}
(correspondente ao riso franco, á gargalhada)
\section{Ahí}
\begin{itemize}
\item {Grp. gram.:adv.}
\end{itemize}
(Escrita usual, mas incorrecta. V. aí)
\section{Ahibi}
\begin{itemize}
\item {fónica:a-i}
\end{itemize}
\begin{itemize}
\item {Grp. gram.:m.}
\end{itemize}
\begin{itemize}
\item {Utilização:Bras}
\end{itemize}
Pequeno molusco comestível, de concha bivalve.
\section{Ahiva}
\begin{itemize}
\item {Grp. gram.:m. f.  e  adj.}
\end{itemize}
\begin{itemize}
\item {Utilização:Bras. do S}
\end{itemize}
Pessôa ou coisa sem valor, insignificante.
(Do tupi)
\section{Ahoai}
\begin{itemize}
\item {Grp. gram.:m.}
\end{itemize}
O mesmo que \textunderscore ahovai\textunderscore .
\section{Ahovai}
\begin{itemize}
\item {Grp. gram.:m.}
\end{itemize}
Planta solânea americana, de uma só fôlha.
\section{Ai!}
\begin{itemize}
\item {Grp. gram.:m.}
\end{itemize}
\begin{itemize}
\item {Grp. gram.:Interj.}
\end{itemize}
Grito de dôr, de alegria.
(design. de dôr ou de alegria)
Anel muito delgado.
\textunderscore Num ai\textunderscore , num instante:«\textunderscore chegariamos a Salvaterra num ai\textunderscore ». Corvo, \textunderscore Anno na Côrte\textunderscore , III, 53.
\section{Ai}
\begin{itemize}
\item {Grp. gram.:m.}
\end{itemize}
Quadrúpede, de marcha muito vagarosa.
\section{Ai}
\begin{itemize}
\item {Grp. gram.:m.}
\end{itemize}
\begin{itemize}
\item {Proveniência:(De \textunderscore Ai\textunderscore , n. p.)}
\end{itemize}
Espécie de vinho francês, no Marne.
\section{Aí}
\begin{itemize}
\item {Grp. gram.:adv.}
\end{itemize}
Nesse lugar.
(Da prep. \textunderscore a\textunderscore  e do adv. ant. \textunderscore i\textunderscore )
\section{Aia}
\begin{itemize}
\item {Grp. gram.:f.}
\end{itemize}
\begin{itemize}
\item {Proveniência:(De \textunderscore aio\textunderscore )}
\end{itemize}
Criada para companhia.
Criada de quarto; criada particular; camareira.
\section{Aiabeba}
\begin{itemize}
\item {Grp. gram.:f.}
\end{itemize}
\begin{itemize}
\item {Utilização:Ant.}
\end{itemize}
O mesmo que \textunderscore arrabil\textunderscore .
\section{Aiabutipita}
\begin{itemize}
\item {Grp. gram.:f.}
\end{itemize}
Árvore brasileira.
\section{Aiai}
\begin{itemize}
\item {Grp. gram.:m.}
\end{itemize}
\begin{itemize}
\item {Utilização:Fam.}
\end{itemize}
Chôro; lamento: \textunderscore fez-lhe cantar o aiai\textunderscore .«\textunderscore ...os seus aiais\textunderscore ». Castilho, \textunderscore Fausto\textunderscore , 140.
\section{Aiáia}
\begin{itemize}
\item {Grp. gram.:f.}
\end{itemize}
\begin{itemize}
\item {Utilização:Bras}
\end{itemize}
Brinquedo.
Vestido de criança.
\section{Aiapana}
\begin{itemize}
\item {Grp. gram.:f.}
\end{itemize}
\begin{itemize}
\item {Utilização:Bras}
\end{itemize}
Planta medicinal, que os Índios applicam contra o veneno das cobras.
\section{Aiapaina}
\begin{itemize}
\item {Grp. gram.:f.}
\end{itemize}
\begin{itemize}
\item {Utilização:Bras}
\end{itemize}
Planta medicinal, que os Índios applicam contra o veneno das cobras.
\section{Aiar}
\begin{itemize}
\item {Grp. gram.:v. i.}
\end{itemize}
\begin{itemize}
\item {Grp. gram.:M.}
\end{itemize}
Dar ais.
O acto de dar ais de agonia:«\textunderscore o aiar dos moribundos...\textunderscore »Castilho, \textunderscore Metam.\textunderscore , 237.
\section{Aibi}
\begin{itemize}
\item {fónica:a-i}
\end{itemize}
\begin{itemize}
\item {Grp. gram.:m.}
\end{itemize}
\begin{itemize}
\item {Utilização:Bras}
\end{itemize}
Pequeno molusco comestível, de concha bivalve.
\section{Aicuna!}
\begin{itemize}
\item {Grp. gram.:interj.}
\end{itemize}
\begin{itemize}
\item {Utilização:Bras. do S}
\end{itemize}
(Designa admiração, enthusiasmo)
\section{Aido}
\begin{itemize}
\item {Grp. gram.:m.}
\end{itemize}
O mesmo que \textunderscore eido\textunderscore .
\section{Aidro}
\begin{itemize}
\item {Grp. gram.:m.}
\end{itemize}
\begin{itemize}
\item {Utilização:Prov.}
\end{itemize}
\begin{itemize}
\item {Utilização:trasm.}
\end{itemize}
Átrio; entrada.
(Metáth. de \textunderscore ádrio\textunderscore , lat. \textunderscore atrium\textunderscore )
\section{Aigoto}
\begin{itemize}
\item {fónica:gô}
\end{itemize}
\begin{itemize}
\item {Grp. gram.:m.}
\end{itemize}
\begin{itemize}
\item {Utilização:Prov.}
\end{itemize}
\begin{itemize}
\item {Utilização:trasm.}
\end{itemize}
O filho da águia.
(Por \textunderscore aguioto\textunderscore , de \textunderscore águia\textunderscore . Cp. \textunderscore perdigoto\textunderscore )
\section{Ai-Jesus}
\begin{itemize}
\item {Grp. gram.:m.}
\end{itemize}
\begin{itemize}
\item {Grp. gram.:Interj.}
\end{itemize}
O mais querido, o predilecto.
(design. de dôr)
\section{Ailanthicultura}
\begin{itemize}
\item {Grp. gram.:f.}
\end{itemize}
\begin{itemize}
\item {Proveniência:(De \textunderscore ailantho\textunderscore  + \textunderscore cultura\textunderscore )}
\end{itemize}
Cultura de ailantho, em cujas fôlhas se cria um bicho da seda.
\section{Ailanthina}
\begin{itemize}
\item {Grp. gram.:f.}
\end{itemize}
Matéria têxtil, fornecida pelo bicho da seda que se cria no ailantho.
\section{Ailantho}
\begin{itemize}
\item {Grp. gram.:m.}
\end{itemize}
Árvore terebinthácea, de que se extrai o chamado verniz do Japão.
\section{Ailanticultura}
\begin{itemize}
\item {Grp. gram.:f.}
\end{itemize}
\begin{itemize}
\item {Proveniência:(De \textunderscore ailantho\textunderscore  + \textunderscore cultura\textunderscore )}
\end{itemize}
Cultura de ailanto, em cujas fôlhas se cria um bicho da seda.
\section{Ailantina}
\begin{itemize}
\item {Grp. gram.:f.}
\end{itemize}
Matéria têxtil, fornecida pelo bicho da seda que se cria no ailanto.
\section{Ailanto}
\begin{itemize}
\item {Grp. gram.:m.}
\end{itemize}
Árvore terebinthácea, de que se extrai o chamado verniz do Japão.
\section{Ailila}
\begin{itemize}
\item {Grp. gram.:m.}
\end{itemize}
\begin{itemize}
\item {Utilização:T. de Moncorvo}
\end{itemize}
Janota.
Pedante.
\section{Aimará}
\begin{itemize}
\item {Grp. gram.:m.}
\end{itemize}
Língua dos indígenas da América, entre o Peru e a Bolívia.
\section{Aimbirés}
\begin{itemize}
\item {Grp. gram.:m.}
\end{itemize}
O mesmo que \textunderscore aimborés\textunderscore .
\section{Aimborés}
\begin{itemize}
\item {Grp. gram.:m. pl.}
\end{itemize}
Selvagens do Brasil, que habitavam nas serranias entre as antigas províncias da Baía, Espírito-Santo e Rio-de-Janeiro.
O mesmo que \textunderscore botocudos\textunderscore , segundo Vivien de Saint-Martin.
\section{Aimorés}
\begin{itemize}
\item {Grp. gram.:m. pl.}
\end{itemize}
Indígenas do Brasil, o mesmo que \textunderscore aimborés\textunderscore .
\section{Ainda}
\begin{itemize}
\item {Grp. gram.:adv.}
\end{itemize}
\begin{itemize}
\item {Proveniência:(Do lat. \textunderscore ab\textunderscore  + \textunderscore inde\textunderscore )}
\end{itemize}
Até agora: \textunderscore ainda não veio\textunderscore .
Até então: \textunderscore ainda não tinha vindo\textunderscore .
Além disso: \textunderscore e ainda outras coisas\textunderscore .
Apesar: \textunderscore ainda que jures não te creio\textunderscore .
\section{Ainhum}
\begin{itemize}
\item {Grp. gram.:m.}
\end{itemize}
\begin{itemize}
\item {Utilização:Bras}
\end{itemize}
Affecção, peculiar á raça negra, e caracterizada pela amputação espontânea de um ou mais dedos do pé.
\section{...ãins}
\begin{itemize}
\item {Grp. gram.:suf. pl.}
\end{itemize}
O mesmo que ...\textunderscore ães\textunderscore .
\section{Ainos}
\begin{itemize}
\item {Grp. gram.:m. pl.}
\end{itemize}
Tribo bárbara do Japão.
\section{Aio}
\begin{itemize}
\item {Grp. gram.:m.}
\end{itemize}
Aquelle que está encarregado de educar crianças illustres ou filhos de gente rica.
Criado grave.
Camareiro.
Escudeiro.
(Do vasc.)
\section{Aipim}
\begin{itemize}
\item {Grp. gram.:m.}
\end{itemize}
\begin{itemize}
\item {Proveniência:(De \textunderscore aipo\textunderscore ?)}
\end{itemize}
Planta brasileira.
Mandioca doce.
\section{Aipo}
\begin{itemize}
\item {Grp. gram.:m.}
\end{itemize}
\begin{itemize}
\item {Proveniência:(Lat. \textunderscore apium\textunderscore )}
\end{itemize}
Planta umbellífera, de applicação culinária.
\section{Aipo}
\begin{itemize}
\item {Grp. gram.:m.}
\end{itemize}
\begin{itemize}
\item {Utilização:T. da Bairrada}
\end{itemize}
O mesmo que \textunderscore apo\textunderscore ^2.
\section{Aira}
\begin{itemize}
\item {Grp. gram.:f.}
\end{itemize}
\begin{itemize}
\item {Proveniência:(Gr. \textunderscore aira\textunderscore , joio)}
\end{itemize}
Gênero de plantas gramíneas.
\section{Airado}
\begin{itemize}
\item {Grp. gram.:adj.}
\end{itemize}
\begin{itemize}
\item {Proveniência:(Do cast. \textunderscore aire\textunderscore )}
\end{itemize}
Aéreo. Desvairado.
Vadio; próprio de vadio: \textunderscore vida airada\textunderscore .
\section{Airão}
\begin{itemize}
\item {Grp. gram.:m.}
\end{itemize}
\begin{itemize}
\item {Proveniência:(Do lat. \textunderscore hirundo\textunderscore ?)}
\end{itemize}
Espécie de andorinha.
\section{Airão}
\begin{itemize}
\item {Grp. gram.:m.}
\end{itemize}
\begin{itemize}
\item {Utilização:Ant.}
\end{itemize}
Ramo de flôres artificiaes, formadas de pedras finas, para enfeite de toucados.
\section{Airar}
\begin{itemize}
\item {fónica:a-i}
\end{itemize}
\begin{itemize}
\item {Grp. gram.:v. i.}
\end{itemize}
\begin{itemize}
\item {Utilização:Ant.}
\end{itemize}
Olhar com ira; odiar.
\section{Aire}
\begin{itemize}
\item {Grp. gram.:m.}
\end{itemize}
Ave marítima da costa de Portugal.
\section{Aire}
\begin{itemize}
\item {Grp. gram.:m.}
\end{itemize}
\begin{itemize}
\item {Utilização:Ant.}
\end{itemize}
O mesmo que \textunderscore ária\textunderscore ^1.
\section{Airi}
\begin{itemize}
\item {Grp. gram.:m.}
\end{itemize}
Espécie de coqueiro do Brasil.
\section{Airiri}
\begin{itemize}
\item {Grp. gram.:m.}
\end{itemize}
Espécie de coqueiro do Brasil.
\section{Airitucum}
\begin{itemize}
\item {Grp. gram.:m.}
\end{itemize}
Linha, que se faz de filamentos de airi, para fabricação de redes.
\section{Airo}
\begin{itemize}
\item {Grp. gram.:m.}
\end{itemize}
Ave aquática, (\textunderscore uria troile\textunderscore , Lin.).
O mesmo que \textunderscore aire\textunderscore ^1?
\section{Airol}
\begin{itemize}
\item {Grp. gram.:adj.}
\end{itemize}
\begin{itemize}
\item {Utilização:Chím.}
\end{itemize}
Substância sólida, derivada do dermatol e usada como antiséptico.
\section{Airosamente}
\begin{itemize}
\item {Grp. gram.:adv.}
\end{itemize}
De modo \textunderscore airoso\textunderscore .
Com donaire.
\section{Airosia}
\begin{itemize}
\item {Grp. gram.:f.}
\end{itemize}
\begin{itemize}
\item {Utilização:Prov.}
\end{itemize}
\begin{itemize}
\item {Utilização:minh.}
\end{itemize}
Aspecto airoso, agradável.
\section{Airosidade}
\begin{itemize}
\item {Grp. gram.:f.}
\end{itemize}
Gentileza; qualidade do que é \textunderscore airoso\textunderscore .
\section{Airoso}
\begin{itemize}
\item {Grp. gram.:adj.}
\end{itemize}
\begin{itemize}
\item {Proveniência:(Do cast. \textunderscore aire\textunderscore )}
\end{itemize}
Elegante, gentil, esbelto, garboso.
\section{Ais}
\textunderscore m. pl.\textunderscore  (?)«\textunderscore ...Cupido, cercado de huma rosa de jacintos, com os ays da mesma flor por rayos\textunderscore ». Vieira, IV, 194.
\section{...ais}
\begin{itemize}
\item {Grp. gram.:suf. pl.}
\end{itemize}
O mesmo que \textunderscore ...aes\textunderscore .
\section{...ais}
\begin{itemize}
\item {Grp. gram.:suf.}
\end{itemize}
\textunderscore pl.\textunderscore  de \textunderscore ...al\textunderscore .
\section{...ãis}
\begin{itemize}
\item {Grp. gram.:suf. pl.}
\end{itemize}
O mesmo que \textunderscore ...ães\textunderscore .
\section{...ãis}
\begin{itemize}
\item {Grp. gram.:suf.}
\end{itemize}
\textunderscore pl.\textunderscore  de vários subst. e adj. terminados em \textunderscore ão\textunderscore .
\section{Aissana}
\begin{itemize}
\item {Grp. gram.:m.}
\end{itemize}
Homem incrédulo?«\textunderscore Assim como os fanáticos e os aissanas se excitam mutuamente...\textunderscore »Camillo, \textunderscore Freira no Subt.\textunderscore , 109.
\section{Aitao}
\begin{itemize}
\item {Grp. gram.:m.}
\end{itemize}
O mesmo ou melhor que \textunderscore aitão\textunderscore . Cf. \textunderscore Peregrinação\textunderscore , LXXXV e LXXXVI.
\section{Aitão}
\begin{itemize}
\item {Grp. gram.:m.}
\end{itemize}
Antigo magistrado chinês, que parece superintendia em negócios commerciaes.
\section{Aito}
\begin{itemize}
\item {Grp. gram.:m.}
\end{itemize}
\begin{itemize}
\item {Utilização:Ant.}
\end{itemize}
O mesmo que \textunderscore auto\textunderscore .
\section{Aitona!}
\begin{itemize}
\item {Grp. gram.:interj.}
\end{itemize}
(Design. de alegria):«\textunderscore aitona! vai haver aqui partida rija\textunderscore ». Camillo, \textunderscore Corja\textunderscore , 166.
\section{Aitónia}
\begin{itemize}
\item {Grp. gram.:f.}
\end{itemize}
Gênero de plantas sapindáceas.
\section{Aiua!}
\begin{itemize}
\item {Grp. gram.:interj.}
\end{itemize}
\begin{itemize}
\item {Utilização:Bras}
\end{itemize}
(Designa gracejo ou mofa)
\section{Aiuê!}
\begin{itemize}
\item {Grp. gram.:interj.}
\end{itemize}
\begin{itemize}
\item {Utilização:Bras}
\end{itemize}
(Designa gracejo ou mofa)
\section{Aiunar}
\begin{itemize}
\item {Grp. gram.:v. i.}
\end{itemize}
\begin{itemize}
\item {Utilização:T. de Miranda}
\end{itemize}
O mesmo que \textunderscore jejuar\textunderscore .
\section{Aíva}
\begin{itemize}
\item {Grp. gram.:m. f.  e  adj.}
\end{itemize}
\begin{itemize}
\item {Utilização:Bras. do S}
\end{itemize}
Pessôa ou coisa sem valor, insignificante.
(Do tupi)
\section{Aivaca}
\begin{itemize}
\item {Grp. gram.:f.}
\end{itemize}
O mesmo que \textunderscore aiveca\textunderscore .
\section{Aivão}
\begin{itemize}
\item {Grp. gram.:m.}
\end{itemize}
Faisão ordinário.
Espécie de andorinha.
\section{Aiveca}
\begin{itemize}
\item {Grp. gram.:f.}
\end{itemize}
Cada uma das duas peças arqueadas, que, na charrua, erguem a terra que se lavra.
\section{Aixe}
\begin{itemize}
\item {Grp. gram.:m.}
\end{itemize}
O mesmo que \textunderscore axe\textunderscore ^1.
\section{Aizôa}
\begin{itemize}
\item {Grp. gram.:f.}
\end{itemize}
Gênero de plantas.
\section{Ajadas}
\begin{itemize}
\item {Grp. gram.:f. pl.}
\end{itemize}
\begin{itemize}
\item {Utilização:Ant.}
\end{itemize}
Dádiva do vassallo ao senhor feudal, em occasiões de festa.
(Por \textunderscore hajadas\textunderscore , de \textunderscore haja\textunderscore , de \textunderscore haver\textunderscore )
\section{Ajaezadamente}
\begin{itemize}
\item {fónica:ja-e}
\end{itemize}
\begin{itemize}
\item {Grp. gram.:adv.}
\end{itemize}
De modo \textunderscore ajaezado\textunderscore .
Á maneira de jaêzes.
\section{Ajaezado}
\begin{itemize}
\item {fónica:ja-e}
\end{itemize}
\begin{itemize}
\item {Grp. gram.:adv.}
\end{itemize}
\begin{itemize}
\item {Proveniência:(De \textunderscore ajaezar\textunderscore )}
\end{itemize}
Que tem jaêzes.
Enfeitado.
\section{Ajaezar}
\begin{itemize}
\item {fónica:ja-e}
\end{itemize}
\begin{itemize}
\item {Grp. gram.:v. t.}
\end{itemize}
\begin{itemize}
\item {Proveniência:(De \textunderscore jaez\textunderscore )}
\end{itemize}
Ornar de jaêzes.
Enfeitar.
\section{Ajaja}
\begin{itemize}
\item {Grp. gram.:f.}
\end{itemize}
\begin{itemize}
\item {Utilização:Açor}
\end{itemize}
Buraco, na quilha dos barcos de pesca, pelo qual cai a água, com que êlles se lavam.
\section{Ajanas}
\begin{itemize}
\item {Grp. gram.:m. pl.}
\end{itemize}
Povo indi*gena das costas de Moçambique.
\section{Ajanotadamente}
\begin{itemize}
\item {Grp. gram.:adv.}
\end{itemize}
Á maneira de janota; como os janotas.
\section{Ajanotado}
\begin{itemize}
\item {Grp. gram.:adj.}
\end{itemize}
Que tem disposições para janota.
Que se veste ou se apresenta semelhando um \textunderscore janota\textunderscore .
\section{Ajanotar-se}
\begin{itemize}
\item {Grp. gram.:v. p.}
\end{itemize}
Tornar-se \textunderscore janota\textunderscore .
\section{Ajans}
\begin{itemize}
\item {Grp. gram.:m. pl.}
\end{itemize}
O mesmo que \textunderscore ajanas\textunderscore .
\section{Ajantarado}
\begin{itemize}
\item {Grp. gram.:adj.}
\end{itemize}
Semelhante a um jantar; abundante como um jantar: \textunderscore almôço ajantarado\textunderscore .
\section{Ajardinado}
\begin{itemize}
\item {Grp. gram.:adj.}
\end{itemize}
Convertido em jardim: \textunderscore terreno ajardinado\textunderscore .
\section{Ajardinar}
\begin{itemize}
\item {Grp. gram.:v. t.}
\end{itemize}
Dar fórma de jardim a; converter em jardim: \textunderscore ajardinar um terreno\textunderscore .
\section{Ajarobá}
\begin{itemize}
\item {Grp. gram.:m.}
\end{itemize}
Peixe do Brasil.
\section{Ajeitadamente}
\begin{itemize}
\item {Grp. gram.:adv.}
\end{itemize}
Com jeito.
Acommodadamente.
\section{Ajeitamento}
\begin{itemize}
\item {Grp. gram.:m.}
\end{itemize}
Acto de \textunderscore ajeitar\textunderscore . Cf. F. Lapa, \textunderscore Proc. de Vin.\textunderscore , 33.
\section{Ajeitar}
\begin{itemize}
\item {Grp. gram.:v. t.}
\end{itemize}
Pôr a jeito; acommodar.
\section{Ajeitivar}
\begin{itemize}
\item {Grp. gram.:v. t.}
\end{itemize}
\begin{itemize}
\item {Utilização:Des.}
\end{itemize}
Pôr a seu jeito.
O mesmo que \textunderscore ajeitar\textunderscore .
\section{Ajenil}
\begin{itemize}
\item {Grp. gram.:m.}
\end{itemize}
Peixe das costas do Algarve.
\section{Ajetivar}
\begin{itemize}
\item {fónica:jé}
\end{itemize}
\begin{itemize}
\item {Grp. gram.:v. t.}
\end{itemize}
\begin{itemize}
\item {Utilização:Prov.}
\end{itemize}
Ajeitar, aperfeiçoar.
(Colhido em Turquel)
\section{Ajicubo}
\begin{itemize}
\item {Grp. gram.:m.}
\end{itemize}
Arbusto do Japão.
\section{Ajimez}
\begin{itemize}
\item {Grp. gram.:m.}
\end{itemize}
Janela arqueada superiormente, bipartida por um columnelo central e vertical.
(Cast. \textunderscore ajimez\textunderscore )
\section{Ajipas}
\begin{itemize}
\item {Grp. gram.:f. pl.}
\end{itemize}
Tubérculos, que se apanham na Bolívia e que são semelhantes aos da dhália.
\section{Ajo}
\begin{itemize}
\item {Utilização:Pop.}
\end{itemize}
\begin{itemize}
\item {Proveniência:(De \textunderscore haja\textunderscore , de \textunderscore haver\textunderscore ?)}
\end{itemize}
Estado, situação: \textunderscore eu no teu ajo, repellia a proposta\textunderscore .
\section{Ajoelhação}
\begin{itemize}
\item {fónica:jo-e}
\end{itemize}
\begin{itemize}
\item {Grp. gram.:f.}
\end{itemize}
Acção de \textunderscore ajoelhar\textunderscore .
\section{Ajoelhada}
\begin{itemize}
\item {Grp. gram.:f.}
\end{itemize}
Planta gramínea.
\section{Ajoelhar}
\begin{itemize}
\item {fónica:jo-e}
\end{itemize}
\begin{itemize}
\item {Grp. gram.:v. t.}
\end{itemize}
\begin{itemize}
\item {Grp. gram.:V. i.}
\end{itemize}
Fazer dobrar os joêlhos.
Pôr o joêlho ou os joêlhos no chão.
\section{Ajorcado}
\begin{itemize}
\item {Grp. gram.:adj.}
\end{itemize}
\begin{itemize}
\item {Utilização:Ant.}
\end{itemize}
Ataviado; loução.
(Por \textunderscore axorcado\textunderscore , de \textunderscore axorca\textunderscore )
\section{Ajornalar}
\begin{itemize}
\item {Grp. gram.:v. t.}
\end{itemize}
Tomar para serviço a jornal.
\section{Ajoujador}
\begin{itemize}
\item {Grp. gram.:v. t.}
\end{itemize}
Que ajouja. Cf. Filinto, I, 85.
\section{Ajoujamento}
\begin{itemize}
\item {Grp. gram.:m.}
\end{itemize}
Acto de \textunderscore ajoujar\textunderscore .
\section{Ajoujante}
\begin{itemize}
\item {Grp. gram.:adj.}
\end{itemize}
\begin{itemize}
\item {Utilização:Bras}
\end{itemize}
O mesmo que \textunderscore ajoujador\textunderscore .
\section{Ajoujar}
\begin{itemize}
\item {Grp. gram.:v. t.}
\end{itemize}
Ligar com ajoujo.
Carregar.
Opprimir.
\section{Ajoujo}
\begin{itemize}
\item {Grp. gram.:m.}
\end{itemize}
\begin{itemize}
\item {Utilização:Bras}
\end{itemize}
Cordão ou corrente, com que se prendem ou jungem dois animaes pelo pescoço.
União violenta.
Espécie de barca, formada por duas canôas emparelhadas.
\section{Ajóvea}
\begin{itemize}
\item {Grp. gram.:f.}
\end{itemize}
Árvore da Guiana, (\textunderscore laurus hexandra\textunderscore , Swertz).
\section{Ajoviar}
\begin{itemize}
\item {Grp. gram.:v. t.}
\end{itemize}
\begin{itemize}
\item {Utilização:Ant.}
\end{itemize}
\begin{itemize}
\item {Grp. gram.:V. i.}
\end{itemize}
Causar assombro a.
Ficar assombrado.
\section{Ajuaga}
\begin{itemize}
\item {Grp. gram.:f.}
\end{itemize}
Tumor nos cascos das bêstas.
\section{Ajuda}
\begin{itemize}
\item {Grp. gram.:f.}
\end{itemize}
\begin{itemize}
\item {Grp. gram.:f.}
\end{itemize}
\begin{itemize}
\item {Utilização:Prov.}
\end{itemize}
\begin{itemize}
\item {Utilização:alent.}
\end{itemize}
Acto de \textunderscore ajudar\textunderscore ; auxílio.
Favor.
Clyster.
Espécie de tributo feudal.
O segundo pastor do rebanho, ou o pastor inferiormente immediato ao rabadão.
\section{Ajudada}
\begin{itemize}
\item {Grp. gram.:f.}
\end{itemize}
\begin{itemize}
\item {Utilização:Prov.}
\end{itemize}
\begin{itemize}
\item {Utilização:alg.}
\end{itemize}
\begin{itemize}
\item {Proveniência:(De \textunderscore ajudar\textunderscore )}
\end{itemize}
Auxílio, que a um agricultor dão muitos outros, em qualquer trabalho de campo.
\section{Ajudadeira}
\begin{itemize}
\item {Grp. gram.:f.}
\end{itemize}
\begin{itemize}
\item {Utilização:Ant.}
\end{itemize}
\begin{itemize}
\item {Proveniência:(De \textunderscore ajudar\textunderscore )}
\end{itemize}
Foro ou pensão, que onerava alguns prazos da sé de Viseu.
Tributo feudal, com que o vassallo auxiliava as despesas do príncipe ou do suzerano.
\section{Ajudadoiro}
\begin{itemize}
\item {Grp. gram.:m.}
\end{itemize}
O mesmo que \textunderscore ajudoiro\textunderscore .
\section{Ajudador}
\begin{itemize}
\item {Grp. gram.:m.}
\end{itemize}
O que ajuda.
\section{Ajudância}
\begin{itemize}
\item {Grp. gram.:f.}
\end{itemize}
\begin{itemize}
\item {Utilização:Bras}
\end{itemize}
Cargo de ajudante. Cf. \textunderscore Diário Official\textunderscore  de 15-VI-901.
\section{Ajudante}
\begin{itemize}
\item {Grp. gram.:m.}
\end{itemize}
\begin{itemize}
\item {Proveniência:(Do lat. \textunderscore adjuvans\textunderscore )}
\end{itemize}
Aquelle que ajuda.
\section{Ajudar}
\begin{itemize}
\item {Grp. gram.:v. t.}
\end{itemize}
\begin{itemize}
\item {Grp. gram.:V. p.}
\end{itemize}
\begin{itemize}
\item {Utilização:Prov.}
\end{itemize}
\begin{itemize}
\item {Utilização:alent.}
\end{itemize}
\begin{itemize}
\item {Proveniência:(Do lat. \textunderscore adjutare\textunderscore )}
\end{itemize}
Auxiliar; soccorrer.
Favorecer.
Poder alguém fazer, só por si, qualquer trabalho.
Aguentar certo pêso.
\section{Ajudeado}
\begin{itemize}
\item {Grp. gram.:adj.}
\end{itemize}
O mesmo que \textunderscore ajudengado\textunderscore .
\section{Ajudengado}
\begin{itemize}
\item {Grp. gram.:adj.}
\end{itemize}
\begin{itemize}
\item {Utilização:Des.}
\end{itemize}
Que tem modos de judeu ou de coisa judaica.
\section{Ajudoiro}
\begin{itemize}
\item {Grp. gram.:m.}
\end{itemize}
\begin{itemize}
\item {Utilização:Ant.}
\end{itemize}
\begin{itemize}
\item {Proveniência:(Do lat. \textunderscore adjutorium\textunderscore )}
\end{itemize}
Soccorro; auxílio.
\section{Ajuga}
\begin{itemize}
\item {Grp. gram.:m.}
\end{itemize}
Planta labiada, annual, que comprehende três secções, (\textunderscore ajugarepta\textunderscore , Lin.).
O mesmo que \textunderscore búgula\textunderscore .
\section{Ajúgeas}
\begin{itemize}
\item {Grp. gram.:f. pl.}
\end{itemize}
\begin{itemize}
\item {Proveniência:(De \textunderscore ajuga\textunderscore )}
\end{itemize}
Tríbo de plantas labiadas.
\section{Ajugoide}
\begin{itemize}
\item {Grp. gram.:adj.}
\end{itemize}
\begin{itemize}
\item {Proveniência:(De \textunderscore ajuga\textunderscore  + gr. \textunderscore eidos\textunderscore )}
\end{itemize}
Semelhante ao ajuga.
\section{Ajuizadamente}
\begin{itemize}
\item {fónica:ju-i}
\end{itemize}
\begin{itemize}
\item {Grp. gram.:adv.}
\end{itemize}
De modo \textunderscore ajuizado\textunderscore ; com tino.
\section{Ajuizado}
\begin{itemize}
\item {fónica:ju-i}
\end{itemize}
\begin{itemize}
\item {Grp. gram.:adj.}
\end{itemize}
\begin{itemize}
\item {Proveniência:(De \textunderscore ajuizar\textunderscore )}
\end{itemize}
Que mostra têr juizo.
Atilado.
\section{Ajuizador}
\begin{itemize}
\item {Grp. gram.:m.}
\end{itemize}
O que ajuiza.
\section{Ajuizar}
\begin{itemize}
\item {fónica:ju-i}
\end{itemize}
\begin{itemize}
\item {Grp. gram.:v. t.}
\end{itemize}
Formar juizo de: julgar.
Dar juizo a, tornar sensato.
Levar a juizo, ao tribunal; tornar objecto de processo, de demanda.
\section{Ajular}
\begin{itemize}
\item {Grp. gram.:v. t.}
\end{itemize}
\begin{itemize}
\item {Utilização:Bras}
\end{itemize}
Sotaventear, impellir para julavento; lançar (o navio) para trás.
\section{Ajumentado}
\begin{itemize}
\item {Grp. gram.:adj.}
\end{itemize}
Que tem apparência ou modos de jumento.
\section{Ajunta}
\begin{itemize}
\item {Grp. gram.:f.}
\end{itemize}
\begin{itemize}
\item {Utilização:Prov.}
\end{itemize}
\begin{itemize}
\item {Utilização:beir.}
\end{itemize}
Acto de \textunderscore ajuntar\textunderscore .
\textunderscore Pão de ajunta\textunderscore , pão de milho, com mistura de algum trigo.
\section{Ajuntadamente}
\begin{itemize}
\item {Grp. gram.:adv.}
\end{itemize}
O mesmo que \textunderscore juntamente\textunderscore .
\section{Ajuntadeira}
\begin{itemize}
\item {Grp. gram.:f.}
\end{itemize}
\begin{itemize}
\item {Proveniência:(De \textunderscore ajuntar\textunderscore )}
\end{itemize}
Mulher, que junta e cose as peças superiores do calçado.
\section{Ajuntadoiro}
\begin{itemize}
\item {Grp. gram.:m.}
\end{itemize}
\begin{itemize}
\item {Proveniência:(De \textunderscore ajuntar\textunderscore )}
\end{itemize}
Lugar, onde se juntam águas pluviaes, ou outras coisas.
\section{Ajuntador}
\begin{itemize}
\item {Grp. gram.:m.}
\end{itemize}
O que ajunta.
\section{Ajuntadouro}
\begin{itemize}
\item {Grp. gram.:m.}
\end{itemize}
\begin{itemize}
\item {Proveniência:(De \textunderscore ajuntar\textunderscore )}
\end{itemize}
Lugar, onde se juntam águas pluviaes, ou outras coisas.
\section{Ajuntamento}
\begin{itemize}
\item {Grp. gram.:m.}
\end{itemize}
Reunião de pessôas.
Acto de \textunderscore ajuntar\textunderscore .
\section{Ajuntar}
\begin{itemize}
\item {Grp. gram.:v. t.}
\end{itemize}
\begin{itemize}
\item {Grp. gram.:v. t.}
\end{itemize}
\begin{itemize}
\item {Utilização:Bras. do N}
\end{itemize}
\begin{itemize}
\item {Proveniência:(De \textunderscore juntar\textunderscore )}
\end{itemize}
Aproximar.
Unir.
Accrescentar.
Colligir.
Economizar (dinheiro, quantias)
Apanhar, levantar.
\section{Ajuntável}
\begin{itemize}
\item {Grp. gram.:adj.}
\end{itemize}
Que se póde \textunderscore ajuntar\textunderscore .
\section{Ajuntoira}
\begin{itemize}
\item {Grp. gram.:f.}
\end{itemize}
\begin{itemize}
\item {Proveniência:(De \textunderscore ajuntar\textunderscore )}
\end{itemize}
Pedra, que atravessa uma parede em toda a sua espessura.
\section{Ajuramentadamente}
\begin{itemize}
\item {Grp. gram.:adv.}
\end{itemize}
Com juramento.
\section{Ajuramentar}
\begin{itemize}
\item {Grp. gram.:v. t.}
\end{itemize}
Fazer jurar; deferir juramento a.
\section{Ajurativa}
\begin{itemize}
\item {Grp. gram.:f.}
\end{itemize}
Arbusto brasileiro.
\section{Ajuru}
\begin{itemize}
\item {Grp. gram.:m.}
\end{itemize}
Árvore fructífera do Brasil.
\section{Ajuru}
\begin{itemize}
\item {Grp. gram.:m.}
\end{itemize}
\begin{itemize}
\item {Utilização:Bras}
\end{itemize}
Designação genérica do papagaio.
(Do tupi \textunderscore a\textunderscore , gente, e \textunderscore juru\textunderscore , boca)
\section{Ajurujuru}
\begin{itemize}
\item {Grp. gram.:m.}
\end{itemize}
\begin{itemize}
\item {Utilização:Bras}
\end{itemize}
Designação genérica do papagaio.
(Do tupi \textunderscore a\textunderscore , gente, e \textunderscore juru\textunderscore , boca)
\section{Ajuso}
\begin{itemize}
\item {Grp. gram.:adv.}
\end{itemize}
\begin{itemize}
\item {Utilização:Ant.}
\end{itemize}
\begin{itemize}
\item {Proveniência:(De \textunderscore juso\textunderscore )}
\end{itemize}
Abaixo.
\section{Ajustador}
\begin{itemize}
\item {Grp. gram.:m.}
\end{itemize}
\begin{itemize}
\item {Proveniência:(De \textunderscore ajuntar\textunderscore )}
\end{itemize}
Aquelle que, nos caminhos de ferro, é encarregado de collocar no respectivo lugar as peças de uma máquina, vagão ou carruagem.
\section{Ajustagem}
\textunderscore f. Bras.\textunderscore  (?)«\textunderscore ...officina de serralharia mecânica e ajustagem para servir de prompto ás pequenas reparações\textunderscore  (náuticas)». Do \textunderscore Jornal de Notícias\textunderscore , do Rio, de 9-IX-907.
\section{Ajustamento}
\begin{itemize}
\item {Grp. gram.:m.}
\end{itemize}
Acto ou effeito de \textunderscore ajustar\textunderscore .
\section{Ajustar}
\begin{itemize}
\item {Grp. gram.:v. t.}
\end{itemize}
\begin{itemize}
\item {Proveniência:(De \textunderscore justo\textunderscore )}
\end{itemize}
Adaptar.
Tornar exacto, justo.
Contratar.
Completar.
Collocar no respectivo lugar (as peças de uma máquina, vagão ou carruagem).
\section{Ajuste}
\begin{itemize}
\item {Grp. gram.:m.}
\end{itemize}
Acto de \textunderscore ajustar\textunderscore .
Contrato, convenção.
\section{Ajustiçar}
\begin{itemize}
\item {Grp. gram.:v. t.}
\end{itemize}
\begin{itemize}
\item {Utilização:Ant.}
\end{itemize}
\begin{itemize}
\item {Proveniência:(De \textunderscore justiça\textunderscore )}
\end{itemize}
Apresentar como justo.
Justiçar.
\section{Ajustura}
\begin{itemize}
\item {Grp. gram.:f.}
\end{itemize}
\begin{itemize}
\item {Proveniência:(Fr. \textunderscore ajusture\textunderscore )}
\end{itemize}
Pequena cavidade numa ferradura, para que esta se adapte facilmente ao pé.
\section{Ajutório}
\begin{itemize}
\item {Grp. gram.:m.}
\end{itemize}
\begin{itemize}
\item {Utilização:Bras}
\end{itemize}
O mesmo que \textunderscore muxirão\textunderscore , quando êste, passado um dia, entra pelo seguinte.
\section{Al}
\begin{itemize}
\item {Grp. gram.:pron.}
\end{itemize}
\begin{itemize}
\item {Utilização:Ant.}
\end{itemize}
Outra coisa.
O mais.
\section{Al}
\begin{itemize}
\item {Grp. gram.:pref.}
\end{itemize}
De origem árabe, que entrou na formação de palavras portuguesas, como \textunderscore Almada\textunderscore .
\section{Al}
(contr. ant. da prep. \textunderscore a\textunderscore  e do art. \textunderscore lo\textunderscore )
\section{...al}
\begin{itemize}
\item {Grp. gram.:suf.}
\end{itemize}
\begin{itemize}
\item {Proveniência:(Do suf. lat. \textunderscore ...alis\textunderscore )}
\end{itemize}
(design. de conveniência ou relação)
\section{Ala}
\begin{itemize}
\item {Grp. gram.:f.}
\end{itemize}
\begin{itemize}
\item {Utilização:Prov.}
\end{itemize}
\begin{itemize}
\item {Utilização:trasm.}
\end{itemize}
\begin{itemize}
\item {Utilização:Prov.}
\end{itemize}
\begin{itemize}
\item {Utilização:trasm.}
\end{itemize}
\begin{itemize}
\item {Utilização:Prov.}
\end{itemize}
\begin{itemize}
\item {Utilização:Mil.}
\end{itemize}
Fileira; renque.
Fachada lateral.
Asa
Pedra de loisa, com que se encimam os muros, para que as pedras miúdas se não derrubem facilmente.
O mesmo que \textunderscore labarêda\textunderscore .
Grupo militar, que fórma uma das divisões de um regimento; metade de um batalhão.
Cada uma das duas extremidades de um exército, formado em ordem de batalha.
\textunderscore Dar ala\textunderscore , dar occasião, ensejo:«\textunderscore os seus nervosismos dão ala ao espirito\textunderscore ». Camillo, \textunderscore Cartas\textunderscore , 132.
\section{Ala!}
\begin{itemize}
\item {Grp. gram.:interj.}
\end{itemize}
Eia! vamos! anda!
(Imp. do v. \textunderscore alar\textunderscore )
\section{Alá}
\begin{itemize}
\item {Grp. gram.:adv.}
\end{itemize}
\begin{itemize}
\item {Utilização:Ant.}
\end{itemize}
O mesmo que \textunderscore lá\textunderscore .
\section{Alabaça}
\begin{itemize}
\item {Grp. gram.:f.}
\end{itemize}
\begin{itemize}
\item {Utilização:Prov.}
\end{itemize}
\begin{itemize}
\item {Utilização:dur.}
\end{itemize}
Pedaço de tábua, com que se veda o rombo de um navio.
\section{Alabancioso}
\begin{itemize}
\item {Grp. gram.:adj.}
\end{itemize}
\begin{itemize}
\item {Utilização:Des.}
\end{itemize}
Que costuma \textunderscore alabar-se\textunderscore .
\section{Alabanda}
\begin{itemize}
\item {Grp. gram.:f.}
\end{itemize}
\begin{itemize}
\item {Proveniência:(De \textunderscore Alabanda\textunderscore , n. p.)}
\end{itemize}
Mármore negro, encontrado por Plínio na cidade daquelle nome.
\section{Alabandina}
\begin{itemize}
\item {Grp. gram.:f.}
\end{itemize}
\begin{itemize}
\item {Proveniência:(De \textunderscore alabanda\textunderscore )}
\end{itemize}
Pedra vermelho-escura, que é um sulfureto de manganés.
\section{Alabão}
\begin{itemize}
\item {Grp. gram.:m.}
\end{itemize}
\begin{itemize}
\item {Utilização:Prov.}
\end{itemize}
\begin{itemize}
\item {Utilização:alent.}
\end{itemize}
\begin{itemize}
\item {Proveniência:(Do ár. \textunderscore al-laban\textunderscore )}
\end{itemize}
Rebanho, que dá leite por meio da ordenha.
\section{Alabarar}
\begin{itemize}
\item {Grp. gram.:v. t.}
\end{itemize}
Denegrir com o fogo.
Queimar.
Consumir.
(Relaciona-se com \textunderscore labareda\textunderscore ?)
\section{Alabarca}
\begin{itemize}
\item {Grp. gram.:m.}
\end{itemize}
\begin{itemize}
\item {Proveniência:(Lat. \textunderscore alabarches\textunderscore )}
\end{itemize}
Recebedor dos direitos de importação, entre os antigos Romanos.
\section{Alabarca}
\begin{itemize}
\item {Utilização:Prov.}
\end{itemize}
O mesmo que \textunderscore abarca\textunderscore .
\section{Alabarda}
\begin{itemize}
\item {Grp. gram.:f.}
\end{itemize}
Arma, composta de longa haste, que termina em ferro largo e ponteagudo, atravessado por outro ferro em fórma de meia lua.
(Do alto al. méd. \textunderscore hëlmbart\textunderscore )
\section{Alabardada}
\begin{itemize}
\item {Grp. gram.:f.}
\end{itemize}
Golpe de alabarda.
\section{Alabardado}
\begin{itemize}
\item {Grp. gram.:adj.}
\end{itemize}
Armado de alabarda.
\section{Alabardar}
\begin{itemize}
\item {Grp. gram.:v. t.}
\end{itemize}
\begin{itemize}
\item {Grp. gram.:V. p.}
\end{itemize}
\begin{itemize}
\item {Utilização:Des.}
\end{itemize}
Armar de alabarda.
Jactanciar-se; gabar-se, alabar-se.
\section{Alabardeiras}
\begin{itemize}
\item {Grp. gram.:f. pl.}
\end{itemize}
\begin{itemize}
\item {Utilização:Prov.}
\end{itemize}
\begin{itemize}
\item {Utilização:minh.}
\end{itemize}
Espécie de tamancos, usados pelas mulheres de Castro-Laboreiro.
\section{Alabardeiro}
\begin{itemize}
\item {Grp. gram.:m.}
\end{itemize}
Aquelle que usa alabarda.
\section{Alabardino}
\begin{itemize}
\item {Grp. gram.:adj.}
\end{itemize}
Que tem feitio de alabarda.
\section{Alabarque}
\begin{itemize}
\item {Grp. gram.:m.}
\end{itemize}
\begin{itemize}
\item {Proveniência:(Lat. \textunderscore alabarches\textunderscore )}
\end{itemize}
Recebedor dos direitos de importação, entre os antigos Romanos.
\section{Alabar-se}
\begin{itemize}
\item {Grp. gram.:v. p.}
\end{itemize}
Têr bazófia; jactanciar-se.
(Cast. \textunderscore alabar\textunderscore )
\section{Alabástrico}
\begin{itemize}
\item {Grp. gram.:adj.}
\end{itemize}
O mesmo que \textunderscore alabastrino\textunderscore .
\section{Alabastrilha}
\begin{itemize}
\item {Grp. gram.:f.}
\end{itemize}
Instrumento náutico, o mesmo que \textunderscore balestilha\textunderscore .
\section{Alabastrino}
\begin{itemize}
\item {Grp. gram.:adj.}
\end{itemize}
Que tem a cór ou outras propriedades do alabastro.
\section{Alabastrita}
\begin{itemize}
\item {Grp. gram.:f.}
\end{itemize}
O mesmo que \textunderscore alabastrite\textunderscore .
\section{Alabastrite}
\begin{itemize}
\item {Grp. gram.:f.}
\end{itemize}
\begin{itemize}
\item {Proveniência:(Gr. \textunderscore alabastrites\textunderscore )}
\end{itemize}
Variedade do sulfato de cal, semelhante ao alabastro.
\section{Alabastro}
\begin{itemize}
\item {Grp. gram.:m.}
\end{itemize}
\begin{itemize}
\item {Utilização:Fig.}
\end{itemize}
\begin{itemize}
\item {Proveniência:(Lat. \textunderscore alabastrum\textunderscore )}
\end{itemize}
Espécie de mármore branco e pouco duro.
Alvura.
\section{Alabirintado}
\begin{itemize}
\item {Grp. gram.:adj.}
\end{itemize}
\begin{itemize}
\item {Utilização:Fig.}
\end{itemize}
Que tem fórma de labirinto.
Confuso. Cf. Camillo, \textunderscore Mulher Fatal\textunderscore , 10.
\section{Alabirintar}
\begin{itemize}
\item {Grp. gram.:v. t.}
\end{itemize}
Dar fórma de labirinto a.
Delinear complicadamente, com sinuosidades.
\section{Alabregado}
\begin{itemize}
\item {Grp. gram.:adj.}
\end{itemize}
Que tem modos de labrego.
\section{Alabyrinthado}
\begin{itemize}
\item {Grp. gram.:adj.}
\end{itemize}
\begin{itemize}
\item {Utilização:Fig.}
\end{itemize}
Que tem fórma de labyrintho.
Confuso. Cf. Camillo, \textunderscore Mulher Fatal\textunderscore , 10.
\section{Alabyrinthar}
\begin{itemize}
\item {Grp. gram.:v. t.}
\end{itemize}
Dar fórma de labyrintho a.
Delinear complicadamente, com sinuosidades.
\section{Alacaiado}
\begin{itemize}
\item {Grp. gram.:adj.}
\end{itemize}
Que tem modos de lacaio.
\section{Alácar}
\begin{itemize}
\item {Grp. gram.:m.}
\end{itemize}
\begin{itemize}
\item {Utilização:Des.}
\end{itemize}
O mesmo que \textunderscore lacre\textunderscore .
\section{Alacar}
\begin{itemize}
\item {Grp. gram.:v. i.}
\end{itemize}
\begin{itemize}
\item {Utilização:Pop.}
\end{itemize}
Vergar, ceder ao pêso ou carga.
\section{Alacoado}
\begin{itemize}
\item {Grp. gram.:adj.}
\end{itemize}
Que tem côr de lacão ou de presunto.
\section{Alacrã}
\begin{itemize}
\item {Grp. gram.:f.}
\end{itemize}
O mesmo que \textunderscore lacrau\textunderscore .
\section{Alacraia}
\begin{itemize}
\item {Grp. gram.:f.}
\end{itemize}
\begin{itemize}
\item {Utilização:Prov.}
\end{itemize}
\begin{itemize}
\item {Utilização:trasm.}
\end{itemize}
O mesmo que \textunderscore alacran\textunderscore .
\section{Alacran}
\begin{itemize}
\item {Grp. gram.:f.}
\end{itemize}
O mesmo que \textunderscore lacrau\textunderscore .
\section{Alacrar}
\begin{itemize}
\item {Grp. gram.:v. i.}
\end{itemize}
\begin{itemize}
\item {Utilização:Prov.}
\end{itemize}
\begin{itemize}
\item {Utilização:minh.}
\end{itemize}
Entanguir-se; não se desenvolver devidamente na parte superior, (falando-se da espiga do milho).
\section{Alacrau}
\begin{itemize}
\item {Grp. gram.:m.}
\end{itemize}
(V.lacrau)
\section{Álacre}
\begin{itemize}
\item {Grp. gram.:adj.}
\end{itemize}
\begin{itemize}
\item {Proveniência:(Lat. \textunderscore alacris\textunderscore )}
\end{itemize}
Enthusiasta, vivo, esperto.
Que tem côr ou aspecto alegre.
\section{Alacreado}
\begin{itemize}
\item {Grp. gram.:adj.}
\end{itemize}
Que tem côr de lacre.
\section{Alacridade}
\begin{itemize}
\item {Grp. gram.:f.}
\end{itemize}
\begin{itemize}
\item {Proveniência:(Lat. \textunderscore alacritas\textunderscore )}
\end{itemize}
Vigor.
Alegria.
Enthusiasmo.
\section{Alado}
\begin{itemize}
\item {Grp. gram.:adj.}
\end{itemize}
\begin{itemize}
\item {Proveniência:(De \textunderscore ala\textunderscore )}
\end{itemize}
Que tem asas.
\section{Aladroado}
\begin{itemize}
\item {Grp. gram.:adj.}
\end{itemize}
Que tem tendência para ladrão.
Que furta alguma coisa.
Cerceado ou deminuido com fraude:«\textunderscore o Carneiro tem os pesos muito aladroados\textunderscore ». Macedo, \textunderscore Burros\textunderscore , 225.
\section{Aladroar}
\begin{itemize}
\item {Grp. gram.:v. t.}
\end{itemize}
Cercear ou deminuir com fraude.
\section{Ala-e-larga!}
\begin{itemize}
\item {Grp. gram.:interj.}
\end{itemize}
\begin{itemize}
\item {Utilização:Náut.}
\end{itemize}
Voz de manobra, para virar de bordo, ou para se dar uma volta com a embarcação, ao atracar a um caes ou a um navio.
\section{Alafé}
\begin{itemize}
\item {Grp. gram.:adj.}
\end{itemize}
\begin{itemize}
\item {Utilização:Ant.}
\end{itemize}
\begin{itemize}
\item {Proveniência:(De \textunderscore a\textunderscore  + \textunderscore la\textunderscore  + \textunderscore fé\textunderscore )}
\end{itemize}
Á fé; em verdade; certamente.
\section{Alagadeira}
\begin{itemize}
\item {Grp. gram.:f.}
\end{itemize}
\begin{itemize}
\item {Utilização:Des.}
\end{itemize}
\begin{itemize}
\item {Proveniência:(De \textunderscore alagar\textunderscore )}
\end{itemize}
Mulher gastadora.
\section{Alagadela}
\begin{itemize}
\item {Grp. gram.:f.}
\end{itemize}
Acto de \textunderscore alagar\textunderscore  ou encher de água. Cf. \textunderscore Museu Techn.\textunderscore , 90.
\section{Alagadiceiro}
\begin{itemize}
\item {Grp. gram.:adj.}
\end{itemize}
\begin{itemize}
\item {Utilização:Bras}
\end{itemize}
Que pasta em terreno alagadiço.
\section{Alagadiço}
\begin{itemize}
\item {Grp. gram.:adj.}
\end{itemize}
Sujeito a sêr alagado.
Pantanoso; encharcado.
\section{Alagador}
\begin{itemize}
\item {Grp. gram.:m.}
\end{itemize}
O que alaga.
\section{Alaela}
\begin{itemize}
\item {Grp. gram.:f.}
\end{itemize}
\begin{itemize}
\item {Proveniência:(Do ár. \textunderscore al-hila\textunderscore )}
\end{itemize}
Arraial moirisco.
\section{Alagamar}
\begin{itemize}
\item {Grp. gram.:m.}
\end{itemize}
\begin{itemize}
\item {Proveniência:(De \textunderscore alagar\textunderscore  + \textunderscore mar\textunderscore . Cp. \textunderscore lagamar\textunderscore )}
\end{itemize}
Pequenina angra, por natureza guarnecida de penedos, e onde entra a maré, perdida a sua maior violência.
\section{Alagamento}
\begin{itemize}
\item {Grp. gram.:m.}
\end{itemize}
Acto de \textunderscore alagar\textunderscore .
\section{Alagar}
\begin{itemize}
\item {Grp. gram.:v. t.}
\end{itemize}
\begin{itemize}
\item {Utilização:Ant.}
\end{itemize}
Converter em lago.
Cobrir de água; inundar.
Encher ou cobrir de qualquer líquido.
Invadir.
Deitar ao chão.
Subverter.
Dissipar.
\section{Alagartado}
\begin{itemize}
\item {Grp. gram.:adj.}
\end{itemize}
Que tem côr de lagarto.
\section{Alagartar}
\begin{itemize}
\item {Grp. gram.:v. t.}
\end{itemize}
Dar côr de lagarto a. Cf. Filinto, III, 211.
\section{Alagartear}
\begin{itemize}
\item {Grp. gram.:v. t.}
\end{itemize}
Pôr traços ou manchas em, á feição de lagartas. Cf. Filinto, IV, 163.
\section{Alagem}
\begin{itemize}
\item {Grp. gram.:f.}
\end{itemize}
Acto de \textunderscore alar\textunderscore .
\section{Alagôa}
\begin{itemize}
\item {Grp. gram.:adj.}
\end{itemize}
(V.lagôa)
\section{Alagoano}
\begin{itemize}
\item {Grp. gram.:f.}
\end{itemize}
\begin{itemize}
\item {Grp. gram.:M.}
\end{itemize}
Relativo ao estado de Alagôas, no Brasil.
Aquelle que é natural de Alagôas.
\section{Alagoar}
\begin{itemize}
\item {Grp. gram.:v. t.}
\end{itemize}
\begin{itemize}
\item {Proveniência:(De \textunderscore alagôa\textunderscore )}
\end{itemize}
Encher ou inundar em fórma de logo: \textunderscore a maré, quando sobe alagôa estas covas\textunderscore .
\section{Alagoso}
\begin{itemize}
\item {Grp. gram.:adj.}
\end{itemize}
\begin{itemize}
\item {Proveniência:(De \textunderscore alagar\textunderscore )}
\end{itemize}
Diz-se do terreno paludoso ou alagado de água.
\section{Alagosta}
\begin{itemize}
\item {fónica:gôs}
\end{itemize}
\begin{itemize}
\item {Grp. gram.:f.}
\end{itemize}
\begin{itemize}
\item {Utilização:Prov.}
\end{itemize}
\begin{itemize}
\item {Utilização:trasm.}
\end{itemize}
\begin{itemize}
\item {Proveniência:(De \textunderscore alagar\textunderscore , e infl. de \textunderscore lagosta\textunderscore ?)}
\end{itemize}
Mulher desgovernada, muito gastadora.
\section{Alagostado}
\begin{itemize}
\item {Grp. gram.:adj.}
\end{itemize}
Que tem côr de lagosta.
\section{Alagostice}
\begin{itemize}
\item {Grp. gram.:f.}
\end{itemize}
\begin{itemize}
\item {Utilização:Prov.}
\end{itemize}
\begin{itemize}
\item {Utilização:trasm.}
\end{itemize}
Esbanjamento, acto de \textunderscore alagosta\textunderscore .
\section{A-la-grande}
\begin{itemize}
\item {Grp. gram.:loc. adv.}
\end{itemize}
\begin{itemize}
\item {Utilização:Ant.}
\end{itemize}
Á grande, com grandeza; ostentosamente. Cf. Filinto, \textunderscore D. Manuel\textunderscore , III, 212.
\section{Alaguna}
\begin{itemize}
\item {Grp. gram.:f.}
\end{itemize}
O mesmo que \textunderscore laguna\textunderscore .
\section{Alahela}
\begin{itemize}
\item {Grp. gram.:f.}
\end{itemize}
\begin{itemize}
\item {Proveniência:(Do ár. \textunderscore al-hila\textunderscore )}
\end{itemize}
Arraial moirisco.
\section{Alale!}
\begin{itemize}
\item {Grp. gram.:interj.}
\end{itemize}
Grito de guerra, entre os antigos Gregos.
\section{Alalhana}
\begin{itemize}
\item {Grp. gram.:adv.}
\end{itemize}
\begin{itemize}
\item {Utilização:Ant.}
\end{itemize}
\begin{itemize}
\item {Proveniência:(De \textunderscore a\textunderscore  + \textunderscore la\textunderscore  + \textunderscore lhano\textunderscore )}
\end{itemize}
Chanmente; com lisura, com clareza.
\section{Alalia}
\begin{itemize}
\item {Grp. gram.:f.}
\end{itemize}
\begin{itemize}
\item {Proveniência:(Do gr. \textunderscore a\textunderscore  priv. e \textunderscore laleo\textunderscore )}
\end{itemize}
Paralysia dos órgãos da voz.
Mutismo accidental.
Impossibilidade de falar.
\section{Alálitho}
\begin{itemize}
\item {Grp. gram.:f.}
\end{itemize}
\begin{itemize}
\item {Proveniência:(De \textunderscore Ala\textunderscore , n. p. + gr. \textunderscore lithos\textunderscore )}
\end{itemize}
Mineral esverdeado, do valle de Ala, em Itália.
\section{Alálito}
\begin{itemize}
\item {Grp. gram.:f.}
\end{itemize}
\begin{itemize}
\item {Proveniência:(De \textunderscore Ala\textunderscore , n. p. + gr. \textunderscore lithos\textunderscore )}
\end{itemize}
Mineral esverdeado, do valle de Ala, em Itália.
\section{Álalo}
\begin{itemize}
\item {Grp. gram.:adj.}
\end{itemize}
\begin{itemize}
\item {Proveniência:(Do gr. \textunderscore a\textunderscore  priv. + \textunderscore lalos\textunderscore )}
\end{itemize}
Que não fala.
\section{Alalonga}
\begin{itemize}
\item {Grp. gram.:m.}
\end{itemize}
Peixe das cercanias de Nice.
\section{Alamal}
\begin{itemize}
\item {Grp. gram.:m}
\end{itemize}
Lugar, plantado de álamos; alameda.
\section{Alamão}
\begin{itemize}
\item {Grp. gram.:m. adj.}
\end{itemize}
\begin{itemize}
\item {Utilização:Prov.}
\end{itemize}
\begin{itemize}
\item {Utilização:alent.}
\end{itemize}
\begin{itemize}
\item {Utilização:Prov.}
\end{itemize}
\begin{itemize}
\item {Utilização:beir.}
\end{itemize}
Indivíduo forte ou corpulento.
Vendedor ambulante de fazendas e lençaria.
(Por \textunderscore alemão\textunderscore )
\section{Alamar}
\begin{itemize}
\item {Grp. gram.:m.}
\end{itemize}
\begin{itemize}
\item {Proveniência:(Do ár. \textunderscore al-amara\textunderscore )}
\end{itemize}
Cordão de requife ou de metal, que guarnece e abotôa a frente de um vestuário.
\section{Alamarado}
\begin{itemize}
\item {Grp. gram.:adj.}
\end{itemize}
Ornado de alamares.
\section{Alambari}
\begin{itemize}
\item {Grp. gram.:m.}
\end{itemize}
Pequeno peixe do Brasil.
\section{Alambazadamente}
\begin{itemize}
\item {Grp. gram.:adj.}
\end{itemize}
De modo \textunderscore alambazado\textunderscore .
\section{Alambazado}
\begin{itemize}
\item {Grp. gram.:adj.}
\end{itemize}
\begin{itemize}
\item {Utilização:Prov.}
\end{itemize}
\begin{itemize}
\item {Utilização:minh.}
\end{itemize}
Lambaz.
Grosseiro.
Orgulhoso.
Casquilho.
\section{Alambazar-se}
\begin{itemize}
\item {Grp. gram.:v. p.}
\end{itemize}
\begin{itemize}
\item {Grp. gram.:v. p.}
\end{itemize}
\begin{itemize}
\item {Utilização:Prov.}
\end{itemize}
\begin{itemize}
\item {Utilização:trasm.}
\end{itemize}
Fazer-se lambaz, glotão.
Comer muito.
Tornar-se desajeitado, grosseiro.
Estatelar-se no chão, cair de bruços.
\section{Alâmbel}
\begin{itemize}
\item {Grp. gram.:m.}
\end{itemize}
\begin{itemize}
\item {Utilização:Des.}
\end{itemize}
Pano de côres, para cobrir mesas, tabuleiros, etc.
\section{Alambicadamente}
\begin{itemize}
\item {Grp. gram.:adv.}
\end{itemize}
De modo \textunderscore alambicado\textunderscore .
\section{Alambicado}
\begin{itemize}
\item {Grp. gram.:adj.}
\end{itemize}
\begin{itemize}
\item {Proveniência:(De \textunderscore alambicar\textunderscore )}
\end{itemize}
Pretensioso; arrebicado.
\section{Alambicamento}
\begin{itemize}
\item {Grp. gram.:m.}
\end{itemize}
Affectação; requinte. Cf. Castilho, \textunderscore Tartufo\textunderscore , XVI.
\section{Alambicar}
\begin{itemize}
\item {Grp. gram.:v. t.}
\end{itemize}
\begin{itemize}
\item {Utilização:Fig.}
\end{itemize}
Destilar no alambique.
Tornar affectado, pretensioso.
Arrebicar.
Requintar.
\section{Alambique}
\begin{itemize}
\item {Grp. gram.:m.}
\end{itemize}
Apparelho de destillação.
(Ár. \textunderscore al-ambique\textunderscore , do gr.)
\section{Alambor}
\begin{itemize}
\item {Grp. gram.:m.}
\end{itemize}
\begin{itemize}
\item {Utilização:Ant.}
\end{itemize}
\begin{itemize}
\item {Proveniência:(De \textunderscore alamborar\textunderscore )}
\end{itemize}
Aumento de espessura na base das construcções de alvenaria.
\section{Alamborar}
\begin{itemize}
\item {Grp. gram.:v. t.}
\end{itemize}
Dar declive a, tornar convexo, alombar.
(Por \textunderscore alomborar\textunderscore , de \textunderscore lombo\textunderscore )
\section{Alambra}
\begin{itemize}
\item {Grp. gram.:f.}
\end{itemize}
\begin{itemize}
\item {Proveniência:(De \textunderscore alambre\textunderscore )}
\end{itemize}
Álamo negro.
Resina, extrahida do choupo.
\section{Alambrado}
\begin{itemize}
\item {Grp. gram.:m.}
\end{itemize}
\begin{itemize}
\item {Proveniência:(De \textunderscore alambrar\textunderscore )}
\end{itemize}
Terreno, cercado de fios de arame.
\section{Alambrador}
\begin{itemize}
\item {Grp. gram.:m.}
\end{itemize}
\begin{itemize}
\item {Proveniência:(De \textunderscore alambrar\textunderscore )}
\end{itemize}
Aquelle que fabrica fios de arame.
\section{Alambrar}
\begin{itemize}
\item {Grp. gram.:v. t.}
\end{itemize}
Cercar (terrenos) com fios de arame.
\section{Alambre}
\begin{itemize}
\item {Grp. gram.:m.}
\end{itemize}
\begin{itemize}
\item {Utilização:Pop.}
\end{itemize}
O mesmo que \textunderscore âmbar\textunderscore , (tendo a mais o art. ár. \textunderscore al\textunderscore )
Pessôa fina, muito esperta, que tem lume no ôlho.
\section{Alambreado}
\begin{itemize}
\item {Grp. gram.:adj.}
\end{itemize}
\begin{itemize}
\item {Utilização:des.}
\end{itemize}
\begin{itemize}
\item {Utilização:Pop.}
\end{itemize}
Que tem côr de alambre.
Que tem muito vinho no estômago.
\section{Alameda}
\begin{itemize}
\item {fónica:mê}
\end{itemize}
\begin{itemize}
\item {Grp. gram.:f.}
\end{itemize}
Lugar, plantado de álamos.
Bosque.
Parque.
Rua de árvores.
\section{Alamedar}
\begin{itemize}
\item {Grp. gram.:v. t.}
\end{itemize}
Converter em alameda.
\section{Alâmel}
\begin{itemize}
\item {Grp. gram.:m.}
\end{itemize}
O mesmo que \textunderscore alâmbel\textunderscore .
\section{Alamira}
\begin{itemize}
\item {fónica:á-la}
\end{itemize}
\begin{itemize}
\item {Grp. gram.:adv.}
\end{itemize}
\begin{itemize}
\item {Proveniência:(De \textunderscore a\textunderscore  + \textunderscore la\textunderscore  + \textunderscore mira\textunderscore )}
\end{itemize}
Á espera. A espreitar.
De prevenção:«\textunderscore Mulei Maluco, estando alamira da determinação do exército...\textunderscore »B. da Cruz, \textunderscore Chrón. D. Sebast.\textunderscore  c. LXII.
\section{Alamiré}
\begin{itemize}
\item {Grp. gram.:m.}
\end{itemize}
O mesmo que \textunderscore lamiré\textunderscore .
\section{Álamo}
\begin{itemize}
\item {Grp. gram.:m.}
\end{itemize}
\begin{itemize}
\item {Proveniência:(Do lat. \textunderscore alnus\textunderscore , sob a infl. de \textunderscore ulmus\textunderscore )}
\end{itemize}
Espécie de choupo, da fam. das salicíneas.
\section{Alamoda}
\begin{itemize}
\item {fónica:á-la}
\end{itemize}
\begin{itemize}
\item {Grp. gram.:adv.}
\end{itemize}
\begin{itemize}
\item {Proveniência:(De \textunderscore a\textunderscore  + \textunderscore la\textunderscore  + \textunderscore moda\textunderscore )}
\end{itemize}
Á moda; segundo a moda.
\section{Alâmpada}
\begin{itemize}
\item {Grp. gram.:f.}
\end{itemize}
(V.lâmpada)
\section{Alampadário}
\begin{itemize}
\item {Grp. gram.:m.}
\end{itemize}
(V.lampadário)
\section{Alampadeiro}
\begin{itemize}
\item {Grp. gram.:m.}
\end{itemize}
(V.lampadeiro)
\section{Alampar}
\begin{itemize}
\item {Grp. gram.:v. i.}
\end{itemize}
\begin{itemize}
\item {Utilização:Gír.}
\end{itemize}
\begin{itemize}
\item {Proveniência:(T. cast.)}
\end{itemize}
Vêr.
\section{Alampreado}
\begin{itemize}
\item {Grp. gram.:adj.}
\end{itemize}
Semelhante á lampreia, no gôsto, côr ou fórma.
\section{Alamutu}
\begin{itemize}
\item {Grp. gram.:m.}
\end{itemize}
Árvore fructífera de Madagáscar.
\section{Alancado}
\begin{itemize}
\item {Grp. gram.:adj.}
\end{itemize}
\begin{itemize}
\item {Utilização:Prov.}
\end{itemize}
\begin{itemize}
\item {Utilização:trasm.}
\end{itemize}
Curvado por um pêso; ajoujado.
\section{Alanceado}
\begin{itemize}
\item {Grp. gram.:adj.}
\end{itemize}
\begin{itemize}
\item {Utilização:Fig.}
\end{itemize}
\begin{itemize}
\item {Proveniência:(De \textunderscore alancear\textunderscore )}
\end{itemize}
Ferido com lança.
Amargurado, torturado.
\section{Alanceador}
\begin{itemize}
\item {Grp. gram.:m.}
\end{itemize}
Aquelle que alanceia.
\section{Alancear}
\begin{itemize}
\item {Grp. gram.:v. t.}
\end{itemize}
\begin{itemize}
\item {Utilização:Fig.}
\end{itemize}
Ferir com lança.
Causar grande dôr a; affligir.
Estimular.
\section{Alanco}
\begin{itemize}
\item {Grp. gram.:m.}
\end{itemize}
\begin{itemize}
\item {Utilização:T. de Lanhoso}
\end{itemize}
Impulso; auxílio de fôrça.
\section{Alandeado}
\begin{itemize}
\item {Grp. gram.:adj.}
\end{itemize}
Semelhante á lande.
\section{Alandro}
\begin{itemize}
\item {Grp. gram.:m.}
\end{itemize}
O mesmo que \textunderscore loendro\textunderscore .
\section{Alandroal}
\begin{itemize}
\item {Grp. gram.:m.}
\end{itemize}
Lugar, plantado de alandros.
\section{Alangiáceas}
\begin{itemize}
\item {Grp. gram.:f. pl.}
\end{itemize}
\begin{itemize}
\item {Proveniência:(De \textunderscore alangião\textunderscore )}
\end{itemize}
Família de plantas, que abrange árvores corpulentas, de fôlhas alternas, pecioladas, simples.
\section{Alangião}
\begin{itemize}
\item {Grp. gram.:m.}
\end{itemize}
Planta, que se tomou como typo das alangiáceas.
\section{Alangio}
\begin{itemize}
\item {Grp. gram.:m.}
\end{itemize}
O mesmo que \textunderscore alangião\textunderscore .
\section{Alanguidar-se}
\begin{itemize}
\item {Grp. gram.:v. p.}
\end{itemize}
\begin{itemize}
\item {Proveniência:(De \textunderscore lânguido\textunderscore )}
\end{itemize}
O mesmo que \textunderscore languescer\textunderscore .
\section{Alanhador}
\begin{itemize}
\item {Grp. gram.:m.}
\end{itemize}
Aquelle que alanha.
\section{Alanhar}
\begin{itemize}
\item {Grp. gram.:v. t.}
\end{itemize}
\begin{itemize}
\item {Proveniência:(Do lat. \textunderscore e-laniare\textunderscore , segundo Cornu)}
\end{itemize}
Fazer lanhos em.
Esfaquear.
Golpear.
Cansar.
Opprimir.
\section{Alânico}
\begin{itemize}
\item {Grp. gram.:adj.}
\end{itemize}
Relativo aos \textunderscore alanos\textunderscore .
\section{Alano}
\begin{itemize}
\item {Grp. gram.:m.}
\end{itemize}
Cão grande de caça grossa; alão.
\section{Alanos}
\begin{itemize}
\item {Grp. gram.:m. pl.}
\end{itemize}
\begin{itemize}
\item {Proveniência:(Lat. \textunderscore Alani\textunderscore )}
\end{itemize}
(não Álanos)
Povo da Scýthia, que no século V assolou a Gállia e a Espanha.
\section{Alanta}
\begin{itemize}
\item {Grp. gram.:f.}
\end{itemize}
\begin{itemize}
\item {Utilização:Náut.}
\end{itemize}
Apparelho, que passa em dois cadernaes pelos gornes da embarcação.
(Por \textunderscore alenta\textunderscore , de \textunderscore alentar\textunderscore ?)
\section{Alanterna}
\begin{itemize}
\item {Grp. gram.:f.}
\end{itemize}
(V.lanterna)
\section{Alanterneiro}
\begin{itemize}
\item {Grp. gram.:m.}
\end{itemize}
(V.lanterneiro)
\section{Alantina}
\begin{itemize}
\item {Grp. gram.:f.}
\end{itemize}
O mesmo que \textunderscore dhalina\textunderscore .
\section{Alanzoador}
\begin{itemize}
\item {Grp. gram.:m.}
\end{itemize}
Aquelle que alanzôa.
\section{Alanzoar}
\begin{itemize}
\item {Grp. gram.:v. t.}
\end{itemize}
\begin{itemize}
\item {Grp. gram.:V. i.}
\end{itemize}
\begin{itemize}
\item {Proveniência:(De \textunderscore zoar\textunderscore , com um pref. incerto)}
\end{itemize}
Dizer á tôa.
Tagarelar.
Fanfar, têr bazófia.
\section{Alão}
\begin{itemize}
\item {Grp. gram.:m.}
\end{itemize}
Cão grande de fila.
Pl. \textunderscore alãos.\textunderscore 
(B. lat. \textunderscore alanus\textunderscore )
\section{Alão}
\begin{itemize}
\item {Grp. gram.:m.}
\end{itemize}
\begin{itemize}
\item {Utilização:Prov.}
\end{itemize}
\begin{itemize}
\item {Utilização:trasm.}
\end{itemize}
\begin{itemize}
\item {Proveniência:(De \textunderscore ala\textunderscore ^1)}
\end{itemize}
Grande pedra de loisa, com que se encimam os muros, para que as pedras miúdas não desabem facilmente.
\section{Alapado}
\begin{itemize}
\item {Grp. gram.:adj.}
\end{itemize}
Escondido em lapa.
Agachado.
\section{Alapar}
\begin{itemize}
\item {Grp. gram.:v. t.}
\end{itemize}
Esconder em lapa.
Occultar, debaixo ou detrás de alguma coisa.
\section{Alapar}
\begin{itemize}
\item {Grp. gram.:adv.}
\end{itemize}
\begin{itemize}
\item {Utilização:Ant.}
\end{itemize}
\begin{itemize}
\item {Proveniência:(De \textunderscore a\textunderscore  + \textunderscore la\textunderscore  + \textunderscore par\textunderscore )}
\end{itemize}
Igualmente.
Ao lado:«\textunderscore ...se moveram a lapar\textunderscore ». \textunderscore Jornada de África\textunderscore , c. VI.
\section{Alaparavante}
\textunderscore adv. Ant.\textunderscore  (?)«\textunderscore ...as cameras [q~] hiam dallaparavante.\textunderscore »\textunderscore Hist. Trág. Marít.\textunderscore , 51.
\section{Alapardadamente}
\begin{itemize}
\item {Grp. gram.:adv.}
\end{itemize}
\begin{itemize}
\item {Proveniência:(De \textunderscore alapardado\textunderscore )}
\end{itemize}
Ás occultas, furtivamente.
\section{Alapardado}
\begin{itemize}
\item {Grp. gram.:adj.}
\end{itemize}
\begin{itemize}
\item {Proveniência:(De \textunderscore alapardar-se\textunderscore )}
\end{itemize}
Alapado.
Escondido atrás ou debaixo de alguma coisa.
\section{Alapardar-se}
\begin{itemize}
\item {Grp. gram.:v. p.}
\end{itemize}
\begin{itemize}
\item {Proveniência:(De \textunderscore láparo\textunderscore ?)}
\end{itemize}
Agachar-se.
Occultar-se.
Abaixar-se para não sêr visto.
\section{Alapoado}
\begin{itemize}
\item {Grp. gram.:adj.}
\end{itemize}
Que tem modos de lapão; grosseiro.
\section{Alaptos}
\begin{itemize}
\item {Grp. gram.:m. pl.}
\end{itemize}
Gênero de insectos hymenópteros, cuja única espécie se encontra na Inglaterra.
\section{Alaque}
\begin{itemize}
\item {Grp. gram.:m.}
\end{itemize}
O mesmo que \textunderscore plintho\textunderscore .
\section{Alaqueca}
\begin{itemize}
\item {Grp. gram.:f.}
\end{itemize}
O mesmo que \textunderscore laqueca\textunderscore . Cf. Castanheda, liv. I, c. 13.
\section{Alar}
\begin{itemize}
\item {Grp. gram.:v. t.}
\end{itemize}
\begin{itemize}
\item {Utilização:Prov.}
\end{itemize}
\begin{itemize}
\item {Utilização:dur.}
\end{itemize}
Formar em alas.
Dar asas a.
Fazer voar.
Içar; levantar.
Arrastar ou guiar (barcos) á sirga.
\section{Alar}
\begin{itemize}
\item {Grp. gram.:v. i.}
\end{itemize}
\begin{itemize}
\item {Utilização:Gír.}
\end{itemize}
Viver.
\section{Alar}
\begin{itemize}
\item {Grp. gram.:v. i.}
\end{itemize}
\begin{itemize}
\item {Utilização:Gír.}
\end{itemize}
\begin{itemize}
\item {Proveniência:(Fr. \textunderscore aller\textunderscore )}
\end{itemize}
Ir.
\section{Alar}
\begin{itemize}
\item {Grp. gram.:adj.}
\end{itemize}
\begin{itemize}
\item {Proveniência:(Do lat. \textunderscore ala\textunderscore )}
\end{itemize}
Que tem fórma de asa.
\section{Alara}
\begin{itemize}
\item {Grp. gram.:f.}
\end{itemize}
\begin{itemize}
\item {Utilização:Ant.}
\end{itemize}
Leque, com que os acólytos, nas festas de igreja, enxotavam as môscas da cabeça e da cara dos celebrantes.
(B. lat. \textunderscore alare\textunderscore )
\section{Alárabe}
\begin{itemize}
\item {Grp. gram.:m.}
\end{itemize}
\begin{itemize}
\item {Utilização:Ant.}
\end{itemize}
Árabe beduíno.
(Cp. \textunderscore alarve\textunderscore )
\section{Alaranjado}
\begin{itemize}
\item {Grp. gram.:adj.}
\end{itemize}
\begin{itemize}
\item {Proveniência:(De \textunderscore alaranjar\textunderscore )}
\end{itemize}
Que tem côr ou fórma de laranja.
\section{Alaranjar}
\begin{itemize}
\item {Grp. gram.:v. t.}
\end{itemize}
Dar côr ou forma de laranja. Cf. Eça, \textunderscore P. Amaro\textunderscore , 61.
\section{Alarar}
\begin{itemize}
\item {Grp. gram.:v. t.}
\end{itemize}
Estender no lar, na lareira.
\section{Alárave}
\begin{itemize}
\item {Grp. gram.:m.}
\end{itemize}
\begin{itemize}
\item {Utilização:Ant.}
\end{itemize}
Árabe beduíno.
(Cp. \textunderscore alarve\textunderscore )
\section{Alarcónia}
\begin{itemize}
\item {Grp. gram.:f.}
\end{itemize}
\begin{itemize}
\item {Proveniência:(De \textunderscore Alarcon\textunderscore , n. p.)}
\end{itemize}
Planta, da fam. das compostas, tribo das senecionídeas.
\section{Alarde}
\begin{itemize}
\item {Grp. gram.:m.}
\end{itemize}
Vanglória; ostentação; apparato; vaidade.
(Ár. \textunderscore al\textunderscore  + \textunderscore ardh\textunderscore )
\section{Alardeador}
\begin{itemize}
\item {Grp. gram.:m.}
\end{itemize}
Aquelle que alardeia.
\section{Alardeamento}
\begin{itemize}
\item {Grp. gram.:m.}
\end{itemize}
Acto de \textunderscore alardear\textunderscore .
\section{Alardear}
\begin{itemize}
\item {Grp. gram.:v. t.}
\end{itemize}
Fazer alarde de.
Ostentar.
Gabar-se de.
\section{Alardo}
\begin{itemize}
\item {Grp. gram.:m.}
\end{itemize}
Revista de tropas, que se fazia annualmente.
Gente de armas, preparada para combate ou para embarque.
Resenha minuciosa.
Caderno ou rol, por onde se faz chamada de pessoal.
O mesmo que \textunderscore alarde\textunderscore .(V.alarde)
\section{Alares}
\begin{itemize}
\item {Grp. gram.:m. pl.}
\end{itemize}
\begin{itemize}
\item {Grp. gram.:m. pl.}
\end{itemize}
\begin{itemize}
\item {Proveniência:(De \textunderscore alar\textunderscore )}
\end{itemize}
Laços de crina de cavallo, para caçar perdizes.
Cabos, com que os pescadores do rio Minho alam para a terra a rede algerife.
\section{Alargadamente}
\begin{itemize}
\item {Grp. gram.:adv.}
\end{itemize}
Com largueza.
\section{Alargador}
\begin{itemize}
\item {Grp. gram.:m.}
\end{itemize}
Aquelle que alarga.
\section{Alargamento}
\begin{itemize}
\item {Grp. gram.:m.}
\end{itemize}
Acto de \textunderscore alargar\textunderscore .
\section{Alargar}
\begin{itemize}
\item {Grp. gram.:v. t.}
\end{itemize}
Tornar largo, extenso.
Dilatar.
Afroixar.
Prolongar.
\section{Alariciano}
\begin{itemize}
\item {Grp. gram.:adj.}
\end{itemize}
Relativo a Alarico. Cf. Herculano, \textunderscore Hist. Port.\textunderscore , IV, 388.
\section{Alarida}
\begin{itemize}
\item {Grp. gram.:f.}
\end{itemize}
(V.alarido)
\section{Alarido}
\begin{itemize}
\item {Grp. gram.:m.}
\end{itemize}
\begin{itemize}
\item {Proveniência:(Do ár. \textunderscore garida\textunderscore )}
\end{itemize}
Gritaria; clamor.
\section{Alarife}
\begin{itemize}
\item {Grp. gram.:m.}
\end{itemize}
\begin{itemize}
\item {Utilização:Ant.}
\end{itemize}
Architecto; mestre de obras.
(Ár. \textunderscore alarif\textunderscore )
\section{Alarma}
\begin{itemize}
\item {Grp. gram.:m.}
\end{itemize}
Grito para chamar ás armas.
Rebate.
Abalo.
Confusão.
Vozearia; tumulto.
(Talvez do it. \textunderscore all'arma\textunderscore )
\section{Alarmante}
\begin{itemize}
\item {Grp. gram.:adj.}
\end{itemize}
Que alarma, assusta, perturba.
\section{Alarmar}
\begin{itemize}
\item {Grp. gram.:v. t.}
\end{itemize}
Pôr em alarma.
Assustar.
Alvorotar.
\section{Alarme}
\begin{itemize}
\item {Grp. gram.:m.}
\end{itemize}
(Corr. de \textunderscore alarma\textunderscore )
\section{Alarmista}
\begin{itemize}
\item {Grp. gram.:m.}
\end{itemize}
Aquelle que se compraz em espalhar boatos alarmantes.
\section{Alarpar-se}
\begin{itemize}
\item {Grp. gram.:v. p.}
\end{itemize}
\begin{itemize}
\item {Utilização:Pop.}
\end{itemize}
Abaixar-se, para não ser visto.
Alapardar-se.
(Metáth. de \textunderscore alap'rar-se\textunderscore , por \textunderscore alaparar-se\textunderscore , de \textunderscore láparo\textunderscore )
\section{Alarvado}
\begin{itemize}
\item {Grp. gram.:adj.}
\end{itemize}
Que tem modos de alarve.
\section{Alarvajado}
\begin{itemize}
\item {Grp. gram.:adj.}
\end{itemize}
Que tem modos de alarve; incivil, descortês.
\section{Alarvaria}
\begin{itemize}
\item {Grp. gram.:f.}
\end{itemize}
Brutalidade; acção de alarve.
\section{Alarve}
\begin{itemize}
\item {Grp. gram.:m.}
\end{itemize}
\begin{itemize}
\item {Utilização:Ant.}
\end{itemize}
Aquelle que é brutal, selvagem, rude.
Comilão.
Árabe beduíno.
(Ár. \textunderscore al-arabi\textunderscore )
\section{Alarvidade}
\begin{itemize}
\item {Grp. gram.:f.}
\end{itemize}
Qualidade de alarve. Cf. Arn. Gama, \textunderscore Motim\textunderscore , 21.
\section{Alarvia}
\begin{itemize}
\item {Grp. gram.:f.}
\end{itemize}
Multidão de alarves.
\section{Alarvice}
\begin{itemize}
\item {Grp. gram.:f.}
\end{itemize}
\begin{itemize}
\item {Utilização:P. us.}
\end{itemize}
Qualidade ou acção de \textunderscore alarve\textunderscore .
\section{Alassar}
\begin{itemize}
\item {Grp. gram.:v. i.}
\end{itemize}
\begin{itemize}
\item {Utilização:Marn.}
\end{itemize}
Destacar-se facilmente (o sal) do casco da marinha. Cf. \textunderscore Museu Techn.\textunderscore , 21.
\section{Alastrado}
\begin{itemize}
\item {Grp. gram.:adj.}
\end{itemize}
\begin{itemize}
\item {Proveniência:(De \textunderscore alastrar\textunderscore )}
\end{itemize}
Disposto como lastro.
\section{Alastramento}
\begin{itemize}
\item {Grp. gram.:m.}
\end{itemize}
Acto de \textunderscore alastrar\textunderscore .
\section{Alastrar}
\begin{itemize}
\item {Grp. gram.:v. t.}
\end{itemize}
Cobrir com lastro, lastrar.
Cobrir, espalhando.
Espalhar; derramar.
Alargar gradualmente.
\section{Alaterna}
\begin{itemize}
\item {Grp. gram.:f.}
\end{itemize}
\begin{itemize}
\item {Proveniência:(Lat. \textunderscore alaternus\textunderscore )}
\end{itemize}
Designação scientífica do aderno.
\section{Alaterno}
\begin{itemize}
\item {Grp. gram.:m.}
\end{itemize}
\begin{itemize}
\item {Proveniência:(Lat. \textunderscore alaternus\textunderscore )}
\end{itemize}
Designação scientífica do aderno.
\section{Alatinadamente}
\begin{itemize}
\item {Grp. gram.:adv.}
\end{itemize}
De modo \textunderscore alatinado\textunderscore .
\section{Alatinado}
\begin{itemize}
\item {Grp. gram.:adj.}
\end{itemize}
\begin{itemize}
\item {Proveniência:(De \textunderscore alatinar\textunderscore )}
\end{itemize}
Que tem fórma ou syntaxe latina.
\section{Alatinar}
\begin{itemize}
\item {Grp. gram.:v. t.}
\end{itemize}
Dar fórma ou syntaxe latina a.
\section{Alatita}
\begin{itemize}
\item {Grp. gram.:f.}
\end{itemize}
\begin{itemize}
\item {Proveniência:(De \textunderscore Ala\textunderscore , n. p.)}
\end{itemize}
Crystal transparente e levemente colorido de verde, que se encontra no valle de Ala, (Piemonte), e noutros pontos.
\section{Alatoar}
\begin{itemize}
\item {Grp. gram.:v. t.}
\end{itemize}
Guarnecer com cintas ou embutidos de latão.
\section{Alauate}
\begin{itemize}
\item {fónica:la-u}
\end{itemize}
\begin{itemize}
\item {Grp. gram.:m.}
\end{itemize}
Espécie de macaco.
\section{Alauda}
\begin{itemize}
\item {Grp. gram.:m.}
\end{itemize}
\begin{itemize}
\item {Proveniência:(Lat. \textunderscore alauda\textunderscore )}
\end{itemize}
Gênero de pássaros conirostros, que comprehende a calhandra, o cochicho, etc.
\section{Alaúde}
\begin{itemize}
\item {Grp. gram.:m.}
\end{itemize}
\begin{itemize}
\item {Proveniência:(Do ár. \textunderscore al-ud\textunderscore )}
\end{itemize}
Antigo instrumento de cordas.
\section{Alaúde}
\begin{itemize}
\item {Grp. gram.:m.}
\end{itemize}
Pequena embarcação, ainda hoje usada na pesca do atum. Cf. Ortigão, \textunderscore Culto da Arte\textunderscore .«...Mandou o Conde logo acerca destas cousas Mose Martim de Pumar em um \textunderscore Alaude...\textunderscore »Azurara, \textunderscore Chrón. do Conde D. Pedro\textunderscore , cap. LXI, p. 417.
\section{Alaudídeos}
\begin{itemize}
\item {Grp. gram.:m. pl.}
\end{itemize}
\begin{itemize}
\item {Proveniência:(De \textunderscore alauda\textunderscore  + gr. \textunderscore eidos\textunderscore )}
\end{itemize}
Família de pássaros, que tira nome do \textunderscore alauda\textunderscore .
\section{Alaudíneos}
\begin{itemize}
\item {Grp. gram.:m. pl.}
\end{itemize}
Sub-família dos \textunderscore alaudídeos\textunderscore .
\section{A-la-una}
\begin{itemize}
\item {Grp. gram.:adv.}
\end{itemize}
\begin{itemize}
\item {Grp. gram.:M.}
\end{itemize}
\begin{itemize}
\item {Utilização:Prov.}
\end{itemize}
\begin{itemize}
\item {Utilização:alent.}
\end{itemize}
Á uma.
Juntamente.
Jôgo de rapazes, em que se emprega o salto e certos estribilhos.
(Loc. cast.)
\section{Alaúna}
\begin{itemize}
\item {Grp. gram.:adv.}
\end{itemize}
\begin{itemize}
\item {Grp. gram.:M.}
\end{itemize}
\begin{itemize}
\item {Utilização:Prov.}
\end{itemize}
\begin{itemize}
\item {Utilização:alent.}
\end{itemize}
Á uma.
Juntamente.
Jôgo de rapazes, em que se emprega o salto e certos estribilhos.
(Loc. cast.)
\section{Alaus}
\begin{itemize}
\item {Grp. gram.:m. pl.}
\end{itemize}
\begin{itemize}
\item {Proveniência:(Gr. \textunderscore alaos\textunderscore )}
\end{itemize}
Gênero de insectos coleópteros pentâmeros.
\section{Alavanca}
\begin{itemize}
\item {Grp. gram.:f.}
\end{itemize}
Barra de ferro ou madeira, para mover ou levantar corpos pesados.
(Cast. \textunderscore palanca\textunderscore , sob a infl. de \textunderscore levantar\textunderscore ?)
\section{Alavanco}
\begin{itemize}
\item {Grp. gram.:m.}
\end{itemize}
O mesmo que \textunderscore adem\textunderscore .
\section{Alavão}
\begin{itemize}
\item {Grp. gram.:m.}
\end{itemize}
O mesmo ou melhor que \textunderscore alabão\textunderscore .
\section{Alavão}
\begin{itemize}
\item {Grp. gram.:m.}
\end{itemize}
\begin{itemize}
\item {Utilização:Prov.}
\end{itemize}
\begin{itemize}
\item {Utilização:alent.}
\end{itemize}
\begin{itemize}
\item {Proveniência:(Do ár. \textunderscore al-laban\textunderscore )}
\end{itemize}
Rebanho, que dá leite por meio da ordenha.
\section{Alavercar}
\begin{itemize}
\item {Grp. gram.:v. t.}
\end{itemize}
\begin{itemize}
\item {Utilização:Ant.}
\end{itemize}
\begin{itemize}
\item {Grp. gram.:v. t.}
\end{itemize}
\begin{itemize}
\item {Utilização:Ant.}
\end{itemize}
\begin{itemize}
\item {Proveniência:(De \textunderscore laverca\textunderscore , alludindo ao vôo descendente da laverca?)}
\end{itemize}
Abaixar; humilhar.
Altercar?«\textunderscore ...num alavercar antre elles.\textunderscore »\textunderscore Aulegrafia\textunderscore , 159.
\section{Alavês}
\begin{itemize}
\item {Grp. gram.:m.  e  adj.}
\end{itemize}
O que é da província de Alava, em Espanha.
\section{Alavoeiro}
\begin{itemize}
\item {Grp. gram.:m.}
\end{itemize}
Aquelle que pastoreia alavões.
\section{Alazão}
\begin{itemize}
\item {Grp. gram.:m.}
\end{itemize}
\begin{itemize}
\item {Grp. gram.:Adj.}
\end{itemize}
Cavallo de côr de canela.
Que tem côr de canela, (falando-se do cavallo).
(Cast. \textunderscore alazan\textunderscore )
\section{Alazarado}
\begin{itemize}
\item {Grp. gram.:adj.}
\end{itemize}
\begin{itemize}
\item {Utilização:Prov.}
\end{itemize}
\begin{itemize}
\item {Utilização:beir.}
\end{itemize}
Que tem muitas dívidas.
(Colhido no Fundão)
\section{Alazeirado}
\begin{itemize}
\item {Grp. gram.:adj.}
\end{itemize}
Que tem lazeira.
\section{Alba}
\begin{itemize}
\item {Grp. gram.:f.}
\end{itemize}
Gênero de poesia provençal.
\section{Albácar}
\begin{itemize}
\item {Grp. gram.:m.}
\end{itemize}
Porta de fortaleza moirisca.
Barbacan.
\section{Albaciga}
\begin{itemize}
\item {Grp. gram.:f.}
\end{itemize}
Arbusto do Chile.
\section{Albacora}
\begin{itemize}
\item {Grp. gram.:f.}
\end{itemize}
\begin{itemize}
\item {Proveniência:(Do ár. \textunderscore al-bacar\textunderscore )}
\end{itemize}
Peixe escômbrida, cujo gênero abrange o atum e outras espécies.
\section{Albafar}
\begin{itemize}
\item {Grp. gram.:m.}
\end{itemize}
\begin{itemize}
\item {Proveniência:(Do ár. ?)}
\end{itemize}
Perfume. Incenso.
Peixe, o mesmo que \textunderscore albafora\textunderscore .
\section{Albafór}
\begin{itemize}
\item {Grp. gram.:m.}
\end{itemize}
\begin{itemize}
\item {Utilização:Ant.}
\end{itemize}
Espécie de perfume, extrahido da raiz de uma planta.
\section{Albafôr}
\begin{itemize}
\item {Grp. gram.:m.}
\end{itemize}
\begin{itemize}
\item {Utilização:Ant.}
\end{itemize}
Espécie de perfume, extrahido da raiz de uma planta.
\section{Albafora}
\begin{itemize}
\item {fónica:fô}
\end{itemize}
\begin{itemize}
\item {Grp. gram.:f.}
\end{itemize}
Grande peixe, das costas de Portugal.
O mesmo que \textunderscore albacora\textunderscore ?
\section{Albanês}
\begin{itemize}
\item {Grp. gram.:adj.}
\end{itemize}
\begin{itemize}
\item {Grp. gram.:M.}
\end{itemize}
Relativo á Albânia.
Homem natural da Albânia.
Língua dêste país.
\section{Albanês}
\begin{itemize}
\item {Grp. gram.:m.}
\end{itemize}
\begin{itemize}
\item {Utilização:Prov.}
\end{itemize}
\begin{itemize}
\item {Utilização:alent.}
\end{itemize}
O mesmo que \textunderscore alvanel\textunderscore .
\section{Albanesa}
\begin{itemize}
\item {fónica:nê}
\end{itemize}
\begin{itemize}
\item {Grp. gram.:f.}
\end{itemize}
Anêmona branca.
(Fem. de \textunderscore albanês\textunderscore ,^1)
\section{Albará}
\begin{itemize}
\item {Grp. gram.:m.}
\end{itemize}
\begin{itemize}
\item {Utilização:Bras}
\end{itemize}
Espécie de cana da Índia.
\section{Albarda}
\begin{itemize}
\item {Grp. gram.:f.}
\end{itemize}
\begin{itemize}
\item {Utilização:Pop.}
\end{itemize}
\begin{itemize}
\item {Utilização:Pesc.}
\end{itemize}
\begin{itemize}
\item {Proveniência:(Do ár. \textunderscore al-bardaa\textunderscore )}
\end{itemize}
Sella grosseira de bêstas de carga.
Jaqueta ou casaco mal feito.
O dorso da Pescada.
\section{Albardada}
\begin{itemize}
\item {Grp. gram.:f.}
\end{itemize}
\begin{itemize}
\item {Utilização:Pop.}
\end{itemize}
\begin{itemize}
\item {Proveniência:(De \textunderscore albardar\textunderscore )}
\end{itemize}
Fatia, envolta em ovos batidos, frita e recoberta de açúcar.
\section{Albardado}
\begin{itemize}
\item {Grp. gram.:adj.}
\end{itemize}
\begin{itemize}
\item {Grp. gram.:adj.}
\end{itemize}
\begin{itemize}
\item {Proveniência:(De \textunderscore albardar\textunderscore )}
\end{itemize}
Que traz albarda.
Diz-se do toiro, que tem sôbre o lombo uma mancha de côr differente da do resto do pêlo, não sendo o toiro malhado nem sardo.
\section{Albardadura}
\begin{itemize}
\item {Grp. gram.:f.}
\end{itemize}
\begin{itemize}
\item {Grp. gram.:f.}
\end{itemize}
Arreios de bêstas, comprehendida a albarda, cabeçada, atafal, retranca, etc.
Acto ou effeito de \textunderscore albardar\textunderscore .
\section{Albardan}
\begin{itemize}
\item {Grp. gram.:f.}
\end{itemize}
\begin{itemize}
\item {Utilização:Ant.}
\end{itemize}
O mesmo que \textunderscore capuz\textunderscore . Cf. Af. el Sabio. \textunderscore Cant. de María\textunderscore , 401.
\section{Albardão}
\begin{itemize}
\item {Grp. gram.:m.}
\end{itemize}
\begin{itemize}
\item {Utilização:Bras}
\end{itemize}
Albarda para montar; albarda grande.
Cadeia de cerros e baixadas; sellada.
\section{Albardar}
\begin{itemize}
\item {Grp. gram.:v. t.}
\end{itemize}
\begin{itemize}
\item {Utilização:Fig.}
\end{itemize}
\begin{itemize}
\item {Utilização:Burl.}
\end{itemize}
\begin{itemize}
\item {Utilização:Pop.}
\end{itemize}
Pôr albarda em.
Ajaezar com albarda.
Cobrir (fatias) com ovos e açúcar.
Cobrir (sardinhas) com ovos para fritura.
Vexar.
Vestir, enroupar.
Fazer á pressa e mal (qualquer coisa).
\section{Albardeira}
\begin{itemize}
\item {Grp. gram.:f.}
\end{itemize}
\begin{itemize}
\item {Grp. gram.:Adj.}
\end{itemize}
Rosa silvestre.
Diz-se da agulha grande, com que se cosem albardas.
\section{Albardeiro}
\begin{itemize}
\item {Grp. gram.:m.}
\end{itemize}
\begin{itemize}
\item {Utilização:Pop.}
\end{itemize}
\begin{itemize}
\item {Grp. gram.:Adj.}
\end{itemize}
\begin{itemize}
\item {Utilização:Bras. de Minas}
\end{itemize}
Aquelle que faz ou vende albardas.
Mau alfaiate.
Que trabalha mal, que é mau artista.
\section{Albardeiro}
\begin{itemize}
\item {Grp. gram.:adj.}
\end{itemize}
\begin{itemize}
\item {Utilização:Prov.}
\end{itemize}
\begin{itemize}
\item {Utilização:trasm.}
\end{itemize}
Mentiroso.
(por \textunderscore aldrabeiro\textunderscore ?)
\section{Albardilha}
Pequena albarda. Cf. Goes, \textunderscore Chrón. de D. Man.\textunderscore , XXXVI.
\section{Albardina}
\begin{itemize}
\item {Grp. gram.:f.}
\end{itemize}
\begin{itemize}
\item {Utilização:Bot.}
\end{itemize}
\begin{itemize}
\item {Proveniência:(De \textunderscore Albardos\textunderscore , n. p.)}
\end{itemize}
Peónia silvestre, vulgar nas abas da serra de Albardos.
\section{Albarela}
\begin{itemize}
\item {Grp. gram.:f.}
\end{itemize}
Cogumelo comestível, que cresce nos castanheiros.
\section{Albarrã}
\begin{itemize}
\item {Grp. gram.:f.}
\end{itemize}
Espécie de cebola, da fam. das liliáceas.
Tôrre saliente em castellos ou muralhas.
(Ár. \textunderscore al-barran\textunderscore )
\section{Albarrada}
\begin{itemize}
\item {Grp. gram.:f.}
\end{itemize}
\begin{itemize}
\item {Utilização:Ant.}
\end{itemize}
\begin{itemize}
\item {Grp. gram.:f.}
\end{itemize}
\begin{itemize}
\item {Utilização:Ant.}
\end{itemize}
\begin{itemize}
\item {Grp. gram.:f.}
\end{itemize}
\begin{itemize}
\item {Utilização:Ant.}
\end{itemize}
\begin{itemize}
\item {Proveniência:(Do ár. \textunderscore al-barrada\textunderscore )}
\end{itemize}
Muro de resguardo.
Vaso para beber.
Vaso com flôres, para ornato de mesas.
\section{Albarran}
\begin{itemize}
\item {Grp. gram.:f.}
\end{itemize}
Espécie de cebola, da fam. das liliáceas.
Tôrre saliente em castellos ou muralhas.
(Ár. \textunderscore al-barran\textunderscore )
\section{Albarrana}
\begin{itemize}
\item {Grp. gram.:f.}
\end{itemize}
O mesmo que \textunderscore albarran\textunderscore .
\section{Albarrão}
\begin{itemize}
\item {Grp. gram.:m.}
\end{itemize}
\begin{itemize}
\item {Utilização:Prov.}
\end{itemize}
Perdigão, que perdeu a fêmea, e anda descasalado no monte.
\section{Albarraz}
\begin{itemize}
\item {Grp. gram.:m.}
\end{itemize}
Designação antiga do \textunderscore paparraz\textunderscore .
\section{Albatroz}
\begin{itemize}
\item {Grp. gram.:m.}
\end{itemize}
Grande ave palmípede, muito voraz.
\section{Albente}
\begin{itemize}
\item {Grp. gram.:adj.}
\end{itemize}
\begin{itemize}
\item {Proveniência:(Lat. \textunderscore albens\textunderscore )}
\end{itemize}
Que branqueja; que alveja.
\section{Albergador}
\begin{itemize}
\item {Grp. gram.:m.}
\end{itemize}
Aquelle que alberga.
\section{Albergagem}
\begin{itemize}
\item {Grp. gram.:f.}
\end{itemize}
Direito, que os senhorios tinham, de sêr albergados pelos seus emphyteutas.
\section{Albergala}
\begin{itemize}
\item {Grp. gram.:f.}
\end{itemize}
(Apparece em documentos antigos, talvez como cópia errada de \textunderscore albergata\textunderscore , por \textunderscore alpercata\textunderscore )
\section{Albergamento}
\begin{itemize}
\item {Grp. gram.:m.}
\end{itemize}
Acto de \textunderscore albergar\textunderscore .
\section{Albergar}
\begin{itemize}
\item {Grp. gram.:v. t.}
\end{itemize}
\begin{itemize}
\item {Grp. gram.:V. i.}
\end{itemize}
\begin{itemize}
\item {Grp. gram.:V. i.}
\end{itemize}
Dar albergue, hospedagem, agasalho, a.
Conter: \textunderscore albergar maus sentimentos\textunderscore .
Estar num albergue; estar hospedado.
Co-habitar.
\section{Albergaria}
\begin{itemize}
\item {Grp. gram.:f.}
\end{itemize}
Lugar, em que se dá albergue; estalagem; hospedaria.
Contrato de hospedagem.
\section{Albergue}
\begin{itemize}
\item {Grp. gram.:m.}
\end{itemize}
\begin{itemize}
\item {Proveniência:(Do ár. \textunderscore al-barga\textunderscore ? Do alt. al. ant. \textunderscore hari-bëre\textunderscore )}
\end{itemize}
Hospedagem.
Hospedaria, albergaria.
\section{Albergueiro}
\begin{itemize}
\item {Grp. gram.:m.}
\end{itemize}
Aquelle que alberga, hospéda, agasalha.
\section{Albérsia}
\begin{itemize}
\item {Grp. gram.:f.}
\end{itemize}
Gênero de plantas amarantáceas.
\section{Alberta}
\begin{itemize}
\item {Grp. gram.:f.}
\end{itemize}
Planta rubiácea, de inflorescência terminal.
\section{Albérteas}
\begin{itemize}
\item {Grp. gram.:f. pl.}
\end{itemize}
Tribo de plantas, que têm por typo a \textunderscore alberta\textunderscore .
\section{Albertina}
\begin{itemize}
\item {Grp. gram.:f.}
\end{itemize}
\begin{itemize}
\item {Proveniência:(De \textunderscore Albertina\textunderscore , n. p.)}
\end{itemize}
Espécie de anêmona.
Espécie de tulipa raiada.
\section{Albertinho}
\begin{itemize}
\item {Grp. gram.:m.}
\end{itemize}
\begin{itemize}
\item {Utilização:T. de Serpa}
\end{itemize}
Pequena infusa, da capacidade de meio alberto.
\section{Albertínia}
\begin{itemize}
\item {Grp. gram.:f.}
\end{itemize}
Arbusto do Brasil.
\section{Albertino}
\begin{itemize}
\item {Grp. gram.:m.}
\end{itemize}
\begin{itemize}
\item {Proveniência:(De \textunderscore Alberto\textunderscore , n. p.)}
\end{itemize}
Moéda antiga, em França e noutros países.
\section{Albertipia}
\begin{itemize}
\item {Grp. gram.:f.}
\end{itemize}
\begin{itemize}
\item {Utilização:Phot.}
\end{itemize}
Processo, com que a matriz se traslada para uma placa de vidro coberta de chromato de potassa.
\section{Alberto}
\begin{itemize}
\item {Grp. gram.:m.}
\end{itemize}
\begin{itemize}
\item {Utilização:T. de Serpa}
\end{itemize}
Pequena infusa, ou meia quarta. Cf. \textunderscore Tradição\textunderscore , II, n.^o 11.
\section{Albertypia}
\begin{itemize}
\item {Grp. gram.:f.}
\end{itemize}
\begin{itemize}
\item {Utilização:Phot.}
\end{itemize}
Processo, com que a matriz se traslada para uma placa de vidro coberta de chromato de potassa.
\section{Albetoça}
\begin{itemize}
\item {Grp. gram.:f.}
\end{itemize}
Embarcação indiana, dizem os lexicógraphos.
Rui de Pina porém fala-nos das \textunderscore albetoças\textunderscore , em que se passeava no Tejo, e poderá daqui inferir-se que o termo se applicou a mais de uma espécie de embarcação, não só oriental.
(Alter. do ár. \textunderscore al-botça?\textunderscore )
\section{Albicante}
\begin{itemize}
\item {Grp. gram.:f.}
\end{itemize}
\begin{itemize}
\item {Grp. gram.:Adj.}
\end{itemize}
\begin{itemize}
\item {Proveniência:(Lat. \textunderscore albicans\textunderscore )}
\end{itemize}
Espécie de anêmona, de fôlhas esbranquiçadas.
Esbranquiçado.
\section{Albicastrense}
\begin{itemize}
\item {Grp. gram.:adj.}
\end{itemize}
\begin{itemize}
\item {Grp. gram.:M.}
\end{itemize}
\begin{itemize}
\item {Proveniência:(Do lat. \textunderscore Albicastrum\textunderscore , n. p.)}
\end{itemize}
Relativo a Castello-Branco.
Habitante de Castello-Branco.
\section{Albicaude}
\begin{itemize}
\item {Grp. gram.:adj.}
\end{itemize}
\begin{itemize}
\item {Proveniência:(Do lat. \textunderscore albus\textunderscore  + \textunderscore cauda\textunderscore )}
\end{itemize}
Que tem cauda branca.
\section{Albicaule}
\begin{itemize}
\item {Grp. gram.:adj.}
\end{itemize}
\begin{itemize}
\item {Proveniência:(Do lat. \textunderscore albus\textunderscore  + \textunderscore caulis\textunderscore )}
\end{itemize}
Que tem tronco branco ou esbranquiçado.
\section{Albicole}
\begin{itemize}
\item {Grp. gram.:adj.}
\end{itemize}
\begin{itemize}
\item {Proveniência:(Do lat. \textunderscore albus\textunderscore  + \textunderscore collum\textunderscore )}
\end{itemize}
Que tem pescoço branco.
\section{Albicolle}
\begin{itemize}
\item {Grp. gram.:adj.}
\end{itemize}
\begin{itemize}
\item {Proveniência:(Do lat. \textunderscore albus\textunderscore  + \textunderscore collum\textunderscore )}
\end{itemize}
Que tem pescoço branco.
\section{Albicorque}
\begin{itemize}
\item {Grp. gram.:m.}
\end{itemize}
\begin{itemize}
\item {Utilização:Prov.}
\end{itemize}
(V.albricoque)
\section{Álbido}
\begin{itemize}
\item {Grp. gram.:adj.}
\end{itemize}
\begin{itemize}
\item {Proveniência:(Lat. \textunderscore albidus\textunderscore )}
\end{itemize}
Tirante a branco, esbranquiçado. Cf. Castilho, \textunderscore Fastos\textunderscore , II, 492.
\section{Albificação}
\begin{itemize}
\item {Grp. gram.:f.}
\end{itemize}
Acto de \textunderscore albificar\textunderscore .
\section{Albificar}
\begin{itemize}
\item {Grp. gram.:v. t.}
\end{itemize}
\begin{itemize}
\item {Proveniência:(Do lat. \textunderscore albus\textunderscore  + \textunderscore facere\textunderscore )}
\end{itemize}
Tornar alvo, branco; branquear.
\section{Albiflor}
\begin{itemize}
\item {Grp. gram.:adj.}
\end{itemize}
\begin{itemize}
\item {Proveniência:(Do lat. \textunderscore albus\textunderscore  + \textunderscore flos\textunderscore , \textunderscore floris\textunderscore )}
\end{itemize}
Que dá flôres brancas.
\section{Albigenses}
\begin{itemize}
\item {Grp. gram.:m. pl.}
\end{itemize}
\begin{itemize}
\item {Proveniência:(De \textunderscore Albiga\textunderscore , n. p. lat. de \textunderscore Albi\textunderscore )}
\end{itemize}
Partido ou seita politico-religiosa, que se diffundiu ao sul da França, mormente em Albí.
\section{Albina}
\begin{itemize}
\item {Grp. gram.:f.}
\end{itemize}
Mineral opaco e terroso da Bohêmia.
\section{Albinágio}
\begin{itemize}
\item {Grp. gram.:m.}
\end{itemize}
\begin{itemize}
\item {Utilização:Jur.}
\end{itemize}
\begin{itemize}
\item {Proveniência:(It. \textunderscore albinaggio\textunderscore )}
\end{itemize}
Direitos, que o Fisco recebia outrora da successão de estrangeiros.
Acto de o Fisco succeder ao estrangeiro, que morre sem testar e sem deixar herdeiros necessários.
\section{Albínia}
\begin{itemize}
\item {Grp. gram.:f.}
\end{itemize}
Insecto díptero, (\textunderscore albinia bucaclis\textunderscore ).
\section{Albinismo}
\begin{itemize}
\item {Grp. gram.:m.}
\end{itemize}
\begin{itemize}
\item {Proveniência:(De \textunderscore albino\textunderscore )}
\end{itemize}
Anomalia orgânica, caracterizada pela deminuição ou ausência da matéria corante da pelle, cabellos e olhos.
Doença das plantas, que lhes torna brancas as partes verdes.
\section{Albino}
\begin{itemize}
\item {Grp. gram.:m.  e  adj.}
\end{itemize}
\begin{itemize}
\item {Proveniência:(Do lat. \textunderscore albus\textunderscore )}
\end{itemize}
Indivíduo que tem \textunderscore albinismo\textunderscore .
\section{Albiões}
\begin{itemize}
\item {Grp. gram.:m. pl.}
\end{itemize}
Antigos habitadores da Inglaterra.
\section{Albiona}
\begin{itemize}
\item {Grp. gram.:f.}
\end{itemize}
Verme annélido, semelhante á sanguesuga.
\section{Albirosado}
\begin{itemize}
\item {fónica:ro}
\end{itemize}
\begin{itemize}
\item {Grp. gram.:adj.}
\end{itemize}
Que tem côr intermédia ao branco e ao rosado.
\section{Albirostro}
\begin{itemize}
\item {fónica:rós}
\end{itemize}
\begin{itemize}
\item {Grp. gram.:adj.}
\end{itemize}
\begin{itemize}
\item {Proveniência:(Do lat. \textunderscore albus\textunderscore  + \textunderscore rostrum\textunderscore )}
\end{itemize}
Que tem o bico ou o focinho branco.
\section{Albirrosado}
\begin{itemize}
\item {Grp. gram.:adj.}
\end{itemize}
Que tem côr intermédia ao branco e ao rosado.
\section{Albirrostro}
\begin{itemize}
\item {Grp. gram.:adj.}
\end{itemize}
\begin{itemize}
\item {Proveniência:(Do lat. \textunderscore albus\textunderscore  + \textunderscore rostrum\textunderscore )}
\end{itemize}
Que tem o bico ou o focinho branco.
\section{Albite}
\begin{itemize}
\item {Grp. gram.:f.}
\end{itemize}
\begin{itemize}
\item {Proveniência:(Do lat. \textunderscore albus\textunderscore )}
\end{itemize}
Uma das espécies dos feldspathos.
\section{Albízia}
\begin{itemize}
\item {Grp. gram.:f.}
\end{itemize}
Árvore leguminosa da Índia, (\textunderscore albízia Lebbek\textunderscore , Roxb.).
\section{Albo}
\begin{itemize}
\item {Grp. gram.:m.}
\end{itemize}
Espécie de salmão, (\textunderscore salmo albula\textunderscore ).
\section{Albói}
\begin{itemize}
\item {Grp. gram.:m.}
\end{itemize}
\begin{itemize}
\item {Utilização:Náut.}
\end{itemize}
Pequena abertura, por onde o ar e a luz penetram nas cobertas dos navios.
\section{Albóio}
\begin{itemize}
\item {Grp. gram.:m.}
\end{itemize}
\begin{itemize}
\item {Utilização:Prov.}
\end{itemize}
\begin{itemize}
\item {Utilização:minh.}
\end{itemize}
O mesmo que \textunderscore alpendre\textunderscore .
Casa grande, mas desprezada ou abandonada.
\section{Alboque}
\begin{itemize}
\item {Grp. gram.:m.}
\end{itemize}
\begin{itemize}
\item {Utilização:Ant.}
\end{itemize}
\begin{itemize}
\item {Proveniência:(Do ár. \textunderscore al-boque\textunderscore )}
\end{itemize}
Instrumento pastoril, de sopro.
\section{Albor}
\begin{itemize}
\item {Grp. gram.:m.}
\end{itemize}
(V.alvor)
\section{Alborcar}
\begin{itemize}
\item {Grp. gram.:v. t.}
\end{itemize}
Receber ou entregar por alborque; trocar. Cf. Castilho, \textunderscore Lírica de Anacreonte\textunderscore .
\section{Albornó}
\begin{itemize}
\item {Grp. gram.:m.}
\end{itemize}
\begin{itemize}
\item {Proveniência:(Do ár. \textunderscore abornoz\textunderscore )}
\end{itemize}
Manto com capuz, usado primeiro pelos árabes, e depois na Europa.
Casaco largo, com capuz ou golla grande.
\section{Albóroque}
\begin{itemize}
\item {Grp. gram.:m.}
\end{itemize}
\begin{itemize}
\item {Proveniência:(Do ar. \textunderscore alboroc\textunderscore )}
\end{itemize}
Refeição, que se dá, quando se faz um contrato.
\section{Alborque}
\begin{itemize}
\item {Grp. gram.:m.}
\end{itemize}
\begin{itemize}
\item {Utilização:Pop.}
\end{itemize}
Permutação; escambo.
Copo de vinho que, na occasião do contrato, qualquer dos contratantes, especialmente o comprador, dá ás pessoas presentes, como sancção do negócio.
(Contr. de \textunderscore albóroque\textunderscore )
\section{Albrecha}
\begin{itemize}
\item {Grp. gram.:f.}
\end{itemize}
\begin{itemize}
\item {Utilização:Prov.}
\end{itemize}
\begin{itemize}
\item {Utilização:trasm.}
\end{itemize}
Espécie de pêssego ou damasco, talvez o mesmo que \textunderscore alperche\textunderscore .
\section{Albricoque}
\begin{itemize}
\item {Grp. gram.:m.}
\end{itemize}
\begin{itemize}
\item {Proveniência:(Do ár. \textunderscore al-barcoque\textunderscore )}
\end{itemize}
Damasco, fruto do albricoqueiro.
\section{Albricoqueiro}
\begin{itemize}
\item {Grp. gram.:m.}
\end{itemize}
Árvore, que dá albricoques; espécie de damasqueiro.
\section{Albuca}
\begin{itemize}
\item {Grp. gram.:m.}
\end{itemize}
\begin{itemize}
\item {Proveniência:(Do lat. \textunderscore albus\textunderscore )}
\end{itemize}
Planta bulbosa, da fam. das liliáceas.
\section{Albudieca}
\begin{itemize}
\item {Grp. gram.:f.}
\end{itemize}
Espécie de melão.
(Ár. \textunderscore albilica\textunderscore )
\section{Albufeira}
\begin{itemize}
\item {Grp. gram.:f.}
\end{itemize}
\begin{itemize}
\item {Utilização:Prov.}
\end{itemize}
\begin{itemize}
\item {Utilização:alent.}
\end{itemize}
Lago, formado pelas águas do mar.
Água ruça de azeitonas.
Represa artificial de águas correntes ou pluviaes, para irrigação.
(Cast. \textunderscore albuhera\textunderscore )
\section{Albugem}
\begin{itemize}
\item {Grp. gram.:f.}
\end{itemize}
\begin{itemize}
\item {Proveniência:(Lat. \textunderscore albugo\textunderscore )}
\end{itemize}
Mancha esbranquiçada, que se fórma nos olhos; belida.
\section{Albuginado}
\begin{itemize}
\item {Grp. gram.:adj.}
\end{itemize}
(V.albugíneo)
\section{Albugínea}
\begin{itemize}
\item {Grp. gram.:f.}
\end{itemize}
\begin{itemize}
\item {Proveniência:(Lat. \textunderscore albuginea\textunderscore )}
\end{itemize}
Esclerótica; a parte branca do globo ocular.
\section{Albugíneo}
\begin{itemize}
\item {Grp. gram.:adj.}
\end{itemize}
O mesmo que \textunderscore albuginoso\textunderscore .
\section{Albuginite}
\begin{itemize}
\item {Grp. gram.:f.}
\end{itemize}
\begin{itemize}
\item {Proveniência:(De \textunderscore albugineo\textunderscore )}
\end{itemize}
Phlegmasïa aguda ou chrónica do tecido albugíneo ou fibroso.
\section{Albuginoso}
\begin{itemize}
\item {Grp. gram.:adj.}
\end{itemize}
\begin{itemize}
\item {Grp. gram.:adj.}
\end{itemize}
\begin{itemize}
\item {Proveniência:(Lat. \textunderscore albuginosus\textunderscore )}
\end{itemize}
Esbranquiçado.
Que, pela côr, mostra ter icterícia.
\section{Albugo}
\begin{itemize}
\item {Grp. gram.:m.}
\end{itemize}
O mesmo que \textunderscore albugem\textunderscore .
\section{Álbum}
\begin{itemize}
\item {Grp. gram.:m.}
\end{itemize}
\begin{itemize}
\item {Utilização:Ant.}
\end{itemize}
\begin{itemize}
\item {Proveniência:(Lat. \textunderscore album\textunderscore )}
\end{itemize}
Livro, cujas fôlhas são molduras de cartão para guardar retratos.
Livro, cujas fôlhas são destinadas a desenhos, versos, lembranças de família e de amigos.
Tábua branca, em que se publicavam os edictos do pretor romano.
\section{Albume}
\begin{itemize}
\item {Grp. gram.:m.}
\end{itemize}
\begin{itemize}
\item {Proveniência:(Lat. \textunderscore albumen\textunderscore )}
\end{itemize}
Substância, que envolve e alimenta o embryão, em algumas sementes.
Clara do ovo.
\section{Albumen}
\begin{itemize}
\item {Grp. gram.:m.}
\end{itemize}
\begin{itemize}
\item {Proveniência:(Lat. \textunderscore albumen\textunderscore )}
\end{itemize}
Substância, que envolve e alimenta o embryão, em algumas sementes.
Clara do ovo.
\section{Albumina}
\begin{itemize}
\item {Grp. gram.:f.}
\end{itemize}
\begin{itemize}
\item {Proveniência:(Do lat. \textunderscore albumen\textunderscore )}
\end{itemize}
Um dos princípios immediatos dos corpos organizados, e que tem a propriedade de se coagular com o calor.--É matéria viscosa e esbranquiçada, composta de carbono, hydrogênio, azote, oxygênio, phósphoro e enxôfre.
\section{Albuminado}
\begin{itemize}
\item {Grp. gram.:adj.}
\end{itemize}
Que contém albumina.
\section{Albuminato}
\begin{itemize}
\item {Grp. gram.:m.}
\end{itemize}
Combinação da albumina com outro corpo.
\section{Albuminiforme}
\begin{itemize}
\item {Grp. gram.:adj.}
\end{itemize}
Semelhante á albumina.
\section{Albuminímetro}
\begin{itemize}
\item {Grp. gram.:m.}
\end{itemize}
\begin{itemize}
\item {Proveniência:(De \textunderscore albumina\textunderscore  + gr. \textunderscore metron\textunderscore )}
\end{itemize}
Instrumento de polarização, para determinar a quantidade de albumina contida num líquido.
\section{Albuminina}
\begin{itemize}
\item {Grp. gram.:f.}
\end{itemize}
\begin{itemize}
\item {Proveniência:(De \textunderscore albumina\textunderscore )}
\end{itemize}
Substância, que se separa da clara do ovo, conservando-se esta, durante um mês aproximadamente, numa temperatura inferior a zero.
\section{Albuminóide}
\begin{itemize}
\item {Grp. gram.:adj.}
\end{itemize}
\begin{itemize}
\item {Proveniência:(De \textunderscore albumina\textunderscore  + gr. \textunderscore eidos\textunderscore )}
\end{itemize}
Diz-se dos corpos, que têm a natureza de albumina, como a fibrina e a caseína.
\section{Albuminómetro}
\begin{itemize}
\item {Grp. gram.:m.}
\end{itemize}
Instrumento de Chímica.
\section{Albuminose}
\begin{itemize}
\item {Grp. gram.:f.}
\end{itemize}
O mesmo que \textunderscore peptona\textunderscore .
\section{Albuminoso}
\begin{itemize}
\item {Grp. gram.:adj.}
\end{itemize}
Que tem albumina.
\section{Albuminuria}
\begin{itemize}
\item {Grp. gram.:f.}
\end{itemize}
\begin{itemize}
\item {Proveniência:(De \textunderscore albumina\textunderscore  + gr. \textunderscore ourein\textunderscore )}
\end{itemize}
Doença, caracterizada pela existência de albumina em as urinas.
\section{Albuminúrico}
\begin{itemize}
\item {Grp. gram.:adj.}
\end{itemize}
\begin{itemize}
\item {Grp. gram.:M.}
\end{itemize}
Relativo á albuminuria.
Aquelle que padece albuminuria.
\section{Albúnea}
\begin{itemize}
\item {Grp. gram.:f.}
\end{itemize}
\begin{itemize}
\item {Proveniência:(De \textunderscore Albunea\textunderscore , n. p.)}
\end{itemize}
Crustáceo, da ordem dos decápodes.
\section{Alburno}
\begin{itemize}
\item {Grp. gram.:m.}
\end{itemize}
\begin{itemize}
\item {Proveniência:(Lat. \textunderscore alburnum\textunderscore )}
\end{itemize}
Entrecasco da árvore.
Camada mais exterior do lenho das árvores e arbustos.
\section{Alca}
\begin{itemize}
\item {Grp. gram.:f.}
\end{itemize}
Nome scientifico de penguim.
\section{Alça}
\begin{itemize}
\item {Grp. gram.:f.}
\end{itemize}
\begin{itemize}
\item {Utilização:Mil.}
\end{itemize}
\begin{itemize}
\item {Utilização:Ant.}
\end{itemize}
\begin{itemize}
\item {Utilização:Ant.}
\end{itemize}
\begin{itemize}
\item {Grp. gram.:Pl.}
\end{itemize}
\begin{itemize}
\item {Utilização:Prov.}
\end{itemize}
\begin{itemize}
\item {Utilização:trasm.}
\end{itemize}
\begin{itemize}
\item {Utilização:Prov.}
\end{itemize}
\begin{itemize}
\item {Proveniência:(De \textunderscore alçar\textunderscore )}
\end{itemize}
Cada uma das duas tíras que seguram as calças, passando por cima do ombro; suspensório.
Asa ou puxadeira.
Pedaço de sola, que os sapateiros adaptam ás fôrmas, para as tornar mais altas.
Peça móvel, graduada, nas espingardas de guerra, para, de combinação com o ponto de mira, se regular o alcance do tiro.
Pequena régua, com que em artilharia se póde variar o ângulo que a alma faz com a linha de mira.
Argola de corda, para cingir qualquer peça do poleame, a bórdo.
Presente, donativo. Cf. Rebelo, \textunderscore Mocidade\textunderscore , II, 22.
Recurso, appellação.
Resguardos de ponte.
Saldo positivo. (Colhido em Turquel)--Na accepção de suspensório das calças, algumas provincias só usam o plural da palavra.
\section{Alcabol}
\begin{itemize}
\item {Grp. gram.:m.}
\end{itemize}
O mesmo que \textunderscore alcabrós\textunderscore .
\section{Alcabós}
\begin{itemize}
\item {Grp. gram.:m.}
\end{itemize}
Nome, que os pescadores de Setúbal dão a um peixe esverdeado, a que na costa do norte chamam \textunderscore cabrão\textunderscore .
O mesmo que \textunderscore cabós\textunderscore .
\section{Alcabrós}
\begin{itemize}
\item {Grp. gram.:m.}
\end{itemize}
Nome, que os pescadores de Setúbal dão a um peixe esverdeado, a que na costa do norte chamam \textunderscore cabrão\textunderscore .
O mesmo que \textunderscore cabós\textunderscore .
\section{Alçação}
\begin{itemize}
\item {Grp. gram.:f.}
\end{itemize}
\begin{itemize}
\item {Utilização:Typ.}
\end{itemize}
\begin{itemize}
\item {Proveniência:(De \textunderscore alçar\textunderscore )}
\end{itemize}
Arte de contar, separar e dobrar os exemplares de cada fôlha impressa.
\section{Alcáçar}
\begin{itemize}
\item {Grp. gram.:m.}
\end{itemize}
O mesmo que \textunderscore alcácer\textunderscore .
\section{Alcaçarel}
\begin{itemize}
\item {Grp. gram.:m.}
\end{itemize}
\begin{itemize}
\item {Utilização:Ant.}
\end{itemize}
O mesmo que \textunderscore alcácer\textunderscore .
\section{Alcaçaria}
\begin{itemize}
\item {Grp. gram.:f.}
\end{itemize}
\begin{itemize}
\item {Proveniência:(Do ár. \textunderscore al-caiçaria\textunderscore )}
\end{itemize}
Fábrica de curtir pelles.
Arruamento de lojas.
Alojamento.
Alcácer.
Lugar, onde os Judeus podiam comprar ou vender gêneros.
\section{Alcacel}
\begin{itemize}
\item {Grp. gram.:m.}
\end{itemize}
\begin{itemize}
\item {Utilização:Prov.}
\end{itemize}
\begin{itemize}
\item {Utilização:trasm.}
\end{itemize}
Cebolas novas, pimenteiros, tomateiros, etc., que se compram para plantar nas hortas.
(Cp. \textunderscore alcacêr\textunderscore )
\section{Alcacel}
\begin{itemize}
\item {Grp. gram.:m.}
\end{itemize}
(V. \textunderscore alcacêr\textunderscore , que é como o povo diz)
\section{Alcácema}
\begin{itemize}
\item {Grp. gram.:f.}
\end{itemize}
\begin{itemize}
\item {Utilização:Ant.}
\end{itemize}
Câmara, em que se recolhiam os tripulantes de caravela, adeante do camarote do mestre.
\section{Alcácer}
\begin{itemize}
\item {Grp. gram.:m.}
\end{itemize}
\begin{itemize}
\item {Utilização:Ant.}
\end{itemize}
\begin{itemize}
\item {Proveniência:(Do ár. \textunderscore al-caçr\textunderscore )}
\end{itemize}
Castello; fortaleza.
Palácio; habitação sumptuosa.
\section{Alcacêr}
\begin{itemize}
\item {Grp. gram.:m.}
\end{itemize}
\begin{itemize}
\item {Utilização:Prov.}
\end{itemize}
\begin{itemize}
\item {Utilização:alent.}
\end{itemize}
\begin{itemize}
\item {Utilização:Açor}
\end{itemize}
\begin{itemize}
\item {Proveniência:(Do ár. \textunderscore al-cacil\textunderscore )}
\end{itemize}
Sementeira de aveia ou cevada, para alimento de animaes.
Terreno, em que cresce trigo, cevada ou centeio.
Terra húmida, com erva para forragens. (Colhido na Graciosa)
\section{Alcacereiro}
\begin{itemize}
\item {Grp. gram.:m.}
\end{itemize}
Guarda de alcácer.
\section{Alcacereno}
\begin{itemize}
\item {Grp. gram.:adj.}
\end{itemize}
O mesmo que \textunderscore alcacerense\textunderscore .
\section{Alcacerense}
\begin{itemize}
\item {Grp. gram.:adj.}
\end{itemize}
\begin{itemize}
\item {Grp. gram.:M.}
\end{itemize}
\begin{itemize}
\item {Proveniência:(De \textunderscore Alcácer\textunderscore , n. p.)}
\end{itemize}
Relativo a Alcácer.
Aquelle que nasceu em Alcácer.
\section{Alcachinado}
\begin{itemize}
\item {Grp. gram.:adj.}
\end{itemize}
\begin{itemize}
\item {Utilização:Pop.}
\end{itemize}
Abatido.
Curvado.
\section{Alcachinar}
\begin{itemize}
\item {Grp. gram.:v. t.}
\end{itemize}
\begin{itemize}
\item {Utilização:Pop.}
\end{itemize}
Curvar, abater.
\section{Alcachofa}
\begin{itemize}
\item {Grp. gram.:f.}
\end{itemize}
O mesmo que \textunderscore alcachofra\textunderscore .
Ornato architectónico, em fórma do pinha.
\section{Alcachofra}
\begin{itemize}
\item {Grp. gram.:f.}
\end{itemize}
Planta hortense, da fam. das compostas.
(Cast. \textunderscore alcachofa\textunderscore )
\section{Alcachofral}
\begin{itemize}
\item {Grp. gram.:m.}
\end{itemize}
Lugar, onde crescem alcachofras.
\section{Alcachofrar}
\begin{itemize}
\item {Grp. gram.:v. t.}
\end{itemize}
Tornar semelhante á alcachofra.
Tornar áspero, crespo.
Fazer bordados crespos em.
\section{Alcáçova}
\begin{itemize}
\item {Grp. gram.:f.}
\end{itemize}
\begin{itemize}
\item {Utilização:Prov.}
\end{itemize}
\begin{itemize}
\item {Utilização:Ant.}
\end{itemize}
\begin{itemize}
\item {Proveniência:(Do ár. \textunderscore al-caçaba\textunderscore )}
\end{itemize}
Castello antigo, fortaleza.
Cova, lapa.
Castello de nau de guerra.
\section{Alçacu}
\begin{itemize}
\item {Grp. gram.:m.}
\end{itemize}
\begin{itemize}
\item {Utilização:Zool.}
\end{itemize}
Espécie de mergulhão, (\textunderscore pediceps minor\textunderscore , Bris.).
\section{Alçacuello}
\begin{itemize}
\item {Grp. gram.:m.}
\end{itemize}
\begin{itemize}
\item {Utilização:Ant.}
\end{itemize}
Collar alto, usado por homens e mulheres, para erguer o rosto, e correspondente á parte levantada do actual cabeção ou volta de clérigo.
(Cast. \textunderscore alzacuello\textunderscore )
\section{Alcaçuz}
\begin{itemize}
\item {Grp. gram.:m.}
\end{itemize}
\begin{itemize}
\item {Proveniência:(Do ár. \textunderscore arc essus\textunderscore )}
\end{itemize}
Planta leguminosa, de raiz amarela e doce.
Raiz dessa planta.
\section{Alcada}
\begin{itemize}
\item {Grp. gram.:f.}
\end{itemize}
\begin{itemize}
\item {Utilização:ant.}
\end{itemize}
\begin{itemize}
\item {Utilização:Gír.}
\end{itemize}
Carapuça.
\section{Alçada}
\begin{itemize}
\item {Grp. gram.:f.}
\end{itemize}
\begin{itemize}
\item {Utilização:Ant.}
\end{itemize}
\begin{itemize}
\item {Proveniência:(De \textunderscore alçar\textunderscore )}
\end{itemize}
Competência, jurisdicção.
Limite da acção ou influência de alguém.
Tribunal collectivo e ambulante que, percorrendo os povos, lhes administrava justiça.
\section{Alcadafe}
\begin{itemize}
\item {Grp. gram.:m.}
\end{itemize}
\begin{itemize}
\item {Proveniência:(Do ár. \textunderscore al-kodaf\textunderscore )}
\end{itemize}
Vaso ou selha, sôbre que o taberneiro mede o vinho e que recebe as verteduras.
\section{Alcade}
\begin{itemize}
\item {Grp. gram.:m.}
\end{itemize}
\begin{itemize}
\item {Utilização:Prov.}
\end{itemize}
\begin{itemize}
\item {Utilização:alent.}
\end{itemize}
Pequeno pássaro conirosto, espécie de picanço, insectívoro e bulhento, que se atreve com aves de maior corpo.
(Talvez do ár. \textunderscore al-cadi\textunderscore , donde o cast. \textunderscore alcalde\textunderscore )
\section{Alcádeas}
\begin{itemize}
\item {Grp. gram.:f. pl.}
\end{itemize}
O mesmo que \textunderscore alcades\textunderscore .
\section{Alcadefe}
\begin{itemize}
\item {Grp. gram.:m.}
\end{itemize}
\begin{itemize}
\item {Utilização:T. de Serpa}
\end{itemize}
Vasilha de barro.
\section{Alcadefe}
\begin{itemize}
\item {Grp. gram.:m.}
\end{itemize}
\begin{itemize}
\item {Proveniência:(Do ár. \textunderscore al-kodaf\textunderscore )}
\end{itemize}
Vaso ou selha, sôbre que o taberneiro mede o vinho e que recebe as verteduras.
\section{Alcades}
\begin{itemize}
\item {Grp. gram.:m. pl.}
\end{itemize}
\begin{itemize}
\item {Proveniência:(De \textunderscore alca\textunderscore  + gr. \textunderscore eidos\textunderscore )}
\end{itemize}
Família de aves, da ordem dos palmipedes, e que têm por typo o gênero alca
\textunderscore ou\textunderscore  penguim.
\section{Alçado}
\begin{itemize}
\item {Grp. gram.:m.}
\end{itemize}
\begin{itemize}
\item {Utilização:Bras}
\end{itemize}
\begin{itemize}
\item {Proveniência:(De \textunderscore alçar\textunderscore )}
\end{itemize}
Projecção vertical, traçado, planta.
Casa ou compartimento typographico, em que se alçam as fôlhas que saem úmidas do prelo.
Diz-se do gado bravo ou que ainda não foi domesticado.
\section{Alçador}
\begin{itemize}
\item {Grp. gram.:m.}
\end{itemize}
Aquelle que alça.
\section{Alçadura}
\begin{itemize}
\item {Grp. gram.:f.}
\end{itemize}
Acto de \textunderscore alçar\textunderscore .
\section{Alçagem}
\begin{itemize}
\item {Grp. gram.:f.}
\end{itemize}
Acto de alçar (fôlhas impressas).
\section{Alcagoita}
\begin{itemize}
\item {Grp. gram.:f.}
\end{itemize}
\begin{itemize}
\item {Utilização:Prov.}
\end{itemize}
\begin{itemize}
\item {Utilização:alg.}
\end{itemize}
Amendoim, fruto.
\section{Alcagote}
\begin{itemize}
\item {Grp. gram.:m.}
\end{itemize}
O mesmo que \textunderscore alcaiote\textunderscore .
(Cp. \textunderscore alcaguete\textunderscore )
\section{Alcaguete}
\begin{itemize}
\item {fónica:gu-ê}
\end{itemize}
\begin{itemize}
\item {Grp. gram.:m.}
\end{itemize}
\begin{itemize}
\item {Utilização:Bras. do S}
\end{itemize}
Alcoviteiro; alcaiote.
(Cast. \textunderscore alcahuete\textunderscore )
\section{Alcaiata}
\begin{itemize}
\item {Grp. gram.:f.}
\end{itemize}
Pequeno utensílio de ferro, ligeiramente cónico, us. por sirgueiros.
\section{Alcaicha}
\begin{itemize}
\item {Grp. gram.:f.}
\end{itemize}
Faixa do costado do navio.
Espaço entre as verdugas e cintas, por fóra dos navios.
Uma ou mais ordens de debrum branco, no collarinho das camisas dos marinheiros.
\section{Alcaico}
\begin{itemize}
\item {Grp. gram.:adj.}
\end{itemize}
\begin{itemize}
\item {Proveniência:(Gr. \textunderscore alkaikos\textunderscore )}
\end{itemize}
Diz-se do verso grego hendecassýllabo.
E diz-se da estrophe de quatro versos, sendo alcaicos os dois primeiros.
\section{Alcaidaria}
\begin{itemize}
\item {Grp. gram.:f.}
\end{itemize}
\begin{itemize}
\item {Grp. gram.:f.}
\end{itemize}
Dignidade de alcaide; funcções de alcaide.
Lugar, onde o alcaide exerce a jurisdicção.
Tributo de dois dinheiros, que se pagava ao alcaide de Alenquer, por cada carga de peixe que ia ao mercado da terra.
\section{Alcaide}
\begin{itemize}
\item {Grp. gram.:m.}
\end{itemize}
\begin{itemize}
\item {Utilização:Ant.}
\end{itemize}
\begin{itemize}
\item {Utilização:Bras}
\end{itemize}
\begin{itemize}
\item {Grp. gram.:Pl.}
\end{itemize}
\begin{itemize}
\item {Utilização:T. do Porto}
\end{itemize}
Antigo governador de castello ou província.
Official de justiça.
Em Espanha, autoridade administrativa.
Cada um dos remadores de uma fusta.
Fazenda, que não tem extracção, que se não vende.
Conjunto de objectos vários e insignificantes de uma tanoaria.
(Do ár.)
\section{Alcaidessa}
\begin{itemize}
\item {fónica:dê}
\end{itemize}
\begin{itemize}
\item {Grp. gram.:f.}
\end{itemize}
Mulher de alcaide.
\section{Alcaidina}
\begin{itemize}
\item {Grp. gram.:f.}
\end{itemize}
\begin{itemize}
\item {Utilização:Ant.}
\end{itemize}
O mesmo que \textunderscore alcaidessa\textunderscore .
\section{Alcaiota}
\textunderscore fem.\textunderscore  de \textunderscore alcaiote\textunderscore .
\section{Alcaiotaria}
\begin{itemize}
\item {Grp. gram.:f.}
\end{itemize}
Officio de alcaiote.
\section{Alcaiote}
\begin{itemize}
\item {Grp. gram.:m.}
\end{itemize}
\begin{itemize}
\item {Proveniência:(Do ár. \textunderscore al-cauad\textunderscore )}
\end{itemize}
Alcoviteiro.
\section{Alcaiotismo}
\begin{itemize}
\item {Grp. gram.:m.}
\end{itemize}
Acções ou modos de alcaiote:«\textunderscore ...alcaiotismo dos amorios estampados e atirados a milhares de leitores\textunderscore ». Camillo, \textunderscore Nov. do Minho\textunderscore , IX, 7.
\section{Alcaixa}
\begin{itemize}
\item {Grp. gram.:f.}
\end{itemize}
\begin{itemize}
\item {Grp. gram.:f.}
\end{itemize}
Faixa do costado do navio.
Espaço entre as verdugas e cintas, por fóra dos navios.
Uma ou mais ordens de debrum branco, no collarinho das camisas dos marinheiros.
\section{Alcaiz}
\begin{itemize}
\item {Grp. gram.:m.}
\end{itemize}
\begin{itemize}
\item {Utilização:Ant.}
\end{itemize}
Livro de recenseamento.
(Do mesmo rad. que \textunderscore alcaidaria\textunderscore ?)
\section{Alcala}
\begin{itemize}
\item {Grp. gram.:f.}
\end{itemize}
\begin{itemize}
\item {Utilização:Ant.}
\end{itemize}
\begin{itemize}
\item {Proveniência:(Do ár. \textunderscore al-quila\textunderscore )}
\end{itemize}
Fio de linha, com que se cosem as redes da pescada.
Espécie de alfaia ou vestido, dado por um principe.
Espécie de pano de armar ou de pano de arrás.
\section{Alçalá}
\begin{itemize}
\item {Grp. gram.:m.}
\end{itemize}
Antigo vaso de barro, em que se dava água, nas portarias dos conventos.
(Talvez do imp. \textunderscore alça\textunderscore , de \textunderscore alçar\textunderscore  + \textunderscore lá\textunderscore )
\section{Alcalada}
\textunderscore f. Ant.\textunderscore  (?)«\textunderscore Sim, biringelas há na praça, alcaladas há na villa\textunderscore ». \textunderscore Eufrosina\textunderscore , V, 2.
\section{Alcaldada}
\begin{itemize}
\item {Grp. gram.:f.}
\end{itemize}
\begin{itemize}
\item {Utilização:Prov.}
\end{itemize}
\begin{itemize}
\item {Utilização:trasm.}
\end{itemize}
\begin{itemize}
\item {Proveniência:(Do cast. \textunderscore alcalde\textunderscore ?)}
\end{itemize}
Notícia extraordinária.
Balela.
Lembrança exótica.
\section{Alcaldamento}
\begin{itemize}
\item {Grp. gram.:m.}
\end{itemize}
\begin{itemize}
\item {Utilização:Ant.}
\end{itemize}
\begin{itemize}
\item {Proveniência:(De \textunderscore alcaldar\textunderscore )}
\end{itemize}
Imposto alfandegário.
\section{Alcaldar}
\begin{itemize}
\item {Grp. gram.:v. t.}
\end{itemize}
\begin{itemize}
\item {Utilização:Ant.}
\end{itemize}
\begin{itemize}
\item {Proveniência:(Do cast. \textunderscore alcalde\textunderscore )}
\end{itemize}
Manifestar na alfândega (mercadorias sujeitas ao imposto do trânsito), para pagar auto do imposto.
\section{Alcalde}
\begin{itemize}
\item {Grp. gram.:m.}
\end{itemize}
\begin{itemize}
\item {Utilização:Ant.}
\end{itemize}
Juiz do concelho, alvazil. Cf. Herculano, \textunderscore Hist. de Port.\textunderscore , IV, 45 e 123.
(Cast. \textunderscore alcalde\textunderscore )
\section{Alcalescência}
\begin{itemize}
\item {Grp. gram.:f.}
\end{itemize}
\begin{itemize}
\item {Proveniência:(De \textunderscore alcalescente\textunderscore )}
\end{itemize}
Passagem ao estado alcalino.
\section{Alcalescente}
\begin{itemize}
\item {Grp. gram.:adj.}
\end{itemize}
\begin{itemize}
\item {Proveniência:(De \textunderscore álcali\textunderscore )}
\end{itemize}
Que tem propriedades alcalinas.
Que passa ao estado alcalino.
\section{Alcali}
\begin{itemize}
\item {Grp. gram.:m.}
\end{itemize}
\begin{itemize}
\item {Proveniência:(Do ár. \textunderscore al-cali\textunderscore )}
\end{itemize}
Grupo de corpos compostos, como a soda, a lithina, a cal.
Planta marinha, de que se extrai álcali.
\section{Alcalí}
\begin{itemize}
\item {Grp. gram.:m.}
\end{itemize}
\begin{itemize}
\item {Proveniência:(Do ár. \textunderscore al-cali\textunderscore )}
\end{itemize}
Grupo de corpos compostos, como a soda, a lithina, a cal.
Planta marinha, de que se extrai álcali.
\section{Alcalificante}
\begin{itemize}
\item {Grp. gram.:adj.}
\end{itemize}
\begin{itemize}
\item {Proveniência:(De \textunderscore alcalificar\textunderscore )}
\end{itemize}
Que manifesta propriedades alcalinas em outra substância.
\section{Alcalificar}
\begin{itemize}
\item {Grp. gram.:v. t.}
\end{itemize}
\begin{itemize}
\item {Proveniência:(De \textunderscore álcali\textunderscore  + lat. \textunderscore facere\textunderscore )}
\end{itemize}
Produzir em (uma substância) propriedades alcalinas.
\section{Alcalígeno}
\begin{itemize}
\item {Grp. gram.:adj.}
\end{itemize}
\begin{itemize}
\item {Proveniência:(De \textunderscore álcali\textunderscore  + gr. \textunderscore gennao\textunderscore )}
\end{itemize}
Que produz álcalis.
\section{Alcalimetria}
\begin{itemize}
\item {Grp. gram.:f.}
\end{itemize}
\begin{itemize}
\item {Proveniência:(De \textunderscore alcalímetro\textunderscore )}
\end{itemize}
Processo, com que se determina a proporção de álcali contido nas sodas ou potassas do commércio.
\section{Alcalimétrico}
\begin{itemize}
\item {Grp. gram.:adj.}
\end{itemize}
Relativo á \textunderscore alcalimetria\textunderscore .
\section{Alcalímetro}
\begin{itemize}
\item {Grp. gram.:m.}
\end{itemize}
\begin{itemize}
\item {Proveniência:(De \textunderscore álcali\textunderscore  + gr. \textunderscore metron\textunderscore )}
\end{itemize}
Instrumento, para medir o álcali contido nas sodas e potassas do commércio.
\section{Alcalinidade}
\begin{itemize}
\item {Grp. gram.:f.}
\end{itemize}
\begin{itemize}
\item {Proveniência:(De \textunderscore alcalino\textunderscore )}
\end{itemize}
Estado de uma substância, que tem propriedades do álcali.
\section{Alcalinizar}
\begin{itemize}
\item {Grp. gram.:v. t.}
\end{itemize}
Tornar alcalino. Cf. \textunderscore Diário do Govêrno\textunderscore , de 5-VI-1908.
\section{Alcalino}
\begin{itemize}
\item {Grp. gram.:adj.}
\end{itemize}
Relativo a \textunderscore álcali\textunderscore .
Que tem álcali.
\section{Alcalização}
\begin{itemize}
\item {Grp. gram.:f.}
\end{itemize}
Acto de \textunderscore alcalizar\textunderscore .
\section{Alcalizar}
\begin{itemize}
\item {Grp. gram.:v. t.}
\end{itemize}
\begin{itemize}
\item {Proveniência:(De \textunderscore álcali\textunderscore )}
\end{itemize}
Deixar em (sal neutro) a parte alcalina, extrahindo-lhe a parte ácida.
\section{Alcalóide}
\begin{itemize}
\item {Grp. gram.:m.}
\end{itemize}
\begin{itemize}
\item {Proveniência:(De \textunderscore álcali\textunderscore  + gr. \textunderscore eidos\textunderscore )}
\end{itemize}
Substância orgânica azotada, que, tendo propriedades alcalinas, neutraliza os saes.
\section{Alçamento}
\begin{itemize}
\item {Grp. gram.:m.}
\end{itemize}
Acto de \textunderscore alçar\textunderscore .
\section{Alcamonia}
\begin{itemize}
\item {Grp. gram.:f.}
\end{itemize}
\begin{itemize}
\item {Proveniência:(Do ár. \textunderscore al-cammon\textunderscore )}
\end{itemize}
Espécie de doce ou de bolos.
\section{Alcânave}
\begin{itemize}
\item {Grp. gram.:adj.}
\end{itemize}
\begin{itemize}
\item {Utilização:Ant.}
\end{itemize}
Relativo ao cânhamo:«\textunderscore não há mais linho alcânave\textunderscore ». \textunderscore Aulegrafia\textunderscore , 78.
(Cp. gr. \textunderscore kannabis\textunderscore , cânhamo)
\section{Alcançadiço}
\begin{itemize}
\item {Grp. gram.:adj.}
\end{itemize}
Que se póde alcançar facilmente.
\section{Alcançador}
\begin{itemize}
\item {Grp. gram.:m.}
\end{itemize}
Aquelle que alcança.
\section{Alcançadura}
\begin{itemize}
\item {Grp. gram.:f.}
\end{itemize}
\begin{itemize}
\item {Proveniência:(De \textunderscore alcançar\textunderscore )}
\end{itemize}
Contusão, que o animal faz em si mesmo, na parte inferior dos membros, tocando com um pé no outro.
\section{Alcançamento}
\begin{itemize}
\item {Grp. gram.:m.}
\end{itemize}
Acto de \textunderscore alcançar\textunderscore .
\section{Alcançar}
\begin{itemize}
\item {Grp. gram.:v. t.}
\end{itemize}
\begin{itemize}
\item {Grp. gram.:V. i.}
\end{itemize}
\begin{itemize}
\item {Proveniência:(De \textunderscore alcance\textunderscore )}
\end{itemize}
Chegar a. Apanhar.
Conseguir.
Avistar: \textunderscore daqui não alcanço o zimbório da Estrêlla\textunderscore .
Entender: \textunderscore alcançar o sentido de um discurso\textunderscore . Prever.
Ficar grávida (a mulher).
\section{Alcâncara}
\begin{itemize}
\item {Grp. gram.:f.}
\end{itemize}
\begin{itemize}
\item {Utilização:Ant.}
\end{itemize}
\begin{itemize}
\item {Utilização:Prov.}
\end{itemize}
\begin{itemize}
\item {Utilização:alg.}
\end{itemize}
Espécie de pandeiro.
Espécie de biscoito, feito do massa de pão e gordura.
\section{Alcancareiro}
\begin{itemize}
\item {Grp. gram.:m.}
\end{itemize}
Tangedor de alcâncara.
\section{Alcançável}
\begin{itemize}
\item {Grp. gram.:adj.}
\end{itemize}
Que se póde alcançar.
\section{Alcance}
\begin{itemize}
\item {Grp. gram.:m.}
\end{itemize}
Acção de alcançar.
Encalço.
Conseguimento.
Intelligência.
Importância.
Desfalque.
Peça, que separa da parede as galerias de cortinas ou reposteiros.
Distância que se alcança.
(Corr. de \textunderscore encalço\textunderscore ? Do lat. \textunderscore ad\textunderscore  + \textunderscore calcem\textunderscore ?)
\section{Alcanchal}
\begin{itemize}
\item {Grp. gram.:m.}
\end{itemize}
\begin{itemize}
\item {Utilização:T. de Borba}
\end{itemize}
Caminho péssimo, intransitável.
\section{Alcanço}
\begin{itemize}
\item {Grp. gram.:m.}
\end{itemize}
\begin{itemize}
\item {Utilização:Pop.}
\end{itemize}
\begin{itemize}
\item {Grp. gram.:m.}
\end{itemize}
\begin{itemize}
\item {Utilização:Pop.}
\end{itemize}
O mesmo que \textunderscore alcance\textunderscore .
O dedo insulado, nos pés das aves de rapina. Cf. Fern. Pereira, \textunderscore Caça de Altan.\textunderscore 
\section{Alcândor}
\begin{itemize}
\item {Grp. gram.:m.}
\end{itemize}
O mesmo que \textunderscore alcândora\textunderscore ; lugar alcantilado. Cf. C. Neto, \textunderscore Saldunes\textunderscore .
\section{Alcândora}
\begin{itemize}
\item {Grp. gram.:f.}
\end{itemize}
Poleiro do falcão.
(Ár. \textunderscore alcandara\textunderscore )
\section{Alcandoradamente}
\begin{itemize}
\item {Grp. gram.:adv.}
\end{itemize}
\begin{itemize}
\item {Proveniência:(De \textunderscore alcandorado\textunderscore )}
\end{itemize}
Em lugar elevado.
\section{Alcandorado}
\begin{itemize}
\item {Grp. gram.:adj.}
\end{itemize}
Collocado em lugar alto.
\section{Alcandorar-se}
\begin{itemize}
\item {Grp. gram.:v. p.}
\end{itemize}
Poisar em alcândora.
Colocar-se alto.
Guindar-se.
\section{Alcanela}
\begin{itemize}
\item {Grp. gram.:f.}
\end{itemize}
Rede, para a pesca da sardinha.
\section{Alcânfor}
\begin{itemize}
\item {Grp. gram.:m.  ou  f.}
\end{itemize}
\begin{itemize}
\item {Utilização:Des.}
\end{itemize}
(V.cânfora)
\section{Alcânfora}
\begin{itemize}
\item {Grp. gram.:m.  ou  f.}
\end{itemize}
\begin{itemize}
\item {Utilização:Des.}
\end{itemize}
(V.cânfora)
\section{Alcanforado}
\begin{itemize}
\item {Grp. gram.:adj.}
\end{itemize}
\begin{itemize}
\item {Utilização:Des.}
\end{itemize}
(V.canforado)
\section{Alcanforeira}
\begin{itemize}
\item {Grp. gram.:f.}
\end{itemize}
\begin{itemize}
\item {Utilização:Bot.}
\end{itemize}
Gênero de lauráceas, que produz o alcânfor.
\section{Alcanforeiro}
\begin{itemize}
\item {Grp. gram.:m.}
\end{itemize}
Vaso, em que se encerra o alcânfor, para cheirar.
\section{Alcântara}
\begin{itemize}
\item {Grp. gram.:f.}
\end{itemize}
\begin{itemize}
\item {Utilização:Des.}
\end{itemize}
\begin{itemize}
\item {Proveniência:(Do ár. \textunderscore al-cantara\textunderscore )}
\end{itemize}
Antiga Ordem militar.
O mesmo que \textunderscore ponte\textunderscore .
\section{Alcantil}
\begin{itemize}
\item {Grp. gram.:m.}
\end{itemize}
Rocha talhada a píque.
Cume.
Sítio alto e escarpado.
\section{Alcantilada}
\begin{itemize}
\item {Grp. gram.:f.}
\end{itemize}
Série de alcantis. Longo despenhadeíro.
\section{Alcantiladamente}
\begin{itemize}
\item {Grp. gram.:adv.}
\end{itemize}
Á maneira de alcantil.
\section{Alcantilado}
\begin{itemize}
\item {Grp. gram.:adj.}
\end{itemize}
Que tem fórma de alcantil.
Talhado a pique; escarpado.
\section{Alcantilar}
\begin{itemize}
\item {Grp. gram.:v. t.}
\end{itemize}
Dar fórma de alcantil a.
Talhar a pique.
\section{Alcantiloso}
\begin{itemize}
\item {Grp. gram.:adj.}
\end{itemize}
(V.alcantilado)
\section{Alcanzia}
\begin{itemize}
\item {Grp. gram.:f.}
\end{itemize}
Bóla de barro, ôca, que se atirava com flôres nas cavalhadas.
Panela de barro, que continha matéria explosiva, e se usava nas guerras antigas.
Mealheiro de barro.
(Cast. \textunderscore alcancia\textunderscore )
\section{Alcanziada}
\begin{itemize}
\item {Grp. gram.:f.}
\end{itemize}
Arremêsso de alcanzia.
\section{Alçapão}
\begin{itemize}
\item {Grp. gram.:m.}
\end{itemize}
\begin{itemize}
\item {Utilização:Bras}
\end{itemize}
\begin{itemize}
\item {Proveniência:(De \textunderscore alçar\textunderscore ?)}
\end{itemize}
Porta ou postigo, que fecha de cima para baixo.
Abertura, que communica um pavimento com outro que lhe fica inferior.
Peça das calças, que, segundo o uso antigo, lhes tapava a abertura anteríor.
\textunderscore Alçapão falso\textunderscore , espécie de portinhola na parte superior de gaiola ou caixa, e que se arma por sí mesma, depois que a victima lhe caiu dentro.
\section{Alcaparra}
\begin{itemize}
\item {Grp. gram.:f.}
\end{itemize}
\begin{itemize}
\item {Proveniência:(Do ár. \textunderscore al-cabar\textunderscore )}
\end{itemize}
Planta hortense, da fam. das capparídeas.
Botão da flôr de alcaparra, applicável como condimento.
\section{Alcaparrado}
\begin{itemize}
\item {Grp. gram.:adj.}
\end{itemize}
Temperado com alcaparra.
Desenfastiado.
\section{Alcaparral}
\begin{itemize}
\item {Grp. gram.:m.}
\end{itemize}
Lugar, onde se criam alcaparras.
\section{Alcaparrar}
\begin{itemize}
\item {Grp. gram.:v. t.}
\end{itemize}
Temperar com alcaparras.
\section{Alcaparreira}
\begin{itemize}
\item {Grp. gram.:f.}
\end{itemize}
O mesmo que \textunderscore alcaparra\textunderscore .
\section{Alçapé}
\begin{itemize}
\item {Grp. gram.:m.}
\end{itemize}
\begin{itemize}
\item {Proveniência:(De \textunderscore alçar\textunderscore  + \textunderscore pé\textunderscore )}
\end{itemize}
Armadilha para caça.
Acto traiçoeiro do lutador, que mete o pé entre as pernas do adversário, para o fazer cair.
Artifício doloso.
\section{Alçaprema}
\begin{itemize}
\item {Grp. gram.:f.}
\end{itemize}
\begin{itemize}
\item {Proveniência:(De \textunderscore alça\textunderscore  + lat. \textunderscore premere\textunderscore )}
\end{itemize}
Alavanca.
Trave ou barrote, a pino, escorando alguma coisa.
Aboiz.
Tenaz de dentista.
Instrumento de ferrador, para apertar o focinho das bêstas.
\section{Alçapremar}
\begin{itemize}
\item {Grp. gram.:v. t.}
\end{itemize}
Elevar com alçaprema.
Apanhar com alçaprema.
Apertar; opprimir.
\section{Alçapreme}
\begin{itemize}
\item {Grp. gram.:m.}
\end{itemize}
\begin{itemize}
\item {Utilização:Prov.}
\end{itemize}
\begin{itemize}
\item {Utilização:trasm.}
\end{itemize}
O mesmo que \textunderscore alçaprema\textunderscore .
\section{Alcaptona}
\begin{itemize}
\item {Grp. gram.:f.}
\end{itemize}
Substância amarela, amorpha, inodora e insípida, contida na urina mórbida.
\section{Alcaptor}
\begin{itemize}
\item {Grp. gram.:m.}
\end{itemize}
O mesmo que \textunderscore alcuptor\textunderscore .
\section{Alcar}
\begin{itemize}
\item {Grp. gram.:m.}
\end{itemize}
\begin{itemize}
\item {Proveniência:(Do ár. \textunderscore al-cara\textunderscore )}
\end{itemize}
Arbusto, da fam. das cistíneas, vulgarmente \textunderscore erva das sete sangrias\textunderscore .
\section{Alçar}
\begin{itemize}
\item {Grp. gram.:v. t.}
\end{itemize}
Tornar alto, altear.
Levantar.
Edificar.
Celebrar, exaltando.
Suspender.
Proclamar, acclamar.
(B. lat. \textunderscore altiare\textunderscore )
\section{Alcaravão}
\begin{itemize}
\item {Grp. gram.:m.}
\end{itemize}
\begin{itemize}
\item {Proveniência:(Do ár. \textunderscore al-caravan\textunderscore )}
\end{itemize}
Ave pernalta, de arribação.
\section{Alcaravia}
\begin{itemize}
\item {Grp. gram.:f.}
\end{itemize}
\begin{itemize}
\item {Proveniência:(Do ár. \textunderscore al-caravia\textunderscore )}
\end{itemize}
Planta herbácea, bisannual, umbellífera.
\section{Alcaravis}
\begin{itemize}
\item {Grp. gram.:m.}
\end{itemize}
\begin{itemize}
\item {Proveniência:(Do ár. \textunderscore al-carabis\textunderscore )}
\end{itemize}
Tubo de ferro, que leva o ar, do folle á forja.
\section{Alcárcova}
\begin{itemize}
\item {Grp. gram.:f.}
\end{itemize}
O mesmo que \textunderscore alcórcova\textunderscore .
\section{Alcaria}
\begin{itemize}
\item {Grp. gram.:f.}
\end{itemize}
\begin{itemize}
\item {Utilização:Ant.}
\end{itemize}
Casa campestre, para guardar instrumentos de lavoira.
(Ár. \textunderscore al-caria\textunderscore )
\section{Alcaria}
\begin{itemize}
\item {Grp. gram.:f.}
\end{itemize}
Gênero de plantas.
\section{Alcarnache}
\begin{itemize}
\item {Grp. gram.:m.}
\end{itemize}
Planta damninha, escalracho.
\section{Alcarovia}
\begin{itemize}
\item {Grp. gram.:f.}
\end{itemize}
\begin{itemize}
\item {Utilização:T. do Porto}
\end{itemize}
O mesmo que \textunderscore alcaravia\textunderscore .
\section{Alcarrada}
\begin{itemize}
\item {Grp. gram.:f.}
\end{itemize}
Movimento da ave de rapina, para empolgar a presa.
\section{Alcarradas}
\begin{itemize}
\item {Grp. gram.:f. pl.}
\end{itemize}
\begin{itemize}
\item {Utilização:ant.}
\end{itemize}
\begin{itemize}
\item {Utilização:Pop.}
\end{itemize}
\begin{itemize}
\item {Proveniência:(Do ár. \textunderscore al-carrata\textunderscore )}
\end{itemize}
Arrecadas, pingentes de orelhas.
\section{Alcarraza}
\begin{itemize}
\item {Grp. gram.:f.}
\end{itemize}
\begin{itemize}
\item {Proveniência:(Do ár. \textunderscore al-corraz\textunderscore  ou \textunderscore al-carraza\textunderscore )}
\end{itemize}
Espécie de moringue.
\section{Alcarroteira}
\begin{itemize}
\item {Grp. gram.:f.}
\end{itemize}
\begin{itemize}
\item {Utilização:Prov.}
\end{itemize}
\begin{itemize}
\item {Utilização:trasm.}
\end{itemize}
Mulher mexeriqueira, onzeneira.
\section{Alcateia}
\begin{itemize}
\item {Grp. gram.:f.}
\end{itemize}
\begin{itemize}
\item {Proveniência:(Do ár. \textunderscore al-cati\textunderscore )}
\end{itemize}
Bando de lobos.
Manada de animaes selvagens.
Quadrilha de bandidos.
\textunderscore Estar de alcateia\textunderscore , estar vigiando, estar alerta.
\section{Alcatifa}
\begin{itemize}
\item {Grp. gram.:f.}
\end{itemize}
\begin{itemize}
\item {Proveniência:(Do ár. \textunderscore al-catifa\textunderscore )}
\end{itemize}
Tapête grande, com que se reveste o chão.
\section{Alcatifado}
\begin{itemize}
\item {Grp. gram.:adj.}
\end{itemize}
Coberto de alcatifa.
\section{Alcatifamento}
\begin{itemize}
\item {Grp. gram.:m.}
\end{itemize}
Acto de \textunderscore alcatifar\textunderscore .
\section{Alcatifar}
\begin{itemize}
\item {Grp. gram.:v. t.}
\end{itemize}
Cobrir com alcatifa.
Cobrir, á semelhança de alcatifa.
\section{Alcatifeiro}
\begin{itemize}
\item {Grp. gram.:m.}
\end{itemize}
Fabricante de alcatifas.
\section{Alcatira}
\begin{itemize}
\item {Grp. gram.:f.}
\end{itemize}
\begin{itemize}
\item {Proveniência:(Do ár. \textunderscore al-catira\textunderscore )}
\end{itemize}
Arbusto leguminoso.
Goma branca, extrahida dêsse arbusto.
Adragantho.
\section{Alcatra}
\begin{itemize}
\item {Grp. gram.:f.}
\end{itemize}
\begin{itemize}
\item {Utilização:Chul.}
\end{itemize}
\begin{itemize}
\item {Proveniência:(Do ár. \textunderscore al-catra\textunderscore )}
\end{itemize}
Lugar, onde acaba o fio do lombo do boi ou vaca.
Pernas traseiras ou ancas do boi.
O mesmo que nádegas.
\section{Alcatrão}
\begin{itemize}
\item {Grp. gram.:m.}
\end{itemize}
\begin{itemize}
\item {Proveniência:(Do ar. \textunderscore al-quitran\textunderscore )}
\end{itemize}
Producto da destillação do pinheiro ou da hulha.
\section{Alcatrate}
\begin{itemize}
\item {Grp. gram.:m.}
\end{itemize}
\begin{itemize}
\item {Proveniência:(Do ár. \textunderscore al-catrat\textunderscore , por \textunderscore al-catarat\textunderscore )}
\end{itemize}
Pranchão, que cobre os topos das aposturas do navio. Cf. \textunderscore Peregrinação\textunderscore , LIX.
\section{Alcatraz}
\begin{itemize}
\item {Grp. gram.:m.}
\end{itemize}
Ave palmípede.
Nome de várias espécies de pelicano.
O mesmo que \textunderscore albatroz\textunderscore .
(\textunderscore T. cast.\textunderscore )
\section{Alcatraz}
\begin{itemize}
\item {Grp. gram.:m.}
\end{itemize}
\begin{itemize}
\item {Utilização:Pop.}
\end{itemize}
\begin{itemize}
\item {Proveniência:(De \textunderscore alcatra\textunderscore ?)}
\end{itemize}
Aquelle que concerta ossos deslocados.
\section{Alcatreiro}
\begin{itemize}
\item {Grp. gram.:adj.}
\end{itemize}
\begin{itemize}
\item {Utilização:Chul.}
\end{itemize}
\begin{itemize}
\item {Proveniência:(De \textunderscore alcatra\textunderscore )}
\end{itemize}
Que tem grandes nádegas.
\section{Alcatroamento}
\begin{itemize}
\item {Grp. gram.:m.}
\end{itemize}
Acto de \textunderscore alcatroar\textunderscore .
\section{Alcatroar}
\begin{itemize}
\item {Grp. gram.:v. i.}
\end{itemize}
Misturar, cobrir, untar, com alcatrão.
\section{Alcatruz}
\begin{itemize}
\item {Grp. gram.:m.}
\end{itemize}
\begin{itemize}
\item {Grp. gram.:Pl.}
\end{itemize}
\begin{itemize}
\item {Utilização:T. do Porto}
\end{itemize}
\begin{itemize}
\item {Proveniência:(Do ár. \textunderscore al-cadus\textunderscore )}
\end{itemize}
Vaso de barro, que levanta a água nas noras.
Botas grossas, largas e mal feitas.
\section{Alcatruzada}
\begin{itemize}
\item {Grp. gram.:f.}
\end{itemize}
Cano de manilhas, que leva a água do caldeirão para a marinha, nas salinas do
Sado.
O conteúdo de um alcatruz ou a porção de líquido que um alcatruz comporta.
\section{Alcatruzar}
\begin{itemize}
\item {Grp. gram.:v. t.}
\end{itemize}
Dar fórma de alcatruz a.
\section{Alcavala}
\begin{itemize}
\item {Grp. gram.:f.}
\end{itemize}
\begin{itemize}
\item {Proveniência:(Do ár. \textunderscore al-cavala\textunderscore )}
\end{itemize}
* Nome antigo de um fruto.
Tributo; imposto forçado.
\section{Alcavaleiro}
\begin{itemize}
\item {Grp. gram.:m.}
\end{itemize}
Antigo arrendatário de alcavalas.
Aquelle que administrava o producto de alcavalas.
\section{Alcazira}
\begin{itemize}
\item {Grp. gram.:f.}
\end{itemize}
\begin{itemize}
\item {Utilização:Ant.}
\end{itemize}
O mesmo que \textunderscore algecira\textunderscore .
\section{Alce}
\begin{itemize}
\item {Grp. gram.:m.}
\end{itemize}
\begin{itemize}
\item {Proveniência:(Lat. \textunderscore alce\textunderscore )}
\end{itemize}
Espécie de veado das regiões do norte.
\section{Álcea}
\begin{itemize}
\item {Grp. gram.:f.}
\end{itemize}
\begin{itemize}
\item {Proveniência:(Gr. \textunderscore alkea\textunderscore )}
\end{itemize}
Planta bis-annual, também chamada malvaísco silvestre.
\section{Alceamento}
\begin{itemize}
\item {Grp. gram.:m.}
\end{itemize}
\begin{itemize}
\item {Utilização:Typ.}
\end{itemize}
\begin{itemize}
\item {Proveniência:(De \textunderscore alcear\textunderscore )}
\end{itemize}
Operação typográphica, que consiste na collocação de supportes, alças e fôlhas recortadas sôbre o estofo de týmpano, para que todos os pontos da fôrma tenham na tiragem o devido valor, segundo o corpo dos caracteres respectivos.
\section{Alcear}
\begin{itemize}
\item {Grp. gram.:v. t.}
\end{itemize}
\begin{itemize}
\item {Utilização:Náut.}
\end{itemize}
Coordenar (as fôlhas de um livro que se prepara para a encadernação).
Guarnecer de alça (uma peça de poleame, etc.).
\section{Alcedídeos}
\begin{itemize}
\item {Grp. gram.:m. pl.}
\end{itemize}
O mesmo que \textunderscore alcedinídeos\textunderscore .
\section{Alcedinídeos}
\begin{itemize}
\item {Grp. gram.:m. pl.}
\end{itemize}
\begin{itemize}
\item {Proveniência:(Do lat. \textunderscore alcedo\textunderscore , \textunderscore alcedinis\textunderscore  + gr. \textunderscore eidos\textunderscore )}
\end{itemize}
Família de aves, que tem por typo o maçarico.
\section{Alcediões}
\begin{itemize}
\item {Grp. gram.:m. pl.}
\end{itemize}
\begin{itemize}
\item {Proveniência:(Do lat. \textunderscore alcedo\textunderscore )}
\end{itemize}
Gênero de coleópteros tetrâmeros.
\section{Alcédone}
\begin{itemize}
\item {Grp. gram.:f.}
\end{itemize}
(V.alcyão)
\section{Alcélafo}
\begin{itemize}
\item {Grp. gram.:m.}
\end{itemize}
\begin{itemize}
\item {Proveniência:(De \textunderscore alce\textunderscore  + \textunderscore élapho\textunderscore )}
\end{itemize}
Espécie de antílope.
\section{Alcélapho}
\begin{itemize}
\item {Grp. gram.:m.}
\end{itemize}
\begin{itemize}
\item {Proveniência:(De \textunderscore alce\textunderscore  + \textunderscore élapho\textunderscore )}
\end{itemize}
Espécie de antílope.
\section{Alchaz}
\begin{itemize}
\item {Grp. gram.:m.}
\end{itemize}
\begin{itemize}
\item {Utilização:Ant.}
\end{itemize}
\begin{itemize}
\item {Proveniência:(Do ár. \textunderscore al\textunderscore  + \textunderscore khaz\textunderscore )}
\end{itemize}
Tecido de seda grossa.
\section{Alchemila}
\begin{itemize}
\item {fónica:que}
\end{itemize}
\begin{itemize}
\item {Grp. gram.:f.}
\end{itemize}
Planta, da fam. das rosáceas.
\section{Alcheria}
\begin{itemize}
\item {Grp. gram.:f.}
\end{itemize}
\begin{itemize}
\item {Utilização:Ant.}
\end{itemize}
\begin{itemize}
\item {Proveniência:(Do ár. ?)}
\end{itemize}
Campo de pasto ou de lavoira.
\section{Alchimia}
\begin{itemize}
\item {fónica:qui}
\end{itemize}
\textunderscore f.\textunderscore  (e der.)
(V. \textunderscore alquimia\textunderscore , etc.)
\section{Alchumoiço}
\begin{itemize}
\item {Grp. gram.:m.}
\end{itemize}
\begin{itemize}
\item {Utilização:Prov.}
\end{itemize}
\begin{itemize}
\item {Utilização:trasm.}
\end{itemize}
Grande chumaço.
\section{Alcicorne}
\begin{itemize}
\item {Grp. gram.:adj.}
\end{itemize}
\begin{itemize}
\item {Proveniência:(De \textunderscore alce\textunderscore  + \textunderscore corno\textunderscore )}
\end{itemize}
Que tem cornos semelhantes aos do alce.
\section{Alcicórnio}
\begin{itemize}
\item {Grp. gram.:adj.}
\end{itemize}
\begin{itemize}
\item {Grp. gram.:M.}
\end{itemize}
O mesmo que \textunderscore alcicorne\textunderscore .
Gênero de fêtos polypódios.
\section{Alcide}
\begin{itemize}
\item {Grp. gram.:m.}
\end{itemize}
\begin{itemize}
\item {Utilização:Gír.}
\end{itemize}
Pão.
\section{Alcilante}
\begin{itemize}
\item {Grp. gram.:m.}
\end{itemize}
\begin{itemize}
\item {Utilização:Gír.}
\end{itemize}
Relógio de senhora.
\section{Alcina}
\begin{itemize}
\item {Grp. gram.:f.}
\end{itemize}
Gênero de plantas, da fam. das heliântheas.
\section{Alcis}
\begin{itemize}
\item {Grp. gram.:m.}
\end{itemize}
Lepidóptero nocturno, semelhante ás phalenas.
\section{Alcmena}
\begin{itemize}
\item {Grp. gram.:f.}
\end{itemize}
\begin{itemize}
\item {Proveniência:(De \textunderscore Alcmena\textunderscore , n. p.)}
\end{itemize}
Planeta telescópico.
\section{Alcobaça}
\begin{itemize}
\item {Grp. gram.:m.}
\end{itemize}
\begin{itemize}
\item {Proveniência:(De \textunderscore Alcobaça\textunderscore , n. p.)}
\end{itemize}
Lenço grande de algodão, usado principalmente por pessôas que cheiram rapé:«\textunderscore aparando as lagrimas no alcobaça\textunderscore ». Camillo, \textunderscore Bras. de Prazins\textunderscore , 109.
\section{Alcobaceira}
\textunderscore f.\textunderscore  (?)«\textunderscore Que alcobaceira invocará o povo com tanta perdição de fruta?\textunderscore »Filinto, II, 226.
\section{Alcobacense}
\begin{itemize}
\item {Grp. gram.:adj.}
\end{itemize}
Relativo á villa ou mosteiro de Alcobaça.
\section{Alcoceifa}
\begin{itemize}
\item {Grp. gram.:f.}
\end{itemize}
\begin{itemize}
\item {Utilização:Ant.}
\end{itemize}
Alcoice; bairro de meretrizes.
(Ár. \textunderscore al-coceifa\textunderscore )
\section{Alcoethina}
\begin{itemize}
\item {Grp. gram.:f.}
\end{itemize}
Explosivo, que se tem experimentado nos automóveis, e que se produz, fazendo actuar o álcool desnaturado no carboneto de cálcio. Cf. Benevides, \textunderscore Automóveis\textunderscore .
\section{Alcoetina}
\begin{itemize}
\item {fónica:co-e}
\end{itemize}
\begin{itemize}
\item {Grp. gram.:f.}
\end{itemize}
Explosivo, que se tem experimentado nos automóveis, e que se produz, fazendo actuar o álcool desnaturado no carboneto de cálcio. Cf. Benevides, \textunderscore Automóveis\textunderscore .
\section{Alcofa}
\begin{itemize}
\item {fónica:cô}
\end{itemize}
\begin{itemize}
\item {Grp. gram.:f.}
\end{itemize}
Cêsto flexivel de vime, de esparto ou de fôlha de palma.
(Ár. \textunderscore al-coffa\textunderscore )
\section{Alcofa}
\begin{itemize}
\item {fónica:cô}
\end{itemize}
\begin{itemize}
\item {Grp. gram.:m.  e  f.}
\end{itemize}
Alcoviteiro, alcoviteira.
(Cp. \textunderscore alcoveta\textunderscore )
\section{Alcofeira}
\begin{itemize}
\item {Grp. gram.:f.}
\end{itemize}
Alcoviteira:«\textunderscore isso era volta de alcofeira\textunderscore ». Camillo, \textunderscore F. do Arcediago\textunderscore , c. XVII.
\section{Alcofinha}
\begin{itemize}
\item {Grp. gram.:m.  e  f.}
\end{itemize}
\begin{itemize}
\item {Proveniência:(De \textunderscore alcofa\textunderscore ^2)}
\end{itemize}
Alcoviteiro, alcoviteira.
\section{Alcofor}
\begin{itemize}
\item {Grp. gram.:m.}
\end{itemize}
\begin{itemize}
\item {Utilização:Ant.}
\end{itemize}
O mesmo que \textunderscore antimónio\textunderscore ^1.
E o mesmo que \textunderscore cânfora\textunderscore .
(Ár. \textunderscore alcafur\textunderscore )
\section{Alcoforar}
\begin{itemize}
\item {Grp. gram.:v. t.}
\end{itemize}
\begin{itemize}
\item {Proveniência:(De \textunderscore alcofor\textunderscore )}
\end{itemize}
(V.canforar)
\section{Alcofra}
\begin{itemize}
\item {Grp. gram.:f.}
\end{itemize}
\begin{itemize}
\item {Utilização:Prov.}
\end{itemize}
\begin{itemize}
\item {Utilização:trasm.}
\end{itemize}
O mesmo que \textunderscore escrófula\textunderscore .
\section{Alcoice}
\begin{itemize}
\item {Grp. gram.:m.}
\end{itemize}
\begin{itemize}
\item {Proveniência:(Do ár. \textunderscore alcaus\textunderscore )}
\end{itemize}
Lugar de prostituição; bordel; lupanar.
\section{Alcoiceiro}
\begin{itemize}
\item {Grp. gram.:m.}
\end{itemize}
Aquelle que tem casa de prostituição.
Aquelle que frequenta alcoices.
\section{Alcomonia}
(V.alcamonia)
\section{Álcool}
\begin{itemize}
\item {Grp. gram.:m.}
\end{itemize}
Líquido, obtido pela destillação de qualquer substância fermentável.
Pl. \textunderscore álcooes\textunderscore .
(Ár. \textunderscore al-cohl\textunderscore )
\section{Alcoól}
\begin{itemize}
\item {Grp. gram.:m.}
\end{itemize}
Líquido, obtido pela destillação de qualquer substância fermentável.
Pl. \textunderscore álcooes\textunderscore .
(Ár. \textunderscore al-cohl\textunderscore )
\section{Alcóol}
\begin{itemize}
\item {Grp. gram.:m.}
\end{itemize}
Líquido, obtido pela destillação de qualquer substância fermentável.
Pl. \textunderscore álcooes\textunderscore .
(Ár. \textunderscore al-cohl\textunderscore )
\section{Alcoolado}
\begin{itemize}
\item {Grp. gram.:m.}
\end{itemize}
O mesmo que \textunderscore alcoolato\textunderscore .
\section{Alcião}
\begin{itemize}
\item {Grp. gram.:m.}
\end{itemize}
\begin{itemize}
\item {Proveniência:(Gr. \textunderscore alcuon\textunderscore )}
\end{itemize}
Ave aquática, maçarico.
\section{Alcion}
\begin{itemize}
\item {Grp. gram.:m.}
\end{itemize}
(V.alcyão)
\section{Alcíona}
\begin{itemize}
\item {Grp. gram.:f.}
\end{itemize}
O mesmo que \textunderscore alcião\textunderscore .
\section{Alcionário}
\begin{itemize}
\item {Grp. gram.:m.}
\end{itemize}
Gênero de pólipos.
\section{Alcíone}
\begin{itemize}
\item {Grp. gram.:f.}
\end{itemize}
O mesmo que \textunderscore alcião\textunderscore :«\textunderscore ...a lamentosa alcíone\textunderscore ». Garrett, \textunderscore Camões\textunderscore .
\section{Alcionelas}
\begin{itemize}
\item {Grp. gram.:f. pl.}
\end{itemize}
\begin{itemize}
\item {Proveniência:(De \textunderscore alcýon\textunderscore )}
\end{itemize}
Gênero de pólypos, da classe dos briozários.
\section{Alcióneo}
\begin{itemize}
\item {Grp. gram.:adj.}
\end{itemize}
\begin{itemize}
\item {Utilização:Fig.}
\end{itemize}
\begin{itemize}
\item {Proveniência:(Lat. \textunderscore alcyoneus\textunderscore )}
\end{itemize}
Relativo ao alcião.
Sereno, agradável.
\section{Alcoolativo}
\begin{itemize}
\item {Grp. gram.:m.}
\end{itemize}
\begin{itemize}
\item {Proveniência:(De \textunderscore alcoolato\textunderscore )}
\end{itemize}
Medicamento alcoólico, para uso externo.
\section{Alcoolato}
\begin{itemize}
\item {Grp. gram.:m.}
\end{itemize}
Líquido, resultante da destillação do álcool sôbre substâncias aromáticas.
Combinação com um sal.
\section{Alcoólatra}
\begin{itemize}
\item {Grp. gram.:m.}
\end{itemize}
Amador de álcool.
\section{Alcoolatura}
\begin{itemize}
\item {Grp. gram.:f.}
\end{itemize}
\begin{itemize}
\item {Proveniência:(De \textunderscore alcoolato\textunderscore )}
\end{itemize}
Líquido, resultante da maceração de matérias vegetaes ou animaes em álcool.
\section{Alcoóleo}
\begin{itemize}
\item {Grp. gram.:m.}
\end{itemize}
Álcool, que se carregou de princípios solúveis de uma ou mais substâncias.
\section{Alcoólico}
\begin{itemize}
\item {Grp. gram.:adj.}
\end{itemize}
\begin{itemize}
\item {Grp. gram.:M.}
\end{itemize}
Relativo ao álcool.
Que tem álcool.
Indivíduo, que abusa de bebidas alcoólicas.
\section{Alcoolismo}
\begin{itemize}
\item {Grp. gram.:m.}
\end{itemize}
Estado mórbido, resultante do abuso de bebidas alcoólicas.
\section{Alcoolista}
\begin{itemize}
\item {Grp. gram.:m.}
\end{itemize}
Aquelle que soffre alcoolismo.
\section{Alcoolito}
\begin{itemize}
\item {Grp. gram.:m.}
\end{itemize}
Solução alcoólica, que não deixa resíduo.
\section{Alcoolizar}
\begin{itemize}
\item {Grp. gram.:v. t.}
\end{itemize}
Misturar com álcool (um líquido qualquer)
Embriagar.
\section{Alcoolizável}
\begin{itemize}
\item {Grp. gram.:adj.}
\end{itemize}
Que se póde alcoolizar. Cf. \textunderscore Techn. Rur.\textunderscore , 181.
\section{Alcoolómetro}
\begin{itemize}
\item {Grp. gram.:m.}
\end{itemize}
O mesmo que \textunderscore alcoómetro\textunderscore .
\section{Alcoomel}
\begin{itemize}
\item {Grp. gram.:m.}
\end{itemize}
Excipiente pharmacêutico, formado de uma parte de álcool e três de mel.
\section{Alcoometria}
\begin{itemize}
\item {Grp. gram.:f.}
\end{itemize}
\begin{itemize}
\item {Proveniência:(De \textunderscore alcoómetro\textunderscore )}
\end{itemize}
Processo, com que se determina a quantidade de álcool, que há num líquido.
\section{Alcoómetro}
\begin{itemize}
\item {Grp. gram.:m.}
\end{itemize}
\begin{itemize}
\item {Proveniência:(De \textunderscore álcool\textunderscore  + gr. \textunderscore metron\textunderscore )}
\end{itemize}
Pesa-licor, espécie de areómetro, instrumento para determinar a quantidade de álcool absoluto, contida em qualquer líquido.
\section{Alcopo}
\begin{itemize}
\item {Grp. gram.:m.}
\end{itemize}
\begin{itemize}
\item {Utilização:Zool.}
\end{itemize}
Gênero de turdídeos, da Ásia meridional.
\section{Alcor}
\begin{itemize}
\item {Grp. gram.:m.}
\end{itemize}
Pequena estrêlla na cauda da Ursa-Maior.
\section{Alcoranista}
\begin{itemize}
\item {Grp. gram.:m.}
\end{itemize}
Aquelle que é entendido nas doutrinas do Alcorão.
Sectário do Alcorão.
\section{Alcorão}
\begin{itemize}
\item {Grp. gram.:m.}
\end{itemize}
\begin{itemize}
\item {Utilização:Ant.}
\end{itemize}
\begin{itemize}
\item {Proveniência:(Do ár. \textunderscore al-coran\textunderscore )}
\end{itemize}
Livro sagrado dos Mahometanos.
Religião mahometana.
Tôrre, de onde se chamam os muçulmanos á oração. Cf. \textunderscore Tombo do Estado da Índia\textunderscore , 225, nos \textunderscore Subsidios\textunderscore  de Felner.
\section{Alcorão}
\begin{itemize}
\item {Grp. gram.:m.}
\end{itemize}
\begin{itemize}
\item {Utilização:Prov.}
\end{itemize}
Dá-se este nome, em algumas povoações do Alentejo, (Estremoz, Évora, etc.), ao sótão que serve para arrecadação de trastes velhos.
(Por \textunderscore alcovão\textunderscore , de \textunderscore alcova\textunderscore , infl. por \textunderscore alcorão\textunderscore ^1 ?)
\section{Alcorca}
\begin{itemize}
\item {Grp. gram.:f.}
\end{itemize}
Regueira, para esgotamento de águas.
Fôsso, que se abre para formar vallados, que resguardem ou limitem uma propriedade rústica. Cf. \textunderscore Código Civil\textunderscore , art. 462.
(Contr. de \textunderscore alcórcova\textunderscore )
\section{Alcorça}
\begin{itemize}
\item {fónica:côr}
\end{itemize}
\begin{itemize}
\item {Grp. gram.:f.}
\end{itemize}
Massa de açúcar, para fazer ou cobrir doce.
\section{Alcorce}
\begin{itemize}
\item {fónica:côr}
\end{itemize}
\begin{itemize}
\item {Grp. gram.:m.}
\end{itemize}
O mesmo que \textunderscore alcorça\textunderscore .
\section{Alcórcova}
\begin{itemize}
\item {Grp. gram.:f.}
\end{itemize}
\begin{itemize}
\item {Utilização:Ant.}
\end{itemize}
\begin{itemize}
\item {Proveniência:(Do ár. ?)}
\end{itemize}
Fôsso dos vallados, com que se tapam as propriedades rústicas.
\section{Alcorcovar}
\textunderscore v. t. Ant.\textunderscore  (e der.)
(V. \textunderscore corcovar\textunderscore , etc.)
\section{Alcornoque}
\begin{itemize}
\item {Grp. gram.:m.}
\end{itemize}
\begin{itemize}
\item {Utilização:Ant.}
\end{itemize}
O mesmo que \textunderscore sobreiro\textunderscore .
(\textunderscore T. cast.\textunderscore )
\section{Alcorque}
\begin{itemize}
\item {Grp. gram.:m.}
\end{itemize}
Calçado, que se usava, com sola de cortiça.
(\textunderscore T. cast.\textunderscore )
\section{Alcorraça}
\begin{itemize}
\item {Grp. gram.:f.}
\end{itemize}
\begin{itemize}
\item {Utilização:Prov.}
\end{itemize}
\begin{itemize}
\item {Utilização:alent.}
\end{itemize}
O mesmo que alcorraz.
\section{Alcorraz}
\begin{itemize}
\item {Grp. gram.:m.}
\end{itemize}
Peixe de Portugal, o mesmo que \textunderscore choupa\textunderscore ^2.
\section{Alcouce}
\begin{itemize}
\item {Grp. gram.:m.}
\end{itemize}
\begin{itemize}
\item {Proveniência:(Do ár. \textunderscore alcaus\textunderscore )}
\end{itemize}
Lugar de prostituição; bordel; lupanar.
\section{Alcouceiro}
\begin{itemize}
\item {Grp. gram.:m.}
\end{itemize}
Aquelle que tem casa de prostituição.
Aquelle que frequenta alcouces.
\section{Alcoucês}
\begin{itemize}
\item {Grp. gram.:m.}
\end{itemize}
\begin{itemize}
\item {Utilização:Ant.}
\end{itemize}
\begin{itemize}
\item {Proveniência:(De \textunderscore alcouço\textunderscore )}
\end{itemize}
Vento do sul.
\section{Alcouço}
\begin{itemize}
\item {Grp. gram.:m.}
\end{itemize}
\begin{itemize}
\item {Utilização:Ant.}
\end{itemize}
O lado do sul.
As regiões que ficam para o sul.
\section{Alcova}
\begin{itemize}
\item {fónica:cô}
\end{itemize}
\begin{itemize}
\item {Grp. gram.:f.}
\end{itemize}
\begin{itemize}
\item {Proveniência:(Do ár. \textunderscore al-cobba\textunderscore )}
\end{itemize}
Pequeno quarto de dormir.
Esconderijo.
\section{Alcoveta}
\begin{itemize}
\item {fónica:vê}
\end{itemize}
\begin{itemize}
\item {Grp. gram.:f.}
\end{itemize}
De alcoveto.
\section{Alcoveto}
\begin{itemize}
\item {fónica:vê}
\end{itemize}
\begin{itemize}
\item {Grp. gram.:m.}
\end{itemize}
\begin{itemize}
\item {Proveniência:(De \textunderscore alcova\textunderscore )}
\end{itemize}
O mesmo que \textunderscore alcaiote\textunderscore .
\section{Alcovista}
\begin{itemize}
\item {Grp. gram.:m.}
\end{itemize}
\begin{itemize}
\item {Proveniência:(De \textunderscore alcova\textunderscore )}
\end{itemize}
Homem femeeiro.
\section{Alcovitar}
\begin{itemize}
\item {Grp. gram.:v. t.}
\end{itemize}
\begin{itemize}
\item {Grp. gram.:V. i.}
\end{itemize}
Auxiliar em relações amorosas.
Inculcar.
Servir de alcoviteiro.
(Por \textunderscore alcovetar\textunderscore , de \textunderscore alcoveto\textunderscore )
\section{Alcovitaria}
\begin{itemize}
\item {Grp. gram.:f.}
\end{itemize}
Vida de alcoviteiro.
\section{Alcoviteira}
(fem. de \textunderscore alcoviteiro\textunderscore )
\section{Alcoviteirice}
\begin{itemize}
\item {Grp. gram.:f.}
\end{itemize}
Officio de alcoviteiro.
Alliciação.
Lenocínio.
Mexerico.
\section{Alcoviteiro}
\begin{itemize}
\item {Grp. gram.:m.}
\end{itemize}
\begin{itemize}
\item {Proveniência:(De \textunderscore alcovitar\textunderscore )}
\end{itemize}
Corretor de prostitutas.
Intermediário de relações amorosas.
Mexeriqueiro.
\section{Alcovitice}
\begin{itemize}
\item {Grp. gram.:f.}
\end{itemize}
O mesmo que \textunderscore alcoviteirice\textunderscore .
\section{Alcrevite}
\begin{itemize}
\item {Grp. gram.:m.}
\end{itemize}
\begin{itemize}
\item {Utilização:Ant.}
\end{itemize}
\begin{itemize}
\item {Proveniência:(Do ár. \textunderscore al-quibrite\textunderscore )}
\end{itemize}
Enxôfre.
\section{Alcunha}
\begin{itemize}
\item {Grp. gram.:f.}
\end{itemize}
\begin{itemize}
\item {Proveniência:(Do ár. \textunderscore al-kunía\textunderscore )}
\end{itemize}
Epitheto, dado a alguém, e derivado de qualquer particularidade phýsica ou moral.
Appellido.
\section{Alcunhar}
\begin{itemize}
\item {Grp. gram.:v. t.}
\end{itemize}
Pôr alcunha a.
Appellidar.
Designar por epitheto.
\section{Alcuptor}
\begin{itemize}
\item {Grp. gram.:m.}
\end{itemize}
\begin{itemize}
\item {Utilização:Ant.}
\end{itemize}
Nome de um peixe.
\section{Alcursar}
\begin{itemize}
\item {Grp. gram.:v. t.}
\end{itemize}
\begin{itemize}
\item {Utilização:Prov.}
\end{itemize}
\begin{itemize}
\item {Utilização:alent.}
\end{itemize}
\begin{itemize}
\item {Grp. gram.:V. i.}
\end{itemize}
\begin{itemize}
\item {Proveniência:(De \textunderscore al\textunderscore  + \textunderscore cursar\textunderscore ?)}
\end{itemize}
Alcançar com a vista; vêr.
Restabelecer-se de uma doença.
\section{Alçuz}
\begin{itemize}
\item {Grp. gram.:m.}
\end{itemize}
\begin{itemize}
\item {Utilização:Ant.}
\end{itemize}
Espécie do cânfora.
\section{Alcyão}
\begin{itemize}
\item {Grp. gram.:m.}
\end{itemize}
\begin{itemize}
\item {Proveniência:(Gr. \textunderscore alcuon\textunderscore )}
\end{itemize}
Ave aquática, maçarico.
\section{Alcýon}
\begin{itemize}
\item {Grp. gram.:m.}
\end{itemize}
(V.alcyão)
\section{Alcýona}
\begin{itemize}
\item {Grp. gram.:f.}
\end{itemize}
O mesmo que \textunderscore alcyão\textunderscore .
\section{Alcyonário}
\begin{itemize}
\item {Grp. gram.:m.}
\end{itemize}
Gênero de pólipos.
\section{Alcýone}
\begin{itemize}
\item {Grp. gram.:f.}
\end{itemize}
O mesmo que \textunderscore alcyão\textunderscore :«\textunderscore ...a lamentosa alcýone\textunderscore ». Garrett, \textunderscore Camões\textunderscore .
\section{Alcyonelas}
\begin{itemize}
\item {Grp. gram.:f. pl.}
\end{itemize}
\begin{itemize}
\item {Proveniência:(De \textunderscore alcýon\textunderscore )}
\end{itemize}
Gênero de pólypos, da classe dos briozários.
\section{Alcyóneo}
\begin{itemize}
\item {Grp. gram.:adj.}
\end{itemize}
\begin{itemize}
\item {Utilização:Fig.}
\end{itemize}
\begin{itemize}
\item {Proveniência:(Lat. \textunderscore alcyoneus\textunderscore )}
\end{itemize}
Relativo ao alcyão.
Sereno, agradável.
\section{Alda}
\begin{itemize}
\item {Grp. gram.:f.}
\end{itemize}
\begin{itemize}
\item {Utilização:Ant.}
\end{itemize}
Medida de comprimento, correspondente a 11 decimetros.
\section{Aldavana}
\begin{itemize}
\item {Grp. gram.:f.}
\end{itemize}
Planta rubiácea da Índia portuguesa, (\textunderscore adina cordifolia\textunderscore , Hork).
\section{Aldéa}
\begin{itemize}
\item {Grp. gram.:f.}
\end{itemize}
(V.aldeia)
\section{Aldeã}
(\textunderscore fem.\textunderscore  de \textunderscore al-deão\textunderscore )
\section{Aldeaga}
\begin{itemize}
\item {Grp. gram.:m.  e  f.}
\end{itemize}
Pessôa, que atrapalha tudo; zaranza; trapalhão.
Tagarela, palrador.
\section{Aldeagante}
\begin{itemize}
\item {Grp. gram.:m.  e  f.}
\end{itemize}
\begin{itemize}
\item {Utilização:Prov.}
\end{itemize}
\begin{itemize}
\item {Utilização:trasm.}
\end{itemize}
Pessôa alegre, desenvolta.
Viandante, caminhante.
\section{Aldeagar}
\begin{itemize}
\item {Grp. gram.:v. i.}
\end{itemize}
\begin{itemize}
\item {Utilização:Prov.}
\end{itemize}
Falar á tôa, tagarelar.
Falar animadamente.
Gracejar ruidosamente.
\section{Aldeamento}
\begin{itemize}
\item {Grp. gram.:m.}
\end{itemize}
\begin{itemize}
\item {Utilização:Bras}
\end{itemize}
Povoação de Índios, sob a direcção de missionário ou de autoridade leiga.
\section{Aldean}
(\textunderscore fem.\textunderscore  de \textunderscore al-deão\textunderscore )
\section{Aldeana}
\begin{itemize}
\item {Grp. gram.:f.}
\end{itemize}
O mesmo que \textunderscore aldean\textunderscore . Cf. Castilho, \textunderscore Felic. pela agr.\textunderscore , 19.
\section{Aldeanmente}
\begin{itemize}
\item {Grp. gram.:adv.}
\end{itemize}
Á maneira de \textunderscore aldeão\textunderscore .
\section{Aldeano}
\begin{itemize}
\item {Grp. gram.:m.}
\end{itemize}
\begin{itemize}
\item {Grp. gram.:Adj.}
\end{itemize}
O mesmo que \textunderscore aldeão\textunderscore . Cf. Filinto, III, 63.
Próprio de aldeia. Cf. Filinto, XIII, 49.
\section{Aldeão}
\begin{itemize}
\item {Grp. gram.:m.}
\end{itemize}
\begin{itemize}
\item {Grp. gram.:Adj.}
\end{itemize}
Aquelle que vive ou mora em aldeia.
Rústico.
Simples; próprio de aldeia.
Pl. \textunderscore aldeãos\textunderscore  ou \textunderscore aldeões\textunderscore .
\section{Aldear}
\begin{itemize}
\item {Grp. gram.:v. t.}
\end{itemize}
Dividir em aldeias.
Povoar de aldeias.
Congregar, formando aldeias.
Reunir num só povo, numa aldeia.
\section{Aldebaran}
\begin{itemize}
\item {Grp. gram.:f.}
\end{itemize}
\begin{itemize}
\item {Proveniência:(Do ár. \textunderscore al-debaran\textunderscore )}
\end{itemize}
Estrêlla de primeira grandeza, na constellação do Toiro.
\section{Aldehídico}
\begin{itemize}
\item {Grp. gram.:adj.}
\end{itemize}
Relativo ao aldeído.
\section{Aldehydato}
\begin{itemize}
\item {fónica:de-i}
\end{itemize}
\begin{itemize}
\item {Grp. gram.:m.}
\end{itemize}
\begin{itemize}
\item {Utilização:Chím.}
\end{itemize}
Corpo derivado do aldehydo pela substituição de um átomo de metal por um de hydrogênio.
\section{Aldehýdico}
\begin{itemize}
\item {Grp. gram.:adj.}
\end{itemize}
Relativo ao aldehydo.
\section{Aldehydo}
\begin{itemize}
\item {Grp. gram.:m.}
\end{itemize}
\begin{itemize}
\item {Utilização:Chím.}
\end{itemize}
\begin{itemize}
\item {Proveniência:(De \textunderscore álcool\textunderscore  e \textunderscore hydrogênio\textunderscore )}
\end{itemize}
Substância derivada dos álcooes pela perda de hydrogênio.
\section{Aldehydo-fórmico}
\begin{itemize}
\item {Grp. gram.:m.}
\end{itemize}
Substância desinfectante, o mesmo que \textunderscore formol\textunderscore .
\section{Aldeia}
\begin{itemize}
\item {Grp. gram.:f.}
\end{itemize}
\begin{itemize}
\item {Utilização:Bras}
\end{itemize}
\begin{itemize}
\item {Utilização:Bras}
\end{itemize}
Pequena povoação, que não tem categoria de villa ou cidade. Povoação rústica.
Campo.
Povoação, composta exclusivamente de aborígenes.
Cada uma das casas que constituem uma povoação de indígenas.
\section{Aldeidato}
\begin{itemize}
\item {fónica:de-i}
\end{itemize}
\begin{itemize}
\item {Grp. gram.:m.}
\end{itemize}
\begin{itemize}
\item {Utilização:Chím.}
\end{itemize}
Corpo derivado do aldehydo pela substituição de um átomo de metal por um de hydrogênio.
\section{Aldeído}
\begin{itemize}
\item {Grp. gram.:m.}
\end{itemize}
\begin{itemize}
\item {Utilização:Chím.}
\end{itemize}
\begin{itemize}
\item {Proveniência:(De \textunderscore álcool\textunderscore  e \textunderscore hydrogênio\textunderscore )}
\end{itemize}
Substância derivada dos álcooes pela perda de hydrogênio.
\section{Aldemenos}
\begin{itemize}
\item {Grp. gram.:adv.}
\end{itemize}
\begin{itemize}
\item {Utilização:Pop.}
\end{itemize}
(Corr. de \textunderscore ao menos\textunderscore ). Cf. Castilho, \textunderscore Escavações\textunderscore , 69.
\section{Aldeola}
\begin{itemize}
\item {Grp. gram.:f.}
\end{itemize}
Pequena aldeia.
\section{Aldeota}
\begin{itemize}
\item {Grp. gram.:f.}
\end{itemize}
O mesmo que \textunderscore aldeola\textunderscore .
\section{Aldevane}
\begin{itemize}
\item {Grp. gram.:m.}
\end{itemize}
Árvore de Damão, (\textunderscore nauclea cordifolia\textunderscore ).
\section{Aldino}
\begin{itemize}
\item {Grp. gram.:adj.}
\end{itemize}
\begin{itemize}
\item {Proveniência:(De \textunderscore aldo\textunderscore )}
\end{itemize}
Diz-se das edições dos célebres impressores Aldos, e dos caracteres que êlles empregaram em typographía.
\section{Aldo}
\begin{itemize}
\item {Grp. gram.:m.}
\end{itemize}
Edição feita pelos Aldos.
Typo próprio das edições aldinas.
\section{Aldraba}
\begin{itemize}
\item {Grp. gram.:f.}
\end{itemize}
\begin{itemize}
\item {Proveniência:(Do ár. \textunderscore addaba\textunderscore )}
\end{itemize}
Peça de ferro na parte anterior de uma porta, servindo para bater nesta, a chamar a attenção de quem está dentro, e para erguer a tranqueta que segura a porta do lado posterior,--(algumas vezes, é destinada a um só dêstes dois effeitos).
Tranqueta, com que se fecha a cana do leme.
\section{Aldrava}
\begin{itemize}
\item {Grp. gram.:f.}
\end{itemize}
\begin{itemize}
\item {Proveniência:(Do ár. \textunderscore addaba\textunderscore )}
\end{itemize}
Peça de ferro na parte anterior de uma porta, servindo para bater nesta, a chamar a attenção de quem está dentro, e para erguer a tranqueta que segura a porta do lado posterior,--(algumas vezes, é destinada a um só dêstes dois effeitos).
Tranqueta, com que se fecha a cana do leme.
\section{Aldravada}
\begin{itemize}
\item {Grp. gram.:f.}
\end{itemize}
Pancada com a aldrava na porta.
\section{Aldravadamente}
\begin{itemize}
\item {Grp. gram.:adv.}
\end{itemize}
Confusamente, atrapalhadamente.
\section{Aldravão}
\begin{itemize}
\item {Grp. gram.:m.}
\end{itemize}
\begin{itemize}
\item {Utilização:Pop.}
\end{itemize}
\begin{itemize}
\item {Utilização:Prov.}
\end{itemize}
\begin{itemize}
\item {Utilização:trasm.}
\end{itemize}
Aldrava grande.
Homem mentiroso, trapaceiro.
Aquelle que fala confusamente, atrapalhadamente, que não é limpo ou perfeito no seu trabalho.
Estafermo, (falando-se de uma mulher)
\section{Aldravar}
\begin{itemize}
\item {Grp. gram.:v. t.}
\end{itemize}
\begin{itemize}
\item {Utilização:Pop.}
\end{itemize}
\begin{itemize}
\item {Grp. gram.:V. i.}
\end{itemize}
\begin{itemize}
\item {Utilização:Pop.}
\end{itemize}
\begin{itemize}
\item {Grp. gram.:V. i.}
\end{itemize}
\begin{itemize}
\item {Utilização:Pop.}
\end{itemize}
Aferrolhar.
Pôr aldrava em.
Executar mal (qualquer coisa).
Falar confusamente.
Mentir muito.
Tocar na aldrava para chamar:«\textunderscore fui adrabar-lhe á porta\textunderscore ». Camillo, \textunderscore Caveira\textunderscore , 325.
\section{Aldravice}
\begin{itemize}
\item {Grp. gram.:f.}
\end{itemize}
\begin{itemize}
\item {Utilização:Pop.}
\end{itemize}
\begin{itemize}
\item {Proveniência:(De \textunderscore aldravar\textunderscore )}
\end{itemize}
Trapaça; patranha.
\section{Aldro}
\begin{itemize}
\item {Grp. gram.:m.}
\end{itemize}
\begin{itemize}
\item {Utilização:T. de Trancoso}
\end{itemize}
O mesmo que \textunderscore adro\textunderscore .
\section{Aldrope}
\begin{itemize}
\item {Grp. gram.:m.}
\end{itemize}
(V.galdrope)
\section{Aldrovanda}
\begin{itemize}
\item {Grp. gram.:f.}
\end{itemize}
\begin{itemize}
\item {Proveniência:(De \textunderscore Aldrovande\textunderscore , n. p.)}
\end{itemize}
Gênero de plantas droseráceas.
Planta aquática dos pântanos da Europa meridional.
\section{Aldrubio}
\begin{itemize}
\item {Grp. gram.:m.}
\end{itemize}
\begin{itemize}
\item {Utilização:Prov.}
\end{itemize}
\begin{itemize}
\item {Utilização:beir.}
\end{itemize}
Indivíduo trapaceiro, burlão.
(Por \textunderscore aldravio\textunderscore , de \textunderscore aldravar\textunderscore )
\section{Alé}
\begin{itemize}
\item {Grp. gram.:interj.}
\end{itemize}
\begin{itemize}
\item {Utilização:Ant.}
\end{itemize}
(V.olé!)
\section{Aleald...}
(V.leald...)
\section{Alear}
\begin{itemize}
\item {Grp. gram.:v. i.}
\end{itemize}
(V. \textunderscore alar\textunderscore ^2)
\section{Aleatoriamente}
\begin{itemize}
\item {Grp. gram.:adv.}
\end{itemize}
De modo \textunderscore aleatório\textunderscore .
\section{Aleatório}
\begin{itemize}
\item {Grp. gram.:adj.}
\end{itemize}
\begin{itemize}
\item {Proveniência:(Lat. \textunderscore aleatorius\textunderscore )}
\end{itemize}
Sujeito a acontecimentos futuros, que podem dar lucro ou perda.
Casual, fortuito.
\section{Alecraia}
\begin{itemize}
\item {Grp. gram.:f.}
\end{itemize}
\begin{itemize}
\item {Utilização:Prov.}
\end{itemize}
\begin{itemize}
\item {Utilização:trasm.}
\end{itemize}
O mesmo que \textunderscore lacrau\textunderscore .
\section{Alecrim}
\begin{itemize}
\item {Grp. gram.:m.}
\end{itemize}
\begin{itemize}
\item {Proveniência:(Do ár. \textunderscore al-iclil\textunderscore )}
\end{itemize}
Arbusto aromático, da fam. das labiadas, (\textunderscore rosmarinus officinalis\textunderscore , Lin.).
\section{Alecrim}
\begin{itemize}
\item {Grp. gram.:m.}
\end{itemize}
Peixe de Portugal.
\section{Alecrinzeiro}
\begin{itemize}
\item {Grp. gram.:m.}
\end{itemize}
O mesmo que \textunderscore alecrim\textunderscore ^1, quando êste é do tamanho de um arbusto.
\section{Alectélia}
\begin{itemize}
\item {Grp. gram.:f.}
\end{itemize}
Gênero de aves gallináceas.
\section{Aléctico}
\begin{itemize}
\item {Grp. gram.:adj.}
\end{itemize}
Relativo á alexia.
Que soffre alexia.
\section{Alecto}
\begin{itemize}
\item {Grp. gram.:m.}
\end{itemize}
\begin{itemize}
\item {Proveniência:(De \textunderscore Alecto\textunderscore , n. p.)}
\end{itemize}
Nome de alguns animaes de diferentes ordens e famílias.
\section{Alector}
\begin{itemize}
\item {Grp. gram.:m.}
\end{itemize}
\begin{itemize}
\item {Proveniência:(Gr. \textunderscore alektor\textunderscore )}
\end{itemize}
Ave gallinácea da América.
\section{Alectória}
\begin{itemize}
\item {Grp. gram.:f.}
\end{itemize}
\begin{itemize}
\item {Proveniência:(Do gr. \textunderscore alektor\textunderscore )}
\end{itemize}
Líchen, de fronde ramosa.
\section{Alectório}
\begin{itemize}
\item {Grp. gram.:adj.}
\end{itemize}
\begin{itemize}
\item {Proveniência:(De \textunderscore alector\textunderscore )}
\end{itemize}
Relativo ao gallo.
Diz-se de uma pedra, que se supunha existir no figado ou estômago dos gallos, e da qual se contavam maravilhas.
\section{Alectoromachia}
\begin{itemize}
\item {fónica:qui}
\end{itemize}
\begin{itemize}
\item {Grp. gram.:f.}
\end{itemize}
\begin{itemize}
\item {Proveniência:(Do gr. \textunderscore alektor\textunderscore  + \textunderscore makhe\textunderscore )}
\end{itemize}
Combate de gallos.
\section{Alectoromancia}
\begin{itemize}
\item {Grp. gram.:f.}
\end{itemize}
\begin{itemize}
\item {Proveniência:(Do gr. \textunderscore alektor\textunderscore  + \textunderscore manteia\textunderscore )}
\end{itemize}
Antiga prática de adivinhação, por meio de um gallo, que ia apanhando grãos dispostos sôbre as letras do alfabeto.
\section{Alectoromante}
\begin{itemize}
\item {Grp. gram.:m.}
\end{itemize}
Aquelle que praticava a alectoromancia.
\section{Alectoromântico}
\begin{itemize}
\item {Grp. gram.:adj.}
\end{itemize}
Relativo á alectoromancia.
\section{Alectoromaquia}
\begin{itemize}
\item {Grp. gram.:f.}
\end{itemize}
\begin{itemize}
\item {Proveniência:(Do gr. \textunderscore alektor\textunderscore  + \textunderscore makhe\textunderscore )}
\end{itemize}
Combate de gallos.
\section{Alectras}
\begin{itemize}
\item {Grp. gram.:f. pl.}
\end{itemize}
Gênero de plantas, da fam. das escrofularíneas.
\section{Aléctrio}
\begin{itemize}
\item {Grp. gram.:m.}
\end{itemize}
\begin{itemize}
\item {Utilização:Bot.}
\end{itemize}
Gênero de sapindáceas.
\section{Alectriomancia}
\begin{itemize}
\item {Grp. gram.:f.}
\end{itemize}
O mesmo que \textunderscore alectoromancia\textunderscore .
\section{Alectruro}
\begin{itemize}
\item {Grp. gram.:adj.}
\end{itemize}
\begin{itemize}
\item {Utilização:Zool.}
\end{itemize}
\begin{itemize}
\item {Proveniência:(Do gr. \textunderscore alektruron\textunderscore , gallo, e \textunderscore oura\textunderscore , cauda)}
\end{itemize}
Que tem cauda parecida á do gallo.
\section{Aléctryo}
\begin{itemize}
\item {Grp. gram.:m.}
\end{itemize}
\begin{itemize}
\item {Utilização:Bot.}
\end{itemize}
Gênero de sapindáceas.
\section{Alectryomancia}
\begin{itemize}
\item {Grp. gram.:f.}
\end{itemize}
O mesmo que \textunderscore alectoromancia\textunderscore .
\section{Alécula}
\begin{itemize}
\item {Grp. gram.:f.}
\end{itemize}
Gênero de coleópteros.
\section{Aledar}
\begin{itemize}
\item {Grp. gram.:v. t.}
\end{itemize}
Tornar ledo. Cf. Rui Barb., \textunderscore Réplica\textunderscore , 157.
\section{Alefanginas}
\begin{itemize}
\item {Grp. gram.:f. pl.}
\end{itemize}
Certas pílulas purgativas.
(Talvez alter. do ár. \textunderscore al-efaui\textunderscore )
\section{Alefriz}
\begin{itemize}
\item {Grp. gram.:m.}
\end{itemize}
\begin{itemize}
\item {Proveniência:(Do ár. \textunderscore al\textunderscore  + \textunderscore feridh\textunderscore )}
\end{itemize}
Fenda, encaixe, em que se pregam os topos do taboado do navio.
\section{Alegrador}
\begin{itemize}
\item {Grp. gram.:adj.}
\end{itemize}
Que alegra.
\section{Alegramento}
\begin{itemize}
\item {Grp. gram.:m.}
\end{itemize}
\begin{itemize}
\item {Utilização:P. us.}
\end{itemize}
O mesmo que \textunderscore alegria\textunderscore ^1.
\section{Alegrão}
\begin{itemize}
\item {Grp. gram.:m.}
\end{itemize}
\begin{itemize}
\item {Utilização:Fam.}
\end{itemize}
Grande alegria.
\section{Alegrar}
\begin{itemize}
\item {Grp. gram.:v. t.}
\end{itemize}
\begin{itemize}
\item {Utilização:Fig.}
\end{itemize}
Tornar alegre.
Aformosear.
Embriagar um pouco.
Abrir um pouco as juntas de (tijolo ou cantarías), limpando-as, para nellas vazar argamassa que lhes dê nova cohesão.
\section{Alegrar}
\begin{itemize}
\item {Grp. gram.:v. t.}
\end{itemize}
Cortar com legra.
\section{Alegras}
\begin{itemize}
\item {Grp. gram.:f. pl.}
\end{itemize}
Rede, que faz parte do apparelho da pesca da sardinha.
\section{Alegre}
\begin{itemize}
\item {Grp. gram.:adj.}
\end{itemize}
\begin{itemize}
\item {Proveniência:(Do lat. \textunderscore alacris\textunderscore )}
\end{itemize}
Que tem alegria; contente; prazenteiro.
Que dá alegria; que agrada.
Folgazão.
Um tanto ébrio.
\section{Alegremente}
\begin{itemize}
\item {Grp. gram.:adv.}
\end{itemize}
De modo \textunderscore alegre\textunderscore , com alegria.
\section{Alegrete}
\begin{itemize}
\item {fónica:grê}
\end{itemize}
\begin{itemize}
\item {Grp. gram.:adj.}
\end{itemize}
\begin{itemize}
\item {Grp. gram.:M.}
\end{itemize}
Um tanto alegre.
Um tanto ébrio.
Canteiro ou receptáculo de madeira, pedra ou argamassa, que se enche de terra e onde se criam plantas.
\section{Alegrete}
\begin{itemize}
\item {fónica:grê}
\end{itemize}
\begin{itemize}
\item {Grp. gram.:m.}
\end{itemize}
\begin{itemize}
\item {Utilização:Ant.}
\end{itemize}
Talvez cota de armas.
(Cp. fr. \textunderscore halecret\textunderscore )
\section{Alegria}
\begin{itemize}
\item {Grp. gram.:f.}
\end{itemize}
\begin{itemize}
\item {Grp. gram.:Pl.}
\end{itemize}
\begin{itemize}
\item {Utilização:Pop.}
\end{itemize}
\begin{itemize}
\item {Proveniência:(De \textunderscore alegre\textunderscore )}
\end{itemize}
Contentamento; prazer moral.
Festa.
Acontecimento feliz.
Testículos de animal.
\section{Alegria}
\begin{itemize}
\item {Grp. gram.:f.}
\end{itemize}
Árvore mexicana, da fam. das liliáceas.
\section{Alegroso}
\begin{itemize}
\item {Grp. gram.:adj.}
\end{itemize}
\begin{itemize}
\item {Utilização:T. de Ceilão}
\end{itemize}
O mesmo que \textunderscore alegre\textunderscore .
\section{Alegrote}
\begin{itemize}
\item {Grp. gram.:adj.}
\end{itemize}
O mesmo que \textunderscore alegrete\textunderscore ^1, adj.
\section{Aleia}
\begin{itemize}
\item {Grp. gram.:f.}
\end{itemize}
\begin{itemize}
\item {Utilização:Gal}
\end{itemize}
\begin{itemize}
\item {Proveniência:(Fr. \textunderscore allée\textunderscore )}
\end{itemize}
Fileira, renque de árvores.
Arruamento de jardim.
\section{Aleia}
\begin{itemize}
\item {Grp. gram.:f.}
\end{itemize}
Elephante sem dentes.
\section{Aleijado}
\begin{itemize}
\item {Grp. gram.:m. e adj.}
\end{itemize}
\begin{itemize}
\item {Proveniência:(De \textunderscore aleijar\textunderscore )}
\end{itemize}
O que tem deformidade phýsica.
\section{Aleijamento}
\begin{itemize}
\item {Grp. gram.:m.}
\end{itemize}
\begin{itemize}
\item {Utilização:Ant.}
\end{itemize}
O mesmo que aleijão.
\section{Aleijão}
\begin{itemize}
\item {Grp. gram.:m.}
\end{itemize}
\begin{itemize}
\item {Proveniência:(Do lat. \textunderscore laesio\textunderscore , \textunderscore laesionis\textunderscore )}
\end{itemize}
Deformidade, defeito phýsico ou moral.
\section{Aleijar}
\begin{itemize}
\item {Grp. gram.:v. t.}
\end{itemize}
Causar aleijão, deformidade a.
Magoar muito.
\section{Aleiloar}
\begin{itemize}
\item {Grp. gram.:v. t.}
\end{itemize}
Pôr em leilão.
\section{Aleiona}
\begin{itemize}
\item {Grp. gram.:f.  e  adj.}
\end{itemize}
\begin{itemize}
\item {Utilização:Ant.}
\end{itemize}
Adúltera.
\section{Aleirado}
\begin{itemize}
\item {Grp. gram.:adj.}
\end{itemize}
\begin{itemize}
\item {Proveniência:(De \textunderscore aleirar\textunderscore )}
\end{itemize}
Dividido em leiras.
\section{Aleirar}
\begin{itemize}
\item {Grp. gram.:v. t.}
\end{itemize}
Dividir em leiras.
\section{Aleitação}
\begin{itemize}
\item {Grp. gram.:f.}
\end{itemize}
Acto de ou effeito de \textunderscore aleitar\textunderscore ^1.
\section{Aleitado}
\begin{itemize}
\item {Grp. gram.:adj.}
\end{itemize}
\begin{itemize}
\item {Utilização:Constr.}
\end{itemize}
\begin{itemize}
\item {Proveniência:(De \textunderscore aleitar\textunderscore ^2)}
\end{itemize}
Diz-se da pedra, a que se preparou a face para leito.
\section{Aleitamento}
\begin{itemize}
\item {Grp. gram.:m.}
\end{itemize}
O mesmo que \textunderscore aleitação\textunderscore .
\section{Aleitar}
\begin{itemize}
\item {Grp. gram.:v. t.}
\end{itemize}
\begin{itemize}
\item {Proveniência:(Lat. \textunderscore lactare\textunderscore )}
\end{itemize}
Criar com leite; amamentar.
Tornar claro, como o leite.
\section{Aleitar}
\begin{itemize}
\item {Grp. gram.:v.}
\end{itemize}
\begin{itemize}
\item {Utilização:t. Constr.}
\end{itemize}
Ajustar e preparar a superfície de (uma pedra), para receber outra em cima.
\section{Aleitativo}
\begin{itemize}
\item {Grp. gram.:adj.}
\end{itemize}
Relativo á aleitação.
\section{Aleive}
\begin{itemize}
\item {Grp. gram.:m.}
\end{itemize}
Fraude.
Calúmnia.
\section{Aleivosamente}
\begin{itemize}
\item {Grp. gram.:adv.}
\end{itemize}
De modo \textunderscore aleivoso\textunderscore .
\section{Aleivosia}
\begin{itemize}
\item {Grp. gram.:f.}
\end{itemize}
Qualidade do que é aleivoso.
Aleive.
\section{Aleivoso}
\begin{itemize}
\item {Grp. gram.:adj.}
\end{itemize}
Que procede com aleive.
Fraudulento.
Em que há aleive.
\section{Aleixar}
\begin{itemize}
\item {Grp. gram.:v. t.}
\end{itemize}
\begin{itemize}
\item {Utilização:Ant.}
\end{itemize}
\begin{itemize}
\item {Proveniência:(Do lat. hyp. \textunderscore adlaxare\textunderscore )}
\end{itemize}
Afastar.
Separar.
Distanciar.
\section{Ale-larga}
\begin{itemize}
\item {Grp. gram.:f.}
\end{itemize}
\begin{itemize}
\item {Utilização:Náut.}
\end{itemize}
\begin{itemize}
\item {Proveniência:(De \textunderscore alar\textunderscore  + \textunderscore largar\textunderscore )}
\end{itemize}
Cabo, com que se recolhe a amarra, até suspender a âncora.
\section{Aleli}
\begin{itemize}
\item {Grp. gram.:m.}
\end{itemize}
\begin{itemize}
\item {Proveniência:(Do ár. \textunderscore al-kiri\textunderscore )}
\end{itemize}
Planta crucífera, de flôres rubras, raiadas de branco ou amarelas, e cheirosas.
Goiveiro.
Flôr de goiveiro.--Por corresponder ao \textunderscore leukoion\textunderscore  dos Gregos, há também quem designe por \textunderscore aleh\textunderscore  o lírio branco ou açucena.
\section{Além}
\begin{itemize}
\item {fónica:á-lém}
\end{itemize}
\begin{itemize}
\item {Grp. gram.:adv.}
\end{itemize}
Mais adeante.
Acolá.
Da parte de lá: \textunderscore além da ponte\textunderscore .
Longe; mais longe.
Para mais.
Ainda em cima.
Afóra.
(Contr. de \textunderscore alli\textunderscore  + ant. \textunderscore ende\textunderscore )
\section{Alemagem}
\textunderscore f.\textunderscore  (?)«\textunderscore ...dous fermosos bateis dalemagens...\textunderscore »(\textunderscore Ms.\textunderscore  do séc. XVII, em poder de Sousa Viterbo).
\section{Alemânico}
\begin{itemize}
\item {Grp. gram.:adj.}
\end{itemize}
\begin{itemize}
\item {Proveniência:(Lat. \textunderscore alemannicus\textunderscore )}
\end{itemize}
Relativo á Alemanha ou aos Alemães.
\section{Alemanisco}
\begin{itemize}
\item {Grp. gram.:adj.}
\end{itemize}
\begin{itemize}
\item {Utilização:Ant.}
\end{itemize}
Relativo á Alemanha: \textunderscore costumes alemaniscos\textunderscore . Cf. Jer. Cardoso, \textunderscore Diction\textunderscore .
\section{Alemanismo}
\begin{itemize}
\item {Grp. gram.:m.}
\end{itemize}
\begin{itemize}
\item {Proveniência:(De \textunderscore alemão\textunderscore )}
\end{itemize}
O mesmo que \textunderscore germanismo\textunderscore .
\section{Alemão}
\begin{itemize}
\item {Grp. gram.:adj.}
\end{itemize}
\begin{itemize}
\item {Grp. gram.:M.}
\end{itemize}
\begin{itemize}
\item {Proveniência:(Lat. \textunderscore alemannus\textunderscore )}
\end{itemize}
Relativo á Alemanha.
Habitante da Alemanha.
Língua dos alemães.
\section{Alembrar}
\begin{itemize}
\item {Grp. gram.:v. t.}
\end{itemize}
O mesmo que \textunderscore lembrar\textunderscore :«\textunderscore alembra-me que outrora conheci...\textunderscore »Castilho, \textunderscore Geórg\textunderscore .
\section{Além-eras}
\begin{itemize}
\item {Grp. gram.:loc. adv.}
\end{itemize}
Na eternidade. Cf. Castilho, \textunderscore Fastos\textunderscore , I, 308.
\section{Além-mar}
\begin{itemize}
\item {Grp. gram.:adv.}
\end{itemize}
No ultramar; além do mar.
\section{Além-mundo}
\begin{itemize}
\item {Grp. gram.:m.}
\end{itemize}
A outra vida, a eternidade. Cf. Castilho, \textunderscore Fastos\textunderscore , I, p. XXIX.
\section{Álemo}
\begin{itemize}
\item {Grp. gram.:m.}
\end{itemize}
O mesmo que \textunderscore álamo\textunderscore . Cf. João de Deus, \textunderscore Fl. do Campo\textunderscore , 130.
\section{Alendroal}
\begin{itemize}
\item {Grp. gram.:m.}
\end{itemize}
O mesmo que \textunderscore loendral\textunderscore :«\textunderscore Pero Rodríguez é do Alendroal.\textunderscore »Camões.
\section{Alentadamente}
\begin{itemize}
\item {Grp. gram.:adv.}
\end{itemize}
De modo \textunderscore alentado\textunderscore .
\section{Alentado}
\begin{itemize}
\item {Grp. gram.:adj.}
\end{itemize}
\begin{itemize}
\item {Grp. gram.:adj.}
\end{itemize}
\begin{itemize}
\item {Proveniência:(De \textunderscore alentar\textunderscore )}
\end{itemize}
Valente, esforçado.
Farto, suculento, bom.
\section{Alentador}
\begin{itemize}
\item {Grp. gram.:m.  e  adj.}
\end{itemize}
O que alenta.
\section{Alentar}
\begin{itemize}
\item {Grp. gram.:v. t.}
\end{itemize}
Dar alento, coragem, esfôrço, a.
Alimentar.
\section{Alentecer}
\begin{itemize}
\item {Grp. gram.:v. i.}
\end{itemize}
Fazer-se lento.
\section{Alentejanismo}
\begin{itemize}
\item {Grp. gram.:m.}
\end{itemize}
Locução privativa do Alentejo.
\section{Alentejano}
\begin{itemize}
\item {Grp. gram.:adj.}
\end{itemize}
\begin{itemize}
\item {Grp. gram.:M.}
\end{itemize}
Relativo ao Alentejo.
Habitante do Alentejo.
\section{Alentejão}
\begin{itemize}
\item {Grp. gram.:m.}
\end{itemize}
\begin{itemize}
\item {Utilização:Prov.}
\end{itemize}
\begin{itemize}
\item {Utilização:extrem.}
\end{itemize}
\begin{itemize}
\item {Utilização:Ant.}
\end{itemize}
O mesmo que \textunderscore alentejano\textunderscore . Cf. \textunderscore Anat. Joc.\textunderscore , I, 123.
\section{Alentejoado}
\begin{itemize}
\item {Grp. gram.:adj.}
\end{itemize}
\begin{itemize}
\item {Utilização:Des.}
\end{itemize}
O mesmo que [[lentejoilado|lentejoilar]].
\section{Alento}
\begin{itemize}
\item {Grp. gram.:m.}
\end{itemize}
\begin{itemize}
\item {Grp. gram.:Pl.}
\end{itemize}
\begin{itemize}
\item {Utilização:Ant.}
\end{itemize}
Hálito, respiração.
Ânimo; esfôrço.
Alimento.
Enthusiasmo.
Orifícios nas ventas dos cavallos.
Ornatos, de que usavam as freiras na toalha da cabeça.
(Cast. \textunderscore aliento\textunderscore )
\section{Áleo}
\begin{itemize}
\item {Grp. gram.:adj.}
\end{itemize}
\begin{itemize}
\item {Utilização:Heráld.}
\end{itemize}
\begin{itemize}
\item {Proveniência:(Do lat. \textunderscore ala\textunderscore )}
\end{itemize}
Alado, (falando-se do leão ou da serpente, representados com asas).
\section{Aleócara}
\begin{itemize}
\item {Grp. gram.:f.}
\end{itemize}
O mesmo que \textunderscore aleócaro\textunderscore .
\section{Aleócaro}
\begin{itemize}
\item {Grp. gram.:m.}
\end{itemize}
\begin{itemize}
\item {Proveniência:(Do gr. \textunderscore alea\textunderscore  + \textunderscore kherasso\textunderscore )}
\end{itemize}
Pequeno insecto coleóptero, que vive nos cogumelos.
\section{Aleonado}
\begin{itemize}
\item {Grp. gram.:adj.}
\end{itemize}
Fulvo; da côr do leão:«\textunderscore um saio de setim aleonado\textunderscore ». B. Pato, \textunderscore Port. na India\textunderscore , 13.
\section{Aleopardado}
\begin{itemize}
\item {Grp. gram.:adj.}
\end{itemize}
\begin{itemize}
\item {Utilização:Heráld.}
\end{itemize}
\begin{itemize}
\item {Proveniência:(De \textunderscore leopardo\textunderscore )}
\end{itemize}
Díz-se do leão, quando, no campo do escudo, se representa com as patas firmes, menos uma de deante, que está erguida na direcção do corpo.
\section{Alepina}
\begin{itemize}
\item {Grp. gram.:f.}
\end{itemize}
Certo estôfo de Alepo.
\section{Alépiro}
\begin{itemize}
\item {Grp. gram.:m.}
\end{itemize}
\begin{itemize}
\item {Proveniência:(Do gr. \textunderscore a\textunderscore  priv. + \textunderscore lepuron\textunderscore )}
\end{itemize}
Pequena planta ramosa da Austrália.
\section{Alepisauro}
\begin{itemize}
\item {fónica:sau}
\end{itemize}
\begin{itemize}
\item {Grp. gram.:m.}
\end{itemize}
\begin{itemize}
\item {Proveniência:(Do gr. \textunderscore a\textunderscore  priv. + \textunderscore lepis\textunderscore  + \textunderscore sauros\textunderscore )}
\end{itemize}
Peixe sem escama, da fam. dos salmonídeos.
\section{Alepissauro}
\begin{itemize}
\item {Grp. gram.:m.}
\end{itemize}
\begin{itemize}
\item {Proveniência:(Do gr. \textunderscore a\textunderscore  priv. + \textunderscore lepis\textunderscore  + \textunderscore sauros\textunderscore )}
\end{itemize}
Peixe sem escama, da fam. dos salmonídeos.
\section{Alepocéfalo}
\begin{itemize}
\item {Grp. gram.:adj.}
\end{itemize}
\begin{itemize}
\item {Proveniência:(Do gr. \textunderscore a\textunderscore  priv. + \textunderscore lepis\textunderscore  + \textunderscore kephale\textunderscore )}
\end{itemize}
Diz-se dos peixes, que não têm escamas na cabeça.
\section{Alepocéphalo}
\begin{itemize}
\item {Grp. gram.:adj.}
\end{itemize}
\begin{itemize}
\item {Proveniência:(Do gr. \textunderscore a\textunderscore  priv. + \textunderscore lepis\textunderscore  + \textunderscore kephale\textunderscore )}
\end{itemize}
Diz-se dos peixes, que não têm escamas na cabeça.
\section{Alépyro}
\begin{itemize}
\item {Grp. gram.:m.}
\end{itemize}
\begin{itemize}
\item {Proveniência:(Do gr. \textunderscore a\textunderscore  priv. + \textunderscore lepuron\textunderscore )}
\end{itemize}
Pequena planta ramosa da Austrália.
\section{Alequeado}
\begin{itemize}
\item {Grp. gram.:adj.}
\end{itemize}
\begin{itemize}
\item {Utilização:Bot.}
\end{itemize}
Que tem fórma de leque, (falando-se de certas fôlhas, como as das palmeiras).
\section{Alequeca}
\begin{itemize}
\item {Grp. gram.:f.}
\end{itemize}
O mesmo que \textunderscore laqueca\textunderscore . Cf. \textunderscore Peregrinação\textunderscore , CVII.
\section{Alerce}
\begin{itemize}
\item {Grp. gram.:m.}
\end{itemize}
\begin{itemize}
\item {Utilização:Des.}
\end{itemize}
O mesmo que \textunderscore lárice\textunderscore .
\section{Alereão}
\begin{itemize}
\item {Grp. gram.:m.}
\end{itemize}
\begin{itemize}
\item {Utilização:Heráld.}
\end{itemize}
\begin{itemize}
\item {Proveniência:(Fr. \textunderscore alerion\textunderscore )}
\end{itemize}
Pequena águia de asas abertas e sem bico nem pés.
\section{Alerta}
\begin{itemize}
\item {fónica:á-ler}
\end{itemize}
\begin{itemize}
\item {Grp. gram.:adv.}
\end{itemize}
\begin{itemize}
\item {Grp. gram.:M.}
\end{itemize}
\begin{itemize}
\item {Grp. gram.:Interj.}
\end{itemize}
\begin{itemize}
\item {Proveniência:(Do it. \textunderscore all'erta\textunderscore )}
\end{itemize}
Com vigilância; attentamente.
Sinal, para se estar vigilante.
Cautela! sentido!
\section{Alertar}
\begin{itemize}
\item {Grp. gram.:v. i.}
\end{itemize}
Pôr-se alerta:«\textunderscore alertam ao retinir das charamelas\textunderscore ». Filinto, \textunderscore D. Man. I\textunderscore , 408.
\section{Alestar}
\begin{itemize}
\item {fónica:lés-tár}
\end{itemize}
\begin{itemize}
\item {Grp. gram.:v. t.}
\end{itemize}
Tornar lesto, desembaraçado, prompto.
\section{Aleta}
\begin{itemize}
\item {fónica:lê}
\end{itemize}
\begin{itemize}
\item {Grp. gram.:f.}
\end{itemize}
\begin{itemize}
\item {Utilização:Archit.}
\end{itemize}
Pequena ala.
Lado de um membro ou \textunderscore pé-direito\textunderscore , collocado entre duas arcadas, a meio das quaes há ordinariamente uma columna ou pilastra. Cada uma das duas asas do nariz.
\section{Aletargado}
\begin{itemize}
\item {Grp. gram.:adj.}
\end{itemize}
Pôsto em lethargo.
\section{Alethargado}
\begin{itemize}
\item {Grp. gram.:adj.}
\end{itemize}
Pôsto em lethargo.
\section{Alethologia}
\begin{itemize}
\item {Grp. gram.:f.}
\end{itemize}
\begin{itemize}
\item {Proveniência:(Do gr. \textunderscore aletheia\textunderscore  + \textunderscore logos\textunderscore )}
\end{itemize}
Tratado ou discurso á cêrca da verdade.
\section{Alethológico}
\begin{itemize}
\item {Grp. gram.:adj.}
\end{itemize}
Relativo á \textunderscore alethologia\textunderscore .
\section{Aleto}
\begin{itemize}
\item {Grp. gram.:m.}
\end{itemize}
Ave de rapina, da India.
(Cp. \textunderscore alecto\textunderscore )
\section{Aletologia}
\begin{itemize}
\item {Grp. gram.:f.}
\end{itemize}
\begin{itemize}
\item {Proveniência:(Do gr. \textunderscore aletheia\textunderscore  + \textunderscore logos\textunderscore )}
\end{itemize}
Tratado ou discurso á cêrca da verdade.
\section{Aletológico}
\begin{itemize}
\item {Grp. gram.:adj.}
\end{itemize}
Relativo á \textunderscore alethologia\textunderscore .
\section{Aletradar-se}
\begin{itemize}
\item {Grp. gram.:v. p.}
\end{itemize}
Fazer-se letrado.
\section{Aletre}
\begin{itemize}
\item {Grp. gram.:m.}
\end{itemize}
Gênero de liliáceas.
\section{Aletríneas}
\begin{itemize}
\item {Grp. gram.:f. pl.}
\end{itemize}
Tribo de plantas liliáceas.
\section{Aletria}
\begin{itemize}
\item {Grp. gram.:f.}
\end{itemize}
\begin{itemize}
\item {Proveniência:(Do ár. \textunderscore al\textunderscore  + \textunderscore itria\textunderscore )}
\end{itemize}
Massa de farínha em fios delgados, espécie de macarrão.
\section{Aletrieiro}
\begin{itemize}
\item {Grp. gram.:m.}
\end{itemize}
Fabricante de aletria.
\section{Aletris}
\begin{itemize}
\item {Grp. gram.:m.}
\end{itemize}
\begin{itemize}
\item {Proveniência:(Do gr. \textunderscore aletris\textunderscore )}
\end{itemize}
Planta liliácea da América do Norte.
\section{Aléu}
\begin{itemize}
\item {Grp. gram.:m.}
\end{itemize}
\begin{itemize}
\item {Utilização:Ant.}
\end{itemize}
Jôgo de truque.
Pau, com que se impelle a bola, no jôgo da choca.
\section{Aléu}
\begin{itemize}
\item {Grp. gram.:m.}
\end{itemize}
\begin{itemize}
\item {Utilização:Prov.}
\end{itemize}
\begin{itemize}
\item {Utilização:alg.}
\end{itemize}
Descanso, alívio.
\section{Aleuria}
\begin{itemize}
\item {Grp. gram.:f.}
\end{itemize}
\begin{itemize}
\item {Proveniência:(Do gr. \textunderscore aleuron\textunderscore )}
\end{itemize}
Cogumelo, que se assemelha a um montículo de farinha.
\section{Aleurisma}
\begin{itemize}
\item {Grp. gram.:m.}
\end{itemize}
Môfo, bolor.
\section{Aleurite}
\begin{itemize}
\item {Grp. gram.:f.}
\end{itemize}
\begin{itemize}
\item {Proveniência:(Gr. \textunderscore aleuritis\textunderscore )}
\end{itemize}
Planta de Ceilão, da fam. das euphorbiáceas.
\section{Aleuromancia}
\begin{itemize}
\item {Grp. gram.:f.}
\end{itemize}
\begin{itemize}
\item {Proveniência:(Do gr. \textunderscore aleuron\textunderscore  + \textunderscore manteia\textunderscore )}
\end{itemize}
Antigo processo de adivinhação, por meio de farinha.
\section{Aleuromante}
\begin{itemize}
\item {Grp. gram.:m.}
\end{itemize}
Aquelle que pratica a aleuromancia.
\section{Aleurómetro}
\begin{itemize}
\item {Grp. gram.:m.}
\end{itemize}
\begin{itemize}
\item {Proveniência:(Do gr. \textunderscore aleuron\textunderscore  + \textunderscore metron\textunderscore )}
\end{itemize}
Pequeno instrumento, para medir a quantidade de glúten que há na farinha.
\section{Aleurona}
\begin{itemize}
\item {Grp. gram.:f.}
\end{itemize}
\begin{itemize}
\item {Proveniência:(Do gr. \textunderscore aleuron\textunderscore , farinha)}
\end{itemize}
Substância pulverulenta, fina e branca, contida nas céllulas de certas plantas, mormente das euphorbiáceas.
\section{Aleúte}
\begin{itemize}
\item {Grp. gram.:m.}
\end{itemize}
Língua hiperbórea, agglutinativa, das ilhas Aleútes, a noroéste da América.
\section{Alevadoiro}
\begin{itemize}
\item {Grp. gram.:m.}
\end{itemize}
Pau, com que se ergue a pedra da moenda.
(Por \textunderscore elevadoiro\textunderscore , de \textunderscore elevar\textunderscore )
\section{Alevadouro}
\begin{itemize}
\item {Grp. gram.:m.}
\end{itemize}
Pau, com que se ergue a pedra da moenda.
(Por \textunderscore elevadoiro\textunderscore , de \textunderscore elevar\textunderscore )
\section{Alevant...}
(V.levant...)
\section{Alevanto}
\begin{itemize}
\item {Grp. gram.:m.}
\end{itemize}
\begin{itemize}
\item {Utilização:Des.}
\end{itemize}
\begin{itemize}
\item {Proveniência:(De \textunderscore alevantar\textunderscore )}
\end{itemize}
Sublevação.
Alvorôto, motim.
\section{Alevedar}
\begin{itemize}
\item {Grp. gram.:v. t.}
\end{itemize}
(V.levedar)
\section{Alexandre}
\begin{itemize}
\item {Grp. gram.:adj.}
\end{itemize}
Diz-se de uma variedade de trigo rijo.
\section{Alexandrino}
\begin{itemize}
\item {Grp. gram.:adj.}
\end{itemize}
\begin{itemize}
\item {Grp. gram.:M.}
\end{itemize}
Relativo a Alexandria.
Habitante de Alexandria.
\section{Alexandrino}
\begin{itemize}
\item {Grp. gram.:adj.}
\end{itemize}
\begin{itemize}
\item {Grp. gram.:M.}
\end{itemize}
\begin{itemize}
\item {Proveniência:(De \textunderscore Alexandre\textunderscore , n. p.)}
\end{itemize}
Diz-se do verso de doze sýllabas, com accento tónico na 6.^a e na 12.^a.
Verso alexandrino.
\section{Alexandrite}
\begin{itemize}
\item {Grp. gram.:f.}
\end{itemize}
Pedra preciosa de côr.
\section{Alexia}
\begin{itemize}
\item {fónica:csi}
\end{itemize}
\begin{itemize}
\item {Grp. gram.:f.}
\end{itemize}
\begin{itemize}
\item {Proveniência:(Do gr. \textunderscore a\textunderscore  priv. + \textunderscore legein\textunderscore )}
\end{itemize}
Impossibilidade pathológica de lêr, em individuos que sabiam lêr e que não perderam o uso da vista.
\section{Alexiaco}
\begin{itemize}
\item {fónica:csi}
\end{itemize}
\begin{itemize}
\item {Grp. gram.:adj.}
\end{itemize}
O mesmo que \textunderscore alexiphármaco\textunderscore .
\section{Alexifármaco}
\begin{itemize}
\item {fónica:csi}
\end{itemize}
\begin{itemize}
\item {Grp. gram.:adj.}
\end{itemize}
\begin{itemize}
\item {Proveniência:(Gr. \textunderscore alexipharmakon\textunderscore )}
\end{itemize}
Diz-se dos remédios, que se applicam contra o veneno ingerido no estômago.
\section{Alexina}
\begin{itemize}
\item {Grp. gram.:f.}
\end{itemize}
Substância batericida, de soro sanguíneo normal.
\section{Alexiphármaco}
\begin{itemize}
\item {fónica:csi}
\end{itemize}
\begin{itemize}
\item {Grp. gram.:adj.}
\end{itemize}
\begin{itemize}
\item {Proveniência:(Gr. \textunderscore alexipharmakon\textunderscore )}
\end{itemize}
Diz-se dos remédios, que se applicam contra o veneno ingerido no estômago.
\section{Aleziriado}
\begin{itemize}
\item {Grp. gram.:adj.}
\end{itemize}
Cheio de lezírias.
\section{Alfa}
\begin{itemize}
\item {Grp. gram.:f.}
\end{itemize}
Planta gramínea da Argélia, (\textunderscore stipa tenacissima\textunderscore ).
\section{Alfa}
\begin{itemize}
\item {Grp. gram.:m.}
\end{itemize}
Sacerdote, entre os negros mahometanos do Senegal.
\section{Alfa}
\begin{itemize}
\item {Grp. gram.:f.}
\end{itemize}
\begin{itemize}
\item {Utilização:Prov.}
\end{itemize}
Marco, entre bens communs e particulares.
Fronteira.
\section{Alfa}
\begin{itemize}
\item {Grp. gram.:f.}
\end{itemize}
\begin{itemize}
\item {Utilização:T. de Moncorvo}
\end{itemize}
O mesmo que \textunderscore chamma\textunderscore .
Côr rosada das faces.
\section{Alfábar}
\begin{itemize}
\item {Grp. gram.:m.}
\end{itemize}
(V.alfâmbar)
\section{Alfabareiro}
\begin{itemize}
\item {Grp. gram.:m.}
\end{itemize}
Fabricante de alfábares.
\section{Alfábega}
\begin{itemize}
\item {Grp. gram.:f.}
\end{itemize}
\begin{itemize}
\item {Utilização:T. de Vizela}
\end{itemize}
O mesmo que \textunderscore alfavaca\textunderscore .
\section{Alfaça}
\begin{itemize}
\item {Grp. gram.:f.}
\end{itemize}
\begin{itemize}
\item {Utilização:Des.}
\end{itemize}
O mesmo que \textunderscore alface\textunderscore . Cf. G. Vicente, \textunderscore Reis Magos\textunderscore .
\section{Alfaçal}
\begin{itemize}
\item {Grp. gram.:m.}
\end{itemize}
Lugar, onde se criam alfaces.
\section{Alface}
\begin{itemize}
\item {Grp. gram.:f.}
\end{itemize}
\begin{itemize}
\item {Proveniência:(Do ár. \textunderscore al-cass\textunderscore )}
\end{itemize}
Planta herbácea, hortense, da fam. das compostas.
\section{Alface-de-cordeiro}
\begin{itemize}
\item {Grp. gram.:f.}
\end{itemize}
Planta valerianácea, (\textunderscore valeriana\textunderscore , \textunderscore locusta olitoria\textunderscore , Brotero).
\section{Alface-do-mar}
\begin{itemize}
\item {Grp. gram.:f.}
\end{itemize}
Espécie de moliço dos sapaes baixos, conhecido também por \textunderscore folhada\textunderscore . Cf. \textunderscore Museu Techn\textunderscore ., 54.
\section{Alfácia}
\begin{itemize}
\item {Grp. gram.:f.}
\end{itemize}
\begin{itemize}
\item {Utilização:T. dos vendilhões de Lisbôa}
\end{itemize}
O mesmo que \textunderscore alface\textunderscore .
\section{Alfacinha}
\begin{itemize}
\item {Grp. gram.:m.}
\end{itemize}
\begin{itemize}
\item {Utilização:Pop.}
\end{itemize}
\begin{itemize}
\item {Proveniência:(De \textunderscore alface\textunderscore )}
\end{itemize}
Habitante de Lisbôa.
\section{Alfaco}
\begin{itemize}
\item {Grp. gram.:m.}
\end{itemize}
Cogumelo de copa vermêlha.
(Ár. \textunderscore al-faque\textunderscore )
\section{Alfádega}
\begin{itemize}
\item {Grp. gram.:f.}
\end{itemize}
\begin{itemize}
\item {Utilização:Prov.}
\end{itemize}
\begin{itemize}
\item {Utilização:dur.}
\end{itemize}
\begin{itemize}
\item {Utilização:Ant.}
\end{itemize}
Mangericão de fôlha larga.
O mesmo que \textunderscore mangerona\textunderscore , segundo um dicc. ms., archivado na Tôrre do Tombo.--Supponho que é alter. de \textunderscore alfábega\textunderscore , uma das fórmas cast., correspondentes á nossa \textunderscore alfavaca\textunderscore . (Cf. C. Viana, \textunderscore Apostilas\textunderscore , vb. \textunderscore alfavaca\textunderscore ).
\section{Alfafa}
\begin{itemize}
\item {Grp. gram.:f.}
\end{itemize}
\begin{itemize}
\item {Utilização:Bras}
\end{itemize}
Nome vulgar da luzerna.
(Cast. \textunderscore alfalfa\textunderscore )
\section{Alfageme}
\begin{itemize}
\item {Grp. gram.:m.}
\end{itemize}
\begin{itemize}
\item {Utilização:Ant.}
\end{itemize}
\begin{itemize}
\item {Proveniência:(Do ár. \textunderscore al-haddgem\textunderscore )}
\end{itemize}
Barbeiro.
Fabricante de espadas, armeiro.
\section{Alfaia}
\begin{itemize}
\item {Grp. gram.:f.}
\end{itemize}
\begin{itemize}
\item {Utilização:Prov.}
\end{itemize}
\begin{itemize}
\item {Utilização:alent.}
\end{itemize}
\begin{itemize}
\item {Proveniência:(Do ár. \textunderscore al-caja\textunderscore )}
\end{itemize}
Utensilio de casas ou pessôas.
Adôrno.
Baixela.
Jóia.
Forquilha com três dentes.
\section{Alfaiamento}
\begin{itemize}
\item {Grp. gram.:m.}
\end{itemize}
Acto de \textunderscore alfaiar\textunderscore .
\section{Alfaiar}
\begin{itemize}
\item {Grp. gram.:v. t.}
\end{itemize}
Guarnecer com alfaias.
Mobilar.
Adornar.
\section{Alfaiata}
\begin{itemize}
\item {Grp. gram.:f.}
\end{itemize}
De \textunderscore alfaiate\textunderscore .
\section{Alfaiatar}
\begin{itemize}
\item {Grp. gram.:v. t.}
\end{itemize}
\begin{itemize}
\item {Grp. gram.:V. i.}
\end{itemize}
Coser ou talhar (peças de vestuário).
Exercer o offício de alfaiate.
\section{Alfaiataria}
\begin{itemize}
\item {Grp. gram.:f.}
\end{itemize}
Officina de alfaiate.
\section{Alfaiate}
\begin{itemize}
\item {Grp. gram.:m.}
\end{itemize}
\begin{itemize}
\item {Utilização:Pop.}
\end{itemize}
\begin{itemize}
\item {Utilização:T. da Bairrada}
\end{itemize}
\begin{itemize}
\item {Proveniência:(Do ár. \textunderscore al-caiate\textunderscore )}
\end{itemize}
Aquelle que faz vestuário para homens.
Áve ribeirinha, (\textunderscore recurvirostra avocetta\textunderscore , Lin.).
Insecto, o mesmo que joaninha.
Insecto aquático, de compridas pernas, chamado também \textunderscore cabra\textunderscore .
\section{Alfaifa}
\begin{itemize}
\item {Grp. gram.:f.}
\end{itemize}
\begin{itemize}
\item {Utilização:Ant.}
\end{itemize}
O mesmo que \textunderscore alfafa\textunderscore .
\section{Alfaique}
\begin{itemize}
\item {Grp. gram.:m.}
\end{itemize}
O mesmo que \textunderscore alfaque\textunderscore . Cf. \textunderscore Peregrinação\textunderscore , XLVIII.
\section{Alfaizar}
\begin{itemize}
\item {Grp. gram.:m.}
\end{itemize}
(V.alfeizar)
\section{Alfalema}
\begin{itemize}
\item {Grp. gram.:m.}
\end{itemize}
\begin{itemize}
\item {Utilização:Ant.}
\end{itemize}
Espécie de turbante ou gorro. Cf. G. Vicente. III, 270.
\section{Alfama}
\begin{itemize}
\item {Grp. gram.:m.}
\end{itemize}
\begin{itemize}
\item {Utilização:Ant.}
\end{itemize}
Bairro ou alcoice de judeus.
Asylo, refúgio.
(Do ár.)
\section{Alfâmar}
\begin{itemize}
\item {Grp. gram.:m.}
\end{itemize}
\begin{itemize}
\item {Utilização:Ant.}
\end{itemize}
O mesmo que \textunderscore alfâmbar\textunderscore .
\section{Alfâmbar}
\begin{itemize}
\item {Grp. gram.:m.}
\end{itemize}
\begin{itemize}
\item {Utilização:Ant.}
\end{itemize}
\begin{itemize}
\item {Proveniência:(Do ár. \textunderscore al-hambal\textunderscore )}
\end{itemize}
Cobertor de lan, vermelho.
\section{Alfambareiro}
\begin{itemize}
\item {Grp. gram.:m.}
\end{itemize}
Aquelle que fabricava alfâmbares.
\section{Alfamista}
\begin{itemize}
\item {Grp. gram.:m.}
\end{itemize}
Habitante de Alfama, em Lisbôa.
Tunante; fadista.
\section{Alfanado}
\begin{itemize}
\item {Grp. gram.:m.}
\end{itemize}
\begin{itemize}
\item {Utilização:Prov.}
\end{itemize}
O mesmo que \textunderscore piadeira\textunderscore .
\section{Alfanar}
\begin{itemize}
\item {Grp. gram.:v. t.}
\end{itemize}
\begin{itemize}
\item {Utilização:Ant.}
\end{itemize}
O mesmo que \textunderscore alfenar\textunderscore .
\section{Alfândega}
\begin{itemize}
\item {Grp. gram.:f.}
\end{itemize}
\begin{itemize}
\item {Utilização:Ant.}
\end{itemize}
\begin{itemize}
\item {Utilização:Pop.}
\end{itemize}
\begin{itemize}
\item {Proveniência:(Do ár. \textunderscore alfondak\textunderscore )}
\end{itemize}
Repartição pública, onde se cobram os direitos de entradas e saídas de mercadorias.
Casa, em que está essa repartição.
Aduana.
Depósito; trapiche.
O mesmo que \textunderscore albergaria\textunderscore .
Casa de muito movimento, de muita azáfama.
\section{Alfandegagem}
\begin{itemize}
\item {Grp. gram.:f.}
\end{itemize}
Acto de \textunderscore alfandegar\textunderscore .
\section{Alfandegamento}
\begin{itemize}
\item {Grp. gram.:m.}
\end{itemize}
\begin{itemize}
\item {Utilização:Bras}
\end{itemize}
O mesmo que \textunderscore alfandegagem\textunderscore .
\section{Alfandegar}
\begin{itemize}
\item {Grp. gram.:v. t.}
\end{itemize}
Despachar na alfândega.
Armazenar na alfândega.
\section{Alfandegário}
\begin{itemize}
\item {Grp. gram.:adj.}
\end{itemize}
O mesmo que \textunderscore alfandegueiro\textunderscore .
\section{Alfandegueiro}
\begin{itemize}
\item {Grp. gram.:adj.}
\end{itemize}
Relativo á alfândega.
Aduaneiro:«\textunderscore preias alfandegueiras.\textunderscore »Camillo, \textunderscore Regicida\textunderscore , 3.^a ed., 11.
\section{Alfaneque}
\begin{itemize}
\item {Grp. gram.:m.}
\end{itemize}
Espécie de falcão.
Pequeno quadrúpede africano.
(Ár. \textunderscore al-faneque\textunderscore )
\section{Alfanjada}
\begin{itemize}
\item {Grp. gram.:f.}
\end{itemize}
Golpe de alfange.
\section{Alfanjado}
\begin{itemize}
\item {Grp. gram.:f. adj.}
\end{itemize}
Semelhante ao alfange.
\section{Alfange}
\begin{itemize}
\item {Grp. gram.:m.}
\end{itemize}
Sabre de fôlha larga e curta.
(Ár. \textunderscore al-canjar\textunderscore )
\section{Alfaque}
\begin{itemize}
\item {Grp. gram.:m.}
\end{itemize}
Banco de areia movediça; recife.
Fundão, produzido pelos ferros que ali garram.
(Cp. \textunderscore alfaqueque\textunderscore )
\section{Alfaqueque}
\begin{itemize}
\item {fónica:quê}
\end{itemize}
\begin{itemize}
\item {Grp. gram.:m.}
\end{itemize}
\begin{itemize}
\item {Utilização:Ant.}
\end{itemize}
\begin{itemize}
\item {Utilização:Prov.}
\end{itemize}
\begin{itemize}
\item {Utilização:alg.}
\end{itemize}
\begin{itemize}
\item {Proveniência:(Do ár. \textunderscore al-faqueque\textunderscore )}
\end{itemize}
Emissário.
Aquelle que ia encarregado de resgatar cativos.
Peixe-gallo.
\section{Alfaquete}
\begin{itemize}
\item {fónica:quê}
\end{itemize}
\begin{itemize}
\item {Grp. gram.:m.}
\end{itemize}
Peixe, o mesmo que \textunderscore al-faqueque\textunderscore , (\textunderscore zeus faber\textunderscore , Lin.).
\section{Alfaqui}
\begin{itemize}
\item {Grp. gram.:m.}
\end{itemize}
Legista e sacerdote entre os Muçulmanos.
(Ár. \textunderscore al-faqui\textunderscore )
\section{Alfaquim}
\begin{itemize}
\item {Grp. gram.:m.}
\end{itemize}
\begin{itemize}
\item {Proveniência:(De \textunderscore al-faque\textunderscore ?)}
\end{itemize}
O mesmo que \textunderscore peixe-gallo\textunderscore .
\section{Alfaquique}
\begin{itemize}
\item {Grp. gram.:m.}
\end{itemize}
O mesmo que \textunderscore alfaquim\textunderscore .
\section{Alfar}
\begin{itemize}
\item {Grp. gram.:v. i.}
\end{itemize}
\begin{itemize}
\item {Utilização:T. de Moncorvo}
\end{itemize}
O mesmo que [[abostar-se|abastar]].
\section{Alfaras}
\begin{itemize}
\item {Grp. gram.:m.}
\end{itemize}
\begin{itemize}
\item {Proveniência:(Do ár. \textunderscore al-faras\textunderscore )}
\end{itemize}
Cavallo árabe, exercitado na guerra.
Cavalleiro destro, bem montado.
\section{Alfarda}
\begin{itemize}
\item {Grp. gram.:f.}
\end{itemize}
\begin{itemize}
\item {Utilização:Ant.}
\end{itemize}
Espécie de vestuário feminino, usado talvez só por mulheres burguesas ou plebeias.
(Ár. \textunderscore al-farda\textunderscore ?)
\section{Alfarema}
\begin{itemize}
\item {Grp. gram.:f.}
\end{itemize}
\begin{itemize}
\item {Utilização:Ant.}
\end{itemize}
\begin{itemize}
\item {Proveniência:(Do ár. \textunderscore al-harem\textunderscore )}
\end{itemize}
Touca ou véu para a cabeça.
\section{Alfareme}
\begin{itemize}
\item {Grp. gram.:m.}
\end{itemize}
O mesmo que \textunderscore alfarema\textunderscore .
\section{Alfarge}
\begin{itemize}
\item {Grp. gram.:m.}
\end{itemize}
\begin{itemize}
\item {Utilização:Ant.}
\end{itemize}
Moínho de vento.
(Da mesma or. que \textunderscore alfarja\textunderscore )
\section{Alfário}
\begin{itemize}
\item {Grp. gram.:adj.}
\end{itemize}
\begin{itemize}
\item {Proveniência:(De \textunderscore alfarás\textunderscore )}
\end{itemize}
Diz-se do cavallo que brinca, saltando e rinchando.
\section{Alfarío}
\begin{itemize}
\item {Grp. gram.:adj.}
\end{itemize}
\begin{itemize}
\item {Proveniência:(De \textunderscore alfarás\textunderscore )}
\end{itemize}
Diz-se do cavallo que brinca, saltando e rinchando.
\section{Alfarja}
\begin{itemize}
\item {Grp. gram.:f.}
\end{itemize}
\begin{itemize}
\item {Utilização:Prov.}
\end{itemize}
\begin{itemize}
\item {Utilização:trasm.}
\end{itemize}
\begin{itemize}
\item {Proveniência:(Do ár. \textunderscore al-farche\textunderscore )}
\end{itemize}
Grande vaso de pedra, em que gira a roda que mói a azeitona.
\section{Alfarrábio}
\begin{itemize}
\item {Grp. gram.:m.}
\end{itemize}
\begin{itemize}
\item {Proveniência:(Do ár. ?)}
\end{itemize}
Livro antigo e pouco prestadio.
\section{Alfarrabista}
\begin{itemize}
\item {Grp. gram.:m.}
\end{itemize}
\begin{itemize}
\item {Proveniência:(De \textunderscore alfarrábio\textunderscore )}
\end{itemize}
Aquelle que lê ou collecciona livros antigos. Aquelle que negocía com êlles.
\section{Alfarreca}
\begin{itemize}
\item {Grp. gram.:f.}
\end{itemize}
\begin{itemize}
\item {Utilização:ant.}
\end{itemize}
\begin{itemize}
\item {Utilização:Gír.}
\end{itemize}
Cabelleira.
(Alter. de \textunderscore alforreca\textunderscore )
\section{Alfarricoque}
\begin{itemize}
\item {Grp. gram.:m.}
\end{itemize}
\begin{itemize}
\item {Utilização:Ant.}
\end{itemize}
Homem de pequena estatura.
(Cp. \textunderscore farricoco\textunderscore )
\section{Alfarroba}
\begin{itemize}
\item {fónica:rô}
\end{itemize}
\begin{itemize}
\item {Grp. gram.:f.}
\end{itemize}
\begin{itemize}
\item {Proveniência:(Do ár. \textunderscore al-carroba\textunderscore )}
\end{itemize}
Fruto da alfarrobeira.
\section{Alfarrobado}
\begin{itemize}
\item {Grp. gram.:adj.}
\end{itemize}
\begin{itemize}
\item {Proveniência:(De \textunderscore alfarrobar\textunderscore )}
\end{itemize}
Diz-se da ameixa e de outras frutas, que se deformaram, fazendo lembrar a alfarroba.
\section{Alfarrobal}
\begin{itemize}
\item {Grp. gram.:m.}
\end{itemize}
Lugar, onde há alfarrobeiras.
\section{Alfarrobar}
\begin{itemize}
\item {Grp. gram.:v. t.}
\end{itemize}
Esfregar (linhas de pesca) com alfarroba verde, para as tornar mais rijas e mais escuras.
\section{Alfarrobeira}
\begin{itemize}
\item {Grp. gram.:f.}
\end{itemize}
\begin{itemize}
\item {Proveniência:(De \textunderscore alfarroba\textunderscore )}
\end{itemize}
Árvore leguminosa, (\textunderscore ceratonia siliqua\textunderscore , Lin.).
\section{Alfar-se}
\begin{itemize}
\item {Grp. gram.:v. p.}
\end{itemize}
\begin{itemize}
\item {Utilização:Prov.}
\end{itemize}
\begin{itemize}
\item {Utilização:trasm.}
\end{itemize}
Adquirir alforra.
Engelhar e secar, antes de formado o grão, (falando-se de searas).
Secar, apresentando malhas, (falando-se de frutos).
\section{Alfarva}
\begin{itemize}
\item {Grp. gram.:f.}
\end{itemize}
Planta daninha, que ataca os trigaes, (\textunderscore trigonella foenum graecum\textunderscore ).
(B. lat. \textunderscore alfarfa\textunderscore )
\section{Alfas}
\begin{itemize}
\item {Grp. gram.:f. pl.}
\end{itemize}
\begin{itemize}
\item {Utilização:Prov.}
\end{itemize}
\begin{itemize}
\item {Utilização:trasm.}
\end{itemize}
Malhas, que atacam as searas e os frutos, fazendo-os secar antes do tempo.
(Cp. \textunderscore alfar-se\textunderscore )
\section{Alfas}
\begin{itemize}
\item {Grp. gram.:f. pl.}
\end{itemize}
\begin{itemize}
\item {Utilização:Prov.}
\end{itemize}
\begin{itemize}
\item {Utilização:trasm.}
\end{itemize}
Evaporações quentes, que afogueiam a cara do transeunte, partindo de uma parede caiada, onde o sol bate de chapa, ou de uma estrada arenosa que o sol aqueceu.
\section{Alfavaca}
\begin{itemize}
\item {Grp. gram.:f.}
\end{itemize}
\begin{itemize}
\item {Utilização:T. do Ribatejo}
\end{itemize}
\begin{itemize}
\item {Proveniência:(Do ár. \textunderscore al-habaq\textunderscore )}
\end{itemize}
Planta labiada, semelhante ao mangericão.
Flôr da oliveira.
\section{Alfavaca-de-cobra}
\begin{itemize}
\item {Grp. gram.:f.}
\end{itemize}
Planta urticácea.
\section{Alfavaca-do-campo}
\begin{itemize}
\item {Grp. gram.:f.}
\end{itemize}
Planta aromática, brasileira.
Segurelha.
\section{Alfazar}
\begin{itemize}
\item {Grp. gram.:m.}
\end{itemize}
\begin{itemize}
\item {Utilização:Ant.}
\end{itemize}
Estrada; caminho largo.
(Talvez do ár. \textunderscore al-fesha\textunderscore )
\section{Alfazema}
\begin{itemize}
\item {Grp. gram.:f.}
\end{itemize}
\begin{itemize}
\item {Proveniência:(Do ár. \textunderscore al-cuzema\textunderscore )}
\end{itemize}
Arbusto odorífero, da fam. das labiadas.
\section{Alfazemar}
\begin{itemize}
\item {Grp. gram.:v. t.}
\end{itemize}
Defumar ou perfumar com alfazema. Cf. Arn. Gama, \textunderscore Motim\textunderscore , 335.
\section{Alfeça}
\begin{itemize}
\item {Grp. gram.:f.}
\end{itemize}
Pedaço de ferro, vazado no centro, e sôbre o qual se assenta a chapa em que se quere fazer abertura por meio de puncção, que, atravessando a chapa, entra na parte vazada.
O cravo ou puncção, com que se faz aquella abertura.
Alvião; picareta.
(Cp. \textunderscore alferce\textunderscore )
\section{Alfece}
\begin{itemize}
\item {Grp. gram.:m.}
\end{itemize}
O mesmo que \textunderscore alfeça\textunderscore .
\section{Alfeirada}
\begin{itemize}
\item {Grp. gram.:f.}
\end{itemize}
\begin{itemize}
\item {Utilização:Prov.}
\end{itemize}
\begin{itemize}
\item {Utilização:alent.}
\end{itemize}
Alfeire.
Rebanho de gados de alfeire.
\section{Alfeire}
\begin{itemize}
\item {Grp. gram.:m.}
\end{itemize}
\begin{itemize}
\item {Proveniência:(Do ár. \textunderscore al-heire\textunderscore )}
\end{itemize}
Curral de porcos.
Gado, que não cria.
Rebanho de ovelhas, que não tiveram nem estão para têr borregos.
Terreno cerrado, em que se recolhem porcos.
\section{Alfeireiro}
\begin{itemize}
\item {Grp. gram.:m.}
\end{itemize}
Aquelle que guarda o alfeire.
\section{Alfeirio}
\begin{itemize}
\item {Grp. gram.:adj.}
\end{itemize}
O mesmo que \textunderscore alfeiro\textunderscore ^1.
\section{Alfeiro}
\begin{itemize}
\item {Grp. gram.:adj.}
\end{itemize}
\begin{itemize}
\item {Grp. gram.:M.}
\end{itemize}
Relativo ao gado que não tem crias.
O mesmo que \textunderscore alfeire\textunderscore .
\section{Alfeiro}
\begin{itemize}
\item {Grp. gram.:adj.}
\end{itemize}
\begin{itemize}
\item {Utilização:Prov.}
\end{itemize}
Tomado do cio.
Irrequieto. (Colhido em Turquel)
(Relaciona-se com \textunderscore alfário\textunderscore ?)
\section{Alfeizar}
\begin{itemize}
\item {Grp. gram.:m.}
\end{itemize}
Pau, que se encaixa nas testeiras da serra.
(Cast. \textunderscore alfeizar\textunderscore )
\section{Alféloa}
\begin{itemize}
\item {Grp. gram.:f.}
\end{itemize}
\begin{itemize}
\item {Proveniência:(Do ár. \textunderscore al-helua\textunderscore )}
\end{itemize}
Massa de açúcar ou melaço em ponto, de que se fabricam objectos de confeitaria.
\section{Alfeloeiro}
\begin{itemize}
\item {Grp. gram.:m.}
\end{itemize}
Aquelle que negocía em alféloa.
Confeiteiro.
\section{Alfena}
\begin{itemize}
\item {Grp. gram.:f.}
\end{itemize}
\begin{itemize}
\item {Proveniência:(Do ár. \textunderscore al-hinna\textunderscore )}
\end{itemize}
Arbusto, da fam. das oleáceas, (\textunderscore ligustrum vulgare\textunderscore ).
\section{Alfenado}
\begin{itemize}
\item {Grp. gram.:adj.}
\end{itemize}
\begin{itemize}
\item {Proveniência:(De \textunderscore alfenar\textunderscore )}
\end{itemize}
Enfeitado pretensiosamente; ajanotado.
\section{Alfenar}
\begin{itemize}
\item {Grp. gram.:v. t.}
\end{itemize}
Tingir com baga ou pós de alfena.
Enfeitar.
Tornar effeminado.
\section{Alfeneiro}
\begin{itemize}
\item {Grp. gram.:m.}
\end{itemize}
(V.alfena)
\section{Alfenheiro}
\begin{itemize}
\item {Grp. gram.:m.}
\end{itemize}
O mesmo que \textunderscore alfeneiro\textunderscore .
\section{Alfeni}
\begin{itemize}
\item {Grp. gram.:m.}
\end{itemize}
O mesmo que \textunderscore alfenim\textunderscore .
\section{Alfenicado}
\begin{itemize}
\item {Grp. gram.:adj.}
\end{itemize}
\begin{itemize}
\item {Utilização:Neol.}
\end{itemize}
Que tem modos de alfenim ou casquilho.
\section{Alfenim}
\begin{itemize}
\item {Grp. gram.:m.}
\end{itemize}
\begin{itemize}
\item {Proveniência:(Do ár. \textunderscore al-fenid\textunderscore )}
\end{itemize}
Massa branca de açúcar e óleo de amêndoas doces.
Pessôa delicada, melindrosa.
Janota, peralta.
\section{Alfeninado}
\begin{itemize}
\item {Grp. gram.:adj.}
\end{itemize}
O mesmo que \textunderscore alfenado\textunderscore .
\section{Alfeninar-se}
\begin{itemize}
\item {Grp. gram.:v. p.}
\end{itemize}
\begin{itemize}
\item {Proveniência:(De \textunderscore alfenim\textunderscore )}
\end{itemize}
Tornar-se frágil, delicado.
Fazer-se effeminado.
\section{Alferce}
\begin{itemize}
\item {Grp. gram.:m.}
\end{itemize}
\begin{itemize}
\item {Proveniência:(Do ár. \textunderscore al-fes\textunderscore )}
\end{itemize}
Alvião.
Picareta.
\section{Alférece}
\begin{itemize}
\item {Grp. gram.:m.}
\end{itemize}
O mesmo que \textunderscore alferes\textunderscore . Cf. \textunderscore Eurico\textunderscore , 94.
\section{Alferena}
\begin{itemize}
\item {Grp. gram.:f.}
\end{itemize}
\begin{itemize}
\item {Utilização:Ant.}
\end{itemize}
\begin{itemize}
\item {Proveniência:(De \textunderscore alferes\textunderscore )}
\end{itemize}
A bandeira, que o alferes levava, em occasião de guerra ou de qualquer expedição militar.
\section{Alferes}
\begin{itemize}
\item {Grp. gram.:m.}
\end{itemize}
\begin{itemize}
\item {Utilização:Ant.}
\end{itemize}
\begin{itemize}
\item {Proveniência:(Do ár. \textunderscore al-feris\textunderscore )}
\end{itemize}
Official do exército português, immediatamente inferior ao tenente.
Peixe dos Açores.
Porta-bandeira.
\textunderscore Fazer pé de alferes\textunderscore , fazer côrte, galantear, namorar.
\section{Alférez}
\begin{itemize}
\item {Grp. gram.:m.}
\end{itemize}
\begin{itemize}
\item {Utilização:Ant.}
\end{itemize}
\begin{itemize}
\item {Proveniência:(Do ár. \textunderscore al-feris\textunderscore )}
\end{itemize}
Official do exército português, immediatamente inferior ao tenente.
Peixe dos Açores.
Porta-bandeira.
\textunderscore Fazer pé de alferes\textunderscore , fazer côrte, galantear, namorar.
\section{Alferga}
\begin{itemize}
\item {Grp. gram.:f.}
\end{itemize}
\begin{itemize}
\item {Utilização:Prov.}
\end{itemize}
Medida de semente de sirgo.
\section{Alferro}
\begin{itemize}
\item {Grp. gram.:f.}
\end{itemize}
\begin{itemize}
\item {Utilização:Prov.}
\end{itemize}
\begin{itemize}
\item {Utilização:alent.}
\end{itemize}
Jôgo de rapazes, também conhecido por \textunderscore porca\textunderscore .
\section{Afétena}
\begin{itemize}
\item {Grp. gram.:f.}
\end{itemize}
\begin{itemize}
\item {Utilização:Obsol.}
\end{itemize}
\begin{itemize}
\item {Proveniência:(Do ár. \textunderscore al-fitna\textunderscore )}
\end{itemize}
Hostilidade; guerra.
\section{Alfil}
\begin{itemize}
\item {Grp. gram.:m.}
\end{itemize}
A peça que no xadrez representa o elephante.
\section{Alfim}
\begin{itemize}
\item {Grp. gram.:adv.}
\end{itemize}
\begin{itemize}
\item {Utilização:Des.}
\end{itemize}
\begin{itemize}
\item {Proveniência:(De \textunderscore al\textunderscore ^3 + \textunderscore fim\textunderscore )}
\end{itemize}
Em-fim.
\section{Alfinago}
\begin{itemize}
\item {Grp. gram.:m.}
\end{itemize}
\begin{itemize}
\item {Utilização:Ant.}
\end{itemize}
Pandilha; biltre.
\section{Alfinetada}
\begin{itemize}
\item {Grp. gram.:f.}
\end{itemize}
Acto de \textunderscore alfinetar\textunderscore .
\section{Alfinetadela}
\begin{itemize}
\item {Grp. gram.:f.}
\end{itemize}
O mesmo que \textunderscore alfinetada\textunderscore .
\section{Alfinetar}
\begin{itemize}
\item {Grp. gram.:v. t.}
\end{itemize}
\begin{itemize}
\item {Utilização:Fig.}
\end{itemize}
Picar com alfinete.
Dar fórma de alfinete a.
Satirizar.
Criticar, magoando.
Ferir com palavras.
\section{Alfinete}
\begin{itemize}
\item {fónica:nê}
\end{itemize}
\begin{itemize}
\item {Grp. gram.:m.}
\end{itemize}
\begin{itemize}
\item {Utilização:Prov.}
\end{itemize}
\begin{itemize}
\item {Utilização:minh.}
\end{itemize}
\begin{itemize}
\item {Grp. gram.:Pl.}
\end{itemize}
\begin{itemize}
\item {Utilização:Jur.}
\end{itemize}
\begin{itemize}
\item {Utilização:ant.}
\end{itemize}
\begin{itemize}
\item {Utilização:Marn.}
\end{itemize}
\begin{itemize}
\item {Utilização:Bras}
\end{itemize}
\begin{itemize}
\item {Proveniência:(Do ár. \textunderscore al-quilele\textunderscore )}
\end{itemize}
Hastezinha de metal, aguçada de um lado, limitada do outro por uma espécie de cabeça do mesmo metal, e que serve para pregar ou segurar peças de vestuário.
Objecto análogo de outra substância, para segurar o cabello das mulheres.
Broche de senhora.
Designação vulgar de um insecto, muito nocivo aos cereaes. Cf. \textunderscore Bibl. da Gente do Campo\textunderscore , 304.
Prestação periódica, prometida pelo marido a sua mulher para despesas particulares e adôrno della.
Crystaes de sal, compridos e finos, que, com o vento quente e sêco, apparecem entre os crystaes de sal commum nas marinhas. Cf. \textunderscore Museu Techn\textunderscore ., 72, 82 e 91.
O mesmo que \textunderscore silena\textunderscore .
\section{Alfinetear}
\begin{itemize}
\item {Grp. gram.:v. t.}
\end{itemize}
O mesmo que \textunderscore alfinetar\textunderscore .
\section{Alfinete-de-toucar}
\begin{itemize}
\item {Grp. gram.:m.}
\end{itemize}
Planta geraniácea, (\textunderscore geranium fulgidum\textunderscore ).
\section{Alfineteira}
\begin{itemize}
\item {Grp. gram.:f.}
\end{itemize}
Pregadeira de alfinetes.
\section{Alfineteiro}
\begin{itemize}
\item {Grp. gram.:m.}
\end{itemize}
Fabricante de alfinetes.
Aquelle que negocía em alfinetes.
O mesmo que \textunderscore alfineteira\textunderscore .
\section{Alfinetes-da-raínha}
\begin{itemize}
\item {Grp. gram.:m. pl.}
\end{itemize}
Antiga contribuição que se Pagava no Pôrto.
\section{Alfir}
\begin{itemize}
\item {Grp. gram.:m.}
\end{itemize}
O mesmo que \textunderscore alfil\textunderscore .
\section{Alfirme}
\begin{itemize}
\item {Grp. gram.:m.}
\end{itemize}
\begin{itemize}
\item {Utilização:Prov.}
\end{itemize}
\begin{itemize}
\item {Utilização:trasm.}
\end{itemize}
\begin{itemize}
\item {Utilização:Prov.}
\end{itemize}
\begin{itemize}
\item {Utilização:alg.}
\end{itemize}
\begin{itemize}
\item {Utilização:alent.}
\end{itemize}
Corda delgada de esparto.
Corda delgada, baraço.
Recinto, formado por cordas, em que as ovelhas são ordenhadas.
\section{Alfirmeira}
\begin{itemize}
\item {Grp. gram.:f.}
\end{itemize}
\begin{itemize}
\item {Utilização:Prov.}
\end{itemize}
\begin{itemize}
\item {Utilização:alg.}
\end{itemize}
Ovelha, que tem por costume evadir-se do alfirme.
\section{Alfitete}
\begin{itemize}
\item {Grp. gram.:m.}
\end{itemize}
\begin{itemize}
\item {Proveniência:(Do ár. \textunderscore al-fitita\textunderscore )}
\end{itemize}
Composição culinária de ovos, açucar, vinho e manteiga.
Pastelão.
Iguaria.
\section{Alfitra}
\begin{itemize}
\item {Grp. gram.:f.}
\end{itemize}
\begin{itemize}
\item {Utilização:Ant.}
\end{itemize}
Tributo, que pagavam os Moiros conquistados, e que constava da décima parte dos seus gados.
\section{Alfobre}
\begin{itemize}
\item {fónica:fô}
\end{itemize}
\begin{itemize}
\item {Grp. gram.:m.}
\end{itemize}
\begin{itemize}
\item {Proveniência:(Do ár. \textunderscore al-hofre\textunderscore )}
\end{itemize}
Viveiro, em que se semeiam plantas, e onde crescem até é sua transplantação.
Canteiro, entre dois regos por onde corre água.
\section{Alfôfre}
\begin{itemize}
\item {Grp. gram.:m.}
\end{itemize}
\begin{itemize}
\item {Utilização:Prov.}
\end{itemize}
\begin{itemize}
\item {Utilização:minh.}
\end{itemize}
\begin{itemize}
\item {Utilização:Ant.}
\end{itemize}
Pequena porção de terra.
(Cp. \textunderscore alfobre\textunderscore )
\section{Alfola}
\begin{itemize}
\item {fónica:fô}
\end{itemize}
\begin{itemize}
\item {Grp. gram.:f.}
\end{itemize}
\begin{itemize}
\item {Utilização:Ant.}
\end{itemize}
Vestuário mais ou menos precioso:«\textunderscore e nom casei por me cobrir de alfollas\textunderscore ». \textunderscore Canc. da Vaticana.\textunderscore 
\section{Alfolla}
\begin{itemize}
\item {fónica:fô}
\end{itemize}
\begin{itemize}
\item {Grp. gram.:f.}
\end{itemize}
\begin{itemize}
\item {Utilização:Ant.}
\end{itemize}
Vestuário mais ou menos precioso:«\textunderscore e nom casei por me cobrir de alfollas\textunderscore ». \textunderscore Canc. da Vaticana.\textunderscore 
\section{Alfombra}
\begin{itemize}
\item {Grp. gram.:f.}
\end{itemize}
\begin{itemize}
\item {Proveniência:(Do ár. \textunderscore al-homra\textunderscore )}
\end{itemize}
Tapete, alcatifa.
Campo arrelvado.
\section{Alfombrar}
\begin{itemize}
\item {Grp. gram.:v. t.}
\end{itemize}
Cobrir de alfombra.
Atapetar.
\section{Alfonsia}
\begin{itemize}
\item {Grp. gram.:f.}
\end{itemize}
Ferrugem das plantas; alfôrra.
\section{Alfonsim}
\begin{itemize}
\item {Grp. gram.:m.}
\end{itemize}
\begin{itemize}
\item {Proveniência:(De \textunderscore Alfonso\textunderscore , n. p.)}
\end{itemize}
Antiga moéda portuguesa.
\section{Alforfilhar}
\begin{itemize}
\item {Grp. gram.:v. i.}
\end{itemize}
\begin{itemize}
\item {Utilização:Ant.}
\end{itemize}
Fugir á socapa; esgueirar-se.
\section{Alforjada}
\begin{itemize}
\item {Grp. gram.:f.}
\end{itemize}
O que está dentro do alforje.
Porção de coisas várias.
\section{Alforjar}
\begin{itemize}
\item {Grp. gram.:v. t.}
\end{itemize}
Meter no alforge.
Meter nas algibeiras; arrecadar.
\section{Alforje}
\begin{itemize}
\item {Grp. gram.:m.}
\end{itemize}
\begin{itemize}
\item {Grp. gram.:Pl.}
\end{itemize}
\begin{itemize}
\item {Utilização:Náut.}
\end{itemize}
\begin{itemize}
\item {Proveniência:(Do ár. \textunderscore al-cordj\textunderscore )}
\end{itemize}
Espécie de saco, fechado nas extremidades, e aberto no meio, por onde se dobra, formando dois compartimentos.
O que o alforge leva.
Saliências nos dois cantos da popa.
\section{Alforjeiro}
\begin{itemize}
\item {Grp. gram.:m.}
\end{itemize}
Homem que traz alforje. Cp. Filinto, XII, 16.
\section{Alforna}
\begin{itemize}
\item {fónica:fór}
\end{itemize}
\begin{itemize}
\item {Grp. gram.:f.}
\end{itemize}
\begin{itemize}
\item {Utilização:Ant.}
\end{itemize}
O mesmo que \textunderscore alforva\textunderscore .
\section{Alfornes}
\begin{itemize}
\item {Grp. gram.:m. pl.}
\end{itemize}
Cabos que, em certas redes, partem da cadoira para as tralhas.
\section{Alfôrra}
\begin{itemize}
\item {Grp. gram.:f.}
\end{itemize}
\begin{itemize}
\item {Proveniência:(Do ár. \textunderscore al-harr\textunderscore )}
\end{itemize}
Cogumelo microscópico, que se desenvolve com a humidade nas searas, ennegrecendo o grão quando vem o calor.
Moléstia das searas, determinada pela presença daquelle cogumelo.
\section{Alforrar}
\begin{itemize}
\item {Grp. gram.:v. i.}
\end{itemize}
Produzir alfôrra; mostrar que tem alfôrra.
\section{Alforreca}
\begin{itemize}
\item {Grp. gram.:f.}
\end{itemize}
\begin{itemize}
\item {Proveniência:(Do ár. \textunderscore al-horreque\textunderscore )}
\end{itemize}
Mollusco, do feitio de umbrella e de tecídos semi-transparentes.
\section{Alforria}
\begin{itemize}
\item {Grp. gram.:f.}
\end{itemize}
\begin{itemize}
\item {Proveniência:(Do ár. ?)}
\end{itemize}
Liberdade, dada ao escravo pelo senhor.
Libertação.
\section{Alforriar}
\begin{itemize}
\item {Grp. gram.:v. t.}
\end{itemize}
Dar alforria a.
Libertar.
\section{Alforva}
\begin{itemize}
\item {fónica:fôr}
\end{itemize}
\begin{itemize}
\item {Grp. gram.:f.}
\end{itemize}
O mesmo que \textunderscore alfarva\textunderscore .
\section{Alforza}
\begin{itemize}
\item {Grp. gram.:f.}
\end{itemize}
\begin{itemize}
\item {Utilização:Ant.}
\end{itemize}
O mesmo que \textunderscore alfurja\textunderscore .
\section{Alfós}
\begin{itemize}
\item {Grp. gram.:m.}
\end{itemize}
\begin{itemize}
\item {Utilização:Ant.}
\end{itemize}
\begin{itemize}
\item {Proveniência:(Do ár. \textunderscore alhauz\textunderscore )}
\end{itemize}
Distrito autónomo.
Arrabalde, aros de uma povoação.
Terreno plano.
\section{Alfóstico}
\begin{itemize}
\item {Grp. gram.:m.}
\end{itemize}
O mesmo que \textunderscore alfóstigo\textunderscore .
\section{Alfóstigo}
\begin{itemize}
\item {Grp. gram.:m.}
\end{itemize}
\begin{itemize}
\item {Proveniência:(Do ár. \textunderscore al-fostaq\textunderscore )}
\end{itemize}
Árvore resinosa, da fam. das terebintháceas, (\textunderscore pistacia vera\textunderscore ).
\section{Alfostigueiro}
\begin{itemize}
\item {Grp. gram.:m.}
\end{itemize}
O mesmo que \textunderscore alfóstigo\textunderscore .
\section{Alfoucim}
\begin{itemize}
\item {Grp. gram.:m.}
\end{itemize}
Peixe dos Açores.
\section{Alfovre}
\begin{itemize}
\item {fónica:fô}
\end{itemize}
\begin{itemize}
\item {Grp. gram.:m.}
\end{itemize}
O mesmo que \textunderscore alfobre\textunderscore  e \textunderscore alfofre\textunderscore .
\section{Alfoz}
\begin{itemize}
\item {Grp. gram.:m.}
\end{itemize}
\begin{itemize}
\item {Utilização:Ant.}
\end{itemize}
\begin{itemize}
\item {Proveniência:(Do ár. \textunderscore alhauz\textunderscore )}
\end{itemize}
Distrito autónomo.
Arrabalde, aros de uma povoação.
Terreno plano.
\section{Alfreces}
\begin{itemize}
\item {Grp. gram.:m. pl.}
\end{itemize}
O mesmo que \textunderscore alfrezes\textunderscore .
\section{Alfrecha}
\begin{itemize}
\item {Grp. gram.:f.}
\end{itemize}
\begin{itemize}
\item {Utilização:T. de Moncorvo}
\end{itemize}
Variedade de pêssego.
\section{Alfrédia}
\begin{itemize}
\item {Grp. gram.:f.}
\end{itemize}
\begin{itemize}
\item {Proveniência:(De \textunderscore Alfredo\textunderscore , n. p.)}
\end{itemize}
Gênero de plantas da Sibéria, da fam. das compostas.
\section{Alfrezes}
\begin{itemize}
\item {Grp. gram.:m. pl.}
\end{itemize}
\begin{itemize}
\item {Utilização:Ant.}
\end{itemize}
Mobília de uma casa.
Panos ricos, próprios para armações.
Certos enfeites de vestuário.
\section{Alfridária}
\begin{itemize}
\item {Grp. gram.:f.}
\end{itemize}
Influência, que se suppunha sêr exercida pelos planetas, durante certo tempo.
\section{Alfrocheiro}
\begin{itemize}
\item {Grp. gram.:m.}
\end{itemize}
Casta de uva duriense e beirôa.
(Por \textunderscore alforjeiro\textunderscore , de \textunderscore alforje\textunderscore ?)
\section{Alfunda}
\begin{itemize}
\item {Grp. gram.:f.}
\end{itemize}
\begin{itemize}
\item {Utilização:Prov.}
\end{itemize}
O mesmo que \textunderscore funda\textunderscore ^1, para atirar pedras. (Colhido em Turquel)
\section{Alfur}
\begin{itemize}
\item {Grp. gram.:m.}
\end{itemize}
Uma das línguas da Malásia.
\section{Alfuras}
\begin{itemize}
\item {Grp. gram.:m. pl.}
\end{itemize}
Tribo meio selvagem das montanhas das Molucas.
\section{Alfurja}
\begin{itemize}
\item {Grp. gram.:f.}
\end{itemize}
Saguão.
Monturo.
(Ár. \textunderscore alfurja\textunderscore )
\section{Alfusqueiro}
\begin{itemize}
\item {Grp. gram.:m.}
\end{itemize}
Casta de uva do distrito de Aveiro.
\section{Alga}
\begin{itemize}
\item {Grp. gram.:f.}
\end{itemize}
\begin{itemize}
\item {Proveniência:(Lat. \textunderscore alga\textunderscore )}
\end{itemize}
Planta cryptogâmica, que vive no fundo ou á superfície das águas.
\section{Algáceo}
\begin{itemize}
\item {Grp. gram.:adj.}
\end{itemize}
Relativo a algas.
\section{Algaço}
\begin{itemize}
\item {Grp. gram.:m.}
\end{itemize}
\begin{itemize}
\item {Proveniência:(De \textunderscore alga\textunderscore )}
\end{itemize}
Designação genérica da vegetação que o mar expelle.
\section{Algália}
\begin{itemize}
\item {Grp. gram.:f.}
\end{itemize}
Sonda ôca, para extracção de urinas, e para outras applicações cirúrgicas.
(B. lat. \textunderscore algalia\textunderscore )
\section{Algália}
\begin{itemize}
\item {Grp. gram.:f.}
\end{itemize}
\begin{itemize}
\item {Proveniência:(Do ár. \textunderscore al-galia\textunderscore )}
\end{itemize}
Quadrúpede, semelhante á marta.
Gato-de-algália.
Almiscareiro.
\section{Algaliar}
\begin{itemize}
\item {Grp. gram.:v. t.}
\end{itemize}
Sondar com algália^1.
\section{Alganame}
\begin{itemize}
\item {Grp. gram.:m.}
\end{itemize}
\begin{itemize}
\item {Utilização:Ant.}
\end{itemize}
\begin{itemize}
\item {Proveniência:(Do ár. \textunderscore al-gannam\textunderscore )}
\end{itemize}
Maioral de pastores.
\section{Algar}
\begin{itemize}
\item {Grp. gram.:m.}
\end{itemize}
Caverna.
Gruta.
Despenhadeiro.
(Ár. \textunderscore al-gar\textunderscore )
\section{Algara}
\begin{itemize}
\item {Grp. gram.:f.}
\end{itemize}
\begin{itemize}
\item {Utilização:Ant.}
\end{itemize}
\begin{itemize}
\item {Proveniência:(Do ár. \textunderscore al-gara\textunderscore )}
\end{itemize}
Expedição militar; sortida; investida.
\section{Algarada}
\begin{itemize}
\item {Grp. gram.:f.}
\end{itemize}
\begin{itemize}
\item {Utilização:Ant.}
\end{itemize}
Algazarra, vozearia. Cf. Paganino, \textunderscore Contos\textunderscore .
Incursão, o mesmo que \textunderscore algara\textunderscore .
\section{Algarafa}
\begin{itemize}
\item {Grp. gram.:f.}
\end{itemize}
\begin{itemize}
\item {Utilização:Ant.}
\end{itemize}
\begin{itemize}
\item {Proveniência:(Do ár. \textunderscore al-garraf\textunderscore )}
\end{itemize}
O mesmo que \textunderscore garrafa\textunderscore .
\section{Algarão}
\begin{itemize}
\item {Grp. gram.:m.}
\end{itemize}
\begin{itemize}
\item {Utilização:Prov.}
\end{itemize}
\begin{itemize}
\item {Utilização:beir.}
\end{itemize}
Grande algar, grande caverna.
\section{Algaravazes}
\begin{itemize}
\item {Grp. gram.:m. pl.}
\end{itemize}
\begin{itemize}
\item {Utilização:Ant.}
\end{itemize}
Orla ou fímbria de vestido talar.
\section{Algaravia}
\begin{itemize}
\item {Grp. gram.:f.}
\end{itemize}
\begin{itemize}
\item {Proveniência:(Do ár. \textunderscore al-arabia\textunderscore )}
\end{itemize}
Linguagem árabe.
Confusão de vozes.
Linguagem diffícil de entender.
\section{Algaraviada}
\begin{itemize}
\item {Grp. gram.:f.}
\end{itemize}
\begin{itemize}
\item {Proveniência:(De \textunderscore algaravia\textunderscore )}
\end{itemize}
Confusão de vozes.
Berreiro.
Imbróglio.
\section{Algaraviar}
\begin{itemize}
\item {Grp. gram.:v. i.}
\end{itemize}
Falar ou escrever confusamente.
Fazer algaravia.
\section{Algaraviz}
\begin{itemize}
\item {Grp. gram.:m.}
\end{itemize}
\begin{itemize}
\item {Utilização:Des.}
\end{itemize}
Cano de ferro, que conduz o ar, dos folles ao ôlho da forja.
\section{Algarejo}
\begin{itemize}
\item {Grp. gram.:m.}
\end{itemize}
\begin{itemize}
\item {Utilização:T. de Serpa}
\end{itemize}
Pequeno algar ou furna, de mato crescido. Cf. \textunderscore Tradição\textunderscore .
\section{Algarido}
\begin{itemize}
\item {Grp. gram.:m.}
\end{itemize}
\begin{itemize}
\item {Utilização:Ant.}
\end{itemize}
O mesmo que \textunderscore alarido\textunderscore .
\section{Algarismeira}
\begin{itemize}
\item {Grp. gram.:f.}
\end{itemize}
\begin{itemize}
\item {Utilização:Prov.}
\end{itemize}
\begin{itemize}
\item {Utilização:trasm.}
\end{itemize}
\begin{itemize}
\item {Proveniência:(De \textunderscore algarismo\textunderscore )}
\end{itemize}
Mulher linguareira ou mexeriqueira, que ao que conta accrescenta sempre alguma coisa.
\section{Algarismo}
\begin{itemize}
\item {Grp. gram.:m.}
\end{itemize}
Cada um dos caracteres ou sinaes, de origem árabe, que representam os números.
\section{Algaroba}
\begin{itemize}
\item {Grp. gram.:f.}
\end{itemize}
\begin{itemize}
\item {Utilização:Bras}
\end{itemize}
Planta medicinal do Rio-Grande-do-Sul, applicada contra as ophthalmias.
\section{Algarve}
\begin{itemize}
\item {Grp. gram.:m.}
\end{itemize}
\begin{itemize}
\item {Utilização:Prov.}
\end{itemize}
\begin{itemize}
\item {Utilização:alent.}
\end{itemize}
\begin{itemize}
\item {Grp. gram.:Pl.}
\end{itemize}
\begin{itemize}
\item {Utilização:Des.}
\end{itemize}
\begin{itemize}
\item {Proveniência:(De \textunderscore Algarve\textunderscore , n. p.)}
\end{itemize}
Dá-se êste nome nas feiras aos productos algarvios ou ao lugar onde se vendem êsses productos, como esteiras, figos, etc.
Tripulantes das reaes galés.
\section{Algarvio}
\begin{itemize}
\item {Grp. gram.:adj.}
\end{itemize}
\begin{itemize}
\item {Utilização:Fig.}
\end{itemize}
\begin{itemize}
\item {Grp. gram.:M.}
\end{itemize}
Relativo ao Algarve.
Tagarela; palrador.
Habitante do Algarve.
Remador.
\section{Algarvismo}
\begin{itemize}
\item {Grp. gram.:m.}
\end{itemize}
Palavra ou locução privativa do Algarve.
\section{Algaz}
\begin{itemize}
\item {Grp. gram.:m.}
\end{itemize}
Fruto de algumas palmeiras.
\section{Algazarra}
\begin{itemize}
\item {Grp. gram.:f.}
\end{itemize}
\begin{itemize}
\item {Proveniência:(Do ár. \textunderscore al-gazarra\textunderscore )}
\end{itemize}
Vozearia; assuada; clamor.
Tumulto.
\section{Algazarrar}
\begin{itemize}
\item {Grp. gram.:v. i.}
\end{itemize}
\begin{itemize}
\item {Utilização:Neol.}
\end{itemize}
Fazer algazarra.
\section{Algazela}
\begin{itemize}
\item {Grp. gram.:f.}
\end{itemize}
(V.gazela)
\section{Algazu}
\begin{itemize}
\item {Grp. gram.:f.}
\end{itemize}
\begin{itemize}
\item {Utilização:Ant.}
\end{itemize}
\begin{itemize}
\item {Proveniência:(Do ár. \textunderscore al-gazu\textunderscore )}
\end{itemize}
Pregão de guerra dos Moiros contra os Christãos.
\section{Algazuna}
\begin{itemize}
\item {Grp. gram.:f.}
\end{itemize}
\begin{itemize}
\item {Utilização:Ant.}
\end{itemize}
Hoste? Cf. Cortesão, \textunderscore Subs\textunderscore .
\section{Álgebra}
\begin{itemize}
\item {Grp. gram.:f.}
\end{itemize}
\begin{itemize}
\item {Proveniência:(Do ár. \textunderscore al-jebre\textunderscore )}
\end{itemize}
Sciência, que generaliza as questões relativas aos números, e representa as grandezas por três espécies de sinaes geraes.
O mesmo que \textunderscore orthopedia\textunderscore .
\section{Algebria}
\begin{itemize}
\item {Grp. gram.:f.}
\end{itemize}
Arte de algebrista. Cf. Filinto, I, 196.
\section{Algebricamente}
\begin{itemize}
\item {fónica:gé}
\end{itemize}
\begin{itemize}
\item {Grp. gram.:adv.}
\end{itemize}
De modo \textunderscore algébrico\textunderscore .
\section{Algébrico}
\begin{itemize}
\item {Grp. gram.:adj.}
\end{itemize}
Relativo á \textunderscore Álgebra\textunderscore .
\section{Algebrista}
\begin{itemize}
\item {Grp. gram.:m.}
\end{itemize}
\begin{itemize}
\item {Proveniência:(De \textunderscore álgebra\textunderscore )}
\end{itemize}
Aquelle que é conhecedor da Álgebra.
Aquelle que medíca fracturas de ossos, ou ossos deslocados.
\section{Algebrizar}
\begin{itemize}
\item {Grp. gram.:v. t.}
\end{itemize}
Encher de fórmulas algébricas.
\section{Algecira}
\begin{itemize}
\item {Grp. gram.:f.}
\end{itemize}
\begin{itemize}
\item {Utilização:Ant.}
\end{itemize}
\begin{itemize}
\item {Proveniência:(Do ár. \textunderscore algezira\textunderscore )}
\end{itemize}
O mesmo que \textunderscore ilha\textunderscore . Cf. G. Vicente, I, 185.
\section{Algedo}
\begin{itemize}
\item {fónica:gê}
\end{itemize}
\begin{itemize}
\item {Grp. gram.:m.}
\end{itemize}
\begin{itemize}
\item {Proveniência:(Do gr. \textunderscore algos\textunderscore )}
\end{itemize}
Inflammação, produzida por gonorreia.
\section{Algela}
\begin{itemize}
\item {Grp. gram.:f.}
\end{itemize}
\begin{itemize}
\item {Proveniência:(Do ár. \textunderscore al-hilla\textunderscore )}
\end{itemize}
Arraial moirisco, em que se armam tendas para pernoitar.
\section{Algema}
\begin{itemize}
\item {Grp. gram.:f. pl.}
\end{itemize}
\begin{itemize}
\item {Proveniência:(Do ár. \textunderscore al-jamia\textunderscore )}
\end{itemize}
Instrumento de ferro, com que se prende alguém pelos pulsos.
Cadeia; grilheta.
Oppressão.
\section{Algemado}
\begin{itemize}
\item {Grp. gram.:adj.}
\end{itemize}
\begin{itemize}
\item {Proveniência:(De \textunderscore algemar\textunderscore )}
\end{itemize}
Preso com algemas.
\section{Algemar}
\begin{itemize}
\item {Grp. gram.:v. t.}
\end{itemize}
Prender com algemas.
Dominar; coagir.
\section{Algemia}
\begin{itemize}
\item {Grp. gram.:f.}
\end{itemize}
\begin{itemize}
\item {Proveniência:(Do ár. \textunderscore al-jamia\textunderscore )}
\end{itemize}
Alteração, produzida nos dialectos da Espanha pelo contacto dos Árabes.
Linguagem mesclada de espanhol e árabe.
Texto espanhol ou português em caracteres arábicos.
\section{Algemiar}
\begin{itemize}
\item {Grp. gram.:v. t.}
\end{itemize}
\begin{itemize}
\item {Grp. gram.:V. i.}
\end{itemize}
\begin{itemize}
\item {Proveniência:(De \textunderscore algemia\textunderscore )}
\end{itemize}
Escrever (texto espanhol) em caracteres arábicos.
Falar ou escrever algemia.
\section{Algenibe}
\begin{itemize}
\item {Grp. gram.:m.}
\end{itemize}
\begin{itemize}
\item {Proveniência:(Do ár. \textunderscore al-genib\textunderscore )}
\end{itemize}
Estrêlla da constellação do Pégaso.
\section{Algente}
\begin{itemize}
\item {Grp. gram.:adj.}
\end{itemize}
\begin{itemize}
\item {Proveniência:(Lat. \textunderscore algens\textunderscore )}
\end{itemize}
Muito frio; glacial.
\section{Algeramolho}
\begin{itemize}
\item {fónica:mô}
\end{itemize}
\begin{itemize}
\item {Grp. gram.:m.}
\end{itemize}
\begin{itemize}
\item {Utilização:Prov.}
\end{itemize}
\begin{itemize}
\item {Utilização:alg.}
\end{itemize}
O mesmo que \textunderscore agermolho\textunderscore .
\section{Algeravia}
\begin{itemize}
\item {Grp. gram.:f.}
\end{itemize}
O mesmo que \textunderscore aljaravia\textunderscore .
\section{Algerevia}
\begin{itemize}
\item {Grp. gram.:f.}
\end{itemize}
\begin{itemize}
\item {Utilização:Ant.}
\end{itemize}
O mesmo que \textunderscore aljaravia\textunderscore .
\section{Algerife}
\begin{itemize}
\item {Grp. gram.:m.}
\end{itemize}
\begin{itemize}
\item {Proveniência:(Do ár. \textunderscore al-zeriba\textunderscore )}
\end{itemize}
Rede grande de arrastar, usada no Minho.
\section{Algerifeiro}
\begin{itemize}
\item {Grp. gram.:m.}
\end{itemize}
\begin{itemize}
\item {Utilização:Prov.}
\end{itemize}
\begin{itemize}
\item {Utilização:minh.}
\end{itemize}
Aquelle que, de Fevereiro a Junho, pesca com algerife no rio Minho.
\section{Algerive}
\begin{itemize}
\item {Grp. gram.:m.}
\end{itemize}
(V.algerife)
\section{Algerós}
\begin{itemize}
\item {Grp. gram.:m.}
\end{itemize}
\begin{itemize}
\item {Proveniência:(Do ár. \textunderscore aljorob\textunderscore )}
\end{itemize}
Cano, por onde se escôam as águas do telhado.
Parte saliente do telhado, para desviar as águas da parede que o sustenta.
\section{...algia}
\begin{itemize}
\item {Grp. gram.:suf.}
\end{itemize}
\begin{itemize}
\item {Proveniência:(Do gr. \textunderscore algos\textunderscore )}
\end{itemize}
(design. de soffrimento, dôr)
\section{Algibe}
\begin{itemize}
\item {Grp. gram.:m.}
\end{itemize}
\begin{itemize}
\item {Utilização:Des.}
\end{itemize}
\begin{itemize}
\item {Utilização:Marn.}
\end{itemize}
\begin{itemize}
\item {Proveniência:(Do ár. \textunderscore al-jubb\textunderscore ?)}
\end{itemize}
Cisterna.
Nome de alguns tanques menores, nas salinas.
\section{Algibeba}
\begin{itemize}
\item {Grp. gram.:fem.}
\end{itemize}
De algibebe.
\section{Algibebe}
\begin{itemize}
\item {Grp. gram.:m.}
\end{itemize}
\begin{itemize}
\item {Proveniência:(Do ár. \textunderscore al-djabeb\textunderscore )}
\end{itemize}
Aquelle que vende fato.
\section{Algibeira}
\begin{itemize}
\item {Grp. gram.:f.}
\end{itemize}
\begin{itemize}
\item {Proveniência:(Do ár. \textunderscore al-jebira\textunderscore ?)}
\end{itemize}
Pequeno saco ou bôlso, que faz parte integrante do fato.
Pequena bôlsa, separada do fato, e que as mulheres do povo prendem á cintura, por baixo dos vestidos.
\section{Algibeta}
\begin{itemize}
\item {fónica:bê}
\end{itemize}
\begin{itemize}
\item {Grp. gram.:f.}
\end{itemize}
\begin{itemize}
\item {Utilização:Ant.}
\end{itemize}
O mesmo que \textunderscore aljubeta\textunderscore .
\section{Algibetaria}
\begin{itemize}
\item {Grp. gram.:f.}
\end{itemize}
\begin{itemize}
\item {Utilização:Ant.}
\end{itemize}
\begin{itemize}
\item {Proveniência:(De \textunderscore algibeta\textunderscore )}
\end{itemize}
Arruamento de algibebes.
\section{Álgico}
\begin{itemize}
\item {Grp. gram.:m.}
\end{itemize}
Uma das línguas faladas pelos Índios da América do Norte.
\section{Álgido}
\begin{itemize}
\item {Grp. gram.:adj.}
\end{itemize}
\begin{itemize}
\item {Proveniência:(Lat. \textunderscore algidus\textunderscore )}
\end{itemize}
Muito frio.
Que faz experimentar viva sensação de frio.
\section{Algirão}
\begin{itemize}
\item {Grp. gram.:m.}
\end{itemize}
Abertura, por onde os peixes entram na rede.
\section{Algo}
\begin{itemize}
\item {Grp. gram.:pron.}
\end{itemize}
\begin{itemize}
\item {Utilização:ind.}
\end{itemize}
\begin{itemize}
\item {Grp. gram.:Adv.}
\end{itemize}
\begin{itemize}
\item {Grp. gram.:M.}
\end{itemize}
\begin{itemize}
\item {Proveniência:(Do lat. \textunderscore aliquot\textunderscore )}
\end{itemize}
Alguma coisa.
Um tanto, um pouco.
Alguma coisa.
Fazenda.
Aquelle que possue ou que é rico.
\section{Algodão}
\begin{itemize}
\item {Grp. gram.:m.}
\end{itemize}
\begin{itemize}
\item {Proveniência:(Do ár. \textunderscore al-eoton\textunderscore )}
\end{itemize}
Pennugem ou tênues filamentos vegetaes, que cercam a semente do algodoeiro.
Fio de algodão.
Tecido de algodão.
Lanugem, que cobre as fôlhas de alguns vegetaes.
\section{Algodão-pólvora}
\begin{itemize}
\item {Grp. gram.:m.}
\end{itemize}
Substância explosiva, obtida pela acção do ácido azótico sôbre o algodão.--Também se lhe dão os nomes de \textunderscore collódio\textunderscore , \textunderscore algodão-collódio\textunderscore , \textunderscore nitro-cellulose\textunderscore , \textunderscore pyróxila\textunderscore , \textunderscore pyróxilo\textunderscore , \textunderscore pyroxilina\textunderscore , \textunderscore xyloidina\textunderscore , \textunderscore nitro-amido\textunderscore , \textunderscore trinitrocellulose\textunderscore , \textunderscore pyroxylol\textunderscore ,
\textunderscore colloxylina\textunderscore , etc.
\section{Algodoal}
\begin{itemize}
\item {Grp. gram.:m.}
\end{itemize}
Lugar, onde se criam algodoeiros.
\section{Algodoar}
\begin{itemize}
\item {Grp. gram.:v. t.}
\end{itemize}
Encher de algodão. Cf. \textunderscore Primo Basilio\textunderscore , 212.
\section{Algodoaria}
\begin{itemize}
\item {Grp. gram.:f.}
\end{itemize}
Fábrica de fiação ou de tecidos de algodão.
\section{Algodoeiro}
\begin{itemize}
\item {Grp. gram.:m.}
\end{itemize}
\begin{itemize}
\item {Grp. gram.:Adj.}
\end{itemize}
Planta malvácea (\textunderscore gossypium\textunderscore , ou \textunderscore xilum\textunderscore ), de que há doze espécies, e que produz o algodão.
Fabricante de tecidos de algodão.
Que diz respeito ao algodão: \textunderscore industria algodoeira\textunderscore .
\section{Algodoeiro-do-mato}
\begin{itemize}
\item {Grp. gram.:m.}
\end{itemize}
O mesmo que \textunderscore panheira\textunderscore .
\section{Algodoim}
\begin{itemize}
\item {Grp. gram.:m.}
\end{itemize}
Espécie de algodoeiro do Brasil.
\section{Algófilo}
\begin{itemize}
\item {Grp. gram.:m.  e  adj.}
\end{itemize}
\begin{itemize}
\item {Utilização:Neol.}
\end{itemize}
\begin{itemize}
\item {Proveniência:(Do gr. \textunderscore algos\textunderscore  + \textunderscore philos\textunderscore )}
\end{itemize}
Aquelle que tem prazer na dôr phýsica, como os mártyres e os penitentes.
\section{Algoide}
\begin{itemize}
\item {Grp. gram.:adj.}
\end{itemize}
\begin{itemize}
\item {Proveniência:(Do lat. \textunderscore alga\textunderscore  + gr. \textunderscore eidos\textunderscore )}
\end{itemize}
Semelhante á alga.
\section{Algol}
\begin{itemize}
\item {Grp. gram.:m.}
\end{itemize}
\begin{itemize}
\item {Proveniência:(Do ár. \textunderscore al-gul\textunderscore )}
\end{itemize}
Uma das estrêllas da constellação de Perseu.
\section{Algologia}
\begin{itemize}
\item {Grp. gram.:f.}
\end{itemize}
Tratado ou estudo das algas.
(Cp. \textunderscore algólogo\textunderscore )
\section{Algológico}
\begin{itemize}
\item {Grp. gram.:adj.}
\end{itemize}
Relativo á \textunderscore algologia\textunderscore .
\section{Algologista}
\begin{itemize}
\item {Grp. gram.:m.}
\end{itemize}
O mesmo que \textunderscore algólogo\textunderscore .
\section{Algólogo}
\begin{itemize}
\item {Grp. gram.:m.}
\end{itemize}
\begin{itemize}
\item {Proveniência:(Do lat. \textunderscore alga\textunderscore  + gr. \textunderscore logos\textunderscore )}
\end{itemize}
Botânico, que se dedica ao estudo das algas.
\section{Algonquino}
\begin{itemize}
\item {Grp. gram.:m.}
\end{itemize}
Grupo de línguas, faladas pelos algonquinos.
\section{Algonquinos}
\begin{itemize}
\item {Grp. gram.:m. pl.}
\end{itemize}
Tríbo selvagem do Canadá.
\section{Algóphilo}
\begin{itemize}
\item {Grp. gram.:m.  e  adj.}
\end{itemize}
\begin{itemize}
\item {Utilização:Neol.}
\end{itemize}
\begin{itemize}
\item {Proveniência:(Do gr. \textunderscore algos\textunderscore  + \textunderscore philos\textunderscore )}
\end{itemize}
Aquelle que tem prazer na dôr phýsica, como os mártyres e os penitentes.
\section{Algor}
\begin{itemize}
\item {Grp. gram.:m.}
\end{itemize}
\begin{itemize}
\item {Proveniência:(Lat. \textunderscore algor\textunderscore )}
\end{itemize}
Viva sensação de frio; frio vehemente.
\section{Algorabão}
\begin{itemize}
\item {Grp. gram.:m.}
\end{itemize}
Espécie de grou.
O mesmo que \textunderscore alcaravão\textunderscore ?
\section{Algorithmia}
\begin{itemize}
\item {Grp. gram.:f.}
\end{itemize}
\begin{itemize}
\item {Proveniência:(De \textunderscore algorithmo\textunderscore )}
\end{itemize}
Parte das Mathemáticas puras, que tem por objecto os números.
\section{Algoríthmico}
\begin{itemize}
\item {Grp. gram.:adj.}
\end{itemize}
Relativo a \textunderscore algorithmia\textunderscore .
\section{Algorithmo}
\begin{itemize}
\item {Grp. gram.:m.}
\end{itemize}
Processo de cálculo.
(Há divergências sôbre a procedência do termo)
\section{Algoritmia}
\begin{itemize}
\item {Grp. gram.:f.}
\end{itemize}
\begin{itemize}
\item {Proveniência:(De \textunderscore algoritmo\textunderscore )}
\end{itemize}
Parte das Mathemáticas puras, que tem por objecto os números.
\section{Algorítmico}
\begin{itemize}
\item {Grp. gram.:adj.}
\end{itemize}
Relativo a \textunderscore algoritmia\textunderscore .
\section{Algoritmo}
\begin{itemize}
\item {Grp. gram.:m.}
\end{itemize}
Processo de cálculo.
(Há divergências sôbre a procedência do termo)
\section{Algorova}
\begin{itemize}
\item {Grp. gram.:f.}
\end{itemize}
Árvore leguminosa do Peru.
\section{Algorovão}
\begin{itemize}
\item {Grp. gram.:m.}
\end{itemize}
\begin{itemize}
\item {Utilização:Prov.}
\end{itemize}
\begin{itemize}
\item {Utilização:alent.}
\end{itemize}
O mesmo que \textunderscore alcaravão\textunderscore .
Espécie de jôgo de rapazes.
Rapaz que, nesse jôgo, é perseguido pelos outros, sem romper a cadeia dos que o cercam.
\section{Algorrém}
\begin{itemize}
\item {Grp. gram.:pron.}
\end{itemize}
\begin{itemize}
\item {Utilização:Ant.}
\end{itemize}
\begin{itemize}
\item {Proveniência:(De \textunderscore algo\textunderscore  + \textunderscore rem\textunderscore )}
\end{itemize}
Alguma coisa.
\section{Algoso}
\begin{itemize}
\item {Grp. gram.:adj.}
\end{itemize}
Que tem algas.
\section{Algovão}
\begin{itemize}
\item {Grp. gram.:m.}
\end{itemize}
Ave ribeirinha. (Colhido na Arruda)
\section{Algoz}
\begin{itemize}
\item {fónica:gôs}
\end{itemize}
\begin{itemize}
\item {Grp. gram.:m.}
\end{itemize}
Carrasco; verdugo.
Homem cruel.
(Ár. \textunderscore al-goz\textunderscore )
\section{Algozar}
\begin{itemize}
\item {Grp. gram.:v. i.}
\end{itemize}
Praticar actos de algoz; fazer morticínio:«\textunderscore os enfrascados naquelle morticínio algozaram por maneira, que foram mortos além de quinhentos\textunderscore ». Filinto, \textunderscore D. Man.\textunderscore , I, 345.
\section{Algozaria}
\begin{itemize}
\item {Grp. gram.:f.}
\end{itemize}
Acto próprio de algoz.
Crueldade, deshumanidade.
\section{Algregue}
\begin{itemize}
\item {Grp. gram.:m.}
\end{itemize}
\begin{itemize}
\item {Utilização:Prov.}
\end{itemize}
\begin{itemize}
\item {Utilização:trasm.}
\end{itemize}
Planta espinhosa das arribas.
\section{Algrivão}
\begin{itemize}
\item {Grp. gram.:m.}
\end{itemize}
O mesmo que \textunderscore algrouvão\textunderscore . Cf. B. Pato.
\section{Algrouvão}
\begin{itemize}
\item {Grp. gram.:m.}
\end{itemize}
\begin{itemize}
\item {Utilização:Prov.}
\end{itemize}
\begin{itemize}
\item {Utilização:extrem.}
\end{itemize}
O mesmo que \textunderscore alcaravão\textunderscore .
\section{Algual}
\begin{itemize}
\item {Grp. gram.:m.}
\end{itemize}
Planta, semelhante ao lírio.
\section{Alguém}
\begin{itemize}
\item {Grp. gram.:pron.}
\end{itemize}
\begin{itemize}
\item {Utilização:ind.}
\end{itemize}
\begin{itemize}
\item {Proveniência:(Lat. \textunderscore aliquem\textunderscore )}
\end{itemize}
Alguma pessôa.
Pessôa importante.
\section{Alguergar}
\begin{itemize}
\item {Grp. gram.:v. t.}
\end{itemize}
\begin{itemize}
\item {Proveniência:(De \textunderscore alguergue\textunderscore )}
\end{itemize}
Ornar com mosaicos, feitos de pequenas pedras.
\section{Alguergue}
\begin{itemize}
\item {Grp. gram.:m.}
\end{itemize}
\begin{itemize}
\item {Proveniência:(Do ár. \textunderscore al-quirque\textunderscore )}
\end{itemize}
Pedras, com que se fazem mosaicos.
Embutidos.
Antigo jôgo de pedrinhas.
Pedrinhas dêsse jôgo.
Pedra, em que se espremem as seiras de lagar de azeite.
\section{Alguervão}
\begin{itemize}
\item {Grp. gram.:m.}
\end{itemize}
\begin{itemize}
\item {Utilização:Prov.}
\end{itemize}
\begin{itemize}
\item {Utilização:alent.}
\end{itemize}
O mesmo que \textunderscore alcaravão\textunderscore .
\section{Algueta}
\begin{itemize}
\item {fónica:guê}
\end{itemize}
\begin{itemize}
\item {Grp. gram.:f.}
\end{itemize}
\begin{itemize}
\item {Proveniência:(De \textunderscore alga\textunderscore )}
\end{itemize}
Planta, da fam. das naiádeas.
\section{Alguidar}
\begin{itemize}
\item {Grp. gram.:m.}
\end{itemize}
Vaso de barro ou metal, em fórma de cone invertido, e cuja boca tem diâmetro maior que o da altura.
(Ár. \textunderscore al-guiddar\textunderscore )
\section{Alguidarada}
\begin{itemize}
\item {Grp. gram.:f.}
\end{itemize}
O que um alguidar póde conter.
\section{Alguirradeira}
\begin{itemize}
\item {Grp. gram.:f.}
\end{itemize}
Apparelho das officinas de cardação, nas fábricas de tecidos. Cf. \textunderscore Inquér. Industr.\textunderscore , 2.^a p., 152.
\section{Algum}
\begin{itemize}
\item {Grp. gram.:adj.}
\end{itemize}
\begin{itemize}
\item {Proveniência:(Do lat. \textunderscore aliquis\textunderscore  + \textunderscore unus\textunderscore )}
\end{itemize}
Um, entre dois ou mais.
Qualquer.
Medíocre.
\section{Algumia}
\begin{itemize}
\item {Grp. gram.:adj.}
\end{itemize}
\begin{itemize}
\item {Utilização:Ant.}
\end{itemize}
Pequeno vaso de barro, espécie de púcara ou caneca. Cf. Marreca, \textunderscore Conde Sober. de Cast.\textunderscore 
\section{Algum-tanto}
\begin{itemize}
\item {Grp. gram.:loc. adv.}
\end{itemize}
Medianamente.
\section{Alguno}
\begin{itemize}
\item {Grp. gram.:adj.}
\end{itemize}
\begin{itemize}
\item {Utilização:Ant.}
\end{itemize}
O mesmo que \textunderscore algum\textunderscore .
\section{Algur}
\begin{itemize}
\item {Grp. gram.:m.  e  adv.}
\end{itemize}
\begin{itemize}
\item {Utilização:Ant.}
\end{itemize}
O mesmo que \textunderscore algures\textunderscore .
\section{Algures}
\begin{itemize}
\item {Grp. gram.:m.}
\end{itemize}
\begin{itemize}
\item {Grp. gram.:Adv.}
\end{itemize}
Algum lugar, alguma parte.
Em alguma parte, em algum lugar.
(Alter. de \textunderscore alhures\textunderscore , por infl. de \textunderscore algo\textunderscore )
\section{...alha}
\begin{itemize}
\item {Grp. gram.:suf.}
\end{itemize}
De inferioridade, alargamento, etc.
\section{Alháceas}
\begin{itemize}
\item {Grp. gram.:f. pl.}
\end{itemize}
O mesmo que \textunderscore alliáceas\textunderscore .
\section{...alhaço}
\begin{itemize}
\item {Grp. gram.:suf.}
\end{itemize}
De comparação.
\section{Alhada}
\begin{itemize}
\item {Grp. gram.:f.}
\end{itemize}
Muítos alhos.
Guisado com alho.
Intriga, embrulhada: \textunderscore meter-se numa alhada\textunderscore .
\section{Alhafa}
\begin{itemize}
\item {Grp. gram.:f.}
\end{itemize}
\begin{itemize}
\item {Utilização:Ant.}
\end{itemize}
Pavor, causado por um precipício.
\section{Alhaima}
\begin{itemize}
\item {Grp. gram.:f.}
\end{itemize}
\begin{itemize}
\item {Utilização:Ant.}
\end{itemize}
\begin{itemize}
\item {Proveniência:(Do ár. ?)}
\end{itemize}
Tenda, para abrigar do ar da noite.
\section{Alhal}
\begin{itemize}
\item {Grp. gram.:m.}
\end{itemize}
Campo, onde crescem alhos.
\section{Alhal}
\begin{itemize}
\item {Grp. gram.:m.}
\end{itemize}
\begin{itemize}
\item {Utilização:Prov.}
\end{itemize}
\begin{itemize}
\item {Utilização:minh.}
\end{itemize}
Lugar na cozinha, onde se faz provisão da lenha, quási sempre por baixo do forno.
(Será accepção ext. ou fig. de \textunderscore alhal\textunderscore ^1?)
\section{Alhambrês}
\begin{itemize}
\item {Grp. gram.:adj.}
\end{itemize}
Relativo á Alhambra. Cf. Júl. Castilho, \textunderscore Lisb. Ant.\textunderscore 
\section{Alhanar}
\begin{itemize}
\item {Grp. gram.:v. t.}
\end{itemize}
Tornar lhano, affável.
Fazer plano.
Igualar.
Assolar.
Resolver.
\section{...alhão}
(aum. do suf. \textunderscore ...alho\textunderscore )
\section{...alhar}
\begin{itemize}
\item {Grp. gram.:suf. v.}
\end{itemize}
\begin{itemize}
\item {Proveniência:(Do suf. \textunderscore ...alho\textunderscore )}
\end{itemize}

\section{Alharca}
\begin{itemize}
\item {Grp. gram.:f.}
\end{itemize}
\begin{itemize}
\item {Utilização:Ant.}
\end{itemize}
\begin{itemize}
\item {Proveniência:(Do ár. \textunderscore al-haraca\textunderscore )}
\end{itemize}
Alarma.
Algara.
\section{Alhas}
\begin{itemize}
\item {Grp. gram.:adj. f. pl.}
\end{itemize}
Diz-se das \textunderscore palhas\textunderscore , com que se designam as fôlhas sêcas dos alhos.
\section{Alheação}
\begin{itemize}
\item {Grp. gram.:f.}
\end{itemize}
Acto de \textunderscore alhear\textunderscore .
\section{Alheado}
\begin{itemize}
\item {Grp. gram.:adj.}
\end{itemize}
\begin{itemize}
\item {Proveniência:(De \textunderscore alhear\textunderscore )}
\end{itemize}
Distrahiado.
Perturbado.
\section{Alheador}
\begin{itemize}
\item {Grp. gram.:m.}
\end{itemize}
Aquelle que alheia.
\section{Alheamento}
\begin{itemize}
\item {Grp. gram.:m.}
\end{itemize}
Estado de quem está alheado.
\section{Alhear}
\begin{itemize}
\item {Grp. gram.:v. t.}
\end{itemize}
Tornar alheio.
Passar para outrem o domínio de; alienar.
Desviar.
Perturbar; alucinar.
\section{Alheável}
\begin{itemize}
\item {Grp. gram.:adj.}
\end{itemize}
Que se póde \textunderscore alhear\textunderscore .
\section{Alheio}
\begin{itemize}
\item {Grp. gram.:adj.}
\end{itemize}
\begin{itemize}
\item {Proveniência:(Lat. \textunderscore alienus\textunderscore )}
\end{itemize}
Estranho; pertencente a outrem.
Impróprio.
Distante.
Opposto.
Privado, isento.
Distrahido.
Mentecapto.
\section{Alheira}
\begin{itemize}
\item {Grp. gram.:f.}
\end{itemize}
Planta, cujo cheiro parece o do alho.
Chouriça trasmontana, temperada com alhos.
\section{Alheiro}
\begin{itemize}
\item {Grp. gram.:m.}
\end{itemize}
Aquelle que negocía em alhos.
\section{Alhela}
\begin{itemize}
\item {Grp. gram.:f.}
\end{itemize}
\begin{itemize}
\item {Proveniência:(Do ár. \textunderscore al-hella\textunderscore )}
\end{itemize}
Agrupamento de aduares, de tendas.
Acampamento.
\section{Alheta}
\begin{itemize}
\item {fónica:lhê}
\end{itemize}
\begin{itemize}
\item {Grp. gram.:f.}
\end{itemize}
\begin{itemize}
\item {Utilização:Prov.}
\end{itemize}
\begin{itemize}
\item {Proveniência:(De \textunderscore alho\textunderscore )}
\end{itemize}
Alho, de cabeça inteiriça, não divida em bolbilhos.
(Colhido em Turquel)
\section{Alheta}
\begin{itemize}
\item {fónica:lhê}
\end{itemize}
\begin{itemize}
\item {Grp. gram.:f.}
\end{itemize}
\begin{itemize}
\item {Utilização:Ant.}
\end{itemize}
\begin{itemize}
\item {Utilização:Constr.}
\end{itemize}
Pista, encalço.
Debrum na parte superior da manga do gibão.
Peças de pau, que formam o prolongamento exterior da popa da embarcação.
Porção de pé direito dos arcos, entre o vão e a columna adherente a esse pé direito, em arcadas estabelecidas nos intercolúmnios.
\section{Alheto}
\begin{itemize}
\item {fónica:lhê}
\end{itemize}
\begin{itemize}
\item {Grp. gram.:m.}
\end{itemize}
O mesmo que \textunderscore alheta\textunderscore ^1.
\section{Álhia}
\begin{itemize}
\item {Grp. gram.:f.}
\end{itemize}
\begin{itemize}
\item {Utilização:Des.}
\end{itemize}
\begin{itemize}
\item {Proveniência:(Do lat. \textunderscore allia\textunderscore , pl. de \textunderscore allius\textunderscore ?)}
\end{itemize}
Accumulação de empregos ou benefícios.
\section{Alho}
\begin{itemize}
\item {Grp. gram.:m.}
\end{itemize}
\begin{itemize}
\item {Utilização:irón.}
\end{itemize}
\begin{itemize}
\item {Utilização:Pop.}
\end{itemize}
\begin{itemize}
\item {Proveniência:(Lat. \textunderscore allium\textunderscore )}
\end{itemize}
Planta hortense, liliácea.
Homem esperto: \textunderscore és um alho\textunderscore .
\section{...alho}
\begin{itemize}
\item {Grp. gram.:suf.}
\end{itemize}
De inferioridade, má qualidade.
\section{Alhodra}
\begin{itemize}
\item {Grp. gram.:f.}
\end{itemize}
\begin{itemize}
\item {Utilização:Ant.}
\end{itemize}
Sequestro ou confisco de terras ou fazendas.
\section{Alho-porro}
\begin{itemize}
\item {fónica:pô}
\end{itemize}
\begin{itemize}
\item {Grp. gram.:m.}
\end{itemize}
Alho bravo, (\textunderscore allium porrum\textunderscore ) maior que o alho commum.
\section{Alhora!}
\textunderscore interj.\textunderscore  (de espanto): \textunderscore alhora o homem! veja o que êlle diz e o que êlle faz\textunderscore . (Ouvido em Ílhavo e vulgar nos Açores)
\section{Alhorca}
\begin{itemize}
\item {Grp. gram.:f.}
\end{itemize}
\begin{itemize}
\item {Utilização:Prov.}
\end{itemize}
O mesmo que \textunderscore trepadeira\textunderscore , ave.
\section{...alhote}
\textunderscore suf.\textunderscore  de comparação.
\section{Alhur}
\begin{itemize}
\item {Grp. gram.:adv.}
\end{itemize}
\begin{itemize}
\item {Utilização:Ant.}
\end{itemize}
O mesmo que \textunderscore alhures\textunderscore .
\section{Alhures}
\begin{itemize}
\item {Grp. gram.:adv.}
\end{itemize}
\begin{itemize}
\item {Utilização:Ant.}
\end{itemize}
\begin{itemize}
\item {Proveniência:(Do lat. \textunderscore aliorsum\textunderscore )}
\end{itemize}
Em outro lugar.
Algures.
\section{Aliás}
\begin{itemize}
\item {Grp. gram.:adv.}
\end{itemize}
\begin{itemize}
\item {Grp. gram.:M.}
\end{itemize}
\begin{itemize}
\item {Proveniência:(Lat. \textunderscore alias\textunderscore )}
\end{itemize}
De outro modo.
Outro lugar:«\textunderscore reforçados de adjutórios de aliás vindos\textunderscore ». Filinto, \textunderscore D. Man.\textunderscore , II, 186.
\section{Aliás}
\begin{itemize}
\item {Grp. gram.:f.}
\end{itemize}
Fêmea do elephante. Cf. G. Viana, \textunderscore Apostilas\textunderscore .
\section{Aliavas}
\begin{itemize}
\item {Grp. gram.:f. pl.}
\end{itemize}
Tributo, que se pagava para sustento das aves e cães, com que as pessôas reaes caçavam.
(Por \textunderscore aliaves\textunderscore , do lat. \textunderscore alere\textunderscore  + \textunderscore avis\textunderscore ?)
\section{Alibambar}
\begin{itemize}
\item {Grp. gram.:v. t.}
\end{itemize}
\begin{itemize}
\item {Utilização:Bras}
\end{itemize}
\begin{itemize}
\item {Utilização:Des.}
\end{itemize}
Prender ao libambo.
Acorrentar.
\section{Alíbil}
\begin{itemize}
\item {Grp. gram.:adj.}
\end{itemize}
\begin{itemize}
\item {Proveniência:(Lat. \textunderscore alibilis\textunderscore )}
\end{itemize}
Próprio para nutrição.
\section{Alibilidade}
\begin{itemize}
\item {Grp. gram.:f.}
\end{itemize}
Qualidade do que é alíbil.
\section{Álica}
\begin{itemize}
\item {Grp. gram.:f.}
\end{itemize}
\begin{itemize}
\item {Proveniência:(Lat. \textunderscore alica\textunderscore )}
\end{itemize}
Espécie de trigo ou cevada, de que os antigos extrahiam uma bebida fermentada, semelhante
á cerveja.
Pão de álica.
Cerveja de álica.
\section{Alicaído}
\begin{itemize}
\item {fónica:á-li}
\end{itemize}
\begin{itemize}
\item {Grp. gram.:adj.}
\end{itemize}
\begin{itemize}
\item {Proveniência:(De \textunderscore ala\textunderscore  + \textunderscore cair\textunderscore )}
\end{itemize}
Que tem asas caídas, pendentes.
Desalentado.
\section{Alicanço}
\begin{itemize}
\item {Grp. gram.:m.}
\end{itemize}
\begin{itemize}
\item {Utilização:Pop.}
\end{itemize}
\begin{itemize}
\item {Utilização:Fig.}
\end{itemize}
O mesmo que \textunderscore licranço\textunderscore .
Homem de má índole. Cf. Rebello, \textunderscore Mocidade\textunderscore , III, 37 e 59.
\section{Alicântara}
\begin{itemize}
\item {Grp. gram.:f.}
\end{itemize}
Espécie de lagartíxa.
\section{Alicante}
\begin{itemize}
\item {Grp. gram.:f.}
\end{itemize}
\begin{itemize}
\item {Grp. gram.:f.}
\end{itemize}
\begin{itemize}
\item {Proveniência:(De \textunderscore Alicante\textunderscore , n. p.)}
\end{itemize}
Casta de uva preta algarvia e andaluza.
Vinho dessa casta.
\section{Alicante-vermelho}
\begin{itemize}
\item {Grp. gram.:m.}
\end{itemize}
Casta de uva avermelhada, variedade de alicante.
\section{Alicantina}
\begin{itemize}
\item {Grp. gram.:f.}
\end{itemize}
Velhacaria; manha.
Fraude.
(Fem. de \textunderscore alicantino\textunderscore )
\section{Alicantinador}
\begin{itemize}
\item {Grp. gram.:m.}
\end{itemize}
O mesmo que \textunderscore alicantineiro\textunderscore .
\section{Alicantineiro}
\begin{itemize}
\item {Grp. gram.:m.}
\end{itemize}
Aquelle que faz alicantinas.
\section{Alicantino}
\begin{itemize}
\item {Grp. gram.:adj.}
\end{itemize}
\begin{itemize}
\item {Grp. gram.:M.}
\end{itemize}
Relativo a Alicante.
Habitante de Alicante.
\section{Alicário}
\begin{itemize}
\item {Grp. gram.:m.}
\end{itemize}
\begin{itemize}
\item {Proveniência:(Lat. \textunderscore alicarius\textunderscore )}
\end{itemize}
Aquelle que fabricava álica.
\section{Alicate}
\begin{itemize}
\item {Grp. gram.:m.}
\end{itemize}
\begin{itemize}
\item {Proveniência:(Do ár. \textunderscore al-laccate\textunderscore )}
\end{itemize}
Espécie de torquez ou tenaz, composta de duas asas, que se cruzam e se movem em tôrno de um eixo commum.
\section{Alicece}
\begin{itemize}
\item {Grp. gram.:m.}
\end{itemize}
(V.alicerce)
\section{Alicerçar}
\begin{itemize}
\item {Grp. gram.:v. t.}
\end{itemize}
Fazer o alicerce de; cimentar.
Basear.
Consolidar.
\section{Alicerce}
\begin{itemize}
\item {Grp. gram.:m.}
\end{itemize}
\begin{itemize}
\item {Proveniência:(Do ár. \textunderscore al-isás\textunderscore )}
\end{itemize}
Parte inferior das paredes, a que fica enterrada. Escavação, para assentar a parede.
Base; fundamento.
\section{Alicercear}
\begin{itemize}
\item {Grp. gram.:v. t.}
\end{itemize}
\begin{itemize}
\item {Utilização:Prov.}
\end{itemize}
O mesmo que \textunderscore alicerçar\textunderscore .
\section{Alicesse}
\begin{itemize}
\item {Grp. gram.:m.}
\end{itemize}
\begin{itemize}
\item {Utilização:Ant.}
\end{itemize}
O mesmo que \textunderscore alicerce\textunderscore . Cf. \textunderscore Eufrosina\textunderscore , 141.
\section{Alicondo}
\begin{itemize}
\item {Grp. gram.:m.}
\end{itemize}
Árvore africana, de casca filamentosa, empregada em tecidos.
\section{Alicorne}
\begin{itemize}
\item {Grp. gram.:m.}
\end{itemize}
Espécie de carvão mineral?
\section{Alicranço}
\begin{itemize}
\item {Grp. gram.:m.}
\end{itemize}
O mesmo que \textunderscore licranço\textunderscore .
\section{Alícula}
\begin{itemize}
\item {Grp. gram.:f.}
\end{itemize}
\begin{itemize}
\item {Proveniência:(Lat. \textunderscore alicula\textunderscore )}
\end{itemize}
Chlâmide curta, usada pelos Romanos das classes inferiores.
\section{Aliculária}
\begin{itemize}
\item {Grp. gram.:f.}
\end{itemize}
\begin{itemize}
\item {Proveniência:(De \textunderscore alicula\textunderscore )}
\end{itemize}
Gênero de plantas hepáticas.
\section{Alidada}
\begin{itemize}
\item {Grp. gram.:f.}
\end{itemize}
O mesmo que \textunderscore alidade\textunderscore .
\section{Alidade}
\begin{itemize}
\item {Grp. gram.:f.}
\end{itemize}
Régua móvel, com uma pínnula em cada extremidade, para visar objectos, cuja direcção se quere fixar, em graphometria.
(Ár. \textunderscore al-idada\textunderscore )
\section{Alidor}
\begin{itemize}
\item {Grp. gram.:m.}
\end{itemize}
Variedade de cravo roxo.
\section{Alienábil}
\begin{itemize}
\item {Grp. gram.:adj.}
\end{itemize}
Fórma alat. de \textunderscore alienável\textunderscore .
\section{Alienabilidade}
\begin{itemize}
\item {Grp. gram.:f.}
\end{itemize}
Qualidade do que é \textunderscore alienável\textunderscore .
\section{Alienação}
\begin{itemize}
\item {Grp. gram.:f.}
\end{itemize}
Acto de \textunderscore alienar\textunderscore .
\section{Alienado}
\begin{itemize}
\item {Grp. gram.:adj.}
\end{itemize}
\begin{itemize}
\item {Grp. gram.:M.}
\end{itemize}
\begin{itemize}
\item {Proveniência:(De \textunderscore alienar\textunderscore )}
\end{itemize}
Que endoideceu.
O mesmo que \textunderscore doido\textunderscore : \textunderscore hospital de alienados\textunderscore .
\section{Alienador}
\begin{itemize}
\item {Grp. gram.:m.}
\end{itemize}
Aquelle que aliena.
\section{Alienamento}
O mesmo que \textunderscore alienação\textunderscore .
\section{Alienar}
\begin{itemize}
\item {Grp. gram.:v. t.}
\end{itemize}
\begin{itemize}
\item {Proveniência:(Lat. \textunderscore alienare\textunderscore )}
\end{itemize}
Tornar alheio; alhear.
Afastar.
Malquistar.
Alucinar.
\section{Alienatário}
\begin{itemize}
\item {Grp. gram.:m.}
\end{itemize}
\begin{itemize}
\item {Proveniência:(De \textunderscore alienar\textunderscore )}
\end{itemize}
Aquelle, a quem se trasm.ttiu posse ou propriedade de alguma coisa.
\section{Alienável}
\begin{itemize}
\item {Grp. gram.:adj.}
\end{itemize}
Que póde sêr alienado ou cedido.
\section{Alienígena}
\begin{itemize}
\item {Grp. gram.:m.  e  f.}
\end{itemize}
\begin{itemize}
\item {Grp. gram.:Adj.}
\end{itemize}
\begin{itemize}
\item {Proveniência:(Lat. \textunderscore alienigena\textunderscore )}
\end{itemize}
Indivíduo de outro país.
Estranho, forasteiro.
\section{Alienista}
\begin{itemize}
\item {Grp. gram.:m.}
\end{itemize}
\begin{itemize}
\item {Grp. gram.:Adj.}
\end{itemize}
\begin{itemize}
\item {Proveniência:(De \textunderscore alienar\textunderscore )}
\end{itemize}
Médico, que trata de doenças mentaes.
Relativo ao tratamento de alienados.
\section{Alifafe}
\begin{itemize}
\item {Grp. gram.:m.}
\end{itemize}
\begin{itemize}
\item {Proveniência:(Do ár. \textunderscore al-nafaque\textunderscore )}
\end{itemize}
Tumor nos cavallos, entre o nervo do jarrete e o osso da perna.
\section{Alifafe}
\begin{itemize}
\item {Grp. gram.:m.}
\end{itemize}
\begin{itemize}
\item {Utilização:Ant.}
\end{itemize}
\begin{itemize}
\item {Proveniência:(Do ár. \textunderscore al-lihaf\textunderscore )}
\end{itemize}
Colcha de cama.
Cobertor.
\section{Alifante}
\begin{itemize}
\item {Grp. gram.:m.}
\end{itemize}
\begin{itemize}
\item {Utilização:ant.}
\end{itemize}
\begin{itemize}
\item {Utilização:Pop.}
\end{itemize}
O mesmo que \textunderscore elephante\textunderscore . Cf. G. Vicente, III, 207, etc. \textunderscore Lusíadas\textunderscore , X, 110.
\section{Alífero}
\begin{itemize}
\item {Grp. gram.:adj.}
\end{itemize}
\begin{itemize}
\item {Proveniência:(Do lat. \textunderscore ala\textunderscore  + \textunderscore ferre\textunderscore )}
\end{itemize}
Que tem asas.
\section{Aliforme}
\begin{itemize}
\item {Grp. gram.:adj.}
\end{itemize}
\begin{itemize}
\item {Proveniência:(Do lat. \textunderscore ala\textunderscore  + \textunderscore forma\textunderscore )}
\end{itemize}
Que tem fórma de asa.
\section{Aligeirar}
\begin{itemize}
\item {Grp. gram.:v. t.}
\end{itemize}
Tornar ligeiro; apressar.
\section{Alígero}
\begin{itemize}
\item {Grp. gram.:adj.}
\end{itemize}
\begin{itemize}
\item {Proveniência:(Do lat. \textunderscore ala\textunderscore  + \textunderscore gerere\textunderscore )}
\end{itemize}
Que tem asas.
Veloz.
\section{Aligulado}
\begin{itemize}
\item {Grp. gram.:adj.}
\end{itemize}
O mesmo que \textunderscore ligulado\textunderscore .
\section{Alijação}
\begin{itemize}
\item {Grp. gram.:f.}
\end{itemize}
(V.alijamento)
\section{Alijamento}
\begin{itemize}
\item {Grp. gram.:m.}
\end{itemize}
Acto de \textunderscore alijar\textunderscore .
\section{Alijar}
\begin{itemize}
\item {Grp. gram.:v. t.}
\end{itemize}
\begin{itemize}
\item {Proveniência:(Do lat. hyp. \textunderscore alleviare\textunderscore )}
\end{itemize}
Lançar fóra da embarcação.
Alliviar.
Desembaraçar-se de.
Arremessar.
\section{Alijo}
\begin{itemize}
\item {Grp. gram.:m.}
\end{itemize}
\begin{itemize}
\item {Utilização:P. us.}
\end{itemize}
\begin{itemize}
\item {Proveniência:(De \textunderscore alijar\textunderscore )}
\end{itemize}
Barcaça ou canôa, que acompanha um navio, para receber a carga que elle alija.
\section{Alim}
\begin{itemize}
\item {Grp. gram.:m.}
\end{itemize}
O mesmo que \textunderscore alime\textunderscore .
\section{Alimal}
\begin{itemize}
\item {Grp. gram.:m.}
\end{itemize}
\begin{itemize}
\item {Utilização:Pop.}
\end{itemize}
O mesmo que \textunderscore animal\textunderscore ^1.
\section{Alimária}
\begin{itemize}
\item {Grp. gram.:f.}
\end{itemize}
Animal irracional.
Bruto.
(Metáth. do lat. \textunderscore animalia\textunderscore )
\section{Alime}
\begin{itemize}
\item {Grp. gram.:m.}
\end{itemize}
Theólogo, entre os Árabes. Cf. Herculano, \textunderscore Hist. de Port.\textunderscore , liv. VII, p. I.
(Cp. \textunderscore ulemás\textunderscore )
\section{Alimentação}
\begin{itemize}
\item {Grp. gram.:f.}
\end{itemize}
Acto de \textunderscore alimentar\textunderscore .
\section{Alimental}
\begin{itemize}
\item {Grp. gram.:adj.}
\end{itemize}
(V.alimentício)
\section{Alimentar}
\begin{itemize}
\item {Grp. gram.:adj.}
\end{itemize}
Relativo ao \textunderscore alimento\textunderscore .
\section{Alimentar}
\begin{itemize}
\item {Grp. gram.:v. t.}
\end{itemize}
Dar alimento a; sustentar: \textunderscore alimentar os filhos\textunderscore .
Conservar.
Manter: \textunderscore alimentar esperanças\textunderscore .
Prover.
\section{Alimentário}
\begin{itemize}
\item {Grp. gram.:adj.}
\end{itemize}
\begin{itemize}
\item {Grp. gram.:M.}
\end{itemize}
\begin{itemize}
\item {Utilização:Jur.}
\end{itemize}
(V.alimentício)
Aquelle a quem se devem prestar alimentos.
\section{Alimenteiro}
\begin{itemize}
\item {Grp. gram.:m.}
\end{itemize}
\begin{itemize}
\item {Utilização:Ant.}
\end{itemize}
\begin{itemize}
\item {Proveniência:(De \textunderscore alimento\textunderscore )}
\end{itemize}
Empregado da casa real, espécie do dispenseiro ou mordomo.
\section{Alimentício}
\begin{itemize}
\item {Grp. gram.:adj.}
\end{itemize}
\begin{itemize}
\item {Proveniência:(De \textunderscore alimento\textunderscore )}
\end{itemize}
Que alimenta, que sustenta; próprio para alimentar: \textunderscore gêneros alimentícios\textunderscore .
\section{Alimentividade}
\begin{itemize}
\item {Grp. gram.:f.}
\end{itemize}
\begin{itemize}
\item {Proveniência:(De um hyp. \textunderscore alimentivo\textunderscore , de \textunderscore alimentar\textunderscore )}
\end{itemize}
Órgão do appetite dos alimentos, segundo alguns phrenologistas.
\section{Alimento}
\begin{itemize}
\item {Grp. gram.:m.}
\end{itemize}
\begin{itemize}
\item {Grp. gram.:Pl.}
\end{itemize}
\begin{itemize}
\item {Proveniência:(Lat. \textunderscore alimentum\textunderscore )}
\end{itemize}
Tudo que, digerido, alimenta ou serve para nutrição.
Aquillo que mantém.
Aquillo que fomenta.
Todas as despesas com o tratamento de uma pessôa.
\section{Alimentoso}
\begin{itemize}
\item {Grp. gram.:m.}
\end{itemize}
(V.alimentício)
\section{Alimónia}
\begin{itemize}
\item {Grp. gram.:f.}
\end{itemize}
\begin{itemize}
\item {Utilização:Ant.}
\end{itemize}
Alimentação, sustento. Cf. \textunderscore Âncora Médica\textunderscore , 14.
\section{Alimpa}
\begin{itemize}
\item {Grp. gram.:f.}
\end{itemize}
\begin{itemize}
\item {Utilização:Prov.}
\end{itemize}
\begin{itemize}
\item {Utilização:minh.}
\end{itemize}
Acção de \textunderscore alimpar\textunderscore  árvores ou campos, cortando ramos supérfluos ou plantas nocivas.
Acto de limpar o milho das suas impurezas, deixando-o cair de alto, em pequenas porções, em dias de vento ou de aragem forte.
\section{Alagite}
\begin{itemize}
\item {Grp. gram.:m.}
\end{itemize}
Variedade de manganês silicífero.
\section{Alamanda}
\begin{itemize}
\item {Grp. gram.:f.}
\end{itemize}
\begin{itemize}
\item {Proveniência:(De \textunderscore Allamand\textunderscore , n. p.)}
\end{itemize}
Gênero de plantas apocýneas.
\section{Alânia}
\begin{itemize}
\item {Grp. gram.:f.}
\end{itemize}
Gênero de plantas leguminosas.
\section{Alanita}
\begin{itemize}
\item {Grp. gram.:f.}
\end{itemize}
Substância negra e vitrea, que corta o crystal.
\section{Alanite}
\begin{itemize}
\item {Grp. gram.:f.}
\end{itemize}
O mesmo que \textunderscore alanita\textunderscore .
\section{Alantíase}
\begin{itemize}
\item {Grp. gram.:f.}
\end{itemize}
\begin{itemize}
\item {Proveniência:(Do gr. \textunderscore allas\textunderscore , salsicha)}
\end{itemize}
Entoxicação alimentar, devida á ingestão de salsichas alteradas.
\section{Alanto}
\begin{itemize}
\item {Grp. gram.:m.}
\end{itemize}
Gênero de insectos hymenópteros.
\section{Alantoico}
\begin{itemize}
\item {Grp. gram.:adj.}
\end{itemize}
(V.alantoidiano)
\section{Alantoide}
\begin{itemize}
\item {Grp. gram.:f.}
\end{itemize}
\begin{itemize}
\item {Proveniência:(Gr. \textunderscore allantoeides\textunderscore )}
\end{itemize}
Membrana do féto dos mammíferos, durante os dois primeiros meses de vida intra-uterína, e da qual ao depois se fórma a bexiga e a placenta.
\section{Alantoidiano}
\begin{itemize}
\item {Grp. gram.:adj.}
\end{itemize}
Relativo á \textunderscore alantoide\textunderscore .
\section{Alantoína}
\begin{itemize}
\item {Grp. gram.:f.}
\end{itemize}
Substância neutra, que se encontra no líquido alantoidiano.
\section{Alantos}
\begin{itemize}
\item {Grp. gram.:m. pl.}
\end{itemize}
\begin{itemize}
\item {Proveniência:(Do gr. \textunderscore allas\textunderscore , \textunderscore allantos\textunderscore )}
\end{itemize}
Gênero de insectos hymenópteros.
\section{Alariz}
\begin{itemize}
\item {Grp. gram.:m.}
\end{itemize}
\begin{itemize}
\item {Proveniência:(De \textunderscore Allariz\textunderscore , n. p.)}
\end{itemize}
Nome de um tecido, que se fabrica na Galliza.
\section{Alécula}
\begin{itemize}
\item {Grp. gram.:f.}
\end{itemize}
Insecto coleóptero heterómero.
\section{Alegação}
\begin{itemize}
\item {Grp. gram.:f.}
\end{itemize}
Acto de \textunderscore alegar\textunderscore .
\section{Alegança}
\begin{itemize}
\item {Grp. gram.:f.}
\end{itemize}
\begin{itemize}
\item {Utilização:Ant.}
\end{itemize}
O mesmo que \textunderscore alegação\textunderscore .
\section{Alegante}
\begin{itemize}
\item {Grp. gram.:m.  e  adj.}
\end{itemize}
\begin{itemize}
\item {Proveniência:(Lat. \textunderscore allegans\textunderscore )}
\end{itemize}
O que alega.
\section{Alegar}
\begin{itemize}
\item {Grp. gram.:v. t.}
\end{itemize}
\begin{itemize}
\item {Proveniência:(Lat. \textunderscore allegare\textunderscore )}
\end{itemize}
Citar; apresentar, como prova.
\section{Alegoria}
\begin{itemize}
\item {Grp. gram.:f.}
\end{itemize}
\begin{itemize}
\item {Proveniência:(Gr. \textunderscore allegoria\textunderscore )}
\end{itemize}
Exposição de uma ideia, sob fórma figurada.
Obra artística ou literária, que representa uma coisa, para dar ideia de outra.
\section{Alegoricamente}
\begin{itemize}
\item {Grp. gram.:adv.}
\end{itemize}
De modo \textunderscore alegórico\textunderscore .
\section{Alegórico}
\begin{itemize}
\item {Grp. gram.:adj.}
\end{itemize}
Que envolve alegoria.
Relativo a \textunderscore alegoria\textunderscore .
\section{Alegorismo}
\begin{itemize}
\item {Grp. gram.:m.}
\end{itemize}
Systema de explicar por alegoria.
\section{Alegorista}
\begin{itemize}
\item {Grp. gram.:m.}
\end{itemize}
Aquelle que faz alegorias.
Aquelle que explica por alegorias escritos de outrem.
\section{Alegorizar}
\begin{itemize}
\item {Grp. gram.:v. t.}
\end{itemize}
Expor, por meio de alegoria.
Explicar em sentido alegórico.
\section{Aleluia}
\begin{itemize}
\item {fónica:á-lé-lúi-a}
\end{itemize}
\begin{itemize}
\item {Grp. gram.:f.}
\end{itemize}
\begin{itemize}
\item {Proveniência:(Hebr. \textunderscore halelu\textunderscore  + \textunderscore iah\textunderscore )}
\end{itemize}
Canto de alegria.
Alegria.
O tempo da Páscoa.
Nome de uma planta, que floresce pela Páscoa.
\section{Aleluiar}
\begin{itemize}
\item {Grp. gram.:v. i.}
\end{itemize}
Cantar a aleluia.
\section{Aleluiático}
\begin{itemize}
\item {Grp. gram.:adj.}
\end{itemize}
Relativo á \textunderscore aleluia\textunderscore .
\section{Aleluítico}
\begin{itemize}
\item {fónica:á-lé-lu-í}
\end{itemize}
\begin{itemize}
\item {Grp. gram.:adj.}
\end{itemize}
\begin{itemize}
\item {Proveniência:(De \textunderscore alleluia\textunderscore )}
\end{itemize}
Que louva, saúda ou celebra.
\section{Alestesia}
\begin{itemize}
\item {Grp. gram.:f.}
\end{itemize}
\begin{itemize}
\item {Utilização:Med.}
\end{itemize}
\begin{itemize}
\item {Proveniência:(Do gr. \textunderscore allos\textunderscore , outro, e \textunderscore aisthesis\textunderscore , sensação)}
\end{itemize}
Estado pathológico, caracterizado pela localização das sensações no lado do corpo, opposto ao que recebeu a impressão, e em ponto symétrico.
\section{Ali}
\begin{itemize}
\item {Grp. gram.:adv.}
\end{itemize}
\begin{itemize}
\item {Proveniência:(Lat. \textunderscore illic\textunderscore )}
\end{itemize}
Lá.
Além; naquelle lugar.
\section{Aliáceas}
\begin{itemize}
\item {Grp. gram.:f. pl.}
\end{itemize}
Plantas, que constituem um grupo das liliáceas, e a que serve de typo o alho.
\section{Aliáceo}
\begin{itemize}
\item {Grp. gram.:adj.}
\end{itemize}
\begin{itemize}
\item {Proveniência:(Do lat. \textunderscore allium\textunderscore )}
\end{itemize}
Relativo ao alho.
Semelhante ao alho.
\section{Aliado}
\begin{itemize}
\item {Grp. gram.:adj.}
\end{itemize}
\begin{itemize}
\item {Grp. gram.:M.}
\end{itemize}
Que se aliou.
Aquelle que entrou numa aliança:«\textunderscore os alliados em Sebastopol\textunderscore ».
\section{Aliagem}
\begin{itemize}
\item {Grp. gram.:f.}
\end{itemize}
(V.aliança)
\section{Aliança}
\begin{itemize}
\item {Grp. gram.:f.}
\end{itemize}
Acto ou effeito de \textunderscore aliar\textunderscore .
Anel de casamento.
\section{Aliançar}
\begin{itemize}
\item {Grp. gram.:v. t.}
\end{itemize}
(V.aliar)
\section{Aliar}
\begin{itemize}
\item {Grp. gram.:v. t.}
\end{itemize}
\begin{itemize}
\item {Proveniência:(Do lat. \textunderscore alligare\textunderscore )}
\end{itemize}
Unir.
Combinar.
Fazer ligação de.
Harmonizar.
Unir por casamento.
Confederar.
Encorporar.
Coadunar.
Agrupar.
\section{Aliária}
\begin{itemize}
\item {Grp. gram.:f.}
\end{itemize}
\begin{itemize}
\item {Proveniência:(Do lat. \textunderscore allium\textunderscore )}
\end{itemize}
Planta crucífera, de cheiro semelhante ao do alho.
\section{Aliável}
\begin{itemize}
\item {Grp. gram.:adj.}
\end{itemize}
Que se póde \textunderscore aliar\textunderscore .
\section{Aliciação}
\begin{itemize}
\item {Grp. gram.:f.}
\end{itemize}
Acto de \textunderscore aliciar\textunderscore .
\section{Aliciador}
\begin{itemize}
\item {Grp. gram.:m.}
\end{itemize}
\begin{itemize}
\item {Grp. gram.:Adj.}
\end{itemize}
Aquelle que alicia.
Aquelle que serve para \textunderscore aliciar\textunderscore .
Que alicia.
\section{Aliciamento}
\begin{itemize}
\item {Grp. gram.:m.}
\end{itemize}
(V.aliciação)
\section{Aliciante}
\begin{itemize}
\item {Grp. gram.:adj.}
\end{itemize}
Que alicia; aliciador.
\section{Aliciar}
\begin{itemize}
\item {Grp. gram.:v. t.}
\end{itemize}
\begin{itemize}
\item {Proveniência:(Do lat. \textunderscore allicere\textunderscore )}
\end{itemize}
Atrahir a si.
Seduzir.
Provocar.
Angariar.
Subornar.
\section{Aliciente}
\begin{itemize}
\item {Grp. gram.:adj.}
\end{itemize}
\begin{itemize}
\item {Grp. gram.:M.}
\end{itemize}
\begin{itemize}
\item {Proveniência:(Lat. \textunderscore alliciens\textunderscore )}
\end{itemize}
Que alicia.
Coisa que alicia.
Seducção.
\section{Aligatór}
\begin{itemize}
\item {Grp. gram.:m.}
\end{itemize}
Gênero de reptis sáurios, em cujas espécies entra o caimão, (\textunderscore lacerta alligator\textunderscore , Lin.).
(Fórma ingl., corr. do cast. \textunderscore el lagarto\textunderscore )
\section{Alilena}
\begin{itemize}
\item {Grp. gram.:f.}
\end{itemize}
\begin{itemize}
\item {Proveniência:(De \textunderscore allilo\textunderscore )}
\end{itemize}
Gás incolor, inflammável, de cheiro desagradável.
\section{Alimpação}
\begin{itemize}
\item {Grp. gram.:f.}
\end{itemize}
O mesmo que \textunderscore alimpa\textunderscore . Cf. \textunderscore Bibl. da Gente do Campo\textunderscore , 311.
\section{Alimpadeiras}
\begin{itemize}
\item {Grp. gram.:f. pl.}
\end{itemize}
\begin{itemize}
\item {Proveniência:(De \textunderscore alimpar\textunderscore )}
\end{itemize}
Abelhas, que vão, adeante do seu bando, limpar o lugar, onde as outras irão trabalhar.
\section{Alimpador}
\begin{itemize}
\item {Grp. gram.:m.}
\end{itemize}
Aquelle que alimpa.
\section{Alimpadura}
\begin{itemize}
\item {Grp. gram.:f.}
\end{itemize}
Acção de \textunderscore alimpar\textunderscore .
Resíduo do que se limpa.
O que fica dos cereaes joeirados.
\section{Alimpamento}
\begin{itemize}
\item {Grp. gram.:m.}
\end{itemize}
Acto de \textunderscore alimpar\textunderscore .
\section{Alimpar}
\begin{itemize}
\item {Grp. gram.:v. t.}
\end{itemize}
Tornar limpo.
O mesmo que \textunderscore limpar\textunderscore .
\section{Alimpo}
\begin{itemize}
\item {Grp. gram.:m.}
\end{itemize}
\begin{itemize}
\item {Utilização:Prov.}
\end{itemize}
\begin{itemize}
\item {Proveniência:(De \textunderscore alimpar\textunderscore )}
\end{itemize}
Ramagem, que se supprime nas árvores, quando se podam. (Colhido em Turquel)
\section{Alindado}
\begin{itemize}
\item {Grp. gram.:adj.}
\end{itemize}
Enfeitado, aformoseado.
\section{Alindamento}
\begin{itemize}
\item {Grp. gram.:m.}
\end{itemize}
Acto de \textunderscore alindar\textunderscore .
\section{Alindar}
\begin{itemize}
\item {Grp. gram.:v. t.}
\end{itemize}
Tornar lindo.
Aformosear.
\section{Alinde}
\begin{itemize}
\item {Grp. gram.:m.}
\end{itemize}
Acto de \textunderscore alindar\textunderscore .
Ornato, enfeite.
\section{Alínea}
\begin{itemize}
\item {Grp. gram.:f.}
\end{itemize}
\begin{itemize}
\item {Proveniência:(Fr. \textunderscore alinéa\textunderscore )}
\end{itemize}
Nova linha escrita, cuja primeira palavra abre parágrapho.
Uma das subdivisões de artigo, desígnadas por \textunderscore a\textunderscore ), \textunderscore b\textunderscore ), \textunderscore c\textunderscore ), etc.
\section{Alinegro}
\begin{itemize}
\item {fónica:nê}
\end{itemize}
\begin{itemize}
\item {Grp. gram.:adj.}
\end{itemize}
\begin{itemize}
\item {Proveniência:(De \textunderscore ala\textunderscore  + \textunderscore negro\textunderscore )}
\end{itemize}
Que tem asas negras.
\section{Alinguetado}
\begin{itemize}
\item {fónica:gu-e}
\end{itemize}
\begin{itemize}
\item {Grp. gram.:adj.}
\end{itemize}
Que tem fórma de língua ou lingueta.
\section{Alinhado}
\begin{itemize}
\item {Grp. gram.:adj.}
\end{itemize}
Pôsto em linha.
Enfileirado.
\section{Alinhador}
\begin{itemize}
\item {Grp. gram.:m.}
\end{itemize}
Aquelle que alinha.
\section{Alinhamento}
\begin{itemize}
\item {Grp. gram.:m.}
\end{itemize}
Acto de \textunderscore alinhar\textunderscore .
Direcção do eixo de uma estrada, de um caminho de ferro, de um canal, etc.
\section{Alinhar}
\begin{itemize}
\item {Grp. gram.:v. t.}
\end{itemize}
Pôr em linha recta.
\section{Alinhavadeira}
\begin{itemize}
\item {Grp. gram.:f.}
\end{itemize}
Mulher, que alinhava em officinas de costura.
\section{Alinhavado}
\begin{itemize}
\item {Grp. gram.:adj.}
\end{itemize}
\begin{itemize}
\item {Proveniência:(De \textunderscore alinhavar\textunderscore )}
\end{itemize}
Cosido a ponto largo.
Mal feito.
\section{Alinhavar}
\begin{itemize}
\item {Grp. gram.:v. t.}
\end{itemize}
\begin{itemize}
\item {Proveniência:(De \textunderscore linha\textunderscore )}
\end{itemize}
Coser a ponto largo.
Preparar.
Executar mal.
\section{Alinhavo}
\begin{itemize}
\item {Grp. gram.:m.}
\end{itemize}
Acção de \textunderscore alinhavar\textunderscore .
Os pontos, com que se alinhava.
Esbôço.
\section{Alinho}
\begin{itemize}
\item {Grp. gram.:m.}
\end{itemize}
\begin{itemize}
\item {Utilização:Ant.}
\end{itemize}
Acto ou effeito de \textunderscore alinhar\textunderscore .
Asseio.
Ornato, atavio.
Decência.
Enrêdo.
Vencilho.
Conservação do que se adquiriu.
\section{Alinita}
\begin{itemize}
\item {Grp. gram.:f.}
\end{itemize}
Bacillo que, segundo um recente descobrimento, (1897), promove o desenvolvimento dos cereaes sem auxílio do azoto.
\section{Alio}
\begin{itemize}
\item {Grp. gram.:m.}
\end{itemize}
Árvore de Damão.
\section{Aliónia}
\begin{itemize}
\item {Grp. gram.:f.}
\end{itemize}
\begin{itemize}
\item {Proveniência:(De \textunderscore Allioni\textunderscore , n. p.)}
\end{itemize}
Género de nyctagíneas.
Gênero de plantas nyctagíneas da América.
\section{Alipata}
\begin{itemize}
\item {Grp. gram.:f.}
\end{itemize}
Árvore das Filippinas, de propriedades venenosas.
\section{Alípede}
\begin{itemize}
\item {Grp. gram.:adj.}
\end{itemize}
\begin{itemize}
\item {Proveniência:(Do lat. \textunderscore ala\textunderscore  + \textunderscore pes\textunderscore )}
\end{itemize}
Que tem asas nos pés.
\section{Alípilo}
\begin{itemize}
\item {Grp. gram.:m.}
\end{itemize}
\begin{itemize}
\item {Proveniência:(Lat. \textunderscore alipilus\textunderscore )}
\end{itemize}
Escravo que, nas thermas romanas, era encarregado de arrancar os pelos de certas partes do corpo dos banhistas, especialmente dos sovacos.
\section{Alipivre}
\begin{itemize}
\item {Grp. gram.:m.}
\end{itemize}
\begin{itemize}
\item {Utilização:Bot.}
\end{itemize}
Antiga designação da semente da nigella. Cf. \textunderscore Desengano da Med.\textunderscore , 175.
\section{Alipotente}
\begin{itemize}
\item {fónica:áli}
\end{itemize}
\begin{itemize}
\item {Grp. gram.:adj.}
\end{itemize}
\begin{itemize}
\item {Proveniência:(Do lat. \textunderscore ala\textunderscore  + \textunderscore potens\textunderscore )}
\end{itemize}
Que tem asas possantes.
\section{Alipta}
\begin{itemize}
\item {Grp. gram.:m.}
\end{itemize}
\begin{itemize}
\item {Proveniência:(Gr. \textunderscore aliptes\textunderscore )}
\end{itemize}
Aquelle que, entre os Romanos e Gregos, untava com essências os que saíam do banho, e os athletas antes do combate.
\section{Alíptica}
\begin{itemize}
\item {Grp. gram.:f.}
\end{itemize}
\begin{itemize}
\item {Proveniência:(Gr. \textunderscore aleiptike\textunderscore )}
\end{itemize}
Arte de applicar unturas medicinaes ou hygiénicas.
\section{Alipto}
\begin{itemize}
\item {Grp. gram.:m.}
\end{itemize}
\begin{itemize}
\item {Proveniência:(Gr. \textunderscore aliptes\textunderscore )}
\end{itemize}
Aquelle que, entre os Romanos e Gregos, untava com essências os que saíam do banho, e os athletas antes do combate.
\section{Aliquanta}
\begin{itemize}
\item {Grp. gram.:adj.}
\end{itemize}
\begin{itemize}
\item {Proveniência:(Lat. \textunderscore aliquanta\textunderscore )}
\end{itemize}
Diz-se da parte que não divide um todo sem deixar resto.
\section{Alíquota}
\begin{itemize}
\item {Grp. gram.:adj.}
\end{itemize}
\begin{itemize}
\item {Utilização:Mús.}
\end{itemize}
\begin{itemize}
\item {Proveniência:(Lat. \textunderscore aliquot\textunderscore )}
\end{itemize}
Diz-se da parte ou quantidade contida noutra, um número exacto de vezes.
Diz-se dos sons harmoniosos, produzidos por um determinado som fundamental.
\section{Alisado}
\begin{itemize}
\item {Grp. gram.:adj.}
\end{itemize}
\begin{itemize}
\item {Proveniência:(Do ant. fr. \textunderscore alis\textunderscore )}
\end{itemize}
Diz-se do vento, que sopra regularmente de Léste para Oéste, entre os Trópicos.
\section{Alisador}
\begin{itemize}
\item {Grp. gram.:m.  e  adj.}
\end{itemize}
Aquelle que alisa.
\section{Alisamento}
\begin{itemize}
\item {Grp. gram.:m.}
\end{itemize}
Acto de \textunderscore alisar\textunderscore .
\section{Alisar}
\begin{itemize}
\item {Grp. gram.:v. t.}
\end{itemize}
\begin{itemize}
\item {Utilização:Gír.}
\end{itemize}
\begin{itemize}
\item {Grp. gram.:v. p.}
\end{itemize}
\begin{itemize}
\item {Utilização:Prov.}
\end{itemize}
\begin{itemize}
\item {Proveniência:(De \textunderscore liso\textunderscore )}
\end{itemize}
Tornar liso.
Igualar.
Amaciar.
Desgastar.
Abrandar; adoçar.
Fartar.
Pentear-se.
\section{Alísio}
\begin{itemize}
\item {Grp. gram.:adj.}
\end{itemize}
\begin{itemize}
\item {Utilização:Bras}
\end{itemize}
O mesmo que \textunderscore alisado\textunderscore .
\section{Alisma}
\begin{itemize}
\item {Grp. gram.:f.}
\end{itemize}
\begin{itemize}
\item {Proveniência:(Gr. \textunderscore alisma\textunderscore )}
\end{itemize}
Planta, de haste articulada e ôca, flores rosadas e fôlhas campanuladas.
\section{Alismáceas}
\begin{itemize}
\item {Grp. gram.:f. pl.}
\end{itemize}
Família de plantas monocotyledóneas perispérmicas, a que serve de typo a alisma.
\section{Alismoide}
\begin{itemize}
\item {Grp. gram.:adj.}
\end{itemize}
\begin{itemize}
\item {Proveniência:(Do gr. \textunderscore alisma\textunderscore  + \textunderscore eidos\textunderscore )}
\end{itemize}
Semelhante á alisma.
\section{Alisonita}
\begin{itemize}
\item {Grp. gram.:f.}
\end{itemize}
O mesmo que \textunderscore alisonite\textunderscore .
\section{Alisonite}
\begin{itemize}
\item {Grp. gram.:f.}
\end{itemize}
\begin{itemize}
\item {Proveniência:(De \textunderscore Alison\textunderscore , n. p.)}
\end{itemize}
Mineral compacto, composto de cobre, chumbo e enxôfre.
\section{Alistado}
\begin{itemize}
\item {Grp. gram.:adj.}
\end{itemize}
\begin{itemize}
\item {Proveniência:(De \textunderscore alistar\textunderscore )}
\end{itemize}
Pôsto em lista.
Recrutado.
\section{Alistamento}
\begin{itemize}
\item {Grp. gram.:m.}
\end{itemize}
Acto de \textunderscore alistar\textunderscore .
\section{Alistão}
\begin{itemize}
\item {Grp. gram.:m.}
\end{itemize}
\begin{itemize}
\item {Proveniência:(Do cast. \textunderscore ariston\textunderscore )}
\end{itemize}
Pedra faceada e esquadreada para cantaria.
\section{Alistar}
\begin{itemize}
\item {Grp. gram.:v. t.}
\end{itemize}
Pôr em lista.
Arrolar.
Recrutar.
\section{Alistridente}
\begin{itemize}
\item {fónica:ális}
\end{itemize}
\begin{itemize}
\item {Grp. gram.:adj.}
\end{itemize}
\begin{itemize}
\item {Proveniência:(Do lat. \textunderscore ala\textunderscore  + \textunderscore stridens\textunderscore )}
\end{itemize}
Que faz estridor com as asas.
\section{Alite}
\begin{itemize}
\item {Grp. gram.:m.}
\end{itemize}
Gênero de batrácios.
\section{Aliteração}
\begin{itemize}
\item {Grp. gram.:f.}
\end{itemize}
\begin{itemize}
\item {Utilização:Philol.}
\end{itemize}
\begin{itemize}
\item {Proveniência:(De \textunderscore aliterar\textunderscore )}
\end{itemize}
Emprêgo de vocábulo que começa pela mesma letra que outro já empregado.
\section{Aliterar}
\begin{itemize}
\item {Grp. gram.:v. t.}
\end{itemize}
\begin{itemize}
\item {Grp. gram.:V. i.}
\end{itemize}
\begin{itemize}
\item {Proveniência:(Do lat. \textunderscore ad\textunderscore  + \textunderscore literam\textunderscore )}
\end{itemize}
Dispor em aliteração.
Formar aliteração.
\section{Alitronco}
\begin{itemize}
\item {fónica:áli}
\end{itemize}
\begin{itemize}
\item {Grp. gram.:m.}
\end{itemize}
\begin{itemize}
\item {Proveniência:(De \textunderscore ala\textunderscore  + \textunderscore tronco\textunderscore )}
\end{itemize}
Parte posterior do tronco dos insectos, onde estão as asas.
\section{Alitúrgico}
\begin{itemize}
\item {Grp. gram.:adj.}
\end{itemize}
\begin{itemize}
\item {Proveniência:(De \textunderscore a\textunderscore  priv. + \textunderscore litúrgico\textunderscore )}
\end{itemize}
Diz-se do dia, que não tem offício próprio na Igreja.
\section{Alivar}
\begin{itemize}
\item {Grp. gram.:v. t.}
\end{itemize}
\begin{itemize}
\item {Utilização:Ant.}
\end{itemize}
O mesmo que \textunderscore alliviar\textunderscore .
\section{Alixado}
\begin{itemize}
\item {Grp. gram.:adj.}
\end{itemize}
Que tem o aspecto ou a aspereza da lixa.
\section{Alizaba}
\begin{itemize}
\item {Grp. gram.:f.}
\end{itemize}
Espécie de túnica, com mangas largas, aberta adeante, e usada pelos Moiros.
(Cp. \textunderscore aljuba\textunderscore )
\section{Alizar}
\begin{itemize}
\item {Grp. gram.:m.}
\end{itemize}
Guarnição de madeira, que reveste as ombreiras das portas e janelas.
Guarda-vassoiras.
Faixa de azulejo ao fundo da parede.
(Ár. \textunderscore al-izar\textunderscore )
\section{Alizari}
\begin{itemize}
\item {Grp. gram.:m.}
\end{itemize}
Nome commercial da raíz da ruiva ou da garança.
\section{Alizarina}
\begin{itemize}
\item {Grp. gram.:f.}
\end{itemize}
\begin{itemize}
\item {Proveniência:(De \textunderscore alizari\textunderscore )}
\end{itemize}
Substância còrante, extrahida da raíz da ruiva.
\section{Alizita}
\begin{itemize}
\item {Grp. gram.:f.}
\end{itemize}
Silicato hydratado de níquel, com um pouco de magnésia e de ferro.
\section{Alizite}
\begin{itemize}
\item {Grp. gram.:f.}
\end{itemize}
Silicato hydratado de níquel, com um pouco de magnésia e de ferro.
\section{Aljafra}
\begin{itemize}
\item {Grp. gram.:f.}
\end{itemize}
\begin{itemize}
\item {Proveniência:(Do ár. \textunderscore al-jafna\textunderscore ?)}
\end{itemize}
Bôlso ou seio das redes de arrastar.
\section{Aljama}
\begin{itemize}
\item {Grp. gram.:f.}
\end{itemize}
\begin{itemize}
\item {Utilização:Ant.}
\end{itemize}
Confraria.
Synagoga.
Ajuntamento de moradores moiros ou judeus, que, á custa de impostos, formavam bairros ou povoações em terra portuguesa.
(Ár. \textunderscore al-jama\textunderscore )
\section{Aljamia}
\begin{itemize}
\item {Grp. gram.:f.}
\end{itemize}
O mesmo que \textunderscore algemia\textunderscore . Cf. D. Lopes, \textunderscore Textos em Aljamia\textunderscore .
\section{Aljamiado}
\begin{itemize}
\item {Grp. gram.:adj.}
\end{itemize}
\begin{itemize}
\item {Utilização:Ant.}
\end{itemize}
\begin{itemize}
\item {Proveniência:(De \textunderscore aljamia\textunderscore )}
\end{itemize}
Instruido, erudito.
\section{Aljarabia}
\begin{itemize}
\item {Grp. gram.:f.}
\end{itemize}
\begin{itemize}
\item {Utilização:Ant.}
\end{itemize}
Espécie de túnica moirisca, ou roupão de meias mangas, com capuz. Cf. Herculano, \textunderscore Lendas\textunderscore , I, 18 e 35.
\section{Aljarás}
\begin{itemize}
\item {Grp. gram.:m.}
\end{itemize}
\begin{itemize}
\item {Utilização:Prov.}
\end{itemize}
\begin{itemize}
\item {Utilização:trasm.}
\end{itemize}
Guizo de cão.
\section{Aljaravia}
\begin{itemize}
\item {Grp. gram.:f.}
\end{itemize}
\begin{itemize}
\item {Utilização:Ant.}
\end{itemize}
Espécie de túnica moirisca, ou roupão de meias mangas, com capuz. Cf. Herculano, \textunderscore Lendas\textunderscore , I, 18 e 35.
\section{Aljarós}
\begin{itemize}
\item {Grp. gram.:m.}
\end{itemize}
(V.algerós)
\section{Aljava}
\begin{itemize}
\item {Grp. gram.:f.}
\end{itemize}
\begin{itemize}
\item {Utilização:Prov.}
\end{itemize}
\begin{itemize}
\item {Utilização:alent.}
\end{itemize}
\begin{itemize}
\item {Proveniência:(Do ár. \textunderscore al-jaba\textunderscore )}
\end{itemize}
Carcaz ou estojo, em que se metiam as setas, e que se levava ao ombro.
Recipiente, em que se leva a negaça para a caça de armadilha.
\section{Aljaveira}
\begin{itemize}
\item {Grp. gram.:f.}
\end{itemize}
\begin{itemize}
\item {Utilização:Ant.}
\end{itemize}
O mesmo que \textunderscore aljava\textunderscore ? Cf. \textunderscore Carta\textunderscore  de Pero Vaz de Caminha a D. Man.
\section{Aljazar}
\begin{itemize}
\item {Grp. gram.:m.}
\end{itemize}
Terreno sêco, rodeado de água do mar.
(Ár. \textunderscore al-jazar\textunderscore )
\section{Aljofaina}
\begin{itemize}
\item {Grp. gram.:f.}
\end{itemize}
\begin{itemize}
\item {Utilização:Ant.}
\end{itemize}
\begin{itemize}
\item {Proveniência:(Do ár. \textunderscore gufaine\textunderscore )}
\end{itemize}
Pequena bacia de lavatório.
\section{Aljôfar}
\begin{itemize}
\item {Grp. gram.:m.}
\end{itemize}
Pérola miúda.
Orvalho.
Lágrimas.
(Ár. \textunderscore al-jauhar\textunderscore )
\section{Aljofarar}
\begin{itemize}
\item {Grp. gram.:v. t.}
\end{itemize}
Cobrir de aljôfar.
Orvalhar.
\section{Aljofareira}
\begin{itemize}
\item {Grp. gram.:f.}
\end{itemize}
Planta, de sementes parecidas a aljôfar.
\section{Aljofrar}
\begin{itemize}
\item {Grp. gram.:v. t.}
\end{itemize}
\begin{itemize}
\item {Proveniência:(De \textunderscore aljôfre\textunderscore )}
\end{itemize}
O mesmo que \textunderscore aljofarar\textunderscore .
\section{Aljôfre}
\begin{itemize}
\item {Grp. gram.:m.}
\end{itemize}
O mesmo que \textunderscore aljôfar\textunderscore .
\section{Aljorce}
\begin{itemize}
\item {Grp. gram.:m.}
\end{itemize}
\begin{itemize}
\item {Utilização:Prov.}
\end{itemize}
\begin{itemize}
\item {Utilização:beir.}
\end{itemize}
\begin{itemize}
\item {Proveniência:(Do ár. \textunderscore al-jaras\textunderscore ?)}
\end{itemize}
Campaínha, chocalho, que se põe ao pescoço das bêstas.
\section{Aljuba}
\begin{itemize}
\item {Grp. gram.:f.}
\end{itemize}
\begin{itemize}
\item {Proveniência:(Do ár. \textunderscore al-jubba\textunderscore )}
\end{itemize}
Gibão.
Vestidura moirisca, talar e sem mangas ou com meias mangas.
\section{Aljubádigo}
\begin{itemize}
\item {Grp. gram.:m.}
\end{itemize}
\begin{itemize}
\item {Utilização:Ant.}
\end{itemize}
O mesmo que \textunderscore carceragem\textunderscore :«\textunderscore dê cinco soldos de aljubádigo\textunderscore ». Herculano, \textunderscore Hist. de Port.\textunderscore , IV, 157.
\section{Aljube}
\begin{itemize}
\item {Grp. gram.:m.}
\end{itemize}
\begin{itemize}
\item {Proveniência:(Do ár. \textunderscore al-jubb\textunderscore )}
\end{itemize}
Caverna.
Prisão, carcere escuro.
\section{Aljubeiro}
\begin{itemize}
\item {Grp. gram.:m.}
\end{itemize}
\begin{itemize}
\item {Utilização:Des.}
\end{itemize}
\begin{itemize}
\item {Proveniência:(De \textunderscore aljube\textunderscore )}
\end{itemize}
Carcereiro.
\section{Aljubeta}
\begin{itemize}
\item {fónica:bê}
\end{itemize}
\textunderscore f.\textunderscore  (dem. de \textunderscore aljuba\textunderscore )
* O mesmo que \textunderscore algibebe\textunderscore . Us. por Camillo.
\section{Aljubeteiro}
\begin{itemize}
\item {Grp. gram.:m.}
\end{itemize}
Aquelle que fazia aljubetas.
\section{Aljuz}
\begin{itemize}
\item {Grp. gram.:m.}
\end{itemize}
Colla, extrahida do cardo matacão.
\section{Allagite}
\begin{itemize}
\item {Grp. gram.:m.}
\end{itemize}
Variedade de manganês silicífero.
\section{Allamanda}
\begin{itemize}
\item {Grp. gram.:f.}
\end{itemize}
\begin{itemize}
\item {Proveniência:(De \textunderscore Allamand\textunderscore , n. p.)}
\end{itemize}
Gênero de plantas apocýneas.
\section{Allânia}
\begin{itemize}
\item {Grp. gram.:f.}
\end{itemize}
Gênero de plantas leguminosas.
\section{Allanita}
\begin{itemize}
\item {Grp. gram.:f.}
\end{itemize}
Substância negra e vitrea, que corta o crystal.
\section{Allanite}
\begin{itemize}
\item {Grp. gram.:f.}
\end{itemize}
O mesmo que \textunderscore allanita\textunderscore .
\section{Allantíase}
\begin{itemize}
\item {Grp. gram.:f.}
\end{itemize}
\begin{itemize}
\item {Proveniência:(Do gr. \textunderscore allas\textunderscore , salsicha)}
\end{itemize}
Entoxicação alimentar, devida á ingestão de salsichas alteradas.
\section{Allanto}
\begin{itemize}
\item {Grp. gram.:m.}
\end{itemize}
Gênero de insectos hymenópteros.
\section{Allantoico}
\begin{itemize}
\item {Grp. gram.:adj.}
\end{itemize}
(V.allantoidiano)
\section{Allantoide}
\begin{itemize}
\item {Grp. gram.:f.}
\end{itemize}
\begin{itemize}
\item {Proveniência:(Gr. \textunderscore allantoeides\textunderscore )}
\end{itemize}
Membrana do féto dos mammíferos, durante os dois primeiros meses de vida intra-uterína, e da qual ao depois se fórma a bexiga e a placenta.
\section{Allantoidiano}
\begin{itemize}
\item {Grp. gram.:adj.}
\end{itemize}
Relativo á \textunderscore allantoide\textunderscore .
\section{Allantoína}
\begin{itemize}
\item {Grp. gram.:f.}
\end{itemize}
Substância neutra, que se encontra no líquido allantoidiano.
\section{Allantos}
\begin{itemize}
\item {Grp. gram.:m. pl.}
\end{itemize}
\begin{itemize}
\item {Proveniência:(Do gr. \textunderscore allas\textunderscore , \textunderscore allantos\textunderscore )}
\end{itemize}
Gênero de insectos hymenópteros.
\section{Allariz}
\begin{itemize}
\item {Grp. gram.:m.}
\end{itemize}
\begin{itemize}
\item {Proveniência:(De \textunderscore Allariz\textunderscore , n. p.)}
\end{itemize}
Nome de um tecido, que se fabrica na Galliza.
\section{Allécula}
\begin{itemize}
\item {Grp. gram.:f.}
\end{itemize}
Insecto coleóptero heterómero.
\section{Allegação}
\begin{itemize}
\item {Grp. gram.:f.}
\end{itemize}
Acto de \textunderscore allegar\textunderscore .
\section{Allegança}
\begin{itemize}
\item {Grp. gram.:f.}
\end{itemize}
\begin{itemize}
\item {Utilização:Ant.}
\end{itemize}
O mesmo que \textunderscore allegação\textunderscore .
\section{Allegante}
\begin{itemize}
\item {Grp. gram.:m.  e  adj.}
\end{itemize}
\begin{itemize}
\item {Proveniência:(Lat. \textunderscore allegans\textunderscore )}
\end{itemize}
O que allega.
\section{Allegar}
\begin{itemize}
\item {Grp. gram.:v. t.}
\end{itemize}
\begin{itemize}
\item {Proveniência:(Lat. \textunderscore allegare\textunderscore )}
\end{itemize}
Citar; apresentar, como prova.
\section{Allegoria}
\begin{itemize}
\item {Grp. gram.:f.}
\end{itemize}
\begin{itemize}
\item {Proveniência:(Gr. \textunderscore allegoria\textunderscore )}
\end{itemize}
Exposição de uma ideia, sob fórma figurada.
Obra artística ou literária, que representa uma coisa, para dar ideia de outra.
\section{Allegoricamente}
\begin{itemize}
\item {Grp. gram.:adv.}
\end{itemize}
De modo \textunderscore allegórico\textunderscore .
\section{Allegórico}
\begin{itemize}
\item {Grp. gram.:adj.}
\end{itemize}
Que envolve allegoria.
Relativo a \textunderscore allegoria\textunderscore .
\section{Allegorismo}
\begin{itemize}
\item {Grp. gram.:m.}
\end{itemize}
Systema de explicar por allegoria.
\section{Allegorista}
\begin{itemize}
\item {Grp. gram.:m.}
\end{itemize}
Aquelle que faz allegorias.
Aquelle que explica por allegorias escritos de outrem.
\section{Allegorizar}
\begin{itemize}
\item {Grp. gram.:v. t.}
\end{itemize}
Expor, por meio de allegoria.
Explicar em sentido allegórico.
\section{Allegreto}
\begin{itemize}
\item {fónica:grê}
\end{itemize}
\begin{itemize}
\item {Grp. gram.:m.}
\end{itemize}
Andamento musical, menos vivo que o \textunderscore allegro\textunderscore .
(\textunderscore T. it.\textunderscore )
\section{Allegro}
\begin{itemize}
\item {Grp. gram.:m.}
\end{itemize}
\begin{itemize}
\item {Grp. gram.:Adv.}
\end{itemize}
Peça musical, de andamento vivo e rápido.
Vivamente, rapidamente.
(\textunderscore T. it.\textunderscore )
\section{Alleluia}
\begin{itemize}
\item {fónica:á-lé-lúi-a}
\end{itemize}
\begin{itemize}
\item {Grp. gram.:f.}
\end{itemize}
\begin{itemize}
\item {Proveniência:(Hebr. \textunderscore halelu\textunderscore  + \textunderscore iah\textunderscore )}
\end{itemize}
Canto de alegria.
Alegria.
O tempo da Páscoa.
Nome de uma planta, que floresce pela Páscoa.
\section{Alleluiar}
\begin{itemize}
\item {Grp. gram.:v. i.}
\end{itemize}
Cantar a alleluia.
\section{Alleluiático}
\begin{itemize}
\item {Grp. gram.:adj.}
\end{itemize}
Relativo á \textunderscore alleluia\textunderscore .
\section{Alleluítico}
\begin{itemize}
\item {fónica:á-lé-lu-í}
\end{itemize}
\begin{itemize}
\item {Grp. gram.:adj.}
\end{itemize}
\begin{itemize}
\item {Proveniência:(De \textunderscore alleluia\textunderscore )}
\end{itemize}
Que louva, saúda ou celebra.
\section{Allesthesia}
\begin{itemize}
\item {Grp. gram.:f.}
\end{itemize}
\begin{itemize}
\item {Utilização:Med.}
\end{itemize}
\begin{itemize}
\item {Proveniência:(Do gr. \textunderscore allos\textunderscore , outro, e \textunderscore aisthesis\textunderscore , sensação)}
\end{itemize}
Estado pathológico, caracterizado pela localização das sensações no lado do corpo, opposto ao que recebeu a impressão, e em ponto symétrico.
\section{Alli}
\begin{itemize}
\item {Grp. gram.:adv.}
\end{itemize}
\begin{itemize}
\item {Proveniência:(Lat. \textunderscore illic\textunderscore )}
\end{itemize}
Lá.
Além; naquelle lugar.
\section{Alliáceas}
\begin{itemize}
\item {Grp. gram.:f. pl.}
\end{itemize}
Plantas, que constituem um grupo das liliáceas, e a que serve de typo o alho.
\section{Alliáceo}
\begin{itemize}
\item {Grp. gram.:adj.}
\end{itemize}
\begin{itemize}
\item {Proveniência:(Do lat. \textunderscore allium\textunderscore )}
\end{itemize}
Relativo ao alho.
Semelhante ao alho.
\section{Alliado}
\begin{itemize}
\item {Grp. gram.:adj.}
\end{itemize}
\begin{itemize}
\item {Grp. gram.:M.}
\end{itemize}
Que se alliou.
Aquelle que entrou numa alliança:«\textunderscore os alliados em Sebastopol\textunderscore ».
\section{Alliagem}
\begin{itemize}
\item {Grp. gram.:f.}
\end{itemize}
(V.alliança)
\section{Alliança}
\begin{itemize}
\item {Grp. gram.:f.}
\end{itemize}
Acto ou effeito de \textunderscore alliar\textunderscore .
Anel de casamento.
\section{Alliançar}
\begin{itemize}
\item {Grp. gram.:v. t.}
\end{itemize}
(V.alliar)
\section{Alliar}
\begin{itemize}
\item {Grp. gram.:v. t.}
\end{itemize}
\begin{itemize}
\item {Proveniência:(Do lat. \textunderscore alligare\textunderscore )}
\end{itemize}
Unir.
Combinar.
Fazer ligação de.
Harmonizar.
Unir por casamento.
Confederar.
Encorporar.
Coadunar.
Agrupar.
\section{Alliária}
\begin{itemize}
\item {Grp. gram.:f.}
\end{itemize}
\begin{itemize}
\item {Proveniência:(Do lat. \textunderscore allium\textunderscore )}
\end{itemize}
Planta crucífera, de cheiro semelhante ao do alho.
\section{Alliável}
\begin{itemize}
\item {Grp. gram.:adj.}
\end{itemize}
Que se póde \textunderscore alliar\textunderscore .
\section{Alliciação}
\begin{itemize}
\item {Grp. gram.:f.}
\end{itemize}
Acto de \textunderscore alliciar\textunderscore .
\section{Alliciador}
\begin{itemize}
\item {Grp. gram.:m.}
\end{itemize}
\begin{itemize}
\item {Grp. gram.:Adj.}
\end{itemize}
Aquelle que allicia.
Aquelle que serve para \textunderscore alliciar\textunderscore .
Que allicia.
\section{Alliciamento}
\begin{itemize}
\item {Grp. gram.:m.}
\end{itemize}
(V.alliciação)
\section{Alliciante}
\begin{itemize}
\item {Grp. gram.:adj.}
\end{itemize}
Que allicia; alliciador.
\section{Alliciar}
\begin{itemize}
\item {Grp. gram.:v. t.}
\end{itemize}
\begin{itemize}
\item {Proveniência:(Do lat. \textunderscore allicere\textunderscore )}
\end{itemize}
Atrahir a si.
Seduzir.
Provocar.
Angariar.
Subornar.
\section{Alliciente}
\begin{itemize}
\item {Grp. gram.:adj.}
\end{itemize}
\begin{itemize}
\item {Grp. gram.:M.}
\end{itemize}
\begin{itemize}
\item {Proveniência:(Lat. \textunderscore alliciens\textunderscore )}
\end{itemize}
Que allicia.
Coisa que allicia.
Seducção.
\section{Alligatór}
\begin{itemize}
\item {Grp. gram.:m.}
\end{itemize}
Gênero de reptis sáurios, em cujas espécies entra o caimão, (\textunderscore lacerta alligator\textunderscore , Lin.).
(Fórma ingl., corr. do cast. \textunderscore el lagarto\textunderscore )
\section{Allilena}
\begin{itemize}
\item {Grp. gram.:f.}
\end{itemize}
\begin{itemize}
\item {Proveniência:(De \textunderscore allilo\textunderscore )}
\end{itemize}
Gás incolor, inflammável, de cheiro desagradável.
\section{Alileno}
\begin{itemize}
\item {Grp. gram.:m.}
\end{itemize}
O mesmo que \textunderscore alilena\textunderscore .
\section{Alilo}
\begin{itemize}
\item {Grp. gram.:m.}
\end{itemize}
\begin{itemize}
\item {Proveniência:(Do lat. \textunderscore allium\textunderscore )}
\end{itemize}
Líquido volátil, de cheiro análogo ao do rábão.
Um aos radicaes chímicos, descoberto primeiramente na essência da mostarda e do alho.
\section{Alite}
\begin{itemize}
\item {Grp. gram.:m.}
\end{itemize}
Primeira estrêlla da cauda da Ursa-Maior.
\section{Aliviação}
\begin{itemize}
\item {Grp. gram.:f.}
\end{itemize}
(V.alívio)
\section{Aliviadamente}
\begin{itemize}
\item {Grp. gram.:adv.}
\end{itemize}
Com alívio.
\section{Aliviador}
\begin{itemize}
\item {Grp. gram.:m.}
\end{itemize}
Aquelle que alivia.
\section{Aliviamento}
\begin{itemize}
\item {Grp. gram.:m.}
\end{itemize}
Acto de aliviar.
\section{Aliviar}
\begin{itemize}
\item {Grp. gram.:v. t.}
\end{itemize}
\begin{itemize}
\item {Grp. gram.:V. i.}
\end{itemize}
\begin{itemize}
\item {Proveniência:(Do lat. hyp. \textunderscore elivigare\textunderscore , segundo Cornu)}
\end{itemize}
Descarregar.
Suavizar, atenuar: \textunderscore aliviar dores\textunderscore .
Tornar mais leve: \textunderscore aliviar a carga\textunderscore .
Desimpedir.
Distrahir.
Consolar: \textunderscore esta notícia aliviou-me\textunderscore .
Isentar.
Serenar; abrandar: \textunderscore o tempo aliviou\textunderscore .
\section{Alívio}
\begin{itemize}
\item {Grp. gram.:m.}
\end{itemize}
Effeito de \textunderscore aliviar\textunderscore .
Descanso.
Consolação.
\section{Alivioso}
\begin{itemize}
\item {Grp. gram.:adj.}
\end{itemize}
Que causa alívio. Cf. Filinto, XXI, 87.
\section{Allileno}
\begin{itemize}
\item {Grp. gram.:m.}
\end{itemize}
O mesmo que \textunderscore allilena\textunderscore .
\section{Allilo}
\begin{itemize}
\item {Grp. gram.:m.}
\end{itemize}
\begin{itemize}
\item {Proveniência:(Do lat. \textunderscore allium\textunderscore )}
\end{itemize}
Líquido volátil, de cheiro análogo ao do rábão.
Um aos radicaes chímicos, descoberto primeiramente na essência da mostarda e do alho.
\section{Alliónia}
\begin{itemize}
\item {Grp. gram.:f.}
\end{itemize}
\begin{itemize}
\item {Proveniência:(De \textunderscore Allioni\textunderscore , n. p.)}
\end{itemize}
Gênero de plantas nyctagíneas da América.
\section{Allith}
\begin{itemize}
\item {Grp. gram.:m.}
\end{itemize}
Primeira estrêlla da cauda da Ursa-Maior.
\section{Alliviacão}
\begin{itemize}
\item {Grp. gram.:f.}
\end{itemize}
(V.allívio)
\section{Alliviadamente}
\begin{itemize}
\item {Grp. gram.:adv.}
\end{itemize}
Com allívio.
\section{Alliviador}
\begin{itemize}
\item {Grp. gram.:m.}
\end{itemize}
Aquelle que allivia.
\section{Alliviamento}
\begin{itemize}
\item {Grp. gram.:m.}
\end{itemize}
Acto de alliviar.
\section{Alliviar}
\begin{itemize}
\item {Grp. gram.:v. t.}
\end{itemize}
\begin{itemize}
\item {Grp. gram.:V. i.}
\end{itemize}
\begin{itemize}
\item {Proveniência:(Do lat. hyp. \textunderscore elivigare\textunderscore , segundo Cornu)}
\end{itemize}
Descarregar.
Suavizar, atenuar: \textunderscore alliviar dores\textunderscore .
Tornar mais leve: \textunderscore alliviar a carga\textunderscore .
Desimpedir.
Distrahir.
Consolar: \textunderscore esta notícia alliviou-me\textunderscore .
Isentar.
Serenar; abrandar: \textunderscore o tempo alliviou\textunderscore .
\section{Allívio}
\begin{itemize}
\item {Grp. gram.:m.}
\end{itemize}
Effeito de \textunderscore alliviar\textunderscore .
Descanso.
Consolação.
\section{Allivioso}
\begin{itemize}
\item {Grp. gram.:adj.}
\end{itemize}
Que causa allívio. Cf. Filinto, XXI, 87.
\section{Allóbroge}
\begin{itemize}
\item {Grp. gram.:m.}
\end{itemize}
O mesmo ou melhor que \textunderscore allóbrogo\textunderscore .
\section{Allobrógico}
\begin{itemize}
\item {Grp. gram.:adj.}
\end{itemize}
Relativo aos allóbrogos; próprio de \textunderscore allóbrogo\textunderscore .
\section{Allóbrogo}
\begin{itemize}
\item {Grp. gram.:m.}
\end{itemize}
\begin{itemize}
\item {Grp. gram.:Pl.}
\end{itemize}
\begin{itemize}
\item {Proveniência:(Do lat. \textunderscore allobrox\textunderscore )}
\end{itemize}
Homem grosseiro, rústico.
Povos antigos da região, que hoje se chama Saboia.
\section{Allócero}
\begin{itemize}
\item {Grp. gram.:m.}
\end{itemize}
\begin{itemize}
\item {Proveniência:(Do gr. \textunderscore allos\textunderscore  + \textunderscore keras\textunderscore )}
\end{itemize}
Insecto coleóptero tetrâmero, originário do Brasil.
\section{Allochezia}
\begin{itemize}
\item {fónica:que}
\end{itemize}
\begin{itemize}
\item {Grp. gram.:f.}
\end{itemize}
Evacuação das matérias fecaes por abertura accidental ou anormal.
\section{Allochiria}
\begin{itemize}
\item {fónica:qui}
\end{itemize}
\begin{itemize}
\item {Grp. gram.:f.}
\end{itemize}
\begin{itemize}
\item {Utilização:Med.}
\end{itemize}
\begin{itemize}
\item {Proveniência:(Do gr. \textunderscore allos\textunderscore , outro, e \textunderscore kheir\textunderscore , mão)}
\end{itemize}
O mesmo que \textunderscore allesthesia\textunderscore .
\section{Allochroito}
\begin{itemize}
\item {Grp. gram.:m.}
\end{itemize}
\begin{itemize}
\item {Proveniência:(Do gr. \textunderscore allokroos\textunderscore )}
\end{itemize}
Mineral, que é uma das variedades da granada.
\section{Allochromasia}
\begin{itemize}
\item {Grp. gram.:f.}
\end{itemize}
\begin{itemize}
\item {Proveniência:(Do gr. \textunderscore allos\textunderscore  + \textunderscore khroma\textunderscore )}
\end{itemize}
Doença, caracterizada por se verem côres differentes das que são realmente.
\section{Allocinesia}
\begin{itemize}
\item {Grp. gram.:f.}
\end{itemize}
\begin{itemize}
\item {Utilização:Med.}
\end{itemize}
\begin{itemize}
\item {Proveniência:(Do gr. \textunderscore allos\textunderscore , outro, e \textunderscore kinesis\textunderscore , movimento)}
\end{itemize}
Perturbação da sensibilidade, em que o doente move um dos membros, quando deseja mover outro.
\section{Allocução}
\begin{itemize}
\item {Grp. gram.:f.}
\end{itemize}
\begin{itemize}
\item {Proveniência:(Lat. \textunderscore allocutio\textunderscore )}
\end{itemize}
Discurso, geralmente breve, pronunciado em occasião solenne.
\section{Allopada}
\begin{itemize}
\item {Grp. gram.:f.}
\end{itemize}
Gênero de insectos do Cabo da Bôa-Esperança.
\section{Allodial}
\begin{itemize}
\item {Grp. gram.:adj.}
\end{itemize}
\begin{itemize}
\item {Proveniência:(De \textunderscore allódio\textunderscore )}
\end{itemize}
Livre de encargos ou de direitos senhoriaes.
\section{Allodialidade}
\begin{itemize}
\item {Grp. gram.:f.}
\end{itemize}
Isenção.
Qualidade do que é \textunderscore allodial\textunderscore .
\section{Allódio}
\begin{itemize}
\item {Grp. gram.:m.}
\end{itemize}
\begin{itemize}
\item {Utilização:Ant.}
\end{itemize}
Propriedades ou bens, isentos de encargos senhoriaes. Cf. Herculano, \textunderscore Quest. Púb.\textunderscore , I, 183 e 185.
(B. Lat. \textunderscore allodium\textunderscore )
\section{Alloé}
\begin{itemize}
\item {Grp. gram.:m.}
\end{itemize}
\begin{itemize}
\item {Proveniência:(Do gr. \textunderscore alloios\textunderscore )}
\end{itemize}
Insecto hymenóptero.
\section{Allógeno}
\begin{itemize}
\item {Grp. gram.:adj.}
\end{itemize}
\begin{itemize}
\item {Proveniência:(Do gr. \textunderscore allos\textunderscore  + \textunderscore genos\textunderscore )}
\end{itemize}
Que é de outra raça.
\section{Allógono}
\begin{itemize}
\item {Grp. gram.:adj.}
\end{itemize}
\begin{itemize}
\item {Proveniência:(Do gr. \textunderscore allos\textunderscore  + \textunderscore gonos\textunderscore )}
\end{itemize}
Diz-se de um crystal, que reúne á fórma de nó a de um decaédro de triângulos escalenos, dos quaes câda um tem o seu ângulo plano obtuso igual á maior incidência das faces do nó.
\section{Alloíte}
\begin{itemize}
\item {Grp. gram.:f.}
\end{itemize}
Variedade de pozolana.
\section{Allom!}
\begin{itemize}
\item {Grp. gram.:interj.}
\end{itemize}
\begin{itemize}
\item {Proveniência:(Fr. \textunderscore allons\textunderscore )}
\end{itemize}
Vamos!
Adeante! Cf. Rebello, \textunderscore Mocidade\textunderscore , III, 96; e Garção, II, 62.
\section{Allomorphia}
\begin{itemize}
\item {Grp. gram.:f.}
\end{itemize}
\begin{itemize}
\item {Proveniência:(Do gr. \textunderscore allos\textunderscore  + \textunderscore morphe\textunderscore )}
\end{itemize}
Passagem de uma fórma para outra, diversa.
Metamorphose.
\section{Allomórphico}
\begin{itemize}
\item {Grp. gram.:adj.}
\end{itemize}
\begin{itemize}
\item {Proveniência:(De \textunderscore allomorphia\textunderscore )}
\end{itemize}
Que tem fórma diversa.
\section{Allomorphite}
\begin{itemize}
\item {Grp. gram.:f.}
\end{itemize}
\begin{itemize}
\item {Proveniência:(De \textunderscore allomorphia\textunderscore )}
\end{itemize}
Variedade de sulfato de barita.
\section{Allonga}
\begin{itemize}
\item {Grp. gram.:f.}
\end{itemize}
\begin{itemize}
\item {Utilização:Ant.}
\end{itemize}
O longo, a margem.
(Contr. de \textunderscore a\textunderscore  + \textunderscore la\textunderscore  + \textunderscore longa\textunderscore )
\section{Allónymo}
\begin{itemize}
\item {Grp. gram.:m.}
\end{itemize}
\begin{itemize}
\item {Proveniência:(Do gr. \textunderscore allos\textunderscore  + \textunderscore onuma\textunderscore )}
\end{itemize}
Aquelle que se serve do nome de outrem, assinando.
\section{Allopatha}
\begin{itemize}
\item {Grp. gram.:m.}
\end{itemize}
\begin{itemize}
\item {Proveniência:(Do gr. \textunderscore allos\textunderscore  + \textunderscore pathos\textunderscore )}
\end{itemize}
Aquelle que exerce a allopathia.--A pronúncia exacta seria \textunderscore alopáta\textunderscore .
\section{Allopathia}
\begin{itemize}
\item {Grp. gram.:f.}
\end{itemize}
\begin{itemize}
\item {Proveniência:(De \textunderscore allopatha\textunderscore )}
\end{itemize}
Systema commum de medicina, que combate as doenças por meios contrários a estas.
\section{Allopáthicamente}
\begin{itemize}
\item {Grp. gram.:adv.}
\end{itemize}
Segundo o systema \textunderscore allopáthico\textunderscore .
\section{Allopáthico}
\begin{itemize}
\item {Grp. gram.:adv.}
\end{itemize}
Relativo á \textunderscore allopathia\textunderscore .
\section{Allophana}
\begin{itemize}
\item {Grp. gram.:f.}
\end{itemize}
O mesmo que \textunderscore allophânio\textunderscore .
\section{Allophanato}
\begin{itemize}
\item {Grp. gram.:m.}
\end{itemize}
Sal, resultante da combinação do ácido allophânico com uma base.
\section{Allophânico}
\begin{itemize}
\item {Grp. gram.:adj.}
\end{itemize}
\begin{itemize}
\item {Proveniência:(De \textunderscore allophana\textunderscore )}
\end{itemize}
Diz-se de um ácido, que não existe, no estado livre.
\section{Allophânio}
\begin{itemize}
\item {Grp. gram.:m.}
\end{itemize}
\begin{itemize}
\item {Utilização:Miner.}
\end{itemize}
\begin{itemize}
\item {Proveniência:(Do gr. \textunderscore allos\textunderscore  + \textunderscore phaino\textunderscore )}
\end{itemize}
Silicato de aluminio hydratado.
\section{Alloplecto}
\begin{itemize}
\item {Grp. gram.:m.}
\end{itemize}
Gênero de plantas escrofularineas.
\section{Allóptero}
\begin{itemize}
\item {Grp. gram.:adj.}
\end{itemize}
\begin{itemize}
\item {Utilização:Ichthyol.}
\end{itemize}
\begin{itemize}
\item {Proveniência:(Do gr. \textunderscore allos\textunderscore , outro, e \textunderscore pleron\textunderscore  asa)}
\end{itemize}
Diz-se dos peixes, cujas barbatanas não têm posição fixa.
\section{Allotriologia}
\begin{itemize}
\item {Grp. gram.:f.}
\end{itemize}
\begin{itemize}
\item {Proveniência:(Do gr. \textunderscore allotrios\textunderscore  + \textunderscore logos\textunderscore )}
\end{itemize}
Applicação de doutrinas, estranhas ao assumpto occorrente.
\section{Allotriophagia}
\begin{itemize}
\item {Grp. gram.:f.}
\end{itemize}
\begin{itemize}
\item {Proveniência:(Do gr. \textunderscore allotrios\textunderscore  + \textunderscore phagein\textunderscore )}
\end{itemize}
Doença, caracterizada pela vontade de comer o que não sustenta ou o que é nocivo.
\section{Allotrióphago}
\begin{itemize}
\item {Grp. gram.:m.}
\end{itemize}
Aquelle que soffre \textunderscore allotriophagia\textunderscore .
\section{Allotriosmia}
\begin{itemize}
\item {Grp. gram.:f.}
\end{itemize}
\begin{itemize}
\item {Utilização:Med.}
\end{itemize}
\begin{itemize}
\item {Proveniência:(Do gr. \textunderscore allotrios\textunderscore  + \textunderscore osme\textunderscore )}
\end{itemize}
Vício de olfato, que consiste em sensações olfativas, paradoxaes.
\section{Allotróphico}
\begin{itemize}
\item {Grp. gram.:adj.}
\end{itemize}
\begin{itemize}
\item {Utilização:Neol.}
\end{itemize}
\begin{itemize}
\item {Proveniência:(Do gr. \textunderscore allos\textunderscore  + \textunderscore trophe\textunderscore )}
\end{itemize}
Que tem differente desenvolvimento.
\section{Allotropia}
\begin{itemize}
\item {Grp. gram.:f.}
\end{itemize}
Qualidade, que alguns corpos simples têm, de se apresentar em differentes estados, a que correspondem propriedades distintas.
(Cp. \textunderscore allótropo\textunderscore )
\section{Allotrópico}
\begin{itemize}
\item {Grp. gram.:adj.}
\end{itemize}
\begin{itemize}
\item {Proveniência:(De \textunderscore allotropia\textunderscore )}
\end{itemize}
O mesmo que \textunderscore allótropo\textunderscore .
\section{Allótropo}
\begin{itemize}
\item {Grp. gram.:adj.}
\end{itemize}
\begin{itemize}
\item {Utilização:Philol.}
\end{itemize}
\begin{itemize}
\item {Proveniência:(Do gr. \textunderscore allos\textunderscore  + \textunderscore tropos\textunderscore )}
\end{itemize}
Diz-se do corpo simples, em que se dá a allotropia.
Diz-se dos vocábulos divergentes, derivados de um só, como \textunderscore mancha\textunderscore , \textunderscore mágoa\textunderscore  e \textunderscore malha\textunderscore , do lat. \textunderscore macula\textunderscore .
\section{Alloxana}
\begin{itemize}
\item {Grp. gram.:f.}
\end{itemize}
\begin{itemize}
\item {Proveniência:(Al. \textunderscore alloxan\textunderscore )}
\end{itemize}
Substância, produzida pela acção do ácido azótico sôbre o ácido úrico.
\section{Alloxanato}
\begin{itemize}
\item {Grp. gram.:m.}
\end{itemize}
Combinação de álcalis com alloxana.
\section{Alloxantina}
\begin{itemize}
\item {Grp. gram.:f.}
\end{itemize}
Producto chímico, resultante da acção do ácido azótico sôbre o ácido úrico.
\section{Alluandita}
\begin{itemize}
\item {Grp. gram.:f.}
\end{itemize}
Phosphato de manganés e de ferro.
\section{Alludir}
\begin{itemize}
\item {Grp. gram.:v. i.}
\end{itemize}
\begin{itemize}
\item {Proveniência:(Lat. \textunderscore alludere\textunderscore )}
\end{itemize}
Referir-se indirectamente, vagamente.
Fazer referência.
\section{Allur}
\begin{itemize}
\item {Grp. gram.:adv.}
\end{itemize}
\begin{itemize}
\item {Utilização:Ant.}
\end{itemize}
O mesmo que \textunderscore alhur\textunderscore .
\section{Allusão}
\begin{itemize}
\item {Grp. gram.:f.}
\end{itemize}
\begin{itemize}
\item {Proveniência:(Lat. \textunderscore allusio\textunderscore )}
\end{itemize}
Acto de \textunderscore alludir\textunderscore .
Referência indirecta, vaga.
\section{Allusivamente}
\begin{itemize}
\item {Grp. gram.:adv.}
\end{itemize}
De modo \textunderscore allusivo\textunderscore .
\section{Allusivo}
\begin{itemize}
\item {Grp. gram.:adj.}
\end{itemize}
Que envolve allusão.
Que diz respeito a alguma coisa.
\section{Alluvial}
\begin{itemize}
\item {Grp. gram.:adj.}
\end{itemize}
Relativo a alluvião.
Formado por alluvião.
\section{Alluviano}
\begin{itemize}
\item {Grp. gram.:adj.}
\end{itemize}
Diz-se do terreno ou do depósito, formado por alluvião.
\section{Alluvião}
\begin{itemize}
\item {Grp. gram.:f.}
\end{itemize}
\begin{itemize}
\item {Utilização:Fig.}
\end{itemize}
\begin{itemize}
\item {Proveniência:(Lat. \textunderscore alluvio\textunderscore )}
\end{itemize}
Inundação.
Enxurrada.
Grande quantidade, ou grande número.
O mesmo que terreno alluviano.
\section{Alma}
\begin{itemize}
\item {Grp. gram.:f.}
\end{itemize}
\begin{itemize}
\item {Proveniência:(Do lat. \textunderscore anima\textunderscore )}
\end{itemize}
Essência immaterial da vida humana: \textunderscore a alma é immortal\textunderscore .
Conjunto das faculdades intellectuaes e moraes do homem.
Índole: \textunderscore alma bem formada\textunderscore .
Vida.
Pessôa: \textunderscore aldeia de cem almas\textunderscore .
Caudilho: \textunderscore é a alma do partido\textunderscore .
Espírito humano.
Colorido, animação: \textunderscore cantar com alma\textunderscore .
Coragem: \textunderscore não tens alma de lhe bater\textunderscore .
Enthusiasmo.
Interior de boca de fogo.
Peça de madeira, na rabeca, por baixo do cavalete.
Parte do carril, entre a cabeça e a patilha.
Pequeno pedaço de cabedal, entre a sola e a palmilha do sapato ou bota.
O mesmo que \textunderscore bomba\textunderscore ^1 de uma escada.
Parte média e mais estreita de uma viga de ferro, com duas partes extremas transversaes.
\section{Almacave}
\begin{itemize}
\item {Grp. gram.:m.}
\end{itemize}
(V.almocave)
\section{Almácega}
\begin{itemize}
\item {Grp. gram.:f.}
\end{itemize}
Pequeno tanque, para receber água da nora ou da chuva.
Casta de uva branca da região do Doiro.
(Ár. \textunderscore al-maçtaca\textunderscore )
\section{Almaço}
\begin{itemize}
\item {Grp. gram.:adj.}
\end{itemize}
\begin{itemize}
\item {Grp. gram.:M.}
\end{itemize}
\begin{itemize}
\item {Proveniência:(De \textunderscore a lo-maço\textunderscore  &lt; \textunderscore al maço\textunderscore  &lt; \textunderscore almaço\textunderscore , por allusão ao modo de se fabricar aquelle papel)}
\end{itemize}
Diz-se de uma espécie de papel grosso.
Papel almaço.
\section{Alma-de-cântaro}
\begin{itemize}
\item {Grp. gram.:f.}
\end{itemize}
\begin{itemize}
\item {Utilização:Pop.}
\end{itemize}
Alma damnada.
\section{Alma-de-canto}
\begin{itemize}
\item {Grp. gram.:f.}
\end{itemize}
Alma insensível, de pedra, o mesmo que \textunderscore alma-de-cântaro\textunderscore . Cf. Filinto, V, 118; e Lusíadas, I, 91.
\section{Alma-de-gato}
\begin{itemize}
\item {Grp. gram.:m.}
\end{itemize}
Ave do Brasil.
\section{Alma-de-mestre}
\begin{itemize}
\item {Grp. gram.:m.}
\end{itemize}
\begin{itemize}
\item {Utilização:marit.}
\end{itemize}
\begin{itemize}
\item {Utilização:Gír.}
\end{itemize}
Fracalhão. Cf. Garrett, \textunderscore Camões\textunderscore , 254.
\section{Almadena}
\begin{itemize}
\item {Grp. gram.:f.}
\end{itemize}
\begin{itemize}
\item {Proveniência:(Do ár. \textunderscore al-madin\textunderscore )}
\end{itemize}
O mesmo que \textunderscore minarete\textunderscore .
\section{Almadia}
\begin{itemize}
\item {Grp. gram.:f.}
\end{itemize}
\begin{itemize}
\item {Proveniência:(Do ár. \textunderscore al-madia\textunderscore )}
\end{itemize}
Embarcação africana, estreita e muito comprida, feita geralmente de um tronco de árvore.
\section{Almadrabilha}
\begin{itemize}
\item {Grp. gram.:f.}
\end{itemize}
O mesmo que \textunderscore almadrava\textunderscore . Cf. Ortigão, \textunderscore Culto da Arte\textunderscore .
\section{Almadraque}
\begin{itemize}
\item {Grp. gram.:m.}
\end{itemize}
\begin{itemize}
\item {Utilização:Ant.}
\end{itemize}
Enxêrga.
Coxim.
Colchão grosseiro de palha.
(\textunderscore T. cast.\textunderscore )
\section{Almadraquexa}
\begin{itemize}
\item {Grp. gram.:f.}
\end{itemize}
\begin{itemize}
\item {Utilização:Ant.}
\end{itemize}
\begin{itemize}
\item {Proveniência:(De \textunderscore almadraque\textunderscore )}
\end{itemize}
Cabeçal; travesseiro.
\section{Almadrava}
\begin{itemize}
\item {Grp. gram.:f.}
\end{itemize}
\begin{itemize}
\item {Proveniência:(Do ár. \textunderscore al-madraba\textunderscore )}
\end{itemize}
Pescaria de atum.
Lugar, onde se reúnem pescadores de atum.
Aparelho da pesca de atum.
\section{Almáfega}
\begin{itemize}
\item {Grp. gram.:f.}
\end{itemize}
\begin{itemize}
\item {Utilização:Ant.}
\end{itemize}
\begin{itemize}
\item {Proveniência:(Do ár. \textunderscore al-mirfaca\textunderscore ?)}
\end{itemize}
Burel branco, que servia para luto.
\section{Almáfego}
\begin{itemize}
\item {Grp. gram.:m.}
\end{itemize}
Casta de uva branca dos districtos de Leiria, Santarém e Lisbôa.
\section{Almafre}
\begin{itemize}
\item {Grp. gram.:m.}
\end{itemize}
\begin{itemize}
\item {Utilização:Ant.}
\end{itemize}
\begin{itemize}
\item {Proveniência:(Do ár. \textunderscore al-migfar\textunderscore )}
\end{itemize}
Parte da armadura, que cobria a cabeça.
\section{Almafreixe}
\begin{itemize}
\item {Grp. gram.:m.}
\end{itemize}
\begin{itemize}
\item {Utilização:Ant.}
\end{itemize}
\begin{itemize}
\item {Proveniência:(Do ár. \textunderscore al-mafrex\textunderscore )}
\end{itemize}
Grande mala de viagem.
\section{Almagege}
\begin{itemize}
\item {Grp. gram.:m.}
\end{itemize}
\begin{itemize}
\item {Utilização:Ant.}
\end{itemize}
O mesmo que \textunderscore tanque\textunderscore .
\section{Almagesto}
\begin{itemize}
\item {Grp. gram.:m.}
\end{itemize}
Compilação de observações astronómicas, feitas por astrónomos antigos.
(Ár. \textunderscore al-magisti\textunderscore , do gr. \textunderscore magiste\textunderscore )
\section{Almagra}
\begin{itemize}
\item {Grp. gram.:f.}
\end{itemize}
(V.almagre)
\section{Almagrado}
\begin{itemize}
\item {Grp. gram.:adj.}
\end{itemize}
Tinto de almagre.
\section{Almagrar}
\begin{itemize}
\item {Grp. gram.:v. t.}
\end{itemize}
\begin{itemize}
\item {Utilização:Prov.}
\end{itemize}
\begin{itemize}
\item {Utilização:trasm.}
\end{itemize}
Tingir com almagre.
Marcar.
Polir.
Encher de nódoas com pancadas (o corpo de alguém).
\section{Almagre}
\begin{itemize}
\item {Grp. gram.:m.}
\end{itemize}
\begin{itemize}
\item {Proveniência:(Do ár. \textunderscore al-magra\textunderscore )}
\end{itemize}
Terra avermelhada, que se emprega em algumas indústrias e em pinturas grosseiras.
\section{Almagro}
\begin{itemize}
\item {Grp. gram.:m.}
\end{itemize}
(V.almagre)
\section{Almaínha}
\begin{itemize}
\item {Grp. gram.:f.}
\end{itemize}
\begin{itemize}
\item {Utilização:Ant.}
\end{itemize}
\begin{itemize}
\item {Proveniência:(Do ár. \textunderscore al-munia\textunderscore )}
\end{itemize}
Quintal cerrado.
Quinta suburbana.
Também se dizia \textunderscore almoínha\textunderscore .
\section{Almalhão}
\begin{itemize}
\item {Grp. gram.:m.}
\end{itemize}
\begin{itemize}
\item {Utilização:Ant.}
\end{itemize}
O mesmo que \textunderscore almalho\textunderscore .
\section{Almalho}
\begin{itemize}
\item {Grp. gram.:m.}
\end{itemize}
\begin{itemize}
\item {Utilização:Pop.}
\end{itemize}
\begin{itemize}
\item {Proveniência:(Do lat. \textunderscore animalculum\textunderscore )}
\end{itemize}
Bezerro, novilho.
\section{Almanaque}
\begin{itemize}
\item {Grp. gram.:m.}
\end{itemize}
Calendário.
Livrinho ou livro que, além do calendário, tem diversas indicações uteis, ou trechos de literatura, ou uma e outra coisa.
\textunderscore Almanach\textunderscore  e \textunderscore almanak\textunderscore  ou \textunderscore almanack\textunderscore  não são fórmas portuguesas.
(Ár. \textunderscore almanak\textunderscore )
\section{Almança}
\begin{itemize}
\item {Grp. gram.:f.}
\end{itemize}
Escudo, dividido em pala, nos brasões.
\section{Almandia}
\begin{itemize}
\item {Grp. gram.:f.}
\end{itemize}
O mesmo que \textunderscore almadia\textunderscore .
\section{Almandina}
\begin{itemize}
\item {Grp. gram.:f.}
\end{itemize}
Pedra preciosa, variedade vermelha de granada.
\section{Almandra}
\begin{itemize}
\item {Grp. gram.:f.}
\end{itemize}
\begin{itemize}
\item {Utilização:Ant.}
\end{itemize}
Colcha; alcatifa.
Coberta de linho cru, bordada a seda.
(Cp. \textunderscore almatrixa\textunderscore )
\section{Almandrilha}
\begin{itemize}
\item {Grp. gram.:f.}
\end{itemize}
Espécie de conta alongada, usada como enfeite por alguns povos africanos; avelório.
\section{Alma-negra}
\begin{itemize}
\item {Grp. gram.:f.}
\end{itemize}
\begin{itemize}
\item {Utilização:Mad}
\end{itemize}
Nome de uma ave, (\textunderscore bulweria Bulweri\textunderscore ), o mesmo que \textunderscore anjinho\textunderscore .
\section{Almânia}
\begin{itemize}
\item {Grp. gram.:f.}
\end{itemize}
\begin{itemize}
\item {Proveniência:(De \textunderscore Almann\textunderscore , n. p.)}
\end{itemize}
Gênero de plantas, da fam. das amarantáceas, e originária da Índia.
\section{Almanicha}
\begin{itemize}
\item {Grp. gram.:m.}
\end{itemize}
\begin{itemize}
\item {Utilização:Prov.}
\end{itemize}
\begin{itemize}
\item {Utilização:trasm.}
\end{itemize}
Homem desalmado.
Alma pequena.
Brejeiro; vadio.
\section{Almanjarra}
\begin{itemize}
\item {Grp. gram.:f.}
\end{itemize}
\begin{itemize}
\item {Grp. gram.:M.}
\end{itemize}
\begin{itemize}
\item {Proveniência:(Do ár. \textunderscore al-majarr\textunderscore )}
\end{itemize}
Pau, a que se atrela o animal que faz andar a atafona ou a nora.
Espécie de grande rôdo, com que se tira a lama das marinhas.
\section{Almanjarrar}
\begin{itemize}
\item {Grp. gram.:v. t.}
\end{itemize}
Tirar com o almanjarra.
\section{Almânnia}
\begin{itemize}
\item {Grp. gram.:f.}
\end{itemize}
\begin{itemize}
\item {Proveniência:(De \textunderscore Almann\textunderscore , n. p.)}
\end{itemize}
Gênero de plantas, da fam. das amarantáceas, e originária da Índia.
\section{Almanxar}
\begin{itemize}
\item {Grp. gram.:m.}
\end{itemize}
\begin{itemize}
\item {Utilização:Prov.}
\end{itemize}
Lugar, onde se secam figos.
(Ár. \textunderscore almanxar\textunderscore )
\section{Alma-parens}
\begin{itemize}
\item {fónica:pá}
\end{itemize}
\begin{itemize}
\item {Grp. gram.:f.}
\end{itemize}
\begin{itemize}
\item {Utilização:Fig.}
\end{itemize}
A pátria.
(Loc. lat.)
\section{Almar}
\begin{itemize}
\item {Grp. gram.:m.}
\end{itemize}
\begin{itemize}
\item {Utilização:Des.}
\end{itemize}
O mesmo que \textunderscore almário\textunderscore , tenda, quitanda.
\section{Almarada}
\begin{itemize}
\item {Grp. gram.:f.}
\end{itemize}
\begin{itemize}
\item {Utilização:Ant.}
\end{itemize}
\begin{itemize}
\item {Proveniência:(Do ár. \textunderscore al-macraz\textunderscore )}
\end{itemize}
Espécie de punhal triangular.
\section{Almarado}
\begin{itemize}
\item {Grp. gram.:adj.}
\end{itemize}
Diz-se do toiro, que tem em volta dos olhos uma circunferência de côr differente da do resto da cabeça.
\section{Almaraz}
\begin{itemize}
\item {Grp. gram.:m.}
\end{itemize}
\begin{itemize}
\item {Proveniência:(Do ár. \textunderscore al-maraje\textunderscore , lugar por onde se sobe)}
\end{itemize}
Nome, com que se designa a orla esquerda do Tejo, desde Cacilhas á Trafaria, \textunderscore ou\textunderscore , antes, a encosta que se eleva daquelle lado do Tejo.
\section{Almárcova}
\begin{itemize}
\item {Grp. gram.:f.}
\end{itemize}
\begin{itemize}
\item {Utilização:Ant.}
\end{itemize}
Especie de cutello:«\textunderscore ...um homem de pé, carniceiro de Lisbôa, lhe deu com uma almarcova na mão do cavallo, o qual caiu logo\textunderscore ». Fern. Lopes, \textunderscore Chrón. de D. Fernando\textunderscore , c. XXXVIII. Cf. Herculano, \textunderscore Lendas\textunderscore , I, 130.
\section{Almárfega}
\begin{itemize}
\item {Grp. gram.:f.}
\end{itemize}
O mesmo que \textunderscore almáfega\textunderscore .
\section{Almarge}
\textunderscore f.\textunderscore  (e der.)
(V. \textunderscore almarje\textunderscore , etc.)
\section{Almário}
\begin{itemize}
\item {Grp. gram.:m.}
\end{itemize}
(V.armário)
\section{Almarjal}
\begin{itemize}
\item {Grp. gram.:m.}
\end{itemize}
Terreno apaulado, que tem almarjem.
\section{Almarje}
\begin{itemize}
\item {Grp. gram.:f.}
\end{itemize}
O mesmo que \textunderscore almarjem\textunderscore .
\section{Almarjeado}
\begin{itemize}
\item {Grp. gram.:adj.}
\end{itemize}
\begin{itemize}
\item {Utilização:Prov.}
\end{itemize}
Diz-se do terreno que, embora cultivado, só produz ervas.
\section{Almarjeal}
\begin{itemize}
\item {Grp. gram.:m.}
\end{itemize}
(V. almarjal)
\section{Almarjem}
\begin{itemize}
\item {Grp. gram.:f.}
\end{itemize}
Erva para pasto.
Prado natural.
Pastagem.
(Cast. \textunderscore almarjo\textunderscore )
\section{Almarjio}
\begin{itemize}
\item {Grp. gram.:adj.}
\end{itemize}
Lançado ao almarjem; que anda no almarjem.
\section{Almarraxa}
\begin{itemize}
\item {Grp. gram.:f.}
\end{itemize}
\begin{itemize}
\item {Utilização:Ant.}
\end{itemize}
\begin{itemize}
\item {Proveniência:(Do ár. \textunderscore al-niráxexa\textunderscore )}
\end{itemize}
Vaso, com orifícios no bojo para borrifar.
\section{Almassa}
\begin{itemize}
\item {Grp. gram.:f.}
\end{itemize}
Casta de uva branca da Arruda.
\section{Almástiga}
\begin{itemize}
\item {Grp. gram.:f.}
\end{itemize}
\begin{itemize}
\item {Utilização:Prov.}
\end{itemize}
\begin{itemize}
\item {Utilização:alent.}
\end{itemize}
Alfobre.
(Da mesma or. que \textunderscore almácega\textunderscore ^1)
\section{Almatrixa}
\begin{itemize}
\item {Grp. gram.:f.}
\end{itemize}
\begin{itemize}
\item {Utilização:Prov.}
\end{itemize}
\begin{itemize}
\item {Utilização:alent.}
\end{itemize}
Almofada embastada, que se põe sobre o albardão.
Cobertura de pelles, com que se apparelham burros e que é ligada por uma silha grosseira de baraço.
(Cp. \textunderscore almadraquexa\textunderscore )
\section{Almazém}
\begin{itemize}
\item {Grp. gram.:m.}
\end{itemize}
Fórma pop. e mais exacta que \textunderscore armazem\textunderscore . Cf. \textunderscore Ethiópia Or.\textunderscore , II, 323; \textunderscore Carta de Guia de Casados\textunderscore , 21; \textunderscore Peregrinação\textunderscore , LXXXVIII; \textunderscore Livro das Monções\textunderscore , n.^o 13.
(Cp. cast. \textunderscore almacén\textunderscore )
\section{Almazona}
\begin{itemize}
\item {Grp. gram.:f.}
\end{itemize}
(Corr. de \textunderscore amazona\textunderscore )
\section{Almece}
\begin{itemize}
\item {Proveniência:(Do ár. \textunderscore al-meice\textunderscore )}
\end{itemize}
(\textunderscore Prov. alent.\textunderscore  e \textunderscore açor.\textunderscore )
Sôro branco, que escorre do queijo de cabras, na primeira pressão, e que, misturado com pedacinhos de coalhada e algum leite, é fervido num tacho para se comer.
\section{Almécega}
\begin{itemize}
\item {Grp. gram.:f.}
\end{itemize}
Resina de lentisco.
Mastique.
Goma do Brasil.
(Do ár.?)
\section{Almecegar}
\begin{itemize}
\item {fónica:mé}
\end{itemize}
\begin{itemize}
\item {Grp. gram.:v. t.}
\end{itemize}
Dar côr de almécega, amarelada, a.
Applicar a almécega a.
\section{Almecegueira}
\begin{itemize}
\item {fónica:mé}
\end{itemize}
\begin{itemize}
\item {Grp. gram.:f.}
\end{itemize}
Arbusto terebintháceo, (\textunderscore pistacia lentiscus\textunderscore ).
\section{Almecibuçu}
\begin{itemize}
\item {Grp. gram.:m.}
\end{itemize}
\begin{itemize}
\item {Utilização:Bras}
\end{itemize}
Árvore silvestre, que dá bôa madeira para construcções.
\section{Almedina}
\begin{itemize}
\item {Grp. gram.:f.}
\end{itemize}
\begin{itemize}
\item {Utilização:Ant.}
\end{itemize}
A parte mais antiga de uma cidade, em sítio elevado e com castelo: \textunderscore em Coimbra, uma das entradas para o bairro alto inda conserva o nome de\textunderscore  "Porta de Almedina". Cf. Herculano, \textunderscore Hist. de Port.\textunderscore , II, 32.
\section{Almegue}
\begin{itemize}
\item {Grp. gram.:m.}
\end{itemize}
Lodeiro? ervaçal?:«\textunderscore ajuntayuos ambas no estendedouro contra o pegado almegue.\textunderscore »\textunderscore Eufrosina\textunderscore , 104.
\section{Almeia}
\begin{itemize}
\item {Grp. gram.:f.}
\end{itemize}
\begin{itemize}
\item {Proveniência:(Do ár. \textunderscore alemia\textunderscore )}
\end{itemize}
Dançarina indiana, destra no canto e na poesia.
\section{Almeia}
\begin{itemize}
\item {Grp. gram.:f.}
\end{itemize}
\begin{itemize}
\item {Proveniência:(Do ár. \textunderscore almeia\textunderscore )}
\end{itemize}
Bálsamo natural, produzido no Oriente e preparado em Marselha.
\section{Almeida}
\begin{itemize}
\item {Grp. gram.:f.}
\end{itemize}
\begin{itemize}
\item {Utilização:Náut.}
\end{itemize}
Abertura, por onde entra a cana do leme.
(Relaciona-se com o ár. \textunderscore al-madin\textunderscore , mina?)
\section{Almêidea}
\begin{itemize}
\item {Grp. gram.:f.}
\end{itemize}
Gênero de plantas rutáceas da América.
\section{Almeidina}
\begin{itemize}
\item {Grp. gram.:f.}
\end{itemize}
Borracha branca de Mossâmedes.
\section{Almeirão}
\begin{itemize}
\item {Grp. gram.:m.}
\end{itemize}
\begin{itemize}
\item {Proveniência:(Do ár. \textunderscore al-miron\textunderscore )}
\end{itemize}
Planta hortense, espécie de chicória, (\textunderscore chicoreum intybus\textunderscore ).
\section{Almeirôa}
\begin{itemize}
\item {Grp. gram.:f.}
\end{itemize}
Planta, semelhante ao almeirão.
\section{Almeitiga}
\begin{itemize}
\item {Grp. gram.:f.}
\end{itemize}
\begin{itemize}
\item {Utilização:Ant.}
\end{itemize}
Almôço, ou refeição, que se dava ao cobrador dos foros reaes.
\section{Almeixar}
\begin{itemize}
\item {Grp. gram.:m.}
\end{itemize}
\begin{itemize}
\item {Utilização:Prov.}
\end{itemize}
\begin{itemize}
\item {Utilização:alg.}
\end{itemize}
O mesmo que \textunderscore almanxar\textunderscore .
\section{Almeixiar}
\begin{itemize}
\item {Grp. gram.:m.}
\end{itemize}
O mesmo que \textunderscore almanxar\textunderscore .
\section{Almeizar}
\begin{itemize}
\item {Grp. gram.:m.}
\end{itemize}
Toalha moirisca, usada nas mesas de alguns dos antigos reis de Portugal.
O mesmo que \textunderscore almiazar\textunderscore .
\section{Almejante}
\begin{itemize}
\item {Grp. gram.:adj.}
\end{itemize}
Que almeja.
\section{Almejar}
\begin{itemize}
\item {Grp. gram.:v. t.}
\end{itemize}
\begin{itemize}
\item {Grp. gram.:V. i.}
\end{itemize}
\begin{itemize}
\item {Proveniência:(De \textunderscore alma\textunderscore )}
\end{itemize}
Desejar com ânsia.
Dar a alma, agonizar.
\section{Almejo}
\begin{itemize}
\item {Grp. gram.:m.}
\end{itemize}
Acto de \textunderscore almejar\textunderscore . Cf. Castilho, \textunderscore Fausto\textunderscore , 111.
\section{Almena}
\begin{itemize}
\item {Grp. gram.:m.}
\end{itemize}
Pêso indiano, equivalente a 1 kilogramma aproximadamente.
(Ár. \textunderscore al-mena\textunderscore )
\section{Almenara}
\begin{itemize}
\item {Grp. gram.:f.}
\end{itemize}
Farol ou facho, que se accendia nas tôrres e castellos para dar sinal ao longe.--Fórma exacta, em vez de \textunderscore minarete\textunderscore .
(Ár. \textunderscore al-menara\textunderscore )
\section{Almendrilhas}
\begin{itemize}
\item {Grp. gram.:f. pl.}
\end{itemize}
\begin{itemize}
\item {Utilização:Prov.}
\end{itemize}
\begin{itemize}
\item {Utilização:trasm.}
\end{itemize}
Arrecadas, brincos.
(Cp. \textunderscore almandrilha\textunderscore )
\section{Almenhaba}
\begin{itemize}
\item {Grp. gram.:f.}
\end{itemize}
Espécie de uva grande e preta.
\section{Almenilha}
\begin{itemize}
\item {Grp. gram.:f.}
\end{itemize}
\begin{itemize}
\item {Utilização:Ant.}
\end{itemize}
Certo ornato que se usava nos vestidos.
(Cast. \textunderscore almenilla\textunderscore )
\section{Almexia}
\begin{itemize}
\item {Grp. gram.:f.}
\end{itemize}
\begin{itemize}
\item {Utilização:Ant.}
\end{itemize}
\begin{itemize}
\item {Proveniência:(Do ár. \textunderscore almahxia\textunderscore )}
\end{itemize}
Espécie de túnica, que cobria o vestuário vulgar.
\section{Almiazar}
\begin{itemize}
\item {Grp. gram.:m.}
\end{itemize}
\begin{itemize}
\item {Utilização:Ant.}
\end{itemize}
\begin{itemize}
\item {Proveniência:(Do cast. \textunderscore almaizar\textunderscore )}
\end{itemize}
Véu com franjas, para ornato de altares.
\section{Almiara}
\begin{itemize}
\item {Grp. gram.:f.}
\end{itemize}
\begin{itemize}
\item {Utilização:Prov.}
\end{itemize}
\begin{itemize}
\item {Utilização:alent.}
\end{itemize}
\begin{itemize}
\item {Proveniência:(Do ár. \textunderscore almiiar\textunderscore )}
\end{itemize}
Mêda de trigo.
\section{Almiça}
\begin{itemize}
\item {Grp. gram.:f.}
\end{itemize}
O mesmo que \textunderscore almece\textunderscore .
\section{Almice}
\begin{itemize}
\item {Grp. gram.:m.}
\end{itemize}
O mesmo que \textunderscore almece\textunderscore .
\section{Almilha}
\begin{itemize}
\item {Grp. gram.:f.}
\end{itemize}
\begin{itemize}
\item {Utilização:Ant.}
\end{itemize}
Peça de vestuário, que se usava entre a camisa e o gibão.
(Talvez dem. de \textunderscore alma\textunderscore )
\section{Alminha}
\begin{itemize}
\item {Grp. gram.:f.}
\end{itemize}
\begin{itemize}
\item {Utilização:Prov.}
\end{itemize}
\begin{itemize}
\item {Utilização:minh.}
\end{itemize}
O mealheiro das almas. Cf. Viana, \textunderscore Apostilas\textunderscore .
\section{Alminhaca}
\begin{itemize}
\item {Grp. gram.:f.}
\end{itemize}
Casta de uva branca algarvia.
\section{Alminhas}
\begin{itemize}
\item {Grp. gram.:f. pl.}
\end{itemize}
\begin{itemize}
\item {Utilização:Pop.}
\end{itemize}
\begin{itemize}
\item {Proveniência:(De \textunderscore alma\textunderscore )}
\end{itemize}
Painel, que representa as almas do Purgatório.
\section{Almiqui}
\begin{itemize}
\item {Grp. gram.:m.}
\end{itemize}
Mammifero roedor da ilha de Cuba.
\section{Almiranta}
\begin{itemize}
\item {Grp. gram.:f.}
\end{itemize}
\begin{itemize}
\item {Grp. gram.:Adj.}
\end{itemize}
Embarcação, que leva a bordo o almirante.
Mulher do almirante.
Relativo ao almirante:«\textunderscore a nau almiranta...\textunderscore »Camillo, \textunderscore Caveira\textunderscore , 80, 84, 86.
\section{Almirantado}
\begin{itemize}
\item {Grp. gram.:m.}
\end{itemize}
Dignidade, pôsto de almirante.
Corporação dos officiaes superíores da armada.
\section{Almirante}
\begin{itemize}
\item {Grp. gram.:m.}
\end{itemize}
\begin{itemize}
\item {Grp. gram.:F.}
\end{itemize}
\begin{itemize}
\item {Proveniência:(Do ár. \textunderscore amir\textunderscore  e um suff. obscuro)}
\end{itemize}
Official general da armada.
Navio, em que vai o almirante.
Variedade de pêra.
\section{Almirantear}
\begin{itemize}
\item {Grp. gram.:v. i.}
\end{itemize}
Fazer offício de almirante.
\section{Almíscar}
\begin{itemize}
\item {Grp. gram.:m.}
\end{itemize}
\begin{itemize}
\item {Proveniência:(De \textunderscore almiscre\textunderscore )}
\end{itemize}
Substância aromática, segregada do almiscareiro.
Nome de uma planta de estufa, o mesmo que \textunderscore almiscareira\textunderscore .
\section{Almiscarado}
\begin{itemize}
\item {Grp. gram.:adj.}
\end{itemize}
\begin{itemize}
\item {Utilização:Fig.}
\end{itemize}
Perfumado com almíscar.
Muito perfumado.
\section{Almiscarar}
\begin{itemize}
\item {Grp. gram.:v. t.}
\end{itemize}
Perfumar com almíscar.
\section{Almiscareira}
\begin{itemize}
\item {Grp. gram.:f.}
\end{itemize}
Planta geraniácea, de aroma semelhante ao do almíscar.
\section{Almiscareiro}
\begin{itemize}
\item {Grp. gram.:m.}
\end{itemize}
Animal asiatico, da ordem dos ruminantes, que tem sob o ventre uma bôlsa natural, donde se extrai o almíscar.
\section{Almiscre}
\begin{itemize}
\item {Grp. gram.:m.}
\end{itemize}
\begin{itemize}
\item {Proveniência:(Do cast. \textunderscore almizcle\textunderscore )}
\end{itemize}
O mesmo que \textunderscore almíscar\textunderscore . Cf. Usque, \textunderscore Tribulações\textunderscore , 17.
\section{Almixar}
\begin{itemize}
\item {Grp. gram.:m.}
\end{itemize}
V. \textunderscore almanxar\textunderscore .
\section{Almízcar}
\begin{itemize}
\item {Grp. gram.:m.}
\end{itemize}
\begin{itemize}
\item {Proveniência:(De \textunderscore almiscre\textunderscore )}
\end{itemize}
Substância aromática, segregada do almiscareiro.
Nome de uma planta de estufa, o mesmo que \textunderscore almiscareira\textunderscore .
\section{Almo}
\begin{itemize}
\item {Grp. gram.:adj.}
\end{itemize}
\begin{itemize}
\item {Proveniência:(Lat. \textunderscore almus\textunderscore , por \textunderscore alimus\textunderscore , de \textunderscore alere\textunderscore )}
\end{itemize}
Que cria, que alimenta.
Bom, benigno.
Venerável.
\section{Almoahedes}
\begin{itemize}
\item {Grp. gram.:m. pl.}
\end{itemize}
O mesmo que \textunderscore almoravides\textunderscore .
\section{Almoçadeira}
\begin{itemize}
\item {Grp. gram.:f.}
\end{itemize}
Chícara grande de almôço.
\section{Almocadém}
\begin{itemize}
\item {Grp. gram.:m.}
\end{itemize}
\begin{itemize}
\item {Utilização:Ant.}
\end{itemize}
Commandante, chefe.
(Ár. \textunderscore al-mocaddem\textunderscore )
\section{Almocafre}
\begin{itemize}
\item {Grp. gram.:m.}
\end{itemize}
\begin{itemize}
\item {Proveniência:(T. cast.)}
\end{itemize}
Sacho de ponta, usado nas minas.
\section{Almoçar}
\begin{itemize}
\item {Grp. gram.:v. t.}
\end{itemize}
\begin{itemize}
\item {Grp. gram.:V. i.}
\end{itemize}
Comer ao almôço.
Tomar o almôço.
\section{Almocávar}
\begin{itemize}
\item {Grp. gram.:m.}
\end{itemize}
O mesmo que \textunderscore almocave\textunderscore .
\section{Almocave}
\begin{itemize}
\item {Grp. gram.:m.}
\end{itemize}
\begin{itemize}
\item {Utilização:Ant.}
\end{itemize}
\begin{itemize}
\item {Proveniência:(Do ár. \textunderscore al-macabir\textunderscore )}
\end{itemize}
Cemitério de moiros.
\section{Almocela}
\begin{itemize}
\item {Grp. gram.:f.}
\end{itemize}
\begin{itemize}
\item {Utilização:Ant.}
\end{itemize}
\begin{itemize}
\item {Proveniência:(Do ár. \textunderscore almozala\textunderscore )}
\end{itemize}
Espécie de capuz.
Cobertor ou manta.
\section{Almoceleiro}
\begin{itemize}
\item {Grp. gram.:m.}
\end{itemize}
Aquelle que fabricava almocelas.
\section{Almôço}
\begin{itemize}
\item {Grp. gram.:m.}
\end{itemize}
\begin{itemize}
\item {Utilização:Fig.}
\end{itemize}
Primeira refeição do dia, menos abundante que o jantar, e que, geralmente, se toma de manhan.
O primeiro acontecimento do dia.
(Contr. de \textunderscore almôrço\textunderscore . V. \textunderscore almôrço\textunderscore )
\section{Almocouvar}
\begin{itemize}
\item {Grp. gram.:m.}
\end{itemize}
Pastor, que, na guarda de um rebanho, tem posição immediatamente inferior á do maioral.
\section{Almocóvar}
\begin{itemize}
\item {Grp. gram.:m.}
\end{itemize}
\begin{itemize}
\item {Utilização:Ant.}
\end{itemize}
Cemitério de judeus.
(Cp. \textunderscore almocave\textunderscore )
\section{Almocrevar}
\begin{itemize}
\item {fónica:cré}
\end{itemize}
\begin{itemize}
\item {Grp. gram.:v. t.}
\end{itemize}
\begin{itemize}
\item {Grp. gram.:V. i.}
\end{itemize}
Transportar em bêstas de almocreve.
Exercer o offício de almocreve.
\section{Almocrevaria}
\begin{itemize}
\item {Grp. gram.:f.}
\end{itemize}
Fórma preferível a \textunderscore almocreveria\textunderscore . Cf. Herculano, \textunderscore Hist. de Port\textunderscore , IV, 418.
\section{Almocreve}
\begin{itemize}
\item {Grp. gram.:m.}
\end{itemize}
\begin{itemize}
\item {Proveniência:(Do ár. \textunderscore almoqueri\textunderscore )}
\end{itemize}
Aquelle que tem por offício conduzir bêstas de carga.
Recoveiro.
Carregador.
\section{Almocreveria}
\begin{itemize}
\item {Grp. gram.:f.}
\end{itemize}
\begin{itemize}
\item {Utilização:Ant.}
\end{itemize}
\begin{itemize}
\item {Proveniência:(De \textunderscore almocreve\textunderscore )}
\end{itemize}
Direito, que os almocreves pagavam pelo exercício da sua indústria.
\section{Almoéda}
\begin{itemize}
\item {Grp. gram.:f.}
\end{itemize}
\begin{itemize}
\item {Proveniência:(Do ár. \textunderscore al-monada\textunderscore )}
\end{itemize}
Venda em público, por arrematação; leilão.
Exposição ou offerta ao público.
\section{Almoedar}
\begin{itemize}
\item {fónica:mo-é}
\end{itemize}
\begin{itemize}
\item {Grp. gram.:v. t.}
\end{itemize}
Pôr em almoéda; vender em hasta pública.
\section{Almoedeiro}
\begin{itemize}
\item {fónica:mo-é}
\end{itemize}
\begin{itemize}
\item {Grp. gram.:m.}
\end{itemize}
O mesmo que \textunderscore leiloeiro\textunderscore . Cf. Camillo, \textunderscore Senh. do P. de Ninães\textunderscore , 168.
\section{Almofaça}
\begin{itemize}
\item {Grp. gram.:f.}
\end{itemize}
Escôva de ferro, para limpar bêstas.
(Cast. \textunderscore almohaza\textunderscore )
\section{Almofaçar}
\begin{itemize}
\item {Grp. gram.:v. t.}
\end{itemize}
Limpar com almofaça.
\section{Almofacilha}
\begin{itemize}
\item {Grp. gram.:f.}
\end{itemize}
Porção de estôpa, que se enrola na barbela, a fim de que o cavallo se não fira.
\section{Almofada}
\begin{itemize}
\item {Grp. gram.:f.}
\end{itemize}
\begin{itemize}
\item {Proveniência:(Do ár. \textunderscore al-micada\textunderscore )}
\end{itemize}
Espécie de saco, cheio de qualquer substância elástica ou molle, e que serve de encôsto, assento, e ornatos, bem como de apoio para as peças de fazenda em que se cose á mão, etc.
Peça de madeira, geralmente rectangular, que resai na face das portas e janelas.
Guarnição de madeira, nos navios, para evitar que os cabos, correndo, se cortem.
Variedade de maçan.
\section{Almofadado}
\begin{itemize}
\item {Grp. gram.:adj.}
\end{itemize}
\begin{itemize}
\item {Grp. gram.:M.}
\end{itemize}
\begin{itemize}
\item {Proveniência:(De \textunderscore almofadar\textunderscore )}
\end{itemize}
Que tem almofadas.
Feito á maneira de almofada.
As almofadas de uma obra de madeira ou pedra.
\section{Almofadar}
\begin{itemize}
\item {Grp. gram.:v. t.}
\end{itemize}
Cobrir, ornar, com almofadas.
Enchumaçar.
Sotopor a um objecto uma substância ou peça que o alteia.
\section{Almofala}
\begin{itemize}
\item {Grp. gram.:f.}
\end{itemize}
\begin{itemize}
\item {Utilização:Ant.}
\end{itemize}
\begin{itemize}
\item {Proveniência:(Do ár. \textunderscore al-mahalla\textunderscore )}
\end{itemize}
Campo, arraial, em que se reside algum tempo.
\section{Almofalla}
\begin{itemize}
\item {Grp. gram.:f.}
\end{itemize}
\begin{itemize}
\item {Utilização:Ant.}
\end{itemize}
\begin{itemize}
\item {Proveniência:(Do ár. \textunderscore al-mahalla\textunderscore )}
\end{itemize}
Campo, arraial, em que se reside algum tempo.
\section{Almofariz}
\begin{itemize}
\item {Grp. gram.:m.}
\end{itemize}
\begin{itemize}
\item {Proveniência:(T. cast.)}
\end{itemize}
Gral, vaso em que se pisa ou esmaga qualquer coisa com pilão.
\section{Almofate}
\begin{itemize}
\item {Grp. gram.:m.}
\end{itemize}
Espécie de furador ou de sovela, com que os correeiros abrem olhos na sola.
\section{Almofeira}
\begin{itemize}
\item {Grp. gram.:f.}
\end{itemize}
Líquido escuro, que escorre de azeitonas em talha.
Água de azeitonas; reima.
\section{Almofia}
\begin{itemize}
\item {Grp. gram.:f.}
\end{itemize}
\begin{itemize}
\item {Utilização:Prov.}
\end{itemize}
\begin{itemize}
\item {Utilização:beir.}
\end{itemize}
\begin{itemize}
\item {Utilização:Ant.}
\end{itemize}
\begin{itemize}
\item {Proveniência:(Do ár. \textunderscore almafia\textunderscore , caçoila)}
\end{itemize}
Espécie de tijela, de fundo largo e bordos quási perpendiculares.
\section{Almofreixe}
\begin{itemize}
\item {Grp. gram.:m.}
\end{itemize}
O mesmo que \textunderscore almafreixe\textunderscore .
\section{Almogama}
\begin{itemize}
\item {Grp. gram.:f.}
\end{itemize}
\begin{itemize}
\item {Utilização:Ant.}
\end{itemize}
\begin{itemize}
\item {Proveniência:(T. cast.)}
\end{itemize}
Última caverna da nau, onde os paus são mais juntos por causa do boleamento da prôa.
\section{Almogaure}
\begin{itemize}
\item {Grp. gram.:m.}
\end{itemize}
O mesmo que \textunderscore almogávar\textunderscore .
\section{Almogávar}
\begin{itemize}
\item {Grp. gram.:m.}
\end{itemize}
\begin{itemize}
\item {Utilização:Ant.}
\end{itemize}
\begin{itemize}
\item {Proveniência:(Do ár. \textunderscore al-mogauir\textunderscore )}
\end{itemize}
Guerreiro, que vivia nos matos, donde assaltava terras de moiros.
\section{Almogavaria}
\begin{itemize}
\item {Grp. gram.:f.}
\end{itemize}
Tropa, expedição, correría de almogávares.
\section{Almogavre}
\begin{itemize}
\item {Grp. gram.:m.}
\end{itemize}
O mesmo que \textunderscore almogávar\textunderscore . Cf. Herculano, \textunderscore Bobo\textunderscore , 60 e 210.
\section{Almoínha}
\begin{itemize}
\item {Grp. gram.:f.}
\end{itemize}
\begin{itemize}
\item {Utilização:Ant.}
\end{itemize}
Pequena fazenda; horta murada.
(Cp. \textunderscore almaínha\textunderscore )
\section{Almoinheiro}
\begin{itemize}
\item {Grp. gram.:m.}
\end{itemize}
\begin{itemize}
\item {Utilização:Ant.}
\end{itemize}
Cultivador de almoínha; hortelão.
\section{Almojávena}
\begin{itemize}
\item {Grp. gram.:f.}
\end{itemize}
\begin{itemize}
\item {Proveniência:(Do ár. \textunderscore al-mojábana\textunderscore )}
\end{itemize}
Espécie de belhó, de farinha e queijo.
\section{Almolina}
\begin{itemize}
\item {Grp. gram.:f.}
\end{itemize}
Jôgo antigo, espécie de cabra-cega. Cf. G. Vicente, I, 133.
\section{Almôndega}
\begin{itemize}
\item {Grp. gram.:f.}
\end{itemize}
\begin{itemize}
\item {Proveniência:(Do ár. \textunderscore al-bondoca\textunderscore )}
\end{itemize}
Bolo de carne picada, com ovos e outros adubos.
\section{Almonjava}
\begin{itemize}
\item {Grp. gram.:f.}
\end{itemize}
\begin{itemize}
\item {Utilização:Ant.}
\end{itemize}
Iguaria de carneiro picado, com toicinho e outros temperos.
(Corr. de \textunderscore almojâvena\textunderscore ?)
\section{Almoqueire}
\begin{itemize}
\item {Grp. gram.:m.}
\end{itemize}
\begin{itemize}
\item {Utilização:Ant.}
\end{itemize}
O mesmo que \textunderscore almocreve\textunderscore .
\section{Almorabitino}
\begin{itemize}
\item {Grp. gram.:m.}
\end{itemize}
Moéda moirisca. Cf. Herculano, \textunderscore Bobo\textunderscore , 128 e 165.
\section{Almoravides}
\begin{itemize}
\item {Grp. gram.:m. pl.}
\end{itemize}
\begin{itemize}
\item {Proveniência:(Do ár. \textunderscore almorábit\textunderscore , ermida)}
\end{itemize}
Seita religiosa e, ao depois, também política, entre os Árabes.
Últimos moiros, que dominaram na Espanha até á conquista de Granada pelos reis cathólicos.
\section{Almorçar}
\begin{itemize}
\item {Grp. gram.:v. t.  e  i.}
\end{itemize}
\begin{itemize}
\item {Utilização:Ant.}
\end{itemize}
\begin{itemize}
\item {Proveniência:(De \textunderscore almôrço\textunderscore )}
\end{itemize}
O mesmo que \textunderscore almoçar\textunderscore . Cf. Filinto, XIX, 32 e 69.
\section{Almôrço}
\begin{itemize}
\item {Grp. gram.:m.}
\end{itemize}
\begin{itemize}
\item {Utilização:Ant.}
\end{itemize}
O mesmo que \textunderscore almôço\textunderscore .
(Cast. \textunderscore almuerzo\textunderscore )
\section{Almorrans}
\begin{itemize}
\item {Grp. gram.:f. pl.}
\end{itemize}
O mesmo que \textunderscore almorreimas\textunderscore .
\section{Almorreimado}
\begin{itemize}
\item {Grp. gram.:adj.}
\end{itemize}
Que tem almorreimas.
\section{Almorreimas}
\begin{itemize}
\item {Grp. gram.:f. pl.}
\end{itemize}
\begin{itemize}
\item {Utilização:Pop.}
\end{itemize}
O mesmo que \textunderscore hemorroidas\textunderscore .
\section{Almorreta}
\begin{itemize}
\item {fónica:rê}
\end{itemize}
\begin{itemize}
\item {Grp. gram.:f.}
\end{itemize}
\begin{itemize}
\item {Utilização:Prov.}
\end{itemize}
\begin{itemize}
\item {Utilização:trasm.}
\end{itemize}
Miudezas fibrosas de porco, curtidas em vinho e alho.
\section{Almotaçadamente}
\begin{itemize}
\item {Grp. gram.:adv.}
\end{itemize}
\begin{itemize}
\item {Utilização:Ant.}
\end{itemize}
Segundo a taxa do almotacé.
\section{Almotaçar}
\begin{itemize}
\item {Grp. gram.:v. t.}
\end{itemize}
\begin{itemize}
\item {Utilização:Ant.}
\end{itemize}
\begin{itemize}
\item {Proveniência:(De \textunderscore almotacé\textunderscore )}
\end{itemize}
Taxar o preço de.
\section{Almotaçaria}
\begin{itemize}
\item {Grp. gram.:f.}
\end{itemize}
\begin{itemize}
\item {Utilização:Ant.}
\end{itemize}
Cargo de almotacé.
Fixação do preço dos gêneros alimentícios, feita pelo almotacé.
\section{Almotacé}
\begin{itemize}
\item {Grp. gram.:m.}
\end{itemize}
\begin{itemize}
\item {Utilização:Ant.}
\end{itemize}
\begin{itemize}
\item {Proveniência:(Do ár. \textunderscore al-mohtacib\textunderscore )}
\end{itemize}
Inspector de pesos e medidas, que taxava o preço dos gêneros alimentícios.
\section{Almotacel}
\begin{itemize}
\item {Grp. gram.:m.}
\end{itemize}
\begin{itemize}
\item {Utilização:Ant.}
\end{itemize}
\begin{itemize}
\item {Proveniência:(Do ár. \textunderscore al-mohtacib\textunderscore )}
\end{itemize}
Inspector de pesos e medidas, que taxava o preço dos gêneros alimentícios.
\section{Almotalia}
\begin{itemize}
\item {Grp. gram.:f.}
\end{itemize}
\begin{itemize}
\item {Utilização:Ant.}
\end{itemize}
O mesmo que \textunderscore almotolia\textunderscore . Cf. \textunderscore Eufrosina\textunderscore , 278.
\section{Almotolia}
\begin{itemize}
\item {Grp. gram.:f.}
\end{itemize}
\begin{itemize}
\item {Proveniência:(Do ár. \textunderscore al-motli\textunderscore )}
\end{itemize}
Pequeno vaso, de fórma cónica, para azeite e outros líquidos, principalmente oleosos.
\section{Almotriga}
\begin{itemize}
\item {Grp. gram.:f.}
\end{itemize}
\begin{itemize}
\item {Utilização:Prov.}
\end{itemize}
\begin{itemize}
\item {Utilização:trasm.}
\end{itemize}
\begin{itemize}
\item {Utilização:beir.}
\end{itemize}
O mesmo que \textunderscore almotolia\textunderscore .
\section{Almoxarifado}
\begin{itemize}
\item {Grp. gram.:m.}
\end{itemize}
Cargo de almoxarife; área da sua jurisdição.
\section{Almoxarife}
\begin{itemize}
\item {Grp. gram.:m.}
\end{itemize}
\begin{itemize}
\item {Utilização:Ant.}
\end{itemize}
\begin{itemize}
\item {Proveniência:(Do ár. \textunderscore al-moxrife\textunderscore )}
\end{itemize}
Administrador de propriedades da Casa Real.
Thesoireiro da Casa Real.
Cobrador de impostos de portagem.
\section{Almoxatre}
\begin{itemize}
\item {Grp. gram.:m.}
\end{itemize}
Nome antigo do sal ammoníaco.
\section{Almuádem}
\begin{itemize}
\item {Grp. gram.:m.}
\end{itemize}
\begin{itemize}
\item {Proveniência:(Do ár. \textunderscore al-muaddin\textunderscore )}
\end{itemize}
Moiro, que, do alto das almenaras, chama o povo á oração.
\section{Almucédia}
\begin{itemize}
\item {Grp. gram.:f.}
\end{itemize}
Uma das estrêllas da constellação da Virgem.
\section{Almudação}
\begin{itemize}
\item {Grp. gram.:f.}
\end{itemize}
Acto de \textunderscore almudar\textunderscore .
\section{Almudada}
\begin{itemize}
\item {Grp. gram.:f.}
\end{itemize}
\begin{itemize}
\item {Utilização:Ant.}
\end{itemize}
Almude de cereaes.
Terra, que leva um almude de semente.
O mesmo que \textunderscore almude\textunderscore .
\section{Almudar}
\begin{itemize}
\item {Grp. gram.:v. t.}
\end{itemize}
Medir aos almudes; encher aos almudes.
\section{Almude}
\begin{itemize}
\item {Grp. gram.:m.}
\end{itemize}
\begin{itemize}
\item {Proveniência:(Do ár. \textunderscore al-mudd\textunderscore )}
\end{itemize}
Medida de capacidade para líquidos, de 12 canadas ou 48 quartilhos.
Antiga medida de cereaes.
\section{Almudeiro}
\begin{itemize}
\item {Grp. gram.:adj.}
\end{itemize}
\begin{itemize}
\item {Utilização:Prov.}
\end{itemize}
Diz-se da vasilha, que tem a capacidade de um almude.
Diz-se da canastra, que póde conter as uvas necessárias para dar um almude de vinho.
\section{Alna}
\begin{itemize}
\item {Grp. gram.:f.}
\end{itemize}
\begin{itemize}
\item {Proveniência:(Do germ. \textunderscore alna\textunderscore ?)}
\end{itemize}
Antiga medida de comprimento.
\section{Alnite}
\begin{itemize}
\item {Grp. gram.:f.}
\end{itemize}
\begin{itemize}
\item {Proveniência:(Do lat. \textunderscore alnus\textunderscore )}
\end{itemize}
Vegetal fóssil, do gênero dos amieiros.
\section{Alno}
\begin{itemize}
\item {Grp. gram.:m.}
\end{itemize}
\begin{itemize}
\item {Proveniência:(Lat. \textunderscore alnus\textunderscore )}
\end{itemize}
Gênero de plantas betuláceas.
\section{Aló}
\begin{itemize}
\item {Grp. gram.:adv.}
\end{itemize}
\begin{itemize}
\item {Proveniência:(De \textunderscore a\textunderscore  + \textunderscore ló\textunderscore )}
\end{itemize}
Para barlavento: para a banda donde sopra o vento.
\section{Alobadado}
\begin{itemize}
\item {Grp. gram.:adj.}
\end{itemize}
\begin{itemize}
\item {Utilização:Prov.}
\end{itemize}
\begin{itemize}
\item {Utilização:trasm.}
\end{itemize}
Diz-se do céu, quando apresenta pequenas nuvens negras e pardas, prenúncio de neve.
\section{A-l'obra!}
\begin{itemize}
\item {Grp. gram.:loc. interj.}
\end{itemize}
Mãos á obra!
Trabalhemos! Cf. Castilho, \textunderscore Fausto\textunderscore , 197.
(Contr. de \textunderscore a la obra\textunderscore )
\section{Alóbroge}
\begin{itemize}
\item {Grp. gram.:m.}
\end{itemize}
O mesmo ou melhor que \textunderscore alóbrogo\textunderscore .
\section{Alobrógico}
\begin{itemize}
\item {Grp. gram.:adj.}
\end{itemize}
Relativo aos alóbrogos; próprio de \textunderscore alóbrogo\textunderscore .
\section{Alóbrogo}
\begin{itemize}
\item {Grp. gram.:m.}
\end{itemize}
\begin{itemize}
\item {Grp. gram.:Pl.}
\end{itemize}
\begin{itemize}
\item {Proveniência:(Do lat. \textunderscore allobrox\textunderscore )}
\end{itemize}
Homem grosseiro, rústico.
Povos antigos da região, que hoje se chama Saboia.
\section{Alocásia}
\begin{itemize}
\item {Grp. gram.:f.}
\end{itemize}
Gênero de plantas aroídeas.
\section{Alócero}
\begin{itemize}
\item {Grp. gram.:m.}
\end{itemize}
\begin{itemize}
\item {Proveniência:(Do gr. \textunderscore allos\textunderscore  + \textunderscore keras\textunderscore )}
\end{itemize}
Insecto coleóptero tetrâmero, originário do Brasil.
\section{Alocinesia}
\begin{itemize}
\item {Grp. gram.:f.}
\end{itemize}
\begin{itemize}
\item {Utilização:Med.}
\end{itemize}
\begin{itemize}
\item {Proveniência:(Do gr. \textunderscore allos\textunderscore , outro, e \textunderscore kinesis\textunderscore , movimento)}
\end{itemize}
Perturbação da sensibilidade, em que o doente move um dos membros, quando deseja mover outro.
\section{Alocroito}
\begin{itemize}
\item {Grp. gram.:m.}
\end{itemize}
\begin{itemize}
\item {Proveniência:(Do gr. \textunderscore allokroos\textunderscore )}
\end{itemize}
Mineral, que é uma das variedades da granada.
\section{Alocromasia}
\begin{itemize}
\item {Grp. gram.:f.}
\end{itemize}
\begin{itemize}
\item {Proveniência:(Do gr. \textunderscore allos\textunderscore  + \textunderscore khroma\textunderscore )}
\end{itemize}
Doença, caracterizada por se verem côres differentes das que são realmente.
\section{Alocução}
\begin{itemize}
\item {Grp. gram.:f.}
\end{itemize}
\begin{itemize}
\item {Proveniência:(Lat. \textunderscore allocutio\textunderscore )}
\end{itemize}
Discurso, geralmente breve, pronunciado em occasião solenne.
\section{Alodapa}
\begin{itemize}
\item {Grp. gram.:f.}
\end{itemize}
Gênero de insectos do Cabo da Bôa-Esperança.
\section{Alodial}
\begin{itemize}
\item {Grp. gram.:adj.}
\end{itemize}
\begin{itemize}
\item {Proveniência:(De \textunderscore allódio\textunderscore )}
\end{itemize}
Livre de encargos ou de direitos senhoriaes.
\section{Alodialidade}
\begin{itemize}
\item {Grp. gram.:f.}
\end{itemize}
Isenção.
Qualidade do que é \textunderscore alodial\textunderscore .
\section{Alódio}
\begin{itemize}
\item {Grp. gram.:m.}
\end{itemize}
\begin{itemize}
\item {Utilização:Ant.}
\end{itemize}
Propriedades ou bens, isentos de encargos senhoriaes. Cf. Herculano, \textunderscore Quest. Púb.\textunderscore , I, 183 e 185.
(B. Lat. \textunderscore allodium\textunderscore )
\section{Aloé}
\begin{itemize}
\item {Grp. gram.:m.}
\end{itemize}
O mesmo que \textunderscore aloés\textunderscore .--Camões lia \textunderscore alóe\textunderscore . \textunderscore Lusiadas\textunderscore , X, 137.
\section{Aloé}
\begin{itemize}
\item {Grp. gram.:m.}
\end{itemize}
\begin{itemize}
\item {Proveniência:(Do gr. \textunderscore alloios\textunderscore )}
\end{itemize}
Insecto hymenóptero.
\section{Aloendro}
\begin{itemize}
\item {Grp. gram.:m.}
\end{itemize}
\begin{itemize}
\item {Proveniência:(Do b. lat. \textunderscore lorandrum\textunderscore )}
\end{itemize}
Arbusto apocýneo, (\textunderscore nerium oleander\textunderscore ).
\section{Aloenses}
\begin{itemize}
\item {Grp. gram.:f. pl.}
\end{itemize}
Festas gregas, em honra de Diana. Cf. Castilho, \textunderscore Fastos\textunderscore , I, 544.
\section{Aloerético}
\begin{itemize}
\item {Grp. gram.:adj.}
\end{itemize}
(V.aloético)
\section{Aloés}
\begin{itemize}
\item {Grp. gram.:m.}
\end{itemize}
\begin{itemize}
\item {Proveniência:(Fr. \textunderscore aloès\textunderscore , do lat. \textunderscore aloe\textunderscore )}
\end{itemize}
Planta liliácea, de folhas encarnadas.
Resina purgativa, que se extrai de muitas especies de aloés.
Madeira aromática da Índia, que não tem relação com o verdadeiro aloés.--A antiga e bôa pronúncia era \textunderscore aloés\textunderscore .
Muitos, hoje, dizem \textunderscore áloès\textunderscore .
\section{Aloéste}
\begin{itemize}
\item {Grp. gram.:adv.}
\end{itemize}
\begin{itemize}
\item {Utilização:Ant.}
\end{itemize}
\begin{itemize}
\item {Proveniência:(De \textunderscore a\textunderscore  + \textunderscore lo\textunderscore  + \textunderscore éste\textunderscore )}
\end{itemize}
Para léste; a léste.
\section{Aloetato}
\begin{itemize}
\item {fónica:lo-e}
\end{itemize}
\begin{itemize}
\item {Grp. gram.:m.}
\end{itemize}
\begin{itemize}
\item {Proveniência:(De \textunderscore aloético\textunderscore )}
\end{itemize}
Sal, resultante da combinação do ácido aloético com uma base.
\section{Aloético}
\begin{itemize}
\item {Grp. gram.:adj.}
\end{itemize}
Que contém \textunderscore aloés\textunderscore .
\section{Aloetina}
\begin{itemize}
\item {fónica:lo-e}
\end{itemize}
\begin{itemize}
\item {Grp. gram.:f.}
\end{itemize}
Suco de aloés purificado.
\section{Alofana}
\begin{itemize}
\item {Grp. gram.:f.}
\end{itemize}
O mesmo que \textunderscore alofânio\textunderscore .
\section{Alofanato}
\begin{itemize}
\item {Grp. gram.:m.}
\end{itemize}
Sal, resultante da combinação do ácido alofânico com uma base.
\section{Alofânico}
\begin{itemize}
\item {Grp. gram.:adj.}
\end{itemize}
\begin{itemize}
\item {Proveniência:(De \textunderscore allophana\textunderscore )}
\end{itemize}
Diz-se de um ácido, que não existe, no estado livre.
\section{Alofânio}
\begin{itemize}
\item {Grp. gram.:m.}
\end{itemize}
\begin{itemize}
\item {Utilização:Miner.}
\end{itemize}
\begin{itemize}
\item {Proveniência:(Do gr. \textunderscore allos\textunderscore  + \textunderscore phaino\textunderscore )}
\end{itemize}
Silicato de alumínio hydratado.
\section{Alofe}
\begin{itemize}
\item {Grp. gram.:m.}
\end{itemize}
Gênero de insectos coleópteros.
\section{Alóforo}
\begin{itemize}
\item {Grp. gram.:m.}
\end{itemize}
\begin{itemize}
\item {Proveniência:(Do gr. \textunderscore alos\textunderscore  + \textunderscore phoros\textunderscore )}
\end{itemize}
Insecto díptero, semelhante á môsca.
\section{Alógeno}
\begin{itemize}
\item {Grp. gram.:adj.}
\end{itemize}
\begin{itemize}
\item {Proveniência:(Do gr. \textunderscore allos\textunderscore  + \textunderscore genos\textunderscore )}
\end{itemize}
Que é de outra raça.
\section{Alogia}
\begin{itemize}
\item {Grp. gram.:f.}
\end{itemize}
\begin{itemize}
\item {Proveniência:(Do gr. \textunderscore a\textunderscore  priv. e \textunderscore logos\textunderscore )}
\end{itemize}
Absurdo, disparate.
\section{Alogianos}
\begin{itemize}
\item {Grp. gram.:m. pl.}
\end{itemize}
\begin{itemize}
\item {Proveniência:(De \textunderscore alogia\textunderscore )}
\end{itemize}
Seita dos que negavam a Jesus a qualidade de Verbo eterno.
\section{Alógico}
\begin{itemize}
\item {Grp. gram.:adj.}
\end{itemize}
\begin{itemize}
\item {Proveniência:(De \textunderscore alogia\textunderscore )}
\end{itemize}
Que não precisa de demonstração, para se vêr que é certo.
\section{Alógono}
\begin{itemize}
\item {Grp. gram.:adj.}
\end{itemize}
\begin{itemize}
\item {Proveniência:(Do gr. \textunderscore allos\textunderscore  + \textunderscore gonos\textunderscore )}
\end{itemize}
Diz-se de um crystal, que reúne á fórma de nó a de um decaédro de triângulos escalenos, dos quaes câda um tem o seu ângulo plano obtuso igual á maior incidência das faces do nó.
\section{Álogos}
\begin{itemize}
\item {Grp. gram.:m. pl.}
\end{itemize}
O mesmo que \textunderscore alogianos\textunderscore .
\section{Aloico}
\begin{itemize}
\item {Grp. gram.:adj.}
\end{itemize}
Diz-se de um ácido, resultante da acção do ácido sulfúrico sôbre o aloés.
\section{Aloilado}
\begin{itemize}
\item {Grp. gram.:adj.}
\end{itemize}
\begin{itemize}
\item {Utilização:Prov.}
\end{itemize}
\begin{itemize}
\item {Utilização:trasm.}
\end{itemize}
Atoleimado; maluco.
\section{Aloína}
\begin{itemize}
\item {Grp. gram.:f.}
\end{itemize}
Substância crystallizável, que constitue o princípio purgativo de aloés.
\section{Aloinado}
\begin{itemize}
\item {fónica:lo-i}
\end{itemize}
\begin{itemize}
\item {Grp. gram.:adj.}
\end{itemize}
\begin{itemize}
\item {Proveniência:(De \textunderscore aloína\textunderscore )}
\end{itemize}
Semelhante ao aloés.
\section{Aloíte}
\begin{itemize}
\item {Grp. gram.:f.}
\end{itemize}
Variedade de pozolana.
\section{Aloirar}
\begin{itemize}
\item {Grp. gram.:v. t.}
\end{itemize}
Tornar loiro ou semelhante a loiro.
\section{Aloisar}
\begin{itemize}
\item {Grp. gram.:v. t.}
\end{itemize}
\begin{itemize}
\item {Grp. gram.:V. p.}
\end{itemize}
\begin{itemize}
\item {Utilização:Prov.}
\end{itemize}
Cobrir com loisa.
Retirar-se para um canto, a fim de dormir, (falando-se de animaes).
(Colhido em Turquel)
\section{Alojação}
\begin{itemize}
\item {Grp. gram.:f.}
\end{itemize}
(V.alojamento)
\section{Alojamento}
\begin{itemize}
\item {Grp. gram.:m.}
\end{itemize}
Acto de \textunderscore alojar\textunderscore .
Lugar, onde alguém ou alguma coisa se aloja.
\section{Alojar}
\begin{itemize}
\item {Grp. gram.:v. t.}
\end{itemize}
Meter em loja.
Recolher.
Agasalhar; hospedar.
Armazenar.
\section{Alojar}
\begin{itemize}
\item {Grp. gram.:v. t.}
\end{itemize}
\begin{itemize}
\item {Utilização:Bras. de Minas}
\end{itemize}
O mesmo que \textunderscore vomitar\textunderscore .
\section{Alôjo}
\begin{itemize}
\item {Grp. gram.:m.}
\end{itemize}
\begin{itemize}
\item {Utilização:Prov.}
\end{itemize}
\begin{itemize}
\item {Utilização:alent.}
\end{itemize}
O mesmo que \textunderscore alojamento\textunderscore .
\section{Alôjo}
\begin{itemize}
\item {Grp. gram.:m.}
\end{itemize}
\begin{itemize}
\item {Utilização:Bras. de Minas}
\end{itemize}
\begin{itemize}
\item {Proveniência:(De \textunderscore alojar\textunderscore ^2)}
\end{itemize}
O mesmo que \textunderscore vómito\textunderscore .
\section{Alom!}
\begin{itemize}
\item {Grp. gram.:interj.}
\end{itemize}
\begin{itemize}
\item {Proveniência:(Fr. \textunderscore allons\textunderscore )}
\end{itemize}
Vamos!
Adeante! Cf. Rebello, \textunderscore Mocidade\textunderscore , III, 96; e Garção, II, 62.
\section{Alomancia}
\begin{itemize}
\item {Grp. gram.:f.}
\end{itemize}
\begin{itemize}
\item {Proveniência:(Do gr. \textunderscore als\textunderscore  + \textunderscore manteia\textunderscore )}
\end{itemize}
Arte de adivinhação por meio do sal.
\section{Alomante}
\begin{itemize}
\item {Grp. gram.:m.}
\end{itemize}
Aquelle que pratica a alomancia.
\section{Alombamento}
\begin{itemize}
\item {Grp. gram.:m.}
\end{itemize}
Acto de \textunderscore alombar\textunderscore .
\section{Alombar}
\begin{itemize}
\item {Grp. gram.:v. t.}
\end{itemize}
Fazer curvo como o lombo.
Arquear.
Derrear.
Pôr lombada em (livros).
\section{Alomborar}
\begin{itemize}
\item {Grp. gram.:v. t.}
\end{itemize}
(V.alamborar)
\section{Alomear}
\begin{itemize}
\item {Grp. gram.:v. t.}
\end{itemize}
\begin{itemize}
\item {Utilização:Pop.}
\end{itemize}
Dizer o nome de, nomear, mencionar.
(Ant. \textunderscore lomear\textunderscore , do lat. \textunderscore nominare\textunderscore )
\section{Alomia}
\begin{itemize}
\item {Grp. gram.:f.}
\end{itemize}
\begin{itemize}
\item {Proveniência:(Do gr. \textunderscore a\textunderscore  priv. e \textunderscore loma\textunderscore )}
\end{itemize}
Planta herbácea, de flôres brancas, originária do México.
\section{Alomorfia}
\begin{itemize}
\item {Grp. gram.:f.}
\end{itemize}
\begin{itemize}
\item {Proveniência:(Do gr. \textunderscore allos\textunderscore  + \textunderscore morphe\textunderscore )}
\end{itemize}
Passagem de uma fórma para outra, diversa. Metamorphose.
\section{Alomórfico}
\begin{itemize}
\item {Grp. gram.:adj.}
\end{itemize}
\begin{itemize}
\item {Proveniência:(De \textunderscore allomorphia\textunderscore )}
\end{itemize}
Que tem fórma diversa.
\section{Alonço}
\begin{itemize}
\item {Grp. gram.:m.}
\end{itemize}
\begin{itemize}
\item {Utilização:Prov.}
\end{itemize}
\begin{itemize}
\item {Utilização:alent.}
\end{itemize}
O mesmo que \textunderscore palonço\textunderscore .
\section{Alonga}
\begin{itemize}
\item {Grp. gram.:f.}
\end{itemize}
\begin{itemize}
\item {Proveniência:(De \textunderscore alongar\textunderscore )}
\end{itemize}
Tubo de vidro, em fórma de fuso, que se adapta ás retortas ou balões, nos laboratórios.
\section{Alonga}
\begin{itemize}
\item {Grp. gram.:f.}
\end{itemize}
\begin{itemize}
\item {Utilização:Ant.}
\end{itemize}
O longo, a margem.
(Contr. de \textunderscore a\textunderscore  + \textunderscore la\textunderscore  + \textunderscore longa\textunderscore )
\section{Alongadamente}
\begin{itemize}
\item {Grp. gram.:adv.}
\end{itemize}
Com demora.
\section{Alongamento}
\begin{itemize}
\item {Grp. gram.:m.}
\end{itemize}
Acto de \textunderscore alongar\textunderscore .
\section{Alongar}
\begin{itemize}
\item {Grp. gram.:v. t.}
\end{itemize}
Tornar longo.
Estender.
Afastar.
Demorar.
Aumentar.
\section{Alónimo}
\begin{itemize}
\item {Grp. gram.:m.}
\end{itemize}
\begin{itemize}
\item {Proveniência:(Do gr. \textunderscore allos\textunderscore  + \textunderscore onuma\textunderscore )}
\end{itemize}
Aquelle que se serve do nome de outrem, assinando.
\section{Alônsoa}
\begin{itemize}
\item {Grp. gram.:f.}
\end{itemize}
\begin{itemize}
\item {Proveniência:(De \textunderscore Alonso\textunderscore , n. p.)}
\end{itemize}
Planta ornamental, originária dos Andes.
\section{Alopata}
\begin{itemize}
\item {Grp. gram.:m.}
\end{itemize}
\begin{itemize}
\item {Proveniência:(Do gr. \textunderscore allos\textunderscore  + \textunderscore pathos\textunderscore )}
\end{itemize}
Aquelle que exerce a alopatia.--A pronúncia exacta seria \textunderscore alopáta\textunderscore .
\section{Alopatia}
\begin{itemize}
\item {Grp. gram.:f.}
\end{itemize}
\begin{itemize}
\item {Proveniência:(De \textunderscore allopatha\textunderscore )}
\end{itemize}
Systema commum de medicina, que combate as doenças por meios contrários a estas.
\section{Alopaticamente}
\begin{itemize}
\item {Grp. gram.:adv.}
\end{itemize}
Segundo o systema \textunderscore alopático\textunderscore .
\section{Alopático}
\begin{itemize}
\item {Grp. gram.:adv.}
\end{itemize}
Relativo á \textunderscore alopatia\textunderscore .
\section{Alopecia}
\begin{itemize}
\item {Grp. gram.:f.}
\end{itemize}
\begin{itemize}
\item {Proveniência:(Gr. \textunderscore alopekia\textunderscore )}
\end{itemize}
Quéda dos cabellos, por doença ou por qualquer accidente.
\section{Alopeciados}
\begin{itemize}
\item {Grp. gram.:m. pl.}
\end{itemize}
O mesmo que alopecianos.
\section{Alopecianos}
\begin{itemize}
\item {Grp. gram.:m. pl.}
\end{itemize}
Grupo de peixes esqualos.
\section{Alopécico}
\begin{itemize}
\item {Grp. gram.:m.}
\end{itemize}
Indivíduo, que soffre de alopecia.
\section{Alopécios}
\begin{itemize}
\item {Grp. gram.:m. pl.}
\end{itemize}
O mesmo que \textunderscore alopecianos\textunderscore .
\section{Alopecura}
\begin{itemize}
\item {Grp. gram.:f.}
\end{itemize}
\begin{itemize}
\item {Proveniência:(Do gr. \textunderscore alopex\textunderscore  + \textunderscore oura\textunderscore )}
\end{itemize}
Planta gramínea.
\section{Alopecuro}
\begin{itemize}
\item {Grp. gram.:m.}
\end{itemize}
O mesmo que \textunderscore alopecura\textunderscore .
\section{Alophe}
\begin{itemize}
\item {Grp. gram.:m.}
\end{itemize}
Gênero de insectos coleópteros.
\section{Alóphoro}
\begin{itemize}
\item {Grp. gram.:m.}
\end{itemize}
\begin{itemize}
\item {Proveniência:(Do gr. \textunderscore alos\textunderscore  + \textunderscore phoros\textunderscore )}
\end{itemize}
Insecto díptero, semelhante á môsca.
\section{Alópia}
\begin{itemize}
\item {Grp. gram.:f.}
\end{itemize}
Concha fina e mais ou menos rugosa.
\section{Aloplecto}
\begin{itemize}
\item {Grp. gram.:m.}
\end{itemize}
Gênero de plantas escrofularineas.
\section{Alóptero}
\begin{itemize}
\item {Grp. gram.:adj.}
\end{itemize}
\begin{itemize}
\item {Utilização:Ichthyol.}
\end{itemize}
\begin{itemize}
\item {Proveniência:(Do gr. \textunderscore allos\textunderscore , outro, e \textunderscore pleron\textunderscore  asa)}
\end{itemize}
Diz-se dos peixes, cujas barbatanas não têm posição fixa.
\section{Aloque}
\begin{itemize}
\item {Grp. gram.:m.}
\end{itemize}
\begin{itemize}
\item {Utilização:Prov.}
\end{itemize}
\begin{itemize}
\item {Utilização:minh.}
\end{itemize}
\begin{itemize}
\item {Proveniência:(T. cast.)}
\end{itemize}
Esconderijo (de peixes principalmente, no rio).
O mesmo que \textunderscore monturo\textunderscore . Cf. Camillo, \textunderscore Cav. em Ruínas\textunderscore , 108.
Boneca de pão e açúcar, em que chucham as crianças, e que também se chama \textunderscore rólha\textunderscore .
\section{Aloquete}
\begin{itemize}
\item {fónica:quê}
\end{itemize}
\begin{itemize}
\item {Grp. gram.:m.}
\end{itemize}
\begin{itemize}
\item {Utilização:Prov.}
\end{itemize}
O mesmo que \textunderscore loquete\textunderscore .
\section{Aloquezia}
\begin{itemize}
\item {Grp. gram.:f.}
\end{itemize}
Evacuação das matérias fecaes por abertura accidental ou anormal.
\section{Aloquiria}
\begin{itemize}
\item {Grp. gram.:f.}
\end{itemize}
\begin{itemize}
\item {Utilização:Med.}
\end{itemize}
\begin{itemize}
\item {Proveniência:(Do gr. \textunderscore allos\textunderscore , outro, e \textunderscore kheir\textunderscore , mão)}
\end{itemize}
O mesmo que \textunderscore allesthesia\textunderscore .
\section{Alorpado}
\begin{itemize}
\item {Grp. gram.:adj.}
\end{itemize}
Que tem modos de lorpa.
Apalermado. Cf. Castilho, \textunderscore Médico á Fôrça\textunderscore .
\section{Alor}
\begin{itemize}
\item {Grp. gram.:m.}
\end{itemize}
\begin{itemize}
\item {Proveniência:(Do fr. \textunderscore allure\textunderscore ?)}
\end{itemize}
Impulso:«\textunderscore Hércules quer que alor se dê aos braços\textunderscore ». Filinto, VII, 244.
Movimentos:«\textunderscore dar aos braços e aos quadris o alor da investida afadistada\textunderscore ». Camillo, \textunderscore Sebenta\textunderscore , IV, 13.
Estímulo, incitamento. Cf. Camillo, \textunderscore Narcót\textunderscore . I, 143.
\section{Alosna}
\begin{itemize}
\item {Grp. gram.:f.}
\end{itemize}
(V.losna)
\section{Alotador}
\begin{itemize}
\item {Grp. gram.:m.}
\end{itemize}
\begin{itemize}
\item {Utilização:Bras. do N}
\end{itemize}
\begin{itemize}
\item {Proveniência:(De \textunderscore lote\textunderscore )}
\end{itemize}
Cavallo de padreação, correspondente a um lote de éguas.
\section{Alotar}
\begin{itemize}
\item {Grp. gram.:v. t.}
\end{itemize}
\begin{itemize}
\item {Utilização:Bras. do N}
\end{itemize}
\begin{itemize}
\item {Proveniência:(De \textunderscore lote\textunderscore )}
\end{itemize}
Velar, para que se não dispersem (éguas que formam um lote, privativo de um cavallo de padreação).
\section{Alote}
\begin{itemize}
\item {Grp. gram.:m.}
\end{itemize}
\begin{itemize}
\item {Proveniência:(De \textunderscore alar\textunderscore )}
\end{itemize}
Pequeno cabo náutico, para alar.
\section{Alotriologia}
\begin{itemize}
\item {Grp. gram.:f.}
\end{itemize}
\begin{itemize}
\item {Proveniência:(Do gr. \textunderscore allotrios\textunderscore  + \textunderscore logos\textunderscore )}
\end{itemize}
Applicação de doutrinas, estranhas ao assumpto occorrente.
\section{Alotriofagia}
\begin{itemize}
\item {Grp. gram.:f.}
\end{itemize}
\begin{itemize}
\item {Proveniência:(Do gr. \textunderscore allotrios\textunderscore  + \textunderscore phagein\textunderscore )}
\end{itemize}
Doença, caracterizada pela vontade de comer o que não sustenta ou o que é nocivo.
\section{Alotriófago}
\begin{itemize}
\item {Grp. gram.:m.}
\end{itemize}
Aquelle que soffre \textunderscore alotriofagia\textunderscore .
\section{Alotriosmia}
\begin{itemize}
\item {Grp. gram.:f.}
\end{itemize}
\begin{itemize}
\item {Utilização:Med.}
\end{itemize}
\begin{itemize}
\item {Proveniência:(Do gr. \textunderscore allotrios\textunderscore  + \textunderscore osme\textunderscore )}
\end{itemize}
Vício de olfato, que consiste em sensações olfativas, paradoxaes.
\section{Alotrófico}
\begin{itemize}
\item {Grp. gram.:adj.}
\end{itemize}
\begin{itemize}
\item {Utilização:Neol.}
\end{itemize}
\begin{itemize}
\item {Proveniência:(Do gr. \textunderscore allos\textunderscore  + \textunderscore trophe\textunderscore )}
\end{itemize}
Que tem differente desenvolvimento.
\section{Alotropia}
\begin{itemize}
\item {Grp. gram.:f.}
\end{itemize}
Qualidade, que alguns corpos simples têm, de se apresentar em differentes estados, a que correspondem propriedades distintas.
(Cp. \textunderscore allótropo\textunderscore )
\section{Alotrópico}
\begin{itemize}
\item {Grp. gram.:adj.}
\end{itemize}
\begin{itemize}
\item {Proveniência:(De \textunderscore allotropia\textunderscore )}
\end{itemize}
O mesmo que \textunderscore alótropo\textunderscore .
\section{Alótropo}
\begin{itemize}
\item {Grp. gram.:adj.}
\end{itemize}
\begin{itemize}
\item {Utilização:Philol.}
\end{itemize}
\begin{itemize}
\item {Proveniência:(Do gr. \textunderscore allos\textunderscore  + \textunderscore tropos\textunderscore )}
\end{itemize}
Diz-se do corpo simples, em que se dá a alotropia.
Diz-se dos vocábulos divergentes, derivados de um só, como \textunderscore mancha\textunderscore , \textunderscore mágoa\textunderscore  e \textunderscore malha\textunderscore , do lat. \textunderscore macula\textunderscore .
\section{Aloucado}
\begin{itemize}
\item {Grp. gram.:adj.}
\end{itemize}
Que tem modos de louco.
\section{Aloucar-se}
\begin{itemize}
\item {Grp. gram.:v. p.}
\end{itemize}
Parecer louco; têr modos de louco.
\section{Alourar}
\begin{itemize}
\item {Grp. gram.:v. t.}
\end{itemize}
Tornar louro ou semelhante a louro.
\section{Alousar}
\begin{itemize}
\item {Grp. gram.:v. t.}
\end{itemize}
\begin{itemize}
\item {Grp. gram.:V. p.}
\end{itemize}
\begin{itemize}
\item {Utilização:Prov.}
\end{itemize}
Cobrir com loisa.
Retirar-se para um canto, a fim de dormir, (falando-se de animaes).
(Colhido em Turquel)
\section{Alovuco}
\begin{itemize}
\item {Grp. gram.:m.}
\end{itemize}
Árvore do Congo.
\section{Alpaca}
\begin{itemize}
\item {Grp. gram.:f.}
\end{itemize}
Ruminante, da fam. dos camelídeos, originário da América do Sul.
Lan da alpaca; tecido de lan da alpaca.
\section{Alfa}
\begin{itemize}
\item {Grp. gram.:m.}
\end{itemize}
\begin{itemize}
\item {Utilização:Fig.}
\end{itemize}
\begin{itemize}
\item {Utilização:Mús.}
\end{itemize}
Primeira letra do alphabeto syríaco e grego.
Principio.
Figura, que, na antiga notação, abrangia dois lugares de um pentagramma e representava duas notas ligadas.
\section{Alfabetação}
\begin{itemize}
\item {Grp. gram.:f.}
\end{itemize}
Acto de \textunderscore alfabetar\textunderscore .
\section{Alfabetadamente}
\begin{itemize}
\item {Grp. gram.:adv.}
\end{itemize}
De modo \textunderscore alfabetado\textunderscore ; pela ordem alfabética.
\section{Alfabetado}
\begin{itemize}
\item {Grp. gram.:adj.}
\end{itemize}
Disposto pela ordem alfabética.
\section{Alfabetador}
\begin{itemize}
\item {Grp. gram.:m.}
\end{itemize}
Aquelle que alfabeta.
\section{Alfabetamento}
\begin{itemize}
\item {Grp. gram.:m.}
\end{itemize}
O mesmo que \textunderscore alfabetação\textunderscore .
\section{Alfabetar}
\begin{itemize}
\item {Grp. gram.:v. t.}
\end{itemize}
Dispor, segundo a ordem das letras do alfabeto.
\section{Alfabetário}
\begin{itemize}
\item {Grp. gram.:adj.}
\end{itemize}
Relativo ao alfabeto.
Que tem alfabeto.
\section{Alfabeticamente}
\begin{itemize}
\item {Grp. gram.:adv.}
\end{itemize}
De modo \textunderscore alfabético\textunderscore .
\section{Alfabético}
\begin{itemize}
\item {Grp. gram.:adj.}
\end{itemize}
Alfabetário.
Disposto segundo a ordem das letras do alfabeto.
\section{Alfabetista}
\begin{itemize}
\item {Grp. gram.:m.}
\end{itemize}
O mesmo que \textunderscore alfabetador\textunderscore .
\section{Alfabeto}
\begin{itemize}
\item {Grp. gram.:m.}
\end{itemize}
\begin{itemize}
\item {Proveniência:(De \textunderscore alpha\textunderscore  + \textunderscore beta\textunderscore , nome das prímeiras letras do alphabeto grego)}
\end{itemize}
Abecedário.
Ordem ou disposição convencional das letras de uma língua.
Conjunto das mesmas letras.
Rudimentos de qualquer sciência ou arte.
Qualquer série convencional.
\section{Alfada}
\begin{itemize}
\item {Grp. gram.:adj. f.}
\end{itemize}
\begin{itemize}
\item {Utilização:Mús.}
\end{itemize}
\begin{itemize}
\item {Utilização:ant.}
\end{itemize}
Dizia-se da figura que também se chamava \textunderscore alpha\textunderscore .
(Cp. \textunderscore alpha\textunderscore )
\section{Alfamoxa}
\begin{itemize}
\item {fónica:mô}
\end{itemize}
\begin{itemize}
\item {Grp. gram.:f.}
\end{itemize}
\begin{itemize}
\item {Utilização:Mús.}
\end{itemize}
\begin{itemize}
\item {Utilização:ant.}
\end{itemize}
Alpha sem pellica. Cf. \textunderscore Diccion. Mús.\textunderscore 
\section{Alfeia}
\begin{itemize}
\item {Grp. gram.:f.}
\end{itemize}
Gênero de plantas malváceas.
Gênero de crustáceos decápodes.
\section{Alfênico}
\begin{itemize}
\item {Grp. gram.:m.}
\end{itemize}
Açúcar cândi.
\section{Alfol}
\begin{itemize}
\item {Grp. gram.:m.}
\end{itemize}
Producto chímico, usado como antiséptico e anti-rheumático.
\section{Alfonsino}
\begin{itemize}
\item {Grp. gram.:m.}
\end{itemize}
Antigo instrumento de cirurgia.
\section{Aloxana}
\begin{itemize}
\item {Grp. gram.:f.}
\end{itemize}
\begin{itemize}
\item {Proveniência:(Al. \textunderscore alloxan\textunderscore )}
\end{itemize}
Substância, produzida pela acção do ácido azótico sôbre o ácido úrico.
\section{Aloxanato}
\begin{itemize}
\item {Grp. gram.:m.}
\end{itemize}
Combinação de álcalis com aloxana.
\section{Aloxantina}
\begin{itemize}
\item {Grp. gram.:f.}
\end{itemize}
Producto chímico, resultante da acção do ácido azótico sôbre o ácido úrico.
\section{Alpalhoeiro}
\begin{itemize}
\item {Grp. gram.:m.}
\end{itemize}
\begin{itemize}
\item {Utilização:Prov.}
\end{itemize}
\begin{itemize}
\item {Utilização:alent.}
\end{itemize}
Homem de Alpalhão.
\section{Alpão}
\begin{itemize}
\item {Grp. gram.:m.}
\end{itemize}
Designação vulgar de uma planta malabárica, (\textunderscore bragantia Wallichii\textunderscore ), que se considera efficaz contra as úlceras e mordeduras de cobra.
\section{Alparavaz}
\begin{itemize}
\item {Grp. gram.:m.}
\end{itemize}
\begin{itemize}
\item {Utilização:Ant.}
\end{itemize}
Franja; sanefa.
\section{Alparca}
\begin{itemize}
\item {Grp. gram.:f.}
\end{itemize}
\begin{itemize}
\item {Proveniência:(Do vasc. \textunderscore abarca\textunderscore )}
\end{itemize}
Sandália, espécie de calçado, em que a sola se ajusta ao pé, por meio de tiras de coiro ou de pano.
\section{Alparcata}
\begin{itemize}
\item {Grp. gram.:f.}
\end{itemize}
(V.alparca)
\section{Alparcateiro}
\begin{itemize}
\item {Grp. gram.:m.}
\end{itemize}
\begin{itemize}
\item {Proveniência:(De \textunderscore alparcata\textunderscore )}
\end{itemize}
O mesmo que \textunderscore alparqueiro\textunderscore .
\section{Alparcheiro}
\begin{itemize}
\item {Grp. gram.:m.}
\end{itemize}
Variedade de uva branca da Arruda.
(Por \textunderscore alpercheiro\textunderscore , de \textunderscore alperche\textunderscore ?)
\section{Alpargata}
\begin{itemize}
\item {Grp. gram.:f.}
\end{itemize}
\begin{itemize}
\item {Proveniência:(T. cast.)}
\end{itemize}
O mesmo que \textunderscore alparca\textunderscore :«\textunderscore as alpargatas semeadas de todo o gênero de pedraria\textunderscore ». Sousa, \textunderscore V. do Arceb\textunderscore .
\section{Alpargataria}
\begin{itemize}
\item {Grp. gram.:f.}
\end{itemize}
Officina de alpargatas.
\section{Alpargateiro}
\begin{itemize}
\item {Grp. gram.:m.}
\end{itemize}
O mesmo que \textunderscore alparqueiro\textunderscore .
\section{Alparluz}
\begin{itemize}
\item {Grp. gram.:m.}
\end{itemize}
\begin{itemize}
\item {Utilização:Ant.}
\end{itemize}
Pára-luz, pára-fogo.
Sanefa.
(Por \textunderscore apara-luz\textunderscore )
\section{Alparqueiro}
\begin{itemize}
\item {Grp. gram.:m.}
\end{itemize}
Aquelle que fazia alparcas.
\section{Alpe}
\begin{itemize}
\item {Grp. gram.:m.}
\end{itemize}
\begin{itemize}
\item {Utilização:Ant.}
\end{itemize}
Travesseiro com almofada.
\section{Álpea}
\begin{itemize}
\item {Grp. gram.:f.}
\end{itemize}
Gênero de coleópteros.
\section{Alpechim}
\begin{itemize}
\item {Grp. gram.:m.}
\end{itemize}
Sumo negro e amargo das azeitonas.
Resíduos líquidos do fabríco do azeite.
(Cast. \textunderscore alpechin\textunderscore )
\section{Alpeiria}
\begin{itemize}
\item {Grp. gram.:f.}
\end{itemize}
\begin{itemize}
\item {Utilização:Prov.}
\end{itemize}
\begin{itemize}
\item {Utilização:minh.}
\end{itemize}
O mesmo que \textunderscore apeiria\textunderscore .
\section{Alpendorada}
\begin{itemize}
\item {Grp. gram.:f.}
\end{itemize}
O mesmo que \textunderscore alpendrada\textunderscore .
\section{Alpendrada}
\begin{itemize}
\item {Grp. gram.:f.}
\end{itemize}
Grande alpendre, sustentado por columnas.
\section{Alpendrar}
\begin{itemize}
\item {Grp. gram.:v. t.}
\end{itemize}
Cobrir com alpendre.
\section{Alpendre}
\begin{itemize}
\item {Grp. gram.:m.}
\end{itemize}
\begin{itemize}
\item {Proveniência:(Do lat. \textunderscore ad pendulum\textunderscore , segundo Cornu)}
\end{itemize}
Telheiro.
Tecto, suspenso por columnas ou pilastras, de um lado pelo menos.
Tecto saliente, cobrindo a entrada de um edifício.
\section{Alpense}
\begin{itemize}
\item {Grp. gram.:adj.}
\end{itemize}
O mesmo que \textunderscore alpino\textunderscore .
\section{Alpercata}
\begin{itemize}
\item {Grp. gram.:f.}
\end{itemize}
(V.alparca)
\section{Alpercateiro}
\begin{itemize}
\item {Grp. gram.:m.}
\end{itemize}
\begin{itemize}
\item {Proveniência:(De \textunderscore alpercata\textunderscore )}
\end{itemize}
O mesmo que \textunderscore alparqueiro\textunderscore .
\section{Alperce}
\begin{itemize}
\item {Grp. gram.:m.}
\end{itemize}
Espécie de damasco grande, de cheiro semelhante ao do pêssego.
\section{Alperceiro}
\begin{itemize}
\item {Grp. gram.:m.}
\end{itemize}
Árvore, que dá o alperce ou alperche.
\section{Alperche}
\begin{itemize}
\item {Grp. gram.:m.}
\end{itemize}
O mesmo ou melhor que \textunderscore alperce\textunderscore .
(Cp. cast. \textunderscore alberchigo\textunderscore )
\section{Alpercheiro}
\begin{itemize}
\item {Grp. gram.:m.}
\end{itemize}
O mesmo que \textunderscore alperceiro\textunderscore .
\section{Alpergata}
\begin{itemize}
\item {Grp. gram.:f.}
\end{itemize}
O mesmo que \textunderscore alpargata\textunderscore .
\section{Alpes}
\begin{itemize}
\item {Grp. gram.:m. pl.}
\end{itemize}
\begin{itemize}
\item {Utilização:Ant.}
\end{itemize}
Pastagens entre montes.
(Parece relacionar-se com \textunderscore Alpes\textunderscore , n. p.)
\section{Alpestre}
\begin{itemize}
\item {Grp. gram.:adj.}
\end{itemize}
\begin{itemize}
\item {Proveniência:(Lat. \textunderscore alpestris\textunderscore )}
\end{itemize}
Relativo ou semelhante aos Alpes.
Fragoso; cheio de montes e despenhadeiros.
Que cresce nas montanhas.
\section{Alpéstrico}
\begin{itemize}
\item {Grp. gram.:adj.}
\end{itemize}
(V.alpestre)
\section{Alpestrino}
\begin{itemize}
\item {Grp. gram.:adj.}
\end{itemize}
O mesmo que \textunderscore alpestre\textunderscore . Cf. \textunderscore Viriato Trág.\textunderscore , I, 20.
\section{Alpha}
\begin{itemize}
\item {Grp. gram.:m.}
\end{itemize}
\begin{itemize}
\item {Utilização:Fig.}
\end{itemize}
\begin{itemize}
\item {Utilização:Mús.}
\end{itemize}
Primeira letra do alphabeto syríaco e grego.
Principio.
Figura, que, na antiga notação, abrangia dois lugares de um pentagramma e representava duas notas ligadas.
\section{Alphabetação}
\begin{itemize}
\item {Grp. gram.:f.}
\end{itemize}
Acto de \textunderscore alphabetar\textunderscore .
\section{Alphabetadamente}
\begin{itemize}
\item {Grp. gram.:adv.}
\end{itemize}
De modo \textunderscore alphabetado\textunderscore ; pela ordem alphabética.
\section{Alphabetado}
\begin{itemize}
\item {Grp. gram.:adj.}
\end{itemize}
Disposto pela ordem alphabética.
\section{Alphabetador}
\begin{itemize}
\item {Grp. gram.:m.}
\end{itemize}
Aquelle que alphabeta.
\section{Alphabetamento}
\begin{itemize}
\item {Grp. gram.:m.}
\end{itemize}
O mesmo que \textunderscore alphabetação\textunderscore .
\section{Alphabetar}
\begin{itemize}
\item {Grp. gram.:v. t.}
\end{itemize}
Dispor, segundo a ordem das letras do alphabeto.
\section{Alphabetário}
\begin{itemize}
\item {Grp. gram.:adj.}
\end{itemize}
Relativo ao alphabeto.
Que tem alphabeto.
\section{Alphabeticamente}
\begin{itemize}
\item {Grp. gram.:adv.}
\end{itemize}
De modo \textunderscore alphabético\textunderscore .
\section{Alphabético}
\begin{itemize}
\item {Grp. gram.:adj.}
\end{itemize}
Alphabetário.
Disposto segundo a ordem das letras do alphabeto.
\section{Alphabetista}
\begin{itemize}
\item {Grp. gram.:m.}
\end{itemize}
O mesmo que \textunderscore alphabetador\textunderscore .
\section{Alphabeto}
\begin{itemize}
\item {Grp. gram.:m.}
\end{itemize}
\begin{itemize}
\item {Proveniência:(De \textunderscore alpha\textunderscore  + \textunderscore beta\textunderscore , nome das prímeiras letras do alphabeto grego)}
\end{itemize}
Abecedário.
Ordem ou disposição convencional das letras de uma língua.
Conjunto das mesmas letras.
Rudimentos de qualquer sciência ou arte.
Qualquer série convencional.
\section{Alphada}
\begin{itemize}
\item {Grp. gram.:adj. f.}
\end{itemize}
\begin{itemize}
\item {Utilização:Mús.}
\end{itemize}
\begin{itemize}
\item {Utilização:ant.}
\end{itemize}
Dizia-se da figura que também se chamava \textunderscore alpha\textunderscore .
(Cp. \textunderscore alpha\textunderscore )
\section{Alphamoxa}
\begin{itemize}
\item {fónica:mô}
\end{itemize}
\begin{itemize}
\item {Grp. gram.:f.}
\end{itemize}
\begin{itemize}
\item {Utilização:Mús.}
\end{itemize}
\begin{itemize}
\item {Utilização:ant.}
\end{itemize}
Alpha sem pellica. Cf. \textunderscore Diccion. Mús.\textunderscore 
\section{Alpheia}
\begin{itemize}
\item {Grp. gram.:f.}
\end{itemize}
Gênero de plantas malváceas.
Gênero de crustáceos decápodes.
\section{Alphênico}
\begin{itemize}
\item {Grp. gram.:m.}
\end{itemize}
Açúcar cândi.
\section{Alphol}
\begin{itemize}
\item {Grp. gram.:m.}
\end{itemize}
Producto chímico, usado como antiséptico e anti-rheumático.
\section{Alphonsino}
\begin{itemize}
\item {Grp. gram.:m.}
\end{itemize}
Antigo instrumento de cirurgia.
\section{Álpico}
\begin{itemize}
\item {Grp. gram.:adj.}
\end{itemize}
(V.alpino)
\section{Alpícola}
\begin{itemize}
\item {Grp. gram.:adj.}
\end{itemize}
\begin{itemize}
\item {Proveniência:(De \textunderscore Alpes\textunderscore , n. p. + lat. \textunderscore colere\textunderscore )}
\end{itemize}
Que vive nos Alpes.
\section{Alpígena}
\begin{itemize}
\item {Grp. gram.:adj.}
\end{itemize}
\begin{itemize}
\item {Utilização:P. us.}
\end{itemize}
O mesmo que \textunderscore alpestre\textunderscore .
\section{Alpim}
\begin{itemize}
\item {Grp. gram.:m.}
\end{itemize}
Planta brasileira.
\section{Alpinar}
\begin{itemize}
\item {Grp. gram.:V. i.}
\end{itemize}
\begin{itemize}
\item {Utilização:Gír. lisb.}
\end{itemize}
Fugir.
\section{Alpínia}
\begin{itemize}
\item {Grp. gram.:f.}
\end{itemize}
\begin{itemize}
\item {Proveniência:(De \textunderscore Alpinio\textunderscore , n. p.)}
\end{itemize}
Gênero de plantas zingiberáceas.
\section{Alpinismo}
\begin{itemize}
\item {Grp. gram.:m.}
\end{itemize}
\begin{itemize}
\item {Utilização:Ext.}
\end{itemize}
Gôsto ou hábito das ascensões aos Alpes.
Gôsto das ascensões ás grandes altitudes.
\section{Alpinista}
\begin{itemize}
\item {Grp. gram.:m.  e  adj.}
\end{itemize}
O que aprecia as excursões aos Alpes ou a outras montanhas.
\section{Alpino}
\begin{itemize}
\item {Grp. gram.:adj.}
\end{itemize}
\begin{itemize}
\item {Proveniência:(Lat. \textunderscore alpinus\textunderscore )}
\end{itemize}
Relativo aos Alpes.
Que nasce ou cresce nos Alpes.
\section{Alpirche}
\begin{itemize}
\item {Grp. gram.:m.}
\end{itemize}
\begin{itemize}
\item {Utilização:Prov.}
\end{itemize}
\begin{itemize}
\item {Utilização:trasm.}
\end{itemize}
O mesmo que \textunderscore alpechim\textunderscore .
\section{Alpista}
\begin{itemize}
\item {Grp. gram.:f.}
\end{itemize}
O mesmo que \textunderscore alpiste\textunderscore .
\section{Alpiste}
\begin{itemize}
\item {Grp. gram.:m.}
\end{itemize}
Planta gramínea, (\textunderscore phaleris canariensis\textunderscore ).
Grãos dessa planta, que se empregam em sustento de pássaros engaiolados ou domésticos.
(Cast. \textunderscore alpiste\textunderscore )
\section{Alpisteiro}
\begin{itemize}
\item {Grp. gram.:m.}
\end{itemize}
Recipiente para alpiste.
\section{Alpondras}
\begin{itemize}
\item {Grp. gram.:f. pl.}
\end{itemize}
Pedras, collocadas, de margem a margem, num regato ou rio, para dar passagem; poldras.
(Por \textunderscore alpoldras\textunderscore , de \textunderscore al\textunderscore  + \textunderscore poldra\textunderscore ?)
\section{Alporama}
\begin{itemize}
\item {Grp. gram.:m.}
\end{itemize}
\begin{itemize}
\item {Proveniência:(De \textunderscore Alpes\textunderscore , n. p. + gr. \textunderscore orama\textunderscore )}
\end{itemize}
Vista dos Alpes, em quadro.
\section{Alporão}
\begin{itemize}
\item {Grp. gram.:m.}
\end{itemize}
\begin{itemize}
\item {Utilização:Ant.}
\end{itemize}
Torre de mesquita; almenara.
Mesquita.
\section{Alporca}
\begin{itemize}
\item {Grp. gram.:f.}
\end{itemize}
Designação vulgar das escrófulas.
Alporque.
(Cp. \textunderscore alporque\textunderscore )
\section{Alporcar}
\begin{itemize}
\item {Grp. gram.:v. t.}
\end{itemize}
\begin{itemize}
\item {Proveniência:(De \textunderscore alporque\textunderscore )}
\end{itemize}
Mergulhar na terra (parte de uma planta, para se reproduzir).
\section{Alporque}
\begin{itemize}
\item {Grp. gram.:m.}
\end{itemize}
Mergulhia.
Ramo, que se mergulha na terra, para sêr reproduzido.
Acto ou effeito de alporcar.
\section{Alporquento}
\begin{itemize}
\item {Grp. gram.:adj.}
\end{itemize}
\begin{itemize}
\item {Proveniência:(De \textunderscore alporca\textunderscore )}
\end{itemize}
Que tem escrófulas.
\section{Alpostiz}
\begin{itemize}
\item {Grp. gram.:m.}
\end{itemize}
Nome, que os pescadores de Buarcos dão ao cabo delgado que amarra, umas ás outras, as testas das redes da pescada.
\section{Alquando}
\begin{itemize}
\item {Grp. gram.:adv.}
\end{itemize}
\begin{itemize}
\item {Proveniência:(Do lat. \textunderscore aliquando\textunderscore )}
\end{itemize}
Algumas vezes. Cf. Filinto, XIII, 33.
\section{Alquebrado}
\begin{itemize}
\item {Grp. gram.:adj.}
\end{itemize}
\begin{itemize}
\item {Proveniência:(De \textunderscore alquebrar\textunderscore )}
\end{itemize}
Enfraquecido; adoentado.
Curvado, por doença ou idade.
\section{Alquebramento}
\begin{itemize}
\item {Grp. gram.:m.}
\end{itemize}
Acto de \textunderscore alquebrar\textunderscore .
\section{Alquebrar}
\begin{itemize}
\item {Grp. gram.:v. i.}
\end{itemize}
\begin{itemize}
\item {Grp. gram.:V. t.}
\end{itemize}
Soffrer curvatura na espinha dorsal, por fraqueza ou doença.
Enfraquecer.
Quebrar (o navio) pelas cintas do costado.
Causar fraqueza a.
Quebrar.
\section{Alquebre}
\begin{itemize}
\item {Grp. gram.:m.}
\end{itemize}
(V.alquebramento)
\section{Alquebre}
\begin{itemize}
\item {Grp. gram.:m.}
\end{itemize}
\begin{itemize}
\item {Utilização:Ant.}
\end{itemize}
O mesmo que \textunderscore alqueire\textunderscore ? Cf. Soropita, \textunderscore Poesias\textunderscore , 99.
\section{Alqueirado}
\begin{itemize}
\item {Grp. gram.:adj.}
\end{itemize}
Medido aos alqueires.
\section{Alqueiramento}
\begin{itemize}
\item {Grp. gram.:m.}
\end{itemize}
Acto de \textunderscore alqueirar\textunderscore .
\section{Alqueirão}
\begin{itemize}
\item {Grp. gram.:m.}
\end{itemize}
\begin{itemize}
\item {Utilização:Prov.}
\end{itemize}
\begin{itemize}
\item {Utilização:alent.}
\end{itemize}
Medida de um alqueire de trigo.
\section{Alqueirar}
\begin{itemize}
\item {Grp. gram.:v. t.}
\end{itemize}
Medir aos alqueires.
Calcular por alqueires (a semeadura ou o producto da terra).
\section{Alqueire}
\begin{itemize}
\item {Grp. gram.:m.}
\end{itemize}
\begin{itemize}
\item {Utilização:Açor}
\end{itemize}
\begin{itemize}
\item {Proveniência:(Do ár. \textunderscore al-queil\textunderscore )}
\end{itemize}
Antiga medida de capacidade, para secos e líquidos.
Terreno, que leva um alqueire de semeadura.
Furo da roda, em que entra o eixo do carro.
\section{Alqueireiro}
\begin{itemize}
\item {Grp. gram.:m.}
\end{itemize}
Fabricante de alqueires e medidas semelhantes.
\section{Alqueivar}
\begin{itemize}
\item {Grp. gram.:v. t.}
\end{itemize}
\begin{itemize}
\item {Proveniência:(Do lat. hyp. \textunderscore evellicare\textunderscore , segundo Cornu)}
\end{itemize}
Pôr de alqueive.
Lavrar (terra que se não semeia por um ou mais annos, para adquirir fôrça productiva).
\section{Alqueive}
\begin{itemize}
\item {Grp. gram.:m.}
\end{itemize}
\begin{itemize}
\item {Proveniência:(De \textunderscore alqueivar\textunderscore )}
\end{itemize}
Terreno alqueivado, ou terra que se lavrou e se deixou em poisio.
Estado da terra alqueivada.
\section{Alquemila}
\begin{itemize}
\item {Grp. gram.:f.}
\end{itemize}
Planta, da fam. das rosáceas.
\section{Alquequenje}
\begin{itemize}
\item {Grp. gram.:m.}
\end{itemize}
\begin{itemize}
\item {Proveniência:(Do ár. \textunderscore al-caquenje\textunderscore )}
\end{itemize}
Planta herbácea e medicinal, da fam. das solanáceas.
\section{Alqueria}
\begin{itemize}
\item {Grp. gram.:f.}
\end{itemize}
\begin{itemize}
\item {Utilização:Ant.}
\end{itemize}
(V. \textunderscore alcaria\textunderscore ^1)
\section{Alquermes}
\begin{itemize}
\item {Grp. gram.:m.}
\end{itemize}
\begin{itemize}
\item {Proveniência:(Do ár. \textunderscore al-quirmiz\textunderscore )}
\end{itemize}
Espécie de licor napolitano.
Confecção pharmacêutica, muito excitante e hoje desusada.
\section{Alqueve}
\textunderscore m.\textunderscore  (e der.) \textunderscore Prov.\textunderscore 
O mesmo que \textunderscore alqueive\textunderscore , etc. (Colhido em Turquel)
\section{Alquiar}
\textunderscore v. i.\textunderscore  (e der)
O mesmo que \textunderscore alquilar\textunderscore , etc.
\section{Alquicé}
\begin{itemize}
\item {Grp. gram.:m.}
\end{itemize}
\begin{itemize}
\item {Utilização:Ant.}
\end{itemize}
\begin{itemize}
\item {Grp. gram.:m.}
\end{itemize}
\begin{itemize}
\item {Utilização:Ant.}
\end{itemize}
\begin{itemize}
\item {Proveniência:(Do ár. \textunderscore al-quicé\textunderscore )}
\end{itemize}
Capa moirisca.
Pequeno enxergão, de que usavam os Árabes.
\section{Alquicel}
\begin{itemize}
\item {Grp. gram.:m.}
\end{itemize}
\begin{itemize}
\item {Utilização:Ant.}
\end{itemize}
\begin{itemize}
\item {Grp. gram.:m.}
\end{itemize}
\begin{itemize}
\item {Utilização:Ant.}
\end{itemize}
\begin{itemize}
\item {Proveniência:(Do ár. \textunderscore al-quicé\textunderscore )}
\end{itemize}
Capa moirisca.
Pequeno enxergão, de que usavam os Árabes.
\section{Alquicér}
\begin{itemize}
\item {Grp. gram.:m.}
\end{itemize}
\begin{itemize}
\item {Utilização:Ant.}
\end{itemize}
\begin{itemize}
\item {Grp. gram.:m.}
\end{itemize}
\begin{itemize}
\item {Utilização:Ant.}
\end{itemize}
\begin{itemize}
\item {Proveniência:(Do ár. \textunderscore al-quicé\textunderscore )}
\end{itemize}
Capa moirisca.
Pequeno enxergão, de que usavam os Árabes.
\section{Alquier}
\begin{itemize}
\item {Grp. gram.:m.}
\end{itemize}
\begin{itemize}
\item {Utilização:Ant.}
\end{itemize}
O mesmo que \textunderscore aluguel\textunderscore .
\section{Alquiéz}
\begin{itemize}
\item {Grp. gram.:m.}
\end{itemize}
\begin{itemize}
\item {Utilização:Ant.}
\end{itemize}
Medida, de que usavam os curtidores, na venda de sola.
(Ár. \textunderscore alquies\textunderscore )
\section{Alquifol}
\begin{itemize}
\item {Grp. gram.:m.}
\end{itemize}
O mesmo que \textunderscore alquifu\textunderscore .
\section{Alquifu}
\begin{itemize}
\item {Grp. gram.:m.}
\end{itemize}
\begin{itemize}
\item {Proveniência:(Do ár. \textunderscore al-quifusxe\textunderscore )}
\end{itemize}
Galena, ou minério de chumbo sulfurado.
Pós de galena, com que as mulheres orientaes pintam as sobrancelhas.
\section{Alquilador}
\begin{itemize}
\item {Grp. gram.:m.}
\end{itemize}
Aquelle que alquila.
\section{Alquilar}
\begin{itemize}
\item {Grp. gram.:v. t.}
\end{itemize}
\begin{itemize}
\item {Proveniência:(De \textunderscore alquilé\textunderscore )}
\end{itemize}
O mesmo que \textunderscore alugar\textunderscore  (especialmente bêstas para transporte).
\section{Alquilaria}
\begin{itemize}
\item {Grp. gram.:f.}
\end{itemize}
Contrato de alquilar.
Profissão de alquilador.
\section{Alquilé}
\begin{itemize}
\item {Grp. gram.:m.}
\end{itemize}
\begin{itemize}
\item {Grp. gram.:m.}
\end{itemize}
\begin{itemize}
\item {Proveniência:(Do ár. \textunderscore alquiré\textunderscore )}
\end{itemize}
O mesmo que \textunderscore aluguer\textunderscore , especialmente de cavalgaduras.
O mesmo que \textunderscore alquilador\textunderscore .
\section{Alquiler}
\begin{itemize}
\item {Grp. gram.:m.}
\end{itemize}
\begin{itemize}
\item {Grp. gram.:m.}
\end{itemize}
\begin{itemize}
\item {Proveniência:(Do ár. \textunderscore alquiré\textunderscore )}
\end{itemize}
O mesmo que \textunderscore aluguer\textunderscore , especialmente de cavalgaduras.
O mesmo que \textunderscore alquilador\textunderscore .
\section{Alquimão}
\begin{itemize}
\item {Grp. gram.:m.}
\end{itemize}
O mesmo que alquimau.
\section{Alquimau}
\begin{itemize}
\item {Grp. gram.:m.}
\end{itemize}
O mesmo que \textunderscore gallinha-sultana\textunderscore .
\section{Alquime}
\begin{itemize}
\item {Grp. gram.:m.}
\end{itemize}
Oiro falso.
Pechisbeque.
(Cp. \textunderscore alquimia\textunderscore )
\section{Alquimia}
\begin{itemize}
\item {Grp. gram.:f.}
\end{itemize}
\begin{itemize}
\item {Proveniência:(Do ár. \textunderscore al\textunderscore  + egýp. \textunderscore kema\textunderscore , sciência por excellência)}
\end{itemize}
Chímica da Idade Média; arte chimérica, que procurava a pedra philosophal e a panaceia universal.
\section{Alquímico}
\begin{itemize}
\item {Grp. gram.:adj.}
\end{itemize}
Relativo á \textunderscore alquímia\textunderscore .
\section{Alquimista}
\begin{itemize}
\item {Grp. gram.:m.}
\end{itemize}
Aquelle que se dedicava a estudos e a trabalhos de alquimia.
\section{Alquinar}
\begin{itemize}
\item {Grp. gram.:v. t.}
\end{itemize}
\begin{itemize}
\item {Utilização:Prov.}
\end{itemize}
\begin{itemize}
\item {Utilização:trasm.}
\end{itemize}
\begin{itemize}
\item {Utilização:Fam.}
\end{itemize}
\textunderscore Estar para as alquinar\textunderscore , estar em risco de ir para a outra vida, com uma indigestão.
Morrer.
\section{Alquitão}
\begin{itemize}
\item {Grp. gram.:m.}
\end{itemize}
\begin{itemize}
\item {Utilização:Ant.}
\end{itemize}
\begin{itemize}
\item {Proveniência:(Do ár. \textunderscore al-quitam\textunderscore ?)}
\end{itemize}
Carreta para transporte de mulheres.
\section{Alquitara}
\begin{itemize}
\item {Grp. gram.:f.}
\end{itemize}
\begin{itemize}
\item {Proveniência:(Do ár. \textunderscore al-cattara\textunderscore )}
\end{itemize}
Apparelho destillatório, semelhante ao alambique, mas sem serpentina.
\section{Alquitarra}
\begin{itemize}
\item {Grp. gram.:f.}
\end{itemize}
\begin{itemize}
\item {Utilização:Prov.}
\end{itemize}
\begin{itemize}
\item {Utilização:trasm.}
\end{itemize}
Alambique.
(Cp. \textunderscore alquitara\textunderscore )
\section{Alquitira}
\begin{itemize}
\item {Grp. gram.:f.}
\end{itemize}
O mesmo que \textunderscore alcatira\textunderscore .
\section{Alquitrave}
\begin{itemize}
\item {Grp. gram.:f.}
\end{itemize}
(Corr. de \textunderscore architrave\textunderscore )
\section{Alquorques}
\begin{itemize}
\item {Grp. gram.:m. pl.}
\end{itemize}
Chapins antigos.
\section{Alrete}
\begin{itemize}
\item {fónica:rê}
\end{itemize}
\begin{itemize}
\item {Grp. gram.:m.}
\end{itemize}
Ave de rapina, semelhante ao corvo.
\section{Alriota}
\begin{itemize}
\item {Grp. gram.:f.}
\end{itemize}
\begin{itemize}
\item {Utilização:Prov.}
\end{itemize}
Galhofa, risota.
\section{Alrotado}
\begin{itemize}
\item {Grp. gram.:adj.}
\end{itemize}
\begin{itemize}
\item {Utilização:Ant.}
\end{itemize}
\begin{itemize}
\item {Proveniência:(De \textunderscore alrotar\textunderscore )}
\end{itemize}
Perseguido com vaias; escarnecido.
\section{Alrotador}
\begin{itemize}
\item {Grp. gram.:m.}
\end{itemize}
Aquelle que alrota.
\section{Alrotar}
\begin{itemize}
\item {Grp. gram.:v. i.}
\end{itemize}
\begin{itemize}
\item {Utilização:Ant.}
\end{itemize}
Fazer algazarra, surriada.
Bradar.
Pedir esmola, com grande clamor.
(Refl. de \textunderscore alvorotar\textunderscore ?)
\section{Alrotar}
\begin{itemize}
\item {Grp. gram.:v. i.}
\end{itemize}
\begin{itemize}
\item {Utilização:Pop.}
\end{itemize}
O mesmo que \textunderscore arrotar\textunderscore .
\section{Alrotaria}
\begin{itemize}
\item {Grp. gram.:f.}
\end{itemize}
\begin{itemize}
\item {Utilização:Ant.}
\end{itemize}
\begin{itemize}
\item {Proveniência:(De \textunderscore alrotar\textunderscore ^2)}
\end{itemize}
Algazarra; vozearia.
Escárneo ruidoso.
\section{Alrute}
\begin{itemize}
\item {Grp. gram.:m.}
\end{itemize}
O mesmo que \textunderscore abelheiro\textunderscore  ou \textunderscore abelharuco\textunderscore .
\section{Alsácia}
\begin{itemize}
\item {Grp. gram.:f.}
\end{itemize}
Espécie de tabaco. Cf. \textunderscore Inquér. Industr.\textunderscore , II, 334.
\section{Alsaciano}
\begin{itemize}
\item {Grp. gram.:adj.}
\end{itemize}
\begin{itemize}
\item {Grp. gram.:m.}
\end{itemize}
Relativo á Alsácia.
Habitante da Alsácia.
\section{Alsine}
\begin{itemize}
\item {Grp. gram.:f.}
\end{itemize}
Planta, que serve de tipo ás \textunderscore alsíneas\textunderscore .
\section{Alsíneas}
\begin{itemize}
\item {Grp. gram.:f.}
\end{itemize}
\begin{itemize}
\item {Proveniência:(Do gr. \textunderscore alsine\textunderscore )}
\end{itemize}
Família de plantas, que têm por typo a orelha-de-toupeira.
\section{Alsodíneas}
\begin{itemize}
\item {Grp. gram.:f. pl.}
\end{itemize}
\begin{itemize}
\item {Proveniência:(Do gr. \textunderscore alsos\textunderscore )}
\end{itemize}
Tribo de plantas violáceas, segundo De-Candolle.
\section{Alsona}
\begin{itemize}
\item {Grp. gram.:f.}
\end{itemize}
Árvore da Guiné portuguesa.
\section{Alstónia}
\begin{itemize}
\item {Grp. gram.:f.}
\end{itemize}
\begin{itemize}
\item {Proveniência:(De \textunderscore Alston\textunderscore , n. p.)}
\end{itemize}
Gênero de plantas apocýneas da Ásia e da Oceânia.
\section{Alstroméria}
\begin{itemize}
\item {Grp. gram.:f.}
\end{itemize}
Planta ornamental.
\section{Alta}
\begin{itemize}
\item {Grp. gram.:f.}
\end{itemize}
\begin{itemize}
\item {Utilização:Ant.}
\end{itemize}
Elevação (de preço, de cotação).
Aumento.
Nota, licença, para sair do hospital.
Demora, paragem.
Espécie de dança, em que se erguem muito os pés.
(Fem. de \textunderscore alto\textunderscore )
\section{Altabaixo}
\begin{itemize}
\item {Grp. gram.:m.}
\end{itemize}
\begin{itemize}
\item {Proveniência:(De \textunderscore alto\textunderscore  + \textunderscore abaixo\textunderscore )}
\end{itemize}
Golpe, de alto abaixo, em esgrima.
Pancada, de alto abaixo.
\section{Alta-e-baixa}
\begin{itemize}
\item {Grp. gram.:f.}
\end{itemize}
\begin{itemize}
\item {Utilização:Prov.}
\end{itemize}
\begin{itemize}
\item {Utilização:alent.}
\end{itemize}
Designação, que se dá a uma mulher ou a uma menina, para a reprehender suavemente.
Designação depreciativa, que se applica a uma mulher, em substituição de termos injuriosos.
\section{Altaforma}
\begin{itemize}
\item {Grp. gram.:f.}
\end{itemize}
\begin{itemize}
\item {Proveniência:(Do ár. \textunderscore altaforma\textunderscore )}
\end{itemize}
Ave de rapina, azul.
\section{Altaico}
\begin{itemize}
\item {Grp. gram.:adj.}
\end{itemize}
Relativo ao Altai ou aos povos do Altai.
\section{Altaíta}
\begin{itemize}
\item {Grp. gram.:f.}
\end{itemize}
Variedade de chumbo, descoberta no Altai.
\section{Altamado}
\begin{itemize}
\item {Grp. gram.:adj.}
\end{itemize}
De todas as qualidades: \textunderscore panos altamados\textunderscore . Cf. Viana, \textunderscore Apostilas\textunderscore .
\section{Altamala}
\begin{itemize}
\item {Grp. gram.:adv.}
\end{itemize}
\begin{itemize}
\item {Utilização:Ant.}
\end{itemize}
Á pressa.
Sem escolha; indistintamente.
(Cp. \textunderscore alt'-e-malo\textunderscore )
\section{Altamente}
\begin{itemize}
\item {Grp. gram.:adv.}
\end{itemize}
Em lugar alto.
Em voz alta.
Em alto grau.
Profundamente: \textunderscore livro altamente philosóphico\textunderscore .
Nobremente.
\section{Altamia}
\begin{itemize}
\item {Grp. gram.:f.}
\end{itemize}
\begin{itemize}
\item {Utilização:Ant.}
\end{itemize}
\begin{itemize}
\item {Proveniência:(Do ár. \textunderscore as-soltamia\textunderscore , segundo Dozy)}
\end{itemize}
Espécie de tijela vidrada.
\section{Altamisa}
\begin{itemize}
\item {Grp. gram.:f.}
\end{itemize}
Planta do Peru, ainda não classificada.
\section{Altanadice}
\begin{itemize}
\item {Grp. gram.:f.}
\end{itemize}
Qualidade de \textunderscore altanado\textunderscore :«\textunderscore eu te farei perder a altanadice\textunderscore ». Castilho, \textunderscore Doente de Scisma\textunderscore , 44.
\section{Altanado}
\begin{itemize}
\item {Grp. gram.:adj.}
\end{itemize}
\begin{itemize}
\item {Grp. gram.:M.}
\end{itemize}
\begin{itemize}
\item {Utilização:Gír.}
\end{itemize}
Leviano.
Indivíduo leviano, estroina.
Juiz.
\section{Altanaria}
\begin{itemize}
\item {Grp. gram.:f.}
\end{itemize}
\begin{itemize}
\item {Proveniência:(De \textunderscore altanar\textunderscore )}
\end{itemize}
Altivez; soberba.
Qualidade da caça que vôa alto.
Caça de aves que vôam alto, como, em geral, as de rapina.
\section{Altanar-se}
\begin{itemize}
\item {Grp. gram.:v. p.}
\end{itemize}
\begin{itemize}
\item {Proveniência:(Do lat. \textunderscore altanus\textunderscore )}
\end{itemize}
Fazer-se leviano, soberbo, alvoroçado.
\section{Altancar}
\begin{itemize}
\item {Grp. gram.:m.}
\end{itemize}
Antigo instrumento de música? Cf. Cortesão, \textunderscore Subsídios\textunderscore .
\section{Altaneiro}
\begin{itemize}
\item {Grp. gram.:adj.}
\end{itemize}
Que se eleva muito.
Soberbo.
Altanado.
(Cast. \textunderscore altanero\textunderscore )
\section{Altar}
\begin{itemize}
\item {Grp. gram.:m.}
\end{itemize}
\begin{itemize}
\item {Utilização:Gír.}
\end{itemize}
\begin{itemize}
\item {Grp. gram.:Pl.}
\end{itemize}
\begin{itemize}
\item {Utilização:T. de Azeméis}
\end{itemize}
Mesa para os sacrifícios, nas religiões pagans.
Mesa, onde se diz missa.
Culto, veneração.
Constellação austral.
Mesa de jantar.
Grandes mamas.
\section{Altaragem}
\begin{itemize}
\item {Grp. gram.:f.}
\end{itemize}
\begin{itemize}
\item {Proveniência:(De \textunderscore altar\textunderscore )}
\end{itemize}
Direitos sôbre as offerendas da igreja.
\section{Altareiro}
\begin{itemize}
\item {Grp. gram.:m.}
\end{itemize}
\begin{itemize}
\item {Proveniência:(De \textunderscore altar\textunderscore )}
\end{itemize}
Aquelle que é beato.
Aquelle que tem tendência para serviços de igreja.
Aquelle que tem a seu cargo a limpeza dos altares.
\section{Altarista}
\begin{itemize}
\item {Grp. gram.:m.}
\end{itemize}
\begin{itemize}
\item {Proveniência:(De \textunderscore altar\textunderscore )}
\end{itemize}
Cónego, que, na basílica do Vaticano, cuida do altar-mór e dos frontaes.
\section{Altar-mór}
\begin{itemize}
\item {Grp. gram.:m.}
\end{itemize}
Altar principal, que fica na extremidade opposta á entrada principal da igreja.
\section{Altarum}
\begin{itemize}
\item {Grp. gram.:m.}
\end{itemize}
\begin{itemize}
\item {Utilização:Prov.}
\end{itemize}
\begin{itemize}
\item {Utilização:alg.}
\end{itemize}
Elevação, eminência.
O mesmo que \textunderscore altaruz\textunderscore .
\section{Altaruz}
\begin{itemize}
\item {Grp. gram.:m.}
\end{itemize}
\begin{itemize}
\item {Utilização:Prov.}
\end{itemize}
\begin{itemize}
\item {Utilização:alent.}
\end{itemize}
\begin{itemize}
\item {Proveniência:(De \textunderscore alto\textunderscore )}
\end{itemize}
Entumescência; tumor.
\section{Altavela}
\begin{itemize}
\item {Grp. gram.:f.}
\end{itemize}
Peixe cartilagíneo, plagióstomo, (\textunderscore trigon pastinaca\textunderscore ).
\section{Alteação}
\begin{itemize}
\item {Grp. gram.:f.}
\end{itemize}
O mesmo que \textunderscore alteamento\textunderscore .
\section{Alteamento}
\begin{itemize}
\item {Grp. gram.:m.}
\end{itemize}
Acto de \textunderscore altear\textunderscore .
\section{Altear}
\begin{itemize}
\item {Grp. gram.:v. t.}
\end{itemize}
\begin{itemize}
\item {Grp. gram.:V. i.}
\end{itemize}
Tornar alto, mais alto.
Elevar, elevar mais.
Tornar-se mais alto.
Subir.
\section{Alteastro}
\begin{itemize}
\item {Grp. gram.:m.}
\end{itemize}
Subgênero de altheia, em que se comprehende o malvaísco.
\section{Alteia}
\begin{itemize}
\item {Grp. gram.:f.}
\end{itemize}
\begin{itemize}
\item {Proveniência:(Lat. \textunderscore althaea\textunderscore )}
\end{itemize}
Planta medicinal, da fam. das malváceas.
\section{Alt'-e-malo}
\begin{itemize}
\item {Grp. gram.:loc. adv.}
\end{itemize}
Indistintamente; no conjunto:«\textunderscore daremos... a toda a roda alto e malo a senhoria\textunderscore ». Tolentino, \textunderscore Obras\textunderscore , I, 178.
\section{Altenado}
\begin{itemize}
\item {Grp. gram.:m.}
\end{itemize}
\begin{itemize}
\item {Utilização:ant.}
\end{itemize}
\begin{itemize}
\item {Utilização:Gír.}
\end{itemize}
\begin{itemize}
\item {Proveniência:(De \textunderscore alto\textunderscore . Cp. \textunderscore altanado\textunderscore )}
\end{itemize}
Amo.
\section{Altênia}
\begin{itemize}
\item {Grp. gram.:f.}
\end{itemize}
\begin{itemize}
\item {Proveniência:(De \textunderscore Althen\textunderscore , n. p.)}
\end{itemize}
Planta, da fam. das naiádeas.
\section{Alterábil}
\begin{itemize}
\item {Grp. gram.:adj.}
\end{itemize}
Fórma alat. de \textunderscore alterável\textunderscore .
\section{Alterabilidade}
\begin{itemize}
\item {Grp. gram.:f.}
\end{itemize}
Qualidade do que é \textunderscore alterável\textunderscore .
\section{Alteração}
\begin{itemize}
\item {Grp. gram.:f.}
\end{itemize}
Acção ou effeito de \textunderscore alterar\textunderscore .
Degeneração.
Decomposição.
Inquietação.
Desordem.
Altercação.
\section{Alteradamente}
\begin{itemize}
\item {Grp. gram.:adv.}
\end{itemize}
Com alteração.
\section{Alterado}
\begin{itemize}
\item {Grp. gram.:adj.}
\end{itemize}
\begin{itemize}
\item {Proveniência:(De \textunderscore alterar\textunderscore )}
\end{itemize}
Falsificado.
Corrompido.
Irritado.
\section{Alterador}
\begin{itemize}
\item {Grp. gram.:m.  e  adj.}
\end{itemize}
O que altera.
\section{Alterante}
\begin{itemize}
\item {Grp. gram.:adj.}
\end{itemize}
Que altera.
\section{Alterar}
\begin{itemize}
\item {Grp. gram.:v. t.}
\end{itemize}
\begin{itemize}
\item {Utilização:Ant.}
\end{itemize}
\begin{itemize}
\item {Proveniência:(Do lat. \textunderscore alter\textunderscore )}
\end{itemize}
Mudar: \textunderscore alterar a ordem pública\textunderscore .
Falsificar; decompor; corromper: \textunderscore alterar o vinho\textunderscore .
Desfigurar.
Excitar; agitar; revoltar; alvorotar.
Tornar irado.
Fazer têr sêde, (donde \textunderscore desalterar\textunderscore , matar a sêde).
\section{Alterativo}
\begin{itemize}
\item {Grp. gram.:adj.}
\end{itemize}
O mesmo que \textunderscore alterante\textunderscore  e \textunderscore alterador\textunderscore .
\section{Alterável}
\begin{itemize}
\item {Grp. gram.:adj.}
\end{itemize}
Que póde sêr alterado.
\section{Altercação}
\begin{itemize}
\item {Grp. gram.:f.}
\end{itemize}
\begin{itemize}
\item {Proveniência:(Lat. \textunderscore altercatio\textunderscore )}
\end{itemize}
Acto de altercar.
\section{Altercado}
\begin{itemize}
\item {Grp. gram.:adj.}
\end{itemize}
Que é objecto de altercação.
\section{Altercador}
\begin{itemize}
\item {Grp. gram.:m.}
\end{itemize}
\begin{itemize}
\item {Proveniência:(Lat. \textunderscore altercator\textunderscore )}
\end{itemize}
Aquelle que alterca.
\section{Altercante}
\begin{itemize}
\item {Grp. gram.:adj.}
\end{itemize}
Que alterca.
\section{Altercar}
\begin{itemize}
\item {Grp. gram.:v. i.}
\end{itemize}
\begin{itemize}
\item {Proveniência:(Lat. \textunderscore altercare\textunderscore )}
\end{itemize}
Disputar apaixonadamente.
Argumentar; discutir com vivacidade.
Provocar polêmicas.
\section{Alterco}
\begin{itemize}
\item {Grp. gram.:m.}
\end{itemize}
O mesmo que \textunderscore altercação\textunderscore . Cf. Castilho, \textunderscore Fastos\textunderscore , III, 357.
\section{Alter-ego}
\begin{itemize}
\item {fónica:áltèrégò}
\end{itemize}
\begin{itemize}
\item {Grp. gram.:m.}
\end{itemize}
Outro eu.
Pessôa, em quem eu deposito a maior confiança.
(Loc. lat.)
\section{Alternação}
\begin{itemize}
\item {Grp. gram.:f.}
\end{itemize}
\begin{itemize}
\item {Proveniência:(Lat. \textunderscore alternatio\textunderscore )}
\end{itemize}
Acto de alternar.
\section{Alternadamente}
\begin{itemize}
\item {Grp. gram.:adv.}
\end{itemize}
De modo \textunderscore alternado\textunderscore .
\section{Alternado}
\begin{itemize}
\item {Grp. gram.:adj.}
\end{itemize}
\begin{itemize}
\item {Proveniência:(De \textunderscore alternar\textunderscore )}
\end{itemize}
Que ora é, ora não é.
Que ora precede, ora succede.
Que ora está de um lado, ora de outro.
\section{Alternador}
\begin{itemize}
\item {Grp. gram.:m.}
\end{itemize}
Aquelle que alterna.
\section{Alternamente}
\begin{itemize}
\item {Grp. gram.:adv.}
\end{itemize}
(V.alternadamente)
\section{Alternância}
\begin{itemize}
\item {Grp. gram.:f.}
\end{itemize}
\begin{itemize}
\item {Proveniência:(De \textunderscore alternar\textunderscore )}
\end{itemize}
Disposição das peças de um verticillo, quando collocadas ao nível dos intervalos de um verticillo superior.
Disposição geológica de alguns depósitos estratificados, separados por uma espessura.
\section{Alternante}
\begin{itemize}
\item {Grp. gram.:adj.}
\end{itemize}
\begin{itemize}
\item {Proveniência:(Lat. \textunderscore alternans\textunderscore )}
\end{itemize}
Que alterna.
\section{Alternantera}
\begin{itemize}
\item {Grp. gram.:f.}
\end{itemize}
\begin{itemize}
\item {Proveniência:(De \textunderscore alterno\textunderscore  + \textunderscore anthera\textunderscore )}
\end{itemize}
Gênero de plantas amarantáceas.
\section{Alternanthera}
\begin{itemize}
\item {Grp. gram.:f.}
\end{itemize}
\begin{itemize}
\item {Proveniência:(De \textunderscore alterno\textunderscore  + \textunderscore anthera\textunderscore )}
\end{itemize}
Gênero de plantas amarantáceas.
\section{Alternar}
\begin{itemize}
\item {Grp. gram.:v. t.}
\end{itemize}
\begin{itemize}
\item {Proveniência:(Lat. \textunderscore alternare\textunderscore )}
\end{itemize}
Revezar; fazer variar successivamente.
Collocar em posições recíprocas.
Dispor em ordem alternada.
\section{Alternária}
\begin{itemize}
\item {Grp. gram.:f.}
\end{itemize}
Gênero de cogumelos.
\section{Alternativa}
\begin{itemize}
\item {Grp. gram.:f.}
\end{itemize}
\begin{itemize}
\item {Utilização:Taur.}
\end{itemize}
\begin{itemize}
\item {Proveniência:(De \textunderscore alternativo\textunderscore )}
\end{itemize}
Successão de duas coisas, cada uma por sua vez.
Opção entre duas coisas.
Formalidade, com que o espada entrega a muleta ao novilheiro, autorizando-o a, em sua substituição, matar o toiro.
\section{Alternativamente}
\begin{itemize}
\item {Grp. gram.:adv.}
\end{itemize}
Revezadamente.
Com alternação.
De modo \textunderscore alternativo\textunderscore .
\section{Alternativo}
\begin{itemize}
\item {Grp. gram.:adj.}
\end{itemize}
\begin{itemize}
\item {Proveniência:(De \textunderscore alternar\textunderscore )}
\end{itemize}
Que vem por sua vez.
Que se faz com alternação.
E diz-se das coisas, das quaes se póde escolher a que mais convenha.
\section{Alternato}
\begin{itemize}
\item {Grp. gram.:m.}
\end{itemize}
Acção ou systema de \textunderscore alternar\textunderscore .
\section{Alternatura}
\begin{itemize}
\item {Grp. gram.:f.}
\end{itemize}
\begin{itemize}
\item {Utilização:P. us.}
\end{itemize}
Acto de \textunderscore alternar\textunderscore .
\section{Alterniflóreo}
\begin{itemize}
\item {Grp. gram.:adj.}
\end{itemize}
\begin{itemize}
\item {Utilização:Bot.}
\end{itemize}
\begin{itemize}
\item {Proveniência:(Do lat. \textunderscore alternus\textunderscore  + \textunderscore flus\textunderscore )}
\end{itemize}
Que tem flôres alternas.
\section{Alternifólio}
\begin{itemize}
\item {Grp. gram.:adj.}
\end{itemize}
\begin{itemize}
\item {Utilização:Bot.}
\end{itemize}
\begin{itemize}
\item {Proveniência:(Do lat. \textunderscore alternus\textunderscore  + \textunderscore folium\textunderscore )}
\end{itemize}
Que tem flôres alternas.
\section{Alternípede}
\begin{itemize}
\item {Grp. gram.:adj.}
\end{itemize}
\begin{itemize}
\item {Utilização:Zool.}
\end{itemize}
\begin{itemize}
\item {Proveniência:(Do lat. \textunderscore alternus\textunderscore  + \textunderscore pés\textunderscore )}
\end{itemize}
Que tem as patas alternativamente de duas côres differentes.
\section{Alternipene}
\begin{itemize}
\item {Grp. gram.:adj.}
\end{itemize}
\begin{itemize}
\item {Utilização:Zool.}
\end{itemize}
\begin{itemize}
\item {Proveniência:(Do lat. \textunderscore alternus\textunderscore  + \textunderscore penna\textunderscore )}
\end{itemize}
Diz-se das fôlhas que têm folíolos alternos em pecíolos communs.
\section{Alternipenne}
\begin{itemize}
\item {Grp. gram.:adj.}
\end{itemize}
\begin{itemize}
\item {Utilização:Zool.}
\end{itemize}
\begin{itemize}
\item {Proveniência:(Do lat. \textunderscore alternus\textunderscore  + \textunderscore penna\textunderscore )}
\end{itemize}
Diz-se das fôlhas que têm folíolos alternos em pecíolos communs.
\section{Alternipétalo}
\begin{itemize}
\item {Grp. gram.:adj.}
\end{itemize}
\begin{itemize}
\item {Utilização:Bot.}
\end{itemize}
\begin{itemize}
\item {Proveniência:(De \textunderscore alternus\textunderscore , lat. + \textunderscore petala\textunderscore )}
\end{itemize}
Diz-se dos órgãos vegetaes, que estão insertos em pontos fronteiros aos intervallos que separam as pétalas.
\section{Alterno}
\begin{itemize}
\item {Grp. gram.:adj.}
\end{itemize}
\begin{itemize}
\item {Utilização:Bot.}
\end{itemize}
\begin{itemize}
\item {Utilização:Geom.}
\end{itemize}
\begin{itemize}
\item {Proveniência:(Lat. \textunderscore alternus\textunderscore )}
\end{itemize}
Revezado.
O mesmo que \textunderscore alternado\textunderscore .
Diz-se das fôlhas, que estão collocadas de dois lados do caule, mas cada uma em posição superior
\textunderscore ou\textunderscore  inferior á correspondente do lado opposto.
Diz-se dos ângulos, que se formam de um e outro lado de uma recta que corta outras duas.
\section{Alterosamente}
\begin{itemize}
\item {Grp. gram.:adv.}
\end{itemize}
De modo \textunderscore alteroso\textunderscore .
\section{Alteroso}
\begin{itemize}
\item {Grp. gram.:adj.}
\end{itemize}
\begin{itemize}
\item {Proveniência:(De \textunderscore alto\textunderscore )}
\end{itemize}
Grandioso; soberbo; majestoso; altaneiro.
\section{Altesa}
\begin{itemize}
\item {fónica:tê}
\end{itemize}
\begin{itemize}
\item {Grp. gram.:f.}
\end{itemize}
\begin{itemize}
\item {Utilização:Prov.}
\end{itemize}
\begin{itemize}
\item {Utilização:alent.}
\end{itemize}
O mesmo que artesa.
\section{Alteza}
\begin{itemize}
\item {Grp. gram.:f.}
\end{itemize}
\begin{itemize}
\item {Proveniência:(De \textunderscore alto\textunderscore )}
\end{itemize}
Qualidade do que é alto.
Altura; elevação.
Grandeza.
Sublimidade.
Antigo título honorífico dos reis portugueses, dos filhos de reis em tempos modernos, e de outros parentes próximos de Casas reaes.
\section{Altheastro}
\begin{itemize}
\item {Grp. gram.:m.}
\end{itemize}
Subgênero de altheia, em que se comprehende o malvaísco.
\section{Altheia}
\begin{itemize}
\item {Grp. gram.:f.}
\end{itemize}
\begin{itemize}
\item {Proveniência:(Lat. \textunderscore althaea\textunderscore )}
\end{itemize}
Planta medicinal, da fam. das malváceas.
\section{Althênia}
\begin{itemize}
\item {Grp. gram.:f.}
\end{itemize}
\begin{itemize}
\item {Proveniência:(De \textunderscore Althen\textunderscore , n. p.)}
\end{itemize}
Planta, da fam. das naiádeas.
\section{Alti...}
\begin{itemize}
\item {Grp. gram.:pref.}
\end{itemize}
(design. de \textunderscore alto\textunderscore )
\section{Altibaixa}
\begin{itemize}
\item {Grp. gram.:f.}
\end{itemize}
\begin{itemize}
\item {Utilização:Ant.}
\end{itemize}
\begin{itemize}
\item {Proveniência:(De \textunderscore altibaixo\textunderscore )}
\end{itemize}
Relêvo em tapeçarias de coiro.
\section{Altibaixo}
\begin{itemize}
\item {Grp. gram.:adj.}
\end{itemize}
(Abrev. de \textunderscore alto e baixo\textunderscore )
\section{Altibôrdo}
\begin{itemize}
\item {Grp. gram.:m.}
\end{itemize}
\begin{itemize}
\item {Utilização:Ant.}
\end{itemize}
Alto bôrdo. Cf. \textunderscore Eufrosina\textunderscore , 270.
\section{Altica}
\begin{itemize}
\item {Grp. gram.:f.}
\end{itemize}
Pequeno insecto coleóptero saltador, muito nocívo ás plantas.
\section{Alticanoro}
\begin{itemize}
\item {Grp. gram.:adj.}
\end{itemize}
Que canta alto. Cf. Filinto, XI, 178.
\section{Áltico}
\begin{itemize}
\item {Grp. gram.:m.}
\end{itemize}
Nome, dado por Commerson a um peixe, que Cuvier chamou \textunderscore salarias\textunderscore .
\section{Alticolúmnio}
\begin{itemize}
\item {Grp. gram.:adj.}
\end{itemize}
\begin{itemize}
\item {Proveniência:(De \textunderscore altocolumna\textunderscore )}
\end{itemize}
Que tem columnas altas.
\section{Altícomo}
\begin{itemize}
\item {Grp. gram.:adj.}
\end{itemize}
\begin{itemize}
\item {Utilização:Bot.}
\end{itemize}
\begin{itemize}
\item {Proveniência:(De \textunderscore altocoma\textunderscore )}
\end{itemize}
Que tem folhagem elevada.
\section{Alticolúnio}
\begin{itemize}
\item {Grp. gram.:adj.}
\end{itemize}
\begin{itemize}
\item {Proveniência:(De \textunderscore alto\textunderscore  + \textunderscore columna\textunderscore )}
\end{itemize}
Que tem columnas altas.
\section{Altícopo}
\begin{itemize}
\item {Grp. gram.:m.}
\end{itemize}
\begin{itemize}
\item {Proveniência:(Do gr. \textunderscore altikos\textunderscore )}
\end{itemize}
Insecto coleóptero tetrâmero.
\section{Alticornígero}
\begin{itemize}
\item {Grp. gram.:adj.}
\end{itemize}
\begin{itemize}
\item {Proveniência:(De \textunderscore alto\textunderscore  + \textunderscore cornígero\textunderscore )}
\end{itemize}
Que tem cornos muito altos.
\section{Altiechoante}
\begin{itemize}
\item {fónica:e-co}
\end{itemize}
\begin{itemize}
\item {Grp. gram.:adj.}
\end{itemize}
Que echôa alto. Cf. Castilho, \textunderscore Fastos\textunderscore , II, 85.
\section{Altiecoante}
\begin{itemize}
\item {Grp. gram.:adj.}
\end{itemize}
Que echôa alto. Cf. Castilho, \textunderscore Fastos\textunderscore , II, 85.
\section{Altigritante}
\begin{itemize}
\item {Grp. gram.:adj.}
\end{itemize}
Que grita em tom muito elevado.
\section{Altiloquência}
\begin{itemize}
\item {fónica:cu-en}
\end{itemize}
\begin{itemize}
\item {Grp. gram.:f.}
\end{itemize}
\begin{itemize}
\item {Proveniência:(De \textunderscore alto\textunderscore  + \textunderscore eloquência\textunderscore )}
\end{itemize}
Estílo elevado.
Locução sublime.
\section{Altiloquente}
\begin{itemize}
\item {fónica:cu-en}
\end{itemize}
\begin{itemize}
\item {Grp. gram.:adj.}
\end{itemize}
Que tem altiloquência.
\section{Altiloquia}
\begin{itemize}
\item {Grp. gram.:f.}
\end{itemize}
Qualidade de altíloquo. Cf. Castilho, \textunderscore Fastos\textunderscore , II, 289.
\section{Altilóquio}
\begin{itemize}
\item {Grp. gram.:m.}
\end{itemize}
O mesmo que \textunderscore altiloquência\textunderscore .
(B. lat. \textunderscore altiloquium\textunderscore )
\section{Altíloquo}
\begin{itemize}
\item {Grp. gram.:adj.}
\end{itemize}
Sublime, elevado.
Que fala com sublimidade.
(Cp. \textunderscore altilóquio\textunderscore )
\section{Altimalo}
\begin{itemize}
\item {Grp. gram.:adv.}
\end{itemize}
O mesmo ou melhor que \textunderscore alt'-e-malo\textunderscore .
\section{Altimetria}
\textunderscore f.\textunderscore  (e der.)
O mesmo que \textunderscore hypsometria\textunderscore , etc.
\section{Altimurado}
\begin{itemize}
\item {Grp. gram.:adj.}
\end{itemize}
\begin{itemize}
\item {Proveniência:(De \textunderscore alto\textunderscore  + \textunderscore murado\textunderscore )}
\end{itemize}
O que tem muros altos.
\section{Altiperno}
\begin{itemize}
\item {Grp. gram.:adj.}
\end{itemize}
\begin{itemize}
\item {Utilização:P. us.}
\end{itemize}
O mesmo que \textunderscore pernalto\textunderscore .
\section{Altiplano}
\begin{itemize}
\item {Grp. gram.:m.}
\end{itemize}
\begin{itemize}
\item {Utilização:Neol.}
\end{itemize}
O mesmo que \textunderscore planalto\textunderscore .
\section{Altipotente}
\begin{itemize}
\item {Grp. gram.:adj.}
\end{itemize}
\begin{itemize}
\item {Proveniência:(Do lat. \textunderscore alte\textunderscore  + \textunderscore potens\textunderscore )}
\end{itemize}
Que tem alto poder; que póde muito.
\section{Altirna}
\begin{itemize}
\item {Grp. gram.:f.}
\end{itemize}
Veste sacerdotal na China. Cf. \textunderscore Peregrinação\textunderscore , CX.
\section{Altirostro}
\begin{itemize}
\item {fónica:rós}
\end{itemize}
\begin{itemize}
\item {Grp. gram.:adj.}
\end{itemize}
\begin{itemize}
\item {Proveniência:(Do lat. \textunderscore altus\textunderscore  + \textunderscore rostrum\textunderscore )}
\end{itemize}
Diz-se dos pássaros trepadores, cujo bico é mais largo que comprido.
\section{Altirrostro}
\begin{itemize}
\item {Grp. gram.:adj.}
\end{itemize}
\begin{itemize}
\item {Proveniência:(Do lat. \textunderscore altus\textunderscore  + \textunderscore rostrum\textunderscore )}
\end{itemize}
Diz-se dos pássaros trepadores, cujo bico é mais largo que comprido.
\section{Altisa}
\begin{itemize}
\item {Grp. gram.:f.}
\end{itemize}
Gênero de insectos coleópteros, que contém 149 espécies.
Doença das vinhas, determinada pela presença de uma daquellas espécies.
\section{Altisonante}
\begin{itemize}
\item {fónica:so}
\end{itemize}
\begin{itemize}
\item {Grp. gram.:adj.}
\end{itemize}
O mesmo que altísono.
\section{Altísono}
\begin{itemize}
\item {fónica:so}
\end{itemize}
\begin{itemize}
\item {Grp. gram.:adj.}
\end{itemize}
\begin{itemize}
\item {Proveniência:(Lat. \textunderscore altisonus\textunderscore )}
\end{itemize}
Que sôa alto.
Pomposo.
\section{Altíssimo}
\begin{itemize}
\item {Grp. gram.:m.}
\end{itemize}
\begin{itemize}
\item {Grp. gram.:Adj.}
\end{itemize}
Deus.
Muito alto.
(Sup. de \textunderscore alto\textunderscore )
\section{Altissonante}
\begin{itemize}
\item {Grp. gram.:adj.}
\end{itemize}
O mesmo que altísono.
\section{Altíssono}
\begin{itemize}
\item {Grp. gram.:adj.}
\end{itemize}
\begin{itemize}
\item {Proveniência:(Lat. \textunderscore altisonus\textunderscore )}
\end{itemize}
Que sôa alto.
Pomposo.
\section{Altista}
\begin{itemize}
\item {Grp. gram.:m.  e  adj.}
\end{itemize}
\begin{itemize}
\item {Utilização:Bras}
\end{itemize}
Bolsista, que joga na alta do câmbio.
\section{Altitonante}
\begin{itemize}
\item {Grp. gram.:adj.}
\end{itemize}
\begin{itemize}
\item {Proveniência:(Lat. \textunderscore altitonans\textunderscore )}
\end{itemize}
Que troveja nas alturas.
Estrondoso.
\section{Altitude}
\begin{itemize}
\item {Grp. gram.:f.}
\end{itemize}
\begin{itemize}
\item {Proveniência:(Lat. \textunderscore altitudo\textunderscore )}
\end{itemize}
Altura, em relação ao nível do mar.
Ângulo, formado pelo horizonte e pelo raio visual dirigido a um astro.
\section{Altívago}
\begin{itemize}
\item {Grp. gram.:adj.}
\end{itemize}
Que vaga no espaço, nas alturas.
(B. lat. \textunderscore altívagus\textunderscore )
\section{Altivamente}
\begin{itemize}
\item {Grp. gram.:adv.}
\end{itemize}
De modo \textunderscore altivo\textunderscore , com altivez.
\section{Altivar}
\begin{itemize}
\item {Grp. gram.:v. t.}
\end{itemize}
\begin{itemize}
\item {Utilização:P. us.}
\end{itemize}
Tornar altivo.
Elevar.
\section{Altivez}
\begin{itemize}
\item {Grp. gram.:adv.}
\end{itemize}
Qualidade do que é altivo.
Orgulho nobre.
Arrogância.
\section{Altiveza}
\begin{itemize}
\item {Grp. gram.:f.}
\end{itemize}
O mesmo que \textunderscore altivez\textunderscore .
\section{Altividade}
\begin{itemize}
\item {Grp. gram.:f.}
\end{itemize}
\begin{itemize}
\item {Utilização:Ant.}
\end{itemize}
O mesmo que \textunderscore altivez\textunderscore .
\section{Altivo}
\begin{itemize}
\item {Grp. gram.:adj.}
\end{itemize}
\begin{itemize}
\item {Proveniência:(De \textunderscore alto\textunderscore )}
\end{itemize}
Elevado.
Brioso.
Orgulhoso; arrogante.
\section{Altivolante}
\begin{itemize}
\item {Grp. gram.:adj.}
\end{itemize}
\begin{itemize}
\item {Proveniência:(Lat. \textunderscore altivolans\textunderscore )}
\end{itemize}
O mesmo que \textunderscore altívolo\textunderscore .
\section{Altívolo}
\begin{itemize}
\item {Grp. gram.:adj.}
\end{itemize}
\begin{itemize}
\item {Proveniência:(Lat. \textunderscore altivolus\textunderscore )}
\end{itemize}
Que vôa alto.
\section{Alto}
\begin{itemize}
\item {Grp. gram.:adj.}
\end{itemize}
\begin{itemize}
\item {Grp. gram.:Adv.}
\end{itemize}
\begin{itemize}
\item {Grp. gram.:M.}
\end{itemize}
\begin{itemize}
\item {Proveniência:(Lat. \textunderscore altus\textunderscore )}
\end{itemize}
Que está muito acima: \textunderscore o alto sol\textunderscore .
Elevado.
Levantado.
Illustre: \textunderscore altas personagens\textunderscore .
Soberbo.
Profundo; grande: \textunderscore o seu alto saber\textunderscore .
Que sôa muito: \textunderscore em voz alta\textunderscore .
Que é diffícil de comprehender: \textunderscore altos mystérios\textunderscore .
Importante: \textunderscore altos interesses\textunderscore .
Supremo.
Principal.
Diz-se de uma região, relativamente á sua parte setentrional: \textunderscore o alto Minho\textunderscore .
Diz-se do rio, na região que banha, ainda longe da foz: \textunderscore o alto Nilo\textunderscore .
Diz-se das regiões elevadas, em relação ao nível do mar ou das planícies adjacentes.
Na parte mais alta.
Em som ou voz alta: \textunderscore falar alto\textunderscore .
Altura: \textunderscore pairam no alto as águias\textunderscore .
Mar alto, mar largo: \textunderscore peixe do alto\textunderscore ; \textunderscore pescador do alto\textunderscore .
\section{Alto!}
\begin{itemize}
\item {Grp. gram.:interj.}
\end{itemize}
\begin{itemize}
\item {Grp. gram.:M.}
\end{itemize}
\begin{itemize}
\item {Proveniência:(Do germ. \textunderscore kalten\textunderscore ?)}
\end{itemize}
(para mandar parar)
Acto de parar ou suspender marcha; \textunderscore fizemos alto\textunderscore ; \textunderscore faça alto\textunderscore .
\section{Alto-e-malo}
\begin{itemize}
\item {Grp. gram.:loc. adv.}
\end{itemize}
O mesmo que \textunderscore alt'-e-malo\textunderscore .
\section{Alto-navarro}
\begin{itemize}
\item {Grp. gram.:m.}
\end{itemize}
Um dos dialectos do vasconço.
\section{Altoplano}
\begin{itemize}
\item {Grp. gram.:m.}
\end{itemize}
O mesmo que \textunderscore planalto\textunderscore . Cf. Cortesão, \textunderscore Subs\textunderscore .
\section{Altor}
\begin{itemize}
\item {Grp. gram.:m.  e  adj.}
\end{itemize}
\begin{itemize}
\item {Proveniência:(Lat. \textunderscore altor\textunderscore )}
\end{itemize}
O que nutre ou sustenta.
\section{Altor}
\begin{itemize}
\item {Grp. gram.:m.}
\end{itemize}
\begin{itemize}
\item {Utilização:Prov.}
\end{itemize}
\begin{itemize}
\item {Utilização:trasm.}
\end{itemize}
O mesmo que \textunderscore altura\textunderscore .
\section{Altosa}
\begin{itemize}
\item {Grp. gram.:f.}
\end{itemize}
\begin{itemize}
\item {Proveniência:(De \textunderscore alto\textunderscore )}
\end{itemize}
Lan comprida.
\section{Altriz}
\begin{itemize}
\item {Grp. gram.:adj. f.}
\end{itemize}
\begin{itemize}
\item {Grp. gram.:F.}
\end{itemize}
\begin{itemize}
\item {Proveniência:(Lat. \textunderscore altrix\textunderscore )}
\end{itemize}
Que nutre, sustenta.
A parte nutritiva de uma substância.
\section{Altruísmo}
\begin{itemize}
\item {Grp. gram.:m.}
\end{itemize}
\begin{itemize}
\item {Proveniência:(Do lat. \textunderscore alter\textunderscore )}
\end{itemize}
Amor ao próximo.
Abnegação.
Philanthropia.
\section{Altruísta}
\begin{itemize}
\item {Grp. gram.:adj.}
\end{itemize}
\begin{itemize}
\item {Grp. gram.:M.}
\end{itemize}
Relativo ao altruísmo.
Dedicado aos seus semelhantes.
Philanthropo.
(Cp. \textunderscore altruísmo\textunderscore )
\section{Altruístico}
\begin{itemize}
\item {Grp. gram.:adj.}
\end{itemize}
Relativo a \textunderscore altruismo\textunderscore .
\section{Altura}
\begin{itemize}
\item {Grp. gram.:f.}
\end{itemize}
\begin{itemize}
\item {Proveniência:(De \textunderscore alto\textunderscore )}
\end{itemize}
Dimensão de um objecto, desde a extremidade inferior á superior: \textunderscore tem 20 metros de altura\textunderscore .
Posição de um corpo, acima de uma superfície.
Elevação de um astro acima do horizonte. Superioridade; eminencia.
Importância.
Profundidade.
Cumeada; \textunderscore na altura do Marão\textunderscore .
Firmamento; \textunderscore o sol esplende na altura\textunderscore .
\section{Aluá}
\begin{itemize}
\item {Grp. gram.:m.}
\end{itemize}
Bebida, us. ao N. do Brasil, e formada por um cozimento fermentado de arroz e algumas gotas de limão.
(Do bundo \textunderscore ualuá\textunderscore ?)
\section{Aluado}
\begin{itemize}
\item {Grp. gram.:adj.}
\end{itemize}
Influenciado pela lua.
Lunático.
Adoidado.
E diz-se dos animaes que andam com cio.
\section{Aluamento}
\begin{itemize}
\item {Grp. gram.:m.}
\end{itemize}
\begin{itemize}
\item {Utilização:Náut.}
\end{itemize}
Córte curvo da esteira ou da parte inferior da vela de gávea.
Cio dos animaes.
(Cp. \textunderscore aluado\textunderscore )
\section{Aluandita}
\begin{itemize}
\item {Grp. gram.:f.}
\end{itemize}
Phosphato de manganés e de ferro.
\section{Aluata}
\begin{itemize}
\item {Grp. gram.:m.}
\end{itemize}
Gênero de macacos, que comprehende os gritadores.
\section{Aluato}
\begin{itemize}
\item {Grp. gram.:m.}
\end{itemize}
O mesmo que \textunderscore aluata\textunderscore .
\section{Alucinação}
\begin{itemize}
\item {Grp. gram.:f.}
\end{itemize}
\begin{itemize}
\item {Proveniência:(Lat. \textunderscore alucinatio\textunderscore )}
\end{itemize}
Acto ou effeito de se alucinar.
Cegueira intellectual.
Illusão; devaneio.
\section{Alucinadamente}
\begin{itemize}
\item {Grp. gram.:adv.}
\end{itemize}
De modo \textunderscore alucinado\textunderscore ; com alucinação.
\section{Alucinado}
\begin{itemize}
\item {Grp. gram.:adj.}
\end{itemize}
Privado da razão.
Apaixonado.
\section{Alucinador}
\begin{itemize}
\item {Grp. gram.:m.  e  adj.}
\end{itemize}
O que alucina.
\section{Alucinar}
\begin{itemize}
\item {Grp. gram.:v. t.}
\end{itemize}
\begin{itemize}
\item {Proveniência:(Lat. \textunderscore alucinari\textunderscore )}
\end{itemize}
Privar da razão; desvairar.
Apaixonar.
Fazer cair em illusão.
\section{Alucitas}
\begin{itemize}
\item {Grp. gram.:f. pl.}
\end{itemize}
\begin{itemize}
\item {Proveniência:(Do lat. \textunderscore ad\textunderscore  + \textunderscore lucere\textunderscore )}
\end{itemize}
Gênero de insectos lepidópteres nocturnos.
\section{Aluda}
\begin{itemize}
\item {Grp. gram.:f.}
\end{itemize}
\begin{itemize}
\item {Utilização:T. de Moncorvo}
\end{itemize}
Formiga alada.
O mesmo que \textunderscore agude\textunderscore ?
\section{Alude}
\begin{itemize}
\item {Grp. gram.:f.}
\end{itemize}
O mesmo ou melhor que \textunderscore avalancha\textunderscore . Cf. Gonç. Guimarães, \textunderscore Geol\textunderscore .
(Cast. \textunderscore alud\textunderscore )
\section{Aludel}
\begin{itemize}
\item {Grp. gram.:m.}
\end{itemize}
Conjunto de vasos que, encaixados uns nos outros formam uma espécie de tubo e têm applicação em Chímica.
\section{Aludir}
\begin{itemize}
\item {Grp. gram.:v. i.}
\end{itemize}
Referir-se indirectamente, vagamente.
Fazer referência.
(Lat. \textunderscore alludere\textunderscore ).
\section{Alugação}
\begin{itemize}
\item {Grp. gram.:f.}
\end{itemize}
\begin{itemize}
\item {Utilização:Ant.}
\end{itemize}
O mesmo que \textunderscore aluguer\textunderscore .
\section{Alugador}
\begin{itemize}
\item {Grp. gram.:m.}
\end{itemize}
Aquelle que aluga.
\section{Alugamento}
\begin{itemize}
\item {Grp. gram.:m.}
\end{itemize}
Acto de \textunderscore alugar\textunderscore ; aluguer.
\section{Alugar}
\begin{itemize}
\item {Grp. gram.:v. t.}
\end{itemize}
\begin{itemize}
\item {Proveniência:(Lat. \textunderscore locare\textunderscore )}
\end{itemize}
Dar ou tomar de aluguer.
Assalariar.
\section{A-lugares}
\begin{itemize}
\item {Grp. gram.:loc. adv.}
\end{itemize}
\begin{itemize}
\item {Utilização:Ant.}
\end{itemize}
Parcialmente.
\section{Alugatriz}
\begin{itemize}
\item {Grp. gram.:f.}
\end{itemize}
Mulher que aluga. Cf. Filinto, II, 204 e 262; IX, 107 e 163.
\section{Alugueiro}
\begin{itemize}
\item {Grp. gram.:m.}
\end{itemize}
\begin{itemize}
\item {Utilização:Ant.}
\end{itemize}
\begin{itemize}
\item {Proveniência:(De \textunderscore alugar\textunderscore )}
\end{itemize}
Aquelle que tomava uma coisa de aluguer.
O mesmo que \textunderscore alugador\textunderscore .
\section{Aluguel}
\begin{itemize}
\item {Grp. gram.:m.}
\end{itemize}
O mesmo que \textunderscore aluguer\textunderscore .
\section{Aluguer}
\begin{itemize}
\item {Grp. gram.:m.}
\end{itemize}
\begin{itemize}
\item {Proveniência:(Do cast. ant. \textunderscore aloguer\textunderscore )}
\end{itemize}
Cessão ou acquisição de um serviço ou objecto por tempo e preço determinado: \textunderscore o aluguer de uma casa\textunderscore ; \textunderscore o aluguer de uma casaca\textunderscore .
Preço da cessão temporária: \textunderscore o aluguer é 100$000 reis\textunderscore .
\section{Aluimento}
\begin{itemize}
\item {fónica:lu-i}
\end{itemize}
\begin{itemize}
\item {Grp. gram.:m.}
\end{itemize}
Acto de \textunderscore aluir\textunderscore .
\section{Aluir}
\begin{itemize}
\item {Grp. gram.:v. t.}
\end{itemize}
\begin{itemize}
\item {Grp. gram.:V. i.}
\end{itemize}
\begin{itemize}
\item {Grp. gram.:V. p.}
\end{itemize}
\begin{itemize}
\item {Proveniência:(Do lat. \textunderscore ab\textunderscore  + \textunderscore luere\textunderscore )}
\end{itemize}
Abalar.
Arruinar.
Ameaçar ruína.
Desmoronar-se.
Dobrar-se, vergar:«\textunderscore sentindo-se aluir pelos joelhos\textunderscore ». Camillo, \textunderscore Brasileira\textunderscore , 119.
\section{Álula}
\begin{itemize}
\item {Grp. gram.:f.}
\end{itemize}
\begin{itemize}
\item {Proveniência:(De \textunderscore ala\textunderscore )}
\end{itemize}
Pequena asa.
\section{Alulemba}
\begin{itemize}
\item {Grp. gram.:f.}
\end{itemize}
Árvore angolense, no Duque-de-Bragança.
\section{Alumador}
\begin{itemize}
\item {Grp. gram.:m.}
\end{itemize}
\begin{itemize}
\item {Utilização:Ant.}
\end{itemize}
Lançarote.
\section{Alumbrados}
\begin{itemize}
\item {Grp. gram.:m. pl.}
\end{itemize}
Nome de uma seita; illuminados. Cf. \textunderscore Luz e Calor\textunderscore , 165.
\section{Alumbrar}
\begin{itemize}
\item {Proveniência:(Do cast. \textunderscore lumbre\textunderscore )}
\end{itemize}
\textunderscore v. t. Ant.\textunderscore  (e der.)
O mesmo que \textunderscore alumiar\textunderscore ^1, etc.
\section{Alúmen}
\begin{itemize}
\item {Grp. gram.:m.}
\end{itemize}
\begin{itemize}
\item {Proveniência:(Lat. \textunderscore alumen\textunderscore )}
\end{itemize}
Sulfato duplo de alumina e potassa.
\section{Alumia}
\begin{itemize}
\item {Grp. gram.:f.}
\end{itemize}
\begin{itemize}
\item {Utilização:Prov.}
\end{itemize}
\begin{itemize}
\item {Utilização:alg.}
\end{itemize}
\begin{itemize}
\item {Proveniência:(De \textunderscore alumiar\textunderscore )}
\end{itemize}
Primeira cava que se dá na vinha. Cf. \textunderscore Techn. Rur.\textunderscore , 571.
\section{Alumiação}
\begin{itemize}
\item {Grp. gram.:f.}
\end{itemize}
(V.illuminação)
\section{Alumiada}
\begin{itemize}
\item {Grp. gram.:f.}
\end{itemize}
\begin{itemize}
\item {Utilização:Gír.}
\end{itemize}
\begin{itemize}
\item {Proveniência:(De \textunderscore alumiar\textunderscore )}
\end{itemize}
Fogueira.
\section{Alumiador}
\begin{itemize}
\item {Grp. gram.:m.  e  adj.}
\end{itemize}
O que alumia.
\section{Alumiamento}
\begin{itemize}
\item {Grp. gram.:m.}
\end{itemize}
(V.illuminação)
\section{Alumiana}
\begin{itemize}
\item {Grp. gram.:f.}
\end{itemize}
Sulfato anhydro de alumina.
\section{Alumiante}
\begin{itemize}
\item {Grp. gram.:adj.}
\end{itemize}
\begin{itemize}
\item {Utilização:P. us.}
\end{itemize}
Que alumia; illuminante.
\section{Alumiar}
\begin{itemize}
\item {Grp. gram.:v. t.}
\end{itemize}
\begin{itemize}
\item {Utilização:Prov.}
\end{itemize}
\begin{itemize}
\item {Utilização:alg.}
\end{itemize}
\begin{itemize}
\item {Proveniência:(Do lat. hyp. \textunderscore ad-luminare\textunderscore )}
\end{itemize}
Dar luz a: \textunderscore alumiar a escada\textunderscore ; \textunderscore alumiar os espectadores\textunderscore .
Esclarecer; illuminar; instruir.
Dar mais luz, cavando, no tronco de (a videira).
\section{Alumiar}
\begin{itemize}
\item {Grp. gram.:v. t.}
\end{itemize}
(V.alomear)
\section{Alumina}
\begin{itemize}
\item {Grp. gram.:f.}
\end{itemize}
\begin{itemize}
\item {Proveniência:(De \textunderscore alúmen\textunderscore )}
\end{itemize}
Óxydo metállico, que fórma a base de todas as argillas.
\section{Aluminagem}
\begin{itemize}
\item {Grp. gram.:f.}
\end{itemize}
\begin{itemize}
\item {Utilização:Phot.}
\end{itemize}
Banho de alumina.
\section{Aluminar}
\begin{itemize}
\item {Grp. gram.:v. t.}
\end{itemize}
\begin{itemize}
\item {Grp. gram.:Adj.}
\end{itemize}
Misturar com alúmen.
Que contém alúmen.
\section{Aluminar}
\begin{itemize}
\item {Grp. gram.:v. t.}
\end{itemize}
O mesmo que \textunderscore alumiar\textunderscore ^1:«\textunderscore tambem se punham em marcha aluminando seus passos\textunderscore ». Cf. \textunderscore Luz e Calor\textunderscore .
\section{Aluminato}
\begin{itemize}
\item {Grp. gram.:m.}
\end{itemize}
Sal, resultante de uma combinação, em que a alumina entra como ácido.
\section{Alumínico}
\begin{itemize}
\item {Grp. gram.:adj.}
\end{itemize}
Diz-se dos sáes, em que a alumina é a base.
\section{Alumínico-silicato}
\begin{itemize}
\item {Grp. gram.:m.}
\end{itemize}
Sal, em cuja composição entra o alumínio e o silício.
\section{Aluminídeos}
\begin{itemize}
\item {Grp. gram.:m. pl.}
\end{itemize}
Fam. mineralógica, que comprehende todas as espécies formadas de alumina.
\section{Aluminífero}
\begin{itemize}
\item {Grp. gram.:adj.}
\end{itemize}
\begin{itemize}
\item {Proveniência:(Do lat. \textunderscore alumen\textunderscore  + \textunderscore ferre\textunderscore )}
\end{itemize}
Que contém alúmen.
\section{Alumínio}
\begin{itemize}
\item {Grp. gram.:m.}
\end{itemize}
Metal, que é o radical da alumina.
\section{Aluminioso}
\begin{itemize}
\item {Grp. gram.:adj.}
\end{itemize}
(V.aluminoso)
\section{Aluminita}
\begin{itemize}
\item {Grp. gram.:f.}
\end{itemize}
O mesmo ou melhor que \textunderscore aluminite\textunderscore .
\section{Aluminite}
\begin{itemize}
\item {Grp. gram.:f.}
\end{itemize}
Variedade de sulfato de alumina.
\section{Aluminoso}
\begin{itemize}
\item {Grp. gram.:adj.}
\end{itemize}
O mesmo que \textunderscore aluminífero\textunderscore .
\section{Aluminotermia}
\begin{itemize}
\item {Grp. gram.:f.}
\end{itemize}
Processo moderno (1902), com que se consegue termo-chimicamente a purificação dos óxydos mineraes.
\section{Aluminotérmico}
\begin{itemize}
\item {Grp. gram.:adj.}
\end{itemize}
Relativo á \textunderscore aluminotermia\textunderscore .
\section{Aluminothermia}
\begin{itemize}
\item {Grp. gram.:f.}
\end{itemize}
Processo moderno (1902), com que se consegue thermo-chimicamente a purificação dos óxydos mineraes.
\section{Aluminothérmico}
\begin{itemize}
\item {Grp. gram.:adj.}
\end{itemize}
Relativo á \textunderscore aluminothermia\textunderscore .
\section{Aluminóxido}
\begin{itemize}
\item {Grp. gram.:m.}
\end{itemize}
Óxido de alumínio.
\section{Aluminóxydo}
\begin{itemize}
\item {Grp. gram.:m.}
\end{itemize}
Óxydo de alumínio.
\section{Alumno}
\begin{itemize}
\item {Grp. gram.:m.}
\end{itemize}
\begin{itemize}
\item {Proveniência:(Lat. \textunderscore alumnus\textunderscore )}
\end{itemize}
Educando.
Discípulo.
Aprendiz.
\section{Alumnol}
\begin{itemize}
\item {Grp. gram.:m.}
\end{itemize}
Composto chímico, usado em Cirurgia, como adstringente e antiséptico.
\section{Alumogênio}
\begin{itemize}
\item {Grp. gram.:m.}
\end{itemize}
Sulfato de alumina hydratado.
\section{Alunação}
\begin{itemize}
\item {Grp. gram.:f.}
\end{itemize}
\begin{itemize}
\item {Utilização:Gal}
\end{itemize}
\begin{itemize}
\item {Proveniência:(Fr. \textunderscore alunation\textunderscore , de \textunderscore alun\textunderscore , alúmen)}
\end{itemize}
Formação de alúmen, natural ou artificialmente.
\section{Alúnico}
\begin{itemize}
\item {Grp. gram.:adj.}
\end{itemize}
\begin{itemize}
\item {Utilização:Gal}
\end{itemize}
\begin{itemize}
\item {Proveniência:(Do fr. \textunderscore alun\textunderscore )}
\end{itemize}
Que contém alúmen.
\section{Alunífero}
\begin{itemize}
\item {Grp. gram.:adj.}
\end{itemize}
(V.aluminífero)
\section{Alunita}
\begin{itemize}
\item {Grp. gram.:f.}
\end{itemize}
Sub-sulfato de alumina e de potassa.
\section{Aluno}
\begin{itemize}
\item {Grp. gram.:m.}
\end{itemize}
\begin{itemize}
\item {Proveniência:(Lat. \textunderscore alumnus\textunderscore )}
\end{itemize}
Educando.
Discípulo.
Aprendiz.
\section{Alur}
\begin{itemize}
\item {Grp. gram.:adv.}
\end{itemize}
\begin{itemize}
\item {Utilização:Ant.}
\end{itemize}
O mesmo que \textunderscore alhur\textunderscore .
\section{Alusão}
\begin{itemize}
\item {Grp. gram.:f.}
\end{itemize}
Acto de \textunderscore aludir\textunderscore .
Referência indirecta, vaga.
(Lat. \textunderscore allusio\textunderscore ).
\section{Alusivamente}
\begin{itemize}
\item {Grp. gram.:adv.}
\end{itemize}
De modo \textunderscore alusivo\textunderscore .
\section{Alusivo}
\begin{itemize}
\item {Grp. gram.:adj.}
\end{itemize}
Que envolve alusão.
Que diz respeito a alguma coisa.
\section{Alustre}
\begin{itemize}
\item {Grp. gram.:m.}
\end{itemize}
\begin{itemize}
\item {Utilização:Prov.}
\end{itemize}
\begin{itemize}
\item {Utilização:minh.}
\end{itemize}
\begin{itemize}
\item {Utilização:trasm.}
\end{itemize}
Relâmpago.
(Cp. \textunderscore lustre\textunderscore )
\section{Alutero}
\begin{itemize}
\item {Grp. gram.:m.}
\end{itemize}
Gênero de peixes plectognathos.
\section{Aluvai!}
\begin{itemize}
\item {Grp. gram.:interj.}
\end{itemize}
\begin{itemize}
\item {Utilização:Bras. do N}
\end{itemize}
Alto lá!
\section{Aluvial}
\begin{itemize}
\item {Grp. gram.:adj.}
\end{itemize}
Relativo a aluvião.
Formado por aluvião.
\section{Aluviano}
\begin{itemize}
\item {Grp. gram.:adj.}
\end{itemize}
Diz-se do terreno ou do depósito, formado por aluvião.
\section{Aluvião}
\begin{itemize}
\item {Grp. gram.:f.}
\end{itemize}
\begin{itemize}
\item {Utilização:Fig.}
\end{itemize}
Inundação.
Enxurrada.
Grande quantidade, ou grande número.
O mesmo que terreno aluviano.
(Lat. \textunderscore alluvio\textunderscore ).
\section{Aluxar}
\begin{itemize}
\item {Grp. gram.:v. t.}
\end{itemize}
Afroixar (uma corda que estava retesada).
\section{Aluz}
\begin{itemize}
\item {Grp. gram.:m.}
\end{itemize}
\begin{itemize}
\item {Utilização:Ant.}
\end{itemize}
Tecido felpudo, para vestuário ou ornato.
\section{Aluziar}
\begin{itemize}
\item {Grp. gram.:v. t.}
\end{itemize}
\begin{itemize}
\item {Utilização:Ant.}
\end{itemize}
Tornar luzidio, lustroso; polir.
(Por \textunderscore aluzidiar\textunderscore , de \textunderscore luzidio\textunderscore )
\section{Aluzir}
\begin{itemize}
\item {Grp. gram.:v. t.}
\end{itemize}
\begin{itemize}
\item {Utilização:Ant.}
\end{itemize}
O mesmo que \textunderscore aluziar\textunderscore .
\section{Alva}
\begin{itemize}
\item {Grp. gram.:f.}
\end{itemize}
\begin{itemize}
\item {Proveniência:(Lat. \textunderscore alba\textunderscore )}
\end{itemize}
Primeiro alvor da manhan.
Vestimenta ecclesiástica, de pano branco. Esclerótica.
Casta de uva alentejana.
\section{Alvacá}
\begin{itemize}
\item {Grp. gram.:m.}
\end{itemize}
Planta malvácea, de que se extrai filaça para fazer sacos.
\section{Alvação}
\begin{itemize}
\item {Grp. gram.:adj.}
\end{itemize}
(V.alvadio)
\section{Alvacento}
\begin{itemize}
\item {Grp. gram.:adj.}
\end{itemize}
\begin{itemize}
\item {Proveniência:(De \textunderscore alvo\textunderscore )}
\end{itemize}
Quási branco; esbranquiçado.
Cinzento-claro.
\section{Alvacora}
\begin{itemize}
\item {Grp. gram.:f.}
\end{itemize}
(V.albacora)
\section{Alvada}
\begin{itemize}
\item {Grp. gram.:f.}
\end{itemize}
\begin{itemize}
\item {Utilização:ant.}
\end{itemize}
\begin{itemize}
\item {Utilização:Gír.}
\end{itemize}
Carapuça.
\section{Alvadio}
\begin{itemize}
\item {Grp. gram.:adj.}
\end{itemize}
O mesmo que \textunderscore alvacento\textunderscore .
\section{Alvado}
\begin{itemize}
\item {Grp. gram.:m.}
\end{itemize}
\begin{itemize}
\item {Proveniência:(Lat. \textunderscore alveatus\textunderscore )}
\end{itemize}
Buraco, por onde as abelhas entram no cortiço. Parte ôca de alguns instrumentos, pela qual se adaptam a outros objectos, por ex. o alvado da enxada.
\section{Alvadurão}
\begin{itemize}
\item {Grp. gram.:m.}
\end{itemize}
Casta de uva branca extremenha e da Bairrada.
\section{Alvaiadar}
\begin{itemize}
\item {Grp. gram.:v. t.}
\end{itemize}
Tingir com alvaiade.
\section{Alvaiade}
\begin{itemize}
\item {Grp. gram.:m.}
\end{itemize}
\begin{itemize}
\item {Proveniência:(Do ár. \textunderscore al-baiade\textunderscore )}
\end{itemize}
Carbonato de chumbo, branco ou amarelado.
\section{Alvalade}
\begin{itemize}
\item {Grp. gram.:m.}
\end{itemize}
\begin{itemize}
\item {Utilização:Ant.}
\end{itemize}
Campo ou páteo murado.
\section{Alvalar}
\begin{itemize}
\item {Grp. gram.:v. t.}
\end{itemize}
Oppor vallado a, oppor-se a?«\textunderscore ...assim quiseram alvallar a corrente da verdade\textunderscore ». Filinto, II, 201.
(Relaciona-se com \textunderscore alvalade\textunderscore ?)
\section{Alvallar}
\begin{itemize}
\item {Grp. gram.:v. t.}
\end{itemize}
Oppor vallado a, oppor-se a?«\textunderscore ...assim quiseram alvallar a corrente da verdade\textunderscore ». Filinto, II, 201.
(Relaciona-se com \textunderscore alvalade\textunderscore ?)
\section{Alvanega}
\begin{itemize}
\item {Grp. gram.:f.}
\end{itemize}
\begin{itemize}
\item {Utilização:Ant.}
\end{itemize}
\begin{itemize}
\item {Proveniência:(Do ár. \textunderscore al-banica\textunderscore )}
\end{itemize}
Coifa; touca.
\section{Alvaneira}
\begin{itemize}
\item {Grp. gram.:f.}
\end{itemize}
\begin{itemize}
\item {Utilização:Prov.}
\end{itemize}
\begin{itemize}
\item {Utilização:trasm.}
\end{itemize}
\begin{itemize}
\item {Proveniência:(De \textunderscore alvanel\textunderscore ? de \textunderscore álveo\textunderscore ?)}
\end{itemize}
Cano de esgôto, para a humidade das estrebarias.
\section{Alvanel}
\begin{itemize}
\item {Grp. gram.:m.}
\end{itemize}
\begin{itemize}
\item {Utilização:Ant.}
\end{itemize}
\begin{itemize}
\item {Utilização:Fig.}
\end{itemize}
\begin{itemize}
\item {Utilização:Prov.}
\end{itemize}
\begin{itemize}
\item {Utilização:beir.}
\end{itemize}
\begin{itemize}
\item {Proveniência:(Do ár. \textunderscore al-banné\textunderscore )}
\end{itemize}
Pedreiro.
Autor de obra tôsca.
Aqueducto muito baixo, ordinariamente feito dentro das minas.
\section{Alvanéu}
\begin{itemize}
\item {Grp. gram.:m.}
\end{itemize}
(V.alvanel)
\section{Alvanhal}
\begin{itemize}
\item {Grp. gram.:m.}
\end{itemize}
\begin{itemize}
\item {Utilização:Prov.}
\end{itemize}
\begin{itemize}
\item {Utilização:trasm.}
\end{itemize}
Fôsso de esgôto.
Draino.
(Cp. \textunderscore alvaneira\textunderscore )
\section{Alvão}
\begin{itemize}
\item {Grp. gram.:m.}
\end{itemize}
Ave, semelhante á andorinha.
(Cp. \textunderscore alvéloa\textunderscore )
\section{Alvar}
\begin{itemize}
\item {Grp. gram.:adj.}
\end{itemize}
\begin{itemize}
\item {Utilização:Fig.}
\end{itemize}
\begin{itemize}
\item {Grp. gram.:F.}
\end{itemize}
\begin{itemize}
\item {Proveniência:(De \textunderscore alvo\textunderscore )}
\end{itemize}
O mesmo que \textunderscore alvacento\textunderscore .
Ingênuo.
Tolo; estúpido.
Casta de uva branca da Bairrada.
\section{Alvará}
\begin{itemize}
\item {Grp. gram.:m.}
\end{itemize}
\begin{itemize}
\item {Utilização:Ant.}
\end{itemize}
\begin{itemize}
\item {Proveniência:(Do ár. \textunderscore al-bará\textunderscore )}
\end{itemize}
Documento, passado por uma autoridade a favor de alguém, nomeando-o para certo cargo, certificando, autorizando ou approvando certos actos ou direitos.
Diploma, rubricado pelo monarcha e assignado pelo ministro, sôbre negócios de interesse público ou particular.
\section{Alvaraça}
\begin{itemize}
\item {Grp. gram.:f.}
\end{itemize}
O mesmo que \textunderscore alvaraço\textunderscore .
\section{Alvaraço}
\begin{itemize}
\item {Grp. gram.:m.}
\end{itemize}
Casta de uva, na região do Doiro e no Minho.
\section{Alvará-de-soltura}
\begin{itemize}
\item {Grp. gram.:m.}
\end{itemize}
\begin{itemize}
\item {Utilização:Prov.}
\end{itemize}
\begin{itemize}
\item {Utilização:trasm.}
\end{itemize}
Mulher estouvada, mexeriqueira, enredadeira.
\section{Alvarádoa}
\begin{itemize}
\item {Grp. gram.:f.}
\end{itemize}
\begin{itemize}
\item {Proveniência:(De \textunderscore Alvarado\textunderscore , n. p.)}
\end{itemize}
Gênero de plantas sapindáceas da América.
\section{Alvaral}
\begin{itemize}
\item {Grp. gram.:m.}
\end{itemize}
\begin{itemize}
\item {Utilização:Ant.}
\end{itemize}
O mesmo que \textunderscore alvará\textunderscore .
\section{Alvaraz}
\begin{itemize}
\item {Grp. gram.:m.}
\end{itemize}
Lepra branca.
(Cast. \textunderscore alvaraz\textunderscore )
\section{Alvarazo}
\begin{itemize}
\item {Grp. gram.:m.}
\end{itemize}
Bostella nos cavallos.
(Cp. \textunderscore alvaraz\textunderscore )
\section{Alvarda}
\begin{itemize}
\item {Grp. gram.:f.}
\end{itemize}
Gênero de gramíneas.
\section{Alvarelha}
\begin{itemize}
\item {fónica:varê}
\end{itemize}
\begin{itemize}
\item {Grp. gram.:f.}
\end{itemize}
\begin{itemize}
\item {Utilização:Prov.}
\end{itemize}
\begin{itemize}
\item {Utilização:minh.}
\end{itemize}
\begin{itemize}
\item {Proveniência:(De \textunderscore alvo\textunderscore )}
\end{itemize}
Entreaberta de bom tempo, em dia tempestuoso.
\section{Alvarelhão}
\begin{itemize}
\item {Grp. gram.:m.}
\end{itemize}
Casta de uva tinta, no Minho, Doiro e Beira.
\section{Alvarenga}
\begin{itemize}
\item {Grp. gram.:f.}
\end{itemize}
\begin{itemize}
\item {Utilização:Bras. do N}
\end{itemize}
\begin{itemize}
\item {Proveniência:(De \textunderscore Alvarenga\textunderscore , appellido?)}
\end{itemize}
Lanchão, para carga e descarga de navios, e para transporte de objectos pesados.
\section{Alvarinho}
\begin{itemize}
\item {Grp. gram.:m.}
\end{itemize}
\begin{itemize}
\item {Utilização:Prov.}
\end{itemize}
\begin{itemize}
\item {Utilização:trasm.}
\end{itemize}
Peixe, espécie de cação.
Casta de uva minhota.
Certo álamo branco.
\section{Alvarinho}
\begin{itemize}
\item {Grp. gram.:m.}
\end{itemize}
\begin{itemize}
\item {Proveniência:(De \textunderscore alvaraz\textunderscore )}
\end{itemize}
Bexigas benignas, que dão no gado ovelhum e cabrum.
\section{Alvarinho}
\begin{itemize}
\item {Grp. gram.:m.}
\end{itemize}
\begin{itemize}
\item {Utilização:Prov.}
\end{itemize}
\begin{itemize}
\item {Utilização:trasm.}
\end{itemize}
\begin{itemize}
\item {Proveniência:(De \textunderscore alvar\textunderscore )}
\end{itemize}
Doidivanas, artola.
\section{Alvaroca}
\begin{itemize}
\item {Grp. gram.:m.}
\end{itemize}
Espécie de uva branca do Minho.
\section{Alvaroco}
\begin{itemize}
\item {Grp. gram.:m.}
\end{itemize}
O mesmo que \textunderscore alvaroca\textunderscore .
\section{Alvarrã}
\begin{itemize}
\item {Grp. gram.:f.}
\end{itemize}
(V.albarran)
\section{Alvarral}
\begin{itemize}
\item {Grp. gram.:m.}
\end{itemize}
\begin{itemize}
\item {Proveniência:(Do ár. \textunderscore al-garbal\textunderscore )}
\end{itemize}
Espécie de peneira.
\section{Alvarran}
\begin{itemize}
\item {Grp. gram.:f.}
\end{itemize}
(V.albarran)
\section{Alvarrão}
\begin{itemize}
\item {Grp. gram.:m.}
\end{itemize}
O mesmo que \textunderscore albarrão\textunderscore .
\section{Alvarudão}
\begin{itemize}
\item {Grp. gram.:m.}
\end{itemize}
O mesmo que \textunderscore alvadurão\textunderscore .
\section{Alvassus}
\begin{itemize}
\item {Grp. gram.:m.}
\end{itemize}
\begin{itemize}
\item {Utilização:Náut.}
\end{itemize}
Compartimento no porão, para guardar cabos, ferragens, pólvora.
Pequeno paiol na popa.
\section{Alvazil}
\begin{itemize}
\item {Grp. gram.:m.}
\end{itemize}
\begin{itemize}
\item {Utilização:Ant.}
\end{itemize}
Governador.
Juiz de primeira instância.
Vereador da câmara.
(Cp. \textunderscore aguazil\textunderscore )
\section{Alveador}
\begin{itemize}
\item {Grp. gram.:m.}
\end{itemize}
Aquelle que alveia.
\section{Alveamento}
\begin{itemize}
\item {Grp. gram.:m.}
\end{itemize}
Acto de \textunderscore alvear\textunderscore .
\section{Alvear}
\begin{itemize}
\item {Grp. gram.:v. t.}
\end{itemize}
\begin{itemize}
\item {Utilização:Ant.}
\end{itemize}
\begin{itemize}
\item {Proveniência:(De \textunderscore alvo\textunderscore )}
\end{itemize}
O mesmo que \textunderscore caiar\textunderscore .
\section{Alveário}
\begin{itemize}
\item {Grp. gram.:m.}
\end{itemize}
\begin{itemize}
\item {Proveniência:(Lat. \textunderscore alvearium\textunderscore )}
\end{itemize}
Colmeia.
Colmeal.
\section{Alveci}
\begin{itemize}
\item {Grp. gram.:m.}
\end{itemize}
\begin{itemize}
\item {Utilização:Ant.}
\end{itemize}
O mesmo que \textunderscore alveici\textunderscore .
\section{Alvedrio}
\begin{itemize}
\item {Grp. gram.:m.}
\end{itemize}
\begin{itemize}
\item {Utilização:Pop.}
\end{itemize}
O mesmo que \textunderscore arbítrio\textunderscore .
(Cast. \textunderscore albedrio\textunderscore )
\section{Alveici}
\begin{itemize}
\item {Grp. gram.:m.}
\end{itemize}
\begin{itemize}
\item {Utilização:Ant.}
\end{itemize}
\begin{itemize}
\item {Proveniência:(Do ár. \textunderscore al-uaxi\textunderscore )}
\end{itemize}
Tecido de seda branca e fina.
\section{Alveiro}
\begin{itemize}
\item {Grp. gram.:adj.}
\end{itemize}
\begin{itemize}
\item {Grp. gram.:M.}
\end{itemize}
\begin{itemize}
\item {Utilização:Prov.}
\end{itemize}
\begin{itemize}
\item {Utilização:trasm.}
\end{itemize}
Que tem côr alva: \textunderscore pão alveiro\textunderscore .
E diz-se do moínho, que só mói pão branco.
Pedra ou marco branco, que serve de ponto de mira.
Pano branco, que serve de avental.
Pano de linho, para cobrir o pão que sai do forno.
\section{Alveitar}
\begin{itemize}
\item {Grp. gram.:m.}
\end{itemize}
\begin{itemize}
\item {Proveniência:(Do ár. \textunderscore al-beitar\textunderscore )}
\end{itemize}
Homem, que, sem diploma de habilitação, trata doenças de animaes.
\section{Alveitar}
\begin{itemize}
\item {Grp. gram.:v. t.}
\end{itemize}
\begin{itemize}
\item {Utilização:Prov.}
\end{itemize}
\begin{itemize}
\item {Utilização:minh.}
\end{itemize}
Indagar, pesquisar.
Averiguar.
(Relaciona-se com \textunderscore aviltar\textunderscore ?)
\section{Alveitarar}
\begin{itemize}
\item {Grp. gram.:v. i.}
\end{itemize}
\begin{itemize}
\item {Grp. gram.:V. t.}
\end{itemize}
\begin{itemize}
\item {Utilização:Fig.}
\end{itemize}
Exercer as funcções de alveitar.
Remediar, mondar, emendar:«\textunderscore o poema alveitarando de aleijões\textunderscore ». Filinto, IX, 62.
\section{Alveitaria}
\begin{itemize}
\item {Grp. gram.:f.}
\end{itemize}
Arte de alveitar.
\section{Alvejante}
\begin{itemize}
\item {Grp. gram.:adj.}
\end{itemize}
Que alveja.
\section{Alvejar}
\begin{itemize}
\item {Grp. gram.:v. i.}
\end{itemize}
\begin{itemize}
\item {Grp. gram.:V. i.}
\end{itemize}
\begin{itemize}
\item {Proveniência:(De \textunderscore alvo\textunderscore )}
\end{itemize}
Branquear.
Tomar como ponto de mira, como alvo.
Branquejar.
Atirar ao alvo.
\section{Alvela}
\begin{itemize}
\item {Grp. gram.:f.}
\end{itemize}
O mesmo que \textunderscore alvéloa\textunderscore .
\section{Alveliço}
\begin{itemize}
\item {Grp. gram.:m.}
\end{itemize}
Espécie de alvéloa.
\section{Alvéloa}
\begin{itemize}
\item {Grp. gram.:f.}
\end{itemize}
\begin{itemize}
\item {Proveniência:(Do lat. hyp. \textunderscore albellula\textunderscore )}
\end{itemize}
Pequeno pássaro conirostro, (\textunderscore motacilla alba\textunderscore ).
\section{Alvenaria}
\begin{itemize}
\item {Grp. gram.:f.}
\end{itemize}
Profissão, arte de pedreiro.
O conjunto das pedras que, ligadas, constituem construcção.
(Cast. ant. \textunderscore albañeria\textunderscore )
\section{Alvende}
\begin{itemize}
\item {Grp. gram.:m.}
\end{itemize}
\begin{itemize}
\item {Utilização:Ant.}
\end{itemize}
O mesmo que \textunderscore alvará\textunderscore .
\section{Alvenel}
\begin{itemize}
\item {Grp. gram.:m.}
\end{itemize}
O mesmo que \textunderscore alvanel\textunderscore .
\section{Alvener}
\begin{itemize}
\item {Grp. gram.:m.}
\end{itemize}
(V.alvanel)
\section{Alvenéu}
\begin{itemize}
\item {Grp. gram.:m.}
\end{itemize}
O mesmo que \textunderscore alvanel\textunderscore :«\textunderscore pedras estendidas pela colhér do alvenéu\textunderscore ». Herculano, \textunderscore Questões Púb.\textunderscore , II, 15.
\section{Álveo}
\begin{itemize}
\item {Grp. gram.:m.}
\end{itemize}
\begin{itemize}
\item {Proveniência:(Lat. \textunderscore alveus\textunderscore )}
\end{itemize}
Leito de (rio ou regato).
Sulco.
Escavação.
\section{Alveolado}
\begin{itemize}
\item {Grp. gram.:adj.}
\end{itemize}
Que tem alvéolos.
\section{Alveolar}
\begin{itemize}
\item {Grp. gram.:adj.}
\end{itemize}
Relativo a alvéolo.
\section{Alveolariforme}
\begin{itemize}
\item {Grp. gram.:adj.}
\end{itemize}
\begin{itemize}
\item {Proveniência:(De \textunderscore alvéolo\textunderscore  + \textunderscore fórma\textunderscore )}
\end{itemize}
Que tem fórma de alvéolo.
\section{Alvéolo}
\begin{itemize}
\item {Grp. gram.:m.}
\end{itemize}
\begin{itemize}
\item {Proveniência:(Lat. \textunderscore alveolus\textunderscore )}
\end{itemize}
Céllula, em que as abelhas depositam as larvas e o mel.
Pequena cavidade, em que se inserem os dentes.
Pequena cavidade.
Casulo.
\section{Alvéolo-dental}
\begin{itemize}
\item {Grp. gram.:m.}
\end{itemize}
Parte da gengiva, que une a raiz do dente ao alvéolo.
\section{Alvéolo-labial}
\begin{itemize}
\item {Grp. gram.:m.}
\end{itemize}
Músculo facial, que nasce nos bordos alveolares dos ossos maxillares.
\section{Alvéolo-nasal}
\begin{itemize}
\item {Grp. gram.:m.}
\end{itemize}
Músculo abaixador da asa do nariz.
\section{Alverca}
\begin{itemize}
\item {Grp. gram.:f.}
\end{itemize}
Terreno pantanoso.
Viveiro de peixes.
Tanque.
(Ár. \textunderscore al-birca\textunderscore )
\section{Alverge}
\begin{itemize}
\item {Grp. gram.:m.}
\end{itemize}
\begin{itemize}
\item {Utilização:Ant.}
\end{itemize}
Pequena tôrre.
(Do ár.?)
\section{Alvergue}
\begin{itemize}
\item {Grp. gram.:m.}
\end{itemize}
Tanque, em que repoisa o líquido, escorrido dos bagaços de azeitona, nos lagares de azeite.
(Cp. \textunderscore alverca\textunderscore )
\section{Alvéroa}
\begin{itemize}
\item {Grp. gram.:f.}
\end{itemize}
\begin{itemize}
\item {Utilização:Prov.}
\end{itemize}
O mesmo que \textunderscore alvéloa\textunderscore .
\section{Alvião}
\begin{itemize}
\item {Grp. gram.:m.}
\end{itemize}
Instrumento de ferro, para desaterros ou para rasgar terra dura.
\section{Alviçarar}
\begin{itemize}
\item {Grp. gram.:v. t.}
\end{itemize}
Noticiar, para receber alvíçaras.
Referir (factos inda não conhecidos). Cf. Camillo, \textunderscore Quéda de um Anjo\textunderscore , 196.
\section{Alvíçaras}
\begin{itemize}
\item {Grp. gram.:f. pl.}
\end{itemize}
Prêmio, que se dá a quem traz boas novas ou entrega coisa que se tinha perdido.
(Ár. \textunderscore al-bixara\textunderscore )
\section{Alviçareiro}
\begin{itemize}
\item {Grp. gram.:m.}
\end{itemize}
Aquelle que pede ou recebe alvíçaras.
Aquelle que as dá ou as promete.
Aquelle que dá bôas novas, pedindo alvíçaras. Aquelle que vigia a chegada dos navios á barra, para dar notícia aos interessados e receber alvíçaras.
\section{Alvidrador}
\begin{itemize}
\item {Grp. gram.:m.}
\end{itemize}
\begin{itemize}
\item {Utilização:Ant.}
\end{itemize}
Aquelle que alvidra.
O mesmo que \textunderscore avaliador\textunderscore .
\section{Alvidrar}
\begin{itemize}
\item {Grp. gram.:v. t.}
\end{itemize}
\begin{itemize}
\item {Utilização:Ant.}
\end{itemize}
\begin{itemize}
\item {Proveniência:(De \textunderscore álvidro\textunderscore )}
\end{itemize}
O mesmo que \textunderscore arbitrar\textunderscore .
\section{Álvidro}
\begin{itemize}
\item {Grp. gram.:m.}
\end{itemize}
\begin{itemize}
\item {Utilização:Ant.}
\end{itemize}
O mesmo que \textunderscore árbitro\textunderscore .
(Cp. \textunderscore alvedrio\textunderscore )
\section{Alvidroso}
\begin{itemize}
\item {Grp. gram.:m.}
\end{itemize}
\begin{itemize}
\item {Utilização:Ant.}
\end{itemize}
\begin{itemize}
\item {Proveniência:(De \textunderscore alvidrar\textunderscore )}
\end{itemize}
Castigo ou pena, applicada a arbítrio do juiz ou varão prudente.
\section{Aliselminto}
\begin{itemize}
\item {Grp. gram.:m.}
\end{itemize}
Gênero de vermes intestinaes.
\section{Alismo}
\begin{itemize}
\item {Grp. gram.:m.}
\end{itemize}
\begin{itemize}
\item {Proveniência:(Do gr. \textunderscore alusmos\textunderscore )}
\end{itemize}
Ansiedade, inquietação mórbida.
\section{Alissíneas}
\begin{itemize}
\item {Grp. gram.:f. pl.}
\end{itemize}
\begin{itemize}
\item {Proveniência:(De \textunderscore allysso\textunderscore )}
\end{itemize}
Tríbo de plantas crucíferas, segundo De-Candolle.
\section{Alisso}
\begin{itemize}
\item {Grp. gram.:m.}
\end{itemize}
\begin{itemize}
\item {Proveniência:(Do gr. \textunderscore a\textunderscore  priv. + \textunderscore luzein\textunderscore )}
\end{itemize}
Planta crucífera, ornamental.
\section{Alitarco}
\begin{itemize}
\item {Grp. gram.:m.}
\end{itemize}
Chefe dos officiaes, encarregados da manutenção da ordem nos jogos olýmpicos.
\section{Alíxia}
\begin{itemize}
\item {Grp. gram.:f.}
\end{itemize}
\begin{itemize}
\item {Proveniência:(Do gr. \textunderscore aluxis\textunderscore )}
\end{itemize}
Gênero de plantas apocýneas.
\section{Alviduco}
\begin{itemize}
\item {Grp. gram.:m.}
\end{itemize}
\begin{itemize}
\item {Proveniência:(Do lat. \textunderscore alvus\textunderscore  + \textunderscore ducere\textunderscore )}
\end{itemize}
Purgante.
\section{Alvilha}
\begin{itemize}
\item {Grp. gram.:f.}
\end{itemize}
\begin{itemize}
\item {Proveniência:(De \textunderscore alvo\textunderscore )}
\end{itemize}
Casta de uva algarvia.
\section{Alvilho}
\begin{itemize}
\item {Grp. gram.:m.}
\end{itemize}
\begin{itemize}
\item {Proveniência:(De \textunderscore alvo\textunderscore )}
\end{itemize}
Uva branca de Miranda.
\section{Alvinitente}
\begin{itemize}
\item {Grp. gram.:adj.}
\end{itemize}
\begin{itemize}
\item {Proveniência:(Do lat. \textunderscore albus\textunderscore  + \textunderscore nitens\textunderscore )}
\end{itemize}
Que brilha, branquejando.
\section{Alvino}
\begin{itemize}
\item {Grp. gram.:adj.}
\end{itemize}
\begin{itemize}
\item {Proveniência:(Do lat. \textunderscore alvus\textunderscore )}
\end{itemize}
Que diz respeito ao baixo ventre.
\section{Alvio}
\begin{itemize}
\item {Grp. gram.:adj.}
\end{itemize}
\begin{itemize}
\item {Utilização:T. de Moncorvo}
\end{itemize}
O mesmo que \textunderscore alvo\textunderscore .
\section{Alvitana}
\begin{itemize}
\item {Grp. gram.:f.}
\end{itemize}
\begin{itemize}
\item {Grp. gram.:f.}
\end{itemize}
\begin{itemize}
\item {Proveniência:(Do ár. \textunderscore al-bitana\textunderscore ?)}
\end{itemize}
Rede larga, de malha miúda; tarrafa.
Cada um dos dois panos exteriores do tresmalho.
\section{Alvitanado}
\begin{itemize}
\item {Grp. gram.:adj.}
\end{itemize}
Que tem malha miúda, como a alvitana.
\section{Alvithórax}
\begin{itemize}
\item {Grp. gram.:adj.}
\end{itemize}
\begin{itemize}
\item {Proveniência:(De \textunderscore alvo\textunderscore  + \textunderscore thórax\textunderscore )}
\end{itemize}
Diz-se do animal que tem o thórax branco.
\section{Alvitórax}
\begin{itemize}
\item {Grp. gram.:adj.}
\end{itemize}
\begin{itemize}
\item {Proveniência:(De \textunderscore alvo\textunderscore  + \textunderscore thórax\textunderscore )}
\end{itemize}
Diz-se do animal que tem o thórax branco.
\section{Alvitrador}
\begin{itemize}
\item {Grp. gram.:m.}
\end{itemize}
Aquelle que alvitra.
\section{Alvitrajado}
\begin{itemize}
\item {Grp. gram.:adj.}
\end{itemize}
Vestido de branco. Cf. Castilho, \textunderscore Fastos\textunderscore , I, 149.
\section{Alvitramento}
\begin{itemize}
\item {Grp. gram.:m.}
\end{itemize}
(V.alvitre)
\section{Alvitrar}
\begin{itemize}
\item {Grp. gram.:v. t.}
\end{itemize}
\begin{itemize}
\item {Proveniência:(De \textunderscore alvitre\textunderscore )}
\end{itemize}
Suggerir; lembrar; propor.
\section{Alvitre}
\begin{itemize}
\item {Grp. gram.:m.}
\end{itemize}
\begin{itemize}
\item {Utilização:Ant.}
\end{itemize}
O mesmo que \textunderscore arbítrio\textunderscore .
Proposta.
Suggestão; lembrança.
Projecto.
Notícia.
(Form. pop. de \textunderscore arbitrio\textunderscore )
\section{Alvitreiro}
\begin{itemize}
\item {Grp. gram.:m.}
\end{itemize}
Aquelle que dá alvitres; alvitrador.
Alviçareiro.
\section{Alvitrista}
\begin{itemize}
\item {Grp. gram.:m.}
\end{itemize}
O mesmo que \textunderscore alvitreiro\textunderscore :«\textunderscore os alvitristas da educação pueril\textunderscore ». Castilho, \textunderscore Fel. pela Agr.\textunderscore , 175.
\section{Alvo}
\begin{itemize}
\item {Grp. gram.:adj.}
\end{itemize}
\begin{itemize}
\item {Grp. gram.:M.}
\end{itemize}
\begin{itemize}
\item {Utilização:Fig.}
\end{itemize}
\begin{itemize}
\item {Proveniência:(Lat. \textunderscore albus\textunderscore )}
\end{itemize}
Branco, límpido, puro.
Diz-se de uma maçan branca e temporan, nas Caldas da Rainha.
A côr branca.
A parte branca do globo do ôlho, esclerótica.
Papel branco, que se toma por ponto de mira, para disparar arma de fogo e acertar o tiro.
Ponto do mira.
Intuito; fim.
Direcção.
\section{Alvo-da-serra}
\begin{itemize}
\item {Grp. gram.:m.}
\end{itemize}
Casta de uva branca dos distritos de Leiria e Lisbôa.
\section{Alvor}
\begin{itemize}
\item {Grp. gram.:m.}
\end{itemize}
\begin{itemize}
\item {Proveniência:(Lat. \textunderscore albor\textunderscore )}
\end{itemize}
O mesmo que \textunderscore alva\textunderscore , primeira luz da manhan.
Alvura.
Brilho.
Peixe de água doce, semelhante á taínha.
\section{Alvorada}
\begin{itemize}
\item {Grp. gram.:f.}
\end{itemize}
\begin{itemize}
\item {Utilização:Fig.}
\end{itemize}
\begin{itemize}
\item {Proveniência:(De \textunderscore alvor\textunderscore )}
\end{itemize}
Crepúsculo matutino.
Canto das aves ao amanhecer.
Toque de trombetas e tambores, nos quartéis militares, de madrugada.
Toque de qualquer música, ao romper da manhan.
O despontar da vida; juventude.
\section{Alvorar}
\begin{itemize}
\item {Grp. gram.:v. i.}
\end{itemize}
O mesmo que \textunderscore alvorecer\textunderscore .
\section{Alvorar}
\begin{itemize}
\item {Grp. gram.:v. i.}
\end{itemize}
\begin{itemize}
\item {Utilização:Ant.}
\end{itemize}
Levantar-se ou empinar-se (a bêsta).
(Por \textunderscore arvorar\textunderscore , de \textunderscore árvore\textunderscore )
\section{Alvorar}
\begin{itemize}
\item {Grp. gram.:v. i.}
\end{itemize}
\begin{itemize}
\item {Utilização:Pop.}
\end{itemize}
Abalar, ir-se embora, fugir.
\section{Alvorario}
\begin{itemize}
\item {Grp. gram.:m.}
\end{itemize}
\begin{itemize}
\item {Utilização:Prov.}
\end{itemize}
\begin{itemize}
\item {Utilização:trasm.}
\end{itemize}
O mesmo que \textunderscore alvarinho\textunderscore ^3.
\section{Alvorecer}
\begin{itemize}
\item {Grp. gram.:v. i.}
\end{itemize}
\begin{itemize}
\item {Utilização:Fig.}
\end{itemize}
\begin{itemize}
\item {Proveniência:(De \textunderscore alvor\textunderscore )}
\end{itemize}
Romper o dia; amanhecer.
Começar a manifestar-se (uma qualidade, uma ideia, um sentimento): \textunderscore alvoreceu nelle o talento poético\textunderscore .
\section{Alvoredo}
\begin{itemize}
\item {Grp. gram.:m.}
\end{itemize}
\begin{itemize}
\item {Utilização:Prov.}
\end{itemize}
\begin{itemize}
\item {Utilização:trasm.}
\end{itemize}
Terreno-árido, estéril.
(Do significado não se infere que seja corrupela de \textunderscore arvoredo\textunderscore )
\section{Alvorejar}
\begin{itemize}
\item {Grp. gram.:v. i.}
\end{itemize}
\begin{itemize}
\item {Grp. gram.:V. t.}
\end{itemize}
\begin{itemize}
\item {Proveniência:(De \textunderscore alvor\textunderscore )}
\end{itemize}
Mostrar-se alvo, alvorecer.
Branquear.
\section{Alvoriado}
\begin{itemize}
\item {Grp. gram.:m.  e  adj.}
\end{itemize}
\begin{itemize}
\item {Utilização:Prov.}
\end{itemize}
\begin{itemize}
\item {Utilização:trasm}
\end{itemize}
\begin{itemize}
\item {Utilização:alent}
\end{itemize}
\begin{itemize}
\item {Utilização:alg.}
\end{itemize}
O que tem cabeça leve; estroina.
\section{Alvoriçar}
\begin{itemize}
\item {Grp. gram.:v. t.}
\end{itemize}
\begin{itemize}
\item {Utilização:Ant.}
\end{itemize}
\begin{itemize}
\item {Grp. gram.:V. p.}
\end{itemize}
\begin{itemize}
\item {Utilização:Prov.}
\end{itemize}
\begin{itemize}
\item {Utilização:trasm.}
\end{itemize}
\begin{itemize}
\item {Proveniência:(De \textunderscore alvoriço\textunderscore )}
\end{itemize}
Fugir com susto.
Debandar.
Arripiar-se, pôr-se a pino, (falando-se do cabello).
\section{Alvoriço}
\begin{itemize}
\item {Grp. gram.:m.}
\end{itemize}
\begin{itemize}
\item {Utilização:Ant.}
\end{itemize}
O mesmo que \textunderscore alvorôço\textunderscore .
\section{Alvoroçadamente}
\begin{itemize}
\item {Grp. gram.:adv.}
\end{itemize}
De modo \textunderscore alvoroçado\textunderscore .
\section{Alvoroçado}
\begin{itemize}
\item {Grp. gram.:adj.}
\end{itemize}
\begin{itemize}
\item {Proveniência:(De \textunderscore alvoroçar\textunderscore )}
\end{itemize}
Agitado.
Enthusiasmado.
\section{Alvoroçador}
\begin{itemize}
\item {Grp. gram.:m.}
\end{itemize}
Aquelle que alvoroça.
\section{Alvoroçamento}
\begin{itemize}
\item {Grp. gram.:m.}
\end{itemize}
Acto de \textunderscore alvoroçar\textunderscore .
\section{Alvoroçar}
\begin{itemize}
\item {Grp. gram.:v. t.}
\end{itemize}
Causar alvorôço a.
Agitar.
Amotinar.
Assustar.
Enthusiasmar: \textunderscore alvoroçou-me o teu triumpho\textunderscore .
\section{Alvorôço}
\begin{itemize}
\item {Grp. gram.:m.}
\end{itemize}
Agitação; perturbação.
Alarma.
Enthusiasmo.
Pressa.
(Cast. \textunderscore alborozo\textunderscore )
\section{Alvorotadamente}
\begin{itemize}
\item {Grp. gram.:adv.}
\end{itemize}
Com alvorôto.
\section{Alvorotador}
\begin{itemize}
\item {Grp. gram.:m.}
\end{itemize}
Aquelle que alvorota.
\section{Alvorotamento}
\begin{itemize}
\item {Grp. gram.:m.}
\end{itemize}
Acto de \textunderscore alvorotar\textunderscore .
\section{Alvorotar}
\begin{itemize}
\item {Grp. gram.:v. t.}
\end{itemize}
\begin{itemize}
\item {Proveniência:(De \textunderscore alvorôto\textunderscore )}
\end{itemize}
O mesmo que \textunderscore alvoroçar\textunderscore .
\section{Alvorôto}
\begin{itemize}
\item {Grp. gram.:m.}
\end{itemize}
Revolta.
Borborinho.
O mesmo que \textunderscore alvorôço\textunderscore .
(Do ár., segundo Guadix e Müller. Cp. \textunderscore alvorôço\textunderscore )
\section{Alvotar}
\begin{itemize}
\item {Grp. gram.:v. i.}
\end{itemize}
\begin{itemize}
\item {Utilização:Açor}
\end{itemize}
\begin{itemize}
\item {Proveniência:(De \textunderscore voto\textunderscore )}
\end{itemize}
Cumprir voto ou promessa: \textunderscore os pescadores andavam pela rua alvotando e pedindo\textunderscore .
\section{Alvura}
\begin{itemize}
\item {Grp. gram.:f.}
\end{itemize}
Qualidade do que é alvo.
Pureza.
\section{Alxaima}
\begin{itemize}
\item {Grp. gram.:f.}
\end{itemize}
\begin{itemize}
\item {Utilização:Ant.}
\end{itemize}
Acampamento moirisco. Cf. \textunderscore Hist. de Tângere\textunderscore , 54, 143, etc.
\section{Alyshelmintho}
\begin{itemize}
\item {Grp. gram.:m.}
\end{itemize}
Gênero de vermes intestinaes.
\section{Alysmo}
\begin{itemize}
\item {Grp. gram.:m.}
\end{itemize}
\begin{itemize}
\item {Proveniência:(Do gr. \textunderscore alusmos\textunderscore )}
\end{itemize}
Ansiedade, inquietação mórbida.
\section{Alyssíneas}
\begin{itemize}
\item {Grp. gram.:f. pl.}
\end{itemize}
\begin{itemize}
\item {Proveniência:(De \textunderscore allysso\textunderscore )}
\end{itemize}
Tríbo de plantas crucíferas, segundo De-Candolle.
\section{Alysso}
\begin{itemize}
\item {Grp. gram.:m.}
\end{itemize}
\begin{itemize}
\item {Proveniência:(Do gr. \textunderscore a\textunderscore  priv. + \textunderscore luzein\textunderscore )}
\end{itemize}
Planta crucífera, ornamental.
\section{Alytarcho}
\begin{itemize}
\item {Grp. gram.:m.}
\end{itemize}
Chefe dos officiaes, encarregados da manutenção da ordem nos jogos olýmpicos.
\section{Alýxia}
\begin{itemize}
\item {Grp. gram.:f.}
\end{itemize}
\begin{itemize}
\item {Proveniência:(Do gr. \textunderscore aluxis\textunderscore )}
\end{itemize}
Gênero de plantas apocýneas.
\section{Alzátia}
\begin{itemize}
\item {Grp. gram.:f.}
\end{itemize}
\begin{itemize}
\item {Proveniência:(De \textunderscore Alzate\textunderscore , n. p.)}
\end{itemize}
Planta do Peru.
\section{A. M.}
Abrev. muito us. em cálculos astronómicos, e que designa \textunderscore ante meridiem\textunderscore , (antes do meio dia).
\section{Ama}
\begin{itemize}
\item {Grp. gram.:f.}
\end{itemize}
Mulher, que amamenta criança alheia.
Aia.
Dona de casa, em relação aos criados.
Governanta.
\textunderscore Ama sêca\textunderscore , criada que trata de crianças de peito, sem as amamentar.
\section{Ama}
\begin{itemize}
\item {Grp. gram.:f.}
\end{itemize}
Arvore da ilha de San-Thomé.
\section{Amábil}
\begin{itemize}
\item {Grp. gram.:adj.}
\end{itemize}
(Fórma alat. de \textunderscore amável\textunderscore )
\section{Amabilidade}
\begin{itemize}
\item {Grp. gram.:f.}
\end{itemize}
\begin{itemize}
\item {Proveniência:(Lat. \textunderscore amabilitas\textunderscore )}
\end{itemize}
Qualidade do que é amável.
Delicadeza; urbanidade.
\section{Amable}
\begin{itemize}
\item {Grp. gram.:adj.}
\end{itemize}
\begin{itemize}
\item {Utilização:P. us.}
\end{itemize}
O mesmo que \textunderscore amável\textunderscore . Cf. Filinto, IX, 143.
\section{Amacacado}
\begin{itemize}
\item {Grp. gram.:adj.}
\end{itemize}
Que tem modos ou feições de macaco.
Próprio de macaco: \textunderscore nariz amacacado\textunderscore .
\section{Amaçarocado}
\begin{itemize}
\item {Grp. gram.:adj.}
\end{itemize}
Que tem fórma de maçaroca.
\section{Amaçarocar}
\begin{itemize}
\item {Grp. gram.:v. t.}
\end{itemize}
Dar fórma de maçaroca a.
\section{Amachucado}
\begin{itemize}
\item {Grp. gram.:adj.}
\end{itemize}
Amarrotado.
Acabrunhado.
\section{Amachucar}
\begin{itemize}
\item {Grp. gram.:v. t.}
\end{itemize}
\begin{itemize}
\item {Utilização:Fam.}
\end{itemize}
\begin{itemize}
\item {Proveniência:(De \textunderscore machuca\textunderscore )}
\end{itemize}
Amassar, amarrotar: \textunderscore amachucar um chapéu\textunderscore . Acabrunhar: \textunderscore a morte da mulher amachucou-o\textunderscore .
\section{Amaciar}
\begin{itemize}
\item {Grp. gram.:v. t.}
\end{itemize}
Tornar macio, abrandar.
\section{Amada}
\begin{itemize}
\item {Grp. gram.:f.}
\end{itemize}
A mulher que se ama.
Namorada.
(Fem. de \textunderscore amado\textunderscore )
\section{Amadar}
\begin{itemize}
\item {Grp. gram.:v. t.}
\end{itemize}
\begin{itemize}
\item {Utilização:Prov.}
\end{itemize}
\begin{itemize}
\item {Utilização:trasm.}
\end{itemize}
Dispor (o linho) em pequenas porções, depois de maçado, para se espadelar.
\section{Amadeirado}
\begin{itemize}
\item {Grp. gram.:adj.}
\end{itemize}
Que tem côr de madeira.
\section{Amadeirar}
\begin{itemize}
\item {Grp. gram.:v. t.}
\end{itemize}
Dar côr de madeira a.
\section{Amadeístas}
\begin{itemize}
\item {Grp. gram.:m. pl.}
\end{itemize}
\begin{itemize}
\item {Proveniência:(De \textunderscore Amadeu\textunderscore , n. p.)}
\end{itemize}
Seita religiosa do século XV.
\section{Amádigo}
\begin{itemize}
\item {Grp. gram.:m.}
\end{itemize}
\begin{itemize}
\item {Utilização:Ant.}
\end{itemize}
\begin{itemize}
\item {Proveniência:(De \textunderscore ama\textunderscore )}
\end{itemize}
Honra, privilégio, que se concedia a quem criava filhos de reis, e aos lugares em que se fazia a criação.
\section{Amadio}
\begin{itemize}
\item {Grp. gram.:m.}
\end{itemize}
O mesmo que \textunderscore amavio\textunderscore :«\textunderscore fazem feitiços e dão amadios a seus maridos, para que lhes queiram o maior bem\textunderscore ». Barros, \textunderscore Espelho de Casados\textunderscore .
\section{Amado}
\begin{itemize}
\item {Grp. gram.:adj.}
\end{itemize}
\begin{itemize}
\item {Grp. gram.:M.}
\end{itemize}
Que se ama.
Querido.
Individuo amado.
\section{Amado}
\begin{itemize}
\item {Grp. gram.:m.}
\end{itemize}
\begin{itemize}
\item {Utilização:Prov.}
\end{itemize}
\begin{itemize}
\item {Utilização:minh.}
\end{itemize}
\begin{itemize}
\item {Proveniência:(De \textunderscore ama\textunderscore )}
\end{itemize}
Período, em que a ama de leite amamenta uma criança.
\section{Amadoiro}
\begin{itemize}
\item {Grp. gram.:adj.}
\end{itemize}
\begin{itemize}
\item {Proveniência:(Do lat. \textunderscore amaturus\textunderscore )}
\end{itemize}
Digno de ser amado; amável.
\section{Amador}
\begin{itemize}
\item {Grp. gram.:m.  e  adj.}
\end{itemize}
\begin{itemize}
\item {Grp. gram.:m.  e  adj.}
\end{itemize}
O que ama.
Aquelle que cultiva uma arte por simples prazer: \textunderscore photógrapho amador\textunderscore .
\section{Amadornar}
\begin{itemize}
\item {Grp. gram.:v. t.}
\end{itemize}
\begin{itemize}
\item {Utilização:Pop.}
\end{itemize}
O mesmo que \textunderscore amadorrar\textunderscore .
\section{Amadorrar}
\begin{itemize}
\item {Grp. gram.:v. t.}
\end{itemize}
(V.amodorrar)
\section{Amadouro}
\begin{itemize}
\item {Grp. gram.:adj.}
\end{itemize}
\begin{itemize}
\item {Proveniência:(Do lat. \textunderscore amaturus\textunderscore )}
\end{itemize}
Digno de ser amado; amável.
\section{Amadrinhar}
\begin{itemize}
\item {Grp. gram.:v. t.}
\end{itemize}
\begin{itemize}
\item {Utilização:Prov.}
\end{itemize}
\begin{itemize}
\item {Utilização:alent.}
\end{itemize}
\begin{itemize}
\item {Utilização:Bras. do S}
\end{itemize}
\begin{itemize}
\item {Proveniência:(De \textunderscore madrinha\textunderscore )}
\end{itemize}
Jungir (um toiro) com boi manso.
Acostumar (animaes muares) a viver com uma égua,
Disciplinar, commandando.
\section{Amadrunhador}
\begin{itemize}
\item {Grp. gram.:m.}
\end{itemize}
Instrumento de ferro, semelhante á pua e com o qual o serralheiro abre nas lâminas a cavidade, em que se há do ajustar a cabeça do prego ou parafuso.
\section{Amadrunhar}
\begin{itemize}
\item {Grp. gram.:v. t.}
\end{itemize}
Abrir com amadrunhador.
\section{Amadurar}
\begin{itemize}
\item {Grp. gram.:v. t.}
\end{itemize}
\begin{itemize}
\item {Grp. gram.:V. i.}
\end{itemize}
Tornar maduro.
O mesmo que \textunderscore amadurecer\textunderscore .
\section{Amadurecer}
\begin{itemize}
\item {Grp. gram.:v. t.  e  i.}
\end{itemize}
Tornar maduro; tornar-se maduro (no sentido próp. e fig.).
\section{Amadurecido}
\begin{itemize}
\item {Grp. gram.:adj.}
\end{itemize}
Que amadureceu.
Maduro.
\section{Amadurecimento}
\begin{itemize}
\item {Grp. gram.:m.}
\end{itemize}
Acto ou effeito de \textunderscore amadurecer\textunderscore .
\section{Âmaga}
\begin{itemize}
\item {Grp. gram.:m.}
\end{itemize}
Ébano das ilhas Filippinas.
\section{Amagar-se}
\begin{itemize}
\item {Grp. gram.:v. p.}
\end{itemize}
\begin{itemize}
\item {Utilização:Prov.}
\end{itemize}
\begin{itemize}
\item {Utilização:alent.}
\end{itemize}
Deitar-se.
Descansar, deitando-se.
Estar prostrado por doença.
(Cast. \textunderscore amagar\textunderscore )
\section{Âmago}
\begin{itemize}
\item {Grp. gram.:m.}
\end{itemize}
A medulla das plantas.
A parte mais intima de uma coisa ou pessôa.
A alma.
A essência: \textunderscore o âmago de uma questão\textunderscore .
\section{Amago}
\begin{itemize}
\item {Grp. gram.:m.}
\end{itemize}
\begin{itemize}
\item {Utilização:Ant.}
\end{itemize}
Acto de ameaçar, para extorquir alguma coisa.
\section{Âmago-furado}
\begin{itemize}
\item {Grp. gram.:m.}
\end{itemize}
\begin{itemize}
\item {Utilização:Bras}
\end{itemize}
Doença, que ataca a planta do tabaco.
\section{Amagotado}
\begin{itemize}
\item {Grp. gram.:adj.}
\end{itemize}
Que está em magotes.
\section{Amainar}
\begin{itemize}
\item {Grp. gram.:v. t.}
\end{itemize}
\begin{itemize}
\item {Grp. gram.:V. i.}
\end{itemize}
\begin{itemize}
\item {Proveniência:(Lat. hyp. \textunderscore ad-maniare\textunderscore )}
\end{itemize}
Abaixar, arrear (a vela da embarcação).
Abater.
Abrandar, afroixar: \textunderscore o vento amainou\textunderscore .
\section{Amajuacas}
\begin{itemize}
\item {Grp. gram.:m. pl.}
\end{itemize}
Tríbo de indígenas do Peru.
\section{Amalacto}
\begin{itemize}
\item {Grp. gram.:m.}
\end{itemize}
\begin{itemize}
\item {Proveniência:(Gr. \textunderscore amalaktos\textunderscore )}
\end{itemize}
Insecto coleóptero tetrâmero.
\section{Amálago}
\begin{itemize}
\item {Grp. gram.:m.}
\end{itemize}
Pimenteira das Antilhas.
\section{Amalancornado}
\begin{itemize}
\item {Grp. gram.:adj.}
\end{itemize}
\begin{itemize}
\item {Utilização:Prov.}
\end{itemize}
\begin{itemize}
\item {Utilização:trasm.}
\end{itemize}
Macambúzio, metido consigo. (Cp. \textunderscore melancolia\textunderscore )
\section{Amalçoar}
\begin{itemize}
\item {Grp. gram.:v. i.}
\end{itemize}
\begin{itemize}
\item {Utilização:Prov.}
\end{itemize}
O mesmo que \textunderscore amaldiçoar\textunderscore .
(Colhido em Turquel)
\section{Amaldiçoadamente}
\begin{itemize}
\item {Grp. gram.:adv.}
\end{itemize}
Com maldição.
\section{Amaldiçoado}
\begin{itemize}
\item {Grp. gram.:adj.}
\end{itemize}
Maldito.
Abominado.
\section{Amaldiçoador}
\begin{itemize}
\item {Grp. gram.:m.}
\end{itemize}
Aquelle que amaldiçôa.
\section{Amaldiçoar}
\begin{itemize}
\item {Grp. gram.:v. t.}
\end{itemize}
Lançar maldição a.
Execrar.
Abominar com palavras de aversão.
\section{Amalecitas}
\begin{itemize}
\item {Grp. gram.:m. pl.}
\end{itemize}
Povo árabe, que a \textunderscore Biblia\textunderscore  diz proceder de \textunderscore Amalec\textunderscore , neto de Esaú.
\section{Amaleitado}
\begin{itemize}
\item {Grp. gram.:adj.}
\end{itemize}
Doente de maleitas, maleitoso. Cf. Camillo, \textunderscore Doze Casam.\textunderscore , 127.
\section{Amalfitano}
\begin{itemize}
\item {Grp. gram.:adj.}
\end{itemize}
Relativo a Amálfi, na Itália.
Diz-se especialmente de um código náutico, regido em Amálfi no séc. X e que foi uma das bases do direito internacional marítimo, na Europa.
\section{Amálgama}
\begin{itemize}
\item {Grp. gram.:m.}
\end{itemize}
\begin{itemize}
\item {Proveniência:(Do gr. \textunderscore malagma\textunderscore )}
\end{itemize}
Liga de mercúrio com outro metal.
Mistura de coisas várias.
Ajuntamento de pessôas de diferentes classes, e qualidades.
\section{Amalgamação}
\begin{itemize}
\item {Grp. gram.:f.}
\end{itemize}
Acto de \textunderscore amalgamar\textunderscore .
\section{Amalgamador}
\begin{itemize}
\item {Grp. gram.:m.}
\end{itemize}
Aquelle que amalgama.
\section{Amalgamar}
\begin{itemize}
\item {Grp. gram.:v. t.}
\end{itemize}
Fazer amálgama de (mercúrio com outro metal). Misturar, reunir, confundir (coisas diversas).
\section{Amalgamento}
\begin{itemize}
\item {Grp. gram.:m.}
\end{itemize}
Acto ou effeito de \textunderscore amalgamar\textunderscore .
\section{Amalgâmico}
\begin{itemize}
\item {Grp. gram.:adj.}
\end{itemize}
Que se póde amalgamar ou combinar. Cf. Camillo, \textunderscore Scenas da Foz\textunderscore , 100.
\section{Amalhadeira}
\begin{itemize}
\item {Grp. gram.:f.}
\end{itemize}
\begin{itemize}
\item {Utilização:Prov.}
\end{itemize}
\begin{itemize}
\item {Utilização:alg.}
\end{itemize}
\begin{itemize}
\item {Proveniência:(De \textunderscore amalhar\textunderscore ^2)}
\end{itemize}
Rêde que amalha os peixes.
\section{Amalhar}
\begin{itemize}
\item {Grp. gram.:v. t.}
\end{itemize}
\begin{itemize}
\item {Utilização:Prov.}
\end{itemize}
\begin{itemize}
\item {Utilização:alg.}
\end{itemize}
\begin{itemize}
\item {Utilização:beir.}
\end{itemize}
\begin{itemize}
\item {Proveniência:(De \textunderscore malha\textunderscore , por \textunderscore malhada\textunderscore )}
\end{itemize}
Conduzir á malhada, meter no redil.
Abrigar.
Levar por bom camínho.
Deitar.
\section{Amalhar}
\begin{itemize}
\item {Grp. gram.:v. t.}
\end{itemize}
Prender na malha; illaquear.
\section{Amalhoar}
\begin{itemize}
\item {Grp. gram.:v. t.}
\end{itemize}
\begin{itemize}
\item {Utilização:Prov.}
\end{itemize}
\begin{itemize}
\item {Utilização:trasm.}
\end{itemize}
Vedar com malhões.
\section{Amalhoar}
\begin{itemize}
\item {Grp. gram.:v. t.}
\end{itemize}
\begin{itemize}
\item {Utilização:Pop.}
\end{itemize}
O mesmo que \textunderscore amalhar\textunderscore ^1.
\section{Âmalo}
\begin{itemize}
\item {Grp. gram.:m.}
\end{itemize}
Gênero de insectos coleópteros tetrâmeros.
\section{Amalocar}
\begin{itemize}
\item {Grp. gram.:v. t.}
\end{itemize}
\begin{itemize}
\item {Utilização:Bras}
\end{itemize}
Reunir em maloca ou aldeia; aldear.
\section{Amaltado}
\begin{itemize}
\item {Grp. gram.:adj.}
\end{itemize}
Reunido em malta.
\section{Amalteia}
\begin{itemize}
\item {Grp. gram.:f.}
\end{itemize}
\begin{itemize}
\item {Proveniência:(De \textunderscore Amaltheia\textunderscore , n. p. myth.)}
\end{itemize}
Fruto de certas rosáceas, segundo Desvaux.
\section{Amaltheia}
\begin{itemize}
\item {Grp. gram.:f.}
\end{itemize}
\begin{itemize}
\item {Proveniência:(De \textunderscore Amaltheia\textunderscore , n. p. myth.)}
\end{itemize}
Fruto de certas rosáceas, segundo Desvaux.
\section{Amalucado}
\begin{itemize}
\item {Grp. gram.:adj.}
\end{itemize}
Que parece maluco.
Que é quási maluco; aparvalhado.
Maníaco.
\section{Amame}
\begin{itemize}
\item {Grp. gram.:adj.}
\end{itemize}
Diz-se do cavallo, que tem duas côres, preta e branca.
\section{Amamentação}
\begin{itemize}
\item {Grp. gram.:f.}
\end{itemize}
Acto de \textunderscore amamentar\textunderscore .
\section{Amamentar}
\begin{itemize}
\item {Grp. gram.:v. t.}
\end{itemize}
\begin{itemize}
\item {Proveniência:(De \textunderscore mama\textunderscore )}
\end{itemize}
Criar ao peito; aleitar; dar de mamar a.
Nutrir; alimentar.
\section{Amamona}
\begin{itemize}
\item {Grp. gram.:f.}
\end{itemize}
\begin{itemize}
\item {Utilização:Bras}
\end{itemize}
Árvore silvestre, de bôa madeira para construcções.
\section{Amamu}
\begin{itemize}
\item {Grp. gram.:m.}
\end{itemize}
Gênero de plantas solâneas.
\section{Aman}
\begin{itemize}
\item {Grp. gram.:m.}
\end{itemize}
Amnistia ou perdão, concedido pelos Muçulmanos a quem não pratíca o Islamismo.
Ablução, usada entre os Turcos.
Tecido de algodão do Levante.
(Ár. \textunderscore aman\textunderscore , protecção)
\section{Amanaja}
\begin{itemize}
\item {Grp. gram.:m.}
\end{itemize}
\begin{itemize}
\item {Utilização:Bras. do N}
\end{itemize}
Espécie de tecido de algodão, em algumas tribos do Amazonas.
\section{Amanajós}
\begin{itemize}
\item {Grp. gram.:m. pl.}
\end{itemize}
Indígenas brasileiros, que habitavam no Maranhão.
\section{Amança}
\begin{itemize}
\item {Grp. gram.:f.}
\end{itemize}
\begin{itemize}
\item {Utilização:Ant.}
\end{itemize}
\begin{itemize}
\item {Proveniência:(De \textunderscore amar\textunderscore )}
\end{itemize}
Qualidade de amante.
Amor.
\section{Amancebado}
\begin{itemize}
\item {Grp. gram.:adj.}
\end{itemize}
Que vive em mancebia.
\section{Amancebar-se}
\begin{itemize}
\item {Grp. gram.:v. p.}
\end{itemize}
\begin{itemize}
\item {Proveniência:(De \textunderscore mancebo\textunderscore )}
\end{itemize}
Juntar-se, em mancebia, com alguém.
Tomar concubina.
\section{Amanchar-se}
\begin{itemize}
\item {Grp. gram.:v. p.}
\end{itemize}
\begin{itemize}
\item {Proveniência:(De \textunderscore mancha\textunderscore )}
\end{itemize}
Estar na mancha ou cama, (falando-se do javali).
\section{Amândala}
\begin{itemize}
\item {Grp. gram.:f.}
\end{itemize}
Nome vulgar de várias conchas.
\section{Amanduri}
\begin{itemize}
\item {Grp. gram.:m.}
\end{itemize}
Espécie de algodão de Alexandria.
\section{Amaneirado}
\begin{itemize}
\item {Grp. gram.:adj.}
\end{itemize}
\begin{itemize}
\item {Proveniência:(De \textunderscore amaneirar\textunderscore )}
\end{itemize}
Afectado.
Presumido.
\section{Amaneirar-se}
\begin{itemize}
\item {Grp. gram.:v. p.}
\end{itemize}
\begin{itemize}
\item {Proveniência:(De \textunderscore maneira\textunderscore )}
\end{itemize}
Tomar modos afectados, modos de presumido.--Expressão afrancesada, quanto ao sentido.
\section{Amanequinar}
\begin{itemize}
\item {Grp. gram.:v. t.}
\end{itemize}
Pintar ou esculpir, sem arte, só á vista e por imitação do manequim.
\section{Amangado}
\begin{itemize}
\item {Grp. gram.:adj.}
\end{itemize}
\begin{itemize}
\item {Proveniência:(De \textunderscore mango\textunderscore ^2)}
\end{itemize}
Que tem orgasmo.
\section{Amanhã}
\begin{itemize}
\item {fónica:á-ma}
\end{itemize}
\begin{itemize}
\item {Grp. gram.:adv.}
\end{itemize}
\begin{itemize}
\item {Grp. gram.:M.}
\end{itemize}
\begin{itemize}
\item {Proveniência:(De \textunderscore manhan\textunderscore )}
\end{itemize}
No dia seguinte ao actual.
Na época immediata a outra.
Dia seguinte.
Época futura.
\section{Amanhação}
\begin{itemize}
\item {Grp. gram.:f.}
\end{itemize}
O mesmo que \textunderscore amanho\textunderscore .
\section{Amanhan}
\begin{itemize}
\item {fónica:á-ma}
\end{itemize}
\begin{itemize}
\item {Grp. gram.:adv.}
\end{itemize}
\begin{itemize}
\item {Grp. gram.:M.}
\end{itemize}
\begin{itemize}
\item {Proveniência:(De \textunderscore manhan\textunderscore )}
\end{itemize}
No dia seguinte ao actual.
Na época immediata a outra.
Dia seguinte.
Época futura.
\section{Amanhar}
\begin{itemize}
\item {Grp. gram.:v. t.}
\end{itemize}
Dar amanho a.
Arranjar; dispor; preparar.
Tratar.
Cultivar: \textunderscore amanhar uma herdade\textunderscore .
(Por \textunderscore amanear\textunderscore , de \textunderscore manear\textunderscore )
\section{Amanhecente}
\begin{itemize}
\item {Grp. gram.:adj.}
\end{itemize}
\begin{itemize}
\item {Utilização:Ant.}
\end{itemize}
Que amanhece.
\section{Amanhecer}
\begin{itemize}
\item {Grp. gram.:v. i.}
\end{itemize}
\begin{itemize}
\item {Utilização:Fig.}
\end{itemize}
Raiar a manhan; romper o dia; esclarecer-se (o dia) com a luz da manhan.
Principiar, manifestar-se: \textunderscore a tendência para o crime já amanheceu naquella criança\textunderscore .
\section{Amanhecido}
\begin{itemize}
\item {Grp. gram.:adj.}
\end{itemize}
Que amanheceu.
\section{Amanhecimento}
\begin{itemize}
\item {Grp. gram.:m.}
\end{itemize}
Acto de \textunderscore amanhecer\textunderscore . Cf. Eça, \textunderscore Padre Amaro\textunderscore , 478.
\section{Amanho}
\begin{itemize}
\item {Grp. gram.:m.}
\end{itemize}
\begin{itemize}
\item {Proveniência:(De \textunderscore amanhar\textunderscore )}
\end{itemize}
Arranjo; preparação.
Alinho.
Utensílio.
Lavoira.
\section{Amanhuçar}
\begin{itemize}
\item {Grp. gram.:v. t.}
\end{itemize}
\begin{itemize}
\item {Utilização:Prov.}
\end{itemize}
\begin{itemize}
\item {Utilização:trasm.}
\end{itemize}
Fazer manhuços de.
\section{Amaniás}
\begin{itemize}
\item {Grp. gram.:m. pl.}
\end{itemize}
Tríbo paraense, descendente dos tupinambás.
\section{Amaninhar}
\begin{itemize}
\item {Grp. gram.:v. t.}
\end{itemize}
Tornar maninho.
\section{Amanita}
\begin{itemize}
\item {Grp. gram.:f.}
\end{itemize}
\begin{itemize}
\item {Proveniência:(De \textunderscore Amanus\textunderscore , n. p.)}
\end{itemize}
Cogumelo do gênero agárico.
\section{Amanitina}
\begin{itemize}
\item {Grp. gram.:f.}
\end{itemize}
Principio venenoso, descoberto na amanita.
\section{Amânoa}
\begin{itemize}
\item {Grp. gram.:f.}
\end{itemize}
Gênero de plantas euphorbiáceas.
\section{Amanonsiado}
\begin{itemize}
\item {Grp. gram.:adj.}
\end{itemize}
\begin{itemize}
\item {Utilização:Bras. do S}
\end{itemize}
Diz-se do cavallo, que é manso sem têr sido montado.
\section{Amansadela}
\begin{itemize}
\item {Grp. gram.:f.}
\end{itemize}
Acto de \textunderscore amansar\textunderscore .
\section{Amansador}
\begin{itemize}
\item {Grp. gram.:m.}
\end{itemize}
Aquelle que amansa.
\section{Amansadura}
\begin{itemize}
\item {Grp. gram.:f.}
\end{itemize}
(V.amansadela)
\section{Amansamento}
\begin{itemize}
\item {Grp. gram.:m.}
\end{itemize}
\begin{itemize}
\item {Utilização:Bras}
\end{itemize}
Preparação, que se faz nas árvores dos seringaes, antes de estender nellas os canequinhos de Flandres para comêço da colheita. Acto de \textunderscore amansar\textunderscore . Cf. Castilho, \textunderscore Fastos\textunderscore , I, p. XXV.
\section{Amansar}
\begin{itemize}
\item {Grp. gram.:v. t.}
\end{itemize}
Tornar manso; domesticar: \textunderscore amansar um toiro\textunderscore . Applacar.
Mitigar: \textunderscore amansar soffrimentos\textunderscore .
\section{Amansia}
\begin{itemize}
\item {Grp. gram.:f.}
\end{itemize}
\begin{itemize}
\item {Utilização:Prov.}
\end{itemize}
Acto de amansar (o toiro).
\section{Amantar}
\begin{itemize}
\item {Grp. gram.:v. t.}
\end{itemize}
Cobrir com manta.
\section{Amantar-se}
\begin{itemize}
\item {Grp. gram.:v. p.}
\end{itemize}
\begin{itemize}
\item {Utilização:Prov.}
\end{itemize}
\begin{itemize}
\item {Utilização:extrem.}
\end{itemize}
\begin{itemize}
\item {Proveniência:(De \textunderscore amante\textunderscore ^1)}
\end{itemize}
O mesmo que \textunderscore amancebar-se\textunderscore .
\section{Amante}
\begin{itemize}
\item {Grp. gram.:m.}
\end{itemize}
\begin{itemize}
\item {Grp. gram.:Adj.}
\end{itemize}
\begin{itemize}
\item {Proveniência:(Lat. \textunderscore amans\textunderscore )}
\end{itemize}
Aquelle que ama.
Namorado.
Aquelle que tem relações illícitas.
Que ama.
\section{Amante}
\begin{itemize}
\item {Grp. gram.:m.}
\end{itemize}
\begin{itemize}
\item {Utilização:Náut.}
\end{itemize}
Cabo grosso, para içar parte do apparelho náutico.
Corrente de ferro, na ostaga da gávea alta.
\textunderscore Amante da bolina\textunderscore , cabo, cujo extremo se liga á testa da vela, tendo no outro extremo um sapatilho que se enfia na pôa da bolina.
\section{Amanteigado}
\begin{itemize}
\item {Grp. gram.:adj.}
\end{itemize}
Que tem côr ou sabor de manteiga.
\section{Amanteigar}
\begin{itemize}
\item {Grp. gram.:v. t.}
\end{itemize}
Tornar brando como manteiga.
Dar côr ou sabor de manteiga a.
\section{Amantelar}
\begin{itemize}
\item {Grp. gram.:v. t.}
\end{itemize}
Fortificar; cercar de muralhas.
(Cp. \textunderscore desmantelar\textunderscore )
\section{Amantético}
\begin{itemize}
\item {Grp. gram.:adj.}
\end{itemize}
\begin{itemize}
\item {Utilização:Chul.}
\end{itemize}
\begin{itemize}
\item {Proveniência:(De \textunderscore amante\textunderscore )}
\end{itemize}
Apaixonado.
Carinhoso.
\section{Amanthina}
\begin{itemize}
\item {Grp. gram.:f.}
\end{itemize}
Essência venenosa, extrahida de um cogumelo e idêntica á \textunderscore amanitina\textunderscore .
\section{Amantiforme}
\begin{itemize}
\item {Grp. gram.:adj.}
\end{itemize}
\begin{itemize}
\item {Utilização:Des.}
\end{itemize}
\begin{itemize}
\item {Proveniência:(De \textunderscore amante\textunderscore  + \textunderscore fórma\textunderscore )}
\end{itemize}
Que tem manifestações de amor ou de affecto.
\section{Amantilhar}
\begin{itemize}
\item {Grp. gram.:v.}
\end{itemize}
\begin{itemize}
\item {Utilização:t. Náut.}
\end{itemize}
Endireitar (as vêrgas) com amantilhos.
\section{Amantilhar}
\begin{itemize}
\item {Grp. gram.:v. t.}
\end{itemize}
\begin{itemize}
\item {Utilização:Gír.}
\end{itemize}
O mesmo que \textunderscore envolver\textunderscore .
\section{Amantilho}
\begin{itemize}
\item {Grp. gram.:m.}
\end{itemize}
\begin{itemize}
\item {Utilização:Náut.}
\end{itemize}
\begin{itemize}
\item {Proveniência:(De \textunderscore amante\textunderscore ^2)}
\end{itemize}
Cabo, que sustenta as vêrgas em posição horizontal.
\section{Amantina}
\begin{itemize}
\item {Grp. gram.:f.}
\end{itemize}
Essência venenosa, extrahida de um cogumelo e idêntica á \textunderscore amanitina\textunderscore .
\section{Amanuensado}
\begin{itemize}
\item {Grp. gram.:m.}
\end{itemize}
Cargo ou funcções de amanuense.
\section{Amanuense}
\begin{itemize}
\item {Grp. gram.:m.}
\end{itemize}
\begin{itemize}
\item {Proveniência:(Lat. \textunderscore amanuensis\textunderscore )}
\end{itemize}
Escrevente.
Secretário.
Copista.
Empregado de repartição pública, encarregado geralmente de fazer cópias e registar diplomas e correspondência official.
\section{Àmão}
\begin{itemize}
\item {Grp. gram.:m.}
\end{itemize}
\begin{itemize}
\item {Proveniência:(De \textunderscore á\textunderscore  + \textunderscore mão\textunderscore )}
\end{itemize}
Alcance fácil; situação próxima:«\textunderscore ...pelo àmão da matéria prima.\textunderscore »Castilho, \textunderscore Fastos\textunderscore , I, 315.
\section{Amapá}
\begin{itemize}
\item {Grp. gram.:m.}
\end{itemize}
Planta brasileira, de suco leitoso e medicinal.
\section{Amar}
\begin{itemize}
\item {Grp. gram.:v. t.}
\end{itemize}
\begin{itemize}
\item {Proveniência:(Lat. \textunderscore amare\textunderscore )}
\end{itemize}
Têr amor a: \textunderscore Dom Pedro amou Inês de Castro\textunderscore .
Querer bem a: \textunderscore os bons filhos amam seus paes\textunderscore .
Gostar muito de: \textunderscore amar as viagens\textunderscore .
Desejar; escolher: \textunderscore amae o campo\textunderscore .
\section{Amaracarpo}
\begin{itemize}
\item {Grp. gram.:m.}
\end{itemize}
\begin{itemize}
\item {Proveniência:(Do gr. \textunderscore amara\textunderscore  + \textunderscore karpos\textunderscore )}
\end{itemize}
Arbusto japonês, da fam. das rubiáceas.
\section{Amaraceno}
\begin{itemize}
\item {Grp. gram.:m.}
\end{itemize}
\begin{itemize}
\item {Utilização:Des.}
\end{itemize}
\begin{itemize}
\item {Proveniência:(De \textunderscore amáraco\textunderscore )}
\end{itemize}
Emplasto, composto de muitas drogas.
\section{Amáraco}
\begin{itemize}
\item {Grp. gram.:m.}
\end{itemize}
\begin{itemize}
\item {Proveniência:(Gr. \textunderscore amarakos\textunderscore )}
\end{itemize}
O mesmo que \textunderscore mangerona\textunderscore .
\section{Amarado}
\begin{itemize}
\item {Grp. gram.:adj.}
\end{itemize}
Cheio de muita água; inundado:«\textunderscore olhos amarados de pranto.\textunderscore »Camillo, \textunderscore Mulher Fatal\textunderscore , 139; \textunderscore Corja\textunderscore , 136.
\section{Amara-dulcis}
\begin{itemize}
\item {fónica:dúl}
\end{itemize}
\begin{itemize}
\item {Grp. gram.:f.}
\end{itemize}
Planta solânea, o mesmo que \textunderscore dulcamara\textunderscore .
(Loc. lat.)
\section{Amaral}
\begin{itemize}
\item {Grp. gram.:f.}
\end{itemize}
\begin{itemize}
\item {Proveniência:(De \textunderscore amaro\textunderscore )}
\end{itemize}
Casta de uva preta, serôdia e muito abundante de ácidos, cultivada na Beira, Minho e Doiro.
\section{Amarália}
\begin{itemize}
\item {Grp. gram.:f.}
\end{itemize}
\begin{itemize}
\item {Proveniência:(De \textunderscore Amaral\textunderscore , n. p.)}
\end{itemize}
Gênero de plantas rubiáceas da África tropical.
\section{Amaramente}
\begin{itemize}
\item {Grp. gram.:adv.}
\end{itemize}
\begin{itemize}
\item {Proveniência:(De \textunderscore amaro\textunderscore )}
\end{itemize}
O mesmo que \textunderscore amargamente\textunderscore .
\section{Amarantáceas}
\begin{itemize}
\item {Grp. gram.:f. pl.}
\end{itemize}
Família de plantas, que tem por typo o gênero amaranto.
(Fem. pl. de \textunderscore amarantáceo\textunderscore )
\section{Amarantáceo}
\begin{itemize}
\item {Grp. gram.:adj.}
\end{itemize}
Relativo ou semelhante ao amaranto.
\section{Amarante}
\begin{itemize}
\item {Grp. gram.:f.}
\end{itemize}
Casta de uva preta da Bairrada.
\section{Amarantina}
\begin{itemize}
\item {Grp. gram.:f.}
\end{itemize}
Planta, da fam. das amarantáceas e semelhante ao amaranto.
\section{Amarantino}
\begin{itemize}
\item {Grp. gram.:adj.}
\end{itemize}
Semelhante ao \textunderscore amaranto\textunderscore .
\section{Amaranto}
\begin{itemize}
\item {Grp. gram.:m.}
\end{itemize}
\begin{itemize}
\item {Proveniência:(Gr. \textunderscore amarantos\textunderscore )}
\end{itemize}
Planta herbácea.
A flôr vermelha e avelludada do amaranto.
\section{Amarar}
\begin{itemize}
\item {Grp. gram.:v. t.}
\end{itemize}
\begin{itemize}
\item {Utilização:Ant.}
\end{itemize}
\begin{itemize}
\item {Grp. gram.:V. i.}
\end{itemize}
\begin{itemize}
\item {Grp. gram.:V. p.}
\end{itemize}
\begin{itemize}
\item {Proveniência:(De \textunderscore mar\textunderscore )}
\end{itemize}
Afastar para o mar largo.
Fazer-se ao mar largo.
Arrasar-se de água, inundar-se: \textunderscore amararam-se-me os olhos\textunderscore .
\section{Amarasmear}
\begin{itemize}
\item {Grp. gram.:v. i.}
\end{itemize}
Mostrar marasmo. Cf. Camillo, \textunderscore Volcões\textunderscore , 207.
\section{Amareado}
\begin{itemize}
\item {Grp. gram.:adj.}
\end{itemize}
\begin{itemize}
\item {Utilização:Prov.}
\end{itemize}
\begin{itemize}
\item {Utilização:trasm.}
\end{itemize}
\begin{itemize}
\item {Proveniência:(De \textunderscore amarear\textunderscore )}
\end{itemize}
Que começa a murchar e a descorar.
Que começa a secar (falando-se de roupa no estendedoiro).
\section{Amarear}
\begin{itemize}
\item {Grp. gram.:V. i.}
\end{itemize}
\begin{itemize}
\item {Utilização:Prov.}
\end{itemize}
\begin{itemize}
\item {Utilização:trasm.}
\end{itemize}
\begin{itemize}
\item {Proveniência:(De \textunderscore marear\textunderscore ^1?)}
\end{itemize}
Começar a murchar e a descorar.
Começar a secar (falando-se de roupa no estendedoiro).
\section{Amarecente}
\begin{itemize}
\item {Grp. gram.:adj.}
\end{itemize}
\begin{itemize}
\item {Proveniência:(De \textunderscore amarecer\textunderscore )}
\end{itemize}
O mesmo ou melhor que \textunderscore amarescente\textunderscore .
\section{Amarecer}
\begin{itemize}
\item {Grp. gram.:v. i.}
\end{itemize}
\begin{itemize}
\item {Utilização:Ant.}
\end{itemize}
\begin{itemize}
\item {Proveniência:(Lat. \textunderscore amarescere\textunderscore )}
\end{itemize}
Sentir amargura.
Compadecer-se, apiedar-se.
\section{Amarela}
\begin{itemize}
\item {Grp. gram.:f.}
\end{itemize}
\begin{itemize}
\item {Utilização:Fam.}
\end{itemize}
Planta, da fam. das polygaláceas.
Peça de oiro. Cf. Camillo, \textunderscore Myst. de. Lisbôa\textunderscore , I, 202.
Variedade de uva.
O mesmo que \textunderscore amarilha\textunderscore .
(Fem. de \textunderscore amarelo\textunderscore )
\section{Amarelado}
\begin{itemize}
\item {Grp. gram.:adj.}
\end{itemize}
Um tanto amarelo.
Descorado.
\section{Amarelante}
\begin{itemize}
\item {Grp. gram.:m.}
\end{itemize}
\begin{itemize}
\item {Grp. gram.:Pl.}
\end{itemize}
\begin{itemize}
\item {Utilização:Gír. de soldados.}
\end{itemize}
Variedade de trigo rijo.
Os botões amarelos das fardas; a fecharia e embutidos metállicos da arma ou do correame.
\section{Amarelão}
\begin{itemize}
\item {Grp. gram.:m.}
\end{itemize}
\begin{itemize}
\item {Utilização:Prov.}
\end{itemize}
\begin{itemize}
\item {Utilização:minh.}
\end{itemize}
Nódoa amarela na roupa.
\section{Amarelecer}
\begin{itemize}
\item {Grp. gram.:v. i.}
\end{itemize}
\begin{itemize}
\item {Grp. gram.:V. t.}
\end{itemize}
Tornar-se amarelo, empallidecer.
Fazer amarelo.
\section{Amarelecido}
\begin{itemize}
\item {Grp. gram.:adj.}
\end{itemize}
Que se tornou amarelo.
\section{Amarelecimento}
\begin{itemize}
\item {Grp. gram.:m.}
\end{itemize}
Acto ou effeito de \textunderscore amarelecer\textunderscore .
\section{Amarelejar}
\begin{itemize}
\item {Grp. gram.:v. i.}
\end{itemize}
Mostrar-se amarelo.
\section{Amarelento}
\begin{itemize}
\item {Grp. gram.:adj.}
\end{itemize}
O mesmo que \textunderscore amarelado\textunderscore .
\section{Amarelidão}
\begin{itemize}
\item {Grp. gram.:f.}
\end{itemize}
Côr amarela; pallidez.
\section{Amarelidez}
\begin{itemize}
\item {Grp. gram.:f.}
\end{itemize}
(V.amarelidão)
\section{Amarelido}
\begin{itemize}
\item {Grp. gram.:m.}
\end{itemize}
\begin{itemize}
\item {Grp. gram.:Adj.}
\end{itemize}
O mesmo que \textunderscore amarelidão\textunderscore .
O mesmo que amarelecido:«\textunderscore a côr amarelecida\textunderscore », Camillo, \textunderscore Mulher Fatal\textunderscore , 176; \textunderscore Ôlho de Vidro\textunderscore , 150.
\section{Amarelo}
\begin{itemize}
\item {Grp. gram.:adj.}
\end{itemize}
Que tem a côr do oiro, da gemma de ôvo, do enxôfre, do açafrão, do gengibre, da casca de limão.
\textunderscore Raça amarela\textunderscore , a raça humana da Ásia oriental, entre cujas características avulta a côr amarela. \textunderscore Febre amarela\textunderscore , doença epidêmica, que ataca o estômago e os intestinos, e torna amarela a côr dos doentes.
(Por \textunderscore ambarelo\textunderscore , de \textunderscore âmbar\textunderscore )
\section{Amarena}
\begin{itemize}
\item {Grp. gram.:f.}
\end{itemize}
\begin{itemize}
\item {Proveniência:(Do gr. \textunderscore marainein\textunderscore )}
\end{itemize}
Planta leguminosa, semelhante ao trevo.
\section{Amarescente}
\begin{itemize}
\item {Grp. gram.:adj}
\end{itemize}
\begin{itemize}
\item {Proveniência:(Lat. \textunderscore amarescens\textunderscore )}
\end{itemize}
Que amarga.
\section{Amarfalhar}
\begin{itemize}
\item {Grp. gram.:v. t.}
\end{itemize}
(V.amarfanhar)
\section{Amarfanhamento}
\begin{itemize}
\item {Grp. gram.:m.}
\end{itemize}
Acto de \textunderscore amarfanhar\textunderscore . Cf. Eça, \textunderscore P. Basílio\textunderscore , 122.
\section{Amarfanhar}
\begin{itemize}
\item {Grp. gram.:v. t.}
\end{itemize}
Machucar; amarrotar: \textunderscore amarfanhar o vestido\textunderscore .
Maltratar.
\section{Amargamente}
\begin{itemize}
\item {Grp. gram.:adv.}
\end{itemize}
De modo \textunderscore amargo\textunderscore .
Com amargura: \textunderscore chorou amargamente\textunderscore .
\section{Amargar}
\begin{itemize}
\item {Grp. gram.:v. i.}
\end{itemize}
\begin{itemize}
\item {Grp. gram.:V. t.}
\end{itemize}
\begin{itemize}
\item {Proveniência:(Do b. lat. \textunderscore amaricare\textunderscore )}
\end{itemize}
Têr sabor amargo: \textunderscore êste fruto amarga\textunderscore .
Tornar amargo.
Soffrer, em expiação ou compensação de: \textunderscore Hás de amargar a bôa vida que levas\textunderscore .
\section{Amargaritão}
\begin{itemize}
\item {Grp. gram.:m.}
\end{itemize}
\begin{itemize}
\item {Utilização:Ant.}
\end{itemize}
\begin{itemize}
\item {Proveniência:(De \textunderscore margarita\textunderscore )}
\end{itemize}
Espécie do pós de concha, applicados na pintura esmaltada.
\section{Amargo}
\begin{itemize}
\item {Grp. gram.:adj.}
\end{itemize}
\begin{itemize}
\item {Grp. gram.:M.}
\end{itemize}
\begin{itemize}
\item {Proveniência:(Do lat. hyp. \textunderscore amaricus\textunderscore )}
\end{itemize}
Acre.
Que tem travo, como o absinto, o fel, a quássia.
Doloroso: \textunderscore tem tido horas amargas\textunderscore .
Triste.
Sabor amargo.
\section{Amargor}
\begin{itemize}
\item {Grp. gram.:m.}
\end{itemize}
\begin{itemize}
\item {Proveniência:(De \textunderscore amargar\textunderscore )}
\end{itemize}
Qualidade do que é amargo.
Amargura.
\section{Amargôs}
\begin{itemize}
\item {Grp. gram.:adj.}
\end{itemize}
\begin{itemize}
\item {Utilização:Ant.}
\end{itemize}
\begin{itemize}
\item {Grp. gram.:M.}
\end{itemize}
\begin{itemize}
\item {Utilização:Prov.}
\end{itemize}
Amargoso.
Amargor: \textunderscore esta fruta tem um amargor esquisito\textunderscore .
\section{Amargosamente}
\begin{itemize}
\item {Grp. gram.:adv.}
\end{itemize}
De modo \textunderscore amargoso\textunderscore .
\section{Amargoseira}
\begin{itemize}
\item {Grp. gram.:f.}
\end{itemize}
\begin{itemize}
\item {Proveniência:(De \textunderscore amargoso\textunderscore )}
\end{itemize}
Planta, da fam. das meliáceas.
\section{Amargoso}
\begin{itemize}
\item {Grp. gram.:adj.}
\end{itemize}
\begin{itemize}
\item {Grp. gram.:M.}
\end{itemize}
\begin{itemize}
\item {Utilização:Bras}
\end{itemize}
\begin{itemize}
\item {Proveniência:(Do b. lat. \textunderscore amaricosus\textunderscore )}
\end{itemize}
O mesmo que \textunderscore amargo\textunderscore .
O mesmo que \textunderscore angelim\textunderscore .
\section{Amargueza}
\begin{itemize}
\item {Grp. gram.:f.}
\end{itemize}
(V.amargura)
\section{Amargura}
\begin{itemize}
\item {Grp. gram.:f.}
\end{itemize}
\begin{itemize}
\item {Utilização:Fig.}
\end{itemize}
Sabor amargo.
Angústia, afflicção.
Azedume, acrimónia.
\section{Amarguradamente}
\begin{itemize}
\item {Grp. gram.:adv.}
\end{itemize}
Com amargura.
\section{Amargurado}
\begin{itemize}
\item {Grp. gram.:adj.}
\end{itemize}
Torturado, angustiado: \textunderscore vida amargurada\textunderscore .
\section{Amargurar}
\begin{itemize}
\item {Grp. gram.:v. t.}
\end{itemize}
\begin{itemize}
\item {Proveniência:(De \textunderscore amargura\textunderscore )}
\end{itemize}
Tornar amargo.
Causar amargura a; affligir, angustiar: \textunderscore o destino amargurou-lhe a existência\textunderscore .
Tornar acrimonioso.
\section{Amaribás}
\begin{itemize}
\item {Grp. gram.:m. pl.}
\end{itemize}
Indígenas do norte do Brasil.
\section{Amaricado}
\begin{itemize}
\item {Grp. gram.:adj.}
\end{itemize}
Que se faz maricas; mulherengo.
\section{Amaricante}
\begin{itemize}
\item {Grp. gram.:adj.}
\end{itemize}
\begin{itemize}
\item {Proveniência:(Do b. lat. \textunderscore amaricans\textunderscore )}
\end{itemize}
O mesmo que \textunderscore amargoso\textunderscore .
\section{Amariçar}
\begin{itemize}
\item {Grp. gram.:v. i.}
\end{itemize}
\begin{itemize}
\item {Utilização:Prov.}
\end{itemize}
\begin{itemize}
\item {Utilização:trasm.}
\end{itemize}
Unir-se, juntar-se muito, (falando-se do gado).
\section{Amaricar-se}
\begin{itemize}
\item {Grp. gram.:v. p.}
\end{itemize}
Tornar-se maricas, mulherengo.
\section{Amaricino}
\begin{itemize}
\item {Grp. gram.:m.}
\end{itemize}
\begin{itemize}
\item {Proveniência:(Lat. \textunderscore amaricinus\textunderscore )}
\end{itemize}
Emplastro, em que entram várias substâncias aromáticas.
\section{Amariço}
\begin{itemize}
\item {Grp. gram.:m.}
\end{itemize}
\begin{itemize}
\item {Utilização:Prov.}
\end{itemize}
\begin{itemize}
\item {Utilização:trasm.}
\end{itemize}
Lugar, onde o gado amariça.
\section{Amaridar}
\begin{itemize}
\item {Grp. gram.:v. i.}
\end{itemize}
\begin{itemize}
\item {Utilização:Fig.}
\end{itemize}
Têr intimidade com alguém.
Dar-se bem com outrem.
(Cp. \textunderscore maridar\textunderscore )
\section{Amarídeo}
\begin{itemize}
\item {Grp. gram.:m.}
\end{itemize}
\begin{itemize}
\item {Proveniência:(Do lat. \textunderscore amarus\textunderscore  + gr. \textunderscore eidos\textunderscore )}
\end{itemize}
Designação pharmacêutica de substância amarga.
\section{Amarídeos}
\begin{itemize}
\item {Grp. gram.:m. pl.}
\end{itemize}
Sub-tribo de insectos coleópteros pentâmeros.
\section{Amarilha}
\begin{itemize}
\item {Grp. gram.:f.}
\end{itemize}
Cachexia aquosa das bêstas.
\section{Amarilho}
\begin{itemize}
\item {Grp. gram.:m.}
\end{itemize}
\begin{itemize}
\item {Utilização:Bras}
\end{itemize}
Atadura, ligadura.
\section{Amarília}
\begin{itemize}
\item {Grp. gram.:f.}
\end{itemize}
O mesmo que \textunderscore amarilha\textunderscore .
\section{Amarílico}
\begin{itemize}
\item {Grp. gram.:adj.}
\end{itemize}
\begin{itemize}
\item {Utilização:Bras}
\end{itemize}
Relativo á febre amarela.--É fórma arbitrária, devida ao zoologista it. Sanarelli.
\section{Amarilidáceas}
\begin{itemize}
\item {Grp. gram.:f. pl.}
\end{itemize}
O mesmo ou melhor que amarilídeas.
\section{Amarilídeas}
\begin{itemize}
\item {Grp. gram.:f. pl.}
\end{itemize}
\begin{itemize}
\item {Proveniência:(Do gr. \textunderscore Amarullis\textunderscore , n. p.)}
\end{itemize}
Família de plantas, que tem por typo a amarílis.
\section{Amarilidiforme}
\begin{itemize}
\item {Grp. gram.:adj.}
\end{itemize}
\begin{itemize}
\item {Proveniência:(Do lat. \textunderscore amaryllis\textunderscore  + \textunderscore forma\textunderscore )}
\end{itemize}
Semelhante á amarílis.
\section{Amarilígeno}
\begin{itemize}
\item {Grp. gram.:adj.}
\end{itemize}
\begin{itemize}
\item {Utilização:Bras}
\end{itemize}
Que produz febre amarela.--Outra fórma errónea, devida a Sanarelli.
\section{Amarilíneas}
\begin{itemize}
\item {Grp. gram.:f. pl.}
\end{itemize}
Grupo de plantas amarilidáceas.
\section{Amarílis}
\begin{itemize}
\item {Grp. gram.:f.}
\end{itemize}
\begin{itemize}
\item {Proveniência:(Gr. \textunderscore Amarullis\textunderscore  n. p.)}
\end{itemize}
Planta ornamental, typo das amarilídeas.
\section{Amarina}
\begin{itemize}
\item {Grp. gram.:f.}
\end{itemize}
\begin{itemize}
\item {Proveniência:(De \textunderscore amaro\textunderscore )}
\end{itemize}
Alcaloide, que se prepara pela acção do ammoníaco sôbre a essência de amêndoas amargas.
\section{Amaríneo}
\begin{itemize}
\item {Grp. gram.:adj.}
\end{itemize}
\begin{itemize}
\item {Proveniência:(De \textunderscore amaro\textunderscore )}
\end{itemize}
Que contém substâncias amargas.
\section{Amarinha}
\begin{itemize}
\item {Grp. gram.:f.}
\end{itemize}
Língua africana, do ramo ethiópico.
\section{Amarinhar}
\begin{itemize}
\item {Grp. gram.:v. t.}
\end{itemize}
\begin{itemize}
\item {Grp. gram.:V. p.}
\end{itemize}
\begin{itemize}
\item {Proveniência:(De \textunderscore marinhar\textunderscore )}
\end{itemize}
Equipar, prover de marinheiros.
Tripular.
Commandar (navio).
Inscrever-se como marinheiro.
\section{Amarinheirar-se}
\begin{itemize}
\item {Grp. gram.:v. p.}
\end{itemize}
\begin{itemize}
\item {Proveniência:(De \textunderscore marinheiro\textunderscore )}
\end{itemize}
O mesmo que [[amarinhar-se|amarinhar]].
\section{Amaríntias}
\begin{itemize}
\item {Grp. gram.:f. pl.}
\end{itemize}
\begin{itemize}
\item {Proveniência:(De \textunderscore Amaryrintho\textunderscore , n. p. de uma cidade grega)}
\end{itemize}
Festas, que se celebravam em Amarinto, em honra de Diana.
\section{Amariolar-se}
\begin{itemize}
\item {Grp. gram.:v. p.}
\end{itemize}
Tornar-se mariola. Cf. Th. Ribeiro, \textunderscore Jornadas\textunderscore , I, 114.
\section{Amaripa}
\begin{itemize}
\item {Grp. gram.:m.}
\end{itemize}
Dialecto da Guiana inglesa.
\section{Amarísias}
\begin{itemize}
\item {Grp. gram.:f. pl.}
\end{itemize}
Festas gregas, em honra de Ceres. Cf. Castilho, \textunderscore Fastos\textunderscore , I, 544.
\section{Amaritude}
\begin{itemize}
\item {Grp. gram.:f.}
\end{itemize}
\begin{itemize}
\item {Proveniência:(Lat. \textunderscore maritudo\textunderscore )}
\end{itemize}
O mesmo que \textunderscore amargura\textunderscore .
\section{Amarlotar}
\begin{itemize}
\item {Grp. gram.:v. t.}
\end{itemize}
\begin{itemize}
\item {Utilização:Ant.}
\end{itemize}
\begin{itemize}
\item {Proveniência:(De \textunderscore marlota\textunderscore )}
\end{itemize}
O mesmo que \textunderscore amarrotar\textunderscore .
\section{Amaro}
\begin{itemize}
\item {Grp. gram.:adj.}
\end{itemize}
\begin{itemize}
\item {Utilização:Poét.}
\end{itemize}
\begin{itemize}
\item {Proveniência:(Lat. \textunderscore amarus\textunderscore )}
\end{itemize}
O mesmo que \textunderscore amargo\textunderscore .
\section{Amarória}
\begin{itemize}
\item {Grp. gram.:f.}
\end{itemize}
Gênero de plantas rutáceas.
\section{Amarotado}
\begin{itemize}
\item {Grp. gram.:adj.}
\end{itemize}
Que tem modos de maroto.
\section{Amarotar-se}
\begin{itemize}
\item {Grp. gram.:v. p.}
\end{itemize}
Fazer-se maroto.
Tomar modos de maroto.
\section{Amarra}
\begin{itemize}
\item {Grp. gram.:f.}
\end{itemize}
\begin{itemize}
\item {Utilização:Gír.}
\end{itemize}
\begin{itemize}
\item {Proveniência:(De \textunderscore amarrar\textunderscore )}
\end{itemize}
Calabre, corda ou corrente de ferro, para prender o navio á âncora ou a um ponto fixo.
Corda, cordel ou corrente, com que se prende alguma coisa.
Cadeia de relógio.
\section{Amarração}
\begin{itemize}
\item {Grp. gram.:f.}
\end{itemize}
Acção de \textunderscore amarrar\textunderscore .
Lugar, onde se amarra um navio ou outra coisa.
Conjunto de ferro e boia, a que o navio se amarra.
Conjunto de amarras, com que um navio se segura, pela prôa e pela popa, a um caes.
\section{Amarrado}
\begin{itemize}
\item {Grp. gram.:adj.}
\end{itemize}
Preso com amarra.
\section{Amarradoiro}
\begin{itemize}
\item {Grp. gram.:m.}
\end{itemize}
Lugar, onde se amarra alguma coisa.
\section{Amarrador}
\begin{itemize}
\item {Grp. gram.:m.}
\end{itemize}
Aquelle que amarra.
\section{Amarradouro}
\begin{itemize}
\item {Grp. gram.:m.}
\end{itemize}
Lugar, onde se amarra alguma coisa.
\section{Amarradura}
\begin{itemize}
\item {Grp. gram.:f.}
\end{itemize}
Cabo, com que se amarra a embarcação.
Amarração.
\section{Amarrar}
\begin{itemize}
\item {Grp. gram.:v. t.}
\end{itemize}
\begin{itemize}
\item {Utilização:Bras}
\end{itemize}
\begin{itemize}
\item {Grp. gram.:Loc.}
\end{itemize}
\begin{itemize}
\item {Utilização:Port}
\end{itemize}
\begin{itemize}
\item {Utilização:Loc. da Guiné}
\end{itemize}
\begin{itemize}
\item {Grp. gram.:V. i.}
\end{itemize}
\begin{itemize}
\item {Utilização:Bras. do S}
\end{itemize}
\begin{itemize}
\item {Utilização:Cyn.}
\end{itemize}
\begin{itemize}
\item {Proveniência:(Do ár. \textunderscore marr\textunderscore )}
\end{itemize}
Segurar com amarra.
Acorrentar.
Ligar.
Contratar ou ajustar (transporte ou carreiras).
\textunderscore Amarrar pano\textunderscore , casar.
Fundear.
Parar.
Ajustar ou apostar (corridas de cavallos).
Diz-se do cão que, ao descobrir caça, pára, fixando nella a vista, até que o caçador se aproxime para a matar: \textunderscore de súbito, o cão amarrou uma perdiz\textunderscore .
\section{Amarreta}
\begin{itemize}
\item {fónica:rê}
\end{itemize}
\begin{itemize}
\item {Grp. gram.:f.}
\end{itemize}
Pequena amarra.
\section{Amarrilho}
\begin{itemize}
\item {Grp. gram.:m.}
\end{itemize}
\begin{itemize}
\item {Proveniência:(De \textunderscore amarra\textunderscore )}
\end{itemize}
Cordão, fio, com que se ata alguma coisa.
\section{Amarroado}
\begin{itemize}
\item {Grp. gram.:adj.}
\end{itemize}
\begin{itemize}
\item {Utilização:Ant.}
\end{itemize}
Teimoso.
\section{Amarroamento}
\begin{itemize}
\item {Grp. gram.:m.}
\end{itemize}
Acto de \textunderscore amarroar\textunderscore .
\section{Amarroar}
\begin{itemize}
\item {Grp. gram.:v. t.}
\end{itemize}
\begin{itemize}
\item {Grp. gram.:V. i.}
\end{itemize}
Bater com marrão.
Andar alquebrado, abatido, meditabundo.
\section{Amarroquinado}
\begin{itemize}
\item {Grp. gram.:adj.}
\end{itemize}
Semelhante ao marroquim.
\section{Amarroquinar}
\begin{itemize}
\item {Grp. gram.:v. t.}
\end{itemize}
Tornar semelhante ao marroquim.
\section{Amarrotado}
\begin{itemize}
\item {Grp. gram.:adj.}
\end{itemize}
Vincado por pressão; amachucado.
Contundido.
\section{Amarrotar}
\begin{itemize}
\item {Grp. gram.:v. t.}
\end{itemize}
Enrugar, encrespar.
Amachucar.
Enxovalhar.
Abater.
(Corr. de \textunderscore amarlotar\textunderscore )
\section{Amartelar}
\begin{itemize}
\item {Grp. gram.:v. t.}
\end{itemize}
\begin{itemize}
\item {Utilização:Prov.}
\end{itemize}
\begin{itemize}
\item {Utilização:minh.}
\end{itemize}
Bater com martello.
Importunar.
Discutir.
Vencer.
O mesmo que \textunderscore amolgar\textunderscore .
\section{Amartellar}
\begin{itemize}
\item {Grp. gram.:v. t.}
\end{itemize}
\begin{itemize}
\item {Utilização:Prov.}
\end{itemize}
\begin{itemize}
\item {Utilização:minh.}
\end{itemize}
Bater com martello.
Importunar.
Discutir.
Vencer.
O mesmo que \textunderscore amolgar\textunderscore .
\section{Amarugem}
\begin{itemize}
\item {Grp. gram.:f.}
\end{itemize}
\begin{itemize}
\item {Proveniência:(De \textunderscore amaro\textunderscore )}
\end{itemize}
Sabor ligeiramente amargo.
\section{Amarujar}
\begin{itemize}
\item {Grp. gram.:v. i.}
\end{itemize}
Sêr ligeiramente amargo.
Tornar-se amargo.
(Por \textunderscore amarejar\textunderscore , de \textunderscore amaro\textunderscore )
\section{Amarujento}
\begin{itemize}
\item {Grp. gram.:adj.}
\end{itemize}
Que amaruja.
\section{Amarulento}
\begin{itemize}
\item {Grp. gram.:adj.}
\end{itemize}
\begin{itemize}
\item {Proveniência:(Lat. \textunderscore amarulentus\textunderscore )}
\end{itemize}
Muito amargo; cheio de amargor.
\section{Amarulhar}
\begin{itemize}
\item {Grp. gram.:v. t.}
\end{itemize}
Tornar marulhoso. Cf. Filinto, VI, 264.
\section{Amaryllidáceas}
\begin{itemize}
\item {Grp. gram.:f. pl.}
\end{itemize}
O mesmo ou melhor que \textunderscore amaryllídeas\textunderscore .
\section{Amaryllídeas}
\begin{itemize}
\item {Grp. gram.:f. pl.}
\end{itemize}
\begin{itemize}
\item {Proveniência:(Do gr. \textunderscore Amarullis\textunderscore , n. p.)}
\end{itemize}
Família de plantas, que tem por typo a amarýllis.
\section{Amaryllidiforme}
\begin{itemize}
\item {Grp. gram.:adj.}
\end{itemize}
\begin{itemize}
\item {Proveniência:(Do lat. \textunderscore amaryllis\textunderscore  + \textunderscore forma\textunderscore )}
\end{itemize}
Semelhante á amarýllis.
\section{Amaryllíneas}
\begin{itemize}
\item {Grp. gram.:f. pl.}
\end{itemize}
Grupo de plantas amaryllidáceas.
\section{Amarýllis}
\begin{itemize}
\item {Grp. gram.:f.}
\end{itemize}
\begin{itemize}
\item {Proveniência:(Gr. \textunderscore Amarullis\textunderscore  n. p.)}
\end{itemize}
Planta ornamental, typo das amaryllídeas.
\section{Amarýnthias}
\begin{itemize}
\item {Grp. gram.:f. pl.}
\end{itemize}
\begin{itemize}
\item {Proveniência:(De \textunderscore Amaryrintho\textunderscore , n. p. de uma cidade grega)}
\end{itemize}
Festas, que se celebravam em Amaryntho, em honra de Diana.
\section{Amarýsias}
\begin{itemize}
\item {Grp. gram.:f. pl.}
\end{itemize}
Festas gregas, em honra de Ceres. Cf. Castilho, \textunderscore Fastos\textunderscore , I, 544.
\section{Amás}
\begin{itemize}
\item {Grp. gram.:m.}
\end{itemize}
\begin{itemize}
\item {Utilização:Ant.}
\end{itemize}
\begin{itemize}
\item {Proveniência:(Fr. \textunderscore amas\textunderscore )}
\end{itemize}
Montão; magote.
\section{Amasatina}
\begin{itemize}
\item {Grp. gram.:f.}
\end{itemize}
\begin{itemize}
\item {Proveniência:(De \textunderscore amoníaco\textunderscore  + \textunderscore isatina\textunderscore )}
\end{itemize}
Substância, que se obtém pela acção do ammoníaco sôbre a isatina.
\section{Amásia}
\begin{itemize}
\item {Grp. gram.:f.}
\end{itemize}
\begin{itemize}
\item {Proveniência:(De \textunderscore amásio\textunderscore )}
\end{itemize}
Concubina.
Amante.
\section{Amasiar-se}
\begin{itemize}
\item {Grp. gram.:v. p.}
\end{itemize}
\begin{itemize}
\item {Proveniência:(De \textunderscore amásio\textunderscore )}
\end{itemize}
O mesmo que \textunderscore amancebar-se\textunderscore .
\section{Amásio}
\begin{itemize}
\item {Grp. gram.:m.}
\end{itemize}
\begin{itemize}
\item {Utilização:Des.}
\end{itemize}
\begin{itemize}
\item {Proveniência:(Lat. \textunderscore amasius\textunderscore )}
\end{itemize}
Amante.
Indivíduo amancebado.
\section{Amasío}
\begin{itemize}
\item {Grp. gram.:m.}
\end{itemize}
O mesmo que \textunderscore mancebia\textunderscore ^1. Cf. Arn. Gama, \textunderscore Última Dona\textunderscore , 34.
\section{Amasónia}
\begin{itemize}
\item {Grp. gram.:f.}
\end{itemize}
\begin{itemize}
\item {Proveniência:(De \textunderscore Amason\textunderscore , n. p.)}
\end{itemize}
Planta herbácea americana, da fam. das verbenáceas.
\section{Amassadeira}
\begin{itemize}
\item {Grp. gram.:f.}
\end{itemize}
Mulher, que amassa farinha para fazer pão.
Máquina de amassar.
Masseira.
\section{Amassadela}
\begin{itemize}
\item {Grp. gram.:f.}
\end{itemize}
Acto de amassar.
Amassadura.
Amolgadura.
\section{Amassado}
\begin{itemize}
\item {Grp. gram.:adj.}
\end{itemize}
\begin{itemize}
\item {Utilização:Fig.}
\end{itemize}
\begin{itemize}
\item {Utilização:Náut.}
\end{itemize}
Formado, constituído:«\textunderscore estrangeiros, amassados de fraudes e mentiras.\textunderscore »Filinto, \textunderscore D. Man. II\textunderscore , 155.
Diz-se das águas turvas, barrentas.
\section{Amassadoiro}
\begin{itemize}
\item {Grp. gram.:m.}
\end{itemize}
Taboleiro ou lugar, onde se amassa.
\section{Amassador}
\begin{itemize}
\item {Grp. gram.:m.}
\end{itemize}
\begin{itemize}
\item {Grp. gram.:m.}
\end{itemize}
Aquelle que amassa.
Lugar, onde se misturam os materiaes que constituem a argamassa.
\section{Amassadouro}
\begin{itemize}
\item {Grp. gram.:m.}
\end{itemize}
Taboleiro ou lugar, onde se amassa.
\section{Amassadura}
\begin{itemize}
\item {Grp. gram.:f.}
\end{itemize}
Acto de \textunderscore amassar\textunderscore .
Fornada.
Pancada; amolgadura.
\section{Amassamento}
\begin{itemize}
\item {Grp. gram.:m.}
\end{itemize}
Acto de \textunderscore amassar\textunderscore .
\section{Amassar}
\begin{itemize}
\item {Grp. gram.:v. t.}
\end{itemize}
\begin{itemize}
\item {Proveniência:(Do lat. \textunderscore massare\textunderscore )}
\end{itemize}
Converter em massa.
Misturar.
Amachucar; achatar.
Deprimir.
\section{Amassaria}
\begin{itemize}
\item {Grp. gram.:f.}
\end{itemize}
Casa, lugar, onde se amassa farinha. Cf. \textunderscore Techn. Rur\textunderscore ., 223.
Trabalho de \textunderscore amassar\textunderscore .
\section{Amassilho}
\begin{itemize}
\item {Grp. gram.:m.}
\end{itemize}
\begin{itemize}
\item {Proveniência:(De \textunderscore amassar\textunderscore )}
\end{itemize}
Porção de farinha, que se amassa de uma vez. Apparelho de amassar.
\section{Amastozoários}
\begin{itemize}
\item {Grp. gram.:m. pl.}
\end{itemize}
\begin{itemize}
\item {Proveniência:(Do gr. \textunderscore a\textunderscore  priv.+ \textunderscore mastos\textunderscore  + \textunderscore zoon\textunderscore )}
\end{itemize}
Animaes vertebrados, que não têm mamas.
\section{Amatado}
\begin{itemize}
\item {Grp. gram.:adj.}
\end{itemize}
\begin{itemize}
\item {Utilização:Prov.}
\end{itemize}
\begin{itemize}
\item {Utilização:trasm.}
\end{itemize}
\begin{itemize}
\item {Proveniência:(De \textunderscore mata\textunderscore ^1)}
\end{itemize}
Cheio de matas, (chagas nas bêstas).
\section{Amatalado}
\begin{itemize}
\item {Grp. gram.:adj.}
\end{itemize}
\begin{itemize}
\item {Utilização:Prov.}
\end{itemize}
\begin{itemize}
\item {Utilização:trasm.}
\end{itemize}
O mesmo que \textunderscore amatado\textunderscore .
\section{Amatalar}
\begin{itemize}
\item {Grp. gram.:v. t.}
\end{itemize}
\begin{itemize}
\item {Utilização:Prov.}
\end{itemize}
\begin{itemize}
\item {Utilização:trasm.}
\end{itemize}
O mesmo que \textunderscore amatar\textunderscore ^2.
\section{Amatalotar-se}
\begin{itemize}
\item {Grp. gram.:v. p.}
\end{itemize}
Tornar-se matalote; amarinhar-se.
Associar-se com matalotes, em viagem ou em serviço de navios.--O sentido depreciativo, indicado por alguns diccion. modernos, é erróneo, e procedente talvez de confusão com \textunderscore amatular-se\textunderscore .
\section{Amatar}
\begin{itemize}
\item {Grp. gram.:v. t.}
\end{itemize}
\begin{itemize}
\item {Utilização:Ant.}
\end{itemize}
\begin{itemize}
\item {Proveniência:(De \textunderscore matar\textunderscore ?)}
\end{itemize}
Pagar; satisfazer: \textunderscore amatar um compromisso\textunderscore .
\section{Amatar}
\begin{itemize}
\item {Grp. gram.:v. t.}
\end{itemize}
\begin{itemize}
\item {Utilização:Prov.}
\end{itemize}
\begin{itemize}
\item {Utilização:trasm.}
\end{itemize}
\begin{itemize}
\item {Proveniência:(De \textunderscore mata\textunderscore ^1)}
\end{itemize}
Encher de matas ou mataduras.
\section{Amatar}
\begin{itemize}
\item {Grp. gram.:v. t.}
\end{itemize}
\begin{itemize}
\item {Utilização:Prov.}
\end{itemize}
\begin{itemize}
\item {Utilização:trasm.}
\end{itemize}
Apagar, (uma luz).
(Relaciona-se com \textunderscore amatar\textunderscore ^1?)
\section{Amathúsia}
\begin{itemize}
\item {Grp. gram.:f.}
\end{itemize}
\begin{itemize}
\item {Proveniência:(Do lat. \textunderscore Amathusia\textunderscore , n. p.)}
\end{itemize}
Gênero de insectos lepidópteros diurnos.
\section{Amatilhar}
\begin{itemize}
\item {Grp. gram.:v. t.}
\end{itemize}
Reunir em matilha.
Emparceirar.
\section{Amatividade}
\begin{itemize}
\item {Grp. gram.:f.}
\end{itemize}
\begin{itemize}
\item {Proveniência:(De \textunderscore amativo\textunderscore )}
\end{itemize}
Tendência, disposição para amar.
\section{Amativo}
\begin{itemize}
\item {Grp. gram.:adj.}
\end{itemize}
Propenso para o amor.
Inclinado a amar.
\section{Amato}
\begin{itemize}
\item {Grp. gram.:m.}
\end{itemize}
Insecto lepidóptero crepuscular.
\section{Amatongas}
\begin{itemize}
\item {Grp. gram.:m. pl.}
\end{itemize}
Povo cafreal em Lourenço-Marques.
\section{Amatoriamente}
\begin{itemize}
\item {Grp. gram.:adv.}
\end{itemize}
De modo \textunderscore amatório\textunderscore .
\section{Amatório}
\begin{itemize}
\item {Grp. gram.:adj.}
\end{itemize}
\begin{itemize}
\item {Proveniência:(Lat. \textunderscore amatorius\textunderscore )}
\end{itemize}
Relativo ao amor.
Erótico.
\section{Amatular-se}
\begin{itemize}
\item {Grp. gram.:v. p.}
\end{itemize}
\begin{itemize}
\item {Proveniência:(De \textunderscore matula\textunderscore )}
\end{itemize}
Juntar-se, abandear-se, com gente de má condição.
\section{Amatúsia}
\begin{itemize}
\item {Grp. gram.:f.}
\end{itemize}
\begin{itemize}
\item {Proveniência:(Do lat. \textunderscore Amathusia\textunderscore , n. p.)}
\end{itemize}
Gênero de insectos lepidópteros diurnos.
\section{Amaurose}
\begin{itemize}
\item {Grp. gram.:f.}
\end{itemize}
\begin{itemize}
\item {Proveniência:(Gr. \textunderscore amaurosis\textunderscore )}
\end{itemize}
Perda completa da vista, por qualquer causa.--Em geral, os diccion. dão de \textunderscore aumaurose\textunderscore  a definição que compete a \textunderscore amelopia\textunderscore .
\section{Amaurótico}
\begin{itemize}
\item {Grp. gram.:adj.}
\end{itemize}
\begin{itemize}
\item {Grp. gram.:M.}
\end{itemize}
Relativo á amaurose.
Aquelle que soffre amaurose.
\section{Amável}
\begin{itemize}
\item {Grp. gram.:adj.}
\end{itemize}
\begin{itemize}
\item {Proveniência:(Lat. \textunderscore amabilis\textunderscore )}
\end{itemize}
Que merece ser amado.
Delicado.
Agradável; lhano.
\section{Amavelmente}
\begin{itemize}
\item {Grp. gram.:adv.}
\end{itemize}
De modo \textunderscore amável\textunderscore .
Com amabilidade.
Delicadamente.
\section{Amavia}
\begin{itemize}
\item {Grp. gram.:f.}
\end{itemize}
\begin{itemize}
\item {Utilização:Ant.}
\end{itemize}
O mesmo que \textunderscore amavio\textunderscore . Cf. \textunderscore Eufrosina\textunderscore , 180.
\section{Amavio}
\begin{itemize}
\item {Grp. gram.:m.}
\end{itemize}
\begin{itemize}
\item {Proveniência:(Lat. \textunderscore amibilia\textunderscore , pl. de \textunderscore amabilis\textunderscore ?)}
\end{itemize}
Filtro, beberagem, que se suppunha despertar amor. Feitiço; encanto.
(Mais us. no pl.)
\section{Amavioso}
\begin{itemize}
\item {Grp. gram.:adj.}
\end{itemize}
\begin{itemize}
\item {Utilização:Ant.}
\end{itemize}
Em que ha amavios, encantos.
Suave, delicado.
Amável.
\section{Amazegues}
\begin{itemize}
\item {Grp. gram.:m. pl.}
\end{itemize}
O mesmo que [[Berberes|berbére]]. Cf. Herculano, \textunderscore Hist. de Port.\textunderscore , I, 49.
\section{Amazelar-se}
\begin{itemize}
\item {Grp. gram.:v. p.}
\end{itemize}
Cobrir-se de mazelas.
\section{Amazia}
\begin{itemize}
\item {Grp. gram.:f.}
\end{itemize}
Carência das mamas.
\section{Amazona}
\begin{itemize}
\item {Grp. gram.:f.}
\end{itemize}
\begin{itemize}
\item {Grp. gram.:Pl.}
\end{itemize}
\begin{itemize}
\item {Proveniência:(Lat. \textunderscore amazona\textunderscore )}
\end{itemize}
Mulher aguerrida.
Mulher, que monta cavallos.
Vestido de montar, para senhoras.
Mulheres guerreiras, indígenas da América do Sul, cuja existência é geralmente considerada lenda e que alguns consideram real. Cf. \textunderscore saussurite\textunderscore .
\section{Amazonense}
\begin{itemize}
\item {Grp. gram.:adj.}
\end{itemize}
\begin{itemize}
\item {Grp. gram.:M.}
\end{itemize}
Relativo á região do Amazonas.
Aquelle que é natural dessa região.
\section{Amazónico}
\begin{itemize}
\item {Grp. gram.:adj.}
\end{itemize}
\begin{itemize}
\item {Proveniência:(Lat. \textunderscore amazonicus\textunderscore )}
\end{itemize}
Que diz respeito a amazona.
\section{Amazónico}
\begin{itemize}
\item {Grp. gram.:adj.}
\end{itemize}
Relativo ao Amazonas ou á região do Amazonas.
\section{Amazónio}
\begin{itemize}
\item {Grp. gram.:adj.}
\end{itemize}
(V. \textunderscore amazónico\textunderscore ^2)
\section{Amazonita}
\begin{itemize}
\item {Grp. gram.:f.}
\end{itemize}
O mesmo ou melhor que \textunderscore amazonite\textunderscore .
\section{Amazonite}
\begin{itemize}
\item {Grp. gram.:f.}
\end{itemize}
\begin{itemize}
\item {Proveniência:(De \textunderscore Amazonas\textunderscore , n. p.)}
\end{itemize}
Variedade de feldspatho.
\section{Amazorrado}
\begin{itemize}
\item {Grp. gram.:adj.}
\end{itemize}
Macambúzio, sorumbático:«\textunderscore que tens, Phebo? que tão amazorrado te vejo?\textunderscore »Filinto, VIII, 11.
\section{Amazúlus}
\begin{itemize}
\item {Grp. gram.:m. pl.}
\end{itemize}
Cafres da costa oriental da África.
\section{Amba}
\begin{itemize}
\item {Grp. gram.:f.}
\end{itemize}
Gênero de plantas, (\textunderscore mangifera índica\textunderscore , Lin.).
\section{Ambacas}
\begin{itemize}
\item {Grp. gram.:m. pl.}
\end{itemize}
Tríbo de Angola.
\section{Ambages}
\begin{itemize}
\item {Grp. gram.:m. pl.}
\end{itemize}
\begin{itemize}
\item {Proveniência:(Lat. \textunderscore ambages\textunderscore )}
\end{itemize}
Rodeios; circunlóquio.
Evasiva.
\section{Ambagioso}
\begin{itemize}
\item {Grp. gram.:adj.}
\end{itemize}
\begin{itemize}
\item {Proveniência:(Lat. \textunderscore ambagiosus\textunderscore )}
\end{itemize}
Que usa ambages.
Em que há ambages.
\section{Ambaíba}
\begin{itemize}
\item {Grp. gram.:f.}
\end{itemize}
O mesmo que \textunderscore ambaúba\textunderscore .
\section{Ambaida}
\begin{itemize}
\item {Grp. gram.:f.}
\end{itemize}
Árvore brasileira, da fam. das urticáceas, e cuja madeira, porosa, é muito inflammável.
\section{Ambalão}
\begin{itemize}
\item {Grp. gram.:m.}
\end{itemize}
Árvore indiana, de frutos amarelos.
Provavelmente o mesmo que \textunderscore ambaló\textunderscore .
\section{Ambaló}
\begin{itemize}
\item {Grp. gram.:m.}
\end{itemize}
\begin{itemize}
\item {Utilização:T. de Gôa}
\end{itemize}
O mesmo que \textunderscore munguengue\textunderscore .
\section{Ambaquista}
\begin{itemize}
\item {Grp. gram.:adj.}
\end{itemize}
Relativo a Ambaca ou aos seus habitantes.
Habitante ou indígena de Ambaca.
\section{Âmbar}
\begin{itemize}
\item {Grp. gram.:m.}
\end{itemize}
\begin{itemize}
\item {Proveniência:(Do ár. \textunderscore anbar\textunderscore )}
\end{itemize}
Substância sólida, parda ou preta, de cheiro semelhante ao do almíscar.
Resina fóssil, semi-transparente e quebradiça, de côr amarela, e usada em boquilhas, rosários, etc.
\section{Ambarages}
\begin{itemize}
\item {Grp. gram.:m. pl.}
\end{itemize}
Nome, que se dava aos servidores ou escravos dos reis de Malaca. Cf. Barros, \textunderscore Déc\textunderscore . II, l. VI, c. 6.
\section{Ambar-cinzento}
\begin{itemize}
\item {Grp. gram.:m.}
\end{itemize}
Concreção, formada no tubo digestivo de um cetáceo, (\textunderscore physeter macrocephalus\textunderscore , Lin.)
\section{Ambárico}
\begin{itemize}
\item {Grp. gram.:adj.}
\end{itemize}
Relativo a âmbar.
Feito de âmbar.
\section{Ambarina}
\begin{itemize}
\item {Grp. gram.:f.}
\end{itemize}
Substância, que se extrai do âmbar pardo.
\section{Ambarino}
\begin{itemize}
\item {Grp. gram.:adj.}
\end{itemize}
Relativo ao \textunderscore âmbar\textunderscore .
\section{Ambaro}
\begin{itemize}
\item {Grp. gram.:m.}
\end{itemize}
Árvore indiana.
O mesmo que \textunderscore ambaló\textunderscore ?
\section{Ambarraja}
\begin{itemize}
\item {Grp. gram.:m.}
\end{itemize}
Soldado da guarda real, nalguns povos asiáticos. Cf. \textunderscore Peregrinação\textunderscore , c. CLXXIV.
(Cp. \textunderscore ambarages\textunderscore )
\section{Ambarvaes}
\begin{itemize}
\item {Grp. gram.:f. pl.}
\end{itemize}
\begin{itemize}
\item {Proveniência:(Do lat. \textunderscore ambarvalia\textunderscore )}
\end{itemize}
Festas romanas, em honra de Ceres, para que a deusa tornasse férteis os campos.
\section{Ambarvais}
\begin{itemize}
\item {Grp. gram.:f. pl.}
\end{itemize}
\begin{itemize}
\item {Proveniência:(Do lat. \textunderscore ambarvalia\textunderscore )}
\end{itemize}
Festas romanas, em honra de Ceres, para que a deusa tornasse férteis os campos.
\section{Ambarval}
\begin{itemize}
\item {Grp. gram.:adj.}
\end{itemize}
\begin{itemize}
\item {Proveniência:(Lat. \textunderscore ambarvalis\textunderscore )}
\end{itemize}
Diz-se da víctima, que, antes de sacrificada nas ambarvaes, era obrigada a dar volta aos campos.
\section{Ambate}
\begin{itemize}
\item {Grp. gram.:m.}
\end{itemize}
\begin{itemize}
\item {Proveniência:(Gr. \textunderscore ambates\textunderscore )}
\end{itemize}
Gênero de insectos coleópteros tetrâmeros, cujas espécies habitam a América intertropical.
\section{Ambaúba}
\begin{itemize}
\item {Grp. gram.:f.}
\end{itemize}
Árvore urticácea da América, de cujo fruto os Índios fazem vinho.
\section{Ambé}
\begin{itemize}
\item {Grp. gram.:m.}
\end{itemize}
Planta parasita do Pará.
\section{Ambel}
\begin{itemize}
\item {Grp. gram.:m.}
\end{itemize}
Planta indiana, semelhante ao nenúfar.
\section{Ambelama}
\begin{itemize}
\item {Grp. gram.:f.}
\end{itemize}
Gênero de plantas apocýneas, de frutos comestiveis e medicinaes, (\textunderscore ambelama acida\textunderscore , Aublet).
\section{Ambeta}
\begin{itemize}
\item {fónica:bê}
\end{itemize}
\begin{itemize}
\item {Grp. gram.:f.}
\end{itemize}
Ave africana.
\section{Ambi}
\begin{itemize}
\item {Grp. gram.:m.}
\end{itemize}
\begin{itemize}
\item {Proveniência:(Do gr. \textunderscore amphi\textunderscore )}
\end{itemize}
Antigo instrumento cirúrgico, para a reducção das luxações da espádua.
\section{Âmbi...}
\begin{itemize}
\item {Grp. gram.:pref.}
\end{itemize}
\begin{itemize}
\item {Proveniência:(Gr. \textunderscore amphi\textunderscore )}
\end{itemize}
Á roda.
De ambos os lados.
(Perde o \textunderscore i\textunderscore  nas palavras começadas por vogal)
\section{Ambia}
\begin{itemize}
\item {Grp. gram.:f.}
\end{itemize}
Betume das Índias, líquido e amarelado.
\section{Ambiar}
\begin{itemize}
\item {Grp. gram.:v. t.}
\end{itemize}
O mesmo que \textunderscore rodear\textunderscore :«\textunderscore disseras que o ambiava um ar divino\textunderscore ». Filinto, VII,
218.
(Má derivação de \textunderscore ambiente\textunderscore , como se fôsse \textunderscore ambiante\textunderscore )
\section{Ambiar}
\begin{itemize}
\item {Grp. gram.:m.}
\end{itemize}
\begin{itemize}
\item {Utilização:Bras}
\end{itemize}
\begin{itemize}
\item {Utilização:ant.}
\end{itemize}
O mesmo que \textunderscore panela\textunderscore .
\section{Ambição}
\begin{itemize}
\item {Grp. gram.:f.}
\end{itemize}
\begin{itemize}
\item {Proveniência:(Lat. \textunderscore ambitio\textunderscore )}
\end{itemize}
Desejo ardente (do poder, glória, riqueza).
Aspiração.
\section{Ambiciar}
\begin{itemize}
\item {Grp. gram.:v. t.}
\end{itemize}
\begin{itemize}
\item {Utilização:Ant.}
\end{itemize}
O mesmo que \textunderscore ambicionar\textunderscore .
\section{Ambicionar}
\begin{itemize}
\item {Grp. gram.:v. t.}
\end{itemize}
\begin{itemize}
\item {Proveniência:(Do lat. \textunderscore ambitio\textunderscore )}
\end{itemize}
Têr ambição de.
Cobiçar.
Desejar intensamente.
\section{Ambicioneiro}
\begin{itemize}
\item {Grp. gram.:m.  e  adj.}
\end{itemize}
\begin{itemize}
\item {Utilização:Bras. de Minas}
\end{itemize}
O mesmo que \textunderscore ambicioso\textunderscore .
\section{Ambiciosamente}
\begin{itemize}
\item {Grp. gram.:adv.}
\end{itemize}
Com ambição.
De modo \textunderscore ambicioso\textunderscore .
\section{Ambicioso}
\begin{itemize}
\item {Grp. gram.:adj.}
\end{itemize}
\begin{itemize}
\item {Grp. gram.:M.}
\end{itemize}
\begin{itemize}
\item {Proveniência:(Lat. \textunderscore ambitiosus\textunderscore )}
\end{itemize}
Que tem ambição.
Aquelle que ambiciona.
\section{Ambidestro}
\begin{itemize}
\item {Grp. gram.:adj.}
\end{itemize}
O mesmo que \textunderscore ambidextro\textunderscore .
\section{Ambidextreza}
\begin{itemize}
\item {Grp. gram.:f.}
\end{itemize}
Qualidade de \textunderscore ambidextro\textunderscore .
\section{Ambidextrismo}
\begin{itemize}
\item {Grp. gram.:m.}
\end{itemize}
O mesmo que \textunderscore ambidextreza\textunderscore .
\section{Ambidextro}
\begin{itemize}
\item {Grp. gram.:adj.}
\end{itemize}
\begin{itemize}
\item {Proveniência:(Do lat. \textunderscore ambi\textunderscore  + \textunderscore dexter\textunderscore )}
\end{itemize}
Que se serve de ambas as mãos, com destreza igual.
\section{Ambiente}
\begin{itemize}
\item {Grp. gram.:adj.}
\end{itemize}
\begin{itemize}
\item {Grp. gram.:M.}
\end{itemize}
\begin{itemize}
\item {Proveniência:(Lat. \textunderscore ambiens\textunderscore )}
\end{itemize}
Que anda ou está á roda de alguma coisa ou pessôa.
O ar que se respira.
Roda, esphera, em que se vive.
\section{Ambiesquerdo}
\begin{itemize}
\item {Grp. gram.:adj.}
\end{itemize}
\begin{itemize}
\item {Proveniência:(De \textunderscore ambi...\textunderscore  + \textunderscore esquerdo\textunderscore )}
\end{itemize}
Desajeitado de ambas as mãos.
Inhábil.
\section{Ambigênia}
\begin{itemize}
\item {Grp. gram.:f.}
\end{itemize}
\begin{itemize}
\item {Utilização:Geom.}
\end{itemize}
\begin{itemize}
\item {Proveniência:(Do gr. \textunderscore amphi\textunderscore  + \textunderscore genos\textunderscore )}
\end{itemize}
Espécie de hypérbole, de cujos ramos um cái fóra e outro dentro da asimptota.
\section{Ambígeno}
\begin{itemize}
\item {Grp. gram.:adj.}
\end{itemize}
\begin{itemize}
\item {Proveniência:(Do lat. \textunderscore ambo\textunderscore  + \textunderscore genus\textunderscore )}
\end{itemize}
Procedente de duas espécies differentes.
\section{Ambiguamente}
\begin{itemize}
\item {Grp. gram.:adj.}
\end{itemize}
Com ambiguidade.
\section{Ambiguidade}
\begin{itemize}
\item {fónica:gu-i}
\end{itemize}
\begin{itemize}
\item {Grp. gram.:f.}
\end{itemize}
\begin{itemize}
\item {Proveniência:(Lat. \textunderscore ambiguitas\textunderscore )}
\end{itemize}
Qualidade do que é ambíguo.
\section{Ambiguifloro}
\begin{itemize}
\item {fónica:gu-i}
\end{itemize}
\begin{itemize}
\item {Grp. gram.:adj.}
\end{itemize}
\begin{itemize}
\item {Proveniência:(De \textunderscore ambíguo\textunderscore  + \textunderscore flôr\textunderscore )}
\end{itemize}
Que tem flôres de corolla ambígua.
\section{Ambíguo}
\begin{itemize}
\item {Grp. gram.:adj.}
\end{itemize}
\begin{itemize}
\item {Proveniência:(Lat. \textunderscore ambiguus\textunderscore )}
\end{itemize}
Que póde têr mais que um sentido: \textunderscore expressão ambígua\textunderscore .
Duvidoso.
Incerto.
\section{Ambinhos}
\begin{itemize}
\item {Grp. gram.:adj. pl.}
\end{itemize}
Fórma carinhosa de \textunderscore ambos\textunderscore :«\textunderscore só entre nós ambinhos\textunderscore ». Castilho, \textunderscore Mil e um Myst.\textunderscore , 61.
\section{Âmbios}
\begin{itemize}
\item {Grp. gram.:m. pl.}
\end{itemize}
Antigo povo cafreal. Cf. Couto, \textunderscore Déc\textunderscore . X, c. 14.
\section{Ambíparo}
\begin{itemize}
\item {Grp. gram.:adj.}
\end{itemize}
\begin{itemize}
\item {Utilização:Bot.}
\end{itemize}
\begin{itemize}
\item {Proveniência:(Do lat. \textunderscore ambo\textunderscore  + \textunderscore pavere\textunderscore )}
\end{itemize}
Diz-se dos botões, de que saem fôlhas e flôres.
\section{Ambira}
\begin{itemize}
\item {Grp. gram.:f.}
\end{itemize}
Instrumento musical dos Ethíopes.
\section{Ambiséxuo}
\begin{itemize}
\item {fónica:sé}
\end{itemize}
\begin{itemize}
\item {Grp. gram.:adj.}
\end{itemize}
Que participa dos dois sexos:«\textunderscore ...um corpo único e ambiséxuo\textunderscore ». Castilho, \textunderscore Metam\textunderscore ., 164.
\section{Ambisséxuo}
\begin{itemize}
\item {Grp. gram.:adj.}
\end{itemize}
Que participa dos dois sexos:«\textunderscore ...um corpo único e ambiséxuo\textunderscore ». Castilho, \textunderscore Metam\textunderscore ., 164.
\section{Âmbito}
\begin{itemize}
\item {Grp. gram.:m.}
\end{itemize}
\begin{itemize}
\item {Proveniência:(Lat. \textunderscore ambitus\textunderscore )}
\end{itemize}
Contôrno.
Recinto.
Peripheria.
\section{Ambívio}
\begin{itemize}
\item {Grp. gram.:m.}
\end{itemize}
\begin{itemize}
\item {Proveniência:(Lat. \textunderscore ambivium\textunderscore )}
\end{itemize}
Encruzilhada, lugar, em que desembocam ou se cruzam duas estradas.
\section{Ambjégua}
\begin{itemize}
\item {Grp. gram.:f.}
\end{itemize}
\begin{itemize}
\item {Utilização:Bras}
\end{itemize}
Óleo vegetal odorífero.--Vejo assim escrito o t., mas o grupo \textunderscore bj\textunderscore  é avêsso
á índole da língua.
\section{Amblema}
\begin{itemize}
\item {Grp. gram.:m.}
\end{itemize}
\begin{itemize}
\item {Proveniência:(Gr. \textunderscore amblema\textunderscore )}
\end{itemize}
Gênero de molluscos acéphalos.
\section{Amblêmidos}
\begin{itemize}
\item {Grp. gram.:m. pl.}
\end{itemize}
\begin{itemize}
\item {Proveniência:(Do gr. \textunderscore amblema\textunderscore  + \textunderscore eidos\textunderscore )}
\end{itemize}
Família de molluscos, que têm por typo o amblema.
\section{Amblígono}
\begin{itemize}
\item {Grp. gram.:adj.}
\end{itemize}
\begin{itemize}
\item {Proveniência:(Do gr. \textunderscore amblus\textunderscore  + \textunderscore gonos\textunderscore )}
\end{itemize}
Que tem ângulos obtusos.
\section{Amblíope}
\begin{itemize}
\item {Grp. gram.:m.}
\end{itemize}
Aquelle que soffre \textunderscore ambliopia\textunderscore .
\section{Ambliópe}
\begin{itemize}
\item {Grp. gram.:m.}
\end{itemize}
Aquelle que soffre \textunderscore ambliopia\textunderscore .
\section{Ambliopia}
\begin{itemize}
\item {Grp. gram.:f.}
\end{itemize}
\begin{itemize}
\item {Proveniência:(Gr. \textunderscore ambluopia\textunderscore )}
\end{itemize}
Enfraquecimento, perturbação, da vista.
\section{Amblíuros}
\begin{itemize}
\item {Grp. gram.:m. pl.}
\end{itemize}
\begin{itemize}
\item {Proveniência:(Do gr. \textunderscore amblus\textunderscore  + \textunderscore oura\textunderscore )}
\end{itemize}
Gênero de peixes fósseis.
\section{Amblose}
\begin{itemize}
\item {Grp. gram.:f.}
\end{itemize}
\begin{itemize}
\item {Proveniência:(Gr. \textunderscore amblosis\textunderscore )}
\end{itemize}
O mesmo que \textunderscore abôrto\textunderscore .
\section{Amblótico}
\begin{itemize}
\item {Grp. gram.:adj.}
\end{itemize}
\begin{itemize}
\item {Grp. gram.:M.}
\end{itemize}
Relativo ao abôrto.
Que produz abôrto.
Substância, que, tomada, póde produzir abôrto.
(Cp. \textunderscore amblose\textunderscore )
\section{Amblýgono}
\begin{itemize}
\item {Grp. gram.:adj.}
\end{itemize}
\begin{itemize}
\item {Proveniência:(Do gr. \textunderscore amblus\textunderscore  + \textunderscore gonos\textunderscore )}
\end{itemize}
Que tem ângulos obtusos.
\section{Amblýope}
\begin{itemize}
\item {Grp. gram.:m.}
\end{itemize}
Aquelle que soffre \textunderscore amblyopia\textunderscore .
\section{Amblyopia}
\begin{itemize}
\item {Grp. gram.:f.}
\end{itemize}
\begin{itemize}
\item {Proveniência:(Gr. \textunderscore ambluopia\textunderscore )}
\end{itemize}
Enfraquecimento, perturbação, da vista.
\section{Amblyúros}
\begin{itemize}
\item {Grp. gram.:m. pl.}
\end{itemize}
\begin{itemize}
\item {Proveniência:(Do gr. \textunderscore amblus\textunderscore  + \textunderscore oura\textunderscore )}
\end{itemize}
Gênero de peixes fósseis.
\section{Ambo}
\begin{itemize}
\item {Grp. gram.:m.}
\end{itemize}
Árvore da Índia portuguesa.
\section{Ambolina}
\begin{itemize}
\item {Grp. gram.:f.}
\end{itemize}
Espécie de tabaco. Cf. \textunderscore Inquér. Industr.\textunderscore , II, p. 320.
\section{Ambolismal}
\begin{itemize}
\item {Grp. gram.:adj.}
\end{itemize}
\begin{itemize}
\item {Utilização:Astrol.}
\end{itemize}
Diz-se do anno composto de treze luas.
\section{Ambom}
\begin{itemize}
\item {Grp. gram.:m.}
\end{itemize}
\begin{itemize}
\item {Utilização:Ant.}
\end{itemize}
Espécie de tribuna ou púlpito e pedra, com duas escadas em sentido opposto, á entrada da capella-mór, nalgumas igrejas do século XII.
(B. lat. \textunderscore ambo\textunderscore , do lat. \textunderscore ambire\textunderscore )
\section{Ambos}
\begin{itemize}
\item {Grp. gram.:adj. pl.}
\end{itemize}
\begin{itemize}
\item {Proveniência:(Lat. \textunderscore ambo\textunderscore )}
\end{itemize}
Um e outro; os dois:«\textunderscore ambos de dois\textunderscore ». Garrett, \textunderscore Catão\textunderscore , 73.«\textunderscore Nós viemos ambos de dous\textunderscore ». Prestes, \textunderscore Autos\textunderscore .«\textunderscore Com ambas mãos\textunderscore ». Filinto, VI, 182.
\section{Ambotraço}
\begin{itemize}
\item {Grp. gram.:m.}
\end{itemize}
Instrumento, para escrever em dois papeis separados.
(Do \textunderscore ambos\textunderscore  + \textunderscore traçar\textunderscore )
\section{Ambrar}
\begin{itemize}
\item {Grp. gram.:v. i.}
\end{itemize}
\begin{itemize}
\item {Utilização:Ant.}
\end{itemize}
Fazer movimentos provocantes com as ancas. Cf. \textunderscore Cancion. da Vaticana\textunderscore , 1185.
\section{Ambrária}
\begin{itemize}
\item {Grp. gram.:f.}
\end{itemize}
Mollusco das costas da França.
\section{Ambre}
\begin{itemize}
\item {Grp. gram.:m.}
\end{itemize}
O mesmo que \textunderscore âmbar\textunderscore .
\section{Ambreada}
\begin{itemize}
\item {Grp. gram.:f.}
\end{itemize}
\begin{itemize}
\item {Proveniência:(De \textunderscore ambre\textunderscore )}
\end{itemize}
Substância que imita âmbar amarelo.
\section{Ambreado}
\begin{itemize}
\item {Grp. gram.:adj.}
\end{itemize}
Perfumado com âmbar.
\section{Ambrear}
\begin{itemize}
\item {Grp. gram.:v. t.}
\end{itemize}
\begin{itemize}
\item {Proveniência:(De \textunderscore ambre\textunderscore )}
\end{itemize}
Perfumar com âmbar.
Dar côr de âmbar a.
Aromatizar.
\section{Ambreína}
\begin{itemize}
\item {Grp. gram.:f.}
\end{itemize}
(V.ambarina)
\section{Ambreta}
\begin{itemize}
\item {fónica:brê}
\end{itemize}
\begin{itemize}
\item {Grp. gram.:f.}
\end{itemize}
\begin{itemize}
\item {Proveniência:(De \textunderscore ambre\textunderscore )}
\end{itemize}
Planta malvácea, cujas sementes têm cheiro semelhante ao do almíscar. Variedade de pêra, que cheira levemente a âmbar.
\section{Âmbria}
\begin{itemize}
\item {Grp. gram.:f.}
\end{itemize}
\begin{itemize}
\item {Utilização:Gír.}
\end{itemize}
Fome.
(Cast. \textunderscore hambre\textunderscore )
\section{Ambrósia}
\begin{itemize}
\item {Grp. gram.:f.}
\end{itemize}
\begin{itemize}
\item {Proveniência:(Gr. \textunderscore ambrosia\textunderscore )}
\end{itemize}
Alimento dos deuses.
Manjar delicioso:«\textunderscore darão, que sendo ambrósia preciosa...\textunderscore »M. Thomás, \textunderscore Insulana\textunderscore , V, 114.
Nome de diversas plantas.--Diz-se vulgarmente \textunderscore ambrosía\textunderscore , mas deve-se dizer \textunderscore ambrósia\textunderscore . Cf. Castilho, \textunderscore Sabichonas\textunderscore , 105; Camões, \textunderscore Lusiadas\textunderscore , X, 4; Filinto, VIII, 268, e XIV, 71.
\section{Ambrosiáceas}
\begin{itemize}
\item {Grp. gram.:f. pl.}
\end{itemize}
Família de plantas herbáceas, que tem por typo o gênero ambrósia.
\section{Ambrosíaco}
\begin{itemize}
\item {Grp. gram.:adj.}
\end{itemize}
Relativo á ambrósia.
Delicioso, doce, como a ambrósia dos deuses.
\section{Ambrosiano}
\begin{itemize}
\item {Grp. gram.:adj.}
\end{itemize}
Relativo a Santo-Ambrósio.
\section{Ambrosínia}
\begin{itemize}
\item {Grp. gram.:f.}
\end{itemize}
Gênero de plantas aroídeas.
\section{Ambrosino}
\begin{itemize}
\item {Grp. gram.:adj.}
\end{itemize}
O mesmo que \textunderscore ambrosíaco\textunderscore .
\section{Ambrósio}
\begin{itemize}
\item {Grp. gram.:adj.}
\end{itemize}
O mesmo que \textunderscore ambrosíaco\textunderscore . Cf. Filinto, IX, 186.
\section{Ambrosnato}
\begin{itemize}
\item {Grp. gram.:m.}
\end{itemize}
\begin{itemize}
\item {Utilização:Bras}
\end{itemize}
Espécie de creme.
\section{Ambroso}
\begin{itemize}
\item {Grp. gram.:m.}
\end{itemize}
\begin{itemize}
\item {Utilização:Bras}
\end{itemize}
\begin{itemize}
\item {Proveniência:(De \textunderscore ambre\textunderscore ?)}
\end{itemize}
Iguaria de farinha de milho, azeite e outros temperos.
\section{Ambu}
\begin{itemize}
\item {Grp. gram.:m.}
\end{itemize}
Fruta silvestre do Brasil, de que se faz doce.
Árvore, que dá esse fruto.
\section{Ambuás}
\begin{itemize}
\item {Grp. gram.:m. pl.}
\end{itemize}
Indígenas do Brasil, na região do Pará.
\section{Ambubaia}
\begin{itemize}
\item {Grp. gram.:f.}
\end{itemize}
\begin{itemize}
\item {Proveniência:(Lat. \textunderscore ambubaia\textunderscore )}
\end{itemize}
Designação das cortesans, que em Roma attrahiam os galanteadores, tocando frauta.
Cortesan, que attrahia homens, tocando e dançando nas ruas:«\textunderscore uma ambubaia ou saltadora grega\textunderscore ». S. Monteiro, \textunderscore Am. de Júlia\textunderscore , 25.
\section{Ambude}
\begin{itemize}
\item {Grp. gram.:m.}
\end{itemize}
\begin{itemize}
\item {Utilização:Ant.}
\end{itemize}
O mesmo que \textunderscore embude\textunderscore ^1.
\section{Ambuelas}
\begin{itemize}
\item {Grp. gram.:m. pl.}
\end{itemize}
Tríbo angolense.
\section{Ambuém-de-obó}
\begin{itemize}
\item {Grp. gram.:m.}
\end{itemize}
Árvore da ilha de San-Thomé.
\section{Âmbula}
\begin{itemize}
\item {Grp. gram.:f.}
\end{itemize}
\begin{itemize}
\item {Proveniência:(Do lat. \textunderscore ampulla\textunderscore ? Do b. lat. \textunderscore ampora\textunderscore , por \textunderscore amphora\textunderscore ?)}
\end{itemize}
Pequeno vaso, de gargalo estreito e bojo largo.
Vaso, em que se guardam os santos óleos.
\section{Ambulacrário}
\begin{itemize}
\item {Grp. gram.:adj.}
\end{itemize}
Relativo a ambulacro.
\section{Ambulacriforme}
\begin{itemize}
\item {Grp. gram.:adj.}
\end{itemize}
\begin{itemize}
\item {Proveniência:(Do lat. \textunderscore ambulacrum\textunderscore  + \textunderscore forma\textunderscore )}
\end{itemize}
Que tem fórma de ambulacro.
\section{Ambulacro}
\begin{itemize}
\item {Grp. gram.:m.}
\end{itemize}
\begin{itemize}
\item {Proveniência:(Lat. \textunderscore ambulacrum\textunderscore )}
\end{itemize}
Lugar, plantado de árvores, em renques regulares.
Mamilhos, em que se implantam os espinhos que cobrem os ouriços, (animaes).
Cada uma das saliências cylindricas, que cobrem a face inferior do corpo dos echinodermes e lhes servem para a locomoção.
\section{Ambulância}
\begin{itemize}
\item {Grp. gram.:f.}
\end{itemize}
\begin{itemize}
\item {Proveniência:(De \textunderscore ambulante\textunderscore )}
\end{itemize}
Hospital móvel, que acompanha fôrças militares.
Provisão de medicamentos, annexa ao serviço de comboios.
Serviço especial de transportes postaes, dirigido por pessoal destacado da direcção dos correios.
\section{Ambulante}
\begin{itemize}
\item {Grp. gram.:adj.}
\end{itemize}
\begin{itemize}
\item {Proveniência:(Lat. \textunderscore ambulans\textunderscore )}
\end{itemize}
Que anda; que não tem lugar fixo.
Que vai de terra em terra ou de rua em rua: \textunderscore vendedor ambulante\textunderscore .
\section{Ambular}
\begin{itemize}
\item {Grp. gram.:v. i.}
\end{itemize}
\begin{itemize}
\item {Utilização:P. us.}
\end{itemize}
\begin{itemize}
\item {Proveniência:(Lat. \textunderscore ambulare\textunderscore )}
\end{itemize}
Passear.
\section{Ambulativo}
\begin{itemize}
\item {Grp. gram.:adj.}
\end{itemize}
\begin{itemize}
\item {Proveniência:(Do lat. \textunderscore ambulare\textunderscore )}
\end{itemize}
Errante; que não tem lugar fixo.
Ambulante.
Vagabundo.
Que não pára num lugar.
\section{Ambulatório}
\begin{itemize}
\item {Grp. gram.:adj.}
\end{itemize}
\begin{itemize}
\item {Proveniência:(Lat. \textunderscore ambulatorius\textunderscore )}
\end{itemize}
Que se move de um lado para outro.
O mesmo que \textunderscore ambulativo\textunderscore .
\section{Ambulatriz}
\begin{itemize}
\item {Grp. gram.:f.}
\end{itemize}
\begin{itemize}
\item {Proveniência:(Lat. \textunderscore ambulatrix\textunderscore )}
\end{itemize}
Nome das prostitutas romanas, que, para atrahir os homens, se andavam mostrando pelas ruas.
\section{Ambúlia}
\begin{itemize}
\item {Grp. gram.:f.}
\end{itemize}
Gênero de plantas primuláceas.
\section{Ambulípede}
\begin{itemize}
\item {Grp. gram.:adj.}
\end{itemize}
\begin{itemize}
\item {Proveniência:(Do lat. \textunderscore ambulare\textunderscore  + \textunderscore pes\textunderscore )}
\end{itemize}
Diz-se dos mammiferos, que têm os pés bem conformados para andar.
\section{Ambundo}
\begin{itemize}
\item {Grp. gram.:m.}
\end{itemize}
(V.quimbundo)
\section{Ambundos}
\begin{itemize}
\item {Grp. gram.:m. pl.}
\end{itemize}
O mesmo que \textunderscore angolas\textunderscore .
\section{Amburbiaes}
\begin{itemize}
\item {Grp. gram.:f. pl.}
\end{itemize}
O mesmo que \textunderscore ambúrbias\textunderscore .
\section{Amburbiais}
\begin{itemize}
\item {Grp. gram.:f. pl.}
\end{itemize}
O mesmo que \textunderscore ambúrbias\textunderscore .
\section{Amburbial}
\begin{itemize}
\item {Grp. gram.:adj.}
\end{itemize}
\begin{itemize}
\item {Proveniência:(De \textunderscore ambúrbias\textunderscore )}
\end{itemize}
Relativo ao sacrifício, que os Romanos faziam, depois de uma procissão em volta da cidade.
\section{Ambúrbias}
\begin{itemize}
\item {Grp. gram.:f.}
\end{itemize}
\begin{itemize}
\item {Proveniência:(Lat. \textunderscore amburbium\textunderscore )}
\end{itemize}
Procissão, que os Romanos faziam em volta da cidade, conduzindo as víctimas de um sacrifício que rematava a festa.
\section{Ambustão}
\begin{itemize}
\item {Grp. gram.:f.}
\end{itemize}
\begin{itemize}
\item {Proveniência:(Lat. \textunderscore ambustio\textunderscore )}
\end{itemize}
Cauterização em roda.
\section{...ame}
\begin{itemize}
\item {Grp. gram.:suf.}
\end{itemize}
\begin{itemize}
\item {Proveniência:(Do suf. lat. \textunderscore ...amen\textunderscore )}
\end{itemize}
(indicativo de reunião, montão, grandeza)
\section{Ameaça}
\begin{itemize}
\item {Grp. gram.:f.}
\end{itemize}
\begin{itemize}
\item {Proveniência:(Lat. \textunderscore minaciae\textunderscore )}
\end{itemize}
Promessa de castigo ou de malefício.
Prenúncio (de mal ou de desgraça).
\section{Ameaçadamente}
\begin{itemize}
\item {Grp. gram.:adv.}
\end{itemize}
Com ameaça.
\section{Ameaçador}
\begin{itemize}
\item {Grp. gram.:adj.}
\end{itemize}
\begin{itemize}
\item {Grp. gram.:M.}
\end{itemize}
Que ameaça.
Aquelle que ameaça.
\section{Ameaçante}
\begin{itemize}
\item {Grp. gram.:adj.}
\end{itemize}
Que ameaça.
\section{Ameaçar}
\begin{itemize}
\item {Grp. gram.:v. t.}
\end{itemize}
\begin{itemize}
\item {Grp. gram.:V. i.}
\end{itemize}
Fazer ameaça a.
Intimidar.
Annunciar castigo ou malefício a.
Estar imminente.
\section{Ameaço}
\begin{itemize}
\item {Grp. gram.:m.}
\end{itemize}
O mesmo que \textunderscore ameaça\textunderscore .
Symptoma ou comêço de um ataque de doença. Cf. \textunderscore Peregrinação\textunderscore , c. XI; Usque, \textunderscore Tribulação\textunderscore , 12.
\section{Amealhador}
\begin{itemize}
\item {Grp. gram.:m.}
\end{itemize}
Aquelle que amealha.
\section{Amealhar}
\begin{itemize}
\item {Grp. gram.:v. t.}
\end{itemize}
\begin{itemize}
\item {Proveniência:(De \textunderscore mealha\textunderscore )}
\end{itemize}
Regatear na compra ou venda.
Dividir em pequenas parcellas.
Juntar, pouco a pouco; economizar.
\section{Amean}
\begin{itemize}
\item {Grp. gram.:f.}
\end{itemize}
\begin{itemize}
\item {Utilização:Prov.}
\end{itemize}
Carrête, que prende o pírtigo ao mangual.
\section{Ameandoca}
\begin{itemize}
\item {Grp. gram.:f.}
\end{itemize}
Árvore medicinal do alto Amazonas.
\section{Amear}
\begin{itemize}
\item {Grp. gram.:v. t.}
\end{itemize}
O mesmo que \textunderscore ameiar\textunderscore .
\section{Ameba}
\begin{itemize}
\item {Grp. gram.:f.}
\end{itemize}
\begin{itemize}
\item {Proveniência:(Do gr. \textunderscore amoibe\textunderscore , que muda)}
\end{itemize}
Sêr vivo, que não tem fórma própria e que ainda se não pôde decidir se é do reino vegetal ou animal.
\section{Amebeu}
\begin{itemize}
\item {Grp. gram.:adj.}
\end{itemize}
\begin{itemize}
\item {Utilização:Poét.}
\end{itemize}
\begin{itemize}
\item {Proveniência:(Lat. \textunderscore amoebeus\textunderscore )}
\end{itemize}
O mesmo que [[dialogado|dialogar]].
Diz-se de um pé de verso latino, composto de duas sýllabas longas, duas breves e uma longa.
\section{Amedeia}
\begin{itemize}
\item {Grp. gram.:f.}
\end{itemize}
Gênero de insectos dípteros.
\section{Amedrontadamente}
\begin{itemize}
\item {Grp. gram.:adv.}
\end{itemize}
Com amedrontamento.
\section{Amedrontado}
\begin{itemize}
\item {Grp. gram.:adj.}
\end{itemize}
Assustado.
\section{Amedrontador}
\begin{itemize}
\item {Grp. gram.:m.}
\end{itemize}
Aquelle que amedronta.
\section{Amedrontamento}
\begin{itemize}
\item {Grp. gram.:m.}
\end{itemize}
Acto de amedrontar.
\section{Amedrontar}
\begin{itemize}
\item {Grp. gram.:v. t.}
\end{itemize}
Causar medo a; assustar.
(Cast. \textunderscore amedrontar\textunderscore )
\section{Ameetade}
\begin{itemize}
\item {Grp. gram.:f.}
\end{itemize}
Fórma archaica de \textunderscore metade\textunderscore .
\section{Amegar}
\begin{itemize}
\item {Grp. gram.:v. t.}
\end{itemize}
\begin{itemize}
\item {Utilização:Ant.}
\end{itemize}
Provavelmente, corr. de \textunderscore amolgar\textunderscore .
\section{Âmego}
\begin{itemize}
\item {Grp. gram.:m.}
\end{itemize}
\begin{itemize}
\item {Utilização:Ant.}
\end{itemize}
O mesmo que \textunderscore âmago\textunderscore . Cf. \textunderscore Eufrosina\textunderscore , 295.
\section{Ameia}
\begin{itemize}
\item {Grp. gram.:f.}
\end{itemize}
\begin{itemize}
\item {Proveniência:(Do lat. \textunderscore ad\textunderscore  + \textunderscore moenia\textunderscore )}
\end{itemize}
Cada um dos pequenos parapeitos separados por intervallos, na parte superior das muralhas e castellos.
\section{Ameiar}
\begin{itemize}
\item {Grp. gram.:v. t.}
\end{itemize}
Guarnecer de ameias.
\section{Ameigadamente}
\begin{itemize}
\item {Grp. gram.:adv.}
\end{itemize}
Com meiguice.
\section{Ameigador}
\begin{itemize}
\item {Grp. gram.:m.}
\end{itemize}
Aquelle que ameiga.
\section{Ameigar}
\begin{itemize}
\item {Grp. gram.:v. t.}
\end{itemize}
\begin{itemize}
\item {Proveniência:(De \textunderscore meigo\textunderscore )}
\end{itemize}
Fazer meiguices a; amimar; acarinhar.
\section{Ameigo}
\begin{itemize}
\item {Grp. gram.:m.}
\end{itemize}
Acto de \textunderscore ameigar\textunderscore . Cf. Filinto, XIII, 218.
\section{Ameija}
\begin{itemize}
\item {Grp. gram.:f.}
\end{itemize}
\begin{itemize}
\item {Utilização:Ant.}
\end{itemize}
O mesmo que \textunderscore amêijoa\textunderscore .
\section{Amêijoa}
\begin{itemize}
\item {Grp. gram.:f.}
\end{itemize}
Mollusco acéphalo e comestível.
\section{Ameijoada}
\begin{itemize}
\item {Grp. gram.:f.}
\end{itemize}
Guisado de amêijoas.
\section{Ameijoada}
\begin{itemize}
\item {Grp. gram.:f.}
\end{itemize}
\begin{itemize}
\item {Proveniência:(De \textunderscore ameijoar\textunderscore )}
\end{itemize}
Redil.
Pastagem, onde o gado passa a noite.
\section{Ameijoada}
\begin{itemize}
\item {Grp. gram.:f.}
\end{itemize}
\begin{itemize}
\item {Utilização:Ant.}
\end{itemize}
\begin{itemize}
\item {Utilização:Bras}
\end{itemize}
Espera, que o caçador faz á caça.
Noite mal dormida, passada ao jôgo. Cf. Pacheco, \textunderscore Promptuário\textunderscore , 25.
(Relaciona-se com \textunderscore ameijoada\textunderscore ^2?)
\section{Ameijoar}
\begin{itemize}
\item {Grp. gram.:v. t.}
\end{itemize}
\begin{itemize}
\item {Grp. gram.:V. i.}
\end{itemize}
\begin{itemize}
\item {Proveniência:(De um thema antigo \textunderscore meijon\textunderscore , por \textunderscore meison\textunderscore , do lat. \textunderscore mansio\textunderscore ?)}
\end{itemize}
Juntar (o gado) na malhada.
Reunir de noite (animaes) ao ar livre.
Recolher-se á ameijoada.
\section{Ameiju}
\begin{itemize}
\item {Grp. gram.:m.}
\end{itemize}
Fruto brasileiro, de polpa vermelha e sabor adocicado e enjoativo.
\section{Ameiva}
\begin{itemize}
\item {Grp. gram.:f.}
\end{itemize}
Reptil sáurio do Brasil, semelhante ao lagarto.
\section{Ameixa}
\begin{itemize}
\item {Grp. gram.:f.}
\end{itemize}
\begin{itemize}
\item {Utilização:Pop.}
\end{itemize}
\begin{itemize}
\item {Proveniência:(Do lat. hyp. \textunderscore damascia\textunderscore  ou \textunderscore damascina\textunderscore , de \textunderscore Damascus\textunderscore , n. p.)}
\end{itemize}
Fruto da ameixoeira.
Bala de arma de fogo.
\section{Ameixal}
\begin{itemize}
\item {Grp. gram.:m.}
\end{itemize}
O mesmo que \textunderscore ameixial\textunderscore .
\section{Ameixeira}
\begin{itemize}
\item {Grp. gram.:f.}
\end{itemize}
O mesmo, ou melhor, que \textunderscore ameixoeira\textunderscore .
\section{Ameixial}
\begin{itemize}
\item {Grp. gram.:m.}
\end{itemize}
\begin{itemize}
\item {Proveniência:(Do b. lat. \textunderscore amexinal\textunderscore )}
\end{itemize}
Lugar, onde crescem ameixoeiras.
\section{Ameixieira}
\begin{itemize}
\item {Grp. gram.:f.}
\end{itemize}
O mesmo que \textunderscore ameixoeira\textunderscore .
\section{Amêixoa}
\begin{itemize}
\item {Grp. gram.:f.}
\end{itemize}
O mesmo que \textunderscore ameixa\textunderscore .
\section{Ameixoal}
\begin{itemize}
\item {Grp. gram.:m.}
\end{itemize}
O mesmo que \textunderscore ameixial\textunderscore .
\section{Ameixoeira}
\begin{itemize}
\item {Grp. gram.:f.}
\end{itemize}
\begin{itemize}
\item {Proveniência:(De \textunderscore amêixoa\textunderscore )}
\end{itemize}
Árvore fructífera, rosácea.
\section{Ameju}
\begin{itemize}
\item {Grp. gram.:m.}
\end{itemize}
(V.ameiju)
\section{Amejuba}
\begin{itemize}
\item {Grp. gram.:f.}
\end{itemize}
Nome de duas espécies de árvores, no Brasil, uma de madeira branca, e outra de madeira escura.
\section{Amela}
\begin{itemize}
\item {Grp. gram.:f.}
\end{itemize}
\begin{itemize}
\item {Proveniência:(Do lat. \textunderscore amellus\textunderscore )}
\end{itemize}
Planta ornamental, da fam. das compostas.
\section{Amelaçar}
\begin{itemize}
\item {Grp. gram.:v. t.}
\end{itemize}
Dar côr de melaço a.
Tornar doce, adocicar:«\textunderscore ficou-lhe dessas leituras uma linguagem amelaçada\textunderscore ». Camillo, \textunderscore Brasileira\textunderscore , 335.
\section{Amelado}
\begin{itemize}
\item {Grp. gram.:adj.}
\end{itemize}
Que tem côr de mel. Cf. Arn. Gama, \textunderscore Motim\textunderscore , 51.
\section{Amelindrar}
\begin{itemize}
\item {Grp. gram.:v. t.}
\end{itemize}
O mesmo que \textunderscore melindrar\textunderscore .
\section{Amella}
\begin{itemize}
\item {Grp. gram.:f.}
\end{itemize}
\begin{itemize}
\item {Proveniência:(Do lat. \textunderscore amellus\textunderscore )}
\end{itemize}
Planta ornamental, da fam. das compostas.
\section{Ameloado}
\begin{itemize}
\item {Grp. gram.:adj.}
\end{itemize}
Semelhante ao melão, no feitio, na côr, no cheiro, ou no gôsto.
\section{Amelopia}
\begin{itemize}
\item {Grp. gram.:f.}
\end{itemize}
\begin{itemize}
\item {Proveniência:(Do gr. \textunderscore a\textunderscore  priv., \textunderscore melas\textunderscore , negro, e \textunderscore ops\textunderscore , ôlho)}
\end{itemize}
Deminuição ou perda parcial da vista.
\section{Amelópico}
\begin{itemize}
\item {Grp. gram.:adj.}
\end{itemize}
Relativo á amelopia.
Que soffre amelopia.
\section{Amelroado}
\begin{itemize}
\item {Grp. gram.:adj.}
\end{itemize}
Que tem a côr do melro.
\section{Amembranado}
\begin{itemize}
\item {Grp. gram.:adj.}
\end{itemize}
Que se parece a uma membrana.
\section{Ámen}
\begin{itemize}
\item {fónica:ámên'}
\end{itemize}
\begin{itemize}
\item {Grp. gram.:adv.}
\end{itemize}
\begin{itemize}
\item {Grp. gram.:M. pl.}
\end{itemize}
\begin{itemize}
\item {Utilização:Fam.}
\end{itemize}
\begin{itemize}
\item {Proveniência:(T. hebr.)}
\end{itemize}
Assim seja.
Demasiada condescendência.
Approvação ou acôrdo incondicional.
\section{Amenaça}
\begin{itemize}
\item {Grp. gram.:f.}
\end{itemize}
\begin{itemize}
\item {Utilização:Ant.}
\end{itemize}
Ameaça.
(Cast. \textunderscore amenaza\textunderscore )
\section{Amência}
\begin{itemize}
\item {Grp. gram.:f.}
\end{itemize}
O mesmo que \textunderscore demência\textunderscore .
\section{Amendina}
\begin{itemize}
\item {Grp. gram.:f.}
\end{itemize}
Matéria albuminoide, isómera da caseína, e contida na amêndoa. Cf. E. Moniz,
\textunderscore Hig. do Estôm.\textunderscore , 186.
\section{Amêndoa}
\begin{itemize}
\item {Grp. gram.:f.}
\end{itemize}
\begin{itemize}
\item {Grp. gram.:Pl.}
\end{itemize}
Fruto da amendoeira.
Semente, contida em caroço.
Qualquer presente, de amêndoas cobertas ou de outro objecto, por occasião da Semana Santa ou da Páscoa.
(B. lat. \textunderscore amendola\textunderscore , do lat. \textunderscore amygdala\textunderscore )
\section{Amendoada}
\begin{itemize}
\item {Grp. gram.:f.}
\end{itemize}
Emulsão de amêndoas.
Bôlo ou doce, em que entram amêndoas.
\section{Amendoado}
\begin{itemize}
\item {Grp. gram.:adj.}
\end{itemize}
Semelhante á amêndoa.
Preparado com amêndoa.
\section{Amendoal}
\begin{itemize}
\item {Grp. gram.:m.}
\end{itemize}
\begin{itemize}
\item {Proveniência:(De \textunderscore amêndoa\textunderscore )}
\end{itemize}
Pomar de amendoeiras.
\section{Amendoeira}
\begin{itemize}
\item {Grp. gram.:f.}
\end{itemize}
\begin{itemize}
\item {Proveniência:(De \textunderscore amêndoa\textunderscore )}
\end{itemize}
Árvore rosácea, typo das amygdaláceas.
\section{Amendoim}
\begin{itemize}
\item {Grp. gram.:m.}
\end{itemize}
\begin{itemize}
\item {Proveniência:(De \textunderscore amêndoa\textunderscore )}
\end{itemize}
Planta leguminosa.
O fruto desta planta.
\section{Amendoína}
\begin{itemize}
\item {Grp. gram.:f.}
\end{itemize}
Espécie de cosmético, preparado com suco de amêndoas.
\section{Amendoirana}
\begin{itemize}
\item {Grp. gram.:f.}
\end{itemize}
Pequeno arbusto medicinal do Brasil.
\section{Amenidade}
\begin{itemize}
\item {Grp. gram.:f.}
\end{itemize}
\begin{itemize}
\item {Proveniência:(Lat. \textunderscore amoenitas\textunderscore )}
\end{itemize}
Qualidade do que é ameno.
\section{Amenidão}
\begin{itemize}
\item {Grp. gram.:f.}
\end{itemize}
O mesmo que \textunderscore amenidade\textunderscore . Cf. J. Dinis, \textunderscore Serões\textunderscore , 266.
\section{Ameninado}
\begin{itemize}
\item {Grp. gram.:adj.}
\end{itemize}
Que tem a apparência de menino.
\section{Ameninar-se}
\begin{itemize}
\item {Grp. gram.:v. p.}
\end{itemize}
Tomar a apparência de menino.
Remoçar.
Fazer-se mimoso.
\section{Amenista}
\begin{itemize}
\item {Grp. gram.:m.}
\end{itemize}
\begin{itemize}
\item {Proveniência:(De \textunderscore ámen\textunderscore )}
\end{itemize}
Aquelle que diz ámen a tudo.
Aquelle que é condescendente.
\section{Amenizar}
\begin{itemize}
\item {Grp. gram.:v. t.}
\end{itemize}
Tornar ameno, aprazivel, suave, brando: \textunderscore amenizar o estudo\textunderscore .
\section{Ameno}
\begin{itemize}
\item {Grp. gram.:adj.}
\end{itemize}
\begin{itemize}
\item {Proveniência:(Lat. \textunderscore amoenus\textunderscore )}
\end{itemize}
Deleitoso, aprazível, suave, delicado: \textunderscore conversa amena\textunderscore .
\section{Amenorreia}
\begin{itemize}
\item {Grp. gram.:f.}
\end{itemize}
\begin{itemize}
\item {Proveniência:(Do gr. \textunderscore a\textunderscore  priv. + \textunderscore men\textunderscore  + \textunderscore rhein\textunderscore )}
\end{itemize}
Supressão do catamênio.
\section{Amenorrheia}
\begin{itemize}
\item {Grp. gram.:f.}
\end{itemize}
\begin{itemize}
\item {Proveniência:(Do gr. \textunderscore a\textunderscore  priv. + \textunderscore men\textunderscore  + \textunderscore rhein\textunderscore )}
\end{itemize}
Supressão do catamênio.
\section{Amenoso}
\begin{itemize}
\item {Grp. gram.:adj.}
\end{itemize}
(V.ameno)
\section{Amenta}
\begin{itemize}
\item {Grp. gram.:f.}
\end{itemize}
Acção de \textunderscore amentar\textunderscore .
Reza por um defunto.
O que se paga ao padre pelas preces, em dia de finados.
Espécie de canto mágico, com que se suppunha attrahir os lobos.
\section{Amentação}
\begin{itemize}
\item {Grp. gram.:f.}
\end{itemize}
Acto de \textunderscore amentar\textunderscore . Cf. Ferrer, \textunderscore Direito Natural\textunderscore , 47.
\section{Amentáceas}
\begin{itemize}
\item {Grp. gram.:f. pl.}
\end{itemize}
\begin{itemize}
\item {Proveniência:(Do lat. \textunderscore amentum\textunderscore )}
\end{itemize}
Ordem de plantas, que têm amentilhos.
\section{Amentáceo}
\begin{itemize}
\item {Grp. gram.:adj.}
\end{itemize}
\begin{itemize}
\item {Utilização:Bot.}
\end{itemize}
\begin{itemize}
\item {Proveniência:(Do lat. \textunderscore amentum\textunderscore )}
\end{itemize}
Diz-se das plantas, cujas flores, ordinariamente unisexuaes, estão dispostas em amentilhos.
\section{Amentador}
\begin{itemize}
\item {Grp. gram.:m.}
\end{itemize}
\begin{itemize}
\item {Proveniência:(De \textunderscore amentar\textunderscore )}
\end{itemize}
Aquelle que amenta.
\section{Amentar}
\begin{itemize}
\item {Grp. gram.:v. t.}
\end{itemize}
O mesmo que \textunderscore dementar\textunderscore .
\section{Amentar}
\begin{itemize}
\item {Grp. gram.:v. t.}
\end{itemize}
Trazer á mente; recordar.
Rememorar (o nome dos finados); rezar por êlles.
Responsar.
Conjurar.
\section{Amentar}
\begin{itemize}
\item {Grp. gram.:v. t.}
\end{itemize}
\begin{itemize}
\item {Proveniência:(Lat. \textunderscore amentare\textunderscore )}
\end{itemize}
Ligar com correias.
\section{Amânia}
\begin{itemize}
\item {Grp. gram.:f.}
\end{itemize}
\begin{itemize}
\item {Proveniência:(De \textunderscore Ammann\textunderscore , n. p.)}
\end{itemize}
Gênero de plantas equatoriaes.
\section{Amelida}
\begin{itemize}
\item {Grp. gram.:f.}
\end{itemize}
Substância branca amorpha, obtida pela acção dos álcalis e dos ácidos sôbre a amelina.
\section{Amelina}
\begin{itemize}
\item {Grp. gram.:f.}
\end{itemize}
Base chímica, obtida pela acção dos ácidos sobre o melam.
\section{Amente}
\begin{itemize}
\item {Grp. gram.:adj.}
\end{itemize}
\begin{itemize}
\item {Utilização:Des.}
\end{itemize}
\begin{itemize}
\item {Proveniência:(Lat. \textunderscore amens\textunderscore )}
\end{itemize}
O mesmo que \textunderscore demente\textunderscore .
\section{Amentífero}
\begin{itemize}
\item {Grp. gram.:adj.}
\end{itemize}
\begin{itemize}
\item {Utilização:Bot.}
\end{itemize}
\begin{itemize}
\item {Proveniência:(Do lat. \textunderscore amentum\textunderscore  + \textunderscore ferre\textunderscore )}
\end{itemize}
Que tem amentilhos.
\section{Amentiforme}
\begin{itemize}
\item {Grp. gram.:adj.}
\end{itemize}
\begin{itemize}
\item {Utilização:Bot.}
\end{itemize}
\begin{itemize}
\item {Proveniência:(Do lat. \textunderscore amentum\textunderscore  + \textunderscore forma\textunderscore )}
\end{itemize}
Que tem fórma de amentilho.
\section{Amentilho}
\begin{itemize}
\item {Grp. gram.:m.}
\end{itemize}
\begin{itemize}
\item {Utilização:Bot.}
\end{itemize}
\begin{itemize}
\item {Proveniência:(Do lat. \textunderscore amentum\textunderscore )}
\end{itemize}
Espiga simples, de flôres unisexuaes acompanhadas de escamas, e que se separa do ramo depois da floração ou quando madura.
\section{Amentio}
\begin{itemize}
\item {Grp. gram.:m.}
\end{itemize}
\begin{itemize}
\item {Utilização:Prov.}
\end{itemize}
\begin{itemize}
\item {Utilização:alg.}
\end{itemize}
Candeia ou flôr do castanheiro.
(Por \textunderscore amentilho\textunderscore )
\section{Amentolia}
\begin{itemize}
\item {Grp. gram.:f.}
\end{itemize}
\begin{itemize}
\item {Utilização:Prov.}
\end{itemize}
\begin{itemize}
\item {Utilização:alg.}
\end{itemize}
O mesmo que \textunderscore almotolia\textunderscore .
\section{Amerceador}
\begin{itemize}
\item {Grp. gram.:m.}
\end{itemize}
Aquelle que se amerceia.
\section{Amerceamento}
\begin{itemize}
\item {Grp. gram.:m.}
\end{itemize}
Acto de \textunderscore amercear-se\textunderscore .
\section{Amercear-se}
\begin{itemize}
\item {Grp. gram.:v. p.}
\end{itemize}
Fazer mercê.
Compadecer-se.
\section{Amerger}
\begin{itemize}
\item {Grp. gram.:v. t.}
\end{itemize}
\begin{itemize}
\item {Utilização:Des.}
\end{itemize}
Abaixar.
Abater; humilhar.
(Relaciona-se provavelmente com lat. \textunderscore emergere\textunderscore )
\section{Amergulhar}
\textunderscore v. t.\textunderscore  (e der.)
O mesmo que \textunderscore mergulhar\textunderscore , etc.
\section{Americana}
\begin{itemize}
\item {Grp. gram.:f.}
\end{itemize}
Espécie de carruagem.
\section{Americanamente}
\begin{itemize}
\item {Grp. gram.:adv.}
\end{itemize}
Á maneira dos Americanos, ou dos Estados-Unidos-da-América.
\section{Americanismo}
\begin{itemize}
\item {Grp. gram.:m.}
\end{itemize}
Admiração pelas coisas da América, especialmente dos Estados-Unidos.
\section{Americanista}
\begin{itemize}
\item {Grp. gram.:m.}
\end{itemize}
Aquelle que é versado em coisas da América, especialmente em línguas, usos, ethnographia.
Partidário dos usos e costumes americanos.
\section{Americanizar}
\begin{itemize}
\item {Grp. gram.:v. t.}
\end{itemize}
Dar modos ou carácter de americano a.
\section{Americano}
\begin{itemize}
\item {Grp. gram.:adj.}
\end{itemize}
\begin{itemize}
\item {Utilização:Restrict.}
\end{itemize}
\begin{itemize}
\item {Grp. gram.:M.}
\end{itemize}
\begin{itemize}
\item {Utilização:Restrict.}
\end{itemize}
Relativo á América.
Que é próprio ou natural da América.
Relativo aos Estados-Unidos.
Aquelle que nasceu na América.
Aquelle que é natural dos Estados-Unidos.
Carro de quatro rodas, puxado por bêstas sôbre carris de ferro.
\section{Americanólogo}
\begin{itemize}
\item {Grp. gram.:m.}
\end{itemize}
Indivíduo perito em idiomas e dialectos indígenas da América.
\section{Américo}
\begin{itemize}
\item {Grp. gram.:adj.}
\end{itemize}
O mesmo que \textunderscore americano\textunderscore :«\textunderscore lá jaz nessa américa terra\textunderscore », Filinto, IV, 5.
\section{Americomania}
\begin{itemize}
\item {Grp. gram.:f.}
\end{itemize}
Admiração excessíva ou affectada por tudo que se refira a usos e costumes da América. Cf. Camillo, \textunderscore Narcót.\textunderscore , II, 210.
\section{Amerim}
\begin{itemize}
\item {Grp. gram.:f. adj.}
\end{itemize}
\begin{itemize}
\item {Proveniência:(De \textunderscore Ameria\textunderscore , n. p.)}
\end{itemize}
Nome, que alguns dão á pêra \textunderscore amorim\textunderscore , suppondo que deriva do lat. \textunderscore amerina\textunderscore .
\section{Amerina}
\begin{itemize}
\item {Grp. gram.:f.}
\end{itemize}
Gênero de plantas borragineas da América.
(Cp. \textunderscore amerim\textunderscore )
\section{Amerujar}
\begin{itemize}
\item {Grp. gram.:v. i.}
\end{itemize}
O mesmo que \textunderscore merujar\textunderscore .
\section{Amesendar-se}
\begin{itemize}
\item {Grp. gram.:v. p.}
\end{itemize}
\begin{itemize}
\item {Grp. gram.:V. p.}
\end{itemize}
Sentar-se á mesa.
Refestelar-se; repotrear-se.
Acocorar-se, alapar-se.
\section{Amesendrar-se}
\begin{itemize}
\item {Grp. gram.:V. p.}
\end{itemize}
O mesmo que \textunderscore amesendar-se\textunderscore . Cf. Camillo, \textunderscore Volcões\textunderscore , 103; \textunderscore Brasileira\textunderscore , 357; \textunderscore Noites de Insómn.\textunderscore  I, 52.
\section{Amesquinhamento}
\begin{itemize}
\item {Grp. gram.:m.}
\end{itemize}
Acto de \textunderscore amesquinhar\textunderscore .
\section{Amesquinhar}
\begin{itemize}
\item {Grp. gram.:v. t.}
\end{itemize}
Tornar mesquinho.
Deprimir; humilhar.
\section{Amestrador}
\begin{itemize}
\item {Grp. gram.:m.}
\end{itemize}
Aquelle que amestra.
\section{Amestrar}
\begin{itemize}
\item {Grp. gram.:v. i.}
\end{itemize}
Tornar mestre.
Industriar.
Ensinar.
\section{Ametábolo}
\begin{itemize}
\item {Grp. gram.:m.  e  adj.}
\end{itemize}
Insecto, que não tem metamorphoses completas, mas só mudanças successivas de pelle, como os hemípteros, os orthópteros, etc.
\section{Ametade}
\begin{itemize}
\item {Grp. gram.:f.}
\end{itemize}
O mesmo que \textunderscore metade\textunderscore :«\textunderscore quási ametade da gente\textunderscore ». Góes, \textunderscore Chrón. de D. Man.\textunderscore , p. I, c. 37. Cf. Castilho, \textunderscore Tartufo\textunderscore , 21.
\section{Ametalar}
\begin{itemize}
\item {Grp. gram.:v. i.}
\end{itemize}
Misturar bom metal.
Ornar com metal.
Dar a apparência de metal a.
\section{Ametamorfose}
\begin{itemize}
\item {Grp. gram.:f.}
\end{itemize}
\begin{itemize}
\item {Proveniência:(Do gr. \textunderscore a\textunderscore  priv. + \textunderscore metamorphosis\textunderscore )}
\end{itemize}
Phenómeno, observado em alguns insectos que, em vez de se metamorfosearem, mudam apenas de pelle.
\section{Ametamorphose}
\begin{itemize}
\item {Grp. gram.:f.}
\end{itemize}
\begin{itemize}
\item {Proveniência:(Do gr. \textunderscore a\textunderscore  priv. + \textunderscore metamorphosis\textunderscore )}
\end{itemize}
Phenómeno, observado em alguns insectos que, em vez de se metamorphosearem, mudam apenas de pelle.
\section{Amethysta}
\begin{itemize}
\item {Grp. gram.:f.}
\end{itemize}
\begin{itemize}
\item {Proveniência:(Do gr. \textunderscore amethustos\textunderscore )}
\end{itemize}
Pedra preciosa, roxa, variedade de quartzo.
\section{Amethýstea}
\begin{itemize}
\item {Grp. gram.:f.}
\end{itemize}
\begin{itemize}
\item {Proveniência:(De \textunderscore amethysta\textunderscore )}
\end{itemize}
Planta labiada, de pequenas flôres roxas.
\section{Amethýstico}
\begin{itemize}
\item {Grp. gram.:adj.}
\end{itemize}
Relativo á amethysta.
Que tem a côr e o brilho da amethysta.
\section{Amethysto}
\begin{itemize}
\item {Grp. gram.:m.}
\end{itemize}
\begin{itemize}
\item {Utilização:Des.}
\end{itemize}
O mesmo ou melhor que \textunderscore amethysta\textunderscore .
\section{Ametista}
\begin{itemize}
\item {Grp. gram.:f.}
\end{itemize}
\begin{itemize}
\item {Proveniência:(Do gr. \textunderscore amethustos\textunderscore )}
\end{itemize}
Pedra preciosa, roxa, variedade de quartzo.
\section{Ametístea}
\begin{itemize}
\item {Grp. gram.:f.}
\end{itemize}
\begin{itemize}
\item {Proveniência:(De \textunderscore amethysta\textunderscore )}
\end{itemize}
Planta labiada, de pequenas flôres roxas.
\section{Ametístico}
\begin{itemize}
\item {Grp. gram.:adj.}
\end{itemize}
Relativo á ametista.
Que tem a côr e o brilho da ametista.
\section{Ametisto}
\begin{itemize}
\item {Grp. gram.:m.}
\end{itemize}
\begin{itemize}
\item {Utilização:Des.}
\end{itemize}
O mesmo ou melhor que \textunderscore ametista\textunderscore .
\section{Ametria}
\begin{itemize}
\item {Grp. gram.:f.}
\end{itemize}
\begin{itemize}
\item {Proveniência:(Do gr. \textunderscore a\textunderscore  priv. e \textunderscore metron\textunderscore )}
\end{itemize}
Ausência de medida.
\section{Ametropia}
\begin{itemize}
\item {Grp. gram.:f.}
\end{itemize}
\begin{itemize}
\item {Proveniência:(De \textunderscore amétropo\textunderscore )}
\end{itemize}
Designação collectiva da myopia e da presbitia.
\section{Amétropo}
\begin{itemize}
\item {Grp. gram.:adj.}
\end{itemize}
\begin{itemize}
\item {Proveniência:(Do gr. \textunderscore a\textunderscore  priv. + \textunderscore metron\textunderscore  + \textunderscore ops\textunderscore )}
\end{itemize}
Diz-se do ôlho, em que o foco do apparelho dióptrico está situado adeante ou atrás do plano retiniano da visão distínta.
\section{Amezinhador}
\begin{itemize}
\item {Grp. gram.:m.}
\end{itemize}
Aquelle que amezinha.
\section{Amezinhar}
\begin{itemize}
\item {Grp. gram.:v. t.}
\end{itemize}
Tratar com mèzinhas.
\section{Amhárico}
\begin{itemize}
\item {Grp. gram.:m.}
\end{itemize}
Um dos dialectos da Abyssínia.
\section{Âmi}
\begin{itemize}
\item {Grp. gram.:m.}
\end{itemize}
O mesmo que \textunderscore âmio\textunderscore .
\section{Âmia}
\begin{itemize}
\item {Grp. gram.:f.}
\end{itemize}
\begin{itemize}
\item {Proveniência:(Gr. \textunderscore amia\textunderscore )}
\end{itemize}
Gênero de peixes ganoides.
\section{Amiádeos}
\begin{itemize}
\item {Grp. gram.:m. pl.}
\end{itemize}
Grupo de peixes, que tem por typo a âmia.
\section{Amial}
\begin{itemize}
\item {Grp. gram.:m.}
\end{itemize}
Lugar plantado de amieiros.
(Por \textunderscore amieiral\textunderscore , de \textunderscore amieiro\textunderscore )
\section{Amiantáceo}
\begin{itemize}
\item {Grp. gram.:adj.}
\end{itemize}
Semelhante ao amianto.
\section{Amiantiforme}
\begin{itemize}
\item {Grp. gram.:adj.}
\end{itemize}
\begin{itemize}
\item {Proveniência:(De \textunderscore amianto\textunderscore  + \textunderscore fórma\textunderscore )}
\end{itemize}
Diz-se de uma espécie de arseniato de cobre.
\section{Amiantino}
\begin{itemize}
\item {Grp. gram.:adj.}
\end{itemize}
Que tem o aspecto do amianto.
\section{Amianto}
\begin{itemize}
\item {Grp. gram.:m.}
\end{itemize}
\begin{itemize}
\item {Proveniência:(Gr. \textunderscore amiantos\textunderscore )}
\end{itemize}
Silicato de magnésia, cujas fibras finíssimas são difficilmente fusíveis e combustíveis.
\section{Amiantoide}
\begin{itemize}
\item {Grp. gram.:adj.}
\end{itemize}
\begin{itemize}
\item {Proveniência:(Do gr. \textunderscore amiantos\textunderscore  + \textunderscore eidos\textunderscore )}
\end{itemize}
O mesmo que \textunderscore amiantino\textunderscore .
\section{Amiba}
\begin{itemize}
\item {Grp. gram.:f.}
\end{itemize}
(V.ameba)
\section{Amibiano}
\begin{itemize}
\item {Grp. gram.:adj.}
\end{itemize}
\begin{itemize}
\item {Utilização:Med.}
\end{itemize}
Em que há amiba: \textunderscore dysenteria amibiana\textunderscore .
\section{Amibo}
\begin{itemize}
\item {Grp. gram.:m.}
\end{itemize}
O mesmo que \textunderscore amiba\textunderscore .
\section{Amiboide}
\begin{itemize}
\item {Grp. gram.:adj.}
\end{itemize}
\begin{itemize}
\item {Proveniência:(Do gr. \textunderscore ameiben\textunderscore  + \textunderscore eidos\textunderscore )}
\end{itemize}
Semelhante ao amibo.
\section{Amiboísmo}
\begin{itemize}
\item {Grp. gram.:m.}
\end{itemize}
Faculdade de executar movimentos amiboides.
O conjunto desses movimentos.
\section{Amiçade}
\begin{itemize}
\item {Grp. gram.:f.}
\end{itemize}
Fórma archaica de \textunderscore amizade\textunderscore .
\section{Amichelar}
\begin{itemize}
\item {Grp. gram.:v. t.}
\end{itemize}
Atar com os michelos.
\section{Amícia}
\begin{itemize}
\item {Grp. gram.:f.}
\end{itemize}
\begin{itemize}
\item {Proveniência:(De \textunderscore Amici\textunderscore , n. p.)}
\end{itemize}
Gênero de plantas leguminosas da América do Norte.
\section{Amicícia}
\begin{itemize}
\item {Grp. gram.:f.}
\end{itemize}
\begin{itemize}
\item {Utilização:Des.}
\end{itemize}
\begin{itemize}
\item {Proveniência:(Lat. \textunderscore amicitia\textunderscore )}
\end{itemize}
O mesmo que \textunderscore amizade\textunderscore .
\section{Amicidade}
\begin{itemize}
\item {Grp. gram.:f.}
\end{itemize}
\begin{itemize}
\item {Utilização:Ant.}
\end{itemize}
\begin{itemize}
\item {Proveniência:(Do lat. \textunderscore amicus\textunderscore )}
\end{itemize}
O mesmo que \textunderscore amizidade\textunderscore .
\section{Amicíssimo}
\begin{itemize}
\item {Grp. gram.:adj.}
\end{itemize}
\begin{itemize}
\item {Proveniência:(Do lat. \textunderscore amicus\textunderscore )}
\end{itemize}
Muito amigo.
\section{Amicto}
\begin{itemize}
\item {Grp. gram.:m.}
\end{itemize}
\begin{itemize}
\item {Proveniência:(Lat. \textunderscore amictus\textunderscore )}
\end{itemize}
Pano branco, que o sacerdote põe aos ombros, antes de vestir a alva.
\section{Amictório}
\begin{itemize}
\item {Grp. gram.:m.}
\end{itemize}
\begin{itemize}
\item {Proveniência:(Lat. \textunderscore amictorium\textunderscore )}
\end{itemize}
Espécie de chale ou lenço, com que as donzellas romanas cobriam o peito.
\section{Amicuanes}
\begin{itemize}
\item {Grp. gram.:f. pl.}
\end{itemize}
\begin{itemize}
\item {Utilização:Bras}
\end{itemize}
O mesmo que [[amazonas|amazona]].
\section{Amículo}
\begin{itemize}
\item {Grp. gram.:m.}
\end{itemize}
\begin{itemize}
\item {Proveniência:(Lat. \textunderscore amiculum\textunderscore )}
\end{itemize}
Pequeno vestido.
Véu.
\section{Amida}
\begin{itemize}
\item {Grp. gram.:f.  ou  m.}
\end{itemize}
\begin{itemize}
\item {Utilização:Chím.}
\end{itemize}
\begin{itemize}
\item {Proveniência:(De \textunderscore am\textunderscore , abrev. de \textunderscore ammoníaco\textunderscore )}
\end{itemize}
Radical hypothético, representativo de um sal de ammoníaco, menos um átomo de água.
\section{Amidálico}
\begin{itemize}
\item {Grp. gram.:adj.}
\end{itemize}
Que tem amido.
\section{Amidão}
\begin{itemize}
\item {Grp. gram.:m.}
\end{itemize}
O mesmo que \textunderscore amido\textunderscore .
\section{Amidina}
\begin{itemize}
\item {Grp. gram.:f.}
\end{itemize}
Princípio immediato do amido.
\section{Amido}
\begin{itemize}
\item {Grp. gram.:m.}
\end{itemize}
\begin{itemize}
\item {Grp. gram.:Pl.}
\end{itemize}
\begin{itemize}
\item {Proveniência:(Do lat. \textunderscore amylum\textunderscore )}
\end{itemize}
Fécula em pó, extrahida dos vegetaes.
Corpos derivados do ammoníaco, pela substituição do hydrogênio por um radical ácido.
\section{Âmido}
\begin{itemize}
\item {Grp. gram.:m.}
\end{itemize}
\begin{itemize}
\item {Grp. gram.:Pl.}
\end{itemize}
\begin{itemize}
\item {Proveniência:(Do lat. \textunderscore amylum\textunderscore )}
\end{itemize}
Fécula em pó, extrahida dos vegetaes.
Corpos derivados do ammoníaco, pela substituição do hydrogênio por um radical ácido.
\section{Amidoado}
\begin{itemize}
\item {Grp. gram.:adj.}
\end{itemize}
Feito de amido.
Que tem amido.
(Cp. \textunderscore amidão\textunderscore )
\section{Amidogênio}
\begin{itemize}
\item {Grp. gram.:m.}
\end{itemize}
O mesmo que \textunderscore amida\textunderscore .
\section{Amidol}
\begin{itemize}
\item {Grp. gram.:m.}
\end{itemize}
Chlorhydrato, que se emprega em photographia.
\section{Amidólico}
\begin{itemize}
\item {Grp. gram.:adj.}
\end{itemize}
Em cuja preparação entra o amido ou outra fécula.
\section{Amídona}
\begin{itemize}
\item {Grp. gram.:f.}
\end{itemize}
O mesmo que \textunderscore amidina\textunderscore .
\section{Amidonar}
\begin{itemize}
\item {Grp. gram.:v. t.}
\end{itemize}
\begin{itemize}
\item {Proveniência:(De \textunderscore amidão\textunderscore )}
\end{itemize}
Preparar com amido.
\section{Amidonita}
\begin{itemize}
\item {Grp. gram.:f.}
\end{itemize}
O mesmo que \textunderscore amidina\textunderscore .
\section{Amidureto}
\begin{itemize}
\item {Grp. gram.:m.}
\end{itemize}
Combinação do amidogênio com um metal.
\section{Amieira}
\begin{itemize}
\item {Grp. gram.:f.}
\end{itemize}
O mesmo que \textunderscore amieiro\textunderscore .
Pequena cesta de corras de amieiro ou castanheiro, com arco por cima.
\section{Amieiral}
\begin{itemize}
\item {Grp. gram.:m.}
\end{itemize}
O mesmo que \textunderscore amial\textunderscore .
\section{Amieiro}
\begin{itemize}
\item {Grp. gram.:m.}
\end{itemize}
Planta betulácea, espécie de salgueiro, (\textunderscore betula alnus\textunderscore ).
\section{Amiga}
\begin{itemize}
\item {Grp. gram.:f.}
\end{itemize}
\begin{itemize}
\item {Proveniência:(Lat. \textunderscore amica\textunderscore )}
\end{itemize}
Mulher, que estima ou ama outra pessôa.
Concubina.
\section{Amigação}
\begin{itemize}
\item {Grp. gram.:f.}
\end{itemize}
Estado ou acto de \textunderscore amigar-se\textunderscore .
\section{Amigaço}
\begin{itemize}
\item {Grp. gram.:m.}
\end{itemize}
\begin{itemize}
\item {Utilização:P. us.}
\end{itemize}
O mesmo que \textunderscore amigalhaço\textunderscore .
\section{Amigalhaço}
\begin{itemize}
\item {Grp. gram.:m.}
\end{itemize}
\begin{itemize}
\item {Utilização:Chul.}
\end{itemize}
Grande amigo.
\section{Amigalhão}
\begin{itemize}
\item {Grp. gram.:m.}
\end{itemize}
\begin{itemize}
\item {Utilização:Chul.}
\end{itemize}
O mesmo que \textunderscore amigalhaço\textunderscore .
\section{Amigalhote}
\begin{itemize}
\item {Grp. gram.:m.}
\end{itemize}
\begin{itemize}
\item {Utilização:Pop.}
\end{itemize}
Amigo que inspira pouca confiança.
\section{Amigamente}
\begin{itemize}
\item {Grp. gram.:adv.}
\end{itemize}
O mesmo que \textunderscore amigavelmente\textunderscore .
\section{Amigar-se}
\begin{itemize}
\item {Grp. gram.:v. p.}
\end{itemize}
Tornar-se amigo.
Amancebar-se.
\section{Amigável}
\begin{itemize}
\item {Grp. gram.:adj.}
\end{itemize}
\begin{itemize}
\item {Proveniência:(Lat. \textunderscore amicabilis\textunderscore )}
\end{itemize}
Dito ou feito com amizade: \textunderscore proposta amigável\textunderscore .
Próprio de amigo.
\section{Amigavelmente}
\begin{itemize}
\item {Grp. gram.:adj.}
\end{itemize}
De modo \textunderscore amigável\textunderscore .
\section{Amigo}
\begin{itemize}
\item {Grp. gram.:m.  e  adj.}
\end{itemize}
\begin{itemize}
\item {Utilização:Prov.}
\end{itemize}
\begin{itemize}
\item {Grp. gram.:Loc.}
\end{itemize}
\begin{itemize}
\item {Utilização:fam.}
\end{itemize}
\begin{itemize}
\item {Proveniência:(Lat. \textunderscore amicus\textunderscore )}
\end{itemize}
O que ama; o que tem amizade: \textunderscore os nossos amigos\textunderscore .
Aquelle que estima, aprecia: \textunderscore amigo de gulodices\textunderscore .
Benévolo.
Aliado.
Amante.
Amásio.
\textunderscore Amigo de Peniche\textunderscore , amigo que não é certo.
\section{Amigo-fechado}
\begin{itemize}
\item {Grp. gram.:m.}
\end{itemize}
O mesmo que \textunderscore chamuar\textunderscore .
\section{Amigote}
\begin{itemize}
\item {Grp. gram.:m.}
\end{itemize}
O mesmo que \textunderscore amigalhote\textunderscore .
\section{Amilá}
\begin{itemize}
\item {Grp. gram.:m.}
\end{itemize}
\begin{itemize}
\item {Utilização:Mús.}
\end{itemize}
\begin{itemize}
\item {Utilização:ant.}
\end{itemize}
O mesmo que \textunderscore lamiré\textunderscore .
\section{Amilhar}
\begin{itemize}
\item {Grp. gram.:v. t.}
\end{itemize}
\begin{itemize}
\item {Utilização:Bras}
\end{itemize}
Tratar com milho, dar rações de milho a: \textunderscore amilhar as gallinhas\textunderscore .
\section{Amilo}
\begin{itemize}
\item {Grp. gram.:m.}
\end{itemize}
\begin{itemize}
\item {Utilização:Ant.}
\end{itemize}
\begin{itemize}
\item {Proveniência:(Lat. \textunderscore amylum\textunderscore )}
\end{itemize}
O mesmo que \textunderscore amido\textunderscore .
\section{Amilombe}
\begin{itemize}
\item {Grp. gram.:f.}
\end{itemize}
Rapariga, ao serviço particular das mulheres de alguns potentados africanos.
\section{Amim}
\begin{itemize}
\item {Grp. gram.:m.}
\end{itemize}
Magistrado administrativo, judicial e fiscal, entre os Moiros.
(Do ár.)
\section{Amimador}
\begin{itemize}
\item {Grp. gram.:m.}
\end{itemize}
Aquelle que amima.
\section{Amimalhar}
\begin{itemize}
\item {Grp. gram.:v. t.}
\end{itemize}
\begin{itemize}
\item {Proveniência:(De \textunderscore mimalho\textunderscore )}
\end{itemize}
Tratar com demasiado mimo.
\section{Amimar}
\begin{itemize}
\item {Grp. gram.:v. t.}
\end{itemize}
Dar mimo a; tratar com mimo.
Acariciar.
\section{Amimia}
\begin{itemize}
\item {Grp. gram.:f.}
\end{itemize}
\begin{itemize}
\item {Utilização:Med.}
\end{itemize}
\begin{itemize}
\item {Proveniência:(Do gr. \textunderscore a\textunderscore  priv. + \textunderscore mimos\textunderscore )}
\end{itemize}
Ausência de mímica ou perda da faculdade de utilizar os gestos como sinaes.
\section{Aminas}
\begin{itemize}
\item {Grp. gram.:f. pl.}
\end{itemize}
\begin{itemize}
\item {Utilização:Chím.}
\end{itemize}
Corpos derivados do ammoníaco pela substituição de um ou mais dos hydrogênios por um ou mais radicaes.
\section{Amíneas}
\begin{itemize}
\item {Grp. gram.:f. pl.}
\end{itemize}
Tribo de plantas, que têm por typo o \textunderscore âmio\textunderscore .
\section{Amineirar}
\begin{itemize}
\item {Grp. gram.:v. t.}
\end{itemize}
\begin{itemize}
\item {Utilização:Bras}
\end{itemize}
Dar feição ou hábitos de mineiro ou do natural de Minas-Geraes.
\section{Amingoeira}
\begin{itemize}
\item {Grp. gram.:f.}
\end{itemize}
\begin{itemize}
\item {Utilização:Prov.}
\end{itemize}
\begin{itemize}
\item {Utilização:trasm.}
\end{itemize}
O mesmo que \textunderscore mangual\textunderscore .
\section{Aminguar}
\textunderscore v. i.\textunderscore  (e der)
O mesmo que \textunderscore minguar\textunderscore , etc.
\section{Amínia}
\begin{itemize}
\item {Grp. gram.:f.}
\end{itemize}
Uva vermelha do Brasil.
\section{Aminículo}
\begin{itemize}
\item {Grp. gram.:m.}
\end{itemize}
\begin{itemize}
\item {Utilização:Ant.}
\end{itemize}
O mesmo que \textunderscore adminículo\textunderscore .
\section{Aminos}
\begin{itemize}
\item {Grp. gram.:m. pl.}
\end{itemize}
O mesmo que \textunderscore aminas\textunderscore .
\section{Aminta}
\begin{itemize}
\item {Grp. gram.:f.}
\end{itemize}
Gênero de insectos dípteros.
\section{Âmio}
\begin{itemize}
\item {Grp. gram.:m.}
\end{itemize}
\begin{itemize}
\item {Proveniência:(Do gr. \textunderscore ammi\textunderscore )}
\end{itemize}
Planta umbellifera, semelhante á cenoira, e também conhecida por \textunderscore bisnaga\textunderscore .
\section{Amioca}
\begin{itemize}
\item {Grp. gram.:m.}
\end{itemize}
Pequeno peixe do Brasil, nas costas de Sergipe.
\section{Amir}
\begin{itemize}
\item {Grp. gram.:m.}
\end{itemize}
O mesmo que \textunderscore emir\textunderscore .
\section{Amisatina}
\begin{itemize}
\item {Grp. gram.:f.}
\end{itemize}
Producto chímico, que se obtém pela acção do ammoníaco sôbre a isatina.
\section{Amiseração}
\begin{itemize}
\item {Grp. gram.:f.}
\end{itemize}
Acto de se amiserar. Cf. Cortesão, \textunderscore Subs.\textunderscore 
\section{Amiserar}
\begin{itemize}
\item {Grp. gram.:v. t.}
\end{itemize}
\begin{itemize}
\item {Proveniência:(De \textunderscore mísero\textunderscore )}
\end{itemize}
O mesmo que \textunderscore commiserar\textunderscore .
\section{Ami-só}
\begin{itemize}
\item {Grp. gram.:m.}
\end{itemize}
Planta venenosa, de uma só fôlha, na ilha de San-Thomé.
\section{Amissão}
\begin{itemize}
\item {Grp. gram.:f.}
\end{itemize}
\begin{itemize}
\item {Proveniência:(Lat. \textunderscore amissio\textunderscore )}
\end{itemize}
O mesmo que \textunderscore perda\textunderscore .
\section{Amissivel}
\begin{itemize}
\item {Grp. gram.:adj.}
\end{itemize}
\begin{itemize}
\item {Proveniência:(Lat. \textunderscore amissibilis\textunderscore )}
\end{itemize}
O que se póde perder.
\section{Amistade}
\begin{itemize}
\item {Grp. gram.:f.}
\end{itemize}
\begin{itemize}
\item {Utilização:Prov.}
\end{itemize}
\begin{itemize}
\item {Utilização:minh.}
\end{itemize}
O mesmo que \textunderscore amizade\textunderscore .
(Cast. \textunderscore amistad\textunderscore )
\section{Amistar}
\begin{itemize}
\item {Grp. gram.:v. t.}
\end{itemize}
Tornar amigo.
Reconciliar:«\textunderscore quando te amistaste com elle\textunderscore ». Camillo, \textunderscore Regicida\textunderscore , 97.
(Cast. \textunderscore amistar\textunderscore )
\section{Amistoso}
\begin{itemize}
\item {Grp. gram.:adj.}
\end{itemize}
\begin{itemize}
\item {Proveniência:(T. cast.)}
\end{itemize}
Amigável; próprio de amigo.
\section{Amisular}
\begin{itemize}
\item {Grp. gram.:v. t.}
\end{itemize}
Pôr mísulas em.
Collocar sôbre mísulas.
\section{Amita}
\begin{itemize}
\item {Grp. gram.:f.}
\end{itemize}
\begin{itemize}
\item {Proveniência:(Do gr. \textunderscore ammos\textunderscore , areia)}
\end{itemize}
Designação genérica dos mineraes formados de grãos redondos.
\section{Amito}
\begin{itemize}
\item {Grp. gram.:m.}
\end{itemize}
\begin{itemize}
\item {Utilização:Açor}
\end{itemize}
O mesmo que \textunderscore gáspea\textunderscore .
\section{Amitre}
\begin{itemize}
\item {Grp. gram.:m.}
\end{itemize}
Gênero de coleópteros.
\section{Amiudadamente}
\begin{itemize}
\item {fónica:mi-u}
\end{itemize}
\begin{itemize}
\item {Grp. gram.:adv.}
\end{itemize}
Amiúde; com frequência.
\section{Amiudado}
\begin{itemize}
\item {fónica:mi-u}
\end{itemize}
\begin{itemize}
\item {Grp. gram.:adj.}
\end{itemize}
Frequente.
\section{Amiudança}
\begin{itemize}
\item {fónica:mi-u}
\end{itemize}
\begin{itemize}
\item {Grp. gram.:f.}
\end{itemize}
\begin{itemize}
\item {Utilização:Ant.}
\end{itemize}
Frequência.
Acto de \textunderscore amiudar\textunderscore ^2.
\section{Amiudar}
\begin{itemize}
\item {fónica:mi-u}
\end{itemize}
\begin{itemize}
\item {Grp. gram.:v. t.}
\end{itemize}
Tornar miúdo.
\section{Amiudar}
\begin{itemize}
\item {fónica:mi-u}
\end{itemize}
\begin{itemize}
\item {Grp. gram.:v. t.}
\end{itemize}
\begin{itemize}
\item {Proveniência:(De \textunderscore amiúde\textunderscore )}
\end{itemize}
Tornar frequente.
Repetir.
\section{Amiúde}
\begin{itemize}
\item {Grp. gram.:adv.}
\end{itemize}
\begin{itemize}
\item {Proveniência:(Do lat. \textunderscore minute\textunderscore )}
\end{itemize}
Frequentemente; repetidas vezes.
\section{A-miúdo}
\begin{itemize}
\item {Grp. gram.:adv.}
\end{itemize}
(Corr. de \textunderscore amiúde\textunderscore ). Cf. Camillo, \textunderscore Retr. de Ricard.\textunderscore , 11.
\section{Amizade}
\begin{itemize}
\item {Grp. gram.:f.}
\end{itemize}
Sentimento de quem é amigo.
Amor.
Dedicação.
Benevolência.
Nome de uma armação de pesca, na costa da Galé.
(Contr. de \textunderscore amizidade\textunderscore , do lat. \textunderscore amicitia\textunderscore )
\section{Amizdade}
\begin{itemize}
\item {Grp. gram.:f.}
\end{itemize}
(Fórma archaica de \textunderscore amizade\textunderscore )
\section{Amizidade}
\begin{itemize}
\item {Grp. gram.:f.}
\end{itemize}
(Fórma pop. de \textunderscore amizade\textunderscore )
(Cp. \textunderscore amicidade\textunderscore )
\section{Ammânia}
\begin{itemize}
\item {Grp. gram.:f.}
\end{itemize}
\begin{itemize}
\item {Proveniência:(De \textunderscore Ammann\textunderscore , n. p.)}
\end{itemize}
Gênero de plantas equatoriaes.
\section{Ammelida}
\begin{itemize}
\item {Grp. gram.:f.}
\end{itemize}
Substância branca amorpha, obtida pela acção dos álcalis e dos ácidos sôbre a ammelina.
\section{Ammelina}
\begin{itemize}
\item {Grp. gram.:f.}
\end{itemize}
Base chímica, obtida pela acção dos ácidos sobre o melam.
\section{Âmmi}
\begin{itemize}
\item {Grp. gram.:m.}
\end{itemize}
O mesmo que \textunderscore âmmio\textunderscore .
\section{Ammíneas}
\begin{itemize}
\item {Grp. gram.:f. pl.}
\end{itemize}
Tribo de plantas, que têm por typo o \textunderscore âmmio\textunderscore .
\section{Âmmio}
\begin{itemize}
\item {Grp. gram.:m.}
\end{itemize}
\begin{itemize}
\item {Proveniência:(Do gr. \textunderscore ammi\textunderscore )}
\end{itemize}
Planta umbellifera, semelhante á cenoira, e também conhecida por \textunderscore bisnaga\textunderscore .
\section{Ammita}
\begin{itemize}
\item {Grp. gram.:f.}
\end{itemize}
\begin{itemize}
\item {Proveniência:(Do gr. \textunderscore ammos\textunderscore , areia)}
\end{itemize}
Designação genérica dos mineraes formados de grãos redondos.
\section{Ammóbata}
\begin{itemize}
\item {Grp. gram.:m.}
\end{itemize}
\begin{itemize}
\item {Proveniência:(Do gr. \textunderscore ammos\textunderscore  + \textunderscore bates\textunderscore )}
\end{itemize}
Serpente da Guiné.
\section{Ammóbio}
\begin{itemize}
\item {Grp. gram.:m.}
\end{itemize}
\begin{itemize}
\item {Proveniência:(Do gr. \textunderscore ammos\textunderscore , areia, e \textunderscore bios\textunderscore , vida)}
\end{itemize}
Gênero de plantas compostas.
\section{Ammodendrão}
\begin{itemize}
\item {Grp. gram.:m.}
\end{itemize}
\begin{itemize}
\item {Proveniência:(Do gr. \textunderscore ammos\textunderscore  + \textunderscore dendron\textunderscore )}
\end{itemize}
Planta leguminosa da Sibéria meridional.
\section{Ammodendro}
\begin{itemize}
\item {Grp. gram.:m.}
\end{itemize}
O mesmo que \textunderscore ammodendrão\textunderscore .
\section{Ammódromo}
\begin{itemize}
\item {Grp. gram.:m.}
\end{itemize}
\begin{itemize}
\item {Proveniência:(Do gr. \textunderscore ammos\textunderscore  + \textunderscore dromos\textunderscore )}
\end{itemize}
Pássaro conirostro, que vive em ilhotas, nas costas do Atlântico.
\section{Ammódyte}
\begin{itemize}
\item {Grp. gram.:m.}
\end{itemize}
\begin{itemize}
\item {Proveniência:(Gr. \textunderscore ammodutes\textunderscore )}
\end{itemize}
Peixe, semelhante
á enguia. Reptil, da fam. das víboras.
\section{Ammódyto}
\begin{itemize}
\item {Grp. gram.:adj.}
\end{itemize}
Que vive na areia ou que se enterra na areia.
(Cp. \textunderscore ammódyte\textunderscore )
\section{Ammólico}
\begin{itemize}
\item {Grp. gram.:adj.}
\end{itemize}
Diz-se dos saes que têm por base a ammolina.
\section{Ammolina}
\begin{itemize}
\item {Grp. gram.:f.}
\end{itemize}
Base salificável, extrahida do óleo animal de Dippel.
\section{Ammónia}
\begin{itemize}
\item {Grp. gram.:f.}
\end{itemize}
O mesmo que \textunderscore ammónio\textunderscore ^1.
\section{Ammoniacado}
\begin{itemize}
\item {Grp. gram.:adj.}
\end{itemize}
Que tem sal \textunderscore ammoníaco\textunderscore .
\section{Ammoniacal}
\begin{itemize}
\item {Grp. gram.:adj.}
\end{itemize}
Que tem ammoníaco ou propriedades do ammoníaco.
\section{Ammoniáceo}
\begin{itemize}
\item {Grp. gram.:adj.}
\end{itemize}
O mesmo que \textunderscore ammoniacado\textunderscore .
\section{Ammoníaco}
\begin{itemize}
\item {Grp. gram.:m.}
\end{itemize}
\begin{itemize}
\item {Grp. gram.:Adj.}
\end{itemize}
\begin{itemize}
\item {Proveniência:(Gr. \textunderscore ammoniakos\textunderscore )}
\end{itemize}
Gás, que se encontra no estado de combinação com os ácidos chlorhýdrico e phosphórico na urina, e com os
ácidos acético e carbónico nas matérias animaes em putrefacção.
O mesmo que \textunderscore ammoniacal\textunderscore .
\section{Ammonialdehydo}
\begin{itemize}
\item {Grp. gram.:m.}
\end{itemize}
Combinação de aldehydo e ammoníaco.
\section{Ammoniato}
\begin{itemize}
\item {Grp. gram.:m.}
\end{itemize}
\begin{itemize}
\item {Proveniência:(De \textunderscore ammónia\textunderscore )}
\end{itemize}
Corpo, resultante da combinação do ammoníaco com um óxydo metállico.
\section{Ammónico...}
\begin{itemize}
\item {Proveniência:(De \textunderscore ammoníaco\textunderscore )}
\end{itemize}
Elemento, que entra na composição de certos adjectivos, para designar uma combinação em que entra ammoníaco: \textunderscore amónico-hýdrico\textunderscore , \textunderscore ammónico-potássico\textunderscore , etc.
\section{Ammoniemia}
\begin{itemize}
\item {Grp. gram.:f.}
\end{itemize}
\begin{itemize}
\item {Proveniência:(Do gr. \textunderscore ammoniakos\textunderscore  + \textunderscore haima\textunderscore )}
\end{itemize}
Doença, determinada pela presença de ammoníaco ou dos seus saes no sangue.
\section{Ammonieto}
\begin{itemize}
\item {Grp. gram.:m.}
\end{itemize}
O mesmo ou melhor que \textunderscore ammoniureto\textunderscore .
\section{Ammonímetro}
\begin{itemize}
\item {Grp. gram.:m.}
\end{itemize}
\begin{itemize}
\item {Proveniência:(Do gr. \textunderscore ammoniakos\textunderscore  + \textunderscore metron\textunderscore )}
\end{itemize}
Apparelho, para a dosagem do ammoníaco.
\section{Ammónio}
\begin{itemize}
\item {Grp. gram.:m.}
\end{itemize}
\begin{itemize}
\item {Proveniência:(De \textunderscore ammoníaco\textunderscore )}
\end{itemize}
Combinação hypothética de um equivalente de azoto e quatro de hydrogênio.
\section{Ammónio}
\begin{itemize}
\item {Grp. gram.:adj.}
\end{itemize}
Relativo a Jupíter Âmmon.
\section{Ammonite}
\begin{itemize}
\item {Grp. gram.:f.}
\end{itemize}
\begin{itemize}
\item {Proveniência:(De \textunderscore Ammon\textunderscore , n. p.)}
\end{itemize}
Gênero de molluscos cephalópodes.
\section{Ammonitídeos}
\begin{itemize}
\item {Grp. gram.:m. pl.}
\end{itemize}
Família de molluscos, que têm por typo a \textunderscore ammonite\textunderscore .
\section{Ammoniureto}
\begin{itemize}
\item {Grp. gram.:m.}
\end{itemize}
O mesmo que \textunderscore ammoniato\textunderscore .
\section{Ammonoíde}
\begin{itemize}
\item {Grp. gram.:adj.}
\end{itemize}
\begin{itemize}
\item {Proveniência:(De \textunderscore Ammon\textunderscore , n. p. + gr. \textunderscore eidos\textunderscore )}
\end{itemize}
Diz-se das conchas, que se assemelham á ammonite.
\section{Ammóphila}
\begin{itemize}
\item {Grp. gram.:f.}
\end{itemize}
\begin{itemize}
\item {Proveniência:(Do gr. \textunderscore ammos\textunderscore  + \textunderscore philos\textunderscore )}
\end{itemize}
Planta leguminosa, muito vulgar em médãos, á beira-mar.
\section{Amnesia}
\begin{itemize}
\item {Grp. gram.:f.}
\end{itemize}
\begin{itemize}
\item {Proveniência:(Do gr. \textunderscore amnesis\textunderscore , de \textunderscore a\textunderscore  priv. + \textunderscore mnaomai\textunderscore )}
\end{itemize}
Perda de memória.
\section{Amnesiar}
\begin{itemize}
\item {Grp. gram.:v. t.}
\end{itemize}
Causar amnesia a.
\section{Amnéstico}
\begin{itemize}
\item {Grp. gram.:adj.}
\end{itemize}
\begin{itemize}
\item {Proveniência:(De \textunderscore amnesia\textunderscore )}
\end{itemize}
Que faz perder a memória.
\section{Amniático}
\begin{itemize}
\item {Grp. gram.:adj.}
\end{itemize}
Relativo ao âmnio: \textunderscore o feto, mergulhando no liquido amniático...\textunderscore 
\section{Amnícola}
\begin{itemize}
\item {Grp. gram.:adj.}
\end{itemize}
\begin{itemize}
\item {Proveniência:(Do lat. \textunderscore amnis\textunderscore  + \textunderscore colere\textunderscore )}
\end{itemize}
Que vive á beira de águas correntes.
\section{Âmnio}
\begin{itemize}
\item {Grp. gram.:m.}
\end{itemize}
O mesmo ou melhor que \textunderscore amnios\textunderscore .
\section{Amniomancia}
\begin{itemize}
\item {Grp. gram.:f.}
\end{itemize}
Systema de adivinhação, fundado na inspecção do saco membranoso, que ás vezes envolve a cabeça do recém-nascido.
(Palavra hybr. do lat. \textunderscore amnium\textunderscore  + gr. \textunderscore manteia\textunderscore .)
\section{Amniomanciano}
\begin{itemize}
\item {Grp. gram.:m.}
\end{itemize}
O mesmo que \textunderscore amniomante\textunderscore .
\section{Amniomante}
\begin{itemize}
\item {Grp. gram.:m.}
\end{itemize}
Aquelle que pratica a amniomancia.
\section{Âmnios}
\begin{itemize}
\item {Grp. gram.:m.}
\end{itemize}
\begin{itemize}
\item {Proveniência:(Gr. \textunderscore amnios\textunderscore )}
\end{itemize}
A mais interna membrana que envolve o feto.
\section{Amniótico}
\begin{itemize}
\item {Grp. gram.:adj.}
\end{itemize}
O mesmo que \textunderscore amniático\textunderscore .
\section{Amnistia}
\begin{itemize}
\item {Grp. gram.:f.}
\end{itemize}
\begin{itemize}
\item {Proveniência:(Do gr. \textunderscore amnestia\textunderscore )}
\end{itemize}
Isenção collectiva de castigo, concedida pelo Estado, para certa ordem de crimes.
\section{Amnistiar}
\begin{itemize}
\item {Grp. gram.:v. t.}
\end{itemize}
Conceder amnistia a.
Perdoar.
\section{Amo}
\begin{itemize}
\item {Grp. gram.:m.}
\end{itemize}
\begin{itemize}
\item {Utilização:Ant.}
\end{itemize}
Dono da casa, em relação aos criados.
Senhor.
Marido da ama (que cria).
Pedagogo.
Hospedeiro.
(Cp. \textunderscore ama\textunderscore ^1)
\section{Amóbata}
\begin{itemize}
\item {Grp. gram.:m.}
\end{itemize}
\begin{itemize}
\item {Proveniência:(Do gr. \textunderscore ammos\textunderscore  + \textunderscore bates\textunderscore )}
\end{itemize}
Serpente da Guiné.
\section{Amóbio}
\begin{itemize}
\item {Grp. gram.:m.}
\end{itemize}
\begin{itemize}
\item {Proveniência:(Do gr. \textunderscore ammos\textunderscore , areia, e \textunderscore bios\textunderscore , vida)}
\end{itemize}
Gênero de plantas compostas.
\section{Amocambar}
\begin{itemize}
\item {Grp. gram.:v. t.}
\end{itemize}
\begin{itemize}
\item {Utilização:Bras}
\end{itemize}
\begin{itemize}
\item {Utilização:Bras. de Minas}
\end{itemize}
Reunir em mocambo ou em mocambos.
Esconder, occultar.
\section{Amochado}
\begin{itemize}
\item {Grp. gram.:adj.}
\end{itemize}
\begin{itemize}
\item {Utilização:Des.}
\end{itemize}
\begin{itemize}
\item {Proveniência:(De \textunderscore mocho\textunderscore )}
\end{itemize}
Adoentado, engerido.
\section{Amochoir-se}
\begin{itemize}
\item {Grp. gram.:v. t.}
\end{itemize}
\begin{itemize}
\item {Utilização:Prov.}
\end{itemize}
\begin{itemize}
\item {Utilização:beir.}
\end{itemize}
\begin{itemize}
\item {Proveniência:(De \textunderscore mocho\textunderscore . Cp. o bras. \textunderscore encorujar-se\textunderscore )}
\end{itemize}
Retrahir-se, tornar-se misanthropo.
Embiocar-se.
Encolher-se, retrahindo-se.
\section{Amodelar}
\begin{itemize}
\item {Grp. gram.:v. t.}
\end{itemize}
O mesmo que \textunderscore modelar\textunderscore ^1.
\section{Amodendrão}
\begin{itemize}
\item {Grp. gram.:m.}
\end{itemize}
\begin{itemize}
\item {Proveniência:(Do gr. \textunderscore ammos\textunderscore  + \textunderscore dendron\textunderscore )}
\end{itemize}
Planta leguminosa da Sibéria meridional.
\section{Amodendro}
\begin{itemize}
\item {Grp. gram.:m.}
\end{itemize}
O mesmo que \textunderscore amodendrão\textunderscore .
\section{Amodernar}
\begin{itemize}
\item {Grp. gram.:v. t.}
\end{itemize}
Tornar moderno; dar feição moderna a. Cf. Filinto, XVII, 139.
\section{Amódite}
\begin{itemize}
\item {Grp. gram.:m.}
\end{itemize}
\begin{itemize}
\item {Proveniência:(Gr. \textunderscore ammodutes\textunderscore )}
\end{itemize}
Peixe, semelhante á enguia.
Reptil, da fam. das víboras.
\section{Amódito}
\begin{itemize}
\item {Grp. gram.:adj.}
\end{itemize}
Que vive na areia ou que se enterra na areia.
(Cp. \textunderscore ammódyte\textunderscore )
\section{Amodorrar}
\begin{itemize}
\item {Grp. gram.:v. t.}
\end{itemize}
Causar modorra a; fazer cair em modorra.
Tornar somnolento.
\section{Amódromo}
\begin{itemize}
\item {Grp. gram.:m.}
\end{itemize}
\begin{itemize}
\item {Proveniência:(Do gr. \textunderscore ammos\textunderscore  + \textunderscore dromos\textunderscore )}
\end{itemize}
Pássaro conirostro, que vive em ilhotas, nas costas do Atlântico.
\section{Amoedação}
\begin{itemize}
\item {fónica:mo-e}
\end{itemize}
\begin{itemize}
\item {Grp. gram.:f.}
\end{itemize}
Acto de \textunderscore amoedar\textunderscore . Cf. Castilho, \textunderscore Fastos\textunderscore , I, 352.
\section{Amoedar}
\begin{itemize}
\item {fónica:mo-e}
\end{itemize}
\begin{itemize}
\item {Grp. gram.:v. t.}
\end{itemize}
\begin{itemize}
\item {Proveniência:(De \textunderscore moéda\textunderscore )}
\end{itemize}
Reduzir a moéda.
Cunhar.
\section{Amoedável}
\begin{itemize}
\item {fónica:mo-e}
\end{itemize}
\begin{itemize}
\item {Grp. gram.:adj.}
\end{itemize}
Que se póde amoedar.
\section{Amoestar}
\begin{itemize}
\item {fónica:mo-es}
\end{itemize}
\textunderscore v. t.\textunderscore  (e der.)
O mesmo que \textunderscore admoestar\textunderscore , etc.
\section{Amófila}
\begin{itemize}
\item {Grp. gram.:f.}
\end{itemize}
\begin{itemize}
\item {Proveniência:(Do gr. \textunderscore ammos\textunderscore  + \textunderscore philos\textunderscore )}
\end{itemize}
Planta leguminosa, muito vulgar em médãos, á beira-mar.
\section{Amofinação}
\begin{itemize}
\item {Grp. gram.:f.}
\end{itemize}
Acto de \textunderscore amofinar\textunderscore .
\section{Amofinador}
\begin{itemize}
\item {Grp. gram.:m.}
\end{itemize}
Aquelle que amofina.
\section{Amofinar}
\begin{itemize}
\item {Grp. gram.:v. t.}
\end{itemize}
Tornar mofino.
Affligir; apoquentar.
\section{Amofinativo}
\begin{itemize}
\item {Grp. gram.:adj.}
\end{itemize}
\begin{itemize}
\item {Utilização:P. us.}
\end{itemize}
Que causa amofinação.
\section{Amoinar}
\begin{itemize}
\item {Grp. gram.:v. i.}
\end{itemize}
\begin{itemize}
\item {Utilização:Gír.}
\end{itemize}
Pedir esmola.
\section{Amoirado}
\begin{itemize}
\item {Grp. gram.:adj.}
\end{itemize}
\begin{itemize}
\item {Utilização:Agr.}
\end{itemize}
Levantado em moirão. Cf. \textunderscore Techn. Rur\textunderscore ., 565.
\section{Amoiriscar}
\begin{itemize}
\item {Grp. gram.:v. t.}
\end{itemize}
\begin{itemize}
\item {Grp. gram.:V. i.}
\end{itemize}
Dar aspecto ou feitio moirisco a.
Ter feição de moiro. Cf. \textunderscore Fenix Renasc.\textunderscore , V, 46.
\section{Amoiroar}
\begin{itemize}
\item {Grp. gram.:v. i.}
\end{itemize}
\begin{itemize}
\item {Utilização:Prov.}
\end{itemize}
\begin{itemize}
\item {Utilização:Prov.}
\end{itemize}
\begin{itemize}
\item {Utilização:beir.}
\end{itemize}
Encostar-se, postar-se: \textunderscore Costumava amoiroar á porta da namorada\textunderscore .
Quedar-se (o gado) nas horas de calor, arquejante e muito junto.
(Provavelmente, de \textunderscore moirão\textunderscore )
\section{Amoitar-se}
\begin{itemize}
\item {Grp. gram.:v. p.}
\end{itemize}
\begin{itemize}
\item {Utilização:Bras}
\end{itemize}
\begin{itemize}
\item {Proveniência:(De moita)}
\end{itemize}
Esconder-se.
\section{Amojada}
\begin{itemize}
\item {Grp. gram.:adj. f.}
\end{itemize}
\begin{itemize}
\item {Utilização:Bras. de Minas}
\end{itemize}
\begin{itemize}
\item {Proveniência:(De \textunderscore amojar\textunderscore )}
\end{itemize}
Muito prenhe.
\section{Amojar}
\begin{itemize}
\item {Grp. gram.:v. t.}
\end{itemize}
\begin{itemize}
\item {Utilização:Ant.}
\end{itemize}
Mungir.
Encher de leite.
\section{Amojo}
\begin{itemize}
\item {Grp. gram.:m.}
\end{itemize}
\begin{itemize}
\item {Utilização:Bot.}
\end{itemize}
\begin{itemize}
\item {Proveniência:(De \textunderscore amojar\textunderscore )}
\end{itemize}
Apojadura; entumecimento, produzido pelo leite nos peitos das mulheres e nas tetas dos animaes.
Estado lactescente dos grãos de cereaes.
\section{Amolação}
\begin{itemize}
\item {Grp. gram.:f.}
\end{itemize}
\begin{itemize}
\item {Utilização:Bras. de Minas}
\end{itemize}
\begin{itemize}
\item {Proveniência:(De \textunderscore amolar\textunderscore ^1)}
\end{itemize}
O mesmo que \textunderscore amoladura\textunderscore .
Incômmodo, maçada.
\section{Amolada}
\begin{itemize}
\item {Grp. gram.:f.}
\end{itemize}
O mesmo que \textunderscore amoladura\textunderscore .
\section{Amoladela}
\begin{itemize}
\item {Grp. gram.:f.}
\end{itemize}
O mesmo que \textunderscore amoladura\textunderscore .
\section{Amolador}
\begin{itemize}
\item {Grp. gram.:m.}
\end{itemize}
Aquelle que amola.
Aquelle que, por offício, amola em rebôlo navalhas, tesoiras, etc.
\section{Amoladura}
\begin{itemize}
\item {Grp. gram.:f.}
\end{itemize}
Acto de \textunderscore amolar\textunderscore ^1.
Resíduo do rebôlo, que fica na água com que se abranda o mesmo rebôlo.
\section{Amolar}
\begin{itemize}
\item {Grp. gram.:v. t.}
\end{itemize}
\begin{itemize}
\item {Utilização:Fam.}
\end{itemize}
\begin{itemize}
\item {Utilização:Bras}
\end{itemize}
\begin{itemize}
\item {Grp. gram.:V. i.}
\end{itemize}
\begin{itemize}
\item {Utilização:Fam.}
\end{itemize}
\begin{itemize}
\item {Proveniência:(Do lat. \textunderscore mola\textunderscore , mó)}
\end{itemize}
Afiar; tornar cortante.
Amolgar.
Meter em difficuldades.
Enganar.
Enfadar, falando.
Molestar, causticar.
Ficar meditando no que se ouviu.
\section{Amolar}
\begin{itemize}
\item {Grp. gram.:v. t.}
\end{itemize}
\begin{itemize}
\item {Utilização:T. de Baião}
\end{itemize}
Fazer recuar.
\section{Amolatar}
\begin{itemize}
\item {Grp. gram.:v. t.}
\end{itemize}
\begin{itemize}
\item {Utilização:Prov.}
\end{itemize}
\begin{itemize}
\item {Utilização:minh.}
\end{itemize}
\begin{itemize}
\item {Proveniência:(De \textunderscore molle\textunderscore )}
\end{itemize}
O mesmo que \textunderscore amolgar\textunderscore .
\section{Amoldar}
\begin{itemize}
\item {Grp. gram.:v. t.}
\end{itemize}
Ajustar ao molde; moldar.
Modelar.
Acostumar.
\section{Amoldável}
\begin{itemize}
\item {Grp. gram.:adj.}
\end{itemize}
Que se póde amoldar.
\section{Amolecar}
\begin{itemize}
\item {Grp. gram.:v. t.}
\end{itemize}
\begin{itemize}
\item {Utilização:Bras}
\end{itemize}
\begin{itemize}
\item {Proveniência:(De \textunderscore moleque\textunderscore )}
\end{itemize}
Tratar indecorosamente.
Rebaixar.
Ridiculizar.
\section{Amolecedor}
\begin{itemize}
\item {Grp. gram.:m.}
\end{itemize}
Aquelle que amolece.
\section{Amolecer}
\begin{itemize}
\item {Grp. gram.:v. t.}
\end{itemize}
\begin{itemize}
\item {Grp. gram.:V. i.}
\end{itemize}
\begin{itemize}
\item {Proveniência:(Lat. \textunderscore mollescere\textunderscore )}
\end{itemize}
Tornar mole.
Enervar.
Abrandar.
Commover.
Tornar-se mole.
\section{Amolecimento}
\begin{itemize}
\item {Grp. gram.:m.}
\end{itemize}
Acção de amolecer.
Brandura.
Enfraquecimento: \textunderscore amolecimento cerebral\textunderscore .
\section{Amolegar}
\begin{itemize}
\item {Grp. gram.:v. t.}
\end{itemize}
Tornar mole.
Amachucar, amolgar. Cp. Filinto. X. 28.
\section{Amolentar}
\begin{itemize}
\item {Grp. gram.:v. i.}
\end{itemize}
\begin{itemize}
\item {Proveniência:(De \textunderscore molle\textunderscore )}
\end{itemize}
O mesmo que amolecer.
\section{Amolestar}
\begin{itemize}
\item {Grp. gram.:v. t.}
\end{itemize}
O mesmo que \textunderscore molestar\textunderscore . Cf. Camillo, \textunderscore Regicida\textunderscore , 99.
\section{Amologadela}
\begin{itemize}
\item {Grp. gram.:f.}
\end{itemize}
O mesmo que \textunderscore amolgadura\textunderscore .
\section{Amolgadura}
\begin{itemize}
\item {Grp. gram.:f.}
\end{itemize}
Acto de \textunderscore amolgar\textunderscore .
Mossa, em objecto amolgado.
\section{Amolgamento}
\begin{itemize}
\item {Grp. gram.:m.}
\end{itemize}
O mesmo que \textunderscore amolgadura\textunderscore .
\section{Amolgar}
\begin{itemize}
\item {Grp. gram.:v. t.}
\end{itemize}
\begin{itemize}
\item {Proveniência:(Do lat. \textunderscore emollicare\textunderscore )}
\end{itemize}
Contundir.
Esmagar.
Abater; achatar.
Impressionar.
Obrigar a ceder; derrotar.
\section{Amolgável}
\begin{itemize}
\item {Grp. gram.:adj.}
\end{itemize}
Que se póde \textunderscore amolgar\textunderscore .
\section{Amólico}
\begin{itemize}
\item {Grp. gram.:adj.}
\end{itemize}
Diz-se dos saes que têm por base a amolina.
\section{Amolina}
\begin{itemize}
\item {Grp. gram.:f.}
\end{itemize}
Base salificável, extrahida do óleo animal de Dippel.
\section{Amollatar}
\begin{itemize}
\item {Grp. gram.:v. t.}
\end{itemize}
\begin{itemize}
\item {Utilização:Prov.}
\end{itemize}
\begin{itemize}
\item {Utilização:minh.}
\end{itemize}
\begin{itemize}
\item {Proveniência:(De \textunderscore molle\textunderscore )}
\end{itemize}
O mesmo que \textunderscore amolgar\textunderscore .
\section{Amollecedor}
\begin{itemize}
\item {Grp. gram.:m.}
\end{itemize}
Aquelle que amollece.
\section{Amollecer}
\begin{itemize}
\item {Grp. gram.:v. t.}
\end{itemize}
\begin{itemize}
\item {Grp. gram.:V. i.}
\end{itemize}
\begin{itemize}
\item {Proveniência:(Lat. \textunderscore mollescere\textunderscore )}
\end{itemize}
Tornar molle.
Enervar.
Abrandar.
Commover.
Tornar-se molle.
\section{Amollecimento}
\begin{itemize}
\item {Grp. gram.:m.}
\end{itemize}
Acção de amollecer.
Brandura.
Enfraquecimento: \textunderscore amollecimento cerebral\textunderscore .
\section{Amollegar}
\begin{itemize}
\item {Grp. gram.:v. t.}
\end{itemize}
Tornar molle.
Amachucar, amolgar. Cp. Filinto. X. 28.
\section{Amollentar}
\begin{itemize}
\item {Grp. gram.:v. i.}
\end{itemize}
\begin{itemize}
\item {Proveniência:(De \textunderscore molle\textunderscore )}
\end{itemize}
O mesmo que amollecer.
\section{Amomáceas}
\begin{itemize}
\item {Grp. gram.:f. pl.}
\end{itemize}
\begin{itemize}
\item {Proveniência:(De \textunderscore amomáceo\textunderscore )}
\end{itemize}
Família de plantas monocotyledóneas, que têm por typo o amomo.
\section{Amomáceo}
\begin{itemize}
\item {Grp. gram.:adj.}
\end{itemize}
Relativo ou semelhante ao \textunderscore amomo\textunderscore .
\section{Amomeáceas}
\begin{itemize}
\item {Grp. gram.:f. pl.}
\end{itemize}
O mesmo que \textunderscore amomáceas\textunderscore .
\section{Amômeas}
\begin{itemize}
\item {Grp. gram.:f. pl.}
\end{itemize}
(V.amomáceas)
\section{Amomo}
\begin{itemize}
\item {Grp. gram.:m.}
\end{itemize}
\begin{itemize}
\item {Proveniência:(Gr. \textunderscore amomon\textunderscore )}
\end{itemize}
Gênero de plantas odoríferas.
\section{Amomocarpo}
\begin{itemize}
\item {Grp. gram.:m.}
\end{itemize}
Fruto fóssil de argilas terciárias.
\section{Amoncalhar}
\begin{itemize}
\item {Grp. gram.:v. t.}
\end{itemize}
\begin{itemize}
\item {Utilização:Prov.}
\end{itemize}
\begin{itemize}
\item {Utilização:minh.}
\end{itemize}
O mesmo que \textunderscore amoncanhar\textunderscore .
\section{Amoncanhar}
\begin{itemize}
\item {Grp. gram.:v. t.}
\end{itemize}
\begin{itemize}
\item {Utilização:Prov.}
\end{itemize}
\begin{itemize}
\item {Utilização:minh.}
\end{itemize}
O mesmo que \textunderscore amarfanhar\textunderscore .
\section{Amónia}
\begin{itemize}
\item {Grp. gram.:f.}
\end{itemize}
O mesmo que \textunderscore amónio\textunderscore ^1.
\section{Amoniacado}
\begin{itemize}
\item {Grp. gram.:adj.}
\end{itemize}
Que tem sal \textunderscore amoníaco\textunderscore .
\section{Amoniacal}
\begin{itemize}
\item {Grp. gram.:adj.}
\end{itemize}
Que tem amoníaco ou propriedades do amoníaco.
\section{Amoniáceo}
\begin{itemize}
\item {Grp. gram.:adj.}
\end{itemize}
O mesmo que \textunderscore amoniacado\textunderscore .
\section{Amoníaco}
\begin{itemize}
\item {Grp. gram.:m.}
\end{itemize}
\begin{itemize}
\item {Grp. gram.:Adj.}
\end{itemize}
\begin{itemize}
\item {Proveniência:(Gr. \textunderscore ammoniakos\textunderscore )}
\end{itemize}
Gás, que se encontra no estado de combinação com os ácidos chlorhýdrico e phosphórico na urina, e com os
ácidos acético e carbónico nas matérias animaes em putrefacção.
O mesmo que \textunderscore amoniacal\textunderscore .
\section{Amonialdeído}
\begin{itemize}
\item {Grp. gram.:m.}
\end{itemize}
Combinação de aldehydo e amoníaco.
\section{Amoniato}
\begin{itemize}
\item {Grp. gram.:m.}
\end{itemize}
\begin{itemize}
\item {Proveniência:(De \textunderscore ammónia\textunderscore )}
\end{itemize}
Corpo, resultante da combinação do amoníaco com um óxydo metállico.
\section{Amónico...}
\begin{itemize}
\item {Proveniência:(De \textunderscore ammoníaco\textunderscore )}
\end{itemize}
Elemento, que entra na composição de certos adjectivos, para designar uma combinação em que entra amoníaco: \textunderscore amónico-hýdrico\textunderscore , \textunderscore amónico-potássico\textunderscore , etc.
\section{Amoniemia}
\begin{itemize}
\item {Grp. gram.:f.}
\end{itemize}
\begin{itemize}
\item {Proveniência:(Do gr. \textunderscore ammoniakos\textunderscore  + \textunderscore haima\textunderscore )}
\end{itemize}
Doença, determinada pela presença de amoníaco ou dos seus saes no sangue.
\section{Amonieto}
\begin{itemize}
\item {Grp. gram.:m.}
\end{itemize}
O mesmo ou melhor que \textunderscore amoniureto\textunderscore .
\section{Amonímetro}
\begin{itemize}
\item {Grp. gram.:m.}
\end{itemize}
\begin{itemize}
\item {Proveniência:(Do gr. \textunderscore ammoniakos\textunderscore  + \textunderscore metron\textunderscore )}
\end{itemize}
Apparelho, para a dosagem do amoníaco.
\section{Amónio}
\begin{itemize}
\item {Grp. gram.:m.}
\end{itemize}
\begin{itemize}
\item {Proveniência:(De \textunderscore ammoníaco\textunderscore )}
\end{itemize}
Combinação hypothética de um equivalente de azoto e quatro de hydrogênio.
\section{Amónio}
\begin{itemize}
\item {Grp. gram.:adj.}
\end{itemize}
Relativo a Jupíter Âmon.
\section{Amonite}
\begin{itemize}
\item {Grp. gram.:f.}
\end{itemize}
\begin{itemize}
\item {Proveniência:(De \textunderscore Ammon\textunderscore , n. p.)}
\end{itemize}
Gênero de molluscos cephalópodes.
\section{Amonitídeos}
\begin{itemize}
\item {Grp. gram.:m. pl.}
\end{itemize}
Família de molluscos, que têm por typo a \textunderscore amonite\textunderscore .
\section{Amoniureto}
\begin{itemize}
\item {Grp. gram.:m.}
\end{itemize}
O mesmo que \textunderscore amoniato\textunderscore .
\section{Amonoíde}
\begin{itemize}
\item {Grp. gram.:adj.}
\end{itemize}
\begin{itemize}
\item {Proveniência:(De \textunderscore Ammon\textunderscore , n. p. + gr. \textunderscore eidos\textunderscore )}
\end{itemize}
Diz-se das conchas, que se assemelham á amonite.
\section{Amontado}
\begin{itemize}
\item {Grp. gram.:adj.}
\end{itemize}
\begin{itemize}
\item {Utilização:Prov.}
\end{itemize}
\begin{itemize}
\item {Utilização:trasm.}
\end{itemize}
\begin{itemize}
\item {Proveniência:(De \textunderscore amontar\textunderscore ^1)}
\end{itemize}
Que anda a monte.
Fugitivo.
\section{Amontanhar}
\begin{itemize}
\item {Grp. gram.:v. i.}
\end{itemize}
Elevar-se como montanha.
Avolumar-se:«\textunderscore deixou amontanhar os callos\textunderscore ». Camillo, \textunderscore Mulher Fatal\textunderscore , 26.
\section{Amontar}
\begin{itemize}
\item {Grp. gram.:v. t.}
\end{itemize}
Dar fórma de monte a.
Deixar ir para o monte.
Fazer andar no monte.
\section{Amontar}
\begin{itemize}
\item {Grp. gram.:v. i.}
\end{itemize}
\begin{itemize}
\item {Utilização:Prov.}
\end{itemize}
\begin{itemize}
\item {Utilização:alent.}
\end{itemize}
\begin{itemize}
\item {Grp. gram.:V. t.  e  i.}
\end{itemize}
\begin{itemize}
\item {Utilização:Prov.}
\end{itemize}
Importar.
Elevar-se: \textunderscore a despesa amontou a mil ducados\textunderscore .
Apparecer, mostrar-se, assomar: \textunderscore quando a lua amontava...\textunderscore 
O mesmo que \textunderscore montar\textunderscore .
\section{Amontijar}
\begin{itemize}
\item {Grp. gram.:v. t.}
\end{itemize}
\begin{itemize}
\item {Utilização:Prov.}
\end{itemize}
\begin{itemize}
\item {Utilização:alent.}
\end{itemize}
Cavar (a terra), formando montijos.
\section{Amontilhar}
\begin{itemize}
\item {Grp. gram.:v. t.}
\end{itemize}
O mesmo que \textunderscore amontijar\textunderscore . Cf. \textunderscore Techn. Rur.\textunderscore , 66.
\section{Amontôa}
\begin{itemize}
\item {Grp. gram.:f.}
\end{itemize}
\begin{itemize}
\item {Proveniência:(De \textunderscore amontoar\textunderscore )}
\end{itemize}
Operação agrícola de chegar o terreno para o pé das plantas que são susceptíveis de raízes adventícias.
\section{Amontoação}
\begin{itemize}
\item {Grp. gram.:f.}
\end{itemize}
Acção de \textunderscore amontoar\textunderscore .
\section{Amontoador}
\begin{itemize}
\item {Grp. gram.:m.}
\end{itemize}
Aquelle que amontôa.
Espécie de charrua simples ou arado de duas aivecas, para levantar a terra e aconchegá-la ás plantas.
\section{Amontoamento}
\begin{itemize}
\item {Grp. gram.:m.}
\end{itemize}
O mesmo que \textunderscore amontoação\textunderscore .
Acumulação; montão.
\section{Amontoar}
\begin{itemize}
\item {Grp. gram.:v. t.}
\end{itemize}
\begin{itemize}
\item {Grp. gram.:V. i.}
\end{itemize}
Pôr em montão.
Juntar desordenadamente, acumular.
Subir, erguer-se, á maneira de montão.
\section{Amonturar}
\begin{itemize}
\item {Grp. gram.:v. t.}
\end{itemize}
Juntar em monturo.
\section{Amoorado}
\begin{itemize}
\item {Grp. gram.:adj.}
\end{itemize}
\begin{itemize}
\item {Utilização:Ant.}
\end{itemize}
O mesmo que \textunderscore enfermo\textunderscore . Cf. Fern. Lopes, \textunderscore Chrón. de D. Fern.\textunderscore 
\section{Amor}
\begin{itemize}
\item {Grp. gram.:m.}
\end{itemize}
\begin{itemize}
\item {Utilização:Ant.}
\end{itemize}
\begin{itemize}
\item {Grp. gram.:Loc. prepos.}
\end{itemize}
\begin{itemize}
\item {Utilização:Fam.}
\end{itemize}
\begin{itemize}
\item {Proveniência:(Lat. \textunderscore amor\textunderscore )}
\end{itemize}
Conjunto de phenómenos cerebraes e affectivos, que constituem o instincto sexual.
Affeição profunda de alguém a indivíduo de sexo differente.
Objecto dessa affeição: \textunderscore és o meu amor\textunderscore .
Affecto a pessôas ou coisas: \textunderscore amor ás riquezas\textunderscore . Paixão.
Enthusiasmo.
Favor, graça, mercê.
\textunderscore Por amor de\textunderscore , por causa de: \textunderscore fugiu logo, por amor de evitar a polícia\textunderscore .
\section{Amora}
\begin{itemize}
\item {Grp. gram.:f.}
\end{itemize}
\begin{itemize}
\item {Proveniência:(Lat. \textunderscore mora\textunderscore , \textunderscore pl.\textunderscore  de \textunderscore morum\textunderscore )}
\end{itemize}
Fruto da amoreira e de algumas espécies de silvas.
\section{Amorado}
\begin{itemize}
\item {Grp. gram.:adj.}
\end{itemize}
Que tem côr de amora.
\section{Amorado}
\begin{itemize}
\item {Grp. gram.:m.  e  adj.}
\end{itemize}
\begin{itemize}
\item {Utilização:Prov.}
\end{itemize}
\begin{itemize}
\item {Utilização:alg.}
\end{itemize}
\begin{itemize}
\item {Proveniência:(De \textunderscore amor\textunderscore )}
\end{itemize}
O mesmo que \textunderscore namorado\textunderscore .
\section{Amorar}
\begin{itemize}
\item {Grp. gram.:v. t.}
\end{itemize}
\begin{itemize}
\item {Utilização:Ant.}
\end{itemize}
\begin{itemize}
\item {Grp. gram.:V. p.}
\end{itemize}
\begin{itemize}
\item {Utilização:Ant.}
\end{itemize}
\begin{itemize}
\item {Proveniência:(De \textunderscore morada\textunderscore )}
\end{itemize}
Esconder; guardar.
Mudar de morada.
Fugir.
\section{Amoratado}
\begin{itemize}
\item {Grp. gram.:adj.}
\end{itemize}
\begin{itemize}
\item {Utilização:Prov.}
\end{itemize}
\begin{itemize}
\item {Utilização:trasm.}
\end{itemize}
Que tem muito amor.
\section{Amorativo}
\begin{itemize}
\item {Grp. gram.:adj.}
\end{itemize}
Próprio para amar:«\textunderscore o eu amorativo\textunderscore ». Camillo, \textunderscore Senh. do P. de Ninães\textunderscore , 30.
\section{Amorável}
\begin{itemize}
\item {Grp. gram.:adj.}
\end{itemize}
Que trata com amor; terno; affável.
Em que há affeição ou ternura.
\section{Amoravelmente}
\begin{itemize}
\item {Grp. gram.:adv.}
\end{itemize}
De modo \textunderscore amorável\textunderscore .
\section{Amordaçar}
\begin{itemize}
\item {Grp. gram.:v. t.}
\end{itemize}
Pôr mordaça em.
Açamar.
Impedir de falar: \textunderscore esta lei veio amordaçar os jornalistas\textunderscore .
\section{Amor-de-hortelão}
\begin{itemize}
\item {Grp. gram.:m.}
\end{itemize}
Planta rubiácea, (\textunderscore galium aparine\textunderscore , Lin.).
\section{Amor-do-campo}
\begin{itemize}
\item {Grp. gram.:m.}
\end{itemize}
\begin{itemize}
\item {Utilização:Bras}
\end{itemize}
Desmódio rasteiro.
\section{Amoreia}
\begin{itemize}
\item {Grp. gram.:f.}
\end{itemize}
Pequeno peixe do Brasil.
\section{Amoreira}
\begin{itemize}
\item {Grp. gram.:f.}
\end{itemize}
Árvore, da fam. das moráceas, e de que são principaes espécies a amoreira branca e a amoreira negra.
(B. lat. \textunderscore moraria\textunderscore )
\section{Amoreiral}
\begin{itemize}
\item {Grp. gram.:m.}
\end{itemize}
Lugar plantado de amoreiras.
\section{Amorenado}
\begin{itemize}
\item {Grp. gram.:adj.}
\end{itemize}
Tirante a moreno; quási moreno.
\section{Amorento}
\begin{itemize}
\item {Grp. gram.:adj.}
\end{itemize}
\begin{itemize}
\item {Utilização:Des.}
\end{itemize}
Que tem amores; enamorado.
\section{Amores}
\begin{itemize}
\item {fónica:mô}
\end{itemize}
\begin{itemize}
\item {Grp. gram.:m. pl.}
\end{itemize}
Namôro.
Objecto amado.
Tempo em que se ama.
(Pl. de \textunderscore amor\textunderscore )
\section{Amores-de-burro}
\begin{itemize}
\item {Grp. gram.:m. pl.}
\end{itemize}
\begin{itemize}
\item {Utilização:Prov.}
\end{itemize}
Planta da fam. das compostas, (\textunderscore kerneria pilosa\textunderscore ).
\section{Amorete}
\begin{itemize}
\item {fónica:mo-rê}
\end{itemize}
\begin{itemize}
\item {Grp. gram.:m.}
\end{itemize}
\begin{itemize}
\item {Utilização:Ant.}
\end{itemize}
Tecido, entrançado de prata.
\section{Amoreuxia}
\begin{itemize}
\item {fónica:csi}
\end{itemize}
\begin{itemize}
\item {Grp. gram.:f.}
\end{itemize}
Género de plantas americanas.
\section{Amorfa}
\begin{itemize}
\item {Grp. gram.:f.}
\end{itemize}
\begin{itemize}
\item {Proveniência:(Do gr. \textunderscore a\textunderscore  priv. e \textunderscore morphe\textunderscore )}
\end{itemize}
Gênero de plantas leguminosas.
\section{Amorfanhar}
\begin{itemize}
\item {Grp. gram.:v. t.}
\end{itemize}
O mesmo que \textunderscore amarfanhar\textunderscore . Cf. J. Dinis, \textunderscore Morgadinha\textunderscore , 295.
\section{Amorfo}
\begin{itemize}
\item {Grp. gram.:adj.}
\end{itemize}
\begin{itemize}
\item {Proveniência:(Do gr. \textunderscore a\textunderscore  priv. + \textunderscore morphe\textunderscore )}
\end{itemize}
Que não tem fórma determinada.
\section{Amorfófalo}
\begin{itemize}
\item {Grp. gram.:m.}
\end{itemize}
Planta dos jardins.
\section{Amorfófito}
\begin{itemize}
\item {Grp. gram.:adj.}
\end{itemize}
\begin{itemize}
\item {Proveniência:(Do gr. \textunderscore a\textunderscore  priv. + \textunderscore morphe\textunderscore  + \textunderscore phuton\textunderscore )}
\end{itemize}
Que tem flôres irregulares ou anormaes.
\section{Amorfosoma}
\begin{itemize}
\item {Grp. gram.:m.}
\end{itemize}
Gênero de insectos coleópteros pentâmeros.
\section{Amoricos}
\begin{itemize}
\item {Grp. gram.:m. pl.}
\end{itemize}
\begin{itemize}
\item {Proveniência:(De \textunderscore amores\textunderscore )}
\end{itemize}
Amores ligeiros: namôro.
\section{Amorífero}
\begin{itemize}
\item {Grp. gram.:adj.}
\end{itemize}
\begin{itemize}
\item {Proveniência:(Do lat. \textunderscore amor\textunderscore  + \textunderscore ferre\textunderscore )}
\end{itemize}
Que encerra amor.
Que provoca amor.
\section{Amorim}
\begin{itemize}
\item {Grp. gram.:f.  e  adj.}
\end{itemize}
Designação de três variedades de pêras, uma das quaes é talvez a conhecida por \textunderscore lambe-lhe-os-dedos\textunderscore .
(Segundo alguns, de \textunderscore Amorim\textunderscore , n. p. de uma pov. minh.; mais provavelmente corr. pop. de \textunderscore amerim\textunderscore )
\section{Amorinhos}
\begin{itemize}
\item {Grp. gram.:m. pl.}
\end{itemize}
O mesmo que \textunderscore amoricos\textunderscore .
\section{Amorio}
\begin{itemize}
\item {Grp. gram.:m.}
\end{itemize}
\begin{itemize}
\item {Utilização:Ant.}
\end{itemize}
O mesmo que \textunderscore amoricos\textunderscore .
Amizade, relações cordiaes.
\section{Amoriscado}
\begin{itemize}
\item {Grp. gram.:adj.}
\end{itemize}
Namorado ou próprio de namorado:«\textunderscore donzellas amoriscadas\textunderscore ». Camillo, \textunderscore Freira no Subt.\textunderscore , 14.«\textunderscore Sorrisos amoriscados\textunderscore ». \textunderscore Idem, Caveira\textunderscore .
\section{Amoriscar-se}
\begin{itemize}
\item {Grp. gram.:v. p.}
\end{itemize}
\begin{itemize}
\item {Proveniência:(T. cast.)}
\end{itemize}
Tomar amorios:«\textunderscore ella amoriscara-se do padre\textunderscore ». Camillo, \textunderscore Eusébio\textunderscore .
\section{Amormado}
\begin{itemize}
\item {Grp. gram.:adj.}
\end{itemize}
Doente de mormo.
Adoentado.
\section{Amornar}
\begin{itemize}
\item {Grp. gram.:v. t.}
\end{itemize}
Tornar morno, tépido.
Aquecer levemente.
\section{Amornecer}
\begin{itemize}
\item {Grp. gram.:v. t.}
\end{itemize}
\begin{itemize}
\item {Grp. gram.:V. i.}
\end{itemize}
O mesmo que \textunderscore amornar\textunderscore .
Ficar morno.
\section{Amornetado}
\begin{itemize}
\item {Grp. gram.:adj. ?}
\end{itemize}
«\textunderscore ...a uso de galantes amornetados\textunderscore ». \textunderscore Aulegrafia\textunderscore , 1.
\section{Amorosa}
\begin{itemize}
\item {Grp. gram.:f.}
\end{itemize}
\begin{itemize}
\item {Utilização:Ant.}
\end{itemize}
\begin{itemize}
\item {Proveniência:(De \textunderscore amoroso\textunderscore )}
\end{itemize}
Música de instrumento de corda, acompanhando motivos melodiosos ou sentimentaes.
Planta medicinal do Brasil.
\section{Amorosamente}
\begin{itemize}
\item {Grp. gram.:adv.}
\end{itemize}
De modo \textunderscore amoroso\textunderscore .
Com amor.
\section{Amorosidade}
\begin{itemize}
\item {Grp. gram.:f.}
\end{itemize}
Qualidade do que é amoroso.
\section{Amoroso}
\begin{itemize}
\item {Grp. gram.:adj.}
\end{itemize}
Que tem amor; carinhoso.
Suave.
Meigo.
\section{Amoroso}
\begin{itemize}
\item {fónica:óssò}
\end{itemize}
\begin{itemize}
\item {Grp. gram.:adj.}
\end{itemize}
\begin{itemize}
\item {Proveniência:(T. it.)}
\end{itemize}
Termo, indicativo de que a peça musical, que êlle precede, deve sêr executada com ternura e graça.
\section{Amor-perfeito}
\begin{itemize}
\item {Grp. gram.:m.}
\end{itemize}
Espécie de violáceas, que abrange plantas, cujas flôres, de cinco pétalas, apresentam as mais variadas côres.
\section{Amorpha}
\begin{itemize}
\item {Grp. gram.:f.}
\end{itemize}
\begin{itemize}
\item {Proveniência:(Do gr. \textunderscore a\textunderscore  priv. e \textunderscore morphe\textunderscore )}
\end{itemize}
Gênero de plantas leguminosas.
\section{Amorphia}
\begin{itemize}
\item {Grp. gram.:f.}
\end{itemize}
\begin{itemize}
\item {Proveniência:(De \textunderscore amorpho\textunderscore )}
\end{itemize}
Deformidade; carência de fórma determinada.
\section{Amorpho}
\begin{itemize}
\item {Grp. gram.:adj.}
\end{itemize}
\begin{itemize}
\item {Proveniência:(Do gr. \textunderscore a\textunderscore  priv. + \textunderscore morphe\textunderscore )}
\end{itemize}
Que não tem fórma determinada.
\section{Amorphóphalo}
\begin{itemize}
\item {Grp. gram.:m.}
\end{itemize}
Planta dos jardins.
\section{Amorphóphyto}
\begin{itemize}
\item {Grp. gram.:adj.}
\end{itemize}
\begin{itemize}
\item {Proveniência:(Do gr. \textunderscore a\textunderscore  priv. + \textunderscore morphe\textunderscore  + \textunderscore phuton\textunderscore )}
\end{itemize}
Que tem flôres irregulares ou anormaes.
\section{Amorphosoma}
\begin{itemize}
\item {Grp. gram.:m.}
\end{itemize}
Gênero de insectos coleópteros pentâmeros.
\section{Amorrinhar-se}
\begin{itemize}
\item {Grp. gram.:v. p.}
\end{itemize}
\begin{itemize}
\item {Utilização:Ext.}
\end{itemize}
Adoecer de morrinha.
Enfraquecer; alquebrar-se.
\section{Amorsegar}
\begin{itemize}
\item {Grp. gram.:v. t.}
\end{itemize}
O mesmo que \textunderscore morsegar\textunderscore .
\section{Amortalhadeira}
\begin{itemize}
\item {Grp. gram.:f.}
\end{itemize}
Mulher, que amortalha defuntos.
\section{Amortalhador}
\begin{itemize}
\item {Grp. gram.:m.  e  adj.}
\end{itemize}
O que amortalha.
\section{Amortalhar}
\begin{itemize}
\item {Grp. gram.:v. t.}
\end{itemize}
Envolver em mortalha.
Vestir com hábito grosseiro, por penitência ou desprendimento.
\section{Amorteamento}
\begin{itemize}
\item {Grp. gram.:m.}
\end{itemize}
Acto ou effeito de \textunderscore amortear\textunderscore .
\section{Amortear}
\begin{itemize}
\item {Grp. gram.:v. t.}
\end{itemize}
\begin{itemize}
\item {Utilização:Des.}
\end{itemize}
\begin{itemize}
\item {Proveniência:(De \textunderscore morte\textunderscore )}
\end{itemize}
Desalentar, desanimar.
\section{Amortecer}
\begin{itemize}
\item {Grp. gram.:v. t.}
\end{itemize}
\begin{itemize}
\item {Grp. gram.:V. i.}
\end{itemize}
Fazer ficar como morto.
Enfraquecer; abrandar.
Desfallecer.
Afroixar: \textunderscore a luz amortecia\textunderscore .
\section{Amortecimento}
\begin{itemize}
\item {Grp. gram.:m.}
\end{itemize}
Acção de amortecer.
Enfraquecimento.
\section{Amortiçar-se}
\begin{itemize}
\item {Grp. gram.:v. p.}
\end{itemize}
Tornar-se mortíço: extinguir-se. Cf. Camillo, \textunderscore Quéda de um Anjo\textunderscore , 223.
\section{Amortido}
\begin{itemize}
\item {Grp. gram.:m.}
\end{itemize}
\begin{itemize}
\item {Utilização:Ant.}
\end{itemize}
O mesmo que \textunderscore pináculo\textunderscore .
\section{Amortificado}
\begin{itemize}
\item {Grp. gram.:adj.}
\end{itemize}
O mesmo que [[amortecido|amortecer]]. Cf. Cortesão, \textunderscore Subs\textunderscore .
\section{Amortificar}
\begin{itemize}
\item {Grp. gram.:v. t.}
\end{itemize}
O mesmo que \textunderscore amortecer\textunderscore :«\textunderscore esta má semente dos infieis mortificaria toda bôa semente\textunderscore ». Azurara, \textunderscore Chrón. de D. João I\textunderscore , c. LXIX.
\section{Amortização}
\begin{itemize}
\item {Grp. gram.:f.}
\end{itemize}
Acção de \textunderscore amortizar\textunderscore .
\section{Amortizar}
\begin{itemize}
\item {Grp. gram.:v. t.}
\end{itemize}
\begin{itemize}
\item {Proveniência:(Do lat. hyp. \textunderscore admortitiare\textunderscore )}
\end{itemize}
Extinguir (dividas) a pouco e pouco, ou em prestações.
Passar (bens) para as chamadas corporações de mão morta.
\section{Amortizável}
\begin{itemize}
\item {Grp. gram.:adj.}
\end{itemize}
Que se póde amortizar; que deve sêr amortizado.
\section{Amorudo}
\begin{itemize}
\item {Grp. gram.:adj.}
\end{itemize}
\begin{itemize}
\item {Utilização:Chul.}
\end{itemize}
Apaixonado.
Inclinado ao amor.
\section{Amossar}
\begin{itemize}
\item {Grp. gram.:v. t.}
\end{itemize}
Fazer mossas em.
\section{Amossegar}
\textunderscore v. t.\textunderscore  (e der.)
O mesmo que \textunderscore amorsegar\textunderscore , etc.
\section{Amostado}
\begin{itemize}
\item {Grp. gram.:adj.}
\end{itemize}
Que sabe a mosto. Cf. \textunderscore Techn. Rur.\textunderscore , 156 e 161.
\section{Amostardado}
\begin{itemize}
\item {Grp. gram.:adj.}
\end{itemize}
Temperado com mostarda. Cf. Th. Ribeiro, \textunderscore Jornadas\textunderscore , I, 309.
\section{Amostra}
\begin{itemize}
\item {Grp. gram.:f.}
\end{itemize}
Acto de \textunderscore amostrar\textunderscore .
Sinal.
Modêlo.
Exposição.
\section{Amostradora}
\begin{itemize}
\item {Grp. gram.:f.}
\end{itemize}
\begin{itemize}
\item {Utilização:Gír.}
\end{itemize}
\begin{itemize}
\item {Proveniência:(De \textunderscore amostrar\textunderscore )}
\end{itemize}
Lanterna.
\section{Amostrar}
\textunderscore v. t.\textunderscore  (e der.)
O mesmo que \textunderscore mostrar\textunderscore , etc.
\section{Amostrinha}
\begin{itemize}
\item {Grp. gram.:f.}
\end{itemize}
\begin{itemize}
\item {Utilização:Bras}
\end{itemize}
\begin{itemize}
\item {Proveniência:(De \textunderscore amostra\textunderscore )}
\end{itemize}
Rapé.
\section{Amota}
\begin{itemize}
\item {Grp. gram.:f.}
\end{itemize}
O mesmo que \textunderscore mota\textunderscore .
\section{Amotar}
\begin{itemize}
\item {Grp. gram.:v. t.}
\end{itemize}
Guarnecer de motas.
\section{Amotinação}
\begin{itemize}
\item {Grp. gram.:f.}
\end{itemize}
Acto de \textunderscore amotinar\textunderscore .
\section{Amotinadamente}
\begin{itemize}
\item {Grp. gram.:adv.}
\end{itemize}
Com motim, com alvorôço.
\section{Amotinador}
\begin{itemize}
\item {Grp. gram.:m.}
\end{itemize}
Aquelle que amotina.
\section{Amotinar}
\begin{itemize}
\item {Grp. gram.:v. t.}
\end{itemize}
Pôr em motim; alvoroçar.
Sublevar; revoltar.
\section{Amotinável}
\begin{itemize}
\item {Grp. gram.:adj.}
\end{itemize}
Que facilmente se amotina.
\section{Amoucado}
\begin{itemize}
\item {Grp. gram.:adj.}
\end{itemize}
Um tanto mouco.
\section{Amoucado}
\begin{itemize}
\item {Grp. gram.:adj.}
\end{itemize}
Feito amouco.
\section{Amouco}
\begin{itemize}
\item {Grp. gram.:m.}
\end{itemize}
\begin{itemize}
\item {Utilização:Fig.}
\end{itemize}
Aquelle que, na Índia, jura morrer pelo seu chefe. Cf. Couto, \textunderscore Déc\textunderscore . IV, l. VII, c. 14.
Homem servil, que em tudo, e á custa de tudo, defende e lisonjeia seus superiores ou chefes.
(Talvez do mal.)
\section{Amourado}
\begin{itemize}
\item {Grp. gram.:adj.}
\end{itemize}
\begin{itemize}
\item {Utilização:Agr.}
\end{itemize}
Levantado em mourão. Cf. \textunderscore Techn. Rur\textunderscore ., 565.
\section{Amouriscar}
\begin{itemize}
\item {Grp. gram.:v. t.}
\end{itemize}
\begin{itemize}
\item {Grp. gram.:V. i.}
\end{itemize}
Dar aspecto ou feitio mourisco a.
Ter feição de mouro. Cf. \textunderscore Fenix Renasc.\textunderscore , V, 46.
\section{Amouroar}
\begin{itemize}
\item {Grp. gram.:v. i.}
\end{itemize}
\begin{itemize}
\item {Utilização:Prov.}
\end{itemize}
\begin{itemize}
\item {Utilização:Prov.}
\end{itemize}
\begin{itemize}
\item {Utilização:beir.}
\end{itemize}
Encostar-se, postar-se: \textunderscore Costumava amouroar á porta da namorada\textunderscore .
Quedar-se (o gado) nas horas de calor, arquejante e muito junto.
(Provavelmente, de \textunderscore mourão\textunderscore )
\section{Amover}
\begin{itemize}
\item {Grp. gram.:v. t.}
\end{itemize}
\begin{itemize}
\item {Proveniência:(Do lat. \textunderscore amovere\textunderscore )}
\end{itemize}
Afastar.
Desapossar.
\section{Amovibilidade}
\begin{itemize}
\item {Grp. gram.:f.}
\end{itemize}
Qualidade do que é amovível.
\section{Amovível}
\begin{itemize}
\item {Grp. gram.:adj.}
\end{itemize}
\begin{itemize}
\item {Proveniência:(De \textunderscore amover\textunderscore )}
\end{itemize}
Que póde ser afastado, transferido.
Transitório.
\section{Âmovo-inamovível}
\begin{itemize}
\item {Grp. gram.:adj.}
\end{itemize}
\begin{itemize}
\item {Proveniência:(De \textunderscore amovível\textunderscore  + \textunderscore inamovível\textunderscore )}
\end{itemize}
Diz-se do apparelho cirúrgico que, applicado no tratamento das fracturas, mantém coaptados os fragmentos ósseos, mas que póde abrir-se, para se examinar o osso.
\section{Amoxamar}
\begin{itemize}
\item {Grp. gram.:v. t.}
\end{itemize}
Secar como moxama.
Tornar magro.
\section{Ampa}
\begin{itemize}
\item {Grp. gram.:m.}
\end{itemize}
Figueira de Madagáscar.
\section{Ampalária}
\begin{itemize}
\item {Grp. gram.:f.}
\end{itemize}
Gênero de molluscos gasterópodes.
\section{Ampallária}
\begin{itemize}
\item {Grp. gram.:f.}
\end{itemize}
Gênero de molluscos gasterópodes.
\section{Amparada}
\begin{itemize}
\item {Grp. gram.:f.}
\end{itemize}
\begin{itemize}
\item {Utilização:Des.}
\end{itemize}
Lugar amparado ou abrigado. Cf. \textunderscore Lendas da Índia\textunderscore , I, 27.
\section{Amparadamente}
\begin{itemize}
\item {Grp. gram.:adv.}
\end{itemize}
Com amparo.
\section{Amparador}
\begin{itemize}
\item {Grp. gram.:m.}
\end{itemize}
Aquelle que ampara.
\section{Amparamento}
\begin{itemize}
\item {Grp. gram.:m.}
\end{itemize}
(V.amparo)
\section{Amparar}
\begin{itemize}
\item {Grp. gram.:v. t.}
\end{itemize}
Servir de amparo a; suster; estear: \textunderscore amparar uma videira\textunderscore .
Defender.
Patrocinar: \textunderscore amparar os desvalidos\textunderscore .
(B. lat. \textunderscore imparare\textunderscore )
\section{Amparável}
\begin{itemize}
\item {Grp. gram.:adj.}
\end{itemize}
\begin{itemize}
\item {Utilização:Des.}
\end{itemize}
Que dá amparo.
Que proporciona bôa fortuna:«\textunderscore altos poderes das amparáveis fadas\textunderscore ». Filinto, VIII, 85.
\section{Amparo}
\begin{itemize}
\item {Grp. gram.:m.}
\end{itemize}
Acção de \textunderscore amparar\textunderscore .
Coisa ou pessoa que ampara: \textunderscore fôste o meu amparo\textunderscore .
Esteio: Auxílio.
Defesa.
Protecção.
Resguardo.
Refúgio.
\section{Ampédio}
\begin{itemize}
\item {Grp. gram.:m.}
\end{itemize}
O mesmo que \textunderscore âmpedo\textunderscore .
\section{Âmpedo}
\begin{itemize}
\item {Grp. gram.:m.}
\end{itemize}
\begin{itemize}
\item {Proveniência:(Do gr. \textunderscore ana\textunderscore  + \textunderscore pedion\textunderscore )}
\end{itemize}
Gênero de insectos coleópteros pentâmeros.
\section{Ampélico}
\begin{itemize}
\item {Grp. gram.:adj.}
\end{itemize}
Diz-se de um ácido, que se obtém dos productos do óleo de xisto rectificado, por meio do ácido nítrico.
\section{Ampelideáceas}
\begin{itemize}
\item {Grp. gram.:f. pl.}
\end{itemize}
O mesmo ou melhor que ampelídeas.
\section{Ampelídeas}
\begin{itemize}
\item {Grp. gram.:f. pl.}
\end{itemize}
\begin{itemize}
\item {Proveniência:(De \textunderscore ampelídeo\textunderscore )}
\end{itemize}
Família de plantas, que têm a vinha por typo.
\section{Ampelídeo}
\begin{itemize}
\item {Grp. gram.:adj.}
\end{itemize}
\begin{itemize}
\item {Proveniência:(Do gr. \textunderscore ampelos\textunderscore  + \textunderscore eidos\textunderscore )}
\end{itemize}
Relativo ou semelhante á vinha.
\section{Ampelina}
\begin{itemize}
\item {Grp. gram.:f.}
\end{itemize}
\begin{itemize}
\item {Proveniência:(Do gr. \textunderscore ampelos\textunderscore )}
\end{itemize}
Oleo amarelo, semelhante ao creosote.
\section{Ampelite}
\begin{itemize}
\item {Grp. gram.:f.}
\end{itemize}
\begin{itemize}
\item {Proveniência:(Do gr. \textunderscore ampelos\textunderscore )}
\end{itemize}
Xisto argilloso, que se usou no tratamento das videiras.
\section{Ampelócera}
\begin{itemize}
\item {Grp. gram.:f.}
\end{itemize}
Gênero de árvores americacas, de fôlhas alternas e flôres hermaphroditas ou polýgamas.
\section{Ampelocisso}
\begin{itemize}
\item {Grp. gram.:f.}
\end{itemize}
\begin{itemize}
\item {Proveniência:(Do gr. \textunderscore ampellos\textunderscore  + lat. \textunderscore cissus\textunderscore )}
\end{itemize}
Um dos gêneros de videiras, em que se divide a família das ampelídeas.
\section{Ampeloderme}
\begin{itemize}
\item {Grp. gram.:m.}
\end{itemize}
Gênero de plantas gramíneas.
\section{Ampelografia}
\begin{itemize}
\item {Grp. gram.:f.}
\end{itemize}
Tratado das vinhas, mais ou menos prático.
(Cp. \textunderscore ampelógrapho\textunderscore )
\section{Ampelógrafo}
\begin{itemize}
\item {Grp. gram.:m.}
\end{itemize}
\begin{itemize}
\item {Proveniência:(Do gr. \textunderscore ampelos\textunderscore  + \textunderscore graphein\textunderscore )}
\end{itemize}
Aquelle que escreve, scientificamente, a respeito das vinhas.
\section{Ampelographia}
\begin{itemize}
\item {Grp. gram.:f.}
\end{itemize}
Tratado das vinhas, mais ou menos prático.
(Cp. \textunderscore ampelógrapho\textunderscore )
\section{Ampelógrapho}
\begin{itemize}
\item {Grp. gram.:m.}
\end{itemize}
\begin{itemize}
\item {Proveniência:(Do gr. \textunderscore ampelos\textunderscore  + \textunderscore graphein\textunderscore )}
\end{itemize}
Aquelle que escreve, scientificamente, a respeito das vinhas.
\section{Ampelologia}
\begin{itemize}
\item {Grp. gram.:f.}
\end{itemize}
\begin{itemize}
\item {Proveniência:(Do gr. \textunderscore ampelos\textunderscore  + \textunderscore logos\textunderscore )}
\end{itemize}
Conjunto dos princípios ou theorias respeitantes á cultura e tratamento da vinha.
\section{Ampelológico}
\begin{itemize}
\item {Grp. gram.:adj.}
\end{itemize}
Relativo á ampelologia.
\section{Ampelópse}
\begin{itemize}
\item {Grp. gram.:m.}
\end{itemize}
\begin{itemize}
\item {Proveniência:(Do gr. \textunderscore ampelos\textunderscore  + \textunderscore opsis\textunderscore )}
\end{itemize}
Gênero de plantas, da fam. das ampelídeas.
\section{Ampeloterapia}
\begin{itemize}
\item {Grp. gram.:f.}
\end{itemize}
Tratamento de doenças por meio das uvas. Cf. E. Monin, \textunderscore Hyg. do Estôm.\textunderscore , 172 e 173.
\section{Ampelotherapia}
\begin{itemize}
\item {Grp. gram.:f.}
\end{itemize}
Tratamento de doenças por meio das uvas. Cf. E. Monin, \textunderscore Hyg. do Estôm.\textunderscore , 172 e 173.
\section{Ampembre}
\begin{itemize}
\item {Grp. gram.:f.}
\end{itemize}
Espécie de cabra africana.
\section{Ampere}
\begin{itemize}
\item {Grp. gram.:m.}
\end{itemize}
O mesmo que \textunderscore ampério\textunderscore .
\section{Ampérea}
\begin{itemize}
\item {Grp. gram.:f.}
\end{itemize}
\begin{itemize}
\item {Proveniência:(De \textunderscore Amper\textunderscore , n. p.)}
\end{itemize}
Gênero de plantas euphorbiáceas.
\section{Amperímetro}
\begin{itemize}
\item {Grp. gram.:m.}
\end{itemize}
\begin{itemize}
\item {Utilização:Phýs.}
\end{itemize}
\begin{itemize}
\item {Proveniência:(De \textunderscore ampério\textunderscore  + gr. \textunderscore metron\textunderscore )}
\end{itemize}
Apparelho, para medir o número de ampérios de uma corrente eléctrica.
\section{Ampério}
\begin{itemize}
\item {Grp. gram.:m.}
\end{itemize}
\begin{itemize}
\item {Utilização:Phýs.}
\end{itemize}
\begin{itemize}
\item {Proveniência:(De \textunderscore Ampèr\textunderscore , n. p. Cp. cast. \textunderscore amperio\textunderscore )}
\end{itemize}
Unidade de medida eléctrica, correspondente á passagem de um culômbio por segundo.
\section{Amperómetro}
\begin{itemize}
\item {Grp. gram.:m.}
\end{itemize}
(V.amperímetro)
\section{Amphacantho}
\begin{itemize}
\item {Grp. gram.:m.}
\end{itemize}
\begin{itemize}
\item {Proveniência:(Do gr. \textunderscore amphi\textunderscore  + \textunderscore akantha\textunderscore )}
\end{itemize}
Peixe do Oceano Índico.
\section{Amphanto}
\begin{itemize}
\item {Grp. gram.:m.}
\end{itemize}
\begin{itemize}
\item {Proveniência:(Do gr. \textunderscore amphi\textunderscore  + \textunderscore anthos\textunderscore )}
\end{itemize}
Receptáculo vegetal, que envolve e protege a flôr, como no figo.
\section{Amphásia}
\begin{itemize}
\item {Grp. gram.:f.}
\end{itemize}
\begin{itemize}
\item {Proveniência:(Do gr. \textunderscore amphi\textunderscore  + \textunderscore asis\textunderscore )}
\end{itemize}
Insecto coleóptero pentâmero da América do Norte.
\section{Amphi...}
\begin{itemize}
\item {Grp. gram.:pref.}
\end{itemize}
\begin{itemize}
\item {Proveniência:(Do gr. \textunderscore amphi\textunderscore )}
\end{itemize}
(design. de dois lados ou de dualidade)
\section{Amphião}
\begin{itemize}
\item {Grp. gram.:m.}
\end{itemize}
\begin{itemize}
\item {Proveniência:(Gr. \textunderscore Amphion\textunderscore )}
\end{itemize}
Crustáceo do Oceano Índico.
\section{Amphibiano}
\begin{itemize}
\item {Grp. gram.:adj.}
\end{itemize}
\begin{itemize}
\item {Utilização:P. us.}
\end{itemize}
\begin{itemize}
\item {Grp. gram.:M. pl.}
\end{itemize}
O mesmo que \textunderscore amphíbio\textunderscore .
O mesmo que \textunderscore batrácios\textunderscore .
\section{Amphíbio}
\begin{itemize}
\item {Grp. gram.:m.  e  adj.}
\end{itemize}
\begin{itemize}
\item {Utilização:Fig.}
\end{itemize}
\begin{itemize}
\item {Proveniência:(Do gr. \textunderscore amphi\textunderscore  + \textunderscore bios\textunderscore )}
\end{itemize}
Diz-se do animal e da planta, que vivem na terra e na água.
Aquelle que sustenta opiniões oppostas ou segue duas profissões differentes.
\section{Amphibiographia}
\begin{itemize}
\item {Grp. gram.:f.}
\end{itemize}
\begin{itemize}
\item {Proveniência:(Do gr. \textunderscore amphibios\textunderscore  + \textunderscore graphein\textunderscore )}
\end{itemize}
Descripção dos animaes amphíbios.
\section{Amphibiologia}
\begin{itemize}
\item {Grp. gram.:f.}
\end{itemize}
\begin{itemize}
\item {Proveniência:(Do gr. \textunderscore amphibios\textunderscore  + \textunderscore logos\textunderscore )}
\end{itemize}
Parte da Zoologia, que trata dos animaes amphíbios.
\section{Amphibiológico}
\begin{itemize}
\item {Grp. gram.:adj.}
\end{itemize}
Relativo á \textunderscore amphibiologia\textunderscore .
\section{Amphibiólogo}
\begin{itemize}
\item {Grp. gram.:m.}
\end{itemize}
Aquelle que se dedica ao estudo da \textunderscore amphibiologia\textunderscore .
\section{Amphíbola}
\begin{itemize}
\item {Grp. gram.:f.}
\end{itemize}
O mesmo que \textunderscore amphíbolo\textunderscore .
\section{Amphibolia}
\begin{itemize}
\item {Grp. gram.:f.}
\end{itemize}
Equívoco, que, segundo Kant, consiste em considerar da mesma fórma e attribuir á mesma faculdade objectos próprios de faculdades differentes.
(Cp. \textunderscore amphíbolo\textunderscore )
\section{Amphibólico}
\begin{itemize}
\item {Grp. gram.:adj.}
\end{itemize}
Diz-se dos mineraes, em que entra o amphíbolo, como parte constituinte.
\section{Amphibolífero}
\begin{itemize}
\item {Grp. gram.:adj.}
\end{itemize}
\begin{itemize}
\item {Proveniência:(De \textunderscore amphibolo\textunderscore , + lat. \textunderscore ferre\textunderscore )}
\end{itemize}
Que encerra amphíbolo.
\section{Amphibolita}
\begin{itemize}
\item {Grp. gram.:f.}
\end{itemize}
Rocha, composta quási exclusivamente de amphíbolo.
\section{Amphíbolo}
\begin{itemize}
\item {Grp. gram.:m.}
\end{itemize}
\begin{itemize}
\item {Proveniência:(Gr. \textunderscore amphibolos\textunderscore )}
\end{itemize}
Substância mineral, composta de sílica, cal, magnésia e, ás vezes, óxydo de ferro e de manganés.
\section{Amphibologia}
\begin{itemize}
\item {Grp. gram.:f.}
\end{itemize}
\begin{itemize}
\item {Proveniência:(Do gr. \textunderscore amphibolos\textunderscore  + \textunderscore logos\textunderscore )}
\end{itemize}
Sentido ambíguo.
Disposição de palavras, que permitte mais de um sentido.
\section{Amphibologicamente}
\begin{itemize}
\item {Grp. gram.:adv.}
\end{itemize}
De modo \textunderscore amphibológico\textunderscore .
\section{Amphibológico}
\begin{itemize}
\item {Grp. gram.:adj.}
\end{itemize}
Que encerra amphibologia.
Ambíguo.
\section{Amphibologista}
\begin{itemize}
\item {Grp. gram.:m.}
\end{itemize}
Aquelle que escreve ou fala amphibologicamente.
\section{Amphiboloide}
\begin{itemize}
\item {Grp. gram.:adj.}
\end{itemize}
\begin{itemize}
\item {Proveniência:(Do gr. \textunderscore amphibolos\textunderscore  + \textunderscore eidos\textunderscore )}
\end{itemize}
Que tem amphíbolo.
\section{Amphibolóstylo}
\begin{itemize}
\item {Grp. gram.:adj.}
\end{itemize}
\begin{itemize}
\item {Utilização:Bot.}
\end{itemize}
Diz-se das plantas, cujo estilete é pouco visível.
\section{Amphíbraco}
\begin{itemize}
\item {Grp. gram.:m.}
\end{itemize}
\begin{itemize}
\item {Proveniência:(Do gr. \textunderscore amphi\textunderscore  + \textunderscore brachus\textunderscore )}
\end{itemize}
Pé de verso grego ou latino, com uma sýllaba longa entre duas breves.
\section{Amphicarpo}
\begin{itemize}
\item {Grp. gram.:m.}
\end{itemize}
\begin{itemize}
\item {Proveniência:(Do gr. \textunderscore amphi\textunderscore  + \textunderscore karpos\textunderscore )}
\end{itemize}
Planta, da fam. das gramíneas.
\section{Amphícomo}
\begin{itemize}
\item {Grp. gram.:m.}
\end{itemize}
Gênero de coleópteros.
\section{Amphicrânia}
\begin{itemize}
\item {Grp. gram.:f.}
\end{itemize}
\begin{itemize}
\item {Proveniência:(Do gr. \textunderscore amphikranos\textunderscore )}
\end{itemize}
Insecto coleóptero pentâmero.
\section{Amphyctiões}
\begin{itemize}
\item {Grp. gram.:m. pl.}
\end{itemize}
\begin{itemize}
\item {Utilização:Ant.}
\end{itemize}
\begin{itemize}
\item {Proveniência:(Do gr. \textunderscore amphiktuon\textunderscore )}
\end{itemize}
Representantes dos Estados gregos, que se reuniam para deliberar sôbre negocios geraes.
\section{Amphyctionia}
\begin{itemize}
\item {Grp. gram.:f.}
\end{itemize}
\begin{itemize}
\item {Proveniência:(Do gr. \textunderscore amphiktuon\textunderscore )}
\end{itemize}
Reunião dos amphyctiões.
Direito de sêr representado nessa assembleia.
\section{Amphyctiónio}
\begin{itemize}
\item {Grp. gram.:adj.}
\end{itemize}
Relativo aos \textunderscore amphyctiões\textunderscore .
\section{Amphyciclo}
\begin{itemize}
\item {Grp. gram.:m.}
\end{itemize}
\begin{itemize}
\item {Utilização:P. us.}
\end{itemize}
\begin{itemize}
\item {Proveniência:(Do gr. \textunderscore amphi\textunderscore  + \textunderscore kuklos\textunderscore )}
\end{itemize}
O crescente da lua.
\section{Amphídases}
\begin{itemize}
\item {Grp. gram.:m. pl.}
\end{itemize}
\begin{itemize}
\item {Proveniência:(Do gr. \textunderscore amphidasus\textunderscore )}
\end{itemize}
Insectos lepidópteros nocturnos.
\section{Amphideão}
\begin{itemize}
\item {Grp. gram.:m.}
\end{itemize}
\begin{itemize}
\item {Utilização:Anat.}
\end{itemize}
Orifício do útero.
\section{Amphidésmio}
\begin{itemize}
\item {Grp. gram.:m.}
\end{itemize}
Gênero de plantas polypódeas.
\section{Amphidesmo}
\begin{itemize}
\item {Grp. gram.:m.}
\end{itemize}
\begin{itemize}
\item {Proveniência:(Do gr. \textunderscore amphi\textunderscore  + \textunderscore desmos\textunderscore )}
\end{itemize}
Mollusco acéphalo.
\section{Amphido}
\begin{itemize}
\item {Grp. gram.:m.}
\end{itemize}
\begin{itemize}
\item {Utilização:Chím.}
\end{itemize}
Sal, de composição ternária, resultante da combinação de um ácido com qualquer base, (seg. Berzélio).
\section{Amphídoro}
\begin{itemize}
\item {Grp. gram.:m.}
\end{itemize}
Gênero de coleópteros.
\section{Amphidoxa}
\begin{itemize}
\item {fónica:csa}
\end{itemize}
\begin{itemize}
\item {Grp. gram.:f.}
\end{itemize}
Gênero de plantas compostas, na África austral.
\section{Amphidromia}
\begin{itemize}
\item {Grp. gram.:f.}
\end{itemize}
\begin{itemize}
\item {Proveniência:(Do gr. \textunderscore amphidromos\textunderscore )}
\end{itemize}
Festa, com que os antigos Gregos saudavam os nascimentos e em que davam nome aos recém-nascidos.
\section{Amphigênias}
\begin{itemize}
\item {Grp. gram.:f. pl.}
\end{itemize}
Vegetaes, que têm desenvolvimento discoide.
(Cp. \textunderscore amphigênio\textunderscore )
\section{Amphigênico}
\begin{itemize}
\item {Grp. gram.:adj.}
\end{itemize}
Relativo ás \textunderscore amphigênias\textunderscore .
\section{Amphigênio}
\begin{itemize}
\item {Grp. gram.:m.}
\end{itemize}
\begin{itemize}
\item {Proveniência:(Do gr. \textunderscore amphi\textunderscore  + \textunderscore genos\textunderscore )}
\end{itemize}
Silicato de potassa e de alumina.
\section{Amphigenita}
\begin{itemize}
\item {Grp. gram.:f.}
\end{itemize}
Basalto, em que o feldspatho é substituido em grande parte pelo amphigênio.
\section{Amphiglossa}
\begin{itemize}
\item {Grp. gram.:f.}
\end{itemize}
\begin{itemize}
\item {Proveniência:(Gr. \textunderscore amphiglossos\textunderscore )}
\end{itemize}
Gênero de plantas da fam. das compostas.
\section{Amphigonia}
\begin{itemize}
\item {Grp. gram.:f.}
\end{itemize}
\begin{itemize}
\item {Proveniência:(T. criado por Spencer)}
\end{itemize}
Geração sexual.
\section{Amphiguri}
\begin{itemize}
\item {Grp. gram.:m.}
\end{itemize}
\begin{itemize}
\item {Proveniência:(Gr. \textunderscore amphigouri\textunderscore )}
\end{itemize}
Discurso ou trecho, feito para sêr inintelligível. Qualquer peça literária, desordenada e sem sentido.
\section{Amphiguricamente}
\begin{itemize}
\item {Grp. gram.:adv.}
\end{itemize}
De modo \textunderscore amphigúrico\textunderscore .
\section{Amphigúrico}
\begin{itemize}
\item {Grp. gram.:adj.}
\end{itemize}
Que encerra amphiguri.
\section{Amphigurítico}
\begin{itemize}
\item {Grp. gram.:adj.}
\end{itemize}
Que tem fórma de amphiguri. Cf. Herculano, \textunderscore Carta a Torresão\textunderscore .
\section{Amphíloco}
\begin{itemize}
\item {Grp. gram.:m.}
\end{itemize}
Gênero de coleópteros.
\section{Amphílopho}
\begin{itemize}
\item {Grp. gram.:m.}
\end{itemize}
\begin{itemize}
\item {Proveniência:(Do gr. \textunderscore amphi\textunderscore  + \textunderscore lophos\textunderscore )}
\end{itemize}
Planta americana, da fam. das begoniáceas.
\section{Amphímacro}
\begin{itemize}
\item {Grp. gram.:m.}
\end{itemize}
\begin{itemize}
\item {Proveniência:(Do gr. \textunderscore amphi\textunderscore  + \textunderscore makros\textunderscore )}
\end{itemize}
Pé de verso grego ou latino com uma sýllaba breve entre duas longas.
\section{Amphímeno}
\begin{itemize}
\item {Grp. gram.:m.}
\end{itemize}
Gênero de leguminosas.
\section{Amphimétrico}
\begin{itemize}
\item {Grp. gram.:adj.}
\end{itemize}
\begin{itemize}
\item {Proveniência:(Do gr. \textunderscore amphi\textunderscore  + \textunderscore metron\textunderscore )}
\end{itemize}
Diz-se de qualquer substância mineral, cujos crystaes offerecem incidência igual em certas faces.
\section{Amphimónia}
\begin{itemize}
\item {Grp. gram.:f.}
\end{itemize}
Gênero de leguminosas.
\section{Amphiodonte}
\begin{itemize}
\item {Grp. gram.:m.}
\end{itemize}
Gênero de peixes da América do Norte.
\section{Amphioxo}
\begin{itemize}
\item {fónica:cso}
\end{itemize}
\begin{itemize}
\item {Grp. gram.:m.}
\end{itemize}
Pequeno peixe, sem crânio nem cérebro, que vive occulto nas areias do Mediterrâneo.
\section{Amphipira}
\begin{itemize}
\item {Grp. gram.:f.}
\end{itemize}
Gênero de lepidópteros.
\section{Amphipneustos}
\begin{itemize}
\item {Grp. gram.:m. pl.}
\end{itemize}
\begin{itemize}
\item {Proveniência:(Do gr. \textunderscore amphi\textunderscore  + \textunderscore pneo\textunderscore )}
\end{itemize}
Classe de reptis, com dois apparelhos respiratórios.
\section{Amphípodes}
\begin{itemize}
\item {Grp. gram.:m. pl.}
\end{itemize}
\begin{itemize}
\item {Proveniência:(Do gr. \textunderscore amphi\textunderscore  + \textunderscore pous\textunderscore , \textunderscore podos\textunderscore )}
\end{itemize}
Ordem ou sub-ordem de crustáceos, que têm duas qualidades de pés, para nadar e saltar.
\section{Amphipogão}
\begin{itemize}
\item {Grp. gram.:m.}
\end{itemize}
Gênero de gramíneas.
\section{Amphipróstylo}
\begin{itemize}
\item {Grp. gram.:m.}
\end{itemize}
Diz-se de alguns templos antigos, com duas ordens de columnas, uma na parte anterior e outra na posterior.
(Do gr.)
\section{Amphíptero}
\begin{itemize}
\item {Grp. gram.:m.}
\end{itemize}
\begin{itemize}
\item {Utilização:Heráld.}
\end{itemize}
Dragão ou serpente com asas de morcego.
\section{Amphisarca}
\begin{itemize}
\item {Grp. gram.:m.}
\end{itemize}
Fruto plurilocular, indehiscente, exteriormente duro e interiormente carnudo.
\section{Amphisbena}
\begin{itemize}
\item {Grp. gram.:f.}
\end{itemize}
\begin{itemize}
\item {Utilização:P. us.}
\end{itemize}
\begin{itemize}
\item {Proveniência:(Do gr. \textunderscore amphi\textunderscore  + \textunderscore bainein\textunderscore )}
\end{itemize}
Serpente, que parece têr duas cabeças, uma em cada extremidade.
\section{Amphiscianos}
\begin{itemize}
\item {Grp. gram.:m. pl.}
\end{itemize}
O mesmo que amphíscios.
\section{Amphíscios}
\begin{itemize}
\item {Grp. gram.:m. pl.}
\end{itemize}
\begin{itemize}
\item {Proveniência:(Do gr. \textunderscore amphi\textunderscore  + \textunderscore skia\textunderscore )}
\end{itemize}
Habitantes das regiões equatoriaes, que umas vezes projectam a sua sombra para o norte e outras para o sul, conforme o Sol está abaixo ou acima do Equador.
\section{Amphiscópia}
\begin{itemize}
\item {Grp. gram.:f.}
\end{itemize}
Gênero de acantáceas brasileiras.
\section{Amphistauro}
\begin{itemize}
\item {Grp. gram.:m.}
\end{itemize}
Gênero de coleópteros.
\section{Amphístomos}
\begin{itemize}
\item {Grp. gram.:m. pl.}
\end{itemize}
Gêneros de gusanos intestinaes.
\section{Amphithálamo}
\begin{itemize}
\item {Grp. gram.:m.}
\end{itemize}
\begin{itemize}
\item {Proveniência:(Lat. \textunderscore ampithalamus\textunderscore )}
\end{itemize}
Compartimento, annexo ao quarto da cama, nas habitações gregas e romanas, e em que dormiam ou trabalhavam as escravas.
\section{Amphitheatral}
\begin{itemize}
\item {Grp. gram.:adj.}
\end{itemize}
Que diz respeito a amphitheatro.
\section{Amphitheátrico}
\begin{itemize}
\item {Grp. gram.:adj.}
\end{itemize}
O mesmo que \textunderscore amphitheatral\textunderscore . Cf. Castilho, \textunderscore Fastos\textunderscore , I,
311 e 312.
\section{Amphitheatro}
\begin{itemize}
\item {Grp. gram.:m.}
\end{itemize}
\begin{itemize}
\item {Proveniência:(Gr. \textunderscore amphitheatron\textunderscore )}
\end{itemize}
Antigo edifício oval ou circular, para espectaculos de feras ou gladiadores, e para representações.
Construcção circular, com degraus, nos theatros, nas escolas, etc.
Os espectadores.
\section{Amphitrite}
\begin{itemize}
\item {Grp. gram.:f.}
\end{itemize}
\begin{itemize}
\item {Utilização:Fig.}
\end{itemize}
\begin{itemize}
\item {Proveniência:(Do gr. \textunderscore Amphitrite\textunderscore , n. p.)}
\end{itemize}
Gênero de vermes marinhos.
O mar.
\section{Amphítropo}
\begin{itemize}
\item {Grp. gram.:adj.}
\end{itemize}
\begin{itemize}
\item {Proveniência:(Do gr. \textunderscore amphi\textunderscore  + \textunderscore trepein\textunderscore )}
\end{itemize}
Diz-se, em Botânica, do embryão recurvado, cujas duas extremidades se dirigem ambas para o hilo.
\section{Amphitryão}
\begin{itemize}
\item {Grp. gram.:m.}
\end{itemize}
\begin{itemize}
\item {Proveniência:(Do gr. \textunderscore Amphitruon\textunderscore , n. p.)}
\end{itemize}
O dono da casa, em que se serve jantar lauto a muitos convidados.
Aquelle que paga as despezas de uma comezaina.
\section{Amphitryónio}
\begin{itemize}
\item {Grp. gram.:adj.}
\end{itemize}
Próprio de amphitryões. Cf. \textunderscore Eufrosina\textunderscore , prólogo.
\section{Amphiúma}
\begin{itemize}
\item {Grp. gram.:f.}
\end{itemize}
Gênero de reptis.
\section{Amphodiplopia}
\begin{itemize}
\item {Grp. gram.:f.}
\end{itemize}
\begin{itemize}
\item {Proveniência:(De \textunderscore amphi\textunderscore  + \textunderscore diplonia\textunderscore )}
\end{itemize}
Vício da visão que apresenta os objectos duplicados em ambos os olhos.
\section{Amphodonte}
\begin{itemize}
\item {Grp. gram.:m.}
\end{itemize}
Gênero de leguminosas.
\section{Âmphora}
\begin{itemize}
\item {Grp. gram.:f.}
\end{itemize}
\begin{itemize}
\item {Proveniência:(Lat. \textunderscore amphora\textunderscore )}
\end{itemize}
Vaso grande de duas asas, para líquidos, usado antigamente entre Gregos e Romanos.
Hoje, vaso semelhante àquelle.
Valva de alguns frutos, que se fendem na época da maturação.
Designação ant. do signo de Aquário.
\section{Amphoral}
\begin{itemize}
\item {Grp. gram.:adj.}
\end{itemize}
\begin{itemize}
\item {Utilização:Poét.}
\end{itemize}
Contido em âmphora.
\section{Amphoricidade}
\begin{itemize}
\item {Grp. gram.:f.}
\end{itemize}
\begin{itemize}
\item {Utilização:Med.}
\end{itemize}
Existência do ruído amphórico dentro do peito.
\section{Amphórico}
\begin{itemize}
\item {Grp. gram.:adj.}
\end{itemize}
\begin{itemize}
\item {Proveniência:(De \textunderscore âmphora\textunderscore )}
\end{itemize}
Diz-se do som, que se ouve dentro do peito auscultado, pela semelhança com o som que se obtém, soprando para dentro de uma âmphora vazia.
\section{Amplamente}
\begin{itemize}
\item {Grp. gram.:adv.}
\end{itemize}
Com largueza.
De modo \textunderscore amplo\textunderscore .
\section{Amplectivo}
\begin{itemize}
\item {Grp. gram.:adj.}
\end{itemize}
\begin{itemize}
\item {Proveniência:(Do lat. \textunderscore amplecti\textunderscore )}
\end{itemize}
Diz-se, em botânica, do órgão que abrange outro.
(Melhor se diria \textunderscore amplexivo\textunderscore ).
\section{Amígdala}
\begin{itemize}
\item {Grp. gram.:f.}
\end{itemize}
\begin{itemize}
\item {Proveniência:(Gr. \textunderscore amugdale\textunderscore , amêndoa)}
\end{itemize}
Cada uma das glândulas ovoides, que, em fórma de amêndoa, existem á entrada da garganta.
\section{Amigdaláceas}
\begin{itemize}
\item {Grp. gram.:f. pl.}
\end{itemize}
\begin{itemize}
\item {Proveniência:(De \textunderscore amýgdala\textunderscore )}
\end{itemize}
Tríbo de plantas, da fam. das rosáceas.
\section{Amigdalar}
\begin{itemize}
\item {Grp. gram.:adj.}
\end{itemize}
Diz-se das rochas que são amigdaloides.
\section{Amigdalato}
\begin{itemize}
\item {Grp. gram.:m.}
\end{itemize}
\begin{itemize}
\item {Utilização:Chím.}
\end{itemize}
Sal, resultante da combinação do ácido amigdálico com uma base.
\section{Amigdáleas}
\begin{itemize}
\item {Grp. gram.:f. pl.}
\end{itemize}
O mesmo que \textunderscore amigdaláceas\textunderscore .
\section{Amigdálico}
\begin{itemize}
\item {Grp. gram.:adj.}
\end{itemize}
Diz-se do ácido, em que entra uma solução aquosa de amigdalina.
\section{Amigdalífero}
\begin{itemize}
\item {Grp. gram.:adj.}
\end{itemize}
\begin{itemize}
\item {Proveniência:(Do lat. \textunderscore amygdala\textunderscore  + \textunderscore ferre\textunderscore )}
\end{itemize}
Que apresenta partes, semelhantes a amêndoas.
\section{Amigdalina}
\begin{itemize}
\item {Grp. gram.:f.}
\end{itemize}
\begin{itemize}
\item {Proveniência:(De \textunderscore amýgdala\textunderscore )}
\end{itemize}
Substância, que se extrái das amêndoas amargas.
\section{Amigdalino}
\begin{itemize}
\item {Grp. gram.:adj.}
\end{itemize}
\begin{itemize}
\item {Proveniência:(De \textunderscore amýgdala\textunderscore )}
\end{itemize}
Relativo a amêndoas.
Feito com amêndoas.
\section{Amigdalita}
\begin{itemize}
\item {Grp. gram.:f.}
\end{itemize}
Rocha, que contém seixos arredondados ou em fórma de amêndoa.
\section{Amigdalite}
\begin{itemize}
\item {Grp. gram.:f.}
\end{itemize}
Inflammação nas amígdalas.
\section{Amigdaloide}
\begin{itemize}
\item {Grp. gram.:m.}
\end{itemize}
\begin{itemize}
\item {Proveniência:(Do gr. \textunderscore amugdale\textunderscore  + \textunderscore eidos\textunderscore )}
\end{itemize}
Pedra, que, dentro da própria substância, tem partes em fórma de amêndoa.
\section{Amigdalóforo}
\begin{itemize}
\item {Grp. gram.:m.}
\end{itemize}
\begin{itemize}
\item {Proveniência:(Do gr. \textunderscore amugdale\textunderscore  + \textunderscore phoros\textunderscore )}
\end{itemize}
Designação scientífica da amendoeira.
\section{Amigdalótomo}
\begin{itemize}
\item {Grp. gram.:m.}
\end{itemize}
\begin{itemize}
\item {Proveniência:(Do gr. \textunderscore amugdale\textunderscore  + \textunderscore tome\textunderscore )}
\end{itemize}
Instrumento cirúrgico, para cortar as amígdalas.
\section{Amigdofinina}
\begin{itemize}
\item {Grp. gram.:f.}
\end{itemize}
Producto pharmacêutico, antiséptico, antireumático e antineurálgico.
\section{Amiláceo}
\begin{itemize}
\item {Grp. gram.:adj.}
\end{itemize}
\begin{itemize}
\item {Proveniência:(Do gr. \textunderscore amulon\textunderscore )}
\end{itemize}
Semelhante ao amido.
Que encerra amido.
\section{Amílase}
\begin{itemize}
\item {Grp. gram.:f.}
\end{itemize}
Fermento, que torna solúvel o amido, saccharificando-o.
\section{Amilene}
\begin{itemize}
\item {Grp. gram.:m.}
\end{itemize}
O mesmo que \textunderscore amylênio\textunderscore .
\section{Amilênio}
\begin{itemize}
\item {Grp. gram.:m.}
\end{itemize}
\begin{itemize}
\item {Proveniência:(De \textunderscore âmylo\textunderscore )}
\end{itemize}
Substância anesthésica, incolor e volátil.
\section{Amileno}
\begin{itemize}
\item {Grp. gram.:m.}
\end{itemize}
(V.amilênio)
\section{Amílico}
\begin{itemize}
\item {Grp. gram.:adj.}
\end{itemize}
\begin{itemize}
\item {Proveniência:(De \textunderscore âmylo\textunderscore )}
\end{itemize}
Diz-se de um álcool, de cheiro activo e sabor picante; dos compostos que derivam dêsse álcool, e dos caracteres dêsses compostos.
\section{Âmilo}
\begin{itemize}
\item {Grp. gram.:m.}
\end{itemize}
\begin{itemize}
\item {Proveniência:(Gr. \textunderscore amulon\textunderscore )}
\end{itemize}
Rad. hyp. dos derivados do álcool amílico.
Hydrogênio carbonatado, extrahido do óleo de batata.
\section{Amiloforme}
\begin{itemize}
\item {Grp. gram.:m.}
\end{itemize}
Combinação de aldehido fórmico e amido.
\section{Amiloide}
\begin{itemize}
\item {Grp. gram.:m.}
\end{itemize}
\begin{itemize}
\item {Proveniência:(Do gr. \textunderscore amulon\textunderscore  + \textunderscore eidos\textunderscore )}
\end{itemize}
Variedade de cellulosa, ou princípio vegetal, de que se compõe a parede das céllulas de certos cotylédones.
\section{Amiostenia}
\begin{itemize}
\item {Grp. gram.:f.}
\end{itemize}
Deminuição da fôrça muscular.
\section{Amires}
\begin{itemize}
\item {Grp. gram.:f.}
\end{itemize}
O mesmo ou melhor que \textunderscore amíris\textunderscore .
\section{Amiridáceas}
\begin{itemize}
\item {Grp. gram.:f. pl.}
\end{itemize}
O mesmo que \textunderscore amirídeas\textunderscore .
\section{Amirídeas}
\begin{itemize}
\item {Grp. gram.:f. pl.}
\end{itemize}
\begin{itemize}
\item {Proveniência:(Do gr. \textunderscore amuron\textunderscore  + \textunderscore eidos\textunderscore )}
\end{itemize}
Grupo de plantas dicotyledóneas, que comprehende o amíris.
\section{Amirina}
\begin{itemize}
\item {Grp. gram.:f.}
\end{itemize}
\begin{itemize}
\item {Proveniência:(De \textunderscore amýris\textunderscore )}
\end{itemize}
Substância branca, descoberta por Bonastre em certa goma.
\section{Amíris}
\begin{itemize}
\item {Grp. gram.:f.}
\end{itemize}
\begin{itemize}
\item {Proveniência:(Do gr. \textunderscore amuron\textunderscore )}
\end{itemize}
Gênero de plantas, da fam. das burseráceas.
\section{Amplexicaule}
\begin{itemize}
\item {Grp. gram.:adj.}
\end{itemize}
\begin{itemize}
\item {Proveniência:(De \textunderscore amplexo\textunderscore  + \textunderscore caule\textunderscore )}
\end{itemize}
Diz-se da parte da planta, que abraça o caule.
\section{Amplexifloro}
\begin{itemize}
\item {Grp. gram.:adj.}
\end{itemize}
\begin{itemize}
\item {Proveniência:(De \textunderscore amplexo\textunderscore  + \textunderscore flôr\textunderscore )}
\end{itemize}
Que abraça a flôr, (falando-se de certos órgãos vegetaes).
\section{Amplexifólio}
\begin{itemize}
\item {Grp. gram.:adj.}
\end{itemize}
Diz-se das plantas, cujas fôlhas são amplexicaules.
\section{Amplexo}
\begin{itemize}
\item {Grp. gram.:m.}
\end{itemize}
\begin{itemize}
\item {Proveniência:(Lat. \textunderscore amplexus\textunderscore )}
\end{itemize}
O mesmo que \textunderscore abraço\textunderscore .
\section{Ampliação}
\begin{itemize}
\item {Grp. gram.:f.}
\end{itemize}
Acto ou effeito de \textunderscore ampliar\textunderscore .
\section{Ampliadamente}
\begin{itemize}
\item {Grp. gram.:adv.}
\end{itemize}
Com ampliação.
\section{Ampliador}
\begin{itemize}
\item {Grp. gram.:m.}
\end{itemize}
Aquelle que amplia.
\section{Ampliar}
\begin{itemize}
\item {Grp. gram.:v. t.}
\end{itemize}
\begin{itemize}
\item {Proveniência:(Lat. ampliare)}
\end{itemize}
Tornar amplo.
Alargar.
Estender.
\section{Ampliatifloro}
\begin{itemize}
\item {Grp. gram.:adj.}
\end{itemize}
\begin{itemize}
\item {Proveniência:(Do lat. \textunderscore ampliatus\textunderscore  + \textunderscore flos\textunderscore )}
\end{itemize}
Diz-se da corôa das synanthéreas, quando composta de flôres com corollas amplificadas.
\section{Ampliatiforme}
\begin{itemize}
\item {Grp. gram.:adj.}
\end{itemize}
\begin{itemize}
\item {Utilização:Bot.}
\end{itemize}
\begin{itemize}
\item {Proveniência:(Do lat. \textunderscore ampliatus\textunderscore  + \textunderscore forma\textunderscore )}
\end{itemize}
Que tem grandes dimensões.
\section{Ampliativo}
\begin{itemize}
\item {Grp. gram.:adj.}
\end{itemize}
Que amplia.
Que serve para ampliar.
\section{Ampliável}
\begin{itemize}
\item {Grp. gram.:adj.}
\end{itemize}
Que póde sêr ampliado.
\section{Amplidão}
\begin{itemize}
\item {Grp. gram.:f.}
\end{itemize}
\begin{itemize}
\item {Proveniência:(Lat. \textunderscore amplitudo\textunderscore )}
\end{itemize}
Qualidade do que é amplo.
Largueza.
Extensão.
\section{Amplificação}
\begin{itemize}
\item {Grp. gram.:f.}
\end{itemize}
Acto de \textunderscore amplificar\textunderscore .
\section{Amplificadamente}
\begin{itemize}
\item {Grp. gram.:adv.}
\end{itemize}
Com amplificação.
\section{Amplificador}
\begin{itemize}
\item {Grp. gram.:m.}
\end{itemize}
Aquelle que amplifica.
\section{Amplificar}
\begin{itemize}
\item {Grp. gram.:v. t.}
\end{itemize}
\begin{itemize}
\item {Proveniência:(Do lat. \textunderscore amplificare\textunderscore )}
\end{itemize}
Fazer amplo.
Dilatar.
Tornar maior o que já era grande.
O mesmo que \textunderscore ampliar\textunderscore .
\section{Amplificativo}
\begin{itemize}
\item {Grp. gram.:adj.}
\end{itemize}
O mesmo que \textunderscore ampliativo\textunderscore .
\section{Amplificável}
\begin{itemize}
\item {Grp. gram.:adj.}
\end{itemize}
O mesmo que \textunderscore ampliável\textunderscore .
\section{Amplitude}
\begin{itemize}
\item {Grp. gram.:f.}
\end{itemize}
O mesmo que \textunderscore amplidão\textunderscore .
\section{Amplo}
\begin{itemize}
\item {Grp. gram.:adj.}
\end{itemize}
\begin{itemize}
\item {Proveniência:(Do lat. \textunderscore amplus\textunderscore )}
\end{itemize}
Espaçoso.
Extenso.
Dilatado.
Largo: \textunderscore estrada ampla\textunderscore .
\section{Ampolla}
\begin{itemize}
\item {fónica:pô}
\end{itemize}
\textunderscore f.\textunderscore  (e der.)
O mesmo que \textunderscore empôla\textunderscore ^1, etc.
\section{Amprom}
\begin{itemize}
\item {Grp. gram.:adv.}
\end{itemize}
\begin{itemize}
\item {Utilização:Ant.}
\end{itemize}
Adeante.
Em direitura.
\section{Ampula}
\begin{itemize}
\item {Grp. gram.:f.}
\end{itemize}
\begin{itemize}
\item {Utilização:Ant.}
\end{itemize}
\begin{itemize}
\item {Proveniência:(Lat. \textunderscore ampulla\textunderscore )}
\end{itemize}
Âmbula.
Redoma.
Frasco.
Galheta.
\section{Ampuláceo}
\begin{itemize}
\item {Grp. gram.:adj.}
\end{itemize}
O mesmo que \textunderscore ampular\textunderscore .
\section{Ampular}
\begin{itemize}
\item {Grp. gram.:adj.}
\end{itemize}
Que tem fórma de ampula.
\section{Ampulheta}
\begin{itemize}
\item {fónica:lhê}
\end{itemize}
\begin{itemize}
\item {Grp. gram.:f.}
\end{itemize}
\begin{itemize}
\item {Utilização:Ant.}
\end{itemize}
Relógio de areia.
Pequena âmbula.
(Cast. \textunderscore ampolleta\textunderscore )
\section{Ampulla}
\begin{itemize}
\item {Grp. gram.:f.}
\end{itemize}
\begin{itemize}
\item {Utilização:Ant.}
\end{itemize}
\begin{itemize}
\item {Proveniência:(Lat. \textunderscore ampulla\textunderscore )}
\end{itemize}
Âmbula.
Redoma.
Frasco.
Galheta.
\section{Ampulláceo}
\begin{itemize}
\item {Grp. gram.:adj.}
\end{itemize}
O mesmo que \textunderscore ampullar\textunderscore .
\section{Ampullar}
\begin{itemize}
\item {Grp. gram.:adj.}
\end{itemize}
Que tem fórma de ampulla.
\section{Amputação}
\begin{itemize}
\item {Grp. gram.:f.}
\end{itemize}
Acto de \textunderscore amputar\textunderscore .
\section{Amputar}
\begin{itemize}
\item {Grp. gram.:v. t.}
\end{itemize}
\begin{itemize}
\item {Proveniência:(Lat. \textunderscore amputare\textunderscore )}
\end{itemize}
Cortar (um membro do corpo).
Mutilar.
Eliminar.
Reduzir.
\section{Amsterdamês}
\begin{itemize}
\item {Grp. gram.:m.}
\end{itemize}
Aquelle que é natural de Amsterdão. Cf. Ortigão. \textunderscore Hollanda\textunderscore .
\section{Amuadamente}
\begin{itemize}
\item {Grp. gram.:adv.}
\end{itemize}
Com amúo.
\section{Amuado}
\begin{itemize}
\item {Grp. gram.:adj.}
\end{itemize}
Que tem amúo, que se afasta ou se retrai desgostoso.
Guardado ou enthesoirado sem render, (falando-se de dinheiro ou riqueza):«\textunderscore tem delles (oiro e diamantes), amuados cúmulos\textunderscore ». Filinto, \textunderscore Vida de D. Man.\textunderscore , I, 370.
\section{Amuador}
\begin{itemize}
\item {Grp. gram.:adj.}
\end{itemize}
Que amua facilmente. Cf. Filinto, XIII, 263.
\section{Amuamento}
\begin{itemize}
\item {Grp. gram.:m.}
\end{itemize}
Acto de \textunderscore amuar\textunderscore . Cf. Filinto, XX, 175.
\section{Amuar}
\begin{itemize}
\item {Grp. gram.:v. t.}
\end{itemize}
\begin{itemize}
\item {Grp. gram.:V. i.  e  v. p.}
\end{itemize}
\begin{itemize}
\item {Utilização:Prov.}
\end{itemize}
\begin{itemize}
\item {Utilização:trasm.}
\end{itemize}
Fazer que (alguém) tenha amúo.
Tomar amúo; mostrar, pelo aspecto, gestos ou silêncio, que está escandalizado ou mal disposto.
Baixar de preço.
(Cp. fr. \textunderscore moue\textunderscore )
\section{Amuganhar}
\begin{itemize}
\item {Grp. gram.:v. t.}
\end{itemize}
\begin{itemize}
\item {Utilização:T. do Fundão}
\end{itemize}
Prostrar com pancadas.
Vencer numa contenda.
\section{Amulatado}
\begin{itemize}
\item {Grp. gram.:adj.}
\end{itemize}
Que tem côr de mulato.
\section{Amulatar-se}
\begin{itemize}
\item {Grp. gram.:v. p.}
\end{itemize}
Tomar a côr de mulato.
\section{Amulético}
\begin{itemize}
\item {Grp. gram.:adj.}
\end{itemize}
Relativo a amuletos.
\section{Amuleto}
\begin{itemize}
\item {fónica:lê}
\end{itemize}
\begin{itemize}
\item {Grp. gram.:m.}
\end{itemize}
\begin{itemize}
\item {Proveniência:(Lat. \textunderscore amuletum\textunderscore )}
\end{itemize}
Objecto, que os supersticiosos julgam desviar malefícios e desgraças, em o trazendo consigo.
Talisman.
Preservativo.
\section{Amulherar-se}
\begin{itemize}
\item {Grp. gram.:v. p.}
\end{itemize}
Tomar modos de mulher; effeminar-se.
\section{Amulherengado}
\begin{itemize}
\item {Grp. gram.:adj.}
\end{itemize}
Que tem modos de mulher; effeminado.
\section{Amulherengar-se}
\begin{itemize}
\item {Grp. gram.:v. p.}
\end{itemize}
Fazer-se mulherengo.
\section{Amumiar}
\begin{itemize}
\item {Grp. gram.:v. t.}
\end{itemize}
Tornar semelhante a múmias.
\section{Amunicionar}
\begin{itemize}
\item {Grp. gram.:v. t.}
\end{itemize}
Prover de munições. Cf. Filinto, \textunderscore Vida de D. Man.\textunderscore , I, 389.
\section{Amúo}
\begin{itemize}
\item {Grp. gram.:m.}
\end{itemize}
\begin{itemize}
\item {Proveniência:(De \textunderscore amuar\textunderscore )}
\end{itemize}
Enfado, mau humor, traduzido no aspecto, silêncio ou gestos.
\section{Amura}
\begin{itemize}
\item {Grp. gram.:f.}
\end{itemize}
\begin{itemize}
\item {Utilização:Náut.}
\end{itemize}
Cabo, com que se mareiam os papafigos e as velas menores, cutellos e varredoiras.
Lugar á prôa, a um e outro bordo, onde ficam os paus da amura.
\section{Amurada}
\begin{itemize}
\item {Grp. gram.:f.}
\end{itemize}
\begin{itemize}
\item {Utilização:Náut.}
\end{itemize}
Prolongamento do costado do navio, acima do pavimento superior.
\section{Amurada}
\begin{itemize}
\item {Grp. gram.:f.}
\end{itemize}
Muro, parede. (Accepção, inventada por Camillo, \textunderscore Myst. de Lisb.\textunderscore , I, 157)
\section{Amurado}
\begin{itemize}
\item {Grp. gram.:adj.}
\end{itemize}
Diz-se que um navio está \textunderscore amurado\textunderscore  a bombordo ou a estibordo, segundo a orientação que toma, para receber o vento da direita ou da esquerda.
\section{Amuralhar}
\begin{itemize}
\item {Grp. gram.:v. t.}
\end{itemize}
Cercar de muralhas.
\section{Amurar}
\begin{itemize}
\item {Grp. gram.:v. t.  e  i.}
\end{itemize}
Retesar; prender a amura.
\section{Amurar}
\begin{itemize}
\item {Grp. gram.:v. t.}
\end{itemize}
O mesmo que \textunderscore amuralhar\textunderscore .
\section{Amurca}
\begin{itemize}
\item {Grp. gram.:f.}
\end{itemize}
\begin{itemize}
\item {Proveniência:(Lat. \textunderscore amurca\textunderscore )}
\end{itemize}
Almofeira, água escura, que escorre da talha das azeitonas.
\section{Amurchecer-se}
\begin{itemize}
\item {Grp. gram.:v. p.}
\end{itemize}
Murchar. Cf. Usque, \textunderscore Tribulações\textunderscore , 22.
\section{Amurilhar-se}
\begin{itemize}
\item {Grp. gram.:v. p.}
\end{itemize}
\begin{itemize}
\item {Utilização:Prov.}
\end{itemize}
\begin{itemize}
\item {Utilização:trasm.}
\end{itemize}
Acantonar-se, amesendar-se em qualquer canto.
Retrahir-se silencioso.
(Cp. \textunderscore morilho\textunderscore )
\section{Amurujar}
\begin{itemize}
\item {Grp. gram.:v. t.}
\end{itemize}
\begin{itemize}
\item {Utilização:Ant.}
\end{itemize}
\begin{itemize}
\item {Proveniência:(De \textunderscore murugem\textunderscore )}
\end{itemize}
Regar.
Cobrir de água; limar (o terreno).
\section{Amusia}
\begin{itemize}
\item {Grp. gram.:f.}
\end{itemize}
\begin{itemize}
\item {Proveniência:(Do gr. \textunderscore a\textunderscore  priv. + \textunderscore mousa\textunderscore )}
\end{itemize}
Perda completa ou parcial da faculdade musical.
\section{Amuso}
\begin{itemize}
\item {Grp. gram.:adj.}
\end{itemize}
\begin{itemize}
\item {Utilização:Neol.}
\end{itemize}
\begin{itemize}
\item {Proveniência:(De \textunderscore a\textunderscore  priv. + \textunderscore musa\textunderscore )}
\end{itemize}
Contrário ás musas.
\section{Amýgdala}
\begin{itemize}
\item {Grp. gram.:f.}
\end{itemize}
\begin{itemize}
\item {Proveniência:(Gr. \textunderscore amugdale\textunderscore , amêndoa)}
\end{itemize}
Cada uma das glândulas ovoides, que, em fórma de amêndoa, existem á entrada da garganta.
\section{Amygdaláceas}
\begin{itemize}
\item {Grp. gram.:f. pl.}
\end{itemize}
\begin{itemize}
\item {Proveniência:(De \textunderscore amýgdala\textunderscore )}
\end{itemize}
Tríbo de plantas, da fam. das rosáceas.
\section{Amygdalar}
\begin{itemize}
\item {Grp. gram.:adj.}
\end{itemize}
Diz-se das rochas que são amygdaloides.
\section{Amygdalato}
\begin{itemize}
\item {Grp. gram.:m.}
\end{itemize}
\begin{itemize}
\item {Utilização:Chím.}
\end{itemize}
Sal, resultante da combinação do ácido amygdálico com uma base.
\section{Amygdáleas}
\begin{itemize}
\item {Grp. gram.:f. pl.}
\end{itemize}
O mesmo que \textunderscore amygdaláceas\textunderscore .
\section{Amygdálico}
\begin{itemize}
\item {Grp. gram.:adj.}
\end{itemize}
Diz-se do ácido, em que entra uma solução aquosa de amygdalina.
\section{Amygdalífero}
\begin{itemize}
\item {Grp. gram.:adj.}
\end{itemize}
\begin{itemize}
\item {Proveniência:(Do lat. \textunderscore amygdala\textunderscore  + \textunderscore ferre\textunderscore )}
\end{itemize}
Que apresenta partes, semelhantes a amêndoas.
\section{Amygdalina}
\begin{itemize}
\item {Grp. gram.:f.}
\end{itemize}
\begin{itemize}
\item {Proveniência:(De \textunderscore amýgdala\textunderscore )}
\end{itemize}
Substância, que se extrái das amêndoas amargas.
\section{Amygdalino}
\begin{itemize}
\item {Grp. gram.:adj.}
\end{itemize}
\begin{itemize}
\item {Proveniência:(De \textunderscore amýgdala\textunderscore )}
\end{itemize}
Relativo a amêndoas.
Feito com amêndoas.
\section{Amygdalita}
\begin{itemize}
\item {Grp. gram.:f.}
\end{itemize}
Rocha, que contém seixos arredondados ou em fórma de amêndoa.
\section{Amygdalite}
\begin{itemize}
\item {Grp. gram.:f.}
\end{itemize}
Inflammação nas amýgdalas.
\section{Amygdaloide}
\begin{itemize}
\item {Grp. gram.:m.}
\end{itemize}
\begin{itemize}
\item {Proveniência:(Do gr. \textunderscore amugdale\textunderscore  + \textunderscore eidos\textunderscore )}
\end{itemize}
Pedra, que, dentro da própria substância, tem partes em fórma de amêndoa.
\section{Amygdalóphoro}
\begin{itemize}
\item {Grp. gram.:m.}
\end{itemize}
\begin{itemize}
\item {Proveniência:(Do gr. \textunderscore amugdale\textunderscore  + \textunderscore phoros\textunderscore )}
\end{itemize}
Designação scientífica da amendoeira.
\section{Amygdalótomo}
\begin{itemize}
\item {Grp. gram.:m.}
\end{itemize}
\begin{itemize}
\item {Proveniência:(Do gr. \textunderscore amugdale\textunderscore  + \textunderscore tome\textunderscore )}
\end{itemize}
Instrumento cirúrgico, para cortar as amýgdalas.
\section{Amygdophinina}
\begin{itemize}
\item {Grp. gram.:f.}
\end{itemize}
Producto pharmacêutico, antiséptico, antireumático e antineurálgico.
\section{Amyláceo}
\begin{itemize}
\item {Grp. gram.:adj.}
\end{itemize}
\begin{itemize}
\item {Proveniência:(Do gr. \textunderscore amulon\textunderscore )}
\end{itemize}
Semelhante ao amido.
Que encerra amido.
\section{Amýlase}
\begin{itemize}
\item {Grp. gram.:f.}
\end{itemize}
Fermento, que torna solúvel o amido, saccharificando-o.
\section{Amylene}
\begin{itemize}
\item {Grp. gram.:m.}
\end{itemize}
O mesmo que \textunderscore amylênio\textunderscore .
\section{Amylênio}
\begin{itemize}
\item {Grp. gram.:m.}
\end{itemize}
\begin{itemize}
\item {Proveniência:(De \textunderscore âmylo\textunderscore )}
\end{itemize}
Substância anesthésica, incolor e volátil.
\section{Amyleno}
\begin{itemize}
\item {Grp. gram.:m.}
\end{itemize}
(V.amylênio)
\section{Amýlico}
\begin{itemize}
\item {Grp. gram.:adj.}
\end{itemize}
\begin{itemize}
\item {Proveniência:(De \textunderscore âmylo\textunderscore )}
\end{itemize}
Diz-se de um álcool, de cheiro activo e sabor picante; dos compostos que derivam dêsse álcool, e dos caracteres dêsses compostos.
\section{Âmylo}
\begin{itemize}
\item {Grp. gram.:m.}
\end{itemize}
\begin{itemize}
\item {Proveniência:(Gr. \textunderscore amulon\textunderscore )}
\end{itemize}
Rad. hyp. dos derivados do álcool amýlico.
Hydrogênio carbonatado, extrahido do óleo de batata.
\section{Amyloforme}
\begin{itemize}
\item {Grp. gram.:m.}
\end{itemize}
Combinação de aldehido fórmico e amido.
\section{Amyloide}
\begin{itemize}
\item {Grp. gram.:m.}
\end{itemize}
\begin{itemize}
\item {Proveniência:(Do gr. \textunderscore amulon\textunderscore  + \textunderscore eidos\textunderscore )}
\end{itemize}
Variedade de cellulosa, ou princípio vegetal, de que se compõe a parede das céllulas de certos cotylédones.
\section{Amyosthenia}
\begin{itemize}
\item {Grp. gram.:f.}
\end{itemize}
Deminuição da fôrça muscular.
\section{Amyres}
\begin{itemize}
\item {Grp. gram.:f.}
\end{itemize}
O mesmo ou melhor que \textunderscore amýris\textunderscore .
\section{Amyridáceas}
\begin{itemize}
\item {Grp. gram.:f. pl.}
\end{itemize}
O mesmo que \textunderscore amyrídeas\textunderscore .
\section{Amyrídeas}
\begin{itemize}
\item {Grp. gram.:f. pl.}
\end{itemize}
\begin{itemize}
\item {Proveniência:(Do gr. \textunderscore amuron\textunderscore  + \textunderscore eidos\textunderscore )}
\end{itemize}
Grupo de plantas dicotyledóneas, que comprehende o amýris.
\section{Amyrina}
\begin{itemize}
\item {Grp. gram.:f.}
\end{itemize}
\begin{itemize}
\item {Proveniência:(De \textunderscore amýris\textunderscore )}
\end{itemize}
Substância branca, descoberta por Bonastre em certa goma.
\section{Amýris}
\begin{itemize}
\item {Grp. gram.:f.}
\end{itemize}
\begin{itemize}
\item {Proveniência:(Do gr. \textunderscore amuron\textunderscore )}
\end{itemize}
Gênero de plantas, da fam. das burseráceas.
\section{An...}
\begin{itemize}
\item {Grp. gram.:pref.}
\end{itemize}
\begin{itemize}
\item {Proveniência:(Do gr. \textunderscore an\textunderscore )}
\end{itemize}
(design. de privação ou negação)
\section{...an}
\textunderscore suf.\textunderscore  \textunderscore f.\textunderscore  de alguns \textunderscore s.\textunderscore  e \textunderscore adj.\textunderscore , cuja terminação masculina é \textunderscore ão\textunderscore .
\section{Ana}
\begin{itemize}
\item {Grp. gram.:f.}
\end{itemize}
(V.alna)
\section{Aná}
\begin{itemize}
\item {Grp. gram.:m.}
\end{itemize}
Moeda de prata da Índia inglesa, correspondente á 16.^a parte da rupia.
\section{Aná}
\begin{itemize}
\item {Grp. gram.:adv.}
\end{itemize}
\begin{itemize}
\item {Utilização:Pharm.}
\end{itemize}
Em partes iguaes.--Us. em receitas.
\section{Anabaptismo}
\begin{itemize}
\item {Grp. gram.:m.}
\end{itemize}
Doutrina dos anabaptistas.
\section{Anabaptista}
\begin{itemize}
\item {Grp. gram.:m.}
\end{itemize}
\begin{itemize}
\item {Proveniência:(Do gr. \textunderscore ana\textunderscore  + \textunderscore baptistes\textunderscore )}
\end{itemize}
Membro de uma seita religiosa, que prègava a repetição do baptismo para quem o tivesse recebido antes do uso da razão.
\section{Anabasa}
\begin{itemize}
\item {Grp. gram.:f.}
\end{itemize}
\begin{itemize}
\item {Proveniência:(Do gr. \textunderscore anabasis\textunderscore )}
\end{itemize}
Pequena planta da Espanha e das margens do Cáspio.
\section{Anabase}
\begin{itemize}
\item {Grp. gram.:m.}
\end{itemize}
O mesmo ou melhor que \textunderscore anabasa\textunderscore .
\section{Anabáseas}
\begin{itemize}
\item {Grp. gram.:f. pl.}
\end{itemize}
\begin{itemize}
\item {Proveniência:(De \textunderscore anabase\textunderscore )}
\end{itemize}
Sub-tribo de plantas salsoláceas.
\section{Anabata}
\begin{itemize}
\item {Grp. gram.:m.}
\end{itemize}
\begin{itemize}
\item {Proveniência:(Gr. \textunderscore anabates\textunderscore )}
\end{itemize}
Cavalleiro, que, nos jogos olýmpicos, disputava o prêmio com dois cavallos.
\section{Anabato}
\begin{itemize}
\item {Grp. gram.:m.}
\end{itemize}
\begin{itemize}
\item {Proveniência:(Do gr. \textunderscore anabates\textunderscore )}
\end{itemize}
Pássaro tenuirostro.
\section{Anabenodáctilo}
\begin{itemize}
\item {Grp. gram.:adj.}
\end{itemize}
\begin{itemize}
\item {Proveniência:(Do gr. \textunderscore anabaino\textunderscore  + \textunderscore daktulos\textunderscore )}
\end{itemize}
Diz-se dos animaes, que tem os dedos conformados para trepar.
\section{Anabenodáctylo}
\begin{itemize}
\item {Grp. gram.:adj.}
\end{itemize}
\begin{itemize}
\item {Proveniência:(Do gr. \textunderscore anabaino\textunderscore  + \textunderscore daktulos\textunderscore )}
\end{itemize}
Diz-se dos animaes, que tem os dedos conformados para trepar.
\section{Anabenosáurios}
\begin{itemize}
\item {fónica:sau}
\end{itemize}
\begin{itemize}
\item {Grp. gram.:m. pl.}
\end{itemize}
\begin{itemize}
\item {Proveniência:(Do gr. \textunderscore anabaino\textunderscore  + \textunderscore saura\textunderscore )}
\end{itemize}
Família de reptis sáurios, que sobem ás árvores.
\section{Anabenossáurios}
\begin{itemize}
\item {Grp. gram.:m. pl.}
\end{itemize}
\begin{itemize}
\item {Proveniência:(Do gr. \textunderscore anabaino\textunderscore  + \textunderscore saura\textunderscore )}
\end{itemize}
Família de reptis sáurios, que sobem ás árvores.
\section{Anabi}
\begin{itemize}
\item {Grp. gram.:m.}
\end{itemize}
Planta brasileira, resínosa e amarga.
\section{Anablastemo}
\begin{itemize}
\item {Grp. gram.:m.}
\end{itemize}
Producção especial da folhagem de alguns lichens.
\section{Anabólia}
\begin{itemize}
\item {Grp. gram.:f.}
\end{itemize}
Gênero de insectos.
\section{Anabolismo}
\begin{itemize}
\item {Grp. gram.:m.}
\end{itemize}
\begin{itemize}
\item {Utilização:Physiol.}
\end{itemize}
Conjunto dos phenómenos de sýnthese orgânica, o contrário de \textunderscore catabolismo\textunderscore .
\section{Anabrochismo}
\begin{itemize}
\item {fónica:quis}
\end{itemize}
\begin{itemize}
\item {Grp. gram.:m.}
\end{itemize}
\begin{itemize}
\item {Proveniência:(Gr. \textunderscore anabrokhismos\textunderscore )}
\end{itemize}
Operação, que se imaginou antigamente, para remediar, por meio de uma agulha e de um cabello, o reviramento dos cílios para o globo do ôlho.
\section{Anabroquismo}
\begin{itemize}
\item {Grp. gram.:m.}
\end{itemize}
\begin{itemize}
\item {Proveniência:(Gr. \textunderscore anabrokhismos\textunderscore )}
\end{itemize}
Operação, que se imaginou antigamente, para remediar, por meio de uma agulha e de um cabello, o reviramento dos cílios para o globo do ôlho.
\section{Anabrose}
\begin{itemize}
\item {Grp. gram.:f.}
\end{itemize}
\begin{itemize}
\item {Proveniência:(Gr. \textunderscore anabrosis\textunderscore )}
\end{itemize}
Corrosão das partes sólidas do organismo animal por um humor acre.
\section{Anabrótico}
\begin{itemize}
\item {Grp. gram.:adj.}
\end{itemize}
\begin{itemize}
\item {Utilização:Ant.}
\end{itemize}
\begin{itemize}
\item {Proveniência:(Gr. \textunderscore anabrotikos\textunderscore )}
\end{itemize}
Corrosivo.
\section{Anacá}
\begin{itemize}
\item {Grp. gram.:m.}
\end{itemize}
\begin{itemize}
\item {Proveniência:(Do gr. \textunderscore anax\textunderscore )}
\end{itemize}
Papagaio do Brasil, de côres vivas e variadas.
\section{Anacâmpsida}
\begin{itemize}
\item {Grp. gram.:f.}
\end{itemize}
\begin{itemize}
\item {Proveniência:(Do gr. \textunderscore anakampsis\textunderscore )}
\end{itemize}
Insecto lepidóptero nocturno.
\section{Anacâmptico}
\begin{itemize}
\item {Grp. gram.:adj.}
\end{itemize}
\begin{itemize}
\item {Proveniência:(Do gr. \textunderscore ana\textunderscore  + \textunderscore kamptein\textunderscore )}
\end{itemize}
Que reflecte a luz ou o som.
Produzido pela reflexão da luz sôbre uma linha ou sôbre uma superfície, (falando-se de curvas geométricas).
\section{Anacâmptida}
\begin{itemize}
\item {Grp. gram.:f.}
\end{itemize}
\begin{itemize}
\item {Proveniência:(Do gr. \textunderscore anakamptein\textunderscore )}
\end{itemize}
Planta, da fam. das orchídeas.
\section{Anacampto}
\begin{itemize}
\item {Grp. gram.:m.}
\end{itemize}
\begin{itemize}
\item {Proveniência:(Gr. \textunderscore anakamptos\textunderscore )}
\end{itemize}
Série de notas descendentes, na melopeia grega.
\section{Anacan}
\begin{itemize}
\item {Grp. gram.:m.}
\end{itemize}
O mesmo que \textunderscore anacá\textunderscore .
\section{Anacandaia}
\begin{itemize}
\item {Grp. gram.:f.}
\end{itemize}
Grande serpente de Surinam, espécie de bôa, que atinge 10 metros de comprimento.
\section{Anacandé}
\begin{itemize}
\item {Grp. gram.:m.}
\end{itemize}
Serpente muito delgada de Madagáscar.
(Provavelmente outra fórma de \textunderscore anacandaia\textunderscore ).
\section{Anacantho}
\begin{itemize}
\item {Grp. gram.:m.}
\end{itemize}
Gênero de peixes esclerodermos, (\textunderscore anacanthus barbatus\textunderscore , Gray).
\section{Anacanto}
\begin{itemize}
\item {Grp. gram.:m.}
\end{itemize}
Gênero de peixes esclerodermos, (\textunderscore anacanthus barbatus\textunderscore , Gray).
\section{Anaçar}
\begin{itemize}
\item {Grp. gram.:v. t.}
\end{itemize}
Revolver, misturar (líquidos).
\section{Anacarado}
\begin{itemize}
\item {Grp. gram.:adj.}
\end{itemize}
\begin{itemize}
\item {Proveniência:(De \textunderscore nácar\textunderscore )}
\end{itemize}
Cheio de rubor, ruborizado. Us. por Camillo.
\section{Anacardeáceas}
\begin{itemize}
\item {Grp. gram.:f.}
\end{itemize}
\begin{itemize}
\item {Proveniência:(De \textunderscore anacardeáceo\textunderscore )}
\end{itemize}
Família de plantas, que tem por typo o anacardo.
\section{Anacardeáceo}
\begin{itemize}
\item {Grp. gram.:adj.}
\end{itemize}
Relativo ou semelhante ao anacardo.
\section{Anacárdeas}
\begin{itemize}
\item {Grp. gram.:f. pl.}
\end{itemize}
O mesmo que \textunderscore anacardeáceas\textunderscore .
\section{Anacardeiro}
\begin{itemize}
\item {Grp. gram.:m.}
\end{itemize}
Árvore oriental, que produz o anacardo.
\section{Anacardina}
\begin{itemize}
\item {Grp. gram.:f.}
\end{itemize}
\begin{itemize}
\item {Proveniência:(De \textunderscore anacardino\textunderscore )}
\end{itemize}
Conserva de anacardos.
\section{Anacardino}
\begin{itemize}
\item {Grp. gram.:adj.}
\end{itemize}
Relativo ao anacardo.
\section{Anacardita}
\begin{itemize}
\item {Grp. gram.:f.}
\end{itemize}
O mesmo que \textunderscore anacardite\textunderscore .
\section{Anacardite}
\begin{itemize}
\item {Grp. gram.:f.}
\end{itemize}
\begin{itemize}
\item {Proveniência:(De \textunderscore anacardo\textunderscore )}
\end{itemize}
Fóssil argilloso, em fórma de coração.
\section{Anacardo}
\begin{itemize}
\item {Grp. gram.:m.}
\end{itemize}
\begin{itemize}
\item {Proveniência:(Do gr. \textunderscore ana\textunderscore  + \textunderscore kardia\textunderscore )}
\end{itemize}
Fruto achatado, quási preto e brilhante, do feitio de coração, e também conhecido por \textunderscore fava de Malaca\textunderscore .
A árvore que o produz.
\section{Anacatarcia}
\begin{itemize}
\item {Grp. gram.:f.}
\end{itemize}
O mesmo que \textunderscore expectoração\textunderscore .
\section{Anacatártico}
\begin{itemize}
\item {Grp. gram.:adj.}
\end{itemize}
\begin{itemize}
\item {Proveniência:(Do gr. \textunderscore anakathartikus\textunderscore )}
\end{itemize}
Que promove a expectoração.
\section{Anacatharcia}
\begin{itemize}
\item {Grp. gram.:f.}
\end{itemize}
O mesmo que \textunderscore expectoração\textunderscore .
\section{Anacathártico}
\begin{itemize}
\item {Grp. gram.:adj.}
\end{itemize}
\begin{itemize}
\item {Proveniência:(Do gr. \textunderscore anakathartikus\textunderscore )}
\end{itemize}
Que promove a expectoração.
\section{Anacefaleóse}
\begin{itemize}
\item {Grp. gram.:f.}
\end{itemize}
\begin{itemize}
\item {Proveniência:(Do gr. \textunderscore anakephalaiosis\textunderscore )}
\end{itemize}
O mesmo que \textunderscore recapitulação\textunderscore .
\section{Anacephaleóse}
\begin{itemize}
\item {Grp. gram.:f.}
\end{itemize}
\begin{itemize}
\item {Proveniência:(Do gr. \textunderscore anakephalaiosis\textunderscore )}
\end{itemize}
O mesmo que \textunderscore recapitulação\textunderscore .
\section{Anachrónico}
\begin{itemize}
\item {Grp. gram.:adj.}
\end{itemize}
\begin{itemize}
\item {Proveniência:(Do gr. \textunderscore ana\textunderscore  + \textunderscore kronos\textunderscore )}
\end{itemize}
Opposto á chronologia.
Contrário aos usos da época a que se refere.
Avêsso aos costumes de hoje.
\section{Anachronismo}
\begin{itemize}
\item {Grp. gram.:m.}
\end{itemize}
Êrro de data, ou facto \textunderscore anachrónico\textunderscore .
\section{Anachronizar}
\begin{itemize}
\item {Grp. gram.:v. t.}
\end{itemize}
\begin{itemize}
\item {Proveniência:(Do gr. \textunderscore ana\textunderscore  + \textunderscore khronizein\textunderscore )}
\end{itemize}
Referir, commetendo anachronismo.
\section{Anacíclico}
\begin{itemize}
\item {Grp. gram.:m.  e  adj.}
\end{itemize}
Diz-se do verso que apresenta o mesmo sentido, quer se leia naturalmente, quer ás avessas.
\section{Anaciclo}
\begin{itemize}
\item {Grp. gram.:m.}
\end{itemize}
Gênero de plantas compostas.
(Contr. de \textunderscore ananthocyclo\textunderscore , do gr. an priv. + \textunderscore anthos\textunderscore  + \textunderscore kuklos\textunderscore )
\section{Anacinema}
\begin{itemize}
\item {Grp. gram.:f.}
\end{itemize}
\begin{itemize}
\item {Utilização:Med.}
\end{itemize}
\begin{itemize}
\item {Proveniência:(Do gr. \textunderscore ana\textunderscore  priv. + \textunderscore kinema\textunderscore , movimento)}
\end{itemize}
Prostração de fôrças, em consequência de exercícios gymnásticos.
\section{Anacistos}
\begin{itemize}
\item {Grp. gram.:m. pl.}
\end{itemize}
\begin{itemize}
\item {Proveniência:(Do gr. \textunderscore ana\textunderscore  + \textunderscore kustis\textunderscore )}
\end{itemize}
Gênero de algas terrestres.
\section{Anaclase}
\begin{itemize}
\item {Grp. gram.:f.}
\end{itemize}
\begin{itemize}
\item {Proveniência:(Do gr. \textunderscore anaklasis\textunderscore )}
\end{itemize}
Inflexão articular.
Desvio, refracção, da luz.
\section{Anaclástica}
\begin{itemize}
\item {Grp. gram.:f.}
\end{itemize}
O mesmo que \textunderscore dióptrica\textunderscore .
\section{Anaclástico}
\begin{itemize}
\item {Grp. gram.:adj.}
\end{itemize}
\begin{itemize}
\item {Proveniência:(Do gr. \textunderscore ana\textunderscore  + \textunderscore klaein\textunderscore )}
\end{itemize}
Relativo á refracção da luz.
\section{Anaclintério}
\begin{itemize}
\item {Grp. gram.:m.}
\end{itemize}
\begin{itemize}
\item {Proveniência:(Gr. \textunderscore anaklinterion\textunderscore )}
\end{itemize}
Espécie de canapé ou de marquesa, entre os antigos.
\section{Anaclisia}
\begin{itemize}
\item {Grp. gram.:f.}
\end{itemize}
\begin{itemize}
\item {Proveniência:(Do gr. \textunderscore klinein\textunderscore )}
\end{itemize}
Posição horizontal de um doente na cama ou em cadeira inclinada.
\section{Anaco}
\begin{itemize}
\item {Grp. gram.:m.}
\end{itemize}
\begin{itemize}
\item {Utilização:Prov.}
\end{itemize}
\begin{itemize}
\item {Utilização:minh.}
\end{itemize}
O mesmo que \textunderscore naco\textunderscore .
\section{Anaçoado}
\begin{itemize}
\item {Grp. gram.:adj.}
\end{itemize}
\begin{itemize}
\item {Utilização:Prov.}
\end{itemize}
\begin{itemize}
\item {Utilização:trasm.}
\end{itemize}
Diz-se de um cavallo manso ou dócil; e diz-se do indivíduo bondoso, bonacheirão.
\section{Anácola}
\begin{itemize}
\item {Grp. gram.:f.}
\end{itemize}
Gênero de insectos coleópteros.
\section{Anacolupa}
\begin{itemize}
\item {Grp. gram.:f.}
\end{itemize}
Planta indiana, applicada contra a epilepsia.
\section{Anacolupo}
\begin{itemize}
\item {Grp. gram.:m.}
\end{itemize}
O mesmo que \textunderscore anacoluppa\textunderscore .
\section{Anacoluppa}
\begin{itemize}
\item {Grp. gram.:f.}
\end{itemize}
Planta indiana, applicada contra a epilepsia.
\section{Anacoluppo}
\begin{itemize}
\item {Grp. gram.:m.}
\end{itemize}
O mesmo que \textunderscore anacoluppa\textunderscore .
\section{Anacoluthia}
\begin{itemize}
\item {Grp. gram.:f.}
\end{itemize}
O mesmo que anacolutho.
\section{Anacolutho}
\begin{itemize}
\item {Grp. gram.:m.}
\end{itemize}
\begin{itemize}
\item {Utilização:Gram.}
\end{itemize}
\begin{itemize}
\item {Proveniência:(Gr. \textunderscore anakolouthos\textunderscore )}
\end{itemize}
Ellipse, em que se emprega um relativo, sem o seu antecente.
\section{Anacolutia}
\begin{itemize}
\item {Grp. gram.:f.}
\end{itemize}
O mesmo que \textunderscore anacoluto\textunderscore .
\section{Anacoluto}
\begin{itemize}
\item {Grp. gram.:m.}
\end{itemize}
\begin{itemize}
\item {Utilização:Gram.}
\end{itemize}
\begin{itemize}
\item {Proveniência:(Gr. \textunderscore anakolouthos\textunderscore )}
\end{itemize}
Ellipse, em que se emprega um relativo, sem o seu antecente.
\section{Anaconda}
\begin{itemize}
\item {Grp. gram.:f.}
\end{itemize}
Grande serpente da América do Sul, que chega a têr mais de dez metros de comprimento.
\section{Anacoreta}
\begin{itemize}
\item {fónica:co-rê}
\end{itemize}
\begin{itemize}
\item {Grp. gram.:m.}
\end{itemize}
\begin{itemize}
\item {Proveniência:(Lat. \textunderscore anachoreta\textunderscore )}
\end{itemize}
O religioso ou penitente, que vive na solidão.
Aquelle que vive insulado de relações sociaes.
\section{Anacoreticamente}
\begin{itemize}
\item {Grp. gram.:adv.}
\end{itemize}
Á maneira de anacoreta.
De modo \textunderscore anacorético\textunderscore .
\section{Anacorético}
\begin{itemize}
\item {Grp. gram.:adj.}
\end{itemize}
Relativo a anacoreta.
\section{Anacoretismo}
\begin{itemize}
\item {Grp. gram.:m.}
\end{itemize}
Systema de vida, seguido pelos anacoretas.
\section{Anacorita}
\begin{itemize}
\item {Grp. gram.:m.}
\end{itemize}
(Fórma ant. de \textunderscore anacoreta\textunderscore ).
\section{Anacreôntica}
\begin{itemize}
\item {Grp. gram.:f.}
\end{itemize}
Composição poética de Anacreonte.
Poesia acommodada ao gôsto de Anacreonte.
\section{Anacreôntico}
\begin{itemize}
\item {Grp. gram.:adj.}
\end{itemize}
\begin{itemize}
\item {Proveniência:(Lat. \textunderscore anacreonticus\textunderscore )}
\end{itemize}
Relativo a Anacreonte; que é do gênero ou gôsto das poesias de Anacreonte.
\section{Anacreontismo}
\begin{itemize}
\item {Grp. gram.:m.}
\end{itemize}
\begin{itemize}
\item {Proveniência:(De \textunderscore Anacreonte\textunderscore , n.p.)}
\end{itemize}
Imitação do gênero poético, adoptado por Anacreonte.
\section{Anacrónico}
\begin{itemize}
\item {Grp. gram.:adj.}
\end{itemize}
\begin{itemize}
\item {Proveniência:(Do gr. \textunderscore ana\textunderscore  + \textunderscore kronos\textunderscore )}
\end{itemize}
Opposto á cronologia.
Contrário aos usos da época a que se refere.
Avêsso aos costumes de hoje.
\section{Anacronismo}
\begin{itemize}
\item {Grp. gram.:m.}
\end{itemize}
Êrro de data, ou facto \textunderscore anacrónico\textunderscore .
\section{Anacruse}
\begin{itemize}
\item {Grp. gram.:f.}
\end{itemize}
\begin{itemize}
\item {Utilização:Mús.}
\end{itemize}
\begin{itemize}
\item {Proveniência:(Do gr. \textunderscore ana\textunderscore , para cima, + \textunderscore kruo\textunderscore , eu bato)}
\end{itemize}
Nota ou notas, com que começa uma melodia, se esta começa no tempo fraco do compasso, como preparação para o tempo forte, em que se realiza o \textunderscore ictus\textunderscore  inicial do rythmo.
\section{Anacrústico}
\begin{itemize}
\item {Grp. gram.:adj.}
\end{itemize}
\begin{itemize}
\item {Utilização:Mús.}
\end{itemize}
Diz-se do rythmo que começa por anacruse.
\section{Anactesia}
\begin{itemize}
\item {Grp. gram.:f.}
\end{itemize}
\begin{itemize}
\item {Proveniência:(Do gr. \textunderscore anaktizein\textunderscore )}
\end{itemize}
O mesmo que \textunderscore convalescença\textunderscore .
\section{Anactésico}
\begin{itemize}
\item {Grp. gram.:adj.}
\end{itemize}
Relativo á \textunderscore anactesia\textunderscore .
Que restaura as fôrças.
\section{Anacýclico}
\begin{itemize}
\item {Grp. gram.:m.  e  adj.}
\end{itemize}
Diz-se do verso que apresenta o mesmo sentido, quer se leia naturalmente, quer ás avessas.
\section{Anacyclo}
\begin{itemize}
\item {Grp. gram.:m.}
\end{itemize}
Gênero de plantas compostas.
(Contr. de \textunderscore ananthocyclo\textunderscore , do gr. an priv. + \textunderscore anthos\textunderscore  + \textunderscore kuklos\textunderscore )
\section{Anacystos}
\begin{itemize}
\item {Grp. gram.:m. pl.}
\end{itemize}
\begin{itemize}
\item {Proveniência:(Do gr. \textunderscore ana\textunderscore  + \textunderscore kustis\textunderscore )}
\end{itemize}
Gênero de algas terrestres.
\section{Anadar}
\begin{itemize}
\item {Grp. gram.:m.}
\end{itemize}
\begin{itemize}
\item {Utilização:Ant.}
\end{itemize}
O mesmo que \textunderscore anadel\textunderscore . Cf. Herculano. \textunderscore Hist. de Port.\textunderscore , IV, 317.
\section{Anadaria}
\begin{itemize}
\item {Grp. gram.:f.}
\end{itemize}
Cargo ou jurisdicção do anadel.
\section{Anaddir}
\begin{itemize}
\item {Grp. gram.:v. t.}
\end{itemize}
\begin{itemize}
\item {Utilização:Ant.}
\end{itemize}
Accrescentar, ajuntar.
(Cast. \textunderscore añadir\textunderscore )
\section{Ânade}
\begin{itemize}
\item {Grp. gram.:f.}
\end{itemize}
\begin{itemize}
\item {Utilização:Ant.}
\end{itemize}
\begin{itemize}
\item {Proveniência:(Lat. \textunderscore anas\textunderscore , \textunderscore anatis\textunderscore )}
\end{itemize}
O mesmo que \textunderscore adem\textunderscore .
\section{Anadeia}
\begin{itemize}
\item {Grp. gram.:f.}
\end{itemize}
\begin{itemize}
\item {Utilização:Ant.}
\end{itemize}
O mesmo que \textunderscore anadaria\textunderscore .
\section{Anadel}
\begin{itemize}
\item {Grp. gram.:m.}
\end{itemize}
\begin{itemize}
\item {Utilização:Ant.}
\end{itemize}
\begin{itemize}
\item {Proveniência:(Do ár. \textunderscore an-nadir\textunderscore )}
\end{itemize}
Chefe de companhia militar; capitão de bèsteiros.
\section{Anadelaria}
\begin{itemize}
\item {Grp. gram.:f.}
\end{itemize}
(V.anadaria)
\section{Anadema}
\begin{itemize}
\item {Grp. gram.:f.}
\end{itemize}
\begin{itemize}
\item {Utilização:Ant.}
\end{itemize}
\begin{itemize}
\item {Proveniência:(Lat. \textunderscore anadema\textunderscore )}
\end{itemize}
Grinalda.
\section{Anadênio}
\begin{itemize}
\item {Grp. gram.:m.}
\end{itemize}
\begin{itemize}
\item {Proveniência:(Do gr. \textunderscore an\textunderscore  priv. + \textunderscore aden\textunderscore )}
\end{itemize}
Planta australiana, da fam. das proteáceas.
\section{Anadeno}
\begin{itemize}
\item {Grp. gram.:m.}
\end{itemize}
\begin{itemize}
\item {Proveniência:(Do gr. \textunderscore anadaiein\textunderscore )}
\end{itemize}
Ave, semelhante ao cuco.
\section{Anadiplose}
\begin{itemize}
\item {Grp. gram.:f.}
\end{itemize}
\begin{itemize}
\item {Proveniência:(Gr. \textunderscore anadiplosis\textunderscore )}
\end{itemize}
Repetição de phrase ou palavra final de um período, no comêço do período seguinte.
\section{Anadir}
\begin{itemize}
\item {Grp. gram.:v. t.}
\end{itemize}
\begin{itemize}
\item {Utilização:Ant.}
\end{itemize}
Accrescentar, ajuntar.
(Cast. \textunderscore añadir\textunderscore )
\section{Anadiu}
\begin{itemize}
\item {Grp. gram.:m.}
\end{itemize}
\begin{itemize}
\item {Utilização:Ant.}
\end{itemize}
Espécie de tecido, próprio para luto. Cf. Alf. Sábio. \textunderscore Cant. de Maria\textunderscore , 122.
\section{A-nado}
\begin{itemize}
\item {Grp. gram.:loc. adv.}
\end{itemize}
Nadando: \textunderscore Passar um rio a-nado\textunderscore .
\section{Anadose}
\begin{itemize}
\item {Grp. gram.:f.}
\end{itemize}
\begin{itemize}
\item {Proveniência:(Gr. \textunderscore anadosis\textunderscore )}
\end{itemize}
Distribuição dos elementos nutritivos na economia animal.
\section{Anádromo}
\begin{itemize}
\item {Grp. gram.:m.}
\end{itemize}
Passagem de humor, das partes inferiores para as posteriores do organismo humano.
\section{Anaduva}
\begin{itemize}
\item {Grp. gram.:f.}
\end{itemize}
(V. \textunderscore adua\textunderscore ^2)
\section{Anadúvia}
\begin{itemize}
\item {Grp. gram.:f.}
\end{itemize}
\begin{itemize}
\item {Utilização:Ant.}
\end{itemize}
(V. \textunderscore adua\textunderscore ^2)
\section{Anaeróbio}
\begin{itemize}
\item {fónica:na-e}
\end{itemize}
\begin{itemize}
\item {Grp. gram.:adj.}
\end{itemize}
\begin{itemize}
\item {Grp. gram.:M.}
\end{itemize}
\begin{itemize}
\item {Proveniência:(Do gr. \textunderscore an\textunderscore  priv. + \textunderscore aer\textunderscore  + \textunderscore bios\textunderscore )}
\end{itemize}
Que póde viver e reproduzir-se fóra do contacto do ar ou do oxygênio livre.
Microorganismo, que se desenvolve nas fermentações do queijo, do leite e do álcool, e que não respira o ar ambiente, mas o oxygênio que extrai dessas fermentações.
\section{Anaeroide}
\begin{itemize}
\item {fónica:na-e}
\end{itemize}
\begin{itemize}
\item {Grp. gram.:m.  e  adj.}
\end{itemize}
\begin{itemize}
\item {Proveniência:(Do gr. \textunderscore an\textunderscore  priv. + \textunderscore aer\textunderscore  + \textunderscore eidos\textunderscore )}
\end{itemize}
O mesmo ou melhor que \textunderscore aneroide\textunderscore .
Diz-se de um barómetro de mostrador, em cuja caixa, de paredes metállicas, se fórma o vácuo.
\section{Anafa}
\begin{itemize}
\item {Grp. gram.:f.}
\end{itemize}
Planta herbácea, da fam. das leguminosas.
\section{Anafaia}
\begin{itemize}
\item {Grp. gram.:f.}
\end{itemize}
\begin{itemize}
\item {Proveniência:(Do ár. \textunderscore an-nafaia\textunderscore )}
\end{itemize}
Primeiros fios do bicho da seda, antes da formação do casulo.
\section{Anafar}
\begin{itemize}
\item {Grp. gram.:v. t.}
\end{itemize}
Alimentar com anafa.
Engordar.
Tornar nédio, pelo bom sustento.
\section{Anáfega}
\begin{itemize}
\item {Grp. gram.:f.}
\end{itemize}
\begin{itemize}
\item {Proveniência:(Do ar. \textunderscore an\textunderscore  + \textunderscore nabica\textunderscore )}
\end{itemize}
Espécie de macieira, que dá frutos doces.
\section{Anafiar}
\begin{itemize}
\item {Grp. gram.:v. t.}
\end{itemize}
\begin{itemize}
\item {Utilização:Des.}
\end{itemize}
O mesmo que \textunderscore anafar\textunderscore . Cf. Filinto, VIII, 22; XVIII, 74.
\section{Anafil}
\begin{itemize}
\item {Grp. gram.:m.}
\end{itemize}
Antiga trombeta moirisca.
Pl.\textunderscore  anafis\textunderscore  e \textunderscore anafiles\textunderscore .
(Ár. \textunderscore annafir\textunderscore )
\section{Anafil}
\begin{itemize}
\item {Grp. gram.:adj.}
\end{itemize}
Diz-se de uma espécie de trigo rijo.
\section{Anagal}
\begin{itemize}
\item {Grp. gram.:f.}
\end{itemize}
O mesmo que \textunderscore anagállis\textunderscore .
\section{Anagalhar}
\begin{itemize}
\item {Grp. gram.:v. t.}
\end{itemize}
\begin{itemize}
\item {Utilização:Prov.}
\end{itemize}
\begin{itemize}
\item {Utilização:trasm.}
\end{itemize}
\begin{itemize}
\item {Grp. gram.:V. p.}
\end{itemize}
\begin{itemize}
\item {Utilização:Prov.}
\end{itemize}
\begin{itemize}
\item {Utilização:beir.}
\end{itemize}
Atar com o nagalho.
Casar-se.
\section{Anagálide}
\begin{itemize}
\item {Grp. gram.:f.}
\end{itemize}
O mesmo que \textunderscore anagális\textunderscore .
\section{Anagális}
\begin{itemize}
\item {Grp. gram.:f.}
\end{itemize}
\begin{itemize}
\item {Proveniência:(Gr. \textunderscore anagallis\textunderscore )}
\end{itemize}
Gênero de plantas primuláceas, a que serve de typo o murrião.
\section{Anagállide}
\begin{itemize}
\item {Grp. gram.:f.}
\end{itemize}
O mesmo que \textunderscore anagállis\textunderscore .
\section{Anagállis}
\begin{itemize}
\item {Grp. gram.:f.}
\end{itemize}
\begin{itemize}
\item {Proveniência:(Gr. \textunderscore anagallis\textunderscore )}
\end{itemize}
Gênero de plantas primuláceas, a que serve de typo o murrião.
\section{Anagenita}
\begin{itemize}
\item {Grp. gram.:f.}
\end{itemize}
Espécie de rocha, que contém fragmentos de pedras ígneas, como granito, pórphyro, etc.
\section{Anagíris}
\begin{itemize}
\item {Grp. gram.:m.}
\end{itemize}
O mesmo que \textunderscore anagiro\textunderscore .
\section{Anagiro}
\begin{itemize}
\item {Grp. gram.:m.}
\end{itemize}
\begin{itemize}
\item {Proveniência:(Gr. \textunderscore anaguros\textunderscore )}
\end{itemize}
Planta leguminosa, de casca e madeira fétidas,(\textunderscore anagýris fétida\textunderscore , Lin.).
\section{Anaglifa}
\begin{itemize}
\item {Grp. gram.:f.}
\end{itemize}
Gênero de plantas compostas da África austral.
\section{Anaglífico}
\begin{itemize}
\item {Grp. gram.:adj.}
\end{itemize}
\begin{itemize}
\item {Proveniência:(De \textunderscore anaglypho\textunderscore )}
\end{itemize}
Diz-se do plano ou superficie, em que se acham figuras esculpidas ou cinzeladas, ou outros objectos, em baixo relêvo.
\section{Anaglifo}
\begin{itemize}
\item {Grp. gram.:m.}
\end{itemize}
\begin{itemize}
\item {Proveniência:(Gr. \textunderscore anagluphos\textunderscore )}
\end{itemize}
Obra em relêvo.
\section{Anagliptico}
\begin{itemize}
\item {Grp. gram.:adj.}
\end{itemize}
O mesmo que \textunderscore anaglífico\textunderscore .
\section{Anaglipto}
\begin{itemize}
\item {Grp. gram.:m.}
\end{itemize}
Gênero de insectos coleópteros longicórneos.
\section{Anagliptografia}
\begin{itemize}
\item {Grp. gram.:f.}
\end{itemize}
\begin{itemize}
\item {Proveniência:(Do gr. \textunderscore anagluptos\textunderscore , relêvo, e \textunderscore graphein\textunderscore , traçar)}
\end{itemize}
Processo de sinaes gráphicos em relêvo, descoberto por Braille, para leitura dos cegos.
\section{Anagliptográfico}
\begin{itemize}
\item {Grp. gram.:adj.}
\end{itemize}
Relativo á \textunderscore anagliptografia\textunderscore .
\section{Anaglypha}
\begin{itemize}
\item {Grp. gram.:f.}
\end{itemize}
Gênero de plantas compostas da África austral.
\section{Anaglýphico}
\begin{itemize}
\item {Grp. gram.:adj.}
\end{itemize}
\begin{itemize}
\item {Proveniência:(De \textunderscore anaglypho\textunderscore )}
\end{itemize}
Diz-se do plano ou superficie, em que se acham figuras esculpidas ou cinzeladas, ou outros objectos, em baixo relêvo.
\section{Anaglypho}
\begin{itemize}
\item {Grp. gram.:m.}
\end{itemize}
\begin{itemize}
\item {Proveniência:(Gr. \textunderscore anagluphos\textunderscore )}
\end{itemize}
Obra em relêvo.
\section{Anaglýptico}
\begin{itemize}
\item {Grp. gram.:adj.}
\end{itemize}
O mesmo que \textunderscore anaglýphico\textunderscore .
\section{Anaglypto}
\begin{itemize}
\item {Grp. gram.:m.}
\end{itemize}
Gênero de insectos coleópteros longicórneos.
\section{Anaglyptographia}
\begin{itemize}
\item {Grp. gram.:f.}
\end{itemize}
\begin{itemize}
\item {Proveniência:(Do gr. \textunderscore anagluptos\textunderscore , relêvo, e \textunderscore graphein\textunderscore , traçar)}
\end{itemize}
Processo de sinaes gráphicos em relêvo, descoberto por Braille, para leitura dos cegos.
\section{Anaglyptográphico}
\begin{itemize}
\item {Grp. gram.:adj.}
\end{itemize}
Relativo á \textunderscore anaglyptographia\textunderscore .
\section{Anagnosigrafia}
\begin{itemize}
\item {Grp. gram.:f.}
\end{itemize}
Arte de ensinar a lêr e a escrever ao mesmo tempo.
\section{Anagnosigráfico}
\begin{itemize}
\item {Grp. gram.:adj.}
\end{itemize}
Relativo á \textunderscore anagnosigrafia\textunderscore .
\section{Anagnosígrafo}
\begin{itemize}
\item {Grp. gram.:m.}
\end{itemize}
Aquelle que pratíca a anagnosigrafia.
\section{Anagnosigraphia}
\begin{itemize}
\item {Grp. gram.:f.}
\end{itemize}
Arte de ensinar a lêr e a escrever ao mesmo tempo.
\section{Anagnosigráphico}
\begin{itemize}
\item {Grp. gram.:adj.}
\end{itemize}
Relativo á \textunderscore anagnosigraphia\textunderscore .
\section{Anagnosígrapho}
\begin{itemize}
\item {Grp. gram.:m.}
\end{itemize}
Aquelle que pratíca a anagnosigraphia.
\section{Anagnoste}
\begin{itemize}
\item {Grp. gram.:m.}
\end{itemize}
\begin{itemize}
\item {Proveniência:(Gr. \textunderscore anagnostes\textunderscore )}
\end{itemize}
Escravo romano, que lia durante os banquetes de seus senhores.
\section{Anágoa}
\begin{itemize}
\item {Grp. gram.:f.}
\end{itemize}
Espécie de saia branca, ás vezes aberta ao lado, e que as mulheres usam sôbre a camisa.
(Cast. \textunderscore enagua\textunderscore )
\section{Anagogia}
\begin{itemize}
\item {Grp. gram.:f.}
\end{itemize}
\begin{itemize}
\item {Proveniência:(Gr. \textunderscore anagogia\textunderscore )}
\end{itemize}
Extase, arrebatamento da alma, na contemplação das coisas divinas.
\section{Anagógico}
\begin{itemize}
\item {Grp. gram.:adj.}
\end{itemize}
Relativo a \textunderscore anagogia\textunderscore .
\section{Anagogismo}
\begin{itemize}
\item {Grp. gram.:m.}
\end{itemize}
Interpretação mýstica dos livros sagrados.
(Cp. \textunderscore anagogia\textunderscore )
\section{Anagogista}
\begin{itemize}
\item {Grp. gram.:m.}
\end{itemize}
Aquelle que se occupa de anagogismo.
\section{Anagrama}
\begin{itemize}
\item {Grp. gram.:m.}
\end{itemize}
\begin{itemize}
\item {Proveniência:(Gr. \textunderscore anagramma\textunderscore )}
\end{itemize}
Palavra ou phrase, feita com as letras de outra.
\section{Anagramaticamente}
\begin{itemize}
\item {Grp. gram.:adv.}
\end{itemize}
De modo \textunderscore anagramático\textunderscore .
\section{Anagramático}
\begin{itemize}
\item {Grp. gram.:adj.}
\end{itemize}
Relativo a \textunderscore anagrama\textunderscore .
Em que há anagrama.
\section{Anagramatismo}
\begin{itemize}
\item {Grp. gram.:m.}
\end{itemize}
Processo ou hábito de fazer anagramas.
\section{Anagramatista}
\begin{itemize}
\item {Grp. gram.:m.}
\end{itemize}
Aquelle que anagramatiza.
\section{Anagramatizar}
\begin{itemize}
\item {Grp. gram.:v. i.}
\end{itemize}
Fazer anagramas.
\section{Anagramma}
\begin{itemize}
\item {Grp. gram.:m.}
\end{itemize}
\begin{itemize}
\item {Proveniência:(Gr. \textunderscore anagramma\textunderscore )}
\end{itemize}
Palavra ou phrase, feita com as letras de outra.
\section{Anagrammaticamente}
\begin{itemize}
\item {Grp. gram.:adv.}
\end{itemize}
De modo \textunderscore anagrammático\textunderscore .
\section{Anagrammático}
\begin{itemize}
\item {Grp. gram.:adj.}
\end{itemize}
Relativo a \textunderscore anagramma\textunderscore .
Em que há anagramma.
\section{Anagrammatismo}
\begin{itemize}
\item {Grp. gram.:m.}
\end{itemize}
Processo ou hábito de fazer anagrammas.
\section{Anagrammatista}
\begin{itemize}
\item {Grp. gram.:m.}
\end{itemize}
Aquelle que anagrammatiza.
\section{Anagrammatizar}
\begin{itemize}
\item {Grp. gram.:v. i.}
\end{itemize}
Fazer anagrammas.
\section{Anágua}
\begin{itemize}
\item {Grp. gram.:f.}
\end{itemize}
Espécie de saia branca, ás vezes aberta ao lado, e que as mulheres usam sôbre a camisa.
(Cast. \textunderscore enagua\textunderscore )
\section{Anágua-de-Vênus}
\begin{itemize}
\item {Grp. gram.:f.}
\end{itemize}
Arbusto ornamental, cujas flôres exhibem quási o feitio de uma pequenina saia branca.
\section{Anaguar}
\begin{itemize}
\item {Grp. gram.:v. t.}
\end{itemize}
\begin{itemize}
\item {Utilização:Prov.}
\end{itemize}
Inundar, cobrir de água.
(Por \textunderscore enaguar\textunderscore , de \textunderscore en...\textunderscore  + \textunderscore água\textunderscore )
\section{Anaguel}
\begin{itemize}
\item {Grp. gram.:m.}
\end{itemize}
\begin{itemize}
\item {Utilização:Prov.}
\end{itemize}
\begin{itemize}
\item {Utilização:trasm.}
\end{itemize}
Taboleiro de cortiça, em que se deitam as tripas e outras miudezas dos porcos que se matam.
\section{Anagýris}
\begin{itemize}
\item {Grp. gram.:m.}
\end{itemize}
O mesmo que \textunderscore anagyro\textunderscore .
\section{Anagyro}
\begin{itemize}
\item {Grp. gram.:m.}
\end{itemize}
\begin{itemize}
\item {Proveniência:(Gr. \textunderscore anaguros\textunderscore )}
\end{itemize}
Planta leguminosa, de casca e madeira fétidas, (\textunderscore anagýris fétida\textunderscore , Lin.)
\section{Anaia}
\begin{itemize}
\item {Grp. gram.:f.}
\end{itemize}
Espécie de talisman entre as cabilas.
Espécie de salvo-conducto.
(Do ár.)
\section{Anaínho}
\begin{itemize}
\item {Grp. gram.:adj.}
\end{itemize}
\begin{itemize}
\item {Utilização:Prov.}
\end{itemize}
\begin{itemize}
\item {Grp. gram.:M.}
\end{itemize}
Que é de pequena marca, (falando-se de animaes): \textunderscore esta gallinha é anaínha\textunderscore .
O mesmo que \textunderscore anão\textunderscore .
(Dem. de \textunderscore anão\textunderscore )
\section{Anajá}
\begin{itemize}
\item {Grp. gram.:m.}
\end{itemize}
\begin{itemize}
\item {Grp. gram.:Pl.}
\end{itemize}
Palmeira e côco, de côr amarela.
Aborígenes brasileiros, que habitavam nos sertões do Pará.
\section{Anajé}
\begin{itemize}
\item {Grp. gram.:m.}
\end{itemize}
\begin{itemize}
\item {Utilização:Bras}
\end{itemize}
Ave do valle do Amazonas.
\section{Anal}
\begin{itemize}
\item {Grp. gram.:adj.}
\end{itemize}
Relativo ao ânus.
\section{Analampo}
\begin{itemize}
\item {Grp. gram.:m.}
\end{itemize}
Insecto coleóptero pentâmero.
\section{Analcime}
\begin{itemize}
\item {Grp. gram.:m.}
\end{itemize}
(V.analcimo)
\section{Analcimo}
\begin{itemize}
\item {Grp. gram.:m.}
\end{itemize}
\begin{itemize}
\item {Proveniência:(Do gr. \textunderscore an\textunderscore  + \textunderscore alkimos\textunderscore )}
\end{itemize}
Nome de um silicato hydratado de soda e alumina.
\section{Analector}
\begin{itemize}
\item {Grp. gram.:m.}
\end{itemize}
Colleccionador de analectos.
\section{Analectos}
\begin{itemize}
\item {Grp. gram.:m. pl.}
\end{itemize}
\begin{itemize}
\item {Proveniência:(Do gr. \textunderscore analekta\textunderscore )}
\end{itemize}
O mesmo que \textunderscore anthologia\textunderscore .
\section{Analema}
\begin{itemize}
\item {Grp. gram.:m.}
\end{itemize}
\begin{itemize}
\item {Proveniência:(Do gr. \textunderscore ana\textunderscore  + \textunderscore lemma\textunderscore )}
\end{itemize}
O mesmo que \textunderscore planisphério\textunderscore .
\section{Analemático}
\begin{itemize}
\item {Grp. gram.:adj.}
\end{itemize}
Relativo ao analema.
\section{Analemma}
\begin{itemize}
\item {Grp. gram.:m.}
\end{itemize}
\begin{itemize}
\item {Proveniência:(Do gr. \textunderscore ana\textunderscore  + \textunderscore lemma\textunderscore )}
\end{itemize}
O mesmo que \textunderscore planisphério\textunderscore .
\section{Analemmático}
\begin{itemize}
\item {Grp. gram.:adj.}
\end{itemize}
Relativo ao analemma.
\section{Analepse}
\begin{itemize}
\item {Grp. gram.:f.}
\end{itemize}
Restauração das fôrças, depois de uma doença; convalescença.
(Cp. \textunderscore analepsia\textunderscore )
\section{Analepsia}
\begin{itemize}
\item {Grp. gram.:f.}
\end{itemize}
\begin{itemize}
\item {Proveniência:(Gr. \textunderscore analepsia\textunderscore )}
\end{itemize}
O mesmo que \textunderscore analepse\textunderscore .
\section{Analéptica}
\begin{itemize}
\item {Grp. gram.:f.}
\end{itemize}
\begin{itemize}
\item {Proveniência:(De \textunderscore analéptico\textunderscore )}
\end{itemize}
Parte da Hygíene, que ensina a restabelecer as fôrças dos convalescentes.
\section{Analéptico}
\begin{itemize}
\item {Grp. gram.:adj.}
\end{itemize}
\begin{itemize}
\item {Proveniência:(Gr. \textunderscore analeptikos\textunderscore )}
\end{itemize}
Que restaura as fôrças.
\section{Analfabético}
\begin{itemize}
\item {Grp. gram.:adj.}
\end{itemize}
\begin{itemize}
\item {Proveniência:(De \textunderscore analphabeto\textunderscore )}
\end{itemize}
Diz-se das línguas, que não têm alfabeto, como o tupi, o quimbundo, etc.
\section{Analfabetismo}
\begin{itemize}
\item {Grp. gram.:m.}
\end{itemize}
Falta de instrucção, qualidade do que é \textunderscore analfabeto\textunderscore .
\section{Analfabeto}
\begin{itemize}
\item {Grp. gram.:m.}
\end{itemize}
\begin{itemize}
\item {Grp. gram.:Adj.}
\end{itemize}
\begin{itemize}
\item {Proveniência:(Do gr. \textunderscore an\textunderscore  priv. + \textunderscore alpha\textunderscore  + \textunderscore beta\textunderscore )}
\end{itemize}
Aquelle que ignora o alphabeto.
Que ignora o alfabeto.
Que é muito ignorante.
\section{Analgene}
\begin{itemize}
\item {Grp. gram.:f.}
\end{itemize}
Substância pharmacêutica anti-neurálgica; o mesmo que \textunderscore benzalgene\textunderscore .
\section{Analgesia}
\begin{itemize}
\item {Grp. gram.:f.}
\end{itemize}
O mesmo que \textunderscore analgia\textunderscore .
\section{Analgésico}
\begin{itemize}
\item {Grp. gram.:adj.}
\end{itemize}
O mesmo que \textunderscore análgico\textunderscore .
\section{Analgesina}
\begin{itemize}
\item {Grp. gram.:f.}
\end{itemize}
O mesmo que \textunderscore antipyrina\textunderscore .
\section{Analgia}
\begin{itemize}
\item {Grp. gram.:f.}
\end{itemize}
\begin{itemize}
\item {Proveniência:(Do gr. \textunderscore an\textunderscore  priv. + \textunderscore algos\textunderscore )}
\end{itemize}
Ausência de dôr.
Insensibilidade.
\section{Análgico}
\begin{itemize}
\item {Grp. gram.:adj.}
\end{itemize}
Relativo á \textunderscore analgia\textunderscore .
\section{Analisador}
\begin{itemize}
\item {Grp. gram.:m.}
\end{itemize}
Aquelle que analisa.
\section{Analisar}
\begin{itemize}
\item {Grp. gram.:v. t.}
\end{itemize}
Fazer analise de.
\section{Analisável}
\begin{itemize}
\item {Grp. gram.:adj.}
\end{itemize}
Que se póde analisar.
\section{Análise}
\begin{itemize}
\item {Grp. gram.:f.}
\end{itemize}
\begin{itemize}
\item {Proveniência:(Gr. \textunderscore analusis\textunderscore )}
\end{itemize}
Decomposição de um todo em partes.
Exame de cada parte de um todo.
Processo philosóphico, com que se sobe dos effeitos
ás causas.
A álgebra.
\section{Analista}
\begin{itemize}
\item {Grp. gram.:m.  e  adj.}
\end{itemize}
\begin{itemize}
\item {Proveniência:(De \textunderscore anályse\textunderscore )}
\end{itemize}
O que analisa.
O que é versado em álgebra.
\section{Analiticamente}
\begin{itemize}
\item {Grp. gram.:adv.}
\end{itemize}
De modo \textunderscore analítico\textunderscore .
\section{Analítico}
\begin{itemize}
\item {Grp. gram.:adj.}
\end{itemize}
Que procede por \textunderscore análise\textunderscore .
\section{Analluvião}
\begin{itemize}
\item {Grp. gram.:f.}
\end{itemize}
Alluvião ou detritos, resultantes da decomposição de rochas.
\section{Analogético}
\begin{itemize}
\item {Grp. gram.:m.  e  adj.}
\end{itemize}
O mesmo que \textunderscore ecléctico\textunderscore .
\section{Analogia}
\begin{itemize}
\item {Grp. gram.:f.}
\end{itemize}
\begin{itemize}
\item {Proveniência:(Gr. \textunderscore analogia\textunderscore )}
\end{itemize}
Ponto de semelhança, entre objectos differentes.
Investigação philosóphica da razão das semelhanças entre as coisas.
Razão da formação de certas palavras.
\section{Analogicamente}
\begin{itemize}
\item {Grp. gram.:adv.}
\end{itemize}
De modo \textunderscore analógico\textunderscore .
\section{Analógico}
\begin{itemize}
\item {Grp. gram.:adj.}
\end{itemize}
Que tem analogia.
Baseado em \textunderscore analogia\textunderscore .
\section{Analogismo}
\begin{itemize}
\item {Grp. gram.:m.}
\end{itemize}
Acto de discorrer por \textunderscore analogia\textunderscore .
\section{Analogista}
\begin{itemize}
\item {Grp. gram.:m.}
\end{itemize}
Aquelle que discorre por \textunderscore analogia\textunderscore .
\section{Analogístico}
\begin{itemize}
\item {Grp. gram.:adj.}
\end{itemize}
\begin{itemize}
\item {Proveniência:(De \textunderscore analogista\textunderscore )}
\end{itemize}
Em que se procede por analogia.
\section{Análogo}
\begin{itemize}
\item {Grp. gram.:adj.}
\end{itemize}
\begin{itemize}
\item {Proveniência:(Gr. \textunderscore analogos\textunderscore )}
\end{itemize}
Que tem analogia.
Baseado em analogia.
\section{Analose}
\begin{itemize}
\item {Grp. gram.:f.}
\end{itemize}
Consumpção pathológica, enfraquecimento, depauperação de fôrças.
\section{Analphabético}
\begin{itemize}
\item {Grp. gram.:adj.}
\end{itemize}
\begin{itemize}
\item {Proveniência:(De \textunderscore analphabeto\textunderscore )}
\end{itemize}
Diz-se das línguas, que não têm alphabeto, como o tupi, o quimbundo, etc.
\section{Analphabetismo}
\begin{itemize}
\item {Grp. gram.:m.}
\end{itemize}
Falta de instrucção, qualidade do que é \textunderscore analphabeto\textunderscore .
\section{Analphabeto}
\begin{itemize}
\item {Grp. gram.:m.}
\end{itemize}
\begin{itemize}
\item {Grp. gram.:Adj.}
\end{itemize}
\begin{itemize}
\item {Proveniência:(Do gr. \textunderscore an\textunderscore  priv. + \textunderscore alpha\textunderscore  + \textunderscore beta\textunderscore )}
\end{itemize}
Aquelle que ignora o alphabeto.
Que ignora o alphabeto.
Que é muito ignorante.
\section{Analuvião}
\begin{itemize}
\item {Grp. gram.:f.}
\end{itemize}
Aluvião ou detritos, resultantes da decomposição de rochas.
\section{Analysador}
\begin{itemize}
\item {Grp. gram.:m.}
\end{itemize}
Aquelle que analysa.
\section{Analysar}
\begin{itemize}
\item {Grp. gram.:v. t.}
\end{itemize}
Fazer analyse de.
\section{Analysável}
\begin{itemize}
\item {Grp. gram.:adj.}
\end{itemize}
Que se póde analysar.
\section{Anályse}
\begin{itemize}
\item {Grp. gram.:f.}
\end{itemize}
\begin{itemize}
\item {Proveniência:(Gr. \textunderscore analusis\textunderscore )}
\end{itemize}
Decomposição de um todo em partes.
Exame de cada parte de um todo.
Processo philosóphico, com que se sobe dos effeitos
ás causas.
A álgebra.
\section{Analysta}
\begin{itemize}
\item {Grp. gram.:m.  e  adj.}
\end{itemize}
\begin{itemize}
\item {Proveniência:(De \textunderscore anályse\textunderscore )}
\end{itemize}
O que analysa.
O que é versado em álgebra.
\section{Analyticamente}
\begin{itemize}
\item {Grp. gram.:adv.}
\end{itemize}
De modo \textunderscore analýtico\textunderscore .
\section{Analýtico}
\begin{itemize}
\item {Grp. gram.:adj.}
\end{itemize}
Que procede por \textunderscore anályse\textunderscore .
\section{Anambés}
\begin{itemize}
\item {Grp. gram.:m. pl.}
\end{itemize}
Tríbo de índios, nas cabeceiras do rio Cururuí, no Brasil.
\section{Aname}
\begin{itemize}
\item {Grp. gram.:m.}
\end{itemize}
A língua do Anão.
\section{Anâmico}
\begin{itemize}
\item {Grp. gram.:adj.}
\end{itemize}
Relativo ao Anão: \textunderscore a língua anâmica\textunderscore .
\section{Anamirtina}
\begin{itemize}
\item {Grp. gram.:f.}
\end{itemize}
Substância gorda, que se extrái do fruto do anamirto.
\section{Anamirto}
\begin{itemize}
\item {Grp. gram.:m.}
\end{itemize}
Planta, da fam. das menispermáceas.
\section{Anamita}
\begin{itemize}
\item {Grp. gram.:m.}
\end{itemize}
Aquelle que é natural de Anão.
O mesmo que \textunderscore aname\textunderscore ^1.
\section{Anamítico}
\begin{itemize}
\item {Grp. gram.:adj.}
\end{itemize}
Relativo aos anamitas.
\section{Anamnese}
\begin{itemize}
\item {Grp. gram.:f.}
\end{itemize}
O mesmo que \textunderscore anamnesia\textunderscore .
\section{Anamnesia}
\begin{itemize}
\item {Grp. gram.:f.}
\end{itemize}
\begin{itemize}
\item {Utilização:Med.}
\end{itemize}
\begin{itemize}
\item {Proveniência:(Do gr. \textunderscore ana\textunderscore  + \textunderscore mnesis\textunderscore )}
\end{itemize}
Acto de recordar o que se finge esquecido. Reminiscência.
Restabelecimento da memória.
Informação sobre o comêço e a evolução de uma moléstia, até o momento da observação médica; história pregressa.
\section{Anamnéstico}
\begin{itemize}
\item {Grp. gram.:adj.}
\end{itemize}
\begin{itemize}
\item {Proveniência:(Gr. \textunderscore anamnestikos\textunderscore )}
\end{itemize}
Que aviva a memória.
\section{Anamorfose}
\begin{itemize}
\item {Grp. gram.:f.}
\end{itemize}
\begin{itemize}
\item {Utilização:Bot.}
\end{itemize}
\begin{itemize}
\item {Proveniência:(Do gr. \textunderscore anamorphosis\textunderscore )}
\end{itemize}
Imagem disforme, que, em certo ponto de vista, parece regular.
Arte de desenhar essa imagem.
Modificação ou degenerescência mórbida, que se dá em algumas plantas.
\section{Anamorfótico}
\begin{itemize}
\item {Grp. gram.:adj.}
\end{itemize}
Relativo á anamorfose.
\section{Anamorphose}
\begin{itemize}
\item {Grp. gram.:f.}
\end{itemize}
\begin{itemize}
\item {Utilização:Bot.}
\end{itemize}
\begin{itemize}
\item {Proveniência:(Do gr. \textunderscore anamorphosis\textunderscore )}
\end{itemize}
Imagem disforme, que, em certo ponto de vista, parece regular.
Arte de desenhar essa imagem.
Modificação ou degenerescência mórbida, que se dá em algumas plantas.
\section{Anamorphótico}
\begin{itemize}
\item {Grp. gram.:adj.}
\end{itemize}
Relativo á anamorphose.
\section{Anan}
(fem. de \textunderscore anão\textunderscore )
Bananeira do Brasil.
\section{Ananabasia}
\begin{itemize}
\item {Grp. gram.:f.}
\end{itemize}
\begin{itemize}
\item {Utilização:Med.}
\end{itemize}
Abasia intermittente e angustiosa, nos neurasthênicos.
\section{Ananás}
\begin{itemize}
\item {Grp. gram.:m.}
\end{itemize}
Planta bromeliácea, intertropical.
O fruto dessa planta.
\section{Ananaseiro}
\begin{itemize}
\item {Grp. gram.:m.}
\end{itemize}
O mesmo que a planta \textunderscore ananás\textunderscore .
\section{Ananastasia}
\begin{itemize}
\item {Grp. gram.:f.}
\end{itemize}
\begin{itemize}
\item {Utilização:Med.}
\end{itemize}
Astasia intermittente e angustiosa.
\section{Ananchite}
\begin{itemize}
\item {fónica:qui}
\end{itemize}
\begin{itemize}
\item {Grp. gram.:f.}
\end{itemize}
\begin{itemize}
\item {Proveniência:(Lat. \textunderscore ananchitis\textunderscore )}
\end{itemize}
Pedra preciosa, hoje desconhecida, e da qual, segundo Plínio, se servem os feiticeiros para invocar os demónios.
\section{Ananciclo}
\begin{itemize}
\item {Grp. gram.:m.}
\end{itemize}
Gênero de insectos coleópteros de Java.
\section{Anancyclo}
\begin{itemize}
\item {Grp. gram.:m.}
\end{itemize}
Gênero de insectos coleópteros de Java.
\section{Anandrário}
\begin{itemize}
\item {Grp. gram.:adj.}
\end{itemize}
Diz-se das flores, cujos tegumentos e pistillos substituem os estames.
(Cp. \textunderscore anândrio\textunderscore )
\section{Anandrino}
\begin{itemize}
\item {Grp. gram.:adj.}
\end{itemize}
O mesmo que \textunderscore anândrio\textunderscore .
\section{Anândrio}
\begin{itemize}
\item {Grp. gram.:adj.}
\end{itemize}
\begin{itemize}
\item {Utilização:Bot.}
\end{itemize}
\begin{itemize}
\item {Proveniência:(Do gr. \textunderscore an\textunderscore  priv. + \textunderscore aner\textunderscore , \textunderscore andros\textunderscore )}
\end{itemize}
Cujas fôlhas não têm órgãos masculinos.
\section{Anandro}
\begin{itemize}
\item {Grp. gram.:adj.}
\end{itemize}
\begin{itemize}
\item {Utilização:Bot.}
\end{itemize}
\begin{itemize}
\item {Proveniência:(Do gr. \textunderscore an\textunderscore  priv. + \textunderscore aner\textunderscore , \textunderscore andros\textunderscore )}
\end{itemize}
Cujas fôlhas não têm órgãos masculinos.
\section{Ananera}
\begin{itemize}
\item {Grp. gram.:m.}
\end{itemize}
\begin{itemize}
\item {Utilização:Bras}
\end{itemize}
Árvore silvestre, de bôa madeira para construcções.
\section{Anani}
\begin{itemize}
\item {Grp. gram.:m.}
\end{itemize}
\begin{itemize}
\item {Utilização:Bras}
\end{itemize}
Designação vulgar de uma planta, (\textunderscore simphonia globulifera\textunderscore ).
\section{Ananicar}
\begin{itemize}
\item {Grp. gram.:v. t.}
\end{itemize}
Tornar anão, pequeno, desprezível.
\section{Anânico}
\begin{itemize}
\item {Grp. gram.:adj.}
\end{itemize}
Que tem fórma de anão.
\section{Ananim}
\begin{itemize}
\item {Grp. gram.:m.}
\end{itemize}
O mesmo que \textunderscore anani\textunderscore .
\section{Ananismo}
\begin{itemize}
\item {Grp. gram.:m.}
\end{itemize}
\begin{itemize}
\item {Proveniência:(De \textunderscore anão\textunderscore )}
\end{itemize}
Desenvolvimento mesquinho e anormal de uma planta.
\section{Anano}
\begin{itemize}
\item {Grp. gram.:adj.}
\end{itemize}
O mesmo que \textunderscore anânico\textunderscore . Cf. \textunderscore Viriato Trág.\textunderscore , XVII, 91.
\section{Ananquite}
\begin{itemize}
\item {Grp. gram.:f.}
\end{itemize}
\begin{itemize}
\item {Proveniência:(Lat. \textunderscore ananchitis\textunderscore )}
\end{itemize}
Pedra preciosa, hoje desconhecida, e da qual, segundo Plínio, se servem os feiticeiros para invocar os demónios.
\section{Ananta}
\begin{itemize}
\item {Grp. gram.:f.}
\end{itemize}
Planta indiana, medicinal.
\section{Anantho}
\begin{itemize}
\item {Grp. gram.:adj.}
\end{itemize}
\begin{itemize}
\item {Proveniência:(Do gr. \textunderscore an\textunderscore  priv. + \textunderscore anthos\textunderscore )}
\end{itemize}
Que não deita flôr.
\section{Ananto}
\begin{itemize}
\item {Grp. gram.:adj.}
\end{itemize}
\begin{itemize}
\item {Proveniência:(Do gr. \textunderscore an\textunderscore  priv. + \textunderscore anthos\textunderscore )}
\end{itemize}
Que não deita flôr.
\section{Anão}
\begin{itemize}
\item {Grp. gram.:m.}
\end{itemize}
\begin{itemize}
\item {Grp. gram.:Adj.}
\end{itemize}
\begin{itemize}
\item {Proveniência:(Lat. \textunderscore nanus\textunderscore )}
\end{itemize}
O homem, de estatura menor que a regular.
Pequeno; enfezado.
\section{Anáfalo}
\begin{itemize}
\item {Grp. gram.:m.}
\end{itemize}
\begin{itemize}
\item {Proveniência:(Gr. \textunderscore anaphalos\textunderscore )}
\end{itemize}
Planta indiana, da fam. das compostas.
\section{Anafe}
\begin{itemize}
\item {Grp. gram.:f.}
\end{itemize}
\begin{itemize}
\item {Utilização:Bot.}
\end{itemize}
O mesmo que \textunderscore corôa-de-rei\textunderscore .
\section{Anafonese}
\begin{itemize}
\item {Grp. gram.:f.}
\end{itemize}
\begin{itemize}
\item {Proveniência:(Do gr. \textunderscore anaphonesis\textunderscore )}
\end{itemize}
Exercício de voz.
Grito.
\section{Anáfora}
\begin{itemize}
\item {Grp. gram.:f.}
\end{itemize}
\begin{itemize}
\item {Proveniência:(Do gr. \textunderscore anaphora\textunderscore )}
\end{itemize}
Repetição de uma palavra no comêço de differentes phrases ou de membros de uma phrase.
\section{Anafórico}
\begin{itemize}
\item {Grp. gram.:adj.}
\end{itemize}
Que contém \textunderscore anáfora\textunderscore .
\section{Anaforismo}
\begin{itemize}
\item {Grp. gram.:m.}
\end{itemize}
Abuso da anáfora.
\section{Anafrodisia}
\begin{itemize}
\item {Grp. gram.:f.}
\end{itemize}
\begin{itemize}
\item {Proveniência:(Do gr. \textunderscore ana\textunderscore  priv. + \textunderscore Aphrodite\textunderscore , n. p.)}
\end{itemize}
Ausência de appetites venéreos.
\section{Anafrodisiaco}
\begin{itemize}
\item {Grp. gram.:adj.}
\end{itemize}
O mesmo que anafrodisiano.
\section{Anafrodisiano}
\begin{itemize}
\item {Grp. gram.:adj.}
\end{itemize}
\begin{itemize}
\item {Proveniência:(De \textunderscore anaphrodisia\textunderscore )}
\end{itemize}
Que tira ou evita appetites venéreos.
\section{Anafrodita}
\begin{itemize}
\item {Grp. gram.:m.  e  adj.}
\end{itemize}
\begin{itemize}
\item {Proveniência:(Do gr. \textunderscore ana\textunderscore  priv. + \textunderscore Aphrodite\textunderscore , n. p.)}
\end{itemize}
O que é insensível ao amor.
\section{Anafroditico}
\begin{itemize}
\item {Grp. gram.:adj.}
\end{itemize}
\begin{itemize}
\item {Proveniência:(De \textunderscore anaphrodita\textunderscore )}
\end{itemize}
Que não é produzido por geração propriamente dita ou por concurso dos sexos.
\section{Anápala}
\begin{itemize}
\item {Grp. gram.:f.}
\end{itemize}
\begin{itemize}
\item {Proveniência:(Gr. \textunderscore anapale\textunderscore )}
\end{itemize}
Dança ou luta de crianças nuas, em Lacedemónia.
\section{Anapéstico}
\begin{itemize}
\item {Grp. gram.:adj.}
\end{itemize}
Composto de \textunderscore anapestos\textunderscore .
\section{Anapesto}
\begin{itemize}
\item {Grp. gram.:m.}
\end{itemize}
\begin{itemize}
\item {Proveniência:(Gr. \textunderscore anapaistos\textunderscore )}
\end{itemize}
Pé de verso latino ou grego, em que entram três sýllabas, sendo breves as duas primeiras e longa a última.
\section{Anapetia}
\begin{itemize}
\item {Grp. gram.:f.}
\end{itemize}
\begin{itemize}
\item {Utilização:Med.}
\end{itemize}
Dilatação dos vasos ou do orifício de certas vísceras.
\section{Anáphalo}
\begin{itemize}
\item {Grp. gram.:m.}
\end{itemize}
\begin{itemize}
\item {Proveniência:(Gr. \textunderscore anaphalos\textunderscore )}
\end{itemize}
Planta indiana, da fam. das compostas.
\section{Anaphe}
\begin{itemize}
\item {Grp. gram.:f.}
\end{itemize}
\begin{itemize}
\item {Utilização:Bot.}
\end{itemize}
O mesmo que \textunderscore corôa-de-rei\textunderscore .
\section{Anaphonese}
\begin{itemize}
\item {Grp. gram.:f.}
\end{itemize}
\begin{itemize}
\item {Proveniência:(Do gr. \textunderscore anaphonesis\textunderscore )}
\end{itemize}
Exercício de voz.
Grito.
\section{Anáphora}
\begin{itemize}
\item {Grp. gram.:f.}
\end{itemize}
\begin{itemize}
\item {Proveniência:(Do gr. \textunderscore anaphora\textunderscore )}
\end{itemize}
Repetição de uma palavra no comêço de differentes phrases ou de membros de uma phrase.
\section{Anaphórico}
\begin{itemize}
\item {Grp. gram.:adj.}
\end{itemize}
Que contém \textunderscore anáphora\textunderscore .
\section{Anaphorismo}
\begin{itemize}
\item {Grp. gram.:m.}
\end{itemize}
Abuso da anáphora.
\section{Anaphrodisia}
\begin{itemize}
\item {Grp. gram.:f.}
\end{itemize}
\begin{itemize}
\item {Proveniência:(Do gr. \textunderscore ana\textunderscore  priv. + \textunderscore Aphrodite\textunderscore , n. p.)}
\end{itemize}
Ausência de appetites venéreos.
\section{Anaphrodisiaco}
\begin{itemize}
\item {Grp. gram.:adj.}
\end{itemize}
O mesmo que \textunderscore anaphrodisiano\textunderscore .
\section{Anaphrodisiano}
\begin{itemize}
\item {Grp. gram.:adj.}
\end{itemize}
\begin{itemize}
\item {Proveniência:(De \textunderscore anaphrodisia\textunderscore )}
\end{itemize}
Que tira ou evita appetites venéreos.
\section{Anaphrodita}
\begin{itemize}
\item {Grp. gram.:m.  e  adj.}
\end{itemize}
\begin{itemize}
\item {Proveniência:(Do gr. \textunderscore ana\textunderscore  priv. + \textunderscore Aphrodite\textunderscore , n. p.)}
\end{itemize}
O que é insensível ao amor.
\section{Anaphroditico}
\begin{itemize}
\item {Grp. gram.:adj.}
\end{itemize}
\begin{itemize}
\item {Proveniência:(De \textunderscore anaphrodita\textunderscore )}
\end{itemize}
Que não é produzido por geração propriamente dita ou por concurso dos sexos.
\section{Anaplasia}
\begin{itemize}
\item {Grp. gram.:f.}
\end{itemize}
O mesmo que \textunderscore anaplastia\textunderscore .
\section{Anaplastia}
\begin{itemize}
\item {Grp. gram.:f.}
\end{itemize}
\begin{itemize}
\item {Proveniência:(Do gr. \textunderscore ana\textunderscore  + \textunderscore plassein\textunderscore )}
\end{itemize}
Arte de restabelecer a fórma normal de uma parte mutilada do corpo.
\section{Anaplástico}
\begin{itemize}
\item {Grp. gram.:adj.}
\end{itemize}
Relativo á \textunderscore anaplastia\textunderscore .
\section{Anaplecto}
\begin{itemize}
\item {Grp. gram.:m.}
\end{itemize}
\begin{itemize}
\item {Proveniência:(Do gr. \textunderscore ana\textunderscore  + \textunderscore plektos\textunderscore )}
\end{itemize}
Insecto orthóptero da América central.
\section{Anaplerose}
\begin{itemize}
\item {Grp. gram.:f.}
\end{itemize}
\begin{itemize}
\item {Utilização:Cir.}
\end{itemize}
O mesmo que \textunderscore próthese\textunderscore .
\section{Anaplerótico}
\begin{itemize}
\item {Grp. gram.:m.  e  adj.}
\end{itemize}
\begin{itemize}
\item {Proveniência:(De \textunderscore anaplerose\textunderscore )}
\end{itemize}
Medicamento antigo, que se suppunha determinar a reproducção das carnes e a cicatrização das feridas.
\section{Anapneuse}
\begin{itemize}
\item {Grp. gram.:f.}
\end{itemize}
O mesmo que \textunderscore respiração\textunderscore .
\section{Anapnoico}
\begin{itemize}
\item {Grp. gram.:adj.}
\end{itemize}
Que facilita a expectoração.
\section{Anaptictico}
\begin{itemize}
\item {Grp. gram.:adj.}
\end{itemize}
\begin{itemize}
\item {Utilização:Philol.}
\end{itemize}
Diz-se da vogal intercalada, que desune duas consoantes, como em \textunderscore carapinteiro\textunderscore  e \textunderscore igonorar\textunderscore , fórmas populares de \textunderscore carpinteiro\textunderscore  e \textunderscore ignorar\textunderscore .
\section{Anaptisia}
\begin{itemize}
\item {Grp. gram.:f.}
\end{itemize}
\begin{itemize}
\item {Utilização:Med.}
\end{itemize}
Corrimento mucoso.
\section{Anaptixe}
\begin{itemize}
\item {fónica:cse}
\end{itemize}
\begin{itemize}
\item {Grp. gram.:f.}
\end{itemize}
\begin{itemize}
\item {Utilização:Philol.}
\end{itemize}
Intercalação de uma vogal, que desune duas consoantes.
\section{Anaptíxico}
\begin{itemize}
\item {Grp. gram.:adj.}
\end{itemize}
(V.anaptictico)
\section{Anarca}
\begin{itemize}
\item {Grp. gram.:m. f.}
\end{itemize}
O mesmo que \textunderscore anarchista\textunderscore . Cf. Filinto, IV, 243.
\section{Anarcha}
\begin{itemize}
\item {Grp. gram.:m. f.}
\end{itemize}
O mesmo que \textunderscore anarchista\textunderscore . Cf. Filinto, IV, 243.
\section{Anarchia}
\begin{itemize}
\item {fónica:qui}
\end{itemize}
\begin{itemize}
\item {Grp. gram.:f.}
\end{itemize}
\begin{itemize}
\item {Proveniência:(Gr. \textunderscore anarkhia\textunderscore )}
\end{itemize}
Falta de govêrno, de chefe.
Negação do princípio de autoridade.
Desordem; confusão.
\section{Anárchico}
\begin{itemize}
\item {fónica:qui}
\end{itemize}
\begin{itemize}
\item {Grp. gram.:adj.}
\end{itemize}
Em que há anarchia.
Que excita á anarchia.
Desordenado, confuso.
\section{Anarchismo}
\begin{itemize}
\item {fónica:quis}
\end{itemize}
\begin{itemize}
\item {Grp. gram.:m.}
\end{itemize}
Systema dos anarchistas.
\section{Anarchista}
\begin{itemize}
\item {fónica:quis}
\end{itemize}
\begin{itemize}
\item {Grp. gram.:m.}
\end{itemize}
Sectário da anarchia.
\section{Anarchizar}
\begin{itemize}
\item {fónica:qui}
\end{itemize}
\begin{itemize}
\item {Grp. gram.:v. t.}
\end{itemize}
\begin{itemize}
\item {Proveniência:(De \textunderscore anarchia\textunderscore )}
\end{itemize}
Tornar anárchico.
Excitar á desordem.
Sublevar.
\section{Anarcotina}
\begin{itemize}
\item {Grp. gram.:f.}
\end{itemize}
Embora pareça o contrário, é o mesmo que \textunderscore narcotina\textunderscore  pura, segundo os mais recentes formulários pharmacêuticos.
\section{Anarete}
\begin{itemize}
\item {Grp. gram.:m.}
\end{itemize}
Gênero de insectos dípteros.
\section{Anárico}
\begin{itemize}
\item {Grp. gram.:adj.}
\end{itemize}
\begin{itemize}
\item {Utilização:Philol.}
\end{itemize}
\begin{itemize}
\item {Proveniência:(Do gr. \textunderscore an\textunderscore priv. + \textunderscore árico\textunderscore )}
\end{itemize}
Diz-se das linguas, que não pertencem á família árica. Cf. Viana, \textunderscore Class. das Línguas\textunderscore , 14.
\section{Anarico}
\begin{itemize}
\item {Grp. gram.:m.}
\end{itemize}
Espécie de peixe ósseo.
\section{Anarmóstico}
\begin{itemize}
\item {Grp. gram.:adj.}
\end{itemize}
\begin{itemize}
\item {Utilização:Miner.}
\end{itemize}
\begin{itemize}
\item {Proveniência:(Do gr. \textunderscore an\textunderscore  + \textunderscore armozein\textunderscore )}
\end{itemize}
Diz-se dos crystaes, cujas faces não são todas produzidas segundo a mesma lei.
\section{Anarquia}
\begin{itemize}
\item {Grp. gram.:f.}
\end{itemize}
\begin{itemize}
\item {Proveniência:(Gr. \textunderscore anarkhia\textunderscore )}
\end{itemize}
Falta de govêrno, de chefe.
Negação do princípio de autoridade.
Desordem; confusão.
\section{Anárquico}
\begin{itemize}
\item {Grp. gram.:adj.}
\end{itemize}
Em que há anarquia.
Que excita á anarquia.
Desordenado, confuso.
\section{Anarquismo}
\begin{itemize}
\item {Grp. gram.:m.}
\end{itemize}
Systema dos anarquistas.
\section{Anarquista}
\begin{itemize}
\item {Grp. gram.:m.}
\end{itemize}
Sectário da anarquia.
\section{Anarquizar}
\begin{itemize}
\item {Grp. gram.:v. t.}
\end{itemize}
\begin{itemize}
\item {Proveniência:(De \textunderscore anarchia\textunderscore )}
\end{itemize}
Tornar anárquico.
Excitar á desordem.
Sublevar.
\section{Anarreia}
\begin{itemize}
\item {Grp. gram.:f.}
\end{itemize}
Affluência de humores ás regiões superiores do organismo humano.
\section{Anarreico}
\begin{itemize}
\item {Grp. gram.:adj.}
\end{itemize}
Relativo á anarreia.
\section{Anarrheia}
\begin{itemize}
\item {Grp. gram.:f.}
\end{itemize}
Affluência de humores ás regiões superiores do organismo humano.
\section{Anarrheico}
\begin{itemize}
\item {Grp. gram.:adj.}
\end{itemize}
Relativo á anarrheia.
\section{Anárrhico}
\begin{itemize}
\item {Grp. gram.:m.}
\end{itemize}
Nome scientífico do gato marinho ou lobo do mar.
\section{Anárrico}
\begin{itemize}
\item {Grp. gram.:m.}
\end{itemize}
Nome scientífico do gato marinho ou lobo do mar.
\section{Anarropia}
\begin{itemize}
\item {Grp. gram.:f.}
\end{itemize}
Tendência para a anarrheia.
\section{Anarrópico}
\begin{itemize}
\item {Grp. gram.:adj.}
\end{itemize}
Relativo á anarropia.
\section{Anarta}
\begin{itemize}
\item {Grp. gram.:f.}
\end{itemize}
Insecto lepidóptero nocturno.
\section{Anarthro}
\begin{itemize}
\item {Grp. gram.:adj.}
\end{itemize}
\begin{itemize}
\item {Utilização:Med.}
\end{itemize}
\begin{itemize}
\item {Proveniência:(Do gr. \textunderscore an\textunderscore priv. + \textunderscore arthron\textunderscore )}
\end{itemize}
Que não tem as articulações bem pronunciadas.
\section{Anartro}
\begin{itemize}
\item {Grp. gram.:adj.}
\end{itemize}
\begin{itemize}
\item {Utilização:Med.}
\end{itemize}
\begin{itemize}
\item {Proveniência:(Do gr. \textunderscore an\textunderscore priv. + \textunderscore arthron\textunderscore )}
\end{itemize}
Que não tem as articulações bem pronunciadas.
\section{Anasarca}
\begin{itemize}
\item {Grp. gram.:f.}
\end{itemize}
\begin{itemize}
\item {Proveniência:(Do gr. \textunderscore ana\textunderscore  + \textunderscore sarx\textunderscore )}
\end{itemize}
Inchação, produzida por infiltração de serosidade no tecido cellular.
\section{Anasarco}
\begin{itemize}
\item {Grp. gram.:adj.}
\end{itemize}
\begin{itemize}
\item {Utilização:Ant.}
\end{itemize}
O mesmo que \textunderscore anasártico\textunderscore .
\section{Anasártico}
\begin{itemize}
\item {Grp. gram.:adj.}
\end{itemize}
\begin{itemize}
\item {Utilização:Ant.}
\end{itemize}
Pertencente á anasarca.
\section{Anaspo}
\begin{itemize}
\item {Grp. gram.:m.}
\end{itemize}
\begin{itemize}
\item {Proveniência:(Do gr. \textunderscore an\textunderscore priv. + \textunderscore aspis\textunderscore )}
\end{itemize}
Pequeno insecto coleóptero heterómero.
\section{Anassal}
\begin{itemize}
\item {Grp. gram.:m.}
\end{itemize}
\begin{itemize}
\item {Utilização:Ant.}
\end{itemize}
Parte do antigo capello de ferro. Cf. \textunderscore Cancion. da Vaticana\textunderscore , 1080.
\section{Anástase}
\begin{itemize}
\item {Grp. gram.:f.}
\end{itemize}
\begin{itemize}
\item {Utilização:Med.}
\end{itemize}
\begin{itemize}
\item {Proveniência:(Do gr. \textunderscore an\textunderscore priv. + \textunderscore stasis\textunderscore )}
\end{itemize}
Transporte dos humores, de uma para outra parte do corpo.
\section{Anastatíceas}
\begin{itemize}
\item {Grp. gram.:f. pl.}
\end{itemize}
Tribo de plantas crucíferas, estabelecida por De-Candolle.
\section{Anastático}
\begin{itemize}
\item {Grp. gram.:adj.}
\end{itemize}
\begin{itemize}
\item {Proveniência:(Do gr. \textunderscore anastasis\textunderscore )}
\end{itemize}
Diz-se do processo, com que se reproduzem, por transporte chímico, textos ou desenhos impressos.
\section{Anastêmonas}
\begin{itemize}
\item {Grp. gram.:f.}
\end{itemize}
Classe de plantas dicotyledóneas.
\section{Anástomas}
\begin{itemize}
\item {Grp. gram.:m.}
\end{itemize}
O mesmo que anástomos.
\section{Anástomos}
\begin{itemize}
\item {Grp. gram.:m. pl.}
\end{itemize}
\begin{itemize}
\item {Proveniência:(Do gr. \textunderscore ana\textunderscore  + \textunderscore stoma\textunderscore )}
\end{itemize}
Gênero de molluscos gasterópodes.
\section{Anastomosar}
\begin{itemize}
\item {Grp. gram.:v. t.}
\end{itemize}
Juntar por \textunderscore anastomose\textunderscore .
\section{Anastomose}
\begin{itemize}
\item {Grp. gram.:f.}
\end{itemize}
\begin{itemize}
\item {Utilização:Anat.}
\end{itemize}
\begin{itemize}
\item {Proveniência:(Gr. \textunderscore anastomosis\textunderscore )}
\end{itemize}
Designação, que se dá ao ponto, em que se abocam dois vasos ou canaes.
\section{Anastomótico}
\begin{itemize}
\item {Grp. gram.:adj.}
\end{itemize}
Relativo á anastomose; que fórma anastomose.
\section{Anástrabo}
\begin{itemize}
\item {Grp. gram.:m.}
\end{itemize}
Gênero de plantas escrofularíneas da África austral.
\section{Anástrofe}
\begin{itemize}
\item {Grp. gram.:f.}
\end{itemize}
\begin{itemize}
\item {Utilização:Rhet.}
\end{itemize}
\begin{itemize}
\item {Proveniência:(Gr. \textunderscore anastrophe\textunderscore )}
\end{itemize}
Inversão da ordem natural das palavras correlativas.
\section{Anastrofia}
\begin{itemize}
\item {Grp. gram.:f.}
\end{itemize}
Inversão esplânchnica, em Cirurgia.
(Cp. \textunderscore anástrophe\textunderscore )
\section{Anástrophe}
\begin{itemize}
\item {Grp. gram.:f.}
\end{itemize}
\begin{itemize}
\item {Utilização:Rhet.}
\end{itemize}
\begin{itemize}
\item {Proveniência:(Gr. \textunderscore anastrophe\textunderscore )}
\end{itemize}
Inversão da ordem natural das palavras correlativas.
\section{Anastrophia}
\begin{itemize}
\item {Grp. gram.:f.}
\end{itemize}
Inversão esplânchnica, em Cirurgia.
(Cp. \textunderscore anástrophe\textunderscore )
\section{Anatar}
\begin{itemize}
\item {Grp. gram.:v. t.}
\end{itemize}
Tornar semelhante á nata.
Cobrir de nata.
\section{Anateirado}
\begin{itemize}
\item {Grp. gram.:adj.}
\end{itemize}
Em que há nateiro; que tem vaza de alluvião:«\textunderscore Carcavelos, terrenos em parte anateirados...\textunderscore »Ferr. Lapa, \textunderscore Alm. do Lavrador\textunderscore , (1869), p. 16.
\section{Anátema}
\begin{itemize}
\item {Grp. gram.:m.}
\end{itemize}
\begin{itemize}
\item {Grp. gram.:Adj.}
\end{itemize}
\begin{itemize}
\item {Proveniência:(Gr. \textunderscore anathema\textunderscore )}
\end{itemize}
Excommunhão.
Maldicção.
Reprovação.
Excommungado; maldito.
\section{Anatematismo}
\begin{itemize}
\item {Grp. gram.:m.}
\end{itemize}
Acto solemne, que envolve \textunderscore anátema\textunderscore .
\section{Anatematização}
\begin{itemize}
\item {Grp. gram.:f.}
\end{itemize}
Acto de \textunderscore anatematizar\textunderscore .
\section{Anatematizador}
\begin{itemize}
\item {Grp. gram.:adj.}
\end{itemize}
Que anatematiza. Cf. Herculano, \textunderscore Opúsc.\textunderscore , III, 14.
\section{Anatematizar}
\begin{itemize}
\item {Grp. gram.:v. t.}
\end{itemize}
\begin{itemize}
\item {Proveniência:(Do gr. \textunderscore anathematizein\textunderscore )}
\end{itemize}
Excommungar; banir da communhão dos fiéis.
Condemnar; reprovar.
\section{Ânates}
\begin{itemize}
\item {Grp. gram.:m.}
\end{itemize}
Doença do ânus.
\section{Anáthema}
\begin{itemize}
\item {Grp. gram.:m.}
\end{itemize}
\begin{itemize}
\item {Grp. gram.:Adj.}
\end{itemize}
\begin{itemize}
\item {Proveniência:(Gr. \textunderscore anathema\textunderscore )}
\end{itemize}
Excommunhão.
Maldicção.
Reprovação.
Excommungado; maldito.
\section{Anathematismo}
\begin{itemize}
\item {Grp. gram.:m.}
\end{itemize}
Acto solemne, que envolve \textunderscore anáthema\textunderscore .
\section{Anathematização}
\begin{itemize}
\item {Grp. gram.:f.}
\end{itemize}
Acto de \textunderscore anathematizar\textunderscore .
\section{Anathematizador}
\begin{itemize}
\item {Grp. gram.:adj.}
\end{itemize}
Que anathematiza. Cf. Herculano, \textunderscore Opúsc.\textunderscore , III, 14.
\section{Anathematizar}
\begin{itemize}
\item {Grp. gram.:v. t.}
\end{itemize}
\begin{itemize}
\item {Proveniência:(Do gr. \textunderscore anathematizein\textunderscore )}
\end{itemize}
Excommungar; banir da communhão dos fiéis.
Condemnar; reprovar.
\section{Anati}
\begin{itemize}
\item {Grp. gram.:m.}
\end{itemize}
Árvore fructífera do Brasil.
Fruta dessa árvore.
\section{Anátides}
\begin{itemize}
\item {Grp. gram.:f. pl.}
\end{itemize}
\begin{itemize}
\item {Proveniência:(Do lat. \textunderscore anas\textunderscore , \textunderscore anatis\textunderscore  + gr. \textunderscore eidos\textunderscore )}
\end{itemize}
Fam. de aves palmípedes, que contém as espécies semelhantes ao pato.
\section{Anatifa}
\begin{itemize}
\item {Grp. gram.:f.}
\end{itemize}
O mesmo que \textunderscore anatifo\textunderscore . Cf. Fil. Simões, \textunderscore Cartas da Beiramar\textunderscore , 279.
\section{Anatifo}
\begin{itemize}
\item {Grp. gram.:m.}
\end{itemize}
\begin{itemize}
\item {Proveniência:(Do lat. \textunderscore anas\textunderscore  + \textunderscore ferre\textunderscore )}
\end{itemize}
Crustáceo cirrípede.
\section{Anatinas}
\begin{itemize}
\item {Grp. gram.:f. pl.}
\end{itemize}
\begin{itemize}
\item {Proveniência:(Do lat. \textunderscore anas\textunderscore , \textunderscore anatis\textunderscore )}
\end{itemize}
Gênero de molluscos acéphalos.
\section{Anatocismo}
\begin{itemize}
\item {Grp. gram.:m.}
\end{itemize}
\begin{itemize}
\item {Proveniência:(Gr. \textunderscore anatokismos\textunderscore )}
\end{itemize}
Capitalização dos juros de quantia emprestada.
\section{Anatólico}
\begin{itemize}
\item {Grp. gram.:m.}
\end{itemize}
Gênero de insectos coleópteros heterómeros.
\section{Anatomia}
\begin{itemize}
\item {Grp. gram.:f.}
\end{itemize}
\begin{itemize}
\item {Utilização:Fig.}
\end{itemize}
\begin{itemize}
\item {Proveniência:(Do gr. \textunderscore ana\textunderscore  + \textunderscore tome\textunderscore )}
\end{itemize}
Arte de dissecar as partes dos corpos organizados.
Doutrina da estructura das substâncias organizadas. Conjunto dos conhecimentos, resultantes da dissecção dos corpos organizados, especialmente do corpo humano.
Dissecação; autópsia.
Exame minucioso.
Anályse crítica.
\section{Anatomicamente}
\begin{itemize}
\item {Grp. gram.:adv.}
\end{itemize}
De modo \textunderscore anatómico\textunderscore .
\section{Anatómico}
\begin{itemize}
\item {Grp. gram.:adj.}
\end{itemize}
\begin{itemize}
\item {Grp. gram.:M.}
\end{itemize}
Relativo á \textunderscore anatomia\textunderscore .
O mesmo que \textunderscore anatomista\textunderscore .
\section{Anatomismo}
\begin{itemize}
\item {Grp. gram.:m.}
\end{itemize}
\begin{itemize}
\item {Utilização:Physiol.}
\end{itemize}
\begin{itemize}
\item {Proveniência:(De \textunderscore anatomia\textunderscore )}
\end{itemize}
Hypótese de que a estructura de determinadas partes explica physicamente os phenómenos vitaes que se dão nessas partes.
\section{Anatomista}
\begin{itemize}
\item {Grp. gram.:m.}
\end{itemize}
Aquelle que se occupa de anatomia.
\section{Anatomização}
\begin{itemize}
\item {Grp. gram.:f.}
\end{itemize}
Acto de \textunderscore anatomizar\textunderscore .
\section{Anatomizar}
\begin{itemize}
\item {Grp. gram.:v. t.}
\end{itemize}
\begin{itemize}
\item {Proveniência:(De \textunderscore anatomia\textunderscore )}
\end{itemize}
Dissecar.
Estudar minuciosamente.
\section{Anátomo-pathológico}
\begin{itemize}
\item {Grp. gram.:adj.}
\end{itemize}
\begin{itemize}
\item {Proveniência:(De \textunderscore anatómico\textunderscore  + \textunderscore pathológico\textunderscore )}
\end{itemize}
Que ao mesmo tempo diz respeito á anatomia e á pathologia.
\section{Anatripsia}
\begin{itemize}
\item {Grp. gram.:f.}
\end{itemize}
\begin{itemize}
\item {Utilização:Cir.}
\end{itemize}
O mesmo que \textunderscore fricção\textunderscore .
\section{Anatrípsico}
\begin{itemize}
\item {Grp. gram.:adj.}
\end{itemize}
O mesmo que \textunderscore anatríptico\textunderscore .
\section{Anatripsiologia}
\begin{itemize}
\item {Grp. gram.:f.}
\end{itemize}
\begin{itemize}
\item {Proveniência:(De \textunderscore anatripsia\textunderscore )}
\end{itemize}
Tratado sôbre as fricções.
\section{Anatríptico}
\begin{itemize}
\item {Grp. gram.:adj.}
\end{itemize}
Que serve para fricções.
\section{Anátropa}
\begin{itemize}
\item {Grp. gram.:f.}
\end{itemize}
\begin{itemize}
\item {Proveniência:(Gr. \textunderscore anatrope\textunderscore )}
\end{itemize}
Náuseas, ânsias.
\section{Anatropia}
\begin{itemize}
\item {Grp. gram.:f.}
\end{itemize}
Estado ou qualidade de \textunderscore anátropo\textunderscore .
\section{Anátropo}
\begin{itemize}
\item {Grp. gram.:adj.}
\end{itemize}
\begin{itemize}
\item {Utilização:Bot.}
\end{itemize}
\begin{itemize}
\item {Proveniência:(Do gr. \textunderscore anatrepein\textunderscore )}
\end{itemize}
Diz-se do óvulo vegetal que se encurva, de fórma que o micrópylo fica ao lado do hilo.
\section{Anavalhar}
\begin{itemize}
\item {Grp. gram.:v. t.}
\end{itemize}
Dar fórma de navalha a.
Ferir com a navalha.
\section{Anaxagórea}
\begin{itemize}
\item {Grp. gram.:f.}
\end{itemize}
\begin{itemize}
\item {Proveniência:(De \textunderscore Anaxágoras\textunderscore , n. p.)}
\end{itemize}
Gênero de plantas anonáceas da Ásia e da América.
\section{Anaxatre}
\begin{itemize}
\item {Grp. gram.:m.}
\end{itemize}
\begin{itemize}
\item {Utilização:Ant.}
\end{itemize}
O mesmo que \textunderscore ammoníaco\textunderscore .
\section{Anaz}
\begin{itemize}
\item {Grp. gram.:m.}
\end{itemize}
\begin{itemize}
\item {Proveniência:(Do gr. \textunderscore anax\textunderscore )}
\end{itemize}
Insecto neuróptero.
\section{Anãzado}
\begin{itemize}
\item {Grp. gram.:adj.}
\end{itemize}
\begin{itemize}
\item {Utilização:Prov.}
\end{itemize}
\begin{itemize}
\item {Proveniência:(De \textunderscore anãzar-se\textunderscore )}
\end{itemize}
Que parece anão; que tem pequena estatura.
\section{Anãzar-se}
\begin{itemize}
\item {Grp. gram.:v. p.}
\end{itemize}
Tornar-se anão.
Apoucar-se. Cf. Cortesão, \textunderscore Subs.\textunderscore 
\section{Anazótico}
\begin{itemize}
\item {Grp. gram.:adj.}
\end{itemize}
\begin{itemize}
\item {Utilização:Chím.}
\end{itemize}
\begin{itemize}
\item {Proveniência:(De \textunderscore ana\textunderscore  priv. + \textunderscore azote\textunderscore )}
\end{itemize}
Diz-se dos corpos que não são azotados.
\section{Anazoturia}
\begin{itemize}
\item {Grp. gram.:f.}
\end{itemize}
\begin{itemize}
\item {Utilização:Med.}
\end{itemize}
Desapparecimento total ou parcial da ureia da urina, em certas doenças.
\section{Anca}
\begin{itemize}
\item {Grp. gram.:f.}
\end{itemize}
\begin{itemize}
\item {Utilização:Des.}
\end{itemize}
Cada uma das proeminências lateraes do corpo humano, da cintura ás coxas; cadeiras; quadril; nádega.
Garupa (nos animaes).
O mesmo que \textunderscore ventre\textunderscore :«\textunderscore deu-me setenta almas geradas da minha anca\textunderscore ». Usque, \textunderscore Tribulações\textunderscore , 8.
\section{...ança}
\begin{itemize}
\item {Grp. gram.:suf.}
\end{itemize}
(indicativo de estado, qualidade, etc.)
\section{Ancado}
\begin{itemize}
\item {Grp. gram.:m.}
\end{itemize}
\begin{itemize}
\item {Grp. gram.:Adj.}
\end{itemize}
Designação antiga de uma doença cavallar, que consiste na contracção dos tendões e músculos, com insensibilidade.
Diz-se do solípede, cujos membros, desviados da sua verdadeira direcção, fazem que êlle fique mais inclinado para deante. Cf. Léon, \textunderscore Arte de Ferrar\textunderscore , 152.
\section{Ançarinha}
\begin{itemize}
\item {Grp. gram.:f.}
\end{itemize}
\begin{itemize}
\item {Utilização:Ant.}
\end{itemize}
O mesmo que \textunderscore cicuta\textunderscore .
(Provavelmente o mesmo que \textunderscore anserina\textunderscore ).
\section{Ancathia}
\begin{itemize}
\item {Grp. gram.:f.}
\end{itemize}
\begin{itemize}
\item {Proveniência:(Gr. \textunderscore ankathia\textunderscore )}
\end{itemize}
Planta asiática, da fam. das compostas.
\section{Ancatia}
\begin{itemize}
\item {Grp. gram.:f.}
\end{itemize}
\begin{itemize}
\item {Proveniência:(Gr. \textunderscore ankathia\textunderscore )}
\end{itemize}
Planta asiática, da fam. das compostas.
\section{Ancear}
\textunderscore v. t.\textunderscore  (e der.)
(V. \textunderscore ansiar\textunderscore , etc.)
\section{Anceias}
\begin{itemize}
\item {Grp. gram.:f. pl.}
\end{itemize}
\begin{itemize}
\item {Proveniência:(De \textunderscore Anceia\textunderscore , n. p.)}
\end{itemize}
Gênero de crustáceos isópodes.
\section{Anceio}
\begin{itemize}
\item {Grp. gram.:m.}
\end{itemize}
(V.anseio)
\section{Ancestral}
\begin{itemize}
\item {Grp. gram.:adj.}
\end{itemize}
\begin{itemize}
\item {Utilização:Gal}
\end{itemize}
\begin{itemize}
\item {Proveniência:(Do fr. ant. \textunderscore ancestre\textunderscore )}
\end{itemize}
Relativo a antecessores, a antepassados; avito.
Antigo.
\section{Anchamente}
\begin{itemize}
\item {Grp. gram.:adv.}
\end{itemize}
De modo \textunderscore ancho\textunderscore , com largueza. Cf. F. Recreio, \textunderscore Bat. de Ourique\textunderscore .
\section{Anchão}
\begin{itemize}
\item {Grp. gram.:m.}
\end{itemize}
\begin{itemize}
\item {Utilização:T. de Gôa}
\end{itemize}
\begin{itemize}
\item {Proveniência:(De \textunderscore ancho\textunderscore )}
\end{itemize}
O mesmo que \textunderscore boião\textunderscore .
\section{Anchílops}
\begin{itemize}
\item {fónica:quí}
\end{itemize}
\begin{itemize}
\item {Grp. gram.:m.}
\end{itemize}
\begin{itemize}
\item {Proveniência:(Gr. \textunderscore ankhilops\textunderscore )}
\end{itemize}
Pequeno tumor no ângulo interior do ôlho.
\section{Ancho}
\begin{itemize}
\item {Grp. gram.:adj.}
\end{itemize}
Largo, amplo.
Vaidoso.
(Cast. \textunderscore ancho\textunderscore )
\section{Anchova}
\begin{itemize}
\item {fónica:xô}
\end{itemize}
\begin{itemize}
\item {Grp. gram.:f.}
\end{itemize}
Pequeno peixe, da ordem dos malacopterýgios.
(Cast. \textunderscore anchoa\textunderscore )
\section{Anchura}
\begin{itemize}
\item {Grp. gram.:f.}
\end{itemize}
\begin{itemize}
\item {Utilização:Ant.}
\end{itemize}
\begin{itemize}
\item {Proveniência:(De \textunderscore ancho\textunderscore )}
\end{itemize}
Largura.
\section{Ância}
\textunderscore f.\textunderscore  (e der.)
(V. \textunderscore ânsia\textunderscore , etc.)
\section{...ância}
\begin{itemize}
\item {Grp. gram.:suf.}
\end{itemize}
(design. de qualidade, continuação, etc.)
\section{Anciã}
(fem. de \textunderscore ancião\textunderscore )
\section{Ancian}
(fem. de \textunderscore ancião\textunderscore )
\section{Anciania}
\begin{itemize}
\item {Grp. gram.:f.}
\end{itemize}
O mesmo que \textunderscore ancianidade\textunderscore .
\section{Ancianidade}
\begin{itemize}
\item {Grp. gram.:f.}
\end{itemize}
Qualidade de ancião.
Antiguidade.
\section{Anciano}
\begin{itemize}
\item {Grp. gram.:adj.}
\end{itemize}
\begin{itemize}
\item {Utilização:Ant.}
\end{itemize}
\begin{itemize}
\item {Grp. gram.:M.}
\end{itemize}
Antigo:«\textunderscore nos tempos mais ancianos...\textunderscore »G. Vicente.
O mesmo que \textunderscore ancião\textunderscore . Cf. Usque, \textunderscore Tribulações\textunderscore , 21.
\section{Ancião}
\begin{itemize}
\item {Grp. gram.:m.}
\end{itemize}
\begin{itemize}
\item {Grp. gram.:Adj.}
\end{itemize}
\begin{itemize}
\item {Proveniência:(Do lat. hyp. \textunderscore antianus\textunderscore )}
\end{itemize}
Homem muito velho.
Antigo; velho.
\section{Ancil}
\begin{itemize}
\item {Grp. gram.:m.}
\end{itemize}
\begin{itemize}
\item {Proveniência:(Lat. \textunderscore ancile\textunderscore )}
\end{itemize}
Pequeno escudo oval, de bronze, usado pelos Romanos.
\section{Ancila}
\begin{itemize}
\item {Grp. gram.:f.}
\end{itemize}
\begin{itemize}
\item {Utilização:Fig.}
\end{itemize}
\begin{itemize}
\item {Proveniência:(Lat. \textunderscore ancilla\textunderscore )}
\end{itemize}
Escrava, serva.
Coisa que serve de auxílio ou subsídio a outra.
\section{Ancile}
\begin{itemize}
\item {Grp. gram.:m.}
\end{itemize}
O mesmo que \textunderscore ancil\textunderscore . Cf. Castilho, \textunderscore Fastos\textunderscore , III, 494.
\section{Ancílio}
\begin{itemize}
\item {Grp. gram.:m.}
\end{itemize}
O mesmo que \textunderscore ancil\textunderscore .
\section{Ancilla}
\begin{itemize}
\item {Grp. gram.:f.}
\end{itemize}
\begin{itemize}
\item {Utilização:Fig.}
\end{itemize}
\begin{itemize}
\item {Proveniência:(Lat. \textunderscore ancilla\textunderscore )}
\end{itemize}
Escrava, serva.
Coisa que serve de auxílio ou subsídio a outra.
\section{Ancilo}
\begin{itemize}
\item {Grp. gram.:m.}
\end{itemize}
O mesmo que \textunderscore ancil\textunderscore .
\section{Anciloblefaria}
\begin{itemize}
\item {Grp. gram.:f.}
\end{itemize}
\begin{itemize}
\item {Utilização:Med.}
\end{itemize}
\begin{itemize}
\item {Proveniência:(De \textunderscore ancyloblépharo\textunderscore )}
\end{itemize}
Juncção pathológica, mais ou menos completa, das pálpebras.
\section{Ancilobléfaro}
\begin{itemize}
\item {Grp. gram.:adj.}
\end{itemize}
\begin{itemize}
\item {Proveniência:(Do gr. \textunderscore ankulos\textunderscore  + \textunderscore blepharon\textunderscore )}
\end{itemize}
Que tem anciloblefaria.
\section{Ancilóceras}
\begin{itemize}
\item {Grp. gram.:f. pl.}
\end{itemize}
Gênero fóssil de molluscos cephalópodes.
\section{Ancilócero}
\begin{itemize}
\item {Grp. gram.:m.}
\end{itemize}
\begin{itemize}
\item {Proveniência:(Do gr. \textunderscore ankulos\textunderscore  + \textunderscore keras\textunderscore )}
\end{itemize}
Gênero de insectos coleópteros.
\section{Anciloglosse}
\begin{itemize}
\item {Grp. gram.:f.}
\end{itemize}
\begin{itemize}
\item {Utilização:Med.}
\end{itemize}
\begin{itemize}
\item {Proveniência:(Do gr. \textunderscore ankulos\textunderscore  + \textunderscore glossa\textunderscore )}
\end{itemize}
Falta de movimento na língua, pela extensão do ligamento.
\section{Anciloide}
\begin{itemize}
\item {Grp. gram.:adj.}
\end{itemize}
\begin{itemize}
\item {Proveniência:(Do gr. \textunderscore ankulos\textunderscore  + \textunderscore eidos\textunderscore )}
\end{itemize}
Que tem fórma de colchete ou gancho.
\section{Ancilosar}
\begin{itemize}
\item {Grp. gram.:v. t.}
\end{itemize}
Causar ancilose a.
\section{Ancilose}
\begin{itemize}
\item {Grp. gram.:f.}
\end{itemize}
\begin{itemize}
\item {Utilização:Med.}
\end{itemize}
\begin{itemize}
\item {Proveniência:(Gr. \textunderscore ankulosis\textunderscore )}
\end{itemize}
Falta de movimento em articulação.
\section{Ancilóstomo}
\begin{itemize}
\item {Grp. gram.:m.}
\end{itemize}
\begin{itemize}
\item {Proveniência:(Do gr. \textunderscore ankulos\textunderscore  + \textunderscore stoma\textunderscore )}
\end{itemize}
Helmintho, próprio da espécie humana.
\section{Ancilótomo}
\begin{itemize}
\item {Grp. gram.:m.}
\end{itemize}
\begin{itemize}
\item {Proveniência:(Do gr. \textunderscore ankulos\textunderscore  + \textunderscore tome\textunderscore )}
\end{itemize}
Qualquer instrumento cortante e recurvo.
\section{Ancinhar}
\begin{itemize}
\item {Grp. gram.:v. t.}
\end{itemize}
\begin{itemize}
\item {Utilização:Prov.}
\end{itemize}
Limpar com ancinho.
\section{Ancinho}
\begin{itemize}
\item {Grp. gram.:m.}
\end{itemize}
\begin{itemize}
\item {Utilização:Prov.}
\end{itemize}
\begin{itemize}
\item {Proveniência:(Do lat. \textunderscore uncinus\textunderscore )}
\end{itemize}
Instrumento agricola, dentado, para juntar palha e para outros usos.
Rede de suspensão, que se emprega principalmente na pesca do berbigão.
\section{Ancípite}
\begin{itemize}
\item {Grp. gram.:adj.}
\end{itemize}
\begin{itemize}
\item {Utilização:Gram.}
\end{itemize}
\begin{itemize}
\item {Utilização:Poét.}
\end{itemize}
Hesitante, duvidoso, vacillante.
Diz-se das consoantes \textunderscore l\textunderscore  e \textunderscore r\textunderscore , em que, havendo contacto imperfeito dos órgãos factores, estes interceptam completamente a passagem do ar em um ponto e a deixam leve em outro.
Que tem duas cabeças ou duas faces.
Que tem dois gumes:«\textunderscore nossa ancípite frâncica arrojámos\textunderscore ». Filinto, XIV, 231.
\section{Anco}
\begin{itemize}
\item {Grp. gram.:m.}
\end{itemize}
Cotovelo ou enseada na costa.
(B. lat. \textunderscore ancus\textunderscore )
\section{Ancobrir}
\begin{itemize}
\item {Grp. gram.:v. t.}
\end{itemize}
Fórma antiga de \textunderscore encobrir\textunderscore .
\section{Ancólia}
\begin{itemize}
\item {Grp. gram.:f.}
\end{itemize}
\begin{itemize}
\item {Proveniência:(Fr. \textunderscore ancolie\textunderscore )}
\end{itemize}
O mesmo que \textunderscore aquilégia\textunderscore .
\section{Ancóneo}
\begin{itemize}
\item {Grp. gram.:adj.}
\end{itemize}
\begin{itemize}
\item {Proveniência:(Do gr. \textunderscore ankon\textunderscore )}
\end{itemize}
Diz-se dos músculos da parte posterior e superior do antebraço.
\section{Âncora}
\begin{itemize}
\item {Grp. gram.:f.}
\end{itemize}
\begin{itemize}
\item {Utilização:Constr.}
\end{itemize}
\begin{itemize}
\item {Proveniência:(Lat. \textunderscore ancora\textunderscore )}
\end{itemize}
Instrumento de ferro que, lançado no fundo da água, segura as embarcações por um cabo a que está prêso.
Peça de ligação, geralmente de ferro.
\section{Ancoração}
\begin{itemize}
\item {Grp. gram.:f.}
\end{itemize}
Acto de ancorar.
Ancoradoiro.
\section{Ancoradoiro}
\begin{itemize}
\item {Grp. gram.:m.}
\end{itemize}
Lugar, onde o navio lança âncora.
\section{Ancoradouro}
\begin{itemize}
\item {Grp. gram.:m.}
\end{itemize}
Lugar, onde o navio lança âncora.
\section{Ancoragem}
\begin{itemize}
\item {Grp. gram.:f.}
\end{itemize}
\begin{itemize}
\item {Utilização:Ant.}
\end{itemize}
Acto de \textunderscore ancorar\textunderscore .
O mesmo que \textunderscore ancoradoiro\textunderscore .
\section{Ancorar}
\begin{itemize}
\item {Grp. gram.:v. i.}
\end{itemize}
Lançar âncora; fundear.
Aportar.
\section{Ancoreta}
\begin{itemize}
\item {fónica:co-rê}
\end{itemize}
\begin{itemize}
\item {Grp. gram.:f.}
\end{itemize}
Pequena âncora.
Pequeno barril chato, usado especialmente a bordo de navios.--São definições correntes em diccion. port.; affirmam porém officiaes de marinha que nunca ouviram \textunderscore ancoreta\textunderscore , no sentido de pequena âncora.
\section{Ancorote}
\begin{itemize}
\item {Grp. gram.:m.}
\end{itemize}
Pequena âncora.
\section{Ancova}
\begin{itemize}
\item {Grp. gram.:f.}
\end{itemize}
Língua falada em Madagáscar.
\section{Ancubi}
\begin{itemize}
\item {Grp. gram.:m.}
\end{itemize}
Ave africana do Humbe.
\section{Ancyloblepharia}
\begin{itemize}
\item {Grp. gram.:f.}
\end{itemize}
\begin{itemize}
\item {Utilização:Med.}
\end{itemize}
\begin{itemize}
\item {Proveniência:(De \textunderscore ancyloblépharo\textunderscore )}
\end{itemize}
Juncção pathológica, mais ou menos completa, das pálpebras.
\section{Ancyloblépharo}
\begin{itemize}
\item {Grp. gram.:adj.}
\end{itemize}
\begin{itemize}
\item {Proveniência:(Do gr. \textunderscore ankulos\textunderscore  + \textunderscore blepharon\textunderscore )}
\end{itemize}
Que tem ancyloblepharia.
\section{Ancylóceras}
\begin{itemize}
\item {Grp. gram.:f. pl.}
\end{itemize}
Gênero fóssil de molluscos cephalópodes.
\section{Ancylócero}
\begin{itemize}
\item {Grp. gram.:m.}
\end{itemize}
\begin{itemize}
\item {Proveniência:(Do gr. \textunderscore ankulos\textunderscore  + \textunderscore keras\textunderscore )}
\end{itemize}
Gênero de insectos coleópteros.
\section{Ancyloglosse}
\begin{itemize}
\item {Grp. gram.:f.}
\end{itemize}
\begin{itemize}
\item {Utilização:Med.}
\end{itemize}
\begin{itemize}
\item {Proveniência:(Do gr. \textunderscore ankulos\textunderscore  + \textunderscore glossa\textunderscore )}
\end{itemize}
Falta de movimento na língua, pela extensão do ligamento.
\section{Ancyloide}
\begin{itemize}
\item {Grp. gram.:adj.}
\end{itemize}
\begin{itemize}
\item {Proveniência:(Do gr. \textunderscore ankulos\textunderscore  + \textunderscore eidos\textunderscore )}
\end{itemize}
Que tem fórma de colchete ou gancho.
\section{Ancylosar}
\begin{itemize}
\item {Grp. gram.:v. t.}
\end{itemize}
Causar ancylose a.
\section{Ancylose}
\begin{itemize}
\item {Grp. gram.:f.}
\end{itemize}
\begin{itemize}
\item {Utilização:Med.}
\end{itemize}
\begin{itemize}
\item {Proveniência:(Gr. \textunderscore ankulosis\textunderscore )}
\end{itemize}
Falta de movimento em articulação.
\section{Ancylóstomo}
\begin{itemize}
\item {Grp. gram.:m.}
\end{itemize}
\begin{itemize}
\item {Proveniência:(Do gr. \textunderscore ankulos\textunderscore  + \textunderscore stoma\textunderscore )}
\end{itemize}
Helmintho, próprio da espécie humana.
\section{Ancylótomo}
\begin{itemize}
\item {Grp. gram.:m.}
\end{itemize}
\begin{itemize}
\item {Proveniência:(Do gr. \textunderscore ankulos\textunderscore  + \textunderscore tome\textunderscore )}
\end{itemize}
Qualquer instrumento cortante e recurvo.
\section{Andá}
\begin{itemize}
\item {Grp. gram.:f.}
\end{itemize}
Árvore euphorbiácea do Brasil.
\section{Anda-açu}
\begin{itemize}
\item {Grp. gram.:m.}
\end{itemize}
Planta oleaginosa do Brasil.
\section{Andábata}
\begin{itemize}
\item {Grp. gram.:m.}
\end{itemize}
\begin{itemize}
\item {Proveniência:(Lat. \textunderscore andabata\textunderscore )}
\end{itemize}
Gladiador que, entre os Romanos, combatia de olhos vendados.
\section{Andaço}
\begin{itemize}
\item {Grp. gram.:m.}
\end{itemize}
\begin{itemize}
\item {Utilização:T. de Estarreja}
\end{itemize}
\begin{itemize}
\item {Proveniência:(De \textunderscore andar\textunderscore )}
\end{itemize}
Doença, que está predominando numa localidade.
Pequena epidemia.
Vaga ou onda, que se levanta imprevista.
\section{Andada}
\begin{itemize}
\item {Grp. gram.:f.}
\end{itemize}
Acto de \textunderscore andar\textunderscore .
Caminhada.
\section{Andadeira}
\begin{itemize}
\item {Grp. gram.:f.}
\end{itemize}
\begin{itemize}
\item {Utilização:Ant.}
\end{itemize}
\begin{itemize}
\item {Utilização:Prov.}
\end{itemize}
\begin{itemize}
\item {Utilização:trasm.}
\end{itemize}
\begin{itemize}
\item {Utilização:Prov.}
\end{itemize}
\begin{itemize}
\item {Utilização:Prov.}
\end{itemize}
\begin{itemize}
\item {Utilização:trasm.}
\end{itemize}
O mesmo que \textunderscore almocreveria\textunderscore .
Brinquedo de rapazes, que consta de uma cana, que tem na extremidade um pauzinho atravessado, o qual, munido de asas de papel, gira, quando se expõe ao vento.
A mó que gira e em cujo centro cai o grão que vem do quelho.
Rapariga, que trabalha com a dobadoira.
Cavalgadura ligeira:«\textunderscore dando de espora á andadeira\textunderscore ». Camillo, \textunderscore Brasileira\textunderscore , 67.
\section{Andadeiras}
\begin{itemize}
\item {Grp. gram.:m. pl.}
\end{itemize}
\begin{itemize}
\item {Proveniência:(De \textunderscore andar\textunderscore )}
\end{itemize}
Tiras de pano, com que se seguram as crianças pela cintura, para as ensinar a andar.
\section{Andadeiro}
\begin{itemize}
\item {Grp. gram.:adj.}
\end{itemize}
Que anda muito.
Em que se póde andar facilmente.
\section{Andador}
\begin{itemize}
\item {Grp. gram.:m.}
\end{itemize}
\begin{itemize}
\item {Grp. gram.:Adj.}
\end{itemize}
Moço de recados.
Aquelle que pede, de porta em porta, esmolas para suffragar as almas do Purgatório.
O mesmo que \textunderscore andadeiro\textunderscore .
(B. lat. \textunderscore andator\textunderscore )
\section{Andadoria}
\begin{itemize}
\item {Grp. gram.:f.}
\end{itemize}
Cargo de andador.
\section{Andadura}
\begin{itemize}
\item {Grp. gram.:f.}
\end{itemize}
Modo de andar.
\section{Andagem}
\begin{itemize}
\item {Grp. gram.:f.}
\end{itemize}
\begin{itemize}
\item {Utilização:Ant.}
\end{itemize}
\begin{itemize}
\item {Proveniência:(De \textunderscore andar\textunderscore )}
\end{itemize}
Casa de um só pavimento ou andar.
\section{Andaia}
\begin{itemize}
\item {Grp. gram.:f.}
\end{itemize}
\begin{itemize}
\item {Utilização:Açor}
\end{itemize}
Producto da destillação de vinho, de graduação mais baixa que a aguardente.
\section{Andaia-açu}
\begin{itemize}
\item {Grp. gram.:f.}
\end{itemize}
Espécie de palmeira do Brasil.
\section{Andaida}
\begin{itemize}
\item {Grp. gram.:f.}
\end{itemize}
\begin{itemize}
\item {Utilização:Ant.}
\end{itemize}
O mesmo que \textunderscore andaina\textunderscore .
\section{Andaimada}
\begin{itemize}
\item {Grp. gram.:f.}
\end{itemize}
O mesmo que \textunderscore andaimaria\textunderscore .
\section{Andaimar}
\begin{itemize}
\item {Grp. gram.:v. t.}
\end{itemize}
\begin{itemize}
\item {Utilização:Neol.}
\end{itemize}
\begin{itemize}
\item {Proveniência:(De \textunderscore andaime\textunderscore )}
\end{itemize}
Preparar a construcção ou a formação de: \textunderscore andaimar uma teoria\textunderscore .
\section{Andaimaria}
\begin{itemize}
\item {Grp. gram.:f.}
\end{itemize}
Conjunto de andaimes.
\section{Andaime}
\begin{itemize}
\item {Grp. gram.:f.}
\end{itemize}
\begin{itemize}
\item {Utilização:Ant.}
\end{itemize}
\begin{itemize}
\item {Proveniência:(Do cast. \textunderscore andamio\textunderscore )}
\end{itemize}
Estrado provisório de madeira, sôbre que trabalham os operários de construcções altas; bailéu.
Galeria alta e exterior de fortaleza ou de outro edificio.
\section{Andaimo}
\begin{itemize}
\item {Grp. gram.:m.}
\end{itemize}
O mesmo que \textunderscore andaime\textunderscore .
\section{Andaina}
\begin{itemize}
\item {Grp. gram.:f.}
\end{itemize}
\begin{itemize}
\item {Proveniência:(De \textunderscore andar\textunderscore ?)}
\end{itemize}
Fileira.
Conjunto de peças de vestuário.
Conjunto de velas da embarcação.
Linha de meios, nas salinas.
Embarcação algarvia, para transporte de pesca, o mesmo que \textunderscore enviada\textunderscore .
\section{Andala}
\begin{itemize}
\item {Grp. gram.:f.}
\end{itemize}
Fôlha grande de palmeira, com que se cobrem cubatas.
\section{Andalim}
\begin{itemize}
\item {Grp. gram.:m.}
\end{itemize}
Espécie de sarguça.
\section{Andaluz}
\begin{itemize}
\item {Grp. gram.:adj.}
\end{itemize}
\begin{itemize}
\item {Grp. gram.:M.}
\end{itemize}
Relativo á Andaluzia.
Homem natural da Andaluzia.
\section{Andaluzita}
\begin{itemize}
\item {Grp. gram.:f.}
\end{itemize}
\begin{itemize}
\item {Proveniência:(De \textunderscore Andaluzia\textunderscore , n. p.)}
\end{itemize}
Mineral, composto de silicato de alumina, combinado com um silicato múltiplo de potassa, cal, magnésia, etc.
\section{Andaluzite}
\begin{itemize}
\item {Grp. gram.:f.}
\end{itemize}
\begin{itemize}
\item {Proveniência:(De \textunderscore Andaluzia\textunderscore , n. p.)}
\end{itemize}
Mineral, composto de silicato de alumina, combinado com um silicato múltiplo de potassa, cal, magnésia, etc.
\section{Andame}
\begin{itemize}
\item {Grp. gram.:m.}
\end{itemize}
O mesmo que \textunderscore andaime\textunderscore .
\section{Andamento}
\begin{itemize}
\item {Grp. gram.:m.}
\end{itemize}
\begin{itemize}
\item {Utilização:Mús.}
\end{itemize}
Acto, modo, de \textunderscore andar\textunderscore .
Proseguimento.
Movimento regular.
\section{Andâmio}
\begin{itemize}
\item {Grp. gram.:m.}
\end{itemize}
\begin{itemize}
\item {Utilização:Ant.}
\end{itemize}
O mesmo que \textunderscore andaime\textunderscore . Cf. Usque, \textunderscore Tribulações\textunderscore , 9.
\section{Andamo}
\begin{itemize}
\item {Grp. gram.:m.}
\end{itemize}
\begin{itemize}
\item {Utilização:Ant.}
\end{itemize}
\begin{itemize}
\item {Proveniência:(De \textunderscore andar\textunderscore . Cp. \textunderscore andaime\textunderscore )}
\end{itemize}
Atalho, carreiro.
\section{Andança}
\begin{itemize}
\item {Grp. gram.:f.}
\end{itemize}
\begin{itemize}
\item {Utilização:Ant.}
\end{itemize}
\begin{itemize}
\item {Utilização:Fam.}
\end{itemize}
\begin{itemize}
\item {Proveniência:(De \textunderscore andar\textunderscore )}
\end{itemize}
O mesmo que \textunderscore andadura\textunderscore .
Faina, lida:«\textunderscore em que andanças envolvem Christo\textunderscore ».
Camillo, \textunderscore Caveira\textunderscore , 461.
\section{Andante}
\begin{itemize}
\item {Grp. gram.:m.}
\end{itemize}
\begin{itemize}
\item {Grp. gram.:Adj.}
\end{itemize}
\begin{itemize}
\item {Utilização:Heráld.}
\end{itemize}
O mesmo que \textunderscore andador\textunderscore .
Que anda.
Vagabundo.
Diz-se do animal, que no campo do escudo se apresenta caminhando.
\section{Andante}
\begin{itemize}
\item {Grp. gram.:m.}
\end{itemize}
\begin{itemize}
\item {Proveniência:(T. it.)}
\end{itemize}
Trecho musical, que não deve ser executado muito depressa, nem muito devagar, mas, sim, entre \textunderscore adagio\textunderscore  e \textunderscore allegro\textunderscore .
\section{Andantesco}
\begin{itemize}
\item {Grp. gram.:adj.}
\end{itemize}
Relativo á cavallaria andante. Cavalleiroso. Cf. Latino, \textunderscore Camões\textunderscore , 270, 271 e 272.
\section{Andantino}
\begin{itemize}
\item {Grp. gram.:m.}
\end{itemize}
\begin{itemize}
\item {Proveniência:(T. it.)}
\end{itemize}
Trecho musical, que se deve executar mais lentamente que o \textunderscore andante\textunderscore .
\section{Andapé}
\begin{itemize}
\item {Grp. gram.:m.}
\end{itemize}
\begin{itemize}
\item {Utilização:Prov.}
\end{itemize}
Andaime baixo, que se poisa no chão; espécie de tarima.
(Colhido em Turquel)
\section{Andaquira}
\begin{itemize}
\item {Grp. gram.:f.}
\end{itemize}
\begin{itemize}
\item {Utilização:Bras}
\end{itemize}
Cera especial, fabricada pela mellipona.
\section{Andar}
\begin{itemize}
\item {Grp. gram.:v. i.}
\end{itemize}
\begin{itemize}
\item {Grp. gram.:V. p.}
\end{itemize}
\begin{itemize}
\item {Grp. gram.:M.}
\end{itemize}
Dar passos, caminhar: \textunderscore andei hoje muito\textunderscore .
Divagar.
Mover-se: \textunderscore o navio anda muito\textunderscore .
Decorrer.
Passar a vida: \textunderscore vamos andando\textunderscore .
Trabalhar, têr occupação: \textunderscore ando na escola\textunderscore .
Proceder: \textunderscore o irmão tem andado mal\textunderscore .
Computar-se: \textunderscore a minha despesa diária anda por 3$000 reis\textunderscore .
Mostrar-se:«\textunderscore andam êsses caminhos cheios de povo\textunderscore ». Camillo, \textunderscore Retr. de Ricard.\textunderscore , 139.
Sentir-se: \textunderscore ando-me satisfeito\textunderscore . Cf. Castilho, \textunderscore Fausto\textunderscore , 18 e 38.
Andadura: \textunderscore o seu andar é elegante\textunderscore .
Pavimento de uma casa: \textunderscore tem cinco andares a casa\textunderscore .
Camada. Cf. \textunderscore Museu Techn.\textunderscore , 36.
(B. lat. \textunderscore andare\textunderscore )
\section{Andarego}
\begin{itemize}
\item {fónica:da-ré}
\end{itemize}
\begin{itemize}
\item {Grp. gram.:adj.}
\end{itemize}
\begin{itemize}
\item {Utilização:Prov.}
\end{itemize}
\begin{itemize}
\item {Utilização:trasm.}
\end{itemize}
Que anda bem.
Ligeiro.
(Cast. \textunderscore andariego\textunderscore )
\section{Andaresa}
\begin{itemize}
\item {Grp. gram.:f.}
\end{itemize}
\begin{itemize}
\item {Proveniência:(T. malgache)}
\end{itemize}
Arbusto verbernáceo da Índia.
\section{Andarilhar}
\begin{itemize}
\item {Grp. gram.:v. i.}
\end{itemize}
Servir de andarilho.
Vaguear. Cf. Gama, \textunderscore Segr. do Ab.\textunderscore , 241.
\section{Andarilho}
\begin{itemize}
\item {Grp. gram.:m.}
\end{itemize}
\begin{itemize}
\item {Utilização:Ant.}
\end{itemize}
\begin{itemize}
\item {Proveniência:(De \textunderscore andar\textunderscore )}
\end{itemize}
O mesmo que \textunderscore andadeiro\textunderscore .
Aquelle que leva cartas ou notícias.
Lacaio, que a pé acompanhava os amos, que iam de carro ou a cavallo.
Rapazito, que, nas toiradas, depois de corrido o animal, percorre a arena, para recolher farpas caídas ou outros objectos.
\section{Andarim}
\begin{itemize}
\item {Grp. gram.:m.}
\end{itemize}
\begin{itemize}
\item {Utilização:Des.}
\end{itemize}
O mesmo que \textunderscore andarilho\textunderscore .
Cf. \textunderscore Anat. Joc.\textunderscore , 13.
\section{Andarovel}
\begin{itemize}
\item {Grp. gram.:m.}
\end{itemize}
\begin{itemize}
\item {Utilização:Ant.}
\end{itemize}
O mesmo que \textunderscore andrebello\textunderscore .
\section{Andas}
\begin{itemize}
\item {Grp. gram.:f. pl.}
\end{itemize}
\begin{itemize}
\item {Utilização:Ant.}
\end{itemize}
\begin{itemize}
\item {Proveniência:(Do lat. \textunderscore amites\textunderscore )}
\end{itemize}
Pernas ou muletas de pau, que têm um estribo ou resalto, em que se apoiam os pés.
Espécie de cama ou liteira, sóbre varaes.
Varaes, em que se colloca a tumba.
Charola.
\section{Andavaes}
\begin{itemize}
\item {Grp. gram.:m. pl.}
\end{itemize}
\begin{itemize}
\item {Utilização:Prov.}
\end{itemize}
Esqueleto desconjuntado de animal, cuja carne foi devorada por cães ou lobos.
(Colhido em Turquel)
\section{Andavais}
\begin{itemize}
\item {Grp. gram.:m. pl.}
\end{itemize}
\begin{itemize}
\item {Utilização:Prov.}
\end{itemize}
Esqueleto desconjuntado de animal, cuja carne foi devorada por cães ou lobos.
(Colhido em Turquel)
\section{Andeiro}
\begin{itemize}
\item {Grp. gram.:adj.}
\end{itemize}
Que anda bem.
\section{Andejar}
\begin{itemize}
\item {Grp. gram.:v. i.}
\end{itemize}
Andar ao acaso; vaguear; sêr andejo.
\section{Andejo}
\begin{itemize}
\item {Grp. gram.:adj.}
\end{itemize}
\begin{itemize}
\item {Proveniência:(De \textunderscore andar\textunderscore )}
\end{itemize}
Andeiro.
Erradio.
Que anda muito na rua, que não pára em casa: \textunderscore aquella mulher é muito andeja\textunderscore .
\section{Andersónia}
\begin{itemize}
\item {Grp. gram.:f.}
\end{itemize}
\begin{itemize}
\item {Proveniência:(De \textunderscore Anderson\textunderscore , n. p.)}
\end{itemize}
Designação de vários gêneros de plantas, pertencentes a fam. diversas.
\section{Andesina}
\begin{itemize}
\item {Grp. gram.:f.}
\end{itemize}
Feldspatho dos Andes.
\section{Andiche}
\begin{itemize}
\item {Grp. gram.:m.}
\end{itemize}
O mesmo que \textunderscore endiche\textunderscore .
\section{Andicola}
\begin{itemize}
\item {Grp. gram.:adj.}
\end{itemize}
Que cresce nos Andes.
Que habita nos Andes.
\section{Andilhas}
\begin{itemize}
\item {Grp. gram.:f. pl.}
\end{itemize}
\begin{itemize}
\item {Proveniência:(De \textunderscore andas\textunderscore )}
\end{itemize}
Cadeirinha, armação de madeira, que ampara sôbre a cavalgadura quem monta sentado.
\section{Andim}
\begin{itemize}
\item {Grp. gram.:m.}
\end{itemize}
\begin{itemize}
\item {Utilização:T. de San-Thomé}
\end{itemize}
O mesmo que \textunderscore desdém\textunderscore .
\section{Andino}
\begin{itemize}
\item {Grp. gram.:adj.}
\end{itemize}
O mesmo que \textunderscore andicola\textunderscore .
Relativo aos Andes.
\section{Andira}
\begin{itemize}
\item {Grp. gram.:f.}
\end{itemize}
Gênero de plantas leguminosas.
\section{Andirá}
\begin{itemize}
\item {Grp. gram.:m.}
\end{itemize}
\begin{itemize}
\item {Utilização:Bras. do N}
\end{itemize}
\begin{itemize}
\item {Proveniência:(T. tupi)}
\end{itemize}
Espécie de veado do valle do Amazonas.
Morcego do Brasil.
\section{Andira-aibaiariba}
\begin{itemize}
\item {Grp. gram.:m.}
\end{itemize}
(V.umari)
\section{Andirás}
\begin{itemize}
\item {Grp. gram.:m. pl.}
\end{itemize}
Aborígenes do Brasil, que habitaram em sertões do Pará.
\section{Andiroba}
\begin{itemize}
\item {Grp. gram.:f.}
\end{itemize}
Fruta da andirobeira.
Andirobeira.
(Do tupi)
\section{Andirobeira}
\begin{itemize}
\item {Grp. gram.:f.}
\end{itemize}
\begin{itemize}
\item {Proveniência:(De \textunderscore andiroba\textunderscore )}
\end{itemize}
Planta meliácea da América, (\textunderscore carapa gujanensis\textunderscore ).
\section{Andito}
\begin{itemize}
\item {Grp. gram.:m.}
\end{itemize}
Caminho estreito, acima do nível e ao lado das ruas, pontes ou caes; pequeno passeio lateral.
(Cast. \textunderscore andito\textunderscore )
\section{Andó}
\begin{itemize}
\item {Grp. gram.:adj.}
\end{itemize}
\begin{itemize}
\item {Utilização:Bras}
\end{itemize}
Diz-se de certo feitio de barba: \textunderscore usava barba andó\textunderscore .
\section{Andôa}
\begin{itemize}
\item {Grp. gram.:f.}
\end{itemize}
Espécie de barro azulado, que se tira na margem esquerda da ria de Aveiro.
\section{Andoar}
\begin{itemize}
\item {Grp. gram.:v. t.}
\end{itemize}
Cobrir com andôa.
\section{Andóbia}
\begin{itemize}
\item {Grp. gram.:f.}
\end{itemize}
Pedra, sôbre que gira a mó, em certos engenhos.
\section{Andolo}
\begin{itemize}
\item {Grp. gram.:m.}
\end{itemize}
Pequeno insecto africano, que vive no sub-solo e é comestível para os indígenas.
\section{Andongos}
\begin{itemize}
\item {Grp. gram.:m. pl.}
\end{itemize}
O mesmo que \textunderscore angolas\textunderscore .
\section{Andor}
\begin{itemize}
\item {Grp. gram.:m.}
\end{itemize}
\begin{itemize}
\item {Utilização:Ant.}
\end{itemize}
Padiola ornamentada, em que se levam imagens nas procissões.
Liteira.
Andas.
(Conc. \textunderscore andor\textunderscore , do sanscr. \textunderscore hindola\textunderscore )
\section{Andor-império}
\begin{itemize}
\item {Grp. gram.:m.  Loc.}
\end{itemize}
\begin{itemize}
\item {Utilização:Loc. de Ílhavo.}
\end{itemize}
Pessôa froixa, mollangueira.
\section{Andorinha}
\begin{itemize}
\item {Grp. gram.:f.}
\end{itemize}
\begin{itemize}
\item {Proveniência:(Do lat. hyp. \textunderscore hirundinea\textunderscore , de \textunderscore hirundo\textunderscore , com metáth.)}
\end{itemize}
Pássaro de arribação, da fam. dos fissirostros.
\section{Andorinha}
\begin{itemize}
\item {Grp. gram.:f.}
\end{itemize}
\begin{itemize}
\item {Utilização:Bras}
\end{itemize}
\begin{itemize}
\item {Utilização:Bras}
\end{itemize}
Carro, para transporte de mobília.
Lancha a vapor.
(Talvez do rad. de \textunderscore andor\textunderscore )
\section{Andorinha-da-serra}
\begin{itemize}
\item {Grp. gram.:f.}
\end{itemize}
\begin{itemize}
\item {Utilização:Mad}
\end{itemize}
Espécie de ave, (\textunderscore micropus-unicolor\textunderscore , Jard.).
\section{Andorinha-do-mar}
\begin{itemize}
\item {Grp. gram.:f.}
\end{itemize}
O mesmo que \textunderscore gaivina\textunderscore .
\section{Andorinhão}
\begin{itemize}
\item {Grp. gram.:m.}
\end{itemize}
\begin{itemize}
\item {Proveniência:(De \textunderscore andorinha\textunderscore )}
\end{itemize}
O mesmo que \textunderscore gaivão\textunderscore ^1, (\textunderscore micropus murinus\textunderscore , Brehm).
\section{Andorinho}
\begin{itemize}
\item {Grp. gram.:m.}
\end{itemize}
\begin{itemize}
\item {Utilização:Náut.}
\end{itemize}
Pequena corda, para pear os estribos das vêrgas dos navios.
Pequena andorinha.
Peça do polcame.
\section{Andorrano}
\begin{itemize}
\item {Grp. gram.:m.}
\end{itemize}
\begin{itemize}
\item {Grp. gram.:Adj.}
\end{itemize}
Habitante de Andorra.
Relativo á cidade de \textunderscore Andorra\textunderscore .
\section{Andradinho}
\begin{itemize}
\item {Grp. gram.:m.}
\end{itemize}
\begin{itemize}
\item {Proveniência:(De \textunderscore Andrade\textunderscore , n. p.)}
\end{itemize}
Arbusto do Brasil.
\section{Andrães}
\begin{itemize}
\item {Grp. gram.:m.}
\end{itemize}
\begin{itemize}
\item {Utilização:Prov.}
\end{itemize}
\begin{itemize}
\item {Utilização:trasm.}
\end{itemize}
Homem gordo e apalermado.
\section{Andraguires}
\begin{itemize}
\item {Grp. gram.:m. pl.}
\end{itemize}
Antigo povo da Malásia. Cf. \textunderscore Peregrinação\textunderscore , c. XVI.
\section{Andrajo}
\begin{itemize}
\item {Grp. gram.:m.}
\end{itemize}
Trapo, farrapo.
(Cast. \textunderscore andrajo\textunderscore )
\section{Andrajosamente}
\begin{itemize}
\item {Grp. gram.:adv.}
\end{itemize}
De modo \textunderscore andrajoso\textunderscore .
\section{Andrajoso}
\begin{itemize}
\item {Grp. gram.:adj.}
\end{itemize}
Coberto de andrajos.
Esfarrapado.
\section{Andréa}
\begin{itemize}
\item {Grp. gram.:f.}
\end{itemize}
\begin{itemize}
\item {Proveniência:(De \textunderscore André\textunderscore , n. p.)}
\end{itemize}
Gênero de musgos.
\section{Andrebello}
\begin{itemize}
\item {Grp. gram.:m.}
\end{itemize}
\begin{itemize}
\item {Utilização:Náut.}
\end{itemize}
Cabo de laborar dos mastaréus e vêrgas.
\section{Andrebelo}
\begin{itemize}
\item {Grp. gram.:m.}
\end{itemize}
\begin{itemize}
\item {Utilização:Náut.}
\end{itemize}
Cabo de laborar dos mastaréus e vêrgas.
\section{Andreia}
\begin{itemize}
\item {Grp. gram.:f.}
\end{itemize}
\begin{itemize}
\item {Proveniência:(De \textunderscore André\textunderscore , n. p.)}
\end{itemize}
Gênero de musgos.
\section{Andrenas}
\begin{itemize}
\item {Grp. gram.:f. pl.}
\end{itemize}
Insectos hymenópteros, da fam. das abelhas.
\section{Andrenoide}
\begin{itemize}
\item {Grp. gram.:adj.}
\end{itemize}
Semelhante ás andrenas.
\section{Andrequicé}
\begin{itemize}
\item {Grp. gram.:m.}
\end{itemize}
Nome, que no Brasil se dá ao malmequer grande.
Espécie de forragem.
\section{Andríala}
\begin{itemize}
\item {Grp. gram.:f.}
\end{itemize}
\begin{itemize}
\item {Utilização:Bot.}
\end{itemize}
Gênero de chicoriáceas.
\section{Andrino}
\begin{itemize}
\item {Grp. gram.:adj.}
\end{itemize}
Dizia-se do cavallo, que tem a côr escuro-azulada da parte superior dos andorinhos.
(Cast. \textunderscore andrino\textunderscore )
\section{Ândrio}
\begin{itemize}
\item {Grp. gram.:m.}
\end{itemize}
\begin{itemize}
\item {Proveniência:(Gr. \textunderscore andreios\textunderscore )}
\end{itemize}
Espécie de serpente.
\section{Andriopétalo}
\begin{itemize}
\item {Grp. gram.:m.}
\end{itemize}
\begin{itemize}
\item {Proveniência:(Do gr. \textunderscore andreios\textunderscore  + \textunderscore petalon\textunderscore )}
\end{itemize}
Planta brasileira, da fam. das proteáceas.
\section{Androceia}
\begin{itemize}
\item {Grp. gram.:f.}
\end{itemize}
O mesmo que \textunderscore androceu\textunderscore .
\section{Andrócera}
\begin{itemize}
\item {Grp. gram.:f.}
\end{itemize}
\begin{itemize}
\item {Utilização:Bot.}
\end{itemize}
Gênero de solâneas.
\section{Androceu}
\begin{itemize}
\item {Grp. gram.:m.}
\end{itemize}
\begin{itemize}
\item {Utilização:Bot.}
\end{itemize}
\begin{itemize}
\item {Proveniência:(Do gr. \textunderscore aner\textunderscore , \textunderscore andros\textunderscore  + \textunderscore oikia\textunderscore )}
\end{itemize}
Conjunto dos estames.
\section{Androcia}
\begin{itemize}
\item {Grp. gram.:f.}
\end{itemize}
O mesmo que \textunderscore androceu\textunderscore .
\section{Androcímbio}
\begin{itemize}
\item {Grp. gram.:m.}
\end{itemize}
\begin{itemize}
\item {Utilização:Bot.}
\end{itemize}
Gênero de melantáceas do Cabo.
\section{Andrócoma}
\begin{itemize}
\item {Grp. gram.:f.}
\end{itemize}
Gênero de plantas cyperáceas de Buenos-Aires.
\section{Andródama}
\begin{itemize}
\item {Grp. gram.:f.}
\end{itemize}
O mesmo que \textunderscore androdamante\textunderscore .
\section{Androdamante}
\begin{itemize}
\item {Grp. gram.:m.}
\end{itemize}
\begin{itemize}
\item {Proveniência:(Gr. \textunderscore androdamas\textunderscore , domador de homens)}
\end{itemize}
Pedra preciosa, côr de prata, a que os antigos attribuíam certas virtudes.
\section{Androdínamo}
\begin{itemize}
\item {Grp. gram.:adj.}
\end{itemize}
\begin{itemize}
\item {Proveniência:(Do gr. \textunderscore aner\textunderscore , \textunderscore andros\textunderscore  + \textunderscore dunamos\textunderscore )}
\end{itemize}
Diz-se das plantas, cujos estames adquirem grande desenvolvimento.
\section{Androdýnamo}
\begin{itemize}
\item {Grp. gram.:adj.}
\end{itemize}
\begin{itemize}
\item {Proveniência:(Do gr. \textunderscore aner\textunderscore , \textunderscore andros\textunderscore  + \textunderscore dunamos\textunderscore )}
\end{itemize}
Diz-se das plantas, cujos estames adquirem grande desenvolvimento.
\section{Andrófago}
\begin{itemize}
\item {Grp. gram.:m.  e  adj.}
\end{itemize}
\begin{itemize}
\item {Proveniência:(Do gr. \textunderscore aner\textunderscore , \textunderscore andros\textunderscore  + \textunderscore phagein\textunderscore )}
\end{itemize}
O mesmo que \textunderscore antropófago\textunderscore .
\section{Androfobia}
\begin{itemize}
\item {Grp. gram.:f.}
\end{itemize}
Qualidade de \textunderscore andrófobo\textunderscore .
\section{Andrófobo}
\begin{itemize}
\item {Grp. gram.:adj.}
\end{itemize}
\begin{itemize}
\item {Proveniência:(Do gr. \textunderscore aner\textunderscore , \textunderscore andros\textunderscore  + \textunderscore phobos\textunderscore )}
\end{itemize}
\begin{itemize}
\item {Proveniência:(Do gr. \textunderscore aner\textunderscore , \textunderscore andros\textunderscore  + \textunderscore petalos\textunderscore )}
\end{itemize}
Que tem repugnancia pelo sexo masculino.
\section{Andrófogo}
\begin{itemize}
\item {Grp. gram.:adj.}
\end{itemize}
\begin{itemize}
\item {Utilização:Des.}
\end{itemize}
\begin{itemize}
\item {Proveniência:(T. hýbr., do gr. \textunderscore aner\textunderscore , \textunderscore andros\textunderscore  + lat. \textunderscore fugere\textunderscore )}
\end{itemize}
O mesmo que \textunderscore andróphobo\textunderscore .
\section{Andróforo}
\begin{itemize}
\item {Grp. gram.:m.}
\end{itemize}
\begin{itemize}
\item {Proveniência:(Do gr. \textunderscore aner\textunderscore , \textunderscore andros\textunderscore  + \textunderscore phero\textunderscore )}
\end{itemize}
Parte adherente dos filetes que sustentam as antheras da flôr.
\section{Androgenesia}
\begin{itemize}
\item {Grp. gram.:f.}
\end{itemize}
\begin{itemize}
\item {Proveniência:(Do gr. \textunderscore aner\textunderscore , \textunderscore andros\textunderscore  + \textunderscore genesis\textunderscore )}
\end{itemize}
Sciência do desenvolvimento phýsico e moral da espécie humana.
\section{Androgenésico}
\begin{itemize}
\item {Grp. gram.:adj.}
\end{itemize}
Relativo á androgenesia.
\section{Androgenia}
\begin{itemize}
\item {Grp. gram.:f.}
\end{itemize}
\begin{itemize}
\item {Proveniência:(Do gr. \textunderscore aner\textunderscore , \textunderscore andros\textunderscore  + \textunderscore genos\textunderscore )}
\end{itemize}
Sequência de descendentes varões.
\section{Androginário}
\begin{itemize}
\item {Grp. gram.:adj.}
\end{itemize}
\begin{itemize}
\item {Proveniência:(De \textunderscore andrógyno\textunderscore )}
\end{itemize}
Diz-se das flôres, dobradas pela transformação das duas espécies de órgãos sexuaes, sem que os tegumentos tenham sido alterados.
\section{Androginia}
\begin{itemize}
\item {Grp. gram.:f.}
\end{itemize}
Qualidade do vegetal andrógino.
\section{Androgínico}
\begin{itemize}
\item {Grp. gram.:adj.}
\end{itemize}
\begin{itemize}
\item {Proveniência:(De \textunderscore andrógyno\textunderscore )}
\end{itemize}
Relativo ou pertencente a uma flôr hermaphrodita.
\section{Androginismo}
\begin{itemize}
\item {Grp. gram.:m.}
\end{itemize}
O mesmo que \textunderscore androginia\textunderscore .
\section{Andrógino}
\begin{itemize}
\item {Grp. gram.:adj.}
\end{itemize}
\begin{itemize}
\item {Proveniência:(Gr. \textunderscore androgunos\textunderscore )}
\end{itemize}
Hermaphrodita; commum ao homem e á mulher.
\section{Androglosso}
\begin{itemize}
\item {Grp. gram.:adj.}
\end{itemize}
\begin{itemize}
\item {Proveniência:(Do gr. \textunderscore aner\textunderscore , \textunderscore andros\textunderscore  + \textunderscore glossa\textunderscore )}
\end{itemize}
Diz-se das aves, que aprendem facilmente a falar.
\section{Androgynário}
\begin{itemize}
\item {Grp. gram.:adj.}
\end{itemize}
\begin{itemize}
\item {Proveniência:(De \textunderscore andrógyno\textunderscore )}
\end{itemize}
Diz-se das flôres, dobradas pela transformação das duas espécies de órgãos sexuaes, sem que os tegumentos tenham sido alterados.
\section{Androgynia}
\begin{itemize}
\item {Grp. gram.:f.}
\end{itemize}
Qualidade do vegetal andrógyno.
\section{Androgýnico}
\begin{itemize}
\item {Grp. gram.:adj.}
\end{itemize}
\begin{itemize}
\item {Proveniência:(De \textunderscore andrógyno\textunderscore )}
\end{itemize}
Relativo ou pertencente a uma flôr hermaphrodita.
\section{Androgynismo}
\begin{itemize}
\item {Grp. gram.:m.}
\end{itemize}
O mesmo que \textunderscore androgynia\textunderscore .
\section{Andrógyno}
\begin{itemize}
\item {Grp. gram.:adj.}
\end{itemize}
\begin{itemize}
\item {Proveniência:(Gr. \textunderscore androgunos\textunderscore )}
\end{itemize}
Hermaphrodita; commum ao homem e á mulher.
\section{Androide}
\begin{itemize}
\item {Grp. gram.:m.}
\end{itemize}
\begin{itemize}
\item {Proveniência:(Do gr. \textunderscore aner\textunderscore , \textunderscore andros\textunderscore  + \textunderscore eidos\textunderscore )}
\end{itemize}
Títere, fantoche.
Boneco, que imita a figura de um homem.
O mesmo que \textunderscore anthropopitheco\textunderscore .
\section{Androido}
\begin{itemize}
\item {Grp. gram.:m.}
\end{itemize}
\begin{itemize}
\item {Proveniência:(Do gr. \textunderscore aner\textunderscore , \textunderscore andros\textunderscore  + \textunderscore eidos\textunderscore )}
\end{itemize}
Títere, fantoche.
Boneco, que imita a figura de um homem.
O mesmo que \textunderscore anthropopitheco\textunderscore .
\section{Andrólatra}
\begin{itemize}
\item {Grp. gram.:m.  e  f.}
\end{itemize}
Pessôa, que tributa culto divino a um homem.
(Cp. \textunderscore androlatria\textunderscore )
\section{Androlatria}
\begin{itemize}
\item {Grp. gram.:f.}
\end{itemize}
\begin{itemize}
\item {Proveniência:(Do gr. \textunderscore aner\textunderscore , \textunderscore andros\textunderscore  + \textunderscore latreia\textunderscore )}
\end{itemize}
Culto divino, tributado a um homem.
\section{Androlepsia}
\begin{itemize}
\item {Grp. gram.:f.}
\end{itemize}
Direito, que os Athenienses tinham, de se apoderar de três habitantes de uma cidade, onde se refugiasse um criminoso, até que êste fôsse punido.
\section{Andrologia}
\begin{itemize}
\item {Grp. gram.:f.}
\end{itemize}
Sciência do homem e, especialmente, das doenças do homem.
\section{Andromania}
\begin{itemize}
\item {Grp. gram.:f.}
\end{itemize}
\begin{itemize}
\item {Proveniência:(Do gr. \textunderscore aner\textunderscore , \textunderscore andros\textunderscore  + \textunderscore mania\textunderscore )}
\end{itemize}
Furor uterino; neurose genital da mulher.
\section{Andromaníaca}
\begin{itemize}
\item {Grp. gram.:adj. f.}
\end{itemize}
Que tem \textunderscore andromania\textunderscore .
\section{Andrómeda}
\begin{itemize}
\item {Grp. gram.:f.}
\end{itemize}
\begin{itemize}
\item {Proveniência:(De \textunderscore Andrómeda\textunderscore , n. p. myth.)}
\end{itemize}
Uma das constellações boreaes.
Arbusto, da fam. das ericáceas.
\section{Andrómina}
\begin{itemize}
\item {Grp. gram.:f.}
\end{itemize}
(V.endrómina)
\section{Andronítide}
\begin{itemize}
\item {Grp. gram.:f.}
\end{itemize}
\begin{itemize}
\item {Proveniência:(Gr. \textunderscore andronitis\textunderscore )}
\end{itemize}
A primeira das duas divisões principaes dos antigos palácios gregos, destinada só aos homens, e por um corredor separada da destinada ás mulheres.
\section{Andronito}
\begin{itemize}
\item {Grp. gram.:m.}
\end{itemize}
O mesmo que \textunderscore andronítide\textunderscore .
\section{Andropado}
\begin{itemize}
\item {Grp. gram.:m.}
\end{itemize}
Espécie de melro africano.
\section{Andropetalário}
\begin{itemize}
\item {Grp. gram.:adj.}
\end{itemize}
\begin{itemize}
\item {Utilização:Bot.}
\end{itemize}
\begin{itemize}
\item {Proveniência:(Do gr. \textunderscore aner\textunderscore , \textunderscore andros\textunderscore  + \textunderscore petalon\textunderscore )}
\end{itemize}
Diz-se das plantas de côres duplas, cujos estames se convertem em pétalas, como nas rosas, camélias, etc.
\section{Andróphago}
\begin{itemize}
\item {Grp. gram.:m.  e  adj.}
\end{itemize}
\begin{itemize}
\item {Proveniência:(Do gr. \textunderscore aner\textunderscore , \textunderscore andros\textunderscore  + \textunderscore phagein\textunderscore )}
\end{itemize}
O mesmo que \textunderscore anthropóphago\textunderscore .
\section{Androphobia}
\begin{itemize}
\item {Grp. gram.:f.}
\end{itemize}
Qualidade de \textunderscore andróphobo\textunderscore .
\section{Andróphobo}
\begin{itemize}
\item {Grp. gram.:adj.}
\end{itemize}
\begin{itemize}
\item {Proveniência:(Do gr. \textunderscore aner\textunderscore , \textunderscore andros\textunderscore  + \textunderscore phobos\textunderscore )}
\end{itemize}
\begin{itemize}
\item {Proveniência:(Do gr. \textunderscore aner\textunderscore , \textunderscore andros\textunderscore  + \textunderscore petalos\textunderscore )}
\end{itemize}
Que tem repugnancia pelo sexo masculino.
\section{Andróphoro}
\begin{itemize}
\item {Grp. gram.:m.}
\end{itemize}
\begin{itemize}
\item {Proveniência:(Do gr. \textunderscore aner\textunderscore , \textunderscore andros\textunderscore  + \textunderscore phero\textunderscore )}
\end{itemize}
Parte adherente dos filetes que sustentam as antheras da flôr.
\section{Andropogão}
\begin{itemize}
\item {Grp. gram.:m.}
\end{itemize}
\begin{itemize}
\item {Proveniência:(Do gr. \textunderscore aner\textunderscore , \textunderscore andros\textunderscore  + \textunderscore pogon\textunderscore )}
\end{itemize}
Planta, da fam. das gramíneas.
\section{Andropogónias}
\begin{itemize}
\item {Grp. gram.:f. pl.}
\end{itemize}
Tribo de gramíneas, que tem por typo o andropogão.
\section{Androsáceo}
\begin{itemize}
\item {Grp. gram.:adj.}
\end{itemize}
\begin{itemize}
\item {Utilização:Bot.}
\end{itemize}
Diz-se de uma espécie de agárico, que no outono cresce nas folhas de certas árvores, especialmente do roble.
\section{Andrósaco}
\begin{itemize}
\item {Grp. gram.:m.}
\end{itemize}
\begin{itemize}
\item {Proveniência:(Do gr. \textunderscore aner\textunderscore , \textunderscore andros\textunderscore  + \textunderscore sacos\textunderscore )}
\end{itemize}
Planta primulácea.
\section{Androsemo}
\begin{itemize}
\item {Grp. gram.:m.}
\end{itemize}
Planta africana, da ordem das hypericáceas.
Planta vulgar e applicável contra os cálculos da bexiga e rins, e conhecida entre o povo por \textunderscore mijadeira\textunderscore , (\textunderscore androsaemum vulgare\textunderscore , Lin.).
\section{Andrótomas}
\begin{itemize}
\item {Grp. gram.:f. pl.}
\end{itemize}
O mesmo que \textunderscore synanthéreas\textunderscore .
\section{Andrótomo}
\begin{itemize}
\item {Grp. gram.:adj.}
\end{itemize}
\begin{itemize}
\item {Utilização:Bot.}
\end{itemize}
Diz-se das plantas, cujos estames estão divididos em duas partes por uma espécie de articulação.
\section{Andu}
\begin{itemize}
\item {Grp. gram.:m.}
\end{itemize}
Fruto de um arbusto leguminoso do Brasil.
\section{Andua}
\begin{itemize}
\item {Grp. gram.:f.}
\end{itemize}
Ave africana, (\textunderscore caryathaix lewingstonia\textunderscore ).
\section{Andudu}
\begin{itemize}
\item {Grp. gram.:m.}
\end{itemize}
Ave africana.
\section{Andurriaes}
\begin{itemize}
\item {Grp. gram.:m. pl.}
\end{itemize}
\begin{itemize}
\item {Utilização:Ant.}
\end{itemize}
Lugares ermos, sem caminho.
Lugares públicos, pouco limpos, mas trilhados por muita gente.
(Cast. \textunderscore andurrial\textunderscore )
\section{Andurriais}
\begin{itemize}
\item {Grp. gram.:m. pl.}
\end{itemize}
\begin{itemize}
\item {Utilização:Ant.}
\end{itemize}
Lugares ermos, sem caminho.
Lugares públicos, pouco limpos, mas trilhados por muita gente.
(Cast. \textunderscore andurrial\textunderscore )
\section{Anduzeiro}
\begin{itemize}
\item {Grp. gram.:m.}
\end{itemize}
Arbusto brasileiro, que produz o andu.
\section{Aneaquis}
\begin{itemize}
\item {Grp. gram.:m. pl.}
\end{itemize}
Indígenas do norte do Brasil.
\section{Anecdota}
\begin{itemize}
\item {Grp. gram.:f.}
\end{itemize}
\begin{itemize}
\item {Proveniência:(Do gr. \textunderscore anekdoton\textunderscore )}
\end{itemize}
Narração rapida de um facto jocoso.
Particularidade divertida, histórica ou imaginária.
\section{Anecdótico}
\begin{itemize}
\item {Grp. gram.:adj.}
\end{itemize}
Relativo a anecdota.
Que encerra anecdota.
\section{Anecdotista}
\begin{itemize}
\item {Grp. gram.:m.}
\end{itemize}
Aquelle que conta anecdotas.
Aquelle que as collecciona.
\section{Anecdotizar}
\begin{itemize}
\item {Grp. gram.:v. t.}
\end{itemize}
\begin{itemize}
\item {Grp. gram.:V. i.}
\end{itemize}
Dar fórma de anecdota a.
Contar, em fórma de anecdota.
Contar anecdotas.
\section{Anectasia}
\begin{itemize}
\item {Grp. gram.:f.}
\end{itemize}
\begin{itemize}
\item {Utilização:Med.}
\end{itemize}
\begin{itemize}
\item {Proveniência:(Do gr. \textunderscore an\textunderscore  priv. + \textunderscore ektasis\textunderscore )}
\end{itemize}
Defficiência de extensão de um órgão.
\section{Anediar}
\begin{itemize}
\item {Grp. gram.:v. t.}
\end{itemize}
Tornar nédio.
Alisar.
\section{Anegaça}
\begin{itemize}
\item {Grp. gram.:f.}
\end{itemize}
(V.negaça)
\section{Anegalhar}
\begin{itemize}
\item {Grp. gram.:v. t.}
\end{itemize}
Atar com negalho ou guita.
\section{Anegalhéis}
\begin{itemize}
\item {Grp. gram.:m.}
\end{itemize}
Planta da serra de Cintra.
\section{Anegar}
\begin{itemize}
\item {Grp. gram.:v. t.}
\end{itemize}
Cobrir de água.
Mergulhar.
Alagar.
Afogar.
(Cast. \textunderscore anegar\textunderscore . Cp. \textunderscore anaguar\textunderscore )
\section{Anegrado}
\begin{itemize}
\item {Grp. gram.:adj.}
\end{itemize}
Um tanto negro.
\section{Anegrejar}
\begin{itemize}
\item {Grp. gram.:v. t.}
\end{itemize}
Tornar negro. Cf. Filinto, XIII, 218.
\section{Anegriscado}
\begin{itemize}
\item {Grp. gram.:adj.}
\end{itemize}
Um tanto negro.
\section{Anel}
\begin{itemize}
\item {Grp. gram.:m.}
\end{itemize}
\begin{itemize}
\item {Utilização:Agr.}
\end{itemize}
\begin{itemize}
\item {Proveniência:(Lat. \textunderscore anellus\textunderscore , dem. de \textunderscore anulus\textunderscore )}
\end{itemize}
Círculo.
Qualquer substância de fórma circular.
Pequeno arco, com que se enfeita o dedo.
Elo.
Cada uma das peças de uma corrente.
Sêllo do anel.
Espiral (de cabello).
\textunderscore Anel de Saturno\textunderscore , faixa circular, que rodeia êste planeta.
Medida de água, correspondente a oito pennas de água: \textunderscore a nascente mal dá três aneis de água\textunderscore .
\section{Anelado}
\begin{itemize}
\item {Grp. gram.:adj.}
\end{itemize}
Que tem fórma de anel.
Encaracolado: \textunderscore cabello anelado\textunderscore .
\section{Anelados}
\begin{itemize}
\item {Grp. gram.:m. pl.}
\end{itemize}
\begin{itemize}
\item {Utilização:Zool.}
\end{itemize}
O mesmo que \textunderscore anélidos\textunderscore .
\section{Aneladura}
\begin{itemize}
\item {Grp. gram.:f.}
\end{itemize}
Acto de \textunderscore anelar\textunderscore .
\section{Anelar}
\begin{itemize}
\item {Grp. gram.:v. t.}
\end{itemize}
Dar fórma de anel a.
Encaracolar.
\section{Anelar}
\begin{itemize}
\item {Grp. gram.:adj.}
\end{itemize}
O mesmo que \textunderscore anular\textunderscore ^1.
\section{Anelasto}
\begin{itemize}
\item {Grp. gram.:m.}
\end{itemize}
Gênero de coleópteros americanos.
\section{Aneléctrico}
\begin{itemize}
\item {Grp. gram.:adj.}
\end{itemize}
\begin{itemize}
\item {Proveniência:(De \textunderscore an\textunderscore  priv. + \textunderscore eléctrico\textunderscore )}
\end{itemize}
Que não póde conservar as propriedades eléctricas.
\section{Aneleira}
\begin{itemize}
\item {Grp. gram.:f.}
\end{itemize}
Caixinha, para guardar anéis.
\section{Anelidários}
\begin{itemize}
\item {Grp. gram.:m. pl.}
\end{itemize}
\begin{itemize}
\item {Proveniência:(De \textunderscore anélidos\textunderscore )}
\end{itemize}
Grupo de helminthos, a que pertence a tênia, e que são semelhantes aos anélidos.
\section{Anelídeo}
\begin{itemize}
\item {Grp. gram.:adj.}
\end{itemize}
Pertencente á classe dos \textunderscore anélidos\textunderscore .
\section{Anélides}
\begin{itemize}
\item {Grp. gram.:m. pl.}
\end{itemize}
\begin{itemize}
\item {Proveniência:(De \textunderscore anel\textunderscore  + gr. \textunderscore eidos\textunderscore )}
\end{itemize}
Classe de animaes vertebrados, a que a sanguessuga e a minhoca servem de typo.
\section{Anélidos}
\begin{itemize}
\item {Grp. gram.:m. pl.}
\end{itemize}
\begin{itemize}
\item {Proveniência:(De \textunderscore anel\textunderscore  + gr. \textunderscore eidos\textunderscore )}
\end{itemize}
Classe de animaes vertebrados, a que a sanguessuga e a minhoca servem de typo.
\section{Aneliforme}
\begin{itemize}
\item {Grp. gram.:adj.}
\end{itemize}
Que tem fórma de anel.
\section{Anelípede}
\begin{itemize}
\item {Grp. gram.:adj.}
\end{itemize}
\begin{itemize}
\item {Proveniência:(Do lat. \textunderscore anellus\textunderscore  + \textunderscore pes\textunderscore , \textunderscore pedis\textunderscore )}
\end{itemize}
Que tem patas em fórma de anel.
\section{Anelóptero}
\begin{itemize}
\item {Grp. gram.:adj.}
\end{itemize}
O mesmo que \textunderscore anelytro\textunderscore .
\section{Anelytro}
\begin{itemize}
\item {Grp. gram.:adj.}
\end{itemize}
\begin{itemize}
\item {Proveniência:(De \textunderscore an\textunderscore  priv. + \textunderscore elytro\textunderscore )}
\end{itemize}
Diz-se dos insectos de quatro asas, das quaes as duas superiores não têm a consistência dos elytros.
\section{Anema}
\begin{itemize}
\item {Grp. gram.:m.}
\end{itemize}
Gênero de insectos coleópteros do Senegal.
\section{Anemarrena}
\begin{itemize}
\item {Grp. gram.:f.}
\end{itemize}
\begin{itemize}
\item {Utilização:Bot.}
\end{itemize}
Gênero de liliáceas.
\section{Anemático}
\begin{itemize}
\item {Grp. gram.:adj.}
\end{itemize}
Dizia-se do animal que não tem sangue, segundo a classificação de Aristóteles.
\section{Anemia}
\begin{itemize}
\item {Grp. gram.:f.}
\end{itemize}
\begin{itemize}
\item {Proveniência:(Do gr. \textunderscore an\textunderscore  priv. + \textunderscore haima\textunderscore )}
\end{itemize}
Deminuição do sangue no organismo animal.
Deminuição dos elementos vivificantes do sangue.
\section{Anemiante}
\begin{itemize}
\item {Grp. gram.:adj.}
\end{itemize}
Que produz anemia.
\section{Anemiar}
\begin{itemize}
\item {Grp. gram.:v. t.}
\end{itemize}
\begin{itemize}
\item {Utilização:Neol.}
\end{itemize}
\begin{itemize}
\item {Utilização:Fig.}
\end{itemize}
Produzir anemia em.
Enfraquecer.
\section{Anemizar}
\begin{itemize}
\item {Grp. gram.:v. t.}
\end{itemize}
O mesmo ou melhor que \textunderscore anemiar\textunderscore . Us. por Rui Barbosa.
\section{Anêmico}
\begin{itemize}
\item {Grp. gram.:adj.}
\end{itemize}
Relativo a \textunderscore anemia\textunderscore .
Que tem anemia.
\section{Anemóbata}
\begin{itemize}
\item {Grp. gram.:m.}
\end{itemize}
\begin{itemize}
\item {Proveniência:(Do gr. \textunderscore anemos\textunderscore  + \textunderscore bates\textunderscore )}
\end{itemize}
O mesmo que \textunderscore funâmbulo\textunderscore .
\section{Anemocórdio}
\begin{itemize}
\item {Grp. gram.:m.}
\end{itemize}
Instrumento de cordas, inventado por Schnell, e cuja vibração era produzida por correntes de ar.
\section{Anemofobia}
\begin{itemize}
\item {Grp. gram.:f.}
\end{itemize}
\begin{itemize}
\item {Proveniência:(Do gr. \textunderscore anemos\textunderscore  + \textunderscore phobein\textunderscore )}
\end{itemize}
Terror mórbido do vento.
\section{Anemografia}
\begin{itemize}
\item {Grp. gram.:f.}
\end{itemize}
Descripção dos ventos.
(Cp. \textunderscore anemógrapho\textunderscore )
\section{Anemógrafo}
\begin{itemize}
\item {Grp. gram.:m.}
\end{itemize}
\begin{itemize}
\item {Proveniência:(Do gr. \textunderscore anemos\textunderscore  + \textunderscore graphein\textunderscore )}
\end{itemize}
Aquelle que descreve os ventos.
\section{Anemographia}
\begin{itemize}
\item {Grp. gram.:f.}
\end{itemize}
Descripção dos ventos.
(Cp. \textunderscore anemógrapho\textunderscore )
\section{Anemógrapho}
\begin{itemize}
\item {Grp. gram.:m.}
\end{itemize}
\begin{itemize}
\item {Proveniência:(Do gr. \textunderscore anemos\textunderscore  + \textunderscore graphein\textunderscore )}
\end{itemize}
Aquelle que descreve os ventos.
\section{Anêmola}
\begin{itemize}
\item {Grp. gram.:f.}
\end{itemize}
\begin{itemize}
\item {Utilização:Pop.}
\end{itemize}
O mesmo que \textunderscore anêmona\textunderscore . Cf. \textunderscore Regul. Pharm.\textunderscore 
\section{Anemologia}
\begin{itemize}
\item {Grp. gram.:f.}
\end{itemize}
\begin{itemize}
\item {Proveniência:(De \textunderscore anemólogo\textunderscore )}
\end{itemize}
Tratado á cêrca dos ventos.
\section{Anemólogo}
\begin{itemize}
\item {Grp. gram.:m.}
\end{itemize}
\begin{itemize}
\item {Proveniência:(Do gr. \textunderscore anemos\textunderscore  + \textunderscore logos\textunderscore )}
\end{itemize}
Aquelle que scientificamente trata dos ventos.
\section{Anemometria}
\begin{itemize}
\item {Grp. gram.:f.}
\end{itemize}
\begin{itemize}
\item {Proveniência:(De \textunderscore anemómetro\textunderscore )}
\end{itemize}
Medida da velocidade e fôrça dos ventos.
\section{Anemométrico}
\begin{itemize}
\item {Grp. gram.:adj.}
\end{itemize}
Relativo á \textunderscore anemometria\textunderscore .
\section{Anemómetro}
\begin{itemize}
\item {Grp. gram.:m.}
\end{itemize}
\begin{itemize}
\item {Proveniência:(Do gr. \textunderscore anemos\textunderscore  + \textunderscore metron\textunderscore )}
\end{itemize}
Instrumento, para medir a fôrça dos ventos.
\section{Anemometrógrafo}
\begin{itemize}
\item {Grp. gram.:m.}
\end{itemize}
\begin{itemize}
\item {Proveniência:(Do gr. \textunderscore anemos\textunderscore  + \textunderscore metron\textunderscore  + \textunderscore graphein\textunderscore )}
\end{itemize}
Apparelho, que consiste num anemómetro adaptado a um pêndulo, o qual faz mover o lápis, que marca no papel as variações successivas do vento e sua duração.
\section{Anemometrógrapho}
\begin{itemize}
\item {Grp. gram.:m.}
\end{itemize}
\begin{itemize}
\item {Proveniência:(Do gr. \textunderscore anemos\textunderscore  + \textunderscore metron\textunderscore  + \textunderscore graphein\textunderscore )}
\end{itemize}
Apparelho, que consiste num anemómetro adaptado a um pêndulo, o qual faz mover o lápis, que marca no papel as variações successivas do vento e sua duração.
\section{Anêmona}
\begin{itemize}
\item {Grp. gram.:f.}
\end{itemize}
\begin{itemize}
\item {Proveniência:(Gr. \textunderscore anemone\textunderscore )}
\end{itemize}
Planta ranunculácea.
A flôr dessa planta.
Zoóphyto, da classe dos pólypos, e o mesmo que \textunderscore actínia\textunderscore .
\section{Anemóneas}
\begin{itemize}
\item {Grp. gram.:f. pl.}
\end{itemize}
\begin{itemize}
\item {Proveniência:(De \textunderscore anêmona\textunderscore )}
\end{itemize}
Tribo de ranunculáceas, segundo De-Candolle.
\section{Anemónico}
\begin{itemize}
\item {Grp. gram.:adj.}
\end{itemize}
Diz-se de um ácido, que se deposita na água destillada da anêmona.
\section{Anemonifoliado}
\begin{itemize}
\item {Grp. gram.:adj.}
\end{itemize}
\begin{itemize}
\item {Utilização:Bot.}
\end{itemize}
Cujas fôlhas se parecem ás da anêmona.
\section{Anemonina}
\begin{itemize}
\item {Grp. gram.:f.}
\end{itemize}
Substância branca, inodora, descoberta nas fôlhas da anêmona.
\section{Anemophobia}
\begin{itemize}
\item {Grp. gram.:f.}
\end{itemize}
\begin{itemize}
\item {Proveniência:(Do gr. \textunderscore anemos\textunderscore  + \textunderscore phobein\textunderscore )}
\end{itemize}
Terror mórbido do vento.
\section{Anemoscópio}
\begin{itemize}
\item {Grp. gram.:m.}
\end{itemize}
\begin{itemize}
\item {Proveniência:(Do gr. \textunderscore anemos\textunderscore  + \textunderscore skopein\textunderscore )}
\end{itemize}
Instrumento, para indicar a direcção dos ventos.
\section{Anemótropo}
\begin{itemize}
\item {Grp. gram.:m.}
\end{itemize}
\begin{itemize}
\item {Proveniência:(Do gr. \textunderscore anemos\textunderscore  + \textunderscore trepein\textunderscore )}
\end{itemize}
Motor de vento, que se applica especialmente ao fabríco do chocolate.
\section{Anenai}
\begin{itemize}
\item {Grp. gram.:m.}
\end{itemize}
Árvore indiana, de fíbras têxteis.
\section{Anencefalia}
\begin{itemize}
\item {Grp. gram.:f.}
\end{itemize}
Estado ou qualidade de anencéfalo.
\section{Anencéfalo}
\begin{itemize}
\item {Grp. gram.:adj.}
\end{itemize}
\begin{itemize}
\item {Proveniência:(De \textunderscore an\textunderscore  priv. + \textunderscore encéphalo\textunderscore )}
\end{itemize}
Diz-se do monstro, que não tem cérebro.
\section{Anencephalia}
\begin{itemize}
\item {Grp. gram.:f.}
\end{itemize}
Estado ou qualidade de anencéphalo.
\section{Anencéphalo}
\begin{itemize}
\item {Grp. gram.:adj.}
\end{itemize}
\begin{itemize}
\item {Proveniência:(De \textunderscore an\textunderscore  priv. + \textunderscore encéphalo\textunderscore )}
\end{itemize}
Diz-se do monstro, que não tem cérebro.
\section{Anenteremia}
\begin{itemize}
\item {Grp. gram.:f.}
\end{itemize}
\begin{itemize}
\item {Utilização:Med.}
\end{itemize}
\begin{itemize}
\item {Proveniência:(Do gr. \textunderscore an\textunderscore  priv. + \textunderscore enteros\textunderscore  + \textunderscore haima\textunderscore )}
\end{itemize}
Falta de sangue nos intestinos.
\section{Anentéreo}
\begin{itemize}
\item {Grp. gram.:adj.}
\end{itemize}
\begin{itemize}
\item {Grp. gram.:M. pl.}
\end{itemize}
\begin{itemize}
\item {Proveniência:(Do gr. \textunderscore an\textunderscore  + \textunderscore enteros\textunderscore )}
\end{itemize}
Diz-se dos infusórios, que não têm canal intestinal.
Infusórios polygástricos sem tubo intestinal.
\section{Anenterotrofia}
\begin{itemize}
\item {Grp. gram.:f.}
\end{itemize}
Deminuição do volume dos intestinos.
\section{Anenterotrophia}
\begin{itemize}
\item {Grp. gram.:f.}
\end{itemize}
Deminuição do volume dos intestinos.
\section{...âneo}
\begin{itemize}
\item {Grp. gram.:suf.}
\end{itemize}
(de qualidade, pertença, etc.)
\section{Anepigrafia}
\begin{itemize}
\item {Grp. gram.:f.}
\end{itemize}
Falta ou desapparecimento de uma inscripção.
Estado ou qualidade de \textunderscore anepígrafo\textunderscore .
\section{Anepígrafo}
\begin{itemize}
\item {Grp. gram.:adj.}
\end{itemize}
\begin{itemize}
\item {Proveniência:(Do gr. \textunderscore an\textunderscore  priv. + \textunderscore epi\textunderscore  + \textunderscore graphein\textunderscore )}
\end{itemize}
Que não tem inscripção ou título, (falando-se de medalhas e baixos-relêvos).
\section{Anepigraphia}
\begin{itemize}
\item {Grp. gram.:f.}
\end{itemize}
Falta ou desapparecimento de uma inscripção.
Estado ou qualidade de \textunderscore anepígrapho\textunderscore .
\section{Anepígrapho}
\begin{itemize}
\item {Grp. gram.:adj.}
\end{itemize}
\begin{itemize}
\item {Proveniência:(Do gr. \textunderscore an\textunderscore priv. + \textunderscore epi\textunderscore  + \textunderscore graphein\textunderscore )}
\end{itemize}
Que não tem inscripção ou título, (falando-se de medalhas e baixos-relêvos).
\section{Anepithymia}
\begin{itemize}
\item {Grp. gram.:f.}
\end{itemize}
Paralysia, que interrompe a communicação entre as vísceras abdominaes e o systema nervoso.
Perda dos desejos sensuaes.
\section{Anepitímia}
\begin{itemize}
\item {Grp. gram.:f.}
\end{itemize}
Paralysia, que interrompe a communicação entre as vísceras abdominaes e o systema nervoso.
Perda dos desejos sensuaes.
\section{Anequim}
\begin{itemize}
\item {Grp. gram.:m.}
\end{itemize}
\begin{itemize}
\item {Proveniência:(De \textunderscore Annequim\textunderscore , n. p.)}
\end{itemize}
Espécie de peixe miúdo.
Peixe plagióstomo, pardo-anegrado.
Bôbo do paço, em tempo de D. Fernando I.
\section{Anerana}
\begin{itemize}
\item {Grp. gram.:f.}
\end{itemize}
Árvore da região do alto Amazonas.
\section{Anereta}
\begin{itemize}
\item {fónica:ne-rê}
\end{itemize}
\begin{itemize}
\item {Grp. gram.:f.}
\end{itemize}
\begin{itemize}
\item {Proveniência:(Gr. \textunderscore anairetes\textunderscore )}
\end{itemize}
Insecto colleóptero pentâmero.
\section{Aneretismo}
\begin{itemize}
\item {Grp. gram.:m.}
\end{itemize}
\begin{itemize}
\item {Utilização:Med.}
\end{itemize}
Falta ou perda de irritabilidade.
\section{Aneróbio}
\begin{itemize}
\item {Grp. gram.:m.}
\end{itemize}
(V.anaeróbio)
\section{Aneroide}
\begin{itemize}
\item {Grp. gram.:m.  e  adj.}
\end{itemize}
\begin{itemize}
\item {Proveniência:(Do gr. \textunderscore an\textunderscore  priv. + \textunderscore aer\textunderscore  + \textunderscore eidos\textunderscore )}
\end{itemize}
Diz-se de um barómetro de mostrador, em cuja caixa, de paredes metállicas, se fórma o vácuo.
\section{Anérveo}
\begin{itemize}
\item {Grp. gram.:adj.}
\end{itemize}
\begin{itemize}
\item {Proveniência:(De \textunderscore a\textunderscore  priv. + \textunderscore nervo\textunderscore )}
\end{itemize}
Diz-se do insecto, cujas asas não têm filetos nervosos.
\section{Anervia}
\begin{itemize}
\item {Grp. gram.:f.}
\end{itemize}
\begin{itemize}
\item {Proveniência:(Do gr. \textunderscore a\textunderscore  priv. + \textunderscore neuron\textunderscore )}
\end{itemize}
Paralysia; falta de acção nervosa.
\section{Anervismo}
\begin{itemize}
\item {Grp. gram.:m.}
\end{itemize}
O mesmo que \textunderscore anervia\textunderscore .
\section{Anesia}
\begin{itemize}
\item {Grp. gram.:f.}
\end{itemize}
Deminuïção ou desapparecimento dos symptomas de uma doença.
\section{Anestesia}
\begin{itemize}
\item {Grp. gram.:f.}
\end{itemize}
\begin{itemize}
\item {Proveniência:(Do gr. \textunderscore an\textunderscore  priv. + \textunderscore aisthesis\textunderscore )}
\end{itemize}
Ausência ou deminuição de sensibilidade.
\section{Anestesiante}
\begin{itemize}
\item {Grp. gram.:adj.}
\end{itemize}
Que anestesia.
\section{Anestesiar}
\begin{itemize}
\item {Grp. gram.:v. t.}
\end{itemize}
\begin{itemize}
\item {Proveniência:(De \textunderscore anesthesia\textunderscore )}
\end{itemize}
Tirar a sensibilidade a.
Deminuir a sensibilidade de.
\section{Anestésica}
\begin{itemize}
\item {Grp. gram.:f.}
\end{itemize}
Lepra vulgar.
\section{Anestésico}
\begin{itemize}
\item {Grp. gram.:adj.}
\end{itemize}
Relativo á \textunderscore anestesia\textunderscore .
\section{Anestético}
\begin{itemize}
\item {Grp. gram.:adj.}
\end{itemize}
\begin{itemize}
\item {Grp. gram.:M.}
\end{itemize}
Que produz anestesia.
Aquillo que produz anesthesia. Cf. Latino, \textunderscore Elogios Acad.\textunderscore  314.
\section{Anesthesia}
\begin{itemize}
\item {Grp. gram.:f.}
\end{itemize}
\begin{itemize}
\item {Proveniência:(Do gr. \textunderscore an\textunderscore  priv. + \textunderscore aisthesis\textunderscore )}
\end{itemize}
Ausência ou deminuição de sensibilidade.
\section{Anesthesiante}
\begin{itemize}
\item {Grp. gram.:adj.}
\end{itemize}
Que anesthesia.
\section{Anesthesiar}
\begin{itemize}
\item {Grp. gram.:v. t.}
\end{itemize}
\begin{itemize}
\item {Proveniência:(De \textunderscore anesthesia\textunderscore )}
\end{itemize}
Tirar a sensibilidade a.
Deminuir a sensibilidade de.
\section{Anesthésica}
\begin{itemize}
\item {Grp. gram.:f.}
\end{itemize}
Lepra vulgar.
\section{Anesthésico}
\begin{itemize}
\item {Grp. gram.:adj.}
\end{itemize}
Relativo á \textunderscore anesthesia\textunderscore .
\section{Anesthético}
\begin{itemize}
\item {Grp. gram.:adj.}
\end{itemize}
\begin{itemize}
\item {Grp. gram.:M.}
\end{itemize}
Que produz anesthesia.
Aquillo que produz anesthesia. Cf. Latino, \textunderscore Elogios Acad.\textunderscore  314.
\section{Anete}
\begin{itemize}
\item {fónica:nê}
\end{itemize}
\begin{itemize}
\item {Grp. gram.:f.}
\end{itemize}
Argola da âncora.
Arganéu.
(Refl. de \textunderscore anel\textunderscore )
\section{Anetho}
\begin{itemize}
\item {Grp. gram.:m.}
\end{itemize}
\begin{itemize}
\item {Proveniência:(Gr. \textunderscore amethon\textunderscore )}
\end{itemize}
Planta umbellífera.
Funcho bastardo.
\section{Aneto}
\begin{itemize}
\item {Grp. gram.:m.}
\end{itemize}
\begin{itemize}
\item {Proveniência:(Gr. \textunderscore amethon\textunderscore )}
\end{itemize}
Planta umbellífera.
Funcho bastardo.
\section{Aneura}
\begin{itemize}
\item {Grp. gram.:f.}
\end{itemize}
\begin{itemize}
\item {Utilização:Bot.}
\end{itemize}
Gênero de hepáticas.
\section{Aneuro}
\begin{itemize}
\item {Grp. gram.:m.}
\end{itemize}
Gênero de hemípteros.
\section{Aneurisma}
\begin{itemize}
\item {Grp. gram.:m.}
\end{itemize}
\begin{itemize}
\item {Proveniência:(Gr. \textunderscore aneurusma\textunderscore )}
\end{itemize}
Tumor, formado no trajecto de uma artéria.
Dilatação das cavidades do coração.
\section{Aneurismal}
\begin{itemize}
\item {Grp. gram.:adj.}
\end{itemize}
Relativo a \textunderscore aneurisma\textunderscore .
\section{Aneurismático}
\begin{itemize}
\item {Grp. gram.:adj.}
\end{itemize}
Que tem aneurisma.
O mesmo que \textunderscore aneurismal\textunderscore .
\section{Aneurostesia}
\begin{itemize}
\item {Grp. gram.:f.}
\end{itemize}
Cessação da acção sensorial dos nervos.
\section{Aneurosthesia}
\begin{itemize}
\item {Grp. gram.:f.}
\end{itemize}
Cessação da acção sensorial dos nervos.
\section{Aneurotrofia}
\begin{itemize}
\item {Grp. gram.:f.}
\end{itemize}
Atrophia do eixo nervoso.
\section{Aneurotrophia}
\begin{itemize}
\item {Grp. gram.:f.}
\end{itemize}
Atrophia do eixo nervoso.
\section{Anexim}
\begin{itemize}
\item {Grp. gram.:m.}
\end{itemize}
\begin{itemize}
\item {Proveniência:(Do ár. \textunderscore an-nanxide\textunderscore )}
\end{itemize}
Dito sentencioso.
Rifão; sentença popular.
\section{Anexirista}
\begin{itemize}
\item {Grp. gram.:adj.}
\end{itemize}
\begin{itemize}
\item {Utilização:Ant.}
\end{itemize}
\begin{itemize}
\item {Utilização:Pop.}
\end{itemize}
\begin{itemize}
\item {Proveniência:(De \textunderscore anexir\textunderscore , por \textunderscore anexim\textunderscore )}
\end{itemize}
Que diz anexins.
Sentencioso.
\section{Anfacanto}
\begin{itemize}
\item {Grp. gram.:m.}
\end{itemize}
\begin{itemize}
\item {Proveniência:(Do gr. \textunderscore amphi\textunderscore  + \textunderscore akantha\textunderscore )}
\end{itemize}
Peixe do Oceano Índico.
\section{Anfanto}
\begin{itemize}
\item {Grp. gram.:m.}
\end{itemize}
\begin{itemize}
\item {Proveniência:(Do gr. \textunderscore amphi\textunderscore  + \textunderscore anthos\textunderscore )}
\end{itemize}
Receptáculo vegetal, que envolve e protege a flôr, como no figo.
\section{Anfásia}
\begin{itemize}
\item {Grp. gram.:f.}
\end{itemize}
\begin{itemize}
\item {Proveniência:(Do gr. \textunderscore amphi\textunderscore  + \textunderscore asis\textunderscore )}
\end{itemize}
Insecto coleóptero pentâmero da América do Norte.
\section{Anfesto}
\begin{itemize}
\item {Grp. gram.:adv.}
\end{itemize}
\begin{itemize}
\item {Utilização:Ant.}
\end{itemize}
Para cima; andando para cima:«\textunderscore e ahi como se vai por esse rio de Coira anfesto...\textunderscore »\textunderscore Tombo do Aro de Lamego\textunderscore , de 1346, f. 51.
\section{Anfi...}
\begin{itemize}
\item {Grp. gram.:pref.}
\end{itemize}
\begin{itemize}
\item {Proveniência:(Do gr. \textunderscore amphi\textunderscore )}
\end{itemize}
(design. de dois lados ou de dualidade)
\section{Anfião}
\begin{itemize}
\item {Grp. gram.:m.}
\end{itemize}
\begin{itemize}
\item {Utilização:Ant.}
\end{itemize}
\begin{itemize}
\item {Proveniência:(Do ár. \textunderscore an-fiun\textunderscore )}
\end{itemize}
O mesmo que \textunderscore ópio\textunderscore .
\section{Anfião}
\begin{itemize}
\item {Grp. gram.:m.}
\end{itemize}
\begin{itemize}
\item {Proveniência:(Gr. \textunderscore Amphion\textunderscore )}
\end{itemize}
Crustáceo do Oceano Índico.
\section{Anfibiano}
\begin{itemize}
\item {Grp. gram.:adj.}
\end{itemize}
\begin{itemize}
\item {Utilização:P. us.}
\end{itemize}
\begin{itemize}
\item {Grp. gram.:M. pl.}
\end{itemize}
O mesmo que \textunderscore anfíbio\textunderscore .
O mesmo que \textunderscore batrácios\textunderscore .
\section{Anfíbio}
\begin{itemize}
\item {Grp. gram.:m.  e  adj.}
\end{itemize}
\begin{itemize}
\item {Utilização:Fig.}
\end{itemize}
\begin{itemize}
\item {Proveniência:(Do gr. \textunderscore amphi\textunderscore  + \textunderscore bios\textunderscore )}
\end{itemize}
Diz-se do animal e da planta, que vivem na terra e na água.
Aquelle que sustenta opiniões oppostas ou segue duas profissões differentes.
\section{Anfibiografia}
\begin{itemize}
\item {Grp. gram.:f.}
\end{itemize}
\begin{itemize}
\item {Proveniência:(Do gr. \textunderscore amphibios\textunderscore  + \textunderscore graphein\textunderscore )}
\end{itemize}
Descripção dos animaes anfíbios.
\section{Anfibiologia}
\begin{itemize}
\item {Grp. gram.:f.}
\end{itemize}
\begin{itemize}
\item {Proveniência:(Do gr. \textunderscore amphibios\textunderscore  + \textunderscore logos\textunderscore )}
\end{itemize}
Parte da Zoologia, que trata dos animaes anfíbios.
\section{Anfibiológico}
\begin{itemize}
\item {Grp. gram.:adj.}
\end{itemize}
Relativo á \textunderscore anfibiologia\textunderscore .
\section{Anfibiólogo}
\begin{itemize}
\item {Grp. gram.:m.}
\end{itemize}
Aquelle que se dedica ao estudo da \textunderscore anfibiologia\textunderscore .
\section{Anfíbola}
\begin{itemize}
\item {Grp. gram.:f.}
\end{itemize}
O mesmo que \textunderscore anfíbolo\textunderscore .
\section{Anfibolia}
\begin{itemize}
\item {Grp. gram.:f.}
\end{itemize}
Equívoco, que, segundo Kant, consiste em considerar da mesma fórma e attribuir á mesma faculdade objectos próprios de faculdades differentes.
(Cp. \textunderscore amphíbolo\textunderscore )
\section{Anfibólico}
\begin{itemize}
\item {Grp. gram.:adj.}
\end{itemize}
Diz-se dos mineraes, em que entra o anfíbolo, como parte constituinte.
\section{Anfibolífero}
\begin{itemize}
\item {Grp. gram.:adj.}
\end{itemize}
\begin{itemize}
\item {Proveniência:(De \textunderscore amphibolo\textunderscore  + lat. \textunderscore ferre\textunderscore )}
\end{itemize}
Que encerra anfíbolo.
\section{Anfibolita}
\begin{itemize}
\item {Grp. gram.:f.}
\end{itemize}
Rocha, composta quási exclusivamente de anfíbolo.
\section{Anfíbolo}
\begin{itemize}
\item {Grp. gram.:m.}
\end{itemize}
\begin{itemize}
\item {Proveniência:(Gr. \textunderscore amphibolos\textunderscore )}
\end{itemize}
Substância mineral, composta de sílica, cal, magnésia e, ás vezes, óxydo de ferro e de manganés.
\section{Anfibologia}
\begin{itemize}
\item {Grp. gram.:f.}
\end{itemize}
\begin{itemize}
\item {Proveniência:(Do gr. \textunderscore amphibolos\textunderscore  + \textunderscore logos\textunderscore )}
\end{itemize}
Sentido ambíguo.
Disposição de palavras, que permitte mais de um sentido.
\section{Anfibologicamente}
\begin{itemize}
\item {Grp. gram.:adv.}
\end{itemize}
De modo \textunderscore anfibológico\textunderscore .
\section{Anfibológico}
\begin{itemize}
\item {Grp. gram.:adj.}
\end{itemize}
Que encerra anfibologia.
Ambíguo.
\section{Anfibologista}
\begin{itemize}
\item {Grp. gram.:m.}
\end{itemize}
Aquelle que escreve ou fala anfibologicamente.
\section{Anfiboloide}
\begin{itemize}
\item {Grp. gram.:adj.}
\end{itemize}
\begin{itemize}
\item {Proveniência:(Do gr. \textunderscore amphibolos\textunderscore  + \textunderscore eidos\textunderscore )}
\end{itemize}
Que tem anfíbolo.
\section{Anfibolóstilo}
\begin{itemize}
\item {Grp. gram.:adj.}
\end{itemize}
\begin{itemize}
\item {Utilização:Bot.}
\end{itemize}
Diz-se das plantas, cujo estilete é pouco visível.
\section{Anfíbraco}
\begin{itemize}
\item {Grp. gram.:m.}
\end{itemize}
\begin{itemize}
\item {Proveniência:(Do gr. \textunderscore amphi\textunderscore  + \textunderscore brachus\textunderscore )}
\end{itemize}
Pé de verso grego ou latino, com uma sýllaba longa entre duas breves.
\section{Anficarpo}
\begin{itemize}
\item {Grp. gram.:m.}
\end{itemize}
\begin{itemize}
\item {Proveniência:(Do gr. \textunderscore amphi\textunderscore  + \textunderscore karpos\textunderscore )}
\end{itemize}
Planta, da fam. das gramíneas.
\section{Anfícomo}
\begin{itemize}
\item {Grp. gram.:m.}
\end{itemize}
Gênero de coleópteros.
\section{Anficrânia}
\begin{itemize}
\item {Grp. gram.:f.}
\end{itemize}
\begin{itemize}
\item {Proveniência:(Do gr. \textunderscore amphikranos\textunderscore )}
\end{itemize}
Insecto coleóptero pentâmero.
\section{Anfictiões}
\begin{itemize}
\item {Grp. gram.:m. pl.}
\end{itemize}
\begin{itemize}
\item {Utilização:Ant.}
\end{itemize}
\begin{itemize}
\item {Proveniência:(Do gr. \textunderscore amphiktuon\textunderscore )}
\end{itemize}
Representantes dos Estados gregos, que se reuniam para deliberar sôbre negocios geraes.
\section{Anfictionia}
\begin{itemize}
\item {Grp. gram.:f.}
\end{itemize}
\begin{itemize}
\item {Proveniência:(Do gr. \textunderscore amphiktuon\textunderscore )}
\end{itemize}
Reunião dos anfictiões.
Direito de sêr representado nessa assembleia.
\section{Anfictiónio}
\begin{itemize}
\item {Grp. gram.:adj.}
\end{itemize}
Relativo aos \textunderscore anfictiões\textunderscore .
\section{Anficiclo}
\begin{itemize}
\item {Grp. gram.:m.}
\end{itemize}
\begin{itemize}
\item {Utilização:P. us.}
\end{itemize}
\begin{itemize}
\item {Proveniência:(Do gr. \textunderscore amphi\textunderscore  + \textunderscore kuklos\textunderscore )}
\end{itemize}
O crescente da lua.
\section{Anfídases}
\begin{itemize}
\item {Grp. gram.:m. pl.}
\end{itemize}
\begin{itemize}
\item {Proveniência:(Do gr. \textunderscore amphidasus\textunderscore )}
\end{itemize}
Insectos lepidópteros nocturnos.
\section{Anfideão}
\begin{itemize}
\item {Grp. gram.:m.}
\end{itemize}
\begin{itemize}
\item {Utilização:Anat.}
\end{itemize}
Orifício do útero.
\section{Anfidésmio}
\begin{itemize}
\item {Grp. gram.:m.}
\end{itemize}
Gênero de plantas polypódeas.
\section{Anfidesmo}
\begin{itemize}
\item {Grp. gram.:m.}
\end{itemize}
\begin{itemize}
\item {Proveniência:(Do gr. \textunderscore amphi\textunderscore  + \textunderscore desmos\textunderscore )}
\end{itemize}
Mollusco acéphalo.
\section{Anfido}
\begin{itemize}
\item {Grp. gram.:m.}
\end{itemize}
\begin{itemize}
\item {Utilização:Chím.}
\end{itemize}
Sal, de composição ternária, resultante da combinação de um
ácido com qualquer base, (seg. Berzélio)
\section{Anfídoro}
\begin{itemize}
\item {Grp. gram.:m.}
\end{itemize}
Gênero de coleópteros.
\section{Anfidoxa}
\begin{itemize}
\item {fónica:csa}
\end{itemize}
\begin{itemize}
\item {Grp. gram.:f.}
\end{itemize}
Gênero de plantas compostas, na África austral.
\section{Anfidromia}
\begin{itemize}
\item {Grp. gram.:f.}
\end{itemize}
\begin{itemize}
\item {Proveniência:(Do gr. \textunderscore amphidromos\textunderscore )}
\end{itemize}
Festa, com que os antigos Gregos saudavam os nascimentos e em que davam nome aos recém-nascidos.
\section{Anfigênias}
\begin{itemize}
\item {Grp. gram.:f. pl.}
\end{itemize}
Vegetaes, que têm desenvolvimento discoide.
(Cp. \textunderscore amphigênio\textunderscore )
\section{Anfigênico}
\begin{itemize}
\item {Grp. gram.:adj.}
\end{itemize}
Relativo ás \textunderscore anfigênias\textunderscore .
\section{Anfigênio}
\begin{itemize}
\item {Grp. gram.:m.}
\end{itemize}
\begin{itemize}
\item {Proveniência:(Do gr. \textunderscore amphi\textunderscore  + \textunderscore genos\textunderscore )}
\end{itemize}
Silicato de potassa e de alumina.
\section{Anfiglossa}
\begin{itemize}
\item {Grp. gram.:f.}
\end{itemize}
\begin{itemize}
\item {Proveniência:(Gr. \textunderscore amphiglossos\textunderscore )}
\end{itemize}
Gênero de plantas da fam. das compostas.
\section{Anfigonia}
\begin{itemize}
\item {Grp. gram.:f.}
\end{itemize}
\begin{itemize}
\item {Proveniência:(T. criado por Spencer)}
\end{itemize}
Geração sexual.
\section{Anfiguri}
\begin{itemize}
\item {Grp. gram.:m.}
\end{itemize}
\begin{itemize}
\item {Proveniência:(Gr. \textunderscore amphigouri\textunderscore )}
\end{itemize}
Discurso ou trecho, feito para sêr inintelligível.
Qualquer peça literária, desordenada e sem sentido.
\section{Anfiguricamente}
\begin{itemize}
\item {Grp. gram.:adv.}
\end{itemize}
De modo \textunderscore anfigúrico\textunderscore .
\section{Anfigúrico}
\begin{itemize}
\item {Grp. gram.:adj.}
\end{itemize}
Que encerra anfiguri.
\section{Anfigurítico}
\begin{itemize}
\item {Grp. gram.:adj.}
\end{itemize}
Que tem fórma de anfiguri. Cf. Herculano, \textunderscore Carta a Torresão\textunderscore .
\section{Anfíloco}
\begin{itemize}
\item {Grp. gram.:m.}
\end{itemize}
Gênero de coleópteros.
\section{Anfílofo}
\begin{itemize}
\item {Grp. gram.:m.}
\end{itemize}
\begin{itemize}
\item {Proveniência:(Do gr. \textunderscore amphi\textunderscore  + \textunderscore lophos\textunderscore )}
\end{itemize}
Planta americana, da fam. das begoniáceas.
\section{Anfímacro}
\begin{itemize}
\item {Grp. gram.:m.}
\end{itemize}
\begin{itemize}
\item {Proveniência:(Do gr. \textunderscore amphi\textunderscore  + \textunderscore makros\textunderscore )}
\end{itemize}
Pé de verso grego ou latino com uma sýllaba breve entre duas longas.
\section{Anfímeno}
\begin{itemize}
\item {Grp. gram.:m.}
\end{itemize}
Gênero de leguminosas.
\section{Anfimétrico}
\begin{itemize}
\item {Grp. gram.:adj.}
\end{itemize}
\begin{itemize}
\item {Proveniência:(Do gr. \textunderscore amphi\textunderscore  + \textunderscore metron\textunderscore )}
\end{itemize}
Diz-se de qualquer substância mineral, cujos crystaes offerecem incidência igual em certas faces.
\section{Anfimónia}
\begin{itemize}
\item {Grp. gram.:f.}
\end{itemize}
Gênero de leguminosas.
\section{Anfiodonte}
\begin{itemize}
\item {Grp. gram.:m.}
\end{itemize}
Gênero de peixes da América do Norte.
\section{Anfioxo}
\begin{itemize}
\item {fónica:cso}
\end{itemize}
\begin{itemize}
\item {Grp. gram.:m.}
\end{itemize}
Pequeno peixe, sem crânio nem cérebro, que vive occulto nas areias do Mediterrâneo.
\section{Anfipira}
\begin{itemize}
\item {Grp. gram.:f.}
\end{itemize}
Gênero de lepidópteros.
\section{Anfipneustos}
\begin{itemize}
\item {Grp. gram.:m. pl.}
\end{itemize}
\begin{itemize}
\item {Proveniência:(Do gr. \textunderscore amphi\textunderscore  + \textunderscore pneo\textunderscore )}
\end{itemize}
Classe de reptis, com dois apparelhos respiratórios.
\section{Anfípodes}
\begin{itemize}
\item {Grp. gram.:m. pl.}
\end{itemize}
\begin{itemize}
\item {Proveniência:(Do gr. \textunderscore amphi\textunderscore  + \textunderscore pous\textunderscore , \textunderscore podos\textunderscore )}
\end{itemize}
Ordem ou sub-ordem de crustáceos, que têm duas qualidades de pés, para nadar e saltar.
\section{Anfipogão}
\begin{itemize}
\item {Grp. gram.:m.}
\end{itemize}
Gênero de gramíneas.
\section{Anfipróstilo}
\begin{itemize}
\item {Grp. gram.:m.}
\end{itemize}
Diz-se de alguns templos antigos, com duas ordens de columnas, uma na parte anterior e outra na posterior.
(Do gr.)
\section{Anfíptero}
\begin{itemize}
\item {Grp. gram.:m.}
\end{itemize}
\begin{itemize}
\item {Utilização:Heráld.}
\end{itemize}
Dragão ou serpente com asas de morcego.
\section{Anfisarca}
\begin{itemize}
\item {Grp. gram.:m.}
\end{itemize}
Fruto plurilocular, indehiscente, exteriormente duro e interiormente carnudo.
\section{Anfisbena}
\begin{itemize}
\item {Grp. gram.:f.}
\end{itemize}
\begin{itemize}
\item {Utilização:P. us.}
\end{itemize}
\begin{itemize}
\item {Proveniência:(Do gr. \textunderscore amphi\textunderscore  + \textunderscore bainein\textunderscore )}
\end{itemize}
Serpente, que parece têr duas cabeças, uma em cada extremidade.
\section{Anfiscianos}
\begin{itemize}
\item {Grp. gram.:m. pl.}
\end{itemize}
O mesmo que anfíscios.
\section{Anfíscios}
\begin{itemize}
\item {Grp. gram.:m. pl.}
\end{itemize}
\begin{itemize}
\item {Proveniência:(Do gr. \textunderscore amphi\textunderscore  + \textunderscore skia\textunderscore )}
\end{itemize}
Habitantes das regiões equatoriaes, que umas vezes projectam a sua sombra para o norte e outras para o sul, conforme o Sol está abaixo ou acima do Equador.
\section{Anfiscópia}
\begin{itemize}
\item {Grp. gram.:f.}
\end{itemize}
Gênero de acantáceas brasileiras.
\section{Anfistauro}
\begin{itemize}
\item {Grp. gram.:m.}
\end{itemize}
Gênero de coleópteros.
\section{Anfístomos}
\begin{itemize}
\item {Grp. gram.:m. pl.}
\end{itemize}
Gêneros de gusanos intestinaes.
\section{Anfitálamo}
\begin{itemize}
\item {Grp. gram.:m.}
\end{itemize}
\begin{itemize}
\item {Proveniência:(Lat. \textunderscore ampithalamus\textunderscore )}
\end{itemize}
Compartimento, annexo ao quarto da cama, nas habitações gregas e romanas, e em que dormiam ou trabalhavam as escravas.
\section{Anfiteatral}
\begin{itemize}
\item {Grp. gram.:adj.}
\end{itemize}
Que diz respeito a anfiteatro.
\section{Anfiteátrico}
\begin{itemize}
\item {Grp. gram.:adj.}
\end{itemize}
O mesmo que \textunderscore anfiteatral\textunderscore . Cf. Castilho, \textunderscore Fastos\textunderscore , I, 311 e 312.
\section{Anfiteatro}
\begin{itemize}
\item {Grp. gram.:m.}
\end{itemize}
\begin{itemize}
\item {Proveniência:(Gr. \textunderscore amphitheatron\textunderscore )}
\end{itemize}
Antigo edifício oval ou circular, para espectaculos de feras
\textunderscore ou\textunderscore  gladiadores, e para representações.
Construcção circular, com degraus, nos theatros, nas escolas, etc.
Os espectadores.
\section{Anfitrite}
\begin{itemize}
\item {Grp. gram.:f.}
\end{itemize}
\begin{itemize}
\item {Utilização:Fig.}
\end{itemize}
\begin{itemize}
\item {Proveniência:(Do gr. \textunderscore Amphitrite\textunderscore , n. p.)}
\end{itemize}
Gênero de vermes marinhos.
O mar.
\section{Anfítropo}
\begin{itemize}
\item {Grp. gram.:adj.}
\end{itemize}
\begin{itemize}
\item {Proveniência:(Do gr. \textunderscore amphi\textunderscore  + \textunderscore trepein\textunderscore )}
\end{itemize}
Diz-se, em Botânica, do embryão recurvado, cujas duas extremidades se dirigem ambas para o hilo.
\section{Anfitrião}
\begin{itemize}
\item {Grp. gram.:m.}
\end{itemize}
\begin{itemize}
\item {Proveniência:(Do gr. \textunderscore Amphitruon\textunderscore , n. p.)}
\end{itemize}
O dono da casa, em que se serve jantar lauto a muitos convidados.
Aquelle que paga as despezas de uma comezaina.
\section{Anfitriónio}
\begin{itemize}
\item {Grp. gram.:adj.}
\end{itemize}
Próprio de anftriões. Cf. \textunderscore Eufrosina\textunderscore , prólogo.
\section{Anfiúma}
\begin{itemize}
\item {Grp. gram.:f.}
\end{itemize}
Gênero de reptis.
\section{Anfodiplopia}
\begin{itemize}
\item {Grp. gram.:f.}
\end{itemize}
\begin{itemize}
\item {Proveniência:(De \textunderscore amphi\textunderscore  + \textunderscore diplonia\textunderscore )}
\end{itemize}
Vício da visão que apresenta os objectos duplicados em ambos os olhos.
\section{Anfodonte}
\begin{itemize}
\item {Grp. gram.:m.}
\end{itemize}
Gênero de leguminosas.
\section{Ânfora}
\begin{itemize}
\item {Grp. gram.:f.}
\end{itemize}
\begin{itemize}
\item {Proveniência:(Lat. \textunderscore amphora\textunderscore )}
\end{itemize}
Vaso grande de duas asas, para líquidos, usado antigamente entre Gregos e Romanos.
Hoje, vaso semelhante àquelle.
Valva de alguns frutos, que se fendem na época da maturação.
Designação ant. do signo de Aquário.
\section{Anforal}
\begin{itemize}
\item {Grp. gram.:adj.}
\end{itemize}
\begin{itemize}
\item {Utilização:Poét.}
\end{itemize}
Contido em ânfora.
\section{Anforicidade}
\begin{itemize}
\item {Grp. gram.:f.}
\end{itemize}
\begin{itemize}
\item {Utilização:Med.}
\end{itemize}
Existência do ruído anfórico dentro do peito.
\section{Anfórico}
\begin{itemize}
\item {Grp. gram.:adj.}
\end{itemize}
\begin{itemize}
\item {Proveniência:(De \textunderscore âmphora\textunderscore )}
\end{itemize}
Diz-se do som, que se ouve dentro do peito auscultado, pela semelhança com o som que se obtém, soprando para dentro de uma ânfora vazia.
\section{Anfracto}
\begin{itemize}
\item {Grp. gram.:m.}
\end{itemize}
\begin{itemize}
\item {Utilização:Ant.}
\end{itemize}
\begin{itemize}
\item {Proveniência:(Lat. \textunderscore anfractus\textunderscore )}
\end{itemize}
O mesmo que \textunderscore anfractuosidade\textunderscore .
\section{Anfractuosidade}
\begin{itemize}
\item {Grp. gram.:f.}
\end{itemize}
Qualidade do que é \textunderscore anfractuoso\textunderscore .
\section{Anfractuoso}
\begin{itemize}
\item {Grp. gram.:adj.}
\end{itemize}
\begin{itemize}
\item {Proveniência:(Lat. \textunderscore anfractuosus\textunderscore )}
\end{itemize}
Que tem sinuosidades e curvaturas em vários sentidos.
Que apresenta elevações e depressões.
\section{Angá}
\begin{itemize}
\item {Grp. gram.:m.}
\end{itemize}
Fruta do Brasil.
Árvore silvestre, que dá aquelle fruto.
\section{Angana}
\begin{itemize}
\item {Grp. gram.:f.}
\end{itemize}
\begin{itemize}
\item {Utilização:Bras}
\end{itemize}
Senhora, mulher do senhor.
A filha mais velha da senhora.
Tratamento, dado pelos pais a suas filhas.
(Do quimbundo \textunderscore ngana\textunderscore , senhor)
\section{Ângar}
\begin{itemize}
\item {Grp. gram.:m.}
\end{itemize}
(V.hangar)
\section{Ângara}
\begin{itemize}
\item {Grp. gram.:m.}
\end{itemize}
\begin{itemize}
\item {Utilização:Ant.}
\end{itemize}
\begin{itemize}
\item {Proveniência:(Gr. \textunderscore angaros\textunderscore )}
\end{itemize}
Correio, portador de despachos, entre os Persas.
\section{Angareira}
\begin{itemize}
\item {Grp. gram.:f.}
\end{itemize}
\begin{itemize}
\item {Utilização:Bras. do N}
\end{itemize}
Pequena rede de malhas miúdas, empregada na pesca da taínha.
(Cp. \textunderscore ângar\textunderscore )
\section{Angarela}
\begin{itemize}
\item {Grp. gram.:f.}
\end{itemize}
\begin{itemize}
\item {Utilização:Prov.}
\end{itemize}
Sebe ou conjunto dos fueiros, com que se ampara a carrada de feno.
\section{Angária}
\begin{itemize}
\item {Grp. gram.:f.}
\end{itemize}
\begin{itemize}
\item {Proveniência:(Lat. \textunderscore angaria\textunderscore )}
\end{itemize}
Requisição, aluguel de bêstas de carga.
\section{Angariação}
\begin{itemize}
\item {Grp. gram.:f.}
\end{itemize}
Acto de \textunderscore angariar\textunderscore .
\section{Angariar}
\begin{itemize}
\item {Grp. gram.:v. t.}
\end{itemize}
\begin{itemize}
\item {Proveniência:(Lat. \textunderscore angariare\textunderscore )}
\end{itemize}
Alliciar; attrahir.
Recrutar.
\section{Angariári}
\begin{itemize}
\item {Grp. gram.:f.}
\end{itemize}
O mesmo que \textunderscore angariária\textunderscore .
\section{Angariária}
\begin{itemize}
\item {Grp. gram.:f.}
\end{itemize}
Árvore do Congo, cuja raíz e madeira são medicinaes.
\section{Angarilha}
\begin{itemize}
\item {Grp. gram.:f.}
\end{itemize}
Capa de vime ou de palha, com que se envolvem frascos ou vasos, para se não partirem, facilmente.
Balsa.
(Cast. \textunderscore angarilla\textunderscore )
\section{Ângela}
\begin{itemize}
\item {Grp. gram.:f.  e  adj.}
\end{itemize}
Variedade de pêra, originária da França.
\section{Angelado}
\begin{itemize}
\item {Grp. gram.:adj.}
\end{itemize}
\begin{itemize}
\item {Utilização:Ant.}
\end{itemize}
O mesmo que \textunderscore angélico\textunderscore ^1.
\section{Angelato}
\begin{itemize}
\item {Grp. gram.:m.}
\end{itemize}
Sal, resultante da combinação do ácido angelícico com uma base.
\section{Angélia}
\begin{itemize}
\item {Grp. gram.:f.}
\end{itemize}
\begin{itemize}
\item {Utilização:Des.}
\end{itemize}
\begin{itemize}
\item {Utilização:Poét.}
\end{itemize}
\begin{itemize}
\item {Proveniência:(De \textunderscore Angélia\textunderscore , n. p. myth.)}
\end{itemize}
Mensageira.
Aurora.
\section{Angélica}
\begin{itemize}
\item {Grp. gram.:f.}
\end{itemize}
\begin{itemize}
\item {Proveniência:(De \textunderscore angélico\textunderscore )}
\end{itemize}
Planta umbellífera, medicinal e aromática.
Planta liliácea, de flôr branca, muito odorífera.
Nome de duas variedades de pêras.
Antigo instrumento, espécie de cravo pequeno, com dezasete cordas e dez teclas.
\section{Angelíca}
\begin{itemize}
\item {Grp. gram.:f.}
\end{itemize}
\begin{itemize}
\item {Utilização:beir.}
\end{itemize}
\begin{itemize}
\item {Utilização:Prov.}
\end{itemize}
\begin{itemize}
\item {Utilização:açor}
\end{itemize}
\begin{itemize}
\item {Utilização:trasm.}
\end{itemize}
O mesmo que \textunderscore geropiga\textunderscore .
\section{Angelical}
\begin{itemize}
\item {Grp. gram.:adj.}
\end{itemize}
O mesmo que \textunderscore angélico\textunderscore ^1.
\section{Angelicamente}
\begin{itemize}
\item {fónica:gé}
\end{itemize}
\begin{itemize}
\item {Grp. gram.:adv.}
\end{itemize}
De modo \textunderscore angélico\textunderscore .
\section{Angelicato}
\begin{itemize}
\item {Grp. gram.:m.}
\end{itemize}
O mesmo que \textunderscore angelato\textunderscore .
\section{Angelíceo}
\begin{itemize}
\item {Grp. gram.:adj.}
\end{itemize}
Semelhante á angélica.
\section{Angelícico}
\begin{itemize}
\item {Grp. gram.:adj.}
\end{itemize}
Diz-se de um ácido, extrahido da raíz da angélica.
\section{Angélico}
\begin{itemize}
\item {Grp. gram.:adj.}
\end{itemize}
\begin{itemize}
\item {Grp. gram.:M.}
\end{itemize}
\begin{itemize}
\item {Proveniência:(Lat. \textunderscore angelicus\textunderscore )}
\end{itemize}
Relativo ou semelhante a anjos.
Puríssimo.
Formosíssimo.
* Planta medicinal do Brasil.
\section{Angélico}
\begin{itemize}
\item {Grp. gram.:adj.}
\end{itemize}
O mesmo que \textunderscore angelícico\textunderscore .
\section{Angelim}
\begin{itemize}
\item {Grp. gram.:m.}
\end{itemize}
Designação de várias árvores leguminosas do Brasil, da Índia e da China.
\section{Angelina}
\begin{itemize}
\item {Grp. gram.:f.}
\end{itemize}
Planeta telescópico, descoberto por Tempel.
\section{Angelino}
\begin{itemize}
\item {Grp. gram.:adj.}
\end{itemize}
O mesmo que \textunderscore angélico\textunderscore ^1.
\section{Angelitude}
\begin{itemize}
\item {Grp. gram.:f.}
\end{itemize}
\begin{itemize}
\item {Utilização:Neol.}
\end{itemize}
Estado de anjo. Cf. Raimundo Correia.
\section{Angelização}
\begin{itemize}
\item {Grp. gram.:f.}
\end{itemize}
Acto ou effeito de \textunderscore angelizar\textunderscore .
Estado comparável ao dos anjos. Cf. Cortesão, \textunderscore Subs.\textunderscore 
\section{Angelizar}
\begin{itemize}
\item {Grp. gram.:v. t.}
\end{itemize}
\begin{itemize}
\item {Proveniência:(Do lat. \textunderscore angelus\textunderscore )}
\end{itemize}
Comparar a anjo.
\section{Angelogonia}
\begin{itemize}
\item {Grp. gram.:f.}
\end{itemize}
Theoria sôbre a origem e natureza dos anjos.
\section{Angelografia}
\begin{itemize}
\item {Grp. gram.:f.}
\end{itemize}
O mesmo que \textunderscore angelologia\textunderscore .
\section{Angelographia}
\begin{itemize}
\item {Grp. gram.:f.}
\end{itemize}
O mesmo que \textunderscore angelologia\textunderscore .
\section{Angelolatria}
\begin{itemize}
\item {Grp. gram.:f.}
\end{itemize}
\begin{itemize}
\item {Proveniência:(Do gr. \textunderscore angelos\textunderscore  + \textunderscore latreia\textunderscore )}
\end{itemize}
Culto prestado aos anjos.
\section{Angelologia}
\begin{itemize}
\item {Grp. gram.:f.}
\end{itemize}
\begin{itemize}
\item {Proveniência:(Do gr. \textunderscore angelos\textunderscore  + \textunderscore logos\textunderscore )}
\end{itemize}
Tratado á cêrca dos anjos.
Crença na existência e intervenção dos anjos.
\section{Angelónia}
\begin{itemize}
\item {Grp. gram.:f.}
\end{itemize}
Gênero de plantas escrofularíneas.
\section{Angelote}
\begin{itemize}
\item {Grp. gram.:m.}
\end{itemize}
\begin{itemize}
\item {Proveniência:(Fr. \textunderscore angelot\textunderscore )}
\end{itemize}
Antiga moéda francesa do tempo de San-Luis.
\section{Angerona}
\begin{itemize}
\item {Grp. gram.:f.}
\end{itemize}
\begin{itemize}
\item {Proveniência:(De \textunderscore Angerona\textunderscore , n. p.)}
\end{itemize}
Insecto lepidóptero nocturno.
\section{Angi}
\begin{itemize}
\item {Grp. gram.:m.}
\end{itemize}
Ave de Angola.
\section{Angiairia}
\begin{itemize}
\item {Grp. gram.:f.}
\end{itemize}
Doença das vias respiratórias.
\section{Angianta}
\begin{itemize}
\item {Grp. gram.:f.}
\end{itemize}
Gênero de plantas synanthéreas.
\section{Angiantha}
\begin{itemize}
\item {Grp. gram.:f.}
\end{itemize}
Gênero de plantas synanthéreas.
\section{Angico}
\begin{itemize}
\item {Grp. gram.:m.}
\end{itemize}
Espécie de acácia do Brasil.
\section{Angiectasia}
\begin{itemize}
\item {Grp. gram.:f.}
\end{itemize}
\begin{itemize}
\item {Utilização:Med.}
\end{itemize}
Dilatação dos vasos sanguíneos.
\section{Angiectásico}
\begin{itemize}
\item {Grp. gram.:adj.}
\end{itemize}
Relativo á \textunderscore angiectasia\textunderscore .
\section{Angina}
\begin{itemize}
\item {Grp. gram.:f.}
\end{itemize}
\begin{itemize}
\item {Proveniência:(Lat. \textunderscore angina\textunderscore )}
\end{itemize}
Affecção inflammatória, mais ou menos intensa, nas fauces, na pharynge, na larynge ou na tracheia.
\section{Anginoso}
\begin{itemize}
\item {Grp. gram.:adj.}
\end{itemize}
Relativo a angina.
\section{Angiocarpo}
\begin{itemize}
\item {Grp. gram.:m.}
\end{itemize}
\begin{itemize}
\item {Proveniência:(Do gr. \textunderscore angeion\textunderscore  + \textunderscore karpos\textunderscore )}
\end{itemize}
Fruto de certas plantas, coberto por órgão estranho, que não permitte conhecê-lo á primeira vista.
Espécie de cogumelo.
\section{Angiocholite}
\begin{itemize}
\item {fónica:co}
\end{itemize}
\begin{itemize}
\item {Grp. gram.:f.}
\end{itemize}
\begin{itemize}
\item {Utilização:Med.}
\end{itemize}
Inflammação dos canaes biliares.
\section{Angiocolite}
\begin{itemize}
\item {Grp. gram.:f.}
\end{itemize}
\begin{itemize}
\item {Utilização:Med.}
\end{itemize}
Inflammação dos canaes biliares.
\section{Angiogalia}
\begin{itemize}
\item {Grp. gram.:f.}
\end{itemize}
Doença do apparelho secretor do leite.
\section{Ângio}
\begin{itemize}
\item {Grp. gram.:m.}
\end{itemize}
Fórma antiga de \textunderscore anjo\textunderscore .
\section{Angiografia}
\begin{itemize}
\item {Grp. gram.:f.}
\end{itemize}
\begin{itemize}
\item {Proveniência:(Do gr. \textunderscore angeion\textunderscore  + \textunderscore graphein\textunderscore )}
\end{itemize}
Descripção dos vasos do corpo humano.
\section{Angiográfico}
\begin{itemize}
\item {Grp. gram.:adj.}
\end{itemize}
Relativo á angiografia.
\section{Angiographia}
\begin{itemize}
\item {Grp. gram.:f.}
\end{itemize}
\begin{itemize}
\item {Proveniência:(Do gr. \textunderscore angeion\textunderscore  + \textunderscore graphein\textunderscore )}
\end{itemize}
Descripção dos vasos do corpo humano.
\section{Angiográphico}
\begin{itemize}
\item {Grp. gram.:adj.}
\end{itemize}
Relativo á angiographia.
\section{Angioleucite}
\begin{itemize}
\item {Grp. gram.:f.}
\end{itemize}
Inflammação dos vasos lympháticos.
\section{Angioleucómoro}
\begin{itemize}
\item {Grp. gram.:m.}
\end{itemize}
\begin{itemize}
\item {Utilização:Anat.}
\end{itemize}
\begin{itemize}
\item {Proveniência:(Do gr. \textunderscore angeion\textunderscore  + \textunderscore leukos\textunderscore  + \textunderscore meros\textunderscore )}
\end{itemize}
Subtypo do angiómero ou metâmero lymphático.
\section{Angiologia}
\begin{itemize}
\item {Grp. gram.:f.}
\end{itemize}
\begin{itemize}
\item {Proveniência:(Do gr. \textunderscore angeion\textunderscore  + \textunderscore logos\textunderscore )}
\end{itemize}
Parte da Anatomia, que trata dos vasos.
\section{Angioma}
\begin{itemize}
\item {Grp. gram.:m.}
\end{itemize}
\begin{itemize}
\item {Utilização:Med.}
\end{itemize}
Tumor, formado por vasos sanguíneos.
\section{Angiomérico}
\begin{itemize}
\item {Grp. gram.:adj.}
\end{itemize}
Relativo ao \textunderscore angiómero\textunderscore .
\section{Angiómero}
\begin{itemize}
\item {Grp. gram.:m.}
\end{itemize}
\begin{itemize}
\item {Utilização:Anat.}
\end{itemize}
\begin{itemize}
\item {Proveniência:(Do gr. \textunderscore angeion\textunderscore  + \textunderscore meros\textunderscore )}
\end{itemize}
Parte vascular do metâmero.
\section{Angionoma}
\begin{itemize}
\item {Grp. gram.:m.}
\end{itemize}
\begin{itemize}
\item {Utilização:Med.}
\end{itemize}
Úlcera nos vasos.
\section{Angiopathia}
\begin{itemize}
\item {Grp. gram.:f.}
\end{itemize}
\begin{itemize}
\item {Proveniência:(Do gr. \textunderscore angeion\textunderscore  + \textunderscore pathos\textunderscore )}
\end{itemize}
Doença dos vasos.
\section{Angiopatia}
\begin{itemize}
\item {Grp. gram.:f.}
\end{itemize}
\begin{itemize}
\item {Proveniência:(Do gr. \textunderscore angeion\textunderscore  + \textunderscore pathos\textunderscore )}
\end{itemize}
Doença dos vasos.
\section{Angiopiria}
\begin{itemize}
\item {Grp. gram.:f.}
\end{itemize}
\begin{itemize}
\item {Utilização:Med.}
\end{itemize}
Febre inflammatória.
\section{Angiopyria}
\begin{itemize}
\item {Grp. gram.:f.}
\end{itemize}
\begin{itemize}
\item {Utilização:Med.}
\end{itemize}
Febre inflammatória.
\section{Angiorragia}
\begin{itemize}
\item {Grp. gram.:f.}
\end{itemize}
\begin{itemize}
\item {Utilização:Med.}
\end{itemize}
Fluxo de sangue por excesso de fôrça.
\section{Angiorreia}
\begin{itemize}
\item {Grp. gram.:f.}
\end{itemize}
Fluxo de sangue por fraqueza.
\section{Angiorrhagia}
\begin{itemize}
\item {Grp. gram.:f.}
\end{itemize}
\begin{itemize}
\item {Utilização:Med.}
\end{itemize}
Fluxo de sangue por excesso de fôrça.
\section{Angiorrheia}
\begin{itemize}
\item {Grp. gram.:f.}
\end{itemize}
Fluxo de sangue por fraqueza.
\section{Angioscopia}
\begin{itemize}
\item {Grp. gram.:f.}
\end{itemize}
Applicação do angioscópio.
\section{Angioscópico}
\begin{itemize}
\item {Grp. gram.:adj.}
\end{itemize}
Relativo á \textunderscore angioscopia\textunderscore .
\section{Angioscópio}
\begin{itemize}
\item {Grp. gram.:m.}
\end{itemize}
\begin{itemize}
\item {Proveniência:(Do gr. \textunderscore angeion\textunderscore  + \textunderscore skopein\textunderscore )}
\end{itemize}
Instrumento, para observar os vasos capillares.
\section{Angiose}
\begin{itemize}
\item {Grp. gram.:f.}
\end{itemize}
\begin{itemize}
\item {Utilização:Med.}
\end{itemize}
\begin{itemize}
\item {Proveniência:(Do gr. \textunderscore angeion\textunderscore )}
\end{itemize}
Designação genérica das enfermidades que têm a sua séde no systema vascular sanguíneo.
\section{Angiospermário}
\begin{itemize}
\item {Grp. gram.:adj.}
\end{itemize}
\begin{itemize}
\item {Utilização:Geol.}
\end{itemize}
Diz-se do período geológico, em que appareceram as plantas angiospérmicas.
\section{Angiosperme}
\begin{itemize}
\item {Grp. gram.:adj.}
\end{itemize}
\begin{itemize}
\item {Proveniência:(Do gr. \textunderscore angeion\textunderscore  + \textunderscore sperma\textunderscore )}
\end{itemize}
Cujas sementes têm pericardo distinto.
\section{Angiospérmia}
\begin{itemize}
\item {Grp. gram.:f.}
\end{itemize}
\begin{itemize}
\item {Proveniência:(Do gr. \textunderscore angeion\textunderscore  + \textunderscore sperma\textunderscore )}
\end{itemize}
Ordem de plantas, (segundo Lin.).
\section{Angiospérmico}
\begin{itemize}
\item {Grp. gram.:adj.}
\end{itemize}
\begin{itemize}
\item {Proveniência:(De \textunderscore angiospérmia\textunderscore )}
\end{itemize}
Diz-se do vegetal, cujos grãos são revestidos por um pericarpo distinto.
\section{Angiósporo}
\begin{itemize}
\item {Grp. gram.:adj.}
\end{itemize}
\begin{itemize}
\item {Proveniência:(Do gr. \textunderscore angeion\textunderscore  + \textunderscore spora\textunderscore )}
\end{itemize}
Diz-se dos cogumelos, cujos espórulos estão situados internamente.
\section{Angióstoma}
\begin{itemize}
\item {Grp. gram.:m.}
\end{itemize}
Gênero de vermes intestinaes.
\section{Angiotenia}
\begin{itemize}
\item {Grp. gram.:f.}
\end{itemize}
\begin{itemize}
\item {Proveniência:(Do gr. \textunderscore angeion\textunderscore  + \textunderscore teinein\textunderscore )}
\end{itemize}
Febre inflammatória, por irritação do systema vascular sanguíneo.
\section{Angiotênico}
\begin{itemize}
\item {Grp. gram.:adj.}
\end{itemize}
\begin{itemize}
\item {Proveniência:(De \textunderscore angiotenia\textunderscore )}
\end{itemize}
Inflammatório.
\section{Angiotite}
\begin{itemize}
\item {Grp. gram.:f.}
\end{itemize}
\begin{itemize}
\item {Utilização:Med.}
\end{itemize}
Inflammação do systema vascular sanguíneo.
\section{Angiotomia}
\begin{itemize}
\item {Grp. gram.:f.}
\end{itemize}
\begin{itemize}
\item {Utilização:Cir.}
\end{itemize}
\begin{itemize}
\item {Proveniência:(Do gr. \textunderscore angeion\textunderscore  + \textunderscore tome\textunderscore )}
\end{itemize}
Disseccação dos vasos.
\section{Angiotómico}
\begin{itemize}
\item {Grp. gram.:adj.}
\end{itemize}
Relativo á angiotomia.
\section{Angiporto}
\begin{itemize}
\item {Grp. gram.:m.}
\end{itemize}
\begin{itemize}
\item {Utilização:Ant.}
\end{itemize}
\begin{itemize}
\item {Proveniência:(Lat. \textunderscore angiportus\textunderscore )}
\end{itemize}
Rua estreita.
\section{Angire}
\begin{itemize}
\item {Grp. gram.:m.}
\end{itemize}
Ruminante de Angola.
\section{Angite}
\begin{itemize}
\item {Grp. gram.:f.}
\end{itemize}
\begin{itemize}
\item {Utilização:Med.}
\end{itemize}
\begin{itemize}
\item {Proveniência:(Do gr. \textunderscore angeion\textunderscore )}
\end{itemize}
Inflammação dos vasos sanguíneos.
\section{Angite}
\begin{itemize}
\item {Grp. gram.:f.}
\end{itemize}
Minério dos Açores. Cf. Flaviense, \textunderscore Diccion. Geogr.\textunderscore 
\section{Anglarita}
\begin{itemize}
\item {Grp. gram.:f.}
\end{itemize}
Crystallização do phosphato de ferro.
\section{Anglesita}
\begin{itemize}
\item {Grp. gram.:f.}
\end{itemize}
\begin{itemize}
\item {Proveniência:(De \textunderscore Anglesey\textunderscore , n. p.)}
\end{itemize}
Sulfato de chumbo natural, que se encontra nas minas de Anglesey.
\section{Anglêuria}
\begin{itemize}
\item {Grp. gram.:f.}
\end{itemize}
Gênero de insectos dípteros.
\section{Anglicanismo}
\begin{itemize}
\item {Grp. gram.:m.}
\end{itemize}
\begin{itemize}
\item {Proveniência:(De \textunderscore anglicano\textunderscore )}
\end{itemize}
Religião official em Inglaterra.
\section{Anglicano}
\begin{itemize}
\item {Grp. gram.:m.}
\end{itemize}
\begin{itemize}
\item {Grp. gram.:Adj.}
\end{itemize}
\begin{itemize}
\item {Proveniência:(De \textunderscore ânglico\textunderscore )}
\end{itemize}
Sectário do anglicanismo.
Relativo ao anglicanismo.
\section{Anglicismo}
\begin{itemize}
\item {Grp. gram.:m.}
\end{itemize}
\begin{itemize}
\item {Proveniência:(De \textunderscore ânglico\textunderscore )}
\end{itemize}
Locução inglesa, introduzida noutra língua.
\section{Anglicizar}
\begin{itemize}
\item {Grp. gram.:v. t.}
\end{itemize}
\begin{itemize}
\item {Proveniência:(De \textunderscore ânglico\textunderscore )}
\end{itemize}
O mesmo que \textunderscore anglizar\textunderscore .
\section{Ânglico}
\begin{itemize}
\item {Grp. gram.:adj.}
\end{itemize}
\begin{itemize}
\item {Proveniência:(De \textunderscore anglo\textunderscore )}
\end{itemize}
O mesmo que \textunderscore inglês\textunderscore .
\section{Anglizar}
\begin{itemize}
\item {Grp. gram.:v. t.}
\end{itemize}
\begin{itemize}
\item {Proveniência:(De \textunderscore anglo\textunderscore )}
\end{itemize}
Dar feição inglesa a. Cf. Garrett, \textunderscore Viagens\textunderscore .
\section{Anglo}
\begin{itemize}
\item {Grp. gram.:m.  e  adj.}
\end{itemize}
O que é da Inglaterra.
(Cp. \textunderscore anglos\textunderscore )
\section{Anglo-americano}
\begin{itemize}
\item {Grp. gram.:adj.}
\end{itemize}
\begin{itemize}
\item {Grp. gram.:M.}
\end{itemize}
Relativo aos Estados-Unidos da América do Norte.
Habitante da América do Norte, mas de origem inglesa.
\section{Anglo-árabe}
\begin{itemize}
\item {Grp. gram.:adj.}
\end{itemize}
Relativo a Árabes e Ingleses.
E diz-se especialmente dos cavallos importados da Arábia para a Inglaterra.
\section{Anglo-brasileiro}
\begin{itemize}
\item {Grp. gram.:adj.}
\end{itemize}
Relativo á Inglaterra e ao Brasil.
\section{Anglo-bretão}
\begin{itemize}
\item {Grp. gram.:adj.}
\end{itemize}
De raça inglesa e bretan.
\section{Anglo-canadense}
\begin{itemize}
\item {Grp. gram.:adj.}
\end{itemize}
Diz-se dos Ingleses que habitam o Canadá.
\section{Anglo-chinês}
\begin{itemize}
\item {Grp. gram.:adj.}
\end{itemize}
Relativo a Ingleses e Chineses.
\section{Anglo-continental}
\begin{itemize}
\item {Grp. gram.:adj.}
\end{itemize}
Relativo á Inglaterra e ao continente.
\section{Anglo-dinamarquês}
\begin{itemize}
\item {Grp. gram.:adj.}
\end{itemize}
Relativo á Inglaterra e á Dinamarca.
\section{Anglo-espanhol}
\begin{itemize}
\item {Grp. gram.:adj.}
\end{itemize}
Relativo á Inglaterra e á Espanha.
\section{Anarmónico}
\begin{itemize}
\item {Grp. gram.:adj.}
\end{itemize}
O mesmo que \textunderscore inarmónico\textunderscore .
\section{Anelação}
\begin{itemize}
\item {Grp. gram.:f.}
\end{itemize}
\begin{itemize}
\item {Proveniência:(Do lat. \textunderscore anhelatio\textunderscore )}
\end{itemize}
Respiração diffícil, curta.
\section{Anelante}
\begin{itemize}
\item {Grp. gram.:adj.}
\end{itemize}
\begin{itemize}
\item {Proveniência:(Lat. \textunderscore anhelans\textunderscore )}
\end{itemize}
Que anela.
\section{Anelar}
\begin{itemize}
\item {Grp. gram.:v. i.}
\end{itemize}
\begin{itemize}
\item {Grp. gram.:V. t.}
\end{itemize}
\begin{itemize}
\item {Proveniência:(Lat. \textunderscore anhelare\textunderscore )}
\end{itemize}
Respirar com difficuldade.
Desejar ardentemente.
\section{Anélito}
\begin{itemize}
\item {Grp. gram.:m.}
\end{itemize}
\begin{itemize}
\item {Proveniência:(Lat. \textunderscore anhelitus\textunderscore )}
\end{itemize}
Hálito; respiração.
Aspiração; desejo intenso.
\section{Anelo}
\begin{itemize}
\item {Grp. gram.:m.}
\end{itemize}
\begin{itemize}
\item {Proveniência:(De \textunderscore anhelar\textunderscore )}
\end{itemize}
Desejo ardente; aspiração.
Ânsia.
Espécie de pudim.
\section{Anglofilia}
\begin{itemize}
\item {Grp. gram.:f.}
\end{itemize}
Qualidade do que é \textunderscore anglófilo\textunderscore .
\section{Anglófilo}
\begin{itemize}
\item {Grp. gram.:m.  e  adj.}
\end{itemize}
\begin{itemize}
\item {Proveniência:(De \textunderscore anglo\textunderscore  + gr. \textunderscore philos\textunderscore )}
\end{itemize}
Amigo dos Ingleses.
\section{Anglofobia}
\begin{itemize}
\item {Grp. gram.:f.}
\end{itemize}
\begin{itemize}
\item {Proveniência:(De \textunderscore anglóphobo\textunderscore )}
\end{itemize}
Ódio aos Ingleses.
\section{Anglófobo}
\begin{itemize}
\item {Grp. gram.:adj.}
\end{itemize}
\begin{itemize}
\item {Proveniência:(De \textunderscore anglo\textunderscore  + gr. \textunderscore phobein\textunderscore )}
\end{itemize}
Que tem ódio aos Ingleses.
\section{Anglo-francês}
\begin{itemize}
\item {Grp. gram.:adj.}
\end{itemize}
Relativo a Franceses e Ingleses.
\section{Anglo-indiano}
\begin{itemize}
\item {Grp. gram.:adj.}
\end{itemize}
Diz-se dos Ingleses que habitam a Índia.
Relativo á Inglaterra e á Índia.
\section{Anglo-luso}
\begin{itemize}
\item {Grp. gram.:adj.}
\end{itemize}
Relativo á Inglaterra e a Portugal: \textunderscore alliança anglo-lusa\textunderscore .
\section{Anglomania}
\begin{itemize}
\item {Grp. gram.:f.}
\end{itemize}
\begin{itemize}
\item {Proveniência:(De \textunderscore anglo\textunderscore  + \textunderscore mania\textunderscore )}
\end{itemize}
Imitação exaggerada do que é inglês.
Paixão pelo que é inglês.
\section{Anglomaníaco}
\begin{itemize}
\item {Grp. gram.:m.}
\end{itemize}
Aquelle que tem \textunderscore anglomania\textunderscore .
\section{Anglomanizar}
\begin{itemize}
\item {Grp. gram.:v. i.  e  pr.}
\end{itemize}
Têr anglomania.
\section{Anglómano}
\begin{itemize}
\item {Grp. gram.:m.  e  adj.}
\end{itemize}
O mesmo que \textunderscore anglomaníaco\textunderscore .
\section{Anglo-normando}
\begin{itemize}
\item {Grp. gram.:m.  e  adj.}
\end{itemize}
Diz-se dos Normandos, que se fundiram com os Anglo-Saxões.
\section{Anglophilia}
\begin{itemize}
\item {Grp. gram.:f.}
\end{itemize}
Qualidade do que é \textunderscore anglóphilo\textunderscore .
\section{Anglóphilo}
\begin{itemize}
\item {Grp. gram.:m.  e  adj.}
\end{itemize}
\begin{itemize}
\item {Proveniência:(De \textunderscore anglo\textunderscore  + gr. \textunderscore philos\textunderscore )}
\end{itemize}
Amigo dos Ingleses.
\section{Anglophobia}
\begin{itemize}
\item {Grp. gram.:f.}
\end{itemize}
\begin{itemize}
\item {Proveniência:(De \textunderscore anglóphobo\textunderscore )}
\end{itemize}
Ódio aos Ingleses.
\section{Anglóphobo}
\begin{itemize}
\item {Grp. gram.:adj.}
\end{itemize}
\begin{itemize}
\item {Proveniência:(De \textunderscore anglo\textunderscore  + gr. \textunderscore phobein\textunderscore )}
\end{itemize}
Que tem ódio aos Ingleses.
\section{Anglo-russo}
\begin{itemize}
\item {Grp. gram.:adj.}
\end{itemize}
Relativo a Ingleses e Russos.
\section{Anglos}
\begin{itemize}
\item {Grp. gram.:m. pl.}
\end{itemize}
Povo germânico, que constituiu um dos elementos da população inglesa.
(Do germ.)
\section{Anglo-saxão}
\begin{itemize}
\item {Grp. gram.:m.}
\end{itemize}
\begin{itemize}
\item {Grp. gram.:Adj.}
\end{itemize}
Língua dos Anglo-Saxões.
Relativo aos Anglo-Saxões.
\section{Anglo-Saxões}
\begin{itemize}
\item {Grp. gram.:m. pl.}
\end{itemize}
Povo, constituido pela fusão dos Anglos e dos Saxões.
\section{Anglo-saxónio}
\begin{itemize}
\item {Grp. gram.:adj.}
\end{itemize}
Relativo aos Anglo-Saxões.
\section{Anglo-Saxónios}
\begin{itemize}
\item {Grp. gram.:m. pl.}
\end{itemize}
O mesmo que \textunderscore Anglo-Saxões\textunderscore .
\section{Anglo-saxonismo}
\begin{itemize}
\item {Grp. gram.:m.}
\end{itemize}
Affeição aos Anglo-Saxões.
\section{Angófora}
\begin{itemize}
\item {Grp. gram.:f.}
\end{itemize}
\begin{itemize}
\item {Proveniência:(Do gr. \textunderscore angos\textunderscore  + \textunderscore phoros\textunderscore )}
\end{itemize}
Gênero de plantas myrtáceas.
\section{Angoja}
\begin{itemize}
\item {Grp. gram.:f.}
\end{itemize}
Antiga pêra Portuguesa, hoje desconhecida. Cf. D. N. do Lião, \textunderscore Descr. do R. de Port.\textunderscore , 62.
\section{Angolano}
\begin{itemize}
\item {Grp. gram.:m.  e  adj.}
\end{itemize}
O mesmo que \textunderscore angolense\textunderscore .
\section{Angolares}
\begin{itemize}
\item {Grp. gram.:m. pl.}
\end{itemize}
\begin{itemize}
\item {Proveniência:(De \textunderscore Angola\textunderscore , n. p.)}
\end{itemize}
Negros, que, originários de Angola, habitam parte do território da ilha de San-Thomé, na África occidental.
\section{Angolas}
\begin{itemize}
\item {Grp. gram.:m. pl.}
\end{itemize}
Indígenas africanos, que deram o seu nome á província de Angola.
\section{Angolense}
\begin{itemize}
\item {Grp. gram.:adj.}
\end{itemize}
\begin{itemize}
\item {Grp. gram.:M.}
\end{itemize}
Relativo a Angola.
Habitante de Angola.
O mesmo que \textunderscore quimbundo\textunderscore .
\section{Angombe}
\begin{itemize}
\item {Grp. gram.:m.}
\end{itemize}
Ave africana.
\section{Angóphora}
\begin{itemize}
\item {Grp. gram.:f.}
\end{itemize}
\begin{itemize}
\item {Proveniência:(Do gr. \textunderscore angos\textunderscore  + \textunderscore phoros\textunderscore )}
\end{itemize}
Gênero de plantas myrtáceas.
\section{Ângora}
\begin{itemize}
\item {Grp. gram.:m.  e  adj.}
\end{itemize}
\begin{itemize}
\item {Proveniência:(De \textunderscore Ângora\textunderscore , n. p.)}
\end{itemize}
Diz-se dos gatos, coêlhos ou cabras, procedentes de Ângora, ou semelhantes a êstes pela finura e comprimento do pêlo.
\section{Angoreta}
\begin{itemize}
\item {fónica:gu-rê}
\end{itemize}
\begin{itemize}
\item {Grp. gram.:f.}
\end{itemize}
\begin{itemize}
\item {Utilização:Prov.}
\end{itemize}
\begin{itemize}
\item {Utilização:dur.}
\end{itemize}
Vasilha redonda ou achatada, feita de aduelas e arcos, como as pipas.
(Cp. \textunderscore ancoreta\textunderscore )
\section{Angra}
\begin{itemize}
\item {Grp. gram.:f.}
\end{itemize}
Enseada.
Pequena baía.
(B. lat. \textunderscore ancra\textunderscore )
\section{Angraçá}
\begin{itemize}
\item {Grp. gram.:m.}
\end{itemize}
Espécie de vestuárío gentílico, na Índia Portuguesa. Cf. Th. Ribeiro, \textunderscore Jornadas\textunderscore , II, 101.
\section{Angrense}
\begin{itemize}
\item {Grp. gram.:adj.}
\end{itemize}
\begin{itemize}
\item {Grp. gram.:M.}
\end{itemize}
Relativo á cidade de Angra.
Habitante de Angra.
\section{Angu}
\begin{itemize}
\item {Grp. gram.:m.}
\end{itemize}
\begin{itemize}
\item {Utilização:Bras}
\end{itemize}
\begin{itemize}
\item {Utilização:Bras. de San-Paulo}
\end{itemize}
Farinha de mandioca cozida.
Farinha de milho ou fubá cozido.
\section{Anguada}
\begin{itemize}
\item {Grp. gram.:f.}
\end{itemize}
\begin{itemize}
\item {Utilização:Ant.}
\end{itemize}
Espécie de alamar. Cf. Fern. Lopes, \textunderscore Chrón. de D. Fern.\textunderscore , XLII.
\section{Angueira}
\begin{itemize}
\item {Grp. gram.:f.}
\end{itemize}
\begin{itemize}
\item {Utilização:Ant.}
\end{itemize}
Aluguel de bêstas de carga.
\section{Anguia}
\begin{itemize}
\item {Grp. gram.:f.}
\end{itemize}
\begin{itemize}
\item {Utilização:Prov.}
\end{itemize}
\begin{itemize}
\item {Utilização:minh.}
\end{itemize}
O mesmo que \textunderscore enguia\textunderscore .
\section{Anguicida}
\begin{itemize}
\item {fónica:gu-i}
\end{itemize}
\begin{itemize}
\item {Grp. gram.:adj.}
\end{itemize}
\begin{itemize}
\item {Proveniência:(Do lat. \textunderscore anguis\textunderscore  + \textunderscore caedere\textunderscore )}
\end{itemize}
Que tem a propriedade de matar cobras.
\section{Anguicomado}
\begin{itemize}
\item {fónica:gu-i}
\end{itemize}
\begin{itemize}
\item {Grp. gram.:adj.}
\end{itemize}
O mesmo que \textunderscore anguícomo\textunderscore .
\section{Anguícomo}
\begin{itemize}
\item {fónica:gu-í}
\end{itemize}
\begin{itemize}
\item {Grp. gram.:adj.}
\end{itemize}
\begin{itemize}
\item {Utilização:Poét.}
\end{itemize}
\begin{itemize}
\item {Proveniência:(Do lat. \textunderscore anguis\textunderscore  + \textunderscore coma\textunderscore )}
\end{itemize}
Coroado de serpentes.
\section{Anguídeo}
\begin{itemize}
\item {fónica:gu-i}
\end{itemize}
\begin{itemize}
\item {Grp. gram.:adj.}
\end{itemize}
\begin{itemize}
\item {Grp. gram.:M. pl.}
\end{itemize}
\begin{itemize}
\item {Proveniência:(Do lat. \textunderscore anguis\textunderscore  + gr. \textunderscore eidos\textunderscore )}
\end{itemize}
Semelhante ao licranço.
Sáurios, que têm por typo o licranço.
\section{Anguífero}
\begin{itemize}
\item {fónica:gu-í}
\end{itemize}
\begin{itemize}
\item {Grp. gram.:adj.}
\end{itemize}
\begin{itemize}
\item {Proveniência:(Do lat. \textunderscore anguis\textunderscore  + \textunderscore ferre\textunderscore )}
\end{itemize}
Que tem ou cria cobras.
\section{Anguiforme}
\begin{itemize}
\item {fónica:gu-i}
\end{itemize}
\begin{itemize}
\item {Grp. gram.:adj.}
\end{itemize}
\begin{itemize}
\item {Proveniência:(Do lat. \textunderscore anguis\textunderscore  + \textunderscore forma\textunderscore )}
\end{itemize}
Que tem fórma de cobra.
\section{Anguígemo}
\begin{itemize}
\item {fónica:gu-í}
\end{itemize}
\begin{itemize}
\item {Grp. gram.:m.}
\end{itemize}
Espécie de serpente.
\section{Anguígeno}
\begin{itemize}
\item {fónica:gu-í}
\end{itemize}
\begin{itemize}
\item {Grp. gram.:adj.}
\end{itemize}
Que gera cobras. Cf. Castilho, \textunderscore Metam.\textunderscore , 149.
\section{Anguiliforme}
\begin{itemize}
\item {fónica:gu-i}
\end{itemize}
\begin{itemize}
\item {Grp. gram.:adj.}
\end{itemize}
\begin{itemize}
\item {Grp. gram.:M. pl.}
\end{itemize}
\begin{itemize}
\item {Proveniência:(Do lat. \textunderscore anguilla\textunderscore  + \textunderscore forma\textunderscore )}
\end{itemize}
Que tem fórma de enguia.
Família de peixes malacopterýgios, a que serve de typo a enguia.
\section{Anguilliforme}
\begin{itemize}
\item {fónica:gu-i}
\end{itemize}
\begin{itemize}
\item {Grp. gram.:adj.}
\end{itemize}
\begin{itemize}
\item {Grp. gram.:M. pl.}
\end{itemize}
\begin{itemize}
\item {Proveniência:(Do lat. \textunderscore anguilla\textunderscore  + \textunderscore forma\textunderscore )}
\end{itemize}
Que tem fórma de enguia.
Família de peixes malacopterýgios, a que serve de typo a enguia.
\section{Anguíllula}
\begin{itemize}
\item {fónica:gu-í}
\end{itemize}
\begin{itemize}
\item {Grp. gram.:f.}
\end{itemize}
Insecto, que origina o apodrecimento das raízes da videira.
Doença das videiras, originada por aquelle insecto.
(Dem. do lat. \textunderscore anguilla\textunderscore )
\section{Anguílula}
\begin{itemize}
\item {fónica:gu-í}
\end{itemize}
\begin{itemize}
\item {Grp. gram.:f.}
\end{itemize}
Insecto, que origina o apodrecimento das raízes da videira.
Doença das videiras, originada por aquelle insecto.
(Dem. do lat. \textunderscore anguilla\textunderscore )
\section{Anguina}
\begin{itemize}
\item {fónica:gu-i}
\end{itemize}
\begin{itemize}
\item {Grp. gram.:f.}
\end{itemize}
\begin{itemize}
\item {Proveniência:(De \textunderscore anguino\textunderscore )}
\end{itemize}
Gênero de cucurbitáceas.
Veia das virilhas, nas cavalgaduras.
\section{Anguino}
\begin{itemize}
\item {fónica:gu-i}
\end{itemize}
\begin{itemize}
\item {Grp. gram.:adj.}
\end{itemize}
\begin{itemize}
\item {Grp. gram.:M. pl.}
\end{itemize}
\begin{itemize}
\item {Proveniência:(Lat. \textunderscore anguinus\textunderscore )}
\end{itemize}
Semelhante á cobra.
Família de reptis, cujo typo é a cobra.
\section{Anguinha}
\begin{itemize}
\item {fónica:gu-i}
\end{itemize}
\begin{itemize}
\item {Grp. gram.:f.}
\end{itemize}
\begin{itemize}
\item {Proveniência:(Lat. \textunderscore anguina\textunderscore )}
\end{itemize}
Reptil escamoso, da ordem dos ophídios.
\section{Anguípede}
\begin{itemize}
\item {fónica:gu-i}
\end{itemize}
\begin{itemize}
\item {Grp. gram.:adj.}
\end{itemize}
\begin{itemize}
\item {Utilização:Ant.}
\end{itemize}
\begin{itemize}
\item {Proveniência:(Do lat. \textunderscore anguis\textunderscore  + \textunderscore pes\textunderscore )}
\end{itemize}
Que tem pés de dragão.
\section{Anguirodente}
\begin{itemize}
\item {fónica:ro}
\end{itemize}
\begin{itemize}
\item {Grp. gram.:adj.}
\end{itemize}
Que rói como a cobra?«\textunderscore ...o remorso angui-rodente...\textunderscore »Filinto, IX, 270.
\section{Anguirrodente}
\begin{itemize}
\item {fónica:gu-i}
\end{itemize}
\begin{itemize}
\item {Grp. gram.:adj.}
\end{itemize}
Que rói como a cobra?«\textunderscore ...o remorso angui-rodente...\textunderscore »Filinto, IX, 270.
\section{Ânguis}
\begin{itemize}
\item {fónica:gu-is}
\end{itemize}
\begin{itemize}
\item {Grp. gram.:m.}
\end{itemize}
\begin{itemize}
\item {Proveniência:(Lat. \textunderscore anguis\textunderscore )}
\end{itemize}
Família zoológica, que estabelece, segundo Cuvier, a transição entre os sáurios e os ophídios.
\section{Anguite}
\begin{itemize}
\item {fónica:gu-i}
\end{itemize}
\begin{itemize}
\item {Grp. gram.:f.}
\end{itemize}
\begin{itemize}
\item {Utilização:Bras. do N}
\end{itemize}
Espécie de angu.
\section{Anguivíferas}
\begin{itemize}
\item {fónica:gu-i}
\end{itemize}
\begin{itemize}
\item {Grp. gram.:f. pl.}
\end{itemize}
\begin{itemize}
\item {Proveniência:(Do lat. \textunderscore anguis\textunderscore  + \textunderscore vipera\textunderscore )}
\end{itemize}
Família de víboras, cujo corpo se parece com o das enguias.
\section{Angulado}
\begin{itemize}
\item {Grp. gram.:adj.}
\end{itemize}
O mesmo que \textunderscore anguloso\textunderscore .
\section{Angular}
\begin{itemize}
\item {Grp. gram.:adj.}
\end{itemize}
\begin{itemize}
\item {Proveniência:(Lat. \textunderscore angularis\textunderscore )}
\end{itemize}
Que tem um ou mais ângulos.
Pertencente a ângulos.
E diz-se da pedra fundamental de um edifício.
\section{Angularidade}
\begin{itemize}
\item {Grp. gram.:f.}
\end{itemize}
\begin{itemize}
\item {Proveniência:(De \textunderscore angular\textunderscore )}
\end{itemize}
Qualidade daquillo que tem ângulo ou ângulos.
\section{Angulário}
\begin{itemize}
\item {Grp. gram.:m.}
\end{itemize}
Instrumento, para medir ângulos.
\section{Angularmente}
\begin{itemize}
\item {Grp. gram.:adv.}
\end{itemize}
\begin{itemize}
\item {Proveniência:(De \textunderscore angular\textunderscore )}
\end{itemize}
Em fórma de ângulo.
\section{Angulema}
\begin{itemize}
\item {Grp. gram.:f.}
\end{itemize}
Tecido de estopa ou de cânhamo, que se fabricava em Angouleme.
\section{Angulete}
\begin{itemize}
\item {fónica:lê}
\end{itemize}
\begin{itemize}
\item {Grp. gram.:m.}
\end{itemize}
\begin{itemize}
\item {Proveniência:(De \textunderscore ângulo\textunderscore )}
\end{itemize}
Cavidade, talhada em ângulo recto.
\section{Angulicollo}
\begin{itemize}
\item {Grp. gram.:adj.}
\end{itemize}
Que tem o pescoço anguloso.
\section{Angulicolo}
\begin{itemize}
\item {Grp. gram.:adj.}
\end{itemize}
Que tem o pescoço anguloso.
\section{Angulifero}
\begin{itemize}
\item {Grp. gram.:adj.}
\end{itemize}
Que tem ou fórma ângulos.
\section{Angulinervado}
\begin{itemize}
\item {Grp. gram.:adj.}
\end{itemize}
\begin{itemize}
\item {Utilização:Bot.}
\end{itemize}
Que tem nervuras angulosas.
\section{Angulirostro}
\begin{itemize}
\item {fónica:rós}
\end{itemize}
\begin{itemize}
\item {Grp. gram.:adj.}
\end{itemize}
\begin{itemize}
\item {Proveniência:(Do lat. \textunderscore angulus\textunderscore  + \textunderscore rostrum\textunderscore )}
\end{itemize}
Diz-se das aves, que têm o bico anguloso.
\section{Angulirrostro}
\begin{itemize}
\item {Grp. gram.:adj.}
\end{itemize}
\begin{itemize}
\item {Proveniência:(Do lat. \textunderscore angulus\textunderscore  + \textunderscore rostrum\textunderscore )}
\end{itemize}
Diz-se das aves, que têm o bico anguloso.
\section{Ângulo}
\begin{itemize}
\item {Grp. gram.:m.}
\end{itemize}
\begin{itemize}
\item {Utilização:Heráld.}
\end{itemize}
\begin{itemize}
\item {Proveniência:(Lat. \textunderscore angulus\textunderscore )}
\end{itemize}
Espaço, comprehendido entre duas linhas rectas, que se encontram: \textunderscore ângulo recto\textunderscore , \textunderscore ângulo agudo\textunderscore , \textunderscore ângulo obtuso\textunderscore .
Canto.
Aresta, linha em que se encontram duas faces incidentes.
Arruela, de fórma angular.
\section{Angúloa}
\begin{itemize}
\item {Grp. gram.:f.}
\end{itemize}
Gênero de orchídeas americanas.
\section{Angulometria}
\begin{itemize}
\item {Grp. gram.:f.}
\end{itemize}
Applicação do \textunderscore angulometro\textunderscore .
\section{Angulométrico}
\begin{itemize}
\item {Grp. gram.:adj.}
\end{itemize}
Relativo á \textunderscore angulómetria\textunderscore .
\section{Angulómetro}
\begin{itemize}
\item {Grp. gram.:m.}
\end{itemize}
\begin{itemize}
\item {Proveniência:(Do lat. \textunderscore angulus\textunderscore  + gr. \textunderscore metron\textunderscore )}
\end{itemize}
Instrumento, para medir ângulos.
\section{Anguloso}
\begin{itemize}
\item {Grp. gram.:adj.}
\end{itemize}
Que tem ângulos.
Que tem esquinas ou saliências ponteagudas: \textunderscore feições angulosas\textunderscore .
\section{Anguri}
\begin{itemize}
\item {Grp. gram.:m.}
\end{itemize}
Planta de Java, cujos grãos são soporíficos.
\section{Angúrria}
\begin{itemize}
\item {Grp. gram.:f.}
\end{itemize}
(V.estranguria):«\textunderscore o mal era a retenção de ourinas, que a Física chama angurria\textunderscore ». Sousa, \textunderscore V. do Arceb.\textunderscore , l. V. c. 1.
\section{Angurus}
\begin{itemize}
\item {Grp. gram.:m. pl.}
\end{itemize}
Uma das tríbos de Moçambique.
\section{Angústia}
\begin{itemize}
\item {Grp. gram.:f.}
\end{itemize}
\begin{itemize}
\item {Proveniência:(Lat. \textunderscore angustia\textunderscore )}
\end{itemize}
Estreiteza.
Apêrto de coração.
Afflicção, agonia.
\section{Angustiadamente}
\begin{itemize}
\item {Grp. gram.:adv.}
\end{itemize}
Com angústia.
\section{Angustiador}
\begin{itemize}
\item {Grp. gram.:adj.}
\end{itemize}
Que angustia.
\section{Angustiante}
\begin{itemize}
\item {Grp. gram.:adj.}
\end{itemize}
O mesmo que \textunderscore angustiador\textunderscore . Cf. J. Dinis, \textunderscore Serões\textunderscore , 256.
\section{Angustiar}
\begin{itemize}
\item {Grp. gram.:v. t.}
\end{itemize}
Causar angústia a.
\section{Angusticlave}
\begin{itemize}
\item {Grp. gram.:f.}
\end{itemize}
\begin{itemize}
\item {Proveniência:(Do lat. \textunderscore angustus\textunderscore  + \textunderscore clavus\textunderscore )}
\end{itemize}
Banda estreita de púrpura, na túnica dos magistrados populares, em Roma.
\section{Angusticlávio}
\begin{itemize}
\item {Grp. gram.:m.}
\end{itemize}
Aquelle que podia usar a \textunderscore angusticlave\textunderscore .
\section{Angustidentado}
\begin{itemize}
\item {Grp. gram.:adj.}
\end{itemize}
\begin{itemize}
\item {Proveniência:(De \textunderscore angusto\textunderscore  + \textunderscore dente\textunderscore )}
\end{itemize}
Que tem os dentes apertados.
\section{Angustifoliado}
\begin{itemize}
\item {Grp. gram.:adj.}
\end{itemize}
\begin{itemize}
\item {Proveniência:(Do lat. \textunderscore angustus\textunderscore  + \textunderscore folium\textunderscore )}
\end{itemize}
Que tem fôlhas estreitas.
\section{Angustímano}
\begin{itemize}
\item {Grp. gram.:adj.}
\end{itemize}
\begin{itemize}
\item {Proveniência:(Do lat. \textunderscore angustus\textunderscore  + \textunderscore manus\textunderscore )}
\end{itemize}
Que tem mãos estreitas.
\section{Angustioso}
\begin{itemize}
\item {Grp. gram.:adj.}
\end{itemize}
Que tem angústia.
Que causa angústia.
Que procede de angústia.
\section{Angustipene}
\begin{itemize}
\item {Grp. gram.:adj.}
\end{itemize}
\begin{itemize}
\item {Proveniência:(Do lat. \textunderscore angustus\textunderscore  + \textunderscore penna\textunderscore )}
\end{itemize}
Que tem asas estreitas.
\section{Angustipenne}
\begin{itemize}
\item {Grp. gram.:adj.}
\end{itemize}
\begin{itemize}
\item {Proveniência:(Do lat. \textunderscore angustus\textunderscore  + \textunderscore penna\textunderscore )}
\end{itemize}
Que tem asas estreitas.
\section{Angustireme}
\begin{itemize}
\item {fónica:rê}
\end{itemize}
\begin{itemize}
\item {Grp. gram.:adj.}
\end{itemize}
\begin{itemize}
\item {Proveniência:(De \textunderscore angusto\textunderscore  + \textunderscore remo\textunderscore )}
\end{itemize}
Diz-se dos insectos que têm as patas em fórma de remos estreitos.
\section{Angustirostro}
\begin{itemize}
\item {fónica:rós}
\end{itemize}
\begin{itemize}
\item {Grp. gram.:adj.}
\end{itemize}
\begin{itemize}
\item {Proveniência:(Do lat. \textunderscore angustus\textunderscore  + \textunderscore rostrum\textunderscore )}
\end{itemize}
Que tem bico agudo.
\section{Angustirreme}
\begin{itemize}
\item {Grp. gram.:adj.}
\end{itemize}
\begin{itemize}
\item {Proveniência:(De \textunderscore angusto\textunderscore  + \textunderscore remo\textunderscore )}
\end{itemize}
Diz-se dos insectos que têm as patas em fórma de remos estreitos.
\section{Angustirrostro}
\begin{itemize}
\item {Grp. gram.:adj.}
\end{itemize}
\begin{itemize}
\item {Proveniência:(Do lat. \textunderscore angustus\textunderscore  + \textunderscore rostrum\textunderscore )}
\end{itemize}
Que tem bico agudo.
\section{Angustita}
\begin{itemize}
\item {Grp. gram.:f.}
\end{itemize}
Variedade de apatita.
\section{Angusto}
\begin{itemize}
\item {Grp. gram.:adj.}
\end{itemize}
\begin{itemize}
\item {Utilização:Ant.}
\end{itemize}
\begin{itemize}
\item {Proveniência:(Lat. \textunderscore angustus\textunderscore )}
\end{itemize}
Apertado, estreito.
\section{Angustura}
\begin{itemize}
\item {Grp. gram.:f.}
\end{itemize}
\begin{itemize}
\item {Utilização:Ant.}
\end{itemize}
\begin{itemize}
\item {Utilização:Bras. da Baía}
\end{itemize}
Casca medicinal, febrifuga e estomacal.
Angústia.
Qualidade do que é angusto.
O mesmo que \textunderscore laranjeira-do-mato\textunderscore .
\section{Anguzada}
\begin{itemize}
\item {Grp. gram.:f.}
\end{itemize}
\begin{itemize}
\item {Utilização:Bras. do N}
\end{itemize}
Grande confusão moral.
Mistura de coisas.
\section{Anguzô}
\begin{itemize}
\item {Grp. gram.:m.}
\end{itemize}
\begin{itemize}
\item {Utilização:Bras. do N}
\end{itemize}
Espécie de esparregado.
\section{...anha}
\begin{itemize}
\item {Grp. gram.:suf. f.}
\end{itemize}
(design. de grandeza, extensão, etc.)
\section{Anhafia}
\begin{itemize}
\item {fónica:na}
\end{itemize}
\begin{itemize}
\item {Grp. gram.:f.}
\end{itemize}
\begin{itemize}
\item {Utilização:Med.}
\end{itemize}
Deminuição ou privação da sensibilidade do tacto.
\section{Anhaga}
\begin{itemize}
\item {Grp. gram.:m.}
\end{itemize}
\begin{itemize}
\item {Utilização:Bras}
\end{itemize}
Nome genérico do Diabo.
\section{Anhanguera}
\begin{itemize}
\item {fónica:gu-ê}
\end{itemize}
\begin{itemize}
\item {Grp. gram.:f.}
\end{itemize}
\begin{itemize}
\item {Utilização:Bras}
\end{itemize}
\begin{itemize}
\item {Grp. gram.:Adj.}
\end{itemize}
\begin{itemize}
\item {Utilização:Fig.}
\end{itemize}
Diabo, que tomou qualquer fórma.
Sêr imaginário.
Destemido.
Resoluto.
\section{Anhanha}
\begin{itemize}
\item {Grp. gram.:m.}
\end{itemize}
O mesmo que \textunderscore anhanho\textunderscore .
\section{Anhanho}
\begin{itemize}
\item {Grp. gram.:m.  e  adj.}
\end{itemize}
(V.inhanha\textunderscore  e \textunderscore inhanho)
\section{Anhaphia}
\begin{itemize}
\item {Grp. gram.:f.}
\end{itemize}
\begin{itemize}
\item {Utilização:Med.}
\end{itemize}
Deminuição ou privação da sensibilidade do tacto.
\section{...anhar}
\begin{itemize}
\item {Grp. gram.:suf.}
\end{itemize}
(de verbos frequentativos)
\section{Anhara}
\begin{itemize}
\item {Grp. gram.:f.}
\end{itemize}
\begin{itemize}
\item {Utilização:T. de Angola}
\end{itemize}
Clareira?:«\textunderscore mato franzino, que por vezes desapparece, para substituir-se por anharas\textunderscore ». Capello e Ivens, \textunderscore De Angola\textunderscore , I, 144.
\section{Anharmónico}
\begin{itemize}
\item {fónica:nar}
\end{itemize}
\begin{itemize}
\item {Grp. gram.:adj.}
\end{itemize}
O mesmo que \textunderscore inharmónico\textunderscore .
\section{Anhelação}
\begin{itemize}
\item {fónica:ne}
\end{itemize}
\begin{itemize}
\item {Grp. gram.:f.}
\end{itemize}
\begin{itemize}
\item {Proveniência:(Do lat. \textunderscore anhelatio\textunderscore )}
\end{itemize}
Respiração diffícil, curta.
\section{Anhelante}
\begin{itemize}
\item {fónica:ne}
\end{itemize}
\begin{itemize}
\item {Grp. gram.:adj.}
\end{itemize}
\begin{itemize}
\item {Proveniência:(Lat. \textunderscore anhelans\textunderscore )}
\end{itemize}
Que anhela.
\section{Anhelar}
\begin{itemize}
\item {fónica:ne}
\end{itemize}
\begin{itemize}
\item {Grp. gram.:v. i.}
\end{itemize}
\begin{itemize}
\item {Grp. gram.:V. t.}
\end{itemize}
\begin{itemize}
\item {Proveniência:(Lat. \textunderscore anhelare\textunderscore )}
\end{itemize}
Respirar com difficuldade.
Desejar ardentemente.
\section{Anhélito}
\begin{itemize}
\item {fónica:né}
\end{itemize}
\begin{itemize}
\item {Grp. gram.:m.}
\end{itemize}
\begin{itemize}
\item {Proveniência:(Lat. \textunderscore anhelitus\textunderscore )}
\end{itemize}
Hálito; respiração.
Aspiração; desejo intenso.
\section{Anhelo}
\begin{itemize}
\item {fónica:né}
\end{itemize}
\begin{itemize}
\item {Grp. gram.:m.}
\end{itemize}
\begin{itemize}
\item {Proveniência:(De \textunderscore anhelar\textunderscore )}
\end{itemize}
Desejo ardente; aspiração.
Ânsia.
Espécie de pudim.
\section{Anhima}
\begin{itemize}
\item {Grp. gram.:f.}
\end{itemize}
Espécie de gralha do Brasil.
\section{Anhinga}
\begin{itemize}
\item {Grp. gram.:m.}
\end{itemize}
Ave palmípede.
\section{Anhingaíba}
\begin{itemize}
\item {Grp. gram.:m.}
\end{itemize}
Arbusto brasileiro, de frutos comestíveis.
\section{Anhisto}
\begin{itemize}
\item {fónica:nis}
\end{itemize}
\begin{itemize}
\item {Grp. gram.:adj.}
\end{itemize}
\begin{itemize}
\item {Utilização:Anat.}
\end{itemize}
\begin{itemize}
\item {Utilização:Bot.}
\end{itemize}
\begin{itemize}
\item {Proveniência:(Do gr. \textunderscore an\textunderscore  priv. + \textunderscore histos\textunderscore , tecido)}
\end{itemize}
Que não tem textura determinada.
Diz-se dos órgãos vegetaes, em que nem com o microscópio se descobrem vestígios de tecido cellular, como succede nos tubos exteriores das confervas.
\section{Anhistorico}
\begin{itemize}
\item {fónica:nis}
\end{itemize}
\begin{itemize}
\item {Grp. gram.:adj.}
\end{itemize}
\begin{itemize}
\item {Proveniência:(De \textunderscore an\textunderscore  priv. + \textunderscore historico\textunderscore )}
\end{itemize}
Contrário á história.
\section{Anho}
\begin{itemize}
\item {Grp. gram.:m.}
\end{itemize}
\begin{itemize}
\item {Proveniência:(Lat. \textunderscore agnus\textunderscore )}
\end{itemize}
Cordeiro.
\section{Anhoto}
\begin{itemize}
\item {fónica:nhô}
\end{itemize}
\begin{itemize}
\item {Grp. gram.:adj.}
\end{itemize}
\begin{itemize}
\item {Utilização:Ant.}
\end{itemize}
Dizia-se do navio, que sustava o seu curso, por qualquer incidente imprevisto.
Vagaroso, ronceiro.
\section{Anhuma}
\begin{itemize}
\item {Grp. gram.:f.}
\end{itemize}
Nome de duas aves ribeirinhas do Brasil.
\section{Anhuma-poca}
\begin{itemize}
\item {Grp. gram.:f.}
\end{itemize}
Espécie de anhuma.
\section{Anhupoca}
\begin{itemize}
\item {Grp. gram.:f.}
\end{itemize}
\begin{itemize}
\item {Utilização:Bras}
\end{itemize}
Pássaro com esporões nas asas, e cujo canto se faz ouvir, da meia-noite em deante.
O mesmo que \textunderscore anhuma-poca\textunderscore ?
\section{Anhýdrico}
\begin{itemize}
\item {fónica:ni}
\end{itemize}
\begin{itemize}
\item {Grp. gram.:adj.}
\end{itemize}
O mesmo que \textunderscore anhydro\textunderscore .
\section{Anhydrido}
\begin{itemize}
\item {fónica:ni}
\end{itemize}
\begin{itemize}
\item {Grp. gram.:m.}
\end{itemize}
\begin{itemize}
\item {Proveniência:(De \textunderscore anhydro\textunderscore )}
\end{itemize}
Termo genérico, que designa os ácidos anhydros, isto é, os que se não combinam com a água.
\section{Anhydrite}
\begin{itemize}
\item {fónica:ni}
\end{itemize}
\begin{itemize}
\item {Grp. gram.:m.}
\end{itemize}
\begin{itemize}
\item {Proveniência:(De \textunderscore anhydro\textunderscore )}
\end{itemize}
Mineral crystallino, formado de sulfato de cal anhydro.
\section{Anhydro}
\begin{itemize}
\item {fónica:ni}
\end{itemize}
\begin{itemize}
\item {Grp. gram.:adj.}
\end{itemize}
\begin{itemize}
\item {Proveniência:(Do gr. \textunderscore an\textunderscore  + \textunderscore hudor\textunderscore )}
\end{itemize}
Que não contém água.
\section{Anhydromelia}
\begin{itemize}
\item {fónica:ni}
\end{itemize}
\begin{itemize}
\item {Grp. gram.:f.}
\end{itemize}
\begin{itemize}
\item {Utilização:Ant.}
\end{itemize}
Falta de líquido na cavidade rachidiana.
\section{Anhydrose}
\begin{itemize}
\item {fónica:ni}
\end{itemize}
\begin{itemize}
\item {Grp. gram.:f.}
\end{itemize}
Deminuição ou ausência da secreção sudoral.
\section{Ani}
\begin{itemize}
\item {Grp. gram.:m.}
\end{itemize}
Ave trepadora americana.
\section{Ania}
\begin{itemize}
\item {Grp. gram.:f.}
\end{itemize}
\begin{itemize}
\item {Proveniência:(Gr. \textunderscore ania\textunderscore )}
\end{itemize}
Orchídea indiana.
\section{Aniagem}
\begin{itemize}
\item {Grp. gram.:f.}
\end{itemize}
Pano grosseiro de linho, para capa de fardos.
\section{Anianas}
\begin{itemize}
\item {Grp. gram.:f. pl.}
\end{itemize}
Indígenas do norte do Brasil.
\section{Aniba}
\begin{itemize}
\item {Grp. gram.:f.}
\end{itemize}
Árvore da Guiana, cujas flôres têm periantho de seis divisões.
\section{Anibás}
\begin{itemize}
\item {Grp. gram.:m. pl.}
\end{itemize}
Indígenas brasileiros, que habitam nos sertões do Pará.
\section{Anichar}
\begin{itemize}
\item {Grp. gram.:v. t.}
\end{itemize}
\begin{itemize}
\item {Utilização:Fam.}
\end{itemize}
Pôr em nicho.
Collocar em posição rendosa: \textunderscore o Governador anichou mais um afilhado\textunderscore .
\section{Anichilar}
\begin{itemize}
\item {fónica:qui}
\end{itemize}
\begin{itemize}
\item {Grp. gram.:v. t.}
\end{itemize}
\begin{itemize}
\item {Utilização:Ant.}
\end{itemize}
O mesmo que \textunderscore aniquilar\textunderscore . Cf. \textunderscore Eufrosina\textunderscore , 73; \textunderscore Aulegrafia\textunderscore , 59.
\section{Anicilho}
\begin{itemize}
\item {Grp. gram.:m.}
\end{itemize}
Pereira americana, cujo fruto lembra o perfume e o sabor do anis.
(Cast. \textunderscore anicillo\textunderscore )
\section{Anicorés}
\begin{itemize}
\item {Grp. gram.:m. pl.}
\end{itemize}
Indígenas do norte do Brasil.
\section{Anídio}
\begin{itemize}
\item {Grp. gram.:m.}
\end{itemize}
\begin{itemize}
\item {Proveniência:(Do gr. \textunderscore an\textunderscore  + \textunderscore eidos\textunderscore )}
\end{itemize}
Monstruosidade orgânica, que tem o aspecto de massa informe; mola.
\section{Anido}
\begin{itemize}
\item {Grp. gram.:m.}
\end{itemize}
\begin{itemize}
\item {Proveniência:(Do gr. \textunderscore an\textunderscore  + \textunderscore eidos\textunderscore )}
\end{itemize}
Monstruosidade orgânica, que tem o aspecto de massa informe; mola.
\section{Anídrico}
\begin{itemize}
\item {Grp. gram.:adj.}
\end{itemize}
O mesmo que \textunderscore anidro\textunderscore .
\section{Anidrido}
\begin{itemize}
\item {Grp. gram.:m.}
\end{itemize}
\begin{itemize}
\item {Proveniência:(De \textunderscore anhydro\textunderscore )}
\end{itemize}
Termo genérico, que designa os ácidos anidros, isto é, os que se não combinam com a água.
\section{Anidrite}
\begin{itemize}
\item {Grp. gram.:m.}
\end{itemize}
\begin{itemize}
\item {Proveniência:(De \textunderscore anhydro\textunderscore )}
\end{itemize}
Mineral crystallino, formado de sulfato de cal anidro.
\section{Anidro}
\begin{itemize}
\item {Grp. gram.:adj.}
\end{itemize}
\begin{itemize}
\item {Proveniência:(Do gr. \textunderscore an\textunderscore  + \textunderscore hudor\textunderscore )}
\end{itemize}
Que não contém água.
\section{Anidromelia}
\begin{itemize}
\item {Grp. gram.:f.}
\end{itemize}
\begin{itemize}
\item {Utilização:Ant.}
\end{itemize}
Falta de líquido na cavidade rachidiana.
\section{Anidrose}
\begin{itemize}
\item {Grp. gram.:f.}
\end{itemize}
\begin{itemize}
\item {Grp. gram.:f.}
\end{itemize}
\begin{itemize}
\item {Utilização:Med.}
\end{itemize}
\begin{itemize}
\item {Proveniência:(Do gr. \textunderscore an\textunderscore  priv. + \textunderscore hidrosis\textunderscore )}
\end{itemize}
Deminuição ou ausência da secreção sudoral.
Falta ou deminuição de suor.
\section{Anielagem}
\begin{itemize}
\item {Grp. gram.:f.}
\end{itemize}
Acto de \textunderscore anielar\textunderscore .
\section{Anielar}
\begin{itemize}
\item {Grp. gram.:v. t.}
\end{itemize}
Esmaltar de nielo.
\section{Aniellagem}
\begin{itemize}
\item {Grp. gram.:f.}
\end{itemize}
Acto de \textunderscore aniellar\textunderscore .
\section{Aniellar}
\begin{itemize}
\item {Grp. gram.:v. t.}
\end{itemize}
Esmaltar de niello.
\section{Aniera}
\begin{itemize}
\item {Grp. gram.:f.}
\end{itemize}
Gênero de coleópteros.
\section{Anigozantho}
\begin{itemize}
\item {Grp. gram.:m.}
\end{itemize}
\begin{itemize}
\item {Proveniência:(Do gr. \textunderscore anoigo\textunderscore  + \textunderscore anthos\textunderscore )}
\end{itemize}
Gênero de plantas irídeas.
\section{Anigozanto}
\begin{itemize}
\item {Grp. gram.:m.}
\end{itemize}
\begin{itemize}
\item {Proveniência:(Do gr. \textunderscore anoigo\textunderscore  + \textunderscore anthos\textunderscore )}
\end{itemize}
Gênero de plantas irídeas.
\section{Anil}
\begin{itemize}
\item {Grp. gram.:m.}
\end{itemize}
\begin{itemize}
\item {Proveniência:(Do ár. \textunderscore annir\textunderscore )}
\end{itemize}
Substância que tinge de azul, extraída de algumas plantas leguminosas.
\section{Anil}
\begin{itemize}
\item {Grp. gram.:adj.}
\end{itemize}
\begin{itemize}
\item {Proveniência:(Lat. \textunderscore anilis\textunderscore )}
\end{itemize}
Relativo a mulher velha.
\section{Anilado}
\begin{itemize}
\item {Grp. gram.:adj.}
\end{itemize}
Que tem côr de anil.
\section{Anilado}
\begin{itemize}
\item {Grp. gram.:adj.}
\end{itemize}
O mesmo que \textunderscore aniellado\textunderscore . Cf. Bluteau.
(Pela absorpção do \textunderscore e\textunderscore  átono no \textunderscore i\textunderscore  igualmente átono. Cp. Viana, \textunderscore Apostilas\textunderscore , vb. \textunderscore anielado\textunderscore )
\section{Anilar}
\begin{itemize}
\item {Grp. gram.:v. t.}
\end{itemize}
Dar côr de anil a; tingir de azul.
\section{Anileira}
\begin{itemize}
\item {Grp. gram.:f.}
\end{itemize}
Árvore, que produz anil.
\section{Anileiro}
\begin{itemize}
\item {Grp. gram.:m.}
\end{itemize}
O mesmo que \textunderscore anileira\textunderscore .
Gênero de árvores, de cujas espécies só um pequeno número fornece anil. Cf. Ficalho, \textunderscore Plantas Úteis\textunderscore , 128.
\section{Anilema}
\begin{itemize}
\item {Grp. gram.:f.}
\end{itemize}
Gênero de plantas commelináceas, de que há 59 espécies na África, Ásia e Austrália.
\section{Anilha}
\begin{itemize}
\item {Grp. gram.:f.}
\end{itemize}
\begin{itemize}
\item {Proveniência:(Lat. \textunderscore anicula\textunderscore )}
\end{itemize}
Pequeno arco.
O mesmo que \textunderscore anilho\textunderscore .
\section{Anilhaçar}
\begin{itemize}
\item {Grp. gram.:v. t.}
\end{itemize}
\begin{itemize}
\item {Utilização:Ant.}
\end{itemize}
Prender com anilhos.
\section{Anilhar}
\begin{itemize}
\item {Grp. gram.:v. t.}
\end{itemize}
Pôr anilhos em.
\section{Anilho}
\begin{itemize}
\item {Grp. gram.:m.}
\end{itemize}
\begin{itemize}
\item {Utilização:Bras}
\end{itemize}
Pequena argola, para enfiar cordões ou para guarnecer ilhós.
Parte da colhera, com que se enlaça o pescoço do animal e que é fechada por um botão.
(Cast. \textunderscore anillo\textunderscore )
\section{Anilido}
\begin{itemize}
\item {Grp. gram.:m.}
\end{itemize}
Corpo, que representa os saes da \textunderscore anilina\textunderscore .
\section{Anilina}
\begin{itemize}
\item {Grp. gram.:f.}
\end{itemize}
\begin{itemize}
\item {Proveniência:(De \textunderscore anil\textunderscore )}
\end{itemize}
Substância líquida, incolor, que se obtém pela combinação do índigo com um excesso de potassa.
\section{Anilocros}
\begin{itemize}
\item {Grp. gram.:m. pl.}
\end{itemize}
Gênero de crustáceos isópodes, que vivem no Mediterrâneo.
\section{Animação}
\begin{itemize}
\item {Grp. gram.:f.}
\end{itemize}
Acto ou effeito de \textunderscore animar\textunderscore .
Vivacidade.
Enthusiasmo.
\section{Animadamente}
\begin{itemize}
\item {Grp. gram.:adv.}
\end{itemize}
Com animação.
\section{Animador}
\begin{itemize}
\item {Grp. gram.:m.}
\end{itemize}
\begin{itemize}
\item {Grp. gram.:Adj.}
\end{itemize}
Aquelle que anima.
Que anima, que estimula: \textunderscore palavras animadoras\textunderscore .
\section{Animadversão}
\begin{itemize}
\item {Grp. gram.:f.}
\end{itemize}
\begin{itemize}
\item {Proveniência:(Lat. \textunderscore animadversio\textunderscore )}
\end{itemize}
Censura.
Castigo.
Ódio.
\section{Animadvertir}
\begin{itemize}
\item {Grp. gram.:v. t.}
\end{itemize}
\begin{itemize}
\item {Utilização:P. us.}
\end{itemize}
\begin{itemize}
\item {Proveniência:(Lat. \textunderscore animadvertere\textunderscore )}
\end{itemize}
Tomar em attenção.
Reprehender; castigar.
\section{Animal}
\begin{itemize}
\item {Grp. gram.:m.}
\end{itemize}
\begin{itemize}
\item {Utilização:Bras. do N}
\end{itemize}
\begin{itemize}
\item {Proveniência:(Lat. \textunderscore animal\textunderscore )}
\end{itemize}
Sêr organizado, que tem sensibilidade e movimento próprio: \textunderscore o homem é o rei dos animaes\textunderscore .
Sêr vivo, irracional: \textunderscore deve-se protecção aos animaes\textunderscore .
Pessôa estúpida: \textunderscore aquelle seu vizinho é um animal\textunderscore .
Indivíduo da raça cavallar, exclusivamente.
\section{Animal}
\begin{itemize}
\item {Grp. gram.:adj.}
\end{itemize}
\begin{itemize}
\item {Proveniência:(Lat. \textunderscore animalis\textunderscore )}
\end{itemize}
Pertencente aos seres que vivem e têm sensibilidade e movimento próprio: \textunderscore o reino animal\textunderscore .
Próprio dos seres irracionaes; que provém dos irracionaes. Material, carnal: \textunderscore a vida animal\textunderscore .
\section{Animalaço}
\begin{itemize}
\item {Grp. gram.:m.}
\end{itemize}
\begin{itemize}
\item {Utilização:Fam.}
\end{itemize}
\begin{itemize}
\item {Proveniência:(De \textunderscore animal\textunderscore )}
\end{itemize}
Grande animal; estupidarrão.
\section{Animalão}
\begin{itemize}
\item {Grp. gram.:m.}
\end{itemize}
O mesmo que \textunderscore animalaço\textunderscore .
\section{Animalco}
\begin{itemize}
\item {Grp. gram.:adj.}
\end{itemize}
Relativo a \textunderscore animalejo\textunderscore ? Cf. Filinto, XIII, 233.
\section{Animalcular}
\begin{itemize}
\item {Grp. gram.:adj.}
\end{itemize}
Relativo a animálculos.
\section{Animalculismo}
\begin{itemize}
\item {Grp. gram.:m.}
\end{itemize}
\begin{itemize}
\item {Proveniência:(De \textunderscore animálculo\textunderscore )}
\end{itemize}
Systema dos que dizem que o embryão é produzido pelos animálculos espermáticos.
\section{Animalculista}
\begin{itemize}
\item {Grp. gram.:m.}
\end{itemize}
Aquelle que segue o animalculismo.
\section{Animálculo}
\begin{itemize}
\item {Grp. gram.:m.}
\end{itemize}
Animal microscópico.
(Dem. de \textunderscore animal\textunderscore )
\section{Animalculovismo}
\begin{itemize}
\item {Grp. gram.:m.}
\end{itemize}
\begin{itemize}
\item {Proveniência:(De \textunderscore animálculo\textunderscore  + \textunderscore ovo\textunderscore )}
\end{itemize}
Systema physiológico, segundo o qual o embryão animal é produzido pela juncção dos animálculos espermáticos e do óvulo da fêmea.
\section{Animalculovista}
\begin{itemize}
\item {Grp. gram.:m.}
\end{itemize}
Sectário do animalculovismo.
\section{Animalejo}
\begin{itemize}
\item {Grp. gram.:m.}
\end{itemize}
\begin{itemize}
\item {Utilização:Fam.}
\end{itemize}
Pequeno animal.
Pessôa estúpida.
\section{Animalesco}
\begin{itemize}
\item {fónica:lês}
\end{itemize}
\begin{itemize}
\item {Grp. gram.:adj.}
\end{itemize}
Relativo aos animaes.
Que participa das qualidades delles.
\section{Animalha}
\begin{itemize}
\item {Grp. gram.:f.}
\end{itemize}
\begin{itemize}
\item {Utilização:Ant.}
\end{itemize}
O mesmo que \textunderscore animália\textunderscore . Cf. Cortesão, \textunderscore Subs.\textunderscore 
\section{Animália}
\begin{itemize}
\item {Grp. gram.:f.}
\end{itemize}
\begin{itemize}
\item {Proveniência:(Lat. \textunderscore animalia\textunderscore )}
\end{itemize}
Bêsta; animal.
Fera.
Alimária.
\section{Animalicida}
\begin{itemize}
\item {Grp. gram.:m.}
\end{itemize}
Matador de animaes.
(Cp. \textunderscore animalicídio\textunderscore )
\section{Animalicídio}
\begin{itemize}
\item {Grp. gram.:m.}
\end{itemize}
\begin{itemize}
\item {Proveniência:(Do lat. \textunderscore animal\textunderscore  + \textunderscore caedere\textunderscore )}
\end{itemize}
Morte violenta de animal.
\section{Animalidade}
\begin{itemize}
\item {Grp. gram.:f.}
\end{itemize}
Attributos do animal.
\section{Animalinho}
\begin{itemize}
\item {Grp. gram.:m.}
\end{itemize}
O mesmo que \textunderscore animalzinho\textunderscore .
\section{Animalismo}
\begin{itemize}
\item {Grp. gram.:m.}
\end{itemize}
Natureza do animal.
\section{Animalista}
\begin{itemize}
\item {Grp. gram.:m.}
\end{itemize}
\begin{itemize}
\item {Proveniência:(De \textunderscore animal\textunderscore )}
\end{itemize}
Artista, que se dedica á pintura ou esculptura de animaes.
\section{Animalito}
\begin{itemize}
\item {Grp. gram.:m.}
\end{itemize}
O mesmo que \textunderscore animálculo\textunderscore . Cf. Garrett, \textunderscore Helena\textunderscore , 8.
\section{Animalização}
\begin{itemize}
\item {Grp. gram.:f.}
\end{itemize}
Acto de \textunderscore animalizar\textunderscore .
\section{Animalizar}
\begin{itemize}
\item {Grp. gram.:v. t.}
\end{itemize}
Converter (alimentos) em substância animal.
\section{Animal-planta}
\begin{itemize}
\item {Grp. gram.:m.}
\end{itemize}
Cada um dos seres, que a sciência ainda não resolveu se pertencem ao mundo vegetal ou ao mundo animal.
\section{Animalzinho}
\begin{itemize}
\item {Grp. gram.:m.}
\end{itemize}
(dem. de \textunderscore animal\textunderscore )
\section{Anima-membeça}
\begin{itemize}
\item {Grp. gram.:f.}
\end{itemize}
Planta arácea do Pará, (\textunderscore maranta aquatica\textunderscore ).
\section{Animante}
\begin{itemize}
\item {Grp. gram.:adj.}
\end{itemize}
\begin{itemize}
\item {Proveniência:(Lat. \textunderscore animans\textunderscore )}
\end{itemize}
Que anima.
\section{Animar}
\begin{itemize}
\item {Grp. gram.:v. t.}
\end{itemize}
\begin{itemize}
\item {Proveniência:(Lat. \textunderscore animare\textunderscore )}
\end{itemize}
Dar vida, acção, movimento, enthusiasmo, coragem, a.
Desenvolver.
\section{Animato}
\begin{itemize}
\item {Grp. gram.:adv.}
\end{itemize}
\begin{itemize}
\item {Utilização:Mús.}
\end{itemize}
\begin{itemize}
\item {Proveniência:(T. it.)}
\end{itemize}
Com animação e calor.
\section{Animatógrafo}
\begin{itemize}
\item {Grp. gram.:m.}
\end{itemize}
\begin{itemize}
\item {Proveniência:(T. hyb., do lat. \textunderscore animatus\textunderscore  + \textunderscore graphein\textunderscore )}
\end{itemize}
Apparelho, formado por uma câmara photográfica especial, e que, baseado no cinematógrafo de Edison, faz projectar numa tela imagens ou quadros em movimento.
Cinematógrafo.
\section{Animatógrapho}
\begin{itemize}
\item {Grp. gram.:m.}
\end{itemize}
\begin{itemize}
\item {Proveniência:(T. hyb., do lat. \textunderscore animatus\textunderscore  + \textunderscore graphein\textunderscore )}
\end{itemize}
Apparelho, formado por uma câmara photográphica especial, e que, baseado no cinematógrapho de Edison, faz projectar numa tela imagens ou quadros em movimento.
Cinematógrapho.
\section{Animável}
\begin{itemize}
\item {Grp. gram.:adj.}
\end{itemize}
Susceptível de sêr animado.
\section{Anime}
\begin{itemize}
\item {Grp. gram.:m.}
\end{itemize}
Espécie de resina.
Goma copal.--Moraes lia \textunderscore ânime\textunderscore .
\section{Animicida}
\begin{itemize}
\item {Grp. gram.:m.}
\end{itemize}
\begin{itemize}
\item {Utilização:Des.}
\end{itemize}
\begin{itemize}
\item {Proveniência:(Do lat. \textunderscore anima\textunderscore  + \textunderscore caedere\textunderscore )}
\end{itemize}
Aquelle que mata a alma.
\section{Anímico}
\begin{itemize}
\item {Grp. gram.:adj.}
\end{itemize}
\begin{itemize}
\item {Proveniência:(Do lat. \textunderscore anima\textunderscore )}
\end{itemize}
Relativo á alma.
\section{Animina}
\begin{itemize}
\item {Grp. gram.:f.}
\end{itemize}
\begin{itemize}
\item {Proveniência:(Do lat. \textunderscore anima\textunderscore )}
\end{itemize}
Uma das quatro bases salificáveis, achadas no óleo animal de Dippel.
\section{Animismo}
\begin{itemize}
\item {Grp. gram.:m.}
\end{itemize}
\begin{itemize}
\item {Proveniência:(Do lat. \textunderscore anima\textunderscore )}
\end{itemize}
Systema dos que consideram a alma como causa de todos os phenómenos vitaes.
\section{Animista}
\begin{itemize}
\item {Grp. gram.:m.}
\end{itemize}
Sectário do \textunderscore animismo\textunderscore .
\section{Ânimo}
\begin{itemize}
\item {Grp. gram.:m.}
\end{itemize}
\begin{itemize}
\item {Proveniência:(Lat. \textunderscore animus\textunderscore )}
\end{itemize}
Espírito.
Vida.
Índole: \textunderscore ânimo ruim\textunderscore .
Valor, coragem: \textunderscore não teve ânimo de o matar\textunderscore .
Intenção: \textunderscore sem ânimo de offensa\textunderscore .
\section{Animosamente}
\begin{itemize}
\item {Grp. gram.:adv.}
\end{itemize}
De modo \textunderscore animoso\textunderscore .
\section{Animosidade}
\begin{itemize}
\item {Grp. gram.:f.}
\end{itemize}
\begin{itemize}
\item {Proveniência:(Lat. \textunderscore animositas\textunderscore )}
\end{itemize}
Coragem.
Malquerença; ódio persistente.
\section{Animoso}
\begin{itemize}
\item {Grp. gram.:adj.}
\end{itemize}
Que tem ânimo; corajoso.
\section{Anina}
\begin{itemize}
\item {Grp. gram.:f.}
\end{itemize}
(Corr. de \textunderscore anilha\textunderscore )
\section{Aninar}
\begin{itemize}
\item {Grp. gram.:v. t.}
\end{itemize}
\begin{itemize}
\item {Utilização:Fam.}
\end{itemize}
\begin{itemize}
\item {Utilização:Náut.}
\end{itemize}
\begin{itemize}
\item {Proveniência:(De \textunderscore nino\textunderscore  por \textunderscore menino\textunderscore )}
\end{itemize}
Embalar.
Rebater (a ponta da cavilha sôbre a arruela).
\section{Aninga}
\begin{itemize}
\item {Grp. gram.:f.}
\end{itemize}
\begin{itemize}
\item {Utilização:Bras. do N}
\end{itemize}
Planta aroídea, de fruto comestível.
\section{Aningaíba}
\begin{itemize}
\item {Grp. gram.:f.}
\end{itemize}
\begin{itemize}
\item {Utilização:Bras}
\end{itemize}
O mesmo que \textunderscore aningaúva\textunderscore .
\section{Anaco}
\begin{itemize}
\item {Grp. gram.:m.  e  adj.}
\end{itemize}
\begin{itemize}
\item {Utilização:Prov.}
\end{itemize}
\begin{itemize}
\item {Utilização:alent.}
\end{itemize}
Animal de um ano.
Exclusivamente, o bode de um ano.
\section{Anais}
\begin{itemize}
\item {Grp. gram.:m. pl.}
\end{itemize}
\begin{itemize}
\item {Proveniência:(Lat. \textunderscore annales\textunderscore )}
\end{itemize}
História ou narração organizada ano por ano.
Publicação periódica de sciências, letras ou artes.
História.
\section{Anal}
\begin{itemize}
\item {Grp. gram.:adj.}
\end{itemize}
\begin{itemize}
\item {Grp. gram.:M.}
\end{itemize}
O mesmo que \textunderscore anual\textunderscore .
Ceremónia religiosa, que se celebra todos os dias, durante um ano. Cf. G. Vicente, I, 254.
\section{Analista}
\begin{itemize}
\item {Grp. gram.:m.}
\end{itemize}
\begin{itemize}
\item {Proveniência:(Do lat. \textunderscore annales\textunderscore )}
\end{itemize}
Aquelle que escreve anais.
\section{Ana-pinta}
\begin{itemize}
\item {Grp. gram.:f.}
\end{itemize}
\begin{itemize}
\item {Utilização:Bras}
\end{itemize}
O mesmo que \textunderscore capitão-do-mato\textunderscore .
\section{Anásia}
\begin{itemize}
\item {Grp. gram.:f.}
\end{itemize}
\begin{itemize}
\item {Utilização:Ant.}
\end{itemize}
O mesmo que \textunderscore annata\textunderscore .
\section{Ana-velha}
\begin{itemize}
\item {Grp. gram.:f.}
\end{itemize}
Pequena ave pernalta do Brasil.
\section{Aneiro}
\begin{itemize}
\item {Grp. gram.:adj.}
\end{itemize}
\begin{itemize}
\item {Proveniência:(De \textunderscore anno\textunderscore )}
\end{itemize}
Dependente da maneira como correr o ano.
Contingente, incerto.
Diz-se das árvores, que num ano produzem muito e no seguinte nada.
\section{Anejo}
\begin{itemize}
\item {Grp. gram.:adj.}
\end{itemize}
O mesmo que \textunderscore anelho\textunderscore .
\section{Anelho}
\begin{itemize}
\item {Grp. gram.:adj.}
\end{itemize}
O mesmo que \textunderscore anaco\textunderscore ^2.
\section{Anesa}
\begin{itemize}
\item {Grp. gram.:f.}
\end{itemize}
\begin{itemize}
\item {Utilização:Prov.}
\end{itemize}
\begin{itemize}
\item {Utilização:minh.}
\end{itemize}
Ano agrícola ou económico.
\section{Anexação}
\begin{itemize}
\item {fónica:csa}
\end{itemize}
\begin{itemize}
\item {Grp. gram.:f.}
\end{itemize}
Acto de \textunderscore anexar\textunderscore .
\section{Anexar}
\begin{itemize}
\item {fónica:csa}
\end{itemize}
\begin{itemize}
\item {Grp. gram.:v. t.}
\end{itemize}
\begin{itemize}
\item {Proveniência:(Lat. \textunderscore annexare\textunderscore )}
\end{itemize}
Juntar, ligar.
\section{Anexidade}
\begin{itemize}
\item {fónica:csi}
\end{itemize}
\begin{itemize}
\item {Grp. gram.:f.}
\end{itemize}
\begin{itemize}
\item {Utilização:Ant.}
\end{itemize}
O mesmo que \textunderscore anexo\textunderscore , m.
\section{Anexionismo}
\begin{itemize}
\item {fónica:csi}
\end{itemize}
\begin{itemize}
\item {Grp. gram.:m.}
\end{itemize}
\begin{itemize}
\item {Proveniência:(Do lat. \textunderscore annexio\textunderscore )}
\end{itemize}
Theoria, segundo a qual se pretende que os pequenos Estados devem reunir-se aos grandes, seus vizinhos, sob pretexto de affinidade de raça, língua, etc.
\section{Anexionista}
\begin{itemize}
\item {fónica:csi}
\end{itemize}
\begin{itemize}
\item {Grp. gram.:m.}
\end{itemize}
Partidário de annexação de um país, de um territorio, a outro.
\section{Anexo}
\begin{itemize}
\item {Grp. gram.:m.}
\end{itemize}
\begin{itemize}
\item {Grp. gram.:Adj.}
\end{itemize}
\begin{itemize}
\item {Proveniência:(Lat. \textunderscore annexus\textunderscore )}
\end{itemize}
Aquillo que está ligado, como accessório.
Dependência.
Ligado, junto.
Encorporado.
Sujeito.
\section{Anício}
\begin{itemize}
\item {Grp. gram.:m.}
\end{itemize}
\begin{itemize}
\item {Utilização:Ant.}
\end{itemize}
Título ou instrumento público, em que se referiam as circunstâncias que haviam precedido o mesmo título.
\section{Aniilar}
\begin{itemize}
\item {Proveniência:(Lat. \textunderscore annihilare\textunderscore )}
\end{itemize}
\textunderscore v. t.\textunderscore  (e der.)
O mesmo que \textunderscore aniquilar\textunderscore , etc.
\section{Aningal}
\begin{itemize}
\item {Grp. gram.:m.}
\end{itemize}
\begin{itemize}
\item {Utilização:Bras}
\end{itemize}
Mato de aninga.
\section{Aningapari}
\begin{itemize}
\item {Grp. gram.:f.}
\end{itemize}
Planta melastomácea do Brasil, (\textunderscore melastoma parviflora\textunderscore , Lamk.).
\section{Aningaúba}
\begin{itemize}
\item {Grp. gram.:f.}
\end{itemize}
O mesmo que \textunderscore aningaúva\textunderscore .
\section{Aningaúva}
\begin{itemize}
\item {Grp. gram.:f.}
\end{itemize}
Planta aracea do Brasil, (\textunderscore philodendron arborescens\textunderscore ).
O mesmo que \textunderscore cipó-do-imbê\textunderscore .
\section{Aninhar}
\begin{itemize}
\item {Grp. gram.:v. t.}
\end{itemize}
\begin{itemize}
\item {Grp. gram.:V. p.}
\end{itemize}
Pôr em ninho.
Agasalhar.
Esconder.
Receber.
Acocorar-se; sentar-se, cruzando as pernas.
\section{Aninho}
\begin{itemize}
\item {Grp. gram.:m.}
\end{itemize}
\begin{itemize}
\item {Utilização:Prov.}
\end{itemize}
\begin{itemize}
\item {Utilização:alent.}
\end{itemize}
Lan da primeira tosquia.
Lan de carneiro ou de ovelha de um ano.
(Cp. \textunderscore anínio\textunderscore )
\section{Anínio}
\begin{itemize}
\item {Grp. gram.:adj.}
\end{itemize}
\begin{itemize}
\item {Utilização:Ant.}
\end{itemize}
Relativo a cordeiro.
(Por \textunderscore agnínio\textunderscore , do lat. \textunderscore agnus\textunderscore )
\section{Aniodol}
\begin{itemize}
\item {Grp. gram.:m.}
\end{itemize}
\begin{itemize}
\item {Utilização:Pharm.}
\end{itemize}
Solução antiséptica, empregada na desinfecção das mãos e dos instrumentos.
\section{Aniquilação}
\begin{itemize}
\item {Grp. gram.:f.}
\end{itemize}
Acto de \textunderscore aniquilar\textunderscore .
\section{Aniquilador}
\begin{itemize}
\item {Grp. gram.:m.}
\end{itemize}
O que aniquila.
\section{Aniquilamento}
\begin{itemize}
\item {Grp. gram.:m.}
\end{itemize}
Effeito de \textunderscore aniquilar\textunderscore .
\section{Aniquilar}
\begin{itemize}
\item {Grp. gram.:v. t.}
\end{itemize}
\begin{itemize}
\item {Proveniência:(Lat. \textunderscore ad\textunderscore  + \textunderscore nihilare\textunderscore , de \textunderscore nihil\textunderscore )}
\end{itemize}
Reduzir a nada; destruir.
Abater.
\section{Aniria}
\begin{itemize}
\item {Grp. gram.:f.}
\end{itemize}
\begin{itemize}
\item {Proveniência:(Do gr. \textunderscore an\textunderscore  priv. + \textunderscore íris\textunderscore )}
\end{itemize}
Falta de íris, no ôlho.
\section{Anis}
\begin{itemize}
\item {Grp. gram.:m.}
\end{itemize}
\begin{itemize}
\item {Proveniência:(Fr. \textunderscore anis\textunderscore , do lat. \textunderscore anisum\textunderscore )}
\end{itemize}
Planta umbellifera.
Semente dessa planta.
Erva-doce.
Licor, aromatizado com a mesma planta.
\section{Anisacanta}
\begin{itemize}
\item {Grp. gram.:f.}
\end{itemize}
Arbusto salsoláceo da Austrália.
\section{Anisacantha}
\begin{itemize}
\item {Grp. gram.:f.}
\end{itemize}
Arbusto salsoláceo da Austrália.
\section{Anisacantho}
\begin{itemize}
\item {Grp. gram.:m.}
\end{itemize}
Gênero de plantas acantháceas.
\section{Anisacanto}
\begin{itemize}
\item {Grp. gram.:m.}
\end{itemize}
Gênero de plantas acantháceas.
\section{Anisantho}
\begin{itemize}
\item {Grp. gram.:adj.}
\end{itemize}
\begin{itemize}
\item {Proveniência:(Do gr. \textunderscore an\textunderscore  + \textunderscore isos\textunderscore  + \textunderscore anthos\textunderscore )}
\end{itemize}
Diz-se das plantas que têm flôres desiguaes.
\section{Anisanto}
\begin{itemize}
\item {Grp. gram.:adj.}
\end{itemize}
\begin{itemize}
\item {Proveniência:(Do gr. \textunderscore an\textunderscore  + \textunderscore isos\textunderscore  + \textunderscore anthos\textunderscore )}
\end{itemize}
Diz-se das plantas que têm flôres desiguaes.
\section{Anisar}
\begin{itemize}
\item {Grp. gram.:v. t.}
\end{itemize}
Preparar com anis.
Dar sabor de anis a.
\section{Anisato}
\begin{itemize}
\item {Grp. gram.:m.}
\end{itemize}
Combinação do ácido anísico com uma base.
\section{Anisathro}
\begin{itemize}
\item {Grp. gram.:m.}
\end{itemize}
Gênero de insectos coleópteros longicórneos.
\section{Anisatro}
\begin{itemize}
\item {Grp. gram.:m.}
\end{itemize}
Gênero de insectos coleópteros longicórneos.
\section{Anis-da-Sibéria}
\begin{itemize}
\item {Grp. gram.:m.}
\end{itemize}
O mesmo que \textunderscore badiana\textunderscore .
\section{Aniseias}
\begin{itemize}
\item {Grp. gram.:f. pl.}
\end{itemize}
\begin{itemize}
\item {Proveniência:(Do gr. \textunderscore anison\textunderscore )}
\end{itemize}
Gênero de plantas, da fam. das convolvuláceas.
\section{Aniseira}
\begin{itemize}
\item {Grp. gram.:f.}
\end{itemize}
Anis.
Terreno semeado de anis.
\section{Anis-estrellado}
\begin{itemize}
\item {Grp. gram.:m.}
\end{itemize}
O mesmo que \textunderscore badiana\textunderscore .
\section{Aniseta}
\begin{itemize}
\item {fónica:zê}
\end{itemize}
\begin{itemize}
\item {Grp. gram.:f.}
\end{itemize}
\begin{itemize}
\item {Utilização:Bras}
\end{itemize}
\begin{itemize}
\item {Proveniência:(Fr. \textunderscore anisette\textunderscore )}
\end{itemize}
Licor de anis.
Árvore fructífera.
\section{Anísico}
\begin{itemize}
\item {Grp. gram.:adj.}
\end{itemize}
Diz-se de um ácido, resultante da acção do ácido azótico sôbre a essência de anis.
\section{Anisidina}
\begin{itemize}
\item {Grp. gram.:f.}
\end{itemize}
\begin{itemize}
\item {Proveniência:(De \textunderscore anis\textunderscore  + \textunderscore ácido\textunderscore )}
\end{itemize}
Substância crystallina, que se combina com os ácidos para formar saes.
\section{Anisilo}
\begin{itemize}
\item {Grp. gram.:m.}
\end{itemize}
\begin{itemize}
\item {Utilização:Chím.}
\end{itemize}
Rad. hypoth. do ácido anísico e outros.
\section{Anisina}
\begin{itemize}
\item {Grp. gram.:f.}
\end{itemize}
Principio estimulante do anis.
\section{Anisocéfalo}
\begin{itemize}
\item {Grp. gram.:adj.}
\end{itemize}
\begin{itemize}
\item {Utilização:Bot.}
\end{itemize}
\begin{itemize}
\item {Proveniência:(Do gr. \textunderscore an\textunderscore  priv. + \textunderscore isos\textunderscore  + \textunderscore kephale\textunderscore )}
\end{itemize}
Diz-se das plantas, cujas flôres fórmam capítulos desiguaes.
\section{Anisocéphalo}
\begin{itemize}
\item {Grp. gram.:adj.}
\end{itemize}
\begin{itemize}
\item {Utilização:Bot.}
\end{itemize}
\begin{itemize}
\item {Proveniência:(Do gr. \textunderscore an\textunderscore  priv. + \textunderscore isos\textunderscore  + \textunderscore kephale\textunderscore )}
\end{itemize}
Diz-se das plantas, cujas flôres fórmam capítulos desiguaes.
\section{Anisócera}
\begin{itemize}
\item {Grp. gram.:f.}
\end{itemize}
O mesmo que \textunderscore anisócero\textunderscore .
\section{Anisócero}
\begin{itemize}
\item {Grp. gram.:m.}
\end{itemize}
\begin{itemize}
\item {Proveniência:(Do gr. \textunderscore an\textunderscore  priv. + \textunderscore isos\textunderscore  + \textunderscore keras\textunderscore )}
\end{itemize}
Gênero de insectos coleópteros pentâmeros.
\section{Anisociclo}
\begin{itemize}
\item {Grp. gram.:m.}
\end{itemize}
\begin{itemize}
\item {Proveniência:(Do gr. \textunderscore an\textunderscore  priv. + \textunderscore isos\textunderscore  + \textunderscore kuklos\textunderscore )}
\end{itemize}
Máquina de guerra, com que a milícia byzantina despedia frechas.
\section{Anisocoria}
\begin{itemize}
\item {Grp. gram.:f.}
\end{itemize}
\begin{itemize}
\item {Proveniência:(Do gr. \textunderscore an\textunderscore  priv. + \textunderscore isos\textunderscore  + \textunderscore kore\textunderscore )}
\end{itemize}
Phenómeno mórbido, caracterizado pela desigualdade das pupillas.
\section{Anisocrépida}
\begin{itemize}
\item {Grp. gram.:f.}
\end{itemize}
Gênero de insectos coleópteros heterómeros.
\section{Anisocyclo}
\begin{itemize}
\item {Grp. gram.:m.}
\end{itemize}
\begin{itemize}
\item {Proveniência:(Do gr. \textunderscore an\textunderscore  priv. + \textunderscore isos\textunderscore  + \textunderscore kuklos\textunderscore )}
\end{itemize}
Máquina de guerra, com que a milícia byzantina despedia frechas.
\section{Anisodáctilo}
\begin{itemize}
\item {Grp. gram.:m.}
\end{itemize}
Gênero de coleópteros.
\section{Anisodáctylo}
\begin{itemize}
\item {Grp. gram.:m.}
\end{itemize}
Gênero de coleópteros.
\section{Anisódero}
\begin{itemize}
\item {Grp. gram.:m.}
\end{itemize}
Gênero de coleópteros.
\section{Anísodo}
\begin{itemize}
\item {Grp. gram.:m.}
\end{itemize}
O mesmo que \textunderscore anisodonte\textunderscore .
\section{Anisodonte}
\begin{itemize}
\item {Grp. gram.:m.}
\end{itemize}
\begin{itemize}
\item {Proveniência:(Do gr. \textunderscore anisos\textunderscore  + \textunderscore odous\textunderscore , \textunderscore odontos\textunderscore )}
\end{itemize}
Gênero de plantas labiadas.
\section{Anisofilo}
\begin{itemize}
\item {Grp. gram.:adj.}
\end{itemize}
\begin{itemize}
\item {Proveniência:(Do gr. \textunderscore an\textunderscore  + \textunderscore isos\textunderscore  + \textunderscore phullon\textunderscore )}
\end{itemize}
Que tem fôlhas desiguaes.
\section{Anisogónio}
\begin{itemize}
\item {Grp. gram.:m.}
\end{itemize}
Gênero de fêtos.
\section{Anisoína}
\begin{itemize}
\item {Grp. gram.:f.}
\end{itemize}
Corpo crystallizavel e volátil, producto da decomposição de cânfora de anis.
\section{Anisónela}
\begin{itemize}
\item {Grp. gram.:f.}
\end{itemize}
Gênero de plantas labiadas.
\section{Anisómero}
\begin{itemize}
\item {Grp. gram.:adj.}
\end{itemize}
\begin{itemize}
\item {Proveniência:(Do gr. \textunderscore anisos\textunderscore  + \textunderscore meros\textunderscore )}
\end{itemize}
Formado de partes desiguaes.
\section{Anisométrico}
\begin{itemize}
\item {Grp. gram.:adj.}
\end{itemize}
\begin{itemize}
\item {Utilização:Miner.}
\end{itemize}
\begin{itemize}
\item {Utilização:Mathem.}
\end{itemize}
\begin{itemize}
\item {Proveniência:(Do gr. \textunderscore an\textunderscore  + \textunderscore isos\textunderscore  + \textunderscore metron\textunderscore )}
\end{itemize}
Diz-se do systema de crystallização, que apresenta três ângulos desiguaes.
Diz-se da projecção axinométrica, quando as três direcções ou eixos principaes têm inclinações desiguaes sôbre o plano de projecção.
\section{Anisometropia}
\begin{itemize}
\item {Grp. gram.:f.}
\end{itemize}
\begin{itemize}
\item {Utilização:Med.}
\end{itemize}
\begin{itemize}
\item {Proveniência:(Do gr. \textunderscore an\textunderscore  priv. + \textunderscore isos\textunderscore  + \textunderscore metron\textunderscore  + \textunderscore ops\textunderscore )}
\end{itemize}
Desigualdade de refracção nos dois olhos.
\section{Anisometrópico}
\begin{itemize}
\item {Grp. gram.:adj.}
\end{itemize}
Que tem \textunderscore anisometropia\textunderscore .
\section{Anisonema}
\begin{itemize}
\item {Grp. gram.:f.}
\end{itemize}
\begin{itemize}
\item {Proveniência:(Do gr. \textunderscore an\textunderscore  + \textunderscore isos\textunderscore  + \textunderscore nema\textunderscore )}
\end{itemize}
Gênero de infusórios.
Planta euphorbiácea do Oriente.
\section{Anisopelme}
\begin{itemize}
\item {Grp. gram.:m.}
\end{itemize}
\begin{itemize}
\item {Proveniência:(Do gr. \textunderscore an\textunderscore  + \textunderscore isos\textunderscore  + \textunderscore pelma\textunderscore )}
\end{itemize}
Gênero de insectos hymenópteros.
\section{Anisopétalo}
\begin{itemize}
\item {Grp. gram.:adj.}
\end{itemize}
\begin{itemize}
\item {Utilização:Bot.}
\end{itemize}
\begin{itemize}
\item {Proveniência:(Do gr. \textunderscore an\textunderscore  + \textunderscore isos\textunderscore  + \textunderscore petale\textunderscore )}
\end{itemize}
Que tem pétalas desiguaes.
\section{Anisophyllo}
\begin{itemize}
\item {Grp. gram.:adj.}
\end{itemize}
\begin{itemize}
\item {Proveniência:(Do gr. \textunderscore an\textunderscore  + \textunderscore isos\textunderscore  + \textunderscore phullon\textunderscore )}
\end{itemize}
Que tem fôlhas desiguaes.
\section{Anisostêmone}
\begin{itemize}
\item {Grp. gram.:adj.}
\end{itemize}
\begin{itemize}
\item {Proveniência:(Do gr. \textunderscore an\textunderscore  priv. + \textunderscore isos\textunderscore  + \textunderscore stemon\textunderscore )}
\end{itemize}
Diz-se da flôr, que não tem igual número de pétalas e estames.
\section{Anisotrópico}
\begin{itemize}
\item {Grp. gram.:adj.}
\end{itemize}
Que não é isotrópico.
\section{Anisto}
\begin{itemize}
\item {Grp. gram.:adj.}
\end{itemize}
\begin{itemize}
\item {Utilização:Anat.}
\end{itemize}
\begin{itemize}
\item {Utilização:Bot.}
\end{itemize}
\begin{itemize}
\item {Proveniência:(Do gr. \textunderscore an\textunderscore  priv. + \textunderscore histos\textunderscore , tecido)}
\end{itemize}
Que não tem textura determinada.
Diz-se dos órgãos vegetaes, em que nem com o microscópio se descobrem vestígios de tecido cellular, como succede nos tubos exteriores das confervas.
\section{Anisulmina}
\begin{itemize}
\item {Grp. gram.:f.}
\end{itemize}
Producto chímico, cinzento, obtido pela acção da potassa sôbre os grãos de anis, depois de certas preparações.
\section{Anisúrico}
\begin{itemize}
\item {Grp. gram.:adj.}
\end{itemize}
Diz-se de um ácido obtido pela acção do chloreto de anisylo sôbre o derivado argêntico do glysocollo.
\section{Anis-verde}
\begin{itemize}
\item {Grp. gram.:m.}
\end{itemize}
\begin{itemize}
\item {Utilização:Bras}
\end{itemize}
O mesmo que \textunderscore erva-doce\textunderscore .
\section{Anisylo}
\begin{itemize}
\item {Grp. gram.:m.}
\end{itemize}
\begin{itemize}
\item {Utilização:Chím.}
\end{itemize}
Rad. hypoth. do ácido anísico e outros.
\section{Anivelar}
\textunderscore v. t.\textunderscore  (e der.)
(V. \textunderscore nivelar\textunderscore , etc.) Cf. Garrett, \textunderscore Helena\textunderscore , 91.
\section{Aniversariamente}
\begin{itemize}
\item {Grp. gram.:adv.}
\end{itemize}
\begin{itemize}
\item {Proveniência:(De \textunderscore anniversario\textunderscore )}
\end{itemize}
Em dia certo de cada ano.
\section{Aniversariante}
\begin{itemize}
\item {Grp. gram.:adj.}
\end{itemize}
\begin{itemize}
\item {Utilização:bras}
\end{itemize}
\begin{itemize}
\item {Utilização:Neol.}
\end{itemize}
Que aniversaria.
\section{Aniversariar}
\begin{itemize}
\item {Grp. gram.:v. i.}
\end{itemize}
\begin{itemize}
\item {Utilização:bras}
\end{itemize}
\begin{itemize}
\item {Utilização:Neol.}
\end{itemize}
Fazer aniversário, celebrar aniversário.
\section{Aniversário}
\begin{itemize}
\item {Grp. gram.:m.}
\end{itemize}
\begin{itemize}
\item {Grp. gram.:Adj.}
\end{itemize}
\begin{itemize}
\item {Proveniência:(Lat. \textunderscore anniversarius\textunderscore )}
\end{itemize}
Volta anual do dia, em que se deu certo acontecimento.
Diz-se do dia, em que se conta um ou mais anos sôbre a data de um acontecimento.
\section{Anixo}
\begin{itemize}
\item {fónica:cso}
\end{itemize}
\begin{itemize}
\item {Grp. gram.:m.}
\end{itemize}
\begin{itemize}
\item {Utilização:Ant.}
\end{itemize}
\begin{itemize}
\item {Proveniência:(Lat. \textunderscore annixus\textunderscore )}
\end{itemize}
Gancho de ferro, preso na extremidade de um pau ou vara.
\section{Anjango}
\begin{itemize}
\item {Grp. gram.:m.}
\end{itemize}
Árvore africana.
\section{Anjeela}
\begin{itemize}
\item {Grp. gram.:f.}
\end{itemize}
Embarcação oriental, formada de duas pirogas, reunidas por uma ponte.
\section{Anjinha}
\begin{itemize}
\item {Grp. gram.:f.}
\end{itemize}
\begin{itemize}
\item {Utilização:Fam.}
\end{itemize}
Menina, que se compara a um anjo:«\textunderscore a minha anjinha benta...\textunderscore »Castilho, \textunderscore Sabichonas\textunderscore , 82.
\section{Anjinho}
\begin{itemize}
\item {Grp. gram.:m.}
\end{itemize}
\begin{itemize}
\item {Utilização:Mad}
\end{itemize}
Ave marítima.
\section{Anjinhos}
\begin{itemize}
\item {Grp. gram.:m. pl.}
\end{itemize}
\begin{itemize}
\item {Proveniência:(Do lat. \textunderscore angere\textunderscore ?)}
\end{itemize}
Instrumento, com que se seguravam os criminosos pelos dedos das mãos.
\textunderscore Ir para os anjinhos\textunderscore , morrer.
\section{Anjo}
\begin{itemize}
\item {Grp. gram.:m.}
\end{itemize}
\begin{itemize}
\item {Utilização:Pop.}
\end{itemize}
\begin{itemize}
\item {Proveniência:(Do lat. \textunderscore angelus\textunderscore .)}
\end{itemize}
Criatura, de natureza puramente espiritual.
Pessôa muito virtuosa, innocente.
Criança.
\textunderscore Papos de anjo\textunderscore , doce de ovos.
\textunderscore Anjo papudo\textunderscore , anjo falso, anjo ridículo.
\textunderscore Anjo patudo\textunderscore , o mesmo que \textunderscore anjo papudo\textunderscore . Cf. B. Pereira, \textunderscore Prosodia\textunderscore , vb. \textunderscore pseudangelus\textunderscore .
\section{Annaco}
\begin{itemize}
\item {Grp. gram.:m.  e  adj.}
\end{itemize}
\begin{itemize}
\item {Utilização:Prov.}
\end{itemize}
\begin{itemize}
\item {Utilização:alent.}
\end{itemize}
Animal de um anno.
Exclusivamente, o bode de um anno.
\section{Annaes}
\begin{itemize}
\item {Grp. gram.:m. pl.}
\end{itemize}
\begin{itemize}
\item {Proveniência:(Lat. \textunderscore annales\textunderscore )}
\end{itemize}
História ou narração organizada anno por anno.
Publicação periódica de sciências, letras ou artes.
História.
\section{Annais}
\begin{itemize}
\item {Grp. gram.:m. pl.}
\end{itemize}
\begin{itemize}
\item {Proveniência:(Lat. \textunderscore annales\textunderscore )}
\end{itemize}
História ou narração organizada anno por anno.
Publicação periódica de sciências, letras ou artes.
História.
\section{Annal}
\begin{itemize}
\item {Grp. gram.:adj.}
\end{itemize}
\begin{itemize}
\item {Grp. gram.:M.}
\end{itemize}
O mesmo que \textunderscore annual\textunderscore .
Ceremónia religiosa, que se celebra todos os dias, durante um anno. Cf. G. Vicente, I, 254.
\section{Annalista}
\begin{itemize}
\item {Grp. gram.:m.}
\end{itemize}
\begin{itemize}
\item {Proveniência:(Do lat. \textunderscore annales\textunderscore )}
\end{itemize}
Aquelle que escreve annaes.
\section{Anna-pinta}
\begin{itemize}
\item {Grp. gram.:f.}
\end{itemize}
\begin{itemize}
\item {Utilização:Bras}
\end{itemize}
O mesmo que \textunderscore capitão-do-mato\textunderscore .
\section{Annásia}
\begin{itemize}
\item {Grp. gram.:f.}
\end{itemize}
\begin{itemize}
\item {Utilização:Ant.}
\end{itemize}
O mesmo que \textunderscore annata\textunderscore .
\section{Annata}
\begin{itemize}
\item {Grp. gram.:f.}
\end{itemize}
\begin{itemize}
\item {Utilização:Ant.}
\end{itemize}
Taxa, que pagavam á autoridade ecclesiástica os que recebiam um benefício, sendo a mesma taxa calculada pela renda de um anno dêsse benefício.
(B. lat. \textunderscore annata\textunderscore )
\section{Annatista}
\begin{itemize}
\item {Grp. gram.:m.}
\end{itemize}
Official, que tinha a seu cargo os livros e despachos relativos ás annatas.
\section{Anna-velha}
\begin{itemize}
\item {Grp. gram.:f.}
\end{itemize}
Pequena ave pernalta do Brasil.
\section{Anneiro}
\begin{itemize}
\item {Grp. gram.:adj.}
\end{itemize}
\begin{itemize}
\item {Proveniência:(De \textunderscore anno\textunderscore )}
\end{itemize}
Dependente da maneira como correr o anno.
Contingente, incerto.
Diz-se das árvores, que num anno produzem muito e no seguinte nada.
\section{Annejo}
\begin{itemize}
\item {Grp. gram.:adj.}
\end{itemize}
O mesmo que \textunderscore annelho\textunderscore .
\section{Annelho}
\begin{itemize}
\item {Grp. gram.:adj.}
\end{itemize}
O mesmo que \textunderscore annaco\textunderscore .
\section{Annequim}
\begin{itemize}
\item {Grp. gram.:m.}
\end{itemize}
\begin{itemize}
\item {Proveniência:(De \textunderscore Annequim\textunderscore , n. p.)}
\end{itemize}
Peixe plagióstomo, pardo-anegrado.
Bôbo do paço, em tempo de D. Fernando I.
\section{Annesa}
\begin{itemize}
\item {Grp. gram.:f.}
\end{itemize}
\begin{itemize}
\item {Utilização:Prov.}
\end{itemize}
\begin{itemize}
\item {Utilização:minh.}
\end{itemize}
Anno agrícola ou económico.
\section{Annexação}
\begin{itemize}
\item {fónica:csa}
\end{itemize}
\begin{itemize}
\item {Grp. gram.:f.}
\end{itemize}
Acto de \textunderscore annexar\textunderscore .
\section{Annexar}
\begin{itemize}
\item {fónica:csa}
\end{itemize}
\begin{itemize}
\item {Grp. gram.:v. t.}
\end{itemize}
\begin{itemize}
\item {Proveniência:(Lat. \textunderscore annexare\textunderscore )}
\end{itemize}
Juntar, ligar.
\section{Annexidade}
\begin{itemize}
\item {fónica:csi}
\end{itemize}
\begin{itemize}
\item {Grp. gram.:f.}
\end{itemize}
\begin{itemize}
\item {Utilização:Ant.}
\end{itemize}
O mesmo que \textunderscore annexo\textunderscore , m.
\section{Annexionismo}
\begin{itemize}
\item {fónica:csi}
\end{itemize}
\begin{itemize}
\item {Grp. gram.:m.}
\end{itemize}
\begin{itemize}
\item {Proveniência:(Do lat. \textunderscore annexio\textunderscore )}
\end{itemize}
Theoria, segundo a qual se pretende que os pequenos Estados devem reunir-se aos grandes, seus vizinhos, sob pretexto de affinidade de raça, língua, etc.
\section{Annexionista}
\begin{itemize}
\item {fónica:csi}
\end{itemize}
\begin{itemize}
\item {Grp. gram.:m.}
\end{itemize}
Partidário de annexação de um país, de um territorio, a outro.
\section{Annexo}
\begin{itemize}
\item {Grp. gram.:m.}
\end{itemize}
\begin{itemize}
\item {Grp. gram.:Adj.}
\end{itemize}
\begin{itemize}
\item {Proveniência:(Lat. \textunderscore annexus\textunderscore )}
\end{itemize}
Aquillo que está ligado, como accessório.
Dependência.
Ligado, junto.
Encorporado.
Sujeito.
\section{Annício}
\begin{itemize}
\item {Grp. gram.:m.}
\end{itemize}
\begin{itemize}
\item {Utilização:Ant.}
\end{itemize}
Título ou instrumento público, em que se referiam as circunstâncias que haviam precedido o mesmo título.
\section{Annihilar}
\begin{itemize}
\item {Proveniência:(Lat. \textunderscore annihilare\textunderscore )}
\end{itemize}
\textunderscore v. t.\textunderscore  (e der.)
O mesmo que \textunderscore aniquilar\textunderscore , etc.
\section{Anninho}
\begin{itemize}
\item {Grp. gram.:m.}
\end{itemize}
\begin{itemize}
\item {Utilização:Prov.}
\end{itemize}
\begin{itemize}
\item {Utilização:alent.}
\end{itemize}
Lan de carneiro ou de ovelha de um anno.
\section{Anniversariamente}
\begin{itemize}
\item {Grp. gram.:adv.}
\end{itemize}
\begin{itemize}
\item {Proveniência:(De \textunderscore anniversario\textunderscore )}
\end{itemize}
Em dia certo de cada anno.
\section{Anniversariante}
\begin{itemize}
\item {Grp. gram.:adj.}
\end{itemize}
\begin{itemize}
\item {Utilização:bras}
\end{itemize}
\begin{itemize}
\item {Utilização:Neol.}
\end{itemize}
Que anniversaria.
\section{Anniversariar}
\begin{itemize}
\item {Grp. gram.:v. i.}
\end{itemize}
\begin{itemize}
\item {Utilização:bras}
\end{itemize}
\begin{itemize}
\item {Utilização:Neol.}
\end{itemize}
Fazer anniversário, celebrar anniversário.
\section{Anniversário}
\begin{itemize}
\item {Grp. gram.:m.}
\end{itemize}
\begin{itemize}
\item {Grp. gram.:Adj.}
\end{itemize}
\begin{itemize}
\item {Proveniência:(Lat. \textunderscore anniversarius\textunderscore )}
\end{itemize}
Volta annual do dia, em que se deu certo acontecimento.
Diz-se do dia, em que se conta um ou mais annos sôbre a data de um acontecimento.
\section{Annixo}
\begin{itemize}
\item {fónica:cso}
\end{itemize}
\begin{itemize}
\item {Grp. gram.:m.}
\end{itemize}
\begin{itemize}
\item {Utilização:Ant.}
\end{itemize}
\begin{itemize}
\item {Proveniência:(Lat. \textunderscore annixus\textunderscore )}
\end{itemize}
Gancho de ferro, preso na extremidade de um pau ou vara.
\section{Anno}
\begin{itemize}
\item {Grp. gram.:m.}
\end{itemize}
\begin{itemize}
\item {Grp. gram.:M. pl.}
\end{itemize}
\begin{itemize}
\item {Proveniência:(Lat. \textunderscore annus\textunderscore )}
\end{itemize}
Tempo, que a Terra gasta numa translação completa á volta do Sol. Espaço de doze meses, que começa em 1 de Janeiro e termina em 31 de Dezembro.
Espaço de doze meses, a começar em qualquer dia.
Idades da vida.
\section{Annojal}
\begin{itemize}
\item {Grp. gram.:adj.}
\end{itemize}
\begin{itemize}
\item {Utilização:Ant.}
\end{itemize}
\begin{itemize}
\item {Proveniência:(De \textunderscore annojo\textunderscore )}
\end{itemize}
Dizia-se do leite de vaca que parira um anno antes; grosso.
\section{Annojo}
\begin{itemize}
\item {Grp. gram.:adj.}
\end{itemize}
\begin{itemize}
\item {Utilização:Prov.}
\end{itemize}
\begin{itemize}
\item {Utilização:trasm.}
\end{itemize}
O mesmo que \textunderscore annaco\textunderscore .
Exclusivamente, o boi de um anno.
\section{Annona}
\begin{itemize}
\item {Grp. gram.:f.}
\end{itemize}
\begin{itemize}
\item {Utilização:Ant.}
\end{itemize}
\begin{itemize}
\item {Proveniência:(Lat. \textunderscore annona\textunderscore )}
\end{itemize}
Provisão de mantimentos.
Colheita dos frutos de um anno.
\section{Annonário}
\begin{itemize}
\item {Grp. gram.:adj.}
\end{itemize}
\begin{itemize}
\item {Proveniência:(Lat. \textunderscore annonarius\textunderscore )}
\end{itemize}
Dizia-se de uma lei romana, relativa a provisões que obstassem á carestia de mantimentos.
E dizia-se das províncias, que pagavam as suas contribuições em cereaes.
\section{Annosidade}
\begin{itemize}
\item {Grp. gram.:f.}
\end{itemize}
Qualidade de \textunderscore annoso\textunderscore .
\section{Annoso}
\begin{itemize}
\item {Grp. gram.:adj.}
\end{itemize}
\begin{itemize}
\item {Proveniência:(Lat. \textunderscore annosus\textunderscore )}
\end{itemize}
Que tem muitos annos: \textunderscore um carvalho annoso\textunderscore .
\section{Annotação}
\begin{itemize}
\item {Grp. gram.:f.}
\end{itemize}
\begin{itemize}
\item {Proveniência:(Lat. \textunderscore annotatio\textunderscore )}
\end{itemize}
Acto ou effeito de annotar.
\section{Annotador}
\begin{itemize}
\item {Proveniência:(Lat. \textunderscore annotator\textunderscore )}
\end{itemize}
Aquelle que annota.
\section{Annotar}
\begin{itemize}
\item {Grp. gram.:v. t.}
\end{itemize}
\begin{itemize}
\item {Proveniência:(Lat. \textunderscore annotare\textunderscore )}
\end{itemize}
Fazer notas a.
Esclarecer com commentários.
\section{Annotino}
\begin{itemize}
\item {Grp. gram.:adj.}
\end{itemize}
\begin{itemize}
\item {Utilização:Des.}
\end{itemize}
\begin{itemize}
\item {Proveniência:(Lat. \textunderscore annotinus\textunderscore )}
\end{itemize}
O mesmo que \textunderscore annual\textunderscore .
\section{Ânnua}
\begin{itemize}
\item {Grp. gram.:f.}
\end{itemize}
\begin{itemize}
\item {Utilização:Ant.}
\end{itemize}
\begin{itemize}
\item {Proveniência:(De \textunderscore ânnuo\textunderscore )}
\end{itemize}
Chamavam-se assim as cartas, em que se referiam os successos de um anno:«\textunderscore em uma ânnua da Companhia de Jesus, se refere...\textunderscore »\textunderscore Luz e Calor\textunderscore , 256.
\section{Annual}
\begin{itemize}
\item {Grp. gram.:adj.}
\end{itemize}
\begin{itemize}
\item {Proveniência:(Lat. \textunderscore annualis\textunderscore )}
\end{itemize}
Que dura um anno.
Que succede uma vez por anno.
Quantia, que se paga annualmente; prestação annual.
\section{Annualidade}
\begin{itemize}
\item {Grp. gram.:f.}
\end{itemize}
Qualidade do que é annual.
Pagamento, que se faz todos os annos.
\section{Annualmente}
\begin{itemize}
\item {Grp. gram.:adv.}
\end{itemize}
De modo annual.
Todos os annos.
De anno a anno.
\section{Annuário}
\begin{itemize}
\item {Grp. gram.:m.}
\end{itemize}
\begin{itemize}
\item {Proveniência:(De \textunderscore anno\textunderscore )}
\end{itemize}
Publicação annual.
\section{Annuência}
\begin{itemize}
\item {Grp. gram.:f.}
\end{itemize}
\begin{itemize}
\item {Proveniência:(De \textunderscore annuente\textunderscore )}
\end{itemize}
Acto de annuir.
\section{Annuente}
\begin{itemize}
\item {Grp. gram.:adj.}
\end{itemize}
\begin{itemize}
\item {Proveniência:(Lat. \textunderscore annuens\textunderscore )}
\end{itemize}
Que annue.
\section{Annuidade}
\begin{itemize}
\item {fónica:nu-i}
\end{itemize}
\begin{itemize}
\item {Grp. gram.:f.}
\end{itemize}
\begin{itemize}
\item {Proveniência:(De \textunderscore annuo\textunderscore )}
\end{itemize}
Annualidade.
Quantia, que, paga annualmente, abrange amortização e juro.
\section{Annuir}
\begin{itemize}
\item {Grp. gram.:v. i.}
\end{itemize}
\begin{itemize}
\item {Proveniência:(Lat. \textunderscore annuere\textunderscore )}
\end{itemize}
Dar consentimento.
Condescender.
Estar de acôrdo.
\section{Annuitário}
\begin{itemize}
\item {fónica:nu-i}
\end{itemize}
\begin{itemize}
\item {Grp. gram.:adj.}
\end{itemize}
Que se amortiza por annuidade.
\section{Annullabilidade}
\begin{itemize}
\item {Grp. gram.:f.}
\end{itemize}
Qualidade de annullável.
\section{Annullação}
\begin{itemize}
\item {Grp. gram.:f.}
\end{itemize}
Acto de \textunderscore annullar\textunderscore .
\section{Annullador}
\begin{itemize}
\item {Grp. gram.:m.}
\end{itemize}
Aquelle que annulla.
\section{Annullante}
\begin{itemize}
\item {Grp. gram.:adj.}
\end{itemize}
\begin{itemize}
\item {Proveniência:(Lat. \textunderscore annullans\textunderscore )}
\end{itemize}
Que annulla.
\section{Annullar}
\begin{itemize}
\item {Grp. gram.:v. t.}
\end{itemize}
\begin{itemize}
\item {Proveniência:(Lat. \textunderscore annullare\textunderscore )}
\end{itemize}
Tornar nullo.
Abolir.
Invalidar: \textunderscore o Ministro annullou um despacho do seu antecessor\textunderscore .
Aniquilar.
\section{Annullativo}
\begin{itemize}
\item {Grp. gram.:adj.}
\end{itemize}
Que annulla.
\section{Annullatório}
\begin{itemize}
\item {Grp. gram.:adj.}
\end{itemize}
Que tem fôrça para annullar.
\section{Annullável}
\begin{itemize}
\item {Grp. gram.:adj.}
\end{itemize}
Que póde sêr annullado.
\section{Annumeração}
\begin{itemize}
\item {Grp. gram.:f.}
\end{itemize}
\begin{itemize}
\item {Proveniência:(Lat. \textunderscore annumeratio\textunderscore )}
\end{itemize}
Acto de annumerar.
\section{Annumerar}
\begin{itemize}
\item {Grp. gram.:v. t.}
\end{itemize}
\begin{itemize}
\item {Utilização:Ant.}
\end{itemize}
\begin{itemize}
\item {Proveniência:(Lat. \textunderscore annumerare\textunderscore )}
\end{itemize}
Addicionar.
Numerar.
\section{Annunciação}
\begin{itemize}
\item {Grp. gram.:f.}
\end{itemize}
\begin{itemize}
\item {Proveniência:(Lat. \textunderscore annunciatio\textunderscore )}
\end{itemize}
Acto de annunciar.
\section{Annunciada}
\begin{itemize}
\item {Grp. gram.:f.}
\end{itemize}
O mesmo que \textunderscore annunciação\textunderscore .
\section{Annunciador}
\begin{itemize}
\item {Grp. gram.:m.}
\end{itemize}
\begin{itemize}
\item {Proveniência:(Lat. \textunderscore annunciator\textunderscore )}
\end{itemize}
Aquelle que annuncía.
\section{Annunciante}
\begin{itemize}
\item {Grp. gram.:m.  e  adj.}
\end{itemize}
\begin{itemize}
\item {Proveniência:(Lat. \textunderscore annuncians\textunderscore )}
\end{itemize}
O que annuncía.
O que manda annúncios para os periódicos.
\section{Annunciar}
\begin{itemize}
\item {Grp. gram.:v. t.}
\end{itemize}
\begin{itemize}
\item {Proveniência:(Lat. \textunderscore annunciare\textunderscore )}
\end{itemize}
Dar notícia de: \textunderscore foi annunciar ao pai o resultado dos seus exames\textunderscore .
Publicar.
Predizer, presagiar: \textunderscore aquellas nuvens annunciam trovoada\textunderscore .
Fazer conhecer por annúncio: \textunderscore annunciar um leilão\textunderscore .
Manifestar.
Revelar: \textunderscore o rapaz annuncía as melhores aptidões\textunderscore .
Prevenir da presença ou da chegada de: \textunderscore o criado foi annunciar o médico\textunderscore .
\section{Annunciativo}
\begin{itemize}
\item {Grp. gram.:adj.}
\end{itemize}
Que annuncia, que contém annúncio.
\section{Annúncio}
\begin{itemize}
\item {Grp. gram.:m.}
\end{itemize}
\begin{itemize}
\item {Proveniência:(Lat. \textunderscore annuncius\textunderscore )}
\end{itemize}
Aviso, que torna conhecido um facto que se suppunha ignorado.
Aviso público: \textunderscore vi hoje um annúncio nos jornaes\textunderscore .
Prognóstico, preságio.
\section{Ânnuo}
\begin{itemize}
\item {Grp. gram.:adj.}
\end{itemize}
\begin{itemize}
\item {Proveniência:(Lat. \textunderscore annuus\textunderscore )}
\end{itemize}
O mesmo que \textunderscore annual\textunderscore .
\section{...ano}
\begin{itemize}
\item {Grp. gram.:suf. adj.}
\end{itemize}
(indic. de pertença, origem, etc.)
\section{Ano}
\begin{itemize}
\item {Grp. gram.:m.}
\end{itemize}
\begin{itemize}
\item {Grp. gram.:M. pl.}
\end{itemize}
\begin{itemize}
\item {Proveniência:(Lat. \textunderscore annus\textunderscore )}
\end{itemize}
Tempo, que a Terra gasta numa translação completa á volta do Sol.
Espaço de doze meses, que começa em 1 de Janeiro e termina em 31 de Dezembro.
Espaço de doze meses, a começar em qualquer dia.
Idades da vida.
\section{Ano}
\begin{itemize}
\item {Grp. gram.:m.}
\end{itemize}
\begin{itemize}
\item {Proveniência:(Lat. \textunderscore anus\textunderscore )}
\end{itemize}
O mesmo que \textunderscore ânus\textunderscore . Cf. Moraes.
Abertura, por onde o intestino recto expelle os excrementos.
\section{Ânoda}
\begin{itemize}
\item {Grp. gram.:f.}
\end{itemize}
Gênero de plantas malváceas.
\section{Anodia}
\begin{itemize}
\item {Grp. gram.:f.}
\end{itemize}
\begin{itemize}
\item {Utilização:Med.}
\end{itemize}
\begin{itemize}
\item {Proveniência:(Do gr. \textunderscore an\textunderscore  priv. + \textunderscore ode\textunderscore )}
\end{itemize}
Mania, que consiste no falar desacertado e indecoroso.
\section{Anodinia}
\begin{itemize}
\item {Grp. gram.:f.}
\end{itemize}
\begin{itemize}
\item {Proveniência:(De \textunderscore anódyno\textunderscore )}
\end{itemize}
Ausência de dores.
\section{Anódino}
\begin{itemize}
\item {Grp. gram.:adj.}
\end{itemize}
\begin{itemize}
\item {Proveniência:(Gr. \textunderscore anodunos\textunderscore )}
\end{itemize}
Que faz cessar dores.
Inoffensivo.
Sem importância; secundário.
\section{Anódio}
\begin{itemize}
\item {Grp. gram.:m.  e  adj.}
\end{itemize}
\begin{itemize}
\item {Utilização:Phýs.}
\end{itemize}
Diz-se do electródio positivo.
(Cp. \textunderscore electródio\textunderscore )
\section{Anodo}
\begin{itemize}
\item {Grp. gram.:m.}
\end{itemize}
(Fórma preferível a \textunderscore anódio\textunderscore . Cp. \textunderscore electrodo\textunderscore )
\section{Anodoncia}
\begin{itemize}
\item {Grp. gram.:f.}
\end{itemize}
\begin{itemize}
\item {Proveniência:(Do gr. \textunderscore an\textunderscore  + \textunderscore odous\textunderscore , \textunderscore odontos\textunderscore )}
\end{itemize}
Falta completa de dentes.
\section{Anodôncio}
\begin{itemize}
\item {Grp. gram.:m.}
\end{itemize}
Gênero de musgos.
\section{Anodontes}
\begin{itemize}
\item {Grp. gram.:m. pl.}
\end{itemize}
\begin{itemize}
\item {Proveniência:(Do gr. \textunderscore an\textunderscore  + \textunderscore odous\textunderscore , \textunderscore odontos\textunderscore )}
\end{itemize}
Molluscos de água doce, caracterizados por não terem dentes na charneira das conchas.
\section{Anodôntico}
\begin{itemize}
\item {Grp. gram.:adj.}
\end{itemize}
Relativo aos anodontes.
\section{Anodynia}
\begin{itemize}
\item {Grp. gram.:f.}
\end{itemize}
\begin{itemize}
\item {Proveniência:(De \textunderscore anódyno\textunderscore )}
\end{itemize}
Ausência de dores.
\section{Anódyno}
\begin{itemize}
\item {Grp. gram.:adj.}
\end{itemize}
\begin{itemize}
\item {Proveniência:(Gr. \textunderscore anodunos\textunderscore )}
\end{itemize}
Que faz cessar dores.
Inoffensivo.
Sem importância; secundário.
\section{Anoema}
\begin{itemize}
\item {fónica:no-ê}
\end{itemize}
\begin{itemize}
\item {Grp. gram.:m.}
\end{itemize}
O porco da Índia, em a nomenclatura de Cuvier.
\section{Anofelíneos}
\begin{itemize}
\item {Grp. gram.:m. pl.}
\end{itemize}
\begin{itemize}
\item {Proveniência:(De \textunderscore anophele\textunderscore )}
\end{itemize}
Insectos nemotóceros.
\section{Anófogo}
\begin{itemize}
\item {Grp. gram.:m.}
\end{itemize}
Gênero de reptis sáurios.
\section{Anoftalmia}
\begin{itemize}
\item {Grp. gram.:f.}
\end{itemize}
\begin{itemize}
\item {Proveniência:(Do gr. \textunderscore an\textunderscore  + \textunderscore ophtalmos\textunderscore )}
\end{itemize}
Privação do apparelho ocular.
\section{Anógina}
\begin{itemize}
\item {Grp. gram.:f.}
\end{itemize}
Gênero de plantas cyperáceas.
\section{Anogueira}
\begin{itemize}
\item {Grp. gram.:f.}
\end{itemize}
\begin{itemize}
\item {Utilização:Prov.}
\end{itemize}
\begin{itemize}
\item {Utilização:alg.}
\end{itemize}
O mesmo que \textunderscore nogueira\textunderscore .
\section{Anogueirado}
\begin{itemize}
\item {Grp. gram.:adj.}
\end{itemize}
Que tem côr de nogueira.
\section{Anógyna}
\begin{itemize}
\item {Grp. gram.:f.}
\end{itemize}
Gênero de plantas cyperáceas.
\section{Anoitar}
\begin{itemize}
\item {Grp. gram.:v. i.}
\end{itemize}
O mesmo que \textunderscore ennoitar\textunderscore .
\section{Anoitecer}
\begin{itemize}
\item {Grp. gram.:v. i.}
\end{itemize}
Ir chegando a noite.
Cair a noite; fazer-se noite.
Escurecer.
\section{Anoitecido}
\begin{itemize}
\item {Grp. gram.:adj.}
\end{itemize}
\begin{itemize}
\item {Proveniência:(De \textunderscore anoitecer\textunderscore )}
\end{itemize}
Em que se fez noite.
Escurecido.
\section{Anojadiço}
\begin{itemize}
\item {Grp. gram.:adj.}
\end{itemize}
Que facilmente se anoja.
\section{Anojado}
\begin{itemize}
\item {Grp. gram.:adj.}
\end{itemize}
Que tem nojo ou tédio.
Desgostoso.
\section{Anojador}
\begin{itemize}
\item {Grp. gram.:adj.}
\end{itemize}
Que anoja.
\section{Anojal}
\begin{itemize}
\item {Grp. gram.:adj.}
\end{itemize}
\begin{itemize}
\item {Utilização:Ant.}
\end{itemize}
\begin{itemize}
\item {Proveniência:(De \textunderscore annojo\textunderscore )}
\end{itemize}
Dizia-se do leite de vaca que parira um anno antes; grosso.
\section{Anojamento}
\begin{itemize}
\item {Grp. gram.:m.}
\end{itemize}
Acto de anojar.
Estado de nojo.
\section{Anojar}
\begin{itemize}
\item {Grp. gram.:v. t.}
\end{itemize}
Causar nojo ou tédio a.
Desgostar.
Enjoar.
Enlutar.
\section{Anojo}
\begin{itemize}
\item {Grp. gram.:m.}
\end{itemize}
O mesmo que \textunderscore anojamento\textunderscore .
\section{Anojo}
\begin{itemize}
\item {Grp. gram.:adj.}
\end{itemize}
\begin{itemize}
\item {Utilização:Prov.}
\end{itemize}
\begin{itemize}
\item {Utilização:trasm.}
\end{itemize}
O mesmo que \textunderscore anaco\textunderscore ^2.
Exclusivamente, o boi de um ano.
\section{Anojoso}
\begin{itemize}
\item {Grp. gram.:adj.}
\end{itemize}
Que anoja.
\section{Anoleno}
\begin{itemize}
\item {Grp. gram.:adj.}
\end{itemize}
\begin{itemize}
\item {Grp. gram.:M. pl.}
\end{itemize}
\begin{itemize}
\item {Proveniência:(Do gr. \textunderscore an\textunderscore  + \textunderscore olene\textunderscore )}
\end{itemize}
Que não tem braços.
Molluscos acéphalos sem braços.
\section{Anólis}
\begin{itemize}
\item {Grp. gram.:m.}
\end{itemize}
Reptil das Antilhas.
\section{Anomalão}
\begin{itemize}
\item {Grp. gram.:m.}
\end{itemize}
Gênero de hymenópteros.
\section{Anómalas}
\begin{itemize}
\item {Grp. gram.:f. pl.}
\end{itemize}
\begin{itemize}
\item {Proveniência:(De \textunderscore anómalo\textunderscore )}
\end{itemize}
Plantas herbáceas, de corolla polypétala e irregular, segundo Tournefort.
\section{Anomalia}
\begin{itemize}
\item {Grp. gram.:f.}
\end{itemize}
\begin{itemize}
\item {Proveniência:(Gr. \textunderscore anomalia\textunderscore )}
\end{itemize}
Estado ou qualidade de anómalo; irregularidade; anormalidade.
Excepção á regra.
Aberração.
Desigualdade.
\section{Anomalifloro}
\begin{itemize}
\item {Grp. gram.:adj.}
\end{itemize}
\begin{itemize}
\item {Proveniência:(De \textunderscore anómalo\textunderscore  + \textunderscore flôr\textunderscore )}
\end{itemize}
Que tem flôres de corolla anómala.
\section{Anomalípede}
\begin{itemize}
\item {Grp. gram.:adj.}
\end{itemize}
\begin{itemize}
\item {Proveniência:(De \textunderscore anómalo\textunderscore  + lat. \textunderscore pes\textunderscore , \textunderscore pedis\textunderscore )}
\end{itemize}
Diz-se do animal, cujas patas são desiguaes.
\section{Anomalístico}
\begin{itemize}
\item {Grp. gram.:adj.}
\end{itemize}
\begin{itemize}
\item {Utilização:Astron.}
\end{itemize}
\begin{itemize}
\item {Proveniência:(De \textunderscore anomalia\textunderscore )}
\end{itemize}
Diz-se do tempo, que a Terra gasta entre duas passagens successivas pelo mesmo ponto da sua órbita.
\section{Anómalo}
\begin{itemize}
\item {Grp. gram.:adj.}
\end{itemize}
\begin{itemize}
\item {Proveniência:(Gr. \textunderscore anomalos\textunderscore )}
\end{itemize}
Irregular.
Opposto á ordem natural.
Anormal.
\section{Anomalóporo}
\begin{itemize}
\item {Grp. gram.:adj.}
\end{itemize}
\begin{itemize}
\item {Proveniência:(De \textunderscore anómalo\textunderscore  + \textunderscore poro\textunderscore )}
\end{itemize}
Que tem poros de differentes tamanhos.
\section{Anomaluro}
\begin{itemize}
\item {Grp. gram.:m.}
\end{itemize}
\begin{itemize}
\item {Proveniência:(Do gr. \textunderscore anomalos\textunderscore  + \textunderscore oura\textunderscore )}
\end{itemize}
Mammífero roedor, de cauda anómala, encontrado em Fernando-Pó.
\section{Anomanthódia}
\begin{itemize}
\item {Grp. gram.:f.}
\end{itemize}
Gênero de plantas rubiáceas.
\section{Anomantódia}
\begin{itemize}
\item {Grp. gram.:f.}
\end{itemize}
Gênero de plantas rubiáceas.
\section{Anomateca}
\begin{itemize}
\item {Grp. gram.:f.}
\end{itemize}
\begin{itemize}
\item {Proveniência:(Do gr. \textunderscore anomos\textunderscore  + \textunderscore atheka\textunderscore )}
\end{itemize}
Gênero de plantas irídeas.
\section{Anomatheca}
\begin{itemize}
\item {Grp. gram.:f.}
\end{itemize}
\begin{itemize}
\item {Proveniência:(Do gr. \textunderscore anomos\textunderscore  + \textunderscore atheka\textunderscore )}
\end{itemize}
Gênero de plantas irídeas.
\section{Anomaza}
\begin{itemize}
\item {Grp. gram.:f.}
\end{itemize}
Gênero de plantas irídeas.
\section{Anómea}
\begin{itemize}
\item {Grp. gram.:f.}
\end{itemize}
Gênero de plantas leguminosas.
\section{Anómia}
\begin{itemize}
\item {Grp. gram.:f.}
\end{itemize}
\begin{itemize}
\item {Proveniência:(Do gr. \textunderscore anomos\textunderscore )}
\end{itemize}
Mollusco bivalve, de concha irregular.
\section{Anomial}
\begin{itemize}
\item {Grp. gram.:adj.}
\end{itemize}
Semelhante á anómia.
\section{Anomianos}
\begin{itemize}
\item {Grp. gram.:m. pl.}
\end{itemize}
\begin{itemize}
\item {Proveniência:(Do gr. \textunderscore a\textunderscore  priv. + \textunderscore nomos\textunderscore )}
\end{itemize}
Herejes, que rejeitavam toda e qualquer lei.
\section{Anominação}
\begin{itemize}
\item {Grp. gram.:f.}
\end{itemize}
\begin{itemize}
\item {Proveniência:(Do lat. \textunderscore ad\textunderscore  + \textunderscore nominatio\textunderscore )}
\end{itemize}
Alteração de uma palavra, para lhe alterar o sentido: \textunderscore relatório\textunderscore , \textunderscore ralatório\textunderscore ; \textunderscore parlamento\textunderscore , \textunderscore palramento\textunderscore .
\section{Anomita}
\begin{itemize}
\item {Grp. gram.:f.}
\end{itemize}
Anómia fóssil.
Mica ferro-magnésica, de côr verde.
\section{Anomite}
\begin{itemize}
\item {Grp. gram.:f.}
\end{itemize}
Anómia fóssil.
\section{Anomocardiostenia}
\begin{itemize}
\item {Grp. gram.:f.}
\end{itemize}
\begin{itemize}
\item {Utilização:Med.}
\end{itemize}
Contracções ou palpitações irregulares do coração.
\section{Anomocardiosthenia}
\begin{itemize}
\item {Grp. gram.:f.}
\end{itemize}
\begin{itemize}
\item {Utilização:Med.}
\end{itemize}
Contracções ou palpitações irregulares do coração.
\section{Anomocarpo}
\begin{itemize}
\item {Grp. gram.:adj.}
\end{itemize}
\begin{itemize}
\item {Proveniência:(Do gr. \textunderscore anomos\textunderscore  + \textunderscore karpos\textunderscore )}
\end{itemize}
Que tem frutos irregulares.
\section{Anomocéfalo}
\begin{itemize}
\item {Grp. gram.:adj.}
\end{itemize}
\begin{itemize}
\item {Proveniência:(Do gr. \textunderscore a\textunderscore  priv. + \textunderscore nomos\textunderscore  + \textunderscore kephale\textunderscore )}
\end{itemize}
Que tem cabeça irregular.
\section{Anomocéphalo}
\begin{itemize}
\item {Grp. gram.:adj.}
\end{itemize}
\begin{itemize}
\item {Proveniência:(Do gr. \textunderscore a\textunderscore  priv. + \textunderscore nomos\textunderscore  + \textunderscore kephale\textunderscore )}
\end{itemize}
Que tem cabeça irregular.
\section{Anomochlôa}
\begin{itemize}
\item {Grp. gram.:f.}
\end{itemize}
Gênero de plantas gramíneas.
\section{Anomoclôa}
\begin{itemize}
\item {Grp. gram.:f.}
\end{itemize}
Gênero de plantas gramíneas.
\section{Anomodontes}
\begin{itemize}
\item {Grp. gram.:m. pl.}
\end{itemize}
\begin{itemize}
\item {Utilização:Zool.}
\end{itemize}
\begin{itemize}
\item {Proveniência:(Do gr. \textunderscore anomos\textunderscore  + \textunderscore odous\textunderscore , \textunderscore odontos\textunderscore )}
\end{itemize}
Vertebrados fósseis, da ordem dos sáurios, de maxillas sem dentes, ou só com dois grandes dentes sem raizes, ou com dentes cónicos em ambas as maxillas, ou com outras anomalias.
\section{Anômphalo}
\begin{itemize}
\item {Grp. gram.:adj.}
\end{itemize}
Que não tem umbigo.
\section{Anomuro}
\begin{itemize}
\item {Grp. gram.:adj.}
\end{itemize}
\begin{itemize}
\item {Utilização:Zool.}
\end{itemize}
Que tem cauda extraordinária ou anómala.
\section{Anona}
\begin{itemize}
\item {Grp. gram.:f.}
\end{itemize}
Gênero de plantas, que serve de typo á família das anonáceas.
Fruto da planta do mesmo nome.
\section{Anona}
\begin{itemize}
\item {Grp. gram.:f.}
\end{itemize}
\begin{itemize}
\item {Utilização:Ant.}
\end{itemize}
\begin{itemize}
\item {Proveniência:(Lat. \textunderscore annona\textunderscore )}
\end{itemize}
Provisão de mantimentos.
Colheita dos frutos de um ano.
\section{Anonáceas}
\begin{itemize}
\item {Grp. gram.:f. pl.}
\end{itemize}
\begin{itemize}
\item {Proveniência:(De \textunderscore anonáceo\textunderscore )}
\end{itemize}
Família de plantas dicotyledóneas.
\section{Anonáceo}
\begin{itemize}
\item {Grp. gram.:adj.}
\end{itemize}
Relativo ou semelhante á anona.
\section{Anonário}
\begin{itemize}
\item {Grp. gram.:adj.}
\end{itemize}
\begin{itemize}
\item {Proveniência:(Lat. \textunderscore annonarius\textunderscore )}
\end{itemize}
Dizia-se de uma lei romana, relativa a provisões que obstassem á carestia de mantimentos.
E dizia-se das províncias, que pagavam as suas contribuições em cereaes.
\section{Anônfalo}
\begin{itemize}
\item {Grp. gram.:adj.}
\end{itemize}
Que não tem umbigo.
\section{Anonimado}
\begin{itemize}
\item {Grp. gram.:m.}
\end{itemize}
Qualidade que é anónimo.
Systema de escrever anonimamente.
\section{Anonimamente}
\begin{itemize}
\item {Grp. gram.:adv.}
\end{itemize}
De modo \textunderscore anónimo\textunderscore .
\section{Anonimato}
\begin{itemize}
\item {Grp. gram.:m.}
\end{itemize}
O mesmo que \textunderscore anonimado\textunderscore .
\section{Anonímia}
\begin{itemize}
\item {Grp. gram.:f.}
\end{itemize}
Qualidade do que é \textunderscore anónimo\textunderscore .
\section{Anónimo}
\begin{itemize}
\item {Grp. gram.:m.}
\end{itemize}
\begin{itemize}
\item {Grp. gram.:Adj.}
\end{itemize}
\begin{itemize}
\item {Proveniência:(Gr. \textunderscore anonumos\textunderscore )}
\end{itemize}
Aquelle que não assina o que escreve.
Que não é assinado; que não declara o nome do autor: \textunderscore artigo anónimo\textunderscore .
E diz-se da sociedade commercial, que não é representada por nome ou firma, mas por um título que designa a sua natureza: \textunderscore Companhia Carris de Ferro\textunderscore .
\section{Anonymado}
\begin{itemize}
\item {Grp. gram.:m.}
\end{itemize}
Qualidade que é anónymo.
Systema de escrever anonymamente.
\section{Anonymamente}
\begin{itemize}
\item {Grp. gram.:adv.}
\end{itemize}
De modo \textunderscore anónymo\textunderscore .
\section{Anonymato}
\begin{itemize}
\item {Grp. gram.:m.}
\end{itemize}
O mesmo que \textunderscore anonymado\textunderscore .
\section{Anonýmia}
\begin{itemize}
\item {Grp. gram.:f.}
\end{itemize}
Qualidade do que é \textunderscore anónymo\textunderscore .
\section{Anónymo}
\begin{itemize}
\item {Grp. gram.:m.}
\end{itemize}
\begin{itemize}
\item {Grp. gram.:Adj.}
\end{itemize}
\begin{itemize}
\item {Proveniência:(Gr. \textunderscore anonumos\textunderscore )}
\end{itemize}
Aquelle que não assina o que escreve.
Que não é assinado; que não declara o nome do autor: \textunderscore artigo anónymo\textunderscore .
E diz-se da sociedade commercial, que não é representada por nome ou firma, mas por um título que designa a sua natureza: \textunderscore Companhia Carris de Ferro\textunderscore .
\section{Anopétalo}
\begin{itemize}
\item {Grp. gram.:adj.}
\end{itemize}
\begin{itemize}
\item {Utilização:Bot.}
\end{itemize}
Que tem pétalas direitas.
\section{Anophelíneos}
\begin{itemize}
\item {Grp. gram.:m. pl.}
\end{itemize}
\begin{itemize}
\item {Proveniência:(De \textunderscore anophele\textunderscore )}
\end{itemize}
Insectos nemotóceros.
\section{Anophele}
\begin{itemize}
\item {Grp. gram.:m.}
\end{itemize}
\begin{itemize}
\item {Proveniência:(Gr. \textunderscore anopheles\textunderscore )}
\end{itemize}
Gênero de mosquitos, que trasm.ttem o sezonismo.
\section{Anóphogo}
\begin{itemize}
\item {Grp. gram.:m.}
\end{itemize}
Gênero de reptis sáurios.
\section{Anophtalmia}
\begin{itemize}
\item {Grp. gram.:f.}
\end{itemize}
\begin{itemize}
\item {Proveniência:(Do gr. \textunderscore an\textunderscore  + \textunderscore ophtalmos\textunderscore )}
\end{itemize}
Privação do apparelho ocular.
\section{Anopistho}
\begin{itemize}
\item {Grp. gram.:adj.}
\end{itemize}
Diz-se dos animaes que não têm extremidade anal.
\section{Anopisto}
\begin{itemize}
\item {Grp. gram.:adj.}
\end{itemize}
Diz-se dos animaes que não têm extremidade anal.
\section{Anoplopo}
\begin{itemize}
\item {Grp. gram.:m.}
\end{itemize}
\begin{itemize}
\item {Proveniência:(Do gr. \textunderscore anoplos\textunderscore  + \textunderscore ops\textunderscore )}
\end{itemize}
Reptil, da fam. dos sáurios.
\section{Anoplos}
\begin{itemize}
\item {Grp. gram.:m. pl.}
\end{itemize}
\begin{itemize}
\item {Proveniência:(Gr. \textunderscore anoplos\textunderscore )}
\end{itemize}
Gênero de insectos coleópteros tetrâmeros.
\section{Anoplista}
\begin{itemize}
\item {Grp. gram.:f.}
\end{itemize}
Gênero de coleópteros.
\section{Anoplódera}
\begin{itemize}
\item {Grp. gram.:f.}
\end{itemize}
Gênero de coleópteros.
\section{Anoplodermo}
\begin{itemize}
\item {Grp. gram.:m.}
\end{itemize}
Gênero de coleópteros.
\section{Anoplóforo}
\begin{itemize}
\item {Grp. gram.:m.}
\end{itemize}
Gênero de coleópteros.
\section{Anoplómero}
\begin{itemize}
\item {Grp. gram.:m.}
\end{itemize}
Gênero do coleópteros.
\section{Anoplóphoro}
\begin{itemize}
\item {Grp. gram.:m.}
\end{itemize}
Gênero de coleópteros.
\section{Anoplostermo}
\begin{itemize}
\item {Grp. gram.:m.}
\end{itemize}
Gênero de coleópteros.
\section{Anoplótero}
\begin{itemize}
\item {Grp. gram.:m.}
\end{itemize}
\begin{itemize}
\item {Proveniência:(Do gr. \textunderscore an\textunderscore  + \textunderscore oplon\textunderscore  + \textunderscore therion\textunderscore )}
\end{itemize}
Gênero de mammíferos fósseis.
\section{Anoplóthero}
\begin{itemize}
\item {Grp. gram.:m.}
\end{itemize}
\begin{itemize}
\item {Proveniência:(Do gr. \textunderscore an\textunderscore  + \textunderscore oplon\textunderscore  + \textunderscore therion\textunderscore )}
\end{itemize}
Gênero de mammíferos fósseis.
\section{Anopsia}
\begin{itemize}
\item {Grp. gram.:f.}
\end{itemize}
\begin{itemize}
\item {Utilização:Des.}
\end{itemize}
\begin{itemize}
\item {Proveniência:(Do gr. \textunderscore an\textunderscore  priv. + \textunderscore ops\textunderscore )}
\end{itemize}
O mesmo que \textunderscore amblyopia\textunderscore .
\section{Anoque}
\begin{itemize}
\item {Grp. gram.:m.}
\end{itemize}
\begin{itemize}
\item {Utilização:Prov.}
\end{itemize}
\begin{itemize}
\item {Utilização:trasm.}
\end{itemize}
\begin{itemize}
\item {Utilização:Bras}
\end{itemize}
\begin{itemize}
\item {Utilização:Prov.}
\end{itemize}
\begin{itemize}
\item {Utilização:trasm.}
\end{itemize}
Vasilha, para curtir coiros.
Selha, em que os sapateiros umedecem a sola.
Apparelho, formado de um coiro quadrado, para fabricação de decoada.
Atoleiro, que as águas do inverno fazem nas ruas.
\section{Anordestear}
\begin{itemize}
\item {Grp. gram.:v. t.}
\end{itemize}
Inclinar (o navio) para nordeste.
\section{Anorexia}
\begin{itemize}
\item {fónica:csi}
\end{itemize}
\begin{itemize}
\item {Grp. gram.:f.}
\end{itemize}
\begin{itemize}
\item {Proveniência:(Do gr. \textunderscore an\textunderscore  + \textunderscore oreksis\textunderscore )}
\end{itemize}
Falta de appetite.
\section{Anorgânico}
\begin{itemize}
\item {Grp. gram.:adj.}
\end{itemize}
O mesmo que \textunderscore inorgânico\textunderscore .
\section{Anorganochímica}
\begin{itemize}
\item {fónica:qui}
\end{itemize}
\begin{itemize}
\item {Grp. gram.:f.}
\end{itemize}
Chímica inorgânica.
\section{Anorganogenia}
\begin{itemize}
\item {Grp. gram.:f.}
\end{itemize}
Tratado da origem dos corpos inorgânicos.
\section{Anorganognosia}
\begin{itemize}
\item {Grp. gram.:f.}
\end{itemize}
\begin{itemize}
\item {Utilização:Hist. Nat.}
\end{itemize}
Conhecimento dos corpos inorgânicos.
\section{Anorganografia}
\begin{itemize}
\item {Grp. gram.:f.}
\end{itemize}
\begin{itemize}
\item {Utilização:P. us.}
\end{itemize}
O mesmo que \textunderscore Mineralogia\textunderscore .
\section{Anorganographia}
\begin{itemize}
\item {Grp. gram.:f.}
\end{itemize}
\begin{itemize}
\item {Utilização:P. us.}
\end{itemize}
O mesmo que \textunderscore Mineralogia\textunderscore .
\section{Anorganologia}
\begin{itemize}
\item {Grp. gram.:f.}
\end{itemize}
\begin{itemize}
\item {Utilização:P. us.}
\end{itemize}
O mesmo que \textunderscore Mineralogia\textunderscore .
\section{Anorganoquímica}
\begin{itemize}
\item {Grp. gram.:f.}
\end{itemize}
Chímica inorgânica.
\section{Anori}
\begin{itemize}
\item {Grp. gram.:m.}
\end{itemize}
\begin{itemize}
\item {Utilização:Bras}
\end{itemize}
O macho do tracajá.
\section{Anormal}
\begin{itemize}
\item {Grp. gram.:adj.}
\end{itemize}
\begin{itemize}
\item {Proveniência:(De \textunderscore a\textunderscore  priv. + \textunderscore normal\textunderscore )}
\end{itemize}
Que está fóra da norma.
Contrário ás regras; irregular.
\section{Anormalidade}
\begin{itemize}
\item {Grp. gram.:f.}
\end{itemize}
Qualidade do que é anormal.
Aquillo que é anormal.
\section{Anormalmente}
\begin{itemize}
\item {Grp. gram.:adv.}
\end{itemize}
De modo \textunderscore anormal\textunderscore .
\section{Anorrhynco}
\begin{itemize}
\item {Grp. gram.:adj.}
\end{itemize}
\begin{itemize}
\item {Utilização:Zool.}
\end{itemize}
Desprovido de nariz.
\section{Anorrinco}
\begin{itemize}
\item {Grp. gram.:adj.}
\end{itemize}
\begin{itemize}
\item {Utilização:Zool.}
\end{itemize}
Desprovido de nariz.
\section{Anortear}
\begin{itemize}
\item {Grp. gram.:v. t.}
\end{itemize}
(V.nortear)
\section{Anorthita}
\begin{itemize}
\item {Grp. gram.:f.}
\end{itemize}
Silicato de alumina e de cal, com pequenas quantidades de ferro, magnésia, soda e potassa.
\section{Anorthoscópio}
\begin{itemize}
\item {Grp. gram.:m.}
\end{itemize}
Instrumento, que produz illusões ópticas pela persistência das impressões luminosas.
\section{Anortita}
\begin{itemize}
\item {Grp. gram.:f.}
\end{itemize}
Silicato de alumina e de cal, com pequenas quantidades de ferro, magnésia, soda e potassa.
\section{Anortoscópio}
\begin{itemize}
\item {Grp. gram.:m.}
\end{itemize}
Instrumento, que produz illusões ópticas pela persistência das impressões luminosas.
\section{Anosfresia}
\begin{itemize}
\item {Grp. gram.:f.}
\end{itemize}
\begin{itemize}
\item {Utilização:Med.}
\end{itemize}
Deminuição ou perda do olfato.
\section{Anosidade}
\begin{itemize}
\item {Grp. gram.:f.}
\end{itemize}
Qualidade de \textunderscore anoso\textunderscore .
\section{Anosmia}
\begin{itemize}
\item {Grp. gram.:f.}
\end{itemize}
\begin{itemize}
\item {Proveniência:(Do gr. \textunderscore an\textunderscore  + \textunderscore osme\textunderscore )}
\end{itemize}
Perda ou deminuição do sentido do olfato.
\section{Anoso}
\begin{itemize}
\item {Grp. gram.:adj.}
\end{itemize}
\begin{itemize}
\item {Proveniência:(Lat. \textunderscore annosus\textunderscore )}
\end{itemize}
Que tem muitos anos: \textunderscore um carvalho anoso\textunderscore .
\section{Anosphresia}
\begin{itemize}
\item {Grp. gram.:f.}
\end{itemize}
\begin{itemize}
\item {Utilização:Med.}
\end{itemize}
Deminuição ou perda do olfato.
\section{Anosteozoário}
\begin{itemize}
\item {Grp. gram.:adj.}
\end{itemize}
\begin{itemize}
\item {Proveniência:(Do gr. \textunderscore an\textunderscore  + \textunderscore osteon\textunderscore  + \textunderscore zoarion\textunderscore )}
\end{itemize}
Diz-se dos animaes que não tem ossos.
\section{Anostose}
\begin{itemize}
\item {Grp. gram.:f.}
\end{itemize}
Atrophia senil dos ossos.
\section{Anostóstomos}
\begin{itemize}
\item {Grp. gram.:m. pl.}
\end{itemize}
\begin{itemize}
\item {Proveniência:(Do gr. \textunderscore anostos\textunderscore  + \textunderscore stoma\textunderscore )}
\end{itemize}
Gênero de insectos orthópteros.
\section{Anotação}
\begin{itemize}
\item {Grp. gram.:f.}
\end{itemize}
\begin{itemize}
\item {Proveniência:(Lat. \textunderscore annotatio\textunderscore )}
\end{itemize}
Acto ou effeito de anotar.
\section{Anotador}
\begin{itemize}
\item {Proveniência:(Lat. \textunderscore annotator\textunderscore )}
\end{itemize}
Aquelle que anota.
\section{Anotar}
\begin{itemize}
\item {Grp. gram.:v. t.}
\end{itemize}
\begin{itemize}
\item {Proveniência:(Lat. \textunderscore annotare\textunderscore )}
\end{itemize}
Fazer notas a.
Esclarecer com commentários.
\section{Anótia}
\begin{itemize}
\item {Grp. gram.:f.}
\end{itemize}
Gênero de hemípteros.
\section{Anotino}
\begin{itemize}
\item {Grp. gram.:adj.}
\end{itemize}
\begin{itemize}
\item {Utilização:Des.}
\end{itemize}
\begin{itemize}
\item {Proveniência:(Lat. \textunderscore annotinus\textunderscore )}
\end{itemize}
O mesmo que \textunderscore anual\textunderscore .
\section{Anoutar}
\begin{itemize}
\item {Grp. gram.:v. i.}
\end{itemize}
O mesmo que \textunderscore ennoitar\textunderscore .
\section{Anoutecer}
\begin{itemize}
\item {Grp. gram.:v. i.}
\end{itemize}
Ir chegando a noite.
Cair a noite; fazer-se noite.
Escurecer.
\section{Anoutecido}
\begin{itemize}
\item {Grp. gram.:adj.}
\end{itemize}
\begin{itemize}
\item {Proveniência:(De \textunderscore anoitecer\textunderscore )}
\end{itemize}
Em que se fez noite.
Escurecido.
\section{Anovar}
\textunderscore v. t.\textunderscore  (e der.)
O mesmo que \textunderscore innovar\textunderscore .
\section{Anovear}
\begin{itemize}
\item {Grp. gram.:v. t.}
\end{itemize}
\begin{itemize}
\item {Utilização:Des.}
\end{itemize}
\begin{itemize}
\item {Utilização:Ant.}
\end{itemize}
\begin{itemize}
\item {Proveniência:(De \textunderscore nove\textunderscore )}
\end{itemize}
Multiplicar por nove.
Obrigar a pagar anóveas.
\section{Anóveas}
\begin{itemize}
\item {Grp. gram.:f. pl.}
\end{itemize}
\begin{itemize}
\item {Utilização:Ant.}
\end{itemize}
\begin{itemize}
\item {Proveniência:(De \textunderscore anovear\textunderscore )}
\end{itemize}
Valor, nove vezes superior ao furto, e que o ladrão tinha que pagar.
\section{Anovelar}
\begin{itemize}
\item {Grp. gram.:v. t.}
\end{itemize}
Dar fórma de novelo a.
\section{Anoxemia}
\begin{itemize}
\item {fónica:cse}
\end{itemize}
\begin{itemize}
\item {Grp. gram.:f.}
\end{itemize}
\begin{itemize}
\item {Proveniência:(Do gr. \textunderscore an\textunderscore  + \textunderscore oxus\textunderscore  + \textunderscore haima\textunderscore )}
\end{itemize}
Falta de oxygenação no sangue.
\section{Anóxia}
\begin{itemize}
\item {Grp. gram.:f.}
\end{itemize}
Gênero de coleópteros.
\section{Anoz}
\begin{itemize}
\item {Grp. gram.:f.}
\end{itemize}
\begin{itemize}
\item {Utilização:Prov.}
\end{itemize}
\begin{itemize}
\item {Utilização:alg.}
\end{itemize}
O mesmo que \textunderscore noz\textunderscore .
\section{Anquilha}
\begin{itemize}
\item {Grp. gram.:f.}
\end{itemize}
\begin{itemize}
\item {Utilização:Ant.}
\end{itemize}
\begin{itemize}
\item {Proveniência:(De \textunderscore anca\textunderscore )}
\end{itemize}
Na Universidade, dava-se este nome a quatro theses finaes, que o defendente escolhia.
\section{Anquílops}
\begin{itemize}
\item {Grp. gram.:m.}
\end{itemize}
\begin{itemize}
\item {Proveniência:(Gr. \textunderscore ankhilops\textunderscore )}
\end{itemize}
Pequeno tumor no ângulo interior do ôlho.
\section{Anquinha}
\begin{itemize}
\item {Grp. gram.:f.}
\end{itemize}
\begin{itemize}
\item {Utilização:Ant.}
\end{itemize}
\begin{itemize}
\item {Proveniência:(De \textunderscore anca\textunderscore )}
\end{itemize}
Algibeira, retesada com barbas de baleia.
\section{Anquinhas}
\begin{itemize}
\item {Grp. gram.:f. pl.}
\end{itemize}
\begin{itemize}
\item {Proveniência:(De \textunderscore anca\textunderscore )}
\end{itemize}
Ancas postiças.
\section{Ansa}
\begin{itemize}
\item {Grp. gram.:f.}
\end{itemize}
Asa. Aso, ensejo: \textunderscore dar ansa á intriga\textunderscore . Cf. Filinto, XIII, 189; XIV, 196; XVI, 34 e 196; Camillo, \textunderscore Caveira\textunderscore , 279; \textunderscore Regicida\textunderscore , 210.
\section{Ânsar}
\begin{itemize}
\item {Grp. gram.:m.}
\end{itemize}
\begin{itemize}
\item {Utilização:Ant.}
\end{itemize}
\begin{itemize}
\item {Proveniência:(Lat. \textunderscore anser\textunderscore )}
\end{itemize}
O mesmo que \textunderscore pato\textunderscore ^1.
\section{Ansarinha-malhada}
\begin{itemize}
\item {Grp. gram.:f.}
\end{itemize}
O mesmo que \textunderscore cegude\textunderscore .
\section{Ansarinho}
\begin{itemize}
\item {Grp. gram.:m.}
\end{itemize}
\begin{itemize}
\item {Utilização:Ant.}
\end{itemize}
\begin{itemize}
\item {Proveniência:(Do lat. \textunderscore anser\textunderscore )}
\end{itemize}
O mesmo que \textunderscore pato\textunderscore ^1.
\section{Anséllia}
\begin{itemize}
\item {Grp. gram.:f.}
\end{itemize}
Gênero de orchídeas.
\section{Anserina}
\begin{itemize}
\item {Grp. gram.:f.}
\end{itemize}
\begin{itemize}
\item {Proveniência:(Lat. \textunderscore anserina\textunderscore )}
\end{itemize}
Gênero de plantas, da fam. das chenopódeas.
\section{Anseríneos}
\begin{itemize}
\item {Grp. gram.:m. pl.}
\end{itemize}
\begin{itemize}
\item {Proveniência:(Do lat. \textunderscore anser\textunderscore )}
\end{itemize}
Tríbo de aves palmípedes, de bico curto.
\section{Ânsia}
\begin{itemize}
\item {Grp. gram.:f.}
\end{itemize}
\begin{itemize}
\item {Grp. gram.:Pl.}
\end{itemize}
\begin{itemize}
\item {Proveniência:(Lat. \textunderscore anxia\textunderscore )}
\end{itemize}
Afflicção.
Estertor.
Desejo ardente.
Perturbação, causada pela incerteza.
Náuseas, prenúncios de vómito.
\section{Ansiamento}
\begin{itemize}
\item {Grp. gram.:m.}
\end{itemize}
O mesmo que \textunderscore ânsia\textunderscore . Cf. Camillo, \textunderscore Noit. de Lam.\textunderscore  112.
\section{Ansiar}
\begin{itemize}
\item {Grp. gram.:v. t.}
\end{itemize}
\begin{itemize}
\item {Grp. gram.:V. i.}
\end{itemize}
Causar ânsia a; fazer soffrer.
Desejar ardentemente.
Ter ânsias.
\section{Ansiedade}
\begin{itemize}
\item {Grp. gram.:f.}
\end{itemize}
\begin{itemize}
\item {Proveniência:(Lat. \textunderscore anxietas\textunderscore )}
\end{itemize}
Angústia.
Incerteza afflictiva.
Desejo ardente.
\section{Ansiforme}
\begin{itemize}
\item {Grp. gram.:adj.}
\end{itemize}
\begin{itemize}
\item {Proveniência:(Do lat. \textunderscore ansa\textunderscore  + \textunderscore forma\textunderscore )}
\end{itemize}
Que tem fórma de ansa ou de asa.
\section{Ansiosamente}
\begin{itemize}
\item {Grp. gram.:adv.}
\end{itemize}
De modo \textunderscore ansioso\textunderscore .
Com ânsia.
\section{Ansioso}
\begin{itemize}
\item {Grp. gram.:adj.}
\end{itemize}
Que tem ânsias.
Afflicto.
\section{Anso}
\begin{itemize}
\item {Grp. gram.:m.}
\end{itemize}
\begin{itemize}
\item {Utilização:Ant.}
\end{itemize}
O mesmo que \textunderscore ansa\textunderscore .
\section{Anspeçada}
\begin{itemize}
\item {Grp. gram.:m.}
\end{itemize}
\begin{itemize}
\item {Utilização:Ant.}
\end{itemize}
Official militar, da graduação mais inferior.
(Corr. do it. \textunderscore lancia spezzata\textunderscore )
\section{Ant...}
\begin{itemize}
\item {Grp. gram.:pref.}
\end{itemize}
Equivalente a \textunderscore ante...\textunderscore  ou \textunderscore anti...\textunderscore , quando a palavra, a que se junta, começa por vogal.
\section{Anta}
\begin{itemize}
\item {Grp. gram.:f.}
\end{itemize}
\begin{itemize}
\item {Utilização:Ant.}
\end{itemize}
\begin{itemize}
\item {Proveniência:(Lat. \textunderscore antae\textunderscore )}
\end{itemize}
Monumento megalíthico, formado de uma grande pedra horizontal sôbre outras, mais pequenas e verticaes.
Dólmen.
Pilastra angular.
Monte, de terra, que servia de demarcação.
\section{Anta}
\begin{itemize}
\item {Grp. gram.:f.}
\end{itemize}
\begin{itemize}
\item {Proveniência:(Do ár. \textunderscore lamt\textunderscore ?)}
\end{itemize}
Espécie de antílope, originário da América do Sul.
A pelle deste animal.
\section{Antagónia}
\begin{itemize}
\item {Grp. gram.:f.}
\end{itemize}
\begin{itemize}
\item {Proveniência:(Do gr. \textunderscore anti\textunderscore  + \textunderscore agon\textunderscore )}
\end{itemize}
Gênero de plantas cucurbitáceas.
\section{Antagónico}
\begin{itemize}
\item {Grp. gram.:adj.}
\end{itemize}
\begin{itemize}
\item {Proveniência:(Do gr. \textunderscore anti\textunderscore  + \textunderscore agon\textunderscore )}
\end{itemize}
Contrário, opposto: \textunderscore ideias antagónicas\textunderscore .
\section{Antagonismo}
\begin{itemize}
\item {Grp. gram.:m.}
\end{itemize}
\begin{itemize}
\item {Proveniência:(Gr. \textunderscore antagonisma\textunderscore )}
\end{itemize}
Opposição de systemas.
Rivalidade.
Incompatibilidade.
\section{Antagonista}
\begin{itemize}
\item {Grp. gram.:m.}
\end{itemize}
\begin{itemize}
\item {Proveniência:(Gr. \textunderscore antagonistes\textunderscore )}
\end{itemize}
Contradictor; impugnador.
Aquelle que se esforça contra alguém ou contra alguma coisa.
\section{Antal}
\begin{itemize}
\item {Grp. gram.:m.}
\end{itemize}
Espécie de mollusco.
\section{Antamba}
\begin{itemize}
\item {Grp. gram.:f.}
\end{itemize}
Espécie de leopardo, referida por Bluteau.
Mammífero de Madagáscar.
\section{Antanaclase}
\begin{itemize}
\item {Grp. gram.:f.}
\end{itemize}
\begin{itemize}
\item {Proveniência:(Gr. \textunderscore antanaklasis\textunderscore )}
\end{itemize}
Emprêgo de palavras semelhantes na fórma ou no som, mas differentes no sentido.
\section{Antanagoge}
\begin{itemize}
\item {Grp. gram.:f.}
\end{itemize}
\begin{itemize}
\item {Proveniência:(Do gr. \textunderscore anti\textunderscore  + \textunderscore anagoge\textunderscore )}
\end{itemize}
Recriminação, com os mesmos argumentos que serviram ao accusador.
\section{Antanho}
\begin{itemize}
\item {Grp. gram.:adv.}
\end{itemize}
No anno próximo passado.
Nos tempos passados.
(Cast. \textunderscore antaño\textunderscore , do lat. \textunderscore ante\textunderscore  + \textunderscore annum\textunderscore )
\section{Antão}
\begin{itemize}
\item {Grp. gram.:adv.}
\end{itemize}
\begin{itemize}
\item {Utilização:Pop.}
\end{itemize}
O mesmo que \textunderscore então\textunderscore .
\section{Antapódose}
\begin{itemize}
\item {Grp. gram.:f.}
\end{itemize}
\begin{itemize}
\item {Utilização:Gram.}
\end{itemize}
\begin{itemize}
\item {Proveniência:(Gr. \textunderscore antapodosis\textunderscore )}
\end{itemize}
Membro de um período, correspondente a outro.
\section{Antar}
\begin{itemize}
\item {Grp. gram.:v. t.}
\end{itemize}
Preparar como pelle de anta.
\section{Antárctia}
\begin{itemize}
\item {Grp. gram.:f.}
\end{itemize}
Gênero de coleópteros.
\section{Antárctico}
\begin{itemize}
\item {Grp. gram.:adj.}
\end{itemize}
\begin{itemize}
\item {Proveniência:(Gr. \textunderscore antarktikos\textunderscore )}
\end{itemize}
Opposto a árctico.
Relativo ao polo meridional ou ás regiões glaciaes do Sul.
\section{Antares}
\begin{itemize}
\item {Grp. gram.:m.}
\end{itemize}
\begin{itemize}
\item {Proveniência:(Fr. \textunderscore antarès\textunderscore )}
\end{itemize}
Estrêlla da constellação do Escorpião.
\section{Antáxia}
\begin{itemize}
\item {fónica:csi}
\end{itemize}
\begin{itemize}
\item {Grp. gram.:f.}
\end{itemize}
Gênero de coleópteros.
\section{Antauge}
\begin{itemize}
\item {Grp. gram.:m.}
\end{itemize}
\begin{itemize}
\item {Utilização:Des.}
\end{itemize}
\begin{itemize}
\item {Proveniência:(De \textunderscore anti...\textunderscore  + \textunderscore auge\textunderscore )}
\end{itemize}
O mesmo que \textunderscore perigeu\textunderscore .
\section{Ante}
\begin{itemize}
\item {Grp. gram.:prep.}
\end{itemize}
\begin{itemize}
\item {Proveniência:(Lat. \textunderscore ante\textunderscore )}
\end{itemize}
Deante de: \textunderscore apresentou-se ante mim\textunderscore .
\section{Ante...}
\begin{itemize}
\item {Grp. gram.:pref.}
\end{itemize}
\begin{itemize}
\item {Proveniência:(Do lat. \textunderscore ante\textunderscore )}
\end{itemize}
(equivalente a \textunderscore antes de\textunderscore )
\section{...ante}
\begin{itemize}
\item {Grp. gram.:suf. adj.}
\end{itemize}
\begin{itemize}
\item {Grp. gram.:Suf. m.}
\end{itemize}
\begin{itemize}
\item {Proveniência:(Do lat. \textunderscore ...ans\textunderscore )}
\end{itemize}
(que se junta ao thema dos v. da 1.^a conjug.).
(indic. de profissão ou seita): \textunderscore negociante\textunderscore .
\section{Ante-acto}
\begin{itemize}
\item {Grp. gram.:m.}
\end{itemize}
Termo proposto para substituir, em linguagem nossa, o fr. \textunderscore lever-de-rideau\textunderscore , pequena peça theatral, que se representa antes da peça principal do espectáculo.
\section{Anteado}
\begin{itemize}
\item {Grp. gram.:m.}
\end{itemize}
O mesmo que \textunderscore enteado\textunderscore .
\section{Anteagora}
\begin{itemize}
\item {Grp. gram.:adv.}
\end{itemize}
\begin{itemize}
\item {Utilização:Ant.}
\end{itemize}
\begin{itemize}
\item {Proveniência:(De \textunderscore ante\textunderscore  + \textunderscore agora\textunderscore )}
\end{itemize}
Ainda há pouco.
Em lugar ou occasião proximamente anterior.
\section{Ante-alcova}
\begin{itemize}
\item {Grp. gram.:f.}
\end{itemize}
Compartimento á frente da alcova. Cf. Th. Ribeiro, \textunderscore Jornadas\textunderscore , I, 295.
\section{Anteâmbulo}
\begin{itemize}
\item {Grp. gram.:m.}
\end{itemize}
\begin{itemize}
\item {Proveniência:(Lat. \textunderscore anteambulo\textunderscore )}
\end{itemize}
Escravo romano, que ia adeante da léctica de seu senhor ou senhora, abrindo caminho por entre a multidão.
\section{Anteàré}
\begin{itemize}
\item {fónica:ré}
\end{itemize}
\begin{itemize}
\item {Grp. gram.:f.}
\end{itemize}
\begin{itemize}
\item {Utilização:Náut.}
\end{itemize}
Parte do navio, comprehendida entre o mastro grande e a popa.
O lugar que uma coisa ou pessôa occupa deante da outra, na direcção da popa.
\section{Anteàrré}
\begin{itemize}
\item {Grp. gram.:f.}
\end{itemize}
\begin{itemize}
\item {Utilização:Náut.}
\end{itemize}
Parte do navio, comprehendida entre o mastro grande e a popa.
O lugar que uma coisa ou pessôa occupa deante da outra, na direcção da popa.
\section{Anteaurora}
\begin{itemize}
\item {Grp. gram.:f.}
\end{itemize}
\begin{itemize}
\item {Proveniência:(De \textunderscore ante\textunderscore  + \textunderscore aurora\textunderscore )}
\end{itemize}
Alva, madrugada.
\section{Anteavante}
\begin{itemize}
\item {Grp. gram.:f.}
\end{itemize}
\begin{itemize}
\item {Utilização:Náut.}
\end{itemize}
Parte do navio, comprehendida entre o mastro grande e a prôa.
O lugar que uma coisa ou pessôa occupa adeante da prôa: \textunderscore a bitácula fica por anteavante da roda do leme\textunderscore .
\section{Ante-bem}
\begin{itemize}
\item {Grp. gram.:m.}
\end{itemize}
\begin{itemize}
\item {Utilização:Prov.}
\end{itemize}
\begin{itemize}
\item {Utilização:minh.}
\end{itemize}
Pequena refeição de pão e vinho, que os trabalhadores do campo tomam entre o almoço e o jantar.
\section{Anteboca}
\begin{itemize}
\item {fónica:bô}
\end{itemize}
\begin{itemize}
\item {Grp. gram.:f.}
\end{itemize}
\begin{itemize}
\item {Proveniência:(De \textunderscore ante\textunderscore  + \textunderscore boca\textunderscore )}
\end{itemize}
A parte anterior da boca.
\section{Antebrachial}
\begin{itemize}
\item {fónica:qui}
\end{itemize}
\begin{itemize}
\item {Grp. gram.:adj.}
\end{itemize}
Relativo ao \textunderscore antebraço\textunderscore .
\section{Antebraço}
\begin{itemize}
\item {Grp. gram.:m.}
\end{itemize}
\begin{itemize}
\item {Proveniência:(Do lat. \textunderscore ante\textunderscore  + \textunderscore brachium\textunderscore )}
\end{itemize}
Parte do braço, entre o cotovelo e o pulso.
\section{Antebraquial}
\begin{itemize}
\item {Grp. gram.:adj.}
\end{itemize}
Relativo ao \textunderscore antebraço\textunderscore .
\section{Antecalva}
\begin{itemize}
\item {Grp. gram.:f.}
\end{itemize}
\begin{itemize}
\item {Proveniência:(De \textunderscore ante\textunderscore  + \textunderscore calva\textunderscore )}
\end{itemize}
Calva, na parte anterior da cabêça.
\section{Antecama}
\begin{itemize}
\item {Grp. gram.:f.}
\end{itemize}
\begin{itemize}
\item {Utilização:Prov.}
\end{itemize}
\begin{itemize}
\item {Utilização:beir.}
\end{itemize}
\begin{itemize}
\item {Proveniência:(De \textunderscore ante\textunderscore  + \textunderscore cama\textunderscore )}
\end{itemize}
Pano, que, á frente da cama, cai desde a borda do catre até o chão, para ornato e para occultar o bacio, que habitualmente se colloca debaixo da cama.
\section{Antecâmara}
\begin{itemize}
\item {Grp. gram.:f.}
\end{itemize}
\begin{itemize}
\item {Proveniência:(De \textunderscore ante\textunderscore  + \textunderscore câmara\textunderscore )}
\end{itemize}
Sala, que precede a principal.
Sala de espera.
Espaço anterior á câmara do navio.
\section{Antecanto}
\begin{itemize}
\item {Grp. gram.:m.}
\end{itemize}
Estribilho que se repete no princípio de cada estrophe.
\section{Antecedência}
\begin{itemize}
\item {Grp. gram.:f.}
\end{itemize}
Estado do que é \textunderscore antecedente\textunderscore .
\section{Antecedente}
\begin{itemize}
\item {Grp. gram.:adj.}
\end{itemize}
\begin{itemize}
\item {Proveniência:(Lat. \textunderscore antecedens\textunderscore )}
\end{itemize}
Que antecede.
Precedente.
\section{Antecedentemente}
\begin{itemize}
\item {Grp. gram.:adv.}
\end{itemize}
Em tempo ou lugar \textunderscore antecedente\textunderscore .
\section{Anteceder}
\begin{itemize}
\item {Grp. gram.:v. t.}
\end{itemize}
\begin{itemize}
\item {Grp. gram.:V. i.}
\end{itemize}
\begin{itemize}
\item {Proveniência:(Lat. \textunderscore antecedere\textunderscore )}
\end{itemize}
Vir antes de.
Estar antes de.
Sêr anterior.
\section{Antecessor}
\begin{itemize}
\item {Grp. gram.:m.}
\end{itemize}
\begin{itemize}
\item {Proveniência:(Lat. \textunderscore antecessor\textunderscore )}
\end{itemize}
Aquelle que antecede.
Antepassado.
\section{Ante-céu}
\begin{itemize}
\item {Grp. gram.:m.}
\end{itemize}
\begin{itemize}
\item {Utilização:Fig.}
\end{itemize}
Grande ventura. Cf. Castilho, \textunderscore Tartufo\textunderscore , 21.
\section{Antécios}
\begin{itemize}
\item {Grp. gram.:m. pl.}
\end{itemize}
O mesmo que \textunderscore antecos\textunderscore .
\section{Antecipação}
\begin{itemize}
\item {Grp. gram.:f.}
\end{itemize}
\begin{itemize}
\item {Proveniência:(Lat. \textunderscore anticipatio\textunderscore )}
\end{itemize}
Acto de antecipar.
\section{Antecipadamente}
\begin{itemize}
\item {Grp. gram.:adv.}
\end{itemize}
Com antecipação.
\section{Antecipar}
\begin{itemize}
\item {Grp. gram.:v. t.}
\end{itemize}
\begin{itemize}
\item {Proveniência:(Lat. \textunderscore anticipare\textunderscore )}
\end{itemize}
Prever.
Dar ou receber antes; adeantar: \textunderscore antecipar um pagamento\textunderscore .
\section{Antecolumna}
\begin{itemize}
\item {Grp. gram.:f.}
\end{itemize}
\begin{itemize}
\item {Proveniência:(De \textunderscore ante\textunderscore  + \textunderscore columna\textunderscore )}
\end{itemize}
Columna á frente, separada de outras.
\section{Antecoluna}
\begin{itemize}
\item {Grp. gram.:f.}
\end{itemize}
\begin{itemize}
\item {Proveniência:(De \textunderscore ante\textunderscore  + \textunderscore columna\textunderscore )}
\end{itemize}
Columna á frente, separada de outras.
\section{Ante-com-ante}
\begin{itemize}
\item {Grp. gram.:loc. adv.}
\end{itemize}
\begin{itemize}
\item {Utilização:Ant.}
\end{itemize}
Depressa. Cf. Jer. Cardoso, \textunderscore Diction\textunderscore .
\section{Anteconhecimento}
\begin{itemize}
\item {Grp. gram.:m.}
\end{itemize}
Previdência.
Conhecimento prévio.
\section{Ante-conto}
\begin{itemize}
\item {Grp. gram.:loc. adv.}
\end{itemize}
\begin{itemize}
\item {Utilização:Prov.}
\end{itemize}
\begin{itemize}
\item {Utilização:minh.}
\end{itemize}
Rapidamente, em menos tempo que o preciso para se contar.
\section{Antecór}
\begin{itemize}
\item {Grp. gram.:m.}
\end{itemize}
\begin{itemize}
\item {Proveniência:(Do lat. \textunderscore ante...\textunderscore  + \textunderscore cor\textunderscore )}
\end{itemize}
O mesmo que \textunderscore antecoração\textunderscore .
\section{Antecoração}
\begin{itemize}
\item {Grp. gram.:m.}
\end{itemize}
\begin{itemize}
\item {Proveniência:(De \textunderscore ante...\textunderscore  + \textunderscore coração\textunderscore )}
\end{itemize}
Tumor no peito do cavallo.
Espécie de carbúnculo mortífero na frente do peito ou por trás da espádua no gado bovino.
\section{Antecoro}
\begin{itemize}
\item {fónica:cô}
\end{itemize}
\begin{itemize}
\item {Grp. gram.:m.}
\end{itemize}
\begin{itemize}
\item {Proveniência:(De \textunderscore ante...\textunderscore  + \textunderscore côro\textunderscore )}
\end{itemize}
Casa, que antecede o côro.
\section{Antecos}
\begin{itemize}
\item {Grp. gram.:m. pl.}
\end{itemize}
\begin{itemize}
\item {Proveniência:(Do gr. \textunderscore anti\textunderscore  + \textunderscore oikos\textunderscore )}
\end{itemize}
Habitantes, que no globo têm o mesmo meridiano e latitude opposta.
\section{Antecuante}
\begin{itemize}
\item {Grp. gram.:adv.}
\end{itemize}
\begin{itemize}
\item {Utilização:Ant.}
\end{itemize}
Para trás:«\textunderscore tórno antecuante\textunderscore ». \textunderscore Eufrosina\textunderscore , act. I, sc. 3.
(Cp. \textunderscore recuar\textunderscore )
\section{Antecuco}
\begin{itemize}
\item {Grp. gram.:m. ?}
\end{itemize}
\begin{itemize}
\item {Utilização:Neol.}
\end{itemize}
Aquelle que casou com mulher já forniziada. Cf. Pacheco, \textunderscore Promptuário\textunderscore , 20.
\section{Antedar}
\begin{itemize}
\item {Grp. gram.:v. t.}
\end{itemize}
Dar antes.
\section{Antedata}
\begin{itemize}
\item {Grp. gram.:f.}
\end{itemize}
Data anterior, falsa, destinada a supprir a verdadeira, que se não indica.
\section{Antedatar}
\begin{itemize}
\item {Grp. gram.:v. t.}
\end{itemize}
Pôr antedata em.
\section{Antedia}
\begin{itemize}
\item {Grp. gram.:adv.}
\end{itemize}
\begin{itemize}
\item {Utilização:Ant.}
\end{itemize}
Antes do dia presente.
\section{Antediluviano}
\begin{itemize}
\item {Grp. gram.:adj.}
\end{itemize}
Que é anterior ao dilúvio.
\section{Antedizer}
\begin{itemize}
\item {Grp. gram.:v. t.}
\end{itemize}
Prognosticar.
Annunciar.
Predizer:«\textunderscore isto é o que tu, Hoseah, avias antedito\textunderscore ». Usque, \textunderscore Tribulações\textunderscore , 17.
\section{Anteface}
\begin{itemize}
\item {Grp. gram.:m.}
\end{itemize}
O mesmo ou melhor que \textunderscore antiface\textunderscore :«\textunderscore a realeza absoluta, com a máscara e anteface de govêrno liberal...\textunderscore »Latino, \textunderscore Elogios Acad.\textunderscore , I, 340.
\section{Antefirma}
\begin{itemize}
\item {Grp. gram.:f.}
\end{itemize}
Palavras de cortesia, que precedem a firma de uma carta.
\section{Antefixos}
\begin{itemize}
\item {Grp. gram.:m. pl.}
\end{itemize}
\begin{itemize}
\item {Proveniência:(Lat. \textunderscore antefixa\textunderscore )}
\end{itemize}
Ornatos, adoptados pela architectura grega, romana e etrusca, e que se collocavam verticalmente á frente das telhas, nas faces lateraes dos templos.
\section{Anteflexão}
\begin{itemize}
\item {Grp. gram.:f.}
\end{itemize}
Deformidade do útero, caracterizada pela inclinação anterior do corpo delle sôbre o collo ou vice-versa.
\section{Antefôsso}
\begin{itemize}
\item {Grp. gram.:m.}
\end{itemize}
Fôsso, em volta da esplanada.
\section{Antegalha}
\begin{itemize}
\item {Grp. gram.:f.}
\end{itemize}
\begin{itemize}
\item {Utilização:Náut.}
\end{itemize}
Tomadoiro, com que se amarra a vela, em occasião de temporal.
\section{Antegostar}
\begin{itemize}
\item {Grp. gram.:v. t.}
\end{itemize}
O mesmo que \textunderscore antegozar\textunderscore .
\section{Antegôsto}
\begin{itemize}
\item {Grp. gram.:m.}
\end{itemize}
O mesmo que \textunderscore antegôzo\textunderscore .
\section{Antéfora}
\begin{itemize}
\item {Grp. gram.:f.}
\end{itemize}
Gênero de plantas gramíneas.
\section{Antegozar}
\begin{itemize}
\item {Grp. gram.:v. t.}
\end{itemize}
Gozar antes; prelibar. Cf. Castilho, \textunderscore Fausto\textunderscore , 210.
\section{Antegôzo}
\begin{itemize}
\item {Grp. gram.:m.}
\end{itemize}
Gôzo antecipado.
Acto de \textunderscore antegozar\textunderscore .
\section{Anteguarda}
\begin{itemize}
\item {Grp. gram.:f.}
\end{itemize}
O mesmo que \textunderscore vanguarda\textunderscore .
\section{Antehistórico}
\begin{itemize}
\item {Grp. gram.:adj.}
\end{itemize}
O mesmo que \textunderscore prehistórico\textunderscore .
\section{Antejo}
\begin{itemize}
\item {Grp. gram.:m.}
\end{itemize}
\begin{itemize}
\item {Utilização:Ant.}
\end{itemize}
O mesmo que \textunderscore entejo\textunderscore . Cf. \textunderscore Cancion. da Vaticana\textunderscore .
\section{Antela}
\begin{itemize}
\item {Grp. gram.:f.}
\end{itemize}
Pequena anta^1.
\section{Antela}
\begin{itemize}
\item {Grp. gram.:f.}
\end{itemize}
\begin{itemize}
\item {Utilização:Bot.}
\end{itemize}
Racimo de flôres, em que as ramificações são largas e visíveis.
\section{Antelação}
\begin{itemize}
\item {Grp. gram.:f.}
\end{itemize}
\begin{itemize}
\item {Utilização:Des.}
\end{itemize}
\begin{itemize}
\item {Proveniência:(Do lat. \textunderscore antelatus\textunderscore )}
\end{itemize}
Preferência.
\section{Antélia}
\begin{itemize}
\item {Grp. gram.:f.}
\end{itemize}
O mesmo que \textunderscore antélio\textunderscore .
\section{Antélice}
\begin{itemize}
\item {Grp. gram.:m.}
\end{itemize}
\begin{itemize}
\item {Utilização:Anat.}
\end{itemize}
\begin{itemize}
\item {Proveniência:(Do gr. \textunderscore anti\textunderscore  + \textunderscore helix\textunderscore )}
\end{itemize}
Elevação do pavilhão da orelha, antes do hélice.
\section{Antélio}
\begin{itemize}
\item {Grp. gram.:m.}
\end{itemize}
\begin{itemize}
\item {Proveniência:(Do gr. \textunderscore anti\textunderscore  + \textunderscore helios\textunderscore )}
\end{itemize}
Claridade reflectida pelo sol no lado opposto a êste astro.
\section{Antelmíntico}
\begin{itemize}
\item {Grp. gram.:adj.}
\end{itemize}
(V.antihelmíntico)
\section{Antelo}
\begin{itemize}
\item {Grp. gram.:adj.}
\end{itemize}
Diz-se da inflorescência dos juncos.
\section{Antelóquio}
\begin{itemize}
\item {Grp. gram.:m.}
\end{itemize}
\begin{itemize}
\item {Proveniência:(Lat. \textunderscore anteloquium\textunderscore )}
\end{itemize}
Prefácio.
Prólogo.
Aquillo que se diz antes.
\section{Antelucano}
\begin{itemize}
\item {Grp. gram.:adj.}
\end{itemize}
\begin{itemize}
\item {Proveniência:(Do lat. \textunderscore ante\textunderscore  + \textunderscore lux\textunderscore )}
\end{itemize}
Que se fez antes da luz do dia.
\section{Antema}
\begin{itemize}
\item {Grp. gram.:f.}
\end{itemize}
Gênero de plantas malváceas.
\section{Antemanhã}
\begin{itemize}
\item {Grp. gram.:f.}
\end{itemize}
\begin{itemize}
\item {Grp. gram.:Adv.}
\end{itemize}
O alvorecer.
Pouco antes de amanhecer.
\section{Antemanhan}
\begin{itemize}
\item {Grp. gram.:f.}
\end{itemize}
\begin{itemize}
\item {Grp. gram.:Adv.}
\end{itemize}
O alvorecer.
Pouco antes de amanhecer.
\section{Antemão}
\begin{itemize}
\item {Grp. gram.:adv.}
\end{itemize}
Previamente.
\section{Antemediário}
\begin{itemize}
\item {Grp. gram.:adj.}
\end{itemize}
\begin{itemize}
\item {Utilização:Bot.}
\end{itemize}
Diz-se das pétalas oppostas ás sépalas do cálice.
\section{Antemeridiano}
\begin{itemize}
\item {Grp. gram.:adj.}
\end{itemize}
Anterior ao meio-dia.
\section{Antemesa}
\begin{itemize}
\item {Grp. gram.:f.}
\end{itemize}
Pano bento, em que dizem Missa os sacerdotes do rito grego.
\section{Antemídeas}
\begin{itemize}
\item {Grp. gram.:f. pl.}
\end{itemize}
\begin{itemize}
\item {Proveniência:(Do gr. \textunderscore anthemis\textunderscore  + \textunderscore eidos\textunderscore )}
\end{itemize}
Tribo de plantas, que tem por typo a ântemis.
\section{Antemina}
\begin{itemize}
\item {Grp. gram.:f.}
\end{itemize}
\begin{itemize}
\item {Proveniência:(De \textunderscore ânthemis\textunderscore )}
\end{itemize}
Princípio estimulante da camomila.
\section{Ântemis}
\begin{itemize}
\item {Grp. gram.:f.}
\end{itemize}
\begin{itemize}
\item {Proveniência:(Gr. \textunderscore anthemis\textunderscore )}
\end{itemize}
Nome scientífico da camomila; macella.
\section{Antemover}
\begin{itemize}
\item {Grp. gram.:v. t.}
\end{itemize}
Mover com antecedência.
\section{Antemural}
\begin{itemize}
\item {Grp. gram.:adj.}
\end{itemize}
\begin{itemize}
\item {Grp. gram.:M.}
\end{itemize}
Relativo ao antemuro.
O mesmo que \textunderscore antemuro\textunderscore .
\section{Antemuralha}
\begin{itemize}
\item {Grp. gram.:f.}
\end{itemize}
O mesmo que \textunderscore antemuro\textunderscore .
\section{Antemurar}
\begin{itemize}
\item {Grp. gram.:v. t.}
\end{itemize}
Fortalecer com antemuros.
\section{Antemuro}
\begin{itemize}
\item {Grp. gram.:m.}
\end{itemize}
Parapeito de fortaleza.
Barbacan.
Obra avançada de fortificação.
\section{Antena}
\begin{itemize}
\item {Grp. gram.:f.}
\end{itemize}
\begin{itemize}
\item {Utilização:Náut.}
\end{itemize}
\begin{itemize}
\item {Utilização:Náut.}
\end{itemize}
\begin{itemize}
\item {Utilização:Zool.}
\end{itemize}
\begin{itemize}
\item {Proveniência:(Lat. \textunderscore antenna\textunderscore )}
\end{itemize}
Vêrga, fixa ao mastro, na qual se prende a vela triangular, (vela latina).
Ligeiro tabique, que determina, em sentido vertical, as divisões interiores do navio, especialmente a dos alojamentos.
Appêndice móvel na cabeça de animaes articulados.
\section{Antenado}
\begin{itemize}
\item {Grp. gram.:adj.}
\end{itemize}
\begin{itemize}
\item {Utilização:Zool.}
\end{itemize}
Que tem antenas.
\section{Antenal}
\begin{itemize}
\item {Grp. gram.:adj.}
\end{itemize}
\begin{itemize}
\item {Utilização:Zool.}
\end{itemize}
\begin{itemize}
\item {Grp. gram.:M.}
\end{itemize}
Relativo ás antenas.
Ave marítima do Cabo da Bôa-Esperança. Cf. Viana, \textunderscore Apostilas\textunderscore .
Cp. \textunderscore entenal\textunderscore .
\section{Antenária}
\begin{itemize}
\item {Grp. gram.:f.}
\end{itemize}
\begin{itemize}
\item {Proveniência:(De \textunderscore antenna\textunderscore )}
\end{itemize}
Gênero de plantas, da fam. das compostas.
\section{Antenífero}
\begin{itemize}
\item {Grp. gram.:adj.}
\end{itemize}
\begin{itemize}
\item {Proveniência:(Do lat. \textunderscore antenna\textunderscore  + \textunderscore ferre\textunderscore )}
\end{itemize}
O mesmo que \textunderscore antenado\textunderscore .
\section{Anteniforme}
\begin{itemize}
\item {Grp. gram.:adj.}
\end{itemize}
\begin{itemize}
\item {Proveniência:(Do lat. \textunderscore antenna\textunderscore  + \textunderscore forma\textunderscore )}
\end{itemize}
Semelhante á antena.
\section{Antenna}
\begin{itemize}
\item {Grp. gram.:f.}
\end{itemize}
\begin{itemize}
\item {Utilização:Náut.}
\end{itemize}
\begin{itemize}
\item {Utilização:Náut.}
\end{itemize}
\begin{itemize}
\item {Utilização:Zool.}
\end{itemize}
\begin{itemize}
\item {Proveniência:(Lat. \textunderscore antenna\textunderscore )}
\end{itemize}
Vêrga, fixa ao mastro, na qual se prende a vela triangular,
(vela latina)
Ligeiro tabique, que determina, em sentido vertical, as divisões interiores do navio, especialmente a dos alojamentos.
Appêndice móvel na cabeça de animaes articulados.
\section{Antennado}
\begin{itemize}
\item {Grp. gram.:adj.}
\end{itemize}
\begin{itemize}
\item {Utilização:Zool.}
\end{itemize}
Que tem antennas.
\section{Antennal}
\begin{itemize}
\item {Grp. gram.:adj.}
\end{itemize}
\begin{itemize}
\item {Utilização:Zool.}
\end{itemize}
\begin{itemize}
\item {Grp. gram.:M.}
\end{itemize}
Relativo ás antennas.
Ave marítima do Cabo da Bôa-Esperança. Cf. Viana, \textunderscore Apostilas\textunderscore .
Cp. \textunderscore entenal\textunderscore .
\section{Antennária}
\begin{itemize}
\item {Grp. gram.:f.}
\end{itemize}
\begin{itemize}
\item {Proveniência:(De \textunderscore antenna\textunderscore )}
\end{itemize}
Gênero de plantas, da fam. das compostas.
\section{Antennífero}
\begin{itemize}
\item {Grp. gram.:adj.}
\end{itemize}
\begin{itemize}
\item {Proveniência:(Do lat. \textunderscore antenna\textunderscore  + \textunderscore ferre\textunderscore )}
\end{itemize}
O mesmo que \textunderscore antennado\textunderscore .
\section{Antenniforme}
\begin{itemize}
\item {Grp. gram.:adj.}
\end{itemize}
\begin{itemize}
\item {Proveniência:(Do lat. \textunderscore antenna\textunderscore  + \textunderscore forma\textunderscore )}
\end{itemize}
Semelhante á antenna.
\section{Antênnula}
\begin{itemize}
\item {Grp. gram.:f.}
\end{itemize}
Appêndice articulado na mandíbula de vários insectos.
(Dem. de \textunderscore antenna\textunderscore )
\section{Antenome}
\begin{itemize}
\item {Grp. gram.:m.}
\end{itemize}
Prenome.
Título, que precede o nome.
\section{Antênula}
\begin{itemize}
\item {Grp. gram.:f.}
\end{itemize}
Appêndice articulado na mandíbula de vários insectos.
(Dem. de \textunderscore antenna\textunderscore )
\section{Antenupcial}
\begin{itemize}
\item {Grp. gram.:adj.}
\end{itemize}
Que antecede as núpcias: \textunderscore contrato antenupcial\textunderscore .
\section{Anteoccupação}
\begin{itemize}
\item {Grp. gram.:f.}
\end{itemize}
Acto de \textunderscore anteoccupar\textunderscore .
Figura de Rethórica, em que se prevê e se destrói a objecção.
\section{Anteoccupar}
\begin{itemize}
\item {Grp. gram.:v. t.}
\end{itemize}
O mesmo que \textunderscore preoccupar\textunderscore .
\section{Anteocupação}
\begin{itemize}
\item {Grp. gram.:f.}
\end{itemize}
Acto de \textunderscore anteocupar\textunderscore .
Figura de Rethórica, em que se prevê e se destrói a objecção.
\section{Anteocupar}
\begin{itemize}
\item {Grp. gram.:v. t.}
\end{itemize}
O mesmo que \textunderscore preoccupar\textunderscore .
\section{Ante-olhos}
\begin{itemize}
\item {Grp. gram.:m. pl.}
\end{itemize}
Placas de coiro, que se collocam ao lado dos olhos das cavalgaduras.
\section{Anteontem}
\begin{itemize}
\item {Grp. gram.:adv.}
\end{itemize}
No dia immediamente anterior ao de ontem.
\section{Antepagar}
\begin{itemize}
\item {Grp. gram.:v. t.}
\end{itemize}
Pagar com antecedência.
\section{Antepaixão}
\begin{itemize}
\item {Grp. gram.:f.}
\end{itemize}
\begin{itemize}
\item {Utilização:Ant.}
\end{itemize}
Preconceito.
Paixão, que precede a razão.
\section{Antepara}
\begin{itemize}
\item {Grp. gram.:f.}
\end{itemize}
\begin{itemize}
\item {Utilização:Ant.}
\end{itemize}
\begin{itemize}
\item {Proveniência:(De \textunderscore anteparar\textunderscore )}
\end{itemize}
Divisão provisória na coberta dos navios.
O mesmo que \textunderscore anteparo\textunderscore .
\section{Anteparar}
\begin{itemize}
\item {Grp. gram.:v. t.}
\end{itemize}
Resguardar.
Defender.
Acautelar.
\section{Anteparo}
\begin{itemize}
\item {Grp. gram.:m.}
\end{itemize}
Acção de \textunderscore anteparar\textunderscore .
Resguardo.
Defesa; precaução.
\section{Anteparto}
\begin{itemize}
\item {Grp. gram.:m.}
\end{itemize}
O tempo immediatamente anterior ao parto.
\section{Antepassado}
\begin{itemize}
\item {Grp. gram.:adj.}
\end{itemize}
\begin{itemize}
\item {Grp. gram.:M.}
\end{itemize}
Que passou antes.
Ascendente.
Antecessor: \textunderscore o patriotismo dos nossos antepassados\textunderscore .
\section{Antepassar}
\begin{itemize}
\item {Grp. gram.:v. t.}
\end{itemize}
Preceder.
\section{Antepasto}
\begin{itemize}
\item {Grp. gram.:m.}
\end{itemize}
Iguaria, que precede a primeira coberta.
Aperitivo.
\section{Antepectoral}
\begin{itemize}
\item {Grp. gram.:adj.}
\end{itemize}
O mesmo que \textunderscore antepeitoral\textunderscore .
\section{Antepeitoral}
\begin{itemize}
\item {Grp. gram.:adj.}
\end{itemize}
\begin{itemize}
\item {Utilização:Anat.}
\end{itemize}
Que está na parte anterior do peito.
\section{Antepenúltimo}
\begin{itemize}
\item {Grp. gram.:adj.}
\end{itemize}
Immediatamente anterior ao penúltimo.
\section{Antepoimento}
\begin{itemize}
\item {fónica:po-i}
\end{itemize}
\begin{itemize}
\item {Grp. gram.:m.}
\end{itemize}
\begin{itemize}
\item {Utilização:Ant.}
\end{itemize}
O mesmo que \textunderscore anteposição\textunderscore .
\section{Antepopa}
\begin{itemize}
\item {Grp. gram.:f.}
\end{itemize}
\begin{itemize}
\item {Utilização:Náut.}
\end{itemize}
Parte anterior da popa.
\section{Antepor}
\begin{itemize}
\item {Grp. gram.:v. t.}
\end{itemize}
\begin{itemize}
\item {Proveniência:(Lat. \textunderscore anteponere\textunderscore )}
\end{itemize}
Pôr antes; preferir.
\section{Anteporta}
\begin{itemize}
\item {Grp. gram.:f.}
\end{itemize}
Porta, que precede outra.
\section{Anteportaria}
\begin{itemize}
\item {Grp. gram.:f.}
\end{itemize}
Construcção alpendrada, á frente da portaria.
\section{Antepôrto}
\begin{itemize}
\item {Grp. gram.:m.}
\end{itemize}
Lugar abrigado, á entrada de alguns portos.
\section{Anteposição}
\begin{itemize}
\item {Grp. gram.:f.}
\end{itemize}
Acto de antepor.
\section{Anteprojecto}
\begin{itemize}
\item {Grp. gram.:m.}
\end{itemize}
Esboço de projecto.
Preliminares de um plano.
\section{Antequanto}
\begin{itemize}
\item {Grp. gram.:adv.}
\end{itemize}
\begin{itemize}
\item {Utilização:Ant.}
\end{itemize}
Em um momento; logo.
\section{Antera}
\begin{itemize}
\item {Grp. gram.:f.}
\end{itemize}
\begin{itemize}
\item {Utilização:Bot.}
\end{itemize}
\begin{itemize}
\item {Proveniência:(Gr. \textunderscore antheros\textunderscore )}
\end{itemize}
Cavidade membranosa nos estames, a qual contém o póllen antes da fecundação.
\section{Ante-real}
\begin{itemize}
\item {Grp. gram.:adj.}
\end{itemize}
Anterior á realidade.
Que precede a realidade. Cf. B. Pato, \textunderscore Livro do Monte\textunderscore .
\section{Anteríceas}
\begin{itemize}
\item {Grp. gram.:f. pl.}
\end{itemize}
Tríbo de plantas, que tira o nome de \textunderscore antérico\textunderscore .
\section{Antérico}
\begin{itemize}
\item {Grp. gram.:m.}
\end{itemize}
\begin{itemize}
\item {Proveniência:(Gr. \textunderscore antherikos\textunderscore )}
\end{itemize}
Planta liliácea, que se cultiva em estufa.
\section{Anterídea}
\begin{itemize}
\item {Grp. gram.:f.}
\end{itemize}
\begin{itemize}
\item {Proveniência:(Do gr. \textunderscore antheros\textunderscore  + \textunderscore eidos\textunderscore )}
\end{itemize}
Órgão masculino de várias plantas cryptogâmicas.
\section{Anterídio}
\begin{itemize}
\item {Grp. gram.:m.}
\end{itemize}
O mesmo que \textunderscore antherídea\textunderscore .
\section{Anterino}
\begin{itemize}
\item {Grp. gram.:adj.}
\end{itemize}
\begin{itemize}
\item {Proveniência:(Do gr. \textunderscore anthos\textunderscore )}
\end{itemize}
Que vive nas flôres.
\section{Anterior}
\begin{itemize}
\item {Grp. gram.:adj.}
\end{itemize}
\begin{itemize}
\item {Proveniência:(Lat. \textunderscore anterior\textunderscore )}
\end{itemize}
Que está adeante.
Que é primeiro, no tempo ou em lugar.
\section{Anterioridade}
\begin{itemize}
\item {Grp. gram.:f.}
\end{itemize}
Qualidade do que é anterior.
\section{Anteriormente}
\begin{itemize}
\item {Grp. gram.:adv.}
\end{itemize}
Antes; em tempo anterior.
\section{Antero-dorsal}
\begin{itemize}
\item {Grp. gram.:adj.}
\end{itemize}
Que está na parte anterior do dorso.
\section{Anterófago}
\begin{itemize}
\item {Grp. gram.:m.}
\end{itemize}
\begin{itemize}
\item {Proveniência:(Do gr. \textunderscore antheros\textunderscore  + \textunderscore phagein\textunderscore )}
\end{itemize}
Gênero de insectos coleópteros.
\section{Antero-inferior}
\begin{itemize}
\item {Grp. gram.:adj.}
\end{itemize}
Relativo á parte anterior e inferior.
\section{Antero-interior}
\begin{itemize}
\item {Grp. gram.:adj.}
\end{itemize}
Que está na parte anterior interna.
\section{Antero-posterior}
\begin{itemize}
\item {Grp. gram.:adj.}
\end{itemize}
Que vai ou está de deante para trás.
\section{Anterosoide}
\begin{itemize}
\item {Grp. gram.:m.}
\end{itemize}
\begin{itemize}
\item {Proveniência:(Do gr. \textunderscore antheros\textunderscore  + \textunderscore eidos\textunderscore )}
\end{itemize}
O mesmo que \textunderscore zoósporo\textunderscore .
\section{Ante-rosto}
\begin{itemize}
\item {Grp. gram.:m.}
\end{itemize}
Página, que precede o frontispicio de uma obra, e que geralmente só contém o título da mesma obra.
\section{Antero-superior}
\begin{itemize}
\item {Grp. gram.:adj.}
\end{itemize}
Situado na parte anterior e superior.
\section{Anterura}
\begin{itemize}
\item {Grp. gram.:f.}
\end{itemize}
\begin{itemize}
\item {Proveniência:(Do gr. \textunderscore antheros\textunderscore  + \textunderscore oura\textunderscore )}
\end{itemize}
Gênero de rubiáceas.
\section{Antes}
\begin{itemize}
\item {Grp. gram.:adv.}
\end{itemize}
\begin{itemize}
\item {Grp. gram.:Loc. prep.}
\end{itemize}
\begin{itemize}
\item {Grp. gram.:Loc. adv.}
\end{itemize}
\begin{itemize}
\item {Grp. gram.:Loc. adv.}
\end{itemize}
\begin{itemize}
\item {Utilização:Pop.}
\end{itemize}
\begin{itemize}
\item {Proveniência:(Lat. \textunderscore ante\textunderscore )}
\end{itemize}
Em tempo anterior; precedentemente: \textunderscore para saber, estudou antes\textunderscore .
De preferência: \textunderscore antes suar que tremer\textunderscore .
Pelo contrario: \textunderscore não lhe tenho ódio, antes o estimo\textunderscore .
\textunderscore Antes de\textunderscore , primeiro que.
\textunderscore De antes\textunderscore  ou \textunderscore d'antes\textunderscore , antigamente.
\textunderscore Em antes\textunderscore , o mesmo que \textunderscore dantes\textunderscore , ou \textunderscore antes\textunderscore . Cf. Júl. Dinis, \textunderscore Pupillas\textunderscore , 21.
\textunderscore Antes que\textunderscore , ainda que: \textunderscore há de negar sempre, antes que o matem\textunderscore .
\section{Antesala}
\begin{itemize}
\item {fónica:sá}
\end{itemize}
\begin{itemize}
\item {Grp. gram.:f.}
\end{itemize}
Sala, que precede a principal.
Sala de espera.
Antecâmara.
\section{Ante-sazão}
\begin{itemize}
\item {Grp. gram.:loc. adv.}
\end{itemize}
Prematuramente.
Antecipadamente.
Antes do tempo próprio.
\section{Ante-scena}
\begin{itemize}
\item {Grp. gram.:f.}
\end{itemize}
Parte do theatro, á frente da scena. Cf. Garrett, \textunderscore Filippa\textunderscore , 83.
\section{Antese}
\begin{itemize}
\item {Grp. gram.:f.}
\end{itemize}
\begin{itemize}
\item {Proveniência:(Gr. \textunderscore anthesis\textunderscore )}
\end{itemize}
O desabrochar das flôres.
\section{Antesentir}
\begin{itemize}
\item {fónica:sen}
\end{itemize}
\begin{itemize}
\item {Grp. gram.:v. t.}
\end{itemize}
O mesmo que \textunderscore presentir\textunderscore .
\section{Antesigma}
\begin{itemize}
\item {Grp. gram.:m.}
\end{itemize}
Letra, que o imperador Cláudio accrescentou ao alphabeto latino.
\section{Antesignano}
\begin{itemize}
\item {fónica:si}
\end{itemize}
\begin{itemize}
\item {Grp. gram.:m.}
\end{itemize}
\begin{itemize}
\item {Proveniência:(Lat. \textunderscore antesignanus\textunderscore )}
\end{itemize}
Porta-bandeira, na milícia romana.
\section{Antesocrático}
\begin{itemize}
\item {fónica:so}
\end{itemize}
\begin{itemize}
\item {Grp. gram.:adj.}
\end{itemize}
Anterior aos tempos de Sócrates.
\section{Antessala}
\begin{itemize}
\item {Grp. gram.:f.}
\end{itemize}
Sala, que precede a principal.
Sala de espera.
Antecâmara.
\section{Antessentir}
\begin{itemize}
\item {Grp. gram.:v. t.}
\end{itemize}
O mesmo que \textunderscore pressentir\textunderscore .
\section{Antessignano}
\begin{itemize}
\item {Grp. gram.:m.}
\end{itemize}
\begin{itemize}
\item {Proveniência:(Lat. \textunderscore antesignanus\textunderscore )}
\end{itemize}
Porta-bandeira, na milícia romana.
\section{Antessocrático}
\begin{itemize}
\item {Grp. gram.:adj.}
\end{itemize}
Anterior aos tempos de Sócrates.
\section{Antestatura}
\begin{itemize}
\item {Grp. gram.:f.}
\end{itemize}
Trincheira ou reparo improvisado, para se disputar o terreno perdido ou defender rapidamente uma passagem.
\section{Antes-tempo}
\begin{itemize}
\item {Grp. gram.:loc. adv.}
\end{itemize}
O mesmo que \textunderscore antetempo\textunderscore . Cf. Filinto, \textunderscore D. Man.\textunderscore , II, 148.
\section{Antestérias}
\begin{itemize}
\item {Grp. gram.:f. pl.}
\end{itemize}
Antigas festas atenienses, em honra de Baccho.
\section{Antestério}
\begin{itemize}
\item {Grp. gram.:m.}
\end{itemize}
Quinto mês do antigo anno ateniense.
\section{Antetempo}
\begin{itemize}
\item {Grp. gram.:adv.}
\end{itemize}
Prematuramente; antes do tempo próprio. Cf. Filinto, \textunderscore D. Man.\textunderscore , II, 161.
\section{Ante-terminal}
\begin{itemize}
\item {Grp. gram.:adj.}
\end{itemize}
Que está antes da extremidade.
\section{Antever}
\begin{itemize}
\item {Grp. gram.:v. t.}
\end{itemize}
Ver antes, prever.
\section{Anteversão}
\begin{itemize}
\item {Grp. gram.:f.}
\end{itemize}
\begin{itemize}
\item {Utilização:Med.}
\end{itemize}
\begin{itemize}
\item {Proveniência:(Lat. \textunderscore anteversio\textunderscore )}
\end{itemize}
Acção de anteverter.
Inclinação do fundo do útero para deante.
\section{Anteverter}
\begin{itemize}
\item {Grp. gram.:v. t.}
\end{itemize}
\begin{itemize}
\item {Grp. gram.:V. i.}
\end{itemize}
\begin{itemize}
\item {Proveniência:(Lat. \textunderscore antevertere\textunderscore )}
\end{itemize}
Inclinar para deante.
Preceder.
Preferir.
Avançar, ir adeante.
\section{Antevéspera}
\begin{itemize}
\item {Grp. gram.:f.}
\end{itemize}
Dia anterior á vespera.
\section{Antevidência}
\begin{itemize}
\item {Grp. gram.:f.}
\end{itemize}
Acto de \textunderscore antever\textunderscore .
\section{Antevidente}
\begin{itemize}
\item {Grp. gram.:adj.}
\end{itemize}
\begin{itemize}
\item {Proveniência:(Do lat. \textunderscore ante\textunderscore  + \textunderscore videns\textunderscore )}
\end{itemize}
Que vê antes.
Previdente.
\section{Antevieiro}
\begin{itemize}
\item {Grp. gram.:adj.}
\end{itemize}
\begin{itemize}
\item {Proveniência:(De \textunderscore ante\textunderscore  + \textunderscore via\textunderscore )}
\end{itemize}
Metediço, intrometido.
\section{Antevisão}
\begin{itemize}
\item {Grp. gram.:f.}
\end{itemize}
Acto de \textunderscore antever\textunderscore . Cf. Serpa, \textunderscore Da Nacionalidade\textunderscore , 34.
\section{Antevoar}
\begin{itemize}
\item {Grp. gram.:v. i.}
\end{itemize}
Voar á frente. Cf. Filinto, IX, 281.
\section{Antevocálico}
\begin{itemize}
\item {Grp. gram.:adj.}
\end{itemize}
\begin{itemize}
\item {Utilização:Gram.}
\end{itemize}
Que está antes de uma vogal.
\section{Anthélia}
\begin{itemize}
\item {Grp. gram.:f.}
\end{itemize}
O mesmo que \textunderscore anthélio\textunderscore .
\section{Anthélice}
\begin{itemize}
\item {Grp. gram.:m.}
\end{itemize}
\begin{itemize}
\item {Utilização:Anat.}
\end{itemize}
\begin{itemize}
\item {Proveniência:(Do gr. \textunderscore anti\textunderscore  + \textunderscore helix\textunderscore )}
\end{itemize}
Elevação do pavilhão da orelha, antes do hélice.
\section{Anthélio}
\begin{itemize}
\item {Grp. gram.:m.}
\end{itemize}
\begin{itemize}
\item {Proveniência:(Do gr. \textunderscore anti\textunderscore  + \textunderscore helios\textunderscore )}
\end{itemize}
Claridade reflectida pelo sol no lado opposto a êste astro.
\section{Anthelmíntico}
\begin{itemize}
\item {Grp. gram.:adj.}
\end{itemize}
(V.antihelmíntico)
\section{Anthelo}
\begin{itemize}
\item {Grp. gram.:adj.}
\end{itemize}
Diz-se da inflorescência dos juncos.
\section{Anthema}
\begin{itemize}
\item {Grp. gram.:f.}
\end{itemize}
Gênero de plantas malváceas.
\section{Anthemídeas}
\begin{itemize}
\item {Grp. gram.:f. pl.}
\end{itemize}
\begin{itemize}
\item {Proveniência:(Do gr. \textunderscore anthemis\textunderscore  + \textunderscore eidos\textunderscore )}
\end{itemize}
Tríbo de plantas, que tem por typo a ânthemis.
\section{Anthemina}
\begin{itemize}
\item {Grp. gram.:f.}
\end{itemize}
\begin{itemize}
\item {Proveniência:(De \textunderscore ânthemis\textunderscore )}
\end{itemize}
Princípio estimulante da camomila.
\section{Ânthemis}
\begin{itemize}
\item {Grp. gram.:f.}
\end{itemize}
\begin{itemize}
\item {Proveniência:(Gr. \textunderscore anthemis\textunderscore )}
\end{itemize}
Nome scientífico da camomila; macella.
\section{Antéphora}
\begin{itemize}
\item {Grp. gram.:f.}
\end{itemize}
Gênero de plantas gramíneas.
\section{Anthera}
\begin{itemize}
\item {Grp. gram.:f.}
\end{itemize}
\begin{itemize}
\item {Utilização:Bot.}
\end{itemize}
\begin{itemize}
\item {Proveniência:(Gr. \textunderscore antheros\textunderscore )}
\end{itemize}
Cavidade membranosa nos estames, a qual contém o póllen antes da fecundação.
\section{Antheríceas}
\begin{itemize}
\item {Grp. gram.:f. pl.}
\end{itemize}
Tríbo de plantas, que tira o nome de \textunderscore anthérico\textunderscore .
\section{Anthérico}
\begin{itemize}
\item {Grp. gram.:m.}
\end{itemize}
\begin{itemize}
\item {Proveniência:(Gr. \textunderscore antherikos\textunderscore )}
\end{itemize}
Planta liliácea, que se cultiva em estufa.
\section{Antherídea}
\begin{itemize}
\item {Grp. gram.:f.}
\end{itemize}
\begin{itemize}
\item {Proveniência:(Do gr. \textunderscore antheros\textunderscore  + \textunderscore eidos\textunderscore )}
\end{itemize}
Órgão masculino de várias plantas cryptogâmicas.
\section{Antherídio}
\begin{itemize}
\item {Grp. gram.:m.}
\end{itemize}
O mesmo que \textunderscore antherídea\textunderscore .
\section{Antherino}
\begin{itemize}
\item {Grp. gram.:adj.}
\end{itemize}
\begin{itemize}
\item {Proveniência:(Do gr. \textunderscore anthos\textunderscore )}
\end{itemize}
Que vive nas flôres.
\section{Antheróphago}
\begin{itemize}
\item {Grp. gram.:m.}
\end{itemize}
\begin{itemize}
\item {Proveniência:(Do gr. \textunderscore antheros\textunderscore  + \textunderscore phagein\textunderscore )}
\end{itemize}
Gênero de insectos coleópteros.
\section{Antherosoide}
\begin{itemize}
\item {Grp. gram.:m.}
\end{itemize}
\begin{itemize}
\item {Proveniência:(Do gr. \textunderscore antheros\textunderscore  + \textunderscore eidos\textunderscore )}
\end{itemize}
O mesmo que \textunderscore zoósporo\textunderscore .
\section{Antherura}
\begin{itemize}
\item {Grp. gram.:f.}
\end{itemize}
\begin{itemize}
\item {Proveniência:(Do gr. \textunderscore antheros\textunderscore  + \textunderscore oura\textunderscore )}
\end{itemize}
Gênero de rubiáceas.
\section{Anthese}
\begin{itemize}
\item {Grp. gram.:f.}
\end{itemize}
\begin{itemize}
\item {Proveniência:(Gr. \textunderscore anthesis\textunderscore )}
\end{itemize}
O desabrochar das flôres.
\section{Anthestérias}
\begin{itemize}
\item {Grp. gram.:f. pl.}
\end{itemize}
Antigas festas athenienses, em honra de Baccho.
\section{Anthestério}
\begin{itemize}
\item {Grp. gram.:m.}
\end{itemize}
Quinto mês do antigo anno atheniense.
\section{Anthia}
\begin{itemize}
\item {Grp. gram.:m.}
\end{itemize}
\begin{itemize}
\item {Proveniência:(Do gr. \textunderscore anthas\textunderscore )}
\end{itemize}
Gênero de insectos coleópteros pentâmeros.
\section{Anthídea}
\begin{itemize}
\item {Grp. gram.:f.}
\end{itemize}
\begin{itemize}
\item {Proveniência:(Do gr. \textunderscore anthedon\textunderscore )}
\end{itemize}
Insecto hymenóptero mellífero.
\section{Anthina}
\begin{itemize}
\item {Grp. gram.:f.}
\end{itemize}
\begin{itemize}
\item {Proveniência:(De \textunderscore anihino\textunderscore )}
\end{itemize}
Gênero de cogumelos.
\section{Anthino}
\begin{itemize}
\item {Grp. gram.:adj.}
\end{itemize}
\begin{itemize}
\item {Proveniência:(Do gr. \textunderscore anthos\textunderscore )}
\end{itemize}
Que contém flôres.
\section{Anthóbio}
\begin{itemize}
\item {Grp. gram.:m.}
\end{itemize}
Gênero de coleópteros.
\section{Anthobrânchio}
\begin{itemize}
\item {fónica:qui}
\end{itemize}
\begin{itemize}
\item {Grp. gram.:adj.}
\end{itemize}
Diz-se dos molluscos, cujas brânchias semelham ramalhetes de flôres.
\section{Anthocéphalo}
\begin{itemize}
\item {Grp. gram.:m.}
\end{itemize}
\begin{itemize}
\item {Utilização:Bot.}
\end{itemize}
Gênero de rubiáceas.
\section{Anthócera}
\begin{itemize}
\item {Grp. gram.:f.}
\end{itemize}
\begin{itemize}
\item {Utilização:Bot.}
\end{itemize}
Gênero de hepáticas.
\section{Anthoclâmide}
\begin{itemize}
\item {Grp. gram.:f.}
\end{itemize}
\begin{itemize}
\item {Utilização:Bot.}
\end{itemize}
Gênero de herbáceas.
\section{Anthocyanina}
\begin{itemize}
\item {Grp. gram.:f.}
\end{itemize}
Substância còrante das flôres rubras, rosadas ou azues.
\section{Anthodendro}
\begin{itemize}
\item {Grp. gram.:m.}
\end{itemize}
\begin{itemize}
\item {Utilização:Bot.}
\end{itemize}
Gênero de hericáceas.
\section{Anthographia}
\begin{itemize}
\item {Grp. gram.:f.}
\end{itemize}
\begin{itemize}
\item {Proveniência:(Do gr. \textunderscore anthos\textunderscore  + \textunderscore graphein\textunderscore )}
\end{itemize}
Linguagem das flôres.
\section{Anthographico}
\begin{itemize}
\item {Grp. gram.:adj.}
\end{itemize}
Relativo á anthographia.
\section{Anthógrapho}
\begin{itemize}
\item {Grp. gram.:m.}
\end{itemize}
Aquelle que é versado em anthographia.
\section{Anthologia}
\begin{itemize}
\item {Grp. gram.:f.}
\end{itemize}
\begin{itemize}
\item {Utilização:Fig.}
\end{itemize}
\begin{itemize}
\item {Proveniência:(Gr. \textunderscore anthologia\textunderscore )}
\end{itemize}
Tratado das flôres.
Collecção de flôres.
Escolha, collecção de poesias.
Collecção de trechos em prosa e verso.
Selecta; chrestomathia.
\section{Anthologista}
\begin{itemize}
\item {Grp. gram.:m.}
\end{itemize}
Aquelle que é versado em anthologia.
Colleccionador de poesias.
\section{Anthóloma}
\begin{itemize}
\item {Grp. gram.:m.}
\end{itemize}
Gênero de plantas tiliáceas.
\section{Antholyza}
\begin{itemize}
\item {Grp. gram.:f.}
\end{itemize}
\begin{itemize}
\item {Proveniência:(Do gr. \textunderscore anthos\textunderscore  + \textunderscore lussa\textunderscore )}
\end{itemize}
Gênero de plantas irídeas.
\section{Anthomania}
\begin{itemize}
\item {Grp. gram.:f.}
\end{itemize}
\begin{itemize}
\item {Proveniência:(Do gr. \textunderscore anthos\textunderscore  + \textunderscore mania\textunderscore )}
\end{itemize}
Paixão pelas flôres.
\section{Anthomaniaco}
\begin{itemize}
\item {Grp. gram.:m.  e  adj.}
\end{itemize}
O que tem \textunderscore anthomania\textunderscore .
\section{Anthómano}
\begin{itemize}
\item {Grp. gram.:m.  e  adj.}
\end{itemize}
O mesmo que \textunderscore anthomaniaco\textunderscore .
\section{Anthomyzídeos}
\begin{itemize}
\item {Grp. gram.:m. pl.}
\end{itemize}
\begin{itemize}
\item {Proveniência:(Do gr. \textunderscore anthos\textunderscore  + \textunderscore mizein\textunderscore )}
\end{itemize}
Insectos dípteros, semelhantes ás moscas ordinárias.
\section{Anthomizidos}
\begin{itemize}
\item {Grp. gram.:m. pl.}
\end{itemize}
O mesmo que \textunderscore anthomyzídeos\textunderscore .
\section{Anthóphago}
\begin{itemize}
\item {Grp. gram.:adj.}
\end{itemize}
\begin{itemize}
\item {Proveniência:(Do gr. \textunderscore anthos\textunderscore  + \textunderscore phagein\textunderscore )}
\end{itemize}
Que come flôres.
\section{Anthóphila}
\begin{itemize}
\item {Grp. gram.:f.}
\end{itemize}
\begin{itemize}
\item {Utilização:Entom.}
\end{itemize}
Gênero de lepidópteros.
\section{Anthóphilo}
\begin{itemize}
\item {Grp. gram.:adj.}
\end{itemize}
\begin{itemize}
\item {Proveniência:(Do gr. \textunderscore anthos\textunderscore  + \textunderscore philos\textunderscore )}
\end{itemize}
Que é amigo das flôres.
Que está habitualmente nas flôres.
\section{Anthóphilos}
\begin{itemize}
\item {Grp. gram.:m. pl.}
\end{itemize}
\begin{itemize}
\item {Proveniência:(Do gr. \textunderscore anthos\textunderscore  + \textunderscore philos\textunderscore )}
\end{itemize}
Nome, que os naturalistas deram a uma família de insectos, com quatro asas venenosas, estendidas, antenas filiformes, abdome redondo e lábio curto.
\section{Anthóphoro}
\begin{itemize}
\item {Grp. gram.:m.}
\end{itemize}
\begin{itemize}
\item {Utilização:Bot.}
\end{itemize}
\begin{itemize}
\item {Grp. gram.:M. pl.}
\end{itemize}
\begin{itemize}
\item {Utilização:Entom.}
\end{itemize}
\begin{itemize}
\item {Proveniência:(Do gr. \textunderscore anthos\textunderscore  + \textunderscore pherein\textunderscore )}
\end{itemize}
Receptáculo floral que, partindo do fundo do cálice, sustenta as pétalas, os estames e o pistillo, segundo De-Candolle.
Insectos, da tribo dos apiários.
\section{Anthopogão}
\begin{itemize}
\item {Grp. gram.:m.}
\end{itemize}
Gênero de gramíneas.
\section{Anthorismo}
\begin{itemize}
\item {Grp. gram.:m.}
\end{itemize}
\begin{itemize}
\item {Utilização:Rhet.}
\end{itemize}
\begin{itemize}
\item {Proveniência:(Do gr. \textunderscore anti\textunderscore  + \textunderscore horismos\textunderscore )}
\end{itemize}
Substituição de uma palavra por outra, que se considera mais enérgica ou mais exacta.
\section{Anthoro}
\begin{itemize}
\item {Grp. gram.:m.}
\end{itemize}
Planta rainunculácea, cujos sucos são venenosos.
(Contr. de \textunderscore anti\textunderscore  + lat. \textunderscore thora\textunderscore )
\section{Anthosoma}
\begin{itemize}
\item {Grp. gram.:f.}
\end{itemize}
Gênero de crustáceos.
\section{Anthosperma}
\begin{itemize}
\item {Grp. gram.:m.}
\end{itemize}
\begin{itemize}
\item {Proveniência:(Do gr. \textunderscore anthos\textunderscore  + \textunderscore sperma\textunderscore )}
\end{itemize}
Pequenas concreções còradas, dispersas no tecido de certas plantas.
\section{Anthospermo}
\begin{itemize}
\item {Grp. gram.:m.}
\end{itemize}
\begin{itemize}
\item {Utilização:Bot.}
\end{itemize}
Gênero de rubiáceas.
(Cp. \textunderscore anthosperma\textunderscore )
\section{Anthostema}
\begin{itemize}
\item {Grp. gram.:f.}
\end{itemize}
\begin{itemize}
\item {Proveniência:(Do gr. \textunderscore anthos\textunderscore  + \textunderscore stema\textunderscore )}
\end{itemize}
Gênero de plantas euphorbiáceas.
\section{Anthóstomo}
\begin{itemize}
\item {Grp. gram.:adj.}
\end{itemize}
\begin{itemize}
\item {Utilização:Hist. Nat.}
\end{itemize}
\begin{itemize}
\item {Proveniência:(Do gr. \textunderscore anthos\textunderscore  + \textunderscore stoma\textunderscore )}
\end{itemize}
Que tem, á volta da boca, apendices que dão aspecto de flôr.
\section{Anthoxantheína}
\begin{itemize}
\item {Grp. gram.:f.}
\end{itemize}
O mesmo que \textunderscore anthoxanthina\textunderscore .
\section{Anthoxanthina}
\begin{itemize}
\item {Grp. gram.:f.}
\end{itemize}
Substância còrante das flôres amarelas.
\section{Anthoxhantho}
\begin{itemize}
\item {Grp. gram.:m.}
\end{itemize}
\begin{itemize}
\item {Proveniência:(Do gr. \textunderscore anthos\textunderscore  + \textunderscore xanthos\textunderscore )}
\end{itemize}
Nome scientifico de uma planta gramínea, vulgarmente conhecida por \textunderscore feno-de-cheiro\textunderscore .
\section{Anthozoários}
\begin{itemize}
\item {Grp. gram.:m. pl.}
\end{itemize}
\begin{itemize}
\item {Proveniência:(Do gr. \textunderscore anthos\textunderscore  + \textunderscore zoarion\textunderscore )}
\end{itemize}
Família de polypeiros.
\section{Anthracena}
\begin{itemize}
\item {Grp. gram.:f.}
\end{itemize}
O mesmo que \textunderscore anthracina\textunderscore .
\section{Anthracia}
\begin{itemize}
\item {Grp. gram.:f.}
\end{itemize}
Affecção análoga ao anthraz.
\section{Anthrácico}
\begin{itemize}
\item {Grp. gram.:adj.}
\end{itemize}
O mesmo que \textunderscore anthrácino\textunderscore .
\section{Anthracífero}
\begin{itemize}
\item {Grp. gram.:adj.}
\end{itemize}
\begin{itemize}
\item {Proveniência:(Do gr. \textunderscore anthrax\textunderscore  + lat. \textunderscore ferre\textunderscore )}
\end{itemize}
Que tem anthracita.
\section{Anthraciforme}
\begin{itemize}
\item {Grp. gram.:adj.}
\end{itemize}
Que tem a apparência de anthraz.
\section{Anthracina}
\begin{itemize}
\item {Grp. gram.:f.}
\end{itemize}
\begin{itemize}
\item {Proveniência:(Do gr. \textunderscore anthrax\textunderscore )}
\end{itemize}
Substância, que se obtém pela destillação do alcatrão de hulha.
\section{Antracina}
\begin{itemize}
\item {Grp. gram.:f.}
\end{itemize}
\begin{itemize}
\item {Proveniência:(Do gr. \textunderscore anthrax\textunderscore )}
\end{itemize}
Substância, que se obtém pela destillação do alcatrão de hulha.
\section{Anthrácino}
\begin{itemize}
\item {Grp. gram.:adj.}
\end{itemize}
\begin{itemize}
\item {Proveniência:(Lat. \textunderscore anthracinus\textunderscore )}
\end{itemize}
Relativo ao anthraz.
\section{Anthracita}
\begin{itemize}
\item {Grp. gram.:f.}
\end{itemize}
\begin{itemize}
\item {Proveniência:(Do gr. \textunderscore anthrax\textunderscore )}
\end{itemize}
Carvão mineral, que arde com difficuldade, sem fumo nem cheiro.
\section{Anthracitoso}
\begin{itemize}
\item {Grp. gram.:adj.}
\end{itemize}
Que contém \textunderscore anthracite\textunderscore .
\section{Anthracnose}
\begin{itemize}
\item {Grp. gram.:f.}
\end{itemize}
\begin{itemize}
\item {Proveniência:(Do gr. \textunderscore anthrax\textunderscore )}
\end{itemize}
Cogumelo parasito, ainda pouco conhecido, que ataca os rebentos das videiras.
Doença das vinhas, determinada pela acção daquelle parasito.
\section{Anthracóide}
\begin{itemize}
\item {Grp. gram.:adj.}
\end{itemize}
\begin{itemize}
\item {Proveniência:(Do gr. \textunderscore anthrax\textunderscore  + \textunderscore eidos\textunderscore )}
\end{itemize}
Que tem a côr de carvão.
Que é semelhante ao anthraz.
\section{Anthracomancia}
\begin{itemize}
\item {Grp. gram.:f.}
\end{itemize}
\begin{itemize}
\item {Proveniência:(Do gr. \textunderscore anthrax\textunderscore  + \textunderscore manteia\textunderscore )}
\end{itemize}
Adivinhação pelo exame do carvão encandescente.
\section{Anthracómetro}
\begin{itemize}
\item {Grp. gram.:m.}
\end{itemize}
\begin{itemize}
\item {Proveniência:(Do gr. \textunderscore anthrax\textunderscore  + \textunderscore metron\textunderscore )}
\end{itemize}
Instrumento, para determinar a quantidade de ácido carbónico, contido num fluido aeriforme.
\section{Anthraconito}
\begin{itemize}
\item {Grp. gram.:m.}
\end{itemize}
\begin{itemize}
\item {Proveniência:(Do gr. \textunderscore anthrax\textunderscore )}
\end{itemize}
Uma das variedades de carbonato de cal.
\section{Anthracose}
\begin{itemize}
\item {Grp. gram.:f.}
\end{itemize}
\begin{itemize}
\item {Proveniência:(Gr. \textunderscore anthracosis\textunderscore )}
\end{itemize}
Doença nos pulmões ou nos brônchios, caracterizada pela presença de uma substância escura, que tem os caracteres do carvão.
\section{Anthracotério}
\begin{itemize}
\item {Grp. gram.:m.}
\end{itemize}
Gênero de mamíferos fósseis.
\section{Anthranílico}
\begin{itemize}
\item {Grp. gram.:adj.}
\end{itemize}
Diz-se de um ácido, obtido pela acção da potassa sobre o indigo.
\section{Anthraz}
\begin{itemize}
\item {Grp. gram.:m.}
\end{itemize}
\begin{itemize}
\item {Proveniência:(Gr. \textunderscore anthrax\textunderscore )}
\end{itemize}
Carbúnculo, tumor gangrenoso e inflammatório.
Insecto díptero.
\section{Anthrenos}
\begin{itemize}
\item {Grp. gram.:m. pl.}
\end{itemize}
\begin{itemize}
\item {Proveniência:(Do gr. \textunderscore antho\textunderscore  + \textunderscore rainein\textunderscore )}
\end{itemize}
Insectos coleópteros cujas larvas atacam as pelles e as collecções entomológicas.
\section{Anthromia}
\begin{itemize}
\item {Grp. gram.:f.}
\end{itemize}
Môsca alongada, que deposita seus ovos em substâncias gordas, especialmente nos queijos, produzindo graves irritações intestinaes, (\textunderscore anthromeya errabunda\textunderscore , Lottiez).
\section{Anthropagogia}
\begin{itemize}
\item {Grp. gram.:f.}
\end{itemize}
\begin{itemize}
\item {Proveniência:(Do gr. \textunderscore anthropos\textunderscore  + \textunderscore agoge\textunderscore )}
\end{itemize}
Pedagogia social, tendente a alargar a acção educativa para fóra da escola e da família.
\section{Anthropeiano}
\begin{itemize}
\item {Grp. gram.:adj.}
\end{itemize}
\begin{itemize}
\item {Utilização:Geol.}
\end{itemize}
\begin{itemize}
\item {Proveniência:(Do gr. \textunderscore anthropeios\textunderscore )}
\end{itemize}
Diz-se do terreno coetâneo do apparecimento do homem.
\section{Anthropina}
\begin{itemize}
\item {Grp. gram.:f.}
\end{itemize}
\begin{itemize}
\item {Proveniência:(Do gr. \textunderscore anthropos\textunderscore )}
\end{itemize}
Mistura de estearina e palmitina, extraída da gordura humana.
\section{Anthropocêntrico}
\begin{itemize}
\item {Grp. gram.:adj.}
\end{itemize}
Diz-se do systema philosóphico, segundo o qual o homem é o centro de todo o universo, sendo-lhe por isso subordinadas todas as coisas e para elle criadas. Cf. Latino. \textunderscore Or. da Corôa\textunderscore , CXXXV.
\section{Anthropocentrismo}
\begin{itemize}
\item {Grp. gram.:m.}
\end{itemize}
Doutrina anthropocêntrica.
\section{Anthropocentrista}
\begin{itemize}
\item {Grp. gram.:m.}
\end{itemize}
Sectário do anthropocentrismo.
\section{Anthropodiceia}
\begin{itemize}
\item {Grp. gram.:f.}
\end{itemize}
A justiça dos homens. Cf. Castilho, \textunderscore Avarento\textunderscore , 363.
\section{Anthropoforme}
\begin{itemize}
\item {Grp. gram.:adj.}
\end{itemize}
\begin{itemize}
\item {Proveniência:(Do gr. \textunderscore anthropos\textunderscore  + lat. \textunderscore forma\textunderscore )}
\end{itemize}
Semelhante ao homem.
\section{Anthropófugo}
\begin{itemize}
\item {Grp. gram.:m.  e  adj.}
\end{itemize}
\begin{itemize}
\item {Proveniência:(Do gr. \textunderscore anthropos\textunderscore  + lat. \textunderscore fugere\textunderscore )}
\end{itemize}
O mesmo que \textunderscore anthropóphobo\textunderscore .
\section{Anthropogenesia}
\begin{itemize}
\item {Grp. gram.:f.}
\end{itemize}
\begin{itemize}
\item {Proveniência:(Gr. \textunderscore anthropogenesis\textunderscore )}
\end{itemize}
Sciência da geração humana.
Tratado dos phenómenos da reproducção do homem.
\section{Anthropogenésico}
\begin{itemize}
\item {Grp. gram.:adj.}
\end{itemize}
Relativo á \textunderscore anthropogenesia\textunderscore .
\section{Anthropogenia}
\begin{itemize}
\item {Grp. gram.:f.}
\end{itemize}
O mesmo que \textunderscore anthropogenesia\textunderscore .
\section{Anthropogênico}
\begin{itemize}
\item {Grp. gram.:adj.}
\end{itemize}
Relativo á \textunderscore anthropogenia\textunderscore .
\section{Anthropoglossa}
\begin{itemize}
\item {Grp. gram.:f.}
\end{itemize}
\begin{itemize}
\item {Utilização:Mús.}
\end{itemize}
\begin{itemize}
\item {Utilização:ant.}
\end{itemize}
\begin{itemize}
\item {Proveniência:(Do gr. \textunderscore anthropos\textunderscore  + \textunderscore glossa\textunderscore )}
\end{itemize}
Registo que, no órgão, se chama hoje \textunderscore voz humana\textunderscore .
\section{Anthropoglyphita}
\begin{itemize}
\item {Grp. gram.:f.}
\end{itemize}
\begin{itemize}
\item {Proveniência:(Do gr. \textunderscore anthropos\textunderscore  + \textunderscore gluphos\textunderscore )}
\end{itemize}
Rocha, cuja configuração natural dá o aspecto de um homem ou de uma cara.
\section{Anthropognosia}
\begin{itemize}
\item {Grp. gram.:f.}
\end{itemize}
Conhecimento da natureza phýsica do homem.
\section{Anthropographia}
\begin{itemize}
\item {Grp. gram.:f.}
\end{itemize}
\begin{itemize}
\item {Proveniência:(Do gr. \textunderscore anthropos\textunderscore  + \textunderscore graphein\textunderscore )}
\end{itemize}
Descripção do homem, como animal.
\section{Anthropogrypho}
\begin{itemize}
\item {Grp. gram.:m.}
\end{itemize}
Homem alado, ou ave com fórma de homem. Cf. A. Dinis, \textunderscore Hyssope\textunderscore , 118.
\section{Anthropoide}
\begin{itemize}
\item {Grp. gram.:adj.}
\end{itemize}
\begin{itemize}
\item {Grp. gram.:M.}
\end{itemize}
\begin{itemize}
\item {Proveniência:(Do gr. \textunderscore anthropos\textunderscore  + \textunderscore eidos\textunderscore )}
\end{itemize}
Semelhante ao homem.
Sêr, imaginado por alguns anthropólogos, como transição do animal para o homem.
\section{Anthropólatra}
\begin{itemize}
\item {Grp. gram.:m.}
\end{itemize}
Aquelle que pratica a anthropolatria.
\section{Anthropolatria}
\begin{itemize}
\item {Grp. gram.:f.}
\end{itemize}
\begin{itemize}
\item {Proveniência:(Do gr. \textunderscore anthropos\textunderscore  + \textunderscore latreia\textunderscore )}
\end{itemize}
Adoração do homem.
\section{Anthropolátrico}
\begin{itemize}
\item {Grp. gram.:adj.}
\end{itemize}
Relativo a anthropolatria.
\section{Anthropólitho}
\begin{itemize}
\item {Grp. gram.:m.}
\end{itemize}
\begin{itemize}
\item {Proveniência:(Do gr. \textunderscore anthropos\textunderscore  + \textunderscore lithos\textunderscore )}
\end{itemize}
Ossos humanos fósseis.
\section{Anthropologia}
\begin{itemize}
\item {Grp. gram.:f.}
\end{itemize}
\begin{itemize}
\item {Proveniência:(De \textunderscore anthropólogo\textunderscore )}
\end{itemize}
História natural do homem.
Locução figurada, que atribue a Deus acções ou qualidades humanas.
\section{Anthropológico}
\begin{itemize}
\item {Grp. gram.:adj.}
\end{itemize}
Relativo á \textunderscore anthropologia\textunderscore .
\section{Anthropologista}
\begin{itemize}
\item {Grp. gram.:f.}
\end{itemize}
Professor ou tratadista de anthropologia.
\section{Anthropólogo}
\begin{itemize}
\item {Grp. gram.:m.}
\end{itemize}
\begin{itemize}
\item {Proveniência:(Do gr. \textunderscore anthropos\textunderscore  + \textunderscore logos\textunderscore )}
\end{itemize}
Aquelle que é versado em anthropologia.
\section{Anthropomancia}
\begin{itemize}
\item {Grp. gram.:f.}
\end{itemize}
\begin{itemize}
\item {Proveniência:(Do gr. \textunderscore anthropos\textunderscore  + \textunderscore manteia\textunderscore )}
\end{itemize}
Processo de adivinhação, usado antigamente, observando-se as entranhas de uma criança ou de um homem recentemente degolado.
\section{Anthropometria}
\begin{itemize}
\item {Grp. gram.:f.}
\end{itemize}
\begin{itemize}
\item {Proveniência:(Do gr. \textunderscore anthropos\textunderscore  + \textunderscore metron\textunderscore )}
\end{itemize}
Estudo comparativo das proporções das differentes partes do corpo humano.
\section{Anthropométrico}
\begin{itemize}
\item {Grp. gram.:adj.}
\end{itemize}
Relativo á anthropometria.
\section{Anthropomórphico}
\begin{itemize}
\item {Grp. gram.:adj.}
\end{itemize}
O mesmo que \textunderscore anthropomorpho\textunderscore .
\section{Anthropomorphismo}
\begin{itemize}
\item {Grp. gram.:m.}
\end{itemize}
\begin{itemize}
\item {Proveniência:(Do gr. \textunderscore anthropos\textunderscore  + \textunderscore morphe\textunderscore )}
\end{itemize}
Systema dos que attribuem a Deus acções ou faculdades humanas.
\section{Anthropomorphista}
\begin{itemize}
\item {Grp. gram.:m.  e  adj.}
\end{itemize}
Sectário do anthropomorphismo.
\section{Anthropomorpho}
\begin{itemize}
\item {Grp. gram.:adj.}
\end{itemize}
\begin{itemize}
\item {Proveniência:(Do gr. \textunderscore anthropos\textunderscore  + \textunderscore morphe\textunderscore )}
\end{itemize}
Que é semelhante ao homem.
(T., proposto por Littré, em substituição do neol. hýbrido \textunderscore anthropoforme\textunderscore ).
\section{Anthroponomia}
\begin{itemize}
\item {Grp. gram.:f.}
\end{itemize}
\begin{itemize}
\item {Proveniência:(Do gr. \textunderscore anthropos\textunderscore  + \textunderscore nomos\textunderscore )}
\end{itemize}
Sciência da formação do homem.
\section{Anthroponómico}
\begin{itemize}
\item {Grp. gram.:adj.}
\end{itemize}
Relativo á anthroponomia.
\section{Anthropophagia}
\begin{itemize}
\item {Grp. gram.:f.}
\end{itemize}
Estado de \textunderscore antropóphago\textunderscore .
\section{Anthropóphago}
\begin{itemize}
\item {Grp. gram.:m.  e  adj.}
\end{itemize}
\begin{itemize}
\item {Proveniência:(Do gr. \textunderscore anthropos\textunderscore  + \textunderscore phagein\textunderscore )}
\end{itemize}
O que come carne humana.
\section{Anthropophobia}
\begin{itemize}
\item {Grp. gram.:f.}
\end{itemize}
\begin{itemize}
\item {Proveniência:(De \textunderscore anthropophobo\textunderscore )}
\end{itemize}
Horror aos homens.
Misanthropia.
\section{Anthropóphobo}
\begin{itemize}
\item {Grp. gram.:m.  e  adj.}
\end{itemize}
\begin{itemize}
\item {Proveniência:(Do gr. \textunderscore anthropos\textunderscore  + \textunderscore phobos\textunderscore )}
\end{itemize}
O que teme ou que aborrece os homens.
Misanthropo.
\section{Anthropopitheco}
\begin{itemize}
\item {Grp. gram.:m.}
\end{itemize}
\begin{itemize}
\item {Proveniência:(Do gr. \textunderscore anthropos\textunderscore , homem, e \textunderscore pithekos\textunderscore , macaco)}
\end{itemize}
Gênero hypothético de animaes fósseis, em que se julgou vêr os precursores dos homens.
\section{Anthroposophia}
\begin{itemize}
\item {Grp. gram.:f.}
\end{itemize}
\begin{itemize}
\item {Proveniência:(Do gr. \textunderscore anthropos\textunderscore  + \textunderscore sophia\textunderscore )}
\end{itemize}
Sciência, que trata da natureza moral do homem.
\section{Anthropotechnia}
\begin{itemize}
\item {Grp. gram.:f.}
\end{itemize}
\begin{itemize}
\item {Proveniência:(Do gr. \textunderscore anthropos\textunderscore  + \textunderscore tekhne\textunderscore )}
\end{itemize}
Arte de aperfeiçoar as faculdades do homem e de as adaptar ás necessidades da vida.
\section{Anthropotheísmo}
\begin{itemize}
\item {Grp. gram.:m.}
\end{itemize}
\begin{itemize}
\item {Proveniência:(Do gr. \textunderscore anthropos\textunderscore  + \textunderscore theos\textunderscore )}
\end{itemize}
Deificação da humanidade.
\section{Anthropotheísta}
\begin{itemize}
\item {Grp. gram.:m.}
\end{itemize}
Aquelle que deifica a humanidade.
\section{Anthropotherapia}
\begin{itemize}
\item {Grp. gram.:f.}
\end{itemize}
Therapêutica das doenças humanas.
\section{Anthropotherápico}
\begin{itemize}
\item {Grp. gram.:adj.}
\end{itemize}
Relativo á anthropotherapia.
\section{Anthropotomia}
\begin{itemize}
\item {Grp. gram.:f.}
\end{itemize}
\begin{itemize}
\item {Proveniência:(Do gr. \textunderscore anthropos\textunderscore  + \textunderscore tome\textunderscore )}
\end{itemize}
Anatomia do homem; dissecação do cadáver humano.
\section{Anthropozoico}
\begin{itemize}
\item {Grp. gram.:adj.}
\end{itemize}
\begin{itemize}
\item {Utilização:Zool.}
\end{itemize}
\begin{itemize}
\item {Proveniência:(Do gr. \textunderscore anthropos\textunderscore  + \textunderscore zoon\textunderscore )}
\end{itemize}
Diz-se da quinta phase do período philogenético, na qual se suppõe que o homem appareceu na terra.
\section{Anthura}
\begin{itemize}
\item {Grp. gram.:f.}
\end{itemize}
Gênero de crustáceos.
(Cp. \textunderscore anthúrio\textunderscore )
\section{Anthúrio}
\begin{itemize}
\item {Grp. gram.:m.}
\end{itemize}
\begin{itemize}
\item {Proveniência:(Do gr. \textunderscore anthos\textunderscore  + \textunderscore oura\textunderscore )}
\end{itemize}
Gênero de plantas aroídeas.
\section{Anthusa}
\begin{itemize}
\item {Grp. gram.:f.}
\end{itemize}
\begin{itemize}
\item {Utilização:Bot.}
\end{itemize}
Gênero de leguminosas.
\section{Anthyllídeas}
\begin{itemize}
\item {Grp. gram.:f. pl.}
\end{itemize}
\begin{itemize}
\item {Proveniência:(De \textunderscore anthýllido\textunderscore )}
\end{itemize}
Gênero de plantas leguminosas.
\section{Anthýllido}
\begin{itemize}
\item {Grp. gram.:m.}
\end{itemize}
\begin{itemize}
\item {Proveniência:(Do gr. \textunderscore anthullis\textunderscore )}
\end{itemize}
Arbusto ornamental.
\section{Anti...}
\begin{itemize}
\item {Grp. gram.:pref.}
\end{itemize}
\begin{itemize}
\item {Proveniência:(Do gr. \textunderscore anti\textunderscore )}
\end{itemize}
(indic. de opposição): \textunderscore antichristão\textunderscore ; \textunderscore antiliberal\textunderscore .
\section{Antia}
\begin{itemize}
\item {Grp. gram.:m.}
\end{itemize}
\begin{itemize}
\item {Proveniência:(Do gr. \textunderscore anthas\textunderscore )}
\end{itemize}
Gênero de insectos coleópteros pentâmeros.
\section{Antiabortivo}
\begin{itemize}
\item {Grp. gram.:adj.}
\end{itemize}
\begin{itemize}
\item {Proveniência:(De \textunderscore anti...\textunderscore  + \textunderscore abortivo\textunderscore )}
\end{itemize}
Contrário ao abôrto.
\section{Antiacadêmico}
\begin{itemize}
\item {Grp. gram.:adj.}
\end{itemize}
Contrário ás práticas e doutrinas de uma academia.
\section{Antiácido}
\begin{itemize}
\item {Grp. gram.:adj.}
\end{itemize}
\begin{itemize}
\item {Proveniência:(De \textunderscore anti...\textunderscore  + \textunderscore ácido\textunderscore )}
\end{itemize}
Que impede o desenvolvimento de ácidos no estômago.
\section{Antiadite}
\begin{itemize}
\item {Grp. gram.:f.}
\end{itemize}
O mesmo que \textunderscore amygdalite\textunderscore .
\section{Antiafrodisíaco}
\begin{itemize}
\item {Grp. gram.:adj.}
\end{itemize}
Contrário á afrodisia.
\section{Antiafrodítico}
\begin{itemize}
\item {Grp. gram.:adj.}
\end{itemize}
O mesmo que \textunderscore afrodisíaco\textunderscore .
\section{Antiagrícola}
\begin{itemize}
\item {Grp. gram.:adj.}
\end{itemize}
Contrário ás regras da agricultura.
\section{Antialcalino}
\begin{itemize}
\item {Grp. gram.:adj.}
\end{itemize}
Que modifica a qualidade alcalina dos humores.
\section{Antialcoólico}
\begin{itemize}
\item {Grp. gram.:adj.}
\end{itemize}
Que modifica ou annulla a acção do álcool.
\section{Antiamarílico}
\begin{itemize}
\item {Grp. gram.:adj.}
\end{itemize}
\begin{itemize}
\item {Utilização:Bras}
\end{itemize}
\begin{itemize}
\item {Proveniência:(T. mal formado. Cp. \textunderscore amarílico\textunderscore )}
\end{itemize}
Applicável contra a febre amarela.
\section{Antiaphrodísiaco}
\begin{itemize}
\item {Grp. gram.:adj.}
\end{itemize}
Contrário á aphrodisia.
\section{Antiaphrodítico}
\begin{itemize}
\item {Grp. gram.:adj.}
\end{itemize}
O mesmo que \textunderscore aphrodisíaco\textunderscore .
\section{Antiapopléctico}
\begin{itemize}
\item {Grp. gram.:adj.}
\end{itemize}
Contrário á apoplexia.
\section{Antiar}
\begin{itemize}
\item {Grp. gram.:f.}
\end{itemize}
O mesmo que \textunderscore antiarina\textunderscore .
\section{Antiarina}
\begin{itemize}
\item {Grp. gram.:f.}
\end{itemize}
Princípio tóxico do antiáris.
\section{Antiáris}
\begin{itemize}
\item {Grp. gram.:m.}
\end{itemize}
Gênero de plantas notáveis, com cujo suco leitoso os indígenas de Java envenenam as suas frechas, (\textunderscore antiaris toxicaria\textunderscore , Rumph.).
\section{Antiaristocrata}
\begin{itemize}
\item {Grp. gram.:m.}
\end{itemize}
Aquelle que é opposto á aristocracia.
\section{Antiaristocrático}
\begin{itemize}
\item {Grp. gram.:adj.}
\end{itemize}
Contrário á aristocracia.
\section{Antiarthrítico}
\begin{itemize}
\item {Grp. gram.:adj.}
\end{itemize}
\begin{itemize}
\item {Utilização:Med.}
\end{itemize}
\begin{itemize}
\item {Proveniência:(De \textunderscore anti...\textunderscore  + \textunderscore arthrítico\textunderscore )}
\end{itemize}
Contrário á gota.
\section{Antiartístico}
\begin{itemize}
\item {Grp. gram.:adj.}
\end{itemize}
Contrário ás artes ou aos princípios da arte.
\section{Antiartrítico}
\begin{itemize}
\item {Grp. gram.:adj.}
\end{itemize}
\begin{itemize}
\item {Utilização:Med.}
\end{itemize}
\begin{itemize}
\item {Proveniência:(De \textunderscore anti...\textunderscore  + \textunderscore arthrítico\textunderscore )}
\end{itemize}
Contrário á gota.
\section{Antiasmático}
\begin{itemize}
\item {Grp. gram.:adj.}
\end{itemize}
Contrário á asma; applicável contra a asma.
\section{Antibácchio}
\begin{itemize}
\item {fónica:qui}
\end{itemize}
\begin{itemize}
\item {Grp. gram.:m.  e  adj}
\end{itemize}
\begin{itemize}
\item {Proveniência:(Lat. \textunderscore antibacchius\textunderscore )}
\end{itemize}
Diz-se de um pé de verso latino, que é o bácchio invertido.
\section{Antibáquio}
\begin{itemize}
\item {Grp. gram.:m.  e  adj}
\end{itemize}
\begin{itemize}
\item {Proveniência:(Lat. \textunderscore antibacchius\textunderscore )}
\end{itemize}
Diz-se de um pé de verso latino, que é o báquio invertido.
\section{Antibilioso}
\begin{itemize}
\item {Grp. gram.:adj.}
\end{itemize}
Diz-se do medicamento, destinado a fazer evacuar a bílis.
\section{Antiblennorrágico}
\begin{itemize}
\item {Grp. gram.:adj.}
\end{itemize}
Applicável contra a \textunderscore blennorragia\textunderscore .
\section{Antiblenorrágico}
\begin{itemize}
\item {Grp. gram.:adj.}
\end{itemize}
Applicável contra a \textunderscore blennorragia\textunderscore .
\section{Antibonapartismo}
\begin{itemize}
\item {Grp. gram.:m.}
\end{itemize}
Systema dos antibonapartistas.
\section{Antibonapartista}
\begin{itemize}
\item {Grp. gram.:m.}
\end{itemize}
Adversario do bonapartismo.
\section{Antibritânico}
\begin{itemize}
\item {Grp. gram.:adj.}
\end{itemize}
Hostil aos Ingleses ou aos interesses da Inglaterra.
\section{Antibritânnico}
\begin{itemize}
\item {Grp. gram.:adj.}
\end{itemize}
Hostil aos Ingleses ou aos interesses da Inglaterra.
\section{Antibula}
\begin{itemize}
\item {Grp. gram.:f.}
\end{itemize}
\begin{itemize}
\item {Proveniência:(De \textunderscore anti...\textunderscore  + \textunderscore bulla\textunderscore )}
\end{itemize}
Bulla de antipapa.
\section{Antibulla}
\begin{itemize}
\item {Grp. gram.:f.}
\end{itemize}
\begin{itemize}
\item {Proveniência:(De \textunderscore anti...\textunderscore  + \textunderscore bulla\textunderscore )}
\end{itemize}
Bulla de antipapa.
\section{Anticachéctico}
\begin{itemize}
\item {fónica:qué}
\end{itemize}
\begin{itemize}
\item {Grp. gram.:m.  e  adj.}
\end{itemize}
Diz-se dos medicamentos applicáveis contra a cachexia.
\section{Anticanceroso}
\begin{itemize}
\item {Grp. gram.:adj.}
\end{itemize}
Diz-se de medicamentos, geralmente arsenicaes, empregados contra o cancro.
\section{Anticaquético}
\begin{itemize}
\item {Grp. gram.:m.  e  adj.}
\end{itemize}
Diz-se dos medicamentos applicáveis contra a cachexia.
\section{Anticardeal}
\begin{itemize}
\item {Grp. gram.:m.}
\end{itemize}
Cardeal, nomeado por antipapa.
\section{Anticárdio}
\begin{itemize}
\item {Grp. gram.:m.}
\end{itemize}
\begin{itemize}
\item {Utilização:Anat.}
\end{itemize}
Parte do tronco, correspondente á abertura superior do estômago.
\section{Anticatarral}
\begin{itemize}
\item {Grp. gram.:adj.}
\end{itemize}
Diz-se do medicamento que combate o catarro.
\section{Anticatártico}
\begin{itemize}
\item {Grp. gram.:adj.}
\end{itemize}
O mesmo que \textunderscore parapurgativo\textunderscore .
\section{Anticategoria}
\begin{itemize}
\item {Grp. gram.:f.}
\end{itemize}
Accusação, opposta a outra.
\section{Anticathártico}
\begin{itemize}
\item {Grp. gram.:adj.}
\end{itemize}
O mesmo que \textunderscore parapurgativo\textunderscore .
\section{Anticathólico}
\begin{itemize}
\item {Grp. gram.:adj.}
\end{itemize}
O mesmo que \textunderscore acathólico\textunderscore .
\section{Anticatólico}
\begin{itemize}
\item {Grp. gram.:adj.}
\end{itemize}
O mesmo que \textunderscore acatólico\textunderscore .
\section{Anticefálgico}
\begin{itemize}
\item {Grp. gram.:adj.}
\end{itemize}
Applicável contra dores de cabeça.
\section{Anticephálgico}
\begin{itemize}
\item {Grp. gram.:adj.}
\end{itemize}
Applicável contra dores de cabeça.
\section{Antichloro}
\begin{itemize}
\item {Grp. gram.:m.}
\end{itemize}
Sulfito de cal, com que se destróem os maus effeitos do excesso de chloro nas fábricas de tecidos e de papel.
\section{Anticholérico}
\begin{itemize}
\item {fónica:co}
\end{itemize}
\begin{itemize}
\item {Grp. gram.:adj.}
\end{itemize}
Applicável contra a chólera.
\section{Antichrese}
\begin{itemize}
\item {Grp. gram.:f.}
\end{itemize}
\begin{itemize}
\item {Utilização:Jur.}
\end{itemize}
\begin{itemize}
\item {Proveniência:(Gr. \textunderscore antikhresis\textunderscore )}
\end{itemize}
O mesmo que contrato de consignação de rendimentos.
\section{Antichrético}
\begin{itemize}
\item {Grp. gram.:adj.}
\end{itemize}
Relativo á antichrese. Cf. Ferrer, \textunderscore Dir. Nat.\textunderscore , 153.
\section{Antichristandade}
\begin{itemize}
\item {Grp. gram.:f.}
\end{itemize}
Conjunto dos países que não professam o christianismo.
\section{Antichristão}
\begin{itemize}
\item {Grp. gram.:m.  e  adj.}
\end{itemize}
Inimigo do christianismo; contrário aos christãos.
\section{Antichristianismo}
\begin{itemize}
\item {Grp. gram.:m.}
\end{itemize}
Doutrina, opposta ao christianismo.
\section{Antichristo}
\begin{itemize}
\item {Grp. gram.:m.}
\end{itemize}
\begin{itemize}
\item {Proveniência:(De \textunderscore anti...\textunderscore  + \textunderscore Christo\textunderscore , n. p.)}
\end{itemize}
O último perseguidor da doutrina de Christo, no fim do mundo.
\section{Antichronismo}
\begin{itemize}
\item {Grp. gram.:m.}
\end{itemize}
O mesmo que \textunderscore anachronismo\textunderscore .
\section{Antichtones}
\begin{itemize}
\item {Grp. gram.:m. pl.}
\end{itemize}
\begin{itemize}
\item {Proveniência:(Do gr. \textunderscore anti\textunderscore  + \textunderscore khton\textunderscore )}
\end{itemize}
O mesmo que [[antípodas|antípoda]].
\section{Anticipar}
\textunderscore v. t.\textunderscore  (e der.)
(V. \textunderscore antecipar\textunderscore , etc.)
\section{Anticívico}
\begin{itemize}
\item {Grp. gram.:adj.}
\end{itemize}
Opposto aos deveres de cidadão.
\section{Anticivilizador}
\begin{itemize}
\item {Grp. gram.:adj.}
\end{itemize}
Contrário á civilização.
\section{Anticivismo}
\begin{itemize}
\item {Grp. gram.:m.}
\end{itemize}
Qualidade ou acto opposto a civismo.
\section{Anticlerical}
\begin{itemize}
\item {Grp. gram.:adj.}
\end{itemize}
Contrário ás ideias do clero.
\section{Anticlinal}
\begin{itemize}
\item {Grp. gram.:adj.}
\end{itemize}
O mesmo que \textunderscore anticlíneo\textunderscore .
\section{Anticlinantho}
\begin{itemize}
\item {Grp. gram.:m.}
\end{itemize}
\begin{itemize}
\item {Utilização:Bot.}
\end{itemize}
Parte inferior do receptáculo das plantas de flôres compostas.
\section{Anticlinanto}
\begin{itemize}
\item {Grp. gram.:m.}
\end{itemize}
\begin{itemize}
\item {Utilização:Bot.}
\end{itemize}
Parte inferior do receptáculo das plantas de flôres compostas.
\section{Anticlíneo}
\begin{itemize}
\item {Grp. gram.:adj.}
\end{itemize}
\begin{itemize}
\item {Utilização:Miner.}
\end{itemize}
\begin{itemize}
\item {Proveniência:(Do gr. \textunderscore anti\textunderscore  + \textunderscore klinein\textunderscore )}
\end{itemize}
Diz-se das linhas de intersecção dos planos de estratificação.
\section{Anticloro}
\begin{itemize}
\item {Grp. gram.:m.}
\end{itemize}
Sulfito de cal, com que se destróem os maus effeitos do excesso de cloro nas fábricas de tecidos e de papel.
\section{Antico}
\begin{itemize}
\item {Grp. gram.:m.}
\end{itemize}
Gênero de coleópteros.
\section{Anticolérico}
\begin{itemize}
\item {Grp. gram.:adj.}
\end{itemize}
Applicável contra a cólera.
\section{Anticólico}
\begin{itemize}
\item {Grp. gram.:adj.}
\end{itemize}
\begin{itemize}
\item {Utilização:Med.}
\end{itemize}
\begin{itemize}
\item {Utilização:ant.}
\end{itemize}
Dizia-se dos medicamentos contra a cólica.
\section{Anticolonista}
\begin{itemize}
\item {Grp. gram.:adj.}
\end{itemize}
\begin{itemize}
\item {Utilização:Neol.}
\end{itemize}
Adversário dos que advogam a vantagem das colónias.
\section{Anticomania}
\begin{itemize}
\item {Grp. gram.:f.}
\end{itemize}
\begin{itemize}
\item {Proveniência:(Do lat. \textunderscore antiquus\textunderscore  e \textunderscore mania\textunderscore )}
\end{itemize}
Gôsto excessivo das coisas antigas.
\section{Anticomercial}
\begin{itemize}
\item {Grp. gram.:adj.}
\end{itemize}
Contrário aos interesses do comércio.
\section{Anticommercial}
\begin{itemize}
\item {Grp. gram.:adj.}
\end{itemize}
Contrário aos interesses do commércio.
\section{Anticoncílio}
\begin{itemize}
\item {Grp. gram.:m.}
\end{itemize}
\begin{itemize}
\item {Proveniência:(De \textunderscore anti...\textunderscore  + \textunderscore concilio\textunderscore )}
\end{itemize}
Associação de livres pensadores, que em 1870 se organizou em Nápoles, para protestar contra as decisões do concílio do Vaticano.
\section{Anticonjugal}
\begin{itemize}
\item {Grp. gram.:adj.}
\end{itemize}
Opposto á maneira como devem viver os conjuges.
\section{Anticonstitucional}
\begin{itemize}
\item {Grp. gram.:adj.}
\end{itemize}
Contrário á constituição politica de um país.
\section{Anticonstitucionalmente}
\begin{itemize}
\item {Grp. gram.:adv.}
\end{itemize}
De modo anticonstitucional.
\section{Anticontagionista}
\begin{itemize}
\item {Grp. gram.:m.  e  adj.}
\end{itemize}
Contrário aos que sustentam que uma doença é contagiosa.
\section{Anticonvulsivo}
\begin{itemize}
\item {Grp. gram.:adj.}
\end{itemize}
Diz-se de medicamentos, que se empregam contra as convulsões.
\section{Antícope}
\begin{itemize}
\item {Grp. gram.:f.}
\end{itemize}
\begin{itemize}
\item {Utilização:Med.}
\end{itemize}
Repercussão.
\section{Anticoposcópio}
\begin{itemize}
\item {Grp. gram.:m.}
\end{itemize}
\begin{itemize}
\item {Utilização:Med.}
\end{itemize}
Instrumento, para praticar a percussão immediata em diversos pontos do thórax.
\section{Anticorrosivo}
\begin{itemize}
\item {Grp. gram.:adj.}
\end{itemize}
Opposto ás substancias corrosivas.
\section{Anticosmético}
\begin{itemize}
\item {Grp. gram.:m.  e  adj.}
\end{itemize}
O que destrói a belleza.
\section{Anticosta}
\begin{itemize}
\item {Grp. gram.:f.}
\end{itemize}
Contracosta.
\section{Anticrepúsculo}
\begin{itemize}
\item {Grp. gram.:m.}
\end{itemize}
Claridade, manifestada no ponto opposto ao do crepúsculo real.
\section{Anticrese}
\begin{itemize}
\item {Grp. gram.:f.}
\end{itemize}
\begin{itemize}
\item {Utilização:Jur.}
\end{itemize}
\begin{itemize}
\item {Proveniência:(Gr. \textunderscore antikhresis\textunderscore )}
\end{itemize}
O mesmo que contrato de consignação de rendimentos.
\section{Anticrético}
\begin{itemize}
\item {Grp. gram.:adj.}
\end{itemize}
Relativo á anticrese. Cf. Ferrer, \textunderscore Dir. Nat.\textunderscore , 153.
\section{Anticristandade}
\begin{itemize}
\item {Grp. gram.:f.}
\end{itemize}
Conjunto dos países que não professam o cristianismo.
\section{Anticristão}
\begin{itemize}
\item {Grp. gram.:m.  e  adj.}
\end{itemize}
Inimigo do cristianismo; contrário aos cristãos.
\section{Anticristianismo}
\begin{itemize}
\item {Grp. gram.:m.}
\end{itemize}
Doutrina, opposta ao cristianismo.
\section{Anticristo}
\begin{itemize}
\item {Grp. gram.:m.}
\end{itemize}
\begin{itemize}
\item {Proveniência:(De \textunderscore anti...\textunderscore  + \textunderscore Christo\textunderscore , n. p.)}
\end{itemize}
O último perseguidor da doutrina de Cristo, no fim do mundo.
\section{Anticrítica}
\begin{itemize}
\item {Grp. gram.:f.}
\end{itemize}
Crítica, com que se refuta outra.
\section{Anticrítico}
\begin{itemize}
\item {Grp. gram.:adj.}
\end{itemize}
Que se oppõe ás regras da crítica.
\section{Anticronismo}
\begin{itemize}
\item {Grp. gram.:m.}
\end{itemize}
O mesmo que \textunderscore anacronismo\textunderscore .
\section{Antictones}
\begin{itemize}
\item {Grp. gram.:m. pl.}
\end{itemize}
\begin{itemize}
\item {Proveniência:(Do gr. \textunderscore anti\textunderscore  + \textunderscore khton\textunderscore )}
\end{itemize}
O mesmo que [[antípodas|antípoda]].
\section{Antidáctilo}
\begin{itemize}
\item {Grp. gram.:adj.}
\end{itemize}
Dizia-se do pé de verso, contrário ao dáctilo pela disposição das sýllabas.
\section{Antidáctylo}
\begin{itemize}
\item {Grp. gram.:adj.}
\end{itemize}
Dizia-se do pé de verso, contrário ao dáctylo pela disposição das sýllabas.
\section{Antidafne}
\begin{itemize}
\item {Grp. gram.:f.}
\end{itemize}
\begin{itemize}
\item {Utilização:Bot.}
\end{itemize}
Gênero de lorantháceas.
\section{Antidaphne}
\begin{itemize}
\item {Grp. gram.:f.}
\end{itemize}
\begin{itemize}
\item {Utilização:Bot.}
\end{itemize}
Gênero de lorantháceas.
\section{Antídea}
\begin{itemize}
\item {Grp. gram.:f.}
\end{itemize}
\begin{itemize}
\item {Proveniência:(Do gr. \textunderscore anthedon\textunderscore )}
\end{itemize}
Insecto hymenóptero mellífero.
\section{Antidemocracia}
\begin{itemize}
\item {Grp. gram.:f.}
\end{itemize}
Systema de antidemocratas.
\section{Antidemocrata}
\begin{itemize}
\item {Grp. gram.:m.}
\end{itemize}
Indivíduo, opposto ás práticas e doutrinas dos democratas.
\section{Antidemocrático}
\begin{itemize}
\item {Grp. gram.:adj.}
\end{itemize}
Relativo á antidemocracia.
\section{Antidemoníaco}
\begin{itemize}
\item {Grp. gram.:adj.}
\end{itemize}
Que contesta a existência dos demónios.
\section{Antidesassimilador}
\begin{itemize}
\item {Grp. gram.:adj.}
\end{itemize}
\begin{itemize}
\item {Utilização:Med.}
\end{itemize}
Diz-se de um medicamento que augmenta a nutrição.
\section{Antidesma}
\begin{itemize}
\item {Grp. gram.:f.}
\end{itemize}
Planta oriental, de cuja casca se fazem cordas.
\section{Antidesnutritivo}
\begin{itemize}
\item {Grp. gram.:adj.}
\end{itemize}
O mesmo que \textunderscore antidesassimilador\textunderscore .
\section{Antidéspota}
\begin{itemize}
\item {Grp. gram.:m.}
\end{itemize}
Aquelle que é contrário ao despotismo.
\section{Antideus}
\begin{itemize}
\item {Grp. gram.:m.}
\end{itemize}
O mesmo que \textunderscore atheu\textunderscore .
\section{Antidiabético}
\begin{itemize}
\item {Grp. gram.:adj.}
\end{itemize}
Applicável contra a diabete.
\section{Antidiaforético}
\begin{itemize}
\item {Grp. gram.:adj.}
\end{itemize}
Diz-se do medicamento opposto aos suores excessivos.
\section{Antidiaphorético}
\begin{itemize}
\item {Grp. gram.:adj.}
\end{itemize}
Diz-se do medicamento opposto aos suores excessivos.
\section{Antidiarreico}
\begin{itemize}
\item {Grp. gram.:adj.}
\end{itemize}
Contrário á diarreia.
\section{Antidiatésico}
\begin{itemize}
\item {Grp. gram.:adj.}
\end{itemize}
Que se applica contra a diátese.
\section{Antidiathésico}
\begin{itemize}
\item {Grp. gram.:adj.}
\end{itemize}
Que se applica contra a diáthese.
\section{Antidiftérico}
\begin{itemize}
\item {Grp. gram.:adj.}
\end{itemize}
Applicável contra a difteria.
\section{Antidigestivo}
\begin{itemize}
\item {Grp. gram.:adj.}
\end{itemize}
Que prejudica a digestão.
\section{Antidiphtérico}
\begin{itemize}
\item {Grp. gram.:adj.}
\end{itemize}
Applicável contra a diphteria.
\section{Antidinástico}
\begin{itemize}
\item {Grp. gram.:adj.}
\end{itemize}
Contrário a uma dinastia ou ás dinastias em geral.
\section{Antidisentérico}
\begin{itemize}
\item {Grp. gram.:adj.}
\end{itemize}
Contrário á disenteria.
\section{Antidivino}
\begin{itemize}
\item {Grp. gram.:adj.}
\end{itemize}
Opposto a tudo que é divino.
\section{Antidogmático}
\begin{itemize}
\item {Grp. gram.:adj.}
\end{itemize}
Opposto aos dogmas.
\section{Antidogmatismo}
\begin{itemize}
\item {Grp. gram.:m.}
\end{itemize}
Systema opposto ao dogmatismo.
\section{Antidotal}
\begin{itemize}
\item {Grp. gram.:adj.}
\end{itemize}
\begin{itemize}
\item {Proveniência:(De \textunderscore antídoto\textunderscore )}
\end{itemize}
Que se emprega como antídoto.
\section{Antidotário}
\begin{itemize}
\item {Grp. gram.:m.}
\end{itemize}
\begin{itemize}
\item {Utilização:Ant.}
\end{itemize}
Livro, que tratava dos antídotos.
\section{Antidotismo}
\begin{itemize}
\item {Grp. gram.:m.}
\end{itemize}
Uso ou abuso de antídotos.
\section{Antídoto}
\begin{itemize}
\item {Grp. gram.:m.}
\end{itemize}
\begin{itemize}
\item {Proveniência:(Gr. \textunderscore antídotos\textunderscore )}
\end{itemize}
Substância, que impede a acção nociva de substância venenosa.
Contraveneno.
\section{Antidramático}
\begin{itemize}
\item {Grp. gram.:adj.}
\end{itemize}
Contrário ás regras da arte scênica.
\section{Antídromo}
\begin{itemize}
\item {Grp. gram.:m.}
\end{itemize}
\begin{itemize}
\item {Utilização:Bot.}
\end{itemize}
\begin{itemize}
\item {Proveniência:(Do gr. \textunderscore anti\textunderscore  + \textunderscore droma\textunderscore )}
\end{itemize}
Espiral, que se enrola em sentido contrário ao da que a precede.
\section{Antídula}
\begin{itemize}
\item {Grp. gram.:f.}
\end{itemize}
Gênero de insectos dípteros.
\section{Antidynástico}
\begin{itemize}
\item {Grp. gram.:adj.}
\end{itemize}
Contrário a uma dynastia ou ás dynastias em geral.
\section{Antidysentérico}
\begin{itemize}
\item {Grp. gram.:adj.}
\end{itemize}
Contrário á dysenteria.
\section{Antieconómico}
\begin{itemize}
\item {Grp. gram.:adj.}
\end{itemize}
Contrário á economia.
Opposto aos princípios da economia política.
\section{Antiemético}
\begin{itemize}
\item {Grp. gram.:adj.}
\end{itemize}
Que é contrário a um emético.
\section{Antiepidêmico}
\begin{itemize}
\item {Grp. gram.:adj.}
\end{itemize}
Contrário a epidemia.
\section{Antiepiléptico}
\begin{itemize}
\item {Grp. gram.:adj.}
\end{itemize}
Diz-se do medicamento que combate a epilepsia.
\section{Antiescorbútico}
\begin{itemize}
\item {Grp. gram.:adj.}
\end{itemize}
Applicável contra o escorbuto.
\section{Antiescrofuloso}
\begin{itemize}
\item {Grp. gram.:adj.}
\end{itemize}
Que se applica contra as escrófulas.
\section{Antiespanhol}
\begin{itemize}
\item {Grp. gram.:adj.}
\end{itemize}
Contrário á Espanha, aos seus interesses, ás suas glórias, etc.
\section{Antiespiritual}
\begin{itemize}
\item {Grp. gram.:adj.}
\end{itemize}
O mesmo que \textunderscore materialista\textunderscore ^1.
\section{Antiespiritualismo}
\begin{itemize}
\item {Grp. gram.:m.}
\end{itemize}
Doutrina opposta ao espiritualismo.
\section{Antiestético}
\begin{itemize}
\item {Grp. gram.:adj.}
\end{itemize}
Opposto á estética; em que não há bom gôsto nem amôr ao que é bello.
\section{Antiesthético}
\begin{itemize}
\item {Grp. gram.:adj.}
\end{itemize}
Opposto á esthética; em que não há bom gôsto nem amôr ao que é bello.
\section{Antiethylina}
\begin{itemize}
\item {Grp. gram.:f.}
\end{itemize}
\begin{itemize}
\item {Utilização:Chím.}
\end{itemize}
Substância, ainda mal conhecida, do elemento anti-alcoólico do sangue de cavallo.
\section{Antietilina}
\begin{itemize}
\item {Grp. gram.:f.}
\end{itemize}
\begin{itemize}
\item {Utilização:Chím.}
\end{itemize}
Substância, ainda mal conhecida, do elemento anti-alcoólico do sangue de cavallo.
\section{Antietimológico}
\begin{itemize}
\item {Grp. gram.:adj.}
\end{itemize}
Contrário á etimologia.
\section{Antietymológico}
\begin{itemize}
\item {Grp. gram.:adj.}
\end{itemize}
Contrário á etymologia.
\section{Antievangélico}
\begin{itemize}
\item {Grp. gram.:adj.}
\end{itemize}
Contrário ao Evangelho.
\section{Antiface}
\begin{itemize}
\item {Grp. gram.:m.}
\end{itemize}
\begin{itemize}
\item {Proveniência:(Do lat. \textunderscore ante\textunderscore  + \textunderscore faciem\textunderscore )}
\end{itemize}
Véu, com que se cobre o rosto.
Máscara.
\section{Antifal}
\begin{itemize}
\item {Grp. gram.:m.}
\end{itemize}
\begin{itemize}
\item {Utilização:Ant.}
\end{itemize}
O mesmo que \textunderscore antiphonário\textunderscore .
\section{Antifaz}
\begin{itemize}
\item {Grp. gram.:m.}
\end{itemize}
(V.antiface)
\section{Antifebril}
\begin{itemize}
\item {Grp. gram.:adj.}
\end{itemize}
Que acalma a febre.
\section{Antifebrina}
\begin{itemize}
\item {Grp. gram.:f.}
\end{itemize}
Preparação pharmacêutica, o mesmo que \textunderscore acetanilido\textunderscore .
\section{Antiflatulento}
\begin{itemize}
\item {Grp. gram.:adj.}
\end{itemize}
Applicável contra flatulencias.
\section{Antifrancês}
\begin{itemize}
\item {Grp. gram.:adj.}
\end{itemize}
Contrário á França, aos interesses ou glórias da França.
\section{Antigaláctico}
\begin{itemize}
\item {Grp. gram.:adj.}
\end{itemize}
Que faz deminuir a secreção do leite.
\section{Antigalha}
\begin{itemize}
\item {Grp. gram.:f.}
\end{itemize}
(V.antiqualha)
\section{Antigalho}
\begin{itemize}
\item {Grp. gram.:m.}
\end{itemize}
Peça para segurar as vêrgas do navio, estando rôta a enxárcia.
\section{Antigalicanismo}
\begin{itemize}
\item {Grp. gram.:m.}
\end{itemize}
Opposição ao galicanismo.
\section{Antigalicano}
\begin{itemize}
\item {Grp. gram.:adj.}
\end{itemize}
Adverso á igreja galicana.
\section{Antigallicanismo}
\begin{itemize}
\item {Grp. gram.:m.}
\end{itemize}
Opposição ao gallicanismo.
\section{Antigallicano}
\begin{itemize}
\item {Grp. gram.:adj.}
\end{itemize}
Adverso á igreja gallicana.
\section{Antigamente}
\begin{itemize}
\item {Grp. gram.:adv.}
\end{itemize}
\begin{itemize}
\item {Proveniência:(De \textunderscore antigo\textunderscore )}
\end{itemize}
No tempo passado; outrora; dantes.
\section{Antigangrenoso}
\begin{itemize}
\item {Grp. gram.:adj.}
\end{itemize}
Que se oppõe ao desenvolvimento da gangrena ou que a cura.
\section{Antigeométrico}
\begin{itemize}
\item {Grp. gram.:adj.}
\end{itemize}
Opposto á geometria.
\section{Antigermânico}
\begin{itemize}
\item {Grp. gram.:adj.}
\end{itemize}
Contrário á Alemanha, aos seus interesses, ás suas glórias, etc.
\section{Antigo}
\begin{itemize}
\item {Grp. gram.:adj.}
\end{itemize}
\begin{itemize}
\item {Grp. gram.:M. pl.}
\end{itemize}
\begin{itemize}
\item {Proveniência:(Lat. \textunderscore antiquus\textunderscore )}
\end{itemize}
Anterior ao tempo de agora.
Que existiu noutro tempo: \textunderscore os antigos Romanos\textunderscore .
Que succedeu outrora: \textunderscore as guerras antigas\textunderscore .
Que já passou.
Opposto a actual: \textunderscore modas antigas\textunderscore .
Os homens de outro tempo: \textunderscore os antigos tinham mais patriotismo que nós\textunderscore .
\section{Antigónia}
\begin{itemize}
\item {Grp. gram.:f.}
\end{itemize}
Gênero de peixes acanthopterýgios.
\section{Antigotoso}
\begin{itemize}
\item {Grp. gram.:adj.}
\end{itemize}
\begin{itemize}
\item {Proveniência:(De \textunderscore anti...\textunderscore  + \textunderscore gotoso\textunderscore )}
\end{itemize}
O mesmo que \textunderscore antiarthrítico\textunderscore .
\section{Antigovernamental}
\begin{itemize}
\item {Grp. gram.:adj.}
\end{itemize}
O mesmo que \textunderscore antiministerial\textunderscore .
\section{Antígrafo}
\begin{itemize}
\item {Grp. gram.:m.}
\end{itemize}
\begin{itemize}
\item {Proveniência:(Do gr. \textunderscore anti\textunderscore  + \textunderscore graphein\textunderscore )}
\end{itemize}
Cópia manuscrita.
Sinal, para separar do texto os commentários ou notas.
\section{Antigramatical}
\begin{itemize}
\item {Grp. gram.:adj.}
\end{itemize}
Opposto á gramática.
\section{Antigrammatical}
\begin{itemize}
\item {Grp. gram.:adj.}
\end{itemize}
Opposto á grammática.
\section{Antígrapho}
\begin{itemize}
\item {Grp. gram.:m.}
\end{itemize}
\begin{itemize}
\item {Proveniência:(Do gr. \textunderscore anti\textunderscore  + \textunderscore graphein\textunderscore )}
\end{itemize}
Cópia manuscrita.
Sinal, para separar do texto os commentários ou notas.
\section{Antigualha}
\begin{itemize}
\item {Grp. gram.:f.}
\end{itemize}
(V.antiqualha)
\section{Antiguidade}
\begin{itemize}
\item {Grp. gram.:f.}
\end{itemize}
\begin{itemize}
\item {Grp. gram.:Pl.}
\end{itemize}
\begin{itemize}
\item {Proveniência:(Lat. \textunderscore antiquitas\textunderscore )}
\end{itemize}
Qualidade de antigo: \textunderscore a antiguidade do mundo\textunderscore .
O tempo antigo: \textunderscore historia da antiguidade\textunderscore .
Os homens de eras remotas: \textunderscore a antiguidade tinha mais fé\textunderscore .
Tempo de serviço num cargo: \textunderscore lista da antiguidade dos magistrados\textunderscore .
Aquillo que diz respeito a povos antigos: \textunderscore a antiguidade da música\textunderscore .
Coisas antigas; instituições antigas.
\section{Antihéctico}
\begin{itemize}
\item {Grp. gram.:adj.}
\end{itemize}
Próprio para combater a febre héctica.
\section{Antihelmínthico}
\begin{itemize}
\item {Grp. gram.:adj.}
\end{itemize}
Contrário aos helmintos; vermífugo.
\section{Antihelmíntico}
\begin{itemize}
\item {Grp. gram.:adj.}
\end{itemize}
Contrário aos helmintos; vermífugo.
\section{Antihemorrágico}
\begin{itemize}
\item {Grp. gram.:adj.}
\end{itemize}
Próprio para combater a hemorragia.
\section{Antihemorroidal}
\begin{itemize}
\item {Grp. gram.:adj.}
\end{itemize}
Applicável contra as hemorroidas.
\section{Antiherpético}
\begin{itemize}
\item {Grp. gram.:adj.}
\end{itemize}
Diz-se dos medicamentos que se applicam contra o herpes.
\section{Antihidrofóbico}
\begin{itemize}
\item {Grp. gram.:adj.}
\end{itemize}
Applicável contra a hidrofobia.
\section{Antihidrópico}
\begin{itemize}
\item {Grp. gram.:adj.}
\end{itemize}
Applicável contra a hidropisia.
\section{Antihigiênico}
\begin{itemize}
\item {Grp. gram.:adj.}
\end{itemize}
Opposto ás leis da higiene.
\section{Antihipnótico}
\begin{itemize}
\item {Grp. gram.:adj.}
\end{itemize}
Que tira o somno.
\section{Antihistérico}
\begin{itemize}
\item {Grp. gram.:adj.}
\end{itemize}
Applicável contra o histerismo.
\section{Antihydrophóbico}
\begin{itemize}
\item {Grp. gram.:adj.}
\end{itemize}
Applicável contra a hydrophobia.
\section{Antihydrópico}
\begin{itemize}
\item {Grp. gram.:adj.}
\end{itemize}
Applicável contra a hydropisia.
\section{Antihygiênico}
\begin{itemize}
\item {Grp. gram.:adj.}
\end{itemize}
Opposto ás leis da hygiene.
\section{Antihypnótico}
\begin{itemize}
\item {Grp. gram.:adj.}
\end{itemize}
Que tira o somno.
\section{Antihystérico}
\begin{itemize}
\item {Grp. gram.:adj.}
\end{itemize}
Applicável contra o hysterismo.
\section{Antiictérico}
\begin{itemize}
\item {Grp. gram.:adj.}
\end{itemize}
Applicável contra a icterícia.
\section{Antiindustrial}
\begin{itemize}
\item {Grp. gram.:adj.}
\end{itemize}
Contrário á prosperidade e melhoramentos da indústria.
\section{Antilactagogo}
\begin{itemize}
\item {Grp. gram.:adj.}
\end{itemize}
Que deminue ou suspende a secreção do leite.
\section{Antilegal}
\begin{itemize}
\item {Grp. gram.:adj.}
\end{itemize}
O mesmo que \textunderscore illegal\textunderscore . Cf. Herculano, \textunderscore Quest. Pub.\textunderscore , I, 241.
\section{Antileitoso}
\begin{itemize}
\item {Grp. gram.:adj.}
\end{itemize}
O mesmo que \textunderscore antigaláctico\textunderscore .
\section{Antiletárgico}
\begin{itemize}
\item {Grp. gram.:adj.}
\end{itemize}
Próprio para combater a letargia.
\section{Antilethárgico}
\begin{itemize}
\item {Grp. gram.:adj.}
\end{itemize}
Próprio para combater a lethargia.
\section{Antilhano}
\begin{itemize}
\item {Grp. gram.:adj.}
\end{itemize}
\begin{itemize}
\item {Grp. gram.:M.}
\end{itemize}
Relativo ás Antilhas.
Habitante das Antilhas.
\section{Antiliberal}
\begin{itemize}
\item {Grp. gram.:adj.}
\end{itemize}
Opposto ás ideias liberaes, á liberdade civil e política.
\section{Antiliberalismo}
\begin{itemize}
\item {Grp. gram.:m.}
\end{itemize}
Systema antiliberal.
\section{Antilídeas}
\begin{itemize}
\item {Grp. gram.:f. pl.}
\end{itemize}
\begin{itemize}
\item {Proveniência:(De \textunderscore anthýllido\textunderscore )}
\end{itemize}
Gênero de plantas leguminosas.
\section{Antílido}
\begin{itemize}
\item {Grp. gram.:m.}
\end{itemize}
\begin{itemize}
\item {Proveniência:(Do gr. \textunderscore anthullis\textunderscore )}
\end{itemize}
Arbusto ornamental.
\section{Antilíthico}
\begin{itemize}
\item {Grp. gram.:adj.}
\end{itemize}
Applicável contra a lithíase.
\section{Antilítico}
\begin{itemize}
\item {Grp. gram.:adj.}
\end{itemize}
Applicável contra a litíase.
\section{Antilocabra}
\begin{itemize}
\item {Grp. gram.:f.}
\end{itemize}
Gênero de antílopes.
\section{Antilogarithmo}
\begin{itemize}
\item {Grp. gram.:m.}
\end{itemize}
\begin{itemize}
\item {Utilização:Mathem.}
\end{itemize}
Complemento de um logarithmo, de um seno, de uma secante, de uma tangente.
Número, de que é logarithmo outro número dado.
\section{Antilogaritmo}
\begin{itemize}
\item {Grp. gram.:m.}
\end{itemize}
\begin{itemize}
\item {Utilização:Mathem.}
\end{itemize}
Complemento de um logaritmo, de um seno, de uma secante, de uma tangente.
Número, de que é logaritmo outro número dado.
\section{Antilogia}
\begin{itemize}
\item {Grp. gram.:f.}
\end{itemize}
\begin{itemize}
\item {Proveniência:(Do gr. \textunderscore anti\textunderscore  + \textunderscore logos\textunderscore )}
\end{itemize}
Opposição, que se dá entre as palavras ou entre as ideias de um livro ou discurso.
\section{Antilógico}
\begin{itemize}
\item {Grp. gram.:adj.}
\end{itemize}
Em que há antilogia.
Opposto á lógica.
\section{Antílopa}
\begin{itemize}
\item {Grp. gram.:f.}
\end{itemize}
O mesmo que \textunderscore antílope\textunderscore . Cf. Camillo, \textunderscore Quéda de um Anjo\textunderscore , 172.
\section{Antílope}
\begin{itemize}
\item {Grp. gram.:m.  e  f.}
\end{itemize}
\begin{itemize}
\item {Proveniência:(Do gr. \textunderscore antholopse\textunderscore ?)}
\end{itemize}
Animal ruminante, de galhos ocos, fórma elegante e carreira veloz.
\section{Antilopianos}
\begin{itemize}
\item {Grp. gram.:m. pl.}
\end{itemize}
Sub-família de mammíferos placentários, a que pertence o antílope.
\section{Antímaco}
\begin{itemize}
\item {Grp. gram.:m.}
\end{itemize}
Gênero de coleópteros.
\section{Antimedical}
\begin{itemize}
\item {Grp. gram.:adj.}
\end{itemize}
Contrário á Medicina. Cf. Ortigão, \textunderscore Praias\textunderscore , 120.
\section{Antimefítico}
\begin{itemize}
\item {Grp. gram.:adj.}
\end{itemize}
Próprio para absorver os miasmas ou neutralizar os maus cheiros.
\section{Antimelancólico}
\begin{itemize}
\item {Grp. gram.:adj.}
\end{itemize}
Efficaz contra a melancolia.
\section{Antimelódico}
\begin{itemize}
\item {Grp. gram.:adj.}
\end{itemize}
Contrário á melodia.
\section{Antimensa}
\begin{itemize}
\item {Grp. gram.:f.}
\end{itemize}
Toalha, que serve de altar, entre os Christãos do Oriente.
\section{Antimephítico}
\begin{itemize}
\item {Grp. gram.:adj.}
\end{itemize}
Próprio para absorver os miasmas ou neutralizar os maus cheiros.
\section{Antimérico}
\begin{itemize}
\item {Grp. gram.:adj.}
\end{itemize}
Relativo ao \textunderscore antímero\textunderscore .
\section{Antímero}
\begin{itemize}
\item {Grp. gram.:m.}
\end{itemize}
\begin{itemize}
\item {Utilização:Anat.}
\end{itemize}
\begin{itemize}
\item {Proveniência:(Do gr. \textunderscore anti\textunderscore  + \textunderscore meros\textunderscore , parte)}
\end{itemize}
Cada uma das partes do corpo humano, considerando-se êste dividido por um plano vertical antero-posterior.
\section{Antimesmerista}
\begin{itemize}
\item {Grp. gram.:m.}
\end{itemize}
Adversário do mesmerismo ou da doutrina do magnetismo animal.
\section{Antimetábole}
\begin{itemize}
\item {Grp. gram.:f.}
\end{itemize}
Formação de uma phrase com palavras de outra.
\section{Antimetalepse}
\begin{itemize}
\item {Grp. gram.:f.}
\end{itemize}
\begin{itemize}
\item {Proveniência:(De \textunderscore anti...\textunderscore  + \textunderscore metalepse\textunderscore )}
\end{itemize}
O mesmo que \textunderscore antimetábole\textunderscore .
\section{Antimetátese}
\begin{itemize}
\item {Grp. gram.:f.}
\end{itemize}
\begin{itemize}
\item {Proveniência:(De \textunderscore anti...\textunderscore  + \textunderscore metáthese\textunderscore )}
\end{itemize}
O mesmo que \textunderscore antimetábole\textunderscore .
\section{Antimetáthese}
\begin{itemize}
\item {Grp. gram.:f.}
\end{itemize}
\begin{itemize}
\item {Proveniência:(De \textunderscore anti...\textunderscore  + \textunderscore metáthese\textunderscore )}
\end{itemize}
O mesmo que \textunderscore antimetábole\textunderscore .
\section{Antimiasmático}
\begin{itemize}
\item {Grp. gram.:adj.}
\end{itemize}
Opposto á influência dos miasmas.
\section{Antimilitar}
\begin{itemize}
\item {Grp. gram.:adj.}
\end{itemize}
Contrário ao espirito militar.
\section{Antiministerial}
\begin{itemize}
\item {Grp. gram.:adj.}
\end{itemize}
Contrário ao Ministério ou ao Govêrno.
\section{Antimonacal}
\begin{itemize}
\item {Grp. gram.:adj.}
\end{itemize}
\begin{itemize}
\item {Proveniência:(De \textunderscore anti...\textunderscore  + \textunderscore monachal\textunderscore )}
\end{itemize}
Contrário ás instituições conventuaes.
\section{Antimonachal}
\begin{itemize}
\item {fónica:cal}
\end{itemize}
\begin{itemize}
\item {Grp. gram.:adj.}
\end{itemize}
\begin{itemize}
\item {Proveniência:(De \textunderscore anti...\textunderscore  + \textunderscore monachal\textunderscore )}
\end{itemize}
Contrário ás instituições conventuaes.
\section{Antimonárchico}
\begin{itemize}
\item {fónica:qui}
\end{itemize}
\begin{itemize}
\item {Grp. gram.:adj.}
\end{itemize}
Contrário ao governo monárchico.
\section{Antimonarchista}
\begin{itemize}
\item {fónica:quís}
\end{itemize}
\begin{itemize}
\item {Grp. gram.:adj.}
\end{itemize}
Inimigo da monarchia.
\section{Antimonárquico}
\begin{itemize}
\item {Grp. gram.:adj.}
\end{itemize}
Contrário ao governo monárquico.
\section{Antimonarquista}
\begin{itemize}
\item {Grp. gram.:adj.}
\end{itemize}
Inimigo da monarquia.
\section{Antimoniado}
\begin{itemize}
\item {Grp. gram.:adj.}
\end{itemize}
Que contém antimónio.
\section{Antimonial}
\begin{itemize}
\item {Grp. gram.:adj.}
\end{itemize}
Relativo ao antimónio; que tem antimónio.
\section{Antimónico}
\begin{itemize}
\item {Grp. gram.:adj.}
\end{itemize}
Diz-se de um ácido, composto de dois equivalentes de antimónio e cinco de oxygênio.
\section{Antimonieto}
\begin{itemize}
\item {fónica:ê}
\end{itemize}
\begin{itemize}
\item {Grp. gram.:m.}
\end{itemize}
Liga de antimónio com outro metal.
\section{Antimonífero}
\begin{itemize}
\item {Grp. gram.:adj.}
\end{itemize}
O mesmo que \textunderscore antimoniado\textunderscore .
\section{Antimónio}
\begin{itemize}
\item {Grp. gram.:m.}
\end{itemize}
Sal, formado pelo ácido antimónico, combinado com uma base.
\section{Antimónio}
\begin{itemize}
\item {Grp. gram.:m.}
\end{itemize}
Metal branco, azulado e quebradiço.
(B. lat. \textunderscore antimonium\textunderscore , talvez do ár. \textunderscore athmoud\textunderscore )
\section{Antimonioso}
\begin{itemize}
\item {Grp. gram.:adj.}
\end{itemize}
Diz-se de um ácido, composto de um equivalente de antimónio e dois de oxygênio.
\section{Antimonite}
\begin{itemize}
\item {Grp. gram.:f.}
\end{itemize}
O mesmo que \textunderscore antimonito\textunderscore .
\section{Antimonito}
\begin{itemize}
\item {Grp. gram.:m.}
\end{itemize}
Sal, formado pela combinação do ácido antimonioso com uma base.
\section{Antimoniureto}
\begin{itemize}
\item {fónica:urê}
\end{itemize}
\begin{itemize}
\item {Grp. gram.:m.}
\end{itemize}
(V.antimonieto)
\section{Antimonóxido}
\begin{itemize}
\item {Grp. gram.:m.}
\end{itemize}
\begin{itemize}
\item {Proveniência:(De \textunderscore antimónio\textunderscore  + \textunderscore óxydo\textunderscore )}
\end{itemize}
Designação genérica dos minérios de antimónio em fórma de óxidos.
\section{Antimonóxydo}
\begin{itemize}
\item {Grp. gram.:m.}
\end{itemize}
\begin{itemize}
\item {Proveniência:(De \textunderscore antimónio\textunderscore  + \textunderscore óxydo\textunderscore )}
\end{itemize}
Designação genérica dos minérios de antimónio em fórma de óxydos.
\section{Antimoral}
\begin{itemize}
\item {Grp. gram.:adj.}
\end{itemize}
O mesmo que \textunderscore immoral\textunderscore .
\section{Antimormoso}
\begin{itemize}
\item {Grp. gram.:adj.}
\end{itemize}
Applicável contra o mormo.
\section{Antina}
\begin{itemize}
\item {Grp. gram.:f.}
\end{itemize}
\begin{itemize}
\item {Proveniência:(De \textunderscore anihino\textunderscore )}
\end{itemize}
Gênero de cogumelos.
\section{Antinacional}
\begin{itemize}
\item {Grp. gram.:adj.}
\end{itemize}
Contrário aos interesses ou ao carácter de uma nação.
\section{Antinarcótico}
\begin{itemize}
\item {Grp. gram.:adj.}
\end{itemize}
Applicável contra o estado narcótico.
\section{Antinatural}
\begin{itemize}
\item {Grp. gram.:adj.}
\end{itemize}
Contrário ás leis da natureza.
\section{Antinefrítico}
\begin{itemize}
\item {Grp. gram.:adj.}
\end{itemize}
Applicável contra a nefrite.
\section{Antinephrítico}
\begin{itemize}
\item {Grp. gram.:adj.}
\end{itemize}
Applicável contra a nephrite.
\section{Antinervoso}
\begin{itemize}
\item {Grp. gram.:adj.}
\end{itemize}
Efficaz contra as neuralgias ou perturbações nervosas.
\section{Antineurálgico}
\begin{itemize}
\item {Grp. gram.:adj.}
\end{itemize}
Próprio para combater neuralgias.
\section{Antino}
\begin{itemize}
\item {Grp. gram.:adj.}
\end{itemize}
\begin{itemize}
\item {Proveniência:(Do gr. \textunderscore anthos\textunderscore )}
\end{itemize}
Que contém flôres.
\section{Antinobiliário}
\begin{itemize}
\item {Grp. gram.:adj.}
\end{itemize}
Opposto á nobreza.
\section{Antinódoa}
\begin{itemize}
\item {Grp. gram.:f.}
\end{itemize}
\begin{itemize}
\item {Utilização:Neol.}
\end{itemize}
\begin{itemize}
\item {Proveniência:(De \textunderscore anti...\textunderscore  + \textunderscore nódoa\textunderscore )}
\end{itemize}
Nome genérico das substâncias que servem para tirar nódoas, especialmente da roupa.
\section{Antinomia}
\begin{itemize}
\item {Grp. gram.:f.}
\end{itemize}
\begin{itemize}
\item {Proveniência:(Do gr. \textunderscore anti\textunderscore  + \textunderscore nomos\textunderscore , lei)}
\end{itemize}
Contradicção entre duas leis.
Opposição recíproca de duas coisas ou pessôas.
\section{Antinómico}
\begin{itemize}
\item {Grp. gram.:adj.}
\end{itemize}
Em que há antinomia.
Opposto.
Contradictório.
\section{Antínoo}
\begin{itemize}
\item {Grp. gram.:m.}
\end{itemize}
\begin{itemize}
\item {Proveniência:(Do gr. \textunderscore Antinous\textunderscore , n. p.)}
\end{itemize}
Uma das constellações boreaes.
\section{Antinupcial}
\begin{itemize}
\item {Grp. gram.:adj.}
\end{itemize}
Contrário ao casamento.
\section{Antiobésico}
\begin{itemize}
\item {Grp. gram.:adj.}
\end{itemize}
Que serve para combater a obesidade.
\section{Antiocheno}
\begin{itemize}
\item {fónica:quê}
\end{itemize}
\begin{itemize}
\item {Grp. gram.:adj.}
\end{itemize}
\begin{itemize}
\item {Grp. gram.:M.}
\end{itemize}
\begin{itemize}
\item {Proveniência:(Lat. \textunderscore anthiochenus\textunderscore )}
\end{itemize}
Relativo á cidade de Antiochia.
Habitante de Antiochia.
\section{Antiodontálgico}
\begin{itemize}
\item {Grp. gram.:adj.}
\end{itemize}
\begin{itemize}
\item {Proveniência:(De \textunderscore anti...\textunderscore  + \textunderscore adentálgico\textunderscore )}
\end{itemize}
Próprio para combater doenças de dentes.
\section{Antiodôntico}
\begin{itemize}
\item {Grp. gram.:m.}
\end{itemize}
O mesmo que \textunderscore antiodontálgico\textunderscore .
\section{Antiopa}
\begin{itemize}
\item {Grp. gram.:f.}
\end{itemize}
Lepidóptero diurno.
\section{Antioqueno}
\begin{itemize}
\item {Grp. gram.:adj.}
\end{itemize}
\begin{itemize}
\item {Grp. gram.:M.}
\end{itemize}
\begin{itemize}
\item {Proveniência:(Lat. \textunderscore anthiochenus\textunderscore )}
\end{itemize}
Relativo á cidade de Antioquia.
Habitante de Antioquia.
\section{Antiorgástico}
\begin{itemize}
\item {Grp. gram.:adj.}
\end{itemize}
Opposto ao orgasmo.
\section{Antipalustre}
\begin{itemize}
\item {Grp. gram.:adj.}
\end{itemize}
\begin{itemize}
\item {Proveniência:(De \textunderscore anti...\textunderscore  + \textunderscore palustre\textunderscore )}
\end{itemize}
Opposto á influência dos miasmas das lagôas.
\section{Antipapa}
\begin{itemize}
\item {Grp. gram.:m.}
\end{itemize}
\begin{itemize}
\item {Proveniência:(De \textunderscore anti...\textunderscore  + \textunderscore papa\textunderscore )}
\end{itemize}
Falso papa, que usurpa a jurisdicção do legítimo.
\section{Antipapado}
\begin{itemize}
\item {Grp. gram.:m.}
\end{itemize}
Dignidade do antipapa.
Tempo do seu govêrno.
\section{Antipapismo}
\begin{itemize}
\item {Grp. gram.:m.}
\end{itemize}
Systema dos que não reconhecem o verdadeiro papa.
\section{Antipapista}
\begin{itemize}
\item {Grp. gram.:m.}
\end{itemize}
Sectário do antipapismo.
\section{Antiparalelismo}
\begin{itemize}
\item {Grp. gram.:m.}
\end{itemize}
Qualidade das linhas antiparalelas.
\section{Antiparalelo}
\begin{itemize}
\item {Grp. gram.:adj.}
\end{itemize}
Diz-se das linhas que, com uma terceira, formam ângulos iguaes, mas em sentido contrário.
\section{Antiparalítico}
\begin{itemize}
\item {Grp. gram.:adj.}
\end{itemize}
Applicável contra a paralisia.
\section{Antiparallelismo}
\begin{itemize}
\item {Grp. gram.:m.}
\end{itemize}
Qualidade das linhas antiparallelas.
\section{Antiparallelo}
\begin{itemize}
\item {Grp. gram.:adj.}
\end{itemize}
Diz-se das linhas que, com uma terceira, formam ângulos iguaes, mas em sentido contrário.
\section{Antiparalýthico}
\begin{itemize}
\item {Grp. gram.:adj.}
\end{itemize}
Applicável contra a paralysia.
\section{Antiparástase}
\begin{itemize}
\item {Grp. gram.:f.}
\end{itemize}
\begin{itemize}
\item {Proveniência:(Do gr. \textunderscore anti\textunderscore  + \textunderscore parastasis\textunderscore )}
\end{itemize}
Allegação de que um accusado seria digno de louvor, se praticasse o acto, de que o accusam.
\section{Antiparlamentar}
\begin{itemize}
\item {Grp. gram.:adj.}
\end{itemize}
Contrário aos usos do parlamento.
\section{Antiparras}
\begin{itemize}
\item {Grp. gram.:f. pl.}
\end{itemize}
\begin{itemize}
\item {Utilização:Prov.}
\end{itemize}
\begin{itemize}
\item {Utilização:alg.}
\end{itemize}
Espécie de polainas.
\section{Antipathia}
\begin{itemize}
\item {Grp. gram.:f.}
\end{itemize}
\begin{itemize}
\item {Proveniência:(Gr. \textunderscore antipatheia\textunderscore )}
\end{itemize}
Aversão instintiva, espontânea.
\section{Antipáthico}
\begin{itemize}
\item {Grp. gram.:adj.}
\end{itemize}
Que inspira \textunderscore antipathia\textunderscore .
\section{Anticéptico}
\begin{itemize}
\item {Grp. gram.:adj.}
\end{itemize}
Opposto ao cepticismo.
\section{Antiespasmódico}
\begin{itemize}
\item {Grp. gram.:adj.}
\end{itemize}
\begin{itemize}
\item {Proveniência:(De \textunderscore anti...\textunderscore  + \textunderscore espasmódico\textunderscore )}
\end{itemize}
Recommendável contra os espasmos.
\section{Antiespiritismo}
\begin{itemize}
\item {Grp. gram.:m.}
\end{itemize}
Qualidade de antiespiritista.
\section{Antiespiritista}
\begin{itemize}
\item {Grp. gram.:m.  e  adj.}
\end{itemize}
Adverso ao espiritismo.
\section{Antiestreptocócico}
\begin{itemize}
\item {Grp. gram.:adj.}
\end{itemize}
Applicável ou bom contra o estreptococo.
\section{Antifernal}
\begin{itemize}
\item {Grp. gram.:adj.}
\end{itemize}
Diz-se dos bens, que o marido dá á mulher em contrato antenupcial.
\section{Antifilosofia}
\begin{itemize}
\item {Grp. gram.:f.}
\end{itemize}
Doutrina antifilosófica.
\section{Antifilosófico}
\begin{itemize}
\item {Grp. gram.:adj.}
\end{itemize}
Opposto á filosofia.
\section{Antifisético}
\begin{itemize}
\item {Grp. gram.:adj.}
\end{itemize}
Applicável contra flatulências.
\section{Antifísico}
\begin{itemize}
\item {Grp. gram.:adj.}
\end{itemize}
Contrário ás leis da natureza.
\section{Antifisiológico}
\begin{itemize}
\item {Grp. gram.:adj.}
\end{itemize}
Contrário ás leis da Fisiologia.
\section{Antiflogístico}
\begin{itemize}
\item {Grp. gram.:adj.}
\end{itemize}
Contrário ás inflammações.
\section{Antífona}
\begin{itemize}
\item {Grp. gram.:f.}
\end{itemize}
\begin{itemize}
\item {Proveniência:(Lat. \textunderscore antiphona\textunderscore )}
\end{itemize}
Versículo, que se entôa antes de um psalmo.
Primeiras palavras de um versículo, que, entoadas, dão o tom ao côro.
\section{Antifonário}
\begin{itemize}
\item {Grp. gram.:m.}
\end{itemize}
Livro ecclesiastico, que contém antífonas, com as notas do respectivo cantochão e outros cantos religiosos.
(B. lat. \textunderscore antiphonarium\textunderscore )
\section{Antifoneiro}
\begin{itemize}
\item {Grp. gram.:m.}
\end{itemize}
\begin{itemize}
\item {Proveniência:(De \textunderscore antifphona\textunderscore )}
\end{itemize}
O chantre, que levanta a antífona.
\section{Antifonia}
\begin{itemize}
\item {Grp. gram.:f.}
\end{itemize}
Designação do canto em oitavas, entre os Gregos.
\section{Antifónico}
\begin{itemize}
\item {Grp. gram.:adj.}
\end{itemize}
Relativo á antífona.
\section{Antífrase}
\begin{itemize}
\item {Grp. gram.:f.}
\end{itemize}
\begin{itemize}
\item {Proveniência:(Gr. \textunderscore antiphrasis\textunderscore )}
\end{itemize}
Emprêgo de uma palavra em sentido opposto ao verdadeiro.
\section{Antiftiríaco}
\begin{itemize}
\item {Grp. gram.:adj.}
\end{itemize}
O mesmo que \textunderscore antiftírico\textunderscore .
\section{Antiftírico}
\begin{itemize}
\item {Grp. gram.:adj.}
\end{itemize}
Diz-se do medicamento que destrói os piolhos.
\section{Antipathizar}
\begin{itemize}
\item {Grp. gram.:v. i.}
\end{itemize}
Sentir antipathia.
\section{Antipatia}
\begin{itemize}
\item {Grp. gram.:f.}
\end{itemize}
\begin{itemize}
\item {Proveniência:(Gr. \textunderscore antipatheia\textunderscore )}
\end{itemize}
Aversão instintiva, espontânea.
\section{Antipático}
\begin{itemize}
\item {Grp. gram.:adj.}
\end{itemize}
Que inspira \textunderscore antipatia\textunderscore .
\section{Antipatizar}
\begin{itemize}
\item {Grp. gram.:v. i.}
\end{itemize}
Sentir antipatia.
\section{Antípato}
\begin{itemize}
\item {Grp. gram.:m.}
\end{itemize}
Gênero de pólypos.
\section{Antipatriota}
\begin{itemize}
\item {Grp. gram.:m.}
\end{itemize}
Aquelle que não tem patriotismo.
\section{Antipatriótico}
\begin{itemize}
\item {Grp. gram.:adj.}
\end{itemize}
Contrário ao patriotismo.
\section{Antipatriotismo}
\begin{itemize}
\item {Grp. gram.:m.}
\end{itemize}
Falta de patriotismo.
\section{Antipelicular}
\begin{itemize}
\item {Grp. gram.:adj.}
\end{itemize}
Que faz cair a caspa da cabeça.
\section{Antipellicular}
\begin{itemize}
\item {Grp. gram.:adj.}
\end{itemize}
Que faz cair a caspa da cabeça.
\section{Antiperiódico}
\begin{itemize}
\item {Grp. gram.:adj.}
\end{itemize}
Recommendável contra doenças periódicas.
\section{Antiperistáltico}
\begin{itemize}
\item {Grp. gram.:adj.}
\end{itemize}
Contrário ao movimento peristáltico.
\section{Antiperístase}
\begin{itemize}
\item {Grp. gram.:f.}
\end{itemize}
\begin{itemize}
\item {Proveniência:(Do gr. \textunderscore anti\textunderscore  + \textunderscore peristasis\textunderscore )}
\end{itemize}
Circumstância, que faz sobresair uma de duas qualidades oppostas.
\section{Antipestilencial}
\begin{itemize}
\item {Grp. gram.:adj.}
\end{itemize}
Que se recommenda contra a peste.
\section{Antipestoso}
\begin{itemize}
\item {Grp. gram.:adj.}
\end{itemize}
\begin{itemize}
\item {Grp. gram.:M.}
\end{itemize}
\begin{itemize}
\item {Proveniência:(De \textunderscore anti...\textunderscore  + \textunderscore peste\textunderscore )}
\end{itemize}
O mesmo que \textunderscore antipestilencial\textunderscore .
Medicamento moderno, contra a peste bubónica.
\section{Antiphernal}
\begin{itemize}
\item {Grp. gram.:adj.}
\end{itemize}
Diz-se dos bens, que o marido dá á mulher em contrato antenupcial.
\section{Antiphilosophia}
\begin{itemize}
\item {Grp. gram.:f.}
\end{itemize}
Doutrina antiphilosóphica.
\section{Antiphilosóphico}
\begin{itemize}
\item {Grp. gram.:adj.}
\end{itemize}
Opposto á philosophia.
\section{Antiphlogístico}
\begin{itemize}
\item {Grp. gram.:adj.}
\end{itemize}
Contrário ás inflammações.
\section{Antíphona}
\begin{itemize}
\item {Grp. gram.:f.}
\end{itemize}
\begin{itemize}
\item {Proveniência:(Lat. \textunderscore antiphona\textunderscore )}
\end{itemize}
Versículo, que se entôa antes de um psalmo.
Primeiras palavras de um versículo, que, entoadas, dão o tom ao côro.
\section{Antiphonal}
\begin{itemize}
\item {Grp. gram.:m.}
\end{itemize}
\begin{itemize}
\item {Utilização:Ant.}
\end{itemize}
O mesmo que \textunderscore antiphonário\textunderscore .
\section{Antiphonário}
\begin{itemize}
\item {Grp. gram.:m.}
\end{itemize}
Livro ecclesiastico, que contém antíphonas, com as notas do respectivo cantochão e outros cantos religiosos.
(B. lat. \textunderscore antiphonarium\textunderscore )
\section{Antiphoneiro}
\begin{itemize}
\item {Grp. gram.:m.}
\end{itemize}
\begin{itemize}
\item {Proveniência:(De \textunderscore antiphona\textunderscore )}
\end{itemize}
O chantre, que levanta a antíphona.
\section{Antiphonia}
\begin{itemize}
\item {Grp. gram.:f.}
\end{itemize}
Designação do canto em oitavas, entre os Gregos.
\section{Antiphónico}
\begin{itemize}
\item {Grp. gram.:adj.}
\end{itemize}
Relativo á antíphona.
\section{Antíphrase}
\begin{itemize}
\item {Grp. gram.:f.}
\end{itemize}
\begin{itemize}
\item {Proveniência:(Gr. \textunderscore antiphrasis\textunderscore )}
\end{itemize}
Emprêgo de uma palavra em sentido opposto ao verdadeiro.
\section{Antiphtiríaco}
\begin{itemize}
\item {Grp. gram.:adj.}
\end{itemize}
O mesmo que \textunderscore antiphtírico\textunderscore .
\section{Antiphtírico}
\begin{itemize}
\item {Grp. gram.:adj.}
\end{itemize}
Diz-se do medicamento que destrói os piolhos.
\section{Antiphysético}
\begin{itemize}
\item {Grp. gram.:adj.}
\end{itemize}
Applicável contra flatulências.
\section{Antiphýsico}
\begin{itemize}
\item {Grp. gram.:adj.}
\end{itemize}
Contrário ás leis da natureza.
\section{Antiphysiológico}
\begin{itemize}
\item {Grp. gram.:adj.}
\end{itemize}
Contrário ás leis da Physiologia.
\section{Antipinturesco}
\begin{itemize}
\item {Grp. gram.:adj.}
\end{itemize}
Que não tem nada de pinturesco.
\section{Antipirético}
\begin{itemize}
\item {Grp. gram.:m.  e  adj.}
\end{itemize}
Diz-se do medicamento febrífugo.
(Cp. \textunderscore antipyrina\textunderscore )
\section{Antipirina}
\begin{itemize}
\item {Grp. gram.:f.}
\end{itemize}
\begin{itemize}
\item {Proveniência:(Do gr. \textunderscore anti\textunderscore  + \textunderscore pur\textunderscore )}
\end{itemize}
Substância medicinal, preparada com o auxílio da anilina, e applicada para deminuir as dôres e abaixar a temperatura em certas febres.
\section{Antipirinismo}
\begin{itemize}
\item {Grp. gram.:m.}
\end{itemize}
Estado mórbido, resultante do abuso da antipirina.
\section{Antipirótico}
\begin{itemize}
\item {Grp. gram.:adj.}
\end{itemize}
Diz-se dos medicamentos contra as queimaduras.
(Cp. \textunderscore antipyrina\textunderscore )
\section{Antiplástico}
\begin{itemize}
\item {Grp. gram.:adj.}
\end{itemize}
Diz-se, em cerâmica, das substâncias que fazem deminuir a qualidade plástica da massa.
\section{Antipleurítico}
\begin{itemize}
\item {Grp. gram.:adj.}
\end{itemize}
Applicável contra a pleurisia.
\section{Antipneumónico}
\begin{itemize}
\item {Grp. gram.:adj.}
\end{itemize}
Diz-se do medicamento contrário á pneumonia.
\section{Antipo}
\begin{itemize}
\item {Grp. gram.:m.}
\end{itemize}
Gênero de coleópteros.
\section{Antípoda}
\begin{itemize}
\item {Grp. gram.:m.}
\end{itemize}
\begin{itemize}
\item {Grp. gram.:Adj.}
\end{itemize}
\begin{itemize}
\item {Proveniência:(Gr. \textunderscore antipous\textunderscore , de \textunderscore anti\textunderscore  + \textunderscore pous\textunderscore , \textunderscore podos\textunderscore )}
\end{itemize}
Habitante da terra, que occupa a extremidade do diâmetro do globo, em relação ao habitante que está na outra extremidade.
Opposto.
\section{Antipodal}
\begin{itemize}
\item {Grp. gram.:adj.}
\end{itemize}
O mesmo que \textunderscore antipodiano\textunderscore .
\section{Antipodiano}
\begin{itemize}
\item {Grp. gram.:adj.}
\end{itemize}
O mesmo que \textunderscore antipódico\textunderscore .
\section{Antipódico}
\begin{itemize}
\item {Grp. gram.:adj.}
\end{itemize}
Relativo aos antípodas.
\section{Antipoéta}
\begin{itemize}
\item {Grp. gram.:adj.}
\end{itemize}
Desaffeiçoado ou adverso aos poétas. Cf. Castilho, \textunderscore Fastos\textunderscore , I, 340; \textunderscore Metam.\textunderscore , 294.
\section{Antipoético}
\begin{itemize}
\item {Grp. gram.:adj.}
\end{itemize}
Opposto á poesia.
\section{Antipolítico}
\begin{itemize}
\item {Grp. gram.:adj.}
\end{itemize}
Que é contrário á bôa politica.
\section{Antipopular}
\begin{itemize}
\item {Grp. gram.:adj.}
\end{itemize}
Contrário ao povo ou aos seus interesses.
\section{Antiprogressista}
\begin{itemize}
\item {Grp. gram.:adj.}
\end{itemize}
Contrário ás ideias de progresso.
\section{Antipróstata}
\begin{itemize}
\item {Grp. gram.:f.}
\end{itemize}
Cada uma das duas glândulazinhas, situadas anteriormente á próstata.
\section{Antiprotestante}
\begin{itemize}
\item {Grp. gram.:m.}
\end{itemize}
Inimigo dos protestantes.
\section{Antipsórico}
\begin{itemize}
\item {Grp. gram.:adj.}
\end{itemize}
Applicável contra a sarna.
\section{Antiptose}
\begin{itemize}
\item {Grp. gram.:f.}
\end{itemize}
\begin{itemize}
\item {Utilização:Gram.}
\end{itemize}
\begin{itemize}
\item {Proveniência:(Do gr. \textunderscore anti\textunderscore  + \textunderscore ptosis\textunderscore )}
\end{itemize}
Emprêgo de um caso por outro.
\section{Antipuritano}
\begin{itemize}
\item {Grp. gram.:adj.}
\end{itemize}
Contrário á seita dos puritanos, na Inglaterra.
\section{Antipútrido}
\begin{itemize}
\item {Grp. gram.:m.  e  adj.}
\end{itemize}
\begin{itemize}
\item {Proveniência:(De \textunderscore anti...\textunderscore  + \textunderscore pútrido\textunderscore )}
\end{itemize}
O que se oppõe á putrefacção.
\section{Antipyrético}
\begin{itemize}
\item {Grp. gram.:m.  e  adj.}
\end{itemize}
Diz-se do medicamento febrífugo.
(Cp. \textunderscore antipyrina\textunderscore )
\section{Antipyrina}
\begin{itemize}
\item {Grp. gram.:f.}
\end{itemize}
\begin{itemize}
\item {Proveniência:(Do gr. \textunderscore anti\textunderscore  + \textunderscore pur\textunderscore )}
\end{itemize}
Substância medicinal, preparada com o auxílio da anilina, e applicada para deminuir as dôres e abaixar a temperatura em certas febres.
\section{Antipyrinismo}
\begin{itemize}
\item {Grp. gram.:m.}
\end{itemize}
Estado mórbido, resultante do abuso da antipyrina.
\section{Antipyrótico}
\begin{itemize}
\item {Grp. gram.:adj.}
\end{itemize}
Diz-se dos medicamentos contra as queimaduras.
(Cp. \textunderscore antipyrina\textunderscore )
\section{Antiquado}
\begin{itemize}
\item {Grp. gram.:adj.}
\end{itemize}
Que se tornou antigo.
Desusado; obsoleto.
\section{Antiqualha}
\begin{itemize}
\item {Grp. gram.:f.}
\end{itemize}
Objecto antigo.
Costumes antigos.
Ferros velhos.
(B. lat. \textunderscore antiqualia\textunderscore )
\section{Antiquar}
\begin{itemize}
\item {Grp. gram.:v. t.}
\end{itemize}
\begin{itemize}
\item {Proveniência:(Lat. \textunderscore antiquare\textunderscore )}
\end{itemize}
Tornar antigo, obsoleto, desusado.
\section{Antiquariato}
\begin{itemize}
\item {Grp. gram.:m.}
\end{itemize}
\begin{itemize}
\item {Utilização:P. us.}
\end{itemize}
Conhecimentos de antiquário.
Lugar, onde se guardam objectos antigos.
\section{Antiquário}
\begin{itemize}
\item {Grp. gram.:m.}
\end{itemize}
\begin{itemize}
\item {Grp. gram.:Adj.}
\end{itemize}
Aquelle que estuda antiguidades.
Aquelle que collecciona objectos antigos.
Antiquado ou desusado:«\textunderscore fato antiquário e rafado\textunderscore ».
Castilho, \textunderscore Avarento\textunderscore .
\section{Antiquíssimo}
\begin{itemize}
\item {fónica:cu-í}
\end{itemize}
\begin{itemize}
\item {Grp. gram.:adj.}
\end{itemize}
\begin{itemize}
\item {Proveniência:(Do lat. \textunderscore antiquus\textunderscore )}
\end{itemize}
Muito antigo.
\section{Antirábico}
\begin{itemize}
\item {fónica:rá}
\end{itemize}
\begin{itemize}
\item {Grp. gram.:adj.}
\end{itemize}
\begin{itemize}
\item {Proveniência:(De \textunderscore anti...\textunderscore  + \textunderscore rábico\textunderscore )}
\end{itemize}
Que é bom contra a hydrophobia ou raiva.
\section{Antirachítico}
\begin{itemize}
\item {fónica:raqui}
\end{itemize}
\begin{itemize}
\item {Grp. gram.:adj.}
\end{itemize}
Diz-se do tratamento ou medicamento contra o rachitismo.
\section{Antiracional}
\begin{itemize}
\item {fónica:ra}
\end{itemize}
\begin{itemize}
\item {Grp. gram.:adj.}
\end{itemize}
\begin{itemize}
\item {Proveniência:(De \textunderscore anti...\textunderscore  + \textunderscore racional\textunderscore )}
\end{itemize}
Contrário á razão.
\section{Antiracionalismo}
\begin{itemize}
\item {fónica:ra}
\end{itemize}
\begin{itemize}
\item {Grp. gram.:adj.}
\end{itemize}
Opposto ás doutrinas dos racionalistas.
\section{Antireal}
\begin{itemize}
\item {fónica:re}
\end{itemize}
\begin{itemize}
\item {Grp. gram.:adj.}
\end{itemize}
Que não é real; imaginário. Cf. Latino, \textunderscore Humboldt\textunderscore , 416.
\section{Antirealismo}
\begin{itemize}
\item {fónica:re}
\end{itemize}
\begin{itemize}
\item {Grp. gram.:m.}
\end{itemize}
Aquillo que é opposto ao realismo.
\section{Antirealista}
\begin{itemize}
\item {fónica:re}
\end{itemize}
\begin{itemize}
\item {Grp. gram.:adj.}
\end{itemize}
Opposto a realista.
\section{Antireformista}
\begin{itemize}
\item {fónica:re}
\end{itemize}
\begin{itemize}
\item {Grp. gram.:adj.}
\end{itemize}
Opposto aos reformistas.
\section{Antiregulamentar}
\begin{itemize}
\item {fónica:re}
\end{itemize}
\begin{itemize}
\item {Grp. gram.:adj.}
\end{itemize}
Opposto aos regulamentos.
\section{Antireligioso}
\begin{itemize}
\item {fónica:re}
\end{itemize}
\begin{itemize}
\item {Grp. gram.:adj.}
\end{itemize}
Contrário á religião.
\section{Antirepublicanismo}
\begin{itemize}
\item {fónica:ré}
\end{itemize}
\begin{itemize}
\item {Grp. gram.:m.}
\end{itemize}
Qualidade de antirepublicano.
\section{Antirepublicano}
\begin{itemize}
\item {fónica:ré}
\end{itemize}
\begin{itemize}
\item {Grp. gram.:adj.}
\end{itemize}
Contrário á república.
\section{Antireumatismal}
\begin{itemize}
\item {fónica:reu}
\end{itemize}
\begin{itemize}
\item {Grp. gram.:adj.}
\end{itemize}
Applicável contra o reumatismo.
\section{Antirevolucionário}
\begin{itemize}
\item {fónica:re}
\end{itemize}
\begin{itemize}
\item {Grp. gram.:adj.}
\end{itemize}
Contrário ás revoluções.
\section{Antirrábico}
\begin{itemize}
\item {Grp. gram.:adj.}
\end{itemize}
\begin{itemize}
\item {Proveniência:(De \textunderscore anti...\textunderscore  + \textunderscore rábico\textunderscore )}
\end{itemize}
Que é bom contra a hydrophobia ou raiva.
\section{Antirraquítico}
\begin{itemize}
\item {Grp. gram.:adj.}
\end{itemize}
Diz-se do tratamento ou medicamento contra o raquitismo.
\section{Antirracional}
\begin{itemize}
\item {Grp. gram.:adj.}
\end{itemize}
\begin{itemize}
\item {Proveniência:(De \textunderscore anti...\textunderscore  + \textunderscore racional\textunderscore )}
\end{itemize}
Contrário á razão.
\section{Antirracionalismo}
\begin{itemize}
\item {Grp. gram.:adj.}
\end{itemize}
Opposto ás doutrinas dos racionalistas.
\section{Antirreal}
\begin{itemize}
\item {Grp. gram.:adj.}
\end{itemize}
Que não é real; imaginário. Cf. Latino, \textunderscore Humboldt\textunderscore , 416.
\section{Antirrealismo}
\begin{itemize}
\item {Grp. gram.:m.}
\end{itemize}
Aquillo que é opposto ao realismo.
\section{Antirrealista}
\begin{itemize}
\item {Grp. gram.:adj.}
\end{itemize}
Opposto a realista.
\section{Antirreformista}
\begin{itemize}
\item {Grp. gram.:adj.}
\end{itemize}
Opposto aos reformistas.
\section{Antirregulamentar}
\begin{itemize}
\item {Grp. gram.:adj.}
\end{itemize}
Opposto aos regulamentos.
\section{Antirreligioso}
\begin{itemize}
\item {Grp. gram.:adj.}
\end{itemize}
Contrário á religião.
\section{Antirrepublicanismo}
\begin{itemize}
\item {Grp. gram.:m.}
\end{itemize}
Qualidade de antirepublicano.
\section{Antirrepublicano}
\begin{itemize}
\item {Grp. gram.:adj.}
\end{itemize}
Contrário á república.
\section{Antirreumatismal}
\begin{itemize}
\item {Grp. gram.:adj.}
\end{itemize}
Applicável contra o reumatismo.
\section{Antirrevolucionário}
\begin{itemize}
\item {Grp. gram.:adj.}
\end{itemize}
Contrário ás revoluções.
\section{Antirrhino}
\begin{itemize}
\item {Grp. gram.:m.}
\end{itemize}
\begin{itemize}
\item {Proveniência:(Gr. \textunderscore antirrhinon\textunderscore )}
\end{itemize}
Planta herbácea.
\section{Antirrino}
\begin{itemize}
\item {Grp. gram.:m.}
\end{itemize}
\begin{itemize}
\item {Proveniência:(Gr. \textunderscore antirrhinon\textunderscore )}
\end{itemize}
Planta herbácea.
\section{Antisátira}
\begin{itemize}
\item {fónica:sá}
\end{itemize}
\begin{itemize}
\item {Grp. gram.:f.}
\end{itemize}
Resposta a uma sátira.
\section{Antiscéptico}
\begin{itemize}
\item {Grp. gram.:adj.}
\end{itemize}
Opposto ao scepticismo.
\section{Antíscios}
\begin{itemize}
\item {Grp. gram.:m. pl.}
\end{itemize}
\begin{itemize}
\item {Utilização:Geogr.}
\end{itemize}
\begin{itemize}
\item {Proveniência:(Do gr. \textunderscore anti\textunderscore  + \textunderscore skia\textunderscore )}
\end{itemize}
Povos, que, ao meio dia, projectam a sua sombra em sentido reciprocamente opposto.
Habitantes de uma zona temperada, em relação aos da outra zona temperada.
\section{Antisemita}
\begin{itemize}
\item {fónica:se}
\end{itemize}
\begin{itemize}
\item {Grp. gram.:m.}
\end{itemize}
Inímigo da raça semítica, especialmente dos Judeus.
\section{Antisemítico}
\begin{itemize}
\item {fónica:se}
\end{itemize}
\begin{itemize}
\item {Grp. gram.:adj.}
\end{itemize}
Relativo aos antisemitas.
Contrário aos Semitas.
\section{Antisemitismo}
\begin{itemize}
\item {fónica:se}
\end{itemize}
\begin{itemize}
\item {Grp. gram.:m.}
\end{itemize}
Ódio aos Semitas; systema dos antisemitas.
\section{Antisepsia}
\begin{itemize}
\item {fónica:se}
\end{itemize}
\begin{itemize}
\item {Grp. gram.:f.}
\end{itemize}
\begin{itemize}
\item {Proveniência:(De \textunderscore anti...\textunderscore  + \textunderscore sepsia\textunderscore )}
\end{itemize}
Applicação de desinfectantes.
\section{Antisepsiador}
\begin{itemize}
\item {fónica:se}
\end{itemize}
\begin{itemize}
\item {Grp. gram.:adj.}
\end{itemize}
Que serve para antisepsiar.
Desinfectante.
\section{Antisepsiar}
\begin{itemize}
\item {fónica:se}
\end{itemize}
\begin{itemize}
\item {Grp. gram.:v. t.}
\end{itemize}
Livrar de sepsia.
Desinfectar.
Sanear.
\section{Antiséptico}
\begin{itemize}
\item {fónica:sé}
\end{itemize}
\begin{itemize}
\item {Grp. gram.:m.  e  adj.}
\end{itemize}
\begin{itemize}
\item {Proveniência:(Do gr. \textunderscore anti...\textunderscore  + \textunderscore septikos\textunderscore )}
\end{itemize}
O que impede a putrefacção.
\section{Antisiálico}
\begin{itemize}
\item {Grp. gram.:m.  e  adj.}
\end{itemize}
O mesmo que \textunderscore antisialógogo\textunderscore .
\section{Antisialógogo}
\begin{itemize}
\item {fónica:si}
\end{itemize}
\begin{itemize}
\item {Grp. gram.:m.  e  adj.}
\end{itemize}
Diz-se do medicamento, com que se combate a salivação.
\section{Antisociabilidade}
\begin{itemize}
\item {fónica:so}
\end{itemize}
\begin{itemize}
\item {Grp. gram.:f.}
\end{itemize}
Falta de sociabilidade.
Índole insociável.
\section{Antisocial}
\begin{itemize}
\item {fónica:so}
\end{itemize}
\begin{itemize}
\item {Grp. gram.:adj.}
\end{itemize}
Contrário á ordem social.
\section{Antisocialismo}
\begin{itemize}
\item {fónica:so}
\end{itemize}
\begin{itemize}
\item {Grp. gram.:m.}
\end{itemize}
Systema opposto ao socialismo.
\section{Antisocialista}
\begin{itemize}
\item {fónica:so}
\end{itemize}
\begin{itemize}
\item {Grp. gram.:m.  e  adj.}
\end{itemize}
Adversário do socialismo.
\section{Antisophista}
\begin{itemize}
\item {fónica:so}
\end{itemize}
\begin{itemize}
\item {Grp. gram.:adj.}
\end{itemize}
Contrário aos sophistas.
\section{Antispasmódico}
\begin{itemize}
\item {Grp. gram.:adj.}
\end{itemize}
\begin{itemize}
\item {Proveniência:(De \textunderscore anti...\textunderscore  + \textunderscore espasmódico\textunderscore )}
\end{itemize}
Recommendável contra os espasmos.
\section{Antispasto}
\begin{itemize}
\item {Grp. gram.:m.}
\end{itemize}
\begin{itemize}
\item {Proveniência:(Gr. \textunderscore antispastos\textunderscore )}
\end{itemize}
Pé de verso, composto de duas sýllabas longas entre duas breves, na poética grega e latina.
\section{Antispiritismo}
\begin{itemize}
\item {Grp. gram.:m.}
\end{itemize}
Qualidade de antispiritista.
\section{Antispiritista}
\begin{itemize}
\item {Grp. gram.:m.  e  adj.}
\end{itemize}
Adverso ao espiritismo.
\section{Antispiritualismo}
\begin{itemize}
\item {Grp. gram.:m.}
\end{itemize}
O mesmo que \textunderscore antiespiritualismo\textunderscore .
\section{Antissátira}
\begin{itemize}
\item {Grp. gram.:f.}
\end{itemize}
Resposta a uma sátira.
\section{Antissemita}
\begin{itemize}
\item {Grp. gram.:m.}
\end{itemize}
Inímigo da raça semítica, especialmente dos Judeus.
\section{Antissemítico}
\begin{itemize}
\item {Grp. gram.:adj.}
\end{itemize}
Relativo aos antisemitas.
Contrário aos Semitas.
\section{Antissemitismo}
\begin{itemize}
\item {Grp. gram.:m.}
\end{itemize}
Ódio aos Semitas; systema dos antisemitas.
\section{Antissepsia}
\begin{itemize}
\item {Grp. gram.:f.}
\end{itemize}
\begin{itemize}
\item {Proveniência:(De \textunderscore anti...\textunderscore  + \textunderscore sepsia\textunderscore )}
\end{itemize}
Applicação de desinfectantes.
\section{Antissepsiador}
\begin{itemize}
\item {Grp. gram.:adj.}
\end{itemize}
Que serve para antisepsiar.
Desinfectante.
\section{Antissepsiar}
\begin{itemize}
\item {Grp. gram.:v. t.}
\end{itemize}
Livrar de sepsia.
Desinfectar.
Sanear.
\section{Antisséptico}
\begin{itemize}
\item {Grp. gram.:m.  e  adj.}
\end{itemize}
\begin{itemize}
\item {Proveniência:(Do gr. \textunderscore anti...\textunderscore  + \textunderscore septikos\textunderscore )}
\end{itemize}
O que impede a putrefacção.
\section{Antissiálico}
\begin{itemize}
\item {Grp. gram.:m.  e  adj.}
\end{itemize}
O mesmo que \textunderscore antisialógogo\textunderscore .
\section{Antissialógogo}
\begin{itemize}
\item {Grp. gram.:m.  e  adj.}
\end{itemize}
Diz-se do medicamento, com que se combate a salivação.
\section{Antissifilítico}
\begin{itemize}
\item {Grp. gram.:adj.}
\end{itemize}
Contrário á sífilis.
\section{Antissimbólico}
\begin{itemize}
\item {Grp. gram.:adj.}
\end{itemize}
Contrário aos símbolos, ás imagens.
\section{Antissociabibilidade}
\begin{itemize}
\item {Grp. gram.:f.}
\end{itemize}
Falta de sociabilidade.
Índole insociável.
\section{Antissocial}
\begin{itemize}
\item {Grp. gram.:adj.}
\end{itemize}
Contrário á ordem social.
\section{Antissocialismo}
\begin{itemize}
\item {Grp. gram.:m.}
\end{itemize}
Systema opposto ao socialismo.
\section{Antissocialista}
\begin{itemize}
\item {Grp. gram.:m.  e  adj.}
\end{itemize}
Adversário do socialismo.
\section{Antissofista}
\begin{itemize}
\item {Grp. gram.:adj.}
\end{itemize}
Contrário aos sofistas.
\section{Antissudoral}
\begin{itemize}
\item {Grp. gram.:adj.}
\end{itemize}
Que modera a transpiração.
\section{Antiste}
\begin{itemize}
\item {Grp. gram.:m.}
\end{itemize}
\begin{itemize}
\item {Proveniência:(Lat. \textunderscore antistes\textunderscore )}
\end{itemize}
Pontífice, grande sacerdote, chefe do templo, entre os antigos Pagãos.
\section{Antístite}
\begin{itemize}
\item {Grp. gram.:m.}
\end{itemize}
O mesmo que \textunderscore antiste\textunderscore .
\section{Antístrofe}
\begin{itemize}
\item {Grp. gram.:f.}
\end{itemize}
\begin{itemize}
\item {Proveniência:(Gr. \textunderscore antistrophe\textunderscore )}
\end{itemize}
Segunda parte da ode antiga.
\section{Antístrophe}
\begin{itemize}
\item {Grp. gram.:f.}
\end{itemize}
\begin{itemize}
\item {Proveniência:(Gr. \textunderscore antistrophe\textunderscore )}
\end{itemize}
Segunda parte da ode antiga.
\section{Antistreptocócico}
\begin{itemize}
\item {Grp. gram.:adj.}
\end{itemize}
Applicável ou bom contra o estreptococo.
\section{Antistrumatico}
\begin{itemize}
\item {Grp. gram.:adj.}
\end{itemize}
O mesmo que \textunderscore antistrumoso\textunderscore .
\section{Antistrumoso}
\begin{itemize}
\item {Grp. gram.:adj.}
\end{itemize}
O mesmo que \textunderscore antiescrofuloso\textunderscore .
\section{Antisudoral}
\begin{itemize}
\item {fónica:su}
\end{itemize}
\begin{itemize}
\item {Grp. gram.:adj.}
\end{itemize}
Que modera a transpiração.
\section{Antisymbólico}
\begin{itemize}
\item {fónica:sim}
\end{itemize}
\begin{itemize}
\item {Grp. gram.:adj.}
\end{itemize}
Contrário aos sýmbolos, ás imagens.
\section{Antisyphilítico}
\begin{itemize}
\item {fónica:si}
\end{itemize}
\begin{itemize}
\item {Grp. gram.:adj.}
\end{itemize}
Contrário á sýphilis.
\section{Antiteatral}
\begin{itemize}
\item {Grp. gram.:adj.}
\end{itemize}
Impróprio de theatro.
Que não tem condições scênicas.
\section{Antiteísmo}
\begin{itemize}
\item {Grp. gram.:m.}
\end{itemize}
Systema, que considera a natureza divina e a humana como essencialmente oppostas.
\section{Antiteísta}
\begin{itemize}
\item {Grp. gram.:m.}
\end{itemize}
Sectário do antitheísmo.
\section{Antitenar}
\begin{itemize}
\item {Grp. gram.:m.}
\end{itemize}
\begin{itemize}
\item {Proveniência:(Do gr. \textunderscore anti\textunderscore  + \textunderscore thenar\textunderscore )}
\end{itemize}
Parte da mão, entre o pulso e a base do dedo minimo.
\section{Antiteológico}
\begin{itemize}
\item {Grp. gram.:adj.}
\end{itemize}
Contrário á teologia.
\section{Antitérmico}
\begin{itemize}
\item {Grp. gram.:adj.}
\end{itemize}
\begin{itemize}
\item {Proveniência:(De \textunderscore anti...\textunderscore  + \textunderscore thérmico\textunderscore )}
\end{itemize}
Opposto a calor.
Que faz baixar a temperatura.
\section{Antítese}
\begin{itemize}
\item {Grp. gram.:f.}
\end{itemize}
\begin{itemize}
\item {Proveniência:(Gr. \textunderscore antithesis\textunderscore )}
\end{itemize}
Opposição entre ideias ou palavras.
\section{Antitetânico}
\begin{itemize}
\item {Grp. gram.:adj.}
\end{itemize}
Applicável contra o tétano.
\section{Antitético}
\begin{itemize}
\item {Grp. gram.:adj.}
\end{itemize}
Que contém \textunderscore antitese\textunderscore .
\section{Antitheatral}
\begin{itemize}
\item {Grp. gram.:adj.}
\end{itemize}
Impróprio de theatro.
Que não tem condições scênicas.
\section{Antitheísmo}
\begin{itemize}
\item {Grp. gram.:m.}
\end{itemize}
Systema, que considera a natureza divina e a humana como essencialmente oppostas.
\section{Antitheísta}
\begin{itemize}
\item {Grp. gram.:m.}
\end{itemize}
Sectário do antitheísmo.
\section{Antithenar}
\begin{itemize}
\item {Grp. gram.:m.}
\end{itemize}
\begin{itemize}
\item {Proveniência:(Do gr. \textunderscore anti\textunderscore  + \textunderscore thenar\textunderscore )}
\end{itemize}
Parte da mão, entre o pulso e a base do dedo minimo.
\section{Antitheológico}
\begin{itemize}
\item {Grp. gram.:adj.}
\end{itemize}
Contrário á theologia.
\section{Antithérmico}
\begin{itemize}
\item {Grp. gram.:adj.}
\end{itemize}
\begin{itemize}
\item {Proveniência:(De \textunderscore anti...\textunderscore  + \textunderscore thérmico\textunderscore )}
\end{itemize}
Opposto a calor.
Que faz baixar a temperatura.
\section{Antíthese}
\begin{itemize}
\item {Grp. gram.:f.}
\end{itemize}
\begin{itemize}
\item {Proveniência:(Gr. \textunderscore antithesis\textunderscore )}
\end{itemize}
Opposição entre ideias ou palavras.
\section{Antithético}
\begin{itemize}
\item {Grp. gram.:adj.}
\end{itemize}
Que contém \textunderscore antithese\textunderscore .
\section{Antitísico}
\begin{itemize}
\item {Grp. gram.:adj.}
\end{itemize}
Applicável contra a tísica.
\section{Antitóxico}
\begin{itemize}
\item {Grp. gram.:adj.}
\end{itemize}
\begin{itemize}
\item {Grp. gram.:M.}
\end{itemize}
Contrário aos tóxicos.
Que serve de antídoto ou contraveneno.
Antídoto.
\section{Antitoxina}
\begin{itemize}
\item {Grp. gram.:f.}
\end{itemize}
Substância, que se fórma no organismo, sob a influência de uma infecção e que possue acção contrária á da respectiva toxina.
\section{Antitrago}
\begin{itemize}
\item {Grp. gram.:m.}
\end{itemize}
\begin{itemize}
\item {Utilização:Anat.}
\end{itemize}
Saliência do pavilhão auricular, fronteira ao trago.
\section{Antitrinitário}
\begin{itemize}
\item {Grp. gram.:adj.}
\end{itemize}
\begin{itemize}
\item {Proveniência:(De \textunderscore anti...\textunderscore  + \textunderscore trinitário\textunderscore )}
\end{itemize}
Contrário ao dogma da Trindade.
\section{Antítropo}
\begin{itemize}
\item {Grp. gram.:adj.}
\end{itemize}
\begin{itemize}
\item {Utilização:Bot.}
\end{itemize}
\begin{itemize}
\item {Proveniência:(Do gr. \textunderscore anti\textunderscore  + \textunderscore tropein\textunderscore )}
\end{itemize}
Diz-se do embryão, cuja radícula é opposta ao grão.
\section{Antiunionista}
\begin{itemize}
\item {Grp. gram.:adj.}
\end{itemize}
O mesmo que \textunderscore antiunitário\textunderscore .
\section{Antiunitário}
\begin{itemize}
\item {Grp. gram.:adj.}
\end{itemize}
Contrário á união de dois ou mais povos.
\section{Antiuniversitário}
\begin{itemize}
\item {Grp. gram.:adj.}
\end{itemize}
Contrário ao systema das Universidades ou de uma Universidade.
\section{Antivariólico}
\begin{itemize}
\item {Grp. gram.:adj.}
\end{itemize}
Applicável contra a varíola.
\section{Antivaticanista}
\begin{itemize}
\item {Grp. gram.:m.  e  adj.}
\end{itemize}
Adversário das ideias políticas e religiosas da Cúria romana.
\section{Antivenéreo}
\begin{itemize}
\item {Grp. gram.:adj.}
\end{itemize}
\begin{itemize}
\item {Proveniência:(De \textunderscore anti...\textunderscore  + \textunderscore venéreo\textunderscore )}
\end{itemize}
O mesmo que \textunderscore Antisyphilítico\textunderscore .
\section{Antiverminoso}
\begin{itemize}
\item {Grp. gram.:adj.}
\end{itemize}
O mesmo que \textunderscore vermífugo\textunderscore .
\section{Antiversista}
\begin{itemize}
\item {Grp. gram.:m.  e  adj.}
\end{itemize}
O que não tem vocação para versos.
Aquelle que não gosta de versos.
\section{Antiviril}
\begin{itemize}
\item {Grp. gram.:adj.}
\end{itemize}
Enervante.
Effeminado.
\section{Antivirulento}
\begin{itemize}
\item {Grp. gram.:adj.}
\end{itemize}
Opposto á propagação do vírus.
\section{Antivivisecção}
\begin{itemize}
\item {fónica:sé}
\end{itemize}
\begin{itemize}
\item {Grp. gram.:f.}
\end{itemize}
Systema ou opinião opposta á vivisecção.
\section{Antiviviseccionista}
\begin{itemize}
\item {fónica:se}
\end{itemize}
\begin{itemize}
\item {Grp. gram.:m.}
\end{itemize}
Partidário da antivivisecção.
\section{Antivivissecção}
\begin{itemize}
\item {Grp. gram.:f.}
\end{itemize}
Systema ou opinião opposta á vivissecção.
\section{Antivivisseccionista}
\begin{itemize}
\item {Grp. gram.:m.}
\end{itemize}
Partidário da antivivissecção.
\section{Antizímico}
\begin{itemize}
\item {Grp. gram.:adj.}
\end{itemize}
\begin{itemize}
\item {Proveniência:(Do gr. \textunderscore anti\textunderscore  + \textunderscore zume\textunderscore )}
\end{itemize}
Contrário á fermentação.
\section{Antizýmico}
\begin{itemize}
\item {Grp. gram.:adj.}
\end{itemize}
\begin{itemize}
\item {Proveniência:(Do gr. \textunderscore anti\textunderscore  + \textunderscore zume\textunderscore )}
\end{itemize}
Contrário á fermentação.
\section{Antliarino}
\begin{itemize}
\item {Grp. gram.:m.}
\end{itemize}
Gênero de coleópteros.
\section{Antóbio}
\begin{itemize}
\item {Grp. gram.:m.}
\end{itemize}
Gênero de coleópteros.
\section{Antobrânquio}
\begin{itemize}
\item {Grp. gram.:adj.}
\end{itemize}
Diz-se dos molluscos, cujas brânchias semelham ramalhetes de flôres.
\section{Antocéphalo}
\begin{itemize}
\item {Grp. gram.:m.}
\end{itemize}
\begin{itemize}
\item {Utilização:Bot.}
\end{itemize}
Gênero de rubiáceas.
\section{Antócera}
\begin{itemize}
\item {Grp. gram.:f.}
\end{itemize}
\begin{itemize}
\item {Utilização:Bot.}
\end{itemize}
Gênero de hepáticas.
\section{Antoclâmide}
\begin{itemize}
\item {Grp. gram.:f.}
\end{itemize}
\begin{itemize}
\item {Utilização:Bot.}
\end{itemize}
Gênero de herbáceas.
\section{Antocianina}
\begin{itemize}
\item {Grp. gram.:f.}
\end{itemize}
Substância còrante das flôres rubras, rosadas ou azues.
\section{Antodendro}
\begin{itemize}
\item {Grp. gram.:m.}
\end{itemize}
\begin{itemize}
\item {Utilização:Bot.}
\end{itemize}
Gênero de hericáceas.
\section{Antografia}
\begin{itemize}
\item {Grp. gram.:f.}
\end{itemize}
\begin{itemize}
\item {Proveniência:(Do gr. \textunderscore anthos\textunderscore  + \textunderscore graphein\textunderscore )}
\end{itemize}
Linguagem das flôres.
\section{Antográfico}
\begin{itemize}
\item {Grp. gram.:adj.}
\end{itemize}
Relativo á antografia.
\section{Antógrafo}
\begin{itemize}
\item {Grp. gram.:m.}
\end{itemize}
Aquelle que é versado em antografia.
\section{Antojadiço}
\begin{itemize}
\item {Grp. gram.:adj.}
\end{itemize}
\begin{itemize}
\item {Utilização:Ant.}
\end{itemize}
\begin{itemize}
\item {Proveniência:(De \textunderscore antojar\textunderscore )}
\end{itemize}
Que tem appetites. Cf. \textunderscore Eufrosina\textunderscore , 218.
\section{Antojador}
\begin{itemize}
\item {Grp. gram.:m.}
\end{itemize}
Aquelle que antoja. Cf. Garrett, \textunderscore D. Branca\textunderscore , 133.
\section{Antófago}
\begin{itemize}
\item {Grp. gram.:adj.}
\end{itemize}
\begin{itemize}
\item {Proveniência:(Do gr. \textunderscore anthos\textunderscore  + \textunderscore phagein\textunderscore )}
\end{itemize}
Que come flôres.
\section{Antófila}
\begin{itemize}
\item {Grp. gram.:f.}
\end{itemize}
\begin{itemize}
\item {Utilização:Entom.}
\end{itemize}
Gênero de lepidópteros.
\section{Antófilo}
\begin{itemize}
\item {Grp. gram.:adj.}
\end{itemize}
\begin{itemize}
\item {Proveniência:(Do gr. \textunderscore anthos\textunderscore  + \textunderscore philos\textunderscore )}
\end{itemize}
Que é amigo das flôres.
Que está habitualmente nas flôres.
\section{Antófilos}
\begin{itemize}
\item {Grp. gram.:m. pl.}
\end{itemize}
\begin{itemize}
\item {Proveniência:(Do gr. \textunderscore anthos\textunderscore  + \textunderscore philos\textunderscore )}
\end{itemize}
Nome, que os naturalistas deram a uma família de insectos, com quatro asas venenosas, estendidas, antenas filiformes, abdome redondo e lábio curto.
\section{Antóforo}
\begin{itemize}
\item {Grp. gram.:m.}
\end{itemize}
\begin{itemize}
\item {Utilização:Bot.}
\end{itemize}
\begin{itemize}
\item {Grp. gram.:M. pl.}
\end{itemize}
\begin{itemize}
\item {Utilização:Entom.}
\end{itemize}
\begin{itemize}
\item {Proveniência:(Do gr. \textunderscore anthos\textunderscore  + \textunderscore pherein\textunderscore )}
\end{itemize}
Receptáculo floral que, partindo do fundo do cálice, sustenta as pétalas, os estames e o pistillo, segundo De-Candolle.
Insectos, da tribo dos apiários.
\section{Antojar}
\begin{itemize}
\item {Grp. gram.:v. t.}
\end{itemize}
\begin{itemize}
\item {Proveniência:(De \textunderscore antojo\textunderscore ^1)}
\end{itemize}
Pôr á vista.
Representar na imaginação.
Figurar.
Appetecer.
\section{Antojo}
\begin{itemize}
\item {Grp. gram.:m.}
\end{itemize}
Apparência; figuração.
Appetite; capricho.
(Cast. \textunderscore antojo\textunderscore , de \textunderscore ante\textunderscore  + \textunderscore ojo\textunderscore )
\section{Antojo}
\begin{itemize}
\item {Grp. gram.:m.}
\end{itemize}
O mesmo que \textunderscore entejo\textunderscore , nojo, repugnância:«\textunderscore já toma antojo ao leite\textunderscore ». D. Franc. Manuel, \textunderscore Apólogos\textunderscore . Cf. Garção, II, 55.
\section{Antolhar}
\begin{itemize}
\item {Grp. gram.:v. t.}
\end{itemize}
O mesmo ou melhor que \textunderscore antojar\textunderscore .
\section{Antôlho}
\begin{itemize}
\item {Grp. gram.:m.}
\end{itemize}
O mesmo que \textunderscore antojo\textunderscore ^1.
\section{Antolhos}
\begin{itemize}
\item {Grp. gram.:m. pl.}
\end{itemize}
\begin{itemize}
\item {Proveniência:(De \textunderscore ante\textunderscore  + \textunderscore ôlho\textunderscore )}
\end{itemize}
Espécie de pala, com que se resguardam da luz os olhos doentes de alguém.
Palas accessórias dos cabrestos, postas de modo que os animaes só possam vêr em frente e para baixo.
O mesmo que \textunderscore ante-olhos\textunderscore .
\section{Antoliza}
\begin{itemize}
\item {Grp. gram.:f.}
\end{itemize}
\begin{itemize}
\item {Proveniência:(Do gr. \textunderscore anthos\textunderscore  + \textunderscore lussa\textunderscore )}
\end{itemize}
Gênero de plantas irídeas.
\section{Antologia}
\begin{itemize}
\item {Grp. gram.:f.}
\end{itemize}
\begin{itemize}
\item {Utilização:Fig.}
\end{itemize}
\begin{itemize}
\item {Proveniência:(Gr. \textunderscore anthologia\textunderscore )}
\end{itemize}
Tratado das flôres.
Collecção de flôres.
Escolha, collecção de poesias.
Collecção de trechos em prosa e verso.
Selecta; chrestomathia.
\section{Antologista}
\begin{itemize}
\item {Grp. gram.:m.}
\end{itemize}
Aquelle que é versado em antologia.
Colleccionador de poesias.
\section{Antóloma}
\begin{itemize}
\item {Grp. gram.:m.}
\end{itemize}
Gênero de plantas tiliáceas.
\section{Antomania}
\begin{itemize}
\item {Grp. gram.:f.}
\end{itemize}
\begin{itemize}
\item {Proveniência:(Do gr. \textunderscore anthos\textunderscore  + \textunderscore mania\textunderscore )}
\end{itemize}
Paixão pelas flôres.
\section{Antomaniaco}
\begin{itemize}
\item {Grp. gram.:m.  e  adj.}
\end{itemize}
O que tem \textunderscore antomania\textunderscore .
\section{Antómano}
\begin{itemize}
\item {Grp. gram.:m.  e  adj.}
\end{itemize}
O mesmo que \textunderscore antomaniaco\textunderscore .
\section{Antomizídeos}
\begin{itemize}
\item {Grp. gram.:m. pl.}
\end{itemize}
\begin{itemize}
\item {Proveniência:(Do gr. \textunderscore anthos\textunderscore  + \textunderscore mizein\textunderscore )}
\end{itemize}
Insectos dípteros, semelhantes ás moscas ordinárias.
\section{Antomizidos}
\begin{itemize}
\item {Grp. gram.:m. pl.}
\end{itemize}
O mesmo que \textunderscore antomizídeos\textunderscore .
\section{Antona}
\begin{itemize}
\item {Grp. gram.:f.}
\end{itemize}
Espécie de tecido antigo.
\section{Antónia}
\begin{itemize}
\item {Grp. gram.:f.}
\end{itemize}
\begin{itemize}
\item {Utilização:Bot.}
\end{itemize}
Gênero de loganiáceas.
\section{Antoniano}
\begin{itemize}
\item {Grp. gram.:adj.}
\end{itemize}
Relativo a Santo-António.
\section{Antonímia}
\begin{itemize}
\item {Grp. gram.:f.}
\end{itemize}
Qualidade dos vocábulos antónimos.
\section{Antonímica}
\begin{itemize}
\item {Grp. gram.:f.}
\end{itemize}
Estudo dos vocábulos antónimos.
\section{Antonímico}
\begin{itemize}
\item {Grp. gram.:f.}
\end{itemize}
Relativo á antonimia.
\section{Antónimo}
\begin{itemize}
\item {Grp. gram.:adj.}
\end{itemize}
\begin{itemize}
\item {Grp. gram.:M.}
\end{itemize}
\begin{itemize}
\item {Proveniência:(Do gr. \textunderscore anti\textunderscore  + \textunderscore onuma\textunderscore )}
\end{itemize}
Diz-se dos vocábulos, que têm significação reciprocamente opposta: \textunderscore cobrir, descobrir\textunderscore ; \textunderscore bom, mau\textunderscore .
Palavra antónima.
\section{Antoninho}
\begin{itemize}
\item {Grp. gram.:m.  e  adj.}
\end{itemize}
Dizia-se dos membros de uma Ordem religiosa, (cónegos regulares de Santo-António).
\section{Antoninho}
\begin{itemize}
\item {Grp. gram.:m.}
\end{itemize}
Variedade de peixe.
\section{Antonino}
\begin{itemize}
\item {Grp. gram.:adj.}
\end{itemize}
(V.antoniano)
\section{Antonista}
\begin{itemize}
\item {Grp. gram.:m.}
\end{itemize}
Partidário de D. António, Prior do Grato, que disputou a Fillipe II de Espanha o throno de Portugal.
\section{Antonomásia}
\begin{itemize}
\item {Grp. gram.:f.}
\end{itemize}
\begin{itemize}
\item {Utilização:Rhet.}
\end{itemize}
\begin{itemize}
\item {Proveniência:(Gr. \textunderscore antonomasia\textunderscore )}
\end{itemize}
Substituição de um nome próprio por um nome commum ou por períphrase; e vice-versa.
\section{Antonomástico}
\begin{itemize}
\item {Grp. gram.:adj.}
\end{itemize}
Que tem \textunderscore antonomásia\textunderscore .
\section{Antôntem}
\begin{itemize}
\item {Grp. gram.:adv.}
\end{itemize}
(V.anteontem)
\section{Antonýmia}
\begin{itemize}
\item {Grp. gram.:f.}
\end{itemize}
Qualidade dos vocábulos antónymos.
\section{Antonýmica}
\begin{itemize}
\item {Grp. gram.:f.}
\end{itemize}
Estudo dos vocábulos antónymos.
\section{Antonýmico}
\begin{itemize}
\item {Grp. gram.:f.}
\end{itemize}
Relativo á \textunderscore antonymia\textunderscore .
\section{Antónymo}
\begin{itemize}
\item {Grp. gram.:adj.}
\end{itemize}
\begin{itemize}
\item {Grp. gram.:M.}
\end{itemize}
\begin{itemize}
\item {Proveniência:(Do gr. \textunderscore anti\textunderscore  + \textunderscore onuma\textunderscore )}
\end{itemize}
Diz-se dos vocábulos, que têm significação reciprocamente opposta: \textunderscore cobrir, descobrir\textunderscore ; \textunderscore bom, mau\textunderscore .
Palavra antónyma.
\section{Antopogão}
\begin{itemize}
\item {Grp. gram.:m.}
\end{itemize}
Gênero de gramíneas.
\section{Antoras}
\begin{itemize}
\item {Grp. gram.:adv.}
\end{itemize}
\begin{itemize}
\item {Utilização:P. us.}
\end{itemize}
\begin{itemize}
\item {Proveniência:(De \textunderscore ante\textunderscore  + \textunderscore horas\textunderscore )}
\end{itemize}
Prematuramente; ante-sazão.
\section{Antorismo}
\begin{itemize}
\item {Grp. gram.:m.}
\end{itemize}
\begin{itemize}
\item {Utilização:Rhet.}
\end{itemize}
\begin{itemize}
\item {Proveniência:(Do gr. \textunderscore anti\textunderscore  + \textunderscore orismos\textunderscore )}
\end{itemize}
Substituição de uma palavra por outra mais correcta ou mais enérgica.
\section{Antoro}
\begin{itemize}
\item {Grp. gram.:m.}
\end{itemize}
Planta rainunculácea, cujos sucos são venenosos.
(Contr. de \textunderscore anti\textunderscore  + lat. \textunderscore thora\textunderscore )
\section{Antosoma}
\begin{itemize}
\item {Grp. gram.:f.}
\end{itemize}
Gênero de crustáceos.
\section{Antosperma}
\begin{itemize}
\item {Grp. gram.:m.}
\end{itemize}
\begin{itemize}
\item {Proveniência:(Do gr. \textunderscore anthos\textunderscore  + \textunderscore sperma\textunderscore )}
\end{itemize}
Pequenas concreções còradas, dispersas no tecido de certas plantas.
\section{Antospermo}
\begin{itemize}
\item {Grp. gram.:m.}
\end{itemize}
\begin{itemize}
\item {Utilização:Bot.}
\end{itemize}
Gênero de rubiáceas.
(Cp. \textunderscore anthosperma\textunderscore )
\section{Antostema}
\begin{itemize}
\item {Grp. gram.:f.}
\end{itemize}
\begin{itemize}
\item {Proveniência:(Do gr. \textunderscore anthos\textunderscore  + \textunderscore stema\textunderscore )}
\end{itemize}
Gênero de plantas euphorbiáceas.
\section{Antóstomo}
\begin{itemize}
\item {Grp. gram.:adj.}
\end{itemize}
\begin{itemize}
\item {Utilização:Hist. Nat.}
\end{itemize}
\begin{itemize}
\item {Proveniência:(Do gr. \textunderscore anthos\textunderscore  + \textunderscore stoma\textunderscore )}
\end{itemize}
Que tem, á volta da boca, apendices que dão aspecto de flôr.
\section{Antoxanteína}
\begin{itemize}
\item {Grp. gram.:f.}
\end{itemize}
O mesmo que \textunderscore antoxantina\textunderscore .
\section{Antoxantina}
\begin{itemize}
\item {Grp. gram.:f.}
\end{itemize}
Substância còrante das flôres amarelas.
\section{Antoxanto}
\begin{itemize}
\item {Grp. gram.:m.}
\end{itemize}
\begin{itemize}
\item {Proveniência:(Do gr. \textunderscore anthos\textunderscore  + \textunderscore xanthos\textunderscore )}
\end{itemize}
Nome scientifico de uma planta gramínea, vulgarmente conhecida por \textunderscore feno-de-cheiro\textunderscore .
\section{Antozoários}
\begin{itemize}
\item {Grp. gram.:m. pl.}
\end{itemize}
\begin{itemize}
\item {Proveniência:(Do gr. \textunderscore anthos\textunderscore  + \textunderscore zoarion\textunderscore )}
\end{itemize}
Família de polypeiros.
\section{Antracena}
\begin{itemize}
\item {Grp. gram.:f.}
\end{itemize}
O mesmo que \textunderscore antracina\textunderscore .
\section{Antracia}
\begin{itemize}
\item {Grp. gram.:f.}
\end{itemize}
Affecção análoga ao antraz.
\section{Antrácico}
\begin{itemize}
\item {Grp. gram.:adj.}
\end{itemize}
O mesmo que \textunderscore antrácino\textunderscore .
\section{Antracífero}
\begin{itemize}
\item {Grp. gram.:adj.}
\end{itemize}
\begin{itemize}
\item {Proveniência:(Do gr. \textunderscore anthrax\textunderscore  + lat. \textunderscore ferre\textunderscore )}
\end{itemize}
Que tem antracita.
\section{Antraciforme}
\begin{itemize}
\item {Grp. gram.:adj.}
\end{itemize}
Que tem a apparência de antraz.
\section{Antrácino}
\begin{itemize}
\item {Grp. gram.:adj.}
\end{itemize}
\begin{itemize}
\item {Proveniência:(Lat. \textunderscore anthracinus\textunderscore )}
\end{itemize}
Relativo ao antraz.
\section{Antracita}
\begin{itemize}
\item {Grp. gram.:f.}
\end{itemize}
\begin{itemize}
\item {Proveniência:(Do gr. \textunderscore anthrax\textunderscore )}
\end{itemize}
Carvão mineral, que arde com difficuldade, sem fumo nem cheiro.
\section{Antracitoso}
\begin{itemize}
\item {Grp. gram.:adj.}
\end{itemize}
Que contém \textunderscore antracite\textunderscore .
\section{Antracnose}
\begin{itemize}
\item {Grp. gram.:f.}
\end{itemize}
\begin{itemize}
\item {Proveniência:(Do gr. \textunderscore anthrax\textunderscore )}
\end{itemize}
Cogumelo parasito, ainda pouco conhecido, que ataca os rebentos das videiras.
Doença das vinhas, determinada pela acção daquelle parasito.
\section{Antracóide}
\begin{itemize}
\item {Grp. gram.:adj.}
\end{itemize}
\begin{itemize}
\item {Proveniência:(Do gr. \textunderscore anthrax\textunderscore  + \textunderscore eidos\textunderscore )}
\end{itemize}
Que tem a côr de carvão.
Que é semelhante ao antraz.
\section{Antracomancia}
\begin{itemize}
\item {Grp. gram.:f.}
\end{itemize}
\begin{itemize}
\item {Proveniência:(Do gr. \textunderscore anthrax\textunderscore  + \textunderscore manteia\textunderscore )}
\end{itemize}
Adivinhação pelo exame do carvão encandescente.
\section{Antracómetro}
\begin{itemize}
\item {Grp. gram.:m.}
\end{itemize}
\begin{itemize}
\item {Proveniência:(Do gr. \textunderscore anthrax\textunderscore  + \textunderscore metron\textunderscore )}
\end{itemize}
Instrumento, para determinar a quantidade de ácido carbónico, contido num fluido aeriforme.
\section{Antraconito}
\begin{itemize}
\item {Grp. gram.:m.}
\end{itemize}
\begin{itemize}
\item {Proveniência:(Do gr. \textunderscore anthrax\textunderscore )}
\end{itemize}
Uma das variedades de carbonato de cal.
\section{Antracose}
\begin{itemize}
\item {Grp. gram.:f.}
\end{itemize}
\begin{itemize}
\item {Proveniência:(Gr. \textunderscore anthracosis\textunderscore )}
\end{itemize}
Doença nos pulmões ou nos brônchios, caracterizada pela presença de uma substância escura, que tem os caracteres do carvão.
\section{Antracotério}
\begin{itemize}
\item {Grp. gram.:m.}
\end{itemize}
Gênero de mamíferos fósseis.
\section{Antranílico}
\begin{itemize}
\item {Grp. gram.:adj.}
\end{itemize}
Diz-se de um ácido, obtido pela acção da potassa sobre o indigo.
\section{Antraré}
\begin{itemize}
\item {fónica:ré}
\end{itemize}
\begin{itemize}
\item {Grp. gram.:f.}
\end{itemize}
\begin{itemize}
\item {Utilização:Náut.}
\end{itemize}
Designação vulgar, a bordo, da \textunderscore antearé\textunderscore .
\section{Antravante}
\begin{itemize}
\item {Grp. gram.:f.}
\end{itemize}
\begin{itemize}
\item {Utilização:Náut.}
\end{itemize}
Designação vulgar da \textunderscore anteavante\textunderscore .
\section{Antraz}
\begin{itemize}
\item {Grp. gram.:m.}
\end{itemize}
\begin{itemize}
\item {Proveniência:(Gr. \textunderscore anthrax\textunderscore )}
\end{itemize}
Carbúnculo, tumor gangrenoso e inflammatório.
Insecto díptero.
\section{Antre}
\begin{itemize}
\item {Grp. gram.:prep.}
\end{itemize}
\begin{itemize}
\item {Utilização:Ant.}
\end{itemize}
O mesmo que \textunderscore entre\textunderscore .
\section{Antrecambadamente}
\begin{itemize}
\item {Grp. gram.:adv.}
\end{itemize}
\begin{itemize}
\item {Utilização:Ant.}
\end{itemize}
Alternadamente. Cf. Cortesão, \textunderscore Subs\textunderscore .
\section{Antrelunho}
\begin{itemize}
\item {Grp. gram.:m.}
\end{itemize}
\begin{itemize}
\item {Utilização:Ant.}
\end{itemize}
O mesmo que \textunderscore interlúnio\textunderscore . Cf. \textunderscore Peregrinação\textunderscore , LXX.
\section{Antremez}
\begin{itemize}
\item {Grp. gram.:m.}
\end{itemize}
\begin{itemize}
\item {Utilização:Ant.}
\end{itemize}
O mesmo que \textunderscore entremez\textunderscore .
\section{Antrenos}
\begin{itemize}
\item {Grp. gram.:m. pl.}
\end{itemize}
\begin{itemize}
\item {Proveniência:(Do gr. \textunderscore antho\textunderscore  + \textunderscore rainein\textunderscore )}
\end{itemize}
Insectos coleópteros cujas larvas atacam as pelles e as collecções entomológicas.
\section{Antrepostos}
\begin{itemize}
\item {Grp. gram.:adj. pl.}
\end{itemize}
\begin{itemize}
\item {Utilização:Prov.}
\end{itemize}
\begin{itemize}
\item {Utilização:minh.}
\end{itemize}
Diz-se dos bois já enjugados, mas ainda não appostos ao carro nem seguros pela soga.
(Talvez por \textunderscore antepostos\textunderscore , de \textunderscore antepor\textunderscore )
\section{Antretalhador}
\begin{itemize}
\item {Grp. gram.:m.}
\end{itemize}
\begin{itemize}
\item {Utilização:Ant.}
\end{itemize}
O mesmo que \textunderscore entalhador\textunderscore .
(Por \textunderscore entretalhadar\textunderscore , de \textunderscore entre\textunderscore  + \textunderscore talhar\textunderscore )
\section{Antro}
\begin{itemize}
\item {Grp. gram.:m.}
\end{itemize}
\begin{itemize}
\item {Proveniência:(Lat. \textunderscore antrum\textunderscore )}
\end{itemize}
Caverna; cova natural, funda e escura.
Habitação miserável e escura.
Abrigo de criminosos.
\section{Antrolhos}
\begin{itemize}
\item {Grp. gram.:m. pl.}
\end{itemize}
(V.antolhos)
\section{Antromia}
\begin{itemize}
\item {Grp. gram.:f.}
\end{itemize}
Môsca alongada, que deposita seus ovos em substâncias gordas, especialmente nos queijos, produzindo graves irritações intestinaes, (\textunderscore anthromeya errabunda\textunderscore , Lottiez).
\section{Antropagogia}
\begin{itemize}
\item {Grp. gram.:f.}
\end{itemize}
\begin{itemize}
\item {Proveniência:(Do gr. \textunderscore anthropos\textunderscore  + \textunderscore agoge\textunderscore )}
\end{itemize}
Pedagogia social, tendente a alargar a acção educativa para fóra da escola e da família.
\section{Antropeiano}
\begin{itemize}
\item {Grp. gram.:adj.}
\end{itemize}
\begin{itemize}
\item {Utilização:Geol.}
\end{itemize}
\begin{itemize}
\item {Proveniência:(Do gr. \textunderscore anthropeios\textunderscore )}
\end{itemize}
Diz-se do terreno coetâneo do apparecimento do homem.
\section{Antropina}
\begin{itemize}
\item {Grp. gram.:f.}
\end{itemize}
\begin{itemize}
\item {Proveniência:(Do gr. \textunderscore anthropos\textunderscore )}
\end{itemize}
Mistura de estearina e palmitina, extraída da gordura humana.
\section{Antropocêntrico}
\begin{itemize}
\item {Grp. gram.:adj.}
\end{itemize}
Diz-se do systema philosóphico, segundo o qual o homem é o centro de todo o universo, sendo-lhe por isso subordinadas todas as coisas e para elle criadas. Cf. Latino. \textunderscore Or. da Corôa\textunderscore , CXXXV.
\section{Antropocentrismo}
\begin{itemize}
\item {Grp. gram.:m.}
\end{itemize}
Doutrina antropocêntrica.
\section{Antropocentrista}
\begin{itemize}
\item {Grp. gram.:m.}
\end{itemize}
Sectário do antropocentrismo.
\section{Antropodiceia}
\begin{itemize}
\item {Grp. gram.:f.}
\end{itemize}
A justiça dos homens. Cf. Castilho, \textunderscore Avarento\textunderscore , 363.
\section{Antropofagia}
\begin{itemize}
\item {Grp. gram.:f.}
\end{itemize}
Estado de \textunderscore antropófago\textunderscore .
\section{Antropófago}
\begin{itemize}
\item {Grp. gram.:m.  e  adj.}
\end{itemize}
\begin{itemize}
\item {Proveniência:(Do gr. \textunderscore anthropos\textunderscore  + \textunderscore phagein\textunderscore )}
\end{itemize}
O que come carne humana.
\section{Antropofobia}
\begin{itemize}
\item {Grp. gram.:f.}
\end{itemize}
\begin{itemize}
\item {Proveniência:(De \textunderscore anthropophobo\textunderscore )}
\end{itemize}
Horror aos homens.
Misanthropia.
\section{Antropófobo}
\begin{itemize}
\item {Grp. gram.:m.  e  adj.}
\end{itemize}
\begin{itemize}
\item {Proveniência:(Do gr. \textunderscore anthropos\textunderscore  + \textunderscore phobos\textunderscore )}
\end{itemize}
O que teme ou que aborrece os homens.
Misanthropo.
\section{Antropoforme}
\begin{itemize}
\item {Grp. gram.:adj.}
\end{itemize}
\begin{itemize}
\item {Proveniência:(Do gr. \textunderscore anthropos\textunderscore  + lat. \textunderscore forma\textunderscore )}
\end{itemize}
Semelhante ao homem.
\section{Antropófugo}
\begin{itemize}
\item {Grp. gram.:m.  e  adj.}
\end{itemize}
\begin{itemize}
\item {Proveniência:(Do gr. \textunderscore anthropos\textunderscore  + lat. \textunderscore fugere\textunderscore )}
\end{itemize}
O mesmo que \textunderscore antropófobo\textunderscore .
\section{Antropogenesia}
\begin{itemize}
\item {Grp. gram.:f.}
\end{itemize}
\begin{itemize}
\item {Proveniência:(Gr. \textunderscore anthropogenesis\textunderscore )}
\end{itemize}
Sciência da geração humana.
Tratado dos phenómenos da reproducção do homem.
\section{Antropogenésico}
\begin{itemize}
\item {Grp. gram.:adj.}
\end{itemize}
Relativo á \textunderscore antropogenesia\textunderscore .
\section{Antropogenia}
\begin{itemize}
\item {Grp. gram.:f.}
\end{itemize}
O mesmo que \textunderscore antropogenesia\textunderscore .
\section{Antropogênico}
\begin{itemize}
\item {Grp. gram.:adj.}
\end{itemize}
Relativo á \textunderscore antropogenia\textunderscore .
\section{Antropoglifita}
\begin{itemize}
\item {Grp. gram.:f.}
\end{itemize}
\begin{itemize}
\item {Proveniência:(Do gr. \textunderscore anthropos\textunderscore  + \textunderscore gluphos\textunderscore )}
\end{itemize}
Rocha, cuja configuração natural dá o aspecto de um homem ou de uma cara.
\section{Antropoglossa}
\begin{itemize}
\item {Grp. gram.:f.}
\end{itemize}
\begin{itemize}
\item {Utilização:Mús.}
\end{itemize}
\begin{itemize}
\item {Utilização:ant.}
\end{itemize}
\begin{itemize}
\item {Proveniência:(Do gr. \textunderscore anthropos\textunderscore  + \textunderscore glossa\textunderscore )}
\end{itemize}
Registo que, no órgão, se chama hoje \textunderscore voz humana\textunderscore .
\section{Antropognosia}
\begin{itemize}
\item {Grp. gram.:f.}
\end{itemize}
Conhecimento da natureza phýsica do homem.
\section{Antropografia}
\begin{itemize}
\item {Grp. gram.:f.}
\end{itemize}
\begin{itemize}
\item {Proveniência:(Do gr. \textunderscore anthropos\textunderscore  + \textunderscore graphein\textunderscore )}
\end{itemize}
Descripção do homem, como animal.
\section{Antropogrifo}
\begin{itemize}
\item {Grp. gram.:m.}
\end{itemize}
Homem alado, ou ave com fórma de homem. Cf. A. Dinis, \textunderscore Hyssope\textunderscore , 118.
\section{Antropoide}
\begin{itemize}
\item {Grp. gram.:adj.}
\end{itemize}
\begin{itemize}
\item {Grp. gram.:M.}
\end{itemize}
\begin{itemize}
\item {Proveniência:(Do gr. \textunderscore anthropos\textunderscore  + \textunderscore eidos\textunderscore )}
\end{itemize}
Semelhante ao homem.
Sêr, imaginado por alguns antropólogos, como transição do animal para o homem.
\section{Antropólatra}
\begin{itemize}
\item {Grp. gram.:m.}
\end{itemize}
Aquelle que pratica a antropolatria.
\section{Antropolatria}
\begin{itemize}
\item {Grp. gram.:f.}
\end{itemize}
\begin{itemize}
\item {Proveniência:(Do gr. \textunderscore anthropos\textunderscore  + \textunderscore latreia\textunderscore )}
\end{itemize}
Adoração do homem.
\section{Antropolátrico}
\begin{itemize}
\item {Grp. gram.:adj.}
\end{itemize}
Relativo a antropolatria.
\section{Antropólito}
\begin{itemize}
\item {Grp. gram.:m.}
\end{itemize}
\begin{itemize}
\item {Proveniência:(Do gr. \textunderscore anthropos\textunderscore  + \textunderscore lithos\textunderscore )}
\end{itemize}
Ossos humanos fósseis.
\section{Antropologia}
\begin{itemize}
\item {Grp. gram.:f.}
\end{itemize}
\begin{itemize}
\item {Proveniência:(De \textunderscore anthropólogo\textunderscore )}
\end{itemize}
História natural do homem.
Locução figurada, que atribue a Deus acções ou qualidades humanas.
\section{Antropológico}
\begin{itemize}
\item {Grp. gram.:adj.}
\end{itemize}
Relativo á \textunderscore antropologia\textunderscore .
\section{Antropologista}
\begin{itemize}
\item {Grp. gram.:f.}
\end{itemize}
Professor ou tratadista de antropologia.
\section{Antropólogo}
\begin{itemize}
\item {Grp. gram.:m.}
\end{itemize}
\begin{itemize}
\item {Proveniência:(Do gr. \textunderscore anthropos\textunderscore  + \textunderscore logos\textunderscore )}
\end{itemize}
Aquelle que é versado em antropologia.
\section{Antropomancia}
\begin{itemize}
\item {Grp. gram.:f.}
\end{itemize}
\begin{itemize}
\item {Proveniência:(Do gr. \textunderscore anthropos\textunderscore  + \textunderscore manteia\textunderscore )}
\end{itemize}
Processo de adivinhação, usado antigamente, observando-se as entranhas de uma criança ou de um homem recentemente degolado.
\section{Antropometria}
\begin{itemize}
\item {Grp. gram.:f.}
\end{itemize}
\begin{itemize}
\item {Proveniência:(Do gr. \textunderscore anthropos\textunderscore  + \textunderscore metron\textunderscore )}
\end{itemize}
Estudo comparativo das proporções das differentes partes do corpo humano.
\section{Antropométrico}
\begin{itemize}
\item {Grp. gram.:adj.}
\end{itemize}
Relativo á antropometria.
\section{Antropomórfico}
\begin{itemize}
\item {Grp. gram.:adj.}
\end{itemize}
O mesmo que \textunderscore antropomorfo\textunderscore .
\section{Antropomorfismo}
\begin{itemize}
\item {Grp. gram.:m.}
\end{itemize}
\begin{itemize}
\item {Proveniência:(Do gr. \textunderscore anthropos\textunderscore  + \textunderscore morphe\textunderscore )}
\end{itemize}
Systema dos que attribuem a Deus acções ou faculdades humanas.
\section{Antropomorfista}
\begin{itemize}
\item {Grp. gram.:m.  e  adj.}
\end{itemize}
Sectário do antropomorfsmo.
\section{Antropomorfo}
\begin{itemize}
\item {Grp. gram.:adj.}
\end{itemize}
\begin{itemize}
\item {Proveniência:(Do gr. \textunderscore anthropos\textunderscore  + \textunderscore morphe\textunderscore )}
\end{itemize}
Que é semelhante ao homem.
(T., proposto por Littré, em substituição do neol. hýbrido \textunderscore antropoforme\textunderscore ).
\section{Antroponomia}
\begin{itemize}
\item {Grp. gram.:f.}
\end{itemize}
\begin{itemize}
\item {Proveniência:(Do gr. \textunderscore anthropos\textunderscore  + \textunderscore nomos\textunderscore )}
\end{itemize}
Sciência da formação do homem.
\section{Antroponómico}
\begin{itemize}
\item {Grp. gram.:adj.}
\end{itemize}
Relativo á antroponomia.
\section{Antropopiteco}
\begin{itemize}
\item {Grp. gram.:m.}
\end{itemize}
\begin{itemize}
\item {Proveniência:(Do gr. \textunderscore anthropos\textunderscore , homem, e \textunderscore pithekos\textunderscore , macaco)}
\end{itemize}
Gênero hypothético de animaes fósseis, em que se julgou vêr os precursores dos homens.
\section{Antroposofia}
\begin{itemize}
\item {Grp. gram.:f.}
\end{itemize}
\begin{itemize}
\item {Proveniência:(Do gr. \textunderscore anthropos\textunderscore  + \textunderscore sophia\textunderscore )}
\end{itemize}
Sciência, que trata da natureza moral do homem.
\section{Antropotecnia}
\begin{itemize}
\item {Grp. gram.:f.}
\end{itemize}
\begin{itemize}
\item {Proveniência:(Do gr. \textunderscore anthropos\textunderscore  + \textunderscore tekhne\textunderscore )}
\end{itemize}
Arte de aperfeiçoar as faculdades do homem e de as adaptar ás necessidades da vida.
\section{Antropoteísmo}
\begin{itemize}
\item {Grp. gram.:m.}
\end{itemize}
\begin{itemize}
\item {Proveniência:(Do gr. \textunderscore anthropos\textunderscore  + \textunderscore theos\textunderscore )}
\end{itemize}
Deificação da humanidade.
\section{Antropoteísta}
\begin{itemize}
\item {Grp. gram.:m.}
\end{itemize}
Aquelle que deifica a humanidade.
\section{Antropoterapia}
\begin{itemize}
\item {Grp. gram.:f.}
\end{itemize}
Therapêutica das doenças humanas.
\section{Antropoterápico}
\begin{itemize}
\item {Grp. gram.:adj.}
\end{itemize}
Relativo á antropoterapia.
\section{Antropotomia}
\begin{itemize}
\item {Grp. gram.:f.}
\end{itemize}
\begin{itemize}
\item {Proveniência:(Do gr. \textunderscore anthropos\textunderscore  + \textunderscore tome\textunderscore )}
\end{itemize}
Anatomia do homem; dissecação do cadáver humano.
\section{Antropozoico}
\begin{itemize}
\item {Grp. gram.:adj.}
\end{itemize}
\begin{itemize}
\item {Utilização:Zool.}
\end{itemize}
\begin{itemize}
\item {Proveniência:(Do gr. \textunderscore anthropos\textunderscore  + \textunderscore zoon\textunderscore )}
\end{itemize}
Diz-se da quinta phase do período philogenético, na qual se suppõe que o homem appareceu na terra.
\section{Antura}
\begin{itemize}
\item {Grp. gram.:f.}
\end{itemize}
Gênero de crustáceos.
(Cp. \textunderscore anthúrio\textunderscore )
\section{Antúrio}
\begin{itemize}
\item {Grp. gram.:m.}
\end{itemize}
\begin{itemize}
\item {Proveniência:(Do gr. \textunderscore anthos\textunderscore  + \textunderscore oura\textunderscore )}
\end{itemize}
Gênero de plantas aroídeas.
\section{Antusa}
\begin{itemize}
\item {Grp. gram.:f.}
\end{itemize}
\begin{itemize}
\item {Utilização:Bot.}
\end{itemize}
Gênero de leguminosas.
\section{Anu}
\begin{itemize}
\item {Grp. gram.:m.}
\end{itemize}
\begin{itemize}
\item {Utilização:Bras}
\end{itemize}
Nome de duas aves trepadoras do Brasil.
Bailarico, espécie de fandango.
\section{Ânua}
\begin{itemize}
\item {Grp. gram.:f.}
\end{itemize}
\begin{itemize}
\item {Utilização:Ant.}
\end{itemize}
\begin{itemize}
\item {Proveniência:(De \textunderscore ânnuo\textunderscore )}
\end{itemize}
Chamavam-se assim as cartas, em que se referiam os successos de um anno:«\textunderscore em uma ânua da Companhia de Jesus, se refere...\textunderscore »\textunderscore Luz e Calor\textunderscore , 256.
\section{Anual}
\begin{itemize}
\item {Grp. gram.:adj.}
\end{itemize}
\begin{itemize}
\item {Proveniência:(Lat. \textunderscore annualis\textunderscore )}
\end{itemize}
Que dura um ano.
Que succede uma vez por ano.
Quantia, que se paga anualmente; prestação anual.
\section{Anualidade}
\begin{itemize}
\item {Grp. gram.:f.}
\end{itemize}
Qualidade do que é anual.
Pagamento, que se faz todos os anos.
\section{Anualmente}
\begin{itemize}
\item {Grp. gram.:adv.}
\end{itemize}
De modo anual.
Todos os anos.
De ano a ano.
\section{Anuário}
\begin{itemize}
\item {Grp. gram.:m.}
\end{itemize}
\begin{itemize}
\item {Proveniência:(De \textunderscore anno\textunderscore )}
\end{itemize}
Publicação anual.
\section{Anúdiva}
\begin{itemize}
\item {Grp. gram.:f.}
\end{itemize}
O mesmo que \textunderscore anaduva\textunderscore .
\section{Anúduva}
\begin{itemize}
\item {Grp. gram.:f.}
\end{itemize}
O mesmo que \textunderscore anaduva\textunderscore . Cf. Herculano, \textunderscore Quest. Púb.\textunderscore , II, 28; \textunderscore Idem\textunderscore , \textunderscore Hist. de Port.\textunderscore  (\textunderscore Passim\textunderscore )
\section{Anuência}
\begin{itemize}
\item {Grp. gram.:f.}
\end{itemize}
\begin{itemize}
\item {Proveniência:(De \textunderscore annuente\textunderscore )}
\end{itemize}
Acto de anuir.
\section{Anuente}
\begin{itemize}
\item {Grp. gram.:adj.}
\end{itemize}
\begin{itemize}
\item {Proveniência:(Lat. \textunderscore annuens\textunderscore )}
\end{itemize}
Que anue.
\section{Anuíba}
\begin{itemize}
\item {Grp. gram.:f.}
\end{itemize}
Espécie de loireiro do Brasil.
\section{Anuidade}
\begin{itemize}
\item {fónica:nu-i}
\end{itemize}
\begin{itemize}
\item {Grp. gram.:f.}
\end{itemize}
\begin{itemize}
\item {Proveniência:(De \textunderscore annuo\textunderscore )}
\end{itemize}
Anualidade.
Quantia, que, paga anualmente, abrange amortização e juro.
\section{Anuir}
\begin{itemize}
\item {Grp. gram.:v. i.}
\end{itemize}
\begin{itemize}
\item {Proveniência:(Lat. \textunderscore annuere\textunderscore )}
\end{itemize}
Dar consentimento.
Condescender.
Estar de acôrdo.
\section{Anuitário}
\begin{itemize}
\item {fónica:nu-i}
\end{itemize}
\begin{itemize}
\item {Grp. gram.:adj.}
\end{itemize}
Que se amortiza por anuidade.
\section{Anulabilidade}
\begin{itemize}
\item {Grp. gram.:f.}
\end{itemize}
Qualidade de anulável.
\section{Anulação}
\begin{itemize}
\item {Grp. gram.:f.}
\end{itemize}
Acto de \textunderscore anular\textunderscore .
\section{Anulador}
\begin{itemize}
\item {Grp. gram.:m.}
\end{itemize}
Aquelle que anula.
\section{Anulante}
\begin{itemize}
\item {Grp. gram.:adj.}
\end{itemize}
\begin{itemize}
\item {Proveniência:(Lat. \textunderscore annullans\textunderscore )}
\end{itemize}
Que anula.
\section{Anular}
\begin{itemize}
\item {Grp. gram.:v. t.}
\end{itemize}
\begin{itemize}
\item {Proveniência:(Lat. \textunderscore annullare\textunderscore )}
\end{itemize}
Tornar nullo.
Abolir.
Invalidar: \textunderscore o Ministro annullou um despacho do seu antecessor\textunderscore .
Aniquilar.
\section{Anular}
\begin{itemize}
\item {Grp. gram.:adj.}
\end{itemize}
\begin{itemize}
\item {Proveniência:(Lat. \textunderscore anularis\textunderscore )}
\end{itemize}
Que tem fórma de anel.
Próprio de anel.
\textunderscore Dedo anular\textunderscore , o dedo em que mais se usa o anel.
\section{Anulativo}
\begin{itemize}
\item {Grp. gram.:adj.}
\end{itemize}
Que anula.
\section{Anulatório}
\begin{itemize}
\item {Grp. gram.:adj.}
\end{itemize}
Que tem fôrça para anular.
\section{Anulável}
\begin{itemize}
\item {Grp. gram.:adj.}
\end{itemize}
Que póde sêr anulado.
\section{Anum}
\begin{itemize}
\item {Grp. gram.:m.}
\end{itemize}
O mesmo que \textunderscore anu\textunderscore .
\section{Anumeração}
\begin{itemize}
\item {Grp. gram.:f.}
\end{itemize}
\begin{itemize}
\item {Proveniência:(Lat. \textunderscore annumeratio\textunderscore )}
\end{itemize}
Acto de anumerar.
\section{Anumerar}
\begin{itemize}
\item {Grp. gram.:v. t.}
\end{itemize}
\begin{itemize}
\item {Utilização:Ant.}
\end{itemize}
\begin{itemize}
\item {Proveniência:(Lat. \textunderscore annumerare\textunderscore )}
\end{itemize}
Addicionar.
Numerar.
\section{Anunciação}
\begin{itemize}
\item {Grp. gram.:f.}
\end{itemize}
\begin{itemize}
\item {Proveniência:(Lat. \textunderscore annunciatio\textunderscore )}
\end{itemize}
Acto de anunciar.
\section{Anunciada}
\begin{itemize}
\item {Grp. gram.:f.}
\end{itemize}
O mesmo que \textunderscore anunciação\textunderscore .
\section{Anunciador}
\begin{itemize}
\item {Grp. gram.:m.}
\end{itemize}
\begin{itemize}
\item {Proveniência:(Lat. \textunderscore annunciator\textunderscore )}
\end{itemize}
Aquelle que anuncía.
\section{Anunciante}
\begin{itemize}
\item {Grp. gram.:m.  e  adj.}
\end{itemize}
\begin{itemize}
\item {Proveniência:(Lat. \textunderscore annuncians\textunderscore )}
\end{itemize}
O que anuncía.
O que manda anúncios para os periódicos.
\section{Anunciar}
\begin{itemize}
\item {Grp. gram.:v. t.}
\end{itemize}
\begin{itemize}
\item {Proveniência:(Lat. \textunderscore annunciare\textunderscore )}
\end{itemize}
Dar notícia de: \textunderscore foi anunciar ao pai o resultado dos seus exames\textunderscore .
Publicar.
Predizer, presagiar: \textunderscore aquellas nuvens anunciam trovoada\textunderscore .
Fazer conhecer por anúncio: \textunderscore annunciar um leilão\textunderscore .
Manifestar.
Revelar: \textunderscore o rapaz anuncía as melhores aptidões\textunderscore .
Prevenir da presença ou da chegada de: \textunderscore o criado foi anunciar o médico\textunderscore .
\section{Anunciativo}
\begin{itemize}
\item {Grp. gram.:adj.}
\end{itemize}
Que anuncia, que contém anúncio.
\section{Anúncio}
\begin{itemize}
\item {Grp. gram.:m.}
\end{itemize}
\begin{itemize}
\item {Proveniência:(Lat. \textunderscore annuncius\textunderscore )}
\end{itemize}
Aviso, que torna conhecido um facto que se suppunha ignorado.
Aviso público: \textunderscore vi hoje um anúncio nos jornaes\textunderscore .
Prognóstico, preságio.
\section{Ânuo}
\begin{itemize}
\item {Grp. gram.:adj.}
\end{itemize}
\begin{itemize}
\item {Proveniência:(Lat. \textunderscore annuus\textunderscore )}
\end{itemize}
O mesmo que \textunderscore anual\textunderscore .
\section{Anuria}
\begin{itemize}
\item {Grp. gram.:f.}
\end{itemize}
\begin{itemize}
\item {Proveniência:(Do gr. \textunderscore an\textunderscore  + \textunderscore ouron\textunderscore )}
\end{itemize}
Suppressão da urina.
\section{Anuro}
\begin{itemize}
\item {Grp. gram.:adj.}
\end{itemize}
\begin{itemize}
\item {Grp. gram.:M. pl.}
\end{itemize}
\begin{itemize}
\item {Proveniência:(Do gr. \textunderscore an\textunderscore  + \textunderscore oura\textunderscore )}
\end{itemize}
Diz-se dos animaes amphíbios sem cauda.
Batrácios de pelle nua, sem cauda, e de corpo obeso.
\section{Ânus}
\begin{itemize}
\item {Grp. gram.:m.}
\end{itemize}
\begin{itemize}
\item {Proveniência:(Lat. \textunderscore anus\textunderscore )}
\end{itemize}
Abertura, por onde o intestino recto expelle os excrementos.
\section{Anuvear}
\textunderscore v. t.\textunderscore  (e der.)
O mesmo ou melhor que \textunderscore anuviar\textunderscore , etc.
\section{Anuviador}
\begin{itemize}
\item {Grp. gram.:m.}
\end{itemize}
Aquelle que anuvia.
\section{Anuviar}
\begin{itemize}
\item {Grp. gram.:v. t.}
\end{itemize}
Cobrir de nuvens; nublar.
\section{Anvalló}
\begin{itemize}
\item {Grp. gram.:m.}
\end{itemize}
(V.ambaló)
\section{Anvaló}
\begin{itemize}
\item {Grp. gram.:m.}
\end{itemize}
(V.ambaló)
\section{Anverso}
\begin{itemize}
\item {Grp. gram.:m.}
\end{itemize}
\begin{itemize}
\item {Proveniência:(Do lat. \textunderscore anteversus\textunderscore )}
\end{itemize}
Face de medalha.
\section{Anvidos}
\begin{itemize}
\item {Grp. gram.:adv.}
\end{itemize}
\begin{itemize}
\item {Utilização:Ant.}
\end{itemize}
\begin{itemize}
\item {Proveniência:(Do lat. \textunderscore invitus\textunderscore )}
\end{itemize}
Contra vontade; constrangidamente. Cf. \textunderscore Cancion. da Vaticana\textunderscore , 680; \textunderscore Cancion. Brancuti\textunderscore , 197; Alf. el Sábio, \textunderscore Cant. de Maria\textunderscore , 55.
\section{Anvidoso}
\begin{itemize}
\item {Grp. gram.:adj.}
\end{itemize}
\begin{itemize}
\item {Utilização:Ant.}
\end{itemize}
Invejoso.
\section{Anvula}
\begin{itemize}
\item {Grp. gram.:f.}
\end{itemize}
Pequena árvore africana, de fôlhas simples, pecioladas, e flôres axillares, muito aromáticas.
\section{Anxiedade}
\begin{itemize}
\item {Grp. gram.:f.}
\end{itemize}
(V.ansiedade)
\section{Anzicos}
\begin{itemize}
\item {Grp. gram.:m. pl.}
\end{itemize}
Antigo povo de cannibaes, na margem direita do Zaire.
\section{Anzina}
\begin{itemize}
\item {Grp. gram.:f.}
\end{itemize}
O mesmo que \textunderscore enzinha\textunderscore , árvore.
(Cast. \textunderscore encina\textunderscore )
\section{Anzol}
\begin{itemize}
\item {Grp. gram.:m.}
\end{itemize}
\begin{itemize}
\item {Utilização:Fig.}
\end{itemize}
Pequeno gancho, terminado em farpa, para segurar a isca, com que se pesca.
Ardil, engano.
\section{Anzolado}
\begin{itemize}
\item {Grp. gram.:adj.}
\end{itemize}
Em fórma de anzol.
\section{Anzoleiro}
\begin{itemize}
\item {Grp. gram.:m.}
\end{itemize}
Fabricante ou vendedor de anzóes.
\section{Anzolo}
\begin{itemize}
\item {Grp. gram.:m.}
\end{itemize}
Bracelete de contas, vidrilhos, ou de outras coisas vistosas, mas de pequeno valor:«\textunderscore com anzolos na boca por não falar\textunderscore ». G. Resende,
\textunderscore Chrón. de D. João II\textunderscore , 2.^a p., c. 188.
\section{Anzom}
\begin{itemize}
\item {Grp. gram.:m.}
\end{itemize}
Árvore da Índia Portuguesa.
\section{Anzonice}
\begin{itemize}
\item {Grp. gram.:f.}
\end{itemize}
\begin{itemize}
\item {Utilização:Des.}
\end{itemize}
O mesmo que \textunderscore onzenice\textunderscore . Cf. Camillo, \textunderscore Sc. da Foz\textunderscore , 167.
\section{Ao}
\begin{itemize}
\item {fónica:âú}
\end{itemize}
(palavra composta da prep. \textunderscore a\textunderscore  e do art. \textunderscore o\textunderscore )
\section{Aoai}
\begin{itemize}
\item {Grp. gram.:m.}
\end{itemize}
O mesmo que \textunderscore aovai\textunderscore .
\section{Aonde}
\begin{itemize}
\item {Grp. gram.:adv.}
\end{itemize}
\begin{itemize}
\item {Proveniência:(De \textunderscore a\textunderscore  + \textunderscore onde\textunderscore )}
\end{itemize}
Para que lugar.
Para o qual lugar.
Para onde.
O mesmo que \textunderscore onde\textunderscore :«\textunderscore e aonde não houver condição...\textunderscore »\textunderscore Eufrosína\textunderscore , II, 6.
\section{Aóplo}
\begin{itemize}
\item {Grp. gram.:m.}
\end{itemize}
\begin{itemize}
\item {Proveniência:(Gr. \textunderscore aoplos\textunderscore )}
\end{itemize}
Orchídea indiana.
\section{Aora}
\begin{itemize}
\item {Grp. gram.:adv.}
\end{itemize}
\begin{itemize}
\item {Utilização:Ant.}
\end{itemize}
O mesmo que \textunderscore agora\textunderscore . Cf. \textunderscore Eufrosina\textunderscore , 203.
\section{Aorístico}
\begin{itemize}
\item {Grp. gram.:adj.}
\end{itemize}
Que diz respeito a \textunderscore aoristo\textunderscore .
\section{Aoristo}
\begin{itemize}
\item {Grp. gram.:m.}
\end{itemize}
\begin{itemize}
\item {Proveniência:(Gr. \textunderscore aoristos\textunderscore )}
\end{itemize}
Um dos tempos da conjugação grega.
Pretérito definido.
\section{Aorta}
\begin{itemize}
\item {Grp. gram.:f.}
\end{itemize}
\begin{itemize}
\item {Proveniência:(Gr. \textunderscore aorte\textunderscore )}
\end{itemize}
Artéria, que sái do ventrículo esquerdo do coração.
\section{Aortectasia}
\begin{itemize}
\item {Grp. gram.:f.}
\end{itemize}
\begin{itemize}
\item {Proveniência:(Do gr. \textunderscore aorte\textunderscore  + \textunderscore ktasis\textunderscore )}
\end{itemize}
Dilatação da aorta.
\section{Aorteurisma}
\begin{itemize}
\item {Grp. gram.:m.}
\end{itemize}
Aneurisma da aorta.
\section{Aórtico}
\begin{itemize}
\item {Grp. gram.:adj.}
\end{itemize}
Relativo á \textunderscore aorta\textunderscore .
\section{Aortite}
\begin{itemize}
\item {Grp. gram.:f.}
\end{itemize}
Inflammação da túnica externa da \textunderscore aorta\textunderscore .
\section{Aortoclasia}
\begin{itemize}
\item {Grp. gram.:f.}
\end{itemize}
\begin{itemize}
\item {Proveniência:(Do gr. \textunderscore aorte\textunderscore  + \textunderscore klasis\textunderscore )}
\end{itemize}
Ruptura da aorta.
\section{Aortoclastia}
\begin{itemize}
\item {Grp. gram.:f.}
\end{itemize}
\begin{itemize}
\item {Proveniência:(Do gr. \textunderscore aorte\textunderscore  + \textunderscore klasis\textunderscore )}
\end{itemize}
Ruptura da aorta.
\section{Aos}
(palavra composta da prep. \textunderscore a\textunderscore  e do art. def. pl. \textunderscore os\textunderscore )
\section{Aosadas}
\begin{itemize}
\item {Grp. gram.:adv.}
\end{itemize}
\begin{itemize}
\item {Utilização:Ant.}
\end{itemize}
Com atrevimento; ousadamente; com arrojo.
Com denodo.
Certamente.
(Cp. cast. \textunderscore osar\textunderscore , ousar)
\section{Aôto}
\begin{itemize}
\item {Grp. gram.:m.}
\end{itemize}
\begin{itemize}
\item {Proveniência:(Do gr. \textunderscore a\textunderscore  + \textunderscore ous\textunderscore , \textunderscore otos\textunderscore )}
\end{itemize}
Espécie de macaco da América.
\section{Aovai}
\begin{itemize}
\item {Grp. gram.:m.}
\end{itemize}
Planta solânea americana, de uma só fôlha.
\section{Apa}
\begin{itemize}
\item {Grp. gram.:f.}
\end{itemize}
Bolo de farinha de arroz e azeite de côco, usado na Ásia.
\section{Apacentar}
\textunderscore v. t.\textunderscore  (e der.)
O mesmo que \textunderscore apascentar\textunderscore .
\section{Apachorrar-se}
\begin{itemize}
\item {Grp. gram.:v. p.}
\end{itemize}
Encher-se de pachorra.
\section{Apacificar}
\textunderscore v. t.\textunderscore  (e der.)
O mesmo que \textunderscore pacificar\textunderscore , etc.
\section{Apadesar}
\begin{itemize}
\item {Grp. gram.:v. t.}
\end{itemize}
(V.empavesar)
\section{Apadrinhador}
\begin{itemize}
\item {Grp. gram.:m.  e  adj.}
\end{itemize}
O que apadrinha.
\section{Apadrinhamento}
\begin{itemize}
\item {Grp. gram.:m.}
\end{itemize}
Acto ou effeito de \textunderscore apadrinhar\textunderscore . Cf. Camillo, \textunderscore Quéda de um Anjo\textunderscore , 143.
\section{Apadrinhar}
\begin{itemize}
\item {Grp. gram.:v. t.}
\end{itemize}
Sêr padrinho de.
Proteger.
Defender.
\section{Apadroar}
\begin{itemize}
\item {Grp. gram.:v. t.}
\end{itemize}
Sêr padroeiro de:«\textunderscore ...de todos os patriarchas que apadroavam conventos\textunderscore ». Camillo, \textunderscore Neta do Arcediago\textunderscore , 77.
\section{Apaga}
\begin{itemize}
\item {Grp. gram.:f.}
\end{itemize}
\begin{itemize}
\item {Utilização:Náut.}
\end{itemize}
\begin{itemize}
\item {Proveniência:(De \textunderscore apagar\textunderscore )}
\end{itemize}
Cada um dos cabos que servem para carregar as testas dos papafigos.
\section{Apagadamente}
\begin{itemize}
\item {Grp. gram.:adv.}
\end{itemize}
Froixamente.
Sem brilho.
\section{Apagador}
\begin{itemize}
\item {Grp. gram.:m.}
\end{itemize}
Aquelle ou aquillo que apaga.
\section{Apagafanóes}
\begin{itemize}
\item {Grp. gram.:m. pl.}
\end{itemize}
\begin{itemize}
\item {Utilização:Náut.}
\end{itemize}
Cabos, com que se colhem as velas das gáveas.
\section{Apagamento}
\begin{itemize}
\item {Grp. gram.:m.}
\end{itemize}
Acto de \textunderscore apagar\textunderscore .
\section{Apagapenões}
\begin{itemize}
\item {Grp. gram.:m. pl.}
\end{itemize}
(V.apagafanóes)
\section{Apagapenóes}
\begin{itemize}
\item {Grp. gram.:m. pl.}
\end{itemize}
O mesmo que \textunderscore apagafanóes\textunderscore .
\section{Apagapenol}
\begin{itemize}
\item {Grp. gram.:m.}
\end{itemize}
(Mais us. no pl. \textunderscore apagapenóes\textunderscore )
\section{Apagar}
\begin{itemize}
\item {Grp. gram.:v. t.}
\end{itemize}
\begin{itemize}
\item {Proveniência:(De \textunderscore pagar\textunderscore )}
\end{itemize}
Extinguir (o fogo, a luz).
Fazer desaparecer a luz de (fogão, candeeiro, etc.).
Applacar: \textunderscore apagar discussões\textunderscore .
Abater; humilhar.
\section{Ápage!}
\begin{itemize}
\item {Grp. gram.:interj.}
\end{itemize}
\begin{itemize}
\item {Proveniência:(Gr. \textunderscore apage\textunderscore )}
\end{itemize}
Fóra d'aqui! some-te!
\section{Apagear}
\begin{itemize}
\item {Grp. gram.:v. t.}
\end{itemize}
Servir de pagem a.
Lisonjear, adular.
\section{Apagogia}
\begin{itemize}
\item {Grp. gram.:f.}
\end{itemize}
\begin{itemize}
\item {Proveniência:(Do gr. \textunderscore apagogè\textunderscore )}
\end{itemize}
Demonstração de uma proposição, pelo absurdo da contrária.
\section{Apaijar}
\begin{itemize}
\item {Grp. gram.:v. t.}
\end{itemize}
\begin{itemize}
\item {Utilização:Pop.}
\end{itemize}
O mesmo que \textunderscore apagear\textunderscore .
\section{Apainelado}
\begin{itemize}
\item {Grp. gram.:adj.}
\end{itemize}
\begin{itemize}
\item {Grp. gram.:M.}
\end{itemize}
\begin{itemize}
\item {Utilização:Constr.}
\end{itemize}
Que tem fórma de painel.
Ornato, formado por molduras, dispostas em fórma de painel.
\section{Apainelamento}
\begin{itemize}
\item {Grp. gram.:m.}
\end{itemize}
Acto de \textunderscore apainelar\textunderscore .
\section{Apainelar}
\begin{itemize}
\item {Grp. gram.:v. t.}
\end{itemize}
Dar fórma de painel a.
Ornar com molduras ou artezões (tecto ou parede).
\section{Apaisanado}
\begin{itemize}
\item {Grp. gram.:adj.}
\end{itemize}
Que tem modos ou aspecto de paisano.
\section{Apaisanar}
\begin{itemize}
\item {Grp. gram.:v. t.}
\end{itemize}
Dar modos, traje de paisano, a.
\section{Apaisar}
\begin{itemize}
\item {Grp. gram.:v. t.}
\end{itemize}
Pintar (um quadro em que entram paisagens).
\section{Apaixonadamente}
\begin{itemize}
\item {Grp. gram.:adv.}
\end{itemize}
De modo \textunderscore apaixonado\textunderscore .
Com paixão.
\section{Apaixonado}
\begin{itemize}
\item {Grp. gram.:adj.}
\end{itemize}
Dominado por paixão.
\section{Apaixonar}
\begin{itemize}
\item {Grp. gram.:v. t.}
\end{itemize}
Causar paixão a.
Exaltar.
Contristar.
\section{Apajar}
\begin{itemize}
\item {Grp. gram.:v.}
\end{itemize}
\begin{itemize}
\item {Utilização:t. Marn.}
\end{itemize}
Bater e alisar (montes de sal) com pajão.
\section{Apajar}
\begin{itemize}
\item {Grp. gram.:v. t.}
\end{itemize}
\begin{itemize}
\item {Utilização:Prov.}
\end{itemize}
\begin{itemize}
\item {Utilização:trasm.}
\end{itemize}
O mesmo que \textunderscore apagear\textunderscore .
\section{Apajar}
\begin{itemize}
\item {Grp. gram.:v. t.}
\end{itemize}
\begin{itemize}
\item {Utilização:Prov.}
\end{itemize}
\begin{itemize}
\item {Utilização:minh.}
\end{itemize}
\begin{itemize}
\item {Proveniência:(De \textunderscore pá\textunderscore . Cp. \textunderscore apajar\textunderscore ^1)}
\end{itemize}
Limpar (cereaes) na eira, lançando-os ao vento com a pá.
\section{Apalaçado}
\begin{itemize}
\item {Grp. gram.:adj.}
\end{itemize}
Que tem aspecto de palácio.
\section{Apalaçar}
\begin{itemize}
\item {Grp. gram.:v. t.}
\end{itemize}
Dar fórma ou grandeza de palácio a.
\section{Apalache}
\begin{itemize}
\item {Grp. gram.:m.}
\end{itemize}
Uma das linguas autóchtonas da América.
\section{Apalachina}
\begin{itemize}
\item {Grp. gram.:f.}
\end{itemize}
Arbusto, que cresce principalmente nos Apalaches, e cujas fôlhas se empregam em infusões medicinaes.
\section{Apalacianar}
\begin{itemize}
\item {Grp. gram.:v. t.}
\end{itemize}
Tornar palaciano.
\section{Apaladar}
\begin{itemize}
\item {Grp. gram.:v. t.}
\end{itemize}
Dar bom sabor a. Cf. \textunderscore Techn. Rur.\textunderscore , 17 e 264.
\section{Apalancado}
\begin{itemize}
\item {Grp. gram.:adj.}
\end{itemize}
Semelhante a palanque.
\section{Apalancado}
\begin{itemize}
\item {Grp. gram.:adj.}
\end{itemize}
Fechado com palancas.
\section{Apalancar}
\begin{itemize}
\item {Grp. gram.:v. t.}
\end{itemize}
Fechar com palancas.
\section{Apalancar}
\begin{itemize}
\item {Grp. gram.:v. t.}
\end{itemize}
Guarnecer com palanques.
\section{Apalancar}
\begin{itemize}
\item {Grp. gram.:v. t.}
\end{itemize}
\begin{itemize}
\item {Utilização:Prov.}
\end{itemize}
\begin{itemize}
\item {Utilização:trasm.}
\end{itemize}
Cavar no verão (a terra), para destruir as ervas nocivas.
\section{Apalancar}
\begin{itemize}
\item {Grp. gram.:v. t.}
\end{itemize}
\begin{itemize}
\item {Utilização:T. do Fundão}
\end{itemize}
Fazer oscillar (um objecto mal seguro em uma das suas extremidades): \textunderscore anda sempre a apalancar um dente que lhe está quási a cair\textunderscore .
\section{Apalátoa}
\begin{itemize}
\item {Grp. gram.:f.}
\end{itemize}
Gênero de plantas leguminosas, cujas espécies vivem principalmente na América tropical.
\section{Apalavrar}
\begin{itemize}
\item {Grp. gram.:v. t.}
\end{itemize}
Ajustar sôbre palavra.
Combinar; contratar.
\section{Apaleador}
\begin{itemize}
\item {Grp. gram.:m.}
\end{itemize}
O que apaleia.
\section{Apaleamento}
\begin{itemize}
\item {Grp. gram.:m.}
\end{itemize}
Acto de \textunderscore apalear\textunderscore .
\section{Apalear}
\begin{itemize}
\item {Grp. gram.:v. t.}
\end{itemize}
\begin{itemize}
\item {Utilização:Des.}
\end{itemize}
\begin{itemize}
\item {Proveniência:(Do lat. \textunderscore palus\textunderscore )}
\end{itemize}
Bater com pau.
Fustigar.
Espancar.
\section{Apalermado}
\begin{itemize}
\item {Grp. gram.:adj.}
\end{itemize}
Que tem modos de palerma.
\section{Apalermar-se}
\begin{itemize}
\item {Grp. gram.:v. p.}
\end{itemize}
Tornar-se palerma.
\section{Apalitros}
\begin{itemize}
\item {Grp. gram.:m. pl.}
\end{itemize}
Família de insectos coleópteros, de antennas filiformes e elytros molles.
\section{Apalmado}
\begin{itemize}
\item {Grp. gram.:adj.}
\end{itemize}
\begin{itemize}
\item {Utilização:Heráld.}
\end{itemize}
Diz-se do escudo, que tem uma mão, mostrando a palma.
\section{Apalmar}
\begin{itemize}
\item {Grp. gram.:v. t.}
\end{itemize}
(V.espalmar)
\section{Apalpação}
\begin{itemize}
\item {Grp. gram.:f.}
\end{itemize}
\begin{itemize}
\item {Utilização:Bras}
\end{itemize}
O mesmo que \textunderscore apalpadela\textunderscore .
\section{Apalpadeira}
\begin{itemize}
\item {Grp. gram.:m.}
\end{itemize}
\begin{itemize}
\item {Proveniência:(De \textunderscore apalpar\textunderscore )}
\end{itemize}
Mulher, que, nas estações aduaneiras e policiaes, verifica se pessôas do seu sexo trazem escondidos objectos, cujo porte ou transporte é vedado.
\section{Apalpadela}
\begin{itemize}
\item {Grp. gram.:f.}
\end{itemize}
Acto de \textunderscore apalpar\textunderscore .
\section{Apalpador}
\begin{itemize}
\item {Grp. gram.:m.}
\end{itemize}
Aquelle que apalpa.
\section{Apalpamento}
\begin{itemize}
\item {Grp. gram.:m.}
\end{itemize}
O mesmo que \textunderscore apalpadela\textunderscore .
\section{Apalpão}
\begin{itemize}
\item {Grp. gram.:m.}
\end{itemize}
O mesmo que \textunderscore apalpadela\textunderscore .
\section{Apalpar}
\begin{itemize}
\item {Grp. gram.:v. t.}
\end{itemize}
\begin{itemize}
\item {Proveniência:(De \textunderscore palpar\textunderscore )}
\end{itemize}
Tactear; tocar com a mão, para examinar com o tacto.
Sondar.
Ensaiar.
Molestar, abater: \textunderscore aquella doença apalpou-o\textunderscore .
\section{Apalpo}
\begin{itemize}
\item {Grp. gram.:m.}
\end{itemize}
(V.apalpadela)
\section{Apalytros}
\begin{itemize}
\item {Grp. gram.:m. pl.}
\end{itemize}
Família de insectos coleópteros, de antennas filiformes e elytros molles.
\section{Apan}
\begin{itemize}
\item {Grp. gram.:m.}
\end{itemize}
Marisco de Cabo-Verde e Senegal.
\section{Apanagem}
\begin{itemize}
\item {Grp. gram.:f.}
\end{itemize}
O mesmo que \textunderscore apanágio\textunderscore . Cf. Macedo, \textunderscore Burros\textunderscore , 274.
\section{Apanágio}
\begin{itemize}
\item {Grp. gram.:m.}
\end{itemize}
\begin{itemize}
\item {Utilização:Ant.}
\end{itemize}
Attributo; propriedade característica.
Pensão, ou propriedade, donde se tirava a pensão, que se dava a filhos segundos e viúvas nobres, em vida.
(B. lat. \textunderscore apanagium\textunderscore )
\section{Apanar}
\begin{itemize}
\item {Grp. gram.:v. t.}
\end{itemize}
\begin{itemize}
\item {Utilização:Ant.}
\end{itemize}
O mesmo que \textunderscore apanhar\textunderscore .
\section{Apanascado}
\begin{itemize}
\item {Grp. gram.:adj.}
\end{itemize}
\begin{itemize}
\item {Utilização:Prov.}
\end{itemize}
\begin{itemize}
\item {Utilização:minh.}
\end{itemize}
\begin{itemize}
\item {Proveniência:(De \textunderscore panasco\textunderscore ?)}
\end{itemize}
Parvo, atoleimado.
\section{Apancado}
\begin{itemize}
\item {Grp. gram.:adj.}
\end{itemize}
\begin{itemize}
\item {Utilização:Fam.}
\end{itemize}
\begin{itemize}
\item {Proveniência:(De \textunderscore pancada\textunderscore )}
\end{itemize}
Palerma; idiota.
\section{Apancamento}
\begin{itemize}
\item {Grp. gram.:m.}
\end{itemize}
Acto de \textunderscore apancar\textunderscore .
\section{Apancar}
\begin{itemize}
\item {Grp. gram.:v.}
\end{itemize}
\begin{itemize}
\item {Utilização:t. Marn.}
\end{itemize}
\begin{itemize}
\item {Utilização:Prov.}
\end{itemize}
\begin{itemize}
\item {Utilização:trasm.}
\end{itemize}
\begin{itemize}
\item {Proveniência:(De \textunderscore panca\textunderscore ? De \textunderscore pancada\textunderscore ?)}
\end{itemize}
Apagar com o ugalho (as pègadas do marnoto) nos meios, ainda molles, das salinas.
Fechar um pouco (uma janela), por causa do calor.
\section{Apancanado}
\begin{itemize}
\item {Grp. gram.:adj.}
\end{itemize}
\begin{itemize}
\item {Utilização:Prov.}
\end{itemize}
\begin{itemize}
\item {Utilização:trasm.}
\end{itemize}
O mesmo que \textunderscore apancado\textunderscore .
\section{Apandar}
\begin{itemize}
\item {Grp. gram.:v. t.}
\end{itemize}
Tornar pando, enconcar.
\section{Apandilhar-se}
\begin{itemize}
\item {Grp. gram.:v. p.}
\end{itemize}
\begin{itemize}
\item {Proveniência:(De \textunderscore pandilha\textunderscore )}
\end{itemize}
Conluiar-se para contratos dolosos.
Abandalhar-se.
Tornar-se vadio.
\section{Apanha}
\begin{itemize}
\item {Grp. gram.:f.}
\end{itemize}
\begin{itemize}
\item {Utilização:Prov.}
\end{itemize}
Acto de \textunderscore apanhar\textunderscore .
Colheita: \textunderscore a apanha da azeitona\textunderscore .
Cada um dos pedaes do tear.
\section{Apanhação}
\begin{itemize}
\item {Grp. gram.:f.}
\end{itemize}
\begin{itemize}
\item {Utilização:Bras}
\end{itemize}
O mesmo que \textunderscore apanha\textunderscore .
\section{Apanhadeira}
\begin{itemize}
\item {Grp. gram.:m.}
\end{itemize}
Mulher, que apanha cereaes, frutos.
Pá de apanhar o lixo que se juntou com a vassoira.
\section{Apanhadiço}
\begin{itemize}
\item {Grp. gram.:adj.}
\end{itemize}
Que se apanha facilmente.
\section{Apanhado}
\begin{itemize}
\item {Grp. gram.:m.}
\end{itemize}
Resumo, epítome: \textunderscore um apanhado de factos\textunderscore .
Refêgo, préga: \textunderscore os apanhados da saia\textunderscore .
\section{Apanhador}
\begin{itemize}
\item {Grp. gram.:m.}
\end{itemize}
Aquelle que apanha.
\section{Apanhadura}
\begin{itemize}
\item {Grp. gram.:f.}
\end{itemize}
(V.apanhamento)
\section{Apanha-gallegos}
\begin{itemize}
\item {Grp. gram.:m.}
\end{itemize}
Espécie de jôgo popular.
\section{Apanhamento}
\begin{itemize}
\item {Grp. gram.:m.}
\end{itemize}
O mesmo que \textunderscore apanha\textunderscore .
O mesmo que \textunderscore apanhado\textunderscore .
\section{Apanhamôscas}
\begin{itemize}
\item {Grp. gram.:f.}
\end{itemize}
\begin{itemize}
\item {Proveniência:(De \textunderscore apanhar\textunderscore  + \textunderscore môsca\textunderscore )}
\end{itemize}
Planta droserácea, (\textunderscore dionaea muscipula\textunderscore ), que fecha as fôlhas, quando nellas poisam insectos, e os mata.
Planta cariophyllácea, que, com a sua viscosidade, prende os insectos,
(\textunderscore silene muscipula\textunderscore ).
\section{Apanhar}
\begin{itemize}
\item {Grp. gram.:v. t.}
\end{itemize}
\begin{itemize}
\item {Utilização:Bras. de Minas}
\end{itemize}
Colher: \textunderscore apanhar a fruta\textunderscore .
Levantar do chão: \textunderscore apanhar um lenço\textunderscore .
Dobrar, arregaçar: \textunderscore apanhar o vestido\textunderscore .
Alcançar; obter: \textunderscore apanhar um emprêgo\textunderscore .
Surprehender; supportar: \textunderscore apanhar uma sova\textunderscore .
Aproveitar.
Roubar.
Pescar com rêde: \textunderscore apanhar peixe\textunderscore .
Caçar com armadilha: \textunderscore apanhar um pisco\textunderscore .
Amparar ou tapar (com as mãos):«\textunderscore ...com o rosto apanhado nas mãos\textunderscore ». Camillo, \textunderscore Brasileira\textunderscore , 164.
Comprar: \textunderscore por que preço apanhaste êsse cavallo\textunderscore ?
(B. lat. \textunderscore apanare\textunderscore )
\section{Apanhia}
\begin{itemize}
\item {Grp. gram.:f.}
\end{itemize}
\begin{itemize}
\item {Utilização:T. de Aveiro}
\end{itemize}
Apanha (da sardinha que sai da rêde rebentada).
\section{Apanho}
\begin{itemize}
\item {Grp. gram.:m.}
\end{itemize}
O mesmo que \textunderscore apanha\textunderscore .
\section{Apanicar}
\begin{itemize}
\item {Grp. gram.:v. t.}
\end{itemize}
\begin{itemize}
\item {Utilização:Prov.}
\end{itemize}
\begin{itemize}
\item {Utilização:beir.}
\end{itemize}
\begin{itemize}
\item {Proveniência:(De \textunderscore pano\textunderscore )}
\end{itemize}
Agasalhar com carinho.
Animar.
Mezinhar.
Tratar com desvelo exaggerado.
\section{Apaniguado}
\begin{itemize}
\item {Grp. gram.:m.}
\end{itemize}
Sectário.
Favorito.
Amouco.
(Por \textunderscore apaniaguado\textunderscore , do cast. \textunderscore paniaguado\textunderscore )
\section{Apanthropia}
\begin{itemize}
\item {Grp. gram.:f.}
\end{itemize}
O mesmo que \textunderscore misanthropia\textunderscore .
\section{Apanthrópico}
\begin{itemize}
\item {Grp. gram.:adj.}
\end{itemize}
Relativo á \textunderscore apanthropia\textunderscore .
\section{Apanthropo}
\begin{itemize}
\item {Grp. gram.:m.}
\end{itemize}
\begin{itemize}
\item {Proveniência:(Do gr. \textunderscore apo\textunderscore  + \textunderscore anthropos\textunderscore )}
\end{itemize}
Aquelle que tem \textunderscore apanthropia\textunderscore .
\section{Apantomancia}
\begin{itemize}
\item {Grp. gram.:f.}
\end{itemize}
Adivinhação ou previsão, por meio das coisas que inesperadamente se mostram.
\section{Apantos}
\begin{itemize}
\item {Grp. gram.:m. pl.}
\end{itemize}
\begin{itemize}
\item {Utilização:Prov.}
\end{itemize}
\begin{itemize}
\item {Utilização:minh.}
\end{itemize}
Graçolas com gestos.
\section{Apantropia}
\begin{itemize}
\item {Grp. gram.:f.}
\end{itemize}
O mesmo que \textunderscore misantropia\textunderscore .
\section{Apantrópico}
\begin{itemize}
\item {Grp. gram.:adj.}
\end{itemize}
Relativo á \textunderscore apantropia\textunderscore .
\section{Apantropo}
\begin{itemize}
\item {Grp. gram.:m.}
\end{itemize}
\begin{itemize}
\item {Proveniência:(Do gr. \textunderscore apo\textunderscore  + \textunderscore anthropos\textunderscore )}
\end{itemize}
Aquelle que tem \textunderscore apantropia\textunderscore .
\section{Apantufadas}
\begin{itemize}
\item {Grp. gram.:f. pl.}
\end{itemize}
\begin{itemize}
\item {Utilização:Ant.}
\end{itemize}
Espécie de chinelas. Cf. \textunderscore Eufrosina\textunderscore , 28.
(Cp. \textunderscore pantufo\textunderscore )
\section{Apantufado}
\begin{itemize}
\item {Grp. gram.:adj.}
\end{itemize}
Que tem fórma de pantufo.
\section{Apantufar-se}
\begin{itemize}
\item {Grp. gram.:v. p.}
\end{itemize}
Calçar pantufos.
\section{Apapá}
\begin{itemize}
\item {Grp. gram.:m.}
\end{itemize}
Peixe do Brasil.
\section{Apaparicar}
\begin{itemize}
\item {Grp. gram.:v. t.}
\end{itemize}
Dar paparicos a; tratar com gulodices.
Apanicar; amimar.
\section{Apapeirar}
\begin{itemize}
\item {Grp. gram.:v. t.}
\end{itemize}
\begin{itemize}
\item {Utilização:T. do Fundão}
\end{itemize}
O mesmo que \textunderscore apaparicar\textunderscore .
\section{Apapoilado}
\begin{itemize}
\item {Grp. gram.:adj.}
\end{itemize}
Que tem côr de papoila.
\section{Apapoulado}
\begin{itemize}
\item {Grp. gram.:adj.}
\end{itemize}
Que tem côr de papoila.
\section{A-par}
\begin{itemize}
\item {Grp. gram.:loc. prep.}
\end{itemize}
\begin{itemize}
\item {Grp. gram.:Loc. adv.}
\end{itemize}
\begin{itemize}
\item {Proveniência:(De \textunderscore a\textunderscore  + \textunderscore par\textunderscore )}
\end{itemize}
Junto; ao lado; ao pé: \textunderscore passou a-par êlle\textunderscore .
Parallelamente.
Ao lado um do outro: \textunderscore os dois passaram a-par\textunderscore .
\section{Apar}
\begin{itemize}
\item {Grp. gram.:m.}
\end{itemize}
Espécie de tatu.
\section{Apara}
\begin{itemize}
\item {Grp. gram.:f.}
\end{itemize}
\begin{itemize}
\item {Grp. gram.:Pl.}
\end{itemize}
\begin{itemize}
\item {Utilização:Marn.}
\end{itemize}
\begin{itemize}
\item {Proveniência:(De \textunderscore aparar\textunderscore )}
\end{itemize}
Aquillo que sai do objecto que se apara ou corta.
Raspa.
Maravalha.
Limalha.
O mesmo que \textunderscore bimbadura\textunderscore .
\section{Aparadeira}
\begin{itemize}
\item {Grp. gram.:f.}
\end{itemize}
\begin{itemize}
\item {Utilização:Prov.}
\end{itemize}
O mesmo que \textunderscore parteira\textunderscore .
O mesmo que \textunderscore arandela\textunderscore .
\section{Aparadela}
\begin{itemize}
\item {Grp. gram.:f.}
\end{itemize}
Acto de \textunderscore aparar\textunderscore .
\section{Aparador}
\begin{itemize}
\item {Grp. gram.:adj.}
\end{itemize}
\begin{itemize}
\item {Grp. gram.:M.}
\end{itemize}
\begin{itemize}
\item {Proveniência:(De \textunderscore aparar\textunderscore )}
\end{itemize}
Que apara.
Mesa, em que se colloca o que é preciso para o serviço da mesa de jantar.
Móvel antigo, geralmente de madeira preciosa, formando uma espécie de armario ou mesa alta, com muitas mas pequenas gavetas para joias, papéis, etc.
\section{Aparagem}
\begin{itemize}
\item {Grp. gram.:f.}
\end{itemize}
Acto de \textunderscore aparar\textunderscore .
\section{Aparaltar}
\textunderscore v. t.\textunderscore  (e der.)
O mesmo que \textunderscore aperaltar\textunderscore , etc.
\section{Aparalvilhar}
\textunderscore v. t.\textunderscore  (e der.)
O mesmo que \textunderscore aperalvilhar\textunderscore , etc.
\section{Aparamentar}
\begin{itemize}
\item {Grp. gram.:v. t.}
\end{itemize}
O mesmo que \textunderscore paramentar\textunderscore .
\section{Aparamentos}
\begin{itemize}
\item {Grp. gram.:m. pl.}
\end{itemize}
O mesmo que [[paramentos|paramento]]. Cf. Viana, \textunderscore Apostilas\textunderscore .
\section{Aparar}
\begin{itemize}
\item {Grp. gram.:v. t.}
\end{itemize}
\begin{itemize}
\item {Utilização:Gír.}
\end{itemize}
\begin{itemize}
\item {Utilização:Bras. de Minas}
\end{itemize}
\begin{itemize}
\item {Proveniência:(De \textunderscore parar\textunderscore )}
\end{itemize}
Tomar, receber (objecto que se atira).
Cercear as asperezas de; alisar.
Cortar as bordas de.
Fazer córte em (pennas ou lápis) para que escrevam.
Acceitar, receber (qualquer coisa).
Tolerar: \textunderscore são graças que eu não aparo\textunderscore .
Adular, bajular.
\section{Aparçar}
\begin{itemize}
\item {Grp. gram.:v. t.}
\end{itemize}
\begin{itemize}
\item {Utilização:Ant.}
\end{itemize}
O mesmo que \textunderscore aparceirar\textunderscore .
\section{Aparceirar}
\begin{itemize}
\item {Grp. gram.:v. t.}
\end{itemize}
Tomar, juntar como parceiro.
Associar.
\section{Aparcelado}
\begin{itemize}
\item {Grp. gram.:adj.}
\end{itemize}
\begin{itemize}
\item {Proveniência:(De \textunderscore parcel\textunderscore )}
\end{itemize}
Que tem parcéis.
\section{Aparcelar}
\begin{itemize}
\item {Grp. gram.:v. t.}
\end{itemize}
Dividir em parcelas; ordenar em parcelas.
Dividir, fraccionar.
\section{Aparcellar}
\begin{itemize}
\item {Grp. gram.:v. t.}
\end{itemize}
Dividir em parcellas; ordenar em parcellas.
Dividir, fraccionar.
\section{Aparentar}
\begin{itemize}
\item {Grp. gram.:v. t.}
\end{itemize}
Tornar parente; ligar por parentesco.
\section{Aparentelar}
\begin{itemize}
\item {Grp. gram.:v. t.}
\end{itemize}
\begin{itemize}
\item {Proveniência:(De \textunderscore parentela\textunderscore )}
\end{itemize}
O mesmo que \textunderscore aparentar\textunderscore ^1.
\section{Aparício}
\begin{itemize}
\item {Grp. gram.:m.}
\end{itemize}
\begin{itemize}
\item {Utilização:Ant.}
\end{itemize}
Festa da Epiphania.
\section{Apariço}
\begin{itemize}
\item {Grp. gram.:m.}
\end{itemize}
\begin{itemize}
\item {Utilização:Ant.}
\end{itemize}
O mesmo que \textunderscore Abril\textunderscore .
\section{Aparinas}
\begin{itemize}
\item {Grp. gram.:f. pl.}
\end{itemize}
\begin{itemize}
\item {Proveniência:(Do gr. \textunderscore aparine\textunderscore )}
\end{itemize}
Gênero de plantas.
\section{Aparo}
\begin{itemize}
\item {Grp. gram.:m.}
\end{itemize}
Acto de aparar.
Córte na penna, para escrever.
Penna de metal, que se adapta a uma caneta.
\section{Aparochianar-se}
\begin{itemize}
\item {fónica:qui}
\end{itemize}
\begin{itemize}
\item {Grp. gram.:v. p.}
\end{itemize}
Tornar-se parochiano, freguês.
\section{Aparoquianar-se}
\begin{itemize}
\item {Grp. gram.:v. p.}
\end{itemize}
Tornar-se paroquiano, freguês.
\section{Aparrado}
\begin{itemize}
\item {Grp. gram.:adj.}
\end{itemize}
Semelhante á parra.
Baixo e largo; atarracado.
\section{Aparrar}
\begin{itemize}
\item {Grp. gram.:v. i.}
\end{itemize}
\begin{itemize}
\item {Utilização:Prov.}
\end{itemize}
\begin{itemize}
\item {Utilização:beir.}
\end{itemize}
\begin{itemize}
\item {Utilização:Ext.}
\end{itemize}
Criar parra.
Criar fôlhas, cobrir-se de fôlhas, enramar-se.
\section{Aparreirado}
\begin{itemize}
\item {Grp. gram.:adj.}
\end{itemize}
Que tem fórma de parreira.
\section{Aparreirar}
\begin{itemize}
\item {Grp. gram.:v. t.}
\end{itemize}
Cercar, cobrir ou plantar de parreira.
\section{Aparta}
\begin{itemize}
\item {Grp. gram.:f.}
\end{itemize}
\begin{itemize}
\item {Utilização:Des.}
\end{itemize}
Acto de \textunderscore apartar\textunderscore .
\section{Apartação}
\begin{itemize}
\item {Grp. gram.:f.}
\end{itemize}
O mesmo que \textunderscore apartamento\textunderscore ^1.
Operação, em que o oiro se separa da prata em que foi engastado.
\section{Apartada}
\begin{itemize}
\item {Grp. gram.:f.}
\end{itemize}
O mesmo que \textunderscore apartamento\textunderscore ^1.
\section{Apartadamente}
\begin{itemize}
\item {Grp. gram.:adv.}
\end{itemize}
Com separação; em lugar \textunderscore apartado\textunderscore .
\section{Apartadiço}
\begin{itemize}
\item {Grp. gram.:m.  e  adj.}
\end{itemize}
\begin{itemize}
\item {Utilização:Prov.}
\end{itemize}
\begin{itemize}
\item {Utilização:alent.}
\end{itemize}
\begin{itemize}
\item {Proveniência:(De \textunderscore apartar\textunderscore )}
\end{itemize}
Diz-se do javardo novo, que já póde viver por si e abandonar a família.
\section{Apartador}
\begin{itemize}
\item {Grp. gram.:m.}
\end{itemize}
Aquelle que aparta.
\section{Apartamento}
\begin{itemize}
\item {Grp. gram.:m.}
\end{itemize}
Acto ou effeito de \textunderscore apartar\textunderscore .
\section{Apartamento}
\begin{itemize}
\item {Grp. gram.:m.}
\end{itemize}
\begin{itemize}
\item {Utilização:Ant.}
\end{itemize}
Compartimento, quarto, parte de um prédio destinado a uma família.
(Cp. fr. \textunderscore appartement\textunderscore )
\section{Apartar}
\begin{itemize}
\item {Grp. gram.:v. t.}
\end{itemize}
\begin{itemize}
\item {Proveniência:(De \textunderscore parte\textunderscore )}
\end{itemize}
Pôr de parte; separar.
Escolher.
Pôr longe.
Pôr em sítio escuro.
Desviar.
Evitar.
Dissuadir.
Desmamar.
Separar (oiro) da prata, em que estava engastado.
\section{Á-parte}
\begin{itemize}
\item {Grp. gram.:adv.}
\end{itemize}
\begin{itemize}
\item {Grp. gram.:M.}
\end{itemize}
\begin{itemize}
\item {Proveniência:(De \textunderscore parte\textunderscore )}
\end{itemize}
Separadamente.
Interrupção; palavra ou phrase, pronunciada emquanto outrem está discursando.
\section{Àparte}
\begin{itemize}
\item {Grp. gram.:adv.}
\end{itemize}
\begin{itemize}
\item {Grp. gram.:M.}
\end{itemize}
\begin{itemize}
\item {Proveniência:(De \textunderscore parte\textunderscore )}
\end{itemize}
Separadamente.
Interrupção; palavra ou phrase, pronunciada emquanto outrem está discursando.
\section{Apartear}
\begin{itemize}
\item {Grp. gram.:v. t.}
\end{itemize}
\begin{itemize}
\item {Utilização:bras}
\end{itemize}
\begin{itemize}
\item {Utilização:Neol.}
\end{itemize}
Dirigir apartes a, interromper com apartes: \textunderscore o orador foi muito aparteado\textunderscore .
\section{Aparvalhado}
\begin{itemize}
\item {Grp. gram.:adj.}
\end{itemize}
\begin{itemize}
\item {Proveniência:(De \textunderscore aparvalhar\textunderscore )}
\end{itemize}
Parvo, idiota.
Desnorteado.
\section{Aparvalhar}
\begin{itemize}
\item {Grp. gram.:v. t.}
\end{itemize}
Tornar parvo, atoleimado, desnorteado, atrapalhado.
\section{Aparvoar}
\begin{itemize}
\item {Grp. gram.:v. t.}
\end{itemize}
\begin{itemize}
\item {Proveniência:(De \textunderscore parvo\textunderscore )}
\end{itemize}
Tornar parvo, idiota.
\section{Apascaçar-se}
\begin{itemize}
\item {Grp. gram.:v. p.}
\end{itemize}
\begin{itemize}
\item {Utilização:Bras}
\end{itemize}
Tornar-se pascácio ou atoleimado.
\section{Apascentamento}
\begin{itemize}
\item {Grp. gram.:m.}
\end{itemize}
Acto ou effeito de \textunderscore apascentar\textunderscore .
Sustento. Cf. Castilho, \textunderscore Metam\textunderscore . XXXIII.
\section{Apascentar}
\begin{itemize}
\item {Grp. gram.:v. t.}
\end{itemize}
\begin{itemize}
\item {Proveniência:(De \textunderscore pascer\textunderscore )}
\end{itemize}
Trazer a pastar.
Pastorear.
Doutrinar.
Recrear: \textunderscore apascentar a vista\textunderscore .
\section{Apascoador}
\begin{itemize}
\item {Grp. gram.:m.  e  adj.}
\end{itemize}
O que apascôa.
\section{Apascoamento}
\begin{itemize}
\item {Grp. gram.:m.}
\end{itemize}
Acto ou effeito de \textunderscore apascoar\textunderscore .
\section{Apascoar}
\begin{itemize}
\item {Grp. gram.:v. t.}
\end{itemize}
\begin{itemize}
\item {Utilização:Ant.}
\end{itemize}
\begin{itemize}
\item {Proveniência:(Do lat. \textunderscore pascuus\textunderscore )}
\end{itemize}
O mesmo que \textunderscore apascentar\textunderscore .
\section{Apaso}
\begin{itemize}
\item {Grp. gram.:m.}
\end{itemize}
Mollusco acéphalo do Senegal.
\section{Apassamanado}
\begin{itemize}
\item {Grp. gram.:adj.}
\end{itemize}
Guarnecido de passamanes.
\section{Apassamanar}
\begin{itemize}
\item {Grp. gram.:v. t.}
\end{itemize}
Guarnecer, enfeitar, com passamanes.
\section{Apassionado}
\begin{itemize}
\item {Grp. gram.:adj.}
\end{itemize}
Fórma antiga de \textunderscore apaixonado\textunderscore . Cf. \textunderscore Aulegrafia\textunderscore , 165.
\section{Apassionar}
\begin{itemize}
\item {Grp. gram.:v. t.}
\end{itemize}
\begin{itemize}
\item {Utilização:Ant.}
\end{itemize}
O mesmo que \textunderscore apaixonar\textunderscore . Cf. \textunderscore Eufrosina\textunderscore , I, 1.
\section{Apassivação}
\begin{itemize}
\item {Grp. gram.:f.}
\end{itemize}
Acto de \textunderscore apassivar\textunderscore .
\section{Apassivado}
\begin{itemize}
\item {Grp. gram.:adj.}
\end{itemize}
\begin{itemize}
\item {Utilização:Gram.}
\end{itemize}
Empregado na voz passiva, ou como se fôsse passivo.
\section{Apassivador}
\begin{itemize}
\item {Grp. gram.:adj.}
\end{itemize}
Que apassiva.
\section{Apassivar}
\begin{itemize}
\item {Grp. gram.:v.}
\end{itemize}
\begin{itemize}
\item {Utilização:t. Gram.}
\end{itemize}
\begin{itemize}
\item {Proveniência:(De \textunderscore passivo\textunderscore )}
\end{itemize}
Empregar passivamente.
\section{Apatacado}
\begin{itemize}
\item {Grp. gram.:adj.}
\end{itemize}
\begin{itemize}
\item {Utilização:Bras. do N}
\end{itemize}
\begin{itemize}
\item {Proveniência:(De \textunderscore pataco\textunderscore )}
\end{itemize}
O mesmo que \textunderscore endinheirado\textunderscore .
\section{Apatetado}
\begin{itemize}
\item {Grp. gram.:adj.}
\end{itemize}
Que tem modos de pateta.
Que é pateta.
\section{Apatetar}
\begin{itemize}
\item {Grp. gram.:v. t.}
\end{itemize}
Tornar pateta.
\section{Apathia}
\begin{itemize}
\item {Grp. gram.:f.}
\end{itemize}
\begin{itemize}
\item {Proveniência:(Gr. \textunderscore apatheia\textunderscore )}
\end{itemize}
Indifferença.
Indolência.
Inacessibilidade a commoções.
\section{Apáthico}
\begin{itemize}
\item {Grp. gram.:adj.}
\end{itemize}
Que tem \textunderscore apathia\textunderscore .
\section{Apathizar}
\begin{itemize}
\item {Grp. gram.:v. t.}
\end{itemize}
Tornar apáthico.
\section{Apatia}
\begin{itemize}
\item {Grp. gram.:f.}
\end{itemize}
\begin{itemize}
\item {Proveniência:(Gr. \textunderscore apatheia\textunderscore )}
\end{itemize}
Indifferença.
Indolência.
Inacessibilidade a commoções.
\section{Apático}
\begin{itemize}
\item {Grp. gram.:adj.}
\end{itemize}
Que tem \textunderscore apatia\textunderscore .
\section{Apatifar}
\begin{itemize}
\item {Grp. gram.:v. t.}
\end{itemize}
Tornar patife, desprezível, reles:«\textunderscore apatifaram-lhe o nome\textunderscore ». \textunderscore Anat. Joc.\textunderscore , I, 306.
\section{Apatita}
\begin{itemize}
\item {Grp. gram.:f.}
\end{itemize}
Phosphato de cal natural, tão transparente, que chegou a confundir-se com pedras preciosas.
\section{Apatite}
\begin{itemize}
\item {Grp. gram.:f.}
\end{itemize}
O mesmo que \textunderscore apatito\textunderscore .
\section{Apatito}
\begin{itemize}
\item {Grp. gram.:m.}
\end{itemize}
(V.apatita)
\section{Apatrulhar-se}
\begin{itemize}
\item {Grp. gram.:v. p.}
\end{itemize}
Meter-se em patrulha.
\section{Apatúria}
\begin{itemize}
\item {Grp. gram.:f.}
\end{itemize}
\begin{itemize}
\item {Proveniência:(Lat. \textunderscore apaturia\textunderscore )}
\end{itemize}
Festa grega, em honra de Vênus. Cf. Castilho, \textunderscore Fastos\textunderscore , I, 539.
\section{Apaular}
\begin{itemize}
\item {fónica:pa-u}
\end{itemize}
\begin{itemize}
\item {Grp. gram.:v. t.}
\end{itemize}
Converter em paul; tornar pantanoso.
\section{Apavesar}
\textunderscore v. t.\textunderscore  (e der.)
O mesmo que \textunderscore empavesar\textunderscore , etc.
\section{Apavonar}
\textunderscore v. t.\textunderscore  (e der.)
O mesmo que \textunderscore empavonar\textunderscore , etc.
\section{Apavorador}
\begin{itemize}
\item {Grp. gram.:adj.}
\end{itemize}
Que apavora. Cf. Arn. Gama, \textunderscore Segr. do Ab.\textunderscore , 201.
\section{Apavoramento}
\begin{itemize}
\item {Grp. gram.:m.}
\end{itemize}
Acto ou effeito de \textunderscore apavorar\textunderscore .
\section{Apavorante}
\begin{itemize}
\item {Grp. gram.:adj.}
\end{itemize}
O mesmo que \textunderscore apavorador\textunderscore .
\section{Apavorar}
\begin{itemize}
\item {Grp. gram.:v. t.}
\end{itemize}
Causar pavor a; assustar; aterrar.
\section{Apaxitar}
\textunderscore v. t.\textunderscore  (?)«\textunderscore Meus bons annos apaxito\textunderscore ». \textunderscore Aulegrafia\textunderscore , 42.
\section{Apazigar}
\begin{itemize}
\item {Grp. gram.:v. t.}
\end{itemize}
\begin{itemize}
\item {Utilização:Ant.}
\end{itemize}
O mesmo que \textunderscore apaziguar\textunderscore . Cf. \textunderscore Luz e Calor\textunderscore , 543.
\section{Apazigo}
\begin{itemize}
\item {Grp. gram.:m.}
\end{itemize}
\begin{itemize}
\item {Proveniência:(De \textunderscore apazigar\textunderscore )}
\end{itemize}
Acto ou effeito de apaziguar. Cf. Camillo, \textunderscore Doze Cas.\textunderscore , 120.
\section{Apaziguadamente}
\begin{itemize}
\item {Grp. gram.:adv.}
\end{itemize}
\begin{itemize}
\item {Proveniência:(De \textunderscore apaziguado\textunderscore )}
\end{itemize}
Com sossêgo; em paz.
\section{Apaziguado}
\begin{itemize}
\item {Grp. gram.:adj.}
\end{itemize}
\begin{itemize}
\item {Proveniência:(De \textunderscore apaziguar\textunderscore )}
\end{itemize}
Sossegado.
Acalmado.
\section{Apaziguador}
\begin{itemize}
\item {Grp. gram.:m.}
\end{itemize}
Aquelle que apazigúa.
\section{Apaziguamento}
\begin{itemize}
\item {Grp. gram.:m.}
\end{itemize}
Acto de \textunderscore apaziguar\textunderscore .
\section{Apaziguar}
\begin{itemize}
\item {Grp. gram.:v. t.}
\end{itemize}
Pôr em paz; aquietar; sossegar.
(Cast. \textunderscore apaciguar\textunderscore )
\section{Apeadeira}
\begin{itemize}
\item {Grp. gram.:f.}
\end{itemize}
\begin{itemize}
\item {Proveniência:(De \textunderscore apear\textunderscore )}
\end{itemize}
Pedra ou tronco, para facilitar o apear-se alguém da cavalgadura e o subir para ella.
\section{Apeadeiro}
\begin{itemize}
\item {Grp. gram.:m.}
\end{itemize}
\begin{itemize}
\item {Proveniência:(De \textunderscore apear\textunderscore )}
\end{itemize}
Lugar, onde o combóio pára algumas vezes, só para deixar ou receber passageiros.
\section{Apeadoiro}
\begin{itemize}
\item {Grp. gram.:m.}
\end{itemize}
O mesmo que \textunderscore apeadeira\textunderscore  e \textunderscore apeadeiro\textunderscore .
\section{Apeadouro}
\begin{itemize}
\item {Grp. gram.:m.}
\end{itemize}
O mesmo que \textunderscore apeadeira\textunderscore  e \textunderscore apeadeiro\textunderscore .
\section{Apeamento}
\begin{itemize}
\item {Grp. gram.:m.}
\end{itemize}
Acto de \textunderscore apear\textunderscore .
\section{Apeanha}
\begin{itemize}
\item {Grp. gram.:f.}
\end{itemize}
O mesmo que \textunderscore peanha\textunderscore .
\section{Apeanhar}
\begin{itemize}
\item {Grp. gram.:v. t.}
\end{itemize}
Dar feitio ou semelhança de peanha a.
\section{Apear}
\begin{itemize}
\item {Grp. gram.:v. t.}
\end{itemize}
\begin{itemize}
\item {Grp. gram.:V. i.}
\end{itemize}
\begin{itemize}
\item {Proveniência:(De \textunderscore pé\textunderscore )}
\end{itemize}
Fazer descer.
Desmontar.
Pôr a pé.
Demolir um predio.
Humilhar.
Apear-se.
\section{Apeçonhar}
\textunderscore v. t.\textunderscore  (e der.)
O mesmo que \textunderscore empeçonhar\textunderscore , etc.
\section{Apeçonhentar}
\begin{itemize}
\item {Grp. gram.:v. t.}
\end{itemize}
\begin{itemize}
\item {Utilização:Ant.}
\end{itemize}
O mesmo que \textunderscore empeçonhar\textunderscore .
\section{Apedado}
\begin{itemize}
\item {Grp. gram.:adj.}
\end{itemize}
O mesmo que \textunderscore pedunculado\textunderscore .
\section{Apedar}
\begin{itemize}
\item {Grp. gram.:v. t.}
\end{itemize}
\begin{itemize}
\item {Utilização:Ant.}
\end{itemize}
\begin{itemize}
\item {Proveniência:(Do lat. \textunderscore pes\textunderscore , \textunderscore pedis\textunderscore )}
\end{itemize}
Segurar pelos pés; pear. Cf. \textunderscore Anat. Joc.\textunderscore , 41.
\section{Apedeuta}
\begin{itemize}
\item {Grp. gram.:m.}
\end{itemize}
\begin{itemize}
\item {Proveniência:(Do gr. \textunderscore a\textunderscore . priv. + \textunderscore paideuein\textunderscore , ensinar)}
\end{itemize}
Indivíduo ignorante, homem sem instrucção.
\section{Apedeutismo}
\begin{itemize}
\item {Grp. gram.:m.}
\end{itemize}
\begin{itemize}
\item {Proveniência:(De \textunderscore apedeuta\textunderscore )}
\end{itemize}
O mesmo que \textunderscore ignorância\textunderscore .
\section{Apedicelado}
\begin{itemize}
\item {Grp. gram.:adj.}
\end{itemize}
Que tem pedúnculo ou pedicelo.
\section{Apedicellado}
\begin{itemize}
\item {Grp. gram.:adj.}
\end{itemize}
Que tem pedúnculo ou pedicello.
\section{Apedoirar}
\begin{itemize}
\item {Grp. gram.:v. t.}
\end{itemize}
\begin{itemize}
\item {Utilização:Prov.}
\end{itemize}
\begin{itemize}
\item {Proveniência:(De \textunderscore pedoiro\textunderscore ^2)}
\end{itemize}
Ajuntar em mealheiro; enthesoirar.
\section{Apedourar}
\begin{itemize}
\item {Grp. gram.:v. t.}
\end{itemize}
\begin{itemize}
\item {Utilização:Prov.}
\end{itemize}
\begin{itemize}
\item {Proveniência:(De \textunderscore pedoiro\textunderscore ^2)}
\end{itemize}
Ajuntar em mealheiro; enthesoirar.
\section{Apedramento}
\begin{itemize}
\item {Grp. gram.:m.}
\end{itemize}
Acto de \textunderscore apedrar\textunderscore .
\section{Apedrar}
\begin{itemize}
\item {Grp. gram.:v. t.}
\end{itemize}
\begin{itemize}
\item {Utilização:Ant.}
\end{itemize}
\begin{itemize}
\item {Utilização:Mod.}
\end{itemize}
\begin{itemize}
\item {Grp. gram.:V. i.}
\end{itemize}
\begin{itemize}
\item {Proveniência:(De \textunderscore pedra\textunderscore )}
\end{itemize}
O mesmo que \textunderscore apedrejar\textunderscore .
O mesmo que \textunderscore empedrar\textunderscore .
Endurecer: \textunderscore a fruta apedrou\textunderscore .
\section{Apedregulhar}
\begin{itemize}
\item {Grp. gram.:v. t.}
\end{itemize}
Encher de pedregulhos.
\section{Apedrejador}
\begin{itemize}
\item {Grp. gram.:m.}
\end{itemize}
Aquelle que apedreja.
\section{Apedrejamento}
\begin{itemize}
\item {Grp. gram.:m.}
\end{itemize}
Acto de \textunderscore apedrejar\textunderscore .
\section{Apedrejar}
\begin{itemize}
\item {Grp. gram.:v. t.}
\end{itemize}
Atirar pedras a.
Lapidar; suppliciar com pedradas.
Correr á pedra.
Offender; insultar.
\section{Apegação}
\begin{itemize}
\item {Grp. gram.:f.}
\end{itemize}
Acto de \textunderscore apegar\textunderscore .
\section{Apegadamente}
\begin{itemize}
\item {Grp. gram.:adv.}
\end{itemize}
Com apêgo.
\section{Apegadas}
\begin{itemize}
\item {Grp. gram.:f. pl.}
\end{itemize}
\begin{itemize}
\item {Utilização:Prov.}
\end{itemize}
\begin{itemize}
\item {Utilização:dur.}
\end{itemize}
Tablado, em que os barqueiros do Doiro se collocam para mover o leme.
\section{Apegadiço}
\begin{itemize}
\item {Grp. gram.:adj.}
\end{itemize}
Que se apega facilmente; viscoso.
\section{Apegador}
\begin{itemize}
\item {Grp. gram.:m.}
\end{itemize}
Aquelle que apega.
\section{Apegamento}
\begin{itemize}
\item {Grp. gram.:m.}
\end{itemize}
Acto de \textunderscore apegar\textunderscore .
\section{Apegar}
\begin{itemize}
\item {Grp. gram.:v. t.}
\end{itemize}
\begin{itemize}
\item {Proveniência:(De \textunderscore pegar\textunderscore )}
\end{itemize}
Ajuntar.
Affeiçoar.
Communicar.
Pôr sob o patrocínio de alguém.
\section{Apègar}
\begin{itemize}
\item {Grp. gram.:v. t.}
\end{itemize}
\begin{itemize}
\item {Utilização:T. de Vouzela}
\end{itemize}
\begin{itemize}
\item {Grp. gram.:V. i.}
\end{itemize}
Meter no pégo.
Tomar pé. Cf. Camillo, \textunderscore Sc. da Hora Final\textunderscore , 59.
Regar (o milho), empoçando com os pés a água em volta do tronco; apoceirar;
Meter-se no pégo.
Formar pégo. Cf. Camillo, \textunderscore Brasileira\textunderscore , 63.
\section{Apêgo}
\begin{itemize}
\item {Grp. gram.:m.}
\end{itemize}
\begin{itemize}
\item {Proveniência:(De \textunderscore apegar\textunderscore )}
\end{itemize}
Temão de charrua.
Affeição.
Afêrro; insistência.
\section{Apeguilhar}
\begin{itemize}
\item {Grp. gram.:v. i.}
\end{itemize}
\begin{itemize}
\item {Utilização:Prov.}
\end{itemize}
Comer apeguilho com pão.
Comer moderadamente.
\section{Apeguilho}
\begin{itemize}
\item {Grp. gram.:m.}
\end{itemize}
\begin{itemize}
\item {Utilização:Prov.}
\end{itemize}
\begin{itemize}
\item {Utilização:beir.}
\end{itemize}
Carne de porco cozida.
\section{Apeiragem}
\begin{itemize}
\item {Grp. gram.:f.}
\end{itemize}
Acto de \textunderscore apeirar\textunderscore .
Conjunto das correias necessárias ao carro ou á charrua.
Trem de lavoira.
Utensílios de casa ou de uma officina; petrechos.
\section{Afacia}
\begin{itemize}
\item {Grp. gram.:f.}
\end{itemize}
\begin{itemize}
\item {Proveniência:(Do gr. \textunderscore a\textunderscore  priv. + \textunderscore phakos\textunderscore )}
\end{itemize}
Defeito ocular, que consiste na falta do crystallino.
\section{Afananto}
\begin{itemize}
\item {Grp. gram.:m.}
\end{itemize}
Gênero de plantas ulmáceas.
\section{Afanésio}
\begin{itemize}
\item {Grp. gram.:m.}
\end{itemize}
Arseniato cúprico, hidratado, solúvel nos ácidos e no ammoníaco.
\section{Afanípteros}
\begin{itemize}
\item {Grp. gram.:m. pl.}
\end{itemize}
\begin{itemize}
\item {Proveniência:(Do gr. \textunderscore aphanes\textunderscore  + \textunderscore pteron\textunderscore )}
\end{itemize}
Ordem de insectos.
\section{Afanita}
\begin{itemize}
\item {Grp. gram.:f.}
\end{itemize}
\begin{itemize}
\item {Proveniência:(Do gr. \textunderscore a\textunderscore  priv. + \textunderscore phanos\textunderscore )}
\end{itemize}
Espécie de rochas amphiboloides.
\section{Afasia}
\begin{itemize}
\item {Grp. gram.:f.}
\end{itemize}
\begin{itemize}
\item {Proveniência:(Do gr. \textunderscore a\textunderscore  priv. + \textunderscore phasis\textunderscore )}
\end{itemize}
Perda total ou parcial da voz.
\section{Afásico}
\begin{itemize}
\item {Grp. gram.:adj.}
\end{itemize}
Que tem afasia.
\section{Afelandra}
\begin{itemize}
\item {Grp. gram.:f.}
\end{itemize}
\begin{itemize}
\item {Proveniência:(Do gr. \textunderscore aphiles\textunderscore  + \textunderscore aner\textunderscore , \textunderscore andros\textunderscore )}
\end{itemize}
Gênero de plantas solâneas.
\section{Afélio}
\begin{itemize}
\item {Grp. gram.:m.}
\end{itemize}
\begin{itemize}
\item {Proveniência:(Do gr. \textunderscore apo\textunderscore  + \textunderscore helios\textunderscore )}
\end{itemize}
O ponto mais afastado, em que um planeta se encontra, em relação ao Sol.
\section{Afemia}
\begin{itemize}
\item {Grp. gram.:f.}
\end{itemize}
O mesmo que \textunderscore afasia\textunderscore .
\section{Aférese}
\begin{itemize}
\item {Grp. gram.:f.}
\end{itemize}
\begin{itemize}
\item {Utilização:Gram.}
\end{itemize}
\begin{itemize}
\item {Proveniência:(Do gr. \textunderscore aphairesis\textunderscore )}
\end{itemize}
Suppressão de sýllaba ou letra no princípio de palavra: \textunderscore batina\textunderscore , por \textunderscore abbatina\textunderscore .
\section{Afidífagos}
\begin{itemize}
\item {Grp. gram.:m. pl.}
\end{itemize}
\begin{itemize}
\item {Proveniência:(Do gr. \textunderscore aphis\textunderscore  + \textunderscore phagein\textunderscore )}
\end{itemize}
Insectos coleópteros da secção dos trímeros.
\section{Afídios}
\begin{itemize}
\item {Grp. gram.:m. pl.}
\end{itemize}
\begin{itemize}
\item {Proveniência:(Do gr. \textunderscore aphis\textunderscore  + \textunderscore eidos\textunderscore )}
\end{itemize}
Nome scientifico dos pulgões, que vivem nos vegetaes e os damnificam.
\section{Afilo}
\begin{itemize}
\item {Grp. gram.:adj.}
\end{itemize}
\begin{itemize}
\item {Proveniência:(Do gr. \textunderscore a\textunderscore  priv. + \textunderscore phullon\textunderscore )}
\end{itemize}
Que não tem fôlhas.
\section{Aflogístico}
\begin{itemize}
\item {Grp. gram.:adj.}
\end{itemize}
\begin{itemize}
\item {Proveniência:(De \textunderscore phlogístico\textunderscore )}
\end{itemize}
Que arde sem chamma.
\section{Afonia}
\begin{itemize}
\item {Grp. gram.:f.}
\end{itemize}
O mesmo que \textunderscore afasia\textunderscore .
\section{Afónico}
\begin{itemize}
\item {Grp. gram.:adj.}
\end{itemize}
Que tem afonia.
\section{Aforia}
\begin{itemize}
\item {Grp. gram.:f.}
\end{itemize}
\begin{itemize}
\item {Utilização:Med.}
\end{itemize}
Esterilidade.
\section{Aforismático}
\begin{itemize}
\item {Grp. gram.:adj.}
\end{itemize}
Relativo a \textunderscore aforismo\textunderscore .
\section{Aforismo}
\begin{itemize}
\item {Grp. gram.:m.}
\end{itemize}
\begin{itemize}
\item {Proveniência:(Gr. \textunderscore aphorismos\textunderscore )}
\end{itemize}
Máxima, sentença, que em poucas palavras encerra princípio de grande alcance.
\section{Aforista}
\begin{itemize}
\item {Grp. gram.:m.}
\end{itemize}
Aquelle que faz ou usa aforismos.
\section{Aforístico}
\begin{itemize}
\item {Grp. gram.:adj.}
\end{itemize}
\begin{itemize}
\item {Proveniência:(Gr. \textunderscore aphoristikos\textunderscore )}
\end{itemize}
Que encerra aforismo.
\section{Afracto}
\begin{itemize}
\item {Grp. gram.:m.}
\end{itemize}
\begin{itemize}
\item {Proveniência:(Gr. \textunderscore aphraktos\textunderscore )}
\end{itemize}
Navio longo, sem coberta, entre os antigos.
\section{Afrodíseas}
\begin{itemize}
\item {Grp. gram.:f. pl.}
\end{itemize}
Antigas festas gregas, em honra de Vênus.
\section{Afrodisia}
\begin{itemize}
\item {Grp. gram.:f.}
\end{itemize}
\begin{itemize}
\item {Proveniência:(Gr. \textunderscore aphrodisia\textunderscore )}
\end{itemize}
Aptidão para a geração.
\section{Afrodisíaco}
\begin{itemize}
\item {Grp. gram.:adj.}
\end{itemize}
\begin{itemize}
\item {Proveniência:(De \textunderscore aphrodisia\textunderscore )}
\end{itemize}
Que restaura as fôrças geradoras.
\section{Afrodisiasmo}
\begin{itemize}
\item {Grp. gram.:m.}
\end{itemize}
Cópula carnal.
(Cp. \textunderscore aphrodisia\textunderscore )
\section{Afrodisiografia}
\begin{itemize}
\item {Grp. gram.:f.}
\end{itemize}
O mesmo que \textunderscore afroditografia\textunderscore .
\section{Afrodisiográfico}
\begin{itemize}
\item {Grp. gram.:adj.}
\end{itemize}
Relativo á \textunderscore afrodisiografia\textunderscore .
\section{Afrodisiógrafo}
\begin{itemize}
\item {Grp. gram.:m.}
\end{itemize}
Aquelle que se occupa de \textunderscore afrodisiografia\textunderscore .
\section{Afroditas}
\begin{itemize}
\item {Grp. gram.:f. pl.}
\end{itemize}
\begin{itemize}
\item {Proveniência:(Do gr. \textunderscore Aphrodite\textunderscore , n. de Vênus)}
\end{itemize}
Designação, dada por alguns naturalistas ás plantas cryptogâmicas.
\section{Afroditografia}
\begin{itemize}
\item {Grp. gram.:f.}
\end{itemize}
Descripção do planeta Vênus.
\section{Afta}
\begin{itemize}
\item {Grp. gram.:f.}
\end{itemize}
\begin{itemize}
\item {Proveniência:(Do gr. \textunderscore aphtai\textunderscore )}
\end{itemize}
Pequena ulceração na mucosa, principalmente dentro da bôca.
\section{Aftoso}
\begin{itemize}
\item {Grp. gram.:adj.}
\end{itemize}
Relativo a aftas.
Que tem aftas.
\section{Apeirar}
\begin{itemize}
\item {Grp. gram.:v. t.}
\end{itemize}
\begin{itemize}
\item {Proveniência:(Do lat. \textunderscore apere\textunderscore , ligar)}
\end{itemize}
Jungir ao carro, ao arado, etc.
\section{Apeiria}
\begin{itemize}
\item {Grp. gram.:f.}
\end{itemize}
\begin{itemize}
\item {Utilização:Prov.}
\end{itemize}
\begin{itemize}
\item {Proveniência:(De \textunderscore apeiro\textunderscore )}
\end{itemize}
Porção de utensílios de lavoira; apeiragem.
\section{Apeiro}
\begin{itemize}
\item {Grp. gram.:m.}
\end{itemize}
\begin{itemize}
\item {Proveniência:(De \textunderscore apeirar\textunderscore )}
\end{itemize}
Correia, que prende a canga ao cabeçalho do carro, arado ou charrua, e que também se chama \textunderscore tamoeiro\textunderscore .
Trem de lavoira.
Utensílios.
Trem completo de caça.
\section{Apejar-se}
\begin{itemize}
\item {Grp. gram.:v. p.}
\end{itemize}
O mesmo que [[pejar-se|pejar]].
\section{Á-pela-manhan}
\begin{itemize}
\item {Grp. gram.:adv.}
\end{itemize}
\begin{itemize}
\item {Utilização:Prov.}
\end{itemize}
\begin{itemize}
\item {Utilização:trasm.}
\end{itemize}
O mesmo que \textunderscore amanhan\textunderscore .
\section{Apelho}
\begin{itemize}
\item {fónica:pê}
\end{itemize}
\begin{itemize}
\item {Grp. gram.:m.}
\end{itemize}
\begin{itemize}
\item {Utilização:Prov.}
\end{itemize}
\begin{itemize}
\item {Utilização:alg.}
\end{itemize}
\begin{itemize}
\item {Utilização:alent.}
\end{itemize}
Usa-se na loc. \textunderscore andar aos apelhos\textunderscore , como quem diz \textunderscore andar em desordem, guerrear\textunderscore .
(Por \textunderscore apêgo\textunderscore ?)
\section{Apelintrado}
\begin{itemize}
\item {Grp. gram.:adj.}
\end{itemize}
Que tem modos de pelintra.
\section{Apenar}
\begin{itemize}
\item {Grp. gram.:v. t.}
\end{itemize}
Punir, impor pena a.
Multar.
Intimar em nome da autoridade, para certo serviço. Cf. Arn. Gama, \textunderscore Motim\textunderscore , 528.
Invocar:«\textunderscore apenaria todas do Olympo as divindades a que os raios... me commetam para estoirar a pulga\textunderscore ». Filinto, XIII, 39.
\section{Apenaris}
\begin{itemize}
\item {Grp. gram.:m. pl.}
\end{itemize}
Indígenas brasileiros, que habitaram na região do Pará.
\section{Apenas}
\begin{itemize}
\item {Grp. gram.:adv.}
\end{itemize}
\begin{itemize}
\item {Grp. gram.:Conj.}
\end{itemize}
\begin{itemize}
\item {Proveniência:(De \textunderscore pena\textunderscore )}
\end{itemize}
Difficilmente.
Sómente.
Logo que: \textunderscore apenas amanheceu, levantei-me\textunderscore .
\section{Apendoar}
\begin{itemize}
\item {Grp. gram.:v. t.}
\end{itemize}
\begin{itemize}
\item {Grp. gram.:V. i.}
\end{itemize}
\begin{itemize}
\item {Utilização:Bras}
\end{itemize}
Guarnecer de pendões.
Têr pendão ou bandeira.
Embandeirar-se.
\section{Apenedado}
\begin{itemize}
\item {Grp. gram.:adj.}
\end{itemize}
Que tem muitos penedos.
\section{Apenhado}
\begin{itemize}
\item {Grp. gram.:adj.}
\end{itemize}
Cheio de penhas.
\section{Apenhar}
\begin{itemize}
\item {Grp. gram.:v. t.}
\end{itemize}
\begin{itemize}
\item {Utilização:Ant.}
\end{itemize}
O mesmo que \textunderscore empenhar\textunderscore .
\section{Apenhascado}
\begin{itemize}
\item {Grp. gram.:adj.}
\end{itemize}
Que tem fórma de penhasco.
Coberto de penhascos.
\section{Apenhorar}
\begin{itemize}
\item {Grp. gram.:v. t.}
\end{itemize}
Dar em penhor, empenhar.
\section{Apeninsulado}
\begin{itemize}
\item {Grp. gram.:adj.}
\end{itemize}
Semelhante a uma península.
\section{Apennulado}
\begin{itemize}
\item {Grp. gram.:adj.}
\end{itemize}
Que tem pênnulas.
\section{Apensionado}
\begin{itemize}
\item {Grp. gram.:adj.}
\end{itemize}
\begin{itemize}
\item {Utilização:Pop.}
\end{itemize}
O mesmo que [[pensionado|pensionr]]; muito occupado, com serviços ou negócios.
\section{Apenulado}
\begin{itemize}
\item {Grp. gram.:adj.}
\end{itemize}
Que tem pênulas.
\section{Apenumbrar}
\begin{itemize}
\item {Grp. gram.:v. t.}
\end{itemize}
\begin{itemize}
\item {Utilização:Neol.}
\end{itemize}
Fazer penumbra em. Cf. Val. Magalhães, \textunderscore Vinte Contos\textunderscore .
\section{Apepinação}
\begin{itemize}
\item {Grp. gram.:f.}
\end{itemize}
\begin{itemize}
\item {Utilização:Chul.}
\end{itemize}
Acto de \textunderscore apepinar\textunderscore .
\section{Apepinado}
\begin{itemize}
\item {Grp. gram.:adj.}
\end{itemize}
Que tem fórma ou sabor de pepino.
\section{Apepinador}
\begin{itemize}
\item {Grp. gram.:m.  e  adj.}
\end{itemize}
\begin{itemize}
\item {Utilização:Chul.}
\end{itemize}
O que apepina, que caçôa.
\section{Apepinar}
\begin{itemize}
\item {Grp. gram.:v. t.}
\end{itemize}
\begin{itemize}
\item {Utilização:Chul.}
\end{itemize}
Ridiculizar; escarnecer de.
\section{Apepsia}
\begin{itemize}
\item {Grp. gram.:f.}
\end{itemize}
\begin{itemize}
\item {Proveniência:(Gr. \textunderscore apepsia\textunderscore )}
\end{itemize}
Dificuldade habitual em digerir.
\section{Apequenado}
\begin{itemize}
\item {Grp. gram.:adj.}
\end{itemize}
Um tanto pequeno ou baixo. Cf. Júlio Dinís, \textunderscore Serões\textunderscore , 158.
\section{Apequenar}
\begin{itemize}
\item {Grp. gram.:v. t.}
\end{itemize}
Tornar pequeno. Cf. Camillo, \textunderscore Quéda de Um Anjo\textunderscore , 262.
\section{Aperaltado}
\begin{itemize}
\item {Grp. gram.:adj.}
\end{itemize}
Que tem modos ou traje de peralta.
\section{Aperaltar}
\begin{itemize}
\item {Grp. gram.:v. t.}
\end{itemize}
Tornar peralta, garrido.
Dar modos de peralta a.
\section{Aperalvilhado}
\begin{itemize}
\item {Grp. gram.:adj.}
\end{itemize}
O mesmo que \textunderscore aperaltado\textunderscore .
\section{Aperalvilhar}
\begin{itemize}
\item {Grp. gram.:v. t.}
\end{itemize}
Tornar peralvilho.
Dar modos ou hábitos de peralvilho a.
\section{Apercá}
\begin{itemize}
\item {Grp. gram.:f.}
\end{itemize}
\begin{itemize}
\item {Utilização:Bras}
\end{itemize}
Pequeno mammífero roedor.
\section{Aperceber}
\begin{itemize}
\item {Grp. gram.:v. t.}
\end{itemize}
\begin{itemize}
\item {Proveniência:(De \textunderscore perceber\textunderscore )}
\end{itemize}
Avistar; vêr ao longe.
Distinguir.
Preparar; pôr em ordem.
Munir: \textunderscore aperceber de armamento uma esquadra\textunderscore .
\section{Apercebimento}
\begin{itemize}
\item {Grp. gram.:m.}
\end{itemize}
Acto de \textunderscore aperceber\textunderscore .
\section{Apercepção}
\begin{itemize}
\item {Grp. gram.:f.}
\end{itemize}
\begin{itemize}
\item {Proveniência:(De \textunderscore percepção\textunderscore )}
\end{itemize}
Sentimento, que cada qual tem, da própria consciência.
\section{Aperceptibilidade}
\begin{itemize}
\item {Grp. gram.:f.}
\end{itemize}
Qualidade do que é \textunderscore aperceptível\textunderscore .
\section{Aperceptível}
\begin{itemize}
\item {Grp. gram.:adj.}
\end{itemize}
\begin{itemize}
\item {Proveniência:(De \textunderscore perceptível\textunderscore )}
\end{itemize}
Que se póde aperceber, avistar, distinguir.
\section{Aperema}
\begin{itemize}
\item {Grp. gram.:m.}
\end{itemize}
\begin{itemize}
\item {Utilização:Bras}
\end{itemize}
Espécie de cágado das regiões do Amazonas.
\section{Aperfeiçoadamente}
\begin{itemize}
\item {Grp. gram.:adv.}
\end{itemize}
De modo \textunderscore aperfeiçoado\textunderscore .
Com perfeição.
\section{Aperfeiçoado}
\begin{itemize}
\item {Grp. gram.:adj.}
\end{itemize}
Que se tornou perfeito.
Que se completou.
\section{Aperfeiçoador}
\begin{itemize}
\item {Grp. gram.:m.}
\end{itemize}
\begin{itemize}
\item {Grp. gram.:Adj.}
\end{itemize}
Aquelle que aperfeiçôa.
Que aperfeiçôa.
\section{Aperfeiçoamento}
\begin{itemize}
\item {Grp. gram.:m.}
\end{itemize}
Acto ou effeito de \textunderscore aperfeiçoar\textunderscore .
\section{Aperfeiçoar}
\begin{itemize}
\item {Grp. gram.:v. t.}
\end{itemize}
\begin{itemize}
\item {Proveniência:(De \textunderscore perfeição\textunderscore )}
\end{itemize}
Tornar perfeito: \textunderscore aperfeiçoar um trabalho\textunderscore .
Melhorar; tornar menos imperfeito.
Completar.
\section{Apergaminhado}
\begin{itemize}
\item {Grp. gram.:adj.}
\end{itemize}
Que tem a apparência de pergaminho.
\section{Aperiantáceo}
\begin{itemize}
\item {Grp. gram.:adj.}
\end{itemize}
O mesmo que \textunderscore aperiantado\textunderscore .
\section{Aperiantado}
\begin{itemize}
\item {Grp. gram.:adj.}
\end{itemize}
\begin{itemize}
\item {Utilização:Bot.}
\end{itemize}
\begin{itemize}
\item {Proveniência:(De \textunderscore a\textunderscore  priv. e \textunderscore perianthado\textunderscore )}
\end{itemize}
Diz-se das flôres, que não têm cálice nem corolla que lhes protejam os órgãos sexuaes.
\section{Aperiantháceo}
\begin{itemize}
\item {Grp. gram.:adj.}
\end{itemize}
O mesmo que \textunderscore aperianthado\textunderscore .
\section{Aperianthado}
\begin{itemize}
\item {Grp. gram.:adj.}
\end{itemize}
\begin{itemize}
\item {Utilização:Bot.}
\end{itemize}
\begin{itemize}
\item {Proveniência:(De \textunderscore a\textunderscore  priv. e \textunderscore perianthado\textunderscore )}
\end{itemize}
Diz-se das flôres, que não têm cálice nem corolla que lhes protejam os órgãos sexuaes.
\section{Apericarpo}
\begin{itemize}
\item {Grp. gram.:adj.}
\end{itemize}
Que não tem pericarpo.
\section{Aperiente}
\begin{itemize}
\item {Grp. gram.:adj.}
\end{itemize}
(V.aperitivo)
\section{Aperispérmico}
\begin{itemize}
\item {Grp. gram.:adj.}
\end{itemize}
Que não tem perisperma.
\section{Aperitivo}
\begin{itemize}
\item {Grp. gram.:m.}
\end{itemize}
\begin{itemize}
\item {Grp. gram.:Adj.}
\end{itemize}
\begin{itemize}
\item {Proveniência:(Do lat. \textunderscore aperire\textunderscore )}
\end{itemize}
Aquillo que abre o appetite.
Que abre os póros.
Que desperta appetite.
\section{Aperitório}
\begin{itemize}
\item {Grp. gram.:m.}
\end{itemize}
Lâmina, com que os fabricantes de alfinetes igualam os arames.
\section{Apernar}
\begin{itemize}
\item {Grp. gram.:v. t.}
\end{itemize}
Prender pelas pernas.
Obrigar.
Fazer tomar compromisso.
\section{Apero}
\begin{itemize}
\item {Grp. gram.:m.}
\end{itemize}
\begin{itemize}
\item {Utilização:Bras}
\end{itemize}
O mesmo que \textunderscore apeiro\textunderscore .
\section{Aperolado}
\begin{itemize}
\item {Grp. gram.:adj.}
\end{itemize}
Semelhante a pérola.
\section{Aperolar}
\begin{itemize}
\item {Grp. gram.:v. t.}
\end{itemize}
Tornar semelhante a pérolas.
\section{Aperrar}
\begin{itemize}
\item {Grp. gram.:v. t.}
\end{itemize}
\begin{itemize}
\item {Proveniência:(De \textunderscore perro\textunderscore )}
\end{itemize}
Engatilhar; levantar (o cão da espingarda).
\section{Aperreação}
\begin{itemize}
\item {Grp. gram.:f.}
\end{itemize}
Acto de \textunderscore aperrear\textunderscore .
\section{Aperreador}
\begin{itemize}
\item {Grp. gram.:m.}
\end{itemize}
Aquelle que aperreia.
\section{Aperreamento}
\begin{itemize}
\item {Grp. gram.:m.}
\end{itemize}
O mesmo que \textunderscore aperreação\textunderscore .
\section{Aperrear}
\begin{itemize}
\item {Grp. gram.:v. t.}
\end{itemize}
\begin{itemize}
\item {Proveniência:(De \textunderscore perro\textunderscore )}
\end{itemize}
Fazer perseguir por cães.
Atormentar.
Opprimir.
\section{Aperronhado}
\begin{itemize}
\item {Grp. gram.:adj.}
\end{itemize}
\begin{itemize}
\item {Utilização:Prov.}
\end{itemize}
\begin{itemize}
\item {Utilização:trasm.}
\end{itemize}
O mesmo que [[aperreado|aperrear]], \textunderscore opprimido\textunderscore .
\section{Apertada}
\begin{itemize}
\item {Grp. gram.:f.}
\end{itemize}
Desfiladeiro; o mesmo que \textunderscore apertado\textunderscore :«\textunderscore entraes numa apertada de rochedos\textunderscore ».
Filinto, VII, 185.
\section{Apertadamente}
\begin{itemize}
\item {Grp. gram.:adv.}
\end{itemize}
Com apêrto.
\section{Apertadela}
\begin{itemize}
\item {Grp. gram.:f.}
\end{itemize}
Acto de \textunderscore apertar\textunderscore . Cf. Arn. Gama, \textunderscore Segr. do Ab.\textunderscore , 233.
\section{Apertado}
\begin{itemize}
\item {Grp. gram.:m.}
\end{itemize}
\begin{itemize}
\item {Utilização:Bras}
\end{itemize}
Lugar estreito, desfiladeiro.
Lugar, onde um caminho ou um rio é apertado.
\section{Apertadoiro}
\begin{itemize}
\item {Grp. gram.:m.}
\end{itemize}
\begin{itemize}
\item {Utilização:Ant.}
\end{itemize}
O mesmo que \textunderscore apertador\textunderscore .
\section{Apertador}
\begin{itemize}
\item {Grp. gram.:m.}
\end{itemize}
Aquillo que aperta.
\section{Apertadouro}
\begin{itemize}
\item {Grp. gram.:m.}
\end{itemize}
\begin{itemize}
\item {Utilização:Ant.}
\end{itemize}
O mesmo que \textunderscore apertador\textunderscore .
\section{Aperta-luvas}
\begin{itemize}
\item {Grp. gram.:m.}
\end{itemize}
Espécie de gancho, para abotoar as luvas.
\section{Apertamento}
\begin{itemize}
\item {Grp. gram.:m.}
\end{itemize}
\begin{itemize}
\item {Utilização:Des.}
\end{itemize}
Acto de \textunderscore apertar\textunderscore .
\section{Apertante}
\begin{itemize}
\item {Grp. gram.:f.}
\end{itemize}
\begin{itemize}
\item {Utilização:Gír.}
\end{itemize}
\begin{itemize}
\item {Proveniência:(De \textunderscore apertar\textunderscore )}
\end{itemize}
Corda.
\section{Apertão}
\begin{itemize}
\item {Grp. gram.:m.}
\end{itemize}
Grande apêrto.
Multidão de pessôas, tão juntas, que se apertam.
\section{Apertar}
\begin{itemize}
\item {Grp. gram.:v. t.}
\end{itemize}
\begin{itemize}
\item {Proveniência:(De \textunderscore perto\textunderscore )}
\end{itemize}
Estreitar: \textunderscore apertar a mão\textunderscore .
Unir muito; comprimir.
Apressar: \textunderscore apertar o passo\textunderscore .
Instar com.
Cingir.
Opprimir.
\section{Aperta-ruão}
\begin{itemize}
\item {Grp. gram.:m.}
\end{itemize}
\begin{itemize}
\item {Utilização:Bras}
\end{itemize}
Planta piperácea, medicinal.
\section{Apêrto}
\begin{itemize}
\item {Grp. gram.:m.}
\end{itemize}
Acção de \textunderscore apertar\textunderscore .
Multidão compacta de gente.
Pressa.
Avareza.
Desgraça: \textunderscore ninguém se veja em tal apêrto\textunderscore .
\section{Apertucho}
\begin{itemize}
\item {Grp. gram.:m.}
\end{itemize}
\begin{itemize}
\item {Utilização:Bras}
\end{itemize}
O mesmo que \textunderscore apêrto\textunderscore .
\section{Apertura}
\begin{itemize}
\item {Grp. gram.:f.}
\end{itemize}
O mesmo que \textunderscore apêrto\textunderscore . Cf. Camillo, \textunderscore Caveira\textunderscore , 242.
\section{Apesar}
\begin{itemize}
\item {Proveniência:(De \textunderscore pesar\textunderscore )}
\end{itemize}
\textunderscore adv.\textunderscore  (seguido da prep. \textunderscore de\textunderscore )
Não obstante.
\section{Apesarar}
\begin{itemize}
\item {Grp. gram.:v. t.}
\end{itemize}
\begin{itemize}
\item {Proveniência:(De \textunderscore pesar\textunderscore )}
\end{itemize}
Tornar pesaroso.
\section{Apesentar}
\begin{itemize}
\item {Grp. gram.:v. t.}
\end{itemize}
\begin{itemize}
\item {Proveniência:(De \textunderscore péso\textunderscore )}
\end{itemize}
Tornar pesado.
\section{Apesinhar}
\begin{itemize}
\item {Grp. gram.:v. t.}
\end{itemize}
O mesmo que \textunderscore espezinhar\textunderscore .
\section{Apespontar}
\begin{itemize}
\item {Grp. gram.:v. t.}
\end{itemize}
O mesmo que \textunderscore pespontar\textunderscore . Cf. Arn. Gama, \textunderscore Últ. Dona\textunderscore , 9 e 262.
\section{Apessoado}
\begin{itemize}
\item {Grp. gram.:adj.}
\end{itemize}
\begin{itemize}
\item {Proveniência:(De \textunderscore pessôa\textunderscore )}
\end{itemize}
(precedido de \textunderscore bem\textunderscore )
Que tem bôa estatura, bôa figura, galhardia, gentileza.
\section{Apestanado}
\begin{itemize}
\item {Grp. gram.:adj.}
\end{itemize}
Que tem pestanas.
Que é de pestanas fartas.
\section{Apestar}
\begin{itemize}
\item {Grp. gram.:v. t.}
\end{itemize}
(V.empestar)
\section{Apesunhado}
\begin{itemize}
\item {Grp. gram.:adj.}
\end{itemize}
Que tem unhas grossas e é mal feito das pernas, (especialmente falando-se de bois e de outros animaes).
\section{Apesunhar}
\begin{itemize}
\item {Grp. gram.:v. i.}
\end{itemize}
\begin{itemize}
\item {Proveniência:(De \textunderscore pesunho\textunderscore )}
\end{itemize}
Têr cambada e grossa a parte inferior das pernas, (o porco e outros animaes).
\section{Apetáleas}
\begin{itemize}
\item {Grp. gram.:f. pl.}
\end{itemize}
\begin{itemize}
\item {Proveniência:(De \textunderscore pétala\textunderscore )}
\end{itemize}
O grupo de plantas dicotyledóneas, que não têm cálice nem corolla.
\section{Apetalifloro}
\begin{itemize}
\item {Grp. gram.:adj.}
\end{itemize}
\begin{itemize}
\item {Proveniência:(De \textunderscore apétalo\textunderscore  + \textunderscore flôr\textunderscore )}
\end{itemize}
Que tem flôres sem corolla.
\section{Apétalo}
\begin{itemize}
\item {Grp. gram.:adj.}
\end{itemize}
\begin{itemize}
\item {Proveniência:(Do gr. \textunderscore a\textunderscore  priv. + \textunderscore petale\textunderscore )}
\end{itemize}
Que não têm pétalas.
Que não tem periantho.
\section{Apetrechar}
\begin{itemize}
\item {Grp. gram.:v. t.}
\end{itemize}
Munir de petrechos.
\section{Apetrecho}
\begin{itemize}
\item {Grp. gram.:m.}
\end{itemize}
O mesmo que [[petrecho|petrechos]].
\section{Aphacia}
\begin{itemize}
\item {Grp. gram.:f.}
\end{itemize}
\begin{itemize}
\item {Proveniência:(Do gr. \textunderscore a\textunderscore  priv. + \textunderscore phakos\textunderscore )}
\end{itemize}
Defeito ocular, que consiste na falta do crystallino.
\section{Aphananto}
\begin{itemize}
\item {Grp. gram.:m.}
\end{itemize}
Gênero de plantas ulmáceas.
\section{Aphanésio}
\begin{itemize}
\item {Grp. gram.:m.}
\end{itemize}
Arseniato cúprico, hidratado, solúvel nos ácidos e no ammoníaco.
\section{Aphanípteros}
\begin{itemize}
\item {Grp. gram.:m. pl.}
\end{itemize}
\begin{itemize}
\item {Proveniência:(Do gr. \textunderscore aphanes\textunderscore  + \textunderscore pteron\textunderscore )}
\end{itemize}
Ordem de insectos.
\section{Aphanita}
\begin{itemize}
\item {Grp. gram.:f.}
\end{itemize}
\begin{itemize}
\item {Proveniência:(Do gr. \textunderscore a\textunderscore  priv. + \textunderscore phanos\textunderscore )}
\end{itemize}
Espécie de rochas amphiboloides.
\section{Aphanite}
\begin{itemize}
\item {Grp. gram.:f.}
\end{itemize}
(V.aphanita)
\section{Aphasia}
\begin{itemize}
\item {Grp. gram.:f.}
\end{itemize}
\begin{itemize}
\item {Proveniência:(Do gr. \textunderscore a\textunderscore  priv. + \textunderscore phasis\textunderscore )}
\end{itemize}
Perda total ou parcial da voz.
\section{Aphásico}
\begin{itemize}
\item {Grp. gram.:adj.}
\end{itemize}
Que tem aphasia.
\section{Aphelandra}
\begin{itemize}
\item {Grp. gram.:f.}
\end{itemize}
\begin{itemize}
\item {Proveniência:(Do gr. \textunderscore aphiles\textunderscore  + \textunderscore aner\textunderscore , \textunderscore andros\textunderscore )}
\end{itemize}
Gênero de plantas solâneas.
\section{Aphélio}
\begin{itemize}
\item {Grp. gram.:m.}
\end{itemize}
\begin{itemize}
\item {Proveniência:(Do gr. \textunderscore apo\textunderscore  + \textunderscore helios\textunderscore )}
\end{itemize}
O ponto mais afastado, em que um planeta se encontra, em relação ao Sol.
\section{Aphemia}
\begin{itemize}
\item {Grp. gram.:f.}
\end{itemize}
O mesmo que \textunderscore aphasia\textunderscore .
\section{Aphérese}
\begin{itemize}
\item {Grp. gram.:f.}
\end{itemize}
\begin{itemize}
\item {Utilização:Gram.}
\end{itemize}
\begin{itemize}
\item {Proveniência:(Do gr. \textunderscore aphairesis\textunderscore )}
\end{itemize}
Suppressão de sýllaba ou letra no princípio de palavra: \textunderscore batina\textunderscore , por \textunderscore abbatina\textunderscore .
\section{Aphídios}
\begin{itemize}
\item {Grp. gram.:m. pl.}
\end{itemize}
\begin{itemize}
\item {Proveniência:(Do gr. \textunderscore aphis\textunderscore  + \textunderscore eidos\textunderscore )}
\end{itemize}
Nome scientifico dos pulgões, que vivem nos vegetaes e os damnificam.
\section{Aphidíphagos}
\begin{itemize}
\item {Grp. gram.:m. pl.}
\end{itemize}
\begin{itemize}
\item {Proveniência:(Do gr. \textunderscore aphis\textunderscore  + \textunderscore phagein\textunderscore )}
\end{itemize}
Insectos coleópteros da secção dos trímeros.
\section{Aphleugmar}
\begin{itemize}
\item {Grp. gram.:v. t.}
\end{itemize}
O mesmo que \textunderscore afleumar\textunderscore .
\section{Aphlogístico}
\begin{itemize}
\item {Grp. gram.:adj.}
\end{itemize}
\begin{itemize}
\item {Proveniência:(De \textunderscore phlogístico\textunderscore )}
\end{itemize}
Que arde sem chamma.
\section{Aphonia}
\begin{itemize}
\item {Grp. gram.:f.}
\end{itemize}
O mesmo que \textunderscore aphasia\textunderscore .
\section{Aphónico}
\begin{itemize}
\item {Grp. gram.:adj.}
\end{itemize}
Que tem aphonia.
\section{Aphono}
\begin{itemize}
\item {Grp. gram.:adj.}
\end{itemize}
(V.aphónico)
\section{Aphoria}
\begin{itemize}
\item {Grp. gram.:f.}
\end{itemize}
\begin{itemize}
\item {Utilização:Med.}
\end{itemize}
Esterilidade.
\section{Aphorismático}
\begin{itemize}
\item {Grp. gram.:adj.}
\end{itemize}
Relativo a \textunderscore aphorismo\textunderscore .
\section{Aphorismo}
\begin{itemize}
\item {Grp. gram.:m.}
\end{itemize}
\begin{itemize}
\item {Proveniência:(Gr. \textunderscore aphorismos\textunderscore )}
\end{itemize}
Máxima, sentença, que em poucas palavras encerra princípio de grande alcance.
\section{Aphorista}
\begin{itemize}
\item {Grp. gram.:m.}
\end{itemize}
Aquelle que faz ou usa aphorismos.
\section{Aphorístico}
\begin{itemize}
\item {Grp. gram.:adj.}
\end{itemize}
\begin{itemize}
\item {Proveniência:(Gr. \textunderscore aphoristikos\textunderscore )}
\end{itemize}
Que encerra aphorismo.
\section{Aphracto}
\begin{itemize}
\item {Grp. gram.:m.}
\end{itemize}
\begin{itemize}
\item {Proveniência:(Gr. \textunderscore aphraktos\textunderscore )}
\end{itemize}
Navio longo, sem coberta, entre os antigos.
\section{Aphrodíseas}
\begin{itemize}
\item {Grp. gram.:f. pl.}
\end{itemize}
Antigas festas gregas, em honra de Vênus.
\section{Aphrodisia}
\begin{itemize}
\item {Grp. gram.:f.}
\end{itemize}
\begin{itemize}
\item {Proveniência:(Gr. \textunderscore aphrodisia\textunderscore )}
\end{itemize}
Aptidão para a geração.
\section{Aphrodisíaco}
\begin{itemize}
\item {Grp. gram.:adj.}
\end{itemize}
\begin{itemize}
\item {Proveniência:(De \textunderscore aphrodisia\textunderscore )}
\end{itemize}
Que restaura as fôrças geradoras.
\section{Aphrodisiasmo}
\begin{itemize}
\item {Grp. gram.:m.}
\end{itemize}
Cópula carnal.
(Cp. \textunderscore aphrodisia\textunderscore )
\section{Aphrodisiographia}
\begin{itemize}
\item {Grp. gram.:f.}
\end{itemize}
O mesmo que \textunderscore aphroditographia\textunderscore .
\section{Aphrodisiográphico}
\begin{itemize}
\item {Grp. gram.:adj.}
\end{itemize}
Relativo á \textunderscore aphrodisiographia\textunderscore .
\section{Aphrodisiógrapho}
\begin{itemize}
\item {Grp. gram.:m.}
\end{itemize}
Aquelle que se occupa de \textunderscore aphrodisiographia\textunderscore .
\section{Aphroditas}
\begin{itemize}
\item {Grp. gram.:f. pl.}
\end{itemize}
\begin{itemize}
\item {Proveniência:(Do gr. \textunderscore Aphrodite\textunderscore , n. de Vênus)}
\end{itemize}
Designação, dada por alguns naturalistas ás plantas cryptogâmicas.
\section{Aphroditographia}
\begin{itemize}
\item {Grp. gram.:f.}
\end{itemize}
Descripção do planeta Vênus.
\section{Aphta}
\begin{itemize}
\item {Grp. gram.:f.}
\end{itemize}
\begin{itemize}
\item {Proveniência:(Do gr. \textunderscore aphtai\textunderscore )}
\end{itemize}
Pequena ulceração na mucosa, principalmente dentro da bôca.
\section{Aphtoso}
\begin{itemize}
\item {Grp. gram.:adj.}
\end{itemize}
Relativo a aphtas.
Que tem aphtas.
\section{Aphyllo}
\begin{itemize}
\item {Grp. gram.:adj.}
\end{itemize}
\begin{itemize}
\item {Proveniência:(Do gr. \textunderscore a\textunderscore  priv. + \textunderscore phullon\textunderscore )}
\end{itemize}
Que não tem fôlhas.
\section{Apiacás}
\begin{itemize}
\item {Grp. gram.:m. pl.}
\end{itemize}
Aborígenes sertanejos de Mato-Grosso.
\section{Apiadar-se}
\begin{itemize}
\item {Grp. gram.:v. p.}
\end{itemize}
O mesmo que \textunderscore apiedar-se\textunderscore . Cf. G. Vicente, III, 264.
\section{Apiahá}
\begin{itemize}
\item {Grp. gram.:m.}
\end{itemize}
\begin{itemize}
\item {Utilização:Ant.}
\end{itemize}
Espécie de estribilho.
\section{Apião}
\begin{itemize}
\item {Grp. gram.:m.}
\end{itemize}
Gênero de coleópteros.
\section{Apiário}
\begin{itemize}
\item {Grp. gram.:adj.}
\end{itemize}
\begin{itemize}
\item {Grp. gram.:M. pl.}
\end{itemize}
\begin{itemize}
\item {Proveniência:(Do lat. \textunderscore apis\textunderscore )}
\end{itemize}
Relativo ás abelhas.
Tribo de insectos, a que pertence a abelha.
\section{Apicaçar}
\textunderscore v. t.\textunderscore  (e der.)
O mesmo que \textunderscore espicaçar\textunderscore , etc.
\section{Apicado}
\begin{itemize}
\item {Grp. gram.:adj.}
\end{itemize}
Que termina superiormente, em fórma estreita; que tem ápice:«\textunderscore na apicada penedia\textunderscore ». Serpa, \textunderscore Solaus\textunderscore , 157.
\section{Apical}
\begin{itemize}
\item {Grp. gram.:adj.}
\end{itemize}
\begin{itemize}
\item {Utilização:Gram.}
\end{itemize}
\begin{itemize}
\item {Proveniência:(De \textunderscore ápice\textunderscore )}
\end{itemize}
Diz-se das letras consoantes, que se pronunciam tocando com o ápice da língua nas gengivas dos incisivos superiores, (\textunderscore l, t, d, n...\textunderscore )
\section{Ápice}
\begin{itemize}
\item {Grp. gram.:m.}
\end{itemize}
\begin{itemize}
\item {Proveniência:(Lat. \textunderscore apex\textunderscore , \textunderscore apicis\textunderscore )}
\end{itemize}
A parte mais elevada de uma coisa.
Cume; vértice.
O mais alto grau.
Cada um dos pontos que se põem sobre uma vogal, para que esta não faça ditongo com outra; trema; cimalha.
\section{Apichelar}
\begin{itemize}
\item {Grp. gram.:v. t.}
\end{itemize}
Dar fórma de pichel a.
\section{Apicholado}
\begin{itemize}
\item {Grp. gram.:adj.}
\end{itemize}
\begin{itemize}
\item {Utilização:Ant.}
\end{itemize}
Embreado, alcatroado.
(Por \textunderscore apixoado\textunderscore , de \textunderscore pixe\textunderscore )
\section{Apiciadura}
\begin{itemize}
\item {Grp. gram.:f.}
\end{itemize}
\begin{itemize}
\item {Proveniência:(De \textunderscore ápice\textunderscore )}
\end{itemize}
Ponto, em que se ligam dois volantes, nos trabalhos de armador.
\section{Apicida}
\begin{itemize}
\item {Grp. gram.:adj.}
\end{itemize}
\begin{itemize}
\item {Proveniência:(Do lat. \textunderscore apis\textunderscore  + \textunderscore caedere\textunderscore )}
\end{itemize}
Que produz a morte das abelhas. Cf. Castilho, \textunderscore Fastos\textunderscore , II, 657.
\section{Apicifloro}
\begin{itemize}
\item {Grp. gram.:adj.}
\end{itemize}
\begin{itemize}
\item {Utilização:Bot.}
\end{itemize}
\begin{itemize}
\item {Proveniência:(Do lat. \textunderscore apex\textunderscore  + \textunderscore flos\textunderscore )}
\end{itemize}
Que tem flôres terminaes.
\section{Apiciforme}
\begin{itemize}
\item {Grp. gram.:adj.}
\end{itemize}
\begin{itemize}
\item {Utilização:Miner.}
\end{itemize}
\begin{itemize}
\item {Proveniência:(Do lat. \textunderscore apex\textunderscore  + \textunderscore forma\textunderscore )}
\end{itemize}
Que tem fórma de ápice.
Que tem fórma de agulhas reunidas em tufo, (falando-se de crystaes).
\section{Apicilar}
\begin{itemize}
\item {Grp. gram.:adj.}
\end{itemize}
Que está no ápice.
\section{Apicoado}
\begin{itemize}
\item {Grp. gram.:adj.}
\end{itemize}
\begin{itemize}
\item {Utilização:Bras}
\end{itemize}
Talhado a pique.
Empinado: \textunderscore escada apicoada\textunderscore .
\section{Apicoado}
\begin{itemize}
\item {Grp. gram.:adj.}
\end{itemize}
Desbastado toscamente a picão.
\section{Apicoar}
\begin{itemize}
\item {Grp. gram.:v. t.}
\end{itemize}
Desbastar a picão.
\section{Apicu}
\begin{itemize}
\item {Grp. gram.:m.}
\end{itemize}
O mesmo que \textunderscore apicum\textunderscore .
\section{Apícula}
\begin{itemize}
\item {Grp. gram.:f.}
\end{itemize}
(V.apículo)
\section{Apiculado}
\begin{itemize}
\item {Grp. gram.:adj.}
\end{itemize}
Que termina em apículo.
\section{Apículo}
\begin{itemize}
\item {Grp. gram.:m.}
\end{itemize}
\begin{itemize}
\item {Proveniência:(Lat. \textunderscore apiculum\textunderscore )}
\end{itemize}
Ponta curta e pouco consistente.
\section{Apicultor}
\begin{itemize}
\item {Grp. gram.:m.}
\end{itemize}
\begin{itemize}
\item {Proveniência:(Do lat. \textunderscore apis\textunderscore  + \textunderscore cultor\textunderscore )}
\end{itemize}
Criador de abelhas.
\section{Apicultura}
\begin{itemize}
\item {Grp. gram.:f.}
\end{itemize}
\begin{itemize}
\item {Proveniência:(Do lat. \textunderscore apis\textunderscore  + \textunderscore cultura\textunderscore )}
\end{itemize}
Criação de abelhas; arte de as criar.
\section{Apicum}
\begin{itemize}
\item {Grp. gram.:m.}
\end{itemize}
\begin{itemize}
\item {Utilização:Bras}
\end{itemize}
Terreno alagadiço, formado á beira-mar pelos resíduos das enchentes.
\section{Apiedador}
\begin{itemize}
\item {Grp. gram.:m.}
\end{itemize}
\begin{itemize}
\item {Grp. gram.:Adj.}
\end{itemize}
\begin{itemize}
\item {Proveniência:(De \textunderscore apiedar\textunderscore )}
\end{itemize}
Aquelle que se apieda.
Que se condói.
\section{Apiedar-se}
\begin{itemize}
\item {Grp. gram.:v. p.}
\end{itemize}
\begin{itemize}
\item {Proveniência:(De \textunderscore piedade\textunderscore )}
\end{itemize}
Têr compaixão, piedade.
\section{Apiforme}
\begin{itemize}
\item {Grp. gram.:adj.}
\end{itemize}
\begin{itemize}
\item {Proveniência:(Do lat. \textunderscore apis\textunderscore  + \textunderscore forma\textunderscore )}
\end{itemize}
Que tem fórma de abelha.
\section{Apilarar}
\begin{itemize}
\item {Grp. gram.:v. t.}
\end{itemize}
Ornar com pilares.
Escorar com pilares.
\section{Apilhar}
\begin{itemize}
\item {Grp. gram.:v. t.}
\end{itemize}
\begin{itemize}
\item {Utilização:Prov.}
\end{itemize}
\begin{itemize}
\item {Utilização:alg.}
\end{itemize}
O mesmo que \textunderscore pilhar\textunderscore .
\section{Apiloar}
\begin{itemize}
\item {Grp. gram.:v. t.}
\end{itemize}
Bater com o pilão em.
\section{Apimentado}
\begin{itemize}
\item {Grp. gram.:adj.}
\end{itemize}
\begin{itemize}
\item {Utilização:Fig.}
\end{itemize}
Temperado com pimenta.
Picante.
Malicioso: \textunderscore cançoneta apimentada\textunderscore .
\section{Apimentar}
\begin{itemize}
\item {Grp. gram.:v. t.}
\end{itemize}
\begin{itemize}
\item {Utilização:Fig.}
\end{itemize}
Temperar com pimenta.
Estimular.
Tornar picante, azêdo.
Tornar malicioso: \textunderscore apimentar uma história\textunderscore .
\section{Apimpolhar-se}
\begin{itemize}
\item {Grp. gram.:v. p.}
\end{itemize}
Encher-se de pimpolhos.
\section{Apinagés}
\begin{itemize}
\item {Grp. gram.:m. pl.}
\end{itemize}
Selvagens das margens do Tocantins.
\section{Apinário}
\begin{itemize}
\item {Grp. gram.:m.}
\end{itemize}
\begin{itemize}
\item {Proveniência:(Lat. \textunderscore apinarius\textunderscore )}
\end{itemize}
Comediante, que, entre os Romanos, representava as sáturas.
\section{Apincelar}
\begin{itemize}
\item {Grp. gram.:v. t.}
\end{itemize}
Dar fórma de pincel a.
Dar mão de tinta ou cal em.
\section{Apingentar}
\begin{itemize}
\item {Grp. gram.:v. t.}
\end{itemize}
Dar fórma de pingente a.
Pôr pingentes em.
\section{Apinhar}
\begin{itemize}
\item {Grp. gram.:v. t.}
\end{itemize}
\begin{itemize}
\item {Utilização:Bras}
\end{itemize}
\begin{itemize}
\item {Proveniência:(De \textunderscore pinha\textunderscore )}
\end{itemize}
Ajuntar.
Empilhar; amontoar.
Encher.
Dar fórma de pinha a.
\section{Apinhoar}
\begin{itemize}
\item {Grp. gram.:v. t.}
\end{itemize}
(V.apinhar)
\section{Apinicado}
\begin{itemize}
\item {Grp. gram.:adj.}
\end{itemize}
(V.espinicado)
\section{Apintalhar}
\begin{itemize}
\item {Grp. gram.:v. t.}
\end{itemize}
\begin{itemize}
\item {Utilização:Ext.}
\end{itemize}
Delimitar com pintalhas ou estacas.
Marcar com enxadadas, aquém e além, o limite ou perimetro de (um terreno).
\section{Ápio}
\begin{itemize}
\item {Grp. gram.:m.}
\end{itemize}
\begin{itemize}
\item {Proveniência:(Lat. \textunderscore apium\textunderscore ?)}
\end{itemize}
Gênero de plantas ericáceas.
\section{Apiocrinita}
\begin{itemize}
\item {Grp. gram.:f.}
\end{itemize}
Polypeiro fóssil.
\section{Apiol}
\begin{itemize}
\item {Grp. gram.:m.}
\end{itemize}
Princípio activo da semente da salsa.
\section{Ápios}
\begin{itemize}
\item {Grp. gram.:m.}
\end{itemize}
\begin{itemize}
\item {Proveniência:(Do gr. \textunderscore apion\textunderscore , pera)}
\end{itemize}
Gênero de plantas leguminosas.
\section{Apiósporo}
\begin{itemize}
\item {Grp. gram.:m.}
\end{itemize}
Gênero de cogumelos.
\section{Apipar}
\begin{itemize}
\item {Grp. gram.:v. t.}
\end{itemize}
Dar fórma de pipa a.
\section{Apiquedar}
\begin{itemize}
\item {Grp. gram.:v. t.}
\end{itemize}
\begin{itemize}
\item {Utilização:Des.}
\end{itemize}
Meter a pique, afundar.
\section{Ápis}
\begin{itemize}
\item {Grp. gram.:f.}
\end{itemize}
\begin{itemize}
\item {Proveniência:(Lat. \textunderscore apis\textunderscore )}
\end{itemize}
Pequena constellação austral.
\section{Apisoador}
\begin{itemize}
\item {Grp. gram.:m.}
\end{itemize}
Aquelle que apisôa.
\section{Apisoamento}
\begin{itemize}
\item {Grp. gram.:m.}
\end{itemize}
Acto de \textunderscore apisoar\textunderscore .
\section{Apisoar}
\begin{itemize}
\item {Grp. gram.:v. t.}
\end{itemize}
O mesmo que \textunderscore pisoar\textunderscore .
\section{Apisto}
\begin{itemize}
\item {Grp. gram.:m.}
\end{itemize}
Caldo forte e grosso, feito do suco de carne picada.
(Cast. \textunderscore apisto\textunderscore )
\section{Apitar}
\begin{itemize}
\item {Grp. gram.:v. i.}
\end{itemize}
\begin{itemize}
\item {Utilização:Fam.}
\end{itemize}
Tocar apito.
Soar como apito.
\textunderscore Ficar a apitar\textunderscore , ficar logrado, não conseguir o que se deseja.
\section{Apito}
\begin{itemize}
\item {Grp. gram.:m.}
\end{itemize}
Pequeno instrumento, com que se assobia; silvo.
(Cast. \textunderscore pito\textunderscore )
\section{Apitu}
\begin{itemize}
\item {Grp. gram.:m.}
\end{itemize}
Planta tuberculosa do Brasil.
\section{Apívoro}
\begin{itemize}
\item {Grp. gram.:adj.}
\end{itemize}
\begin{itemize}
\item {Proveniência:(Do lat. \textunderscore apis\textunderscore  + \textunderscore vorare\textunderscore )}
\end{itemize}
Que come abelhas.
\section{Apixolado}
\begin{itemize}
\item {Grp. gram.:adj.}
\end{itemize}
\begin{itemize}
\item {Utilização:Ant.}
\end{itemize}
Embreado, alcatroado.
(Por \textunderscore apixoado\textunderscore , de \textunderscore pixe\textunderscore )
\section{Aplacação}
\begin{itemize}
\item {Grp. gram.:f.}
\end{itemize}
Acto de \textunderscore aplacar\textunderscore .
\section{Aplacador}
\begin{itemize}
\item {Grp. gram.:m.}
\end{itemize}
Aquelle que aplaca.
\section{Aplacar}
\begin{itemize}
\item {Grp. gram.:v. t.}
\end{itemize}
\begin{itemize}
\item {Proveniência:(Do lat. \textunderscore placare\textunderscore )}
\end{itemize}
Tornar plácido; tranquillizar; serenar: \textunderscore aplacar as ondas\textunderscore ; \textunderscore aplacar dores\textunderscore .
Apagar: \textunderscore aplacar um incêndio\textunderscore .
Adquirir a bôa disposição de: \textunderscore conseguiu aplacar a sogra\textunderscore .
\section{Aplacável}
\begin{itemize}
\item {Grp. gram.:adj.}
\end{itemize}
Que póde sêr aplacado.
\section{Aplacentário}
\begin{itemize}
\item {Grp. gram.:adj.}
\end{itemize}
\begin{itemize}
\item {Proveniência:(De \textunderscore a\textunderscore  priv. + \textunderscore placenta\textunderscore )}
\end{itemize}
Diz-se do animal que, depois de gerado, se desloca do corpo materno e vem completar no exterior o seu desenvolvimento.
\section{Aplainado}
\begin{itemize}
\item {Grp. gram.:adj.}
\end{itemize}
\begin{itemize}
\item {Utilização:Fig.}
\end{itemize}
Alisado com plaina.
Liso, plano.
\section{Aplainamento}
\begin{itemize}
\item {Grp. gram.:m.}
\end{itemize}
Acto de \textunderscore aplainar\textunderscore .
\section{Aplainar}
\begin{itemize}
\item {Grp. gram.:v. t.}
\end{itemize}
\begin{itemize}
\item {Proveniência:(De \textunderscore plaina\textunderscore )}
\end{itemize}
Alisar com plaina.
O mesmo que \textunderscore aplanar\textunderscore .
\section{Aplanação}
\begin{itemize}
\item {Grp. gram.:f.}
\end{itemize}
O mesmo que \textunderscore aplanamento\textunderscore .
\section{Aplanador}
\begin{itemize}
\item {Grp. gram.:m.}
\end{itemize}
Aquelle que aplana.
\section{Aplanamento}
\begin{itemize}
\item {Grp. gram.:m.}
\end{itemize}
Acto de \textunderscore aplanar\textunderscore .
\section{Aplanar}
\begin{itemize}
\item {Grp. gram.:v. t.}
\end{itemize}
Tornar plano; nivelar; igualar.
Resolver, desviar: \textunderscore aplanar difficuldades\textunderscore .
Facilitar: \textunderscore aplanar um emprehendimento\textunderscore .
\section{Aplastado}
\begin{itemize}
\item {Grp. gram.:adj.}
\end{itemize}
\begin{itemize}
\item {Utilização:Bras}
\end{itemize}
Cansado, fatigado.
(Cp. \textunderscore emplastro\textunderscore )
\section{Aplastar}
\begin{itemize}
\item {Grp. gram.:v.}
\end{itemize}
\begin{itemize}
\item {Utilização:t. Náut.}
\end{itemize}
Desfraldar (vela) Desferrar (o pano do navio).
\section{Aplástico}
\begin{itemize}
\item {Grp. gram.:adj.}
\end{itemize}
\begin{itemize}
\item {Proveniência:(De \textunderscore a\textunderscore  priv. + \textunderscore plástico\textunderscore )}
\end{itemize}
Que não tem plasticidade.
\section{Aplastrar}
\begin{itemize}
\item {Grp. gram.:v. i.}
\end{itemize}
\begin{itemize}
\item {Utilização:Bras}
\end{itemize}
O mesmo que \textunderscore abombar\textunderscore .
\section{Aplebear-se}
\begin{itemize}
\item {Grp. gram.:v. p.}
\end{itemize}
\begin{itemize}
\item {Proveniência:(De \textunderscore plebe\textunderscore )}
\end{itemize}
Tomar modos de plebeu.
Rebaixar-se.
\section{Aples}
\begin{itemize}
\item {Grp. gram.:prep.}
\end{itemize}
\begin{itemize}
\item {Utilização:Ant.}
\end{itemize}
O mesmo que \textunderscore apres\textunderscore .
\section{Aplestia}
\begin{itemize}
\item {Grp. gram.:f.}
\end{itemize}
\begin{itemize}
\item {Proveniência:(Gr. \textunderscore aplestia\textunderscore )}
\end{itemize}
Fome insaciável.
\section{Áplido}
\begin{itemize}
\item {Grp. gram.:m.}
\end{itemize}
Gênero de cogumelos.
\section{Aplísia}
\begin{itemize}
\item {Grp. gram.:f.}
\end{itemize}
O mesmo ou melhor que \textunderscore aplísio\textunderscore . Cf. Fil. Simões, \textunderscore Beiramar\textunderscore , 271.
\section{Aplísio}
\begin{itemize}
\item {Grp. gram.:m.}
\end{itemize}
\begin{itemize}
\item {Proveniência:(Do gr. \textunderscore aplusia\textunderscore )}
\end{itemize}
Nome scientifico dos molluscos gasterópodes, de corpo semelhante ao dos caracoes.
\section{Aplócero}
\begin{itemize}
\item {Grp. gram.:adj.}
\end{itemize}
\begin{itemize}
\item {Utilização:Zool.}
\end{itemize}
Que tem antennas simples.
\section{Aplocnêmia}
\begin{itemize}
\item {Grp. gram.:f.}
\end{itemize}
Gênero de coleópteros.
\section{Aplódero}
\begin{itemize}
\item {Grp. gram.:m.}
\end{itemize}
Gênero de coleópteros.
\section{Aplodonte}
\begin{itemize}
\item {Grp. gram.:m.}
\end{itemize}
Gênero de musgos.
\section{Aplofilo}
\begin{itemize}
\item {Grp. gram.:m.}
\end{itemize}
Espécie de arruda.
\section{Aplomado}
\begin{itemize}
\item {Grp. gram.:adj.}
\end{itemize}
Diz-se do toiro, que corresponde difficilmente ao cite.
\section{Aplomia}
\begin{itemize}
\item {Grp. gram.:f.}
\end{itemize}
Gênero de dípteros.
\section{Aploperistómeas}
\begin{itemize}
\item {Grp. gram.:f. pl.}
\end{itemize}
\begin{itemize}
\item {Utilização:Bot.}
\end{itemize}
\begin{itemize}
\item {Proveniência:(Do gr. \textunderscore aploos\textunderscore  + \textunderscore peri\textunderscore  + \textunderscore stoma\textunderscore )}
\end{itemize}
Classe da fam. dos musgos.
\section{Aplophyllo}
\begin{itemize}
\item {Grp. gram.:m.}
\end{itemize}
Espécie de arruda.
\section{Aplóscelo}
\begin{itemize}
\item {Grp. gram.:m.}
\end{itemize}
Gênero de coleópteros.
\section{Aplóstomo}
\begin{itemize}
\item {Grp. gram.:adj.}
\end{itemize}
\begin{itemize}
\item {Proveniência:(Do gr. \textunderscore aploos\textunderscore  + \textunderscore stoma\textunderscore )}
\end{itemize}
Que tem abertura ou bôca simples.
\section{Aplotomia}
\begin{itemize}
\item {Grp. gram.:f.}
\end{itemize}
\begin{itemize}
\item {Proveniência:(Do gr. \textunderscore aploos\textunderscore  + \textunderscore tome\textunderscore )}
\end{itemize}
Pequena incisão.
\section{Apluda}
\begin{itemize}
\item {Grp. gram.:f.}
\end{itemize}
Gênero de plantas gramíneas.
\section{Aplumar}
\textunderscore v. t.\textunderscore  (e der.)
O mesmo que \textunderscore aprumar\textunderscore , etc.
\section{Aplustre}
\begin{itemize}
\item {Grp. gram.:m.}
\end{itemize}
\begin{itemize}
\item {Utilização:Ant.}
\end{itemize}
\begin{itemize}
\item {Proveniência:(Lat. \textunderscore aplustre\textunderscore )}
\end{itemize}
Ornato da popa de um navio.
\section{Aplustro}
\begin{itemize}
\item {Grp. gram.:m.}
\end{itemize}
O mesmo que \textunderscore aplustre\textunderscore .
\section{Aplýsia}
\begin{itemize}
\item {Grp. gram.:f.}
\end{itemize}
O mesmo ou melhor que \textunderscore aplýsio\textunderscore . Cf. Fil. Simões, \textunderscore Beiramar\textunderscore , 271.
\section{Aplýsio}
\begin{itemize}
\item {Grp. gram.:m.}
\end{itemize}
\begin{itemize}
\item {Proveniência:(Do gr. \textunderscore aplusia\textunderscore )}
\end{itemize}
Nome scientifico dos molluscos gasterópodes, de corpo semelhante ao dos caracoes.
\section{Apnéa}
\begin{itemize}
\item {Grp. gram.:f.}
\end{itemize}
\begin{itemize}
\item {Proveniência:(Gr. \textunderscore apnoia\textunderscore )}
\end{itemize}
Suspensão temporária ou pausa da respiração em certos casos mórbidos.
\section{Apneia}
\begin{itemize}
\item {Grp. gram.:f.}
\end{itemize}
\begin{itemize}
\item {Proveniência:(Gr. \textunderscore apnoia\textunderscore )}
\end{itemize}
Suspensão temporária ou pausa da respiração em certos casos mórbidos.
\section{Apo}
\begin{itemize}
\item {Grp. gram.:m.}
\end{itemize}
\begin{itemize}
\item {Proveniência:(Gr. \textunderscore apous\textunderscore )}
\end{itemize}
Pequena constellação austral.
Crustáceo branchiópode.
Designação desusada da \textunderscore ave-do-paraíso\textunderscore .
\section{Apo}
\begin{itemize}
\item {Grp. gram.:m.}
\end{itemize}
\begin{itemize}
\item {Utilização:Prov.}
\end{itemize}
Haste horizontal ou levemente inclinada, que é a parte principal da charrua, e á qual se prendem as outras peças dêsse instrumento agrícola.
(Cp. \textunderscore apeiro\textunderscore )
\section{Ápoa}
\begin{itemize}
\item {Grp. gram.:f.}
\end{itemize}
Serpente do Brasil.
\section{Ápoca}
\begin{itemize}
\item {Grp. gram.:f.}
\end{itemize}
\begin{itemize}
\item {Utilização:Jur.}
\end{itemize}
\begin{itemize}
\item {Proveniência:(Lat. \textunderscore apocha\textunderscore )}
\end{itemize}
Designação genérica de qualquer bilhete, em que um devedor confessa ter recebido uma quantia e se obriga a pagá-la. Cf. F. Borges, \textunderscore Dicc. Jur.\textunderscore 
\section{Apocalbase}
\begin{itemize}
\item {Grp. gram.:f.}
\end{itemize}
Resina de uma planta euphorbiácea, com que os Africanos ervam as frechas.
\section{Apocalíptico}
\begin{itemize}
\item {Grp. gram.:adj.}
\end{itemize}
\begin{itemize}
\item {Utilização:Fig.}
\end{itemize}
Relativo ao Apocalipse.
Diffícil de comprehender.
Obscuro; sibyllino.
\section{Apocalýptico}
\begin{itemize}
\item {Grp. gram.:adj.}
\end{itemize}
\begin{itemize}
\item {Utilização:Fig.}
\end{itemize}
Relativo ao Apocalypse.
Diffícil de comprehender.
Obscuro; sibyllino.
\section{Apocapnismo}
\begin{itemize}
\item {Grp. gram.:m.}
\end{itemize}
\begin{itemize}
\item {Utilização:Med.}
\end{itemize}
Fumigação de vapores odoríferos.
\section{Apocarpado}
\begin{itemize}
\item {Grp. gram.:adj.}
\end{itemize}
\begin{itemize}
\item {Utilização:Bot.}
\end{itemize}
Diz-se dos ovários, em que ha apocarpo.
\section{Apocarpo}
\begin{itemize}
\item {Grp. gram.:m.}
\end{itemize}
\begin{itemize}
\item {Utilização:Bot.}
\end{itemize}
\begin{itemize}
\item {Proveniência:(Do gr. \textunderscore apo\textunderscore  + \textunderscore karpos\textunderscore )}
\end{itemize}
Disposição dos ovários de um vegetal, quando separados entre si.
\section{Apocatástase}
\begin{itemize}
\item {Grp. gram.:f.}
\end{itemize}
\begin{itemize}
\item {Utilização:Astron.}
\end{itemize}
\begin{itemize}
\item {Proveniência:(Gr. \textunderscore apokatástasis\textunderscore )}
\end{itemize}
Revolução periódica de um astro.
\section{Apoceirar}
\begin{itemize}
\item {Grp. gram.:v. t.}
\end{itemize}
Fazer poço ou escavação á roda de (uma planta) para a regar.
\section{Apocenose}
\begin{itemize}
\item {Grp. gram.:f.}
\end{itemize}
\begin{itemize}
\item {Utilização:Med.}
\end{itemize}
\begin{itemize}
\item {Proveniência:(Gr. \textunderscore apokenosis\textunderscore )}
\end{itemize}
Evacuação contra a natureza.
Hemorragia simples.
\section{Apochylismo}
\begin{itemize}
\item {fónica:qui}
\end{itemize}
\begin{itemize}
\item {Grp. gram.:m.}
\end{itemize}
O mesmo que \textunderscore arrobe\textunderscore .
\section{Apocina}
\begin{itemize}
\item {Grp. gram.:f.}
\end{itemize}
Princípio activo, extrahido da raiz do apócino.
\section{Apocináceas}
\begin{itemize}
\item {Grp. gram.:f. pl.}
\end{itemize}
O mesmo ou melhor que apocíneas.
\section{Apocíneas}
\begin{itemize}
\item {Grp. gram.:f. pl.}
\end{itemize}
\begin{itemize}
\item {Proveniência:(De \textunderscore apócyno\textunderscore )}
\end{itemize}
Plantas dicotyledóneas, monopétalas, hypogýneas.
\section{Apocinina}
\begin{itemize}
\item {Grp. gram.:f.}
\end{itemize}
Princípio activo do cânhamo do Canadá.
\section{Apócino}
\begin{itemize}
\item {Grp. gram.:m.}
\end{itemize}
\begin{itemize}
\item {Proveniência:(Do gr. \textunderscore apo\textunderscore  + \textunderscore kuon\textunderscore )}
\end{itemize}
Gênero de plantas, que serve de typo ás apocýneas.
\section{Apocopado}
\begin{itemize}
\item {Grp. gram.:adj.}
\end{itemize}
\begin{itemize}
\item {Utilização:Gram.}
\end{itemize}
Diz-se da palavra, em que falta a letra ou sýllaba final.
\section{Apocopar}
\begin{itemize}
\item {Grp. gram.:v.}
\end{itemize}
\begin{itemize}
\item {Utilização:t. Gram.}
\end{itemize}
\begin{itemize}
\item {Proveniência:(De \textunderscore apócope\textunderscore )}
\end{itemize}
Cortar letra ou sýllaba no fim de (palavra).
\section{Apócope}
\begin{itemize}
\item {Grp. gram.:f.}
\end{itemize}
\begin{itemize}
\item {Utilização:Gram.}
\end{itemize}
\begin{itemize}
\item {Proveniência:(Gr. \textunderscore apokope\textunderscore )}
\end{itemize}
Suppressão de uma letra ou sýllaba no fim da palavra: \textunderscore mármor\textunderscore .
\section{Apocópico}
\begin{itemize}
\item {Grp. gram.:adj.}
\end{itemize}
\begin{itemize}
\item {Utilização:Gram.}
\end{itemize}
Relativo a apócope.
Em que há apócope. Cf. J. Ribeiro, \textunderscore Dicc. Gram.\textunderscore , 68.
\section{Apocrenato}
\begin{itemize}
\item {Grp. gram.:m.}
\end{itemize}
\begin{itemize}
\item {Utilização:Chím.}
\end{itemize}
Sal, resultante da combinação do ácido apocrênico com uma base.
\section{Apocrênico}
\begin{itemize}
\item {Grp. gram.:adj.}
\end{itemize}
\begin{itemize}
\item {Utilização:Chím.}
\end{itemize}
Diz-se de um ácido, que se separa dos depósitos côr de ocre das águas ferruginosas.
\section{Apócrifa}
\begin{itemize}
\item {Grp. gram.:f.}
\end{itemize}
Gênero de coleópteros.
\section{Apocrifamente}
\begin{itemize}
\item {Grp. gram.:adv.}
\end{itemize}
Com documentos falsos.
\section{Apócrifo}
\begin{itemize}
\item {Grp. gram.:adj.}
\end{itemize}
\begin{itemize}
\item {Proveniência:(Gr. \textunderscore apokruphos\textunderscore )}
\end{itemize}
Diz-se de obra, cuja authenticidade não está provada.
\section{Apocrisia}
\begin{itemize}
\item {Grp. gram.:f.}
\end{itemize}
\begin{itemize}
\item {Utilização:Med.}
\end{itemize}
\begin{itemize}
\item {Proveniência:(Do gr. \textunderscore apo\textunderscore  + \textunderscore krisis\textunderscore )}
\end{itemize}
Excremento ou evacuação, que apparece com symptomas de crise.
\section{Apocrisiário}
\begin{itemize}
\item {Grp. gram.:m.}
\end{itemize}
\begin{itemize}
\item {Utilização:Ant.}
\end{itemize}
\begin{itemize}
\item {Proveniência:(Lat. \textunderscore apocrisiarius\textunderscore )}
\end{itemize}
Serventuário de igreja.
Procurador de causas ecclesiásticas nos paços dos imperantes.
\section{Apocrístico}
\begin{itemize}
\item {Grp. gram.:adj.}
\end{itemize}
\begin{itemize}
\item {Utilização:Med.}
\end{itemize}
Que expelle os humores.
Adstringente.
(Cp. \textunderscore apocrisia\textunderscore )
\section{Apócrypha}
\begin{itemize}
\item {Grp. gram.:f.}
\end{itemize}
Gênero de coleópteros.
\section{Apocryphamente}
\begin{itemize}
\item {Grp. gram.:adv.}
\end{itemize}
Com documentos falsos.
\section{Apócrypho}
\begin{itemize}
\item {Grp. gram.:adj.}
\end{itemize}
\begin{itemize}
\item {Proveniência:(Gr. \textunderscore apokruphos\textunderscore )}
\end{itemize}
Diz-se de obra, cuja authenticidade não está provada.
\section{Apocyna}
\begin{itemize}
\item {Grp. gram.:f.}
\end{itemize}
Princípio activo, extrahido da raiz do apócyno.
\section{Apocynáceas}
\begin{itemize}
\item {Grp. gram.:f. pl.}
\end{itemize}
O mesmo ou melhor que apocýneas.
\section{Apocýneas}
\begin{itemize}
\item {Grp. gram.:f. pl.}
\end{itemize}
\begin{itemize}
\item {Proveniência:(De \textunderscore apócyno\textunderscore )}
\end{itemize}
Plantas dicotyledóneas, monopétalas, hypogýneas.
\section{Apocynina}
\begin{itemize}
\item {Grp. gram.:f.}
\end{itemize}
Princípio activo do cânhamo do Canadá.
\section{Apócyno}
\begin{itemize}
\item {Grp. gram.:m.}
\end{itemize}
\begin{itemize}
\item {Proveniência:(Do gr. \textunderscore apo\textunderscore  + \textunderscore kuon\textunderscore )}
\end{itemize}
Gênero de plantas, que serve de typo ás apocýneas.
\section{Apodadeira}
\begin{itemize}
\item {Grp. gram.:f.}
\end{itemize}
\begin{itemize}
\item {Proveniência:(De \textunderscore apodar\textunderscore )}
\end{itemize}
Mulher, que dirige apodos, que escarnece ou moteja.
\section{Apodador}
\begin{itemize}
\item {Grp. gram.:m.}
\end{itemize}
Aquelle que apoda.
\section{Apodadura}
\begin{itemize}
\item {Grp. gram.:f.}
\end{itemize}
O mesmo que \textunderscore ápodo\textunderscore .
\section{Apodar}
\begin{itemize}
\item {Grp. gram.:v. t.}
\end{itemize}
\begin{itemize}
\item {Proveniência:(Do lat. \textunderscore putare\textunderscore )}
\end{itemize}
Dirigir apodos a; zombar, escarnecer de.
Comparar; classificar depreciativamente: \textunderscore apodaram-me de ingrato\textunderscore .
\section{Ápode}
\begin{itemize}
\item {Grp. gram.:adj.}
\end{itemize}
\begin{itemize}
\item {Grp. gram.:M.}
\end{itemize}
\begin{itemize}
\item {Proveniência:(Gr. \textunderscore apous\textunderscore )}
\end{itemize}
Que não tem pés.
Espécie de andorinha marítima.
Ordem de peixes.
\section{Apodecto}
\begin{itemize}
\item {Grp. gram.:m.}
\end{itemize}
Magistrado grego, que tinha a seu cargo o serviço das contribuições.
\section{Apodema}
\begin{itemize}
\item {Grp. gram.:m.}
\end{itemize}
Parte superior dos insectos, que adhere ao thorax.
\section{Apodengado}
\begin{itemize}
\item {Grp. gram.:adj.}
\end{itemize}
Semelhante a podengo.
\section{Apoderamento}
\begin{itemize}
\item {Grp. gram.:m.}
\end{itemize}
Acto de \textunderscore apoderar-se\textunderscore .
\section{Apoderar-se}
\begin{itemize}
\item {Grp. gram.:v. p.}
\end{itemize}
\begin{itemize}
\item {Proveniência:(De \textunderscore poder\textunderscore )}
\end{itemize}
Tomar posse; assenhorear-se.
\section{Apódero}
\begin{itemize}
\item {Grp. gram.:m.}
\end{itemize}
\begin{itemize}
\item {Proveniência:(Do gr. \textunderscore apoderein\textunderscore )}
\end{itemize}
Insecto coleóptero tetrâmero.
\section{Apodia}
\begin{itemize}
\item {Grp. gram.:f.}
\end{itemize}
\begin{itemize}
\item {Proveniência:(De \textunderscore ápode\textunderscore )}
\end{itemize}
Falta de pés.
\section{Apodíctico}
\begin{itemize}
\item {Grp. gram.:adj.}
\end{itemize}
\begin{itemize}
\item {Proveniência:(Do gr. \textunderscore apodeiktikos\textunderscore )}
\end{itemize}
Evidente.
\section{Apodioxe}
\begin{itemize}
\item {Grp. gram.:f.}
\end{itemize}
\begin{itemize}
\item {Proveniência:(Gr. \textunderscore apodioxis\textunderscore )}
\end{itemize}
Repulsão de um argumento, como absurdo.
\section{Apoditério}
\begin{itemize}
\item {Grp. gram.:m.}
\end{itemize}
\begin{itemize}
\item {Proveniência:(Lat. \textunderscore apodyterium\textunderscore )}
\end{itemize}
Recinto nas antigas casas de banhos, em que se despiam as roupas e se penduravam em cabides.
\section{Apôdo}
\begin{itemize}
\item {Grp. gram.:m.}
\end{itemize}
\begin{itemize}
\item {Proveniência:(De \textunderscore apodar\textunderscore )}
\end{itemize}
Comparação ridícula.
Motejo; zombaria.
\section{Ápodo}
\begin{itemize}
\item {Grp. gram.:m.  e  adj.}
\end{itemize}
(V.ápode)
\section{Apodógino}
\begin{itemize}
\item {Grp. gram.:m.}
\end{itemize}
\begin{itemize}
\item {Utilização:Bot.}
\end{itemize}
\begin{itemize}
\item {Proveniência:(Do gr. \textunderscore a\textunderscore  priv. + \textunderscore pous\textunderscore  + \textunderscore gune\textunderscore )}
\end{itemize}
Que não tem o ovário na base.
\section{Apodógyno}
\begin{itemize}
\item {Grp. gram.:m.}
\end{itemize}
\begin{itemize}
\item {Utilização:Bot.}
\end{itemize}
\begin{itemize}
\item {Proveniência:(Do gr. \textunderscore a\textunderscore  priv. + \textunderscore pous\textunderscore  + \textunderscore gune\textunderscore )}
\end{itemize}
Que não tem o ovário na base.
\section{Apodonte}
\begin{itemize}
\item {Grp. gram.:m.}
\end{itemize}
Gênero de peixes.
\section{Apodópnico}
\begin{itemize}
\item {Grp. gram.:adj.}
\end{itemize}
Diz-se do medicamento, que restabelece a respiração em caso de asphyxia.
\section{Apódose}
\begin{itemize}
\item {Grp. gram.:f.}
\end{itemize}
\begin{itemize}
\item {Utilização:Gram.}
\end{itemize}
\begin{itemize}
\item {Proveniência:(Gr. \textunderscore apodosis\textunderscore )}
\end{itemize}
Segunda parte de um período, em relação á primeira, cujo sentido completa.
\section{Apodrecido}
\begin{itemize}
\item {Grp. gram.:adj.}
\end{itemize}
Que apodreceu.
\section{Apodrecimento}
\begin{itemize}
\item {Grp. gram.:m.}
\end{itemize}
\begin{itemize}
\item {Utilização:Fig.}
\end{itemize}
\begin{itemize}
\item {Proveniência:(De \textunderscore apodrecer\textunderscore )}
\end{itemize}
Putrefacção.
Perversão.
\section{Apodrentar}
\begin{itemize}
\item {Grp. gram.:v. t.}
\end{itemize}
\begin{itemize}
\item {Proveniência:(De \textunderscore podre\textunderscore )}
\end{itemize}
O mesmo que \textunderscore apodrecer\textunderscore .
\section{Apodrido}
\begin{itemize}
\item {Grp. gram.:adj.}
\end{itemize}
Que começa a apodrecer. Cf. Th. Ribeiro, \textunderscore Jornadas\textunderscore , I, 340.
\section{Apodrir}
\begin{itemize}
\item {Grp. gram.:v. i.}
\end{itemize}
\begin{itemize}
\item {Utilização:Prov.}
\end{itemize}
Começar a apodrecer.
Apodrecer.
\section{Apodytério}
\begin{itemize}
\item {Grp. gram.:m.}
\end{itemize}
\begin{itemize}
\item {Proveniência:(Lat. \textunderscore apodyterium\textunderscore )}
\end{itemize}
Recinto nas antigas casas de banhos, em que se despiam as roupas e se penduravam em cabides.
\section{Apogão}
\begin{itemize}
\item {Grp. gram.:m.}
\end{itemize}
\begin{itemize}
\item {Utilização:Bot.}
\end{itemize}
Gênero de chicoriáceas.
\section{Apogastro}
\begin{itemize}
\item {Grp. gram.:adj.}
\end{itemize}
\begin{itemize}
\item {Proveniência:(Do gr. \textunderscore a\textunderscore  priv. + \textunderscore pous\textunderscore  + \textunderscore gaster\textunderscore )}
\end{itemize}
Diz-se dos molluscos, cujo ventre é desprovido de pés.
\section{Apogeu}
\begin{itemize}
\item {Grp. gram.:m.}
\end{itemize}
\begin{itemize}
\item {Utilização:Fig.}
\end{itemize}
\begin{itemize}
\item {Proveniência:(Gr. \textunderscore apogaios\textunderscore )}
\end{itemize}
Ponto, em que a Lua está mais distante da Terra.
O grau mais elevado: \textunderscore o apogeu da libertinagem\textunderscore .
\section{Apogístico}
\begin{itemize}
\item {Grp. gram.:adj.}
\end{itemize}
Relativo ao apogeu.
\section{Apogitagoara}
\begin{itemize}
\item {Grp. gram.:f.}
\end{itemize}
\begin{itemize}
\item {Utilização:Bras}
\end{itemize}
Planta rutácea, medicinal, do Brasil.
\section{Apogónia}
\begin{itemize}
\item {Grp. gram.:f.}
\end{itemize}
Gênero de coleópteros.
\section{Apógrafo}
\begin{itemize}
\item {Grp. gram.:m.}
\end{itemize}
\begin{itemize}
\item {Proveniência:(Gr. \textunderscore apographon\textunderscore )}
\end{itemize}
Cópia de um escrito original, (por opposição a \textunderscore autógrapho\textunderscore ).
Instrumento para copiar desenhos.
\section{Apógrapho}
\begin{itemize}
\item {Grp. gram.:m.}
\end{itemize}
\begin{itemize}
\item {Proveniência:(Gr. \textunderscore apographon\textunderscore )}
\end{itemize}
Cópia de um escrito original, (por opposição a \textunderscore autógrapho\textunderscore ).
Instrumento para copiar desenhos.
\section{Apohial}
\begin{itemize}
\item {Grp. gram.:m.}
\end{itemize}
\begin{itemize}
\item {Utilização:Anat.}
\end{itemize}
\begin{itemize}
\item {Proveniência:(Do gr. \textunderscore apo\textunderscore , volta, e \textunderscore hyal\textunderscore )}
\end{itemize}
Peça inferior do meio arco hyoídeo.
Pequeno côrno do hyoide.
\section{Apohyal}
\begin{itemize}
\item {Grp. gram.:m.}
\end{itemize}
\begin{itemize}
\item {Utilização:Anat.}
\end{itemize}
\begin{itemize}
\item {Proveniência:(Do gr. \textunderscore apo\textunderscore , volta, e \textunderscore hyal\textunderscore )}
\end{itemize}
Peça inferior do meio arco hyoídeo.
Pequeno côrno do hyoide.
\section{Apoiado}
\begin{itemize}
\item {Grp. gram.:m.}
\end{itemize}
\begin{itemize}
\item {Grp. gram.:Interj.}
\end{itemize}
\begin{itemize}
\item {Proveniência:(De \textunderscore apoiar\textunderscore )}
\end{itemize}
Applauso.
Apoio.
Muito bem!
\section{Apoiar}
\begin{itemize}
\item {Grp. gram.:v. t.}
\end{itemize}
Dar apoio a; applaudir.
Sustentar.
Basear.
Ajudar.
Confirmar.
Confiar.
Patrocinar: \textunderscore apoiar uma pretensão\textunderscore .
\section{Apoimento}
\begin{itemize}
\item {fónica:po-i}
\end{itemize}
\begin{itemize}
\item {Grp. gram.:m.}
\end{itemize}
\begin{itemize}
\item {Utilização:Ant.}
\end{itemize}
\begin{itemize}
\item {Proveniência:(De \textunderscore poêr\textunderscore , por \textunderscore pôr\textunderscore )}
\end{itemize}
Posição; postura.
\section{Apoio}
\begin{itemize}
\item {Grp. gram.:m.}
\end{itemize}
\begin{itemize}
\item {Proveniência:(Do lat. \textunderscore podium\textunderscore )}
\end{itemize}
Sustentáculo; base.
Protecção.
Applauso.
Ajuda.
Prova.
\section{Apojadura}
\begin{itemize}
\item {Grp. gram.:f.}
\end{itemize}
\begin{itemize}
\item {Proveniência:(De \textunderscore apojar\textunderscore )}
\end{itemize}
Grande afluência de leite no seio da mulher.
\section{Apojar}
\begin{itemize}
\item {Grp. gram.:v. i.}
\end{itemize}
\begin{itemize}
\item {Utilização:Prov.}
\end{itemize}
\begin{itemize}
\item {Utilização:alg.}
\end{itemize}
\begin{itemize}
\item {Utilização:Bras}
\end{itemize}
\begin{itemize}
\item {Proveniência:(Do lat. \textunderscore podium\textunderscore )}
\end{itemize}
Encher-se de leite ou de outro líquido.
Demorar-se.
Fazer chegar (o novilho), segunda vez, á teta da mãe, para se tirar o apojo.
\section{Apojo}
\begin{itemize}
\item {Grp. gram.:m.}
\end{itemize}
\begin{itemize}
\item {Utilização:Bras}
\end{itemize}
O leite mais grosso que se tira da vaca, depois de se tirar o primeiro, que é pouco espesso. Cf. Coruja, \textunderscore Coll. de Vocáb.\textunderscore 
\section{Apolainado}
\begin{itemize}
\item {Grp. gram.:adj.}
\end{itemize}
Que traz polainas. Cf. Camillo, \textunderscore Quéda\textunderscore , 111.
\section{Apolar}
\begin{itemize}
\item {Grp. gram.:adj.}
\end{itemize}
\begin{itemize}
\item {Utilização:Zool.}
\end{itemize}
\begin{itemize}
\item {Proveniência:(De \textunderscore a\textunderscore  priv. + \textunderscore polar\textunderscore )}
\end{itemize}
Diz-se da céllula nervosa arredondada. Cf. Max. Lemos, \textunderscore Zool\textunderscore .
\section{Apoldrada}
\begin{itemize}
\item {Grp. gram.:adj. f.}
\end{itemize}
Diz-se da égua que cria poldros.
\section{Apoleação}
\begin{itemize}
\item {Grp. gram.:f.}
\end{itemize}
Acto de \textunderscore apolear\textunderscore .
\section{Apolear}
\begin{itemize}
\item {Grp. gram.:v. t.}
\end{itemize}
O mesmo que \textunderscore polear\textunderscore .
\section{Apolegar}
\begin{itemize}
\item {Grp. gram.:v. t.}
\end{itemize}
Machucar, apertar com os dedos ou com as mãos. Cf. \textunderscore Peregrinação\textunderscore , XCVII.
(Cp. \textunderscore pollegar\textunderscore )
\section{Apolegar}
\begin{itemize}
\item {Grp. gram.:v. t.}
\end{itemize}
\begin{itemize}
\item {Utilização:Ant.}
\end{itemize}
O mesmo ou melhor que \textunderscore empolgar\textunderscore .
O mesmo que \textunderscore apolejar\textunderscore . Cf. \textunderscore Anat. Joc.\textunderscore , I, 136.
\section{Apolejador}
\begin{itemize}
\item {Grp. gram.:m.}
\end{itemize}
Aquelle que apolleja.
\section{Apolejar}
\begin{itemize}
\item {Grp. gram.:v. t.}
\end{itemize}
\begin{itemize}
\item {Proveniência:(Do lat. \textunderscore pollex\textunderscore )}
\end{itemize}
Amassar com os dedos.
\section{Apolentar}
\begin{itemize}
\item {Grp. gram.:v. t.}
\end{itemize}
Engordar com polenta.
\section{Apolentar}
\begin{itemize}
\item {Grp. gram.:v. t.}
\end{itemize}
\begin{itemize}
\item {Utilização:Prov.}
\end{itemize}
\begin{itemize}
\item {Utilização:beir.}
\end{itemize}
Palpar (a fruta) com os dedos, a vêr se está madura.
(Cp. \textunderscore apollejar\textunderscore )
\section{Apólice}
\begin{itemize}
\item {Grp. gram.:f.}
\end{itemize}
\begin{itemize}
\item {Proveniência:(Ingl. \textunderscore policy\textunderscore )}
\end{itemize}
Certificado de obrigação mercantil ou financeira.
Acção de Companhia.
Documento de seguro contra incêndio ou contra outro sinistro.
\section{Apolíneo}
\begin{itemize}
\item {Grp. gram.:adj.}
\end{itemize}
\begin{itemize}
\item {Proveniência:(Lat. \textunderscore apollineus\textunderscore )}
\end{itemize}
Relativo a Apolo.
Formoso como Apontalo.
\section{Apolínico}
\begin{itemize}
\item {Grp. gram.:adj.}
\end{itemize}
Relativo a Apolo; apolíneo. Cf. Castilho, \textunderscore Fastos\textunderscore , II, 619.
\section{Apolisina}
\begin{itemize}
\item {Grp. gram.:f.}
\end{itemize}
Preparado pharmacêutico, antipyrético e analgésico.
\section{Apolitana}
\begin{itemize}
\item {Grp. gram.:f.}
\end{itemize}
\begin{itemize}
\item {Utilização:Prov.}
\end{itemize}
\begin{itemize}
\item {Utilização:alent.}
\end{itemize}
\begin{itemize}
\item {Utilização:Fig.}
\end{itemize}
Reunião do terno, duque e ás do mesmo naipe, no jôgo dos três setes.
Azar; infelicidade.
(Por \textunderscore napolitana\textunderscore , fem. de \textunderscore napolitano\textunderscore )
\section{Apollegar}
\begin{itemize}
\item {Grp. gram.:v. t.}
\end{itemize}
\begin{itemize}
\item {Utilização:Ant.}
\end{itemize}
O mesmo ou melhor que \textunderscore empolgar\textunderscore .
O mesmo que \textunderscore apollejar\textunderscore . Cf. \textunderscore Anat. Joc.\textunderscore , I, 136.
\section{Apollejador}
\begin{itemize}
\item {Grp. gram.:m.}
\end{itemize}
Aquelle que apolleja.
\section{Apollejar}
\begin{itemize}
\item {Grp. gram.:v. t.}
\end{itemize}
\begin{itemize}
\item {Proveniência:(Do lat. \textunderscore pollex\textunderscore )}
\end{itemize}
Amassar com os dedos.
\section{Apollíneo}
\begin{itemize}
\item {Grp. gram.:adj.}
\end{itemize}
\begin{itemize}
\item {Proveniência:(Lat. \textunderscore apollineus\textunderscore )}
\end{itemize}
Relativo a Apollo.
Formoso como Apollo.
\section{Apollínico}
\begin{itemize}
\item {Grp. gram.:adj.}
\end{itemize}
Relativo a Apollo; apollíneo. Cf. Castilho, \textunderscore Fastos\textunderscore , II, 619.
\section{Apollóneas}
\begin{itemize}
\item {Grp. gram.:f. pl.}
\end{itemize}
Antigas festas gregas em honra de Apollo.
\section{Apollónicon}
\begin{itemize}
\item {Grp. gram.:m.}
\end{itemize}
Órgão de manivela, espécie de grande realejo, inventado em 1824.
\section{Apollónion}
\begin{itemize}
\item {Grp. gram.:m.}
\end{itemize}
Espécie desusada de piano de dois teclados, com alguns registos de órgão e, na parte superior, uma figura automática que tocava frauta.
\section{Apologal}
\begin{itemize}
\item {Grp. gram.:adj.}
\end{itemize}
Relativo a apólogos.
Que contém apólogo.
\section{Apologético}
\begin{itemize}
\item {Grp. gram.:adj.}
\end{itemize}
\begin{itemize}
\item {Proveniência:(Gr. \textunderscore apologetikos\textunderscore )}
\end{itemize}
Que contém apologia.
\section{Apologia}
\begin{itemize}
\item {Grp. gram.:f.}
\end{itemize}
\begin{itemize}
\item {Proveniência:(Gr. \textunderscore apologia\textunderscore )}
\end{itemize}
Discurso, para justificar ou defender.
Elogio, encómio.
\section{Apológico}
\begin{itemize}
\item {Grp. gram.:adj.}
\end{itemize}
(V.apologético)
\section{Apologista}
\begin{itemize}
\item {Grp. gram.:m.  ou  f.}
\end{itemize}
Quem faz apologia.
\section{Apologizar}
\begin{itemize}
\item {Grp. gram.:v. t.}
\end{itemize}
Fazer a apologia de. Cf. Camillo, \textunderscore Narcót.\textunderscore , I, 88.
\section{Apólogo}
\begin{itemize}
\item {Grp. gram.:m.}
\end{itemize}
\begin{itemize}
\item {Proveniência:(Gr. \textunderscore apologos\textunderscore )}
\end{itemize}
Allegoria moral, em que figuram, falando, animaes ou coisas inanimadas.
\section{Apolóneas}
\begin{itemize}
\item {Grp. gram.:f. pl.}
\end{itemize}
Antigas festas gregas em honra de Apolo.
\section{Apolónicon}
\begin{itemize}
\item {Grp. gram.:m.}
\end{itemize}
Órgão de manivela, espécie de grande realejo, inventado em 1824.
\section{Apolónion}
\begin{itemize}
\item {Grp. gram.:m.}
\end{itemize}
Espécie desusada de piano de dois teclados, com alguns registos de órgão e, na parte superior, uma figura automática que tocava frauta.
\section{Apoltronar-se}
\begin{itemize}
\item {Grp. gram.:v. p.}
\end{itemize}
Tornar-se poltrão, cobarde.
\section{Apoltronar-se}
\begin{itemize}
\item {Grp. gram.:v. p.}
\end{itemize}
Sentar-se em poltrona.
\section{Apolvilhar}
\textunderscore v. t.\textunderscore  (e der.)
O mesmo que \textunderscore polvilhar\textunderscore , etc.
\section{Apolysina}
\begin{itemize}
\item {Grp. gram.:f.}
\end{itemize}
Preparado pharmacêutico, antipyrético e analgésico.
\section{Apomecometria}
\begin{itemize}
\item {Grp. gram.:f.}
\end{itemize}
\begin{itemize}
\item {Proveniência:(De \textunderscore apomecómetro\textunderscore )}
\end{itemize}
Arte de medir a distância e avaliar a natureza dos objectos afastados.
\section{Apomecómetro}
\begin{itemize}
\item {Grp. gram.:m.}
\end{itemize}
\begin{itemize}
\item {Proveniência:(Do gr. \textunderscore apo\textunderscore  + \textunderscore mekos\textunderscore  + \textunderscore metron\textunderscore )}
\end{itemize}
Instrumento, para medir a distância de objectos muito afastados.
\section{Apomorfina}
\begin{itemize}
\item {Grp. gram.:f.}
\end{itemize}
Medicamento, que se emprega como vomitivo, em caso de envenenamento.
\section{Apomorphina}
\begin{itemize}
\item {Grp. gram.:f.}
\end{itemize}
Medicamento, que se emprega como vomitivo, em caso de envenenamento.
\section{Apompar}
\begin{itemize}
\item {Grp. gram.:v. t.}
\end{itemize}
Tornar pomposo. Cf. Camillo, \textunderscore Canc. Al.\textunderscore , 173.
\section{Aponeurologia}
\begin{itemize}
\item {Grp. gram.:f.}
\end{itemize}
\begin{itemize}
\item {Proveniência:(Do gr. \textunderscore aponeurosis\textunderscore  + \textunderscore logos\textunderscore )}
\end{itemize}
Parte da Anatomia, que trata das aponeuroses.
\section{Aponeurose}
\begin{itemize}
\item {Grp. gram.:f.}
\end{itemize}
\begin{itemize}
\item {Proveniência:(Gr. \textunderscore aponeurosis\textunderscore )}
\end{itemize}
Membrana consistente e fibrosa, que envolve os músculos ou lhes serve de intersecção.
\section{Aponeurótico}
\begin{itemize}
\item {Grp. gram.:m.}
\end{itemize}
Relativo a \textunderscore aponeurose\textunderscore .
\section{Aponeurótomo}
\begin{itemize}
\item {Grp. gram.:m.}
\end{itemize}
\begin{itemize}
\item {Proveniência:(Do gr. \textunderscore aponeurosis\textunderscore  + \textunderscore tome\textunderscore )}
\end{itemize}
Instrumento cirúrgico, para dividir a aponeurose abdominal.
\section{Aponevrose}
\textunderscore f.\textunderscore  (e der.)
(V. \textunderscore aponeurose\textunderscore , etc.)
\section{Aponitrose}
\begin{itemize}
\item {Grp. gram.:f.}
\end{itemize}
Acto de polvilhar com nitro uma chaga ou úlcera.
\section{Apontadamente}
\begin{itemize}
\item {Grp. gram.:adv.}
\end{itemize}
Pontualmente; rigorosamente.
\section{Apontado}
\begin{itemize}
\item {Grp. gram.:adj.}
\end{itemize}
Que tem ponta; que termina em ponta. Cf. Filinto, \textunderscore D. Man.\textunderscore , II, 24.
Casquilho, garrido. Cf. \textunderscore Eufrosina\textunderscore , 249.
\section{Apontador}
\begin{itemize}
\item {Grp. gram.:m.}
\end{itemize}
\begin{itemize}
\item {Proveniência:(De \textunderscore apontar\textunderscore )}
\end{itemize}
Aquelle que aponta.
Aquelle que faz pontas de instrumentos.
Aquelle que faz pontaria.
Aquelle que faz o rol e aponta o serviço e as faltas de certos trabalhadores. Livro, em que se apontam serviços ou faltas de trabalhadores.
O homem, que serve de ponto nos theatros.
\section{Apontamento}
\begin{itemize}
\item {Grp. gram.:m.}
\end{itemize}
\begin{itemize}
\item {Proveniência:(De \textunderscore apontar\textunderscore ^2)}
\end{itemize}
Registo; nota; lembrança.
Escrita do que se há de fazer.
Plano.
\section{Apontar}
\begin{itemize}
\item {Grp. gram.:v. t.}
\end{itemize}
\begin{itemize}
\item {Grp. gram.:V. i.}
\end{itemize}
Aguçar; fazer a ponta a.
Começar a apparecer: \textunderscore quando a lua apontava...\textunderscore 
\section{Apontar}
\begin{itemize}
\item {Grp. gram.:v. t.}
\end{itemize}
\begin{itemize}
\item {Proveniência:(De \textunderscore ponto\textunderscore )}
\end{itemize}
Indicar; marcar; notar.
Sugerir.
Tomar nota de (letras de câmbio).
Preparar.
Bosquejar; mencionar de leve.
\section{Apontear}
\textunderscore v. t.\textunderscore  (e der.)
(V. \textunderscore pontear\textunderscore , etc.)
\section{Apontoado}
\begin{itemize}
\item {Grp. gram.:adj.}
\end{itemize}
Seguro com pontos largos.
\section{Apontoar}
\begin{itemize}
\item {Grp. gram.:v. t.}
\end{itemize}
Segurar com pontos largos.
\section{Apontoar}
\begin{itemize}
\item {Grp. gram.:v. t.}
\end{itemize}
Encher, segurar com pontões.
Citar a ponto, a propósito. Cf. Filinto, XVI, 267.
\section{Apophonia}
\begin{itemize}
\item {Grp. gram.:f.}
\end{itemize}
\begin{itemize}
\item {Utilização:Gram.}
\end{itemize}
Alteração do valor phonético da vogal de um radical, sem sêr por influência da vogal final: \textunderscore tijôlo\textunderscore , \textunderscore tijólos\textunderscore .
(Cp. \textunderscore metaphonia\textunderscore )
\section{Apophoretos}
\begin{itemize}
\item {Grp. gram.:m. pl.}
\end{itemize}
\begin{itemize}
\item {Proveniência:(Gr. \textunderscore apophoreta\textunderscore )}
\end{itemize}
Dádivas, que se distribuiam nas saturnaes.
\section{Apophtegma}
\begin{itemize}
\item {Grp. gram.:m.}
\end{itemize}
\begin{itemize}
\item {Proveniência:(Gr. \textunderscore apophtegma\textunderscore )}
\end{itemize}
Dito sentencioso de pessôa illustre.
\section{Apophtegmatismo}
\begin{itemize}
\item {Grp. gram.:m.}
\end{itemize}
Emprêgo ou uso de apophtegmas.
\section{Apóphyge}
\begin{itemize}
\item {Grp. gram.:f.}
\end{itemize}
\begin{itemize}
\item {Utilização:Ant.}
\end{itemize}
\begin{itemize}
\item {Proveniência:(Gr. \textunderscore apophuge\textunderscore , saída)}
\end{itemize}
Parte de uma columna, immediatamente superior á base.
Cinta de ferro, que abraça a columna, para que esta se não parta.
\section{Apóphyse}
\begin{itemize}
\item {Grp. gram.:f.}
\end{itemize}
\begin{itemize}
\item {Utilização:Anat.}
\end{itemize}
\begin{itemize}
\item {Proveniência:(Gr. \textunderscore apophusis\textunderscore )}
\end{itemize}
Parte saliente de um osso ou de um órgão.
\section{Apophysiário}
\begin{itemize}
\item {Grp. gram.:adj.}
\end{itemize}
Relativo a apóphyse.
\section{Apoplanésia}
\begin{itemize}
\item {Grp. gram.:f.}
\end{itemize}
\begin{itemize}
\item {Proveniência:(Do gr. \textunderscore apoplanaein\textunderscore )}
\end{itemize}
Planta leguminosa.
\section{Apopléctico}
\begin{itemize}
\item {Grp. gram.:adj.}
\end{itemize}
\begin{itemize}
\item {Proveniência:(Gr. \textunderscore apoplektikos\textunderscore )}
\end{itemize}
Relativo a \textunderscore apoplexia\textunderscore .
Sujeito a apoplexia.
Irritado; acalorado.
\section{Apoplexia}
\begin{itemize}
\item {fónica:plè-sí}
\end{itemize}
\begin{itemize}
\item {Grp. gram.:f.}
\end{itemize}
\begin{itemize}
\item {Proveniência:(Gr. \textunderscore apoplexia\textunderscore )}
\end{itemize}
Doença, que produz a perda das sensações e do movimento.
Moléstia, que ataca as videiras, secando-lhes o fruto depois de desenvolvido.
\section{Apoquentação}
\begin{itemize}
\item {Grp. gram.:f.}
\end{itemize}
Acto ou effeito de \textunderscore apoquentar\textunderscore .
\section{Apoquentador}
\begin{itemize}
\item {Grp. gram.:adj.}
\end{itemize}
\begin{itemize}
\item {Grp. gram.:M.}
\end{itemize}
Que apoquenta; impertinente.
Aquelle que apoquenta.
\section{Apoquentar}
\begin{itemize}
\item {Grp. gram.:v. t.}
\end{itemize}
Opprimir, affligir, importunar, molestar.
(Por \textunderscore apouquentar\textunderscore , de \textunderscore pouco\textunderscore )
\section{Apoquilismo}
\begin{itemize}
\item {Grp. gram.:m.}
\end{itemize}
O mesmo que \textunderscore arrobe\textunderscore .
\section{Aporfiar}
\textunderscore v. t.\textunderscore  (e der.)
O mesmo que \textunderscore porfiar\textunderscore ^2, etc.
\section{Aporia}
\begin{itemize}
\item {Grp. gram.:f.}
\end{itemize}
\begin{itemize}
\item {Proveniência:(Gr. \textunderscore aporia\textunderscore )}
\end{itemize}
Figura de Rhetórica, com que o orador mostra hesitar sôbre o que há de dizer.
\section{Aporisma}
\begin{itemize}
\item {Grp. gram.:m.}
\end{itemize}
\begin{itemize}
\item {Utilização:Med.}
\end{itemize}
Extravazação do sangue.
\section{Aporismar}
\begin{itemize}
\item {Grp. gram.:v. t.}
\end{itemize}
O mesmo que \textunderscore apostemar\textunderscore .
\section{Áporo}
\begin{itemize}
\item {Grp. gram.:m.}
\end{itemize}
\begin{itemize}
\item {Proveniência:(Gr. \textunderscore aporos\textunderscore )}
\end{itemize}
Problema de difficil resolução.
Planta, da fam. das orchídeas.
Insecto hymenóptero.
\section{Aporobrânchio}
\begin{itemize}
\item {fónica:qui}
\end{itemize}
\begin{itemize}
\item {Grp. gram.:m.}
\end{itemize}
\begin{itemize}
\item {Grp. gram.:Adj.}
\end{itemize}
\begin{itemize}
\item {Proveniência:(Do gr. \textunderscore aporos\textunderscore  + \textunderscore brankhia\textunderscore )}
\end{itemize}
Animal articulado, da classe dos arachnídeos.
Mollusco cephalópode.
Que tem guelras pouco desenvolvidas.
\section{Aporobrânquio}
\begin{itemize}
\item {Grp. gram.:m.}
\end{itemize}
\begin{itemize}
\item {Grp. gram.:Adj.}
\end{itemize}
\begin{itemize}
\item {Proveniência:(Do gr. \textunderscore aporos\textunderscore  + \textunderscore brankhia\textunderscore )}
\end{itemize}
Animal articulado, da classe dos arachnídeos.
Mollusco cephalópode.
Que tem guelras pouco desenvolvidas.
\section{Aporosa}
\begin{itemize}
\item {Grp. gram.:f.}
\end{itemize}
Gênero de plantas euphorbiáceas.
\section{Aporrear}
\begin{itemize}
\item {Grp. gram.:v. t.}
\end{itemize}
\begin{itemize}
\item {Utilização:Des.}
\end{itemize}
\begin{itemize}
\item {Grp. gram.:V. i.}
\end{itemize}
\begin{itemize}
\item {Utilização:Bras}
\end{itemize}
Espancar; dar pancadas em.
Diz-se do cavallo mal domado ou que se não conseguiu domar.
(Cast. \textunderscore aporrear\textunderscore )
\section{Aporrechar}
\begin{itemize}
\item {Grp. gram.:v. t.}
\end{itemize}
\begin{itemize}
\item {Utilização:T. de Moncorvo}
\end{itemize}
Apoquentar; opprimir; aporrinhar.
(Cp. \textunderscore aporretar\textunderscore )
\section{Aporretar}
\begin{itemize}
\item {Grp. gram.:v. t.}
\end{itemize}
Desancar com porrete.
\section{Aporrinhar}
\begin{itemize}
\item {Grp. gram.:v. t.}
\end{itemize}
\begin{itemize}
\item {Utilização:Pop.}
\end{itemize}
Apoquentar; affligir.
(Cp. \textunderscore aporrear\textunderscore )
\section{Aportada}
\begin{itemize}
\item {Grp. gram.:f.}
\end{itemize}
\begin{itemize}
\item {Utilização:Ant.}
\end{itemize}
O mesmo que \textunderscore aportamento\textunderscore .
\section{Aportamento}
\begin{itemize}
\item {Grp. gram.:m.}
\end{itemize}
Acção de \textunderscore aportar\textunderscore .
\section{Aportar}
\begin{itemize}
\item {Grp. gram.:v. i.}
\end{itemize}
Chegar ao porto.
Fundear; lançar âncora.
\section{Aportelado}
\begin{itemize}
\item {Grp. gram.:m.}
\end{itemize}
\begin{itemize}
\item {Utilização:Ant.}
\end{itemize}
\begin{itemize}
\item {Proveniência:(De \textunderscore portela\textunderscore )}
\end{itemize}
Juiz da vintena.
\section{Aportilhar}
\begin{itemize}
\item {Grp. gram.:v. t.}
\end{itemize}
Fazer portilhas em.
Abrir fendas em.
\section{Aportuguesadamente}
\begin{itemize}
\item {Grp. gram.:adv.}
\end{itemize}
Semelhantemente a português.
\section{Aportuguesado}
\begin{itemize}
\item {Grp. gram.:adv.}
\end{itemize}
Que tem fórma portuguesa.
\section{Aportuguesamento}
\begin{itemize}
\item {Grp. gram.:m.}
\end{itemize}
Acto ou effeito de \textunderscore aportuguesar\textunderscore .
\section{Aportuguesar}
\begin{itemize}
\item {Grp. gram.:v. t.}
\end{itemize}
Tornar português; dar feição portuguesa a: \textunderscore aportuguesando«tourisme», temos«turismo»\textunderscore .
\section{Aportuguesável}
\begin{itemize}
\item {Grp. gram.:adj.}
\end{itemize}
Que se póde aportuguesar.
\section{Após}
\begin{itemize}
\item {Grp. gram.:prep.}
\end{itemize}
\begin{itemize}
\item {Grp. gram.:Adv.}
\end{itemize}
\begin{itemize}
\item {Proveniência:(Do lat. \textunderscore post\textunderscore )}
\end{itemize}
Depois de: \textunderscore correu após êlle\textunderscore .
Depois.
\section{Aposarco}
\begin{itemize}
\item {Grp. gram.:adj.}
\end{itemize}
\begin{itemize}
\item {Utilização:Med.}
\end{itemize}
\begin{itemize}
\item {Proveniência:(Do gr. \textunderscore apo\textunderscore  + \textunderscore sarkos\textunderscore )}
\end{itemize}
Que faz crescer a carne da ferida.
\section{Aposcepsia}
\begin{itemize}
\item {Grp. gram.:f.}
\end{itemize}
\begin{itemize}
\item {Utilização:Med.}
\end{itemize}
Passagem súbita dos humores, de uma para outra parte do corpo.
\section{Aposentação}
\begin{itemize}
\item {Grp. gram.:f.}
\end{itemize}
Acto de \textunderscore aposentar\textunderscore .
Estado de quem se aposentou.
\section{Aposentador}
\begin{itemize}
\item {Grp. gram.:m.}
\end{itemize}
Aquelle que aposenta.
\section{Aposentadoria}
\begin{itemize}
\item {Grp. gram.:f.}
\end{itemize}
\begin{itemize}
\item {Proveniência:(De \textunderscore aposentar\textunderscore )}
\end{itemize}
Hospedagem: albergaria.
Aposentação.
\section{Aposentamento}
\begin{itemize}
\item {Grp. gram.:m.}
\end{itemize}
O mesmo que \textunderscore aposentação\textunderscore .
\section{Aposentar}
\begin{itemize}
\item {Grp. gram.:v. t.}
\end{itemize}
\begin{itemize}
\item {Proveniência:(De \textunderscore aposento\textunderscore )}
\end{itemize}
Hospedar; dar abrigo a.
Jubilar; conceder officialmente a (alguém) o direito, ou impor o dever, de deixar o exercício de funcções públicas, sem deixar de receber o vencimento, no todo ou em parte.
\section{Aposento}
\begin{itemize}
\item {Grp. gram.:m.}
\end{itemize}
Casa; moradia.
Agasalho.
Compartimento de casa; quarto.
(Cast. \textunderscore aposento\textunderscore )
\section{Aposiopese}
\begin{itemize}
\item {Grp. gram.:f.}
\end{itemize}
\begin{itemize}
\item {Utilização:Rhet.}
\end{itemize}
\begin{itemize}
\item {Proveniência:(Gr. \textunderscore aposiopesis\textunderscore )}
\end{itemize}
Reticência.
Interrupção de phrase.
\section{Aposítico}
\begin{itemize}
\item {Grp. gram.:adj.}
\end{itemize}
\begin{itemize}
\item {Utilização:Med.}
\end{itemize}
Que faz cessar o appetite.
\section{Apospasmo}
\begin{itemize}
\item {Grp. gram.:m.}
\end{itemize}
Solução de continuidade nos ligamentos do organismo humano.
\section{Apossar}
\begin{itemize}
\item {Grp. gram.:v. t.}
\end{itemize}
\begin{itemize}
\item {Grp. gram.:V. p.}
\end{itemize}
Pôr de posse.
Tomar posse, apoderar-se.
\section{Apossuir-se}
\begin{itemize}
\item {Grp. gram.:v. p.}
\end{itemize}
Apossar-se:«\textunderscore ilhas, de que se tinha apossuido\textunderscore ». Filinto, \textunderscore D. Man.\textunderscore , II, 283.
\section{Aposta}
\begin{itemize}
\item {Grp. gram.:f.}
\end{itemize}
\begin{itemize}
\item {Proveniência:(De \textunderscore apostar\textunderscore )}
\end{itemize}
Ajuste entre pessôas, que affirmam coisas differentes, devendo, a que não acertar, dar á outra uma quantia ou coisa determinada.
A coisa ou quantia que se aposta.
Desafio.
\section{Apostadamente}
\begin{itemize}
\item {Grp. gram.:adv.}
\end{itemize}
De propósito; determinadamente.
\section{Apostador}
\begin{itemize}
\item {Grp. gram.:m.}
\end{itemize}
Aquelle que aposta.
\section{Apostar}
\begin{itemize}
\item {Grp. gram.:v. t.}
\end{itemize}
\begin{itemize}
\item {Utilização:Ant.}
\end{itemize}
\begin{itemize}
\item {Grp. gram.:V. i.}
\end{itemize}
\begin{itemize}
\item {Proveniência:(De \textunderscore postar\textunderscore )}
\end{itemize}
Fazer aposta de: \textunderscore apostou cinco tostões\textunderscore .
Sustentar: \textunderscore aposto que chove amanhan\textunderscore .
Arriscar.
Preparar; pôr em ordem.
Fazer aposta:«\textunderscore apostarei dobrado contra singelo, em como...\textunderscore »Castilho, \textunderscore Sabichonas\textunderscore , 34.
\section{Apóstase}
\begin{itemize}
\item {Grp. gram.:f.}
\end{itemize}
\begin{itemize}
\item {Proveniência:(Gr. \textunderscore apostasis\textunderscore )}
\end{itemize}
Formação do abscesso.
\section{Apostasia}
\begin{itemize}
\item {Grp. gram.:f.}
\end{itemize}
\begin{itemize}
\item {Proveniência:(Gr. \textunderscore apostasia\textunderscore )}
\end{itemize}
Mudança de religião; abjuração.
Acto de abandonar um partido ou uma opinião.
\section{Apostásia}
\begin{itemize}
\item {Grp. gram.:f.}
\end{itemize}
Gênero de plantas, que servem de typo ás apostasiáceas.
\section{Apostasiáceas}
\begin{itemize}
\item {Grp. gram.:f. pl.}
\end{itemize}
\begin{itemize}
\item {Proveniência:(De \textunderscore apostásia\textunderscore )}
\end{itemize}
Família de plantas phanerogâmicas, vivazes, originárias da Índia.
\section{Apóstata}
\begin{itemize}
\item {Grp. gram.:adj.}
\end{itemize}
\begin{itemize}
\item {Proveniência:(Do gr. \textunderscore apostates\textunderscore )}
\end{itemize}
Aquelle que apostatou.
\section{Apostatar}
\begin{itemize}
\item {Grp. gram.:v. i.}
\end{itemize}
\begin{itemize}
\item {Proveniência:(De \textunderscore apóstata\textunderscore )}
\end{itemize}
Mudar de religião.
Mudar de partido.
\section{Apostático}
\begin{itemize}
\item {Grp. gram.:adj.}
\end{itemize}
Relativo a \textunderscore apostasia\textunderscore .
\section{Apostema}
\begin{itemize}
\item {Grp. gram.:m.}
\end{itemize}
\begin{itemize}
\item {Proveniência:(Gr. \textunderscore apostema\textunderscore )}
\end{itemize}
O mesmo que \textunderscore abscesso\textunderscore .
\section{Apostemar}
\begin{itemize}
\item {Grp. gram.:v. t.}
\end{itemize}
\begin{itemize}
\item {Grp. gram.:V. i.}
\end{itemize}
Corromper, estragar.
Criar apostema.
\section{Apostemático}
\begin{itemize}
\item {Grp. gram.:adj.}
\end{itemize}
Relativo ao apostema.
\section{Apostemeira}
\begin{itemize}
\item {Grp. gram.:f.}
\end{itemize}
\begin{itemize}
\item {Proveniência:(De \textunderscore apostema\textunderscore )}
\end{itemize}
Planta do Maranhão.
\section{Apostemoso}
\begin{itemize}
\item {Grp. gram.:adj.}
\end{itemize}
Relativo ao apostema.
\section{Apostiçar}
\begin{itemize}
\item {Grp. gram.:v. t.}
\end{itemize}
\begin{itemize}
\item {Utilização:Prov.}
\end{itemize}
\begin{itemize}
\item {Utilização:minh.}
\end{itemize}
\begin{itemize}
\item {Proveniência:(De \textunderscore pôsto\textunderscore )}
\end{itemize}
Enjeitar.
\section{Apostila}
\begin{itemize}
\item {Grp. gram.:f.}
\end{itemize}
\begin{itemize}
\item {Proveniência:(De \textunderscore postilla\textunderscore )}
\end{itemize}
Addicionamento, annotação a um escrito.
Aditamento a um diploma official.
Commentário.
Recommendação, á margem de um requerimento.
\section{Apostilador}
\begin{itemize}
\item {Grp. gram.:m.}
\end{itemize}
Aquelle que apostila.
\section{Apostilar}
\begin{itemize}
\item {Grp. gram.:v. t.}
\end{itemize}
\begin{itemize}
\item {Proveniência:(De \textunderscore apostilla\textunderscore )}
\end{itemize}
Notar.
Explicar.
Fazer apostilas a: \textunderscore Gonçalves Viana apostilou os diccionários\textunderscore .
\section{Apostilha}
\begin{itemize}
\item {Grp. gram.:f.}
\end{itemize}
O mesmo que \textunderscore apostilla\textunderscore .
\section{Apostilla}
\begin{itemize}
\item {Grp. gram.:f.}
\end{itemize}
\begin{itemize}
\item {Proveniência:(De \textunderscore postilla\textunderscore )}
\end{itemize}
Addicionamento, annotação a um escrito.
Aditamento a um diploma official.
Commentário.
Recommendação, á margem de um requerimento.
\section{Apostillador}
\begin{itemize}
\item {Grp. gram.:m.}
\end{itemize}
Aquelle que apostilla.
\section{Apostillar}
\begin{itemize}
\item {Grp. gram.:v. t.}
\end{itemize}
\begin{itemize}
\item {Proveniência:(De \textunderscore apostilla\textunderscore )}
\end{itemize}
Notar.
Explicar.
Fazer apostillas a: \textunderscore Gonçalves Viana apostillou os diccionários\textunderscore .
\section{Apostolado}
\begin{itemize}
\item {Grp. gram.:m.}
\end{itemize}
\begin{itemize}
\item {Utilização:Ant.}
\end{itemize}
Missão de apóstolo.
Grupo dos apóstolos.
Propagação de uma doutrina.
Representante ou enviado de um príncipe, para tratar certos negócios.
\section{Apostolar}
\begin{itemize}
\item {Grp. gram.:v. i.}
\end{itemize}
\begin{itemize}
\item {Grp. gram.:V. t.}
\end{itemize}
\begin{itemize}
\item {Proveniência:(De \textunderscore apóstolo\textunderscore )}
\end{itemize}
Prègar o Evangelho.
Prègar como apóstolo.
Diffundir prègando.
Vulgarizar, falando ou escrevendo.
\section{Aparatar}
\begin{itemize}
\item {Grp. gram.:v. t.}
\end{itemize}
\begin{itemize}
\item {Proveniência:(De \textunderscore apparato\textunderscore )}
\end{itemize}
Tornar aparatoso.
Adornar.
\section{Aparato}
\begin{itemize}
\item {Grp. gram.:m.}
\end{itemize}
Apresentação pomposa.
Preparação para festa.
Esplendor; magnificência; luxo.
Apresto.
Reunião de elementos para uma composição.
Apparelho, em que está o puado, nas officinas de cardagem. Cf. \textunderscore Inquér. Indust.\textunderscore , p. II, l. 111, 67.
\section{Aparatosamente}
\begin{itemize}
\item {Grp. gram.:adv.}
\end{itemize}
De modo \textunderscore aparatoso\textunderscore .
Com aparato.
\section{Aparatoso}
\begin{itemize}
\item {Grp. gram.:adj.}
\end{itemize}
Em que há aparato.
Magnificente.
\section{Aparecente}
\begin{itemize}
\item {Grp. gram.:adj.}
\end{itemize}
Que começa a aparecer.
\section{Aparecer}
\begin{itemize}
\item {Grp. gram.:v. i.}
\end{itemize}
\begin{itemize}
\item {Proveniência:(Do lat. \textunderscore apparere\textunderscore )}
\end{itemize}
Fazer-se vêr.
Apresentar-se; comparecer: \textunderscore não pude apparecer na assembleia\textunderscore .
Revelar-se.
Acontecer.
\section{Aparecido}
\begin{itemize}
\item {Grp. gram.:adj.}
\end{itemize}
Que apareceu.
\section{Aparecimento}
\begin{itemize}
\item {Grp. gram.:m.}
\end{itemize}
Acto de \textunderscore aparecer\textunderscore .
\section{Aparelhadamente}
\begin{itemize}
\item {Grp. gram.:adv.}
\end{itemize}
De modo \textunderscore aparelhado\textunderscore .
Com preparação.
\section{Aparelhado}
\begin{itemize}
\item {Grp. gram.:adv.}
\end{itemize}
\begin{itemize}
\item {Utilização:Prov.}
\end{itemize}
\begin{itemize}
\item {Utilização:minh.}
\end{itemize}
Preparado, disposto: \textunderscore aparelhado para combate\textunderscore .
Destinado.
Enfeitado; arreado: \textunderscore um cavallo aparelhado\textunderscore .
Concertado.
Abastecido: \textunderscore um cesto aparelhado\textunderscore .
\section{Aparelhador}
\begin{itemize}
\item {Grp. gram.:m.}
\end{itemize}
Aquelle que aparelha.
O encarregado de certas obras, immediatamente inferior ao architecto ou ao mestre.
\section{Aparelhamento}
\begin{itemize}
\item {Grp. gram.:m.}
\end{itemize}
Acto ou effeito de \textunderscore aparelhar\textunderscore .
Aparelho.
\section{Aparelhar}
\begin{itemize}
\item {Grp. gram.:v. t.}
\end{itemize}
\begin{itemize}
\item {Proveniência:(De \textunderscore apparelho\textunderscore )}
\end{itemize}
Preparar.
Tornar disposto; aprestar.
Pôr arreios em (cavalgaduras).
Desbastar (madeira ou pedra) para certas obras.
Enfeitar.
Concertar.
\section{Aparelho}
\begin{itemize}
\item {Grp. gram.:m.}
\end{itemize}
\begin{itemize}
\item {Utilização:Náut.}
\end{itemize}
\begin{itemize}
\item {Proveniência:(De um lat. hypoth. \textunderscore appariculum\textunderscore ? Cp. \textunderscore parelho\textunderscore , fr. \textunderscore pareil\textunderscore )}
\end{itemize}
Acto de aparelhar.
Preparo.
Arreios de cavalgadura.
Conjunto de utensílios náuticos.
Aprestos béllicos.
Baixella; alfaias; peças do serviço culinário.
Máchina para levantar pesos.
Primeira camada de óleo no pano que se vai pintar.
Conjunto dos objectos necessários para uma operação cirúrgica.
Corda com anzoes, que se atravessa no ponto do rio, por onde se espera que o peixe passe.
Conjunto dos órgãos que, num corpo organizado, cooperam para a mesma funcção. Desbaste ou córte de pedras para revestimento de cantarias.
Trem de lavoira; apeiro.
Conjunto dos mastros, paus, mastaréus e respectivas vêrgas, pano, cabos fixos e cabos de laborar, próprios de uma embarcação, e pelos quaes se determina a classificação do navio: \textunderscore apparelho de escuna\textunderscore ; \textunderscore apparelho de brigue\textunderscore .
\section{Aparência}
\begin{itemize}
\item {Grp. gram.:f.}
\end{itemize}
\begin{itemize}
\item {Proveniência:(Lat. \textunderscore apparentia\textunderscore )}
\end{itemize}
Aquillo que se mostra á primeira vista; exterioridade; aspecto: \textunderscore aparência de bôa pessôa\textunderscore .
Probabilidade.
Fingimento; disfarce.
\section{Aparentar}
\begin{itemize}
\item {Grp. gram.:v. t.}
\end{itemize}
\begin{itemize}
\item {Proveniência:(De \textunderscore apparente\textunderscore )}
\end{itemize}
Dar aparência de: \textunderscore aparentar innocência\textunderscore .
Fingir.
\section{Aparente}
\begin{itemize}
\item {Grp. gram.:adj.}
\end{itemize}
\begin{itemize}
\item {Proveniência:(Lat. \textunderscore apparens\textunderscore )}
\end{itemize}
Que aparece.
Evidente.
Semelhante.
Exterior.
Que só existe na aparencia.
\section{Aparentemente}
\begin{itemize}
\item {Grp. gram.:adv.}
\end{itemize}
De modo \textunderscore aparente\textunderscore .
\section{Aparição}
\begin{itemize}
\item {Grp. gram.:f.}
\end{itemize}
\begin{itemize}
\item {Proveniência:(Lat. \textunderscore apparitio\textunderscore )}
\end{itemize}
O mesmo que \textunderscore aparecimento\textunderscore .
Principio.
Fantasma: \textunderscore acredita em aparições\textunderscore .
\section{Apelação}
\begin{itemize}
\item {Grp. gram.:f.}
\end{itemize}
\begin{itemize}
\item {Proveniência:(Lat. \textunderscore appellatio\textunderscore )}
\end{itemize}
Recurso para tribunal superior.
Acto de recorrer ou de soccorrer-se.
\section{Apelação}
\textunderscore f. Ant.\textunderscore  (?)«\textunderscore ...nos chegamos tão perto dellas\textunderscore (embarcações), \textunderscore que lhe enxergamos toda a apelação dos remos e conhecemos que erão galeotas de Turcos\textunderscore ». \textunderscore Peregrinação\textunderscore , V.
\section{Apelamento}
\begin{itemize}
\item {Grp. gram.:m.}
\end{itemize}
(V. \textunderscore apelação\textunderscore ^1)
\section{Apelante}
\begin{itemize}
\item {Grp. gram.:adj.}
\end{itemize}
\begin{itemize}
\item {Proveniência:(Lat. \textunderscore appellans\textunderscore )}
\end{itemize}
Aquelle que apela.
Recorrente.
\section{Apelar}
\begin{itemize}
\item {Grp. gram.:v. t.}
\end{itemize}
\begin{itemize}
\item {Grp. gram.:V. i.}
\end{itemize}
\begin{itemize}
\item {Grp. gram.:V. p.}
\end{itemize}
\begin{itemize}
\item {Proveniência:(Lat. \textunderscore appellare\textunderscore )}
\end{itemize}
Invocar em soccorro.
Recorrer da decisão de um tribunal inferior para outro superior: \textunderscore o réu apelou da sentença\textunderscore .
Chamar-se, têr o nome de, têr por nome. Cf. Sousa Monteiro, \textunderscore Elog. do Lat.\textunderscore 
\section{Apelativamente}
\begin{itemize}
\item {Grp. gram.:adv.}
\end{itemize}
De modo \textunderscore apelativo\textunderscore .
\section{Apelativo}
\begin{itemize}
\item {Grp. gram.:adj.}
\end{itemize}
\begin{itemize}
\item {Utilização:Gram.}
\end{itemize}
\begin{itemize}
\item {Proveniência:(Lat. \textunderscore appellativus\textunderscore )}
\end{itemize}
Diz-se do nome que é commum aos indivíduos de uma espécie ou classe.
\section{Apelatório}
\begin{itemize}
\item {Grp. gram.:adj.}
\end{itemize}
\begin{itemize}
\item {Proveniência:(Lat. \textunderscore appellatorius\textunderscore )}
\end{itemize}
Relativo á apelação.
\section{Apelável}
\begin{itemize}
\item {Grp. gram.:adj.}
\end{itemize}
De que se póde \textunderscore apelar\textunderscore .
\section{Apelidação}
\begin{itemize}
\item {Grp. gram.:f.}
\end{itemize}
Acto de \textunderscore apelidar\textunderscore .
\section{Apelidar}
\begin{itemize}
\item {Grp. gram.:v. t.}
\end{itemize}
\begin{itemize}
\item {Utilização:Ant.}
\end{itemize}
\begin{itemize}
\item {Proveniência:(Lat. \textunderscore ad-pellitare\textunderscore , freq. de \textunderscore appellare\textunderscore )}
\end{itemize}
Designar por apelido.
Cognominar; nomear; alcunhar.
Apregoar.
Chamar em auxílio.
Convocar para a guerra. Cf. Hercul., \textunderscore Hist. de Port.\textunderscore , IV, 41.
\section{Apelido}
\begin{itemize}
\item {Grp. gram.:m.}
\end{itemize}
\begin{itemize}
\item {Utilização:Ant.}
\end{itemize}
Sobrenome.
Alcunha.
Designação particular de certas coisas.
Apêlo ás armas, ou obrigação de pegar em armas, quando se annunciasse a aproximação de inimigos. Cf. G. Henriques, \textunderscore Alenquer\textunderscore .
(B. lat. \textunderscore appellitus\textunderscore )
\section{Apêlo}
\begin{itemize}
\item {Grp. gram.:m.}
\end{itemize}
\begin{itemize}
\item {Proveniência:(De \textunderscore appellar\textunderscore )}
\end{itemize}
Apelação.
Chamamento; invocação.
Convite ou suggestão, para se prestar auxílio ou subsídio.
\section{Apendente}
\begin{itemize}
\item {Grp. gram.:adj.}
\end{itemize}
\begin{itemize}
\item {Utilização:Bot.}
\end{itemize}
\begin{itemize}
\item {Proveniência:(De \textunderscore appender\textunderscore )}
\end{itemize}
Diz-se do grão vegetal, quando o hilo, ao nivel da placenta ou pouco mais ou menos, está por baixo do ponto mais elevado do grão.
\section{Apender}
\begin{itemize}
\item {Grp. gram.:v. t.}
\end{itemize}
\begin{itemize}
\item {Proveniência:(Lat. \textunderscore appendere\textunderscore )}
\end{itemize}
O mesmo que \textunderscore apensar\textunderscore :«\textunderscore lembra Chateaubriand quando, no poema dos Mártyres, apende aos quadros vistossissimos as austeridades da nova lei\textunderscore ». Camillo.
\section{Apêndice}
\begin{itemize}
\item {Grp. gram.:m.}
\end{itemize}
\begin{itemize}
\item {Utilização:Anat.}
\end{itemize}
\begin{itemize}
\item {Proveniência:(Lat. \textunderscore appendix\textunderscore )}
\end{itemize}
Parte annexa a uma obra.
Accessório.
Accrescentamento.
Parte pendente ou dependente de outra.
Aquillo que nos animaes se não considera essencial ao seu organismo.
Prolongamento das flôres e das fôlhas, que acompanha o pedúnculo até quási a sua inserção na haste ou no ramo.
Parte adherente ou contínua de corpo a que não é essencial: \textunderscore apêndice esternal\textunderscore ; \textunderscore apêndice íleo-cecal\textunderscore .
\section{Apendiceado}
\begin{itemize}
\item {Grp. gram.:adj.}
\end{itemize}
Que tem apêndices.
\section{Apendiciforme}
\begin{itemize}
\item {Grp. gram.:adj.}
\end{itemize}
\begin{itemize}
\item {Proveniência:(Do lat. \textunderscore appendix\textunderscore  + \textunderscore forma\textunderscore )}
\end{itemize}
Que tem fórma de apêndice.
\section{Apendicite}
\begin{itemize}
\item {Grp. gram.:f.}
\end{itemize}
\begin{itemize}
\item {Utilização:Med.}
\end{itemize}
\begin{itemize}
\item {Proveniência:(De \textunderscore appêndice\textunderscore )}
\end{itemize}
Excrescência carnosa, vulgarmente conhecida por \textunderscore verrugas\textunderscore , \textunderscore cravos\textunderscore , etc.
Inflammação do apêndice íleo-cecal.
\section{Apendiculado}
\begin{itemize}
\item {Grp. gram.:adj.}
\end{itemize}
Que termina em apendículo.
\section{Apendicular}
\begin{itemize}
\item {Grp. gram.:adj.}
\end{itemize}
\begin{itemize}
\item {Proveniência:(De \textunderscore appendículo\textunderscore )}
\end{itemize}
Relativo a apêndice.
Que não é essencial ao todo, de que faz parte.
\section{Apendículo}
\begin{itemize}
\item {Grp. gram.:m.}
\end{itemize}
(dem. de \textunderscore apêndice\textunderscore )
\section{Apendigastro}
\begin{itemize}
\item {Grp. gram.:adj.}
\end{itemize}
Diz-se dos animaes que têm o abdome em fórma de apêndice.
\section{Apendix}
\begin{itemize}
\item {Grp. gram.:m.}
\end{itemize}
(V.apêndice)
\section{Apensa}
\begin{itemize}
\item {Grp. gram.:f.}
\end{itemize}
Acção e effeito de erguer novamente as varas de uma videira, por terem caído ou sido abaladas pela redra, pêso do fruto, fôrça dos ventos, etc.
(Fem. de \textunderscore appenso\textunderscore )
\section{Apensar}
\begin{itemize}
\item {Grp. gram.:v. t.}
\end{itemize}
\begin{itemize}
\item {Proveniência:(De \textunderscore apenso\textunderscore )}
\end{itemize}
Juntar; annexar; accrescentar: \textunderscore apensar um requerimento a um processo\textunderscore .
\section{Apenso}
\begin{itemize}
\item {Grp. gram.:m.}
\end{itemize}
\begin{itemize}
\item {Grp. gram.:Adj.}
\end{itemize}
\begin{itemize}
\item {Utilização:Ant.}
\end{itemize}
\begin{itemize}
\item {Proveniência:(Lat. \textunderscore appensus\textunderscore )}
\end{itemize}
Aquillo que se apensa.
Junto; annexo.
Pendente.
\section{Apertinente}
\begin{itemize}
\item {Grp. gram.:adj.}
\end{itemize}
Conciliador:«\textunderscore sois atilada e apertinente\textunderscore ». Garrett, \textunderscore Arco de Sant'Anna\textunderscore , I, 114.
\section{Apetecedor}
\begin{itemize}
\item {Grp. gram.:m.}
\end{itemize}
Aquelle ou aquillo que apetece ou que se apetece.
\section{Apetecer}
\begin{itemize}
\item {Grp. gram.:v. t.}
\end{itemize}
\begin{itemize}
\item {Grp. gram.:V. i.}
\end{itemize}
\begin{itemize}
\item {Proveniência:(Do lat. \textunderscore appetere\textunderscore )}
\end{itemize}
Têr apetite de.
Pretender.
Desejar: \textunderscore não apeteço a glória\textunderscore .
Causar apetite: \textunderscore êste guisado apetece\textunderscore .
\section{Apetecível}
\begin{itemize}
\item {Grp. gram.:adj.}
\end{itemize}
Digno de sêr apetecido.
\section{Apetência}
\begin{itemize}
\item {Grp. gram.:f.}
\end{itemize}
\begin{itemize}
\item {Proveniência:(Lat. \textunderscore appetentia\textunderscore )}
\end{itemize}
O mesmo que \textunderscore apetite\textunderscore .
\section{Apetente}
\begin{itemize}
\item {Grp. gram.:adj.}
\end{itemize}
\begin{itemize}
\item {Proveniência:(Lat. \textunderscore appetens\textunderscore )}
\end{itemize}
Que apetece.
\section{Apetibilidade}
\begin{itemize}
\item {Grp. gram.:f.}
\end{itemize}
\begin{itemize}
\item {Utilização:P. us.}
\end{itemize}
Qualidade do que é apetecível.
\section{Apetir}
\begin{itemize}
\item {Grp. gram.:v. t.}
\end{itemize}
\begin{itemize}
\item {Utilização:Ant.}
\end{itemize}
O mesmo que \textunderscore apetecer\textunderscore . Cf. \textunderscore Aulegrafia\textunderscore , 182.
\section{Apetitar}
\begin{itemize}
\item {Grp. gram.:v. t.}
\end{itemize}
\begin{itemize}
\item {Proveniência:(De \textunderscore appetite\textunderscore )}
\end{itemize}
Causar apetite a.
Tentar, cativar.
\section{Apetite}
\begin{itemize}
\item {Grp. gram.:m.}
\end{itemize}
\begin{itemize}
\item {Proveniência:(Lat. \textunderscore appetitus\textunderscore )}
\end{itemize}
Desejo.
Ambição.
Predilecção.
Sensualidade.
\section{Apetitível}
\begin{itemize}
\item {Grp. gram.:adj.}
\end{itemize}
\begin{itemize}
\item {Utilização:P. us.}
\end{itemize}
O mesmo que \textunderscore apetecível\textunderscore .
\section{Apetitivo}
\begin{itemize}
\item {Grp. gram.:adj.}
\end{itemize}
Que tem apetite.
Sensual.
Que desperta apetite.
\section{Apetito}
\begin{itemize}
\item {Grp. gram.:m.}
\end{itemize}
\begin{itemize}
\item {Utilização:Des.}
\end{itemize}
O mesmo que apetite. Cf. \textunderscore Lusíadas\textunderscore , X, 5; \textunderscore Eufrosina\textunderscore , 89 e 132.
\section{Apetitosamente}
\begin{itemize}
\item {Grp. gram.:adv.}
\end{itemize}
De modo \textunderscore apetitoso\textunderscore .
Despertando apetite.
\section{Apetitoso}
\begin{itemize}
\item {Grp. gram.:adj.}
\end{itemize}
Que desperta apetite.
Cubiçoso.
Que tem grande desejo.
Caprichoso.
Digno de sêr apetecido.
Supérfluo.
\section{Aplaudente}
\begin{itemize}
\item {Grp. gram.:adj.}
\end{itemize}
\begin{itemize}
\item {Proveniência:(Lat. \textunderscore applaudens\textunderscore )}
\end{itemize}
Que aplaude.
\section{Apostolicamente}
\begin{itemize}
\item {Grp. gram.:adv.}
\end{itemize}
De modo apostólico.
Á maneira de apóstolo.
\section{Apostolicidade}
\begin{itemize}
\item {Grp. gram.:f.}
\end{itemize}
\begin{itemize}
\item {Proveniência:(De \textunderscore apostólico\textunderscore )}
\end{itemize}
Conformidade de doutrina com a dos apóstolos.
\section{Apostólico}
\begin{itemize}
\item {Grp. gram.:adj.}
\end{itemize}
\begin{itemize}
\item {Grp. gram.:M.}
\end{itemize}
Relativo aos apóstolos.
Procedente dos Apóstolos.
Que depende da Santa-Sé, ou é relativo a ella.
Papa; Pontífice. Cf. \textunderscore Port. Mon. Hist.\textunderscore , \textunderscore Script.\textunderscore , 246.
\section{Apostolinos}
\begin{itemize}
\item {Grp. gram.:m. pl.}
\end{itemize}
Ordem religiosa de Gênova, supprimida por Xisto V.
\section{Apostolizar}
\begin{itemize}
\item {Grp. gram.:v. t.}
\end{itemize}
O mesmo que \textunderscore apostolar\textunderscore .
\section{Apóstolo}
\begin{itemize}
\item {Grp. gram.:m.}
\end{itemize}
\begin{itemize}
\item {Grp. gram.:M. pl.}
\end{itemize}
\begin{itemize}
\item {Utilização:Jur.}
\end{itemize}
Cada um dos doze discípulos de Jesus.
Aquelle que primeiro prègou o Christianismo num país.
O propagador de uma doutrina.
Missionário exemplar.
Certos herejes que, inculcando virtude e abnegação, usavam hábito de frades e se entregavam a todos os vícios.
Dimissórias, dadas pelos Bispos aos seus diocesanos.
Delegados, embaixadores, núncios.
Certidão authêntica de appellação ou recurso.
\section{Apostrofar}
\begin{itemize}
\item {Grp. gram.:v. t.}
\end{itemize}
Interromper com apóstrofe.
Dirigir apóstrofes a.
\section{Apostrofar}
\begin{itemize}
\item {Grp. gram.:v.}
\end{itemize}
\begin{itemize}
\item {Utilização:t. Gram.}
\end{itemize}
Pôr apóstrofo em.
\section{Apóstrofe}
\begin{itemize}
\item {Grp. gram.:f.}
\end{itemize}
\begin{itemize}
\item {Utilização:Rhet.}
\end{itemize}
\begin{itemize}
\item {Proveniência:(Gr. \textunderscore apostrophe\textunderscore )}
\end{itemize}
Interrupção, que o orador faz, dirigindo-se a coisas ou pessôas, reaes ou ficticias.
Interpellação directa e imprevista.
\section{Apóstrofe}
\begin{itemize}
\item {Grp. gram.:f.}
\end{itemize}
\begin{itemize}
\item {Proveniência:(Gr. \textunderscore apostrophos\textunderscore )}
\end{itemize}
Sinal gráphico, que na escrita designa elisão de letra ou letras.
\section{Apóstrofo}
\begin{itemize}
\item {Grp. gram.:m.}
\end{itemize}
\begin{itemize}
\item {Proveniência:(Gr. \textunderscore apostrophos\textunderscore )}
\end{itemize}
Sinal gráphico, que na escrita designa elisão de letra ou letras.
\section{Apostrophar}
\begin{itemize}
\item {Grp. gram.:v. t.}
\end{itemize}
Interromper com apóstrophe.
Dirigir apóstrophes a.
\section{Apostrophar}
\begin{itemize}
\item {Grp. gram.:v.}
\end{itemize}
\begin{itemize}
\item {Utilização:t. Gram.}
\end{itemize}
Pôr apóstropho em.
\section{Apóstrophe}
\begin{itemize}
\item {Grp. gram.:f.}
\end{itemize}
\begin{itemize}
\item {Utilização:Rhet.}
\end{itemize}
\begin{itemize}
\item {Proveniência:(Gr. \textunderscore apostrophe\textunderscore )}
\end{itemize}
Interrupção, que o orador faz, dirigindo-se a coisas ou pessôas, reaes ou ficticias.
Interpellação directa e imprevista.
\section{Apóstrophe}
\begin{itemize}
\item {Grp. gram.:f.}
\end{itemize}
\begin{itemize}
\item {Proveniência:(Gr. \textunderscore apostrophos\textunderscore )}
\end{itemize}
Sinal gráphico, que na escrita designa elisão de letra ou letras.
\section{Apóstropho}
\begin{itemize}
\item {Grp. gram.:m.}
\end{itemize}
\begin{itemize}
\item {Proveniência:(Gr. \textunderscore apostrophos\textunderscore )}
\end{itemize}
Sinal gráphico, que na escrita designa elisão de letra ou letras.
\section{Apostura}
\begin{itemize}
\item {Grp. gram.:f.}
\end{itemize}
O mesmo ou melhor que \textunderscore atitude\textunderscore . Cf. San-Luís, \textunderscore Glossario\textunderscore .
\section{Aposturas}
\begin{itemize}
\item {Grp. gram.:f. pl.}
\end{itemize}
\begin{itemize}
\item {Utilização:Ant.}
\end{itemize}
\begin{itemize}
\item {Proveniência:(De \textunderscore postura\textunderscore )}
\end{itemize}
Peças das balisas e madeiras que formam o costado do navio, para cima da cinta.
\section{Apotas}
\begin{itemize}
\item {Grp. gram.:m. pl.}
\end{itemize}
Indígenas do alto Amazonas.
\section{Apoteca}
\begin{itemize}
\item {Grp. gram.:f.}
\end{itemize}
\begin{itemize}
\item {Utilização:Ant.}
\end{itemize}
\begin{itemize}
\item {Proveniência:(Gr. \textunderscore apotheke\textunderscore )}
\end{itemize}
Corpo fructifero, formado nos lichens por um receptáculo e pelos órgãos reproductores.
Despensa ou depósito de gêneros alimentícios, especialmente de vinho.
\section{Apotécio}
\begin{itemize}
\item {Grp. gram.:m.}
\end{itemize}
(V.apoteca)
\section{Apotelesma}
\begin{itemize}
\item {Grp. gram.:f.}
\end{itemize}
\begin{itemize}
\item {Utilização:Med.}
\end{itemize}
Terminação de uma doença.
\section{Apotelesmática}
\begin{itemize}
\item {Grp. gram.:f.}
\end{itemize}
O mesmo que \textunderscore astrologia\textunderscore .
\section{Apotelesmático}
\begin{itemize}
\item {Grp. gram.:adj.}
\end{itemize}
Relativo á apotelesmática.
\section{Apotema}
\begin{itemize}
\item {Grp. gram.:m.}
\end{itemize}
\begin{itemize}
\item {Utilização:Geom.}
\end{itemize}
\begin{itemize}
\item {Utilização:Chím.}
\end{itemize}
\begin{itemize}
\item {Proveniência:(Do gr. \textunderscore apo\textunderscore  + \textunderscore titheme\textunderscore )}
\end{itemize}
Linha perpendicular, tirada do centro para qualquer lado de um polýgono regular.
Precipitado escuro, que se fórma na dissolução dos extractos vegetaes.
\section{Apotentado}
\begin{itemize}
\item {Grp. gram.:adj.}
\end{itemize}
O mesmo que \textunderscore poderoso\textunderscore :«\textunderscore ...hum rei mui rico e apotentado\textunderscore ». Filinto, \textunderscore D. Man.\textunderscore , I, 229.
\section{Apoteóse}
\begin{itemize}
\item {Grp. gram.:f.}
\end{itemize}
\begin{itemize}
\item {Proveniência:(Gr. \textunderscore apotheosis\textunderscore )}
\end{itemize}
Acção de collocar alguém em o número dos deuses.
Glorificação.
Elogio extraordinário.
\section{Apoteótico}
\begin{itemize}
\item {Grp. gram.:adj.}
\end{itemize}
Que contém apoteóse.
Muito elogioso.
\section{Apoterapia}
\begin{itemize}
\item {Grp. gram.:f.}
\end{itemize}
O mesmo que \textunderscore terapêutica\textunderscore .
\section{Apótese}
\begin{itemize}
\item {Grp. gram.:f.}
\end{itemize}
\begin{itemize}
\item {Utilização:Cir.}
\end{itemize}
\begin{itemize}
\item {Proveniência:(Gr. \textunderscore apothesis\textunderscore )}
\end{itemize}
Posição, que deve dar-se a um membro fracturado, depois de ligada a fractura.
\section{Apotheca}
\begin{itemize}
\item {Grp. gram.:f.}
\end{itemize}
\begin{itemize}
\item {Utilização:Ant.}
\end{itemize}
\begin{itemize}
\item {Proveniência:(Gr. \textunderscore apotheke\textunderscore )}
\end{itemize}
Corpo fructifero, formado nos lichens por um receptáculo e pelos órgãos reproductores.
Despensa ou depósito de gêneros alimentícios, especialmente de vinho.
\section{Apothécio}
\begin{itemize}
\item {Grp. gram.:m.}
\end{itemize}
(V.apotheca)
\section{Apothema}
\begin{itemize}
\item {Grp. gram.:m.}
\end{itemize}
\begin{itemize}
\item {Utilização:Geom.}
\end{itemize}
\begin{itemize}
\item {Utilização:Chím.}
\end{itemize}
\begin{itemize}
\item {Proveniência:(Do gr. \textunderscore apo\textunderscore  + \textunderscore titheme\textunderscore )}
\end{itemize}
Linha perpendicular, tirada do centro para qualquer lado de um polýgono regular.
Precipitado escuro, que se fórma na dissolução dos extractos vegetaes.
\section{Apotheóse}
\begin{itemize}
\item {Grp. gram.:f.}
\end{itemize}
\begin{itemize}
\item {Proveniência:(Gr. \textunderscore apotheosis\textunderscore )}
\end{itemize}
Acção de collocar alguém em o número dos deuses.
Glorificação.
Elogio extraordinário.
\section{Apotheótico}
\begin{itemize}
\item {Grp. gram.:adj.}
\end{itemize}
Que contém apotheóse.
Muito elogioso.
\section{Apotherapia}
\begin{itemize}
\item {Grp. gram.:f.}
\end{itemize}
O mesmo que \textunderscore therapêutica\textunderscore .
\section{Apóthese}
\begin{itemize}
\item {Grp. gram.:f.}
\end{itemize}
\begin{itemize}
\item {Utilização:Cir.}
\end{itemize}
\begin{itemize}
\item {Proveniência:(Gr. \textunderscore apothesis\textunderscore )}
\end{itemize}
Posição, que deve dar-se a um membro fracturado, depois de ligada a fractura.
\section{Apotiacorava}
\begin{itemize}
\item {Grp. gram.:f.}
\end{itemize}
Planta euphorbiácea do Pará.
\section{Apótomo}
\begin{itemize}
\item {Grp. gram.:m.}
\end{itemize}
\begin{itemize}
\item {Utilização:Mús.}
\end{itemize}
\begin{itemize}
\item {Utilização:Mathem.}
\end{itemize}
Intervallo entre dois tons.
Differença entre duas quantidades incommensuráveis.
\section{Apotos}
\begin{itemize}
\item {Grp. gram.:m. pl.}
\end{itemize}
O mesmo que \textunderscore Apotas\textunderscore .
\section{Apoucadamente}
\begin{itemize}
\item {Grp. gram.:adv.}
\end{itemize}
Com apoucamento.
\section{Apoucado}
\begin{itemize}
\item {Grp. gram.:adj.}
\end{itemize}
Pouco desenvolvido.
Tímido.
Imbecil.
\section{Apoucador}
\begin{itemize}
\item {Grp. gram.:m.}
\end{itemize}
Aquelle que apouca.
\section{Apoucamento}
\begin{itemize}
\item {Grp. gram.:m.}
\end{itemize}
Acto de \textunderscore apoucar\textunderscore .
\section{Apoucar}
\begin{itemize}
\item {Grp. gram.:v. t.}
\end{itemize}
Reduzir a pouco; deminuir.
Rebaixar.
Amesquinhar; desdenhar: \textunderscore apoucar o talento de alguém\textunderscore .
\section{Apouquentar}
\textunderscore v. t.\textunderscore  (e der.)
O mesmo que \textunderscore apoquentar\textunderscore , etc.
\section{Apousentar}
\begin{itemize}
\item {Grp. gram.:v. t.}
\end{itemize}
\begin{itemize}
\item {Utilização:Ant.}
\end{itemize}
(V.aposentar)
\section{Apoutar}
\begin{itemize}
\item {Grp. gram.:v. t.}
\end{itemize}
(V.poutar)
\section{Apózema}
\begin{itemize}
\item {Grp. gram.:f.}
\end{itemize}
\begin{itemize}
\item {Proveniência:(Gr. \textunderscore apozema\textunderscore )}
\end{itemize}
Cozimento de substâncias vegetaes, a que se juntam outras substâncias que o clarificam e adoçam.
\section{Apparatar}
\begin{itemize}
\item {Grp. gram.:v. t.}
\end{itemize}
\begin{itemize}
\item {Proveniência:(De \textunderscore apparato\textunderscore )}
\end{itemize}
Tornar apparatoso.
Adornar.
\section{Apparato}
\begin{itemize}
\item {Grp. gram.:m.}
\end{itemize}
Apresentação pomposa.
Preparação para festa.
Esplendor; magnificência; luxo.
Apresto.
Reunião de elementos para uma composição.
Apparelho, em que está o puado, nas officinas de cardagem. Cf. \textunderscore Inquér. Indust.\textunderscore , p. II, l. 111, 67.
\section{Apparatosamente}
\begin{itemize}
\item {Grp. gram.:adv.}
\end{itemize}
De modo \textunderscore apparatoso\textunderscore .
Com apparato.
\section{Apparatoso}
\begin{itemize}
\item {Grp. gram.:adj.}
\end{itemize}
Em que há apparato.
Magnificente.
\section{Apparecente}
\begin{itemize}
\item {Grp. gram.:adj.}
\end{itemize}
Que começa a apparecer.
\section{Apparecer}
\begin{itemize}
\item {Grp. gram.:v. i.}
\end{itemize}
\begin{itemize}
\item {Proveniência:(Do lat. \textunderscore apparere\textunderscore )}
\end{itemize}
Fazer-se vêr.
Apresentar-se; comparecer: \textunderscore não pude apparecer na assembleia\textunderscore .
Revelar-se.
Acontecer.
\section{Apparecido}
\begin{itemize}
\item {Grp. gram.:adj.}
\end{itemize}
Que appareceu.
\section{Apparecimento}
\begin{itemize}
\item {Grp. gram.:m.}
\end{itemize}
Acto de \textunderscore apparecer\textunderscore .
\section{Apparelhadamente}
\begin{itemize}
\item {Grp. gram.:adv.}
\end{itemize}
De modo \textunderscore apparelhado\textunderscore .
Com preparação.
\section{Apparelhado}
\begin{itemize}
\item {Grp. gram.:adv.}
\end{itemize}
\begin{itemize}
\item {Utilização:Prov.}
\end{itemize}
\begin{itemize}
\item {Utilização:minh.}
\end{itemize}
Preparado, disposto: \textunderscore apparelhado para combate\textunderscore .
Destinado.
Enfeitado; arreado: \textunderscore um cavallo apparelhado\textunderscore .
Concertado.
Abastecido: \textunderscore um cesto apparelhado\textunderscore .
\section{Apparelhador}
\begin{itemize}
\item {Grp. gram.:m.}
\end{itemize}
Aquelle que apparelha.
O encarregado de certas obras, immediatamente inferior ao architecto ou ao mestre.
\section{Apparelhamento}
\begin{itemize}
\item {Grp. gram.:m.}
\end{itemize}
Acto ou effeito de \textunderscore apparelhar\textunderscore .
Apparelho.
\section{Apparelhar}
\begin{itemize}
\item {Grp. gram.:v. t.}
\end{itemize}
\begin{itemize}
\item {Proveniência:(De \textunderscore apparelho\textunderscore )}
\end{itemize}
Preparar.
Tornar disposto; aprestar.
Pôr arreios em (cavalgaduras).
Desbastar (madeira ou pedra) para certas obras.
Enfeitar.
Concertar.
\section{Apparelho}
\begin{itemize}
\item {fónica:pa-rê}
\end{itemize}
\begin{itemize}
\item {Grp. gram.:m.}
\end{itemize}
\begin{itemize}
\item {Utilização:Náut.}
\end{itemize}
\begin{itemize}
\item {Proveniência:(De um lat. hypoth. \textunderscore appariculum\textunderscore ? Cp. \textunderscore parelho\textunderscore , fr. \textunderscore pareil\textunderscore )}
\end{itemize}
Acto de apparelhar.
Preparo.
Arreios de cavalgadura.
Conjunto de utensílios náuticos.
Aprestos béllicos.
Baixella; alfaias; peças do serviço culinário.
Máchina para levantar pesos.
Primeira camada de óleo no pano que se vai pintar.
Conjunto dos objectos necessários para uma operação cirúrgica.
Corda com anzoes, que se atravessa no ponto do rio, por onde se espera que o peixe passe.
Conjunto dos órgãos que, num corpo organizado, cooperam para a mesma funcção. Desbaste ou córte de pedras para revestimento de cantarias.
Trem de lavoira; apeiro.
Conjunto dos mastros, paus, mastaréus e respectivas vêrgas, pano, cabos fixos e cabos de laborar, próprios de uma embarcação, e pelos quaes se determina a classificação do navio: \textunderscore apparelho de escuna\textunderscore ; \textunderscore apparelho de brigue\textunderscore .
\section{Apparência}
\begin{itemize}
\item {Grp. gram.:f.}
\end{itemize}
\begin{itemize}
\item {Proveniência:(Lat. \textunderscore apparentia\textunderscore )}
\end{itemize}
Aquillo que se mostra á primeira vista; exterioridade; aspecto: \textunderscore apparência de bôa pessôa\textunderscore .
Probabilidade.
Fingimento; disfarce.
\section{Apparentar}
\begin{itemize}
\item {Grp. gram.:v. t.}
\end{itemize}
\begin{itemize}
\item {Proveniência:(De \textunderscore apparente\textunderscore )}
\end{itemize}
Dar apparência de: \textunderscore apparentar innocência\textunderscore .
Fingir.
\section{Apparente}
\begin{itemize}
\item {Grp. gram.:adj.}
\end{itemize}
\begin{itemize}
\item {Proveniência:(Lat. \textunderscore apparens\textunderscore )}
\end{itemize}
Que apparece.
Evidente.
Semelhante.
Exterior.
Que só existe na apparencia.
\section{Apparentemente}
\begin{itemize}
\item {Grp. gram.:adv.}
\end{itemize}
De modo \textunderscore apparente\textunderscore .
\section{Apparição}
\begin{itemize}
\item {Grp. gram.:f.}
\end{itemize}
\begin{itemize}
\item {Proveniência:(Lat. \textunderscore apparitio\textunderscore )}
\end{itemize}
O mesmo que \textunderscore apparecimento\textunderscore .
Principio.
Fantasma: \textunderscore acredita em apparições\textunderscore .
\section{Appellação}
\begin{itemize}
\item {Grp. gram.:f.}
\end{itemize}
\begin{itemize}
\item {Proveniência:(Lat. \textunderscore appellatio\textunderscore )}
\end{itemize}
Recurso para tribunal superior.
Acto de recorrer ou de soccorrer-se.
\section{Appellação}
\textunderscore f. Ant.\textunderscore  (?)«\textunderscore ...nos chegamos tão perto dellas\textunderscore  (embarcações), \textunderscore que lhe enxergamos toda a apellação dos remos e conhecemos que erão galeotas de Turcos\textunderscore ». \textunderscore Peregrinação\textunderscore , V.
\section{Appellamento}
\begin{itemize}
\item {Grp. gram.:m.}
\end{itemize}
(V. \textunderscore appellação\textunderscore ^1)
\section{Appellante}
\begin{itemize}
\item {Grp. gram.:adj.}
\end{itemize}
\begin{itemize}
\item {Proveniência:(Lat. \textunderscore appellans\textunderscore )}
\end{itemize}
Aquelle que appella.
Recorrente.
\section{Appellar}
\begin{itemize}
\item {Grp. gram.:v. t.}
\end{itemize}
\begin{itemize}
\item {Grp. gram.:V. i.}
\end{itemize}
\begin{itemize}
\item {Grp. gram.:V. p.}
\end{itemize}
\begin{itemize}
\item {Proveniência:(Lat. \textunderscore appellare\textunderscore )}
\end{itemize}
Invocar em soccorro.
Recorrer da decisão de um tribunal inferior para outro superior: \textunderscore o réu appellou da sentença\textunderscore .
Chamar-se, têr o nome de, têr por nome. Cf. Sousa Monteiro, \textunderscore Elog. do Lat.\textunderscore 
\section{Appellativamente}
\begin{itemize}
\item {Grp. gram.:adv.}
\end{itemize}
De modo \textunderscore appellativo\textunderscore .
\section{Appellativo}
\begin{itemize}
\item {Grp. gram.:adj.}
\end{itemize}
\begin{itemize}
\item {Utilização:Gram.}
\end{itemize}
\begin{itemize}
\item {Proveniência:(Lat. \textunderscore appellativus\textunderscore )}
\end{itemize}
Diz-se do nome que é commum aos indivíduos de uma espécie ou classe.
\section{Appellatório}
\begin{itemize}
\item {Grp. gram.:adj.}
\end{itemize}
\begin{itemize}
\item {Proveniência:(Lat. \textunderscore appellatorius\textunderscore )}
\end{itemize}
Relativo á appellação.
\section{Appellável}
\begin{itemize}
\item {Grp. gram.:adj.}
\end{itemize}
De que se póde \textunderscore appellar\textunderscore .
\section{Appellidação}
\begin{itemize}
\item {Grp. gram.:f.}
\end{itemize}
Acto de \textunderscore appellidar\textunderscore .
\section{Appellidar}
\begin{itemize}
\item {Grp. gram.:v. t.}
\end{itemize}
\begin{itemize}
\item {Utilização:Ant.}
\end{itemize}
\begin{itemize}
\item {Proveniência:(Lat. \textunderscore ad-pellitare\textunderscore , freq. de \textunderscore appellare\textunderscore )}
\end{itemize}
Designar por appellido.
Cognominar; nomear; alcunhar.
Apregoar.
Chamar em auxílio.
Convocar para a guerra. Cf. Hercul., \textunderscore Hist. de Port.\textunderscore , IV, 41.
\section{Appellido}
\begin{itemize}
\item {Grp. gram.:m.}
\end{itemize}
\begin{itemize}
\item {Utilização:Ant.}
\end{itemize}
Sobrenome.
Alcunha.
Designação particular de certas coisas.
Appêllo ás armas, ou obrigação de pegar em armas, quando se annunciasse a aproximação de inimigos. Cf. G. Henriques, \textunderscore Alenquer\textunderscore .
(B. lat. \textunderscore appellitus\textunderscore )
\section{Appêllo}
\begin{itemize}
\item {Grp. gram.:m.}
\end{itemize}
\begin{itemize}
\item {Proveniência:(De \textunderscore appellar\textunderscore )}
\end{itemize}
Appellação.
Chamamento; invocação.
Convite ou suggestão, para se prestar auxílio ou subsídio.
\section{Appendente}
\begin{itemize}
\item {Grp. gram.:adj.}
\end{itemize}
\begin{itemize}
\item {Utilização:Bot.}
\end{itemize}
\begin{itemize}
\item {Proveniência:(De \textunderscore appender\textunderscore )}
\end{itemize}
Diz-se do grão vegetal, quando o hilo, ao nivel da placenta ou pouco mais ou menos, está por baixo do ponto mais elevado do grão.
\section{Appender}
\begin{itemize}
\item {Grp. gram.:v. t.}
\end{itemize}
\begin{itemize}
\item {Proveniência:(Lat. \textunderscore appendere\textunderscore )}
\end{itemize}
O mesmo que \textunderscore appensar\textunderscore :«\textunderscore lembra Chateaubriand, quando, no poema dos Mártyres, appende aos quadros vistosissimos as austeridades da nova lei\textunderscore ». Camillo.
\section{Appêndice}
\begin{itemize}
\item {Grp. gram.:m.}
\end{itemize}
\begin{itemize}
\item {Utilização:Anat.}
\end{itemize}
\begin{itemize}
\item {Proveniência:(Lat. \textunderscore appendix\textunderscore )}
\end{itemize}
Parte annexa a uma obra.
Accessório.
Accrescentamento.
Parte pendente ou dependente de outra.
Aquillo que nos animaes se não considera essencial ao seu organismo.
Prolongamento das flôres e das fôlhas, que acompanha o pedúnculo até quási a sua inserção na haste ou no ramo.
Parte adherente ou contínua de corpo a que não é essencial: \textunderscore appêndice esternal\textunderscore ; \textunderscore appêndice íleo-cecal\textunderscore .
\section{Appendiceado}
\begin{itemize}
\item {Grp. gram.:adj.}
\end{itemize}
Que tem appêndices.
\section{Appendiciforme}
\begin{itemize}
\item {Grp. gram.:adj.}
\end{itemize}
\begin{itemize}
\item {Proveniência:(Do lat. \textunderscore appendix\textunderscore  + \textunderscore forma\textunderscore )}
\end{itemize}
Que tem fórma de appêndice.
\section{Appendicite}
\begin{itemize}
\item {Grp. gram.:f.}
\end{itemize}
\begin{itemize}
\item {Utilização:Med.}
\end{itemize}
\begin{itemize}
\item {Proveniência:(De \textunderscore appêndice\textunderscore )}
\end{itemize}
Excrescência carnosa, vulgarmente conhecida por \textunderscore verrugas\textunderscore , \textunderscore cravos\textunderscore , etc.
Inflammação do appêndice íleo-cecal.
\section{Appendiculado}
\begin{itemize}
\item {Grp. gram.:adj.}
\end{itemize}
Que termina em appendículo.
\section{Appendicular}
\begin{itemize}
\item {Grp. gram.:adj.}
\end{itemize}
\begin{itemize}
\item {Proveniência:(De \textunderscore appendículo\textunderscore )}
\end{itemize}
Relativo a appêndice.
Que não é essencial ao todo, de que faz parte.
\section{Appendículo}
\begin{itemize}
\item {Grp. gram.:m.}
\end{itemize}
(dem. de \textunderscore appêndice\textunderscore )
\section{Appendigastro}
\begin{itemize}
\item {Grp. gram.:adj.}
\end{itemize}
Diz-se dos animaes que têm o abdome em fórma de appêndice.
\section{Appendix}
\begin{itemize}
\item {Grp. gram.:m.}
\end{itemize}
(V.appêndice)
\section{Appensa}
\begin{itemize}
\item {Grp. gram.:f.}
\end{itemize}
Acção e effeito de erguer novamente as varas de uma videira, por terem caído ou sido abaladas pela redra, pêso do fruto, fôrça dos ventos, etc.
(Fem. de \textunderscore appenso\textunderscore )
\section{Appensar}
\begin{itemize}
\item {Grp. gram.:v. t.}
\end{itemize}
\begin{itemize}
\item {Proveniência:(De \textunderscore apenso\textunderscore )}
\end{itemize}
Juntar; annexar; accrescentar: \textunderscore appensar um requerimento a um processo\textunderscore .
\section{Appenso}
\begin{itemize}
\item {Grp. gram.:m.}
\end{itemize}
\begin{itemize}
\item {Grp. gram.:Adj.}
\end{itemize}
\begin{itemize}
\item {Utilização:Ant.}
\end{itemize}
\begin{itemize}
\item {Proveniência:(Lat. \textunderscore appensus\textunderscore )}
\end{itemize}
Aquillo que se appensa.
Junto; annexo.
Pendente.
\section{Appertinente}
\begin{itemize}
\item {Grp. gram.:adj.}
\end{itemize}
Conciliador:«\textunderscore sois atilada e appertinente\textunderscore ». Garrett, \textunderscore Arco de Sant'Anna\textunderscore , I, 114.
\section{Appetecedor}
\begin{itemize}
\item {Grp. gram.:m.}
\end{itemize}
Aquelle ou aquillo que appetece ou que se appetece.
\section{Appetecer}
\begin{itemize}
\item {Grp. gram.:v. t.}
\end{itemize}
\begin{itemize}
\item {Grp. gram.:V. i.}
\end{itemize}
\begin{itemize}
\item {Proveniência:(Do lat. \textunderscore appetere\textunderscore )}
\end{itemize}
Têr appetite de.
Pretender.
Desejar: \textunderscore não appeteço a glória\textunderscore .
Causar appetite: \textunderscore êste guisado appetece\textunderscore .
\section{Appetecível}
\begin{itemize}
\item {Grp. gram.:adj.}
\end{itemize}
Digno de sêr appetecido.
\section{Appetência}
\begin{itemize}
\item {Grp. gram.:f.}
\end{itemize}
\begin{itemize}
\item {Proveniência:(Lat. \textunderscore appetentia\textunderscore )}
\end{itemize}
O mesmo que \textunderscore appetite\textunderscore .
\section{Appetente}
\begin{itemize}
\item {Grp. gram.:adj.}
\end{itemize}
\begin{itemize}
\item {Proveniência:(Lat. \textunderscore appetens\textunderscore )}
\end{itemize}
Que appetece.
\section{Appetibilidade}
\begin{itemize}
\item {Grp. gram.:f.}
\end{itemize}
\begin{itemize}
\item {Utilização:P. us.}
\end{itemize}
Qualidade do que é appetecível.
\section{Appetir}
\begin{itemize}
\item {Grp. gram.:v. t.}
\end{itemize}
\begin{itemize}
\item {Utilização:Ant.}
\end{itemize}
O mesmo que \textunderscore appetecer\textunderscore . Cf. \textunderscore Aulegrafia\textunderscore , 182.
\section{Appetitar}
\begin{itemize}
\item {Grp. gram.:v. t.}
\end{itemize}
\begin{itemize}
\item {Proveniência:(De \textunderscore appetite\textunderscore )}
\end{itemize}
Causar appetite a.
Tentar, cativar.
\section{Appetite}
\begin{itemize}
\item {Grp. gram.:m.}
\end{itemize}
\begin{itemize}
\item {Proveniência:(Lat. \textunderscore appetitus\textunderscore )}
\end{itemize}
Desejo.
Ambição.
Predilecção.
Sensualidade.
\section{Appetitível}
\begin{itemize}
\item {Grp. gram.:adj.}
\end{itemize}
\begin{itemize}
\item {Utilização:P. us.}
\end{itemize}
O mesmo que \textunderscore appetecível\textunderscore .
\section{Appetitivo}
\begin{itemize}
\item {Grp. gram.:adj.}
\end{itemize}
Que tem appetite.
Sensual.
Que desperta appetite.
\section{Appetito}
\begin{itemize}
\item {Grp. gram.:m.}
\end{itemize}
\begin{itemize}
\item {Utilização:Des.}
\end{itemize}
O mesmo que appetite. Cf. \textunderscore Lusíadas\textunderscore , X, 5; \textunderscore Eufrosina\textunderscore , 89 e 132.
\section{Appetitosamente}
\begin{itemize}
\item {Grp. gram.:adv.}
\end{itemize}
De modo \textunderscore appetitoso\textunderscore .
Despertando appetite.
\section{Appetitoso}
\begin{itemize}
\item {Grp. gram.:adj.}
\end{itemize}
Que desperta appetite.
Cubiçoso.
Que tem grande desejo.
Caprichoso.
Digno de sêr appetecido.
Supérfluo.
\section{Applaudente}
\begin{itemize}
\item {Grp. gram.:adj.}
\end{itemize}
\begin{itemize}
\item {Proveniência:(Lat. \textunderscore applaudens\textunderscore )}
\end{itemize}
Que applaude.
\section{Aplaudidamente}
\begin{itemize}
\item {Grp. gram.:adv.}
\end{itemize}
Com aplauso.
\section{Aplaudidor}
\begin{itemize}
\item {Grp. gram.:m.  e  adj.}
\end{itemize}
O que aplaude.
\section{Aplaudir}
\begin{itemize}
\item {Grp. gram.:v. t.}
\end{itemize}
\begin{itemize}
\item {Grp. gram.:V. i.}
\end{itemize}
\begin{itemize}
\item {Proveniência:(Lat. \textunderscore applaudere\textunderscore )}
\end{itemize}
Dar aplauso a.
Louvar; elogiar.
Bater palmas, festejando: \textunderscore da plateia, ninguém aplaudiu\textunderscore .
\section{Aplausível}
\begin{itemize}
\item {Grp. gram.:adj.}
\end{itemize}
Plausível; digno de aplauso.
\section{Aplauso}
\begin{itemize}
\item {Grp. gram.:m.}
\end{itemize}
\begin{itemize}
\item {Proveniência:(Lat. \textunderscore applausus\textunderscore )}
\end{itemize}
Acto de aplaudir.
Acclamação.
Elogio público.
Louvor.
Demonstração alegre e ruidosa de approvação: \textunderscore o orador recebeu muitos aplausos\textunderscore .
\section{Aplicabilidade}
\begin{itemize}
\item {Grp. gram.:f.}
\end{itemize}
Qualidade do que é \textunderscore aplicável\textunderscore .
\section{Aplicação}
\begin{itemize}
\item {Grp. gram.:f.}
\end{itemize}
\begin{itemize}
\item {Proveniência:(Lat. \textunderscore applicatio\textunderscore )}
\end{itemize}
Acto ou effeito de aplicar.
Adaptação.
Destino.
Concentração de espirito: \textunderscore estudar com aplicação\textunderscore .
Obra de passamanaria.
\section{Aplicadamente}
\begin{itemize}
\item {Grp. gram.:adv.}
\end{itemize}
Com aplicação.
\section{Aplicando}
\begin{itemize}
\item {Grp. gram.:adj.}
\end{itemize}
O mesmo que \textunderscore aplicável\textunderscore .
\section{Aplicante}
\begin{itemize}
\item {Grp. gram.:adj.}
\end{itemize}
\begin{itemize}
\item {Proveniência:(Lat. \textunderscore applicans\textunderscore )}
\end{itemize}
Que aplica.
\section{Aplicar}
\begin{itemize}
\item {Grp. gram.:v. t.}
\end{itemize}
\begin{itemize}
\item {Proveniência:(Lat. \textunderscore applicare\textunderscore )}
\end{itemize}
Juntar; adaptar; sobrepor: \textunderscore aplicar uma cataplasma\textunderscore .
Empregar: \textunderscore aplicar mal o tempo\textunderscore .
Receitar: \textunderscore aplicar um xarope\textunderscore .
Realizar.
Infligir: \textunderscore o juiz aplicou-lhe pena leve\textunderscore .
\section{Aplicativo}
\begin{itemize}
\item {Grp. gram.:adj.}
\end{itemize}
O mesmo que \textunderscore aplicável\textunderscore .
\section{Aplicável}
\begin{itemize}
\item {Grp. gram.:adj.}
\end{itemize}
\begin{itemize}
\item {Proveniência:(De \textunderscore applicar\textunderscore )}
\end{itemize}
Que póde sêr aplicado.
\section{Applaudidamente}
\begin{itemize}
\item {Grp. gram.:adv.}
\end{itemize}
Com applauso.
\section{Applaudidor}
\begin{itemize}
\item {Grp. gram.:m.  e  adj.}
\end{itemize}
O que applaude.
\section{Applaudir}
\begin{itemize}
\item {Grp. gram.:v. t.}
\end{itemize}
\begin{itemize}
\item {Grp. gram.:V. i.}
\end{itemize}
\begin{itemize}
\item {Proveniência:(Lat. \textunderscore applaudere\textunderscore )}
\end{itemize}
Dar applauso a.
Louvar; elogiar.
Bater palmas, festejando: \textunderscore da plateia, ninguém applaudiu\textunderscore .
\section{Applausível}
\begin{itemize}
\item {Grp. gram.:adj.}
\end{itemize}
Plausível; digno de applauso.
\section{Applauso}
\begin{itemize}
\item {Grp. gram.:m.}
\end{itemize}
\begin{itemize}
\item {Proveniência:(Lat. \textunderscore applausus\textunderscore )}
\end{itemize}
Acto de applaudir.
Acclamação.
Elogio público.
Louvor.
Demonstração alegre e ruidosa de approvação: \textunderscore o orador recebeu muitos applausos\textunderscore .
\section{Applicabilidade}
\begin{itemize}
\item {Grp. gram.:f.}
\end{itemize}
Qualidade do que é \textunderscore applicável\textunderscore .
\section{Applicação}
\begin{itemize}
\item {Grp. gram.:f.}
\end{itemize}
\begin{itemize}
\item {Proveniência:(Lat. \textunderscore applicatio\textunderscore )}
\end{itemize}
Acto ou effeito de applicar.
Adaptação.
Destino.
Concentração de espirito: \textunderscore estudar com applicação\textunderscore .
Obra de passamanaria.
\section{Applicadamente}
\begin{itemize}
\item {Grp. gram.:adv.}
\end{itemize}
Com applicação.
\section{Applicando}
\begin{itemize}
\item {Grp. gram.:adj.}
\end{itemize}
O mesmo que \textunderscore applicável\textunderscore .
\section{Applicante}
\begin{itemize}
\item {Grp. gram.:adj.}
\end{itemize}
\begin{itemize}
\item {Proveniência:(Lat. \textunderscore applicans\textunderscore )}
\end{itemize}
Que applica.
\section{Applicar}
\begin{itemize}
\item {Grp. gram.:v. t.}
\end{itemize}
\begin{itemize}
\item {Proveniência:(Lat. \textunderscore applicare\textunderscore )}
\end{itemize}
Juntar; adaptar; sobrepor: \textunderscore applicar uma cataplasma\textunderscore .
Empregar: \textunderscore applicar mal o tempo\textunderscore .
Receitar: \textunderscore applicar um xarope\textunderscore .
Realizar.
Infligir: \textunderscore o juiz applicou-lhe pena leve\textunderscore .
\section{Applicativo}
\begin{itemize}
\item {Grp. gram.:adj.}
\end{itemize}
O mesmo que \textunderscore applicável\textunderscore .
\section{Applicável}
\begin{itemize}
\item {Grp. gram.:adj.}
\end{itemize}
\begin{itemize}
\item {Proveniência:(De \textunderscore applicar\textunderscore )}
\end{itemize}
Que póde sêr applicado.
\section{Appoer}
\begin{itemize}
\item {Grp. gram.:v. t.}
\end{itemize}
(Fórma antiga de \textunderscore appor\textunderscore )
Accusar.
\section{Appoggiatura}
\begin{itemize}
\item {Grp. gram.:f.}
\end{itemize}
\begin{itemize}
\item {Utilização:Mús.}
\end{itemize}
\begin{itemize}
\item {Proveniência:(T. it.)}
\end{itemize}
Nóta rápida, antes da nota principal.
\section{Appor}
\begin{itemize}
\item {Grp. gram.:v. i.}
\end{itemize}
\begin{itemize}
\item {Proveniência:(Lat. \textunderscore apponere\textunderscore )}
\end{itemize}
Sobrepor; juxtapor.
Applicar.
Jungir (bois ou vacas) a um carro.
Meter (cavalgaduras) aos varaes da carruagem. Cf. Camillo, \textunderscore Cav. em Ruin.\textunderscore , 112.
\section{Apposição}
\begin{itemize}
\item {Grp. gram.:f.}
\end{itemize}
\begin{itemize}
\item {Utilização:Gram.}
\end{itemize}
\begin{itemize}
\item {Proveniência:(Lat. \textunderscore appositio\textunderscore )}
\end{itemize}
Acção de appor.
Relação de dois substantivos separados por vírgula, um dos quaes se refere ao outro.
\section{Appositadamente}
\begin{itemize}
\item {Grp. gram.:adv.}
\end{itemize}
Judiciosamente, sensatamente. Cf. Cortesão, \textunderscore Subs\textunderscore .
\section{Appositivo}
\begin{itemize}
\item {Grp. gram.:adj.}
\end{itemize}
\begin{itemize}
\item {Proveniência:(Lat. \textunderscore appositivus\textunderscore )}
\end{itemize}
Que tem apposição.
\section{Appósito}
\begin{itemize}
\item {Grp. gram.:adj.}
\end{itemize}
\begin{itemize}
\item {Grp. gram.:M.}
\end{itemize}
O mesmo que \textunderscore apposto\textunderscore .
Parche ou ligadura, que se faz adherir a um ferimento ou chaga, até que passe a inflammação ou até que haja cicatrização. Cf. Filinto, IV, 208.
\section{Appostamente}
\begin{itemize}
\item {Grp. gram.:adv.}
\end{itemize}
\begin{itemize}
\item {Proveniência:(De \textunderscore apposto\textunderscore )}
\end{itemize}
Em bôa ordem; convenientemente.
Com asseio.
\section{Appostar}
\begin{itemize}
\item {Grp. gram.:v. t.}
\end{itemize}
\begin{itemize}
\item {Utilização:Ant.}
\end{itemize}
\begin{itemize}
\item {Proveniência:(De \textunderscore apposto\textunderscore )}
\end{itemize}
Collocar ao pé.
Acommodar; dispor bem.
\section{Apposto}
\begin{itemize}
\item {Grp. gram.:adj.}
\end{itemize}
\begin{itemize}
\item {Grp. gram.:M.}
\end{itemize}
\begin{itemize}
\item {Proveniência:(Lat. \textunderscore appositus\textunderscore )}
\end{itemize}
Bem feito de corpo, airoso, gentil.
Aquillo que está ligado por apposição.
\section{Appostura}
\begin{itemize}
\item {Grp. gram.:f.}
\end{itemize}
\begin{itemize}
\item {Proveniência:(De \textunderscore apposto\textunderscore )}
\end{itemize}
Gentileza; garbo.
\section{Apprehendedor}
\begin{itemize}
\item {Grp. gram.:m.}
\end{itemize}
\begin{itemize}
\item {Proveniência:(De \textunderscore apprehender\textunderscore )}
\end{itemize}
Aquelle que apprehende.
\section{Apprehender}
\begin{itemize}
\item {Grp. gram.:v. t.}
\end{itemize}
\begin{itemize}
\item {Grp. gram.:V. i.}
\end{itemize}
\begin{itemize}
\item {Proveniência:(Lat. \textunderscore apprehendere\textunderscore )}
\end{itemize}
Fazer apprehensão de; tomar: \textunderscore apprehender contrabando\textunderscore .
Prender.
Preoccupar-se; scismar.
\section{Apprehensão}
\begin{itemize}
\item {Grp. gram.:f.}
\end{itemize}
\begin{itemize}
\item {Proveniência:(Lat. \textunderscore apprehensio\textunderscore )}
\end{itemize}
Acto de tomar ou prender.
Receio.
Scisma: \textunderscore a sua principal doença é a apprehensão\textunderscore .
Percepção.
\section{Apprehensibilidade}
\begin{itemize}
\item {Grp. gram.:f.}
\end{itemize}
Qualidade do que é \textunderscore apprehensível\textunderscore .
\section{Apprehensivamente}
\begin{itemize}
\item {Grp. gram.:adv.}
\end{itemize}
De modo \textunderscore apprehensivo\textunderscore .
Com apprehensão.
\section{Apprehensível}
\begin{itemize}
\item {Grp. gram.:adj.}
\end{itemize}
\begin{itemize}
\item {Proveniência:(Lat. \textunderscore apprehensibilis\textunderscore )}
\end{itemize}
Que póde sêr apprehendido.
\section{Apprehensivo}
\begin{itemize}
\item {Grp. gram.:adj.}
\end{itemize}
\begin{itemize}
\item {Proveniência:(Do lat. \textunderscore apprehensus\textunderscore )}
\end{itemize}
Que apprehende.
Receoso.
Scismático.
\section{Apprehensor}
\begin{itemize}
\item {Grp. gram.:m.}
\end{itemize}
\begin{itemize}
\item {Proveniência:(Do lat. \textunderscore apprehensus\textunderscore )}
\end{itemize}
Aquelle que apprehende.
\section{Apprehensório}
\begin{itemize}
\item {Grp. gram.:adj.}
\end{itemize}
\begin{itemize}
\item {Proveniência:(Do lat. \textunderscore apprehensus\textunderscore )}
\end{itemize}
Que serve para apprehender.
\section{Apprender}
\begin{itemize}
\item {Grp. gram.:v. t.}
\end{itemize}
\begin{itemize}
\item {Grp. gram.:V. i.}
\end{itemize}
\begin{itemize}
\item {Proveniência:(Lat. \textunderscore apprehendere\textunderscore )}
\end{itemize}
Adquirir conhecimento de: \textunderscore apprender música\textunderscore .
Conservar na memória: \textunderscore apprender um conto\textunderscore .
Fixar facilmente na memória qualquer coisa: \textunderscore êste rapaz apprende bem\textunderscore .
Prestar-se ao ensino, á educação.
\section{Apprendiz}
\begin{itemize}
\item {Grp. gram.:m.}
\end{itemize}
Aquelle que apprende offício ou arte; principiante.
Aquelle que é pouco intelligente ou que tem pouca experiência.
Primeiro grau da Maçonaria.
(B. lat. \textunderscore apprehendivus\textunderscore )
\section{Apprendizado}
\begin{itemize}
\item {Grp. gram.:m.}
\end{itemize}
\begin{itemize}
\item {Proveniência:(De \textunderscore apprendiz\textunderscore )}
\end{itemize}
Acção de apprender.
Tirocínio.
Tempo que se gasta, apprendendo. Cf. Latino, \textunderscore Elogios\textunderscore , 345.
\section{Apprendizagem}
\begin{itemize}
\item {Grp. gram.:f.}
\end{itemize}
\begin{itemize}
\item {Utilização:Gal}
\end{itemize}
(V.apprendizado)
\section{Approbativamente}
\begin{itemize}
\item {Grp. gram.:adv.}
\end{itemize}
De modo \textunderscore approbativo\textunderscore .
Com approvação.
\section{Approbativo}
\begin{itemize}
\item {Grp. gram.:adj.}
\end{itemize}
\begin{itemize}
\item {Proveniência:(Do lat. \textunderscore approbare\textunderscore )}
\end{itemize}
Que contém approvação.
\section{Approbatório}
\begin{itemize}
\item {Grp. gram.:adj.}
\end{itemize}
(V.approbativo)
\section{Appropinquação}
\begin{itemize}
\item {Grp. gram.:f.}
\end{itemize}
Acto de \textunderscore appropinquar\textunderscore .
\section{Appropinquar}
\begin{itemize}
\item {Grp. gram.:v. t.}
\end{itemize}
\begin{itemize}
\item {Proveniência:(Lat. \textunderscore appropinquare\textunderscore )}
\end{itemize}
O mesmo que \textunderscore aproximar\textunderscore .
\section{Appropriação}
\begin{itemize}
\item {Grp. gram.:f.}
\end{itemize}
Acto de \textunderscore appropriar\textunderscore .
\section{Appropriadamente}
\begin{itemize}
\item {Grp. gram.:adv.}
\end{itemize}
Com propriedade.
Judiciosamente.
\section{Appropriador}
\begin{itemize}
\item {Grp. gram.:adj.}
\end{itemize}
\begin{itemize}
\item {Grp. gram.:M.}
\end{itemize}
Que appropria.
Official de chapelaria, que faz a appropriagem.
\section{Appropriagem}
\begin{itemize}
\item {Grp. gram.:f.}
\end{itemize}
\begin{itemize}
\item {Proveniência:(De \textunderscore appropriar\textunderscore )}
\end{itemize}
Acabamento do chapéu, depois que vem da fula.
\section{Appropriar}
\begin{itemize}
\item {Grp. gram.:v. t.}
\end{itemize}
\begin{itemize}
\item {Proveniência:(Lat. \textunderscore appropriare\textunderscore )}
\end{itemize}
Tornar próprio.
Accomodar.
Applicar.
Attribuir.
\section{Approvação}
\begin{itemize}
\item {Grp. gram.:f.}
\end{itemize}
\begin{itemize}
\item {Proveniência:(Lat. \textunderscore approbatio\textunderscore )}
\end{itemize}
Acto de \textunderscore approvar\textunderscore .
\section{Approvadamente}
\begin{itemize}
\item {Grp. gram.:adv.}
\end{itemize}
Com approvação.
\section{Approvado}
\begin{itemize}
\item {Grp. gram.:adj.}
\end{itemize}
Que teve approvação.
Admittido.
Julgado apto.
\section{Approvador}
\begin{itemize}
\item {Grp. gram.:m.}
\end{itemize}
\begin{itemize}
\item {Proveniência:(Lat. \textunderscore approbator\textunderscore )}
\end{itemize}
Aquelle que approva.
\section{Approvar}
\begin{itemize}
\item {Grp. gram.:v. t.}
\end{itemize}
\begin{itemize}
\item {Proveniência:(Lat. \textunderscore approbare\textunderscore )}
\end{itemize}
Julgar bom: \textunderscore approvar uma acção\textunderscore .
Louvar.
Ratificar.
Consentir em.
Julgar habilitado (o estudante) na disciplina em que foi examinado.
\section{Approvativo}
\begin{itemize}
\item {Grp. gram.:adj.}
\end{itemize}
(V.approbativo)
\section{Approvável}
\begin{itemize}
\item {Grp. gram.:adj.}
\end{itemize}
\begin{itemize}
\item {Proveniência:(Lat. \textunderscore approbabilis\textunderscore )}
\end{itemize}
Digno de sêr approvado.
\section{Approxar}
\begin{itemize}
\item {Grp. gram.:v. i.}
\end{itemize}
\begin{itemize}
\item {Utilização:Ant.}
\end{itemize}
Fazer approxes.
\section{Approxe}
\begin{itemize}
\item {Grp. gram.:m.}
\end{itemize}
\begin{itemize}
\item {Utilização:Des.}
\end{itemize}
\begin{itemize}
\item {Proveniência:(Fr. \textunderscore approche\textunderscore )}
\end{itemize}
Entricheiramento, para facilitar a aproximação ás praças sitiadas.
Investida.
\section{Appulso}
\begin{itemize}
\item {Grp. gram.:m.}
\end{itemize}
\begin{itemize}
\item {Proveniência:(Lat. \textunderscore appulsus\textunderscore )}
\end{itemize}
Passagem da Lua junto de outro astro.
\section{Appúnia}
\begin{itemize}
\item {Grp. gram.:f.}
\end{itemize}
Gênero de plantas rubiáceas da América.
\section{Apoer}
\begin{itemize}
\item {Grp. gram.:v. t.}
\end{itemize}
(Fórma antiga de \textunderscore apor\textunderscore )
Accusar.
\section{Apor}
\begin{itemize}
\item {Grp. gram.:v. i.}
\end{itemize}
\begin{itemize}
\item {Proveniência:(Lat. \textunderscore apponere\textunderscore )}
\end{itemize}
Sobrepor; juxtapor.
Aplicar.
Jungir (bois ou vacas) a um carro.
Meter (cavalgaduras) aos varaes da carruagem. Cf. Camillo, \textunderscore Cav. em Ruin.\textunderscore , 112.
\section{Aposição}
\begin{itemize}
\item {Grp. gram.:f.}
\end{itemize}
\begin{itemize}
\item {Utilização:Gram.}
\end{itemize}
\begin{itemize}
\item {Proveniência:(Lat. \textunderscore appositio\textunderscore )}
\end{itemize}
Acção de apor.
Relação de dois substantivos separados por vírgula, um dos quaes se refere ao outro.
\section{Apositadamente}
\begin{itemize}
\item {Grp. gram.:adv.}
\end{itemize}
Judiciosamente, sensatamente. Cf. Cortesão, \textunderscore Subs\textunderscore .
\section{Apositivo}
\begin{itemize}
\item {Grp. gram.:adj.}
\end{itemize}
\begin{itemize}
\item {Proveniência:(Lat. \textunderscore appositivus\textunderscore )}
\end{itemize}
Que tem aposição.
\section{Apósito}
\begin{itemize}
\item {Grp. gram.:adj.}
\end{itemize}
\begin{itemize}
\item {Grp. gram.:M.}
\end{itemize}
O mesmo que \textunderscore aposto\textunderscore .
Parche ou ligadura, que se faz adherir a um ferimento ou chaga, até que passe a inflammação ou até que haja cicatrizacão. Cf. Filinto, IV, 208.
\section{Apostamente}
\begin{itemize}
\item {Grp. gram.:adv.}
\end{itemize}
\begin{itemize}
\item {Proveniência:(De \textunderscore apposto\textunderscore )}
\end{itemize}
Em bôa ordem; convenientemente.
Com asseio.
\section{Apostar}
\begin{itemize}
\item {Grp. gram.:v. t.}
\end{itemize}
\begin{itemize}
\item {Utilização:Ant.}
\end{itemize}
\begin{itemize}
\item {Proveniência:(De \textunderscore apposto\textunderscore )}
\end{itemize}
Collocar ao pé.
Acommodar; dispor bem.
\section{Aposto}
\begin{itemize}
\item {fónica:pôs}
\end{itemize}
\begin{itemize}
\item {Grp. gram.:adj.}
\end{itemize}
\begin{itemize}
\item {Grp. gram.:M.}
\end{itemize}
\begin{itemize}
\item {Proveniência:(Lat. \textunderscore appositus\textunderscore )}
\end{itemize}
Bem feito de corpo, airoso, gentil.
Aquillo que está ligado por aposição.
\section{Apostura}
\begin{itemize}
\item {Grp. gram.:f.}
\end{itemize}
\begin{itemize}
\item {Proveniência:(De \textunderscore apposto\textunderscore )}
\end{itemize}
Gentileza; garbo.
\section{Apraxia}
\begin{itemize}
\item {Grp. gram.:f.}
\end{itemize}
\begin{itemize}
\item {Utilização:Med.}
\end{itemize}
Perda da faculdade de apreciar as fórmas dos objectos.
\section{Aprazador}
\begin{itemize}
\item {Grp. gram.:m.}
\end{itemize}
Aquelle que apraza.
\section{Aprazamento}
\begin{itemize}
\item {Grp. gram.:m.}
\end{itemize}
Acto de aprazar.
\section{Aprazar}
\begin{itemize}
\item {Grp. gram.:v. t.}
\end{itemize}
Marcar prazo para se fazer (alguma coisa): \textunderscore aprazar uma tarefa\textunderscore .
Convocar, para tempo determinado: \textunderscore aprazar uma reunião\textunderscore .
Designar.
Adiar.
\section{Aprazente}
\begin{itemize}
\item {Grp. gram.:adj.}
\end{itemize}
\begin{itemize}
\item {Utilização:Des.}
\end{itemize}
Que apraz.
\section{Aprazer}
\begin{itemize}
\item {Grp. gram.:v. i.}
\end{itemize}
\begin{itemize}
\item {Proveniência:(De \textunderscore prazer\textunderscore )}
\end{itemize}
Sêr aprazível; agradar.
Causar deleite.
\section{Aprazerado}
\begin{itemize}
\item {Grp. gram.:adj.}
\end{itemize}
\begin{itemize}
\item {Utilização:Ant.}
\end{itemize}
Dado a prazeres.
Prazenteiro, alegre.
\section{Aprazibilidade}
\begin{itemize}
\item {Grp. gram.:f.}
\end{itemize}
Qualidade do que é \textunderscore aprazível\textunderscore .
\section{Aprazimento}
\begin{itemize}
\item {Grp. gram.:m.}
\end{itemize}
\begin{itemize}
\item {Proveniência:(De \textunderscore aprazer\textunderscore )}
\end{itemize}
Agrado.
Contentamento; prazer.
Beneplácito.
\section{Aprazível}
\begin{itemize}
\item {Grp. gram.:adj.}
\end{itemize}
Que apraz.
\section{Aprazivelmente}
\begin{itemize}
\item {Grp. gram.:adv.}
\end{itemize}
De modo \textunderscore aprazivel\textunderscore .
\section{Apre!}
\begin{itemize}
\item {Grp. gram.:interj.}
\end{itemize}
Irra!
Fóra!
Vai-te!
Apage! oh!
\section{Apreá}
\begin{itemize}
\item {Grp. gram.:m.}
\end{itemize}
\begin{itemize}
\item {Utilização:Bras}
\end{itemize}
\begin{itemize}
\item {Utilização:Zool.}
\end{itemize}
Espécie de roedor.
\section{Aprear}
\begin{itemize}
\item {Grp. gram.:v. t.}
\end{itemize}
O mesmo que \textunderscore prear\textunderscore . Cf. Filinto, \textunderscore D. Man.\textunderscore , I, 238.
\section{Apreçador}
\begin{itemize}
\item {Grp. gram.:m.}
\end{itemize}
Aquelle que apreça.
\section{Apreçamento}
\begin{itemize}
\item {Grp. gram.:m.}
\end{itemize}
Acto de \textunderscore apreçar\textunderscore .
\section{Apreçar}
\begin{itemize}
\item {Grp. gram.:v. t.}
\end{itemize}
Marcar o preço de.
Perguntar o preço de.
Avaliar.
\section{Apreciação}
\begin{itemize}
\item {Grp. gram.:f.}
\end{itemize}
Acto de \textunderscore apreciar\textunderscore .
\section{Apreciador}
\begin{itemize}
\item {Grp. gram.:m.}
\end{itemize}
Aquelle que aprecía.
\section{Apreciadura}
\begin{itemize}
\item {Grp. gram.:f.}
\end{itemize}
\begin{itemize}
\item {Utilização:Ant.}
\end{itemize}
Coisa certa e determinada.
Postura camarária. Cf. Herculano, \textunderscore Hist. de Port.\textunderscore , IV, 395.
\section{Apreciar}
\begin{itemize}
\item {Grp. gram.:v. t.}
\end{itemize}
\begin{itemize}
\item {Proveniência:(Lat. \textunderscore appretiare\textunderscore )}
\end{itemize}
Dar merecimento a; estimar.
Avaliar.
Julgar.
\section{Apreciativo}
\begin{itemize}
\item {Grp. gram.:adj.}
\end{itemize}
Que denota apreciação.
\section{Apreciável}
\begin{itemize}
\item {Grp. gram.:adj.}
\end{itemize}
Que é digno de aprêço.
\section{Aprêço}
\begin{itemize}
\item {Grp. gram.:m.}
\end{itemize}
\begin{itemize}
\item {Proveniência:(De \textunderscore apreçar\textunderscore )}
\end{itemize}
Estima, consideração.
\section{Apreendedor}
\begin{itemize}
\item {Grp. gram.:m.}
\end{itemize}
\begin{itemize}
\item {Proveniência:(De \textunderscore apprehender\textunderscore )}
\end{itemize}
Aquelle que apreende.
\section{Apreender}
\begin{itemize}
\item {Grp. gram.:v. t.}
\end{itemize}
\begin{itemize}
\item {Grp. gram.:V. i.}
\end{itemize}
\begin{itemize}
\item {Proveniência:(Lat. \textunderscore apprehendere\textunderscore )}
\end{itemize}
Fazer apreensão de; tomar: \textunderscore apreender contrabando\textunderscore .
Prender.
Preoccupar-se; scismar.
\section{Apreensão}
\begin{itemize}
\item {Grp. gram.:f.}
\end{itemize}
\begin{itemize}
\item {Proveniência:(Lat. \textunderscore apprehensio\textunderscore )}
\end{itemize}
Acto de tomar ou prender.
Receio.
Scisma: \textunderscore a sua principal doença é a apreensão\textunderscore .
Percepção.
\section{Apreensibilidade}
\begin{itemize}
\item {Grp. gram.:f.}
\end{itemize}
Qualidade do que é \textunderscore apreensível\textunderscore .
\section{Apreensivamente}
\begin{itemize}
\item {Grp. gram.:adv.}
\end{itemize}
De modo \textunderscore apreensivo\textunderscore .
Com apreensão.
\section{Apreensível}
\begin{itemize}
\item {Grp. gram.:adj.}
\end{itemize}
\begin{itemize}
\item {Proveniência:(Lat. \textunderscore apprehensibilis\textunderscore )}
\end{itemize}
Que póde sêr apreendido.
\section{Apreensivo}
\begin{itemize}
\item {Grp. gram.:adj.}
\end{itemize}
\begin{itemize}
\item {Proveniência:(Do lat. \textunderscore apprehensus\textunderscore )}
\end{itemize}
Que apreende.
Receoso.
Scismático.
\section{Apreensor}
\begin{itemize}
\item {Grp. gram.:m.}
\end{itemize}
\begin{itemize}
\item {Proveniência:(Do lat. \textunderscore apprehensus\textunderscore )}
\end{itemize}
Aquelle que apreende.
\section{Apreensório}
\begin{itemize}
\item {Grp. gram.:adj.}
\end{itemize}
\begin{itemize}
\item {Proveniência:(Do lat. \textunderscore apprehensus\textunderscore )}
\end{itemize}
Que serve para apreender.
\section{Apregoador}
\begin{itemize}
\item {Grp. gram.:m.}
\end{itemize}
Aquelle que apregôa.
\section{Apregoar}
\begin{itemize}
\item {Grp. gram.:v. t.}
\end{itemize}
Publicar por pregão.
Annunciar em alta voz.
Proclamar; divulgar: \textunderscore apregoar calúmnias\textunderscore .
(B. lat. \textunderscore appreconare\textunderscore )
\section{Apremar}
\begin{itemize}
\item {Grp. gram.:v. t.}
\end{itemize}
\begin{itemize}
\item {Proveniência:(De \textunderscore prema\textunderscore )}
\end{itemize}
O mesmo que \textunderscore opprimir\textunderscore .
\section{Aprender}
\begin{itemize}
\item {Grp. gram.:v. t.}
\end{itemize}
\begin{itemize}
\item {Grp. gram.:V. i.}
\end{itemize}
\begin{itemize}
\item {Proveniência:(Lat. \textunderscore apprehendere\textunderscore )}
\end{itemize}
Adquirir conhecimento de: \textunderscore aprender música\textunderscore .
Conservar na memória: \textunderscore aprender um conto\textunderscore .
Fixar facilmente na memória qualquer coisa: \textunderscore êste rapaz aprende bem\textunderscore .
Prestar-se ao ensino, á educação.
\section{Aprendiz}
\begin{itemize}
\item {Grp. gram.:m.}
\end{itemize}
Aquelle que aprende offício ou arte; principiante.
Aquelle que é pouco intelligente ou que tem pouca experiência.
Primeiro grau da Maçonaria.
(B. lat. \textunderscore apprehendivus\textunderscore )
\section{Aprendizado}
\begin{itemize}
\item {Grp. gram.:m.}
\end{itemize}
\begin{itemize}
\item {Proveniência:(De \textunderscore apprendiz\textunderscore )}
\end{itemize}
Acção de aprender.
Tirocínio.
Tempo que se gasta, aprendendo. Cf. Latino, \textunderscore Elogios\textunderscore , 345.
\section{Aprendizagem}
\begin{itemize}
\item {Grp. gram.:f.}
\end{itemize}
\begin{itemize}
\item {Utilização:Gal}
\end{itemize}
(V.aprendizado)
\section{Apres}
\begin{itemize}
\item {Grp. gram.:prep.}
\end{itemize}
\begin{itemize}
\item {Utilização:Ant.}
\end{itemize}
Junto de: em poder de.
(Cf. francês \textunderscore auprès\textunderscore )
\section{Apresador}
\begin{itemize}
\item {Grp. gram.:m.}
\end{itemize}
\begin{itemize}
\item {Proveniência:(De \textunderscore apresar\textunderscore )}
\end{itemize}
Aquelle que apresa.
\section{Apresamento}
\begin{itemize}
\item {Grp. gram.:m.}
\end{itemize}
Acto de \textunderscore apresar\textunderscore .
\section{Apresar}
\begin{itemize}
\item {Grp. gram.:v. t.}
\end{itemize}
Tomar como presa, capturar; agarrar; apprehender.
\section{Apresenciar}
\begin{itemize}
\item {Grp. gram.:v. t.}
\end{itemize}
O mesmo que \textunderscore presenciar\textunderscore . Cf. Filinto, XI, 267.
\section{Apresentação}
\begin{itemize}
\item {Grp. gram.:f.}
\end{itemize}
Acto de \textunderscore apresentar\textunderscore .
(B. lat. \textunderscore apresentactio\textunderscore )
\section{Apresentador}
\begin{itemize}
\item {Grp. gram.:m.}
\end{itemize}
Aquelle que apresenta.
\section{Apresentante}
\begin{itemize}
\item {Grp. gram.:adj.}
\end{itemize}
Que apresenta.
\section{Apresentar}
\begin{itemize}
\item {Grp. gram.:v. t.}
\end{itemize}
\begin{itemize}
\item {Proveniência:(Lat. \textunderscore praesentare\textunderscore )}
\end{itemize}
Tornar presente.
Pôr á vista: \textunderscore apresentou-lhe o filho\textunderscore .
Manifestar: \textunderscore apresenta sinaes de desgôsto\textunderscore .
Offerecer.
Conferir benefício ecclesiástico a: \textunderscore apresentar um párocho\textunderscore .
Estender.
Exprimir: \textunderscore apresento-lhe os meus sentimentos\textunderscore .
Mostrar, recommendar.
\section{Apresentável}
\begin{itemize}
\item {Grp. gram.:adj.}
\end{itemize}
Digno de sêr apresentado.
\section{Apresigar}
\begin{itemize}
\item {Grp. gram.:v. t.}
\end{itemize}
\begin{itemize}
\item {Proveniência:(De \textunderscore apresigo\textunderscore )}
\end{itemize}
O mesmo que \textunderscore apeguilhar\textunderscore .
\section{Apresigo}
\begin{itemize}
\item {Grp. gram.:m.}
\end{itemize}
O mesmo que \textunderscore presigo\textunderscore .
\section{Apresilhar}
\begin{itemize}
\item {Grp. gram.:v. t.}
\end{itemize}
Prender com presilha.
Guarnecer com cordões de trancelim.
\section{Apressadamente}
\begin{itemize}
\item {Grp. gram.:adv.}
\end{itemize}
Com pressa, á pressa, de modo \textunderscore apressado\textunderscore .
\section{Apressado}
\begin{itemize}
\item {Grp. gram.:adj.}
\end{itemize}
\begin{itemize}
\item {Grp. gram.:Adv.}
\end{itemize}
Que tem pressa.
Ligeiro.
Apressadamente.
\section{Apressador}
\begin{itemize}
\item {Grp. gram.:m.}
\end{itemize}
Aquelle que apressa.
\section{Apressamento}
\begin{itemize}
\item {Grp. gram.:m.}
\end{itemize}
Acto de \textunderscore apressar\textunderscore .
\section{Apressar}
\begin{itemize}
\item {Grp. gram.:v. t.}
\end{itemize}
Dar pressa a.
Tornar rápido: \textunderscore apressar o passo\textunderscore .
Fazer com rapidez.
Estimular; obrigar a proceder com pressa.
\section{Apressuradamente}
\begin{itemize}
\item {Grp. gram.:adv.}
\end{itemize}
Com apressuramento.
\section{Apressuramento}
\begin{itemize}
\item {Grp. gram.:m.}
\end{itemize}
Acto de \textunderscore apressurar\textunderscore .
\section{Apressurar}
\begin{itemize}
\item {Grp. gram.:v. t.}
\end{itemize}
Tornar pressuroso.
Afadigar.
Aviár.
Executar rapidamente.
\section{Aprestador}
\begin{itemize}
\item {Grp. gram.:m.}
\end{itemize}
Aquelle que apresta.
\section{Aprestamar}
\begin{itemize}
\item {Grp. gram.:v. t.}
\end{itemize}
\begin{itemize}
\item {Utilização:Ant.}
\end{itemize}
Dar em apréstamo.
\section{Aprestamento}
\begin{itemize}
\item {Grp. gram.:m.}
\end{itemize}
Acto de \textunderscore aprestar\textunderscore .
\section{Apréstamo}
\begin{itemize}
\item {Grp. gram.:m.}
\end{itemize}
\begin{itemize}
\item {Utilização:Jur.}
\end{itemize}
Consignação de frutos, imposta nalguma herdade, para pagamento de certos encargos.
Herdade, sujeita a êsse ónus.
O mesmo que \textunderscore préstamo\textunderscore .
\section{Aprestar}
\begin{itemize}
\item {Grp. gram.:v. t.}
\end{itemize}
\begin{itemize}
\item {Proveniência:(De \textunderscore presto\textunderscore )}
\end{itemize}
Apparelhar; apromptar.
Fazer prestes.
Aperceber.
\section{Apreste}
\begin{itemize}
\item {Grp. gram.:m.}
\end{itemize}
O mesmo que \textunderscore apresto\textunderscore . Cf. Castilho, \textunderscore Fastos\textunderscore , III, 153.
\section{Aprestes}
\begin{itemize}
\item {Grp. gram.:adj.}
\end{itemize}
\begin{itemize}
\item {Utilização:Ant.}
\end{itemize}
O mesmo que \textunderscore prestes\textunderscore .
\section{Apresto}
\begin{itemize}
\item {Grp. gram.:m.}
\end{itemize}
\begin{itemize}
\item {Proveniência:(De \textunderscore aprestar\textunderscore )}
\end{itemize}
Preparativo.
Tudo que serve de instrumento ou meio para a realização de certos actos.
\section{Apricar}
\begin{itemize}
\item {Grp. gram.:v. t.}
\end{itemize}
\begin{itemize}
\item {Utilização:Ant.}
\end{itemize}
O mesmo que \textunderscore applicar\textunderscore .
\section{Aprico}
\begin{itemize}
\item {Grp. gram.:adj.}
\end{itemize}
\begin{itemize}
\item {Utilização:Des.}
\end{itemize}
\begin{itemize}
\item {Proveniência:(Lat. \textunderscore apricus\textunderscore )}
\end{itemize}
Abrigado. Cf. Filinto, IX, 54 e 80.
\section{A-primas}
\begin{itemize}
\item {Grp. gram.:loc. adv.}
\end{itemize}
\begin{itemize}
\item {Utilização:Ant.}
\end{itemize}
Em primeiro lugar.
\section{Aprimoradamente}
\begin{itemize}
\item {Grp. gram.:adv.}
\end{itemize}
Com primor.
\section{Aprimorado}
\begin{itemize}
\item {Grp. gram.:adj.}
\end{itemize}
\begin{itemize}
\item {Proveniência:(De \textunderscore aprimorar\textunderscore )}
\end{itemize}
Feito com primor.
Perfeito.
Completo.
\section{Aprimorador}
\begin{itemize}
\item {Grp. gram.:m.}
\end{itemize}
Aquelle que aprimora. Cf. Castilho, \textunderscore Fausto\textunderscore , X.
\section{Aprimorar}
\begin{itemize}
\item {Grp. gram.:v. t.}
\end{itemize}
Fazer com primor.
Tornar primoroso.
Aperfeiçoar.
Acompanhar de delicadeza.
\section{Aprincesar-se}
\begin{itemize}
\item {Grp. gram.:v. p.}
\end{itemize}
Tomar modos de princesa.
\section{Apriscar}
\begin{itemize}
\item {Grp. gram.:v. t.}
\end{itemize}
\begin{itemize}
\item {Utilização:Des.}
\end{itemize}
Meter no aprisco.
\section{Aprisco}
\begin{itemize}
\item {Grp. gram.:m.}
\end{itemize}
\begin{itemize}
\item {Utilização:Prov.}
\end{itemize}
\begin{itemize}
\item {Utilização:trasm.}
\end{itemize}
Curral.
Albergue.
Choupana.
Caverna.
Abegoaria.
Propriedade insignificante, nas arribas.
(Cast. \textunderscore aprisco\textunderscore )
\section{Aprisionado}
\begin{itemize}
\item {Grp. gram.:adj.}
\end{itemize}
\begin{itemize}
\item {Proveniência:(De \textunderscore aprisionar\textunderscore )}
\end{itemize}
Prisioneiro.
Encarcerado.
Submisso, sujeito.
\section{Aprisionador}
\begin{itemize}
\item {Grp. gram.:m.}
\end{itemize}
Aquelle que aprisiona.
\section{Aprisionamento}
\begin{itemize}
\item {Grp. gram.:m.}
\end{itemize}
Acto de \textunderscore aprisionar\textunderscore .
\section{Aprisionar}
\begin{itemize}
\item {Grp. gram.:v. i.}
\end{itemize}
\begin{itemize}
\item {Proveniência:(De \textunderscore prisão\textunderscore )}
\end{itemize}
Fazer prisioneiro.
Meter em prisão; prender; apresar.
\section{Aprisoar}
\begin{itemize}
\item {Grp. gram.:v. t.}
\end{itemize}
(V.aprisionar)
\section{Aproamento}
\begin{itemize}
\item {Grp. gram.:m.}
\end{itemize}
Acto de \textunderscore aproar\textunderscore .
\section{Aproar}
\begin{itemize}
\item {Grp. gram.:v. t.}
\end{itemize}
\begin{itemize}
\item {Grp. gram.:V. i.}
\end{itemize}
Dirigir com a prôa: \textunderscore aproar um barco\textunderscore .
Dirigir a prôa, por effeito do leme ou das velas ou por effeito natural: \textunderscore aproar para Léste\textunderscore .
\section{Aprobativamente}
\begin{itemize}
\item {Grp. gram.:adv.}
\end{itemize}
De modo \textunderscore aprobativo\textunderscore .
Com aprovação.
\section{Aprobativo}
\begin{itemize}
\item {Grp. gram.:adj.}
\end{itemize}
\begin{itemize}
\item {Proveniência:(Do lat. \textunderscore approbare\textunderscore )}
\end{itemize}
Que contém aprovação.
\section{Aprobatório}
\begin{itemize}
\item {Grp. gram.:adj.}
\end{itemize}
(V.aprobativo)
\section{Aproejar}
\begin{itemize}
\item {Grp. gram.:v. i.}
\end{itemize}
O mesmo que \textunderscore aproar\textunderscore .
\section{Aprofeitar}
\begin{itemize}
\item {Grp. gram.:v. t.}
\end{itemize}
\begin{itemize}
\item {Utilização:Ant.}
\end{itemize}
O mesmo que \textunderscore aproveitar\textunderscore .
\section{Aprofundar}
\begin{itemize}
\item {Grp. gram.:v. t.}
\end{itemize}
O mesmo que \textunderscore profundar\textunderscore .
\section{Apira}
\begin{itemize}
\item {Grp. gram.:f.}
\end{itemize}
\begin{itemize}
\item {Proveniência:(De \textunderscore apyro\textunderscore )}
\end{itemize}
Barro infusível, espécie de porcelana, de que se fazem objectos de loiça e de esculptura.
\section{Apirético}
\begin{itemize}
\item {Grp. gram.:adj.}
\end{itemize}
\begin{itemize}
\item {Proveniência:(De \textunderscore apyrexia\textunderscore )}
\end{itemize}
Que não tem febre.
\section{Apirexia}
\begin{itemize}
\item {fónica:csi}
\end{itemize}
\begin{itemize}
\item {Grp. gram.:f.}
\end{itemize}
\begin{itemize}
\item {Proveniência:(Gr. \textunderscore apurexia\textunderscore )}
\end{itemize}
Estado do enfêrmo, nos intervallos dos accessos febris.
\section{Apiro}
\begin{itemize}
\item {Grp. gram.:adj.}
\end{itemize}
\begin{itemize}
\item {Proveniência:(Gr. \textunderscore apuros\textunderscore )}
\end{itemize}
Que resiste ao fogo.
\section{Apromptar}
\begin{itemize}
\item {Grp. gram.:v. t.}
\end{itemize}
\begin{itemize}
\item {Proveniência:(De \textunderscore prompto\textunderscore )}
\end{itemize}
Tornar prompto.
Dispor; apparelhar; preparar.
Concluir.
\section{Aprónia}
\begin{itemize}
\item {Grp. gram.:f.}
\end{itemize}
Espécie de bryónia.
\section{Aprontar}
\begin{itemize}
\item {Grp. gram.:v. t.}
\end{itemize}
\begin{itemize}
\item {Proveniência:(De \textunderscore prompto\textunderscore )}
\end{itemize}
Tornar pronto.
Dispor; apparelhar; preparar.
Concluir.
\section{Apropexia}
\begin{itemize}
\item {Grp. gram.:f.}
\end{itemize}
(Fórma ant. de \textunderscore apoplexia\textunderscore . Cf. \textunderscore Eufrosina\textunderscore , 182)
\section{Apropinquação}
\begin{itemize}
\item {Grp. gram.:f.}
\end{itemize}
Acto de \textunderscore apropinquar\textunderscore .
\section{Apropinquar}
\begin{itemize}
\item {Grp. gram.:v. t.}
\end{itemize}
\begin{itemize}
\item {Proveniência:(Lat. \textunderscore appropinquare\textunderscore )}
\end{itemize}
O mesmo que \textunderscore aproximar\textunderscore .
\section{Apropositadamente}
\begin{itemize}
\item {Grp. gram.:adv.}
\end{itemize}
De modo \textunderscore apropositado\textunderscore .
Judiciosamente.
\section{Apropositado}
\begin{itemize}
\item {Grp. gram.:adj.}
\end{itemize}
Que vem a propósito; conveniente, opportuno.
\section{Apropositar}
\begin{itemize}
\item {Grp. gram.:v. t.}
\end{itemize}
\begin{itemize}
\item {Utilização:Des.}
\end{itemize}
Adaptar; appropriar.
Dizer ou fazer a propósito.
\section{A-propósito}
\begin{itemize}
\item {Grp. gram.:loc. adv.}
\end{itemize}
Em occasião opportuna; convenientemente; a talho de foice.
\section{Apropriação}
\begin{itemize}
\item {Grp. gram.:f.}
\end{itemize}
Acto de \textunderscore apropriar\textunderscore .
\section{Apropriadamente}
\begin{itemize}
\item {Grp. gram.:adv.}
\end{itemize}
Com propriedade.
Judiciosamente.
\section{Apropriador}
\begin{itemize}
\item {Grp. gram.:adj.}
\end{itemize}
\begin{itemize}
\item {Grp. gram.:M.}
\end{itemize}
Que apropria.
Official de chapelaria, que faz a apropriagem.
\section{Apropriagem}
\begin{itemize}
\item {Grp. gram.:f.}
\end{itemize}
\begin{itemize}
\item {Proveniência:(De \textunderscore appropriar\textunderscore )}
\end{itemize}
Acabamento do chapéu, depois que vem da fula.
\section{Apropriar}
\begin{itemize}
\item {Grp. gram.:v. t.}
\end{itemize}
\begin{itemize}
\item {Proveniência:(Lat. \textunderscore appropriare\textunderscore )}
\end{itemize}
Tornar próprio.
Acomodar.
Aplicar.
Atribuir.
\section{Aprosado}
\begin{itemize}
\item {Grp. gram.:adj.}
\end{itemize}
Diz-se do verso, em que não há poesia, ou que se parece á prosa. Cf. Filinto, IV, 218.
\section{Aprosexía}
\begin{itemize}
\item {fónica:csi}
\end{itemize}
\begin{itemize}
\item {Grp. gram.:f.}
\end{itemize}
\begin{itemize}
\item {Proveniência:(Do gr. \textunderscore a\textunderscore  priv. + \textunderscore prosexein\textunderscore )}
\end{itemize}
Impossibilidade de fixar a attenção.
\section{Aprostatotrofia}
\begin{itemize}
\item {Grp. gram.:f.}
\end{itemize}
\begin{itemize}
\item {Utilização:Med.}
\end{itemize}
Atrofia da próstata.
\section{Aprostatotrophia}
\begin{itemize}
\item {Grp. gram.:f.}
\end{itemize}
\begin{itemize}
\item {Utilização:Med.}
\end{itemize}
Atrophia da próstata.
\section{Aprovação}
\begin{itemize}
\item {Grp. gram.:f.}
\end{itemize}
\begin{itemize}
\item {Proveniência:(Lat. \textunderscore approbatio\textunderscore )}
\end{itemize}
Acto de \textunderscore aprovar\textunderscore .
\section{Aprovadamente}
\begin{itemize}
\item {Grp. gram.:adv.}
\end{itemize}
Com aprovação.
\section{Aprovado}
\begin{itemize}
\item {Grp. gram.:adj.}
\end{itemize}
Que teve aprovação.
Admittido.
Julgado apto.
\section{Aprovador}
\begin{itemize}
\item {Grp. gram.:m.}
\end{itemize}
\begin{itemize}
\item {Proveniência:(Lat. \textunderscore approbator\textunderscore )}
\end{itemize}
Aquelle que aprova.
\section{Aprovar}
\begin{itemize}
\item {Grp. gram.:v. t.}
\end{itemize}
\begin{itemize}
\item {Proveniência:(Lat. \textunderscore approbare\textunderscore )}
\end{itemize}
Julgar bom: \textunderscore aprovar uma acção\textunderscore .
Louvar.
Ratificar.
Consentir em.
Julgar habilitado (o estudante) na disciplina em que foi examinado.
\section{Aprovativo}
\begin{itemize}
\item {Grp. gram.:adj.}
\end{itemize}
(V.aprobativo)
\section{Aprovável}
\begin{itemize}
\item {Grp. gram.:adj.}
\end{itemize}
\begin{itemize}
\item {Proveniência:(Lat. \textunderscore approbabilis\textunderscore )}
\end{itemize}
Digno de sêr aprovado.
\section{Aproveitação}
\begin{itemize}
\item {Grp. gram.:f.}
\end{itemize}
Acto de \textunderscore aproveitar\textunderscore .
\section{Aproveitadamente}
\begin{itemize}
\item {Grp. gram.:adv.}
\end{itemize}
Económicamente.
Com proveito.
\section{Aproveitado}
\begin{itemize}
\item {Grp. gram.:adj.}
\end{itemize}
\begin{itemize}
\item {Proveniência:(De \textunderscore aproveitar\textunderscore )}
\end{itemize}
Que aproveita tudo, ou de tudo sabe tirar proveito; que tem tido aproveitamento.
Que dá proveito.
Que foi utilizado.
\section{Aproveitador}
\begin{itemize}
\item {Grp. gram.:m.}
\end{itemize}
Aquelle que aproveita.
\section{Aproveitamento}
\begin{itemize}
\item {Grp. gram.:m.}
\end{itemize}
Acto de \textunderscore aproveitar\textunderscore .
\section{Aproveitar}
\begin{itemize}
\item {Grp. gram.:v. t.}
\end{itemize}
\begin{itemize}
\item {Grp. gram.:V. i.}
\end{itemize}
\begin{itemize}
\item {Proveniência:(Do lat. hyp. \textunderscore profectare\textunderscore )}
\end{itemize}
Tirar proveito de.
Tornar proveitoso, tornar útil.
Sêr proveitoso, útil: \textunderscore isso não me aproveita\textunderscore .
Lucrar: \textunderscore hás de aproveitar muito com isso\textunderscore .
\section{Aproveitável}
\begin{itemize}
\item {Grp. gram.:adj.}
\end{itemize}
Que é digno de se aproveitar.
\section{Aproveitível}
\begin{itemize}
\item {Grp. gram.:adj.}
\end{itemize}
\begin{itemize}
\item {Utilização:Ant.}
\end{itemize}
O mesmo que \textunderscore aproveitável\textunderscore . Cf. Gasp. Correia, \textunderscore Lendas da Índia\textunderscore .
\section{Aprovisionamento}
\begin{itemize}
\item {Grp. gram.:m.}
\end{itemize}
Acto ou effeito de \textunderscore aprovisionar\textunderscore .
\section{Aprovisionar}
\begin{itemize}
\item {Grp. gram.:v. t.}
\end{itemize}
\begin{itemize}
\item {Proveniência:(De \textunderscore provisão\textunderscore )}
\end{itemize}
Prover; abastecer.
\section{Aproxar}
\begin{itemize}
\item {Grp. gram.:v. i.}
\end{itemize}
\begin{itemize}
\item {Utilização:Ant.}
\end{itemize}
Fazer aproxes.
\section{Aproxe}
\begin{itemize}
\item {Grp. gram.:m.}
\end{itemize}
\begin{itemize}
\item {Utilização:Des.}
\end{itemize}
\begin{itemize}
\item {Proveniência:(Fr. \textunderscore approche\textunderscore )}
\end{itemize}
Entricheiramento, para facilitar a aproximação ás praças sitiadas.
Investida.
\section{Aproximação}
\begin{itemize}
\item {fónica:si}
\end{itemize}
\begin{itemize}
\item {Grp. gram.:f.}
\end{itemize}
Acto de \textunderscore aproximar\textunderscore .
\section{Aproximadamente}
\begin{itemize}
\item {fónica:si}
\end{itemize}
\begin{itemize}
\item {Grp. gram.:adv.}
\end{itemize}
Com aproximação.
\section{Aproximar}
\begin{itemize}
\item {fónica:si}
\end{itemize}
\begin{itemize}
\item {Grp. gram.:v. t.}
\end{itemize}
Tornar próximo; pôr perto.
Relacionar.
Apressar.
Comparar.
\section{Aproximativo}
\begin{itemize}
\item {fónica:si}
\end{itemize}
\begin{itemize}
\item {Grp. gram.:adj.}
\end{itemize}
Feito por aproximação.
\section{Aprumar}
\begin{itemize}
\item {Grp. gram.:v. t.}
\end{itemize}
\begin{itemize}
\item {Utilização:Fig.}
\end{itemize}
Pôr a prumo.
Endireitar.
Tornar altivo.
\section{Aprumo}
\begin{itemize}
\item {Grp. gram.:m.}
\end{itemize}
Effeito de aprumar.
Altivez.
\section{Apsará}
\begin{itemize}
\item {Grp. gram.:f.}
\end{itemize}
Nympha do céu ou do paraíso dos Índios.
\section{Apseudo}
\begin{itemize}
\item {Grp. gram.:m.}
\end{itemize}
Gênero de crustáceos.
\section{Ápsida}
\begin{itemize}
\item {Grp. gram.:f.}
\end{itemize}
O mesmo que \textunderscore ápside\textunderscore .
\section{Ápside}
\begin{itemize}
\item {Grp. gram.:f.}
\end{itemize}
(V.ábside)
\section{Apsidíolo}
\begin{itemize}
\item {Grp. gram.:m.}
\end{itemize}
\begin{itemize}
\item {Utilização:Ant.}
\end{itemize}
Capella accessória á capella-mór.
(Cp. \textunderscore ápside\textunderscore )
\section{Apsiforoides}
\begin{itemize}
\item {Grp. gram.:m. pl.}
\end{itemize}
Gênero de peixes, cujo corpo é coberto de coiraça escamosa.
\section{Apsiphoroides}
\begin{itemize}
\item {Grp. gram.:m. pl.}
\end{itemize}
Gênero de peixes, cujo corpo é coberto de coiraça escamosa.
\section{Aptá}
\begin{itemize}
\item {Grp. gram.:f.}
\end{itemize}
O mesmo que \textunderscore apteira\textunderscore .
\section{Aptamente}
\begin{itemize}
\item {Grp. gram.:adv.}
\end{itemize}
Com aptidão.
\section{Aptar}
\begin{itemize}
\item {Grp. gram.:v. t.}
\end{itemize}
\begin{itemize}
\item {Utilização:Des.}
\end{itemize}
\begin{itemize}
\item {Proveniência:(Lat. \textunderscore aptare\textunderscore )}
\end{itemize}
Tornar apto.
Adaptar.
\section{Apteira}
\begin{itemize}
\item {Grp. gram.:f.}
\end{itemize}
\begin{itemize}
\item {Proveniência:(De \textunderscore aptá\textunderscore )}
\end{itemize}
Árvore intertropical, (\textunderscore bauhinia parvifolia\textunderscore ), cuja casca se emprega no fabríco de cordas, e de cujas fôlhas se fazem vulgarmente cigarros na Índia Portuguesa.
\section{Aptenoditas}
\begin{itemize}
\item {Grp. gram.:f. pl.}
\end{itemize}
\begin{itemize}
\item {Proveniência:(Do gr. \textunderscore apten\textunderscore  + \textunderscore duein\textunderscore )}
\end{itemize}
Aves palmípedes.
\section{Apterantho}
\begin{itemize}
\item {Grp. gram.:m.}
\end{itemize}
\begin{itemize}
\item {Utilização:Bot.}
\end{itemize}
Gênero de asclepiádeas.
\section{Apteranto}
\begin{itemize}
\item {Grp. gram.:m.}
\end{itemize}
\begin{itemize}
\item {Utilização:Bot.}
\end{itemize}
Gênero de asclepiádeas.
\section{Apterigianos}
\begin{itemize}
\item {Grp. gram.:m. pl.}
\end{itemize}
\begin{itemize}
\item {Proveniência:(Do gr. \textunderscore a\textunderscore  priv. + \textunderscore pterux\textunderscore )}
\end{itemize}
Molluscos, que não têm órgão especial para a natação.
\section{Apterino}
\begin{itemize}
\item {Grp. gram.:m.}
\end{itemize}
Gênero de dípteros.
\section{Aptério}
\begin{itemize}
\item {Grp. gram.:m.}
\end{itemize}
\begin{itemize}
\item {Proveniência:(Do gr. \textunderscore a\textunderscore  priv. + \textunderscore pteron\textunderscore )}
\end{itemize}
Edificio grego, desprovido de columnas.
\section{Áptero}
\begin{itemize}
\item {Grp. gram.:adj.}
\end{itemize}
\begin{itemize}
\item {Grp. gram.:M. pl.}
\end{itemize}
\begin{itemize}
\item {Proveniência:(Gr. \textunderscore apteros\textunderscore )}
\end{itemize}
Que não tem asas.
Certos insectos, desprovidos de asas.
\section{Apterologia}
\begin{itemize}
\item {Grp. gram.:f.}
\end{itemize}
Tratado dos insectos ápteros.
\section{Apterológico}
\begin{itemize}
\item {Grp. gram.:adj.}
\end{itemize}
Relativo á apterologia.
\section{Apterólogo}
\begin{itemize}
\item {Grp. gram.:m.}
\end{itemize}
\begin{itemize}
\item {Proveniência:(Do gr. \textunderscore apteros\textunderscore  + \textunderscore logos\textunderscore )}
\end{itemize}
Aquelle que se occupa de apterologia.
\section{Apterónoto}
\begin{itemize}
\item {Grp. gram.:m.}
\end{itemize}
\begin{itemize}
\item {Proveniência:(Do gr. \textunderscore a\textunderscore  priv. + \textunderscore pteron\textunderscore )}
\end{itemize}
Espécie de enguia sem barbatana dorsal.
\section{Apteruros}
\begin{itemize}
\item {Grp. gram.:m. pl.}
\end{itemize}
\begin{itemize}
\item {Proveniência:(Do gr. \textunderscore a\textunderscore  priv. + \textunderscore pteros\textunderscore  + \textunderscore oura\textunderscore )}
\end{itemize}
Crustáceos decápodes.
Espécie de arraia.
\section{Apterygianos}
\begin{itemize}
\item {Grp. gram.:m. pl.}
\end{itemize}
\begin{itemize}
\item {Proveniência:(Do gr. \textunderscore a\textunderscore  priv. + \textunderscore pterux\textunderscore )}
\end{itemize}
Molluscos, que não têm órgão especial para a natação.
\section{Aptidão}
\begin{itemize}
\item {Grp. gram.:f.}
\end{itemize}
Qualidade do que é \textunderscore apto\textunderscore .
Capacidade; disposição.
\section{Aptificar}
\textunderscore v. t.\textunderscore  (e der.)
O mesmo que \textunderscore aptar\textunderscore , etc.
\section{Aptitude}
\begin{itemize}
\item {Grp. gram.:f.}
\end{itemize}
(V.aptidão)
\section{Apto}
\begin{itemize}
\item {Grp. gram.:adj.}
\end{itemize}
\begin{itemize}
\item {Proveniência:(Lat. \textunderscore aptus\textunderscore )}
\end{itemize}
Idóneo; hábil; capaz.
\section{Aptó}
\begin{itemize}
\item {Grp. gram.:m.}
\end{itemize}
Árvore indiana.
\section{Apuado}
\begin{itemize}
\item {Grp. gram.:adj.}
\end{itemize}
Que tem puas.
Que tem fórma de pua; ponteagudo.
Suppliciado com puas.
\section{Apuamento}
\begin{itemize}
\item {Grp. gram.:m.}
\end{itemize}
Acto de \textunderscore apuar\textunderscore .
\section{Apuar}
\begin{itemize}
\item {Grp. gram.:v. t.}
\end{itemize}
Cravar com puas.
Armar com bicos.
Suppliciar com puas.
\section{Apuava}
\begin{itemize}
\item {Grp. gram.:adj.}
\end{itemize}
\begin{itemize}
\item {Utilização:Bras}
\end{itemize}
Diz-se do cavallo espantado ou pouco dócil.
O mesmo que \textunderscore azuá\textunderscore .
\section{Apuí}
\begin{itemize}
\item {Grp. gram.:m.}
\end{itemize}
Planta urticácea do Pará.
\section{Apular}
\begin{itemize}
\item {Grp. gram.:v. t.}
\end{itemize}
\begin{itemize}
\item {Utilização:T. do Fundão}
\end{itemize}
Aparar ou apanhar no ar (um objecto que vai cair).
(Talvez de \textunderscore pulo\textunderscore )
\section{Apuleia}
\begin{itemize}
\item {Grp. gram.:f.}
\end{itemize}
Gênero de plantas leguminosas da América tropical.
\section{Apulso}
\begin{itemize}
\item {Grp. gram.:m.}
\end{itemize}
\begin{itemize}
\item {Proveniência:(Lat. \textunderscore appulsus\textunderscore )}
\end{itemize}
Passagem da Lua junto de outro astro.
\section{Apunhalado}
\begin{itemize}
\item {Grp. gram.:adj.}
\end{itemize}
\begin{itemize}
\item {Grp. gram.:adj.}
\end{itemize}
Ferido ou morto com punhal.
Diz-se de uma espécie de pombos, que têm uma malha vermelha no peito.
\section{Apunhalante}
\begin{itemize}
\item {Grp. gram.:adj.}
\end{itemize}
\begin{itemize}
\item {Utilização:Neol.}
\end{itemize}
\begin{itemize}
\item {Proveniência:(De \textunderscore apunhalar\textunderscore )}
\end{itemize}
Que magôa muito; que offende, que afflige.
\section{Apunhalar}
\begin{itemize}
\item {Grp. gram.:v. t.}
\end{itemize}
Ferir com punhal; matar com punhal.
Offender gravemente com palavras.
Pungir, magoar muito.
\section{Apunhar}
\begin{itemize}
\item {Grp. gram.:V. i.}
\end{itemize}
\textunderscore v. t.\textunderscore  (e der.)
(V. \textunderscore empunhar\textunderscore , etc.)
Bater com os punhos. Cf. \textunderscore Eufrosina\textunderscore , 21.
\section{Apúnia}
\begin{itemize}
\item {Grp. gram.:f.}
\end{itemize}
Gênero de plantas rubiáceas da América.
\section{Apupada}
\begin{itemize}
\item {Grp. gram.:f.}
\end{itemize}
Acto de \textunderscore apupar\textunderscore .
\section{Apupar}
\begin{itemize}
\item {Grp. gram.:v. t.}
\end{itemize}
\begin{itemize}
\item {Grp. gram.:V. i.}
\end{itemize}
Escarnecer, perseguir com apupos.
Tocar (uma buzina ou trompa), para os monteiros se reunirem:«\textunderscore apupa, apupa, clarim\textunderscore ». Cast., \textunderscore Outono\textunderscore , 165.
\section{Apupo}
\begin{itemize}
\item {Grp. gram.:m.}
\end{itemize}
\begin{itemize}
\item {Proveniência:(Do lat. \textunderscore upupa\textunderscore ?)}
\end{itemize}
Arruaça.
Vozearia de troça; vaia.
\section{Apuração}
\begin{itemize}
\item {Grp. gram.:f.}
\end{itemize}
Acto de \textunderscore apurar\textunderscore .
\section{Apuradamente}
\begin{itemize}
\item {Grp. gram.:adv.}
\end{itemize}
De modo \textunderscore apurado\textunderscore ; com apuro.
\section{Apurado}
\begin{itemize}
\item {Grp. gram.:adj.}
\end{itemize}
Escolhido como melhor.
Delicado: \textunderscore gosto apurado\textunderscore .
Bem vestido; vestido com esmêro.
Esgotado: \textunderscore paciência apurada\textunderscore .
Difficultoso: \textunderscore circunstâncias apuradas\textunderscore .
Em que há apuro, elegância: \textunderscore linguagem apurada\textunderscore .
\section{Apurador}
\begin{itemize}
\item {Grp. gram.:m.  e  adj.}
\end{itemize}
O que apura.
\section{Apuramento}
\begin{itemize}
\item {Grp. gram.:m.}
\end{itemize}
Acto e effeito de \textunderscore apurar\textunderscore .
Contagem, liquidação: \textunderscore apuramento de uma votação\textunderscore .
\section{Apurar}
\begin{itemize}
\item {Grp. gram.:v. t.}
\end{itemize}
\begin{itemize}
\item {Utilização:Bras. de Minas}
\end{itemize}
\begin{itemize}
\item {Grp. gram.:adj.}
\end{itemize}
\begin{itemize}
\item {Proveniência:(De \textunderscore apurar\textunderscore )}
\end{itemize}
Tornar puro.
Aperfeiçoar.
Escolher.
Concluir.
Averiguar: \textunderscore apurar a verdade\textunderscore .
Discutir.
Suavizar.
Recensear.
Tornar distinto, elegante.
Esgotar: \textunderscore apurar a paciência\textunderscore .
Vender, liquidar.Apurativo,
Purificante; depurativo.
\section{Apuridar-se}
\begin{itemize}
\item {Grp. gram.:v. p.}
\end{itemize}
\begin{itemize}
\item {Utilização:Ant.}
\end{itemize}
\begin{itemize}
\item {Proveniência:(De \textunderscore puridade\textunderscore )}
\end{itemize}
Segredar.
\section{Apuro}
\begin{itemize}
\item {Grp. gram.:m.}
\end{itemize}
\begin{itemize}
\item {Grp. gram.:m.}
\end{itemize}
Acto de \textunderscore apurar\textunderscore .
Apuramento.
Correcção no vestir, no falar, etc.
Miséria; situação angustiosa: \textunderscore viu-se em apuros\textunderscore .
Somma de quantias apuradas.
\section{Apurpurado}
\begin{itemize}
\item {Grp. gram.:adj.}
\end{itemize}
Que tem côr de púrpura.
\section{Ápus}
\begin{itemize}
\item {Grp. gram.:m.}
\end{itemize}
Constellação austral.
Nome de um pequeno pássaro.
\section{Aputar-se}
\begin{itemize}
\item {Grp. gram.:v. p.}
\end{itemize}
\begin{itemize}
\item {Utilização:alent}
\end{itemize}
\begin{itemize}
\item {Utilização:Pleb.}
\end{itemize}
O mesmo que \textunderscore amancebar-se\textunderscore .
\section{Apútega}
\begin{itemize}
\item {Grp. gram.:f.}
\end{itemize}
\begin{itemize}
\item {Utilização:Prov.}
\end{itemize}
\begin{itemize}
\item {Utilização:alent.}
\end{itemize}
O mesmo que \textunderscore pútega\textunderscore .
\section{Aputé-jubá}
\begin{itemize}
\item {Grp. gram.:m.}
\end{itemize}
Periquito da América.
\section{Apyra}
\begin{itemize}
\item {Grp. gram.:f.}
\end{itemize}
\begin{itemize}
\item {Proveniência:(De \textunderscore apyro\textunderscore )}
\end{itemize}
Barro infusível, espécie de porcelana, de que se fazem objectos de loiça e de esculptura.
\section{Apyrético}
\begin{itemize}
\item {Grp. gram.:adj.}
\end{itemize}
\begin{itemize}
\item {Proveniência:(De \textunderscore apyrexia\textunderscore )}
\end{itemize}
Que não tem febre.
\section{Apyrexia}
\begin{itemize}
\item {fónica:csi}
\end{itemize}
\begin{itemize}
\item {Grp. gram.:f.}
\end{itemize}
\begin{itemize}
\item {Proveniência:(Gr. \textunderscore apurexia\textunderscore )}
\end{itemize}
Estado do enfêrmo, nos intervallos dos accessos febris.
\section{Apyro}
\begin{itemize}
\item {Grp. gram.:adj.}
\end{itemize}
\begin{itemize}
\item {Proveniência:(Gr. \textunderscore apuros\textunderscore )}
\end{itemize}
Que resiste ao fogo.
\section{Aquadrelamento}
\begin{itemize}
\item {Grp. gram.:m.}
\end{itemize}
\begin{itemize}
\item {Utilização:Ant.}
\end{itemize}
\begin{itemize}
\item {Proveniência:(De \textunderscore aquadrelar\textunderscore )}
\end{itemize}
Enumeração, rol.
\section{Aquadrelar}
\begin{itemize}
\item {Grp. gram.:v. t.}
\end{itemize}
\begin{itemize}
\item {Utilização:Ant.}
\end{itemize}
Dividir em quadrelas.
\section{Aquadrilhamento}
\begin{itemize}
\item {Grp. gram.:m.}
\end{itemize}
Acto de \textunderscore aquadrilhar\textunderscore .
\section{Aquadrilhar}
\begin{itemize}
\item {Grp. gram.:v. t.}
\end{itemize}
Formar em quadrilha.
Alistar.
\section{Aquadrimar-se}
\begin{itemize}
\item {Grp. gram.:v. p.}
\end{itemize}
\begin{itemize}
\item {Utilização:Prov.}
\end{itemize}
\begin{itemize}
\item {Utilização:minh.}
\end{itemize}
O mesmo que \textunderscore acadimar-se\textunderscore .
\section{Aquando}
\begin{itemize}
\item {Grp. gram.:conj.}
\end{itemize}
\begin{itemize}
\item {Utilização:pop.}
\end{itemize}
\begin{itemize}
\item {Utilização:Ant.}
\end{itemize}
Ao mesmo tempo que.
Quando.
\section{Aquaquá}
\begin{itemize}
\item {Grp. gram.:m.}
\end{itemize}
Espécie de sapo do Brasil.
\section{Aquarela}
\begin{itemize}
\item {Grp. gram.:f.}
\end{itemize}
(V.aguarela)
\section{Aquário}
\begin{itemize}
\item {Grp. gram.:m.}
\end{itemize}
\begin{itemize}
\item {Grp. gram.:Adj.}
\end{itemize}
\begin{itemize}
\item {Proveniência:(Lat. \textunderscore aquarius\textunderscore )}
\end{itemize}
Reservatório, onde se conservam ou criam plantas ou animaes que vivem na água.
Um dos signos do zodíaco.
Que vive na água.
\section{Aquartalado}
\begin{itemize}
\item {Grp. gram.:adj.}
\end{itemize}
Diz-se do cavallo, que tem os quartos fortes e baixos.
(Por \textunderscore aquartelado\textunderscore , de \textunderscore quartel\textunderscore  por \textunderscore quarto\textunderscore )
\section{Aquartelado}
\begin{itemize}
\item {Grp. gram.:adj.}
\end{itemize}
Dividido em quartéis.
Recolhido em quartéis.
\section{Aquartelamento}
\begin{itemize}
\item {Grp. gram.:m.}
\end{itemize}
Acto de \textunderscore aquartelar\textunderscore .
\section{Aquartelar}
\begin{itemize}
\item {Grp. gram.:v. t.}
\end{itemize}
\begin{itemize}
\item {Utilização:Heráld.}
\end{itemize}
\begin{itemize}
\item {Grp. gram.:V. i.}
\end{itemize}
Meter em quartéis.
Alojar; aboletar.
Dividir em quartéis.
Alojar-se em quartéis.
\section{Aquartilhar}
\begin{itemize}
\item {Grp. gram.:v. t.}
\end{itemize}
Vender por miúdo, aos quartilhos.
\section{Aquático}
\begin{itemize}
\item {Grp. gram.:adj.}
\end{itemize}
\begin{itemize}
\item {Proveniência:(Lat. \textunderscore aquaticus\textunderscore )}
\end{itemize}
Pertencente á água.
Que vive na água ou sôbre a água.
\section{Aquátil}
\begin{itemize}
\item {Grp. gram.:adj.}
\end{itemize}
\begin{itemize}
\item {Proveniência:(Lat. \textunderscore aquatilis\textunderscore )}
\end{itemize}
O mesmo que \textunderscore aquático\textunderscore .
\section{Aquebrantar}
\begin{itemize}
\item {Grp. gram.:v. t.}
\end{itemize}
(V.quebrantar)
\section{Aquecedor}
\begin{itemize}
\item {Grp. gram.:m.}
\end{itemize}
Maquinismo para aquecer casas, etc.
\section{Aquecer}
\begin{itemize}
\item {fónica:qué-cêr}
\end{itemize}
\begin{itemize}
\item {Grp. gram.:v. t.}
\end{itemize}
\begin{itemize}
\item {Grp. gram.:V. i.}
\end{itemize}
\begin{itemize}
\item {Proveniência:(Lat. \textunderscore calescere\textunderscore )}
\end{itemize}
Tornar quente.
Exaltar; irritar.
Tornar-se quente.
Exaltar-se.
\section{Aquecer}
\begin{itemize}
\item {Grp. gram.:v. i.}
\end{itemize}
\begin{itemize}
\item {Utilização:Ant.}
\end{itemize}
O mesmo que \textunderscore acaecer\textunderscore .
\section{Aquecimento}
\begin{itemize}
\item {fónica:qué}
\end{itemize}
\begin{itemize}
\item {Grp. gram.:m.}
\end{itemize}
Acto ou effeito de \textunderscore aquecer\textunderscore .
\section{Aquecível}
\begin{itemize}
\item {fónica:qué}
\end{itemize}
\begin{itemize}
\item {Grp. gram.:adj.}
\end{itemize}
Que se póde aquecer.
\section{Aquecodola}
\begin{itemize}
\item {Grp. gram.:f.}
\end{itemize}
\begin{itemize}
\item {Utilização:Prov.}
\end{itemize}
\begin{itemize}
\item {Utilização:trasm.}
\end{itemize}
Tosa, tunda.
\section{Aquedar}
\begin{itemize}
\item {Grp. gram.:v. t.}
\end{itemize}
\begin{itemize}
\item {Grp. gram.:V. i.}
\end{itemize}
Tornar quêdo ou sossegado.
Parar; o mesmo que \textunderscore quedar\textunderscore .
\section{Aquedático}
\textunderscore adj.\textunderscore  (?)«\textunderscore ...aquedáticas tormas vai fugando\textunderscore ». \textunderscore Viriato Trág.\textunderscore , XIV, 84.
\section{Aqueducto}
\begin{itemize}
\item {Grp. gram.:m.}
\end{itemize}
\begin{itemize}
\item {Utilização:Anat.}
\end{itemize}
\begin{itemize}
\item {Proveniência:(Lat. \textunderscore aquaeductus\textunderscore )}
\end{itemize}
Encanamento por onde correm águas.
Nome de vários canaes, como \textunderscore aqueducto de Fallópio\textunderscore , no interior do rochedo, a communicação do terceiro ventrículo com o quarto, etc.
\section{Aqueduto}
\begin{itemize}
\item {Grp. gram.:m.}
\end{itemize}
\begin{itemize}
\item {Utilização:Anat.}
\end{itemize}
\begin{itemize}
\item {Proveniência:(Lat. \textunderscore aquaeductus\textunderscore )}
\end{itemize}
Encanamento por onde correm águas.
Nome de vários canaes, como \textunderscore aqueduto de Fallópio\textunderscore , no interior do rochedo, a communicação do terceiro ventrículo com o quarto, etc.
\section{Aqueijar-se}
\begin{itemize}
\item {Grp. gram.:v. p.}
\end{itemize}
\begin{itemize}
\item {Utilização:Ant.}
\end{itemize}
Apressar-se.
\section{Aqueivar-se}
\begin{itemize}
\item {Grp. gram.:v. p.}
\end{itemize}
\begin{itemize}
\item {Utilização:Ant.}
\end{itemize}
Aquietar-se, serenar.
\section{Aqueixar-se}
\begin{itemize}
\item {Grp. gram.:v. p.}
\end{itemize}
\begin{itemize}
\item {Utilização:Ant.}
\end{itemize}
Queixar-se. Cf. \textunderscore Eufrosina\textunderscore , 340.
\section{Aquela}
\begin{itemize}
\item {Grp. gram.:f.}
\end{itemize}
\begin{itemize}
\item {Utilização:Fam.}
\end{itemize}
\begin{itemize}
\item {Proveniência:(De \textunderscore aquelle\textunderscore )}
\end{itemize}
Manía.
Ideia, opinião.
Acanhamento, ceremónia.
\section{Aquelar}
\begin{itemize}
\item {Grp. gram.:v. t.}
\end{itemize}
\begin{itemize}
\item {Utilização:Prov.}
\end{itemize}
\begin{itemize}
\item {Utilização:minh.}
\end{itemize}
\begin{itemize}
\item {Proveniência:(De \textunderscore aquella\textunderscore )}
\end{itemize}
Arranjar; fazer.
Atinar com.
\section{Aquele}
\begin{itemize}
\item {Grp. gram.:pron.}
\end{itemize}
\begin{itemize}
\item {Proveniência:(Do lat. \textunderscore eccum\textunderscore  + \textunderscore ille\textunderscore )}
\end{itemize}
(designativo da pessôa ou objecto que está um pouco distante de quem fala)
\section{Aquelho}
\begin{itemize}
\item {Grp. gram.:pron.}
\end{itemize}
\begin{itemize}
\item {Utilização:Ant.}
\end{itemize}
Aquillo.
\section{Aquella}
\begin{itemize}
\item {Grp. gram.:f.}
\end{itemize}
\begin{itemize}
\item {Utilização:Fam.}
\end{itemize}
\begin{itemize}
\item {Proveniência:(De \textunderscore aquelle\textunderscore )}
\end{itemize}
Manía.
Ideia, opinião.
Acanhamento, ceremónia.
\section{Aquellar}
\begin{itemize}
\item {Grp. gram.:v. t.}
\end{itemize}
\begin{itemize}
\item {Utilização:Prov.}
\end{itemize}
\begin{itemize}
\item {Utilização:minh.}
\end{itemize}
\begin{itemize}
\item {Proveniência:(De \textunderscore aquella\textunderscore )}
\end{itemize}
Arranjar; fazer.
Atinar com.
\section{Aquelle}
\begin{itemize}
\item {Grp. gram.:pron.}
\end{itemize}
\begin{itemize}
\item {Proveniência:(Do lat. \textunderscore eccum\textunderscore  + \textunderscore ille\textunderscore )}
\end{itemize}
(designativo da pessôa ou objecto que está um pouco distante de quem fala)
\section{Aquelloutro}
\begin{itemize}
\item {Grp. gram.:adj.  e  pron.}
\end{itemize}
(contr. de \textunderscore aquelle\textunderscore  e \textunderscore outro\textunderscore )
\section{Aqueloutro}
\begin{itemize}
\item {Grp. gram.:adj.  e  pron.}
\end{itemize}
(contr. de \textunderscore aquele\textunderscore  e \textunderscore outro\textunderscore )
\section{Àquém}
\begin{itemize}
\item {Grp. gram.:adv.  e  pron.}
\end{itemize}
\begin{itemize}
\item {Proveniência:(Do lat. \textunderscore eccum\textunderscore  + \textunderscore inde\textunderscore ?)}
\end{itemize}
Do lado de cá.
Inferiormente; menos.
\section{Aqueme}
\begin{itemize}
\item {Grp. gram.:m.}
\end{itemize}
\begin{itemize}
\item {Utilização:Ant.}
\end{itemize}
Maioral, chefe.
\section{Aquemeneres}
\begin{itemize}
\item {Grp. gram.:adv.}
\end{itemize}
O mesmo que \textunderscore aquemunéris\textunderscore .
\section{Aquemunéris}
\begin{itemize}
\item {Grp. gram.:adv.}
\end{itemize}
\begin{itemize}
\item {Utilização:Fam.}
\end{itemize}
\begin{itemize}
\item {Proveniência:(Do al. \textunderscore ia\textunderscore  + \textunderscore mein\textunderscore  + \textunderscore herr\textunderscore , sim, meu senhor)}
\end{itemize}
Assim mesmo; exactamente.
\section{Aquênio}
\begin{itemize}
\item {Grp. gram.:m.}
\end{itemize}
\begin{itemize}
\item {Utilização:Bot.}
\end{itemize}
\begin{itemize}
\item {Proveniência:(Do gr. \textunderscore a\textunderscore  priv. + \textunderscore khainein\textunderscore )}
\end{itemize}
Fruto monospermo, cujo pericarpo é distinto do tegumento próprio da semente.
\section{Aquênio}
\begin{itemize}
\item {Grp. gram.:adj.}
\end{itemize}
\begin{itemize}
\item {Grp. gram.:M.}
\end{itemize}
Diz-se de um período geológico, criado por Dumond.
Terreno na base da série infra-cretácea, constituído por um conjunto de areias brancas ou ferruginosas e de argila, que cobrem directamente as camadas carboníferas.
(Por \textunderscore aachênio\textunderscore , de \textunderscore Aachen\textunderscore , n. p. al. de Aix-la-Chapelle)
\section{Aquentamento}
\begin{itemize}
\item {Grp. gram.:m.}
\end{itemize}
Acto de \textunderscore aquentar\textunderscore .
\section{Aquentar}
\begin{itemize}
\item {Grp. gram.:v. t.}
\end{itemize}
Tornar quente.
Animar; estimular.
\section{Áqueo}
\begin{itemize}
\item {Grp. gram.:adj.}
\end{itemize}
O mesmo que \textunderscore aquoso\textunderscore .
\section{Aquerenciar-se}
\begin{itemize}
\item {Grp. gram.:v. p.}
\end{itemize}
\begin{itemize}
\item {Utilização:Bras}
\end{itemize}
\begin{itemize}
\item {Proveniência:(De \textunderscore querença\textunderscore )}
\end{itemize}
Acostumar-se a determinado lugar.
Acostumar-se a acompanhar outro ou a viver com outro, (falando-se de animaes).
\section{Aqueronte}
\begin{itemize}
\item {Grp. gram.:m.}
\end{itemize}
\begin{itemize}
\item {Proveniência:(Lat. \textunderscore acheron, acherontis\textunderscore )}
\end{itemize}
O inferno mythológico.
\section{Aqueronteu}
\begin{itemize}
\item {Grp. gram.:adj.}
\end{itemize}
O mesmo que \textunderscore aquerôntico\textunderscore .
\section{Aquerôntia}
\begin{itemize}
\item {Grp. gram.:f.}
\end{itemize}
Insecto, cuja larva ataca e destrói as flôres do tabaco.
\section{Aquerôntico}
\begin{itemize}
\item {Grp. gram.:adj.}
\end{itemize}
Relativo ao aqueronte.
\section{Aquesse}
\begin{itemize}
\item {fónica:quê}
\end{itemize}
\begin{itemize}
\item {Grp. gram.:pron.}
\end{itemize}
\begin{itemize}
\item {Utilização:Ant.}
\end{itemize}
O mesmo que \textunderscore êsse\textunderscore .
\section{Aqueste}
\begin{itemize}
\item {fónica:quês}
\end{itemize}
\begin{itemize}
\item {Grp. gram.:pron.}
\end{itemize}
\begin{itemize}
\item {Utilização:Ant.}
\end{itemize}
O mesmo que \textunderscore êste\textunderscore .
\section{Aquesto}
\begin{itemize}
\item {fónica:quês}
\end{itemize}
\begin{itemize}
\item {Grp. gram.:pron.}
\end{itemize}
\begin{itemize}
\item {Utilização:Ant.}
\end{itemize}
O mesmo que \textunderscore isto\textunderscore .
\section{Aquetar}
\begin{itemize}
\item {fónica:qué}
\end{itemize}
\begin{itemize}
\item {Grp. gram.:v. t.}
\end{itemize}
\begin{itemize}
\item {Utilização:Prov.}
\end{itemize}
\begin{itemize}
\item {Utilização:beir.}
\end{itemize}
O mesmo que \textunderscore aquietar\textunderscore .
\section{Aqueu}
\begin{itemize}
\item {Grp. gram.:adj.}
\end{itemize}
\begin{itemize}
\item {Grp. gram.:M.}
\end{itemize}
\begin{itemize}
\item {Grp. gram.:Pl.}
\end{itemize}
\begin{itemize}
\item {Proveniência:(Lat. \textunderscore achaeus\textunderscore )}
\end{itemize}
Relativo á Achaia; achaico.
Habitante da Achaia.
Antigo povo da Grécia, o mesmo que \textunderscore achivos\textunderscore .
\section{Aqui}
\begin{itemize}
\item {Grp. gram.:adv.}
\end{itemize}
\begin{itemize}
\item {Proveniência:(Do lat. \textunderscore eccum\textunderscore  + \textunderscore hic\textunderscore )}
\end{itemize}
Neste lugar.
Nesta occasião.
A êste lugar: \textunderscore veio aqui\textunderscore .
Nisto.
\section{Áqui}
\begin{itemize}
\item {Grp. gram.:m.}
\end{itemize}
Planta trepadeira da ilha de San-Thomé.
\section{Aquícola}
\begin{itemize}
\item {fónica:cu-i}
\end{itemize}
\begin{itemize}
\item {Grp. gram.:adj.}
\end{itemize}
\begin{itemize}
\item {Grp. gram.:adj.}
\end{itemize}
\begin{itemize}
\item {Grp. gram.:M.}
\end{itemize}
\begin{itemize}
\item {Proveniência:(Do lat. \textunderscore aqua\textunderscore  + \textunderscore colere\textunderscore )}
\end{itemize}
Relativo á aquicultura.
Que vive na água.
Habitante da água.
\section{Aquicultura}
\begin{itemize}
\item {fónica:cu-i}
\end{itemize}
\begin{itemize}
\item {Grp. gram.:f.}
\end{itemize}
\begin{itemize}
\item {Proveniência:(Do lat. \textunderscore aqua\textunderscore  + \textunderscore cultura\textunderscore )}
\end{itemize}
Tratamento dos rios, lagos e esteiros, para a bôa producção piscatória.
\section{Aqui-del-rei!}
\begin{itemize}
\item {Grp. gram.:interj.}
\end{itemize}
Grito de soccorro, em que se invocava a fôrça pública.
\section{Aquiescência}
\begin{itemize}
\item {Grp. gram.:f.}
\end{itemize}
Acto de \textunderscore aquiescêr\textunderscore .
\section{Aquiescêr}
\begin{itemize}
\item {Grp. gram.:v. t.}
\end{itemize}
\begin{itemize}
\item {Proveniência:(Lat. \textunderscore acquiescere\textunderscore )}
\end{itemize}
Annuir, transigir.
\section{Aquietação}
\begin{itemize}
\item {Grp. gram.:f.}
\end{itemize}
Acto ou effeito de \textunderscore aquietar\textunderscore .
\section{Aquietador}
\begin{itemize}
\item {Grp. gram.:m.  e  adj.}
\end{itemize}
O que aquieta.
\section{Aquietar}
\begin{itemize}
\item {Grp. gram.:v. t.}
\end{itemize}
\begin{itemize}
\item {Grp. gram.:V. i.}
\end{itemize}
Tornar quieto.
Apaziguar; serenar: tranquillizar.
Aquietar-se:«\textunderscore não aquieta o pó\textunderscore ». Padre Vieira.
\section{Aquífero}
\begin{itemize}
\item {Grp. gram.:adj.}
\end{itemize}
\begin{itemize}
\item {Proveniência:(Do lat. \textunderscore aqua\textunderscore  + \textunderscore ferre\textunderscore )}
\end{itemize}
Que tem água.
\section{Aquifoliáceas}
\begin{itemize}
\item {Grp. gram.:f. pl.}
\end{itemize}
Família de plantas, que tem por typo o aquifólio.
\section{Aquifólio}
\begin{itemize}
\item {Grp. gram.:m.}
\end{itemize}
\begin{itemize}
\item {Proveniência:(Lat. \textunderscore aquifolium\textunderscore )}
\end{itemize}
Nome scientifico do azevinho.
\section{Aquilão}
\begin{itemize}
\item {Grp. gram.:m.}
\end{itemize}
\begin{itemize}
\item {Proveniência:(Lat. \textunderscore aquilo\textunderscore )}
\end{itemize}
Vento norte.
Região setentrional.
O vento do Nordeste, segundo a Náutica antiga.
\section{Aquilão}
\begin{itemize}
\item {Grp. gram.:m.}
\end{itemize}
\begin{itemize}
\item {Utilização:Bras}
\end{itemize}
Unguento, semelhante ao basalicão.
\section{Aquilária}
\begin{itemize}
\item {Grp. gram.:f.}
\end{itemize}
Árvore indiana.
Monstruosidade vegetal, caracterizada pela ausência accidental dos lábios em corollas que normalmente os têm.
O mesmo que \textunderscore águila\textunderscore .
\section{Aquilaríneas}
\begin{itemize}
\item {Grp. gram.:f. pl.}
\end{itemize}
\begin{itemize}
\item {Proveniência:(De \textunderscore aquilária\textunderscore )}
\end{itemize}
Familia de plantas dicotyledóneas, de casca flexível, ramos lisos e pecíolos curtos.
\section{Aquilatador}
\begin{itemize}
\item {Grp. gram.:m.}
\end{itemize}
Aquelle que aquilata.
\section{Aquilatar}
\begin{itemize}
\item {Grp. gram.:v. t.}
\end{itemize}
Determinar o quilate de.
Avaliar: apreciar.
Aperfeiçoar.
\section{Aquíleas}
\begin{itemize}
\item {Grp. gram.:f. pl.}
\end{itemize}
\begin{itemize}
\item {Proveniência:(De \textunderscore achilleia\textunderscore )}
\end{itemize}
Grupo de plantas corymbíferas, segundo Jussieu.
\section{Aquilégia}
\begin{itemize}
\item {Grp. gram.:f.}
\end{itemize}
\begin{itemize}
\item {Proveniência:(Do lat. \textunderscore agua\textunderscore  + \textunderscore legere\textunderscore )}
\end{itemize}
Planta ornamental, ranunculácea.
\section{Aquilégio}
\begin{itemize}
\item {Grp. gram.:m.}
\end{itemize}
\begin{itemize}
\item {Proveniência:(Lat. \textunderscore aquillex\textunderscore )}
\end{itemize}
Indivíduo que, entre os Romanos, descobria nascentes de água, ou tinha a seu cargo a guarda e fiscalização das fontes.
\section{Aquileia}
\begin{itemize}
\item {Grp. gram.:f.}
\end{itemize}
\begin{itemize}
\item {Proveniência:(Gr. \textunderscore akhilleia\textunderscore )}
\end{itemize}
Planta, de flôres radiadas, dispostas em corymbo.
\section{Aquileico}
\begin{itemize}
\item {Grp. gram.:adj.}
\end{itemize}
Diz-se do ácido que existe na aquileia.
\section{Aquileja}
\begin{itemize}
\item {Grp. gram.:f.}
\end{itemize}
O mesmo que \textunderscore aquileia\textunderscore .
\section{Aquilhado}
\begin{itemize}
\item {Grp. gram.:adj.}
\end{itemize}
Que tem quilha.
Semelhante a quilha.
\section{Aquilhão}
\begin{itemize}
\item {Grp. gram.:m.}
\end{itemize}
\begin{itemize}
\item {Utilização:Prov.}
\end{itemize}
\begin{itemize}
\item {Utilização:trasm.}
\end{itemize}
O jôgo do moínho.
\section{Aquilia}
\begin{itemize}
\item {Grp. gram.:f.}
\end{itemize}
\begin{itemize}
\item {Proveniência:(Do gr. \textunderscore a\textunderscore  priv. + \textunderscore kheilos\textunderscore , lábio)}
\end{itemize}
Monstruosidade, caracterizada pela falta de lábios.
\section{Aquilífero}
\begin{itemize}
\item {Grp. gram.:adj.}
\end{itemize}
\begin{itemize}
\item {Utilização:Ant.}
\end{itemize}
\begin{itemize}
\item {Grp. gram.:M.}
\end{itemize}
\begin{itemize}
\item {Utilização:Ant.}
\end{itemize}
\begin{itemize}
\item {Proveniência:(Lat. \textunderscore aquilifer\textunderscore )}
\end{itemize}
Que tem pintadas as águias ou as armas romanas.
Porta-bandeira.
\section{Aquilino}
\begin{itemize}
\item {Grp. gram.:adj.}
\end{itemize}
\begin{itemize}
\item {Proveniência:(Do lat. \textunderscore aquila\textunderscore )}
\end{itemize}
Próprio da águia: pertencente á águia.
Recurvo, como o bico da águia: \textunderscore nariz aquilino\textunderscore .
Penetrante, como os olhos de águia: \textunderscore olhar aquilino\textunderscore .
\section{Aquillo}
\begin{itemize}
\item {Grp. gram.:pron.}
\end{itemize}
\begin{itemize}
\item {Proveniência:(Do lat. \textunderscore eccum\textunderscore  + \textunderscore illud\textunderscore )}
\end{itemize}
Aquella coisa.
Aquellas coisas.
Aquella pessôa.
\section{Aquilo}
\begin{itemize}
\item {Grp. gram.:pron.}
\end{itemize}
\begin{itemize}
\item {Proveniência:(Do lat. \textunderscore eccum\textunderscore  + \textunderscore illud\textunderscore )}
\end{itemize}
Aquela coisa.
Aquelas coisas.
Aquela pessôa.
\section{Aquilombado}
\begin{itemize}
\item {Grp. gram.:adj.}
\end{itemize}
\begin{itemize}
\item {Utilização:Bras}
\end{itemize}
Que se refugiou em quilombo.
\section{Aquilombar}
\begin{itemize}
\item {Grp. gram.:v. t.}
\end{itemize}
\begin{itemize}
\item {Utilização:Bras}
\end{itemize}
Reünir em quilombo (escravos fugitivos).
\section{Aquilonal}
\begin{itemize}
\item {Grp. gram.:adj.}
\end{itemize}
O mesmo que \textunderscore aquilonar\textunderscore .
\section{Aquilonar}
\begin{itemize}
\item {Grp. gram.:adj.}
\end{itemize}
\begin{itemize}
\item {Proveniência:(Lat. \textunderscore aquilonaris\textunderscore )}
\end{itemize}
O mesmo que \textunderscore aquilónio\textunderscore .
\section{Aquilónio}
\begin{itemize}
\item {Grp. gram.:adj.}
\end{itemize}
\begin{itemize}
\item {Proveniência:(Lat. \textunderscore aquilonius\textunderscore )}
\end{itemize}
Relativo ao aquilão^1.
\section{Aquinhoador}
\begin{itemize}
\item {Grp. gram.:m.}
\end{itemize}
Aquelle que aquinhôa.
\section{Aquinhoamento}
\begin{itemize}
\item {Grp. gram.:m.}
\end{itemize}
Acto de \textunderscore aquinhoar\textunderscore .
\section{Aquinhoar}
\begin{itemize}
\item {Grp. gram.:v. t.}
\end{itemize}
Dividir em quinhões.
Dar em quinhão.
Partilhar.
Tomar parte de.
\section{Á-qui-qui}
\begin{itemize}
\item {fónica:cu-i-cu-i}
\end{itemize}
\begin{itemize}
\item {Grp. gram.:adv.}
\end{itemize}
\begin{itemize}
\item {Utilização:Chul.}
\end{itemize}
Isso mesmo, exactamente.
\section{Aquiqui}
\begin{itemize}
\item {Grp. gram.:m.}
\end{itemize}
Espécie de macaco do Brazil.
\section{Aquiranto}
\begin{itemize}
\item {Grp. gram.:m.}
\end{itemize}
\begin{itemize}
\item {Proveniência:(Do gr. \textunderscore akhuronanthos\textunderscore )}
\end{itemize}
Gênero de plantas amarantáceas.
\section{Aquirastro}
\begin{itemize}
\item {Grp. gram.:m.}
\end{itemize}
\begin{itemize}
\item {Proveniência:(Do gr. \textunderscore akhuronastron\textunderscore )}
\end{itemize}
Planta do grupo das chicoriáceas, e cujo cálice tem a fórma de martinete.
\section{Aquirente}
\begin{itemize}
\item {Grp. gram.:adj.}
\end{itemize}
Que aquire.
\section{Aquirição}
\begin{itemize}
\item {Grp. gram.:f.}
\end{itemize}
\begin{itemize}
\item {Utilização:Ant.}
\end{itemize}
O mesmo que \textunderscore aquisição\textunderscore ^2.
\section{Aquiridor}
\begin{itemize}
\item {Grp. gram.:m.}
\end{itemize}
O que aquire.
\section{Aquirimento}
\begin{itemize}
\item {Grp. gram.:m.}
\end{itemize}
\begin{itemize}
\item {Utilização:Ant.}
\end{itemize}
O mesmo que \textunderscore aquisição\textunderscore ^2.
\section{Aquirir}
\begin{itemize}
\item {Grp. gram.:v. t.}
\end{itemize}
\begin{itemize}
\item {Utilização:Ant.}
\end{itemize}
O mesmo que \textunderscore adquirir\textunderscore .
\section{Aquiro}
\begin{itemize}
\item {Grp. gram.:m.}
\end{itemize}
\begin{itemize}
\item {Proveniência:(Do gr. \textunderscore a\textunderscore  priv. + \textunderscore ckeir\textunderscore , mão)}
\end{itemize}
Peixe pleuronecto, semelhante ao linguado.
\section{Aquiróforo}
\begin{itemize}
\item {Grp. gram.:m.}
\end{itemize}
\begin{itemize}
\item {Proveniência:(Do gr. \textunderscore akhuronphoros\textunderscore )}
\end{itemize}
Gênero de plantas compostas.
\section{Aquirófito}
\begin{itemize}
\item {Grp. gram.:adj.}
\end{itemize}
\begin{itemize}
\item {Proveniência:(Do gr. \textunderscore akhuronphuton\textunderscore )}
\end{itemize}
Diz-se da planta, cuja flôr é composta de palhetas.
\section{Aquirospermo}
\begin{itemize}
\item {Proveniência:(Do gr. \textunderscore akhuronsperma\textunderscore )}
\end{itemize}
Gênero de plantas labiadas.
\section{Aquisição}
\begin{itemize}
\item {Grp. gram.:f.}
\end{itemize}
O mesmo que \textunderscore acquisição\textunderscore .
\section{Aquisição}
\begin{itemize}
\item {Grp. gram.:f.}
\end{itemize}
Acto ou effeito de \textunderscore aquirir\textunderscore .
\section{Aquista}
\begin{itemize}
\item {fónica:cu-is}
\end{itemize}
\begin{itemize}
\item {Grp. gram.:m.  e  f.}
\end{itemize}
\begin{itemize}
\item {Proveniência:(Do lat. \textunderscore aqua\textunderscore )}
\end{itemize}
Pessôa, que faz uso de águas medicinaes, na localidade onde ellas nascem.
\section{Aquistar}
\begin{itemize}
\item {Grp. gram.:v. t.}
\end{itemize}
Adquirir, ganhar:«\textunderscore que grande autoridade logo aquista\textunderscore », \textunderscore Lusíadas\textunderscore , VII, 59.
(Cp. \textunderscore acquistar\textunderscore )
\section{Aquivo}
\begin{itemize}
\item {Grp. gram.:m. e adj.}
\end{itemize}
\begin{itemize}
\item {Proveniência:(Lat. \textunderscore achivus\textunderscore )}
\end{itemize}
O mesmo que \textunderscore grego\textunderscore  da Thessália ou do Peloponneso; grego.
\section{Aquosidade}
\begin{itemize}
\item {Grp. gram.:m.}
\end{itemize}
Qualidade de que é \textunderscore aquoso\textunderscore .
\section{Aquoso}
\begin{itemize}
\item {Grp. gram.:adj.}
\end{itemize}
\begin{itemize}
\item {Proveniência:(Lat. \textunderscore aquosus\textunderscore )}
\end{itemize}
Que tem água.
Semelhante á água.
Da natureza da água.
\section{Aquotiar}
\textunderscore v. t.\textunderscore  (e der)
O mesmo que \textunderscore acotiar\textunderscore , etc.
\section{Ar}
\begin{itemize}
\item {Grp. gram.:m.}
\end{itemize}
\begin{itemize}
\item {Proveniência:(Lat. \textunderscore aer\textunderscore )}
\end{itemize}
Fluido transparente e invisível, que fórma a atmosphera.
O espaço sôbre a terra.
Viração.
Clima: \textunderscore aquella região tem bom ar\textunderscore .
Apparência.
Figura; parecença: \textunderscore tem ar de bôa pessôa\textunderscore .
Ataque de paralysia: \textunderscore deu-lhe um ar\textunderscore .
Vácuo.
\section{...ar}
\begin{itemize}
\item {Grp. gram.:suf. verbal  suf. subst.  e  adj.}
\end{itemize}

\section{Ara}
\begin{itemize}
\item {Grp. gram.:f.}
\end{itemize}
\begin{itemize}
\item {Proveniência:(Lat. \textunderscore ara\textunderscore )}
\end{itemize}
Altar.
Pedra do altar.
Lugar do sacrifício.
Constellação austral.
\section{Ará}
\begin{itemize}
\item {Grp. gram.:m.}
\end{itemize}
Unidade das medidas de capacidade para sólidos, no territorio de Damão, e igual a 32 parás.
\section{Araan!}
\begin{itemize}
\item {Grp. gram.:interj.}
\end{itemize}
\begin{itemize}
\item {Utilização:bras}
\end{itemize}
(designativa de saudação ou de surpresa agradável)
(Do tupi)
\section{Arabata}
\begin{itemize}
\item {Grp. gram.:m.}
\end{itemize}
\begin{itemize}
\item {Grp. gram.:F.}
\end{itemize}
Espécie de macaco.
Cotovía da América.
\section{Árabe}
\begin{itemize}
\item {Grp. gram.:m.}
\end{itemize}
\begin{itemize}
\item {Grp. gram.:Adj.}
\end{itemize}
\begin{itemize}
\item {Proveniência:(Lat. \textunderscore arabs\textunderscore , \textunderscore arabis\textunderscore )}
\end{itemize}
Aquelle que é natural da Arábia.
Idioma dos Árabes e de alguns povos muçulmanos.
Relativo á Arábia.
\section{Arabesca}
\begin{itemize}
\item {fónica:bês}
\end{itemize}
\begin{itemize}
\item {Grp. gram.:f.}
\end{itemize}
\begin{itemize}
\item {Utilização:Ant.}
\end{itemize}
Ornamento, usado em esteiras e formado de 10 ou 12 fios.
Mulhér árabe. Cf. Pant. de Aveiro, \textunderscore Itiner.\textunderscore , 23.^o, (2.^a ed.).
(Cp. \textunderscore arabesco\textunderscore )
\section{Arabesco}
\begin{itemize}
\item {fónica:bês}
\end{itemize}
\begin{itemize}
\item {Grp. gram.:m.}
\end{itemize}
\begin{itemize}
\item {Grp. gram.:Adj.}
\end{itemize}
\begin{itemize}
\item {Proveniência:(De \textunderscore árabe\textunderscore )}
\end{itemize}
Ornato, que imita fôlhas, flôres ou frutos, em pintura e em esculptura.
Relativo a Árabes.
\section{Arabeta}
\begin{itemize}
\item {fónica:bê}
\end{itemize}
\begin{itemize}
\item {Grp. gram.:f.}
\end{itemize}
\begin{itemize}
\item {Proveniência:(Do gr. \textunderscore arabeein\textunderscore )}
\end{itemize}
Insecto díptero.
Planta crucífera.
\section{Arabi}
\begin{itemize}
\item {Grp. gram.:m.}
\end{itemize}
O mesmo que \textunderscore rabbino\textunderscore .
\section{Arabiado}
\begin{itemize}
\item {Grp. gram.:m.}
\end{itemize}
\begin{itemize}
\item {Utilização:Ant.}
\end{itemize}
\begin{itemize}
\item {Proveniência:(De \textunderscore árabe\textunderscore )}
\end{itemize}
Tributo, que os Judeus pagavam á Corôa.
\section{Arábico}
\begin{itemize}
\item {Grp. gram.:adj.}
\end{itemize}
\begin{itemize}
\item {Grp. gram.:M.}
\end{itemize}
Pertencente á Arábia.
A língua árabe.
\section{Arabídeas}
\begin{itemize}
\item {Grp. gram.:f. pl.}
\end{itemize}
\begin{itemize}
\item {Proveniência:(De \textunderscore árabis\textunderscore )}
\end{itemize}
Tríbo de plantas, da fam. das crucíferas, segundo De-Candolle.
\section{Arábigo}
\begin{itemize}
\item {Grp. gram.:m.  e  adj.}
\end{itemize}
O mesmo que \textunderscore arábico\textunderscore . Cf. Castilho, \textunderscore Metam.\textunderscore , XXVII.
\section{Arabina}
\begin{itemize}
\item {Grp. gram.:f.}
\end{itemize}
\begin{itemize}
\item {Proveniência:(De \textunderscore árabe\textunderscore )}
\end{itemize}
Princípio immediato, e solúvel na água, da goma arábica.
\section{Árabis}
\begin{itemize}
\item {Grp. gram.:f.}
\end{itemize}
Designação scientífica de uma planta crucífera, que serve de typo ás arabídeas.
\section{Arabismo}
\begin{itemize}
\item {Grp. gram.:m.}
\end{itemize}
Locução, própria da lingua árabe.
\section{Arabista}
\begin{itemize}
\item {Grp. gram.:m.}
\end{itemize}
Aquelle que conhece bem a língua árabe.
\section{Arabizante}
\begin{itemize}
\item {Grp. gram.:m.}
\end{itemize}
O mesmo que \textunderscore arabista\textunderscore .
\section{Arabizar}
\begin{itemize}
\item {Grp. gram.:v. t.}
\end{itemize}
\begin{itemize}
\item {Grp. gram.:V. i.}
\end{itemize}
Dar feição árabe a.
Dedicar-se a estudos arábicos.
Imitar a linguagem árabe.
\section{Arabote}
\begin{itemize}
\item {Grp. gram.:m.}
\end{itemize}
\begin{itemize}
\item {Utilização:Açor}
\end{itemize}
\begin{itemize}
\item {Proveniência:(Do ingl. \textunderscore water-boot\textunderscore )}
\end{itemize}
Barco, que leva água, aos navios.
\section{Arabu}
\begin{itemize}
\item {Grp. gram.:m.}
\end{itemize}
\begin{itemize}
\item {Utilização:Bras}
\end{itemize}
Pirão, feito de ovos de tartaruga ou de tracajá, batidos com açúcar e farinha.
\section{Arabutan}
\begin{itemize}
\item {Grp. gram.:m.}
\end{itemize}
Árvore leguminosa, que produz o chamado pau-brasil.
\section{Araca}
\begin{itemize}
\item {Grp. gram.:f.}
\end{itemize}
\begin{itemize}
\item {Proveniência:(Do ár. \textunderscore araque\textunderscore )}
\end{itemize}
Bebida alcoólica, preparada na Índia e na América com a fermentação do arroz principalmente.
\section{Araçá}
\begin{itemize}
\item {Grp. gram.:m.}
\end{itemize}
Árvore myrtácea da América.
O fruto dessa árvore.
\section{Araçá-goiaba}
\begin{itemize}
\item {Grp. gram.:m.}
\end{itemize}
\begin{itemize}
\item {Utilização:Bras}
\end{itemize}
O mesmo que \textunderscore goiaba\textunderscore .
\section{Aracaí}
\begin{itemize}
\item {Grp. gram.:m.}
\end{itemize}
\begin{itemize}
\item {Utilização:Bras}
\end{itemize}
Planta medicinal.
\section{Aracambuz}
\begin{itemize}
\item {Grp. gram.:m.}
\end{itemize}
\begin{itemize}
\item {Utilização:Bras}
\end{itemize}
Cruzeta, em que descansa a vêrga da mezena, em as jangadas.
Armação, em que se penduram os apparelhos da pesca na jangada.
\section{Aracamiri}
\begin{itemize}
\item {Grp. gram.:m.}
\end{itemize}
Arbusto do Brasil.
\section{Araçanga}
\begin{itemize}
\item {Grp. gram.:f.}
\end{itemize}
\begin{itemize}
\item {Utilização:Bras}
\end{itemize}
Cacete curto, com que os jangadeiros matam o peixe já ferrado no anzol. Cf. Alves Câmara, \textunderscore Constr. Navaes\textunderscore .
\section{Araçanhuna}
\begin{itemize}
\item {Grp. gram.:f.}
\end{itemize}
\begin{itemize}
\item {Utilização:Bras}
\end{itemize}
Árvore fructífera do mato virgem, de fruto semelhante á jaboticaba.
\section{Aracarangá}
\begin{itemize}
\item {Grp. gram.:m.}
\end{itemize}
Espécie de papagaio do Brasil.
\section{Aração}
\begin{itemize}
\item {Grp. gram.:f.}
\end{itemize}
\begin{itemize}
\item {Utilização:Bras}
\end{itemize}
Acto de comer soffregamente.
Fome excessiva.
\section{Araça-piroca}
\begin{itemize}
\item {Grp. gram.:m.}
\end{itemize}
Árvore silvestre do Brasil.
\section{Aracá-poca}
\begin{itemize}
\item {Grp. gram.:m.}
\end{itemize}
Árvore silvestre do Brasil.
\section{Araçareiro}
\begin{itemize}
\item {Grp. gram.:m.}
\end{itemize}
O mesmo que \textunderscore araçazeiro\textunderscore .
\section{Araçari}
\begin{itemize}
\item {Grp. gram.:m.}
\end{itemize}
\begin{itemize}
\item {Grp. gram.:m.}
\end{itemize}
\begin{itemize}
\item {Utilização:Bras}
\end{itemize}
O mesmo ou melhor que \textunderscore arassari\textunderscore .
Nome de várias aves trepadoras.
(Do tupi)
\section{Araçás}
\begin{itemize}
\item {Grp. gram.:m.}
\end{itemize}
O mesmo que \textunderscore araçá\textunderscore .
\section{Aracati}
\begin{itemize}
\item {Grp. gram.:m.}
\end{itemize}
\begin{itemize}
\item {Utilização:Bras}
\end{itemize}
Nome, que no Ceará se dá a um vento forte, que á noite sopra do Nordeste, no verão.
\section{Araçazada}
\begin{itemize}
\item {Grp. gram.:f.}
\end{itemize}
\begin{itemize}
\item {Utilização:Bras}
\end{itemize}
Doce de araçá.
\section{Araçazeiro}
\begin{itemize}
\item {Grp. gram.:m.}
\end{itemize}
(V.araçá)
\section{Aracazinho}
\begin{itemize}
\item {Grp. gram.:m.}
\end{itemize}
Arbusto brasileiro, (\textunderscore davia fragrans\textunderscore ).
\section{Aráceas}
\begin{itemize}
\item {Grp. gram.:f. pl.}
\end{itemize}
\begin{itemize}
\item {Proveniência:(Do gr. \textunderscore aron\textunderscore )}
\end{itemize}
Família de plantas monocotyledóneas, cujo typo é o jarro.
\section{Aráchida}
\begin{itemize}
\item {fónica:qui}
\end{itemize}
\begin{itemize}
\item {Grp. gram.:f.}
\end{itemize}
Planta trepadeira, (\textunderscore arachis hypogoea\textunderscore ), cultivada na América e na França.
\section{Arachídico}
\begin{itemize}
\item {fónica:qui}
\end{itemize}
\begin{itemize}
\item {Grp. gram.:adj.}
\end{itemize}
Diz-se de um ácido, extrahido do óleo da aráchida.
\section{Arachina}
\begin{itemize}
\item {fónica:qui}
\end{itemize}
\begin{itemize}
\item {Grp. gram.:f.}
\end{itemize}
Éther glycérico, extrahido do ácido arachídico.
\section{Arachis}
\begin{itemize}
\item {Grp. gram.:m.}
\end{itemize}
O mesmo que \textunderscore amendoim\textunderscore . Cf. Capello, \textunderscore Benguela\textunderscore , I, 69; II, 249.
\section{Arachneólitha}
\begin{itemize}
\item {Grp. gram.:f.}
\end{itemize}
\begin{itemize}
\item {Proveniência:(Do gr. \textunderscore arakhnaios\textunderscore  + \textunderscore lithos\textunderscore )}
\end{itemize}
Caranguejo fóssil.
\section{Arachnídeos}
\begin{itemize}
\item {Grp. gram.:m. pl.}
\end{itemize}
\begin{itemize}
\item {Proveniência:(Do gr. \textunderscore arakhne\textunderscore  + \textunderscore eidos\textunderscore )}
\end{itemize}
Segunda classe dos animaes articulados, cuja cabeça e thorax formam uma só peça, e cuja parte posterior é uma série de anéis ou uma massa glandulosa.
\section{Arachnidos}
\begin{itemize}
\item {Grp. gram.:m. pl.}
\end{itemize}
(V.arachnídeos)
\section{Arachnite}
\begin{itemize}
\item {Grp. gram.:f.}
\end{itemize}
Inflammação da arachnoide.
\section{Arachnodérmico}
\begin{itemize}
\item {Grp. gram.:adj.}
\end{itemize}
\begin{itemize}
\item {Proveniência:(Do gr. \textunderscore arakhne\textunderscore  + \textunderscore derma\textunderscore )}
\end{itemize}
Que tem a pelle fina, como teia de aranha.
\section{Arachnogenose}
\begin{itemize}
\item {Grp. gram.:f.}
\end{itemize}
Doença, resultante da picada ou da introducção de uma aranha em o nosso organismo.
\section{Arachnoide}
\begin{itemize}
\item {Grp. gram.:f.}
\end{itemize}
\begin{itemize}
\item {Utilização:Anat.}
\end{itemize}
\begin{itemize}
\item {Proveniência:(Do gr. \textunderscore arakhne\textunderscore  + \textunderscore eidos\textunderscore )}
\end{itemize}
Membrana delgada e transparente, entre a dura-máter e a pia-máter.
\section{Arachnoídeo}
\begin{itemize}
\item {Grp. gram.:adj.}
\end{itemize}
Relativo á \textunderscore arachnoide\textunderscore .
\section{Arachnoidiano}
\begin{itemize}
\item {Grp. gram.:adj.}
\end{itemize}
O mesmo que \textunderscore arachnoídeo\textunderscore .
\section{Arachnoidite}
\begin{itemize}
\item {Grp. gram.:f.}
\end{itemize}
O mesmo que \textunderscore arachnite\textunderscore .
\section{Arachnologia}
\begin{itemize}
\item {Grp. gram.:f.}
\end{itemize}
Parte da Entomologia, que trata das aranhas.
(Cp. \textunderscore arachnólogo\textunderscore )
\section{Arachnólogo}
\begin{itemize}
\item {Grp. gram.:m.}
\end{itemize}
\begin{itemize}
\item {Proveniência:(Do gr. \textunderscore arakhne\textunderscore  + \textunderscore logos\textunderscore )}
\end{itemize}
Aquelle que escreve á cêrca de aranhas.
\section{Aracirana}
\begin{itemize}
\item {Grp. gram.:f.}
\end{itemize}
Nome de uma ave brasileira.
\section{Aracis}
\begin{itemize}
\item {Grp. gram.:m. pl.}
\end{itemize}
\begin{itemize}
\item {Utilização:Bras}
\end{itemize}
Tribo selvagem, que habitou em Mato-Grosso.
\section{Aracneólita}
\begin{itemize}
\item {Grp. gram.:f.}
\end{itemize}
\begin{itemize}
\item {Proveniência:(Do gr. \textunderscore arakhnaios\textunderscore  + \textunderscore lithos\textunderscore )}
\end{itemize}
Caranguejo fóssil.
\section{Aracnídeos}
\begin{itemize}
\item {Grp. gram.:m. pl.}
\end{itemize}
\begin{itemize}
\item {Proveniência:(Do gr. \textunderscore arakhne\textunderscore  + \textunderscore eidos\textunderscore )}
\end{itemize}
Segunda classe dos animaes articulados, cuja cabeça e thorax formam uma só peça, e cuja parte posterior é uma série de anéis ou uma massa glandulosa.
\section{Aracnidos}
\begin{itemize}
\item {Grp. gram.:m. pl.}
\end{itemize}
(V.aracnídeos)
\section{Aracnite}
\begin{itemize}
\item {Grp. gram.:f.}
\end{itemize}
Inflammação da aracnoide.
\section{Aracnodérmico}
\begin{itemize}
\item {Grp. gram.:adj.}
\end{itemize}
\begin{itemize}
\item {Proveniência:(Do gr. \textunderscore arakhne\textunderscore  + \textunderscore derma\textunderscore )}
\end{itemize}
Que tem a pelle fina, como teia de aranha.
\section{Aracnogenose}
\begin{itemize}
\item {Grp. gram.:f.}
\end{itemize}
Doença, resultante da picada ou da introducção de uma aranha em o nosso organismo.
\section{Aracnoide}
\begin{itemize}
\item {Grp. gram.:f.}
\end{itemize}
\begin{itemize}
\item {Utilização:Anat.}
\end{itemize}
\begin{itemize}
\item {Proveniência:(Do gr. \textunderscore arakhne\textunderscore  + \textunderscore eidos\textunderscore )}
\end{itemize}
Membrana delgada e transparente, entre a dura-máter e a pia-máter.
\section{Aracnoídeo}
\begin{itemize}
\item {Grp. gram.:adj.}
\end{itemize}
Relativo á \textunderscore aracnoide\textunderscore .
\section{Aracnoidiano}
\begin{itemize}
\item {Grp. gram.:adj.}
\end{itemize}
O mesmo que \textunderscore aracnoídeo\textunderscore .
\section{Aracnoidite}
\begin{itemize}
\item {Grp. gram.:f.}
\end{itemize}
O mesmo que \textunderscore aracnite\textunderscore .
\section{Aracnologia}
\begin{itemize}
\item {Grp. gram.:f.}
\end{itemize}
Parte da Entomologia, que trata das aranhas.
(Cp. \textunderscore arachnólogo\textunderscore )
\section{Aracnólogo}
\begin{itemize}
\item {Grp. gram.:m.}
\end{itemize}
\begin{itemize}
\item {Proveniência:(Do gr. \textunderscore arakhne\textunderscore  + \textunderscore logos\textunderscore )}
\end{itemize}
Aquelle que escreve á cêrca de aranhas.
\section{Aracoá}
\begin{itemize}
\item {Grp. gram.:f.}
\end{itemize}
(V.araquan)
\section{Aracu}
\begin{itemize}
\item {Grp. gram.:m.}
\end{itemize}
Peixe do Brasil.
\section{Aracuan}
\begin{itemize}
\item {Grp. gram.:m.  e  f.}
\end{itemize}
O mesmo que \textunderscore araquan\textunderscore .
\section{Arada}
\begin{itemize}
\item {Grp. gram.:f.}
\end{itemize}
Acto ou effeito de arar; aradura.
Lavoira.
\section{Arádega}
\begin{itemize}
\item {Grp. gram.:f.}
\end{itemize}
\begin{itemize}
\item {Proveniência:(De \textunderscore arar\textunderscore )}
\end{itemize}
Antigo tributo de pão, que se pagava ao mosteiro de Alcobaça.
\section{Aradeira}
\begin{itemize}
\item {Grp. gram.:f.}
\end{itemize}
\begin{itemize}
\item {Utilização:Prov.}
\end{itemize}
\begin{itemize}
\item {Utilização:trasm.}
\end{itemize}
O mesmo que \textunderscore hera\textunderscore .
(Por \textunderscore heredeira\textunderscore , metáth. de \textunderscore hedereira\textunderscore , do lat. \textunderscore hedera\textunderscore )
\section{Aradelo}
\begin{itemize}
\item {fónica:dê}
\end{itemize}
\begin{itemize}
\item {Grp. gram.:m.}
\end{itemize}
\begin{itemize}
\item {Utilização:Prov.}
\end{itemize}
A constellação da Ursa-Maior.
(Colhido em Turquel)
\section{Arado}
\begin{itemize}
\item {Grp. gram.:m.}
\end{itemize}
\begin{itemize}
\item {Utilização:Náut.}
\end{itemize}
\begin{itemize}
\item {Proveniência:(Lat. \textunderscore aratrum\textunderscore )}
\end{itemize}
Instrumento, para lavrar a terra.
Nome que se dá ao chamado \textunderscore busca-vida\textunderscore , quando se emprega em procurar uma âncora ou outro objecto, que esteja debaixo da água e se não veja.
\section{Arado}
\begin{itemize}
\item {Grp. gram.:adj.}
\end{itemize}
\begin{itemize}
\item {Utilização:Bras}
\end{itemize}
Esfomeado.
(Cp. \textunderscore aração\textunderscore )
\section{Aradoira}
\begin{itemize}
\item {Grp. gram.:f.}
\end{itemize}
\begin{itemize}
\item {Utilização:Ant.}
\end{itemize}
\begin{itemize}
\item {Proveniência:(De \textunderscore arar\textunderscore )}
\end{itemize}
Um dia de lavoira ou de vessada.
\section{Aradoiro}
\begin{itemize}
\item {Grp. gram.:m.}
\end{itemize}
\begin{itemize}
\item {Utilização:Ant.}
\end{itemize}
O mesmo que \textunderscore arado\textunderscore ^1.
\section{Arador}
\begin{itemize}
\item {Grp. gram.:m.}
\end{itemize}
\begin{itemize}
\item {Proveniência:(Lat. \textunderscore arator\textunderscore )}
\end{itemize}
Aquelle que ara.
\section{Aradoura}
\begin{itemize}
\item {Grp. gram.:f.}
\end{itemize}
\begin{itemize}
\item {Utilização:Ant.}
\end{itemize}
\begin{itemize}
\item {Proveniência:(De \textunderscore arar\textunderscore )}
\end{itemize}
Um dia de lavoura ou de vessada.
\section{Aradouro}
\begin{itemize}
\item {Grp. gram.:m.}
\end{itemize}
\begin{itemize}
\item {Utilização:Ant.}
\end{itemize}
O mesmo que \textunderscore arado\textunderscore ^1.
\section{Aradura}
\begin{itemize}
\item {Grp. gram.:f.}
\end{itemize}
Acto de \textunderscore arar\textunderscore .
Terra arada.
\section{Aráes}
\begin{itemize}
\item {Grp. gram.:m. pl.}
\end{itemize}
Nação indígena do Brasil, que dominava em Goiás.
\section{Aragano}
\begin{itemize}
\item {Grp. gram.:adj.}
\end{itemize}
\begin{itemize}
\item {Utilização:Bras}
\end{itemize}
Diz-se do cavallo espantadiço ou diffícil de sêr dominado.
(Cast. \textunderscore haragán\textunderscore )
\section{Aragão}
\begin{itemize}
\item {Grp. gram.:m.}
\end{itemize}
\begin{itemize}
\item {Utilização:Bras. do Rio}
\end{itemize}
\begin{itemize}
\item {Proveniência:(De \textunderscore Aragão\textunderscore , n. p.)}
\end{itemize}
Sino grande da igreja de San-Francisco de Paula, que dá o toque para se fecharem os estabelecimentos do Rio-de-Janeiro.
\section{Aragem}
\begin{itemize}
\item {Grp. gram.:f.}
\end{itemize}
\begin{itemize}
\item {Proveniência:(De \textunderscore ar\textunderscore )}
\end{itemize}
Viração; vento brando.
Bafejo.
\section{Arágoa}
\begin{itemize}
\item {Grp. gram.:f.}
\end{itemize}
Gênero de plantas escrofularíneas.
\section{Aragoês}
\begin{itemize}
\item {Grp. gram.:m.  e  adj.}
\end{itemize}
O mesmo ou melhor que \textunderscore aragonês\textunderscore .
\section{Aragonês}
\begin{itemize}
\item {Grp. gram.:m.}
\end{itemize}
\begin{itemize}
\item {Grp. gram.:Adj.}
\end{itemize}
\begin{itemize}
\item {Proveniência:(De \textunderscore Aragon\textunderscore , n. p. cast.)}
\end{itemize}
Aquelle que nasceu em Aragão.
Dialecto de Aragão.
Casta de uva preta.
Relativo a Aragão.
\section{Aragonesa}
\begin{itemize}
\item {Grp. gram.:f.}
\end{itemize}
Casta de uva alentejana.
\section{Aragonita}
\begin{itemize}
\item {Grp. gram.:f.}
\end{itemize}
\begin{itemize}
\item {Proveniência:(De \textunderscore Aragon\textunderscore , n. p. cast.)}
\end{itemize}
Carbonato calcáreo crystallizável.
\section{Aragonite}
\begin{itemize}
\item {Grp. gram.:f.}
\end{itemize}
(V.aragonita)
\section{Araguágua}
\begin{itemize}
\item {Grp. gram.:m.}
\end{itemize}
Espadarte do Brasil.
\section{Araguari}
\begin{itemize}
\item {Grp. gram.:m.}
\end{itemize}
\begin{itemize}
\item {Utilização:Bras}
\end{itemize}
Espécie de arara.
\section{Araguato}
\begin{itemize}
\item {Grp. gram.:m.}
\end{itemize}
Macaco ruivo do Orenoque.
\section{Araiané!}
\begin{itemize}
\item {Grp. gram.:interj.}
\end{itemize}
\begin{itemize}
\item {Utilização:bras}
\end{itemize}
(design. de aborrecimento, causado pela repetição enfadonha de uma notícia já muito sabida)
\section{Araicás}
\begin{itemize}
\item {Grp. gram.:m. pl.}
\end{itemize}
Uma das tribos indígenas do norte do Brasil.
\section{Aráis}
\begin{itemize}
\item {Grp. gram.:m. pl.}
\end{itemize}
Nação indígena do Brasil, que dominava em Goiás.
\section{Aral}
\begin{itemize}
\item {Grp. gram.:m.}
\end{itemize}
\begin{itemize}
\item {Utilização:Ant.}
\end{itemize}
\begin{itemize}
\item {Proveniência:(De \textunderscore arar\textunderscore )}
\end{itemize}
Terra arroteada, própria para cultura.
\section{Aralha}
\begin{itemize}
\item {Grp. gram.:f.}
\end{itemize}
Novilha de dois annos.
Palha de alhos.
\section{Arália}
\begin{itemize}
\item {Grp. gram.:f.}
\end{itemize}
Gênero de plantas, que serve de typo ás araliáceas.
\section{Araliáceas}
\begin{itemize}
\item {Grp. gram.:f. pl.}
\end{itemize}
\begin{itemize}
\item {Proveniência:(De \textunderscore arália\textunderscore )}
\end{itemize}
Familia de plantas dicotyledóneas, polypétalas, sem estipulas.
\section{Aramá}
\begin{itemize}
\item {Grp. gram.:adv.}
\end{itemize}
\begin{itemize}
\item {Utilização:Ant.}
\end{itemize}
O mesmo que \textunderscore eramá\textunderscore .
\section{Aramaca}
\begin{itemize}
\item {Grp. gram.:f.}
\end{itemize}
Espécie de linguado das costas do Brasil.
\section{Aramada}
\begin{itemize}
\item {Grp. gram.:adj. f.}
\end{itemize}
\begin{itemize}
\item {Utilização:Prov.}
\end{itemize}
\begin{itemize}
\item {Utilização:alent.}
\end{itemize}
\begin{itemize}
\item {Proveniência:(De \textunderscore arame\textunderscore )}
\end{itemize}
Diz-se da lanterna, que tem resguardos de lata á borda dos vidros.
\section{Aramador}
\begin{itemize}
\item {Grp. gram.:m.}
\end{itemize}
\begin{itemize}
\item {Utilização:Bras}
\end{itemize}
Fabricante de rede de arame.
Alambrador.
\section{Aramagem}
\begin{itemize}
\item {Grp. gram.:f.}
\end{itemize}
Gradeamento de arame.
\section{Aramar}
\begin{itemize}
\item {Grp. gram.:v. t.}
\end{itemize}
\begin{itemize}
\item {Utilização:Bras}
\end{itemize}
Fabricar ou gradear com arame.
\section{Aramata}
\begin{itemize}
\item {Grp. gram.:f.}
\end{itemize}
Árvore da Guiana inglesa.
\section{Arame}
\begin{itemize}
\item {Grp. gram.:m.}
\end{itemize}
\begin{itemize}
\item {Utilização:Gír.}
\end{itemize}
\begin{itemize}
\item {Utilização:ant.}
\end{itemize}
\begin{itemize}
\item {Utilização:Gír.}
\end{itemize}
\begin{itemize}
\item {Grp. gram.:Pl.}
\end{itemize}
\begin{itemize}
\item {Utilização:Prov.}
\end{itemize}
\begin{itemize}
\item {Utilização:alent.}
\end{itemize}
\begin{itemize}
\item {Utilização:Bras}
\end{itemize}
\begin{itemize}
\item {Utilização:Chul.}
\end{itemize}
\begin{itemize}
\item {Utilização:Fam.}
\end{itemize}
\begin{itemize}
\item {Proveniência:(Do lat. \textunderscore hyp\textunderscore . \textunderscore aeramen\textunderscore )}
\end{itemize}
Liga de cobre com zinco ou com outros metaes.
Fio de latão; ferro puxado á fieira.
Loiça de metal amarelo:«\textunderscore todo o gênero de arame para a fábrica dos doces\textunderscore ». M. Bernardes, \textunderscore N. Floresta\textunderscore .
Navalha.
Espada.
Guardas de lata, acaneladas, nos vidros de certas lanternas.
O mesmo que \textunderscore dinheiro\textunderscore .
\textunderscore Ir aos arames\textunderscore , irritar-se.
\section{Arameiro}
\begin{itemize}
\item {Grp. gram.:m.}
\end{itemize}
Aquelle que trabalha em arame.
\section{Aramenha}
\begin{itemize}
\item {Grp. gram.:f.}
\end{itemize}
\begin{itemize}
\item {Utilização:Prov.}
\end{itemize}
\begin{itemize}
\item {Utilização:dur.}
\end{itemize}
Planta, conhecida também por \textunderscore erva-babosa\textunderscore .
Armadilha, de fórma cónica, para apanhar pássaros.
\section{Arameu}
\begin{itemize}
\item {Grp. gram.:m.}
\end{itemize}
Grupo de línguas semíticas.
\section{Arâmico}
\begin{itemize}
\item {Grp. gram.:m.}
\end{itemize}
O mesmo que \textunderscore arameu\textunderscore .
\section{Aramina}
\begin{itemize}
\item {Grp. gram.:f.}
\end{itemize}
\begin{itemize}
\item {Utilização:Bras}
\end{itemize}
Fibra têxtil do carrapicho.
\section{Aramio}
\begin{itemize}
\item {Grp. gram.:m.}
\end{itemize}
\begin{itemize}
\item {Utilização:Ant.}
\end{itemize}
Terra ou geira que se lavra num dia.
\section{Aramoso}
\begin{itemize}
\item {Grp. gram.:adj.}
\end{itemize}
\begin{itemize}
\item {Proveniência:(De \textunderscore arame\textunderscore )}
\end{itemize}
Diz-se do anel das agaríceas, quando, em vez de formar membrana, é constituido por filamentos separados.
\section{Arancu}
\begin{itemize}
\item {Grp. gram.:m.}
\end{itemize}
\begin{itemize}
\item {Utilização:Prov.}
\end{itemize}
\begin{itemize}
\item {Utilização:trasm.}
\end{itemize}
O mesmo que \textunderscore pyrilampo\textunderscore .
\section{Arandela}
\begin{itemize}
\item {Grp. gram.:f.}
\end{itemize}
\begin{itemize}
\item {Utilização:Ant.}
\end{itemize}
\begin{itemize}
\item {Utilização:Bras. do Rio}
\end{itemize}
Peça de metal ou loiça, que se põe na boca do castiçal, para aparar os pingos da vela.
Copos da lança.
Collar de folhos.
Braço ou bico de gás, preso á parede.
(Cast. \textunderscore arandela\textunderscore )
\section{Arando}
\begin{itemize}
\item {Grp. gram.:m.}
\end{itemize}
Gênero de árvores ericáceas e medicinaes do Brasil.
\section{Araneano}
\begin{itemize}
\item {Grp. gram.:adj.}
\end{itemize}
\begin{itemize}
\item {Proveniência:(Do lat. \textunderscore aranea\textunderscore )}
\end{itemize}
Semelhante a uma aranha.
\section{Araneides}
\begin{itemize}
\item {Grp. gram.:m. pl.}
\end{itemize}
O mesmo que \textunderscore arachnídeos\textunderscore .
\section{Araneídeos}
\begin{itemize}
\item {Grp. gram.:m. pl.}
\end{itemize}
O mesmo que \textunderscore araneides\textunderscore .
\section{Araneífero}
\begin{itemize}
\item {Grp. gram.:adj.}
\end{itemize}
\begin{itemize}
\item {Proveniência:(Do lat. \textunderscore aranea\textunderscore  + \textunderscore ferre\textunderscore )}
\end{itemize}
Que tem teias de aranha.
\section{Araneiforme}
\begin{itemize}
\item {Grp. gram.:adj.}
\end{itemize}
\begin{itemize}
\item {Proveniência:(Do lat. \textunderscore aranea\textunderscore  + \textunderscore forma\textunderscore )}
\end{itemize}
Semelhante á aranha.
\section{Araneografia}
\begin{itemize}
\item {Grp. gram.:f.}
\end{itemize}
O mesmo que \textunderscore araneologia\textunderscore .
\section{Araneographia}
\begin{itemize}
\item {Grp. gram.:f.}
\end{itemize}
O mesmo que \textunderscore araneologia\textunderscore .
\section{Araneologia}
\begin{itemize}
\item {Grp. gram.:f.}
\end{itemize}
O mesmo que \textunderscore arachnologia\textunderscore .
\section{Arangão}
\begin{itemize}
\item {Grp. gram.:m.}
\end{itemize}
\begin{itemize}
\item {Utilização:Pop.}
\end{itemize}
(V. \textunderscore artesão\textunderscore ^1)
\section{Aranha}
\begin{itemize}
\item {Grp. gram.:f.}
\end{itemize}
\begin{itemize}
\item {Utilização:Bras}
\end{itemize}
\begin{itemize}
\item {Utilização:Náut.}
\end{itemize}
\begin{itemize}
\item {Proveniência:(Lat. \textunderscore aranea\textunderscore )}
\end{itemize}
Animal articulado, de abdome grosso, com fieiras, de que sái a substância com que o animal fórma uma teia muito delgada.
Nome de um peixe.
Lustre para velas.
Viatura leve, de rodas altas, para sêr puxada por um só cavallo.
Peça de arame, em que repoisa a pantalha.
Peça de arame, com que se suspendem pratos numa parede.
Planta dos jardins, (\textunderscore gloriosa superba\textunderscore ) Peça de ferro, no fim da cadeia do travão.
Refôrço das velas, formado por um cabo com sapatilho no seio, e cujos chicotes desfiados se cosem á vela, cobrindo-se de tiras de lona.
Lagariça de madeira, com prensa de pau, usada pelos pequenos lavradores de Carcavelos.
\textunderscore Andar ás aranhas\textunderscore , andar desnorteado, á tôa.
\section{Aranhagato}
\begin{itemize}
\item {Grp. gram.:m.}
\end{itemize}
Árvore silvestre do Brasil, também chamada \textunderscore vinhático\textunderscore .
\section{Aranhão}
\begin{itemize}
\item {Grp. gram.:m.}
\end{itemize}
Aranha grande.
\section{Aranheira}
\begin{itemize}
\item {Grp. gram.:f.}
\end{itemize}
\begin{itemize}
\item {Utilização:Prov.}
\end{itemize}
Teia de aranha.
\section{Aranheiro}
\begin{itemize}
\item {Grp. gram.:m.}
\end{itemize}
(V.aranhol)
\section{Aranhento}
\begin{itemize}
\item {Grp. gram.:adj.}
\end{itemize}
Próprio de aranha.
Em que há aranhas: \textunderscore casa aranhenta\textunderscore .
\section{Aranhiço}
\begin{itemize}
\item {Grp. gram.:m.}
\end{itemize}
\begin{itemize}
\item {Grp. gram.:Pl.}
\end{itemize}
Pequena aranha.
Peixe, semelhante a um polvo pequeno e que se colhe nos arrifes da beira-mar.
Conjuncto de arcos salientes das abóbadas ogivaes.
\section{Aranhol}
\begin{itemize}
\item {Grp. gram.:m.}
\end{itemize}
\begin{itemize}
\item {Proveniência:(De \textunderscore aranha\textunderscore )}
\end{itemize}
Buraco, em que se recolhem as aranhas.
Armadilha, para apanhar pássaros, semelhante á aranha.
\section{Aranhola}
\begin{itemize}
\item {Grp. gram.:f.}
\end{itemize}
\begin{itemize}
\item {Proveniência:(De \textunderscore aranha\textunderscore )}
\end{itemize}
Caranguejo grande, com a coiraça cheia de bicos.
\section{Aranhoso}
\begin{itemize}
\item {Grp. gram.:adj.}
\end{itemize}
\begin{itemize}
\item {Proveniência:(De \textunderscore aranha\textunderscore )}
\end{itemize}
Diz-se dos pêlos longos, finos e entrecruzados, como os fios da teia de aranha.
O mesmo que \textunderscore aramoso\textunderscore .
\section{Aranhota}
\begin{itemize}
\item {Grp. gram.:f.}
\end{itemize}
\begin{itemize}
\item {Utilização:Gír.}
\end{itemize}
\begin{itemize}
\item {Proveniência:(De \textunderscore aranha\textunderscore )}
\end{itemize}
Sardinha.
\section{Aranhuço}
\begin{itemize}
\item {Grp. gram.:m.}
\end{itemize}
\begin{itemize}
\item {Utilização:Pop.}
\end{itemize}
O mesmo que \textunderscore aranhão\textunderscore .
Espécie de peixe, (\textunderscore trachinus araneus\textunderscore , Cuv.).
\section{Aranoso}
\begin{itemize}
\item {Grp. gram.:adj.}
\end{itemize}
(V.aranhoso)
\section{Aranzel}
\begin{itemize}
\item {Grp. gram.:m.}
\end{itemize}
\begin{itemize}
\item {Utilização:Ant.}
\end{itemize}
\begin{itemize}
\item {Grp. gram.:M.  e  f.}
\end{itemize}
\begin{itemize}
\item {Utilização:Prov.}
\end{itemize}
\begin{itemize}
\item {Utilização:trasm.}
\end{itemize}
Discurso prolixo e enfadonho; lenga-lenga.
Formulário.
Tarifa alfandegária.
Pessôa fraca.
(Cast. \textunderscore arancel\textunderscore )
\section{Arão}
\begin{itemize}
\item {Grp. gram.:m.}
\end{itemize}
\begin{itemize}
\item {Proveniência:(Do gr. \textunderscore aron\textunderscore )}
\end{itemize}
O mesmo que \textunderscore jarro\textunderscore ^2 (planta).
\section{Arapabaca}
\begin{itemize}
\item {Grp. gram.:f.}
\end{itemize}
\begin{itemize}
\item {Utilização:Bras}
\end{itemize}
O mesmo que \textunderscore lombrigueira\textunderscore .
\section{Arapapá}
\begin{itemize}
\item {Grp. gram.:m.}
\end{itemize}
Ave ribeirinha do Brasil.
(Do tupi)
\section{Arapene}
\begin{itemize}
\item {Grp. gram.:m.}
\end{itemize}
\begin{itemize}
\item {Proveniência:(Lat. \textunderscore arapennis\textunderscore )}
\end{itemize}
Quadrado de vinte e quatro palmos por banda, (entre os Godos). Cf. Herculano, \textunderscore Eurico\textunderscore , 262.
\section{Arapenne}
\begin{itemize}
\item {Grp. gram.:m.}
\end{itemize}
\begin{itemize}
\item {Proveniência:(Lat. \textunderscore arapennis\textunderscore )}
\end{itemize}
Quadrado de vinte e quatro palmos por banda, (entre os Godos). Cf. Herculano, \textunderscore Eurico\textunderscore , 262.
\section{Arapinga}
\begin{itemize}
\item {Grp. gram.:m.}
\end{itemize}
Ave do Brasil.
\section{Arapiraca}
\begin{itemize}
\item {Grp. gram.:f.}
\end{itemize}
Árvore do Brasil.
\section{Arapoca}
\begin{itemize}
\item {Grp. gram.:f.}
\end{itemize}
Árvore rutácea do Brasil.
\section{Araponga}
\begin{itemize}
\item {Grp. gram.:f.}
\end{itemize}
\begin{itemize}
\item {Utilização:Bras}
\end{itemize}
Ave branca do Brasil, notável pelo som metállico do seu canto.
Pessôa, que tem voz estridente ou que fala, gritando.
(Corr. do tupi \textunderscore guiparong\textunderscore )
\section{Arapuá}
\begin{itemize}
\item {Grp. gram.:f.}
\end{itemize}
Abelha grande e negra do Brasil.
\section{Arapuca}
\begin{itemize}
\item {Grp. gram.:f.}
\end{itemize}
\begin{itemize}
\item {Utilização:Fig.}
\end{itemize}
Armadilha, com que no Brasil se apanham pássaros.
Casa velha ou esburacada.
\section{Araquan}
\begin{itemize}
\item {Grp. gram.:m.  e  f.}
\end{itemize}
\begin{itemize}
\item {Utilização:Bras}
\end{itemize}
\begin{itemize}
\item {Proveniência:(T. tupi)}
\end{itemize}
Nome commum a três espécies de gallináceas.
\section{Araque}
\begin{itemize}
\item {Grp. gram.:m.}
\end{itemize}
(V.araca)
\section{Aráquida}
\begin{itemize}
\item {Grp. gram.:f.}
\end{itemize}
Planta trepadeira, (\textunderscore arachis hypogoea\textunderscore ), cultivada na América e na França.
\section{Araquídico}
\begin{itemize}
\item {Grp. gram.:adj.}
\end{itemize}
Diz-se de um ácido, extrahido do óleo da aráquida.
\section{Araquina}
\begin{itemize}
\item {Grp. gram.:f.}
\end{itemize}
Éther glycérico, extrahido do ácido araquídico.
\section{Arar}
\begin{itemize}
\item {Grp. gram.:v. t.}
\end{itemize}
\begin{itemize}
\item {Proveniência:(Lat. \textunderscore arare\textunderscore )}
\end{itemize}
Lavrar, sulcar (a terra).
Navegar.
\section{Arara}
\begin{itemize}
\item {Grp. gram.:f.}
\end{itemize}
\begin{itemize}
\item {Grp. gram.:M.  e  f.}
\end{itemize}
\begin{itemize}
\item {Utilização:Bras}
\end{itemize}
\begin{itemize}
\item {Utilização:pop.}
\end{itemize}
Ave trepadora, espécie de papagaio.
Espécie de amaranto.
Mentira; balela.
Pateta.
Caloiro.
\section{Arará}
\begin{itemize}
\item {Grp. gram.:m.}
\end{itemize}
Árvore da ilha de Cuba.
Ave aquática do Rio-Grande-do-Sul.
Espécie de térmite.
\section{Arara-bóia}
\begin{itemize}
\item {Grp. gram.:f.}
\end{itemize}
Espécie de serpente da região do Amazonas.
\section{Arara-canga}
\begin{itemize}
\item {Grp. gram.:f.}
\end{itemize}
\begin{itemize}
\item {Utilização:Bras}
\end{itemize}
A arara vermelha.
\section{Ararama}
\begin{itemize}
\item {Grp. gram.:f.}
\end{itemize}
Grande arara preta.
Árvore do Brasil, própria para construcções.
\section{Araranan}
\begin{itemize}
\item {Grp. gram.:m.}
\end{itemize}
Peixe do Brasil.
\section{Ararari}
\begin{itemize}
\item {Grp. gram.:m.}
\end{itemize}
Árvore medicinal do Alto-Amazonas.
\section{Ararapá}
\begin{itemize}
\item {Grp. gram.:m.}
\end{itemize}
\begin{itemize}
\item {Utilização:Bras}
\end{itemize}
Ave nocturna das regiões do Amazonas.
\section{Ararás}
\begin{itemize}
\item {Grp. gram.:m. pl.}
\end{itemize}
Índios do Brasil, entre o rio Madeira e o Tapojós.
\section{Araraúna}
\begin{itemize}
\item {Grp. gram.:f.}
\end{itemize}
O mesmo que \textunderscore araruna\textunderscore .
\section{Araribá}
\begin{itemize}
\item {Grp. gram.:f.}
\end{itemize}
Árvore brasileira, rubiácea, de casca rubra, empregada em tinturaria pelos Índios.
Ruivinha.
\section{Araribina}
\begin{itemize}
\item {Grp. gram.:f.}
\end{itemize}
Alcaloide crystallizável da casca da araribá.
\section{Araribuna}
\begin{itemize}
\item {Grp. gram.:f.}
\end{itemize}
Árvore brasileira.
\section{Araroba}
\begin{itemize}
\item {Grp. gram.:f.}
\end{itemize}
Planta leguminosa do Brasil.
\section{Araruás}
\begin{itemize}
\item {Grp. gram.:m. pl.}
\end{itemize}
Índios do Brasil, nas margens do Japurá.
\section{Araruna}
\begin{itemize}
\item {Grp. gram.:f.}
\end{itemize}
\begin{itemize}
\item {Proveniência:(T. tupi)}
\end{itemize}
Espécie de arara, de côr azul-ferrete.
\section{Araruta}
\begin{itemize}
\item {Grp. gram.:f.}
\end{itemize}
\begin{itemize}
\item {Proveniência:(Do ingl. \textunderscore arrow-root\textunderscore )}
\end{itemize}
Fécula alimentícia, extrahida do rizoma de plantas amomáceas.
\section{Arás}
\begin{itemize}
\item {Grp. gram.:m. pl.}
\end{itemize}
Gênero de aves, a que pertencem as araras.
\section{Arasari}
\begin{itemize}
\item {Grp. gram.:m.}
\end{itemize}
(V.arassari)
\section{Arassari}
\begin{itemize}
\item {Grp. gram.:m.}
\end{itemize}
\begin{itemize}
\item {Utilização:Bras}
\end{itemize}
Nome de várias aves trepadoras.
(Do tupi)
\section{Arataca}
\begin{itemize}
\item {Grp. gram.:f.}
\end{itemize}
\begin{itemize}
\item {Utilização:Bras}
\end{itemize}
\begin{itemize}
\item {Proveniência:(T. tupi)}
\end{itemize}
Espécie de armadilha, para apanhar animaes silvestres.
\section{Arataia}
\begin{itemize}
\item {Grp. gram.:f.}
\end{itemize}
Árvore do Brasil.
\section{Aratanha}
\begin{itemize}
\item {Grp. gram.:f.}
\end{itemize}
\begin{itemize}
\item {Utilização:Bras}
\end{itemize}
Pequena vaca.
Pequeno camarão.
Pequeno sapo.
\section{Araticu}
\begin{itemize}
\item {Grp. gram.:m.}
\end{itemize}
Designação de várias árvores do Brasil.
Fruto dessas árvores.
\section{Araticueiro}
\begin{itemize}
\item {Grp. gram.:m.}
\end{itemize}
O mesmo ou melhor que \textunderscore araticuzeiro\textunderscore .
\section{Araticum}
\begin{itemize}
\item {Grp. gram.:m.}
\end{itemize}
O mesmo que \textunderscore araticu\textunderscore .
\section{Araticuzeiro}
\begin{itemize}
\item {Grp. gram.:m.}
\end{itemize}
(V. \textunderscore araticu\textunderscore , árvore)
\section{Aratingui}
\begin{itemize}
\item {Grp. gram.:m.}
\end{itemize}
Árvore do Brasil.
\section{Aratório}
\begin{itemize}
\item {Grp. gram.:adj.}
\end{itemize}
\begin{itemize}
\item {Proveniência:(Lat. \textunderscore aratorius\textunderscore )}
\end{itemize}
Que pertence ao arado ou á lavoira.
\section{Aratriforme}
\begin{itemize}
\item {Grp. gram.:adj.}
\end{itemize}
\begin{itemize}
\item {Proveniência:(Do lat. \textunderscore aratrum\textunderscore  + \textunderscore forma\textunderscore )}
\end{itemize}
Semelhante ao arado.
\section{Aratu}
\begin{itemize}
\item {Grp. gram.:m.}
\end{itemize}
\begin{itemize}
\item {Utilização:Bras}
\end{itemize}
Espécie de caranguejo.
\section{Arau}
\begin{itemize}
\item {Grp. gram.:m.}
\end{itemize}
O mesmo que \textunderscore airo\textunderscore .
\section{Arauaris}
\begin{itemize}
\item {Grp. gram.:m. pl.}
\end{itemize}
Indígenas do norte do Brasil.
\section{Araucânio}
\begin{itemize}
\item {Grp. gram.:m.}
\end{itemize}
Língua holophrástica dos Araucanos ou Chilenos.
\section{Araucano}
\begin{itemize}
\item {Grp. gram.:m.}
\end{itemize}
Língua holophrástica dos Araucanos ou Chilenos.
\section{Araucanos}
\begin{itemize}
\item {Grp. gram.:m. pl.}
\end{itemize}
\begin{itemize}
\item {Proveniência:(De \textunderscore Araucânia\textunderscore , n. p.)}
\end{itemize}
Aborígenes do Chile.
\section{Araucária}
\begin{itemize}
\item {Grp. gram.:f.}
\end{itemize}
\begin{itemize}
\item {Proveniência:(De \textunderscore Arauco\textunderscore , n. p.)}
\end{itemize}
Planta conífera das regiões tropicaes.
\section{Araucariado}
\begin{itemize}
\item {Grp. gram.:adj.}
\end{itemize}
Parecido á araucária.
\section{Araucaritas}
\begin{itemize}
\item {Grp. gram.:f. pl.}
\end{itemize}
\begin{itemize}
\item {Proveniência:(De \textunderscore araucária\textunderscore )}
\end{itemize}
Gênero de coníferas fósseis.
\section{Araucas}
\begin{itemize}
\item {Grp. gram.:m. pl.}
\end{itemize}
Uma das tríbos selvagens da Guiana inglesa.
\section{Araúja}
\begin{itemize}
\item {Grp. gram.:f.}
\end{itemize}
Árvore brasileira, de grandes flôres brancas e côr de rosa, (\textunderscore arauja sericifera\textunderscore , Brot.).
\section{Aráuja}
\begin{itemize}
\item {Grp. gram.:f.}
\end{itemize}
Árvore brasileira, de grandes flôres brancas e côr de rosa, (\textunderscore arauja sericifera\textunderscore , Brot.).
\section{Araújo}
\begin{itemize}
\item {Grp. gram.:m.}
\end{itemize}
\begin{itemize}
\item {Utilização:Prov.}
\end{itemize}
\begin{itemize}
\item {Utilização:minh.}
\end{itemize}
O mesmo que \textunderscore arujo\textunderscore , argueiro.
\section{Araújo}
\begin{itemize}
\item {Grp. gram.:m.}
\end{itemize}
O mesmo que \textunderscore araúja\textunderscore .
\section{Arauto}
\begin{itemize}
\item {Grp. gram.:m.}
\end{itemize}
Funccionário, que, na Idade-Média, annunciava as funcções públicas, ou era encarregado de declarar a guerra a povos estrangeiros.
Pregoeiro.
Postilhão.
(B. lat. \textunderscore haraldus\textunderscore )
\section{Araveça}
\begin{itemize}
\item {Grp. gram.:f.}
\end{itemize}
\begin{itemize}
\item {Proveniência:(De \textunderscore arar\textunderscore  + ?)}
\end{itemize}
Espécie de charrua, com uma só aiveca, que póde mudar-se de um para outro lado.
\section{Arável}
\begin{itemize}
\item {Grp. gram.:adj.}
\end{itemize}
\begin{itemize}
\item {Proveniência:(Lat. \textunderscore arabilis\textunderscore )}
\end{itemize}
Que póde sêr arado ou lavrado.
\section{Aravela}
\begin{itemize}
\item {Grp. gram.:f.}
\end{itemize}
\begin{itemize}
\item {Utilização:T. da Bairrada}
\end{itemize}
O mesmo que \textunderscore rabela\textunderscore .
\section{Aravia}
\begin{itemize}
\item {Grp. gram.:f.}
\end{itemize}
\begin{itemize}
\item {Utilização:Deprec.}
\end{itemize}
Língua árabe.
Línguagem rasteira.
Expressão obscura.
Algaravia.
(Por \textunderscore arabia\textunderscore , de \textunderscore árabe\textunderscore )
\section{Araxá}
\begin{itemize}
\item {Grp. gram.:m.}
\end{itemize}
\begin{itemize}
\item {Utilização:Bras}
\end{itemize}
\begin{itemize}
\item {Proveniência:(T. tupi-guarani, de \textunderscore ara\textunderscore  + \textunderscore xá\textunderscore )}
\end{itemize}
Planalto.
\section{Araxixu}
\begin{itemize}
\item {Grp. gram.:m.}
\end{itemize}
\begin{itemize}
\item {Utilização:Bras}
\end{itemize}
\begin{itemize}
\item {Proveniência:(T. tupi)}
\end{itemize}
O mesmo que \textunderscore erva-moira\textunderscore .
\section{Arazóia}
\begin{itemize}
\item {Grp. gram.:f.}
\end{itemize}
\begin{itemize}
\item {Utilização:Bras}
\end{itemize}
Fraldão de pennas, usado por mulheres indígenas.
\section{Arbi}
\begin{itemize}
\item {Grp. gram.:m.}
\end{itemize}
O mesmo que \textunderscore arbim\textunderscore .
\section{Arbim}
\begin{itemize}
\item {Grp. gram.:m.}
\end{itemize}
Pano grosseiro, antigo.
Antigo traje de camponês.
(Por \textunderscore arabim\textunderscore , de \textunderscore árabe\textunderscore ?)
\section{Arbitrador}
\begin{itemize}
\item {Grp. gram.:m.}
\end{itemize}
\begin{itemize}
\item {Proveniência:(Lat. \textunderscore arbitrator\textunderscore )}
\end{itemize}
Aquelle que \textunderscore arbitra\textunderscore .
\section{Arbitragem}
\begin{itemize}
\item {Grp. gram.:f.}
\end{itemize}
\begin{itemize}
\item {Proveniência:(De \textunderscore arbitrar\textunderscore )}
\end{itemize}
Julgamento, feito por árbitro, ou árbitros.
\section{Arbitral}
\begin{itemize}
\item {Grp. gram.:adj.}
\end{itemize}
\begin{itemize}
\item {Grp. gram.:adj.}
\end{itemize}
\begin{itemize}
\item {Proveniência:(Lat. \textunderscore arbitralis\textunderscore )}
\end{itemize}
Relativo a árbitros.
Feito por árbitros: \textunderscore decisão arbitral\textunderscore .
O mesmo que \textunderscore arbitrário\textunderscore . Cf. Filinto, V, 63.
\section{Arbitralmente}
\begin{itemize}
\item {Grp. gram.:adv.}
\end{itemize}
Por meio de árbitros.
\section{Arbitramento}
\begin{itemize}
\item {Grp. gram.:m.}
\end{itemize}
Acto de \textunderscore arbitrar\textunderscore .
\section{Arbitrar}
\begin{itemize}
\item {Grp. gram.:v. t.}
\end{itemize}
\begin{itemize}
\item {Proveniência:(Lat. \textunderscore arbitrare\textunderscore )}
\end{itemize}
Julgar com árbitros.
Determinar por arbítrio.
\section{Arbitrariamente}
\begin{itemize}
\item {Grp. gram.:adv.}
\end{itemize}
De modo \textunderscore arbitrário\textunderscore .
\section{Arbitrariedade}
\begin{itemize}
\item {Grp. gram.:f.}
\end{itemize}
Procedimento arbitrário; capricho.
\section{Arbitrário}
\begin{itemize}
\item {Grp. gram.:adj.}
\end{itemize}
\begin{itemize}
\item {Proveniência:(Lat. \textunderscore arbitrarius\textunderscore )}
\end{itemize}
Procedente de arbítrio.
Que não tem regras.
Não permittido.
Despótico.
\section{Arbitrativo}
\begin{itemize}
\item {Grp. gram.:adj.}
\end{itemize}
Que depende de arbítrio.
\section{Arbitreiro}
\begin{itemize}
\item {Grp. gram.:m.}
\end{itemize}
O mesmo que \textunderscore arbitrista\textunderscore .
\section{Arbítrio}
\begin{itemize}
\item {Grp. gram.:m.}
\end{itemize}
\begin{itemize}
\item {Proveniência:(Lat. \textunderscore arbitrium\textunderscore )}
\end{itemize}
Resolução, dependente da vontade.
Julgamento de árbitros.
Opinião.
Meio.
Alvitre.
\section{Arbitrista}
\begin{itemize}
\item {Grp. gram.:m.}
\end{itemize}
\begin{itemize}
\item {Proveniência:(De \textunderscore arbítrio\textunderscore )}
\end{itemize}
Aquelle que planeia meios extraordinários para conseguir um fim.
Aquelle que regula ou determina qualquer coisa:«\textunderscore arbitrista das modas\textunderscore ». \textunderscore Anat. Joc.\textunderscore , pról.
\section{Árbitro}
\begin{itemize}
\item {Grp. gram.:m.}
\end{itemize}
\begin{itemize}
\item {Proveniência:(Lat. \textunderscore arbiter\textunderscore )}
\end{itemize}
Aquelle que resolve questões, por consenso das partes litigantes.
Senhor absoluto.
Modêlo, exemplar: \textunderscore Petrónio, árbitro das elegâncias...\textunderscore 
\section{Arboés}
\begin{itemize}
\item {Grp. gram.:m. pl.}
\end{itemize}
\begin{itemize}
\item {Utilização:Açor}
\end{itemize}
O mesmo que \textunderscore clarabóia\textunderscore .
\section{Arbóis}
\begin{itemize}
\item {Grp. gram.:m. pl.}
\end{itemize}
\begin{itemize}
\item {Utilização:Açor}
\end{itemize}
O mesmo que \textunderscore clarabóia\textunderscore .
\section{Arbóreo}
\begin{itemize}
\item {Grp. gram.:adj.}
\end{itemize}
\begin{itemize}
\item {Proveniência:(Lat. \textunderscore arboreus\textunderscore )}
\end{itemize}
Relativo a árvore.
Semelhante a árvore ou que tem a altura de uma árvore ordinária: \textunderscore fêto arbóreo\textunderscore .
\section{Arborescência}
\begin{itemize}
\item {Grp. gram.:f.}
\end{itemize}
Qualidade do que é arborescente.
\section{Arborescente}
\begin{itemize}
\item {Grp. gram.:adj.}
\end{itemize}
\begin{itemize}
\item {Proveniência:(Lat. \textunderscore arborescens\textunderscore )}
\end{itemize}
Diz-se das plantas herbáceas, cujos ramos adquirem consistência, como os das árvores.
Que toma as proporções de uma árvore.
\section{Arborescer}
\begin{itemize}
\item {Grp. gram.:v. i.}
\end{itemize}
\begin{itemize}
\item {Proveniência:(Lat. \textunderscore arborescere\textunderscore )}
\end{itemize}
Tornar-se árvore; crescer como a árvore.
\section{Arborícola}
\begin{itemize}
\item {Grp. gram.:adj.}
\end{itemize}
\begin{itemize}
\item {Proveniência:(Do lat. \textunderscore arbor\textunderscore  + \textunderscore colere\textunderscore )}
\end{itemize}
Que vive nas árvores.
\section{Arboricultor}
\begin{itemize}
\item {Grp. gram.:m.}
\end{itemize}
\begin{itemize}
\item {Proveniência:(Do lat. \textunderscore arbor\textunderscore  + \textunderscore cultor\textunderscore )}
\end{itemize}
Aquelle que trata da cultura das árvores.
\section{Arboricultura}
\begin{itemize}
\item {Grp. gram.:f.}
\end{itemize}
\begin{itemize}
\item {Proveniência:(Do lat. \textunderscore arbor\textunderscore  + \textunderscore cultura\textunderscore )}
\end{itemize}
Cultura das árvores.
\section{Arboriforme}
\begin{itemize}
\item {Grp. gram.:adj.}
\end{itemize}
\begin{itemize}
\item {Proveniência:(Do lat. \textunderscore arbor\textunderscore  + \textunderscore forma\textunderscore )}
\end{itemize}
Que tem fórma de árvore.
\section{Arborista}
\begin{itemize}
\item {Grp. gram.:m.}
\end{itemize}
(V.arboricultor)
\section{Arborização}
\begin{itemize}
\item {Grp. gram.:f.}
\end{itemize}
Acto de \textunderscore arborizar\textunderscore .
\section{Arborizado}
\begin{itemize}
\item {Grp. gram.:adj.}
\end{itemize}
\begin{itemize}
\item {Utilização:Miner.}
\end{itemize}
Plantado de árvores: \textunderscore a Praça de Affonso de Albuquerque é arborizada\textunderscore .
Diz-se dos mineraes, que apresentam veios ramificados.
\section{Arborizar}
\begin{itemize}
\item {Grp. gram.:v. t.}
\end{itemize}
\begin{itemize}
\item {Proveniência:(Do lat. \textunderscore arbor\textunderscore )}
\end{itemize}
Plantar árvores em.
\section{Arbúscula}
\begin{itemize}
\item {Grp. gram.:f.}
\end{itemize}
\begin{itemize}
\item {Proveniência:(Lat. \textunderscore arbuscula\textunderscore )}
\end{itemize}
O mesmo ou melhor que \textunderscore arbúsculo\textunderscore .
\section{Arbuscular}
\begin{itemize}
\item {Grp. gram.:adj.}
\end{itemize}
\begin{itemize}
\item {Proveniência:(De \textunderscore arbúsculo\textunderscore )}
\end{itemize}
Ramificado como uma árvore.
\section{Arbúsculo}
\begin{itemize}
\item {Grp. gram.:m.}
\end{itemize}
Pequeno arbusto.
(Cp. \textunderscore arbúscula\textunderscore )
\section{Arbústeo}
\begin{itemize}
\item {Grp. gram.:adj.}
\end{itemize}
Pertencente á classe dos arbustos.
Relativo a arbusto.
\section{Arbustiforme}
\begin{itemize}
\item {Grp. gram.:adj.}
\end{itemize}
\begin{itemize}
\item {Proveniência:(Do lat. \textunderscore arbustum\textunderscore  + \textunderscore forma\textunderscore )}
\end{itemize}
Que tem fórma de arbusto.
\section{Arbustivo}
\begin{itemize}
\item {Grp. gram.:adj.}
\end{itemize}
\begin{itemize}
\item {Proveniência:(Lat. \textunderscore arbustivus\textunderscore )}
\end{itemize}
Relativo a arbustos.
\section{Arbusto}
\begin{itemize}
\item {Grp. gram.:m.}
\end{itemize}
\begin{itemize}
\item {Proveniência:(Lat. \textunderscore arbustum\textunderscore )}
\end{itemize}
Pequena árvore.
\section{Arbutáceas}
\begin{itemize}
\item {Grp. gram.:f. pl.}
\end{itemize}
\begin{itemize}
\item {Proveniência:(De \textunderscore árbuto\textunderscore )}
\end{itemize}
Família de plantas, que têm por typo o medronheiro.
\section{Arbutina}
\begin{itemize}
\item {Grp. gram.:f.}
\end{itemize}
\begin{itemize}
\item {Proveniência:(De \textunderscore árbuto\textunderscore )}
\end{itemize}
Medicamento diurético, applicado contra o catarro vesical.
\section{Árbuto}
\begin{itemize}
\item {Grp. gram.:m.}
\end{itemize}
\begin{itemize}
\item {Proveniência:(Lat. \textunderscore arbutum\textunderscore )}
\end{itemize}
Gênero de plantas, a que pertence o medronheiro.
\section{Arca}
\begin{itemize}
\item {Grp. gram.:f.}
\end{itemize}
\begin{itemize}
\item {Utilização:Prov.}
\end{itemize}
\begin{itemize}
\item {Utilização:trasm.}
\end{itemize}
\begin{itemize}
\item {Utilização:Ant.}
\end{itemize}
\begin{itemize}
\item {Grp. gram.:Loc.}
\end{itemize}
\begin{itemize}
\item {Proveniência:(Lat. \textunderscore arca\textunderscore )}
\end{itemize}
Grande caixa, de tampa chata.
Cofre.
Thesoiro.
Reservatório.
Peito.
O mesmo que \textunderscore abraço\textunderscore .
Gênero de molluscos bivalves.
Costado. Cêsto de gávea.
Monte de pedras, que serve de baliza ou marco.
O mesmo que \textunderscore anta\textunderscore ^1.
\textunderscore Arca de Noé\textunderscore , embarcação, em que Noé se salvou do dilúvio, segundo a narração bíblica.
\section{Arçã}
\begin{itemize}
\item {Grp. gram.:f.}
\end{itemize}
\begin{itemize}
\item {Utilização:Prov.}
\end{itemize}
\begin{itemize}
\item {Utilização:trasm.}
\end{itemize}
O mesmo que \textunderscore tomilho\textunderscore  ou \textunderscore rosmaninho\textunderscore .
\section{Arcabém}
\begin{itemize}
\item {Grp. gram.:m.}
\end{itemize}
Parte posterior das grades de vêrga, que constituem a sebe de um carro.
\section{Arcaboiço}
\begin{itemize}
\item {Grp. gram.:m.}
\end{itemize}
\begin{itemize}
\item {Proveniência:(De \textunderscore arca\textunderscore  + ?)}
\end{itemize}
Peito.
Madeiramento de uma construcção.
Esqueleto.
\section{Arcabouço}
\begin{itemize}
\item {Grp. gram.:m.}
\end{itemize}
\begin{itemize}
\item {Proveniência:(De \textunderscore arca\textunderscore  + ?)}
\end{itemize}
Peito.
Madeiramento de uma construcção.
Esqueleto.
\section{Arcabuz}
\begin{itemize}
\item {Grp. gram.:m.}
\end{itemize}
\begin{itemize}
\item {Proveniência:(Do neerl. \textunderscore haakbus\textunderscore , caixa de gancho)}
\end{itemize}
Antiga arma de fogo, de cano curto e largo.
\section{Arcabuzaço}
\begin{itemize}
\item {Grp. gram.:m.}
\end{itemize}
O mesmo que \textunderscore arcabuzada\textunderscore .
\section{Arcabuzada}
\begin{itemize}
\item {Grp. gram.:f.}
\end{itemize}
Tiro de arcabuz.
\section{Arcabuzamento}
\begin{itemize}
\item {Grp. gram.:m.}
\end{itemize}
Acto ou effeito de \textunderscore arcabuzar\textunderscore .
\section{Arcabuzar}
\begin{itemize}
\item {Grp. gram.:v. t.}
\end{itemize}
Matar com tiros de arcabuz.
Espingardear.
\section{Arcabuzaria}
\begin{itemize}
\item {Grp. gram.:f.}
\end{itemize}
Descarga de arcabuzes.
Tropa armada de arcabuzes.
\section{Arcabuzear}
\begin{itemize}
\item {Grp. gram.:v. t.}
\end{itemize}
O mesmo que \textunderscore arcabuzar\textunderscore .
\section{Arcabuzeiro}
\begin{itemize}
\item {Grp. gram.:m.}
\end{itemize}
Aquelle que fabríca arcabuzes.
Aquelle que se arma com êlles.
\section{Arcabuzeta}
\begin{itemize}
\item {fónica:zê}
\end{itemize}
\begin{itemize}
\item {Grp. gram.:m.}
\end{itemize}
Pequeno arcabuz, usado outrora por cavalleiros.
\section{Arcáceos}
\begin{itemize}
\item {Grp. gram.:m. pl.}
\end{itemize}
O mesmo que \textunderscore arcádeos\textunderscore .
\section{Arcada}
\begin{itemize}
\item {Grp. gram.:f.}
\end{itemize}
Série de arcos.
Abóbada arqueada.
Corrida do arco sôbre as cordas de um instrumento musical.
Movimento do peito, quando se respira com ânsia.
\section{Árcade}
\begin{itemize}
\item {Grp. gram.:m.}
\end{itemize}
\begin{itemize}
\item {Proveniência:(Lat. \textunderscore arcas\textunderscore , \textunderscore árcadis\textunderscore )}
\end{itemize}
Aquelle que é natural da Arcádia.
Membro de certas academias que se chamaram Arcádias.
\section{Arcádeos}
\begin{itemize}
\item {Grp. gram.:m. pl.}
\end{itemize}
\begin{itemize}
\item {Proveniência:(Do lat. \textunderscore arca\textunderscore  + gr. \textunderscore eidos\textunderscore )}
\end{itemize}
Família de molluscos.
\section{Arcades-ambo}
\begin{itemize}
\item {fónica:árcadès-ambó}
\end{itemize}
\begin{itemize}
\item {Grp. gram.:adj. pl.}
\end{itemize}
Tal um, tal outro; ambos do mesmo jaêz.
(Loc. lat.)
\section{Arcádico}
\begin{itemize}
\item {Grp. gram.:adj.}
\end{itemize}
\begin{itemize}
\item {Proveniência:(Lat. \textunderscore arcadicus\textunderscore )}
\end{itemize}
Relativo á Arcádia ou ás academias dêste nome.
\section{Arcadismo}
\begin{itemize}
\item {Grp. gram.:m.}
\end{itemize}
Influência literária das Arcádias.
\section{Arcado}
\begin{itemize}
\item {Grp. gram.:adj.}
\end{itemize}
O mesmo que [[arqueado|arquear]].
\section{Árcado}
\begin{itemize}
\item {Grp. gram.:m.}
\end{itemize}
O mesmo que \textunderscore árcade\textunderscore .
\section{Arcadura}
\begin{itemize}
\item {Grp. gram.:f.}
\end{itemize}
\begin{itemize}
\item {Proveniência:(De \textunderscore arcar\textunderscore )}
\end{itemize}
O mesmo que \textunderscore curvatura\textunderscore .
\section{Arcaico}
\begin{itemize}
\item {Grp. gram.:adj.}
\end{itemize}
\begin{itemize}
\item {Proveniência:(Do gr. \textunderscore arkhaios\textunderscore )}
\end{itemize}
Que contém arcaismo.
Antiquado: \textunderscore expressão arcaica\textunderscore .
\section{Arcainha}
\begin{itemize}
\item {Grp. gram.:f.}
\end{itemize}
\begin{itemize}
\item {Utilização:Prov.}
\end{itemize}
\begin{itemize}
\item {Utilização:beir.}
\end{itemize}
Pequena anta ou dólmen.
(Cp. \textunderscore arca\textunderscore )
\section{Arcaísmo}
\begin{itemize}
\item {Grp. gram.:m.}
\end{itemize}
\begin{itemize}
\item {Proveniência:(Lat. \textunderscore archaismus\textunderscore )}
\end{itemize}
Modo antiquado de falar ou de escrever.
Locução arcaica.
\section{Arcaísta}
\begin{itemize}
\item {Grp. gram.:m.  e  adj.}
\end{itemize}
O que emprega arcaísmos.
\section{Arcaístico}
\begin{itemize}
\item {Grp. gram.:adj.}
\end{itemize}
O mesmo que \textunderscore arcaico\textunderscore .
\section{Arcaizar-se}
\begin{itemize}
\item {Grp. gram.:v. p.}
\end{itemize}
Tornar-se arcaico, desusado, obsoleto.
\section{Arcal}
\begin{itemize}
\item {Grp. gram.:m.}
\end{itemize}
O mesmo que arcale.
\section{Arcale}
\begin{itemize}
\item {Grp. gram.:m.}
\end{itemize}
Espécie de esteva.
\section{Arcalião}
\begin{itemize}
\item {Grp. gram.:m.}
\end{itemize}
Espécie de dormideira.
\section{Arçan}
\begin{itemize}
\item {Grp. gram.:f.}
\end{itemize}
\begin{itemize}
\item {Utilização:Prov.}
\end{itemize}
\begin{itemize}
\item {Utilização:trasm.}
\end{itemize}
O mesmo que \textunderscore tomilho\textunderscore  ou \textunderscore rosmaninho\textunderscore .
\section{Arcane}
\begin{itemize}
\item {Grp. gram.:m.}
\end{itemize}
Composição metállica, que serve para estanhar metaes.
\section{Arcânea}
\begin{itemize}
\item {Grp. gram.:f.}
\end{itemize}
Espécie de crustáceo decápode.
\section{Arcangélica}
\begin{itemize}
\item {Grp. gram.:f.}
\end{itemize}
Gênero de plantas umbellíferas, pouco differentes da angélica.
\section{Arcangélico}
\begin{itemize}
\item {Grp. gram.:adj.}
\end{itemize}
\begin{itemize}
\item {Proveniência:(Lat. \textunderscore archangelicus\textunderscore )}
\end{itemize}
Relativo a arcanjo.
\section{Arçanha}
\begin{itemize}
\item {Grp. gram.:f.}
\end{itemize}
\begin{itemize}
\item {Utilização:Prov.}
\end{itemize}
\begin{itemize}
\item {Utilização:trasm.}
\end{itemize}
O mesmo que \textunderscore arçan\textunderscore .
\section{Arçanhal}
\begin{itemize}
\item {Grp. gram.:m.}
\end{itemize}
\begin{itemize}
\item {Utilização:Prov.}
\end{itemize}
\begin{itemize}
\item {Utilização:trasm.}
\end{itemize}
Campo de arçanhas.
\section{Arcanidade}
\begin{itemize}
\item {Grp. gram.:f.}
\end{itemize}
\begin{itemize}
\item {Utilização:Ant.}
\end{itemize}
O mesmo que \textunderscore arcano\textunderscore .
\section{Arcanita}
\begin{itemize}
\item {Grp. gram.:f.}
\end{itemize}
Sulfato de potassa míneral, que se acha em solução nas águas.
\section{Arcanjo}
\begin{itemize}
\item {Grp. gram.:m.}
\end{itemize}
\begin{itemize}
\item {Proveniência:(Do lat. \textunderscore archangelus\textunderscore )}
\end{itemize}
Anjo, de ordem superior.
\section{Arcano}
\begin{itemize}
\item {Grp. gram.:m.}
\end{itemize}
\begin{itemize}
\item {Grp. gram.:Adj.}
\end{itemize}
\begin{itemize}
\item {Proveniência:(Do lat. \textunderscore arcanus\textunderscore )}
\end{itemize}
Segrêdo, mystério.
Mysterioso; occulto.
\section{Arção}
\begin{itemize}
\item {Grp. gram.:m.}
\end{itemize}
Peça arqueada, que limita a sella adeante e atrás.
(Cast. \textunderscore arzon\textunderscore )
\section{Arcar}
\begin{itemize}
\item {Grp. gram.:v. t.}
\end{itemize}
O mesmo que \textunderscore arquear\textunderscore .
\section{Arcar}
\begin{itemize}
\item {Grp. gram.:v. i.}
\end{itemize}
\begin{itemize}
\item {Proveniência:(De \textunderscore arca\textunderscore )}
\end{itemize}
Lutar: \textunderscore arcar com difficuldades\textunderscore .
\section{Arcaria}
\begin{itemize}
\item {Grp. gram.:f.}
\end{itemize}
Série de arcos.
O mesmo que \textunderscore arcada\textunderscore .
\section{Arcário}
\begin{itemize}
\item {Grp. gram.:m.}
\end{itemize}
\begin{itemize}
\item {Proveniência:(Lat. \textunderscore arcarius\textunderscore )}
\end{itemize}
O encarregado de cofre público ou de cofre de communidade; thesoireiro.
Cobrador ou recebedor de impostos, no tempo do Império Romano. Cf. Herculano, \textunderscore Hist. de Port.\textunderscore , IV, 27.
\section{Arcas-encoiradas}
\begin{itemize}
\item {Grp. gram.:f. pl.}
\end{itemize}
Hábito de dissimular; fingimento; impostura.
\section{Arcatura}
\begin{itemize}
\item {Grp. gram.:f.}
\end{itemize}
\begin{itemize}
\item {Proveniência:(Lat. \textunderscore arcatura\textunderscore )}
\end{itemize}
Arcada fingida, que se usava na architectura romana.
\section{Arcaz}
\begin{itemize}
\item {Grp. gram.:m.}
\end{itemize}
Arca grande com gavetões.
(Cast. \textunderscore arcaz\textunderscore )
\section{Arcebispado}
\begin{itemize}
\item {Grp. gram.:m.}
\end{itemize}
Dignidade de Arcebispo.
Território, em que êlle exerce sua jurisdicção.
Residencia de Arcebispo.
\section{Arcebispal}
\begin{itemize}
\item {Grp. gram.:adj.}
\end{itemize}
O mesmo que \textunderscore archiepiscopal\textunderscore .
\section{Arcebispo}
\begin{itemize}
\item {Grp. gram.:m.}
\end{itemize}
\begin{itemize}
\item {Proveniência:(Do lat. \textunderscore archiepiscopus\textunderscore )}
\end{itemize}
Prelado, que tem Bispos suffragâneos.
\section{Arcediagado}
\begin{itemize}
\item {Grp. gram.:m.}
\end{itemize}
Dignidade de arcediago.
\section{Arcediago}
\begin{itemize}
\item {Grp. gram.:m.}
\end{itemize}
\begin{itemize}
\item {Utilização:Ant.}
\end{itemize}
\begin{itemize}
\item {Proveniência:(Do lat. \textunderscore archidiaconus\textunderscore )}
\end{itemize}
Dignitário dos cabidos.
Primeiro entre os diáconos.
\section{Arcediano}
\begin{itemize}
\item {Grp. gram.:m.}
\end{itemize}
\begin{itemize}
\item {Utilização:Ant.}
\end{itemize}
O mesmo que \textunderscore arcediago\textunderscore . Cf. \textunderscore Canc. da Vaticana\textunderscore .
\section{Arcela}
\begin{itemize}
\item {Grp. gram.:f.}
\end{itemize}
Gênero de protozoários.
\section{Arcelíneo}
\begin{itemize}
\item {Grp. gram.:adj.}
\end{itemize}
\begin{itemize}
\item {Grp. gram.:M. pl.}
\end{itemize}
Relativo ou semelhante á arcela.
Família de infusórios polygástricos, sem canal alimentar, com uma só abertura no corpo e appêndices variáveis.
\section{Arcella}
\begin{itemize}
\item {Grp. gram.:f.}
\end{itemize}
Gênero de protozoários.
\section{Arcellíneo}
\begin{itemize}
\item {Grp. gram.:adj.}
\end{itemize}
\begin{itemize}
\item {Grp. gram.:M. pl.}
\end{itemize}
Relativo ou semelhante á arcella.
Família de infusórios polygástricos, sem canal alimentar, com uma só abertura no corpo e appêndices variáveis.
\section{Arcer}
\begin{itemize}
\item {Grp. gram.:v. i.}
\end{itemize}
O mesmo que \textunderscore arder\textunderscore :«\textunderscore arço de ver nossas cousas hirem todas ao revés\textunderscore ». Gonç. Dias, \textunderscore Sextilhas\textunderscore . Cf. \textunderscore Aulegrafia\textunderscore , 40.
(Por \textunderscore arser\textunderscore , do lat. \textunderscore arsus\textunderscore ?)
\section{Árcera}
\begin{itemize}
\item {Grp. gram.:f.}
\end{itemize}
\begin{itemize}
\item {Proveniência:(Lat. \textunderscore arcera\textunderscore )}
\end{itemize}
Carro romano, espécie de léctica.
\section{Arcésthida}
\begin{itemize}
\item {Grp. gram.:f.}
\end{itemize}
\begin{itemize}
\item {Proveniência:(Do gr. \textunderscore arkesthis\textunderscore )}
\end{itemize}
Fruto do zimbro.
Fruto semelhante ao do zimbro.
\section{Arcéstida}
\begin{itemize}
\item {Grp. gram.:f.}
\end{itemize}
\begin{itemize}
\item {Proveniência:(Do gr. \textunderscore arkesthis\textunderscore )}
\end{itemize}
Fruto do zimbro.
Fruto semelhante ao do zimbro.
\section{Arcete}
\begin{itemize}
\item {fónica:cê}
\end{itemize}
\begin{itemize}
\item {Grp. gram.:m.}
\end{itemize}
\begin{itemize}
\item {Proveniência:(De \textunderscore arco\textunderscore )}
\end{itemize}
Serra, para cortar pedras.
\section{Archa}
\begin{itemize}
\item {Grp. gram.:f.}
\end{itemize}
\begin{itemize}
\item {Proveniência:(Do lat. \textunderscore ascia\textunderscore )}
\end{itemize}
Arma antiga, de que usavam os guardas do paço.
\section{Archaico}
\begin{itemize}
\item {fónica:cai}
\end{itemize}
\begin{itemize}
\item {Grp. gram.:adj.}
\end{itemize}
\begin{itemize}
\item {Proveniência:(Do gr. \textunderscore arkhaios\textunderscore )}
\end{itemize}
Que contém archaismo.
Antiquado: \textunderscore expressão archaica\textunderscore .
\section{Archaísmo}
\begin{itemize}
\item {fónica:ca}
\end{itemize}
\begin{itemize}
\item {Grp. gram.:m.}
\end{itemize}
\begin{itemize}
\item {Proveniência:(Lat. \textunderscore archaismus\textunderscore )}
\end{itemize}
Modo antiquado de falar ou de escrever.
Locução archaica.
\section{Archaísta}
\begin{itemize}
\item {fónica:ca}
\end{itemize}
\begin{itemize}
\item {Grp. gram.:m.  e  adj.}
\end{itemize}
O que emprega archaísmos.
\section{Archaístico}
\begin{itemize}
\item {fónica:ca}
\end{itemize}
\begin{itemize}
\item {Grp. gram.:adj.}
\end{itemize}
O mesmo que \textunderscore archaico\textunderscore .
\section{Archaizar-se}
\begin{itemize}
\item {fónica:ca-i}
\end{itemize}
\begin{itemize}
\item {Grp. gram.:v. p.}
\end{itemize}
Tornar-se archaico, desusado, obsoleto.
\section{Archangélica}
\begin{itemize}
\item {fónica:can}
\end{itemize}
\begin{itemize}
\item {Grp. gram.:f.}
\end{itemize}
Gênero de plantas umbellíferas, pouco differentes da angélica.
\section{Archangélico}
\begin{itemize}
\item {fónica:can}
\end{itemize}
\begin{itemize}
\item {Grp. gram.:adj.}
\end{itemize}
\begin{itemize}
\item {Proveniência:(Lat. \textunderscore archangelicus\textunderscore )}
\end{itemize}
Relativo a archanjo.
\section{Archanjo}
\begin{itemize}
\item {fónica:can}
\end{itemize}
\begin{itemize}
\item {Grp. gram.:m.}
\end{itemize}
\begin{itemize}
\item {Proveniência:(Do lat. \textunderscore archangelus\textunderscore )}
\end{itemize}
Anjo, de ordem superior.
\section{Archegónio}
\begin{itemize}
\item {fónica:que}
\end{itemize}
\begin{itemize}
\item {Grp. gram.:m.}
\end{itemize}
O órgão feminino das algas, das plantas hepáticas e das criptogâmicas.
\section{Archeiro}
\begin{itemize}
\item {Grp. gram.:m.}
\end{itemize}
\begin{itemize}
\item {Utilização:Gír.}
\end{itemize}
\begin{itemize}
\item {Proveniência:(De \textunderscore archa\textunderscore )}
\end{itemize}
Alabardeiro.
Aquelle que se arma com archa.
Ébrio.
\section{Archeogeologia}
\begin{itemize}
\item {fónica:que}
\end{itemize}
\begin{itemize}
\item {Grp. gram.:f.}
\end{itemize}
Geologia prehistórica.
\section{Archeographia}
\begin{itemize}
\item {fónica:que}
\end{itemize}
\begin{itemize}
\item {Grp. gram.:f.}
\end{itemize}
Descripção dos monumentos antigos.
(Cp. \textunderscore archeógrapho\textunderscore )
\section{Archeógrapho}
\begin{itemize}
\item {fónica:que}
\end{itemize}
\begin{itemize}
\item {Grp. gram.:m.}
\end{itemize}
\begin{itemize}
\item {Proveniência:(Do gr. \textunderscore arkhaios\textunderscore  + \textunderscore graphein\textunderscore )}
\end{itemize}
Aquelle que descreve monumentos antigos.
\section{Archeolíthico}
\begin{itemize}
\item {fónica:que-o}
\end{itemize}
\begin{itemize}
\item {Grp. gram.:adj.}
\end{itemize}
\begin{itemize}
\item {Proveniência:(Do gr. \textunderscore arkhaios\textunderscore  + \textunderscore lithos\textunderscore )}
\end{itemize}
Relativo ás rochas das primeiras idades geológicas.
\section{Archeologia}
\begin{itemize}
\item {fónica:que-o}
\end{itemize}
\begin{itemize}
\item {Grp. gram.:f.}
\end{itemize}
Estudo de coisas antigas.
(Cp. \textunderscore archeólogo\textunderscore )
\section{Archeológico}
\begin{itemize}
\item {fónica:que-o}
\end{itemize}
\begin{itemize}
\item {Grp. gram.:adj.}
\end{itemize}
Relativo á \textunderscore Archeologia\textunderscore .
\section{Archeólogo}
\begin{itemize}
\item {fónica:que}
\end{itemize}
\begin{itemize}
\item {Grp. gram.:m.}
\end{itemize}
\begin{itemize}
\item {Proveniência:(Do gr. \textunderscore arkhaios\textunderscore  + \textunderscore logos\textunderscore )}
\end{itemize}
Aquelle que se dedica á Archeologia ou é versado nella.
\section{Archeozoítico}
\begin{itemize}
\item {Grp. gram.:adj.}
\end{itemize}
\begin{itemize}
\item {Utilização:Geol.}
\end{itemize}
\begin{itemize}
\item {Proveniência:(Do gr. \textunderscore arkhaios\textunderscore  + \textunderscore zoon\textunderscore )}
\end{itemize}
Diz-se da primeira phase do período philogenético, na qual só havia na terra animaes invertebrados.
\section{Archeta}
\begin{itemize}
\item {fónica:chê}
\end{itemize}
\begin{itemize}
\item {Grp. gram.:f.}
\end{itemize}
\begin{itemize}
\item {Utilização:Ant.}
\end{itemize}
\begin{itemize}
\item {Proveniência:(De \textunderscore arca\textunderscore , sob infl. do fr. \textunderscore arche\textunderscore )}
\end{itemize}
Mealheiro para receber esmolas.
Arca de bufarinheiro.
\section{Archete}
\begin{itemize}
\item {fónica:chê}
\end{itemize}
\begin{itemize}
\item {Grp. gram.:m.}
\end{itemize}
\begin{itemize}
\item {Proveniência:(De \textunderscore arca\textunderscore , sob infl. do fr. \textunderscore arche\textunderscore )}
\end{itemize}
Pequena arca. Urna cinerária.
\section{Archete}
\begin{itemize}
\item {fónica:chê}
\end{itemize}
\begin{itemize}
\item {Grp. gram.:m.}
\end{itemize}
\begin{itemize}
\item {Utilização:Prov.}
\end{itemize}
\begin{itemize}
\item {Utilização:alent.}
\end{itemize}
\begin{itemize}
\item {Utilização:Constr.}
\end{itemize}
Ornato, em fórma de arco, nas tapeçarias antigas.
Instrumento cirúrgico, usado na lithotrícia.
Vêrga de porta ou janela, feita de tijolo e em arco.
Contrapadieira, ou a pedra que perfaz o resto da espessura da parede, atrás da padieira de madeira de porta ou de janela.
(Fr. archet).
\section{Archétypo}
\begin{itemize}
\item {fónica:qué}
\end{itemize}
\begin{itemize}
\item {Grp. gram.:m.}
\end{itemize}
\begin{itemize}
\item {Proveniência:(Lat. \textunderscore archetypum\textunderscore )}
\end{itemize}
Modêlo dos seres criados.
Exemplar.
Padrão.
\section{Archeu}
\begin{itemize}
\item {fónica:queu}
\end{itemize}
\begin{itemize}
\item {Grp. gram.:m.}
\end{itemize}
\begin{itemize}
\item {Utilização:Des.}
\end{itemize}
\begin{itemize}
\item {Proveniência:(Do gr. \textunderscore arkheios\textunderscore )}
\end{itemize}
Vigor, energia. Cf. Cortesão, \textunderscore Subs.\textunderscore 
\section{Archi...}
\begin{itemize}
\item {fónica:qui}
\end{itemize}
\begin{itemize}
\item {Proveniência:(Do gr. \textunderscore arkhein\textunderscore )}
\end{itemize}
\textunderscore pref.\textunderscore  de superioridade: \textunderscore conheço-o: é um jornalista architolo\textunderscore .
\section{Archiapóstata}
\begin{itemize}
\item {fónica:qui}
\end{itemize}
\begin{itemize}
\item {Grp. gram.:m.}
\end{itemize}
O maior dos apóstatas.
\section{Archiatro}
\begin{itemize}
\item {fónica:qui}
\end{itemize}
\begin{itemize}
\item {Grp. gram.:m.}
\end{itemize}
\begin{itemize}
\item {Proveniência:(Lat. \textunderscore archiatrus\textunderscore )}
\end{itemize}
O médico principal.
O médico do rei.
\section{Archibancada}
\begin{itemize}
\item {fónica:qui}
\end{itemize}
\begin{itemize}
\item {Grp. gram.:f.}
\end{itemize}
\begin{itemize}
\item {Utilização:Bras}
\end{itemize}
O mesmo que \textunderscore archibanco\textunderscore .
\section{Archibanco}
\begin{itemize}
\item {fónica:qui}
\end{itemize}
\begin{itemize}
\item {Grp. gram.:m.}
\end{itemize}
\begin{itemize}
\item {Proveniência:(De \textunderscore archi...\textunderscore  + \textunderscore banco\textunderscore )}
\end{itemize}
O maior banco de uma casa; banco grande de costas.
\section{Archiconfraria}
\begin{itemize}
\item {fónica:qui}
\end{itemize}
\begin{itemize}
\item {Grp. gram.:f.}
\end{itemize}
\begin{itemize}
\item {Proveniência:(De \textunderscore archi...\textunderscore  + \textunderscore confraria\textunderscore )}
\end{itemize}
Confraria principal.
\section{Archicýthara}
\begin{itemize}
\item {fónica:qui}
\end{itemize}
\begin{itemize}
\item {Grp. gram.:f.}
\end{itemize}
\begin{itemize}
\item {Utilização:Des.}
\end{itemize}
Instrumento de dois cravelhames e vinte e duas cordas.
\section{Archidiácono}
\begin{itemize}
\item {fónica:qui}
\end{itemize}
\begin{itemize}
\item {Grp. gram.:m.}
\end{itemize}
(V.Arcediago)
\section{Archidiocesano}
\begin{itemize}
\item {fónica:qui}
\end{itemize}
\begin{itemize}
\item {Grp. gram.:adj.}
\end{itemize}
Relativo a \textunderscore archidiocese\textunderscore .
\section{Archidiocese}
\begin{itemize}
\item {fónica:qui}
\end{itemize}
\begin{itemize}
\item {Grp. gram.:f.}
\end{itemize}
\begin{itemize}
\item {Proveniência:(De \textunderscore archi...\textunderscore  + \textunderscore diocese\textunderscore )}
\end{itemize}
Diocese, que tem outras suffragâneas.
Arcebispado.
\section{Archidivino}
\begin{itemize}
\item {fónica:qui}
\end{itemize}
\begin{itemize}
\item {Grp. gram.:adj.}
\end{itemize}
Superiormente divino. Cf. Castilho, \textunderscore Sabichonas\textunderscore , 231.
\section{Archiducado}
\begin{itemize}
\item {fónica:qui}
\end{itemize}
\begin{itemize}
\item {Grp. gram.:m.}
\end{itemize}
Dignidade ou território de \textunderscore Archiduque\textunderscore .
\section{Archiducal}
\begin{itemize}
\item {Grp. gram.:adj.}
\end{itemize}
(\textunderscore qui\textunderscore )
Pertencente a \textunderscore Archiduque\textunderscore .
\section{Archiduque}
\begin{itemize}
\item {fónica:qui}
\end{itemize}
\begin{itemize}
\item {Grp. gram.:m.}
\end{itemize}
\begin{itemize}
\item {Proveniência:(De \textunderscore archi...\textunderscore  + \textunderscore duque\textunderscore )}
\end{itemize}
Título dos Príncipes da Casa de Áustria.
\section{Archiduquesa}
\begin{itemize}
\item {fónica:qui}
\end{itemize}
\begin{itemize}
\item {Grp. gram.:f.}
\end{itemize}
Mulher de Archiduque.
Título das Princesas da Casa de Áustria.
\section{Archiepiscopado}
\begin{itemize}
\item {fónica:qui}
\end{itemize}
\begin{itemize}
\item {Grp. gram.:m.}
\end{itemize}
O mesmo que \textunderscore arcebispado\textunderscore . Cf. Herculano, \textunderscore Hist. de Port.\textunderscore , III, 417.
\section{Archiepiscopal}
\begin{itemize}
\item {fónica:qui}
\end{itemize}
\begin{itemize}
\item {Grp. gram.:adj.}
\end{itemize}
\begin{itemize}
\item {Proveniência:(De \textunderscore archi...\textunderscore  + \textunderscore episcopal\textunderscore )}
\end{itemize}
Relativo a Arcebispo.
\section{Archilaúde}
\begin{itemize}
\item {fónica:qui}
\end{itemize}
\begin{itemize}
\item {Grp. gram.:m.}
\end{itemize}
\begin{itemize}
\item {Proveniência:(Fr. archilute)}
\end{itemize}
Instrumento músico do cravelhame e dezasete cordas.
\section{Archilevita}
\begin{itemize}
\item {fónica:qui}
\end{itemize}
\begin{itemize}
\item {Grp. gram.:m.}
\end{itemize}
\begin{itemize}
\item {Proveniência:(De \textunderscore archi...\textunderscore  + \textunderscore levita\textunderscore )}
\end{itemize}
Chefe dos levitas, entre os Hebreus.
\section{Archilymphático}
\begin{itemize}
\item {fónica:qui}
\end{itemize}
\begin{itemize}
\item {Grp. gram.:adj.}
\end{itemize}
\begin{itemize}
\item {Utilização:Med.}
\end{itemize}
Lymphático em alto grau.
\section{Archimagiro}
\begin{itemize}
\item {fónica:qui}
\end{itemize}
\begin{itemize}
\item {Grp. gram.:m.}
\end{itemize}
\begin{itemize}
\item {Utilização:Ant.}
\end{itemize}
\begin{itemize}
\item {Proveniência:(Lat. \textunderscore archimagirus\textunderscore )}
\end{itemize}
Chefe de cozinha; chefe de cozinheiros.
\section{Archimago}
\begin{itemize}
\item {fónica:qui}
\end{itemize}
\begin{itemize}
\item {Grp. gram.:m.}
\end{itemize}
\begin{itemize}
\item {Proveniência:(De \textunderscore archi...\textunderscore  + \textunderscore mago\textunderscore )}
\end{itemize}
Chefe dos magos.
Chefe da religião, entre os antigos Persas.
\section{Archimandrita}
\begin{itemize}
\item {fónica:qui}
\end{itemize}
\begin{itemize}
\item {Grp. gram.:f.}
\end{itemize}
\begin{itemize}
\item {Proveniência:(Gr. \textunderscore arkhimandrites\textunderscore )}
\end{itemize}
Abbade de certos conventos.
\section{Archimandritado}
\begin{itemize}
\item {fónica:qui}
\end{itemize}
\begin{itemize}
\item {Grp. gram.:m.}
\end{itemize}
Dignidade de archimandrita.
\section{Archimimo}
\begin{itemize}
\item {fónica:qui}
\end{itemize}
\begin{itemize}
\item {Grp. gram.:m.}
\end{itemize}
\begin{itemize}
\item {Utilização:Ant.}
\end{itemize}
\begin{itemize}
\item {Proveniência:(Lat. \textunderscore archimimus\textunderscore )}
\end{itemize}
Chefe dos que representavam pantomimas.
\section{Archiministro}
\begin{itemize}
\item {fónica:qui}
\end{itemize}
\begin{itemize}
\item {Grp. gram.:m.}
\end{itemize}
\begin{itemize}
\item {Utilização:Des.}
\end{itemize}
\begin{itemize}
\item {Proveniência:(De \textunderscore archi...\textunderscore  + \textunderscore ministro\textunderscore )}
\end{itemize}
O primeiro Ministro.
\section{Archimorto}
\begin{itemize}
\item {fónica:qui}
\end{itemize}
\begin{itemize}
\item {Grp. gram.:adj.}
\end{itemize}
\begin{itemize}
\item {Utilização:Des.}
\end{itemize}
Que está bem morto; morto há muito tempo.
\section{Archimosteiro}
\begin{itemize}
\item {fónica:qui}
\end{itemize}
\begin{itemize}
\item {Grp. gram.:m.}
\end{itemize}
\begin{itemize}
\item {Proveniência:(De \textunderscore archi...\textunderscore  + \textunderscore mosteiro\textunderscore )}
\end{itemize}
Mosteiro principal de uma Ordem religiosa.
\section{Archinotário}
\begin{itemize}
\item {fónica:qui}
\end{itemize}
\begin{itemize}
\item {Grp. gram.:m.}
\end{itemize}
\begin{itemize}
\item {Utilização:Ant.}
\end{itemize}
\begin{itemize}
\item {Proveniência:(De \textunderscore archi...\textunderscore  + \textunderscore notário\textunderscore )}
\end{itemize}
Chefe dos notários.
\section{Archipélago}
\begin{itemize}
\item {fónica:qui}
\end{itemize}
\begin{itemize}
\item {Grp. gram.:m.}
\end{itemize}
\begin{itemize}
\item {Proveniência:(Do gr. \textunderscore arkhi\textunderscore  + \textunderscore pelagos\textunderscore )}
\end{itemize}
Grupo de ilhas, pouco distantes umas das outras: \textunderscore o archipélago dos Açores\textunderscore .
\section{Archipirata}
\begin{itemize}
\item {fónica:qui}
\end{itemize}
\begin{itemize}
\item {Grp. gram.:m.}
\end{itemize}
\begin{itemize}
\item {Utilização:Fig.}
\end{itemize}
\begin{itemize}
\item {Proveniência:(Lat. \textunderscore archipirata\textunderscore )}
\end{itemize}
Chefe de piratas.
Agiota, usurário.
\section{Archipotente}
\begin{itemize}
\item {fónica:qui}
\end{itemize}
\begin{itemize}
\item {Grp. gram.:adj.}
\end{itemize}
Poderosíssimo. Cf. Castilho, \textunderscore Fausto\textunderscore , 275.
\section{Archipresbýtero}
\begin{itemize}
\item {fónica:qui}
\end{itemize}
\begin{itemize}
\item {Grp. gram.:m. (e der.)}
\end{itemize}
O mesmo que \textunderscore arcipreste\textunderscore , etc.
\section{Archiprior}
\begin{itemize}
\item {fónica:qui}
\end{itemize}
\begin{itemize}
\item {Grp. gram.:m.}
\end{itemize}
\begin{itemize}
\item {Proveniência:(De \textunderscore archi...\textunderscore  + \textunderscore prior\textunderscore )}
\end{itemize}
Título do Grão-Mestre dos Templários.
\section{Archipriorado}
\begin{itemize}
\item {fónica:qui}
\end{itemize}
\begin{itemize}
\item {Grp. gram.:m.}
\end{itemize}
Dignidade de Archiprior.
\section{Archipropheta}
\begin{itemize}
\item {fónica:qui}
\end{itemize}
\begin{itemize}
\item {Grp. gram.:m.}
\end{itemize}
\begin{itemize}
\item {Proveniência:(De \textunderscore archi...\textunderscore  + \textunderscore propheta\textunderscore )}
\end{itemize}
O principal dos prophetas.
\section{Archipulha}
\begin{itemize}
\item {fónica:qui}
\end{itemize}
\begin{itemize}
\item {Grp. gram.:m.}
\end{itemize}
Grandíssimo pulha. Cf. Camillo, \textunderscore Noites de Insómn.\textunderscore , VIII, 93.
\section{Archisistro}
\begin{itemize}
\item {fónica:quissis}
\end{itemize}
\begin{itemize}
\item {Grp. gram.:m.}
\end{itemize}
Antigo instrumento musical.
\section{Archisophista}
\begin{itemize}
\item {fónica:quisso}
\end{itemize}
\begin{itemize}
\item {Grp. gram.:m.}
\end{itemize}
Sofista em alto grau.
\section{Architectar}
\begin{itemize}
\item {fónica:qui}
\end{itemize}
\begin{itemize}
\item {Grp. gram.:v. t.}
\end{itemize}
\begin{itemize}
\item {Proveniência:(De \textunderscore architecto\textunderscore )}
\end{itemize}
Edificar: \textunderscore architectar um palácio\textunderscore .
Planear; idear: \textunderscore architectar uma empresa\textunderscore .
\section{Architecto}
\begin{itemize}
\item {fónica:qui}
\end{itemize}
\begin{itemize}
\item {Grp. gram.:m.}
\end{itemize}
\begin{itemize}
\item {Proveniência:(Lat. \textunderscore architectus\textunderscore )}
\end{itemize}
Aquelle que dirige construcções de edifícios.
Aquelle que planeia; aquelle que fantasia.
\section{Architectónica}
\begin{itemize}
\item {fónica:qui}
\end{itemize}
\begin{itemize}
\item {Grp. gram.:f.}
\end{itemize}
O mesmo que \textunderscore architectura\textunderscore .
\section{Architectónico}
\begin{itemize}
\item {fónica:qui}
\end{itemize}
\begin{itemize}
\item {Grp. gram.:adj.}
\end{itemize}
\begin{itemize}
\item {Proveniência:(Lat. \textunderscore architectonicus\textunderscore )}
\end{itemize}
Relativo á architectura.
\section{Architectonographia}
\begin{itemize}
\item {fónica:qui}
\end{itemize}
\begin{itemize}
\item {Grp. gram.:f.}
\end{itemize}
Arte de descrever edifícios.
(Cp. \textunderscore architectonógrapho\textunderscore )
\section{Architectonógrapho}
\begin{itemize}
\item {fónica:qui}
\end{itemize}
\begin{itemize}
\item {Grp. gram.:m.}
\end{itemize}
\begin{itemize}
\item {Proveniência:(Do gr. \textunderscore arkhitektonos\textunderscore  + \textunderscore graphein\textunderscore )}
\end{itemize}
Aquelle que faz a descripção de edifícios.
\section{Architector}
\begin{itemize}
\item {fónica:qui}
\end{itemize}
\begin{itemize}
\item {Grp. gram.:m.}
\end{itemize}
\begin{itemize}
\item {Utilização:Ant.}
\end{itemize}
\begin{itemize}
\item {Proveniência:(Lat. \textunderscore architector\textunderscore )}
\end{itemize}
O mesmo que \textunderscore architecto\textunderscore .
\section{Architectura}
\begin{itemize}
\item {fónica:qui}
\end{itemize}
\begin{itemize}
\item {Grp. gram.:f.}
\end{itemize}
\begin{itemize}
\item {Proveniência:(Lat. \textunderscore architectura\textunderscore )}
\end{itemize}
Arte de construir edifícios.
Contextura.
Plano, projecto.
\section{Architectural}
\begin{itemize}
\item {fónica:qui}
\end{itemize}
\begin{itemize}
\item {Grp. gram.:adj.}
\end{itemize}
Relativo á \textunderscore architectura\textunderscore .
\section{Architecturista}
\begin{itemize}
\item {fónica:qui}
\end{itemize}
\begin{itemize}
\item {Grp. gram.:m.}
\end{itemize}
\begin{itemize}
\item {Proveniência:(De \textunderscore architectura\textunderscore )}
\end{itemize}
Pintor, que tem por especialidade a reproducção de edifícios em suas telas.
\section{Architolo}
\begin{itemize}
\item {fónica:qui}
\end{itemize}
\begin{itemize}
\item {Grp. gram.:m.  e  adj.}
\end{itemize}
\begin{itemize}
\item {Proveniência:(De \textunderscore archi...\textunderscore  + \textunderscore tolo\textunderscore )}
\end{itemize}
Tolo no mais alto grau.
\section{Architravado}
\begin{itemize}
\item {fónica:qui}
\end{itemize}
\begin{itemize}
\item {Grp. gram.:adj.}
\end{itemize}
Ornado de architrave.
\section{Architrave}
\begin{itemize}
\item {fónica:qui}
\end{itemize}
\begin{itemize}
\item {Grp. gram.:f.}
\end{itemize}
\begin{itemize}
\item {Utilização:Archit.}
\end{itemize}
\begin{itemize}
\item {Proveniência:(De \textunderscore archi...\textunderscore  + \textunderscore trave\textunderscore )}
\end{itemize}
Parte inferior de um entablamento, entre o friso e o capitel.
\section{Architriclino}
\begin{itemize}
\item {fónica:qui}
\end{itemize}
\begin{itemize}
\item {Grp. gram.:m.}
\end{itemize}
\begin{itemize}
\item {Proveniência:(Lat. \textunderscore architriclinus\textunderscore )}
\end{itemize}
Chefe de escanções ou dos que servem á mesa.
Mordomo.
\section{Architrovão}
\begin{itemize}
\item {fónica:qui}
\end{itemize}
\begin{itemize}
\item {Grp. gram.:m.}
\end{itemize}
\begin{itemize}
\item {Proveniência:(De \textunderscore archi...\textunderscore  + \textunderscore trovão\textunderscore )}
\end{itemize}
Antiga máquina de cobre, que arremessava projécteis com grande estrondo.
\section{Archivar}
\begin{itemize}
\item {fónica:qui}
\end{itemize}
\begin{itemize}
\item {Grp. gram.:v. t.}
\end{itemize}
Recolher em archivo.
Guardar; conservar.
\section{Archiviola}
\begin{itemize}
\item {fónica:qui}
\end{itemize}
\begin{itemize}
\item {Grp. gram.:f.}
\end{itemize}
\begin{itemize}
\item {Proveniência:(De \textunderscore archi...\textunderscore  + \textunderscore viola\textunderscore )}
\end{itemize}
Antigo instrumento músico, que se compunha do uma espécie de cravo, a que se adaptava o maquinismo de uma viola.
\section{Archivista}
\begin{itemize}
\item {fónica:qui}
\end{itemize}
\begin{itemize}
\item {Grp. gram.:m.}
\end{itemize}
Aquelle que tem archivo a seu cargo.
\section{Archivo}
\begin{itemize}
\item {fónica:qui}
\end{itemize}
\begin{itemize}
\item {Grp. gram.:m.}
\end{itemize}
\begin{itemize}
\item {Utilização:Fig.}
\end{itemize}
\begin{itemize}
\item {Proveniência:(Lat. \textunderscore archivum\textunderscore )}
\end{itemize}
Lugar, onde se guardam documentos escritos: \textunderscore Archivo da Tôrre do Tombo\textunderscore .
Cartório.
Deposito.
Pessôa de grande memória.
\section{Archivolta}
\begin{itemize}
\item {fónica:qui}
\end{itemize}
\begin{itemize}
\item {Grp. gram.:f.}
\end{itemize}
\begin{itemize}
\item {Utilização:Archit.}
\end{itemize}
Contôrno, que acompanha o arco.
(B. lat. \textunderscore archivoltum\textunderscore )
\section{Archivulgar}
\begin{itemize}
\item {fónica:qui}
\end{itemize}
\begin{itemize}
\item {Grp. gram.:adj.}
\end{itemize}
Extremamente vulgar. Cf. Castilho, \textunderscore Sabichonas\textunderscore , 70.
\section{Archontado}
\begin{itemize}
\item {fónica:con}
\end{itemize}
\begin{itemize}
\item {Grp. gram.:m.}
\end{itemize}
Título ou cargo de archonte.
\section{Archonte}
\begin{itemize}
\item {fónica:con}
\end{itemize}
\begin{itemize}
\item {Grp. gram.:m.}
\end{itemize}
\begin{itemize}
\item {Proveniência:(Lat. \textunderscore archon\textunderscore , \textunderscore archontis\textunderscore )}
\end{itemize}
Antigo magistrado grego, que, antes de Sólon, tinha o poder de legislar e, ao depois, foi simples executor das leis.
\section{Archoptose}
\begin{itemize}
\item {fónica:co}
\end{itemize}
\begin{itemize}
\item {Grp. gram.:f.}
\end{itemize}
\begin{itemize}
\item {Utilização:Med.}
\end{itemize}
Saída ou quéda do intestino recto.
\section{Archorrheia}
\begin{itemize}
\item {fónica:co}
\end{itemize}
\begin{itemize}
\item {Grp. gram.:f.}
\end{itemize}
\begin{itemize}
\item {Utilização:Med.}
\end{itemize}
Hemorragia no ânus.
\section{Archorrheico}
\begin{itemize}
\item {fónica:co}
\end{itemize}
\begin{itemize}
\item {Grp. gram.:adj.}
\end{itemize}
Relativo a archorrheia.
\section{Archotada}
\begin{itemize}
\item {Grp. gram.:f.}
\end{itemize}
Cortejo nocturno, illuminado com archotes.
\section{Archote}
\begin{itemize}
\item {Grp. gram.:m.}
\end{itemize}
\begin{itemize}
\item {Utilização:Gír.}
\end{itemize}
\begin{itemize}
\item {Proveniência:(De \textunderscore archa\textunderscore )}
\end{itemize}
Pedaço de corda de esparto, untada de breu, e que se accende para alumiar.
Quartilho de vinho.
\section{Archoteiro}
\begin{itemize}
\item {Grp. gram.:m.}
\end{itemize}
Fabricante ou vendedor de archotes.
\section{Arcífero}
\begin{itemize}
\item {Grp. gram.:adj.}
\end{itemize}
\begin{itemize}
\item {Proveniência:(Lat. \textunderscore arcifer\textunderscore )}
\end{itemize}
Armado de arco.
\section{Arciforme}
\begin{itemize}
\item {Grp. gram.:adj.}
\end{itemize}
\begin{itemize}
\item {Proveniência:(Do lat. \textunderscore arcus\textunderscore  + \textunderscore forma\textunderscore )}
\end{itemize}
Que tem fórma de arco.
\section{Arcio}
\begin{itemize}
\item {Grp. gram.:m.}
\end{itemize}
Gênero de plantas vivazes, sem haste e de fôlhas redondas.
\section{Arcipotente}
\begin{itemize}
\item {Grp. gram.:adj.}
\end{itemize}
\begin{itemize}
\item {Proveniência:(Lat. \textunderscore arcipotens\textunderscore )}
\end{itemize}
Hábil em manejar o arco. Cf. Filinto. X, 12.
\section{Arciprestádego}
\begin{itemize}
\item {Grp. gram.:m.}
\end{itemize}
\begin{itemize}
\item {Utilização:Ant.}
\end{itemize}
O mesmo que \textunderscore arciprestado\textunderscore .
\section{Arciprestado}
\begin{itemize}
\item {Grp. gram.:m.}
\end{itemize}
Dignidade de Arcipreste.
Território, em que a jurisdicção dêste se exerce.
\section{Arciprestal}
\begin{itemize}
\item {Grp. gram.:adj.}
\end{itemize}
Relativo a \textunderscore Arcipreste\textunderscore .
\section{Arcipreste}
\begin{itemize}
\item {Grp. gram.:m.}
\end{itemize}
\begin{itemize}
\item {Proveniência:(Do lat. \textunderscore archipresbyter\textunderscore )}
\end{itemize}
Párocho, com jurisdicção superior á de outros.
\section{Arcitenente}
\begin{itemize}
\item {Grp. gram.:adj.}
\end{itemize}
\begin{itemize}
\item {Proveniência:(Do lat. \textunderscore arcus\textunderscore  + \textunderscore tenens\textunderscore )}
\end{itemize}
Que se figura com arco na mão.
\section{Arco}
\begin{itemize}
\item {Grp. gram.:m.}
\end{itemize}
\begin{itemize}
\item {Proveniência:(Lat. \textunderscore arcus\textunderscore )}
\end{itemize}
Parte de qualquer curva.
Haste flexível, recurvada por uma corda que se lhe prende ás extremidades e que, retesada, impelle a seta.
Pequena vara, guarnecida de crina, com que se ferem as cordas de certos instrumentos músicos: \textunderscore o arco da rabeca\textunderscore .
Curva de abóbada.
Qualquer peça de fórma anular ou circular: \textunderscore as crianças jogam o arco\textunderscore .
Cada uma das duas partes do sinal orthográphico, chamado parênthese.
\section{Arcobalista}
\begin{itemize}
\item {Grp. gram.:f.}
\end{itemize}
\begin{itemize}
\item {Utilização:Ant.}
\end{itemize}
\begin{itemize}
\item {Proveniência:(Lat. \textunderscore arcuballista\textunderscore )}
\end{itemize}
Pequena catapulta, manejada por dois homens.
\section{Arcoballista}
\begin{itemize}
\item {Grp. gram.:f.}
\end{itemize}
\begin{itemize}
\item {Utilização:Ant.}
\end{itemize}
\begin{itemize}
\item {Proveniência:(Lat. \textunderscore arcuballista\textunderscore )}
\end{itemize}
Pequena catapulta, manejada por dois homens.
\section{Arcobotante}
\begin{itemize}
\item {Grp. gram.:m.}
\end{itemize}
\begin{itemize}
\item {Proveniência:(Do fr. \textunderscore arc-boutant\textunderscore )}
\end{itemize}
Construcção exterior, terminada em arco e que ampara parede ou abóbada.
Caminho em floresta, com abóbada arqueada, formada pelo cruzamento da ramaria.
\section{Arco-celeste}
\begin{itemize}
\item {Grp. gram.:m.}
\end{itemize}
O mesmo que \textunderscore arco-íris\textunderscore .
\section{Arco-da-chuva}
\begin{itemize}
\item {Grp. gram.:m.}
\end{itemize}
\begin{itemize}
\item {Utilização:Pop.}
\end{itemize}
O mesmo que \textunderscore arco-íris\textunderscore .
\section{Arco-da-velha}
\begin{itemize}
\item {Grp. gram.:m.}
\end{itemize}
\begin{itemize}
\item {Utilização:Pop.}
\end{itemize}
O mesmo que \textunderscore arco-íris\textunderscore .
\section{Arco-de-Deus}
\begin{itemize}
\item {Grp. gram.:m.}
\end{itemize}
\begin{itemize}
\item {Utilização:Pop.}
\end{itemize}
O mesmo que \textunderscore arco-íris\textunderscore .
\section{Arco-íris}
\begin{itemize}
\item {Grp. gram.:m.}
\end{itemize}
Meteóro luminoso, em fórma de arco, que apresenta as sete côres do espectro solar.
\section{Arcontado}
\begin{itemize}
\item {Grp. gram.:m.}
\end{itemize}
Título ou cargo de arconte.
\section{Arconte}
\begin{itemize}
\item {Grp. gram.:m.}
\end{itemize}
\begin{itemize}
\item {Proveniência:(Lat. \textunderscore archon\textunderscore , \textunderscore archontis\textunderscore )}
\end{itemize}
Antigo magistrado grego, que, antes de Sólon, tinha o poder de legislar e, ao depois, foi simples executor das leis.
\section{Arcoptose}
\begin{itemize}
\item {Grp. gram.:f.}
\end{itemize}
\begin{itemize}
\item {Utilização:Med.}
\end{itemize}
Saída ou quéda do intestino recto.
\section{Arcorreia}
\begin{itemize}
\item {Grp. gram.:f.}
\end{itemize}
\begin{itemize}
\item {Utilização:Med.}
\end{itemize}
Hemorragia no ânus.
\section{Arcorreico}
\begin{itemize}
\item {Grp. gram.:adj.}
\end{itemize}
Relativo a arcorreia.
\section{Arcoso}
\begin{itemize}
\item {Grp. gram.:m.}
\end{itemize}
\begin{itemize}
\item {Utilização:Gír.}
\end{itemize}
\begin{itemize}
\item {Proveniência:(De \textunderscore arco\textunderscore )}
\end{itemize}
Anel.
\section{Arco-verde}
\begin{itemize}
\item {Grp. gram.:m.}
\end{itemize}
Árvore do Brasil, o mesmo que \textunderscore ipé\textunderscore .
\section{Arcoze}
\begin{itemize}
\item {Grp. gram.:f.}
\end{itemize}
Espécie de mineral quási ferruginoso.
\section{Arctação}
\begin{itemize}
\item {Grp. gram.:f.}
\end{itemize}
\begin{itemize}
\item {Proveniência:(Lat. \textunderscore arctatio\textunderscore )}
\end{itemize}
Apêrto de um canal natural, no organismo humano.
\section{Arctar}
\begin{itemize}
\item {Grp. gram.:v. t.}
\end{itemize}
\begin{itemize}
\item {Proveniência:(Lat. \textunderscore arctare\textunderscore )}
\end{itemize}
Apertar; contrahir.
\section{Árctico}
\begin{itemize}
\item {Grp. gram.:adj.}
\end{itemize}
\begin{itemize}
\item {Proveniência:(Lat. \textunderscore arcticus\textunderscore )}
\end{itemize}
Boreal; setentrional: \textunderscore o pólo árctico\textunderscore .
\section{Arctícola}
\begin{itemize}
\item {Grp. gram.:adj.}
\end{itemize}
\begin{itemize}
\item {Utilização:Entom.}
\end{itemize}
\begin{itemize}
\item {Proveniência:(Do lat. \textunderscore arcticus\textunderscore  + \textunderscore colere\textunderscore )}
\end{itemize}
Diz-se dos insectos que vivem perto do pólo árctico.
\section{Árctoa}
\begin{itemize}
\item {Grp. gram.:f.}
\end{itemize}
Gênero de musgo.
\section{Arctocéfalo}
\begin{itemize}
\item {Grp. gram.:m.}
\end{itemize}
O mesmo que \textunderscore foca\textunderscore ^3.
\section{Arctocéphalo}
\begin{itemize}
\item {Grp. gram.:m.}
\end{itemize}
O mesmo que \textunderscore phoca\textunderscore .
\section{Arctos}
\begin{itemize}
\item {Grp. gram.:m.}
\end{itemize}
\begin{itemize}
\item {Proveniência:(Lat. \textunderscore arctos\textunderscore )}
\end{itemize}
O mesmo que a constellação \textunderscore Ursa-Maior\textunderscore , ou a \textunderscore Maior\textunderscore  e a \textunderscore Menor\textunderscore  juntamente.
\section{Arcturo}
\begin{itemize}
\item {Grp. gram.:m.}
\end{itemize}
\begin{itemize}
\item {Proveniência:(Lat. \textunderscore arcturus\textunderscore )}
\end{itemize}
Estrêlla de primeira grandeza da constellação do \textunderscore Boieiro\textunderscore , na cauda da \textunderscore Ursa-Maior\textunderscore .
\section{Arcual}
\begin{itemize}
\item {Grp. gram.:adj.}
\end{itemize}
Que tem fórma de arco.
\section{Arculho}
\begin{itemize}
\item {Grp. gram.:m.}
\end{itemize}
\begin{itemize}
\item {Utilização:Prov.}
\end{itemize}
\begin{itemize}
\item {Utilização:trasm.}
\end{itemize}
Rede com arcos, para pescar peixe.
\section{Arcumferência}
\begin{itemize}
\item {Utilização:P. us.}
\end{itemize}
O mesmo que \textunderscore circumferência\textunderscore .
\section{Arcunferência}
\begin{itemize}
\item {Utilização:P. us.}
\end{itemize}
O mesmo que \textunderscore circumferência\textunderscore .
\section{Arda}
\begin{itemize}
\item {Grp. gram.:f.}
\end{itemize}
Esquilo, da ordem dos roëdores.
\section{Árdea}
\begin{itemize}
\item {Grp. gram.:f.}
\end{itemize}
\begin{itemize}
\item {Proveniência:(Lat. \textunderscore ardea\textunderscore )}
\end{itemize}
Designação genérica e scientífica das garças.
\section{Ardego}
\begin{itemize}
\item {fónica:dê}
\end{itemize}
\begin{itemize}
\item {Grp. gram.:adj.}
\end{itemize}
\begin{itemize}
\item {Proveniência:(De \textunderscore arder\textunderscore )}
\end{itemize}
Fogoso: \textunderscore cavallo ardego\textunderscore .
Irritável.
Árduo: \textunderscore trabalho ardego\textunderscore .
\section{Ardeleão}
\begin{itemize}
\item {Grp. gram.:m.}
\end{itemize}
\begin{itemize}
\item {Proveniência:(Lat. \textunderscore ardelio\textunderscore )}
\end{itemize}
Homem intrometido, metediço. Cf. A. Costa, \textunderscore Três Mundos\textunderscore , 75.
\section{Ardência}
\begin{itemize}
\item {Grp. gram.:f.}
\end{itemize}
Qualidade do que é ardente ou que causa ardor.
Ardor.
Vivacidade.
Sabor acre de algumas substâncias fermentadas.
\section{Ardente}
\begin{itemize}
\item {Grp. gram.:adj.}
\end{itemize}
\begin{itemize}
\item {Proveniência:(Lat. \textunderscore ardens\textunderscore )}
\end{itemize}
Que arde.
Que requeima: \textunderscore sol ardente\textunderscore .
Acre.
Que produz muito calor.
Enérgico; intenso; vivo: \textunderscore fantasia ardente\textunderscore .
\section{Ardentemente}
\begin{itemize}
\item {Grp. gram.:adv.}
\end{itemize}
De modo \textunderscore ardente\textunderscore .
\section{Ardentia}
\begin{itemize}
\item {Grp. gram.:f.}
\end{itemize}
\begin{itemize}
\item {Proveniência:(De \textunderscore ardente\textunderscore )}
\end{itemize}
Phosphorecência do mar.
\section{Ardentoso}
\begin{itemize}
\item {Grp. gram.:adj.}
\end{itemize}
\begin{itemize}
\item {Proveniência:(De \textunderscore ardente\textunderscore )}
\end{itemize}
Híspido, que causa ardor e inflammação.
\section{Arder}
\begin{itemize}
\item {Grp. gram.:v. i.}
\end{itemize}
\begin{itemize}
\item {Grp. gram.:V. t.}
\end{itemize}
\begin{itemize}
\item {Utilização:Prov.}
\end{itemize}
\begin{itemize}
\item {Utilização:extrem.}
\end{itemize}
\begin{itemize}
\item {Proveniência:(Lat. \textunderscore ardere\textunderscore )}
\end{itemize}
Estar em chamma: \textunderscore a lenha arde\textunderscore .
Inflammar-se.
Exaltar-se.
Têr grande calor.
Sentir desejo vehemente.
Brilhar: \textunderscore o Sol arde\textunderscore .
Têr sabor acre.
Sêr devastado.
Grassar.
Desbaratar-se: \textunderscore toda a riqueza ardeu\textunderscore .
Queimar, abrasar:«\textunderscore torva a feição lhe arde as entranhas.\textunderscore »Filinto, XI, 236
Gafar-se (a azeitona madura).
\section{Arderela}
\begin{itemize}
\item {Grp. gram.:f.}
\end{itemize}
Espécie de melharuco.
\section{Ardidamente}
\begin{itemize}
\item {Grp. gram.:adv.}
\end{itemize}
\begin{itemize}
\item {Proveniência:(De \textunderscore arder\textunderscore )}
\end{itemize}
Corajosamente.
Com ardor.
\section{Ardidez}
\begin{itemize}
\item {Grp. gram.:f.}
\end{itemize}
O mesmo que \textunderscore ardideza\textunderscore . Cf. Filinto, XVIII, 54.
\section{Ardideza}
\begin{itemize}
\item {Grp. gram.:f.}
\end{itemize}
\begin{itemize}
\item {Utilização:Ant.}
\end{itemize}
O mesmo que \textunderscore ardimento\textunderscore ^2.
\section{Ardido}
\begin{itemize}
\item {Proveniência:(De \textunderscore arder\textunderscore )}
\end{itemize}
Queimado.
Fermentado.
\section{Ardido}
\begin{itemize}
\item {Grp. gram.:adj.}
\end{itemize}
Corajoso; valente; audaz:«\textunderscore qual cão de caçador, sagaz e ardido\textunderscore ». Lusíadas, IX, 74.
\section{Ardífero}
\begin{itemize}
\item {Grp. gram.:adj.}
\end{itemize}
\begin{itemize}
\item {Utilização:P. us.}
\end{itemize}
Que produz ardor.
\section{Ardil}
\begin{itemize}
\item {Grp. gram.:m.}
\end{itemize}
\begin{itemize}
\item {Proveniência:(Do b. lat. \textunderscore artitus\textunderscore ?)}
\end{itemize}
Manha; astúcia.
Subtileza.
\section{Ardileza}
\begin{itemize}
\item {Grp. gram.:f.}
\end{itemize}
(V.ardil)
\section{Ardilosamente}
\begin{itemize}
\item {Grp. gram.:adj.}
\end{itemize}
De modo \textunderscore ardiloso\textunderscore .
Com ardil; astuciosamente.
\section{Ardiloso}
\begin{itemize}
\item {Grp. gram.:adj.}
\end{itemize}
Que usa de ardis.
Astucioso.
Sagaz.
Velhaco.
\section{Ardimento}
\begin{itemize}
\item {Grp. gram.:m.}
\end{itemize}
O mesmo que \textunderscore ardência\textunderscore .
\section{Ardimento}
\begin{itemize}
\item {Grp. gram.:m.}
\end{itemize}
Coragem.
(Cp. \textunderscore ardido\textunderscore ^2)
\section{Ardina}
\begin{itemize}
\item {Grp. gram.:f.}
\end{itemize}
\begin{itemize}
\item {Utilização:Gír.}
\end{itemize}
\begin{itemize}
\item {Grp. gram.:M.}
\end{itemize}
\begin{itemize}
\item {Utilização:Gír. de Lisbôa.}
\end{itemize}
O mesmo que \textunderscore ardose\textunderscore .
Rapaz, especialmente o que pelas ruas vende bilhetes postaes, papel, etc.
\section{Ardingo}
\begin{itemize}
\item {Grp. gram.:m.}
\end{itemize}
Antigo magistrado na Lusitânia, talvez o mesmo que \textunderscore gardingo\textunderscore .
\section{Ardísia}
\begin{itemize}
\item {Grp. gram.:f.}
\end{itemize}
Gênero de plantas ardisiáceas, segundo Jussieu.
Secção de primuláceas, segundo Payer.
\section{Ardisiáceas}
\begin{itemize}
\item {Grp. gram.:f. pl.}
\end{itemize}
\begin{itemize}
\item {Proveniência:(De \textunderscore ardísia\textunderscore )}
\end{itemize}
Familia de plantas dicotyledóneas, encorporada hoje nas mirsineáceas.
\section{Ardor}
\begin{itemize}
\item {Grp. gram.:m.}
\end{itemize}
\begin{itemize}
\item {Proveniência:(Lat. \textunderscore ardor\textunderscore )}
\end{itemize}
Calor forte.
Paixão.
Energia.
Vivacidade.
Sabor picante.
\section{Ardosa}
\begin{itemize}
\item {Grp. gram.:f.}
\end{itemize}
O mesmo que \textunderscore ardose\textunderscore .
\section{Ardose}
\begin{itemize}
\item {Grp. gram.:f.}
\end{itemize}
\begin{itemize}
\item {Utilização:Gír.}
\end{itemize}
\begin{itemize}
\item {Proveniência:(De \textunderscore ardor\textunderscore )}
\end{itemize}
Aguardente.
\section{Ardósia}
\begin{itemize}
\item {Grp. gram.:f.}
\end{itemize}
\begin{itemize}
\item {Proveniência:(Fr. \textunderscore ardoise\textunderscore )}
\end{itemize}
Pedra cinzenta escura, que se divide em lâminas e que se applica em cobertura de casas, em quadros sôbre que se escreve nas escolas, etc.
\section{Ardosieira}
\begin{itemize}
\item {Grp. gram.:f.}
\end{itemize}
Rocha de ardósia.
\section{Arduamente}
\begin{itemize}
\item {Grp. gram.:adv.}
\end{itemize}
De modo \textunderscore árduo\textunderscore .
Difficultosamente.
\section{Ardume}
\begin{itemize}
\item {Grp. gram.:m.}
\end{itemize}
O mesmo que \textunderscore ardor\textunderscore . Cf. \textunderscore Bibl. da Gente do Campo\textunderscore , 517.
\section{Árduo}
\begin{itemize}
\item {Grp. gram.:adj.}
\end{itemize}
\begin{itemize}
\item {Proveniência:(Lat. \textunderscore arduus\textunderscore )}
\end{itemize}
Escarpado: \textunderscore serrania árdua\textunderscore .
Espinhoso; diffícil; trabalhoso; custoso: \textunderscore empresa árdua\textunderscore .
\section{Arduosidade}
\begin{itemize}
\item {Grp. gram.:f.}
\end{itemize}
\begin{itemize}
\item {Utilização:Des.}
\end{itemize}
\begin{itemize}
\item {Proveniência:(De um hypoth. \textunderscore arduoso\textunderscore , de \textunderscore árduo\textunderscore )}
\end{itemize}
Grande difficuldade; grande trabalho.
\section{Ardura}
\begin{itemize}
\item {Grp. gram.:f.}
\end{itemize}
(V.ardor)
\section{Are}
\begin{itemize}
\item {Grp. gram.:m.}
\end{itemize}
\begin{itemize}
\item {Proveniência:(Lat. \textunderscore area\textunderscore )}
\end{itemize}
Medida de superfície, de cem metros quadrados.
\section{Área}
\begin{itemize}
\item {Grp. gram.:f.}
\end{itemize}
\begin{itemize}
\item {Proveniência:(Lat. \textunderscore area\textunderscore )}
\end{itemize}
Superfície plana, delimitada: \textunderscore a área do velódromo\textunderscore .
Espaço.
Campo em que se exerce determinada actividade: \textunderscore a área das sciências naturaes\textunderscore .
Espaço, que um raio vector de um astro percorre em certo tempo.
\section{Aréa}
\begin{itemize}
\item {Grp. gram.:f.}
\end{itemize}
\begin{itemize}
\item {Utilização:Fam.}
\end{itemize}
\begin{itemize}
\item {Proveniência:(Lat. \textunderscore arena\textunderscore )}
\end{itemize}
Substância mineral, granulosa ou pulverulenta, que se acumula nas praias, no leito dos rios, etc.
Qualquer pó.
Grânulos calcáreos da urina.
Patétice; doidice; toleima.
\section{Areação}
\begin{itemize}
\item {Grp. gram.:f.}
\end{itemize}
Acto de \textunderscore arear\textunderscore .
\section{Areado}
\begin{itemize}
\item {Grp. gram.:adj.}
\end{itemize}
\begin{itemize}
\item {Utilização:Bot.}
\end{itemize}
Coberto de areia.
Limpo, por se têr esfregado com areia ou com outro pó: \textunderscore garfos areados\textunderscore .
Refinado, (falando-se do açúcar).
Diz-se das fôlhas, que têm pontuações, produzidas pela areia espalhada no solo.
\section{Areado}
\begin{itemize}
\item {Grp. gram.:adj.}
\end{itemize}
\begin{itemize}
\item {Utilização:Pop.}
\end{itemize}
O mesmo que \textunderscore ousado\textunderscore . Cf. Camillo, \textunderscore Brasileira\textunderscore , 288.
\section{Areador}
\begin{itemize}
\item {Grp. gram.:m.}
\end{itemize}
Operário, que areia açúcar.
\section{Areal}
\begin{itemize}
\item {Grp. gram.:m.}
\end{itemize}
\begin{itemize}
\item {Utilização:Prov.}
\end{itemize}
\begin{itemize}
\item {Utilização:trasm.}
\end{itemize}
Lugar, onde há muita areia.
Praia.
O mesmo que \textunderscore poisio\textunderscore .
\section{Areamento}
\begin{itemize}
\item {Grp. gram.:m.}
\end{itemize}
Acto de \textunderscore arear\textunderscore .
\section{Arear}
\begin{itemize}
\item {Grp. gram.:v. t.}
\end{itemize}
\begin{itemize}
\item {Utilização:Ant.}
\end{itemize}
\begin{itemize}
\item {Utilização:Pop.}
\end{itemize}
Cobrir com areia.
Esfregar com areia ou outro pó, limpando: \textunderscore arear metaes\textunderscore .
Refinar (açúcar).
Estontear, tornar pateta.--Nesta última accepção, parece corr. de \textunderscore ourar\textunderscore .
\section{Areática}
\begin{itemize}
\item {Grp. gram.:f.}
\end{itemize}
\begin{itemize}
\item {Utilização:Ant.}
\end{itemize}
\begin{itemize}
\item {Proveniência:(Do lat. \textunderscore area\textunderscore , eira)}
\end{itemize}
O mesmo que \textunderscore eirádiga\textunderscore .
\section{Areca}
\begin{itemize}
\item {Grp. gram.:f.}
\end{itemize}
Árvore, da fam. das palmeiras.
Fruto da arequeira.
\section{Arecal}
\begin{itemize}
\item {Grp. gram.:m.}
\end{itemize}
Bosque de arecas.
\section{Arecíneas}
\begin{itemize}
\item {Grp. gram.:f. pl.}
\end{itemize}
\begin{itemize}
\item {Proveniência:(De \textunderscore arecineo\textunderscore )}
\end{itemize}
Tríbo de plantas, da fam. das palmeiras.
\section{Arecíneo}
\begin{itemize}
\item {Grp. gram.:adj.}
\end{itemize}
Relativo ou semelhante á areca.
\section{Arecuno}
\begin{itemize}
\item {Grp. gram.:m.}
\end{itemize}
Dialecto da Guiana Inglesa.
\section{Areeiro}
\begin{itemize}
\item {Grp. gram.:m.}
\end{itemize}
\begin{itemize}
\item {Proveniência:(Do lat. \textunderscore arenarius\textunderscore )}
\end{itemize}
Areal. Lugar, donde se extrai areia.
Pequeno vaso, com areia que se deita sôbre a escrita para a secar.
\section{Areento}
\begin{itemize}
\item {Grp. gram.:adj.}
\end{itemize}
Que tem muita areia.
\section{Arefacção}
\begin{itemize}
\item {Grp. gram.:f.}
\end{itemize}
\begin{itemize}
\item {Proveniência:(Do lat. \textunderscore arefacere\textunderscore )}
\end{itemize}
Dissecação das substâncias que se hão de reduzir a pó.
\section{Areia}
\begin{itemize}
\item {Grp. gram.:f.}
\end{itemize}
\begin{itemize}
\item {Utilização:Fam.}
\end{itemize}
\begin{itemize}
\item {Proveniência:(Lat. \textunderscore arena\textunderscore )}
\end{itemize}
Substância mineral, granulosa ou pulverulenta, que se acumula nas praias, no leito dos rios, etc.
Qualquer pó.
Grânulos calcáreos da urina.
Patétice; doidice; toleima.
\section{Areião}
\begin{itemize}
\item {Grp. gram.:m.}
\end{itemize}
\begin{itemize}
\item {Utilização:Bras}
\end{itemize}
\begin{itemize}
\item {Proveniência:(De \textunderscore areia\textunderscore )}
\end{itemize}
Grande areal.
\section{Areínho}
\begin{itemize}
\item {Grp. gram.:m.}
\end{itemize}
\begin{itemize}
\item {Utilização:Prov.}
\end{itemize}
\begin{itemize}
\item {Utilização:minh.}
\end{itemize}
Pequeno areal, á beira de um rio.
O mesmo que \textunderscore ínsua\textunderscore .
Banco de areia, que se cobre e descobre com as marés.
\section{Areio}
\begin{itemize}
\item {Grp. gram.:m.}
\end{itemize}
\begin{itemize}
\item {Utilização:Prov.}
\end{itemize}
\begin{itemize}
\item {Utilização:dur.}
\end{itemize}
\begin{itemize}
\item {Proveniência:(De \textunderscore areia\textunderscore )}
\end{itemize}
Lugar, onde o leito do Doiro tem muita areia.
\section{Areísca}
\begin{itemize}
\item {Grp. gram.:f.}
\end{itemize}
\begin{itemize}
\item {Proveniência:(De \textunderscore areia\textunderscore )}
\end{itemize}
Espécie de argamassa, feita de areia e solão.
O mesmo ou melhor que \textunderscore grés\textunderscore .
\section{Areísco}
\begin{itemize}
\item {Grp. gram.:adj.}
\end{itemize}
\begin{itemize}
\item {Proveniência:(De \textunderscore areia\textunderscore )}
\end{itemize}
O mesmo que \textunderscore arisco\textunderscore ^1.
\section{Arejamento}
\begin{itemize}
\item {Grp. gram.:m.}
\end{itemize}
Acto ou effeito de \textunderscore arejar\textunderscore .
Arejo.
\section{Arejar}
\begin{itemize}
\item {Grp. gram.:v. t.}
\end{itemize}
\begin{itemize}
\item {Grp. gram.:V. i.}
\end{itemize}
\begin{itemize}
\item {Utilização:T. da Bairrada}
\end{itemize}
Expor ao ar; ventilar: \textunderscore arejar um quarto\textunderscore .
Tomar ar novo: \textunderscore foi arejar para o campo\textunderscore .
Avellar-se: \textunderscore esta fruta arejou\textunderscore .
Branquear uma parte do cabello, por motivo de doença.
\section{Arejo}
\begin{itemize}
\item {Grp. gram.:m.}
\end{itemize}
Acto de \textunderscore arejar\textunderscore .
Aragem.
Ventilação.
Doença dos vegetaes, que faz secar o fruto da oliveira.
Mau olhado, quebranto. Cf. Camillo, \textunderscore Ôlho de Vidro\textunderscore , 76.
\section{Arelhana}
\begin{itemize}
\item {Grp. gram.:f.}
\end{itemize}
Cinturão asiático, em que se leva dinheiro ou se penduram adagas.
\section{Arena}
\begin{itemize}
\item {Grp. gram.:f.}
\end{itemize}
\begin{itemize}
\item {Proveniência:(Lat. \textunderscore arena\textunderscore )}
\end{itemize}
Parte areada do amphitheatro, onde combatiam gladiadores, feras, etc.
Terreno circular, fechado, onde se correm toiros e se dão outros espectáculos.
Lugar de contenda.
Discussão.
\section{Arenação}
\begin{itemize}
\item {Grp. gram.:f.}
\end{itemize}
\begin{itemize}
\item {Utilização:Med.}
\end{itemize}
O mesmo que \textunderscore areação\textunderscore .
Acto de cobrir com areia quente e sêca um membro, cuja artéria principal foi laqueada, ou a totalidade do corpo.
\section{Arenáceo}
\begin{itemize}
\item {Grp. gram.:adj.}
\end{itemize}
\begin{itemize}
\item {Proveniência:(Lat. \textunderscore arenaceus\textunderscore )}
\end{itemize}
Relativo á areia.
\section{Arenado}
\begin{itemize}
\item {Grp. gram.:adj.}
\end{itemize}
Coberto de areia. Cf. Arn. Gama, \textunderscore Motim\textunderscore , 185.
\section{Arenária}
\begin{itemize}
\item {Grp. gram.:f.}
\end{itemize}
\begin{itemize}
\item {Proveniência:(De \textunderscore arenário\textunderscore )}
\end{itemize}
Gênero de plantas caryophylláceas.
\section{Arenário}
\begin{itemize}
\item {Grp. gram.:adj.}
\end{itemize}
\begin{itemize}
\item {Proveniência:(Lat. \textunderscore arenarius\textunderscore )}
\end{itemize}
Que cresce em terrenos arenosos.
\section{Arenato}
\begin{itemize}
\item {Grp. gram.:adj.}
\end{itemize}
\begin{itemize}
\item {Proveniência:(Lat. \textunderscore arenatus\textunderscore )}
\end{itemize}
Em cuja composição entra areia.
\section{Arenga}
\begin{itemize}
\item {Grp. gram.:f.}
\end{itemize}
\begin{itemize}
\item {Grp. gram.:M.}
\end{itemize}
\begin{itemize}
\item {Utilização:Prov.}
\end{itemize}
\begin{itemize}
\item {Utilização:trasm.}
\end{itemize}
Allocução.
Aranzel; discurso fastidioso.
Altercação.
Trabalhador reles, ou que finge trabalhar e não faz quási nada.
(Cp. cast. \textunderscore arenga\textunderscore )
\section{Arenga}
\begin{itemize}
\item {Grp. gram.:f.}
\end{itemize}
\begin{itemize}
\item {Utilização:Prov.}
\end{itemize}
\begin{itemize}
\item {Utilização:dur.}
\end{itemize}
Fluxo menstrual.
\section{Arengador}
\begin{itemize}
\item {Grp. gram.:m.}
\end{itemize}
Aquelle que arenga.
\section{Arengar}
\begin{itemize}
\item {Grp. gram.:v. t.}
\end{itemize}
\begin{itemize}
\item {Grp. gram.:V. i.}
\end{itemize}
\begin{itemize}
\item {Utilização:Prov.}
\end{itemize}
\begin{itemize}
\item {Utilização:trasm.}
\end{itemize}
\begin{itemize}
\item {Proveniência:(De \textunderscore arenga\textunderscore ^1)}
\end{itemize}
Pronunciar diffusamente.
Fazer arenga.
Illudir, fingindo que trabalha e não fazendo quási nada.
\section{Arengueiro}
\begin{itemize}
\item {Grp. gram.:m.}
\end{itemize}
O mesmo que \textunderscore arengador\textunderscore .
\section{Arenicalcinzamento}
\begin{itemize}
\item {Grp. gram.:m.}
\end{itemize}
\begin{itemize}
\item {Utilização:Neol.}
\end{itemize}
Acto de cobrir os cadáveres com camadas de areia conchilífera, ossos, cinza e carvão, como se praticava no valle do Tejo, em tempos prehistóricos.
\section{Arenícola}
\begin{itemize}
\item {Grp. gram.:m.  e  adj.}
\end{itemize}
\begin{itemize}
\item {Grp. gram.:M.}
\end{itemize}
\begin{itemize}
\item {Proveniência:(Do lat. \textunderscore arena\textunderscore  + \textunderscore colere\textunderscore )}
\end{itemize}
O que vive em terreno arenoso.
O mesmo que \textunderscore biscalongo\textunderscore .
\section{Arenífero}
\begin{itemize}
\item {Grp. gram.:adj.}
\end{itemize}
\begin{itemize}
\item {Proveniência:(Do lat. \textunderscore arena\textunderscore  + \textunderscore ferre\textunderscore )}
\end{itemize}
Que contém areia.
\section{Areniforme}
\begin{itemize}
\item {Grp. gram.:adj.}
\end{itemize}
\begin{itemize}
\item {Proveniência:(Do lat. \textunderscore arena\textunderscore  + \textunderscore forma\textunderscore )}
\end{itemize}
Semelhante á areia.
\section{Arenito}
\begin{itemize}
\item {Grp. gram.:m.}
\end{itemize}
Saibro, cujos grãos foram ligados entre si por um cimento qualquer.
O mesmo ou melhor que \textunderscore grés\textunderscore . Cf. G. Guimarães, \textunderscore Geol\textunderscore .
\section{Arenoso}
\begin{itemize}
\item {Grp. gram.:adj.}
\end{itemize}
\begin{itemize}
\item {Proveniência:(Lat. \textunderscore arenosus\textunderscore )}
\end{itemize}
Coberto de areia; areento.
Misturado com areia.
Que tem aspecto de areia.
\section{Arenque}
\begin{itemize}
\item {Grp. gram.:m.}
\end{itemize}
\begin{itemize}
\item {Grp. gram.:Pl.}
\end{itemize}
\begin{itemize}
\item {Utilização:Gír.}
\end{itemize}
O mesmo que \textunderscore harenque\textunderscore .
Os dedos da mão.
\section{Arensar}
\begin{itemize}
\item {Grp. gram.:v. i.}
\end{itemize}
Diz-se do cysne, quando solta a voz. Cf. Castilho, \textunderscore Fastos\textunderscore , III, 324.
\section{Aréola}
\begin{itemize}
\item {Grp. gram.:f.}
\end{itemize}
\begin{itemize}
\item {Proveniência:(Lat. \textunderscore areola\textunderscore )}
\end{itemize}
Canteiro de jardim.
Círculo pigmentado, em volta da glândula mamal.
Círculo, que envolve borbulhas.
Círculo, que rodeia a lua.
\section{Areóla}
\begin{itemize}
\item {Grp. gram.:f.}
\end{itemize}
\begin{itemize}
\item {Utilização:T. da Bairrada}
\end{itemize}
\begin{itemize}
\item {Proveniência:(De \textunderscore areia\textunderscore )}
\end{itemize}
Terra areenta; arneiro.
\section{Areolação}
\begin{itemize}
\item {fónica:aré-o}
\end{itemize}
\begin{itemize}
\item {Grp. gram.:f.}
\end{itemize}
\begin{itemize}
\item {Proveniência:(De \textunderscore areolar\textunderscore )}
\end{itemize}
Fórma, que apresentam as fibras de qualquer tecido cellular.
\section{Areolado}
\begin{itemize}
\item {fónica:aré-o}
\end{itemize}
\begin{itemize}
\item {Grp. gram.:adj.}
\end{itemize}
O mesmo que \textunderscore areolar\textunderscore .
\section{Areolar}
\begin{itemize}
\item {fónica:aré-o}
\end{itemize}
\begin{itemize}
\item {Grp. gram.:adj.}
\end{itemize}
Que tem aréolas.
\section{Areometria}
\begin{itemize}
\item {Grp. gram.:f.}
\end{itemize}
\begin{itemize}
\item {Proveniência:(De \textunderscore areómetro\textunderscore )}
\end{itemize}
Cálculo da densidade dos líquidos.
\section{Areométrico}
\begin{itemize}
\item {Grp. gram.:adj.}
\end{itemize}
Relativo ao \textunderscore areómetro\textunderscore .
\section{Areómetro}
\begin{itemize}
\item {Grp. gram.:m.}
\end{itemize}
\begin{itemize}
\item {Proveniência:(Do gr. \textunderscore araios\textunderscore  + \textunderscore metron\textunderscore )}
\end{itemize}
Instrumento, para determinar a densidade ou o pêso específico dos líquidos.
\section{Areopagita}
\begin{itemize}
\item {Grp. gram.:m.}
\end{itemize}
\begin{itemize}
\item {Proveniência:(Lat. \textunderscore areopagita\textunderscore )}
\end{itemize}
Membro do areópago.
\section{Areópago}
\begin{itemize}
\item {Grp. gram.:m.}
\end{itemize}
\begin{itemize}
\item {Utilização:Ext.}
\end{itemize}
\begin{itemize}
\item {Proveniência:(Gr. \textunderscore areiopagos\textunderscore )}
\end{itemize}
Tribunal atheniense.
Assembleia de magistrados, sábios, etc.
\section{Areoso}
\begin{itemize}
\item {Grp. gram.:adj.}
\end{itemize}
O mesmo que \textunderscore arenoso\textunderscore .
\section{Areossístilo}
\begin{itemize}
\item {Grp. gram.:m.}
\end{itemize}
\begin{itemize}
\item {Utilização:Archit.}
\end{itemize}
\begin{itemize}
\item {Proveniência:(Do gr. \textunderscore araios\textunderscore  + \textunderscore sustulos\textunderscore )}
\end{itemize}
Distribuição de columnas, cujos espaços são sístilos.
\section{Areostilo}
\begin{itemize}
\item {Grp. gram.:m.}
\end{itemize}
\begin{itemize}
\item {Utilização:Archit.}
\end{itemize}
\begin{itemize}
\item {Proveniência:(Do gr. \textunderscore araios\textunderscore  + \textunderscore stulos\textunderscore )}
\end{itemize}
Intercolúmnio de grande largura, usado só na architectura toscana, quando as architraves eram de madeira.
\section{Areostylo}
\begin{itemize}
\item {Grp. gram.:m.}
\end{itemize}
\begin{itemize}
\item {Utilização:Archit.}
\end{itemize}
\begin{itemize}
\item {Proveniência:(Do gr. \textunderscore araios\textunderscore  + \textunderscore stulos\textunderscore )}
\end{itemize}
Intercolúmnio de grande largura, usado só na architectura toscana, quando as architraves eram de madeira.
\section{Areosýstylo}
\begin{itemize}
\item {fónica:sis}
\end{itemize}
\begin{itemize}
\item {Grp. gram.:m.}
\end{itemize}
\begin{itemize}
\item {Utilização:Archit.}
\end{itemize}
\begin{itemize}
\item {Proveniência:(Do gr. \textunderscore araios\textunderscore  + \textunderscore sustulos\textunderscore )}
\end{itemize}
Distribuição de columnas, cujos espaços são sýstylos.
\section{Areotectónica}
\begin{itemize}
\item {Grp. gram.:f.}
\end{itemize}
\begin{itemize}
\item {Proveniência:(Do gr. \textunderscore areios\textunderscore  + \textunderscore tektonike\textunderscore )}
\end{itemize}
Arte, que trata do ataque e defesa das praças de guerra.
\section{Arepa}
\begin{itemize}
\item {Grp. gram.:f.}
\end{itemize}
Empada, feita de farinha de milho com carne de porco e muito usada em alguns pontos da Espanha.
\section{Arepabas}
\begin{itemize}
\item {Grp. gram.:m. pl.}
\end{itemize}
Selvagens da América do Norte.
\section{Arequeira}
\begin{itemize}
\item {Grp. gram.:f.}
\end{itemize}
O mesmo que \textunderscore areca\textunderscore .
Antiga unidade monetária, nas communidades de Gôa.
\section{Areranha}
\begin{itemize}
\item {Grp. gram.:f.}
\end{itemize}
Animal aquático ou amphíbio do Brasil.
\section{Aresco}
\begin{itemize}
\item {Grp. gram.:m.}
\end{itemize}
Gênero de coleópteros.
\section{Aresta}
\begin{itemize}
\item {Grp. gram.:f.}
\end{itemize}
\begin{itemize}
\item {Utilização:Geom.}
\end{itemize}
\begin{itemize}
\item {Utilização:Prov.}
\end{itemize}
\begin{itemize}
\item {Utilização:Prov.}
\end{itemize}
\begin{itemize}
\item {Proveniência:(Do lat. \textunderscore arista\textunderscore )}
\end{itemize}
Filete sêco e delgado, que nasce das palhetas floraes das plantas gramíneas.
Pragana.
Coisa de pouca monta.
Intersecção de dois planos que formam ângulo diedro.
Linha, que separa as duas vertentes principaes de uma montanha.
Esquina.
Partícula inútil, que cái da estriga, quando esta se fia, e que faz parte dos tomentos.
O mesmo que \textunderscore argueiro\textunderscore .
\section{Aresteiro}
\begin{itemize}
\item {Grp. gram.:m.}
\end{itemize}
\begin{itemize}
\item {Proveniência:(De \textunderscore aresto\textunderscore )}
\end{itemize}
Jurisconsulto, que allega casos julgados.
\section{Arestim}
\begin{itemize}
\item {Grp. gram.:m.}
\end{itemize}
Eczema dos animaes equídeos.
\section{Aresto}
\begin{itemize}
\item {Grp. gram.:m.}
\end{itemize}
\begin{itemize}
\item {Utilização:Jur.}
\end{itemize}
Caso julgado; decisão judicial.
Solução de uma difficuldade.
(Cp. \textunderscore arresto\textunderscore )
\section{Arestoso}
\begin{itemize}
\item {Grp. gram.:adj.}
\end{itemize}
Que tem arestas.
\section{Arestudo}
\begin{itemize}
\item {Grp. gram.:adj.}
\end{itemize}
Que tem aresta, que é áspero, (falando-se do linho que teve pouco curtimento). Cf. \textunderscore Bibl. da Gente do Campo\textunderscore , 299.
\section{Aretologia}
\begin{itemize}
\item {Grp. gram.:f.}
\end{itemize}
Parte da Philosophia moral, que trata da virtude, sua natureza e meios de a possuir.
\section{Aretológico}
\begin{itemize}
\item {Grp. gram.:adj.}
\end{itemize}
Relativo á Aretologia.
\section{Aretólogo}
\begin{itemize}
\item {Grp. gram.:m.}
\end{itemize}
Aquelle que se dedica ao estudo da Aretologia.
\section{Areu}
\begin{itemize}
\item {Grp. gram.:adj.}
\end{itemize}
Embaraçado, que não sabe o que fazer:«\textunderscore viu-se areu com leões e crocodilos\textunderscore ». Filinto, X, 145.
\section{Areumático}
\begin{itemize}
\item {fónica:arreu}
\end{itemize}
\begin{itemize}
\item {Grp. gram.:adj.}
\end{itemize}
\begin{itemize}
\item {Proveniência:(De \textunderscore a\textunderscore  priv. + \textunderscore reumático\textunderscore )}
\end{itemize}
Que não foi atacado de reumatismo.
\section{Arevaco}
\begin{itemize}
\item {Grp. gram.:m.}
\end{itemize}
Idioma, falado nas Guianas inglesa e hollandesa.
\section{Arevessado}
\textunderscore m. Ant.\textunderscore  (?)«\textunderscore ...leua corenta, e dous arevessados\textunderscore ». \textunderscore Livro Náutico\textunderscore , ms. da Bibl. Nac. de Lisbôa.
\section{Arfada}
\begin{itemize}
\item {Grp. gram.:f.}
\end{itemize}
Acto de \textunderscore arfar\textunderscore .
\section{Arfador}
\begin{itemize}
\item {Grp. gram.:adj.}
\end{itemize}
\begin{itemize}
\item {Proveniência:(De \textunderscore arfar\textunderscore )}
\end{itemize}
Que é gingão. Cf. Filinto, VI, 7.
\section{Arfadura}
\begin{itemize}
\item {Grp. gram.:f.}
\end{itemize}
O mesmo que \textunderscore arfagem\textunderscore .
\section{Arfagem}
\begin{itemize}
\item {Grp. gram.:f.}
\end{itemize}
O mesmo que \textunderscore arfada\textunderscore .
\section{Arfante}
\begin{itemize}
\item {Grp. gram.:adj.}
\end{itemize}
Que arfa.
\section{Arfar}
\begin{itemize}
\item {Grp. gram.:v. i.}
\end{itemize}
\begin{itemize}
\item {Utilização:Náut.}
\end{itemize}
Respirar com difficuldade.
Offegar.
Baloiçar.
Dar baloiço, de prôa á popa.
(Cast. \textunderscore arfar\textunderscore )
\section{Arfece}
\begin{itemize}
\item {Grp. gram.:adj.}
\end{itemize}
\begin{itemize}
\item {Utilização:Ant.}
\end{itemize}
O mesmo que \textunderscore refece\textunderscore .
\section{Arféloa}
\begin{itemize}
\item {Grp. gram.:f.}
\end{itemize}
O mesmo que \textunderscore alféloa\textunderscore .
\section{Arféria}
\begin{itemize}
\item {Grp. gram.:f.}
\end{itemize}
Vaso com vinho ou água, para as libações, em honra dos deuses infernaes, na Roma antiga.
(Lat. \textunderscore arfería\textunderscore ).
\section{Arfil}
\begin{itemize}
\item {Grp. gram.:m.}
\end{itemize}
Uma das peças do jôgo de xadrez, mais conhecida por \textunderscore elephante\textunderscore .
\section{Arga}
\begin{itemize}
\item {Grp. gram.:f.}
\end{itemize}
Fruto africano.
Gênero de insectos coleópteros.
Leite de égua fermentado, bebida usada na Tartária.
\section{Argaceiro}
\begin{itemize}
\item {Grp. gram.:m.}
\end{itemize}
\begin{itemize}
\item {Utilização:Prov.}
\end{itemize}
\begin{itemize}
\item {Utilização:minh.}
\end{itemize}
Homem que se emprega na apanha do argaço.
\section{Argaço}
\begin{itemize}
\item {Grp. gram.:m.}
\end{itemize}
\begin{itemize}
\item {Utilização:T. de Monção}
\end{itemize}
O mesmo que \textunderscore alga\textunderscore .
Caruma sêca.
(Por \textunderscore algaço\textunderscore , de \textunderscore alga\textunderscore )
\section{Argadilho}
\begin{itemize}
\item {Grp. gram.:m.}
\end{itemize}
\begin{itemize}
\item {Utilização:Prov.}
\end{itemize}
\begin{itemize}
\item {Utilização:trasm.}
\end{itemize}
O mesmo que \textunderscore dobadoira\textunderscore .
(Cast. \textunderscore argadil\textunderscore )
\section{Argal}
\begin{itemize}
\item {Grp. gram.:m.}
\end{itemize}
O mesmo que \textunderscore argau\textunderscore ^1.
\section{Argala}
\begin{itemize}
\item {Grp. gram.:f.}
\end{itemize}
Espécie de cegonha da Índia.
\section{Argali}
\begin{itemize}
\item {Grp. gram.:m.}
\end{itemize}
\begin{itemize}
\item {Proveniência:(T. mongol)}
\end{itemize}
Carneiro da Sibéria.
\section{Argamassa}
\begin{itemize}
\item {Grp. gram.:f.}
\end{itemize}
\begin{itemize}
\item {Utilização:Gír.}
\end{itemize}
Cimento, formado de cal, areia e água.
Comida.
\section{Argamassador}
\begin{itemize}
\item {Grp. gram.:m.}
\end{itemize}
Aquelle que argamassa.
\section{Argamassar}
\begin{itemize}
\item {Grp. gram.:v. t.}
\end{itemize}
Unir, tapar, com argamassa.
\section{Argamula}
\begin{itemize}
\item {Grp. gram.:f.}
\end{itemize}
\begin{itemize}
\item {Utilização:Prov.}
\end{itemize}
\begin{itemize}
\item {Utilização:beir.}
\end{itemize}
Planta campesina, que se aproveita em sustento de porcos.
\section{Argana}
\begin{itemize}
\item {Grp. gram.:f.}
\end{itemize}
\begin{itemize}
\item {Utilização:Ant.}
\end{itemize}
\begin{itemize}
\item {Utilização:Prov.}
\end{itemize}
Guindaste. Máquina de guerra, para arremêsso de combustíveis.
Espinha de peixe.
(Cast. \textunderscore argano\textunderscore )
\section{Arganaça}
\begin{itemize}
\item {Grp. gram.:f.}
\end{itemize}
\begin{itemize}
\item {Utilização:Prov.}
\end{itemize}
O mesmo que \textunderscore arganaz\textunderscore .
\section{Arganaz}
\begin{itemize}
\item {Grp. gram.:m.}
\end{itemize}
\begin{itemize}
\item {Proveniência:(De \textunderscore argana\textunderscore )}
\end{itemize}
Rato silvestre.
Homem alto.
\section{Arganel}
\begin{itemize}
\item {Grp. gram.:m.}
\end{itemize}
O mesmo que \textunderscore arganéu\textunderscore .
\section{Arganéo}
\begin{itemize}
\item {Grp. gram.:m.}
\end{itemize}
\begin{itemize}
\item {Utilização:Prov.}
\end{itemize}
Argola, em que se prendem as cordas da artilharia.
Argola da âncora.
Peça de ferro flexível, que se espeta no focinho do porco, torcendo as duas extremidades uma na outra, para que o animal não possa fossar.
(Dem. do hyp. \textunderscore argano\textunderscore , do lat. \textunderscore arganum\textunderscore )
\section{Arganéu}
\begin{itemize}
\item {Grp. gram.:m.}
\end{itemize}
\begin{itemize}
\item {Utilização:Prov.}
\end{itemize}
Argola, em que se prendem as cordas da artilharia.
Argola da âncora.
Peça de ferro flexível, que se espeta no focinho do porco, torcendo as duas extremidades uma na outra, para que o animal não possa fossar.
(Dem. do hyp. \textunderscore argano\textunderscore , do lat. \textunderscore arganum\textunderscore )
\section{Argânia}
\begin{itemize}
\item {Grp. gram.:f.}
\end{itemize}
Gênero de plantas sapotáceas.
\section{Argão}
\begin{itemize}
\item {Grp. gram.:f.}
\end{itemize}
\begin{itemize}
\item {Utilização:Ant.}
\end{itemize}
Espécie de alforge.
\section{Argão}
\begin{itemize}
\item {Grp. gram.:m.}
\end{itemize}
O mesmo que \textunderscore argau\textunderscore ^1.
\section{Argau}
\begin{itemize}
\item {Grp. gram.:m.}
\end{itemize}
\begin{itemize}
\item {Proveniência:(Do b. lat. \textunderscore arganum\textunderscore )}
\end{itemize}
Canudo de folha ou de cana, com que se tiram líquidos de vasilhas.
\section{Argau}
\begin{itemize}
\item {Grp. gram.:m.}
\end{itemize}
\begin{itemize}
\item {Utilização:Ant.}
\end{itemize}
Roupão de luto.
Espécie de garnacha, que os ecclesiásticos regulares usavam de inverno sôbre o hábito.
\section{Argavaço}
\begin{itemize}
\item {Grp. gram.:m.}
\end{itemize}
\begin{itemize}
\item {Utilização:Prov.}
\end{itemize}
\begin{itemize}
\item {Utilização:trasm.}
\end{itemize}
O mesmo que \textunderscore gravato\textunderscore .
\section{Argel}
\begin{itemize}
\item {Grp. gram.:adj.}
\end{itemize}
\begin{itemize}
\item {Utilização:Bras. do N}
\end{itemize}
\begin{itemize}
\item {Utilização:Ant.}
\end{itemize}
\begin{itemize}
\item {Grp. gram.:M.}
\end{itemize}
\begin{itemize}
\item {Utilização:ant.}
\end{itemize}
\begin{itemize}
\item {Utilização:Pop.}
\end{itemize}
Diz-se do cavallo que tem pés brancos.
Desajeitado, desmazelado.
Desgraçado, mofino.
Barulho, ajuntamento tumultuoso.
(Cast. \textunderscore argel\textunderscore )
\section{Argelino}
\begin{itemize}
\item {Grp. gram.:m.}
\end{itemize}
\begin{itemize}
\item {Grp. gram.:M.}
\end{itemize}
Relativo a Argel.
Habitante de Argel.
\section{Argêmon}
\begin{itemize}
\item {Grp. gram.:m.}
\end{itemize}
\begin{itemize}
\item {Proveniência:(Gr. \textunderscore argemone\textunderscore )}
\end{itemize}
Úlcera da córnea, arredondada e superficial.
Papoila espinhosa.
\section{Argêmone}
\begin{itemize}
\item {Grp. gram.:f.}
\end{itemize}
\begin{itemize}
\item {Proveniência:(Gr. \textunderscore argemone\textunderscore )}
\end{itemize}
Úlcera da córnea, arredondada e superficial.
Papoila espinhosa.
\section{Argempel}
\begin{itemize}
\item {Grp. gram.:m.}
\end{itemize}
\begin{itemize}
\item {Proveniência:(Do lat. \textunderscore argentum\textunderscore  + \textunderscore pellis\textunderscore )}
\end{itemize}
Coiro lavrado e prateado.
\section{Argência}
\begin{itemize}
\item {Grp. gram.:f.}
\end{itemize}
Numerário?«\textunderscore ...um cidadão de Mans, capão de sua argência...\textunderscore »Filinto, VI, 364, (ed. Paris).
(Será êrro typogr., por \textunderscore agência\textunderscore ?)
\section{Argentador}
\begin{itemize}
\item {Grp. gram.:m.  e  adj.}
\end{itemize}
O que argenta.
\section{Argentão}
\begin{itemize}
\item {Grp. gram.:m.}
\end{itemize}
\begin{itemize}
\item {Proveniência:(De \textunderscore argento\textunderscore )}
\end{itemize}
Liga de cobre, estanho e níqel.
\section{Argentar}
\begin{itemize}
\item {Grp. gram.:v. t.}
\end{itemize}
\begin{itemize}
\item {Proveniência:(Do lat. \textunderscore argentum\textunderscore )}
\end{itemize}
Pratear: tornar branco.
\section{Argentaria}
\begin{itemize}
\item {Grp. gram.:f.}
\end{itemize}
\begin{itemize}
\item {Proveniência:(De \textunderscore argento\textunderscore )}
\end{itemize}
Guarnição de prata.
Baixella de prata.
\section{Argentário}
\begin{itemize}
\item {Grp. gram.:m.}
\end{itemize}
\begin{itemize}
\item {Proveniência:(Lat. \textunderscore argentarius\textunderscore )}
\end{itemize}
Homem rico.
Guarda-prata.
\section{Argentato}
\begin{itemize}
\item {Grp. gram.:m.}
\end{itemize}
Sal, produzido pelo óxydo de prata.
\section{Argentear}
\begin{itemize}
\item {Grp. gram.:v. t.}
\end{itemize}
(V.argentar)
\section{Argênteo}
\begin{itemize}
\item {Grp. gram.:adj.}
\end{itemize}
\begin{itemize}
\item {Proveniência:(Lat. \textunderscore argenteus\textunderscore )}
\end{itemize}
Feito de Prata.
Brilhante como prata.
Que sôa como prata.
\section{Argenticérulo}
\begin{itemize}
\item {Grp. gram.:adj.}
\end{itemize}
\begin{itemize}
\item {Proveniência:(Do lat. \textunderscore argentum\textunderscore  + \textunderscore caerulus\textunderscore )}
\end{itemize}
Que participa das côres prateada e azul. Cf. Castilho, \textunderscore Fastos\textunderscore , III, 177.
\section{Argêntico}
\begin{itemize}
\item {Grp. gram.:adj.}
\end{itemize}
\begin{itemize}
\item {Proveniência:(Do lat. \textunderscore argentum\textunderscore )}
\end{itemize}
Diz-se do óxydo e dos saes, que têm por base a prata.
\section{Argentífero}
\begin{itemize}
\item {Grp. gram.:adj.}
\end{itemize}
\begin{itemize}
\item {Proveniência:(Do lat. \textunderscore argentum\textunderscore  + \textunderscore ferre\textunderscore )}
\end{itemize}
Que contém prata.
\section{Argentífico}
\begin{itemize}
\item {Grp. gram.:adj.}
\end{itemize}
\begin{itemize}
\item {Proveniência:(Do lat. \textunderscore argentum\textunderscore  + \textunderscore facere\textunderscore )}
\end{itemize}
Que converte em prata.
\section{Argentifólio}
\begin{itemize}
\item {Grp. gram.:adj.}
\end{itemize}
\begin{itemize}
\item {Proveniência:(Do lat. \textunderscore argentum\textunderscore  + \textunderscore folium\textunderscore )}
\end{itemize}
Que tem fôlhas prateadas.
\section{Argentifronte}
\begin{itemize}
\item {Grp. gram.:adj.}
\end{itemize}
Que tem a fronte prateada. Cf. Filinto, IV, 258.
\section{Argentina}
\begin{itemize}
\item {Grp. gram.:f.}
\end{itemize}
\begin{itemize}
\item {Proveniência:(De \textunderscore argentino\textunderscore )}
\end{itemize}
Planta rosácea.
Peixe, da fam. dos salmões.
\section{Argentino}
\begin{itemize}
\item {Grp. gram.:adj.}
\end{itemize}
\begin{itemize}
\item {Grp. gram.:M.}
\end{itemize}
\begin{itemize}
\item {Proveniência:(Lat. \textunderscore argentinus\textunderscore )}
\end{itemize}
O mesmo que \textunderscore argênteo\textunderscore .
Habitante da República Argentina ou do Rio-da-Prata.
\section{Argento}
\begin{itemize}
\item {Grp. gram.:m.}
\end{itemize}
\begin{itemize}
\item {Utilização:Poét.}
\end{itemize}
\begin{itemize}
\item {Grp. gram.:Adj.}
\end{itemize}
\begin{itemize}
\item {Proveniência:(Lat. \textunderscore argentum\textunderscore )}
\end{itemize}
Designação alatinada da prata.
O mar.
O mesmo que \textunderscore argênteo\textunderscore . Cf. Filinto, XIV, 21.
\section{Argeste}
\begin{itemize}
\item {Grp. gram.:m.}
\end{itemize}
\begin{itemize}
\item {Utilização:Náut.}
\end{itemize}
\begin{itemize}
\item {Utilização:ant.}
\end{itemize}
\begin{itemize}
\item {Proveniência:(Lat. \textunderscore argestes\textunderscore )}
\end{itemize}
Vento de Noroéste.
\section{Argila}
\begin{itemize}
\item {Grp. gram.:f.}
\end{itemize}
\begin{itemize}
\item {Utilização:Fig.}
\end{itemize}
\begin{itemize}
\item {Proveniência:(Lat. \textunderscore argilla\textunderscore )}
\end{itemize}
Substância terrosa, esbranquiçada, formada de sílica e alumina.
Barro.
Fragilidade.
\section{Argiláceo}
\begin{itemize}
\item {Grp. gram.:adj.}
\end{itemize}
\begin{itemize}
\item {Proveniência:(Lat. \textunderscore argillaceus\textunderscore )}
\end{itemize}
O mesmo que \textunderscore argiloso\textunderscore .
\section{Argileira}
\begin{itemize}
\item {Grp. gram.:f.}
\end{itemize}
Lugar, donde se extrái a argila; barreiro.
\section{Argilífero}
\begin{itemize}
\item {Grp. gram.:adj.}
\end{itemize}
\begin{itemize}
\item {Proveniência:(Do lat. \textunderscore argilla\textunderscore  + \textunderscore ferre\textunderscore )}
\end{itemize}
Que contém argila.
\section{Argiliforme}
\begin{itemize}
\item {Grp. gram.:adj.}
\end{itemize}
\begin{itemize}
\item {Proveniência:(Do lat. \textunderscore argilla\textunderscore  + \textunderscore forma\textunderscore )}
\end{itemize}
Semelhante á argila.
\section{Argilita}
\begin{itemize}
\item {Grp. gram.:f.}
\end{itemize}
O mesmo ou melhor que \textunderscore argilite\textunderscore .
\section{Argilite}
\begin{itemize}
\item {Grp. gram.:f.}
\end{itemize}
\begin{itemize}
\item {Proveniência:(De \textunderscore argilla\textunderscore )}
\end{itemize}
Xisto argiloso.
\section{Argilla}
\begin{itemize}
\item {Grp. gram.:f.}
\end{itemize}
\begin{itemize}
\item {Utilização:Fig.}
\end{itemize}
\begin{itemize}
\item {Proveniência:(Lat. \textunderscore argilla\textunderscore )}
\end{itemize}
Substância terrosa, esbranquiçada, formada de sílica e alumina.
Barro.
Fragilidade.
\section{Argilláceo}
\begin{itemize}
\item {Grp. gram.:adj.}
\end{itemize}
\begin{itemize}
\item {Proveniência:(Lat. \textunderscore argillaceus\textunderscore )}
\end{itemize}
O mesmo que \textunderscore argilloso\textunderscore .
\section{Argilleira}
\begin{itemize}
\item {Grp. gram.:f.}
\end{itemize}
Lugar, donde se extrái a argilla; barreiro.
\section{Argillífero}
\begin{itemize}
\item {Grp. gram.:adj.}
\end{itemize}
\begin{itemize}
\item {Proveniência:(Do lat. \textunderscore argilla\textunderscore  + \textunderscore ferre\textunderscore )}
\end{itemize}
Que contém argilla.
\section{Argilliforme}
\begin{itemize}
\item {Grp. gram.:adj.}
\end{itemize}
\begin{itemize}
\item {Proveniência:(Do lat. \textunderscore argilla\textunderscore  + \textunderscore forma\textunderscore )}
\end{itemize}
Semelhante á argilla.
\section{Argillita}
\begin{itemize}
\item {Grp. gram.:f.}
\end{itemize}
O mesmo ou melhor que \textunderscore argillite\textunderscore .
\section{Argillite}
\begin{itemize}
\item {Grp. gram.:f.}
\end{itemize}
\begin{itemize}
\item {Proveniência:(De \textunderscore argilla\textunderscore )}
\end{itemize}
Xisto argiloso.
\section{Argilloide}
\begin{itemize}
\item {Grp. gram.:adj.}
\end{itemize}
\begin{itemize}
\item {Proveniência:(Do gr. \textunderscore argillos\textunderscore  + \textunderscore eidos\textunderscore )}
\end{itemize}
O mesmo que \textunderscore argilliforme\textunderscore .
\section{Argillolíthico}
\begin{itemize}
\item {Grp. gram.:adj.}
\end{itemize}
Que é da natureza do argillólitho.
\section{Argillólitho}
\begin{itemize}
\item {Grp. gram.:m.}
\end{itemize}
\begin{itemize}
\item {Proveniência:(Do gr. \textunderscore argillos\textunderscore  + \textunderscore lithos\textunderscore )}
\end{itemize}
Argilla sedimentária, que se petrificou.
\section{Argillomicáceo}
\begin{itemize}
\item {Grp. gram.:adj.}
\end{itemize}
\begin{itemize}
\item {Utilização:Geol.}
\end{itemize}
Em que entra argilla e mica.
\section{Argillóphyro}
\begin{itemize}
\item {Grp. gram.:m.}
\end{itemize}
Espécie de pórphyro decomposto.
\section{Argilloso}
\begin{itemize}
\item {Grp. gram.:adj.}
\end{itemize}
\begin{itemize}
\item {Proveniência:(Lat. \textunderscore argillosus\textunderscore )}
\end{itemize}
Que tem argilla.
Que é da natureza da argilla.
\section{Argilófiro}
\begin{itemize}
\item {Grp. gram.:m.}
\end{itemize}
Espécie de pórfiro decomposto.
\section{Argiloide}
\begin{itemize}
\item {Grp. gram.:adj.}
\end{itemize}
\begin{itemize}
\item {Proveniência:(Do gr. \textunderscore argillos\textunderscore  + \textunderscore eidos\textunderscore )}
\end{itemize}
O mesmo que \textunderscore argiliforme\textunderscore .
\section{Argilólito}
\begin{itemize}
\item {Grp. gram.:m.}
\end{itemize}
\begin{itemize}
\item {Proveniência:(Do gr. \textunderscore argillos\textunderscore  + \textunderscore lithos\textunderscore )}
\end{itemize}
Argilla sedimentária, que se petrificou.
\section{Argilolítico}
\begin{itemize}
\item {Grp. gram.:adj.}
\end{itemize}
Que é da natureza do argilólito.
\section{Argilomicáceo}
\begin{itemize}
\item {Grp. gram.:adj.}
\end{itemize}
\begin{itemize}
\item {Utilização:Geol.}
\end{itemize}
Em que entra argila e mica.
\section{Argiloso}
\begin{itemize}
\item {Grp. gram.:adj.}
\end{itemize}
\begin{itemize}
\item {Proveniência:(Lat. \textunderscore argillosus\textunderscore )}
\end{itemize}
Que tem argila.
Que é da natureza da argila.
\section{Argina}
\begin{itemize}
\item {Grp. gram.:f.}
\end{itemize}
Gênero de insectos lepidópteros diurnos.
\section{Argipampa}
\begin{itemize}
\item {Grp. gram.:f.}
\end{itemize}
\begin{itemize}
\item {Utilização:T. do Fundão}
\end{itemize}
Qualquer móvel desmesurado e feio.
\section{Argite}
\begin{itemize}
\item {Grp. gram.:f.}
\end{itemize}
\begin{itemize}
\item {Proveniência:(Lat. \textunderscore argitis\textunderscore )}
\end{itemize}
Variedade de uva branca, muito apreciada entre os antigos Romanos. Cf. Castilho, \textunderscore Georg.\textunderscore , 81.
\section{Argivo}
\begin{itemize}
\item {Grp. gram.:m.}
\end{itemize}
\begin{itemize}
\item {Proveniência:(Lat. \textunderscore argivus\textunderscore )}
\end{itemize}
O mesmo que \textunderscore grego\textunderscore .
\section{Argo}
\begin{itemize}
\item {Grp. gram.:m.}
\end{itemize}
\begin{itemize}
\item {Proveniência:(Lat. \textunderscore Argo\textunderscore , \textunderscore Argus\textunderscore , n. p. do navio, em que embarcaram os argonautas)}
\end{itemize}
Constellação austral.
\section{Argo}
\begin{itemize}
\item {Grp. gram.:m.}
\end{itemize}
\begin{itemize}
\item {Utilização:Fig.}
\end{itemize}
\begin{itemize}
\item {Proveniência:(Lat. \textunderscore Argus\textunderscore , n. p.)}
\end{itemize}
Personagem mythológica de cem olhos.
Pessôa, que vê muito, que observa bem.
\section{Argofilo}
\begin{itemize}
\item {Grp. gram.:m.}
\end{itemize}
\begin{itemize}
\item {Proveniência:(Do gr. \textunderscore argo\textunderscore  + \textunderscore phulon\textunderscore )}
\end{itemize}
Arbusto da Nova-Escócia.
\section{Argola}
\begin{itemize}
\item {Grp. gram.:f.}
\end{itemize}
\begin{itemize}
\item {Proveniência:(Do ár. \textunderscore al-gole\textunderscore )}
\end{itemize}
Pequeno arco. Anel de ferro, a que se prende alguma coisa.
Arrecada.
Aldrava.
\section{Argolada}
\begin{itemize}
\item {Grp. gram.:f.}
\end{itemize}
\begin{itemize}
\item {Utilização:T. do Fundão}
\end{itemize}
\begin{itemize}
\item {Proveniência:(De \textunderscore argola\textunderscore )}
\end{itemize}
Pancada com a aldrava da porta.
Pancada com a ponta ferrada de um cacete; cacetada.
\section{Argolado}
\begin{itemize}
\item {Grp. gram.:adj.}
\end{itemize}
Munido de argola: \textunderscore varapau argolado\textunderscore .
\section{Argolagem}
\begin{itemize}
\item {Grp. gram.:f.}
\end{itemize}
Systema e conjuncto de argolas, nos antigos engenhos de açúcar.
\section{Argolão}
\begin{itemize}
\item {Grp. gram.:m.}
\end{itemize}
Argola grande.
\section{Argolar}
\begin{itemize}
\item {Grp. gram.:v. t.}
\end{itemize}
Prender com argolas.
Pôr argolas em.
Enfiar (estribos) até á cava do pé.
\section{Argoleiro}
\begin{itemize}
\item {Grp. gram.:m.}
\end{itemize}
Fabricante de argolas.
\section{Argolinha}
\begin{itemize}
\item {Grp. gram.:f.}
\end{itemize}
Jôgo popular, o mesmo que \textunderscore pampolinha\textunderscore .
\section{Argolista}
\begin{itemize}
\item {Grp. gram.:m.}
\end{itemize}
\begin{itemize}
\item {Utilização:T. de Setúbal}
\end{itemize}
Gymnasta, que trabalha em argolas suspensas.
Aquelle que dos barcos baldeia o sal para dentro dos vapores.
\section{Argomas}
\begin{itemize}
\item {Grp. gram.:f. pl.}
\end{itemize}
\begin{itemize}
\item {Utilização:Prov.}
\end{itemize}
Ramagem miúda, eliminada das árvores quando se limpam.
\section{Argonar}
\begin{itemize}
\item {Grp. gram.:v. i.}
\end{itemize}
\begin{itemize}
\item {Utilização:Prov.}
\end{itemize}
\begin{itemize}
\item {Utilização:trasm.}
\end{itemize}
Apanhar hortaliça no campo.
(Metáth. de \textunderscore agronar\textunderscore , de \textunderscore agro\textunderscore ^1)
\section{Argonauta}
\begin{itemize}
\item {Grp. gram.:m.}
\end{itemize}
\begin{itemize}
\item {Utilização:Ext.}
\end{itemize}
\begin{itemize}
\item {Utilização:Zool.}
\end{itemize}
\begin{itemize}
\item {Proveniência:(Lat. \textunderscore argonantae\textunderscore )}
\end{itemize}
Cada um dos navegadores, que, segundo as lendas gregas, foram á Cólchida em a nau \textunderscore Argo\textunderscore .
Navegante ousado.
Mollúsculo cephalópode.
\section{Argonautáceo}
\begin{itemize}
\item {Grp. gram.:adj.}
\end{itemize}
Diz-se dos molluscos, semelhantes ao que se chama argonauta.
\section{Argonáutico}
\begin{itemize}
\item {Grp. gram.:adj.}
\end{itemize}
Relativo aos argonautas.
\section{Argonautídeos}
\begin{itemize}
\item {Grp. gram.:m. pl.}
\end{itemize}
\begin{itemize}
\item {Proveniência:(Do gr. \textunderscore argonautes\textunderscore  + \textunderscore eidos\textunderscore )}
\end{itemize}
Fam. de molluscos, que têm por typo o \textunderscore argonauta\textunderscore .
\section{Argonina}
\begin{itemize}
\item {Grp. gram.:f.}
\end{itemize}
\begin{itemize}
\item {Utilização:Chím.}
\end{itemize}
Combinação albuminada de prata, de propriedades bactericidas.
\section{Argophyllo}
\begin{itemize}
\item {Grp. gram.:m.}
\end{itemize}
\begin{itemize}
\item {Proveniência:(Do gr. \textunderscore argo\textunderscore  + \textunderscore phulon\textunderscore )}
\end{itemize}
Arbusto da Nova-Escócia.
\section{Argos}
\begin{itemize}
\item {Grp. gram.:m.}
\end{itemize}
\begin{itemize}
\item {Proveniência:(Lat. \textunderscore Argo\textunderscore , \textunderscore Argus\textunderscore , n. p. do navio, em que embarcaram os argonautas)}
\end{itemize}
Constellação austral.
\section{Argos}
\begin{itemize}
\item {Grp. gram.:m.}
\end{itemize}
\begin{itemize}
\item {Utilização:Fig.}
\end{itemize}
\begin{itemize}
\item {Proveniência:(Lat. \textunderscore Argus\textunderscore , n. p.)}
\end{itemize}
Personagem mythológica de cem olhos.
Pessôa, que vê muito, que observa bem.
\section{Argúcia}
\begin{itemize}
\item {Grp. gram.:f.}
\end{itemize}
\begin{itemize}
\item {Proveniência:(Lat. \textunderscore argutia\textunderscore )}
\end{itemize}
Agudeza de espírito.
Argumento subtil.
Chiste.
\section{Arguciar}
\begin{itemize}
\item {Grp. gram.:v. i.}
\end{itemize}
Empregar argúcias.
\section{Arguciosamente}
\begin{itemize}
\item {Grp. gram.:adv.}
\end{itemize}
De modo \textunderscore argucioso\textunderscore ; com argúcia.
\section{Argília}
\begin{itemize}
\item {Grp. gram.:f.}
\end{itemize}
Gênero de plantas bignoniáceas.
\section{Argira}
\begin{itemize}
\item {Grp. gram.:f.}
\end{itemize}
\begin{itemize}
\item {Proveniência:(Do gr. \textunderscore arguros\textunderscore )}
\end{itemize}
Gênero de insectos dípteros.
\section{Argirântemo}
\begin{itemize}
\item {Grp. gram.:adj.}
\end{itemize}
\begin{itemize}
\item {Proveniência:(Do gr. \textunderscore arguros\textunderscore  + \textunderscore anthema\textunderscore )}
\end{itemize}
Que tem flôres brancas como a prata.
\section{Argiráspides}
\begin{itemize}
\item {Grp. gram.:m. pl.}
\end{itemize}
\begin{itemize}
\item {Proveniência:(Do gr. \textunderscore arguros\textunderscore  + \textunderscore aspis\textunderscore )}
\end{itemize}
Soldados de Alexandre Magno, que usavam escudos brancos.
\section{Argireia}
\begin{itemize}
\item {Grp. gram.:f.}
\end{itemize}
Gênero de plantas convolvuláceas.
\section{Argiria}
\begin{itemize}
\item {Grp. gram.:f.}
\end{itemize}
\begin{itemize}
\item {Proveniência:(Do gr. \textunderscore arguros\textunderscore , prata)}
\end{itemize}
Depósito de prata numa parte da pelle ou das mucosas, em resultado do abuso dos saes de prata.
\section{Argiríase}
\begin{itemize}
\item {Grp. gram.:f.}
\end{itemize}
O mesmo que \textunderscore argiria\textunderscore .
\section{Argírico}
\begin{itemize}
\item {Grp. gram.:adj.}
\end{itemize}
Relativo á prata.
\section{Argirita}
\begin{itemize}
\item {Grp. gram.:f.}
\end{itemize}
Pedra metállica, também designada por margarida de prata.
\section{Argirite}
\begin{itemize}
\item {Grp. gram.:f.}
\end{itemize}
(V.argirita)
\section{Argiritrose}
\begin{itemize}
\item {Grp. gram.:f.}
\end{itemize}
Prata antinomiada, prata vermelha.
\section{Argiro}
\begin{itemize}
\item {Grp. gram.:m.}
\end{itemize}
Gênero de dípteros.
\section{Argirocéfalo}
\begin{itemize}
\item {Grp. gram.:adj.}
\end{itemize}
\begin{itemize}
\item {Proveniência:(Do gr. \textunderscore arguros\textunderscore  + \textunderscore kephale\textunderscore )}
\end{itemize}
Que tem cabeça branca.
\section{Argirócomo}
\begin{itemize}
\item {Grp. gram.:adj.}
\end{itemize}
\begin{itemize}
\item {Proveniência:(Do gr. \textunderscore arguros\textunderscore  + \textunderscore kome\textunderscore )}
\end{itemize}
Que tem cabelleira branca, (falando-se de cometas).
\section{Argirócrata}
\begin{itemize}
\item {Grp. gram.:m.}
\end{itemize}
\begin{itemize}
\item {Utilização:bras}
\end{itemize}
\begin{itemize}
\item {Utilização:Neol.}
\end{itemize}
\begin{itemize}
\item {Proveniência:(Do gr. \textunderscore arguros\textunderscore  + \textunderscore kratos\textunderscore )}
\end{itemize}
Indivíduo opulento, argentário.
\section{Argirodendro}
\begin{itemize}
\item {Grp. gram.:m.}
\end{itemize}
\begin{itemize}
\item {Proveniência:(Do gr. \textunderscore arguros\textunderscore  + \textunderscore dendron\textunderscore )}
\end{itemize}
Nome de várias plantas.
\section{Argirofilo}
\begin{itemize}
\item {Grp. gram.:adj.}
\end{itemize}
\begin{itemize}
\item {Proveniência:(Do gr. \textunderscore arguros\textunderscore  + \textunderscore phullon\textunderscore )}
\end{itemize}
Que tem fôlhas brancas como a prata.
\section{Argiróforo}
\begin{itemize}
\item {Grp. gram.:m.}
\end{itemize}
\begin{itemize}
\item {Utilização:Med.}
\end{itemize}
\begin{itemize}
\item {Utilização:ant.}
\end{itemize}
\begin{itemize}
\item {Proveniência:(Do gr. \textunderscore arguros\textunderscore  + \textunderscore phoros\textunderscore )}
\end{itemize}
Espécie de antídoto.
\section{Argirólito}
\begin{itemize}
\item {Grp. gram.:m.}
\end{itemize}
\begin{itemize}
\item {Proveniência:(Do gr. \textunderscore arguros\textunderscore  + \textunderscore lithos\textunderscore )}
\end{itemize}
Pedra de prata, ou que parece prata.
Pedra preciosa, descrita pelos antigos, mas cuja natureza se não sabe ao certo, suppondo-se que fôsse uma aventurina oriental.
\section{Argiropeia}
\begin{itemize}
\item {Grp. gram.:f.}
\end{itemize}
\begin{itemize}
\item {Proveniência:(Do gr. \textunderscore arguros\textunderscore  + \textunderscore poiein\textunderscore )}
\end{itemize}
Supposta arte de fazer prata.
\section{Argirose}
\begin{itemize}
\item {Grp. gram.:f.}
\end{itemize}
\begin{itemize}
\item {Proveniência:(Do gr. \textunderscore arguros\textunderscore , prata)}
\end{itemize}
O mais commum dos minérios de prata, malleável, pardo e de brilho metállico.
Prata sulfurada.
O mesmo que \textunderscore argiria\textunderscore .
\section{Argirostigmado}
\begin{itemize}
\item {Grp. gram.:adj.}
\end{itemize}
\begin{itemize}
\item {Utilização:Bot.}
\end{itemize}
\begin{itemize}
\item {Proveniência:(Do gr. \textunderscore arguros\textunderscore  + \textunderscore stigma\textunderscore )}
\end{itemize}
Que tem manchas brancas.
\section{Argiróstomo}
\begin{itemize}
\item {Grp. gram.:adj.}
\end{itemize}
\begin{itemize}
\item {Utilização:Zool.}
\end{itemize}
\begin{itemize}
\item {Proveniência:(Do gr. \textunderscore arguros\textunderscore  + \textunderscore stoma\textunderscore )}
\end{itemize}
Que tem a bôca da côr da prata.
\section{Argirotirso}
\begin{itemize}
\item {Grp. gram.:m.}
\end{itemize}
Combinação natural do sulfureto de prata com o antimónio.
Antimónio sulfurado.
Prata vermelha.
\section{Argucioso}
\begin{itemize}
\item {Grp. gram.:adj.}
\end{itemize}
Que usa de argúcias.
Que contém argúcia: \textunderscore um dito argucioso\textunderscore .
\section{Argueireiro}
\begin{itemize}
\item {Grp. gram.:adj.}
\end{itemize}
\begin{itemize}
\item {Utilização:Fig.}
\end{itemize}
Que procura argueiros.
Minucioso.
\section{Argueirinha}
\begin{itemize}
\item {Grp. gram.:adj. f.}
\end{itemize}
\begin{itemize}
\item {Proveniência:(De \textunderscore argueiro\textunderscore )}
\end{itemize}
Diz-se da pedra de cevar, com que se tiram argueiros dos olhos.
\section{Argueiro}
\begin{itemize}
\item {Grp. gram.:m.}
\end{itemize}
Palhinha; aresta.
Coisa insignificante.
Árvore do Brasil.
\section{Arguente}
\begin{itemize}
\item {fónica:gu-en}
\end{itemize}
\begin{itemize}
\item {Grp. gram.:adj.}
\end{itemize}
\begin{itemize}
\item {Proveniência:(Lat. \textunderscore arguens\textunderscore )}
\end{itemize}
Que argúe.
\section{Arguês}
\begin{itemize}
\item {Grp. gram.:m.}
\end{itemize}
Casta de uva minhota.
(Por \textunderscore areguês\textunderscore , de \textunderscore Aregos\textunderscore , n. p.?)
\section{Arguição}
\begin{itemize}
\item {fónica:gu-i}
\end{itemize}
\begin{itemize}
\item {Grp. gram.:f.}
\end{itemize}
Acto de \textunderscore arguir\textunderscore .
\section{Arguiço}
\begin{itemize}
\item {Grp. gram.:m.}
\end{itemize}
\begin{itemize}
\item {Utilização:Prov.}
\end{itemize}
\begin{itemize}
\item {Utilização:minh.}
\end{itemize}
Caruma sêca; gravanha.
(Cp. \textunderscore argueiro\textunderscore )
\section{Arguidor}
\begin{itemize}
\item {fónica:gu-i}
\end{itemize}
\begin{itemize}
\item {Grp. gram.:m.}
\end{itemize}
\begin{itemize}
\item {Grp. gram.:Adj.}
\end{itemize}
\begin{itemize}
\item {Proveniência:(De \textunderscore arguir\textunderscore )}
\end{itemize}
Aquelle que argúe.
O mesmo que \textunderscore arguente\textunderscore .
\section{Arguilheiro}
\begin{itemize}
\item {Grp. gram.:m.}
\end{itemize}
\begin{itemize}
\item {Utilização:Prov.}
\end{itemize}
\begin{itemize}
\item {Utilização:trasm.}
\end{itemize}
Diligente em serviço próprio ou alheio; fura-vidas.
\section{Arguir}
\begin{itemize}
\item {Grp. gram.:m.}
\end{itemize}
Casta de uva preta do concelho de Caminha.
\section{Arguir}
\begin{itemize}
\item {fónica:gu-ir}
\end{itemize}
\begin{itemize}
\item {Grp. gram.:v. t.}
\end{itemize}
\begin{itemize}
\item {Grp. gram.:V. i.}
\end{itemize}
\begin{itemize}
\item {Proveniência:(Lat. \textunderscore arguere\textunderscore )}
\end{itemize}
Impugnar.
Censurar.
Accusar.
Argumentar.
\section{Arguitivamente}
\begin{itemize}
\item {fónica:gu-i}
\end{itemize}
\begin{itemize}
\item {Grp. gram.:adv.}
\end{itemize}
De modo \textunderscore arguitivo\textunderscore .
Com arguição.
\section{Arguitivo}
\begin{itemize}
\item {fónica:gu-i}
\end{itemize}
\begin{itemize}
\item {Grp. gram.:adj.}
\end{itemize}
Que contém arguição.
Accusatório.
\section{Arguiz}
\begin{itemize}
\item {Grp. gram.:m.}
\end{itemize}
Casta de uva. Cf. A. A. Aguiar, \textunderscore Processos de Vin.\textunderscore , 43.
O mesmo que \textunderscore arguir\textunderscore ^1?
\section{Argumentação}
\begin{itemize}
\item {Grp. gram.:f.}
\end{itemize}
\begin{itemize}
\item {Proveniência:(Lat. \textunderscore argumentatio\textunderscore )}
\end{itemize}
Conjunto de argumentos.
Acto de argumentar.
\section{Argumentador}
\begin{itemize}
\item {Grp. gram.:m.}
\end{itemize}
\begin{itemize}
\item {Proveniência:(Lat. \textunderscore argumentator\textunderscore )}
\end{itemize}
Aquelle que argumenta.
\section{Argumentante}
\begin{itemize}
\item {Grp. gram.:adj.}
\end{itemize}
O mesmo que \textunderscore arguente\textunderscore .
\section{Argumentar}
\begin{itemize}
\item {Grp. gram.:v. i.}
\end{itemize}
\begin{itemize}
\item {Proveniência:(Lat. \textunderscore argumentari\textunderscore )}
\end{itemize}
Usar de argumentos.
Discutir.
Tirar illações ou consequências.
\section{Argumentativo}
\begin{itemize}
\item {Grp. gram.:adj.}
\end{itemize}
Que encerra argumento.
Semelhante a argumento.
\section{Argumento}
\begin{itemize}
\item {Grp. gram.:m.}
\end{itemize}
\begin{itemize}
\item {Proveniência:(Lat. \textunderscore argumentum\textunderscore )}
\end{itemize}
Raciocínio, que de uma ou mais preposições tira uma consequência.
Prova.
Exposição resumida de uma obra.
Altercação.
\section{Árgus}
\begin{itemize}
\item {Grp. gram.:m.}
\end{itemize}
Gênero de aranhas.
Espécie de lagarto.
Gênero de molluscos acéphalos.
Gênero de gallináceas.
O mesmo que \textunderscore argos\textunderscore ^2.
(Cp. \textunderscore argos\textunderscore ^2)
\section{Argutamente}
\begin{itemize}
\item {Grp. gram.:adv.}
\end{itemize}
De modo \textunderscore arguto\textunderscore .
Com argúcia.
\section{Arguto}
\begin{itemize}
\item {Grp. gram.:adj.}
\end{itemize}
\begin{itemize}
\item {Proveniência:(Lat. \textunderscore argutus\textunderscore )}
\end{itemize}
Afinado.
De som agudo.
Subtil.
Espirituoso.
\section{Argýlia}
\begin{itemize}
\item {Grp. gram.:f.}
\end{itemize}
Gênero de plantas bignoniáceas.
\section{Argynna}
\begin{itemize}
\item {Grp. gram.:f.}
\end{itemize}
Gênero de insectos lepidópteros diurnos.
\section{Argyra}
\begin{itemize}
\item {Grp. gram.:f.}
\end{itemize}
\begin{itemize}
\item {Proveniência:(Do gr. \textunderscore arguros\textunderscore )}
\end{itemize}
Gênero de insectos dípteros.
\section{Argyrânthemo}
\begin{itemize}
\item {Grp. gram.:adj.}
\end{itemize}
\begin{itemize}
\item {Proveniência:(Do gr. \textunderscore arguros\textunderscore  + \textunderscore anthema\textunderscore )}
\end{itemize}
Que tem flôres brancas como a prata.
\section{Argyráspides}
\begin{itemize}
\item {Grp. gram.:m. pl.}
\end{itemize}
\begin{itemize}
\item {Proveniência:(Do gr. \textunderscore arguros\textunderscore  + \textunderscore aspis\textunderscore )}
\end{itemize}
Soldados de Alexandre Magno, que usavam escudos brancos.
\section{Argyreia}
\begin{itemize}
\item {Grp. gram.:f.}
\end{itemize}
Gênero de plantas convolvuláceas.
\section{Argyria}
\begin{itemize}
\item {Grp. gram.:f.}
\end{itemize}
\begin{itemize}
\item {Proveniência:(Do gr. \textunderscore arguros\textunderscore , prata)}
\end{itemize}
Depósito de prata numa parte da pelle ou das mucosas, em resultado do abuso dos saes de prata.
\section{Argyríase}
\begin{itemize}
\item {Grp. gram.:f.}
\end{itemize}
O mesmo que \textunderscore argyria\textunderscore .
\section{Argýrico}
\begin{itemize}
\item {Grp. gram.:adj.}
\end{itemize}
Relativo á prata.
\section{Argyrita}
\begin{itemize}
\item {Grp. gram.:f.}
\end{itemize}
Pedra metállica, também designada por margarida de prata.
\section{Argyrite}
\begin{itemize}
\item {Grp. gram.:f.}
\end{itemize}
(V.argyrita)
\section{Argyrithrose}
\begin{itemize}
\item {Grp. gram.:f.}
\end{itemize}
Prata antinomiada, prata vermelha.
\section{Argyro}
\begin{itemize}
\item {Grp. gram.:m.}
\end{itemize}
Gênero de dípteros.
\section{Argyrocéphalo}
\begin{itemize}
\item {Grp. gram.:adj.}
\end{itemize}
\begin{itemize}
\item {Proveniência:(Do gr. \textunderscore arguros\textunderscore  + \textunderscore kephale\textunderscore )}
\end{itemize}
Que tem cabeça branca.
\section{Argyrócomo}
\begin{itemize}
\item {Grp. gram.:adj.}
\end{itemize}
\begin{itemize}
\item {Proveniência:(Do gr. \textunderscore arguros\textunderscore  + \textunderscore kome\textunderscore )}
\end{itemize}
Que tem cabelleira branca, (falando-se de cometas).
\section{Argyrócrata}
\begin{itemize}
\item {Grp. gram.:m.}
\end{itemize}
\begin{itemize}
\item {Utilização:bras}
\end{itemize}
\begin{itemize}
\item {Utilização:Neol.}
\end{itemize}
\begin{itemize}
\item {Proveniência:(Do gr. \textunderscore arguros\textunderscore  + \textunderscore kratos\textunderscore )}
\end{itemize}
Indivíduo opulento, argentário.
\section{Argyrodendro}
\begin{itemize}
\item {Grp. gram.:m.}
\end{itemize}
\begin{itemize}
\item {Proveniência:(Do gr. \textunderscore arguros\textunderscore  + \textunderscore dendron\textunderscore )}
\end{itemize}
Nome de várias plantas.
\section{Argyrólitho}
\begin{itemize}
\item {Grp. gram.:m.}
\end{itemize}
\begin{itemize}
\item {Proveniência:(Do gr. \textunderscore arguros\textunderscore  + \textunderscore lithos\textunderscore )}
\end{itemize}
Pedra de prata, ou que parece prata.
Pedra preciosa, descrita pelos antigos, mas cuja natureza se não sabe ao certo, suppondo-se que fôsse uma aventurina oriental.
\section{Argyropeia}
\begin{itemize}
\item {Grp. gram.:f.}
\end{itemize}
\begin{itemize}
\item {Proveniência:(Do gr. \textunderscore arguros\textunderscore  + \textunderscore poiein\textunderscore )}
\end{itemize}
Supposta arte de fazer prata.
\section{Argyrophillo}
\begin{itemize}
\item {Grp. gram.:adj.}
\end{itemize}
\begin{itemize}
\item {Proveniência:(Do gr. \textunderscore arguros\textunderscore  + \textunderscore phullon\textunderscore )}
\end{itemize}
Que tem fôlhas brancas como a prata.
\section{Argyróphoro}
\begin{itemize}
\item {Grp. gram.:m.}
\end{itemize}
\begin{itemize}
\item {Utilização:Med.}
\end{itemize}
\begin{itemize}
\item {Utilização:ant.}
\end{itemize}
\begin{itemize}
\item {Proveniência:(Do gr. \textunderscore arguros\textunderscore  + \textunderscore phoros\textunderscore )}
\end{itemize}
Espécie de antídoto.
\section{Argyrose}
\begin{itemize}
\item {Grp. gram.:f.}
\end{itemize}
\begin{itemize}
\item {Proveniência:(Do gr. \textunderscore arguros\textunderscore , prata)}
\end{itemize}
O mais commum dos minérios de prata, malleável, pardo e de brilho metállico.
Prata sulfurada.
O mesmo que \textunderscore argyria\textunderscore .
\section{Argyrostigmado}
\begin{itemize}
\item {Grp. gram.:adj.}
\end{itemize}
\begin{itemize}
\item {Utilização:Bot.}
\end{itemize}
\begin{itemize}
\item {Proveniência:(Do gr. \textunderscore arguros\textunderscore  + \textunderscore stigma\textunderscore )}
\end{itemize}
Que tem manchas brancas.
\section{Argyróstomo}
\begin{itemize}
\item {Grp. gram.:adj.}
\end{itemize}
\begin{itemize}
\item {Utilização:Zool.}
\end{itemize}
\begin{itemize}
\item {Proveniência:(Do gr. \textunderscore arguros\textunderscore  + \textunderscore stoma\textunderscore )}
\end{itemize}
Que tem a bôca da côr da prata.
\section{Argyrothyrso}
\begin{itemize}
\item {Grp. gram.:m.}
\end{itemize}
Combinação natural do sulfureto de prata com o antimónio.
Antimónio sulfurado.
Prata vermelha.
\section{Arhizotónico}
\begin{itemize}
\item {fónica:ri}
\end{itemize}
\begin{itemize}
\item {Grp. gram.:adj.}
\end{itemize}
\begin{itemize}
\item {Utilização:Philol.}
\end{itemize}
\begin{itemize}
\item {Proveniência:(De \textunderscore a\textunderscore  priv. + \textunderscore rhizotónico\textunderscore )}
\end{itemize}
Diz-se das fórmas verbaes, cuja sýllaba tónica está na terminação ou desinência, como em \textunderscore copiamos\textunderscore , \textunderscore copiarei\textunderscore .
\section{Ária}
\begin{itemize}
\item {Grp. gram.:f.}
\end{itemize}
\begin{itemize}
\item {Proveniência:(It. \textunderscore ária\textunderscore )}
\end{itemize}
Peça de música para uma só voz.
Cantiga.
\section{Ária}
\begin{itemize}
\item {Grp. gram.:f.}
\end{itemize}
\begin{itemize}
\item {Utilização:Prov.}
\end{itemize}
Doairo, bom aspecto.
Physionomia. Cp. Camillo, \textunderscore Cavar em Ruinas\textunderscore , 224.
\section{...ária}
\begin{itemize}
\item {Grp. gram.:suf.}
\end{itemize}
(fem. de \textunderscore ...ário\textunderscore )
\section{...aría}
\begin{itemize}
\item {Grp. gram.:suf. f.}
\end{itemize}
De quantidade, repetição, etc.: \textunderscore cavallaria\textunderscore ; \textunderscore infantaria\textunderscore ; \textunderscore parçaria\textunderscore .
\section{Aríaco}
\begin{itemize}
\item {Grp. gram.:adj.}
\end{itemize}
O mesmo que \textunderscore ariano\textunderscore ^2.
\section{Ariádna}
\begin{itemize}
\item {Grp. gram.:f.}
\end{itemize}
\begin{itemize}
\item {Proveniência:(Do gr. \textunderscore Ariadne\textunderscore , n. p.)}
\end{itemize}
Espécie de aranha.
Uma das estrêllas da constellação da \textunderscore Corôa\textunderscore .
\section{Ariádne}
\begin{itemize}
\item {Grp. gram.:f.}
\end{itemize}
\begin{itemize}
\item {Proveniência:(Do gr. \textunderscore Ariadne\textunderscore , n. p.)}
\end{itemize}
Espécie de aranha.
Uma das estrêllas da constellação da \textunderscore Corôa\textunderscore .
\section{Arianismo}
\begin{itemize}
\item {Grp. gram.:m.}
\end{itemize}
Seita religiosa dos Arianos.
\section{Ariano}
\begin{itemize}
\item {Grp. gram.:m.}
\end{itemize}
\begin{itemize}
\item {Proveniência:(De \textunderscore Ario\textunderscore , n. p.)}
\end{itemize}
Sectário do heresiarcha Ario, que negava a consubstancialidade do Padre com o Filho, no dogma da Trindade.
\section{Ariano}
\begin{itemize}
\item {Grp. gram.:adj.}
\end{itemize}
\begin{itemize}
\item {Grp. gram.:M.}
\end{itemize}
Relativo aos Árias.
Língua dos Árias.
\section{Árias}
\begin{itemize}
\item {Grp. gram.:m. pl.}
\end{itemize}
Povos antiquíssimos, que se estabeleceram no Industão e iniciaram a civilização indo-europeia.
\section{Ariaucanes}
\begin{itemize}
\item {Grp. gram.:m. pl.}
\end{itemize}
Indígenas do Brasil, nas margens do Madeira.
\section{Aricar}
\begin{itemize}
\item {Grp. gram.:v. t.}
\end{itemize}
\begin{itemize}
\item {Utilização:Prov.}
\end{itemize}
\begin{itemize}
\item {Proveniência:(De \textunderscore arar\textunderscore )}
\end{itemize}
Lavrar de leve, para arrancar ervas damninhas.
\section{Arícia}
\begin{itemize}
\item {Grp. gram.:f.}
\end{itemize}
Gênero de insectos aricíneos.
\section{Aricíneos}
\begin{itemize}
\item {Grp. gram.:m. pl.}
\end{itemize}
\begin{itemize}
\item {Proveniência:(De \textunderscore aricia\textunderscore )}
\end{itemize}
Tríbo de insectos dípteros.
\section{Árico}
\begin{itemize}
\item {Grp. gram.:adj.}
\end{itemize}
O mesmo que \textunderscore ariano\textunderscore ^2.
\section{Aricori}
\begin{itemize}
\item {Grp. gram.:f.}
\end{itemize}
Árvore brasileira, da fam. das palmeiras.
\section{Aricunanes}
\begin{itemize}
\item {Grp. gram.:m. pl.}
\end{itemize}
Tríbo de Índios, nas margens do Madeira.
\section{Aridez}
\begin{itemize}
\item {Grp. gram.:f.}
\end{itemize}
\begin{itemize}
\item {Proveniência:(Lat. \textunderscore ariditas\textunderscore )}
\end{itemize}
Qualidade do que é árido.
Secura.
Esterilidade.
\section{Árido}
\begin{itemize}
\item {Grp. gram.:adj.}
\end{itemize}
\begin{itemize}
\item {Proveniência:(Lat. \textunderscore aridus\textunderscore )}
\end{itemize}
Estéril.
Sêco: \textunderscore campo árido\textunderscore .
Desagradável; fastidioso: \textunderscore leituras áridas\textunderscore .
\section{Áriès}
\begin{itemize}
\item {Grp. gram.:m.}
\end{itemize}
\begin{itemize}
\item {Proveniência:(Lat. \textunderscore aries\textunderscore )}
\end{itemize}
O mesmo que \textunderscore Carneiro\textunderscore ^1, uma das constellações do Zodíaco.
\section{Arieta}
\begin{itemize}
\item {fónica:ê}
\end{itemize}
\begin{itemize}
\item {Grp. gram.:f.}
\end{itemize}
Pequena ária.
\section{Arietária}
\begin{itemize}
\item {Grp. gram.:f.}
\end{itemize}
\begin{itemize}
\item {Proveniência:(De \textunderscore aríete\textunderscore )}
\end{itemize}
O mesmo que \textunderscore saxífraga\textunderscore .
\section{Arietário}
\begin{itemize}
\item {Grp. gram.:adj.}
\end{itemize}
\begin{itemize}
\item {Proveniência:(Lat. \textunderscore arietarius\textunderscore )}
\end{itemize}
Relativo ou semelhante ao \textunderscore aríete\textunderscore :«\textunderscore outras testugens arietárias tinham\textunderscore ».
\textunderscore Viriato Tragico\textunderscore , II, 17.
\section{Aríete}
\begin{itemize}
\item {Grp. gram.:m.}
\end{itemize}
\begin{itemize}
\item {Proveniência:(Do lat. \textunderscore aries\textunderscore , \textunderscore arietis\textunderscore )}
\end{itemize}
Antiga máquina de guerra.
\section{Arietino}
\begin{itemize}
\item {Grp. gram.:adj.}
\end{itemize}
Pertencente ao aríete.
Relativo a carneiro.
\section{Arife}
\begin{itemize}
\item {Grp. gram.:m.}
\end{itemize}
\begin{itemize}
\item {Utilização:Gír.}
\end{itemize}
O mesmo que \textunderscore tesoira\textunderscore .
\section{Arilado}
\begin{itemize}
\item {Grp. gram.:adj.}
\end{itemize}
Que tem arilo.
\section{Arilário}
\begin{itemize}
\item {Grp. gram.:adj.}
\end{itemize}
Semelhante ao arilo.
\section{Arilho}
\begin{itemize}
\item {Grp. gram.:m.}
\end{itemize}
(V.arilo)
\section{Arilho}
\begin{itemize}
\item {Grp. gram.:m.}
\end{itemize}
\begin{itemize}
\item {Utilização:T. de Setúbal}
\end{itemize}
\begin{itemize}
\item {Proveniência:(De \textunderscore ar\textunderscore )}
\end{itemize}
Vento fraco e frio.
\section{Arillado}
\begin{itemize}
\item {Grp. gram.:adj.}
\end{itemize}
Que tem arillo.
\section{Arillário}
\begin{itemize}
\item {Grp. gram.:adj.}
\end{itemize}
Semelhante ao arillo.
\section{Arillo}
\begin{itemize}
\item {Grp. gram.:m.}
\end{itemize}
Grão sêcco da uva.
Graínha.
Appêndice do funículo, que cobre certas sementes.
(B. lat. \textunderscore arillus\textunderscore )
\section{Arilo}
\begin{itemize}
\item {Grp. gram.:m.}
\end{itemize}
Grão sêcco da uva.
Graínha.
Appêndice do funículo, que cobre certas sementes.
(B. lat. \textunderscore arillus\textunderscore )
\section{Arimbo}
\begin{itemize}
\item {Grp. gram.:m.}
\end{itemize}
O mesmo que \textunderscore arimo\textunderscore .
\section{Arimo}
\begin{itemize}
\item {Grp. gram.:m.}
\end{itemize}
Quinta para cultura agrícola, em Angola.
\section{Arimónio}
\begin{itemize}
\item {Grp. gram.:m.}
\end{itemize}
\begin{itemize}
\item {Utilização:Ant.}
\end{itemize}
O mesmo que \textunderscore arimono\textunderscore .
\section{Arimono}
\begin{itemize}
\item {Grp. gram.:m.}
\end{itemize}
\begin{itemize}
\item {Utilização:Ant.}
\end{itemize}
Espécie de cadeirinha.
\section{Arinas}
\begin{itemize}
\item {Grp. gram.:m. pl.}
\end{itemize}
Indígenas da Guiana brasileira.
\section{Aringa}
\begin{itemize}
\item {Grp. gram.:f.}
\end{itemize}
\begin{itemize}
\item {Proveniência:(T. cafreal)}
\end{itemize}
Campo fortificado, entre os indígenas da África.
\section{Aringão}
\begin{itemize}
\item {Grp. gram.:m.}
\end{itemize}
(V. \textunderscore artesão\textunderscore ^1)
\section{Arinos}
\begin{itemize}
\item {Grp. gram.:m. pl.}
\end{itemize}
Antiga nação de Índios do Brasil, que dominavam nas margens do rio do mesmo nome.
\section{Arinque}
\begin{itemize}
\item {Grp. gram.:m.}
\end{itemize}
\begin{itemize}
\item {Utilização:Náut.}
\end{itemize}
\begin{itemize}
\item {Grp. gram.:Pl.}
\end{itemize}
\begin{itemize}
\item {Utilização:T. de Aveiro}
\end{itemize}
Cabo, que prende a bóia á âncora.
Fluctuadores ou bóias nas calas da rede de cercar e alar.
\section{Arinta}
\begin{itemize}
\item {Grp. gram.:f.}
\end{itemize}
O mesmo que \textunderscore arinto\textunderscore .
\section{Arinto}
\begin{itemize}
\item {Grp. gram.:m.}
\end{itemize}
Espécie de uva branca.
Também se diz arinto o vinho produzido por essa uva.
\section{Arinto-preto}
\begin{itemize}
\item {Grp. gram.:m.}
\end{itemize}
Casta de uva da região do Doiro.
\section{...ário}
\begin{itemize}
\item {Grp. gram.:suf. m.  e  adj.}
\end{itemize}
\begin{itemize}
\item {Proveniência:(Do lat. \textunderscore ...arius\textunderscore )}
\end{itemize}
(para designar qualidade, profissão, collectividade, etc.)
\section{Ariolomancia}
\begin{itemize}
\item {Grp. gram.:f.}
\end{itemize}
Supposta arte de adivinhar, por meio de ídolos. Cf. Castilho, \textunderscore Fastos\textunderscore , III, 314.
\section{Ariosca}
\begin{itemize}
\item {Grp. gram.:f.}
\end{itemize}
(V.arriosca)
\section{Aripar}
\begin{itemize}
\item {fónica:ri}
\end{itemize}
\begin{itemize}
\item {Grp. gram.:v. i.}
\end{itemize}
Surribar a terra das ostreiras, para apanhar pérolas.
(Talvez do lat. \textunderscore ripa\textunderscore )
\section{Aripeiro}
\begin{itemize}
\item {fónica:ri}
\end{itemize}
\begin{itemize}
\item {Grp. gram.:m.}
\end{itemize}
\begin{itemize}
\item {Proveniência:(De \textunderscore aripar\textunderscore )}
\end{itemize}
Aquelle que aripa.
\section{Ariperana}
\begin{itemize}
\item {Grp. gram.:f.}
\end{itemize}
Árvore brasileira.
\section{Aripo}
\begin{itemize}
\item {fónica:ri}
\end{itemize}
\begin{itemize}
\item {Grp. gram.:m.}
\end{itemize}
Acto de \textunderscore aripar\textunderscore .
\section{Ariquenas}
\begin{itemize}
\item {Grp. gram.:m. pl.}
\end{itemize}
Indígenas do Brasil, nas margens do Madeira.
\section{Ariramba}
\begin{itemize}
\item {Grp. gram.:f.}
\end{itemize}
Ave ribeirinha do norte do Brasil.
\section{Ariranha}
\begin{itemize}
\item {Grp. gram.:f.}
\end{itemize}
\begin{itemize}
\item {Utilização:Bras}
\end{itemize}
Mammífero, semelhante á lontra.
\section{Ariri}
\begin{itemize}
\item {Grp. gram.:m.}
\end{itemize}
\begin{itemize}
\item {Utilização:Bras}
\end{itemize}
Palmeira medicinal.
\section{Arísaro}
\begin{itemize}
\item {Grp. gram.:m.}
\end{itemize}
\begin{itemize}
\item {Proveniência:(Do gr. \textunderscore aris\textunderscore  + \textunderscore aron\textunderscore )}
\end{itemize}
Planta, da fam. das aroídeas.
\section{Arisca}
\begin{itemize}
\item {Grp. gram.:f.}
\end{itemize}
O mesmo que \textunderscore areísca\textunderscore .
\section{Ariscar}
\begin{itemize}
\item {Grp. gram.:v. t.}
\end{itemize}
\begin{itemize}
\item {Grp. gram.:V. i.}
\end{itemize}
\begin{itemize}
\item {Proveniência:(De \textunderscore arisco\textunderscore )}
\end{itemize}
Recusar:«\textunderscore ignora os numes que o favor lhe ariscam\textunderscore ». Filinto, VI, 168.
Sêr arisco:«\textunderscore de mim, Chloé, te ariscas\textunderscore ». Id., X, 264.
\section{Arisco}
\begin{itemize}
\item {Grp. gram.:adj.}
\end{itemize}
\begin{itemize}
\item {Grp. gram.:Pl.}
\end{itemize}
\begin{itemize}
\item {Utilização:Bras}
\end{itemize}
Arenoso.
Áspero; bravio.
Esquivo: \textunderscore donzella arisca\textunderscore .
Certos terrenos, propícios á mandioca.
(Por \textunderscore areísco\textunderscore , de \textunderscore areia\textunderscore )
\section{Arisco}
\begin{itemize}
\item {Grp. gram.:m.}
\end{itemize}
\begin{itemize}
\item {Utilização:T. da Bairrada}
\end{itemize}
Tordo.
Pisco.
\section{Arismética}
\begin{itemize}
\item {Grp. gram.:f.}
\end{itemize}
\begin{itemize}
\item {Utilização:Ant.}
\end{itemize}
O mesmo que \textunderscore arithmética\textunderscore .
(Provn. \textunderscore arismetica\textunderscore )
\section{Aristado}
\begin{itemize}
\item {Grp. gram.:adj.}
\end{itemize}
\begin{itemize}
\item {Proveniência:(Do lat. \textunderscore arista\textunderscore )}
\end{itemize}
Que tem aresta.
\section{Aristarcho}
\begin{itemize}
\item {fónica:co}
\end{itemize}
\begin{itemize}
\item {Grp. gram.:m.}
\end{itemize}
\begin{itemize}
\item {Proveniência:(Lat. \textunderscore Aristarchus\textunderscore , n. p.)}
\end{itemize}
Censor severo; crítico.
\section{Aristarco}
\begin{itemize}
\item {Grp. gram.:m.}
\end{itemize}
\begin{itemize}
\item {Proveniência:(Lat. \textunderscore Aristarchus\textunderscore , n. p.)}
\end{itemize}
Censor severo; crítico.
\section{Aristária}
\begin{itemize}
\item {Grp. gram.:f.}
\end{itemize}
\begin{itemize}
\item {Proveniência:(Do lat. \textunderscore arista\textunderscore )}
\end{itemize}
Gênero de algas.
Gênero de gramíneas.
\section{Aristiforme}
\begin{itemize}
\item {Grp. gram.:adj.}
\end{itemize}
\begin{itemize}
\item {Proveniência:(Do lat. \textunderscore arista\textunderscore  + \textunderscore forma\textunderscore )}
\end{itemize}
Que tem fórma de aresta.
\section{Aristocracia}
\begin{itemize}
\item {Grp. gram.:f.}
\end{itemize}
\begin{itemize}
\item {Proveniência:(Do gr. \textunderscore aristokrateia\textunderscore )}
\end{itemize}
Govêrno, exercido por pessôas nobres.
Fidalguia; nobreza.
Superioridade.
\section{Aristocráta}
\begin{itemize}
\item {Grp. gram.:m.}
\end{itemize}
\begin{itemize}
\item {Grp. gram.:Adj.}
\end{itemize}
O indivíduo que pertence á aristocracia.
Fidalgo.
(V.aristocrático)
\section{Aristócrata}
\begin{itemize}
\item {Grp. gram.:m.}
\end{itemize}
\begin{itemize}
\item {Grp. gram.:Adj.}
\end{itemize}
O indivíduo que pertence á aristocracia.
Fidalgo.
(V.aristocrático)
\section{Aristocraticamente}
\begin{itemize}
\item {Grp. gram.:adv.}
\end{itemize}
De modo \textunderscore aristocrático\textunderscore .
Á maneira de fidalgo.
\section{Aristocrático}
\begin{itemize}
\item {Grp. gram.:adj.}
\end{itemize}
\begin{itemize}
\item {Proveniência:(Gr. \textunderscore aristokratikos\textunderscore )}
\end{itemize}
Relativo á aristocracia.
Nobre.
\section{Aristocratismo}
\begin{itemize}
\item {Grp. gram.:m.}
\end{itemize}
Maneiras, princípios, de aristocrata.
\section{Aristocratizar}
\begin{itemize}
\item {Grp. gram.:v. t.}
\end{itemize}
Tornar aristocrático.
Dar fóros de aristocrata a.
\section{Aristodemocracia}
\begin{itemize}
\item {Grp. gram.:f.}
\end{itemize}
\begin{itemize}
\item {Proveniência:(Do gr. \textunderscore aristos\textunderscore  + \textunderscore democrateia\textunderscore )}
\end{itemize}
Govêrno, em que a nobreza e o povo tomam parte.
\section{Aristodemocrata}
\begin{itemize}
\item {Grp. gram.:m.}
\end{itemize}
Partidário da aristodemocracia.
\section{Aristodemócrata}
\begin{itemize}
\item {Grp. gram.:m.}
\end{itemize}
Partidário da aristodemocracia.
\section{Aristodemocrático}
\begin{itemize}
\item {Grp. gram.:adj.}
\end{itemize}
Relativo á aristodemocracia.
\section{Aristofanesco}
\begin{itemize}
\item {Grp. gram.:adj.}
\end{itemize}
O mesmo que \textunderscore aristofânico\textunderscore .
\section{Aristofaniano}
\begin{itemize}
\item {Grp. gram.:adj.}
\end{itemize}
Diz-se dos versos de sete pés e meio, por terem sido muito usados por Aristófanes.
\section{Aristofânico}
\begin{itemize}
\item {Grp. gram.:adj.}
\end{itemize}
Relativo a Aristófanes.
Feito, segundo o gôsto ou o estylo de Aristófanes.
\section{Aristol}
\begin{itemize}
\item {Grp. gram.:m.}
\end{itemize}
Medicamento, obtido pela acção do iodo em solução alcalina, ou pelo iodeto de potássio, em presença dos hipochloritos sobre o thymol.
\section{Aristolóchia}
\begin{itemize}
\item {fónica:qui}
\end{itemize}
\begin{itemize}
\item {Grp. gram.:f.}
\end{itemize}
\begin{itemize}
\item {Proveniência:(Lat. \textunderscore aristolochia\textunderscore )}
\end{itemize}
Planta dicotyledónea, medicinal.
\section{Aristolochiáceas}
\begin{itemize}
\item {fónica:qui}
\end{itemize}
\begin{itemize}
\item {Grp. gram.:f. pl.}
\end{itemize}
Familia de plantas, que tem por typo a \textunderscore aristolóchia\textunderscore .
\section{Aristolóquia}
\begin{itemize}
\item {Grp. gram.:f.}
\end{itemize}
\begin{itemize}
\item {Proveniência:(Lat. \textunderscore aristolochia\textunderscore )}
\end{itemize}
Planta dicotyledónea, medicinal.
\section{Aristoloquiáceas}
\begin{itemize}
\item {Grp. gram.:f. pl.}
\end{itemize}
Familia de plantas, que tem por typo a \textunderscore aristolóquia\textunderscore .
\section{Aríston}
\begin{itemize}
\item {Grp. gram.:m.}
\end{itemize}
\begin{itemize}
\item {Utilização:Mús.}
\end{itemize}
Moderno instrumento de manivela, espécie de realejo.
\section{Aristophanesco}
\begin{itemize}
\item {Grp. gram.:adj.}
\end{itemize}
O mesmo que \textunderscore aristophânico\textunderscore .
\section{Aristophaniano}
\begin{itemize}
\item {Grp. gram.:adj.}
\end{itemize}
Diz-se dos versos de sete pés e meio, por terem sido muito usados por Aristóphanes.
\section{Aristophânico}
\begin{itemize}
\item {Grp. gram.:adj.}
\end{itemize}
Relativo a Aristóphanes.
Feito, segundo o gôsto ou o estylo de Aristóphanes.
\section{Aristoso}
\begin{itemize}
\item {Grp. gram.:adj.}
\end{itemize}
\begin{itemize}
\item {Proveniência:(Lat. \textunderscore aristosus\textunderscore )}
\end{itemize}
O mesmo que \textunderscore aristado\textunderscore .
\section{Aristotélia}
\begin{itemize}
\item {Grp. gram.:f.}
\end{itemize}
\begin{itemize}
\item {Proveniência:(De \textunderscore Aristóteles\textunderscore , n. p.)}
\end{itemize}
Gênero de plantas camelliáceas.
\section{Aristotélico}
\begin{itemize}
\item {Grp. gram.:adj.}
\end{itemize}
Relativo a Aristóteles ou á sua doutrina.
\section{Aristotelismo}
\begin{itemize}
\item {Grp. gram.:m.}
\end{itemize}
Doutrina de Aristóteles.
\section{Aritá}
\begin{itemize}
\item {Grp. gram.:m.}
\end{itemize}
Árvore brasileira, de que os Índios fazem apparelhos de mesa.
\section{Arithmancia}
\begin{itemize}
\item {Grp. gram.:f.}
\end{itemize}
O mesmo que \textunderscore arithmomancia\textunderscore .
\section{Arithmética}
\begin{itemize}
\item {Grp. gram.:f.}
\end{itemize}
\begin{itemize}
\item {Proveniência:(Gr. \textunderscore arithmetike\textunderscore )}
\end{itemize}
Sciência dos números.
Tratado desta sciência.
\section{Arithmeticamente}
\begin{itemize}
\item {fónica:mé}
\end{itemize}
\begin{itemize}
\item {Grp. gram.:adv.}
\end{itemize}
Segundo as regras da Arithmética.
\section{Arithmético}
\begin{itemize}
\item {Grp. gram.:adj.}
\end{itemize}
\begin{itemize}
\item {Grp. gram.:M.}
\end{itemize}
\begin{itemize}
\item {Proveniência:(Lat. \textunderscore arithmeticus\textunderscore )}
\end{itemize}
Relativo á Arithmética.
Fundado na Arithmética.
Aquelle que a ensina ou é nella versado.
\section{Arithmographia}
\begin{itemize}
\item {Grp. gram.:f.}
\end{itemize}
Arte de exprimir por sinaes convencionaes as quantidades, cuja composição é conhecida.
(Cp. \textunderscore arithmógrapho\textunderscore )
\section{Arithmógrapho}
\begin{itemize}
\item {Grp. gram.:m.}
\end{itemize}
\begin{itemize}
\item {Proveniência:(Do gr. \textunderscore arithmos\textunderscore  + \textunderscore graphein\textunderscore )}
\end{itemize}
Instrumento, para operar mecanicamente operações arithméticas.
\section{Arithmologia}
\begin{itemize}
\item {Grp. gram.:f.}
\end{itemize}
\begin{itemize}
\item {Proveniência:(Do gr. \textunderscore arithmos\textunderscore  + \textunderscore logos\textunderscore )}
\end{itemize}
Sciência, que tem por objecto os numeros, a medição das grandezas em geral.
\section{Arithmomancia}
\begin{itemize}
\item {Grp. gram.:f.}
\end{itemize}
\begin{itemize}
\item {Proveniência:(Do gr. \textunderscore arithmos\textunderscore  + \textunderscore manteia\textunderscore )}
\end{itemize}
Arte de adivinhar pelos números.
\section{Arithmomântico}
\begin{itemize}
\item {Grp. gram.:adj.}
\end{itemize}
Relativo á Arithmomancia.
\section{Arithmometria}
\begin{itemize}
\item {Grp. gram.:f.}
\end{itemize}
Arte de traçar divisões logaríthmicas sôbre o arithmómetro.
\section{Arithmométrico}
\begin{itemize}
\item {Grp. gram.:adj.}
\end{itemize}
Relativo á Arithmometria.
\section{Arithmómetro}
\begin{itemize}
\item {Grp. gram.:m.}
\end{itemize}
\begin{itemize}
\item {Proveniência:(Do gr. \textunderscore arithmos\textunderscore  + \textunderscore metron\textunderscore )}
\end{itemize}
Instrumento ou máquina, em que estão traçadas as divisões logaríthmicas e que serve para cálculos arithméticos.
\section{Arithmoplanímetro}
\begin{itemize}
\item {Grp. gram.:m.}
\end{itemize}
Instrumento, inventado por Lalanne, para resolver os mais difíceis cálculos geométricos e trigonométricos.
\section{Aritmancia}
\begin{itemize}
\item {Grp. gram.:f.}
\end{itemize}
O mesmo que \textunderscore aritmomancia\textunderscore .
\section{Aritmética}
\begin{itemize}
\item {Grp. gram.:f.}
\end{itemize}
\begin{itemize}
\item {Proveniência:(Gr. \textunderscore arithmetike\textunderscore )}
\end{itemize}
Sciência dos números.
Tratado desta sciência.
\section{Aritmografia}
\begin{itemize}
\item {Grp. gram.:f.}
\end{itemize}
Arte de exprimir por sinaes convencionaes as quantidades, cuja composição é conhecida.
(Cp. \textunderscore arithmógrapho\textunderscore )
\section{Aritmógrafo}
\begin{itemize}
\item {Grp. gram.:m.}
\end{itemize}
\begin{itemize}
\item {Proveniência:(Do gr. \textunderscore arithmos\textunderscore  + \textunderscore graphein\textunderscore )}
\end{itemize}
Instrumento, para operar mecanicamente operações arithméticas.
\section{Aritmologia}
\begin{itemize}
\item {Grp. gram.:f.}
\end{itemize}
\begin{itemize}
\item {Proveniência:(Do gr. \textunderscore arithmos\textunderscore  + \textunderscore logos\textunderscore )}
\end{itemize}
Sciência, que tem por objecto os numeros, a medição das grandezas em geral.
\section{Aritmomancia}
\begin{itemize}
\item {Grp. gram.:f.}
\end{itemize}
\begin{itemize}
\item {Proveniência:(Do gr. \textunderscore arithmos\textunderscore  + \textunderscore manteia\textunderscore )}
\end{itemize}
Arte de adivinhar pelos números.
\section{Aritmomântico}
\begin{itemize}
\item {Grp. gram.:adj.}
\end{itemize}
Relativo á Aritmomancia.
\section{Aritmometria}
\begin{itemize}
\item {Grp. gram.:f.}
\end{itemize}
Arte de traçar divisões logaríthmicas sôbre o aritmómetro.
\section{Aritmométrico}
\begin{itemize}
\item {Grp. gram.:adj.}
\end{itemize}
Relativo á Aritmometria.
\section{Aritmómetro}
\begin{itemize}
\item {Grp. gram.:m.}
\end{itemize}
\begin{itemize}
\item {Proveniência:(Do gr. \textunderscore arithmos\textunderscore  + \textunderscore metron\textunderscore )}
\end{itemize}
Instrumento ou máquina, em que estão traçadas as divisões logaríthmicas e que serve para cálculos arithméticos.
\section{Aritmoplanímetro}
\begin{itemize}
\item {Grp. gram.:m.}
\end{itemize}
Instrumento, inventado por Lalanne, para resolver os mais difíceis cálculos geométricos e trigonométricos.
\section{Arjão}
\begin{itemize}
\item {Grp. gram.:m.}
\end{itemize}
\begin{itemize}
\item {Utilização:Prov.}
\end{itemize}
\begin{itemize}
\item {Utilização:minh.}
\end{itemize}
Pau, em que se empa a videira ou que ampara ervilhas, feijões, etc.
Casta de uva.
\section{Arjoada}
\begin{itemize}
\item {Grp. gram.:f.}
\end{itemize}
Videiras, empadas de arjão.
\section{Arjoar}
\begin{itemize}
\item {Grp. gram.:v. t.}
\end{itemize}
Amparar com arjão.
\section{Arjoona}
\begin{itemize}
\item {Grp. gram.:f.}
\end{itemize}
Gênero de santaláceas.
\section{Arjunça}
\begin{itemize}
\item {Grp. gram.:f.}
\end{itemize}
Suco glutinoso, extrahído, por incisão de uma espécie de cardo, (\textunderscore carlinga gommifera\textunderscore ), suco que os passarinheiros empregam na feitura de viscos para caça de aves.
\section{Arlequim}
\begin{itemize}
\item {Grp. gram.:m.}
\end{itemize}
\begin{itemize}
\item {Utilização:Fig.}
\end{itemize}
\begin{itemize}
\item {Proveniência:(Do it. \textunderscore arlecchino\textunderscore )}
\end{itemize}
Personagem que, na antiga comédia italiana, usava traje de várias cores.
Saltimbanco.
Palhaço.
Homem, que muda facilmente de opinião; catavento.
\section{Arlequinada}
\begin{itemize}
\item {Grp. gram.:f.}
\end{itemize}
Acção, própria de arlequim.
Modos de arlequim.
Festa de arlequins.
Procedimento ridículo.
\section{Arlequíneo}
\begin{itemize}
\item {Grp. gram.:adj.}
\end{itemize}
Diz-se dos animaes de cores variadas.
\section{Arlesiano}
\begin{itemize}
\item {Grp. gram.:adj.}
\end{itemize}
\begin{itemize}
\item {Grp. gram.:M.}
\end{itemize}
Relativo a Arles.
Habitante de Arles.
\section{Arlota}
\begin{itemize}
\item {Grp. gram.:f.}
\end{itemize}
\begin{itemize}
\item {Utilização:Ant.}
\end{itemize}
Mulher vagabunda. Cf. \textunderscore Cancion. da Vaticana\textunderscore .
\section{Arlotão}
\begin{itemize}
\item {Grp. gram.:m.}
\end{itemize}
\begin{itemize}
\item {Utilização:Ant.}
\end{itemize}
Homem vagabundo, vadío.
\section{Arlotia}
\begin{itemize}
\item {Grp. gram.:f.}
\end{itemize}
\begin{itemize}
\item {Utilização:Ant.}
\end{itemize}
Qualidade de arlota ou arlotão.
\section{Arma}
\begin{itemize}
\item {Grp. gram.:f.}
\end{itemize}
\begin{itemize}
\item {Grp. gram.:Pl.}
\end{itemize}
\begin{itemize}
\item {Utilização:Ant.}
\end{itemize}
\begin{itemize}
\item {Proveniência:(Lat. \textunderscore arma\textunderscore )}
\end{itemize}
Instrumento offensivo ou defensivo.
Qualquer meio de aggressão.
Classe de tropa: \textunderscore a arma de artilharia\textunderscore .
Cornos. Distintivo de nobreza.
\textunderscore Armas ligeiras\textunderscore , roupas leves:«\textunderscore achey a rapariga em armas ligeiras\textunderscore ». \textunderscore Eufrosina\textunderscore , 149.
\section{Armação}
\begin{itemize}
\item {Grp. gram.:f.}
\end{itemize}
\begin{itemize}
\item {Utilização:Prov.}
\end{itemize}
\begin{itemize}
\item {Utilização:alent.}
\end{itemize}
Acto ou effeito de \textunderscore armar\textunderscore .
Cornos.
Madeiramento de um edifício.
Contextura.
Apparelhos náuticos.
Petrechos de pesca.
Peças fixas de madeira em loja commercial, para têr ou expor as mercadorias.
Guarnição de paredes, arcos, essas, etc.
Equipamento do navio.
Armadilha, para caçar pombos bravos.
Apparelho permanente para a pesca do atum e da sardinha.
\section{Armada}
\begin{itemize}
\item {Grp. gram.:f.}
\end{itemize}
\begin{itemize}
\item {Utilização:Ant.}
\end{itemize}
\begin{itemize}
\item {Utilização:Ant.}
\end{itemize}
\begin{itemize}
\item {Proveniência:(Lat. \textunderscore armata\textunderscore )}
\end{itemize}
Esquadra de uma nação.
Conjunto de navios de guerra, pertencentes a uma potência.
Exército naval.
Exército de terra.
Cilada.
Armadilha.
\section{Armadilha}
\begin{itemize}
\item {Grp. gram.:f.}
\end{itemize}
\begin{itemize}
\item {Proveniência:(De \textunderscore armar\textunderscore )}
\end{itemize}
Qualquer artifício para apanhar caça.
Lôgro astucioso; cilada.
\section{Armadilho}
\begin{itemize}
\item {Grp. gram.:m.}
\end{itemize}
\begin{itemize}
\item {Proveniência:(De \textunderscore armado\textunderscore )}
\end{itemize}
Gênero de mammiferos, da ordem dos tatus.
Gênero de crustáceos.
\section{Armado}
\begin{itemize}
\item {Grp. gram.:adj.}
\end{itemize}
\begin{itemize}
\item {Utilização:Heráld.}
\end{itemize}
Munido de arma.
Apparelhado.
Acautelado.
Diz-se do animal, que, no escudo, apresenta dentes e garras com esmalte.
\section{Armadoiras}
\begin{itemize}
\item {Grp. gram.:f. pl.}
\end{itemize}
\begin{itemize}
\item {Proveniência:(De \textunderscore armar\textunderscore )}
\end{itemize}
Fasquias no costado do navio em construcção, para fixar as escoras que o suspendem.
\section{Armador}
\begin{itemize}
\item {Grp. gram.:m.}
\end{itemize}
Aquelle que arma.
Proprietário de navios mercantes.
Aquelle que adorna igrejas, casas, etc.
\section{Armadouras}
\begin{itemize}
\item {Grp. gram.:f. pl.}
\end{itemize}
\begin{itemize}
\item {Proveniência:(De \textunderscore armar\textunderscore )}
\end{itemize}
Fasquias no costado do navio em construcção, para fixar as escoras que o suspendem.
\section{Armadura}
\begin{itemize}
\item {Grp. gram.:f.}
\end{itemize}
\begin{itemize}
\item {Proveniência:(Lat. \textunderscore armatura\textunderscore )}
\end{itemize}
Conjuncto de armas.
Vestidura de antigos guerreiros.
Armação de edifícios.
Chapa metállica de condensador electrico.
Lâminas de ferro doce, que se collocam junto aos pólos dos magnetes, para que não percam a fôrça attractriva.
Pontas dos animaes.
\section{Armamentário}
\begin{itemize}
\item {Grp. gram.:adj.}
\end{itemize}
Relativo a armamento. Cf. Castilho, na \textunderscore Livraria Cláss.\textunderscore , X, 80.
\section{Armamento}
\begin{itemize}
\item {Grp. gram.:m.}
\end{itemize}
Acção de \textunderscore armar\textunderscore .
Preparativos de guerra.
Conjunto de armas.
Depósito de armas.
Amuleto, composto do sino-saimão, meia-lua e coração, de ferro ou aço.
\section{Armando}
\begin{itemize}
\item {Grp. gram.:m.}
\end{itemize}
Papas aperitivas para cavallos.
\section{Armânia}
\begin{itemize}
\item {Grp. gram.:f.}
\end{itemize}
\begin{itemize}
\item {Utilização:Bot.}
\end{itemize}
Gênero de compostas.
\section{Armão}
\begin{itemize}
\item {Grp. gram.:m.}
\end{itemize}
\begin{itemize}
\item {Proveniência:(Fr. \textunderscore armon\textunderscore )}
\end{itemize}
Jôgo deanteiro das carrêtas de artilharia.
Peça da carroça, a que se prende a extremidade mais grossa do temão.
\section{Armar}
\begin{itemize}
\item {Grp. gram.:v. t.}
\end{itemize}
\begin{itemize}
\item {Grp. gram.:V. i.}
\end{itemize}
\begin{itemize}
\item {Proveniência:(Lat. \textunderscore armare\textunderscore )}
\end{itemize}
Munir de armas: \textunderscore armar soldados\textunderscore .
Cobrir com armadura.
Fortalecer.
Fabricar.
Adornar: \textunderscore armar uma sala\textunderscore .
Tramar: \textunderscore armar calúmnias\textunderscore .
Equipar; apparelhar.
Fazer armadilha.
Quadrar.
\section{Armaria}
\begin{itemize}
\item {Grp. gram.:f.}
\end{itemize}
Depósito de armas.
Heráldica.
\section{Armarinheiro}
\begin{itemize}
\item {Grp. gram.:m.}
\end{itemize}
Proprietario de armarinho.
\section{Armarinho}
\begin{itemize}
\item {Grp. gram.:m.}
\end{itemize}
\begin{itemize}
\item {Utilização:Bras}
\end{itemize}
\begin{itemize}
\item {Proveniência:(De \textunderscore armário\textunderscore )}
\end{itemize}
Pequena casa de negócio; loja de capella.
\section{Armário}
\begin{itemize}
\item {Grp. gram.:m.}
\end{itemize}
\begin{itemize}
\item {Proveniência:(Lat. \textunderscore armarium\textunderscore )}
\end{itemize}
Móvel de madeira, com prateleiras, para guardar objectos de uso doméstico.
Receptáculo, no vão de parede, com prateleiras.
\section{Armatoste}
\begin{itemize}
\item {Grp. gram.:m.}
\end{itemize}
\begin{itemize}
\item {Proveniência:(De \textunderscore armar\textunderscore  + \textunderscore toste\textunderscore )}
\end{itemize}
Engenho, com que primitivamente se armavam as béstas.
\section{Armatura}
\begin{itemize}
\item {Grp. gram.:f.}
\end{itemize}
Placas metállicas, que fazem parte dos condensadores.
(Cp. \textunderscore armadura\textunderscore )
\section{Armazém}
\begin{itemize}
\item {Grp. gram.:m.}
\end{itemize}
\begin{itemize}
\item {Grp. gram.:Pl.}
\end{itemize}
\begin{itemize}
\item {Proveniência:(Do ár. \textunderscore al-makhem\textunderscore )}
\end{itemize}
Depósito de mercadorias.
Depósito de fornecimentos para a guerra.
Taberna.
Mercearia.
* \textunderscore Armazens\textunderscore  geraes, serviços dos caminhos de ferro, que se reduzem a comprar, conservar e distribuir pelos outros serviços os materiaes e outros objectos necessarios.
\section{Armazenado}
\begin{itemize}
\item {Grp. gram.:adj.}
\end{itemize}
Guardado em armazém.
\section{Armazenagem}
\begin{itemize}
\item {Grp. gram.:f.}
\end{itemize}
Acção de \textunderscore armazenar\textunderscore .
Aquillo que se paga pelo depósito de mercadorias em alfândegas, e noutras estações de despacho.
\section{Armazenar}
\begin{itemize}
\item {Grp. gram.:v. t.}
\end{itemize}
Pôr em armazém.
Depositar.
Conservar.
Reunir.
\section{Armazenário}
\begin{itemize}
\item {Grp. gram.:m.}
\end{itemize}
\begin{itemize}
\item {Utilização:Bras. de Pernambuco}
\end{itemize}
Negociante, que toma de arrendamento grandes armazens, onde deposita açúcar que compra para revender ou exportar.
\section{Armazenista}
\begin{itemize}
\item {Grp. gram.:m.}
\end{itemize}
\begin{itemize}
\item {Utilização:Bras}
\end{itemize}
Fiel de armazém.
Encarregado de armazém.
\section{Armeiro}
\begin{itemize}
\item {Grp. gram.:m.}
\end{itemize}
Aquelle que vende armas; aquelle que as fabríca ou concerta.
Alfageme.
Depósito de armas.
\section{Armela}
\begin{itemize}
\item {Grp. gram.:f.}
\end{itemize}
\begin{itemize}
\item {Proveniência:(Lat. \textunderscore armilla\textunderscore )}
\end{itemize}
Argola de ferro.
Bracelete.
\section{Armelina}
\begin{itemize}
\item {Grp. gram.:f.}
\end{itemize}
Pelle branca de armelino.
(Cp. \textunderscore armelino\textunderscore )
\section{Armelino}
\begin{itemize}
\item {Grp. gram.:m.}
\end{itemize}
\begin{itemize}
\item {Proveniência:(Do lat. \textunderscore armelinus\textunderscore )}
\end{itemize}
Animal asiático, o mesmo que \textunderscore arminho\textunderscore .
\section{Armella}
\begin{itemize}
\item {Grp. gram.:f.}
\end{itemize}
\begin{itemize}
\item {Proveniência:(Lat. \textunderscore armilla\textunderscore )}
\end{itemize}
Argola de ferro.
Bracelete.
\section{Armello}
\begin{itemize}
\item {Grp. gram.:m.}
\end{itemize}
\begin{itemize}
\item {Utilização:Prov.}
\end{itemize}
\begin{itemize}
\item {Utilização:minh.}
\end{itemize}
Armadilha, para apanhar pássaros.
(Cp. \textunderscore armella\textunderscore )
\section{Armelo}
\begin{itemize}
\item {Grp. gram.:m.}
\end{itemize}
\begin{itemize}
\item {Utilização:Prov.}
\end{itemize}
\begin{itemize}
\item {Utilização:minh.}
\end{itemize}
Armadilha, para apanhar pássaros.
(Cp. \textunderscore armella\textunderscore )
\section{Armenha}
\begin{itemize}
\item {Grp. gram.:f.}
\end{itemize}
O mesmo que \textunderscore aramenha\textunderscore .
\section{Armênico}
\begin{itemize}
\item {Grp. gram.:m.  e  adj.}
\end{itemize}
O mesmo que \textunderscore armênio\textunderscore .
\section{Armênio}
\begin{itemize}
\item {Grp. gram.:m.}
\end{itemize}
\begin{itemize}
\item {Grp. gram.:Adj.}
\end{itemize}
O idioma da Armênia.
Aquelle que é natural da Armênia.
Relativo á Armênia.
\section{Armenista}
\begin{itemize}
\item {Grp. gram.:m.}
\end{itemize}
Aquelle que se dedica especialmente ao estudo da língua armênica.
\section{Armental}
\begin{itemize}
\item {Grp. gram.:adj.}
\end{itemize}
\begin{itemize}
\item {Proveniência:(Lat. \textunderscore armentalis\textunderscore )}
\end{itemize}
Relativo a armento.
\section{Armentário}
\begin{itemize}
\item {Grp. gram.:m.}
\end{itemize}
\begin{itemize}
\item {Proveniência:(Lat. \textunderscore armentarius\textunderscore )}
\end{itemize}
O mesmo que \textunderscore pastor\textunderscore . Cf. Filinto, XIV, 146.
Tratador dos cavallos, no exército romano. Cf. \textunderscore Viriato Trág.\textunderscore , II, 22.
\section{Armentio}
\begin{itemize}
\item {Grp. gram.:m.}
\end{itemize}
\begin{itemize}
\item {Proveniência:(Do lat. \textunderscore armentivus\textunderscore )}
\end{itemize}
O mesmo que \textunderscore armento\textunderscore .
\section{Armento}
\begin{itemize}
\item {Grp. gram.:m.}
\end{itemize}
\begin{itemize}
\item {Proveniência:(Lat. \textunderscore armentum\textunderscore )}
\end{itemize}
O mesmo que \textunderscore rebanho\textunderscore ^1.
\section{Armentoso}
\begin{itemize}
\item {Grp. gram.:adj.}
\end{itemize}
\begin{itemize}
\item {Proveniência:(Lat. \textunderscore armentosus\textunderscore )}
\end{itemize}
Que possue muito gado.
\section{Arméria}
\begin{itemize}
\item {Grp. gram.:f.}
\end{itemize}
Gênero de plantas plumbagíneas.
\section{Arméu}
\begin{itemize}
\item {Grp. gram.:m.}
\end{itemize}
Manojo de lan, estopa ou linho, que se põe de uma vez na roca.
\section{Armezim}
\begin{itemize}
\item {Grp. gram.:m.}
\end{itemize}
Espécie de tafetá de Bengala.
\section{Armiclausa}
\begin{itemize}
\item {Grp. gram.:f.}
\end{itemize}
\begin{itemize}
\item {Proveniência:(Do lat. \textunderscore arma\textunderscore  + \textunderscore clausus\textunderscore )}
\end{itemize}
Túnica curta, usada pelos antigos Romanos e aberta adeante e atrás, desde a cintura.
\section{Armífero}
\begin{itemize}
\item {Grp. gram.:adj.}
\end{itemize}
O mesmo que \textunderscore armígero\textunderscore .
\section{Armígero}
\begin{itemize}
\item {Grp. gram.:adj.}
\end{itemize}
\begin{itemize}
\item {Grp. gram.:M.}
\end{itemize}
\begin{itemize}
\item {Proveniência:(Lat. \textunderscore armiger\textunderscore )}
\end{itemize}
Que traz armas.
Soldado.
\section{Armila}
\begin{itemize}
\item {Grp. gram.:f.}
\end{itemize}
\begin{itemize}
\item {Proveniência:(Lat. \textunderscore armilla\textunderscore )}
\end{itemize}
Armela.
Manilha.
Cada um dos anéis da base de columna dórica.
\section{Armilado}
\begin{itemize}
\item {Grp. gram.:adj.}
\end{itemize}
\begin{itemize}
\item {Proveniência:(De \textunderscore armilla\textunderscore )}
\end{itemize}
Diz-se dos animaes, rodeados de um anel ou banda, distinta do resto do corpo.
\section{Armilar}
\begin{itemize}
\item {Grp. gram.:adj.}
\end{itemize}
\begin{itemize}
\item {Proveniência:(De \textunderscore armilla\textunderscore )}
\end{itemize}
Que tem armilas.
Que é formado de círculos representativos dos da esphera celeste: \textunderscore esphera armilar\textunderscore .
\section{Armilária}
\begin{itemize}
\item {Grp. gram.:f.}
\end{itemize}
\begin{itemize}
\item {Proveniência:(De \textunderscore armilla\textunderscore )}
\end{itemize}
Tríbo de cogumelos agaricíneos.
\section{Armilha}
\begin{itemize}
\item {Grp. gram.:f.}
\end{itemize}
\begin{itemize}
\item {Utilização:Ant.}
\end{itemize}
\begin{itemize}
\item {Proveniência:(De \textunderscore arma\textunderscore )}
\end{itemize}
Vestidura curta, que se usava por baixo das armas.
\section{Armilha}
\begin{itemize}
\item {Grp. gram.:f.}
\end{itemize}
(V.armadilha)
\section{Armilheiriça}
\begin{itemize}
\item {Grp. gram.:f.}
\end{itemize}
\begin{itemize}
\item {Utilização:T. da Bairrada}
\end{itemize}
Alvéloa de peito amarelo, (\textunderscore motacilla flava\textunderscore ), o mesmo que \textunderscore boieira\textunderscore .
\section{Armilheiro}
\begin{itemize}
\item {Grp. gram.:m.}
\end{itemize}
Espécie de formão, o mesmo que \textunderscore badame\textunderscore .
\section{Armilla}
\begin{itemize}
\item {Grp. gram.:f.}
\end{itemize}
\begin{itemize}
\item {Proveniência:(Lat. \textunderscore armilla\textunderscore )}
\end{itemize}
Armella.
Manilha.
Cada um dos anéis da base de columna dórica.
\section{Armillado}
\begin{itemize}
\item {Grp. gram.:adj.}
\end{itemize}
\begin{itemize}
\item {Proveniência:(De \textunderscore armilla\textunderscore )}
\end{itemize}
Diz-se dos animaes, rodeados de um anel ou banda, distinta do resto do corpo.
\section{Armillar}
\begin{itemize}
\item {Grp. gram.:adj.}
\end{itemize}
\begin{itemize}
\item {Proveniência:(De \textunderscore armilla\textunderscore )}
\end{itemize}
Que tem armillas.
Que é formado de círculos representativos dos da esphera celeste: \textunderscore esphera armillar\textunderscore .
\section{Armillária}
\begin{itemize}
\item {Grp. gram.:f.}
\end{itemize}
\begin{itemize}
\item {Proveniência:(De \textunderscore armilla\textunderscore )}
\end{itemize}
Tríbo de cogumelos agaricíneos.
\section{Armilústria}
\begin{itemize}
\item {Grp. gram.:f.}
\end{itemize}
\begin{itemize}
\item {Proveniência:(Lat. \textunderscore armilustria\textunderscore )}
\end{itemize}
Festa, que os Romanos celebravam no Aventino, passando-se revista ás legiões, e offerecendo-se um sacrifício pela prosperidade das mesmas legiões.
\section{Armilustro}
\begin{itemize}
\item {Grp. gram.:m.}
\end{itemize}
O mesmo que \textunderscore armilústria\textunderscore .
\section{Armim}
\begin{itemize}
\item {Grp. gram.:m.}
\end{itemize}
(V.armino)
\section{Arminado}
\begin{itemize}
\item {Grp. gram.:adj.}
\end{itemize}
Que tem armino.
\section{Arminhado}
\begin{itemize}
\item {Grp. gram.:adj.}
\end{itemize}
Guarnecido de arminho.
Branco, com pontos negros.
\section{Arminho}
\begin{itemize}
\item {Grp. gram.:m.}
\end{itemize}
\begin{itemize}
\item {Utilização:Fig.}
\end{itemize}
\begin{itemize}
\item {Utilização:Heráld.}
\end{itemize}
\begin{itemize}
\item {Grp. gram.:Pl.}
\end{itemize}
Mammífero, da fam. dos mustelídeos, cuja pelle macia é, no inverno, alvíssima.
Pelle do arminho.
Brancura.
Objecto fofo, macio.
Corpo de prata, semeado de pequenas cruzes de sinople, dispostas em xadrez. Cp. L. Ribeiro, \textunderscore Trat. de Armaria\textunderscore .
Títulos de nobreza.
(B. lat. \textunderscore arminius\textunderscore )
\section{Armino}
\begin{itemize}
\item {Grp. gram.:m.}
\end{itemize}
Malha de cabellos no casco do cavallo.
(Cp. \textunderscore arminho\textunderscore )
\section{Armipotência}
\begin{itemize}
\item {Grp. gram.:f.}
\end{itemize}
Qualidade de armipotente.
\section{Armipotente}
\begin{itemize}
\item {Grp. gram.:adj.}
\end{itemize}
\begin{itemize}
\item {Proveniência:(Lat. \textunderscore armipotens\textunderscore )}
\end{itemize}
Poderoso em armas.
Guerreiro.
\section{Armísono}
\begin{itemize}
\item {fónica:so}
\end{itemize}
\begin{itemize}
\item {Grp. gram.:adj.}
\end{itemize}
\begin{itemize}
\item {Proveniência:(Lat. \textunderscore armisonus\textunderscore )}
\end{itemize}
Que sôa como as armas, quando se embatem.
\section{Armíssono}
\begin{itemize}
\item {Grp. gram.:adj.}
\end{itemize}
\begin{itemize}
\item {Proveniência:(Lat. \textunderscore armisonus\textunderscore )}
\end{itemize}
Que sôa como as armas, quando se embatem.
\section{Armista}
\begin{itemize}
\item {Grp. gram.:m.}
\end{itemize}
\begin{itemize}
\item {Proveniência:(De \textunderscore arma\textunderscore )}
\end{itemize}
Aquelle que é perito em Armaria.
\section{Armistício}
\begin{itemize}
\item {Grp. gram.:m.}
\end{itemize}
\begin{itemize}
\item {Proveniência:(Do lat. \textunderscore arma\textunderscore  + \textunderscore stare\textunderscore )}
\end{itemize}
Suspensão de guerra; tréguas curtas.
\section{Armistrondo}
\begin{itemize}
\item {Grp. gram.:m.}
\end{itemize}
\begin{itemize}
\item {Utilização:Neol.}
\end{itemize}
Estrondo das armas.
O retinir das espadas que se cruzam. Cf. C. Neto, \textunderscore Saldunes\textunderscore .
\section{Armo}
\begin{itemize}
\item {Grp. gram.:m.}
\end{itemize}
(V.arméu)
\section{Armões}
\begin{itemize}
\item {Grp. gram.:m. pl.}
\end{itemize}
\begin{itemize}
\item {Utilização:Prov.}
\end{itemize}
\begin{itemize}
\item {Utilização:trasm.}
\end{itemize}
\begin{itemize}
\item {Proveniência:(De \textunderscore arma\textunderscore )}
\end{itemize}
Pernas ou braços, robustos.
\section{Armolão}
\begin{itemize}
\item {Grp. gram.:m.}
\end{itemize}
\begin{itemize}
\item {Utilização:Bras}
\end{itemize}
O mesmo que \textunderscore espinafre\textunderscore .
\section{Armole}
\begin{itemize}
\item {Grp. gram.:f.}
\end{itemize}
\begin{itemize}
\item {Proveniência:(Do lat. \textunderscore olus\textunderscore  + \textunderscore mollis\textunderscore )}
\end{itemize}
Hortaliça, com propriedades semelhantes ás do espinafre.
\section{Armoles}
\begin{itemize}
\item {Grp. gram.:f.}
\end{itemize}
\begin{itemize}
\item {Proveniência:(Do lat. \textunderscore olus\textunderscore  + \textunderscore mollis\textunderscore )}
\end{itemize}
Hortaliça, com propriedades semelhantes ás do espinafre.
\section{Armorácia}
\begin{itemize}
\item {Grp. gram.:f.}
\end{itemize}
\begin{itemize}
\item {Proveniência:(Gr. \textunderscore armorakia\textunderscore )}
\end{itemize}
Planta crucifera, espécie de rábano bravo.
\section{Armorejado}
\begin{itemize}
\item {Grp. gram.:adj.}
\end{itemize}
O mesmo que \textunderscore armoriado\textunderscore . Cf. Camillo, \textunderscore Estrêl. Fun.\textunderscore , pról.
\section{Armoriado}
\begin{itemize}
\item {Grp. gram.:adj.}
\end{itemize}
\begin{itemize}
\item {Utilização:Heráld.}
\end{itemize}
\begin{itemize}
\item {Proveniência:(De \textunderscore armoriar\textunderscore )}
\end{itemize}
Que tem armas ou brasões, applicados, pintados ou esculpidos.
\section{Armoriar}
\begin{itemize}
\item {Grp. gram.:v.}
\end{itemize}
\begin{itemize}
\item {Utilização:t. Heráld.}
\end{itemize}
\begin{itemize}
\item {Proveniência:(Fr. \textunderscore armorier\textunderscore )}
\end{itemize}
Pôr armas ou brasões em; empregar os sýmbolos da nobreza em. Cf. Camillo, \textunderscore Suicida\textunderscore , 7.
\section{Armorial}
\begin{itemize}
\item {Grp. gram.:m.}
\end{itemize}
\begin{itemize}
\item {Grp. gram.:Adj.}
\end{itemize}
\begin{itemize}
\item {Proveniência:(Fr. \textunderscore armorial\textunderscore )}
\end{itemize}
Livro, em que estão registados os brasões.
Relativo á Armaria ou a brasões.
\section{Armórico}
\begin{itemize}
\item {Grp. gram.:adj.}
\end{itemize}
O mesmo que \textunderscore bretão\textunderscore ^1.
\section{Armósia}
\begin{itemize}
\item {Grp. gram.:f.}
\end{itemize}
\begin{itemize}
\item {Utilização:Bot.}
\end{itemize}
Gênero de compostas.
\section{Armur}
\begin{itemize}
\item {Grp. gram.:m.}
\end{itemize}
\begin{itemize}
\item {Proveniência:(Fr. \textunderscore armure\textunderscore )}
\end{itemize}
Espécie de tecido mais ou menos transparente. Cf. \textunderscore Inquér. Industr.\textunderscore , l. III, 37.
\section{Armuzello}
\begin{itemize}
\item {Grp. gram.:m.}
\end{itemize}
Espécie de rede.
\section{Armuzelo}
\begin{itemize}
\item {fónica:zê}
\end{itemize}
\begin{itemize}
\item {Grp. gram.:m.}
\end{itemize}
Espécie de rede.
\section{Arnabutos}
\begin{itemize}
\item {Grp. gram.:m. pl.}
\end{itemize}
Aborígenes brasileiros, que habitam no Pará.
\section{Arnado}
\begin{itemize}
\item {Grp. gram.:m.}
\end{itemize}
O mesmo que \textunderscore arneiro\textunderscore .
\section{Arnal}
\begin{itemize}
\item {Grp. gram.:adj.}
\end{itemize}
\begin{itemize}
\item {Utilização:Prov.}
\end{itemize}
\begin{itemize}
\item {Utilização:minh.}
\end{itemize}
\begin{itemize}
\item {Grp. gram.:M.}
\end{itemize}
\begin{itemize}
\item {Utilização:Prov.}
\end{itemize}
\begin{itemize}
\item {Utilização:minh.}
\end{itemize}
Relativo a areia.
Que cresce na areia: \textunderscore o mato arnal\textunderscore .
Tojo, que cresce na areia.
(Por \textunderscore arenal\textunderscore , do lat. \textunderscore arena\textunderscore )
\section{Arnás}
\begin{itemize}
\item {Grp. gram.:m.}
\end{itemize}
\begin{itemize}
\item {Utilização:Prov.}
\end{itemize}
\begin{itemize}
\item {Utilização:trasm.}
\end{itemize}
\begin{itemize}
\item {Grp. gram.:Adj.}
\end{itemize}
Qualidade do que come muito ou tem bôa boca.
Estômago.
Robusto.
(Cp. \textunderscore arnela\textunderscore )
\section{Arnasudo}
\begin{itemize}
\item {Grp. gram.:adj.}
\end{itemize}
\begin{itemize}
\item {Utilização:Prov.}
\end{itemize}
\begin{itemize}
\item {Utilização:trasm.}
\end{itemize}
\begin{itemize}
\item {Proveniência:(De \textunderscore arnás\textunderscore )}
\end{itemize}
Que tem bom estômago.
Que é robusto.
\section{Arnecan}
\begin{itemize}
\item {Grp. gram.:f.}
\end{itemize}
Planta do Brasil.
\section{Arneira}
\begin{itemize}
\item {Grp. gram.:f.}
\end{itemize}
Madeira do Brasil.
\section{Arneiro}
\begin{itemize}
\item {Grp. gram.:m.}
\end{itemize}
\begin{itemize}
\item {Proveniência:(Lat. \textunderscore arenarius\textunderscore )}
\end{itemize}
Lugar arenoso, estéril.
\section{Arnela}
\begin{itemize}
\item {Grp. gram.:f.}
\end{itemize}
Resto de um dente na gengiva.
(Por \textunderscore arenela\textunderscore , do lat. \textunderscore arena\textunderscore )
\section{Arnês}
\begin{itemize}
\item {Grp. gram.:m.}
\end{itemize}
\begin{itemize}
\item {Utilização:Fig.}
\end{itemize}
Antiga armadura completa de um guerreiro.
Apparelho de cavallo.
Protecção.
(Cast. \textunderscore arnés\textunderscore )
\section{Arnesar}
\begin{itemize}
\item {Grp. gram.:v. t.}
\end{itemize}
Cobrir com arnês.
\section{Arneutéria}
\begin{itemize}
\item {Grp. gram.:f.}
\end{itemize}
\begin{itemize}
\item {Utilização:Des.}
\end{itemize}
Arte de natação.
\section{Arnica}
\begin{itemize}
\item {Grp. gram.:f.}
\end{itemize}
\begin{itemize}
\item {Proveniência:(Do lat. \textunderscore ptarmica\textunderscore )}
\end{itemize}
Planta, da fam. das compostas, vulgarmente conhecida por \textunderscore espirradeira\textunderscore .
Tintura, extrahida dessa planta, macerada em álcool.
\section{Arnicina}
\begin{itemize}
\item {Grp. gram.:f.}
\end{itemize}
Resina, extrahida da arnica.
\section{Arnilhas}
\begin{itemize}
\item {Grp. gram.:m.}
\end{itemize}
\begin{itemize}
\item {Utilização:Prov.}
\end{itemize}
\begin{itemize}
\item {Utilização:beir.}
\end{itemize}
Criança enfesada e magra.
\section{Arnito-cerceal}
\begin{itemize}
\item {Grp. gram.:m.}
\end{itemize}
Casta de uva minhota.
\section{Arnoso}
\begin{itemize}
\item {Grp. gram.:m.}
\end{itemize}
(V.arneiro)
\section{Aro}
\begin{itemize}
\item {Grp. gram.:m.}
\end{itemize}
\begin{itemize}
\item {Utilização:Carp.}
\end{itemize}
Pequeno arco.
Anel, círculo.
Abertura circular.
Subúrbios de cidade ou terra importante.
O mesmo que \textunderscore cincho\textunderscore ^1.
Peça quadrangular de madeira, em fórma de moldura ou caixilho, com que se guarnecem os vãos das janelas.
\textunderscore Aro de pedraria\textunderscore , aro em que se movem ou trabalham as portas das janelas.
\textunderscore Aro de golla\textunderscore , aro em que se movem ou trabalham os caixilhos ou vidraças das janelas.
\section{Aro}
\begin{itemize}
\item {Grp. gram.:m.}
\end{itemize}
O mesmo que \textunderscore arão\textunderscore .
\section{Aroaquis}
\begin{itemize}
\item {Grp. gram.:m. pl.}
\end{itemize}
Índios do Brasil, que dominavam nas margens do Rio-Negro.
\section{Aroeira}
\begin{itemize}
\item {Grp. gram.:f.}
\end{itemize}
Planta brasileira.
O mesmo que \textunderscore lentisco\textunderscore .
\section{Aroeiro}
\begin{itemize}
\item {Grp. gram.:m.}
\end{itemize}
O mesmo que \textunderscore aroeira\textunderscore .
\section{Aroídeas}
\begin{itemize}
\item {Grp. gram.:f. pl.}
\end{itemize}
\begin{itemize}
\item {Proveniência:(Do gr. \textunderscore aron\textunderscore  + \textunderscore eidos\textunderscore )}
\end{itemize}
Família de plantas, o mesmo que \textunderscore aráceas\textunderscore .
\section{Arola}
\begin{itemize}
\item {Grp. gram.:f.}
\end{itemize}
\begin{itemize}
\item {Utilização:Prov.}
\end{itemize}
Arriosca; armadilha.
\section{Arolas}
\begin{itemize}
\item {Grp. gram.:m.}
\end{itemize}
\begin{itemize}
\item {Utilização:Prov.}
\end{itemize}
\begin{itemize}
\item {Utilização:trasm.}
\end{itemize}
Sujeito sem valor.
\section{Aroma}
\begin{itemize}
\item {Grp. gram.:m.}
\end{itemize}
\begin{itemize}
\item {Proveniência:(Gr. \textunderscore aroma\textunderscore )}
\end{itemize}
Essência odorífera de várias substâncias vegetaes.
Perfume agradável.
Fragrância.
\section{Aromadendro}
\begin{itemize}
\item {Grp. gram.:m.}
\end{itemize}
Gênero de magnoleáceas.
\section{Aromal}
\begin{itemize}
\item {Grp. gram.:adj.}
\end{itemize}
\begin{itemize}
\item {Utilização:Neol.}
\end{itemize}
Relativo a aromas.
\section{Aromar}
\begin{itemize}
\item {Grp. gram.:v. t.}
\end{itemize}
\begin{itemize}
\item {Utilização:Prov.}
\end{itemize}
\begin{itemize}
\item {Utilização:trasm.}
\end{itemize}
O mesmo que \textunderscore aromatizar\textunderscore .
\section{Arômatas}
\begin{itemize}
\item {Grp. gram.:m. pl}
\end{itemize}
\begin{itemize}
\item {Utilização:Ant.}
\end{itemize}
O mesmo que [[aromas|aroma]].
\section{Aromaticidade}
\begin{itemize}
\item {Grp. gram.:f.}
\end{itemize}
Qualidade do que é aromático.
\section{Aromático}
\begin{itemize}
\item {Grp. gram.:adj.}
\end{itemize}
\begin{itemize}
\item {Proveniência:(Gr. \textunderscore aromatikos\textunderscore )}
\end{itemize}
Que tem aroma.
\section{Aromatização}
\begin{itemize}
\item {Grp. gram.:f.}
\end{itemize}
Acto de \textunderscore aromatizar\textunderscore .
\section{Aromatizante}
\begin{itemize}
\item {Grp. gram.:adj.}
\end{itemize}
Que aromatiza.
\section{Aromatizar}
\begin{itemize}
\item {Grp. gram.:v. t.}
\end{itemize}
\begin{itemize}
\item {Proveniência:(Lat. \textunderscore aromatizare\textunderscore )}
\end{itemize}
Tornar aromático.
\section{Aromato}
\begin{itemize}
\item {Grp. gram.:m.}
\end{itemize}
\begin{itemize}
\item {Proveniência:(De \textunderscore aroma\textunderscore )}
\end{itemize}
Parte dos vegetaes odoríferos, empregada no fabrico de perfumes.
\section{Aromatáforo}
\begin{itemize}
\item {Grp. gram.:m.}
\end{itemize}
Escravo que levava os aromas do seu senhor.
\section{Aromatóphoro}
\begin{itemize}
\item {Grp. gram.:m.}
\end{itemize}
Escravo que levava os aromas do seu senhor.
\section{Aromoso}
\begin{itemize}
\item {Grp. gram.:adj.}
\end{itemize}
\begin{itemize}
\item {Utilização:Neol.}
\end{itemize}
O mesmo que \textunderscore aromático\textunderscore .
\section{Aronco}
\begin{itemize}
\item {Grp. gram.:m.}
\end{itemize}
Gênero de rosáceas.
\section{Arónea}
\begin{itemize}
\item {Grp. gram.:f.}
\end{itemize}
Gênero de pomáceas.
\section{Aroscina}
\begin{itemize}
\item {Grp. gram.:f.}
\end{itemize}
Medicamento mydriático, quatro vezes mais activo que a atropina.
\section{Arouca}
\begin{itemize}
\item {Grp. gram.:f.}
\end{itemize}
Planta da serra de Cintra.
\section{Arouquesa}
\begin{itemize}
\item {Grp. gram.:adj. f.}
\end{itemize}
Diz-se da vaca, que é criada nas terras de Arouca.
\section{Arpão}
\begin{itemize}
\item {Grp. gram.:m.}
\end{itemize}
Instrumento, com que se pescam os grandes peixes.
Fisga.
Gancho, para aferrar embarcações.
Arma indiana.
(Cast. \textunderscore arpon\textunderscore )
\section{Arpar}
\begin{itemize}
\item {Grp. gram.:v. t.}
\end{itemize}
O mesmo que \textunderscore arpear\textunderscore .
\section{Arpear}
\begin{itemize}
\item {Grp. gram.:v. t.}
\end{itemize}
Arpoar. Cf. Filinto, \textunderscore D. Man.\textunderscore , 278.
\section{Arpejar}
\begin{itemize}
\item {Grp. gram.:v. i.}
\end{itemize}
Fazer arpejos.
\section{Arpejo}
\begin{itemize}
\item {Grp. gram.:m.}
\end{itemize}
\begin{itemize}
\item {Proveniência:(It. \textunderscore arpeggio\textunderscore )}
\end{itemize}
Acorde de sons sucessivos, em instrumento de cordas.
\section{Arpentagem}
\begin{itemize}
\item {Grp. gram.:f.}
\end{itemize}
\begin{itemize}
\item {Proveniência:(Fr. \textunderscore arpentage\textunderscore . Cp. \textunderscore arpente\textunderscore )}
\end{itemize}
Medida da superfície das terras.
\section{Arpente}
\begin{itemize}
\item {Grp. gram.:m.}
\end{itemize}
Antiga medida agrária, entre os Gállios, adoptada pelos Romanos.
Geira.
(Cp. \textunderscore arapenne\textunderscore )
\section{Arpento}
\begin{itemize}
\item {Grp. gram.:m.}
\end{itemize}
Antiga medida agrária, entre os Gállios, adoptada pelos Romanos.
Geira.
(Cp. \textunderscore arapenne\textunderscore )
\section{Arpéo}
\begin{itemize}
\item {Grp. gram.:m.}
\end{itemize}
Pequeno arpão.
Fisga.
Fateixa.
(Cast. \textunderscore arpeo\textunderscore )
\section{Arpéu}
\begin{itemize}
\item {Grp. gram.:m.}
\end{itemize}
Pequeno arpão.
Fisga.
Fateixa.
(Cast. \textunderscore arpeo\textunderscore )
\section{Arpicórdio}
\begin{itemize}
\item {Grp. gram.:m.}
\end{itemize}
\begin{itemize}
\item {Utilização:Mús.}
\end{itemize}
Espécie de cravo antigo, cujas cordas eram superiormente verticaes ao teclado, dando quási o aspecto de harpa.
\section{Arpista}
\begin{itemize}
\item {Grp. gram.:m.}
\end{itemize}
\begin{itemize}
\item {Utilização:T. do Fundão}
\end{itemize}
Vento muito frio.
\section{Arpoação}
\begin{itemize}
\item {Grp. gram.:f.}
\end{itemize}
Acto de \textunderscore arpoar\textunderscore .
\section{Arpoador}
\begin{itemize}
\item {Grp. gram.:m.}
\end{itemize}
Aquelle que arpôa.
\section{Arpoar}
\begin{itemize}
\item {Grp. gram.:v. i.}
\end{itemize}
Segurar com o arpão.
Arremessá-lo contra.
Ferir com êlle.
Agarrar.
Seduzir.
\section{Arpoeira}
\begin{itemize}
\item {Grp. gram.:f.}
\end{itemize}
Corda do arpão.
\section{Arqueação}
\begin{itemize}
\item {Grp. gram.:f.}
\end{itemize}
Medição de vasilhas arqueadas.
Capacidade do navio.
Acto ou effeito de \textunderscore arquear\textunderscore .
\section{Arqueador}
\begin{itemize}
\item {Grp. gram.:m.}
\end{itemize}
Aquelle que arqueia.
\section{Arqueadura}
\begin{itemize}
\item {Grp. gram.:f.}
\end{itemize}
\begin{itemize}
\item {Proveniência:(De \textunderscore arquear\textunderscore )}
\end{itemize}
Curvatura em arco.
\section{Arqueamento}
\begin{itemize}
\item {Grp. gram.:m.}
\end{itemize}
O mesmo que \textunderscore arqueadura\textunderscore .
\section{Arquear}
\begin{itemize}
\item {Grp. gram.:v. t.}
\end{itemize}
Curvar em fórma de arco.
Medir a capacidade de.
Dobrar, curvar; tornar flexível.
\section{Arquegónio}
\begin{itemize}
\item {Grp. gram.:m.}
\end{itemize}
O órgão feminino das algas, das plantas hepáticas e das criptogâmicas.
\section{Arqueio}
\begin{itemize}
\item {Grp. gram.:m.}
\end{itemize}
(V.arqueação)
\section{Arqueiro}
\begin{itemize}
\item {Grp. gram.:m.}
\end{itemize}
\begin{itemize}
\item {Utilização:Ant.}
\end{itemize}
\begin{itemize}
\item {Proveniência:(De \textunderscore arco\textunderscore  e de \textunderscore arca\textunderscore )}
\end{itemize}
Aquelle que faz ou vende arcos; tanoeiro.
Aquelle que faz ou vende arcas.
Thesoireiro.
\section{Arquejamento}
\begin{itemize}
\item {Grp. gram.:m.}
\end{itemize}
O mesmo que \textunderscore arquejo\textunderscore .
\section{Arquejante}
\begin{itemize}
\item {Grp. gram.:adj.}
\end{itemize}
\begin{itemize}
\item {Proveniência:(De \textunderscore arquejar\textunderscore ^1)}
\end{itemize}
Que arqueja.
\section{Arquejar}
\begin{itemize}
\item {Grp. gram.:v. i.}
\end{itemize}
\begin{itemize}
\item {Proveniência:(De \textunderscore arca\textunderscore )}
\end{itemize}
Arfar.
Offegar.
Ansiar.
\section{Arquejar}
\begin{itemize}
\item {Grp. gram.:v. t.}
\end{itemize}
\begin{itemize}
\item {Utilização:Des.}
\end{itemize}
\begin{itemize}
\item {Proveniência:(De \textunderscore arco\textunderscore )}
\end{itemize}
O mesmo que \textunderscore arquear\textunderscore .
\section{Arquejo}
\begin{itemize}
\item {Grp. gram.:m.}
\end{itemize}
Acto de \textunderscore arquejar\textunderscore ^1.
\section{Arquel}
\begin{itemize}
\item {Grp. gram.:m.}
\end{itemize}
Arbusto, da fam. das apocýneas.
\section{Arquelha}
\begin{itemize}
\item {fónica:quê}
\end{itemize}
\begin{itemize}
\item {Grp. gram.:f.}
\end{itemize}
\begin{itemize}
\item {Utilização:Ant.}
\end{itemize}
\begin{itemize}
\item {Proveniência:(De \textunderscore arco\textunderscore )}
\end{itemize}
Pavilhão da cama.
Mosquiteiro.
Sobrecéu.
\section{Arqueogeologia}
\begin{itemize}
\item {Grp. gram.:f.}
\end{itemize}
Geologia prehistórica.
\section{Arqueografia}
\begin{itemize}
\item {Grp. gram.:f.}
\end{itemize}
Descripção dos monumentos antigos.
(Cp. \textunderscore archeógrapho\textunderscore )
\section{Arqueógrafo}
\begin{itemize}
\item {Grp. gram.:m.}
\end{itemize}
\begin{itemize}
\item {Proveniência:(Do gr. \textunderscore arkhaios\textunderscore  + \textunderscore graphein\textunderscore )}
\end{itemize}
Aquelle que descreve monumentos antigos.
\section{Arqueolítico}
\begin{itemize}
\item {Grp. gram.:adj.}
\end{itemize}
\begin{itemize}
\item {Proveniência:(Do gr. \textunderscore arkhaios\textunderscore  + \textunderscore lithos\textunderscore )}
\end{itemize}
Relativo ás rochas das primeiras idades geológicas.
\section{Arqueolologia}
\begin{itemize}
\item {Grp. gram.:f.}
\end{itemize}
Estudo de coisas antigas.
(Cp. \textunderscore archeólogo\textunderscore )
\section{Arqueológico}
\begin{itemize}
\item {Grp. gram.:adj.}
\end{itemize}
Relativo á \textunderscore Arqueologia\textunderscore .
\section{Arqueólogo}
\begin{itemize}
\item {Grp. gram.:m.}
\end{itemize}
\begin{itemize}
\item {Proveniência:(Do gr. \textunderscore arkhaios\textunderscore  + \textunderscore logos\textunderscore )}
\end{itemize}
Aquelle que se dedica á Arqueologia ou é versado nella.
\section{Arqueozoítico}
\begin{itemize}
\item {Grp. gram.:adj.}
\end{itemize}
\begin{itemize}
\item {Utilização:Geol.}
\end{itemize}
\begin{itemize}
\item {Proveniência:(Do gr. \textunderscore arkhaios\textunderscore  + \textunderscore zoon\textunderscore )}
\end{itemize}
Diz-se da primeira phase do período philogenético, na qual só havia na terra animaes invertebrados.
\section{Arqueta}
\begin{itemize}
\item {fónica:quê}
\end{itemize}
\begin{itemize}
\item {Grp. gram.:f.}
\end{itemize}
Pequena arca; caixinha.
\section{Arquete}
\begin{itemize}
\item {fónica:quê}
\end{itemize}
\begin{itemize}
\item {Grp. gram.:m.}
\end{itemize}
Pequeno arco para toca instrumentos de corda.
Tumba, urna cinerária.
\section{Arquétipo}
\begin{itemize}
\item {Grp. gram.:m.}
\end{itemize}
Modêlo dos seres criados.
Exemplar.
Padrão.
(Lat. \textunderscore archetypum\textunderscore ).
\section{Arqueu}
\begin{itemize}
\item {Grp. gram.:m.}
\end{itemize}
\begin{itemize}
\item {Utilização:Des.}
\end{itemize}
\begin{itemize}
\item {Proveniência:(Do gr. \textunderscore arkheios\textunderscore )}
\end{itemize}
Vigor, energia. Cf. Cortesão, Subs.
\section{Arquiapóstata}
\begin{itemize}
\item {Grp. gram.:m.}
\end{itemize}
O maior dos apóstatas.
\section{Arquiatro}
\begin{itemize}
\item {Grp. gram.:m.}
\end{itemize}
O médico principal.
O médico do rei.
(Lat. \textunderscore archiatrus\textunderscore ).
\section{Arquibancada}
\begin{itemize}
\item {Grp. gram.:f.}
\end{itemize}
\begin{itemize}
\item {Utilização:Bras}
\end{itemize}
O mesmo que \textunderscore arquibanco\textunderscore .
\section{Arquibanco}
\begin{itemize}
\item {Grp. gram.:m.}
\end{itemize}
\begin{itemize}
\item {Proveniência:(De \textunderscore archi...\textunderscore  + \textunderscore banco\textunderscore )}
\end{itemize}
O maior banco de uma casa; banco grande de costas.
\section{Arquiconfraria}
\begin{itemize}
\item {Grp. gram.:f.}
\end{itemize}
\begin{itemize}
\item {Proveniência:(De \textunderscore archi...\textunderscore  + \textunderscore confraria\textunderscore )}
\end{itemize}
Confraria principal.
\section{Arquicítara}
\begin{itemize}
\item {Grp. gram.:f.}
\end{itemize}
\begin{itemize}
\item {Utilização:Des.}
\end{itemize}
Instrumento de dois cravelhames e vinte e duas cordas.
\section{Arquidiácono}
\begin{itemize}
\item {Grp. gram.:m.}
\end{itemize}
(V.Arcediago)
\section{Arquidiocesano}
\begin{itemize}
\item {Grp. gram.:adj.}
\end{itemize}
Relativo a \textunderscore arquidiocese\textunderscore .
\section{Arquidiocese}
\begin{itemize}
\item {Grp. gram.:f.}
\end{itemize}
\begin{itemize}
\item {Proveniência:(De \textunderscore archi...\textunderscore  + \textunderscore diocese\textunderscore )}
\end{itemize}
Diocese, que tem outras suffragâneas.
Arcebispado.
\section{Arquidivino}
\begin{itemize}
\item {Grp. gram.:adj.}
\end{itemize}
Superiormente divino. Cf. Castilho, \textunderscore Sabichonas\textunderscore , 231.
\section{Arquiducado}
\begin{itemize}
\item {Grp. gram.:m.}
\end{itemize}
Dignidade ou território de \textunderscore Arquiduque\textunderscore .
\section{Arquiducal}
\begin{itemize}
\item {Grp. gram.:adj.}
\end{itemize}
Pertencente a \textunderscore Arquiduque\textunderscore .
\section{Arquiduque}
\begin{itemize}
\item {Grp. gram.:m.}
\end{itemize}
\begin{itemize}
\item {Proveniência:(De \textunderscore archi...\textunderscore  + \textunderscore duque\textunderscore )}
\end{itemize}
Título dos Príncipes da Casa de Áustria.
\section{Arquiduquesa}
\begin{itemize}
\item {Grp. gram.:f.}
\end{itemize}
Mulher de Arquiduque.
Título das Princesas da Casa de Áustria.
\section{Arquiepiscopado}
\begin{itemize}
\item {Grp. gram.:m.}
\end{itemize}
O mesmo que \textunderscore arcebispado\textunderscore . Cf. Herculano, \textunderscore Hist. de Port.\textunderscore , III, 417.
\section{Arquiepiscopal}
\begin{itemize}
\item {Grp. gram.:adj.}
\end{itemize}
\begin{itemize}
\item {Proveniência:(De \textunderscore archi...\textunderscore  + \textunderscore episcopal\textunderscore )}
\end{itemize}
Relativo a Arcebispo.
\section{Arquilaúde}
\begin{itemize}
\item {Grp. gram.:m.}
\end{itemize}
\begin{itemize}
\item {Proveniência:(Fr. archilute)}
\end{itemize}
Instrumento músico do cravelhame e dezasete cordas.
\section{Arquilevita}
\begin{itemize}
\item {Grp. gram.:m.}
\end{itemize}
\begin{itemize}
\item {Proveniência:(De \textunderscore archi...\textunderscore  + \textunderscore levita\textunderscore )}
\end{itemize}
Chefe dos levitas, entre os Hebreus.
\section{Arquilho}
\begin{itemize}
\item {Grp. gram.:m.}
\end{itemize}
\begin{itemize}
\item {Proveniência:(De \textunderscore arco\textunderscore )}
\end{itemize}
Arco delgado de madeira ou metal, nos bombos e tambores, sôbre o qual se retesa o papel, que outro arco comprime por meio de parafusos ou cordagem.
\section{Arquilinfático}
\begin{itemize}
\item {Grp. gram.:adj.}
\end{itemize}
\begin{itemize}
\item {Utilização:Med.}
\end{itemize}
Linfático em alto grau.
\section{Arquimagiro}
\begin{itemize}
\item {Grp. gram.:m.}
\end{itemize}
\begin{itemize}
\item {Utilização:Ant.}
\end{itemize}
\begin{itemize}
\item {Proveniência:(Lat. \textunderscore archimagirus\textunderscore )}
\end{itemize}
Chefe de cozinha; chefe de cozinheiros.
\section{Arquimago}
\begin{itemize}
\item {Grp. gram.:m.}
\end{itemize}
\begin{itemize}
\item {Proveniência:(De \textunderscore archi...\textunderscore  + \textunderscore mago\textunderscore )}
\end{itemize}
Chefe dos magos.
Chefe da religião, entre os antigos Persas.
\section{Arquimandrita}
\begin{itemize}
\item {Grp. gram.:f.}
\end{itemize}
\begin{itemize}
\item {Proveniência:(Gr. \textunderscore arkhimandrites\textunderscore )}
\end{itemize}
Abbade de certos conventos.
\section{Arquimandritado}
\begin{itemize}
\item {Grp. gram.:m.}
\end{itemize}
Dignidade de arquimandrita.
\section{Arquimimo}
\begin{itemize}
\item {Grp. gram.:m.}
\end{itemize}
\begin{itemize}
\item {Utilização:Ant.}
\end{itemize}
\begin{itemize}
\item {Proveniência:(Lat. \textunderscore archimimus\textunderscore )}
\end{itemize}
Chefe dos que representavam pantomimas.
\section{Arquiministro}
\begin{itemize}
\item {Grp. gram.:m.}
\end{itemize}
\begin{itemize}
\item {Utilização:Des.}
\end{itemize}
\begin{itemize}
\item {Proveniência:(De \textunderscore archi...\textunderscore  + \textunderscore ministro\textunderscore )}
\end{itemize}
O primeiro Ministro.
\section{Arquimorto}
\begin{itemize}
\item {Grp. gram.:adj.}
\end{itemize}
\begin{itemize}
\item {Utilização:Des.}
\end{itemize}
Que está bem morto; morto há muito tempo.
\section{Arquimosteiro}
\begin{itemize}
\item {Grp. gram.:m.}
\end{itemize}
\begin{itemize}
\item {Proveniência:(De \textunderscore archi...\textunderscore  + \textunderscore mosteiro\textunderscore )}
\end{itemize}
Mosteiro principal de uma Ordem religiosa.
\section{Arquinha}
\begin{itemize}
\item {Grp. gram.:f.}
\end{itemize}
\begin{itemize}
\item {Proveniência:(De \textunderscore arca\textunderscore )}
\end{itemize}
Mollusco acéphalo.
\section{Arquinhos}
\begin{itemize}
\item {Grp. gram.:m. pl.}
\end{itemize}
Espécie de jôgo popular, o mesmo que \textunderscore paus-mandados\textunderscore .
\section{Arquinotário}
\begin{itemize}
\item {Grp. gram.:m.}
\end{itemize}
\begin{itemize}
\item {Utilização:Ant.}
\end{itemize}
\begin{itemize}
\item {Proveniência:(De \textunderscore archi...\textunderscore  + \textunderscore notário\textunderscore )}
\end{itemize}
Chefe dos notários.
\section{Arquipélago}
\begin{itemize}
\item {Grp. gram.:m.}
\end{itemize}
\begin{itemize}
\item {Proveniência:(Do gr. \textunderscore arkhi\textunderscore  + \textunderscore pelagos\textunderscore )}
\end{itemize}
Grupo de ilhas, pouco distantes umas das outras: \textunderscore o arquipélago dos Açores\textunderscore .
\section{Arquipirata}
\begin{itemize}
\item {Grp. gram.:m.}
\end{itemize}
\begin{itemize}
\item {Utilização:Fig.}
\end{itemize}
\begin{itemize}
\item {Proveniência:(Lat. \textunderscore archipirata\textunderscore )}
\end{itemize}
Chefe de piratas.
Agiota, usurário.
\section{Arquipotente}
\begin{itemize}
\item {Grp. gram.:adj.}
\end{itemize}
Poderosíssimo. Cf. Castilho, \textunderscore Fausto\textunderscore , 275.
\section{Arquipresbítero}
\begin{itemize}
\item {Grp. gram.:m. (e der.)}
\end{itemize}
O mesmo que \textunderscore arcipreste\textunderscore , etc.
\section{Arquiprior}
\begin{itemize}
\item {Grp. gram.:m.}
\end{itemize}
\begin{itemize}
\item {Proveniência:(De \textunderscore archi...\textunderscore  + \textunderscore prior\textunderscore )}
\end{itemize}
Título do Grão-Mestre dos Templários.
\section{Arquipriorado}
\begin{itemize}
\item {Grp. gram.:m.}
\end{itemize}
Dignidade de Arquiprior.
\section{Arquiprofeta}
\begin{itemize}
\item {Grp. gram.:m.}
\end{itemize}
\begin{itemize}
\item {Proveniência:(De \textunderscore archi...\textunderscore  + \textunderscore propheta\textunderscore )}
\end{itemize}
O principal dos profetas.
\section{Arquipulha}
\begin{itemize}
\item {Grp. gram.:m.}
\end{itemize}
Grandíssimo pulha. Cf. Camillo, \textunderscore Noites de Insómn\textunderscore ., VIII, 93.
\section{Arquissistro}
\begin{itemize}
\item {Grp. gram.:m.}
\end{itemize}
Antigo instrumento musical.
\section{Arquissofista}
\begin{itemize}
\item {Grp. gram.:m.}
\end{itemize}
Sofista em alto grau.
\section{Arquitaria}
\begin{itemize}
\item {Grp. gram.:f.}
\end{itemize}
\begin{itemize}
\item {Utilização:Ant.}
\end{itemize}
Ucharia da casa real.
(Por \textunderscore arquetaria\textunderscore , de \textunderscore arqueta\textunderscore )
\section{Arquitectar}
\begin{itemize}
\item {Grp. gram.:v. t.}
\end{itemize}
\begin{itemize}
\item {Proveniência:(De \textunderscore architecto\textunderscore )}
\end{itemize}
Edificar: \textunderscore arquitectar um palácio\textunderscore .
Planear; idear: \textunderscore arquitectar uma empresa\textunderscore .
\section{Arquitecto}
\begin{itemize}
\item {Grp. gram.:m.}
\end{itemize}
\begin{itemize}
\item {Proveniência:(Lat. \textunderscore architectus\textunderscore )}
\end{itemize}
Aquelle que dirige construcções de edifícios.
Aquelle que planeia; aquelle que fantasia.
\section{Arquitectónica}
\begin{itemize}
\item {Grp. gram.:f.}
\end{itemize}
O mesmo que \textunderscore arquitectura\textunderscore .
\section{Arquitectónico}
\begin{itemize}
\item {Grp. gram.:adj.}
\end{itemize}
\begin{itemize}
\item {Proveniência:(Lat. \textunderscore architectonicus\textunderscore )}
\end{itemize}
Relativo á arquitectura.
\section{Arquitectonografia}
\begin{itemize}
\item {Grp. gram.:f.}
\end{itemize}
Arte de descrever edifícios.
(Cp. \textunderscore architectonógrapho\textunderscore )
\section{Arquitectonógrafo}
\begin{itemize}
\item {Grp. gram.:m.}
\end{itemize}
\begin{itemize}
\item {Proveniência:(Do gr. \textunderscore arkhitektonos\textunderscore  + \textunderscore graphein\textunderscore )}
\end{itemize}
Aquelle que faz a descripção de edifícios.
\section{Arquitector}
\begin{itemize}
\item {Grp. gram.:m.}
\end{itemize}
\begin{itemize}
\item {Utilização:Ant.}
\end{itemize}
\begin{itemize}
\item {Proveniência:(Lat. \textunderscore architector\textunderscore )}
\end{itemize}
O mesmo que \textunderscore arquitecto\textunderscore .
\section{Arquitectura}
\begin{itemize}
\item {Grp. gram.:f.}
\end{itemize}
\begin{itemize}
\item {Proveniência:(Lat. \textunderscore architectura\textunderscore )}
\end{itemize}
Arte de construir edifícios.
Contextura.
Plano, projecto.
\section{Arquitectural}
\begin{itemize}
\item {Grp. gram.:adj.}
\end{itemize}
Relativo á \textunderscore arquitectura\textunderscore .
\section{Arquitecturista}
\begin{itemize}
\item {Grp. gram.:m.}
\end{itemize}
\begin{itemize}
\item {Proveniência:(De \textunderscore architectura\textunderscore )}
\end{itemize}
Pintor, que tem por especialidade a reproducção de edifícios em suas telas.
\section{Arquitolo}
\begin{itemize}
\item {Grp. gram.:m.  e  adj.}
\end{itemize}
\begin{itemize}
\item {Proveniência:(De \textunderscore archi...\textunderscore  + \textunderscore tolo\textunderscore )}
\end{itemize}
Tolo no mais alto grau.
\section{Arquitravado}
\begin{itemize}
\item {Grp. gram.:adj.}
\end{itemize}
Ornado de arquitrave.
\section{Arquitrave}
\begin{itemize}
\item {Grp. gram.:f.}
\end{itemize}
\begin{itemize}
\item {Utilização:Archit.}
\end{itemize}
\begin{itemize}
\item {Proveniência:(De \textunderscore archi...\textunderscore  + \textunderscore trave\textunderscore )}
\end{itemize}
Parte inferior de um entablamento, entre o friso e o capitel.
\section{Arquitriclino}
\begin{itemize}
\item {Grp. gram.:m.}
\end{itemize}
\begin{itemize}
\item {Proveniência:(Lat. \textunderscore architriclinus\textunderscore )}
\end{itemize}
Chefe de escanções ou dos que servem á mesa.
Mordomo.
\section{Arquitrovão}
\begin{itemize}
\item {Grp. gram.:m.}
\end{itemize}
\begin{itemize}
\item {Proveniência:(De \textunderscore archi...\textunderscore  + \textunderscore trovão\textunderscore )}
\end{itemize}
Antiga máquina de cobre, que arremessava projécteis com grande estrondo.
\section{Arquivar}
\begin{itemize}
\item {Grp. gram.:v. t.}
\end{itemize}
Recolher em arquivo.
Guardar; conservar.
\section{Arquiviola}
\begin{itemize}
\item {Grp. gram.:f.}
\end{itemize}
\begin{itemize}
\item {Proveniência:(De \textunderscore archi...\textunderscore  + \textunderscore viola\textunderscore )}
\end{itemize}
Antigo instrumento músico, que se compunha do uma espécie de cravo, a que se adaptava o maquinismo de uma viola.
\section{Arquivista}
\begin{itemize}
\item {Grp. gram.:m.}
\end{itemize}
Aquelle que tem arquivo a seu cargo.
\section{Arquivo}
\begin{itemize}
\item {Grp. gram.:m.}
\end{itemize}
\begin{itemize}
\item {Utilização:Fig.}
\end{itemize}
\begin{itemize}
\item {Proveniência:(Lat. \textunderscore archivum\textunderscore )}
\end{itemize}
Lugar, onde se guardam documentos escritos: \textunderscore Arquivo da Tôrre do Tombo\textunderscore .
Cartório.
Deposito.
Pessôa de grande memória.
\section{Arquivolta}
\begin{itemize}
\item {Grp. gram.:f.}
\end{itemize}
\begin{itemize}
\item {Utilização:Archit.}
\end{itemize}
Contôrno, que acompanha o arco.
(B. lat. \textunderscore archivoltum\textunderscore )
\section{Arquivulgar}
\begin{itemize}
\item {Grp. gram.:adj.}
\end{itemize}
Extremamente vulgar. Cf. Castilho, \textunderscore Sabichonas\textunderscore , 70.
\section{Arrabal}
\begin{itemize}
\item {Grp. gram.:m.}
\end{itemize}
\begin{itemize}
\item {Utilização:Ant.}
\end{itemize}
O mesmo que \textunderscore arrabalde\textunderscore .
\section{Arrabalde}
\begin{itemize}
\item {Grp. gram.:m.}
\end{itemize}
\begin{itemize}
\item {Proveniência:(Do ár. \textunderscore ar-rabadh\textunderscore )}
\end{itemize}
Cercanias; arredores.
Parte extrema de uma povoação.
Vizinhanças; aro.
\section{Arrabaldeiro}
\begin{itemize}
\item {Grp. gram.:m.}
\end{itemize}
\begin{itemize}
\item {Grp. gram.:Adj.}
\end{itemize}
Aquelle, que vive no arrabalde.
Relativo a arrabalde.
\section{Arrabeca}
\begin{itemize}
\item {Grp. gram.:f.}
\end{itemize}
\begin{itemize}
\item {Utilização:Ant.}
\end{itemize}
O mesmo que \textunderscore rabeca\textunderscore .
\section{Arrabeirar}
\begin{itemize}
\item {Grp. gram.:v. t.}
\end{itemize}
\begin{itemize}
\item {Utilização:Prov.}
\end{itemize}
\begin{itemize}
\item {Utilização:trasm.}
\end{itemize}
\begin{itemize}
\item {Utilização:Ext.}
\end{itemize}
Tirar as rabeiras a (os cereaes na eira).
Concluir.
\section{Arrabel}
\begin{itemize}
\item {Grp. gram.:m.}
\end{itemize}
O mesmo que \textunderscore arrabil\textunderscore .
\section{Arrábido}
\begin{itemize}
\item {Grp. gram.:m.}
\end{itemize}
Frade do convento da Arrábida.
Frade da mesma Ordem, mas noutro convento.
\section{Arrabil}
\begin{itemize}
\item {Grp. gram.:m.}
\end{itemize}
\begin{itemize}
\item {Proveniência:(Do ár. \textunderscore ar-rabeb\textunderscore )}
\end{itemize}
Antigo instrumento músico de uma ou duas cordas, entre os Árabes e, depois, de três, na Idade-Média.
\section{Arrabileiro}
\begin{itemize}
\item {Grp. gram.:m.}
\end{itemize}
Aquelle que tangia arrabil.
\section{Arrabilete}
\begin{itemize}
\item {fónica:lê}
\end{itemize}
\begin{itemize}
\item {Grp. gram.:m.}
\end{itemize}
Pequeno arrabil.
\section{Arrabio}
\begin{itemize}
\item {Grp. gram.:m.}
\end{itemize}
Ave ribeirinha, (\textunderscore dafila acuta\textunderscore , Lin.) e o mesmo que \textunderscore rabijunco\textunderscore .
\section{Arrabujar-se}
\begin{itemize}
\item {Grp. gram.:v. p.}
\end{itemize}
\begin{itemize}
\item {Proveniência:(De \textunderscore rabujar\textunderscore )}
\end{itemize}
Tornar-se rabugento.
\section{Arrabunhar}
\begin{itemize}
\item {Grp. gram.:v. t.}
\end{itemize}
\begin{itemize}
\item {Utilização:Prov.}
\end{itemize}
\begin{itemize}
\item {Utilização:beir.}
\end{itemize}
O mesmo que \textunderscore arranhar\textunderscore .
(Cp. gall. \textunderscore rabuñar\textunderscore )
\section{Arraca}
\begin{itemize}
\item {Grp. gram.:f.}
\end{itemize}
\begin{itemize}
\item {Utilização:T. da Índ. Port}
\end{itemize}
Aguardente destillada do melaço, da sura, e do arroz.
\section{Arracacha}
\begin{itemize}
\item {Grp. gram.:m.}
\end{itemize}
\begin{itemize}
\item {Utilização:Bras}
\end{itemize}
Gênero de umbellíferas alimentares.
\section{Arraçado}
\begin{itemize}
\item {Grp. gram.:adj.}
\end{itemize}
Semelhante a certa raça.
Que participa de certa raça. Cf. \textunderscore Bibl. da Gente do Campo\textunderscore , 268.
\section{Arraçar}
\begin{itemize}
\item {Grp. gram.:v. t.}
\end{itemize}
\begin{itemize}
\item {Grp. gram.:V. t.}
\end{itemize}
Conseguir (bôas crias), cruzando de bôa raça com outra que não era bôa.
Sêr de bôa raça.
Sair á raça dos pais.
\section{Arracimar-se}
\begin{itemize}
\item {Grp. gram.:v. p.}
\end{itemize}
Encher-se de racimos.
Tomar fórma de cacho.
\section{Arraçoamento}
\begin{itemize}
\item {Grp. gram.:m.}
\end{itemize}
Acto de \textunderscore arraçoar\textunderscore .
\section{Arraçoar}
\begin{itemize}
\item {Grp. gram.:v. t.}
\end{itemize}
Dar ração a.
Dividir em rações.
Alimentar.
\section{Arraes}
\begin{itemize}
\item {Grp. gram.:m.}
\end{itemize}
(V.arrais)
\section{Arrafim}
\begin{itemize}
\item {Grp. gram.:m.}
\end{itemize}
\begin{itemize}
\item {Utilização:Ant.}
\end{itemize}
Presumpção de valente.
\section{Arraia}
\begin{itemize}
\item {Grp. gram.:f.}
\end{itemize}
Peixe, o mesmo que \textunderscore raia\textunderscore ^1.
\section{Arraia}
\begin{itemize}
\item {Grp. gram.:f.}
\end{itemize}
Fronteira de um país, o mesmo que \textunderscore raia\textunderscore ^2.
\section{Arraia}
\begin{itemize}
\item {Grp. gram.:f.}
\end{itemize}
Plebe.
(Ár. \textunderscore ar-rai\textunderscore )
\section{Arraiada}
\begin{itemize}
\item {Grp. gram.:f.}
\end{itemize}
Acto de arraiar; aurora. Cf. Garrett, \textunderscore D. Branca\textunderscore , 106.
\section{Arraial}
\begin{itemize}
\item {Grp. gram.:m.}
\end{itemize}
\begin{itemize}
\item {Utilização:Bras. da Baía}
\end{itemize}
\begin{itemize}
\item {Utilização:Prov.}
\end{itemize}
\begin{itemize}
\item {Utilização:alg.}
\end{itemize}
Acampamento.
Agglomeração festiva de povo.
Lugar, em que se juntam romeiros e em que há tendas provisórias, abarracamentos de comestíveis, ornamentações, música, etc.
Aldeóla, lugarejo.
Conjunto de barracas ou pequenas casas á beira do mar ou de um rio, nas quaes se abrigam pescadores e os respectivos apparelhos. Cf. Baldaque, \textunderscore Pesc. em Portugal\textunderscore .
\section{Arraialesco}
\begin{itemize}
\item {Grp. gram.:adj.}
\end{itemize}
Relativo a arraial.
Próprio de arraial.
\section{Arraiano}
\begin{itemize}
\item {Grp. gram.:adj.}
\end{itemize}
\begin{itemize}
\item {Proveniência:(De \textunderscore arraia\textunderscore ^2)}
\end{itemize}
Que mora na arraia ou fronteira.
Que é natural da fronteira.
\section{Arraião}
\begin{itemize}
\item {Grp. gram.:m.}
\end{itemize}
(V.murta)
\section{Arraião}
\begin{itemize}
\item {Grp. gram.:m.}
\end{itemize}
\begin{itemize}
\item {Proveniência:(De \textunderscore arraia\textunderscore ^1)}
\end{itemize}
Peixe dos Açores.
\section{Arraiar}
\begin{itemize}
\item {Grp. gram.:v. i.}
\end{itemize}
(V. \textunderscore raiar\textunderscore ^1)
\section{Arraigada}
\begin{itemize}
\item {Grp. gram.:f.}
\end{itemize}
\begin{itemize}
\item {Utilização:Náut.}
\end{itemize}
\begin{itemize}
\item {Proveniência:(De \textunderscore arraigar\textunderscore )}
\end{itemize}
Base, por onde a língua se prende ao osso hyoide.
Parte, por onde qualquer membro do animal se prende ao corpo.
Garganta do mastro ou parte superior do mastro propriamente dito, reforçada por uma chapa de ferro.
Garganta dos amantilhos.
Correntes de ferro, que aguentam o cesto da gávea para a garganta do mastro real.
\section{Arraigado}
\begin{itemize}
\item {Grp. gram.:adj.}
\end{itemize}
\begin{itemize}
\item {Utilização:Prov.}
\end{itemize}
\begin{itemize}
\item {Proveniência:(De \textunderscore arraigar\textunderscore )}
\end{itemize}
Diz-se do indivíduo que, em determinada terra, possue bens de raiz.
(Colhido em Turquel)
\section{Arraigar}
\begin{itemize}
\item {Grp. gram.:v. t.}
\end{itemize}
\begin{itemize}
\item {Grp. gram.:V. i.}
\end{itemize}
\begin{itemize}
\item {Proveniência:(Do lat. \textunderscore radicare\textunderscore )}
\end{itemize}
Firmar pela raíz; enraïzar.
Lançar raizes.
\section{Arraigota}
\begin{itemize}
\item {Grp. gram.:f.}
\end{itemize}
\begin{itemize}
\item {Utilização:Prov.}
\end{itemize}
\begin{itemize}
\item {Utilização:alent.}
\end{itemize}
Tronco sêco ou raiz, boa para queimar.
(Cp. \textunderscore arraigar\textunderscore )
\section{Arraiar}
\begin{itemize}
\item {Grp. gram.:v.}
\end{itemize}
\begin{itemize}
\item {Utilização:t. Agr.}
\end{itemize}
Decotar a rama velha de (bacêllo).
\section{Arrais}
\begin{itemize}
\item {Grp. gram.:m.}
\end{itemize}
\begin{itemize}
\item {Proveniência:(Do ár. \textunderscore ar-rai\textunderscore )}
\end{itemize}
Aquelle que commanda um barco.
Patrão de lancha.
\section{Arráiz}
\begin{itemize}
\item {Grp. gram.:m.}
\end{itemize}
\begin{itemize}
\item {Proveniência:(Do ár. \textunderscore ar-rai\textunderscore )}
\end{itemize}
Aquelle que commanda um barco.
Patrão de lancha.
\section{Arralentar}
\begin{itemize}
\item {Grp. gram.:v. t.}
\end{itemize}
Tornar ralo.
Desbastar (plantações).
\section{Arramada}
\begin{itemize}
\item {Grp. gram.:f.}
\end{itemize}
O mesmo que \textunderscore ramada\textunderscore .
\section{Arramalhar}
\begin{itemize}
\item {Grp. gram.:v. i.}
\end{itemize}
O mesmo que \textunderscore ramalhar\textunderscore .
Esconder-se debaixo de ramos, (falando-se de reptis).
Agitar-se na rede, (falando-se de peixes).
Chegar quási, aproximar-se, orçar: \textunderscore já arramalha pelos seus cincoenta annos\textunderscore .
\section{Arramar}
\begin{itemize}
\item {Grp. gram.:v. t.}
\end{itemize}
O mesmo que \textunderscore derramar\textunderscore  e \textunderscore espalhar\textunderscore .
E o mesmo que \textunderscore enramar\textunderscore .
\section{Arramar-se}
\begin{itemize}
\item {Grp. gram.:v. p.}
\end{itemize}
\begin{itemize}
\item {Utilização:Ant.}
\end{itemize}
\begin{itemize}
\item {Proveniência:(De \textunderscore rama\textunderscore )}
\end{itemize}
Abrigar-se; recolher-se, acolher-se.
\section{Arrampadoiro}
\begin{itemize}
\item {Grp. gram.:m.}
\end{itemize}
\begin{itemize}
\item {Utilização:Ant.}
\end{itemize}
\begin{itemize}
\item {Proveniência:(De \textunderscore rampa\textunderscore )}
\end{itemize}
Terra inculta, susceptível de sêr arroteada.
Vertente, encosta.
\section{Arrampadouro}
\begin{itemize}
\item {Grp. gram.:m.}
\end{itemize}
\begin{itemize}
\item {Utilização:Ant.}
\end{itemize}
\begin{itemize}
\item {Proveniência:(De \textunderscore rampa\textunderscore )}
\end{itemize}
Terra inculta, susceptível de sêr arroteada.
Vertente, encosta.
\section{Arran}
\begin{itemize}
\item {Grp. gram.:f.}
\end{itemize}
\begin{itemize}
\item {Utilização:Pop.}
\end{itemize}
O mesmo que \textunderscore ran\textunderscore :«\textunderscore um pobre velho cria arrans na barriga se bebe água\textunderscore ». Filinto.
\section{Arranca}
\begin{itemize}
\item {Grp. gram.:f.}
\end{itemize}
\begin{itemize}
\item {Utilização:Prov.}
\end{itemize}
Acto de \textunderscore arrancar\textunderscore .
Haste de planta ou vergôntea que se arrancou do chão: \textunderscore carvão de arranca\textunderscore .
Ramo ou galho que, com a mão, se separou da árvore.
\section{Arrancada}
\begin{itemize}
\item {Grp. gram.:f.}
\end{itemize}
Acto de arrancar.
Saída violenta.
Movimento inesperado.
Terreno, donde se arrancaram raizes, para sêr cultivado.
Briga.
Expedição militar.
\section{Arrancadamente}
\begin{itemize}
\item {Grp. gram.:adv.}
\end{itemize}
De arrancada; impetuosamente.
\section{Arrancadela}
\begin{itemize}
\item {Grp. gram.:f.}
\end{itemize}
O mesmo que \textunderscore arrancada\textunderscore .
\section{Arrancado}
\begin{itemize}
\item {Grp. gram.:adj.}
\end{itemize}
Impetuoso.
Dirigido com fôrça. Cf. Filinto, \textunderscore D. Man.\textunderscore , I, 27 e 292.
\section{Arrancador}
\begin{itemize}
\item {Grp. gram.:m.}
\end{itemize}
Aquelle que arranca.
Utensílio para arrancar batatas.
\section{Arrancadura}
\begin{itemize}
\item {Grp. gram.:f.}
\end{itemize}
Acto de \textunderscore arrancar\textunderscore .
Porção que se arranca de uma vez.
\section{Arrancamento}
\begin{itemize}
\item {Grp. gram.:m.}
\end{itemize}
O mesmo que \textunderscore arrancada\textunderscore .
\section{Arrancanes}
\begin{itemize}
\item {Grp. gram.:m. pl.}
\end{itemize}
\begin{itemize}
\item {Utilização:Ant.}
\end{itemize}
Arrecadas.
\section{Arranca-pinheiros}
\begin{itemize}
\item {Grp. gram.:m.}
\end{itemize}
\begin{itemize}
\item {Utilização:Pop.}
\end{itemize}
O mesmo que \textunderscore elephante\textunderscore .
(Colhido na Bairrada)
\section{Arrancar}
\begin{itemize}
\item {Grp. gram.:v. t.}
\end{itemize}
\begin{itemize}
\item {Grp. gram.:V. i.}
\end{itemize}
\begin{itemize}
\item {Proveniência:(Do lat. \textunderscore eruncare\textunderscore )}
\end{itemize}
Tirar com fôrça.
Desarraigar: \textunderscore arrancar árvores\textunderscore .
Extorquir: \textunderscore arrancar dinheiro aos incautos\textunderscore .
Separar: \textunderscore arrancou-o da minha companhia\textunderscore .
Libertar: \textunderscore arrancar da prisão\textunderscore .
Obrigar a manifestar-se: \textunderscore arrancou-me a confissão\textunderscore .
Conseguir.
Pôr em fuga.
Saír de repente.
Avançar impetuosamente: \textunderscore arrancou para êlle\textunderscore .
Puxar de repente por um objecto: \textunderscore arrancou da espada\textunderscore .
Agonizar.
\section{Arranca-sonda}
\begin{itemize}
\item {Grp. gram.:f.}
\end{itemize}
Instrumento de mineração, para tirar as sondas de mina, quando estas se entalam ou se partem nos furos.
\section{Arranca-tubos}
\begin{itemize}
\item {Grp. gram.:m.}
\end{itemize}
Ferramenta de mineiro, composta de duas peças horizontaes e que a sonda tira dos furos, girando em certo sentido.
\section{Arranchar}
\begin{itemize}
\item {Grp. gram.:v. t.}
\end{itemize}
\begin{itemize}
\item {Grp. gram.:V. i.}
\end{itemize}
Reunir em ranchos.
Admittir á mesa das refeições.
Formar rancho.
Abandear-se; associar-se.
\section{Arranco}
\begin{itemize}
\item {Grp. gram.:m.}
\end{itemize}
Acto ou effeito de \textunderscore arrancar\textunderscore .
Movimento impetuoso para partir ou para acommeter.
Ímpeto.
Ânsia.
Agonia.
Arquejo.
\section{Arrancoar}
\begin{itemize}
\item {Grp. gram.:v. i.}
\end{itemize}
\begin{itemize}
\item {Utilização:Ant.}
\end{itemize}
\begin{itemize}
\item {Proveniência:(De \textunderscore arranco\textunderscore ?)}
\end{itemize}
Queixar-se.
Aggravar-se.
\section{Arrancorar-se}
\begin{itemize}
\item {Grp. gram.:v. p.}
\end{itemize}
\begin{itemize}
\item {Proveniência:(De \textunderscore rancor\textunderscore )}
\end{itemize}
Tornar-se rancoroso.
\section{Arranha}
\begin{itemize}
\item {Grp. gram.:f.}
\end{itemize}
\begin{itemize}
\item {Proveniência:(De \textunderscore arranhar\textunderscore ?)}
\end{itemize}
Apparelho para a pesca do polvo, usado na ria de Vigo.
\section{Arranhadela}
\begin{itemize}
\item {Grp. gram.:f.}
\end{itemize}
O mesmo que \textunderscore arranhadura\textunderscore .
\section{Arranhador}
\begin{itemize}
\item {Grp. gram.:m.}
\end{itemize}
Aquelle que arranha.
\section{Arranhadura}
\begin{itemize}
\item {Grp. gram.:f.}
\end{itemize}
Acto ou effeito de \textunderscore arranhar\textunderscore .
\section{Arranha-lobos}
\begin{itemize}
\item {Grp. gram.:m. pl.}
\end{itemize}
\begin{itemize}
\item {Utilização:Prov.}
\end{itemize}
\begin{itemize}
\item {Utilização:minh.}
\end{itemize}
Planta, (\textunderscore genista berberidea\textunderscore , Lge.).
\section{Arranhão}
\begin{itemize}
\item {Grp. gram.:m.}
\end{itemize}
O mesmo que \textunderscore arranhadura\textunderscore .
\section{Arranhar}
\begin{itemize}
\item {Grp. gram.:v. t.}
\end{itemize}
\begin{itemize}
\item {Grp. gram.:V. i.}
\end{itemize}
Ferir levemente com as unhas ou com a ponta de algum instrumento.
Tocar mal (um instrumento).
Conhecer pouco (uma língua, uma disciplina).
Sêr áspero.
(Cp. \textunderscore arañar\textunderscore )
\section{Arranhosa}
\begin{itemize}
\item {Grp. gram.:f.}
\end{itemize}
Nome, que se dava a uma planta, de cuja baga se extrahia tinta.
\section{Arranjadeiro}
\begin{itemize}
\item {Grp. gram.:adj.}
\end{itemize}
Methódico, cuidadoso. Cf. Castilho, \textunderscore Fausto\textunderscore , 215.
\section{Arranjadela}
\begin{itemize}
\item {Grp. gram.:f.}
\end{itemize}
O mesmo que \textunderscore arranjamento\textunderscore . Cf. Eça, \textunderscore P. Amaro\textunderscore , 505.
\section{Arranjamento}
\begin{itemize}
\item {Grp. gram.:m.}
\end{itemize}
Acto ou effeito de \textunderscore arranjar\textunderscore .
\section{Arranjar}
\begin{itemize}
\item {Grp. gram.:v. t.}
\end{itemize}
\begin{itemize}
\item {Proveniência:(Fr. \textunderscore arranger\textunderscore )}
\end{itemize}
Pôr em ordem; dispor: \textunderscore arranjar os livros na estante\textunderscore .
Obter: \textunderscore arranjar um emprêgo\textunderscore .
Conciliar.
Adornar.
\section{Arranjista}
\begin{itemize}
\item {Grp. gram.:m.}
\end{itemize}
\begin{itemize}
\item {Proveniência:(De \textunderscore arranjo\textunderscore )}
\end{itemize}
Homem activo, fura-vidas.
Especulador.
\section{Arranjo}
\begin{itemize}
\item {Grp. gram.:m.}
\end{itemize}
\begin{itemize}
\item {Utilização:Gal}
\end{itemize}
Acto de \textunderscore arranjar\textunderscore .
Economia.
Mobília da casa.
Aconchego.
Conveniência: \textunderscore faz-me arranjo\textunderscore .
A ordem, por que podem ser collocadas as letras, em Álgebra, relacionando umas com outras.
\section{Arranque}
\begin{itemize}
\item {Grp. gram.:m.}
\end{itemize}
\begin{itemize}
\item {Utilização:Archit.}
\end{itemize}
O mesmo que \textunderscore arranco\textunderscore .
Acto de avançar com ímpeto, (falando-se do toiro ou do toireiro).
Parte, onde começa a curvatura de uma abóbada.
Impulso da máquina do combóio, ou o acto de ella começar a marchar.
\section{...arrão}
\begin{itemize}
\item {Grp. gram.:suf. m.}
\end{itemize}
(designativo de aumento, grandeza): \textunderscore mansarrão\textunderscore .
\section{Arrapazado}
\begin{itemize}
\item {Grp. gram.:adj.}
\end{itemize}
Que tem modos de rapaz.
\section{Arrapazar-se}
\begin{itemize}
\item {Grp. gram.:v. p.}
\end{itemize}
Adquirir maneiras de rapaz.
\section{Arrapinar}
\textunderscore v. t.\textunderscore  (e der.)
O mesmo que \textunderscore rapinar\textunderscore .
\section{Arraposado}
\begin{itemize}
\item {Grp. gram.:adj.}
\end{itemize}
Que tem manhas de raposa.
\section{Arraposar-se}
\begin{itemize}
\item {Grp. gram.:v. p.}
\end{itemize}
Têr manhas como a raposa.
Usar fingimento ou velhacaria.
\section{Arraquir}
\begin{itemize}
\item {Grp. gram.:m.}
\end{itemize}
\begin{itemize}
\item {Utilização:Prov.}
\end{itemize}
\begin{itemize}
\item {Utilização:beir.}
\end{itemize}
Pau com diversos galhos, que serve de cabide aos pastores.
(Do ár.?)
\section{Arrar}
\begin{itemize}
\item {Grp. gram.:v. t.}
\end{itemize}
Zangar?«\textunderscore Detem-se tanto que tenho medo arrar meu amo\textunderscore ». \textunderscore Eufrosina\textunderscore , 103.
\section{Arrarar}
\begin{itemize}
\item {Grp. gram.:v. t.}
\end{itemize}
\begin{itemize}
\item {Utilização:Des.}
\end{itemize}
Tornar raro.
\section{Arras}
\begin{itemize}
\item {Grp. gram.:f. pl.}
\end{itemize}
\begin{itemize}
\item {Proveniência:(Lat. \textunderscore arrha\textunderscore )}
\end{itemize}
Penhor.
Dinheiro, dado em sinal ou garantia de um contrato.
Bens dotaes, que o noivo assegura por contrato á esposa.
\section{Arrás}
\begin{itemize}
\item {Grp. gram.:m.}
\end{itemize}
\begin{itemize}
\item {Proveniência:(De \textunderscore Arras\textunderscore , n. p.)}
\end{itemize}
Tapeçaria antiga, para ornar paredes de salas ou galerias.
\section{Arrasa}
\begin{itemize}
\item {Grp. gram.:f.}
\end{itemize}
Acto de \textunderscore arrasar\textunderscore  (medidas). Cf. Aguiar, \textunderscore Processos de Vin.\textunderscore , 42.
\section{Arrasadeira}
\begin{itemize}
\item {Grp. gram.:f.}
\end{itemize}
O mesmo que \textunderscore rasoira\textunderscore .
\section{Arrasador}
\begin{itemize}
\item {Grp. gram.:m.}
\end{itemize}
\begin{itemize}
\item {Utilização:Prov.}
\end{itemize}
\begin{itemize}
\item {Utilização:dur.}
\end{itemize}
Aquelle que arrasa.
Official de chapeleiro, que arrasa o feltro.
O mesmo que \textunderscore rasoira\textunderscore .
\section{Arrasadura}
\begin{itemize}
\item {Grp. gram.:f.}
\end{itemize}
Acto de \textunderscore arrasar\textunderscore .
Aquillo que sobeja da medida, depois de rasa.
\section{Arrasamento}
\begin{itemize}
\item {Grp. gram.:m.}
\end{itemize}
Acto ou effeito de \textunderscore arrasar\textunderscore .
Demolição.
\section{Arrasar}
\begin{itemize}
\item {Grp. gram.:v. t.}
\end{itemize}
\begin{itemize}
\item {Utilização:Náut.}
\end{itemize}
Tornar raso.
Aplanar.
Nivelar com o chão.
Deitar ao chão; destruir; arruinar: \textunderscore arrasar uma casa\textunderscore .
Estragar.
Humilhar.
Encher completamente: \textunderscore arrasar um decalitro\textunderscore .
Perder de vista:«\textunderscore nas alturas do Pôrto, avistámos a cidade, e, minutos depois, arrasámo-la\textunderscore ».
\section{Arrasista}
\begin{itemize}
\item {Grp. gram.:m.}
\end{itemize}
Fabricante de panos de arrás.
\section{Arrasoirar}
\begin{itemize}
\item {Grp. gram.:v. i.}
\end{itemize}
Nivelar com rasoira, rasoirar.
\section{Arrasourar}
\begin{itemize}
\item {Grp. gram.:v. i.}
\end{itemize}
Nivelar com rasoira, rasoirar.
\section{Arrassar}
\begin{itemize}
\item {Grp. gram.:m.}
\end{itemize}
Corrente metállica, usada nas armações de pesca para atracar as redes.
\section{Arrassaz}
\begin{itemize}
\item {Grp. gram.:m.}
\end{itemize}
Tralha inferior da rede de certas embarcações de pesca, no Barreiro e no Seixal.
(Cp. \textunderscore arrassar\textunderscore )
\section{Arrasta}
\begin{itemize}
\item {Grp. gram.:f.}
\end{itemize}
\begin{itemize}
\item {Utilização:Prov.}
\end{itemize}
\begin{itemize}
\item {Utilização:trasm.}
\end{itemize}
\begin{itemize}
\item {Utilização:T. do Ribatejo}
\end{itemize}
\begin{itemize}
\item {Proveniência:(De \textunderscore arrastar\textunderscore )}
\end{itemize}
O mesmo que \textunderscore zorra\textunderscore ^1.
Corda, com que laçam os toiros nos cornos.
\section{Arrastadamente}
\begin{itemize}
\item {Grp. gram.:adv.}
\end{itemize}
Com arrastamento.
\section{Arrastadeira}
\begin{itemize}
\item {Grp. gram.:f.}
\end{itemize}
\begin{itemize}
\item {Proveniência:(De \textunderscore arrastar\textunderscore )}
\end{itemize}
Vaso quási chato, em que os doentes podem defecar, quando deitados.
\section{Arrastadeiro}
\begin{itemize}
\item {Grp. gram.:adj.}
\end{itemize}
Que arrasta; rasteiro.
\section{Arrastador}
\begin{itemize}
\item {Grp. gram.:m.}
\end{itemize}
\begin{itemize}
\item {Utilização:Mad}
\end{itemize}
O mesmo que \textunderscore ascensor\textunderscore .
\section{Arrastadura}
\begin{itemize}
\item {Grp. gram.:f.}
\end{itemize}
O mesmo que \textunderscore arrastamento\textunderscore .
\section{Arrastamento}
\begin{itemize}
\item {Grp. gram.:m.}
\end{itemize}
Acto de \textunderscore arrastar\textunderscore .
\section{Arrastão}
\begin{itemize}
\item {Grp. gram.:m.}
\end{itemize}
\begin{itemize}
\item {Utilização:Pesc.}
\end{itemize}
Esfôrço impetuoso para arrastar.
Repellão.
Saco de rede, que se arrasta pelo fundo da água, a reboque dos vapores de pesca.
\section{Arrasta-pé}
\begin{itemize}
\item {Grp. gram.:m.}
\end{itemize}
\begin{itemize}
\item {Utilização:Bras}
\end{itemize}
\begin{itemize}
\item {Utilização:chul.}
\end{itemize}
O mesmo que \textunderscore baile\textunderscore .
\section{Arrastar}
\begin{itemize}
\item {Grp. gram.:v. t.}
\end{itemize}
\begin{itemize}
\item {Grp. gram.:V. i.}
\end{itemize}
\begin{itemize}
\item {Grp. gram.:Loc.}
\end{itemize}
\begin{itemize}
\item {Utilização:fam.}
\end{itemize}
Levar de rastos.
Conduzir á fôrça.
Attrahir.
Mover com difficuldade.
Compelir.
Ir de rojo; rastejar.
\textunderscore Arrastar a asa\textunderscore , galantear, fazer namôro. Cf. B. Pato, \textunderscore Paquita\textunderscore , 165.
\section{Arraste}
\begin{itemize}
\item {Grp. gram.:m.}
\end{itemize}
O mesmo que \textunderscore arrasto\textunderscore .
\section{Arrasto}
\begin{itemize}
\item {Grp. gram.:m.}
\end{itemize}
Acto de \textunderscore arrastar\textunderscore .
Miséria.
Apparelho volante da rede de arrastar, composto de saco e alares.
\section{Arrastrar}
\textunderscore v. t.\textunderscore  (e der.)
O mesmo que \textunderscore arrastar\textunderscore . Cf. Cast., \textunderscore Metam.\textunderscore  124; \textunderscore Eufrosina\textunderscore , 338; \textunderscore Luz e Calor\textunderscore , 13.
\section{Arrastre}
\begin{itemize}
\item {Grp. gram.:m.}
\end{itemize}
\begin{itemize}
\item {Proveniência:(De \textunderscore arrastrar\textunderscore ?)}
\end{itemize}
Apparelho cylíndrico, em que se põe o minério argentífero para o reduzir a pó e peneirá-lo.
\section{Arrate}
\begin{itemize}
\item {Grp. gram.:m.}
\end{itemize}
Fórma pop. de arrátel.
\section{Arrátel}
\begin{itemize}
\item {Grp. gram.:m.}
\end{itemize}
\begin{itemize}
\item {Proveniência:(Do ár. \textunderscore ar-ratle\textunderscore )}
\end{itemize}
Pêso antigo, equivalente a 459 grammas.
\section{Arratelar}
\begin{itemize}
\item {Grp. gram.:v. t.}
\end{itemize}
Pesar aos arráteis.
Dividir em porções de arrátel.
\section{Arratem}
\begin{itemize}
\item {Grp. gram.:m.}
\end{itemize}
\begin{itemize}
\item {Utilização:Ant.}
\end{itemize}
O mesmo que \textunderscore arrátel\textunderscore . Cf. \textunderscore Peregrinação\textunderscore , XCVII.
\section{Arrazoadamente}
\begin{itemize}
\item {Grp. gram.:adv.}
\end{itemize}
Com razão.
Com arrazoamento.
\section{Arrazoado}
\begin{itemize}
\item {Grp. gram.:m.}
\end{itemize}
\begin{itemize}
\item {Proveniência:(De \textunderscore arrazoar\textunderscore )}
\end{itemize}
Discurso.
Defesa.
\section{Arrazoador}
\begin{itemize}
\item {Grp. gram.:m.}
\end{itemize}
Aquelle que arrazoa.
\section{Arrazoamento}
\begin{itemize}
\item {Grp. gram.:m.}
\end{itemize}
Acto de \textunderscore arrazoar\textunderscore .
\section{Arrazoar}
\begin{itemize}
\item {Grp. gram.:v. t.}
\end{itemize}
\begin{itemize}
\item {Grp. gram.:V. i.}
\end{itemize}
\begin{itemize}
\item {Proveniência:(De \textunderscore razão\textunderscore )}
\end{itemize}
Defender ou expor, allegando razões.
Censurar.
Discorrer.
Conversar.
Discutir.
\section{Arre!}
\begin{itemize}
\item {Grp. gram.:Interj.}
\end{itemize}
\begin{itemize}
\item {Utilização:pleb.}
\end{itemize}
\begin{itemize}
\item {Proveniência:(Do ár. \textunderscore harre\textunderscore , t. expletivo)}
\end{itemize}
(designativa de cólera ou enfado, e expressão com que se incitam as bêstas a caminhar).
\section{Arreador}
\begin{itemize}
\item {Grp. gram.:m.}
\end{itemize}
\begin{itemize}
\item {Utilização:Bras}
\end{itemize}
O mesmo que \textunderscore arreeiro\textunderscore .
\section{Arreamento}
\begin{itemize}
\item {Grp. gram.:m.}
\end{itemize}
Mobília.
Adereços.
Acto de \textunderscore arrear\textunderscore ^1.
\section{Arrear}
\begin{itemize}
\item {Grp. gram.:v. t.}
\end{itemize}
Pôr os arreios a; apparelhar.
Adereçar.
Mobilar.
\section{Arrear}
\begin{itemize}
\item {Grp. gram.:v. t.}
\end{itemize}
\begin{itemize}
\item {Utilização:Náut.}
\end{itemize}
\begin{itemize}
\item {Grp. gram.:V. i.}
\end{itemize}
Baixar. Amainar: \textunderscore arrear velas\textunderscore .
Largar successivamente, a pouco e pouco, (um cabo, uma linha, uma rede).
Desviar, inclinar:«\textunderscore o chapéu arreado para a nuca\textunderscore ». Camillo, \textunderscore Hom. de Brios\textunderscore , 54.
\section{Arrearia}
\begin{itemize}
\item {Grp. gram.:f.}
\end{itemize}
Casa de arreeiros.
Arreeirada.
Vida de arreeiros. Cf. J. Castilho, \textunderscore Lisb. Ant.\textunderscore 
\section{Arreata}
\begin{itemize}
\item {Grp. gram.:f.}
\end{itemize}
\begin{itemize}
\item {Utilização:Açor}
\end{itemize}
\begin{itemize}
\item {Proveniência:(De \textunderscore arreatar\textunderscore )}
\end{itemize}
Correia, corda ou cabresto, com que se conduzem as bêstas.
O mesmo que \textunderscore mastro\textunderscore .
\section{Arreatadura}
\begin{itemize}
\item {Grp. gram.:f.}
\end{itemize}
Acto de \textunderscore arreatar\textunderscore .
\section{Arreatar}
\begin{itemize}
\item {Grp. gram.:v. t.}
\end{itemize}
\begin{itemize}
\item {Proveniência:(De \textunderscore reatar\textunderscore )}
\end{itemize}
Atar com muitas voltas.
Prender com arreata.
\section{Arreaz}
\begin{itemize}
\item {Grp. gram.:f.}
\end{itemize}
Fivela, por onde passam as correias dos estribos.
(Talvez do ár. \textunderscore orua\textunderscore )
\section{Arrebadela}
\begin{itemize}
\item {Grp. gram.:f.}
\end{itemize}
Pequena armadilha de pesca com anzoes que sobrenadam.
(Por \textunderscore arribadela\textunderscore , de \textunderscore arriba\textunderscore ?)
\section{Arrebanhador}
\begin{itemize}
\item {Grp. gram.:m.}
\end{itemize}
Aquelle que arrebanha.
\section{Arrebanhar}
\begin{itemize}
\item {Grp. gram.:v. t.}
\end{itemize}
Juntar em rebanho.
Reunir.
\section{Arrebanhar}
\begin{itemize}
\item {Grp. gram.:v. t.}
\end{itemize}
Praticar o arrebanho em (terras semeadas).
\section{Arrebanho}
\begin{itemize}
\item {Grp. gram.:m.}
\end{itemize}
\begin{itemize}
\item {Utilização:T. da Bairrada}
\end{itemize}
Operação agrícola, em que o arado leva atravessada na traseira do temão uma vassoira que aplana os camalhões e cobre as sementes, á proporção que o arado vai abrindo sulcos na terra já semeada.
(Por \textunderscore abarranho\textunderscore , de \textunderscore barrer\textunderscore , por \textunderscore varrer\textunderscore ?)
\section{Arrebatadamente}
\begin{itemize}
\item {Grp. gram.:adv.}
\end{itemize}
Com arrebatamento.
\section{Arrebatado}
\begin{itemize}
\item {Grp. gram.:adj.}
\end{itemize}
\begin{itemize}
\item {Proveniência:(De \textunderscore arrebatar\textunderscore )}
\end{itemize}
Irritado.
Irascível.
\section{Arrebatador}
\begin{itemize}
\item {Grp. gram.:m.}
\end{itemize}
\begin{itemize}
\item {Grp. gram.:Adj.}
\end{itemize}
Aquelle que arrebata.
Que arrebata.
Que causa enthusiasmo ou êxtase.
\section{Arrebatamento}
\begin{itemize}
\item {Grp. gram.:m.}
\end{itemize}
Furor súbito.
Excitação.
Enlêvo.
Acto de \textunderscore arrebatar\textunderscore .
\section{Arrebatante}
\begin{itemize}
\item {Grp. gram.:adj.}
\end{itemize}
\begin{itemize}
\item {Proveniência:(De \textunderscore arrebatar\textunderscore )}
\end{itemize}
Diz-se, em Heráldica, do lobo, em relação á raposa.
\section{Arrebatapunhadas}
\begin{itemize}
\item {Grp. gram.:m.}
\end{itemize}
\begin{itemize}
\item {Utilização:Fam.}
\end{itemize}
Valentão; espadachim.
Desordeiro.
\section{Arrebatar}
\begin{itemize}
\item {Grp. gram.:v. t.}
\end{itemize}
Tirar com violência.
Roubar: \textunderscore arrebataram-lhe o filho\textunderscore .
Irritar.
Levar de repente.
Arrancar.
Maravilhar; extasiar: \textunderscore êste panorama arrebata-me\textunderscore .
(Cast. \textunderscore arrebatar\textunderscore )
\section{Arrebate}
\begin{itemize}
\item {Grp. gram.:m.}
\end{itemize}
O mesmo que \textunderscore arrebato\textunderscore .
\section{Arrebatinha}
\begin{itemize}
\item {Grp. gram.:f.}
\end{itemize}
Acto de \textunderscore arrebatar\textunderscore .
(Cp. \textunderscore rebatinha\textunderscore )
\section{Arrebato}
\begin{itemize}
\item {Grp. gram.:m.}
\end{itemize}
\begin{itemize}
\item {Utilização:Ant.}
\end{itemize}
\begin{itemize}
\item {Grp. gram.:Loc. adv.}
\end{itemize}
Acção de arrebatar.
De \textunderscore arrebato\textunderscore , arrebatadamente; de repente.
\section{Arrebatosamente}
\begin{itemize}
\item {Grp. gram.:adv.}
\end{itemize}
\begin{itemize}
\item {Utilização:Ant.}
\end{itemize}
O mesmo que \textunderscore arrebatadamente\textunderscore .
\section{Arrebém}
\begin{itemize}
\item {Grp. gram.:m.}
\end{itemize}
\begin{itemize}
\item {Utilização:Náut.}
\end{itemize}
Pequeno cabo, de vários usos a bórdo.
\section{Arrebenta-boi}
\begin{itemize}
\item {Grp. gram.:m.}
\end{itemize}
\begin{itemize}
\item {Utilização:Prov.}
\end{itemize}
\begin{itemize}
\item {Utilização:alent.}
\end{itemize}
Nome vulgar de duas espécies de jarro (planta).
O mesmo que \textunderscore luca\textunderscore ,^2 ou \textunderscore reineta\textunderscore .
\section{Arrebentação}
\begin{itemize}
\item {Grp. gram.:f.}
\end{itemize}
\begin{itemize}
\item {Utilização:Agr.}
\end{itemize}
Marulho das ondas contra a praia ou contra um recife.
Acto de \textunderscore arrebentar\textunderscore .
Acto de abotoar ou de lançar gomos. Cf. F. Lapa, \textunderscore Rev. Agr.\textunderscore 
\section{Arrebenta-cavallo}
\begin{itemize}
\item {Grp. gram.:m.}
\end{itemize}
Nome de uma planta brasileira, nociva aos cavallos.
\section{Arrebentadiço}
\begin{itemize}
\item {Grp. gram.:adj.}
\end{itemize}
Que é susceptível de arrebentar.
\section{Arrebentamento}
\begin{itemize}
\item {Grp. gram.:m.}
\end{itemize}
Estrondo daquillo que arrebenta.
Acto de \textunderscore arrebentar\textunderscore .
\section{Arrebentante}
\begin{itemize}
\item {Grp. gram.:adj.}
\end{itemize}
\begin{itemize}
\item {Utilização:Heráld.}
\end{itemize}
Diz-se do lobo ou raposa, na postura de leão rompente.
\section{Arrebentão}
\begin{itemize}
\item {Grp. gram.:m.}
\end{itemize}
(V. \textunderscore rebentão\textunderscore ^1)
\section{Arrebentar}
\begin{itemize}
\item {Grp. gram.:v. t.}
\end{itemize}
Estoirar.
O mesmo que \textunderscore rebentar\textunderscore .
Apparecer subitamente: \textunderscore já sabem a notícia que hoje arrebentou\textunderscore ?
\section{Arrebento}
\begin{itemize}
\item {Grp. gram.:m.}
\end{itemize}
(V.rebento)
\section{Arrebicado}
\begin{itemize}
\item {Grp. gram.:adj.}
\end{itemize}
Em que há arrebique ou affectação: \textunderscore estilo arrebicado\textunderscore .
\section{Arrebicar}
\begin{itemize}
\item {Grp. gram.:v. t.}
\end{itemize}
Ornar com arrebique.
Alindar, com affectação: \textunderscore a solteirona arrebicava-se muito\textunderscore .
\section{Arrebique}
\begin{itemize}
\item {Grp. gram.:m.}
\end{itemize}
\begin{itemize}
\item {Proveniência:(Do ár. \textunderscore rebique\textunderscore )}
\end{itemize}
Cosmético.
Enfeite exaggerado, ridículo.
\section{Arrebitado}
\begin{itemize}
\item {Grp. gram.:adj.}
\end{itemize}
Revirado para cima: \textunderscore nariz arrebitado\textunderscore .
\section{Arrebitar}
\begin{itemize}
\item {Grp. gram.:v. t.}
\end{itemize}
\begin{itemize}
\item {Utilização:Prov.}
\end{itemize}
\begin{itemize}
\item {Utilização:Ext.}
\end{itemize}
\begin{itemize}
\item {Proveniência:(De \textunderscore rebitar\textunderscore )}
\end{itemize}
Revirar para cima a ponta, a aba de: \textunderscore arrebitar o chapéu\textunderscore .
Lançar água por uma bica, (falando-se de fontes ou chafarizes).
Urinar para longe.
\section{Arrebite}
\begin{itemize}
\item {Grp. gram.:m.}
\end{itemize}
\begin{itemize}
\item {Utilização:Prov.}
\end{itemize}
\begin{itemize}
\item {Utilização:minh.}
\end{itemize}
Criança esperta, muito viva.
\section{Arrebito}
\begin{itemize}
\item {Grp. gram.:m.}
\end{itemize}
\begin{itemize}
\item {Proveniência:(De \textunderscore arrebitar\textunderscore )}
\end{itemize}
Configuração do nariz arrebitado.
Soberba.
\section{Arrebol}
\begin{itemize}
\item {Grp. gram.:m.}
\end{itemize}
\begin{itemize}
\item {Grp. gram.:Pl.}
\end{itemize}
Vermelhidão da aurora.
Rosiclér.
Côr avermelhada do poente, em seguida ao sol-posto.
\textunderscore Arreboes\textunderscore  ou \textunderscore arreboles\textunderscore :«\textunderscore arreboles do sol posto\textunderscore ». G. Resende, \textunderscore Cancion.\textunderscore 
(Cast. \textunderscore arrebol\textunderscore )
\section{Arrebolar}
\begin{itemize}
\item {Grp. gram.:v. t.}
\end{itemize}
\begin{itemize}
\item {Proveniência:(De \textunderscore rebolar\textunderscore )}
\end{itemize}
Tornar redondo; dar feitio de bola a.
\section{Arrebolar}
\begin{itemize}
\item {Grp. gram.:v. t.}
\end{itemize}
Dar côr de arrebol a.
\section{Arreburrinho}
\begin{itemize}
\item {fónica:á}
\end{itemize}
\begin{itemize}
\item {Grp. gram.:m.}
\end{itemize}
\begin{itemize}
\item {Utilização:Fam.}
\end{itemize}
\begin{itemize}
\item {Proveniência:(De \textunderscore arre\textunderscore  + \textunderscore burrinho\textunderscore , dem. de \textunderscore burro\textunderscore )}
\end{itemize}
Brinquedo de rapazes, que se baloiçam numa prancha movida sôbre um ponto de apoio.
Pessôa, que obedece cegamente a outra.
\section{Arreçã}
\begin{itemize}
\item {Grp. gram.:f.}
\end{itemize}
\begin{itemize}
\item {Utilização:Prov.}
\end{itemize}
\begin{itemize}
\item {Utilização:trasm.}
\end{itemize}
O mesmo que \textunderscore arçan\textunderscore . Cf. G. Junqueiro, \textunderscore Simples\textunderscore .
\section{Arrecabe}
\begin{itemize}
\item {Grp. gram.:m.}
\end{itemize}
Corda com que se puxa a rede de arrastar.
\section{Arrecada}
\begin{itemize}
\item {Grp. gram.:f.}
\end{itemize}
\begin{itemize}
\item {Proveniência:(Do ár. \textunderscore al-carrata\textunderscore )}
\end{itemize}
Enfeite, de ordinario em fórma de argola, para as orelhas.
\section{Arrecadação}
\begin{itemize}
\item {Grp. gram.:f.}
\end{itemize}
Lugar, onde se arrecada.
Guarda; depósito.
Cobrança: \textunderscore arrecadação de impostos\textunderscore .
Prisão.
Acto de \textunderscore arrecadar\textunderscore .
\section{Arrecadador}
\begin{itemize}
\item {Grp. gram.:m.}
\end{itemize}
Aquelle que arrecada.
\section{Arrecadamento}
\begin{itemize}
\item {Grp. gram.:m.}
\end{itemize}
Acto de \textunderscore arrecadar\textunderscore .
\section{Arrecadar}
\begin{itemize}
\item {Grp. gram.:v. t.}
\end{itemize}
Têr em lugar seguro.
Depositar.
Guardar.
Cobrar.
Tomar posse de.
Alcançar.
Prender.
(B. lat. \textunderscore recaptare\textunderscore )
\section{Arrecádea}
\begin{itemize}
\item {Grp. gram.:f.}
\end{itemize}
\begin{itemize}
\item {Utilização:Pop.}
\end{itemize}
O mesmo que \textunderscore arrecada\textunderscore .
\section{Arreçal}
\begin{itemize}
\item {Grp. gram.:m.}
\end{itemize}
\begin{itemize}
\item {Utilização:Pesc.}
\end{itemize}
Tralha inferior dos quartos das armações redondas de Peniche.
\section{Arreçanhal}
\begin{itemize}
\item {Grp. gram.:m.}
\end{itemize}
O mesmo que \textunderscore arçanhal\textunderscore .
\section{Arrecear}
\begin{itemize}
\item {Grp. gram.:v. t.}
\end{itemize}
(V.recear)
\section{Arrecear-se}
\begin{itemize}
\item {Grp. gram.:v. p.}
\end{itemize}
\begin{itemize}
\item {Proveniência:(De \textunderscore recear\textunderscore )}
\end{itemize}
Têr receio.
\section{Arreceber}
\begin{itemize}
\item {Grp. gram.:v. t.}
\end{itemize}
\begin{itemize}
\item {Utilização:Pop.}
\end{itemize}
O mesmo que \textunderscore receber\textunderscore : \textunderscore arrecebeu-a por mulher\textunderscore .
\section{Arreceio}
\begin{itemize}
\item {Grp. gram.:m.}
\end{itemize}
(V.receio)
\section{Arrecife}
\begin{itemize}
\item {Grp. gram.:m.}
\end{itemize}
(V.recife)
\section{Arrecolher}
\begin{itemize}
\item {Grp. gram.:v. t.}
\end{itemize}
\begin{itemize}
\item {Utilização:Pop.}
\end{itemize}
O mesmo que \textunderscore recolher\textunderscore .
\section{Arrecova}
\begin{itemize}
\item {Grp. gram.:f.}
\end{itemize}
\begin{itemize}
\item {Proveniência:(De \textunderscore recova\textunderscore )}
\end{itemize}
Bagagens, carga.
\section{Arrecuar}
\textunderscore v. i.\textunderscore  (e der.)
O mesmo que \textunderscore recuar\textunderscore , etc.
\section{Arrecuas}
\begin{itemize}
\item {Grp. gram.:f. pl.}
\end{itemize}
Us. na loc. adv. \textunderscore ás arrecuas\textunderscore , andando para trás, recuando.
\section{Arreda!}
\begin{itemize}
\item {Grp. gram.:interj.}
\end{itemize}
Fóra! Para trás!
(Imper. de \textunderscore arredar\textunderscore )
\section{Arredadamente}
\begin{itemize}
\item {Grp. gram.:adv.}
\end{itemize}
\begin{itemize}
\item {Proveniência:(De \textunderscore arredado\textunderscore )}
\end{itemize}
Em lugar distante; distanciadamente.
De longe.
Raras vezes.
\section{Arredado}
\begin{itemize}
\item {Grp. gram.:adj.}
\end{itemize}
\begin{itemize}
\item {Proveniência:(De \textunderscore arredar\textunderscore )}
\end{itemize}
Afastado; distante: \textunderscore num lugar arredado da cidade\textunderscore .
\section{Arredamento}
\begin{itemize}
\item {Grp. gram.:m.}
\end{itemize}
Acto de \textunderscore arredar\textunderscore .
\section{Arredar}
\begin{itemize}
\item {Grp. gram.:v. t.}
\end{itemize}
\begin{itemize}
\item {Proveniência:(Do lat. \textunderscore retro\textunderscore )}
\end{itemize}
Remover para trás.
Afastar.
Separar.
Pôr de parte.
\section{Arredavel}
\begin{itemize}
\item {Grp. gram.:adj.}
\end{itemize}
Que se póde arredar.
\section{Arredio}
\begin{itemize}
\item {Grp. gram.:adj.}
\end{itemize}
\begin{itemize}
\item {Proveniência:(Do lat. hyp. \textunderscore erralivus\textunderscore )}
\end{itemize}
Desviado.
Separado.
Que anda longe dos lugares que frequentava, ou da companhia que tinha.
Tresmalhado.
\section{Arrédo}
\begin{itemize}
\item {Grp. gram.:adv.}
\end{itemize}
\begin{itemize}
\item {Utilização:Ant.}
\end{itemize}
O mesmo que \textunderscore arredadamente\textunderscore .
\section{Arrêdo}
\begin{itemize}
\item {Grp. gram.:adj.}
\end{itemize}
\begin{itemize}
\item {Utilização:Ant.}
\end{itemize}
O mesmo que \textunderscore arredado\textunderscore .
(\textunderscore Part. irr.\textunderscore  de \textunderscore arredar\textunderscore )
\section{Arredoiça}
\begin{itemize}
\item {Grp. gram.:f.}
\end{itemize}
O mesmo que \textunderscore redoiça\textunderscore .
\section{Arredoma}
\begin{itemize}
\item {Grp. gram.:f.}
\end{itemize}
\begin{itemize}
\item {Utilização:Des.}
\end{itemize}
O mesmo que \textunderscore redoma\textunderscore . Cf. Usque, \textunderscore Tribulações\textunderscore , 24, v.^o
\section{Arre-dom-macho}
\begin{itemize}
\item {Grp. gram.:m.}
\end{itemize}
Planta da serra de Cintra.
\section{Arredonda}
\begin{itemize}
\item {Grp. gram.:adv.}
\end{itemize}
\begin{itemize}
\item {Utilização:Ant.}
\end{itemize}
\begin{itemize}
\item {Proveniência:(De \textunderscore redondo\textunderscore )}
\end{itemize}
Em redór; em tôrno.
\section{Arredondado}
\begin{itemize}
\item {Grp. gram.:adj.}
\end{itemize}
\begin{itemize}
\item {Proveniência:(De \textunderscore arredondar\textunderscore )}
\end{itemize}
Que tem fórma circular ou fórma de bóla.
\section{Arredondamento}
\begin{itemize}
\item {Grp. gram.:m.}
\end{itemize}
Acto de \textunderscore arredondar\textunderscore .
Divisão administrativa, em França. Cf. Castilho, \textunderscore Colóq. Ald.\textunderscore , 307 e 390.
\section{Arredondar}
\begin{itemize}
\item {Grp. gram.:v. t.}
\end{itemize}
Tornar redondo.
Dar fórma circular a.
Completar, dar número redondo a: \textunderscore arredondar uma quantia\textunderscore .
Aperfeiçoar.
Pôr em relêvo.
Tornar harmonioso (o período, a phrase).
\section{Arredór}
\begin{itemize}
\item {Grp. gram.:adv.}
\end{itemize}
\begin{itemize}
\item {Grp. gram.:Adj.}
\end{itemize}
\begin{itemize}
\item {Grp. gram.:Pl.}
\end{itemize}
\begin{itemize}
\item {Proveniência:(De \textunderscore redór\textunderscore )}
\end{itemize}
Ao redór, em redór, em volta.
Circunvizinho:«\textunderscore os campos arredores de Safim\textunderscore ». Filinto, \textunderscore D. Man.\textunderscore , II, 304.
Arrabaldes; aros; subúrbios: \textunderscore nos arredóres da cidade\textunderscore .
\section{Arredrar}
\begin{itemize}
\item {Grp. gram.:v. t.}
\end{itemize}
(V.redrar)
\section{Arreeirada}
\begin{itemize}
\item {Grp. gram.:f.}
\end{itemize}
Acto ou palavra indecorosa, própria de arreeiro.
\section{Arreeirado}
\begin{itemize}
\item {Grp. gram.:adj.}
\end{itemize}
Que tem modos de arreeiro.
\section{Arreeirático}
\begin{itemize}
\item {Grp. gram.:adj.}
\end{itemize}
Próprio de arreeiro.
\section{Arreeiro}
\begin{itemize}
\item {Grp. gram.:m.}
\end{itemize}
\begin{itemize}
\item {Utilização:Fig.}
\end{itemize}
\begin{itemize}
\item {Proveniência:(De \textunderscore arre\textunderscore )}
\end{itemize}
Aquelle que conduz ou guia bêstas de aluguel; almocreve.
Alquilador.
Homem mal educado, que usa palavras indecorosas ou insultantes.
\section{Arrefanhar}
\begin{itemize}
\item {Grp. gram.:v. t.}
\end{itemize}
Tirar das mãos de outrem com violência.
(Cp. \textunderscore arrepanhar\textunderscore )
\section{Arrefeçar}
\begin{itemize}
\item {Grp. gram.:v. t.}
\end{itemize}
\begin{itemize}
\item {Proveniência:(De \textunderscore refece\textunderscore )}
\end{itemize}
Aviltar. Vender por baixo preço.
\section{Arrefecedor}
\begin{itemize}
\item {Grp. gram.:adj.}
\end{itemize}
Que faz arrefecer. Cf. Eça, \textunderscore P. Amaro\textunderscore , 306.
\section{Arrefecer}
\begin{itemize}
\item {fónica:fé-cêr}
\end{itemize}
\begin{itemize}
\item {Grp. gram.:v. i.}
\end{itemize}
\begin{itemize}
\item {Grp. gram.:V. t.}
\end{itemize}
\begin{itemize}
\item {Proveniência:(Do lat. \textunderscore refrigescere\textunderscore )}
\end{itemize}
Tornar-se frio; perder o calor: \textunderscore o tempo arrefeceu\textunderscore .
Desanimar.
Perder a energia: \textunderscore arrefeceu-lhe o entusiasmo\textunderscore .
Moderar (o zêlo, o enthusiasmo, a actividade de).
\section{Arrefecido}
\begin{itemize}
\item {fónica:fé}
\end{itemize}
\begin{itemize}
\item {Grp. gram.:adj.}
\end{itemize}
Que arrefeceu.
\section{Arrefecimento}
\begin{itemize}
\item {fónica:fé}
\end{itemize}
\begin{itemize}
\item {Grp. gram.:m.}
\end{itemize}
Acto de \textunderscore arrefecer\textunderscore .
\section{Arrefém}
\begin{itemize}
\item {Grp. gram.:m.}
\end{itemize}
\begin{itemize}
\item {Utilização:Ant.}
\end{itemize}
O mesmo que \textunderscore refém\textunderscore .
\section{Arrefentado}
\begin{itemize}
\item {Grp. gram.:adj.}
\end{itemize}
Um tanto frio.
\section{Arrefentar}
\begin{itemize}
\item {Grp. gram.:v. t.}
\end{itemize}
\begin{itemize}
\item {Utilização:Pop.}
\end{itemize}
\begin{itemize}
\item {Proveniência:(De \textunderscore arrefecer\textunderscore )}
\end{itemize}
Tornar um tanto frio.
\section{Arreferir}
\begin{itemize}
\item {Grp. gram.:v. t.}
\end{itemize}
\begin{itemize}
\item {Utilização:Ant.}
\end{itemize}
O mesmo que \textunderscore referir\textunderscore .
\section{Arregaçada}
\begin{itemize}
\item {Grp. gram.:f.}
\end{itemize}
\begin{itemize}
\item {Proveniência:(De \textunderscore arregaçar\textunderscore )}
\end{itemize}
Porção, com que se enche o regaço.
Quantidade, que póde conter-se no regaço.
Grande porção.
\section{Arregaçar}
\begin{itemize}
\item {Grp. gram.:v. t.}
\end{itemize}
\begin{itemize}
\item {Proveniência:(De \textunderscore regaço\textunderscore )}
\end{itemize}
Puxar, dobrar para cima (o vestuário, a parte anterior das mangas).
Colher as bordas de (um vestido).
\section{Arregacha}
\begin{itemize}
\item {Grp. gram.:f.}
\end{itemize}
\begin{itemize}
\item {Utilização:Prov.}
\end{itemize}
\begin{itemize}
\item {Utilização:alent.}
\end{itemize}
Ave, semelhante á gallinhola, mas mais pequena.
\section{Arregalado}
\begin{itemize}
\item {Grp. gram.:adj.}
\end{itemize}
Muito aberto, com admiração ou espanto: \textunderscore olhos arregalados\textunderscore .
\section{Arregalar}
\begin{itemize}
\item {Grp. gram.:v. t.}
\end{itemize}
\begin{itemize}
\item {Proveniência:(De \textunderscore regalo\textunderscore )}
\end{itemize}
Abrir muito, com satisfação ou espanto (os olhos).
\section{Arreganhada}
\begin{itemize}
\item {Grp. gram.:f.}
\end{itemize}
Nome commum a vários peixes, da fam. dos esqualos.
Peixe plagióstomo, do gênero \textunderscore Bocage e Capello\textunderscore .
\section{Arreganhado}
\begin{itemize}
\item {Grp. gram.:adj.}
\end{itemize}
\begin{itemize}
\item {Utilização:Prov.}
\end{itemize}
Trêmulo de frio.
\section{Arreganhar}
\begin{itemize}
\item {Grp. gram.:v. t.}
\end{itemize}
\begin{itemize}
\item {Grp. gram.:V. i.}
\end{itemize}
\begin{itemize}
\item {Grp. gram.:V. p.}
\end{itemize}
\begin{itemize}
\item {Utilização:Prov.}
\end{itemize}
Mostrar (os dentes), abrindo os lábios, com expressão de riso ou de cólera.
Gretar: \textunderscore a fruta arreganhou\textunderscore .
Tremer com frio.
(Cast. \textunderscore regañar\textunderscore )
\section{Arreganho}
\begin{itemize}
\item {Grp. gram.:m.}
\end{itemize}
Acto de \textunderscore arreganhar\textunderscore .
Altivez; intrepidez: \textunderscore combater com arreganho\textunderscore .
\section{Arregateiras}
\begin{itemize}
\item {Grp. gram.:f. pl.}
\end{itemize}
\begin{itemize}
\item {Utilização:Prov.}
\end{itemize}
\begin{itemize}
\item {Utilização:trasm.}
\end{itemize}
\begin{itemize}
\item {Proveniência:(De \textunderscore rêgo\textunderscore )}
\end{itemize}
Pequenos animaes, que, como as toupeiras, revolvem as hortas e lameiros, formando montículos de terra.
Montículos de terra, formados por esses animaes ou pelas toupeiras.
\section{Arregimentado}
\begin{itemize}
\item {Grp. gram.:adj.}
\end{itemize}
\begin{itemize}
\item {Utilização:Fig.}
\end{itemize}
Que tem lugar num regimento.
Associado.
\section{Arregimentar}
\begin{itemize}
\item {Grp. gram.:v. t.}
\end{itemize}
Reunir em regimento, enfileirar.
Associar.
\section{Arregoar}
\begin{itemize}
\item {Grp. gram.:v. t.}
\end{itemize}
\begin{itemize}
\item {Grp. gram.:V. i.}
\end{itemize}
Abrir regos em.
Fender.
Tomar a fórma de rêgo; tornar-se cavo longitudinalmente.
\section{Arreguilar}
\begin{itemize}
\item {Grp. gram.:v. t.}
\end{itemize}
\begin{itemize}
\item {Utilização:Prov.}
\end{itemize}
\begin{itemize}
\item {Utilização:minh.}
\end{itemize}
O mesmo que \textunderscore arregalar\textunderscore .
\section{Arreigar}
\textunderscore v. t.\textunderscore  (e der.)
O mesmo que \textunderscore arraigar\textunderscore , etc.
\section{Arreio}
\begin{itemize}
\item {Grp. gram.:m.}
\end{itemize}
Apparelho de bêstas; jaez.
Enfeite.
(Cast. \textunderscore arreo\textunderscore )
\section{Arreitado}
\begin{itemize}
\item {Grp. gram.:adj.}
\end{itemize}
\begin{itemize}
\item {Utilização:Prov.}
\end{itemize}
\begin{itemize}
\item {Utilização:trasm.}
\end{itemize}
Garrido, aperaltado.
\section{Arreitar}
\begin{itemize}
\item {Grp. gram.:v. t.}
\end{itemize}
\begin{itemize}
\item {Utilização:Chul.}
\end{itemize}
\begin{itemize}
\item {Proveniência:(Do lat. \textunderscore arrectus\textunderscore )}
\end{itemize}
Excitar desejos venéreos em.
Estimular sensualmente.
\section{Arreiteta}
\begin{itemize}
\item {fónica:tê}
\end{itemize}
\begin{itemize}
\item {Grp. gram.:f.}
\end{itemize}
\begin{itemize}
\item {Utilização:Prov.}
\end{itemize}
Almotolia.
\section{Arrejeitar}
\begin{itemize}
\item {Grp. gram.:v. t.}
\end{itemize}
\begin{itemize}
\item {Proveniência:(De \textunderscore rejeitar\textunderscore )}
\end{itemize}
Arremessar para longe.
\section{Arrelhada}
\begin{itemize}
\item {Grp. gram.:f.}
\end{itemize}
\begin{itemize}
\item {Proveniência:(De \textunderscore relha\textunderscore )}
\end{itemize}
Raspadeira, instrumento de ferro, para limpar o arado.
\section{Arrelhador}
\begin{itemize}
\item {Grp. gram.:m.}
\end{itemize}
\begin{itemize}
\item {Utilização:Bras. do N}
\end{itemize}
\begin{itemize}
\item {Proveniência:(De \textunderscore arrelhar\textunderscore )}
\end{itemize}
Tira de coiro ou relho, com que se amarra o bezerro á perna da vaca, quando esta é ordenhada.
\section{Arrelhar}
\begin{itemize}
\item {Grp. gram.:v. t.}
\end{itemize}
\begin{itemize}
\item {Utilização:Bras. do N}
\end{itemize}
\begin{itemize}
\item {Proveniência:(De \textunderscore relho\textunderscore )}
\end{itemize}
Amarrar com relho ou tira de coiro (o bezerro) á vaca, quando esta é ordenhada.
\section{Arrelia}
\begin{itemize}
\item {Grp. gram.:f.}
\end{itemize}
Zanga.
Enguiço.
Máu agoiro.
\section{Arreliado}
\begin{itemize}
\item {Grp. gram.:adj.}
\end{itemize}
Que tem arrelia.
\section{Arreliador}
\begin{itemize}
\item {Grp. gram.:m.}
\end{itemize}
\begin{itemize}
\item {Grp. gram.:Adj.}
\end{itemize}
Aquelle que arrelia.
Que arrelia, que impacienta.
\section{Arreliante}
\begin{itemize}
\item {Grp. gram.:adj.}
\end{itemize}
\begin{itemize}
\item {Proveniência:(De \textunderscore arreliar\textunderscore )}
\end{itemize}
Que produz arrelia.
\section{Arreliar}
\begin{itemize}
\item {Grp. gram.:v. t.}
\end{itemize}
Fazer arrelia a.
Impacientar.
\section{Arrelica}
\begin{itemize}
\item {Grp. gram.:f.}
\end{itemize}
\begin{itemize}
\item {Utilização:Pop.}
\end{itemize}
O mesmo que \textunderscore arrelíquia\textunderscore .
\section{Arrelioso}
\begin{itemize}
\item {Grp. gram.:adj.}
\end{itemize}
Que causa arrelia; impertinente.
\section{Arrelique}
\begin{itemize}
\item {Grp. gram.:m.}
\end{itemize}
\begin{itemize}
\item {Utilização:Prov.}
\end{itemize}
\begin{itemize}
\item {Utilização:minh.}
\end{itemize}
Nome de uma planta, (\textunderscore vicia angustifolia\textunderscore , Lin.).
\section{Arrelíquia}
\begin{itemize}
\item {Grp. gram.:f.}
\end{itemize}
(V.relíquia)
\section{Arrelvado}
\begin{itemize}
\item {Grp. gram.:adj.}
\end{itemize}
Que tem relva; coberto de relva, de verdura ou de flôres:«\textunderscore o campo arrelvado de violetas\textunderscore ». Garrett, \textunderscore D. Branca\textunderscore , 114.
\section{Arrelvar-se}
\begin{itemize}
\item {Grp. gram.:v. p.}
\end{itemize}
Cobrir-se de relva; mostrar-se verdejante, (falando-se de campo ou terreno).
\section{Arremangado}
\begin{itemize}
\item {Grp. gram.:adj.}
\end{itemize}
Que arregaçou as mangas.
\section{Arremangar}
\begin{itemize}
\item {Grp. gram.:v. i.  e  p.}
\end{itemize}
\begin{itemize}
\item {Utilização:Prov.}
\end{itemize}
\begin{itemize}
\item {Utilização:alg.}
\end{itemize}
\begin{itemize}
\item {Proveniência:(De \textunderscore re...\textunderscore  + \textunderscore manga\textunderscore )}
\end{itemize}
Arregaçar as mangas.
Fingir que se trabalha.
\section{Arremansar-se}
\begin{itemize}
\item {Grp. gram.:v. p.}
\end{itemize}
Ficar em remanso.
\section{Arrematação}
\begin{itemize}
\item {Grp. gram.:f.}
\end{itemize}
Acto de \textunderscore arrematar\textunderscore .
\section{Arrematador}
\begin{itemize}
\item {Grp. gram.:m.}
\end{itemize}
\begin{itemize}
\item {Grp. gram.:Adj.}
\end{itemize}
Aquelle que arremata.
Que arremata.
\section{Arrematante}
\begin{itemize}
\item {Grp. gram.:m.  e  adj.}
\end{itemize}
Arrematador; que arremata.
\section{Arrematar}
\begin{itemize}
\item {Grp. gram.:v. t.}
\end{itemize}
\begin{itemize}
\item {Grp. gram.:V. i.}
\end{itemize}
O mesmo que \textunderscore rematar\textunderscore ; concluir.
Fazer remate de pontos em (costura).
Atar em nó (o cabello) Sachar pela segunda vez (milho).
Acabar.
(Cp. \textunderscore rematar\textunderscore )
\section{Arrematar}
\begin{itemize}
\item {Grp. gram.:v. t.}
\end{itemize}
\begin{itemize}
\item {Utilização:Prov.}
\end{itemize}
\begin{itemize}
\item {Utilização:alg.}
\end{itemize}
\begin{itemize}
\item {Grp. gram.:V. i.}
\end{itemize}
\begin{itemize}
\item {Utilização:Prov.}
\end{itemize}
\begin{itemize}
\item {Utilização:alg.}
\end{itemize}
Adjudicar em leilão a quem mais dá.
Comprar em leilão.
Tomar de arrendamento em almoéda.
Berrar como leiloeiro. Cf. Filinto, X, 128.
Amaldiçoar, rogar pragas a.
Rogar pragas; dizer palavras injuriosas ou obscenas.
(Por \textunderscore arramatar\textunderscore , de \textunderscore ramo\textunderscore )
\section{Arremate}
\begin{itemize}
\item {Grp. gram.:m.}
\end{itemize}
Acto de \textunderscore arrematar\textunderscore ^1. Remate. Cf. Garrett, \textunderscore Camões\textunderscore , 43.
\section{Arremeção}
\begin{itemize}
\item {Grp. gram.:m.}
\end{itemize}
\begin{itemize}
\item {Utilização:Ant.}
\end{itemize}
Medida agrária de 19 {1/2} palmos, ou 4^{m},30.
\section{Arremedador}
\begin{itemize}
\item {Grp. gram.:m.}
\end{itemize}
Aquelle que arremeda.
\section{Arremedar}
\begin{itemize}
\item {Grp. gram.:v. t.}
\end{itemize}
\begin{itemize}
\item {Proveniência:(Do lat. \textunderscore re...\textunderscore  + \textunderscore imitari\textunderscore )}
\end{itemize}
Imitar ridiculamente; macaquear.
Contrafazer; semelhar: \textunderscore arremedar o estilo de alguém\textunderscore .
\section{Arremedilho}
\begin{itemize}
\item {Grp. gram.:m.}
\end{itemize}
\begin{itemize}
\item {Utilização:Ant.}
\end{itemize}
Estribilho.
Cantiga.
Antiga e faceta representação scênica.
(Dem. de \textunderscore arremêdo\textunderscore )
\section{Arremêdo}
\begin{itemize}
\item {Grp. gram.:m.}
\end{itemize}
Acto de \textunderscore arremedar\textunderscore .
\section{Arremelgar}
\begin{itemize}
\item {Grp. gram.:v. t.}
\end{itemize}
\begin{itemize}
\item {Utilização:Pop.}
\end{itemize}
\begin{itemize}
\item {Proveniência:(De \textunderscore remelgado\textunderscore )}
\end{itemize}
Abrir muito os olhos.
\section{Arremenicar}
\begin{itemize}
\item {Grp. gram.:v. i.}
\end{itemize}
O mesmo que \textunderscore remenicar\textunderscore .
\section{Arremessadamente}
\begin{itemize}
\item {Grp. gram.:adv.}
\end{itemize}
Com arremêsso.
\section{Arremessado}
\begin{itemize}
\item {fónica:mé}
\end{itemize}
\begin{itemize}
\item {Grp. gram.:adj.}
\end{itemize}
\begin{itemize}
\item {Utilização:Prov.}
\end{itemize}
\begin{itemize}
\item {Utilização:trasm.}
\end{itemize}
Bem provido, bem abastecido; que tem a casa bem cheia; que tem a algibeira bem quente.
\section{Arremessador}
\begin{itemize}
\item {Grp. gram.:m.}
\end{itemize}
Aquelle que arremessa.
\section{Arremessamento}
\begin{itemize}
\item {Grp. gram.:m.}
\end{itemize}
O mesmo que \textunderscore arremêsso\textunderscore .
\section{Arremessão}
\begin{itemize}
\item {Grp. gram.:m.}
\end{itemize}
\begin{itemize}
\item {Proveniência:(De \textunderscore arremêsso\textunderscore )}
\end{itemize}
Aquillo que se arremessa; arma que se atira.
\section{Arremessar}
\begin{itemize}
\item {Grp. gram.:v. t.}
\end{itemize}
\begin{itemize}
\item {Proveniência:(Do lat. \textunderscore remissus\textunderscore )}
\end{itemize}
Arrojar com força.
Lançar para longe.
Expulsar.
Fazer andar com ímpeto.
\section{Arremêsso}
\begin{itemize}
\item {Grp. gram.:m.}
\end{itemize}
Acto ou effeito de \textunderscore arremessar\textunderscore .
Arma, objecto, que se arremessa.
Ataque.
Ameaça.
Gesto repulsivo: \textunderscore fez-lhe um arremêsso\textunderscore .
Temeridade.
\section{Arremetedor}
\begin{itemize}
\item {Grp. gram.:m.  e  adj.}
\end{itemize}
\begin{itemize}
\item {Proveniência:(De \textunderscore arremeter\textunderscore )}
\end{itemize}
O que arremete.
\section{Arremetedura}
\begin{itemize}
\item {Grp. gram.:f.}
\end{itemize}
(V.arremetida)
\section{Arremetente}
\begin{itemize}
\item {Grp. gram.:adj.}
\end{itemize}
Que arremete.
\section{Arremeter}
\begin{itemize}
\item {Grp. gram.:v. i.}
\end{itemize}
\begin{itemize}
\item {Grp. gram.:V. t.}
\end{itemize}
\begin{itemize}
\item {Proveniência:(Lat. \textunderscore remittere\textunderscore )}
\end{itemize}
Investir; arrojar-se precipitadamente.
Adiantar-se impetuosamente: \textunderscore arremeteu contra o grupo\textunderscore .
Açular (animaes).
\section{Arremetida}
\begin{itemize}
\item {Grp. gram.:f.}
\end{itemize}
Acto de \textunderscore arremeter\textunderscore .
\section{Arremetimento}
\begin{itemize}
\item {Grp. gram.:m.}
\end{itemize}
O mesmo que \textunderscore arremetida\textunderscore .
\section{Arreminação}
\begin{itemize}
\item {Grp. gram.:f.}
\end{itemize}
Acto de \textunderscore arreminar-se\textunderscore .
\section{Arreminar-se}
\begin{itemize}
\item {Grp. gram.:v. p.}
\end{itemize}
\begin{itemize}
\item {Utilização:Pop.}
\end{itemize}
Irritar-se, ameaçando.
(Cp. lat. \textunderscore minari\textunderscore )
\section{Arrenda}
\begin{itemize}
\item {Grp. gram.:f.}
\end{itemize}
\begin{itemize}
\item {Utilização:Prov.}
\end{itemize}
O mesmo que \textunderscore redra\textunderscore .
\section{Arrendação}
\begin{itemize}
\item {Grp. gram.:f.}
\end{itemize}
(V.arrendamento)
\section{Arrendada}
\begin{itemize}
\item {Grp. gram.:f.}
\end{itemize}
\begin{itemize}
\item {Utilização:Prov.}
\end{itemize}
\begin{itemize}
\item {Utilização:beir.}
\end{itemize}
Belga lavradía.
\section{Arrendado}
\begin{itemize}
\item {Grp. gram.:adj.}
\end{itemize}
Que se arrendou: \textunderscore uma casa arrendada\textunderscore .
\section{Arrendador}
\begin{itemize}
\item {Grp. gram.:m.}
\end{itemize}
Aquelle que arrenda.
\section{Arrendados}
\begin{itemize}
\item {Grp. gram.:m. pl.}
\end{itemize}
\begin{itemize}
\item {Proveniência:(De \textunderscore arrendar\textunderscore ^3)}
\end{itemize}
Lavores, em fórma de renda.
Ornatos delicados e miúdos, em architectura.
\section{Arrendamento}
\begin{itemize}
\item {Grp. gram.:m.}
\end{itemize}
\begin{itemize}
\item {Proveniência:(De \textunderscore arrendar\textunderscore ^1)}
\end{itemize}
Contrato, em que alguém cede a outrem, por certo tempo e determinado preço, a fruição de um prédio.
O título dêsse contrato.
O preço, por que se cede a fruição temporária de um prédio.
\section{Arrendar}
\begin{itemize}
\item {Grp. gram.:v. t.}
\end{itemize}
\begin{itemize}
\item {Proveniência:(De \textunderscore renda\textunderscore ^2)}
\end{itemize}
Dar de renda.
Tomar de arrendamento.
\section{Arrendar}
\begin{itemize}
\item {Grp. gram.:v. t.}
\end{itemize}
Sujeitar (o cavallo) á rédea.
(Por \textunderscore arredear\textunderscore , de \textunderscore rédea\textunderscore )
\section{Arrendar}
\begin{itemize}
\item {Grp. gram.:v. t.}
\end{itemize}
\begin{itemize}
\item {Proveniência:(De \textunderscore renda\textunderscore ^1)}
\end{itemize}
Guarnecer com rendas.
Rendilhar.
\section{Arrendar}
\begin{itemize}
\item {Grp. gram.:v. t.}
\end{itemize}
\begin{itemize}
\item {Proveniência:(De \textunderscore arrenda\textunderscore )}
\end{itemize}
Redrar; cavar segunda vez, (vinhas ou milharaes), para lhe tirar a erva.
\section{Arrendatário}
\begin{itemize}
\item {Grp. gram.:m.}
\end{itemize}
\begin{itemize}
\item {Proveniência:(De \textunderscore arrendar\textunderscore ^1)}
\end{itemize}
Aquelle que toma de arrendamento um prédio.
\section{Arrendável}
\begin{itemize}
\item {Grp. gram.:adj.}
\end{itemize}
Que se póde \textunderscore arrendar\textunderscore .
\section{Arrendilhado}
\begin{itemize}
\item {Grp. gram.:adj.}
\end{itemize}
O mesmo que [[rendilhado|rendilhar]]. Cf. Th. Ribeiro, \textunderscore Jornadas\textunderscore , I, 401.
\section{Arrendo}
\begin{itemize}
\item {Grp. gram.:m.}
\end{itemize}
\begin{itemize}
\item {Utilização:Prov.}
\end{itemize}
\begin{itemize}
\item {Utilização:minh.}
\end{itemize}
O mesmo que \textunderscore arrendamento\textunderscore .
\section{Arrenega}
\begin{itemize}
\item {Grp. gram.:f.}
\end{itemize}
\begin{itemize}
\item {Utilização:Prov.}
\end{itemize}
\begin{itemize}
\item {Utilização:alg.}
\end{itemize}
\begin{itemize}
\item {Utilização:Ant.}
\end{itemize}
O mesmo que \textunderscore arrenegação\textunderscore .
Folga, descanso.
\section{Arrenegação}
\begin{itemize}
\item {Grp. gram.:f.}
\end{itemize}
Apostasia.
Acto de \textunderscore arrenegar\textunderscore .
\section{Arrenegada}
\begin{itemize}
\item {Grp. gram.:f.}
\end{itemize}
\begin{itemize}
\item {Proveniência:(De \textunderscore arrenegar\textunderscore )}
\end{itemize}
Jôgo de cartas, também conhecido por \textunderscore zanga\textunderscore .
\section{Arrenegado}
\begin{itemize}
\item {Grp. gram.:adj.}
\end{itemize}
Zangado.
Irritado.
\section{Arrenegador}
\begin{itemize}
\item {Grp. gram.:m.}
\end{itemize}
Aquelle que arrenega.
\section{Arrenegar}
\begin{itemize}
\item {Grp. gram.:v. t.}
\end{itemize}
\begin{itemize}
\item {Grp. gram.:V. i.}
\end{itemize}
\begin{itemize}
\item {Grp. gram.:V. p.}
\end{itemize}
\begin{itemize}
\item {Proveniência:(De \textunderscore renegar\textunderscore )}
\end{itemize}
Renegar; abjurar.
Detestar.
Amaldiçoar.
Têr aversão: \textunderscore arrenégo de tal vida\textunderscore .
Irritar-se, zangar-se.
\section{Arrenêgo}
\begin{itemize}
\item {Grp. gram.:m.}
\end{itemize}
Acto de \textunderscore arrenegar-se\textunderscore .
\section{Arrenóptero}
\begin{itemize}
\item {Grp. gram.:m.}
\end{itemize}
Gênero de musgos americanos.
\section{Arrenoquia}
\begin{itemize}
\item {Grp. gram.:f.}
\end{itemize}
Faculdade excepcional, que têm certos animaes, como as abelhas, de pôr ovos que não procedem da fecundação e de que só sáem indivíduos machos.
\section{Arrepanhar}
\begin{itemize}
\item {Grp. gram.:v. t.}
\end{itemize}
\begin{itemize}
\item {Proveniência:(De \textunderscore re...\textunderscore  + \textunderscore apanhar\textunderscore )}
\end{itemize}
Engelhar; refegar.
Economizar com avareza.
Roubar; arrebatar.
\section{Arrepelação}
\begin{itemize}
\item {Grp. gram.:f.}
\end{itemize}
Acto de \textunderscore arrepelar\textunderscore .
\section{Arrepelada}
\begin{itemize}
\item {Grp. gram.:f.}
\end{itemize}
Briga.
Acto de \textunderscore arrepelar\textunderscore .
\section{Arrepeladela}
\begin{itemize}
\item {Grp. gram.:f.}
\end{itemize}
(V.arrepelão)
\section{Arrepelador}
\begin{itemize}
\item {Grp. gram.:adj.}
\end{itemize}
Que arrepela.
\section{Arrepelamento}
\begin{itemize}
\item {Grp. gram.:m.}
\end{itemize}
Acto de \textunderscore arrepelar\textunderscore .
\section{Arrepelão}
\begin{itemize}
\item {Grp. gram.:m.}
\end{itemize}
Repelão.
Acto de \textunderscore arrepelar\textunderscore .
\section{Arrepelar}
\begin{itemize}
\item {Grp. gram.:v. t.}
\end{itemize}
\begin{itemize}
\item {Grp. gram.:V. p.}
\end{itemize}
\begin{itemize}
\item {Proveniência:(De \textunderscore re...\textunderscore  + \textunderscore pelar\textunderscore )}
\end{itemize}
Puxar (os cabellos).
Arrancar (pêlos, pennas).
Beliscar.
Lastimar-se.
Arrancar os cabellos.
\section{Arrepender-se}
\begin{itemize}
\item {Grp. gram.:v. p.}
\end{itemize}
\begin{itemize}
\item {Proveniência:(Do lat. \textunderscore re...\textunderscore  + \textunderscore poenitere\textunderscore )}
\end{itemize}
Têr pesar (de faltas próprias).
Mudar de opinião.
\section{Arrependido}
\begin{itemize}
\item {Grp. gram.:adj.}
\end{itemize}
Que se arrependeu.
\section{Arrependimento}
\begin{itemize}
\item {Grp. gram.:m.}
\end{itemize}
Acto de \textunderscore arrepender-se\textunderscore .
\section{Arrepeso}
\begin{itemize}
\item {fónica:pê}
\end{itemize}
\begin{itemize}
\item {Grp. gram.:adj.}
\end{itemize}
\begin{itemize}
\item {Utilização:Pop.}
\end{itemize}
O mesmo que \textunderscore arrependido\textunderscore .
\section{Arrepia}
\begin{itemize}
\item {Grp. gram.:f.}
\end{itemize}
\begin{itemize}
\item {Utilização:Pop.}
\end{itemize}
\begin{itemize}
\item {Proveniência:(De \textunderscore arrepiar\textunderscore )}
\end{itemize}
Certa música de viola, para acompanhamento de dança desenvolta.
\section{Arrepiacabello}
\begin{itemize}
\item {fónica:bê}
\end{itemize}
\begin{itemize}
\item {Grp. gram.:adv.}
\end{itemize}
\begin{itemize}
\item {Grp. gram.:M.}
\end{itemize}
\begin{itemize}
\item {Proveniência:(De \textunderscore arrepiar\textunderscore  + \textunderscore cabello\textunderscore )}
\end{itemize}
Em sentido contrário.
Pessôa ríspida, intratável.
\section{Arrepiacabelo}
\begin{itemize}
\item {fónica:bê}
\end{itemize}
\begin{itemize}
\item {Grp. gram.:adv.}
\end{itemize}
\begin{itemize}
\item {Grp. gram.:M.}
\end{itemize}
\begin{itemize}
\item {Proveniência:(De \textunderscore arrepiar\textunderscore  + \textunderscore cabello\textunderscore )}
\end{itemize}
Em sentido contrário.
Pessôa ríspida, intratável.
\section{Arrepiado}
\begin{itemize}
\item {Grp. gram.:adj.}
\end{itemize}
\begin{itemize}
\item {Proveniência:(De \textunderscore arrepiar\textunderscore )}
\end{itemize}
Diz-se da ave, a que o chumbo do caçador tirou algumas pennas, e que, depois de subir verticalmente, fecha as asas e cái morta.
Que, por mêdo, sente arrepiarem-se-lhe as carnes e o cabello.
\section{Arrepiadura}
\begin{itemize}
\item {Grp. gram.:f.}
\end{itemize}
O mesmo que \textunderscore arrepiamento\textunderscore .
\section{Arrepiamento}
\begin{itemize}
\item {Grp. gram.:m.}
\end{itemize}
Acto ou effeito de \textunderscore arrepiar\textunderscore .
\section{Arrepiar}
\begin{itemize}
\item {Grp. gram.:v. t.}
\end{itemize}
\begin{itemize}
\item {Grp. gram.:V. i.}
\end{itemize}
\begin{itemize}
\item {Proveniência:(Lat. \textunderscore horripilare\textunderscore )}
\end{itemize}
Levantar, encrespar (o cabello).
Causar horror a; fazer tremer.
Enrugar.
\textunderscore Arrepiar caminho\textunderscore  ou \textunderscore carreira\textunderscore , voltar para trás, desandar; desdizer-se.
\textunderscore Arrepiar peixe\textunderscore , golpeá-lo, salgá-lo e prepará-lo para o conservar.
Causar arrepios.
Erguer-se ferida e cair morta, (falando-se da ave).
\section{Arrepio}
\begin{itemize}
\item {Grp. gram.:m.}
\end{itemize}
\begin{itemize}
\item {Utilização:T. da Bairrada}
\end{itemize}
\begin{itemize}
\item {Proveniência:(De \textunderscore arrepiar\textunderscore )}
\end{itemize}
Calefrio.
Acto de arrepiar-se alguém.
Direcção inversa da que costuma têr o cabello, o pêlo, etc.
Acto de um jogador ter feito três jogos a fio, e depois o contrário fazer quatro seguidamente, ganhando a partida.
\textunderscore Ao arrepio\textunderscore , contra a maré, contra a corrente:«\textunderscore lançaram-se á corrente e foram ao arrepio, até...\textunderscore »Camillo, \textunderscore Retr. de Ricard.\textunderscore , 132.
\section{Arrepique}
\begin{itemize}
\item {Grp. gram.:m.}
\end{itemize}
O mesmo que \textunderscore arrebique\textunderscore . Cf. Camillo, \textunderscore Quéda de Um Anjo\textunderscore , 120; \textunderscore Aulegrafia\textunderscore , 1.
\section{Arrepolhado}
\begin{itemize}
\item {Grp. gram.:adj.}
\end{itemize}
Que tem aspecto de repolho.
\section{Arrepolhar}
\begin{itemize}
\item {Grp. gram.:v. t.}
\end{itemize}
Dar fórma de repolho a.
Entufar.
\section{Arrequi}
\begin{itemize}
\item {Grp. gram.:m.}
\end{itemize}
\begin{itemize}
\item {Utilização:Ant.}
\end{itemize}
Jôgo de cartas.
\section{Arrequifar}
\begin{itemize}
\item {Grp. gram.:v. t.}
\end{itemize}
Ornar de requifes. Cf. Garrett, \textunderscore Viagens\textunderscore , I, 8.
\section{Arrequife}
\begin{itemize}
\item {Grp. gram.:m.}
\end{itemize}
\begin{itemize}
\item {Proveniência:(Do ár. \textunderscore arricfe\textunderscore )}
\end{itemize}
Ponta de ferro na extremidade de um pau, para limpar o algodão.
\section{Arresinar-se}
\begin{itemize}
\item {Grp. gram.:v. i.}
\end{itemize}
\begin{itemize}
\item {Utilização:Gír.}
\end{itemize}
Horrorizar-se.
(Cp. \textunderscore rezingar\textunderscore )
\section{Arrestante}
\begin{itemize}
\item {Grp. gram.:m.  e  f.}
\end{itemize}
\begin{itemize}
\item {Proveniência:(De \textunderscore arrestar\textunderscore )}
\end{itemize}
Pessôa, que requere arresto.
\section{Arrestar}
\begin{itemize}
\item {Grp. gram.:v. t.}
\end{itemize}
Fazer arresto em.
(B. lat. \textunderscore arrestare\textunderscore )
\section{Arresto}
\begin{itemize}
\item {Grp. gram.:m.}
\end{itemize}
\begin{itemize}
\item {Proveniência:(De \textunderscore arrestar\textunderscore )}
\end{itemize}
Aprehensão de objectos ou bens, por auctorização judicial.
Penhóra; embargo.
\section{Arrestralar}
\begin{itemize}
\item {Grp. gram.:v. t.}
\end{itemize}
\begin{itemize}
\item {Utilização:Prov.}
\end{itemize}
\begin{itemize}
\item {Utilização:trasm.}
\end{itemize}
Rapar com a navalha (a cara).
Assentar forte bofetada em (a cara de alguém).
\section{Arretar}
\begin{itemize}
\item {Grp. gram.:v. t.}
\end{itemize}
\begin{itemize}
\item {Proveniência:(De \textunderscore arrêto\textunderscore )}
\end{itemize}
Fazer voltar para trás.
Sustar a marcha de (animaes, rebanho, etc.).
\section{Arrêto}
\begin{itemize}
\item {Grp. gram.:m.}
\end{itemize}
Acto de arretar.
Botaréu; pequeno muro que sustém a pressão de terreno declive.
\section{Arreu}
\begin{itemize}
\item {Grp. gram.:adv.}
\end{itemize}
\begin{itemize}
\item {Utilização:Ant.}
\end{itemize}
A fio; continuadamente.
\section{Arreumático}
\begin{itemize}
\item {Grp. gram.:adj.}
\end{itemize}
\begin{itemize}
\item {Proveniência:(De \textunderscore a\textunderscore  priv. + \textunderscore reumático\textunderscore )}
\end{itemize}
Que não foi atacado de reumatismo.
\section{Arrevém}
\begin{itemize}
\item {Grp. gram.:m.}
\end{itemize}
\begin{itemize}
\item {Utilização:Ant.}
\end{itemize}
O mesmo que \textunderscore arrebém\textunderscore .
\section{Arrevesadamente}
\begin{itemize}
\item {Grp. gram.:adv.}
\end{itemize}
Ao revés.
\section{Arrevesado}
\begin{itemize}
\item {Grp. gram.:adj.}
\end{itemize}
\begin{itemize}
\item {Proveniência:(De \textunderscore arrevesar\textunderscore )}
\end{itemize}
Pôsto ás avessas.
Intrincado; obscuro: \textunderscore estilo arrevesado\textunderscore .
\section{Arrevesar}
\begin{itemize}
\item {Grp. gram.:v. t.}
\end{itemize}
Pôr ao revés, ás avessas, do avesso.
Tornar embaraçado, obscuro.
Dar sentido contrário a.
Revesar.
Arrevessar.
(Cp. \textunderscore arrevessar\textunderscore )
\section{Arrevessar}
\begin{itemize}
\item {Grp. gram.:v. t.}
\end{itemize}
\begin{itemize}
\item {Proveniência:(Do lat. \textunderscore reversus\textunderscore )}
\end{itemize}
Vomitar.
É o mesmo que \textunderscore arrevesar\textunderscore .
\section{Arrevêsso}
\begin{itemize}
\item {Grp. gram.:adj.}
\end{itemize}
O mesmo que \textunderscore arrevesado\textunderscore .
(Part. irr. de \textunderscore arrevessar\textunderscore )
\section{Arrenal}
\begin{itemize}
\item {Grp. gram.:m.}
\end{itemize}
Medicamento, constituído por uma preparação arsenical.
(Cp. lat. \textunderscore arrhenicum\textunderscore )
\section{Arrepsia}
\begin{itemize}
\item {Grp. gram.:f.}
\end{itemize}
\begin{itemize}
\item {Proveniência:(Gr. \textunderscore arrhepsia\textunderscore )}
\end{itemize}
Incerteza; hesitação.
\section{Arrhas}
\begin{itemize}
\item {Grp. gram.:f. pl.}
\end{itemize}
O mesmo que \textunderscore arras\textunderscore .
\section{Arrhenal}
\begin{itemize}
\item {Grp. gram.:m.}
\end{itemize}
Medicamento, constituído por uma preparação arsenical.
(Cp. lat. \textunderscore arrhenicum\textunderscore )
\section{Arrhepsia}
\begin{itemize}
\item {Grp. gram.:f.}
\end{itemize}
\begin{itemize}
\item {Proveniência:(Gr. \textunderscore arrhepsia\textunderscore )}
\end{itemize}
Incerteza; hesitação.
\section{Arrhizo}
\begin{itemize}
\item {Grp. gram.:adj.}
\end{itemize}
\begin{itemize}
\item {Proveniência:(Do gr. \textunderscore a\textunderscore  priv. + \textunderscore rhiza\textunderscore )}
\end{itemize}
Que não tem raiz ou radícula.
\section{Arriar}
\begin{itemize}
\item {Grp. gram.:v.}
\end{itemize}
\begin{itemize}
\item {Utilização:t. Marn.}
\end{itemize}
Inutilizar (compartimentos destinados para a evaporação).
\section{Arriba}
\begin{itemize}
\item {Grp. gram.:f.}
\end{itemize}
\begin{itemize}
\item {Grp. gram.:Adv.}
\end{itemize}
\begin{itemize}
\item {Proveniência:(Do lat. \textunderscore ad\textunderscore  + \textunderscore ripam\textunderscore )}
\end{itemize}
O mesmo que \textunderscore riba\textunderscore .
Fraga á beira-mar.
Acima.
Adeante.
\section{Arribação}
\begin{itemize}
\item {Grp. gram.:f.}
\end{itemize}
Acto de \textunderscore arribar\textunderscore .
\section{Arribada}
\begin{itemize}
\item {Grp. gram.:f.}
\end{itemize}
\begin{itemize}
\item {Utilização:Prov.}
\end{itemize}
\begin{itemize}
\item {Utilização:dur.}
\end{itemize}
\begin{itemize}
\item {Utilização:Prov.}
\end{itemize}
\begin{itemize}
\item {Utilização:minh.}
\end{itemize}
O mesmo que \textunderscore arribação\textunderscore .
O mesmo que \textunderscore sapada\textunderscore .
Borda de um campo sôbre um caminho público; orla do campo, em talude, sem parede.
\section{Arribadeiro}
\begin{itemize}
\item {Grp. gram.:m.}
\end{itemize}
\begin{itemize}
\item {Utilização:Pesc.}
\end{itemize}
\begin{itemize}
\item {Proveniência:(De \textunderscore arribar\textunderscore )}
\end{itemize}
Corda, que se ala, do mar para a terra, depois do lançamento da rede de arrastar.
\section{Arribadiço}
\begin{itemize}
\item {Grp. gram.:adj.}
\end{itemize}
\begin{itemize}
\item {Utilização:Fig.}
\end{itemize}
\begin{itemize}
\item {Proveniência:(De \textunderscore arribar\textunderscore )}
\end{itemize}
Diz-se das aves de arribação.
Adventício; intruso.
\section{Arribana}
\begin{itemize}
\item {Grp. gram.:f.}
\end{itemize}
Choupana.
Curral.
Pequena casa, coberta de colmo.
\section{Arribar}
\begin{itemize}
\item {Grp. gram.:v. i.}
\end{itemize}
\begin{itemize}
\item {Proveniência:(Do b. lat. \textunderscore ad-ripare\textunderscore )}
\end{itemize}
Chegar a um pôrto, por motivo de fôrça maior.
Ancorar.
Virar para sotavento.
Melhorar; restabelecer-se: \textunderscore o meu doente arribou\textunderscore .
Chegar.
\section{Arribe}
\begin{itemize}
\item {Grp. gram.:m.}
\end{itemize}
\begin{itemize}
\item {Utilização:Bras}
\end{itemize}
\begin{itemize}
\item {Proveniência:(De \textunderscore arribar\textunderscore )}
\end{itemize}
Acto de \textunderscore arribar\textunderscore .
Chegada.
Importação.
\section{Arribozes}
\begin{itemize}
\item {Grp. gram.:m. pl.}
\end{itemize}
\begin{itemize}
\item {Utilização:Prov.}
\end{itemize}
\begin{itemize}
\item {Proveniência:(De \textunderscore riba\textunderscore )}
\end{itemize}
Fragas; ribas muito escarpadas.
\section{Arriçar}
\begin{itemize}
\item {Grp. gram.:v. t.}
\end{itemize}
\begin{itemize}
\item {Grp. gram.:V. i.}
\end{itemize}
\begin{itemize}
\item {Utilização:Ant.}
\end{itemize}
O mesmo que \textunderscore arrizar\textunderscore ^1 e \textunderscore eriçar\textunderscore .
Tornar-se rijo e robusto. Cf. \textunderscore Cancion. da Vaticana\textunderscore .
\section{Arricaveiro}
\begin{itemize}
\item {Grp. gram.:m.}
\end{itemize}
\begin{itemize}
\item {Utilização:Ant.}
\end{itemize}
\begin{itemize}
\item {Proveniência:(Do ár. \textunderscore ar-ricabe\textunderscore )}
\end{itemize}
Segundo Viterbo, chamava-se assim o soldado agricultor, que só em tempo de guerra era obrigado a prestar serviços de guarnição; mas, segundo Dozy, era o escudeiro ou aquelle que segurava o estribo.
\section{Arridar}
\begin{itemize}
\item {Grp. gram.:v.}
\end{itemize}
\begin{itemize}
\item {Utilização:t. Náut.}
\end{itemize}
Prender as arridas nos botões de.
Segurar com as arridas.
\section{Arridas}
\begin{itemize}
\item {Grp. gram.:f. pl.}
\end{itemize}
\begin{itemize}
\item {Utilização:Náut.}
\end{itemize}
Cordéis, que prendem os toldos ás bordas dos escaleres.
\section{Arrieira}
\begin{itemize}
\item {Grp. gram.:f.}
\end{itemize}
\begin{itemize}
\item {Proveniência:(Fr. \textunderscore arrière\textunderscore )}
\end{itemize}
Espécie de carbúnculo mortífero, que se desenvolve no intestino recto do gado bovino. Cf. Baganha, \textunderscore Vacas Leiteiras\textunderscore , 230.
\section{Arrieiro}
\textunderscore m.\textunderscore  (e der.)
(V. \textunderscore arreeiro\textunderscore , etc.)
\section{Arriel}
\begin{itemize}
\item {Grp. gram.:m.}
\end{itemize}
Barra de prata.
Pequena argola de oiro.
Arrecada.
(Cast. \textunderscore riel\textunderscore )
\section{Arrifana}
\begin{itemize}
\item {Grp. gram.:f.}
\end{itemize}
\begin{itemize}
\item {Utilização:Prov.}
\end{itemize}
\begin{itemize}
\item {Utilização:alent.}
\end{itemize}
\begin{itemize}
\item {Proveniência:(De \textunderscore Arrifana\textunderscore , n. p.)}
\end{itemize}
Pano fino de linho.
\section{Arrifana}
\begin{itemize}
\item {Grp. gram.:f.}
\end{itemize}
\begin{itemize}
\item {Utilização:Açor}
\end{itemize}
Série de arrifes.
\section{Arrifar-se}
\begin{itemize}
\item {Grp. gram.:v. p.}
\end{itemize}
\begin{itemize}
\item {Utilização:Ant.}
\end{itemize}
\begin{itemize}
\item {Proveniência:(De \textunderscore rifa\textunderscore )}
\end{itemize}
Jogar aos dados.
Têr brio.
\section{Arrife}
\begin{itemize}
\item {Grp. gram.:m.}
\end{itemize}
\begin{itemize}
\item {Utilização:Açor}
\end{itemize}
\begin{itemize}
\item {Utilização:T. de Alcanena}
\end{itemize}
Desbaste de arvoredo em linha recta, formando uma aberta de alguns metros de largura; sesmo. Cf. \textunderscore Gaz. das Ald.\textunderscore , n.^o 106.
Tênue camada de terreno, em que apparecem, aqui e ali, cabeçotes de rocha subjacente.
Penedia, cortada a prumo.
(Do ár.?)
\section{Arrifeiro}
\begin{itemize}
\item {Grp. gram.:m.}
\end{itemize}
\begin{itemize}
\item {Utilização:Açor}
\end{itemize}
\begin{itemize}
\item {Proveniência:(De \textunderscore Arrifes\textunderscore , n. p. de uma freguesia suburbana de Ponta-Delgada)}
\end{itemize}
Homem rude, boçal.
\section{Arrigar}
\begin{itemize}
\item {Grp. gram.:v. t.}
\end{itemize}
\begin{itemize}
\item {Utilização:Prov.}
\end{itemize}
\begin{itemize}
\item {Utilização:trasm.}
\end{itemize}
Tirar da terra (o linho), para o ripar e enriar.
\section{Arrijar}
\begin{itemize}
\item {Grp. gram.:v. t.  e  i.}
\end{itemize}
(V.enrijar)
\section{Arrilhada}
\begin{itemize}
\item {Grp. gram.:f.}
\end{itemize}
\begin{itemize}
\item {Utilização:Prov.}
\end{itemize}
Bico de ferro da aguilhada.
Espécie de raspador, composto de um ferro triangular, com cabo de madeira, e que serve para arrancar da rocha a minhoca de água salgada, destinada a isco.
\section{Arrimadiço}
\begin{itemize}
\item {Grp. gram.:adj.}
\end{itemize}
Que se arrima ou costuma arrimar-se.
\section{Arrimão}
\begin{itemize}
\item {Grp. gram.:m.}
\end{itemize}
\begin{itemize}
\item {Utilização:Ant.}
\end{itemize}
Espécie de vela grande de embarcação.
\section{Arrimar}
\begin{itemize}
\item {Grp. gram.:v. t.}
\end{itemize}
\begin{itemize}
\item {Utilização:Fam.}
\end{itemize}
\begin{itemize}
\item {Proveniência:(De \textunderscore rima\textunderscore ^3)}
\end{itemize}
Encostar.
Pôr em rima.
Deixar de lado.
Arrumar.
Dar, bater.
\section{Arrimo}
\begin{itemize}
\item {Grp. gram.:m.}
\end{itemize}
\begin{itemize}
\item {Utilização:Des.}
\end{itemize}
\begin{itemize}
\item {Proveniência:(De \textunderscore arrimar\textunderscore )}
\end{itemize}
Encôsto.
Amparo; auxílio.
Haveres, fazenda.
\section{Arrincão}
\begin{itemize}
\item {Grp. gram.:m.}
\end{itemize}
\begin{itemize}
\item {Utilização:T. de Lisbôa}
\end{itemize}
Vigota, que, no telhado de mais de duas vertentes, vai da parte superior do cunhal á cumeeira.
(Cp. \textunderscore rincão\textunderscore )
\section{Arrinção}
\begin{itemize}
\item {Grp. gram.:m.}
\end{itemize}
\begin{itemize}
\item {Utilização:Pop.}
\end{itemize}
(V. \textunderscore artesão\textunderscore ^1)
\section{Arrincar}
\begin{itemize}
\item {Grp. gram.:v. t.}
\end{itemize}
(Corr. de \textunderscore arrancar\textunderscore )
\section{Arrincoar}
\begin{itemize}
\item {Grp. gram.:v. t.}
\end{itemize}
\begin{itemize}
\item {Grp. gram.:V. p.}
\end{itemize}
Recolher em rincão.
Encurralar.
Tornar-se misanthropo, desviar-se da convivência.
\section{Arrinconar}
\begin{itemize}
\item {Grp. gram.:v. t.}
\end{itemize}
\begin{itemize}
\item {Utilização:Bras}
\end{itemize}
O mesmo que \textunderscore arrincoar\textunderscore .
\section{Arringa-iba}
\begin{itemize}
\item {Grp. gram.:f.}
\end{itemize}
Planta venenosa da Índia.
\section{Arringar}
\textunderscore v. t.\textunderscore  (e der.)
(V. \textunderscore arraigar\textunderscore , etc.)
\section{Arrinho}
\begin{itemize}
\item {Grp. gram.:m.}
\end{itemize}
\begin{itemize}
\item {Utilização:Ant.}
\end{itemize}
Enseada, onde é facil e copiosa a pesca dos sáveis e lampreias.
\section{Arrió}
\begin{itemize}
\item {Grp. gram.:m.}
\end{itemize}
O mesmo que \textunderscore arriós\textunderscore .
\section{Arriol}
\begin{itemize}
\item {Grp. gram.:m.}
\end{itemize}
(V.arriós)
\section{Arriós}
\begin{itemize}
\item {Grp. gram.:m.}
\end{itemize}
Pedrinha redonda, com que se jogava o alguergue.
Peloiros de arcabuz.
Fava amargosa do Brasil, de casca grossa e cinzenta.
(Talvez do vasc. Cp. Viana, \textunderscore Apostilas\textunderscore )
\section{Arriosca}
\begin{itemize}
\item {Grp. gram.:f.}
\end{itemize}
Lôgro; esparrela.
Cilada, falcatrua.
\section{Arripar}
\textunderscore v. i.\textunderscore  (e der.)
Surribar a terra das ostreiras, para apanhar pérolas.
(Talvez do lat. \textunderscore ripa\textunderscore )
\section{Arripeiro}
\begin{itemize}
\item {Grp. gram.:m.}
\end{itemize}
\begin{itemize}
\item {Proveniência:(De \textunderscore aripar\textunderscore )}
\end{itemize}
Aquelle que arripa.
\section{Arripiar}
\textunderscore v. t.\textunderscore  (e der.)
O mesmo que \textunderscore arrepiar\textunderscore , etc.
\section{Arripo}
\begin{itemize}
\item {Grp. gram.:m.}
\end{itemize}
Acto de \textunderscore arripar\textunderscore .
\section{Arriscadamente}
\begin{itemize}
\item {Grp. gram.:adv.}
\end{itemize}
Com risco, com perigo.
\section{Arriscar}
\begin{itemize}
\item {Grp. gram.:v. t.}
\end{itemize}
Pôr em risco; aventurar; sujeitar á sorte: \textunderscore arriscar uma libra no jôgo\textunderscore .
\section{Arrispidar-se}
\begin{itemize}
\item {Grp. gram.:v. p.}
\end{itemize}
Tornar-se ríspido, intratável.
\section{Arrizar}
\begin{itemize}
\item {Grp. gram.:v.}
\end{itemize}
\begin{itemize}
\item {Utilização:t. Náut.}
\end{itemize}
Atar com os rizes.
Meter nos rizes.
Prender com cordas.
\section{Arrizar}
\begin{itemize}
\item {Grp. gram.:v. i.}
\end{itemize}
O mesmo que \textunderscore arriçar\textunderscore .
\section{Arrizo}
\begin{itemize}
\item {Grp. gram.:adj.}
\end{itemize}
\begin{itemize}
\item {Proveniência:(Do gr. \textunderscore a\textunderscore  priv. + \textunderscore rhiza\textunderscore )}
\end{itemize}
Que não tem raiz ou radícula.
\section{Arrizotónico}
\begin{itemize}
\item {Grp. gram.:adj.}
\end{itemize}
\begin{itemize}
\item {Utilização:Philol.}
\end{itemize}
\begin{itemize}
\item {Proveniência:(De \textunderscore a\textunderscore  priv. + \textunderscore rhizotónico\textunderscore )}
\end{itemize}
Diz-se das fórmas verbaes, cuja sýllaba tónica está na terminação ou desinência, como em \textunderscore copiamos\textunderscore , \textunderscore copiarei\textunderscore .
\section{Arro}
\begin{itemize}
\item {Grp. gram.:m.}
\end{itemize}
\begin{itemize}
\item {Utilização:Ant.}
\end{itemize}
Lodo, lama.
\section{Arroba}
\begin{itemize}
\item {fónica:rô}
\end{itemize}
\begin{itemize}
\item {Grp. gram.:f.}
\end{itemize}
\begin{itemize}
\item {Proveniência:(Do ár. \textunderscore ar-robe\textunderscore )}
\end{itemize}
Antigo pêso, igual a trinta e dois arráteis.
\section{Arrobação}
\begin{itemize}
\item {Grp. gram.:f.}
\end{itemize}
\begin{itemize}
\item {Utilização:Bras. do N}
\end{itemize}
Acto de \textunderscore arrobar\textunderscore  ou pesar por arroba.
\textunderscore Rês de bôa arrobação\textunderscore , a rês que tem muito pêso ou muita carne.
\section{Arrobamento}
\begin{itemize}
\item {Grp. gram.:m.}
\end{itemize}
Acto de \textunderscore arrobar\textunderscore .
\section{Arrobar}
\begin{itemize}
\item {Grp. gram.:v. t.}
\end{itemize}
\begin{itemize}
\item {Utilização:Fig.}
\end{itemize}
Pesar ou medir por arroba.
Avaliar, á simples vista.
\section{Arrobe}
\begin{itemize}
\item {fónica:ro}
\end{itemize}
\begin{itemize}
\item {Grp. gram.:m.}
\end{itemize}
\begin{itemize}
\item {Proveniência:(Do ár. \textunderscore ar-robe\textunderscore )}
\end{itemize}
Xarope, produzido pelo mosto da uva, concentrado pela acção do fogo.
Geleia ou conserva de frutas.
\section{Arrobo}
\begin{itemize}
\item {Grp. gram.:m.}
\end{itemize}
(V.arroubo)
\section{Arrobustar-se}
\begin{itemize}
\item {Grp. gram.:v. p.}
\end{itemize}
\begin{itemize}
\item {Utilização:P. us.}
\end{itemize}
Tornar-se robusto.
\section{Arrocado}
\begin{itemize}
\item {Grp. gram.:adj.}
\end{itemize}
Que tem fórma de roca. Cf. Rebello, \textunderscore Mocid. de D. João V\textunderscore , I, 176.
\section{Arrocava}
\begin{itemize}
\item {Grp. gram.:f.}
\end{itemize}
O mesmo ou melhor que \textunderscore arrocova\textunderscore . Cf. Herculano, \textunderscore Hist. de Port.\textunderscore , IV, 106.
\section{Arrochada}
\begin{itemize}
\item {Grp. gram.:f.}
\end{itemize}
Pancada com arrocho; cacetada.
\section{Arrochador}
\begin{itemize}
\item {Grp. gram.:m.}
\end{itemize}
Aquelle que arrocha.
\section{Arrochadura}
\begin{itemize}
\item {Grp. gram.:f.}
\end{itemize}
Acto de \textunderscore arrochar\textunderscore .
\section{Arrochar}
\begin{itemize}
\item {Grp. gram.:v. t.}
\end{itemize}
\begin{itemize}
\item {Utilização:P. us.}
\end{itemize}
Atar, apertando com arrocho.
Comprimir fortemente.
Bater com arrocho.
\section{Arrocheiro}
\begin{itemize}
\item {Grp. gram.:m.}
\end{itemize}
\begin{itemize}
\item {Proveniência:(De \textunderscore arrocho\textunderscore )}
\end{itemize}
Arreeiro, almocreve.
\section{Arrochelar}
\begin{itemize}
\item {Grp. gram.:v. t.}
\end{itemize}
\begin{itemize}
\item {Utilização:Des.}
\end{itemize}
\begin{itemize}
\item {Proveniência:(De \textunderscore Rochella\textunderscore , n. p.)}
\end{itemize}
Fortificar.
\section{Arrochellar}
\begin{itemize}
\item {Grp. gram.:v. t.}
\end{itemize}
\begin{itemize}
\item {Utilização:Des.}
\end{itemize}
\begin{itemize}
\item {Proveniência:(De \textunderscore Rochella\textunderscore , n. p.)}
\end{itemize}
Fortificar.
\section{Arrocho}
\begin{itemize}
\item {fónica:rô}
\end{itemize}
\begin{itemize}
\item {Grp. gram.:m.}
\end{itemize}
Acto de \textunderscore arrochar\textunderscore .
Pau curto e torto, para apertar as cordas com que se ata um volume, cargas, etc.
Cacete.
Bordão, com que se espanca.
\section{Arrociar}
\begin{itemize}
\item {Grp. gram.:v. t.  e  i.}
\end{itemize}
O mesmo que \textunderscore rociar\textunderscore .
\section{Arrocova}
\begin{itemize}
\item {Grp. gram.:m.}
\end{itemize}
\begin{itemize}
\item {Utilização:Ant.}
\end{itemize}
\begin{itemize}
\item {Proveniência:(Do ár. \textunderscore ar-rocaba\textunderscore )}
\end{itemize}
O mesmo que \textunderscore arricaveiro\textunderscore .
Sentinela.
\section{Arrodelar}
\begin{itemize}
\item {Grp. gram.:v. t.}
\end{itemize}
Cobrir, armar, com rodela.
Dar fórma de rodela a.
\section{Arrofo}
\begin{itemize}
\item {fónica:rô}
\end{itemize}
\begin{itemize}
\item {Grp. gram.:m.}
\end{itemize}
Buraco, no remate da tarrafa.
\section{Arrogação}
\begin{itemize}
\item {Grp. gram.:f.}
\end{itemize}
Acto de \textunderscore arrogar\textunderscore .
\section{Arrogador}
\begin{itemize}
\item {Grp. gram.:m.}
\end{itemize}
Aquelle que arroga.
\section{Arrogância}
\begin{itemize}
\item {Grp. gram.:f.}
\end{itemize}
\begin{itemize}
\item {Proveniência:(Lat. \textunderscore arrogantia\textunderscore )}
\end{itemize}
Orgulho.
Soberba.
Insolência.
Altivez.
\section{Arrogante}
\begin{itemize}
\item {Grp. gram.:adj.}
\end{itemize}
\begin{itemize}
\item {Proveniência:(Lat. \textunderscore arrogans\textunderscore )}
\end{itemize}
Que tem arrogância.
Intrépido.
Majestoso.
\section{Arrogantemente}
\begin{itemize}
\item {Grp. gram.:adv.}
\end{itemize}
De modo \textunderscore arrogante\textunderscore ; com arrogância.
\section{Arrogar}
\begin{itemize}
\item {Grp. gram.:v. t.}
\end{itemize}
\begin{itemize}
\item {Proveniência:(Lat. \textunderscore arrogare\textunderscore )}
\end{itemize}
Tomar como próprio.
Appropriar-se de.
\section{Arrogar}
\begin{itemize}
\item {Grp. gram.:v. i.}
\end{itemize}
\begin{itemize}
\item {Utilização:T. da Bairrada}
\end{itemize}
Dirigir-se ou localizar-se, (falando de qualquer soffrimento): \textunderscore a inflamação arrogou para o peito\textunderscore .
\section{Arrogia}
\begin{itemize}
\item {Grp. gram.:f.}
\end{itemize}
\begin{itemize}
\item {Utilização:Ant.}
\end{itemize}
\begin{itemize}
\item {Proveniência:(De \textunderscore arrojo\textunderscore ?)}
\end{itemize}
Acto de apossar-se violentamente.
\section{Arrôgo}
\begin{itemize}
\item {Grp. gram.:m.}
\end{itemize}
\begin{itemize}
\item {Utilização:Des.}
\end{itemize}
O mesmo que \textunderscore arrogância\textunderscore .
\section{Arroiar}
\begin{itemize}
\item {Grp. gram.:v. i.}
\end{itemize}
Correr, brotar, como arroio.
\section{Arroio}
\begin{itemize}
\item {Grp. gram.:m.}
\end{itemize}
\begin{itemize}
\item {Proveniência:(Do lat. \textunderscore arrujium\textunderscore )}
\end{itemize}
Ribeiro; regato, que não tem permanência.
Pequena corrente de qualquer líquido.
\section{Arroios}
\begin{itemize}
\item {Grp. gram.:m. pl.}
\end{itemize}
Planta, talvez o mesmo que \textunderscore marroios\textunderscore .
\section{Arrojada}
\begin{itemize}
\item {Grp. gram.:f.}
\end{itemize}
\begin{itemize}
\item {Utilização:Ant.}
\end{itemize}
Namorada.
(Cp. \textunderscore arrojado\textunderscore )
\section{Arrojadamente}
\begin{itemize}
\item {Grp. gram.:adv.}
\end{itemize}
Com arrojo.
\section{Arrojadiço}
\begin{itemize}
\item {Grp. gram.:adj.}
\end{itemize}
Que se póde arrojar.
Ousado, temerário.
\section{Arrojado}
\begin{itemize}
\item {Grp. gram.:adj.}
\end{itemize}
\begin{itemize}
\item {Grp. gram.:M.}
\end{itemize}
\begin{itemize}
\item {Utilização:Ant.}
\end{itemize}
Destemido; valente.
Ousado; impetuoso.
Namorado. Cf. Castilho, \textunderscore Misanthropo\textunderscore , 97 e 109.
\section{Arrojador}
\begin{itemize}
\item {Grp. gram.:m.}
\end{itemize}
Aquelle que arroja.
\section{Arrojamento}
\begin{itemize}
\item {Grp. gram.:m.}
\end{itemize}
Acto de \textunderscore arrojar\textunderscore .
\section{Arrojão}
\begin{itemize}
\item {Grp. gram.:m.}
\end{itemize}
(V. \textunderscore rojão\textunderscore ^1)
\section{Arrojar}
\begin{itemize}
\item {Grp. gram.:v. t.}
\end{itemize}
\begin{itemize}
\item {Grp. gram.:V. p.}
\end{itemize}
\begin{itemize}
\item {Proveniência:(De \textunderscore rojar\textunderscore )}
\end{itemize}
Levar de rojo.
Arremessar.
Arrastar.
Ousar, atrever-se.
\section{Arrojeitar}
\begin{itemize}
\item {Grp. gram.:v. i.}
\end{itemize}
Arremessar o arrojeito.
(Por \textunderscore arrejeitar\textunderscore )
\section{Arrojeito}
\begin{itemize}
\item {Grp. gram.:m.}
\end{itemize}
\begin{itemize}
\item {Utilização:Prov.}
\end{itemize}
\begin{itemize}
\item {Proveniência:(De \textunderscore arrojeitar\textunderscore )}
\end{itemize}
Pau grosso, que se arremessa.
\section{Arrojo}
\begin{itemize}
\item {Grp. gram.:m.}
\end{itemize}
\begin{itemize}
\item {Utilização:Pop.}
\end{itemize}
\begin{itemize}
\item {Utilização:Prov.}
\end{itemize}
\begin{itemize}
\item {Utilização:trasm.}
\end{itemize}
\begin{itemize}
\item {Grp. gram.:Pl.}
\end{itemize}
Acto de \textunderscore arrojar\textunderscore .
Audácia; afoiteza.
Tumor, nascida.
Forcado de lavoira.
Destroços de naufrágio, que vêm á praia.
\section{Arrolador}
\begin{itemize}
\item {Grp. gram.:m.}
\end{itemize}
Aquelle que arrola.
\section{Arrolamento}
\begin{itemize}
\item {Grp. gram.:m.}
\end{itemize}
Acto de \textunderscore arrolar\textunderscore ^1.
\section{Arrolar}
\begin{itemize}
\item {Grp. gram.:v. t.}
\end{itemize}
Inscrever em rol.
Inventariar; relacionar.
\section{Arrolar}
\begin{itemize}
\item {Grp. gram.:v. t.}
\end{itemize}
(V.enrolar)
\section{Arrolar}
\begin{itemize}
\item {Grp. gram.:v. i.}
\end{itemize}
(V.arrulhar)
\section{Arrolhar}
\begin{itemize}
\item {Grp. gram.:v. t.}
\end{itemize}
Tapar com rôlha.
\section{Arrôlo}
\begin{itemize}
\item {Grp. gram.:m.}
\end{itemize}
Canto monótono, com que se adormentam crianças.
(Cp. \textunderscore arrolar\textunderscore ^3)
\section{Arromanar}
\begin{itemize}
\item {Grp. gram.:v. t.}
\end{itemize}
\begin{itemize}
\item {Utilização:Prov.}
\end{itemize}
\begin{itemize}
\item {Utilização:trasm.}
\end{itemize}
Pesar com a balança romana; arratelar.
\section{Arromançar}
\textunderscore v. t.\textunderscore  (e der.)
(V. \textunderscore romancear\textunderscore , etc.)
\section{Arromba}
\begin{itemize}
\item {Grp. gram.:f.}
\end{itemize}
\begin{itemize}
\item {Proveniência:(De \textunderscore arrombar\textunderscore )}
\end{itemize}
Canção para viola.
\textunderscore De arromba\textunderscore , diz-se de uma coisa excellente ou que espanta.
\section{Arrombada}
\begin{itemize}
\item {Grp. gram.:f.}
\end{itemize}
\begin{itemize}
\item {Utilização:Gír. do Pôrto.}
\end{itemize}
Rombo.
Acto de \textunderscore arrombar\textunderscore .
Borda falsa do navio.
Mulher que não é virgem.
\section{Arrombadela}
\begin{itemize}
\item {Grp. gram.:f.}
\end{itemize}
O mesmo que \textunderscore arrombamento\textunderscore .
\section{Arrombador}
\begin{itemize}
\item {Grp. gram.:m.}
\end{itemize}
Aquelle que arromba.
\section{Arrombamento}
\begin{itemize}
\item {Grp. gram.:m.}
\end{itemize}
Acto de \textunderscore arrombar\textunderscore .
\section{Arrombar}
\begin{itemize}
\item {Grp. gram.:v. t.}
\end{itemize}
Fazer rombo a.
Romper.
Despedaçar.
Arruinar.
Vencer.
\section{Arromper}
\begin{itemize}
\item {Grp. gram.:v. t.}
\end{itemize}
\begin{itemize}
\item {Utilização:Ant.}
\end{itemize}
\begin{itemize}
\item {Proveniência:(De \textunderscore romper\textunderscore )}
\end{itemize}
O mesmo que \textunderscore arrotear\textunderscore .
\section{Arrosetado}
\begin{itemize}
\item {Grp. gram.:adj.}
\end{itemize}
Que tem fórma de roseta.
\section{Arrostar}
\begin{itemize}
\item {Grp. gram.:v. t.  e  i.}
\end{itemize}
\begin{itemize}
\item {Grp. gram.:V. p.}
\end{itemize}
Fazer rosto a; resistir.
Encarar de frente.
Encontrar-se de frente. Cf. Filinto. \textunderscore D. Man.\textunderscore , I, 225; Sousa, \textunderscore Vida do Arceb.\textunderscore , I, 49.
\section{Arrôta}
\begin{itemize}
\item {Grp. gram.:f.}
\end{itemize}
\begin{itemize}
\item {Utilização:T. da Bairrada}
\end{itemize}
O mesmo que \textunderscore arroteia\textunderscore .
\section{Arrotador}
\begin{itemize}
\item {Grp. gram.:m.}
\end{itemize}
Aquelle que arrota.
Fanfarrão.
\section{Arrotadura}
\begin{itemize}
\item {Grp. gram.:f.}
\end{itemize}
\begin{itemize}
\item {Utilização:Náut.}
\end{itemize}
Volta de um cabo, com que se liga um mastro a um madeiro, para o fortificar.
\section{Arrotar}
\begin{itemize}
\item {Grp. gram.:v. i.}
\end{itemize}
\begin{itemize}
\item {Grp. gram.:V. t.}
\end{itemize}
\begin{itemize}
\item {Proveniência:(Do lat. \textunderscore ructare\textunderscore )}
\end{itemize}
Dar arrotos.
Vangloriar-se.
Alardear: \textunderscore arrotar postas de pescada\textunderscore , vangloriar-se de rico, de homem que não precisa dos outros.
\section{Arrotéa}
\begin{itemize}
\item {Grp. gram.:f.}
\end{itemize}
(V.arroteia)
\section{Arroteado}
\begin{itemize}
\item {Grp. gram.:adj.}
\end{itemize}
Que se arroteou.
Desbravado.
\section{Arroteador}
\begin{itemize}
\item {Grp. gram.:m.}
\end{itemize}
Aquelle que arroteia.
\section{Arroteamento}
\begin{itemize}
\item {Grp. gram.:m.}
\end{itemize}
Acto de \textunderscore arrotear\textunderscore .
\section{Arrotear}
\begin{itemize}
\item {Grp. gram.:v. t.}
\end{itemize}
\begin{itemize}
\item {Proveniência:(De \textunderscore arroteia\textunderscore )}
\end{itemize}
Romper (terreno inculto).
Desbravar, para cultivar.
Educar.
\section{Arroteia}
\begin{itemize}
\item {Grp. gram.:f.}
\end{itemize}
\begin{itemize}
\item {Utilização:Prov.}
\end{itemize}
\begin{itemize}
\item {Utilização:alg.}
\end{itemize}
\begin{itemize}
\item {Proveniência:(Do b. lat. \textunderscore arruptela\textunderscore , do lat. \textunderscore ruptus\textunderscore , roto)}
\end{itemize}
Terra, que se rompeu de novo, para começar a sêr cultivada.
Queima do mato, em terra não lavrada, para esta se adubar com a cinza.
\section{Arrôto}
\begin{itemize}
\item {Grp. gram.:m.}
\end{itemize}
\begin{itemize}
\item {Proveniência:(De \textunderscore arrotar\textunderscore )}
\end{itemize}
Gases, que sáem do estômago com ruído.
\section{Arrotova}
\begin{itemize}
\item {Grp. gram.:m.}
\end{itemize}
\begin{itemize}
\item {Utilização:Ant.}
\end{itemize}
\begin{itemize}
\item {Proveniência:(Do ár. \textunderscore ar-rotabe\textunderscore )}
\end{itemize}
O mesmo que \textunderscore arrocova\textunderscore .
\section{Arroubamento}
\begin{itemize}
\item {Grp. gram.:m.}
\end{itemize}
Acto de \textunderscore arroubar\textunderscore .
Arrebatamento; êxtase, enlêvo.
\section{Arroubar}
\begin{itemize}
\item {Grp. gram.:v. t.}
\end{itemize}
Arrebatar; extasiar; enlevar.
Assombrar.
(Cp. \textunderscore roubar\textunderscore )
\section{Arroubo}
\begin{itemize}
\item {Grp. gram.:m.}
\end{itemize}
\begin{itemize}
\item {Proveniência:(De \textunderscore arroubar\textunderscore )}
\end{itemize}
Enlêvo; êxtase; encanto.
\section{Arrouçar}
\begin{itemize}
\item {Grp. gram.:v. t.}
\end{itemize}
\begin{itemize}
\item {Utilização:Prov.}
\end{itemize}
Arrastar.
\section{Arroupar}
\begin{itemize}
\item {Grp. gram.:v. t.}
\end{itemize}
(V.enroupar)
\section{Arroxar}
\begin{itemize}
\item {Grp. gram.:v. t.}
\end{itemize}
O mesmo que \textunderscore arroxear\textunderscore .
\section{Arroxeado}
\begin{itemize}
\item {Grp. gram.:adj.}
\end{itemize}
Que se arroxeou.
Tirante a roxo.
\section{Arroxear}
\begin{itemize}
\item {Grp. gram.:v. t.}
\end{itemize}
Tornar roxo.
\section{Arroz}
\begin{itemize}
\item {fónica:rôs}
\end{itemize}
\begin{itemize}
\item {Grp. gram.:m.}
\end{itemize}
\begin{itemize}
\item {Grp. gram.:Adj.}
\end{itemize}
\begin{itemize}
\item {Utilização:Prov.}
\end{itemize}
\begin{itemize}
\item {Utilização:trasm.}
\end{itemize}
\begin{itemize}
\item {Proveniência:(Do ár. \textunderscore ar-roze\textunderscore )}
\end{itemize}
Planta gramínea.
O grão dessa planta.
Preparação culinária, em que entra o arroz, como parte principal.
Diz-se de uma variedade de feijão.
\section{Arrozal}
\begin{itemize}
\item {Grp. gram.:m.}
\end{itemize}
Lugar, onde se cultiva arroz.
\section{Arrozalva}
\begin{itemize}
\item {Grp. gram.:f.}
\end{itemize}
\begin{itemize}
\item {Utilização:Bras}
\end{itemize}
Farinha de arroz.
\section{Arroz-do-mato}
\begin{itemize}
\item {Grp. gram.:m.}
\end{itemize}
\begin{itemize}
\item {Utilização:Bras}
\end{itemize}
O mesmo que \textunderscore arrózia\textunderscore .
\section{Arrozeira}
\begin{itemize}
\item {Grp. gram.:f.}
\end{itemize}
O mesmo que \textunderscore arrozal\textunderscore .
\section{Arrozeiro}
\begin{itemize}
\item {Grp. gram.:m.}
\end{itemize}
\begin{itemize}
\item {Grp. gram.:Adj.}
\end{itemize}
Aquelle que cultiva arroz.
Negociante de arroz.
Que gosta muito de arroz.
\section{Arrózia}
\begin{itemize}
\item {Grp. gram.:f.}
\end{itemize}
Gênero de plantas gramíneas, originárias do Brasil.
\section{Arruaça}
\begin{itemize}
\item {Grp. gram.:f.}
\end{itemize}
\begin{itemize}
\item {Proveniência:(De \textunderscore arruar\textunderscore )}
\end{itemize}
Motim nas ruas; alvorôto, tumulto popular.
\section{Arruaçar}
\begin{itemize}
\item {Grp. gram.:v. i.}
\end{itemize}
Fazer arruaça.
\section{Arruaceiro}
\begin{itemize}
\item {Grp. gram.:m.}
\end{itemize}
Aquelle que faz arruaça.
\section{Arruadeira}
\begin{itemize}
\item {Grp. gram.:f.}
\end{itemize}
\begin{itemize}
\item {Proveniência:(De \textunderscore arruar\textunderscore )}
\end{itemize}
Mulher, que anda muito na rua.
Rameira.
\section{Arruador}
\begin{itemize}
\item {Grp. gram.:m.}
\end{itemize}
\begin{itemize}
\item {Proveniência:(De \textunderscore arruar\textunderscore )}
\end{itemize}
Vadio.
Arruaceiro.
Aquelle que arrua ou tem a seu cargo o alinhamento das construcções.
\section{Arruamento}
\begin{itemize}
\item {Grp. gram.:m.}
\end{itemize}
Série de edifícios ou estabelecimentos, dispostos ao longo ou aos lados de uma rua.
Acto de \textunderscore arruar\textunderscore .
\section{Arruante}
\begin{itemize}
\item {Grp. gram.:m.}
\end{itemize}
\begin{itemize}
\item {Utilização:T. de Lanhoso}
\end{itemize}
\begin{itemize}
\item {Proveniência:(De \textunderscore arruar\textunderscore )}
\end{itemize}
Aquelle que anda nas esfolhadas, disfarçado, a fazer galanteios.
\section{Arruar}
\begin{itemize}
\item {Grp. gram.:v. t.}
\end{itemize}
\begin{itemize}
\item {Grp. gram.:V. i.}
\end{itemize}
Dividir em ruas.
Distribuir pelas ruas.
Alinhar (passeios e ruas).
Passear ostentosamente.
Vadiar.
\section{Arruar}
\begin{itemize}
\item {Grp. gram.:v. i.}
\end{itemize}
\begin{itemize}
\item {Proveniência:(T. onom.)}
\end{itemize}
O mesmo que \textunderscore grunhir\textunderscore .
\section{Arruçado}
\begin{itemize}
\item {Grp. gram.:adj.}
\end{itemize}
Que arruçou.
Que encaneceu.
\section{Arruçar}
\begin{itemize}
\item {Grp. gram.:v. i.}
\end{itemize}
Tornar-se ruço, encanecer.
\section{Arruda}
\begin{itemize}
\item {Grp. gram.:f.}
\end{itemize}
\begin{itemize}
\item {Proveniência:(Do lat. \textunderscore ruta\textunderscore )}
\end{itemize}
Gênero de plantas rutáceas, (\textunderscore ruta graveolens\textunderscore ).
\section{Arruda-dos-muros}
\begin{itemize}
\item {Grp. gram.:f.}
\end{itemize}
Planta medicinal do Brasil, (\textunderscore asplenium ruta-muraria\textunderscore , Lin.).
\section{Arrudão}
\begin{itemize}
\item {Grp. gram.:m.}
\end{itemize}
Planta, do gênero arruda.
\section{Arrudia}
\begin{itemize}
\item {Grp. gram.:f.}
\end{itemize}
Planta clusiácea do Brasil.
\section{Arruela}
\begin{itemize}
\item {Grp. gram.:f.}
\end{itemize}
\begin{itemize}
\item {Utilização:Heráld.}
\end{itemize}
\begin{itemize}
\item {Utilização:Prov.}
\end{itemize}
\begin{itemize}
\item {Utilização:extrem.}
\end{itemize}
Círculo, em fórma de moéda, nos escudos heráldicos.
Ornato, como o besante, mas de côr e não de metal.
Pedaço de prata, vasado pelos ourives no tijolo.
Chapa de ferro, na ponta da cavilha.
Poço, em que recolhem as águas dos terrenos mais altos, para dali se escoarem por sargetas.
(Cp. \textunderscore rodela\textunderscore )
\section{Arruelado}
\begin{itemize}
\item {Grp. gram.:adj.}
\end{itemize}
Que tem arruelas.
\section{Arrufada}
\begin{itemize}
\item {Grp. gram.:f.}
\end{itemize}
\begin{itemize}
\item {Proveniência:(De \textunderscore arrufar\textunderscore )}
\end{itemize}
Bolo fofo, de farinha, ovos e açúcar.
\section{Arrufadamente}
\begin{itemize}
\item {Grp. gram.:adv.}
\end{itemize}
Com arrufo.
\section{Arrufadiço}
\begin{itemize}
\item {Grp. gram.:adj.}
\end{itemize}
Que facilmente se arrufa.
\section{Arrufanado}
\begin{itemize}
\item {Grp. gram.:adj.}
\end{itemize}
(V.arrufianado)
\section{Arrufar}
\begin{itemize}
\item {Grp. gram.:v. t.}
\end{itemize}
\begin{itemize}
\item {Grp. gram.:V. p.}
\end{itemize}
\begin{itemize}
\item {Proveniência:(Do al. \textunderscore rupfen\textunderscore )}
\end{itemize}
Irritar; tornar agastado.
Rufar.
Encrespar-se.
Entufar-se.
Desavir-se.
Mostrar mau modo, calando o motivo.
\section{Arrufianado}
\begin{itemize}
\item {Grp. gram.:adj.}
\end{itemize}
Que tem modos de rufião.
Próprio de rufião.
\section{Arrufo}
\begin{itemize}
\item {Grp. gram.:m.}
\end{itemize}
Acto de \textunderscore arrufar\textunderscore .
Mau humor.
Despeito.
Agastamento passageiro, entre pessôas que se astimam; amuo.
\section{Arrugar}
\textunderscore v. t.\textunderscore  (e der.)
O mesmo que \textunderscore enrugar\textunderscore , etc.
\section{Arrúgia}
\begin{itemize}
\item {Grp. gram.:f.}
\end{itemize}
\begin{itemize}
\item {Proveniência:(Lat. \textunderscore arrugia\textunderscore )}
\end{itemize}
Canal, para escoamento de águas nas minas.
\section{Arruído}
\begin{itemize}
\item {Grp. gram.:m.}
\end{itemize}
Clamor, vozearia confusa.
Desordem com gritos.
Festa ostentosa.
Ruído.
(Cp. \textunderscore ruído\textunderscore )
\section{Arruinador}
\begin{itemize}
\item {fónica:ru-i}
\end{itemize}
\begin{itemize}
\item {Grp. gram.:m.}
\end{itemize}
Aquelle que arruína.
\section{Arruinamento}
\begin{itemize}
\item {fónica:ru-i}
\end{itemize}
\begin{itemize}
\item {Grp. gram.:m.}
\end{itemize}
Acto de \textunderscore arruinar\textunderscore .
\section{Arruinar}
\begin{itemize}
\item {fónica:ru-i}
\end{itemize}
\begin{itemize}
\item {Grp. gram.:v. t.}
\end{itemize}
\begin{itemize}
\item {Grp. gram.:V. i.}
\end{itemize}
Causar ruína a: \textunderscore o tempo arruinou o palácio\textunderscore .
Demolir.
Estragar.
Tornar pobre: \textunderscore o jôgo arruína\textunderscore .
Desacreditar.
Fazer perder a saúde.
Cair em ruína, desmoronar-se:«\textunderscore este pagode arruinou e caiu\textunderscore ». \textunderscore Ethiópia Or.\textunderscore , II, 82.
\section{Arruivado}
\begin{itemize}
\item {Grp. gram.:adj.}
\end{itemize}
Tirante a ruivo.
\section{Arruivascado}
\begin{itemize}
\item {Grp. gram.:adj.}
\end{itemize}
Tirante a ruivo.
\section{Arrular}
\begin{itemize}
\item {Grp. gram.:v. i.}
\end{itemize}
O mesmo que \textunderscore arrulhar\textunderscore .
\section{Arrulhar}
\begin{itemize}
\item {Grp. gram.:v. i.}
\end{itemize}
Cantar como as rôlas e os pombos.
Galantear alguém.
Acalentar crianças.
(Cast. \textunderscore arrollar\textunderscore )
\section{Arrulho}
\begin{itemize}
\item {Grp. gram.:m.}
\end{itemize}
Acto de \textunderscore arrulhar\textunderscore .
Canto ou toada, com que se adormentam crianças.
\section{Arrumação}
\begin{itemize}
\item {Grp. gram.:f.}
\end{itemize}
\begin{itemize}
\item {Utilização:Náut.}
\end{itemize}
Acto de \textunderscore arrumar\textunderscore .
Conjunto dos sinaes atmosphéricos, pelos quaes o marítimo conhece no mar que, por determinado rumo, se encontra terra próxima.
\section{Arrumadeira}
\begin{itemize}
\item {Grp. gram.:adj. f.}
\end{itemize}
\begin{itemize}
\item {Grp. gram.:F.}
\end{itemize}
\begin{itemize}
\item {Utilização:Bras}
\end{itemize}
\begin{itemize}
\item {Proveniência:(De \textunderscore arrumar\textunderscore )}
\end{itemize}
Diz-se da mulher, cuidadosa na arrumação ou bôa disposição do mobiliário doméstico.
Criada de quartos, criada para serviços domesticos, que não sejam cozinhar nem esfregar e limpar soalhos:«\textunderscore aluga-se uma bôa arrumadeira, de nacionalidade portuguesa...\textunderscore »\textunderscore Jorn. do Com.\textunderscore , do Rio, de 12-V-912.
\section{Arrumadela}
\begin{itemize}
\item {Grp. gram.:f.}
\end{itemize}
(V.arrumação). Cf. Eça, \textunderscore P. Amaro\textunderscore , 466.
\section{Arrumador}
\begin{itemize}
\item {Grp. gram.:m.}
\end{itemize}
\begin{itemize}
\item {Utilização:Bras}
\end{itemize}
Aquelle que arruma.
Criado de quartos ou incumbido da disposição e limpeza do mobiliário:«\textunderscore Aluga-se um moço português para arrumador\textunderscore ». \textunderscore Jorn. do Com.\textunderscore , de 12-V-912.
\section{Arrumamento}
\begin{itemize}
\item {Grp. gram.:m.}
\end{itemize}
O mesmo que \textunderscore arrumação\textunderscore .
\section{Arrumar}
\begin{itemize}
\item {Grp. gram.:v. t.}
\end{itemize}
\begin{itemize}
\item {Utilização:Pop.}
\end{itemize}
Collocar convenientemente; pôr em ordem: \textunderscore arrumar livros\textunderscore .
Pôr de lado, deixar.
Dar: \textunderscore arrumar um pontapé\textunderscore .
Empregar num offício ou indústria: \textunderscore arrumar um rapaz numa fábrica\textunderscore .
Conseguir o casamento de: \textunderscore afinal, arrumou a filha\textunderscore .
Dirigir em certo rumo.
(Por \textunderscore arrimar\textunderscore )
\section{Arrumo}
\begin{itemize}
\item {Grp. gram.:m.}
\end{itemize}
O mesmo que \textunderscore arrumação\textunderscore .
\section{Arrunhado}
\begin{itemize}
\item {Grp. gram.:adj.}
\end{itemize}
\begin{itemize}
\item {Utilização:Prov.}
\end{itemize}
\begin{itemize}
\item {Utilização:trasm.}
\end{itemize}
\begin{itemize}
\item {Proveniência:(De \textunderscore arrunhar\textunderscore )}
\end{itemize}
Que tem a saúde gasta; arruinado.
\section{Arrunhar}
\begin{itemize}
\item {Grp. gram.:v. t.}
\end{itemize}
\begin{itemize}
\item {Utilização:Prov.}
\end{itemize}
\begin{itemize}
\item {Utilização:ant.}
\end{itemize}
Arruinar.
Rasgar, abrir:«\textunderscore a corrente fazia arrunhar enormes campas de areia\textunderscore ». \textunderscore Museu Techn.\textunderscore , 52. Cf. Barros, \textunderscore Déc. II\textunderscore , l. I, c. 5; \textunderscore Eufrosina\textunderscore , 271.
\section{Arsenal}
\begin{itemize}
\item {Grp. gram.:m.}
\end{itemize}
\begin{itemize}
\item {Proveniência:(Do ár. \textunderscore dar-cenáa\textunderscore )}
\end{itemize}
Estabelecimento, onde se fabricam e reparam navios.
Depósito de petrechos de guerra.
Archivo.
\section{Arseníaco}
\begin{itemize}
\item {Grp. gram.:adj.}
\end{itemize}
Diz-se de um ácido composto de arsênico e oxygênio.
\section{Arseniado}
\begin{itemize}
\item {Grp. gram.:adj.}
\end{itemize}
O mesmo que \textunderscore arsenicado\textunderscore .
\section{Arseniatado}
\begin{itemize}
\item {Grp. gram.:adj.}
\end{itemize}
Em que há arseniato.
\section{Arseniato}
\begin{itemize}
\item {Grp. gram.:m.}
\end{itemize}
Sal, composto de ácido arsênico e uma base.
\section{Arsenicado}
\begin{itemize}
\item {Grp. gram.:adj.}
\end{itemize}
Combinado com arsênico.
\section{Arsenical}
\begin{itemize}
\item {Grp. gram.:adj.}
\end{itemize}
Que tem arsênico.
Relativo a arsênico.
\section{Arsenicíase}
\begin{itemize}
\item {Grp. gram.:f.}
\end{itemize}
Entoxicação arsenical chrónica.
\section{Arsenicismo}
\begin{itemize}
\item {Grp. gram.:m.}
\end{itemize}
\begin{itemize}
\item {Utilização:Neol.}
\end{itemize}
Doença, produzida pelo uso ou abuso do arsênico.
\section{Arsenicita}
\begin{itemize}
\item {Grp. gram.:f.}
\end{itemize}
O mesmo ou melhor que arsenicite.
\section{Arsenicite}
\begin{itemize}
\item {Grp. gram.:f.}
\end{itemize}
\begin{itemize}
\item {Proveniência:(De \textunderscore arsênico\textunderscore )}
\end{itemize}
Arseniato de cal.
\section{Arsênico}
\begin{itemize}
\item {Grp. gram.:m.}
\end{itemize}
\begin{itemize}
\item {Grp. gram.:Adj.}
\end{itemize}
\begin{itemize}
\item {Proveniência:(Gr. \textunderscore arsenikon\textunderscore )}
\end{itemize}
Arsênio.
Ácido arsenioso.
O mesmo que \textunderscore arseníaco\textunderscore .
\section{Arsenieto}
\begin{itemize}
\item {fónica:ê}
\end{itemize}
\begin{itemize}
\item {Grp. gram.:m.}
\end{itemize}
\begin{itemize}
\item {Utilização:Chím.}
\end{itemize}
Combinação de arsênico com outro corpo simples.
\section{Arsenífero}
\begin{itemize}
\item {Grp. gram.:adj.}
\end{itemize}
(V.arseniado)
\section{Arsênio}
\begin{itemize}
\item {Grp. gram.:m.}
\end{itemize}
Metal pardo, luzidio, e muito friável.
O mesmo que \textunderscore arsênico\textunderscore .
\section{Arsenioso}
\begin{itemize}
\item {Grp. gram.:adj.}
\end{itemize}
\begin{itemize}
\item {Proveniência:(De \textunderscore arsênio\textunderscore )}
\end{itemize}
Diz-se do ácido, que resulta da combinação do arsênico e do oxygênio.
\section{Arsenito}
\begin{itemize}
\item {Grp. gram.:m.}
\end{itemize}
O mesmo que \textunderscore arseniato\textunderscore .
\section{Arseniurado}
\begin{itemize}
\item {Grp. gram.:adj.}
\end{itemize}
Diz-se, indevidamente, de qualquer metal arsenicado.
(Cp. \textunderscore arseniureto\textunderscore )
\section{Arseniureto}
\begin{itemize}
\item {Grp. gram.:m.}
\end{itemize}
(V.arsenieto)
\section{Arsenizita}
\begin{itemize}
\item {Grp. gram.:f.}
\end{itemize}
(V.arsenicita)
\section{Arsenoterapia}
\begin{itemize}
\item {Grp. gram.:f.}
\end{itemize}
\begin{itemize}
\item {Utilização:Med.}
\end{itemize}
Tratamento médico por meio do arsênico.
\section{Arsenoterápico}
\begin{itemize}
\item {Grp. gram.:adj.}
\end{itemize}
Relativo á \textunderscore arsenotherapia\textunderscore .
\section{Arsenotherapia}
\begin{itemize}
\item {Grp. gram.:f.}
\end{itemize}
\begin{itemize}
\item {Utilização:Med.}
\end{itemize}
Tratamento médico por meio do arsênico.
\section{Arsenotherápico}
\begin{itemize}
\item {Grp. gram.:adj.}
\end{itemize}
Relativo á \textunderscore arsenotherapia\textunderscore .
\section{Arses}
\begin{itemize}
\item {Grp. gram.:m.}
\end{itemize}
Ave africana, da fam. dos dentirostros.
\section{Ársis}
\begin{itemize}
\item {Grp. gram.:f.}
\end{itemize}
\begin{itemize}
\item {Proveniência:(Gr. \textunderscore arsis\textunderscore )}
\end{itemize}
Elevação de tom.
Elevação da voz.
\section{Artabótris}
\begin{itemize}
\item {Grp. gram.:m.}
\end{itemize}
\begin{itemize}
\item {Utilização:Bot.}
\end{itemize}
Gênero de anonáceas.
\section{Ártabros}
\begin{itemize}
\item {Grp. gram.:m. pl.}
\end{itemize}
Povos do Noroéste da Espanha, na vizinhança do Finisterra, que se chamou Ártabro. Cf. Herculano, \textunderscore Hist. de Port.\textunderscore , I, 16.
\section{Artâmia}
\begin{itemize}
\item {Grp. gram.:f.}
\end{itemize}
Gênero de aves africanas.
\section{Artanita}
\begin{itemize}
\item {Grp. gram.:f.}
\end{itemize}
\begin{itemize}
\item {Proveniência:(Do gr. \textunderscore artos\textunderscore )}
\end{itemize}
Planta medicinal, da fam. das primuláceas, (\textunderscore cyclamen europaeum\textunderscore , Lin.).
\section{Artão}
\begin{itemize}
\item {Grp. gram.:m.}
\end{itemize}
\begin{itemize}
\item {Utilização:Gír.}
\end{itemize}
Pão.
\section{Arte}
\begin{itemize}
\item {Grp. gram.:f.}
\end{itemize}
\begin{itemize}
\item {Utilização:Pesc.}
\end{itemize}
\begin{itemize}
\item {Proveniência:(Lat. \textunderscore ars\textunderscore , \textunderscore artis\textunderscore )}
\end{itemize}
Conjunto de preceitos, para bem dizer ou fazer qualquer coisa.
Livro, tratado, que contém aquelles preceitos.
Artifício.
Habilidade.
Ardil.
Maldade.
Offício.
Modo.
\textunderscore Arte de pesca\textunderscore , apparelho de rede de arrastar.
\section{Artecida}
\begin{itemize}
\item {Grp. gram.:m.}
\end{itemize}
\begin{itemize}
\item {Utilização:Fig.}
\end{itemize}
Aquelle que maltrata a arte. Cf. Cortesão, \textunderscore Subs.\textunderscore 
\section{Artefacto}
\begin{itemize}
\item {Grp. gram.:m.}
\end{itemize}
\begin{itemize}
\item {Proveniência:(De \textunderscore arte\textunderscore  + \textunderscore facto\textunderscore )}
\end{itemize}
Producto de artes mechânicas.
\section{Arteiramente}
\begin{itemize}
\item {Grp. gram.:adv.}
\end{itemize}
De modo \textunderscore arteiro\textunderscore .
\section{Arteirice}
\begin{itemize}
\item {Grp. gram.:f.}
\end{itemize}
\begin{itemize}
\item {Proveniência:(De \textunderscore arteiro\textunderscore )}
\end{itemize}
Manha; astúcia; ardil.
\section{Arteiro}
\begin{itemize}
\item {Grp. gram.:adj.}
\end{itemize}
Que tem arte.
Manhoso; astuto.
\section{Arteiroso}
\begin{itemize}
\item {Grp. gram.:adj.}
\end{itemize}
\begin{itemize}
\item {Utilização:Des.}
\end{itemize}
\begin{itemize}
\item {Proveniência:(De \textunderscore arteiro\textunderscore )}
\end{itemize}
Destro; fino; manhoso.
\section{Artejano}
\begin{itemize}
\item {Grp. gram.:m.}
\end{itemize}
\begin{itemize}
\item {Utilização:Des.}
\end{itemize}
O mesmo que \textunderscore artífice\textunderscore :«\textunderscore qualquer artejano de Roma\textunderscore ». Vieira, \textunderscore Cartas\textunderscore .
(Cast. \textunderscore artesano\textunderscore )
\section{Artelete}
\begin{itemize}
\item {fónica:lê}
\end{itemize}
\begin{itemize}
\item {Grp. gram.:m.}
\end{itemize}
Guisado, pastel ou torta, de pedaços de ave ou vitella.
(Cast. \textunderscore artelete\textunderscore )
\section{Artelharia}
\begin{itemize}
\item {Grp. gram.:f.}
\end{itemize}
\begin{itemize}
\item {Grp. gram.:Pl.}
\end{itemize}
\begin{itemize}
\item {Utilização:Ant.}
\end{itemize}
\begin{itemize}
\item {Proveniência:(De \textunderscore artilhar\textunderscore )}
\end{itemize}
Material de guerra, constituido por vários gêneros de bocas de fogo.
Tropa, empregada no serviço da artilharia.
Uma das classes do exército.
Preparativo para uma aggressão verbal ou discussão.
Qualquer meio poderoso de ataque ou defesa.
O mesmo ou melhor que \textunderscore artilharia\textunderscore . Cf. \textunderscore Peregrinação\textunderscore , c. I, 3; Tenreiro, \textunderscore Itiner.\textunderscore , c. XL; \textunderscore Inéd. da Hist. de Port.\textunderscore , I, 547.
Trastes de casa, mobília.
\section{Artelho}
\begin{itemize}
\item {fónica:tê}
\end{itemize}
\begin{itemize}
\item {Grp. gram.:m.}
\end{itemize}
\begin{itemize}
\item {Proveniência:(Lat. \textunderscore articulus\textunderscore )}
\end{itemize}
Extremidade inferior, saliente e arredondada, dos ossos da perna, na sua articulação com o pé.
Tornozelo.
\section{Arte-mágica}
\begin{itemize}
\item {Grp. gram.:f.}
\end{itemize}
\begin{itemize}
\item {Utilização:Pop.}
\end{itemize}
O mesmo que \textunderscore magia\textunderscore .
Arte de feiticeiro; manigâncias.
\section{Artemágico}
\begin{itemize}
\item {Grp. gram.:m.}
\end{itemize}
\begin{itemize}
\item {Utilização:Pop.}
\end{itemize}
Feiticeiro; nigromante:«\textunderscore os artemágicos e as bruxas e feiticeiras aproveitam-se de braços de defuntos\textunderscore ». Bernardes, \textunderscore N. Floresta\textunderscore , II, 242.
\section{Arte-maior}
\begin{itemize}
\item {Grp. gram.:f.}
\end{itemize}
Diz-se de \textunderscore arte-maior\textunderscore  o verso castelhano e português de onze sýllabas.
\section{Artemão}
\begin{itemize}
\item {Grp. gram.:m.}
\end{itemize}
\begin{itemize}
\item {Proveniência:(Lat. \textunderscore artemon\textunderscore )}
\end{itemize}
Vela mestra de navio.
\section{Artêmia}
\begin{itemize}
\item {Grp. gram.:f.}
\end{itemize}
Gênero de molluscos dos pântanos de água salgada.
\section{Artemije}
\begin{itemize}
\item {Grp. gram.:f.}
\end{itemize}
\begin{itemize}
\item {Utilização:Prov.}
\end{itemize}
\begin{itemize}
\item {Utilização:beir.}
\end{itemize}
O mesmo que \textunderscore artemísia\textunderscore .
\section{Artemísia}
\begin{itemize}
\item {Grp. gram.:f.}
\end{itemize}
\begin{itemize}
\item {Proveniência:(Lat. \textunderscore artemisia\textunderscore )}
\end{itemize}
Gênero de plantas, da fam. das compostas.
\section{Artemisila}
\begin{itemize}
\item {Grp. gram.:f.}
\end{itemize}
Planta silvestre, do gênero artemísia.
\section{Artemisina}
\begin{itemize}
\item {Grp. gram.:f.}
\end{itemize}
Princípio amargo, que se extrái da artemísia.
\section{Artena}
\begin{itemize}
\item {Grp. gram.:f.}
\end{itemize}
Ave aquática palmípede.
\section{Artena}
\begin{itemize}
\item {Grp. gram.:f.}
\end{itemize}
Gênero de aranhas.
\section{Artequim}
\begin{itemize}
\item {Grp. gram.:m.}
\end{itemize}
Fruta indiana, que se applica contra a lepra.
\section{Artéria}
\begin{itemize}
\item {Grp. gram.:f.}
\end{itemize}
\begin{itemize}
\item {Utilização:Anat.}
\end{itemize}
\begin{itemize}
\item {Proveniência:(Lat. \textunderscore arteria\textunderscore )}
\end{itemize}
Cada um dos vasos, que levam sangue a todas as partes do corpo.
Grande via de communicação: \textunderscore o Chiado é uma das principaes artérias de Lisbôa\textunderscore .
\section{Arteríaco}
\begin{itemize}
\item {Grp. gram.:adj.}
\end{itemize}
\begin{itemize}
\item {Proveniência:(De \textunderscore artéria\textunderscore )}
\end{itemize}
Diz-se do medicamento, applicável ás doenças da tracheia e da larynge.
\section{Arterial}
\begin{itemize}
\item {Grp. gram.:adj.}
\end{itemize}
Relativo á artéria.
\section{Arterialização}
\begin{itemize}
\item {Grp. gram.:f.}
\end{itemize}
Acto de \textunderscore arterializar\textunderscore .
\section{Arterializar}
\begin{itemize}
\item {Grp. gram.:v. t.}
\end{itemize}
Transformar (o sangue venoso) em sangue arterial.
\section{Arterioesclerose}
\begin{itemize}
\item {fónica:té}
\end{itemize}
\begin{itemize}
\item {Grp. gram.:f.}
\end{itemize}
\begin{itemize}
\item {Utilização:Med.}
\end{itemize}
Esclerose das túnicas arteriaes.
\section{Arteriografia}
\begin{itemize}
\item {Grp. gram.:f.}
\end{itemize}
\begin{itemize}
\item {Proveniência:(Do gr. \textunderscore arteria\textunderscore  + \textunderscore graphein\textunderscore )}
\end{itemize}
Parte da Anatomia, que descreve o systema arterial.
\section{Arteriographia}
\begin{itemize}
\item {Grp. gram.:f.}
\end{itemize}
\begin{itemize}
\item {Proveniência:(Do gr. \textunderscore arteria\textunderscore  + \textunderscore graphein\textunderscore )}
\end{itemize}
Parte da Anatomia, que descreve o systema arterial.
\section{Arteríola}
\begin{itemize}
\item {Grp. gram.:f.}
\end{itemize}
Pequena artéria.
\section{Arteriologia}
\begin{itemize}
\item {Grp. gram.:f.}
\end{itemize}
\begin{itemize}
\item {Proveniência:(Do gr. \textunderscore arteria\textunderscore  + \textunderscore logos\textunderscore )}
\end{itemize}
Tratado do systema arterial.
\section{Arteriomérico}
\begin{itemize}
\item {Grp. gram.:adj.}
\end{itemize}
Relativo ao \textunderscore arteriómero\textunderscore .
\section{Arteriómero}
\begin{itemize}
\item {Grp. gram.:m.}
\end{itemize}
\begin{itemize}
\item {Proveniência:(Do gr. \textunderscore arteria\textunderscore  + \textunderscore meros\textunderscore )}
\end{itemize}
Parte do systema arterial representada no metâmero.
\section{Arteriorisma}
\begin{itemize}
\item {Grp. gram.:m.}
\end{itemize}
Dilatação anormal de uma artéria.
O mesmo que \textunderscore aneurisma\textunderscore .
\section{Arterioso}
\begin{itemize}
\item {Grp. gram.:adj.}
\end{itemize}
(V.arterial)
\section{Arteriotomia}
\begin{itemize}
\item {Grp. gram.:f.}
\end{itemize}
\begin{itemize}
\item {Proveniência:(Gr. \textunderscore  arteriotomia\textunderscore )}
\end{itemize}
Sangria numa artéria.
\section{Arterite}
\begin{itemize}
\item {Grp. gram.:f.}
\end{itemize}
Inflammação nas artérias.
\section{Artesa}
\begin{itemize}
\item {Grp. gram.:f.}
\end{itemize}
Caixote de quatro faces iguaes, que vai estreitando para o fundo e serve para amassadoiro do pão.
\section{Artesano}
\begin{itemize}
\item {Grp. gram.:m.}
\end{itemize}
\begin{itemize}
\item {Utilização:Ant.}
\end{itemize}
O mesmo que \textunderscore artesão\textunderscore ^1.
\section{Artesão}
\begin{itemize}
\item {Grp. gram.:m.}
\end{itemize}
\begin{itemize}
\item {Utilização:Ant.}
\end{itemize}
O mesmo que \textunderscore artífice\textunderscore . Cf. Filinto, \textunderscore D. Man.\textunderscore , III, 72 e 117.
(Cast. \textunderscore artesano\textunderscore )
\section{Artesão}
\begin{itemize}
\item {Grp. gram.:m.}
\end{itemize}
\begin{itemize}
\item {Proveniência:(De \textunderscore artesa\textunderscore )}
\end{itemize}
Lavor entre molduras, nas abóbadas, voltas de arcos e tectos.
\section{Artesiano}
\begin{itemize}
\item {Grp. gram.:adj.}
\end{itemize}
Diz-se dos poços, abertos por broca ou sonda, e donde repuxa a água acima do nível do solo.
(B. lat. \textunderscore artesianus\textunderscore )
\section{Artesoar}
\begin{itemize}
\item {Grp. gram.:v. t.}
\end{itemize}
Ornar com artesões.
\section{Artesonar}
\begin{itemize}
\item {Grp. gram.:v. t.}
\end{itemize}
O mesmo que \textunderscore artesoar\textunderscore .
\section{Arthanitha}
\begin{itemize}
\item {Grp. gram.:f.}
\end{itemize}
Planta medicinal, (\textunderscore cyclamen europaeum\textunderscore , Lin.)
\section{Arthena}
\begin{itemize}
\item {Grp. gram.:f.}
\end{itemize}
Gênero de aranhas.
\section{Arthralgia}
\begin{itemize}
\item {Grp. gram.:f.}
\end{itemize}
\begin{itemize}
\item {Proveniência:(Do gr. \textunderscore arthron\textunderscore  + \textunderscore algos\textunderscore )}
\end{itemize}
Dôres nas articulações.
\section{Arthrite}
\begin{itemize}
\item {Grp. gram.:f.}
\end{itemize}
\begin{itemize}
\item {Proveniência:(Gr. \textunderscore arthritis\textunderscore )}
\end{itemize}
Inflammação nas articulações.
\section{Arthrítico}
\begin{itemize}
\item {Grp. gram.:adj.}
\end{itemize}
\begin{itemize}
\item {Proveniência:(Gr. \textunderscore arthritikos\textunderscore )}
\end{itemize}
Relativo á arthrite.
Que padece arthrite.
\section{Arthritina}
\begin{itemize}
\item {Grp. gram.:f.}
\end{itemize}
\begin{itemize}
\item {Proveniência:(De \textunderscore arthrite\textunderscore )}
\end{itemize}
Nome de um medicamento contra o arthritismo, a diáthese úrica, etc.
\section{Arthritismo}
\begin{itemize}
\item {Grp. gram.:m.}
\end{itemize}
Estado de arthrítico, arthrite; diáthese arthrítica.
\section{Arthrocéphalo}
\begin{itemize}
\item {Grp. gram.:adj.}
\end{itemize}
\begin{itemize}
\item {Proveniência:(Do gr. \textunderscore arthron\textunderscore  + \textunderscore kephale\textunderscore )}
\end{itemize}
Diz-se dos crustáceos, que têm a cabeça separada do thorax.
\section{Arthrodia}
\begin{itemize}
\item {Grp. gram.:f.}
\end{itemize}
\begin{itemize}
\item {Proveniência:(Gr. \textunderscore arthrodia\textunderscore )}
\end{itemize}
Articulação, resultante do encaixe de uma pequena saliência óssea em uma pequena cavidade.
\section{Arthrodial}
\begin{itemize}
\item {Grp. gram.:adj.}
\end{itemize}
Relativo á \textunderscore arthrodia\textunderscore .
\section{Arthroídeas}
\begin{itemize}
\item {Grp. gram.:f. pl.}
\end{itemize}
\begin{itemize}
\item {Proveniência:(Do gr. \textunderscore arthron\textunderscore  + \textunderscore eidos\textunderscore )}
\end{itemize}
Seres orgânicos, compostos de filamentos articulados.
Fam. de plantas aquáticas, segundo Bory.
\section{Arthrolóbio}
\begin{itemize}
\item {Grp. gram.:m.}
\end{itemize}
\begin{itemize}
\item {Proveniência:(Do gr. \textunderscore arthron\textunderscore , articulação, e \textunderscore lobos\textunderscore , vagem)}
\end{itemize}
Gênero de plantas leguminosas.
\section{Arthrologia}
\begin{itemize}
\item {Grp. gram.:f.}
\end{itemize}
\begin{itemize}
\item {Proveniência:(Do gr. \textunderscore arthron\textunderscore  + \textunderscore logos\textunderscore )}
\end{itemize}
Tratado das articulações.
\section{Arthrológico}
\begin{itemize}
\item {Grp. gram.:adj.}
\end{itemize}
Relativo á \textunderscore arthrologia\textunderscore .
\section{Arthromeral}
\begin{itemize}
\item {Grp. gram.:adj.}
\end{itemize}
Relativo ao \textunderscore arthrómero\textunderscore .
\section{Arthromérico}
\begin{itemize}
\item {Grp. gram.:adj.}
\end{itemize}
Relativo ao \textunderscore arthrómero\textunderscore .
\section{Arthrómero}
\begin{itemize}
\item {Grp. gram.:m.}
\end{itemize}
\begin{itemize}
\item {Proveniência:(Do gr. \textunderscore arthron\textunderscore  + \textunderscore meros\textunderscore )}
\end{itemize}
Parte ligamentosa do metâmero.
\section{Arthropathia}
\begin{itemize}
\item {Grp. gram.:f.}
\end{itemize}
\begin{itemize}
\item {Proveniência:(Do gr. \textunderscore arthron\textunderscore  + \textunderscore pathos\textunderscore )}
\end{itemize}
Doença nas articulações.
\section{Arthrópode}
\begin{itemize}
\item {Grp. gram.:m.}
\end{itemize}
\begin{itemize}
\item {Grp. gram.:Pl.}
\end{itemize}
\begin{itemize}
\item {Proveniência:(Do gr. \textunderscore arthron\textunderscore  + \textunderscore pous\textunderscore , \textunderscore podos\textunderscore )}
\end{itemize}
Planta herbácea da Austrália.
Typo zoológico, a que pertencem os bichos da seda.
Grupo de animaes marínhos, em que se comprehendem os crustáceos.
\section{Arthropódio}
\begin{itemize}
\item {Grp. gram.:m.}
\end{itemize}
O mesmo que \textunderscore arthrópode\textunderscore .
\section{Arthropyose}
\begin{itemize}
\item {Grp. gram.:f.}
\end{itemize}
\begin{itemize}
\item {Proveniência:(Do gr. \textunderscore arthron\textunderscore  + \textunderscore puon\textunderscore )}
\end{itemize}
Suppuração de uma articulação.
\section{Arthrozoário}
\begin{itemize}
\item {Grp. gram.:adj.}
\end{itemize}
\begin{itemize}
\item {Proveniência:(Do gr. \textunderscore arthron\textunderscore  + \textunderscore zoarion\textunderscore )}
\end{itemize}
O mesmo que \textunderscore articulado\textunderscore  (animal).
\section{Artice}
\begin{itemize}
\item {Grp. gram.:f.}
\end{itemize}
\begin{itemize}
\item {Utilização:Ant.}
\end{itemize}
\begin{itemize}
\item {Proveniência:(De \textunderscore arte\textunderscore )}
\end{itemize}
O mesmo que \textunderscore arteirice\textunderscore .
\section{Articida}
\begin{itemize}
\item {Grp. gram.:m.}
\end{itemize}
\begin{itemize}
\item {Utilização:Fig.}
\end{itemize}
Aquelle que maltrata a arte. Cf. Cortesão, \textunderscore Subs.\textunderscore 
\section{Articulação}
\begin{itemize}
\item {Grp. gram.:f.}
\end{itemize}
\begin{itemize}
\item {Proveniência:(Lat. \textunderscore articulatio\textunderscore )}
\end{itemize}
Juntura natural de ossos.
União entre peças de um apparelho ou máquina.
Reunião dos artículos dos animaes articulados.
Som articulado da voz.
Pronúncia das palavras.
Exposição ou allegação, por meio de parágraphos separados e, geralmente, numerados.
\section{Articulação}
\begin{itemize}
\item {Grp. gram.:f.}
\end{itemize}
\begin{itemize}
\item {Utilização:Prov.}
\end{itemize}
Discussão.
Descompostura.
(Talvez corr. de \textunderscore altercação\textunderscore )
\section{Articuladamente}
\begin{itemize}
\item {Grp. gram.:adv.}
\end{itemize}
\begin{itemize}
\item {Proveniência:(De \textunderscore articulado\textunderscore )}
\end{itemize}
Claramente.
Por meio de artigos ou parágraphos separados.
\section{Articulado}
\begin{itemize}
\item {Grp. gram.:m.}
\end{itemize}
\begin{itemize}
\item {Grp. gram.:M. pl.}
\end{itemize}
\begin{itemize}
\item {Utilização:Zool.}
\end{itemize}
Exposição jurídica em parágraphos ou artigos.
Grande divisão do reino animal.
\section{Articulante}
\begin{itemize}
\item {Grp. gram.:m.}
\end{itemize}
\begin{itemize}
\item {Proveniência:(Lat. \textunderscore articulans\textunderscore )}
\end{itemize}
Aquelle que articula.
\section{Articular}
\begin{itemize}
\item {Grp. gram.:adj.}
\end{itemize}
\begin{itemize}
\item {Proveniência:(Lat. \textunderscore articularis\textunderscore )}
\end{itemize}
Relativo ás articulações.
\section{Articular}
\begin{itemize}
\item {Grp. gram.:v. t.}
\end{itemize}
\begin{itemize}
\item {Utilização:Jur.}
\end{itemize}
Levar (os ossos articulares) á sua posição natural.
Reunir pelas articulações.
Expor em artículos ou em parágraphos separados: \textunderscore articular o libello\textunderscore .
Juntar por meio de cadeias.
Pronunciar: \textunderscore articular palavras\textunderscore .
\section{Articular}
\begin{itemize}
\item {Grp. gram.:v. i.}
\end{itemize}
\begin{itemize}
\item {Utilização:Prov.}
\end{itemize}
O mesmo que \textunderscore discutir\textunderscore .
(Alter. de \textunderscore altercar\textunderscore ?)
\section{Articulável}
\begin{itemize}
\item {Grp. gram.:adj.}
\end{itemize}
Que se póde \textunderscore articular\textunderscore .
\section{Articulista}
\begin{itemize}
\item {Grp. gram.:m.}
\end{itemize}
\begin{itemize}
\item {Proveniência:(De \textunderscore artículo\textunderscore )}
\end{itemize}
Aquelle que escreve artigos em jornaes.
\section{Artículo}
\begin{itemize}
\item {Grp. gram.:m.}
\end{itemize}
\begin{itemize}
\item {Proveniência:(Lat. \textunderscore articulus\textunderscore )}
\end{itemize}
Juntura dos ossos.
Phalange dos dedos.
Espaço entre os nós de um caule.
Segmento dos appêndices dos animaes articulados.
Artigo, divisão de um trabalho literário, scientífico ou forense.
\section{Articuloso}
\begin{itemize}
\item {Grp. gram.:adj.}
\end{itemize}
Que tem artículos ou que é composto de artículos.
\section{Artife}
\begin{itemize}
\item {Grp. gram.:m.}
\end{itemize}
\begin{itemize}
\item {Utilização:ant.}
\end{itemize}
\begin{itemize}
\item {Utilização:Gír.}
\end{itemize}
Pão.
\section{Artífice}
\begin{itemize}
\item {Grp. gram.:m.}
\end{itemize}
\begin{itemize}
\item {Proveniência:(Lat. \textunderscore artifex\textunderscore )}
\end{itemize}
Aquelle que exerce uma arte mecânica.
Operário.
Autor de um artefacto.
Inventor.
\section{Artificial}
\begin{itemize}
\item {Grp. gram.:adj.}
\end{itemize}
\begin{itemize}
\item {Grp. gram.:M.}
\end{itemize}
\begin{itemize}
\item {Utilização:Ant.}
\end{itemize}
\begin{itemize}
\item {Proveniência:(Lat. \textunderscore artificialis\textunderscore )}
\end{itemize}
Produzido por arte ou indústria.
Dissimulado.
Artífice.
\section{Artificialidade}
\begin{itemize}
\item {Grp. gram.:f.}
\end{itemize}
Qualidade do que é \textunderscore artificial\textunderscore .
\section{Artificialismo}
\begin{itemize}
\item {Grp. gram.:m.}
\end{itemize}
O mesmo que \textunderscore artificialidade\textunderscore . Cf. Camillo, \textunderscore Narcót.\textunderscore , II, 229.
\section{Artificialmente}
\begin{itemize}
\item {Grp. gram.:adv.}
\end{itemize}
De modo \textunderscore artificial\textunderscore .
\section{Artificiar}
\begin{itemize}
\item {Grp. gram.:v. t.}
\end{itemize}
Fazer com artifício.
Máquinar.
\section{Artifício}
\begin{itemize}
\item {Grp. gram.:m.}
\end{itemize}
\begin{itemize}
\item {Proveniência:(Lat. \textunderscore artificium\textunderscore )}
\end{itemize}
Meios, com que se obtém um artefacto.
Producto de arte.
Habilidade.
Astúcia.
Fingimento.
Trabalho pyrotéchnico.
\section{Artificiosamente}
\begin{itemize}
\item {Grp. gram.:adv.}
\end{itemize}
De modo \textunderscore artificioso\textunderscore .
Com artifício.
\section{Artificioso}
\begin{itemize}
\item {Grp. gram.:adj.}
\end{itemize}
\begin{itemize}
\item {Proveniência:(Lat. \textunderscore artificiosus\textunderscore )}
\end{itemize}
Que tem artifício.
Em que há artifício.
\section{Artigar}
\begin{itemize}
\item {Grp. gram.:v. t.}
\end{itemize}
\begin{itemize}
\item {Utilização:Prov.}
\end{itemize}
\begin{itemize}
\item {Utilização:dur.}
\end{itemize}
Allegar, deduzindo por artigos; articular.
\section{Artigo}
\begin{itemize}
\item {Grp. gram.:m.}
\end{itemize}
\begin{itemize}
\item {Proveniência:(Lat. \textunderscore articulus\textunderscore )}
\end{itemize}
Palavra, que precede os substantivos ou palavras substantivadas, para lhes dar feição definida ou indefinida.
Cada um dos assumptos, em que se divide um escrito.
Cada um dos parágraphos, em que se dividem certas petições, contestações e outros escritos forenses. Conjunctura: \textunderscore em artigo de morte\textunderscore .
Cada um dos escritos de uma fôlha periódica, mais extensos que uma simples notícia.
Cada uma das divisões de um trabalho, numeradas.
Objecto de mercadoria: \textunderscore estabelecimento com muitos artigos\textunderscore .
Cada um dos pontos doutrinários do Credo.
\section{Artiguelho}
\begin{itemize}
\item {fónica:guê}
\end{itemize}
\begin{itemize}
\item {Grp. gram.:m.}
\end{itemize}
\begin{itemize}
\item {Utilização:Deprec.}
\end{itemize}
Artigo insignificante; insignificante escrito, publicado em gazeta.
\section{Artilhado}
\begin{itemize}
\item {Grp. gram.:adj.}
\end{itemize}
Guarnecido de peças de artilharia: \textunderscore navio artilhado\textunderscore .
\section{Artilhamento}
\begin{itemize}
\item {Grp. gram.:m.}
\end{itemize}
Acto de \textunderscore artilhar\textunderscore .
\section{Artilhar}
\begin{itemize}
\item {Grp. gram.:v. t.}
\end{itemize}
\begin{itemize}
\item {Utilização:Fig.}
\end{itemize}
Guarnecer com artilharia.
Preparar com argumentos.
(B. lat. \textunderscore artillare\textunderscore )
\section{Artilharia}
\begin{itemize}
\item {Grp. gram.:f.}
\end{itemize}
\begin{itemize}
\item {Grp. gram.:Pl.}
\end{itemize}
\begin{itemize}
\item {Utilização:Ant.}
\end{itemize}
\begin{itemize}
\item {Proveniência:(De \textunderscore artilhar\textunderscore )}
\end{itemize}
Material de guerra, constituido por vários gêneros de bocas de fogo.
Tropa, empregada no serviço da artilharia.
Uma das classes do exército.
Preparativo para uma aggressão verbal ou discussão.
Qualquer meio poderoso de ataque ou defesa.
Trastes de casa, mobília.
\section{Artilheiro}
\begin{itemize}
\item {Grp. gram.:m.}
\end{itemize}
Soldado de artilharia.
\section{Artimanha}
\begin{itemize}
\item {Grp. gram.:f.}
\end{itemize}
\begin{itemize}
\item {Proveniência:(De \textunderscore arte\textunderscore  + \textunderscore manha\textunderscore )}
\end{itemize}
Artifício.
Ardil; astúcia.
\section{Artimão}
\begin{itemize}
\item {Grp. gram.:m.}
\end{itemize}
(V.artemão)
\section{Artim-graxa}
\begin{itemize}
\item {Grp. gram.:m.}
\end{itemize}
Certo mineral, que se descobriu nas margens do Zézere.
\section{Artiodáctilos}
\begin{itemize}
\item {Grp. gram.:m. pl.}
\end{itemize}
\begin{itemize}
\item {Utilização:Zool.}
\end{itemize}
\begin{itemize}
\item {Proveniência:(Do gr. \textunderscore artios\textunderscore , par + \textunderscore daktulos\textunderscore , dedo)}
\end{itemize}
Ordem de mamíferos, que comprehende os animaes, cujos dedos são em número par, como o boi, o veado, o javali.
\section{Artiodáctylos}
\begin{itemize}
\item {Grp. gram.:m. pl.}
\end{itemize}
\begin{itemize}
\item {Utilização:Zool.}
\end{itemize}
\begin{itemize}
\item {Proveniência:(Do gr. \textunderscore artios\textunderscore , par + \textunderscore daktulos\textunderscore , dedo)}
\end{itemize}
Ordem de mamíferos, que comprehende os animaes, cujos dedos são em número par, como o boi, o veado, o javali.
\section{Artiozooário}
\begin{itemize}
\item {Grp. gram.:m.}
\end{itemize}
\begin{itemize}
\item {Utilização:Zool.}
\end{itemize}
\begin{itemize}
\item {Proveniência:(Do gr. \textunderscore artios\textunderscore  + \textunderscore zoon\textunderscore )}
\end{itemize}
Animal de configuração symétrica, ou de fórma symétrica bilateral.
\section{Artista}
\begin{itemize}
\item {Grp. gram.:m.}
\end{itemize}
\begin{itemize}
\item {Utilização:Pop.}
\end{itemize}
\begin{itemize}
\item {Grp. gram.:Adj.}
\end{itemize}
Aquelle que cultiva artes liberaes.
Aquelle que faz profissão de uma arte.
Operário.
Que tem gênio.
Que ama as artes.
Engenhoso.
Manhoso.
(B. lat. \textunderscore artista\textunderscore )
\section{Artisticamente}
\begin{itemize}
\item {Grp. gram.:adv.}
\end{itemize}
De modo \textunderscore artístico\textunderscore .
\section{Artístico}
\begin{itemize}
\item {Grp. gram.:adj.}
\end{itemize}
\begin{itemize}
\item {Proveniência:(De \textunderscore artista\textunderscore )}
\end{itemize}
Que diz respeito ás artes.
Feito com arte.
\section{Artocarpáceas}
\begin{itemize}
\item {Grp. gram.:f. pl.}
\end{itemize}
O mesmo ou melhor que \textunderscore artocárpeas\textunderscore . Cf. Latino, \textunderscore Humboldt\textunderscore , 180 e 198.
\section{Artocárpeas}
\begin{itemize}
\item {Grp. gram.:f. pl.}
\end{itemize}
Família de plantas, que tem por typo o \textunderscore artocarpo\textunderscore .
\section{Artocarpo}
\begin{itemize}
\item {Grp. gram.:m.}
\end{itemize}
\begin{itemize}
\item {Proveniência:(Do gr. \textunderscore artos\textunderscore  + \textunderscore karpos\textunderscore )}
\end{itemize}
Gênero de plantas urticáceas.
Árvore da Oceânia, mais conhecida por \textunderscore árvore-do-pão\textunderscore .
\section{Artócopo}
\begin{itemize}
\item {Grp. gram.:m.}
\end{itemize}
\begin{itemize}
\item {Proveniência:(Lat. \textunderscore artocopus\textunderscore )}
\end{itemize}
Escravo, que, entre os Romanos, partia o pão á mesa.
\section{Artódio}
\begin{itemize}
\item {Grp. gram.:m.}
\end{itemize}
\begin{itemize}
\item {Proveniência:(Do gr. \textunderscore artos\textunderscore  + \textunderscore eidos\textunderscore )}
\end{itemize}
Gênero de coleópteros.
\section{Artófago}
\begin{itemize}
\item {Grp. gram.:adj.}
\end{itemize}
\begin{itemize}
\item {Proveniência:(Do gr. \textunderscore arthos\textunderscore  + \textunderscore phagein\textunderscore )}
\end{itemize}
Que prefere o pão a outro alimento.
\section{Artóforo}
\begin{itemize}
\item {Grp. gram.:m.}
\end{itemize}
\begin{itemize}
\item {Utilização:Ant.}
\end{itemize}
\begin{itemize}
\item {Proveniência:(Do gr. \textunderscore artos\textunderscore  + \textunderscore phoros\textunderscore )}
\end{itemize}
Cofre, em que se guardavam as hóstias consagradas.
\section{Artola}
\begin{itemize}
\item {Grp. gram.:f. ?}
\end{itemize}
\begin{itemize}
\item {Utilização:Bras}
\end{itemize}
Espécie de liteira? Cf. V. de Taunay, \textunderscore Ret. de Laguna\textunderscore , 14.
\section{Artola}
\begin{itemize}
\item {Grp. gram.:m.}
\end{itemize}
\begin{itemize}
\item {Utilização:Prov.}
\end{itemize}
\begin{itemize}
\item {Utilização:beir.}
\end{itemize}
Estroina; valdevinos.
Janota com pouco juízo.
(Cp. \textunderscore arlota\textunderscore )
\section{Artolar}
\begin{itemize}
\item {Grp. gram.:v. i.}
\end{itemize}
Viver como artola; vadiar.
\section{Artólatra}
\begin{itemize}
\item {Grp. gram.:m.}
\end{itemize}
\begin{itemize}
\item {Proveniência:(Do gr. \textunderscore artos\textunderscore  + \textunderscore latrenein\textunderscore )}
\end{itemize}
Adorador do pão.
\section{Artolatria}
\begin{itemize}
\item {Grp. gram.:f.}
\end{itemize}
Adoração do pão.
(Cp. \textunderscore artólatra\textunderscore )
\section{Artolice}
\begin{itemize}
\item {Grp. gram.:f.}
\end{itemize}
Acto ou modos de artola.
\section{Artólitho}
\begin{itemize}
\item {Grp. gram.:m.}
\end{itemize}
\begin{itemize}
\item {Proveniência:(Do gr. \textunderscore artos\textunderscore  + \textunderscore lithos\textunderscore )}
\end{itemize}
Concreção pétrea, em fórma de pão, e privativa dos terrenos terciários.
\section{Artólito}
\begin{itemize}
\item {Grp. gram.:m.}
\end{itemize}
\begin{itemize}
\item {Proveniência:(Do gr. \textunderscore artos\textunderscore  + \textunderscore lithos\textunderscore )}
\end{itemize}
Concreção pétrea, em fórma de pão, e privativa dos terrenos terciários.
\section{Artomel}
\begin{itemize}
\item {Grp. gram.:m.}
\end{itemize}
\begin{itemize}
\item {Proveniência:(Do gr. \textunderscore artos\textunderscore , e de \textunderscore mel\textunderscore )}
\end{itemize}
Cataplasma de pão e mel.
\section{Artoméli}
\begin{itemize}
\item {Grp. gram.:m.}
\end{itemize}
(V.artomel)
\section{Artonomia}
\begin{itemize}
\item {Grp. gram.:f.}
\end{itemize}
\begin{itemize}
\item {Proveniência:(Do gr. \textunderscore artos\textunderscore  + \textunderscore nomos\textunderscore )}
\end{itemize}
Arte de fabricar pão.
\section{Artonómico}
\begin{itemize}
\item {Grp. gram.:adj.}
\end{itemize}
Relativo á artonomia.
\section{Artóphago}
\begin{itemize}
\item {Grp. gram.:adj.}
\end{itemize}
\begin{itemize}
\item {Proveniência:(Do gr. \textunderscore arthos\textunderscore  + \textunderscore phagein\textunderscore )}
\end{itemize}
Que prefere o pão a outro alimento.
\section{Artóphoro}
\begin{itemize}
\item {Grp. gram.:m.}
\end{itemize}
\begin{itemize}
\item {Utilização:Ant.}
\end{itemize}
\begin{itemize}
\item {Proveniência:(Do gr. \textunderscore artos\textunderscore  + \textunderscore phoros\textunderscore )}
\end{itemize}
Cofre, em que se guardavam as hóstias consagradas.
\section{Artopta}
\begin{itemize}
\item {Grp. gram.:m.}
\end{itemize}
\begin{itemize}
\item {Proveniência:(Lat. \textunderscore artopta\textunderscore , que Freund julga substituição apócrypha de \textunderscore artocopus\textunderscore )}
\end{itemize}
O mesmo que \textunderscore artócopo\textunderscore .
\section{Artoso}
\begin{itemize}
\item {Grp. gram.:adj.}
\end{itemize}
\begin{itemize}
\item {Utilização:Prov.}
\end{itemize}
Que tem arte; que é habilidoso.
(Colhido em Turquel)
\section{Artralgia}
\begin{itemize}
\item {Grp. gram.:f.}
\end{itemize}
\begin{itemize}
\item {Proveniência:(Do gr. \textunderscore arthron\textunderscore  + \textunderscore algos\textunderscore )}
\end{itemize}
Dôres nas articulações.
\section{Artrite}
\begin{itemize}
\item {Grp. gram.:f.}
\end{itemize}
\begin{itemize}
\item {Proveniência:(Gr. \textunderscore arthritis\textunderscore )}
\end{itemize}
Inflammação nas articulações.
\section{Artrítico}
\begin{itemize}
\item {Grp. gram.:adj.}
\end{itemize}
\begin{itemize}
\item {Proveniência:(Gr. \textunderscore arthritikos\textunderscore )}
\end{itemize}
Relativo á arthrite.
Que padece arthrite.
\section{Artritina}
\begin{itemize}
\item {Grp. gram.:f.}
\end{itemize}
\begin{itemize}
\item {Proveniência:(De \textunderscore arthrite\textunderscore )}
\end{itemize}
Nome de um medicamento contra o artritismo, a diáthese úrica, etc.
\section{Artritismo}
\begin{itemize}
\item {Grp. gram.:m.}
\end{itemize}
Estado de artrítico, artrite; diáthese artrítica.
\section{Artrocéfalo}
\begin{itemize}
\item {Grp. gram.:adj.}
\end{itemize}
\begin{itemize}
\item {Proveniência:(Do gr. \textunderscore arthron\textunderscore  + \textunderscore kephale\textunderscore )}
\end{itemize}
Diz-se dos crustáceos, que têm a cabeça separada do thorax.
\section{Artrodia}
\begin{itemize}
\item {Grp. gram.:f.}
\end{itemize}
\begin{itemize}
\item {Proveniência:(Gr. \textunderscore arthrodia\textunderscore )}
\end{itemize}
Articulação, resultante do encaixe de uma pequena saliência óssea em uma pequena cavidade.
\section{Artrodial}
\begin{itemize}
\item {Grp. gram.:adj.}
\end{itemize}
Relativo á \textunderscore artrodia\textunderscore .
\section{Artroídeas}
\begin{itemize}
\item {Grp. gram.:f. pl.}
\end{itemize}
\begin{itemize}
\item {Proveniência:(Do gr. \textunderscore arthron\textunderscore  + \textunderscore eidos\textunderscore )}
\end{itemize}
Seres orgânicos, compostos de filamentos articulados.
Fam. de plantas aquáticas, segundo Bory.
\section{Artrolóbio}
\begin{itemize}
\item {Grp. gram.:m.}
\end{itemize}
\begin{itemize}
\item {Proveniência:(Do gr. \textunderscore arthron\textunderscore , articulação, e \textunderscore lobos\textunderscore , vagem)}
\end{itemize}
Gênero de plantas leguminosas.
\section{Artrologia}
\begin{itemize}
\item {Grp. gram.:f.}
\end{itemize}
\begin{itemize}
\item {Proveniência:(Do gr. \textunderscore arthron\textunderscore  + \textunderscore logos\textunderscore )}
\end{itemize}
Tratado das articulações.
\section{Artrológico}
\begin{itemize}
\item {Grp. gram.:adj.}
\end{itemize}
Relativo á \textunderscore artrologia\textunderscore .
\section{Artromeral}
\begin{itemize}
\item {Grp. gram.:adj.}
\end{itemize}
Relativo ao \textunderscore artrómero\textunderscore .
\section{Artromérico}
\begin{itemize}
\item {Grp. gram.:adj.}
\end{itemize}
Relativo ao \textunderscore artrómero\textunderscore .
\section{Artrómero}
\begin{itemize}
\item {Grp. gram.:m.}
\end{itemize}
\begin{itemize}
\item {Proveniência:(Do gr. \textunderscore arthron\textunderscore  + \textunderscore meros\textunderscore )}
\end{itemize}
Parte ligamentosa do metâmero.
\section{Artropatia}
\begin{itemize}
\item {Grp. gram.:f.}
\end{itemize}
\begin{itemize}
\item {Proveniência:(Do gr. \textunderscore arthron\textunderscore  + \textunderscore pathos\textunderscore )}
\end{itemize}
Doença nas articulações.
\section{Artropiose}
\begin{itemize}
\item {Grp. gram.:f.}
\end{itemize}
\begin{itemize}
\item {Proveniência:(Do gr. \textunderscore arthron\textunderscore  + \textunderscore puon\textunderscore )}
\end{itemize}
Suppuração de uma articulação.
\section{Artrópode}
\begin{itemize}
\item {Grp. gram.:m.}
\end{itemize}
\begin{itemize}
\item {Grp. gram.:Pl.}
\end{itemize}
\begin{itemize}
\item {Proveniência:(Do gr. \textunderscore arthron\textunderscore  + \textunderscore pous\textunderscore , \textunderscore podos\textunderscore )}
\end{itemize}
Planta herbácea da Austrália.
Typo zoológico, a que pertencem os bichos da seda.
Grupo de animaes marínhos, em que se comprehendem os crustáceos.
\section{Artropódio}
\begin{itemize}
\item {Grp. gram.:m.}
\end{itemize}
O mesmo que \textunderscore artrópode\textunderscore .
\section{Artrozoário}
\begin{itemize}
\item {Grp. gram.:adj.}
\end{itemize}
\begin{itemize}
\item {Proveniência:(Do gr. \textunderscore arthron\textunderscore  + \textunderscore zoarion\textunderscore )}
\end{itemize}
O mesmo que \textunderscore articulado\textunderscore  (animal).
\section{Aru}
\begin{itemize}
\item {Grp. gram.:m.}
\end{itemize}
\begin{itemize}
\item {Utilização:Bras}
\end{itemize}
Espécie de sapo das regiões do Amazonas.
\section{Aruá}
\begin{itemize}
\item {Grp. gram.:adj.}
\end{itemize}
\begin{itemize}
\item {Utilização:Bras}
\end{itemize}
\begin{itemize}
\item {Grp. gram.:M.}
\end{itemize}
Desconfiado.
Espantadiço.
Pouco dócil.
Pequeno mollusco brasileiro.
(Do guarani?)
\section{Aruanan}
\begin{itemize}
\item {Grp. gram.:m.}
\end{itemize}
Peixe do Purus, no Brasil.
\section{Aruanás}
\begin{itemize}
\item {Grp. gram.:m. pl.}
\end{itemize}
Indígenas brasileiros das margens do Juruá.
\section{Aruans}
\begin{itemize}
\item {Grp. gram.:m. pl.}
\end{itemize}
Tribo indígena, que habitou no Pará.
\section{Aruaquis}
\begin{itemize}
\item {Grp. gram.:m. pl.}
\end{itemize}
Tribo de selvagens, que habitou no Pará.
\section{Aruba}
\begin{itemize}
\item {Grp. gram.:f.}
\end{itemize}
Arbusto da Guiana.
\section{Arubé}
\begin{itemize}
\item {Grp. gram.:m.}
\end{itemize}
\begin{itemize}
\item {Utilização:Bras. do Pará}
\end{itemize}
Massa de mandioca, com sal, alho e pimenta, a qual se desfaz em môlho de peixe, para servir do tempêro á mesa.
\section{Arucerrana}
\begin{itemize}
\item {Grp. gram.:f.}
\end{itemize}
Árvore silvestre do Brasil.
\section{Arujo}
\begin{itemize}
\item {Grp. gram.:m.}
\end{itemize}
\begin{itemize}
\item {Utilização:Prov.}
\end{itemize}
Argueiro.
Faúla.
Pequena maravalha, que caiu na comida.
(Cast. \textunderscore orujo\textunderscore )
\section{Árula}
\begin{itemize}
\item {Grp. gram.:f.}
\end{itemize}
\begin{itemize}
\item {Proveniência:(Lat. \textunderscore arula\textunderscore )}
\end{itemize}
Pequeno altar.
\section{Árum}
\begin{itemize}
\item {Grp. gram.:m.}
\end{itemize}
\begin{itemize}
\item {Proveniência:(Lat. \textunderscore arum\textunderscore )}
\end{itemize}
Gênero de plantas, que serve de typo ás aroídeas, e que é vulgarmente conhecido por \textunderscore jarro\textunderscore ; o mesmo que \textunderscore arão\textunderscore .
\section{Aruman}
\begin{itemize}
\item {Grp. gram.:m.}
\end{itemize}
\begin{itemize}
\item {Utilização:Bras}
\end{itemize}
Espécie do junco, com que se fazem paneiros, balaios, etc.
\section{Arunco}
\begin{itemize}
\item {Grp. gram.:m.}
\end{itemize}
\begin{itemize}
\item {Proveniência:(Lat. \textunderscore arruncus\textunderscore )}
\end{itemize}
Planta rosácea, mais conhecida por \textunderscore barba de cabra\textunderscore .
Espécie de sapo.
\section{Arunda}
\begin{itemize}
\item {Grp. gram.:f.}
\end{itemize}
\begin{itemize}
\item {Proveniência:(Do lat. \textunderscore arundo\textunderscore )}
\end{itemize}
Gramínea, que se emprega para fixar terra areenta.
\section{Arundina}
\begin{itemize}
\item {Grp. gram.:f.}
\end{itemize}
\begin{itemize}
\item {Proveniência:(Do lat. \textunderscore arundo\textunderscore )}
\end{itemize}
Gênero de plantas orientaes, da fam. das orchídeas.
\section{Arundináceas}
\begin{itemize}
\item {Grp. gram.:f. pl.}
\end{itemize}
\begin{itemize}
\item {Proveniência:(De \textunderscore arundináceo\textunderscore )}
\end{itemize}
Tribo de plantas da fam. das gramíneas, e de caule fistuloso.
\section{Arundináceo}
\begin{itemize}
\item {Grp. gram.:adj.}
\end{itemize}
\begin{itemize}
\item {Proveniência:(Lat. \textunderscore arundinaceus\textunderscore )}
\end{itemize}
Relativo ou semelhante á cana.
\section{Arundinela}
\begin{itemize}
\item {Grp. gram.:f.}
\end{itemize}
\begin{itemize}
\item {Proveniência:(Do lat. \textunderscore arundo\textunderscore )}
\end{itemize}
Gênero de plantas gramíneas.
\section{Arundíneo}
\begin{itemize}
\item {Grp. gram.:adj.}
\end{itemize}
\begin{itemize}
\item {Proveniência:(Lat. \textunderscore arundineus\textunderscore )}
\end{itemize}
Feito de cana.
\section{Arundinoso}
\begin{itemize}
\item {Grp. gram.:adj.}
\end{itemize}
\begin{itemize}
\item {Proveniência:(Lat. \textunderscore arundinosus\textunderscore )}
\end{itemize}
Que produz canas.
Abundante de canas.
\section{Arurão}
\begin{itemize}
\item {Grp. gram.:m.}
\end{itemize}
Grande jacaré do Brasil.
\section{Aruspicação}
\begin{itemize}
\item {Grp. gram.:f.}
\end{itemize}
Sciência dos arúspices.
\section{Aruspicatório}
\begin{itemize}
\item {Grp. gram.:adj.}
\end{itemize}
Relativo aos arúspices.
\section{Arúspice}
\begin{itemize}
\item {Grp. gram.:m.}
\end{itemize}
\begin{itemize}
\item {Proveniência:(Lat. \textunderscore aruspex\textunderscore )}
\end{itemize}
Sacerdote romano, que fazia prognósticos, consultando as entranhas das víctimas.
\section{Aruspicina}
\begin{itemize}
\item {Grp. gram.:f.}
\end{itemize}
\begin{itemize}
\item {Proveniência:(Lat. \textunderscore aruspicina\textunderscore )}
\end{itemize}
Sciência ou arte dos arúspices; o mesmo que \textunderscore aruspicação\textunderscore .
\section{Aruspicínia}
\begin{itemize}
\item {Grp. gram.:f.}
\end{itemize}
O mesmo que \textunderscore aruspicina\textunderscore . Cf. Castilho, \textunderscore Fastos\textunderscore , III, 314 e 315.
\section{Aruspicino}
\begin{itemize}
\item {Grp. gram.:adj.}
\end{itemize}
\begin{itemize}
\item {Proveniência:(Lat. \textunderscore aruspicinus\textunderscore )}
\end{itemize}
Relativo aos arúspices.
\section{Aruspício}
\begin{itemize}
\item {Grp. gram.:m.}
\end{itemize}
\begin{itemize}
\item {Proveniência:(Lat. \textunderscore aruspicium\textunderscore )}
\end{itemize}
Prognóstico, feito pelos arúspices.
\section{Aruspicismo}
\begin{itemize}
\item {Grp. gram.:m.}
\end{itemize}
O mesmo que \textunderscore aruspicina\textunderscore .
\section{Arval}
\begin{itemize}
\item {Grp. gram.:adj.}
\end{itemize}
\begin{itemize}
\item {Proveniência:(Lat. \textunderscore arvalis\textunderscore )}
\end{itemize}
Relativo a terras cultivadas.
Campesino.
\section{Arvela}
\begin{itemize}
\item {Grp. gram.:f.}
\end{itemize}
(Fórma pop. de \textunderscore alvéloa\textunderscore )
\section{Arvelas}
\begin{itemize}
\item {Grp. gram.:f. pl.}
\end{itemize}
Argolas que os marinheiros metem nas cavilhas, para segurar melhor as chavetas.
\section{Arveliça}
\begin{itemize}
\item {Grp. gram.:f.}
\end{itemize}
O mesmo que \textunderscore alvéloa\textunderscore .
\section{Arvelicha}
\begin{itemize}
\item {Grp. gram.:f.}
\end{itemize}
\begin{itemize}
\item {Utilização:Prov.}
\end{itemize}
\begin{itemize}
\item {Proveniência:(De \textunderscore arvela\textunderscore )}
\end{itemize}
Espécie de alvéola.
\section{Arvelinha}
\begin{itemize}
\item {Grp. gram.:f.}
\end{itemize}
Espécie de alvéola, (\textunderscore motacilla sulphurea\textunderscore , Bechst.).
(Cp. \textunderscore arvela\textunderscore ^1)
\section{Arvéloa}
\begin{itemize}
\item {Grp. gram.:f.}
\end{itemize}
O mesmo que \textunderscore alvéloa\textunderscore .
\section{Arvense}
\begin{itemize}
\item {Grp. gram.:adj.}
\end{itemize}
\begin{itemize}
\item {Proveniência:(Do lat. \textunderscore arvum\textunderscore )}
\end{itemize}
Que cresce em terras semeadas.
\section{Arvião}
\begin{itemize}
\item {Grp. gram.:m.}
\end{itemize}
O mesmo que \textunderscore gaivão\textunderscore ^1.
\section{Arvícola}
\begin{itemize}
\item {Grp. gram.:m.}
\end{itemize}
\begin{itemize}
\item {Utilização:Zool.}
\end{itemize}
\begin{itemize}
\item {Proveniência:(Do lat. \textunderscore arvum\textunderscore  + \textunderscore colere\textunderscore )}
\end{itemize}
Habitante do campo. Lavrador.
Mammífero roedor, da fam. dos murídeos.
\section{Arvicultura}
\begin{itemize}
\item {Grp. gram.:f.}
\end{itemize}
\begin{itemize}
\item {Proveniência:(Do lat. \textunderscore arvum\textunderscore  + \textunderscore cultura\textunderscore )}
\end{itemize}
Cultura de cereaes.
Cultura dos campos.
\section{Arvingel}
\begin{itemize}
\item {Grp. gram.:m.}
\end{itemize}
Espécie de embarcação do Tejo.
\section{Arvoamento}
\begin{itemize}
\item {Grp. gram.:m.}
\end{itemize}
Acto de \textunderscore arvoar\textunderscore .
\section{Arvoar}
\begin{itemize}
\item {Grp. gram.:v. t.  e  i.}
\end{itemize}
\begin{itemize}
\item {Proveniência:(Do lat. \textunderscore herbulare\textunderscore )}
\end{itemize}
Entontecer.
\section{Arvoragem}
\begin{itemize}
\item {Grp. gram.:f.}
\end{itemize}
Acto de \textunderscore arvorar\textunderscore .
\section{Arvorar}
\begin{itemize}
\item {Grp. gram.:v. t.}
\end{itemize}
\begin{itemize}
\item {Grp. gram.:V. i.}
\end{itemize}
\begin{itemize}
\item {Proveniência:(De \textunderscore árvore\textunderscore )}
\end{itemize}
Arborizar. Erguer perpendicularmente; hastear.
Desfraldar: \textunderscore arvorar uma bandeira\textunderscore .
Elevar a um cargo.
Fugir.
\section{Árvore}
\begin{itemize}
\item {Grp. gram.:f.}
\end{itemize}
\begin{itemize}
\item {Proveniência:(Lat. \textunderscore arbor\textunderscore )}
\end{itemize}
Vegetal lenhoso, de tronco elevado, com ramos na parte superior.
Peça principal de uma roda ou máquina.
Mastro.
Ramificações de uma família, descritas em fórma de árvore.
Nome de diversas crystallizações.
Navio.
\section{Arvorecente}
\begin{itemize}
\item {Grp. gram.:adj.}
\end{itemize}
(V.arborescente)
\section{Arvorecer}
\begin{itemize}
\item {Grp. gram.:v. i.}
\end{itemize}
\begin{itemize}
\item {Proveniência:(Lat. \textunderscore arborescere\textunderscore )}
\end{itemize}
Tomar as dimensões de árvore.
\section{Arvorecido}
\begin{itemize}
\item {Grp. gram.:adj.}
\end{itemize}
Que arvoreceu.
\section{Arvoreda}
\begin{itemize}
\item {Grp. gram.:f.}
\end{itemize}
\begin{itemize}
\item {Utilização:Ant.}
\end{itemize}
O mesmo que \textunderscore arvoredo\textunderscore . Cf. Usque, \textunderscore Tribulações\textunderscore , 10.
\section{Árvore-da-independência}
\begin{itemize}
\item {Grp. gram.:f.}
\end{itemize}
\begin{itemize}
\item {Utilização:Bras}
\end{itemize}
Planta ornamental, cujas fôlhas são matizadas de amarelo e verde.
\section{Árvore-da-prata}
\begin{itemize}
\item {Grp. gram.:f.}
\end{itemize}
Variedade de proteia, cujos rebentos têm folhas brancas, como a prata.
\section{Árvore-da-preguiça}
\begin{itemize}
\item {Grp. gram.:f.}
\end{itemize}
\begin{itemize}
\item {Utilização:Bras}
\end{itemize}
O mesmo que \textunderscore umbaúba\textunderscore .
\section{Árvore-da-vaca}
\begin{itemize}
\item {Grp. gram.:f.}
\end{itemize}
Planta artocárpea da América, (\textunderscore bosimum galactodendron\textunderscore ).
\section{Árvore-de-castidade}
\begin{itemize}
\item {Grp. gram.:f.}
\end{itemize}
O mesmo que \textunderscore agnocasto\textunderscore . Cf. \textunderscore Desengano da Med.\textunderscore , 71.
\section{Árvore-de-cato}
\begin{itemize}
\item {Grp. gram.:f.}
\end{itemize}
Nome do pau-ferro, na Índia Portuguesa.
\section{Árvore-de-coral}
\begin{itemize}
\item {Grp. gram.:f.}
\end{itemize}
Planta leguminosa, (\textunderscore erythrina poianthes\textunderscore , Brot.).
\section{Árvore-de-fôlha-parida}
\begin{itemize}
\item {Grp. gram.:f.}
\end{itemize}
Árvore burserácea da Índia, (\textunderscore garuga pinnata\textunderscore , Roxb.).
\section{Arvoredo}
\begin{itemize}
\item {fónica:vorê}
\end{itemize}
\begin{itemize}
\item {Grp. gram.:m.}
\end{itemize}
\begin{itemize}
\item {Proveniência:(De \textunderscore árvore\textunderscore )}
\end{itemize}
Lugar, onde vegetam árvores; conjunto de árvores; bosque.
Mastreação do navio.
\section{Árvore-do-nilo}
\begin{itemize}
\item {Grp. gram.:f.}
\end{itemize}
\begin{itemize}
\item {Utilização:Bras}
\end{itemize}
Planta aquática.
\section{Arvore-do-pão}
\begin{itemize}
\item {Grp. gram.:f.}
\end{itemize}
(V.artocarpo)
\section{Árvore-do-paraíso}
\begin{itemize}
\item {Grp. gram.:f.}
\end{itemize}
\begin{itemize}
\item {Utilização:Bras}
\end{itemize}
Espécie de oliveira, também chamada \textunderscore oliveira-da-bohêmia\textunderscore .
O mesmo que \textunderscore thuia\textunderscore .
\section{Árvore-do-ponto}
\begin{itemize}
\item {Grp. gram.:f.}
\end{itemize}
\begin{itemize}
\item {Utilização:T. de Coimbra}
\end{itemize}
O mesmo que \textunderscore tulipeiro\textunderscore , por começar a florir quando se põe ponto nos trabalhos universitários, para começarem as férias maiores.
\section{Arvorejar-se}
\begin{itemize}
\item {Grp. gram.:v. p.}
\end{itemize}
Cobrir-se de árvores, nascidas sem cultura.
\section{Arvorescer}
\textunderscore v. t.\textunderscore  (e der.)(V.arvorecer)
\section{Arvoreta}
\begin{itemize}
\item {fónica:vorê}
\end{itemize}
\begin{itemize}
\item {Grp. gram.:f.}
\end{itemize}
Pequena árvore; arvorezinha.
\section{Arvoriforme}
\begin{itemize}
\item {Grp. gram.:adj.}
\end{itemize}
O mesmo que \textunderscore arboriforme\textunderscore .
\section{Arxar}
\begin{itemize}
\item {Grp. gram.:v. t.}
\end{itemize}
Redrar (a vinha).
\section{Aritenoide}
\begin{itemize}
\item {Grp. gram.:m.  e  adj.}
\end{itemize}
\begin{itemize}
\item {Utilização:Anat.}
\end{itemize}
\begin{itemize}
\item {Proveniência:(Do gr. \textunderscore arutaina\textunderscore  + \textunderscore eidos\textunderscore )}
\end{itemize}
Diz-se das pequenas cartilagens, na parte póstero-superior da larynge.
\section{Arrítmico}
\begin{itemize}
\item {Grp. gram.:adj.}
\end{itemize}
\begin{itemize}
\item {Utilização:Med.}
\end{itemize}
\begin{itemize}
\item {Proveniência:(De \textunderscore arythmo\textunderscore )}
\end{itemize}
Que tem falta de ritmo.
Que tem pulsações irregulares.
\section{Arritmo}
\begin{itemize}
\item {Grp. gram.:m.}
\end{itemize}
\begin{itemize}
\item {Proveniência:(Do gr. \textunderscore a\textunderscore  priv. + \textunderscore ruthmos\textunderscore )}
\end{itemize}
Falta de ritmo.
Irregularidade nas pulsações.
\section{Arytenoide}
\begin{itemize}
\item {Grp. gram.:m.  e  adj.}
\end{itemize}
\begin{itemize}
\item {Utilização:Anat.}
\end{itemize}
\begin{itemize}
\item {Proveniência:(Do gr. \textunderscore arutaina\textunderscore  + \textunderscore eidos\textunderscore )}
\end{itemize}
Diz-se das pequenas cartilagens, na parte póstero-superior da larynge.
\section{Arýthmico}
\begin{itemize}
\item {fónica:ri}
\end{itemize}
\begin{itemize}
\item {Grp. gram.:adj.}
\end{itemize}
\begin{itemize}
\item {Utilização:Med.}
\end{itemize}
\begin{itemize}
\item {Proveniência:(De \textunderscore arythmo\textunderscore )}
\end{itemize}
Que tem falta de rythmo.
Que tem pulsações irregulares.
\section{Arythmo}
\begin{itemize}
\item {fónica:ri}
\end{itemize}
\begin{itemize}
\item {Grp. gram.:m.}
\end{itemize}
\begin{itemize}
\item {Proveniência:(Do gr. \textunderscore a\textunderscore  priv. + \textunderscore ruthmos\textunderscore )}
\end{itemize}
Falta de rythmo.
Irregularidade nas pulsações.
\section{Arzanefe}
\begin{itemize}
\item {Grp. gram.:m.}
\end{itemize}
O mesmo que \textunderscore arzenefe\textunderscore .
\section{Arzegaia}
\begin{itemize}
\item {Grp. gram.:f.}
\end{itemize}
Espécie de azagaia, com um ferro em cada extremo, usada pelos estradiotas, que a arremessavam, conservando-a presa por uma corda.
(Cp. \textunderscore azagaia\textunderscore )
\section{Arzenefe}
\begin{itemize}
\item {Grp. gram.:m.}
\end{itemize}
Designação, que os alquimistas deram ao sulfureto amarelo de arsênico.
\section{Arzila}
\begin{itemize}
\item {Grp. gram.:f.}
\end{itemize}
Nome vulgar de uma espécie de arraia, (peixe).
\section{Arzola}
\begin{itemize}
\item {Grp. gram.:f.}
\end{itemize}
Planta annual, da fam. das compostas, (\textunderscore xanthium spinosum\textunderscore ).
\section{Ás}
(contr. da prep. \textunderscore a\textunderscore  e do art. \textunderscore as\textunderscore )
\section{Ás}
\begin{itemize}
\item {Grp. gram.:m.}
\end{itemize}
\begin{itemize}
\item {Proveniência:(Lat. \textunderscore as\textunderscore )}
\end{itemize}
O mesmo que \textunderscore asse\textunderscore .
Carta de jogar, que tem um só ponto marcado.
\section{As}
\begin{itemize}
\item {fónica:âs}
\end{itemize}
\begin{itemize}
\item {Grp. gram.:f. pl.}
\end{itemize}
(\textunderscore pron.\textunderscore  e \textunderscore art. def.\textunderscore )
\section{Asa}
\begin{itemize}
\item {Grp. gram.:f.}
\end{itemize}
\begin{itemize}
\item {Utilização:Constr.}
\end{itemize}
\begin{itemize}
\item {Utilização:Prov.}
\end{itemize}
\begin{itemize}
\item {Utilização:dur.}
\end{itemize}
\begin{itemize}
\item {Grp. gram.:Pl.}
\end{itemize}
\begin{itemize}
\item {Proveniência:(Do lat. \textunderscore ansa\textunderscore )}
\end{itemize}
Appêndice, recurvado em fórma de argola, de certos utensílios domésticos.
Membro, guarnecido de pennas, que serve ás aves para voarem.
Appêndice membranoso de alguns insectos e peixes.
Peça de metal, applicada nos guarda-ventos, em que fazem o effeito de puxador.
Ansa, occasião.
Ligeireza.
Partes lateraes, que guarnecem as ventas.
Pétalas lateraes, na flôr das papilionáceas.
\section{Asa-branca}
\begin{itemize}
\item {Grp. gram.:f.}
\end{itemize}
\begin{itemize}
\item {Utilização:Bras. do N}
\end{itemize}
Espécie de pomba brava.
\section{Asada}
\begin{itemize}
\item {Grp. gram.:f.}
\end{itemize}
\begin{itemize}
\item {Utilização:Prov.}
\end{itemize}
\begin{itemize}
\item {Utilização:alent.}
\end{itemize}
Vasilha com asas.
\section{Asa-de-corvo}
\begin{itemize}
\item {Grp. gram.:f.}
\end{itemize}
Variedade de trigo rijo.
\section{Asado}
\begin{itemize}
\item {Grp. gram.:m.}
\end{itemize}
\begin{itemize}
\item {Grp. gram.:Adj.}
\end{itemize}
Vaso com asa.
Que tem asas.
\section{Asado}
\begin{itemize}
\item {Grp. gram.:adj.}
\end{itemize}
\begin{itemize}
\item {Proveniência:(De \textunderscore aso\textunderscore )}
\end{itemize}
Jeitoso; cômmodo.
Bem feito: \textunderscore um cesto bem asado\textunderscore .
\section{Asador}
\begin{itemize}
\item {Grp. gram.:m.  e  adj.}
\end{itemize}
\begin{itemize}
\item {Utilização:Des.}
\end{itemize}
\begin{itemize}
\item {Proveniência:(De \textunderscore asar\textunderscore ^2)}
\end{itemize}
O que dá aso.
\section{Asagres}
\begin{itemize}
\item {Grp. gram.:m. pl.}
\end{itemize}
\begin{itemize}
\item {Utilização:Prov.}
\end{itemize}
\begin{itemize}
\item {Utilização:trasm.}
\end{itemize}
Uvas verdes.
(Metáth. de \textunderscore agrases\textunderscore , pl. de \textunderscore agrás\textunderscore )
\section{Asal-azêdo}
\begin{itemize}
\item {Grp. gram.:m.}
\end{itemize}
Casta de uva minhota.
\section{Asal-doce}
\begin{itemize}
\item {Grp. gram.:m.}
\end{itemize}
Casta de uva minhota.
\section{Asalveada}
\begin{itemize}
\item {fónica:sal}
\end{itemize}
\begin{itemize}
\item {Grp. gram.:adj. f.}
\end{itemize}
Diz-se da corolla, semelhante á da salva.
\section{Asa-negra}
\begin{itemize}
\item {Grp. gram.:f.}
\end{itemize}
\begin{itemize}
\item {Utilização:Pop.}
\end{itemize}
Pessôa, que constantemente embaraça ou prejudica outra.
\section{Asaprol}
\begin{itemize}
\item {Grp. gram.:m.}
\end{itemize}
\begin{itemize}
\item {Utilização:Chím.}
\end{itemize}
Naphtylsulfato de cálcio.
\section{Asar}
\begin{itemize}
\item {Grp. gram.:m.}
\end{itemize}
(v. \textunderscore azar\textunderscore ^1)
\section{Asar}
\begin{itemize}
\item {Grp. gram.:v. t.}
\end{itemize}
\begin{itemize}
\item {Grp. gram.:V. p.}
\end{itemize}
Dar aso ou ensejo a.
Vir a propósito.
Proporcionar-se.
Tornar-se jeitoso.
Accommodar-se.
\section{Asarca}
\begin{itemize}
\item {Grp. gram.:f.}
\end{itemize}
Gênero de orchídeas.
\section{Asareidas}
\begin{itemize}
\item {Grp. gram.:f. pl.}
\end{itemize}
\begin{itemize}
\item {Proveniência:(Do gr. \textunderscore asaron\textunderscore  + \textunderscore eidos\textunderscore )}
\end{itemize}
Família de plantas, ás quaes o ásaro serve de typo.
\section{Asarina}
\begin{itemize}
\item {Grp. gram.:f.}
\end{itemize}
Princípio acre, contido no ásaro.
\section{Ásaro}
\begin{itemize}
\item {Grp. gram.:m.}
\end{itemize}
\begin{itemize}
\item {Proveniência:(Lat. \textunderscore asarum\textunderscore )}
\end{itemize}
Planta vivaz e medicinal.
\section{Asarona}
\begin{itemize}
\item {Grp. gram.:f.}
\end{itemize}
Substância crystallizável da raiz sêca do ásaro.
\section{Ás-avessas}
\begin{itemize}
\item {Grp. gram.:loc. adv.}
\end{itemize}
\begin{itemize}
\item {Proveniência:(De \textunderscore ás\textunderscore ^1 + \textunderscore avesso\textunderscore )}
\end{itemize}
Do avêsso. Ao contrário; de modo inverso.
\section{Asbestino}
\begin{itemize}
\item {Grp. gram.:adj.}
\end{itemize}
Que diz respeito ao asbesto.
\section{Asbesto}
\begin{itemize}
\item {Grp. gram.:m.}
\end{itemize}
\begin{itemize}
\item {Proveniência:(Gr. \textunderscore asbestos\textunderscore )}
\end{itemize}
Substância mineral, composta principalmente de silicato de cal e magnésia.
\section{Asbolina}
\begin{itemize}
\item {Grp. gram.:f.}
\end{itemize}
\begin{itemize}
\item {Proveniência:(Do gr. \textunderscore asbole\textunderscore )}
\end{itemize}
Oleo, extrahído da fuligem das chaminés.
\section{Asca}
\begin{itemize}
\item {Grp. gram.:f.}
\end{itemize}
O mesmo que \textunderscore asco\textunderscore .
\section{Ascá}
\begin{itemize}
\item {Grp. gram.:m.}
\end{itemize}
\begin{itemize}
\item {Utilização:Prov.}
\end{itemize}
\begin{itemize}
\item {Utilização:minh.}
\end{itemize}
Cabo, para suspender as redes da pescada.
\section{Ascálafos}
\begin{itemize}
\item {Grp. gram.:m. pl.}
\end{itemize}
\begin{itemize}
\item {Proveniência:(Do gr. \textunderscore Askalaphos\textunderscore , n. p.)}
\end{itemize}
Insectos neurópteros.
\section{Ascálaphos}
\begin{itemize}
\item {Grp. gram.:m. pl.}
\end{itemize}
\begin{itemize}
\item {Proveniência:(Do gr. \textunderscore Askalaphos\textunderscore , n. p.)}
\end{itemize}
Insectos neurópteros.
\section{Ascarento}
\begin{itemize}
\item {Grp. gram.:adj.}
\end{itemize}
(V.asqueroso)
\section{Ascari}
\begin{itemize}
\item {Grp. gram.:m.}
\end{itemize}
\begin{itemize}
\item {Utilização:Ant.}
\end{itemize}
Espécie de pano. Cf. \textunderscore Cancion. da Vaticana\textunderscore .
\section{Ascaricida}
\begin{itemize}
\item {Grp. gram.:f.}
\end{itemize}
\begin{itemize}
\item {Proveniência:(Do gr. \textunderscore askaris\textunderscore  + lat. \textunderscore caedere\textunderscore )}
\end{itemize}
Planta vermífuga, da fam. das compostas.
\section{Ascáridas}
\begin{itemize}
\item {Grp. gram.:f. pl.}
\end{itemize}
O mesmo que \textunderscore ascárides\textunderscore .
\section{Ascárides}
\begin{itemize}
\item {Grp. gram.:m.}
\end{itemize}
\begin{itemize}
\item {Proveniência:(Do gr. \textunderscore askaris\textunderscore )}
\end{itemize}
Lombrigas, vermes intestinaes, de corpo arredondado, e aguçado nas duas extremidades.
\section{Ascaridíase}
\begin{itemize}
\item {Grp. gram.:f.}
\end{itemize}
\begin{itemize}
\item {Utilização:Med.}
\end{itemize}
Enterite verminosa, produzida por ascárides.
\section{Ascaule}
\begin{itemize}
\item {Grp. gram.:m.}
\end{itemize}
Nome, que entre os Gregos se dava á gaita de folles.
\section{Ás-cavaleiras}
\begin{itemize}
\item {Grp. gram.:loc. adv.}
\end{itemize}
O mesmo que \textunderscore ás-cavallinhas\textunderscore .
\section{Ás-cavalinhas}
\begin{itemize}
\item {Grp. gram.:loc. adv.}
\end{itemize}
O mesmo que \textunderscore ás-cavallitas\textunderscore .
\section{Ás-cavalitas}
\begin{itemize}
\item {Grp. gram.:loc. adv.}
\end{itemize}
\begin{itemize}
\item {Proveniência:(De \textunderscore cavallo\textunderscore )}
\end{itemize}
Aos hombros, sôbre o pescoço.
\section{Ás-cavalleiras}
\begin{itemize}
\item {Grp. gram.:loc. adv.}
\end{itemize}
O mesmo que \textunderscore ás-cavallinhas\textunderscore .
\section{Ás-cavallinhas}
\begin{itemize}
\item {Grp. gram.:loc. adv.}
\end{itemize}
O mesmo que \textunderscore ás-cavallitas\textunderscore .
\section{Ás-cavallitas}
\begin{itemize}
\item {Grp. gram.:loc. adv.}
\end{itemize}
\begin{itemize}
\item {Proveniência:(De \textunderscore cavallo\textunderscore )}
\end{itemize}
Aos hombros, sôbre o pescoço.
\section{Ás-cegas}
\begin{itemize}
\item {Grp. gram.:loc. adv.}
\end{itemize}
\begin{itemize}
\item {Proveniência:(De \textunderscore cego\textunderscore )}
\end{itemize}
Cegamente.
\section{Ascendência}
\begin{itemize}
\item {Grp. gram.:f.}
\end{itemize}
\begin{itemize}
\item {Proveniência:(De \textunderscore ascendente\textunderscore )}
\end{itemize}
Acção de elevar-se.
Superioridade.
Linha das gerações anteriores de um indivíduo ou de uma família.
Antepassados.
Raça.
\section{Ascendente}
\begin{itemize}
\item {Grp. gram.:adj.}
\end{itemize}
\begin{itemize}
\item {Grp. gram.:M.}
\end{itemize}
\begin{itemize}
\item {Proveniência:(Lat. \textunderscore ascendens\textunderscore )}
\end{itemize}
Que ascende; que se eleva.
Que aumenta; que cresce.
Em caminhos de ferro, diz-se de tudo que vem ou está do lado onde começa a linha férrea.
Antepassado.
Qualquer parente em linha recta ascendente.
\section{Ascender}
\begin{itemize}
\item {Grp. gram.:v. i.}
\end{itemize}
\begin{itemize}
\item {Proveniência:(Lat. \textunderscore ascendere\textunderscore )}
\end{itemize}
Subir: \textunderscore o sol ascende do Oriente\textunderscore .
Elevar-se: \textunderscore ascender ás culminâncias do poder\textunderscore .
\section{Ascendimento}
\begin{itemize}
\item {Grp. gram.:m.}
\end{itemize}
Acto de \textunderscore ascender\textunderscore .
\section{Ascensão}
\begin{itemize}
\item {Grp. gram.:f.}
\end{itemize}
\begin{itemize}
\item {Utilização:Astron.}
\end{itemize}
\begin{itemize}
\item {Proveniência:(Lat. \textunderscore ascensio\textunderscore )}
\end{itemize}
Acto de ascender, de subir.
Festa ecclesiástica, que commemora a subida de Christo ao Céu.
O dia dessa festa.
\textunderscore Ascensão recta\textunderscore  ou \textunderscore obliqua de um astro\textunderscore , o ponto do equador celeste, que se eleva com êsse astro, na esphera recta ou obliqua.
\section{Ascensional}
\begin{itemize}
\item {Grp. gram.:adj.}
\end{itemize}
\begin{itemize}
\item {Proveniência:(Do lat. \textunderscore ascensio\textunderscore )}
\end{itemize}
Relativo a ascensão.
Que obriga a subir.
Que se realiza, subindo.
\section{Ascensionário}
\begin{itemize}
\item {Grp. gram.:adj.}
\end{itemize}
Que ascende, que sobe.
(Cp. \textunderscore ascensional\textunderscore )
\section{Ascensionista}
\begin{itemize}
\item {Grp. gram.:m.  e  f.}
\end{itemize}
\begin{itemize}
\item {Utilização:Bras}
\end{itemize}
Pessôa, que faz ascensões em balão, ou de outra maneira, a pontos elevados.
\section{Ascenso}
\begin{itemize}
\item {Grp. gram.:m.}
\end{itemize}
(V.ascensão)
\section{Ascensor}
\begin{itemize}
\item {Grp. gram.:m.}
\end{itemize}
\begin{itemize}
\item {Proveniência:(Lat. \textunderscore ascensor\textunderscore )}
\end{itemize}
Aquelle que eleva.
Apparelho mecânico, que transporta, elevando, fardos ou pessôas.
\section{Ascese}
\begin{itemize}
\item {Grp. gram.:f.}
\end{itemize}
\begin{itemize}
\item {Proveniência:(Gr. \textunderscore askesis\textunderscore )}
\end{itemize}
Prática de devoção e meditação religiosa.
\section{Asceta}
\begin{itemize}
\item {Grp. gram.:m.  e  f.}
\end{itemize}
\begin{itemize}
\item {Proveniência:(Gr. \textunderscore asketes\textunderscore )}
\end{itemize}
Pessôa, que se dedica inteiramente a exercícios espirituaes, mortificando o corpo e insulando-se da sociedade.
\section{Ascetano}
\begin{itemize}
\item {Grp. gram.:m.}
\end{itemize}
\begin{itemize}
\item {Utilização:Ant.}
\end{itemize}
O mesmo que \textunderscore ascetério\textunderscore .
\section{Ascetério}
\begin{itemize}
\item {Grp. gram.:m.}
\end{itemize}
\begin{itemize}
\item {Proveniência:(Gr. \textunderscore asketeria\textunderscore )}
\end{itemize}
Lugar, onde vivem ascetas; mosteiro.
\section{Ascética}
\begin{itemize}
\item {Grp. gram.:f.}
\end{itemize}
\begin{itemize}
\item {Proveniência:(De \textunderscore ascético\textunderscore )}
\end{itemize}
Doutrina dos ascetas.
\section{Ascético}
\begin{itemize}
\item {Grp. gram.:adj.}
\end{itemize}
Relativo ao ascetismo ou aos ascetas.
Contemplativo, devoto.
\section{Ascetismo}
\begin{itemize}
\item {Grp. gram.:m.}
\end{itemize}
Ascese.
Systema moral dos ascetas.
\section{Ascidiado}
\begin{itemize}
\item {Grp. gram.:adj.}
\end{itemize}
\begin{itemize}
\item {Utilização:Bot.}
\end{itemize}
Diz-se das fôlhas, que terminam em appêndice oco e dilatado.
(Cp. \textunderscore ascídias\textunderscore )
\section{Ascídias}
\begin{itemize}
\item {Grp. gram.:f. pl.}
\end{itemize}
\begin{itemize}
\item {Proveniência:(Gr. \textunderscore askidion\textunderscore )}
\end{itemize}
Mollúsculos tuniciários, da ordem dos acéphalos.
\section{Ascidoblásteo}
\begin{itemize}
\item {Grp. gram.:adj.}
\end{itemize}
\begin{itemize}
\item {Utilização:Bot.}
\end{itemize}
\begin{itemize}
\item {Proveniência:(Do gr. \textunderscore askidion\textunderscore  + \textunderscore blassein\textunderscore )}
\end{itemize}
Cujo embryão não é dividido.
\section{Áscios}
\begin{itemize}
\item {Grp. gram.:m. pl.}
\end{itemize}
\begin{itemize}
\item {Proveniência:(Do gr. \textunderscore a\textunderscore  priv. + \textunderscore skia\textunderscore , sombra)}
\end{itemize}
Habitantes da zona tórrida, que, ao meio-dia, não projectam sombra, porque o sol lhes fica a prumo.
\section{Asciro}
\begin{itemize}
\item {Grp. gram.:m.}
\end{itemize}
\begin{itemize}
\item {Proveniência:(Gr. \textunderscore askuron\textunderscore )}
\end{itemize}
Arruda brava.
\section{Ascite}
\begin{itemize}
\item {Grp. gram.:f.}
\end{itemize}
\begin{itemize}
\item {Proveniência:(Gr. \textunderscore askites\textunderscore )}
\end{itemize}
Hydropisia, resultante da accumulação de serosidade no peritonéu.
\section{Ascítico}
\begin{itemize}
\item {Grp. gram.:adj.}
\end{itemize}
Relativo á ascite.
Que tem ascite.
\section{Asclépia}
\begin{itemize}
\item {Grp. gram.:f.}
\end{itemize}
Planta, o mesmo que \textunderscore flôr-de-cera\textunderscore .
\section{Asclepiáceas}
\begin{itemize}
\item {Grp. gram.:f. pl.}
\end{itemize}
(V.asclepiádeas)
\section{Asclepíadas}
\begin{itemize}
\item {Grp. gram.:f. pl.}
\end{itemize}
O mesmo que \textunderscore asclepiádeas\textunderscore .
\section{Asclepíade}
\begin{itemize}
\item {Grp. gram.:f.}
\end{itemize}
Planta trepadeira, que serve de typo ás asclepiádeas.
\section{Asclepiádeas}
\begin{itemize}
\item {Grp. gram.:f. pl.}
\end{itemize}
\begin{itemize}
\item {Proveniência:(Do lat. \textunderscore Asclepias\textunderscore , n. p.)}
\end{itemize}
Família de plantas dicotyledóneas, trepadeiras.
\section{Asclepiadeu}
\begin{itemize}
\item {Grp. gram.:adj.}
\end{itemize}
\begin{itemize}
\item {Proveniência:(De \textunderscore Asklepiades\textunderscore , n. p.)}
\end{itemize}
Diz-se do verso grego e latino, composto de um espondeu, dois choriambos e um jambo.
\section{Asclero}
\begin{itemize}
\item {Grp. gram.:m.}
\end{itemize}
\begin{itemize}
\item {Proveniência:(Do gr. \textunderscore a\textunderscore  priv. + \textunderscore skleros\textunderscore )}
\end{itemize}
Insecto coleóptero heterómero.
\section{Asco}
\begin{itemize}
\item {Grp. gram.:m.}
\end{itemize}
\begin{itemize}
\item {Proveniência:(Do gr. \textunderscore aiskhos\textunderscore ?)}
\end{itemize}
Aversão.
Tédio; enjôo.
\section{...asco}
\begin{itemize}
\item {Grp. gram.:suf.}
\end{itemize}
(de significação deminutiva umas vezes, e outras explicativa)
\section{Ascóforos}
\begin{itemize}
\item {Grp. gram.:m. pl.}
\end{itemize}
\begin{itemize}
\item {Proveniência:(Do gr. \textunderscore askos\textunderscore  + \textunderscore phoros\textunderscore )}
\end{itemize}
Família de cogumelos, que tem utrículos.
\section{Ascoitar}
\begin{itemize}
\item {Grp. gram.:v. t.}
\end{itemize}
\begin{itemize}
\item {Utilização:minh}
\end{itemize}
\begin{itemize}
\item {Utilização:Pop.}
\end{itemize}
O mesmo que \textunderscore escutar\textunderscore .
\section{Ascoma}
\begin{itemize}
\item {Grp. gram.:f.}
\end{itemize}
Pelle, que se põe nos remos, para se não gastarem muito, roçando na borda do barco.
\section{Ascomicetes}
\begin{itemize}
\item {Grp. gram.:m. pl.}
\end{itemize}
O mesmo que \textunderscore ascóforos\textunderscore .
\section{Ascomycetes}
\begin{itemize}
\item {Grp. gram.:m. pl.}
\end{itemize}
O mesmo que \textunderscore ascóforos\textunderscore .
\section{Ascóphoros}
\begin{itemize}
\item {Grp. gram.:m. pl.}
\end{itemize}
\begin{itemize}
\item {Proveniência:(Do gr. \textunderscore askos\textunderscore  + \textunderscore phoros\textunderscore )}
\end{itemize}
Família de cogumelos, que tem utrículos.
\section{Ascorosamente}
\begin{itemize}
\item {Grp. gram.:adv.}
\end{itemize}
(V.asquerosamente)
\section{Ascoroso}
\begin{itemize}
\item {Grp. gram.:adj.}
\end{itemize}
(V.asqueroso)
\section{Áscua}
\begin{itemize}
\item {Grp. gram.:f.}
\end{itemize}
\begin{itemize}
\item {Proveniência:(T. cast.)}
\end{itemize}
Brasa viva.
\section{Ascuma}
\begin{itemize}
\item {Grp. gram.:f.}
\end{itemize}
Pequena lança antiga, para arremêsso.
\section{Ascumada}
\begin{itemize}
\item {Grp. gram.:f.}
\end{itemize}
Golpe de ascuma. Cf. R. Barbosa, \textunderscore Réplica\textunderscore , 157.
\section{Ascuna}
\begin{itemize}
\item {Grp. gram.:f.}
\end{itemize}
O mesmo que \textunderscore ascunha\textunderscore .
\section{Ascunha}
\begin{itemize}
\item {Grp. gram.:f.}
\end{itemize}
\begin{itemize}
\item {Utilização:Ant.}
\end{itemize}
O mesmo que \textunderscore ascuma\textunderscore .
\section{Ascyro}
\begin{itemize}
\item {Grp. gram.:m.}
\end{itemize}
\begin{itemize}
\item {Proveniência:(Gr. \textunderscore askuron\textunderscore )}
\end{itemize}
Arruda brava.
\section{Ás-declaradas}
\begin{itemize}
\item {Grp. gram.:loc. adv.}
\end{itemize}
Publicamente, ás claras. Cf. Filinto, \textunderscore D. Man.\textunderscore , II, 174.
\section{Aselha}
\begin{itemize}
\item {fónica:zê}
\end{itemize}
\begin{itemize}
\item {Grp. gram.:f.}
\end{itemize}
Pequena asa.
\section{Aselho}
\begin{itemize}
\item {fónica:zê}
\end{itemize}
\begin{itemize}
\item {Grp. gram.:f.}
\end{itemize}
Gênero de crustáceas isópodes, o mesmo que \textunderscore asello\textunderscore .
\section{Asélidos}
\begin{itemize}
\item {Grp. gram.:m. pl.}
\end{itemize}
\begin{itemize}
\item {Proveniência:(De \textunderscore asello\textunderscore  + gr. \textunderscore eidos\textunderscore )}
\end{itemize}
Classe de crustáceos isópodes.
\section{Aséllidos}
\begin{itemize}
\item {Grp. gram.:m. pl.}
\end{itemize}
\begin{itemize}
\item {Proveniência:(De \textunderscore asello\textunderscore  + gr. \textunderscore eidos\textunderscore )}
\end{itemize}
Classe de crustáceos isópodes.
\section{Asello}
\begin{itemize}
\item {Grp. gram.:m.}
\end{itemize}
\begin{itemize}
\item {Proveniência:(Lat. \textunderscore asellus\textunderscore )}
\end{itemize}
Pequeno crustáceo isópode, de água doce.
\section{Asellos}
\begin{itemize}
\item {Grp. gram.:m. pl.}
\end{itemize}
Duas estrêllas do signo de Câncer.
(Pl. de \textunderscore asello\textunderscore )
\section{Aselo}
\begin{itemize}
\item {Grp. gram.:m.}
\end{itemize}
\begin{itemize}
\item {Proveniência:(Lat. \textunderscore asellus\textunderscore )}
\end{itemize}
Pequeno crustáceo isópode, de água doce.
\section{Aselos}
\begin{itemize}
\item {Grp. gram.:m. pl.}
\end{itemize}
Duas estrêllas do signo de Câncer.
(Pl. de \textunderscore asello\textunderscore )
\section{Asemia}
\begin{itemize}
\item {fónica:se}
\end{itemize}
\begin{itemize}
\item {Grp. gram.:f.}
\end{itemize}
\begin{itemize}
\item {Proveniência:(Do gr. \textunderscore a\textunderscore  priv. + \textunderscore semeion\textunderscore )}
\end{itemize}
Impossibilidade de utilizar os sinaes da linguagem falada ou mímica, quer para exprimir, quer para comprehender ideias.
\section{Asêmio}
\begin{itemize}
\item {fónica:sê}
\end{itemize}
\begin{itemize}
\item {Grp. gram.:m.  e  adj.}
\end{itemize}
O que padece de asemia.
\section{Asépalo}
\begin{itemize}
\item {fónica:sé}
\end{itemize}
\begin{itemize}
\item {Grp. gram.:adj.}
\end{itemize}
\begin{itemize}
\item {Utilização:Bot.}
\end{itemize}
Que não tem sépalas.
\section{Asepsia}
\begin{itemize}
\item {fónica:se}
\end{itemize}
\begin{itemize}
\item {Grp. gram.:f.}
\end{itemize}
\begin{itemize}
\item {Grp. gram.:f.}
\end{itemize}
\begin{itemize}
\item {Utilização:Med.}
\end{itemize}
\begin{itemize}
\item {Proveniência:(Do gr. \textunderscore a\textunderscore  priv. + \textunderscore sepein\textunderscore )}
\end{itemize}
O mesmo que \textunderscore antisepsia\textunderscore .
Méthodo de afastar da economia humana todos os germes, capazes de produzir infecção.
\section{Aséptico}
\begin{itemize}
\item {fónica:sé}
\end{itemize}
\begin{itemize}
\item {Grp. gram.:adj.}
\end{itemize}
Relativo a \textunderscore asepsia\textunderscore .
\section{Aséptulina}
\begin{itemize}
\item {fónica:sé}
\end{itemize}
\begin{itemize}
\item {Grp. gram.:f.}
\end{itemize}
Solução medicamentosa, que se emprega no tratamento da tuberculose.
\section{Asevia}
\begin{itemize}
\item {Grp. gram.:f.}
\end{itemize}
Espécie de peixe do mar.
\section{Asexo}
\begin{itemize}
\item {fónica:assé}
\end{itemize}
\begin{itemize}
\item {Grp. gram.:adj.}
\end{itemize}
\begin{itemize}
\item {Proveniência:(De \textunderscore a\textunderscore  priv. + \textunderscore sexo\textunderscore )}
\end{itemize}
Que não tem sexo.
\section{Asexuado}
\begin{itemize}
\item {fónica:sé}
\end{itemize}
\begin{itemize}
\item {Grp. gram.:adj.}
\end{itemize}
O mesmo que \textunderscore asexual\textunderscore .
\section{Asexual}
\begin{itemize}
\item {fónica:sé}
\end{itemize}
\begin{itemize}
\item {Grp. gram.:adj.}
\end{itemize}
(V.asexo)
\section{Asialia}
\begin{itemize}
\item {fónica:si}
\end{itemize}
\begin{itemize}
\item {Grp. gram.:f.}
\end{itemize}
\begin{itemize}
\item {Proveniência:(Do gr. \textunderscore a\textunderscore  priv. + \textunderscore sialon\textunderscore )}
\end{itemize}
Falta de secreção salivar.
\section{Asiano}
\begin{itemize}
\item {Grp. gram.:adj.}
\end{itemize}
\begin{itemize}
\item {Grp. gram.:adj.}
\end{itemize}
O mesmo que \textunderscore asiático\textunderscore . Cf. Pant. de Aveiro, \textunderscore Itiner.\textunderscore , 15, (2.^a ed.).
(V.asiático)
\section{Asiarca}
\begin{itemize}
\item {Grp. gram.:m.}
\end{itemize}
\begin{itemize}
\item {Proveniência:(Gr. \textunderscore asiarkhes\textunderscore )}
\end{itemize}
Grão-sacerdote e presidente dos espectáculos e combates na província romana da Ásia.
\section{Asiarcha}
\begin{itemize}
\item {fónica:ca}
\end{itemize}
\begin{itemize}
\item {Grp. gram.:m.}
\end{itemize}
\begin{itemize}
\item {Proveniência:(Gr. \textunderscore asiarkhes\textunderscore )}
\end{itemize}
Grão-sacerdote e presidente dos espectáculos e combates na província romana da Ásia.
\section{Asiática}
\begin{itemize}
\item {Grp. gram.:f.}
\end{itemize}
\begin{itemize}
\item {Proveniência:(De \textunderscore asiático\textunderscore )}
\end{itemize}
Espécie de anêmona.
\section{Asiaticamente}
\begin{itemize}
\item {Grp. gram.:adj.}
\end{itemize}
De módo asiático; á maneira dos Asiáticos.
Magnificentemente.
\section{Asiaticismo}
\begin{itemize}
\item {Grp. gram.:m.}
\end{itemize}
Vocábulo, procedente de língua asiatica.
\section{Asiático}
\begin{itemize}
\item {Grp. gram.:adj.}
\end{itemize}
\begin{itemize}
\item {Grp. gram.:M.}
\end{itemize}
\begin{itemize}
\item {Proveniência:(Lat. \textunderscore asiaticus\textunderscore )}
\end{itemize}
Relativo á Ásia.
Diffuso e pomposo, (falando-se do estilo).
Aquelle que é natural da Ásia.
\section{Asiatismo}
\begin{itemize}
\item {Grp. gram.:m.}
\end{itemize}
Estílo pomposo e diffuso.
(Cp. \textunderscore asiático\textunderscore )
\section{Asidas}
\begin{itemize}
\item {Grp. gram.:f. pl.}
\end{itemize}
\begin{itemize}
\item {Utilização:Prov.}
\end{itemize}
\begin{itemize}
\item {Utilização:minh.}
\end{itemize}
\begin{itemize}
\item {Proveniência:(De \textunderscore asir\textunderscore )}
\end{itemize}
Peguilhos, que há no fundo do rio Minho, e em que as redes se podem prender e rasgar.
\section{Asigmático}
\begin{itemize}
\item {fónica:si}
\end{itemize}
\begin{itemize}
\item {Grp. gram.:adj.}
\end{itemize}
\begin{itemize}
\item {Utilização:Gram.}
\end{itemize}
\begin{itemize}
\item {Proveniência:(De \textunderscore a\textunderscore  priv. e \textunderscore sigmático\textunderscore )}
\end{itemize}
Que perdeu o \textunderscore s\textunderscore ; que não tem \textunderscore s\textunderscore .
\section{Asilo}
\begin{itemize}
\item {Grp. gram.:m.}
\end{itemize}
\begin{itemize}
\item {Proveniência:(Gr. \textunderscore asilos\textunderscore )}
\end{itemize}
Insecto díptero.
Moscardo.
\section{Asimina}
\begin{itemize}
\item {Grp. gram.:f.}
\end{itemize}
Fruto das plantas anonáceas.
\section{Asimineiro}
\begin{itemize}
\item {Grp. gram.:m.}
\end{itemize}
\begin{itemize}
\item {Proveniência:(De \textunderscore asimina\textunderscore )}
\end{itemize}
Gênero de plantas anonáceas.
\section{Asinário}
\begin{itemize}
\item {Grp. gram.:adj.}
\end{itemize}
\begin{itemize}
\item {Proveniência:(Lat. \textunderscore asinarius\textunderscore )}
\end{itemize}
Relativo a asno; próprio de asno.
\section{Asini-auricular}
\begin{itemize}
\item {Grp. gram.:adj.}
\end{itemize}
Que tem orelhas, como as de burro. Cf. Herculano, \textunderscore Bobo\textunderscore , 30.
\section{Asinino}
\begin{itemize}
\item {Grp. gram.:adj.}
\end{itemize}
\begin{itemize}
\item {Proveniência:(Lat. \textunderscore asininus\textunderscore )}
\end{itemize}
Relativo a asno; próprio de asno.
Estúpido.
\section{Asir}
\begin{itemize}
\item {Grp. gram.:v. t.}
\end{itemize}
\begin{itemize}
\item {Utilização:P. us.}
\end{itemize}
\begin{itemize}
\item {Proveniência:(De \textunderscore asa\textunderscore ?)}
\end{itemize}
Agarrar, empunhar.
\section{Asma}
\begin{itemize}
\item {Grp. gram.:f.}
\end{itemize}
\begin{itemize}
\item {Proveniência:(Gr. \textunderscore asthma\textunderscore )}
\end{itemize}
Difficuldade de respirar, que se manifesta por accessos irregulares.
\section{Asmático}
\begin{itemize}
\item {Grp. gram.:adj.}
\end{itemize}
\begin{itemize}
\item {Grp. gram.:M.}
\end{itemize}
\begin{itemize}
\item {Proveniência:(Gr. \textunderscore asthmatikos\textunderscore )}
\end{itemize}
Relativo á asma.
Que tem asma.
Aquelle que padece asma.
\section{Asmento}
\begin{itemize}
\item {Grp. gram.:m.  e  adj.}
\end{itemize}
(V.asmático)
\section{Asmo}
\begin{itemize}
\item {Grp. gram.:adj.}
\end{itemize}
O mesmo que \textunderscore ázimo\textunderscore .
\section{Asna}
\begin{itemize}
\item {Grp. gram.:f.}
\end{itemize}
\begin{itemize}
\item {Utilização:Constr.}
\end{itemize}
\begin{itemize}
\item {Utilização:Heráld.}
\end{itemize}
\begin{itemize}
\item {Utilização:Prov.}
\end{itemize}
\begin{itemize}
\item {Utilização:trasm.}
\end{itemize}
\begin{itemize}
\item {Proveniência:(Lat. \textunderscore asina\textunderscore )}
\end{itemize}
Peça de madeira, que fórma um ângulo, em cuja ponta assenta a viga mestra, ou o pau de fileira.
Ângulo, formado por duas barras, que se afastam inferiormente.
Fêmea do asno, burra; mula.
\section{Asnada}
\begin{itemize}
\item {Grp. gram.:f.}
\end{itemize}
Manada de asnos.
Asneira. Cf. \textunderscore Eufrosina\textunderscore , 335.
\section{Asnal}
\begin{itemize}
\item {Grp. gram.:adj.}
\end{itemize}
\begin{itemize}
\item {Proveniência:(Lat. \textunderscore asinalis\textunderscore )}
\end{itemize}
Próprio de asno; bestial.
\section{Asnalmente}
\begin{itemize}
\item {Grp. gram.:adv.}
\end{itemize}
De modo \textunderscore asnal\textunderscore .
Asnaticamente.
\section{Asnamente}
\begin{itemize}
\item {Grp. gram.:adj.}
\end{itemize}
Á maneira de asno; tolamente.
\section{Asnamento}
\begin{itemize}
\item {Grp. gram.:m.}
\end{itemize}
\begin{itemize}
\item {Proveniência:(De um hypoth. \textunderscore asnar\textunderscore , de \textunderscore asna\textunderscore )}
\end{itemize}
Conjunto de asnas, nas construcções.
Vigamento do telhado.
\section{Asnaria}
\begin{itemize}
\item {Grp. gram.:f.}
\end{itemize}
\begin{itemize}
\item {Proveniência:(De \textunderscore asna\textunderscore )}
\end{itemize}
O mesmo que \textunderscore asnamento\textunderscore .
\section{Asnaria}
\begin{itemize}
\item {Grp. gram.:f.}
\end{itemize}
\begin{itemize}
\item {Proveniência:(De \textunderscore asno\textunderscore )}
\end{itemize}
O mesmo que \textunderscore asnada\textunderscore .
\section{Asnas}
\begin{itemize}
\item {Grp. gram.:f. pl.}
\end{itemize}
\begin{itemize}
\item {Utilização:Prov.}
\end{itemize}
Resguardo de madeira ou de pedra, para que as ribeiras não invadam os terrenos marginaes.
(Pl. de \textunderscore asna\textunderscore ?)
\section{Asnaticamente}
\begin{itemize}
\item {Grp. gram.:adv.}
\end{itemize}
De modo \textunderscore asnático\textunderscore .
\section{Asfaltado}
\begin{itemize}
\item {Grp. gram.:adj.}
\end{itemize}
Coberto de asfalto.
\section{Asfaltador}
\begin{itemize}
\item {Grp. gram.:m.}
\end{itemize}
Operário que asfalta.
\section{Asfaltar}
\begin{itemize}
\item {Grp. gram.:v. t.}
\end{itemize}
Cobrir com asfalto.
\section{Asfaltaria}
\begin{itemize}
\item {Grp. gram.:f.}
\end{itemize}
Fábrica de asfalto.
\section{Asfalto}
\begin{itemize}
\item {Grp. gram.:m.}
\end{itemize}
\begin{itemize}
\item {Proveniência:(Gr. \textunderscore asphaltos\textunderscore )}
\end{itemize}
Betume escuro, lustroso e friável, que se encontra especialmente no lago Asfaltite.
Mistura de diversos hydrocarbonetos, que fórma uma substância glutinosa, que endurece com o frio.
Lugar, revestido por essa mistura, accrecida de areia.
\section{Asfixia}
\begin{itemize}
\item {fónica:csi}
\end{itemize}
\begin{itemize}
\item {Grp. gram.:f.}
\end{itemize}
\begin{itemize}
\item {Proveniência:(Gr. \textunderscore asphuxia\textunderscore )}
\end{itemize}
Suppressão da respiração.
Estado de morte apparente ou imminente, por estrangulação, submersão na água, ou por immersão em atmosphera impregnada de gases impróprios para a vida.
\section{Asfixiante}
\begin{itemize}
\item {fónica:csi}
\end{itemize}
\begin{itemize}
\item {Grp. gram.:adj.}
\end{itemize}
Que asfixia.
\section{Asfixiar}
\begin{itemize}
\item {fónica:csi}
\end{itemize}
\begin{itemize}
\item {Grp. gram.:v. t.}
\end{itemize}
Causar asfixia a; suffocar.
\section{Asfíxico}
\begin{itemize}
\item {fónica:csi}
\end{itemize}
\begin{itemize}
\item {Grp. gram.:adj.}
\end{itemize}
Que produz asfixia.
Que tem o carácter da asfixia.
\section{Asfixioso}
\begin{itemize}
\item {fónica:csi}
\end{itemize}
\begin{itemize}
\item {Grp. gram.:adj.}
\end{itemize}
Que causa asfixia.
\section{Asfodelo}
\begin{itemize}
\item {Grp. gram.:m.}
\end{itemize}
\begin{itemize}
\item {Proveniência:(Gr. \textunderscore asphodelos\textunderscore )}
\end{itemize}
Planta liliácea, de raiz tuberiforme.
\section{Asnático}
\begin{itemize}
\item {Grp. gram.:adj.}
\end{itemize}
O mesmo que \textunderscore asnal\textunderscore .
\section{Asnear}
\begin{itemize}
\item {Grp. gram.:v. i.}
\end{itemize}
Dizer ou fazer asneiras.
\section{Asnega}
\begin{itemize}
\item {Grp. gram.:f.}
\end{itemize}
\begin{itemize}
\item {Utilização:Ant.}
\end{itemize}
Talvez o mesmo que \textunderscore questão\textunderscore  ou \textunderscore contenda\textunderscore . Cf. \textunderscore Anat. Joc.\textunderscore , I, 435.
\section{Asneira}
\begin{itemize}
\item {Grp. gram.:f.}
\end{itemize}
\begin{itemize}
\item {Proveniência:(De \textunderscore asno\textunderscore )}
\end{itemize}
Burrice.
Tolice; inépcia; disparate.
Acção ou palavra obscena.
\section{Asneirada}
\begin{itemize}
\item {Grp. gram.:f.}
\end{itemize}
\begin{itemize}
\item {Utilização:Prov.}
\end{itemize}
\begin{itemize}
\item {Utilização:alg.}
\end{itemize}
Grande asneira.
\section{Asneirão}
\begin{itemize}
\item {Grp. gram.:m.}
\end{itemize}
Grande asno.
Toleirão.
\section{Asneirista}
\begin{itemize}
\item {Grp. gram.:adj.}
\end{itemize}
Que se compraz em dizer asneiras.
Em que há muitas asneiras. Cf. Filinto, I, 143.
\section{Asneiro}
\begin{itemize}
\item {Grp. gram.:m.}
\end{itemize}
\begin{itemize}
\item {Grp. gram.:Adj.}
\end{itemize}
\begin{itemize}
\item {Proveniência:(De \textunderscore asno\textunderscore )}
\end{itemize}
Burriqueiro.
Aquelle que trata de asnos.
Asnal.
Diz-se da bêsta, que procede de cavallo e burra.
\section{Asneirola}
\begin{itemize}
\item {Grp. gram.:f.}
\end{itemize}
\begin{itemize}
\item {Proveniência:(De \textunderscore asneira\textunderscore )}
\end{itemize}
Expressão indecente; obscenidade.
\section{Asnice}
\begin{itemize}
\item {Grp. gram.:f.}
\end{itemize}
O mesmo que \textunderscore asnidade\textunderscore .
\section{Asnidade}
\begin{itemize}
\item {Grp. gram.:f.}
\end{itemize}
(V.asneira)
\section{Asnil}
\begin{itemize}
\item {Grp. gram.:adj.}
\end{itemize}
\begin{itemize}
\item {Utilização:Des.}
\end{itemize}
\begin{itemize}
\item {Grp. gram.:M.}
\end{itemize}
O mesmo que \textunderscore asnal\textunderscore .
Peixe das costas do Algarve, (\textunderscore pseudo-helotes Guntheri\textunderscore , Capello).
\section{Asnis}
\begin{itemize}
\item {Grp. gram.:m. pl.}
\end{itemize}
\begin{itemize}
\item {Utilização:Ant.}
\end{itemize}
\begin{itemize}
\item {Proveniência:(De \textunderscore asno\textunderscore )}
\end{itemize}
Arreios.
\section{Asno}
\begin{itemize}
\item {Grp. gram.:m.}
\end{itemize}
\begin{itemize}
\item {Utilização:Fig.}
\end{itemize}
\begin{itemize}
\item {Proveniência:(Lat. \textunderscore asinus\textunderscore )}
\end{itemize}
Burro, jumento.
Pessôa estúpida, imbecil, ignorante.
\section{Aso}
\begin{itemize}
\item {Grp. gram.:m.}
\end{itemize}
Ensejo; meio; occasião; pretexto.
Jeito.
Causa.
(Cp. \textunderscore asa\textunderscore )
\section{Asor}
\begin{itemize}
\item {Grp. gram.:m.}
\end{itemize}
Instrumento músico, usado entre Hebreus e que, suppõe-se, teria dez cordas.--A \textunderscore Vulgata\textunderscore  chama-lhe \textunderscore psaltério de dez cordas\textunderscore .
\section{Aspa}
\begin{itemize}
\item {Grp. gram.:f.}
\end{itemize}
\begin{itemize}
\item {Grp. gram.:Pl.}
\end{itemize}
\begin{itemize}
\item {Utilização:Gram.}
\end{itemize}
\begin{itemize}
\item {Proveniência:(Do ant. al. \textunderscore haspa\textunderscore ?)}
\end{itemize}
Antigo instrumento de supplício, em fórma de X ou de cruz de Santo-André.
Cruzamento de madeira, com aquella fórma, para construcções.
Cruz de pano, que se punha nos sambenitos.
Insígnia heráldica, em fórma de X.
Asas de moinho de vento.
Comas, traços curvos, que separam de um texto as citações ou palavras dignas de nota especial.
\section{Aspado}
\begin{itemize}
\item {Grp. gram.:adj.}
\end{itemize}
\begin{itemize}
\item {Utilização:Gram.}
\end{itemize}
Collocado entre aspas.
\section{Aspálatho}
\begin{itemize}
\item {Grp. gram.:m.}
\end{itemize}
\begin{itemize}
\item {Proveniência:(Lat. \textunderscore aspalathus\textunderscore )}
\end{itemize}
Pequena árvore espinhosa, de raiz medicinal.
\section{Aspálato}
\begin{itemize}
\item {Grp. gram.:m.}
\end{itemize}
\begin{itemize}
\item {Proveniência:(Lat. \textunderscore aspalathus\textunderscore )}
\end{itemize}
Pequena árvore espinhosa, de raiz medicinal.
\section{Aspar}
\begin{itemize}
\item {Grp. gram.:v. t.}
\end{itemize}
\begin{itemize}
\item {Utilização:Gram.}
\end{itemize}
Crucificar na aspa.
Maltratar.
Collocar entre aspas.
\section{Aspar}
\begin{itemize}
\item {Grp. gram.:v. t.}
\end{itemize}
Expungir, riscar, eliminar:«\textunderscore ...a Polónia fôsse aspada de entre os potentados...\textunderscore »Camilo, \textunderscore Sc. da Foz\textunderscore , 22.
(Relaciona-se com \textunderscore raspar\textunderscore ?)
\section{Asparagina}
\begin{itemize}
\item {Grp. gram.:f.}
\end{itemize}
\begin{itemize}
\item {Proveniência:(Do lat. \textunderscore asparagus\textunderscore )}
\end{itemize}
Substância neutra, extrahida do espargo.
Medicamento diurético.
\section{Asparagíneas}
\begin{itemize}
\item {Grp. gram.:f. pl.}
\end{itemize}
\begin{itemize}
\item {Proveniência:(Do lat. \textunderscore asparagus\textunderscore )}
\end{itemize}
Tribo de plantas liliáceas, que têm por typo o espargo commum.
\section{Aspárago}
\begin{itemize}
\item {Grp. gram.:m.}
\end{itemize}
(V.espargo)
\section{Aspartato}
\begin{itemize}
\item {Grp. gram.:m.}
\end{itemize}
\begin{itemize}
\item {Utilização:Chím.}
\end{itemize}
\begin{itemize}
\item {Proveniência:(De \textunderscore aspártico\textunderscore )}
\end{itemize}
Combinação do ácido aspártico com uma base.
\section{Aspártico}
\begin{itemize}
\item {Grp. gram.:adj.}
\end{itemize}
\begin{itemize}
\item {Utilização:Chím.}
\end{itemize}
\begin{itemize}
\item {Proveniência:(De \textunderscore aspárago\textunderscore )}
\end{itemize}
Diz-se de um ácido, produzido pela metamorphose da asparagina ou pela transformação de certos saes ammoniacaes.
\section{Aspásia}
\begin{itemize}
\item {Grp. gram.:f.}
\end{itemize}
\begin{itemize}
\item {Proveniência:(Do gr. \textunderscore aspasomai\textunderscore , abraçar)}
\end{itemize}
Gênero do orchídeas.
\section{Aspecto}
\begin{itemize}
\item {Grp. gram.:m.}
\end{itemize}
\begin{itemize}
\item {Proveniência:(Lat. \textunderscore aspectus\textunderscore )}
\end{itemize}
A parte exterior das coisas.
Aquillo que se vê.
Inspecção.
Apparência.
Semblante: \textunderscore homem de torvo aspecto\textunderscore .
\section{Aspegrênia}
\begin{itemize}
\item {Grp. gram.:f.}
\end{itemize}
Gênero de orchídeas.
Erva epíphyta do Peru.
\section{Aspeito}
\begin{itemize}
\item {Grp. gram.:m.}
\end{itemize}
O mesmo que \textunderscore aspecto\textunderscore :«\textunderscore viu de perto o aspeito doente e lívido do padre\textunderscore ». Camillo, \textunderscore Caveira\textunderscore , 259.
\section{Asperamente}
\begin{itemize}
\item {Grp. gram.:adv.}
\end{itemize}
De modo áspero, severo, rude: \textunderscore respondeu-lhe asperamente\textunderscore .
\section{Ás-perdidas}
\begin{itemize}
\item {Grp. gram.:loc. adv.}
\end{itemize}
O mesmo que \textunderscore perdidamente\textunderscore . Cf. Filinto, XIX, 220.
\section{Aspereza}
\begin{itemize}
\item {Grp. gram.:f.}
\end{itemize}
Qualidade do que é áspero, escabroso: \textunderscore a aspereza da encosta\textunderscore .
Amargor.
Rudeza; severidade: \textunderscore tratou-me com aspereza\textunderscore .
Dureza no estilo.
Desigualdade nos toques de um quadro.
Rispidez de sons.
\section{Asperger}
\begin{itemize}
\item {Grp. gram.:v. t.}
\end{itemize}
(V.aspergir)
\section{Asperges}
\begin{itemize}
\item {Grp. gram.:m.}
\end{itemize}
\begin{itemize}
\item {Proveniência:(Do lat. \textunderscore aspergere\textunderscore )}
\end{itemize}
Aspersão com água benta.
Momento, em que, nos offícios religiosos, se asperge água benta.
Antíphona, que se canta antes da Missa, durante a aspersão da água benta e que começa pelas palavras \textunderscore asperges me, Domine\textunderscore .
\textunderscore Capa de asperges\textunderscore , a capa que reveste o sacerdote, durante a aspersão da água benta.
\section{Aspergilário}
\begin{itemize}
\item {Grp. gram.:adj.}
\end{itemize}
O mesmo que \textunderscore aspergiliforme\textunderscore .
\section{Aspergiliforme}
\begin{itemize}
\item {Grp. gram.:adj.}
\end{itemize}
\begin{itemize}
\item {Proveniência:(Do b. lat. \textunderscore aspergillum\textunderscore  + lat. \textunderscore forma\textunderscore )}
\end{itemize}
Que tem fórma de hyssope.
\section{Aspergillário}
\begin{itemize}
\item {Grp. gram.:adj.}
\end{itemize}
O mesmo que \textunderscore aspergilliforme\textunderscore .
\section{Aspergilliforme}
\begin{itemize}
\item {Grp. gram.:adj.}
\end{itemize}
\begin{itemize}
\item {Proveniência:(Do b. lat. \textunderscore aspergillum\textunderscore  + lat. \textunderscore forma\textunderscore )}
\end{itemize}
Que tem fórma de hyssope.
\section{Aspergillo}
\begin{itemize}
\item {Grp. gram.:m.}
\end{itemize}
Órgão vegetal, semelhante a um hyssope.
Gênero de cogumelos.
Borrifador de água lustral, em certas festas religiosas de Roma.
(B. lat. \textunderscore aspergillum\textunderscore )
\section{Aspergillose}
\begin{itemize}
\item {Grp. gram.:f.}
\end{itemize}
Doença, produzida pelos cogumelos, chamados aspergillos.
\section{Aspergilo}
\begin{itemize}
\item {Grp. gram.:m.}
\end{itemize}
Órgão vegetal, semelhante a um hyssope.
Gênero de cogumelos.
Borrifador de água lustral, em certas festas religiosas de Roma.
(B. lat. \textunderscore aspergillum\textunderscore )
\section{Aspergilose}
\begin{itemize}
\item {Grp. gram.:f.}
\end{itemize}
Doença, produzida pelos cogumelos, chamados aspergillos.
\section{Aspergimento}
\begin{itemize}
\item {Grp. gram.:m.}
\end{itemize}
(V.aspersão)
\section{Aspergir}
\begin{itemize}
\item {Grp. gram.:v. t.}
\end{itemize}
\begin{itemize}
\item {Proveniência:(Lat. \textunderscore aspergere\textunderscore )}
\end{itemize}
Borrifar.
Orvalhar.
Espalhar líquido em fórma de chuva sôbre.
\section{Aspericorne}
\begin{itemize}
\item {Grp. gram.:adj.}
\end{itemize}
O mesmo que \textunderscore aspericórneo\textunderscore .
\section{Aspericórneo}
\begin{itemize}
\item {Grp. gram.:adj.}
\end{itemize}
\begin{itemize}
\item {Proveniência:(De \textunderscore áspero\textunderscore  + \textunderscore córneo\textunderscore )}
\end{itemize}
Diz-se das plantas, cujas antennas têm pêlos ásperos.
\section{Asperidade}
\begin{itemize}
\item {Grp. gram.:f.}
\end{itemize}
\begin{itemize}
\item {Proveniência:(Lat. \textunderscore asperitas\textunderscore )}
\end{itemize}
O mesmo que \textunderscore aspereza\textunderscore .
\section{Asperidão}
\begin{itemize}
\item {Grp. gram.:f.}
\end{itemize}
O mesmo que \textunderscore aspereza\textunderscore .
\section{Asperifólio}
\begin{itemize}
\item {Grp. gram.:adj.}
\end{itemize}
\begin{itemize}
\item {Proveniência:(Do lat. \textunderscore asper\textunderscore  + \textunderscore folium\textunderscore )}
\end{itemize}
Que tem fôlhas ásperas.
\section{Asperifoliáceas}
\begin{itemize}
\item {Grp. gram.:f. pl.}
\end{itemize}
O mesmo que \textunderscore borragíneas\textunderscore .
\section{Asperifólias}
\begin{itemize}
\item {Grp. gram.:f. pl.}
\end{itemize}
(Veja \textunderscore asperifoliáceas\textunderscore )
\section{Asperíssimo}
\begin{itemize}
\item {Grp. gram.:adj.}
\end{itemize}
O mesmo que \textunderscore aspérrimo\textunderscore . Cf. Sousa, \textunderscore Vida do Arceb.\textunderscore , I, 118.
\section{Aspermado}
\begin{itemize}
\item {Grp. gram.:adj.}
\end{itemize}
O mesmo que \textunderscore aspermo\textunderscore .
\section{Aspermatismo}
\begin{itemize}
\item {Grp. gram.:m.}
\end{itemize}
\begin{itemize}
\item {Proveniência:(De \textunderscore aspermo\textunderscore )}
\end{itemize}
Difficuldade ou impossibilidade de ejacular esperma.
\section{Aspermia}
\begin{itemize}
\item {Grp. gram.:f.}
\end{itemize}
\begin{itemize}
\item {Proveniência:(De \textunderscore aspermo\textunderscore )}
\end{itemize}
Estado de uma planta, que não dá sementes.
Esterilidade no homem.
\section{Aspermo}
\begin{itemize}
\item {Grp. gram.:adj.}
\end{itemize}
\begin{itemize}
\item {Proveniência:(Gr. \textunderscore aspermos\textunderscore )}
\end{itemize}
Que não produz grãos.
Que não tem esperma.
\section{Áspero}
\begin{itemize}
\item {Grp. gram.:adj.}
\end{itemize}
\begin{itemize}
\item {Utilização:Fig.}
\end{itemize}
\begin{itemize}
\item {Utilização:Pint.}
\end{itemize}
\begin{itemize}
\item {Proveniência:(Lat. \textunderscore asper\textunderscore )}
\end{itemize}
Que tem superfície desigual, incômmoda ao tacto: \textunderscore madeira áspera\textunderscore .
Rijo.
Escabroso, fragoso: \textunderscore ásperas penedias\textunderscore .
Azêdo, acre, desagradável ao paladar: \textunderscore vinho áspero\textunderscore .
Severo, duro: \textunderscore reprehensão áspera\textunderscore .
Desagradável ao ouvido: \textunderscore voz áspera\textunderscore .
Desharmónico, desagradável á vista.
\section{Asperococco}
\begin{itemize}
\item {Grp. gram.:m.}
\end{itemize}
Gênero de algas.
\section{Asperococo}
\begin{itemize}
\item {Grp. gram.:m.}
\end{itemize}
Gênero de algas.
\section{Aspérrimo}
\begin{itemize}
\item {Grp. gram.:adj.}
\end{itemize}
(sup. de \textunderscore áspero\textunderscore )
\section{Aspersão}
\begin{itemize}
\item {Grp. gram.:f.}
\end{itemize}
\begin{itemize}
\item {Proveniência:(Lat. \textunderscore aspersio\textunderscore )}
\end{itemize}
Acção de aspergir.
\section{Aspersar}
\begin{itemize}
\item {Grp. gram.:v. t.}
\end{itemize}
\begin{itemize}
\item {Utilização:Ant.}
\end{itemize}
O mesmo que \textunderscore aspergir\textunderscore . Cf. Cortesão, \textunderscore Subs\textunderscore .
\section{Aspersório}
\begin{itemize}
\item {Grp. gram.:m.}
\end{itemize}
\begin{itemize}
\item {Proveniência:(Do lat. \textunderscore aspersus\textunderscore )}
\end{itemize}
O mesmo que \textunderscore hyssope\textunderscore .
\section{Asperugo}
\begin{itemize}
\item {Grp. gram.:m.}
\end{itemize}
\begin{itemize}
\item {Proveniência:(Lat. \textunderscore asperugo\textunderscore )}
\end{itemize}
Gênero de plantas borragineas.
\section{Aspérula}
\begin{itemize}
\item {Grp. gram.:f.}
\end{itemize}
\begin{itemize}
\item {Proveniência:(Do lat. \textunderscore asper\textunderscore )}
\end{itemize}
Planta medicinal, da fam. das rubiáceas.
\section{Aspes}
\begin{itemize}
\item {Grp. gram.:m. pl.}
\end{itemize}
\begin{itemize}
\item {Utilização:Bras}
\end{itemize}
Raios da roda, no engenho de açúcar.
(Por \textunderscore aspas\textunderscore )
\section{Asphaltado}
\begin{itemize}
\item {Grp. gram.:adj.}
\end{itemize}
Coberto de asphalto.
\section{Asphaltador}
\begin{itemize}
\item {Grp. gram.:m.}
\end{itemize}
Operário que asphalta.
\section{Asphaltar}
\begin{itemize}
\item {Grp. gram.:v. t.}
\end{itemize}
Cobrir com asphalto.
\section{Asphaltaria}
\begin{itemize}
\item {Grp. gram.:f.}
\end{itemize}
Fábrica de asphalto.
\section{Asphalto}
\begin{itemize}
\item {Grp. gram.:m.}
\end{itemize}
\begin{itemize}
\item {Proveniência:(Gr. \textunderscore asphaltos\textunderscore )}
\end{itemize}
Betume escuro, lustroso e friável, que se encontra especialmente no lago Asphaltite.
Mistura de diversos hydrocarbonetos, que fórma uma substância glutinosa, que endurece com o frio.
Lugar, revestido por essa mistura, accrecida de areia.
\section{Asphodelo}
\begin{itemize}
\item {Grp. gram.:m.}
\end{itemize}
\begin{itemize}
\item {Proveniência:(Gr. \textunderscore asphodelos\textunderscore )}
\end{itemize}
Planta liliácea, de raiz tuberiforme.
\section{Asphyxia}
\begin{itemize}
\item {Grp. gram.:f.}
\end{itemize}
\begin{itemize}
\item {Proveniência:(Gr. \textunderscore asphuxia\textunderscore )}
\end{itemize}
Suppressão da respiração.
Estado de morte apparente ou imminente, por estrangulação, submersão na água, ou por immersão em atmosphera impregnada de gases impróprios para a vida.
\section{Asphyxiante}
\begin{itemize}
\item {Grp. gram.:adj.}
\end{itemize}
Que asphyxia.
\section{Asphyxiar}
\begin{itemize}
\item {Grp. gram.:v. t.}
\end{itemize}
Causar asphyxia a; suffocar.
\section{Asphýxico}
\begin{itemize}
\item {Grp. gram.:adj.}
\end{itemize}
Que produz asphyxia.
Que tem o carácter da asphyxia.
\section{Asphyxioso}
\begin{itemize}
\item {Grp. gram.:adj.}
\end{itemize}
Que causa asphyxia.
\section{Áspide}
\begin{itemize}
\item {Grp. gram.:f.  ou  m.}
\end{itemize}
\begin{itemize}
\item {Grp. gram.:f.  ou  m.}
\end{itemize}
\begin{itemize}
\item {Proveniência:(Gr. \textunderscore aspis\textunderscore )}
\end{itemize}
Pequena cobra, venenosa e semelhante á víbora.
Pessôa maledicente.
Antiga peça de artilharia.
\section{Aspídia}
\begin{itemize}
\item {Grp. gram.:f.}
\end{itemize}
\begin{itemize}
\item {Proveniência:(Gr. \textunderscore aspidion\textunderscore )}
\end{itemize}
Insecto lepidóptero nocturno.
Gênero de fêtos.
\section{Aspidiáceo}
\begin{itemize}
\item {Grp. gram.:adj.}
\end{itemize}
O mesmo que \textunderscore aspidiado\textunderscore .
\section{Aspidiado}
\begin{itemize}
\item {Grp. gram.:adj.}
\end{itemize}
Semelhante á aspídia.
\section{Aspídio}
\begin{itemize}
\item {Grp. gram.:m.}
\end{itemize}
(V.aspídia)
\section{Aspidiota}
\begin{itemize}
\item {Grp. gram.:m.}
\end{itemize}
\begin{itemize}
\item {Proveniência:(Do gr. \textunderscore aspidion\textunderscore )}
\end{itemize}
Gênero de crustáceos.
\section{Aspidisce}
\begin{itemize}
\item {Grp. gram.:f.}
\end{itemize}
\begin{itemize}
\item {Utilização:Ant.}
\end{itemize}
\begin{itemize}
\item {Proveniência:(Gr. \textunderscore aspidiske\textunderscore )}
\end{itemize}
Gênero de infusórios.
Esphincter.
\section{Aspidiscina}
\begin{itemize}
\item {Grp. gram.:f.}
\end{itemize}
\begin{itemize}
\item {Proveniência:(De \textunderscore aspidisce\textunderscore )}
\end{itemize}
Família de infusórios, que abrange só o gênero aspidisce.
\section{Aspidisco}
\begin{itemize}
\item {Grp. gram.:m.}
\end{itemize}
(V.aspidisce)
\section{Aspidistra}
\begin{itemize}
\item {Grp. gram.:f.}
\end{itemize}
Gênero de plantas ornamentaes, de fôlha larga.
\section{Aspidocéfalo}
\begin{itemize}
\item {Grp. gram.:adj.}
\end{itemize}
\begin{itemize}
\item {Proveniência:(Do gr. \textunderscore aspis\textunderscore  + \textunderscore kephale\textunderscore )}
\end{itemize}
Que tem a cabeça guarnecida de placas.
\section{Aspidocéphalo}
\begin{itemize}
\item {Grp. gram.:adj.}
\end{itemize}
\begin{itemize}
\item {Proveniência:(Do gr. \textunderscore aspis\textunderscore  + \textunderscore kephale\textunderscore )}
\end{itemize}
Que tem a cabeça guarnecida de placas.
\section{Aspidosperma}
\begin{itemize}
\item {Grp. gram.:m.}
\end{itemize}
\begin{itemize}
\item {Proveniência:(Do gr. \textunderscore aspidion\textunderscore  + \textunderscore sperma\textunderscore )}
\end{itemize}
Árvore do Brasil, de ramos chatos ou curvos e casca saborosa.
\section{Aspidospermina}
\begin{itemize}
\item {Grp. gram.:f.}
\end{itemize}
\begin{itemize}
\item {Proveniência:(De \textunderscore aspidosperma\textunderscore )}
\end{itemize}
Medicamento contra a dispneia.
\section{Aspília}
\begin{itemize}
\item {Grp. gram.:f.}
\end{itemize}
Gênero de plantas tropicaes.
\section{Aspilota}
\begin{itemize}
\item {Grp. gram.:f.}
\end{itemize}
Pedra preciosa, da côr da prata.
\section{Aspiração}
\begin{itemize}
\item {Grp. gram.:f.}
\end{itemize}
\begin{itemize}
\item {Utilização:Gram.}
\end{itemize}
\begin{itemize}
\item {Proveniência:(Lat. \textunderscore aspiratio\textunderscore )}
\end{itemize}
Acção de aspirar.
Desejo vehemente.
Pronunciação guttural.
\section{Aspirador}
\begin{itemize}
\item {Grp. gram.:m.}
\end{itemize}
\begin{itemize}
\item {Proveniência:(De \textunderscore aspirar\textunderscore )}
\end{itemize}
Apparelho, para aspirar a água de um reservatório.
Mecanismo, para estabelecer uma corrente de ar em espaço limitado.
\section{Aspirante}
\begin{itemize}
\item {Grp. gram.:adj.}
\end{itemize}
\begin{itemize}
\item {Grp. gram.:M.}
\end{itemize}
\begin{itemize}
\item {Proveniência:(De \textunderscore aspirar\textunderscore )}
\end{itemize}
Que aspira.
Que absorve: \textunderscore bomba aspirante\textunderscore .
Pequena graduação militar e burocrática: \textunderscore um aspirante de Fazenda\textunderscore .
\section{Aspirar}
\begin{itemize}
\item {Grp. gram.:v. t.}
\end{itemize}
\begin{itemize}
\item {Utilização:Gram.}
\end{itemize}
\begin{itemize}
\item {Grp. gram.:V. i.}
\end{itemize}
\begin{itemize}
\item {Proveniência:(Lat. \textunderscore aspirare\textunderscore )}
\end{itemize}
Attrahir (o ar) aos pulmões.
Absorver; chupar.
Pronunciar gutturalmente.
Têr desejo vehemente.
Têr pretensão: \textunderscore aspirar á mão da rapariga\textunderscore .
\section{Aspirativo}
\begin{itemize}
\item {Grp. gram.:adj.}
\end{itemize}
\begin{itemize}
\item {Proveniência:(De \textunderscore aspirar\textunderscore )}
\end{itemize}
Que deve pronunciar-se com aspiração.
\section{Aspirina}
\begin{itemize}
\item {Grp. gram.:f.}
\end{itemize}
Medicamento antipyrético e analgésico.
\section{Asplênia}
\begin{itemize}
\item {Grp. gram.:f.}
\end{itemize}
\begin{itemize}
\item {Proveniência:(Gr. \textunderscore asplenon\textunderscore )}
\end{itemize}
Gênero de fêtos.
\section{Aspleniáceas}
\begin{itemize}
\item {Grp. gram.:f. pl.}
\end{itemize}
\begin{itemize}
\item {Proveniência:(De \textunderscore asplênia\textunderscore )}
\end{itemize}
Gênero de fêtos, segundo Presl.
\section{Asplenioide}
\begin{itemize}
\item {Grp. gram.:adj.}
\end{itemize}
Semelhante á asplênia.
\section{Asplênio}
\begin{itemize}
\item {Grp. gram.:m.}
\end{itemize}
O mesmo que \textunderscore asplênia\textunderscore .
\section{Aspondílico}
\begin{itemize}
\item {Grp. gram.:adj.}
\end{itemize}
\begin{itemize}
\item {Proveniência:(De \textunderscore a\textunderscore  priv. e \textunderscore espondýlico\textunderscore )}
\end{itemize}
Que não tem natureza de vértebra.
\section{Aspondýlico}
\begin{itemize}
\item {Grp. gram.:adj.}
\end{itemize}
\begin{itemize}
\item {Proveniência:(De \textunderscore a\textunderscore  priv. e \textunderscore espondýlico\textunderscore )}
\end{itemize}
Que não tem natureza de vértebra.
\section{Áspora}
\begin{itemize}
\item {Grp. gram.:adj. f.}
\end{itemize}
\begin{itemize}
\item {Proveniência:(Do gr. \textunderscore a\textunderscore  priv. + \textunderscore sporos\textunderscore )}
\end{itemize}
Diz-se da planta que não tem corpúsculos reproductores.
\section{Aspre}
\begin{itemize}
\item {Grp. gram.:m.}
\end{itemize}
Antiga moéda de prata nos países muçulmanos.
\section{Aspredo}
\begin{itemize}
\item {fónica:prê}
\end{itemize}
\begin{itemize}
\item {Grp. gram.:m.}
\end{itemize}
Peixe de água dôce.
(Por \textunderscore asperedo\textunderscore , de \textunderscore áspero\textunderscore ?)
\section{Asquerosamente}
\begin{itemize}
\item {Grp. gram.:adv.}
\end{itemize}
De modo \textunderscore asqueroso\textunderscore .
\section{Asquerosidade}
\begin{itemize}
\item {Grp. gram.:f.}
\end{itemize}
Qualidade do que é \textunderscore asqueroso\textunderscore .
\section{Asqueroso}
\begin{itemize}
\item {Grp. gram.:adj.}
\end{itemize}
Que causa asco.
Sujo; repelente: \textunderscore baiuca asquerosa\textunderscore .
Tôrpe.
Infame: \textunderscore procedimento asqueroso\textunderscore .
(Cast. \textunderscore asqueroso\textunderscore )
\section{Assa}
\begin{itemize}
\item {Grp. gram.:f.}
\end{itemize}
Suco vegetal concreto.
\section{Assabão}
\begin{itemize}
\item {Grp. gram.:m.}
\end{itemize}
\begin{itemize}
\item {Utilização:Prov.}
\end{itemize}
O mesmo que \textunderscore sabão\textunderscore ^1.
\section{Assaborar}
\begin{itemize}
\item {Grp. gram.:v. t.}
\end{itemize}
Tornar saboroso, dar bom sabor a.
\section{Assaborear}
\textunderscore v. t.\textunderscore  (e der.)
O mesmo que \textunderscore saborear\textunderscore , etc.
\section{Assacadilha}
\begin{itemize}
\item {Grp. gram.:f.}
\end{itemize}
\begin{itemize}
\item {Proveniência:(De \textunderscore assacar\textunderscore )}
\end{itemize}
Imputação aleivosa.
\section{Assacador}
\begin{itemize}
\item {Grp. gram.:m.  e  adj.}
\end{itemize}
O que assaca.
\section{Assacar}
\begin{itemize}
\item {Grp. gram.:v. t.}
\end{itemize}
Imputar aleivosamente; inventar (calúmnia).
(Cp. \textunderscore sacar\textunderscore )
\section{Assacate}
\begin{itemize}
\item {Grp. gram.:m.}
\end{itemize}
Sebo, extrahido do mesentério das reses.
\section{Assacu}
\begin{itemize}
\item {Grp. gram.:m.}
\end{itemize}
(V.açacu)
\section{Assadeira}
\begin{itemize}
\item {Grp. gram.:f.}
\end{itemize}
Mulher, que assa castanhas.
Vaso de barro, em que se assam castanhas; assador.
Vaso, em que se assa carne ou peixe.
\section{Assadeiro}
\begin{itemize}
\item {Grp. gram.:m.}
\end{itemize}
\begin{itemize}
\item {Utilização:T. do Fundão}
\end{itemize}
\begin{itemize}
\item {Grp. gram.:Adj.}
\end{itemize}
O mesmo que \textunderscore assador\textunderscore .
Mulher velha, gorda e feia.
Próprio para assar.
\section{Assado}
\begin{itemize}
\item {Grp. gram.:adj.}
\end{itemize}
\begin{itemize}
\item {Utilização:Pop.}
\end{itemize}
\begin{itemize}
\item {Grp. gram.:M.}
\end{itemize}
\begin{itemize}
\item {Utilização:Fam.}
\end{itemize}
\begin{itemize}
\item {Proveniência:(De \textunderscore assar\textunderscore )}
\end{itemize}
Que tem inflammacão nas virilhas, pelo calor, pela nutrição ou pelo andar.
Que tem inflammação nos sovacos, na barbela ou em quaesquer refegos da pelle.
Peça de carne assada.
Lance, difficuldade: \textunderscore não te queiras vêr em taes assados\textunderscore . Cf. Castilho, \textunderscore Misanthropo\textunderscore , 162.
\section{Assador}
\begin{itemize}
\item {Grp. gram.:m.}
\end{itemize}
Aquelle que assa.
Vaso, utensílio, em que se assam castanhas.
\section{Assadura}
\begin{itemize}
\item {Grp. gram.:f.}
\end{itemize}
\begin{itemize}
\item {Utilização:Prov.}
\end{itemize}
\begin{itemize}
\item {Utilização:Prov.}
\end{itemize}
\begin{itemize}
\item {Utilização:trasm.}
\end{itemize}
Acção de assar.
Pedaço de carne assada, ou que póde assar-se de uma vez.
Pedaço de carne de porco, que se dá de presente por occasião da matança.
Lombo de porco.
(B. lat. \textunderscore assatura\textunderscore )
\section{Assa-fétida}
\begin{itemize}
\item {Grp. gram.:f.}
\end{itemize}
\begin{itemize}
\item {Proveniência:(De \textunderscore assa\textunderscore  + \textunderscore fétido\textunderscore )}
\end{itemize}
Planta umbellífera.
Goma resinosa, que se extrái dessa planta.
\section{Assafiado}
\begin{itemize}
\item {Grp. gram.:adj.}
\end{itemize}
\begin{itemize}
\item {Utilização:Prov.}
\end{itemize}
\begin{itemize}
\item {Utilização:trasm.}
\end{itemize}
\begin{itemize}
\item {Proveniência:(De \textunderscore safio\textunderscore ?)}
\end{itemize}
Opprimido com trabalho.
\section{Assafio}
\begin{itemize}
\item {Grp. gram.:m.}
\end{itemize}
\begin{itemize}
\item {Utilização:T. de Aveiro}
\end{itemize}
O mesmo que \textunderscore safio\textunderscore ^1.
\section{Assalariação}
\begin{itemize}
\item {Grp. gram.:f.}
\end{itemize}
Acto de \textunderscore assalariar\textunderscore . Cf. Arn. Gama, \textunderscore Motim\textunderscore , 266.
\section{Assalariado}
\begin{itemize}
\item {Grp. gram.:adj.}
\end{itemize}
Que trabalha por salário.
\section{Assalariador}
\begin{itemize}
\item {Grp. gram.:m.}
\end{itemize}
Aquelle que assalaria.
\section{Assalariamento}
\begin{itemize}
\item {Grp. gram.:m.}
\end{itemize}
Acção de \textunderscore assalariar\textunderscore .
\section{Assalariar}
\begin{itemize}
\item {Grp. gram.:v. t.}
\end{itemize}
Convencionar por salário os serviços de.
Têr ao serviço por salário.
Remunerar por serviço deshonroso.
\section{Assalmoado}
\begin{itemize}
\item {Grp. gram.:adj.}
\end{itemize}
O mesmo que \textunderscore assalmonado\textunderscore .
\section{Assalmonado}
\begin{itemize}
\item {Grp. gram.:adj.}
\end{itemize}
Semelhante ao salmão; que tem a côr do salmão.
\section{Açaloiado}
\begin{itemize}
\item {Grp. gram.:adj.}
\end{itemize}
Que tem modos de salôio.
Rude.
\section{Assaloiado}
\begin{itemize}
\item {Grp. gram.:adj.}
\end{itemize}
Que tem modos de salôio.
Rude.
\section{Assaltada}
\begin{itemize}
\item {Grp. gram.:f.}
\end{itemize}
Acto de \textunderscore assaltar\textunderscore .
\section{Assaltador}
\begin{itemize}
\item {Grp. gram.:m.}
\end{itemize}
Aquelle que assalta.
\section{Assaltar}
\begin{itemize}
\item {Grp. gram.:v. t.}
\end{itemize}
Atacar de súbito.
Investir com ímpeto: \textunderscore assaltar uma fortaleza\textunderscore .
Surprehender.
Acommeter á traição: \textunderscore assaltar um viandante\textunderscore .
Impressionar de repente; occorrer a: \textunderscore assaltou-me uma tentação\textunderscore .
(Cp. \textunderscore saltar\textunderscore )
\section{Assaltear}
\begin{itemize}
\item {Grp. gram.:v. t.}
\end{itemize}
O mesmo que \textunderscore assaltar\textunderscore .
\section{Assalto}
\begin{itemize}
\item {Grp. gram.:m.}
\end{itemize}
\begin{itemize}
\item {Proveniência:(De \textunderscore assaltar\textunderscore )}
\end{itemize}
O mesmo que \textunderscore assaltada\textunderscore .
Ataque; investida; acommetimento inesperado.
Instância.
Tentação.
Combate simulado, em esgrima.
Espécie de jôgo de tabuleiro.
\section{Assalveada}
\begin{itemize}
\item {Grp. gram.:adj. f.}
\end{itemize}
Diz-se da corolla, semelhante á da salva.
\section{Assambarcador}
\begin{itemize}
\item {Grp. gram.:m.}
\end{itemize}
Aquelle que assambarca.
\section{Assambarcamento}
\begin{itemize}
\item {Grp. gram.:m.}
\end{itemize}
Acto ou effeito de \textunderscore assambarcar\textunderscore .
\section{Assambarcar}
\begin{itemize}
\item {Grp. gram.:v. t.}
\end{itemize}
\begin{itemize}
\item {Proveniência:(De \textunderscore sambarca\textunderscore ? De \textunderscore a\textunderscore  + \textunderscore si\textunderscore  + \textunderscore abarcar\textunderscore ?)}
\end{itemize}
Chamar a si, privando outros da respectiva vantagem.
Monopolizar.
\section{Assami}
\begin{itemize}
\item {Grp. gram.:m.}
\end{itemize}
Língua, falada no Assão, ao Nordeste da Índia.
\section{Assane}
\begin{itemize}
\item {Grp. gram.:m.}
\end{itemize}
Árvore indiana, (\textunderscore briedelia spinosa\textunderscore ), de grandes dimensões, cuja madeira resistente é muito empregada em estacarias de poços e caes.
\section{Assanha}
\begin{itemize}
\item {Grp. gram.:f.}
\end{itemize}
\begin{itemize}
\item {Utilização:Prov.}
\end{itemize}
\begin{itemize}
\item {Utilização:dur.}
\end{itemize}
Acto de assanhar-se; irritação.
Altercação.
\section{Assanhadiço}
\begin{itemize}
\item {Grp. gram.:adj.}
\end{itemize}
Que facilmente se assanha.
Irascível.
\section{Assanhado}
\begin{itemize}
\item {Grp. gram.:adj.}
\end{itemize}
Que tem sanha; irritado.
\section{Assanhamento}
\begin{itemize}
\item {Grp. gram.:m.}
\end{itemize}
Acção de \textunderscore assanhar\textunderscore .
\section{Assanhar}
\begin{itemize}
\item {Grp. gram.:v. t.}
\end{itemize}
Encher de sanha; irritar; enfurecer.
Aggravar: \textunderscore assanhar uma ferida\textunderscore .
Avermelhar.
\section{Assanho}
\begin{itemize}
\item {Grp. gram.:m.}
\end{itemize}
(V.assanhamento)
\section{Assapado}
\begin{itemize}
\item {Grp. gram.:adj.}
\end{itemize}
\begin{itemize}
\item {Proveniência:(De \textunderscore assapar\textunderscore )}
\end{itemize}
O mesmo que \textunderscore acaçapado\textunderscore .
\section{Assapar}
\begin{itemize}
\item {Grp. gram.:v. i.}
\end{itemize}
\begin{itemize}
\item {Utilização:Prov.}
\end{itemize}
\begin{itemize}
\item {Proveniência:(De \textunderscore sapo\textunderscore ?)}
\end{itemize}
Alapar-se, esconder-se.
Cair, desmoronar-se.
\section{Assa-peixe}
\begin{itemize}
\item {Grp. gram.:f.}
\end{itemize}
Planta urticácea do Brasil.
\section{Assaquia}
\begin{itemize}
\item {Grp. gram.:f.}
\end{itemize}
\begin{itemize}
\item {Utilização:Prov.}
\end{itemize}
\begin{itemize}
\item {Utilização:alent.}
\end{itemize}
O mesmo que \textunderscore saguão\textunderscore .
\section{Assar}
\begin{itemize}
\item {Grp. gram.:v. t.}
\end{itemize}
\begin{itemize}
\item {Proveniência:(Lat. \textunderscore assare\textunderscore )}
\end{itemize}
Submeter á acção directa do fogo em sêco.
Queimar.
Tornar muito quente.
\section{Assaranzar-se}
\begin{itemize}
\item {Grp. gram.:v. p.}
\end{itemize}
\begin{itemize}
\item {Utilização:Bras}
\end{itemize}
O mesmo que \textunderscore zaranzar\textunderscore .
\section{Assarapantado}
\begin{itemize}
\item {Grp. gram.:adj.}
\end{itemize}
Espantado.
Confundido.
\section{Assarapantar}
\begin{itemize}
\item {Grp. gram.:v. t.}
\end{itemize}
\begin{itemize}
\item {Utilização:Pop.}
\end{itemize}
Espantar; assustar.
Atrapalhar.
\section{Assarapanto}
\begin{itemize}
\item {Grp. gram.:m.}
\end{itemize}
\begin{itemize}
\item {Proveniência:(De \textunderscore assarapantar\textunderscore )}
\end{itemize}
Grande espanto.
\section{Assarapolhado}
\begin{itemize}
\item {Grp. gram.:adj.}
\end{itemize}
\begin{itemize}
\item {Utilização:Pop.}
\end{itemize}
Atrapalhado, atarantado.
\section{Assarapolhar}
\begin{itemize}
\item {Grp. gram.:v. t.}
\end{itemize}
\begin{itemize}
\item {Utilização:Pop.}
\end{itemize}
Sobresaltar.
Atordoar, confundir.
\section{Assarasi}
\begin{itemize}
\item {Grp. gram.:m.}
\end{itemize}
Moéda de oiro indiana, do valor aproximado de 7$000 reis.
\section{Assaria}
\begin{itemize}
\item {Grp. gram.:f.}
\end{itemize}
Espécie de uva, o mesmo que \textunderscore assario\textunderscore .
\section{Assarilhado}
\begin{itemize}
\item {Grp. gram.:adj.}
\end{itemize}
Que tem fórma de sarilho.
\section{Assarina}
\begin{itemize}
\item {Grp. gram.:f.}
\end{itemize}
(V.anserina)
\section{Assario}
\begin{itemize}
\item {Grp. gram.:m.}
\end{itemize}
\begin{itemize}
\item {Proveniência:(Lat. \textunderscore assarius\textunderscore ?)}
\end{itemize}
O mesmo ou melhor que \textunderscore asserio\textunderscore .
\section{Assás}
\begin{itemize}
\item {Grp. gram.:adv.}
\end{itemize}
(V.assaz)
\section{Assassi}
\begin{itemize}
\item {Grp. gram.:m.}
\end{itemize}
Peixe cartilaginoso do Mar-Vermelho.
\section{Assassinado}
\begin{itemize}
\item {Grp. gram.:adj.}
\end{itemize}
\begin{itemize}
\item {Proveniência:(De \textunderscore assassinar\textunderscore )}
\end{itemize}
Morto por alguém.
\section{Assassinador}
\begin{itemize}
\item {Grp. gram.:m.}
\end{itemize}
(V.assassino)
\section{Assassinamento}
\begin{itemize}
\item {Grp. gram.:m.}
\end{itemize}
(V.assassínio)
\section{Assassinar}
\begin{itemize}
\item {Grp. gram.:v. t.}
\end{itemize}
\begin{itemize}
\item {Proveniência:(De \textunderscore assassino\textunderscore )}
\end{itemize}
Matar traiçoeiramente, com premeditação.
Matar (alguém).
\section{Assassinato}
\begin{itemize}
\item {Grp. gram.:m.}
\end{itemize}
(V.assassínio)
\section{Assassínio}
\begin{itemize}
\item {Grp. gram.:m.}
\end{itemize}
Acto de \textunderscore assassinar\textunderscore .
\section{Assassino}
\begin{itemize}
\item {Grp. gram.:m.}
\end{itemize}
\begin{itemize}
\item {Grp. gram.:Adj.}
\end{itemize}
Aquelle que mata traiçoeiramente ou com premeditação.
Indivíduo, que mata alguém.
Que assassina.
(B. lat. \textunderscore assassini\textunderscore )
\section{Assativo}
\begin{itemize}
\item {Grp. gram.:adv.}
\end{itemize}
Próprio para assar.
\section{Assaz}
\begin{itemize}
\item {Grp. gram.:adv.}
\end{itemize}
\begin{itemize}
\item {Proveniência:(Do lat. \textunderscore ad\textunderscore  + \textunderscore satien\textunderscore )}
\end{itemize}
Suficientemente.
Quanto é preciso.
\section{Assazoar}
\begin{itemize}
\item {Grp. gram.:v. t.}
\end{itemize}
O mesmo que \textunderscore assazonar\textunderscore .
\section{Assazonar}
\begin{itemize}
\item {Grp. gram.:v. t.}
\end{itemize}
(V.sazonar)
\section{Asse}
\begin{itemize}
\item {Grp. gram.:m.}
\end{itemize}
\begin{itemize}
\item {Proveniência:(Lat. \textunderscore as\textunderscore , \textunderscore assis\textunderscore )}
\end{itemize}
Antiga moéda romana de cobre.
\section{Asseadamente}
\begin{itemize}
\item {Grp. gram.:adv.}
\end{itemize}
De modo \textunderscore asseado\textunderscore .
Com asseio.
\section{Asseado}
\begin{itemize}
\item {Grp. gram.:adj.}
\end{itemize}
Que tem asseio.
Que traja com asseio.
Limpo.
\section{Asseamento}
\begin{itemize}
\item {Grp. gram.:m.}
\end{itemize}
Acção de \textunderscore assear\textunderscore .
\section{Assear}
\begin{itemize}
\item {Grp. gram.:v. t.}
\end{itemize}
\begin{itemize}
\item {Proveniência:(De \textunderscore asseio\textunderscore )}
\end{itemize}
Tornar limpo.
Enfeitar.
Vestir de bons fatos.
\section{Ássecla}
\begin{itemize}
\item {Grp. gram.:f.}
\end{itemize}
\begin{itemize}
\item {Proveniência:(Lat. \textunderscore assecla\textunderscore )}
\end{itemize}
Sectário; partidário.
\section{Assecuratório}
\begin{itemize}
\item {Grp. gram.:adj.}
\end{itemize}
\begin{itemize}
\item {Utilização:bras}
\end{itemize}
\begin{itemize}
\item {Utilização:Jur.}
\end{itemize}
Que assegura; que garante.
\section{Assedadeira}
\begin{itemize}
\item {Grp. gram.:f.}
\end{itemize}
Mulher que asseda linho.
\section{Assedado}
\begin{itemize}
\item {Grp. gram.:adj.}
\end{itemize}
Que se limpou nos sedeiros.
Macio como seda.
\section{Assedador}
\begin{itemize}
\item {Grp. gram.:m.}
\end{itemize}
Aquelle que asseda.
\section{Assedagem}
\begin{itemize}
\item {Grp. gram.:f.}
\end{itemize}
\begin{itemize}
\item {Proveniência:(De \textunderscore assedar\textunderscore )}
\end{itemize}
Operação, que tem por fim endireitar e apurar os filamentos do linho, separando delles as arestas e outras substâncias estranhas, na indústria doméstica de fiação e tecelagem.
\section{Assedar}
\begin{itemize}
\item {Grp. gram.:v. t.}
\end{itemize}
Limpar nos sedeiros: \textunderscore assedar linho\textunderscore .
Tornar macio como seda.
\section{Assedentado}
\begin{itemize}
\item {Grp. gram.:adj.}
\end{itemize}
\begin{itemize}
\item {Proveniência:(De \textunderscore sedento\textunderscore )}
\end{itemize}
Que tem sede.
\section{Assediador}
\begin{itemize}
\item {Grp. gram.:m.}
\end{itemize}
Aquelle que assedía.
\section{Assediante}
\begin{itemize}
\item {Grp. gram.:adj.}
\end{itemize}
Que assedía.
\section{Assediar}
\begin{itemize}
\item {Grp. gram.:v. t.}
\end{itemize}
\begin{itemize}
\item {Proveniência:(Lat. hyp. \textunderscore obsidiare\textunderscore )}
\end{itemize}
Pôr assédio a: \textunderscore assediar uma praça\textunderscore .
Importunar; molestar com pretensões insistentes: \textunderscore assediou-me com pedidos\textunderscore .
\section{Assedilhado}
\begin{itemize}
\item {Grp. gram.:adj.}
\end{itemize}
\begin{itemize}
\item {Proveniência:(De \textunderscore sede\textunderscore )}
\end{itemize}
O mesmo que \textunderscore sequioso\textunderscore .
\section{Assédio}
\begin{itemize}
\item {Grp. gram.:m.}
\end{itemize}
\begin{itemize}
\item {Utilização:Fig.}
\end{itemize}
Operações militares, em frente ou á volta de uma praça, para a tomar.
Cêrco; sitio.
Insistência impertinente junto de alguém.
(B. lat. \textunderscore assedium\textunderscore )
\section{Asseellar}
\begin{itemize}
\item {Grp. gram.:v. t.}
\end{itemize}
\begin{itemize}
\item {Utilização:Ant.}
\end{itemize}
O mesmo que \textunderscore sellar\textunderscore ^1.
\section{Asseetar}
\begin{itemize}
\item {Grp. gram.:v. t.}
\end{itemize}
\begin{itemize}
\item {Utilização:Ant.}
\end{itemize}
O mesmo que \textunderscore assetear\textunderscore .
\section{Asseguração}
\begin{itemize}
\item {Grp. gram.:f.}
\end{itemize}
Acto de \textunderscore assegurar\textunderscore .
\section{Asseguradamente}
\begin{itemize}
\item {Grp. gram.:adv.}
\end{itemize}
Com segurança.
\section{Assegurador}
\begin{itemize}
\item {Grp. gram.:m.}
\end{itemize}
Aquelle que assegura.
\section{Assegurar}
\begin{itemize}
\item {Grp. gram.:v. t.}
\end{itemize}
Certificar; affirmar com segurança: \textunderscore asseguro-lhe que o conheço\textunderscore .
(Cp. \textunderscore segurar\textunderscore )
\section{Asseio}
\begin{itemize}
\item {Grp. gram.:m.}
\end{itemize}
Limpeza.
Perfeição: \textunderscore escrever com asseio\textunderscore .
Esmero no vestir.
(Cast. \textunderscore aseo\textunderscore )
\section{Assejo}
\begin{itemize}
\item {Grp. gram.:m.}
\end{itemize}
\begin{itemize}
\item {Utilização:Ant.}
\end{itemize}
Ensejo.
\section{Asselar}
\begin{itemize}
\item {Grp. gram.:v. t.}
\end{itemize}
\begin{itemize}
\item {Utilização:Des.}
\end{itemize}
\begin{itemize}
\item {Proveniência:(De \textunderscore sêllo\textunderscore )}
\end{itemize}
Sellar.
Validar.
Confirmar.
\section{Assellar}
\begin{itemize}
\item {Grp. gram.:v. t.}
\end{itemize}
\begin{itemize}
\item {Utilização:Des.}
\end{itemize}
\begin{itemize}
\item {Proveniência:(De \textunderscore sêllo\textunderscore )}
\end{itemize}
Sellar.
Validar.
Confirmar.
\section{Asselvajado}
\begin{itemize}
\item {Grp. gram.:adj.}
\end{itemize}
Que tem modos de selvagem.
\section{Asselvajar}
\begin{itemize}
\item {Grp. gram.:v. t.}
\end{itemize}
Tornar selvagem, brutal, grosseiro.
\section{Assembléa}
\begin{itemize}
\item {Grp. gram.:f.}
\end{itemize}
O mesmo que \textunderscore assembleia\textunderscore .
\section{Assembleia}
\begin{itemize}
\item {Grp. gram.:f.}
\end{itemize}
\begin{itemize}
\item {Proveniência:(Fr. \textunderscore assemblée\textunderscore )}
\end{itemize}
Reunião de pessôas.
Sociedade; corporação.
Lugar ou casa, onde se reúne gente para se divertir.
Planta umbellífera, de flôres brancas ou avermelhadas.
\section{Assemelhação}
\begin{itemize}
\item {Grp. gram.:f.}
\end{itemize}
Acto de \textunderscore assemelhar\textunderscore .
\section{Assemelhar}
\begin{itemize}
\item {Grp. gram.:v. t.}
\end{itemize}
Tornar semelhante: \textunderscore aquella acção assemelha-o a um bandido\textunderscore .
Imitar.
Julgar semelhante.
Parecer semelhante a: \textunderscore o estilo dêle semelha o de Camillo\textunderscore .
(Cp. \textunderscore assimilar\textunderscore )
\section{Assemia}
\begin{itemize}
\item {Grp. gram.:f.}
\end{itemize}
\begin{itemize}
\item {Proveniência:(Do gr. \textunderscore a\textunderscore  priv. + \textunderscore semeion\textunderscore )}
\end{itemize}
Impossibilidade de utilizar os sinaes da linguagem falada ou mímica, quer para exprimir, quer para comprehender ideias.
\section{Assêmio}
\begin{itemize}
\item {Grp. gram.:m.  e  adj.}
\end{itemize}
O que padece de asemia.
\section{Assencar}
\begin{itemize}
\item {Grp. gram.:v. i.}
\end{itemize}
\begin{itemize}
\item {Utilização:Prov.}
\end{itemize}
\begin{itemize}
\item {Utilização:alent.}
\end{itemize}
Acertar, bater em cheio.
(Corr. de \textunderscore assentar\textunderscore ?)
\section{Assenhorado}
\begin{itemize}
\item {Grp. gram.:adj.}
\end{itemize}
Que tem modos de senhora; efeminado.
\section{Assenhorar-se}
\begin{itemize}
\item {Grp. gram.:v. p.}
\end{itemize}
\begin{itemize}
\item {Utilização:Prov.}
\end{itemize}
\begin{itemize}
\item {Utilização:trasm.}
\end{itemize}
O mesmo que \textunderscore assenhorear-se\textunderscore .
\section{Assenhorear-se}
\begin{itemize}
\item {Grp. gram.:v. p.}
\end{itemize}
Tornar-se senhor.
Tomar posse.
Entrar no domínio.
(Cp. \textunderscore senhorear\textunderscore )
\section{Assenso}
\begin{itemize}
\item {Grp. gram.:m.}
\end{itemize}
O mesmo que \textunderscore assentimento\textunderscore .
\section{Assentada}
\begin{itemize}
\item {Grp. gram.:f.}
\end{itemize}
\begin{itemize}
\item {Utilização:Prov.}
\end{itemize}
\begin{itemize}
\item {Utilização:Prov.}
\end{itemize}
\begin{itemize}
\item {Utilização:trasm.}
\end{itemize}
\begin{itemize}
\item {Proveniência:(De \textunderscore assentado\textunderscore )}
\end{itemize}
Secção forense, para depoimento de testemunhas.
Termo, que se lavra, do depoimento de testemunhas.
Acto de assentar-se.
Tempo, em que se está sentado: \textunderscore li todo o livro de uma assentada\textunderscore .
Porção de terreno plano.
Assembleia de eleitores.
\section{Assentadamente}
\begin{itemize}
\item {Grp. gram.:adv.}
\end{itemize}
\begin{itemize}
\item {Proveniência:(De \textunderscore assentar\textunderscore )}
\end{itemize}
Determinadamente.
Com prudência.
\section{Assentadeira}
\begin{itemize}
\item {Grp. gram.:f.}
\end{itemize}
\begin{itemize}
\item {Proveniência:(De \textunderscore assentar\textunderscore )}
\end{itemize}
Um dos maquinismos das fábricas de tecidos. Cf. \textunderscore Inquér. Industr.\textunderscore , II p., l. 3.^o, 208.
\section{Assentador}
\begin{itemize}
\item {Grp. gram.:m.}
\end{itemize}
\begin{itemize}
\item {Proveniência:(De \textunderscore assentar\textunderscore )}
\end{itemize}
Operário, que se emprega exclusivamente na conservação de via e obras, em caminhos de ferro.
\section{Assentadura}
\begin{itemize}
\item {Grp. gram.:f.}
\end{itemize}
\begin{itemize}
\item {Utilização:Veter.}
\end{itemize}
\begin{itemize}
\item {Proveniência:(De \textunderscore assentar\textunderscore )}
\end{itemize}
Compressão, produzida pela cava superior da ferradura na face plantar do casco dos solípedes.
\section{Assentamento}
\begin{itemize}
\item {Grp. gram.:m.}
\end{itemize}
Acção ou effeito de \textunderscore assentar\textunderscore .
\section{Assentar}
\begin{itemize}
\item {Grp. gram.:v. t.}
\end{itemize}
\begin{itemize}
\item {Grp. gram.:V. i.}
\end{itemize}
\begin{itemize}
\item {Grp. gram.:V. p.}
\end{itemize}
Fazer sentar.
Firmar.
Registra: \textunderscore assentar uma dívida\textunderscore .
Julgar; resolver.
Collocar.
Combinar: \textunderscore assentar um negócio\textunderscore .
Estabelecer: \textunderscore assentar residência\textunderscore .
Tornar firme: \textunderscore assentar o atêrro\textunderscore .
\textunderscore Assentar a mão\textunderscore , adquirir destreza com o exercicio.
Sossegar.
Tornar-se bem comportado: \textunderscore o rapaz assentou\textunderscore .
Tomar a devida situação, por effeito do trânsito, (\textunderscore falando-se de terreno\textunderscore ).
Tomar resolução: \textunderscore e assentámos nisto\textunderscore .
Sentar-se.
(Cp.\textunderscore sentar\textunderscore )
\section{Assente}
\begin{itemize}
\item {Grp. gram.:adj.}
\end{itemize}
\begin{itemize}
\item {Proveniência:(De \textunderscore assentar\textunderscore )}
\end{itemize}
Firme.
Resolvido: \textunderscore opinião assente\textunderscore .
\section{Assentimento}
\begin{itemize}
\item {Grp. gram.:m.}
\end{itemize}
Acto de \textunderscore assentir\textunderscore .
\section{Assentir}
\begin{itemize}
\item {Grp. gram.:v. i.}
\end{itemize}
\begin{itemize}
\item {Proveniência:(Lat. \textunderscore assentire\textunderscore )}
\end{itemize}
Consentir; concordar; annuir.
\section{Assentista}
\begin{itemize}
\item {Grp. gram.:m.}
\end{itemize}
\begin{itemize}
\item {Utilização:Ant.}
\end{itemize}
\begin{itemize}
\item {Proveniência:(De \textunderscore assentar\textunderscore )}
\end{itemize}
Fornecedor de mantimentos para as tropas, mediante quantia assentada.
\section{Assento}
\begin{itemize}
\item {Grp. gram.:m.}
\end{itemize}
\begin{itemize}
\item {Proveniência:(De \textunderscore assentar\textunderscore )}
\end{itemize}
Objecto em que a gente se senta; banco, cadeira, etc.
Base.
Nádegas.
Lugar, em que alguma coisa está assente.
Sítio.
Residencia.
Juizo; bom senso: \textunderscore é preciso que você tenha assento\textunderscore .
Descanso: \textunderscore tenho trabalhado sem assento\textunderscore .
Acôrdo.
Registo; termo de qualquer acto official: \textunderscore lavrar um assento.\textunderscore 
\section{Assépalo}
\begin{itemize}
\item {Grp. gram.:adj.}
\end{itemize}
\begin{itemize}
\item {Utilização:Bot.}
\end{itemize}
Que não tem sépalas.
\section{Assepsia}
\begin{itemize}
\item {Grp. gram.:f.}
\end{itemize}
\begin{itemize}
\item {Grp. gram.:f.}
\end{itemize}
\begin{itemize}
\item {Utilização:Med.}
\end{itemize}
\begin{itemize}
\item {Proveniência:(Do gr. \textunderscore a\textunderscore  priv. + \textunderscore sepein\textunderscore )}
\end{itemize}
O mesmo que \textunderscore antissepsia\textunderscore .
Método de afastar da economia humana todos os germes, capazes de produzir infecção.
\section{Asséptico}
\begin{itemize}
\item {Grp. gram.:adj.}
\end{itemize}
Relativo a \textunderscore assepsia\textunderscore .
\section{Asséptulina}
\begin{itemize}
\item {Grp. gram.:f.}
\end{itemize}
Solução medicamentosa, que se emprega no tratamento da tuberculose.
\section{Asserção}
\begin{itemize}
\item {Grp. gram.:f.}
\end{itemize}
\begin{itemize}
\item {Proveniência:(Lat. \textunderscore assertio\textunderscore )}
\end{itemize}
Affirmação; allegação.
\section{Asserenar}
\textunderscore v. t.\textunderscore  (e der.)
(V. \textunderscore serenar\textunderscore , etc.)
\section{Asserio}
\begin{itemize}
\item {Grp. gram.:m.}
\end{itemize}
\begin{itemize}
\item {Grp. gram.:Adj.}
\end{itemize}
\begin{itemize}
\item {Utilização:Prov.}
\end{itemize}
\begin{itemize}
\item {Utilização:alent.}
\end{itemize}
Uva branca da Extremadura, Alentejo e Algarve.
Diz-se de certos legumes, quando são de bôa qualidade.
(Cp. \textunderscore assario\textunderscore )
\section{Assertivo}
\begin{itemize}
\item {Grp. gram.:adj.}
\end{itemize}
O mesmo que \textunderscore affirmativo\textunderscore .
\section{Asserto}
\begin{itemize}
\item {Grp. gram.:m.}
\end{itemize}
\begin{itemize}
\item {Proveniência:(Lat. \textunderscore assertum\textunderscore )}
\end{itemize}
O mesmo que \textunderscore affirmação\textunderscore .
\section{Assertoar}
\begin{itemize}
\item {Grp. gram.:v. t.}
\end{itemize}
\begin{itemize}
\item {Proveniência:(De \textunderscore sertum\textunderscore ?)}
\end{itemize}
Talhar, dispor (coletes, casacos, etc.), de fórma que uma banda se sobreponha a outra.
\section{Assertor}
\begin{itemize}
\item {Grp. gram.:m.}
\end{itemize}
\begin{itemize}
\item {Utilização:Des.}
\end{itemize}
\begin{itemize}
\item {Proveniência:(Lat. \textunderscore assertor\textunderscore )}
\end{itemize}
Aquelle que assevera.
\section{Assertório}
\begin{itemize}
\item {Grp. gram.:adj.}
\end{itemize}
(V.assertivo)
\section{Assertuar}
\begin{itemize}
\item {Grp. gram.:v. t.}
\end{itemize}
O mesmo ou melhor que \textunderscore assertoar\textunderscore .
\section{Assessor}
\begin{itemize}
\item {Grp. gram.:m.}
\end{itemize}
\begin{itemize}
\item {Proveniência:(Lat. \textunderscore assessor\textunderscore )}
\end{itemize}
Adjunto; auxiliar.
Antigo funccionário, que acompanhava os embaixadores.
Antigo magistrado, que auxiliava os juízes leigos.
\section{Assessório}
\begin{itemize}
\item {Grp. gram.:adj.}
\end{itemize}
\begin{itemize}
\item {Proveniência:(Lat. \textunderscore assessorius\textunderscore )}
\end{itemize}
Relativo ao assessor.
\section{Assestar}
\begin{itemize}
\item {Grp. gram.:v. t.}
\end{itemize}
Apontar; dirigir contra alguma coisa: \textunderscore assestar o binóculo\textunderscore .
(B. lat. \textunderscore assistare\textunderscore )
\section{Assesto}
\begin{itemize}
\item {Grp. gram.:m.}
\end{itemize}
Acção de \textunderscore assestar\textunderscore .
\section{Assetar}
\textunderscore v. t.\textunderscore  (e der.)
O mesmo que \textunderscore assetear\textunderscore , etc. Cf. \textunderscore Eufrosina\textunderscore , 181; Usque, \textunderscore Tribulações\textunderscore , 17.
\section{Asseteador}
\begin{itemize}
\item {Grp. gram.:m.}
\end{itemize}
Aquelle que asseteia.
\section{Assetear}
\begin{itemize}
\item {Grp. gram.:v. t.}
\end{itemize}
\begin{itemize}
\item {Utilização:Fig.}
\end{itemize}
Matar ou ferir com seta.
Atacar; injuriar; molestar.
\section{Assetinação}
\begin{itemize}
\item {Grp. gram.:f.}
\end{itemize}
Acção de \textunderscore assetinar\textunderscore .
\section{Assetinado}
\begin{itemize}
\item {Grp. gram.:adj.}
\end{itemize}
\begin{itemize}
\item {Proveniência:(De \textunderscore assetinar\textunderscore )}
\end{itemize}
Macio e lustroso como setim: \textunderscore papel assetinado\textunderscore .
\section{Assetinador}
\begin{itemize}
\item {Grp. gram.:m.}
\end{itemize}
Aquelle que assetina.
\section{Assetinar}
\begin{itemize}
\item {Grp. gram.:v. t.}
\end{itemize}
Tornar macio e lustroso como setim.
Amaciar.
Calandrar.
\section{Asseveração}
\begin{itemize}
\item {Grp. gram.:f.}
\end{itemize}
\begin{itemize}
\item {Proveniência:(Lat. \textunderscore asseveratio\textunderscore )}
\end{itemize}
Acto de asseverar.
\section{Asseveradamente}
\begin{itemize}
\item {Grp. gram.:adv.}
\end{itemize}
Com asseveração.
\section{Asseverador}
\begin{itemize}
\item {Grp. gram.:m.}
\end{itemize}
Aquelle que assevera.
\section{Asseverante}
\begin{itemize}
\item {Grp. gram.:adj.}
\end{itemize}
Que assevera.
\section{Asseverar}
\begin{itemize}
\item {Grp. gram.:v. t.}
\end{itemize}
\begin{itemize}
\item {Proveniência:(Lat. \textunderscore asseverare\textunderscore )}
\end{itemize}
Affirmar.
Certificar; assegurar.
\section{Asseverativo}
\begin{itemize}
\item {Grp. gram.:adj.}
\end{itemize}
\begin{itemize}
\item {Proveniência:(De \textunderscore asseverar\textunderscore )}
\end{itemize}
Affirmativo.
\section{Assexo}
\begin{itemize}
\item {Grp. gram.:adj.}
\end{itemize}
\begin{itemize}
\item {Proveniência:(De \textunderscore a\textunderscore  priv. + \textunderscore sexo\textunderscore )}
\end{itemize}
Que não tem sexo.
\section{Assexuado}
\begin{itemize}
\item {Grp. gram.:adj.}
\end{itemize}
O mesmo que \textunderscore assexual\textunderscore .
\section{Assexual}
\begin{itemize}
\item {Grp. gram.:adj.}
\end{itemize}
(V.assexo)
\section{Assi}
\begin{itemize}
\item {Grp. gram.:adv.}
\end{itemize}
\begin{itemize}
\item {Utilização:Ant.}
\end{itemize}
O mesmo que \textunderscore assim\textunderscore .
\section{Assialia}
\begin{itemize}
\item {Grp. gram.:f.}
\end{itemize}
\begin{itemize}
\item {Proveniência:(Do gr. \textunderscore a\textunderscore  priv. + \textunderscore sialon\textunderscore )}
\end{itemize}
Falta de secreção salivar.
\section{Assibilação}
\begin{itemize}
\item {Grp. gram.:f.}
\end{itemize}
Acto de \textunderscore assibilar\textunderscore .
\section{Assibilar}
\begin{itemize}
\item {Grp. gram.:v.}
\end{itemize}
\begin{itemize}
\item {Utilização:t. Gram.}
\end{itemize}
Tornar sibilante.
\section{Assiduamente}
\begin{itemize}
\item {Grp. gram.:adv.}
\end{itemize}
Com assiduidade; de modo \textunderscore assíduo\textunderscore .
\section{Assiduidade}
\begin{itemize}
\item {fónica:du-i}
\end{itemize}
\begin{itemize}
\item {Grp. gram.:f.}
\end{itemize}
Qualidade do que é \textunderscore assíduo\textunderscore .
\section{Assíduo}
\begin{itemize}
\item {Grp. gram.:adj.}
\end{itemize}
Que apparece com frequência onde tem a obrigação de apparecer: \textunderscore funccionário assíduo\textunderscore .
Constante no trabalho.
Pontual.
Diligente.
Incessante: trabalho assíduo.
Frequente.
(Lat. \textunderscore assiduus\textunderscore ).
\section{Assigmático}
\begin{itemize}
\item {Grp. gram.:adj.}
\end{itemize}
\begin{itemize}
\item {Utilização:Gram.}
\end{itemize}
\begin{itemize}
\item {Proveniência:(De \textunderscore a\textunderscore  priv. e \textunderscore sigmático\textunderscore )}
\end{itemize}
Que perdeu o \textunderscore s\textunderscore ; que não tem \textunderscore s\textunderscore .
\section{Assignalar}
\textunderscore v. t.\textunderscore  (e der.)
(V. \textunderscore assinalar\textunderscore , etc.)
\section{Assignar}
\textunderscore v. t.\textunderscore  e \textunderscore i.\textunderscore  (e der.)
(V. \textunderscore assinar\textunderscore , etc.)
\section{Assim}
\begin{itemize}
\item {Grp. gram.:adj.}
\end{itemize}
\begin{itemize}
\item {Grp. gram.:Conj.}
\end{itemize}
\begin{itemize}
\item {Proveniência:(Do lat. \textunderscore ad\textunderscore  + \textunderscore sic\textunderscore ?)}
\end{itemize}
Dêste, dêsse, daquelle modo.
Do mesmo modo.
Logo: \textunderscore assim que o vi, levantei-me\textunderscore .
\section{Assim-assim}
\begin{itemize}
\item {Grp. gram.:loc. adv.}
\end{itemize}
Soffrivelmente; nem bem nem mal.
\section{Assim-como-assim}
\begin{itemize}
\item {Grp. gram.:loc. adv.}
\end{itemize}
Nesse caso; visto isso; já agora: \textunderscore «assim-como-assim, o melhor é diser tudo»\textunderscore . Camillo, \textunderscore Retr. de Ricard.\textunderscore , 15. Cf. \textunderscore Eufrosina\textunderscore , 124.
\section{Assimilabilidade}
\begin{itemize}
\item {Grp. gram.:f.}
\end{itemize}
Qualidade do que é \textunderscore assimilável\textunderscore .
\section{Assimilação}
\begin{itemize}
\item {Grp. gram.:f.}
\end{itemize}
\begin{itemize}
\item {Utilização:Gram.}
\end{itemize}
\begin{itemize}
\item {Proveniência:(Lat. \textunderscore assimilatio\textunderscore )}
\end{itemize}
Acto de assimilar.
Funcção orgânica, com que os seres vivos transformam os alimentos em substância própria.
Apropriação (de ideias ou de fórmas).
Identificação, por euphonia, de uma letra por outra, que a segue ou precede.
Transformação ou confusão de palavras que se assemelham.
\section{Assimilador}
\begin{itemize}
\item {Grp. gram.:adj.}
\end{itemize}
Que produz \textunderscore assimilação\textunderscore .
\section{Assimilar}
\begin{itemize}
\item {Grp. gram.:v. t.}
\end{itemize}
\begin{itemize}
\item {Proveniência:(Lat. \textunderscore assimilare\textunderscore )}
\end{itemize}
Produzir assimilação em.
Apropriar; compenetrar-se de: \textunderscore assimilar o estilo de outrem\textunderscore .
Tornar semelhante.
\section{Assimilativo}
\begin{itemize}
\item {Grp. gram.:adj.}
\end{itemize}
\begin{itemize}
\item {Proveniência:(De \textunderscore assimilar\textunderscore )}
\end{itemize}
Relativo á assimilação.
Assimilador.
\section{Assimilável}
\begin{itemize}
\item {Grp. gram.:adj.}
\end{itemize}
Que póde sêr assimilado.
\section{Assimilhar}
\textunderscore v. t.\textunderscore  (e der.)
(V. \textunderscore assemelhar\textunderscore , etc.)
\section{Assim-mesmo}
\begin{itemize}
\item {Grp. gram.:loc. adv.}
\end{itemize}
\begin{itemize}
\item {Utilização:Des.}
\end{itemize}
Também; igualmente.
\section{Assinação}
\begin{itemize}
\item {Grp. gram.:f.}
\end{itemize}
Acto de \textunderscore assinar\textunderscore .
Consignação de rendimentos.
\section{Assinado}
\begin{itemize}
\item {Grp. gram.:m.}
\end{itemize}
\begin{itemize}
\item {Grp. gram.:Adj.}
\end{itemize}
\begin{itemize}
\item {Proveniência:(De \textunderscore assinar\textunderscore )}
\end{itemize}
Documento, que tem assinatura.
Em que há assinatura.
\section{Assinaladamente}
\begin{itemize}
\item {Grp. gram.:adv.}
\end{itemize}
De modo \textunderscore assinalado\textunderscore .
\section{Assinalado}
\begin{itemize}
\item {Grp. gram.:adj.}
\end{itemize}
\begin{itemize}
\item {Proveniência:(De \textunderscore assinalar\textunderscore )}
\end{itemize}
Que se assinalou.
Notável, célebre: \textunderscore varões assinalados\textunderscore .
\section{Assinalador}
\begin{itemize}
\item {Grp. gram.:m.}
\end{itemize}
Aquelle que assinala.
\section{Assinalamento}
\begin{itemize}
\item {Grp. gram.:m.}
\end{itemize}
Acto de \textunderscore assinalar\textunderscore .
\section{Assinalar}
\begin{itemize}
\item {Grp. gram.:v. t.}
\end{itemize}
Pôr sinal em.
Tomar nota de.
Dar sinal de.
Distinguir.
Especificar.
\section{Assinalável}
\begin{itemize}
\item {Grp. gram.:adj.}
\end{itemize}
Que se póde \textunderscore assinalar\textunderscore .
\section{Assinamento}
\begin{itemize}
\item {Grp. gram.:m.}
\end{itemize}
(V.assinação)
\section{Assinante}
\begin{itemize}
\item {Grp. gram.:m.}
\end{itemize}
Aquelle que assina.
Subscritor: \textunderscore os assinantes de um jornal\textunderscore .
\section{Assinar}
\begin{itemize}
\item {Grp. gram.:v. t.}
\end{itemize}
\begin{itemize}
\item {Grp. gram.:V. i.}
\end{itemize}
\begin{itemize}
\item {Proveniência:(Lat. \textunderscore assignare\textunderscore )}
\end{itemize}
Firmar com o seu nome ou sinal.
Demarcar.
Distinguir; assinalar.
Indicar.
Destinar.
Aprazar.
Ajustar.
Subscrever o nome em favor de (uma publicação): \textunderscore assinar um jornal\textunderscore .
Escrever o próprio nome.
Tornar-se subscritor ou destinatário de uma publicação: \textunderscore assinar para um jornal\textunderscore .
\section{Assinatura}
\begin{itemize}
\item {Grp. gram.:f.}
\end{itemize}
Firma.
Nome que se escreveu.
Direito adquirido pela assinatura de um contrato.
Preço dêsse direito: \textunderscore a assinatura são 2$000 reis\textunderscore .
Acção de \textunderscore assinar\textunderscore .
\section{Assinável}
\begin{itemize}
\item {Grp. gram.:adj.}
\end{itemize}
Que se póde \textunderscore assinar\textunderscore .
\section{Assingelar}
\begin{itemize}
\item {Grp. gram.:v. t.}
\end{itemize}
Tornar singelo, simplificar. Cf. Filinto, XI, 279.
\section{Assipôndio}
\begin{itemize}
\item {Grp. gram.:m.}
\end{itemize}
\begin{itemize}
\item {Proveniência:(Lat. \textunderscore assipondium\textunderscore )}
\end{itemize}
Moéda romana, de cobre.
\section{Assiriano}
\begin{itemize}
\item {Grp. gram.:adj.}
\end{itemize}
O mesmo que assírico.
\section{Assírico}
\begin{itemize}
\item {Grp. gram.:adj.}
\end{itemize}
Relativo á Assíria.
\section{Assirio}
\begin{itemize}
\item {Grp. gram.:m.}
\end{itemize}
Uva branca de Borba, provavelmente a mesma que \textunderscore asserío\textunderscore .
\section{Assírio}
\begin{itemize}
\item {Grp. gram.:m.}
\end{itemize}
\begin{itemize}
\item {Grp. gram.:Adj.}
\end{itemize}
Aquelle que é natural da Assíria.
Dialecto da Assíria.
Relativo á Assíria.
\section{Assiriologia}
\begin{itemize}
\item {Grp. gram.:f.}
\end{itemize}
Estudo da Archeologia e Philologia da Assíria.
\section{Assiriológico}
\begin{itemize}
\item {Grp. gram.:adj.}
\end{itemize}
Relativo á assiriologia.
\section{Assiriólogo}
\begin{itemize}
\item {Grp. gram.:m.}
\end{itemize}
\begin{itemize}
\item {Proveniência:(De \textunderscore Assýria\textunderscore , n. p. + gr. \textunderscore logos\textunderscore )}
\end{itemize}
Aquelle que se dedica á assiriologia.
\section{Assisadeira}
\begin{itemize}
\item {Grp. gram.:f.}
\end{itemize}
\begin{itemize}
\item {Utilização:Prov.}
\end{itemize}
\begin{itemize}
\item {Utilização:trasm.}
\end{itemize}
\begin{itemize}
\item {Proveniência:(De \textunderscore assisado\textunderscore )}
\end{itemize}
Mulher, que tem má lingua, que murmura de tudo.
\section{Assisado}
\begin{itemize}
\item {Grp. gram.:adj.}
\end{itemize}
Que tem siso; sensato; prudente.
\section{Assísio}
\begin{itemize}
\item {Grp. gram.:m.}
\end{itemize}
\begin{itemize}
\item {Utilização:Ant.}
\end{itemize}
\begin{itemize}
\item {Proveniência:(Do lat. \textunderscore ad\textunderscore  + \textunderscore sessus\textunderscore )}
\end{itemize}
Coadjutor de cónegos; beneficiado.
\section{Assistência}
\begin{itemize}
\item {Grp. gram.:f.}
\end{itemize}
\begin{itemize}
\item {Grp. gram.:f.}
\end{itemize}
\begin{itemize}
\item {Proveniência:(De \textunderscore assistente\textunderscore )}
\end{itemize}
Acto de assistir.
Presença.
Auxílio; amparo.
Assiduïdade.
Moradia.
Conjunto de assistentes: \textunderscore a assistência applaudíu\textunderscore .
\section{Assistente}
\begin{itemize}
\item {Grp. gram.:adj.}
\end{itemize}
\begin{itemize}
\item {Grp. gram.:M.}
\end{itemize}
\begin{itemize}
\item {Grp. gram.:F.}
\end{itemize}
\begin{itemize}
\item {Utilização:Bras}
\end{itemize}
Que assiste.
Diz-se do médico que trata de um doente, acompanhando a doença.
Morador.
Aquelle que está presente.
Adjunto; assessor.
O mesmo que \textunderscore parteira\textunderscore .
\section{Assistida}
\begin{itemize}
\item {Grp. gram.:adj. f.}
\end{itemize}
\begin{itemize}
\item {Utilização:Bras}
\end{itemize}
Diz-se da mulher que está no período do catamênio.
\section{Assistir}
\begin{itemize}
\item {Grp. gram.:v. i.}
\end{itemize}
\begin{itemize}
\item {Grp. gram.:V. t.}
\end{itemize}
\begin{itemize}
\item {Proveniência:(Lat. \textunderscore assistere\textunderscore )}
\end{itemize}
Habitar.
Estar presente.
Fazer companhia.
Prestar soccorro.
Acompanhar.
Patrocinar.
\section{Assitiar}
\begin{itemize}
\item {Grp. gram.:v. t.}
\end{itemize}
O mesmo que \textunderscore sitiar\textunderscore . Cf. Filinto, XX, 208.
\section{Asso}
\begin{itemize}
\item {Grp. gram.:adj.}
\end{itemize}
\begin{itemize}
\item {Utilização:Bras}
\end{itemize}
O mesmo que \textunderscore albino\textunderscore .
\section{Assoalhado}
\begin{itemize}
\item {Grp. gram.:adj.}
\end{itemize}
Que tem soalho.
\section{Assoalhador}
\begin{itemize}
\item {Grp. gram.:m.}
\end{itemize}
Aquelle que assoalha.
\section{Assoalhadura}
\begin{itemize}
\item {Grp. gram.:f.}
\end{itemize}
O mesmo que \textunderscore assoalhamento\textunderscore .
\section{Assoalhamento}
\begin{itemize}
\item {Grp. gram.:m.}
\end{itemize}
Acto ou effeito de \textunderscore assoalhar\textunderscore .
\section{Assoalhar}
\begin{itemize}
\item {Grp. gram.:v. t.}
\end{itemize}
\begin{itemize}
\item {Proveniência:(De \textunderscore som\textunderscore )}
\end{itemize}
Divulgar: \textunderscore assoalhar um boato\textunderscore .
Ostentar: \textunderscore assoalhar riquezas\textunderscore .
\section{Assoalhar}
\begin{itemize}
\item {Grp. gram.:v. t.}
\end{itemize}
\begin{itemize}
\item {Proveniência:(De \textunderscore soalho\textunderscore )}
\end{itemize}
Fazer soalho em; solhar: \textunderscore assoalhar uma casa\textunderscore .
\section{Assoalhar}
\begin{itemize}
\item {Grp. gram.:v. t.}
\end{itemize}
Expor ao sol: \textunderscore assoalhar a roupa\textunderscore .
\section{Assoalho}
\begin{itemize}
\item {Grp. gram.:m.}
\end{itemize}
O mesmo que \textunderscore soalho\textunderscore ^1.
\section{Assoante}
\begin{itemize}
\item {Grp. gram.:adj.}
\end{itemize}
\begin{itemize}
\item {Proveniência:(Lat. \textunderscore assonans\textunderscore )}
\end{itemize}
Que tem assonância.
\section{Assoar}
\begin{itemize}
\item {Grp. gram.:v. t.}
\end{itemize}
Limpar de mucosidades (o nariz); esmoncar.
(Cp. \textunderscore soar\textunderscore ^1)
\section{Assoberbado}
\begin{itemize}
\item {Grp. gram.:adj.}
\end{itemize}
Que tem modos de pessôa soberba.
\section{Assoberbar}
\begin{itemize}
\item {Grp. gram.:v. t.}
\end{itemize}
\begin{itemize}
\item {Grp. gram.:v. p.}
\end{itemize}
Tratar com soberba; humilhar.
Dominar.
Estar superior a.
Tornar-se soberbo.
\section{Assobiada}
\begin{itemize}
\item {Grp. gram.:f.}
\end{itemize}
\begin{itemize}
\item {Proveniência:(De \textunderscore assobiar\textunderscore )}
\end{itemize}
Assuada, apupada, com assobios.
\section{Assobiadeira}
\begin{itemize}
\item {Grp. gram.:f.}
\end{itemize}
\begin{itemize}
\item {Proveniência:(De \textunderscore assobiar\textunderscore )}
\end{itemize}
Ave de arribação.
O mesmo que \textunderscore piadeira\textunderscore .
\section{Assobiado}
\begin{itemize}
\item {Grp. gram.:adj.}
\end{itemize}
\begin{itemize}
\item {Utilização:Prov.}
\end{itemize}
\begin{itemize}
\item {Utilização:trasm.}
\end{itemize}
\begin{itemize}
\item {Proveniência:(De \textunderscore assobio\textunderscore )}
\end{itemize}
Descòrado pelo frio; entanguido.
\section{Assobiador}
\begin{itemize}
\item {Grp. gram.:m.}
\end{itemize}
Aquelle que assobia.
\section{Assobiante}
\begin{itemize}
\item {Grp. gram.:adj.}
\end{itemize}
Que assobia. Cf. Filinto, II, 10.
\section{Assobiar}
\begin{itemize}
\item {Grp. gram.:V. i.}
\end{itemize}
\begin{itemize}
\item {Grp. gram.:V. t.}
\end{itemize}
\begin{itemize}
\item {Proveniência:(Do lat. \textunderscore ad\textunderscore  + \textunderscore sibilare\textunderscore )}
\end{itemize}
Soltar assobios.
Sibilar: \textunderscore o vento assobia\textunderscore .
Imitar, assobiando: \textunderscore assobiar a«Marselhesa»\textunderscore .
Escarnecer, dirigir vaias a.
\section{Assobio}
\begin{itemize}
\item {Grp. gram.:m.}
\end{itemize}
\begin{itemize}
\item {Proveniência:(De \textunderscore assobiar\textunderscore )}
\end{itemize}
Pequeno instrumento, com que se assobia; apito.
Silvo.
Som agudo, expellido dos lábios.
\section{Assobradar}
\begin{itemize}
\item {Grp. gram.:v. t.}
\end{itemize}
\begin{itemize}
\item {Proveniência:(De \textunderscore sobrado\textunderscore )}
\end{itemize}
O mesmo que \textunderscore solhar\textunderscore ^1.
\section{Associação}
\begin{itemize}
\item {Grp. gram.:f.}
\end{itemize}
Acto de associar-se.
Reunião de pessôas para um fim commum.
Sociedade.
Connexão.
Agrupamento psychológico (de ideias).
\section{Associadamente}
\begin{itemize}
\item {Grp. gram.:adv.}
\end{itemize}
Em sociedade.
\section{Associalização}
\begin{itemize}
\item {Grp. gram.:f.}
\end{itemize}
Acto ou effeito de \textunderscore assocializar\textunderscore .
\section{Assocializar}
\begin{itemize}
\item {Grp. gram.:v. t.}
\end{itemize}
Tornar social.
Tornar fraternal, irmanar.
\section{Associar}
\begin{itemize}
\item {Grp. gram.:v. t.}
\end{itemize}
\begin{itemize}
\item {Proveniência:(Lat. \textunderscore associare\textunderscore )}
\end{itemize}
Reunir em sociedade.
Aggregar.
Tomar como sócio.
Alliar.
\section{Assoguilhar}
\begin{itemize}
\item {Grp. gram.:v. t.}
\end{itemize}
\begin{itemize}
\item {Utilização:Ant.}
\end{itemize}
O mesmo que \textunderscore guarnecer\textunderscore .
\section{Assolação}
\begin{itemize}
\item {Grp. gram.:f.}
\end{itemize}
Acção de \textunderscore assolar\textunderscore .
\section{Assolador}
\begin{itemize}
\item {Grp. gram.:m.  e  adj.}
\end{itemize}
O que assola.
\section{Assolamento}
\begin{itemize}
\item {Grp. gram.:m.}
\end{itemize}
(V.assolação)
\section{Assolapar}
\begin{itemize}
\item {Grp. gram.:v. t.}
\end{itemize}
\begin{itemize}
\item {Utilização:Fig.}
\end{itemize}
\begin{itemize}
\item {Proveniência:(De \textunderscore solapa\textunderscore )}
\end{itemize}
Formar lapa em.
Escavar.
Minar.
Arruinar.
Disfarçar; occultar.
* \textunderscore v. t.\textunderscore  (e der.)
O mesmo que \textunderscore solapar\textunderscore , etc.
\section{Assolar}
\begin{itemize}
\item {Grp. gram.:v. t.}
\end{itemize}
\begin{itemize}
\item {Grp. gram.:V. p.}
\end{itemize}
\begin{itemize}
\item {Utilização:Prov.}
\end{itemize}
\begin{itemize}
\item {Utilização:alent.}
\end{itemize}
\begin{itemize}
\item {Proveniência:(Lat. \textunderscore assolare\textunderscore )}
\end{itemize}
Arrasar.
Destruir.
Reduzir a pó: \textunderscore assolar uma cidade\textunderscore .
Devastar; talar: \textunderscore assolar os campos\textunderscore .
Acachapar-se, coser-se com o solo ou com a terra.
\section{Assoldadadamente}
\begin{itemize}
\item {Grp. gram.:adv.}
\end{itemize}
Com sôldo, com salário.
\section{Assoldadar}
\begin{itemize}
\item {Grp. gram.:v. t.}
\end{itemize}
\begin{itemize}
\item {Proveniência:(De \textunderscore soldada\textunderscore )}
\end{itemize}
Assalariar.
Ajustar para serviço, por soldada.
Alistar no serviço militar, a sôldo.
\section{Assolear}
\begin{itemize}
\item {Grp. gram.:v. i.}
\end{itemize}
\begin{itemize}
\item {Utilização:Bras}
\end{itemize}
\begin{itemize}
\item {Proveniência:(De \textunderscore sol\textunderscore )}
\end{itemize}
Fatigar-se, por têr andado ao sol.
\section{Assoleimar}
\begin{itemize}
\item {Grp. gram.:v. t.}
\end{itemize}
\begin{itemize}
\item {Utilização:Prov.}
\end{itemize}
\begin{itemize}
\item {Proveniência:(De \textunderscore sol\textunderscore )}
\end{itemize}
Estiolar; queimar.
\section{Assolhar}
\textunderscore v. t.\textunderscore  (e der.)
O mesmo que \textunderscore solhar\textunderscore ^1, etc.
\section{Assòlhar}
\begin{itemize}
\item {Grp. gram.:v. t.}
\end{itemize}
\begin{itemize}
\item {Utilização:Prov.}
\end{itemize}
Assoalhar, divulgar.
\section{Assolto}
\begin{itemize}
\item {Grp. gram.:adj.}
\end{itemize}
O mesmo que \textunderscore absolto\textunderscore . Cf. Sousa, \textunderscore Vida do Arceb.\textunderscore , II, 301.
\section{Assolver}
\textunderscore v. t.\textunderscore  (e der.)
Fórma antiga de absolver, etc. Cf. G. Vicente, \textunderscore Inês Pereira\textunderscore .
\section{Assomada}
\begin{itemize}
\item {Grp. gram.:f.}
\end{itemize}
\begin{itemize}
\item {Proveniência:(De \textunderscore assomar\textunderscore )}
\end{itemize}
Acto de \textunderscore assomar\textunderscore .
Cumeada; vértice do monte.
\section{Assomadamente}
\begin{itemize}
\item {Grp. gram.:adv.}
\end{itemize}
Com assomo, com ira.
\section{Assomado}
\begin{itemize}
\item {Grp. gram.:adj.}
\end{itemize}
\begin{itemize}
\item {Utilização:Bras}
\end{itemize}
Irritado, colérico.
Espantadiço; que se assusta facilmente.
\section{Assomar}
\begin{itemize}
\item {Grp. gram.:v. i.}
\end{itemize}
\begin{itemize}
\item {Grp. gram.:V. t.}
\end{itemize}
\begin{itemize}
\item {Grp. gram.:V. p.}
\end{itemize}
\begin{itemize}
\item {Proveniência:(Do lat. \textunderscore ad\textunderscore  + \textunderscore summum\textunderscore )}
\end{itemize}
Subir á cumeada.
Apparecer em ponto alto: \textunderscore assomava a Lua\textunderscore .
Começar a mostrar-se.
Chegar: \textunderscore assomou á janela\textunderscore .
Causar irritação a.
Irritar-se.
Animar-se com o vinho.
\section{Assombração}
\begin{itemize}
\item {Grp. gram.:f.}
\end{itemize}
\begin{itemize}
\item {Utilização:Bras}
\end{itemize}
\begin{itemize}
\item {Utilização:Bras. de San-Paulo}
\end{itemize}
\begin{itemize}
\item {Proveniência:(De \textunderscore assombrar\textunderscore )}
\end{itemize}
Susto, causado pelo encontro ou apparição de coisas sobrenaturaes; terror, procedente de causa inexplicável.
Apparição fantástica, que produz terror.
\section{Assombradiço}
\begin{itemize}
\item {Grp. gram.:adj.}
\end{itemize}
Que se assombra facilmente.
\section{Assombramento}
\begin{itemize}
\item {Grp. gram.:m.}
\end{itemize}
Acto de \textunderscore assombrar\textunderscore .
\section{Assombrar}
\begin{itemize}
\item {Grp. gram.:v. t.}
\end{itemize}
\begin{itemize}
\item {Grp. gram.:V. i.}
\end{itemize}
\begin{itemize}
\item {Proveniência:(De \textunderscore sombra\textunderscore )}
\end{itemize}
Dar sombra a; tornar sombrio.
Causar assombro a.
Atemorizar.
Abalar com commoção eléctrica.
Produzir admiração, espanto.
\section{Assombreamento}
\begin{itemize}
\item {Grp. gram.:m.}
\end{itemize}
Acto de \textunderscore assombrear\textunderscore .
\section{Assombrear}
\begin{itemize}
\item {Grp. gram.:v. t.}
\end{itemize}
\begin{itemize}
\item {Grp. gram.:v. t.}
\end{itemize}
\begin{itemize}
\item {Grp. gram.:V. i.}
\end{itemize}
O mesmo que \textunderscore sombrear\textunderscore .
Dar sombra a.
Manchar.
Desgostar.
Dar sombreado a uma tela, a um desenho, etc.
\section{Assombro}
\begin{itemize}
\item {Grp. gram.:m.}
\end{itemize}
\begin{itemize}
\item {Proveniência:(De \textunderscore assombrar\textunderscore )}
\end{itemize}
Admiração excessiva; espanto.
Terror.
Aquillo que produz espanto, terror, admiração.
Aquillo que é raro, prodigioso.
\section{Assombrosamente}
\begin{itemize}
\item {Grp. gram.:adv.}
\end{itemize}
De modo \textunderscore assombroso\textunderscore .
\section{Assombroso}
\begin{itemize}
\item {Grp. gram.:adj.}
\end{itemize}
Que causa assombro.
\section{Assomo}
\begin{itemize}
\item {Grp. gram.:m.}
\end{itemize}
Acto de \textunderscore assomar\textunderscore .
Indício; presumpção.
Irritação, agastamento. Cf. Filinto, XIII, 86.
\section{Assonância}
\begin{itemize}
\item {Grp. gram.:f.}
\end{itemize}
\begin{itemize}
\item {Proveniência:(De \textunderscore assonante\textunderscore )}
\end{itemize}
Conformidade ou aproximação euphónica entre as vogaes tónicas de duas palavras.
\section{Assonante}
\begin{itemize}
\item {Grp. gram.:adj.}
\end{itemize}
(V.assoante)
\section{Assonar-se}
\begin{itemize}
\item {Grp. gram.:v. p.}
\end{itemize}
\begin{itemize}
\item {Utilização:Des.}
\end{itemize}
Reunir-se para fazer assuada.
(Cp. lat. \textunderscore sonare\textunderscore )
\section{Assonjo}
\begin{itemize}
\item {Grp. gram.:m.}
\end{itemize}
\begin{itemize}
\item {Utilização:Ant.}
\end{itemize}
Catadupa.
Cataracta.
\section{Assono}
\begin{itemize}
\item {Grp. gram.:m.}
\end{itemize}
Planta indiana, o mesmo que \textunderscore mareta\textunderscore .
\section{Assonsar}
\begin{itemize}
\item {Grp. gram.:v. i.}
\end{itemize}
\begin{itemize}
\item {Utilização:Bras. do S}
\end{itemize}
Abombar um pouco.
\section{Assopeado}
\begin{itemize}
\item {Grp. gram.:adj.}
\end{itemize}
\begin{itemize}
\item {Utilização:Prov.}
\end{itemize}
\begin{itemize}
\item {Utilização:trasm.}
\end{itemize}
Acabrunhado.
Afrontado.
Perseguido por crèdores.
(Cp. \textunderscore sopear\textunderscore )
\section{Assopear}
\begin{itemize}
\item {Grp. gram.:v. t.}
\end{itemize}
O mesmo que \textunderscore sopear\textunderscore .
\section{Assopradela}
\begin{itemize}
\item {Grp. gram.:f.}
\end{itemize}
Acção de \textunderscore assoprar\textunderscore .
\section{Assoprador}
\begin{itemize}
\item {Grp. gram.:m.}
\end{itemize}
Aquelle que assopra.
\section{Assopradura}
\begin{itemize}
\item {Grp. gram.:f.}
\end{itemize}
(V.assopro)
\section{Assoprar}
\begin{itemize}
\item {Grp. gram.:v. t.  e  i.}
\end{itemize}
\begin{itemize}
\item {Utilização:Gír.}
\end{itemize}
O mesmo que \textunderscore soprar\textunderscore .
Denunciar.
\section{Assopro}
\begin{itemize}
\item {Grp. gram.:m.}
\end{itemize}
O mesmo que \textunderscore sôpro\textunderscore .
\section{Assoreamento}
\begin{itemize}
\item {Grp. gram.:m.}
\end{itemize}
\begin{itemize}
\item {Proveniência:(De \textunderscore assorear\textunderscore )}
\end{itemize}
Acervo de terras ou de areias, occasionado por enchentes ou por obras de arte.
\section{Assorear}
\begin{itemize}
\item {Grp. gram.:v. t.}
\end{itemize}
\begin{itemize}
\item {Grp. gram.:V. i.}
\end{itemize}
\begin{itemize}
\item {Proveniência:(De \textunderscore so...\textunderscore  + \textunderscore arear\textunderscore , de \textunderscore areia\textunderscore )}
\end{itemize}
Produzir assoreamento em: \textunderscore as cheias assorearam o Mondego\textunderscore .
Soffrer assoreamento: \textunderscore o Tejo tem assoreado nalguns pontos\textunderscore .
\section{Assossegar}
\begin{itemize}
\item {Grp. gram.:v. t.}
\end{itemize}
O mesmo que \textunderscore sossegar\textunderscore .
\section{Assovelado}
\begin{itemize}
\item {Grp. gram.:adj.}
\end{itemize}
Que tem fórma de sovela.
\section{Assovelar}
\begin{itemize}
\item {Grp. gram.:v. t.}
\end{itemize}
Dar fórma de sovela a.
Furar com sovela.
Picar com sovela.
Espicaçar.
Irritar.
\section{Assoviar}
\begin{itemize}
\item {Grp. gram.:v. t.}
\end{itemize}
O mesmo que \textunderscore assobiar\textunderscore .
\section{Assovinar}
\begin{itemize}
\item {Grp. gram.:v. t.}
\end{itemize}
Picar, furar, com sovina.
Assovelar.
Espicaçar; estimular.
Irritar.
\section{Assovinhar}
\begin{itemize}
\item {Grp. gram.:v. t.}
\end{itemize}
\begin{itemize}
\item {Utilização:Prov.}
\end{itemize}
\begin{itemize}
\item {Utilização:trasm.}
\end{itemize}
Coser mal, dando ponto aquém e ponto além, para apressar o trabalho.
(Relaciona-se com \textunderscore assovinar\textunderscore ?)
\section{Assuada}
\begin{itemize}
\item {Grp. gram.:f.}
\end{itemize}
Reunião de gente armada.
Desordem; motim.
Pessôas, que se agrupam, para a perpetração de um crime.
Vozearia.
Arruaça; gritaria insultuosa.
(B. lat. \textunderscore assunata\textunderscore )
\section{Assuar}
\begin{itemize}
\item {Grp. gram.:v. t.}
\end{itemize}
\begin{itemize}
\item {Utilização:Des.}
\end{itemize}
Insultar com vaias.
Amotinar para desordem ou crime.
(B. lat. \textunderscore assunare\textunderscore )
\section{Assubir}
\begin{itemize}
\item {Grp. gram.:v. i.}
\end{itemize}
O mesmo que \textunderscore subir\textunderscore . Cf. Castilho, \textunderscore Outono\textunderscore , 249.
\section{Assucceder}
\begin{itemize}
\item {Grp. gram.:v. i.}
\end{itemize}
\begin{itemize}
\item {Utilização:ant.}
\end{itemize}
\begin{itemize}
\item {Utilização:Pop.}
\end{itemize}
O mesmo que \textunderscore succeder\textunderscore .
\section{Assuceder}
\begin{itemize}
\item {Grp. gram.:v. i.}
\end{itemize}
\begin{itemize}
\item {Utilização:ant.}
\end{itemize}
\begin{itemize}
\item {Utilização:Pop.}
\end{itemize}
O mesmo que \textunderscore suceder\textunderscore .
\section{Assujeitar}
\textunderscore v. t.\textunderscore  (e der.)
(Fórma pop. de \textunderscore sujeitar\textunderscore , etc.)
\section{Assumir}
\begin{itemize}
\item {Grp. gram.:v. t.}
\end{itemize}
\begin{itemize}
\item {Proveniência:(Lat. \textunderscore assumere\textunderscore )}
\end{itemize}
Tomar sôbre si. Tomar para si; avocar.
Entrar no exercício de: \textunderscore assumir as suas funcções de juiz\textunderscore .
\section{Assumpção}
\begin{itemize}
\item {Grp. gram.:f.}
\end{itemize}
\begin{itemize}
\item {Proveniência:(Lat. \textunderscore assumptio\textunderscore )}
\end{itemize}
Acto de assumir.
Elevação a uma dignidade superior.
Elevação de Nossa-Senhora aos Céus; festa, com que a Igreja celebra esta elevação.
Acto, em que a Divindade encarnou em a natureza humana.
\section{Assumpcionistas}
\begin{itemize}
\item {Grp. gram.:m. pl.}
\end{itemize}
\begin{itemize}
\item {Proveniência:(De \textunderscore Assumpção\textunderscore , n. p.)}
\end{itemize}
Congregação religiosa, fundada em Nimes, em 1847, e chamada por Pio X a collaborar nas missões do Oriente.
\section{Assumptível}
\begin{itemize}
\item {Grp. gram.:adj.}
\end{itemize}
\begin{itemize}
\item {Proveniência:(De \textunderscore assumpto\textunderscore )}
\end{itemize}
Que póde sêr assumido.
\section{Assumptivo}
\begin{itemize}
\item {Grp. gram.:adj.}
\end{itemize}
\begin{itemize}
\item {Proveniência:(De \textunderscore assumpto\textunderscore )}
\end{itemize}
Que se assume.
Que se adopta.
\section{Assumpto}
\begin{itemize}
\item {Grp. gram.:m.}
\end{itemize}
\begin{itemize}
\item {Proveniência:(Lat. \textunderscore assumptus\textunderscore )}
\end{itemize}
Objecto; matéria, de que se trata: \textunderscore o assumpto de um discurso\textunderscore .
Thema.
\section{Assunado}
\begin{itemize}
\item {Grp. gram.:adj.}
\end{itemize}
\begin{itemize}
\item {Utilização:Ant.}
\end{itemize}
Junto, reunido.
\section{Assunamento}
\begin{itemize}
\item {Grp. gram.:m.}
\end{itemize}
Acto de \textunderscore assunar\textunderscore .
\section{Assunar}
\begin{itemize}
\item {Grp. gram.:v. t.}
\end{itemize}
\begin{itemize}
\item {Utilização:Ant.}
\end{itemize}
Juntar.
Amotinar; assuar.
(B. lat. \textunderscore assunare\textunderscore )
\section{Assunção}
\begin{itemize}
\item {Grp. gram.:f.}
\end{itemize}
\begin{itemize}
\item {Proveniência:(Lat. \textunderscore assumptio\textunderscore )}
\end{itemize}
Acto de assumir.
Elevação a uma dignidade superior.
Elevação de Nossa-Senhora aos Céus; festa, com que a Igreja celebra esta elevação.
Acto, em que a Divindade encarnou em a natureza humana.
\section{Assuncionistas}
\begin{itemize}
\item {Grp. gram.:m. pl.}
\end{itemize}
\begin{itemize}
\item {Proveniência:(De \textunderscore Assumpção\textunderscore , n. p.)}
\end{itemize}
Congregação religiosa, fundada em Nimes, em 1847, e chamada por Pio X a collaborar nas missões do Oriente.
\section{Assuntar}
\begin{itemize}
\item {Grp. gram.:v. i.}
\end{itemize}
\begin{itemize}
\item {Utilização:Bras}
\end{itemize}
\begin{itemize}
\item {Proveniência:(De \textunderscore assumpto\textunderscore ?)}
\end{itemize}
Dar attenção.
\section{Assuntível}
\begin{itemize}
\item {Grp. gram.:adj.}
\end{itemize}
\begin{itemize}
\item {Proveniência:(De \textunderscore assumpto\textunderscore )}
\end{itemize}
Que póde sêr assumido.
\section{Assuntivo}
\begin{itemize}
\item {Grp. gram.:adj.}
\end{itemize}
\begin{itemize}
\item {Proveniência:(De \textunderscore assumpto\textunderscore )}
\end{itemize}
Que se assume.
Que se adopta.
\section{Assunto}
\begin{itemize}
\item {Grp. gram.:m.}
\end{itemize}
\begin{itemize}
\item {Proveniência:(Lat. \textunderscore assumptus\textunderscore )}
\end{itemize}
Objecto; matéria, de que se trata: \textunderscore o assumpto de um discurso\textunderscore .
Thema.
\section{Assupá}
\begin{itemize}
\item {Grp. gram.:m.}
\end{itemize}
Arbusto brasileiro da região amazónica.
\section{Assurgir}
\begin{itemize}
\item {Grp. gram.:v. i.}
\end{itemize}
\begin{itemize}
\item {Utilização:Des.}
\end{itemize}
O mesmo que \textunderscore surgir\textunderscore .
\section{Assuso}
\begin{itemize}
\item {Grp. gram.:adv.}
\end{itemize}
\begin{itemize}
\item {Utilização:Ant.}
\end{itemize}
\begin{itemize}
\item {Proveniência:(De \textunderscore suso\textunderscore )}
\end{itemize}
Acima.
\section{Assustadamente}
\begin{itemize}
\item {Grp. gram.:adv.}
\end{itemize}
Com susto.
De modo \textunderscore assustado\textunderscore .
\section{Assustadiço}
\begin{itemize}
\item {Grp. gram.:adj.}
\end{itemize}
Que se assusta facilmente.
\section{Assustado}
\begin{itemize}
\item {Grp. gram.:adj.}
\end{itemize}
Que se assustou.
Intimidado.
\section{Assustador}
\begin{itemize}
\item {Grp. gram.:m.}
\end{itemize}
\begin{itemize}
\item {Grp. gram.:Adj.}
\end{itemize}
Aquelle que assusta.
Que assusta.
\section{Assustar}
\begin{itemize}
\item {Grp. gram.:v. t.}
\end{itemize}
Causar susto a; intimidar.
\section{Assustoso}
\begin{itemize}
\item {Grp. gram.:adj.}
\end{itemize}
Que causa susto; que faz medo.
\section{Assutilar}
\begin{itemize}
\item {Grp. gram.:v. t.}
\end{itemize}
\begin{itemize}
\item {Utilização:Des.}
\end{itemize}
Tornar subtil, subtilizar.
(Por \textunderscore assubtilar\textunderscore )
\section{Assuxar}
\begin{itemize}
\item {Grp. gram.:v. t.}
\end{itemize}
O mesmo que \textunderscore suxar\textunderscore . Cf. \textunderscore Eufrosina\textunderscore , 109.
\section{Assyriano}
\begin{itemize}
\item {Grp. gram.:adj.}
\end{itemize}
O mesmo que assýrico.
\section{Assýrico}
\begin{itemize}
\item {Grp. gram.:adj.}
\end{itemize}
Relativo á Assýria.
\section{Assýrio}
\begin{itemize}
\item {Grp. gram.:m.}
\end{itemize}
\begin{itemize}
\item {Grp. gram.:Adj.}
\end{itemize}
Aquelle que é natural da Assýria.
Dialecto da Assýria.
Relativo á Assýria.
\section{Assyriologia}
\begin{itemize}
\item {Grp. gram.:f.}
\end{itemize}
Estudo da Archeologia e Philologia da Assýria.
\section{Assyriológico}
\begin{itemize}
\item {Grp. gram.:adj.}
\end{itemize}
Relativo á assyriologia.
\section{Assyriólogo}
\begin{itemize}
\item {Grp. gram.:m.}
\end{itemize}
\begin{itemize}
\item {Proveniência:(De \textunderscore Assýria\textunderscore , n. p. + gr. \textunderscore logos\textunderscore )}
\end{itemize}
Aquelle que se dedica á assyriologia.
\section{Asta!}
\begin{itemize}
\item {Grp. gram.:interj.}
\end{itemize}
(para fazer recuar os bois jungidos)
\section{Astacites}
\begin{itemize}
\item {Grp. gram.:m. pl.}
\end{itemize}
\begin{itemize}
\item {Proveniência:(De \textunderscore astaco\textunderscore )}
\end{itemize}
Família de crustáceos, semelhantes aos caranguejos.
\section{Ástaco}
\begin{itemize}
\item {Grp. gram.:m.}
\end{itemize}
\begin{itemize}
\item {Proveniência:(Lat. \textunderscore astacus\textunderscore )}
\end{itemize}
Nome scientífico do caranquejo.
\section{Astarteia}
\begin{itemize}
\item {Grp. gram.:f.}
\end{itemize}
\begin{itemize}
\item {Proveniência:(De \textunderscore Astarte\textunderscore , n. p.)}
\end{itemize}
Arbusto myrtáceo da Austrália.
Gênero de molluscos acéphalos.
\section{Astasia}
\begin{itemize}
\item {Grp. gram.:f.}
\end{itemize}
\begin{itemize}
\item {Utilização:Med.}
\end{itemize}
\begin{itemize}
\item {Proveniência:(Gr. \textunderscore astasia\textunderscore )}
\end{itemize}
Impossibilidade de se manter o corpo erecto.
\section{Astático}
\begin{itemize}
\item {Grp. gram.:adj.}
\end{itemize}
\begin{itemize}
\item {Proveniência:(Do gr. \textunderscore a\textunderscore  + \textunderscore statikos\textunderscore )}
\end{itemize}
Que não é estável.
\section{Asteísmo}
\begin{itemize}
\item {Grp. gram.:m.}
\end{itemize}
\begin{itemize}
\item {Proveniência:(Gr. \textunderscore asteisma\textunderscore )}
\end{itemize}
Expressão graciosa, levemente irónica.
\section{Astela}
\begin{itemize}
\item {Grp. gram.:f.}
\end{itemize}
\begin{itemize}
\item {Proveniência:(Do lat. \textunderscore hasta\textunderscore )}
\end{itemize}
Apparelho cirúrgico, que se applica a membros fracturados.
\section{Astélia}
\begin{itemize}
\item {Grp. gram.:f.}
\end{itemize}
\begin{itemize}
\item {Proveniência:(De \textunderscore Astele\textunderscore , n. p.)}
\end{itemize}
Planta herbácea, vivaz.
\section{Asictos}
\begin{itemize}
\item {Grp. gram.:m.}
\end{itemize}
Pedra preciosa, conhecida dos antigos.
\section{Asilado}
\begin{itemize}
\item {Grp. gram.:adj.}
\end{itemize}
\begin{itemize}
\item {Grp. gram.:M.}
\end{itemize}
Que vive recolhido num asilo.
Indivíduo asilado.
\section{Asilar}
\begin{itemize}
\item {Grp. gram.:v. t.}
\end{itemize}
Dar asilo a.
Albergar, abrigar.
\section{Asilo}
\begin{itemize}
\item {Grp. gram.:m.}
\end{itemize}
\begin{itemize}
\item {Proveniência:(Lat. \textunderscore asylum\textunderscore )}
\end{itemize}
Lugar inviolável, em que antigamente se buscava refúgio.
Abrigo.
Protecção.
Retiro.
Estabelecimento de caridade, para educar crianças pobres ou recolher vadios, inválidos, etc.
\section{Assimbolia}
\begin{itemize}
\item {Grp. gram.:f.}
\end{itemize}
O mesmo que \textunderscore asemia\textunderscore .
\section{Assimetria}
\begin{itemize}
\item {Grp. gram.:f.}
\end{itemize}
\begin{itemize}
\item {Proveniência:(Do gr. \textunderscore a\textunderscore  priv. + \textunderscore symetria\textunderscore )}
\end{itemize}
Falta de simetria.
\section{Assimétrico}
\begin{itemize}
\item {Grp. gram.:adj.}
\end{itemize}
Em que há \textunderscore assimetria\textunderscore .
\section{Assimptota}
\begin{itemize}
\item {Grp. gram.:f.}
\end{itemize}
\begin{itemize}
\item {Utilização:Geom.}
\end{itemize}
\begin{itemize}
\item {Proveniência:(Do gr. \textunderscore a\textunderscore  priv. + \textunderscore symptotos\textunderscore )}
\end{itemize}
Linha recta, que se aproxima indefinidamente de uma curva, sem poder tocá-la.
\section{Assimptótico}
\begin{itemize}
\item {Grp. gram.:adj.}
\end{itemize}
Relativo á \textunderscore assimptota\textunderscore .
\section{Assindético}
\begin{itemize}
\item {Grp. gram.:adj.}
\end{itemize}
Que tem assíndeto, ou em que há assíndeto.
\section{Assíndeton}
\begin{itemize}
\item {Grp. gram.:m.}
\end{itemize}
\begin{itemize}
\item {Utilização:Gram.}
\end{itemize}
\begin{itemize}
\item {Proveniência:(Gr. \textunderscore asundetos\textunderscore )}
\end{itemize}
Suppressão da conjuncção copulativa entre phrases, ou entre partes de uma phrase.
\section{Assinergia}
\begin{itemize}
\item {Grp. gram.:f.}
\end{itemize}
Falta de energia.
\section{Assistásia}
\begin{itemize}
\item {Grp. gram.:f.}
\end{itemize}
Gênero de plantas acantháceas.
\section{Assistolia}
\begin{itemize}
\item {Grp. gram.:f.}
\end{itemize}
\begin{itemize}
\item {Utilização:Med.}
\end{itemize}
\begin{itemize}
\item {Proveniência:(Do gr. \textunderscore a\textunderscore  priv. + \textunderscore sustole\textunderscore )}
\end{itemize}
Insufficiência ou falta da sýstole do coração.
\section{Astema}
\begin{itemize}
\item {Grp. gram.:m.}
\end{itemize}
\begin{itemize}
\item {Proveniência:(Do gr. \textunderscore a\textunderscore  priv. + \textunderscore stemma\textunderscore )}
\end{itemize}
Planta dos Andes.
\section{Astemma}
\begin{itemize}
\item {Grp. gram.:m.}
\end{itemize}
\begin{itemize}
\item {Proveniência:(Do gr. \textunderscore a\textunderscore  priv. + \textunderscore stemma\textunderscore )}
\end{itemize}
Planta dos Andes.
\section{Astenia}
\begin{itemize}
\item {Grp. gram.:f.}
\end{itemize}
\begin{itemize}
\item {Proveniência:(Gr. \textunderscore astheneia\textunderscore )}
\end{itemize}
Fraqueza; debilidade.
\section{Astênico}
\begin{itemize}
\item {Grp. gram.:adj.}
\end{itemize}
Que padece \textunderscore astenia\textunderscore .
\section{Astenopia}
\begin{itemize}
\item {Grp. gram.:f.}
\end{itemize}
\begin{itemize}
\item {Proveniência:(Do gr. \textunderscore a\textunderscore  priv. + \textunderscore stenos\textunderscore  + \textunderscore ops\textunderscore )}
\end{itemize}
Cansaço occasional da vista, determinado pela applicação della.
\section{Astenopira}
\begin{itemize}
\item {Grp. gram.:f.}
\end{itemize}
\begin{itemize}
\item {Utilização:Med.}
\end{itemize}
\begin{itemize}
\item {Proveniência:(Do gr. \textunderscore asthenes\textunderscore  + \textunderscore puros\textunderscore )}
\end{itemize}
Febre, acompanhada de prostração de fôrças.
\section{Áster}
\begin{itemize}
\item {Grp. gram.:m.}
\end{itemize}
\begin{itemize}
\item {Proveniência:(Gr. \textunderscore aster\textunderscore )}
\end{itemize}
Gênero de plantas vivazes, de que há muitas espécies cultivadas em jardins.
\section{Asteracanta}
\begin{itemize}
\item {Grp. gram.:f.}
\end{itemize}
\begin{itemize}
\item {Proveniência:(Do gr. \textunderscore aster\textunderscore  + \textunderscore akantha\textunderscore )}
\end{itemize}
Planta acanthácea, de longas fôlhas, originária da Índia.
\section{Asteracantha}
\begin{itemize}
\item {Grp. gram.:f.}
\end{itemize}
\begin{itemize}
\item {Proveniência:(Do gr. \textunderscore aster\textunderscore  + \textunderscore akantha\textunderscore )}
\end{itemize}
Planta acanthácea, de longas fôlhas, originária da Índia.
\section{Asterela}
\begin{itemize}
\item {Grp. gram.:f.}
\end{itemize}
\begin{itemize}
\item {Proveniência:(Do gr. \textunderscore aster\textunderscore )}
\end{itemize}
Gênero de cogumelos.
\section{Astéria}
\begin{itemize}
\item {Grp. gram.:f.}
\end{itemize}
\begin{itemize}
\item {Proveniência:(Do gr. \textunderscore aster\textunderscore )}
\end{itemize}
Espécie de opala, que tem a qualidade do asterismo.
\section{Astéria}
\begin{itemize}
\item {Grp. gram.:f.}
\end{itemize}
\begin{itemize}
\item {Proveniência:(Gr. \textunderscore asterios\textunderscore )}
\end{itemize}
Zoóphyto radiário, mais conhecido por \textunderscore estrêlla-do-mar\textunderscore .
\section{Astérico}
\begin{itemize}
\item {Grp. gram.:adj.}
\end{itemize}
\begin{itemize}
\item {Utilização:Anat.}
\end{itemize}
\begin{itemize}
\item {Proveniência:(De \textunderscore astérion\textunderscore )}
\end{itemize}
Diz-se do ângulo póstero-inferior dos parietaes.
\section{Asterídeos}
\begin{itemize}
\item {Grp. gram.:m. pl.}
\end{itemize}
\begin{itemize}
\item {Proveniência:(Do gr. \textunderscore asterios\textunderscore  + \textunderscore eidos\textunderscore )}
\end{itemize}
Família de radiários, a que serve de typo a astéria^2.
\section{Asterino}
\begin{itemize}
\item {Grp. gram.:m.}
\end{itemize}
\begin{itemize}
\item {Utilização:Bras}
\end{itemize}
Planta annual.
\section{Astérion}
\begin{itemize}
\item {Grp. gram.:m.}
\end{itemize}
\begin{itemize}
\item {Utilização:Anat.}
\end{itemize}
\begin{itemize}
\item {Proveniência:(Do gr. \textunderscore aster\textunderscore )}
\end{itemize}
Cruzamento das três suturas cranianas, a occípito-parietal, a mastoido-parietal e a mastoido-occipital.
\section{Asterisco}
\begin{itemize}
\item {Grp. gram.:m.}
\end{itemize}
\begin{itemize}
\item {Proveniência:(Gr. \textunderscore asteriskos\textunderscore )}
\end{itemize}
Sinal, que, em fórma de estrêlla, tem na escrita significação convencional, ou serve de remissão.
\section{Asterismo}
\begin{itemize}
\item {Grp. gram.:m.}
\end{itemize}
\begin{itemize}
\item {Proveniência:(Gr. \textunderscore asterismos\textunderscore )}
\end{itemize}
Constellação.
Qualidade, que alguns mineraes possuem, de apresentar a imagem de uma estrêlla de quatro ou seis raios.
\section{Asternal}
\begin{itemize}
\item {Grp. gram.:adj.}
\end{itemize}
\begin{itemize}
\item {Utilização:Anat.}
\end{itemize}
\begin{itemize}
\item {Proveniência:(Do gr. \textunderscore a\textunderscore  priv. + \textunderscore sternon\textunderscore )}
\end{itemize}
Diz-se das costelas, que se não articulam com o esterno.
\section{Asterodermo}
\begin{itemize}
\item {Grp. gram.:m.}
\end{itemize}
\begin{itemize}
\item {Proveniência:(Do gr. \textunderscore aster\textunderscore  + \textunderscore derma\textunderscore )}
\end{itemize}
Gênero de peixes, da fam. das raias.
\section{Asteroide}
\begin{itemize}
\item {Grp. gram.:m.}
\end{itemize}
\begin{itemize}
\item {Grp. gram.:Adj.}
\end{itemize}
\begin{itemize}
\item {Proveniência:(Do gr. \textunderscore aster\textunderscore  + \textunderscore eidos\textunderscore )}
\end{itemize}
Pequeno planeta.
Pequeno corpo cósmico, que percorre o espaço, como as estrêllas cadentes e os aerólithos.
Semelhante a uma estrêlla.
\section{Asteroma}
\begin{itemize}
\item {Grp. gram.:m.}
\end{itemize}
Gênero de cogumelos.
\section{Asteróscopo}
\begin{itemize}
\item {Grp. gram.:m.}
\end{itemize}
\begin{itemize}
\item {Proveniência:(Do gr. \textunderscore aster\textunderscore  + \textunderscore skopein\textunderscore )}
\end{itemize}
Insecto lepidóptero nocturno.
\section{Asthenia}
\begin{itemize}
\item {Grp. gram.:f.}
\end{itemize}
\begin{itemize}
\item {Proveniência:(Gr. \textunderscore astheneia\textunderscore )}
\end{itemize}
Fraqueza; debilidade.
\section{Asthênico}
\begin{itemize}
\item {Grp. gram.:adj.}
\end{itemize}
Que padece \textunderscore asthenia\textunderscore .
\section{Asthenopyra}
\begin{itemize}
\item {Grp. gram.:f.}
\end{itemize}
\begin{itemize}
\item {Utilização:Med.}
\end{itemize}
\begin{itemize}
\item {Proveniência:(Do gr. \textunderscore asthenes\textunderscore  + \textunderscore puros\textunderscore )}
\end{itemize}
Febre, acompanhada de prostração de fôrças.
\section{Asthma}
\textunderscore f.\textunderscore  (e der.)
O mesmo que \textunderscore asma\textunderscore , etc.
\section{Astidâmia}
\begin{itemize}
\item {Grp. gram.:f.}
\end{itemize}
Gênero de plantas umbellíferas.
\section{Astigmação}
\begin{itemize}
\item {Grp. gram.:f.}
\end{itemize}
\begin{itemize}
\item {Utilização:Phot.}
\end{itemize}
\begin{itemize}
\item {Proveniência:(Do gr. \textunderscore a\textunderscore  priv. + \textunderscore stigma\textunderscore )}
\end{itemize}
Aberração de posição da imagem nas matrizas.
\section{Astigmatismo}
\begin{itemize}
\item {Grp. gram.:m.}
\end{itemize}
\begin{itemize}
\item {Proveniência:(Do gr. \textunderscore a\textunderscore  priv. + \textunderscore stigma\textunderscore )}
\end{itemize}
Perturbação da vista, por defeito na curvatura do crystallino.
\section{Astil}
\begin{itemize}
\item {Grp. gram.:m.}
\end{itemize}
\begin{itemize}
\item {Utilização:Ant.}
\end{itemize}
Medida agrária de 25 palmos de largo.
Segundo outros, esta medida correspondia á \textunderscore aquilhada\textunderscore , e orçava por 5^{m},50.
(Por \textunderscore hastil\textunderscore , de \textunderscore haste\textunderscore )
\section{Astim}
\begin{itemize}
\item {Grp. gram.:m.}
\end{itemize}
(V.astil)
\section{Astinomia}
\begin{itemize}
\item {Grp. gram.:f.}
\end{itemize}
Dignidade ou funcções de \textunderscore astínomo\textunderscore .
\section{Astínomo}
\begin{itemize}
\item {Grp. gram.:m.}
\end{itemize}
Magistrado grego, que tinha a seu cargo a polícia das ruas.
\section{Astma}
\textunderscore f.\textunderscore  (e der.)
O mesmo que \textunderscore asma\textunderscore , etc.
\section{Asto}
\begin{itemize}
\item {Grp. gram.:m.}
\end{itemize}
\begin{itemize}
\item {Utilização:Ant.}
\end{itemize}
\begin{itemize}
\item {Grp. gram.:Adj.}
\end{itemize}
\begin{itemize}
\item {Proveniência:(Lat. ant. \textunderscore astus\textunderscore )}
\end{itemize}
O mesmo que \textunderscore astúcia\textunderscore .
O mesmo que \textunderscore astuto\textunderscore .
\section{Astomia}
\begin{itemize}
\item {Grp. gram.:f.}
\end{itemize}
\begin{itemize}
\item {Proveniência:(Do gr. \textunderscore a\textunderscore  priv. + \textunderscore stoma\textunderscore )}
\end{itemize}
Ausência congênita da bôca.
\section{Ástomos}
\begin{itemize}
\item {Grp. gram.:m. pl.}
\end{itemize}
\begin{itemize}
\item {Proveniência:(Do gr. \textunderscore a\textunderscore  priv. + \textunderscore stoma\textunderscore )}
\end{itemize}
Musgos, cuja cápsula não tem abertura.
\section{Astragalectomia}
\begin{itemize}
\item {Grp. gram.:f.}
\end{itemize}
\begin{itemize}
\item {Utilização:Cir.}
\end{itemize}
Extracção do astrágalo.
\section{Astragália}
\begin{itemize}
\item {Grp. gram.:f.}
\end{itemize}
Perfil de cornija ou contôrno de moldura, terminado em \textunderscore astrágalo\textunderscore .
\section{Astragaliano}
\begin{itemize}
\item {Grp. gram.:adj.}
\end{itemize}
Relativo ao \textunderscore astrágalo\textunderscore .
\section{Astrágalo}
\begin{itemize}
\item {Grp. gram.:m.}
\end{itemize}
\begin{itemize}
\item {Utilização:Anat.}
\end{itemize}
\begin{itemize}
\item {Utilização:Constr.}
\end{itemize}
\begin{itemize}
\item {Utilização:Artilh.}
\end{itemize}
\begin{itemize}
\item {Utilização:Bot.}
\end{itemize}
\begin{itemize}
\item {Utilização:Constr.}
\end{itemize}
\begin{itemize}
\item {Proveniência:(Gr. \textunderscore astrágalos\textunderscore )}
\end{itemize}
Um dos ossos do tarso.
Moldura, que cerca a parte superior do fuste de uma columna.
Filete, em volta do canhão, junto á bôca.
Planta leguminosa, papilionácea.
Tondinho de pequenas dimensões.
\section{Astragaloide}
\begin{itemize}
\item {Grp. gram.:adj.}
\end{itemize}
\begin{itemize}
\item {Proveniência:(Do gr. \textunderscore astragalos\textunderscore  + \textunderscore eidos\textunderscore )}
\end{itemize}
Que tem semelhança com o astrágalo.
\section{Astragalomancia}
\begin{itemize}
\item {Grp. gram.:f.}
\end{itemize}
\begin{itemize}
\item {Proveniência:(Do gr. \textunderscore astragalos\textunderscore  + \textunderscore manteia\textunderscore )}
\end{itemize}
Systema de adivinhação, praticado pelos antigos Gregos com dados, em que se marcavam as letras do alphabeto.
\section{Astral}
\begin{itemize}
\item {Grp. gram.:adj.}
\end{itemize}
\begin{itemize}
\item {Grp. gram.:M.}
\end{itemize}
\begin{itemize}
\item {Proveniência:(Lat. \textunderscore astralis\textunderscore )}
\end{itemize}
Relativo aos astros; sideral.
Região fantástica, povoada de miragens e sombras que, segundo os occultistas, são observadas por videntes, possessos e hypnóticos.
\section{Astralização}
\begin{itemize}
\item {Grp. gram.:f.}
\end{itemize}
Acto de \textunderscore astralizar-se\textunderscore .
\section{Astralizar-se}
\begin{itemize}
\item {Grp. gram.:v. p.}
\end{itemize}
\begin{itemize}
\item {Utilização:Neol.}
\end{itemize}
Viver no astral; sêr vidente.
\section{Astrança}
\begin{itemize}
\item {Grp. gram.:f.}
\end{itemize}
O mesmo que \textunderscore astrância\textunderscore . Cf. \textunderscore Desengano da Med.\textunderscore , 36.
\section{Astrância}
\begin{itemize}
\item {Grp. gram.:f.}
\end{itemize}
\begin{itemize}
\item {Proveniência:(Do gr. \textunderscore astron\textunderscore )}
\end{itemize}
Planta umbellífera.
\section{Astrápea}
\begin{itemize}
\item {Grp. gram.:f.}
\end{itemize}
Gênero de plantas americanas, de flôres rosadas.
\section{Ástrapo}
\begin{itemize}
\item {Grp. gram.:m.}
\end{itemize}
\begin{itemize}
\item {Proveniência:(Gr. \textunderscore astrape\textunderscore )}
\end{itemize}
Gênero de peixes eléctricos, da fam. das tremelgas.
\section{Astre}
\begin{itemize}
\item {Grp. gram.:m.}
\end{itemize}
\begin{itemize}
\item {Utilização:Ant.}
\end{itemize}
\begin{itemize}
\item {Proveniência:(Gr. \textunderscore aster\textunderscore )}
\end{itemize}
Fortuna; bôa sorte; bôa estrêlla.
\section{Astréa}
\begin{itemize}
\item {Grp. gram.:f.}
\end{itemize}
\begin{itemize}
\item {Proveniência:(Gr. \textunderscore Astraia\textunderscore , n. p.)}
\end{itemize}
Uma das constellações do Zodíaco, também chamada \textunderscore Virgem\textunderscore .
Nome de um planeta telescópico, descoberto em 1845.
Pólypo pétreo, de superfície estrellada.
\section{Astréfia}
\begin{itemize}
\item {Grp. gram.:f.}
\end{itemize}
Gênero de plantas valerianáceas.
\section{Astrego}
\begin{itemize}
\item {fónica:trê}
\end{itemize}
\begin{itemize}
\item {Grp. gram.:m.}
\end{itemize}
\begin{itemize}
\item {Utilização:Ant.}
\end{itemize}
Ligação de parentesco.
Obrigação.
Respeito.
(Cp. \textunderscore adstricto\textunderscore )
\section{Astreia}
\begin{itemize}
\item {Grp. gram.:f.}
\end{itemize}
\begin{itemize}
\item {Proveniência:(Gr. \textunderscore Astraia\textunderscore , n. p.)}
\end{itemize}
Uma das constellações do Zodíaco, também chamada \textunderscore Virgem\textunderscore .
Nome de um planeta telescópico, descoberto em 1845.
Pólypo pétreo, de superfície estrellada.
\section{Ástreo}
\begin{itemize}
\item {Grp. gram.:adj.}
\end{itemize}
O mesmo que \textunderscore ástrico\textunderscore .
\section{Astreópora}
\begin{itemize}
\item {Grp. gram.:f.}
\end{itemize}
Gênero de polypeiros, vizinhos das madréporas.
\section{Astréphia}
\begin{itemize}
\item {Grp. gram.:f.}
\end{itemize}
Gênero de plantas valerianáceas.
\section{Astrever-se}
\begin{itemize}
\item {Grp. gram.:v. p.}
\end{itemize}
(Fórma pop. de \textunderscore atrever-se\textunderscore )
\section{Astribordo}
\begin{itemize}
\item {Grp. gram.:m.}
\end{itemize}
\begin{itemize}
\item {Utilização:Ant.}
\end{itemize}
O mesmo que \textunderscore estibordo\textunderscore . Cf. \textunderscore Peregrinação\textunderscore , XL.
\section{Ástrico}
\begin{itemize}
\item {Grp. gram.:adj.}
\end{itemize}
\begin{itemize}
\item {Proveniência:(Gr. \textunderscore astrikos\textunderscore )}
\end{itemize}
O mesmo que \textunderscore astral\textunderscore .
Cheio de astros; constellado.
\section{Astricto}
\begin{itemize}
\item {Grp. gram.:adj.}
\end{itemize}
(V.adstricto)
\section{Astrífero}
\begin{itemize}
\item {Grp. gram.:adj.}
\end{itemize}
\begin{itemize}
\item {Proveniência:(Do lat. \textunderscore astrum\textunderscore  + \textunderscore ferre\textunderscore )}
\end{itemize}
Que tem astros.
\section{Astrígero}
\begin{itemize}
\item {Grp. gram.:adj.}
\end{itemize}
\begin{itemize}
\item {Proveniência:(Do lat. \textunderscore astrum\textunderscore  + \textunderscore gerere\textunderscore )}
\end{itemize}
O mesmo que \textunderscore astrífero\textunderscore .
\section{Astringir}
\textunderscore v. t.\textunderscore  (e der.)
(V. \textunderscore adstringir\textunderscore , etc.)
\section{Astriste}
\begin{itemize}
\item {Grp. gram.:f.}
\end{itemize}
Pedra preciosa, semelhante a crystal.
\section{Astro}
\begin{itemize}
\item {Grp. gram.:m.}
\end{itemize}
\begin{itemize}
\item {Utilização:Fig.}
\end{itemize}
\begin{itemize}
\item {Proveniência:(Lat. \textunderscore astrum\textunderscore )}
\end{itemize}
Qualquer corpo celeste, que tem marcha regular.
Pessôa illustre.
\section{Astrobolismo}
\begin{itemize}
\item {Grp. gram.:m.}
\end{itemize}
Paralysia súbita, que se attribuia á influência dos astros.
\section{Astróbolo}
\begin{itemize}
\item {Grp. gram.:m.}
\end{itemize}
Nome, que os antigos deram ao feldspatho nacarado, do que faziam uso em operações de magia.
\section{Astrocário}
\begin{itemize}
\item {Grp. gram.:m.}
\end{itemize}
\begin{itemize}
\item {Proveniência:(Do gr. \textunderscore astron\textunderscore  + \textunderscore karuon\textunderscore )}
\end{itemize}
Gênero de palmeiras, cujos frutos são comestíveis.
\section{Astrocárpeas}
\begin{itemize}
\item {Grp. gram.:f. pl.}
\end{itemize}
\begin{itemize}
\item {Proveniência:(De \textunderscore astrocarpo\textunderscore )}
\end{itemize}
Tribo de plantas resedáceas.
\section{Astrocarpo}
\begin{itemize}
\item {Grp. gram.:m.}
\end{itemize}
\begin{itemize}
\item {Proveniência:(Do gr. \textunderscore astron\textunderscore  + \textunderscore karpos\textunderscore )}
\end{itemize}
Gênero de plantas, que serve de typo ás astrocárpeas.
\section{Astrocinologia}
\begin{itemize}
\item {Grp. gram.:f.}
\end{itemize}
\begin{itemize}
\item {Proveniência:(Do gr. \textunderscore astron\textunderscore  + \textunderscore kuon\textunderscore  + \textunderscore logos\textunderscore )}
\end{itemize}
Tratado dos dias caniculares.
\section{Astrocinológico}
\begin{itemize}
\item {Grp. gram.:adj.}
\end{itemize}
Relativo á astrocinologia.
\section{Astrocynologia}
\begin{itemize}
\item {Grp. gram.:f.}
\end{itemize}
\begin{itemize}
\item {Proveniência:(Do gr. \textunderscore astron\textunderscore  + \textunderscore kuon\textunderscore  + \textunderscore logos\textunderscore )}
\end{itemize}
Tratado dos dias caniculares.
\section{Astrocynológico}
\begin{itemize}
\item {Grp. gram.:adj.}
\end{itemize}
Relativo á astrocynologia.
\section{Astrodendro}
\begin{itemize}
\item {Grp. gram.:m.}
\end{itemize}
O mesmo que \textunderscore estercúlia\textunderscore .
\section{Astroderme}
\begin{itemize}
\item {Grp. gram.:m.}
\end{itemize}
Gênero de peixes.
\section{Astrodinâmica}
\begin{itemize}
\item {Grp. gram.:f.}
\end{itemize}
\begin{itemize}
\item {Proveniência:(De \textunderscore astro\textunderscore  + \textunderscore dynámica\textunderscore )}
\end{itemize}
Sciência das fôrças que movem os astros.
\section{Astrodynâmica}
\begin{itemize}
\item {Grp. gram.:f.}
\end{itemize}
\begin{itemize}
\item {Proveniência:(De \textunderscore astro\textunderscore  + \textunderscore dynámica\textunderscore )}
\end{itemize}
Sciência das fôrças que movem os astros.
\section{Astrofísica}
\begin{itemize}
\item {Grp. gram.:f.}
\end{itemize}
Física dos astros.
\section{Astrofobia}
\begin{itemize}
\item {Grp. gram.:f.}
\end{itemize}
\begin{itemize}
\item {Proveniência:(Do gr. \textunderscore astron\textunderscore  + \textunderscore phobein\textunderscore )}
\end{itemize}
Medo mórbido dos relâmpagos e trovões.
\section{Astrófobo}
\begin{itemize}
\item {Grp. gram.:m.}
\end{itemize}
Aquelle que soffre astrofobia.
\section{Astroide}
\begin{itemize}
\item {Grp. gram.:m.  e  adj.}
\end{itemize}
(V.asteroide)
\section{Astroites}
\begin{itemize}
\item {Grp. gram.:m. pl.}
\end{itemize}
Polypeiros com céllulas estrelladas.
\section{Astrola}
\begin{itemize}
\item {Grp. gram.:m.}
\end{itemize}
Gênero de molluscos.
\section{Astrolábio}
\begin{itemize}
\item {Grp. gram.:m.}
\end{itemize}
Antigo instrumento, para observações astraes.
(B. lat. \textunderscore astrolabium\textunderscore )
\section{Astrólatra}
\begin{itemize}
\item {Grp. gram.:m.}
\end{itemize}
Aquelle que pratica a astrolatria.
\section{Astrolatria}
\begin{itemize}
\item {Grp. gram.:f.}
\end{itemize}
\begin{itemize}
\item {Proveniência:(Do gr. \textunderscore astron\textunderscore  + \textunderscore latreia\textunderscore )}
\end{itemize}
Adoração dos astros.
\section{Astróloga}
\begin{itemize}
\item {Grp. gram.:f.}
\end{itemize}
Mulher, que cultivava a Astrologia. Cf. \textunderscore Viriato Trág.\textunderscore , XIII, 82.
\section{Astrologia}
\begin{itemize}
\item {Grp. gram.:f.}
\end{itemize}
\begin{itemize}
\item {Proveniência:(Gr. \textunderscore astrologia\textunderscore )}
\end{itemize}
Supposta arte de lêr o futuro nos astros.
\section{Astrologicamente}
\begin{itemize}
\item {Grp. gram.:adv.}
\end{itemize}
De modo \textunderscore astrológico\textunderscore .
\section{Astrológico}
\begin{itemize}
\item {Grp. gram.:m.}
\end{itemize}
Relativo á \textunderscore Astrologia\textunderscore .
\section{Astrólogo}
\begin{itemize}
\item {Grp. gram.:m.}
\end{itemize}
Aquelle que se dedicava á \textunderscore Astrologia\textunderscore .
\section{Astromancia}
\begin{itemize}
\item {Grp. gram.:f.}
\end{itemize}
\begin{itemize}
\item {Proveniência:(Do gr. \textunderscore astron\textunderscore  + \textunderscore manteia\textunderscore )}
\end{itemize}
Arte de adivinhar, por meio dos astros.
\section{Astrometria}
\begin{itemize}
\item {Grp. gram.:f.}
\end{itemize}
\begin{itemize}
\item {Proveniência:(De \textunderscore astrómetro\textunderscore )}
\end{itemize}
Arte de medir o diâmetro apparente dos astros e a sua distância.
\section{Astrometrico}
\begin{itemize}
\item {Grp. gram.:adj.}
\end{itemize}
Relativo á \textunderscore astrometria\textunderscore .
\section{Astrómetro}
\begin{itemize}
\item {Grp. gram.:m.}
\end{itemize}
\begin{itemize}
\item {Proveniência:(Do gr. \textunderscore astron\textunderscore  + \textunderscore metron\textunderscore )}
\end{itemize}
Instrumento, para medir os diâmetros apparentes e as distâncias dos astros.
\section{Astrónia}
\begin{itemize}
\item {Grp. gram.:f.}
\end{itemize}
Gênero de plantas melastomáceas.
\section{Astrónio}
\begin{itemize}
\item {Grp. gram.:m.}
\end{itemize}
Gênero de plantas terebinthaceas.
\section{Astronomia}
\begin{itemize}
\item {Grp. gram.:f.}
\end{itemize}
\begin{itemize}
\item {Proveniência:(Gr. \textunderscore astronomia\textunderscore )}
\end{itemize}
Sciência, que estuda a constituição e movimento dos astros.
\section{Astronomicamente}
\begin{itemize}
\item {Grp. gram.:adv.}
\end{itemize}
\begin{itemize}
\item {Proveniência:(De \textunderscore astronómico\textunderscore )}
\end{itemize}
Segundo os princípios da Astronomia.
\section{Astronómico}
\begin{itemize}
\item {Grp. gram.:adj.}
\end{itemize}
\begin{itemize}
\item {Proveniência:(Gr. \textunderscore astronomikos\textunderscore )}
\end{itemize}
Que diz respeito, que pertence, á Astronomia.
\section{Astrónomo}
\begin{itemize}
\item {Grp. gram.:m.}
\end{itemize}
\begin{itemize}
\item {Proveniência:(Gr. \textunderscore astronomos\textunderscore )}
\end{itemize}
Aquelle que conhece e pratica a Astronomia.
\section{Astrophobia}
\begin{itemize}
\item {Grp. gram.:f.}
\end{itemize}
\begin{itemize}
\item {Proveniência:(Do gr. \textunderscore astron\textunderscore  + \textunderscore phobein\textunderscore )}
\end{itemize}
Medo mórbido dos relâmpagos e trovões.
\section{Astróphobo}
\begin{itemize}
\item {Grp. gram.:m.}
\end{itemize}
Aquelle que soffre astrophobia.
\section{Astrophýsica}
\begin{itemize}
\item {Grp. gram.:f.}
\end{itemize}
Phýsica dos astros.
\section{Astroscopia}
\begin{itemize}
\item {Grp. gram.:f.}
\end{itemize}
\begin{itemize}
\item {Proveniência:(Do gr. \textunderscore astron\textunderscore  + \textunderscore skopein\textunderscore )}
\end{itemize}
Observação dos astros.
\section{Astroscópio}
\begin{itemize}
\item {Grp. gram.:m.}
\end{itemize}
Antigo instrumento, para observação dos astros.
(Cp. \textunderscore astroscopia\textunderscore )
\section{Astrosia}
\begin{itemize}
\item {Grp. gram.:f.}
\end{itemize}
\begin{itemize}
\item {Utilização:Ant.}
\end{itemize}
\begin{itemize}
\item {Proveniência:(De \textunderscore astro\textunderscore )}
\end{itemize}
Superstição.
Acontecimento, em que se suppunha influírem os astros.
\section{Astroso}
\begin{itemize}
\item {Grp. gram.:adj.}
\end{itemize}
\begin{itemize}
\item {Proveniência:(Lat. \textunderscore astrosus\textunderscore )}
\end{itemize}
Que nasceu sob a influência de mau astro; infeliz.
Funesto; fúnebre.
\section{Astrostática}
\begin{itemize}
\item {Grp. gram.:f.}
\end{itemize}
\begin{itemize}
\item {Proveniência:(De \textunderscore astro\textunderscore  + \textunderscore estática\textunderscore )}
\end{itemize}
Parte da Astronomia, que trata do volume e da respectiva distância dos astros.
\section{Astúcia}
\begin{itemize}
\item {Grp. gram.:f.}
\end{itemize}
\begin{itemize}
\item {Proveniência:(Lat. \textunderscore astutia\textunderscore )}
\end{itemize}
Manha; habilidade para o mal.
Sagacidade.
Estratagema.
\section{Astuciar}
\begin{itemize}
\item {Grp. gram.:v. t.}
\end{itemize}
\begin{itemize}
\item {Grp. gram.:V. i.}
\end{itemize}
Planear com astúcia.
Servir-se de astúcia.
\section{Astuciosamente}
\begin{itemize}
\item {Grp. gram.:adv.}
\end{itemize}
De modo \textunderscore astucioso\textunderscore .
Com astúcia.
\section{Astucioso}
\begin{itemize}
\item {Grp. gram.:adj.}
\end{itemize}
Que tem astúcia: \textunderscore homem astucioso\textunderscore .
Em que há astúcia: \textunderscore procedimento astucioso\textunderscore .
\section{Ástur}
\begin{itemize}
\item {Grp. gram.:m.}
\end{itemize}
Ave de rapina, espécie de falcão.
O mesmo que \textunderscore asturiano\textunderscore .
\section{Asturiano}
\begin{itemize}
\item {Grp. gram.:m.}
\end{itemize}
\begin{itemize}
\item {Grp. gram.:Adj.}
\end{itemize}
Aquelle que é natural das Astúrias.
Dialecto das Astúrias.
Relativo ás Astúrias.
\section{Asturina}
\begin{itemize}
\item {Grp. gram.:f.}
\end{itemize}
Gênero de ave de rapina.
\section{Astutamente}
\begin{itemize}
\item {Grp. gram.:adv.}
\end{itemize}
De modo \textunderscore astuto\textunderscore .
Astuciosamente.
\section{Astuto}
\begin{itemize}
\item {Grp. gram.:adj.}
\end{itemize}
\begin{itemize}
\item {Proveniência:(Lat. \textunderscore astutus\textunderscore )}
\end{itemize}
Que tem astúcia; em que há astúcia; astucioso.
\section{Astydâmia}
\begin{itemize}
\item {Grp. gram.:f.}
\end{itemize}
Gênero de plantas umbellíferas.
\section{Astynomia}
\begin{itemize}
\item {Grp. gram.:f.}
\end{itemize}
Dignidade ou funcções de \textunderscore astýnomo\textunderscore .
\section{Astýnomo}
\begin{itemize}
\item {Grp. gram.:m.}
\end{itemize}
Magistrado grego, que tinha a seu cargo a polícia das ruas.
\section{Ás-vintes}
\begin{itemize}
\item {Grp. gram.:loc. adv.}
\end{itemize}
\begin{itemize}
\item {Utilização:Prov.}
\end{itemize}
\begin{itemize}
\item {Utilização:minh.}
\end{itemize}
Muito depressa.
\section{Asyctos}
\begin{itemize}
\item {Grp. gram.:m.}
\end{itemize}
Pedra preciosa, conhecida dos antigos.
\section{Asylado}
\begin{itemize}
\item {Grp. gram.:adj.}
\end{itemize}
\begin{itemize}
\item {Grp. gram.:M.}
\end{itemize}
Que vive recolhido num asylo.
Indivíduo asylado.
\section{Asylar}
\begin{itemize}
\item {Grp. gram.:v. t.}
\end{itemize}
Dar asylo a.
Albergar, abrigar.
\section{Asylo}
\begin{itemize}
\item {Grp. gram.:m.}
\end{itemize}
\begin{itemize}
\item {Proveniência:(Lat. \textunderscore asylum\textunderscore )}
\end{itemize}
Lugar inviolável, em que antigamente se buscava refúgio.
Abrigo.
Protecção.
Retiro.
Estabelecimento de caridade, para educar crianças pobres ou recolher vadios, inválidos, etc.
\section{Asymbolia}
\begin{itemize}
\item {fónica:sim}
\end{itemize}
\begin{itemize}
\item {Grp. gram.:f.}
\end{itemize}
O mesmo que \textunderscore asemia\textunderscore .
\section{Asymetria}
\begin{itemize}
\item {fónica:si}
\end{itemize}
\begin{itemize}
\item {Grp. gram.:f.}
\end{itemize}
\begin{itemize}
\item {Proveniência:(Do gr. \textunderscore a\textunderscore  priv. + \textunderscore symetria\textunderscore )}
\end{itemize}
Falta de symetria.
\section{Asymétrico}
\begin{itemize}
\item {fónica:si}
\end{itemize}
\begin{itemize}
\item {Grp. gram.:adj.}
\end{itemize}
Em que há \textunderscore asymetria\textunderscore .
\section{Asymptota}
\begin{itemize}
\item {fónica:sim}
\end{itemize}
\begin{itemize}
\item {Grp. gram.:f.}
\end{itemize}
\begin{itemize}
\item {Utilização:Geom.}
\end{itemize}
\begin{itemize}
\item {Proveniência:(Do gr. \textunderscore a\textunderscore  priv. + \textunderscore symptotos\textunderscore )}
\end{itemize}
Linha recta, que se aproxima indefinidamente de uma curva, sem poder tocá-la.
\section{Asymptótico}
\begin{itemize}
\item {fónica:sim}
\end{itemize}
\begin{itemize}
\item {Grp. gram.:adj.}
\end{itemize}
Relativo á \textunderscore asymptota\textunderscore .
\section{Asyndético}
\begin{itemize}
\item {fónica:sin}
\end{itemize}
\begin{itemize}
\item {Grp. gram.:adj.}
\end{itemize}
Que tem asýndeto, ou em que há asýndeto.
\section{Asýndeto}
\begin{itemize}
\item {Grp. gram.:m.}
\end{itemize}
O mesmo ou melhor que \textunderscore asýndeton\textunderscore .
\section{Asýndeton}
\begin{itemize}
\item {fónica:sin}
\end{itemize}
\begin{itemize}
\item {Grp. gram.:m.}
\end{itemize}
\begin{itemize}
\item {Utilização:Gram.}
\end{itemize}
\begin{itemize}
\item {Proveniência:(Gr. \textunderscore asundetos\textunderscore )}
\end{itemize}
Suppressão da conjuncção copulativa entre phrases, ou entre partes de uma phrase.
\section{Asynergia}
\begin{itemize}
\item {fónica:si}
\end{itemize}
\begin{itemize}
\item {Grp. gram.:f.}
\end{itemize}
Falta de energia.
\section{Asystásia}
\begin{itemize}
\item {fónica:sis}
\end{itemize}
\begin{itemize}
\item {Grp. gram.:f.}
\end{itemize}
Gênero de plantas acantháceas.
\section{Asystolia}
\begin{itemize}
\item {fónica:sis}
\end{itemize}
\begin{itemize}
\item {Grp. gram.:f.}
\end{itemize}
\begin{itemize}
\item {Utilização:Med.}
\end{itemize}
\begin{itemize}
\item {Proveniência:(Do gr. \textunderscore a\textunderscore  priv. + \textunderscore sustole\textunderscore )}
\end{itemize}
Insufficiência ou falta da sýstole do coração.
\section{Ata}
\begin{itemize}
\item {Grp. gram.:f.}
\end{itemize}
Planta fructífera do Brasil, (\textunderscore anona squamosa\textunderscore ).
Fruta da ateira.
O mesmo que \textunderscore fruta-de-conde\textunderscore , no Ceará.
\section{Atá}
\begin{itemize}
\item {Grp. gram.:prep.  e  adv.}
\end{itemize}
\begin{itemize}
\item {Utilização:Ant.}
\end{itemize}
O mesmo que \textunderscore até\textunderscore .
\section{Atabacado}
\begin{itemize}
\item {Grp. gram.:adj.}
\end{itemize}
Que é da côr do tabaco.
\section{Atabafadamente}
\begin{itemize}
\item {Grp. gram.:adv.}
\end{itemize}
Ás occultas.
Com suffocação.
\section{Atabafador}
\begin{itemize}
\item {Grp. gram.:m.}
\end{itemize}
Aquelle que atabafa.
\section{Atabafamento}
\begin{itemize}
\item {Grp. gram.:m.}
\end{itemize}
Acção de \textunderscore atabafar\textunderscore .
\section{Atabafar}
\begin{itemize}
\item {Grp. gram.:v. t.}
\end{itemize}
\begin{itemize}
\item {Grp. gram.:V. i.}
\end{itemize}
Abafar. Encobrir.
O mesmo que \textunderscore empalmar\textunderscore . Cf. Castilho, \textunderscore Fausto\textunderscore , 224.
Respirar com difficuldade.
\section{Atabafeia}
\begin{itemize}
\item {Grp. gram.:f.}
\end{itemize}
O mesmo ou melhor que \textunderscore tabafeia\textunderscore .
\section{Atabal}
\begin{itemize}
\item {Grp. gram.:m.}
\end{itemize}
O mesmo que \textunderscore atabale\textunderscore .
\section{Atabalaque}
\begin{itemize}
\item {Grp. gram.:m.}
\end{itemize}
O mesmo que \textunderscore atabale\textunderscore .
\section{Atabale}
\begin{itemize}
\item {Grp. gram.:m.}
\end{itemize}
O mesmo que \textunderscore timbale\textunderscore .
(Cast. \textunderscore atabal\textunderscore )
\section{Atabaleiro}
\begin{itemize}
\item {Grp. gram.:m.}
\end{itemize}
O mesmo que \textunderscore timbaleiro\textunderscore .
\section{Atabalhoadamente}
\begin{itemize}
\item {Grp. gram.:adv.}
\end{itemize}
De modo \textunderscore atabalhoado\textunderscore .
\section{Atabalhoado}
\begin{itemize}
\item {Grp. gram.:adj.}
\end{itemize}
\begin{itemize}
\item {Proveniência:(De \textunderscore atabalhoar\textunderscore )}
\end{itemize}
Despropositado.
Precipitado.
\section{Atabalhoamento}
\begin{itemize}
\item {Grp. gram.:m.}
\end{itemize}
Acção ou effeito de \textunderscore atabalhoar\textunderscore .
\section{Atabalhoar}
\begin{itemize}
\item {Grp. gram.:v. t.}
\end{itemize}
Fazer, dizer, sem ordem nem propósito; atrapalhar.
\section{Atabaque}
\begin{itemize}
\item {Grp. gram.:m.}
\end{itemize}
Instrumento oriental.
Designação ant. do \textunderscore atabale\textunderscore .
\section{Atabaque}
\begin{itemize}
\item {Grp. gram.:m.}
\end{itemize}
Aio de príncipes, em algumas côrtes orientaes.
\section{Atabaqueiro}
\begin{itemize}
\item {Grp. gram.:m.}
\end{itemize}
Tangedor do \textunderscore atabaque\textunderscore ^1.
\section{Atabarda}
\begin{itemize}
\item {Grp. gram.:f.}
\end{itemize}
Certa vestidura antiga.
\section{Atabefe}
\begin{itemize}
\item {Grp. gram.:m.}
\end{itemize}
(V.tabefe)
\section{Atabernadamente}
\begin{itemize}
\item {Grp. gram.:adv.}
\end{itemize}
Á maneira de taberna ou de taberneiro.
\section{Atabernado}
\begin{itemize}
\item {Grp. gram.:adj.}
\end{itemize}
Que tem aspecto de taberna.
\section{Atabernar}
\begin{itemize}
\item {Grp. gram.:v. t.}
\end{itemize}
Vender em taberna.
Vender por miúdo.
Converter em taberna.
Tornar grosseiro.
\section{Atabuar}
\begin{itemize}
\item {Grp. gram.:v. t.}
\end{itemize}
\begin{itemize}
\item {Utilização:Prov.}
\end{itemize}
\begin{itemize}
\item {Proveniência:(De \textunderscore tábua\textunderscore )}
\end{itemize}
Empanzinar; empaturrar.
\section{Atabucar}
\begin{itemize}
\item {Grp. gram.:v. t.}
\end{itemize}
\begin{itemize}
\item {Utilização:Ant.}
\end{itemize}
Alliciar.
Enganar.
\section{Atabular}
\begin{itemize}
\item {Grp. gram.:v. t.}
\end{itemize}
\begin{itemize}
\item {Utilização:Bras}
\end{itemize}
\begin{itemize}
\item {Grp. gram.:V. i.}
\end{itemize}
\begin{itemize}
\item {Utilização:Prov.}
\end{itemize}
Estugar; apressar.
Questionar em voz alta; falar á tôa, alanzoar.
\section{Atada}
\begin{itemize}
\item {Grp. gram.:f.}
\end{itemize}
\begin{itemize}
\item {Proveniência:(De \textunderscore atacar\textunderscore )}
\end{itemize}
Atacador; cordão ou correia, com que se ataca ou aperta uma peça do vestuário.
\section{Atacadas}
\begin{itemize}
\item {Grp. gram.:f. pl.}
\end{itemize}
Barrotes, que se pregam no costado do navio, para obrigar a madeira a ir ao seu lugar.
\section{Atacadista}
\begin{itemize}
\item {Grp. gram.:m.}
\end{itemize}
\begin{itemize}
\item {Utilização:Bras}
\end{itemize}
Negociante que vende por atacado.
\section{Atacador}
\begin{itemize}
\item {Grp. gram.:m.}
\end{itemize}
Aquelle ou aquillo que ataca.
Cordão ou correia, com que se ataca uma peça de vestuário.
\section{Atacaniça}
\begin{itemize}
\item {Grp. gram.:f.}
\end{itemize}
O mesmo que \textunderscore tacaniça\textunderscore .
\section{Atacante}
\begin{itemize}
\item {Grp. gram.:adj.}
\end{itemize}
Que ataca.
\section{Atação}
\begin{itemize}
\item {Grp. gram.:f.}
\end{itemize}
Acto de \textunderscore atar\textunderscore  (mólhos, feixes, etc.). Cf. P. Moraes, \textunderscore Econ. Rur.\textunderscore , 257.
\section{Atacar}
\begin{itemize}
\item {Grp. gram.:v. t.}
\end{itemize}
Prender com ataca; apertar: \textunderscore atacar umas botas\textunderscore .
Offender: \textunderscore atacar a honra de alguém\textunderscore .
Acometer; assaltar: \textunderscore atacar um regimento\textunderscore .
Impugnar: \textunderscore atacar um systema\textunderscore .
Carregar (\textunderscore arma de fogo\textunderscore ).
Arregaçar (as calças).
(B. lat. \textunderscore attacare\textunderscore )
\section{Atacoar}
\begin{itemize}
\item {Grp. gram.:v. t.}
\end{itemize}
\begin{itemize}
\item {Proveniência:(De \textunderscore tacão\textunderscore )}
\end{itemize}
Pôr tacões em.
Concertar mal.
\section{Atadeiro}
\begin{itemize}
\item {Grp. gram.:adj.}
\end{itemize}
Que serve para atar: \textunderscore vime atadeiro\textunderscore .
\section{Atadilho}
\begin{itemize}
\item {Grp. gram.:m.}
\end{itemize}
\begin{itemize}
\item {Proveniência:(De \textunderscore atar\textunderscore )}
\end{itemize}
Parte inferior da guitarra, onde estão os botões ou pregos, que seguram as cordas.
\section{Atadinho}
\begin{itemize}
\item {Grp. gram.:adj.}
\end{itemize}
\begin{itemize}
\item {Utilização:Fam.}
\end{itemize}
\begin{itemize}
\item {Proveniência:(De \textunderscore atado\textunderscore )}
\end{itemize}
Acanhado, tímido.
\section{Atado}
\begin{itemize}
\item {Grp. gram.:m.}
\end{itemize}
\begin{itemize}
\item {Grp. gram.:Adj.}
\end{itemize}
\begin{itemize}
\item {Proveniência:(De \textunderscore atar\textunderscore )}
\end{itemize}
Feixe; mólho.
Que não tem desembaraço.
Tímido.
\section{Atador}
\begin{itemize}
\item {Grp. gram.:m.}
\end{itemize}
Aquelle que ata.
\section{Atadura}
\begin{itemize}
\item {Grp. gram.:f.}
\end{itemize}
Aquillo com que se ata.
Ligadura.
Acção de \textunderscore atar\textunderscore .
\section{Ataes}
\begin{itemize}
\item {Grp. gram.:prep.}
\end{itemize}
\begin{itemize}
\item {Utilização:Ant.}
\end{itemize}
O mesmo que \textunderscore atá\textunderscore .
\section{Atafal}
\begin{itemize}
\item {Grp. gram.:m.}
\end{itemize}
\begin{itemize}
\item {Proveniência:(Do ár. \textunderscore at-tafar\textunderscore )}
\end{itemize}
Retranca da cavalgadura.
\section{Atafegar}
\begin{itemize}
\item {Grp. gram.:v. t.}
\end{itemize}
\begin{itemize}
\item {Utilização:Prov.}
\end{itemize}
Suffocar, abafar.
\section{Atafêgo}
\begin{itemize}
\item {Grp. gram.:m.}
\end{itemize}
Acto de \textunderscore atafegar\textunderscore .
Suffocação.
\section{Atafera}
\begin{itemize}
\item {Grp. gram.:f.}
\end{itemize}
\begin{itemize}
\item {Proveniência:(Do ár. \textunderscore tafaha\textunderscore )}
\end{itemize}
Cordão de esparto, de que se fazem asas de seirões.
\section{Atafiinda}
\begin{itemize}
\item {Grp. gram.:f.}
\end{itemize}
\begin{itemize}
\item {Proveniência:(De \textunderscore atar\textunderscore  + \textunderscore findar\textunderscore , seg. Car. Michaëlis)}
\end{itemize}
Antigo artifício de trovadores, que consistia na feitura de cantigas ou estrophes encadeadas com auxílio de um estribilho.
\section{Atafina}
\begin{itemize}
\item {Grp. gram.:f.}
\end{itemize}
\begin{itemize}
\item {Utilização:Prov.}
\end{itemize}
Espécie de gancho, que se abre e se fecha, como os que seguram o relógio.
\section{Atafona}
\begin{itemize}
\item {Grp. gram.:f.}
\end{itemize}
\begin{itemize}
\item {Proveniência:(Do ár. \textunderscore at-tahona\textunderscore )}
\end{itemize}
Moínho manual ou movido por bêstas.
Azenha.
\section{Atafoneiro}
\begin{itemize}
\item {Grp. gram.:m.}
\end{itemize}
Aquelle que tem atafona; aquelle que a dirige.
\section{Atafular-se}
\begin{itemize}
\item {Grp. gram.:v. p.}
\end{itemize}
Tornar-se taful.
\section{Atafulhado}
\begin{itemize}
\item {Grp. gram.:adj.}
\end{itemize}
\begin{itemize}
\item {Proveniência:(De \textunderscore atafulhar\textunderscore )}
\end{itemize}
Muito cheio.
\section{Atafulhamento}
\begin{itemize}
\item {Grp. gram.:m.}
\end{itemize}
Acção de \textunderscore atafulhar\textunderscore .
\section{Atafulhar}
\begin{itemize}
\item {Grp. gram.:v. t.}
\end{itemize}
\begin{itemize}
\item {Utilização:Pop.}
\end{itemize}
Encher muito.
(Corr. de \textunderscore atapulhar\textunderscore )
\section{Ataganhar}
\begin{itemize}
\item {Grp. gram.:v. t.}
\end{itemize}
\begin{itemize}
\item {Utilização:Prov.}
\end{itemize}
\begin{itemize}
\item {Utilização:trasm.}
\end{itemize}
Afogar, apertando a garganta; estrangular.
Tirar a respiração a, obstruindo a larynge.
\section{Atagantar}
\begin{itemize}
\item {Grp. gram.:v. t.}
\end{itemize}
Maltratar com tagante.
Opprimir.
\section{Atagantar}
\begin{itemize}
\item {Grp. gram.:v. t.}
\end{itemize}
\begin{itemize}
\item {Utilização:Prov.}
\end{itemize}
O mesmo que \textunderscore atarantar\textunderscore .
\section{Ataimado}
\begin{itemize}
\item {Grp. gram.:adj.}
\end{itemize}
Fino? ardiloso?:«\textunderscore estes muito ataimados caem em piores atoleiros\textunderscore ». \textunderscore Aulegrafia\textunderscore , 63.
\section{Atal}
\begin{itemize}
\item {Grp. gram.:adv.}
\end{itemize}
\begin{itemize}
\item {Utilização:Ant.}
\end{itemize}
\begin{itemize}
\item {Proveniência:(De \textunderscore a\textunderscore  + \textunderscore tal\textunderscore )}
\end{itemize}
Sob condição; contanto que.
\section{Atalaia}
\begin{itemize}
\item {Grp. gram.:f.}
\end{itemize}
\begin{itemize}
\item {Grp. gram.:M.}
\end{itemize}
\begin{itemize}
\item {Proveniência:(Do ár. \textunderscore at-talia\textunderscore )}
\end{itemize}
Sentinela, vigia.
Ponto elevado, donde se vigia.
Observação; precaução: \textunderscore estar de atalaia\textunderscore .
Homem que vigia.
\section{Atalaiar}
\begin{itemize}
\item {Grp. gram.:v. t.}
\end{itemize}
Pôr atalaias em.
Observar, vigiar.
Pôr de sôbre-aviso.
\section{Atalancar-se}
\begin{itemize}
\item {Grp. gram.:v. p.}
\end{itemize}
\begin{itemize}
\item {Utilização:Prov.}
\end{itemize}
\begin{itemize}
\item {Utilização:trasm.}
\end{itemize}
Endividar-se muito.
(Cp. \textunderscore entalar\textunderscore )
\section{Atalanta}
\begin{itemize}
\item {Grp. gram.:f.}
\end{itemize}
\begin{itemize}
\item {Proveniência:(De \textunderscore Atalanta\textunderscore , n. p. myth.)}
\end{itemize}
Espécie de borboleta.
Pequeno planeta, descoberto há poucos annos, entre Marte e Júpiter.
\section{Atalante}
\begin{itemize}
\item {Grp. gram.:m.}
\end{itemize}
\begin{itemize}
\item {Utilização:Prov.}
\end{itemize}
\begin{itemize}
\item {Utilização:trasm.}
\end{itemize}
\begin{itemize}
\item {Proveniência:(De \textunderscore talante\textunderscore )}
\end{itemize}
Desejo intempestivo de possuir alguma coisa.
\section{Atalântia}
\begin{itemize}
\item {Grp. gram.:f.}
\end{itemize}
Gênero de plantas rutáceas.
\section{Atalhada}
\begin{itemize}
\item {Grp. gram.:f.}
\end{itemize}
\begin{itemize}
\item {Proveniência:(De \textunderscore atalhar\textunderscore )}
\end{itemize}
Córte, que se faz nas matas, para evitar propagação de incêndio; aceiro; sesmo.
\section{Atalhador}
\begin{itemize}
\item {Grp. gram.:m.}
\end{itemize}
Aquelle que atalha.
\section{Atalhamento}
\begin{itemize}
\item {Grp. gram.:m.}
\end{itemize}
Acção de \textunderscore atalhar\textunderscore .
\section{Atalhar}
\begin{itemize}
\item {Grp. gram.:v. t.}
\end{itemize}
\begin{itemize}
\item {Grp. gram.:V. i.}
\end{itemize}
\begin{itemize}
\item {Utilização:Des.}
\end{itemize}
Impedir.
Interromper: \textunderscore atalhar um incêndio\textunderscore .
Obviar a.
Estreitar, resumir.
Ficar confuso, embaraçado. Cf. Brito, \textunderscore Elogios dos Reis\textunderscore .
(Cp. \textunderscore talhar\textunderscore )
\section{Atalho}
\begin{itemize}
\item {Grp. gram.:m.}
\end{itemize}
\begin{itemize}
\item {Proveniência:(De \textunderscore atalhar\textunderscore )}
\end{itemize}
Caminho estreito, que, por fóra de estrada commum, encurta as distâncias.
Embaraçoso, estôrvo.
Obra de madeira, com que se reduz a área de uma praça, para concentrar a defesa.
Remate; córte. Cf. Filinto, VII, 251.
\section{Atalocha}
\begin{itemize}
\item {Grp. gram.:f.}
\end{itemize}
\begin{itemize}
\item {Utilização:Prov.}
\end{itemize}
\begin{itemize}
\item {Utilização:alent.}
\end{itemize}
Utensílio, com que os estucadores estendem a massa nas paredes.
\section{Atamado}
\begin{itemize}
\item {Grp. gram.:adj.}
\end{itemize}
\begin{itemize}
\item {Utilização:Ant.}
\end{itemize}
Dizia-se de certos panos.
(Por \textunderscore atamarado\textunderscore ?)
\section{Atamanca}
\begin{itemize}
\item {Grp. gram.:f.}
\end{itemize}
\begin{itemize}
\item {Utilização:Agr.}
\end{itemize}
\begin{itemize}
\item {Proveniência:(De \textunderscore atamancar\textunderscore )}
\end{itemize}
Mergulhia de varas, deixando de pé a planta mãe.
\section{Atamancador}
\begin{itemize}
\item {Grp. gram.:m.}
\end{itemize}
Aquelle que atamanca.
\section{Atamancar}
\begin{itemize}
\item {Grp. gram.:v. t.}
\end{itemize}
\begin{itemize}
\item {Proveniência:(De \textunderscore tamanco\textunderscore )}
\end{itemize}
Concertar grosseiramente; realizar mal.
\section{Atamarado}
\begin{itemize}
\item {Grp. gram.:adj.}
\end{itemize}
Que tem côr de tâmara.
\section{Atambor}
\begin{itemize}
\item {Grp. gram.:m.}
\end{itemize}
\begin{itemize}
\item {Proveniência:(Do ár. \textunderscore at-tambul\textunderscore )}
\end{itemize}
O mesmo que \textunderscore betle\textunderscore . Cf. \textunderscore Roteiro de V. da Gama\textunderscore .
\section{Atambor}
\begin{itemize}
\item {Grp. gram.:m.}
\end{itemize}
O mesmo que \textunderscore tambor\textunderscore .
\section{Atamento}
\begin{itemize}
\item {Grp. gram.:m.}
\end{itemize}
\begin{itemize}
\item {Utilização:Fam.}
\end{itemize}
\begin{itemize}
\item {Utilização:Ant.}
\end{itemize}
Acanhamento, timidez.
Acção de \textunderscore atar\textunderscore .
\section{Atamísquea}
\begin{itemize}
\item {Grp. gram.:f.}
\end{itemize}
Gênero de plantas capparidáceas.
\section{Atanado}
\begin{itemize}
\item {Grp. gram.:m.}
\end{itemize}
\begin{itemize}
\item {Proveniência:(De \textunderscore atanar\textunderscore )}
\end{itemize}
Casca de árvores, que, reduzida a pó, serve no curtimento dos coiros.
Coiro atanado.
\section{Atanajura}
\begin{itemize}
\item {Grp. gram.:f.}
\end{itemize}
Grande formiga, vulgar no Brasil.
\section{Atanar}
\begin{itemize}
\item {Grp. gram.:v. t.}
\end{itemize}
\begin{itemize}
\item {Proveniência:(Do al. \textunderscore tann\textunderscore )}
\end{itemize}
Curtir com casca de carvalho ou de algumas outras árvores.
\section{Atanazar}
\begin{itemize}
\item {Grp. gram.:v. t.}
\end{itemize}
O mesmo que \textunderscore atenazar\textunderscore . Cf. \textunderscore Peregrinação\textunderscore , XXIII.
\section{Atanchar}
\textunderscore v. t.\textunderscore  (e der.)
O mesmo que \textunderscore tanchar\textunderscore , etc.
\section{Atanoado}
\begin{itemize}
\item {Grp. gram.:m.  e  adj.}
\end{itemize}
\begin{itemize}
\item {Utilização:Prov.}
\end{itemize}
\begin{itemize}
\item {Utilização:beir.}
\end{itemize}
O mesmo que \textunderscore atanado\textunderscore .
\section{Atapetar}
\begin{itemize}
\item {Grp. gram.:v. t.}
\end{itemize}
Cobrir com tapête.
Cobrir, á maneira de tapête: \textunderscore atapetar de flôres\textunderscore .
\section{Atapu}
\begin{itemize}
\item {Grp. gram.:m.}
\end{itemize}
\begin{itemize}
\item {Utilização:Bras}
\end{itemize}
O mesmo que \textunderscore uatapu\textunderscore .
\section{Atapulhar}
\begin{itemize}
\item {Grp. gram.:v. t.}
\end{itemize}
Meter tapulho em.
Tapar.
Encher muito.
\section{Ataque}
\begin{itemize}
\item {Grp. gram.:m.}
\end{itemize}
Acçao de \textunderscore atacar\textunderscore .
Investida; assalto.
Injúria.
Accusação.
Discussão.
Incómmodo súbito: \textunderscore deu-lhe um ataque\textunderscore .
\section{Ataqueiras}
\begin{itemize}
\item {Grp. gram.:f. pl.}
\end{itemize}
\begin{itemize}
\item {Utilização:Pop.}
\end{itemize}
\begin{itemize}
\item {Proveniência:(De \textunderscore atacar\textunderscore )}
\end{itemize}
Difficuldades, embaraços, talas.
\section{Ataqueiro}
\begin{itemize}
\item {Grp. gram.:m.}
\end{itemize}
Vendedor ou fabricante de atacas.
\section{Atar}
\begin{itemize}
\item {Grp. gram.:v. t.}
\end{itemize}
\begin{itemize}
\item {Proveniência:(Lat. \textunderscore aptare\textunderscore )}
\end{itemize}
Ligar, cingir com cordão, fita, etc., apertando.
Estreitar.
Impedir: \textunderscore a compaixão atou-lhe os braços\textunderscore .
Atrelar.
Submeter.
Expor com coherência e nexo.
\section{Atarabebê}
\begin{itemize}
\item {Grp. gram.:m.}
\end{itemize}
\begin{itemize}
\item {Utilização:Bras}
\end{itemize}
Ornato e borla de penas, na extremidade da tangapena.
(Do tupi \textunderscore atar\textunderscore , ornato, e \textunderscore bebê\textunderscore , que vôa)
\section{Atarahu}
\begin{itemize}
\item {Grp. gram.:m.}
\end{itemize}
\begin{itemize}
\item {Utilização:Bras}
\end{itemize}
Furor; cólera.
\section{Ataranhado}
\begin{itemize}
\item {Grp. gram.:adj.}
\end{itemize}
\begin{itemize}
\item {Utilização:Prov.}
\end{itemize}
\begin{itemize}
\item {Utilização:beir.}
\end{itemize}
Atrapalhado, atarantado.
\section{Atarantação}
\begin{itemize}
\item {Grp. gram.:f.}
\end{itemize}
Acto de \textunderscore atarantar\textunderscore .
\section{Atarantar}
\begin{itemize}
\item {Grp. gram.:v. t.}
\end{itemize}
\begin{itemize}
\item {Utilização:Pop.}
\end{itemize}
Estontear; confundir; atrapalhar.
Perturbar.
(Cast. \textunderscore atarantar\textunderscore )
\section{Ataranto}
\begin{itemize}
\item {Grp. gram.:m.}
\end{itemize}
(V.atarantação)
\section{Ataraú}
\begin{itemize}
\item {Grp. gram.:m.}
\end{itemize}
\begin{itemize}
\item {Utilização:Bras}
\end{itemize}
Furor; cólera.
\section{Ataraxia}
\begin{itemize}
\item {fónica:csi}
\end{itemize}
\begin{itemize}
\item {Grp. gram.:f.}
\end{itemize}
\begin{itemize}
\item {Proveniência:(Gr. \textunderscore ataraxia\textunderscore )}
\end{itemize}
Ausência de perturbação; tranquillidade de espírito.
\section{Atardar}
\begin{itemize}
\item {Grp. gram.:v. t.}
\end{itemize}
\begin{itemize}
\item {Proveniência:(De \textunderscore tarde\textunderscore )}
\end{itemize}
Demorar, atrasar. Cf. Filinto, X, 261.
\section{Atarefado}
\begin{itemize}
\item {Grp. gram.:adj.}
\end{itemize}
Occupado em tarefa.
Azafamado.
\section{Atarefar}
\begin{itemize}
\item {Grp. gram.:v. t.}
\end{itemize}
Dar tarefa a.
Sobrecarregar de trabalho.
\section{Ataroucado}
\begin{itemize}
\item {Grp. gram.:adj.}
\end{itemize}
\begin{itemize}
\item {Proveniência:(De \textunderscore ataroucar\textunderscore )}
\end{itemize}
Adoidado, apalermado.
\section{Ataroucar}
\begin{itemize}
\item {Grp. gram.:v. t.}
\end{itemize}
Tornar idiota.
Fazer tolamente:«\textunderscore fez um soneto chapado! assim os ataroucava eu...\textunderscore »Filinto, VIII, 49.
\section{Atarracado}
\begin{itemize}
\item {Grp. gram.:adj.}
\end{itemize}
Baixo e grosso.
Achaparrado.
\section{Atarracar}
\begin{itemize}
\item {Grp. gram.:v. t.}
\end{itemize}
\begin{itemize}
\item {Proveniência:(Do ár. \textunderscore at-taraca\textunderscore )}
\end{itemize}
Preparar (ferradura), apertando-a com martelo.
Apertar.
Confundir, perturbar.
\section{Atarrachador}
\begin{itemize}
\item {Grp. gram.:m.}
\end{itemize}
Instrumento, para \textunderscore atarrachar\textunderscore .
\section{Atarrachar}
\begin{itemize}
\item {Grp. gram.:v. t.}
\end{itemize}
Apertar com tarracha.
\section{Atarrafado}
\begin{itemize}
\item {Grp. gram.:adj.}
\end{itemize}
\begin{itemize}
\item {Utilização:Pop.}
\end{itemize}
Coberto com tarrafa.
Que tem capa ou manta esburacada.
\section{Atartarugado}
\begin{itemize}
\item {Grp. gram.:adj.}
\end{itemize}
Que tem côr de tartaruga.
\section{Atascadeiro}
\begin{itemize}
\item {Grp. gram.:m.}
\end{itemize}
Lamaçal, atoleiro.
(Cast. \textunderscore atascadero\textunderscore )
\section{Atascal}
\begin{itemize}
\item {Grp. gram.:m.}
\end{itemize}
\begin{itemize}
\item {Utilização:bras}
\end{itemize}
\begin{itemize}
\item {Utilização:Neol.}
\end{itemize}
O mesmo que \textunderscore atoleiro\textunderscore .
(Cp. \textunderscore atascar\textunderscore )
\section{Atascar}
\begin{itemize}
\item {Grp. gram.:v. t.}
\end{itemize}
Meter em atoleiro.
(Cast. \textunderscore atascar\textunderscore )
\section{Atasqueiro}
\begin{itemize}
\item {Grp. gram.:m.}
\end{itemize}
(Contr. de \textunderscore atascadeiro\textunderscore )
\section{Atassalhador}
\begin{itemize}
\item {Grp. gram.:m.}
\end{itemize}
Aquelle que atassalha.
\section{Atassalhadura}
\begin{itemize}
\item {Grp. gram.:f.}
\end{itemize}
Acção de \textunderscore atassalhar\textunderscore .
\section{Atassalhar}
\begin{itemize}
\item {Grp. gram.:v. t.}
\end{itemize}
Cortar em tassalhos.
Espedaçar.
Rasgar.
Retalhar.
Derrotar.
Desacreditar: \textunderscore atassalhar a fama de alguém\textunderscore .
\section{Atassim}
\begin{itemize}
\item {Grp. gram.:m.}
\end{itemize}
\begin{itemize}
\item {Proveniência:(De \textunderscore atar\textunderscore  + ?)}
\end{itemize}
Fio, que cose as malhas da rede á tralha.
\section{Atataranhar-se}
\begin{itemize}
\item {Grp. gram.:v. p.}
\end{itemize}
Tornar-se momentânea e accidentalmente tataranha.
(Colhido em Turquel)
\section{Ataúde}
\begin{itemize}
\item {Grp. gram.:m.}
\end{itemize}
\begin{itemize}
\item {Proveniência:(Do ár. \textunderscore al-tabute\textunderscore )}
\end{itemize}
Caixão funerário.
Féretro: tumba.
Sepulcro.
\section{Atauxiar}
\begin{itemize}
\item {Grp. gram.:v. t.}
\end{itemize}
(V.tauxiar)
\section{Atavanado}
\begin{itemize}
\item {Grp. gram.:adj.}
\end{itemize}
\begin{itemize}
\item {Proveniência:(De \textunderscore tavão\textunderscore )}
\end{itemize}
Diz-se do cavallo preto ou escuro, com pintas nas ancas ou nas espáduas.
\section{Atavernar}
\textunderscore v. t.\textunderscore  (e der.)
O mesmo que \textunderscore atabernar\textunderscore , etc.
\section{Ataviador}
\begin{itemize}
\item {Grp. gram.:m.}
\end{itemize}
Aquelle que atavia.
\section{Ataviamento}
\begin{itemize}
\item {Grp. gram.:m.}
\end{itemize}
Acção de \textunderscore ataviar\textunderscore .
\section{Ataviar}
\begin{itemize}
\item {Grp. gram.:v. t.}
\end{itemize}
\begin{itemize}
\item {Proveniência:(Do rad. angl. sax. \textunderscore tawian\textunderscore , arrotear)}
\end{itemize}
Enfeitar, adornar.
Aformosear.
\section{Atávico}
\begin{itemize}
\item {Grp. gram.:adj.}
\end{itemize}
\begin{itemize}
\item {Proveniência:(Do lat. \textunderscore atavus\textunderscore )}
\end{itemize}
Produzido por atavismo: \textunderscore vício atávico\textunderscore .
\section{Atavio}
\begin{itemize}
\item {Grp. gram.:m.}
\end{itemize}
\begin{itemize}
\item {Proveniência:(De \textunderscore ataviar\textunderscore )}
\end{itemize}
Enfeite, adôrno.
Apparelho.
\section{Atavismo}
\begin{itemize}
\item {Grp. gram.:m.}
\end{itemize}
Propriedade, que têm os animaes e os vegetaes, de trasm.ttir caracteres seus aos descendentes, com intervallo de uma ou mais gerações.
Semelhança com os avós.
(Cp. \textunderscore atávico\textunderscore )
\section{Atavolado}
\begin{itemize}
\item {Grp. gram.:m.}
\end{itemize}
\begin{itemize}
\item {Utilização:Ant.}
\end{itemize}
\begin{itemize}
\item {Proveniência:(De \textunderscore távola\textunderscore )}
\end{itemize}
Jôgo de mesa.
\section{Atavonado}
\begin{itemize}
\item {Grp. gram.:adj.}
\end{itemize}
(V.atavanado)
\section{Ataxia}
\begin{itemize}
\item {fónica:csi}
\end{itemize}
\begin{itemize}
\item {Grp. gram.:f.}
\end{itemize}
\begin{itemize}
\item {Utilização:Med.}
\end{itemize}
\begin{itemize}
\item {Proveniência:(Gr. \textunderscore ataxia\textunderscore )}
\end{itemize}
Irregularidade, desordem, falta de coordenação, nos movimentos voluntários, contrastando com a integridade da fôrça muscular.
\section{Atáxico}
\begin{itemize}
\item {fónica:csi}
\end{itemize}
\begin{itemize}
\item {Grp. gram.:adj.}
\end{itemize}
Em que há \textunderscore ataxia\textunderscore .
\section{Atazanar}
\begin{itemize}
\item {Grp. gram.:v. t.}
\end{itemize}
Espicaçar.
Estimular.
Importunar.
(Do ár.?)
\section{Ate}
\begin{itemize}
\item {Grp. gram.:m.}
\end{itemize}
Gênero de orchídeas.
\section{Até}
\begin{itemize}
\item {Grp. gram.:prep.}
\end{itemize}
\begin{itemize}
\item {Grp. gram.:Adv.}
\end{itemize}
\begin{itemize}
\item {Proveniência:(Do lat. \textunderscore hactenus\textunderscore )}
\end{itemize}
(indicativa de um termo no espaço, no tempo, nas acções, etc.: \textunderscore ir até o Pôrto\textunderscore ; \textunderscore soffreu até hoje\textunderscore ; \textunderscore chegar até á última degradação\textunderscore )
Ainda.
Também: \textunderscore até os adversários o admiram\textunderscore .
\section{Ateador}
\begin{itemize}
\item {Grp. gram.:m.}
\end{itemize}
Aquelle que ateia.
\section{Atear}
\begin{itemize}
\item {Grp. gram.:v. t.}
\end{itemize}
\begin{itemize}
\item {Proveniência:(Do lat. \textunderscore teda\textunderscore )}
\end{itemize}
Lançar fogo a.
Avivar (fogo).
Incendiar.
Excitar; fomentar: \textunderscore atear discórdias\textunderscore .
\section{Atechnia}
\begin{itemize}
\item {Grp. gram.:f.}
\end{itemize}
\begin{itemize}
\item {Proveniência:(Do gr. \textunderscore a\textunderscore  priv. + \textunderscore tekhne\textunderscore )}
\end{itemize}
Ausência de árte.
\section{Atecnia}
\begin{itemize}
\item {Grp. gram.:f.}
\end{itemize}
\begin{itemize}
\item {Proveniência:(Do gr. \textunderscore a\textunderscore  priv. + \textunderscore tekhne\textunderscore )}
\end{itemize}
Ausência de árte.
\section{Atedágua}
\begin{itemize}
\item {Grp. gram.:f.}
\end{itemize}
Espécie de rosmaninho, que cheira a incenso.
\section{Atediar}
\begin{itemize}
\item {Grp. gram.:v. t.}
\end{itemize}
O mesmo que \textunderscore entediar\textunderscore .
\section{Ateigamento}
\begin{itemize}
\item {Grp. gram.:m.}
\end{itemize}
Acto de \textunderscore ateigar\textunderscore .
\section{Ateigar}
\begin{itemize}
\item {Grp. gram.:v. t.}
\end{itemize}
\begin{itemize}
\item {Utilização:Ant.}
\end{itemize}
\begin{itemize}
\item {Proveniência:(De \textunderscore teiga\textunderscore )}
\end{itemize}
Medir com teiga.
Avaliar a ôlho.
Encher muito.
\section{Ateimar}
\begin{itemize}
\item {Grp. gram.:v. i.}
\end{itemize}
(V.teimar)
\section{Ateira}
\begin{itemize}
\item {Grp. gram.:f.}
\end{itemize}
\begin{itemize}
\item {Proveniência:(De \textunderscore ata\textunderscore )}
\end{itemize}
Árvore anonácea da América e da Índia, (\textunderscore anona squamosa\textunderscore , Lin.).
\section{Ateiran}
\begin{itemize}
\item {Grp. gram.:f.}
\end{itemize}
Árvore indiana.
\section{Ateiró}
\begin{itemize}
\item {Grp. gram.:m.}
\end{itemize}
(V.teiró)
\section{Atela}
\begin{itemize}
\item {Grp. gram.:f.}
\end{itemize}
Gênero de insectos coleópteros pentâmeros.
\section{Átele}
\begin{itemize}
\item {Grp. gram.:m.}
\end{itemize}
Espécie de macaco do Brasil.
\section{Atelectasia}
\begin{itemize}
\item {Grp. gram.:f.}
\end{itemize}
\begin{itemize}
\item {Utilização:Med.}
\end{itemize}
\begin{itemize}
\item {Proveniência:(Do gr. \textunderscore a\textunderscore  priv. + \textunderscore tele\textunderscore  + \textunderscore ektasis\textunderscore )}
\end{itemize}
Falta de dilatação.
Distensão incompleta dos pulmões nos recém-nascidos.
Endurecimento vermelho nos pulmões das crianças.
\section{Atelépode}
\begin{itemize}
\item {Grp. gram.:adj.}
\end{itemize}
\begin{itemize}
\item {Utilização:Zool.}
\end{itemize}
\begin{itemize}
\item {Proveniência:(Do gr. \textunderscore a\textunderscore  priv. + \textunderscore tele\textunderscore  + \textunderscore pous\textunderscore , \textunderscore podos\textunderscore )}
\end{itemize}
A que falta o dedo pollegar ou outro dedo.
\section{Atálamo}
\begin{itemize}
\item {Grp. gram.:adj.}
\end{itemize}
\begin{itemize}
\item {Utilização:Bot.}
\end{itemize}
\begin{itemize}
\item {Proveniência:(Do gr. \textunderscore a\textunderscore  priv. + \textunderscore thalamos\textunderscore )}
\end{itemize}
Diz-se dos lichens, que não têm conceptáculos.
\section{Atamanta}
\begin{itemize}
\item {Grp. gram.:f.}
\end{itemize}
\begin{itemize}
\item {Proveniência:(De \textunderscore Athamas\textunderscore , n. p.)}
\end{itemize}
Gênero de plantas umbellíferas.
\section{Atamantina}
\begin{itemize}
\item {Grp. gram.:f.}
\end{itemize}
Substância, encontrada na raiz das sementes quási maduras da atamanta.
\section{Ateísmo}
\begin{itemize}
\item {Grp. gram.:m.}
\end{itemize}
Doutrina dos ateus.
\section{Ateísta}
\begin{itemize}
\item {Grp. gram.:m.}
\end{itemize}
(V.ateu)
\section{Ateístico}
\begin{itemize}
\item {Grp. gram.:adj.}
\end{itemize}
Relativo aos ateístas.
\section{Atelanas}
\begin{itemize}
\item {Grp. gram.:f. pl.}
\end{itemize}
\begin{itemize}
\item {Proveniência:(De \textunderscore Atella\textunderscore , n. p.)}
\end{itemize}
Farças populares, usadas entre os antigos Romanos.
\section{Atelhamento}
\begin{itemize}
\item {Grp. gram.:m.}
\end{itemize}
Acto de \textunderscore atelhar\textunderscore .
\section{Atelhar}
\begin{itemize}
\item {Grp. gram.:v. t.}
\end{itemize}
\begin{itemize}
\item {Utilização:Bras. do N}
\end{itemize}
Cobrir de telhas.
\section{Atélia}
\begin{itemize}
\item {Grp. gram.:f.}
\end{itemize}
\begin{itemize}
\item {Utilização:Med.}
\end{itemize}
Ausência de mamillas.
\section{Atelinas}
\begin{itemize}
\item {Grp. gram.:f. pl.}
\end{itemize}
Última classe de vegetaes, segundo Link.
\section{Atellanas}
\begin{itemize}
\item {Grp. gram.:f. pl.}
\end{itemize}
\begin{itemize}
\item {Proveniência:(De \textunderscore Atella\textunderscore , n. p.)}
\end{itemize}
Farças populares, usadas entre os antigos Romanos.
\section{Atelócero}
\begin{itemize}
\item {Grp. gram.:m.}
\end{itemize}
\begin{itemize}
\item {Proveniência:(Do gr. \textunderscore ateles\textunderscore  + \textunderscore keras\textunderscore )}
\end{itemize}
Insecto hemíptero, originário do Senegal.
\section{Atém}
\begin{itemize}
\item {Grp. gram.:adv.}
\end{itemize}
\begin{itemize}
\item {Utilização:Ant.}
\end{itemize}
O mesmo que \textunderscore até\textunderscore .
\section{Atemorizadamente}
\begin{itemize}
\item {Grp. gram.:adv.}
\end{itemize}
Com temor.
\section{Atemorizador}
\begin{itemize}
\item {Grp. gram.:m.  e  adj.}
\end{itemize}
O que atemoriza.
\section{Atemorizamento}
\begin{itemize}
\item {Grp. gram.:m.}
\end{itemize}
Acção de \textunderscore atemorizar\textunderscore .
\section{Atemorizar}
\begin{itemize}
\item {Grp. gram.:v. t.}
\end{itemize}
Causar temor a.
Aterrar.
Espavorir.
\section{Atempação}
\begin{itemize}
\item {Grp. gram.:f.}
\end{itemize}
Acção de \textunderscore atempar\textunderscore .
\section{Atempadamente}
\begin{itemize}
\item {Grp. gram.:adv.}
\end{itemize}
Com \textunderscore atempação\textunderscore .
\section{Atempado}
\begin{itemize}
\item {Grp. gram.:adj.}
\end{itemize}
\begin{itemize}
\item {Utilização:Vit.}
\end{itemize}
Diz-se das varas da vinha, que vingaram e se desenvolveram.
\section{Atempar}
\begin{itemize}
\item {Grp. gram.:v. t.}
\end{itemize}
\begin{itemize}
\item {Proveniência:(De \textunderscore tempo\textunderscore )}
\end{itemize}
Marcar prazo a.
\section{Atemperar}
\begin{itemize}
\item {Grp. gram.:v. t.}
\end{itemize}
(V.temperar)
\section{Atemperar}
\begin{itemize}
\item {Grp. gram.:v. i.  e  p.}
\end{itemize}
Atempar; contemporizar:«\textunderscore como a graça do Senhor se atempera e acommoda ao nosso modo...\textunderscore ». \textunderscore Luz e Calor\textunderscore , 400.
\section{A-tempo}
\begin{itemize}
\item {Grp. gram.:loc. adv.}
\end{itemize}
\begin{itemize}
\item {Grp. gram.:M.}
\end{itemize}
Opportunamente.
Em bôa hora.
Opportunidade, ensejo.
\section{Atenazar}
\begin{itemize}
\item {Grp. gram.:v. t.}
\end{itemize}
O mesmo que \textunderscore atazanar\textunderscore .
\section{Atença}
\begin{itemize}
\item {Grp. gram.:f.}
\end{itemize}
Confiança.
Acção de \textunderscore ater-se\textunderscore . Cf. Filinto, XII, 254.
\section{Atendar}
\begin{itemize}
\item {Grp. gram.:v. i.}
\end{itemize}
Levantar tenda; acampar.
\section{Ateneias}
\begin{itemize}
\item {Grp. gram.:f. pl.}
\end{itemize}
Festas gregas, em honra de Minerva. Cf. Castilho. \textunderscore Fastos\textunderscore , I,
543.
\section{Ateneu}
\begin{itemize}
\item {Grp. gram.:m.}
\end{itemize}
\begin{itemize}
\item {Proveniência:(Do lat. \textunderscore athenaeum\textunderscore )}
\end{itemize}
Lugar público, em que os literatos gregos liam as suas obras.
Estabelecimento de instrucção, não official.
Academia.
\section{Ateniense}
\begin{itemize}
\item {Grp. gram.:adj.}
\end{itemize}
\begin{itemize}
\item {Grp. gram.:M.}
\end{itemize}
\begin{itemize}
\item {Proveniência:(Lat. \textunderscore atheniensis\textunderscore )}
\end{itemize}
Relativo a Atenas.
Habitante de Atenas.
\section{Atenrar}
\begin{itemize}
\item {Grp. gram.:v. t.}
\end{itemize}
Tornar tenro: \textunderscore a cozedura atenra a carne\textunderscore .
\section{Atens}
\begin{itemize}
\item {Grp. gram.:adv.}
\end{itemize}
\begin{itemize}
\item {Utilização:Ant.}
\end{itemize}
O mesmo que \textunderscore até\textunderscore .
\section{Atequipera}
\begin{itemize}
\item {fónica:pê}
\end{itemize}
\begin{itemize}
\item {Grp. gram.:f.}
\end{itemize}
Variedade de pêra, também conhecida por \textunderscore fidalga\textunderscore . Cf. Carvalho da Costa, \textunderscore Chor. Port.\textunderscore , I, 425; Bluteau, vb. \textunderscore atequi-pera\textunderscore .
(Se não é palavra composta de \textunderscore até\textunderscore  + \textunderscore aqui\textunderscore  + \textunderscore pêra\textunderscore , póde têr vindo do árabe. Cf. Dozy)
\section{Aterandra}
\begin{itemize}
\item {Grp. gram.:f.}
\end{itemize}
Gênero de plantas asclepiádeas.
\section{Atérica}
\begin{itemize}
\item {Grp. gram.:f.}
\end{itemize}
Insecto lepidóptero diurno.
\section{Aterina}
\begin{itemize}
\item {Grp. gram.:f.}
\end{itemize}
\begin{itemize}
\item {Proveniência:(Do gr. \textunderscore athera\textunderscore )}
\end{itemize}
Nome scientífico do peixe-rei.
\section{Aterlondar}
\begin{itemize}
\item {Grp. gram.:v. t.}
\end{itemize}
\begin{itemize}
\item {Utilização:Prov.}
\end{itemize}
\begin{itemize}
\item {Utilização:trasm.}
\end{itemize}
\begin{itemize}
\item {Proveniência:(Do cast. \textunderscore atolondrar\textunderscore )}
\end{itemize}
O mesmo que \textunderscore atordoar\textunderscore .
\section{Atermal}
\begin{itemize}
\item {Grp. gram.:adj.}
\end{itemize}
Diz-se das águas mineraes frias.
O mesmo que \textunderscore atérmico\textunderscore .
\section{Atermaneidade}
\begin{itemize}
\item {Grp. gram.:f.}
\end{itemize}
Carácter dos corpos atérmanos.
\section{Atérmano}
\begin{itemize}
\item {Grp. gram.:adj.}
\end{itemize}
\begin{itemize}
\item {Proveniência:(Do gr. \textunderscore a\textunderscore  priv. + \textunderscore therme\textunderscore )}
\end{itemize}
Diz-se dos corpos que são impenetráveis ao calor.
\section{Atermar}
\begin{itemize}
\item {Grp. gram.:v. t.}
\end{itemize}
\begin{itemize}
\item {Proveniência:(De \textunderscore termo\textunderscore )}
\end{itemize}
O mesmo que \textunderscore atempar\textunderscore :«\textunderscore empresas atermadas não podem ser gostosas\textunderscore ». \textunderscore Aulegrafia\textunderscore , 131.
\section{Atérmico}
\begin{itemize}
\item {Grp. gram.:adj.}
\end{itemize}
\begin{itemize}
\item {Utilização:Phýs.}
\end{itemize}
\begin{itemize}
\item {Proveniência:(Do gr. \textunderscore a\textunderscore  priv. + \textunderscore therme\textunderscore )}
\end{itemize}
Que se não deixa atravessar pelo calor.
\section{Ateroma}
\begin{itemize}
\item {Grp. gram.:m.}
\end{itemize}
\begin{itemize}
\item {Utilização:Med.}
\end{itemize}
\begin{itemize}
\item {Proveniência:(Gr. \textunderscore astheroma\textunderscore )}
\end{itemize}
Nome, que se dava ao cisto sebáceo, formado no pescoço ou nas artérias do pescoço.
\section{Ateromatoso}
\begin{itemize}
\item {Grp. gram.:adj.}
\end{itemize}
Que soffre ateroma.
\section{Aterrado}
\begin{itemize}
\item {Grp. gram.:m.}
\end{itemize}
\begin{itemize}
\item {Utilização:Bras}
\end{itemize}
\begin{itemize}
\item {Proveniência:(De \textunderscore aterrar\textunderscore ^2)}
\end{itemize}
Lugar, que se aterrou.
\section{Aterrador}
\begin{itemize}
\item {Grp. gram.:adj.}
\end{itemize}
Que causa terror.
\section{Aterraplenar}
\begin{itemize}
\item {Grp. gram.:v. t.}
\end{itemize}
(V.terraplenar)
\section{Aterrar}
\begin{itemize}
\item {Grp. gram.:v. t.}
\end{itemize}
\begin{itemize}
\item {Grp. gram.:V. i.}
\end{itemize}
\begin{itemize}
\item {Proveniência:(Do lat. \textunderscore terrere\textunderscore )}
\end{itemize}
Causar terror a; atemorizar.
Fazer medo a.
\section{Aterrar}
\begin{itemize}
\item {Grp. gram.:v. t.}
\end{itemize}
\begin{itemize}
\item {Utilização:Ant.}
\end{itemize}
\begin{itemize}
\item {Utilização:Neol.}
\end{itemize}
Cobrir com terra.
Altear com terra.
Arrasar.
Descer á terra (um aeroplano): \textunderscore o aeroplano aterrou sem incidente desagradável\textunderscore .
\section{Atêrro}
\begin{itemize}
\item {Grp. gram.:m.}
\end{itemize}
Acção ou effeito de \textunderscore aterrar\textunderscore ^2.
Porção de terra ou de entulho, destinada a nivelar ou altear um terreno.
\section{Aterroar}
\begin{itemize}
\item {Grp. gram.:v. t.}
\end{itemize}
\begin{itemize}
\item {Utilização:Prov.}
\end{itemize}
\begin{itemize}
\item {Utilização:alg.}
\end{itemize}
\begin{itemize}
\item {Proveniência:(De \textunderscore terrão\textunderscore =\textunderscore torrão\textunderscore )}
\end{itemize}
Cobrir com terra (moreias de mato).
\section{Aterrorar}
\begin{itemize}
\item {Grp. gram.:v. t.}
\end{itemize}
\begin{itemize}
\item {Utilização:Prov.}
\end{itemize}
O mesmo que \textunderscore aterrorizar\textunderscore .
\section{Aterrorizador}
\begin{itemize}
\item {Grp. gram.:adj.}
\end{itemize}
Que aterroriza.
\section{Aterrorizar}
\begin{itemize}
\item {Grp. gram.:v. t.}
\end{itemize}
Causar terror a.
\section{Ater-se}
\begin{itemize}
\item {Grp. gram.:v. p.}
\end{itemize}
\begin{itemize}
\item {Proveniência:(De \textunderscore a\textunderscore  + \textunderscore ter\textunderscore )}
\end{itemize}
Encostar-se, acostar-se: \textunderscore ater-se aos empenhos\textunderscore .
Têr confiança.
\section{Ateruro}
\begin{itemize}
\item {Grp. gram.:m.}
\end{itemize}
Gênero de plantas aroídeas.
\section{Atesar}
\begin{itemize}
\item {Grp. gram.:v. t.  e  i.}
\end{itemize}
(V.entesar)
\section{Atestador}
\begin{itemize}
\item {Grp. gram.:m.}
\end{itemize}
Vasilha, com que se atestam ou se acabam de encher as pipas e tonéis.
\section{Atestamento}
\begin{itemize}
\item {Grp. gram.:m.}
\end{itemize}
Acto de \textunderscore atestar\textunderscore  (vasilhas).
\section{Atestar}
\begin{itemize}
\item {Grp. gram.:v. t.}
\end{itemize}
Encher até ao testo.
Encher completamente.
\section{Atesto}
\begin{itemize}
\item {Grp. gram.:m.}
\end{itemize}
\begin{itemize}
\item {Proveniência:(De \textunderscore atestar\textunderscore )}
\end{itemize}
Operação de encher o vazio, deixado pela evaporação, dentro dos cascos que contêm vinho.
\section{Atetose}
\begin{itemize}
\item {Grp. gram.:f.}
\end{itemize}
\begin{itemize}
\item {Utilização:Med.}
\end{itemize}
\begin{itemize}
\item {Proveniência:(Do gr. \textunderscore a\textunderscore  priv. + \textunderscore thetos\textunderscore , fixo)}
\end{itemize}
Impossibilidade de manter em posição fixa os dedos, executando
êstes movimento involuntário.
\section{Atetósico}
\begin{itemize}
\item {Grp. gram.:adj.}
\end{itemize}
Relativo á atetose.
\section{Ateu}
\begin{itemize}
\item {Grp. gram.:m.}
\end{itemize}
\begin{itemize}
\item {Proveniência:(Lat. \textunderscore atheus\textunderscore )}
\end{itemize}
Aquelle que não crê na existência de Deus.
Ímpio.
\section{Atextar}
\begin{itemize}
\item {Grp. gram.:v. i.}
\end{itemize}
\begin{itemize}
\item {Utilização:Prov.}
\end{itemize}
\begin{itemize}
\item {Utilização:dur.}
\end{itemize}
\begin{itemize}
\item {Proveniência:(De \textunderscore texto\textunderscore )}
\end{itemize}
Discutir, altercar.
\section{Atexto}
\begin{itemize}
\item {Grp. gram.:m.}
\end{itemize}
\begin{itemize}
\item {Utilização:Prov.}
\end{itemize}
\begin{itemize}
\item {Utilização:dur.}
\end{itemize}
Acto de atextar; altercação.
\section{Athálamo}
\begin{itemize}
\item {Grp. gram.:adj.}
\end{itemize}
\begin{itemize}
\item {Utilização:Bot.}
\end{itemize}
\begin{itemize}
\item {Proveniência:(Do gr. \textunderscore a\textunderscore  priv. + \textunderscore thalamos\textunderscore )}
\end{itemize}
Diz-se dos lichens, que não têm conceptáculos.
\section{Athamanta}
\begin{itemize}
\item {Grp. gram.:f.}
\end{itemize}
\begin{itemize}
\item {Proveniência:(De \textunderscore Athamas\textunderscore , n. p.)}
\end{itemize}
Gênero de plantas umbellíferas.
\section{Athamantina}
\begin{itemize}
\item {Grp. gram.:f.}
\end{itemize}
Substância, encontrada na raiz das sementes quási maduras da athamanta.
\section{Athanásia}
\begin{itemize}
\item {Grp. gram.:f.}
\end{itemize}
(V.tanaceto)
\section{Atheísmo}
\begin{itemize}
\item {Grp. gram.:m.}
\end{itemize}
Doutrina dos atheus.
\section{Atheísta}
\begin{itemize}
\item {Grp. gram.:m.}
\end{itemize}
(V.atheu)
\section{Atheístico}
\begin{itemize}
\item {Grp. gram.:adj.}
\end{itemize}
Relativo aos atheístas.
\section{Athélia}
\begin{itemize}
\item {Grp. gram.:f.}
\end{itemize}
\begin{itemize}
\item {Utilização:Med.}
\end{itemize}
Ausência de mamillas.
\section{Atheneias}
\begin{itemize}
\item {Grp. gram.:f. pl.}
\end{itemize}
Festas gregas, em honra de Minerva. Cf. Castilho. \textunderscore Fastos\textunderscore , I,
543.
\section{Atheneu}
\begin{itemize}
\item {Grp. gram.:m.}
\end{itemize}
\begin{itemize}
\item {Proveniência:(Do lat. \textunderscore athenaeum\textunderscore )}
\end{itemize}
Lugar público, em que os literatos gregos liam as suas obras.
Estabelecimento de instrucção, não official.
Academia.
\section{Atheniense}
\begin{itemize}
\item {Grp. gram.:adj.}
\end{itemize}
\begin{itemize}
\item {Grp. gram.:M.}
\end{itemize}
\begin{itemize}
\item {Proveniência:(Lat. \textunderscore atheniensis\textunderscore )}
\end{itemize}
Relativo a Athenas.
Habitante de Athenas.
\section{Atherandra}
\begin{itemize}
\item {Grp. gram.:f.}
\end{itemize}
Gênero de plantas asclepiádeas.
\section{Atherina}
\begin{itemize}
\item {Grp. gram.:f.}
\end{itemize}
\begin{itemize}
\item {Proveniência:(Do gr. \textunderscore athera\textunderscore )}
\end{itemize}
Nome scientífico do peixe-rei.
\section{Athermal}
\begin{itemize}
\item {Grp. gram.:adj.}
\end{itemize}
Diz-se das águas mineraes frias.
O mesmo que \textunderscore athérmico\textunderscore .
\section{Athermaneidade}
\begin{itemize}
\item {Grp. gram.:f.}
\end{itemize}
Carácter dos corpos athérmanos.
\section{Athérmano}
\begin{itemize}
\item {Grp. gram.:adj.}
\end{itemize}
\begin{itemize}
\item {Proveniência:(Do gr. \textunderscore a\textunderscore  priv. + \textunderscore therme\textunderscore )}
\end{itemize}
Diz-se dos corpos que são impenetráveis ao calor.
\section{Athérmico}
\begin{itemize}
\item {Grp. gram.:adj.}
\end{itemize}
\begin{itemize}
\item {Utilização:Phýs.}
\end{itemize}
\begin{itemize}
\item {Proveniência:(Do gr. \textunderscore a\textunderscore  priv. + \textunderscore therme\textunderscore )}
\end{itemize}
Que se não deixa atravessar pelo calor.
\section{Atheroma}
\begin{itemize}
\item {Grp. gram.:m.}
\end{itemize}
\begin{itemize}
\item {Utilização:Med.}
\end{itemize}
\begin{itemize}
\item {Proveniência:(Gr. \textunderscore astheroma\textunderscore )}
\end{itemize}
Nome, que se dava ao cisto sebáceo, formado no pescoço ou nas artérias do pescoço.
\section{Atheromatoso}
\begin{itemize}
\item {Grp. gram.:adj.}
\end{itemize}
Que soffre atheroma.
\section{Atheruro}
\begin{itemize}
\item {Grp. gram.:m.}
\end{itemize}
Gênero de plantas aroídeas.
\section{Athetose}
\begin{itemize}
\item {Grp. gram.:f.}
\end{itemize}
\begin{itemize}
\item {Utilização:Med.}
\end{itemize}
\begin{itemize}
\item {Proveniência:(Do gr. \textunderscore a\textunderscore  priv. + \textunderscore thetos\textunderscore , fixo)}
\end{itemize}
Impossibilidade de manter em posição fixa os dedos, executando
êstes movimento involuntário.
\section{Athetósico}
\begin{itemize}
\item {Grp. gram.:adj.}
\end{itemize}
Relativo á athetose.
\section{Atheu}
\begin{itemize}
\item {Grp. gram.:m.}
\end{itemize}
\begin{itemize}
\item {Proveniência:(Lat. \textunderscore atheus\textunderscore )}
\end{itemize}
Aquelle que não crê na existência de Deus.
Ímpio.
\section{Athleta}
\begin{itemize}
\item {Grp. gram.:m.}
\end{itemize}
\begin{itemize}
\item {Proveniência:(Gr. \textunderscore athletes\textunderscore )}
\end{itemize}
Aquelle que se exercitava na luta, para combater em festas solennes.
Lutador.
Homem valente.
Campeão.
\section{Athlética}
\begin{itemize}
\item {Grp. gram.:f.}
\end{itemize}
\begin{itemize}
\item {Proveniência:(Gr. \textunderscore athletike\textunderscore )}
\end{itemize}
Arte de athleta.
\section{Athleticamente}
\begin{itemize}
\item {Grp. gram.:adv.}
\end{itemize}
De modo \textunderscore athlético\textunderscore .
Á maneira de athleta.
\section{Athlético}
\begin{itemize}
\item {Grp. gram.:adj.}
\end{itemize}
\begin{itemize}
\item {Proveniência:(Gr. \textunderscore atletikos\textunderscore )}
\end{itemize}
Relativo a athleta; próprio de athleta.
Vigoroso.
\section{Athlóteta}
\begin{itemize}
\item {Grp. gram.:m.}
\end{itemize}
\begin{itemize}
\item {Proveniência:(Do gr. \textunderscore athlothetes\textunderscore )}
\end{itemize}
Magistrado grego, que presidia aos jogos gymnásticos, velando pela ordem e pela decência e conferindo prêmios.
\section{Athorácico}
\begin{itemize}
\item {Grp. gram.:adj.}
\end{itemize}
\begin{itemize}
\item {Grp. gram.:M. pl.}
\end{itemize}
\begin{itemize}
\item {Proveniência:(De \textunderscore a\textunderscore  priv. + \textunderscore thorácico\textunderscore )}
\end{itemize}
Que não tem thorax.
Crustáceos decápodes que parece não terem thorax.
\section{Athrepsia}
\begin{itemize}
\item {Grp. gram.:f.}
\end{itemize}
\begin{itemize}
\item {Utilização:Med.}
\end{itemize}
\begin{itemize}
\item {Proveniência:(Do gr. \textunderscore a\textunderscore  priv. + \textunderscore threpsis\textunderscore , nutrição)}
\end{itemize}
Deperecimento lento e progressivo dos recém-nascidos, causado por grandes perturbações das funcções nutritivas.
Deperecimento lento e progressivo dos recém-nascidos, resultante de profundas alterações do trabalho nutritivo. Cf. R. Galvão, \textunderscore Vocab.\textunderscore 
\section{Athymia}
\begin{itemize}
\item {Grp. gram.:f.}
\end{itemize}
\begin{itemize}
\item {Proveniência:(Do gr. \textunderscore a\textunderscore  priv. + \textunderscore thumos\textunderscore )}
\end{itemize}
Desânimo, abatimento.
\section{Atiá}
\begin{itemize}
\item {Grp. gram.:m.}
\end{itemize}
Moéda portuguesa de Dio, equivalente a 6-1/2 reis de Portugal.
\section{Atibar}
\begin{itemize}
\item {Grp. gram.:v. t.}
\end{itemize}
\begin{itemize}
\item {Utilização:Prov.}
\end{itemize}
\begin{itemize}
\item {Utilização:beir.}
\end{itemize}
Enfraquecer: \textunderscore atibar o vinho com água\textunderscore .
Abrandar a temperatura de: \textunderscore atibar a água quente com água fria\textunderscore .
(Por \textunderscore atibiar\textunderscore , de \textunderscore tíbio\textunderscore )
\section{Atibecer}
\begin{itemize}
\item {Grp. gram.:v. i.}
\end{itemize}
\begin{itemize}
\item {Utilização:Ant.}
\end{itemize}
O mesmo que \textunderscore entibecer\textunderscore .
\section{Atibiar}
\begin{itemize}
\item {Grp. gram.:v. t.}
\end{itemize}
(V.entibiar)
\section{Atiça}
\begin{itemize}
\item {Grp. gram.:m.}
\end{itemize}
\begin{itemize}
\item {Utilização:Prov.}
\end{itemize}
\begin{itemize}
\item {Utilização:alent.}
\end{itemize}
\begin{itemize}
\item {Proveniência:(De \textunderscore atiçar\textunderscore )}
\end{itemize}
Indivíduo que instiga outros a brigarem.
\section{Atiçador}
\begin{itemize}
\item {Grp. gram.:m.}
\end{itemize}
Aquelle que atiça.
\section{Atiçamento}
\begin{itemize}
\item {Grp. gram.:m.}
\end{itemize}
Acção de \textunderscore atiçar\textunderscore .
\section{Atiçar}
\begin{itemize}
\item {Grp. gram.:v. t.}
\end{itemize}
\begin{itemize}
\item {Proveniência:(De \textunderscore tição\textunderscore )}
\end{itemize}
Avivar, atear (o fogo).
Fomentar.
Irritar.
Assolar.
\section{...ático}
\begin{itemize}
\item {Grp. gram.:suf. adj.}
\end{itemize}
(indicativo de qualidade, pertença ou relação: \textunderscore lunático\textunderscore ; \textunderscore asiático...\textunderscore )
\section{Atiçoar}
\begin{itemize}
\item {Grp. gram.:v. t.}
\end{itemize}
Queimar com tições.
\section{Atido}
\begin{itemize}
\item {Grp. gram.:adj.}
\end{itemize}
Que se atém.
Confiado, esperançado.
\section{Atigrado}
\begin{itemize}
\item {Grp. gram.:adj.}
\end{itemize}
Mosqueado como a pelle do tigre.
\section{Atijolado}
\begin{itemize}
\item {Grp. gram.:adj.}
\end{itemize}
Que tem côr de tijolo.
\section{Atijolar}
\begin{itemize}
\item {Grp. gram.:v. t.}
\end{itemize}
\begin{itemize}
\item {Utilização:Bras}
\end{itemize}
\begin{itemize}
\item {Proveniência:(De \textunderscore tijolo\textunderscore )}
\end{itemize}
Ladrilhar.
\section{Atiladamente}
\begin{itemize}
\item {Grp. gram.:adv.}
\end{itemize}
De modo \textunderscore atilado\textunderscore .
\section{Atilado}
\begin{itemize}
\item {Grp. gram.:adj.}
\end{itemize}
\begin{itemize}
\item {Proveniência:(De \textunderscore atilar\textunderscore )}
\end{itemize}
Escrupuloso.
Ajuizado, discreto.
Elegante.
Correcto.
\section{Atilamento}
\begin{itemize}
\item {Grp. gram.:m.}
\end{itemize}
Qualidade do que é atilado.
Acção de \textunderscore atilar\textunderscore .
\section{Atilar}
\begin{itemize}
\item {Grp. gram.:v. t.}
\end{itemize}
Pôr o til em.
Executar com cuidado.
Aperfeiçoar.
Tornar hábil, esperto.
\section{Atilho}
\begin{itemize}
\item {Grp. gram.:m.}
\end{itemize}
Fita, cordão, para atar.
\section{Atiliano}
\begin{itemize}
\item {Grp. gram.:adj.}
\end{itemize}
\begin{itemize}
\item {Utilização:Jur.}
\end{itemize}
\begin{itemize}
\item {Proveniência:(Lat. \textunderscore atilianus\textunderscore )}
\end{itemize}
Diz-se do encargo da tutela dativa.
\section{Atim}
\begin{itemize}
\item {Grp. gram.:m.}
\end{itemize}
Ave aquática do Brasil.
\section{Atimar}
\begin{itemize}
\item {Grp. gram.:v. t.}
\end{itemize}
\begin{itemize}
\item {Utilização:Açor}
\end{itemize}
\begin{itemize}
\item {Utilização:ant.}
\end{itemize}
\begin{itemize}
\item {Grp. gram.:V. i.}
\end{itemize}
\begin{itemize}
\item {Utilização:Prov.}
\end{itemize}
\begin{itemize}
\item {Utilização:minh.}
\end{itemize}
\begin{itemize}
\item {Proveniência:(Do lat. \textunderscore ultimare\textunderscore ?)}
\end{itemize}
O mesmo que \textunderscore ultimar\textunderscore .
Levar a cabo.
Emprehender.
Acertar: \textunderscore atimar com o caminho\textunderscore .
\section{Atimia}
\begin{itemize}
\item {Grp. gram.:f.}
\end{itemize}
\begin{itemize}
\item {Proveniência:(Do gr. \textunderscore a\textunderscore  priv. + \textunderscore thumos\textunderscore )}
\end{itemize}
Desânimo, abatimento.
\section{Atinadamente}
\begin{itemize}
\item {Grp. gram.:adv.}
\end{itemize}
Com tino.
\section{Atinar}
\begin{itemize}
\item {Grp. gram.:v. t.  e  i.}
\end{itemize}
Executar com tino.
Descobrir pelo tino ou por conjectura: \textunderscore atinar com os motivos da zanga\textunderscore .
Recordar.
Encaminhar-se por algum indício.
\section{Atincal}
\begin{itemize}
\item {Grp. gram.:m.}
\end{itemize}
(V.tincal)
\section{Atinga}
\begin{itemize}
\item {Grp. gram.:f.}
\end{itemize}
Peixe dos mares do Brasil.
\section{Atino}
\begin{itemize}
\item {Grp. gram.:m.}
\end{itemize}
Acto de atinar.
O mesmo que \textunderscore tino\textunderscore ^1.
\section{Atintar}
\begin{itemize}
\item {Grp. gram.:v. t.}
\end{itemize}
Dar ligeira mão de tinta em.
\section{Atiplado}
\begin{itemize}
\item {Grp. gram.:adj.}
\end{itemize}
Que tem voz semelhante ao tiple. Cf. Filinto, IX, 55.
\section{Atiplar}
\begin{itemize}
\item {Grp. gram.:v. t.}
\end{itemize}
Dar voz de tiple a. Cf. Filinto, VIII, 94.
\section{Atirada}
\begin{itemize}
\item {Grp. gram.:f.}
\end{itemize}
Acto de atirar ou disparar.
\section{Atiradiço}
\begin{itemize}
\item {Grp. gram.:adj.}
\end{itemize}
\begin{itemize}
\item {Utilização:Fam.}
\end{itemize}
\begin{itemize}
\item {Proveniência:(De \textunderscore atirar\textunderscore )}
\end{itemize}
Petulante.
Aquelle que se atira a aventuras amorosas.
\section{Atirador}
\begin{itemize}
\item {Grp. gram.:m.}
\end{itemize}
\begin{itemize}
\item {Proveniência:(De \textunderscore atirar\textunderscore )}
\end{itemize}
Aquelle que atira; aquelle que dispara arma de fogo.
Soldado italiano.
\section{Atirar}
\begin{itemize}
\item {Grp. gram.:v. t.}
\end{itemize}
\begin{itemize}
\item {Grp. gram.:V. i.}
\end{itemize}
Arremessar; arrojar com rapidez; lançar: \textunderscore atirar pedras\textunderscore .
Disparar arma de fogo: \textunderscore atirar contra alguém\textunderscore .
Soltar arma de arremêsso.
Dar coices.
Participar de certas qualidades; ter tendência ou propensão: \textunderscore aquelle estôfo tem uma côr que atira para verde\textunderscore ; \textunderscore há gente que, desde criança, atira para a maldade\textunderscore .
\section{Atirar}
\begin{itemize}
\item {Grp. gram.:v. t.}
\end{itemize}
\begin{itemize}
\item {Utilização:Bras. do N}
\end{itemize}
Convidar para dançar.
(Próthese de \textunderscore tirar\textunderscore ?)
\section{Atitar}
\begin{itemize}
\item {Grp. gram.:v. i.}
\end{itemize}
\begin{itemize}
\item {Proveniência:(T. onom)}
\end{itemize}
Soltar gritos agudos.
Silvar.
\section{Atito}
\begin{itemize}
\item {Grp. gram.:m.}
\end{itemize}
\begin{itemize}
\item {Proveniência:(T. onom)}
\end{itemize}
Grito agudo, mormente das aves quando embravecidas. Cf. \textunderscore Eufrosina\textunderscore , 263.
\section{Atitude}
\begin{itemize}
\item {Grp. gram.:f.}
\end{itemize}
\begin{itemize}
\item {Utilização:Neol.}
\end{itemize}
\begin{itemize}
\item {Proveniência:(Lat. \textunderscore aptitudo\textunderscore )}
\end{itemize}
Postura, modo de têr o corpo.
Propósito; significação de um propósito.
Norma de procedimento: \textunderscore é esta a nossa atitude\textunderscore .
\section{Atlanta}
\begin{itemize}
\item {Grp. gram.:f.}
\end{itemize}
Gênero de molluscos gasterópodes.
\section{Atlante}
\begin{itemize}
\item {Grp. gram.:m.}
\end{itemize}
\begin{itemize}
\item {Utilização:Fig.}
\end{itemize}
Figura de homem, que sustenta cornija, esphera, etc.
Homem robusto.
Pessoa que tem a seu cargo negócios graves.
Gênero de molluscos, de concha em forma de espera.
Sustentáculo. Cf. Garrett, \textunderscore Port. na Bal.\textunderscore , 102.
\section{Atlântico}
\begin{itemize}
\item {Grp. gram.:adj.}
\end{itemize}
Relativo ao Atlas ou ao Atlântico.
Que vive no Atlântico.
\section{Atlântida}
\begin{itemize}
\item {Grp. gram.:f.}
\end{itemize}
\begin{itemize}
\item {Proveniência:(De \textunderscore atlante\textunderscore )}
\end{itemize}
Família de molluscos.
\section{Atlas}
\begin{itemize}
\item {Grp. gram.:m.}
\end{itemize}
\begin{itemize}
\item {Utilização:Anat.}
\end{itemize}
\begin{itemize}
\item {Proveniência:(Gr. \textunderscore Atlas\textunderscore , n. p.)}
\end{itemize}
Collecção de cartas geográphicas.
Collecção de estampas, elucidativas de obra, a que estão annexas.
A primeira vértebra superior.
\section{Atleta}
\begin{itemize}
\item {Grp. gram.:m.}
\end{itemize}
\begin{itemize}
\item {Proveniência:(Gr. \textunderscore athletes\textunderscore )}
\end{itemize}
Aquelle que se exercitava na luta, para combater em festas solennes.
Lutador.
Homem valente.
Campeão.
\section{Atlética}
\begin{itemize}
\item {Grp. gram.:f.}
\end{itemize}
\begin{itemize}
\item {Proveniência:(Gr. \textunderscore athletike\textunderscore )}
\end{itemize}
Arte de atleta.
\section{Atleticamente}
\begin{itemize}
\item {Grp. gram.:adv.}
\end{itemize}
De modo \textunderscore atlético\textunderscore .
Á maneira de atleta.
\section{Atlético}
\begin{itemize}
\item {Grp. gram.:adj.}
\end{itemize}
\begin{itemize}
\item {Proveniência:(Gr. \textunderscore atletikos\textunderscore )}
\end{itemize}
Relativo a atleta; próprio de atleta.
Vigoroso.
\section{Atloide}
\begin{itemize}
\item {Grp. gram.:adj.}
\end{itemize}
\begin{itemize}
\item {Grp. gram.:M.}
\end{itemize}
\begin{itemize}
\item {Proveniência:(De \textunderscore atlas\textunderscore  + gr. \textunderscore eidos\textunderscore )}
\end{itemize}
O mesmo que \textunderscore atloídeo\textunderscore .
A vértebra atlas.
\section{Atloídeo}
\begin{itemize}
\item {Grp. gram.:adj.}
\end{itemize}
\begin{itemize}
\item {Utilização:Anat.}
\end{itemize}
Relativo á vértebra atlas.
(Cp. \textunderscore atloide\textunderscore )
\section{Atloidiano}
\begin{itemize}
\item {Grp. gram.:adj.}
\end{itemize}
O mesmo que \textunderscore atloídeo\textunderscore .
\section{Atlóteta}
\begin{itemize}
\item {Grp. gram.:m.}
\end{itemize}
\begin{itemize}
\item {Proveniência:(Do gr. \textunderscore athlothetes\textunderscore )}
\end{itemize}
Magistrado grego, que presidia aos jogos gymnásticos, velando pela ordem e pela decência e conferindo prêmios.
\section{Atmidométrico}
\begin{itemize}
\item {Grp. gram.:adj.}
\end{itemize}
Relativo ao \textunderscore atmidómetro\textunderscore .
\section{Atmidómetro}
\begin{itemize}
\item {Grp. gram.:m.}
\end{itemize}
O mesmo que \textunderscore atmómetro\textunderscore .
\section{Atmómetro}
\begin{itemize}
\item {Grp. gram.:m.}
\end{itemize}
\begin{itemize}
\item {Proveniência:(Do gr. \textunderscore atmos\textunderscore  + \textunderscore metron\textunderscore )}
\end{itemize}
Instrumento, com que se mede a evaporação.
\section{Atmosfera}
\begin{itemize}
\item {Grp. gram.:f.}
\end{itemize}
\begin{itemize}
\item {Proveniência:(Do gr. \textunderscore atmos\textunderscore  + \textunderscore sphaira\textunderscore )}
\end{itemize}
Ar, camada gasosa, que envolve a terra.
Ambiente: \textunderscore viver numa atmosfera viciada\textunderscore .
Horizonte.
Invólucro fluído de qualquer astro.
Ambiente moral: \textunderscore a atmosfera da corrupção\textunderscore .
Medida dinâmica de gases e vapores, equivalente á pressão, exercida sôbre a superfície de um centímetro quadrado por uma coluna de mercúrio de 0^{m},76 de altura.
\section{Atmosférico}
\begin{itemize}
\item {Grp. gram.:adj.}
\end{itemize}
Relativo á \textunderscore atmosfera\textunderscore .
\section{Atmosphera}
\begin{itemize}
\item {Grp. gram.:f.}
\end{itemize}
\begin{itemize}
\item {Proveniência:(Do gr. \textunderscore atmos\textunderscore  + \textunderscore sphaira\textunderscore )}
\end{itemize}
Ar, camada gasosa, que envolve a terra.
Ambiente: \textunderscore viver numa atmosphera viciada\textunderscore .
Horizonte.
Invólucro fluído de qualquer astro.
Ambiente moral: \textunderscore a atmosphera da corrupção\textunderscore .
Medida dynâmica de gases e vapores, equivalente á pressão, exercida sôbre a superfície de um centímetro quadrado por uma columna de mercúrio de 0^{m},76 de altura.
\section{Atmosphérico}
\begin{itemize}
\item {Grp. gram.:adj.}
\end{itemize}
Relativo á \textunderscore atmosphéra\textunderscore .
\section{...ato}
\begin{itemize}
\item {Grp. gram.:suf., m.}
\end{itemize}
\begin{itemize}
\item {Grp. gram.:Suf. m.  e  adj.}
\end{itemize}
(indicativo, em Chímica, de um sal, formado pela reacção de um ácido; designativo de cargo, jurisdicção, etc.)
(significando o mesmo que \textunderscore ...ado\textunderscore )
\section{Atôa}
\begin{itemize}
\item {Grp. gram.:adj.}
\end{itemize}
\begin{itemize}
\item {Utilização:Bras}
\end{itemize}
Que não tem objecto ou fim; irreflectido.
Inútil. Cf. M. Soares, \textunderscore Diccion. Bras.\textunderscore 
(Cap. \textunderscore tôa\textunderscore )
\section{Á-tôa}
\begin{itemize}
\item {Grp. gram.:loc. adv.}
\end{itemize}
Sem reflexão.
Ao acaso.
\section{Atoada}
\begin{itemize}
\item {Grp. gram.:f.}
\end{itemize}
\begin{itemize}
\item {Utilização:Ant.}
\end{itemize}
\begin{itemize}
\item {Proveniência:(De \textunderscore toar\textunderscore )}
\end{itemize}
Boato; notícia vaga.
\section{Atoagem}
\begin{itemize}
\item {Grp. gram.:?}
\end{itemize}
Acção de \textunderscore atoar\textunderscore .
\section{Atoalhado}
\begin{itemize}
\item {Grp. gram.:adj.}
\end{itemize}
Adamascado.
Que tem lavor próprio de toalhas.
Coberto com toalha: \textunderscore mesa atoalhada\textunderscore .
\section{Atoalhar}
\begin{itemize}
\item {Grp. gram.:v. t.}
\end{itemize}
Cobrir com toalha.
\section{Atoamente}
\begin{itemize}
\item {Grp. gram.:adv.}
\end{itemize}
\begin{itemize}
\item {Utilização:Bras}
\end{itemize}
\begin{itemize}
\item {Proveniência:(De \textunderscore atôa\textunderscore )}
\end{itemize}
De modo \textunderscore irreflectido\textunderscore ; á tôa.
\section{Atoar}
\begin{itemize}
\item {Grp. gram.:v. t.}
\end{itemize}
\begin{itemize}
\item {Grp. gram.:V. i.}
\end{itemize}
\begin{itemize}
\item {Utilização:Pop.}
\end{itemize}
Levar á tôa, a reboque, á sirga.
Rebocar.
Teimar em se não mover (um animal).
\section{Atoarda}
\begin{itemize}
\item {Grp. gram.:f.}
\end{itemize}
(Corr. de \textunderscore atoada\textunderscore , ou metáth. de \textunderscore atroada\textunderscore )
\section{Atobá}
\begin{itemize}
\item {Grp. gram.:m.}
\end{itemize}
Ave aquática do Brasil.
\section{Atocaiar}
\begin{itemize}
\item {Grp. gram.:v. t.}
\end{itemize}
\begin{itemize}
\item {Utilização:Bras}
\end{itemize}
\begin{itemize}
\item {Proveniência:(De \textunderscore tocaia\textunderscore )}
\end{itemize}
Assaltar nas sombras ou no ermo.
Esconder-se para atacar de surpresa.
\section{Atocalto}
\begin{itemize}
\item {Grp. gram.:m.}
\end{itemize}
Aranha americana, cuja teia é de fios variegados.
\section{Atochador}
\begin{itemize}
\item {Grp. gram.:m.}
\end{itemize}
Aquelle que atocha.
Instrumento de \textunderscore atochar\textunderscore .
\section{Atochar}
\begin{itemize}
\item {Grp. gram.:v. t.}
\end{itemize}
\begin{itemize}
\item {Proveniência:(De \textunderscore tocho\textunderscore )}
\end{itemize}
Apertar.
Entalar.
Segurar com tocho ou cunha.
Atulhar.
\section{Atôcho}
\begin{itemize}
\item {Grp. gram.:m.}
\end{itemize}
Pau ou cunha, com que se atocha.
O mesmo que \textunderscore tocho\textunderscore .
\section{Atocia}
\begin{itemize}
\item {Grp. gram.:f.}
\end{itemize}
\begin{itemize}
\item {Proveniência:(Do gr. \textunderscore a\textunderscore  priv. + \textunderscore tokos\textunderscore )}
\end{itemize}
Esterilidade da mulher.
\section{Atócio}
\begin{itemize}
\item {Grp. gram.:m.}
\end{itemize}
Medicamento, de que se dizia que obstava á fecundação na mulher.
(Cp. \textunderscore atocia\textunderscore )
\section{Atoladamente}
\begin{itemize}
\item {Grp. gram.:adv.}
\end{itemize}
Á maneira de tolo.
\section{Atoladamente}
\begin{itemize}
\item {Grp. gram.:adv.}
\end{itemize}
\begin{itemize}
\item {Proveniência:(De \textunderscore atolado\textunderscore ^1)}
\end{itemize}
Á maneira do que se atolou.
\section{Atoladiço}
\begin{itemize}
\item {Grp. gram.:adj.}
\end{itemize}
\begin{itemize}
\item {Proveniência:(De \textunderscore atolar\textunderscore )}
\end{itemize}
Que fórma atoleiro.
\section{Atolado}
\begin{itemize}
\item {Grp. gram.:adj.}
\end{itemize}
Que se atolou.
Atascado.
\section{Atolado}
\begin{itemize}
\item {Grp. gram.:adj.}
\end{itemize}
Que tem modos de tolo; atoleimado.
\section{Atoladoiro}
\begin{itemize}
\item {Grp. gram.:m.}
\end{itemize}
\begin{itemize}
\item {Utilização:Prov.}
\end{itemize}
\begin{itemize}
\item {Utilização:trasm.}
\end{itemize}
O mesmo que \textunderscore atoleiro\textunderscore .
\section{Atoladouro}
\begin{itemize}
\item {Grp. gram.:m.}
\end{itemize}
\begin{itemize}
\item {Utilização:Prov.}
\end{itemize}
\begin{itemize}
\item {Utilização:trasm.}
\end{itemize}
O mesmo que \textunderscore atoleiro\textunderscore .
\section{Atolambado}
\begin{itemize}
\item {Grp. gram.:adj.}
\end{itemize}
\begin{itemize}
\item {Utilização:Pop.}
\end{itemize}
O mesmo que \textunderscore atoleimado\textunderscore . Cf. Camillo, \textunderscore Volcões de Lama\textunderscore , 102.
\section{Atolambar}
\begin{itemize}
\item {Grp. gram.:v. t.}
\end{itemize}
Causar toleima a. Cf. Cortesão, \textunderscore Subs\textunderscore .
\section{Atolamento}
\begin{itemize}
\item {Grp. gram.:m.}
\end{itemize}
Acto de \textunderscore atolar\textunderscore .
\section{Atolar}
\begin{itemize}
\item {Grp. gram.:v. t.}
\end{itemize}
Meter em atoleiros.
Enterrar no lodo; atascar.
(Cast. \textunderscore atollar\textunderscore )
\section{Atoleco}
\begin{itemize}
\item {Grp. gram.:adj.}
\end{itemize}
\begin{itemize}
\item {Utilização:Pop.}
\end{itemize}
O mesmo que \textunderscore atoleimado\textunderscore . Cf. Castilho, \textunderscore Sabichonas\textunderscore .
\section{Atoledo}
\begin{itemize}
\item {Grp. gram.:m.}
\end{itemize}
\begin{itemize}
\item {Utilização:Bras. do S}
\end{itemize}
O mesmo que \textunderscore atoleiro\textunderscore .
\section{Atoleimado}
\begin{itemize}
\item {Grp. gram.:adj.}
\end{itemize}
O mesmo que \textunderscore atolado\textunderscore ^2.
\section{Atoleimar-se}
\begin{itemize}
\item {Grp. gram.:v. p.}
\end{itemize}
\begin{itemize}
\item {Proveniência:(De \textunderscore toleima\textunderscore )}
\end{itemize}
Têr maneiras de tolo.
\section{Atoleiro}
\begin{itemize}
\item {Grp. gram.:m.}
\end{itemize}
\begin{itemize}
\item {Proveniência:(De \textunderscore atolar\textunderscore )}
\end{itemize}
Lodaçal; pântano.
Rebaixamento.
Embaraço, de que é diffícil sair.
\section{Atólico}
\begin{itemize}
\item {Grp. gram.:adj.}
\end{itemize}
(Fórma pop. de \textunderscore attónito\textunderscore )
\section{Atomatar}
\begin{itemize}
\item {Grp. gram.:v. t.}
\end{itemize}
\begin{itemize}
\item {Utilização:Pop.}
\end{itemize}
\begin{itemize}
\item {Utilização:Bras}
\end{itemize}
Tornar vermelho (alguém) como um tomate.
Confundir, envergonhar.
Esborrachar como a um tomate; pisar, abater.
\section{Atombar}
\begin{itemize}
\item {Grp. gram.:v. t.}
\end{itemize}
Arrolar; reduzir a tombo.
\section{Atomicidade}
\begin{itemize}
\item {Grp. gram.:f.}
\end{itemize}
\begin{itemize}
\item {Utilização:Phýs.}
\end{itemize}
\begin{itemize}
\item {Proveniência:(De \textunderscore atómico\textunderscore )}
\end{itemize}
Propriedade, que tem o átomo de um corpo, de se unir a átomo de outro.
Valência máxima e fixa.
\section{Atómico}
\begin{itemize}
\item {Grp. gram.:adj.}
\end{itemize}
Relativo a \textunderscore átomo\textunderscore .
\section{Atomismo}
\begin{itemize}
\item {Grp. gram.:m.}
\end{itemize}
Systema philosóphico, que explica a constituição do universo por meio de princípios chamados átomos.
\section{Atomista}
\begin{itemize}
\item {Grp. gram.:m.}
\end{itemize}
Sectário do atomismo.
\section{Atomistico}
\begin{itemize}
\item {Grp. gram.:adj.}
\end{itemize}
O mesmo que \textunderscore atómico\textunderscore .
E diz-se da theoria do atomismo.
\section{Atomizar}
\begin{itemize}
\item {Grp. gram.:v. t.}
\end{itemize}
Reduzir a átomos. Cf. Cortesão, \textunderscore Subs\textunderscore .
\section{Átomo}
\begin{itemize}
\item {Grp. gram.:m.}
\end{itemize}
\begin{itemize}
\item {Grp. gram.:Pl.}
\end{itemize}
\begin{itemize}
\item {Proveniência:(Lat. \textunderscore atomus\textunderscore )}
\end{itemize}
Partícula, que se considera o último grau da divisão da matéria.
Coisa excessivamente pequena.
Insignificância.
Corpúsculos, que se vêem movendo-se no espaço, quando banhados por uma réstea de luz.
Curto espaço:«\textunderscore num átomo de tempo...\textunderscore »Filinto, \textunderscore D. Man.\textunderscore , I, 193.
Momento, occasião:«\textunderscore nesse átomo, entra o Conde\textunderscore ». Filinto, XXI, 206.
\section{Atomologia}
\begin{itemize}
\item {Grp. gram.:f.}
\end{itemize}
\begin{itemize}
\item {Proveniência:(Do gr. \textunderscore atomos\textunderscore  + \textunderscore logos\textunderscore )}
\end{itemize}
Estudo das forças, que as moléculas da matéria exercem umas sôbre as outras.
Tratado da theoria atomística.
\section{Atomológico}
\begin{itemize}
\item {Grp. gram.:adj.}
\end{itemize}
Relativo á \textunderscore atomologia\textunderscore .
\section{Atomologista}
\begin{itemize}
\item {Grp. gram.:m.}
\end{itemize}
Aquelle que se dedica ao estudo da \textunderscore atomologia\textunderscore .
\section{Atondo}
\begin{itemize}
\item {Grp. gram.:m.}
\end{itemize}
\begin{itemize}
\item {Utilização:Ant.}
\end{itemize}
\begin{itemize}
\item {Proveniência:(Do b. lat. \textunderscore adtonitum\textunderscore )}
\end{itemize}
Alfaia, traste de uso. Cf. Herculano, \textunderscore Opúsc.\textunderscore , III, 279.
\section{Atonelado}
\begin{itemize}
\item {Grp. gram.:adj.}
\end{itemize}
Que tem fórma de tonel.
\section{Atonia}
\begin{itemize}
\item {Grp. gram.:f.}
\end{itemize}
\begin{itemize}
\item {Proveniência:(Gr. \textunderscore atonia\textunderscore )}
\end{itemize}
Fraqueza; frouxidão; debilidade geral.
Inércia.
\section{Atónico}
\begin{itemize}
\item {Grp. gram.:adj.}
\end{itemize}
Relativo á \textunderscore atonia\textunderscore .
\section{Atónico}
\begin{itemize}
\item {Grp. gram.:adj.}
\end{itemize}
O mesmo que \textunderscore átono\textunderscore .
\section{Átono}
\begin{itemize}
\item {Grp. gram.:adj.}
\end{itemize}
\begin{itemize}
\item {Utilização:Gram.}
\end{itemize}
\begin{itemize}
\item {Proveniência:(Do gr. \textunderscore a\textunderscore  + \textunderscore tonos\textunderscore )}
\end{itemize}
Que não tem accento tónico; não accentuado: \textunderscore vogal átona\textunderscore .
\section{Atontadamente}
\begin{itemize}
\item {Grp. gram.:adv.}
\end{itemize}
O mesmo que \textunderscore tontamente\textunderscore .
\section{Atontar}
\begin{itemize}
\item {Grp. gram.:v. t.}
\end{itemize}
\begin{itemize}
\item {Proveniência:(De \textunderscore tonto\textunderscore )}
\end{itemize}
O mesmo que \textunderscore entontecer\textunderscore .
\section{Atontear}
\begin{itemize}
\item {Grp. gram.:v. t.}
\end{itemize}
(V.atontar)
\section{Atopar}
\begin{itemize}
\item {Grp. gram.:v. t.}
\end{itemize}
\begin{itemize}
\item {Utilização:Prov.}
\end{itemize}
\begin{itemize}
\item {Utilização:trasm.}
\end{itemize}
O mesmo que \textunderscore topar\textunderscore .
\section{Atopetar}
\begin{itemize}
\item {Grp. gram.:v.}
\end{itemize}
\begin{itemize}
\item {Utilização:t. Náut.}
\end{itemize}
Levantar, içar até o tope (do navio).
(Cp. \textunderscore topetar\textunderscore )
\section{Atora}
\begin{itemize}
\item {Grp. gram.:f.}
\end{itemize}
\begin{itemize}
\item {Utilização:Bras}
\end{itemize}
\begin{itemize}
\item {Proveniência:(De \textunderscore atorar\textunderscore )}
\end{itemize}
Pedaço de pau, cortado em peças regulares.
Tôro.
\section{Atorácico}
\begin{itemize}
\item {Grp. gram.:adj.}
\end{itemize}
\begin{itemize}
\item {Grp. gram.:M. pl.}
\end{itemize}
\begin{itemize}
\item {Proveniência:(De \textunderscore a\textunderscore  priv. + \textunderscore thorácico\textunderscore )}
\end{itemize}
Que não tem torax.
Crustáceos decápodes que parece não terem torax.
\section{Atorar}
\begin{itemize}
\item {Grp. gram.:v. t.}
\end{itemize}
Dividir em toros.
\section{Atorçalado}
\begin{itemize}
\item {Grp. gram.:adj.}
\end{itemize}
Guarnecido de torçal.
\section{Atorçalar}
\begin{itemize}
\item {Grp. gram.:v. t.}
\end{itemize}
Guarnecer com torçal; bordar com torçal.
\section{Atorcedor}
\begin{itemize}
\item {Grp. gram.:m.}
\end{itemize}
Aquelle que replica ou accusa. Cf. Usque, \textunderscore Tribulações\textunderscore , 32.
\section{Atordoadamente}
\begin{itemize}
\item {Grp. gram.:adv.}
\end{itemize}
Com atordoamento.
Com perturbação.
\section{Atordoado}
\begin{itemize}
\item {Grp. gram.:adj.}
\end{itemize}
Perturbado.
Confundido.
\section{Atordoador}
\begin{itemize}
\item {Grp. gram.:adj.}
\end{itemize}
Que atordoa. Cf. Júl. Dinis, \textunderscore Morgadinha\textunderscore , 112 e 253.
\section{Atordoamento}
\begin{itemize}
\item {Grp. gram.:m.}
\end{itemize}
Effeito de \textunderscore atordoar\textunderscore .
\section{Atordoante}
\begin{itemize}
\item {Grp. gram.:adj.}
\end{itemize}
O mesmo que \textunderscore atordoador\textunderscore .
\section{Atordoar}
\begin{itemize}
\item {Grp. gram.:v. t.}
\end{itemize}
Perturbar os sentidos de.
Estontear.
Causar admiração a.
(Cp. \textunderscore aturdir\textunderscore )
\section{Atormentadamente}
\begin{itemize}
\item {Grp. gram.:adv.}
\end{itemize}
Com tormento.
\section{Atormentador}
\begin{itemize}
\item {Grp. gram.:m.}
\end{itemize}
Aquelle que atormenta.
\section{Atormentar}
\begin{itemize}
\item {Grp. gram.:v. t.}
\end{itemize}
\begin{itemize}
\item {Proveniência:(De \textunderscore tormento\textunderscore )}
\end{itemize}
Dar tormento a; torturar.
Mortificar; affligir.
Agitar.
\section{Atormentativo}
\begin{itemize}
\item {Grp. gram.:adj.}
\end{itemize}
Que causa tormento.
\section{Atorrear}
\begin{itemize}
\item {Grp. gram.:v. t.}
\end{itemize}
Guarnecer de torres.
\section{Atortemelado}
\begin{itemize}
\item {Grp. gram.:adj.}
\end{itemize}
\begin{itemize}
\item {Proveniência:(De \textunderscore torto\textunderscore ?)}
\end{itemize}
Que não tem fimeza no andar; que anda aos ziguezagues:«\textunderscore os mais veleiros levavam-no esfalfado, cambaleando, atortemelado\textunderscore ». Camillo, \textunderscore Brasileira\textunderscore , 62.
\section{Atoucado}
\begin{itemize}
\item {Grp. gram.:adj.}
\end{itemize}
Semelhante a touca.
\section{Atoucinhado}
\begin{itemize}
\item {Grp. gram.:adj.}
\end{itemize}
Semelhante ao toucinho.
Gordo.
\section{Atoxicar}
\begin{itemize}
\item {Grp. gram.:v. t.}
\end{itemize}
(V.entoxicar)
\section{Atóxico}
\begin{itemize}
\item {Grp. gram.:adj.}
\end{itemize}
\begin{itemize}
\item {Proveniência:(De \textunderscore a\textunderscore  priv. + \textunderscore tóxico\textunderscore )}
\end{itemize}
Que não tem veneno.
\section{Atrabile}
\begin{itemize}
\item {Grp. gram.:f.}
\end{itemize}
\begin{itemize}
\item {Proveniência:(Do lat. \textunderscore ater\textunderscore  + \textunderscore bilis\textunderscore )}
\end{itemize}
Imaginário humor ou bílis negra, que se suppunha sêr a causa da melancolia.
\section{Atrabiliário}
\begin{itemize}
\item {Grp. gram.:adj.}
\end{itemize}
Que tem atrabílis.
Melancólico.
Colérico.
\section{Atrabilioso}
\begin{itemize}
\item {Grp. gram.:adj.}
\end{itemize}
O mesmo que \textunderscore atrabiliário\textunderscore .
\section{Atrabílis}
\begin{itemize}
\item {Grp. gram.:f.}
\end{itemize}
\begin{itemize}
\item {Proveniência:(Do lat. \textunderscore ater\textunderscore  + \textunderscore bilis\textunderscore )}
\end{itemize}
Imaginário humor ou bílis negra, que se suppunha sêr a causa da melancolia.
\section{Atracação}
\begin{itemize}
\item {Grp. gram.:f.}
\end{itemize}
Acção de \textunderscore atracar\textunderscore .
\section{Atracadela}
\begin{itemize}
\item {Grp. gram.:f.}
\end{itemize}
O mesmo que \textunderscore atracão\textunderscore .
\section{Atracador}
\begin{itemize}
\item {Grp. gram.:m.}
\end{itemize}
Aquelle que atraca.
\section{Atracão}
\begin{itemize}
\item {Grp. gram.:m.}
\end{itemize}
\begin{itemize}
\item {Utilização:Pop.}
\end{itemize}
\begin{itemize}
\item {Proveniência:(De \textunderscore atracar\textunderscore )}
\end{itemize}
Encontrão.
Impertinência.
\section{Atracar}
\begin{itemize}
\item {Grp. gram.:v. t.}
\end{itemize}
\begin{itemize}
\item {Utilização:Pop.}
\end{itemize}
\begin{itemize}
\item {Proveniência:(T. cast.)}
\end{itemize}
Amarrar.
Encostar (um barco a outro).
Aproximar-se impertinentemente de, para pedir ou suggestionar: \textunderscore atracou-o mesmo na rua, para lhe pedir cinco tostões\textunderscore .
\section{Atractóbolo}
\begin{itemize}
\item {Grp. gram.:m.}
\end{itemize}
Gênero de cogumelos.
\section{Atractosomo}
\begin{itemize}
\item {Grp. gram.:adj.}
\end{itemize}
\begin{itemize}
\item {Utilização:Zool.}
\end{itemize}
Que tem o corpo fusiforme.
\section{Atrafegar-se}
\begin{itemize}
\item {Grp. gram.:v. p.}
\end{itemize}
Fatigar-se.
Meter-se em tráfegos.
\section{Atraiçoadamente}
\begin{itemize}
\item {Grp. gram.:adv.}
\end{itemize}
Com traição.
\section{Atraiçoador}
\begin{itemize}
\item {Grp. gram.:m.}
\end{itemize}
Aquelle que atraiçôa.
\section{Atraiçoar}
\begin{itemize}
\item {Grp. gram.:v. t.}
\end{itemize}
Tratar com traição; enganar.
\section{Atralhoar}
\begin{itemize}
\item {Grp. gram.:v. t.}
\end{itemize}
Meter á charrua (toiros castrados).
\section{Atramar}
\begin{itemize}
\item {Grp. gram.:v. i.}
\end{itemize}
\begin{itemize}
\item {Utilização:Prov.}
\end{itemize}
\begin{itemize}
\item {Utilização:trasm.}
\end{itemize}
\begin{itemize}
\item {Grp. gram.:V. p.}
\end{itemize}
O mesmo que \textunderscore agostar-se\textunderscore .
Mostrar a trama, tornar-se ralo, (falando-se de um tecido):«\textunderscore damasco que o pó, a humidade e o tempo tinham feito atramar-se.\textunderscore »Corvo, \textunderscore Anno na Côrte\textunderscore , III, 83.
\section{Atramar}
\begin{itemize}
\item {Grp. gram.:v. i.}
\end{itemize}
O mesmo que \textunderscore atremar\textunderscore .--(É corrente, no \textunderscore Cancion. de Resende\textunderscore )
\section{Atramentária}
\begin{itemize}
\item {Grp. gram.:f.}
\end{itemize}
\begin{itemize}
\item {Proveniência:(Do lat. \textunderscore atramentum\textunderscore )}
\end{itemize}
Sulfato de ferro.
\section{Atramentário}
\begin{itemize}
\item {Grp. gram.:adj.}
\end{itemize}
\begin{itemize}
\item {Grp. gram.:M.}
\end{itemize}
Relativo a \textunderscore atramento\textunderscore .
Tinteiro em que os Romanos tinham o atramento.
Vasilha, em que os pintores e os sapateiros tinham um verniz escuro.
\section{Atramento}
\begin{itemize}
\item {Grp. gram.:m.}
\end{itemize}
\begin{itemize}
\item {Proveniência:(Lat. \textunderscore atramentum\textunderscore )}
\end{itemize}
Tinta, com que os Romanos escreviam.
Líquido escuro, que servia para pintar e envernizar.
\section{Atrancada}
\begin{itemize}
\item {Grp. gram.:f.}
\end{itemize}
\begin{itemize}
\item {Utilização:Prov.}
\end{itemize}
\begin{itemize}
\item {Utilização:minh.}
\end{itemize}
\begin{itemize}
\item {Proveniência:(De \textunderscore atrancar\textunderscore )}
\end{itemize}
Montão ou rima que na véspera de San João, se faz de cancelos, caniços, etc., que por pirraça se tiram das propriedades alheias.
\section{Atrancar}
\begin{itemize}
\item {Grp. gram.:v. t.}
\end{itemize}
(V.trancar)
\section{Atranco}
\begin{itemize}
\item {Grp. gram.:m.}
\end{itemize}
Coisa que atranca.
\section{Atranqueirado}
\begin{itemize}
\item {Grp. gram.:adj.}
\end{itemize}
Que tem tranqueira.
\section{Atrapalhação}
\begin{itemize}
\item {Grp. gram.:f.}
\end{itemize}
\begin{itemize}
\item {Utilização:Pop.}
\end{itemize}
\begin{itemize}
\item {Proveniência:(De \textunderscore atrapalhar\textunderscore )}
\end{itemize}
Confusão.
Embaraço; acanhamento.
\section{Atrapalhadamente}
\begin{itemize}
\item {Grp. gram.:adv.}
\end{itemize}
De modo \textunderscore atrapalhado\textunderscore .
Com atrapalhação.
\section{Atrapalhado}
\begin{itemize}
\item {Grp. gram.:adj.}
\end{itemize}
Que se atrapalhou.
Atordoado, perturbado.
\section{Atrapalhador}
\begin{itemize}
\item {Grp. gram.:m.}
\end{itemize}
Trapalhão.
Aquelle que atrapalha.
\section{Atrapalhar}
\begin{itemize}
\item {Grp. gram.:v. t.}
\end{itemize}
\begin{itemize}
\item {Grp. gram.:V. i.}
\end{itemize}
Perturbar; confundir.
Embaraçar.
Fazer ou dizer mal de: \textunderscore atrapalhar uma história\textunderscore .
Produzir confusão.
(Cp. \textunderscore trapalhão\textunderscore ^2)
\section{Atrapar}
\begin{itemize}
\item {Grp. gram.:v. t.}
\end{itemize}
\begin{itemize}
\item {Utilização:Prov.}
\end{itemize}
\begin{itemize}
\item {Utilização:trasm.}
\end{itemize}
\begin{itemize}
\item {Utilização:Ext.}
\end{itemize}
Agarrar na carreira.
Concluir.
(Cast. \textunderscore atrapar\textunderscore )
\section{Atrás}
\begin{itemize}
\item {Grp. gram.:adv.}
\end{itemize}
Detrás; no lugar posterior.
Anteriormente.
Em posição pior que a de outrem.
\section{Atrasadamente}
\begin{itemize}
\item {Grp. gram.:adv.}
\end{itemize}
De modo \textunderscore atrasado\textunderscore .
Anteriormente.
\section{Atrasado}
\begin{itemize}
\item {Grp. gram.:adj.}
\end{itemize}
Que se atrasou.
Retardado.
\section{Atrasador}
\begin{itemize}
\item {Grp. gram.:m.}
\end{itemize}
\begin{itemize}
\item {Grp. gram.:Adj.}
\end{itemize}
\begin{itemize}
\item {Utilização:Pop.}
\end{itemize}
Aquelle que atrasa.
Que atrasa.
Antiquado; ordinário.
\section{Atrasamento}
\begin{itemize}
\item {Grp. gram.:m.}
\end{itemize}
(V.atraso)
\section{Atrasar}
\begin{itemize}
\item {Grp. gram.:v. t.}
\end{itemize}
\begin{itemize}
\item {Grp. gram.:V. i.  e  p.}
\end{itemize}
Pôr atrás.
Demorar; dilatar: \textunderscore atrasar um pagamento\textunderscore .
Embaraçar o desenvolvimento de: \textunderscore o inverno atrasou os trigaes\textunderscore .
Prejudicar.
Andar ou mover-se mais lentamente do que convém, (falando-se de um relógio ou do seu maquinismo): \textunderscore êste relógio atrasa\textunderscore .
\section{Atrasmente}
\begin{itemize}
\item {fónica:trás}
\end{itemize}
\begin{itemize}
\item {Grp. gram.:adv.}
\end{itemize}
\begin{itemize}
\item {Utilização:Prov.}
\end{itemize}
\begin{itemize}
\item {Utilização:trasm.}
\end{itemize}
\begin{itemize}
\item {Proveniência:(De \textunderscore atrás\textunderscore )}
\end{itemize}
Anteriormente; algum tempo antes.
\section{Atraso}
\begin{itemize}
\item {Grp. gram.:m.}
\end{itemize}
\begin{itemize}
\item {Proveniência:(De \textunderscore atrasar\textunderscore )}
\end{itemize}
Acção ou effeito de atrasar.
Decadência.
\section{Atravancadamente}
\begin{itemize}
\item {Grp. gram.:adv.}
\end{itemize}
Com atravancamento.
\section{Atravancamento}
\begin{itemize}
\item {Grp. gram.:m.}
\end{itemize}
Acto ou effeito de \textunderscore atravancar\textunderscore .
\section{Atravancar}
\begin{itemize}
\item {Grp. gram.:v. t.}
\end{itemize}
Impedir com travanca.
Embaraçar, estorvar: \textunderscore atravancar uma empresa\textunderscore .
\section{Atravancarruas}
\begin{itemize}
\item {Grp. gram.:m.}
\end{itemize}
\begin{itemize}
\item {Utilização:T. da Bairrada}
\end{itemize}
O mesmo que \textunderscore trancarruas\textunderscore .
\section{Atravanco}
\begin{itemize}
\item {Grp. gram.:m.}
\end{itemize}
Acto ou effeito de \textunderscore atravancar\textunderscore .
Embaraço, travanca. Cf. Camillo, \textunderscore Cav. em Ruinas\textunderscore , 78 e 114; Filinto, VI, 272.
\section{Através}
\begin{itemize}
\item {Grp. gram.:adv.}
\end{itemize}
\begin{itemize}
\item {Grp. gram.:Loc. prep.}
\end{itemize}
\begin{itemize}
\item {Proveniência:(De \textunderscore través\textunderscore )}
\end{itemize}
De lado a lado.
\textunderscore Através de\textunderscore , por entre, pelo centro, de lado a lado de: \textunderscore através da cidade\textunderscore .
\section{Atravessadamente}
\begin{itemize}
\item {Grp. gram.:adv.}
\end{itemize}
\begin{itemize}
\item {Proveniência:(De \textunderscore atravessar\textunderscore )}
\end{itemize}
Contrariamente.
Ao través.
\section{Atravessadeira}
\begin{itemize}
\item {Grp. gram.:f.}
\end{itemize}
\begin{itemize}
\item {Utilização:T. de Coimbra}
\end{itemize}
\begin{itemize}
\item {Proveniência:(De \textunderscore atravessar\textunderscore )}
\end{itemize}
Mulher, que, ás portas da cidade, compra gêneros destinados ao mercado, revendendo-os neste, por maior preço do que se os fornecedores os expusessem directamente á venda. Cf. \textunderscore Conimbricense\textunderscore , de 7--X--99.
\section{Atravessadiço}
\begin{itemize}
\item {Grp. gram.:adj.}
\end{itemize}
Que se atravessa, que se oppõe.
\section{Atravessado}
\begin{itemize}
\item {Grp. gram.:adj.}
\end{itemize}
\begin{itemize}
\item {Utilização:Fam.}
\end{itemize}
Travesso, desinquieto.
Pouco leal.
\section{Atravessadoiro}
\begin{itemize}
\item {Grp. gram.:m.}
\end{itemize}
Caminho, que atravessa terras lavradías.
Atalho.
\section{Atravessador}
\begin{itemize}
\item {Grp. gram.:m.}
\end{itemize}
Aquelle que atravessa.
\section{Atravessadouro}
\begin{itemize}
\item {Grp. gram.:m.}
\end{itemize}
Caminho, que atravessa terras lavradías.
Atalho.
\section{Atravessar}
\begin{itemize}
\item {Grp. gram.:v. t.}
\end{itemize}
\begin{itemize}
\item {Grp. gram.:V. i.}
\end{itemize}
\begin{itemize}
\item {Utilização:Náut.}
\end{itemize}
Passar através de: atravessar um campo.
Pôr ao través: \textunderscore atravessar um pau na estrada\textunderscore .
Pôr obliquamente.
Cruzar: \textunderscore atravessar os mares\textunderscore .
Traspassar.
Estar ao través de.
Commover muito: \textunderscore dores, que atravessam o coração\textunderscore .
Impedir.
Interromper.
Collocar adeante.
Dispor o velame, de fórma que aguente o navio parado.
\section{Atravincado}
\begin{itemize}
\item {Grp. gram.:adj.}
\end{itemize}
\begin{itemize}
\item {Utilização:Des.}
\end{itemize}
\begin{itemize}
\item {Proveniência:(De \textunderscore travinca\textunderscore )}
\end{itemize}
Bem seguro.
\section{Atrecer-se}
\begin{itemize}
\item {Grp. gram.:v. p.}
\end{itemize}
\begin{itemize}
\item {Utilização:Prov.}
\end{itemize}
\begin{itemize}
\item {Utilização:trasm.}
\end{itemize}
Tolher-se de frio.
\section{Atreguar}
\begin{itemize}
\item {Grp. gram.:v. i.}
\end{itemize}
O mesmo que \textunderscore atreguar-se\textunderscore .
\section{Atreguar-se}
\begin{itemize}
\item {Grp. gram.:v. p.}
\end{itemize}
Ajustar tréguas.
\section{Atregulhadamente}
\begin{itemize}
\item {Grp. gram.:adj.}
\end{itemize}
\begin{itemize}
\item {Utilização:Prov.}
\end{itemize}
\begin{itemize}
\item {Utilização:trasm.}
\end{itemize}
Apressadamente.
Atrapalhadamente.
\section{Atregulhar-se}
\begin{itemize}
\item {Grp. gram.:v. p.}
\end{itemize}
\begin{itemize}
\item {Utilização:Prov.}
\end{itemize}
\begin{itemize}
\item {Utilização:trasm.}
\end{itemize}
Apressar-se.
Atrapalhar-se, meter os pés pelas mãos.
\section{Atreiçoar}
\begin{itemize}
\item {Grp. gram.:v. t.}
\end{itemize}
\begin{itemize}
\item {Utilização:ant.}
\end{itemize}
\begin{itemize}
\item {Utilização:Pop.}
\end{itemize}
Sêr pérfido, refalsado para com. Cf. \textunderscore Peregrinação\textunderscore , XIV.
(Por \textunderscore atraiçoar\textunderscore )
\section{Atreladamente}
\begin{itemize}
\item {Grp. gram.:adv.}
\end{itemize}
Com trela.
\section{Atrelagem}
\begin{itemize}
\item {Grp. gram.:f.}
\end{itemize}
Apparelho, para atrelar a maquina ás carruagens de caminhos de ferro.
\section{Atrelar}
\begin{itemize}
\item {Grp. gram.:v. t.}
\end{itemize}
Prender com trela.
Prender.
Seduzir.
Dominar.
\section{Atrema}
\begin{itemize}
\item {Grp. gram.:f.}
\end{itemize}
\begin{itemize}
\item {Proveniência:(Gr. \textunderscore atremes\textunderscore )}
\end{itemize}
Planta umbellífera da América do Norte.
\section{Atremar}
\begin{itemize}
\item {Grp. gram.:v. i.}
\end{itemize}
\begin{itemize}
\item {Utilização:Pop.}
\end{itemize}
\begin{itemize}
\item {Utilização:Mad}
\end{itemize}
Proceder com acêrto.
Discorrer bem, têr tino:«\textunderscore esta minha cabeça já não atrema\textunderscore ». Camillo, \textunderscore Filha do Regicida\textunderscore , 20.
Prestar ouvidos, dar attenção.
(Talvez metáth. de \textunderscore atermar\textunderscore , de \textunderscore termo\textunderscore )
\section{Atrenado}
\begin{itemize}
\item {Grp. gram.:adj.}
\end{itemize}
\begin{itemize}
\item {Utilização:Ant.}
\end{itemize}
Triplicado.
(Por \textunderscore aternado\textunderscore , do lat. \textunderscore terni\textunderscore )
\section{Atrepa}
\begin{itemize}
\item {Grp. gram.:f.}
\end{itemize}
\begin{itemize}
\item {Utilização:Prov.}
\end{itemize}
O mesmo que \textunderscore atrepadeira\textunderscore .
\section{Atrepadeira}
\begin{itemize}
\item {Grp. gram.:f.}
\end{itemize}
\begin{itemize}
\item {Utilização:Prov.}
\end{itemize}
Ave, o mesmo que \textunderscore trepadeira\textunderscore .
\section{Atrepar}
\begin{itemize}
\item {Grp. gram.:v. i.}
\end{itemize}
O mesmo que \textunderscore trepar\textunderscore ^1.
\section{Atrepsia}
\begin{itemize}
\item {Grp. gram.:f.}
\end{itemize}
\begin{itemize}
\item {Utilização:Med.}
\end{itemize}
\begin{itemize}
\item {Proveniência:(Do gr. \textunderscore a\textunderscore  priv. + \textunderscore threpsis\textunderscore , nutrição)}
\end{itemize}
Deperecimento lento e progressivo dos recém-nascidos, causado por grandes perturbações das funcções nutritivas.
Deperecimento lento e progressivo dos recém-nascidos, resultante de profundas alterações do trabalho nutritivo. Cf. R. Galvão, \textunderscore Vocab.\textunderscore 
\section{Atresia}
\begin{itemize}
\item {Grp. gram.:f.}
\end{itemize}
\begin{itemize}
\item {Utilização:Med.}
\end{itemize}
\begin{itemize}
\item {Proveniência:(Do gr. \textunderscore a\textunderscore  priv. + \textunderscore tresis\textunderscore , abertura)}
\end{itemize}
Occlusão ou estreitamento de canal ou orifício natural do corpo.
\section{Atrever-se}
\begin{itemize}
\item {Grp. gram.:v. p.}
\end{itemize}
\begin{itemize}
\item {Proveniência:(Do lat. \textunderscore attribuere sibi\textunderscore ?)}
\end{itemize}
Têr ousadía.
Afoitar-se.
Arrostar: \textunderscore atrever-se com um lobo\textunderscore .
\section{Atreves-te}
\begin{itemize}
\item {Grp. gram.:m.}
\end{itemize}
\begin{itemize}
\item {Utilização:Prov.}
\end{itemize}
\begin{itemize}
\item {Utilização:trasm.}
\end{itemize}
Um jôgo de conca.
\section{Atrevidaço}
\begin{itemize}
\item {Grp. gram.:adj.}
\end{itemize}
\begin{itemize}
\item {Utilização:Pop.}
\end{itemize}
Muito atrevido.
\section{Atrevidamente}
\begin{itemize}
\item {Grp. gram.:adv.}
\end{itemize}
De modo \textunderscore atrevido\textunderscore .
Com atrevimento.
\section{Atrevidas}
\begin{itemize}
\item {Grp. gram.:f. pl.}
\end{itemize}
Us. na loc. adv. \textunderscore ás atrevidas\textunderscore , atrevidamente:«\textunderscore mais ás claras e atrevidas\textunderscore ». Filinto, \textunderscore D. Man.\textunderscore , II, 172.
\section{Atrevidete}
\begin{itemize}
\item {fónica:dê}
\end{itemize}
\begin{itemize}
\item {Grp. gram.:adj.}
\end{itemize}
(Dem. fam. de \textunderscore atrevido\textunderscore )
\section{Atrevido}
\begin{itemize}
\item {Grp. gram.:adj.}
\end{itemize}
Que se atreve.
Ousado.
Petulante.
Insolente.
\section{Atanger}
\begin{itemize}
\item {Grp. gram.:v. t.}
\end{itemize}
\begin{itemize}
\item {Utilização:Ant.}
\end{itemize}
O mesmo que \textunderscore atingir\textunderscore .
\section{Atélabo}
\begin{itemize}
\item {Grp. gram.:m.}
\end{itemize}
\begin{itemize}
\item {Proveniência:(Gr. \textunderscore attelabos\textunderscore )}
\end{itemize}
Insecto, espécie de pequeno gafanhoto.
\section{Atença}
\begin{itemize}
\item {Grp. gram.:f.}
\end{itemize}
Espera; expectativa:«\textunderscore ...na attença de outras galas\textunderscore ». Filinto, XII, 254.
\section{Atenção}
\begin{itemize}
\item {Grp. gram.:f.}
\end{itemize}
\begin{itemize}
\item {Proveniência:(Lat. \textunderscore attentio\textunderscore )}
\end{itemize}
Acto de applicar o espírito a.
Cuidado.
Estudo.
Urbanidade: \textunderscore tratar alguém com atenção\textunderscore .
\section{Atenciosamente}
\begin{itemize}
\item {Grp. gram.:adv.}
\end{itemize}
De modo \textunderscore atencioso\textunderscore .
Delicadamente, urbanamente.
\section{Atencioso}
\begin{itemize}
\item {Grp. gram.:adj.}
\end{itemize}
Feito com atenção: \textunderscore estudo atencioso\textunderscore .
Que presta atenção.
Delicado, urbano.
\section{Atenda}
\begin{itemize}
\item {Grp. gram.:f.}
\end{itemize}
\begin{itemize}
\item {Utilização:Ant.}
\end{itemize}
\begin{itemize}
\item {Proveniência:(De \textunderscore attender\textunderscore )}
\end{itemize}
O mesmo que \textunderscore atença\textunderscore ^2.
\section{Atender}
\begin{itemize}
\item {Grp. gram.:v. t.}
\end{itemize}
\begin{itemize}
\item {Utilização:Ant.}
\end{itemize}
\begin{itemize}
\item {Proveniência:(Lat. \textunderscore attendere\textunderscore )}
\end{itemize}
Dar atenção a: \textunderscore atender um aviso\textunderscore .
Advertir.
Tomar em consideração.
Observar.
Deferir, despachar: \textunderscore o ministro atendeu o requerente\textunderscore .
Esperar, aguardar.
\section{Atendível}
\begin{itemize}
\item {Grp. gram.:adj.}
\end{itemize}
\begin{itemize}
\item {Proveniência:(De \textunderscore attender\textunderscore )}
\end{itemize}
Que merece atenção.
\section{Atentadamente}
\begin{itemize}
\item {Grp. gram.:adv.}
\end{itemize}
O mesmo que \textunderscore atentamente\textunderscore .
\section{Atentado}
\begin{itemize}
\item {Grp. gram.:m.}
\end{itemize}
\begin{itemize}
\item {Proveniência:(De \textunderscore attentar\textunderscore )}
\end{itemize}
Acção criminosa; offensa ás leis ou á moral.
\section{Atentamente}
\begin{itemize}
\item {Grp. gram.:adv.}
\end{itemize}
\begin{itemize}
\item {Proveniência:(De \textunderscore attento\textunderscore )}
\end{itemize}
Com atenção.
\section{Atentar}
\begin{itemize}
\item {Grp. gram.:v. t.  e  i.}
\end{itemize}
\begin{itemize}
\item {Utilização:Bras. do N}
\end{itemize}
\begin{itemize}
\item {Proveniência:(De \textunderscore attento\textunderscore )}
\end{itemize}
Vêr com atenção; observar bem.
Considerar.
Cuidar de.
Reflectir.
Irritar.
Atormentar.
\section{Atentar}
\begin{itemize}
\item {Grp. gram.:v. t.}
\end{itemize}
\begin{itemize}
\item {Grp. gram.:V. i.}
\end{itemize}
\begin{itemize}
\item {Proveniência:(Lat. \textunderscore attentare\textunderscore )}
\end{itemize}
Intentar.
Commeter atentado.
\section{Atentatório}
\begin{itemize}
\item {Grp. gram.:adj.}
\end{itemize}
Em que há atentado.
\section{Atêntego}
\begin{itemize}
\item {Grp. gram.:adj.}
\end{itemize}
\begin{itemize}
\item {Utilização:Ant.}
\end{itemize}
O mesmo que atento? Cf. G. Vicente, I, 144.
\section{Atentivo}
\begin{itemize}
\item {Grp. gram.:adj.}
\end{itemize}
Em que há atenção:«\textunderscore observação attentiva\textunderscore ». Camillo, \textunderscore Rom. de um H. Rico\textunderscore , 48.
\section{Atento}
\begin{itemize}
\item {Grp. gram.:adj.}
\end{itemize}
\begin{itemize}
\item {Proveniência:(Lat. \textunderscore attentus\textunderscore )}
\end{itemize}
Que atende.
Que presta atenção.
Estudioso; applicado.
\section{Atenuação}
\begin{itemize}
\item {Grp. gram.:f.}
\end{itemize}
\begin{itemize}
\item {Proveniência:(Lat. \textunderscore attenuatio\textunderscore )}
\end{itemize}
Acção de atenuar.
\section{Atenuadamente}
\begin{itemize}
\item {Grp. gram.:adv.}
\end{itemize}
Com atenuação.
\section{Atenuante}
\begin{itemize}
\item {Grp. gram.:adj.}
\end{itemize}
\begin{itemize}
\item {Grp. gram.:F.}
\end{itemize}
\begin{itemize}
\item {Proveniência:(Lat. \textunderscore attenuans\textunderscore )}
\end{itemize}
Que atenua.
Circunstância atenuante.
\section{Atenuar}
\begin{itemize}
\item {Grp. gram.:v. t.}
\end{itemize}
\begin{itemize}
\item {Proveniência:(Lat. \textunderscore attenuare\textunderscore )}
\end{itemize}
Fazer tênue.
Adelgaçar.
Deminuir: \textunderscore atenuar necessidades\textunderscore .
Debilitar.
\section{Atenuativo}
\begin{itemize}
\item {Grp. gram.:adj.}
\end{itemize}
Que serve para atenuar.
\section{Atenuável}
\begin{itemize}
\item {Grp. gram.:adj.}
\end{itemize}
Que se póde atenuar.
\section{Atestação}
\begin{itemize}
\item {Grp. gram.:f.}
\end{itemize}
\begin{itemize}
\item {Proveniência:(Lat. \textunderscore attestatio\textunderscore )}
\end{itemize}
Acto de atestar.
\section{Atestado}
\begin{itemize}
\item {Grp. gram.:m.}
\end{itemize}
Acto de \textunderscore atestar\textunderscore .
Aquillo que se atesta.
Documento, em que se atesta alguma coisa: \textunderscore atestado de bom comportamento\textunderscore .
\section{Atestante}
\begin{itemize}
\item {Grp. gram.:m.  e  f.}
\end{itemize}
Pessôa que atesta.
\section{Atestar}
\begin{itemize}
\item {Grp. gram.:v. t.}
\end{itemize}
\begin{itemize}
\item {Proveniência:(Lat. \textunderscore attestari\textunderscore )}
\end{itemize}
Affirmar como testemunha.
Certificar por escrito.
Demonstrar.
\section{Aticismo}
\begin{itemize}
\item {Grp. gram.:m.}
\end{itemize}
\begin{itemize}
\item {Proveniência:(Lat. \textunderscore aticismus\textunderscore )}
\end{itemize}
Elegância de linguagem.
\section{Atico}
\begin{itemize}
\item {Grp. gram.:adj.}
\end{itemize}
\begin{itemize}
\item {Grp. gram.:M.}
\end{itemize}
\begin{itemize}
\item {Proveniência:(Lat. \textunderscore atticus\textunderscore )}
\end{itemize}
Relativo á Ática.
Conforme ao atticismo; elegante; puro: \textunderscore estilo ático\textunderscore .
Dialecto da Ática.
Lanço de parede no entablamento de um edifício, para lhe dar mais relevo, ou esconder a vista do telhado, ou servir de parapeito aos terraços.
\section{Aticurga}
\begin{itemize}
\item {Grp. gram.:f.}
\end{itemize}
\begin{itemize}
\item {Proveniência:(De \textunderscore atticurgo\textunderscore )}
\end{itemize}
Coluna ou pilastra ática, com quatro faces iguaes.
\section{Aticurgo}
\begin{itemize}
\item {Grp. gram.:adj.}
\end{itemize}
\begin{itemize}
\item {Proveniência:(Gr. \textunderscore attikourges\textunderscore )}
\end{itemize}
Diz-se do estilo architectónico, próprio da Ática ou de Athenas.
\section{Atinente}
\begin{itemize}
\item {Grp. gram.:adj.}
\end{itemize}
\begin{itemize}
\item {Proveniência:(Lat. \textunderscore attinens\textunderscore )}
\end{itemize}
Pertencente, relativo, concernente.
\section{Atingir}
\begin{itemize}
\item {Grp. gram.:v. t.}
\end{itemize}
\begin{itemize}
\item {Proveniência:(Lat. \textunderscore attingere\textunderscore , de \textunderscore ad\textunderscore  + \textunderscore tangere\textunderscore )}
\end{itemize}
Chegar a.
Tocar de leve.
Dizer respeito a: \textunderscore essa prohibição não atinge os carroceiros\textunderscore .
Perceber.
Conseguir.
\section{Atingível}
\begin{itemize}
\item {Grp. gram.:adj.}
\end{itemize}
Que póde sêr atingido.
\section{Atonitamente}
\begin{itemize}
\item {Grp. gram.:adv.}
\end{itemize}
Com espanto; de modo atónito.
\section{Atónito}
\begin{itemize}
\item {Grp. gram.:adj.}
\end{itemize}
\begin{itemize}
\item {Proveniência:(Lat. \textunderscore attonitus\textunderscore )}
\end{itemize}
Espantado; estupefacto.
Admirado; assombrado.
\section{Atracção}
\begin{itemize}
\item {Grp. gram.:f.}
\end{itemize}
\begin{itemize}
\item {Proveniência:(Lat. \textunderscore attractio\textunderscore )}
\end{itemize}
Acto de atrair.
Sympathia.
\section{Atractividade}
\begin{itemize}
\item {Grp. gram.:f.}
\end{itemize}
Qualidade de atractivo.
\section{Atractivo}
\begin{itemize}
\item {Grp. gram.:adj.}
\end{itemize}
\begin{itemize}
\item {Grp. gram.:M.}
\end{itemize}
\begin{itemize}
\item {Proveniência:(Lat. \textunderscore attractivus\textunderscore )}
\end{itemize}
Que tem a qualidade de atrair.
Qualidade de atrair.
Encanto, formosura: \textunderscore os atractivos da minha Maria\textunderscore .
\section{Atraente}
\begin{itemize}
\item {Grp. gram.:adj.}
\end{itemize}
\begin{itemize}
\item {Proveniência:(Lat. \textunderscore attrahens\textunderscore )}
\end{itemize}
Que atrai.
\section{Atraidor}
\begin{itemize}
\item {fónica:tra-í}
\end{itemize}
\begin{itemize}
\item {Grp. gram.:Adj.}
\end{itemize}
Aquelle que atrai.
Que atrai.
\section{Atraimento}
\begin{itemize}
\item {fónica:tra-í}
\end{itemize}
\begin{itemize}
\item {Grp. gram.:m.}
\end{itemize}
(V.atracção)
\section{Atrair}
\begin{itemize}
\item {Grp. gram.:v. t.}
\end{itemize}
\begin{itemize}
\item {Proveniência:(Lat. \textunderscore attrahere\textunderscore )}
\end{itemize}
Trazer para si; puxar para si: \textunderscore atrair a attenção\textunderscore .
Chamar.
Induzir: \textunderscore atrair para a desgraça\textunderscore .
Suscitar.
Fazer adherir a uma opinião.
\section{Atrectação}
\begin{itemize}
\item {Grp. gram.:f.}
\end{itemize}
\begin{itemize}
\item {Utilização:Des.}
\end{itemize}
\begin{itemize}
\item {Proveniência:(Lat. \textunderscore attrectalis\textunderscore )}
\end{itemize}
O mesmo que \textunderscore apropriação\textunderscore . Cf. Cortesão, \textunderscore Subs.\textunderscore 
\section{Atreito}
\begin{itemize}
\item {Grp. gram.:adj.}
\end{itemize}
\begin{itemize}
\item {Proveniência:(Lat. \textunderscore attractus\textunderscore )}
\end{itemize}
Que tem inclinação para alguma coisa: \textunderscore atreito a jogar\textunderscore .
Costumado.
Exposto: \textunderscore atreito a indigestões\textunderscore .
\section{Atrevimento}
\begin{itemize}
\item {Grp. gram.:m.}
\end{itemize}
Petulância.
Insolência.
Acção de \textunderscore atrever-se\textunderscore .
\section{Atribuição}
\begin{itemize}
\item {fónica:bu-i}
\end{itemize}
\begin{itemize}
\item {Grp. gram.:f.}
\end{itemize}
Acto de \textunderscore atribuir\textunderscore .
\section{Atribuidor}
\begin{itemize}
\item {fónica:bu-i}
\end{itemize}
\begin{itemize}
\item {Grp. gram.:m.}
\end{itemize}
Aquelle que atribue.
\section{Atribuir}
\begin{itemize}
\item {Grp. gram.:v. t.}
\end{itemize}
\begin{itemize}
\item {Proveniência:(Lat. \textunderscore attribuere\textunderscore )}
\end{itemize}
Imputar, referir: \textunderscore atribuir defeitos a alguém\textunderscore .
Conferir.
Apropriar.
\section{Atribuível}
\begin{itemize}
\item {Grp. gram.:adj.}
\end{itemize}
Que se deve ou se póde atribuir.
\section{Atribulação}
\begin{itemize}
\item {Grp. gram.:f.}
\end{itemize}
O mesmo que \textunderscore tribulação\textunderscore .
\section{Atribuladamente}
\begin{itemize}
\item {Grp. gram.:adv.}
\end{itemize}
Com tribulação.
\section{Atribulador}
\begin{itemize}
\item {Grp. gram.:m.}
\end{itemize}
\begin{itemize}
\item {Grp. gram.:Adj.}
\end{itemize}
Aquelle que atribula.
Que atribula.
\section{Atribular}
\begin{itemize}
\item {Grp. gram.:v. t.}
\end{itemize}
\begin{itemize}
\item {Proveniência:(Do lat. \textunderscore tribulare\textunderscore )}
\end{itemize}
Causar tribulação a.
Angustiar.
Inquietar.
\section{Atributar}
\begin{itemize}
\item {Grp. gram.:v. t.}
\end{itemize}
(V.tributar)
\section{Atributivo}
\begin{itemize}
\item {Grp. gram.:adj.}
\end{itemize}
\begin{itemize}
\item {Proveniência:(De \textunderscore attribuir\textunderscore )}
\end{itemize}
Que atribue.
Que indica um atributo.
\section{Atributo}
\begin{itemize}
\item {Grp. gram.:m.}
\end{itemize}
\begin{itemize}
\item {Utilização:Gram.}
\end{itemize}
\begin{itemize}
\item {Proveniência:(Lat. \textunderscore attributus\textunderscore )}
\end{itemize}
Aquillo que é próprio de alguém ou de alguma coisa.
Aquillo que se affirma ou se nega do sujeito.
Propriedade, qualidade.
Sýmbolo.
\section{Atricado}
\begin{itemize}
\item {Grp. gram.:adj.}
\end{itemize}
\begin{itemize}
\item {Utilização:Prov.}
\end{itemize}
\begin{itemize}
\item {Proveniência:(Do lat. \textunderscore tricare\textunderscore . Cp. \textunderscore trigar-se\textunderscore )}
\end{itemize}
Alegremente occupado num trabalho manual, com diligência.
(Colhido em \textunderscore Turquel\textunderscore )
\section{Atrição}
\begin{itemize}
\item {Grp. gram.:f.}
\end{itemize}
\begin{itemize}
\item {Utilização:Theol.}
\end{itemize}
\begin{itemize}
\item {Proveniência:(Lat. \textunderscore attritio\textunderscore )}
\end{itemize}
Atrito.
Contracção (do estômago).
Pequeno ferimento.
Desgaste.
Pesar de têr offendido a Divindade, causado pelo medo da punição.
\section{Atricaude}
\begin{itemize}
\item {Grp. gram.:adj.}
\end{itemize}
\begin{itemize}
\item {Utilização:Zool.}
\end{itemize}
\begin{itemize}
\item {Proveniência:(Do lat. \textunderscore ater\textunderscore  + \textunderscore cauda\textunderscore )}
\end{itemize}
Que tem cauda negra.
\section{Atríchia}
\begin{itemize}
\item {fónica:qui}
\end{itemize}
\begin{itemize}
\item {Grp. gram.:f.}
\end{itemize}
\begin{itemize}
\item {Proveniência:(Do gr. \textunderscore a\textunderscore  priv. + \textunderscore thrix\textunderscore )}
\end{itemize}
Falta de pêlos ou cabelos.
\section{Atricular}
\begin{itemize}
\item {Grp. gram.:v. i.}
\end{itemize}
\begin{itemize}
\item {Utilização:T. da Bairrada}
\end{itemize}
Parolar.
Questionar.
(Metáth. de \textunderscore articular\textunderscore ^2)
\section{Atriense}
\begin{itemize}
\item {Grp. gram.:m.}
\end{itemize}
\begin{itemize}
\item {Utilização:Ant.}
\end{itemize}
\begin{itemize}
\item {Proveniência:(Lat. \textunderscore atriensis\textunderscore )}
\end{itemize}
Porteiro; guarda do pátio.
\section{Atrigado}
\begin{itemize}
\item {Grp. gram.:adj.}
\end{itemize}
\begin{itemize}
\item {Utilização:Prov.}
\end{itemize}
\begin{itemize}
\item {Utilização:beir.}
\end{itemize}
Que tem côr de trigo.
Amarelado por doença, adoentado.
\section{Atrigar-se}
\begin{itemize}
\item {Grp. gram.:v. p.}
\end{itemize}
O mesmo que \textunderscore trigar-se\textunderscore :«\textunderscore Não se atrigue\textunderscore ». Camillo, \textunderscore Brasileira\textunderscore , 206.
\section{Atrigueirado}
\begin{itemize}
\item {Grp. gram.:adj.}
\end{itemize}
Quási trigueiro; tirante a trigueiro.
\section{Atril}
\begin{itemize}
\item {Grp. gram.:m.}
\end{itemize}
\begin{itemize}
\item {Utilização:Des.}
\end{itemize}
Estante do côro.
(Cast. \textunderscore atril\textunderscore )
\section{Atrincheirar}
\begin{itemize}
\item {Grp. gram.:v. t.}
\end{itemize}
(V.entrincheirar)
\section{Átrio}
\begin{itemize}
\item {Grp. gram.:m.}
\end{itemize}
\begin{itemize}
\item {Proveniência:(Lat. \textunderscore atrium\textunderscore )}
\end{itemize}
Pórtico romano, coberto, no interior do edifício.
Pátio; espaço defeso, na frente dos edifícios.
\section{Atríolo}
\begin{itemize}
\item {Grp. gram.:m.}
\end{itemize}
\begin{itemize}
\item {Proveniência:(Lat. \textunderscore atriolum\textunderscore )}
\end{itemize}
Pequeno átrio.
\section{Atrípede}
\begin{itemize}
\item {Grp. gram.:adj.}
\end{itemize}
\begin{itemize}
\item {Utilização:Zool.}
\end{itemize}
\begin{itemize}
\item {Proveniência:(Do lat. \textunderscore ater\textunderscore  + \textunderscore pes\textunderscore , \textunderscore pedis\textunderscore )}
\end{itemize}
Que tem pés negros.
\section{Atríquia}
\begin{itemize}
\item {Grp. gram.:f.}
\end{itemize}
\begin{itemize}
\item {Proveniência:(Do gr. \textunderscore a\textunderscore  priv. + \textunderscore thrix\textunderscore )}
\end{itemize}
Falta de pêlos ou cabelos.
\section{Atristar}
\begin{itemize}
\item {Grp. gram.:v. t.}
\end{itemize}
\begin{itemize}
\item {Grp. gram.:V. i.  e  p.}
\end{itemize}
O mesmo que \textunderscore entristecer\textunderscore .
Tornar-se triste. Cf. Usque, \textunderscore Tribulações\textunderscore , 21.
\section{Atrito}
\begin{itemize}
\item {Grp. gram.:m.}
\end{itemize}
\begin{itemize}
\item {Grp. gram.:Pl.}
\end{itemize}
\begin{itemize}
\item {Grp. gram.:Adj.}
\end{itemize}
\begin{itemize}
\item {Proveniência:(Lat. \textunderscore attritus\textunderscore )}
\end{itemize}
Contacto de dois corpos ásperos ou duros, roçando um pelo outro; fricção.
Difficuldades: \textunderscore encontrar atritos\textunderscore .
Que tem atrição: \textunderscore peccador atrito\textunderscore .
\section{Atro}
\begin{itemize}
\item {Grp. gram.:m.}
\end{itemize}
\begin{itemize}
\item {Proveniência:(Lat. \textunderscore ater\textunderscore )}
\end{itemize}
Negro.
Tenebroso; medonho.
\section{Atroada}
\begin{itemize}
\item {Grp. gram.:f.}
\end{itemize}
\begin{itemize}
\item {Proveniência:(De \textunderscore atroar\textunderscore )}
\end{itemize}
Grande ruido; estrondo.
\section{Atroador}
\begin{itemize}
\item {Grp. gram.:m.}
\end{itemize}
\begin{itemize}
\item {Grp. gram.:Adj.}
\end{itemize}
Aquelle que atrôa.
Que atrôa.
\section{Atroamento}
\begin{itemize}
\item {Grp. gram.:m.}
\end{itemize}
Acção e effeito de \textunderscore atroar\textunderscore .
\section{Atroar}
\begin{itemize}
\item {Grp. gram.:v. t.}
\end{itemize}
\begin{itemize}
\item {Grp. gram.:V. i.}
\end{itemize}
Abalar por effeito de estrondo.
Fazer retumbar: \textunderscore atroar os ares\textunderscore .
Atordoar; perturbar (o sentido da audição).
Molestar com pancada (os cascos das bêstas, quando se ferram).
Retumbar; fazer grande estrondo: \textunderscore as descargas atroavam\textunderscore .
(Cp. \textunderscore troar\textunderscore )
\section{Atroce}
\begin{itemize}
\item {Grp. gram.:adj.}
\end{itemize}
(Fórma ant. de \textunderscore atroz\textunderscore . Cf. \textunderscore Lusíadas\textunderscore , I, 85)
\section{Atrocemente}
\begin{itemize}
\item {Grp. gram.:adv.}
\end{itemize}
\begin{itemize}
\item {Utilização:Des.}
\end{itemize}
O mesmo que \textunderscore atrozmente\textunderscore . Cp. Filinto, \textunderscore D. Man.\textunderscore , I, 209.
\section{Atrocidade}
\begin{itemize}
\item {Grp. gram.:f.}
\end{itemize}
\begin{itemize}
\item {Proveniência:(Lat. \textunderscore atrocitas\textunderscore )}
\end{itemize}
Qualidade do que é atroz.
Barbaridade.
\section{Atroçoar}
\begin{itemize}
\item {Grp. gram.:v. t.}
\end{itemize}
Dividir em troços; fragmentar:«\textunderscore engenho para atroçoar o anil\textunderscore ». \textunderscore Inquér. Industr.\textunderscore , p. II, 1. I, 72.
\section{Atrofia}
\begin{itemize}
\item {Grp. gram.:f.}
\end{itemize}
\begin{itemize}
\item {Proveniência:(Gr. \textunderscore atrophia\textunderscore )}
\end{itemize}
Definhamento, por falta de nutrição; enfraquecimento.
Decadência.
\section{Atrofiante}
\begin{itemize}
\item {Grp. gram.:adj.}
\end{itemize}
Que atrofia.
\section{Atrofiar}
\begin{itemize}
\item {Grp. gram.:v. t.}
\end{itemize}
Causar atrofia a.
\section{Atrófico}
\begin{itemize}
\item {Grp. gram.:adj.}
\end{itemize}
Que padece atrofia.
\section{Atrombetado}
\begin{itemize}
\item {Grp. gram.:adj.}
\end{itemize}
Semelhante a uma trombeta.
\section{Atronar}
\begin{itemize}
\item {Grp. gram.:v. i.}
\end{itemize}
\begin{itemize}
\item {Utilização:Prov.}
\end{itemize}
\begin{itemize}
\item {Proveniência:(Do lat. \textunderscore tonare\textunderscore )}
\end{itemize}
O mesmo que \textunderscore trovejar\textunderscore .
\section{Atronchado}
\begin{itemize}
\item {Grp. gram.:adj.}
\end{itemize}
\begin{itemize}
\item {Utilização:Prov.}
\end{itemize}
\begin{itemize}
\item {Utilização:trasm.}
\end{itemize}
\begin{itemize}
\item {Proveniência:(De \textunderscore troncho\textunderscore )}
\end{itemize}
Atarracado e forte, (falando-se de um homem).
\section{Atrôo}
\begin{itemize}
\item {Grp. gram.:m.}
\end{itemize}
Acto de \textunderscore atroar\textunderscore .
\section{Átropa}
\begin{itemize}
\item {Grp. gram.:f.}
\end{itemize}
\begin{itemize}
\item {Proveniência:(De \textunderscore Átropos\textunderscore , n. p.)}
\end{itemize}
Gênero de plantas solâneas.
\section{Atropar}
\begin{itemize}
\item {Grp. gram.:v. t.}
\end{itemize}
Guarnecer de tropa.
Reunir em tropa.
\section{Atropeladamente}
\begin{itemize}
\item {Grp. gram.:adv.}
\end{itemize}
Com tropel; desordenadamente.
\section{Atropelamento}
\begin{itemize}
\item {Grp. gram.:m.}
\end{itemize}
Acto de \textunderscore atropelar\textunderscore .
\section{Atropelante}
\begin{itemize}
\item {Grp. gram.:adj.}
\end{itemize}
Que atropela.
\section{Atropelar}
\begin{itemize}
\item {Grp. gram.:v. t.}
\end{itemize}
\begin{itemize}
\item {Proveniência:(De \textunderscore tropel\textunderscore )}
\end{itemize}
Calcar, passando por cima: \textunderscore o cavalleiro atropelou uma criança\textunderscore .
Deitar ao chão.
Empurrar.
Desprezar.
Postergar: \textunderscore atropelar direitos\textunderscore .
\section{Atropêlo}
\begin{itemize}
\item {Grp. gram.:m.}
\end{itemize}
O mesmo que \textunderscore atropelamento\textunderscore .
\section{Atrophia}
\begin{itemize}
\item {Grp. gram.:f.}
\end{itemize}
\begin{itemize}
\item {Proveniência:(Gr. \textunderscore atrophia\textunderscore )}
\end{itemize}
Definhamento, por falta de nutrição; enfraquecimento.
Decadência.
\section{Atrophiante}
\begin{itemize}
\item {Grp. gram.:adj.}
\end{itemize}
Que atrophia.
\section{Atrophiar}
\begin{itemize}
\item {Grp. gram.:v. t.}
\end{itemize}
Causar atrophia a.
\section{Atróphico}
\begin{itemize}
\item {Grp. gram.:adj.}
\end{itemize}
Que padece atrophia.
\section{Atropilhar}
\begin{itemize}
\item {Grp. gram.:v. t.}
\end{itemize}
Reunir (cavallos) em tropilha.
\section{Atropina}
\begin{itemize}
\item {Grp. gram.:f.}
\end{itemize}
Alcaloide, extrahido da belladona.
(Do \textunderscore átropa\textunderscore , parte do nome botânico da belladona)
\section{Átropos}
\begin{itemize}
\item {Grp. gram.:f.}
\end{itemize}
\begin{itemize}
\item {Proveniência:(De \textunderscore Atropos\textunderscore , n. p. myth.)}
\end{itemize}
Borboleta nocturna.
\section{Atróptero}
\begin{itemize}
\item {Grp. gram.:adj.}
\end{itemize}
\begin{itemize}
\item {Proveniência:(Do lat. \textunderscore ater\textunderscore  + gr. \textunderscore pteron\textunderscore )}
\end{itemize}
Diz-se das aves que têm asas negras.
\section{Atroviscado}
\begin{itemize}
\item {Grp. gram.:adj.}
\end{itemize}
\begin{itemize}
\item {Utilização:Prov.}
\end{itemize}
\begin{itemize}
\item {Utilização:trasm.}
\end{itemize}
Que tem côr da casca do trovisco maçado.
E diz-se do pão, quando um calor muito forte e repentino lhe torna, no forno, ennegrecida e amargosa a côdea.
\section{Atroz}
\begin{itemize}
\item {Grp. gram.:adj.}
\end{itemize}
\begin{itemize}
\item {Proveniência:(Lat. \textunderscore atrox\textunderscore , de \textunderscore ater\textunderscore )}
\end{itemize}
Feroz; muito cruel.
Tormentoso.
Monstruoso: \textunderscore crime atroz\textunderscore .
\section{Atrozmente}
\begin{itemize}
\item {Grp. gram.:adv.}
\end{itemize}
De modo \textunderscore atroz\textunderscore .
\section{Atrusar}
\begin{itemize}
\item {Grp. gram.:v. t.}
\end{itemize}
\begin{itemize}
\item {Utilização:Fam.}
\end{itemize}
Encafuar; encaixar. Cf. Castilho, \textunderscore Doente de Scisma\textunderscore .
(Relaciona-se provavelmente com \textunderscore intruso\textunderscore )
\section{Atrutado}
\begin{itemize}
\item {Grp. gram.:adj.}
\end{itemize}
Que tem malhas como a truta.
\section{Attanger}
\begin{itemize}
\item {Grp. gram.:v. t.}
\end{itemize}
\begin{itemize}
\item {Utilização:Ant.}
\end{itemize}
O mesmo que \textunderscore atingir\textunderscore .
\section{Attélabo}
\begin{itemize}
\item {Grp. gram.:m.}
\end{itemize}
\begin{itemize}
\item {Proveniência:(Gr. \textunderscore attelabos\textunderscore )}
\end{itemize}
Insecto, espécie de pequeno gafanhoto.
\section{Attença}
\begin{itemize}
\item {Grp. gram.:f.}
\end{itemize}
Espera; expectativa:«\textunderscore ...na attença de outras galas\textunderscore ». Filinto, XII, 254.
\section{Attenção}
\begin{itemize}
\item {Grp. gram.:f.}
\end{itemize}
\begin{itemize}
\item {Proveniência:(Lat. \textunderscore attentio\textunderscore )}
\end{itemize}
Acto de applicar o espírito a.
Cuidado.
Estudo.
Urbanidade: \textunderscore tratar alguém com attenção\textunderscore .
\section{Attenciosamente}
\begin{itemize}
\item {Grp. gram.:adv.}
\end{itemize}
De modo \textunderscore attencioso\textunderscore .
Delicadamente, urbanamente.
\section{Attencioso}
\begin{itemize}
\item {Grp. gram.:adj.}
\end{itemize}
Feito com attenção: \textunderscore estudo attencioso\textunderscore .
Que presta attenção.
Delicado, urbano.
\section{Attenda}
\begin{itemize}
\item {Grp. gram.:f.}
\end{itemize}
\begin{itemize}
\item {Utilização:Ant.}
\end{itemize}
\begin{itemize}
\item {Proveniência:(De \textunderscore attender\textunderscore )}
\end{itemize}
O mesmo que \textunderscore attença\textunderscore .
\section{Attender}
\begin{itemize}
\item {Grp. gram.:v. t.}
\end{itemize}
\begin{itemize}
\item {Utilização:Ant.}
\end{itemize}
\begin{itemize}
\item {Proveniência:(Lat. \textunderscore attendere\textunderscore )}
\end{itemize}
Dar attenção a: \textunderscore attender um aviso\textunderscore .
Advertir.
Tomar em consideração.
Observar.
Deferir, despachar: \textunderscore o ministro attendeu o requerente\textunderscore .
Esperar, aguardar.
\section{Attendível}
\begin{itemize}
\item {Grp. gram.:adj.}
\end{itemize}
\begin{itemize}
\item {Proveniência:(De \textunderscore attender\textunderscore )}
\end{itemize}
Que merece attenção.
\section{Attentadamente}
\begin{itemize}
\item {Grp. gram.:adv.}
\end{itemize}
O mesmo que \textunderscore attentamente\textunderscore .
\section{Attentado}
\begin{itemize}
\item {Grp. gram.:m.}
\end{itemize}
\begin{itemize}
\item {Proveniência:(De \textunderscore attentar\textunderscore )}
\end{itemize}
Acção criminosa; offensa ás leis ou á moral.
\section{Attentamente}
\begin{itemize}
\item {Grp. gram.:adv.}
\end{itemize}
\begin{itemize}
\item {Proveniência:(De \textunderscore attento\textunderscore )}
\end{itemize}
Com attenção.
\section{Attentar}
\begin{itemize}
\item {Grp. gram.:v. t.  e  i.}
\end{itemize}
\begin{itemize}
\item {Utilização:Bras. do N}
\end{itemize}
\begin{itemize}
\item {Proveniência:(De \textunderscore attento\textunderscore )}
\end{itemize}
Vêr com attenção; observar bem.
Considerar.
Cuidar de.
Reflectir.
Irritar.
Atormentar.
\section{Attentar}
\begin{itemize}
\item {Grp. gram.:v. t.}
\end{itemize}
\begin{itemize}
\item {Grp. gram.:V. i.}
\end{itemize}
\begin{itemize}
\item {Proveniência:(Lat. \textunderscore attentare\textunderscore )}
\end{itemize}
Intentar.
Commeter attentado.
\section{Attrahimento}
\begin{itemize}
\item {Grp. gram.:m.}
\end{itemize}
(V.atracção)
\section{Attrahir}
\begin{itemize}
\item {Grp. gram.:v. t.}
\end{itemize}
\begin{itemize}
\item {Proveniência:(Lat. \textunderscore attrahere\textunderscore )}
\end{itemize}
Trazer para si; puxar para si: \textunderscore attrahir a attenção\textunderscore .
Chamar.
Induzir: \textunderscore attrahir para a desgraça\textunderscore .
Suscitar.
Fazer adherir a uma opinião.
\section{Attrectação}
\begin{itemize}
\item {Grp. gram.:f.}
\end{itemize}
\begin{itemize}
\item {Utilização:Des.}
\end{itemize}
\begin{itemize}
\item {Proveniência:(Lat. \textunderscore attrectalis\textunderscore )}
\end{itemize}
O mesmo que \textunderscore apropriação\textunderscore . Cf. Cortesão, \textunderscore Subs.\textunderscore 
\section{Attreito}
\begin{itemize}
\item {Grp. gram.:adj.}
\end{itemize}
\begin{itemize}
\item {Proveniência:(Lat. \textunderscore attractus\textunderscore )}
\end{itemize}
Que tem inclinação para alguma coisa: \textunderscore attreito a jogar\textunderscore .
Costumado.
Exposto: \textunderscore attreito a indigestões\textunderscore .
\section{Attribuição}
\begin{itemize}
\item {fónica:bu-i}
\end{itemize}
\begin{itemize}
\item {Grp. gram.:f.}
\end{itemize}
Acto de \textunderscore attribuir\textunderscore .
\section{Attribuidor}
\begin{itemize}
\item {fónica:bu-i}
\end{itemize}
\begin{itemize}
\item {Grp. gram.:m.}
\end{itemize}
Aquelle que attribue.
\section{Attribuir}
\begin{itemize}
\item {Grp. gram.:v. t.}
\end{itemize}
\begin{itemize}
\item {Proveniência:(Lat. \textunderscore attribuere\textunderscore )}
\end{itemize}
Imputar, referir: \textunderscore attribuir defeitos a alguém\textunderscore .
Conferir.
Apropriar.
\section{Attribuível}
\begin{itemize}
\item {Grp. gram.:adj.}
\end{itemize}
Que se deve ou se póde attribuir.
\section{Attributivo}
\begin{itemize}
\item {Grp. gram.:adj.}
\end{itemize}
\begin{itemize}
\item {Proveniência:(De \textunderscore attribuir\textunderscore )}
\end{itemize}
Que attribue.
Que indica um attributo.
\section{Attributo}
\begin{itemize}
\item {Grp. gram.:m.}
\end{itemize}
\begin{itemize}
\item {Utilização:Gram.}
\end{itemize}
\begin{itemize}
\item {Proveniência:(Lat. \textunderscore attributus\textunderscore )}
\end{itemize}
Aquillo que é próprio de alguém ou de alguma coisa.
Aquillo que se affirma ou se nega do sujeito.
Propriedade, qualidade.
Sýmbolo.
\section{Attrição}
\begin{itemize}
\item {Grp. gram.:f.}
\end{itemize}
\begin{itemize}
\item {Utilização:Theol.}
\end{itemize}
\begin{itemize}
\item {Proveniência:(Lat. \textunderscore attritio\textunderscore )}
\end{itemize}
Attrito.
Contracção (do estômago).
Pequeno ferimento.
Desgaste.
Pesar de têr offendido a Divindade, causado pelo medo da punição.
\section{Attrito}
\begin{itemize}
\item {Grp. gram.:m.}
\end{itemize}
\begin{itemize}
\item {Grp. gram.:Pl.}
\end{itemize}
\begin{itemize}
\item {Grp. gram.:Adj.}
\end{itemize}
\begin{itemize}
\item {Proveniência:(Lat. \textunderscore attritus\textunderscore )}
\end{itemize}
Contacto de dois corpos ásperos ou duros, roçando um pelo outro; fricção.
Difficuldades: \textunderscore encontrar attritos\textunderscore .
Que tem attrição: \textunderscore peccador attrito\textunderscore .
\section{Atuar}
\begin{itemize}
\item {Grp. gram.:v. t.}
\end{itemize}
Tratar por tu.
\section{Atuarro}
\begin{itemize}
\item {Grp. gram.:m.}
\end{itemize}
Pequeno atum, que ainda não desova.
\section{Atucanar}
\begin{itemize}
\item {Grp. gram.:v. t.}
\end{itemize}
\begin{itemize}
\item {Utilização:Bras. do N}
\end{itemize}
Incommodar com insistência.
Irritar.
\section{Atueira}
\begin{itemize}
\item {Grp. gram.:f.}
\end{itemize}
Rede para pescar atuns.
\section{Atufar}
\begin{itemize}
\item {Grp. gram.:v. t.}
\end{itemize}
\begin{itemize}
\item {Proveniência:(De \textunderscore tufo\textunderscore )}
\end{itemize}
Entufar.
Encher.
Mergulhar.
Lançar dentro.
\section{Atuir}
\begin{itemize}
\item {Grp. gram.:v. t.}
\end{itemize}
\begin{itemize}
\item {Utilização:Prov.}
\end{itemize}
\begin{itemize}
\item {Utilização:trasm.}
\end{itemize}
Obstruir; entupir.
\section{Atulhadamente}
\begin{itemize}
\item {Grp. gram.:adv.}
\end{itemize}
Com atulhamento.
\section{Atulhado}
\begin{itemize}
\item {Grp. gram.:adj.}
\end{itemize}
Entulhado, cheio.
\section{Atulhamento}
\begin{itemize}
\item {Grp. gram.:m.}
\end{itemize}
Acto ou effeito de \textunderscore atulhar\textunderscore .
\section{Atulhar}
\begin{itemize}
\item {Grp. gram.:v. t.}
\end{itemize}
O mesmo que \textunderscore entulhar\textunderscore .
\section{Atulho}
\begin{itemize}
\item {Grp. gram.:m.}
\end{itemize}
Acto de \textunderscore atulhar\textunderscore ; atulhamento. Cf. Filinto, IX, 146.
\section{Atum}
\begin{itemize}
\item {Grp. gram.:m.}
\end{itemize}
\begin{itemize}
\item {Proveniência:(Do lat. \textunderscore thunnus\textunderscore )}
\end{itemize}
Peixe, da fam. dos escômbridas.
\section{Atumultuador}
\begin{itemize}
\item {Grp. gram.:m.}
\end{itemize}
Aquelle que atumultua.
\section{Atumultuar}
\begin{itemize}
\item {Grp. gram.:v. t.}
\end{itemize}
Pôr em tumulto; amotinar.
\section{Atundir}
\begin{itemize}
\item {Grp. gram.:v. t.}
\end{itemize}
(V.contundir)
\section{Atundo}
\begin{itemize}
\item {Grp. gram.:m.}
\end{itemize}
Espécie de gaiola africana. Cf. Serpa Pinto, I, 233.
\section{Atuneira}
\begin{itemize}
\item {Grp. gram.:f.}
\end{itemize}
Dorna grande, para salgar atum.
\section{Atupir}
\begin{itemize}
\item {Grp. gram.:v. t.}
\end{itemize}
(V.entupir)
\section{Aturá}
\begin{itemize}
\item {Grp. gram.:m.}
\end{itemize}
\begin{itemize}
\item {Utilização:Bras. do N}
\end{itemize}
Grande cêsto cylíndrico, muito alto, para transporte de productos ruraes.
(Do tupi)
\section{Aturadamente}
\begin{itemize}
\item {Grp. gram.:adv.}
\end{itemize}
\begin{itemize}
\item {Proveniência:(De \textunderscore aturar\textunderscore )}
\end{itemize}
Com perseverança.
\section{Aturado}
\begin{itemize}
\item {Grp. gram.:adj.}
\end{itemize}
Persistente, constante: \textunderscore trabalho aturado\textunderscore .
\section{Aturadoiro}
\begin{itemize}
\item {Grp. gram.:adj.}
\end{itemize}
Que póde aturar, que é resistente, duradoiro. Cf. Lapa, \textunderscore Processos de Vin.\textunderscore , 38.
\section{Aturadouro}
\begin{itemize}
\item {Grp. gram.:adj.}
\end{itemize}
Que póde aturar, que é resistente, duradoiro. Cf. Lapa, \textunderscore Processos de Vin.\textunderscore , 38.
\section{Aturador}
\begin{itemize}
\item {Grp. gram.:m.}
\end{itemize}
Aquelle que atura.
\section{Aturamento}
\begin{itemize}
\item {Grp. gram.:m.}
\end{itemize}
\begin{itemize}
\item {Utilização:Des.}
\end{itemize}
Acção de \textunderscore aturar\textunderscore .
\section{Atipia}
\begin{itemize}
\item {Grp. gram.:f.}
\end{itemize}
\begin{itemize}
\item {Utilização:Med.}
\end{itemize}
\begin{itemize}
\item {Proveniência:(Do gr. \textunderscore a\textunderscore  priv. + \textunderscore tupos\textunderscore )}
\end{itemize}
Irregularidade nos accessos de moléstias periódicas.
\section{Atípico}
\begin{itemize}
\item {Grp. gram.:adj.}
\end{itemize}
\begin{itemize}
\item {Proveniência:(De \textunderscore atypia\textunderscore )}
\end{itemize}
Diz-se das febres intermittentes, cujos accessos não têm regularidade.
\section{Aturar}
\begin{itemize}
\item {Grp. gram.:v. t.}
\end{itemize}
\begin{itemize}
\item {Grp. gram.:V. i.}
\end{itemize}
\begin{itemize}
\item {Proveniência:(Do lat. \textunderscore obturare\textunderscore ?)}
\end{itemize}
Supportar; soffrer com resignação; tolerar: \textunderscore aturar a impertinência da espôsa\textunderscore .
Conservar.
Prolongar.
Sustentar.
Perseverar; continuar.
\section{Aturável}
\begin{itemize}
\item {Grp. gram.:adj.}
\end{itemize}
Que se póde \textunderscore aturar\textunderscore .
\section{Aturdido}
\begin{itemize}
\item {Grp. gram.:adj.}
\end{itemize}
Atordoado:«\textunderscore o bom Lara aturdido ficou\textunderscore ». Dinis. \textunderscore Hyssope\textunderscore .
\section{Aturdidor}
\begin{itemize}
\item {Grp. gram.:adj.}
\end{itemize}
Que aturde, que estonteia.
\section{Aturdimento}
\begin{itemize}
\item {Grp. gram.:m.}
\end{itemize}
Effeito de \textunderscore aturdir\textunderscore .
\section{Aturdir}
\begin{itemize}
\item {Grp. gram.:v. t.}
\end{itemize}
\begin{itemize}
\item {Proveniência:(Do lat. hyp. \textunderscore extordire\textunderscore )}
\end{itemize}
Perturbar; estontear.
Espantar; causar maravilha a.
Intimidar.
Tornar irreflexivo.
\section{Aturrear}
\begin{itemize}
\item {Grp. gram.:v. i.}
\end{itemize}
\begin{itemize}
\item {Utilização:Prov.}
\end{itemize}
\begin{itemize}
\item {Utilização:trasm.}
\end{itemize}
Fazer muito barulho aos ouvidos de alguém.
(Relaciona-se com \textunderscore turra\textunderscore ?)
\section{Atutar}
\begin{itemize}
\item {Grp. gram.:v. i.}
\end{itemize}
\begin{itemize}
\item {Utilização:Prov.}
\end{itemize}
\begin{itemize}
\item {Utilização:minh.}
\end{itemize}
Amolgar.
Amachucar.
\section{Atypia}
\begin{itemize}
\item {Grp. gram.:f.}
\end{itemize}
\begin{itemize}
\item {Utilização:Med.}
\end{itemize}
\begin{itemize}
\item {Proveniência:(Do gr. \textunderscore a\textunderscore  priv. + \textunderscore tupos\textunderscore )}
\end{itemize}
Irregularidade nos accessos de moléstias periódicas.
\section{Atýpico}
\begin{itemize}
\item {Grp. gram.:adj.}
\end{itemize}
\begin{itemize}
\item {Proveniência:(De \textunderscore atypia\textunderscore )}
\end{itemize}
Diz-se das febres intermittentes, cujos accessos não têm regularidade.
\section{Auaduri}
\begin{itemize}
\item {Grp. gram.:m.}
\end{itemize}
\begin{itemize}
\item {Utilização:Bras}
\end{itemize}
O mesmo que \textunderscore abiorama\textunderscore .
\section{Auatá}
\begin{itemize}
\item {Grp. gram.:adv.}
\end{itemize}
\begin{itemize}
\item {Utilização:Bras}
\end{itemize}
Ao acaso.
\section{Auati}
\begin{itemize}
\item {Grp. gram.:m.}
\end{itemize}
Árvore da Guiana, de casca e sementes medicinaes.
\section{Aubriécia}
\begin{itemize}
\item {Grp. gram.:f.}
\end{itemize}
\begin{itemize}
\item {Proveniência:(De \textunderscore Aubriete\textunderscore , n. p.)}
\end{itemize}
Gênero de plantas crucíferas.
\section{Aução}
\begin{itemize}
\item {Grp. gram.:f.}
\end{itemize}
\begin{itemize}
\item {Utilização:Ant.}
\end{itemize}
O mesmo que \textunderscore acção\textunderscore . Cf. \textunderscore Eufrosina\textunderscore , 317, 319 e 321; \textunderscore Peregrinação\textunderscore , XXX; Barros, etc.
\section{Aucuba}
\begin{itemize}
\item {Grp. gram.:f.}
\end{itemize}
\begin{itemize}
\item {Proveniência:(T. japon.)}
\end{itemize}
Gênero de plantas ampelídeas.
\section{Aucúpio}
\begin{itemize}
\item {Grp. gram.:m.}
\end{itemize}
\begin{itemize}
\item {Proveniência:(Lat. \textunderscore aucupium\textunderscore )}
\end{itemize}
Caça de aves, por meio de armadilhas.
\section{Audácia}
\begin{itemize}
\item {Grp. gram.:f.}
\end{itemize}
\begin{itemize}
\item {Proveniência:(Lat. \textunderscore audacia\textunderscore )}
\end{itemize}
Impulso da alma para actos diffíceis ou perigosos.
Ousadia.
Valor; intrepidez.
Atrevimento; petulância.
\section{Audaciosamente}
\begin{itemize}
\item {Grp. gram.:adv.}
\end{itemize}
Com audácia.
\section{Audacioso}
\begin{itemize}
\item {Grp. gram.:adj.}
\end{itemize}
Que tem audácia; audaz.
Que requere audácia.
\section{Audacíssimo}
\begin{itemize}
\item {Grp. gram.:adj.}
\end{itemize}
(\textunderscore sup. de audaz\textunderscore )
\section{Audaz}
\begin{itemize}
\item {Grp. gram.:adj.}
\end{itemize}
\begin{itemize}
\item {Proveniência:(Lat. \textunderscore audax\textunderscore )}
\end{itemize}
Que tem audácia: \textunderscore homem audaz\textunderscore .
Em que há audácia: \textunderscore procedimento audaz\textunderscore .
\section{Audazmente}
\begin{itemize}
\item {Grp. gram.:adv.}
\end{itemize}
De modo \textunderscore audaz\textunderscore .
\section{Aúde}
\begin{itemize}
\item {Grp. gram.:f.}
\end{itemize}
\begin{itemize}
\item {Utilização:Prov.}
\end{itemize}
\begin{itemize}
\item {Utilização:beir.}
\end{itemize}
O mesmo que \textunderscore agude\textunderscore .
\section{Aúdes}
\begin{itemize}
\item {Grp. gram.:interj.}
\end{itemize}
(Voz expletiva, us. na loc. fam. \textunderscore aúdes\textunderscore , \textunderscore que é coisa leve\textunderscore , como quem diz: \textunderscore passe por lá muito bem\textunderscore )
(Colhido em Lamego)
\section{Audição}
\begin{itemize}
\item {Grp. gram.:f.}
\end{itemize}
\begin{itemize}
\item {Proveniência:(Lat. \textunderscore auditio\textunderscore )}
\end{itemize}
Percepção dos sons por meio do ouvido.
Acto de ouvir, de escutar: \textunderscore audição de uma leitura\textunderscore .
\section{Audiença}
\begin{itemize}
\item {Grp. gram.:f.}
\end{itemize}
\begin{itemize}
\item {Utilização:Ant.}
\end{itemize}
O mesmo que \textunderscore audiência\textunderscore :«\textunderscore vem todos cá á audiença\textunderscore ». G. Vicente, \textunderscore Juiz da Beira\textunderscore .
\section{Audiência}
\begin{itemize}
\item {Grp. gram.:f.}
\end{itemize}
\begin{itemize}
\item {Proveniência:(Lat. \textunderscore audientia\textunderscore )}
\end{itemize}
Audição.
Attenção, que se dá a quem fala.
Acção de receber as pessôas que nos pretendem falar.
Sessão de tribunal.
\section{Audiente}
\begin{itemize}
\item {Grp. gram.:adj.}
\end{itemize}
\begin{itemize}
\item {Utilização:Des.}
\end{itemize}
\begin{itemize}
\item {Proveniência:(Lat. \textunderscore audiens\textunderscore )}
\end{itemize}
Que ouve.
\section{Auditivo}
\begin{itemize}
\item {Grp. gram.:adj.}
\end{itemize}
\begin{itemize}
\item {Proveniência:(Do lat. \textunderscore audire\textunderscore )}
\end{itemize}
Que pertence ao ouvido.
\section{Audito}
\begin{itemize}
\item {Grp. gram.:m.}
\end{itemize}
\begin{itemize}
\item {Proveniência:(Lat. \textunderscore anditus\textunderscore )}
\end{itemize}
Acção de ouvir.
\section{Auditor}
\begin{itemize}
\item {Grp. gram.:m.}
\end{itemize}
\begin{itemize}
\item {Proveniência:(Lat. \textunderscore auditor\textunderscore )}
\end{itemize}
Aquelle que ouve.
Magistrado, encarregado de informar uma repartição sôbre a applicação das leis a casos occorrentes; ouvidor.
Magistrado judicial, adjunto a tribunaes militares ou de marinha.
Assessor do núncio.
\section{Auditoria}
\begin{itemize}
\item {Grp. gram.:f.}
\end{itemize}
Cargo de auditor.
Casa, lugar, onde o auditor exerce suas funcções.
\section{Auditório}
\begin{itemize}
\item {Grp. gram.:m.}
\end{itemize}
\begin{itemize}
\item {Grp. gram.:Adj.}
\end{itemize}
\begin{itemize}
\item {Utilização:Des.}
\end{itemize}
\begin{itemize}
\item {Proveniência:(Lat. \textunderscore auditorium\textunderscore )}
\end{itemize}
Reunião de pessôas, para ouvirem oradores ou para assistirem a uma audiência ou sessão.
Os ouvintes.
Lugar, onde se reúnem os ouvintes.
O mesmo que \textunderscore auditivo\textunderscore .
\section{Audível}
\begin{itemize}
\item {Grp. gram.:adj.}
\end{itemize}
\begin{itemize}
\item {Proveniência:(Do lat. \textunderscore audire\textunderscore )}
\end{itemize}
Que póde sêr ouvido.
\section{Auferir}
\begin{itemize}
\item {Grp. gram.:v. t.}
\end{itemize}
\begin{itemize}
\item {Proveniência:(Lat. \textunderscore auferre\textunderscore )}
\end{itemize}
Colher: \textunderscore auferir desenganos\textunderscore .
Lucrar: \textunderscore auferir vantagens\textunderscore .
\section{Auferível}
\begin{itemize}
\item {Grp. gram.:adj.}
\end{itemize}
Que se póde auferir.
\section{Auga}
\textunderscore f.\textunderscore  (e der.)
(Corr. pop. e \textunderscore ant.\textunderscore  de \textunderscore água\textunderscore , etc.)
\section{Augada}
\begin{itemize}
\item {Grp. gram.:f.}
\end{itemize}
\begin{itemize}
\item {Utilização:Pop.}
\end{itemize}
\begin{itemize}
\item {Proveniência:(De \textunderscore auga\textunderscore )}
\end{itemize}
O mesmo que \textunderscore aguada\textunderscore .
\section{Augalhar}
\begin{itemize}
\item {Grp. gram.:v. t.}
\end{itemize}
\begin{itemize}
\item {Utilização:T. da Bairrada}
\end{itemize}
\begin{itemize}
\item {Proveniência:(De \textunderscore auga\textunderscore )}
\end{itemize}
Borrifar no estendedoiro (roupa lavada, para a còrar).
\section{Auge}
\begin{itemize}
\item {Grp. gram.:m.}
\end{itemize}
O ponto mais elevado; o maior grau; apogeu: \textunderscore no auge do desespêro\textunderscore .
(Ár. \textunderscore auje\textunderscore )
\section{Auges}
\begin{itemize}
\item {Grp. gram.:m. pl.}
\end{itemize}
Aborígenes, que habitaram no Maranhão.
\section{Augir}
\begin{itemize}
\item {Grp. gram.:v. t.}
\end{itemize}
Tocar o auge de:«\textunderscore augem a perfeição\textunderscore ». J. Castilho, \textunderscore Ermitério\textunderscore , 113.
\section{Augmento}
\textunderscore m.\textunderscore  (e der.)
(V. \textunderscore aumento\textunderscore , etc.)
\section{Augueiro}
\begin{itemize}
\item {Grp. gram.:m.}
\end{itemize}
\begin{itemize}
\item {Utilização:Pop.}
\end{itemize}
\begin{itemize}
\item {Proveniência:(De \textunderscore auga\textunderscore )}
\end{itemize}
O mesmo que \textunderscore agueiro\textunderscore .
\section{Augural}
\begin{itemize}
\item {Grp. gram.:adj.}
\end{itemize}
\begin{itemize}
\item {Proveniência:(Lat. \textunderscore auguralis\textunderscore )}
\end{itemize}
Relativo ao áugure.
\section{Augurar}
\begin{itemize}
\item {Grp. gram.:v. t.}
\end{itemize}
\begin{itemize}
\item {Proveniência:(Lat. \textunderscore augurari\textunderscore )}
\end{itemize}
Presagiar; predizer: \textunderscore augurar desgraças\textunderscore .
Conjecturar.
\section{Auguratório}
\begin{itemize}
\item {Grp. gram.:m.}
\end{itemize}
\begin{itemize}
\item {Proveniência:(Lat. \textunderscore auguratorium\textunderscore )}
\end{itemize}
Lugar, em que se reuniam os áugures.
\section{Auguratriz}
\begin{itemize}
\item {Grp. gram.:f.}
\end{itemize}
\begin{itemize}
\item {Grp. gram.:Adj. f.}
\end{itemize}
\begin{itemize}
\item {Proveniência:(Lat. \textunderscore auguratrix\textunderscore )}
\end{itemize}
Mulher, que exercia funcções de áugures.
Que augura: \textunderscore bruxa auguratriz\textunderscore .
\section{Áugure}
\begin{itemize}
\item {Grp. gram.:m.}
\end{itemize}
\begin{itemize}
\item {Proveniência:(Lat. \textunderscore augur\textunderscore )}
\end{itemize}
Sacerdote romano, que tirava preságios do vôo e do canto das aves.
Adivinho.
Agoireiro.
\section{Augúrio}
\begin{itemize}
\item {Grp. gram.:m.}
\end{itemize}
\begin{itemize}
\item {Proveniência:(Lat. \textunderscore augurium\textunderscore )}
\end{itemize}
Prognóstico; agoiro.
\section{Augustal}
\begin{itemize}
\item {Grp. gram.:m.}
\end{itemize}
\begin{itemize}
\item {Proveniência:(Lat. \textunderscore augustalis\textunderscore )}
\end{itemize}
Antiga moéda de oiro, na Sicília.
\section{Augustamente}
\begin{itemize}
\item {Grp. gram.:adv.}
\end{itemize}
De modo \textunderscore augusto\textunderscore .
\section{Augustano}
\begin{itemize}
\item {Grp. gram.:adj.}
\end{itemize}
\begin{itemize}
\item {Proveniência:(Lat. \textunderscore augustanus\textunderscore )}
\end{itemize}
Relativo ao Imperador Augusto ou ao seu tempo.
\section{Augustinho}
\begin{itemize}
\item {Grp. gram.:m.}
\end{itemize}
\begin{itemize}
\item {Utilização:Typ.}
\end{itemize}
\begin{itemize}
\item {Proveniência:(Lat. \textunderscore augustinus\textunderscore )}
\end{itemize}
Um dos typos de letra.
\section{Augusto}
\begin{itemize}
\item {Grp. gram.:adj.}
\end{itemize}
\begin{itemize}
\item {Proveniência:(Lat. \textunderscore augustus\textunderscore )}
\end{itemize}
Respeitável.
Magnifico.
Solenne.
\section{Auiqui}
\begin{itemize}
\item {Grp. gram.:m.}
\end{itemize}
\begin{itemize}
\item {Utilização:Bras}
\end{itemize}
Espécie de formiga das regiões do Amazonas.
\section{Aula}
\begin{itemize}
\item {Grp. gram.:f.}
\end{itemize}
\begin{itemize}
\item {Utilização:Ant.}
\end{itemize}
\begin{itemize}
\item {Proveniência:(Lat. \textunderscore aula\textunderscore )}
\end{itemize}
Sala, em que se recebem lições.
Prelecção.
Côrte.
Pátio.
A parte mais interior do santuário ou da capella-mór.
\section{Aulátrida}
\begin{itemize}
\item {Grp. gram.:f.}
\end{itemize}
\begin{itemize}
\item {Proveniência:(Do gr. \textunderscore aulatris\textunderscore , \textunderscore aulatridos\textunderscore )}
\end{itemize}
O mesmo que \textunderscore auletriz\textunderscore .
\section{Aulete}
\begin{itemize}
\item {Grp. gram.:m.}
\end{itemize}
Tocador de aulo.
\section{Aulética}
\begin{itemize}
\item {Grp. gram.:f.}
\end{itemize}
Arte de tocar aulo.
\section{Aulético}
\begin{itemize}
\item {Grp. gram.:adj.}
\end{itemize}
Relativo ao aulo.
\section{Auletriz}
\begin{itemize}
\item {Grp. gram.:f.}
\end{itemize}
Tocadora de aulo. Cf. Latino. \textunderscore Or. da Corôa\textunderscore , CXLIII.
\section{Aulicismo}
\begin{itemize}
\item {Grp. gram.:m.}
\end{itemize}
\begin{itemize}
\item {Utilização:Neol.}
\end{itemize}
Qualidade de \textunderscore áulico\textunderscore . Cf. Rui Barbosa, \textunderscore Réplica\textunderscore , 157.
\section{Áulico}
\begin{itemize}
\item {Grp. gram.:adj.}
\end{itemize}
\begin{itemize}
\item {Grp. gram.:M.}
\end{itemize}
\begin{itemize}
\item {Proveniência:(Lat. \textunderscore aulicus\textunderscore )}
\end{itemize}
Relativo á côrte.
Cortesão.
\section{Aulido}
\begin{itemize}
\item {Grp. gram.:m.}
\end{itemize}
Grito; uivo; berro de animaes.
(Cast. \textunderscore aullido\textunderscore )
\section{Aulista}
\begin{itemize}
\item {Grp. gram.:m.}
\end{itemize}
Estudante; aquelle que frequenta aulas.
\section{Aulo}
\begin{itemize}
\item {Grp. gram.:m.}
\end{itemize}
Designação da frauta, entre os Gregos.
\section{Aulodia}
\begin{itemize}
\item {Grp. gram.:f.}
\end{itemize}
Canto, acompanhado de aulo.
\section{Aumentação}
\begin{itemize}
\item {Grp. gram.:f.}
\end{itemize}
O mesmo que \textunderscore aumento\textunderscore .
\section{Aumentador}
\begin{itemize}
\item {Grp. gram.:m.  e  adj.}
\end{itemize}
O que aumenta.
\section{Aumentar}
\begin{itemize}
\item {Grp. gram.:v. t.}
\end{itemize}
\begin{itemize}
\item {Grp. gram.:V. i.}
\end{itemize}
\begin{itemize}
\item {Proveniência:(Lat. \textunderscore augmentare\textunderscore )}
\end{itemize}
Tornar maior; amplificar: \textunderscore aumentar affectos\textunderscore .
Engrandecer.
Tornar próspero: \textunderscore a fortuna aumentou-lhe o negócio\textunderscore .
Aggravar.
Progredir.
Crescer.
Melhorar.
Prosperar.
\section{Aumentar}
\begin{itemize}
\item {Grp. gram.:v. t.}
\end{itemize}
\begin{itemize}
\item {Utilização:Prov.}
\end{itemize}
O mesmo que \textunderscore inventar\textunderscore  e \textunderscore ementar\textunderscore ^2.
\section{Aumentativo}
\begin{itemize}
\item {Grp. gram.:adj.}
\end{itemize}
Que aumenta.
\section{Aumentável}
\begin{itemize}
\item {Grp. gram.:adj.}
\end{itemize}
Que póde sêr aumentado.
\section{Aumento}
\begin{itemize}
\item {Grp. gram.:m.}
\end{itemize}
\begin{itemize}
\item {Proveniência:(Lat. \textunderscore augmentum\textunderscore )}
\end{itemize}
Acto ou effeito de aumentar.
\section{Aunar}
\begin{itemize}
\item {fónica:a-u}
\end{itemize}
\begin{itemize}
\item {Grp. gram.:v. t.}
\end{itemize}
\begin{itemize}
\item {Utilização:Des.}
\end{itemize}
\begin{itemize}
\item {Proveniência:(De \textunderscore uno\textunderscore )}
\end{itemize}
Reunir num todo.
\section{Aura}
\begin{itemize}
\item {Grp. gram.:f.}
\end{itemize}
\begin{itemize}
\item {Utilização:Med.}
\end{itemize}
\begin{itemize}
\item {Proveniência:(Lat. \textunderscore aura\textunderscore )}
\end{itemize}
Vento brando e suave; aragem.
Popularidade, bôa acceitação.
Fama, rumor.
Hálito.
Phenómenos, que precedem ataque nervoso, especialmente epiléptico.
\section{Aurana}
\begin{itemize}
\item {Grp. gram.:f.}
\end{itemize}
\begin{itemize}
\item {Utilização:Bras}
\end{itemize}
\begin{itemize}
\item {Proveniência:(Do guar. \textunderscore ai\textunderscore , chaga, e \textunderscore ran\textunderscore , semelhante)}
\end{itemize}
Espécie de morpheia, que se manifesta por manchas, que lavram todo o corpo.
\section{Auranciáceas}
\begin{itemize}
\item {Grp. gram.:f. pl.}
\end{itemize}
\begin{itemize}
\item {Proveniência:(De \textunderscore auranciáceo\textunderscore )}
\end{itemize}
Família de plantas dicotyledóneas polypétalas, a que servo de typo a laranjeira.
\section{Auranciáceo}
\begin{itemize}
\item {Grp. gram.:adj.}
\end{itemize}
\begin{itemize}
\item {Proveniência:(De \textunderscore aurantium\textunderscore , n. bot. da laranjeira)}
\end{itemize}
Semelhante ou relativo á laranjeira.
\section{Aurantiáceas}
\begin{itemize}
\item {Grp. gram.:f. pl.}
\end{itemize}
O mesmo que \textunderscore auranciáceas\textunderscore .
\section{Aurantiáceo}
\begin{itemize}
\item {Grp. gram.:adj.}
\end{itemize}
O mesmo que \textunderscore auranciáceo\textunderscore .
\section{Aurantina}
\begin{itemize}
\item {Grp. gram.:f.}
\end{itemize}
Princípio amargo das laranjas.
(Cp. \textunderscore aurantiáceo\textunderscore )
\section{Aurato}
\begin{itemize}
\item {Grp. gram.:m.}
\end{itemize}
\begin{itemize}
\item {Proveniência:(Do lat. \textunderscore aurum\textunderscore )}
\end{itemize}
Sal, produzido pela combinação do ácido áurico com uma base.
\section{Aurécia}
\begin{itemize}
\item {Grp. gram.:f.}
\end{itemize}
\begin{itemize}
\item {Utilização:Prov.}
\end{itemize}
\begin{itemize}
\item {Utilização:beir.}
\end{itemize}
A aragem fresca dos campos depois de regados, em manhans de verão.
(Cp. \textunderscore aura\textunderscore )
\section{Aurélia}
\begin{itemize}
\item {Grp. gram.:f.}
\end{itemize}
\begin{itemize}
\item {Proveniência:(Do lat. \textunderscore aurum\textunderscore )}
\end{itemize}
Gênero de plantas herbáceas.
Chrysálida.
Gênero de zoóphytos.
\section{Áureo}
\begin{itemize}
\item {Grp. gram.:adj.}
\end{itemize}
\begin{itemize}
\item {Grp. gram.:M.}
\end{itemize}
\begin{itemize}
\item {Proveniência:(Lat. \textunderscore aureus\textunderscore )}
\end{itemize}
Que tem côr de oiro; que é de oiro.
Brilhante: \textunderscore idade áurea da literatura\textunderscore .
Doirado.
Magnífico.
Nobre.
De muito valor.
Antiga moéda de oiro portuguesa.
Moéda romana de oiro, igual a cinco denários.
\section{Auréola}
\begin{itemize}
\item {Grp. gram.:f.}
\end{itemize}
\begin{itemize}
\item {Proveniência:(Lat. \textunderscore aureola\textunderscore )}
\end{itemize}
Círculo luminoso, que orna a cabeça dos santos, nas suas imagens.
Diadema.
Glória: \textunderscore a auréola dos sábios\textunderscore .
\section{Aureolar}
\begin{itemize}
\item {Grp. gram.:v. t.}
\end{itemize}
Ornar com auréola.
Abrilhantar.
\section{Aureolar}
\begin{itemize}
\item {Grp. gram.:adj.}
\end{itemize}
Que tem fórma de auréola.
\section{Aureolização}
\begin{itemize}
\item {Grp. gram.:f.}
\end{itemize}
Acto de \textunderscore aureolizar\textunderscore .
\section{Aureolizar}
\begin{itemize}
\item {Grp. gram.:v. t.}
\end{itemize}
\begin{itemize}
\item {Utilização:Neol.}
\end{itemize}
O mesmo que \textunderscore aureolar\textunderscore ^1.
\section{Auricalco}
\begin{itemize}
\item {Grp. gram.:m.}
\end{itemize}
\begin{itemize}
\item {Proveniência:(Lat. \textunderscore auricalchum\textunderscore )}
\end{itemize}
Metal finissimo, hoje desconhecido.
\section{Auri-cerúleo}
\begin{itemize}
\item {Grp. gram.:adj.}
\end{itemize}
\begin{itemize}
\item {Proveniência:(De \textunderscore áureo\textunderscore  + \textunderscore cerúleo\textunderscore )}
\end{itemize}
Que Participa do azul e da côr do oiro.
\section{Auricídia}
\begin{itemize}
\item {Grp. gram.:f.}
\end{itemize}
Sêde ou cubiça de oiro.
\section{Auriclýpeo}
\begin{itemize}
\item {Grp. gram.:adj.}
\end{itemize}
\begin{itemize}
\item {Proveniência:(Do lat. \textunderscore aurum\textunderscore  + \textunderscore clypeus\textunderscore )}
\end{itemize}
Que tem escudo doirado. Cf. Filinto, XVI, 264.
\section{Áurico}
\begin{itemize}
\item {Grp. gram.:adj.}
\end{itemize}
\begin{itemize}
\item {Proveniência:(Do lat. \textunderscore aurum\textunderscore )}
\end{itemize}
Diz-se de um ácido, que é o peróxido de oiro.
\section{Auricolor}
\begin{itemize}
\item {Grp. gram.:adj.}
\end{itemize}
\begin{itemize}
\item {Proveniência:(Do lat. \textunderscore aurum\textunderscore  + \textunderscore color\textunderscore )}
\end{itemize}
Que tem côr de oiro.
\section{Aurícomo}
\begin{itemize}
\item {Grp. gram.:adj.}
\end{itemize}
\begin{itemize}
\item {Proveniência:(Lat. \textunderscore auricomus\textunderscore )}
\end{itemize}
Que tem cabellos doirados.
\section{Auricórneo}
\begin{itemize}
\item {Grp. gram.:adj.}
\end{itemize}
\begin{itemize}
\item {Utilização:Zool.}
\end{itemize}
\begin{itemize}
\item {Proveniência:(Do lat. \textunderscore aurum\textunderscore  + \textunderscore cornu\textunderscore )}
\end{itemize}
Que tem antennas amarelas, côr de oiro.
\section{Auricrinante}
\begin{itemize}
\item {Grp. gram.:adj.}
\end{itemize}
O mesmo que \textunderscore auricrinito\textunderscore .
\section{Auricrinito}
\begin{itemize}
\item {Grp. gram.:adj.}
\end{itemize}
Que tem trança doirada. Cf. Filinto, V, 56; II, 235.
\section{Aurícula}
\begin{itemize}
\item {Grp. gram.:f.}
\end{itemize}
\begin{itemize}
\item {Utilização:Bot.}
\end{itemize}
\begin{itemize}
\item {Proveniência:(Lat. \textunderscore auricula\textunderscore )}
\end{itemize}
Pavilhão do ouvido.
Cada uma das cavidades superiores do coração.
Appêndice arredondado, na base de certas fôlhas de plantas.
Tufo de pennas, na cabeça de certas aves.
Planta primulácea.
Mollusco gasterópode.
\section{Auriculado}
\begin{itemize}
\item {Grp. gram.:adj.}
\end{itemize}
Que tem aurículas.
\section{Auricular}
\begin{itemize}
\item {Grp. gram.:adj.}
\end{itemize}
\begin{itemize}
\item {Proveniência:(Lat. \textunderscore auricularis\textunderscore )}
\end{itemize}
Que diz respeito ao ouvido: \textunderscore a confissão auricular\textunderscore .
\section{Auriculífero}
\begin{itemize}
\item {Grp. gram.:adj.}
\end{itemize}
\begin{itemize}
\item {Proveniência:(Do lat. \textunderscore auricula\textunderscore  + \textunderscore ferre\textunderscore )}
\end{itemize}
Que tem aurículas.
\section{Auriculiforme}
\begin{itemize}
\item {Grp. gram.:adj.}
\end{itemize}
\begin{itemize}
\item {Proveniência:(Do lat. \textunderscore auricula\textunderscore  + \textunderscore forma\textunderscore )}
\end{itemize}
Que tem a fórma de orelha pequena.
\section{Auriculista}
\begin{itemize}
\item {Grp. gram.:m.}
\end{itemize}
\begin{itemize}
\item {Proveniência:(De \textunderscore auricula\textunderscore )}
\end{itemize}
Médico, que trata especialmente de doenças de ouvidos.
\section{Aurículo}
\begin{itemize}
\item {Grp. gram.:m.}
\end{itemize}
O mesmo que \textunderscore aurícula\textunderscore .
Mollusco, também conhecido por \textunderscore aurícula\textunderscore .
\section{Auriculoso}
\begin{itemize}
\item {Grp. gram.:adj.}
\end{itemize}
Que tem aurículas.
\section{Auri-dulce}
\begin{itemize}
\item {Grp. gram.:adj.}
\end{itemize}
Que é doce ou suave e tem côr de oiro:«\textunderscore auri-dulce humor da cepa\textunderscore ». Filinto, XI, 19.
\section{Auriense}
\begin{itemize}
\item {Grp. gram.:adj.}
\end{itemize}
Relativo a Orense, na Galliza. Cf. Herculano, \textunderscore Hist. de Port.\textunderscore , II, 426.
\section{Auri-esplendente}
\begin{itemize}
\item {Grp. gram.:adj.}
\end{itemize}
Que brilha como o oiro.
\section{Aurífero}
\begin{itemize}
\item {Grp. gram.:adj.}
\end{itemize}
\begin{itemize}
\item {Proveniência:(Lat. \textunderscore aurifer\textunderscore )}
\end{itemize}
Que contém oiro.
\section{Aurificação}
\begin{itemize}
\item {Grp. gram.:f.}
\end{itemize}
\begin{itemize}
\item {Proveniência:(De \textunderscore aurificar\textunderscore )}
\end{itemize}
Obturação dos dentes furados, feita com fôlhas de oiro.
\section{Aurificar}
\begin{itemize}
\item {Grp. gram.:v. t.}
\end{itemize}
\begin{itemize}
\item {Proveniência:(Do lat. \textunderscore aurum\textunderscore  + \textunderscore facere\textunderscore )}
\end{itemize}
Obturar com fôlhas de oiro (os dentes furados).
\section{Aurífice}
\begin{itemize}
\item {Grp. gram.:m.}
\end{itemize}
\begin{itemize}
\item {Proveniência:(Lat. \textunderscore aurifex\textunderscore )}
\end{itemize}
Aquelle que trabalha em oiro; ourives.
\section{Aurifício}
\begin{itemize}
\item {Grp. gram.:adj.}
\end{itemize}
Que fabríca oiro; que converte em oiro.
Com que se adquire oiro:«\textunderscore empresa aurificia\textunderscore ». Camillo, \textunderscore N. de Insómn.\textunderscore , IV, 84.
(Cp. \textunderscore aurífico\textunderscore )
\section{Aurífico}
\begin{itemize}
\item {Grp. gram.:adj.}
\end{itemize}
\begin{itemize}
\item {Proveniência:(Do lat. \textunderscore aurum\textunderscore  + \textunderscore facere\textunderscore )}
\end{itemize}
Que tem oiro ou a côr do oiro.
Que converte em oiro.
\section{Auriflama}
\begin{itemize}
\item {Grp. gram.:f.}
\end{itemize}
\begin{itemize}
\item {Proveniência:(T. composto á imitação do fr. \textunderscore oriflamme\textunderscore , cujo étimo não está bem determinado)}
\end{itemize}
Antigo estandarte vermelho dos reis de França.
Bandeira.
\section{Auriflamante}
\begin{itemize}
\item {Grp. gram.:adj.}
\end{itemize}
Que tem a côr da auriflamma.
\section{Auriflamma}
\begin{itemize}
\item {Grp. gram.:f.}
\end{itemize}
\begin{itemize}
\item {Proveniência:(T. composto á imitação do fr. \textunderscore oriflamme\textunderscore , cujo étimo não está bem determinado)}
\end{itemize}
Antigo estandarte vermelho dos reis de França.
Bandeira.
\section{Auriflammante}
\begin{itemize}
\item {Grp. gram.:adj.}
\end{itemize}
Que tem a côr da auriflamma.
\section{Auriforme}
\begin{itemize}
\item {Grp. gram.:adj.}
\end{itemize}
\begin{itemize}
\item {Proveniência:(Do lat. \textunderscore auris\textunderscore  + \textunderscore forma\textunderscore )}
\end{itemize}
Diz-se das conchas bivalves, que têm a fórma de orelha.
\section{Aurifrigiado}
\begin{itemize}
\item {Grp. gram.:adj.}
\end{itemize}
O mesmo que \textunderscore aurifrigiato\textunderscore .
\section{Aurifrigiato}
\begin{itemize}
\item {Grp. gram.:adj.}
\end{itemize}
\begin{itemize}
\item {Proveniência:(Do b. lat. \textunderscore aurifrigia\textunderscore , fímbria doirada)}
\end{itemize}
Orlado de oiro. Cf. \textunderscore Ritual dos Cistercienses\textunderscore .
\section{Aurifrísio}
\begin{itemize}
\item {Grp. gram.:m.}
\end{itemize}
Espécie de águia; águia marinha.
\section{Aurifulgente}
\begin{itemize}
\item {Grp. gram.:adj.}
\end{itemize}
\begin{itemize}
\item {Proveniência:(Do lat. \textunderscore aurum\textunderscore  + \textunderscore fulgere\textunderscore )}
\end{itemize}
Que brilha como oiro.
\section{Aurifúlgido}
\begin{itemize}
\item {Grp. gram.:adj.}
\end{itemize}
O mesmo que \textunderscore aurifulgente\textunderscore . Cf. Ruy Barbosa, \textunderscore Réplica\textunderscore , 157.
\section{Auriga}
\begin{itemize}
\item {Grp. gram.:m.}
\end{itemize}
\begin{itemize}
\item {Proveniência:(Lat. \textunderscore auriga\textunderscore )}
\end{itemize}
Cocheiro.
Uma das constellações boreaes.
\section{Aurigário}
\begin{itemize}
\item {Grp. gram.:m.}
\end{itemize}
\begin{itemize}
\item {Proveniência:(Lat. \textunderscore aurigarius\textunderscore )}
\end{itemize}
Aquelle que, entre os Romanos, guiava carros em corridas, vestido de quatro côres diversas, como representando as quatro estações do anno.
\section{Aurigastro}
\begin{itemize}
\item {Grp. gram.:adj.}
\end{itemize}
\begin{itemize}
\item {Proveniência:(Do lat. \textunderscore aurum\textunderscore  + \textunderscore gaster\textunderscore )}
\end{itemize}
Diz-se dos animaes, que têm o ventre amarelado.
\section{Aurigemante}
\begin{itemize}
\item {Grp. gram.:adj.}
\end{itemize}
Que brilha como oiro e pedras preciosas. Cf. Castilho, \textunderscore Fastos\textunderscore , I, 85.
\section{Aurigemmante}
\begin{itemize}
\item {Grp. gram.:adj.}
\end{itemize}
Que brilha como oiro e pedras preciosas. Cf. Castilho, \textunderscore Fastos\textunderscore , I, 85.
\section{Aurígero}
\begin{itemize}
\item {Grp. gram.:adj.}
\end{itemize}
\begin{itemize}
\item {Proveniência:(Lat. \textunderscore auriger\textunderscore )}
\end{itemize}
O mesmo que \textunderscore aurífero\textunderscore . Cf. Castilho, \textunderscore Fastos\textunderscore , II, 167.
\section{Auriginoso}
\begin{itemize}
\item {Grp. gram.:adj.}
\end{itemize}
\begin{itemize}
\item {Utilização:Med.}
\end{itemize}
\begin{itemize}
\item {Proveniência:(Lat. \textunderscore auriginosus\textunderscore )}
\end{itemize}
Que tem côr de oiro.
Diz-se da febre acompanhada de icterícia.
\section{Aurilavrado}
\begin{itemize}
\item {Grp. gram.:adj.}
\end{itemize}
Que é de oiro com lavores:«\textunderscore caixas de rapé aurilavradas\textunderscore ». Crespo, \textunderscore Miniat.\textunderscore , 10.
\section{Auriluzente}
\begin{itemize}
\item {Grp. gram.:adj.}
\end{itemize}
O mesmo que \textunderscore aurifulgente\textunderscore . Cf. Macedo, \textunderscore Oriente\textunderscore , I, 43.
\section{Aurimesclado}
\begin{itemize}
\item {Grp. gram.:adj.}
\end{itemize}
Que tem côr de oiro, á mistura. Cf. Castilho, \textunderscore Metam.\textunderscore , 150; Filinto, V, 75.
\section{Aurínia}
\begin{itemize}
\item {Grp. gram.:f.}
\end{itemize}
Gênero de plantas crucíferas.
\section{Auripurpúreo}
\begin{itemize}
\item {Grp. gram.:adj.}
\end{itemize}
Que tem côr de oiro e púrpura. Cf. Castilho, \textunderscore Fastos\textunderscore , I, 109.
\section{Aurir}
\begin{itemize}
\item {Grp. gram.:v. i.}
\end{itemize}
\begin{itemize}
\item {Utilização:T. da Bairrada}
\end{itemize}
\begin{itemize}
\item {Proveniência:(De \textunderscore aura\textunderscore . Cp. \textunderscore oirar\textunderscore ^1)}
\end{itemize}
Fugir allucinadamente.
Allucinar-se; oirar.
\section{Aurirosado}
\begin{itemize}
\item {fónica:ro}
\end{itemize}
\begin{itemize}
\item {Grp. gram.:adj.}
\end{itemize}
O mesmo que \textunderscore auriróseo\textunderscore .
\section{Auriróseo}
\begin{itemize}
\item {fónica:ró}
\end{itemize}
\begin{itemize}
\item {Grp. gram.:adj.}
\end{itemize}
\begin{itemize}
\item {Proveniência:(Do lat. \textunderscore aurum\textunderscore  + \textunderscore rosa\textunderscore )}
\end{itemize}
Que tem côr de oiro e rosa.
\section{Aurirrosado}
\begin{itemize}
\item {Grp. gram.:adj.}
\end{itemize}
O mesmo que \textunderscore auriróseo\textunderscore .
\section{Aurirróseo}
\begin{itemize}
\item {Grp. gram.:adj.}
\end{itemize}
\begin{itemize}
\item {Proveniência:(Do lat. \textunderscore aurum\textunderscore  + \textunderscore rosa\textunderscore )}
\end{itemize}
Que tem côr de oiro e rosa.
\section{Auriscalpo}
\begin{itemize}
\item {Grp. gram.:m.}
\end{itemize}
\begin{itemize}
\item {Proveniência:(Lat. \textunderscore auriscalpium\textunderscore )}
\end{itemize}
Sonda, instrumento, para limpar as orelhas.
\section{Aurisplendente}
\begin{itemize}
\item {Grp. gram.:adj.}
\end{itemize}
O mesmo que \textunderscore auri-esplendente\textunderscore . Cf. Macedo, \textunderscore Oriente\textunderscore , II, 31.
\section{Aurito}
\begin{itemize}
\item {Grp. gram.:adj.}
\end{itemize}
\begin{itemize}
\item {Proveniência:(Lat. \textunderscore auritus\textunderscore )}
\end{itemize}
Que tem orelhas grandes.
Que ouve bem.
\section{Auritrêmulo}
\begin{itemize}
\item {Grp. gram.:adj.}
\end{itemize}
Que tremeluz, brilhando como oiro. Cf. Castilho, \textunderscore Fastos\textunderscore , II, 41.
\section{Auriverde}
\begin{itemize}
\item {Grp. gram.:adj.}
\end{itemize}
\begin{itemize}
\item {Proveniência:(De \textunderscore áureo\textunderscore  + \textunderscore verde\textunderscore )}
\end{itemize}
Que participa das duas côres, a verde e a do oiro.
\section{Aurívoro}
\begin{itemize}
\item {Grp. gram.:adj.}
\end{itemize}
\begin{itemize}
\item {Utilização:Fig.}
\end{itemize}
\begin{itemize}
\item {Proveniência:(Do lat. \textunderscore aurum\textunderscore  + \textunderscore vorare\textunderscore )}
\end{itemize}
Que devora oiro.
Gastador, dissipador.
\section{Auroferrífero}
\begin{itemize}
\item {Grp. gram.:adj.}
\end{itemize}
\begin{itemize}
\item {Proveniência:(Do lat. \textunderscore aurum\textunderscore  + \textunderscore ferrum\textunderscore  + \textunderscore ferre\textunderscore )}
\end{itemize}
Que contém oiro e ferro.
\section{Auroque}
\begin{itemize}
\item {Grp. gram.:m.}
\end{itemize}
Boi selvagem, espécie de bisão, que abundou outrora nas florestas europeias e hoje se encontra apenas no Cáucaso e nalgumas florestas do nordeste da Europa.
\section{Aurora}
\begin{itemize}
\item {Grp. gram.:f.}
\end{itemize}
\begin{itemize}
\item {Proveniência:(Lat. \textunderscore aurora\textunderscore )}
\end{itemize}
Claridade, que antecede o nascimento do sol.
Princípio da vida; juventude: \textunderscore a pequena está na sua aurora\textunderscore .
Phenómeno luminoso das regiões polares.
Início; comêço: \textunderscore na aurora da Renascença\textunderscore .
Oriente.
Rosiclér.
\section{Auroral}
\begin{itemize}
\item {Grp. gram.:adj.}
\end{itemize}
Que diz respeito á aurora.
\section{Aurorar}
\begin{itemize}
\item {Grp. gram.:v. t.}
\end{itemize}
\begin{itemize}
\item {Utilização:Neol.}
\end{itemize}
Illuminar como a aurora.
Clarificar.
Tornar brilhante.
\section{Aurorecer}
\begin{itemize}
\item {Grp. gram.:v. i.}
\end{itemize}
\begin{itemize}
\item {Proveniência:(De \textunderscore aurora\textunderscore )}
\end{itemize}
Começar a romper o dia.
\section{Aurorescer}
\begin{itemize}
\item {Grp. gram.:v. i.}
\end{itemize}
\begin{itemize}
\item {Proveniência:(De \textunderscore aurora\textunderscore )}
\end{itemize}
Começar a romper o dia.
\section{Auruncos}
\begin{itemize}
\item {Grp. gram.:m. pl.}
\end{itemize}
\begin{itemize}
\item {Proveniência:(Lat. \textunderscore aurunci\textunderscore )}
\end{itemize}
Antigos habitantes do Lácio.
\section{Auscultação}
\begin{itemize}
\item {Grp. gram.:f.}
\end{itemize}
Acção de \textunderscore auscultar\textunderscore .
\section{Auscultador}
\begin{itemize}
\item {Grp. gram.:m.}
\end{itemize}
Aquelle que ausculta.
\section{Auscultar}
\begin{itemize}
\item {Grp. gram.:v. t.}
\end{itemize}
\begin{itemize}
\item {Proveniência:(Lat. \textunderscore auscultare\textunderscore )}
\end{itemize}
Applicar o ouvido a (peito, ventre, costas, cabeça, pescoço, etc.) para conhecer os ruidos que se produzem interiormente.
\section{Ausência}
\begin{itemize}
\item {Grp. gram.:f.}
\end{itemize}
\begin{itemize}
\item {Grp. gram.:Pl.}
\end{itemize}
\begin{itemize}
\item {Proveniência:(Lat. \textunderscore absentia\textunderscore )}
\end{itemize}
Afastamento.
Carência.
Falta de comparência.
O que se diz da pessôa ausente: \textunderscore fazem-lhe bôas ausências\textunderscore .
\section{Ausentar-se}
\begin{itemize}
\item {Grp. gram.:v. p.}
\end{itemize}
\begin{itemize}
\item {Proveniência:(De \textunderscore ausente\textunderscore )}
\end{itemize}
Afastar-se.
Partir; ir-se.
\section{Ausente}
\begin{itemize}
\item {Grp. gram.:adj.}
\end{itemize}
\begin{itemize}
\item {Grp. gram.:M.}
\end{itemize}
\begin{itemize}
\item {Proveniência:(Lat. \textunderscore absens\textunderscore )}
\end{itemize}
Afastado; que não está presente; distante: \textunderscore um amigo ausente\textunderscore .
Aquelle que deixou o seu domicílio, afastando-se para longe.
\section{Ausia}
\begin{itemize}
\item {Grp. gram.:f.}
\end{itemize}
\begin{itemize}
\item {Utilização:Prov.}
\end{itemize}
O mesmo que \textunderscore ousio\textunderscore .
\section{Ausio}
\begin{itemize}
\item {Grp. gram.:m.}
\end{itemize}
\begin{itemize}
\item {Utilização:T. de Lanhoso}
\end{itemize}
Ausência.
Solidão.
Tristeza.
\section{Auso}
\begin{itemize}
\item {Grp. gram.:m.}
\end{itemize}
\begin{itemize}
\item {Proveniência:(Lat. \textunderscore ausus\textunderscore )}
\end{itemize}
(V.ousio)
\section{Ausões}
\begin{itemize}
\item {Grp. gram.:m. pl.}
\end{itemize}
\begin{itemize}
\item {Utilização:Ext.}
\end{itemize}
\begin{itemize}
\item {Proveniência:(Lat. \textunderscore ausones\textunderscore )}
\end{itemize}
Antigo povo da Itália.
O mesmo que [[Italianos|italiano]].
\section{Ausónio}
\begin{itemize}
\item {Grp. gram.:adj.}
\end{itemize}
O mesmo que \textunderscore itálico\textunderscore . Cf. \textunderscore Viriato Trág.\textunderscore , XI, 31.
\section{Áuspice}
\begin{itemize}
\item {Grp. gram.:m.}
\end{itemize}
\begin{itemize}
\item {Proveniência:(Lat. \textunderscore auspex\textunderscore )}
\end{itemize}
Áugure; arúspice.
\section{Auspicar}
\begin{itemize}
\item {Grp. gram.:v. t.}
\end{itemize}
\begin{itemize}
\item {Utilização:Des.}
\end{itemize}
\begin{itemize}
\item {Proveniência:(Lat. \textunderscore auspicari\textunderscore )}
\end{itemize}
Prognosticar; predizer.
\section{Auspiciar}
\begin{itemize}
\item {Grp. gram.:v. t.}
\end{itemize}
Fazer auspício de; prognosticar, augurar.
\section{Auspício}
\begin{itemize}
\item {Grp. gram.:m.}
\end{itemize}
\begin{itemize}
\item {Grp. gram.:Pl.}
\end{itemize}
\begin{itemize}
\item {Grp. gram.:Adj.}
\end{itemize}
\begin{itemize}
\item {Proveniência:(Lat. \textunderscore auspicium\textunderscore )}
\end{itemize}
O mesmo que \textunderscore augúrio\textunderscore .
Patrocínio; direcção: \textunderscore fazer uma publicação sob os auspicíos do Govêrno\textunderscore .
O mesmo que \textunderscore auspicioso\textunderscore , Cf. Filinto, XVI, 316.
\section{Auspiciosamente}
\begin{itemize}
\item {Grp. gram.:adv.}
\end{itemize}
De modo \textunderscore auspicioso\textunderscore .
\section{Auspicioso}
\begin{itemize}
\item {Grp. gram.:adj.}
\end{itemize}
\begin{itemize}
\item {Proveniência:(De \textunderscore auspicio\textunderscore )}
\end{itemize}
Bem agoirado; prometedor: \textunderscore auspicioso emprehendimento\textunderscore .
\section{Austaga}
\begin{itemize}
\item {fónica:a-us}
\end{itemize}
\begin{itemize}
\item {Grp. gram.:f.}
\end{itemize}
\begin{itemize}
\item {Utilização:Ant.}
\end{itemize}
\begin{itemize}
\item {Proveniência:(De \textunderscore aúste\textunderscore )}
\end{itemize}
Apparelho ou cabo para içar vela.
\section{Austar}
\begin{itemize}
\item {fónica:a-us}
\end{itemize}
\begin{itemize}
\item {Grp. gram.:v. t.}
\end{itemize}
\begin{itemize}
\item {Utilização:Ant.}
\end{itemize}
\begin{itemize}
\item {Proveniência:(De \textunderscore aúste\textunderscore )}
\end{itemize}
Lançar, arremessar (cabos náuticos, calabretes, etc.). Cf. \textunderscore Peregrinação\textunderscore , LIII.
\section{Aúste}
\begin{itemize}
\item {Grp. gram.:m.}
\end{itemize}
\begin{itemize}
\item {Utilização:Ant.}
\end{itemize}
Cabo de navio, amarra.
\section{Austeramente}
\begin{itemize}
\item {Grp. gram.:adv.}
\end{itemize}
De modo \textunderscore austero\textunderscore .
\section{Austeridade}
\begin{itemize}
\item {Grp. gram.:f.}
\end{itemize}
\begin{itemize}
\item {Proveniência:(Lat. \textunderscore austeritas\textunderscore )}
\end{itemize}
Qualidade do que é austero.
\section{Austerismo}
\begin{itemize}
\item {Grp. gram.:m.}
\end{itemize}
Excesso de austeridade.
\section{Austerizar}
\begin{itemize}
\item {Grp. gram.:v. t.}
\end{itemize}
Tornar austero.
\section{Austero}
\begin{itemize}
\item {Grp. gram.:adj.}
\end{itemize}
\begin{itemize}
\item {Proveniência:(Lat. \textunderscore austerus\textunderscore )}
\end{itemize}
Severo.
Que é rígido em carácter ou costumes.
Penoso.
Acerbo.
Sério.
Ríspido.
Adstringente.
Grosseiro.
Áspero.
Escuro.
\section{Auster-se}
\begin{itemize}
\item {Grp. gram.:v. p.}
\end{itemize}
\begin{itemize}
\item {Utilização:Ant.}
\end{itemize}
O mesmo que [[abster-se|abster]].
\section{Austinado}
\begin{itemize}
\item {Grp. gram.:adj.}
\end{itemize}
\begin{itemize}
\item {Utilização:Des.}
\end{itemize}
\begin{itemize}
\item {Utilização:Ant.}
\end{itemize}
O mesmo que \textunderscore desaustinado\textunderscore .
Contumaz, impenitente. Cf. G. Vicente, I, 252.
(Alter. de \textunderscore obstinado\textunderscore )
\section{Austinente}
\begin{itemize}
\item {Grp. gram.:adj.}
\end{itemize}
\begin{itemize}
\item {Utilização:Ant.}
\end{itemize}
O mesmo que \textunderscore abstinente\textunderscore .
\section{Austral}
\begin{itemize}
\item {Grp. gram.:adj.}
\end{itemize}
\begin{itemize}
\item {Proveniência:(Lat. \textunderscore australis\textunderscore )}
\end{itemize}
Que está do lado do austro; que fica da banda do Sul: \textunderscore constellação austral\textunderscore .
\section{Australasiano}
\begin{itemize}
\item {Grp. gram.:adj.}
\end{itemize}
O mesmo que \textunderscore austrálio\textunderscore .
\section{Australásio}
\begin{itemize}
\item {Grp. gram.:adj.}
\end{itemize}
O mesmo que \textunderscore austrálio\textunderscore .
\section{Austrália}
\begin{itemize}
\item {Grp. gram.:f.}
\end{itemize}
\begin{itemize}
\item {Utilização:Bot.}
\end{itemize}
Mimosa, de folha constante.
\section{Australiano}
\begin{itemize}
\item {Grp. gram.:m.}
\end{itemize}
\begin{itemize}
\item {Grp. gram.:Adj.}
\end{itemize}
Habitante da Austrália.
Pertencente á Austrália: \textunderscore clima australiano\textunderscore .
\section{Austrálico}
\begin{itemize}
\item {Grp. gram.:m.}
\end{itemize}
Gênero de insectos coleópteros tetrâmeros.
\section{Austrálio}
\begin{itemize}
\item {Grp. gram.:adj.}
\end{itemize}
Que é do sul da Ásia; australiano.
\section{Australita}
\begin{itemize}
\item {Grp. gram.:f.}
\end{itemize}
Mineral cinzento, composto de alumínio, silício e ferro.
\section{Austríaco}
\begin{itemize}
\item {Grp. gram.:m.}
\end{itemize}
\begin{itemize}
\item {Grp. gram.:Adj.}
\end{itemize}
Aquelle que é natural da Áustria.
Relativo á Áustria.
\section{Austrífero}
\begin{itemize}
\item {Grp. gram.:adj.}
\end{itemize}
\begin{itemize}
\item {Utilização:Des.}
\end{itemize}
\begin{itemize}
\item {Proveniência:(Lat. \textunderscore austrifer\textunderscore )}
\end{itemize}
Que traz chuva ou vento do Sul.
\section{Austrino}
\begin{itemize}
\item {Grp. gram.:adj.}
\end{itemize}
\begin{itemize}
\item {Proveniência:(Lat. \textunderscore austrinus\textunderscore )}
\end{itemize}
O mesmo que \textunderscore austral\textunderscore .
\section{Austro}
\begin{itemize}
\item {Grp. gram.:m.}
\end{itemize}
\begin{itemize}
\item {Proveniência:(Lat. \textunderscore auster\textunderscore )}
\end{itemize}
Sul; vento do Sul.
\section{Austro-húngaro}
\begin{itemize}
\item {Grp. gram.:adj.}
\end{itemize}
Diz-se do Império que comprehende a Áustria e a Hungria.
\section{Autália}
\begin{itemize}
\item {Grp. gram.:f.}
\end{itemize}
Gênero de insectos coleópteros pentâmeros.
\section{Autarchia}
\begin{itemize}
\item {fónica:qui}
\end{itemize}
\begin{itemize}
\item {Grp. gram.:f.}
\end{itemize}
\begin{itemize}
\item {Utilização:Neol.}
\end{itemize}
Govêrno autónomo; autonomia.
\section{Autarcia}
\begin{itemize}
\item {Grp. gram.:f.}
\end{itemize}
Tranquilidade de espírito.
Satisfação íntima.
Frugalidade, sobriedade.
\section{Autarquia}
\begin{itemize}
\item {Grp. gram.:f.}
\end{itemize}
\begin{itemize}
\item {Utilização:Neol.}
\end{itemize}
Govêrno autónomo; autonomia.
\section{Autem-genuit}
\begin{itemize}
\item {fónica:áutêum-gênuid}
\end{itemize}
\begin{itemize}
\item {Grp. gram.:m.}
\end{itemize}
Longa e fastidiosa relação; narração enfadonha.
(Loc. lat. da Bíblia)
\section{Autêntica}
\begin{itemize}
\item {Grp. gram.:f.}
\end{itemize}
\begin{itemize}
\item {Proveniência:(De \textunderscore authêntico\textunderscore )}
\end{itemize}
Certidão, carta autêntica.
Certificado de relíquia ou milagre.
\section{Autenticação}
\begin{itemize}
\item {Grp. gram.:f.}
\end{itemize}
Acto de \textunderscore autenticar\textunderscore . Cf. Herculano, \textunderscore Hist. de Port.\textunderscore , IV, 209.
\section{Autenticamente}
\begin{itemize}
\item {Grp. gram.:adv.}
\end{itemize}
De modo \textunderscore autêntico\textunderscore .
\section{Autenticar}
\begin{itemize}
\item {Grp. gram.:v. t.}
\end{itemize}
Tornar autêntico.
\section{Autenticidade}
\begin{itemize}
\item {Grp. gram.:f.}
\end{itemize}
Qualidade do que é \textunderscore autêntico\textunderscore .
\section{Autêntico}
\begin{itemize}
\item {Grp. gram.:adj.}
\end{itemize}
\begin{itemize}
\item {Proveniência:(Lat. \textunderscore authenticus\textunderscore )}
\end{itemize}
Tornado certo por testemunho solenne: \textunderscore narração autêntica\textunderscore .
Legalizado, digno de acreditar-se.
Que é do autor a quem se atribue: \textunderscore obra autêntica\textunderscore .
Diz-se de quatro dos oito tons do cantochão.--Os outros quatros dizem-se \textunderscore plagaes\textunderscore .
\section{Autepsa}
\begin{itemize}
\item {Grp. gram.:f.}
\end{itemize}
\begin{itemize}
\item {Proveniência:(Lat. \textunderscore authepsa\textunderscore )}
\end{itemize}
Antigo vaso ou máquina culinária, com duas divisões, uma em que se accendia o fogo e outra em que se coziam as iguarias.
\section{Authêntica}
\begin{itemize}
\item {Grp. gram.:f.}
\end{itemize}
\begin{itemize}
\item {Proveniência:(De \textunderscore authêntico\textunderscore )}
\end{itemize}
Certidão, carta authêntica.
Certificado de relíquia ou milagre.
\section{Authenticação}
\begin{itemize}
\item {Grp. gram.:f.}
\end{itemize}
Acto de \textunderscore authenticar\textunderscore . Cf. Herculano, \textunderscore Hist. de Port.\textunderscore , IV, 209.
\section{Authenticamente}
\begin{itemize}
\item {Grp. gram.:adv.}
\end{itemize}
De modo \textunderscore authêntico\textunderscore .
\section{Authenticar}
\begin{itemize}
\item {Grp. gram.:v. t.}
\end{itemize}
Tornar authêntico.
\section{Authenticidade}
\begin{itemize}
\item {Grp. gram.:f.}
\end{itemize}
Qualidade do que é \textunderscore authêntico\textunderscore .
\section{Authêntico}
\begin{itemize}
\item {Grp. gram.:adj.}
\end{itemize}
\begin{itemize}
\item {Proveniência:(Lat. \textunderscore authenticus\textunderscore )}
\end{itemize}
Tornado certo por testemunho solenne: \textunderscore narração authêntica\textunderscore .
Legalizado, digno de acreditar-se.
Que é do autor a quem se atribue: \textunderscore obra authêntica\textunderscore .
Diz-se de quatro dos oito tons do cantochão.--Os outros quatros dizem-se \textunderscore plagaes\textunderscore .
\section{Authepsa}
\begin{itemize}
\item {Grp. gram.:f.}
\end{itemize}
\begin{itemize}
\item {Proveniência:(Lat. \textunderscore authepsa\textunderscore )}
\end{itemize}
Antigo vaso ou máquina culinária, com duas divisões, uma em que se accendia o fogo e outra em que se coziam as iguarias.
\section{Autivo}
\textunderscore adj.\textunderscore  (e der.)
(Fórma ant. de \textunderscore activo\textunderscore , etc.)
\section{Auto}
\begin{itemize}
\item {Grp. gram.:m.}
\end{itemize}
\begin{itemize}
\item {Utilização:Ant.}
\end{itemize}
\begin{itemize}
\item {Grp. gram.:Pl.}
\end{itemize}
\begin{itemize}
\item {Proveniência:(Lat. \textunderscore actus\textunderscore )}
\end{itemize}
Solennidade.
Acção.
Narração escrita e authenticada de qualquer acto: \textunderscore auto da posse de um juiz\textunderscore .
O mesmo que \textunderscore acto\textunderscore .
Antiga composição dramática: \textunderscore os autos de Gil Vicente\textunderscore .
Conjunto das peças de um processo forense; processo.
\section{Auto...}
\begin{itemize}
\item {Grp. gram.:pref.}
\end{itemize}
\begin{itemize}
\item {Proveniência:(Gr. \textunderscore autos\textunderscore )}
\end{itemize}
Próprio.
De si mesmo, por si mesmo: \textunderscore autobiographia\textunderscore .
\section{Autobiografar}
\begin{itemize}
\item {Grp. gram.:v. t.}
\end{itemize}
Fazer a autobiografia de.
\section{Autobiografia}
\begin{itemize}
\item {Grp. gram.:f.}
\end{itemize}
Vida de um indivíduo, escrita por êlle próprio.
(Cp. \textunderscore autobiógrapho\textunderscore )
\section{Autobiográfico}
\begin{itemize}
\item {Grp. gram.:adj.}
\end{itemize}
Relativo a autobiografia.
\section{Autobiógrafo}
\begin{itemize}
\item {Grp. gram.:m.}
\end{itemize}
\begin{itemize}
\item {Proveniência:(Do gr. \textunderscore autos\textunderscore  + \textunderscore bios\textunderscore  + \textunderscore graphein\textunderscore )}
\end{itemize}
Autor da sua própria biografia.
\section{Autobiographar}
\begin{itemize}
\item {Grp. gram.:v. t.}
\end{itemize}
Fazer a autobiographia de.
\section{Autobiographia}
\begin{itemize}
\item {Grp. gram.:f.}
\end{itemize}
Vida de um indivíduo, escrita por êlle próprio.
(Cp. \textunderscore autobiógrapho\textunderscore )
\section{Autobiográphico}
\begin{itemize}
\item {Grp. gram.:adj.}
\end{itemize}
Relativo a autobiographia.
\section{Autobiógrapho}
\begin{itemize}
\item {Grp. gram.:m.}
\end{itemize}
\begin{itemize}
\item {Proveniência:(Do gr. \textunderscore autos\textunderscore  + \textunderscore bios\textunderscore  + \textunderscore graphein\textunderscore )}
\end{itemize}
Autor da sua própria biographia.
\section{Autocéfalo}
\begin{itemize}
\item {Grp. gram.:m.}
\end{itemize}
\begin{itemize}
\item {Grp. gram.:Adj.}
\end{itemize}
\begin{itemize}
\item {Proveniência:(Gr. \textunderscore autokephalos\textunderscore )}
\end{itemize}
Bispo grego, que não estava sujeito ao Patriarcha.
Independente, autónomo, que se governa por si.
\section{Autocéphalo}
\begin{itemize}
\item {Grp. gram.:m.}
\end{itemize}
\begin{itemize}
\item {Grp. gram.:Adj.}
\end{itemize}
\begin{itemize}
\item {Proveniência:(Gr. \textunderscore autokephalos\textunderscore )}
\end{itemize}
Bispo grego, que não estava sujeito ao Patriarcha.
Independente, autónomo, que se governa por si.
\section{Autóchtone}
\begin{itemize}
\item {Grp. gram.:m.  e  adj.}
\end{itemize}
\begin{itemize}
\item {Proveniência:(Lat. \textunderscore autochthones\textunderscore )}
\end{itemize}
Indígena; aborígene.
\section{Autochthoneidade}
\begin{itemize}
\item {Grp. gram.:f.}
\end{itemize}
O mesmo que \textunderscore autochthonia\textunderscore .
\section{Autochthonia}
\begin{itemize}
\item {Grp. gram.:f.}
\end{itemize}
Qualidade de autóchthone.
\section{Autochthonismo}
\begin{itemize}
\item {Grp. gram.:m.}
\end{itemize}
O mesmo que \textunderscore autochthonia\textunderscore .
\section{Autocinesia}
\begin{itemize}
\item {Grp. gram.:f.}
\end{itemize}
\begin{itemize}
\item {Proveniência:(Do gr. \textunderscore autos\textunderscore  + \textunderscore kinein\textunderscore )}
\end{itemize}
Propriedade, que a matéria viva tem, segundo Bouchout, de se mover por si própria, sem músculos, nem fibras contrácteis apparentes.
\section{Autoclave}
\begin{itemize}
\item {Grp. gram.:f.}
\end{itemize}
\begin{itemize}
\item {Proveniência:(Do gr. \textunderscore autos\textunderscore  + lat. \textunderscore clavis\textunderscore )}
\end{itemize}
Vaso ou marmita, usada especialmente nas pharmácias para fazerem cocções sem evaporação.
\section{Autoclínica}
\begin{itemize}
\item {Grp. gram.:f.}
\end{itemize}
\begin{itemize}
\item {Proveniência:(De \textunderscore auto...\textunderscore  + \textunderscore clínica\textunderscore )}
\end{itemize}
Estudo de uma doença, feito pelo próprio doente.
\section{Autoclismo}
\begin{itemize}
\item {Grp. gram.:m.}
\end{itemize}
Apparelho, formado principalmente por uma caixa de ferro, collocada acima da boca das latrinas, e da qual jorra água para as lavar.
\section{Autocopista}
\begin{itemize}
\item {Grp. gram.:m.}
\end{itemize}
\begin{itemize}
\item {Proveniência:(De \textunderscore auto...\textunderscore  + \textunderscore copista\textunderscore )}
\end{itemize}
Máquina, para fazer cópias de qualquer escrita.
\section{Autocracia}
\begin{itemize}
\item {Grp. gram.:f.}
\end{itemize}
Poder absoluto de um monarcha.
(Cp. \textunderscore autócrata\textunderscore )
\section{Autócrata}
\begin{itemize}
\item {Grp. gram.:m.}
\end{itemize}
\begin{itemize}
\item {Proveniência:(Gr. \textunderscore autokrates\textunderscore )}
\end{itemize}
Soberano absoluto, independente de qualquer Constituição política.
\section{Autocrático}
\begin{itemize}
\item {Grp. gram.:adj.}
\end{itemize}
Relativo a \textunderscore autócrata\textunderscore .
\section{Autocrítica}
\begin{itemize}
\item {Grp. gram.:f.}
\end{itemize}
Critica, que alguém faz de si próprio ou das suas obras.
\section{Autóctone}
\begin{itemize}
\item {Grp. gram.:m.  e  adj.}
\end{itemize}
\begin{itemize}
\item {Proveniência:(Lat. \textunderscore autochthones\textunderscore )}
\end{itemize}
Indígena; aborígene.
\section{Autoctoneidade}
\begin{itemize}
\item {Grp. gram.:f.}
\end{itemize}
O mesmo que \textunderscore autoctonia\textunderscore .
\section{Autoctonia}
\begin{itemize}
\item {Grp. gram.:f.}
\end{itemize}
Qualidade de autóctone.
\section{Autoctonismo}
\begin{itemize}
\item {Grp. gram.:m.}
\end{itemize}
O mesmo que \textunderscore autoctonia\textunderscore .
\section{Auto-de-fé}
\begin{itemize}
\item {Grp. gram.:m.}
\end{itemize}
Solennidade inquisitorial, em que apareciam os penitenciados, applicando-se-lhes as penas.
\section{Autodidacta}
\begin{itemize}
\item {Grp. gram.:m.}
\end{itemize}
\begin{itemize}
\item {Utilização:Neol.}
\end{itemize}
\begin{itemize}
\item {Proveniência:(De \textunderscore auto...\textunderscore  + \textunderscore didacta\textunderscore )}
\end{itemize}
Aquelle que pratica a autodidáctica.
\section{Autodidáctica}
\begin{itemize}
\item {Grp. gram.:f.}
\end{itemize}
\begin{itemize}
\item {Utilização:Neol.}
\end{itemize}
\begin{itemize}
\item {Proveniência:(De \textunderscore auto...\textunderscore  + \textunderscore didáctica\textunderscore )}
\end{itemize}
Arte de ensinar, dirigindo livremente o processo de ensino.
\section{Autodidáctico}
\begin{itemize}
\item {Grp. gram.:adj.}
\end{itemize}
Relativo á \textunderscore autodidáctica\textunderscore .
\section{Autodidácto}
\begin{itemize}
\item {Grp. gram.:adj.}
\end{itemize}
\begin{itemize}
\item {Grp. gram.:M.}
\end{itemize}
Que aprende sem mestre, que é mestre de si próprio.
O mesmo que \textunderscore autodidacta\textunderscore .
\section{Autodidaxia}
\begin{itemize}
\item {Grp. gram.:f.}
\end{itemize}
Acto de estudar sem mestre.
Aptidão para aprender sem mestre.
(Cp. \textunderscore autodidacta\textunderscore )
\section{Autodinamia}
\begin{itemize}
\item {Grp. gram.:f.}
\end{itemize}
\begin{itemize}
\item {Proveniência:(Do gr. \textunderscore autos\textunderscore  + \textunderscore dunamos\textunderscore )}
\end{itemize}
Propriedade daquillo que se move por fôrça própria.
\section{Autodinâmico}
\begin{itemize}
\item {Grp. gram.:adj.}
\end{itemize}
Relativo á \textunderscore autodinamia\textunderscore .
\section{Autodynamia}
\begin{itemize}
\item {Grp. gram.:f.}
\end{itemize}
\begin{itemize}
\item {Proveniência:(Do gr. \textunderscore autos\textunderscore  + \textunderscore dunamos\textunderscore )}
\end{itemize}
Propriedade daquillo que se move por fôrça própria.
\section{Autodynâmico}
\begin{itemize}
\item {Grp. gram.:adj.}
\end{itemize}
Relativo á \textunderscore autodynamia\textunderscore .
\section{Auto-educação}
\begin{itemize}
\item {Grp. gram.:f.}
\end{itemize}
Educação, independente de normas.
\section{Autofecundação}
\begin{itemize}
\item {Grp. gram.:f.}
\end{itemize}
\begin{itemize}
\item {Utilização:Bot.}
\end{itemize}
Propriedade das plantas que se fecundam com o próprio póllen.
\section{Autofecundo}
\begin{itemize}
\item {Grp. gram.:adj.}
\end{itemize}
Que tem a propriedade da autofecundação.
\section{Autogamia}
\begin{itemize}
\item {Grp. gram.:f.}
\end{itemize}
\begin{itemize}
\item {Utilização:Zool.}
\end{itemize}
\begin{itemize}
\item {Proveniência:(De \textunderscore autógamo\textunderscore )}
\end{itemize}
Hermaphroditismo, em que a disposição do duplo apparelho reproductor é tal, que o órgão macho póde operar sobre o órgão feminino, de modo que o indivíduo se fecunde a si próprio.--É phenómeno, que só se dá em alguns vermes.
\section{Autógamo}
\begin{itemize}
\item {Grp. gram.:adj.}
\end{itemize}
\begin{itemize}
\item {Proveniência:(Do gr. \textunderscore autos\textunderscore  + \textunderscore gamos\textunderscore )}
\end{itemize}
Que tem a qualidade ou estado de autogamia.
\section{Autogasogênio}
\begin{itemize}
\item {Grp. gram.:m.}
\end{itemize}
Espécie de candeeiro, que por si mesmo produz o gás que alimenta a chamma.
\section{Autogêneo}
\begin{itemize}
\item {Grp. gram.:adj.}
\end{itemize}
\begin{itemize}
\item {Utilização:Bot.}
\end{itemize}
\begin{itemize}
\item {Proveniência:(Do gr. \textunderscore autos\textunderscore  + \textunderscore genes\textunderscore )}
\end{itemize}
Que existe por si próprio.
Cujos bolbos dão fôlhas, antes de metidos na terra, (falando-se do narciso).
\section{Autogênese}
\begin{itemize}
\item {Grp. gram.:f.}
\end{itemize}
\begin{itemize}
\item {Proveniência:(Do gr. \textunderscore autos\textunderscore  + \textunderscore genesis\textunderscore )}
\end{itemize}
Geração espontânea.
\section{Autogenia}
\begin{itemize}
\item {Grp. gram.:f.}
\end{itemize}
Qualidade de autogêneo.
\section{Autognose}
\begin{itemize}
\item {Grp. gram.:f.}
\end{itemize}
\begin{itemize}
\item {Proveniência:(Do gr. \textunderscore autos\textunderscore  + \textunderscore gnosis\textunderscore )}
\end{itemize}
Conhecimento de si próprio.
\section{Autognosia}
\begin{itemize}
\item {Grp. gram.:f.}
\end{itemize}
O mesmo que \textunderscore autognose\textunderscore .
\section{Autogovêrno}
\begin{itemize}
\item {Grp. gram.:m.}
\end{itemize}
Govêrno autónomo.
\section{Autografar}
\begin{itemize}
\item {Grp. gram.:v. t.}
\end{itemize}
\begin{itemize}
\item {Proveniência:(De \textunderscore autógrapho\textunderscore )}
\end{itemize}
Reproduzir por autografia.
\section{Autografia}
\begin{itemize}
\item {Grp. gram.:f.}
\end{itemize}
Reproducção fiel de uma escrita.
Processo para obter rapidamente a reproducção de um manuscrito.
(Cp. \textunderscore autógrapho\textunderscore )
\section{Autográfico}
\begin{itemize}
\item {Grp. gram.:adj.}
\end{itemize}
Relativo á \textunderscore autografia\textunderscore .
\section{Autógrafo}
\begin{itemize}
\item {Grp. gram.:m.}
\end{itemize}
\begin{itemize}
\item {Grp. gram.:Adj.}
\end{itemize}
\begin{itemize}
\item {Proveniência:(Gr. \textunderscore autographos\textunderscore )}
\end{itemize}
Escrito, feito pelo próprio autor.
Original; que é escrito pelo próprio autor.
\section{Autografomania}
\begin{itemize}
\item {Grp. gram.:f.}
\end{itemize}
Mania de colleccionar escritos autógrafos.
\section{Autografomaníaco}
\begin{itemize}
\item {Grp. gram.:m.  e  adj.}
\end{itemize}
O que tem a \textunderscore autografomania\textunderscore .
\section{Autografófilo}
\begin{itemize}
\item {Grp. gram.:m.}
\end{itemize}
\begin{itemize}
\item {Proveniência:(Do gr. \textunderscore autographos\textunderscore  + \textunderscore philos\textunderscore )}
\end{itemize}
Aquelle que sabe apreciar e colligir bons autógrafos.
\section{Autographar}
\begin{itemize}
\item {Grp. gram.:v. t.}
\end{itemize}
\begin{itemize}
\item {Proveniência:(De \textunderscore autógrapho\textunderscore )}
\end{itemize}
Reproduzir por autographia.
\section{Autographia}
\begin{itemize}
\item {Grp. gram.:f.}
\end{itemize}
Reproducção fiel de uma escrita.
Processo para obter rapidamente a reproducção de um manuscrito.
(Cp. \textunderscore autógrapho\textunderscore )
\section{Autográphico}
\begin{itemize}
\item {Grp. gram.:adj.}
\end{itemize}
Relativo á \textunderscore autographia\textunderscore .
\section{Autógrapho}
\begin{itemize}
\item {Grp. gram.:m.}
\end{itemize}
\begin{itemize}
\item {Grp. gram.:Adj.}
\end{itemize}
\begin{itemize}
\item {Proveniência:(Gr. \textunderscore autographos\textunderscore )}
\end{itemize}
Escrito, feito pelo próprio autor.
Original; que é escrito pelo próprio autor.
\section{Autographomania}
\begin{itemize}
\item {Grp. gram.:f.}
\end{itemize}
Mania de colleccionar escritos autógraphos.
\section{Autographomaníaco}
\begin{itemize}
\item {Grp. gram.:m.  e  adj.}
\end{itemize}
O que tem a \textunderscore autographomania\textunderscore .
\section{Autographóphilo}
\begin{itemize}
\item {Grp. gram.:m.}
\end{itemize}
\begin{itemize}
\item {Proveniência:(Do gr. \textunderscore autographos\textunderscore  + \textunderscore philos\textunderscore )}
\end{itemize}
Aquelle que sabe apreciar e colligir bons autógraphos.
\section{Auto-idolatria}
\begin{itemize}
\item {Grp. gram.:f.}
\end{itemize}
Amor exaggerado de si próprio; admiração das próprias obras ou inventos:«\textunderscore arrojos de auto-idolatria, censuraveis até quando justos\textunderscore ». Castilho, \textunderscore Montalverne\textunderscore .
\section{Auto-infecção}
\begin{itemize}
\item {Grp. gram.:f.}
\end{itemize}
Infecção espontânea do organismo ou de parte do organismo humano.
\section{Auto-intoxicação}
\begin{itemize}
\item {Grp. gram.:f.}
\end{itemize}
Estado do indivíduo ou do organismo, em que se produziu intoxicação, independentemente de acção externa.
\section{Auctor}
\begin{itemize}
\item {Grp. gram.:m.}
\end{itemize}
\begin{itemize}
\item {Utilização:Prov.}
\end{itemize}
\begin{itemize}
\item {Proveniência:(Lat. \textunderscore auctor\textunderscore )}
\end{itemize}
(que é fórma erudita, mas opposta á phonética e ás tradições da língua).
O mesmo, nos seus der.
Causa principal de uma coisa.
Inventor.
Aquelle que escreveu obra literária ou scientífica.
Fundador.
Aquelle que intenta demanda judicial.
Aquelle, de que procede ou nasce alguém ou alguma coisa.
O vallador mais experiente, encarregado de dirigir os outros.
\section{Autofagia}
\begin{itemize}
\item {Grp. gram.:f.}
\end{itemize}
\begin{itemize}
\item {Proveniência:(Do gr. \textunderscore autos\textunderscore  + \textunderscore phagein\textunderscore )}
\end{itemize}
Estado do animal, que sustenta a vida á custa da própria substância.
\section{Autofagismo}
\begin{itemize}
\item {Grp. gram.:m.}
\end{itemize}
O mesmo que \textunderscore autofagia\textunderscore .
\section{Autófago}
\begin{itemize}
\item {Grp. gram.:adj.}
\end{itemize}
Que tem \textunderscore autofagia\textunderscore .
\section{Autofonia}
\begin{itemize}
\item {Grp. gram.:f.}
\end{itemize}
\begin{itemize}
\item {Utilização:Med.}
\end{itemize}
\begin{itemize}
\item {Proveniência:(Do gr. \textunderscore autos\textunderscore  + \textunderscore phone\textunderscore )}
\end{itemize}
Resonância da própria voz, observada por um médico que ausculta um doente.
\section{Autofónico}
\begin{itemize}
\item {Grp. gram.:adj.}
\end{itemize}
Relativo á \textunderscore autofonia\textunderscore .
\section{Autofografia}
\begin{itemize}
\item {Grp. gram.:f.}
\end{itemize}
\begin{itemize}
\item {Proveniência:(De \textunderscore auto...\textunderscore  + \textunderscore photographia\textunderscore )}
\end{itemize}
Processo fotográfico, com que se reproduzem desenhos e escritos num papel impregnado de chloreto de prata.
\section{Autolábio}
\begin{itemize}
\item {Grp. gram.:m.}
\end{itemize}
Pinça, que se aperta por si própria, em virtude da elasticidade dos seus braços.
\section{Autólatra}
\begin{itemize}
\item {Grp. gram.:m.}
\end{itemize}
Aquelle que tem \textunderscore autolatria\textunderscore .
\section{Autolatria}
\begin{itemize}
\item {Grp. gram.:f.}
\end{itemize}
\begin{itemize}
\item {Proveniência:(Do gr. \textunderscore autos\textunderscore  + \textunderscore latreia\textunderscore )}
\end{itemize}
Adoração de si próprio; auto-idolatria.
\section{Automachia}
\begin{itemize}
\item {fónica:qui}
\end{itemize}
\begin{itemize}
\item {Grp. gram.:f.}
\end{itemize}
Contradicção consigo próprio.
\section{Automalita}
\begin{itemize}
\item {Grp. gram.:f.}
\end{itemize}
Variedade de aluminato de zinco.
\section{Automaquia}
\begin{itemize}
\item {Grp. gram.:f.}
\end{itemize}
Contradicção consigo próprio.
\section{Automatário}
\begin{itemize}
\item {Grp. gram.:m.}
\end{itemize}
\begin{itemize}
\item {Proveniência:(Lat. \textunderscore automatarius\textunderscore )}
\end{itemize}
Aquelle que faz autómatos.
\section{Automatia}
\begin{itemize}
\item {Grp. gram.:f.}
\end{itemize}
Estado do que é \textunderscore autómato\textunderscore .
\section{Automaticamente}
\begin{itemize}
\item {Grp. gram.:adv.}
\end{itemize}
De modo \textunderscore automático\textunderscore .
\section{Automático}
\begin{itemize}
\item {Grp. gram.:adj.}
\end{itemize}
Inconsciente, próprio de \textunderscore autómato\textunderscore : \textunderscore movimento automático\textunderscore .
\section{Automatismo}
\begin{itemize}
\item {Grp. gram.:m.}
\end{itemize}
(V.automatia)
\section{Automatista}
\begin{itemize}
\item {Grp. gram.:m.}
\end{itemize}
O mesmo que \textunderscore automatário\textunderscore .
\section{Automatizar}
\begin{itemize}
\item {Grp. gram.:v. t.}
\end{itemize}
Tornar automático. Cf. \textunderscore Museu Techn.\textunderscore , 21.
\section{Autómato}
\begin{itemize}
\item {Grp. gram.:m.}
\end{itemize}
\begin{itemize}
\item {Proveniência:(Gr. \textunderscore automatos\textunderscore )}
\end{itemize}
Figura, que imita os movimentos dos seres animados.
Maquinismo, posto em movimento por meios mecânicos.
Pessôa inconsciente, cujos actos obedecem á vontade alheia ou não são precedidos de reflexão.
\section{Automaturgo}
\begin{itemize}
\item {Grp. gram.:m.}
\end{itemize}
(V.automatário)
\section{Automedonte}
\begin{itemize}
\item {Grp. gram.:m.}
\end{itemize}
\begin{itemize}
\item {Utilização:Fig.}
\end{itemize}
\begin{itemize}
\item {Proveniência:(De \textunderscore Automedonte\textunderscore , n. p.)}
\end{itemize}
Conductor hábil de uma carruagem. Cf. Garrett, \textunderscore Helena\textunderscore , 28.
\section{Autómetro}
\begin{itemize}
\item {Grp. gram.:m.}
\end{itemize}
\begin{itemize}
\item {Proveniência:(Do gr. \textunderscore autos\textunderscore  + \textunderscore metron\textunderscore )}
\end{itemize}
Instrumento topográphico, que serve, ao mesmo tempo, para levantamento de plantas e nivelamentos.
\section{Automnesia}
\begin{itemize}
\item {Grp. gram.:f.}
\end{itemize}
\begin{itemize}
\item {Proveniência:(Do gr. \textunderscore autos\textunderscore  + \textunderscore mnesis\textunderscore )}
\end{itemize}
Vestigio, que a memória conserva, segundo Ampère, do sentimento que a alma teve da sua própria actividade.
\section{Automnestia}
\begin{itemize}
\item {Grp. gram.:f.}
\end{itemize}
\begin{itemize}
\item {Proveniência:(Do gr. \textunderscore autos\textunderscore  + \textunderscore mnesis\textunderscore )}
\end{itemize}
Vestigio, que a memória conserva, segundo Ampère, do sentimento que a alma teve da sua própria actividade.
\section{Automobilismo}
\begin{itemize}
\item {Grp. gram.:m.}
\end{itemize}
\begin{itemize}
\item {Proveniência:(De \textunderscore automóbil\textunderscore , por \textunderscore automóvel\textunderscore )}
\end{itemize}
Systema de vehículos automóveis.
\section{Automobilista}
\begin{itemize}
\item {Grp. gram.:m.}
\end{itemize}
Aquelle que se dedica ao automobilismo.
\section{Automobilístico}
\begin{itemize}
\item {Grp. gram.:adj.}
\end{itemize}
Relativo ao automobilismo: \textunderscore desastre automobilistico\textunderscore .
\section{Automotor}
\begin{itemize}
\item {Grp. gram.:m.}
\end{itemize}
O mesmo ou melhor que \textunderscore automóvel\textunderscore . Cf. F. Lapa, \textunderscore Phýs. e Chím.\textunderscore , 74.
\section{Automóvel}
\begin{itemize}
\item {Grp. gram.:adj.}
\end{itemize}
\begin{itemize}
\item {Grp. gram.:M.}
\end{itemize}
\begin{itemize}
\item {Proveniência:(De \textunderscore auto...\textunderscore  + \textunderscore móvel\textunderscore )}
\end{itemize}
Que se move por si, automaticamente.
Vehículo, que se move mecanicamente.
\section{Autónimo}
\begin{itemize}
\item {Grp. gram.:adj.}
\end{itemize}
\begin{itemize}
\item {Proveniência:(Do gr. \textunderscore autos\textunderscore  + \textunderscore onuma\textunderscore )}
\end{itemize}
Diz-se da obra, que tem o verdadeiro nome do seu autor.
Diz-se do autor, que assinou a sua obra com o seu verdadeiro nome.
\section{Autonomia}
\begin{itemize}
\item {Grp. gram.:f.}
\end{itemize}
\begin{itemize}
\item {Proveniência:(Gr. \textunderscore autonomia\textunderscore )}
\end{itemize}
Faculdade, que tem um pais conquistado, de se administrar segundo as suas leis.
Independência administrativa.
Liberdade moral ou intellectual.
\section{Autonomicamente}
\begin{itemize}
\item {Grp. gram.:adv.}
\end{itemize}
De modo \textunderscore autonómico\textunderscore .
\section{Autonómico}
\begin{itemize}
\item {Grp. gram.:adj.}
\end{itemize}
Que tem autonomia.
Relativo á \textunderscore autonomia\textunderscore .
\section{Autonomista}
\begin{itemize}
\item {Grp. gram.:m.  e  adj.}
\end{itemize}
Partidário da autonomia da sua terra.
\section{Autónomo}
\begin{itemize}
\item {Grp. gram.:adj.}
\end{itemize}
\begin{itemize}
\item {Proveniência:(Do gr. \textunderscore autos\textunderscore  + \textunderscore nomos\textunderscore )}
\end{itemize}
Que se governa por si.
Independente; livre.
Que tem autonomia.
\section{Autónymo}
\begin{itemize}
\item {Grp. gram.:adj.}
\end{itemize}
\begin{itemize}
\item {Proveniência:(Do gr. \textunderscore autos\textunderscore  + \textunderscore onuma\textunderscore )}
\end{itemize}
Diz-se da obra, que tem o verdadeiro nome do seu autor.
Diz-se do autor, que assinou a sua obra com o seu verdadeiro nome.
\section{Autopata}
\begin{itemize}
\item {Grp. gram.:m.}
\end{itemize}
Aquelle que tem \textunderscore autopatia\textunderscore .
\section{Autopatha}
\begin{itemize}
\item {Grp. gram.:m.}
\end{itemize}
Aquelle que tem \textunderscore autopathia\textunderscore .
\section{Autopathia}
\begin{itemize}
\item {Grp. gram.:f.}
\end{itemize}
\begin{itemize}
\item {Proveniência:(Do gr. \textunderscore autos\textunderscore  + \textunderscore pathos\textunderscore )}
\end{itemize}
Egoísmo exaggerado.
Insensibilidade, perante os males e o bem de outrem.
\section{Autopatia}
\begin{itemize}
\item {Grp. gram.:f.}
\end{itemize}
\begin{itemize}
\item {Proveniência:(Do gr. \textunderscore autos\textunderscore  + \textunderscore pathos\textunderscore )}
\end{itemize}
Egoísmo exaggerado.
Insensibilidade, perante os males e o bem de outrem.
\section{Autopesador}
\begin{itemize}
\item {Grp. gram.:m.}
\end{itemize}
\begin{itemize}
\item {Proveniência:(De \textunderscore auto...\textunderscore  + \textunderscore pesar\textunderscore )}
\end{itemize}
Espécie de balança, que indica por si, sem auxílio de ninguém, o pêso de fardos e outros objectos.
\section{Autophagia}
\begin{itemize}
\item {Grp. gram.:f.}
\end{itemize}
\begin{itemize}
\item {Proveniência:(Do gr. \textunderscore autos\textunderscore  + \textunderscore phagein\textunderscore )}
\end{itemize}
Estado do animal, que sustenta a vida á custa da própria substância.
\section{Autophagismo}
\begin{itemize}
\item {Grp. gram.:m.}
\end{itemize}
O mesmo que \textunderscore autophagia\textunderscore .
\section{Autóphago}
\begin{itemize}
\item {Grp. gram.:adj.}
\end{itemize}
Que tem \textunderscore autophagia\textunderscore .
\section{Autophonia}
\begin{itemize}
\item {Grp. gram.:f.}
\end{itemize}
\begin{itemize}
\item {Utilização:Med.}
\end{itemize}
\begin{itemize}
\item {Proveniência:(Do gr. \textunderscore autos\textunderscore  + \textunderscore phone\textunderscore )}
\end{itemize}
Resonância da própria voz, observada por um médico que ausculta um doente.
\section{Autophónico}
\begin{itemize}
\item {Grp. gram.:adj.}
\end{itemize}
Relativo á \textunderscore autophonia\textunderscore .
\section{Autophotographia}
\begin{itemize}
\item {Grp. gram.:f.}
\end{itemize}
\begin{itemize}
\item {Proveniência:(De \textunderscore auto...\textunderscore  + \textunderscore photographia\textunderscore )}
\end{itemize}
Processo photográphico, com que se reproduzem desenhos e escritos num papel impregnado de chloreto de prata.
\section{Autópiro}
\begin{itemize}
\item {Grp. gram.:m.}
\end{itemize}
\begin{itemize}
\item {Proveniência:(Lat. \textunderscore autopyrus\textunderscore )}
\end{itemize}
Pão grosseiro, muito usado entre os antigos Romanos.
\section{Autopistia}
\begin{itemize}
\item {Grp. gram.:f.}
\end{itemize}
\begin{itemize}
\item {Utilização:Theol.}
\end{itemize}
Crença immediata, indiscutível, que não soffre demonstrações.
\section{Autoplastia}
\begin{itemize}
\item {Grp. gram.:f.}
\end{itemize}
\begin{itemize}
\item {Utilização:Cir.}
\end{itemize}
\begin{itemize}
\item {Proveniência:(Do gr. \textunderscore autos\textunderscore  + \textunderscore plassein\textunderscore )}
\end{itemize}
Restauração de uma parte do corpo, pela applicação de uma parte da pelle do mesmo corpo.
\section{Autoplástico}
\begin{itemize}
\item {Grp. gram.:adj.}
\end{itemize}
Relativo á \textunderscore autoplastia\textunderscore .
\section{Autopse}
\begin{itemize}
\item {Grp. gram.:f.}
\end{itemize}
O mesmo que \textunderscore autópsia\textunderscore .
\section{Autópsia}
\begin{itemize}
\item {Grp. gram.:f.}
\end{itemize}
\begin{itemize}
\item {Proveniência:(Gr. \textunderscore autopsia\textunderscore )}
\end{itemize}
Inspecção de si mesmo.
Observação interior.
Exame médico das partes de um cadáver; necropsia.
\section{Autopsía}
\begin{itemize}
\item {Grp. gram.:f.}
\end{itemize}
\begin{itemize}
\item {Proveniência:(Gr. \textunderscore autopsia\textunderscore )}
\end{itemize}
Inspecção de si mesmo.
Observação interior.
Exame médico das partes de um cadáver; necropsia.
\section{Autopsiar}
\begin{itemize}
\item {Grp. gram.:v. t.}
\end{itemize}
Fazer autópsia a.
\section{Autóptico}
\begin{itemize}
\item {Grp. gram.:adj.}
\end{itemize}
Relativo á \textunderscore autópsia\textunderscore .
\section{Autópyro}
\begin{itemize}
\item {Grp. gram.:m.}
\end{itemize}
\begin{itemize}
\item {Proveniência:(Lat. \textunderscore autopyrus\textunderscore )}
\end{itemize}
Pão grosseiro, muito usado entre os antigos Romanos.
\section{Autor}
\begin{itemize}
\item {Grp. gram.:m.}
\end{itemize}
\begin{itemize}
\item {Utilização:Prov.}
\end{itemize}
\begin{itemize}
\item {Proveniência:(Lat. \textunderscore auctor\textunderscore )}
\end{itemize}
Causa principal de uma coisa.
Inventor.
Aquelle que escreveu obra literária ou scientífica.
Fundador.
Aquelle que intenta demanda judicial.
Aquelle, de que procede ou nasce alguém ou alguma coisa.
O vallador mais experiente, encarregado de dirigir os outros.
\section{Autora}
(fem. de \textunderscore autor\textunderscore )
\section{Autoral}
\begin{itemize}
\item {Grp. gram.:adj.}
\end{itemize}
Relativo aos autores de obras literárias ou scientíficas:«\textunderscore a lei de 1 de Agosto de 1898 define e garante os direitos autoraes\textunderscore ». \textunderscore Diário Official do Brasil\textunderscore , de 3-VIII-98.
\section{Auto-regulador}
\begin{itemize}
\item {Grp. gram.:adj.}
\end{itemize}
Diz-se de um dýnamo tetrapolar, inventado no Brasil em 1901.
\section{Autoria}
\begin{itemize}
\item {Grp. gram.:f.}
\end{itemize}
Qualidade de autor num pleito.
Presença do autor em audiência.
\textunderscore Chamar á autoria\textunderscore , invocar a responsabilidade de \textunderscore autoridade\textunderscore .
\section{Autoridade}
\begin{itemize}
\item {Grp. gram.:f.}
\end{itemize}
\begin{itemize}
\item {Proveniência:(Lat. \textunderscore auctoritas\textunderscore )}
\end{itemize}
Poder de mandar.
Domínio.
Vontade própria.
Poder público.
Aquelle que exerce êste poder: \textunderscore as autoridades administrativas\textunderscore .
Crédito, importância.
Pessôa, que tem crédito.
Pessôa ou texto, que se invoca em reforço de uma opinião: \textunderscore citar autoridades\textunderscore .
\section{Autoritário}
\begin{itemize}
\item {Grp. gram.:adj.}
\end{itemize}
\begin{itemize}
\item {Proveniência:(Do lat. \textunderscore auctoritas\textunderscore )}
\end{itemize}
Relativo á autoridade.
Dominador; despótico: \textunderscore carácter autoritário\textunderscore .
\section{Autoritarismo}
\begin{itemize}
\item {Grp. gram.:m.}
\end{itemize}
Systema de autoritário; despotismo.
\section{Autorização}
\begin{itemize}
\item {Grp. gram.:f.}
\end{itemize}
Poder, que se recebeu para fazer alguma coisa.
Acção de \textunderscore autorizar\textunderscore .
\section{Autorizadamente}
\begin{itemize}
\item {Grp. gram.:adv.}
\end{itemize}
De modo \textunderscore autorizado\textunderscore .
\section{Autorizado}
\begin{itemize}
\item {Grp. gram.:adj.}
\end{itemize}
Que tem autoridade.
Que recebe autorização.
\section{Autorizador}
\begin{itemize}
\item {Grp. gram.:m.  e  adj.}
\end{itemize}
O que autoriza.
\section{Autorizamento}
\begin{itemize}
\item {Grp. gram.:m.}
\end{itemize}
Acção de \textunderscore autorizar\textunderscore .
\section{Autorizar}
\begin{itemize}
\item {Grp. gram.:v. t.}
\end{itemize}
Dar autoridade a.
Dar permissão a.
Validar.
Justificar.
(B. lat. \textunderscore autorizare\textunderscore )
\section{Autorizável}
\begin{itemize}
\item {Grp. gram.:adj.}
\end{itemize}
Que póde sêr autorizado.
\section{Autosauro}
\begin{itemize}
\item {fónica:sau}
\end{itemize}
\begin{itemize}
\item {Grp. gram.:adj.}
\end{itemize}
\begin{itemize}
\item {Proveniência:(Do gr. \textunderscore autos\textunderscore  + \textunderscore sauros\textunderscore )}
\end{itemize}
Que se parece com um lagarto.
\section{Autoscopia}
\begin{itemize}
\item {Grp. gram.:f.}
\end{itemize}
\begin{itemize}
\item {Proveniência:(Do gr. \textunderscore autos\textunderscore  + \textunderscore skopein\textunderscore )}
\end{itemize}
Exame ou auscultação de si próprio.
\section{Autoscópio}
\begin{itemize}
\item {Grp. gram.:m.}
\end{itemize}
Instrumento médico de autoscopia.
\section{Autosita}
\begin{itemize}
\item {Grp. gram.:f.}
\end{itemize}
O mesmo que \textunderscore autosito\textunderscore .
\section{Autosito}
\begin{itemize}
\item {Grp. gram.:m.}
\end{itemize}
\begin{itemize}
\item {Utilização:Terat.}
\end{itemize}
\begin{itemize}
\item {Proveniência:(Gr. \textunderscore autositos\textunderscore )}
\end{itemize}
Monstro simples, que, fóra do ventre materno, póde viver por si.
\section{Autossauro}
\begin{itemize}
\item {Grp. gram.:adj.}
\end{itemize}
\begin{itemize}
\item {Proveniência:(Do gr. \textunderscore autos\textunderscore  + \textunderscore sauros\textunderscore )}
\end{itemize}
Que se parece com um lagarto.
\section{Auto-sugestão}
\begin{itemize}
\item {Grp. gram.:f.}
\end{itemize}
Acto ou effeito de \textunderscore auto-suggestionar\textunderscore .
\section{Auto-sugestionar}
\begin{itemize}
\item {Grp. gram.:v. t.}
\end{itemize}
Suggerir a si próprio.
\section{Auto-suggestão}
\begin{itemize}
\item {Grp. gram.:f.}
\end{itemize}
Acto ou effeito de \textunderscore auto-suggestionar\textunderscore .
\section{Auto-suggestionar}
\begin{itemize}
\item {Grp. gram.:v. t.}
\end{itemize}
Suggerir a si próprio.
\section{Autotelia}
\begin{itemize}
\item {Grp. gram.:f.}
\end{itemize}
Faculdade de determinar por si mesmo o fim das suas acções.
\section{Autotomia}
\begin{itemize}
\item {Grp. gram.:f.}
\end{itemize}
\begin{itemize}
\item {Proveniência:(Do gr. \textunderscore autos\textunderscore  + \textunderscore tome\textunderscore )}
\end{itemize}
Mutilação espontânea, que se observa em certos animaes (insectos, crustáceos, etc.) que, para escapar ao maior perigo, abandonam parte do corpo.
\section{Autuação}
\begin{itemize}
\item {Grp. gram.:f.}
\end{itemize}
Acto de \textunderscore autuar\textunderscore .
\section{Autuar}
\begin{itemize}
\item {Grp. gram.:v. t.}
\end{itemize}
Reduzir a auto.
Reunir em processo; processar.
\section{Autumnação}
\begin{itemize}
\item {Grp. gram.:f.}
\end{itemize}
\begin{itemize}
\item {Proveniência:(Do lat. \textunderscore autumnus\textunderscore )}
\end{itemize}
Influência do outono na vegetação.
\section{Autumnal}
\begin{itemize}
\item {Grp. gram.:adj.}
\end{itemize}
Cf. Filinto, XIV, 217.(V.outonal)
\section{Autumno}
\begin{itemize}
\item {Grp. gram.:m.}
\end{itemize}
\begin{itemize}
\item {Utilização:Ant.}
\end{itemize}
O mesmo que \textunderscore outono\textunderscore .
\section{Auxese}
\begin{itemize}
\item {fónica:csé}
\end{itemize}
\begin{itemize}
\item {Grp. gram.:f.}
\end{itemize}
\begin{itemize}
\item {Proveniência:(Gr. \textunderscore auxesis\textunderscore )}
\end{itemize}
Exaggeração; hypérbole.
\section{Auxiliadamente}
\begin{itemize}
\item {fónica:si}
\end{itemize}
\begin{itemize}
\item {Grp. gram.:adv.}
\end{itemize}
Com auxílio.
\section{Auxiliador}
\begin{itemize}
\item {fónica:si}
\end{itemize}
\begin{itemize}
\item {Grp. gram.:m.  e  adj.}
\end{itemize}
\begin{itemize}
\item {Proveniência:(Lat. \textunderscore auxiliador\textunderscore )}
\end{itemize}
O que auxilía.
\section{Auxiliante}
\begin{itemize}
\item {fónica:si}
\end{itemize}
\begin{itemize}
\item {Grp. gram.:adj}
\end{itemize}
Que auxilía.
\section{Auxiliar}
\begin{itemize}
\item {fónica:si}
\end{itemize}
\begin{itemize}
\item {Grp. gram.:adj.}
\end{itemize}
\begin{itemize}
\item {Grp. gram.:M.}
\end{itemize}
\begin{itemize}
\item {Proveniência:(Lat. \textunderscore auxiliaris\textunderscore )}
\end{itemize}
O mesmo que \textunderscore auxiliante\textunderscore .
O mesmo que \textunderscore auxiliador\textunderscore .
\section{Auxiliar}
\begin{itemize}
\item {fónica:si}
\end{itemize}
\begin{itemize}
\item {Grp. gram.:v. t.}
\end{itemize}
\begin{itemize}
\item {Proveniência:(Lat. \textunderscore auxiliare\textunderscore )}
\end{itemize}
Dar auxílio a; soccorrer: \textunderscore auxiliar a pobreza\textunderscore .
Dar ensejo a; servir de meio para.
Ajudar: \textunderscore auxiliar uma empresa\textunderscore .
\section{Auxiliariamente}
\begin{itemize}
\item {fónica:si}
\end{itemize}
\begin{itemize}
\item {Grp. gram.:adv.}
\end{itemize}
(V.auxiliarmente)
\section{Auxiliário}
\begin{itemize}
\item {fónica:si}
\end{itemize}
\begin{itemize}
\item {Grp. gram.:adj.}
\end{itemize}
\begin{itemize}
\item {Proveniência:(Lat. \textunderscore auxiliarius\textunderscore )}
\end{itemize}
O mesmo que \textunderscore auxiliar\textunderscore ^1.
\section{Auxiliarmente}
\begin{itemize}
\item {fónica:si}
\end{itemize}
\begin{itemize}
\item {Grp. gram.:adv.}
\end{itemize}
Á maneira de auxílio.
\section{Auxílio}
\begin{itemize}
\item {fónica:si}
\end{itemize}
\begin{itemize}
\item {Grp. gram.:m.}
\end{itemize}
\begin{itemize}
\item {Proveniência:(Lat. \textunderscore auxilium\textunderscore )}
\end{itemize}
Soccorro.
Ajuda.
Subsídio.
Amparo.
\section{Auxitano}
\begin{itemize}
\item {fónica:csi}
\end{itemize}
\begin{itemize}
\item {Grp. gram.:adj.}
\end{itemize}
\begin{itemize}
\item {Utilização:Des.}
\end{itemize}
Relativo a Cádiz. Cf. Herculano, \textunderscore Hist. de Port.\textunderscore , III, 149.
\section{Auxómetro}
\begin{itemize}
\item {fónica:csó}
\end{itemize}
\begin{itemize}
\item {Grp. gram.:m.}
\end{itemize}
\begin{itemize}
\item {Proveniência:(Do gr. \textunderscore auxos\textunderscore  + \textunderscore metron\textunderscore )}
\end{itemize}
Instrumento, com que se mede o aumento, que produzem as lentes convergentes.
\section{Avacate}
\begin{itemize}
\item {Grp. gram.:m.}
\end{itemize}
O mesmo que \textunderscore abacate\textunderscore .
Abacateiro. Cf. Ficalho, \textunderscore Plantas úteis\textunderscore , 247.
\section{Avache}
\begin{itemize}
\item {Grp. gram.:m.}
\end{itemize}
\begin{itemize}
\item {Utilização:Ant.}
\end{itemize}
Pássaro?:«\textunderscore mais vale um avache que dois te darei\textunderscore ». \textunderscore Eufrosina\textunderscore , 61.
\section{Avacuado}
\begin{itemize}
\item {Grp. gram.:adj.}
\end{itemize}
\begin{itemize}
\item {Utilização:Prov.}
\end{itemize}
\begin{itemize}
\item {Utilização:alent.}
\end{itemize}
\begin{itemize}
\item {Proveniência:(De \textunderscore avacuar\textunderscore )}
\end{itemize}
Extenuado, prostrado.
\section{Avacuar}
\begin{itemize}
\item {Grp. gram.:v. t.}
\end{itemize}
\begin{itemize}
\item {Utilização:Prov.}
\end{itemize}
\begin{itemize}
\item {Utilização:alent.}
\end{itemize}
Prostrar.
Extenuar.
(Por \textunderscore evacuar\textunderscore ?)
\section{Avadana}
\begin{itemize}
\item {Grp. gram.:m.}
\end{itemize}
Apólogo búdico. Cf. V. Abreu \textunderscore Contos da Índia\textunderscore , 24.
\section{Avagarar}
\begin{itemize}
\item {Grp. gram.:v. t.}
\end{itemize}
\begin{itemize}
\item {Proveniência:(De \textunderscore vagar\textunderscore )}
\end{itemize}
Tornar vagaroso; retardar. Cf. Filinto, \textunderscore D. Man.\textunderscore , I, 340.
\section{Aval}
\begin{itemize}
\item {Grp. gram.:m.}
\end{itemize}
\begin{itemize}
\item {Utilização:Jur.}
\end{itemize}
\begin{itemize}
\item {Proveniência:(Fr. \textunderscore aval\textunderscore )}
\end{itemize}
Obrigação de pagar uma letra, contrahida por terceiro, assinando documento, que se annexa á letra.
Caução de uma letra de câmbio, consignada na mesma letra.
Caução separada da letra, e constante de um documento, (carta de aval).
\section{Avaladar}
\begin{itemize}
\item {Grp. gram.:v. t.}
\end{itemize}
Rodear com valados.
\section{Avalancha}
\begin{itemize}
\item {Grp. gram.:f.}
\end{itemize}
\begin{itemize}
\item {Utilização:Gal}
\end{itemize}
\begin{itemize}
\item {Proveniência:(Fr. \textunderscore avalanche\textunderscore )}
\end{itemize}
(que deveria substituir-se por \textunderscore alude\textunderscore )
Massa de neve, que rola do monte, derrubando quanto encontra.
Invasão súbita de gente ou de animaes.
Quéda estrondosa de objectos pesados.
\section{Avalentoar-se}
\begin{itemize}
\item {Grp. gram.:v. p.}
\end{itemize}
\begin{itemize}
\item {Utilização:Bras}
\end{itemize}
Tornar-se valentão.
Insubordinar-se.
\section{Avaliação}
\begin{itemize}
\item {Grp. gram.:f.}
\end{itemize}
Apreciação.
Valor, determinado por quem avalia: \textunderscore avaliação de uma propriedade\textunderscore .
Acto de \textunderscore avaliar\textunderscore .
\section{Avaliador}
\begin{itemize}
\item {Grp. gram.:m.}
\end{itemize}
Aquelle que avalia.
\section{Avaliamento}
\begin{itemize}
\item {Grp. gram.:m.}
\end{itemize}
O mesmo que \textunderscore avaliação\textunderscore .
\section{Avaliança}
\begin{itemize}
\item {Grp. gram.:f.}
\end{itemize}
\begin{itemize}
\item {Utilização:Ant.}
\end{itemize}
O mesmo que \textunderscore avaliação\textunderscore .
\section{Avaliar}
\begin{itemize}
\item {Grp. gram.:v. t.}
\end{itemize}
\begin{itemize}
\item {Proveniência:(De \textunderscore valía\textunderscore )}
\end{itemize}
Determinar a valia ou o valor de: \textunderscore avaliar um relógio\textunderscore .
Apreciar o merecimento de: \textunderscore avaliar um escritor\textunderscore .
Reconhecer a força de.
Estimar; calcular.
\section{Avalladar}
\begin{itemize}
\item {Grp. gram.:v. t.}
\end{itemize}
Rodear com vallados.
\section{Avaluar}
\textunderscore v. t.\textunderscore  (e der.)
O mesmo que \textunderscore avaliar\textunderscore , etc.
\section{Avambraços}
\begin{itemize}
\item {Grp. gram.:m. pl.}
\end{itemize}
\begin{itemize}
\item {Proveniência:(Do fr. \textunderscore avant\textunderscore  + \textunderscore bras\textunderscore )}
\end{itemize}
Parte da antiga armadura, para livrar dos golpes os braços.
\section{Avançada}
\begin{itemize}
\item {Grp. gram.:f.}
\end{itemize}
Assalto, investida.
Deanteira.
Acção de \textunderscore avançar\textunderscore .
\section{Avançadamente}
\begin{itemize}
\item {Grp. gram.:adv.}
\end{itemize}
\begin{itemize}
\item {Proveniência:(De \textunderscore avançado\textunderscore )}
\end{itemize}
Com avanço.
\section{Avançado}
\begin{itemize}
\item {Grp. gram.:adj.}
\end{itemize}
Que vai na frente.
Progressivo.
Radical; muito liberal: \textunderscore ideias avançadas\textunderscore .
\section{Avançamento}
\begin{itemize}
\item {Grp. gram.:m.}
\end{itemize}
Parte saliente de um edifício.
Effeito de \textunderscore avançar\textunderscore .
\section{Avançar}
\begin{itemize}
\item {Grp. gram.:v. i.}
\end{itemize}
\begin{itemize}
\item {Grp. gram.:V. t.}
\end{itemize}
Andar para a frente; adeantar-se.
Investir: \textunderscore avançou contra mim\textunderscore .
Fazer saliência.
Continuar.
Progredir.
Fazer andar para deante.
Executar.
Aventurar, ousar: \textunderscore avançar uma affirmação\textunderscore .
(Cast. \textunderscore avanzare\textunderscore )
\section{Avance}
\begin{itemize}
\item {Grp. gram.:m.}
\end{itemize}
O mesmo que \textunderscore avanço\textunderscore .
\section{Avanço}
\begin{itemize}
\item {Grp. gram.:m.}
\end{itemize}
Acção de \textunderscore avançar\textunderscore .
Melhoria.
Deanteira.
Interesse.
Adeantamento de dinheiro.
\section{Avangar}
\begin{itemize}
\item {Grp. gram.:v. i.}
\end{itemize}
\begin{itemize}
\item {Utilização:Prov.}
\end{itemize}
\begin{itemize}
\item {Utilização:trasm.}
\end{itemize}
Pender a um lado com o pêso.
Diz-se especialmente das árvores avergadas de fruto.
\section{Avangelho}
\begin{itemize}
\item {Grp. gram.:m.}
\end{itemize}
\begin{itemize}
\item {Utilização:Ant.}
\end{itemize}
O mesmo que \textunderscore Evangelho\textunderscore .
\section{Avanguarda}
\begin{itemize}
\item {Grp. gram.:f.}
\end{itemize}
\begin{itemize}
\item {Utilização:Des.}
\end{itemize}
\begin{itemize}
\item {Proveniência:(Fr. \textunderscore avant-garde\textunderscore )}
\end{itemize}
O mesmo ou melhor que \textunderscore vanguarda\textunderscore .
\section{Avania}
\begin{itemize}
\item {Grp. gram.:f.}
\end{itemize}
\begin{itemize}
\item {Proveniência:(Gr. mod. \textunderscore avania\textunderscore )}
\end{itemize}
Vexação, que os Turcos faziam aos Christãos.
Afronta pública.
\section{Avantado}
\begin{itemize}
\item {Grp. gram.:adj.}
\end{itemize}
\begin{itemize}
\item {Utilização:Ant.}
\end{itemize}
\begin{itemize}
\item {Proveniência:(De \textunderscore àvante\textunderscore )}
\end{itemize}
Rico.
Engrandecido.
\section{Avantagem}
\begin{itemize}
\item {Grp. gram.:f.}
\end{itemize}
(V.vantagem)
\section{Avantairo}
\begin{itemize}
\item {Grp. gram.:m.}
\end{itemize}
\begin{itemize}
\item {Utilização:Ant.}
\end{itemize}
O mesmo que \textunderscore inventário\textunderscore . Cf. G. Vicente, I.
\section{Avantajadamente}
\begin{itemize}
\item {Grp. gram.:adv.}
\end{itemize}
De modo \textunderscore avantajado\textunderscore .
Com vantagem.
\section{Avantajado}
\begin{itemize}
\item {Grp. gram.:adj.}
\end{itemize}
\begin{itemize}
\item {Proveniência:(De \textunderscore avantajar\textunderscore )}
\end{itemize}
Que tem vantagem, superioridade.
Que excede o que é ordinário: \textunderscore estatura avantajada\textunderscore .
\section{Avantajar}
\begin{itemize}
\item {Grp. gram.:v. t.}
\end{itemize}
\begin{itemize}
\item {Grp. gram.:V. i.}
\end{itemize}
Têr vantagem sôbre.
Exceder.
Tornar superior a.
Melhorar.
Elevar.
Progredir.
\section{Àvante}
\begin{itemize}
\item {Grp. gram.:adv.}
\end{itemize}
\begin{itemize}
\item {Utilização:Náut.}
\end{itemize}
\begin{itemize}
\item {Grp. gram.:Interj.}
\end{itemize}
\begin{itemize}
\item {Grp. gram.:Prep.}
\end{itemize}
\begin{itemize}
\item {Utilização:Ant.}
\end{itemize}
\begin{itemize}
\item {Proveniência:(Do lat. \textunderscore ab\textunderscore  + \textunderscore ante\textunderscore )}
\end{itemize}
Adeante.
Para a frente: \textunderscore caminhar àvante\textunderscore .
\textunderscore Dar por de avante\textunderscore , encher o velame pelo outro bordo, passando com a prôa pela linha do vento.
Eia!; sus!.
Mais acima.
\section{Avantesma}
\begin{itemize}
\item {Grp. gram.:f.}
\end{itemize}
(V.abantesma)
\section{Avaqueirado}
\begin{itemize}
\item {Grp. gram.:adj.}
\end{itemize}
Que tem modos de vaqueiro.
\section{Avaramente}
\begin{itemize}
\item {Grp. gram.:adv.}
\end{itemize}
O mesmo que \textunderscore avarentamente\textunderscore .
\section{Avarandado}
\begin{itemize}
\item {Grp. gram.:adj.}
\end{itemize}
\begin{itemize}
\item {Utilização:Bras. do N}
\end{itemize}
\begin{itemize}
\item {Grp. gram.:M.}
\end{itemize}
Que tem varanda, (falando-se de prédios).
Prédio com varanda.
\section{Avarcas}
\begin{itemize}
\item {Grp. gram.:f. pl.}
\end{itemize}
O mesmo que \textunderscore abarcas\textunderscore .
\section{Avarentamente}
\begin{itemize}
\item {Grp. gram.:adv.}
\end{itemize}
\begin{itemize}
\item {Proveniência:(De \textunderscore avarento\textunderscore )}
\end{itemize}
Com avareza.
\section{Avarento}
\begin{itemize}
\item {Grp. gram.:m.  e  adj.}
\end{itemize}
\begin{itemize}
\item {Proveniência:(De \textunderscore avaro\textunderscore )}
\end{itemize}
O que tem avareza.
\section{Avareza}
\begin{itemize}
\item {Grp. gram.:f.}
\end{itemize}
\begin{itemize}
\item {Proveniência:(Do lat. \textunderscore avaritia\textunderscore )}
\end{itemize}
Excessivo desejo de acumular riquezas.
Mesquinhez.
Zelos.
\section{Avaria}
\begin{itemize}
\item {Grp. gram.:f.}
\end{itemize}
\begin{itemize}
\item {Utilização:Med.}
\end{itemize}
\begin{itemize}
\item {Proveniência:(Do ár. \textunderscore avar\textunderscore )}
\end{itemize}
Prejuizo, causado a um navio ou á sua carga.
Despesa, feita com o salvamento de embarcação.
Direito, que se pagava, para conservação do pôrto, em que se lançava ferro.
Qualquer prejuizo; estragos.
O mesmo que \textunderscore avarigênese\textunderscore .
\section{Avariado}
(V.variado)
\section{Avariado}
\begin{itemize}
\item {Grp. gram.:adj.}
\end{itemize}
\begin{itemize}
\item {Utilização:Med.}
\end{itemize}
Damnificado, estragado: \textunderscore peixe avariado\textunderscore .
Que sofre avaria ou avarigênese.
\section{Avariar}
\begin{itemize}
\item {Grp. gram.:v. t.}
\end{itemize}
\begin{itemize}
\item {Grp. gram.:V. i.}
\end{itemize}
Causar avaria a.
Prejudicar.
Damnificar-se; estragar-se: \textunderscore o trigo avariou no celleiro\textunderscore .
\section{Avarícia}
\begin{itemize}
\item {Grp. gram.:f.}
\end{itemize}
\begin{itemize}
\item {Utilização:Des.}
\end{itemize}
O mesmo que \textunderscore avareza\textunderscore . Cf. Barros, \textunderscore Déc.\textunderscore  III, 262.
\section{Avarigênese}
\begin{itemize}
\item {Grp. gram.:f.}
\end{itemize}
\begin{itemize}
\item {Utilização:Med.}
\end{itemize}
Conjunto do micróbio da avariose e da neisserose.
\section{Avariose}
\begin{itemize}
\item {Grp. gram.:f.}
\end{itemize}
\begin{itemize}
\item {Utilização:Med.}
\end{itemize}
Nova designação da sýphilis.
Infecção especifica ou grande avaria.
\section{Avarismo}
\begin{itemize}
\item {Grp. gram.:m.}
\end{itemize}
\begin{itemize}
\item {Proveniência:(De \textunderscore avaro\textunderscore )}
\end{itemize}
Avareza systemática.
\section{Avaro}
\begin{itemize}
\item {Grp. gram.:adj.}
\end{itemize}
\begin{itemize}
\item {Proveniência:(Lat. \textunderscore avarus\textunderscore )}
\end{itemize}
O mesmo que \textunderscore avarento\textunderscore .
\section{Avassalador}
\begin{itemize}
\item {Grp. gram.:m.}
\end{itemize}
Aquelle que avassalla.
\section{Avassalante}
\begin{itemize}
\item {Grp. gram.:adj.}
\end{itemize}
Que avassalla.
\section{Avassalar}
\begin{itemize}
\item {Grp. gram.:v. t.}
\end{itemize}
Tornar vassallo.
Dominar.
Opprimir.
\section{Avassallador}
\begin{itemize}
\item {Grp. gram.:m.}
\end{itemize}
Aquelle que avassalla.
\section{Avassallante}
\begin{itemize}
\item {Grp. gram.:adj.}
\end{itemize}
Que avassalla.
\section{Avassallar}
\begin{itemize}
\item {Grp. gram.:v. t.}
\end{itemize}
Tornar vassallo.
Dominar.
Opprimir.
\section{Avatara}
\begin{itemize}
\item {Grp. gram.:m.}
\end{itemize}
\begin{itemize}
\item {Utilização:Fig.}
\end{itemize}
Designação genérica das encarnações divinas, na theogonia brahmânica.
Transformação.
(Do sânscr.)
\section{Avati}
\begin{itemize}
\item {Grp. gram.:m.}
\end{itemize}
Milho indígena do Brasil.
\section{Ave}
\begin{itemize}
\item {Grp. gram.:f.}
\end{itemize}
\begin{itemize}
\item {Utilização:Bras}
\end{itemize}
\begin{itemize}
\item {Proveniência:(Lat. \textunderscore avis\textunderscore )}
\end{itemize}
Animal vertebrado, de respiração e circulação duplas, coração dividido em quatro cavidades, e a pelle coberta de pennas, bico córneo e desdentado.
Pessôa astuta e ladra.
\section{Avè!}
\begin{itemize}
\item {Grp. gram.:interj.}
\end{itemize}
\begin{itemize}
\item {Grp. gram.:M.}
\end{itemize}
\begin{itemize}
\item {Proveniência:(Lat. \textunderscore ave\textunderscore , imp. do defect. \textunderscore avere\textunderscore )}
\end{itemize}
(para saudar)
Saudação.
\section{Aveaco}
\begin{itemize}
\item {Grp. gram.:m.}
\end{itemize}
\begin{itemize}
\item {Utilização:Ant.}
\end{itemize}
Pão de aveia.
\section{Aveado}
\begin{itemize}
\item {Grp. gram.:adj.}
\end{itemize}
\begin{itemize}
\item {Utilização:P. us.}
\end{itemize}
Que tem veia de doido ou veneta.
Adoidado.
\section{Aveal}
\begin{itemize}
\item {Grp. gram.:m.}
\end{itemize}
Campo, semeado de aveia.
\section{Avècer}
\begin{itemize}
\item {Grp. gram.:v. i.}
\end{itemize}
\begin{itemize}
\item {Utilização:Prov.}
\end{itemize}
\begin{itemize}
\item {Utilização:minh.}
\end{itemize}
\begin{itemize}
\item {Proveniência:(Do lat. \textunderscore valescere\textunderscore )}
\end{itemize}
Correr, decorrer.
\section{Avecla}
\begin{itemize}
\item {Grp. gram.:f.}
\end{itemize}
\begin{itemize}
\item {Utilização:Ant.}
\end{itemize}
Caminho estreito, azinhaga.
\section{Avecoínha}
\begin{itemize}
\item {Grp. gram.:f.}
\end{itemize}
\begin{itemize}
\item {Utilização:Prov.}
\end{itemize}
O mesmo que \textunderscore abibe\textunderscore .
\section{Avectos}
\begin{itemize}
\item {Grp. gram.:m. pl.}
\end{itemize}
\begin{itemize}
\item {Utilização:Ant.}
\end{itemize}
Alfaias de igreja? Cf. \textunderscore Testamento de D. Mumadona\textunderscore .
\section{Aveediço}
\begin{itemize}
\item {Grp. gram.:m.}
\end{itemize}
\begin{itemize}
\item {Utilização:Ant.}
\end{itemize}
Vindiço, estranho.
\section{Ave-fria}
\begin{itemize}
\item {Grp. gram.:f.}
\end{itemize}
\begin{itemize}
\item {Utilização:Prov.}
\end{itemize}
O mesmo que \textunderscore abibe\textunderscore .
\section{Aveia}
\begin{itemize}
\item {Grp. gram.:f.}
\end{itemize}
\begin{itemize}
\item {Proveniência:(Lat. \textunderscore avena\textunderscore )}
\end{itemize}
Planta gramínea, alimentícia.
\section{Aveirense}
\begin{itemize}
\item {Grp. gram.:adj.}
\end{itemize}
\begin{itemize}
\item {Grp. gram.:M.}
\end{itemize}
Relativo a Aveiro.
Habitante de Aveiro.
\section{Avejan}
\begin{itemize}
\item {Grp. gram.:f.}
\end{itemize}
\begin{itemize}
\item {Utilização:Pop.}
\end{itemize}
O mesmo que \textunderscore avejão\textunderscore .
\section{Avejão}
\begin{itemize}
\item {Grp. gram.:m.}
\end{itemize}
\begin{itemize}
\item {Utilização:Pop.}
\end{itemize}
Visão; abantesma.
Pessôa muito encorpada.
(Por \textunderscore abujão\textunderscore , do lat. \textunderscore abusio\textunderscore )
\section{Ável}
\begin{itemize}
\item {Grp. gram.:m.}
\end{itemize}
O mesmo que \textunderscore ávela\textunderscore .
\section{Ávela}
\begin{itemize}
\item {Grp. gram.:f.}
\end{itemize}
Arroz torrado, que se usa na Índia.
\section{Avelã}
\begin{itemize}
\item {Grp. gram.:f.}
\end{itemize}
\begin{itemize}
\item {Proveniência:(Lat. \textunderscore avellana\textunderscore )}
\end{itemize}
Fruto da aveleira.
\section{Avelado}
\begin{itemize}
\item {Grp. gram.:adj.}
\end{itemize}
\begin{itemize}
\item {Utilização:Prov.}
\end{itemize}
\begin{itemize}
\item {Utilização:minh.}
\end{itemize}
Um tanto húmido; lento.
\section{Avelal}
\begin{itemize}
\item {Grp. gram.:m.}
\end{itemize}
O mesmo que \textunderscore avelanal\textunderscore .
\section{Avelamento}
\begin{itemize}
\item {Grp. gram.:m.}
\end{itemize}
Acto de \textunderscore avelar\textunderscore . Cf. \textunderscore Techn. Rur.\textunderscore , 87.
\section{Avelanado}
\begin{itemize}
\item {Grp. gram.:adj.}
\end{itemize}
Que tem côr de avelã.
\section{Avelanal}
\begin{itemize}
\item {Grp. gram.:m.}
\end{itemize}
\begin{itemize}
\item {Proveniência:(De \textunderscore avellan\textunderscore )}
\end{itemize}
Lugar, onde crescem aveleiras.
\section{Avelaneira}
\begin{itemize}
\item {Grp. gram.:f.}
\end{itemize}
(V.aveleira)
\section{Avelanzeira}
\begin{itemize}
\item {Grp. gram.:f.}
\end{itemize}
O mesmo que \textunderscore aveleira\textunderscore . Cf. Camillo, \textunderscore Scenas da Foz\textunderscore , 71.
\section{Avelar}
\begin{itemize}
\item {Grp. gram.:v. i.}
\end{itemize}
\begin{itemize}
\item {Utilização:Pop.}
\end{itemize}
\begin{itemize}
\item {Grp. gram.:M.}
\end{itemize}
\begin{itemize}
\item {Proveniência:(De \textunderscore avellan\textunderscore )}
\end{itemize}
Engelhar, secando-se casca.
Encarquilhar.
Envelhecer.
Sêr velho, e ir durando, sem alteração de aspecto nem de fôrças.
O mesmo que \textunderscore avelanal\textunderscore .
\section{Aveleira}
\begin{itemize}
\item {Grp. gram.:f.}
\end{itemize}
\begin{itemize}
\item {Proveniência:(De \textunderscore avellan\textunderscore )}
\end{itemize}
Árvore amentácea, cujo fruto é uma glande.
\section{Aveleiral}
\begin{itemize}
\item {Grp. gram.:m.}
\end{itemize}
O mesmo que \textunderscore avelanal\textunderscore .
\section{Avelhacado}
\begin{itemize}
\item {Grp. gram.:adj.}
\end{itemize}
Que é um tanto velhaco.
\section{Avelhado}
\begin{itemize}
\item {Grp. gram.:adj.}
\end{itemize}
Que se vai tornando velho.
\section{Avelhar}
\begin{itemize}
\item {Grp. gram.:v. i.  e  p.}
\end{itemize}
O mesmo que [[avelhentar-se|avelhentar]].
\section{Avelhentado}
\begin{itemize}
\item {Grp. gram.:adj.}
\end{itemize}
Um tanto velho.
Que se avelhentou prematuramente.
\section{Avelhentador}
\begin{itemize}
\item {Grp. gram.:adj.}
\end{itemize}
Que faz avelhentar.
\section{Avelhentar}
\begin{itemize}
\item {Grp. gram.:v. t.}
\end{itemize}
Tornar velho prematuramente.
Abater o vigor de (gente moça).
\section{Avellã}
\begin{itemize}
\item {Grp. gram.:f.}
\end{itemize}
\begin{itemize}
\item {Proveniência:(Lat. \textunderscore avellana\textunderscore )}
\end{itemize}
Fruto da avelleira.
\section{Avellal}
\begin{itemize}
\item {Grp. gram.:m.}
\end{itemize}
O mesmo que \textunderscore avellanal\textunderscore .
\section{Avellamento}
\begin{itemize}
\item {Grp. gram.:m.}
\end{itemize}
Acto de \textunderscore avellar\textunderscore . Cf. \textunderscore Techn. Rur.\textunderscore , 87.
\section{Avellan}
\begin{itemize}
\item {Grp. gram.:f.}
\end{itemize}
\begin{itemize}
\item {Proveniência:(Lat. \textunderscore avellana\textunderscore )}
\end{itemize}
Fruto da avelleira.
\section{Avellanado}
\begin{itemize}
\item {Grp. gram.:adj.}
\end{itemize}
Que tem côr de avellan.
\section{Avellanal}
\begin{itemize}
\item {Grp. gram.:m.}
\end{itemize}
\begin{itemize}
\item {Proveniência:(De \textunderscore avellan\textunderscore )}
\end{itemize}
Lugar, onde crescem avelleiras.
\section{Avellaneira}
\begin{itemize}
\item {Grp. gram.:f.}
\end{itemize}
(V.avelleira)
\section{Avellanzeira}
\begin{itemize}
\item {Grp. gram.:f.}
\end{itemize}
O mesmo que \textunderscore avelleira\textunderscore . Cf. Camillo, \textunderscore Scenas da Foz\textunderscore , 71.
\section{Avellar}
\begin{itemize}
\item {Grp. gram.:v. i.}
\end{itemize}
\begin{itemize}
\item {Utilização:Pop.}
\end{itemize}
\begin{itemize}
\item {Grp. gram.:M.}
\end{itemize}
\begin{itemize}
\item {Proveniência:(De \textunderscore avellan\textunderscore )}
\end{itemize}
Engelhar, secando-se casca.
Encarquilhar.
Envelhecer.
Sêr velho, e ir durando, sem alteração de aspecto nem de fôrças.
O mesmo que \textunderscore avellanal\textunderscore .
\section{Avelleira}
\begin{itemize}
\item {Grp. gram.:f.}
\end{itemize}
\begin{itemize}
\item {Proveniência:(De \textunderscore avellan\textunderscore )}
\end{itemize}
Árvore amentácea, cujo fruto é uma glande.
\section{Avelleiral}
\begin{itemize}
\item {Grp. gram.:m.}
\end{itemize}
O mesmo que \textunderscore avellanal\textunderscore .
\section{Avelludadeira}
\begin{itemize}
\item {Grp. gram.:f.}
\end{itemize}
O mesmo que \textunderscore avelludadora\textunderscore .
\section{Avelludado}
\begin{itemize}
\item {Grp. gram.:adj.}
\end{itemize}
Macio como velludo.
\section{Avelludadora}
\begin{itemize}
\item {Grp. gram.:f.}
\end{itemize}
Mulher, que em certas fábricas é encarregada de avelludar os tecidos. Cf. \textunderscore Inquér. Industr.\textunderscore , p. II, 108,
109 e 158.
\section{Avelludar}
\begin{itemize}
\item {Grp. gram.:v. t.}
\end{itemize}
Dar apparência de velludo a.
\section{Avelórios}
\begin{itemize}
\item {Grp. gram.:m. pl.}
\end{itemize}
Contas de vidro ou missanga.
Bagatelas.
(Cast. \textunderscore abelorio\textunderscore )
\section{Avelós}
\begin{itemize}
\item {Grp. gram.:m.}
\end{itemize}
\begin{itemize}
\item {Utilização:Bras. do N}
\end{itemize}
Planta euphorbiácea, medicinal.
\section{Avélroa}
\begin{itemize}
\item {Grp. gram.:f.}
\end{itemize}
\begin{itemize}
\item {Utilização:Prov.}
\end{itemize}
O mesmo que \textunderscore alvéloa\textunderscore .
(Metáth. de \textunderscore arvéloa\textunderscore )
\section{Aveludadeira}
\begin{itemize}
\item {Grp. gram.:f.}
\end{itemize}
O mesmo que \textunderscore aveludadora\textunderscore .
\section{Aveludado}
\begin{itemize}
\item {Grp. gram.:adj.}
\end{itemize}
Macio como veludo.
\section{Aveludadora}
\begin{itemize}
\item {Grp. gram.:f.}
\end{itemize}
Mulher, que em certas fábricas é encarregada de aveludar os tecidos. Cf. \textunderscore Inquér. Industr.\textunderscore , p. II, 108, 109 e 158.
\section{Aveludar}
\begin{itemize}
\item {Grp. gram.:v. t.}
\end{itemize}
Dar apparência de velludo a.
\section{Avèmaria}
\begin{itemize}
\item {Grp. gram.:f.}
\end{itemize}
\begin{itemize}
\item {Grp. gram.:Pl.}
\end{itemize}
\begin{itemize}
\item {Proveniência:(De \textunderscore avè\textunderscore  + \textunderscore Maria\textunderscore , n. p.)}
\end{itemize}
Oração, que os Christãos consagram á Virgem Maria.
Conta de rosário, menor que o padre-nosso.
Tempo, que se gasta em rezar uma avèmaria.
Cada um dos tres toques, que o sino da igreja dá por dia, para de cada vez se rezar três vezes a avèmaria.
Tarde, boca da noite: \textunderscore encontraram-se ás avemarias\textunderscore .
Hora, em que o sino dá o toque das últimas avèmarias dêsse dia.
\section{Avemaria}
\begin{itemize}
\item {Grp. gram.:f.}
\end{itemize}
\begin{itemize}
\item {Grp. gram.:Pl.}
\end{itemize}
\begin{itemize}
\item {Proveniência:(De \textunderscore avè\textunderscore  + \textunderscore Maria\textunderscore , n. p.)}
\end{itemize}
Oração, que os Christãos consagram á Virgem Maria.
Conta de rosário, menor que o padre-nosso.
Tempo, que se gasta em rezar uma avèmaria.
Cada um dos tres toques, que o sino da igreja dá por dia, para de cada vez se rezar três vezes a avèmaria.
Tarde, boca da noite: \textunderscore encontraram-se ás avemarias\textunderscore .
Hora, em que o sino dá o toque das últimas avèmarias dêsse dia.
\section{Avena}
\begin{itemize}
\item {Grp. gram.:f.}
\end{itemize}
\begin{itemize}
\item {Proveniência:(Lat. \textunderscore avena\textunderscore )}
\end{itemize}
Antiga frauta pastoril.
O mesmo que \textunderscore aveia\textunderscore .
\section{Avenáceas}
\begin{itemize}
\item {Grp. gram.:f.}
\end{itemize}
\begin{itemize}
\item {Proveniência:(De \textunderscore avenáceo\textunderscore )}
\end{itemize}
Tríbo de plantas gramíneas.
\section{Avenáceo}
\begin{itemize}
\item {Grp. gram.:adj.}
\end{itemize}
\begin{itemize}
\item {Proveniência:(Do lat. \textunderscore avena\textunderscore )}
\end{itemize}
Relativo ou semelhante á aveia.
\section{Avenaína}
\begin{itemize}
\item {Grp. gram.:f.}
\end{itemize}
\begin{itemize}
\item {Proveniência:(Do lat. \textunderscore avena\textunderscore )}
\end{itemize}
Glúten da aveia.
\section{Avenca}
\begin{itemize}
\item {Grp. gram.:f.}
\end{itemize}
\begin{itemize}
\item {Proveniência:(Do lat. \textunderscore vinca\textunderscore )}
\end{itemize}
Planta herbácea, medicinal, também chamada \textunderscore capillária\textunderscore .
\section{Avença}
\begin{itemize}
\item {Grp. gram.:f.}
\end{itemize}
Ajuste: \textunderscore o taberneiro pagou por avença a contribuição\textunderscore .
Conciliação entre litigantes.
Concórdia.
(Cp. \textunderscore aveniência\textunderscore )
\section{Avençadura}
\begin{itemize}
\item {Grp. gram.:f.}
\end{itemize}
\begin{itemize}
\item {Utilização:Náut.}
\end{itemize}
\begin{itemize}
\item {Utilização:ant.}
\end{itemize}
Enxárcia real.
\section{Avençal}
\begin{itemize}
\item {Grp. gram.:m. ,  f.  e  adj.}
\end{itemize}
\begin{itemize}
\item {Proveniência:(De \textunderscore avença\textunderscore )}
\end{itemize}
Pessôa, que está avençada.
\section{Avencão}
\begin{itemize}
\item {Grp. gram.:m.}
\end{itemize}
Gênero de plantas, da mesma fam. da avenca.
\section{Avençar}
\begin{itemize}
\item {Grp. gram.:v. i.}
\end{itemize}
\begin{itemize}
\item {Grp. gram.:V. p.}
\end{itemize}
Fazer contracto de avença.
Obrigar-se por avença: \textunderscore avençar-se com a Companhia das Aguas\textunderscore .
\section{Avendiço}
\begin{itemize}
\item {Grp. gram.:adj.}
\end{itemize}
\begin{itemize}
\item {Utilização:Ant.}
\end{itemize}
O mesmo que \textunderscore adventicio\textunderscore .
\section{Avenenar}
\begin{itemize}
\item {Grp. gram.:v. t.}
\end{itemize}
(V.envenenar)
\section{Avenida}
\begin{itemize}
\item {Grp. gram.:f.}
\end{itemize}
\begin{itemize}
\item {Utilização:Ant.}
\end{itemize}
\begin{itemize}
\item {Proveniência:(Do lat. \textunderscore venire\textunderscore , sob o infl. do fr. \textunderscore avenue\textunderscore )}
\end{itemize}
Caminho, estrada, que leva a determinado lugar.
Rua, ladeada de árvores.
Alameda.
Chamada, assalto:«\textunderscore resistida foi a primeira fúria da avenida\textunderscore ». \textunderscore Viriato Trág.\textunderscore , XVII, 63.
\section{Aveniência}
\begin{itemize}
\item {Grp. gram.:f.}
\end{itemize}
\begin{itemize}
\item {Utilização:Ant.}
\end{itemize}
Contrato, ajuste, combinação, avença.
(B. lat. \textunderscore advenientia\textunderscore )
\section{Avenina}
\begin{itemize}
\item {Grp. gram.:f.}
\end{itemize}
Essência aromática, excitante, contida na aveia. Cf. E. Monin, \textunderscore Hyg. do Estôm.\textunderscore , 51.
\section{Avenose}
\begin{itemize}
\item {Grp. gram.:f.}
\end{itemize}
\begin{itemize}
\item {Proveniência:(Do lat. \textunderscore avena\textunderscore )}
\end{itemize}
Medicamento, em que entra uma mistura de farinha de aveia.
\section{Aventejado}
\begin{itemize}
\item {Grp. gram.:adj.}
\end{itemize}
O mesmo que \textunderscore avantajado\textunderscore , afortunado. Cf. D. Bernardes, \textunderscore Lima\textunderscore , 97.
\section{Avental}
\begin{itemize}
\item {Grp. gram.:m.}
\end{itemize}
\begin{itemize}
\item {Utilização:Prov.}
\end{itemize}
\begin{itemize}
\item {Utilização:alent.}
\end{itemize}
Resguardo de pano ou de coiro, que se põe por deante do fato, para evitar que êste se suje ou se estrague.
Peça de ornato, pendente da cintura das damas, por deante das saias.
Resguardo, que, em algumas carruagens, livra da chuva e da lama quem vai nêsses vehículos.
Pedaço de madeira, de cortiça ou de chapéu, que se põe na barriga dos carneiros ou dos bodes, para lhes impedir a cópula.
(Por \textunderscore avantal\textunderscore , de \textunderscore àvante\textunderscore )
\section{Aventar}
\begin{itemize}
\item {Grp. gram.:v. t.}
\end{itemize}
Expor, agitar, ao vento: \textunderscore aventar o trigo na eira\textunderscore .
Atirar; deitar fóra.
Enunciar, aventurar (uma proposição, uma ideia).
Entrever, perceber ao longe.
\section{Aventejar}
\begin{itemize}
\item {Grp. gram.:v. i.}
\end{itemize}
\begin{itemize}
\item {Utilização:Prov.}
\end{itemize}
\begin{itemize}
\item {Utilização:alent.}
\end{itemize}
\begin{itemize}
\item {Proveniência:(De \textunderscore vento\textunderscore )}
\end{itemize}
Procurar as emanações da caça, trazidas pelo vento, (falando-se do cão).
\section{Avento}
\begin{itemize}
\item {Grp. gram.:m.}
\end{itemize}
\begin{itemize}
\item {Utilização:Ant.}
\end{itemize}
O mesmo que \textunderscore advento\textunderscore :«\textunderscore tudo tem o seu tempo e os nabos em avento\textunderscore ». \textunderscore Eufrosina\textunderscore , act. I, sc. 1.
\section{Aventura}
\begin{itemize}
\item {Grp. gram.:f.}
\end{itemize}
O que succede imprevistamente.
Acção arriscada.
Proêza guerreira dos cavalleiros andantes.
Acaso.
(Cp. \textunderscore ventura\textunderscore )
\section{Aventurar}
\begin{itemize}
\item {Grp. gram.:v. t.}
\end{itemize}
Sujeitar a ventura; arriscar.
Aventar, expor (uma proposição, uma ideia).
Tornar venturoso:«\textunderscore venturas suas me aventurariam.\textunderscore »Castilho, \textunderscore Camões\textunderscore , 27.
\section{Aventureiro}
\begin{itemize}
\item {Grp. gram.:m.}
\end{itemize}
\begin{itemize}
\item {Grp. gram.:Adj.}
\end{itemize}
Aquelle que procura aventuras.
Vagabundo.
Pessôa temerária.
Aquelle que não tem conhecidos meios de vida.
Que vive de aventuras.
Arriscado, incerto, precário.
\section{Aventurina}
\begin{itemize}
\item {Grp. gram.:f.}
\end{itemize}
\begin{itemize}
\item {Proveniência:(Fr. \textunderscore aventurine\textunderscore )}
\end{itemize}
Conta de vidro, mesclada de limalha de cobre.
Quartzo semitransparente, colorido de palhetas amarelas, verdes ou encarnadas.
\section{Aventurinado}
\begin{itemize}
\item {Grp. gram.:adj.}
\end{itemize}
Parecido á \textunderscore venturina\textunderscore .
\section{Aventurosamente}
\begin{itemize}
\item {Grp. gram.:adv.}
\end{itemize}
De modo \textunderscore aventuroso\textunderscore .
\section{Aventuroso}
\begin{itemize}
\item {Grp. gram.:m.}
\end{itemize}
Que se aventura.
Arriscado.
\section{Averano}
\begin{itemize}
\item {Grp. gram.:m.}
\end{itemize}
\begin{itemize}
\item {Proveniência:(De \textunderscore ave\textunderscore  + \textunderscore verão\textunderscore )}
\end{itemize}
Pássaro do Brasil, ave de verão.
\section{Averbação}
\begin{itemize}
\item {Grp. gram.:f.}
\end{itemize}
(V.averbamento)
\section{Averbadamente}
\begin{itemize}
\item {Grp. gram.:adv.}
\end{itemize}
Por meio de averbamento.
\section{Averbamento}
\begin{itemize}
\item {Grp. gram.:m.}
\end{itemize}
Acção de \textunderscore averbar\textunderscore .
\section{Averbar}
\begin{itemize}
\item {Grp. gram.:v. t.}
\end{itemize}
\begin{itemize}
\item {Utilização:Gram.}
\end{itemize}
\begin{itemize}
\item {Proveniência:(De \textunderscore verbo\textunderscore  e de \textunderscore verba\textunderscore )}
\end{itemize}
Escrever em verba, á margem de (um titulo): \textunderscore averbar inscripções\textunderscore .
Registar.
Apodar, accusar: \textunderscore averbar de inepto\textunderscore .
Empregar como verbo.
\section{Averduga}
\begin{itemize}
\item {Grp. gram.:f.}
\end{itemize}
O mesmo que \textunderscore averdugada\textunderscore .
\section{Averdugada}
\begin{itemize}
\item {Grp. gram.:f.}
\end{itemize}
\begin{itemize}
\item {Utilização:Ant.}
\end{itemize}
Sáia larga, com arcos.
Crinolina; merinaque.
\section{Averdugado}
\begin{itemize}
\item {Grp. gram.:adj.}
\end{itemize}
\begin{itemize}
\item {Utilização:Prov.}
\end{itemize}
\begin{itemize}
\item {Utilização:alent.}
\end{itemize}
\begin{itemize}
\item {Proveniência:(De \textunderscore averdugar\textunderscore )}
\end{itemize}
Flexível; que se dobra facilmente.
\section{Averdugar}
\begin{itemize}
\item {Grp. gram.:v. i.}
\end{itemize}
\begin{itemize}
\item {Utilização:Prov.}
\end{itemize}
\begin{itemize}
\item {Utilização:alent.}
\end{itemize}
\begin{itemize}
\item {Proveniência:(De \textunderscore verde\textunderscore )}
\end{itemize}
Sêr flexível.
Vergar ao pêso da fruta, (falando-se de árvores).
\section{Averdungado}
\begin{itemize}
\item {Grp. gram.:adj.}
\end{itemize}
Esverdeado; tirante a verde.
\section{Ave-real}
\begin{itemize}
\item {Grp. gram.:f.}
\end{itemize}
Ave pernalta e ribeirinha, de formoso tufo de pennas ou sedas na cabeça, (\textunderscore ardea pavonia\textunderscore , Lin.).
\section{Avergalhar}
\begin{itemize}
\item {Grp. gram.:v. t.}
\end{itemize}
Bater com vergalho em. Cf. Arn. Gama, \textunderscore Motim\textunderscore , 516.
\section{Avergar}
\begin{itemize}
\item {Grp. gram.:v. i.  e  p.}
\end{itemize}
Inclinar-se com o pêso; vergar: \textunderscore árvores avergadas de fruto\textunderscore .
\section{Avergoar}
\begin{itemize}
\item {Grp. gram.:v. t.}
\end{itemize}
Fazer vergões em.
Maltratar; espancar.
\section{Avergonhar}
\begin{itemize}
\item {Grp. gram.:v. t.}
\end{itemize}
(V.envergonhar). Cf. Usque, \textunderscore Tribulações\textunderscore , 41.
\section{Averiguação}
\begin{itemize}
\item {Grp. gram.:f.}
\end{itemize}
Acção de \textunderscore averiguar\textunderscore .
\section{Averiguadamente}
\begin{itemize}
\item {Grp. gram.:adv.}
\end{itemize}
\begin{itemize}
\item {Proveniência:(De \textunderscore averiguar\textunderscore )}
\end{itemize}
Com provas certas.
\section{Averiguador}
\begin{itemize}
\item {Grp. gram.:m.  e  adj.}
\end{itemize}
O que averigua.
\section{Averiguar}
\begin{itemize}
\item {Grp. gram.:v. t.}
\end{itemize}
\begin{itemize}
\item {Grp. gram.:V. i.}
\end{itemize}
\begin{itemize}
\item {Utilização:Prov.}
\end{itemize}
\begin{itemize}
\item {Utilização:minh.}
\end{itemize}
\begin{itemize}
\item {Proveniência:(Do lat. \textunderscore ad\textunderscore  + \textunderscore verificare\textunderscore ?)}
\end{itemize}
Investigar; inquirir: \textunderscore vou averiguar o que há\textunderscore .
Determinar a verdade de; verificar: \textunderscore averiguei que estava enganado\textunderscore . Experimentar.
Concertar, combinar.
Expellir as secundinas.
\section{Averiguável}
\begin{itemize}
\item {Grp. gram.:adj.}
\end{itemize}
Que se póde \textunderscore averiguar\textunderscore .
\section{Avermelhado}
\begin{itemize}
\item {Grp. gram.:adj.}
\end{itemize}
Tirante a vermelho.
\section{Avermelhar}
\begin{itemize}
\item {Grp. gram.:v. t.}
\end{itemize}
Tornar vermelho, tornar um tanto vermelho.
\section{Avernal}
\begin{itemize}
\item {Grp. gram.:adj.}
\end{itemize}
Infernal.
Relativo ao \textunderscore averno\textunderscore : \textunderscore lagos avernaes\textunderscore .
\section{Avérneo}
\begin{itemize}
\item {Grp. gram.:adj.}
\end{itemize}
O mesmo que \textunderscore avernal\textunderscore . Cf. Filinto, XVI, 61.
\section{Averno}
\begin{itemize}
\item {Grp. gram.:m.}
\end{itemize}
\begin{itemize}
\item {Grp. gram.:Adj.}
\end{itemize}
\begin{itemize}
\item {Proveniência:(Lat. \textunderscore avernus\textunderscore )}
\end{itemize}
Inferno.
O mesmo que \textunderscore avernal\textunderscore .
\section{Avernoso}
\begin{itemize}
\item {Grp. gram.:adj.}
\end{itemize}
(V.avernal)
\section{Averroísmo}
\begin{itemize}
\item {Grp. gram.:m.}
\end{itemize}
Doutrina philosóphica de Averroes.
\section{Averrugar}
\begin{itemize}
\item {Grp. gram.:v. t.}
\end{itemize}
(V.enverrugar)
\section{Aversamente}
\begin{itemize}
\item {Grp. gram.:adv.}
\end{itemize}
\begin{itemize}
\item {Proveniência:(De \textunderscore averso\textunderscore )}
\end{itemize}
Com aversão.
\section{Aversão}
\begin{itemize}
\item {Grp. gram.:f.}
\end{itemize}
\begin{itemize}
\item {Proveniência:(Lat. \textunderscore aversio\textunderscore )}
\end{itemize}
Sentimento, que nos afasta de uma pessôa ou coisa.
Antipathia.
Ódio.
\section{Aversário}
\begin{itemize}
\item {Grp. gram.:adj.}
\end{itemize}
\begin{itemize}
\item {Utilização:Ant.}
\end{itemize}
O mesmo que \textunderscore adversário\textunderscore .
\section{Averso}
\begin{itemize}
\item {Grp. gram.:adj.}
\end{itemize}
O mesmo que \textunderscore adverso\textunderscore . Cf. C. Lobo, \textunderscore Sát. de Juv.\textunderscore , I, 47.
\section{Averter}
\begin{itemize}
\item {Grp. gram.:v. t.}
\end{itemize}
\begin{itemize}
\item {Proveniência:(Lat. \textunderscore advertere\textunderscore )}
\end{itemize}
Desviar do seu curso. Cf. Castilho, \textunderscore Fastos\textunderscore , III, 137.
\section{Avespa}
\begin{itemize}
\item {Grp. gram.:f.}
\end{itemize}
\begin{itemize}
\item {Utilização:Pop.}
\end{itemize}
O mesmo que \textunderscore vespa\textunderscore .
\section{Avessada}
\begin{itemize}
\item {Grp. gram.:f.}
\end{itemize}
\begin{itemize}
\item {Proveniência:(De \textunderscore avessado\textunderscore )}
\end{itemize}
Correia, com que se prendia o falcão á vara.
\section{Avessado}
\begin{itemize}
\item {Grp. gram.:adj.}
\end{itemize}
\begin{itemize}
\item {Utilização:T. da Bairrada}
\end{itemize}
\begin{itemize}
\item {Proveniência:(Do lat. \textunderscore adversatus\textunderscore )}
\end{itemize}
O mesmo que \textunderscore arrevesado\textunderscore , (falando-se de nomes ou palavras que custa pronunciar).
\section{Avessamente}
\begin{itemize}
\item {Grp. gram.:adv.}
\end{itemize}
\begin{itemize}
\item {Proveniência:(De \textunderscore avesso\textunderscore )}
\end{itemize}
Ás avessas.
De má vontade.
\section{Avessar}
\begin{itemize}
\item {Grp. gram.:v. t.}
\end{itemize}
Tornar avesso.
\section{Avessas}
\begin{itemize}
\item {Grp. gram.:f. pl.}
\end{itemize}
\begin{itemize}
\item {Grp. gram.:Loc. adv.}
\end{itemize}
\begin{itemize}
\item {Proveniência:(De \textunderscore avesso\textunderscore )}
\end{itemize}
Aquillo que é contrário.
\textunderscore Ás avessas\textunderscore , ao contrário, em sentido opposto: \textunderscore trazia o casaco ás avessas\textunderscore .
\section{Avessedo}
\begin{itemize}
\item {fónica:sê}
\end{itemize}
\begin{itemize}
\item {Grp. gram.:m.}
\end{itemize}
\begin{itemize}
\item {Utilização:Prov.}
\end{itemize}
\begin{itemize}
\item {Utilização:trasm.}
\end{itemize}
Encosta de montanha, do lado do norte.
(Cp. \textunderscore avesso\textunderscore . V. \textunderscore soalheiro\textunderscore )
\section{Avessia}
\begin{itemize}
\item {Grp. gram.:f.}
\end{itemize}
Qualidade do que é avesso.
\section{Avessidade}
\begin{itemize}
\item {Grp. gram.:f.}
\end{itemize}
Qualidade de avesso. Cf. Alex. Lobo, III, 311.
\section{Avesso}
\begin{itemize}
\item {fónica:vê}
\end{itemize}
\begin{itemize}
\item {Grp. gram.:m.}
\end{itemize}
\begin{itemize}
\item {Grp. gram.:Adj.}
\end{itemize}
\begin{itemize}
\item {Proveniência:(Lat. \textunderscore adversus\textunderscore )}
\end{itemize}
A parte opposta á superfície ou parte principal de um objecto.
O que há de occulto num carácter.
Reverso.
O lado mau.
Defeito; êrro.
Contrário.
Opposto ao que deve sêr.
Mau.
\section{Avestruz}
\begin{itemize}
\item {Proveniência:(Do lat. \textunderscore avis\textunderscore  + \textunderscore struthio\textunderscore )}
\end{itemize}
\textunderscore f.\textunderscore  (\textunderscore m.\textunderscore , segundo outros)
Grande ave, da ordem das pernaltas.
\section{Avetoninha}
\begin{itemize}
\item {Grp. gram.:f.}
\end{itemize}
\begin{itemize}
\item {Utilização:Prov.}
\end{itemize}
O mesmo que \textunderscore abibe\textunderscore .
\section{Avexar}
\textunderscore v. t.\textunderscore  (e der.)
O mesmo que \textunderscore vexar\textunderscore , etc.
\section{Avezadamente}
\begin{itemize}
\item {Grp. gram.:adv.}
\end{itemize}
\begin{itemize}
\item {Proveniência:(De \textunderscore avesado\textunderscore )}
\end{itemize}
Por vezo.
\section{Avezado}
\begin{itemize}
\item {Grp. gram.:adj.}
\end{itemize}
Acostumado, habituado.
\section{Avezamente}
\begin{itemize}
\item {Grp. gram.:adv.}
\end{itemize}
O mesmo que \textunderscore avezadamente\textunderscore . Cf. Filinto, \textunderscore D. Man.\textunderscore , I, 61.
\section{Avezar}
\begin{itemize}
\item {Grp. gram.:v. t.}
\end{itemize}
\begin{itemize}
\item {Grp. gram.:V. p.}
\end{itemize}
\begin{itemize}
\item {Utilização:Prov.}
\end{itemize}
\begin{itemize}
\item {Utilização:trasm.}
\end{itemize}
Produzir vezo em; acostumar.
Adquirir vezo, acostumar-se: \textunderscore avezou-se a dormir muito\textunderscore .
Apparecer, estar presente.
\section{Avezar}
\begin{itemize}
\item {Grp. gram.:v. t.}
\end{itemize}
\begin{itemize}
\item {Utilização:Gír.}
\end{itemize}
Têr, possuir: \textunderscore não aveza cheta\textunderscore .
(Alter. de \textunderscore haver\textunderscore )
\section{Avezimão}
\begin{itemize}
\item {Grp. gram.:m.}
\end{itemize}
\begin{itemize}
\item {Utilização:Ant.}
\end{itemize}
Avejão? abantesma? Cf. G. Vicente, I, 254.
\section{Avezinha}
\begin{itemize}
\item {Grp. gram.:f.}
\end{itemize}
(Dem. de \textunderscore ave\textunderscore )
\section{Aviação}
\begin{itemize}
\item {Grp. gram.:f.}
\end{itemize}
\begin{itemize}
\item {Proveniência:(Fr. \textunderscore aviation\textunderscore , do lat. \textunderscore avis\textunderscore )}
\end{itemize}
Navegação aérea.
\section{Aviado}
\begin{itemize}
\item {Grp. gram.:m.}
\end{itemize}
\begin{itemize}
\item {Utilização:Bras}
\end{itemize}
\begin{itemize}
\item {Proveniência:(De \textunderscore aviar\textunderscore )}
\end{itemize}
Negociante por conta alheia.
Mascate, que, por conta dos negociantes da costa, vai fazer negócio no sertão.--É também t. afr. Cf. Capello, \textunderscore Benguella\textunderscore , I, 15, 17, 363.
\section{Aviador}
\begin{itemize}
\item {Grp. gram.:m.}
\end{itemize}
\begin{itemize}
\item {Proveniência:(Fr. \textunderscore aviateur\textunderscore )}
\end{itemize}
Indivíduo, que se occupa da aviação.
\section{Aviador}
\begin{itemize}
\item {Grp. gram.:m.}
\end{itemize}
Aquelle que avia: \textunderscore aviador de receituário\textunderscore .
\section{Aviamento}
\begin{itemize}
\item {Grp. gram.:m.}
\end{itemize}
\begin{itemize}
\item {Grp. gram.:Pl.}
\end{itemize}
Acção ou effeito de aviar.
Auxílio.
Preparativo.
Utensílios de lavoira.
Linhas, botões, forros e outras miudezas accessórias em trabalhos de costura.
\section{Aviar}
\begin{itemize}
\item {Grp. gram.:v. t.}
\end{itemize}
\begin{itemize}
\item {Utilização:Fam.}
\end{itemize}
\begin{itemize}
\item {Grp. gram.:V. p.}
\end{itemize}
\begin{itemize}
\item {Proveniência:(De \textunderscore via\textunderscore )}
\end{itemize}
Pôr a caminho; expedir: \textunderscore aviar encommendas\textunderscore .
Apromptar: \textunderscore aviar incumbências\textunderscore .
Preparar.
Despedir.
Apressar, abreviar: \textunderscore avia isso, que é tarde\textunderscore .
Pôr em difficuldades, dar cabo de.
Apressar-se; realizar sem demora qualquer serviço ou acto: \textunderscore anda, avia-te, que já deu meio-dia\textunderscore .
\section{Aviário}
\begin{itemize}
\item {Grp. gram.:m.}
\end{itemize}
\begin{itemize}
\item {Proveniência:(Lat. \textunderscore aviarium\textunderscore )}
\end{itemize}
Viveiro de aves.
\section{Aviário}
\begin{itemize}
\item {Grp. gram.:adj.}
\end{itemize}
\begin{itemize}
\item {Proveniência:(Lat. \textunderscore aviarius\textunderscore )}
\end{itemize}
Relativo a aves.
\section{Avicênia}
\begin{itemize}
\item {Grp. gram.:f.}
\end{itemize}
\begin{itemize}
\item {Proveniência:(De \textunderscore Avicena\textunderscore , n. p.)}
\end{itemize}
Arbusto, cuja casca se emprega em curtumes.
\section{Avicenista}
\begin{itemize}
\item {Grp. gram.:m.}
\end{itemize}
Médico, sectário de Avicena.
\section{Aviceptologia}
\begin{itemize}
\item {Grp. gram.:f.}
\end{itemize}
\begin{itemize}
\item {Proveniência:(Do lat. \textunderscore avis\textunderscore  + \textunderscore ceptus\textunderscore  + gr. \textunderscore logos\textunderscore )}
\end{itemize}
Arte de caçar aves com laços ou armadilhas.
\section{Aviceptológico}
\begin{itemize}
\item {Grp. gram.:adj.}
\end{itemize}
Relativo á \textunderscore aviceptologia\textunderscore .
\section{Avicida}
\begin{itemize}
\item {Grp. gram.:m.  e  adj.}
\end{itemize}
\begin{itemize}
\item {Proveniência:(Do lat. \textunderscore avis\textunderscore  + \textunderscore caedere\textunderscore )}
\end{itemize}
O que mata aves.
\section{Avicínio}
\begin{itemize}
\item {Grp. gram.:m.}
\end{itemize}
\begin{itemize}
\item {Utilização:Mús.}
\end{itemize}
\begin{itemize}
\item {Proveniência:(Do lat. \textunderscore avis\textunderscore  + \textunderscore canere\textunderscore )}
\end{itemize}
Registo dos órgãos antigos, que imitava o chilrear dos pássaros.
\section{Avícola}
\begin{itemize}
\item {Grp. gram.:m.}
\end{itemize}
O mesmo que \textunderscore avicultor\textunderscore .
\section{Avícula}
\begin{itemize}
\item {Grp. gram.:f.}
\end{itemize}
\begin{itemize}
\item {Proveniência:(Lat. \textunderscore avicula\textunderscore )}
\end{itemize}
Pequena ave.
Gênero de mollúsculos, cuja concha tem semelhança com a cauda da andorinha.
\section{Aviculado}
\begin{itemize}
\item {Grp. gram.:adj.}
\end{itemize}
Semelhante ao mollúsculo avícula.
\section{Aviculário}
\begin{itemize}
\item {Grp. gram.:adj.}
\end{itemize}
\begin{itemize}
\item {Grp. gram.:M.}
\end{itemize}
\begin{itemize}
\item {Proveniência:(Lat. \textunderscore avicularius\textunderscore )}
\end{itemize}
Relativo a aves.
Que devora aves ou vive em o ninho dellas.
Aquelle que trata de aves.
\section{Aviculinha}
\begin{itemize}
\item {Grp. gram.:f.}
\end{itemize}
\begin{itemize}
\item {Proveniência:(De \textunderscore avicula\textunderscore )}
\end{itemize}
Mollúsco acéphalo.
\section{Avicultor}
\begin{itemize}
\item {Grp. gram.:m.}
\end{itemize}
\begin{itemize}
\item {Proveniência:(Do lat. \textunderscore avis\textunderscore  + \textunderscore cultor\textunderscore )}
\end{itemize}
Aquelle que cria aves.
\section{Avicultura}
\begin{itemize}
\item {Grp. gram.:f.}
\end{itemize}
\begin{itemize}
\item {Proveniência:(Do lat. \textunderscore avis\textunderscore  + \textunderscore cultura\textunderscore )}
\end{itemize}
Criação de aves.
\section{Avidamente}
\begin{itemize}
\item {Grp. gram.:adv.}
\end{itemize}
\begin{itemize}
\item {Proveniência:(De \textunderscore ávido\textunderscore )}
\end{itemize}
Com avidez.
\section{Avidar}
\begin{itemize}
\item {Grp. gram.:v. t.}
\end{itemize}
\begin{itemize}
\item {Utilização:Prov.}
\end{itemize}
\begin{itemize}
\item {Utilização:minh.}
\end{itemize}
\begin{itemize}
\item {Grp. gram.:V. i.}
\end{itemize}
\begin{itemize}
\item {Proveniência:(De \textunderscore vide\textunderscore )}
\end{itemize}
Plantas videiras em.
Plantar videiras.
\section{Avidez}
\begin{itemize}
\item {Grp. gram.:f.}
\end{itemize}
Qualidade do que é \textunderscore ávido\textunderscore .
\section{Ávido}
\begin{itemize}
\item {Grp. gram.:adj.}
\end{itemize}
\begin{itemize}
\item {Proveniência:(Lat. \textunderscore avidus\textunderscore )}
\end{itemize}
Que deseja ardentemente.
Sôfrego.
Avaro.
\section{Avieirado}
\begin{itemize}
\item {Grp. gram.:adj.}
\end{itemize}
Que tem vieiras.
\section{Avigorar}
\begin{itemize}
\item {Grp. gram.:v. t.}
\end{itemize}
Dar vigor a.
Consolidar.
\section{Ávila}
\begin{itemize}
\item {Grp. gram.:f.}
\end{itemize}
O mesmo que \textunderscore ávela\textunderscore . Cf. o \textunderscore Dicc. de nomes, versos e vozes...\textunderscore , ms. da Tôrre do Tombo.
\section{Avilar}
\begin{itemize}
\item {Grp. gram.:v. t.}
\end{itemize}
\begin{itemize}
\item {Utilização:Ant.}
\end{itemize}
\begin{itemize}
\item {Proveniência:(De \textunderscore vil\textunderscore )}
\end{itemize}
O mesmo que \textunderscore aviltar\textunderscore .
\section{Avilanar-se}
\begin{itemize}
\item {Grp. gram.:v. p.}
\end{itemize}
Tornar-se vilão.
Degenerar dos seus méritos ou nobreza.
\section{Avillanar-se}
\begin{itemize}
\item {Grp. gram.:v. p.}
\end{itemize}
Tornar-se villão.
Degenerar dos seus méritos ou nobreza.
\section{Aviltação}
\begin{itemize}
\item {Grp. gram.:f.}
\end{itemize}
(V.aviltamento)
\section{Aviltadamente}
\begin{itemize}
\item {Grp. gram.:adv.}
\end{itemize}
Com aviltamento.
\section{Aviltador}
\begin{itemize}
\item {Grp. gram.:adj.}
\end{itemize}
O mesmo que \textunderscore aviltante\textunderscore .
\section{Aviltamento}
\begin{itemize}
\item {Grp. gram.:m.}
\end{itemize}
Acção ou effeito de \textunderscore aviltar\textunderscore .
\section{Aviltante}
\begin{itemize}
\item {Grp. gram.:adj.}
\end{itemize}
Que avilta.
\section{Aviltar}
\begin{itemize}
\item {Grp. gram.:v. t.}
\end{itemize}
\begin{itemize}
\item {Proveniência:(Do lat. \textunderscore vilitare\textunderscore )}
\end{itemize}
Tornar vil; rebaixar.
Desprezar.
Deshonrar.
Humilhar.
\section{Aviltoso}
\begin{itemize}
\item {Grp. gram.:adj.}
\end{itemize}
Que avilta. Cf. Filinto, XVIII, 247.
\section{Avinagradamente}
\begin{itemize}
\item {Grp. gram.:adv.}
\end{itemize}
Com azedume; de modo \textunderscore avinagrado\textunderscore .
\section{Avinagrado}
\begin{itemize}
\item {Grp. gram.:adj.}
\end{itemize}
Que tem vinagre.
Temperado com vinagre.
Azêdo.
\section{Avinagrar}
\begin{itemize}
\item {Grp. gram.:v. t.}
\end{itemize}
Temperar com vinagre.
Dar sabor de vinagre a.
Azedar.
Irritar.
Constranger.
\section{Avincar}
\begin{itemize}
\item {Grp. gram.:v. t.}
\end{itemize}
O mesmo que \textunderscore vincar\textunderscore . Cf. Crespo, \textunderscore Nocturnos\textunderscore , 132.
\section{Avindador}
\begin{itemize}
\item {Grp. gram.:adj.}
\end{itemize}
\begin{itemize}
\item {Utilização:Ant.}
\end{itemize}
\begin{itemize}
\item {Proveniência:(De \textunderscore avindo\textunderscore )}
\end{itemize}
Avindor.
\section{Avindeiro}
\begin{itemize}
\item {Grp. gram.:m.}
\end{itemize}
O mesmo que \textunderscore avindor\textunderscore .
\section{Avindo}
\begin{itemize}
\item {Grp. gram.:adj.}
\end{itemize}
\begin{itemize}
\item {Grp. gram.:M.}
\end{itemize}
\begin{itemize}
\item {Utilização:Prov.}
\end{itemize}
\begin{itemize}
\item {Utilização:trasm.}
\end{itemize}
\begin{itemize}
\item {Proveniência:(De \textunderscore avir\textunderscore )}
\end{itemize}
Que se aveio.
Congraçado, harmonizado.
Cliente, freguês.
\section{Avindor}
\begin{itemize}
\item {Grp. gram.:m.  e  adj.}
\end{itemize}
\begin{itemize}
\item {Proveniência:(De \textunderscore avindo\textunderscore )}
\end{itemize}
Mediador.
Aquelle que trata de harmonizar pessôas litigantes.
\section{Avinhado}
\begin{itemize}
\item {Grp. gram.:adj.}
\end{itemize}
\begin{itemize}
\item {Grp. gram.:M.}
\end{itemize}
Impregnado do vinho.
Que tem sabor ou cheiro a vinho.
Ébrio.
Ave canora do Brasil; curió.
\section{Avincular}
\begin{itemize}
\item {Grp. gram.:v. t.}
\end{itemize}
O mesmo que \textunderscore vincular\textunderscore ^1. Cf. Filinto, \textunderscore D. Man.\textunderscore , I, 143.
\section{Avinhar}
\begin{itemize}
\item {Grp. gram.:v. t.}
\end{itemize}
\begin{itemize}
\item {Proveniência:(De \textunderscore vinho\textunderscore )}
\end{itemize}
Misturar com vinho.
Impregnar de vinho.
Embeber de vinho.
Dar sabor e cheiro de vinho a.
Emborrachar.
\section{Avio}
\begin{itemize}
\item {Grp. gram.:m.}
\end{itemize}
\begin{itemize}
\item {Utilização:Prov.}
\end{itemize}
\begin{itemize}
\item {Utilização:alent.}
\end{itemize}
\begin{itemize}
\item {Proveniência:(De \textunderscore aviar\textunderscore )}
\end{itemize}
O mesmo que \textunderscore aviamento\textunderscore .
Provisão de mantimentos.
\section{Aviolado}
\begin{itemize}
\item {Grp. gram.:adj.}
\end{itemize}
\begin{itemize}
\item {Proveniência:(De \textunderscore viola\textunderscore )}
\end{itemize}
Feito com flôres de violeta.
Que tem forma de viola.
\section{Avir}
\begin{itemize}
\item {Grp. gram.:v. t.}
\end{itemize}
\begin{itemize}
\item {Grp. gram.:V. i.}
\end{itemize}
\begin{itemize}
\item {Utilização:Ant.}
\end{itemize}
\begin{itemize}
\item {Grp. gram.:V. p.}
\end{itemize}
\begin{itemize}
\item {Proveniência:(Lat. \textunderscore advenire\textunderscore )}
\end{itemize}
Ajustar; acordar em; combinar.
Succeder; acontecer.
Advir.
Combinar-se; entender-se.
Acommodar-se, pôr-se de acôrdo.
\section{Avisadamente}
\begin{itemize}
\item {Grp. gram.:adv.}
\end{itemize}
\begin{itemize}
\item {Proveniência:(De \textunderscore avisar\textunderscore )}
\end{itemize}
Com acêrto.
\section{Avisado}
\begin{itemize}
\item {Grp. gram.:adj.}
\end{itemize}
\begin{itemize}
\item {Proveniência:(De \textunderscore aviso\textunderscore ^2)}
\end{itemize}
Prudente, discreto, ajuizado.
Que tomou uma resolução.
\section{Avisador}
\begin{itemize}
\item {Grp. gram.:m.  e  adj.}
\end{itemize}
O que avisa.
\section{Avisamento}
\begin{itemize}
\item {Grp. gram.:m.}
\end{itemize}
\begin{itemize}
\item {Utilização:Ant.}
\end{itemize}
Parecer; conselho.
Precaução.
Provisão.
O mesmo que \textunderscore aviso\textunderscore ^1:«\textunderscore dai avisamento a todos que çarrem muito bem as portas\textunderscore ». Azurara, \textunderscore Crón. de D. João I\textunderscore , C. LXXII.
\section{Avisança}
\begin{itemize}
\item {Grp. gram.:f.}
\end{itemize}
\begin{itemize}
\item {Utilização:Ant.}
\end{itemize}
O mesmo que \textunderscore aviso\textunderscore ^1.
\section{Avisar}
\begin{itemize}
\item {Grp. gram.:v. t.}
\end{itemize}
\begin{itemize}
\item {Proveniência:(Do b. lat. \textunderscore advisare\textunderscore )}
\end{itemize}
Noticiar, annunciar; fazer saber.
\section{Aviso}
\begin{itemize}
\item {Grp. gram.:m.}
\end{itemize}
Acto de \textunderscore avisar\textunderscore .
Embarcação, para troca de communicações ou para descobrimento dos inimigos.
\section{Aviso}
\begin{itemize}
\item {Grp. gram.:m.}
\end{itemize}
\begin{itemize}
\item {Proveniência:(Do lat. \textunderscore ad\textunderscore  + \textunderscore visum\textunderscore )}
\end{itemize}
Opinião; conceito.
Conselho.
\section{Avissuga}
\begin{itemize}
\item {Grp. gram.:f.}
\end{itemize}
\begin{itemize}
\item {Proveniência:(De \textunderscore ave\textunderscore  + \textunderscore sugar\textunderscore )}
\end{itemize}
Insecto pupíparo, que vive nas aves.
\section{Avistar}
\begin{itemize}
\item {Grp. gram.:v. t.}
\end{itemize}
\begin{itemize}
\item {Proveniência:(De \textunderscore vista\textunderscore )}
\end{itemize}
Vêr ao longe; começar a vêr ao longe; entrever.
Defrontar.
\section{Avistável}
\begin{itemize}
\item {Grp. gram.:adj.}
\end{itemize}
Que se póde avistar.
\section{Avisuga}
\begin{itemize}
\item {fónica:su}
\end{itemize}
\begin{itemize}
\item {Grp. gram.:f.}
\end{itemize}
\begin{itemize}
\item {Proveniência:(De \textunderscore ave\textunderscore  + \textunderscore sugar\textunderscore )}
\end{itemize}
Insecto pupíparo, que vive nas aves.
\section{Avititado}
\begin{itemize}
\item {Grp. gram.:m.}
\end{itemize}
\begin{itemize}
\item {Utilização:Ant.}
\end{itemize}
\begin{itemize}
\item {Proveniência:(Do lat. \textunderscore vita\textunderscore )}
\end{itemize}
Emprazamento ou arrendamento, por uma vida ou certo número de vidas.
\section{Avito}
\begin{itemize}
\item {Grp. gram.:adj.}
\end{itemize}
\begin{itemize}
\item {Proveniência:(Lat. \textunderscore avitus\textunderscore )}
\end{itemize}
Que procede do avós ou de antepassados.
\section{Avitualhar}
\begin{itemize}
\item {Grp. gram.:v. t.}
\end{itemize}
Prover de vitualhas.
Fornecer de mantimentos.
\section{Aviú}
\begin{itemize}
\item {Grp. gram.:m.}
\end{itemize}
\begin{itemize}
\item {Utilização:bras. do N}
\end{itemize}
Espécie de camarão do Tocantins.
\section{Aviuzar}
\begin{itemize}
\item {fónica:vi-u}
\end{itemize}
\begin{itemize}
\item {Grp. gram.:v. t.}
\end{itemize}
\begin{itemize}
\item {Utilização:Prov.}
\end{itemize}
\begin{itemize}
\item {Utilização:trasm.}
\end{itemize}
\begin{itemize}
\item {Utilização:beir.}
\end{itemize}
O mesmo que \textunderscore enviesar\textunderscore .
\section{Avivador}
\begin{itemize}
\item {Grp. gram.:m.}
\end{itemize}
\begin{itemize}
\item {Grp. gram.:Adj.}
\end{itemize}
Aquelle que aviva.
Que aviva.
\section{Avivar}
\begin{itemize}
\item {Grp. gram.:v. t.}
\end{itemize}
Dar vivacidado a, tornar mais vivo: \textunderscore avivar uma luz\textunderscore .
Despertar.
Aggravar: \textunderscore avivar uma chaga\textunderscore .
Apressar: \textunderscore avivar o passo\textunderscore .
Realçar.
Guarnecer de vivos.
\section{Aviventador}
\begin{itemize}
\item {Grp. gram.:m.}
\end{itemize}
\begin{itemize}
\item {Grp. gram.:Adj.}
\end{itemize}
Aquelle que aviventa.
Que aviventa.
\section{Aviventar}
\begin{itemize}
\item {Grp. gram.:v. t.}
\end{itemize}
\begin{itemize}
\item {Proveniência:(De \textunderscore vivo\textunderscore )}
\end{itemize}
Fomentar a vida em.
Reanimar; fortalecer.
\section{Avizinhamento}
\begin{itemize}
\item {Grp. gram.:m.}
\end{itemize}
Acção de \textunderscore avizinhar\textunderscore .
\section{Avizinhar}
\begin{itemize}
\item {Grp. gram.:v. t.}
\end{itemize}
\begin{itemize}
\item {Grp. gram.:V. i.}
\end{itemize}
Tornar vizinho; aproximar.
Confinar com.
Aproximar-se.
Confinar. Cf. Garrett, \textunderscore Romanc.\textunderscore , I, 48.
\section{Avo}
\begin{itemize}
\item {Grp. gram.:m.}
\end{itemize}
Palavra, que, junta aos números cardinaes, de dez para cima, indica as partes em que se divide um todo.
Insignificância, bagatela.
Moéda, em Macau e Timor.
(Cast. \textunderscore avo\textunderscore )
\section{Avô}
\begin{itemize}
\item {Grp. gram.:m.}
\end{itemize}
\begin{itemize}
\item {Proveniência:(Lat. hyp. \textunderscore abolus\textunderscore . Cp. cast. \textunderscore abuelo\textunderscore )}
\end{itemize}
Pai do pai ou da mãe.
\textunderscore Pl.\textunderscore  (\textunderscore avós\textunderscore )
Pais dos pais.
Antepassados.
\section{Avó}
\begin{itemize}
\item {Grp. gram.:f.}
\end{itemize}
\begin{itemize}
\item {Proveniência:(Do lat. hyp. \textunderscore abola\textunderscore . Cf. cast. \textunderscore abuela\textunderscore )}
\end{itemize}
Mãe do pae ou da mãe.
\section{Avóa}
\begin{itemize}
\item {Grp. gram.:f.}
\end{itemize}
\begin{itemize}
\item {Utilização:Ant.}
\end{itemize}
O mesmo que \textunderscore avó\textunderscore .
\section{Avoaçar}
\begin{itemize}
\item {Grp. gram.:v. i.}
\end{itemize}
(V.esvoaçar)
\section{Avoadinha}
\begin{itemize}
\item {Grp. gram.:f.}
\end{itemize}
\begin{itemize}
\item {Utilização:T. de Odemira}
\end{itemize}
Planta medicinal, (\textunderscore conyza ambigua\textunderscore . D. C.).
\section{Avoado}
\begin{itemize}
\item {Grp. gram.:adj.}
\end{itemize}
\begin{itemize}
\item {Utilização:Bras}
\end{itemize}
\begin{itemize}
\item {Proveniência:(De \textunderscore avoar\textunderscore )}
\end{itemize}
Tonto.
Atoleimado.
Imprudente.
\section{Avoamento}
\begin{itemize}
\item {Grp. gram.:m.}
\end{itemize}
\begin{itemize}
\item {Utilização:Ant.}
\end{itemize}
\begin{itemize}
\item {Proveniência:(De \textunderscore avoar\textunderscore )}
\end{itemize}
Vôo.
Elevação do pensamento.
\section{Avoante}
\begin{itemize}
\item {Grp. gram.:f.}
\end{itemize}
\begin{itemize}
\item {Utilização:Bras}
\end{itemize}
\begin{itemize}
\item {Proveniência:(De \textunderscore avoar\textunderscore )}
\end{itemize}
Espécie de pomba.
\section{Avoar}
\begin{itemize}
\item {Grp. gram.:v. i.}
\end{itemize}
O mesmo que \textunderscore voar\textunderscore . Cf. Camillo, \textunderscore Regicida\textunderscore , XIII.
\section{Avocação}
\begin{itemize}
\item {Grp. gram.:f.}
\end{itemize}
Acção de \textunderscore avocar\textunderscore .
\section{Avocar}
\begin{itemize}
\item {Grp. gram.:v. t.}
\end{itemize}
\begin{itemize}
\item {Utilização:Jur.}
\end{itemize}
\begin{itemize}
\item {Proveniência:(Lat. \textunderscore avocare\textunderscore )}
\end{itemize}
Attrahir; chamar a si.
Deslocar de um para outro tribunal (uma causa).
\section{Avocatório}
\begin{itemize}
\item {Grp. gram.:adj.}
\end{itemize}
Que serve para \textunderscore avocar\textunderscore .
\section{Avocatura}
\begin{itemize}
\item {Grp. gram.:f.}
\end{itemize}
(V.avocação)
\section{Avocável}
\begin{itemize}
\item {Grp. gram.:adj.}
\end{itemize}
Que se póde avocar.
\section{Avoceta}
\begin{itemize}
\item {Grp. gram.:f.}
\end{itemize}
\begin{itemize}
\item {Proveniência:(It. \textunderscore avocetta\textunderscore )}
\end{itemize}
Ave palmípede.
\section{Avoejar}
\begin{itemize}
\item {Grp. gram.:v. t.}
\end{itemize}
O mesmo que \textunderscore voejar\textunderscore . Cf. Camillo, \textunderscore Estrêl. Fun.\textunderscore , 65 e 133.
\section{Avoejo}
\begin{itemize}
\item {fónica:vo-ei}
\end{itemize}
\begin{itemize}
\item {Grp. gram.:m.}
\end{itemize}
\begin{itemize}
\item {Utilização:Neol.}
\end{itemize}
\begin{itemize}
\item {Proveniência:(De \textunderscore avoejar\textunderscore )}
\end{itemize}
Desenho caprichoso ou fantástico em cerâmica e em peças de charão.
\section{Avoenga}
\begin{itemize}
\item {Grp. gram.:f.}
\end{itemize}
\begin{itemize}
\item {Utilização:Ant.}
\end{itemize}
\begin{itemize}
\item {Proveniência:(De \textunderscore avoengo\textunderscore )}
\end{itemize}
Herança.
Direito de succeder em bens de ascendentes.
\section{Avoengado}
\begin{itemize}
\item {Grp. gram.:adj.}
\end{itemize}
Relativo a avoengos.
Próprio de tempos afastados. Cf. Filinto, \textunderscore D. Man.\textunderscore , I, 324.
\section{Avoengo}
\begin{itemize}
\item {Grp. gram.:adj.}
\end{itemize}
\begin{itemize}
\item {Grp. gram.:M. pl.}
\end{itemize}
\begin{itemize}
\item {Proveniência:(De \textunderscore avô\textunderscore )}
\end{itemize}
Que procede dos avós: relativo aos avós.
Avito.
Antepassados.
\section{Avogacia}
\begin{itemize}
\item {Grp. gram.:f.}
\end{itemize}
\begin{itemize}
\item {Utilização:Ant.}
\end{itemize}
O mesmo que \textunderscore advocacia\textunderscore . Cf. \textunderscore Luz e Calor\textunderscore , 524.
\section{Avogado}
\begin{itemize}
\item {Grp. gram.:m.}
\end{itemize}
\begin{itemize}
\item {Utilização:Ant.}
\end{itemize}
O mesmo que \textunderscore advogado\textunderscore . Cf. \textunderscore Eufrosina\textunderscore , 36.
\section{Avolumar}
\begin{itemize}
\item {Grp. gram.:v. t.}
\end{itemize}
Aumentar em volume.
Tornar maior.
Aumentar.
Engrandecer.
\section{Avonda!}
\begin{itemize}
\item {Grp. gram.:interj.}
\end{itemize}
(V.bond)
\section{Avondamento}
\begin{itemize}
\item {Grp. gram.:m.}
\end{itemize}
\begin{itemize}
\item {Utilização:Ant.}
\end{itemize}
O mesmo que \textunderscore abundância\textunderscore .
\section{Avonde}
\begin{itemize}
\item {Grp. gram.:adv.}
\end{itemize}
\begin{itemize}
\item {Proveniência:(Do lat. \textunderscore abunde\textunderscore )}
\end{itemize}
Abundantemente.
\section{Avondo}
\begin{itemize}
\item {Grp. gram.:adv.}
\end{itemize}
\begin{itemize}
\item {Utilização:Ant.}
\end{itemize}
O mesmo que \textunderscore avonde\textunderscore :«\textunderscore sabe sciênda avondo\textunderscore ». G. Vicente, \textunderscore Auto da Feira\textunderscore .
\section{Avondoso}
\begin{itemize}
\item {Grp. gram.:adj.}
\end{itemize}
\begin{itemize}
\item {Utilização:Ant.}
\end{itemize}
O mesmo que \textunderscore abundante\textunderscore :«\textunderscore prazer avondoso\textunderscore ». G. Vicente, \textunderscore Carta a D. João III\textunderscore .
\section{Avozear}
\begin{itemize}
\item {Grp. gram.:v. t.}
\end{itemize}
\begin{itemize}
\item {Utilização:Des.}
\end{itemize}
Acclamar em altas vozes.
(Cp. \textunderscore vozear\textunderscore )
\section{Ávrego}
\begin{itemize}
\item {Grp. gram.:m.}
\end{itemize}
\begin{itemize}
\item {Utilização:Ant.}
\end{itemize}
O mesmo que \textunderscore áfrico\textunderscore  (vento).
\section{Avulsão}
\begin{itemize}
\item {Grp. gram.:f.}
\end{itemize}
\begin{itemize}
\item {Proveniência:(Lat. \textunderscore avulsio\textunderscore )}
\end{itemize}
Acção de extrahir com violência.
\section{Avulso}
\begin{itemize}
\item {Grp. gram.:adj.}
\end{itemize}
\begin{itemize}
\item {Proveniência:(Lat. \textunderscore avulsus\textunderscore )}
\end{itemize}
Arrancado com violência.
Separado.
Desligado do corpo ou da collecção, a que pertence: \textunderscore fôlha avulsa\textunderscore .
Insulado.
Desirmanado.
\section{Avultações}
\begin{itemize}
\item {Grp. gram.:f. pl.}
\end{itemize}
\begin{itemize}
\item {Utilização:Pop.}
\end{itemize}
Parecenças. Cf. Camillo, \textunderscore Caveira\textunderscore , 113 e 164.
\section{Avultante}
\begin{itemize}
\item {Grp. gram.:adj.}
\end{itemize}
Que avulta. Cp. Castilho, \textunderscore Fastos\textunderscore , III, 181.
\section{Avultar}
\begin{itemize}
\item {Grp. gram.:v. t.}
\end{itemize}
\begin{itemize}
\item {Grp. gram.:V. i.}
\end{itemize}
Dar vulto a; pôr em relevo.
Avolumar.
Exaggerar.
Crescer.
Sobresair.
\section{Avultoso}
\begin{itemize}
\item {Grp. gram.:adj.}
\end{itemize}
\begin{itemize}
\item {Proveniência:(De \textunderscore avultar\textunderscore )}
\end{itemize}
Que avulta, que sobresai.
\section{Avuncular}
\begin{itemize}
\item {Grp. gram.:adj.}
\end{itemize}
\begin{itemize}
\item {Proveniência:(Do lat. \textunderscore avunculus\textunderscore )}
\end{itemize}
Relativo ao tio ou á tia.
\section{Avunculicida}
\begin{itemize}
\item {Grp. gram.:m.}
\end{itemize}
Aquelle que perpetrou avunculicídio.
\section{Avunculicídio}
\begin{itemize}
\item {Grp. gram.:m.}
\end{itemize}
\begin{itemize}
\item {Proveniência:(Do lat. \textunderscore avunculus\textunderscore  + \textunderscore caedere\textunderscore )}
\end{itemize}
Acto de quem assassina seu tio materno. Cf. a Revista \textunderscore Movimento Médico\textunderscore ,
VI, 148.
\section{Axá}
\begin{itemize}
\item {Grp. gram.:m.}
\end{itemize}
Oração, que os Moiros fazem a Deus, antes de se deitarem na cama. Cf. Barros, \textunderscore Déc.\textunderscore , II, l. 10, c. 6.
\section{Axadrezado}
\begin{itemize}
\item {Grp. gram.:adj.}
\end{itemize}
Semelhante ao tabuleiro do xadrez.
\section{Axantho}
\begin{itemize}
\item {Grp. gram.:m.}
\end{itemize}
\begin{itemize}
\item {Proveniência:(Do gr. \textunderscore axon\textunderscore  + \textunderscore anthos\textunderscore )}
\end{itemize}
Planta rubiácea da Malásia.
\section{Axanto}
\begin{itemize}
\item {Grp. gram.:m.}
\end{itemize}
\begin{itemize}
\item {Proveniência:(Do gr. \textunderscore axon\textunderscore  + \textunderscore anthos\textunderscore )}
\end{itemize}
Planta rubiácea da Malásia.
\section{Axe}
\begin{itemize}
\item {Grp. gram.:m.}
\end{itemize}
\begin{itemize}
\item {Utilização:Infant.}
\end{itemize}
Ferimento.
Dôr.
\section{Axe}
\begin{itemize}
\item {Grp. gram.:f.}
\end{itemize}
\begin{itemize}
\item {Proveniência:(Lat. \textunderscore axis\textunderscore )}
\end{itemize}
Linha imaginária, que passa pelo centro de um corpo circular; o mesmo que \textunderscore eixo\textunderscore .
\section{Axi!}
\begin{itemize}
\item {Grp. gram.:interj.}
\end{itemize}
\begin{itemize}
\item {Utilização:Bras}
\end{itemize}
(Designa tédio, aversão, repugnância)
\section{Axi}
\begin{itemize}
\item {Grp. gram.:m.}
\end{itemize}
Pimenta da Guiné.
\section{Axial}
\begin{itemize}
\item {Grp. gram.:adj.}
\end{itemize}
\begin{itemize}
\item {Proveniência:(Do lat. \textunderscore axis\textunderscore )}
\end{itemize}
Relativo a eixo.
Que serve de eixo.
Que tem fórma de eixo.
\section{Axialmente}
\begin{itemize}
\item {Grp. gram.:adv.}
\end{itemize}
De modo \textunderscore axial\textunderscore .
Á maneira de eixo.
\section{Axículo}
\begin{itemize}
\item {fónica:csi}
\end{itemize}
\begin{itemize}
\item {Grp. gram.:m.}
\end{itemize}
\begin{itemize}
\item {Proveniência:(Lat. \textunderscore axiculus\textunderscore )}
\end{itemize}
Pequeno eixo.
\section{Axífero}
\begin{itemize}
\item {fónica:csi}
\end{itemize}
\begin{itemize}
\item {Grp. gram.:adj.}
\end{itemize}
\begin{itemize}
\item {Proveniência:(Do lat. \textunderscore axis\textunderscore  + \textunderscore ferre\textunderscore )}
\end{itemize}
Que tem eixo.
\section{Axiforme}
\begin{itemize}
\item {fónica:csi}
\end{itemize}
\begin{itemize}
\item {Grp. gram.:adj.}
\end{itemize}
\begin{itemize}
\item {Proveniência:(Do lat. \textunderscore axis\textunderscore  + \textunderscore forma\textunderscore )}
\end{itemize}
Que tem fórma de eixo.
\section{Axífugo}
\begin{itemize}
\item {fónica:csi}
\end{itemize}
\begin{itemize}
\item {Grp. gram.:adj.}
\end{itemize}
\begin{itemize}
\item {Proveniência:(Do lat. \textunderscore axis\textunderscore  + \textunderscore fugere\textunderscore )}
\end{itemize}
O mesmo que \textunderscore centrífugo\textunderscore .
\section{Axígrafo}
\begin{itemize}
\item {fónica:csi}
\end{itemize}
\begin{itemize}
\item {Grp. gram.:m.}
\end{itemize}
Variedade de cal carbonatada.
\section{Axígrapho}
\begin{itemize}
\item {fónica:csi}
\end{itemize}
\begin{itemize}
\item {Grp. gram.:m.}
\end{itemize}
Variedade de cal carbonatada.
\section{Áxil}
\begin{itemize}
\item {fónica:csil}
\end{itemize}
\begin{itemize}
\item {Grp. gram.:adj.}
\end{itemize}
\begin{itemize}
\item {Utilização:Bot.}
\end{itemize}
\begin{itemize}
\item {Proveniência:(Do lat. \textunderscore axis\textunderscore )}
\end{itemize}
Relativo ao eixo de uma planta.
\section{Axila}
\begin{itemize}
\item {fónica:csi}
\end{itemize}
\begin{itemize}
\item {Grp. gram.:f.}
\end{itemize}
\begin{itemize}
\item {Utilização:Bot.}
\end{itemize}
\begin{itemize}
\item {Proveniência:(Lat. \textunderscore axilla\textunderscore )}
\end{itemize}
Cavidade inferior á juncção do braço com o ombro; sovaco.
Ângulo, formado por dois ramos, ou por uma fôlha com o ramo, ou por um ramo com o caule.
\section{Axilante}
\begin{itemize}
\item {fónica:csi}
\end{itemize}
\begin{itemize}
\item {Grp. gram.:adj.}
\end{itemize}
\begin{itemize}
\item {Utilização:Bot.}
\end{itemize}
\begin{itemize}
\item {Proveniência:(De \textunderscore axilla\textunderscore )}
\end{itemize}
Diz-se da fôlha, cuja axila tem um botão ou ramo, sendo preciso distingui-la das que se desenvolvem nesse botão ou ramo.
\section{Axilla}
\begin{itemize}
\item {fónica:csi}
\end{itemize}
\begin{itemize}
\item {Grp. gram.:f.}
\end{itemize}
\begin{itemize}
\item {Utilização:Bot.}
\end{itemize}
\begin{itemize}
\item {Proveniência:(Lat. \textunderscore axilla\textunderscore )}
\end{itemize}
Cavidade inferior á juncção do braço com o ombro; sovaco.
Ângulo, formado por dois ramos, ou por uma fôlha com o ramo, ou por um ramo com o caule.
\section{Axillante}
\begin{itemize}
\item {Grp. gram.:adj.}
\end{itemize}
\begin{itemize}
\item {Utilização:Bot.}
\end{itemize}
\begin{itemize}
\item {Proveniência:(De \textunderscore axilla\textunderscore )}
\end{itemize}
Diz-se da fôlha, cuja axilla tem um botão ou ramo, sendo preciso distingui-la das que se desenvolvem nesse botão ou ramo.
\section{Axilar}
\begin{itemize}
\item {fónica:csi}
\end{itemize}
\begin{itemize}
\item {Grp. gram.:adj.}
\end{itemize}
Relativo á \textunderscore axila\textunderscore .
Que está ou cresce na axila.
\section{Axilibarbudo}
\begin{itemize}
\item {fónica:csi}
\end{itemize}
\begin{itemize}
\item {Grp. gram.:adj.}
\end{itemize}
\begin{itemize}
\item {Proveniência:(De \textunderscore axilla\textunderscore  + \textunderscore barbudo\textunderscore )}
\end{itemize}
Que tem pêlos na axilla, (tratando-se de fôlhas vegetaes).
\section{Axilifloro}
\begin{itemize}
\item {fónica:csi}
\end{itemize}
\begin{itemize}
\item {Grp. gram.:adj.}
\end{itemize}
\begin{itemize}
\item {Proveniência:(De \textunderscore axilla\textunderscore  + \textunderscore flôr\textunderscore )}
\end{itemize}
Diz-se das plantas, que têm flôres axilares.
\section{Axillar}
\begin{itemize}
\item {fónica:csi}
\end{itemize}
\begin{itemize}
\item {Grp. gram.:adj.}
\end{itemize}
Relativo á \textunderscore axilla\textunderscore .
Que está ou cresce na axilla.
\section{Axillibarbudo}
\begin{itemize}
\item {fónica:csi}
\end{itemize}
\begin{itemize}
\item {Grp. gram.:adj.}
\end{itemize}
\begin{itemize}
\item {Proveniência:(De \textunderscore axilla\textunderscore  + \textunderscore barbudo\textunderscore )}
\end{itemize}
Que tem pêlos na axilla, (tratando-se de fôlhas vegetaes).
\section{Axillifloro}
\begin{itemize}
\item {fónica:csi}
\end{itemize}
\begin{itemize}
\item {Grp. gram.:adj.}
\end{itemize}
\begin{itemize}
\item {Proveniência:(De \textunderscore axilla\textunderscore  + \textunderscore flôr\textunderscore )}
\end{itemize}
Diz-se das plantas, que têm flôres axillares.
\section{Áxilo}
\begin{itemize}
\item {fónica:csi}
\end{itemize}
\begin{itemize}
\item {Grp. gram.:adj.}
\end{itemize}
\begin{itemize}
\item {Proveniência:(Do gr. \textunderscore a\textunderscore  priv. + \textunderscore xulon\textunderscore )}
\end{itemize}
Que não produz madeira, como certos vegetaes cellulares.
\section{Aximez}
\begin{itemize}
\item {Grp. gram.:m.}
\end{itemize}
\begin{itemize}
\item {Proveniência:(Do ár. \textunderscore ax-mese\textunderscore )}
\end{itemize}
(Fórma preferível a \textunderscore ajimez\textunderscore . V. \textunderscore ajimez\textunderscore )
\section{Aximez}
\begin{itemize}
\item {Grp. gram.:m.}
\end{itemize}
Janela arqueada superiormente, bipartida por um columnelo central e vertical.
(Cast. \textunderscore ajimez\textunderscore )
\section{Axinela}
\begin{itemize}
\item {fónica:csi}
\end{itemize}
\begin{itemize}
\item {Grp. gram.:f.}
\end{itemize}
Esponja do mar Adriático.
\section{Axinella}
\begin{itemize}
\item {fónica:csi}
\end{itemize}
\begin{itemize}
\item {Grp. gram.:f.}
\end{itemize}
Esponja do mar Adriático.
\section{Axiniano}
\begin{itemize}
\item {fónica:csi}
\end{itemize}
\begin{itemize}
\item {Grp. gram.:adj.}
\end{itemize}
\begin{itemize}
\item {Proveniência:(Do gr. \textunderscore axine\textunderscore )}
\end{itemize}
Diz-se do jade verde, de que os selvagens da América do Sul e da Oceânia fazem machados.
\section{Axinita}
\begin{itemize}
\item {fónica:csi}
\end{itemize}
\begin{itemize}
\item {Grp. gram.:f.}
\end{itemize}
\begin{itemize}
\item {Proveniência:(Do gr. \textunderscore axine\textunderscore )}
\end{itemize}
Espécie de turmalina, silicato aluminoso, violáceo, cujos crystaes apresentam a configuração de um ferro de machado.
\section{Axinite}
\begin{itemize}
\item {fónica:csi}
\end{itemize}
\begin{itemize}
\item {Grp. gram.:f.}
\end{itemize}
(V.axinita)
\section{Axinomancia}
\begin{itemize}
\item {fónica:csi}
\end{itemize}
\begin{itemize}
\item {Grp. gram.:f.}
\end{itemize}
\begin{itemize}
\item {Proveniência:(Do gr. \textunderscore axine\textunderscore  + \textunderscore manteia\textunderscore )}
\end{itemize}
Antiga arte de adivinhar, por meio de um machado.
\section{Axinomântico}
\begin{itemize}
\item {fónica:csi}
\end{itemize}
\begin{itemize}
\item {Grp. gram.:adj.}
\end{itemize}
Relativo á \textunderscore axinomância\textunderscore .
\section{Axioma}
\begin{itemize}
\item {Grp. gram.:m.}
\end{itemize}
\begin{itemize}
\item {Proveniência:(Gr. \textunderscore axioma\textunderscore )}
\end{itemize}
Proposição, cuja verdade é evidente.
Máxima.
\section{Axiomancia}
\begin{itemize}
\item {fónica:csi}
\end{itemize}
\begin{itemize}
\item {Grp. gram.:f.}
\end{itemize}
O mesmo que \textunderscore axinomancia\textunderscore . Cf. Castilho, \textunderscore Fastos\textunderscore , III, 323.
\section{Axiomático}
\begin{itemize}
\item {Grp. gram.:adj.}
\end{itemize}
Incontestável; evidente: \textunderscore verdade axiomática\textunderscore .
Que tem o carácter de \textunderscore axioma\textunderscore .
\section{Axiómetro}
\begin{itemize}
\item {fónica:csi}
\end{itemize}
\begin{itemize}
\item {Grp. gram.:m.}
\end{itemize}
\begin{itemize}
\item {Proveniência:(Do gr. \textunderscore axon\textunderscore  + \textunderscore metron\textunderscore )}
\end{itemize}
Instrumento náutico, que faz conhecer a posição da roda do leme, indicando assim a direcção da barra.
\section{Axiomórfico}
\begin{itemize}
\item {fónica:csi}
\end{itemize}
\begin{itemize}
\item {Grp. gram.:adj.}
\end{itemize}
Diz-se de uma variedade de cal carbonatada.
\section{Axiomórphico}
\begin{itemize}
\item {fónica:csi}
\end{itemize}
\begin{itemize}
\item {Grp. gram.:adj.}
\end{itemize}
Diz-se de uma variedade de cal carbonatada.
\section{Axípeto}
\begin{itemize}
\item {fónica:csi}
\end{itemize}
\begin{itemize}
\item {Grp. gram.:adj.}
\end{itemize}
\begin{itemize}
\item {Proveniência:(Do lat. \textunderscore axis\textunderscore  + \textunderscore petere\textunderscore )}
\end{itemize}
O mesmo que \textunderscore centrípeto\textunderscore .
\section{Áxis}
\begin{itemize}
\item {fónica:csis}
\end{itemize}
\begin{itemize}
\item {Grp. gram.:m.}
\end{itemize}
\begin{itemize}
\item {Proveniência:(Lat. \textunderscore axis\textunderscore )}
\end{itemize}
Segunda vértebra cervical.
Ruminante asiático, espécie de veado.
\section{Á-xis}
\begin{itemize}
\item {Grp. gram.:m.}
\end{itemize}
\begin{itemize}
\item {Utilização:Ant.}
\end{itemize}
O abecê, o princípio das coisas:«\textunderscore farteei tornar ao axis\textunderscore ».
Chiado, \textunderscore Obras\textunderscore , 187.«\textunderscore Quando V. M. pegou no a. x., já eu tinha de cór...\textunderscore »J. Daniel, \textunderscore Casa de Pasto\textunderscore .
\section{Àxis}
\begin{itemize}
\item {Grp. gram.:m.}
\end{itemize}
\begin{itemize}
\item {Utilização:Ant.}
\end{itemize}
O abecê, o princípio das coisas:«\textunderscore farteei tornar ao axis\textunderscore ».
Chiado, \textunderscore Obras\textunderscore , 187.«\textunderscore Quando V. M. pegou no a. x., já eu tinha de cór...\textunderscore »J. Daniel, \textunderscore Casa de Pasto\textunderscore .
\section{Axófito}
\begin{itemize}
\item {fónica:csó}
\end{itemize}
\begin{itemize}
\item {Grp. gram.:m.}
\end{itemize}
\begin{itemize}
\item {Proveniência:(Do gr. \textunderscore axon\textunderscore  + gr. \textunderscore phuton\textunderscore )}
\end{itemize}
Nome, dado por alguns botânicos á haste da planta.
\section{Axóide}
\begin{itemize}
\item {fónica:csói}
\end{itemize}
\begin{itemize}
\item {Grp. gram.:m.}
\end{itemize}
\begin{itemize}
\item {Grp. gram.:Adj.}
\end{itemize}
\begin{itemize}
\item {Proveniência:(Do gr. \textunderscore axon\textunderscore  + \textunderscore eidos\textunderscore )}
\end{itemize}
Áxis, segunda vértebra cervical.
Que tem fórma de eixo.
\section{Axoídeo}
\begin{itemize}
\item {fónica:csó}
\end{itemize}
\begin{itemize}
\item {Grp. gram.:adj.}
\end{itemize}
\begin{itemize}
\item {Proveniência:(Do gr. \textunderscore axon\textunderscore  + \textunderscore eidos\textunderscore )}
\end{itemize}
Que tem fórma de eixo.
\section{Axone}
\begin{itemize}
\item {fónica:csó}
\end{itemize}
\begin{itemize}
\item {Grp. gram.:m.}
\end{itemize}
\begin{itemize}
\item {Utilização:Hist. Nat.}
\end{itemize}
\begin{itemize}
\item {Proveniência:(Gr. \textunderscore axon\textunderscore )}
\end{itemize}
Prolongamento das fibras nervosas, que exercem funcções noutras fibras.
\section{Axonométrico}
\begin{itemize}
\item {fónica:csó}
\end{itemize}
\begin{itemize}
\item {Grp. gram.:adj.}
\end{itemize}
\begin{itemize}
\item {Proveniência:(Gr. \textunderscore axon\textunderscore  + \textunderscore metron\textunderscore )}
\end{itemize}
Diz-se da perspectiva, que é uma projecção orthogonal sôbre um plano oblíquo ás três dimensões do corpo que se se não divide em direcção transversal e parallela á base.
(Do gr. \textunderscore axon\textunderscore  + \textunderscore tomos\textunderscore )tem de reproduzir.
\section{Axóphyto}
\begin{itemize}
\item {Grp. gram.:m.}
\end{itemize}
\begin{itemize}
\item {Proveniência:(Do gr. \textunderscore axon\textunderscore  + gr. \textunderscore phuton\textunderscore )}
\end{itemize}
Nome, dado por alguns botânicos á haste da planta.
\section{Axorar}
\begin{itemize}
\item {Grp. gram.:v. t.}
\end{itemize}
\begin{itemize}
\item {Utilização:Ant.}
\end{itemize}
\begin{itemize}
\item {Proveniência:(Do ár. \textunderscore axura\textunderscore ? Mais provavelmente do ingl. \textunderscore shore\textunderscore )}
\end{itemize}
Expulsar.
Arruinar.
Desbaratar:«\textunderscore á légoa vós axorareis quantos Almograves vierem\textunderscore ». \textunderscore Aulegrafia\textunderscore , 135. Fazer evacuar (uma nau)
\section{Axorca}
\begin{itemize}
\item {Grp. gram.:f.}
\end{itemize}
\begin{itemize}
\item {Proveniência:(Do ár. \textunderscore ax-xorca\textunderscore )}
\end{itemize}
Argola, com que, em alguns povos incultos, se enfeitam pernas ou braços.
Manilha.
\section{Axótomo}
\begin{itemize}
\item {fónica:csó}
\end{itemize}
\begin{itemize}
\item {Grp. gram.:adj.}
\end{itemize}
\begin{itemize}
\item {Proveniência:(Do gr. \textunderscore axon\textunderscore  + \textunderscore tomos\textunderscore )}
\end{itemize}
Diz-se do crystal, que se não divide em direcção transversal e parallela á base.
\section{Axúngia}
\begin{itemize}
\item {fónica:csun}
\end{itemize}
\begin{itemize}
\item {Grp. gram.:f.}
\end{itemize}
\begin{itemize}
\item {Proveniência:(Lat. \textunderscore axungia\textunderscore )}
\end{itemize}
Substância gordurosa, com que se untavam os eixos dos carros.
Gordura de porco, para usos pharmacêuticos; banha.
\section{Áxylo}
\begin{itemize}
\item {Grp. gram.:adj.}
\end{itemize}
\begin{itemize}
\item {Proveniência:(Do gr. \textunderscore a\textunderscore  priv. + \textunderscore xulon\textunderscore )}
\end{itemize}
Que não produz madeira, como certos vegetaes cellulares.
\section{Az}
\begin{itemize}
\item {Grp. gram.:f.}
\end{itemize}
\begin{itemize}
\item {Utilização:Ant.}
\end{itemize}
\begin{itemize}
\item {Proveniência:(Do lat. \textunderscore acies\textunderscore ?)}
\end{itemize}
Esquadrão.
Ala do exército.
Arraial.
\section{Azabumbado}
\begin{itemize}
\item {Grp. gram.:adj.}
\end{itemize}
Que tem fórma de zabumba.
Batido como um zabumba.
Banzado: \textunderscore ficou azabumbado com a notícia\textunderscore .
\section{Azabumbar}
\begin{itemize}
\item {Grp. gram.:v. t.}
\end{itemize}
\begin{itemize}
\item {Grp. gram.:V. i.}
\end{itemize}
\begin{itemize}
\item {Proveniência:(De \textunderscore zabumba\textunderscore )}
\end{itemize}
Aturdir.
Bater em.
Embatucar.
\section{Azáfama}
\begin{itemize}
\item {Grp. gram.:f.}
\end{itemize}
\begin{itemize}
\item {Proveniência:(Do ár. \textunderscore azzahma\textunderscore )}
\end{itemize}
Affluência de negócios.
Apêrto de gente.
Pressa.
Atrapalhação.
\section{Azafamadamente}
\begin{itemize}
\item {Grp. gram.:adv.}
\end{itemize}
De modo \textunderscore azafamado\textunderscore .
Com azáfama.
\section{Azafamado}
\begin{itemize}
\item {Grp. gram.:adj.}
\end{itemize}
Que tem azáfama.
Apressado.
Atarefado.
\section{Azafamar-se}
\begin{itemize}
\item {Grp. gram.:v. p.}
\end{itemize}
Têr azáfama.
Trabalhar com actividade.
\section{Azaga}
\begin{itemize}
\item {Grp. gram.:f.}
\end{itemize}
\begin{itemize}
\item {Utilização:Ant.}
\end{itemize}
O mesmo que \textunderscore rètaguarda\textunderscore . Cf. Herculano, \textunderscore Hist. de Port.\textunderscore , IV, 415.
\section{Azagaia}
\begin{itemize}
\item {Grp. gram.:f.}
\end{itemize}
Lança curta.
(Do berb.)
\section{Azagaiada}
\begin{itemize}
\item {Grp. gram.:f.}
\end{itemize}
Golpe de azagaia.
\section{Azagaiar}
\begin{itemize}
\item {Grp. gram.:v. t.}
\end{itemize}
Ferir ou matar com azagaia.
\section{Azagal}
\begin{itemize}
\item {Grp. gram.:m.}
\end{itemize}
\begin{itemize}
\item {Utilização:Prov.}
\end{itemize}
O mesmo que \textunderscore zagal\textunderscore .
\section{Azagre}
\begin{itemize}
\item {Grp. gram.:m.}
\end{itemize}
\begin{itemize}
\item {Utilização:Prov.}
\end{itemize}
\begin{itemize}
\item {Utilização:Trasm.}
\end{itemize}
(Corr. de \textunderscore uzagre\textunderscore )
\section{Azaina}
\begin{itemize}
\item {Grp. gram.:f.}
\end{itemize}
Antiga trombeta de guerra.
\section{Azal}
\begin{itemize}
\item {Grp. gram.:m.}
\end{itemize}
Espécie de uva branca minhota.
\section{Azálea}
\begin{itemize}
\item {Grp. gram.:f.}
\end{itemize}
\begin{itemize}
\item {Proveniência:(Gr. \textunderscore azaleos\textunderscore )}
\end{itemize}
Planta ericínea, muito apreciada pela elegância das suas flôres.
\section{Azaleáceas}
\begin{itemize}
\item {Grp. gram.:f. pl.}
\end{itemize}
\begin{itemize}
\item {Proveniência:(De \textunderscore azálea\textunderscore )}
\end{itemize}
Fam. de plantas, muito próxima das ericíneas.
\section{Azamar}
\begin{itemize}
\item {Grp. gram.:m.}
\end{itemize}
O mesmo que \textunderscore vermelhão\textunderscore .
\section{Azamboado}
\begin{itemize}
\item {Grp. gram.:adj.}
\end{itemize}
\begin{itemize}
\item {Proveniência:(De \textunderscore azamboar\textunderscore )}
\end{itemize}
Áspero.
Insípido, como a zambôa.
Atoleimado.
\section{Azamboar}
\begin{itemize}
\item {Grp. gram.:v. t.}
\end{itemize}
\begin{itemize}
\item {Proveniência:(De \textunderscore zambôa\textunderscore )}
\end{itemize}
Entontecer.
Tornar insípido.
\section{Azambuj...}
(V.zambuj...)
\section{Azambuja}
\begin{itemize}
\item {Grp. gram.:adj. f.}
\end{itemize}
\begin{itemize}
\item {Utilização:Prov.}
\end{itemize}
\begin{itemize}
\item {Utilização:minh.}
\end{itemize}
Diz-se de uma variedade de oliveira, (\textunderscore olea europaea oleaster\textunderscore ).
\section{Azanegue}
\begin{itemize}
\item {Grp. gram.:m.}
\end{itemize}
Outra fórma de \textunderscore azenegue\textunderscore . Cf. Barros, \textunderscore Déc.\textunderscore  I, l. III, c. 8.
\section{Azangar}
\begin{itemize}
\item {Grp. gram.:v. t.}
\end{itemize}
\begin{itemize}
\item {Utilização:Prov.}
\end{itemize}
Agoirar.
Enguiçar.
Infelicitar: \textunderscore aquella bruxa azangou-me\textunderscore .
\section{Azango}
\begin{itemize}
\item {Grp. gram.:m.}
\end{itemize}
\begin{itemize}
\item {Utilização:Prov.}
\end{itemize}
\begin{itemize}
\item {Utilização:minh.}
\end{itemize}
Azar; desdita; sorte má.
(Por \textunderscore azaigo\textunderscore , metáth. de \textunderscore aziago\textunderscore ?)
\section{Azaóla}
\begin{itemize}
\item {Grp. gram.:f.}
\end{itemize}
Género de plantas sapotáceas.
\section{Azaqui}
\begin{itemize}
\item {Grp. gram.:m.}
\end{itemize}
\begin{itemize}
\item {Proveniência:(Do ár. \textunderscore az-zaqui\textunderscore )}
\end{itemize}
Tributo, que entre nós pagaram os Moiros, e que consistia na décima parte dos frutos da terra.
\section{Azar}
\begin{itemize}
\item {Grp. gram.:m.}
\end{itemize}
\begin{itemize}
\item {Proveniência:(Do ár. \textunderscore az-zahr\textunderscore )}
\end{itemize}
Sorte.
Aventura infeliz.
\section{Azar}
\begin{itemize}
\item {Grp. gram.:m.}
\end{itemize}
Antiga moeda de Ormuz.
\section{Azar}
\begin{itemize}
\item {Grp. gram.:m.}
\end{itemize}
Casta de uva branca de Basto.
\section{Azara}
\begin{itemize}
\item {Grp. gram.:m.}
\end{itemize}
Gênero de mollúscos.
Gênero de plantas bixáceas do Chile e do Brasil meridional.
\section{Azara}
\begin{itemize}
\item {Grp. gram.:m.}
\end{itemize}
\begin{itemize}
\item {Utilização:Ant.}
\end{itemize}
O mesmo que \textunderscore azar\textunderscore ^1:«\textunderscore azara te veio\textunderscore ». G. Vicente, I, 343.
\section{Azaranzar-se}
\begin{itemize}
\item {Grp. gram.:v. p.}
\end{itemize}
\begin{itemize}
\item {Utilização:Bras}
\end{itemize}
O mesmo que \textunderscore zaranzar\textunderscore .
\section{Azarcão}
\begin{itemize}
\item {Grp. gram.:m.}
\end{itemize}
(V.zarcão)
\section{Azareiro}
\begin{itemize}
\item {Grp. gram.:m.}
\end{itemize}
(V.azereiro)
\section{Azares}
\begin{itemize}
\item {Grp. gram.:m.}
\end{itemize}
Planta trepadeira, cujas flôres brancas formam cacho aromático.
\section{Azaria}
\begin{itemize}
\item {Grp. gram.:f.}
\end{itemize}
\begin{itemize}
\item {Utilização:Ant.}
\end{itemize}
O mesmo que \textunderscore escaramuça\textunderscore . Cf. Herculano, \textunderscore Hist. de Port.\textunderscore , IV, 408 e 415.
\section{Azarma}
\begin{itemize}
\item {Grp. gram.:f.}
\end{itemize}
Provavelmente, um animal qualquer:«\textunderscore ...lhe enviou em presente seis alãos e seis sabujos... e trinta azarmas, todas com contas e anguadas de prata\textunderscore ». Fern. Lopes, \textunderscore Chrón. de D. Fernando\textunderscore , cap. XLIII.
\section{Azarola}
\begin{itemize}
\item {Grp. gram.:f.}
\end{itemize}
\begin{itemize}
\item {Proveniência:(Do ár. \textunderscore az-zarora\textunderscore )}
\end{itemize}
Fruto acídulo do azaroleiro.
\section{Azaroleira}
\begin{itemize}
\item {Grp. gram.:f.}
\end{itemize}
O mesmo que \textunderscore azaroleiro\textunderscore .
\section{Azaroleiro}
\begin{itemize}
\item {Grp. gram.:m.}
\end{itemize}
\begin{itemize}
\item {Proveniência:(De \textunderscore azarola\textunderscore )}
\end{itemize}
Árvore rosácea, da tríbo das pomáceas.
\section{Azarucha}
\begin{itemize}
\item {Grp. gram.:f.}
\end{itemize}
\begin{itemize}
\item {Utilização:Ant.}
\end{itemize}
O mesmo que \textunderscore azaruja\textunderscore .
\section{Azaruja}
\begin{itemize}
\item {Grp. gram.:f.}
\end{itemize}
\begin{itemize}
\item {Utilização:Ant.}
\end{itemize}
Herdade.
\section{Azcuma}
\begin{itemize}
\item {Grp. gram.:f.}
\end{itemize}
O mesmo que \textunderscore ascuma\textunderscore .
\section{Azebrado}
\begin{itemize}
\item {Grp. gram.:adj.}
\end{itemize}
Coberto de azebre ou verdete:«\textunderscore o punho azebrado da espada\textunderscore ». Camillo, \textunderscore Brasileira\textunderscore , 64.
\section{Azebrar}
\begin{itemize}
\item {Grp. gram.:v. t.}
\end{itemize}
Cobrir de azebre.
\section{Azebre}
\begin{itemize}
\item {fónica:zê}
\end{itemize}
\begin{itemize}
\item {Grp. gram.:m.}
\end{itemize}
\begin{itemize}
\item {Utilização:Pop.}
\end{itemize}
\begin{itemize}
\item {Proveniência:(Do ár. \textunderscore aç-cibrar\textunderscore )}
\end{itemize}
Aloés.
Mistura de hydrato e de carbonato de cobre, que se fórma na superfície dos objectos de cobre, por effeito da humidade do ar ou dos liquidos; verdete.
Finura.
Malicia: gaiatice.
\section{Azêda}
\begin{itemize}
\item {Grp. gram.:f.}
\end{itemize}
\begin{itemize}
\item {Proveniência:(De \textunderscore azêdo\textunderscore )}
\end{itemize}
Designação de várias plantas oxalídeas.
Planta hortícola, de gôsto ácido, de que se extrai o chamado \textunderscore sal-de-azêdas\textunderscore .
\section{Azêda-brava}
\begin{itemize}
\item {Grp. gram.:f.}
\end{itemize}
Planta africana, de fôlhas comestíveis, que habita terrenos arenosos.
\section{Azedador}
\begin{itemize}
\item {Grp. gram.:m.}
\end{itemize}
\begin{itemize}
\item {Grp. gram.:Adj.}
\end{itemize}
Aquelle que azéda.
Que azéda.
\section{Azedamente}
\begin{itemize}
\item {fónica:zê}
\end{itemize}
\begin{itemize}
\item {Grp. gram.:adv.}
\end{itemize}
\begin{itemize}
\item {Proveniência:(De \textunderscore azêdo\textunderscore )}
\end{itemize}
Com azedume.
\section{Azedamento}
\begin{itemize}
\item {Grp. gram.:m.}
\end{itemize}
Effeito de \textunderscore azedar\textunderscore .
\section{Azedar}
\begin{itemize}
\item {Grp. gram.:v. t.}
\end{itemize}
Tornar azêdo.
Causar mau humor a; irritar.
\section{Azedeira}
\begin{itemize}
\item {Grp. gram.:f.}
\end{itemize}
(V.azêda)
\section{Azederaco}
\begin{itemize}
\item {Grp. gram.:m.}
\end{itemize}
Árvore tropical, de fruto venenoso.
(Do persa)
\section{Azederaque}
\begin{itemize}
\item {Grp. gram.:m.}
\end{itemize}
Árvore tropical, de fruto venenoso.
(Do persa)
\section{Azedete}
\begin{itemize}
\item {fónica:dê}
\end{itemize}
\begin{itemize}
\item {Grp. gram.:adj.}
\end{itemize}
(Dem. de \textunderscore azêdo\textunderscore )
\section{Azedia}
\begin{itemize}
\item {Grp. gram.:f.}
\end{itemize}
O mesmo que \textunderscore azedume\textunderscore .
\section{Azedinha}
\begin{itemize}
\item {Grp. gram.:f.}
\end{itemize}
\begin{itemize}
\item {Proveniência:(De \textunderscore azêda\textunderscore )}
\end{itemize}
Planta vulgar, mais pequena e mais ácida que a azêda.
Fruta do Brasil.
\section{Azedinha-do-brejo}
\begin{itemize}
\item {Grp. gram.:f.}
\end{itemize}
Planta begoniácea, (\textunderscore begonia-acida\textunderscore ).
\section{Azêdo}
\begin{itemize}
\item {Grp. gram.:adj.}
\end{itemize}
\begin{itemize}
\item {Utilização:Fig.}
\end{itemize}
\begin{itemize}
\item {Grp. gram.:M.}
\end{itemize}
\begin{itemize}
\item {Proveniência:(Lat. \textunderscore acetum\textunderscore )}
\end{itemize}
Que tem sabor ácido, áspero, como o do vinagre, etc.
Corrompido por fermentação.
Rude: \textunderscore temperamento azêdo\textunderscore .
Irado
O mesmo que \textunderscore azedume\textunderscore .
\section{Azedum}
\begin{itemize}
\item {Grp. gram.:m.}
\end{itemize}
\begin{itemize}
\item {Utilização:Pop.}
\end{itemize}
O mesmo que \textunderscore azedume\textunderscore .
\section{Azedume}
\begin{itemize}
\item {Grp. gram.:m.}
\end{itemize}
Qualidade do que é azêdo.
Sabor ácido.
Agastamento.
Ira.
\section{Azeirado}
\begin{itemize}
\item {Grp. gram.:adj.}
\end{itemize}
Mordaz, azêdo. Cf. Herculano, \textunderscore Bobo\textunderscore , 30.
(Cp. \textunderscore acerar\textunderscore )
\section{Azeitada}
\begin{itemize}
\item {Grp. gram.:f.}
\end{itemize}
\begin{itemize}
\item {Proveniência:(De \textunderscore azeitar\textunderscore )}
\end{itemize}
Porção de azeite, com que se tempera a comida, ou que se entorna por acaso.
\section{Azeitadeira}
\begin{itemize}
\item {Grp. gram.:f.}
\end{itemize}
\begin{itemize}
\item {Proveniência:(De \textunderscore azeitar\textunderscore )}
\end{itemize}
Apparelho, para azeitar a lan, em fábricas de tecidos. Cf. \textunderscore Inquér. Industr.\textunderscore , p. II, l. I, 72.
\section{Azeitado}
\begin{itemize}
\item {Grp. gram.:adj.}
\end{itemize}
Teperado com azeite.
\section{Azeitão}
\begin{itemize}
\item {Grp. gram.:adj.}
\end{itemize}
\begin{itemize}
\item {Utilização:Bras. do N}
\end{itemize}
\begin{itemize}
\item {Proveniência:(De \textunderscore azeite\textunderscore )}
\end{itemize}
Preto e lustroso.
\section{Azeitar}
\begin{itemize}
\item {Grp. gram.:v. t.}
\end{itemize}
\begin{itemize}
\item {Utilização:Bras}
\end{itemize}
\begin{itemize}
\item {Utilização:fam.}
\end{itemize}
\begin{itemize}
\item {Proveniência:(De \textunderscore azeite\textunderscore )}
\end{itemize}
Temperar com azeite.
Untar com azeite ou outro óleo.
Namorar.
\section{Azeite}
\begin{itemize}
\item {Grp. gram.:m.}
\end{itemize}
\begin{itemize}
\item {Grp. gram.:Pl.}
\end{itemize}
\begin{itemize}
\item {Utilização:Fam.}
\end{itemize}
\begin{itemize}
\item {Proveniência:(Do ár. \textunderscore al-zait\textunderscore )}
\end{itemize}
Óleo, que se extrai da azeitona.
O fruto da oliveira: \textunderscore o olival produziu muito azeite\textunderscore .
Óleo, extrahido de outras plantas e de alguns animaes: \textunderscore azeite de baleia\textunderscore .
Mau humor: \textunderscore está hoje com os seus azeites\textunderscore .
\section{Azeiteira}
\begin{itemize}
\item {Grp. gram.:f.}
\end{itemize}
Almotolia, vaso para azeite.
\section{Azeiteiro}
\begin{itemize}
\item {Grp. gram.:m.}
\end{itemize}
\begin{itemize}
\item {Utilização:Bras}
\end{itemize}
\begin{itemize}
\item {Grp. gram.:Adj.}
\end{itemize}
Aquelle que vende ou fabríca azeite.
Rufião.
Relativo ao azeite.
\section{Azeitona}
\begin{itemize}
\item {Grp. gram.:f.}
\end{itemize}
\begin{itemize}
\item {Proveniência:(Do ár. \textunderscore az-zeituna\textunderscore )}
\end{itemize}
O fruto da oliveira.
\section{Azeitona}
\begin{itemize}
\item {Grp. gram.:f.}
\end{itemize}
Árvore da ilha de San-Thomé, (\textunderscore sideroxylon densiflorum\textunderscore ).
\section{Azeitonado}
\begin{itemize}
\item {Grp. gram.:adj.}
\end{itemize}
Que tem côr de azeitona.
\section{Azeitona-rei}
\begin{itemize}
\item {Grp. gram.:f.}
\end{itemize}
Variedade de oliveira e seu fruto, (\textunderscore olea regia\textunderscore , Ros.).
\section{Azeitoneira}
\begin{itemize}
\item {Grp. gram.:f.}
\end{itemize}
\begin{itemize}
\item {Utilização:Prov.}
\end{itemize}
\begin{itemize}
\item {Utilização:beir.}
\end{itemize}
Prato, ou vaso, em que se servem azeitonas.
O mesmo que \textunderscore oliveira\textunderscore .
\section{Azeitoneiro}
\begin{itemize}
\item {Grp. gram.:m.}
\end{itemize}
Aquelle que vende azeitonas.
Azeitoneira.
\section{Azêmala}
\begin{itemize}
\item {Grp. gram.:f.}
\end{itemize}
(V.azêmula)
\section{Azemel}
\begin{itemize}
\item {Grp. gram.:m.}
\end{itemize}
\begin{itemize}
\item {Utilização:Ant.}
\end{itemize}
\begin{itemize}
\item {Utilização:Prov.}
\end{itemize}
\begin{itemize}
\item {Utilização:trasm.}
\end{itemize}
Almocreve.
Arraial.
Abarracamento.
Rancho. Cf. Viterbo, \textunderscore Elucid\textunderscore .
Homem desleal; traidor.
\section{Azêmela}
\begin{itemize}
\item {Grp. gram.:f.}
\end{itemize}
\begin{itemize}
\item {Proveniência:(Do ár. \textunderscore az-zemila\textunderscore )}
\end{itemize}
Bêsta de carga, que fórma récua com outras.
Bêsta velha e cansada.
\section{Azêmola}
\begin{itemize}
\item {Grp. gram.:f.}
\end{itemize}
\begin{itemize}
\item {Proveniência:(Do ár. \textunderscore az-zemila\textunderscore )}
\end{itemize}
Bêsta de carga, que fórma récua com outras.
Bêsta velha e cansada.
\section{Azêmula}
\begin{itemize}
\item {Grp. gram.:f.}
\end{itemize}
\begin{itemize}
\item {Proveniência:(Do ár. \textunderscore az-zemila\textunderscore )}
\end{itemize}
Bêsta de carga, que fórma récua com outras.
Bêsta velha e cansada.
\section{Azenegue}
\begin{itemize}
\item {Grp. gram.:m.}
\end{itemize}
\begin{itemize}
\item {Grp. gram.:Pl.}
\end{itemize}
A língua dos Berberes; o Berbere.
Tríbos moiriscas do Sahará Occidental e do Noroéste da África. Cf. \textunderscore Lusíadas\textunderscore , V, 6; Azurara, \textunderscore Descobr. e Conq. da Guiné\textunderscore .
\section{Azenha}
\begin{itemize}
\item {Grp. gram.:f.}
\end{itemize}
Moínho de rodízio, movido por água.
(Cast. \textunderscore aceña\textunderscore )
\section{Azentio}
\begin{itemize}
\item {Grp. gram.:m.}
\end{itemize}
\begin{itemize}
\item {Utilização:Prov.}
\end{itemize}
\begin{itemize}
\item {Utilização:trasm.}
\end{itemize}
Terreno alagadiço.
\section{Ázeo}
\begin{itemize}
\item {Grp. gram.:m.}
\end{itemize}
\begin{itemize}
\item {Utilização:Ant.}
\end{itemize}
\begin{itemize}
\item {Utilização:Prov.}
\end{itemize}
\begin{itemize}
\item {Utilização:beir.}
\end{itemize}
\begin{itemize}
\item {Proveniência:(Lat. \textunderscore acinus\textunderscore )}
\end{itemize}
Bago de uva.
Cada um dos grupos de uvas, que constituem um cacho.
\section{Azerado}
\begin{itemize}
\item {Grp. gram.:adj.}
\end{itemize}
\begin{itemize}
\item {Proveniência:(De \textunderscore azerar\textunderscore )}
\end{itemize}
Que tem côr de aço.
\section{Azerar}
\begin{itemize}
\item {Grp. gram.:v. t.}
\end{itemize}
Dar côr de aço a (caracteres impressos) das fôlhas que se cortam.
O mesmo que \textunderscore acerar\textunderscore .
\section{Azeredo}
\begin{itemize}
\item {fónica:zerê}
\end{itemize}
\begin{itemize}
\item {Grp. gram.:m.}
\end{itemize}
Mata de azereiros.
\section{Azereiro}
\begin{itemize}
\item {Grp. gram.:m.}
\end{itemize}
\begin{itemize}
\item {Proveniência:(Do lat. \textunderscore acer\textunderscore ?)}
\end{itemize}
Árvore rosácea, da tríbo das amygdaláceas.
\section{Azerola}
\begin{itemize}
\item {Grp. gram.:f.}
\end{itemize}
(V.azarola)
\section{Azeroleira}
\begin{itemize}
\item {Grp. gram.:f.}
\end{itemize}
(V.azaroleiro)
\section{Azervada}
\begin{itemize}
\item {Grp. gram.:f.}
\end{itemize}
\begin{itemize}
\item {Utilização:Ant.}
\end{itemize}
Estacada.
Tapume; reunião de azerves.
\section{Azerve}
\begin{itemize}
\item {Grp. gram.:m.}
\end{itemize}
\begin{itemize}
\item {Proveniência:(Do ár. \textunderscore az-zerb\textunderscore )}
\end{itemize}
Resguardo de ramos contra o vento.
\section{Azevan}
\begin{itemize}
\item {Grp. gram.:f.}
\end{itemize}
O mesmo que \textunderscore ascuma\textunderscore . Cf. Garrett, \textunderscore Arco de Sant'Anna\textunderscore , I, 60; II, 93.
\section{Azevão}
\begin{itemize}
\item {Grp. gram.:m.}
\end{itemize}
O mesmo que \textunderscore azevan\textunderscore .
\section{Azevém}
\begin{itemize}
\item {Grp. gram.:m.}
\end{itemize}
Planta gramínea, vulgar, (\textunderscore solium perenne\textunderscore , Lin.).
\section{Azevia}
\begin{itemize}
\item {Grp. gram.:f.}
\end{itemize}
Espécie do linguado, (\textunderscore solea azevia\textunderscore ).
\section{Azevichado}
\begin{itemize}
\item {Grp. gram.:adj.}
\end{itemize}
Que tem côr de azeviche.
\section{Azevichar}
\begin{itemize}
\item {Grp. gram.:v. t.}
\end{itemize}
Dar côr de azeviche a.
\section{Azeviche}
\begin{itemize}
\item {Grp. gram.:m.}
\end{itemize}
\begin{itemize}
\item {Proveniência:(Do ár. \textunderscore açabach\textunderscore )}
\end{itemize}
Substância mineral, muito negra, luzidía e frágil.
Objecto muito negro.
\textunderscore De azeviche\textunderscore , que é muito negro: \textunderscore cabellos de azeviche\textunderscore .
\section{Azevieiro}
\begin{itemize}
\item {Grp. gram.:adj.}
\end{itemize}
Esperto.
Malicioso; agaiatado: olhos azevieiros.
Libertino.
(Por \textunderscore azebrieiro\textunderscore , de \textunderscore azêbre\textunderscore ?)
\section{Azevinheiro}
\begin{itemize}
\item {Grp. gram.:m.}
\end{itemize}
\begin{itemize}
\item {Utilização:Prov.}
\end{itemize}
\begin{itemize}
\item {Utilização:trasm.}
\end{itemize}
O mesmo que \textunderscore azevinho\textunderscore .
\section{Azevinho}
\begin{itemize}
\item {Grp. gram.:m.}
\end{itemize}
\begin{itemize}
\item {Utilização:Prov.}
\end{itemize}
Arbusto espinhoso, (\textunderscore ilex aquifolium\textunderscore ).
Casta de uva miúda.
\section{Azevre}
\begin{itemize}
\item {Grp. gram.:m.}
\end{itemize}
O mesmo que \textunderscore azebre\textunderscore . Cf. Camillo, \textunderscore Brasileira\textunderscore , 64.
\section{Azia}
\begin{itemize}
\item {Grp. gram.:f.}
\end{itemize}
\begin{itemize}
\item {Utilização:Pop.}
\end{itemize}
Azedume do estômago.
\textunderscore Azia de queixos\textunderscore , vontade de comer, bom appetite.
(Contr. de \textunderscore azedia\textunderscore )
\section{Aziago}
\begin{itemize}
\item {Grp. gram.:adj.}
\end{itemize}
Nefasto.
Que faz recear desgraça; que é de mau agoiro: \textunderscore dia aziago\textunderscore .
(Cast. \textunderscore aciago\textunderscore )
\section{Aziar}
\begin{itemize}
\item {Grp. gram.:m.}
\end{itemize}
\begin{itemize}
\item {Utilização:Fig.}
\end{itemize}
\begin{itemize}
\item {Proveniência:(Do ár. \textunderscore az-ziare\textunderscore )}
\end{itemize}
Espécie de torniquete, com que os alveitares apertam o focinho das bêstas, para as terem immóveis.
Tormento; coisa afflictiva.
\section{Azeúme}
\begin{itemize}
\item {Grp. gram.:m.}
\end{itemize}
Azedia.
Má disposição de espírito.
(Alter. de \textunderscore azedume\textunderscore )
\section{Ázigos}
\begin{itemize}
\item {Grp. gram.:m.}
\end{itemize}
\begin{itemize}
\item {Proveniência:(Gr. \textunderscore azugos\textunderscore )}
\end{itemize}
Veia, que se estende da columna lombar até á veia cava superior.
Nome de outra veia, que sobe pelo lado esquerdo da columna vertebral, terminando na outra veia do mesmo nome.
\section{Azimbre}
\begin{itemize}
\item {Grp. gram.:m.}
\end{itemize}
\begin{itemize}
\item {Utilização:Prov.}
\end{itemize}
\begin{itemize}
\item {Utilização:trasm.}
\end{itemize}
O mesmo que \textunderscore cimbre\textunderscore . Cf. Herculano, \textunderscore Abóbada\textunderscore .
\section{Azímela}
\begin{itemize}
\item {Grp. gram.:f.}
\end{itemize}
\begin{itemize}
\item {Utilização:Ant.}
\end{itemize}
O mesmo que \textunderscore azêmula\textunderscore .
\section{Azimita}
\begin{itemize}
\item {Grp. gram.:m.}
\end{itemize}
\begin{itemize}
\item {Proveniência:(De \textunderscore ázymo\textunderscore )}
\end{itemize}
Aquelle que se serve de pão ázimo.
\section{Ázimo}
\begin{itemize}
\item {Grp. gram.:adj.}
\end{itemize}
\begin{itemize}
\item {Grp. gram.:M.}
\end{itemize}
\begin{itemize}
\item {Proveniência:(Gr. \textunderscore azumos\textunderscore )}
\end{itemize}
Diz-so do pão, que não fermentou.
Pão, que não fermentou.
\section{Azímute}
\begin{itemize}
\item {Grp. gram.:m.}
\end{itemize}
\begin{itemize}
\item {Proveniência:(Do ár. \textunderscore as-samt\textunderscore )}
\end{itemize}
Círculo vertical, que passa por determinado ponto.
Arco do círculo do horizonte, comprehendido entre o meridiano do lugar e o círculo vertical que passa por um corpo celeste.
\section{Azimutal}
\begin{itemize}
\item {Grp. gram.:adj.}
\end{itemize}
Relativo ao azímute.
\section{Azinaga}
\begin{itemize}
\item {Grp. gram.:f.}
\end{itemize}
\begin{itemize}
\item {Utilização:Ant.}
\end{itemize}
O mesmo que \textunderscore azinhaga\textunderscore .
\section{Azincho}
\begin{itemize}
\item {Grp. gram.:m.}
\end{itemize}
\begin{itemize}
\item {Utilização:Des.}
\end{itemize}
O mesmo que \textunderscore cincho\textunderscore ^1.
\section{Azinha}
\begin{itemize}
\item {Grp. gram.:adv.}
\end{itemize}
\begin{itemize}
\item {Utilização:Ant.}
\end{itemize}
\begin{itemize}
\item {Proveniência:(It. \textunderscore agina\textunderscore )}
\end{itemize}
Depressa.
\section{Azinha}
\begin{itemize}
\item {Grp. gram.:f.}
\end{itemize}
Fruto da \textunderscore azinheira\textunderscore .
\section{Azinhá}
\begin{itemize}
\item {Grp. gram.:m.}
\end{itemize}
Pássaro do Brasil.
\section{Azinhaga}
\begin{itemize}
\item {Grp. gram.:f.}
\end{itemize}
Caminho estreito, entre propriedades rústicas, e ladeado de muros ou vallados altos; canada.
(Talvez de \textunderscore azinho\textunderscore )
\section{Azinhal}
\begin{itemize}
\item {Grp. gram.:m.}
\end{itemize}
\begin{itemize}
\item {Proveniência:(De \textunderscore azinho\textunderscore )}
\end{itemize}
Lugar, onde crescem azinheiras.
\section{Azinhavrar}
\begin{itemize}
\item {Grp. gram.:v. t.}
\end{itemize}
\begin{itemize}
\item {Proveniência:(De \textunderscore azinhavre\textunderscore )}
\end{itemize}
O mesmo que \textunderscore azebrar\textunderscore .
\section{Azinhavre}
\begin{itemize}
\item {Grp. gram.:m.}
\end{itemize}
\begin{itemize}
\item {Proveniência:(Do ár. \textunderscore az-zinjar\textunderscore )}
\end{itemize}
O mesmo que \textunderscore azebre\textunderscore .
\section{Azinheira}
\begin{itemize}
\item {Grp. gram.:f.}
\end{itemize}
O mesmo que \textunderscore azinheiro\textunderscore .
\section{Azinheiral}
\begin{itemize}
\item {Grp. gram.:m.}
\end{itemize}
O mesmo que \textunderscore azinhal\textunderscore . Cf. Castilho, \textunderscore Fastos\textunderscore , I, 95; II, 35.
\section{Azinheiro}
\begin{itemize}
\item {Grp. gram.:m.}
\end{itemize}
Espécie de carvalho, (\textunderscore quercus ilex\textunderscore ).
(Cp. \textunderscore azinho\textunderscore )
\section{Azinho}
\begin{itemize}
\item {Grp. gram.:m.}
\end{itemize}
\begin{itemize}
\item {Proveniência:(De um hyp. lat. \textunderscore ilicinus\textunderscore , de \textunderscore ilex\textunderscore )}
\end{itemize}
O mesmo que \textunderscore azinheiro\textunderscore .
\section{Azinhoso}
\begin{itemize}
\item {Grp. gram.:adj.}
\end{itemize}
Que tem azinhos.
\section{...ázio}
\begin{itemize}
\item {Grp. gram.:suf.}
\end{itemize}
\begin{itemize}
\item {Utilização:pop.}
\end{itemize}
\begin{itemize}
\item {Proveniência:(Lat. \textunderscore ...aceus\textunderscore )}
\end{itemize}
(designativo de grandeza): \textunderscore um copázio de vinho\textunderscore .
\section{Aziumado}
\begin{itemize}
\item {fónica:zi-u}
\end{itemize}
\begin{itemize}
\item {Grp. gram.:adj.}
\end{itemize}
\begin{itemize}
\item {Proveniência:(De \textunderscore aziumar\textunderscore )}
\end{itemize}
Irritado.
\section{Aziumar}
\begin{itemize}
\item {fónica:zi-u}
\end{itemize}
\begin{itemize}
\item {Grp. gram.:v. t.}
\end{itemize}
\begin{itemize}
\item {Grp. gram.:V. i.}
\end{itemize}
\begin{itemize}
\item {Proveniência:(De \textunderscore aziúme\textunderscore )}
\end{itemize}
Causar azedume a; irritar.
Azedar-se: \textunderscore o leite aziumou\textunderscore .
\section{Aziúme}
\begin{itemize}
\item {Grp. gram.:m.}
\end{itemize}
Azedia.
Má disposição de espírito.
(Alter. de \textunderscore azedume\textunderscore )
\section{Azmo}
\begin{itemize}
\item {Grp. gram.:adj.}
\end{itemize}
\begin{itemize}
\item {Utilização:Pop.}
\end{itemize}
O mesmo que \textunderscore ázymo\textunderscore .
\section{Azoada}
\begin{itemize}
\item {Grp. gram.:f.}
\end{itemize}
O mesmo que \textunderscore azoamento\textunderscore . Cf. Filinto, X, 134.
\section{Azoado}
\begin{itemize}
\item {Grp. gram.:adj.}
\end{itemize}
\begin{itemize}
\item {Proveniência:(De \textunderscore azoar\textunderscore )}
\end{itemize}
Atordoado.
Zangado.
\section{Azoamento}
\begin{itemize}
\item {Grp. gram.:m.}
\end{itemize}
Acto de \textunderscore azoar\textunderscore . Cf. Filinto, IX, 106; XIX, 78.
\section{Azoar}
\begin{itemize}
\item {Grp. gram.:v. t.}
\end{itemize}
\begin{itemize}
\item {Grp. gram.:V. i.}
\end{itemize}
\begin{itemize}
\item {Proveniência:(Do lat. \textunderscore ad-sonare\textunderscore )}
\end{itemize}
Atordoar.
Enfadar; tornar zangado.
Zangar-se.
\section{Azoico}
\begin{itemize}
\item {Grp. gram.:adj.}
\end{itemize}
\begin{itemize}
\item {Utilização:Geol.}
\end{itemize}
\begin{itemize}
\item {Proveniência:(Do gr. \textunderscore a\textunderscore  priv. + \textunderscore zoon\textunderscore )}
\end{itemize}
Que não é fossilífero, (falando-se de terrenos primitivos, isto é, formados antes dos seres organizados, embora em alguns terrenos, denominados azoicos, tenham apparecído vestígios do corpos organizados).
\section{Azoinante}
\begin{itemize}
\item {Grp. gram.:adj.}
\end{itemize}
Que azoina. Cf. Filinto, V, 112; X, 224.
\section{Azoinar}
\begin{itemize}
\item {Grp. gram.:v. t.}
\end{itemize}
Incommodar ou importunar, falando muito: \textunderscore azoinar os ouvidos de alguém\textunderscore .
(Talvez do lat. hyp. \textunderscore adsoniare\textunderscore )
\section{Azola}
\begin{itemize}
\item {Grp. gram.:f.}
\end{itemize}
Gênero de plantas cryptogâmicas, vizinho dos fêtos.
\section{Azolla}
\begin{itemize}
\item {Grp. gram.:f.}
\end{itemize}
Gênero de plantas cryptogâmicas, vizinho dos fêtos.
\section{Azomarato}
\begin{itemize}
\item {Grp. gram.:m.}
\end{itemize}
Sal, resultante da combinação do ácido azomárico com uma base.
\section{Azomárico}
\begin{itemize}
\item {Grp. gram.:adj.}
\end{itemize}
Diz-se de um ácido, que se obtém, fazendo ferver a colophana com uma grande porção de ácido azótico.
\section{Azombado}
\begin{itemize}
\item {Grp. gram.:adj.}
\end{itemize}
\begin{itemize}
\item {Utilização:Bras. de Minas}
\end{itemize}
Preoccupado.
\section{Azoodinamia}
\begin{itemize}
\item {Grp. gram.:f.}
\end{itemize}
\begin{itemize}
\item {Proveniência:(Do gr. \textunderscore a\textunderscore  priv. + \textunderscore zoon\textunderscore  + \textunderscore dunamos\textunderscore )}
\end{itemize}
Cessação ou perda das forças vitaes, por doença ou pela idade.
\section{Azoodinâmico}
\begin{itemize}
\item {Grp. gram.:adj.}
\end{itemize}
Relativo á \textunderscore azoodinamia\textunderscore .
\section{Azoodínamo}
\begin{itemize}
\item {Grp. gram.:m.}
\end{itemize}
Aquelle que soffre \textunderscore azoodinamia\textunderscore .
\section{Azoodynamia}
\begin{itemize}
\item {Grp. gram.:f.}
\end{itemize}
\begin{itemize}
\item {Proveniência:(Do gr. \textunderscore a\textunderscore  priv. + \textunderscore zoon\textunderscore  + \textunderscore dunamos\textunderscore )}
\end{itemize}
Cessação ou perda das forças vitaes, por doença ou pela idade.
\section{Azoodynâmico}
\begin{itemize}
\item {Grp. gram.:adj.}
\end{itemize}
Relativo á \textunderscore azoodynamia\textunderscore .
\section{Azoodýnamo}
\begin{itemize}
\item {Grp. gram.:m.}
\end{itemize}
Aquelle que soffre \textunderscore azoodynamia\textunderscore .
\section{Azoofilia}
\begin{itemize}
\item {Grp. gram.:f.}
\end{itemize}
Qualidade de azoófilo.
\section{Azoofílico}
\begin{itemize}
\item {Grp. gram.:adj.}
\end{itemize}
Relativo á azoofilia.
\section{Azoófilo}
\begin{itemize}
\item {Grp. gram.:adj.}
\end{itemize}
\begin{itemize}
\item {Proveniência:(Do gr. \textunderscore a\textunderscore  priv. + \textunderscore zoon\textunderscore  + \textunderscore philos\textunderscore )}
\end{itemize}
Que não gosta dos seres vivos, ou que tem affeição mórbida aos objectos inanimados. Cf. Sousa Martins, \textunderscore Nosogr.\textunderscore 
\section{Azoophilia}
\begin{itemize}
\item {Grp. gram.:f.}
\end{itemize}
Qualidade de azoóphilo.
\section{Azoophílico}
\begin{itemize}
\item {Grp. gram.:adj.}
\end{itemize}
Relativo á azoophilia.
\section{Azoóphilo}
\begin{itemize}
\item {Grp. gram.:adj.}
\end{itemize}
\begin{itemize}
\item {Proveniência:(Do gr. \textunderscore a\textunderscore  priv. + \textunderscore zoon\textunderscore  + \textunderscore philos\textunderscore )}
\end{itemize}
Que não gosta dos seres vivos, ou que tem affeição mórbida aos objectos inanimados. Cf. Sousa Martins, \textunderscore Nosogr.\textunderscore 
\section{Azoótico}
\begin{itemize}
\item {Grp. gram.:adj.}
\end{itemize}
\begin{itemize}
\item {Proveniência:(Do gr. \textunderscore a\textunderscore  priv. + \textunderscore zoon\textunderscore )}
\end{itemize}
Que não tem vestígios ou restos de seres organizados.
\section{Azoque}
\begin{itemize}
\item {Grp. gram.:m.}
\end{itemize}
Mercador árabe. Cf. Herculano, \textunderscore Lendas e Narrativas\textunderscore , 15.
\section{Azoraque}
\begin{itemize}
\item {Grp. gram.:m.}
\end{itemize}
\begin{itemize}
\item {Utilização:Bot.}
\end{itemize}
Nome pop. do \textunderscore corvolvulus tricolor\textunderscore .
Bons-dias.
\section{Azoratado}
\begin{itemize}
\item {Grp. gram.:adj.}
\end{itemize}
Que é doidivanas ou estroina.
(Cp. \textunderscore azoratar\textunderscore )
\section{Azoratar}
\begin{itemize}
\item {Grp. gram.:v. t.}
\end{itemize}
O mesmo que \textunderscore entontecer\textunderscore .
\section{Azoreira}
\begin{itemize}
\item {Grp. gram.:f.}
\end{itemize}
\begin{itemize}
\item {Utilização:Ant.}
\end{itemize}
Mata para lenha.
(Relaciona-se provavelmente com \textunderscore azereiro\textunderscore )
\section{Azorela}
\begin{itemize}
\item {Grp. gram.:f.}
\end{itemize}
Planta umbelífera da África austral.
\section{Azorragada}
\begin{itemize}
\item {Grp. gram.:f.}
\end{itemize}
Pancada com azorrague.
\section{Azorragamento}
\begin{itemize}
\item {Grp. gram.:m.}
\end{itemize}
Acto de \textunderscore azorragar\textunderscore . Cf. Arn. Gama, \textunderscore Motim\textunderscore , 378.
\section{Azorragar}
\begin{itemize}
\item {Grp. gram.:v. t.}
\end{itemize}
Açoitar com azorrague.
\section{Azorrague}
\begin{itemize}
\item {Grp. gram.:m.}
\end{itemize}
\begin{itemize}
\item {Utilização:Ext.}
\end{itemize}
Chicote. Látego, formado de uma ou mais correias enlaçadas.
Punição.
Flagello.
\section{Azorrar}
\begin{itemize}
\item {Grp. gram.:v. t.}
\end{itemize}
\begin{itemize}
\item {Proveniência:(De \textunderscore zorra\textunderscore )}
\end{itemize}
Arrastar pesadamente.
\section{Azotação}
\begin{itemize}
\item {Grp. gram.:f.}
\end{itemize}
Acto de \textunderscore azotar\textunderscore .
\section{Azotado}
\begin{itemize}
\item {Grp. gram.:adj.}
\end{itemize}
Que contém azoto.
\section{Azotar}
\begin{itemize}
\item {Grp. gram.:v. t.}
\end{itemize}
Tornar azotado.
\section{Azotato}
\begin{itemize}
\item {Grp. gram.:m.}
\end{itemize}
\begin{itemize}
\item {Proveniência:(De \textunderscore azoto\textunderscore )}
\end{itemize}
Sal, resultante da combinação do ácido azótico com uma base.
\section{Azote}
\begin{itemize}
\item {Grp. gram.:m.}
\end{itemize}
(V.azoto)
\section{Azote}
\begin{itemize}
\item {Grp. gram.:m.}
\end{itemize}
\begin{itemize}
\item {Proveniência:(Fr. \textunderscore azote\textunderscore , do gr. \textunderscore a\textunderscore  priv. + \textunderscore zoe\textunderscore , \textunderscore vida\textunderscore )}
\end{itemize}
Corpo simples, gasoso, que constitue a maior parte do ar atmosphérico.
\section{Azoteto}
\begin{itemize}
\item {fónica:tê}
\end{itemize}
\begin{itemize}
\item {Grp. gram.:m.}
\end{itemize}
Qualquer combinação de azoto com um radical ou com outro corpo simples.
\section{Azótico}
\begin{itemize}
\item {Grp. gram.:adj.}
\end{itemize}
\begin{itemize}
\item {Proveniência:(De \textunderscore azoto\textunderscore )}
\end{itemize}
Diz-se do ácido, que é uma combinação de azoto com o oxygênio.
\section{Azotito}
\begin{itemize}
\item {Grp. gram.:m.}
\end{itemize}
\begin{itemize}
\item {Proveniência:(De \textunderscore azoto\textunderscore )}
\end{itemize}
Sal, formado pela combinação do ácido azotoso com uma base.
\section{Azoto}
\begin{itemize}
\item {Grp. gram.:m.}
\end{itemize}
\begin{itemize}
\item {Proveniência:(Fr. \textunderscore azote\textunderscore , do gr. \textunderscore a\textunderscore  priv. + \textunderscore zoe\textunderscore , \textunderscore vida\textunderscore )}
\end{itemize}
Corpo simples, gasoso, que constitue a maior parte do ar atmosphérico.
\section{Azotoso}
\begin{itemize}
\item {Grp. gram.:adj.}
\end{itemize}
\begin{itemize}
\item {Proveniência:(De \textunderscore azoto\textunderscore )}
\end{itemize}
Diz-se de um ácido, resultante de uma combinação do oxigênio com azoto, mas menos oxygenado que o ácido azótico.
\section{Azotureto}
\begin{itemize}
\item {Grp. gram.:m.}
\end{itemize}
(V.azoteto)
\section{Azoturia}
\begin{itemize}
\item {Grp. gram.:f.}
\end{itemize}
\begin{itemize}
\item {Proveniência:(De \textunderscore azoto\textunderscore  + gr. \textunderscore ourein\textunderscore )}
\end{itemize}
Doença, caracterizada pela perda excessiva de ureia, (principio azotado)
\section{Azotúrico}
\begin{itemize}
\item {Grp. gram.:adj.}
\end{itemize}
\begin{itemize}
\item {Grp. gram.:M.}
\end{itemize}
Relativo á azoturia.
Aquelle que padece azoturia.
\section{Azougadamente}
\begin{itemize}
\item {Grp. gram.:adv.}
\end{itemize}
De modo \textunderscore azougado\textunderscore .
\section{Azougado}
\begin{itemize}
\item {Grp. gram.:adj.}
\end{itemize}
Vivo, esperto, finório.
\section{Azougar}
\begin{itemize}
\item {Grp. gram.:v. t.}
\end{itemize}
\begin{itemize}
\item {Utilização:Fig.}
\end{itemize}
\begin{itemize}
\item {Grp. gram.:V. i.}
\end{itemize}
\begin{itemize}
\item {Utilização:Açor}
\end{itemize}
\begin{itemize}
\item {Utilização:Mad}
\end{itemize}
\begin{itemize}
\item {Utilização:Prov.}
\end{itemize}
Juntar com azougue.
Tornar vivo, esperto.
Fazer murchar as fôlhas de.
Apodrecer, (falando-se de batatas, laranjas, etc.).
Morrer, (falando-se de animaes).
Saturar-se de água.
Definhar.
\section{Azougue}
\begin{itemize}
\item {Grp. gram.:m.}
\end{itemize}
\begin{itemize}
\item {Utilização:Fig.}
\end{itemize}
\begin{itemize}
\item {Proveniência:(Do ár. \textunderscore az-zoca\textunderscore )}
\end{itemize}
O mesmo que \textunderscore mercúrio\textunderscore .
Pessôa esperta, ladina.
Esperteza, finura.
Planta euphorbiácea do Brasil.
\section{Azongue-dos-pobres}
\begin{itemize}
\item {Grp. gram.:m.}
\end{itemize}
\begin{itemize}
\item {Utilização:Bras}
\end{itemize}
Cucurbitácea medicinal.
\section{Aztecas}
\begin{itemize}
\item {Grp. gram.:m. pl.}
\end{itemize}
Indígenas do México.
\section{Azucrim}
\begin{itemize}
\item {Grp. gram.:m.}
\end{itemize}
\begin{itemize}
\item {Utilização:Bras. do N}
\end{itemize}
Pessôa importuna, maçadora.
\section{Azucrinar}
\begin{itemize}
\item {Grp. gram.:v. t.  e  i.}
\end{itemize}
\begin{itemize}
\item {Utilização:Bras. do N}
\end{itemize}
Importunar, maçar.
\section{Azul}
\begin{itemize}
\item {Grp. gram.:adj.}
\end{itemize}
\begin{itemize}
\item {Utilização:Fig.}
\end{itemize}
\begin{itemize}
\item {Utilização:Prov.}
\end{itemize}
\begin{itemize}
\item {Utilização:trasm.}
\end{itemize}
\begin{itemize}
\item {Grp. gram.:M.}
\end{itemize}
Que tem uma das côres do espectro solar, parecida á do céu sem nuvens.
Muito assustado, muito embaraçado: \textunderscore viu-se azul\textunderscore .
Bebedor.
A côr azul.
Cada uma das gradações desta côr: \textunderscore azul-ferrete, azul-claro...\textunderscore 
O céu.
\section{Azulado}
\begin{itemize}
\item {Grp. gram.:adj.}
\end{itemize}
Que tem côr azul.
Um tanto azul.
\section{Azulador}
\begin{itemize}
\item {Grp. gram.:m.}
\end{itemize}
Aquelle que azula.
\section{Azulão}
\begin{itemize}
\item {Grp. gram.:m.}
\end{itemize}
\begin{itemize}
\item {Proveniência:(De \textunderscore azul\textunderscore )}
\end{itemize}
Ave brasileira, de côr anilada.
Árvore tropical.
\section{Azular}
\begin{itemize}
\item {Grp. gram.:v. t.}
\end{itemize}
\begin{itemize}
\item {Grp. gram.:V. i.}
\end{itemize}
\begin{itemize}
\item {Utilização:Bras}
\end{itemize}
\begin{itemize}
\item {Utilização:pop.}
\end{itemize}
\begin{itemize}
\item {Utilização:minh}
\end{itemize}
\begin{itemize}
\item {Utilização:Gír.}
\end{itemize}
Dar côr azul a.
Voar pelo azul do céu.
Desapparecer: esgueirar-se.
Beber vinho.
\section{Azul-claro}
\begin{itemize}
\item {Grp. gram.:adj.}
\end{itemize}
Tirante a azul e branco.
\section{Azulego}
\begin{itemize}
\item {fónica:lê}
\end{itemize}
\begin{itemize}
\item {Grp. gram.:adj.}
\end{itemize}
\begin{itemize}
\item {Utilização:Bras. do S}
\end{itemize}
\begin{itemize}
\item {Proveniência:(De \textunderscore azul\textunderscore )}
\end{itemize}
Diz-se do cavallo sarapintado de preto e branco, parecendo, a distancia, azul.
\section{Azulejador}
\begin{itemize}
\item {Grp. gram.:m.}
\end{itemize}
Aquelle que azuleja.
\section{Azulejar}
\begin{itemize}
\item {Grp. gram.:v. t.}
\end{itemize}
Guarnecer de azulejos.
\section{Azulejar}
\begin{itemize}
\item {Grp. gram.:v. t.}
\end{itemize}
\begin{itemize}
\item {Grp. gram.:V. i.}
\end{itemize}
Tornar azul, azular.
Tornar-se azul; deixar vêr a sua côr azul:«\textunderscore o céu azulejava\textunderscore ».
\section{Azulejo}
\begin{itemize}
\item {Grp. gram.:m.}
\end{itemize}
Ladrilho vidrado, com desenhos de côres, em que predomina o azul.
(Provavelmente, do ár. \textunderscore az-zulaca\textunderscore , de \textunderscore zulaca\textunderscore , rad. que significa \textunderscore adherir\textunderscore )
\section{Azul-escuro}
\begin{itemize}
\item {Grp. gram.:adj.}
\end{itemize}
Tirante a azul e escuro.
\section{Azul-ferrete}
\begin{itemize}
\item {Grp. gram.:adj.}
\end{itemize}
\begin{itemize}
\item {Grp. gram.:M.}
\end{itemize}
Cuja côr azul é muito carregada, quási preta.
A côr azul, muito carregada, quási preta.
\section{Azulina}
\begin{itemize}
\item {Grp. gram.:f.}
\end{itemize}
\begin{itemize}
\item {Proveniência:(De \textunderscore azul\textunderscore )}
\end{itemize}
Matéria còrante azul, derivada do ácido phênico e da anilina.
\section{Azulino}
\begin{itemize}
\item {Grp. gram.:adj.}
\end{itemize}
\begin{itemize}
\item {Grp. gram.:M.}
\end{itemize}
\begin{itemize}
\item {Proveniência:(De \textunderscore azul\textunderscore )}
\end{itemize}
Que tem côr azul.
Espécie de tordo de Caiena.
\section{Azulmato}
\begin{itemize}
\item {Grp. gram.:m.}
\end{itemize}
\begin{itemize}
\item {Proveniência:(De \textunderscore azúlmico\textunderscore )}
\end{itemize}
Combinação do ácido azúlmico com uma base.
\section{Azúlmico}
\begin{itemize}
\item {Grp. gram.:adj.}
\end{itemize}
\begin{itemize}
\item {Proveniência:(De \textunderscore azoto\textunderscore  + \textunderscore úlmico\textunderscore )}
\end{itemize}
Diz-se de um ácido, que é uma substância cinzenta, úlmica e azotada.
\section{Azulmina}
\begin{itemize}
\item {Grp. gram.:f.}
\end{itemize}
\begin{itemize}
\item {Proveniência:(De \textunderscore azoto\textunderscore  + \textunderscore ulmina\textunderscore )}
\end{itemize}
Massa úlmica, inodora e negra.
\section{Azulóio}
\begin{itemize}
\item {Grp. gram.:adj.}
\end{itemize}
Tirante a azul e lóio.
\section{Azumbrar}
\begin{itemize}
\item {Grp. gram.:v. t.}
\end{itemize}
Dobrar, curvar, vergar.
(Infl. de \textunderscore zumbrir-se\textunderscore )
\section{Azurecho}
\begin{itemize}
\item {fónica:zurê}
\end{itemize}
\begin{itemize}
\item {Grp. gram.:m.}
\end{itemize}
\begin{itemize}
\item {Utilização:Ant.}
\end{itemize}
O mesmo que \textunderscore azulejo\textunderscore .
(Cp. fr. \textunderscore azur\textunderscore )
\section{Azurita}
\begin{itemize}
\item {Grp. gram.:f.}
\end{itemize}
\begin{itemize}
\item {Proveniência:(Fr. \textunderscore azurite\textunderscore )}
\end{itemize}
Carbonato de cobre, de côr azul.
\section{Azurracha}
\begin{itemize}
\item {Grp. gram.:f.}
\end{itemize}
\begin{itemize}
\item {Utilização:Ant.}
\end{itemize}
\begin{itemize}
\item {Proveniência:(Do ár. \textunderscore az-zallaje\textunderscore )}
\end{itemize}
Barco de um remo, que se usava no Doiro.
\section{Azurrar}
\begin{itemize}
\item {Grp. gram.:v. i.}
\end{itemize}
(V. \textunderscore zurrar\textunderscore ^1)
\section{Ázygos}
\begin{itemize}
\item {Grp. gram.:m.}
\end{itemize}
\begin{itemize}
\item {Proveniência:(Gr. \textunderscore azugos\textunderscore )}
\end{itemize}
Veia, que se estende da columna lombar até á veia cava superior.
Nome de outra veia, que sobe pelo lado esquerdo da columna vertebral, terminando na outra veia do mesmo nome.
\section{Azymita}
\begin{itemize}
\item {Grp. gram.:m.}
\end{itemize}
\begin{itemize}
\item {Proveniência:(De \textunderscore ázymo\textunderscore )}
\end{itemize}
Aquelle que se serve de pão ázymo.
\section{Ázymo}
\begin{itemize}
\item {Grp. gram.:adj.}
\end{itemize}
\begin{itemize}
\item {Grp. gram.:M.}
\end{itemize}
\begin{itemize}
\item {Proveniência:(Gr. \textunderscore azumos\textunderscore )}
\end{itemize}
\end{document}