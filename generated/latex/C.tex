
\begin{itemize}
\item {Proveniência: }
\end{itemize}\documentclass{article}
\usepackage[portuguese]{babel}
\title{C}
\begin{document}
Relativo a Byzâncio ou ao Baixo-Imperio.
Súbtil e fútil, como as questões theológicas da corte de Byzâncio.
Estilo ou arte, que se cultivou no Baixo-Império.
Habitante de Byzâncio.
\section{Çaboga}
\begin{itemize}
\item {Grp. gram.:f.}
\end{itemize}
\begin{itemize}
\item {Proveniência:(Do ár. \textunderscore çaboga\textunderscore )}
\end{itemize}
O mesmo que \textunderscore sàvelha\textunderscore .
\section{Çacaí}
\begin{itemize}
\item {Grp. gram.:m.}
\end{itemize}
\begin{itemize}
\item {Utilização:Bras. do N}
\end{itemize}
\begin{itemize}
\item {Proveniência:(Do guar. \textunderscore içacaí\textunderscore )}
\end{itemize}
Graveto.
Galho sêco de árvore.
Accendalha.
\section{Çacre}
\begin{itemize}
\item {Grp. gram.:m.}
\end{itemize}
\begin{itemize}
\item {Proveniência:(Do ár. \textunderscore çaqre\textunderscore )}
\end{itemize}
Espécie de falcão.
Antigo e grande canhão.
\section{Çáfara}
\begin{itemize}
\item {Grp. gram.:f.}
\end{itemize}
\begin{itemize}
\item {Proveniência:(De \textunderscore çáfaro\textunderscore )}
\end{itemize}
Terreno sáfaro; penhasco.
\section{Çáfaro}
\begin{itemize}
\item {Grp. gram.:adj.}
\end{itemize}
\begin{itemize}
\item {Utilização:Des.}
\end{itemize}
\begin{itemize}
\item {Proveniência:(Do ár. \textunderscore çahra\textunderscore , deserto)}
\end{itemize}
Inculto, agreste.
Estéril.
Bravo.
Estranho, alheio, distante.
Mal morigerado. Cf. Sousa, \textunderscore Vida do Arceb.\textunderscore , 121.
\section{Çáfeo}
\begin{itemize}
\item {Grp. gram.:adj.}
\end{itemize}
\begin{itemize}
\item {Utilização:Ant.}
\end{itemize}
Reles, desprezível, o mesmo que \textunderscore sáfio\textunderscore :«\textunderscore ...se não fosse casar ante co mais çáfeo bargante que come pão e cebola.\textunderscore »G. Vicente, \textunderscore Inês Pereira\textunderscore .
\section{Çafra}
\begin{itemize}
\item {Grp. gram.:f.}
\end{itemize}
\begin{itemize}
\item {Proveniência:(Do ár. \textunderscore çafr\textunderscore )}
\end{itemize}
Pó de um ácido de cobalto, próprio para a fabricação do vidro azul.
\section{Çafranina}
\begin{itemize}
\item {Grp. gram.:f.}
\end{itemize}
Um dos productos da hulha.
(Cp. \textunderscore açafrão\textunderscore )
\section{Çagu}
\begin{itemize}
\item {Grp. gram.:m.}
\end{itemize}
Substância amylácea, extrahida da parte central das hastes de algumas palmeiras.
Espécie de licor, destillado pelos ramos de palmeira, e usado na Índia.
Substância farinácea, extrahida de algumas plantas e de que fazem pão os Japoneses.
(Mal. \textunderscore sagu\textunderscore )
\section{Çaju}
\begin{itemize}
\item {Grp. gram.:m.}
\end{itemize}
\begin{itemize}
\item {Proveniência:(Do guar. \textunderscore çay-guazu\textunderscore )}
\end{itemize}
Pequeno macaco do Brasil.
\section{Çalama}
\begin{itemize}
\item {Grp. gram.:f.}
\end{itemize}
O mesmo que \textunderscore çalamaleque\textunderscore . Cf. Dom. Vieira, \textunderscore Diccion.\textunderscore 
\section{Çarça-parrilha}
\begin{itemize}
\item {Grp. gram.:f.}
\end{itemize}
\begin{itemize}
\item {Proveniência:(Do cast. \textunderscore zarza\textunderscore  + \textunderscore Parillo\textunderscore , n. p.)}
\end{itemize}
Planta, de origem americana, cuja raíz é depurativa e sudorífica.
Planta indígena, também conhecida por \textunderscore legação\textunderscore .
O mesmo que \textunderscore japecanga\textunderscore .
\section{Çamarra}
\begin{itemize}
\item {Grp. gram.:f.}
\end{itemize}
\begin{itemize}
\item {Utilização:Prov.}
\end{itemize}
\begin{itemize}
\item {Utilização:trasm.}
\end{itemize}
\begin{itemize}
\item {Utilização:Prov.}
\end{itemize}
\begin{itemize}
\item {Utilização:minh.}
\end{itemize}
\begin{itemize}
\item {Grp. gram.:M.}
\end{itemize}
\begin{itemize}
\item {Utilização:Deprec.}
\end{itemize}
Chimarra.
Vestuário antigo e rústico de pelles de ovelha.
Pello de ovelha ou carneiro, em quanto conserva a lan.
Traje ecclesiástico, chimarra.
Homem carcunda; a giba do carcunda.
Costado, costas: \textunderscore foi-lhe á çamarra\textunderscore .
Padre.
(Cast. \textunderscore zamarra\textunderscore )
\section{Çamarrão}
\begin{itemize}
\item {Grp. gram.:m.}
\end{itemize}
\begin{itemize}
\item {Utilização:Prov.}
\end{itemize}
\begin{itemize}
\item {Utilização:beir.}
\end{itemize}
\begin{itemize}
\item {Utilização:T. de Turquel}
\end{itemize}
Grande samarra.
Mulhér pública, coiro.
Homem muito gordo.
\section{Çamarreiro}
\begin{itemize}
\item {Grp. gram.:m.}
\end{itemize}
\begin{itemize}
\item {Proveniência:(De \textunderscore çamarra\textunderscore )}
\end{itemize}
Negociante de pelles de ovelha e carneiro.
\section{Çamarrinho}
\begin{itemize}
\item {Grp. gram.:m.}
\end{itemize}
Variedade de uva preta.
\section{Çamarro}
\begin{itemize}
\item {Grp. gram.:m.}
\end{itemize}
O mesmo que \textunderscore çamarra\textunderscore . Cf. G. Vicente, I.
\section{Çambuco}
\begin{itemize}
\item {Grp. gram.:m.}
\end{itemize}
Pequena embarcação indiana.
\section{Çanca}
\begin{itemize}
\item {Grp. gram.:f.}
\end{itemize}
\begin{itemize}
\item {Utilização:Prov.}
\end{itemize}
\begin{itemize}
\item {Utilização:trasm.}
\end{itemize}
\begin{itemize}
\item {Utilização:Náut.}
\end{itemize}
Cimalha convexa, que liga uma parede a um tecto.
Parte do telhado, assente sôbre a espessura da parede.
O mesmo que \textunderscore chanca\textunderscore .
Parte do corrimão, que sai fóra do talabardão.
(Cast. \textunderscore zanca\textunderscore )
\section{Çancadilha}
\begin{itemize}
\item {Grp. gram.:f.}
\end{itemize}
\begin{itemize}
\item {Utilização:Prov.}
\end{itemize}
\begin{itemize}
\item {Utilização:trasm.}
\end{itemize}
Cambapé.
Cunha, para calçar pontões.
Acaso; bambúrrio.
(Cast. \textunderscore zancadilla\textunderscore )
\section{Çancarrão}
\begin{itemize}
\item {Grp. gram.:m.}
\end{itemize}
\begin{itemize}
\item {Grp. gram.:Adj.}
\end{itemize}
Sanco grande.
Desajeitado; estrambótico.
Feio.
Ignorante, lerdo.
(Cast. \textunderscore zancarrón\textunderscore )
\section{Çanga}
\begin{itemize}
\item {Grp. gram.:f.}
\end{itemize}
\begin{itemize}
\item {Utilização:Bras. do S}
\end{itemize}
\begin{itemize}
\item {Proveniência:(Do cast. \textunderscore zanja\textunderscore )}
\end{itemize}
Escavação funda, produzida num terreno pela chuva ou por correntes subterrâneas.
\section{Çanja}
\begin{itemize}
\item {Grp. gram.:f.}
\end{itemize}
Abertura, feita para escoamento de água.
Sargeta, valeta.
Rêgo entre os bacellos. Cp. \textunderscore sanga\textunderscore ^2.
(Cast. \textunderscore zanja\textunderscore )
\section{Çapa}
\begin{itemize}
\item {Grp. gram.:f.}
\end{itemize}
\begin{itemize}
\item {Utilização:Prov.}
\end{itemize}
\begin{itemize}
\item {Utilização:trasm.}
\end{itemize}
\begin{itemize}
\item {Utilização:Fig.}
\end{itemize}
Pá, com que se ergue a terra que se escavou.
Abertura de fossos, trincheiras e galerias subterrâneas etc., geralmente para se accommeter uma praça ao abrigo dos sitiados.
Trabalho de sapador.
Alluvião.
Trabalho occulto, ardiloso.
Ardil.
(Cp. cast. \textunderscore zapa\textunderscore  e o b. lat. \textunderscore zappa\textunderscore )
\section{Çapar}
\begin{itemize}
\item {Grp. gram.:v. i.}
\end{itemize}
Trabalhar com a sapa.
Executar trabalhos de sapa.
\section{Çapata}
\begin{itemize}
\item {Grp. gram.:f.}
\end{itemize}
\begin{itemize}
\item {Utilização:Náut.}
\end{itemize}
Chinela de coiro.
Peça de madeira sôbre um pilar, para reforçar ou equilibrar a trave que assenta nella.
Pequena bigota, com furo no meio, e em fórma de sapato.
Poleame, que se firma no chicote dos cabrestos, estais, etc.
Rodela de camurça, na chave dos instrumentos músicos.
Calço de pedra ou supplemento saliente á base de uma parede, para a fortificar.
O mesmo que \textunderscore berma\textunderscore .
(Cast. \textunderscore zapata\textunderscore )
\section{Çapatada}
\begin{itemize}
\item {Grp. gram.:f.}
\end{itemize}
\begin{itemize}
\item {Utilização:Pop.}
\end{itemize}
Pancada com o sapato.
Pancada, que o gato dá com a pata.
\section{Çapatadinha}
\begin{itemize}
\item {Grp. gram.:f.}
\end{itemize}
Espécie de jôgo popular.
\section{Çapataria}
\begin{itemize}
\item {Grp. gram.:f.}
\end{itemize}
Arte ou estabelecimento de sapateiro.
Arruamento de sapateiros.
\section{Çapateada}
\begin{itemize}
\item {Grp. gram.:f.}
\end{itemize}
Acto ou effeito de çapatear.
\section{Çapateado}
\begin{itemize}
\item {Grp. gram.:m.}
\end{itemize}
\begin{itemize}
\item {Proveniência:(De \textunderscore çapatear\textunderscore )}
\end{itemize}
Sapateada.
Dança popular, em que se faz grande ruído com os tacões do calçado.
\section{Çapatear}
\begin{itemize}
\item {Grp. gram.:v. i.}
\end{itemize}
\begin{itemize}
\item {Grp. gram.:V. t.}
\end{itemize}
\begin{itemize}
\item {Proveniência:(De \textunderscore sapato\textunderscore )}
\end{itemize}
Bater no chão com o salto do calçado.
Executar (uma dança), fazendo grande ruído com o calçado ou só com os saltos do calçado.
\section{Çapateia}
\begin{itemize}
\item {Grp. gram.:f.}
\end{itemize}
\begin{itemize}
\item {Proveniência:(De \textunderscore çapatear\textunderscore )}
\end{itemize}
Dança popular dos Açores.
\section{Çapateira}
\begin{itemize}
\item {Grp. gram.:f.}
\end{itemize}
Mulhér de sapateiro.
Mulhér, que faz sapatos.
Designação genérica do várias plantas melastomáceas.
Nome de alguns crustáceos decápodes.
\section{Çapateiral}
\begin{itemize}
\item {Grp. gram.:adj.}
\end{itemize}
\begin{itemize}
\item {Utilização:Chul.}
\end{itemize}
Próprio de sapateiro.
Que tem modos de sapateiro.
\section{Çapateiro}
\begin{itemize}
\item {Grp. gram.:m.}
\end{itemize}
\begin{itemize}
\item {Utilização:T. de Penafiel}
\end{itemize}
Aquelle que faz sapatos ou trabalha em calçado.
Vendedor de calçado.
Insecto orthóptero, também chamado \textunderscore fèdevelha\textunderscore .
\section{Çapateta}
\begin{itemize}
\item {fónica:tê}
\end{itemize}
\begin{itemize}
\item {Grp. gram.:f.}
\end{itemize}
\begin{itemize}
\item {Proveniência:(De \textunderscore sapata\textunderscore )}
\end{itemize}
Chinela.
Ruído, produzido pelos tacões quando se anda.
\section{Çapatilha}
\begin{itemize}
\item {Grp. gram.:f.}
\end{itemize}
\begin{itemize}
\item {Proveniência:(De \textunderscore sapata\textunderscore )}
\end{itemize}
Sapata dos instrumentos músicos.
*\textunderscore Chapel.\textunderscore 
Peça de ferro, com que os fulistas recalcam os chapéus, para dar unidade e consistência ao pêlo.
\section{Çapatilho}
\begin{itemize}
\item {Grp. gram.:m.}
\end{itemize}
\begin{itemize}
\item {Proveniência:(De \textunderscore sapato\textunderscore )}
\end{itemize}
Arco de ferro, canelado exteriormente, que se firma nos chicotes dos cabos náuticos, etc.
Primeira fôlha sêca, que se tira da cana do açúcar, quando esta se limpa.
\section{Çapatinho}
\begin{itemize}
\item {Grp. gram.:m.}
\end{itemize}
Sapato pequeno e delicado.
Espécie de jôgo popular.
\section{Çapatola}
\begin{itemize}
\item {Grp. gram.:f.}
\end{itemize}
\begin{itemize}
\item {Grp. gram.:M.}
\end{itemize}
Sapato grande e mal feito.
O mesmo que \textunderscore remendão\textunderscore .
\section{Çapatorra}
\begin{itemize}
\item {fónica:tô}
\end{itemize}
\begin{itemize}
\item {Grp. gram.:f.}
\end{itemize}
O mesmo que \textunderscore çapatorro\textunderscore .
\section{Çapatorro}
\begin{itemize}
\item {fónica:tô}
\end{itemize}
\begin{itemize}
\item {Grp. gram.:m.}
\end{itemize}
Sapato grosseiro e malfeito.
\section{Çapatrancas}
\begin{itemize}
\item {Grp. gram.:f. pl.}
\end{itemize}
\begin{itemize}
\item {Utilização:Bras}
\end{itemize}
Sapatos grossos, sapatorros.
\section{Çape!}
\begin{itemize}
\item {Grp. gram.:interj.}
\end{itemize}
(usada para afugentar os gatos)
\section{Çarabanda}
\begin{itemize}
\item {Grp. gram.:f.}
\end{itemize}
\begin{itemize}
\item {Utilização:Pop.}
\end{itemize}
Dança antiga, popular e desenvolta.
Censura, reprehensão.
(Cast. \textunderscore zarabanda\textunderscore )
\section{Çarabandear}
\begin{itemize}
\item {Grp. gram.:v. i.}
\end{itemize}
\begin{itemize}
\item {Grp. gram.:V. i.}
\end{itemize}
Dançar a çarabanda.
Dançar.
\section{Çaragoça}
\begin{itemize}
\item {fónica:gô}
\end{itemize}
\begin{itemize}
\item {Grp. gram.:f.}
\end{itemize}
\begin{itemize}
\item {Proveniência:(De \textunderscore Zaragoza\textunderscore , n. p.)}
\end{itemize}
Tecido grosso de lan escura.
O mesmo que \textunderscore mandrião\textunderscore , ave.
\section{Çaragoçana}
\begin{itemize}
\item {Grp. gram.:f.}
\end{itemize}
\begin{itemize}
\item {Proveniência:(De \textunderscore çaragoçano\textunderscore )}
\end{itemize}
Espécie de ameixa comprida, escura e saborosa.
\section{Çaragoçano}
\begin{itemize}
\item {Grp. gram.:adj.}
\end{itemize}
\begin{itemize}
\item {Grp. gram.:M.}
\end{itemize}
Relativo a Saragoça.
Habitante de Saragoça.
\section{Çarapatel}
\begin{itemize}
\item {Grp. gram.:m.}
\end{itemize}
\begin{itemize}
\item {Utilização:Prov.}
\end{itemize}
\begin{itemize}
\item {Utilização:trasm.}
\end{itemize}
Iguaria, preparada com sangue, fígado, rim, bofe e coração de porco ou carneiro, com caldo.
Confusão, balbúrdia.
(Cast. \textunderscore zarapatel\textunderscore )
\section{Çarçal}
\begin{itemize}
\item {Grp. gram.:m.}
\end{itemize}
Silvado.
Lugar, onde crescem sarças.
\section{Çarçoso}
\begin{itemize}
\item {Grp. gram.:adj.}
\end{itemize}
Que tem çarças, que tem espinhos; que produz sarças.
\section{Çarpar}
\begin{itemize}
\item {Grp. gram.:v. t.}
\end{itemize}
\begin{itemize}
\item {Grp. gram.:V. i.}
\end{itemize}
Erguer (uma âncora).
Levantar ferro, navegar.
(Cast. \textunderscore zarpar\textunderscore )
\section{Celerado}
\begin{itemize}
\item {Grp. gram.:adj.}
\end{itemize}
\begin{itemize}
\item {Grp. gram.:M.}
\end{itemize}
\begin{itemize}
\item {Proveniência:(Lat. \textunderscore sceleratus\textunderscore )}
\end{itemize}
Que praticou grande crime.
Que é capaz de grandes crimes.
Perverso.
Que revela grande perversidade.
Indivíduo celerado.
\section{Cenário}
\begin{itemize}
\item {Grp. gram.:m.}
\end{itemize}
\begin{itemize}
\item {Proveniência:(Lat. \textunderscore scenarium\textunderscore )}
\end{itemize}
Decoração teatral.
Conjunto de bastidores e vistas, apropriadas aos factos que se representam.
\section{Cenedesmo}
\begin{itemize}
\item {Grp. gram.:m.}
\end{itemize}
\begin{itemize}
\item {Proveniência:(Do gr. \textunderscore skene\textunderscore  + \textunderscore desmos\textunderscore )}
\end{itemize}
Gênero de algas microscópicas.
\section{Cênico}
\begin{itemize}
\item {Grp. gram.:adj.}
\end{itemize}
Relativo á cena; teatral.
\section{Cenografia}
\begin{itemize}
\item {Grp. gram.:f.}
\end{itemize}
\begin{itemize}
\item {Utilização:Restrict.}
\end{itemize}
\begin{itemize}
\item {Proveniência:(Lat. \textunderscore scenographia\textunderscore )}
\end{itemize}
Arte de desenhar os lugares, os edifícios, etc., alargando-os ou estreitando-os, segundo as regras da perspectiva.
Arte de pintar as decorações de um teatro.
Conjunto dos objectos representados.
\section{Cenograficamente}
\begin{itemize}
\item {Grp. gram.:adv.}
\end{itemize}
De modo cenográfico, ou segundo as regras da cenografia.
\section{Cenográfico}
\begin{itemize}
\item {Grp. gram.:adj.}
\end{itemize}
Relativo á cenografia.
\section{Cenógrafo}
\begin{itemize}
\item {Grp. gram.:m.}
\end{itemize}
\begin{itemize}
\item {Proveniência:(Do gr. \textunderscore skene\textunderscore  + \textunderscore graphein\textunderscore )}
\end{itemize}
Aquele que pratíca a cenografia.
Aquele que pinta o cenario.
\section{Cenopégia}
\begin{itemize}
\item {Grp. gram.:f.}
\end{itemize}
\begin{itemize}
\item {Proveniência:(Lat. \textunderscore scenopegia\textunderscore )}
\end{itemize}
A festa dos tabernáculos, entre os Judeus, que com ela celebravam a sua estada de quarenta anos no deserto.
\section{Cepa}
\begin{itemize}
\item {fónica:cê}
\end{itemize}
\begin{itemize}
\item {Grp. gram.:f.}
\end{itemize}
\begin{itemize}
\item {Proveniência:(De \textunderscore cepo\textunderscore )}
\end{itemize}
Tronco de videira.
Parte inferior das árvores, incluindo as raízes, de que se faz carvão.
\textunderscore Estar sempre na cepa torta\textunderscore , ou \textunderscore não passar da cepa torta\textunderscore , não melhorar de posição, não progredir, não apprender.
\section{Cepa}
\begin{itemize}
\item {Grp. gram.:f.}
\end{itemize}
\begin{itemize}
\item {Proveniência:(Do gr. \textunderscore skepe\textunderscore )}
\end{itemize}
Gênero de árvores indianas.
\section{Cepticamente}
\begin{itemize}
\item {Grp. gram.:adv.}
\end{itemize}
De modo céptico; com cepticismo; de modo pirrónico.
\section{Céptico}
\begin{itemize}
\item {Grp. gram.:adj.}
\end{itemize}
\begin{itemize}
\item {Grp. gram.:M.}
\end{itemize}
\begin{itemize}
\item {Proveniência:(Gr. \textunderscore skeptikos\textunderscore )}
\end{itemize}
Diz-se dos filósofos, cujo dogma principal era duvidar de tudo; descrente.
Sectário do cepticismo.
Indivíduo descrente ou que duvída de tudo.
\section{Ceptrígero}
\begin{itemize}
\item {Grp. gram.:adj.}
\end{itemize}
\begin{itemize}
\item {Proveniência:(Lat. \textunderscore sceptriger\textunderscore )}
\end{itemize}
Que usa ceptro.
\section{Ciascopia}
\begin{itemize}
\item {Grp. gram.:f.}
\end{itemize}
\begin{itemize}
\item {Proveniência:(Do gr. \textunderscore skia\textunderscore  + \textunderscore skopein\textunderscore )}
\end{itemize}
Determinação da refracção do ôlho pelo estudo das sombras que se observam no campo pupilar com o auxílio do oftalmoscópio.
\section{Ciática}
\begin{itemize}
\item {Grp. gram.:f.}
\end{itemize}
\begin{itemize}
\item {Proveniência:(De \textunderscore sciático\textunderscore )}
\end{itemize}
Dôr ciática.
\section{Ciático}
\begin{itemize}
\item {Grp. gram.:adj.}
\end{itemize}
\begin{itemize}
\item {Proveniência:(Lat. \textunderscore sciaticus\textunderscore )}
\end{itemize}
Relativo ás ancas ou á parte superior da coxa.
Diz-se do nervo mais grosso de todo o organismo animal.
Diz-se da dôr, que se fixa nesse nervo, ocupando a parte posterior da coxa e da perna.
\section{Ciente}
\begin{itemize}
\item {Grp. gram.:adj.}
\end{itemize}
\begin{itemize}
\item {Proveniência:(Lat. \textunderscore sciens\textunderscore )}
\end{itemize}
Que tem conhecimento de alguma coisa; que sabe: \textunderscore ficou ciente da recomendação\textunderscore .
Que tem ciência
\section{Cientemente}
\begin{itemize}
\item {Grp. gram.:adv.}
\end{itemize}
\begin{itemize}
\item {Proveniência:(De \textunderscore ciente\textunderscore )}
\end{itemize}
Com ciência, com conhecimento.
Adrede, de caso pensado.
\section{Cientificamente}
\begin{itemize}
\item {Grp. gram.:adv.}
\end{itemize}
De modo cientifico.
Segundo os processos ou preceitos da ciência; com ciência.
\section{Cientificar}
\begin{itemize}
\item {Grp. gram.:v. t.}
\end{itemize}
\begin{itemize}
\item {Utilização:Neol.}
\end{itemize}
\begin{itemize}
\item {Proveniência:(Do lat. \textunderscore sciens\textunderscore  + \textunderscore facere\textunderscore )}
\end{itemize}
Tornar ciente.
\section{Científico}
\begin{itemize}
\item {Grp. gram.:adj.}
\end{itemize}
\begin{itemize}
\item {Proveniência:(Do lat. \textunderscore scientia\textunderscore  + \textunderscore facere\textunderscore )}
\end{itemize}
Relativo á ciência: \textunderscore progressos científicos\textunderscore .
Que mostra ciência.
\section{Cientista}
\begin{itemize}
\item {Grp. gram.:m.}
\end{itemize}
\begin{itemize}
\item {Utilização:Neol.}
\end{itemize}
\begin{itemize}
\item {Proveniência:(Do lat. \textunderscore scientia\textunderscore )}
\end{itemize}
Aquele que se ocupa de ciências ou de uma ciência.
\section{Cieropia}
\begin{itemize}
\item {Grp. gram.:f.}
\end{itemize}
\begin{itemize}
\item {Proveniência:(Do gr. \textunderscore skieros\textunderscore  + \textunderscore ops\textunderscore )}
\end{itemize}
Doença da vista, que faz vêr os objectos com uma côr mais carregada que a real.
\section{Cila}
\begin{itemize}
\item {Grp. gram.:f.}
\end{itemize}
\begin{itemize}
\item {Proveniência:(Lat. \textunderscore scilla\textunderscore )}
\end{itemize}
Gênero de plantas liliáceas.
Espécie de narciso, (\textunderscore pancratium guyanensis\textunderscore ).
\section{Cilítico}
\begin{itemize}
\item {Grp. gram.:adj.}
\end{itemize}
Que contém suco de cila ou que é feito com suco de cila.
\section{Cilitina}
\begin{itemize}
\item {Grp. gram.:f.}
\end{itemize}
Princípio acre, que se encontra no suco da cila.
\section{Cilito}
\begin{itemize}
\item {Grp. gram.:m.}
\end{itemize}
Vinho de cila, preparado nas boticas.
\section{Cinco}
\begin{itemize}
\item {Grp. gram.:adj.}
\end{itemize}
\begin{itemize}
\item {Grp. gram.:M.}
\end{itemize}
\begin{itemize}
\item {Proveniência:(Do lat. \textunderscore quinque\textunderscore )}
\end{itemize}
Diz-se do número cardinal, formado de quatro e mais um; quinto.
O algarismo representativo dêsse número.
Carta de jogar ou peça do dominó, que tem cinco pontos.
Aquelle ou aquillo que numa série de cinco occupa o último lugar.
\section{Cinco}
\begin{itemize}
\item {Grp. gram.:m.}
\end{itemize}
\begin{itemize}
\item {Proveniência:(Do gr. \textunderscore skinkos\textunderscore )}
\end{itemize}
Gênero do reptis sáurios.
\section{Cindir}
\begin{itemize}
\item {Grp. gram.:v. t.}
\end{itemize}
O mesmo que \textunderscore escindir\textunderscore .
\section{Cintila}
\begin{itemize}
\item {Grp. gram.:f.}
\end{itemize}
\begin{itemize}
\item {Proveniência:(Lat. \textunderscore scintilla\textunderscore )}
\end{itemize}
O mesmo que \textunderscore centelha\textunderscore :«\textunderscore ...uma cintila resistente de instincto feminil.\textunderscore »Camillo, \textunderscore Brasileira\textunderscore , 388.
\section{Cintilação}
\begin{itemize}
\item {Grp. gram.:f.}
\end{itemize}
\begin{itemize}
\item {Proveniência:(Do lat. \textunderscore scintillatio\textunderscore )}
\end{itemize}
Acto ou efeito de cintilar; brilho intenso.
\section{Cintilante}
\begin{itemize}
\item {Grp. gram.:adj.}
\end{itemize}
\begin{itemize}
\item {Proveniência:(Lat. \textunderscore scintillans\textunderscore )}
\end{itemize}
Que cintila; muito brilhante; deslumbrante.
\section{Cintilar}
\begin{itemize}
\item {Grp. gram.:v. i.}
\end{itemize}
\begin{itemize}
\item {Grp. gram.:V. t.}
\end{itemize}
\begin{itemize}
\item {Proveniência:(Lat. \textunderscore scintillare\textunderscore )}
\end{itemize}
Têr brilho, semelhante ao das centelhas.
Brilhar, tremendo; tremeluzir: \textunderscore cintilam estrêlas\textunderscore .
Faiscar.
Brilhar muito.
Resplandecer.
Irradiar, difundir luminosamente:«\textunderscore está cintilando diferentes brilhos.\textunderscore »\textunderscore Luz e Calor\textunderscore , 327.«\textunderscore Cintilaram áscuas de júbilo os olhos da menina.\textunderscore »Camillo, \textunderscore Caveira\textunderscore , 162.
\section{Cintilómetro}
\begin{itemize}
\item {Grp. gram.:m.}
\end{itemize}
\begin{itemize}
\item {Proveniência:(Do lat. \textunderscore scintilla\textunderscore  + gr. \textunderscore metron\textunderscore )}
\end{itemize}
Instrumento, inventado por Arago, para apreciar a intensidade da cintilação dos astros.
\section{Ciografia}
\begin{itemize}
\item {Grp. gram.:f.}
\end{itemize}
\begin{itemize}
\item {Proveniência:(Lat. \textunderscore sciographia\textunderscore )}
\end{itemize}
Desenho de um edifício, que se representa cortado longitudinalmente ou transversalmente, para se lhe poder observar a disposição interior.
Arte de conhecer as horas pela sombra dos astros.
\section{Ciográfico}
\begin{itemize}
\item {Grp. gram.:adj.}
\end{itemize}
Relativo á ciografia.
\section{Ciógrafo}
\begin{itemize}
\item {Grp. gram.:m.}
\end{itemize}
\begin{itemize}
\item {Proveniência:(Do gr. \textunderscore skia\textunderscore  + \textunderscore graphein\textunderscore )}
\end{itemize}
Aquele que conhece ou pratica a ciografia.
\section{Ciótico}
\begin{itemize}
\item {Grp. gram.:adj.}
\end{itemize}
\begin{itemize}
\item {Proveniência:(Do gr. \textunderscore skia\textunderscore  + \textunderscore opteskein\textunderscore )}
\end{itemize}
Relativo á visão na sombra.
\section{Ciras}
\begin{itemize}
\item {Grp. gram.:f. pl.}
\end{itemize}
O mesmo que \textunderscore cirofórias\textunderscore .
\section{Círias}
\begin{itemize}
\item {Grp. gram.:f. pl.}
\end{itemize}
(V.cirofórias)
\section{Cirofórias}
\begin{itemize}
\item {Grp. gram.:f. pl.}
\end{itemize}
\begin{itemize}
\item {Proveniência:(Do gr. \textunderscore skiros\textunderscore  + \textunderscore phoros\textunderscore )}
\end{itemize}
Festas, que os Ateneienses celebrava em honra de Minerva, durante as quaes se enramavam cabanas, e em que os mancebos, nos jogos respectivos, sustinham nas mãos cepas carregadas de uvas.
\section{Cirofório}
\begin{itemize}
\item {Grp. gram.:m.}
\end{itemize}
Mês, em que os Atenienses celebravam as cirofórias e que era o duodécimo do ano ático.
\section{Ciros}
\begin{itemize}
\item {Grp. gram.:m. pl.}
\end{itemize}
\begin{itemize}
\item {Proveniência:(Lat. \textunderscore scyri\textunderscore )}
\end{itemize}
Tríbo combatente nas guerras góticas da Espanha. Cf. Herculano, \textunderscore Eurico\textunderscore , c. IV.
\section{Cisma}
\begin{itemize}
\item {Grp. gram.:m.}
\end{itemize}
\begin{itemize}
\item {Proveniência:(Do lat. \textunderscore schisma\textunderscore )}
\end{itemize}
Separação, que um indivíduo ou uma colectividade faz, de uma religião ou de indivíduos que obedecem a um chefe religioso, não reconhecido por aqueles.
Separação do povo judeu em dois reinos.
Separação de crenças políticas ou literárias.
\section{Cismar}
\begin{itemize}
\item {Grp. gram.:v. t.}
\end{itemize}
\begin{itemize}
\item {Grp. gram.:V. i.}
\end{itemize}
\begin{itemize}
\item {Grp. gram.:M.}
\end{itemize}
Pensar muito em.
Meditar, preoccupar-se.
Andar apreensivo.
Ideia fixa, cisma.
(Or. duvidosa. Relacionar-se-á com o cast. \textunderscore ensimismar-se\textunderscore ? Neste caso, deveríamos escrever \textunderscore sismar\textunderscore )
\section{Cismaticamente}
\begin{itemize}
\item {Grp. gram.:adv.}
\end{itemize}
\begin{itemize}
\item {Proveniência:(De \textunderscore cismático\textunderscore ^2)}
\end{itemize}
De modo cismático.
Com cisma^2; como quem devaneia.
Á maneira de quem é apreensivo.
\section{Cismático}
\begin{itemize}
\item {Grp. gram.:adj.}
\end{itemize}
\begin{itemize}
\item {Grp. gram.:M.}
\end{itemize}
\begin{itemize}
\item {Utilização:Ant.}
\end{itemize}
\begin{itemize}
\item {Proveniência:(Do lat. \textunderscore schismaticus\textunderscore )}
\end{itemize}
Que segue um cisma.
Relativo a cisma^2.
O mesmo que \textunderscore castelhano\textunderscore .
\section{Cismático}
\begin{itemize}
\item {Grp. gram.:adj.}
\end{itemize}
\begin{itemize}
\item {Proveniência:(De \textunderscore cismar\textunderscore )}
\end{itemize}
Que anda apreensivo.
Que medita, sem objecto determinado.
Que devaneia.
\section{Citas}
\begin{itemize}
\item {Grp. gram.:m. Pl.}
\end{itemize}
\begin{itemize}
\item {Proveniência:(Lat. \textunderscore Scythae\textunderscore )}
\end{itemize}
Designação genérica dos povos nómades do norte da Europa e da Ásia.
\section{Cítico}
\begin{itemize}
\item {Grp. gram.:adj.}
\end{itemize}
Relativo aos Citas.
\section{Citissa}
\begin{itemize}
\item {Grp. gram.:f.}
\end{itemize}
\begin{itemize}
\item {Proveniência:(Lat. \textunderscore scythissa\textunderscore )}
\end{itemize}
Mulhér, natural da Cítia.
\section{Çocata}
\begin{itemize}
\item {Grp. gram.:f.}
\end{itemize}
Ferro manipulado e considerado inutil, especialmente o que serviu em caminhos de ferro, e que se aproveita para ser refundido e entregue de novo ao commercio.
(Cast. \textunderscore zocata\textunderscore )
\section{Çóco}
\begin{itemize}
\item {Grp. gram.:m.}
\end{itemize}
\begin{itemize}
\item {Utilização:Ant.}
\end{itemize}
O mesmo que \textunderscore tamanco\textunderscore .
O mesmo que \textunderscore chapim\textunderscore ^1.
(Cp. cast. \textunderscore zueco\textunderscore )
\section{Çumarento}
\begin{itemize}
\item {Grp. gram.:adj.}
\end{itemize}
Que tem çumo ou muito çumo: \textunderscore laranja çumarenta\textunderscore .
\section{Çumoso}
\begin{itemize}
\item {Grp. gram.:adj.}
\end{itemize}
Que tem çumo; çumarento.
\section{Çuro}
\begin{itemize}
\item {Grp. gram.:adj.}
\end{itemize}
Que não tem rabo; derrabado: \textunderscore uma gallinha çura\textunderscore .
(Cast. \textunderscore zuro\textunderscore )
\section{Çurra}
\begin{itemize}
\item {Grp. gram.:f.}
\end{itemize}
\begin{itemize}
\item {Utilização:Pop.}
\end{itemize}
\begin{itemize}
\item {Proveniência:(De \textunderscore surrar\textunderscore )}
\end{itemize}
Pancadaria; tunda.
\section{Çurrão}
\begin{itemize}
\item {Grp. gram.:m.}
\end{itemize}
\begin{itemize}
\item {Utilização:Pleb.}
\end{itemize}
Bolsa ou saco de coiro, destinado especialmente a farnel de pastores.
Fato sujo e gasto.
O mesmo que \textunderscore prostituta\textunderscore  reles.
(Cast. \textunderscore zurrón\textunderscore )
\section{Çurro}
\begin{itemize}
\item {Grp. gram.:m.}
\end{itemize}
\begin{itemize}
\item {Utilização:Prov.}
\end{itemize}
\begin{itemize}
\item {Utilização:trasm.}
\end{itemize}
Sujidade no rosto, nas mãos ou nos pés, especialmente a que provém do suór.
O mesmo que \textunderscore café\textunderscore .
(Cp. \textunderscore churdo\textunderscore  e \textunderscore surrar\textunderscore )
\section{Cianuria}
\begin{itemize}
\item {Grp. gram.:f.}
\end{itemize}
\begin{itemize}
\item {Utilização:Med.}
\end{itemize}
\begin{itemize}
\item {Proveniência:(Do gr. \textunderscore kuanos\textunderscore  + \textunderscore ouron\textunderscore )}
\end{itemize}
Emissão de urinas azuladas.
\section{Ciclocéfalo}
\begin{itemize}
\item {Grp. gram.:m.}
\end{itemize}
\begin{itemize}
\item {Utilização:Terat.}
\end{itemize}
\begin{itemize}
\item {Grp. gram.:Pl.}
\end{itemize}
\begin{itemize}
\item {Utilização:Zool.}
\end{itemize}
\begin{itemize}
\item {Proveniência:(Do gr. \textunderscore kuklos\textunderscore  + \textunderscore kephale\textunderscore )}
\end{itemize}
Monstro de uma só órbita, dois olhos contíguos e nariz atrophiado.
Tríbo de insectos pentâmeros, no systema de Cuvier.
\section{Cinobasto}
\begin{itemize}
\item {Grp. gram.:m.}
\end{itemize}
Fruto ácido, adstringente, da roseira canina.
\section{Cinogenético}
\begin{itemize}
\item {Grp. gram.:adj.}
\end{itemize}
Relativo á cinogênese.
\section{Cinogenia}
\begin{itemize}
\item {Grp. gram.:f.}
\end{itemize}
O mesmo que \textunderscore cenogênese\textunderscore .
\section{Cinogênico}
\begin{itemize}
\item {Grp. gram.:adj.}
\end{itemize}
Relativo á cinogenia; o mesmo que \textunderscore cinogenético\textunderscore .
\section{Cirtómetro}
\begin{itemize}
\item {Grp. gram.:m.}
\end{itemize}
\begin{itemize}
\item {Utilização:Med.}
\end{itemize}
\begin{itemize}
\item {Proveniência:(Do gr. \textunderscore kurtos\textunderscore  + \textunderscore metron\textunderscore )}
\end{itemize}
Instrumento, para medir as saliências mórbidas do corpo, essencialmente as saliências do thórax.
\section{Cistectasia}
\begin{itemize}
\item {Grp. gram.:f.}
\end{itemize}
\begin{itemize}
\item {Utilização:Med.}
\end{itemize}
\begin{itemize}
\item {Proveniência:(Do gr. \textunderscore kustis\textunderscore  + \textunderscore ektasis\textunderscore )}
\end{itemize}
Dilatação da bexiga.
\section{Cistectomia}
\begin{itemize}
\item {Grp. gram.:f.}
\end{itemize}
\begin{itemize}
\item {Utilização:Med.}
\end{itemize}
\begin{itemize}
\item {Proveniência:(Do gr. \textunderscore kustis\textunderscore  + \textunderscore ektome\textunderscore )}
\end{itemize}
Ablação de parte ou da totalidade da bexiga.
\section{Cistencéfalo}
\begin{itemize}
\item {Grp. gram.:m.}
\end{itemize}
\begin{itemize}
\item {Utilização:Terat.}
\end{itemize}
\begin{itemize}
\item {Proveniência:(Do gr. \textunderscore kustis\textunderscore  + \textunderscore enkephalon\textunderscore )}
\end{itemize}
Monstro, cujo encéfalo é substituido por uma vesícula.
\section{Cistodinia}
\begin{itemize}
\item {Grp. gram.:f.}
\end{itemize}
\begin{itemize}
\item {Utilização:Med.}
\end{itemize}
\begin{itemize}
\item {Proveniência:(Do gr. \textunderscore kustis\textunderscore  + \textunderscore odune\textunderscore )}
\end{itemize}
Dôr reumática na túnica muscular da bexiga.
\section{Cistoemia}
\begin{itemize}
\item {fónica:to-e}
\end{itemize}
\begin{itemize}
\item {Grp. gram.:f.}
\end{itemize}
\begin{itemize}
\item {Utilização:Med.}
\end{itemize}
\begin{itemize}
\item {Proveniência:(Do gr. \textunderscore kustis\textunderscore  + \textunderscore haima\textunderscore )}
\end{itemize}
Affluxo do sangue para a bexiga.
\section{Cistopexia}
\begin{itemize}
\item {fónica:csi}
\end{itemize}
\begin{itemize}
\item {Grp. gram.:f.}
\end{itemize}
\begin{itemize}
\item {Utilização:Med.}
\end{itemize}
\begin{itemize}
\item {Proveniência:(Do gr. \textunderscore kustis\textunderscore  + \textunderscore pexis\textunderscore )}
\end{itemize}
Fixação da parede anterior da bexiga á parede abdominal, por cima da sínfise pubiana.
\section{Cistoplastia}
\begin{itemize}
\item {Grp. gram.:f.}
\end{itemize}
\begin{itemize}
\item {Utilização:Med.}
\end{itemize}
\begin{itemize}
\item {Proveniência:(Do gr. \textunderscore kustis\textunderscore  + \textunderscore plassein\textunderscore )}
\end{itemize}
Operação da fístula vesico-vaginal, por autoplastia.
\section{Cistoptose}
\begin{itemize}
\item {Grp. gram.:f.}
\end{itemize}
\begin{itemize}
\item {Utilização:Med.}
\end{itemize}
\begin{itemize}
\item {Proveniência:(Do gr. \textunderscore kustis\textunderscore  + \textunderscore ptosis\textunderscore )}
\end{itemize}
Prolapso da membrana interna da bexiga.
\section{Cistorragia}
\begin{itemize}
\item {Grp. gram.:f.}
\end{itemize}
\begin{itemize}
\item {Utilização:Med.}
\end{itemize}
\begin{itemize}
\item {Proveniência:(Do gr. \textunderscore kustis\textunderscore  + \textunderscore rhageins\textunderscore )}
\end{itemize}
Hemorragia vesical.
\section{Cistorrafia}
\begin{itemize}
\item {Grp. gram.:f.}
\end{itemize}
\begin{itemize}
\item {Utilização:Med.}
\end{itemize}
\begin{itemize}
\item {Proveniência:(Do gr. \textunderscore kustis\textunderscore  + \textunderscore raphe\textunderscore )}
\end{itemize}
Sutura da bexiga.
\section{Citogênese}
\begin{itemize}
\item {Grp. gram.:f.}
\end{itemize}
\begin{itemize}
\item {Utilização:Biol.}
\end{itemize}
\begin{itemize}
\item {Proveniência:(Do gr. \textunderscore kutos\textunderscore  + \textunderscore genesis\textunderscore )}
\end{itemize}
Formação das células.
\section{Cyanuria}
\begin{itemize}
\item {Grp. gram.:f.}
\end{itemize}
\begin{itemize}
\item {Utilização:Med.}
\end{itemize}
\begin{itemize}
\item {Proveniência:(Do gr. \textunderscore kuanos\textunderscore  + \textunderscore ouron\textunderscore )}
\end{itemize}
Emissão de urinas azuladas.
\section{Cyclocéphalo}
\begin{itemize}
\item {Grp. gram.:m.}
\end{itemize}
\begin{itemize}
\item {Utilização:Terat.}
\end{itemize}
\begin{itemize}
\item {Grp. gram.:Pl.}
\end{itemize}
\begin{itemize}
\item {Utilização:Zool.}
\end{itemize}
\begin{itemize}
\item {Proveniência:(Do gr. \textunderscore kuklos\textunderscore  + \textunderscore kephale\textunderscore )}
\end{itemize}
Monstro de uma só órbita, dois olhos contíguos e nariz atrophiado.
Tríbo de insectos pentâmeros, no systema de Cuvier.
\section{Cynobasto}
\begin{itemize}
\item {Grp. gram.:m.}
\end{itemize}
Fruto ácido, adstringente, da roseira canina.
\section{Cynogenético}
\begin{itemize}
\item {Grp. gram.:adj.}
\end{itemize}
Relativo á cynogênese.
\section{Cynogenia}
\begin{itemize}
\item {Grp. gram.:f.}
\end{itemize}
O mesmo que \textunderscore cytogênese\textunderscore .
\section{Cynogênico}
\begin{itemize}
\item {Grp. gram.:adj.}
\end{itemize}
Relativo á cynogenia; o mesmo que \textunderscore cynogenético\textunderscore .
\section{Cyrtómetro}
\begin{itemize}
\item {Grp. gram.:m.}
\end{itemize}
\begin{itemize}
\item {Utilização:Med.}
\end{itemize}
\begin{itemize}
\item {Proveniência:(Do gr. \textunderscore kurtos\textunderscore  + \textunderscore metron\textunderscore )}
\end{itemize}
Instrumento, para medir as saliências mórbidas do corpo, essencialmente[**typo corrigido] as saliências do thórax.
\section{Cystectasia}
\begin{itemize}
\item {Grp. gram.:f.}
\end{itemize}
\begin{itemize}
\item {Utilização:Med.}
\end{itemize}
\begin{itemize}
\item {Proveniência:(Do gr. \textunderscore kustis\textunderscore  + \textunderscore ektasis\textunderscore )}
\end{itemize}
Dilatação da bexiga.
\section{Cystectomia}
\begin{itemize}
\item {Grp. gram.:f.}
\end{itemize}
\begin{itemize}
\item {Utilização:Med.}
\end{itemize}
\begin{itemize}
\item {Proveniência:(Do gr. \textunderscore kustis\textunderscore  + \textunderscore ektome\textunderscore )}
\end{itemize}
Ablação de parte ou da totalidade da bexiga.
\section{Cystencéphalo}
\begin{itemize}
\item {Grp. gram.:m.}
\end{itemize}
\begin{itemize}
\item {Utilização:Terat.}
\end{itemize}
\begin{itemize}
\item {Proveniência:(Do gr. \textunderscore kustis\textunderscore  + \textunderscore enkephalon\textunderscore )}
\end{itemize}
Monstro, cujo encéphalo é substituido por uma vesícula.
\section{Cystodynia}
\begin{itemize}
\item {Grp. gram.:f.}
\end{itemize}
\begin{itemize}
\item {Utilização:Med.}
\end{itemize}
\begin{itemize}
\item {Proveniência:(Do gr. \textunderscore kustis\textunderscore  + \textunderscore odune\textunderscore )}
\end{itemize}
Dôr rheumática na túnica muscular da bexiga.
\section{Cysthoemia}
\begin{itemize}
\item {Grp. gram.:f.}
\end{itemize}
\begin{itemize}
\item {Utilização:Med.}
\end{itemize}
\begin{itemize}
\item {Proveniência:(Do gr. \textunderscore kustis\textunderscore  + \textunderscore haima\textunderscore )}
\end{itemize}
Affluxo do sangue para a bexiga.
\section{Cystopexia}
\begin{itemize}
\item {Grp. gram.:f.}
\end{itemize}
\begin{itemize}
\item {Utilização:Med.}
\end{itemize}
\begin{itemize}
\item {Proveniência:(Do gr. \textunderscore kustis\textunderscore  + \textunderscore pexis\textunderscore )}
\end{itemize}
Fixação da parede anterior da bexiga á parede abdominal, por cima da sýmphise pubiana.
\section{Cystoplastia}
\begin{itemize}
\item {Grp. gram.:f.}
\end{itemize}
\begin{itemize}
\item {Utilização:Med.}
\end{itemize}
\begin{itemize}
\item {Proveniência:(Do gr. \textunderscore kustis\textunderscore  + \textunderscore plassein\textunderscore )}
\end{itemize}
Operação da fístula vesico-vaginal, por autoplastia.
\section{Cystoptose}
\begin{itemize}
\item {Grp. gram.:f.}
\end{itemize}
\begin{itemize}
\item {Utilização:Med.}
\end{itemize}
\begin{itemize}
\item {Proveniência:(Do gr. \textunderscore kustis\textunderscore  + \textunderscore ptosis\textunderscore )}
\end{itemize}
Prolapso da membrana interna da bexiga.
\section{Cystorrhagia}
\begin{itemize}
\item {Grp. gram.:f.}
\end{itemize}
\begin{itemize}
\item {Utilização:Med.}
\end{itemize}
\begin{itemize}
\item {Proveniência:(Do gr. \textunderscore kustis\textunderscore  + \textunderscore rhageins\textunderscore )}
\end{itemize}
Hemorrhagia vesical.
\section{Cystorrhaphia}
\begin{itemize}
\item {Grp. gram.:f.}
\end{itemize}
\begin{itemize}
\item {Utilização:Med.}
\end{itemize}
\begin{itemize}
\item {Proveniência:(Do gr. \textunderscore kustis\textunderscore  + \textunderscore raphe\textunderscore )}
\end{itemize}
Sutura da bexiga.
\section{Cytogênese}
\begin{itemize}
\item {Grp. gram.:f.}
\end{itemize}
\begin{itemize}
\item {Utilização:Biol.}
\end{itemize}
\begin{itemize}
\item {Proveniência:(Do gr. \textunderscore kutos\textunderscore  + \textunderscore genesis\textunderscore )}
\end{itemize}
Formação das céllulas.
\section{Càvaterra}
\begin{itemize}
\item {Utilização:Prov.}
\end{itemize}
\begin{itemize}
\item {Utilização:dur.}
\end{itemize}
O mesmo que \textunderscore toupeira\textunderscore .
\section{Caçaca}
\begin{itemize}
\item {Grp. gram.:f.}
\end{itemize}
\begin{itemize}
\item {Utilização:Ant.}
\end{itemize}
O mesmo que \textunderscore casaca\textunderscore ^1.
\section{Caçacão}
\begin{itemize}
\item {Grp. gram.:m.}
\end{itemize}
\begin{itemize}
\item {Utilização:Ant.}
\end{itemize}
O mesmo que \textunderscore casacão\textunderscore . Cf. B. Pereira, \textunderscore Prosódia\textunderscore , vb. \textunderscore braccae\textunderscore .
\section{Caça-rabo}
\begin{itemize}
\item {Grp. gram.:m.}
\end{itemize}
\begin{itemize}
\item {Utilização:Prov.}
\end{itemize}
O mesmo que \textunderscore saca-rabo\textunderscore  ou \textunderscore mangusto\textunderscore .
\section{Çacotorino}
\begin{itemize}
\item {Grp. gram.:m.}
\end{itemize}
Habitante de Çacotorá ou Socotorá. Cf. \textunderscore Rot. do Mar-Verm.\textunderscore , 16.
\section{Cacula}
\begin{itemize}
\item {Grp. gram.:m.}
\end{itemize}
\begin{itemize}
\item {Utilização:Pharm.}
\end{itemize}
Medicamento tónico, composto da cola e outras substâncias.
\section{Cadaverização}
\begin{itemize}
\item {Grp. gram.:f.}
\end{itemize}
Acto ou effeito de cadaverizar:«\textunderscore da cadaverisação e autólyse da medulla espinhal.\textunderscore »\textunderscore Rev. da Univ. de Coímbra\textunderscore , I, 67.
\section{Cadaverizar}
\begin{itemize}
\item {Grp. gram.:v. t.}
\end{itemize}
\begin{itemize}
\item {Utilização:Neol.}
\end{itemize}
Reduzir a cadáver.
Extinguir a acção vital de (uma parte do organismo).
\section{Cadmeu}
\begin{itemize}
\item {Grp. gram.:adj.}
\end{itemize}
\begin{itemize}
\item {Proveniência:(Lat. \textunderscore cadmeus\textunderscore )}
\end{itemize}
O mesmo ou melhór que \textunderscore cádmeo\textunderscore .
\section{Caduca}
\begin{itemize}
\item {Grp. gram.:f.}
\end{itemize}
\begin{itemize}
\item {Utilização:Physiol.}
\end{itemize}
\begin{itemize}
\item {Proveniência:(Do lat. \textunderscore cadere\textunderscore )}
\end{itemize}
Membrana, que envolve o ovo, formada á custa de muco uterino.
\section{Cacosmia}
\begin{itemize}
\item {Grp. gram.:f.}
\end{itemize}
\begin{itemize}
\item {Utilização:Med.}
\end{itemize}
\begin{itemize}
\item {Proveniência:(Do gr. \textunderscore kakos\textunderscore  + \textunderscore osme\textunderscore )}
\end{itemize}
Perversão do olfacto, que leva o doente a apreciar cheiros desagradáveis.
Percepção habitual de mau cheiro, por alucinação do olfacto.
\section{Caetano}
\begin{itemize}
\item {Grp. gram.:m.}
\end{itemize}
Frade de certa Ordem monástica: \textunderscore vamos á igreja dos caetanos\textunderscore .
\section{Cafuinho}
\begin{itemize}
\item {Grp. gram.:m.}
\end{itemize}
O mesmo que \textunderscore cafuínha\textunderscore .
\section{Caiota}
\begin{itemize}
\item {Grp. gram.:f.}
\end{itemize}
\begin{itemize}
\item {Utilização:Bot.}
\end{itemize}
O mesmo que \textunderscore chuchu\textunderscore . Cf. P. Coutinho, \textunderscore Flora\textunderscore , 599.
\section{Cajibá}
\begin{itemize}
\item {Grp. gram.:m.}
\end{itemize}
\begin{itemize}
\item {Utilização:Bras}
\end{itemize}
Piaçaba grossa e da melhór qualidade.
\section{Cajueiro}
\begin{itemize}
\item {Grp. gram.:m.}
\end{itemize}
Nome de algumas árvores, arbustos e plantas rasteiras, da fam. das terebintháceas, e procedentes da América do Sul.
\section{Calcífugo}
\begin{itemize}
\item {Grp. gram.:adj.}
\end{itemize}
\begin{itemize}
\item {Utilização:Bot.}
\end{itemize}
\begin{itemize}
\item {Proveniência:(Do lat. \textunderscore calx\textunderscore , \textunderscore calcis\textunderscore  + \textunderscore fugere\textunderscore )}
\end{itemize}
Diz-se das plantas, que se não dão bem nos terrenos calcários.
\section{Calcinhas}
\begin{itemize}
\item {Grp. gram.:f. pl.}
\end{itemize}
Calças curtas, usadas por mulheres, debaixo das saias.
\section{Calcoxisto}
\begin{itemize}
\item {Grp. gram.:m.}
\end{itemize}
\begin{itemize}
\item {Utilização:Miner.}
\end{itemize}
Espécie de xisto de calcário granuloso e crystallino, com mistura de mica e quartzo.
\section{Calembureiro}
\begin{itemize}
\item {Grp. gram.:m.}
\end{itemize}
O mesmo que \textunderscore calemburista\textunderscore . Cf. Júl. Castilho, \textunderscore Lisb. Ant.\textunderscore 
\section{Calemburístico}
\begin{itemize}
\item {Grp. gram.:adj.}
\end{itemize}
Relativo a calemburista ou a calembur.
\section{Calicó}
\begin{itemize}
\item {Grp. gram.:m.}
\end{itemize}
\begin{itemize}
\item {Utilização:Ant.}
\end{itemize}
Tela branca, indiana, de algodão.
\section{Calosoma}
\begin{itemize}
\item {fónica:sô}
\end{itemize}
\begin{itemize}
\item {Grp. gram.:m.}
\end{itemize}
Coleóptero, espécie de carocha grande dos pinhaes, (\textunderscore calosoma sycophanta\textunderscore , Lin.). Cf. P. Moraes, \textunderscore Zool. Elem.\textunderscore , 584.
\section{Calosomo}
\begin{itemize}
\item {fónica:sô}
\end{itemize}
\begin{itemize}
\item {Grp. gram.:m.}
\end{itemize}
Coleóptero, espécie de carocha grande dos pinhaes, (\textunderscore calosoma sycophanta\textunderscore , Lin.). Cf. P. Moraes, \textunderscore Zool. Elem.\textunderscore , 584.
\section{Calleira}
\begin{itemize}
\item {Grp. gram.:f.}
\end{itemize}
\begin{itemize}
\item {Utilização:Prov.}
\end{itemize}
\begin{itemize}
\item {Utilização:alent.}
\end{itemize}
\begin{itemize}
\item {Proveniência:(De \textunderscore callo\textunderscore )}
\end{itemize}
Pedaço de coiro, com que os ceifeiros resguardam a mão esquerda contra os golpes da foice, e que está ligado ás dedeiras por uma correia.
\section{Camaquema}
\begin{itemize}
\item {Grp. gram.:f.}
\end{itemize}
Ave de rapina, africana.
\section{Camarina}
\begin{itemize}
\item {Grp. gram.:f.}
\end{itemize}
Árvore de Timor, que dá boa madeira para construcções. Cf. \textunderscore Século\textunderscore , de 30-VII-911.
\section{Cambeiro}
\begin{itemize}
\item {Grp. gram.:adj.}
\end{itemize}
\begin{itemize}
\item {Utilização:Veter.}
\end{itemize}
Diz-se do joêlho da cavalgadura, quando se afasta muito para o lado de fóra.
(Cp. \textunderscore cambado\textunderscore )
\section{Cambula}
\begin{itemize}
\item {Grp. gram.:f.}
\end{itemize}
\begin{itemize}
\item {Utilização:Bras}
\end{itemize}
Nome de um peixe.
\section{Campaneiro}
\begin{itemize}
\item {Grp. gram.:m.}
\end{itemize}
\begin{itemize}
\item {Utilização:Ant.}
\end{itemize}
Tangedor de campana; sineiro. Cf. \textunderscore Port. Ant. e Mod.\textunderscore 
\section{Campenomia}
\begin{itemize}
\item {Grp. gram.:f.}
\end{itemize}
Parte da Grammática, que trata da flexão das palavras; ptoseonomia.
\section{Cana-brava}
\begin{itemize}
\item {Grp. gram.:f.}
\end{itemize}
\begin{itemize}
\item {Utilização:Bras. do N}
\end{itemize}
Planta dos terrenos encharcados.
\section{Canal}
\begin{itemize}
\item {Grp. gram.:m.}
\end{itemize}
\begin{itemize}
\item {Utilização:Prov.}
\end{itemize}
\begin{itemize}
\item {Utilização:trasm.}
\end{itemize}
\begin{itemize}
\item {Proveniência:(De \textunderscore cana\textunderscore )}
\end{itemize}
O mesmo que \textunderscore canavial\textunderscore .
\section{Canavês}
\begin{itemize}
\item {Grp. gram.:adj.}
\end{itemize}
Diz-se do animal bovídeo da região do Marco-de-Canaveses. Cf. \textunderscore Port. au Point de Vue Agr.\textunderscore , 237 e 252.
\section{Candoroso}
\begin{itemize}
\item {Grp. gram.:adj.}
\end{itemize}
Que tem candor; alvo:«\textunderscore ...teu collo candoroso.\textunderscore »A. Gouveia, \textunderscore Elegias e Carmes\textunderscore , 227.
\section{Canelim}
\begin{itemize}
\item {Grp. gram.:m.}
\end{itemize}
Amêndoa confeitada, cujo núcleo é um pedaço de canela.--Na amêndoa confeitada, chamada \textunderscore canelão\textunderscore , o núcleo é de cidrão.
\section{Canhamaça}
\begin{itemize}
\item {Grp. gram.:f.}
\end{itemize}
Semente de cânhamo.
\section{Capanema}
\begin{itemize}
\item {Grp. gram.:m.}
\end{itemize}
\begin{itemize}
\item {Utilização:Bras. do Rio}
\end{itemize}
Líquido formicida.
\section{Caparsão}
\begin{itemize}
\item {Grp. gram.:m.}
\end{itemize}
\begin{itemize}
\item {Utilização:Ant.}
\end{itemize}
O mesmo que \textunderscore caparazão\textunderscore . Cf. B. Pereira, \textunderscore Prosódia\textunderscore , vb. \textunderscore huricalia\textunderscore .
\section{Capejuba}
\begin{itemize}
\item {Grp. gram.:f.}
\end{itemize}
\begin{itemize}
\item {Utilização:Bras. do Maranhão}
\end{itemize}
Espécie de macaco aloirado.
\section{Capuão}
\begin{itemize}
\item {Grp. gram.:m.}
\end{itemize}
\begin{itemize}
\item {Utilização:Bras}
\end{itemize}
O mesmo que \textunderscore capão\textunderscore ^2.
\section{Caquí}
\begin{itemize}
\item {Grp. gram.:m.}
\end{itemize}
Pano de algodão, usado em vestuário de militares, carteiros, etc.
\section{Cará-da-costa}
\begin{itemize}
\item {Grp. gram.:m.}
\end{itemize}
\begin{itemize}
\item {Utilização:Ant.}
\end{itemize}
O mesmo que \textunderscore inhame\textunderscore .
\section{Carafo}
\begin{itemize}
\item {Grp. gram.:m.}
\end{itemize}
\begin{itemize}
\item {Utilização:Ant.}
\end{itemize}
Indivíduo, que se occupa de transacções ou permutações monetárias; cambiador. Cf. Tenreiro, \textunderscore Itiner.\textunderscore , 29, (ed. 1829)
\section{Caramulo}
\begin{itemize}
\item {Grp. gram.:m.}
\end{itemize}
\begin{itemize}
\item {Utilização:T. da Bairrada}
\end{itemize}
Montão; qualquer eminência: \textunderscore um caramulo de areia\textunderscore ; \textunderscore caiu e fez um caramulo na testa\textunderscore .
\section{Carão-de-moça}
\begin{itemize}
\item {Grp. gram.:m.}
\end{itemize}
Casta de uva branca da Madeira.
\section{Carapo}
\begin{itemize}
\item {Grp. gram.:m.}
\end{itemize}
\begin{itemize}
\item {Utilização:Ant.}
\end{itemize}
Espécie de tela?:«\textunderscore hũa taboa debuxada em carapo branco.\textunderscore »(De um testamento do séc. XVII)
\section{Carcarejadura}
\begin{itemize}
\item {Grp. gram.:f.}
\end{itemize}
\begin{itemize}
\item {Proveniência:(De \textunderscore carcarejar\textunderscore , por \textunderscore cacarejar\textunderscore )}
\end{itemize}
O mesmo que \textunderscore cacarejo\textunderscore . Cf. B. Pereira, \textunderscore Prosódia\textunderscore , vb. \textunderscore pericarphismus\textunderscore .
\section{Cardiectasia}
\begin{itemize}
\item {Grp. gram.:f.}
\end{itemize}
\begin{itemize}
\item {Utilização:Med.}
\end{itemize}
\begin{itemize}
\item {Proveniência:(Do gr. \textunderscore kardia\textunderscore  + \textunderscore ektasis\textunderscore )}
\end{itemize}
Dilatação total ou parcial do coração.
\section{Cardiograma}
\begin{itemize}
\item {Grp. gram.:m.}
\end{itemize}
\begin{itemize}
\item {Utilização:Med.}
\end{itemize}
\begin{itemize}
\item {Proveniência:(Do gr. \textunderscore kardia\textunderscore  + \textunderscore gramma\textunderscore )}
\end{itemize}
Curva, obtida pelo cardiógrafo.
\section{Cardiogramma}
\begin{itemize}
\item {Grp. gram.:m.}
\end{itemize}
\begin{itemize}
\item {Utilização:Med.}
\end{itemize}
\begin{itemize}
\item {Proveniência:(Do gr. \textunderscore kardia\textunderscore  + \textunderscore gramma\textunderscore )}
\end{itemize}
Curva, obtida pelo cardiógrapho.
\section{Cardiopericardite}
\begin{itemize}
\item {Grp. gram.:f.}
\end{itemize}
\begin{itemize}
\item {Utilização:Med.}
\end{itemize}
Inflammação do pericárdio e do coração.
\section{Cardioptose}
\begin{itemize}
\item {Grp. gram.:f.}
\end{itemize}
\begin{itemize}
\item {Utilização:Med.}
\end{itemize}
\begin{itemize}
\item {Proveniência:(Do gr. \textunderscore kardia\textunderscore  + \textunderscore ptosis\textunderscore )}
\end{itemize}
Deslocamento do coração, para baixo.
\section{Cardiorhaphia}
\begin{itemize}
\item {Grp. gram.:f.}
\end{itemize}
\begin{itemize}
\item {Utilização:Med.}
\end{itemize}
\begin{itemize}
\item {Proveniência:(Do gr. \textunderscore kardia\textunderscore  + \textunderscore raphe\textunderscore )}
\end{itemize}
Sutura das feridas do coração.
\section{Cardiorhexia}
\begin{itemize}
\item {fónica:csi}
\end{itemize}
\begin{itemize}
\item {Grp. gram.:f.}
\end{itemize}
\begin{itemize}
\item {Utilização:Med.}
\end{itemize}
\begin{itemize}
\item {Proveniência:(Do gr. \textunderscore kardia\textunderscore  + \textunderscore rhexis\textunderscore )}
\end{itemize}
Rotura do coração.
\section{Cardiorrafia}
\begin{itemize}
\item {Grp. gram.:f.}
\end{itemize}
\begin{itemize}
\item {Utilização:Med.}
\end{itemize}
\begin{itemize}
\item {Proveniência:(Do gr. \textunderscore kardia\textunderscore  + \textunderscore raphe\textunderscore )}
\end{itemize}
Sutura das feridas do coração.
\section{Cardiorrexia}
\begin{itemize}
\item {fónica:csi}
\end{itemize}
\begin{itemize}
\item {Grp. gram.:f.}
\end{itemize}
\begin{itemize}
\item {Utilização:Med.}
\end{itemize}
\begin{itemize}
\item {Proveniência:(Do gr. \textunderscore kardia\textunderscore  + \textunderscore rhexis\textunderscore )}
\end{itemize}
Rotura do coração.
\section{Carnalite}
\begin{itemize}
\item {Grp. gram.:f.}
\end{itemize}
Espécie de chloreto de magnésio e potássio hydratado natural.
\section{Carpectomia}
\begin{itemize}
\item {Grp. gram.:f.}
\end{itemize}
\begin{itemize}
\item {Utilização:Med.}
\end{itemize}
\begin{itemize}
\item {Proveniência:(Do gr. \textunderscore kargos\textunderscore  + \textunderscore ek\textunderscore  + \textunderscore tome\textunderscore )}
\end{itemize}
Secção total ou parcial dos ossos do corpo.
\section{Carpição}
\begin{itemize}
\item {Grp. gram.:f.}
\end{itemize}
\begin{itemize}
\item {Utilização:Bras}
\end{itemize}
Acto de tratar e desmoitar uma roça.
\section{Carteirista}
\begin{itemize}
\item {Grp. gram.:m.}
\end{itemize}
Gatuno, que é vezeiro em roubos de carteiras com valores.
\section{Carulha}
\begin{itemize}
\item {Grp. gram.:f.}
\end{itemize}
O mesmo que \textunderscore carula\textunderscore . Cf. \textunderscore Port. Ant. e Mod.\textunderscore , vb. \textunderscore garulha\textunderscore .
\section{Caryophillada-maiór}
\begin{itemize}
\item {Grp. gram.:f.}
\end{itemize}
Planta, o mesmo que \textunderscore sanícula\textunderscore  ou \textunderscore erva-benta\textunderscore .
\section{Casadinhos}
\begin{itemize}
\item {Grp. gram.:m. pl.}
\end{itemize}
\begin{itemize}
\item {Utilização:Bot.}
\end{itemize}
Planta, o mesmo que \textunderscore lágrima-de-sangue\textunderscore . Cf. P. Coutinho, 229.
\section{Cascalhento}
\begin{itemize}
\item {Grp. gram.:adj.}
\end{itemize}
Diz-se do terreno, em que abunda cascalho.
\section{Cassacão}
\begin{itemize}
\item {Grp. gram.:m.}
\end{itemize}
\begin{itemize}
\item {Utilização:Ant.}
\end{itemize}
O mesmo que \textunderscore casacão\textunderscore . Cf. B. Pereira, \textunderscore Prosódia\textunderscore , vb. \textunderscore cucullus\textunderscore .
\section{Cataglóssio}
\begin{itemize}
\item {Grp. gram.:m.}
\end{itemize}
\begin{itemize}
\item {Proveniência:(Do gr. \textunderscore kata\textunderscore  + \textunderscore glossa\textunderscore )}
\end{itemize}
Instrumento, para fazer baixar a língua.
\section{Cataglosso}
\begin{itemize}
\item {Grp. gram.:m.}
\end{itemize}
\begin{itemize}
\item {Utilização:Med.}
\end{itemize}
\begin{itemize}
\item {Proveniência:(Do gr. \textunderscore kata\textunderscore  + \textunderscore glossa\textunderscore )}
\end{itemize}
Instrumento, para fazer baixar a língua.
\section{Catapúcia-menór}
\begin{itemize}
\item {Grp. gram.:f.}
\end{itemize}
Planta, o mesmo que \textunderscore tartago\textunderscore , (\textunderscore euphorbia lathyris\textunderscore , Lin.).
\section{Catanear}
\begin{itemize}
\item {Grp. gram.:v. i.}
\end{itemize}
Bater com catana.
Discutir:«\textunderscore muito ha hi que catanear nos dois termos\textunderscore ». Filinto, \textunderscore Fáb. de Lafont.\textunderscore 
\section{Cataracteiro}
\begin{itemize}
\item {Grp. gram.:m.  e  adj.}
\end{itemize}
\begin{itemize}
\item {Utilização:Des.}
\end{itemize}
O que soffre cataractas. Cf. L. Sousa, \textunderscore Hist. de S. Dom.\textunderscore , I, 234.
\section{Cavacão}
\begin{itemize}
\item {Grp. gram.:m.}
\end{itemize}
\begin{itemize}
\item {Utilização:Fam.}
\end{itemize}
Acto de mostrar grande zanga ou agastamento: \textunderscore deu um cavacão\textunderscore !
\section{Cebocéfalo}
\begin{itemize}
\item {Grp. gram.:m.}
\end{itemize}
\begin{itemize}
\item {Utilização:Terat.}
\end{itemize}
\begin{itemize}
\item {Proveniência:(Do gr. \textunderscore kebos\textunderscore  + \textunderscore kephale\textunderscore )}
\end{itemize}
Monstro, que, por têr os olhos muito juntos e aparelho nasal atrofiado, dá aparência de macaco.
\section{Cebocéphalo}
\begin{itemize}
\item {Grp. gram.:m.}
\end{itemize}
\begin{itemize}
\item {Utilização:Terat.}
\end{itemize}
\begin{itemize}
\item {Proveniência:(Do gr. \textunderscore kebos\textunderscore  + \textunderscore kephale\textunderscore )}
\end{itemize}
Monstro, que, por têr os olhos muito juntos e apparelho nasal atrophiado, dá apparência de macaco.
\section{Celestite}
\begin{itemize}
\item {Grp. gram.:f.}
\end{itemize}
Variedade de estrôncio, o mesmo que \textunderscore celestina\textunderscore .
\section{Celestito}
\begin{itemize}
\item {Grp. gram.:m.}
\end{itemize}
\begin{itemize}
\item {Utilização:Miner.}
\end{itemize}
Variedade de estrôncio, o mesmo que \textunderscore celestina\textunderscore .
\section{Celticidade}
\begin{itemize}
\item {Grp. gram.:f.}
\end{itemize}
Qualidade do que é céltico.
\section{Cem-pés}
\begin{itemize}
\item {Grp. gram.:m.}
\end{itemize}
Insecto myriápode, o mesmo que \textunderscore cerra-cancella\textunderscore .
\section{Centezinho}
\begin{itemize}
\item {Grp. gram.:adj.}
\end{itemize}
Diz-se de uma espécie de centeio temporão. Cf. \textunderscore Port. au Point de Vue Agr.\textunderscore , 582.
\section{Centola}
\begin{itemize}
\item {Grp. gram.:f.}
\end{itemize}
O mesmo que \textunderscore santola\textunderscore . Cf. B. Pereira, \textunderscore Prosódia\textunderscore , vb. \textunderscore celtina\textunderscore , \textunderscore testudo\textunderscore , etc.
\section{Centrar}
\begin{itemize}
\item {Grp. gram.:v. t.}
\end{itemize}
\begin{itemize}
\item {Utilização:T. de Mecân}
\end{itemize}
\begin{itemize}
\item {Proveniência:(De \textunderscore centro\textunderscore )}
\end{itemize}
Determinar um centro geométrico ou phýsico em.
Fazer coincidir uma série de centros, para formar (eixo geométrico) com outro eixo: \textunderscore centrar granadas na alma das peças de artilharia\textunderscore .
\section{Centrífuga}
\begin{itemize}
\item {Grp. gram.:f.}
\end{itemize}
\begin{itemize}
\item {Proveniência:(De \textunderscore centrífugo\textunderscore )}
\end{itemize}
Pequeno apparelho, para imprimir movimento rotatório a certos objectos, como garrafas, tubos, etc.
\section{Centrifugar}
\begin{itemize}
\item {Grp. gram.:v. t.}
\end{itemize}
\begin{itemize}
\item {Utilização:Neol.}
\end{itemize}
Applicar a centrífuga a.
\section{Calcosina}
\begin{itemize}
\item {Grp. gram.:f.}
\end{itemize}
O mesmo que \textunderscore calcosite\textunderscore .
\section{Calcosite}
\begin{itemize}
\item {Grp. gram.:f.}
\end{itemize}
\begin{itemize}
\item {Proveniência:(Do gr. \textunderscore khalkos\textunderscore )}
\end{itemize}
Sulfureto de cobre mineral.
\section{Calcosito}
\begin{itemize}
\item {Grp. gram.:m.}
\end{itemize}
\begin{itemize}
\item {Proveniência:(Do gr. \textunderscore khalkos\textunderscore )}
\end{itemize}
Sulfureto de cobre mineral.
\section{Cefalómelo}
\begin{itemize}
\item {Grp. gram.:m.}
\end{itemize}
\begin{itemize}
\item {Utilização:Terat.}
\end{itemize}
\begin{itemize}
\item {Proveniência:(Do gr. \textunderscore kephale\textunderscore  + \textunderscore melos\textunderscore )}
\end{itemize}
Monstro, que tem um ou mais membros na cabeça.
\section{Ceboletas-de-frança}
\begin{itemize}
\item {Grp. gram.:f. pl.}
\end{itemize}
Planta liliácea, (\textunderscore allium schaenoprasum\textunderscore , Lin.).
\section{Cefalópago}
\begin{itemize}
\item {Grp. gram.:m.}
\end{itemize}
\begin{itemize}
\item {Utilização:Terat.}
\end{itemize}
\begin{itemize}
\item {Proveniência:(Do gr. \textunderscore kephale\textunderscore  + \textunderscore pageis\textunderscore )}
\end{itemize}
Monstro, composto de dois indivíduos, ligados pelas cabeças.
\section{Cefalotripsia}
\begin{itemize}
\item {Grp. gram.:f.}
\end{itemize}
\begin{itemize}
\item {Utilização:Med.}
\end{itemize}
Operação cirúrgica, com o cefalótribo.
\section{Cephalómelo}
\begin{itemize}
\item {Grp. gram.:m.}
\end{itemize}
\begin{itemize}
\item {Utilização:Terat.}
\end{itemize}
\begin{itemize}
\item {Proveniência:(Do gr. \textunderscore kephale\textunderscore  + \textunderscore melos\textunderscore )}
\end{itemize}
Monstro, que tem um ou mais membros na cabeça.
\section{Cephalópago}
\begin{itemize}
\item {Grp. gram.:m.}
\end{itemize}
\begin{itemize}
\item {Utilização:Terat.}
\end{itemize}
\begin{itemize}
\item {Proveniência:(Do gr. \textunderscore kephale\textunderscore  + \textunderscore pageis\textunderscore )}
\end{itemize}
Monstro, composto de dois indivíduos, ligados pelas cabeças.
\section{Cephalotripsia}
\begin{itemize}
\item {Grp. gram.:f.}
\end{itemize}
\begin{itemize}
\item {Utilização:Med.}
\end{itemize}
Operação cirúrgica, com o cephalótribo.
\section{Cerada}
\begin{itemize}
\item {Grp. gram.:f.}
\end{itemize}
\begin{itemize}
\item {Utilização:Ant.}
\end{itemize}
O mesmo que \textunderscore enceradura\textunderscore . Cf. \textunderscore Palmeirim de Ingl.\textunderscore , III, 34.
\section{Cercoito}
\begin{itemize}
\item {Grp. gram.:m.}
\end{itemize}
\begin{itemize}
\item {Utilização:Ant.}
\end{itemize}
O mesmo que \textunderscore circuito\textunderscore . Cf. \textunderscore Rot. do Mar-Vermelho\textunderscore , 98 e 99.
\section{Cerraceiro}
\begin{itemize}
\item {Grp. gram.:m.}
\end{itemize}
\begin{itemize}
\item {Utilização:Prov.}
\end{itemize}
\begin{itemize}
\item {Proveniência:(De \textunderscore cerração\textunderscore )}
\end{itemize}
Nevoeiro espêsso.
\section{Cerradão}
\begin{itemize}
\item {Grp. gram.:m.}
\end{itemize}
\begin{itemize}
\item {Utilização:Bras}
\end{itemize}
\begin{itemize}
\item {Proveniência:(De \textunderscore cerrado\textunderscore )}
\end{itemize}
Mata xeróphila dos planaltos, na qual as árvores são mais densas e menos tortuosas, e mais variada a flora, do que nas matas chamadas \textunderscore cerrados\textunderscore .
\section{Certão}
\begin{itemize}
\item {Grp. gram.:m.}
\end{itemize}
\begin{itemize}
\item {Utilização:Ant.}
\end{itemize}
(Outra fórma de \textunderscore sertão\textunderscore ). Cf. \textunderscore Rot. do Mar-Vermelho\textunderscore , 178; B. Pereira, \textunderscore Prosodia\textunderscore ; etc.
\section{Cervicite}
\begin{itemize}
\item {Grp. gram.:f.}
\end{itemize}
\begin{itemize}
\item {Utilização:Med.}
\end{itemize}
\begin{itemize}
\item {Proveniência:(Do lat. \textunderscore cervix\textunderscore )}
\end{itemize}
Inflammação do collo uterino; metrite, localizada no collo do útero.
\section{Cerzir}
\textunderscore v. t.\textunderscore  (e der.)
O mesmo que \textunderscore serzir\textunderscore , etc.--Os que rejeitam a etym. latina \textunderscore sarcire\textunderscore , preferem \textunderscore cerzir\textunderscore . Cf. G. Vianna, \textunderscore Voc. Ortogr. e Rem.\textunderscore 
\section{Chá-inglês}
\begin{itemize}
\item {Grp. gram.:m.}
\end{itemize}
\begin{itemize}
\item {Utilização:Bot.}
\end{itemize}
Planta, o mesmo que \textunderscore erva-do-chá\textunderscore .
\section{Chalcosina}
\begin{itemize}
\item {fónica:cal}
\end{itemize}
\begin{itemize}
\item {Grp. gram.:f.}
\end{itemize}
O mesmo que \textunderscore chalcosite\textunderscore .
\section{Chalcosite}
\begin{itemize}
\item {fónica:cal}
\end{itemize}
\begin{itemize}
\item {Grp. gram.:f.}
\end{itemize}
\begin{itemize}
\item {Proveniência:(Do gr. \textunderscore khalkos\textunderscore )}
\end{itemize}
Sulfureto de cobre mineral.
\section{Chalcosito}
\begin{itemize}
\item {fónica:cal}
\end{itemize}
\begin{itemize}
\item {Grp. gram.:m.}
\end{itemize}
\begin{itemize}
\item {Proveniência:(Do gr. \textunderscore khalkos\textunderscore )}
\end{itemize}
Sulfureto de cobre mineral.
\section{Charéu}
\begin{itemize}
\item {Grp. gram.:m.}
\end{itemize}
Peixe vulgar, na costa do centro e norte do Brasil.
\section{Charuto-do-rei}
\begin{itemize}
\item {Grp. gram.:m.}
\end{itemize}
\begin{itemize}
\item {Utilização:Bot.}
\end{itemize}
Planta solanácea, (\textunderscore nicotiana glauca\textunderscore , Grahm).
\section{Chegadinho}
\begin{itemize}
\item {Grp. gram.:m.}
\end{itemize}
\begin{itemize}
\item {Utilização:T. de Grândola}
\end{itemize}
O mesmo que \textunderscore grippe\textunderscore .
\section{Cheleano}
\begin{itemize}
\item {Grp. gram.:adj.}
\end{itemize}
\begin{itemize}
\item {Utilização:Anthrop.}
\end{itemize}
Relativo á estação prehistórica de Chelles, (França).
Diz-se das coisas e seres animados dos primeiros tempos da época geológica quaternária.
\section{Chelleano}
\begin{itemize}
\item {Grp. gram.:adj.}
\end{itemize}
\begin{itemize}
\item {Utilização:Anthrop.}
\end{itemize}
Relativo á estação prehistórica de Chelles, (França).
Diz-se das coisas e seres animados dos primeiros tempos da época geológica quaternária.
\section{Chibeiro}
\begin{itemize}
\item {Grp. gram.:m.}
\end{itemize}
\begin{itemize}
\item {Utilização:Prov.}
\end{itemize}
O mesmo que \textunderscore cabreiro\textunderscore .
Negociante ou cortador de carne de chibo. (Colhido na Bairrada)
\section{Chiclan}
\begin{itemize}
\item {Grp. gram.:adj.}
\end{itemize}
\begin{itemize}
\item {Utilização:Veter.}
\end{itemize}
Diz-se do cavallo, que apresenta um só testículo.
\section{Chintel}
\begin{itemize}
\item {Grp. gram.:m.}
\end{itemize}
\begin{itemize}
\item {Utilização:T. da Bairrada}
\end{itemize}
O mesmo que \textunderscore chantel\textunderscore .
\section{Chiota}
\begin{itemize}
\item {fónica:qui}
\end{itemize}
\begin{itemize}
\item {Grp. gram.:m.  e  f.}
\end{itemize}
Habitante da ilha de Chios.
\section{Chipanzé}
\begin{itemize}
\item {Grp. gram.:m.}
\end{itemize}
\begin{itemize}
\item {Utilização:Pop.}
\end{itemize}
O mesmo que \textunderscore chimpanzé\textunderscore . Cf. P. Moraes, \textunderscore Zool. Elem.\textunderscore  142.
\section{Chloritoxisto}
\begin{itemize}
\item {Grp. gram.:m.}
\end{itemize}
Rocha primitiva, em que predomina a chlorite, o quartzo e o feldspatho.
\section{Chlorização}
\begin{itemize}
\item {Grp. gram.:f.}
\end{itemize}
Acto ou effeito de \textunderscore chlorizar\textunderscore .
\section{Chlorizar}
\begin{itemize}
\item {Grp. gram.:v. t.}
\end{itemize}
Transformar em chloro.
Dar a côr ou os caracteres de chloro a.
\section{Cholangiostomia}
\begin{itemize}
\item {fónica:co}
\end{itemize}
\begin{itemize}
\item {Grp. gram.:f.}
\end{itemize}
\begin{itemize}
\item {Utilização:Med.}
\end{itemize}
\begin{itemize}
\item {Proveniência:(Do gr. \textunderscore khole\textunderscore  + \textunderscore angeion\textunderscore  + \textunderscore tome\textunderscore )}
\end{itemize}
Operação, com que se insere um conducto biliar para a pelle.
\section{Cholecystectomia}
\begin{itemize}
\item {fónica:co}
\end{itemize}
\begin{itemize}
\item {Grp. gram.:f.}
\end{itemize}
\begin{itemize}
\item {Utilização:Med.}
\end{itemize}
\begin{itemize}
\item {Proveniência:(Do gr. \textunderscore khole\textunderscore  + \textunderscore kustis\textunderscore  + \textunderscore ektome\textunderscore )}
\end{itemize}
Extirpação da vesícula biliar.
\section{Cholecystenterostomia}
\begin{itemize}
\item {fónica:co}
\end{itemize}
\begin{itemize}
\item {Grp. gram.:f.}
\end{itemize}
\begin{itemize}
\item {Utilização:Med.}
\end{itemize}
\begin{itemize}
\item {Proveniência:(Do gr. \textunderscore khole\textunderscore  + \textunderscore kustis\textunderscore  + \textunderscore enteron\textunderscore  + \textunderscore stoma\textunderscore )}
\end{itemize}
Operação, que consiste em fazer a abertura da vesícula biliar para o intestino.
\section{Cholemia}
\begin{itemize}
\item {fónica:co}
\end{itemize}
\begin{itemize}
\item {Grp. gram.:f.}
\end{itemize}
O mesmo ou melhór que \textunderscore cholihemia\textunderscore .
\section{Cholicystite}
\begin{itemize}
\item {fónica:co}
\end{itemize}
\begin{itemize}
\item {Grp. gram.:f.}
\end{itemize}
\begin{itemize}
\item {Utilização:Med.}
\end{itemize}
\begin{itemize}
\item {Proveniência:(Do gr. \textunderscore khole\textunderscore  + \textunderscore kustis\textunderscore )}
\end{itemize}
Inflammação da vesícula biliar.
\section{Cholohemia}
\begin{itemize}
\item {fónica:co}
\end{itemize}
\begin{itemize}
\item {Grp. gram.:f.}
\end{itemize}
O mesmo que \textunderscore cholihemia\textunderscore .
\section{Chondrite}
\begin{itemize}
\item {fónica:con}
\end{itemize}
\begin{itemize}
\item {Grp. gram.:f.}
\end{itemize}
\begin{itemize}
\item {Utilização:Med.}
\end{itemize}
\begin{itemize}
\item {Proveniência:(De \textunderscore chondro\textunderscore )}
\end{itemize}
Inflammação de uma cartilagem.
\section{Chordite}
\begin{itemize}
\item {fónica:cor}
\end{itemize}
\begin{itemize}
\item {Grp. gram.:f.}
\end{itemize}
\begin{itemize}
\item {Utilização:Med.}
\end{itemize}
\begin{itemize}
\item {Proveniência:(Do gr. \textunderscore khorde\textunderscore )}
\end{itemize}
Inflammação das cordas vocais.
\section{Chorebispo}
\begin{itemize}
\item {fónica:co}
\end{itemize}
\begin{itemize}
\item {Grp. gram.:m.}
\end{itemize}
\begin{itemize}
\item {Proveniência:(Lat. \textunderscore chorepiscopus\textunderscore )}
\end{itemize}
Aquelle que, na primitiva Igreja, fazia as vezes de Bispo; Arcediago.
\section{Choroidite}
\begin{itemize}
\item {fónica:co}
\end{itemize}
\begin{itemize}
\item {Grp. gram.:f.}
\end{itemize}
\begin{itemize}
\item {Utilização:Med.}
\end{itemize}
Inflammação da choróide.
\section{Choupo}
\begin{itemize}
\item {Grp. gram.:m.}
\end{itemize}
Peixe esparóide, também conhecido por \textunderscore choupa\textunderscore .
\section{Christadelphos}
\begin{itemize}
\item {Grp. gram.:m. pl.}
\end{itemize}
Seita religiosa dos Estados-Unidos.
\section{Chromidrose}
\begin{itemize}
\item {fónica:cro}
\end{itemize}
\begin{itemize}
\item {Grp. gram.:f.}
\end{itemize}
\begin{itemize}
\item {Utilização:Med.}
\end{itemize}
\begin{itemize}
\item {Proveniência:(Do gr. \textunderscore kroma\textunderscore  + \textunderscore hidrosis\textunderscore )}
\end{itemize}
Suor còrado.
\section{Chupim}
\begin{itemize}
\item {Grp. gram.:m.}
\end{itemize}
\begin{itemize}
\item {Utilização:Bras. de San-Paulo}
\end{itemize}
O mesmo ou melhór que \textunderscore chopim\textunderscore .
Indivíduo interesseiro, que casa com mulhér abastada, para viver á custa della.
\section{Chymoso}
\begin{itemize}
\item {fónica:qui}
\end{itemize}
\begin{itemize}
\item {Grp. gram.:adj.}
\end{itemize}
Relativo ao chymo.
\section{Cicate}
\begin{itemize}
\item {Grp. gram.:m.}
\end{itemize}
\begin{itemize}
\item {Utilização:Des.}
\end{itemize}
O mesmo que \textunderscore acicate\textunderscore .
\section{Cicieira}
\begin{itemize}
\item {Grp. gram.:f.}
\end{itemize}
\begin{itemize}
\item {Utilização:Prov.}
\end{itemize}
\begin{itemize}
\item {Proveniência:(De \textunderscore cieiro\textunderscore ? Ou de \textunderscore ciciar\textunderscore ?)}
\end{itemize}
Vento fresco.
\section{Cimério}
\begin{itemize}
\item {Grp. gram.:adj.}
\end{itemize}
Espêsso; profundo:«\textunderscore cimérios sonhos...\textunderscore »M. Bernárdez, \textunderscore N. Floresta\textunderscore .
\section{Cimmério}
\begin{itemize}
\item {Grp. gram.:adj.}
\end{itemize}
Espêsso; profundo:«\textunderscore cimmérios sonhos...\textunderscore »M. Bernárdez, \textunderscore N. Floresta\textunderscore .
\section{Cincoésma}
\begin{itemize}
\item {Grp. gram.:f.}
\end{itemize}
\begin{itemize}
\item {Utilização:Ant.}
\end{itemize}
\begin{itemize}
\item {Proveniência:(Do lat. \textunderscore quinquagesima\textunderscore )}
\end{itemize}
Festa de Pentecostes ou do Espírito Santo, cincoenta dias depois da Páscoa. Cf. \textunderscore Rev. Lus.\textunderscore , XVI, 2.
\section{Circumfusa}
\begin{itemize}
\item {Grp. gram.:f.}
\end{itemize}
\begin{itemize}
\item {Utilização:Philos.}
\end{itemize}
\begin{itemize}
\item {Proveniência:(Do lat. \textunderscore circumfusus\textunderscore )}
\end{itemize}
Conjunto dos agentes phýsicos externos, (clima, atmosphera, habitação, etc.).
\section{Circunfusa}
\begin{itemize}
\item {Grp. gram.:f.}
\end{itemize}
\begin{itemize}
\item {Utilização:Philos.}
\end{itemize}
\begin{itemize}
\item {Proveniência:(Do lat. \textunderscore circumfusus\textunderscore )}
\end{itemize}
Conjunto dos agentes físicos externos, (clima, atmosfera, habitação, etc.).
\section{Cirrhose}
\begin{itemize}
\item {Grp. gram.:f.}
\end{itemize}
\begin{itemize}
\item {Utilização:Med.}
\end{itemize}
\begin{itemize}
\item {Proveniência:(Do gr. \textunderscore kirrhos\textunderscore , amarelo, porque o fígado, em tal estado, apresenta ás vezes granulações amarelas)}
\end{itemize}
Moléstia, caracterizada pela esclerose do fígado, com aumento do volume dêsse órgão, hypertrophia do baço, etc.
\section{Cirrose}
\begin{itemize}
\item {Grp. gram.:f.}
\end{itemize}
\begin{itemize}
\item {Utilização:Med.}
\end{itemize}
\begin{itemize}
\item {Proveniência:(Do gr. \textunderscore kirrhos\textunderscore , amarelo, porque o fígado, em tal estado, apresenta ás vezes granulações amarelas)}
\end{itemize}
Moléstia, caracterizada pela esclerose do fígado, com aumento do volume desse órgão, hipertrofia do baço, etc.
\section{Cirsotomia}
\begin{itemize}
\item {Grp. gram.:f.}
\end{itemize}
\begin{itemize}
\item {Utilização:Med.}
\end{itemize}
\begin{itemize}
\item {Proveniência:(Do gr. \textunderscore kirsos\textunderscore  + \textunderscore tome\textunderscore )}
\end{itemize}
Extirpação de varizes.
\section{Cloritoxisto}
\begin{itemize}
\item {Grp. gram.:m.}
\end{itemize}
Rocha primitiva, em que predomina a clorite, o quartzo e o feldspato.
\section{Clorização}
\begin{itemize}
\item {Grp. gram.:f.}
\end{itemize}
Acto ou effeito de \textunderscore clorizar\textunderscore .
\section{Clorizar}
\begin{itemize}
\item {Grp. gram.:v. t.}
\end{itemize}
Transformar em cloro.
Dar a côr ou os caracteres de cloro a.
\section{Colangiostomia}
\begin{itemize}
\item {Grp. gram.:f.}
\end{itemize}
\begin{itemize}
\item {Utilização:Med.}
\end{itemize}
\begin{itemize}
\item {Proveniência:(Do gr. \textunderscore khole\textunderscore  + \textunderscore angeion\textunderscore  + \textunderscore tome\textunderscore )}
\end{itemize}
Operação, com que se insere um conducto biliar para a pele.
\section{Colecistectomia}
\begin{itemize}
\item {Grp. gram.:f.}
\end{itemize}
\begin{itemize}
\item {Utilização:Med.}
\end{itemize}
\begin{itemize}
\item {Proveniência:(Do gr. \textunderscore khole\textunderscore  + \textunderscore kustis\textunderscore  + \textunderscore ektome\textunderscore )}
\end{itemize}
Extirpação da vesícula biliar.
\section{Colecistenterostomia}
\begin{itemize}
\item {Grp. gram.:f.}
\end{itemize}
\begin{itemize}
\item {Utilização:Med.}
\end{itemize}
\begin{itemize}
\item {Proveniência:(Do gr. \textunderscore khole\textunderscore  + \textunderscore kustis\textunderscore  + \textunderscore enteron\textunderscore  + \textunderscore stoma\textunderscore )}
\end{itemize}
Operação, que consiste em fazer a abertura da vesícula biliar para o intestino.
\section{Colemia}
\begin{itemize}
\item {Grp. gram.:f.}
\end{itemize}
O mesmo ou melhór que \textunderscore cholihemia\textunderscore .
\section{Colicistite}
\begin{itemize}
\item {Grp. gram.:f.}
\end{itemize}
\begin{itemize}
\item {Utilização:Med.}
\end{itemize}
\begin{itemize}
\item {Proveniência:(Do gr. \textunderscore khole\textunderscore  + \textunderscore kustis\textunderscore )}
\end{itemize}
Inflamação da vesícula biliar.
\section{Coloemia}
\begin{itemize}
\item {Grp. gram.:f.}
\end{itemize}
O mesmo que \textunderscore cholihemia\textunderscore .
\section{Condrite}
\begin{itemize}
\item {Grp. gram.:f.}
\end{itemize}
\begin{itemize}
\item {Utilização:Med.}
\end{itemize}
\begin{itemize}
\item {Proveniência:(De \textunderscore condro\textunderscore )}
\end{itemize}
Inflamação de uma cartilagem.
\section{Cordite}
\begin{itemize}
\item {Grp. gram.:f.}
\end{itemize}
\begin{itemize}
\item {Utilização:Med.}
\end{itemize}
\begin{itemize}
\item {Proveniência:(Do gr. \textunderscore khorde\textunderscore )}
\end{itemize}
Inflamação das cordas vocais.
\section{Corebispo}
\begin{itemize}
\item {Grp. gram.:m.}
\end{itemize}
\begin{itemize}
\item {Proveniência:(Lat. \textunderscore chorepiscopus\textunderscore )}
\end{itemize}
Aquele que, na primitiva Igreja, fazia as vezes de Bispo; Arcediago.
\section{Coroidite}
\begin{itemize}
\item {Grp. gram.:f.}
\end{itemize}
\begin{itemize}
\item {Utilização:Med.}
\end{itemize}
Inflamação da coróide.
\section{Cristadelfos}
\begin{itemize}
\item {Grp. gram.:m. pl.}
\end{itemize}
Seita religiosa dos Estados-Unidos.
\section{Cromidrose}
\begin{itemize}
\item {Grp. gram.:f.}
\end{itemize}
\begin{itemize}
\item {Utilização:Med.}
\end{itemize}
\begin{itemize}
\item {Proveniência:(Do gr. \textunderscore kroma\textunderscore  + \textunderscore hidrosis\textunderscore )}
\end{itemize}
Suor còrado.
\section{Claim}
\begin{itemize}
\item {fónica:kleime}
\end{itemize}
\begin{itemize}
\item {Grp. gram.:m.}
\end{itemize}
\begin{itemize}
\item {Proveniência:(T. ingl.)}
\end{itemize}
Determinado volume e área de terreno metallífero. Cf. \textunderscore Decreto\textunderscore  de 24-IV-1911.--Usa-se principalmente na África Or. Port., e apparece em documentos officiaes como se fôsse fórma portuguesa. Esta seria \textunderscore cleime\textunderscore .
\section{Clárkia}
\begin{itemize}
\item {Grp. gram.:f.}
\end{itemize}
Gênero de plantas de jardim.
\section{Clarinetista}
\begin{itemize}
\item {Grp. gram.:m.}
\end{itemize}
Tocador de clarinete.
\section{Cleptofobia}
\begin{itemize}
\item {Grp. gram.:f.}
\end{itemize}
\begin{itemize}
\item {Utilização:Med.}
\end{itemize}
\begin{itemize}
\item {Proveniência:(Do gr. \textunderscore klepto\textunderscore  + \textunderscore phobos\textunderscore )}
\end{itemize}
Mêdo mórbido de roubar.
\section{Cleptophobia}
\begin{itemize}
\item {Grp. gram.:f.}
\end{itemize}
\begin{itemize}
\item {Utilização:Med.}
\end{itemize}
\begin{itemize}
\item {Proveniência:(Do gr. \textunderscore klepto\textunderscore  + \textunderscore phobos\textunderscore )}
\end{itemize}
Mêdo mórbido de roubar.
\section{Chliantho}
\begin{itemize}
\item {Grp. gram.:m.}
\end{itemize}
Gênero de plantas de jardim.
\section{Clianto}
\begin{itemize}
\item {Grp. gram.:m.}
\end{itemize}
Gênero de plantas de jardim.
\section{Clitorismo}
\begin{itemize}
\item {Grp. gram.:m.}
\end{itemize}
\begin{itemize}
\item {Utilização:Med.}
\end{itemize}
\begin{itemize}
\item {Proveniência:(De \textunderscore clitóris\textunderscore )}
\end{itemize}
Perversão da mulhér que, tendo clitóris volumoso, satisfaz com outra desejos genésicos.
\section{Coaxada}
\begin{itemize}
\item {Grp. gram.:f.}
\end{itemize}
O mesmo que \textunderscore coaxo\textunderscore .
\section{Cobaltite}
\begin{itemize}
\item {Grp. gram.:f.}
\end{itemize}
Sulfureto de arsênico e cobalto.
\section{Cobaltito}
\begin{itemize}
\item {Grp. gram.:m.}
\end{itemize}
Sulfureto de arsênico e cobalto.
\section{Cocainismo}
\begin{itemize}
\item {fónica:ca-i}
\end{itemize}
\begin{itemize}
\item {Grp. gram.:m.}
\end{itemize}
\begin{itemize}
\item {Utilização:Med.}
\end{itemize}
Entoxicação pela cocaína.
\section{Cóccus}
\begin{itemize}
\item {Grp. gram.:m.}
\end{itemize}
\begin{itemize}
\item {Utilização:Med.}
\end{itemize}
(V.côcco)
\section{Coccygotomia}
\begin{itemize}
\item {Grp. gram.:f.}
\end{itemize}
\begin{itemize}
\item {Utilização:Med.}
\end{itemize}
\begin{itemize}
\item {Proveniência:(Do gr. \textunderscore kokkus\textunderscore , \textunderscore kokkigos\textunderscore  + \textunderscore tome\textunderscore )}
\end{itemize}
Secção do cóccyx, para alargamento da bacia.
\section{Cocigotomia}
\begin{itemize}
\item {Grp. gram.:f.}
\end{itemize}
\begin{itemize}
\item {Utilização:Med.}
\end{itemize}
\begin{itemize}
\item {Proveniência:(Do gr. \textunderscore kokkus\textunderscore , \textunderscore kokkigos\textunderscore  + \textunderscore tome\textunderscore )}
\end{itemize}
Secção do cócix, para alargamento da bacia.
\section{Cocheiral}
\begin{itemize}
\item {Grp. gram.:adj.}
\end{itemize}
Relativo a cocheiro: \textunderscore mãos cocheiraes\textunderscore .
\section{Coisica}
\begin{itemize}
\item {Grp. gram.:f.}
\end{itemize}
\begin{itemize}
\item {Utilização:Bras. de Minas}
\end{itemize}
Pouca coisa.
Bagatela.
\section{Coitadice}
\begin{itemize}
\item {Grp. gram.:f.}
\end{itemize}
\begin{itemize}
\item {Utilização:Ant.}
\end{itemize}
\begin{itemize}
\item {Proveniência:(De \textunderscore coitado\textunderscore ?)}
\end{itemize}
Fraqueza; cobardia. Cf. Frei M. da Esperança, \textunderscore Hist. Seráf.\textunderscore , II, 350.
\section{Coleocele}
\begin{itemize}
\item {Grp. gram.:m.}
\end{itemize}
\begin{itemize}
\item {Utilização:Med.}
\end{itemize}
\begin{itemize}
\item {Proveniência:(Do gr. \textunderscore koleos\textunderscore  + \textunderscore kele\textunderscore )}
\end{itemize}
Hérnia vaginal.
\section{Coleoptose}
\begin{itemize}
\item {Grp. gram.:f.}
\end{itemize}
\begin{itemize}
\item {Utilização:Med.}
\end{itemize}
\begin{itemize}
\item {Proveniência:(Do gr. \textunderscore koleos\textunderscore  + \textunderscore ptosis\textunderscore )}
\end{itemize}
Quéda ou prolapso da vagina.
\section{Coleorhexia}
\begin{itemize}
\item {fónica:ré,csi}
\end{itemize}
\begin{itemize}
\item {Grp. gram.:f.}
\end{itemize}
\begin{itemize}
\item {Utilização:Med.}
\end{itemize}
\begin{itemize}
\item {Proveniência:(Do gr. \textunderscore koleos\textunderscore  + \textunderscore rhexis\textunderscore )}
\end{itemize}
Rotura da vagina.
\section{Coleorrexia}
\begin{itemize}
\item {fónica:csi}
\end{itemize}
\begin{itemize}
\item {Grp. gram.:f.}
\end{itemize}
\begin{itemize}
\item {Utilização:Med.}
\end{itemize}
\begin{itemize}
\item {Proveniência:(Do gr. \textunderscore koleos\textunderscore  + \textunderscore rhexis\textunderscore )}
\end{itemize}
Rotura da vagina.
\section{Coleostegnose}
\begin{itemize}
\item {Grp. gram.:f.}
\end{itemize}
\begin{itemize}
\item {Utilização:Med.}
\end{itemize}
\begin{itemize}
\item {Proveniência:(Do gr. \textunderscore koleos\textunderscore  + \textunderscore stegnosis\textunderscore )}
\end{itemize}
Estreitamento da vagina.
\section{Colínsia}
\begin{itemize}
\item {Grp. gram.:f.}
\end{itemize}
Gênero de plantas de jardim.
\section{Collemia}
\begin{itemize}
\item {Grp. gram.:f.}
\end{itemize}
\begin{itemize}
\item {Utilização:Med.}
\end{itemize}
\begin{itemize}
\item {Proveniência:(Do gr. \textunderscore kolle\textunderscore  + \textunderscore haima\textunderscore )}
\end{itemize}
Estado mórbido, em que o sangue apresenta substâncias collóides.
\section{Colopathia}
\begin{itemize}
\item {Grp. gram.:f.}
\end{itemize}
\begin{itemize}
\item {Utilização:Med.}
\end{itemize}
\begin{itemize}
\item {Proveniência:(Do gr. \textunderscore kolon\textunderscore  + \textunderscore pathos\textunderscore )}
\end{itemize}
Moléstia do cólon.
\section{Colopatia}
\begin{itemize}
\item {Grp. gram.:f.}
\end{itemize}
\begin{itemize}
\item {Utilização:Med.}
\end{itemize}
\begin{itemize}
\item {Proveniência:(Do gr. \textunderscore kolon\textunderscore  + \textunderscore pathos\textunderscore )}
\end{itemize}
Moléstia do cólon.
\section{Colopexia}
\begin{itemize}
\item {fónica:csi}
\end{itemize}
\begin{itemize}
\item {Grp. gram.:f.}
\end{itemize}
\begin{itemize}
\item {Utilização:Med.}
\end{itemize}
\begin{itemize}
\item {Proveniência:(Do gr. \textunderscore kolon\textunderscore  + \textunderscore pexis\textunderscore )}
\end{itemize}
Fixação do cólon na parede abdominal anterior.
\section{Colossucorreia}
\begin{itemize}
\item {Grp. gram.:f.}
\end{itemize}
\begin{itemize}
\item {Utilização:Med.}
\end{itemize}
Hipersecreção da mucosa do intestino grosso.
\section{Colosuccorhéa}
\begin{itemize}
\item {fónica:su}
\end{itemize}
\begin{itemize}
\item {Grp. gram.:f.}
\end{itemize}
\begin{itemize}
\item {Utilização:Med.}
\end{itemize}
Hypersecreção da mucosa do intestino grosso.
\section{Colotifo}
\begin{itemize}
\item {Grp. gram.:m.}
\end{itemize}
\begin{itemize}
\item {Utilização:Med.}
\end{itemize}
\begin{itemize}
\item {Proveniência:(De \textunderscore cólon\textunderscore  + \textunderscore typho\textunderscore )}
\end{itemize}
Febre tifóide, caracterizada por se darem no intestino grosso as ulcerações.
\section{Colotomia}
\begin{itemize}
\item {Grp. gram.:f.}
\end{itemize}
\begin{itemize}
\item {Utilização:Med.}
\end{itemize}
\begin{itemize}
\item {Proveniência:(Do gr. \textunderscore kolon\textunderscore  + \textunderscore tome\textunderscore )}
\end{itemize}
Formação de ânus artificial, por abertura no cólon.
\section{Colotypho}
\begin{itemize}
\item {Grp. gram.:m.}
\end{itemize}
\begin{itemize}
\item {Utilização:Med.}
\end{itemize}
\begin{itemize}
\item {Proveniência:(De \textunderscore cólon\textunderscore  + \textunderscore typho\textunderscore )}
\end{itemize}
Febre typhóide, caracterizada por se darem no intestino grosso as ulcerações.
\section{Colpite}
\begin{itemize}
\item {Grp. gram.:f.}
\end{itemize}
\begin{itemize}
\item {Utilização:Med.}
\end{itemize}
\begin{itemize}
\item {Proveniência:(Do gr. \textunderscore kolpos\textunderscore )}
\end{itemize}
Inflammação da vagina; vaginite.
\section{Colpocele}
\begin{itemize}
\item {Grp. gram.:m.}
\end{itemize}
\begin{itemize}
\item {Proveniência:(Do gr. \textunderscore kolpos\textunderscore  + \textunderscore kele\textunderscore )}
\end{itemize}
O mesmo que \textunderscore coleocele\textunderscore .
\section{Colpocistomia}
\begin{itemize}
\item {Grp. gram.:f.}
\end{itemize}
\begin{itemize}
\item {Utilização:Med.}
\end{itemize}
\begin{itemize}
\item {Proveniência:(Do gr. \textunderscore kolpos\textunderscore  + \textunderscore kustis\textunderscore  + \textunderscore tome\textunderscore )}
\end{itemize}
Incisão do colo da bexiga pela vagina.
\section{Colpocystomia}
\begin{itemize}
\item {Grp. gram.:f.}
\end{itemize}
\begin{itemize}
\item {Utilização:Med.}
\end{itemize}
\begin{itemize}
\item {Proveniência:(Do gr. \textunderscore kolpos\textunderscore  + \textunderscore kustis\textunderscore  + \textunderscore tome\textunderscore )}
\end{itemize}
Incisão do collo da bexiga pela vagina.
\section{Colpohysterectomia}
\begin{itemize}
\item {Grp. gram.:f.}
\end{itemize}
\begin{itemize}
\item {Utilização:Med.}
\end{itemize}
\begin{itemize}
\item {Proveniência:(Do gr. \textunderscore kolpos\textunderscore  + \textunderscore hustera\textunderscore  + \textunderscore ektome\textunderscore )}
\end{itemize}
Hysterectomia da vagina.
\section{Colpohysteropexia}
\begin{itemize}
\item {fónica:csi}
\end{itemize}
\begin{itemize}
\item {Grp. gram.:f.}
\end{itemize}
\begin{itemize}
\item {Utilização:Med.}
\end{itemize}
\begin{itemize}
\item {Proveniência:(Do gr. \textunderscore kolpos\textunderscore  + \textunderscore hustera\textunderscore  + \textunderscore pexis\textunderscore )}
\end{itemize}
Operação, com que se fixa o collo do útero á parede vaginal.
\section{Colpohysterostomia}
\begin{itemize}
\item {Grp. gram.:f.}
\end{itemize}
\begin{itemize}
\item {Utilização:Med.}
\end{itemize}
\begin{itemize}
\item {Proveniência:(Do gr. \textunderscore kolpos\textunderscore  + \textunderscore hustera\textunderscore  + \textunderscore stoma\textunderscore )}
\end{itemize}
Incisão das paredes vaginaes e uterinas, postas em contacto.
\section{Colpoisterectomia}
\begin{itemize}
\item {fónica:po-is}
\end{itemize}
\begin{itemize}
\item {Grp. gram.:f.}
\end{itemize}
\begin{itemize}
\item {Utilização:Med.}
\end{itemize}
\begin{itemize}
\item {Proveniência:(Do gr. \textunderscore kolpos\textunderscore  + \textunderscore hustera\textunderscore  + \textunderscore ektome\textunderscore )}
\end{itemize}
Histerectomia da vagina.
\section{Colpoisteropexia}
\begin{itemize}
\item {fónica:po-is,csi}
\end{itemize}
\begin{itemize}
\item {Grp. gram.:f.}
\end{itemize}
\begin{itemize}
\item {Utilização:Med.}
\end{itemize}
\begin{itemize}
\item {Proveniência:(Do gr. \textunderscore kolpos\textunderscore  + \textunderscore hustera\textunderscore  + \textunderscore pexis\textunderscore )}
\end{itemize}
Operação, com que se fixa o colo do útero á parede vaginal.
\section{Colpoisterostomia}
\begin{itemize}
\item {fónica:po-is}
\end{itemize}
\begin{itemize}
\item {Grp. gram.:f.}
\end{itemize}
\begin{itemize}
\item {Utilização:Med.}
\end{itemize}
\begin{itemize}
\item {Proveniência:(Do gr. \textunderscore kolpos\textunderscore  + \textunderscore hustera\textunderscore  + \textunderscore stoma\textunderscore )}
\end{itemize}
Incisão das paredes vaginaes e uterinas, postas em contacto.
\section{Colpoperineoplastia}
\begin{itemize}
\item {Grp. gram.:f.}
\end{itemize}
\begin{itemize}
\item {Utilização:Med.}
\end{itemize}
\begin{itemize}
\item {Proveniência:(Do gr. \textunderscore kolpos\textunderscore  + \textunderscore perinaion\textunderscore  + \textunderscore plassein\textunderscore )}
\end{itemize}
Operação, para aumentar a espessura do perineu, deminuindo o orifício vulvar.
\section{Colpoperineorhaphia}
\begin{itemize}
\item {fónica:ra}
\end{itemize}
\begin{itemize}
\item {Grp. gram.:f.}
\end{itemize}
\begin{itemize}
\item {Utilização:Med.}
\end{itemize}
\begin{itemize}
\item {Proveniência:(Do gr. \textunderscore kolpos\textunderscore  + \textunderscore perinaion\textunderscore  + \textunderscore raphe\textunderscore )}
\end{itemize}
Operação, com que se restaura o perineu, suturando parte da parede vaginal.
\section{Colpoperineorrafia}
\begin{itemize}
\item {Grp. gram.:f.}
\end{itemize}
\begin{itemize}
\item {Utilização:Med.}
\end{itemize}
\begin{itemize}
\item {Proveniência:(Do gr. \textunderscore kolpos\textunderscore  + \textunderscore perinaion\textunderscore  + \textunderscore raphe\textunderscore )}
\end{itemize}
Operação, com que se restaura o perineu, suturando parte da parede vaginal.
\section{Colpoptose}
\begin{itemize}
\item {Grp. gram.:f.}
\end{itemize}
\begin{itemize}
\item {Utilização:Med.}
\end{itemize}
\begin{itemize}
\item {Proveniência:(Do gr. \textunderscore kolpos\textunderscore  + \textunderscore ptosis\textunderscore )}
\end{itemize}
O mesmo que \textunderscore coleoptose\textunderscore .
\section{Colpotomia}
\begin{itemize}
\item {Grp. gram.:f.}
\end{itemize}
\begin{itemize}
\item {Utilização:Med.}
\end{itemize}
\begin{itemize}
\item {Proveniência:(Do gr. \textunderscore kolpos\textunderscore  + \textunderscore tome\textunderscore )}
\end{itemize}
Incisão da vagina.
\section{Colubrão}
\begin{itemize}
\item {Grp. gram.:m.}
\end{itemize}
Constellação, também conhecida por \textunderscore serpentária\textunderscore  e \textunderscore serpentário\textunderscore . Cf. B. Pereira, \textunderscore Prosodia\textunderscore , vb. \textunderscore palma serpentarii\textunderscore .
\section{Columnar}
\begin{itemize}
\item {Grp. gram.:v. t.}
\end{itemize}
\begin{itemize}
\item {Utilização:Neol.}
\end{itemize}
Dispor em columnas.
Dar fórma de columnas a.
\section{Colunar}
\begin{itemize}
\item {Grp. gram.:v. t.}
\end{itemize}
\begin{itemize}
\item {Utilização:Neol.}
\end{itemize}
Dispor em colunas.
Dar fórma de colunas a.
\section{Colemia}
\begin{itemize}
\item {Grp. gram.:f.}
\end{itemize}
\begin{itemize}
\item {Utilização:Med.}
\end{itemize}
\begin{itemize}
\item {Proveniência:(Do gr. \textunderscore kolle\textunderscore  + \textunderscore haima\textunderscore )}
\end{itemize}
Estado mórbido, em que o sangue apresenta substâncias collóides.
\section{Cómea}
\begin{itemize}
\item {Grp. gram.:f.}
\end{itemize}
Gênero de plantas de jardim.
\section{Comilôa}
\begin{itemize}
\item {Grp. gram.:f.}
\end{itemize}
Mulhér, que come muito. Cf. B. Pereira, \textunderscore Prosodia\textunderscore , vb. \textunderscore edax\textunderscore .
(Fem. de \textunderscore comilão\textunderscore )
\section{Comilona}
\begin{itemize}
\item {Grp. gram.:f.}
\end{itemize}
O mesmo e mais usado que \textunderscore comilôa\textunderscore .
\section{Comprativo}
\begin{itemize}
\item {Grp. gram.:adj.}
\end{itemize}
\begin{itemize}
\item {Utilização:Neol.}
\end{itemize}
\begin{itemize}
\item {Proveniência:(De \textunderscore comprar\textunderscore )}
\end{itemize}
Relativo a compra.
Que serve para comprar: \textunderscore valor comprativo da moéda\textunderscore .
\section{Compratório}
\begin{itemize}
\item {Grp. gram.:adj.}
\end{itemize}
\begin{itemize}
\item {Utilização:Neol.}
\end{itemize}
\begin{itemize}
\item {Proveniência:(De \textunderscore comprar\textunderscore )}
\end{itemize}
Que serve para comprar.
Que tem como effeito a compra: \textunderscore a efficácia compratória do dinheiro\textunderscore .
\section{Compridamente}
\begin{itemize}
\item {Grp. gram.:adv.}
\end{itemize}
\begin{itemize}
\item {Utilização:Des.}
\end{itemize}
\begin{itemize}
\item {Proveniência:(De \textunderscore comprir\textunderscore )}
\end{itemize}
Completamente; perfeitamente.
\section{Compridão}
\begin{itemize}
\item {Grp. gram.:f.}
\end{itemize}
\begin{itemize}
\item {Proveniência:(De \textunderscore comprido\textunderscore )}
\end{itemize}
O mesmo que \textunderscore comprimento\textunderscore :«\textunderscore a compridão\textunderscore  (do barco) \textunderscore o faz fraco\textunderscore ». \textunderscore Liv. da Fáb. das Naus\textunderscore , 208. Cf. Castilho, \textunderscore Pal. de um Crente\textunderscore , 62.
\section{Contabilista}
\begin{itemize}
\item {Grp. gram.:m.}
\end{itemize}
\begin{itemize}
\item {Utilização:Neol.}
\end{itemize}
Aquelle que é perito em contabilidade:«\textunderscore dar autoridade jurídica aos contabilistas, chamados a intervir...\textunderscore »\textunderscore Diário-do-Gov.\textunderscore , de 29-V-911.
\section{Contagional}
\begin{itemize}
\item {Grp. gram.:adj.}
\end{itemize}
\begin{itemize}
\item {Proveniência:(De \textunderscore contagião\textunderscore )}
\end{itemize}
Relativo a contágio.
Que se realiza por contágio:«\textunderscore transmissão contagional\textunderscore ». R. Jorge, \textunderscore Bol. dos Serv. Sanit.\textunderscore , I.
\section{Contẽença}
\begin{itemize}
\item {Grp. gram.:f.}
\end{itemize}
\begin{itemize}
\item {Utilização:Ant.}
\end{itemize}
Comportamento; porte. Cf. \textunderscore Port. Mon. Hist.\textunderscore , \textunderscore Script.\textunderscore , 241.
\section{Contra-alisado}
\begin{itemize}
\item {Grp. gram.:m.}
\end{itemize}
Vento, que sopra do lado contrário ao dos alisados.
\section{Contra-safra}
\begin{itemize}
\item {Grp. gram.:f.}
\end{itemize}
\begin{itemize}
\item {Utilização:Agr.}
\end{itemize}
Anno de intervallo, em que não há safra ou colheita. Cf. \textunderscore Port. au Point de Vue Agr.\textunderscore , 488.
\section{Contratista}
\begin{itemize}
\item {Grp. gram.:m.}
\end{itemize}
\begin{itemize}
\item {Utilização:Neol.}
\end{itemize}
Aquelle que contrata com o Estado, com uma Companhia ou com outra collectividade, trabalhos de construcção ou viação por empreitada ou por tarefa. Cf. o respectivo \textunderscore Regulamento\textunderscore  official.
\section{Coprologia}
\begin{itemize}
\item {Grp. gram.:f.}
\end{itemize}
\begin{itemize}
\item {Utilização:Med.}
\end{itemize}
\begin{itemize}
\item {Proveniência:(Do gr. \textunderscore kopros\textunderscore  + \textunderscore logos\textunderscore )}
\end{itemize}
Estudo das matérias fecaes.
\section{Coqueluchóide}
\begin{itemize}
\item {Grp. gram.:f.}
\end{itemize}
\begin{itemize}
\item {Utilização:Med.}
\end{itemize}
Diz-se de uma tosse, análoga á coqueluche.
\section{Coralíneo}
\begin{itemize}
\item {Grp. gram.:adj.}
\end{itemize}
Que é da natureza do coral.
Que tem origem nos coraes.
\section{Corcovadura}
\begin{itemize}
\item {Grp. gram.:f.}
\end{itemize}
Acto ou effeito de corcovar.
Corcunda. Cf. B. Pereira, \textunderscore Prosodia\textunderscore , vb. \textunderscore lordosis\textunderscore .
\section{Corcunhas, de}
\begin{itemize}
\item {Grp. gram.:loc. adv.}
\end{itemize}
\begin{itemize}
\item {Utilização:Prov.}
\end{itemize}
\begin{itemize}
\item {Utilização:trasm.}
\end{itemize}
De cócoras.
(Talvez de \textunderscore cócoras\textunderscore  &lt; \textunderscore cocorunhas\textunderscore  &lt; \textunderscore cocrunhas\textunderscore  &lt; \textunderscore corcunhas\textunderscore )
\section{Corectopia}
\begin{itemize}
\item {Grp. gram.:f.}
\end{itemize}
\begin{itemize}
\item {Utilização:Med.}
\end{itemize}
\begin{itemize}
\item {Proveniência:(Do gr. \textunderscore kore\textunderscore  + \textunderscore ek\textunderscore  + \textunderscore topos\textunderscore )}
\end{itemize}
Situação anómala da pupilla, fóra do centro da íris.
\section{Cornelina}
\begin{itemize}
\item {Grp. gram.:f.}
\end{itemize}
O mesmo que \textunderscore cornalina\textunderscore . Cf. B. Pereira, \textunderscore Prosodia\textunderscore , vb. \textunderscore onix\textunderscore .
\section{Coronarite}
\begin{itemize}
\item {Grp. gram.:f.}
\end{itemize}
\begin{itemize}
\item {Utilização:Med.}
\end{itemize}
Arterite, nas artérias coronaes.
\section{Cororô}
\begin{itemize}
\item {Grp. gram.:m.}
\end{itemize}
\begin{itemize}
\item {Utilização:Bras. do Rio}
\end{itemize}
Camada de arroz esturrado, que fica adherente á panela ou a outra vasilha, em que é cozinhado.
\section{Covelite}
\begin{itemize}
\item {Grp. gram.:f.}
\end{itemize}
Espécie de calcosina azul escura.
\section{Covellite}
\begin{itemize}
\item {Grp. gram.:f.}
\end{itemize}
Espécie de chalcosina azul escura.
\section{Coxeira}
\begin{itemize}
\item {Grp. gram.:f.}
\end{itemize}
\begin{itemize}
\item {Proveniência:(De \textunderscore coxo\textunderscore ^1)}
\end{itemize}
Manqueira de animaes.
\section{Coxite}
\begin{itemize}
\item {Grp. gram.:f.}
\end{itemize}
\begin{itemize}
\item {Utilização:Med.}
\end{itemize}
\begin{itemize}
\item {Proveniência:(De \textunderscore coxa\textunderscore )}
\end{itemize}
Arthrite aguda coxo-femoral.
\section{Cranho}
\begin{itemize}
\item {Grp. gram.:m.}
\end{itemize}
\begin{itemize}
\item {Utilização:Ant.}
\end{itemize}
O mesmo que \textunderscore crânio\textunderscore . Cf. \textunderscore Rev. Lus.\textunderscore , XV, 2.
\section{Cranioclasia}
\begin{itemize}
\item {Grp. gram.:f.}
\end{itemize}
\begin{itemize}
\item {Utilização:Med.}
\end{itemize}
\begin{itemize}
\item {Proveniência:(Do gr. \textunderscore kranion\textunderscore  + \textunderscore klasis\textunderscore )}
\end{itemize}
Esmagamento da cabeça do feto na cavidade pelviana.
\section{Cranioclasta}
\begin{itemize}
\item {Grp. gram.:m.}
\end{itemize}
\begin{itemize}
\item {Utilização:Med.}
\end{itemize}
Instrumento, para extrahir a cabeça do feto, depois de se lhe reduzir o volume, sem a esmagar completamente.
(Cp. \textunderscore cranioclasia\textunderscore )
\section{Craniópago}
\begin{itemize}
\item {Grp. gram.:m.}
\end{itemize}
\begin{itemize}
\item {Utilização:Terat.}
\end{itemize}
\begin{itemize}
\item {Proveniência:(Do gr. \textunderscore kranion\textunderscore  + \textunderscore pageis\textunderscore )}
\end{itemize}
Nome dos monstros duplos, ligados pela extremidade cephálica.
\section{Craniotabes}
\begin{itemize}
\item {Grp. gram.:f.}
\end{itemize}
\begin{itemize}
\item {Utilização:Med.}
\end{itemize}
\begin{itemize}
\item {Proveniência:(De \textunderscore crânio\textunderscore  + \textunderscore tabes\textunderscore )}
\end{itemize}
Deformação dos ossos do crânio, com depressão dos tecidos.
\section{Crapiela}
\begin{itemize}
\item {Grp. gram.:f.}
\end{itemize}
\begin{itemize}
\item {Utilização:Burl.}
\end{itemize}
O mesmo que \textunderscore piela\textunderscore , bebedeira.
\section{Cremnofobia}
\begin{itemize}
\item {Grp. gram.:f.}
\end{itemize}
\begin{itemize}
\item {Utilização:Med.}
\end{itemize}
\begin{itemize}
\item {Proveniência:(Do gr. \textunderscore kremnos\textunderscore  + \textunderscore phobos\textunderscore )}
\end{itemize}
Mêdo mórbido dos precipícios.
\section{Cremnophobia}
\begin{itemize}
\item {Grp. gram.:f.}
\end{itemize}
\begin{itemize}
\item {Utilização:Med.}
\end{itemize}
\begin{itemize}
\item {Proveniência:(Do gr. \textunderscore kremnos\textunderscore  + \textunderscore phobos\textunderscore )}
\end{itemize}
Mêdo mórbido dos precipícios.
\section{Crépis}
\begin{itemize}
\item {Grp. gram.:m.}
\end{itemize}
Gênero de plantas de jardins.
\section{Crines}
\begin{itemize}
\item {Grp. gram.:m. pl.}
\end{itemize}
\begin{itemize}
\item {Utilização:Des.}
\end{itemize}
\begin{itemize}
\item {Proveniência:(Lat. \textunderscore crines\textunderscore )}
\end{itemize}
Cabellos; cabelleira. Cf. G. P. Castro, \textunderscore Ulysseia\textunderscore , VIII, 150.
\section{Crista-de-gallo}
\begin{itemize}
\item {Grp. gram.:f.}
\end{itemize}
O mesmo que \textunderscore gallacrista\textunderscore .
\section{Cristata}
\begin{itemize}
\item {Grp. gram.:f.}
\end{itemize}
O mesmo que \textunderscore gallacrista\textunderscore .
\section{Crucite}
\begin{itemize}
\item {Grp. gram.:f.}
\end{itemize}
\begin{itemize}
\item {Proveniência:(Do lat. \textunderscore crux\textunderscore , \textunderscore crucis\textunderscore )}
\end{itemize}
O mesmo que \textunderscore macla\textunderscore , também chamada \textunderscore pedra de cruz\textunderscore , porque os prismas, em que crystalliza, têm no interior uma espécie de cruz.
\section{Crianestesia}
\begin{itemize}
\item {Grp. gram.:f.}
\end{itemize}
\begin{itemize}
\item {Utilização:Med.}
\end{itemize}
\begin{itemize}
\item {Proveniência:(Do gr. \textunderscore kruos\textunderscore  + \textunderscore an\textunderscore  + \textunderscore aisthesis\textunderscore )}
\end{itemize}
Anestesia pelo frio.
\section{Criptorquídeo}
\begin{itemize}
\item {Grp. gram.:adj.}
\end{itemize}
\begin{itemize}
\item {Utilização:Veter.}
\end{itemize}
\begin{itemize}
\item {Proveniência:(Do gr. \textunderscore khruptos\textunderscore  + \textunderscore orkhis\textunderscore )}
\end{itemize}
Diz-se do cavalo, cujos testículos mal se vêem, por estarem muito recolhidos.
\section{Cristalofobia}
\begin{itemize}
\item {Grp. gram.:f.}
\end{itemize}
\begin{itemize}
\item {Utilização:Med.}
\end{itemize}
\begin{itemize}
\item {Proveniência:(Do gr. \textunderscore krustallos\textunderscore  + \textunderscore phobos\textunderscore )}
\end{itemize}
Terror mórbido dos objectos de vidro.
\section{Cruorina}
\begin{itemize}
\item {Grp. gram.:f.}
\end{itemize}
\begin{itemize}
\item {Proveniência:(Do lat. \textunderscore cruor\textunderscore )}
\end{itemize}
O mesmo que \textunderscore hematosina\textunderscore .
\section{Crupal}
\begin{itemize}
\item {Grp. gram.:adj.}
\end{itemize}
\begin{itemize}
\item {Utilização:Med.}
\end{itemize}
\begin{itemize}
\item {Proveniência:(De \textunderscore crupe\textunderscore )}
\end{itemize}
Diz-se da inflammação, acompanhada de falsas membranas: \textunderscore enterite crupal\textunderscore .
\section{Cruzes!}
\begin{itemize}
\item {Grp. gram.:interj.}
\end{itemize}
(Serve para afugentar o demónio; para indicar espanto ou repulsão, etc.)
\section{Cruzes, canhoto!}
\begin{itemize}
\item {Grp. gram.:interj.}
\end{itemize}
(Serve para afugentar o demónio; para indicar espanto ou repulsão, etc.)
\section{Cryanesthesia}
\begin{itemize}
\item {Grp. gram.:f.}
\end{itemize}
\begin{itemize}
\item {Utilização:Med.}
\end{itemize}
\begin{itemize}
\item {Proveniência:(Do gr. \textunderscore kruos\textunderscore  + \textunderscore an\textunderscore  + \textunderscore aisthesis\textunderscore )}
\end{itemize}
Anestesia pelo frio.
\section{Cryptorchídeo}
\begin{itemize}
\item {fónica:qui}
\end{itemize}
\begin{itemize}
\item {Grp. gram.:adj.}
\end{itemize}
\begin{itemize}
\item {Utilização:Veter.}
\end{itemize}
\begin{itemize}
\item {Proveniência:(Do gr. \textunderscore khruptos\textunderscore  + \textunderscore orkhis\textunderscore )}
\end{itemize}
Diz-se do cavallo, cujos testículos mal se vêem, por estarem muito recolhidos.
\section{Crystallophobia}
\begin{itemize}
\item {Grp. gram.:f.}
\end{itemize}
\begin{itemize}
\item {Utilização:Med.}
\end{itemize}
\begin{itemize}
\item {Proveniência:(Do gr. \textunderscore krustallos\textunderscore  + \textunderscore phobos\textunderscore )}
\end{itemize}
Terror mórbido dos objectos de vidro.
\section{Cuco-rabilongo}
\begin{itemize}
\item {Grp. gram.:m.}
\end{itemize}
Ave, também conhecida por \textunderscore pêga-cuca\textunderscore , (\textunderscore cocytes glandarius\textunderscore ). Cf. P. Moraes, \textunderscore Zool. Elem.\textunderscore , 296.
\section{Cúfea}
\begin{itemize}
\item {Grp. gram.:f.}
\end{itemize}
Gênero de plantas de jardim.
\section{Cupido}
\begin{itemize}
\item {Grp. gram.:m.}
\end{itemize}
\begin{itemize}
\item {Utilização:Bras. do Rio}
\end{itemize}
Formiga branca, o mesmo que \textunderscore cupim\textunderscore .
\section{Cúphea}
\begin{itemize}
\item {Grp. gram.:f.}
\end{itemize}
Gênero de plantas de jardim.
\section{Cupro-níquel}
\begin{itemize}
\item {Grp. gram.:m.}
\end{itemize}
Arseniureto de níquel natural.
\section{Curau}
\begin{itemize}
\item {Utilização:Bras. de San-Paulo}
\end{itemize}
Espécie de angu, feito de milho verde, moído e cosido com açúcar.
\section{Cureta}
\begin{itemize}
\item {fónica:curê}
\end{itemize}
\begin{itemize}
\item {Grp. gram.:f.}
\end{itemize}
Instrumento cirúrgico, em fórma de colhér ou pá, para extrahir do organismo animal um corpo estranho.
\section{Curicurê}
\begin{itemize}
\item {Grp. gram.:m.}
\end{itemize}
\begin{itemize}
\item {Utilização:Bras}
\end{itemize}
Espécie de lagarta, muito nociva aos algodoaes.
\section{Curral}
\begin{itemize}
\item {Utilização:T. da Madeira}
\end{itemize}
\begin{itemize}
\item {Utilização:Ant.}
\end{itemize}
Valle entre montes, de communicação diffícil.
Residência ou capital de alguns régulos africanos.
\section{Cursante}
\begin{itemize}
\item {Grp. gram.:m.}
\end{itemize}
\begin{itemize}
\item {Proveniência:(De \textunderscore cursar\textunderscore )}
\end{itemize}
O mesmo que \textunderscore cursista\textunderscore :«\textunderscore qual cursante das aulas...\textunderscore »Filinto, \textunderscore Fáb. de Lafont.\textunderscore , II, 142.
\section{Cutelaço}
\begin{itemize}
\item {Grp. gram.:m.}
\end{itemize}
Grande cutelo de magarefe. Cf. B. Pereira, \textunderscore Prosodia\textunderscore , vb. \textunderscore secespita\textunderscore .
\section{Cutellaço}
\begin{itemize}
\item {Grp. gram.:m.}
\end{itemize}
Grande cutello de magarefe. Cf. B. Pereira, \textunderscore Prosodia\textunderscore , vb. \textunderscore secespita\textunderscore .
\section{Cutisação}
\begin{itemize}
\item {Grp. gram.:f.}
\end{itemize}
\begin{itemize}
\item {Utilização:Med.}
\end{itemize}
\begin{itemize}
\item {Proveniência:(De \textunderscore cútis\textunderscore )}
\end{itemize}
Passagem de uma mucosa a estado análogo ao da pelle.
\section{Cutite}
\begin{itemize}
\item {Grp. gram.:f.}
\end{itemize}
\begin{itemize}
\item {Utilização:Med.}
\end{itemize}
\begin{itemize}
\item {Proveniência:(De \textunderscore cútis\textunderscore )}
\end{itemize}
Inflammação da derme, na erysipela.
\section{C}
\begin{itemize}
\item {fónica:cê}
\end{itemize}
\begin{itemize}
\item {Grp. gram.:m.}
\end{itemize}
\begin{itemize}
\item {Grp. gram.:Adj.}
\end{itemize}
\begin{itemize}
\item {Utilização:Mús.}
\end{itemize}
Terceira letra do alphabeto português.
Sinal de compasso quaternário, em música.
Sinal de \textunderscore cem\textunderscore  na numeração romana.
Terceiro, (falando-se de um número ou de um objecto, que faz parte de uma série).
Que é de terceira classe, (falando-se de carruagens dos caminhos de ferro).
Segundo grau da escala, na antiga notação romana.
\section{Cá}
\begin{itemize}
\item {Grp. gram.:adv.}
\end{itemize}
\begin{itemize}
\item {Proveniência:(Do lat. \textunderscore eccum\textunderscore  + \textunderscore hac\textunderscore )}
\end{itemize}
Aqui; neste logar.
Para este logar.
Entre nós.
A nós.
A mim.
\section{Ca}
\begin{itemize}
\item {Grp. gram.:conj.}
\end{itemize}
\begin{itemize}
\item {Utilização:Ant.}
\end{itemize}
Que, porquê.
(Ainda us. no Algarve. Do ant. fr. \textunderscore ca\textunderscore , hoje \textunderscore car\textunderscore , se não é alter. do port. \textunderscore que\textunderscore )
\section{Cã}
\begin{itemize}
\item {Grp. gram.:f.}
\end{itemize}
(V. \textunderscore can\textunderscore ^1)
\section{Caá}
\begin{itemize}
\item {Grp. gram.:m.  e  f.}
\end{itemize}
\begin{itemize}
\item {Utilização:Bras}
\end{itemize}
Designação genérica dos vegetaes.
(Do tupi)
\section{Caá-ataia}
\begin{itemize}
\item {Grp. gram.:f.}
\end{itemize}
Planta purgativa do Brasil.
\section{Caaba}
\begin{itemize}
\item {Grp. gram.:f.}
\end{itemize}
\begin{itemize}
\item {Proveniência:(Do ár. \textunderscore kaabet\textunderscore , casa)}
\end{itemize}
Edifício religioso em Meca, da particular veneração dos Muçulmanos.
\section{Caá-guacuba}
\begin{itemize}
\item {Grp. gram.:f.}
\end{itemize}
Arvoreta brasileira, cujas flores têm a fórma e o cheiro das da tília.
\section{Caaiguazu}
\begin{itemize}
\item {Grp. gram.:m.}
\end{itemize}
Espécie de tatu do Brasil.
\section{Caama}
\begin{itemize}
\item {Grp. gram.:m.}
\end{itemize}
Quadrúpede do gênero antilope.
\section{Caá-membeca}
\begin{itemize}
\item {Grp. gram.:f.}
\end{itemize}
\begin{itemize}
\item {Utilização:Bras}
\end{itemize}
Planta polygálea e medicinal.
\section{Caá-opiá}
\begin{itemize}
\item {Grp. gram.:m.}
\end{itemize}
\begin{itemize}
\item {Utilização:Bras}
\end{itemize}
O mesmo que \textunderscore pau-de-lacre\textunderscore .
\section{Caá-puan}
\begin{itemize}
\item {Grp. gram.:m.}
\end{itemize}
\begin{itemize}
\item {Utilização:Bras}
\end{itemize}
Ilha de mato.
\section{Caarina}
\begin{itemize}
\item {Grp. gram.:f.}
\end{itemize}
\begin{itemize}
\item {Utilização:Bras}
\end{itemize}
Raiz da mandioca.
\section{Caá-tininga}
\begin{itemize}
\item {Grp. gram.:f.}
\end{itemize}
Árvore medicinal do Alto-Amazonas.
\section{Caavurana}
\begin{itemize}
\item {Grp. gram.:f.}
\end{itemize}
Planta solânea medicinal do Brasil.
\section{Caba}
\begin{itemize}
\item {Grp. gram.:f.}
\end{itemize}
Insecto do Brasil, espécie de abelha.
\section{Cabaça}
\begin{itemize}
\item {Grp. gram.:f.}
\end{itemize}
\begin{itemize}
\item {Utilização:Pop.}
\end{itemize}
Fruto de uma planta cucurbitácea, mais ou menos parecido a uma grande pêra, e formado geralmente de dois bojos, o superior dos quaes é mais pequeno que o inferior.
Vasilha, em que se transforma esse fruto depois de sêco e que serve principalmente para vinho.
Vaso, utensílio, enfeite, que tem a fórma de cabaça.
Espécie de abóbora.
Variedade de pêra.
Brinco ou pingente das orelhas.
(Cp. cast. \textunderscore calabaza\textunderscore )
\section{Cabaça}
\begin{itemize}
\item {Grp. gram.:m.}
\end{itemize}
\begin{itemize}
\item {Utilização:bras}
\end{itemize}
Aquelle dos gêmeos, que nasce em segundo lugar.--O que nasce primeiro diz-se \textunderscore caculo\textunderscore .
\section{Cabaçada}
\begin{itemize}
\item {Grp. gram.:f.}
\end{itemize}
Líquido, que uma cabaça ou cabaço contém ou póde conter: \textunderscore embriagar-se com uma cabaçada de vinho\textunderscore .
\section{Cabaçal}
\begin{itemize}
\item {Grp. gram.:adj.}
\end{itemize}
\begin{itemize}
\item {Proveniência:(De \textunderscore cabaço\textunderscore )}
\end{itemize}
Diz-se de uma variedade de maçans grandes e de uma variedade de pêra pouco apreciada.
\section{Cabação}
\begin{itemize}
\item {Grp. gram.:m.}
\end{itemize}
\begin{itemize}
\item {Utilização:T. da Chamusca}
\end{itemize}
\begin{itemize}
\item {Proveniência:(De \textunderscore cabaça\textunderscore )}
\end{itemize}
Pimento grande.
\section{Cabaceira}
\begin{itemize}
\item {Grp. gram.:f.}
\end{itemize}
\begin{itemize}
\item {Utilização:Pesc.}
\end{itemize}
\begin{itemize}
\item {Utilização:T. da Guiné e de San-Thomé}
\end{itemize}
\begin{itemize}
\item {Proveniência:(De \textunderscore cabaça\textunderscore )}
\end{itemize}
Planta cucurbitácea, que produz cabaças.
Espécie de rêde de saco, com armadilha na bôca.
O mesmo que \textunderscore embondeiro\textunderscore .
\section{Cabaceiro}
\begin{itemize}
\item {Grp. gram.:m.}
\end{itemize}
(V.cabaceira)
\section{Cabacinha}
\begin{itemize}
\item {Grp. gram.:f.}
\end{itemize}
\begin{itemize}
\item {Utilização:Bras}
\end{itemize}
\begin{itemize}
\item {Grp. gram.:Pl.}
\end{itemize}
\begin{itemize}
\item {Proveniência:(De \textunderscore cabaça\textunderscore )}
\end{itemize}
Planta cucurbitácea, variedade de cabaceira.
Fruto dessa planta, distinto da cabaça, não só por sêr mais pequeno, mas também por têr a superfície listrada ou verrugosa.
Bóla de cera, cheia de água, para os folguedos de Entrudo.
Pequenos pingentes ou brincos das orelhas.
\section{Cabacinha-riscada}
\begin{itemize}
\item {Grp. gram.:f.}
\end{itemize}
Variedade de pêra beirôa, muito apreciada.
\section{Cabacinho}
\begin{itemize}
\item {Grp. gram.:m.}
\end{itemize}
\begin{itemize}
\item {Proveniência:(De \textunderscore cabaço\textunderscore ^1)}
\end{itemize}
Planta cucurbitácea do Brasil.
\section{Cabaço}
\begin{itemize}
\item {Grp. gram.:m.}
\end{itemize}
\begin{itemize}
\item {Utilização:Prov.}
\end{itemize}
\begin{itemize}
\item {Utilização:minh.}
\end{itemize}
\begin{itemize}
\item {Utilização:Prov.}
\end{itemize}
\begin{itemize}
\item {Utilização:minh.}
\end{itemize}
\begin{itemize}
\item {Utilização:T. de Coímbra e do Brasil}
\end{itemize}
O mesmo que \textunderscore cabaça\textunderscore ^1.
Cabaça oblonga, de bojos pouco salientes.
Nome de várias plantas cucurbitáceas do Brasil.
Regador, feito de lata ou de cabaça sêca, com que se extrái água de poços e represas, lançando-a para um sulco, que a leva ao terreno que se deseja regar.
Nome de um peixe.
Parte cylíndrica horizontal de vessadoiro, limitada adeante pela relha, continuada atrás pela rabiça e formando com esta ângulo obtuso.
Medida de líquidos, equivalente a um cântaro ou a 24 quartilhos.
Variedade de pêra ordinária.
Marçano de mercearia.
Aprendiz de caixeiro.
(Cp. \textunderscore cabaça\textunderscore ^1)
\section{Cabaço}
\begin{itemize}
\item {Grp. gram.:m.}
\end{itemize}
\begin{itemize}
\item {Utilização:bras}
\end{itemize}
\begin{itemize}
\item {Utilização:Pop.}
\end{itemize}
O mesmo que \textunderscore virgindade\textunderscore .
(Do quimbundo \textunderscore cabásu\textunderscore )
\section{Cabaço}
\begin{itemize}
\item {Grp. gram.:m.}
\end{itemize}
\begin{itemize}
\item {Utilização:Pop.}
\end{itemize}
\textunderscore Dar o cabaço\textunderscore , recusar a mão da noiva a quem a pediu.
Arrepender-se de têr promettido casamento; faltar á promessa de casamento.
\textunderscore Levar o cabaço\textunderscore , receber a recusa de um pedido de casamento.
(Cp. o cast. \textunderscore llevar calabazas\textunderscore , com o mesmo significado)
\section{Cabaçuda}
\begin{itemize}
\item {Grp. gram.:f.}
\end{itemize}
\begin{itemize}
\item {Utilização:Bras}
\end{itemize}
\begin{itemize}
\item {Proveniência:(De \textunderscore cabaço\textunderscore ^3)}
\end{itemize}
Mulher virgem.
\section{Cabaçudo}
\begin{itemize}
\item {Grp. gram.:m.  e  adj.}
\end{itemize}
\begin{itemize}
\item {Utilização:Bras}
\end{itemize}
Novato, simples, ingênuo, como as vírgens.
(Cp. \textunderscore cabaçuda\textunderscore )
\section{Cabada}
\begin{itemize}
\item {Grp. gram.:f.}
\end{itemize}
Farda militar dos Gregos modernos.
\section{Cabadela}
\begin{itemize}
\item {Grp. gram.:f.}
\end{itemize}
(V.cabidela)
\section{Cabaia}
\begin{itemize}
\item {Grp. gram.:f.}
\end{itemize}
\begin{itemize}
\item {Proveniência:(Do ár. \textunderscore cabá\textunderscore )}
\end{itemize}
Vestuário de grandes mangas, aberto ao lado e usado na China e noutros povos orientaes.
\section{Cabaíbas}
\begin{itemize}
\item {Grp. gram.:m. pl.}
\end{itemize}
Tríbo de índios do Brasil, nas margens do rio Arinos.
\section{Cabal}
\begin{itemize}
\item {Grp. gram.:adj.}
\end{itemize}
\begin{itemize}
\item {Proveniência:(De \textunderscore cabo\textunderscore )}
\end{itemize}
Completo; rigoroso; pleno: \textunderscore resposta cabal\textunderscore .
\section{Cabal}
\begin{itemize}
\item {Grp. gram.:m.}
\end{itemize}
Supposta alimária de Java, a cujos ossos os Jaus attribuíam virtudes sobrenaturaes. Cf. Barros, Déc. II, l. VI, c. 2.
\section{Cabala}
\begin{itemize}
\item {Grp. gram.:f.}
\end{itemize}
\begin{itemize}
\item {Proveniência:(Hebr. \textunderscore kabala\textunderscore ?)}
\end{itemize}
Systema hebraico de interpretação bíblica.
Sciência occulta.
Conlúio ou intriga secreta, entre indivíduos que conspiram para o mesmo fim.
\section{Cábala}
\begin{itemize}
\item {Grp. gram.:f.}
\end{itemize}
\begin{itemize}
\item {Proveniência:(Hebr. \textunderscore kabala\textunderscore ?)}
\end{itemize}
Systema hebraico de interpretação bíblica.
Sciência occulta.
Conlúio ou intriga secreta, entre indivíduos que conspiram para o mesmo fim.
\section{Cabalacaxengo}
\begin{itemize}
\item {Grp. gram.:m.}
\end{itemize}
Pássaro conirostro da África occidental.
\section{Cabalar}
\begin{itemize}
\item {Grp. gram.:v. i.}
\end{itemize}
Fazer cabalas, entrar em cabalas.
\section{Cabaleta}
\begin{itemize}
\item {fónica:lê}
\end{itemize}
\begin{itemize}
\item {Grp. gram.:f.}
\end{itemize}
\begin{itemize}
\item {Proveniência:(It. \textunderscore cabaletta\textunderscore )}
\end{itemize}
Trecho ligeiro de música, de rythmo animado.
\section{Cabalim}
\begin{itemize}
\item {Grp. gram.:m.}
\end{itemize}
Espécie de tecido antigo.
\section{Cabalino}
\begin{itemize}
\item {Grp. gram.:adj.}
\end{itemize}
\begin{itemize}
\item {Proveniência:(Lat. \textunderscore caballinus\textunderscore )}
\end{itemize}
Relativo ao cavallo Pégaso.
E diz-se de um aloés, que se usou, ou se suppôs sêr usado, em Veterinária.
\section{Cabalista}
\begin{itemize}
\item {Grp. gram.:m.  e  f.}
\end{itemize}
\begin{itemize}
\item {Proveniência:(De \textunderscore cabala\textunderscore )}
\end{itemize}
Pessôa, dada a sciências occultas, á cabala.
\section{Cabalisticamente}
\begin{itemize}
\item {Grp. gram.:adv.}
\end{itemize}
De modo \textunderscore cabalístico\textunderscore .
\section{Cabalístico}
\begin{itemize}
\item {Grp. gram.:adj.}
\end{itemize}
\begin{itemize}
\item {Proveniência:(De \textunderscore cabalista\textunderscore )}
\end{itemize}
Relativo á cabala dos Hebreus ou ás artes occultas, á magia.
Secreto, mysterioso.
\section{Caballino}
\begin{itemize}
\item {Grp. gram.:adj.}
\end{itemize}
\begin{itemize}
\item {Proveniência:(Lat. \textunderscore caballinus\textunderscore )}
\end{itemize}
Relativo ao cavallo Pégaso.
E diz-se de um aloés, que se usou, ou se suppôs sêr usado, em Veterinária.
\section{Cabalmente}
\begin{itemize}
\item {Grp. gram.:adv.}
\end{itemize}
De modo \textunderscore cabal\textunderscore .
\section{Cabana}
\begin{itemize}
\item {Grp. gram.:f.}
\end{itemize}
\begin{itemize}
\item {Utilização:Ant.}
\end{itemize}
\begin{itemize}
\item {Proveniência:(Do b. lat. \textunderscore capanna\textunderscore )}
\end{itemize}
Pequena casa rústica; choupana; tugúrio; choça; colmado.
Rebanho, manada.
\section{Cabanada}
\begin{itemize}
\item {Grp. gram.:f.}
\end{itemize}
\begin{itemize}
\item {Utilização:Bras}
\end{itemize}
\begin{itemize}
\item {Proveniência:(De \textunderscore cabano\textunderscore ^2)}
\end{itemize}
Revolução popular do Pará, em 1832.
Sedição de Pernambuco e Alagôas, de 1832 a 1835.
\section{Cabanagem}
\begin{itemize}
\item {Grp. gram.:f.}
\end{itemize}
\begin{itemize}
\item {Utilização:Bras. do N}
\end{itemize}
\begin{itemize}
\item {Utilização:Bras}
\end{itemize}
\begin{itemize}
\item {Proveniência:(De \textunderscore cabano\textunderscore ^2)}
\end{itemize}
Partido dos cabanos em Alagôas.
Acto de cabano, selvajaria.
\section{Cabanal}
\begin{itemize}
\item {Grp. gram.:m.}
\end{itemize}
\begin{itemize}
\item {Utilização:Prov.}
\end{itemize}
\begin{itemize}
\item {Utilização:trasm.}
\end{itemize}
\begin{itemize}
\item {Proveniência:(De \textunderscore cabana\textunderscore )}
\end{itemize}
Abrigo, coberto de colmo ou de telha, junto ás eiras, e no qual os lavradores recolhem, contra o mau tempo, os cereaes que hão de sêr debulhados.
Alpendre, sob o qual os lavradores abrigam o carro, lenha, etc.
Coberto, nas feiras, em que os tendeiros, ourives e outros expõem á venda as suas mercadorias.
\section{Cabane}
\begin{itemize}
\item {Grp. gram.:m.}
\end{itemize}
Um dos maquinismos das fábricas de tecidos. Cf. \textunderscore Inquér. Industr.\textunderscore , P. II, l. 3.^o, 41.
\section{Cabaneira}
\begin{itemize}
\item {Grp. gram.:f.}
\end{itemize}
\begin{itemize}
\item {Utilização:Prov.}
\end{itemize}
\begin{itemize}
\item {Utilização:minh.}
\end{itemize}
Mulher, que vive em cabana.
Mulher pobre.
Mulher solteira.
(Cp. \textunderscore cabaneiro\textunderscore )
\section{Cabaneiro}
\begin{itemize}
\item {Grp. gram.:m.}
\end{itemize}
\begin{itemize}
\item {Proveniência:(Do b. lat. \textunderscore capannarius\textunderscore )}
\end{itemize}
Aquelle que vive em cabana.
Homem pobre.
Grande cesto de vimes.
\section{Cabanejo}
\begin{itemize}
\item {Grp. gram.:m.}
\end{itemize}
Grande cesto de vêrga, também chamado \textunderscore cabaneiro\textunderscore .
\section{Cabanil}
\begin{itemize}
\item {Grp. gram.:m.}
\end{itemize}
\begin{itemize}
\item {Utilização:Prov.}
\end{itemize}
\begin{itemize}
\item {Utilização:beir.}
\end{itemize}
Resguardo de pedra ou madeira, em volta de uma planta para que o gado a não damnifique.
(Cp. \textunderscore cabaneiro\textunderscore )
\section{Cabanilho}
\begin{itemize}
\item {Grp. gram.:m.}
\end{itemize}
\begin{itemize}
\item {Utilização:Prov.}
\end{itemize}
\begin{itemize}
\item {Utilização:beir.}
\end{itemize}
Cesto com tampa, próprio para transporte de frutas.
(Cp. \textunderscore cabano\textunderscore ^1)
\section{Cabano}
\begin{itemize}
\item {Grp. gram.:adj.}
\end{itemize}
\begin{itemize}
\item {Grp. gram.:M.}
\end{itemize}
Diz-se do boi, cujas pontas são horizontaes ou um tanto voltadas para baixo.
E diz-se do cavallo, que tem as orelhas derrubadas.
O mesmo que \textunderscore cabaneiro\textunderscore , (cesto).
\section{Cabano}
\begin{itemize}
\item {Grp. gram.:m.}
\end{itemize}
\begin{itemize}
\item {Utilização:Bras}
\end{itemize}
Membro de certa facção política em Alagôas, Pernambuco e Pará.
\section{Cabarbanda}
\begin{itemize}
\item {Grp. gram.:f.}
\end{itemize}
\begin{itemize}
\item {Utilização:Ant.}
\end{itemize}
Cinto, tecido com oiro ou prata, e usado por Persas e Mongóes.
\section{Cabarro}
\begin{itemize}
\item {Grp. gram.:m.}
\end{itemize}
Inclinação das paredes de um vaso de latão, dando a êste a fórma de um cóne truncado.
\section{Cabasu}
\begin{itemize}
\item {Grp. gram.:m.}
\end{itemize}
Espécie de tatu.
\section{Cabaz}
\begin{itemize}
\item {Grp. gram.:m.}
\end{itemize}
\begin{itemize}
\item {Utilização:Pop.}
\end{itemize}
Cêsto de junco, de vêrga, de cana, etc., geralmente com tampa e asa arqueada.
Caixa de lata, para transporte de comidas.
Bebida alcoólica, formada de café, vinho, açúcar e canela.
(B. lat. \textunderscore cabacius\textunderscore , talvez do lat. \textunderscore capax\textunderscore )
\section{Cabazada}
\begin{itemize}
\item {Grp. gram.:f.}
\end{itemize}
Aquillo que enche um cabaz.
\section{Cabdel}
\textunderscore m.\textunderscore  (pl. \textunderscore cabdeles\textunderscore )
Designação antiga do que hoje se diz \textunderscore almirante\textunderscore .
(Cp. \textunderscore coudel\textunderscore )
\section{Cabe}
\begin{itemize}
\item {Grp. gram.:m.}
\end{itemize}
\begin{itemize}
\item {Utilização:Ant.}
\end{itemize}
\begin{itemize}
\item {Proveniência:(De \textunderscore caber\textunderscore )}
\end{itemize}
Distância entre duas bólas no jôgo do aro.
O passar a bóla além da raia dêsse jôgo.
Ardíl; astúcia inesperada, para obter um fim.
Ensejo, aso:«\textunderscore não deu cabe a demoras\textunderscore ». Filinto, \textunderscore D. Man.\textunderscore , I, 242.
\section{Cabear}
\begin{itemize}
\item {Grp. gram.:v. i.}
\end{itemize}
\begin{itemize}
\item {Proveniência:(De \textunderscore cabo\textunderscore )}
\end{itemize}
Agitar (o cavallo) a cauda, quando o picam.
\section{Cabeça}
\begin{itemize}
\item {fónica:bê}
\end{itemize}
\begin{itemize}
\item {Grp. gram.:f.}
\end{itemize}
\begin{itemize}
\item {Grp. gram.:M.}
\end{itemize}
\begin{itemize}
\item {Utilização:Gír. de Lisbôa.}
\end{itemize}
A parte mais elevada do corpo humano, e a parte mais anterior dos animaes irracionaes, a qual contém o cérebro, os órgãos da vista, audição, paladar e olfato.
A parte da cabeça, que se cobre de cabellos.
Intelligência; talento; tino: \textunderscore aquelle rapaz tem cabeça\textunderscore .
Pessôa intelligente ou sabedora: \textunderscore é uma grande cabeça\textunderscore .
Chefe. (Nesta accepção, \textunderscore cabeça\textunderscore  emprega-se muitas vezes como \textunderscore m.\textunderscore )
Lugar principal de uma circumscripção territorial: \textunderscore a cabeça do concelho\textunderscore .
Indivíduo ou animal, considerados numericamente: \textunderscore tem vinte cabeças no curral\textunderscore .
Extremidade superior de um objecto.
A parte superior de um objecto, mais larga que o todo.
Frente de um cortejo.
Titulo de um capitulo.
Objecto, que, pela sua fórma e collocação sôbre outros, dá ideia de uma cabeça pequena ou grande.
\textunderscore Cabeça de vento\textunderscore , indivíduo estouvado, leviano.
\textunderscore Cabeça de arvela\textunderscore , ou \textunderscore cabeça de andorinha\textunderscore , (o mesmo significado).
\textunderscore Cabeça de pau\textunderscore , indivíduo, que, em nome de outros, arremata objectos em leilão, entrando na partilha do lucro, depois de revendidos.
(B. lat. \textunderscore capitia\textunderscore , de \textunderscore caput\textunderscore )
\section{Cabeça-baixa}
\begin{itemize}
\item {Grp. gram.:m.  e  f.}
\end{itemize}
\begin{itemize}
\item {Utilização:Bras. do N}
\end{itemize}
Porco ou porca.
\section{Cabeça-chata}
\begin{itemize}
\item {Grp. gram.:m.  e  f.}
\end{itemize}
\begin{itemize}
\item {Utilização:Bras. do Maranhão}
\end{itemize}
Indivíduo do Ceará.
\section{Cabeçada}
\begin{itemize}
\item {Grp. gram.:f.}
\end{itemize}
\begin{itemize}
\item {Utilização:Pop.}
\end{itemize}
\begin{itemize}
\item {Proveniência:(De \textunderscore cabeça\textunderscore )}
\end{itemize}
Pancada com a cabeça.
Asneira, disparate.
Correias, com que se cinge e guarnece a cabeça das cavalgaduras.
Movimento súbito, para cima, da cabeça da cavalgadura.
Canastrada á cabeça.
Movimento de cabeça, affirmando ou approvando:«\textunderscore o major com cabeçadas affirmativas...\textunderscore »Camillo, \textunderscore Brasileira\textunderscore , 50.
\section{Cabeça-de-água}
\begin{itemize}
\item {Grp. gram.:f.}
\end{itemize}
\begin{itemize}
\item {Utilização:Pop.}
\end{itemize}
O mesmo que \textunderscore hydrocephalia\textunderscore .
\section{Cabeça-de-casal}
\begin{itemize}
\item {Grp. gram.:m.  e  f.}
\end{itemize}
Pessôa, encarregada de arrolar e descrever os bens de uma herança.
\section{Cabeça-de-frade}
\begin{itemize}
\item {Grp. gram.:f.}
\end{itemize}
\begin{itemize}
\item {Utilização:Bras}
\end{itemize}
Planta cactácea e alimentícia.
\section{Cabeça-de-moleque}
\begin{itemize}
\item {Grp. gram.:f.}
\end{itemize}
\begin{itemize}
\item {Utilização:Bras}
\end{itemize}
O mesmo que \textunderscore tejuco\textunderscore .
\section{Cabeça-de-negro}
\begin{itemize}
\item {Grp. gram.:m.}
\end{itemize}
\begin{itemize}
\item {Utilização:Bras}
\end{itemize}
Fruto de uma árvore anonácea do México.
O mesmo que \textunderscore tejuco\textunderscore .
\section{Cabeça-de-prego}
\begin{itemize}
\item {Grp. gram.:m.}
\end{itemize}
\begin{itemize}
\item {Utilização:Bras. do N}
\end{itemize}
O mesmo que \textunderscore girino\textunderscore .
\section{Cabeça-de-rubim}
\begin{itemize}
\item {Grp. gram.:m.}
\end{itemize}
\begin{itemize}
\item {Utilização:Bras}
\end{itemize}
Pássaro esverdeado de poupa carmezim.
\section{Cabeça-de-tremoço}
\begin{itemize}
\item {Grp. gram.:f.}
\end{itemize}
\begin{itemize}
\item {Utilização:Serralh.}
\end{itemize}
Espécie de parafuso.
\section{Cabeça-dura}
\begin{itemize}
\item {Grp. gram.:m.}
\end{itemize}
Peixe marítimo do Brasil.
\section{Cabeçal}
\begin{itemize}
\item {Grp. gram.:m.}
\end{itemize}
\begin{itemize}
\item {Utilização:Prov.}
\end{itemize}
\begin{itemize}
\item {Utilização:trasm.}
\end{itemize}
\begin{itemize}
\item {Proveniência:(De \textunderscore cabeça\textunderscore )}
\end{itemize}
Almofada, em que descansa a cabeça; cabeceira.
Chumaço, em volta de uma ferida, por baixo da ligadura.
Cada um dos quatro paus, que sustentavam a caixa dos antigos coches.
A peça de madeira, que segura o sino e que lhe fica sobreposta.
\section{Cabeçalha}
\begin{itemize}
\item {Grp. gram.:f.}
\end{itemize}
\begin{itemize}
\item {Utilização:Prov.}
\end{itemize}
\begin{itemize}
\item {Utilização:minh.}
\end{itemize}
\begin{itemize}
\item {Proveniência:(De \textunderscore cabeça\textunderscore )}
\end{itemize}
Temão de carro.
A extremidade deanteira do mesmo temão.
\section{Cabeçalhão}
\begin{itemize}
\item {Grp. gram.:m.}
\end{itemize}
\begin{itemize}
\item {Utilização:Prov.}
\end{itemize}
\begin{itemize}
\item {Utilização:trasm.}
\end{itemize}
Parte deanteira e curva da cabeçalha.
\section{Cabeçalho}
\begin{itemize}
\item {Grp. gram.:m.}
\end{itemize}
\begin{itemize}
\item {Proveniência:(De \textunderscore cabeça\textunderscore )}
\end{itemize}
O mesmo que \textunderscore cabeçalha\textunderscore .
Cabeceira.
Titulo (de capitulo, de artigo, etc.).
Titulo (de jornal) com o sub-titulo e annexos, que estão sempre compostos, para todos os números da folha.
Palavreado, que serve de exórdio a uma carta ou a outro escrito. Cf. Filinto, XII, 305.
\section{Cabeção}
\begin{itemize}
\item {Grp. gram.:m.}
\end{itemize}
\begin{itemize}
\item {Utilização:Ant.}
\end{itemize}
\begin{itemize}
\item {Proveniência:(De \textunderscore cabeça\textunderscore )}
\end{itemize}
Golla larga e pendente.
Colleira de sacerdote.
Collarinho largo de senhoras.
Cabresto, com duas rédeas de lan e um pequeno arco de ferro, para domar e governar uma cavalgadura, sem lhe ferir a bôca.
Almofadão.
Vinheta no frontispício de um livro.
\section{Cabeçaria}
\begin{itemize}
\item {Grp. gram.:f.}
\end{itemize}
\begin{itemize}
\item {Proveniência:(De \textunderscore cabeça\textunderscore )}
\end{itemize}
Pedras para alicerces, grosseiramente apparelhadas.
\section{Cabeceador}
\begin{itemize}
\item {Grp. gram.:m.  e  adj.}
\end{itemize}
O que cabeceia.
\section{Cabecear}
\begin{itemize}
\item {Grp. gram.:v. i.}
\end{itemize}
Mover a cabeça.
Deixá-la pender e erguê-la, alternadamente: \textunderscore cabecear com somno\textunderscore .
Inclinar-se.
\section{Cabeceira}
\begin{itemize}
\item {Grp. gram.:f.}
\end{itemize}
\begin{itemize}
\item {Utilização:Ant.}
\end{itemize}
\begin{itemize}
\item {Grp. gram.:Pl.}
\end{itemize}
\begin{itemize}
\item {Proveniência:(De \textunderscore cabeça\textunderscore )}
\end{itemize}
Almofada, lugar, em que descansa a cabeça.
Lado da cama, para onde se deita a cabeça.
O lado mais estreito da mesa, quando oblonga.
Lugar, occupado á mesa pelo amphitryão ou pelo mais graduado dos convivas.
Lado da sepultura, para onde fica a cabeça do defunto.
Ornato ou contraforte, em fórma de cordão, na lombada de alguns livros encadernados.
O princípio ou a primeira linha (de uma relação).
Um dos compartimentos das marinhas.
Chefe, caudilho.
Região vizinha da nascente de um rio: \textunderscore nas cabeceiras do Nilo\textunderscore .
\section{Cabeceiro}
\begin{itemize}
\item {Grp. gram.:m.}
\end{itemize}
\begin{itemize}
\item {Utilização:T. da Bairrada}
\end{itemize}
Cada uma das extremidades de uma leira de terra.
(Cp. \textunderscore cabeceira\textunderscore )
\section{Cabecel}
\begin{itemize}
\item {Grp. gram.:m.}
\end{itemize}
\begin{itemize}
\item {Proveniência:(De \textunderscore cabeça\textunderscore )}
\end{itemize}
Quinhoeiro principal de uma propriedade indívisa, o qual tinha a seu cargo o pagamento integral do respectivo foro ao senhorio directo, e os pagamentos do rendimento aos demais quinhoeiros.
Cabeçalho ou vinheta, por cima de um escrito.
\section{Cabecelaria}
\begin{itemize}
\item {Grp. gram.:f.}
\end{itemize}
Encargo de cabecel.
\section{Cabecilha}
\begin{itemize}
\item {Grp. gram.:m.}
\end{itemize}
\begin{itemize}
\item {Proveniência:(De \textunderscore cabeça\textunderscore )}
\end{itemize}
Caudilho, chefe.
\section{Cabecinha}
\begin{itemize}
\item {Grp. gram.:f.}
\end{itemize}
\begin{itemize}
\item {Proveniência:(De \textunderscore cabeça\textunderscore )}
\end{itemize}
Farinha, que fica no peneiro, depois de separada a sêmea, quando se peneira o rolão.
Nome madeirense, que, acompanhado de epíthetos, designa várias aves: \textunderscore cabecinha encarnada\textunderscore  e \textunderscore cabecinha rosada\textunderscore , o pintasilgo; \textunderscore cabecinha negra\textunderscore , a toutinegra.
\section{Cabecinha-encarnada}
\begin{itemize}
\item {Grp. gram.:f.}
\end{itemize}
\begin{itemize}
\item {Utilização:Mad}
\end{itemize}
O mesmo que \textunderscore pintasilgo\textunderscore .
\section{Cabecinha-rosada}
\begin{itemize}
\item {Grp. gram.:f.}
\end{itemize}
O mesmo que \textunderscore cabecinha-encarnada\textunderscore .
\section{Cabeço}
\begin{itemize}
\item {fónica:bê}
\end{itemize}
\begin{itemize}
\item {Grp. gram.:m.}
\end{itemize}
O ponto arredondado, e mais alto, de um monte.
Monte pequeno; oiteiro.
(B. lat. \textunderscore capitium\textunderscore , do lat. \textunderscore caput\textunderscore )
\section{Cabeçorra}
\begin{itemize}
\item {fónica:çô}
\end{itemize}
\begin{itemize}
\item {Grp. gram.:f.}
\end{itemize}
\begin{itemize}
\item {Utilização:Pop.}
\end{itemize}
Cabeça grande.
\section{Cabeçote}
\begin{itemize}
\item {Grp. gram.:m.}
\end{itemize}
\begin{itemize}
\item {Proveniência:(De \textunderscore cabeça\textunderscore )}
\end{itemize}
Cada uma das testeiras do banco, sôbre que trabalham marceneiros e carpinteiros.
\section{Cabeçudo}
\begin{itemize}
\item {Grp. gram.:adj.}
\end{itemize}
\begin{itemize}
\item {Utilização:Fig.}
\end{itemize}
\begin{itemize}
\item {Grp. gram.:M.}
\end{itemize}
\begin{itemize}
\item {Utilização:Pop.}
\end{itemize}
\begin{itemize}
\item {Proveniência:(De \textunderscore cabeça\textunderscore )}
\end{itemize}
Que tem cabeça grande.
Teimoso; casmurro.
Homem teimoso.
Peixe de Portugal.
O mesmo que \textunderscore girino\textunderscore .
\section{Cabecumbe}
\begin{itemize}
\item {Grp. gram.:m.}
\end{itemize}
Árvore de Cabinda, própria para construcções e mobília.
\section{Cabedaes}
\begin{itemize}
\item {Grp. gram.:m. pl.}
\end{itemize}
Designação que, em Lisbôa, se dá aos desempenos de marceneiro.
(Cp. \textunderscore cabedal\textunderscore )
\section{Cabedais}
\begin{itemize}
\item {Grp. gram.:m. pl.}
\end{itemize}
Designação que, em Lisbôa, se dá aos desempenos de marceneiro.
(Cp. \textunderscore cabedal\textunderscore )
\section{Cabedal}
\begin{itemize}
\item {Grp. gram.:m.}
\end{itemize}
\begin{itemize}
\item {Grp. gram.:Adj.}
\end{itemize}
\begin{itemize}
\item {Utilização:Ant.}
\end{itemize}
\begin{itemize}
\item {Proveniência:(Do lat. \textunderscore capitalis\textunderscore )}
\end{itemize}
Capital; conjunto de coisas de valor.
Riqueza; haveres.
Cópia de conhecimentos.
Estimação.
Coiro, próprio para se manufacturar calçado; sola.
Aquillo que é objecto do commércio.
Recurso, poder.
O mesmo que \textunderscore caudal\textunderscore .
\section{Cabedaleiro}
\begin{itemize}
\item {Grp. gram.:m.}
\end{itemize}
\begin{itemize}
\item {Utilização:P. us.}
\end{itemize}
\begin{itemize}
\item {Proveniência:(De \textunderscore cabedal\textunderscore )}
\end{itemize}
Aquelle que negocía com dinheiro alheio.
\section{Cabedar}
\begin{itemize}
\item {Grp. gram.:v. i.}
\end{itemize}
\begin{itemize}
\item {Utilização:Prov.}
\end{itemize}
\begin{itemize}
\item {Utilização:alent.}
\end{itemize}
Tocar por sorte, caber.
(Por \textunderscore cabidar\textunderscore , de \textunderscore cabido\textunderscore , \textunderscore part.\textunderscore  de \textunderscore caber\textunderscore )
\section{Cabedela}
\begin{itemize}
\item {Grp. gram.:f.}
\end{itemize}
O mesmo que \textunderscore cabidela\textunderscore . Cf. \textunderscore Filodemo\textunderscore , II, 7.
\section{Cabedelo}
\begin{itemize}
\item {fónica:dê}
\end{itemize}
\begin{itemize}
\item {Grp. gram.:m.}
\end{itemize}
\begin{itemize}
\item {Proveniência:(Do lat. \textunderscore capitellum\textunderscore )}
\end{itemize}
Pequeno cabo, cabeço de areia, junto á foz dos rios.
\section{Cabedulho}
\begin{itemize}
\item {Grp. gram.:m.}
\end{itemize}
O mesmo ou melhor que \textunderscore cadabulho\textunderscore .
\section{Cabeia}
\begin{itemize}
\item {Grp. gram.:f.}
\end{itemize}
Árvore euphorbiácea da África occidental.
\section{Cabeira}
\begin{itemize}
\item {Grp. gram.:f.}
\end{itemize}
\begin{itemize}
\item {Utilização:Carp.}
\end{itemize}
Trabalho de moldura em tectos e soalhos.
\section{Cabeiro}
\begin{itemize}
\item {Grp. gram.:adj.}
\end{itemize}
Que está no cabo, que é o último.
\section{Cabela}
\begin{itemize}
\item {Grp. gram.:f.}
\end{itemize}
Árvore anonácea da África occidental, (\textunderscore xilopia aethiopica\textunderscore , Rich.).
\section{Cabeladura}
\begin{itemize}
\item {Grp. gram.:f.}
\end{itemize}
\begin{itemize}
\item {Proveniência:(De \textunderscore cabello\textunderscore )}
\end{itemize}
Cabeleira; encabeladura.
\section{Cabelame}
\begin{itemize}
\item {Grp. gram.:m.}
\end{itemize}
\begin{itemize}
\item {Utilização:Bot.}
\end{itemize}
\begin{itemize}
\item {Proveniência:(De \textunderscore cabello\textunderscore )}
\end{itemize}
Conjunto das radículas de uma planta.
\section{Cabeleira}
\begin{itemize}
\item {Grp. gram.:f.}
\end{itemize}
\begin{itemize}
\item {Utilização:Agr.}
\end{itemize}
\begin{itemize}
\item {Grp. gram.:M.}
\end{itemize}
\begin{itemize}
\item {Proveniência:(De \textunderscore cabello\textunderscore )}
\end{itemize}
Conjunto dos cabelos compridos que crescem na cabeça.
Cabelos postiços; chinó.
Nebulosidade, mais ou menos luminosa, que circunda o núcleo dos cometas.
Crina.
Radículas, á flôr da terra.
Homem apaixonado por ideias antigas.
\section{Cabeleireira}
\begin{itemize}
\item {Grp. gram.:f.}
\end{itemize}
Mulher, que trabalha em cabelo, fazendo cabeleiras ou penteando mulheres.
(Fem. de \textunderscore cabelleireiro\textunderscore )
\section{Cabeleireiro}
\begin{itemize}
\item {Grp. gram.:m.}
\end{itemize}
\begin{itemize}
\item {Proveniência:(De \textunderscore cabelleira\textunderscore )}
\end{itemize}
Aquele que trabalha em cabeleiras.
Aquele que por ofício corta ou penteia o cabelo dos outros.
\section{Cabeleiro}
\begin{itemize}
\item {Grp. gram.:m.}
\end{itemize}
\begin{itemize}
\item {Utilização:Prov.}
\end{itemize}
Um cabelo.
\section{Cabelinho}
\begin{itemize}
\item {Grp. gram.:m.}
\end{itemize}
\begin{itemize}
\item {Grp. gram.:Loc.}
\end{itemize}
\begin{itemize}
\item {Utilização:fam.}
\end{itemize}
Cabelo pequeno.
\textunderscore Têr cabelinho na venta\textunderscore , têr mau gênio, sêr irritável.
\section{Cabelladura}
\begin{itemize}
\item {Grp. gram.:f.}
\end{itemize}
\begin{itemize}
\item {Proveniência:(De \textunderscore cabello\textunderscore )}
\end{itemize}
Cabelleira; encabelladura.
\section{Cabellame}
\begin{itemize}
\item {Grp. gram.:m.}
\end{itemize}
\begin{itemize}
\item {Utilização:Bot.}
\end{itemize}
\begin{itemize}
\item {Proveniência:(De \textunderscore cabello\textunderscore )}
\end{itemize}
Conjunto das radículas de uma planta.
\section{Cabelleira}
\begin{itemize}
\item {Grp. gram.:f.}
\end{itemize}
\begin{itemize}
\item {Utilização:Agr.}
\end{itemize}
\begin{itemize}
\item {Grp. gram.:M.}
\end{itemize}
\begin{itemize}
\item {Proveniência:(De \textunderscore cabello\textunderscore )}
\end{itemize}
Conjunto dos cabellos compridos que crescem na cabeça.
Cabellos postiços; chinó.
Nebulosidade, mais ou menos luminosa, que circunda o núcleo dos cometas.
Crina.
Radículas, á flôr da terra.
Homem apaixonado por ideias antigas.
\section{Cabelleireira}
\begin{itemize}
\item {Grp. gram.:f.}
\end{itemize}
Mulher, que trabalha em cabello, fazendo cabelleiras ou penteando mulheres.
(Fem. de \textunderscore cabelleireiro\textunderscore )
\section{Cabelleireiro}
\begin{itemize}
\item {Grp. gram.:m.}
\end{itemize}
\begin{itemize}
\item {Proveniência:(De \textunderscore cabelleira\textunderscore )}
\end{itemize}
Aquelle que trabalha em cabelleiras.
Aquelle que por offício corta ou penteia o cabello dos outros.
\section{Cabelleiro}
\begin{itemize}
\item {Grp. gram.:m.}
\end{itemize}
\begin{itemize}
\item {Utilização:Prov.}
\end{itemize}
Um cabello.
\section{Cabellinho}
\begin{itemize}
\item {Grp. gram.:m.}
\end{itemize}
\begin{itemize}
\item {Grp. gram.:Loc.}
\end{itemize}
\begin{itemize}
\item {Utilização:fam.}
\end{itemize}
Cabello pequeno.
\textunderscore Têr cabellinho na venta\textunderscore , têr mau gênio, sêr irritável.
\section{Cabello}
\begin{itemize}
\item {fónica:bê}
\end{itemize}
\begin{itemize}
\item {Grp. gram.:m.}
\end{itemize}
\begin{itemize}
\item {Utilização:Ext.}
\end{itemize}
\begin{itemize}
\item {Grp. gram.:Loc. adv.}
\end{itemize}
\begin{itemize}
\item {Proveniência:(Do lat. \textunderscore capillus\textunderscore )}
\end{itemize}
Pêlos, que crescem sôbre a cabeça humana.
Pêlos, que nascem em qualquer parte do corpo humano.
Pêlos compridos de alguns animaes.
Cada um dos pêlos que nascem na cabeça.
Delgada mola de aço, que regula o movimento dos relógios de algibeira.
\textunderscore Pelos cabellos\textunderscore , de má vontade, com sacrifício.
\textunderscore Têr cabellos no coração\textunderscore , sêr insensível, cruel.
\section{Cabello-de-negro}
\begin{itemize}
\item {Grp. gram.:m.}
\end{itemize}
\begin{itemize}
\item {Utilização:Bras}
\end{itemize}
Planta medicinal.
\section{Cabellos-de-Vênus}
\begin{itemize}
\item {Grp. gram.:m. pl.}
\end{itemize}
\begin{itemize}
\item {Utilização:Bras}
\end{itemize}
O mesmo que \textunderscore nigella\textunderscore .
\section{Cabello-vermelho}
\begin{itemize}
\item {Grp. gram.:m.}
\end{itemize}
Espécie de algá da ria de Aveiro, (\textunderscore geranium rubrum\textunderscore , Ag.).
\section{Cabelluda}
\begin{itemize}
\item {Grp. gram.:f.}
\end{itemize}
\begin{itemize}
\item {Proveniência:(De \textunderscore cabelludo\textunderscore )}
\end{itemize}
Árvore myrtácea do Brasil.
\section{Cabelludo}
\begin{itemize}
\item {Grp. gram.:adj.}
\end{itemize}
\begin{itemize}
\item {Grp. gram.:M.}
\end{itemize}
\begin{itemize}
\item {Utilização:Bras}
\end{itemize}
Que tem muito cabello ou cabello comprido.
\textunderscore Coiro cabelludo\textunderscore , pelle que cobre o crânio, e em que nasce o cabello.
Espécie de lagarta de pêlos compridos.
\section{Cabellugem}
\begin{itemize}
\item {Grp. gram.:f.}
\end{itemize}
\begin{itemize}
\item {Utilização:Des.}
\end{itemize}
O mesmo que \textunderscore cabelladura\textunderscore .
\section{Cabelo}
\begin{itemize}
\item {fónica:bê}
\end{itemize}
\begin{itemize}
\item {Grp. gram.:m.}
\end{itemize}
\begin{itemize}
\item {Utilização:Ext.}
\end{itemize}
\begin{itemize}
\item {Grp. gram.:Loc. adv.}
\end{itemize}
\begin{itemize}
\item {Proveniência:(Do lat. \textunderscore capillus\textunderscore )}
\end{itemize}
Pêlos, que crescem sôbre a cabeça humana.
Pêlos, que nascem em qualquer parte do corpo humano.
Pêlos compridos de alguns animaes.
Cada um dos pêlos que nascem na cabeça.
Delgada mola de aço, que regula o movimento dos relógios de algibeira.
\textunderscore Pelos cabelos\textunderscore , de má vontade, com sacrifício.
\textunderscore Têr cabelos no coração\textunderscore , sêr insensível, cruel.
\section{Cabeluda}
\begin{itemize}
\item {Grp. gram.:f.}
\end{itemize}
\begin{itemize}
\item {Proveniência:(De \textunderscore cabelludo\textunderscore )}
\end{itemize}
Árvore myrtácea do Brasil.
\section{Cabeludo}
\begin{itemize}
\item {Grp. gram.:adj.}
\end{itemize}
\begin{itemize}
\item {Grp. gram.:M.}
\end{itemize}
\begin{itemize}
\item {Utilização:Bras}
\end{itemize}
Que tem muito cabelo ou cabelo comprido.
\textunderscore Coiro cabeludo\textunderscore , pelle que cobre o crânio, e em que nasce o cabelo.
Espécie de lagarta de pêlos compridos.
\section{Cabelugem}
\begin{itemize}
\item {Grp. gram.:f.}
\end{itemize}
\begin{itemize}
\item {Utilização:Des.}
\end{itemize}
O mesmo que \textunderscore cabeladura\textunderscore .
\section{Cabenda}
\begin{itemize}
\item {Grp. gram.:f.}
\end{itemize}
Gênero de árvores angolenses de Cazengo.
\section{Caber}
\begin{itemize}
\item {Grp. gram.:v. i.}
\end{itemize}
\begin{itemize}
\item {Grp. gram.:M.}
\end{itemize}
\begin{itemize}
\item {Utilização:Ant.}
\end{itemize}
\begin{itemize}
\item {Utilização:Ant.}
\end{itemize}
\begin{itemize}
\item {Proveniência:(Do lat. \textunderscore capere\textunderscore )}
\end{itemize}
Poder sêr comprehendido, poder estar dentro.
Poder exprimir-se ou realizar-se: \textunderscore caber no possível\textunderscore .
Sêr compatível.
Competir; pertencer em partilha; vir por sorte ou turno: \textunderscore couberam-me três belgas por herança\textunderscore .
Sêr admissível; sêr opportuno.
Quinhão, sorte, legítima.
Capital, que se emprega em bem-feitorias de um prédio.
\section{Cabera}
\begin{itemize}
\item {Grp. gram.:f.}
\end{itemize}
\begin{itemize}
\item {Proveniência:(De \textunderscore Cabera\textunderscore , n. p.)}
\end{itemize}
Insecto lepidóptero nocturno.
\section{Cabesnalha}
\begin{itemize}
\item {Grp. gram.:f.}
\end{itemize}
\begin{itemize}
\item {Utilização:Prov.}
\end{itemize}
\begin{itemize}
\item {Utilização:trasm.}
\end{itemize}
O mesmo que \textunderscore cabeçalha\textunderscore  (do carro).
\section{Cabi}
\begin{itemize}
\item {Grp. gram.:m.}
\end{itemize}
Árvore de Cabinda, própria para construcções.
\section{Çabi}
\begin{itemize}
\item {Grp. gram.:m.}
\end{itemize}
O mesmo que \textunderscore çabri\textunderscore .
\section{Cabicanca}
\begin{itemize}
\item {Grp. gram.:f.}
\end{itemize}
\begin{itemize}
\item {Utilização:Prov.}
\end{itemize}
\begin{itemize}
\item {Utilização:beir.}
\end{itemize}
Pássaro fantástico, nas superstições populares. (Colhido em Celorico da Beira)
\section{Cabida}
\begin{itemize}
\item {Grp. gram.:f.}
\end{itemize}
Acto ou effeito de \textunderscore caber\textunderscore .
Cabimento.
Acceitação, valimento, intimidade de relações.
\section{Cabide}
\begin{itemize}
\item {Grp. gram.:m.}
\end{itemize}
Móvel, em que se pendura fato, chapéus, etc.
(Do ár.? Cp. G. Viana, \textunderscore Apostilas\textunderscore )
\section{Cabidela}
\begin{itemize}
\item {Grp. gram.:f.}
\end{itemize}
\begin{itemize}
\item {Proveniência:(Do ár. \textunderscore cabol\textunderscore , fígado)}
\end{itemize}
Reunião de fígado, pescoço, pernas e outras miudezas de aves.
Guisado, feito com essas miudezas e sangue das mesmas aves.
\section{Cabido}
\begin{itemize}
\item {Grp. gram.:m.}
\end{itemize}
\begin{itemize}
\item {Utilização:Ant.}
\end{itemize}
\begin{itemize}
\item {Utilização:Ant.}
\end{itemize}
\begin{itemize}
\item {Proveniência:(Do lat. \textunderscore capitulum\textunderscore )}
\end{itemize}
Conjunto, corporação, dos cónegos de uma cathedral.
Assembleia, celebrada por uma Ordem religiosa.
Alpendre, annexo a uma igreja; galilé.
\section{Cabido}
\begin{itemize}
\item {Grp. gram.:adj.}
\end{itemize}
\begin{itemize}
\item {Proveniência:(De \textunderscore caber\textunderscore )}
\end{itemize}
Que tem cabimento: \textunderscore foi bem cabida a observação\textunderscore .
\section{Cabido}
\begin{itemize}
\item {Grp. gram.:m.}
\end{itemize}
\begin{itemize}
\item {Utilização:Prov.}
\end{itemize}
O mesmo que \textunderscore cabide\textunderscore .
\section{Cabidoal}
\begin{itemize}
\item {Grp. gram.:adj.}
\end{itemize}
\begin{itemize}
\item {Utilização:Ant.}
\end{itemize}
\begin{itemize}
\item {Grp. gram.:M.}
\end{itemize}
\begin{itemize}
\item {Utilização:Ant.}
\end{itemize}
Relativo a cabido.
Relativo a assembleia dos parochianos.
Procurador das assembleias parochiaes.
\section{Cabídola}
\begin{itemize}
\item {Grp. gram.:f.}
\end{itemize}
\begin{itemize}
\item {Utilização:Ant.}
\end{itemize}
\begin{itemize}
\item {Proveniência:(Lat. \textunderscore capitula\textunderscore , pl. de \textunderscore capitulum\textunderscore )}
\end{itemize}
Letra maiúscula.
\section{Cabidual}
\begin{itemize}
\item {Grp. gram.:adj.}
\end{itemize}
\begin{itemize}
\item {Utilização:Ant.}
\end{itemize}
\begin{itemize}
\item {Grp. gram.:M.}
\end{itemize}
\begin{itemize}
\item {Utilização:Ant.}
\end{itemize}
Relativo a cabido.
Relativo a assembleia dos parochianos.
Procurador das assembleias parochiaes.
\section{Cabila}
\begin{itemize}
\item {Grp. gram.:f.}
\end{itemize}
Designação genérica de várias tríbos da África setentrional.
(Ár. \textunderscore cabila\textunderscore )
\section{Cabilangau}
\begin{itemize}
\item {Grp. gram.:m.}
\end{itemize}
Arvoreta angolense de Pungo-Andongo.
\section{Cabilda}
\begin{itemize}
\item {Grp. gram.:f.}
\end{itemize}
O mesmo que \textunderscore cabila\textunderscore .
\section{Cabimento}
\begin{itemize}
\item {Grp. gram.:m.}
\end{itemize}
\begin{itemize}
\item {Proveniência:(De \textunderscore caber\textunderscore )}
\end{itemize}
O mesmo que \textunderscore cabida\textunderscore .
Lugar; acceitação, recebimento.
Conveniência; opportunidade.
\section{Cabina}
\begin{itemize}
\item {Grp. gram.:f.}
\end{itemize}
\begin{itemize}
\item {Utilização:Gal}
\end{itemize}
\begin{itemize}
\item {Proveniência:(Fr. \textunderscore cabine\textunderscore )}
\end{itemize}
Pequeno compartimento nos navios mercantes.
\section{Cabinda}
\begin{itemize}
\item {Grp. gram.:m.}
\end{itemize}
Língua angolense do território de Cabinda.
\section{Cabindas}
\begin{itemize}
\item {Grp. gram.:m. pl.}
\end{itemize}
Fiotes, que habitam sôbre o Zaire.
\section{Cabineiro}
\begin{itemize}
\item {Grp. gram.:m.}
\end{itemize}
\begin{itemize}
\item {Utilização:Neol.}
\end{itemize}
Aquelle que trata de cabinas.
\section{Cabirto}
\begin{itemize}
\item {Grp. gram.:m.}
\end{itemize}
(Metáth. de \textunderscore cabrito\textunderscore , us. em Melgaço)
\section{Cabis}
\begin{itemize}
\item {Grp. gram.:m.}
\end{itemize}
\begin{itemize}
\item {Utilização:Bras. de Minas}
\end{itemize}
Casaco de homem.
\section{Cabisalva}
\begin{itemize}
\item {Grp. gram.:f.}
\end{itemize}
\begin{itemize}
\item {Proveniência:(De \textunderscore cabeça\textunderscore  + \textunderscore alvo\textunderscore . Cp. \textunderscore cabisbaixo\textunderscore )}
\end{itemize}
Ave de rapina.
\section{Cabisbaixo}
\begin{itemize}
\item {Grp. gram.:adj.}
\end{itemize}
\begin{itemize}
\item {Proveniência:(De \textunderscore cabeça\textunderscore  + \textunderscore baixo\textunderscore )}
\end{itemize}
Que traz a cabeça baixa, inclinada.
Abatido.
Vexado.
\section{Cabiscaído}
\begin{itemize}
\item {Grp. gram.:adj.}
\end{itemize}
O mesmo que \textunderscore cabisbaixo\textunderscore . Cf. Vieira, \textunderscore Cartas\textunderscore , I, 429.
\section{Cabíscol}
\begin{itemize}
\item {Grp. gram.:m.}
\end{itemize}
\begin{itemize}
\item {Utilização:Ant.}
\end{itemize}
Chantre.
(Cast. \textunderscore capiscol\textunderscore )
\section{Cabiúna}
\begin{itemize}
\item {Grp. gram.:f.}
\end{itemize}
Espécie de jacarandá.
\section{Cabixis}
\begin{itemize}
\item {Grp. gram.:m. pl.}
\end{itemize}
Aborígenes de Mato-Grosso.
\section{Cabo}
\begin{itemize}
\item {Grp. gram.:m.}
\end{itemize}
\begin{itemize}
\item {Proveniência:(Do lat. \textunderscore caput\textunderscore )}
\end{itemize}
Chefe, caudilho, cabeça.
Elevação de terra que, em fórma de ponta, entra pelo mar.
Fim, lugar extremo.
\textunderscore Cabo de guerra\textunderscore , antigo official superior do exército.
\textunderscore Cabo de esquadra\textunderscore , extinta graduação inferior, na milícia.
\textunderscore Razões de cabo de esquadra\textunderscore , razões disparatadas.
\section{Cabo}
\begin{itemize}
\item {Grp. gram.:m.}
\end{itemize}
\begin{itemize}
\item {Proveniência:(Do lat. \textunderscore capulus\textunderscore )}
\end{itemize}
Tudo que prende qualquer coisa ou tudo aquillo por onde qualquer coisa se segura: \textunderscore o cabo da vassoira\textunderscore .
Cauda.
Cada uma das cordas mais ou menos grossas, que se empregam nos navios.
Feixe de fios metállicos, para transmissão de telegrammas.
\section{Cabo}
\begin{itemize}
\item {Grp. gram.:m.}
\end{itemize}
\begin{itemize}
\item {Utilização:Ant.}
\end{itemize}
\begin{itemize}
\item {Proveniência:(De \textunderscore caber\textunderscore )}
\end{itemize}
Lugar em que uma pessóa ou coisa cabe ou está: \textunderscore venha aqui para o meu cabo\textunderscore .
\textunderscore Por seu cabo\textunderscore , por sua vez, por seu turno.
\section{Caboboata}
\begin{itemize}
\item {Grp. gram.:f.}
\end{itemize}
Pequena planta africana, da fam. das labiadas, herbácea, annual, de fôlhas ovaes-oblongas, com pêlos hirsutos e flôres em espigas na extremidade do caule.
\section{Cabocir}
\begin{itemize}
\item {Grp. gram.:m.}
\end{itemize}
\begin{itemize}
\item {Proveniência:(Do port. \textunderscore cabeceira\textunderscore ?)}
\end{itemize}
Chefe indígena, na Guiné setentrional.
\section{Cabocla}
\begin{itemize}
\item {fónica:bô}
\end{itemize}
\begin{itemize}
\item {Grp. gram.:f.}
\end{itemize}
\begin{itemize}
\item {Proveniência:(De \textunderscore caboclo\textunderscore )}
\end{itemize}
Espécie de rôla do Brasil.
Mulher, da casta dos caboclos.
\section{Caboclada}
\begin{itemize}
\item {Grp. gram.:f.}
\end{itemize}
\begin{itemize}
\item {Utilização:Bras}
\end{itemize}
Classe de caboclos.
Agrupamento de caboclos.
\section{Caboclinho}
\begin{itemize}
\item {Grp. gram.:m.}
\end{itemize}
\begin{itemize}
\item {Utilização:Bras}
\end{itemize}
Pássaro do Brasil, notável pelo seu canto.
\section{Caboclismo}
\begin{itemize}
\item {Grp. gram.:m.}
\end{itemize}
\begin{itemize}
\item {Utilização:Bras}
\end{itemize}
Acto ou sentimento próprio de caboclo.
\section{Caboclo}
\begin{itemize}
\item {fónica:bô}
\end{itemize}
\begin{itemize}
\item {Grp. gram.:adj.}
\end{itemize}
\begin{itemize}
\item {Utilização:Bras}
\end{itemize}
\begin{itemize}
\item {Grp. gram.:M.}
\end{itemize}
\begin{itemize}
\item {Utilização:Bras}
\end{itemize}
Que tem côr acobreada.
Mestiço.
O mesmo que \textunderscore carijó\textunderscore .
(Do tupi \textunderscore caá-boc\textunderscore )
\section{Cabo-de-sovela}
\begin{itemize}
\item {Grp. gram.:m.}
\end{itemize}
Variedade de pêra ordinária.
\section{Cabódi}
\begin{itemize}
\item {Grp. gram.:m.}
\end{itemize}
Arbusto africano, da fam. das malváceas.
\section{Cabograma}
\begin{itemize}
\item {Grp. gram.:m.}
\end{itemize}
\begin{itemize}
\item {Utilização:Neol.}
\end{itemize}
Telegramma submarino.
\section{Cabogramma}
\begin{itemize}
\item {Grp. gram.:m.}
\end{itemize}
\begin{itemize}
\item {Utilização:Neol.}
\end{itemize}
Telegramma submarino.
\section{Cabolama-tende}
\begin{itemize}
\item {Grp. gram.:m.}
\end{itemize}
Arbusto africano, que fórma moitas, com flôres cordiformes, de corolla branca.
\section{Cabole}
\begin{itemize}
\item {Grp. gram.:m.}
\end{itemize}
Árvore africana, da fam. das loganiáceas e muito semelhante ao mabole.
\section{Cabolebole}
\begin{itemize}
\item {Grp. gram.:m.}
\end{itemize}
Arbusto africano, annual.
\section{Cabomba}
\begin{itemize}
\item {Grp. gram.:f.}
\end{itemize}
O mesmo que \textunderscore cabombo\textunderscore .
\section{Cabombo}
\begin{itemize}
\item {Grp. gram.:m.}
\end{itemize}
Arbusto africano, de fôlhas inteiras e frutos semelhantes a laranjas.
\section{Càbonegro}
\begin{itemize}
\item {fónica:nê}
\end{itemize}
\begin{itemize}
\item {Grp. gram.:m.}
\end{itemize}
Espécie de palmeira americana.
\section{Caboquenas}
\begin{itemize}
\item {Grp. gram.:m. pl.}
\end{itemize}
Selvagens, que habitam no Pará.
\section{Caboraíba}
\begin{itemize}
\item {Grp. gram.:f.}
\end{itemize}
\begin{itemize}
\item {Utilização:Bras}
\end{itemize}
Espécie de óleo.
\section{Caboré}
\begin{itemize}
\item {Grp. gram.:m.}
\end{itemize}
\begin{itemize}
\item {Utilização:Bras}
\end{itemize}
\begin{itemize}
\item {Utilização:Fig.}
\end{itemize}
\begin{itemize}
\item {Grp. gram.:M.  e  f.}
\end{itemize}
Espécie de mocho.
Pequena panela de barro, para serviço de cozinha.
Caboclo de pouca idade.
Homem gordo e de baixa estatura.
Mestiço de negro e índio.
Pessôa trigueira, tirante a caboclo.
(Do tupi)
\section{Caborge}
\begin{itemize}
\item {Grp. gram.:m.}
\end{itemize}
\begin{itemize}
\item {Utilização:Bras. do N}
\end{itemize}
O mesmo que \textunderscore feitiçaria\textunderscore .
\section{Cabortar}
\begin{itemize}
\item {Grp. gram.:v. i.}
\end{itemize}
\begin{itemize}
\item {Utilização:Bras}
\end{itemize}
Mentir.
O mesmo que \textunderscore cabortear\textunderscore .
\section{Cabortear}
\begin{itemize}
\item {Grp. gram.:v. i.}
\end{itemize}
\begin{itemize}
\item {Utilização:Bras. do S}
\end{itemize}
\begin{itemize}
\item {Proveniência:(De \textunderscore caborteiro\textunderscore )}
\end{itemize}
Proceder mal, como um caborteiro.
\section{Caborteiro}
\begin{itemize}
\item {Grp. gram.:m.  e  adj.}
\end{itemize}
\begin{itemize}
\item {Utilização:Bras. do S}
\end{itemize}
Indivíduo velhaco, manhoso.
\section{Cabós}
\begin{itemize}
\item {Grp. gram.:m.}
\end{itemize}
Nome de várias espécies de peixes marítimos.
(B. lat. \textunderscore cabos\textunderscore )
\section{Cabos-brancos}
\begin{itemize}
\item {Grp. gram.:adj.}
\end{itemize}
\begin{itemize}
\item {Utilização:Bras}
\end{itemize}
Diz-se do cavallo que tem brancos os quatro pés.
\section{Caboseira}
\begin{itemize}
\item {Grp. gram.:f.}
\end{itemize}
\begin{itemize}
\item {Proveniência:(De \textunderscore cabós\textunderscore )}
\end{itemize}
Nome de um peixe dos Açores.
\section{Caboseiro}
\begin{itemize}
\item {Grp. gram.:m.}
\end{itemize}
Cana, apparelhada para a pesca das caboseiras.
\section{Cabos-negros}
\begin{itemize}
\item {Grp. gram.:adj.}
\end{itemize}
\begin{itemize}
\item {Utilização:Bras}
\end{itemize}
Diz-se do cavallo que tem negros os quatro pés.
\section{Cabotagem}
\begin{itemize}
\item {Grp. gram.:f.}
\end{itemize}
\begin{itemize}
\item {Proveniência:(De \textunderscore cabo\textunderscore )}
\end{itemize}
Navegação costeira, (entre cabos ou entre portos da mesma região).
\section{Cabotar}
\begin{itemize}
\item {Grp. gram.:v. i.}
\end{itemize}
Fazer cabotagem.
\section{Cabotinagem}
\begin{itemize}
\item {Grp. gram.:f.}
\end{itemize}
\begin{itemize}
\item {Utilização:Neol.}
\end{itemize}
Vida ou costumes de cabotino.
\section{Cabotinismo}
\begin{itemize}
\item {Grp. gram.:m.}
\end{itemize}
\begin{itemize}
\item {Utilização:Neol.}
\end{itemize}
O mesmo que \textunderscore cabotinagem\textunderscore .
\section{Cabotino}
\begin{itemize}
\item {Grp. gram.:m.}
\end{itemize}
\begin{itemize}
\item {Utilização:Neol.}
\end{itemize}
\begin{itemize}
\item {Proveniência:(Fr. \textunderscore cabotin\textunderscore )}
\end{itemize}
Comediante ambulante.
Mau comediante.
\section{Caboucador}
\begin{itemize}
\item {Grp. gram.:m.}
\end{itemize}
Aquelle que cabouca.
\section{Caboucar}
\begin{itemize}
\item {Grp. gram.:v. t.}
\end{itemize}
\begin{itemize}
\item {Utilização:Fig.}
\end{itemize}
\begin{itemize}
\item {Grp. gram.:V. i.}
\end{itemize}
Abrir caboucos em.
Assentar nos caboucos.
Assentar.
Iniciar.
Abrir caboucos. Cf. Camillo, \textunderscore Perfil\textunderscore , 110.
\section{Cabouco}
\begin{itemize}
\item {Grp. gram.:m.}
\end{itemize}
\begin{itemize}
\item {Utilização:Prov.}
\end{itemize}
Fôsso; cova comprida, em que se assentam alicerces.
O espaço em que gira o rodízio da azenha.
Estribo de pau.
(Por \textunderscore cavouco\textunderscore , de \textunderscore cavo\textunderscore  + ?)
\section{Cabouja-anganga}
\begin{itemize}
\item {Grp. gram.:f.}
\end{itemize}
Ave pernalta do Brasil.
\section{Caboupa}
\begin{itemize}
\item {Grp. gram.:f.}
\end{itemize}
Árvore da Guiné, cuja casca é purgativa.
\section{Cabouqueiro}
\begin{itemize}
\item {Grp. gram.:m.}
\end{itemize}
Aquelle que faz caboucos.
Aquelle que escava; cavador.
Aquelle que trabalha em minas.
\section{Cabo-verde}
\begin{itemize}
\item {Grp. gram.:m.  e  f.}
\end{itemize}
\begin{itemize}
\item {Utilização:Bras}
\end{itemize}
Mestiço de negro e índio; caboré.
\section{Caboverdeano}
\begin{itemize}
\item {Grp. gram.:adj.}
\end{itemize}
\begin{itemize}
\item {Grp. gram.:M.}
\end{itemize}
Relativo ao archipélago de Cabo-Verde.
Habitante de Cabo-Verde.
\section{Cabra}
\begin{itemize}
\item {Grp. gram.:f.}
\end{itemize}
\begin{itemize}
\item {Utilização:T. de Coimbra}
\end{itemize}
\begin{itemize}
\item {Utilização:T. da Bairrada}
\end{itemize}
\begin{itemize}
\item {Utilização:Pop.}
\end{itemize}
\begin{itemize}
\item {Utilização:Fig.}
\end{itemize}
\begin{itemize}
\item {Grp. gram.:M.  e  f.}
\end{itemize}
\begin{itemize}
\item {Utilização:Gír.}
\end{itemize}
\begin{itemize}
\item {Proveniência:(Lat. \textunderscore capra\textunderscore )}
\end{itemize}
Animal mammífero, da ordem dos ruminantes, fêmea de bode.
Guindaste.
Espécie de pequeno peixe avermelhado, também conhecido por \textunderscore cabrita\textunderscore  ou \textunderscore cabritinha\textunderscore .
Sineta universitária, que annuncía o comêço e o fim do serviço escolar diário.
Variedade de insecto, também conhecido por \textunderscore alfaiate\textunderscore .
Mulher dissoluta.
Mulher de mau gênio ou que berra muito.
Denunciante; polícia.
\section{Cabra}
\begin{itemize}
\item {Grp. gram.:m.  e  f.}
\end{itemize}
\begin{itemize}
\item {Utilização:Bras}
\end{itemize}
\begin{itemize}
\item {Utilização:Bras. riograndense}
\end{itemize}
Mestiço, filho de mulato e negra ou viceversa.
Indivíduo; qualquer sujeito.
\section{Cabra}
\begin{itemize}
\item {Grp. gram.:f.}
\end{itemize}
Árvore da ilha de San-Thomé, (\textunderscore trema guineensis\textunderscore ).
\section{Cabra-cega}
\begin{itemize}
\item {Grp. gram.:f.}
\end{itemize}
Jôgo ou folguedo de sala, em que uma pessôa, de olhos vendados, se esforça por apanhar outra, para sêr por esta substituída.
\section{Cabrada}
\begin{itemize}
\item {Grp. gram.:f.}
\end{itemize}
Rebanho de cabras.
\section{Cabralhada}
\begin{itemize}
\item {Grp. gram.:f.}
\end{itemize}
\begin{itemize}
\item {Utilização:Bras. do N}
\end{itemize}
O mesmo que \textunderscore cabroeira\textunderscore .
\section{Cabralino}
\begin{itemize}
\item {Grp. gram.:adj.}
\end{itemize}
\begin{itemize}
\item {Proveniência:(De \textunderscore Cabral\textunderscore , n. p.)}
\end{itemize}
Relativo ao Govêrno cabralista.
\textunderscore Á cabralina\textunderscore , á fôrça; semelhantemente ao Govêrno de Costa-Cabral.
\section{Cabralismo}
\begin{itemize}
\item {Grp. gram.:m.}
\end{itemize}
Partido político, que preponderou em Portugal, durante o ministério de \textunderscore Cabral\textunderscore , (marquês de Thomar).
\section{Cabralista}
\begin{itemize}
\item {Grp. gram.:m.}
\end{itemize}
Sectário do cabralismo.
\section{Cabramo}
\begin{itemize}
\item {Grp. gram.:m.}
\end{itemize}
\begin{itemize}
\item {Proveniência:(Do lat. \textunderscore caput\textunderscore  + \textunderscore premere\textunderscore )}
\end{itemize}
Corda, que se prende a uma das pontas e ao pé ou á mão do boi, para que não fuja.
\section{Cabranaz}
\begin{itemize}
\item {Grp. gram.:m.}
\end{itemize}
\begin{itemize}
\item {Utilização:Des.}
\end{itemize}
Bode grande.
Mulato.
\section{Cabrão}
\begin{itemize}
\item {Grp. gram.:m.}
\end{itemize}
\begin{itemize}
\item {Utilização:Pop.}
\end{itemize}
\begin{itemize}
\item {Utilização:Pop.}
\end{itemize}
Bode.
Marido, a quem a mulher é infiel.
Criança, que berra muito.
(B. lat. \textunderscore capro\textunderscore )
\section{Cabre}
\begin{itemize}
\item {Grp. gram.:m.}
\end{itemize}
\begin{itemize}
\item {Utilização:Açor}
\end{itemize}
Incidente do jôgo da emboca, em que o jogador bate com a sua bóla na do jogador contrário, para que êste não possa atravessar com a sua o respectivo arco.
\section{Cabre}
\begin{itemize}
\item {Grp. gram.:m.}
\end{itemize}
O mesmo que \textunderscore cábrea\textunderscore .
\section{Cábrea}
\begin{itemize}
\item {Grp. gram.:f.}
\end{itemize}
Corda grossa, que serve de amarreta de navio.
Espécie de guindaste.
(Talvez do lat. \textunderscore caprea\textunderscore )
\section{Cabreado}
\begin{itemize}
\item {Grp. gram.:adj.}
\end{itemize}
\begin{itemize}
\item {Proveniência:(De \textunderscore cabra\textunderscore ^1)}
\end{itemize}
Diz-se do cavallo, que se representa nos brasões levantado sôbre os pés posteriores.
\section{Cabrear}
\begin{itemize}
\item {Grp. gram.:v. i.}
\end{itemize}
Empinar-se (o cavallo), erguer-se nos pés, como faz a cabra. Cf. Garrett, \textunderscore Fábulas\textunderscore , 70.
\section{Cabreira}
\begin{itemize}
\item {Grp. gram.:f.}
\end{itemize}
Mulher, que guarda cabras.
Planta leguminosa, (\textunderscore scorpiurus sulcata\textunderscore , Lin.).
\section{Cabreiro}
\begin{itemize}
\item {Grp. gram.:m.}
\end{itemize}
\begin{itemize}
\item {Grp. gram.:Adj.}
\end{itemize}
Aquelle que guarda cabras, pastor de cabras.
Que guarda cabras:«\textunderscore porque a gente cabreira em tudo quer attentar\textunderscore ». G. Vicente, \textunderscore Clérigo da Beira\textunderscore .
Diz-se de uma variedade de queijo.
\section{Cabrejar}
\begin{itemize}
\item {Grp. gram.:v. i.}
\end{itemize}
\begin{itemize}
\item {Utilização:Prov.}
\end{itemize}
Brincar desenvoltamente. (Colhido em Turquel)
\section{Cabrestante}
\begin{itemize}
\item {Grp. gram.:m.}
\end{itemize}
\begin{itemize}
\item {Proveniência:(De \textunderscore cabra\textunderscore , guindaste?)}
\end{itemize}
Espécie de sarilho, em que se enrolam cabos, para erguer as âncoras e outros pesos.
\section{Cabrestear}
\begin{itemize}
\item {Grp. gram.:v. i.}
\end{itemize}
\begin{itemize}
\item {Utilização:Bras}
\end{itemize}
Deixar-se conduzir pelo cabresto sem difficuldade, (falando-se de cavallos).
\section{Cabresteiro}
\begin{itemize}
\item {Grp. gram.:m.}
\end{itemize}
\begin{itemize}
\item {Grp. gram.:Adj.}
\end{itemize}
Aquelle que faz cabrestos.
Que se deixa levar pelo cabresto.
Submisso, dócil.
\section{Cabrestilho}
\begin{itemize}
\item {Grp. gram.:m.}
\end{itemize}
\begin{itemize}
\item {Proveniência:(De \textunderscore cabresto\textunderscore )}
\end{itemize}
Cabresto pequeno.
\textunderscore Meias de cabrestilho\textunderscore , antigas meias curtas e sem pé, usadas por camponesas.
\section{Cabresto}
\begin{itemize}
\item {fónica:brês}
\end{itemize}
\begin{itemize}
\item {Grp. gram.:m.}
\end{itemize}
\begin{itemize}
\item {Utilização:Náut.}
\end{itemize}
\begin{itemize}
\item {Proveniência:(Do lat. \textunderscore capistrum\textunderscore )}
\end{itemize}
Corda ou correia, com que se prendem e conduzem as cavalgaduras, sem freio.
Boi manso, que serve de guia aos toiros.
Cabo ou corrente, que segura o gurupés ao beque.
Prisão.
Prepúcio.
O mesmo que \textunderscore socairo\textunderscore ^1, correia, corrente ou corda, que prende o cabeçalho á canga.
\section{Çabri}
\begin{itemize}
\item {Grp. gram.:m.}
\end{itemize}
\begin{itemize}
\item {Utilização:Ant.}
\end{itemize}
Espécie de pano indiano.
\section{Cabrião}
\begin{itemize}
\item {Grp. gram.:m.}
\end{itemize}
\begin{itemize}
\item {Proveniência:(De \textunderscore Cabrion\textunderscore , n. p.)}
\end{itemize}
Pessôa, que importuna ou molesta sem cessar.
\section{Cabril}
\begin{itemize}
\item {Grp. gram.:m.}
\end{itemize}
\begin{itemize}
\item {Grp. gram.:Adj.}
\end{itemize}
\begin{itemize}
\item {Proveniência:(Lat. \textunderscore caprile\textunderscore )}
\end{itemize}
Curral do cabras.
Áspero, agreste. Cf. Filinto, XIII, 266.
\section{Cabrilha}
\begin{itemize}
\item {Grp. gram.:f.}
\end{itemize}
\begin{itemize}
\item {Proveniência:(De \textunderscore cabra\textunderscore ^1)}
\end{itemize}
Pequena cábrea.
Pau, com que se move o cabrestante.
Bimbarra.
Apparelho para elevar água.
\section{Cabrim}
\begin{itemize}
\item {Grp. gram.:m.}
\end{itemize}
Pelle curtida de cabra.
\section{Cabrinha}
\begin{itemize}
\item {Grp. gram.:f.}
\end{itemize}
Cabra pequena.
O mesmo que \textunderscore cabra\textunderscore ^1 ou \textunderscore cabrita\textunderscore , peixe.
\section{Cabriola}
\begin{itemize}
\item {Grp. gram.:f.}
\end{itemize}
\begin{itemize}
\item {Utilização:Prov.}
\end{itemize}
\begin{itemize}
\item {Utilização:Fig.}
\end{itemize}
Salto de cabra.
Cambalhota.
Rapariga turbulenta.
Mulher descarada ou dissoluta. (Colhido em Turquel)
Mudança rápida de opinião.
\section{Cabriolar}
\begin{itemize}
\item {Grp. gram.:v. i.}
\end{itemize}
Dar cabriolas.
\section{Cabriolé}
\begin{itemize}
\item {Grp. gram.:m.}
\end{itemize}
\begin{itemize}
\item {Proveniência:(Fr. \textunderscore cabriolet\textunderscore )}
\end{itemize}
Carruagem leve de duas rodas, puxada por um cavallo.
\section{Cabrita}
\begin{itemize}
\item {Grp. gram.:f.}
\end{itemize}
\begin{itemize}
\item {Utilização:T. do Fundão}
\end{itemize}
\begin{itemize}
\item {Utilização:Prov.}
\end{itemize}
\begin{itemize}
\item {Utilização:minh.}
\end{itemize}
\begin{itemize}
\item {Utilização:Prov.}
\end{itemize}
\begin{itemize}
\item {Utilização:dur.}
\end{itemize}
\begin{itemize}
\item {Utilização:Prov.}
\end{itemize}
\begin{itemize}
\item {Utilização:minh.}
\end{itemize}
\begin{itemize}
\item {Utilização:Prov.}
\end{itemize}
\begin{itemize}
\item {Utilização:minh.}
\end{itemize}
\begin{itemize}
\item {Utilização:T. de Barcelos}
\end{itemize}
\begin{itemize}
\item {Grp. gram.:Loc.}
\end{itemize}
\begin{itemize}
\item {Utilização:minh}
\end{itemize}
\begin{itemize}
\item {Proveniência:(De \textunderscore cabrito\textunderscore )}
\end{itemize}
Cabra pequena.
Antiga máquina de guerra, que arremessava pedras.
Pequeno peixe, que também se diz \textunderscore cabra\textunderscore  ou \textunderscore cabrinha\textunderscore .
Inflammação dos olhos.
Pirraça.
Costume de aquelle, que compra uma junta de bois em feira, pagar uma conveniente quantidade de vinho a todos os que entraram na transacção, quer como partes principaes, quer como secundárias.
Chôro ou amuo de crianças.
(Colhido em Barcelos)
Espécie de cunha, que aperta o encedoiro do mangual contra o pírtigo.
Sinal, feito de linhas enroladas, para indicar o comêço de uma meada.
\textunderscore Em cabrita\textunderscore , diz-se da fila de malhadores de espigas, quando vence em destreza e esfôrço a fila contrária.
\textunderscore Ás cabritas\textunderscore , aos ombros, ás cavalleiras.
\section{Cabritalho}
\begin{itemize}
\item {Grp. gram.:m.}
\end{itemize}
Casta ordinária de uva preta.
\section{Cabritão}
\begin{itemize}
\item {Grp. gram.:m.}
\end{itemize}
Planta medicinal de Cabo-Verde (\textunderscore lotus purpureus\textunderscore , Web.).
\section{Cabritar}
\begin{itemize}
\item {Grp. gram.:v. i.}
\end{itemize}
Saltar como os cabritos.
\section{Cabriteiro}
\begin{itemize}
\item {Grp. gram.:m.}
\end{itemize}
\begin{itemize}
\item {Proveniência:(De \textunderscore cabrita\textunderscore )}
\end{itemize}
Aquelle que fazia máquinas de guerra, chamadas cabritas.
\section{Cabritela}
\begin{itemize}
\item {Grp. gram.:f.}
\end{itemize}
Casta de uva tinta de Azeitão, talvez a mesma que \textunderscore tinta-castellan\textunderscore .
\section{Cabritino}
\begin{itemize}
\item {Grp. gram.:adj.}
\end{itemize}
Relativo a cabrito.
\section{Cabrito}
\begin{itemize}
\item {Grp. gram.:m.}
\end{itemize}
\begin{itemize}
\item {Proveniência:(De \textunderscore cabro\textunderscore )}
\end{itemize}
Pequeno bode.
\section{Cabriúva}
\begin{itemize}
\item {Grp. gram.:f.}
\end{itemize}
Árvore leguminosa do Brasil.
\section{Cabro}
\begin{itemize}
\item {Grp. gram.:m.}
\end{itemize}
\begin{itemize}
\item {Utilização:Des.}
\end{itemize}
\begin{itemize}
\item {Proveniência:(Lat. \textunderscore caper\textunderscore )}
\end{itemize}
O mesmo que \textunderscore bode\textunderscore ^1.
\section{Cabroada}
\begin{itemize}
\item {Grp. gram.:f.}
\end{itemize}
Chusma de cabrões. Cf. Filinto, IX, 145.
\section{Cabrocar}
\begin{itemize}
\item {Grp. gram.:v. t.}
\end{itemize}
Roçar ou ceifar (mato)
\section{Cabrocha}
\begin{itemize}
\item {Grp. gram.:m.  e  f.}
\end{itemize}
\begin{itemize}
\item {Utilização:Bras}
\end{itemize}
Indivíduo, ainda novo, da casta dos cabras.
(Cp. \textunderscore cabra\textunderscore ^2)
\section{Cabroeira}
\begin{itemize}
\item {Grp. gram.:f.}
\end{itemize}
\begin{itemize}
\item {Utilização:Bras}
\end{itemize}
\begin{itemize}
\item {Proveniência:(De \textunderscore cabra\textunderscore ^2)}
\end{itemize}
Malta de indivíduos, chamados cabras.
\section{Cabroeiro}
\begin{itemize}
\item {Grp. gram.:m.}
\end{itemize}
\begin{itemize}
\item {Utilização:Bras}
\end{itemize}
O mesmo que \textunderscore cabroeira\textunderscore .
\section{Cabronaz}
\begin{itemize}
\item {Grp. gram.:m.}
\end{itemize}
Grande cabrão. Cf. Cortesão, \textunderscore Subs\textunderscore .
\section{Cabrua}
\begin{itemize}
\item {Grp. gram.:f.}
\end{itemize}
\begin{itemize}
\item {Utilização:Ant.}
\end{itemize}
\begin{itemize}
\item {Proveniência:(De \textunderscore cabrum\textunderscore )}
\end{itemize}
Pellataria de cabra ou de bode.
\section{Cabrué}
\begin{itemize}
\item {Grp. gram.:m.}
\end{itemize}
\begin{itemize}
\item {Utilização:Bras}
\end{itemize}
Árvore silvestre de madeira rija e cinzenta.
\section{Cabrum}
\begin{itemize}
\item {Grp. gram.:adj.}
\end{itemize}
\begin{itemize}
\item {Proveniência:(De \textunderscore cabra\textunderscore ^1)}
\end{itemize}
Relativo a cabras ou bodes.
\section{Cabrunco}
\begin{itemize}
\item {Grp. gram.:m.}
\end{itemize}
(Fórma pop. de \textunderscore carbúnculo\textunderscore )
\section{Cabuchão}
\begin{itemize}
\item {Grp. gram.:m.}
\end{itemize}
\begin{itemize}
\item {Utilização:Bras}
\end{itemize}
Aquillo que tem fórma cónica.
Gênero de molluscos, (\textunderscore pielopsis\textunderscore ).
(Cast. \textunderscore cabujon\textunderscore )
\section{Cabucho}
\begin{itemize}
\item {Grp. gram.:m.}
\end{itemize}
Ponta superior dos pães de açúcar.
(Alter. de \textunderscore capucho\textunderscore )
\section{Cabúia}
\begin{itemize}
\item {Grp. gram.:f.}
\end{itemize}
Planta filamentosa da América do Norte.
\section{Cabul}
\begin{itemize}
\item {Grp. gram.:m.}
\end{itemize}
Arvoreta hypericácea da África portuguesa, (\textunderscore psorospermum febrifugum\textunderscore , Spach.).
\section{Cábula}
\begin{itemize}
\item {Grp. gram.:f.}
\end{itemize}
\begin{itemize}
\item {Grp. gram.:M.}
\end{itemize}
\begin{itemize}
\item {Grp. gram.:Adj.}
\end{itemize}
Defeito do estudante, que não é assíduo nas aulas.
Ardil, com que se falta a uma obrigação.
Aquelle que falta ás aulas, a que tem o dever de assistir.
Ardiloso, manhoso, para se esquivar, ao cumprimento de um dever.
(Alter. de \textunderscore cabala\textunderscore , que melhor se dirá \textunderscore cábala\textunderscore )
\section{Cabúla}
\begin{itemize}
\item {Grp. gram.:f.}
\end{itemize}
\begin{itemize}
\item {Utilização:T. de Alcanena}
\end{itemize}
Montão de feixes de trigo, de vides, etc.
(Cp. \textunderscore cogulo\textunderscore )
\section{Cabular}
\begin{itemize}
\item {Grp. gram.:v. i.}
\end{itemize}
Sêr cábula, mandriar (o estudante).
Trapacear nas aulas.
\section{Cabuleté}
\begin{itemize}
\item {Grp. gram.:m.}
\end{itemize}
\begin{itemize}
\item {Utilização:Bras}
\end{itemize}
Bisbórria, bigorrilhas, homem da ralé.
\section{Cabulice}
\begin{itemize}
\item {Grp. gram.:f.}
\end{itemize}
Acto ou qualidade de cábula.
\section{Cabuloso}
\begin{itemize}
\item {Grp. gram.:adj.}
\end{itemize}
\begin{itemize}
\item {Utilização:Bras}
\end{itemize}
\begin{itemize}
\item {Proveniência:(De \textunderscore cábula\textunderscore )}
\end{itemize}
O mesmo que \textunderscore manhoso\textunderscore .
\section{Cabundá}
\begin{itemize}
\item {Grp. gram.:m.}
\end{itemize}
\begin{itemize}
\item {Utilização:Bras}
\end{itemize}
\begin{itemize}
\item {Proveniência:(T. guar.)}
\end{itemize}
Escravo fugitivo e ladrão.
\section{Cabundo}
\begin{itemize}
\item {Grp. gram.:m.}
\end{itemize}
Árvore angolense.
\section{Cabungo}
\begin{itemize}
\item {Grp. gram.:m.}
\end{itemize}
\begin{itemize}
\item {Utilização:bras}
\end{itemize}
\begin{itemize}
\item {Utilização:Ant.}
\end{itemize}
\begin{itemize}
\item {Utilização:Fig.}
\end{itemize}
Bispote.
Pessôa pouco limpa ou desprezível.
(Talvez t. afr.)
\section{Cabungueira}
\begin{itemize}
\item {Grp. gram.:f.}
\end{itemize}
\begin{itemize}
\item {Utilização:Ant.}
\end{itemize}
Mulher, que despejava ou lavava cabungos.
\section{Cabureira}
\begin{itemize}
\item {Grp. gram.:f.}
\end{itemize}
Árvore do bálsamo, no Perú.
(Cast. \textunderscore caburera\textunderscore )
\section{Caburo}
\begin{itemize}
\item {Grp. gram.:m.}
\end{itemize}
Espécie de coruja no Brasil.
\section{Caca}
\begin{itemize}
\item {Grp. gram.:f.}
\end{itemize}
\begin{itemize}
\item {Utilização:Infant.}
\end{itemize}
\begin{itemize}
\item {Proveniência:(Do lat. \textunderscore cacare\textunderscore )}
\end{itemize}
Excrementos.
Immundície.
\section{Caça}
\begin{itemize}
\item {Grp. gram.:f.}
\end{itemize}
Acto de caçar.
Animaes que se caçam.
Investigação.
Perseguição.
Conjunto das redes de um barco; apparelho de cincoenta a oitenta redes.
\section{Cá-cá}
\begin{itemize}
\item {Grp. gram.:m.}
\end{itemize}
Espécie de jôgo popular.
\section{Caçabe}
\begin{itemize}
\item {Grp. gram.:m.}
\end{itemize}
Recebedoria de rendas miúdas, na Índia portuguesa.
\section{Caçabe}
\begin{itemize}
\item {Grp. gram.:m.}
\end{itemize}
\begin{itemize}
\item {Utilização:Bras}
\end{itemize}
Farinha ou pão de raíz de mandioca.
\section{Cacabóia}
\begin{itemize}
\item {Grp. gram.:f.}
\end{itemize}
Serpente amphíbia do Brasil.
\section{Cacaborrada}
\begin{itemize}
\item {Grp. gram.:f.}
\end{itemize}
\begin{itemize}
\item {Utilização:Pleb.}
\end{itemize}
\begin{itemize}
\item {Proveniência:(De \textunderscore caca\textunderscore  + \textunderscore borrado\textunderscore )}
\end{itemize}
Asneira, disparate.
\section{Cacachim}
\begin{itemize}
\item {Grp. gram.:m.}
\end{itemize}
\begin{itemize}
\item {Utilização:T. da Bairrada}
\end{itemize}
O mesmo que \textunderscore megengra\textunderscore .
\section{Cacada}
\begin{itemize}
\item {Grp. gram.:f.}
\end{itemize}
\begin{itemize}
\item {Utilização:Prov.}
\end{itemize}
\begin{itemize}
\item {Utilização:trasm.}
\end{itemize}
\begin{itemize}
\item {Proveniência:(De \textunderscore caca\textunderscore )}
\end{itemize}
\textunderscore Cacada de ovos\textunderscore , grande porção delles, encontrados em sítio esconso, onde as gallinhas se tivessem acostumado a pôr.
\section{Cacada}
\begin{itemize}
\item {Grp. gram.:f.}
\end{itemize}
Porção de cacos ou trastes velhos.
Montão de coisas de pouco valor.
\section{Caçada}
\begin{itemize}
\item {Grp. gram.:f.}
\end{itemize}
Acto ou effeito de \textunderscore caçar\textunderscore .
Animaes que se caçaram.
\section{Caçadeira}
\begin{itemize}
\item {Grp. gram.:f.}
\end{itemize}
Pequena arma, própria para caçar.
Jaquetão, próprio para caçador.
Pequeno barco, de fundo chato e borda baixa, para caça de aves aquáticas.
\section{Caçadeiro}
\begin{itemize}
\item {Grp. gram.:adj.}
\end{itemize}
Próprio para a caça: \textunderscore espingarda caçadeira\textunderscore .
Que gosta de caça.
Que caça.
\section{Caçado}
\begin{itemize}
\item {Grp. gram.:adj.}
\end{itemize}
\begin{itemize}
\item {Utilização:Fam.}
\end{itemize}
Apanhado na caça.
Destro na caça, acostumado á caça (falando-se de cães).
Perito, experiente, matreiro.
\section{Caçador}
\begin{itemize}
\item {Grp. gram.:adj.}
\end{itemize}
\begin{itemize}
\item {Grp. gram.:M.}
\end{itemize}
Que caça.
Aquelle que caça.
Aquelle que exerce a profissão de caçar.
Soldado de infantaria ou cavallaria, que combate insulado ou por pelotões.
\section{Caçador-viajante}
\begin{itemize}
\item {Grp. gram.:m.}
\end{itemize}
Espécie de jôgo popular.
\section{Cacaforro}
\begin{itemize}
\item {fónica:fô}
\end{itemize}
\begin{itemize}
\item {Grp. gram.:m.}
\end{itemize}
\begin{itemize}
\item {Utilização:Prov.}
\end{itemize}
\begin{itemize}
\item {Utilização:trasm.}
\end{itemize}
Espécie de cogumelo muito molle, sem pé e que contém uma substância viscosa que, depois de sêca, se faz em pó acafetado.
\section{Cacaieiro}
\begin{itemize}
\item {Grp. gram.:adj.}
\end{itemize}
\begin{itemize}
\item {Utilização:Bras}
\end{itemize}
Que traz cacaio.
\section{Cacaio}
\begin{itemize}
\item {Grp. gram.:m.}
\end{itemize}
\begin{itemize}
\item {Utilização:Bras}
\end{itemize}
Alforge ou saco de viagem, prêso por baixo dos braços e pendurado nas costas.
\section{Çacalado}
\begin{itemize}
\item {Grp. gram.:adj.}
\end{itemize}
Açacalado. Cf. \textunderscore Viriato Trág.\textunderscore , I, 55.
\section{Cacália}
\begin{itemize}
\item {Grp. gram.:f.}
\end{itemize}
\begin{itemize}
\item {Proveniência:(Gr. \textunderscore kakalia\textunderscore )}
\end{itemize}
Gênero de plantas, da fam. das compostas.
\section{Caçamba}
\begin{itemize}
\item {Grp. gram.:f.}
\end{itemize}
\begin{itemize}
\item {Utilização:Bras}
\end{itemize}
\begin{itemize}
\item {Utilização:Ext.}
\end{itemize}
Alcatruz.
Balde, preso numa corda enrolada num sarilho ou nora, para tirar água dos poços.
Qualquer balde.
Estribo em fórma de chinelo.
Espécie de cangalhas.
\section{Caçamulo}
\begin{itemize}
\item {Grp. gram.:m.}
\end{itemize}
\begin{itemize}
\item {Utilização:Prov.}
\end{itemize}
Espiga de milho, depois de esbagoada.
(Corr. de \textunderscore cascabulho\textunderscore ?)
\section{Cacangélico}
\begin{itemize}
\item {Grp. gram.:m.}
\end{itemize}
Membro de uma seita lutherana, da qual se dizia que estava em communicação com os anjos.
\section{Caçante}
\begin{itemize}
\item {Grp. gram.:adj.}
\end{itemize}
Diz-se do animal, que nos brasões se representa em acção de caçar.
\section{Cação}
\begin{itemize}
\item {Grp. gram.:m.}
\end{itemize}
\begin{itemize}
\item {Utilização:T. de Vouzela}
\end{itemize}
\begin{itemize}
\item {Utilização:Pleb.}
\end{itemize}
Peixe marítimo, mustelídeo, (\textunderscore mustelus vulgaris\textunderscore ).
Cada uma das duas pedras lateraes, postas em comêço de arco abatido sôbre a torça, para que a parede não carregue sôbre esta.
Rameira.
\section{Cação}
\begin{itemize}
\item {Grp. gram.:m.}
\end{itemize}
\begin{itemize}
\item {Utilização:Bras}
\end{itemize}
Chibé, em que o sal substitue o açúcar, e em que se deita pimenta.
\section{Caçapear}
\begin{itemize}
\item {Grp. gram.:v. i.}
\end{itemize}
\begin{itemize}
\item {Utilização:Prov.}
\end{itemize}
\begin{itemize}
\item {Utilização:alent.}
\end{itemize}
\begin{itemize}
\item {Proveniência:(De \textunderscore caçapo\textunderscore )}
\end{itemize}
Diz-se do andar ordinário do coêlho.
\section{Caçapeira}
\begin{itemize}
\item {Grp. gram.:f.}
\end{itemize}
\begin{itemize}
\item {Utilização:Prov.}
\end{itemize}
\begin{itemize}
\item {Utilização:alent.}
\end{itemize}
\begin{itemize}
\item {Proveniência:(De \textunderscore caçapo\textunderscore )}
\end{itemize}
Ninho de coêlhos.
\section{Caçapeiro}
\begin{itemize}
\item {Grp. gram.:m.}
\end{itemize}
\begin{itemize}
\item {Utilização:Prov.}
\end{itemize}
Nome vulgar da \textunderscore dedaleira\textunderscore .
(Relaciona-se talvez com \textunderscore cachapeira\textunderscore )
\section{Caçapelho}
\begin{itemize}
\item {fónica:pê}
\end{itemize}
\begin{itemize}
\item {Grp. gram.:m.}
\end{itemize}
\begin{itemize}
\item {Utilização:T. do Fundão}
\end{itemize}
Corrida aos saltos, num só pé.
\section{Caçapo}
\begin{itemize}
\item {Grp. gram.:m.}
\end{itemize}
Coêlho novo.
Homem baixo e grosso.
(Cp. cast. \textunderscore gazapo\textunderscore )
\section{Caçapo}
\begin{itemize}
\item {Grp. gram.:m.}
\end{itemize}
\begin{itemize}
\item {Utilização:Bras}
\end{itemize}
Espécie de trigo, de cuja palha se fazem os afamados chapeus de Florença.
\section{Caçar}
\begin{itemize}
\item {Grp. gram.:v. t.}
\end{itemize}
\begin{itemize}
\item {Utilização:Náut.}
\end{itemize}
\begin{itemize}
\item {Proveniência:(De um lat. hyp. \textunderscore captiare\textunderscore , de \textunderscore captus\textunderscore )}
\end{itemize}
Procurar ou perseguir (animaes), para os matar ou apanhar vivos.
O mesmo que \textunderscore pescar\textunderscore :«\textunderscore não lhes lembrando que era defeso caçar cõ cóqua... matárão alguns peixes\textunderscore ». \textunderscore Alvará\textunderscore  de D. Sebast., in \textunderscore Rev. Lus.\textunderscore , XV, 123.
Apanhar, conseguir.
Colher, atar (velas, cabos, etc.).
\section{Cacará}
\begin{itemize}
\item {Grp. gram.:m.}
\end{itemize}
\begin{itemize}
\item {Utilização:Bras}
\end{itemize}
Bisbórria.
\section{Cacaracá}
\begin{itemize}
\item {Grp. gram.:m.}
\end{itemize}
\begin{itemize}
\item {Utilização:Pop.}
\end{itemize}
\textunderscore De cacaracá\textunderscore ; de pouca monta; insignificante.
\section{Cacarecos}
\begin{itemize}
\item {Grp. gram.:m. pl.}
\end{itemize}
\begin{itemize}
\item {Utilização:Bras}
\end{itemize}
O mesmo que \textunderscore cacaréus\textunderscore .
\section{Cacarejador}
\begin{itemize}
\item {Grp. gram.:adj.}
\end{itemize}
Que cacareja.
Chocalheiro.
\section{Cacarejar}
\begin{itemize}
\item {Grp. gram.:v. i.}
\end{itemize}
\begin{itemize}
\item {Utilização:Fig.}
\end{itemize}
\begin{itemize}
\item {Proveniência:(T. onom.)}
\end{itemize}
Cantar (a gallinha, e ainda outras aves que lhe imitam o canto).
Tagarelar, enfadando.
\section{Cacarejo}
\begin{itemize}
\item {Grp. gram.:m.}
\end{itemize}
Acto de \textunderscore cacarejar\textunderscore .
Canto da gallinha, depois de pôr o ovo.
\section{Cacarel}
\begin{itemize}
\item {Grp. gram.:m.  e  f.}
\end{itemize}
\begin{itemize}
\item {Utilização:Prov.}
\end{itemize}
Pessôa leviana e ingênua, que acha graça a tudo.
\section{Cacarelho}
\begin{itemize}
\item {fónica:carê}
\end{itemize}
\begin{itemize}
\item {Grp. gram.:m.}
\end{itemize}
\begin{itemize}
\item {Utilização:T. de Barcelos}
\end{itemize}
Indivíduo gago ou tartamudo.
\section{Caçarema}
\begin{itemize}
\item {Grp. gram.:f.}
\end{itemize}
Pequena formiga preta da Baía.
\section{Cacareno}
\begin{itemize}
\item {Grp. gram.:m.}
\end{itemize}
\begin{itemize}
\item {Utilização:Bras}
\end{itemize}
Traste velho ou muito usado.
Coisa de pouco valor.
(Cp. \textunderscore cacaréus\textunderscore )
\section{Cacaréos}
\begin{itemize}
\item {Grp. gram.:m. pl.}
\end{itemize}
\begin{itemize}
\item {Proveniência:(De \textunderscore caco\textunderscore )}
\end{itemize}
Cacos, trastes velhos.
\section{Caçarete}
\begin{itemize}
\item {fónica:çarê}
\end{itemize}
\begin{itemize}
\item {Grp. gram.:m.}
\end{itemize}
\begin{itemize}
\item {Proveniência:(De \textunderscore caçar\textunderscore )}
\end{itemize}
Espécie de rede de arrastar.
\section{Cacaréus}
\begin{itemize}
\item {Grp. gram.:m. pl.}
\end{itemize}
\begin{itemize}
\item {Proveniência:(De \textunderscore caco\textunderscore )}
\end{itemize}
Cacos, trastes velhos.
\section{Cacaria}
\begin{itemize}
\item {Grp. gram.:f.}
\end{itemize}
Monte de cacos.
Porção de objectos velhos, inúteis.
\section{Cacaria}
\begin{itemize}
\item {Grp. gram.:f.}
\end{itemize}
\begin{itemize}
\item {Utilização:Bras}
\end{itemize}
\begin{itemize}
\item {Proveniência:(De \textunderscore Cacaria\textunderscore , n. p. da ilha, onde houve uma quadrilha de ladrões)}
\end{itemize}
Corja de ladrões; espelunca de ladrões.
\section{Cacáo}
\begin{itemize}
\item {Grp. gram.:m.}
\end{itemize}
Fruto, em fórma de amêndoa, em que se contém uma cápsula, que é a base do chocolate.
Árvore, que dá esse fruto.
(Or. mexicana)
\section{Caçarola}
\begin{itemize}
\item {Grp. gram.:f.}
\end{itemize}
\begin{itemize}
\item {Proveniência:(De \textunderscore caço\textunderscore )}
\end{itemize}
Caçoila; tacho de barro.
\section{Caçarote}
\begin{itemize}
\item {Grp. gram.:m.}
\end{itemize}
\begin{itemize}
\item {Utilização:T. de Aveiro}
\end{itemize}
Uma das quarteladas dos calões, com malha mais larga que a da alcanela, e menos que a do regalo.
\section{Caçarreta}
\begin{itemize}
\item {fónica:rê}
\end{itemize}
\begin{itemize}
\item {Grp. gram.:m.}
\end{itemize}
\begin{itemize}
\item {Utilização:Prov.}
\end{itemize}
\begin{itemize}
\item {Utilização:alent.}
\end{itemize}
\begin{itemize}
\item {Proveniência:(De \textunderscore caçar\textunderscore )}
\end{itemize}
Ruim caçador.
\section{Cacarrusso}
\begin{itemize}
\item {Grp. gram.:m.}
\end{itemize}
\begin{itemize}
\item {Utilização:T. de Avis}
\end{itemize}
\begin{itemize}
\item {Proveniência:(De \textunderscore caco\textunderscore )}
\end{itemize}
Qualquer vasilha velha.
\section{Caca-sebo}
\begin{itemize}
\item {Grp. gram.:m.}
\end{itemize}
\begin{itemize}
\item {Utilização:Bras}
\end{itemize}
\begin{itemize}
\item {Utilização:pop.}
\end{itemize}
Alfarrabista.
(Da alcunha de um alfarrabista)
\section{Cacateira}
\begin{itemize}
\item {Grp. gram.:f.}
\end{itemize}
Cucurbitácea da Índia portuguesa, (\textunderscore momordica charantia\textunderscore , Lin.).
\section{Cacatapuias}
\begin{itemize}
\item {Grp. gram.:m. pl.}
\end{itemize}
Indígenas da Guiana brasileira.
\section{Cacatório}
\begin{itemize}
\item {Grp. gram.:adj.}
\end{itemize}
\begin{itemize}
\item {Proveniência:(Do lat. \textunderscore cacare\textunderscore )}
\end{itemize}
Diz-se do medicamento que produz dejecções alvinas.
\section{Cacatu}
\begin{itemize}
\item {Grp. gram.:m.}
\end{itemize}
O mesmo que \textunderscore cacatua\textunderscore .
\section{Cacatua}
\begin{itemize}
\item {Grp. gram.:f.}
\end{itemize}
\begin{itemize}
\item {Proveniência:(Do mal. \textunderscore kakatua\textunderscore )}
\end{itemize}
Ave trepadora, semelhante ao papagaio.
\section{Cacáu}
\begin{itemize}
\item {Grp. gram.:m.}
\end{itemize}
Fruto, em fórma de amêndoa, em que se contém uma cápsula, que é a base do chocolate.
Árvore, que dá esse fruto.
(Or. mexicana)
\section{Cacaual}
\begin{itemize}
\item {Grp. gram.:m.}
\end{itemize}
Lugar, onde crescem cacaueiros.
\section{Cacaueiro}
\begin{itemize}
\item {Grp. gram.:m.}
\end{itemize}
Árvore esterculiácea, que produz cacau.
\section{Cacauzeiro}
\begin{itemize}
\item {Grp. gram.:m.}
\end{itemize}
O mesmo que \textunderscore cacaueiro\textunderscore .
\section{Caçave}
\begin{itemize}
\item {Grp. gram.:m.}
\end{itemize}
\begin{itemize}
\item {Utilização:Bras}
\end{itemize}
Farinha ou pão de raíz de mandioca.
\section{Cácea}
\begin{itemize}
\item {Grp. gram.:f.}
\end{itemize}
Acção de \textunderscore cacear\textunderscore .
\section{Cacear}
\begin{itemize}
\item {Grp. gram.:v. i.}
\end{itemize}
\begin{itemize}
\item {Utilização:Náut.}
\end{itemize}
\begin{itemize}
\item {Proveniência:(De \textunderscore caça\textunderscore )}
\end{itemize}
Caçar, garrar, descair (o navio).
\section{Cacebi}
\begin{itemize}
\item {Grp. gram.:m.}
\end{itemize}
\begin{itemize}
\item {Utilização:Ant.}
\end{itemize}
Espécie de taficira.
\section{Caceco}
\begin{itemize}
\item {Grp. gram.:m.}
\end{itemize}
Árvore de Angola.
\section{Caceia}
\begin{itemize}
\item {Grp. gram.:f.}
\end{itemize}
\begin{itemize}
\item {Proveniência:(De \textunderscore caça\textunderscore )}
\end{itemize}
Conjunto das redes que, amarradas entre si, os barcos de pesca lançam no alto mar.
\section{Cacembe}
\begin{itemize}
\item {Grp. gram.:m.}
\end{itemize}
\begin{itemize}
\item {Proveniência:(T. lund.)}
\end{itemize}
Arbusto angolense, de caule herbáceo e flôres roxas.
\section{Cacequeza}
\begin{itemize}
\item {Grp. gram.:f.}
\end{itemize}
\begin{itemize}
\item {Proveniência:(T. lund.)}
\end{itemize}
Arbusto angolense, de flôres amareladas.
\section{Cacera}
\begin{itemize}
\item {Grp. gram.:f.}
\end{itemize}
Planta comestível, na Índia.
\section{Caceta}
\begin{itemize}
\item {Grp. gram.:f.}
\end{itemize}
Espécie de vaso pharmaceutico, com um ralo no fundo.
(B. lat. \textunderscore capsetta\textunderscore , do lat. \textunderscore capsa\textunderscore )
\section{Cacetada}
\begin{itemize}
\item {Grp. gram.:f.}
\end{itemize}
Acção de bater com cacete; pancada de cacete.
\section{Cacetar}
\begin{itemize}
\item {Grp. gram.:v. t.}
\end{itemize}
Bater com cacete, espancar.
\section{Cacete}
\begin{itemize}
\item {fónica:cê}
\end{itemize}
\begin{itemize}
\item {Grp. gram.:m.}
\end{itemize}
\begin{itemize}
\item {Utilização:T. do Porto}
\end{itemize}
\begin{itemize}
\item {Utilização:Bras}
\end{itemize}
\begin{itemize}
\item {Grp. gram.:Adj.}
\end{itemize}
\begin{itemize}
\item {Utilização:Bras}
\end{itemize}
Pau curto e grosso.
Bordão, grosso numa das extremidades.
Moca.
Bordão.
Pão de trigo, sôbre o comprido.
Maçada, impertinência.
Maçador, impertinente.
(Provavelmente, dem. de \textunderscore caço\textunderscore , no sentido de moca)
\section{Caceteação}
\begin{itemize}
\item {Grp. gram.:f.}
\end{itemize}
\begin{itemize}
\item {Utilização:Bras}
\end{itemize}
Acto de cacetear; maçada.
\section{Cacetear}
\begin{itemize}
\item {Grp. gram.:v. t.}
\end{itemize}
\begin{itemize}
\item {Utilização:Bras}
\end{itemize}
Importunar.
(Cp. \textunderscore caceteiro\textunderscore )
\section{Caceteiro}
\begin{itemize}
\item {Grp. gram.:m.}
\end{itemize}
\begin{itemize}
\item {Utilização:Bras}
\end{itemize}
\begin{itemize}
\item {Proveniência:(De \textunderscore cacete\textunderscore )}
\end{itemize}
Aquelle que costuma trazer cacete.
Desordeiro.
Valentão.
Homem importuno.
\section{Cacha}
\begin{itemize}
\item {Grp. gram.:f.}
\end{itemize}
\begin{itemize}
\item {Utilização:P. us.}
\end{itemize}
\begin{itemize}
\item {Proveniência:(De \textunderscore cachar\textunderscore )}
\end{itemize}
Aquillo que se pratíca ás occultas.
Assimilação.
Ardil.
\section{Cacha}
\begin{itemize}
\item {Grp. gram.:f.}
\end{itemize}
\begin{itemize}
\item {Utilização:Prov.}
\end{itemize}
\begin{itemize}
\item {Utilização:trasm.}
\end{itemize}
\begin{itemize}
\item {Utilização:Ext.}
\end{itemize}
Pano da Índia, com que se fazem tangas.
Moéda indiana.
Metade de um lenço, cortado diagonalmente.
Metade de um fruto ou de qualquer coisa.
\section{Cachaça}
\begin{itemize}
\item {Grp. gram.:f.}
\end{itemize}
\begin{itemize}
\item {Utilização:Bras}
\end{itemize}
\begin{itemize}
\item {Grp. gram.:M.}
\end{itemize}
\begin{itemize}
\item {Utilização:Bras}
\end{itemize}
\begin{itemize}
\item {Utilização:fig.}
\end{itemize}
Aguardente, que se extrai das bôrras do melaço e das limpaduras do suco da cana de açúcar.
Espuma, produzida pela primeira fervura do suco da cana de açúcar.
Paixão predominante.
Bêbedo.
(Cast. \textunderscore cachaza\textunderscore )
\section{Cachação}
\begin{itemize}
\item {Grp. gram.:m.}
\end{itemize}
Pancada no cachaço.
\section{Cachaceira}
\begin{itemize}
\item {Grp. gram.:f.}
\end{itemize}
Grande cachaço.
Correia, que passa por detrás das orelhas da cavalgadura.
\section{Cachaceira}
\begin{itemize}
\item {Grp. gram.:f.}
\end{itemize}
\begin{itemize}
\item {Utilização:Bras}
\end{itemize}
\begin{itemize}
\item {Proveniência:(De \textunderscore cachaça\textunderscore )}
\end{itemize}
Lugar, em que se junta a cachaça, tirada das caldeiras do açúcar.
Bebedeira.
\section{Cachaceiro}
\begin{itemize}
\item {Grp. gram.:adj.}
\end{itemize}
\begin{itemize}
\item {Utilização:Pop.}
\end{itemize}
\begin{itemize}
\item {Proveniência:(De \textunderscore cachaço\textunderscore )}
\end{itemize}
Soberbo, arrogante, cachaçudo.
\section{Cachaceiro}
\begin{itemize}
\item {Grp. gram.:adj.}
\end{itemize}
\begin{itemize}
\item {Utilização:Bras}
\end{itemize}
Dado ao abuso da cachaça e que com ella se embriaga.
\section{Cachacipançudo}
\begin{itemize}
\item {fónica:chá}
\end{itemize}
\begin{itemize}
\item {Grp. gram.:adj.}
\end{itemize}
\begin{itemize}
\item {Proveniência:(De \textunderscore cachaço\textunderscore  + \textunderscore pança\textunderscore )}
\end{itemize}
Que tem cachaço muito volumoso. Cf. Garrett, \textunderscore D. Branca\textunderscore , 38.
\section{Cachaço}
\begin{itemize}
\item {Grp. gram.:m.}
\end{itemize}
\begin{itemize}
\item {Utilização:Pop.}
\end{itemize}
\begin{itemize}
\item {Utilização:bras}
\end{itemize}
\begin{itemize}
\item {Utilização:Ant.}
\end{itemize}
\begin{itemize}
\item {Utilização:Mad}
\end{itemize}
\begin{itemize}
\item {Proveniência:(De \textunderscore cacho\textunderscore ^2)}
\end{itemize}
Parte posterior do pescoço.
Pescoço grosso.
Soberba; arrogância.
Porco gordo, cevado. Cf. Quevedo, \textunderscore Aff. Afr.\textunderscore ; M. Soares, \textunderscore Diccion. Bras\textunderscore .
Varrão, porco de cobrição.
Pancada ou sôco, na parte posterior do pescoço. Cf. Camillo, \textunderscore Corja\textunderscore , 161.
\section{Cachaçudo}
\begin{itemize}
\item {Grp. gram.:adj.}
\end{itemize}
\begin{itemize}
\item {Utilização:Prov.}
\end{itemize}
\begin{itemize}
\item {Grp. gram.:M.}
\end{itemize}
\begin{itemize}
\item {Utilização:Prov.}
\end{itemize}
Soberbo, orgulhoso.
Que olha os outros com altivez.
Homem rico e poderoso.
(Colhido em Turquel)
\section{Cachada}
\begin{itemize}
\item {Grp. gram.:f.}
\end{itemize}
Alqueive; queima do mato, para adubar as terras.
\section{Cachafosgo}
\begin{itemize}
\item {fónica:fôs}
\end{itemize}
\begin{itemize}
\item {Grp. gram.:m.}
\end{itemize}
\begin{itemize}
\item {Utilização:Prov.}
\end{itemize}
\begin{itemize}
\item {Utilização:trasm.}
\end{itemize}
Buraco, feito na terra, e a que se não vê o fundo.
(Cp. \textunderscore cachar\textunderscore ^2)
\section{Cachafrelho}
\begin{itemize}
\item {fónica:frê}
\end{itemize}
\begin{itemize}
\item {Grp. gram.:m.}
\end{itemize}
\begin{itemize}
\item {Utilização:Prov.}
\end{itemize}
\begin{itemize}
\item {Utilização:trasm.}
\end{itemize}
Malsim.
Guarda-fiscal.
Fiscal do real de água.
\section{Cachafrilhas}
\begin{itemize}
\item {Grp. gram.:m.}
\end{itemize}
\begin{itemize}
\item {Utilização:Prov.}
\end{itemize}
\begin{itemize}
\item {Utilização:trasm.}
\end{itemize}
Indivíduo alto, magro e de má índole.
\section{Cachafundo}
\begin{itemize}
\item {Grp. gram.:m.}
\end{itemize}
\begin{itemize}
\item {Utilização:Prov.}
\end{itemize}
\begin{itemize}
\item {Proveniência:(De \textunderscore cachar\textunderscore ^2 + \textunderscore fundo\textunderscore ?)}
\end{itemize}
Mergulho de nadador. (Colhido em Barcelos)
\section{Cachagens}
\begin{itemize}
\item {Grp. gram.:f. pl.}
\end{itemize}
Ossos das fossas nasaes.
Guelras do peixe.
\section{Cachaíle}
\begin{itemize}
\item {Grp. gram.:m.}
\end{itemize}
Insecto africano, que constrói a sua habitação nas árvores.
\section{Cachalote}
\begin{itemize}
\item {Grp. gram.:m.}
\end{itemize}
\begin{itemize}
\item {Proveniência:(Fr. \textunderscore cacholot\textunderscore )}
\end{itemize}
Cetáceo, semelhante á baleia, mas com dentes.
\section{Cachambu}
\begin{itemize}
\item {Grp. gram.:m.}
\end{itemize}
\begin{itemize}
\item {Utilização:Bras. de Minas}
\end{itemize}
\begin{itemize}
\item {Utilização:Bras. de Goiás}
\end{itemize}
O mesmo que \textunderscore zabumba\textunderscore .
Barril, tapado com uma pelle esticada.
Espécie de dança, que se executa ao som do tambor. Cf. Verg. Franco, \textunderscore Viagem\textunderscore .
\section{Cachamola}
\begin{itemize}
\item {Grp. gram.:f.}
\end{itemize}
\begin{itemize}
\item {Utilização:Pop.}
\end{itemize}
Traços, que designam a cabeça no \textunderscore jôgo do homem\textunderscore .
(Alt. de \textunderscore cachimónia\textunderscore ?)
\section{Cachamorra}
\begin{itemize}
\item {fónica:mô}
\end{itemize}
\begin{itemize}
\item {Grp. gram.:f.}
\end{itemize}
Moca.
\section{Cachamorrada}
\begin{itemize}
\item {Grp. gram.:f.}
\end{itemize}
Pancada com cachamorra; mocada.
\section{Cachanolas}
\begin{itemize}
\item {Grp. gram.:f. pl.}
\end{itemize}
\begin{itemize}
\item {Utilização:T. da Guarda}
\end{itemize}
O mesmo que \textunderscore castanholas\textunderscore , instrumento.
\section{Cachão}
\begin{itemize}
\item {Grp. gram.:m.}
\end{itemize}
\begin{itemize}
\item {Proveniência:(Do lat. \textunderscore coctio\textunderscore ?)}
\end{itemize}
Borbulhão, borbotão.
Fervura.
\section{Cachapeira}
\begin{itemize}
\item {Grp. gram.:f.}
\end{itemize}
\begin{itemize}
\item {Utilização:Prov.}
\end{itemize}
\begin{itemize}
\item {Utilização:trasm.}
\end{itemize}
\begin{itemize}
\item {Utilização:Prov.}
\end{itemize}
Erva rasteira, de fôlhas crespas e recortadas, e de cujo centro sái uma haste lenhosa, de que os rapazes fazem frechas.
O mesmo que \textunderscore cachaporra\textunderscore .
(Colhido em Turquel)
(Cp. \textunderscore acachapar\textunderscore )
\section{Cachapim}
\begin{itemize}
\item {Grp. gram.:m.}
\end{itemize}
\begin{itemize}
\item {Utilização:Prov.}
\end{itemize}
\begin{itemize}
\item {Utilização:alent.}
\end{itemize}
O mesmo que \textunderscore megengra\textunderscore .
\section{Cachaporra}
\begin{itemize}
\item {fónica:pô}
\end{itemize}
\begin{itemize}
\item {Grp. gram.:f.}
\end{itemize}
\begin{itemize}
\item {Utilização:Pleb.}
\end{itemize}
O mesmo que \textunderscore cachamorra\textunderscore .
(Cast. \textunderscore cachiporra\textunderscore )
\section{Cachaporrada}
\begin{itemize}
\item {Grp. gram.:f.}
\end{itemize}
\begin{itemize}
\item {Utilização:Pleb.}
\end{itemize}
O mesmo que \textunderscore cachamorrada\textunderscore .
\section{Cachapução}
\begin{itemize}
\item {Grp. gram.:m.}
\end{itemize}
\begin{itemize}
\item {Utilização:Prov.}
\end{itemize}
\begin{itemize}
\item {Utilização:trasm.}
\end{itemize}
Grande cachapuço.
\section{Cachapuço}
\begin{itemize}
\item {Grp. gram.:m.}
\end{itemize}
\begin{itemize}
\item {Utilização:Prov.}
\end{itemize}
\begin{itemize}
\item {Utilização:trasm.}
\end{itemize}
Mergulho, de cabeça para baixo.
(Cp. \textunderscore cachapuz!\textunderscore )
\section{Cachapuz!}
\begin{itemize}
\item {Grp. gram.:interj.}
\end{itemize}
\begin{itemize}
\item {Proveniência:(T. onom.)}
\end{itemize}
Voz, significativa de quéda com estrondo ou quéda de chofre.
\section{Cachar}
\begin{itemize}
\item {Grp. gram.:v. t.}
\end{itemize}
\begin{itemize}
\item {Utilização:Prov.}
\end{itemize}
\begin{itemize}
\item {Utilização:minh.}
\end{itemize}
Arrotear, desbravar.
\section{Cachar}
\begin{itemize}
\item {Grp. gram.:v. i.}
\end{itemize}
\begin{itemize}
\item {Utilização:Des.}
\end{itemize}
\begin{itemize}
\item {Grp. gram.:V. t.}
\end{itemize}
Praticar occultamente um acto.
Esconder, tapar:«\textunderscore o resto do corpo cachavão com panos de seda\textunderscore ». Filinto, \textunderscore D. Man.\textunderscore , I, 379.
\section{Cacharamba}
\begin{itemize}
\item {Grp. gram.:f.}
\end{itemize}
\begin{itemize}
\item {Utilização:Des.}
\end{itemize}
O mesmo que \textunderscore bebedeira\textunderscore .
\section{Cacharambado}
\begin{itemize}
\item {Grp. gram.:adj.}
\end{itemize}
\begin{itemize}
\item {Utilização:Des.}
\end{itemize}
Bêbedo.
\section{Cachari}
\begin{itemize}
\item {Grp. gram.:m.}
\end{itemize}
O mesmo que \textunderscore caril\textunderscore .
\section{Cacharolete}
\begin{itemize}
\item {fónica:lê}
\end{itemize}
\begin{itemize}
\item {Grp. gram.:m.}
\end{itemize}
Bebida alcoólica, formada pela mistura de diversos licores.
\section{Cacharós}
\begin{itemize}
\item {Grp. gram.:m.}
\end{itemize}
\begin{itemize}
\item {Utilização:Prov.}
\end{itemize}
\begin{itemize}
\item {Utilização:trasm.}
\end{itemize}
Casa velha e feia, grande mas desconfortável.
\section{Cacharro}
\begin{itemize}
\item {Grp. gram.:m.}
\end{itemize}
\begin{itemize}
\item {Utilização:Prov.}
\end{itemize}
\begin{itemize}
\item {Utilização:trasm.}
\end{itemize}
\begin{itemize}
\item {Proveniência:(T. cast.)}
\end{itemize}
Jarro, gomil.
\section{Cachatim}
\begin{itemize}
\item {Grp. gram.:m.}
\end{itemize}
Variedade de goma asiática.
\section{Cache}
\begin{itemize}
\item {Grp. gram.:m.}
\end{itemize}
Moéda chinesa, correspondente á décima parte do condarim.
\section{Cacheadeira}
\begin{itemize}
\item {Grp. gram.:f.}
\end{itemize}
\begin{itemize}
\item {Utilização:Prov.}
\end{itemize}
\begin{itemize}
\item {Utilização:minh.}
\end{itemize}
\begin{itemize}
\item {Proveniência:(De \textunderscore cachear\textunderscore ^3)}
\end{itemize}
O mesmo que \textunderscore apalpadeira\textunderscore .
\section{Cacheado}
\begin{itemize}
\item {Grp. gram.:adj.}
\end{itemize}
\begin{itemize}
\item {Utilização:Bras}
\end{itemize}
\begin{itemize}
\item {Proveniência:(De \textunderscore cachear\textunderscore ^1)}
\end{itemize}
Espigado em cachos.
Penteado em fórma de cachos.
Encrespado, crespo.
\section{Cachear}
\begin{itemize}
\item {Grp. gram.:v. i.}
\end{itemize}
\begin{itemize}
\item {Proveniência:(De \textunderscore cacho\textunderscore ^1)}
\end{itemize}
Cobrir-se de cachos; produzir cachos.
\section{Cachear}
\begin{itemize}
\item {Grp. gram.:v. t.}
\end{itemize}
\begin{itemize}
\item {Utilização:Prov.}
\end{itemize}
\begin{itemize}
\item {Utilização:trasm.}
\end{itemize}
\begin{itemize}
\item {Proveniência:(De \textunderscore cacho\textunderscore ^2)}
\end{itemize}
Têr cóito com, machear, (falando-se dos machos que, no cóito, seguram com o bico ou com os dentes o pescoço da fêmea).
\section{Cachear}
\begin{itemize}
\item {Grp. gram.:v. i.}
\end{itemize}
\begin{itemize}
\item {Proveniência:(De \textunderscore cacha\textunderscore ^1)}
\end{itemize}
Exercer as funcções de cacheadeira ou apalpadeira.
\section{Cacheira}
\begin{itemize}
\item {Grp. gram.:f.}
\end{itemize}
Cacete, moca.
Pau tôsco.
\section{Cacheirada}
\begin{itemize}
\item {Grp. gram.:f.}
\end{itemize}
Pancada com cacheira.
\section{Cacheiro}
\begin{itemize}
\item {Grp. gram.:m.}
\end{itemize}
O mesmo que \textunderscore cacheira\textunderscore .
\section{Cacheiro}
\begin{itemize}
\item {Grp. gram.:adj.}
\end{itemize}
\begin{itemize}
\item {Utilização:Prov.}
\end{itemize}
\begin{itemize}
\item {Utilização:trasm.}
\end{itemize}
\begin{itemize}
\item {Proveniência:(De \textunderscore cacha\textunderscore ^1)}
\end{itemize}
Diz-se do ouriço, (animal), que se esconde sobre os espinhos próprios.
Ardiloso, astuto.
\section{Cachemira}
\begin{itemize}
\item {Grp. gram.:f.}
\end{itemize}
\begin{itemize}
\item {Proveniência:(De \textunderscore Cachemira\textunderscore , n. p.)}
\end{itemize}
Tecido de lan fina, fabricado na Índia.
\section{Cachené}
\begin{itemize}
\item {Grp. gram.:m.}
\end{itemize}
\begin{itemize}
\item {Utilização:Neol.}
\end{itemize}
\begin{itemize}
\item {Proveniência:(Do fr. \textunderscore cacher\textunderscore  + \textunderscore nez\textunderscore )}
\end{itemize}
Manta, ou lenço, com que se agasalha o pescoço.
\section{Cacherá}
\begin{itemize}
\item {Grp. gram.:m.}
\end{itemize}
Nome que, em Melres, se dá ao \textunderscore pintarroxo\textunderscore .
\section{Cacherim}
\begin{itemize}
\item {Grp. gram.:m.}
\end{itemize}
\begin{itemize}
\item {Utilização:Bras}
\end{itemize}
Navalha ou canivete.
Faca velha ou usada.
Cabo de faca.
\section{Cacheringuengue}
\begin{itemize}
\item {Grp. gram.:m.}
\end{itemize}
\begin{itemize}
\item {Utilização:Bras}
\end{itemize}
Pequena faca, velha ou muito usada.
\section{Cacheta}
\begin{itemize}
\item {fónica:chê}
\end{itemize}
\begin{itemize}
\item {Grp. gram.:f.}
\end{itemize}
\begin{itemize}
\item {Utilização:Bras}
\end{itemize}
\begin{itemize}
\item {Proveniência:(De \textunderscore cacha\textunderscore ^1)}
\end{itemize}
Acto de ficar, por cálculo, em ponto baixo, no jôgo do sete-e-meio.
\section{Cachético}
\begin{itemize}
\item {fónica:qué}
\end{itemize}
\begin{itemize}
\item {Grp. gram.:adj.}
\end{itemize}
Que tem cachèxía.
\section{Cachexia}
\begin{itemize}
\item {fónica:quécsí}
\end{itemize}
\begin{itemize}
\item {Grp. gram.:f.}
\end{itemize}
\begin{itemize}
\item {Proveniência:(Gr. \textunderscore kakhexia\textunderscore )}
\end{itemize}
Fraqueza geral do organismo.
Abatimento senil.
\section{Cachia}
\begin{itemize}
\item {Grp. gram.:f.}
\end{itemize}
Flôr da esponjeira; esponja.
\section{Cachiar}
\begin{itemize}
\item {Grp. gram.:v. t.}
\end{itemize}
\begin{itemize}
\item {Utilização:Prov.}
\end{itemize}
\begin{itemize}
\item {Utilização:trasm.}
\end{itemize}
Arrotear (terras).
(Cp. \textunderscore cachar\textunderscore ^1)
\section{Cachicama}
\begin{itemize}
\item {Grp. gram.:m.}
\end{itemize}
(V.armadilho)
\section{Cachicha!}
\begin{itemize}
\item {Grp. gram.:interj.}
\end{itemize}
\begin{itemize}
\item {Utilização:Prov.}
\end{itemize}
\begin{itemize}
\item {Utilização:trasm.}
\end{itemize}
(Designa repugnância ou nojo)
\section{Cachichi}
\begin{itemize}
\item {Grp. gram.:adj.}
\end{itemize}
\begin{itemize}
\item {Utilização:Bras}
\end{itemize}
Diz-se da aguardente de inferior ou má qualidade.
\section{Cachiço}
\begin{itemize}
\item {Grp. gram.:m.}
\end{itemize}
\begin{itemize}
\item {Utilização:Prov.}
\end{itemize}
\begin{itemize}
\item {Utilização:Beir.}
\end{itemize}
\begin{itemize}
\item {Utilização:T. do Fundão}
\end{itemize}
\begin{itemize}
\item {Utilização:Prov.}
\end{itemize}
\begin{itemize}
\item {Utilização:minh.}
\end{itemize}
\begin{itemize}
\item {Utilização:Prov.}
\end{itemize}
\begin{itemize}
\item {Utilização:dur.}
\end{itemize}
Robalo pequeno.
Bagaço de uva.
Palha miúda, cortada pelo mangual, na occasião da malha.
O mesmo que \textunderscore graveto\textunderscore .
Espiga de milho, depois de debulhada.
\section{Cachilras}
\begin{itemize}
\item {Grp. gram.:m. pl.}
\end{itemize}
\begin{itemize}
\item {Utilização:Gír.}
\end{itemize}
Seios.
\section{Cachiman}
\begin{itemize}
\item {Grp. gram.:f.}
\end{itemize}
Árvore anonácea, das Antilhas.
\section{Cachimana}
\begin{itemize}
\item {Grp. gram.:f.}
\end{itemize}
\begin{itemize}
\item {Proveniência:(De \textunderscore cachar\textunderscore  + \textunderscore manha\textunderscore )}
\end{itemize}
Ardil, artimanha.
\section{Cachimbada}
\begin{itemize}
\item {Grp. gram.:f.}
\end{itemize}
Porção de tabaco, que se mete no cachimbo.
Fumaça do cachimbo.
\section{Cachimbador}
\begin{itemize}
\item {Grp. gram.:m.  e  adj.}
\end{itemize}
O que cachimba.
\section{Cachimbante}
\begin{itemize}
\item {Grp. gram.:adj.}
\end{itemize}
Que cachimba. Cf. Filinto, III, 247.
\section{Cachimbar}
\begin{itemize}
\item {Grp. gram.:v. i.}
\end{itemize}
\begin{itemize}
\item {Utilização:Pop.}
\end{itemize}
\begin{itemize}
\item {Grp. gram.:V. t.}
\end{itemize}
\begin{itemize}
\item {Utilização:Bras}
\end{itemize}
\begin{itemize}
\item {Grp. gram.:V. p.}
\end{itemize}
\begin{itemize}
\item {Proveniência:(De \textunderscore cachimbo\textunderscore )}
\end{itemize}
Fumar por cachimbo.
Não fazer caso, votar desprêzo.
Meditar, ponderar.
(a mesma significação).
\section{Cachimbo}
\begin{itemize}
\item {Grp. gram.:m.}
\end{itemize}
\begin{itemize}
\item {Utilização:Bras. de Pernambuco}
\end{itemize}
\begin{itemize}
\item {Grp. gram.:Pl.}
\end{itemize}
\begin{itemize}
\item {Utilização:Gír.}
\end{itemize}
\begin{itemize}
\item {Proveniência:(Do químb. \textunderscore quixima\textunderscore )}
\end{itemize}
Apparelho de fumador, composto de um fornilho, em que se deita tabaco, e de um tubo por onde se sorve o fumo.
Peça de ferro, em que entra o espigão do leme da porta.
Buraco, em que se encaixa a vela do castiçal.
Bebida, preparada com aguardente e mel.
Pés.
\section{Cachimónia}
\begin{itemize}
\item {Grp. gram.:f.}
\end{itemize}
\begin{itemize}
\item {Utilização:Pop.}
\end{itemize}
Cabeça.
Capacidade, juizo.
(Cp. \textunderscore cacheira\textunderscore  e \textunderscore cachola\textunderscore )
\section{Cachinada}
\begin{itemize}
\item {Grp. gram.:f.}
\end{itemize}
\begin{itemize}
\item {Proveniência:(De \textunderscore cachinar\textunderscore )}
\end{itemize}
Gargalhada por motejo.
\section{Cachinador}
\begin{itemize}
\item {Grp. gram.:m.}
\end{itemize}
Aquelle que cachina.
\section{Cachinar}
\begin{itemize}
\item {Grp. gram.:v. i.}
\end{itemize}
\begin{itemize}
\item {Proveniência:(Do lat. \textunderscore cachinnare\textunderscore , t. onom.)}
\end{itemize}
Rir ruidosamente; soltar gargalhadas de escárneo.
\section{Cachinche}
\begin{itemize}
\item {Grp. gram.:m.}
\end{itemize}
\begin{itemize}
\item {Utilização:Bras}
\end{itemize}
O mesmo que \textunderscore cachinguelê\textunderscore .
\section{Cachineses}
\begin{itemize}
\item {Grp. gram.:m. pl.}
\end{itemize}
Selvagens, que habitavam em Minas-Geraes.
\section{Cachingar}
\begin{itemize}
\item {Grp. gram.:v. i.}
\end{itemize}
\begin{itemize}
\item {Utilização:Bras. do N}
\end{itemize}
O mesmo que \textunderscore coxear\textunderscore .
\section{Cachinguelê}
\begin{itemize}
\item {Grp. gram.:m.}
\end{itemize}
\begin{itemize}
\item {Utilização:Bras}
\end{itemize}
\begin{itemize}
\item {Utilização:Fig.}
\end{itemize}
Animal, da ordem dos roedores, (\textunderscore sciurus aestuans\textunderscore , Lin.).
Indivíduo magro, esperto e mexediço.
\section{Cachinha}
\begin{itemize}
\item {Grp. gram.:f.}
\end{itemize}
\begin{itemize}
\item {Utilização:Prov.}
\end{itemize}
\begin{itemize}
\item {Utilização:trasm.}
\end{itemize}
\begin{itemize}
\item {Proveniência:(De \textunderscore cachar\textunderscore ^2)}
\end{itemize}
Conlúio; combinação secreta.
Acôrdo mútuo.
\section{Cachirim}
\begin{itemize}
\item {Grp. gram.:m.}
\end{itemize}
\begin{itemize}
\item {Utilização:Bras. do N}
\end{itemize}
\begin{itemize}
\item {Utilização:Bras. do Rio}
\end{itemize}
Caldo de beiju, diluído em água.
Licor fermentado, extrahido da mandioca por destillação.
\section{Cacho}
\begin{itemize}
\item {Grp. gram.:m.}
\end{itemize}
\begin{itemize}
\item {Utilização:Prov.}
\end{itemize}
\begin{itemize}
\item {Utilização:trasm.}
\end{itemize}
\begin{itemize}
\item {Utilização:Prov.}
\end{itemize}
\begin{itemize}
\item {Grp. gram.:Pl.}
\end{itemize}
\begin{itemize}
\item {Utilização:Prov.}
\end{itemize}
\begin{itemize}
\item {Utilização:alent.}
\end{itemize}
\begin{itemize}
\item {Grp. gram.:Loc.}
\end{itemize}
\begin{itemize}
\item {Utilização:fam.}
\end{itemize}
\begin{itemize}
\item {Proveniência:(Do lat. \textunderscore capulus\textunderscore ?)}
\end{itemize}
Conjunto de flôres ou frutos, sustentados por pecíolos e disposto em escádeas num eixo commum.
Reunião de objectos, dispostos à maneira de cacho.
Pequena porção, bocado.
Cacho de uvas: \textunderscore foi á vinha e comeu dois cachos\textunderscore .
Espigas ou réstias de espigas, que resistem á primeira debulha, e que se juntam para formar \textunderscore eiras de cachos\textunderscore .
\textunderscore Bêbedo como um cacho\textunderscore , muito bêbedo.
\section{Cacho}
\begin{itemize}
\item {Grp. gram.:m.}
\end{itemize}
\begin{itemize}
\item {Utilização:Des.}
\end{itemize}
O mesmo que \textunderscore pescoço\textunderscore .
\section{Cachoante}
\begin{itemize}
\item {Grp. gram.:adj.}
\end{itemize}
Que cachôa.
\section{Cachoar}
\begin{itemize}
\item {Grp. gram.:v. i.}
\end{itemize}
\begin{itemize}
\item {Utilização:Neol.}
\end{itemize}
Formar cachão, formar cachoeira.
\section{Cacho-de-pedra}
\begin{itemize}
\item {Grp. gram.:m.}
\end{itemize}
\begin{itemize}
\item {Utilização:T. de Aveiro}
\end{itemize}
Alga marinha, (\textunderscore fucus variculosus\textunderscore , Lin.).
\section{Cachoeira}
\begin{itemize}
\item {Grp. gram.:f.}
\end{itemize}
\begin{itemize}
\item {Proveniência:(De \textunderscore cachão\textunderscore )}
\end{itemize}
Corrente de água, que se despenha, levantando cachão. Catadupa; catarata.
\section{Cachola}
\begin{itemize}
\item {Grp. gram.:f.}
\end{itemize}
\begin{itemize}
\item {Utilização:Pop.}
\end{itemize}
\begin{itemize}
\item {Utilização:Náut.}
\end{itemize}
\begin{itemize}
\item {Utilização:Prov.}
\end{itemize}
\begin{itemize}
\item {Utilização:Prov.}
\end{itemize}
\begin{itemize}
\item {Utilização:Prov.}
\end{itemize}
\begin{itemize}
\item {Utilização:trasm.}
\end{itemize}
Cabeça, cachimónia.
Pau, que se prega sobre o calcês, para obstar á introducção da água nos encaixes dos madeiros.
Fígado de porco ou de outro animal.
Cabeça de sardinha e de outros peixes.
O mesmo que \textunderscore cacholo\textunderscore .
(Cp. cast. \textunderscore cholla\textunderscore )
\section{Cacholada}
\begin{itemize}
\item {Grp. gram.:f.}
\end{itemize}
Porção de cachola ou guisado de bofes, fígado, sangue e banha de porco; sarrabulho.
\section{Cacholeta}
\begin{itemize}
\item {fónica:lê}
\end{itemize}
\begin{itemize}
\item {Grp. gram.:f.}
\end{itemize}
\begin{itemize}
\item {Utilização:Pop.}
\end{itemize}
\begin{itemize}
\item {Proveniência:(De \textunderscore cachola\textunderscore )}
\end{itemize}
Pancada na cabeça.
Offensa.
Censura.
\section{Cacholo}
\begin{itemize}
\item {fónica:chô}
\end{itemize}
\begin{itemize}
\item {Grp. gram.:m.}
\end{itemize}
\begin{itemize}
\item {Utilização:Prov.}
\end{itemize}
\begin{itemize}
\item {Utilização:trasm.}
\end{itemize}
Jôgo de rapazes.
\section{Cacholote}
\begin{itemize}
\item {Grp. gram.:m.}
\end{itemize}
(V.cachalote)
\section{Cachombo}
\begin{itemize}
\item {Grp. gram.:m.}
\end{itemize}
Espécie de mocho africano.
\section{Cachonceira}
\begin{itemize}
\item {Grp. gram.:f.}
\end{itemize}
\begin{itemize}
\item {Utilização:Ant.}
\end{itemize}
\begin{itemize}
\item {Proveniência:(De \textunderscore cacho\textunderscore ^1)}
\end{itemize}
Cabello comprido, em anéis.
\section{Cachonda}
\begin{itemize}
\item {Grp. gram.:adj. f.}
\end{itemize}
\begin{itemize}
\item {Utilização:Prov.}
\end{itemize}
\begin{itemize}
\item {Utilização:trasm.}
\end{itemize}
\begin{itemize}
\item {Utilização:beir.}
\end{itemize}
Diz-se da fêmea, que anda na sazão de ir ao macho, (especialmente falando-se de cadellas).
(Cp. cast. \textunderscore cachondo\textunderscore )
\section{Cachofelho}
\begin{itemize}
\item {fónica:fê}
\end{itemize}
\begin{itemize}
\item {Grp. gram.:m.}
\end{itemize}
\begin{itemize}
\item {Utilização:Prov.}
\end{itemize}
\begin{itemize}
\item {Utilização:minh.}
\end{itemize}
Casa pequena, acanhada. Cf. O. Pratt, \textunderscore Ling. Minh.\textunderscore 
\section{Cachondé}
\begin{itemize}
\item {Grp. gram.:m.}
\end{itemize}
O mesmo que \textunderscore cachundé\textunderscore .
\section{Cachopa}
\begin{itemize}
\item {Grp. gram.:f.}
\end{itemize}
Rapariga.
(Cp. \textunderscore cachopo\textunderscore ^1)
\section{Cachopa}
\begin{itemize}
\item {Grp. gram.:f.}
\end{itemize}
Casta de uva do Doiro.
\section{Cachoparrão}
\begin{itemize}
\item {Grp. gram.:m.}
\end{itemize}
\begin{itemize}
\item {Proveniência:(De \textunderscore cachopo\textunderscore ^1)}
\end{itemize}
Rapagão.
\section{Cachopeiro}
\begin{itemize}
\item {Grp. gram.:m.}
\end{itemize}
Variedade de figo branco e grande.
\section{Cachopelho}
\begin{itemize}
\item {fónica:pê}
\end{itemize}
\begin{itemize}
\item {Grp. gram.:m.}
\end{itemize}
\begin{itemize}
\item {Utilização:Prov.}
\end{itemize}
\begin{itemize}
\item {Utilização:minh.}
\end{itemize}
Casa pequena, acanhada. Cf. O. Pratt, \textunderscore Ling. Minh.\textunderscore 
\section{Cachopice}
\begin{itemize}
\item {Grp. gram.:f.}
\end{itemize}
\begin{itemize}
\item {Proveniência:(De \textunderscore cachopo\textunderscore ^1)}
\end{itemize}
Qualidade de cachopo ou cachopa.
Rapaziada, acção própria de rapazes.
\section{Cachopo}
\begin{itemize}
\item {fónica:chô}
\end{itemize}
\begin{itemize}
\item {Grp. gram.:m.}
\end{itemize}
\begin{itemize}
\item {Utilização:Pop.}
\end{itemize}
O mesmo que \textunderscore rapaz\textunderscore ^1.
\section{Cachopo}
\begin{itemize}
\item {fónica:chô}
\end{itemize}
\begin{itemize}
\item {Grp. gram.:m.}
\end{itemize}
Baixio, escolho.
Perigo; obstáculo.
Revéses.
\section{Cachopo}
\begin{itemize}
\item {fónica:chô}
\end{itemize}
\begin{itemize}
\item {Grp. gram.:m.}
\end{itemize}
Variedade de figueira algarvia.
\section{Cachopucho}
\begin{itemize}
\item {Grp. gram.:m.}
\end{itemize}
Nome de uma droga de Guzerate.
\section{Cachorra}
\begin{itemize}
\item {fónica:chô}
\end{itemize}
\begin{itemize}
\item {Grp. gram.:f.}
\end{itemize}
\begin{itemize}
\item {Utilização:Fig.}
\end{itemize}
\begin{itemize}
\item {Proveniência:(De \textunderscore cachorro\textunderscore )}
\end{itemize}
Cadela muito nova.
Filha, ainda tenra, de outros animaes do gênero canino.
Espécie de peixe, semelhante ao atum.
Mulher maliciosa ou má.
Antiga peça de artilharia. Cf. \textunderscore Livro das Monções\textunderscore , n.^o 13.
\section{Cachorrada}
\begin{itemize}
\item {Grp. gram.:f.}
\end{itemize}
\begin{itemize}
\item {Utilização:Carp.}
\end{itemize}
\begin{itemize}
\item {Proveniência:(De \textunderscore cachorro\textunderscore )}
\end{itemize}
Bando de cães pequenos.
Conjunto dos cachorros de uma construcção.
Acto indecoroso.
Gente reles.
\section{Cachorrado}
\begin{itemize}
\item {Grp. gram.:adj.}
\end{itemize}
Sustentado pelos cachorros de uma construcção.
\section{Cachorreira}
\begin{itemize}
\item {Grp. gram.:f.}
\end{itemize}
\begin{itemize}
\item {Utilização:Ant.}
\end{itemize}
Cabelleira muito comprida.
(Cp. \textunderscore cachonceira\textunderscore )
\section{Cachorro}
\begin{itemize}
\item {fónica:chô}
\end{itemize}
\begin{itemize}
\item {Grp. gram.:m.}
\end{itemize}
Cão novo e pequeno.
Filho recém-nascido do lobo, do leão, ou de animaes congêneres.
Escora, ou peça saliente que, numa construcção, sustenta uma cimalha, friso, sacada, etc.
Escora do navio, no estaleiro.
Pau, que bate na calha da atafona, para fazer cair o grão.
Homem ordinário, mau.
Rapaz travesso, turbulento.
Peixe de Portugal.
(Cp. lat. \textunderscore catulus\textunderscore )
\section{Cachorro-de-água}
\begin{itemize}
\item {Grp. gram.:m.}
\end{itemize}
\begin{itemize}
\item {Utilização:Bras}
\end{itemize}
Quadrúpede amphíbio dos rios centraes do Brasil.
\section{Cachorros-de-prôa}
\begin{itemize}
\item {Grp. gram.:m. pl.}
\end{itemize}
\begin{itemize}
\item {Utilização:Gír.}
\end{itemize}
Seios de mulher.
\section{Cachotão}
\begin{itemize}
\item {Grp. gram.:m.}
\end{itemize}
Árvore de Lourenço-Marques.
\section{Cachu}
\begin{itemize}
\item {Grp. gram.:m.}
\end{itemize}
O mesmo que \textunderscore catechu\textunderscore .
\section{Cachuça}
\begin{itemize}
\item {Grp. gram.:f.}
\end{itemize}
\begin{itemize}
\item {Utilização:Prov.}
\end{itemize}
\begin{itemize}
\item {Utilização:minh.}
\end{itemize}
A fêmea do melro.
\section{Cachucha}
\begin{itemize}
\item {Grp. gram.:f.}
\end{itemize}
Certa dança espanhola.
Pequenina embarcação, em alguns portos da América.
(Cast. \textunderscore cachucha\textunderscore )
\section{Cachucho}
\begin{itemize}
\item {Grp. gram.:m.}
\end{itemize}
\begin{itemize}
\item {Utilização:Prov.}
\end{itemize}
\begin{itemize}
\item {Utilização:beir.}
\end{itemize}
\begin{itemize}
\item {Utilização:Gír.}
\end{itemize}
Peixe vulgar, da fam. dos pristipomátidas.
Nome de uma planta.
Medulla das pennas.
Papelota no cabello, formando anéis.
Chaveta da dobradiça.
Anel grosso de oiro.
(Cp. cast. \textunderscore cachucho\textunderscore )
\section{Cachudo}
\begin{itemize}
\item {Grp. gram.:m.  e  adj.}
\end{itemize}
\begin{itemize}
\item {Grp. gram.:Adj.}
\end{itemize}
\begin{itemize}
\item {Proveniência:(De \textunderscore cacho\textunderscore ^1)}
\end{itemize}
Casta de uva preta minhota.
Variedade de trigo rijo.
Casta de uva branca alentejana.
Que dá grandes cachos: \textunderscore parreira cachuda\textunderscore .
\section{Cachul}
\begin{itemize}
\item {Grp. gram.:m.}
\end{itemize}
Planta americana, espécie de verónica.
\section{Cachumba}
\begin{itemize}
\item {Grp. gram.:f.}
\end{itemize}
\begin{itemize}
\item {Utilização:Bras. do Rio}
\end{itemize}
Inflammação nas parótidas; trasorelho.
\section{Cachundé}
\begin{itemize}
\item {Grp. gram.:m.}
\end{itemize}
Grãos vegetaes ou confecções, que se trazem na bôca para lhe communicar bom cheiro.
(Cp. \textunderscore cachu\textunderscore )
\section{Cachutânico}
\begin{itemize}
\item {Grp. gram.:adj.}
\end{itemize}
Diz-se de um ácido, extrahido da \textunderscore acácia catechu\textunderscore .
\section{Cachutânnico}
\begin{itemize}
\item {Grp. gram.:adj.}
\end{itemize}
Diz-se de um ácido, extrahido da \textunderscore acácia catechu\textunderscore .
\section{Cachútico}
\begin{itemize}
\item {Grp. gram.:adj.}
\end{itemize}
Diz-se de um ácido, extrahido do \textunderscore cachu\textunderscore .
\section{Cacica}
\begin{itemize}
\item {Grp. gram.:f.}
\end{itemize}
\begin{itemize}
\item {Utilização:Bras}
\end{itemize}
Licor, o mesmo que \textunderscore cachirim\textunderscore .
\section{Cacifeiro}
\begin{itemize}
\item {Grp. gram.:m.}
\end{itemize}
\begin{itemize}
\item {Proveniência:(De \textunderscore cacifo\textunderscore )}
\end{itemize}
Cónego thesoireiro do cabido de Coimbra.
\section{Cacifo}
\begin{itemize}
\item {Grp. gram.:m.}
\end{itemize}
\begin{itemize}
\item {Utilização:Ant.}
\end{itemize}
\begin{itemize}
\item {Proveniência:(Do ár. \textunderscore cafiz\textunderscore )}
\end{itemize}
Cofre, caixa.
Gaveta.
Buraca, no jôgo da bola.
Recanto.
Pequeno armário, aberto na parede.
Medida de capacidade, equivalente ao celamim.
\section{Cacifro}
\begin{itemize}
\item {Grp. gram.:m.}
\end{itemize}
O mesmo que \textunderscore cacifo\textunderscore . Cf. Camillo, \textunderscore Myst. de Lisbôa\textunderscore , I, 206.
\section{Cacim}
\begin{itemize}
\item {Grp. gram.:m.}
\end{itemize}
\begin{itemize}
\item {Proveniência:(De \textunderscore caço\textunderscore )}
\end{itemize}
Pequeno caço, para uso de tintureiros.
\section{Cacimba}
\begin{itemize}
\item {Grp. gram.:f.}
\end{itemize}
\begin{itemize}
\item {Proveniência:(Do quimb. \textunderscore quixima\textunderscore )}
\end{itemize}
Nevoeiro úmido, que se fórma em alguns pontos de África.
Chuva miúda; relento.
Cova, destinada a receber a água dos terrenos pantanosos.
Em Angola, poço, que recebe a água pluvial, filtrada por terrenos circunjacentes, e da qual se servem as povoações.
\section{Cacimbado}
\begin{itemize}
\item {Grp. gram.:adj.}
\end{itemize}
\begin{itemize}
\item {Utilização:Bras}
\end{itemize}
\begin{itemize}
\item {Proveniência:(De \textunderscore cacimbar\textunderscore )}
\end{itemize}
Diz-se do terreno, onde há cacimbas ou póços.
Encharcado nuns pontos e noutros não.
Que tem barro próprio para olaria.
\section{Cacimbão}
\begin{itemize}
\item {Grp. gram.:m.}
\end{itemize}
\begin{itemize}
\item {Utilização:Bras}
\end{itemize}
Cova ou poço grande, onde se juntou agua. Cp. Sílv. Romero, \textunderscore Contos\textunderscore .
\section{Cacimbar}
\begin{itemize}
\item {Grp. gram.:v. i.}
\end{itemize}
\begin{itemize}
\item {Utilização:Bras}
\end{itemize}
Cair cacimba.
Encher-se de água (um terreno), formando pôças aquém e além.
\section{Cacimbeiro}
\begin{itemize}
\item {Grp. gram.:m.}
\end{itemize}
Aquelle que faz as covas, chamadas \textunderscore cacimbas\textunderscore .
\section{Cacimbo}
\begin{itemize}
\item {Grp. gram.:m.}
\end{itemize}
\begin{itemize}
\item {Utilização:T. de Angola}
\end{itemize}
O mesmo que \textunderscore cacimba\textunderscore , relento.
\section{Cácimo}
\begin{itemize}
\item {Grp. gram.:m.}
\end{itemize}
\begin{itemize}
\item {Utilização:Prov.}
\end{itemize}
\begin{itemize}
\item {Utilização:minh.}
\end{itemize}
Espécie de planta, (\textunderscore senecio silvaticus\textunderscore , Lin.), com que o povo envenena o peixe dos rios.
\section{Cacique}
\begin{itemize}
\item {Grp. gram.:m.}
\end{itemize}
\begin{itemize}
\item {Proveniência:(T. cast.)}
\end{itemize}
Chefe ou autoridade superior, entre os indígenas de várias regiões americanas.
Aquelle que tem influência politica numa ou mais localidades, e costuma arrebanhar os eleitores, para votações políticas ou administrativas.
Pássaro brasileiro das regiões do Amazonas.
\section{Caciz}
\begin{itemize}
\item {Grp. gram.:m.}
\end{itemize}
\begin{itemize}
\item {Utilização:Ant.}
\end{itemize}
\begin{itemize}
\item {Proveniência:(Do ár. \textunderscore caaxix\textunderscore )}
\end{itemize}
Vedor, ou homem nobre, em alguns Estados africanos.
Sacerdote moirisco, na África oriental. Cf. Pant. de Aveiro, \textunderscore Itiner.\textunderscore , 65 v.^o, (2.^a ed.)
\section{Caco}
\begin{itemize}
\item {Grp. gram.:m.}
\end{itemize}
\begin{itemize}
\item {Utilização:Fig.}
\end{itemize}
\begin{itemize}
\item {Utilização:Fam.}
\end{itemize}
\begin{itemize}
\item {Utilização:Bras}
\end{itemize}
\begin{itemize}
\item {Proveniência:(Lat. \textunderscore calculus\textunderscore )}
\end{itemize}
Pedaço de loiça, vidro, etc.
Traste velho, de pouco valor.
Cabeça, intelligência.
Pessôa velha e doente.
Humor solidificado do nariz.
Pó, a que se reduz o tabaco, depois de torrado ao fogo e moído em um caco de loiça.
\section{Caço}
\begin{itemize}
\item {Grp. gram.:m.}
\end{itemize}
\begin{itemize}
\item {Utilização:Des.}
\end{itemize}
\begin{itemize}
\item {Utilização:Prov.}
\end{itemize}
\begin{itemize}
\item {Utilização:trasm.}
\end{itemize}
Colhér grande, com que se tira o azeite da talha, a sopa da terrina, etc.
Frigideira de barro com cabo.
Vaso culinário de latão, convexo por baixo, com um cabo mais longo que o da sertan, e no qual se costuma aquecer o leite.
(Cast. \textunderscore cazó\textunderscore )
\section{Caçoada}
\begin{itemize}
\item {Grp. gram.:f.}
\end{itemize}
Acção de \textunderscore caçoar\textunderscore .
Motejo, troça.
\section{Caçoador}
\begin{itemize}
\item {Grp. gram.:adj.}
\end{itemize}
Que caçoa.
\section{Caçoante}
\begin{itemize}
\item {Grp. gram.:adj.}
\end{itemize}
Que caçoa.
\section{Caçoar}
\begin{itemize}
\item {Grp. gram.:v. t.  e  i.}
\end{itemize}
Zombar; troçar; motejar.
(Por \textunderscore cançoar\textunderscore , de \textunderscore canção\textunderscore )
\section{Cacochondrite}
\begin{itemize}
\item {fónica:con}
\end{itemize}
\begin{itemize}
\item {Grp. gram.:f.}
\end{itemize}
\begin{itemize}
\item {Proveniência:(Do gr. \textunderscore kakos\textunderscore  + \textunderscore khondros\textunderscore )}
\end{itemize}
Espécie de serpente venenosa.
\section{Cacochymia}
\begin{itemize}
\item {fónica:qui}
\end{itemize}
\begin{itemize}
\item {Grp. gram.:f.}
\end{itemize}
Estado do que é \textunderscore cacochymo\textunderscore .
\section{Cacochýmico}
\begin{itemize}
\item {fónica:qui}
\end{itemize}
\begin{itemize}
\item {Grp. gram.:adj.}
\end{itemize}
Relativo á \textunderscore cacochymia\textunderscore .
\section{Cacochymo}
\begin{itemize}
\item {fónica:qui}
\end{itemize}
\begin{itemize}
\item {Grp. gram.:adj.}
\end{itemize}
\begin{itemize}
\item {Proveniência:(Do gr. \textunderscore kakos\textunderscore  + \textunderscore khimos\textunderscore )}
\end{itemize}
Que tem compleição débil.
\section{Cacoco}
\begin{itemize}
\item {Grp. gram.:m.}
\end{itemize}
Espécie de ave de rapina, africana.
\section{Cacocondrite}
\begin{itemize}
\item {Grp. gram.:f.}
\end{itemize}
\begin{itemize}
\item {Proveniência:(Do gr. \textunderscore kakos\textunderscore  + \textunderscore khondros\textunderscore )}
\end{itemize}
Espécie de serpente venenosa.
\section{Cacoeiro}
\begin{itemize}
\item {Grp. gram.:m.}
\end{itemize}
Planta do cacau, cacaueiro.
\section{Cacoépia}
\begin{itemize}
\item {fónica:co-e}
\end{itemize}
\begin{itemize}
\item {Grp. gram.:f.}
\end{itemize}
\begin{itemize}
\item {Utilização:Neol.}
\end{itemize}
\begin{itemize}
\item {Proveniência:(Do gr. \textunderscore kakos\textunderscore  + \textunderscore epos\textunderscore )}
\end{itemize}
Pronuncia viciosa.
\section{Cacoête}
\begin{itemize}
\item {Grp. gram.:m.}
\end{itemize}
\begin{itemize}
\item {Proveniência:(Gr. \textunderscore kacoethes\textunderscore )}
\end{itemize}
Mau hábito de torcer a cara ou fazer gestos e ademanes desagradáveis.
Defeito phýsico de gaguejar.
\section{Cacófago}
\begin{itemize}
\item {Grp. gram.:adj.}
\end{itemize}
\begin{itemize}
\item {Proveniência:(Do gr. \textunderscore kakos\textunderscore  + \textunderscore phaigen\textunderscore )}
\end{itemize}
Que come coisas repugnantes.
\section{Cacófaton}
\begin{itemize}
\item {Grp. gram.:m.}
\end{itemize}
O mesmo que \textunderscore cacofonia\textunderscore .
\section{Cacofonia}
\begin{itemize}
\item {Grp. gram.:f.}
\end{itemize}
\begin{itemize}
\item {Proveniência:(Gr. \textunderscore kakophonia\textunderscore )}
\end{itemize}
Som desagradável ou palavra obscena, resultante da união de letras ou sýllabas de palavras differentes.
\section{Cacofoniar}
\begin{itemize}
\item {Grp. gram.:v. i.}
\end{itemize}
Produzir cacofonia. Cf. Rui Barb., \textunderscore Réplica\textunderscore , 10.
\section{Cacofónico}
\begin{itemize}
\item {Grp. gram.:adj.}
\end{itemize}
Em que há cacofonia.
\section{Cacogênese}
\begin{itemize}
\item {Grp. gram.:f.}
\end{itemize}
\begin{itemize}
\item {Proveniência:(Do gr. \textunderscore kakos\textunderscore  + \textunderscore genesis\textunderscore )}
\end{itemize}
Monstruosidade de nascença.
\section{Cacografia}
\begin{itemize}
\item {Grp. gram.:f.}
\end{itemize}
\begin{itemize}
\item {Proveniência:(Gr. \textunderscore kakographia\textunderscore )}
\end{itemize}
Êrro orthográfico.
\section{Cacográfico}
\begin{itemize}
\item {Grp. gram.:adj.}
\end{itemize}
Relativo à \textunderscore cacografia\textunderscore .
\section{Cacographia}
\begin{itemize}
\item {Grp. gram.:f.}
\end{itemize}
\begin{itemize}
\item {Proveniência:(Gr. \textunderscore kakographia\textunderscore )}
\end{itemize}
Êrro orthográphico.
\section{Cacográphico}
\begin{itemize}
\item {Grp. gram.:adj.}
\end{itemize}
Relativo à \textunderscore cacographia\textunderscore .
\section{Caçoila}
\begin{itemize}
\item {Grp. gram.:f.}
\end{itemize}
\begin{itemize}
\item {Utilização:Náut.}
\end{itemize}
Vaso cylindrico de barro, mais largo do que alto, para nelle se cozerem alimentos.
Vaso, em que se queimam perfumes.
Peça de poleame.
(Cast. \textunderscore cazuela\textunderscore )
\section{Caçoilada}
\begin{itemize}
\item {Grp. gram.:f.}
\end{itemize}
Iguaria, que se faz em caçoila ou caçoilo.
\section{Caçoilo}
\begin{itemize}
\item {Grp. gram.:m.}
\end{itemize}
\begin{itemize}
\item {Utilização:Prov.}
\end{itemize}
\begin{itemize}
\item {Utilização:beir.}
\end{itemize}
\begin{itemize}
\item {Utilização:Náut.}
\end{itemize}
Pequena caçoila.
Pequena bóla de pau, furada, que facilita o movimento dos cabos.
\section{Caçoiro}
\begin{itemize}
\item {Grp. gram.:m.}
\end{itemize}
Rodela de madeira ou cortiça, que se introduz na cana da roca, para lhe formar o bôjo.
\section{Caçoísta}
\begin{itemize}
\item {Grp. gram.:m.  e  f.}
\end{itemize}
Pessôa, que gosta de caçoar.
\section{Cacola}
\begin{itemize}
\item {Grp. gram.:m.}
\end{itemize}
Reptil africano, da ordem dos sáurios.
\section{Caçola}
\begin{itemize}
\item {fónica:çô}
\end{itemize}
\begin{itemize}
\item {Grp. gram.:f.}
\end{itemize}
(V.caçoila)
\section{Caçoleta}
\begin{itemize}
\item {fónica:lê}
\end{itemize}
\begin{itemize}
\item {Grp. gram.:f.}
\end{itemize}
\begin{itemize}
\item {Utilização:Bras. do N}
\end{itemize}
\begin{itemize}
\item {Proveniência:(De \textunderscore caçola\textunderscore )}
\end{itemize}
Fuzil de espingarda antiga.
Cápsula de matéria fulminante, nas armas de percussão.
Vaso, em que os ourives recozem o oiro e a prata.
Pequena frigideira.
Espécie de medalha, que as mulheres usam ao pescoço, ou que serve de berloque na corrente dos relógios.
\section{Cacologia}
\begin{itemize}
\item {Grp. gram.:f.}
\end{itemize}
Êrro de locução.
(Cp. \textunderscore cacólogo\textunderscore )
\section{Cacológico}
\begin{itemize}
\item {Grp. gram.:adj.}
\end{itemize}
Em que há cacologia.
Relativo à cacologia.
\section{Cacólogo}
\begin{itemize}
\item {Grp. gram.:m.}
\end{itemize}
\begin{itemize}
\item {Proveniência:(Do gr. \textunderscore kakos\textunderscore  + \textunderscore logos\textunderscore )}
\end{itemize}
Aquelle que commete cacologias.
\section{Cacome}
\begin{itemize}
\item {Grp. gram.:m.}
\end{itemize}
Arbusto de Moçambique.
\section{Caçonaes}
\begin{itemize}
\item {Grp. gram.:m. pl.}
\end{itemize}
\begin{itemize}
\item {Utilização:Prov.}
\end{itemize}
\begin{itemize}
\item {Utilização:alg.}
\end{itemize}
\begin{itemize}
\item {Proveniência:(De \textunderscore cação\textunderscore )}
\end{itemize}
Rede de emmalhar cações.
\section{Caçonais}
\begin{itemize}
\item {Grp. gram.:m. pl.}
\end{itemize}
\begin{itemize}
\item {Utilização:Prov.}
\end{itemize}
\begin{itemize}
\item {Utilização:alg.}
\end{itemize}
\begin{itemize}
\item {Proveniência:(De \textunderscore cação\textunderscore )}
\end{itemize}
Rede de emmalhar cações.
\section{Caçoneira}
\begin{itemize}
\item {Grp. gram.:f.}
\end{itemize}
Planta angolense, (\textunderscore euphorbía rhipzaloides\textunderscore ).
\section{Caçonetes}
\begin{itemize}
\item {fónica:nê}
\end{itemize}
\begin{itemize}
\item {Grp. gram.:m. pl.}
\end{itemize}
\begin{itemize}
\item {Utilização:Náut.}
\end{itemize}
Paus torneados, que se encaixavam nos punhos dos joanetes, para as escotas não puderem correr.
(Por \textunderscore calçonete\textunderscore , de \textunderscore calço\textunderscore ?)
\section{Cacongo}
\begin{itemize}
\item {Grp. gram.:m.}
\end{itemize}
Arbusto africano, monocotyledóneo, de fôlhas oppostas e flôres hemaphroditas.
Espécie de salmão africano.
\section{Cacongos}
\begin{itemize}
\item {Grp. gram.:m. pl.}
\end{itemize}
Congueses da margem direita do Zaire.
\section{Caconso}
\begin{itemize}
\item {Grp. gram.:adj.}
\end{itemize}
\begin{itemize}
\item {Utilização:Pop.}
\end{itemize}
\begin{itemize}
\item {Utilização:T. do Fundão}
\end{itemize}
Abatido por doença, achacado.
Sonso, fingido.
\section{Cacopathia}
\begin{itemize}
\item {Grp. gram.:f.}
\end{itemize}
\begin{itemize}
\item {Proveniência:(Do gr. \textunderscore kakos\textunderscore  + \textunderscore pathos\textunderscore )}
\end{itemize}
Dôr maligna.
Doença de mau carácter.
\section{Cacopatia}
\begin{itemize}
\item {Grp. gram.:f.}
\end{itemize}
\begin{itemize}
\item {Proveniência:(Do gr. \textunderscore kakos\textunderscore  + \textunderscore pathos\textunderscore )}
\end{itemize}
Dôr maligna.
Doença de mau carácter.
\section{Cacóphago}
\begin{itemize}
\item {Grp. gram.:adj.}
\end{itemize}
\begin{itemize}
\item {Proveniência:(Do gr. \textunderscore kakos\textunderscore  + \textunderscore phaigen\textunderscore )}
\end{itemize}
Que come coisas repugnantes.
\section{Cacóphaton}
\begin{itemize}
\item {Grp. gram.:m.}
\end{itemize}
O mesmo que \textunderscore cacophonia\textunderscore .
\section{Cacophonia}
\begin{itemize}
\item {Grp. gram.:f.}
\end{itemize}
\begin{itemize}
\item {Proveniência:(Gr. \textunderscore kakophonia\textunderscore )}
\end{itemize}
Som desagradável ou palavra obscena, resultante da união de letras ou sýllabas de palavras differentes.
\section{Cacophoniar}
\begin{itemize}
\item {Grp. gram.:v. i.}
\end{itemize}
Produzir cacophonia. Cf. Rui Barb., \textunderscore Réplica\textunderscore , 10.
\section{Cacophónico}
\begin{itemize}
\item {Grp. gram.:adj.}
\end{itemize}
Em que há cacophonia.
\section{Cacoquimia}
\begin{itemize}
\item {Grp. gram.:f.}
\end{itemize}
Estado do que é \textunderscore cacoquimo\textunderscore .
\section{Cacoquímico}
\begin{itemize}
\item {Grp. gram.:adj.}
\end{itemize}
Relativo á \textunderscore cacoquimia\textunderscore .
\section{Cacoquímo}
\begin{itemize}
\item {Grp. gram.:adj.}
\end{itemize}
\begin{itemize}
\item {Proveniência:(Do gr. \textunderscore kakos\textunderscore  + \textunderscore khimos\textunderscore )}
\end{itemize}
Que tem compleição débil.
\section{Cacório}
\begin{itemize}
\item {Grp. gram.:adj.}
\end{itemize}
\begin{itemize}
\item {Utilização:Bras}
\end{itemize}
\begin{itemize}
\item {Utilização:chul.}
\end{itemize}
Esperto; sagaz; astuto.
\section{Cacosfixia}
\begin{itemize}
\item {fónica:csi}
\end{itemize}
\begin{itemize}
\item {Grp. gram.:f.}
\end{itemize}
\begin{itemize}
\item {Utilização:Med.}
\end{itemize}
\begin{itemize}
\item {Proveniência:(Do gr. \textunderscore kakos\textunderscore  + \textunderscore sphugmos\textunderscore )}
\end{itemize}
Irregularidade de pulso.
\section{Cacósmia}
\begin{itemize}
\item {Grp. gram.:f.}
\end{itemize}
\begin{itemize}
\item {Proveniência:(Do gr. \textunderscore kakos\textunderscore  + \textunderscore osme\textunderscore )}
\end{itemize}
Arbusto do Peru, mal cheiroso.
\section{Cacoso}
\begin{itemize}
\item {fónica:cô}
\end{itemize}
\begin{itemize}
\item {Grp. gram.:adj.}
\end{itemize}
\begin{itemize}
\item {Utilização:Prov.}
\end{itemize}
\begin{itemize}
\item {Utilização:trasm.}
\end{itemize}
\begin{itemize}
\item {Utilização:Prov.}
\end{itemize}
\begin{itemize}
\item {Utilização:trasm.}
\end{itemize}
\begin{itemize}
\item {Proveniência:(De \textunderscore caco\textunderscore )}
\end{itemize}
Ranhoso.
Sujo.
Encardido, (falando-se de lenços ou de outros panos).
Velho, esboicelado, (falando-se de pratos ou outras vasilhas de barro).
\section{Cacosphyxia}
\begin{itemize}
\item {Grp. gram.:f.}
\end{itemize}
\begin{itemize}
\item {Utilização:Med.}
\end{itemize}
\begin{itemize}
\item {Proveniência:(Do gr. \textunderscore kakos\textunderscore  + \textunderscore sphugmos\textunderscore )}
\end{itemize}
Irregularidade de pulso.
\section{Cacóstomo}
\begin{itemize}
\item {Grp. gram.:adj.}
\end{itemize}
\begin{itemize}
\item {Utilização:Med.}
\end{itemize}
\begin{itemize}
\item {Proveniência:(Do gr. \textunderscore kakos\textunderscore , mau, e \textunderscore stoma\textunderscore , bôca)}
\end{itemize}
Que tem mau cheiro na bôca; que tem hálito fétido.
\section{Cacotanasia}
\begin{itemize}
\item {Grp. gram.:f.}
\end{itemize}
\begin{itemize}
\item {Proveniência:(Do gr. \textunderscore kakos\textunderscore  + \textunderscore thanatos\textunderscore )}
\end{itemize}
Morte afflictiva.
\section{Caçote}
\begin{itemize}
\item {Grp. gram.:m.}
\end{itemize}
\begin{itemize}
\item {Utilização:Bras}
\end{itemize}
\begin{itemize}
\item {Utilização:Bras}
\end{itemize}
Antigo saial de soldados.
Pequena ran.
Pessôa imberbe.
(Por \textunderscore calçote\textunderscore , de \textunderscore calça\textunderscore ?)
\section{Cacotechnia}
\begin{itemize}
\item {Grp. gram.:f.}
\end{itemize}
\begin{itemize}
\item {Proveniência:(Do gr. \textunderscore kakos\textunderscore  + \textunderscore teckhne\textunderscore )}
\end{itemize}
Falta de arte.
\section{Cacotecnia}
\begin{itemize}
\item {Grp. gram.:f.}
\end{itemize}
\begin{itemize}
\item {Proveniência:(Do gr. \textunderscore kakos\textunderscore  + \textunderscore teckhne\textunderscore )}
\end{itemize}
Falta de arte.
\section{Cacothanasia}
\begin{itemize}
\item {Grp. gram.:f.}
\end{itemize}
\begin{itemize}
\item {Proveniência:(Do gr. \textunderscore kakos\textunderscore  + \textunderscore thanatos\textunderscore )}
\end{itemize}
Morte afflictiva.
\section{Cacothymia}
\begin{itemize}
\item {Grp. gram.:f.}
\end{itemize}
Perturbação das faculdades moraes.
\section{Cacotimia}
\begin{itemize}
\item {Grp. gram.:f.}
\end{itemize}
Perturbação das faculdades moraes.
\section{Cacotrofia}
\begin{itemize}
\item {Grp. gram.:f.}
\end{itemize}
\begin{itemize}
\item {Proveniência:(Do gr. \textunderscore kakos\textunderscore  + \textunderscore trophe\textunderscore )}
\end{itemize}
Má alimentação; defeito nas funcções da nutrição.
\section{Cacotrophia}
\begin{itemize}
\item {Grp. gram.:f.}
\end{itemize}
\begin{itemize}
\item {Proveniência:(Do gr. \textunderscore kakos\textunderscore  + \textunderscore trophe\textunderscore )}
\end{itemize}
Má alimentação; defeito nas funcções da nutrição.
\section{Caçougue}
\begin{itemize}
\item {Grp. gram.:m.}
\end{itemize}
Ave africana, (\textunderscore micronisus monogrammicus\textunderscore ).
\section{Caçoula}
\begin{itemize}
\item {Grp. gram.:f.}
\end{itemize}
\begin{itemize}
\item {Utilização:Náut.}
\end{itemize}
Vaso cylindrico de barro, mais largo do que alto, para nelle se cozerem alimentos.
Vaso, em que se queimam perfumes.
Peça de poleame.
(Cast. \textunderscore cazuela\textunderscore )
\section{Caçoulada}
\begin{itemize}
\item {Grp. gram.:f.}
\end{itemize}
Iguaria, que se faz em caçoula ou caçoulo.
\section{Caçoulo}
\begin{itemize}
\item {Grp. gram.:m.}
\end{itemize}
\begin{itemize}
\item {Utilização:Prov.}
\end{itemize}
\begin{itemize}
\item {Utilização:beir.}
\end{itemize}
\begin{itemize}
\item {Utilização:Náut.}
\end{itemize}
Pequena caçoula.
Pequena bóla de pau, furada, que facilita o movimento dos cabos.
\section{Caçouro}
\begin{itemize}
\item {Grp. gram.:m.}
\end{itemize}
Rodela de madeira ou cortiça, que se introduz na cana da roca, para lhe formar o bôjo.
\section{Cacozelia}
\begin{itemize}
\item {Grp. gram.:f.}
\end{itemize}
\begin{itemize}
\item {Proveniência:(Do gr. \textunderscore kakos\textunderscore , e \textunderscore zêlo\textunderscore )}
\end{itemize}
Imitação viciosa.
Zêlo exaggerado.
\section{Cactáceas}
\begin{itemize}
\item {Grp. gram.:f. pl.}
\end{itemize}
O mesmo ou melhor que \textunderscore cácteas\textunderscore .
\section{Cácteas}
\begin{itemize}
\item {Grp. gram.:f. pl.}
\end{itemize}
Família de plantas, que têm por typo o cacto.
\section{Cacto}
\begin{itemize}
\item {Grp. gram.:m.}
\end{itemize}
\begin{itemize}
\item {Proveniência:(Gr. \textunderscore kaktos\textunderscore )}
\end{itemize}
Nome de variadissimas plantas, de caule esphérico ou anguloso ou foliáceo ou de peças articuladas, e que dão flôres grandes e de côres vivas.
\section{Caçuá}
\begin{itemize}
\item {Grp. gram.:m.}
\end{itemize}
\begin{itemize}
\item {Utilização:Bras}
\end{itemize}
Seirão de cipó para cangalhas.
Espécie de rede, de malhas largas.
\section{Cacuala}
\begin{itemize}
\item {Grp. gram.:f.}
\end{itemize}
Planta de Angola, de caule verde-amarelado, espinhoso e herbáceo.
\section{Cacuata}
\begin{itemize}
\item {Grp. gram.:m.}
\end{itemize}
Dignitário dos sobas, na África.
\section{Cacubi}
\begin{itemize}
\item {Grp. gram.:m.}
\end{itemize}
Espécie de cobra.
\section{Cacuco}
\begin{itemize}
\item {Grp. gram.:m.}
\end{itemize}
O mesmo que \textunderscore cacumbu\textunderscore ^1.
\section{Cacueme}
\begin{itemize}
\item {Grp. gram.:m.}
\end{itemize}
Árvore angolense, de fôlhas simples e flôres representadas por um cálice de três sépalas e uma corolla gamopétala.
\section{Caçula}
\begin{itemize}
\item {Grp. gram.:m.}
\end{itemize}
\begin{itemize}
\item {Utilização:Bras. do S}
\end{itemize}
\begin{itemize}
\item {Proveniência:(Do quimb. cazule)}
\end{itemize}
Filho mais novo.
\section{Caçula}
\begin{itemize}
\item {Grp. gram.:f.}
\end{itemize}
\begin{itemize}
\item {Utilização:Bras}
\end{itemize}
\begin{itemize}
\item {Proveniência:(Do quimb. \textunderscore cuçula\textunderscore )}
\end{itemize}
Acto de \textunderscore secar\textunderscore  ou moer milho no pilão, a braços.
\section{Caculage}
\begin{itemize}
\item {Grp. gram.:m.}
\end{itemize}
\begin{itemize}
\item {Utilização:Bras}
\end{itemize}
Planta medicinal.
\section{Caçulê}
\begin{itemize}
\item {Grp. gram.:m.}
\end{itemize}
\begin{itemize}
\item {Utilização:Bras}
\end{itemize}
O mesmo que \textunderscore caçula\textunderscore ^1, mas menos usado.
\section{Caculo}
\begin{itemize}
\item {Grp. gram.:m.}
\end{itemize}
\begin{itemize}
\item {Utilização:Bras}
\end{itemize}
Aquelle dos gêmeos que nasceu primeiro.
\section{Caculo}
\begin{itemize}
\item {Grp. gram.:m.}
\end{itemize}
Ave africana, (\textunderscore scops capensis\textunderscore , Smith).
\section{Caculo}
\begin{itemize}
\item {Grp. gram.:m.}
\end{itemize}
\begin{itemize}
\item {Utilização:Bras}
\end{itemize}
O mesmo que \textunderscore cogulo\textunderscore .
\section{Cacumá}
\begin{itemize}
\item {Grp. gram.:m.}
\end{itemize}
Arbusto medicinal da ilha de San-Thomé.
(No museu da \textunderscore Socied. de Geog. de Lisbôa\textunderscore , lê-se \textunderscore cacuma\textunderscore )
\section{Cacumbi}
\begin{itemize}
\item {Grp. gram.:m.}
\end{itemize}
\begin{itemize}
\item {Utilização:Bras}
\end{itemize}
O mesmo que \textunderscore cacumbu\textunderscore ^2.
\section{Cacumbu}
\begin{itemize}
\item {Grp. gram.:m.}
\end{itemize}
\begin{itemize}
\item {Utilização:Bras}
\end{itemize}
Enxada ou machado já gasto.
Metade do dia santo, que vai da quinta-feira á sexta-feira da Semana Santa.
O mesmo que \textunderscore cachirim\textunderscore .
\section{Cacumbu}
\begin{itemize}
\item {Grp. gram.:m.}
\end{itemize}
\begin{itemize}
\item {Utilização:Bras}
\end{itemize}
Dança de negros, o mesmo que \textunderscore cachambu\textunderscore .
\section{Cacume}
\begin{itemize}
\item {Grp. gram.:m.}
\end{itemize}
\begin{itemize}
\item {Utilização:Des.}
\end{itemize}
\begin{itemize}
\item {Proveniência:(Lat. \textunderscore cacumen\textunderscore )}
\end{itemize}
A parte mais elevada de tudo o que termina em ponta.
\section{Cacúmen}
\begin{itemize}
\item {Grp. gram.:m.}
\end{itemize}
\begin{itemize}
\item {Utilização:Des.}
\end{itemize}
\begin{itemize}
\item {Proveniência:(Lat. \textunderscore cacumen\textunderscore )}
\end{itemize}
A parte mais elevada de tudo o que termina em ponta.
\section{Cacuminal}
\begin{itemize}
\item {Grp. gram.:adj.}
\end{itemize}
\begin{itemize}
\item {Proveniência:(Do lat. \textunderscore cacumen\textunderscore )}
\end{itemize}
Diz-se de uma classe de consoantes do alphabeto dravídico e do sanscrítico; e diz-se, em phonética geral, das letras apicaes, (\textunderscore t\textunderscore , \textunderscore d\textunderscore , \textunderscore s\textunderscore , \textunderscore z\textunderscore , \textunderscore n\textunderscore , \textunderscore l\textunderscore , \textunderscore r\textunderscore ).
\section{Cacunda}
\begin{itemize}
\item {Grp. gram.:f.}
\end{itemize}
\begin{itemize}
\item {Utilização:Bras}
\end{itemize}
Costas.
(Alt. de \textunderscore carcunda\textunderscore )
\section{Cacunda}
\begin{itemize}
\item {Grp. gram.:f.}
\end{itemize}
\begin{itemize}
\item {Utilização:Bras}
\end{itemize}
Espécie de vinhático.
\section{Cacundê}
\begin{itemize}
\item {Grp. gram.:m.}
\end{itemize}
\begin{itemize}
\item {Utilização:Bras}
\end{itemize}
Lavor, com que se guarnecem saias e camisas de mulher, e que consiste em coser tiras de pano sôbre um desenho feito naquellas peças de roupa, fazendo-se depois desapparecer o desenho e cortando-se o excedente.
\section{Cacundeiro}
\begin{itemize}
\item {Grp. gram.:m.  e  adj.}
\end{itemize}
\begin{itemize}
\item {Utilização:Bras}
\end{itemize}
\begin{itemize}
\item {Utilização:Fig.}
\end{itemize}
\begin{itemize}
\item {Proveniência:(De \textunderscore cacunda\textunderscore ^1)}
\end{itemize}
Carregador, moço de fretes.
Homem da ínfima classe.
\section{Cacundo}
\begin{itemize}
\item {Grp. gram.:m.}
\end{itemize}
\begin{itemize}
\item {Grp. gram.:M.  e  adj.}
\end{itemize}
\begin{itemize}
\item {Utilização:Bras}
\end{itemize}
O mesmo que \textunderscore cacunda\textunderscore ^1.
O mesmo que \textunderscore carcunda\textunderscore .
\section{Cacuri}
\begin{itemize}
\item {Grp. gram.:m.}
\end{itemize}
\begin{itemize}
\item {Utilização:Bras}
\end{itemize}
O mesmo que \textunderscore jiqui\textunderscore .
\section{Cacurichiche}
\begin{itemize}
\item {Grp. gram.:m.}
\end{itemize}
\begin{itemize}
\item {Proveniência:(T. lund.)}
\end{itemize}
Ave angolense, pernalta, da fam. das macrodáctylas, preta, ligeiramente bronzeada.
\section{Caçurrento}
\begin{itemize}
\item {Grp. gram.:adj.}
\end{itemize}
Que tem muito çurro; sujo, emporcalhado.
\section{Caçurro}
\begin{itemize}
\item {Grp. gram.:m.}
\end{itemize}
\begin{itemize}
\item {Utilização:Prov.}
\end{itemize}
Çurro, porcaria, falta de limpeza.
\section{Cada}
\begin{itemize}
\item {fónica:câ}
\end{itemize}
\begin{itemize}
\item {Grp. gram.:adj. invar.}
\end{itemize}
\begin{itemize}
\item {Proveniência:(Do gr. \textunderscore kata\textunderscore )}
\end{itemize}
(indicativo de que uma collectividade de coisas ou pessôas deve sêr considerada separadamente em todas as coisas ou indivíduos que a compõem, ou em todos os seus sentidos: \textunderscore a cada hora\textunderscore )
Tal; tão fóra do commum: \textunderscore êste homem tem cada mania\textunderscore !
\section{Cadabulho}
\begin{itemize}
\item {Grp. gram.:m.}
\end{itemize}
\begin{itemize}
\item {Utilização:Prov.}
\end{itemize}
\begin{itemize}
\item {Utilização:alent.}
\end{itemize}
\begin{itemize}
\item {Utilização:beir.}
\end{itemize}
Pequeno espaço de terra, que ficou sem lavra, por não lhe poder chegar o arado, como succede junto de árvores, paredes, etc. e que tem de se cavar para a sementeira.
(Por \textunderscore cavadulho\textunderscore , de \textunderscore cavar\textunderscore )
\section{Cadaço}
\begin{itemize}
\item {Grp. gram.:m.}
\end{itemize}
(V.cadarço)
\section{Cadafalso}
\begin{itemize}
\item {Grp. gram.:m.}
\end{itemize}
\begin{itemize}
\item {Utilização:Açor}
\end{itemize}
Tablado.
Andaime.
Estrado, erguido em lugar público, para nelle se exporem ou executarem os condemnados.
Casa, destinada ás festas do Espírito-Santo.
(Cp. \textunderscore catafalco\textunderscore )
\section{Cadamo}
\begin{itemize}
\item {Grp. gram.:m.}
\end{itemize}
\begin{itemize}
\item {Utilização:T. de Timor}
\end{itemize}
\begin{itemize}
\item {Proveniência:(Do ár. \textunderscore kadam\textunderscore )}
\end{itemize}
Piloto.
\section{Cadaquê}
\begin{itemize}
\item {Grp. gram.:Loc. conj.}
\end{itemize}
Cada vez que, todas as vezes que.
\section{Cadarço}
\begin{itemize}
\item {Grp. gram.:m.}
\end{itemize}
Barbilho, cordão de anafaia.
Tecido de anafaia.
Nastro; cadilho; cadaxo.
(Cast. \textunderscore cadarzo\textunderscore )
\section{Cadarrão}
\begin{itemize}
\item {Grp. gram.:m.}
\end{itemize}
\begin{itemize}
\item {Utilização:Ant.}
\end{itemize}
Catarro grande?:«\textunderscore depois dos cadarrões, que foram doenças geraes.\textunderscore »Sousa, \textunderscore Vida do Arceb.\textunderscore , V, 19.
(Por \textunderscore catarrão\textunderscore , de \textunderscore catarro\textunderscore ?)
\section{Cadaste}
\begin{itemize}
\item {Grp. gram.:m.}
\end{itemize}
\begin{itemize}
\item {Utilização:Náut.}
\end{itemize}
Peça da popa, em que assentam as dobradiças do leme.
(Cast. \textunderscore codaste\textunderscore , de \textunderscore coda\textunderscore )
\section{Cadastral}
\begin{itemize}
\item {Grp. gram.:adj.}
\end{itemize}
Relativo ao \textunderscore cadastro\textunderscore .
\section{Cadastrar}
\begin{itemize}
\item {Grp. gram.:v. t.}
\end{itemize}
Fazer o cadastro de.
\section{Cadastro}
\begin{itemize}
\item {Grp. gram.:m.}
\end{itemize}
\begin{itemize}
\item {Proveniência:(Do b. lat. \textunderscore cadastrum\textunderscore )}
\end{itemize}
Registo público do valor, natureza e confrontações de prédios rústicos.
Medida e avaliação official, contida naquelle registo.
Recenseamento dos cidadãos, segundo os seus haveres, profissão, etc.
Registo policial de criminosos.
\section{Cadava}
\begin{itemize}
\item {Grp. gram.:f.}
\end{itemize}
Conjunto dos troncos do mato, que ficam de pé depois das queimadas e ainda servem para lenha.
(Cast. \textunderscore cadava\textunderscore )
\section{Cadaval}
\begin{itemize}
\item {Grp. gram.:m.}
\end{itemize}
\begin{itemize}
\item {Utilização:Prov.}
\end{itemize}
\begin{itemize}
\item {Utilização:ant.}
\end{itemize}
Lugar, onde ficam cadavas, depois das queimadas.
\section{Cadáver}
\begin{itemize}
\item {Grp. gram.:m.}
\end{itemize}
\begin{itemize}
\item {Utilização:Fig.}
\end{itemize}
\begin{itemize}
\item {Utilização:Bras}
\end{itemize}
\begin{itemize}
\item {Utilização:chul.}
\end{itemize}
\begin{itemize}
\item {Proveniência:(Lat. \textunderscore cadaver\textunderscore )}
\end{itemize}
Corpo sem vida, especialmente do ente racional.
Aquelle que está tão fraco ou decadente, que parece extinguir-se-lhe a vida.
Aquillo que se tornou obsoleto ou se extinguiu.
O mesmo que \textunderscore crèdor\textunderscore .
\textunderscore Enterrar o cadáver\textunderscore , pagar a dívida.
\section{Cadavérico}
\begin{itemize}
\item {Grp. gram.:adj.}
\end{itemize}
Relativo a cadáver.
Que tem apparência de cadáver.
\section{Cadaverina}
\begin{itemize}
\item {Grp. gram.:f.}
\end{itemize}
\begin{itemize}
\item {Utilização:bras}
\end{itemize}
\begin{itemize}
\item {Utilização:Neol.}
\end{itemize}
Ptomaína, que se encontra nos cadáveres.
\section{Cadaveroso}
\begin{itemize}
\item {Grp. gram.:adj.}
\end{itemize}
\begin{itemize}
\item {Utilização:Des.}
\end{itemize}
\begin{itemize}
\item {Proveniência:(Lat. \textunderscore cadaverosus\textunderscore )}
\end{itemize}
Próprio de cadáver; cadavérico.
\section{Cadaxo}
\begin{itemize}
\item {Grp. gram.:m.}
\end{itemize}
\begin{itemize}
\item {Utilização:Prov.}
\end{itemize}
O mesmo que \textunderscore cadexo\textunderscore .
\section{Cade}
\begin{itemize}
\item {Grp. gram.:m.}
\end{itemize}
\begin{itemize}
\item {Utilização:Bras}
\end{itemize}
Espécie de zimbro, (\textunderscore juniperus oxycedros\textunderscore , Lin.).
\section{Cadéa}
\begin{itemize}
\item {Grp. gram.:f.}
\end{itemize}
(V.cadeia)
\section{Cadeado}
\begin{itemize}
\item {Grp. gram.:m.}
\end{itemize}
\begin{itemize}
\item {Proveniência:(Lat. \textunderscore catenatus\textunderscore )}
\end{itemize}
Fechadura móvel.
Corrente, formada de fuzis.
\section{Cadeeiro}
\begin{itemize}
\item {Grp. gram.:m.}
\end{itemize}
\begin{itemize}
\item {Utilização:Ant.}
\end{itemize}
\begin{itemize}
\item {Proveniência:(De \textunderscore cadéa\textunderscore )}
\end{itemize}
Carcereiro.
\section{Cadeia}
\begin{itemize}
\item {Grp. gram.:f.}
\end{itemize}
\begin{itemize}
\item {Utilização:Constr.}
\end{itemize}
\begin{itemize}
\item {Proveniência:(Lat. \textunderscore catena\textunderscore )}
\end{itemize}
Corrente, formada de anéis metállicos.
Qualquer ligame.
Algemas de condemnado.
Lugar de prisão, cárcere.
Cativeiro.
Sujeição por affecto, dever ou interesse.
Série de objectos semelhantes.
Continuidade.
Série de pessôas, em fila, de maneira que possam transmittir um objecto, de mão em mão.
Movimento de dança, em que os dançantes, andando uns á volta da sala e em sentido invérso de outros, formam cruzamento ou anéis com a linha que seguem.
Prancha de madeira, atravessada em cruz, na mesa do carro.
Peça de madeira, em que, deante dos vãos das portas ou janelas, se assenta uma das extremidades dos barrotes do sobrado.
Pequena travéssa, collocada interiormente no tampo harmónico dos violinos e outros instrumentos, para reforçar o centro do mesmo tampo e augmentar a vibração.
\textunderscore Ponto de cadeia\textunderscore , modo de coser ou bordar, em que os pontos tomam a fórma dos anéis de cadeia.
\section{Cadeira}
\begin{itemize}
\item {Grp. gram.:f.}
\end{itemize}
\begin{itemize}
\item {Utilização:Ext.}
\end{itemize}
\begin{itemize}
\item {Grp. gram.:Pl.}
\end{itemize}
\begin{itemize}
\item {Proveniência:(Do lat. \textunderscore cathedra\textunderscore )}
\end{itemize}
Assento com costas, para uma pessôa.
\textunderscore Cadeira de baloiço\textunderscore , cadeira ou poltrona, que se faz baloiçar, com um simples movimento do corpo.
Funcções de professor.
Dignidade ecclesiástica.
Aula, disciplina que se ensina.
Séde.
Quadris, nádegas.
\section{Cadeira}
\begin{itemize}
\item {Grp. gram.:f.}
\end{itemize}
Árvore da Guiné portuguesa.
\section{Cadeirado}
\begin{itemize}
\item {Grp. gram.:m.}
\end{itemize}
Fila de cadeiras, ligadas á parede de um côro, de uma capella, de uma aula, etc. Cf. Gabr. Pereira, \textunderscore Estudos Ebor.\textunderscore , I, 4.
\section{Cadeiral}
\begin{itemize}
\item {Grp. gram.:m.}
\end{itemize}
O mesmo que \textunderscore cadeirado\textunderscore .
\section{Cadeirão}
\begin{itemize}
\item {Grp. gram.:m.}
\end{itemize}
Cadeira grande.
\section{Cadeireiro}
\begin{itemize}
\item {Grp. gram.:m.}
\end{itemize}
Marceneiro, que se occupa especialmente de cadeiras.
\section{Cadeirinha}
\begin{itemize}
\item {Grp. gram.:f.}
\end{itemize}
\begin{itemize}
\item {Proveniência:(De \textunderscore cadeira\textunderscore )}
\end{itemize}
Espécie de liteira, conduzida por homens.
Cruzeta, formada pelas mãos de duas pessôas, para que outra se sente nella.
\section{Cadeixo}
\begin{itemize}
\item {Grp. gram.:m.}
\end{itemize}
\begin{itemize}
\item {Utilização:Prov.}
\end{itemize}
Livro velho, alfarrábio.
(Alter. de \textunderscore códice\textunderscore ?)
\section{Cadeixo}
\begin{itemize}
\item {Grp. gram.:m.}
\end{itemize}
(V.cadexo)
\section{Cadela}
\begin{itemize}
\item {Grp. gram.:f.}
\end{itemize}
\begin{itemize}
\item {Proveniência:(Lat. \textunderscore catella\textunderscore )}
\end{itemize}
Fêmea do cão.
\section{Cadeleira}
\begin{itemize}
\item {Grp. gram.:f.}
\end{itemize}
\begin{itemize}
\item {Proveniência:(De \textunderscore cadella\textunderscore )}
\end{itemize}
Casta de uva trasmontana.
\section{Cadeleiro}
\begin{itemize}
\item {Grp. gram.:adj.}
\end{itemize}
\begin{itemize}
\item {Utilização:Prov.}
\end{itemize}
\begin{itemize}
\item {Utilização:beir.}
\end{itemize}
Femeeiro.
\section{Cadelinha}
\begin{itemize}
\item {Grp. gram.:f.}
\end{itemize}
\begin{itemize}
\item {Utilização:Prov.}
\end{itemize}
\begin{itemize}
\item {Utilização:alent.}
\end{itemize}
Mollusco bivalve, semelhante á amêijoa.
O mesmo que \textunderscore bicha-cadela\textunderscore .
\section{Cadelinhas}
\begin{itemize}
\item {Grp. gram.:f. pl.}
\end{itemize}
\begin{itemize}
\item {Utilização:Pop.}
\end{itemize}
Fragmentos de conchas, misturados na areia das praias.
(Cp. \textunderscore cadellinha\textunderscore )
\section{Cadella}
\begin{itemize}
\item {Grp. gram.:f.}
\end{itemize}
\begin{itemize}
\item {Proveniência:(Lat. \textunderscore catella\textunderscore )}
\end{itemize}
Fêmea do cão.
\section{Cadelleira}
\begin{itemize}
\item {Grp. gram.:f.}
\end{itemize}
\begin{itemize}
\item {Proveniência:(De \textunderscore cadella\textunderscore )}
\end{itemize}
Casta de uva trasmontana.
\section{Cadelleiro}
\begin{itemize}
\item {Grp. gram.:adj.}
\end{itemize}
\begin{itemize}
\item {Utilização:Prov.}
\end{itemize}
\begin{itemize}
\item {Utilização:beir.}
\end{itemize}
Femeeiro.
\section{Cadellinha}
\begin{itemize}
\item {Grp. gram.:f.}
\end{itemize}
\begin{itemize}
\item {Utilização:Prov.}
\end{itemize}
\begin{itemize}
\item {Utilização:alent.}
\end{itemize}
Mollusco bivalve, semelhante á amêijoa.
O mesmo que \textunderscore bicha-cadela\textunderscore .
\section{Cadellinhas}
\begin{itemize}
\item {Grp. gram.:f. pl.}
\end{itemize}
\begin{itemize}
\item {Utilização:Pop.}
\end{itemize}
Fragmentos de conchas, misturados na areia das praias.
(Cp. \textunderscore cadellinha\textunderscore )
\section{Cadello}
\begin{itemize}
\item {Grp. gram.:m.}
\end{itemize}
\begin{itemize}
\item {Utilização:Des.}
\end{itemize}
\begin{itemize}
\item {Utilização:Prov.}
\end{itemize}
\begin{itemize}
\item {Utilização:minh.}
\end{itemize}
\begin{itemize}
\item {Proveniência:(Lat. \textunderscore catellus\textunderscore )}
\end{itemize}
Cão pequeno.
Cruzeta de pau, presa ao adelhão e sacudida pela mó em movimento.
\section{Cadellona}
\begin{itemize}
\item {Grp. gram.:f.}
\end{itemize}
O mesmo que \textunderscore rameira\textunderscore . Cf. Filinto, XIII, 98.
\section{Cadelo}
\begin{itemize}
\item {fónica:dê}
\end{itemize}
\begin{itemize}
\item {Grp. gram.:m.}
\end{itemize}
\begin{itemize}
\item {Utilização:Des.}
\end{itemize}
\begin{itemize}
\item {Utilização:Prov.}
\end{itemize}
\begin{itemize}
\item {Utilização:minh.}
\end{itemize}
\begin{itemize}
\item {Proveniência:(Lat. \textunderscore catellus\textunderscore )}
\end{itemize}
Cão pequeno.
Cruzeta de pau, presa ao adelhão e sacudida pela mó em movimento.
\section{Cadelona}
\begin{itemize}
\item {Grp. gram.:f.}
\end{itemize}
O mesmo que \textunderscore rameira\textunderscore . Cf. Filinto, XIII, 98.
\section{Cadena}
\begin{itemize}
\item {Grp. gram.:f.}
\end{itemize}
\begin{itemize}
\item {Utilização:Bras. do S}
\end{itemize}
Meio engenhoso de tirar dos chifres do toiro, sem perigo, o laço em que se acha preso.
(Cast. \textunderscore cadena\textunderscore )
\section{Cadência}
\begin{itemize}
\item {Grp. gram.:f.}
\end{itemize}
\begin{itemize}
\item {Proveniência:(De \textunderscore cadente\textunderscore )}
\end{itemize}
Harmonia na disposição das palavras.
Suavidade de estilo.
Tendência, vocação.
Regularidade de movimentos.
Pausa de uma phrase musical.
Trilo.
\section{Cadenciar}
\begin{itemize}
\item {Grp. gram.:v. t.}
\end{itemize}
Dar cadência a.
\section{Cadencioso}
\begin{itemize}
\item {Grp. gram.:adj.}
\end{itemize}
Que tem cadência.
\section{Cadeneta}
\begin{itemize}
\item {fónica:nê}
\end{itemize}
\begin{itemize}
\item {Grp. gram.:f.}
\end{itemize}
\begin{itemize}
\item {Utilização:Ant.}
\end{itemize}
\begin{itemize}
\item {Proveniência:(Do cast. \textunderscore cadena\textunderscore )}
\end{itemize}
Bordado a ponto de cadeia.
\section{Cadenetilha}
\begin{itemize}
\item {Grp. gram.:f.}
\end{itemize}
\begin{itemize}
\item {Utilização:Ant.}
\end{itemize}
\begin{itemize}
\item {Proveniência:(De \textunderscore cadeneta\textunderscore )}
\end{itemize}
Trancelim, canotilho.
\section{Cadenilha}
\begin{itemize}
\item {Grp. gram.:f.}
\end{itemize}
\begin{itemize}
\item {Utilização:Ant.}
\end{itemize}
\begin{itemize}
\item {Proveniência:(Do cast. \textunderscore cadenilla\textunderscore )}
\end{itemize}
Renda estreita a ponto de cadeia.
\section{Cadente}
\begin{itemize}
\item {Grp. gram.:adj.}
\end{itemize}
\begin{itemize}
\item {Proveniência:(Lat. \textunderscore cadens\textunderscore )}
\end{itemize}
Que tem cadência.
Que vái caindo: \textunderscore estrêllas cadentes\textunderscore .
\section{Caderna}
\begin{itemize}
\item {Grp. gram.:f.}
\end{itemize}
\begin{itemize}
\item {Utilização:Heráld.}
\end{itemize}
\begin{itemize}
\item {Proveniência:(Do lat. \textunderscore quaterni\textunderscore )}
\end{itemize}
Reunião de quatro peças semelhantes, em um escudo.
\section{Cadernal}
\begin{itemize}
\item {Grp. gram.:m.}
\end{itemize}
\begin{itemize}
\item {Utilização:Náut.}
\end{itemize}
\begin{itemize}
\item {Proveniência:(Do lat. \textunderscore quaterni\textunderscore )}
\end{itemize}
Quadrado de madeira, em que gira independente uma roldana múltipla, em navio.
Apparelho, para erguer pontes levadiças.
\section{Caderneta}
\begin{itemize}
\item {fónica:nê}
\end{itemize}
\begin{itemize}
\item {Grp. gram.:f.}
\end{itemize}
\begin{itemize}
\item {Proveniência:(De \textunderscore caderno\textunderscore )}
\end{itemize}
Livrinho de lembranças.
Livrete, em que se regista o serviço e o procedimento dos militares.
Registo de depósitos de dinheiro, e das quantias que o depositante vai levantando, por conta do dinheiro que depositou.
Fascículo ou parte de uma obra, que se distribue a assinantes.
\section{Caderno}
\begin{itemize}
\item {Grp. gram.:m.}
\end{itemize}
\begin{itemize}
\item {Proveniência:(Do lat. \textunderscore quaterni\textunderscore )}
\end{itemize}
Porção de fôlhas de papel, sobrepostas, formando pequeno livro.
Caderneta, livro de apontamentos.
Conjunto de cinco fôlhas de papel, em branco, dobradas por fórma, que cada uma constitue quatro laudas: \textunderscore vai-me comprar dois cadernos de papel almaço\textunderscore .
\textunderscore Caderno de encargos\textunderscore , o que contém as condições de um contrato.
\section{Cadete}
\begin{itemize}
\item {fónica:dé}
\end{itemize}
\begin{itemize}
\item {Grp. gram.:m.}
\end{itemize}
\begin{itemize}
\item {Utilização:Prov.}
\end{itemize}
\begin{itemize}
\item {Utilização:trasm.}
\end{itemize}
Antiga designação dos filhos segundos de pessôas nobres.
Soldado nobre, que na milícia gozava certos privilégios.
Soldado que, dispensado do serviço militar, cursa escolas superiores.
Homem aperaltado.
\section{Cadexo}
\begin{itemize}
\item {Grp. gram.:m.}
\end{itemize}
Trôço de linha ou de retrós.
Madeixa de cabello, separada da respectiva cabelladura.
(Cast. \textunderscore cadejo\textunderscore )
\section{Cádi}
\begin{itemize}
\item {Grp. gram.:m.}
\end{itemize}
Magistrado judicial, entre os Muçulmanos.
(Cp. \textunderscore alcaide\textunderscore )
\section{Çadi}
\begin{itemize}
\item {Grp. gram.:m.}
\end{itemize}
Antiga moéda de Ormuz.
\section{Cadia-abuça}
\begin{itemize}
\item {Grp. gram.:f.}
\end{itemize}
Árvore de Moçambique.
\section{Cadialunginga}
\begin{itemize}
\item {Grp. gram.:f.}
\end{itemize}
Árvore angolense, de fôlhas simples, alternas, e flôres auxiliares, papilionáceas.
\section{Cadilha}
\begin{itemize}
\item {Grp. gram.:f.}
\end{itemize}
Fios do urdume sem trama; cadilhos.
\section{Cadilho}
\begin{itemize}
\item {Grp. gram.:m.}
\end{itemize}
\begin{itemize}
\item {Utilização:Prov.}
\end{itemize}
\begin{itemize}
\item {Utilização:minh.}
\end{itemize}
Fio, para prender ou amarrar qualquer coisa.
(Cp. \textunderscore cadilhos\textunderscore )
\section{Cadilhos}
\begin{itemize}
\item {Grp. gram.:m. pl.}
\end{itemize}
\begin{itemize}
\item {Utilização:Prolóq.}
\end{itemize}
Primeiros e últimos fios do urdume, que não levam trama, e fórmam uma espécie de franja.
Franja de toalhas, tapetes, etc.
Guarnição.
\textunderscore Quem tem filhos tem cadilhos\textunderscore , os pais têm sempre cuidados.
(Cast. \textunderscore cadillos\textunderscore )
\section{Cadima}
\begin{itemize}
\item {Grp. gram.:f.}
\end{itemize}
\begin{itemize}
\item {Utilização:Ant.}
\end{itemize}
Estrada principal.
\section{Cadime}
\begin{itemize}
\item {Grp. gram.:m.}
\end{itemize}
Tábuas recurvas do costado do navio, que dão volta á prôa.
\section{Cadimo}
\begin{itemize}
\item {Grp. gram.:adj.}
\end{itemize}
\begin{itemize}
\item {Proveniência:(Do ár. cadim)}
\end{itemize}
Destro.
Ardiloso.
Usual; frequentado.
\section{Cadina}
\begin{itemize}
\item {Grp. gram.:f.}
\end{itemize}
Designação de cada uma das sete esposas legitimas, que o sultão da Turquia póde têr.
Mulher do cádi.
(Do turco)
\section{Cadinhar}
\begin{itemize}
\item {Grp. gram.:v. t.}
\end{itemize}
Fundir em cadinho.
\section{Cadinho}
\begin{itemize}
\item {Grp. gram.:m.}
\end{itemize}
\begin{itemize}
\item {Proveniência:(Do lat. \textunderscore catinum\textunderscore )}
\end{itemize}
Vaso de barro, em que se fundem metaes e outros mineraes.
\section{Cadino}
\begin{itemize}
\item {Grp. gram.:adj.}
\end{itemize}
(Corr. trasm. de \textunderscore cadimo\textunderscore )
\section{Cadivo}
\begin{itemize}
\item {Grp. gram.:adj.}
\end{itemize}
\begin{itemize}
\item {Utilização:Des.}
\end{itemize}
O mesmo que \textunderscore caduco\textunderscore .
\section{Cadixe}
\begin{itemize}
\item {Grp. gram.:m.}
\end{itemize}
Cavallo árabe, de raça especial.
\section{Cádmeo}
\begin{itemize}
\item {Grp. gram.:adj.}
\end{itemize}
\begin{itemize}
\item {Proveniência:(De \textunderscore Cadmo\textunderscore , n. p.)}
\end{itemize}
Diz-se do primitivo alphabeto dos Gregos.
\section{Cadmia}
\begin{itemize}
\item {Grp. gram.:f.}
\end{itemize}
\begin{itemize}
\item {Proveniência:(Lat. \textunderscore cadmia\textunderscore )}
\end{itemize}
Substância, que contém zinco, ferro, etc.
Zinco oxydado.
\section{Cádmio}
\begin{itemize}
\item {Grp. gram.:m.}
\end{itemize}
\begin{itemize}
\item {Proveniência:(Gr. \textunderscore kadmion\textunderscore )}
\end{itemize}
Metal pardacento, e malleável, que póde facilmente reduzir-se a fôlhas e fios delgados.
\section{Cado}
\begin{itemize}
\item {Grp. gram.:m.}
\end{itemize}
\begin{itemize}
\item {Proveniência:(Gr. \textunderscore kados\textunderscore )}
\end{itemize}
Antigo vaso para líquidos.
Medida de capacidade, entre os antigos.
\section{Cadoiço}
\begin{itemize}
\item {Grp. gram.:m.}
\end{itemize}
\begin{itemize}
\item {Utilização:Prov.}
\end{itemize}
\begin{itemize}
\item {Utilização:minh.}
\end{itemize}
Aloque vasto e fundo.
(Cast. \textunderscore cadozo\textunderscore )
\section{Cadoira}
\begin{itemize}
\item {Grp. gram.:f.}
\end{itemize}
Cabo de linho, amarrado nos punhos das redes da pescada, e que serve para as alar acima.
\section{Cadora}
\begin{itemize}
\item {Grp. gram.:adv.}
\end{itemize}
\begin{itemize}
\item {Utilização:Ant.}
\end{itemize}
Em cada hora. Cf. \textunderscore Aulegrafia\textunderscore , 4.
\section{Cadós}
\begin{itemize}
\item {Grp. gram.:m.}
\end{itemize}
\begin{itemize}
\item {Utilização:Prov.}
\end{itemize}
\begin{itemize}
\item {Utilização:beir.}
\end{itemize}
\begin{itemize}
\item {Utilização:Prov.}
\end{itemize}
\begin{itemize}
\item {Utilização:dur.}
\end{itemize}
\begin{itemize}
\item {Proveniência:(Do cast. ant. \textunderscore cadoso\textunderscore )}
\end{itemize}
Pequena cova, que serve no jôgo da pela.
Covil; toca.
Barril de lixo.
Lugar, donde se não póde sair.
Peixe de água doce.
Homem gasto ou extenuado, por doença ou excessos.
Mulher velha e trôpega.
\section{Cadosete}
\begin{itemize}
\item {fónica:zê}
\end{itemize}
\begin{itemize}
\item {Grp. gram.:m.}
\end{itemize}
\begin{itemize}
\item {Proveniência:(De \textunderscore cadós\textunderscore )}
\end{itemize}
Gênero de peixes abdominaes de água doce.
\section{Cadouço}
\begin{itemize}
\item {Grp. gram.:m.}
\end{itemize}
\begin{itemize}
\item {Utilização:Prov.}
\end{itemize}
\begin{itemize}
\item {Utilização:minh.}
\end{itemize}
Aloque vasto e fundo.
(Cast. \textunderscore cadozo\textunderscore )
\section{Cadoura}
\begin{itemize}
\item {Grp. gram.:f.}
\end{itemize}
Cabo de linho, amarrado nos punhos das redes da pescada, e que serve para as alar acima.
\section{Cadoxe}
\begin{itemize}
\item {Grp. gram.:m.}
\end{itemize}
\begin{itemize}
\item {Proveniência:(Do hebr. \textunderscore kagadar\textunderscore )}
\end{itemize}
Um dos graus superiores da maçonaria, no rito escocês.
\section{Cadoxo}
\begin{itemize}
\item {fónica:dô}
\end{itemize}
\begin{itemize}
\item {Grp. gram.:m.}
\end{itemize}
\begin{itemize}
\item {Utilização:T. de Avintes}
\end{itemize}
O mesmo que \textunderscore cadexo\textunderscore .
Qualquer objecto que serve de núcleo a um novelo.
\section{Cadraço}
\begin{itemize}
\item {Grp. gram.:m.}
\end{itemize}
\begin{itemize}
\item {Utilização:Prov.}
\end{itemize}
\begin{itemize}
\item {Grp. gram.:m.}
\end{itemize}
\begin{itemize}
\item {Utilização:Prov.}
\end{itemize}
\begin{itemize}
\item {Utilização:beir.}
\end{itemize}
O mesmo que \textunderscore bagaço\textunderscore .
Bagaço de uvas.
\section{Caduca}
\begin{itemize}
\item {Grp. gram.:f.}
\end{itemize}
Enfeite, usado nas orelhas por bailadeiras indianas. Cf. Th. Ribeiro, \textunderscore Jornadas\textunderscore , II, 104.
\section{Caducante}
\begin{itemize}
\item {Grp. gram.:adj.}
\end{itemize}
Que caduca.
\section{Caducar}
\begin{itemize}
\item {Grp. gram.:v. i.}
\end{itemize}
Fazer-se caduco.
Envelhecer.
Perder as fôrças.
Deixar de têr valor; tornar-se nullo.
\section{Caducário}
\begin{itemize}
\item {Grp. gram.:adj.}
\end{itemize}
\begin{itemize}
\item {Proveniência:(Lat. \textunderscore caducarius\textunderscore )}
\end{itemize}
Relativo a coisas caducas.
Que diz respeito a bens que deixaram de têr dono.
\section{Caduceador}
\begin{itemize}
\item {Grp. gram.:m.}
\end{itemize}
\begin{itemize}
\item {Utilização:Ant.}
\end{itemize}
\begin{itemize}
\item {Proveniência:(Lat. \textunderscore caduceator\textunderscore )}
\end{itemize}
Parlamentário.
Arauto.
\section{Caduceu}
\begin{itemize}
\item {Grp. gram.:m.}
\end{itemize}
\begin{itemize}
\item {Proveniência:(Lat. \textunderscore caduceum\textunderscore )}
\end{itemize}
Vara delgada e lisa, terminada em duas asas e rodeada por duas serpentes, a qual, sendo insígnia de Mercúrio, era também usada pelos antigos parlamentários.
\section{Caducidade}
\begin{itemize}
\item {Grp. gram.:f.}
\end{itemize}
Qualidade do que é \textunderscore caduco\textunderscore .
\section{Caducífero}
\begin{itemize}
\item {Grp. gram.:adj.}
\end{itemize}
\begin{itemize}
\item {Proveniência:(Do lat. \textunderscore caduceum\textunderscore  + \textunderscore ferre\textunderscore )}
\end{itemize}
Que leva caduceu.
\section{Caduco}
\begin{itemize}
\item {Grp. gram.:adj.}
\end{itemize}
\begin{itemize}
\item {Proveniência:(Lat. \textunderscore caducus\textunderscore )}
\end{itemize}
Que cái; que vai cair.
Que perdeu fôrças, viço, valor, crédito: \textunderscore homem caduco\textunderscore .
Transitório, que desapparece breve: \textunderscore vegetação caduca\textunderscore .
Que se tornou nullo: \textunderscore costume caduco\textunderscore .
\section{Caduquez}
\begin{itemize}
\item {Grp. gram.:f.}
\end{itemize}
(V.caducidade)
\section{Caduquice}
\begin{itemize}
\item {Grp. gram.:f.}
\end{itemize}
\begin{itemize}
\item {Utilização:P. us.}
\end{itemize}
O mesmo que \textunderscore caduquez\textunderscore .
\section{Cadurça}
\begin{itemize}
\item {Grp. gram.:f.}
\end{itemize}
O mesmo que \textunderscore carduça\textunderscore . Cf. \textunderscore Inquér. Indust.\textunderscore , P. II, l. 2.^o, 122.
(Methát. de \textunderscore carduça\textunderscore )
\section{Caeira}
\begin{itemize}
\item {Grp. gram.:f.}
\end{itemize}
Matilha, o mesmo que \textunderscore queira\textunderscore .
\section{Caeiro}
\begin{itemize}
\item {Grp. gram.:m.}
\end{itemize}
(V.caieiro)
\section{Caer}
\begin{itemize}
\item {Grp. gram.:v. i.}
\end{itemize}
\begin{itemize}
\item {Utilização:Ant.}
\end{itemize}
O mesmo que \textunderscore cair\textunderscore .
Caber em sorte ou por herança.
\section{Caes}
\begin{itemize}
\item {Grp. gram.:m.}
\end{itemize}
\begin{itemize}
\item {Proveniência:(Do b. lat. \textunderscore caium\textunderscore . Cp. câmbrico \textunderscore cale\textunderscore )}
\end{itemize}
Elevação de terra, ordinariamente lageada e murada, que á beira de um rio ou de um pôrto, é destinada ao embarque ou desembarque de pessôas ou mercadorias.
Parte das estações de caminhos de ferro, em que se descarregam mercadorias, e se apeiam ou embarcam os passageiros.
\section{Caans}
\begin{itemize}
\item {Grp. gram.:m. pl.}
\end{itemize}
Nome de algumas tríbos indígenas do Brasil, em Mato-Grosso.
\section{Caetano}
\begin{itemize}
\item {Grp. gram.:m.}
\end{itemize}
Árvore silvestre do Brasil.
\section{Caetê}
\begin{itemize}
\item {Grp. gram.:m.}
\end{itemize}
Planta brasileira, de fôlhas largas; bananeira do mato.
Mato bravo ou espinhoso.
\section{Caetés}
\begin{itemize}
\item {fónica:ca-e}
\end{itemize}
\begin{itemize}
\item {Grp. gram.:m. pl.}
\end{itemize}
Nome de várias tribos ferozes, exterminadas no Brasil pelos Tupinambás.
\section{Caetetu}
\begin{itemize}
\item {Grp. gram.:m.}
\end{itemize}
O mesmo que \textunderscore caititu\textunderscore .
\section{Cafajestada}
\begin{itemize}
\item {Grp. gram.:f.}
\end{itemize}
\begin{itemize}
\item {Utilização:Bras}
\end{itemize}
Acto de \textunderscore cafajeste\textunderscore .
Grupo de cafajestes.
\section{Cafajeste}
\begin{itemize}
\item {fónica:gês}
\end{itemize}
\begin{itemize}
\item {Grp. gram.:m.}
\end{itemize}
\begin{itemize}
\item {Utilização:Bras}
\end{itemize}
Homem de ínfima condição.
Indivíduo sem préstimo.
Aquelle que não é estudante e que, em gíria escolar de Coimbra, se chama \textunderscore futrica\textunderscore .
\section{Cafanga}
\begin{itemize}
\item {Grp. gram.:f.}
\end{itemize}
\begin{itemize}
\item {Utilização:Bras}
\end{itemize}
\begin{itemize}
\item {Utilização:Chul.}
\end{itemize}
\begin{itemize}
\item {Utilização:Bras. do Ceará}
\end{itemize}
Desdém fingido por aquillo que se deseja.
Recusa apparente daquillo que se offerece.
Embuste.
Defeito, balda.
\section{Cafangar}
\begin{itemize}
\item {Grp. gram.:v. i.}
\end{itemize}
\begin{itemize}
\item {Utilização:Bras. do Ceará}
\end{itemize}
\begin{itemize}
\item {Proveniência:(De \textunderscore cafanga\textunderscore )}
\end{itemize}
Notar defeitos em quem os não tem.
Zombar.
\section{Çafar}
\begin{itemize}
\item {Grp. gram.:v. t.}
\end{itemize}
\begin{itemize}
\item {Grp. gram.:v. t.}
\end{itemize}
\begin{itemize}
\item {Grp. gram.:V. p.}
\end{itemize}
\begin{itemize}
\item {Utilização:Fam.}
\end{itemize}
Tirar, puxando.
Tirar.
Extrahir.
Subtrahir; furtar.
Livrar, desembaraçar.
Gastar pelo attrito ou uso.
Desgastar.
Escapar, fugir.
(Cp. cast. \textunderscore zafar\textunderscore )
\section{Cáfaro}
\begin{itemize}
\item {Grp. gram.:adj.}
\end{itemize}
\begin{itemize}
\item {Proveniência:(Do ár. \textunderscore çahra\textunderscore , deserto)}
\end{itemize}
(Escrita exacta, em vez de \textunderscore sáfaro\textunderscore . Cf. Garrett, \textunderscore Flôres sem Fruto\textunderscore , 70)
\section{Cafarreiro}
\begin{itemize}
\item {Grp. gram.:m.}
\end{itemize}
Cobrador de cafarro.
\section{Cafarro}
\begin{itemize}
\item {Grp. gram.:m.}
\end{itemize}
Tributo, que se pagava na Terra-Santa.
(Do ár.)
\section{Cafazeste}
\begin{itemize}
\item {fónica:zês}
\end{itemize}
\begin{itemize}
\item {Grp. gram.:m.}
\end{itemize}
O mesmo que \textunderscore cafajeste\textunderscore .
\section{Café}
\begin{itemize}
\item {Grp. gram.:m.}
\end{itemize}
\begin{itemize}
\item {Utilização:Gír.}
\end{itemize}
\begin{itemize}
\item {Proveniência:(Do ár. \textunderscore cahua\textunderscore )}
\end{itemize}
Semente de cafezeiro.
Infusão dessa semente, depois de torrada e moída.
Estabelecimento, em que se toma café e outras bebidas; botequim.
Cafeeiro.
\textunderscore Café frio\textunderscore , chícara de vinho.
\section{Cafedório}
\begin{itemize}
\item {Grp. gram.:m.}
\end{itemize}
\begin{itemize}
\item {Utilização:Fam.}
\end{itemize}
Café ordinário, aguado e sem-sabor.
\section{Cafeeiral}
\begin{itemize}
\item {Grp. gram.:m.}
\end{itemize}
Plantação de cafeeiros.
\section{Cafeeiro}
\begin{itemize}
\item {Grp. gram.:m.}
\end{itemize}
\begin{itemize}
\item {Proveniência:(De \textunderscore café\textunderscore )}
\end{itemize}
Arbusto sempre verde, que produz o café.
\section{Café-eugênio}
\begin{itemize}
\item {Grp. gram.:m.}
\end{itemize}
Fruto de certas plantas myrtáceas.
\section{Cafeico}
\begin{itemize}
\item {Grp. gram.:adj.}
\end{itemize}
Diz-se de um ácido, que se descobriu no café.
\section{Cafeidina}
\begin{itemize}
\item {Grp. gram.:f.}
\end{itemize}
\begin{itemize}
\item {Proveniência:(De \textunderscore café\textunderscore )}
\end{itemize}
Alcaloide, que se obtém pela reacção do barito sôbre a cafeína.
\section{Cafeína}
\begin{itemize}
\item {Grp. gram.:f.}
\end{itemize}
\begin{itemize}
\item {Proveniência:(De \textunderscore café\textunderscore )}
\end{itemize}
Princípio crystallizável, que se desenvolve no café pela torrefacção.
\section{Cafelana}
\begin{itemize}
\item {Grp. gram.:f.}
\end{itemize}
\begin{itemize}
\item {Utilização:Bras}
\end{itemize}
Cafezal extenso.
\section{Cafelista}
\begin{itemize}
\item {Grp. gram.:m.}
\end{itemize}
\begin{itemize}
\item {Utilização:Bras}
\end{itemize}
O mesmo que \textunderscore cafezista\textunderscore .
\section{Cafeona}
\begin{itemize}
\item {Grp. gram.:f.}
\end{itemize}
\begin{itemize}
\item {Proveniência:(De \textunderscore café\textunderscore )}
\end{itemize}
Óleo aromático, extrahido do café torrado.
\section{Cafequezu}
\begin{itemize}
\item {Grp. gram.:m.}
\end{itemize}
Árvore sapotácea de Angola.
\section{Caferana}
\begin{itemize}
\item {Grp. gram.:f.}
\end{itemize}
\begin{itemize}
\item {Utilização:Bras}
\end{itemize}
Planta medicinal, gencianácea, (\textunderscore tachia guynanenis\textunderscore ).
\section{Cafetal}
\begin{itemize}
\item {Grp. gram.:m.}
\end{itemize}
O mesmo que \textunderscore cafeeiral\textunderscore . Cf. Garrett, \textunderscore Helena\textunderscore , 68.
\section{Cafetan}
\begin{itemize}
\item {Grp. gram.:m.}
\end{itemize}
\begin{itemize}
\item {Proveniência:(Fr. \textunderscore cafetan\textunderscore )}
\end{itemize}
Túnica, debruada de pelles, usada no Oriente, e que o sultão da Turquia offerece como distincção.
\section{Café-tânico}
\begin{itemize}
\item {Grp. gram.:adj.}
\end{itemize}
Diz-se de um ácido, extrahido do café.
\section{Café-tânnico}
\begin{itemize}
\item {Grp. gram.:adj.}
\end{itemize}
Diz-se de um ácido, extrahido do café.
\section{Cafeteira}
\begin{itemize}
\item {Grp. gram.:f.}
\end{itemize}
\begin{itemize}
\item {Proveniência:(De \textunderscore café\textunderscore )}
\end{itemize}
Vaso de loiça ou metal, em que se faz a infusão do café ou em que êste se leva á mesa.
\section{Cafezal}
\begin{itemize}
\item {Grp. gram.:m.}
\end{itemize}
O mesmo que \textunderscore cafeeiral\textunderscore .
\section{Cafezeiro}
\begin{itemize}
\item {Grp. gram.:m.}
\end{itemize}
O mesmo que \textunderscore cafeeiro\textunderscore .
\section{Cafezista}
\begin{itemize}
\item {Grp. gram.:m.  e  f.}
\end{itemize}
\begin{itemize}
\item {Grp. gram.:M.}
\end{itemize}
\begin{itemize}
\item {Utilização:Bras}
\end{itemize}
Pessôa, que gosta muito de café.
Proprietário de plantações de café; plantador de café.
\section{Cafifa}
\begin{itemize}
\item {Grp. gram.:m.  e  f.}
\end{itemize}
\begin{itemize}
\item {Utilização:Bras. do N}
\end{itemize}
Pessôa infeliz ao jôgo.
Pessôa, a quem o jogador attribue a sua má sorte.
(Cp. \textunderscore cafife\textunderscore )
\section{Cafife}
\begin{itemize}
\item {Grp. gram.:m.}
\end{itemize}
\begin{itemize}
\item {Utilização:Bras}
\end{itemize}
\begin{itemize}
\item {Proveniência:(Do quimb. \textunderscore kafifi\textunderscore , que vê pouco?)}
\end{itemize}
Série de contrariedades.
Achaque, morrinha.
\section{Cafifice}
\begin{itemize}
\item {Grp. gram.:f.}
\end{itemize}
O mesmo que \textunderscore cafifismo\textunderscore .
\section{Cafifismo}
\begin{itemize}
\item {Grp. gram.:m.}
\end{itemize}
\begin{itemize}
\item {Utilização:Bras}
\end{itemize}
Estado de quem soffre cafife.
\section{Cáfila}
\begin{itemize}
\item {Grp. gram.:f.}
\end{itemize}
\begin{itemize}
\item {Proveniência:(Do ár. \textunderscore kafila\textunderscore )}
\end{itemize}
Caravana.
Grande número de camelos que transportam mercadorias.
Bando; matulagem, corja.
\section{Cafinfim}
\begin{itemize}
\item {Grp. gram.:m.}
\end{itemize}
\begin{itemize}
\item {Utilização:Bras. do N}
\end{itemize}
Piolho de gallinha.
Membro de certo partido politico.
\section{Çafio}
\begin{itemize}
\item {Grp. gram.:m.}
\end{itemize}
(Fórma exacta, em vez da usual, \textunderscore safio\textunderscore )
(Cast. \textunderscore rafio\textunderscore )
\section{Çáfio}
\begin{itemize}
\item {Grp. gram.:adj.}
\end{itemize}
Reles:«\textunderscore ...irás pesar a culpa do bilingue tapuia ou çáfio negro nas trêmulas balanças\textunderscore ». Filinto, V, 4.«\textunderscore ...Elle, moquenco, çáfio e desdenhoso\textunderscore ». \textunderscore Idem\textunderscore , X, 135.
\section{Cáfir-zuló}
\begin{itemize}
\item {Grp. gram.:m.}
\end{itemize}
Dialecto da Zambézia, pertencente ao grupo banto.
\section{Cafiz}
\begin{itemize}
\item {Grp. gram.:m.}
\end{itemize}
Antiga medida de capacidade para sólidos.
\section{Çafões}
\begin{itemize}
\item {Grp. gram.:m. pl.}
\end{itemize}
\begin{itemize}
\item {Utilização:Prov.}
\end{itemize}
\begin{itemize}
\item {Utilização:alent.}
\end{itemize}
Meias-calças de pelles.
(Cast. \textunderscore zahones\textunderscore )
\section{Cafoto}
\begin{itemize}
\item {fónica:fô}
\end{itemize}
\begin{itemize}
\item {Grp. gram.:m.}
\end{itemize}
Arbusto africano, (\textunderscore tephrosia vogelu\textunderscore , Hook.), alto, de flôres vistosas, cujas fôlhas e caules esmagados os negros lançam na água para envenenar os peixes.
\section{Cafoto}
\begin{itemize}
\item {fónica:fô}
\end{itemize}
\begin{itemize}
\item {Grp. gram.:m.}
\end{itemize}
\begin{itemize}
\item {Utilização:Bras}
\end{itemize}
Latrina.
\section{Cafra}
\begin{itemize}
\item {Grp. gram.:f.}
\end{itemize}
\begin{itemize}
\item {Utilização:Des.}
\end{itemize}
Mulher da Cafraria. Cf. \textunderscore Hist. Trág. Marit.\textunderscore , 27.
(Fem. de \textunderscore cafre\textunderscore )
\section{Cafral}
\begin{itemize}
\item {Grp. gram.:adj.}
\end{itemize}
(V.cafreal)
\section{Cafraria}
\begin{itemize}
\item {Grp. gram.:f.}
\end{itemize}
Multidão de cafres.
\section{Cafre}
\begin{itemize}
\item {Grp. gram.:m.}
\end{itemize}
\begin{itemize}
\item {Utilização:Fig.}
\end{itemize}
\begin{itemize}
\item {Utilização:Prov.}
\end{itemize}
\begin{itemize}
\item {Utilização:minh.}
\end{itemize}
\begin{itemize}
\item {Proveniência:(Do ár. \textunderscore kafir\textunderscore )}
\end{itemize}
Habitante da Cafraria.
Língua da Cafraria.
Homem rude, bárbaro.
Homem avarento, sovina. (Colhido em Barcelos)
\section{Cafreal}
\begin{itemize}
\item {Grp. gram.:adj.}
\end{itemize}
Relativo aos cafres ou próprio delles.
\section{Cafrice}
\begin{itemize}
\item {Grp. gram.:f.}
\end{itemize}
Acção própria de cafre; barbaridade, crueldade.
\section{Cafrinho}
\begin{itemize}
\item {Grp. gram.:m.}
\end{itemize}
Arroz preto de Timor.
\section{Cáften}
\begin{itemize}
\item {Grp. gram.:m.}
\end{itemize}
\begin{itemize}
\item {Utilização:Bras}
\end{itemize}
Aquelle que tem commércio de meretrizes.
(Relaciona-se com o ingl. \textunderscore captain\textunderscore ?)
\section{Caftina}
\begin{itemize}
\item {Grp. gram.:f.}
\end{itemize}
\begin{itemize}
\item {Utilização:Bras}
\end{itemize}
\begin{itemize}
\item {Proveniência:(De \textunderscore cáften\textunderscore )}
\end{itemize}
Mulher, que tem negócio de meretrizes.
\section{Cafua}
\begin{itemize}
\item {Grp. gram.:f.}
\end{itemize}
\begin{itemize}
\item {Utilização:Bras. da Baía}
\end{itemize}
Cova; antro.
Esconderijo.
Habitação miserável.
Quarto, que, nos collégios, serve de prisão a estudantes.
\section{Cafuão}
\begin{itemize}
\item {Grp. gram.:m.}
\end{itemize}
\begin{itemize}
\item {Utilização:Açor}
\end{itemize}
\begin{itemize}
\item {Proveniência:(De \textunderscore cafua\textunderscore )}
\end{itemize}
Tulha subterrânea, usada na ilha de San-Miguel.
\section{Cafuca}
\begin{itemize}
\item {Grp. gram.:f.}
\end{itemize}
\begin{itemize}
\item {Utilização:Bras}
\end{itemize}
Cova de carvão de madeira.
\section{Cafuínha}
\begin{itemize}
\item {Grp. gram.:m.  e  f.}
\end{itemize}
\begin{itemize}
\item {Utilização:Des.}
\end{itemize}
Pessôa avarenta, fuínha.
\section{Cafula-tungo}
\begin{itemize}
\item {Grp. gram.:m.}
\end{itemize}
\begin{itemize}
\item {Proveniência:(T. lund.)}
\end{itemize}
Pequena árvore angolense, de fôlhas verde-escuras e flôres em espigas terminaes.
\section{Cafulo}
\begin{itemize}
\item {Grp. gram.:m.}
\end{itemize}
\begin{itemize}
\item {Utilização:Prov.}
\end{itemize}
\begin{itemize}
\item {Utilização:trasm.}
\end{itemize}
Carolo do milho.
\section{Cafundó}
\begin{itemize}
\item {Grp. gram.:m.}
\end{itemize}
\begin{itemize}
\item {Utilização:Bras}
\end{itemize}
Lugar ermo e distante, de accesso diffícil, ordinariamente entre montanhas.
O mesmo que \textunderscore cafua\textunderscore , quarto para prisão de collegiaes.
\section{Cafundório}
\begin{itemize}
\item {Grp. gram.:m.}
\end{itemize}
\begin{itemize}
\item {Utilização:Bras}
\end{itemize}
O mesmo que \textunderscore cafundó\textunderscore .
\section{Cafuné}
\begin{itemize}
\item {Grp. gram.:m.}
\end{itemize}
\begin{itemize}
\item {Utilização:Bras}
\end{itemize}
\begin{itemize}
\item {Utilização:Bras}
\end{itemize}
Estalído, que se dá com as unhas sôbre a cabeça de alguém, para o adormentar.
Pequeno dendê, intercalado nos grandes.
\section{Cafungar}
\begin{itemize}
\item {Grp. gram.:v. t.}
\end{itemize}
\begin{itemize}
\item {Utilização:Bras}
\end{itemize}
Investigar, procurar minuciosamente.
(Por \textunderscore cafucar\textunderscore , de \textunderscore cafuca\textunderscore ?)
\section{Cafunge}
\begin{itemize}
\item {Grp. gram.:m.}
\end{itemize}
\begin{itemize}
\item {Utilização:Bras}
\end{itemize}
Moleque travêsso e larápio.
Gatuno.
\section{Cafurna}
\begin{itemize}
\item {Grp. gram.:f.}
\end{itemize}
O mesmo que \textunderscore cafua\textunderscore .
(Refl. de \textunderscore furna\textunderscore )
\section{Cafus}
\begin{itemize}
\item {Grp. gram.:m.  e  adj.}
\end{itemize}
(V.cafusa)
\section{Cafusa}
\begin{itemize}
\item {Grp. gram.:m.  e  adj.}
\end{itemize}
\begin{itemize}
\item {Utilização:Bras}
\end{itemize}
Filho ou filha de mulato e preta, ou de preto e mulata.
Descendente de preto e de índio na América.
\section{Cafuso}
\begin{itemize}
\item {Grp. gram.:m.}
\end{itemize}
O mesmo que \textunderscore cafusa\textunderscore .
\section{Caga}
\begin{itemize}
\item {Grp. gram.:m.}
\end{itemize}
\begin{itemize}
\item {Utilização:Pleb.}
\end{itemize}
\begin{itemize}
\item {Grp. gram.:F.}
\end{itemize}
\begin{itemize}
\item {Proveniência:(De \textunderscore cagar\textunderscore )}
\end{itemize}
Homem lamecha.
Homem, que se encoleriza com um motejo.
O mesmo que \textunderscore caca\textunderscore . Cf. G. Vicente, I, 224.
\section{Çaga}
\begin{itemize}
\item {Grp. gram.:f.}
\end{itemize}
\begin{itemize}
\item {Utilização:Des.}
\end{itemize}
O mesmo que \textunderscore encalço\textunderscore  ou \textunderscore retaguarda\textunderscore : \textunderscore ir na çaga de alguém\textunderscore . Cf. Herculano, \textunderscore Hist. de Port.\textunderscore , IV, 415.
\section{Caga-andando}
\begin{itemize}
\item {Grp. gram.:m.}
\end{itemize}
\begin{itemize}
\item {Utilização:ant.}
\end{itemize}
\begin{itemize}
\item {Utilização:Chul.}
\end{itemize}
Indivíduo, que anda muito devagar, fazendo resair muito as nádegas.
\section{Cagaçal}
\begin{itemize}
\item {Grp. gram.:m.}
\end{itemize}
\begin{itemize}
\item {Utilização:Pleb.}
\end{itemize}
\begin{itemize}
\item {Utilização:Prov.}
\end{itemize}
\begin{itemize}
\item {Utilização:alent.}
\end{itemize}
Sitio, onde se deitam excrementos.
Pessôa ordinária, vil.
Olival pequeno.
(Cp. \textunderscore cagaço\textunderscore )
\section{Cagaçal}
\begin{itemize}
\item {Grp. gram.:m.}
\end{itemize}
Cêrco, que as toninhas fazem á sardinha quando a perseguem.
\section{Cagação}
\begin{itemize}
\item {Grp. gram.:adj.}
\end{itemize}
\begin{itemize}
\item {Utilização:Pleb.}
\end{itemize}
Que tem cagaço, que é medroso.
\section{Cagaço}
\begin{itemize}
\item {Grp. gram.:m.}
\end{itemize}
\begin{itemize}
\item {Utilização:Pleb.}
\end{itemize}
\begin{itemize}
\item {Proveniência:(De \textunderscore cagar\textunderscore )}
\end{itemize}
Mêdo, susto, terror.
\section{Cagada}
\begin{itemize}
\item {Grp. gram.:f.}
\end{itemize}
\begin{itemize}
\item {Utilização:Pleb.}
\end{itemize}
\begin{itemize}
\item {Proveniência:(De \textunderscore cagar\textunderscore )}
\end{itemize}
Acto de defecar; dejecção.
\section{Cagadela}
\begin{itemize}
\item {Grp. gram.:f.}
\end{itemize}
O mesmo que \textunderscore cagada\textunderscore .
Dejecção de mosca ou de outro insecto em superfície limpa.
\section{Cagadinha}
\begin{itemize}
\item {Grp. gram.:f.}
\end{itemize}
\begin{itemize}
\item {Utilização:Prov.}
\end{itemize}
\begin{itemize}
\item {Utilização:alent.}
\end{itemize}
Variedade de roman.
\section{Cagádo}
\begin{itemize}
\item {Grp. gram.:m.}
\end{itemize}
O mesmo que \textunderscore mandrião\textunderscore , ave.
\section{Cágado}
\begin{itemize}
\item {Grp. gram.:m.}
\end{itemize}
\begin{itemize}
\item {Grp. gram.:M.  e  adj.}
\end{itemize}
\begin{itemize}
\item {Utilização:Pop.}
\end{itemize}
\begin{itemize}
\item {Proveniência:(Do lat. hyp. \textunderscore cacitus\textunderscore ?)}
\end{itemize}
Espécie de tartaruga de água doce.
Chapuz para os cabos do leme.
Sujeito finório, manhoso.
Preguiçoso.
\section{Cagadoiro}
\begin{itemize}
\item {Grp. gram.:m.}
\end{itemize}
O mesmo que \textunderscore cagatório\textunderscore .
\section{Cagadol}
\begin{itemize}
\item {Grp. gram.:m.}
\end{itemize}
Árvore de Damão.
\section{Caga-fogo}
\begin{itemize}
\item {Grp. gram.:m.}
\end{itemize}
\begin{itemize}
\item {Utilização:Bras}
\end{itemize}
Abelha, de corpo delgado e negro.
O mesmo que \textunderscore caga-lume\textunderscore .
\section{Cagaforra}
\begin{itemize}
\item {fónica:fô}
\end{itemize}
\begin{itemize}
\item {Grp. gram.:f.}
\end{itemize}
\begin{itemize}
\item {Utilização:Prov.}
\end{itemize}
\begin{itemize}
\item {Utilização:beir.}
\end{itemize}
O mesmo que \textunderscore cacaforro\textunderscore .
\section{Cagaita}
\begin{itemize}
\item {Grp. gram.:f.}
\end{itemize}
Fruto da cagaiteira.
\section{Cagaiteira}
\begin{itemize}
\item {Grp. gram.:f.}
\end{itemize}
Árvore fructífera do Brasil, (\textunderscore eugenia dysenterica\textunderscore , Mart.).
\section{Cagalhão}
\begin{itemize}
\item {Grp. gram.:m.}
\end{itemize}
\begin{itemize}
\item {Utilização:Pleb.}
\end{itemize}
\begin{itemize}
\item {Proveniência:(De \textunderscore cagar\textunderscore )}
\end{itemize}
Porção consistente de excremento.
\section{Cagalho}
\begin{itemize}
\item {Grp. gram.:m.}
\end{itemize}
Ave africana.
\section{Cagalhoada}
\begin{itemize}
\item {Grp. gram.:f.}
\end{itemize}
\begin{itemize}
\item {Utilização:Burl.}
\end{itemize}
Porção de coisas miúdas e insignificantes.
\section{Cagalismo}
\begin{itemize}
\item {Grp. gram.:m.}
\end{itemize}
\begin{itemize}
\item {Utilização:Pleb.}
\end{itemize}
Qualidade de caga.
\section{Cagalizar}
\begin{itemize}
\item {Grp. gram.:v. t.}
\end{itemize}
\begin{itemize}
\item {Utilização:Pleb.}
\end{itemize}
Tornar caga.
\section{Caga-lume}
\begin{itemize}
\item {Grp. gram.:m.}
\end{itemize}
\begin{itemize}
\item {Utilização:Pleb.}
\end{itemize}
O mesmo que \textunderscore pyrilampo\textunderscore .
\section{Cagamasso}
\begin{itemize}
\item {Grp. gram.:m.}
\end{itemize}
Nome de uma erva, que crescia especialmente nos coitos de Alcobaça.
\section{Caga-merdeira}
\begin{itemize}
\item {Grp. gram.:f.}
\end{itemize}
\begin{itemize}
\item {Utilização:Ant.}
\end{itemize}
Excremento. Cf. G. Vicente, \textunderscore Auto da Barca\textunderscore .
\section{Caganapo}
\begin{itemize}
\item {Grp. gram.:m.}
\end{itemize}
O mesmo que \textunderscore caganato\textunderscore . (Colhido na Régua)
\section{Caganato}
\begin{itemize}
\item {Grp. gram.:m.}
\end{itemize}
\begin{itemize}
\item {Utilização:Prov.}
\end{itemize}
\begin{itemize}
\item {Utilização:dur.}
\end{itemize}
Larva da ran; o mesmo que \textunderscore girino\textunderscore .
\section{Caganefa}
\begin{itemize}
\item {Grp. gram.:f.}
\end{itemize}
\begin{itemize}
\item {Utilização:Gír.}
\end{itemize}
Espingarda.
\section{Caganeira}
\begin{itemize}
\item {Grp. gram.:f.}
\end{itemize}
\begin{itemize}
\item {Utilização:Pleb.}
\end{itemize}
\begin{itemize}
\item {Utilização:alg.}
\end{itemize}
\begin{itemize}
\item {Utilização:Fig.}
\end{itemize}
O mesmo que \textunderscore diarreia\textunderscore .
Porção de tolices.
(Cp. \textunderscore cagão\textunderscore )
\section{Caganeta}
\begin{itemize}
\item {fónica:nê}
\end{itemize}
\begin{itemize}
\item {Grp. gram.:f.}
\end{itemize}
\begin{itemize}
\item {Utilização:Pleb.}
\end{itemize}
\begin{itemize}
\item {Grp. gram.:M.  e  f.}
\end{itemize}
\begin{itemize}
\item {Utilização:T. de Barcelos}
\end{itemize}
O mesmo que \textunderscore caganita\textunderscore .
Pessôa muito baixa, muito atarracada.
O mesmo que \textunderscore cagarola\textunderscore .
\section{Caganifância}
\begin{itemize}
\item {Grp. gram.:f.}
\end{itemize}
\begin{itemize}
\item {Utilização:Pleb.}
\end{itemize}
Coisa de pouco valor, insignificância, bagatela.
\section{Caganita}
\begin{itemize}
\item {Grp. gram.:f.}
\end{itemize}
\begin{itemize}
\item {Utilização:Pleb.}
\end{itemize}
\begin{itemize}
\item {Proveniência:(De \textunderscore cagar\textunderscore )}
\end{itemize}
Excremento de certos animaes, em fórma de pequeninas bolas.
\section{Caganito}
\begin{itemize}
\item {Grp. gram.:m.}
\end{itemize}
\begin{itemize}
\item {Utilização:Pleb.}
\end{itemize}
Indivíduo muito baixo, de pequena estatura.
(Cp. \textunderscore caganita\textunderscore )
\section{Cagão}
\begin{itemize}
\item {Grp. gram.:m.}
\end{itemize}
\begin{itemize}
\item {Utilização:Pleb.}
\end{itemize}
\begin{itemize}
\item {Utilização:Fig.}
\end{itemize}
\begin{itemize}
\item {Proveniência:(De \textunderscore cagar\textunderscore )}
\end{itemize}
Aquelle que tem diarreia.
Criança, que defeca muito ou amiúde.
Homem medroso.
\section{Cagar}
\begin{itemize}
\item {Grp. gram.:v. t.  e  i.}
\end{itemize}
\begin{itemize}
\item {Utilização:Pleb.}
\end{itemize}
\begin{itemize}
\item {Proveniência:(Lat. \textunderscore cacare\textunderscore )}
\end{itemize}
Defecar; expellir qualquer coisa pelo ânus.
Emporcalhar.
Têr desprêzo por qualquer coisa: \textunderscore estou-me cagando para isso\textunderscore .«\textunderscore Em ti mijo, em ti cago, ó formosura\textunderscore ». Bocage.
\section{Cagarola}
\begin{itemize}
\item {Grp. gram.:m.  e  f.}
\end{itemize}
\begin{itemize}
\item {Utilização:Pleb.}
\end{itemize}
\begin{itemize}
\item {Proveniência:(De \textunderscore cagar\textunderscore )}
\end{itemize}
Pessôa, que se assusta com qualquer coisa.
\section{Cagarra}
\begin{itemize}
\item {Grp. gram.:f.}
\end{itemize}
\begin{itemize}
\item {Utilização:Mad}
\end{itemize}
\begin{itemize}
\item {Utilização:Açor}
\end{itemize}
O mesmo que \textunderscore pardoca\textunderscore .
Espécie de gaivota.
\section{Cagarralo}
\begin{itemize}
\item {Grp. gram.:m.}
\end{itemize}
Designação popular do girino.
\section{Cagarrão}
\begin{itemize}
\item {Grp. gram.:m.}
\end{itemize}
\begin{itemize}
\item {Utilização:Pleb.}
\end{itemize}
\begin{itemize}
\item {Utilização:Gír.}
\end{itemize}
\begin{itemize}
\item {Proveniência:(De \textunderscore cagar\textunderscore )}
\end{itemize}
Homem muito medroso.
Penico grande.
Cadeia, prisão.
\section{Cagarraz}
\begin{itemize}
\item {Grp. gram.:m.}
\end{itemize}
Espécie de \textunderscore mergulhão\textunderscore , ave, (\textunderscore podiceps nigricollis\textunderscore , Brehm.).
\section{Cagarreta}
\begin{itemize}
\item {fónica:rê}
\end{itemize}
\begin{itemize}
\item {Grp. gram.:m.  e  f.}
\end{itemize}
\begin{itemize}
\item {Utilização:Chul.}
\end{itemize}
Pessôa muito baixa, atarracada.
\section{Cagarria}
\begin{itemize}
\item {Grp. gram.:f.}
\end{itemize}
O mesmo que \textunderscore cagarrinha\textunderscore . Cf. Filinto, XII, 174.
\section{Cagarrinha}
\begin{itemize}
\item {Grp. gram.:f.}
\end{itemize}
Pequeno peixe de água doce.
\section{Cagarrufa}
\begin{itemize}
\item {Grp. gram.:f.}
\end{itemize}
\begin{itemize}
\item {Utilização:Gír.}
\end{itemize}
O mesmo que \textunderscore caganefa\textunderscore .
\section{Caga-sebo}
\begin{itemize}
\item {Grp. gram.:m.}
\end{itemize}
Pequena ave do Brasil.
O mesmo que \textunderscore caca-sebo\textunderscore .
\section{Cagatório}
\begin{itemize}
\item {Grp. gram.:m.}
\end{itemize}
\begin{itemize}
\item {Utilização:Pleb.}
\end{itemize}
Latrina.
\section{Cagião}
\begin{itemize}
\item {Grp. gram.:m.}
\end{itemize}
\begin{itemize}
\item {Utilização:Ant.}
\end{itemize}
O mesmo que \textunderscore cajão\textunderscore .
\section{Caginga}
\begin{itemize}
\item {Grp. gram.:f.}
\end{itemize}
Boné semi-esphérico, com duas grandes saliências cónicas, caídas ao lado, fabricado de fibras de malváceas e leguminosas por indígenas da África oriental.
\section{Cagoans}
\begin{itemize}
\item {Grp. gram.:m. pl.}
\end{itemize}
\begin{itemize}
\item {Utilização:Bras}
\end{itemize}
Selvagens, que habitaram em San-Paulo.
\section{Cagolulo}
\begin{itemize}
\item {Grp. gram.:m.}
\end{itemize}
Ave colombina da África occidental.
\section{Cagom}
\begin{itemize}
\item {Grp. gram.:m.}
\end{itemize}
\begin{itemize}
\item {Utilização:Prov.}
\end{itemize}
\begin{itemize}
\item {Utilização:minh.}
\end{itemize}
O mesmo que \textunderscore pisca-longa\textunderscore .
\section{Cagona}
(\textunderscore fem.\textunderscore  de \textunderscore cagão\textunderscore )
\section{Cagoro}
\begin{itemize}
\item {Grp. gram.:m.}
\end{itemize}
\begin{itemize}
\item {Utilização:T. de Moçambique}
\end{itemize}
O mesmo que \textunderscore pau-ferro\textunderscore .
\section{Cagote}
\begin{itemize}
\item {Grp. gram.:m.}
\end{itemize}
\begin{itemize}
\item {Utilização:Prov.}
\end{itemize}
\begin{itemize}
\item {Utilização:dur.}
\end{itemize}
O mesmo que \textunderscore caganeira\textunderscore .
\section{Cagosanga}
\begin{itemize}
\item {Grp. gram.:f.}
\end{itemize}
Nome da ipecacuanha no Brasil.
\section{Çaguão}
\begin{itemize}
\item {Grp. gram.:m.}
\end{itemize}
Pátio estreito e descoberto, no interior de um edifício.
Espécie de alpendre, á entrada dos conventos.
(Cast. \textunderscore zaguan\textunderscore )
\section{Cágueda}
\begin{itemize}
\item {Grp. gram.:f.}
\end{itemize}
\begin{itemize}
\item {Utilização:Prov.}
\end{itemize}
\begin{itemize}
\item {Utilização:alent.}
\end{itemize}
\begin{itemize}
\item {Proveniência:(De \textunderscore cáguedo\textunderscore , por \textunderscore cágado\textunderscore )}
\end{itemize}
Travinca, com que ás vezes se prende o chocalho á colleira.
Travinca, que se junta a estrevenga pelas extremidades.
\section{Caguedol}
\begin{itemize}
\item {Grp. gram.:m.}
\end{itemize}
Árvore de Moçambique, o mesmo que \textunderscore cagadol\textunderscore .
\section{Cagueiro}
\begin{itemize}
\item {Grp. gram.:m.}
\end{itemize}
\begin{itemize}
\item {Utilização:Pleb.}
\end{itemize}
\begin{itemize}
\item {Proveniência:(De \textunderscore cagar\textunderscore )}
\end{itemize}
Ânus.
\section{Caguetas!}
\begin{itemize}
\item {fónica:guê}
\end{itemize}
\begin{itemize}
\item {Grp. gram.:interj.}
\end{itemize}
\begin{itemize}
\item {Utilização:Pleb.}
\end{itemize}
Ora!
\section{Caguincha}
\begin{itemize}
\item {Grp. gram.:m.}
\end{itemize}
\begin{itemize}
\item {Utilização:Prov.}
\end{itemize}
O mesmo que \textunderscore caguinchas\textunderscore .
Homem importuno, fastidioso. (Colhido em Turquel)
\section{Caguinchas}
\begin{itemize}
\item {Grp. gram.:m.}
\end{itemize}
\begin{itemize}
\item {Utilização:Pleb.}
\end{itemize}
O mesmo que \textunderscore cagarola\textunderscore .
\section{Caguincho}
\begin{itemize}
\item {Grp. gram.:m.}
\end{itemize}
\begin{itemize}
\item {Utilização:Bras}
\end{itemize}
O dois de paus das cartas de jogar.
O mesmo que \textunderscore caguinchas\textunderscore .
\section{Cagulo}
\begin{itemize}
\item {Grp. gram.:m.}
\end{itemize}
\begin{itemize}
\item {Utilização:Prov.}
\end{itemize}
O mesmo que \textunderscore cogulo\textunderscore .
(Colhido na Guarda)
\section{Cahans}
\begin{itemize}
\item {Grp. gram.:m. pl.}
\end{itemize}
Nome de algumas tríbos indígenas do Brasil, em Mato-Grosso.
\section{Caatinga}
\begin{itemize}
\item {Grp. gram.:f.}
\end{itemize}
\begin{itemize}
\item {Utilização:Bras}
\end{itemize}
\begin{itemize}
\item {Grp. gram.:f.}
\end{itemize}
Terra pantanosa, em que cresce a piassabeira.
O mesmo ou melhór que \textunderscore catinga\textunderscore ^2.
\section{Cahatinga}
\begin{itemize}
\item {Grp. gram.:f.}
\end{itemize}
\begin{itemize}
\item {Utilização:Bras}
\end{itemize}
Terra pantanosa, em que cresce a piassabeira.
\section{Cahique}
\begin{itemize}
\item {Grp. gram.:m.}
\end{itemize}
(V.caíque)
\section{Cahir}
\textunderscore v. i.\textunderscore  (e der.)
(Fórma incorrecta, por \textunderscore cair\textunderscore , etc.)
\section{Caia}
\begin{itemize}
\item {Grp. gram.:f.}
\end{itemize}
\begin{itemize}
\item {Utilização:T. de Macau}
\end{itemize}
\begin{itemize}
\item {Proveniência:(Do japon. \textunderscore ka\textunderscore , mosquito, e \textunderscore ya\textunderscore , recinto)}
\end{itemize}
O mesmo que \textunderscore mosquiteiro\textunderscore .
\section{Caiação}
\begin{itemize}
\item {Grp. gram.:f.}
\end{itemize}
Acto ou effeito de caiar.
\section{Caiada}
\begin{itemize}
\item {Grp. gram.:f.}
\end{itemize}
Pássaro dentirostro, o mesmo que \textunderscore tanjasno\textunderscore .
\section{Caiadeira}
\begin{itemize}
\item {Grp. gram.:f.}
\end{itemize}
Mulher, que se emprega em caiar.
\section{Caiadela}
\begin{itemize}
\item {Grp. gram.:f.}
\end{itemize}
\begin{itemize}
\item {Utilização:Fig.}
\end{itemize}
\begin{itemize}
\item {Proveniência:(De \textunderscore caiar\textunderscore )}
\end{itemize}
Caiação.
Mão de cal.
Acção de branquear a pelle com cosméticos.
\section{Caiado}
\begin{itemize}
\item {Grp. gram.:m.}
\end{itemize}
\begin{itemize}
\item {Utilização:Prov.}
\end{itemize}
\begin{itemize}
\item {Utilização:alent.}
\end{itemize}
Peixe dos Açores, também conhecido por \textunderscore clérigo\textunderscore .
Serviço de caiar: \textunderscore hoje, cá em casa, é dia de caiados\textunderscore .
\section{Caiador}
\begin{itemize}
\item {Grp. gram.:m.}
\end{itemize}
\begin{itemize}
\item {Proveniência:(De \textunderscore caiar\textunderscore )}
\end{itemize}
Aquelle que caia.
\section{Caiadura}
\begin{itemize}
\item {Grp. gram.:f.}
\end{itemize}
O mesmo que \textunderscore caiadela\textunderscore .
\section{Caiala}
\begin{itemize}
\item {Grp. gram.:f.}
\end{itemize}
O mesmo que \textunderscore vielo\textunderscore .
\section{Caiala-camochi}
\begin{itemize}
\item {Grp. gram.:m.}
\end{itemize}
Planta herbácea e leguminosa de Angola, (\textunderscore uraria picta\textunderscore , Desv.).
\section{Caiana}
\begin{itemize}
\item {Grp. gram.:f.  e  adj.}
\end{itemize}
\begin{itemize}
\item {Utilização:Bras}
\end{itemize}
Variedade de cana de açúcar.
\section{Caiané}
\begin{itemize}
\item {Grp. gram.:m.}
\end{itemize}
Árvore oleaginosa do Brasil.
(Caminhoá lê \textunderscore caianê\textunderscore )
\section{Caiapiá}
\begin{itemize}
\item {Grp. gram.:m.}
\end{itemize}
Raiz medicinal do Brasil.
\section{Caiapós}
\begin{itemize}
\item {Grp. gram.:m. pl.}
\end{itemize}
Aguerrida nação de Índios do Brasil, que dominavam nas capitanias de Goiás, San-Paulo e Minas-Geraes.
\section{Caiar}
\begin{itemize}
\item {Grp. gram.:v. t.}
\end{itemize}
\begin{itemize}
\item {Utilização:Fig.}
\end{itemize}
Branquear com cal, diluida em água.
Branquear (a pelle) com cosméticos.
Simular, mascarar.
(Contr. de \textunderscore calear\textunderscore , de \textunderscore cal\textunderscore )
\section{Caiarara}
\begin{itemize}
\item {Grp. gram.:m.}
\end{itemize}
\begin{itemize}
\item {Utilização:Bras}
\end{itemize}
Espécie de macaco.
\section{Caibo}
\begin{itemize}
\item {Grp. gram.:m.}
\end{itemize}
\begin{itemize}
\item {Utilização:Pop.}
\end{itemize}
O mesmo que \textunderscore cabo\textunderscore ^3, lugar ou espaço onde se cabe.
\section{Cãibra}
\begin{itemize}
\item {Grp. gram.:f.}
\end{itemize}
(V.câimbra)
\section{Caibrada}
\begin{itemize}
\item {Grp. gram.:f.}
\end{itemize}
Pancada com caibro.
\section{Caibral}
\begin{itemize}
\item {Grp. gram.:adj.}
\end{itemize}
\begin{itemize}
\item {Proveniência:(De \textunderscore caibro\textunderscore ^1)}
\end{itemize}
Relativo a caibros.
Diz-se de um prego grande, com que se fixam caibros ou madeira grossa.
\section{Caibramento}
\begin{itemize}
\item {Grp. gram.:m.}
\end{itemize}
Conjunto dos caibros de um telhado.
Acto de \textunderscore caibrar\textunderscore .
\section{Caibrar}
\begin{itemize}
\item {Grp. gram.:v. t.}
\end{itemize}
Pôr caibros em: \textunderscore caibrar uma casa, um telhado\textunderscore .
\section{Caibro}
\begin{itemize}
\item {Grp. gram.:m.}
\end{itemize}
\begin{itemize}
\item {Utilização:Prov.}
\end{itemize}
\begin{itemize}
\item {Utilização:minh.}
\end{itemize}
\begin{itemize}
\item {Proveniência:(Do lat. des. \textunderscore capreus\textunderscore ?)}
\end{itemize}
Pau, quadrado e grosso, com que se liga o frechal á cumeeira da construcção, e sôbre o qual assentam as ripas.
Cada uma das peças da roda do carro, separadas pelo meão.
Cada um dos barrotes, que formam a latada, assentando nos esteios.
\section{Caibro}
\begin{itemize}
\item {Grp. gram.:m.}
\end{itemize}
\begin{itemize}
\item {Utilização:Bras}
\end{itemize}
Um par de quaesquer objectos, especialmente duas espigas de milho, ligadas entre si pela própria palha.
\section{Caiçá}
\begin{itemize}
\item {Grp. gram.:m.}
\end{itemize}
\begin{itemize}
\item {Utilização:Bras}
\end{itemize}
O mesmo que \textunderscore caiçara\textunderscore .
\section{Caicai}
\begin{itemize}
\item {Grp. gram.:m.}
\end{itemize}
\begin{itemize}
\item {Utilização:Bras}
\end{itemize}
Espécie de rede de pescar.
\section{Caiçara}
\begin{itemize}
\item {Grp. gram.:f.}
\end{itemize}
\begin{itemize}
\item {Utilização:Bras}
\end{itemize}
\begin{itemize}
\item {Utilização:Bras. do N}
\end{itemize}
\begin{itemize}
\item {Grp. gram.:M.}
\end{itemize}
\begin{itemize}
\item {Utilização:Bras. do N}
\end{itemize}
\begin{itemize}
\item {Proveniência:(Do guar. \textunderscore caá-içá\textunderscore )}
\end{itemize}
Arvoredo morto, de que restam troncos e forquilhas.
Mólho de ramos, que se deita na água, para attrahir peixes.
Curral.
Caboclo ruim, sem préstimo.
Cobra venenosa.
\section{Caiçu}
\begin{itemize}
\item {Grp. gram.:m.}
\end{itemize}
Pequena ave do Brasil.
\section{Caiçuma}
\begin{itemize}
\item {Grp. gram.:f.}
\end{itemize}
Bebida, feita de mandioca cozida e fermentada, muito apreciada por algumas tríbos da América do Sul.
\section{Caída}
\begin{itemize}
\item {Grp. gram.:f.}
\end{itemize}
\begin{itemize}
\item {Utilização:Des.}
\end{itemize}
\begin{itemize}
\item {Proveniência:(De \textunderscore cair\textunderscore )}
\end{itemize}
Quéda.
\section{Caideiro}
\begin{itemize}
\item {fónica:ca-i}
\end{itemize}
\begin{itemize}
\item {Grp. gram.:adj.}
\end{itemize}
\begin{itemize}
\item {Proveniência:(De \textunderscore cair\textunderscore )}
\end{itemize}
Caduco.
\section{Caidiço}
\begin{itemize}
\item {fónica:ca-i}
\end{itemize}
\begin{itemize}
\item {Grp. gram.:adj.}
\end{itemize}
(V.caideiro)
\section{Caído}
\begin{itemize}
\item {Grp. gram.:adj.}
\end{itemize}
\begin{itemize}
\item {Grp. gram.:M. pl.}
\end{itemize}
\begin{itemize}
\item {Proveniência:(De \textunderscore cair\textunderscore )}
\end{itemize}
Abatido; triste: \textunderscore andar muito caído\textunderscore .
Que constitue dívida, que é devido, ou cujo pagamento se atrasou: \textunderscore juros caídos\textunderscore .
Rendas vencidas e não pagas.
Restos, desperdícios.
\section{Caidor}
\begin{itemize}
\item {fónica:ca-i}
\end{itemize}
\begin{itemize}
\item {Grp. gram.:adj.}
\end{itemize}
\begin{itemize}
\item {Utilização:Des.}
\end{itemize}
Que cai. Cf. Filinto II, 300.
\section{Caiebabinongo}
\begin{itemize}
\item {Grp. gram.:m.}
\end{itemize}
Pássaro tenuirostro da África.
\section{Caieira}
\begin{itemize}
\item {Grp. gram.:f.}
\end{itemize}
\begin{itemize}
\item {Proveniência:(De \textunderscore caiar\textunderscore )}
\end{itemize}
Fábrica de cal.
Lugar, onde se calcina a cal.
\section{Caieiro}
\begin{itemize}
\item {Grp. gram.:m.}
\end{itemize}
\begin{itemize}
\item {Proveniência:(De \textunderscore caiar\textunderscore )}
\end{itemize}
O mesmo que \textunderscore caiador\textunderscore .
\section{Caienas}
\begin{itemize}
\item {Grp. gram.:m. pl.}
\end{itemize}
Tríbo de índios da América do Norte.
\section{Caieque}
\begin{itemize}
\item {Grp. gram.:m.}
\end{itemize}
\begin{itemize}
\item {Proveniência:(T. afr.)}
\end{itemize}
Larva do salalé.
\section{Caigeira}
\begin{itemize}
\item {fónica:ca-i}
\end{itemize}
\begin{itemize}
\item {Grp. gram.:f.}
\end{itemize}
\begin{itemize}
\item {Utilização:Prov.}
\end{itemize}
\begin{itemize}
\item {Utilização:minh.}
\end{itemize}
\begin{itemize}
\item {Proveniência:(Do lat. hyp. \textunderscore caliginaria\textunderscore , de \textunderscore caligo\textunderscore , \textunderscore caliginis\textunderscore )}
\end{itemize}
O mesmo que \textunderscore nevoeiro\textunderscore .
\section{Caim!}
(t. onom., que designa o latido doloroso do cão)
\section{Caim}
\begin{itemize}
\item {Grp. gram.:m.}
\end{itemize}
\begin{itemize}
\item {Utilização:Açor}
\end{itemize}
Homem mau.
Patife.
(de \textunderscore Caim\textunderscore , n. p. bíbl.)
\section{Caimacão}
\begin{itemize}
\item {Grp. gram.:m.}
\end{itemize}
\begin{itemize}
\item {Proveniência:(Do ár. \textunderscore kaim\textunderscore  + \textunderscore mekam\textunderscore )}
\end{itemize}
Nome de certos funccionários turcos.
\section{Caimal}
\begin{itemize}
\item {Grp. gram.:m.}
\end{itemize}
\begin{itemize}
\item {Proveniência:(T. malab.)}
\end{itemize}
Antiga designação dos senhores e príncipes, no Malabar.
\section{Caimão}
\begin{itemize}
\item {Grp. gram.:m.}
\end{itemize}
\begin{itemize}
\item {Proveniência:(Fr. \textunderscore caïman\textunderscore )}
\end{itemize}
Espécie de \textunderscore alligator\textunderscore .
\section{Caimão}
\begin{itemize}
\item {Grp. gram.:m.}
\end{itemize}
O mesmo que \textunderscore caimal\textunderscore .
\section{Câimbas}
\begin{itemize}
\item {Grp. gram.:f. pl.}
\end{itemize}
\begin{itemize}
\item {Utilização:Açor}
\end{itemize}
Segmentos lateraes da roda do carro, na ilha de San-Miguel.
\section{Caimbeiro}
\begin{itemize}
\item {Grp. gram.:m.}
\end{itemize}
\begin{itemize}
\item {Utilização:Ant.}
\end{itemize}
\begin{itemize}
\item {Proveniência:(De \textunderscore cãimbas\textunderscore ?)}
\end{itemize}
Carpinteiro de carros?
\section{Cãimbo}
\begin{itemize}
\item {Grp. gram.:m.}
\end{itemize}
Vara, com que se apanha fruta; o mesmo que \textunderscore cambo\textunderscore . Cf. Filinto, VII, 148.
\section{Câimbra}
\begin{itemize}
\item {Grp. gram.:f.}
\end{itemize}
\begin{itemize}
\item {Grp. gram.:Pl.}
\end{itemize}
\begin{itemize}
\item {Proveniência:(Do norr. ant. \textunderscore klampi\textunderscore , segundo Körting)}
\end{itemize}
Breca; contracção involuntária do tecido muscular.
O mesmo que \textunderscore cambras\textunderscore .
\section{Câimbro}
\begin{itemize}
\item {Grp. gram.:m.}
\end{itemize}
\begin{itemize}
\item {Utilização:Bras}
\end{itemize}
Um par de quaesquer coisas unidas.
(Cp. \textunderscore caibro\textunderscore ^2)
\section{Caimento}
\begin{itemize}
\item {fónica:ca-i}
\end{itemize}
\begin{itemize}
\item {Grp. gram.:m.}
\end{itemize}
Acção de caír.
Decadência.
\section{Caimiri}
\begin{itemize}
\item {Grp. gram.:m.}
\end{itemize}
Espécie de macaco da América.
\section{Caimiteiro}
\begin{itemize}
\item {Grp. gram.:m.}
\end{itemize}
Árvore das Antilhas.
\section{Cainana}
\begin{itemize}
\item {Grp. gram.:f.}
\end{itemize}
\begin{itemize}
\item {Utilização:Bras}
\end{itemize}
Espécie de cobra, não venenosa.
\section{Cainca}
\begin{itemize}
\item {Grp. gram.:f.}
\end{itemize}
Planta chinchonácea do Brasil.
\section{Cainça}
\begin{itemize}
\item {Grp. gram.:f.}
\end{itemize}
\begin{itemize}
\item {Proveniência:(De \textunderscore cão\textunderscore )}
\end{itemize}
Ajuntamento de cães, canzoada.
\section{Cainçada}
\begin{itemize}
\item {Grp. gram.:f.}
\end{itemize}
\begin{itemize}
\item {Proveniência:(De \textunderscore cão\textunderscore )}
\end{itemize}
Ajuntamento de cães, canzoada.
\section{Cainçalha}
\begin{itemize}
\item {Grp. gram.:f.}
\end{itemize}
\begin{itemize}
\item {Proveniência:(De \textunderscore cão\textunderscore )}
\end{itemize}
Ajuntamento de cães, canzoada.
\section{Cainçar}
\begin{itemize}
\item {Grp. gram.:v. i.}
\end{itemize}
\begin{itemize}
\item {Utilização:Prov.}
\end{itemize}
\begin{itemize}
\item {Utilização:minh.}
\end{itemize}
\begin{itemize}
\item {Proveniência:(De \textunderscore cainça\textunderscore )}
\end{itemize}
Andar com o cio (a cadella), andar cachonda.
\section{Caincenta}
\begin{itemize}
\item {Grp. gram.:adj. f.}
\end{itemize}
\begin{itemize}
\item {Utilização:Prov.}
\end{itemize}
\begin{itemize}
\item {Utilização:minh.}
\end{itemize}
\begin{itemize}
\item {Proveniência:(De \textunderscore cainçar\textunderscore )}
\end{itemize}
Diz-se da cadella que anda cachonda. (Colhido em Barcelos)
\section{Cainhar}
\begin{itemize}
\item {fónica:ca-i}
\end{itemize}
\begin{itemize}
\item {Grp. gram.:v. i.}
\end{itemize}
\begin{itemize}
\item {Proveniência:(De \textunderscore caim\textunderscore ^1)}
\end{itemize}
Latir dolorosamenente.
\section{Cainheza}
\begin{itemize}
\item {fónica:ca-i}
\end{itemize}
\begin{itemize}
\item {Grp. gram.:f.}
\end{itemize}
\begin{itemize}
\item {Proveniência:(De \textunderscore caínho\textunderscore )}
\end{itemize}
Sovinice, avareza.
\section{Caínho}
\begin{itemize}
\item {Grp. gram.:adj.}
\end{itemize}
\begin{itemize}
\item {Grp. gram.:M.}
\end{itemize}
\begin{itemize}
\item {Proveniência:(De \textunderscore cão\textunderscore )}
\end{itemize}
Próprio de cão.
Avarento, mesquinho.
Casta de uva preta.
\section{Caínho-branco}
\begin{itemize}
\item {Grp. gram.:m.}
\end{itemize}
Casta de uva minhota.
\section{Cainiti}
\begin{itemize}
\item {Grp. gram.:m.}
\end{itemize}
Planta sapotácea da América.
\section{Cainito}
\begin{itemize}
\item {Grp. gram.:m.}
\end{itemize}
Árvore fructífera do Brasil, (\textunderscore chrysophyllum cainito\textunderscore , Lin.).
O mesmo que \textunderscore cainiti\textunderscore ?
\section{Caio}
\begin{itemize}
\item {Grp. gram.:m.}
\end{itemize}
(V.caiadela)
\section{Caiongo}
\begin{itemize}
\item {Grp. gram.:adj.}
\end{itemize}
\begin{itemize}
\item {Utilização:Bras}
\end{itemize}
Enfraquecido, avelhentado.
\section{Caiorro}
\begin{itemize}
\item {fónica:ô}
\end{itemize}
\begin{itemize}
\item {Grp. gram.:m.}
\end{itemize}
\begin{itemize}
\item {Utilização:Prov.}
\end{itemize}
\begin{itemize}
\item {Utilização:alent.}
\end{itemize}
Pião sem cabeça.
\section{Caipira}
\begin{itemize}
\item {Grp. gram.:m.}
\end{itemize}
\begin{itemize}
\item {Utilização:Prov.}
\end{itemize}
\begin{itemize}
\item {Utilização:minh.}
\end{itemize}
\begin{itemize}
\item {Utilização:Bras}
\end{itemize}
Nome depreciativo, com que os Realistas designavam cada um dos Constitucionaes, durante as lutas civis de 1828-1834.
Homem sovina, avarento.
Homem do mato, rústico, labrego.
(Alter. do tupi \textunderscore curupira\textunderscore )
\section{Caipirada}
\begin{itemize}
\item {Grp. gram.:f.}
\end{itemize}
\begin{itemize}
\item {Utilização:Bras}
\end{itemize}
Acto de caipira; rusticidade.
Grupo de caipiras; a classe dos caipiras.
\section{Caipora}
\begin{itemize}
\item {Grp. gram.:m.}
\end{itemize}
\begin{itemize}
\item {Utilização:Bras}
\end{itemize}
\begin{itemize}
\item {Grp. gram.:M.  e  f.}
\end{itemize}
\begin{itemize}
\item {Utilização:Bras}
\end{itemize}
\begin{itemize}
\item {Utilização:Fig.}
\end{itemize}
\begin{itemize}
\item {Grp. gram.:Adj.}
\end{itemize}
Fogo fátuo.
Ente fantástico que, segundo a crendice popular, percorre as estradas, tornando infeliz quem encontra.
Pessôa, cuja presença prejudica o bom andamento dos negócios de outrem.
Infeliz.
(Corr. do tupi \textunderscore caapora\textunderscore )
\section{Caiporismo}
\begin{itemize}
\item {Grp. gram.:m.}
\end{itemize}
\begin{itemize}
\item {Utilização:Bras}
\end{itemize}
\begin{itemize}
\item {Proveniência:(De \textunderscore caipora\textunderscore )}
\end{itemize}
Má sorte, infelicidade.
\section{Caíque}
\begin{itemize}
\item {Grp. gram.:m.}
\end{itemize}
Pequeno navio de dois mastros, com velas triangulares.
Nome de varias outras embarcações.
(Turc. \textunderscore kaik\textunderscore )
\section{Cair}
\begin{itemize}
\item {Grp. gram.:v. i.}
\end{itemize}
\begin{itemize}
\item {Proveniência:(Do lat. \textunderscore cadere\textunderscore )}
\end{itemize}
Ir abaixo, em virtude do próprio pêso, ou por têr perdido o equilibrio.
Estar sobranceiro; pender.
Acontecer: \textunderscore a Páscoa êste anno caiu a 17 de Março\textunderscore .
Sujeitar-se; incorrer: \textunderscore cair sob a alçada da lei\textunderscore .
Sêr surprehendido, enganado.
Descer: \textunderscore caiu da elevada situação que occupava\textunderscore .
\section{Cãira}
\begin{itemize}
\item {Grp. gram.:f.}
\end{itemize}
Antiga medida portuguesa, correspondente a três quartas de alqueire.
(Provavelmente, relaciona-se com \textunderscore alqueire\textunderscore )
\section{Cairara}
\begin{itemize}
\item {Grp. gram.:adj.}
\end{itemize}
\begin{itemize}
\item {Utilização:Bras. do N}
\end{itemize}
\begin{itemize}
\item {Grp. gram.:M.}
\end{itemize}
Muito grande.
Variedade de macaco do Amazonas.
\section{Cairel}
\begin{itemize}
\item {Grp. gram.:f.}
\end{itemize}
Fita, galão, para debruar; debrum.
Resguardo.
Borda; beira: \textunderscore no cairel do abysmo\textunderscore .
(Alt. de \textunderscore quadrela\textunderscore )
\section{Cairela}
\begin{itemize}
\item {Grp. gram.:f.}
\end{itemize}
\begin{itemize}
\item {Utilização:Prov.}
\end{itemize}
\begin{itemize}
\item {Utilização:alent.}
\end{itemize}
O mesmo que \textunderscore coirela\textunderscore .
\section{Cairelar}
\begin{itemize}
\item {Grp. gram.:v. t.}
\end{itemize}
Pôr cairel a; debruar.
(Cp. \textunderscore acairelar\textunderscore )
\section{Cairi}
\begin{itemize}
\item {Grp. gram.:m.}
\end{itemize}
\begin{itemize}
\item {Utilização:Bras}
\end{itemize}
Guisado de gallinha com pimenta, pevide de abóbora, etc.
\section{Cairina}
\begin{itemize}
\item {Grp. gram.:f.}
\end{itemize}
Medicamento enérgico antifebril.
\section{Cairiris}
\begin{itemize}
\item {Grp. gram.:m. pl.}
\end{itemize}
Numerosa tríbo de Índios do Brasil, que viviam na cordilheira Borborema, quando os Portugueses aportaram á América.
\section{Cairo}
\begin{itemize}
\item {Grp. gram.:m.}
\end{itemize}
\begin{itemize}
\item {Proveniência:(Do malab. \textunderscore kayar\textunderscore , ou do tamul \textunderscore cayiru\textunderscore )}
\end{itemize}
Filamentos de noz de coco, resistentes, próprios para cordas.
Cordel, que prende os testicos da serra.
\section{Cãiro}
\begin{itemize}
\item {Grp. gram.:m.}
\end{itemize}
\begin{itemize}
\item {Utilização:Prov.}
\end{itemize}
\begin{itemize}
\item {Utilização:trasm.}
\end{itemize}
\begin{itemize}
\item {Proveniência:(Do lat. \textunderscore canarius\textunderscore , de \textunderscore canis\textunderscore , cão)}
\end{itemize}
Dente canino, colmilho.
\section{Cairuá}
\begin{itemize}
\item {Grp. gram.:m.}
\end{itemize}
Ave do Brasil.
\section{Cais}
\begin{itemize}
\item {Grp. gram.:m.}
\end{itemize}
\begin{itemize}
\item {Proveniência:(Do b. lat. \textunderscore caium\textunderscore . Cp. câmbrico \textunderscore cale\textunderscore )}
\end{itemize}
Elevação de terra, ordinariamente lageada e murada, que á beira de um rio ou de um pôrto, é destinada ao embarque ou desembarque de pessôas ou mercadorias.
Parte das estações de caminhos de ferro, em que se descarregam mercadorias, e se apeiam ou embarcam os passageiros.
\section{Caité}
\begin{itemize}
\item {Grp. gram.:m.}
\end{itemize}
Planta medicinal do Brasil.
\section{Caitetu}
\begin{itemize}
\item {Grp. gram.:m.}
\end{itemize}
(V.caititu)
\section{Caititu}
\begin{itemize}
\item {Grp. gram.:m.}
\end{itemize}
\begin{itemize}
\item {Utilização:Bras}
\end{itemize}
Mammífero pachiderme, indígena da América.
Engenho para fazer farinha, movido á mão.
\section{Caíva}
\begin{itemize}
\item {Grp. gram.:f.}
\end{itemize}
\begin{itemize}
\item {Utilização:Bras}
\end{itemize}
Mato, cujo terreno é impróprio para cultura.
(Do tupi \textunderscore caa\textunderscore  + \textunderscore aiva\textunderscore )
\section{Caixa}
\begin{itemize}
\item {Grp. gram.:f.}
\end{itemize}
\begin{itemize}
\item {Utilização:Constr.}
\end{itemize}
\begin{itemize}
\item {Utilização:T. de calceteiro}
\end{itemize}
\begin{itemize}
\item {Grp. gram.:M.}
\end{itemize}
\begin{itemize}
\item {Proveniência:(Do lat. \textunderscore capsa\textunderscore )}
\end{itemize}
Arca, móvel quadrilongo de madeira, para guardar ou transportar fato, mercadorias, etc.
Boceta.
Estojo.
Cofre forte, em que os banqueiros, capitalistas, negociantes, etc., guardam dinheiro e documentos importantes.
O vão, em que se acha collocada uma escada, dentro de casa.
A parte de um terreno, que vai sêr calcetada.
Parte do theatro em que estão os camarins dos actores.
Taboleiro, com subdivisões, para distribuição e guarda de caracteres typográphicos.
Qualquer objecto ou peça, que contém ou resguarda outra.
\textunderscore Caixa forte\textunderscore , ou \textunderscore caixa clara\textunderscore , ou \textunderscore caixa de guerra\textunderscore , ou simplesmente \textunderscore caixa\textunderscore , tambor de cylindro baixo, usado especialmente nos regimentos de infantaria.
\textunderscore Caixa tarola\textunderscore , o mesmo que \textunderscore tarola\textunderscore .
\textunderscore Caixa chata\textunderscore , o mesmo que \textunderscore tarola\textunderscore .
\textunderscore Caixa de rufo\textunderscore , tambor de cylindro mais alto que o da \textunderscore caixa forte\textunderscore .
\textunderscore Caixa de resonância\textunderscore , o corpo principal da maior parte dos instrumentos de corda.
\textunderscore Caixa de música\textunderscore , instrumento mecânico de música, que consiste numa caixa com um cylindro, crivado de pontas, as quaes, quando o cylindro gira, ferem uma espécie de teclado, composto de finas lâminas de aço.
\textunderscore Caixa de ar\textunderscore , espaço entre o solo e o vigamento de um edifício.
\textunderscore Caixa-de-água\textunderscore , o mesmo que \textunderscore arca-de-água\textunderscore , ou \textunderscore mãe-de água\textunderscore . Cf. B. Pereira, vb. \textunderscore cataracta\textunderscore .
\textunderscore Caixa económica\textunderscore , estabelecimento público, em que se recebem em depósito economias dos depositantes, pagando-se a êstes determinado juro.
Aquelle que, numa casa commercial, tem a seu cargo cobranças e pagamentos.
Livro, em que se registam entradas e saídas de fundos.
\section{Caixamarim}
\begin{itemize}
\item {Grp. gram.:m.}
\end{itemize}
\begin{itemize}
\item {Proveniência:(De \textunderscore caixa\textunderscore  + \textunderscore marinho\textunderscore )}
\end{itemize}
Pequena embarcação costeira.
\section{Caixanas}
\begin{itemize}
\item {Grp. gram.:m. pl.}
\end{itemize}
Tríbo do alto Amazonas.
\section{Caixão}
\begin{itemize}
\item {Grp. gram.:m.}
\end{itemize}
\begin{itemize}
\item {Utilização:Náut.}
\end{itemize}
\begin{itemize}
\item {Utilização:Prov.}
\end{itemize}
Caixa grande.
Espaço entre a almeida e a cabeça do leme.
Féretro, caixa abaulada, para encerrar defuntos.
Espécie de armadilha para caça.
\section{Caixão}
\begin{itemize}
\item {Grp. gram.:m.}
\end{itemize}
Árvore da ilha de San-Thomé.
\section{Caixaria}
\begin{itemize}
\item {Grp. gram.:f.}
\end{itemize}
\begin{itemize}
\item {Proveniência:(De \textunderscore caixa\textunderscore )}
\end{itemize}
Grande porção de caixas.
Profissão de caixeiro.
\section{Caixeirada}
\begin{itemize}
\item {Grp. gram.:f.}
\end{itemize}
\begin{itemize}
\item {Utilização:Deprec.}
\end{itemize}
Classe dos caixeiros.
Multidão de caixeiros.
\section{Caixeiria}
\begin{itemize}
\item {Grp. gram.:f.}
\end{itemize}
(V.caixaria)
\section{Caixeiro}
\begin{itemize}
\item {Grp. gram.:m.}
\end{itemize}
\begin{itemize}
\item {Proveniência:(De \textunderscore caixa\textunderscore )}
\end{itemize}
Aquelle que faz caixas.
Aquelle que, nas casas commerciaes, está encarregado da venda a retalho.
O encarregado de uma caixa commercial; guarda-livros; o caixa.
\section{Caixela}
\begin{itemize}
\item {Grp. gram.:f.}
\end{itemize}
\begin{itemize}
\item {Utilização:Prov.}
\end{itemize}
\begin{itemize}
\item {Utilização:beir.}
\end{itemize}
\begin{itemize}
\item {Proveniência:(Do lat. \textunderscore capsella\textunderscore )}
\end{itemize}
Caixa das hóstias.
\section{Caixella}
\begin{itemize}
\item {Grp. gram.:f.}
\end{itemize}
\begin{itemize}
\item {Utilização:Prov.}
\end{itemize}
\begin{itemize}
\item {Utilização:beir.}
\end{itemize}
\begin{itemize}
\item {Proveniência:(Do lat. \textunderscore capsella\textunderscore )}
\end{itemize}
Caixa das hóstias.
\section{Caixeta}
\begin{itemize}
\item {fónica:xê}
\end{itemize}
\begin{itemize}
\item {Grp. gram.:f.}
\end{itemize}
\begin{itemize}
\item {Utilização:Bras}
\end{itemize}
Caixa pequena.
Árvore silvestre.
\section{Caixilharia}
\begin{itemize}
\item {Grp. gram.:f.}
\end{itemize}
Conjunto de caixilhos.
\section{Caixilho}
\begin{itemize}
\item {Grp. gram.:m.}
\end{itemize}
\begin{itemize}
\item {Grp. gram.:Pl.}
\end{itemize}
\begin{itemize}
\item {Utilização:Gír.}
\end{itemize}
\begin{itemize}
\item {Proveniência:(De \textunderscore caixa\textunderscore )}
\end{itemize}
Moldura de vidros.
Moldura para quadro ou para estampa.
Moldura.
Olhos.
\section{Caixilho-focal}
\begin{itemize}
\item {Grp. gram.:m.}
\end{itemize}
O caixilho das matrizes photográphicas, correspondente ao fr. \textunderscore chassis\textunderscore .
\section{Caixola}
\begin{itemize}
\item {Grp. gram.:f.}
\end{itemize}
\begin{itemize}
\item {Utilização:Bras}
\end{itemize}
Pequena caixa.
\section{Caixota}
\begin{itemize}
\item {Grp. gram.:f.}
\end{itemize}
\begin{itemize}
\item {Utilização:Bras}
\end{itemize}
Caixa pequena.
\section{Caixotão}
\begin{itemize}
\item {Grp. gram.:m.}
\end{itemize}
\begin{itemize}
\item {Proveniência:(De \textunderscore caixote\textunderscore )}
\end{itemize}
Caixote grande.
Tecto côncavo e quadrado, que abrange relevos e outras decorações.
\section{Caixotaria}
\begin{itemize}
\item {Grp. gram.:f.}
\end{itemize}
Estabelecimento, onde se fazem ou se vendem caixotes.
\section{Caixote}
\begin{itemize}
\item {Grp. gram.:m.}
\end{itemize}
\begin{itemize}
\item {Utilização:Marcen.}
\end{itemize}
Caixa pequena, tôsca.
A parte superior de um guarda-vestidos, antes de guarnecida.
\section{Caixoteiro}
\begin{itemize}
\item {Grp. gram.:m.}
\end{itemize}
Aquelle que faz caixotes.
\section{Caixotim}
\begin{itemize}
\item {Grp. gram.:m.}
\end{itemize}
\begin{itemize}
\item {Proveniência:(De \textunderscore caixote\textunderscore )}
\end{itemize}
Cada uma das divisões da caixa typográphica.
\section{Cajá}
\begin{itemize}
\item {Grp. gram.:m.}
\end{itemize}
Fruta da cajazeira.
A cajazeira.
\section{Cajadada}
\begin{itemize}
\item {Grp. gram.:f.}
\end{itemize}
Pancada com cajado.
\section{Cajadela}
\begin{itemize}
\item {Grp. gram.:f.}
\end{itemize}
O mesmo que \textunderscore cajadada\textunderscore .
\section{Cajadil}
\begin{itemize}
\item {Grp. gram.:m.}
\end{itemize}
Árvore angolense, sempre verde, de fôlhas simples e flôres axillares, muito miúdas.
\section{Cajado}
\begin{itemize}
\item {Grp. gram.:m.}
\end{itemize}
\begin{itemize}
\item {Utilização:Fig.}
\end{itemize}
Bordão de pastor, com a extremidade superior arqueada.
Bastão.
Amparo.
\section{Cajado-de-San-José}
\begin{itemize}
\item {Grp. gram.:m.}
\end{itemize}
O mesmo que \textunderscore pau-de-San-José\textunderscore .
\section{Cajaeiro}
\begin{itemize}
\item {Grp. gram.:m.}
\end{itemize}
O mesmo que \textunderscore cajazeiro\textunderscore . Cf. Ed. Magalhães, \textunderscore Hyg. Alim.\textunderscore , I, 345.
\section{Cajalala}
\begin{itemize}
\item {Grp. gram.:f.}
\end{itemize}
Arbusto angolense, de caule tortuoso, fôlhas simples, serreadas, e flôres miúdas, inodoras.
\section{Cajamanga}
\begin{itemize}
\item {Grp. gram.:m.}
\end{itemize}
Planta ornamental, de fruto comestível.
\section{Cajano}
\begin{itemize}
\item {Grp. gram.:m.}
\end{itemize}
Planta papilionácea.
\section{Cajão}
\begin{itemize}
\item {Grp. gram.:m.}
\end{itemize}
\begin{itemize}
\item {Utilização:Ant.}
\end{itemize}
Damno; desastre.
(Cp. \textunderscore caso\textunderscore )
\section{Cajati}
\begin{itemize}
\item {Grp. gram.:m.}
\end{itemize}
Arbusto do Brasil.
\section{Cajazeira}
\begin{itemize}
\item {Grp. gram.:f.}
\end{itemize}
\begin{itemize}
\item {Utilização:Bras}
\end{itemize}
Árvore anacardiácea, (\textunderscore spondias lutea\textunderscore , Lin.), chamada \textunderscore munguengue\textunderscore  em Angola; \textunderscore amboló\textunderscore  em Gôa; \textunderscore cajá\textunderscore , \textunderscore acajá\textunderscore  e \textunderscore taparebá\textunderscore  em vários pontos do Brasil.
\section{Cajazeiro}
\begin{itemize}
\item {Grp. gram.:m.}
\end{itemize}
O mesmo que \textunderscore cajazeira\textunderscore .
\section{Cajepute}
\begin{itemize}
\item {Grp. gram.:m.}
\end{itemize}
\begin{itemize}
\item {Proveniência:(Do mal. \textunderscore kain-púti\textunderscore )}
\end{itemize}
Planta myrtácea.
Óleo verde e medicinal, que se extrai das fôlhas dessa planta.
\section{Cajetilha}
\begin{itemize}
\item {Grp. gram.:m.}
\end{itemize}
\begin{itemize}
\item {Utilização:Bras. do S}
\end{itemize}
\begin{itemize}
\item {Proveniência:(De \textunderscore cajeta\textunderscore , que na República Argentina significa \textunderscore janota\textunderscore )}
\end{itemize}
Janota da cidade; peralvilho.
\section{Cajinga}
\begin{itemize}
\item {Grp. gram.:m.}
\end{itemize}
Barrete, que alguns sertanejos de Angola fabricam com mabella. Cf. Capello e Ivens, I, 49.
\section{Cajitas}
\begin{itemize}
\item {Grp. gram.:m.}
\end{itemize}
\begin{itemize}
\item {Utilização:Prov.}
\end{itemize}
\begin{itemize}
\item {Utilização:trasm.}
\end{itemize}
O mesmo que \textunderscore canejo\textunderscore ^2.
\section{Cajo}
\begin{itemize}
\item {Grp. gram.:m.}
\end{itemize}
\begin{itemize}
\item {Utilização:Ant.}
\end{itemize}
Acaso? cajão?:«\textunderscore Ponho por cajo que alguem vem como eu vim agora.\textunderscore »G. Vicente, \textunderscore Inês Pereira\textunderscore .
\section{Cajom}
\begin{itemize}
\item {Grp. gram.:m.}
\end{itemize}
(V.cajão)
\section{Caju}
\textunderscore m.\textunderscore  (e der.)
O mesmo que \textunderscore acaju\textunderscore , etc.
Cajueiro.
O mesmo que \textunderscore cajueiro\textunderscore .
(Do tupi)
\section{Cajuada}
\begin{itemize}
\item {Grp. gram.:f.}
\end{itemize}
\begin{itemize}
\item {Utilização:Bras}
\end{itemize}
\begin{itemize}
\item {Proveniência:(De \textunderscore caju\textunderscore )}
\end{itemize}
Bebida refrigerante, feita com sumo de caju, água e açúcar.
\section{Cajuí}
\begin{itemize}
\item {Grp. gram.:m.}
\end{itemize}
\begin{itemize}
\item {Utilização:Bras. do N}
\end{itemize}
Variedade de caju.
\section{Cajúri}
\begin{itemize}
\item {Grp. gram.:m.}
\end{itemize}
Árvore intertropical, (\textunderscore phoenix silvestris\textunderscore ).
\section{Cajurubeba}
\begin{itemize}
\item {Grp. gram.:f.}
\end{itemize}
\begin{itemize}
\item {Utilização:Bras}
\end{itemize}
Planta medicinal.
\section{Cajuso}
\begin{itemize}
\item {Grp. gram.:adv.}
\end{itemize}
\begin{itemize}
\item {Utilização:Ant.}
\end{itemize}
Por acaso.
Eventualmente.
(Cp. \textunderscore cajão\textunderscore )
\section{Cajuzeiro}
\begin{itemize}
\item {Grp. gram.:m.}
\end{itemize}
O mesmo que \textunderscore cajueiro\textunderscore .
\section{Cal}
\begin{itemize}
\item {Grp. gram.:f.}
\end{itemize}
\begin{itemize}
\item {Proveniência:(Do lat. \textunderscore calx\textunderscore )}
\end{itemize}
Protóxydo de cálcio.
Substancia que, combinada com o ácido carbónico, fórma a pedra e o mármore, e, combinada com o ácido sulfúrico, fórma o gesso.
\section{Cal}
\begin{itemize}
\item {Grp. gram.:f.}
\end{itemize}
\begin{itemize}
\item {Utilização:T. de Aveiro}
\end{itemize}
O mesmo que \textunderscore cale\textunderscore , braço da ria.
\section{Cala}
\begin{itemize}
\item {Grp. gram.:f.}
\end{itemize}
\begin{itemize}
\item {Proveniência:(De \textunderscore calar\textunderscore )}
\end{itemize}
Pequena enseada entre rochedos.
Abertura em frutos ou outros objectos, para os conhecer interiormente.
Calada, silêncio:«\textunderscore por bom mo venderam e eu o tomei á cala de sua bôa fama.\textunderscore »Camões, \textunderscore Seleuco\textunderscore , pról.
\section{Cala}
\begin{itemize}
\item {Grp. gram.:f.}
\end{itemize}
Corda de esparto, para alar ou arrastar certas redes, fixadas nos calões.
\section{Cala}
\begin{itemize}
\item {Grp. gram.:f.}
\end{itemize}
\begin{itemize}
\item {Utilização:T. de Aveiro}
\end{itemize}
(V. \textunderscore cal\textunderscore ^2)
\section{Çalá}
\begin{itemize}
\item {Grp. gram.:m.}
\end{itemize}
\begin{itemize}
\item {Proveniência:(Do ár. \textunderscore çalá\textunderscore )}
\end{itemize}
O mesmo que \textunderscore çalamaleque\textunderscore .
\section{Calabaça}
\begin{itemize}
\item {Grp. gram.:f.}
\end{itemize}
(V. \textunderscore cabaça\textunderscore ^1)
\section{Calabaceira}
\textunderscore f.\textunderscore  (\textunderscore T. de San-Thomé e da Guiné\textunderscore )
O mesmo que \textunderscore embondeiro\textunderscore .
(Cp. \textunderscore cabaceira\textunderscore )
\section{Calabarina}
\begin{itemize}
\item {Grp. gram.:f.}
\end{itemize}
Producto pharmacêutico, obtido do extracto da fava do Calabar.
\section{Calabarino}
\begin{itemize}
\item {Grp. gram.:m.}
\end{itemize}
(V.calabarina)
\section{Calaboço}
\begin{itemize}
\item {fónica:bô}
\end{itemize}
\begin{itemize}
\item {Grp. gram.:m.}
\end{itemize}
O mesmo que \textunderscore calaboiço\textunderscore .
\section{Calaboiço}
\begin{itemize}
\item {Grp. gram.:m.}
\end{itemize}
Prisão subterrânea.
Cárcere.
Lugar de prisão provisória ou preventiva.
Lugar sombrio.
(Cast. \textunderscore calabozo\textunderscore )
\section{Calabouço}
\begin{itemize}
\item {Grp. gram.:m.}
\end{itemize}
Prisão subterrânea.
Cárcere.
Lugar de prisão provisória ou preventiva.
Lugar sombrio.
(Cast. \textunderscore calabozo\textunderscore )
\section{Calabre}
\begin{itemize}
\item {Grp. gram.:m.}
\end{itemize}
Corda grossa, cabo.
Amarra.
\section{Calabreada}
\begin{itemize}
\item {Grp. gram.:f.}
\end{itemize}
Acção de \textunderscore calabrear\textunderscore .
\section{Calabreadura}
\begin{itemize}
\item {Grp. gram.:f.}
\end{itemize}
O mesmo que \textunderscore calabreada\textunderscore .
\section{Calabrear}
\begin{itemize}
\item {Grp. gram.:v. t.}
\end{itemize}
\begin{itemize}
\item {Utilização:Des.}
\end{itemize}
Adubar.
Adulterar (vinhos).
Preparar.
Confundir: \textunderscore calabrear milho e trigo\textunderscore .
Perverter.
Governar:«\textunderscore calabrear a vida\textunderscore ». \textunderscore Aulegrafia\textunderscore , 161.
\section{Calabrês}
\begin{itemize}
\item {Grp. gram.:adj.}
\end{itemize}
\begin{itemize}
\item {Grp. gram.:M.}
\end{itemize}
Relativo á Calábria.
Próprio dos salteadores da Calábria.
Habitante da Calábria.
\section{Calabrete}
\begin{itemize}
\item {fónica:brê}
\end{itemize}
\begin{itemize}
\item {Grp. gram.:m.}
\end{itemize}
O mesmo que \textunderscore calabrote\textunderscore .
\section{Calábria}
\begin{itemize}
\item {Grp. gram.:f.}
\end{itemize}
Espécie de videira brava, brasileira.
\section{Calábrico}
\begin{itemize}
\item {Grp. gram.:adj.}
\end{itemize}
\begin{itemize}
\item {Proveniência:(Lat. \textunderscore calabricus\textunderscore )}
\end{itemize}
Relativo á Calábria.
\section{Calabrote}
\begin{itemize}
\item {Grp. gram.:m.}
\end{itemize}
Calabre pouco grosso.
\section{Calabroteado}
\begin{itemize}
\item {Grp. gram.:adj.}
\end{itemize}
\begin{itemize}
\item {Utilização:Náut.}
\end{itemize}
\begin{itemize}
\item {Proveniência:(De \textunderscore calabrote\textunderscore )}
\end{itemize}
Diz-se do cabo, formado por nove cordões, ou por três cabos de três cordões cochados.
\section{Calaburço}
\begin{itemize}
\item {Grp. gram.:m.}
\end{itemize}
\begin{itemize}
\item {Utilização:Prov.}
\end{itemize}
\begin{itemize}
\item {Utilização:alent.}
\end{itemize}
Bilha ou infusa, com a asa e a bôca ou gargalo partidos, mas com o bôjo inteiro.
\section{Calaburo}
\begin{itemize}
\item {Grp. gram.:m.}
\end{itemize}
Grande árvore liliácea da ilha de San-Domingos.
\section{Calaça}
\begin{itemize}
\item {Grp. gram.:f.}
\end{itemize}
\begin{itemize}
\item {Utilização:Ant.}
\end{itemize}
Foro, que consistia numa porção de carne.
\section{Calaça}
\begin{itemize}
\item {Grp. gram.:f.}
\end{itemize}
\begin{itemize}
\item {Utilização:Prov.}
\end{itemize}
\begin{itemize}
\item {Grp. gram.:M.}
\end{itemize}
\begin{itemize}
\item {Utilização:Prov.}
\end{itemize}
\begin{itemize}
\item {Utilização:minh.}
\end{itemize}
Preguiça.
Homem preguiçoso, calaceiro.
\section{Calacala}
\begin{itemize}
\item {Grp. gram.:f.}
\end{itemize}
Árvore do Congo. O mesmo que \textunderscore calaguala\textunderscore ?
\section{Calaçaria}
\begin{itemize}
\item {Grp. gram.:f.}
\end{itemize}
Qualidade ou vida de calaceiro; ociosidade.
\section{Calacear}
\begin{itemize}
\item {Grp. gram.:v. i.}
\end{itemize}
\begin{itemize}
\item {Proveniência:(De \textunderscore calaça\textunderscore ^2)}
\end{itemize}
Mandriar; viver na ociosidade, ou á custa de outrem.
\section{Calaceirar}
\begin{itemize}
\item {Grp. gram.:v. i.}
\end{itemize}
(V.calacear)
\section{Calaceirice}
\begin{itemize}
\item {Grp. gram.:f.}
\end{itemize}
O mesmo ou melhor que \textunderscore calacice\textunderscore . Cf. Camillo, \textunderscore Corja\textunderscore , c. X.
\section{Calaceiro}
\begin{itemize}
\item {Grp. gram.:m.}
\end{itemize}
\begin{itemize}
\item {Proveniência:(De \textunderscore calaça\textunderscore ^2)}
\end{itemize}
Mandrião; vadio.
Homem guloso.
Parasito.
Frascário, femeeiro:«\textunderscore maridos calaceiros de criadas\textunderscore ». F. Manuel, \textunderscore Carta de Guia\textunderscore , 163.
\section{Calacice}
\begin{itemize}
\item {Grp. gram.:f.}
\end{itemize}
Qualidade de calaceiro.
\section{Calacorda}
\begin{itemize}
\item {fónica:cá}
\end{itemize}
\begin{itemize}
\item {Grp. gram.:f.}
\end{itemize}
\begin{itemize}
\item {Utilização:Ant.}
\end{itemize}
\begin{itemize}
\item {Proveniência:(De \textunderscore calar\textunderscore  + \textunderscore corda\textunderscore )}
\end{itemize}
Toque de tambor, feito como sinal, para chegar a corda do morrão ao mosquete.
\section{Calacre}
\begin{itemize}
\item {Grp. gram.:m.}
\end{itemize}
\begin{itemize}
\item {Utilização:Prov.}
\end{itemize}
\begin{itemize}
\item {Utilização:trasm.}
\end{itemize}
Dívida.
Embaraço.
\section{Calada}
\begin{itemize}
\item {Grp. gram.:f.}
\end{itemize}
\begin{itemize}
\item {Proveniência:(De \textunderscore calar\textunderscore )}
\end{itemize}
Cessação de ruído, silêncio: \textunderscore na calada da noite\textunderscore .
\section{Caladamente}
\begin{itemize}
\item {Grp. gram.:adv.}
\end{itemize}
\begin{itemize}
\item {Proveniência:(De \textunderscore calado\textunderscore )}
\end{itemize}
Silenciosamente.
Occultamente.
\section{Caladáris}
\begin{itemize}
\item {Grp. gram.:m.}
\end{itemize}
Pano de algodão, com listas pretas e encarnadas, procedente da Índia.
\section{Caladião}
\begin{itemize}
\item {Grp. gram.:m.}
\end{itemize}
O mesmo que \textunderscore caládio\textunderscore .
\section{Caladigão}
\begin{itemize}
\item {Grp. gram.:m.}
\end{itemize}
Tribunal ou sala de audiência, entre os Chineses. Cf. \textunderscore Peregrinação\textunderscore , CIII.
\section{Caládio}
\begin{itemize}
\item {Grp. gram.:m.}
\end{itemize}
Gênero de plantas aráceas do Brasil e das Antilhas.
\section{Calado}
\begin{itemize}
\item {Grp. gram.:m.}
\end{itemize}
Dialecto, falado nas montanhas que cercam Dili, em Timor.
\section{Calado}
\begin{itemize}
\item {Grp. gram.:m.}
\end{itemize}
\begin{itemize}
\item {Utilização:Náut.}
\end{itemize}
Distância vertical, da quilha do navio á linha de fluctuação.
Espaço, occupado pelo navio dentro de água.
(Cp. \textunderscore calar\textunderscore )
\section{Calado}
\begin{itemize}
\item {Grp. gram.:adj.}
\end{itemize}
\begin{itemize}
\item {Proveniência:(De \textunderscore calar\textunderscore )}
\end{itemize}
Que não diz nada.
Silencioso.
\section{Calador}
\begin{itemize}
\item {Grp. gram.:m.}
\end{itemize}
\begin{itemize}
\item {Proveniência:(De \textunderscore cala\textunderscore ^2)}
\end{itemize}
Tripulante, que deita a rede ao mar.
Aquelle que vai arreando as cordas ou cabos da rede de cercar e alar.
\section{Caladura}
\begin{itemize}
\item {Grp. gram.:f.}
\end{itemize}
\begin{itemize}
\item {Proveniência:(De \textunderscore calar\textunderscore )}
\end{itemize}
Acção de calar.
O mesmo que \textunderscore cala\textunderscore ^1.
\section{Calafate}
\begin{itemize}
\item {Grp. gram.:m.}
\end{itemize}
\begin{itemize}
\item {Utilização:Bras}
\end{itemize}
\begin{itemize}
\item {Proveniência:(De \textunderscore calafetar\textunderscore )}
\end{itemize}
Aquelle que se occupa em calafetar embarcações.
Pássaro da Austrália.
\section{Calafetação}
\begin{itemize}
\item {Grp. gram.:f.}
\end{itemize}
O mesmo que \textunderscore calafetagem\textunderscore .
\section{Calafetador}
\begin{itemize}
\item {Grp. gram.:m.}
\end{itemize}
\begin{itemize}
\item {Proveniência:(De \textunderscore calafetar\textunderscore )}
\end{itemize}
Instrumento, com que se calafeta.
\section{Calafetagem}
\begin{itemize}
\item {Grp. gram.:f.}
\end{itemize}
Acção de \textunderscore calafetar\textunderscore .
Estôpa ou outra substância, com que se calafeta.
\section{Calafetamento}
\begin{itemize}
\item {Grp. gram.:m.}
\end{itemize}
(V.calafetagem)
\section{Calafetar}
\begin{itemize}
\item {Grp. gram.:v. t.}
\end{itemize}
\begin{itemize}
\item {Proveniência:(Do ár. \textunderscore kalafa\textunderscore )}
\end{itemize}
Tapar com estôpa as fendas, junturas ou buracos de (navios).
Tapar com estôpa ou outra substância as junturas de aduelas e tampos de (pipas, tonéis, etc.).
Tapar com trapos, papéis, etc., a abertura de (quartos, salas), para impedir a entrada do vento ou do ar.
\section{Calafetear}
\begin{itemize}
\item {Grp. gram.:v. t.}
\end{itemize}
(V.calafetar)
\section{Calafeto}
\begin{itemize}
\item {fónica:fê}
\end{itemize}
\begin{itemize}
\item {Grp. gram.:m.}
\end{itemize}
Acção de calafetar.
Substância, com que se calafeta.
Resguardo contra o frio.
\section{Calafrio}
\begin{itemize}
\item {Grp. gram.:m.}
\end{itemize}
(V.calefrio)
\section{Calagem}
\begin{itemize}
\item {Grp. gram.:f.}
\end{itemize}
Mistura de cal na terra, para certas culturas.
\section{Calagoiça}
\begin{itemize}
\item {Grp. gram.:f.}
\end{itemize}
\begin{itemize}
\item {Utilização:Prov.}
\end{itemize}
\begin{itemize}
\item {Utilização:trasm.}
\end{itemize}
Foice roçadoira, de cabo curto.
\section{Calagoiçada}
\begin{itemize}
\item {Grp. gram.:f.}
\end{itemize}
Pancada com \textunderscore calagoiça\textunderscore  ou \textunderscore calagoiço\textunderscore .
\section{Calagoiço}
\begin{itemize}
\item {Grp. gram.:m.}
\end{itemize}
\begin{itemize}
\item {Utilização:Prov.}
\end{itemize}
\begin{itemize}
\item {Utilização:trasm.}
\end{itemize}
Instrumento análogo á calagoiça, mas de volta mais fechada e de cabo mais longo.
\section{Calagouça}
\begin{itemize}
\item {Grp. gram.:f.}
\end{itemize}
\begin{itemize}
\item {Utilização:Prov.}
\end{itemize}
\begin{itemize}
\item {Utilização:trasm.}
\end{itemize}
Foice roçadoira, de cabo curto.
\section{Calagouçada}
\begin{itemize}
\item {Grp. gram.:f.}
\end{itemize}
Pancada com \textunderscore calagoiça\textunderscore  ou \textunderscore calagoiço\textunderscore .
\section{Calagouço}
\begin{itemize}
\item {Grp. gram.:m.}
\end{itemize}
\begin{itemize}
\item {Utilização:Prov.}
\end{itemize}
\begin{itemize}
\item {Utilização:trasm.}
\end{itemize}
Instrumento análogo á calagoiça, mas de volta mais fechada e de cabo mais longo.
\section{Calaguala}
\begin{itemize}
\item {Grp. gram.:f.}
\end{itemize}
Fêto americano, medicinal.
\section{Calaim}
\begin{itemize}
\item {Grp. gram.:m.}
\end{itemize}
\begin{itemize}
\item {Proveniência:(Do ár. \textunderscore calaí\textunderscore )}
\end{itemize}
Estanho indiano.
\section{Calajar}
\begin{itemize}
\item {Grp. gram.:m.}
\end{itemize}
Árvore indiana, muito applicada em construcções, (\textunderscore strichnos nux vomica\textunderscore ).
\section{Calala}
\begin{itemize}
\item {Grp. gram.:m.}
\end{itemize}
Chefe guerreiro em algumas tríbos de Angola.
\section{Calalanza}
\begin{itemize}
\item {Grp. gram.:f.}
\end{itemize}
Árvore angolense, muito importante pela qualidade da sua madeira.
\section{Calalu}
\begin{itemize}
\item {Grp. gram.:m.}
\end{itemize}
Planta malvácea, (\textunderscore hibiscus esculentus\textunderscore ).
\section{Calaluz}
\begin{itemize}
\item {Grp. gram.:m.}
\end{itemize}
Pequena embarcação indiana.
\section{Calamaço}
\begin{itemize}
\item {Grp. gram.:m.}
\end{itemize}
\begin{itemize}
\item {Utilização:Des.}
\end{itemize}
Tecido lustroso de lan.
Durante:«\textunderscore deixo um vestido de calamaço negro...\textunderscore »(De um testamento de 1693)
(Cp. fr. \textunderscore calmande\textunderscore )
\section{Çalamaleque}
\begin{itemize}
\item {Grp. gram.:m.}
\end{itemize}
\begin{itemize}
\item {Utilização:pop.}
\end{itemize}
\begin{itemize}
\item {Utilização:Fig.}
\end{itemize}
\begin{itemize}
\item {Proveniência:(Do ár. \textunderscore salam\textunderscore  + \textunderscore haleik\textunderscore )}
\end{itemize}
Saudação, entre os Turcos.
Mesura exaggerada.
Grande reverência; cumprimentos affectados.
\section{Calamão}
\begin{itemize}
\item {Grp. gram.:m.}
\end{itemize}
Ave indiana, verde e violácea..--E talvez aportuguesamento inútil do cast. \textunderscore calamón\textunderscore . Corresponde-lhe o port. \textunderscore camão\textunderscore , ou \textunderscore alquimão\textunderscore .
\section{Calamar}
\begin{itemize}
\item {Grp. gram.:m.}
\end{itemize}
Peixe da costa do Algarve.
\section{Calamate}
\begin{itemize}
\item {Grp. gram.:m.}
\end{itemize}
Arbusto angolense, sarmentoso, cujos frutos são bagas vermelhas, semelhantes ás da erva-moira.
\section{Calamaulo}
\begin{itemize}
\item {Grp. gram.:m.}
\end{itemize}
\begin{itemize}
\item {Proveniência:(Do gr. \textunderscore kalamos\textunderscore  + \textunderscore aulos\textunderscore )}
\end{itemize}
Designação antiga da frauta simples, feita de cana.
\section{Calambá}
\begin{itemize}
\item {Grp. gram.:m.}
\end{itemize}
(V.calambaque)
\section{Calambaque}
\begin{itemize}
\item {Grp. gram.:m.}
\end{itemize}
\begin{itemize}
\item {Proveniência:(Do mal. \textunderscore kalambak\textunderscore )}
\end{itemize}
Substância vegetal, aromática, que se tem confundido com \textunderscore calambuco\textunderscore .
\section{Calambuca}
\begin{itemize}
\item {Grp. gram.:f.}
\end{itemize}
O mesmo que \textunderscore calambuco\textunderscore .
\section{Calambuco}
\begin{itemize}
\item {Grp. gram.:m.}
\end{itemize}
Árvore odorífera do Oriente, espécie de euphórbio, cuja madeira é muito usada nas artes.
Madeira dessa árvore. Cf. Vieira, IX, 223.
\section{Calambuque}
\begin{itemize}
\item {Grp. gram.:m.}
\end{itemize}
(Outra fórma de \textunderscore calambuco\textunderscore )
\section{Calâmeas}
\begin{itemize}
\item {Grp. gram.:f.}
\end{itemize}
\begin{itemize}
\item {Proveniência:(De \textunderscore cálamo\textunderscore )}
\end{itemize}
Tríbo de palmeiras, na classificação de Kunth.
\section{Calamei}
\begin{itemize}
\item {Grp. gram.:m.}
\end{itemize}
\begin{itemize}
\item {Utilização:Ant.}
\end{itemize}
Espécie de mite.
\section{Çalameiro}
\begin{itemize}
\item {Grp. gram.:adj.}
\end{itemize}
\begin{itemize}
\item {Utilização:Prov.}
\end{itemize}
\begin{itemize}
\item {Utilização:alg.}
\end{itemize}
Lisonjeiro, adulador.
(Cast. \textunderscore zalamero\textunderscore )
\section{Calamento}
\begin{itemize}
\item {Grp. gram.:m.}
\end{itemize}
\begin{itemize}
\item {Utilização:Náut.}
\end{itemize}
Acto de \textunderscore calar\textunderscore .
O mesmo que \textunderscore cala\textunderscore ^2.
Porção de cabo, necessária para um barco fundear.
\section{Calamidade}
\begin{itemize}
\item {Grp. gram.:f.}
\end{itemize}
\begin{itemize}
\item {Proveniência:(Lat. \textunderscore calamitas\textunderscore )}
\end{itemize}
Desgraça, extensiva a muita gente.
Infortúnio público.
Grande desgraça.
\section{Calamídeo}
\begin{itemize}
\item {Grp. gram.:adj.}
\end{itemize}
\begin{itemize}
\item {Proveniência:(Do gr. \textunderscore kalamos\textunderscore  + \textunderscore eidos\textunderscore )}
\end{itemize}
Que tem fórma de penna.
\section{Calamífero}
\begin{itemize}
\item {Grp. gram.:adj.}
\end{itemize}
\begin{itemize}
\item {Proveniência:(Do lat. \textunderscore calamus\textunderscore  + \textunderscore ferre\textunderscore )}
\end{itemize}
Que tem colmo.
\section{Calamiforme}
\begin{itemize}
\item {Grp. gram.:adj.}
\end{itemize}
\begin{itemize}
\item {Proveniência:(Do lat. \textunderscore calamus\textunderscore  + \textunderscore forma\textunderscore )}
\end{itemize}
Que tem fórma de colmo.
\section{Calamina}
\begin{itemize}
\item {Grp. gram.:f.}
\end{itemize}
Designação antiga de um óxydo de zinco carbonatado.
Terra bituminosa, para purificar o cobre.
(B. lat. \textunderscore calamina\textunderscore )
\section{Calaminar}
\begin{itemize}
\item {Grp. gram.:adj.}
\end{itemize}
Dizia-se de uma pedra, (calamina), que, feita em pó, se applicava contra doenças de olhos.
(B. lat. \textunderscore calaminaris\textunderscore )
\section{Calaminta}
\begin{itemize}
\item {Grp. gram.:f.}
\end{itemize}
\begin{itemize}
\item {Proveniência:(Gr. \textunderscore calaminthe\textunderscore )}
\end{itemize}
Planta aromática, labiada.
Poejo.
Neveda maior.
\section{Calamintha}
\begin{itemize}
\item {Grp. gram.:f.}
\end{itemize}
\begin{itemize}
\item {Proveniência:(Gr. \textunderscore calaminthe\textunderscore )}
\end{itemize}
Planta aromática, labiada.
Poejo.
Neveda maior.
\section{Calamistrar}
\begin{itemize}
\item {Grp. gram.:v. t.}
\end{itemize}
\begin{itemize}
\item {Proveniência:(De \textunderscore calamistro\textunderscore )}
\end{itemize}
Frisar, tornar crespo (o cabello).
\section{Calamistro}
\begin{itemize}
\item {Grp. gram.:m.}
\end{itemize}
\begin{itemize}
\item {Proveniência:(Lat. \textunderscore calamistrum\textunderscore )}
\end{itemize}
Ferro com que os antigos frisavam o cabello.
\section{Calamita}
\begin{itemize}
\item {Grp. gram.:f.}
\end{itemize}
\begin{itemize}
\item {Utilização:Ant.}
\end{itemize}
\begin{itemize}
\item {Proveniência:(Do lat. \textunderscore calamus\textunderscore )}
\end{itemize}
Espécie de estoraque.
Planta fóssil equisetínea dos terrenos carboníferos.
Bússola.
\section{Calamites}
\begin{itemize}
\item {Grp. gram.:m.}
\end{itemize}
O mesmo ou melhor que \textunderscore calamita\textunderscore . Cf. dr. G. Guimarães, \textunderscore Elem. de Geologia\textunderscore , 208.
\section{Calamitosamente}
\begin{itemize}
\item {Grp. gram.:adv.}
\end{itemize}
De modo \textunderscore calamitoso\textunderscore .
\section{Calamitoso}
\begin{itemize}
\item {Grp. gram.:adj.}
\end{itemize}
\begin{itemize}
\item {Proveniência:(Lat. \textunderscore calamitosus\textunderscore )}
\end{itemize}
Em que há calamidade.
Que traz calamidade.
Funesto.
\section{Cálamo}
\begin{itemize}
\item {Grp. gram.:m.}
\end{itemize}
\begin{itemize}
\item {Proveniência:(Lat. \textunderscore calamus\textunderscore )}
\end{itemize}
Cáule de cereaes.
Penna de escrever.
Frauta.
Estilo.
Planta arácea, medicinal, (\textunderscore acorus calamus\textunderscore ).
\section{Calamocada}
\begin{itemize}
\item {Grp. gram.:f.}
\end{itemize}
\begin{itemize}
\item {Utilização:Pleb.}
\end{itemize}
\begin{itemize}
\item {Proveniência:(De um rad. incerto e \textunderscore mocada\textunderscore )}
\end{itemize}
Pancada na cabeça.
\section{Calamute}
\begin{itemize}
\item {Grp. gram.:m.}
\end{itemize}
Antiga embarcação indiana.
\section{Calandra}
\begin{itemize}
\item {Grp. gram.:f.}
\end{itemize}
\begin{itemize}
\item {Proveniência:(Gr. \textunderscore kalandra\textunderscore )}
\end{itemize}
Máquina cylíndrica, para lustrar ou assetinar tecidos ou papel.
\section{Calandrado}
\begin{itemize}
\item {Grp. gram.:adj.}
\end{itemize}
Assetinado com a calandra.
\section{Calandragem}
\begin{itemize}
\item {Grp. gram.:f.}
\end{itemize}
Acção de \textunderscore calandrar\textunderscore .
\section{Calandrar}
\begin{itemize}
\item {Grp. gram.:v. t.}
\end{itemize}
Lustrar, assetinar, com a calandra.
\section{Calandreiro}
\begin{itemize}
\item {Grp. gram.:m.}
\end{itemize}
Aquelle que calandra.
\section{Calandrínia}
\begin{itemize}
\item {Grp. gram.:f.}
\end{itemize}
\begin{itemize}
\item {Proveniência:(De \textunderscore Calandrini\textunderscore , n. p.)}
\end{itemize}
Gênero de plantas portuláceas.
\section{Calange}
\begin{itemize}
\item {Grp. gram.:m.}
\end{itemize}
Ave gallinácea da África.
\section{Calango}
\begin{itemize}
\item {Grp. gram.:m.}
\end{itemize}
\begin{itemize}
\item {Utilização:Bras}
\end{itemize}
Espécie de lagarto.
\section{Calangro}
\begin{itemize}
\item {Grp. gram.:m.}
\end{itemize}
\begin{itemize}
\item {Utilização:Bras}
\end{itemize}
\begin{itemize}
\item {Proveniência:(De \textunderscore Calango\textunderscore , n. p. do chefe desse grupo)}
\end{itemize}
O mesmo que \textunderscore calango\textunderscore .
Membro de um grupo de salteadores, que infestaram o Pará, de 1873 a 1880.
\section{Calanja}
\begin{itemize}
\item {Grp. gram.:f.}
\end{itemize}
Antigo pêso de Ceilão, correspondente a pouco mais de 4 kilogrammas.
\section{Calão}
\begin{itemize}
\item {Grp. gram.:m.}
\end{itemize}
Linguagem baixa, peculiar a fadistas, larápios, ciganos, etc.
Bohêmio; gíria.
(Cast. \textunderscore caló\textunderscore )
\section{Calão}
\begin{itemize}
\item {Grp. gram.:m.}
\end{itemize}
\begin{itemize}
\item {Proveniência:(De \textunderscore cala\textunderscore ^2?)}
\end{itemize}
Grande lancha, de bôca aberta e oito ou dez remos por banda, empregada especialmente na pesca do atum.
\section{Calão}
\begin{itemize}
\item {Grp. gram.:m.}
\end{itemize}
\begin{itemize}
\item {Proveniência:(De \textunderscore cale\textunderscore )}
\end{itemize}
Telha grande, que se emprega em revestir o fundo dos regos de água, para que esta não seja absorvida pela terra.
\section{Calão}
\begin{itemize}
\item {Grp. gram.:m.}
\end{itemize}
\begin{itemize}
\item {Utilização:Prov.}
\end{itemize}
Homem indolente, calaceiro.
\section{Calapita}
\begin{itemize}
\item {Grp. gram.:f.}
\end{itemize}
Concreção, que se fórma nas nozes de côco.
\section{Calar}
\begin{itemize}
\item {Grp. gram.:v. t.}
\end{itemize}
\begin{itemize}
\item {Utilização:Prov.}
\end{itemize}
\begin{itemize}
\item {Utilização:trasm.}
\end{itemize}
\begin{itemize}
\item {Grp. gram.:V. i.}
\end{itemize}
\begin{itemize}
\item {Utilização:Pesc.}
\end{itemize}
\begin{itemize}
\item {Proveniência:(Lat. \textunderscore chalare\textunderscore )}
\end{itemize}
Abaixar.
Fazer estar em silêncio: \textunderscore calar as crianças\textunderscore .
Penetrar, cortar.
Rachar com faca (uma melancia), para se verificar se está madura.
Meter no fundo.
Collocar em lugar próprio.
Cortar as medranças de (melões e melancias), para que bracejem para os lados.
Descer.
Guardar silêncio.
Lançar á água uma rede de galeão.
\section{Calasse}
\begin{itemize}
\item {Grp. gram.:m.}
\end{itemize}
Árvore de Damão.
\section{Caláthide}
\begin{itemize}
\item {Grp. gram.:f.}
\end{itemize}
\begin{itemize}
\item {Utilização:Bot.}
\end{itemize}
\begin{itemize}
\item {Proveniência:(Gr. \textunderscore kalathis\textunderscore , açafate)}
\end{itemize}
Reunião de pequenas flôres sôbre um receptáculo commum.
\section{Calathiforme}
\begin{itemize}
\item {Grp. gram.:adj.}
\end{itemize}
\begin{itemize}
\item {Proveniência:(Do lat. \textunderscore calathus\textunderscore  + \textunderscore forma\textunderscore )}
\end{itemize}
Que tem fórma de açafate.
\section{Calátide}
\begin{itemize}
\item {Grp. gram.:f.}
\end{itemize}
\begin{itemize}
\item {Utilização:Bot.}
\end{itemize}
\begin{itemize}
\item {Proveniência:(Gr. \textunderscore kalathis\textunderscore , açafate)}
\end{itemize}
Reunião de pequenas flôres sôbre um receptáculo commum.
\section{Calatiforme}
\begin{itemize}
\item {Grp. gram.:adj.}
\end{itemize}
\begin{itemize}
\item {Proveniência:(Do lat. \textunderscore calathus\textunderscore  + \textunderscore forma\textunderscore )}
\end{itemize}
Que tem fórma de açafate.
\section{Calatrão}
\begin{itemize}
\item {Grp. gram.:m.}
\end{itemize}
\begin{itemize}
\item {Utilização:T. do Fundão}
\end{itemize}
\begin{itemize}
\item {Utilização:T. da Bairrada}
\end{itemize}
Mulher encorpada e feia.
Rameira reles.
(Cp. \textunderscore culatrona\textunderscore )
\section{Calatrava}
\begin{itemize}
\item {Grp. gram.:f.}
\end{itemize}
\begin{itemize}
\item {Proveniência:(De \textunderscore Calatrava\textunderscore , n. p.)}
\end{itemize}
Antiga Ordem militar de Castella.
\section{Calatravense}
\begin{itemize}
\item {Grp. gram.:adj.}
\end{itemize}
Relativo á Ordem militar de Calatrava.
\section{Calatravo}
\begin{itemize}
\item {Grp. gram.:m.}
\end{itemize}
Cavalleiro ou frade da Ordem de Calatrava.
\section{Calatróia}
\begin{itemize}
\item {Grp. gram.:f.}
\end{itemize}
\begin{itemize}
\item {Utilização:Prov.}
\end{itemize}
\begin{itemize}
\item {Utilização:alent.}
\end{itemize}
Sopa de azeite e cebola.
\section{Calau}
\begin{itemize}
\item {Grp. gram.:m.}
\end{itemize}
Ave de bico enorme, da divisão dos syndáctylos.
\section{Calávea}
\begin{itemize}
\item {Grp. gram.:f.}
\end{itemize}
Árvore de Samatra, de casca filamentosa.
\section{Calaveira}
\begin{itemize}
\item {Grp. gram.:f.}
\end{itemize}
\begin{itemize}
\item {Utilização:Prov.}
\end{itemize}
\begin{itemize}
\item {Utilização:alent.}
\end{itemize}
\begin{itemize}
\item {Utilização:Ant.}
\end{itemize}
\begin{itemize}
\item {Grp. gram.:M.  e  adj.}
\end{itemize}
\begin{itemize}
\item {Utilização:Prov.}
\end{itemize}
\begin{itemize}
\item {Utilização:alent.}
\end{itemize}
Caveira.
Estouvado, extravagante.
(Cast. \textunderscore calavera\textunderscore )
\section{Calca}
\begin{itemize}
\item {Grp. gram.:f.}
\end{itemize}
\begin{itemize}
\item {Utilização:Des.}
\end{itemize}
Acção de \textunderscore calcar\textunderscore .
\section{Calça}
\begin{itemize}
\item {Grp. gram.:f.}
\end{itemize}
\begin{itemize}
\item {Proveniência:(De \textunderscore calçar\textunderscore )}
\end{itemize}
Peça de vestuário. (V. \textunderscore calças\textunderscore )
Atilho ou fita, que se põe nas pernas das gallinhas e outras aves domésticas, para as distinguir das alheias.
Malha branca nas mãos ou nos pés dos equídeos, acima da corôa dos cascos.
\section{Calçada}
\begin{itemize}
\item {Grp. gram.:f.}
\end{itemize}
\begin{itemize}
\item {Proveniência:(De \textunderscore calçar\textunderscore )}
\end{itemize}
Caminho ou rua empedrada.
Rua íngreme.
\section{Calçada}
\begin{itemize}
\item {Grp. gram.:f.}
\end{itemize}
\begin{itemize}
\item {Utilização:Ant.}
\end{itemize}
\begin{itemize}
\item {Proveniência:(De \textunderscore calça\textunderscore )}
\end{itemize}
Pancada com calça ou meia, que se encheu de areia ou terra.
\section{Calcadeira}
\begin{itemize}
\item {Grp. gram.:f.}
\end{itemize}
\begin{itemize}
\item {Proveniência:(De \textunderscore calcar\textunderscore )}
\end{itemize}
Pau, com que os moleiros atacam ou calcam a farinha nos sacos.
\section{Calçadeira}
\begin{itemize}
\item {Grp. gram.:f.}
\end{itemize}
\begin{itemize}
\item {Proveniência:(De \textunderscore calçar\textunderscore )}
\end{itemize}
Utensílio, com que se facilita o calçar sapatos.
\section{Calçado}
\begin{itemize}
\item {Grp. gram.:adj.}
\end{itemize}
\begin{itemize}
\item {Grp. gram.:M.}
\end{itemize}
\begin{itemize}
\item {Proveniência:(Lat. \textunderscore calceatus\textunderscore )}
\end{itemize}
Empedrado.
Que tem malhas nos pés ou nas mãos, (falando-se de animaes).
Peça de vestuário, que cobre os pés por todos os lados.
\section{Calcadoiro}
\begin{itemize}
\item {Grp. gram.:m.}
\end{itemize}
\begin{itemize}
\item {Proveniência:(De \textunderscore calcar\textunderscore )}
\end{itemize}
Lugar, em que se calca.
Eira, em que se debulham cereaes.
Cereaes, que estão na eira para sêr debulhados.
\section{Calcador}
\begin{itemize}
\item {Grp. gram.:m.  e  adj.}
\end{itemize}
\begin{itemize}
\item {Grp. gram.:M.}
\end{itemize}
O que calca.
Peça das máquinas de costura, com a qual se segura o tecido que se cose.
\section{Calçador}
\begin{itemize}
\item {Grp. gram.:m.}
\end{itemize}
Aquelle que calça.
Calçadeira.
\section{Calcadouro}
\begin{itemize}
\item {Grp. gram.:m.}
\end{itemize}
\begin{itemize}
\item {Proveniência:(De \textunderscore calcar\textunderscore )}
\end{itemize}
Lugar, em que se calca.
Eira, em que se debulham cereaes.
Cereaes, que estão na eira para sêr debulhados.
\section{Calcadura}
\begin{itemize}
\item {Grp. gram.:f.}
\end{itemize}
Acção de \textunderscore calcar\textunderscore .
\section{Calçadura}
\begin{itemize}
\item {Grp. gram.:f.}
\end{itemize}
\begin{itemize}
\item {Proveniência:(De \textunderscore calçar\textunderscore )}
\end{itemize}
Lugar, occupado pelo calcanhar no calçado.
\section{Calcamano}
\begin{itemize}
\item {Grp. gram.:m.}
\end{itemize}
\begin{itemize}
\item {Utilização:Bras}
\end{itemize}
Designação pop. e depreciativa dos Italianos em geral.
\section{Calcamar}
\begin{itemize}
\item {Grp. gram.:m.}
\end{itemize}
\begin{itemize}
\item {Proveniência:(De \textunderscore calcar\textunderscore  + \textunderscore mar\textunderscore )}
\end{itemize}
Ave aquática dos mares da África.
Pássaro do Brasil.
\section{Calcamento}
\begin{itemize}
\item {Grp. gram.:m.}
\end{itemize}
(V.calcadura)
\section{Calçamento}
\begin{itemize}
\item {Grp. gram.:m.}
\end{itemize}
\begin{itemize}
\item {Utilização:Des.}
\end{itemize}
Acção de \textunderscore calçar\textunderscore .
\section{Calcaneano}
\begin{itemize}
\item {Grp. gram.:adj.}
\end{itemize}
Relativo ao calcâneo.
\section{Calcâneo}
\begin{itemize}
\item {Grp. gram.:m.}
\end{itemize}
\begin{itemize}
\item {Proveniência:(Lat. \textunderscore calcaneum\textunderscore )}
\end{itemize}
Osso, que fórma o calcanhar.
\section{Calcanha}
\begin{itemize}
\item {Grp. gram.:f.}
\end{itemize}
\begin{itemize}
\item {Utilização:Bras}
\end{itemize}
Varredeira de engenhos.
\section{Calcanhar}
\begin{itemize}
\item {Grp. gram.:m.}
\end{itemize}
\begin{itemize}
\item {Proveniência:(Do lat. \textunderscore calcaneum\textunderscore )}
\end{itemize}
A parte posterior do pé.
A parte do calçado, correspondente àquella parte do pé.
\section{Calcanheira}
\begin{itemize}
\item {Grp. gram.:f.}
\end{itemize}
\begin{itemize}
\item {Proveniência:(De \textunderscore calcanhar\textunderscore )}
\end{itemize}
Parte do sapato, da bota ou da meia, que se adapta ao calcanhar.
Dobra, que se faz na parte posterior dos sapatos, para se calçarem como chinelos.
\section{Calcantes}
\begin{itemize}
\item {Grp. gram.:m. pl.}
\end{itemize}
\begin{itemize}
\item {Utilização:Gír.}
\end{itemize}
\begin{itemize}
\item {Proveniência:(De \textunderscore calcar\textunderscore )}
\end{itemize}
Sapatos.
Pés.
\section{Calcão}
\begin{itemize}
\item {Grp. gram.:m.}
\end{itemize}
\begin{itemize}
\item {Utilização:Prov.}
\end{itemize}
\begin{itemize}
\item {Utilização:trasm.}
\end{itemize}
O mesmo que \textunderscore calcadeira\textunderscore .
\section{Calção}
\begin{itemize}
\item {Grp. gram.:m.}
\end{itemize}
\begin{itemize}
\item {Proveniência:(De \textunderscore calça\textunderscore )}
\end{itemize}
Calças, que descem até o joêlho, ou pouco abaixo do joêlho.
Pennas, que revestem as pernas de algumas aves.
\section{Çalçaparrilha}
\begin{itemize}
\item {Grp. gram.:f.}
\end{itemize}
(Fórma exacta, em vez da mais usada, \textunderscore salsa-parrilha\textunderscore )
(Cast. \textunderscore zarzaparrilla\textunderscore )
\section{Calcar}
\begin{itemize}
\item {Grp. gram.:v. t.}
\end{itemize}
\begin{itemize}
\item {Proveniência:(Lat. \textunderscore calcare\textunderscore )}
\end{itemize}
Pisar; moer; contundir.
Atropelar.
Desprezar.
Comprimir.
O mesmo que \textunderscore decalcar\textunderscore .
\section{Calçar}
\begin{itemize}
\item {Grp. gram.:v. t.}
\end{itemize}
\begin{itemize}
\item {Grp. gram.:V. i.}
\end{itemize}
\begin{itemize}
\item {Proveniência:(Lat. \textunderscore calceare\textunderscore , de \textunderscore calceus\textunderscore )}
\end{itemize}
Revestir pés, pernas ou mãos com (o vestuário que lhes corresponde): \textunderscore calçar meias\textunderscore ; \textunderscore calçar luvas\textunderscore .
Fornecer calçado a: \textunderscore o tio é que o veste e o calça\textunderscore .
Empedrar, calcetar: \textunderscore calçar uma rua\textunderscore .
Segurar, erguer com calço: \textunderscore calçar a roda do carro\textunderscore .
Revestir com aço a parte cortante de (instrumento, ferramenta): \textunderscore calçar o machado\textunderscore .
Ajustar-se bem.
Têr calçado.
\section{Calcaré}
\begin{itemize}
\item {Grp. gram.:m.}
\end{itemize}
\begin{itemize}
\item {Utilização:Prov.}
\end{itemize}
\begin{itemize}
\item {Utilização:minh.}
\end{itemize}
O mesmo que \textunderscore codorniz\textunderscore .
(Cp. \textunderscore calcoré\textunderscore )
\section{Calcário}
\begin{itemize}
\item {Grp. gram.:adj.}
\end{itemize}
\begin{itemize}
\item {Grp. gram.:M.}
\end{itemize}
\begin{itemize}
\item {Proveniência:(Lat. \textunderscore calcarius\textunderscore )}
\end{itemize}
Que tem cal, em que há cal: \textunderscore terreno calcário\textunderscore .
Rocha, formada pelo carbonato de cálcio.
\section{Calcarização}
\begin{itemize}
\item {Grp. gram.:f.}
\end{itemize}
Acto de \textunderscore calcarizar\textunderscore .
\section{Calcarizar}
\begin{itemize}
\item {Grp. gram.:v. t.}
\end{itemize}
\begin{itemize}
\item {Utilização:Neol.}
\end{itemize}
Tornar calcário; misturar com cal.
\section{Calcas}
\begin{itemize}
\item {Grp. gram.:f. pl.}
\end{itemize}
\begin{itemize}
\item {Utilização:Gír. Lisb.}
\end{itemize}
Botas.
(Cp. \textunderscore calcantes\textunderscore )
\section{Calças}
\begin{itemize}
\item {Grp. gram.:f. pl.}
\end{itemize}
Vestuário para homem, que começa na cintura, dividindo-se por baixo do tronco em dois canos que rodeiam as pernas, estendendo-se até os pés.
Vestuário identico, mas mais curto, para mulheres.
(Pl. de \textunderscore calça\textunderscore )
\section{Calce}
\begin{itemize}
\item {Grp. gram.:m.}
\end{itemize}
(V.calço)
\section{Calcedónia}
\begin{itemize}
\item {Grp. gram.:f.}
\end{itemize}
\begin{itemize}
\item {Proveniência:(Do gr. \textunderscore Kalkedon\textunderscore , n. p.)}
\end{itemize}
Pedra preciosa, espécie de ágatha.
\section{Calcedónio}
\begin{itemize}
\item {Grp. gram.:adj.}
\end{itemize}
Que tem o aspecto da calcedónia.
\section{Calceiforme}
\begin{itemize}
\item {Grp. gram.:adj.}
\end{itemize}
\begin{itemize}
\item {Proveniência:(Do lat. \textunderscore calceus\textunderscore  + \textunderscore forma\textunderscore )}
\end{itemize}
Que tem fórma de sapato.
\section{Calceiro}
\begin{itemize}
\item {Grp. gram.:m.}
\end{itemize}
Fabricante de calças.
\section{Cálceo}
\begin{itemize}
\item {Grp. gram.:m.}
\end{itemize}
\begin{itemize}
\item {Utilização:Ant.}
\end{itemize}
\begin{itemize}
\item {Proveniência:(Lat. \textunderscore calceus\textunderscore )}
\end{itemize}
Sapato.
\section{Calceolária}
\begin{itemize}
\item {Grp. gram.:f.}
\end{itemize}
\begin{itemize}
\item {Proveniência:(Do lat. \textunderscore calceolus\textunderscore )}
\end{itemize}
Planta medicinal e ornamental, da fam. das escrofularíneas.
\section{Calcês}
\begin{itemize}
\item {Grp. gram.:m.}
\end{itemize}
\begin{itemize}
\item {Utilização:Náut.}
\end{itemize}
\begin{itemize}
\item {Proveniência:(Do lat. \textunderscore carchesium\textunderscore )}
\end{itemize}
Parte quadrada do mastro ou mastaréu, desde a roman para cima, e na qual encapella a enxárcia real.
\section{Calceta}
\begin{itemize}
\item {fónica:cê}
\end{itemize}
\begin{itemize}
\item {Grp. gram.:f.}
\end{itemize}
\begin{itemize}
\item {Grp. gram.:M.}
\end{itemize}
\begin{itemize}
\item {Grp. gram.:F. pl.}
\end{itemize}
\begin{itemize}
\item {Utilização:Prov.}
\end{itemize}
\begin{itemize}
\item {Proveniência:(De \textunderscore calçar\textunderscore )}
\end{itemize}
Grilheta, argola, com que se prendia a perna do condemnado.
Trabalho forçado de condemnados.
O condemnado a trabalhos forçados.
Peúgas.
\section{Calcetar}
\begin{itemize}
\item {Grp. gram.:v. t.}
\end{itemize}
\begin{itemize}
\item {Proveniência:(De \textunderscore calçar\textunderscore )}
\end{itemize}
Empedrar; calçar, revestir, com pedras justapostas.
\section{Calcetaria}
\begin{itemize}
\item {Grp. gram.:f.}
\end{itemize}
Trabalho, profissão, de calceteiro.
\section{Calceteiro}
\begin{itemize}
\item {Grp. gram.:m.}
\end{itemize}
\begin{itemize}
\item {Proveniência:(De \textunderscore calcetar\textunderscore )}
\end{itemize}
Aquelle que trabalha no empedramento de estradas, ruas, pátios, etc.
\section{Calceteiro}
\begin{itemize}
\item {Grp. gram.:m.}
\end{itemize}
\begin{itemize}
\item {Utilização:Ant.}
\end{itemize}
Fabricante de calças.
Calceiro. Cf. Bluteau.
\section{Cálcico}
\begin{itemize}
\item {Grp. gram.:adj.}
\end{itemize}
\begin{itemize}
\item {Proveniência:(Do lat. \textunderscore calx\textunderscore )}
\end{itemize}
Relativo á cal.
\section{Calcídeo}
\begin{itemize}
\item {Grp. gram.:m.}
\end{itemize}
\begin{itemize}
\item {Proveniência:(De \textunderscore cálcio\textunderscore )}
\end{itemize}
Designação genérica dos metaes semelhantes ao cálcio.
\section{Calcídicas}
\begin{itemize}
\item {Grp. gram.:f. pl.}
\end{itemize}
Salas, onde se vendiam refrescos, nos tribunaes romanos.
\section{Calcídios}
\begin{itemize}
\item {Grp. gram.:m. pl.}
\end{itemize}
\begin{itemize}
\item {Proveniência:(De \textunderscore cálcido\textunderscore )}
\end{itemize}
Insectos, que têm o cálcido por typo.
\section{Cálcido}
\begin{itemize}
\item {Grp. gram.:m.}
\end{itemize}
\begin{itemize}
\item {Proveniência:(De \textunderscore cálcio\textunderscore , por terem aquelles animaes uma côr metállica acobreada)}
\end{itemize}
Gênero de insectos hymenópteros.
Reptil sáurio.
\section{Calcífero}
\begin{itemize}
\item {Grp. gram.:adj.}
\end{itemize}
O mesmo que \textunderscore calcário\textunderscore .
\section{Calcificação}
\begin{itemize}
\item {Grp. gram.:f.}
\end{itemize}
\begin{itemize}
\item {Proveniência:(De \textunderscore calcificar\textunderscore )}
\end{itemize}
Qualidade dos tecidos que, tendo sido molles, tomam a consistência e a côr da cal.
\section{Calcificar-se}
\begin{itemize}
\item {Grp. gram.:v. p.}
\end{itemize}
\begin{itemize}
\item {Utilização:Med.}
\end{itemize}
\begin{itemize}
\item {Proveniência:(Do lat. \textunderscore calx\textunderscore  + \textunderscore facere\textunderscore )}
\end{itemize}
Tomar anormalmente a côr e a consistência da cal, por effeito pathológico.
\section{Calcímetro}
\begin{itemize}
\item {Grp. gram.:m.}
\end{itemize}
\begin{itemize}
\item {Proveniência:(T. hýbr., do lat. \textunderscore calx\textunderscore  + gr. \textunderscore metron\textunderscore )}
\end{itemize}
Instrumento, para avaliar a proporção, em que a cal entra num terreno destinado á cultura da vinha.
\section{Cálcimo}
\begin{itemize}
\item {Grp. gram.:m.}
\end{itemize}
\begin{itemize}
\item {Utilização:Prov.}
\end{itemize}
\begin{itemize}
\item {Utilização:trasm.}
\end{itemize}
Espécie de planta, (\textunderscore verbascum virgatum\textunderscore , With.) com que o povo envenena o peixe dos rios.
(Cp. \textunderscore cácimo\textunderscore )
\section{Calcina}
\begin{itemize}
\item {Grp. gram.:f.}
\end{itemize}
O mesmo que \textunderscore calcinação\textunderscore . Cf. \textunderscore Inquér. Industr.\textunderscore , p. II, l. I, 290, 294 e 298.
\section{Calcinação}
\begin{itemize}
\item {Grp. gram.:f.}
\end{itemize}
Acção ou effeito de \textunderscore calcinar\textunderscore .
\section{Calcinar}
\begin{itemize}
\item {Grp. gram.:v. t.}
\end{itemize}
\begin{itemize}
\item {Proveniência:(Do lat. \textunderscore calcinare\textunderscore )}
\end{itemize}
Transformar em cal por meio do fogo.
Transformar em óxydos.
Aquecer muito; abrasar: \textunderscore o sol calcina os areaes\textunderscore .
Reduzir a cinzas ou carvão, queimado.
Cauterizar.
\section{Calcinatório}
\begin{itemize}
\item {Grp. gram.:adj.}
\end{itemize}
Com que se póde calcinar.
\section{Calcinável}
\begin{itemize}
\item {Grp. gram.:adj.}
\end{itemize}
Que póde calcinar-se.
\section{Calcinitro}
\begin{itemize}
\item {Grp. gram.:m.}
\end{itemize}
\begin{itemize}
\item {Proveniência:(Do lat. \textunderscore calx\textunderscore  + \textunderscore nitrum\textunderscore )}
\end{itemize}
Azotato de cal.
\section{Cálcio}
\begin{itemize}
\item {Grp. gram.:m.}
\end{itemize}
\begin{itemize}
\item {Proveniência:(Do lat. \textunderscore calx\textunderscore )}
\end{itemize}
Metal branco e amarelado, que se extrai da cal.
Metal, que, combinado com o oxygênio, constitue a cal.
\section{Calcite}
\begin{itemize}
\item {Grp. gram.:f.}
\end{itemize}
\begin{itemize}
\item {Proveniência:(Do lat. \textunderscore calx\textunderscore )}
\end{itemize}
Carbonato natural, espécie de calcário.
\section{Calcítrapa}
\begin{itemize}
\item {Grp. gram.:f.}
\end{itemize}
\begin{itemize}
\item {Utilização:Bot.}
\end{itemize}
Espécie de centáurea, (\textunderscore centaurea calcitrapa\textunderscore ).
\section{Calcitrar}
\begin{itemize}
\item {Grp. gram.:v. i.}
\end{itemize}
\begin{itemize}
\item {Utilização:Ant.}
\end{itemize}
\begin{itemize}
\item {Proveniência:(Lat. \textunderscore calcitrare\textunderscore )}
\end{itemize}
Pernear, escabujar. Cp. \textunderscore recalcitrar\textunderscore .
\section{Calco}
\begin{itemize}
\item {Grp. gram.:m.}
\end{itemize}
\begin{itemize}
\item {Proveniência:(De \textunderscore calcar\textunderscore )}
\end{itemize}
Desenho ou gravura, que se reproduz, collocando-lhe em cima um papel transparente e seguindo-lhe os traços com uma penna.
Papel molhado, que se estende sôbre uma inscripção lapidar, para se obter cópia pelo vestígio dos relevos no papel.
\section{Calço}
\begin{itemize}
\item {Grp. gram.:m.}
\end{itemize}
\begin{itemize}
\item {Utilização:T. do Fundão}
\end{itemize}
\begin{itemize}
\item {Utilização:Prov.}
\end{itemize}
\begin{itemize}
\item {Utilização:dur.}
\end{itemize}
\begin{itemize}
\item {Proveniência:(De \textunderscore calçar\textunderscore )}
\end{itemize}
Pedra, cunha, pedaço de madeira ou de outra substância, que se põe debaixo de um objecto, para o firmar ou para o elevar, ou para o nivelar.
Miolo de pão; bolo mal cozido.
Batoréu ou arrêto, que sustenta terras em socalco.
\section{Calçonipo}
\begin{itemize}
\item {Grp. gram.:m.}
\end{itemize}
\begin{itemize}
\item {Utilização:T. da Régua}
\end{itemize}
Calças curtas.
\section{Calcópteros}
\begin{itemize}
\item {Grp. gram.:m. pl.}
\end{itemize}
\begin{itemize}
\item {Proveniência:(De \textunderscore cálcio\textunderscore  + gr. \textunderscore pteron\textunderscore )}
\end{itemize}
Insectos, que têm asas bronzeadas.
\section{Calcoré}
\begin{itemize}
\item {Grp. gram.:m.}
\end{itemize}
\begin{itemize}
\item {Utilização:Prov.}
\end{itemize}
\begin{itemize}
\item {Utilização:minh.}
\end{itemize}
O mesmo que \textunderscore codorniz\textunderscore .
(Talvez t. onom.)
\section{Calcorreada}
\begin{itemize}
\item {Grp. gram.:f.}
\end{itemize}
\begin{itemize}
\item {Utilização:Pop.}
\end{itemize}
Acção de \textunderscore calcorrear\textunderscore .
\section{Calcorreador}
\begin{itemize}
\item {Grp. gram.:m.}
\end{itemize}
\begin{itemize}
\item {Proveniência:(De \textunderscore calcorrear\textunderscore )}
\end{itemize}
Aquelle que anda muito a pé; andarilho.
\section{Calcorrear}
\begin{itemize}
\item {Grp. gram.:v. i.}
\end{itemize}
\begin{itemize}
\item {Utilização:Fam.}
\end{itemize}
Andar a pé; caminhar muito.
(Cp. \textunderscore calcorros\textunderscore )
\section{Calcorros}
\begin{itemize}
\item {fónica:cô}
\end{itemize}
\begin{itemize}
\item {Grp. gram.:m. pl.}
\end{itemize}
\begin{itemize}
\item {Utilização:Prov.}
\end{itemize}
Sapatos.
(Cast. \textunderscore calcorros\textunderscore )
\section{Calcos}
\begin{itemize}
\item {Grp. gram.:m. pl.}
\end{itemize}
\begin{itemize}
\item {Utilização:ant.}
\end{itemize}
\begin{itemize}
\item {Utilização:Gír.}
\end{itemize}
Sapatos.
(Cp. \textunderscore calcorros\textunderscore )
\section{Calçotas}
\begin{itemize}
\item {Grp. gram.:f. pl.}
\end{itemize}
(Dem. de \textunderscore calças\textunderscore )
\section{Calçote}
\begin{itemize}
\item {Grp. gram.:m.}
\end{itemize}
(V.calçotas)
\section{Calçudo}
\begin{itemize}
\item {Grp. gram.:adj.}
\end{itemize}
\begin{itemize}
\item {Proveniência:(De \textunderscore calça\textunderscore )}
\end{itemize}
Que tem as calças compridas.
E diz-se da ave, que tem as pernas cobertas de pennas.
\section{Calculação}
\begin{itemize}
\item {Grp. gram.:f.}
\end{itemize}
\begin{itemize}
\item {Utilização:Des.}
\end{itemize}
Acto de calcular.
\section{Calculadamente}
\begin{itemize}
\item {Grp. gram.:adv.}
\end{itemize}
Com cálculo.
Com premeditações.
\section{Calculador}
\begin{itemize}
\item {Grp. gram.:m.  e  adj.}
\end{itemize}
\begin{itemize}
\item {Proveniência:(Lat. \textunderscore calculator\textunderscore )}
\end{itemize}
O que calcúla.
\section{Calculante}
\begin{itemize}
\item {Grp. gram.:adj.}
\end{itemize}
\begin{itemize}
\item {Proveniência:(Lat. \textunderscore calculans\textunderscore )}
\end{itemize}
Que calcula, que faz cálculos.
\section{Calcular}
\begin{itemize}
\item {Grp. gram.:v. t.}
\end{itemize}
\begin{itemize}
\item {Proveniência:(Lat. \textunderscore calculare\textunderscore )}
\end{itemize}
Determinar, por meio de operação mathemática.
Contar.
Avaliar.
Conjecturar, presumir: \textunderscore calcúlo que choverá hoje\textunderscore .
Prever.
\section{Calculável}
\begin{itemize}
\item {Grp. gram.:adj.}
\end{itemize}
Que se póde calcular.
\section{Calculista}
\begin{itemize}
\item {Grp. gram.:m.  e  f.}
\end{itemize}
(V.calculador)
\section{Cálculo}
\begin{itemize}
\item {Grp. gram.:m.}
\end{itemize}
\begin{itemize}
\item {Proveniência:(Lat. \textunderscore calculus\textunderscore )}
\end{itemize}
Concreção dura, que se fórma na bexiga.
Acção de calcular.
Plano.
Operação, para achar o resultado da combinação de certos números
\textunderscore ou\textunderscore  quantidades.
Uma das partes da Mathemática, que se occupa da resolução de problemas arithméticos ou algébricos.
\section{Calculoso}
\begin{itemize}
\item {Grp. gram.:adj.}
\end{itemize}
\begin{itemize}
\item {Utilização:Des.}
\end{itemize}
Que padece de cálculos na bexiga.
\section{Calda}
\begin{itemize}
\item {Grp. gram.:f.}
\end{itemize}
\begin{itemize}
\item {Proveniência:(Lat. \textunderscore calda\textunderscore , por \textunderscore calida\textunderscore , fem. de \textunderscore calidus\textunderscore )}
\end{itemize}
Espécie de xarope, usado principalmente em confeitarias.
Acto de tornar encandescente o ferro, para o transformar ou applicar a vários trabalhos.
\section{Calda}
\begin{itemize}
\item {Grp. gram.:f.}
\end{itemize}
\begin{itemize}
\item {Utilização:Prov.}
\end{itemize}
\begin{itemize}
\item {Utilização:trasm.}
\end{itemize}
\begin{itemize}
\item {Utilização:beir.}
\end{itemize}
Sova; tunda.
(Relaciona-se com \textunderscore calda\textunderscore ^1?)
\section{Caldaça}
\begin{itemize}
\item {Grp. gram.:f.}
\end{itemize}
\begin{itemize}
\item {Utilização:Pop.}
\end{itemize}
\begin{itemize}
\item {Utilização:Gír.}
\end{itemize}
\begin{itemize}
\item {Proveniência:(Lat. \textunderscore caldacia\textunderscore )}
\end{itemize}
Caldo mal feito, pouco temperado, aguado.
Vinho.
* \textunderscore M.\textunderscore  (us. no Torrão)
Borrachão.
\section{Caldagem}
\begin{itemize}
\item {Grp. gram.:f.}
\end{itemize}
O mesmo que \textunderscore calagem\textunderscore .
\section{Caldário}
\begin{itemize}
\item {Grp. gram.:adj.}
\end{itemize}
Relativo a caldas.
\section{Caldas}
\begin{itemize}
\item {Grp. gram.:f. pl.}
\end{itemize}
Águas thermaes, \textunderscore ou\textunderscore , antes, o lugar onde ellas nascem.
(Pl. de \textunderscore calda\textunderscore ^1)
\section{Caldeação}
\begin{itemize}
\item {Grp. gram.:f.}
\end{itemize}
Acto de \textunderscore caldear\textunderscore .
\section{Caldeamento}
\begin{itemize}
\item {Grp. gram.:m.}
\end{itemize}
Acto ou effeito de \textunderscore caldear\textunderscore .
\section{Caldear}
\begin{itemize}
\item {Grp. gram.:v. t.}
\end{itemize}
\begin{itemize}
\item {Utilização:Fig.}
\end{itemize}
Tornar rubro por meio de fogo.
Pôr em brasa.
Temperar.
Ligar, reforçando-as, duas substâncias metállicas, encandescentes.
Misturar, juntar com água.
Mestiçar.
\section{Caldeia-quina}
\begin{itemize}
\item {Grp. gram.:f.}
\end{itemize}
Planta da serra de Sintra.
\section{Caldeira}
\begin{itemize}
\item {Grp. gram.:f.}
\end{itemize}
\begin{itemize}
\item {Proveniência:(Lat. \textunderscore caldaria\textunderscore , por \textunderscore calidária\textunderscore , fem. de \textunderscore calidarius\textunderscore )}
\end{itemize}
Grande vaso metállico, para aquecimento de água, producção de vapor, cozedura de alimentos, etc.
Depressão do terreno, no fundo de uma cisterna, tanque, lagôa, etc.
Pequena cova, á roda dos pés das árvores, para juntar águas de chuva ou de rega.
Cada um dos dois compartimentos, inferiores ao caldeiro, nas marinhas do Sado.
Pequena doca ou abrigo natural, para embarcações pequenas.
\section{Caldeirada}
\begin{itemize}
\item {Grp. gram.:f.}
\end{itemize}
\begin{itemize}
\item {Proveniência:(De \textunderscore caldeira\textunderscore )}
\end{itemize}
Porção de líquido, que se deita em uma caldeira.
Pancada de água.
Porção de líquido, que se despeja em um vaso.
Guisado de peixe, feito em caldeira, panelão ou tacho.
\section{Caldeirão}
\begin{itemize}
\item {Grp. gram.:m.}
\end{itemize}
\begin{itemize}
\item {Utilização:Prov.}
\end{itemize}
\begin{itemize}
\item {Utilização:alent.}
\end{itemize}
\begin{itemize}
\item {Utilização:Bras. do N}
\end{itemize}
\begin{itemize}
\item {Utilização:Bras. do N}
\end{itemize}
\begin{itemize}
\item {Utilização:Bras. do S}
\end{itemize}
\begin{itemize}
\item {Proveniência:(De \textunderscore caldeira\textunderscore )}
\end{itemize}
Caldeira grande.
Sinal de suspensão, em música.
Cântaro de cobre ou de latão, quando serve para água.
* Depósito, que recebe as águas do govêrno da salina, para a distribuir pelas peças.
Tanque natural nos lagedos, onde se reúne a água pluvial.
Redemoínho nos rios.
Escavação, feita no campo ou nas estradas pela chuva ou pelo piso de animaes.
\section{Caldeirar}
\begin{itemize}
\item {Grp. gram.:v. t.}
\end{itemize}
Meter em caldeira.
Caldear peros em (vinho) Cf. F. Lapa, \textunderscore Alm. do Lavr.\textunderscore 
\section{Caldeiraria}
\begin{itemize}
\item {Grp. gram.:f.}
\end{itemize}
\begin{itemize}
\item {Utilização:Fig.}
\end{itemize}
\begin{itemize}
\item {Proveniência:(De \textunderscore caldeira\textunderscore )}
\end{itemize}
Arruamento de caldeireiros.
Lugar, em que se faz grande ruído.
\section{Caldeireiro}
\begin{itemize}
\item {Grp. gram.:m.}
\end{itemize}
\begin{itemize}
\item {Utilização:Bras}
\end{itemize}
\begin{itemize}
\item {Proveniência:(De \textunderscore caldeira\textunderscore )}
\end{itemize}
Aquelle que faz ou vende caldeiras e, geralmente, utensílios de metal.
Aquelle que trabalha nas caldeiras dos engenhos de açúcar.
\section{Caldeirinha}
\begin{itemize}
\item {Grp. gram.:f.}
\end{itemize}
\begin{itemize}
\item {Utilização:Prov.}
\end{itemize}
\begin{itemize}
\item {Utilização:alg.}
\end{itemize}
\begin{itemize}
\item {Proveniência:(De \textunderscore caldeira\textunderscore )}
\end{itemize}
Vaso de água benta.
O mesmo que \textunderscore megengra\textunderscore .
\section{Caldeiro}
\begin{itemize}
\item {Grp. gram.:m.}
\end{itemize}
\begin{itemize}
\item {Proveniência:(Lat. \textunderscore caldarius\textunderscore , contr. de \textunderscore calidarius\textunderscore )}
\end{itemize}
Vaso, com que se tira água das cisternas ou poços.
Pequena caldeira, para cozinhar.
Segunda bacia rectangular, nas salinas, separada do viveiro por um dique; o mesmo que \textunderscore algibé\textunderscore .
\section{Caldeiro}
\begin{itemize}
\item {Grp. gram.:adj.}
\end{itemize}
Diz-se do toiro que tem as hastes um tanto baixas e menos unidas que as dos gaiolos.
\section{Caldeta}
\begin{itemize}
\item {fónica:dê}
\end{itemize}
\begin{itemize}
\item {Grp. gram.:f.}
\end{itemize}
\begin{itemize}
\item {Proveniência:(De \textunderscore caldo\textunderscore )}
\end{itemize}
Pitéu algarvio, espécie de sopa com amêijoas e varios adubos.
\section{Caldivana}
\begin{itemize}
\item {Grp. gram.:f.}
\end{itemize}
(V.caldaça)
\section{Caldo}
\begin{itemize}
\item {Grp. gram.:m.}
\end{itemize}
\begin{itemize}
\item {Utilização:Prov.}
\end{itemize}
\begin{itemize}
\item {Proveniência:(Lat. \textunderscore caldus\textunderscore , contr. de \textunderscore calidus\textunderscore )}
\end{itemize}
Substância líquida e alimentícia, preparada pela cocção de carne ou de outros adubos, a que ás vezes se juntam legumes.
Hortaliça, couves.
\section{Caldoça}
\begin{itemize}
\item {Grp. gram.:f.}
\end{itemize}
O mesmo que \textunderscore caldaça\textunderscore .
\section{Caldorro}
\begin{itemize}
\item {fónica:dô}
\end{itemize}
\begin{itemize}
\item {Grp. gram.:m.}
\end{itemize}
\begin{itemize}
\item {Utilização:Pop.}
\end{itemize}
(V.caldaça)
\section{Caldoso}
\begin{itemize}
\item {Grp. gram.:adj.}
\end{itemize}
Que tem muita calda.
\section{Calducha}
\begin{itemize}
\item {Grp. gram.:f.}
\end{itemize}
\begin{itemize}
\item {Utilização:Pop.}
\end{itemize}
(V.caldaça)
\section{Caldudo}
\begin{itemize}
\item {Grp. gram.:m.}
\end{itemize}
\begin{itemize}
\item {Utilização:Prov.}
\end{itemize}
\begin{itemize}
\item {Utilização:beir.}
\end{itemize}
\begin{itemize}
\item {Proveniência:(De \textunderscore caldo\textunderscore )}
\end{itemize}
Caldo de castanhas piladas, espécie de pureia, que se usa á sôbre-mesa.
\section{Cale}
\begin{itemize}
\item {Grp. gram.:f.}
\end{itemize}
\begin{itemize}
\item {Utilização:Prov.}
\end{itemize}
\begin{itemize}
\item {Utilização:minh.}
\end{itemize}
\begin{itemize}
\item {Proveniência:(Do lat. \textunderscore canalis\textunderscore )}
\end{itemize}
Rêgo ou encaixe em peça comprida de madeira, como a \textunderscore cale\textunderscore  ou \textunderscore calha\textunderscore  da azenha.
Barco, de fundo chato, para navegação fluvial.
\section{Caleadela}
\begin{itemize}
\item {Grp. gram.:f.}
\end{itemize}
Acto de \textunderscore calear\textunderscore .
\section{Calear}
\begin{itemize}
\item {Grp. gram.:v. t.}
\end{itemize}
\begin{itemize}
\item {Utilização:Prov.}
\end{itemize}
\begin{itemize}
\item {Proveniência:(De \textunderscore cal\textunderscore )}
\end{itemize}
O mesmo que \textunderscore caiar\textunderscore .
\section{Caleça}
\begin{itemize}
\item {Grp. gram.:f.}
\end{itemize}
\begin{itemize}
\item {Utilização:Bras}
\end{itemize}
\begin{itemize}
\item {Proveniência:(T. commum a todas as línguas esclavónicas)}
\end{itemize}
Antiga carruagem, própria para jornada.
Sege; o mesmo que \textunderscore caleche\textunderscore .
\section{Caleceiro}
\begin{itemize}
\item {Grp. gram.:m.}
\end{itemize}
Guia de caleça.
\section{Caleche}
\begin{itemize}
\item {Grp. gram.:m.  e  f.}
\end{itemize}
\begin{itemize}
\item {Proveniência:(Fr. \textunderscore calèche\textunderscore . Cp. \textunderscore caleça\textunderscore )}
\end{itemize}
Termo afrancesado, que designa uma carruagem de quatro rodas e dois assentos, aberta por deante.
\section{Caleço}
\begin{itemize}
\item {Grp. gram.:m.}
\end{itemize}
\begin{itemize}
\item {Utilização:Gír.}
\end{itemize}
\begin{itemize}
\item {Utilização:Prov.}
\end{itemize}
\begin{itemize}
\item {Utilização:alent.}
\end{itemize}
\begin{itemize}
\item {Proveniência:(De \textunderscore cale\textunderscore )}
\end{itemize}
Quartilho.
Copo de vinho.
\section{Calefacção}
\begin{itemize}
\item {Grp. gram.:f.}
\end{itemize}
\begin{itemize}
\item {Proveniência:(Lat. \textunderscore calefactio\textunderscore )}
\end{itemize}
Acto de aquecer.
\section{Calefaciente}
\begin{itemize}
\item {Grp. gram.:adj.}
\end{itemize}
\begin{itemize}
\item {Proveniência:(Lat. \textunderscore calefaciens\textunderscore )}
\end{itemize}
Que faz aquecer.
\section{Calefactor}
\begin{itemize}
\item {Grp. gram.:m.}
\end{itemize}
Apparelho de aquecimento. Cf. \textunderscore Techn. Rur.\textunderscore , 259.
\section{Calefrio}
\begin{itemize}
\item {Grp. gram.:m.}
\end{itemize}
Contracção rápida da pelle, com sensação de frio.
Frio interior, acompanhado de tremura.
Arrepio.
(Cp. cast. \textunderscore calofrio\textunderscore )
\section{Calege}
\begin{itemize}
\item {Grp. gram.:f.}
\end{itemize}
O mesmo que \textunderscore caleça\textunderscore . Cf. Camillo, \textunderscore Caveira\textunderscore , 347.
\section{Calego}
\begin{itemize}
\item {Grp. gram.:m.}
\end{itemize}
\begin{itemize}
\item {Utilização:Ant.}
\end{itemize}
O mesmo que \textunderscore caleça\textunderscore . Cf. M. Bernardes, \textunderscore Armas da Castidade\textunderscore , 302.
\section{Calei}
\begin{itemize}
\item {Grp. gram.:m.}
\end{itemize}
Indivíduo que, junto de alguns sobas angolenses, dá audiência permanente ao povo. Cf. Capello e Ivens, I, 173.
\section{Caleidoscópio}
\begin{itemize}
\item {Grp. gram.:m.}
\end{itemize}
(V.calidoscópio)
\section{Caleira}
\begin{itemize}
\item {Grp. gram.:f.}
\end{itemize}
\begin{itemize}
\item {Proveniência:(De \textunderscore cale\textunderscore )}
\end{itemize}
Cano, para esgôto das águas do telhado.
Telha.
Tronco, com um sulco profundo em sentido longitudinal, ou construcção análoga, para derivação de líquidos.
Peça de madeira, em fórma de telha, o mesmo que \textunderscore adelhão\textunderscore .
O mesmo que \textunderscore calão\textunderscore ^3.
\section{Caleiro}
\begin{itemize}
\item {Grp. gram.:m.}
\end{itemize}
\begin{itemize}
\item {Utilização:Prov.}
\end{itemize}
\begin{itemize}
\item {Utilização:minh.}
\end{itemize}
O mesmo que \textunderscore caleira\textunderscore .
Caleira de pedra, para encanamento de águas potáveis.
Designação do pescador dos rios, em opposição ao do mar.
\section{Caleiro}
\begin{itemize}
\item {Grp. gram.:m.}
\end{itemize}
Aquelle que tem fornos de cal por sua conta.
Operário dos fornos de cal. Cf. \textunderscore Inquér. Industr.\textunderscore , p. II, l. II, 252.
\section{Calele}
\begin{itemize}
\item {Grp. gram.:m.}
\end{itemize}
\begin{itemize}
\item {Proveniência:(T. lund.)}
\end{itemize}
Árvore angolense, de copa semi-esphérica, fôlhas serreadas, e frutos axillares em pequenos cachos.
\section{Calema}
\begin{itemize}
\item {Grp. gram.:f.}
\end{itemize}
\begin{itemize}
\item {Proveniência:(T. afr.?)}
\end{itemize}
Especial ondulação do mar, nas costas da África.
Arrebentação do mar na costa, em consequência da ondulação que vem do largo.
\section{Çalema}
\begin{itemize}
\item {Grp. gram.:f.}
\end{itemize}
\begin{itemize}
\item {Utilização:Ant.}
\end{itemize}
\begin{itemize}
\item {Proveniência:(Do ár. \textunderscore çalam\textunderscore )}
\end{itemize}
Peixe esparoide.
Saudação, cumprimentos, o mesmo que \textunderscore salama\textunderscore . Cf. Latino, \textunderscore Elog. Acad.\textunderscore 
\section{Calembe-lembe}
\begin{itemize}
\item {Grp. gram.:m.}
\end{itemize}
Planta angolense, aquática, (\textunderscore nymphaea stellata\textunderscore ?), cujo limbo, na página superior, é de vivo carmesim.
\section{Calembur}
\begin{itemize}
\item {Grp. gram.:m.}
\end{itemize}
\begin{itemize}
\item {Utilização:Neol.}
\end{itemize}
\begin{itemize}
\item {Proveniência:(Fr. \textunderscore calembour\textunderscore )}
\end{itemize}
Jôgo de palavras que, sendo differentes na significação, são semelhantes no som, dando lugar a equívocos. Cf. Castilho, \textunderscore Sabichonas\textunderscore , 119.
\section{Calemburar}
\begin{itemize}
\item {Grp. gram.:v. i.}
\end{itemize}
Fazer calembures. Cf. Castilho, \textunderscore Sabichonas\textunderscore , 120.
\section{Calemburaria}
\begin{itemize}
\item {Grp. gram.:f.}
\end{itemize}
Dom de calemburar. Cf. Castilho, \textunderscore Sabichonas\textunderscore , 120.
\section{Calemburgar}
\begin{itemize}
\item {Grp. gram.:v. t.}
\end{itemize}
Dizer ou exprimir, formando calemburgo. Cf. Camillo, \textunderscore Caveira\textunderscore , 126.
\section{Calemburgo}
\begin{itemize}
\item {Grp. gram.:m.}
\end{itemize}
O mesmo que \textunderscore calembur\textunderscore . Cf. Filinto, VIII, 129; Castilho, \textunderscore Sabichonas\textunderscore , 121; Pato, \textunderscore Ciprestes\textunderscore , 178.
\section{Calemburista}
\begin{itemize}
\item {Grp. gram.:m.}
\end{itemize}
Aquelle que faz calembures.
Aquelle que gosta de falar equivocamente, por gracejo.
\section{Calemburizar}
\begin{itemize}
\item {Grp. gram.:v. i.}
\end{itemize}
O mesmo que \textunderscore calemburar\textunderscore . Cf. Camillo, \textunderscore Volcões\textunderscore , 190.
\section{Calemute}
\begin{itemize}
\item {Grp. gram.:m.}
\end{itemize}
(V.calamute)
\section{Calendar}
\begin{itemize}
\item {Grp. gram.:adj.}
\end{itemize}
\begin{itemize}
\item {Proveniência:(Lat. \textunderscore calendaris\textunderscore )}
\end{itemize}
Relativo ás calendas. Cf. Castilho, \textunderscore Fastos\textunderscore , I, 242.
\section{Calendário}
\begin{itemize}
\item {Grp. gram.:m.}
\end{itemize}
\begin{itemize}
\item {Proveniência:(Lat. \textunderscore calendarium\textunderscore )}
\end{itemize}
Tabella, folhinha ou livrinho, em que se indicam os dias, semanas e meses do anno, festas religiosas, phases da lua, dias de gala, etc.
Almanaque.
\section{Calendarista}
\begin{itemize}
\item {Grp. gram.:m.}
\end{itemize}
Aquelle que faz calendários.
\section{Calendas}
\begin{itemize}
\item {Grp. gram.:f. pl.}
\end{itemize}
\begin{itemize}
\item {Proveniência:(Lat. \textunderscore calendae\textunderscore )}
\end{itemize}
Primeiro dia de cada mês.
\section{Calêndula}
\begin{itemize}
\item {Grp. gram.:f.}
\end{itemize}
Designação scientífica da planta, vulgarmente conhecida por \textunderscore maravilhas\textunderscore , (\textunderscore calendula officinalis\textunderscore ).
\section{Calenduláceas}
\begin{itemize}
\item {Grp. gram.:f. pl.}
\end{itemize}
Família de plantas, a que serve de typo a calêndula.
\section{Calendulina}
\begin{itemize}
\item {Grp. gram.:f.}
\end{itemize}
Substância, que se extrai da calêndula.
\section{Calentura}
\begin{itemize}
\item {Grp. gram.:f.}
\end{itemize}
Accesso febril, de que enfermam ás vezes os navegantes, nas regiões intertropicaes.
(Cast. \textunderscore calentura\textunderscore )
\section{Calepino}
\begin{itemize}
\item {Grp. gram.:m.}
\end{itemize}
\begin{itemize}
\item {Utilização:P. us.}
\end{itemize}
\begin{itemize}
\item {Utilização:Fig.}
\end{itemize}
\begin{itemize}
\item {Proveniência:(De \textunderscore Calepino\textunderscore , n. p. do autor de um diccionário célebre)}
\end{itemize}
Vocabulário.
Livro de lembranças; agenda.
\section{Calete}
\begin{itemize}
\item {fónica:lê}
\end{itemize}
\begin{itemize}
\item {Grp. gram.:m.}
\end{itemize}
Qualidade, gênero, estôfo: \textunderscore não quero gente de tal calete\textunderscore . Cf. \textunderscore Aulegrafia\textunderscore , 44.
\section{Caleu}
\begin{itemize}
\item {Grp. gram.:m.}
\end{itemize}
Animal asiático. Cf. \textunderscore Peregrinação\textunderscore , LXXIII.
\section{Calha}
\begin{itemize}
\item {Grp. gram.:f.}
\end{itemize}
\begin{itemize}
\item {Proveniência:(Do lat. \textunderscore callis\textunderscore )}
\end{itemize}
Sulco, pequeno rêgo, aberto ou moldado em pedra, madeira, barro ou metal, para facilitar o curso de qualquer objecto.
Carril.
Cale.
Espaço, entre os paulitos do jôgo de bilhar.
Jôgo de rapazes, ou disco de loiça ou de ferro, que, tendo-se um pé levantado, se impelle com o outro pé, pelas divisões que, para esse jôgo, se traçam na terra.
\section{Calha-calha}
\begin{itemize}
\item {Grp. gram.:m.}
\end{itemize}
\begin{itemize}
\item {Utilização:Prov.}
\end{itemize}
\begin{itemize}
\item {Utilização:alent.}
\end{itemize}
Jôgo de rapazes.
\section{Calhadoiro}
\begin{itemize}
\item {Grp. gram.:m.}
\end{itemize}
\begin{itemize}
\item {Proveniência:(De \textunderscore calhar\textunderscore )}
\end{itemize}
Lugar, em que os jogadores da bola firmam os pés.
\section{Calhadouro}
\begin{itemize}
\item {Grp. gram.:m.}
\end{itemize}
\begin{itemize}
\item {Proveniência:(De \textunderscore calhar\textunderscore )}
\end{itemize}
Lugar, em que os jogadores da bola firmam os pés.
\section{Calhamaçada}
\begin{itemize}
\item {Grp. gram.:f.}
\end{itemize}
Porção de calhamaços.
\section{Calhamaço}
\begin{itemize}
\item {Grp. gram.:m.}
\end{itemize}
\begin{itemize}
\item {Utilização:Pop.}
\end{itemize}
\begin{itemize}
\item {Utilização:Chul.}
\end{itemize}
Livro grande e antigo.
Mulher gorda e feia.
(Corr. de \textunderscore canhamaço\textunderscore )
\section{Calhamandreiro}
\begin{itemize}
\item {Grp. gram.:m.}
\end{itemize}
\begin{itemize}
\item {Utilização:Prov.}
\end{itemize}
\begin{itemize}
\item {Utilização:trasm.}
\end{itemize}
Membro do partido legitimista.
\section{Calhambeque}
\begin{itemize}
\item {Grp. gram.:m.}
\end{itemize}
\begin{itemize}
\item {Utilização:Fam.}
\end{itemize}
Pequena embarcação costeira.
Traste velho; objecto de pouco valor.
\section{Calhambola}
\begin{itemize}
\item {Grp. gram.:m.  e  f.}
\end{itemize}
\begin{itemize}
\item {Utilização:Bras}
\end{itemize}
Pessôa escrava, que fugiu para o sertão.
(Do tupi)
\section{Calhandra}
\begin{itemize}
\item {Grp. gram.:f.}
\end{itemize}
\begin{itemize}
\item {Proveniência:(Gr. \textunderscore kalandra\textunderscore )}
\end{itemize}
Espécie de cotovia, de bico forte e vôo rasteiro.
\section{Calhandreira}
\begin{itemize}
\item {Grp. gram.:f.}
\end{itemize}
\begin{itemize}
\item {Utilização:Prov.}
\end{itemize}
Mulher, que despeja calhandros.
Prostituta. Cf. Camillo, \textunderscore Caveira\textunderscore , 467.
Bisbilhoteira. (Colhido em Turquel)
\section{Calhandreiro}
\begin{itemize}
\item {Grp. gram.:m.}
\end{itemize}
Aquelle que despeja calhandros.
\section{Calhandro}
\begin{itemize}
\item {Grp. gram.:m.}
\end{itemize}
Grande vaso cylíndrico, em que se juntam immundícies.
(Corr. de \textunderscore colondro\textunderscore ?)
\section{Calháo}
\begin{itemize}
\item {Grp. gram.:m.}
\end{itemize}
Pedaço de rocha dura.
Pedra solta.
(Cast. \textunderscore callao\textunderscore )
\section{Calhão}
\begin{itemize}
\item {Grp. gram.:m.}
\end{itemize}
\begin{itemize}
\item {Proveniência:(De \textunderscore calha\textunderscore )}
\end{itemize}
Segunda das divisões, que se traçam no chão, para o jôgo da calha.
\section{Calhar}
\begin{itemize}
\item {Grp. gram.:v. i.}
\end{itemize}
\begin{itemize}
\item {Utilização:Gír.}
\end{itemize}
Entrar em calha; correr pela calha.
Ajustar-se.
Vir a tempo; sêr opportuno.
Acontecer.
Agradar, aprazer.
\section{Calhariz}
\begin{itemize}
\item {Grp. gram.:f.}
\end{itemize}
Casta de uva.
\section{Calhastroz}
\begin{itemize}
\item {Grp. gram.:m.}
\end{itemize}
\begin{itemize}
\item {Utilização:Prov.}
\end{itemize}
\begin{itemize}
\item {Utilização:dur.}
\end{itemize}
Pessôa alta e desajeitada. Cf. Camillo, \textunderscore Narcót.\textunderscore , I, 229.
\section{Calhau}
\begin{itemize}
\item {Grp. gram.:m.}
\end{itemize}
Pedaço de rocha dura.
Pedra solta.
(Cast. \textunderscore callao\textunderscore )
\section{Calhe}
\begin{itemize}
\item {Grp. gram.:m.}
\end{itemize}
\begin{itemize}
\item {Utilização:Prov.}
\end{itemize}
\begin{itemize}
\item {Utilização:trasm.}
\end{itemize}
(V.calha)
Cale de madeira, que, na azenha, leva a água ás pennas do rodízio.
\section{Calheira}
\begin{itemize}
\item {Grp. gram.:f.}
\end{itemize}
\begin{itemize}
\item {Utilização:Prov.}
\end{itemize}
\begin{itemize}
\item {Utilização:trasm.}
\end{itemize}
Quelho do moínho.
\section{Calheiro}
\begin{itemize}
\item {Grp. gram.:m.}
\end{itemize}
\begin{itemize}
\item {Utilização:Prov.}
\end{itemize}
\begin{itemize}
\item {Utilização:trasm.}
\end{itemize}
O mesmo que \textunderscore calhe\textunderscore .
\section{Calhestro}
\begin{itemize}
\item {Grp. gram.:m.}
\end{itemize}
\begin{itemize}
\item {Utilização:Prov.}
\end{itemize}
Habilidade, jeito: \textunderscore o Manoel não tem calhestro para o officio\textunderscore .
Verosemelhança: \textunderscore Isso que dizes não tem calhestro\textunderscore .
(Colhido em Turquel)
\section{Calheta}
\begin{itemize}
\item {fónica:lhê}
\end{itemize}
\begin{itemize}
\item {Grp. gram.:f.}
\end{itemize}
\begin{itemize}
\item {Proveniência:(De \textunderscore cala\textunderscore )}
\end{itemize}
Angra estreita.
\section{Calhoada}
\begin{itemize}
\item {Grp. gram.:f.}
\end{itemize}
\begin{itemize}
\item {Utilização:Prov.}
\end{itemize}
\begin{itemize}
\item {Utilização:Prov.}
\end{itemize}
\begin{itemize}
\item {Utilização:beir.}
\end{itemize}
\begin{itemize}
\item {Proveniência:(De \textunderscore calhau\textunderscore )}
\end{itemize}
O mesmo que \textunderscore pedrada\textunderscore .
Monte de calhaus.
Quéda no solo ou sôbre pedras.
\section{Calhondro}
\begin{itemize}
\item {Grp. gram.:m.}
\end{itemize}
\begin{itemize}
\item {Utilização:Prov.}
\end{itemize}
O mesmo que \textunderscore colondro\textunderscore . Cf. \textunderscore Bibl. da G. do Campo\textunderscore , 245 e 271.
\section{Calhorda}
\begin{itemize}
\item {Grp. gram.:m.  e  adj.}
\end{itemize}
Homem desprezível, patife.
\section{Calhordas}
\begin{itemize}
\item {Grp. gram.:m.  e  adj.}
\end{itemize}
Homem desprezível, patife.
\section{Calhorra}
\begin{itemize}
\item {fónica:lhô}
\end{itemize}
\begin{itemize}
\item {Grp. gram.:f.}
\end{itemize}
\begin{itemize}
\item {Utilização:Prov.}
\end{itemize}
O mesmo que \textunderscore canhorra\textunderscore .
\section{Calhorro}
\begin{itemize}
\item {fónica:lhô}
\end{itemize}
\begin{itemize}
\item {Grp. gram.:adj.}
\end{itemize}
\begin{itemize}
\item {Utilização:T. da Guarda}
\end{itemize}
Atrevido.
Malandro.
\section{Calhostros}
\begin{itemize}
\item {fónica:lhôs}
\end{itemize}
\begin{itemize}
\item {Grp. gram.:m. pl.}
\end{itemize}
\begin{itemize}
\item {Utilização:Prov.}
\end{itemize}
\begin{itemize}
\item {Utilização:alent.}
\end{itemize}
Primeiro leite dos animaes.
\section{Cáli}
\begin{itemize}
\item {Grp. gram.:m.}
\end{itemize}
Planta ebenácea, de frutos comestíveis, (\textunderscore diospyros kali\textunderscore , Lin.).
\section{Cáli}
\begin{itemize}
\item {Grp. gram.:m.}
\end{itemize}
O mesmo que \textunderscore potassa\textunderscore .
\section{Caliastro}
\begin{itemize}
\item {Grp. gram.:m.}
\end{itemize}
\begin{itemize}
\item {Proveniência:(Do gr. \textunderscore kalos\textunderscore  + \textunderscore aster\textunderscore )}
\end{itemize}
Espécie de astéria, zoóphito.
\section{Calibrador}
\begin{itemize}
\item {Grp. gram.:m.}
\end{itemize}
\begin{itemize}
\item {Proveniência:(De \textunderscore calibrar\textunderscore )}
\end{itemize}
Padrão, com que se mede o calibre das bôcas de fogo ou de quaesquer tubos.
Apparelho agrícola, para separar as sementes, segundo o calibre ou grossura dellas.
\section{Calibragem}
\begin{itemize}
\item {Grp. gram.:f.}
\end{itemize}
\begin{itemize}
\item {Utilização:Agr.}
\end{itemize}
Acto de calibrar ou escolher os bagos de trigo, por meio de crivos ou tararas. Cf. \textunderscore Gaz. dos Lavr.\textunderscore , I, 17.
\section{Calibrar}
\begin{itemize}
\item {Grp. gram.:v. t.}
\end{itemize}
Medir o calibre de; dar o conveniente calibre a.
\section{Calibre}
\begin{itemize}
\item {Grp. gram.:m.}
\end{itemize}
\begin{itemize}
\item {Proveniência:(Do ár. \textunderscore kalab\textunderscore  + lat. \textunderscore libra\textunderscore ?)}
\end{itemize}
Diâmetro, capacidade, de um tubo.
Diâmetro da bala, determinado pelo diâmetro de uma bôca de fogo.
Dimensão; tamanho.
Calibrador.
Utensílio recortado, para moldar estuque.
\section{Caliça}
\begin{itemize}
\item {Grp. gram.:f.}
\end{itemize}
\begin{itemize}
\item {Utilização:Gír.}
\end{itemize}
\begin{itemize}
\item {Proveniência:(De \textunderscore cal\textunderscore )}
\end{itemize}
Fragmentos de argamassa, cal, etc.
Dinheiro em prata.
\section{Caliçada}
\begin{itemize}
\item {Grp. gram.:f.}
\end{itemize}
\begin{itemize}
\item {Utilização:Prov.}
\end{itemize}
Porção de caliça.
Pancada com caliço.
\section{Calicânti}
\begin{itemize}
\item {Grp. gram.:m.}
\end{itemize}
Árvore indiana, (\textunderscore acacia sundra\textunderscore ), semelhante ao pau-ferro, do qual se distingue apenas por têr menor duração e dimensão, e pelas fendas do seu invólucro cortical.
\section{Cálice}
\begin{itemize}
\item {Grp. gram.:m.}
\end{itemize}
\begin{itemize}
\item {Proveniência:(Lat. \textunderscore calix\textunderscore )}
\end{itemize}
Vaso, que serve na Missa, para a consagração do vinho.
Copo com pé.
\section{Cálice}
\begin{itemize}
\item {Grp. gram.:m.}
\end{itemize}
\begin{itemize}
\item {Proveniência:(Lat. \textunderscore calyx\textunderscore )}
\end{itemize}
Invólucro da flôr, que contém a corolla e os orgãos sexuaes.
\section{Calícera}
\begin{itemize}
\item {Grp. gram.:f.}
\end{itemize}
\begin{itemize}
\item {Proveniência:(Do gr. \textunderscore kalos\textunderscore  + \textunderscore keras\textunderscore )}
\end{itemize}
Insecto díptero.
\section{Calícero}
\begin{itemize}
\item {Grp. gram.:f.}
\end{itemize}
Insecto hymenóptero.
(Cp. \textunderscore calícera\textunderscore ^1)
\section{Calicifloras}
\begin{itemize}
\item {fónica:cá}
\end{itemize}
\begin{itemize}
\item {Grp. gram.:f. pl.}
\end{itemize}
\begin{itemize}
\item {Proveniência:(De \textunderscore cálice\textunderscore ^2 + \textunderscore flôr\textunderscore )}
\end{itemize}
Classe de plantas, cuja corolla polypétala é inserta com os estames sôbre o cálice.
\section{Caliciforme}
\begin{itemize}
\item {fónica:cá}
\end{itemize}
\begin{itemize}
\item {Grp. gram.:adj.}
\end{itemize}
\begin{itemize}
\item {Proveniência:(De \textunderscore cálice\textunderscore ^1 + \textunderscore fórma\textunderscore )}
\end{itemize}
Que tem fórma de cálice.
\section{Calicinal}
\begin{itemize}
\item {Grp. gram.:adj.}
\end{itemize}
O mesmo que \textunderscore calicinar\textunderscore .
\section{Calicinar}
\begin{itemize}
\item {Grp. gram.:adj.}
\end{itemize}
\begin{itemize}
\item {Proveniência:(De \textunderscore calícino\textunderscore )}
\end{itemize}
Relativo ao cálice das flôres.
\section{Calícino}
\begin{itemize}
\item {Grp. gram.:adj.}
\end{itemize}
\begin{itemize}
\item {Proveniência:(De \textunderscore cálice\textunderscore ^2)}
\end{itemize}
Relativo ao cálice das flôres.
\section{Caliço}
\begin{itemize}
\item {Grp. gram.:m.}
\end{itemize}
\begin{itemize}
\item {Utilização:Prov.}
\end{itemize}
\begin{itemize}
\item {Utilização:Prov.}
\end{itemize}
\begin{itemize}
\item {Utilização:alg.}
\end{itemize}
O mesmo que \textunderscore caliça\textunderscore .
Torrão.
Variedade de calcário tufoso, leve e resistente, empregado em construcções.
\section{Caliculado}
\begin{itemize}
\item {Grp. gram.:adj.}
\end{itemize}
Que tem calículo.
\section{Calículo}
\begin{itemize}
\item {Grp. gram.:m.}
\end{itemize}
\begin{itemize}
\item {Proveniência:(Lat. \textunderscore calyculus\textunderscore )}
\end{itemize}
Pequeno cálice das flôres.
\section{Calidade}
\begin{itemize}
\item {Grp. gram.:f.}
\end{itemize}
\begin{itemize}
\item {Utilização:Ant.}
\end{itemize}
Qualidade. Cf. \textunderscore Eufrosina\textunderscore , 146; Rui de Pina, etc.
\section{Cálido}
\begin{itemize}
\item {Grp. gram.:adj.}
\end{itemize}
\begin{itemize}
\item {Proveniência:(Lat. \textunderscore calidus\textunderscore )}
\end{itemize}
Quente.
Ardente; fogoso.
\section{Calidoscópio}
\begin{itemize}
\item {Grp. gram.:m.}
\end{itemize}
\begin{itemize}
\item {Proveniência:(Do gr. \textunderscore kalos\textunderscore  + \textunderscore eidos\textunderscore  + \textunderscore skopein\textunderscore )}
\end{itemize}
Instrumento de Phýsica, formado de pequenos espelhos inclinados, e que, a cada movimento, apresenta combinações variadas e agradáveis.
\section{Califa}
\begin{itemize}
\item {Grp. gram.:m.}
\end{itemize}
Chefe, soberano muçulmano.
(Ár. \textunderscore califa\textunderscore )
\section{Califado}
\begin{itemize}
\item {Grp. gram.:m.}
\end{itemize}
Jurisdicção de califa.
Território, governado por um califa.
Tempo que dura o govêrno de um califa.
\section{Calífero}
\begin{itemize}
\item {Grp. gram.:adj.}
\end{itemize}
\begin{itemize}
\item {Proveniência:(De \textunderscore cal\textunderscore  + lat. \textunderscore ferre\textunderscore )}
\end{itemize}
Diz-se do forno, em que fabricam a cal.
\section{Califórnia}
\begin{itemize}
\item {Grp. gram.:f.}
\end{itemize}
\begin{itemize}
\item {Utilização:T. de Barcelos}
\end{itemize}
Caverna profunda.
(Relaciona-se com \textunderscore furna\textunderscore ?)
\section{Californiano}
\begin{itemize}
\item {Grp. gram.:adj.}
\end{itemize}
\begin{itemize}
\item {Grp. gram.:M.}
\end{itemize}
Relativo á Califórnia.
Habitante da Califórnia.
\section{Califórnica}
\begin{itemize}
\item {Grp. gram.:f.}
\end{itemize}
\begin{itemize}
\item {Proveniência:(De \textunderscore Califórnia\textunderscore , n. p.)}
\end{itemize}
Espécie de videira, cujas uvas têm pequenos bagos.
\section{Califórnica}
\begin{itemize}
\item {Grp. gram.:f.}
\end{itemize}
\begin{itemize}
\item {Utilização:Prov.}
\end{itemize}
\begin{itemize}
\item {Utilização:alent.}
\end{itemize}
Carrêta, que póde despejar-se, soltando-se do cabeçalho o taboleiro.
\section{Califórnios}
\begin{itemize}
\item {Grp. gram.:m. pl.}
\end{itemize}
Indígenas da Califórnia.
\section{Cáliga}
\begin{itemize}
\item {Grp. gram.:f.}
\end{itemize}
\begin{itemize}
\item {Proveniência:(Lat. \textunderscore caliga\textunderscore )}
\end{itemize}
Sandália, guarnecida de prégos, usada pelos antigos soldados romanos.
\section{Caligante}
\begin{itemize}
\item {Grp. gram.:adj.}
\end{itemize}
\begin{itemize}
\item {Proveniência:(Lat. \textunderscore caligans\textunderscore )}
\end{itemize}
Que perturba a vista.
Que causa vertigens, (falando-se de coisas muito altas, donde mal se póde olhar para baixo):«\textunderscore janelas caligantes\textunderscore ». Filinto, IV, 242.
\section{Caligem}
\begin{itemize}
\item {Grp. gram.:f.}
\end{itemize}
\begin{itemize}
\item {Proveniência:(Lat. \textunderscore caligo\textunderscore , \textunderscore caliginis\textunderscore )}
\end{itemize}
Nevoeiro espêsso.
Escuridão; trevas.
Cataracta (dos olhos).
\section{Caliginoso}
\begin{itemize}
\item {Grp. gram.:adj.}
\end{itemize}
\begin{itemize}
\item {Proveniência:(Lat. \textunderscore caliginosus\textunderscore )}
\end{itemize}
Muito escuro, tenebroso.
\section{Calígula}
\begin{itemize}
\item {Grp. gram.:f.}
\end{itemize}
\begin{itemize}
\item {Proveniência:(Lat. \textunderscore caligula\textunderscore )}
\end{itemize}
Pelle, que cobre o tarso das aves.
\section{Calilu}
\begin{itemize}
\item {Grp. gram.:m.}
\end{itemize}
Iguaria picante de San-Thomé.
\section{Calim}
\begin{itemize}
\item {Grp. gram.:m.}
\end{itemize}
Liga de chumbo e estanho, feita na China.
(Cp. \textunderscore calaim\textunderscore )
\section{Calimba}
\begin{itemize}
\item {Grp. gram.:f.}
\end{itemize}
\begin{itemize}
\item {Utilização:Pesc.}
\end{itemize}
Primeiro enrêdo do saco, no apparelho da chávega.
(Cp. \textunderscore calimbé\textunderscore )
\section{Calimbá}
\begin{itemize}
\item {Grp. gram.:m.}
\end{itemize}
\begin{itemize}
\item {Utilização:Bras}
\end{itemize}
O mesmo que \textunderscore calumbá\textunderscore .
\section{Calimbé}
\begin{itemize}
\item {Grp. gram.:m.}
\end{itemize}
\begin{itemize}
\item {Proveniência:(T. afr.?)}
\end{itemize}
Espécie de tanga dos negros da Guiana.
\section{Calimburgo}
\begin{itemize}
\item {Grp. gram.:m.}
\end{itemize}
(Fórma incorrecta, em vez de \textunderscore calembur\textunderscore , mas usada por Garrett, Benalcanfor, etc.)
\section{Calime}
\begin{itemize}
\item {Grp. gram.:m.}
\end{itemize}
A parte delgada do navio, entre a linha de água e o gio grande.
Bóia, que se amarra no fundo do saco da rede de cercar e alar.
\section{Calimeira}
\begin{itemize}
\item {Grp. gram.:f.}
\end{itemize}
\begin{itemize}
\item {Proveniência:(De \textunderscore calime\textunderscore )}
\end{itemize}
Pequena lancha, que acompanha o copo da armação de pesca de atum e sardinha.
\section{Calimérida}
\begin{itemize}
\item {Grp. gram.:f.}
\end{itemize}
\begin{itemize}
\item {Proveniência:(Do gr. \textunderscore kalos\textunderscore  + \textunderscore meris\textunderscore )}
\end{itemize}
Planta siberiana, da fam. das compostas.
\section{Calinada}
\begin{itemize}
\item {Grp. gram.:f.}
\end{itemize}
Acto ou dito próprio de calino.
\section{Calinite}
\begin{itemize}
\item {Grp. gram.:f.}
\end{itemize}
Alúmen potássico ou pedra ume, que se emprega em Medicina, como adstringente.
\section{Calino}
\begin{itemize}
\item {Grp. gram.:m.  e  adj.}
\end{itemize}
\begin{itemize}
\item {Utilização:Neol.}
\end{itemize}
\begin{itemize}
\item {Proveniência:(Fr. \textunderscore calin\textunderscore )}
\end{itemize}
Estúpido; bronco.
Indolente.
\section{Calino}
\begin{itemize}
\item {Grp. gram.:adj.}
\end{itemize}
\begin{itemize}
\item {Utilização:Prov.}
\end{itemize}
\begin{itemize}
\item {Utilização:trasm.}
\end{itemize}
Quente.
\section{Cala}
\begin{itemize}
\item {Grp. gram.:f.}
\end{itemize}
Gênero de plantas aráceas, (\textunderscore calla palustris\textunderscore , Lin.).
\section{Calaico}
\begin{itemize}
\item {Grp. gram.:adj.}
\end{itemize}
\begin{itemize}
\item {Grp. gram.:M.}
\end{itemize}
\begin{itemize}
\item {Proveniência:(Lat. \textunderscore callaici\textunderscore )}
\end{itemize}
Relativo á Galécia, hoje Galiza e Minho.
Galego.
Habitante da Galécia. Cf. Herculano, \textunderscore Hist. de Port.\textunderscore  I, 15 e 43.
\section{Cale}
\begin{itemize}
\item {Grp. gram.:f.}
\end{itemize}
\begin{itemize}
\item {Utilização:Ant.}
\end{itemize}
\begin{itemize}
\item {Proveniência:(Lat. \textunderscore callis\textunderscore )}
\end{itemize}
Rua.
\section{Caleja}
\begin{itemize}
\item {Grp. gram.:f.}
\end{itemize}
\begin{itemize}
\item {Utilização:Prov.}
\end{itemize}
\begin{itemize}
\item {Utilização:trasm.}
\end{itemize}
\begin{itemize}
\item {Utilização:Ant.}
\end{itemize}
\begin{itemize}
\item {Proveniência:(T. cast.)}
\end{itemize}
Pequena rua; beco.
\section{Caleja}
\begin{itemize}
\item {Grp. gram.:f.}
\end{itemize}
O mesmo que \textunderscore calosidade\textunderscore . Cf. \textunderscore Viriato Trág.\textunderscore , I, 31.
\section{Calejado}
\begin{itemize}
\item {Grp. gram.:adj.}
\end{itemize}
\begin{itemize}
\item {Utilização:Fig.}
\end{itemize}
Que tem calos.
Experiente.
Matreiro.
\section{Calejador}
\begin{itemize}
\item {Grp. gram.:adj.}
\end{itemize}
Que caleja.
\section{Calejão}
\begin{itemize}
\item {Grp. gram.:m.}
\end{itemize}
\begin{itemize}
\item {Utilização:Ant.}
\end{itemize}
\begin{itemize}
\item {Proveniência:(De \textunderscore calleja\textunderscore )}
\end{itemize}
Rua larga.
\section{Calejar}
\begin{itemize}
\item {Grp. gram.:v. t.}
\end{itemize}
\begin{itemize}
\item {Proveniência:(De \textunderscore callo\textunderscore )}
\end{itemize}
Produzir calos em; tornar caloso.
Endurecer; tornar insensível.
\section{Calejo}
\begin{itemize}
\item {Grp. gram.:m.}
\end{itemize}
\begin{itemize}
\item {Utilização:Prov.}
\end{itemize}
\begin{itemize}
\item {Utilização:beir.}
\end{itemize}
Pequena rua, viela, travessa.
(Cp. \textunderscore calleja\textunderscore ^1)
\section{Cália}
\begin{itemize}
\item {Grp. gram.:f.}
\end{itemize}
\begin{itemize}
\item {Proveniência:(Do gr. \textunderscore kallos\textunderscore )}
\end{itemize}
Insecto tetrâmero, longicórneo.
\section{Caliandra}
\begin{itemize}
\item {Grp. gram.:f.}
\end{itemize}
\begin{itemize}
\item {Utilização:Bot.}
\end{itemize}
Planta do Amazonas.
\section{Calianira}
\begin{itemize}
\item {Grp. gram.:f.}
\end{itemize}
\begin{itemize}
\item {Proveniência:(De \textunderscore Callianira\textunderscore , n. p.)}
\end{itemize}
Zoóphito gelatinoso e transparente.
\section{Caliântemo}
\begin{itemize}
\item {Grp. gram.:m.}
\end{itemize}
\begin{itemize}
\item {Proveniência:(Do gr. \textunderscore kallos\textunderscore  + \textunderscore anthemon\textunderscore )}
\end{itemize}
Planta ranunculácea, vivaz.
\section{Calicarpo}
\begin{itemize}
\item {Grp. gram.:m.}
\end{itemize}
\begin{itemize}
\item {Proveniência:(Do gr. \textunderscore kallos\textunderscore  + \textunderscore karpos\textunderscore )}
\end{itemize}
Gênero de plantas verbenáceas.
\section{Calichromo}
\begin{itemize}
\item {Grp. gram.:adj.}
\end{itemize}
\begin{itemize}
\item {Grp. gram.:M. pl.}
\end{itemize}
\begin{itemize}
\item {Proveniência:(Do gr. \textunderscore kallos\textunderscore  + \textunderscore khroma\textunderscore )}
\end{itemize}
Que tem côres belas.
Gênero de pássaros, de formosas côres.
\section{Calicida}
\begin{itemize}
\item {Grp. gram.:m.}
\end{itemize}
\begin{itemize}
\item {Proveniência:(Do lat. \textunderscore callum\textunderscore  + \textunderscore caedere\textunderscore )}
\end{itemize}
Medicamento, que destrói calos.
Coricida.
\section{Calicromo}
\begin{itemize}
\item {Grp. gram.:adj.}
\end{itemize}
\begin{itemize}
\item {Grp. gram.:M. pl.}
\end{itemize}
\begin{itemize}
\item {Proveniência:(Do gr. \textunderscore kallos\textunderscore  + \textunderscore khroma\textunderscore )}
\end{itemize}
Que tem côres bellas.
Gênero de pássaros, de formosas côres.
\section{Calídeos}
\begin{itemize}
\item {Grp. gram.:m. pl.}
\end{itemize}
\begin{itemize}
\item {Proveniência:(Do gr. \textunderscore kallos\textunderscore  + \textunderscore eidos\textunderscore )}
\end{itemize}
Gênero de insectos longicórneos, a que pertence a cália.
\section{Cálido}
\begin{itemize}
\item {Grp. gram.:adj.}
\end{itemize}
\begin{itemize}
\item {Proveniência:(Lat. \textunderscore callidus\textunderscore )}
\end{itemize}
Astuto, sagaz. Cp. Garção, II, 12.
\section{Califonia}
\begin{itemize}
\item {Grp. gram.:f.}
\end{itemize}
\begin{itemize}
\item {Proveniência:(Do gr. \textunderscore kallos\textunderscore  + \textunderscore phone\textunderscore )}
\end{itemize}
Beleza da voz.
\section{Caligrafia}
\begin{itemize}
\item {Grp. gram.:f.}
\end{itemize}
Arte de bem escrever á mão.
Maneira de escrever.
(Cp. \textunderscore callígrapho\textunderscore )
\section{Caligráfico}
\begin{itemize}
\item {Grp. gram.:adj.}
\end{itemize}
Relativo á caligrafia: \textunderscore caprichos caligráficos\textunderscore .
\section{Calígrafo}
\begin{itemize}
\item {Grp. gram.:m.}
\end{itemize}
\begin{itemize}
\item {Proveniência:(Do gr. \textunderscore kallos\textunderscore  + \textunderscore graphein\textunderscore )}
\end{itemize}
Aquele que sabe caligrafia.
Aquele que a ensina.
\section{Calilogia}
\begin{itemize}
\item {Grp. gram.:f.}
\end{itemize}
\begin{itemize}
\item {Proveniência:(Do gr. \textunderscore kallos\textunderscore  + \textunderscore logos\textunderscore )}
\end{itemize}
Beleza ou elegância de expressão.
\section{Calínico}
\begin{itemize}
\item {Grp. gram.:m.}
\end{itemize}
\begin{itemize}
\item {Proveniência:(Do gr. \textunderscore kallos\textunderscore  + \textunderscore nike\textunderscore )}
\end{itemize}
Bailado grego, que se executava ao som de frautas.
\section{Calipedia}
\begin{itemize}
\item {Grp. gram.:f.}
\end{itemize}
\begin{itemize}
\item {Proveniência:(Gr. \textunderscore kallipaidía\textunderscore )}
\end{itemize}
Conjunto de preceitos ou conselhos, para a procriação de filhos formosos.
\section{Calipédico}
\begin{itemize}
\item {Grp. gram.:adj.}
\end{itemize}
Relativo á calipedia.
\section{Calipígico}
\begin{itemize}
\item {Grp. gram.:adj.}
\end{itemize}
O mesmo que \textunderscore calipígio\textunderscore .
\section{Calipígio}
\begin{itemize}
\item {Grp. gram.:adj.}
\end{itemize}
\begin{itemize}
\item {Proveniência:(Do gr. \textunderscore kallos\textunderscore  + \textunderscore puge\textunderscore )}
\end{itemize}
Que tem formosas nádegas, (epitheto de Vênus).
\section{Caliptra}
\begin{itemize}
\item {Grp. gram.:f.}
\end{itemize}
\begin{itemize}
\item {Proveniência:(Lat. \textunderscore caliptra\textunderscore )}
\end{itemize}
Espécie de touca, usada pelas mulheres da antiga Roma.
\section{Calique}
\begin{itemize}
\item {Grp. gram.:m.}
\end{itemize}
\begin{itemize}
\item {Utilização:Gír.}
\end{itemize}
Dinheiro.
\section{Calismo}
\begin{itemize}
\item {Grp. gram.:m.}
\end{itemize}
\begin{itemize}
\item {Utilização:Med.}
\end{itemize}
\begin{itemize}
\item {Proveniência:(Do ár. \textunderscore cali\textunderscore )}
\end{itemize}
Conjunto dos accidentes mórbidos, causados pela potassa.
\section{Calista}
\begin{itemize}
\item {Grp. gram.:m.}
\end{itemize}
Aquele que cura ou extrai calos; pedicuro.
\section{Calístefo}
\begin{itemize}
\item {Grp. gram.:m.}
\end{itemize}
Gênero de plantas compostas, originárias da China ou Japão.
\section{Calistenia}
\begin{itemize}
\item {Grp. gram.:f.}
\end{itemize}
\begin{itemize}
\item {Proveniência:(Do gr. \textunderscore kallos\textunderscore  + \textunderscore sthenos\textunderscore )}
\end{itemize}
Collecção de preceitos gymnásticos, para uso do sexo feminino.
\section{Calisto}
\begin{itemize}
\item {Grp. gram.:m.}
\end{itemize}
\begin{itemize}
\item {Utilização:Fam.}
\end{itemize}
\begin{itemize}
\item {Proveniência:(De um n. p.?)}
\end{itemize}
Indivíduo, a cuja presença o jogador infeliz attribue a sua má sorte.
\section{Calite}
\begin{itemize}
\item {Grp. gram.:m.}
\end{itemize}
\begin{itemize}
\item {Utilização:Prov.}
\end{itemize}
\begin{itemize}
\item {Utilização:minh.}
\end{itemize}
Qualidade, casta, o mesmo que \textunderscore calete\textunderscore : \textunderscore esta fruta é de mau calite\textunderscore .
(Cp. lat. \textunderscore qualitas\textunderscore )
\section{Calítrico}
\begin{itemize}
\item {Grp. gram.:m.}
\end{itemize}
\begin{itemize}
\item {Proveniência:(Gr. \textunderscore kallítrikos\textunderscore )}
\end{itemize}
Gênero de fêtos.
\section{Cálix}
\begin{itemize}
\item {Grp. gram.:m.}
\end{itemize}
O mesmo que \textunderscore cálice\textunderscore ^1.
\section{Caliz}
\begin{itemize}
\item {Grp. gram.:m.}
\end{itemize}
\begin{itemize}
\item {Utilização:Bras}
\end{itemize}
\begin{itemize}
\item {Proveniência:(De \textunderscore cale\textunderscore )}
\end{itemize}
Cale de madeira, usada em engenhos de açúcar.
\section{Calla}
\begin{itemize}
\item {Grp. gram.:f.}
\end{itemize}
Gênero de plantas aráceas, (\textunderscore calla palustris\textunderscore , Lin.).
\section{Callaico}
\begin{itemize}
\item {Grp. gram.:adj.}
\end{itemize}
\begin{itemize}
\item {Grp. gram.:M.}
\end{itemize}
\begin{itemize}
\item {Proveniência:(Lat. \textunderscore callaici\textunderscore )}
\end{itemize}
Relativo á Gallécia, hoje Galliza e Minho.
Gallego.
Habitante da Gallécia. Cf. Herculano, \textunderscore Hist. de Port.\textunderscore  I, 15 e 43.
\section{Calle}
\begin{itemize}
\item {Grp. gram.:f.}
\end{itemize}
\begin{itemize}
\item {Utilização:Ant.}
\end{itemize}
\begin{itemize}
\item {Proveniência:(Lat. \textunderscore callis\textunderscore )}
\end{itemize}
Rua.
\section{Calleja}
\begin{itemize}
\item {Grp. gram.:f.}
\end{itemize}
\begin{itemize}
\item {Utilização:Prov.}
\end{itemize}
\begin{itemize}
\item {Utilização:trasm.}
\end{itemize}
\begin{itemize}
\item {Utilização:Ant.}
\end{itemize}
\begin{itemize}
\item {Proveniência:(T. cast.)}
\end{itemize}
Pequena rua; beco.
\section{Calleja}
\begin{itemize}
\item {Grp. gram.:f.}
\end{itemize}
O mesmo que \textunderscore callosidade\textunderscore . Cf. \textunderscore Viriato Trág.\textunderscore , I, 31.
\section{Callejado}
\begin{itemize}
\item {Grp. gram.:adj.}
\end{itemize}
\begin{itemize}
\item {Utilização:Fig.}
\end{itemize}
Que tem callos.
Experiente.
Matreiro.
\section{Callejador}
\begin{itemize}
\item {Grp. gram.:adj.}
\end{itemize}
Que calleja.
\section{Callejão}
\begin{itemize}
\item {Grp. gram.:m.}
\end{itemize}
\begin{itemize}
\item {Utilização:Ant.}
\end{itemize}
\begin{itemize}
\item {Proveniência:(De \textunderscore calleja\textunderscore )}
\end{itemize}
Rua larga.
\section{Callejar}
\begin{itemize}
\item {Grp. gram.:v. t.}
\end{itemize}
\begin{itemize}
\item {Proveniência:(De \textunderscore callo\textunderscore )}
\end{itemize}
Produzir callos em; tornar calloso.
Endurecer; tornar insensível.
\section{Callejo}
\begin{itemize}
\item {Grp. gram.:m.}
\end{itemize}
\begin{itemize}
\item {Utilização:Prov.}
\end{itemize}
\begin{itemize}
\item {Utilização:beir.}
\end{itemize}
Pequena rua, viella, travessa.
(Cp. \textunderscore calleja\textunderscore ^1)
\section{Cállia}
\begin{itemize}
\item {Grp. gram.:f.}
\end{itemize}
\begin{itemize}
\item {Proveniência:(Do gr. \textunderscore kallos\textunderscore )}
\end{itemize}
Insecto tetrâmero, longicórneo.
\section{Calliandra}
\begin{itemize}
\item {Grp. gram.:f.}
\end{itemize}
\begin{itemize}
\item {Utilização:Bot.}
\end{itemize}
Planta do Amazonas.
\section{Callianira}
\begin{itemize}
\item {Grp. gram.:f.}
\end{itemize}
\begin{itemize}
\item {Proveniência:(De \textunderscore Callianira\textunderscore , n. p.)}
\end{itemize}
Zoóphito gelatinoso e transparente.
\section{Calliânthemo}
\begin{itemize}
\item {Grp. gram.:m.}
\end{itemize}
\begin{itemize}
\item {Proveniência:(Do gr. \textunderscore kallos\textunderscore  + \textunderscore anthemon\textunderscore )}
\end{itemize}
Planta ranunculácea, vivaz.
\section{Callicarpo}
\begin{itemize}
\item {Grp. gram.:m.}
\end{itemize}
\begin{itemize}
\item {Proveniência:(Do gr. \textunderscore kallos\textunderscore  + \textunderscore karpos\textunderscore )}
\end{itemize}
Gênero de plantas verbenáceas.
\section{Callicida}
\begin{itemize}
\item {Grp. gram.:m.}
\end{itemize}
\begin{itemize}
\item {Proveniência:(Do lat. \textunderscore callum\textunderscore  + \textunderscore caedere\textunderscore )}
\end{itemize}
Medicamento, que destrói callos.
Coricida.
\section{Callídeos}
\begin{itemize}
\item {Grp. gram.:m. pl.}
\end{itemize}
\begin{itemize}
\item {Proveniência:(Do gr. \textunderscore kallos\textunderscore  + \textunderscore eidos\textunderscore )}
\end{itemize}
Gênero de insectos longicórneos, a que pertence a cállia.
\section{Cállido}
\begin{itemize}
\item {Grp. gram.:adj.}
\end{itemize}
\begin{itemize}
\item {Proveniência:(Lat. \textunderscore callidus\textunderscore )}
\end{itemize}
Astuto, sagaz. Cp. Garção, II, 12.
\section{Calligraphia}
\begin{itemize}
\item {Grp. gram.:f.}
\end{itemize}
Arte de bem escrever á mão.
Maneira de escrever.
(Cp. \textunderscore callígrapho\textunderscore )
\section{Calligráphico}
\begin{itemize}
\item {Grp. gram.:adj.}
\end{itemize}
Relativo á calligraphia: \textunderscore caprichos calligráphicos\textunderscore .
\section{Callígrapho}
\begin{itemize}
\item {Grp. gram.:m.}
\end{itemize}
\begin{itemize}
\item {Proveniência:(Do gr. \textunderscore kallos\textunderscore  + \textunderscore graphein\textunderscore )}
\end{itemize}
Aquelle que sabe calligraphia.
Aquelle que a ensina.
\section{Callilogia}
\begin{itemize}
\item {Grp. gram.:f.}
\end{itemize}
\begin{itemize}
\item {Proveniência:(Do gr. \textunderscore kallos\textunderscore  + \textunderscore logos\textunderscore )}
\end{itemize}
Belleza ou elegância de expressão.
\section{Callínico}
\begin{itemize}
\item {Grp. gram.:m.}
\end{itemize}
\begin{itemize}
\item {Proveniência:(Do gr. \textunderscore kallos\textunderscore  + \textunderscore nike\textunderscore )}
\end{itemize}
Bailado grego, que se executava ao som de frautas.
\section{Callipedia}
\begin{itemize}
\item {Grp. gram.:f.}
\end{itemize}
\begin{itemize}
\item {Proveniência:(Gr. \textunderscore kallipaidía\textunderscore )}
\end{itemize}
Conjunto de preceitos ou conselhos, para a procriação de filhos formosos.
\section{Callipédico}
\begin{itemize}
\item {Grp. gram.:adj.}
\end{itemize}
Relativo á callipedia.
\section{Calliphonia}
\begin{itemize}
\item {Grp. gram.:f.}
\end{itemize}
\begin{itemize}
\item {Proveniência:(Do gr. \textunderscore kallos\textunderscore  + \textunderscore phone\textunderscore )}
\end{itemize}
Belleza da voz.
\section{Callipýgico}
\begin{itemize}
\item {Grp. gram.:adj.}
\end{itemize}
O mesmo que \textunderscore callipýgio\textunderscore .
\section{Callipýgio}
\begin{itemize}
\item {Grp. gram.:adj.}
\end{itemize}
\begin{itemize}
\item {Proveniência:(Do gr. \textunderscore kallos\textunderscore  + \textunderscore puge\textunderscore )}
\end{itemize}
Que tem formosas nádegas, (epitheto de Vênus).
\section{Callista}
\begin{itemize}
\item {Grp. gram.:m.}
\end{itemize}
Aquelle que cura ou extrai callos; pedicuro.
\section{Callístepho}
\begin{itemize}
\item {Grp. gram.:m.}
\end{itemize}
Gênero de plantas compostas, originárias da China ou Japão.
\section{Callisthenia}
\begin{itemize}
\item {Grp. gram.:f.}
\end{itemize}
\begin{itemize}
\item {Proveniência:(Do gr. \textunderscore kallos\textunderscore  + \textunderscore sthenos\textunderscore )}
\end{itemize}
Collecção de preceitos gymnásticos, para uso do sexo feminino.
\section{Callisto}
\begin{itemize}
\item {Grp. gram.:m.}
\end{itemize}
\begin{itemize}
\item {Utilização:Fam.}
\end{itemize}
\begin{itemize}
\item {Proveniência:(De um n. p.?)}
\end{itemize}
Indivíduo, a cuja presença o jogador infeliz attribue a sua má sorte.
\section{Callítrico}
\begin{itemize}
\item {Grp. gram.:m.}
\end{itemize}
\begin{itemize}
\item {Proveniência:(Gr. \textunderscore kallítrikos\textunderscore )}
\end{itemize}
Gênero de fêtos.
\section{Callo}
\begin{itemize}
\item {Grp. gram.:m.}
\end{itemize}
\begin{itemize}
\item {Utilização:Prov.}
\end{itemize}
\begin{itemize}
\item {Utilização:alent.}
\end{itemize}
\begin{itemize}
\item {Utilização:Fig.}
\end{itemize}
\begin{itemize}
\item {Utilização:Prov.}
\end{itemize}
\begin{itemize}
\item {Utilização:dur.}
\end{itemize}
\begin{itemize}
\item {Proveniência:(Lat. \textunderscore callum\textunderscore )}
\end{itemize}
Endurecimento da pelle em determinado ponto, por compressão ou fricção continuada.
Pequeno tumor, duro e circunscrito, nos tornozelos ou nos dedos dos pés.
Substância dura, que une os ossos fracturados.
Extensão de terreno argilloso, entre terrenos de outra formação.
Indifferença, insensibilidade, proveniente do hábito.
Endurecimento, formado em roda de videiras pelas raizes das varas que se cortaram na poda.
Extremidade dos ramos da ferradura.
\section{Callocéphalo}
\begin{itemize}
\item {Grp. gram.:adj.}
\end{itemize}
\begin{itemize}
\item {Grp. gram.:M.}
\end{itemize}
\begin{itemize}
\item {Utilização:Zool.}
\end{itemize}
Que tem cabeça formosa.
Mammífero, da ordem das phocas.
\section{Callódemo}
\begin{itemize}
\item {Grp. gram.:m.}
\end{itemize}
\begin{itemize}
\item {Proveniência:(Do gr. \textunderscore kallos\textunderscore  + \textunderscore demas\textunderscore )}
\end{itemize}
Insecto coleóptero da Austrália.
\section{Calloeira}
\begin{itemize}
\item {Grp. gram.:f.}
\end{itemize}
\begin{itemize}
\item {Utilização:Prov.}
\end{itemize}
\begin{itemize}
\item {Utilização:trasm.}
\end{itemize}
\begin{itemize}
\item {Proveniência:(De \textunderscore callo\textunderscore )}
\end{itemize}
Tumor no pescoço dos bois, causado pela molhelha.
\section{Callómetro}
\begin{itemize}
\item {Grp. gram.:m.}
\end{itemize}
\begin{itemize}
\item {Proveniência:(Do gr. \textunderscore kallos\textunderscore  + \textunderscore metron\textunderscore )}
\end{itemize}
Supposta medida de belleza.
\section{Callongo}
\begin{itemize}
\item {Grp. gram.:m.}
\end{itemize}
\begin{itemize}
\item {Proveniência:(T. lund.)}
\end{itemize}
Árvore angolense.
\section{Callosidade}
\begin{itemize}
\item {Grp. gram.:f.}
\end{itemize}
\begin{itemize}
\item {Proveniência:(Lat. \textunderscore callositas\textunderscore )}
\end{itemize}
Dureza callosa.
Qualidade daquillo que tem callos.
\section{Calloso}
\begin{itemize}
\item {Grp. gram.:adj.}
\end{itemize}
\begin{itemize}
\item {Proveniência:(Lat. \textunderscore callosus\textunderscore )}
\end{itemize}
Que tem callos.
\section{Calluna}
\begin{itemize}
\item {Grp. gram.:f.}
\end{itemize}
\begin{itemize}
\item {Proveniência:(Do gr. \textunderscore kallunein\textunderscore , varrer)}
\end{itemize}
Gênero de plantas ericáceas.
\section{Calma}
\begin{itemize}
\item {Grp. gram.:f.}
\end{itemize}
Calor da atmosphera.
Hora do dia, em que há mais calor.
Calmaria, cessação de agitação no mar.
Serenidade.
(B. lat. \textunderscore cauma\textunderscore , do gr. \textunderscore kauma\textunderscore )
\section{Calmândulas}
\begin{itemize}
\item {Grp. gram.:f. pl.}
\end{itemize}
(V.camândulas)Cf. Camillo, \textunderscore Brilh. do Bras\textunderscore .
\section{Calmante}
\begin{itemize}
\item {Grp. gram.:adj.}
\end{itemize}
\begin{itemize}
\item {Grp. gram.:M.}
\end{itemize}
\begin{itemize}
\item {Proveniência:(De \textunderscore calmar\textunderscore ^1)}
\end{itemize}
Que acalma, que abranda, que tranquilliza.
Medicamento, que mitiga dores ou excitações nervosas.
\section{Calmão}
\begin{itemize}
\item {Grp. gram.:adj.}
\end{itemize}
\begin{itemize}
\item {Utilização:Ant.}
\end{itemize}
\begin{itemize}
\item {Proveniência:(De \textunderscore calmo\textunderscore )}
\end{itemize}
Que está em calmaria, (falando-se do mar).
\section{Calmão}
\begin{itemize}
\item {Grp. gram.:m.}
\end{itemize}
\begin{itemize}
\item {Proveniência:(T. da Guiné port.)}
\end{itemize}
Meia cabaça, cortada longitudinalmente, para tirar água de uma vasilha para outra, ou do fundo dos barcos.
\section{Calmar}
\begin{itemize}
\item {Grp. gram.:v. t.}
\end{itemize}
O mesmo que \textunderscore acalmar\textunderscore .
\section{Calmar}
\begin{itemize}
\item {Grp. gram.:m.}
\end{itemize}
\begin{itemize}
\item {Proveniência:(Do lat. \textunderscore calamarius\textunderscore )}
\end{itemize}
Mollusco cephalópode, conhecido vulgarmente por \textunderscore chôco\textunderscore .
\section{Calmar}
\begin{itemize}
\item {Grp. gram.:v. t.}
\end{itemize}
\begin{itemize}
\item {Utilização:Gír.}
\end{itemize}
Dar pancadas em.
(Cp. \textunderscore calmar\textunderscore ^1)
\section{Calmaria}
\begin{itemize}
\item {Grp. gram.:f.}
\end{itemize}
\begin{itemize}
\item {Utilização:Fig.}
\end{itemize}
\begin{itemize}
\item {Proveniência:(De \textunderscore calma\textunderscore )}
\end{itemize}
Cessação do vento e do movimento das ondas.
Calma.
Calor sem vento.
Ausência de notícias e de factos importantes; tranquillidade geral.
\section{Calmeirão}
\begin{itemize}
\item {Grp. gram.:adj.}
\end{itemize}
\begin{itemize}
\item {Utilização:Gír.}
\end{itemize}
\begin{itemize}
\item {Grp. gram.:M.}
\end{itemize}
\begin{itemize}
\item {Utilização:Prov.}
\end{itemize}
\begin{itemize}
\item {Proveniência:(De \textunderscore calmo\textunderscore )}
\end{itemize}
Preguiçoso.
Indivíduo corpulento, mas mollangueirão e pouco atilado. (Colhido em Turquel)
\section{Calmeiro}
\begin{itemize}
\item {Grp. gram.:adj.}
\end{itemize}
\begin{itemize}
\item {Utilização:Bras}
\end{itemize}
Diz-se da embarcação que navega com pouco vento, quási com calmaria.
\section{Calmo}
\begin{itemize}
\item {Grp. gram.:adj.}
\end{itemize}
\begin{itemize}
\item {Proveniência:(De \textunderscore calma\textunderscore )}
\end{itemize}
Quente, calmoso.
Que está em calmaria, sossegado.
\section{Calmoroso}
\begin{itemize}
\item {Grp. gram.:adj.}
\end{itemize}
O mesmo que \textunderscore calmoso\textunderscore . Cf. Macedo, \textunderscore Meditação\textunderscore , 124.
\section{Calmorrear}
\begin{itemize}
\item {Grp. gram.:v. t.}
\end{itemize}
\begin{itemize}
\item {Utilização:Chul.}
\end{itemize}
Espancar.
(Cp. \textunderscore calmar\textunderscore ^3)
\section{Calmoso}
\begin{itemize}
\item {Grp. gram.:adj.}
\end{itemize}
Em que há calma; quente (falando-se do tempo ou do ar).
\section{Calmuco}
\begin{itemize}
\item {Grp. gram.:m.}
\end{itemize}
Língua uralo-altaica da Ásia setentrional.
\section{Calo}
\begin{itemize}
\item {Grp. gram.:m.}
\end{itemize}
\begin{itemize}
\item {Utilização:Prov.}
\end{itemize}
\begin{itemize}
\item {Utilização:alent.}
\end{itemize}
\begin{itemize}
\item {Utilização:Fig.}
\end{itemize}
\begin{itemize}
\item {Utilização:Prov.}
\end{itemize}
\begin{itemize}
\item {Utilização:dur.}
\end{itemize}
\begin{itemize}
\item {Proveniência:(Lat. \textunderscore callum\textunderscore )}
\end{itemize}
Endurecimento da pele em determinado ponto, por compressão ou fricção continuada.
Pequeno tumor, duro e circunscrito, nos tornozelos ou nos dedos dos pés.
Substância dura, que une os ossos fracturados.
Extensão de terreno argiloso, entre terrenos de outra formação.
Indiferença, insensibilidade, proveniente do hábito.
Endurecimento, formado em roda de videiras pelas raizes das varas que se cortaram na poda.
Extremidade dos ramos da ferradura.
\section{Caló}
\begin{itemize}
\item {Grp. gram.:m.}
\end{itemize}
Linguagem dos ciganos da Espanha.
Nome, que os ciganos da Espanha dão a qualquer indivíduo masculino da sua raça.
Pl. \textunderscore calés\textunderscore .
Fem. sing. \textunderscore calli\textunderscore , pl. \textunderscore callias\textunderscore .
\section{Caló}
\begin{itemize}
\item {Grp. gram.:m.}
\end{itemize}
Variedade de arroz indiano.
\section{Calocéfalo}
\begin{itemize}
\item {Grp. gram.:adj.}
\end{itemize}
\begin{itemize}
\item {Grp. gram.:M.}
\end{itemize}
\begin{itemize}
\item {Utilização:Zool.}
\end{itemize}
Que tem cabeça formosa.
Mamífero, da ordem das focas.
\section{Calódemo}
\begin{itemize}
\item {Grp. gram.:m.}
\end{itemize}
\begin{itemize}
\item {Proveniência:(Do gr. \textunderscore kallos\textunderscore  + \textunderscore demas\textunderscore )}
\end{itemize}
Insecto coleóptero da Austrália.
\section{Caloeira}
\begin{itemize}
\item {Grp. gram.:f.}
\end{itemize}
\begin{itemize}
\item {Utilização:Prov.}
\end{itemize}
\begin{itemize}
\item {Utilização:trasm.}
\end{itemize}
\begin{itemize}
\item {Proveniência:(De \textunderscore callo\textunderscore )}
\end{itemize}
Tumor no pescoço dos bois, causado pela molhelha.
\section{Calões}
\begin{itemize}
\item {Grp. gram.:m. pl.}
\end{itemize}
\begin{itemize}
\item {Utilização:T. de Aveiro}
\end{itemize}
Extremidades dos panos de rede, ligados á bôca da rede de pescar e alar.
\section{Caloête}
\begin{itemize}
\item {Grp. gram.:m.}
\end{itemize}
\begin{itemize}
\item {Utilização:Ant.}
\end{itemize}
Pau, para empalar.
(Do malaialim \textunderscore kaluekki\textunderscore )
\section{Calofilo}
\begin{itemize}
\item {Grp. gram.:adj.}
\end{itemize}
\begin{itemize}
\item {Utilização:Bot.}
\end{itemize}
\begin{itemize}
\item {Proveniência:(Do gr. \textunderscore kalos\textunderscore  ou \textunderscore kallos\textunderscore  + \textunderscore phullon\textunderscore )}
\end{itemize}
Que tem fôlhas belas.
\section{Calogio}
\begin{itemize}
\item {Grp. gram.:m.}
\end{itemize}
\begin{itemize}
\item {Utilização:Bras}
\end{itemize}
Quarto escuso, para entrevistas amorosas.
Zungu.
(Cp. \textunderscore caloji\textunderscore )
\section{Caló-gondó}
\begin{itemize}
\item {Grp. gram.:m.}
\end{itemize}
Pequena árvore da Índia portuguesa.
\section{Çaloio}
\begin{itemize}
\item {Grp. gram.:m.  e  adj.}
\end{itemize}
\begin{itemize}
\item {Utilização:Fig.}
\end{itemize}
\begin{itemize}
\item {Utilização:Ant.}
\end{itemize}
\begin{itemize}
\item {Proveniência:(De \textunderscore çaloio\textunderscore , nome ár. de um tributo que em Lisbôa pagavam os padeiros moiros)}
\end{itemize}
Camponês ou aldeão dos arrabaldes de Lisbôa.
Aldeão.
Grosseiro.
Rústico.
Finório, velhaco.
Diz-se de um pão, feito de uma variedade de trigo durázio, que se cultiva perto de Lisbôa.
Moiro, originário de Salé.
\section{Caloira}
\begin{itemize}
\item {Grp. gram.:f.}
\end{itemize}
\begin{itemize}
\item {Utilização:Prov.}
\end{itemize}
\begin{itemize}
\item {Utilização:trasm.}
\end{itemize}
O mesmo que \textunderscore preguiça\textunderscore .
\section{Caloirada}
\begin{itemize}
\item {Grp. gram.:f.}
\end{itemize}
\begin{itemize}
\item {Utilização:Deprec.}
\end{itemize}
Turba de caloiros; os caloiros.
\section{Caloirice}
\begin{itemize}
\item {Grp. gram.:f.}
\end{itemize}
\begin{itemize}
\item {Utilização:Fam.}
\end{itemize}
Ingenuidade ou tolice de caloiro.
\section{Caloiro}
\begin{itemize}
\item {Grp. gram.:m.}
\end{itemize}
Estudante de disciplinas preparatórias.
Aquelle que é noviço em qualquer coisa.
Aquelle que é acanhado, lorpa.
\section{Caloji}
\begin{itemize}
\item {Grp. gram.:m.}
\end{itemize}
\begin{itemize}
\item {Utilização:Bras}
\end{itemize}
O mesmo que \textunderscore zungu\textunderscore .
\section{Calolo}
\begin{itemize}
\item {Grp. gram.:m.}
\end{itemize}
Árvore angolense, (\textunderscore phoenix spinosa\textunderscore ).
\section{Calombo}
\begin{itemize}
\item {Grp. gram.:m.}
\end{itemize}
\begin{itemize}
\item {Utilização:Bras}
\end{itemize}
\begin{itemize}
\item {Utilização:Bras}
\end{itemize}
Tumor, inchaço duro, em qualquer parte do corpo.
Qualquer líquido coalhado em granulações.
(Talvez or. afr.)
\section{Calombro}
\begin{itemize}
\item {Grp. gram.:m.}
\end{itemize}
\begin{itemize}
\item {Utilização:Prov.}
\end{itemize}
O mesmo que \textunderscore colondro\textunderscore . Cf. \textunderscore Bibl. da G. do Campo\textunderscore , 271.
\section{Calomel}
\begin{itemize}
\item {Grp. gram.:m.}
\end{itemize}
\begin{itemize}
\item {Proveniência:(Do gr. \textunderscore kalos\textunderscore  ou \textunderscore kallos\textunderscore  + \textunderscore melas\textunderscore )}
\end{itemize}
Protochloreto de mercúrio, na antiga nomenclatura chímica.
\section{Calomelanos}
\begin{itemize}
\item {Grp. gram.:m. pl.}
\end{itemize}
\begin{itemize}
\item {Proveniência:(De \textunderscore calomel\textunderscore )}
\end{itemize}
Sub-chloreto de mercúrio.
\section{Calómetro}
\begin{itemize}
\item {Grp. gram.:m.}
\end{itemize}
\begin{itemize}
\item {Proveniência:(Do gr. \textunderscore kallos\textunderscore  + \textunderscore metron\textunderscore )}
\end{itemize}
Suposta medida de beleza.
\section{Calona}
\begin{itemize}
\item {Grp. gram.:f.}
\end{itemize}
\begin{itemize}
\item {Utilização:Gír.}
\end{itemize}
\begin{itemize}
\item {Proveniência:(De \textunderscore calão\textunderscore , cigano)}
\end{itemize}
Mulher desprezível.
\section{Calonde}
\begin{itemize}
\item {Grp. gram.:m.}
\end{itemize}
Arbusto leguminoso de Angola.
\section{Calondro}
\begin{itemize}
\item {Grp. gram.:m.}
\end{itemize}
\begin{itemize}
\item {Utilização:Prov.}
\end{itemize}
\begin{itemize}
\item {Utilização:trasm.}
\end{itemize}
O mesmo que \textunderscore colondro\textunderscore .
\section{Calongo}
\begin{itemize}
\item {Grp. gram.:m.}
\end{itemize}
\begin{itemize}
\item {Proveniência:(T. lund.)}
\end{itemize}
Árvore angolense.
\section{Calophyllo}
\begin{itemize}
\item {Grp. gram.:adj.}
\end{itemize}
\begin{itemize}
\item {Utilização:Bot.}
\end{itemize}
\begin{itemize}
\item {Proveniência:(Do gr. \textunderscore kalos\textunderscore  ou \textunderscore kallos\textunderscore  + \textunderscore phullon\textunderscore )}
\end{itemize}
Que tem fôlhas bellas.
\section{Calóptero}
\begin{itemize}
\item {Grp. gram.:adj.}
\end{itemize}
\begin{itemize}
\item {Proveniência:(Do gr. \textunderscore kalos\textunderscore  ou \textunderscore kallos\textunderscore  + \textunderscore pteron\textunderscore )}
\end{itemize}
Que tem bellas asas.
\section{Caloqueio}
\begin{itemize}
\item {Grp. gram.:m.}
\end{itemize}
Pássaro dentirostro da África.
\section{Calor}
\begin{itemize}
\item {Grp. gram.:m.}
\end{itemize}
\begin{itemize}
\item {Utilização:Fig.}
\end{itemize}
\begin{itemize}
\item {Utilização:Gír.}
\end{itemize}
\begin{itemize}
\item {Proveniência:(Lat. \textunderscore calor\textunderscore )}
\end{itemize}
Qualidade daquillo que está quente, ou de quem está quente.
Sensação, que experimentamos, na proximidade de um corpo quente.
Elevação de temperatura, produzida pelo sol.
Vivacidade, animação: \textunderscore falar com calor\textunderscore .
Sova, tunda, critica ou reprehensão severa: \textunderscore apanhou um calor\textunderscore .
\section{Caloria}
\begin{itemize}
\item {Grp. gram.:f.}
\end{itemize}
Unidade, com que, em Phýsica, se mede a quantidade de calor.
\section{Caloricidade}
\begin{itemize}
\item {Grp. gram.:f.}
\end{itemize}
Faculdade, que têm os corpos vivos, de desenvolver certa quantidade de calórico.
\section{Calórico}
\begin{itemize}
\item {Grp. gram.:m.}
\end{itemize}
\begin{itemize}
\item {Proveniência:(De \textunderscore calor\textunderscore )}
\end{itemize}
Princípio do calor, isto é, propriedade, que a matéria tem, de, em certas condições, se manifestar pelo calor.
\section{Calorífero}
\begin{itemize}
\item {Grp. gram.:adj.}
\end{itemize}
\begin{itemize}
\item {Grp. gram.:M.}
\end{itemize}
\begin{itemize}
\item {Proveniência:(Do lat. \textunderscore calor\textunderscore  + \textunderscore ferre\textunderscore )}
\end{itemize}
Que tem calor.
Maquinismo, com que se aquece uma casa, compartimentos de combóio, etc.
Fogão de sala.
\section{Calorificação}
\begin{itemize}
\item {Grp. gram.:f.}
\end{itemize}
\begin{itemize}
\item {Proveniência:(De \textunderscore calorificar\textunderscore )}
\end{itemize}
Desenvolvimento de calor nos corpos vivos.
\section{Calorificar}
\begin{itemize}
\item {Grp. gram.:v. t.}
\end{itemize}
\begin{itemize}
\item {Proveniência:(Do lat. \textunderscore calor\textunderscore  + \textunderscore facere\textunderscore )}
\end{itemize}
Communicar calor a.
\section{Calorífico}
\begin{itemize}
\item {Grp. gram.:adj.}
\end{itemize}
\begin{itemize}
\item {Grp. gram.:M.}
\end{itemize}
\begin{itemize}
\item {Proveniência:(Lat. \textunderscore calorificus\textunderscore )}
\end{itemize}
Que tem a fôrça de produzir calor.
Apparelho, que produz calor.
\section{Calorífugo}
\begin{itemize}
\item {Grp. gram.:adj.}
\end{itemize}
\begin{itemize}
\item {Proveniência:(Do lat. \textunderscore calor\textunderscore  + \textunderscore fugere\textunderscore )}
\end{itemize}
Que evita o calor.
\section{Calorimetria}
\begin{itemize}
\item {Grp. gram.:f.}
\end{itemize}
Parte da Phýsica, que se occupa da medição do calórico livre.
(Cp. \textunderscore calorímetro\textunderscore )
\section{Calorimétrico}
\begin{itemize}
\item {Grp. gram.:adj.}
\end{itemize}
Relativo á calorimetria.
\section{Calorímetro}
\begin{itemize}
\item {Grp. gram.:m.}
\end{itemize}
\begin{itemize}
\item {Proveniência:(Do lat. \textunderscore calor\textunderscore  + gr. \textunderscore metron\textunderscore )}
\end{itemize}
Instrumento, com que se mede o calórico específico de um corpo.
\section{Calorosamente}
\begin{itemize}
\item {Grp. gram.:adv.}
\end{itemize}
\begin{itemize}
\item {Proveniência:(De \textunderscore caloroso\textunderscore )}
\end{itemize}
Com calor, com enthusiasmo.
\section{Caloroso}
\begin{itemize}
\item {Grp. gram.:adj.}
\end{itemize}
\begin{itemize}
\item {Proveniência:(De \textunderscore calor\textunderscore )}
\end{itemize}
Calmoso.
Enérgico, enthusiasta.
\section{Calosidade}
\begin{itemize}
\item {Grp. gram.:f.}
\end{itemize}
\begin{itemize}
\item {Proveniência:(Lat. \textunderscore callositas\textunderscore )}
\end{itemize}
Dureza calosa.
Qualidade daquilo que tem calos.
\section{Caloso}
\begin{itemize}
\item {Grp. gram.:adj.}
\end{itemize}
\begin{itemize}
\item {Proveniência:(Lat. \textunderscore callosus\textunderscore )}
\end{itemize}
Que tem calos.
\section{Calota}
\begin{itemize}
\item {Grp. gram.:f.}
\end{itemize}
\begin{itemize}
\item {Utilização:Mathem.}
\end{itemize}
Parte de uma esphera ou cylindro, comprehendida entre planos parallelos.
(B. lat. \textunderscore calota\textunderscore )
\section{Calote}
\begin{itemize}
\item {Grp. gram.:m.}
\end{itemize}
\begin{itemize}
\item {Utilização:Fam.}
\end{itemize}
Dívida, que se não pagou ou que se contrai sem tenção de a pagar.
\section{Calotear}
\begin{itemize}
\item {Grp. gram.:v. t.}
\end{itemize}
\begin{itemize}
\item {Grp. gram.:V. i.}
\end{itemize}
\begin{itemize}
\item {Proveniência:(De \textunderscore calote\textunderscore )}
\end{itemize}
Não pagar o que deve a.
Fazer calotes.
\section{Caloteirismo}
\begin{itemize}
\item {Grp. gram.:m.}
\end{itemize}
Hábito de caloteiro.
\section{Caloteiro}
\begin{itemize}
\item {Grp. gram.:m.}
\end{itemize}
Aquelle que caloteia.
\section{Calotismo}
\begin{itemize}
\item {Grp. gram.:m.}
\end{itemize}
(V.caloteirismo)
\section{Calouro}
\begin{itemize}
\item {Grp. gram.:m.}
\end{itemize}
Estudante de disciplinas preparatórias.
Aquelle que é noviço em qualquer coisa.
Aquelle que é acanhado, lorpa.
\section{Calpa}
\begin{itemize}
\item {Grp. gram.:f.}
\end{itemize}
\begin{itemize}
\item {Utilização:Bot.}
\end{itemize}
\begin{itemize}
\item {Proveniência:(Gr. \textunderscore kalpe\textunderscore )}
\end{itemize}
Urna dos musgos.
Vaso, urna funerária. Cf. \textunderscore Viriato Trág.\textunderscore , IV, 114.
\section{Cálpar}
\begin{itemize}
\item {Grp. gram.:m.}
\end{itemize}
\begin{itemize}
\item {Proveniência:(Lat. \textunderscore calpar\textunderscore )}
\end{itemize}
Vinho novo, que os Romanos offereciam a Júpiter, antes de o provar.
Vaso romano para líquidos, especialmente para vinho.
\section{Calpúrnia}
\begin{itemize}
\item {Grp. gram.:f.}
\end{itemize}
\begin{itemize}
\item {Proveniência:(De \textunderscore Calpúrnio\textunderscore , n. p.)}
\end{itemize}
Arbusto leguminoso da Índia e Cabo da Bôa-Esperança.
\section{Calque}
\begin{itemize}
\item {Grp. gram.:m.}
\end{itemize}
O mesmo que \textunderscore calco\textunderscore .
\section{Calraxo}
\begin{itemize}
\item {Grp. gram.:m.}
\end{itemize}
Erva dos prados, tambem conhecida por \textunderscore língua-de-ovelha\textunderscore .
\section{Calta}
\begin{itemize}
\item {Grp. gram.:f.}
\end{itemize}
\begin{itemize}
\item {Proveniência:(Lat. \textunderscore caltha\textunderscore )}
\end{itemize}
Planta ranunculácea, de flôres amarelas.
\section{Cal'-te!}
\begin{itemize}
\item {Grp. gram.:interj.}
\end{itemize}
\begin{itemize}
\item {Utilização:Prov.}
\end{itemize}
(Indicação de ameaça)
(Por \textunderscore cala-te\textunderscore , de \textunderscore calar\textunderscore )
\section{Calte!}
\begin{itemize}
\item {Grp. gram.:interj.}
\end{itemize}
\begin{itemize}
\item {Utilização:Prov.}
\end{itemize}
(Indicação de ameaça)
(Por \textunderscore cala-te\textunderscore , de \textunderscore calar\textunderscore )
\section{Caltha}
\begin{itemize}
\item {Grp. gram.:f.}
\end{itemize}
\begin{itemize}
\item {Proveniência:(Lat. \textunderscore caltha\textunderscore )}
\end{itemize}
Planta ranunculácea, de flôres amarelas.
\section{Cálthula}
\begin{itemize}
\item {Grp. gram.:f.}
\end{itemize}
\begin{itemize}
\item {Proveniência:(Lat. \textunderscore calthula\textunderscore )}
\end{itemize}
Vestido amarelo, usado por matronas romanas.
\section{Cáltula}
\begin{itemize}
\item {Grp. gram.:f.}
\end{itemize}
\begin{itemize}
\item {Proveniência:(Lat. \textunderscore calthula\textunderscore )}
\end{itemize}
Vestido amarelo, usado por matronas romanas.
\section{Caluda!}
\begin{itemize}
\item {Grp. gram.:interj.}
\end{itemize}
\begin{itemize}
\item {Proveniência:(De \textunderscore calar\textunderscore )}
\end{itemize}
(para impor silêncio)
\section{Caluête}
\begin{itemize}
\item {Grp. gram.:m.}
\end{itemize}
\begin{itemize}
\item {Utilização:Ant.}
\end{itemize}
Pau, para empalar.
(Do malaialim \textunderscore kaluekki\textunderscore )
\section{Caluga}
\begin{itemize}
\item {Grp. gram.:f.}
\end{itemize}
Cachaço e espádua do porco.
(Cp. \textunderscore caluva\textunderscore )
\section{Caluiana}
\begin{itemize}
\item {Grp. gram.:m.}
\end{itemize}
Uma das línguas da África occidental.
\section{Calum}
\begin{itemize}
\item {Grp. gram.:m.}
\end{itemize}
Casta de uva.
\section{Calumba}
\begin{itemize}
\item {Grp. gram.:f.}
\end{itemize}
Planta menispérmácea, medicinal, (\textunderscore menispermum palmatum\textunderscore , Lin.).
\section{Calumbá}
\begin{itemize}
\item {Grp. gram.:m.}
\end{itemize}
\begin{itemize}
\item {Utilização:Bras}
\end{itemize}
Suco, que se extraíu da cana.
Cocho do caldo, nos engenhos de açúcar.
\section{Calumba-do-brasil}
\begin{itemize}
\item {Grp. gram.:f.}
\end{itemize}
Planta rutácea, medicinal, (\textunderscore simaba calumba\textunderscore ).
\section{Calumberembe}
\begin{itemize}
\item {Grp. gram.:m.}
\end{itemize}
Reptil angolense.
\section{Calíbio}
\begin{itemize}
\item {Grp. gram.:m.}
\end{itemize}
\begin{itemize}
\item {Proveniência:(Gr. \textunderscore kalubion\textunderscore )}
\end{itemize}
Fruto, em fórma de cápsula.
\section{Calibita}
\begin{itemize}
\item {Grp. gram.:m.}
\end{itemize}
\begin{itemize}
\item {Proveniência:(Gr. \textunderscore kalubites\textunderscore )}
\end{itemize}
Christão, que vivia insulado em choça.
\section{Calicândria}
\begin{itemize}
\item {Grp. gram.:f.}
\end{itemize}
\begin{itemize}
\item {Proveniência:(Do gr. \textunderscore kalux\textunderscore  + \textunderscore aner\textunderscore , \textunderscore andros\textunderscore )}
\end{itemize}
Gênero de plantas, que tem mais de dez estames insertos no cálice.
\section{Calicantáceas}
\begin{itemize}
\item {Grp. gram.:f. pl.}
\end{itemize}
O mesmo ou melhor que \textunderscore calicânteas\textunderscore .
\section{Calicânteas}
\begin{itemize}
\item {Grp. gram.:f. pl.}
\end{itemize}
\begin{itemize}
\item {Proveniência:(De \textunderscore calycantho\textunderscore )}
\end{itemize}
Ordem de plantas, que compreende as onagrárias e outras, e tem por tipo o calicanto.
\section{Calicântemo}
\begin{itemize}
\item {Grp. gram.:adj.}
\end{itemize}
\begin{itemize}
\item {Proveniência:(Do gr. \textunderscore kalux\textunderscore  + \textunderscore anthema\textunderscore )}
\end{itemize}
Que tem cálice semelhante á corola.
\section{Calicanto}
\begin{itemize}
\item {Grp. gram.:m.}
\end{itemize}
\begin{itemize}
\item {Proveniência:(Do gr. \textunderscore kalux\textunderscore  + \textunderscore anthos\textunderscore )}
\end{itemize}
Formosa planta, originária da América do Norte.
\section{Calícera}
\begin{itemize}
\item {Grp. gram.:f.}
\end{itemize}
\begin{itemize}
\item {Proveniência:(Do gr. \textunderscore kalux\textunderscore  + \textunderscore keras\textunderscore )}
\end{itemize}
Gênero de plantas dicotiledóneas.
\section{Calicéreas}
\begin{itemize}
\item {Grp. gram.:f. pl.}
\end{itemize}
Fam. de plantas, que têm por tipo a \textunderscore calícera\textunderscore .
\section{Caliptérios}
\begin{itemize}
\item {Grp. gram.:m. pl.}
\end{itemize}
\begin{itemize}
\item {Proveniência:(Gr. \textunderscore kalupterion\textunderscore )}
\end{itemize}
Pennas curtas na parte inferior da cauda das aves.
\section{Caliptra}
\begin{itemize}
\item {Grp. gram.:f.}
\end{itemize}
\begin{itemize}
\item {Utilização:Bot.}
\end{itemize}
\begin{itemize}
\item {Proveniência:(Gr. \textunderscore kaluptra\textunderscore )}
\end{itemize}
Cápsula, que apresentam certas plantas, como os musgos.
\section{Caliptrado}
\begin{itemize}
\item {Grp. gram.:adj.}
\end{itemize}
\begin{itemize}
\item {Utilização:Bot.}
\end{itemize}
Que tem caliptra.
\section{Calistégia}
\begin{itemize}
\item {Grp. gram.:f.}
\end{itemize}
\begin{itemize}
\item {Proveniência:(Do gr. \textunderscore kalux\textunderscore  + \textunderscore stegein\textunderscore )}
\end{itemize}
Gênero de plantas convolvuláceas.
\section{Calumbi}
\begin{itemize}
\item {Grp. gram.:m.}
\end{itemize}
Planta rosácea de Angola, (\textunderscore rubus pinnatus\textunderscore , Ol.?).
\section{Calúmnia}
\begin{itemize}
\item {Grp. gram.:f.}
\end{itemize}
\begin{itemize}
\item {Proveniência:(Lat. \textunderscore calumnia\textunderscore )}
\end{itemize}
Imputação falsa, offensiva da reputação e crédito de alguém.
Os calumniadores: \textunderscore calou-se a calúmnia\textunderscore .
\section{Calumniado}
\begin{itemize}
\item {Grp. gram.:adj.}
\end{itemize}
Que é objecto de calúmnia.
\section{Calumniador}
\begin{itemize}
\item {Grp. gram.:m.}
\end{itemize}
Aquelle que calumnía.
\section{Calumniar}
\begin{itemize}
\item {Grp. gram.:v. t.}
\end{itemize}
\begin{itemize}
\item {Proveniência:(Lat. \textunderscore calumniari\textunderscore )}
\end{itemize}
Offender com calúmnias; diffamar, fazendo accusações falsas.
\section{Calumniável}
\begin{itemize}
\item {Grp. gram.:adj.}
\end{itemize}
\begin{itemize}
\item {Proveniência:(De \textunderscore calumniar\textunderscore )}
\end{itemize}
Que póde sêr objecto de calúmnia.
\section{Calumniosamente}
\begin{itemize}
\item {Grp. gram.:adv.}
\end{itemize}
De modo calumnioso.
\section{Calumnioso}
\begin{itemize}
\item {Grp. gram.:adj.}
\end{itemize}
\begin{itemize}
\item {Proveniência:(Lat. \textunderscore calumniosus\textunderscore )}
\end{itemize}
Que contém calúmnia.
Que serve para calumniar.
\section{Caluna}
\begin{itemize}
\item {Grp. gram.:f.}
\end{itemize}
\begin{itemize}
\item {Proveniência:(Do gr. \textunderscore kallunein\textunderscore , varrer)}
\end{itemize}
Gênero de plantas ericáceas.
\section{Calundu}
\begin{itemize}
\item {Grp. gram.:m.}
\end{itemize}
\begin{itemize}
\item {Utilização:Bras}
\end{itemize}
Mau humor.
Arrufo.
Irascibilidade.
(Talvez t. afr.)
\section{Calunga}
\begin{itemize}
\item {Grp. gram.:f.}
\end{itemize}
\begin{itemize}
\item {Grp. gram.:M.}
\end{itemize}
\begin{itemize}
\item {Utilização:Bras}
\end{itemize}
\begin{itemize}
\item {Utilização:Bras. da Baía}
\end{itemize}
\begin{itemize}
\item {Utilização:Bras. do Rio}
\end{itemize}
\begin{itemize}
\item {Utilização:Bras. de Cabofrio}
\end{itemize}
Planta rutácea do Brasil.
Boneco.
Ratoneiro.
Qualquer rato pequeno, murganho.
Ratinho preto do mato.
Peixe, o mesmo que \textunderscore pargo\textunderscore .
\section{Calungage}
\begin{itemize}
\item {Grp. gram.:f.}
\end{itemize}
\begin{itemize}
\item {Utilização:Bras}
\end{itemize}
\begin{itemize}
\item {Proveniência:(De \textunderscore calunga\textunderscore )}
\end{itemize}
O mesmo que \textunderscore vadiagem\textunderscore .
\section{Calungueira}
\begin{itemize}
\item {Grp. gram.:f.}
\end{itemize}
\begin{itemize}
\item {Utilização:Bras}
\end{itemize}
Barco de pesca no alto mar.
(Talvez do t. afr. \textunderscore calunga\textunderscore , mar)
\section{Calungueiro}
\begin{itemize}
\item {Grp. gram.:m.  e  adj.}
\end{itemize}
\begin{itemize}
\item {Utilização:Bras. do Rio}
\end{itemize}
Pescador de pargo.
\section{Calúnia}
\begin{itemize}
\item {Grp. gram.:f.}
\end{itemize}
\begin{itemize}
\item {Proveniência:(Lat. \textunderscore calumnia\textunderscore )}
\end{itemize}
Imputação falsa, ofensiva da reputação e crédito de alguém.
Os caluniadores: \textunderscore calou-se a calúnia\textunderscore .
\section{Caluniado}
\begin{itemize}
\item {Grp. gram.:adj.}
\end{itemize}
Que é objecto de calúnia.
\section{Caluniador}
\begin{itemize}
\item {Grp. gram.:m.}
\end{itemize}
Aquele que calunía.
\section{Caluniar}
\begin{itemize}
\item {Grp. gram.:v. t.}
\end{itemize}
\begin{itemize}
\item {Proveniência:(Lat. \textunderscore calumniari\textunderscore )}
\end{itemize}
Ofender com calúnias; difamar, fazendo acusações falsas.
\section{Caluniável}
\begin{itemize}
\item {Grp. gram.:adj.}
\end{itemize}
\begin{itemize}
\item {Proveniência:(De \textunderscore calumniar\textunderscore )}
\end{itemize}
Que póde sêr objecto de calúnia.
\section{Caluniosamente}
\begin{itemize}
\item {Grp. gram.:adv.}
\end{itemize}
De modo calunioso.
\section{Calunioso}
\begin{itemize}
\item {Grp. gram.:adj.}
\end{itemize}
\begin{itemize}
\item {Proveniência:(Lat. \textunderscore calumniosus\textunderscore )}
\end{itemize}
Que contém calúnia.
Que serve para caluniar.
\section{Calurange}
\begin{itemize}
\item {Grp. gram.:m.}
\end{itemize}
Árvore umbellífera medicinal, de Angola.
\section{Calussangé}
\begin{itemize}
\item {Grp. gram.:m.}
\end{itemize}
Árvore burserácea de Angola, (\textunderscore commiphora longebracteata\textunderscore , Engl.).
\section{Calussange}
\begin{itemize}
\item {Grp. gram.:m.}
\end{itemize}
Planta angolense, medicinal, da fam. das umbellíferas.
\section{Caluva}
\begin{itemize}
\item {Grp. gram.:f.}
\end{itemize}
\begin{itemize}
\item {Utilização:Prov.}
\end{itemize}
\begin{itemize}
\item {Utilização:beir.}
\end{itemize}
O mesmo que \textunderscore caluga\textunderscore .
(Por \textunderscore colluva\textunderscore , de \textunderscore collo\textunderscore ? ou por \textunderscore calluva\textunderscore , de \textunderscore callo?\textunderscore )
\section{Calva}
\begin{itemize}
\item {Grp. gram.:f.}
\end{itemize}
\begin{itemize}
\item {Utilização:Fig.}
\end{itemize}
\begin{itemize}
\item {Proveniência:(Lat. \textunderscore calva\textunderscore )}
\end{itemize}
Parte da cabeça, donde caiu o cabello.
Clareira.
Defeitos, culpas: \textunderscore pôr-lhe a calva á mostra\textunderscore .
\section{Calvar}
\begin{itemize}
\item {Grp. gram.:v. t.}
\end{itemize}
(V.calvejar)
\section{Calvário}
\begin{itemize}
\item {Grp. gram.:m.}
\end{itemize}
\begin{itemize}
\item {Utilização:Fig.}
\end{itemize}
\begin{itemize}
\item {Proveniência:(Lat. \textunderscore calvarium\textunderscore )}
\end{itemize}
Lugar da crucificação de Christo.
Elevação, representativa dêsse lugar.
Moéda de prata, no tempo de D. João III, com o valor de 400 reis.
Trabalhos.
Martýrio.
Qualquer elevação de terreno, diffícil de subir. Cf. Rebello, \textunderscore Mocidade\textunderscore , III, 44.
\section{Calveira}
\begin{itemize}
\item {Grp. gram.:f.}
\end{itemize}
(V.caveira)
\section{Calvejar}
\begin{itemize}
\item {Grp. gram.:v. t.}
\end{itemize}
\begin{itemize}
\item {Grp. gram.:V. i.}
\end{itemize}
Fazer calvo.
Tornar-se calvo.
\section{Calvez}
\begin{itemize}
\item {Grp. gram.:f.}
\end{itemize}
\begin{itemize}
\item {Proveniência:(Lat. \textunderscore calvities\textunderscore )}
\end{itemize}
Estado de quem é calvo.
\section{Calvície}
\begin{itemize}
\item {Grp. gram.:f.}
\end{itemize}
\begin{itemize}
\item {Proveniência:(Lat. \textunderscore calvities\textunderscore )}
\end{itemize}
Estado de quem é calvo.
\section{Calvil-urimo}
\begin{itemize}
\item {Grp. gram.:m.}
\end{itemize}
\begin{itemize}
\item {Proveniência:(T. lund.)}
\end{itemize}
Árvore angolense, rica em borracha.
\section{Calvim}
\begin{itemize}
\item {Grp. gram.:m.}
\end{itemize}
\begin{itemize}
\item {Utilização:Ant.}
\end{itemize}
Manilha, para conducção de água.
\section{Calvinismo}
\begin{itemize}
\item {Grp. gram.:m.}
\end{itemize}
Systema religioso, fundado por Calvino.
\section{Calvinista}
\begin{itemize}
\item {Grp. gram.:m.}
\end{itemize}
Sectário do calvinismo.
\section{Calvo}
\begin{itemize}
\item {Grp. gram.:adj.}
\end{itemize}
\begin{itemize}
\item {Utilização:Fig.}
\end{itemize}
\begin{itemize}
\item {Grp. gram.:M.}
\end{itemize}
\begin{itemize}
\item {Proveniência:(Lat. \textunderscore calvus\textunderscore )}
\end{itemize}
Que não tem cabellos na cabeça ou em parte della.
Mal dissimulado, evidente: \textunderscore intriga muito calva\textunderscore .
Diz-se de uma variedade de pêssego, de casca lisa e glabra, sem adherência da polpa ao caroço.
Indivíduo calvo.
\section{Calvura}
\begin{itemize}
\item {Grp. gram.:f.}
\end{itemize}
O mesmo que \textunderscore calvície\textunderscore . Cf. Usque, \textunderscore Tribulações\textunderscore , 21, V.^o
\section{Calýbio}
\begin{itemize}
\item {Grp. gram.:m.}
\end{itemize}
\begin{itemize}
\item {Proveniência:(Gr. \textunderscore kalubion\textunderscore )}
\end{itemize}
Fruto, em fórma de cápsula.
\section{Calybita}
\begin{itemize}
\item {Grp. gram.:m.}
\end{itemize}
\begin{itemize}
\item {Proveniência:(Gr. \textunderscore kalubites\textunderscore )}
\end{itemize}
Christão, que vivia insulado em choça.
\section{Calycândria}
\begin{itemize}
\item {Grp. gram.:f.}
\end{itemize}
\begin{itemize}
\item {Proveniência:(Do gr. \textunderscore kalux\textunderscore  + \textunderscore aner\textunderscore , \textunderscore andros\textunderscore )}
\end{itemize}
Gênero de plantas, que tem mais de dez estames insertos no cálice.
\section{Calycantháceas}
\begin{itemize}
\item {Grp. gram.:f. pl.}
\end{itemize}
O mesmo ou melhor que \textunderscore calycântheas\textunderscore .
\section{Calycântheas}
\begin{itemize}
\item {Grp. gram.:f. pl.}
\end{itemize}
\begin{itemize}
\item {Proveniência:(De \textunderscore calycantho\textunderscore )}
\end{itemize}
Ordem de plantas, que comprehende as onagrárias e outras, e tem por typo o calycantho.
\section{Calycânthemo}
\begin{itemize}
\item {Grp. gram.:adj.}
\end{itemize}
\begin{itemize}
\item {Proveniência:(Do gr. \textunderscore kalux\textunderscore  + \textunderscore anthema\textunderscore )}
\end{itemize}
Que tem cálice semelhante á corolla.
\section{Calycantho}
\begin{itemize}
\item {Grp. gram.:m.}
\end{itemize}
\begin{itemize}
\item {Proveniência:(Do gr. \textunderscore kalux\textunderscore  + \textunderscore anthos\textunderscore )}
\end{itemize}
Formosa planta, originária da América do Norte.
\section{Cályce}
\begin{itemize}
\item {Grp. gram.:m.}
\end{itemize}
\begin{itemize}
\item {Proveniência:(Lat. \textunderscore calyx\textunderscore )}
\end{itemize}
Invólucro da flôr, que contém a corolla e os orgãos sexuaes.
\section{Calýcera}
\begin{itemize}
\item {Grp. gram.:f.}
\end{itemize}
\begin{itemize}
\item {Proveniência:(Do gr. \textunderscore kalux\textunderscore  + \textunderscore keras\textunderscore )}
\end{itemize}
Gênero de plantas dicotyledóneas.
\section{Calycéreas}
\begin{itemize}
\item {Grp. gram.:f. pl.}
\end{itemize}
Fam. de plantas, que têm por typo a \textunderscore calýcera\textunderscore .
\section{Calyptérios}
\begin{itemize}
\item {Grp. gram.:m. pl.}
\end{itemize}
\begin{itemize}
\item {Proveniência:(Gr. \textunderscore kalupterion\textunderscore )}
\end{itemize}
Pennas curtas na parte inferior da cauda das aves.
\section{Calyptra}
\begin{itemize}
\item {Grp. gram.:f.}
\end{itemize}
\begin{itemize}
\item {Utilização:Bot.}
\end{itemize}
\begin{itemize}
\item {Proveniência:(Gr. \textunderscore kaluptra\textunderscore )}
\end{itemize}
Cápsula, que apresentam certas plantas, como os musgos.
\section{Calyptrado}
\begin{itemize}
\item {Grp. gram.:adj.}
\end{itemize}
\begin{itemize}
\item {Utilização:Bot.}
\end{itemize}
Que tem calyptra.
\section{Calystégia}
\begin{itemize}
\item {Grp. gram.:f.}
\end{itemize}
\begin{itemize}
\item {Proveniência:(Do gr. \textunderscore kalux\textunderscore  + \textunderscore stegein\textunderscore )}
\end{itemize}
Gênero de plantas convolvuláceas.
\section{Calyx}
\begin{itemize}
\item {Grp. gram.:m.}
\end{itemize}
O mesmo que \textunderscore cálice\textunderscore ^2.
\section{Cama}
\begin{itemize}
\item {Grp. gram.:f.}
\end{itemize}
\begin{itemize}
\item {Utilização:Prov.}
\end{itemize}
\begin{itemize}
\item {Utilização:dur.}
\end{itemize}
\begin{itemize}
\item {Utilização:Bras}
\end{itemize}
Objecto ou objectos, sôbre que alguém ou um animal se deita, ou sôbre que se póde deitar.
Móvel, em que habitualmente se dorme; leito; enxêrga, colchão.
Barra do leito.
Conjunto de coisas macias e flexíveis, sôbre que se collocam objectos melindrosos ou frágeis.
Camada.
Pequena elevação de terra lavrada, para certas sementeiras.
Effeito de acamar.
O mesmo que \textunderscore mergulhia\textunderscore .
Leito do rio.
(Med. lat. \textunderscore cama\textunderscore . Cp. gr. \textunderscore khamai\textunderscore )
\section{Camacari}
\begin{itemize}
\item {Grp. gram.:m.}
\end{itemize}
\begin{itemize}
\item {Utilização:Bras}
\end{itemize}
Árvore, cuja goma é vermífuga.
\section{Camacheiro}
\begin{itemize}
\item {Grp. gram.:m.}
\end{itemize}
\begin{itemize}
\item {Utilização:Mad}
\end{itemize}
\begin{itemize}
\item {Proveniência:(De \textunderscore Camacho\textunderscore , n. p.)}
\end{itemize}
Vento de Léste.
\section{Camachilra}
\begin{itemize}
\item {Grp. gram.:f.}
\end{itemize}
O mesmo que \textunderscore cambaxira\textunderscore .
\section{Camachirra}
\begin{itemize}
\item {Grp. gram.:f.}
\end{itemize}
O mesmo que \textunderscore cambaxira\textunderscore .
\section{Camada}
\begin{itemize}
\item {Grp. gram.:f.}
\end{itemize}
\begin{itemize}
\item {Utilização:Bras}
\end{itemize}
\begin{itemize}
\item {Proveniência:(De \textunderscore cama\textunderscore )}
\end{itemize}
Objecto ou objectos, estendidos uniformemente, formando superfície horizontal ou proximamente horizontal.
Ataque de sezões, bexigas, sarna, etc.
Classe: \textunderscore as camadas sociaes\textunderscore .
Grande quantidade.
Cada uma das partes differentes que, na atmosphera, nos vegetaes e no globo terrestre, indicam respectivamente a densidade e antiguidade de formação ou constituição.
Extensão vasta e lisa de terreno.
\section{Camafeio}
\begin{itemize}
\item {Grp. gram.:m.}
\end{itemize}
\begin{itemize}
\item {Utilização:Ant.}
\end{itemize}
O mesmo que \textunderscore camafeu\textunderscore . Cf. \textunderscore Eufrosina\textunderscore , 219.
\section{Camafeu}
\begin{itemize}
\item {Grp. gram.:m.}
\end{itemize}
\begin{itemize}
\item {Utilização:Pop.}
\end{itemize}
\begin{itemize}
\item {Utilização:Ant.}
\end{itemize}
\begin{itemize}
\item {Proveniência:(Do b. lat. \textunderscore camahotus\textunderscore )}
\end{itemize}
Pedra preciosa, com duas camadas differentes na côr, sôbre uma das quaes se lavrou uma figura em relêvo, á qual serve de fundo a outra camada.
Mulher muito feia.
Sêllo dos reis de Portugal.
Effígie dos reis nas moédas.
\section{Camafonge}
\begin{itemize}
\item {Grp. gram.:m.}
\end{itemize}
\begin{itemize}
\item {Utilização:Bras}
\end{itemize}
Moleque travesso.
Ente desprezível.
(Talvez or. afr.)
\section{Camaísma}
\begin{itemize}
\item {Grp. gram.:m.}
\end{itemize}
\begin{itemize}
\item {Utilização:Bras}
\end{itemize}
Planta, de que fazem frechas os indígenas.
\section{Camal}
\begin{itemize}
\item {Grp. gram.:m.}
\end{itemize}
Antiga peça de armadura, que, cobrindo o elmo, descaía sôbre os ombros.
(Provn. \textunderscore camal\textunderscore , de \textunderscore cap\textunderscore , cabeça, e \textunderscore malh\textunderscore , malha)
\section{Camalassana}
\begin{itemize}
\item {Grp. gram.:f.}
\end{itemize}
Flôr indiana, aquática e gigântea, de pedúnculo encarnado, (\textunderscore nimphaea alba\textunderscore ).
\section{Camáldula}
\begin{itemize}
\item {Grp. gram.:f.}
\end{itemize}
Convento de camáldulos.
\section{Camáldulas}
\begin{itemize}
\item {Grp. gram.:f. pl.}
\end{itemize}
\begin{itemize}
\item {Proveniência:(De \textunderscore camáldulo\textunderscore )}
\end{itemize}
O mesmo que \textunderscore camândulas\textunderscore .
\section{Camaldulense}
\begin{itemize}
\item {Grp. gram.:adj.}
\end{itemize}
Relativo á Ordem monástica dos camáldulos.
\section{Camáldulo}
\begin{itemize}
\item {Grp. gram.:m.}
\end{itemize}
\begin{itemize}
\item {Proveniência:(De \textunderscore Camáldoli\textunderscore , n. p.)}
\end{itemize}
Religioso de uma Ordem monástica, fundada por San-Romualdo em Camáldoli, na Toscana.
\section{Camaleão}
\begin{itemize}
\item {Grp. gram.:m.}
\end{itemize}
\begin{itemize}
\item {Utilização:Fig.}
\end{itemize}
\begin{itemize}
\item {Proveniência:(Do gr. \textunderscore kamaileon\textunderscore )}
\end{itemize}
Lagarto da costa oriental da África, coberto por uma espécie de lixa, cujas rugosidades mudam de cores, em que predomina o verde e o vermelho.
Reptil da ordem das lagartixas, (\textunderscore camaeleo vulgaris\textunderscore ).
Aquelle que muda facilmente de opiniões; catavento.
\textunderscore Camaleão-mineral\textunderscore , mistura de manganato de potássio com sesquióxydo de manganés.
\section{Camaleão}
\begin{itemize}
\item {Grp. gram.:m.}
\end{itemize}
\begin{itemize}
\item {Utilização:Bras. do N}
\end{itemize}
\begin{itemize}
\item {Proveniência:(De \textunderscore camalhão\textunderscore ^1?)}
\end{itemize}
Escavação nas estradas, causada pela passagem de tropas ou de carros.
\section{Camalete}
\begin{itemize}
\item {fónica:lê}
\end{itemize}
\begin{itemize}
\item {Grp. gram.:m.}
\end{itemize}
\begin{itemize}
\item {Utilização:Ant.}
\end{itemize}
Pequena peça de artilharia, o mesmo que \textunderscore camelete\textunderscore . Cf. Lopo de S. Coutinho, \textunderscore Hist. do C. de Dio\textunderscore , l. II, c. 17.
\section{Camalha}
\begin{itemize}
\item {Grp. gram.:f.}
\end{itemize}
Capuz de lan que, cobrindo a cabeça das mulheres, lhes cái sôbre os hombros.
(Cp. \textunderscore camal\textunderscore )
\section{Camalhão}
\begin{itemize}
\item {Grp. gram.:m.}
\end{itemize}
\begin{itemize}
\item {Proveniência:(De \textunderscore cama\textunderscore )}
\end{itemize}
Pequena elevação ou camada de terra, disposta para sementeira, entre dois regos.
\section{Camalhão}
\begin{itemize}
\item {Grp. gram.:m.}
\end{itemize}
Camalho grande.
\section{Camalho}
\begin{itemize}
\item {Grp. gram.:m.}
\end{itemize}
\begin{itemize}
\item {Utilização:Ant.}
\end{itemize}
O mesmo que \textunderscore camal\textunderscore .
\section{Camalote}
\begin{itemize}
\item {Utilização:Bras. do S}
\end{itemize}
Ervaçal nas margens dos rios.
Ilhota fluctuante, formada de troncos soltos, raízes, etc., e que desce os grandes rios, á mercê das correntes.
\section{Camândulas}
\begin{itemize}
\item {Grp. gram.:f. pl.}
\end{itemize}
Rosário de grandes contas.
(Corr. de \textunderscore camáldulas\textunderscore )
\section{Camanho}
\begin{itemize}
\item {Grp. gram.:adj.}
\end{itemize}
\begin{itemize}
\item {Utilização:Ant.}
\end{itemize}
\begin{itemize}
\item {Proveniência:(Do lat. \textunderscore quam magnus\textunderscore )}
\end{itemize}
Quão grande.
\section{Camansai}
\begin{itemize}
\item {Grp. gram.:m.}
\end{itemize}
Árvore das Filippinas, cuja madeira se emprega em construcções navaes.
\section{Camanturai}
\begin{itemize}
\item {Grp. gram.:m.}
\end{itemize}
Árvore indiana, de casca medicinal.
\section{Camão}
\begin{itemize}
\item {Grp. gram.:m.}
\end{itemize}
Ave aquática, de bico agudo e pennas azues.
\section{Camapu}
\begin{itemize}
\item {Grp. gram.:m.}
\end{itemize}
Planta solânea do Brasil.
Fruto dessa planta.
\section{Camar}
\begin{itemize}
\item {Grp. gram.:v. t.}
\end{itemize}
O mesmo que \textunderscore acamar\textunderscore . Cf. Filinto, IV, 328.
\section{Câmara}
\begin{itemize}
\item {Grp. gram.:f.}
\end{itemize}
\begin{itemize}
\item {Proveniência:(Do lat. \textunderscore camera\textunderscore )}
\end{itemize}
Compartimento de uma casa.
Aposento; quarto de dormir.
Corporação de Vereadores, de Deputados, de commerciantes.
Apparelho óptico.
Tribunal (ecclesiastico).
Edifício, onde funcciona a Vereação.
Edifício das côrtes.
\textunderscore Câmara óptica\textunderscore , caixa, com um óculo de lente convergente, pelo qual se vê ampliada uma estampa que esteja dentro da caixa.
\textunderscore Câmara ardente\textunderscore , compartimento ou sala illuminada por velas, e em que se deposita o defunto, antes do saimento.
\section{Camará}
\begin{itemize}
\item {Grp. gram.:m.}
\end{itemize}
Planta verbenácea do Brasil.
\section{Camará-bravo}
\begin{itemize}
\item {Grp. gram.:m.}
\end{itemize}
\begin{itemize}
\item {Utilização:Bot.}
\end{itemize}
O mesmo que \textunderscore official-da-sala\textunderscore .
\section{Camaracubo}
\begin{itemize}
\item {Grp. gram.:m.}
\end{itemize}
Planta do Brasil.
\section{Camarada}
\begin{itemize}
\item {Grp. gram.:f.}
\end{itemize}
\begin{itemize}
\item {Utilização:Bras. do N}
\end{itemize}
\begin{itemize}
\item {Grp. gram.:M.  e  f.}
\end{itemize}
\begin{itemize}
\item {Utilização:Bras. do N}
\end{itemize}
\begin{itemize}
\item {Proveniência:(De \textunderscore câmara\textunderscore )}
\end{itemize}
Companheiro de quarto.
Collega, condiscipulo.
Cada um dos indivíduos que exercem a mesma profissão.
Soldado.
Militar, que está impedido no serviço particular de um official do exército.
Arreeiro.
Pessôa amancebada.
\section{Camaradagem}
\begin{itemize}
\item {Grp. gram.:f.}
\end{itemize}
\begin{itemize}
\item {Proveniência:(De \textunderscore camarada\textunderscore )}
\end{itemize}
Convivência amigável entre pessôas que têm a mesma occupação.
\section{Camaradaria}
\begin{itemize}
\item {Grp. gram.:f.}
\end{itemize}
O mesmo que \textunderscore camaradagem\textunderscore . Cf. Castilho, \textunderscore Fastos\textunderscore , III, 484.
\section{Camaradinha}
\begin{itemize}
\item {Grp. gram.:f.}
\end{itemize}
\begin{itemize}
\item {Utilização:Bras. de Minas}
\end{itemize}
Planta medicinal.
\section{Camará-do-mato}
\begin{itemize}
\item {Grp. gram.:m.}
\end{itemize}
\begin{itemize}
\item {Utilização:Bot.}
\end{itemize}
O mesmo que \textunderscore pau-pereiro\textunderscore .
\section{Camarajapo}
\begin{itemize}
\item {Grp. gram.:m.}
\end{itemize}
Espécie de hortelan do Brasil.
\section{Camaralengo}
\textunderscore m.\textunderscore  (ant.)(V.camarlengo)
\section{Camarambaia}
\begin{itemize}
\item {Grp. gram.:f.}
\end{itemize}
Planta onagrária do Brasil.
\section{Camaranchão}
\begin{itemize}
\item {Grp. gram.:m.}
\end{itemize}
Obra avançada, numa fortificação.
(Aument. de \textunderscore camarancha\textunderscore , de \textunderscore câmara\textunderscore  + \textunderscore ancho\textunderscore )
\section{Camarano}
\begin{itemize}
\item {Grp. gram.:adj.}
\end{itemize}
\begin{itemize}
\item {Utilização:Bot.}
\end{itemize}
\begin{itemize}
\item {Proveniência:(De \textunderscore câmara\textunderscore )}
\end{itemize}
Diz-se das plantas que, exteriormente, apresentam um sulco longitudinal e, interiormente, uma placenta lateral, correspondente a êsse sulco.
\section{Camarão}
\begin{itemize}
\item {Grp. gram.:m.}
\end{itemize}
\begin{itemize}
\item {Proveniência:(Do lat. \textunderscore cammarus\textunderscore )}
\end{itemize}
Pequeno crustáceo decápode.
Antigo vaso de loiça.
Gancho, com que se suspendem do tecto candeeiros, lustres, etc.
Certa qualidade de prego.
O mesmo que \textunderscore pau-carga\textunderscore .
\section{Camararés}
\begin{itemize}
\item {Grp. gram.:m. pl.}
\end{itemize}
Tríbo de Índios do Brasil, que viviam nas margens de um affluente do rio Madeira.
\section{Camararia}
\begin{itemize}
\item {Grp. gram.:f.}
\end{itemize}
Cargo de camareiro.
\section{Camarário}
\begin{itemize}
\item {Grp. gram.:adj.}
\end{itemize}
\begin{itemize}
\item {Grp. gram.:M.}
\end{itemize}
Que diz respeito á câmara: \textunderscore resoluções camarárias\textunderscore .
Antiga dignidade ecclesiástica.
\section{Câmaras}
\begin{itemize}
\item {Grp. gram.:f. pl.}
\end{itemize}
O mesmo que \textunderscore cambras\textunderscore .
\section{Camarata}
\begin{itemize}
\item {Grp. gram.:f.}
\end{itemize}
\begin{itemize}
\item {Proveniência:(Lat. \textunderscore camerata\textunderscore )}
\end{itemize}
Série de leitos numa só sala.
\section{Camaratão}
\begin{itemize}
\item {Grp. gram.:m.}
\end{itemize}
\begin{itemize}
\item {Proveniência:(De \textunderscore Camarate\textunderscore , n. p.)}
\end{itemize}
Casta de uva branca de Ourém.
\section{Camarate}
\begin{itemize}
\item {Grp. gram.:f.}
\end{itemize}
\begin{itemize}
\item {Proveniência:(De \textunderscore Camarate\textunderscore , n. p.)}
\end{itemize}
Casta de uva branca, temporan e muito dôce.
\section{Camarate-tinto}
\begin{itemize}
\item {Grp. gram.:m.}
\end{itemize}
Casta de uva preta.
\section{Camaratinga}
\begin{itemize}
\item {Grp. gram.:f.}
\end{itemize}
Planta trepadeira do Brasil.
\section{Camarçada}
\begin{itemize}
\item {Grp. gram.:f.}
\end{itemize}
\begin{itemize}
\item {Utilização:Pop.}
\end{itemize}
\begin{itemize}
\item {Proveniência:(De \textunderscore camarço\textunderscore )}
\end{itemize}
Porção de achaques: \textunderscore uma camarçada de defluxos\textunderscore .
\section{Camarção}
\begin{itemize}
\item {Grp. gram.:m.}
\end{itemize}
Pequeno bosque de arbustos silvestres.
Terra areenta, que só produz plantas silvestres.
O mesmo que \textunderscore médão\textunderscore .
\section{Camarço}
\begin{itemize}
\item {Grp. gram.:m.}
\end{itemize}
\begin{itemize}
\item {Utilização:Pop.}
\end{itemize}
\begin{itemize}
\item {Utilização:Ant.}
\end{itemize}
\begin{itemize}
\item {Utilização:Gír.}
\end{itemize}
Desgraça.
Doença.
\textunderscore Ficar camarço\textunderscore , não fazer vasa nenhuma, no jôgo dos centos:«\textunderscore ficaremos esta noite camarço\textunderscore ». R. Lobo, \textunderscore Côrte na Aldeia\textunderscore , I, 63.
Febre muito ardente, febrão.
Tostão.
(Por \textunderscore queimarço\textunderscore , de \textunderscore queimar\textunderscore ?)
\section{Camareira}
\begin{itemize}
\item {Grp. gram.:f.}
\end{itemize}
\begin{itemize}
\item {Proveniência:(De \textunderscore câmara\textunderscore )}
\end{itemize}
Senhora, que faz serviço na câmara da Rainha, da Princesa, etc.
Mulher, que faz serviço em botequins.
Planta, o mesmo que \textunderscore tripa-de ovelha\textunderscore .
\section{Camareiro}
\begin{itemize}
\item {Grp. gram.:m.}
\end{itemize}
\begin{itemize}
\item {Utilização:Ant.}
\end{itemize}
\begin{itemize}
\item {Proveniência:(De \textunderscore câmara\textunderscore )}
\end{itemize}
O mesmo que \textunderscore camarista\textunderscore .
Coadjutor do Abbade de convento.
Dignitário da côrte pontifícia.
Bacio de quarto, para urinar.
Espécie de rêde, o mesmo que \textunderscore camaroeiro\textunderscore .
\section{Camarento}
\begin{itemize}
\item {Grp. gram.:adj.}
\end{itemize}
Que tem a doença das câmaras.
\section{Camareta}
\begin{itemize}
\item {fónica:marê}
\end{itemize}
\begin{itemize}
\item {Grp. gram.:f.}
\end{itemize}
\begin{itemize}
\item {Utilização:Prov.}
\end{itemize}
\begin{itemize}
\item {Utilização:minh.}
\end{itemize}
\begin{itemize}
\item {Proveniência:(De \textunderscore câmara\textunderscore )}
\end{itemize}
Quarto, em que só cabe a enxerga e em que o pescador se deita.
\section{Camaria}
\begin{itemize}
\item {Grp. gram.:f.}
\end{itemize}
Espécie de urze rasteira, que dá camarinhas:«\textunderscore que se embrenhou numas moitas de camarias\textunderscore ». Corvo, \textunderscore Anno na Côrte\textunderscore , III, 27.
\section{Camarilha}
\begin{itemize}
\item {Grp. gram.:f.}
\end{itemize}
\begin{itemize}
\item {Proveniência:(De \textunderscore câmara\textunderscore )}
\end{itemize}
Cortesãos, que, lisonjeando o monarcha, influem nocivamente nos negócios públicos.
\section{Camarim}
\begin{itemize}
\item {Grp. gram.:m.}
\end{itemize}
\begin{itemize}
\item {Proveniência:(Do it. \textunderscore camerino\textunderscore )}
\end{itemize}
Pequena câmara.
\section{Camarina}
\begin{itemize}
\item {Grp. gram.:f.}
\end{itemize}
\begin{itemize}
\item {Utilização:Des.}
\end{itemize}
Pequena câmara, camarim.
\section{Camarinha}
\begin{itemize}
\item {Grp. gram.:f.}
\end{itemize}
\begin{itemize}
\item {Utilização:Náut.}
\end{itemize}
\begin{itemize}
\item {Utilização:Bras. do N}
\end{itemize}
\begin{itemize}
\item {Grp. gram.:Pl.}
\end{itemize}
\begin{itemize}
\item {Utilização:Fig.}
\end{itemize}
\begin{itemize}
\item {Proveniência:(De \textunderscore câmara\textunderscore )}
\end{itemize}
Casa da ré.
Quarto de dormir.
Pequena prateleira, ao canto de uma sala ou quarto.
Frutos pequenos e redondos de várias plantas.
Gotas pequeninas.
\section{Camarinhado}
\begin{itemize}
\item {Grp. gram.:adj.}
\end{itemize}
Que tem fórma de camarinhas.
\section{Camarinheira}
\begin{itemize}
\item {Grp. gram.:f.}
\end{itemize}
Planta empetrácea, que produz camarinhas.
\section{Camarista}
\begin{itemize}
\item {Grp. gram.:m.}
\end{itemize}
\begin{itemize}
\item {Proveniência:(De \textunderscore câmara\textunderscore )}
\end{itemize}
Fidalgo, ao serviço de pessôas reaes.
Vereador municipal.
\section{Camarlengado}
\begin{itemize}
\item {Grp. gram.:m.}
\end{itemize}
Dignidade de camarlengo.
\section{Camarlengo}
\begin{itemize}
\item {Grp. gram.:m.  e  adj.}
\end{itemize}
Diz-se do Cardeal, que preside á Câmara Apostólica.
(B. lat. \textunderscore camarlengus\textunderscore , do germ.)
\section{Camarneira}
\begin{itemize}
\item {Grp. gram.:f.}
\end{itemize}
\begin{itemize}
\item {Utilização:Prov.}
\end{itemize}
Emprega-se como termo da comparação, para designar uma leguminosa, avergada de vagens: \textunderscore o ervilhal todo élle é uma camarneira\textunderscore .
(Colhido em Turquel)
(Por \textunderscore camarinheira\textunderscore )
\section{Camaroeiro}
\begin{itemize}
\item {Grp. gram.:f.}
\end{itemize}
Rede, para pescar camarões.
Objecto, em fórma de camaroeiro, que se emprega para annunciar a proximidade ou continuação de temporaes, içando-se em lugar patente ou elevado.
\section{Camarota}
\begin{itemize}
\item {Grp. gram.:f.}
\end{itemize}
Gênero de insectos dípteros.
\section{Camarote}
\begin{itemize}
\item {Grp. gram.:m.}
\end{itemize}
Pequena câmara de navio.
Cada um dos repartimentos, dispostos em andares, á volta de uma sala de espectáculos.
\section{Camaroteiro}
\begin{itemize}
\item {Grp. gram.:m.}
\end{itemize}
\begin{itemize}
\item {Proveniência:(De \textunderscore camarote\textunderscore )}
\end{itemize}
Aquelle que vende bilhetes, para admissão nos camarotes e em outros lugares de theatro.
\section{Camaroto}
\begin{itemize}
\item {fónica:marô}
\end{itemize}
\begin{itemize}
\item {Grp. gram.:m.}
\end{itemize}
\begin{itemize}
\item {Proveniência:(De \textunderscore câmara\textunderscore )}
\end{itemize}
Gênero de insectos coleópteros, originários da América.
\section{Camarra}
\textunderscore f.\textunderscore  (e der.)
O mesmo ou melhor que \textunderscore samarra\textunderscore , etc.
\section{Camartelada}
\begin{itemize}
\item {Grp. gram.:f.}
\end{itemize}
Pancada de camartelo.
\section{Camartelador}
\begin{itemize}
\item {Grp. gram.:adj.}
\end{itemize}
Que camartela; que desmorona:«\textunderscore bemdita sejas tu, geração camarteladora!\textunderscore »Herculano, \textunderscore Quest. Públ.\textunderscore , II, 48.
\section{Camartelar}
\begin{itemize}
\item {Grp. gram.:v. t.}
\end{itemize}
Bater, partir, ou destruir com martelo.
\section{Camartelo}
\begin{itemize}
\item {Grp. gram.:m.}
\end{itemize}
\begin{itemize}
\item {Proveniência:(De \textunderscore martelo\textunderscore  com um pref. obscuro)}
\end{itemize}
Grande martelo, que termina de um lado em ponta ou gume, e do outro em fórma arredondada ou quadrangular.
\section{Camásia}
\begin{itemize}
\item {Grp. gram.:f.}
\end{itemize}
Gênero de plantas liliáceas.
\section{Camassino}
\begin{itemize}
\item {Grp. gram.:m.}
\end{itemize}
Lingua uralo-altaica do grupo samoiedo.
\section{Camastralho}
\begin{itemize}
\item {Grp. gram.:m.}
\end{itemize}
\begin{itemize}
\item {Utilização:Prov.}
\end{itemize}
\begin{itemize}
\item {Utilização:alent.}
\end{itemize}
Cama pobre, feita no chão.
\section{Camau}
\begin{itemize}
\item {Grp. gram.:m.}
\end{itemize}
O mesmo que \textunderscore gallinha-sultana\textunderscore .
\section{Camauro}
\begin{itemize}
\item {Grp. gram.:m.}
\end{itemize}
\begin{itemize}
\item {Utilização:Ant.}
\end{itemize}
Barrete, usado pelos Papas, encobrindo-lhes as orelhas.
(B. lat. \textunderscore camaurum\textunderscore )
\section{Camba}
\begin{itemize}
\item {Grp. gram.:f.}
\end{itemize}
\begin{itemize}
\item {Utilização:Ant.}
\end{itemize}
\begin{itemize}
\item {Proveniência:(Do lat. \textunderscore campe\textunderscore )}
\end{itemize}
Peça curva das rodas dos carros; pina.
Nesga.
Moínho de mão.
Pequena cambota.
\section{Camba}
\begin{itemize}
\item {Grp. gram.:f.}
\end{itemize}
\begin{itemize}
\item {Utilização:Bras}
\end{itemize}
\begin{itemize}
\item {Grp. gram.:M.  e  f.}
\end{itemize}
O mesmo que \textunderscore mucamba\textunderscore ^1.
Índio desprezível.
\section{Cambacece}
\begin{itemize}
\item {Grp. gram.:m.}
\end{itemize}
Arbusto angolense, elegante e ramoso.
\section{Cambada}
\begin{itemize}
\item {Grp. gram.:f.}
\end{itemize}
\begin{itemize}
\item {Proveniência:(De \textunderscore camba\textunderscore )}
\end{itemize}
Porção de objectos, enfiados ou pendurados em uma cana, cordel, etc.
Grande quantidade.
Corja; canalha.
\section{Cambadela}
\begin{itemize}
\item {Grp. gram.:f.}
\end{itemize}
\begin{itemize}
\item {Proveniência:(De \textunderscore cambar\textunderscore )}
\end{itemize}
O mesmo que \textunderscore cambalhota\textunderscore .
\section{Cambado}
\begin{itemize}
\item {Grp. gram.:adj.}
\end{itemize}
\begin{itemize}
\item {Proveniência:(De \textunderscore cambar\textunderscore ^1)}
\end{itemize}
Que tem pernas tortas.
Torto, acalcanhado, (falando-se do calçado).
\section{Cambadoiro}
\begin{itemize}
\item {Grp. gram.:m.}
\end{itemize}
\begin{itemize}
\item {Utilização:Prov.}
\end{itemize}
\begin{itemize}
\item {Utilização:dur.}
\end{itemize}
\begin{itemize}
\item {Proveniência:(De \textunderscore cambar\textunderscore ^2)}
\end{itemize}
Desvio do rumo, feito pelos barqueiros, quando passam do um lado do rio, onde a corrente é mais forte, para outro, onde a navegação é mais fácil.
\section{Cambador}
\begin{itemize}
\item {Grp. gram.:m.}
\end{itemize}
\begin{itemize}
\item {Utilização:Des.}
\end{itemize}
\begin{itemize}
\item {Proveniência:(De \textunderscore cambar\textunderscore ^2)}
\end{itemize}
O mesmo que \textunderscore cambiador\textunderscore .
\section{Cambadouro}
\begin{itemize}
\item {Grp. gram.:m.}
\end{itemize}
\begin{itemize}
\item {Utilização:Prov.}
\end{itemize}
\begin{itemize}
\item {Utilização:dur.}
\end{itemize}
\begin{itemize}
\item {Proveniência:(De \textunderscore cambar\textunderscore ^2)}
\end{itemize}
Desvio do rumo, feito pelos barqueiros, quando passam do um lado do rio, onde a corrente é mais forte, para outro, onde a navegação é mais fácil.
\section{Cambaia}
\begin{itemize}
\item {Grp. gram.:f.}
\end{itemize}
\begin{itemize}
\item {Utilização:Marn.}
\end{itemize}
Desabamento do muro das salinas.
\section{Cambaiate}
\begin{itemize}
\item {Grp. gram.:adj.}
\end{itemize}
\begin{itemize}
\item {Utilização:Ant.}
\end{itemize}
Relativo a Cambaia; fabricado em Cambaia. Cf. \textunderscore Lembrança das Cousas da India\textunderscore , 56, nos \textunderscore Subsídios\textunderscore  de Felner.
\section{Cambaico}
\begin{itemize}
\item {Grp. gram.:adj.}
\end{itemize}
Relativo a Cambaia. Cf. \textunderscore Lusiadas\textunderscore , X, 60.
\section{Cambaio}
\begin{itemize}
\item {Grp. gram.:adj.}
\end{itemize}
O mesmo que \textunderscore cambado\textunderscore .
\section{Cambais}
\begin{itemize}
\item {Grp. gram.:m.}
\end{itemize}
\begin{itemize}
\item {Utilização:Ant.}
\end{itemize}
\begin{itemize}
\item {Proveniência:(De \textunderscore Cambaia\textunderscore , n. p.)}
\end{itemize}
Espécie de vestuário militar.
\section{Cambal}
\begin{itemize}
\item {Grp. gram.:m.}
\end{itemize}
Resguardo de pano, madeira ou farinha, para que se não espalhe a farinha que se vai moendo.
(Cp. \textunderscore camba\textunderscore ^1)
\section{Cambalacho}
\begin{itemize}
\item {Grp. gram.:m.}
\end{itemize}
\begin{itemize}
\item {Proveniência:(Do rad. de \textunderscore cambar\textunderscore ^2)}
\end{itemize}
Permutação.
Troca ardilosa.
Ardil, tramóia.
Conluio.
\section{Cambaleante}
\begin{itemize}
\item {Grp. gram.:adj.}
\end{itemize}
Que cambaleia.
\section{Cambalear}
\begin{itemize}
\item {Grp. gram.:v. i.}
\end{itemize}
\begin{itemize}
\item {Proveniência:(Do rad. de \textunderscore cambar\textunderscore ^1)}
\end{itemize}
Caminhar sem firmeza; oscillar andando.
\section{Cambaleio}
\begin{itemize}
\item {Grp. gram.:m.}
\end{itemize}
Acto de \textunderscore cambalear\textunderscore .
\section{Cambalenga}
\begin{itemize}
\item {Grp. gram.:f.}
\end{itemize}
O mesmo que \textunderscore camolenga\textunderscore .
\section{Cambalhão}
\begin{itemize}
\item {Grp. gram.:m.}
\end{itemize}
\begin{itemize}
\item {Utilização:Prov.}
\end{itemize}
\begin{itemize}
\item {Utilização:dur.}
\end{itemize}
Espaço de terra, que os maus cavadores e redradores deixam em cru.
(Por \textunderscore camalhão\textunderscore ^1?)
\section{Cambalheira}
\begin{itemize}
\item {Grp. gram.:f.}
\end{itemize}
\begin{itemize}
\item {Utilização:Prov.}
\end{itemize}
\begin{itemize}
\item {Utilização:trasm.}
\end{itemize}
\begin{itemize}
\item {Utilização:beir.}
\end{itemize}
O mesmo que \textunderscore gramalheira\textunderscore .
\section{Cambalhota}
\begin{itemize}
\item {Grp. gram.:f.}
\end{itemize}
\begin{itemize}
\item {Proveniência:(Do rad. de \textunderscore cambar\textunderscore ^1)}
\end{itemize}
Volta, que se dá com o corpo, baixando a cabeça ou firmando-a no chão, e levantando as pernas posteriormente, para caírem do outro lado.
Trambolhão.
\section{Cambalhotar}
\begin{itemize}
\item {Grp. gram.:v. i.}
\end{itemize}
Dar cambalhotas. Cf. Filinto, VII, 110.
\section{Cambaluço}
\begin{itemize}
\item {Grp. gram.:m.}
\end{itemize}
\begin{itemize}
\item {Utilização:Prov.}
\end{itemize}
\begin{itemize}
\item {Utilização:trasm.}
\end{itemize}
Acto de cair de bruços.
Grande tombo.
(Cp. \textunderscore cambaluz\textunderscore )
\section{Cambaluz}
\begin{itemize}
\item {Grp. gram.:m.}
\end{itemize}
\begin{itemize}
\item {Utilização:Prov.}
\end{itemize}
\begin{itemize}
\item {Utilização:beir.}
\end{itemize}
Quéda.
Contrariedade; desastre.
(Cp. \textunderscore cambalhota\textunderscore )
\section{Cambambaxilo}
\begin{itemize}
\item {Grp. gram.:m.}
\end{itemize}
Árvore angolense, da fam. das verbenáceas, de frutos negros e comestíveis, semelhantes ás azeitonas.
\section{Cambango}
\begin{itemize}
\item {Grp. gram.:m.}
\end{itemize}
Ave gallinácea de Angola. Cp. Capello e Ivens, II, 358.
\section{Cambão}
\begin{itemize}
\item {Grp. gram.:m.}
\end{itemize}
\begin{itemize}
\item {Utilização:Bras}
\end{itemize}
\begin{itemize}
\item {Utilização:Prov.}
\end{itemize}
\begin{itemize}
\item {Utilização:T. de Aveiro}
\end{itemize}
\begin{itemize}
\item {Utilização:Bras. do N}
\end{itemize}
\begin{itemize}
\item {Utilização:Gír. de Lisbôa.}
\end{itemize}
Apparelho, com que se ligam duas juntas de bois ao mesmo carro.
Pau, a que se ligam as bêstas que fazem mover a nora ou a atafona.
Pau com gancho, para apanhar fruta.
Junta de bois.
Vara, que, no engenho para tirar água, chamado \textunderscore burra\textunderscore , sustenta a pedra, que contrabalança o pêso do balde.
Cada uma das duas cordas, fixas no meio do punho do remo, no barco de pesca, e que, divididas em sete chicotes e puxadas por sete dos homens da companha, servem para remar.
Pau, que se pendura ao pescoço de um animal, para que se não distancie muito, nem penetre em roças ou cerrados.
Conluio entre vários indivíduos, para não disputarem entre si, nos lanços dos leilões, e repartirem os lucros da revenda.
\section{Cambapé}
\begin{itemize}
\item {Grp. gram.:m.}
\end{itemize}
\begin{itemize}
\item {Utilização:Pop.}
\end{itemize}
\begin{itemize}
\item {Proveniência:(De \textunderscore cambar\textunderscore ^2 + \textunderscore pé\textunderscore )}
\end{itemize}
Ardil, com que se mete o pé ou a perna entre as de outrem, para o fazer cair.
Armadilha, cilada.
\section{Cambar}
\begin{itemize}
\item {Grp. gram.:v. i.}
\end{itemize}
Cambalear; entortar as pernas.
(Cp. \textunderscore camba\textunderscore ^1)
\section{Cambar}
\begin{itemize}
\item {Grp. gram.:v. t.}
\end{itemize}
\begin{itemize}
\item {Utilização:Des.}
\end{itemize}
\begin{itemize}
\item {Grp. gram.:V. i.}
\end{itemize}
\begin{itemize}
\item {Utilização:Prov.}
\end{itemize}
\begin{itemize}
\item {Utilização:Bras}
\end{itemize}
\begin{itemize}
\item {Utilização:dur.}
\end{itemize}
O mesmo que \textunderscore cambiar\textunderscore .
Passar de um lugar para outro.
Mudar de rumo.
Mudar de um bordo para outro (o vento, as escotas das velas latinas, etc.).
\section{Cambará}
\begin{itemize}
\item {Grp. gram.:m.}
\end{itemize}
O mesmo que \textunderscore camará\textunderscore .
\section{Cambareira}
\begin{itemize}
\item {Grp. gram.:f.}
\end{itemize}
Árvore silvestre, intertropical, de tronco tortuoso.
\section{Cambarro}
\begin{itemize}
\item {Grp. gram.:m.}
\end{itemize}
\begin{itemize}
\item {Utilização:Prov.}
\end{itemize}
\begin{itemize}
\item {Utilização:trasm.}
\end{itemize}
Alpendre para palha.
\section{Cambás}
\begin{itemize}
\item {Grp. gram.:m.}
\end{itemize}
\begin{itemize}
\item {Utilização:Ant.}
\end{itemize}
Cobertura acolchoada, com que se poupava o corpo aos golpes de armas brancas.
\section{Cambaxira}
\begin{itemize}
\item {Grp. gram.:f.}
\end{itemize}
Pássaro harmonioso do Brasil.
\section{Cambeba}
\begin{itemize}
\item {Grp. gram.:f.}
\end{itemize}
\begin{itemize}
\item {Utilização:Bras}
\end{itemize}
Espécie de tartaruga do norte do Brasil.
\section{Cambebus}
\begin{itemize}
\item {Grp. gram.:m. pl.}
\end{itemize}
O mesmo que \textunderscore Omáguas\textunderscore .
\section{Cambeia}
\begin{itemize}
\item {Grp. gram.:f.}
\end{itemize}
\begin{itemize}
\item {Proveniência:(De \textunderscore cambar\textunderscore ^1)}
\end{itemize}
Ruína, produzida pelos vendavaes nos muros das salinas.
Bôca, que êsses muros apresentam, no lugar do desmoronamento.
\section{Cambeirada}
\begin{itemize}
\item {Grp. gram.:f.}
\end{itemize}
\begin{itemize}
\item {Utilização:Prov.}
\end{itemize}
Arremêsso de cambeiras, ou acto de enfarinhar com cambeiras, no Carnaval.
Porção de cambeiras.
Pequena porção de farinha.
\section{Cambeiral}
\begin{itemize}
\item {Grp. gram.:m.}
\end{itemize}
(V.cambal)
\section{Cambeiras}
\begin{itemize}
\item {Grp. gram.:f. pl.}
\end{itemize}
\begin{itemize}
\item {Utilização:T. da Bairrada}
\end{itemize}
\begin{itemize}
\item {Proveniência:(De \textunderscore camba\textunderscore )}
\end{itemize}
A farinha, mais fina, que se evola da mó, poisando nas paredes e objectos circunjacentes.
Anteparo de madeira, á frente da mó do moínho.
\section{Cambeiro}
\begin{itemize}
\item {Grp. gram.:m.}
\end{itemize}
\begin{itemize}
\item {Utilização:Prov.}
\end{itemize}
\begin{itemize}
\item {Utilização:beir.}
\end{itemize}
\begin{itemize}
\item {Utilização:T. da Guarda}
\end{itemize}
Tronco alto e esguio de pinheiro, que, em certas noites festivas, como a de San-João, se fixa num lugar ou praça, pendurando-se-lhe dos galhos vides ou ramos, a que se deita fogo, para illuminar o sítio.
Loiceiro.
Cada um dos pequenos cepos, que ficam ao lado da andadeira do moínho.
(Cp. \textunderscore cambão\textunderscore )
\section{Cambembe}
\begin{itemize}
\item {Grp. gram.:adj.}
\end{itemize}
\begin{itemize}
\item {Utilização:Bras}
\end{itemize}
O mesmo que \textunderscore cambado\textunderscore ; desajeitado, trangalhadanças.
\section{Cambeta}
\begin{itemize}
\item {fónica:bê}
\end{itemize}
\begin{itemize}
\item {Grp. gram.:m.  e  adj.}
\end{itemize}
(V.cambaio)
\section{Cambetas}
\begin{itemize}
\item {Grp. gram.:m. pl.}
\end{itemize}
Selvagens, que habitaram no Pará.
\section{Cambetear}
\begin{itemize}
\item {Grp. gram.:v. i.}
\end{itemize}
Andar como cambeta; coxear.
\section{Cambiador}
\begin{itemize}
\item {Grp. gram.:m.}
\end{itemize}
\begin{itemize}
\item {Utilização:Des.}
\end{itemize}
(V.cambista)
\section{Cambial}
\begin{itemize}
\item {Grp. gram.:adj.}
\end{itemize}
\begin{itemize}
\item {Grp. gram.:M.}
\end{itemize}
Relativo a \textunderscore câmbio\textunderscore .
Letra, sacada numa praça sôbre outra.
\section{Cambiante}
\begin{itemize}
\item {Grp. gram.:adj.}
\end{itemize}
\begin{itemize}
\item {Grp. gram.:M.}
\end{itemize}
\begin{itemize}
\item {Proveniência:(De \textunderscore cambiar\textunderscore )}
\end{itemize}
Que cambía; que passa de uma côr para outra.
Irisado.
Que é de furta-côres.
Que é de côr indecisa.
Gradação de côres.
Côr indecisa.
\section{Cambiar}
\begin{itemize}
\item {Grp. gram.:v. t.}
\end{itemize}
\begin{itemize}
\item {Grp. gram.:V. i.}
\end{itemize}
Permutar (moéda ou letras de um país pelas de outro).
Fazer a sorte de câmbio a (toiros).
Mudar de côres.
Fazer mudança de opiniões, systemas, etc.
(B. lat. \textunderscore cambiare\textunderscore )
\section{Cambica}
\begin{itemize}
\item {Grp. gram.:f.}
\end{itemize}
\begin{itemize}
\item {Utilização:Bras}
\end{itemize}
Alimento, feito de uma fruta macerada em água fria com açúcar.
(Talvez do tupi \textunderscore cameric\textunderscore , amassar)
\section{Cambiço}
\begin{itemize}
\item {Grp. gram.:m.}
\end{itemize}
\begin{itemize}
\item {Utilização:Prov.}
\end{itemize}
\begin{itemize}
\item {Utilização:trasm.}
\end{itemize}
Espécie de temão, que vai da grade e do canamão do trilho ao jugo dos bois.
(Cp. \textunderscore cambixo\textunderscore )
\section{Cambindas}
\begin{itemize}
\item {Grp. gram.:f.}
\end{itemize}
\begin{itemize}
\item {Utilização:Bras}
\end{itemize}
Espécie de dança, em que os dançadores estão acocorados, movendo-se ao som da música.
(Provavelmente por \textunderscore cabindas\textunderscore )
\section{Câmbio}
\begin{itemize}
\item {Grp. gram.:m.}
\end{itemize}
\begin{itemize}
\item {Proveniência:(De \textunderscore cambiar\textunderscore )}
\end{itemize}
Permutação; escambo.
Negociação de moédas, letras, metaes preciosos, notas de banco, etc.
Valor relativo dos objectos que se cambiam, ou lucro que o cambista aufere da permutação de valores.
Negociação mercantil, em que alguém cede a outrem o direito de receber uma quantia, em lugar differente daquelle em que se faz o contrato.
Sorte de toireiro, em que êste engana o toiro com um movimento de corpo para o lado direito, inclinando a cabeça para o lado opposto e pregando então as bandarilhas.
\section{Cambio}
\begin{itemize}
\item {Grp. gram.:m.}
\end{itemize}
Árvore de Damão, (\textunderscore careya arborea\textunderscore ).
\section{Cambiroto}
\begin{itemize}
\item {fónica:birô}
\end{itemize}
\begin{itemize}
\item {Grp. gram.:m.}
\end{itemize}
\begin{itemize}
\item {Utilização:Bras}
\end{itemize}
Cabeço de terra firme, na região do Acre.
\section{Cambismo}
\begin{itemize}
\item {Grp. gram.:m.}
\end{itemize}
\begin{itemize}
\item {Utilização:Fin.}
\end{itemize}
Influência do câmbio.
\section{Cambista}
\begin{itemize}
\item {Grp. gram.:m.}
\end{itemize}
\begin{itemize}
\item {Proveniência:(De \textunderscore câmbio\textunderscore )}
\end{itemize}
Aquelle que tem estabelecimento, em que se negociam papéis de crédito ou se fazem permutações monetárias.
\section{Cambitar}
\begin{itemize}
\item {Grp. gram.:v. t.}
\end{itemize}
\begin{itemize}
\item {Utilização:Bras. do N}
\end{itemize}
Transportar (cana de açúcar) do campo para o engenho.
\section{Cambito}
\begin{itemize}
\item {Grp. gram.:m.}
\end{itemize}
\begin{itemize}
\item {Utilização:Bras}
\end{itemize}
\begin{itemize}
\item {Utilização:Prov.}
\end{itemize}
\begin{itemize}
\item {Utilização:minh.}
\end{itemize}
\begin{itemize}
\item {Utilização:Bras. do N}
\end{itemize}
Pernil de porco.
Posta de arraia sêca.
Gancho de pau.
(Cp. it. \textunderscore gambetta\textunderscore , de \textunderscore gamba\textunderscore , perna)
\section{Câmbium}
\begin{itemize}
\item {Grp. gram.:m.}
\end{itemize}
Suco mucilaginoso, que se observa, tirando, na primavera, a casca de uma planta dicotyledónea, e que parece sêr a seiva elaborada.
\section{Cambixo}
\begin{itemize}
\item {Grp. gram.:m.}
\end{itemize}
\begin{itemize}
\item {Utilização:Des.}
\end{itemize}
\begin{itemize}
\item {Proveniência:(De \textunderscore cambo\textunderscore )}
\end{itemize}
Coisa torta, desajeitada.
\section{Cambo}
\begin{itemize}
\item {Grp. gram.:m.}
\end{itemize}
\begin{itemize}
\item {Utilização:Prov.}
\end{itemize}
\begin{itemize}
\item {Utilização:beir.}
\end{itemize}
\begin{itemize}
\item {Utilização:Prov.}
\end{itemize}
\begin{itemize}
\item {Utilização:beir.}
\end{itemize}
Pau com gancho para apanhar fruta; cambão, ladra.
Cambada, enfiada.
Braço das balanças antigas.
Pau com forquilha, com que se ampara a armação de latadas.
\section{Cambôa}
\begin{itemize}
\item {Grp. gram.:f.}
\end{itemize}
\begin{itemize}
\item {Proveniência:(De \textunderscore cambar\textunderscore ^2)}
\end{itemize}
Pequeno lago artificial, junto ao mar, em que a preamar deixa entrar o peixe miúdo.
\section{Cambôa}
\begin{itemize}
\item {Grp. gram.:f.}
\end{itemize}
\begin{itemize}
\item {Utilização:Bras}
\end{itemize}
O mesmo que \textunderscore gambôa\textunderscore ^1.
\section{Camboada}
\begin{itemize}
\item {Grp. gram.:f.}
\end{itemize}
Acto de \textunderscore camboar\textunderscore .
\section{Camboar}
\begin{itemize}
\item {Grp. gram.:v. i.}
\end{itemize}
\begin{itemize}
\item {Utilização:Prov.}
\end{itemize}
\begin{itemize}
\item {Utilização:trasm.}
\end{itemize}
\begin{itemize}
\item {Utilização:dur.}
\end{itemize}
\begin{itemize}
\item {Proveniência:(De \textunderscore cambão\textunderscore )}
\end{itemize}
Jungir ao carro duas ou três juntas de bois, para subir ladeira.
\section{Camboatá}
\begin{itemize}
\item {Grp. gram.:m.}
\end{itemize}
\begin{itemize}
\item {Utilização:Bras}
\end{itemize}
Peixe de água dôce.
Árvore sapindácea.
\section{Cambocá}
\begin{itemize}
\item {Grp. gram.:m.}
\end{itemize}
Fruta do Brasil.
\section{Camboeira}
\begin{itemize}
\item {Grp. gram.:f.}
\end{itemize}
\begin{itemize}
\item {Utilização:Prov.}
\end{itemize}
\begin{itemize}
\item {Utilização:minh.}
\end{itemize}
\begin{itemize}
\item {Proveniência:(De \textunderscore cambôa\textunderscore ^2)}
\end{itemize}
Rede, que se emprega na pesca da cambôa.
\section{Cambógia}
\begin{itemize}
\item {Grp. gram.:f.}
\end{itemize}
Goma resina, extrahida de várias plantas de Sião, Índia, etc.
(Cp. \textunderscore cambojano\textunderscore )
\section{Cambói}
\begin{itemize}
\item {Grp. gram.:m.}
\end{itemize}
Nome de uma fruta do Brasil; o mesmo que \textunderscore cambúi\textunderscore ?
\section{Cambojano}
\begin{itemize}
\item {Grp. gram.:adj.}
\end{itemize}
\begin{itemize}
\item {Grp. gram.:M.}
\end{itemize}
Relativo a Camboja.
Habitante de Camboja.
Língua dêste país.
\section{Cambola}
\begin{itemize}
\item {Grp. gram.:f.}
\end{itemize}
\begin{itemize}
\item {Utilização:T. de Moçambique}
\end{itemize}
Corda, feita de fibras vegetaes.
\section{Cambolação}
\begin{itemize}
\item {Grp. gram.:f.}
\end{itemize}
\begin{itemize}
\item {Utilização:T. de Angola}
\end{itemize}
Engajamento de comitivas de carregadores do interior da África.
\section{Cambolim}
\begin{itemize}
\item {Grp. gram.:m.}
\end{itemize}
Tecido grosso da Pérsia.
Manta dêsse tecido.
\section{Cambona}
\begin{itemize}
\item {Grp. gram.:f.}
\end{itemize}
\begin{itemize}
\item {Utilização:Náut.}
\end{itemize}
\begin{itemize}
\item {Utilização:Náut.}
\end{itemize}
\begin{itemize}
\item {Grp. gram.:Adj. f.}
\end{itemize}
\begin{itemize}
\item {Proveniência:(De \textunderscore cambar\textunderscore )}
\end{itemize}
Mudança rápida das velas.
Mudança rápida de rumo, na direcção das velas.
\textunderscore Fazer cambona\textunderscore , dar uma reviravolta.
Dizia-se da embarcação que se inclina para o lado, por falta de lastro.
\section{Cambondo}
\begin{itemize}
\item {Grp. gram.:adj.}
\end{itemize}
\begin{itemize}
\item {Utilização:Bras}
\end{itemize}
Amancebado.
\section{Cambonja}
\begin{itemize}
\item {Grp. gram.:f.}
\end{itemize}
Ave pernalta, originária de Angola.
\section{Cambonzo}
\begin{itemize}
\item {Grp. gram.:m.}
\end{itemize}
Gato bravo, africano.
\section{Cambota}
\begin{itemize}
\item {Grp. gram.:f.}
\end{itemize}
\begin{itemize}
\item {Proveniência:(De \textunderscore cambar\textunderscore ^1)}
\end{itemize}
Molde semi-circular, para armação de abóbadas ou arcos.
\section{Cambotas}
\begin{itemize}
\item {Grp. gram.:m.}
\end{itemize}
\begin{itemize}
\item {Utilização:Prov.}
\end{itemize}
\begin{itemize}
\item {Utilização:beir.}
\end{itemize}
Aquelle que tem as pernas tortas.
(Cp. \textunderscore cambo\textunderscore )
\section{Cambra}
\begin{itemize}
\item {Grp. gram.:f.}
\end{itemize}
\begin{itemize}
\item {Utilização:Pop.}
\end{itemize}
\begin{itemize}
\item {Utilização:Ant.}
\end{itemize}
O mesmo que \textunderscore câmara\textunderscore .
Abóbada.
Construcção arqueada.
\section{Cambraia}
\begin{itemize}
\item {Grp. gram.:f.}
\end{itemize}
\begin{itemize}
\item {Utilização:Bras. do N}
\end{itemize}
\begin{itemize}
\item {Proveniência:(De \textunderscore Cambray\textunderscore , n. p.)}
\end{itemize}
Tecido fino e transparente, de linho ou algodão.
Carneiro branco.
\section{Cambraieta}
\begin{itemize}
\item {fónica:ê}
\end{itemize}
\begin{itemize}
\item {Grp. gram.:f.}
\end{itemize}
Cambraia ordinária.
\section{Cambraínha}
\begin{itemize}
\item {Grp. gram.:f.}
\end{itemize}
\begin{itemize}
\item {Utilização:Bras}
\end{itemize}
\begin{itemize}
\item {Utilização:Gír. de Lisb.}
\end{itemize}
Espécie de cambraia, um pouco superior á cambraieta.
Aguardente incolor.
\section{Cambra-mutete}
\begin{itemize}
\item {Grp. gram.:f.}
\end{itemize}
Ave africana, côr de canela, granívora.
\section{Cambrão}
\begin{itemize}
\item {Grp. gram.:m.}
\end{itemize}
\begin{itemize}
\item {Proveniência:(Do lat. \textunderscore crabro\textunderscore )}
\end{itemize}
Espécie de vespa grande.
\section{Cambrão}
\begin{itemize}
\item {Grp. gram.:m.}
\end{itemize}
Fruto da cambroeira.
(Cast. \textunderscore cambron\textunderscore )
\section{Cambrar}
\begin{itemize}
\item {Grp. gram.:v. t.}
\end{itemize}
\begin{itemize}
\item {Utilização:Ant.}
\end{itemize}
\begin{itemize}
\item {Proveniência:(De \textunderscore cambra\textunderscore )}
\end{itemize}
Abobadar, construir, arqueando.
\section{Cambras}
\begin{itemize}
\item {Grp. gram.:f. pl.}
\end{itemize}
\begin{itemize}
\item {Utilização:Pop.}
\end{itemize}
Soltura de ventre, diarreia.
(Por \textunderscore cãimbras\textunderscore , de \textunderscore cãimbra\textunderscore ?)
\section{Cambriano}
\begin{itemize}
\item {Grp. gram.:adj.}
\end{itemize}
\begin{itemize}
\item {Utilização:Geol.}
\end{itemize}
\begin{itemize}
\item {Proveniência:(De \textunderscore Câmbria\textunderscore , n. p.)}
\end{itemize}
Diz-se de uma espécie de terreno paleozoico.
\section{Câmbrico}
\begin{itemize}
\item {Grp. gram.:m.}
\end{itemize}
\begin{itemize}
\item {Grp. gram.:Adj.}
\end{itemize}
\begin{itemize}
\item {Utilização:Geol.}
\end{itemize}
\begin{itemize}
\item {Proveniência:(De \textunderscore Câmbria\textunderscore , n. p.)}
\end{itemize}
Idioma céltico, que comprehendia três dialectos principaes: o \textunderscore câmbico\textunderscore  ou \textunderscore gallês\textunderscore , o \textunderscore córnico\textunderscore  e o \textunderscore armórico\textunderscore  ou \textunderscore bretão\textunderscore .
Relativo ao país de Galles.
O mesmo que \textunderscore cambriano\textunderscore .
\section{Cambrina}
\begin{itemize}
\item {Grp. gram.:f.}
\end{itemize}
\begin{itemize}
\item {Utilização:Prov.}
\end{itemize}
\begin{itemize}
\item {Utilização:trasm.}
\end{itemize}
Caramelo.
\section{Cambroeira}
\begin{itemize}
\item {Grp. gram.:f.}
\end{itemize}
Planta espinhosa, da fam. das solâneas.
(Cp. cast. \textunderscore cambronera\textunderscore )
\section{Cambuaaca}
\begin{itemize}
\item {Grp. gram.:f.}
\end{itemize}
Pássaro dentirostro da África.
Ave trepadora africana.
\section{Cambucá}
\begin{itemize}
\item {Grp. gram.:m.}
\end{itemize}
Planta myrtácea do Brasil.
Fruto dessa planta.
\section{Cambucazeiro}
\begin{itemize}
\item {Grp. gram.:m.}
\end{itemize}
\begin{itemize}
\item {Utilização:Bras}
\end{itemize}
\begin{itemize}
\item {Proveniência:(De \textunderscore cambucá\textunderscore )}
\end{itemize}
Árvore myrtácea da América; o mesmo que \textunderscore cambucá\textunderscore .
\section{Cambuci}
\begin{itemize}
\item {Grp. gram.:m.}
\end{itemize}
\begin{itemize}
\item {Utilização:Bras}
\end{itemize}
\begin{itemize}
\item {Proveniência:(T. tupi)}
\end{itemize}
Árvore myrtácea da América.
Fruto dessa árvore.
\section{Cambudice}
\begin{itemize}
\item {Grp. gram.:f.}
\end{itemize}
Qualidade daquillo que é cambudo.
\section{Cambudo}
\begin{itemize}
\item {Grp. gram.:adj.}
\end{itemize}
\begin{itemize}
\item {Proveniência:(De \textunderscore camba\textunderscore )}
\end{itemize}
Adunco.
\section{Cambuí}
\begin{itemize}
\item {Grp. gram.:m.}
\end{itemize}
\begin{itemize}
\item {Utilização:Bras}
\end{itemize}
Fruto de cambuizeiro.
O mesmo que \textunderscore cambuizeiro\textunderscore .
\section{Cambuizeiro}
\begin{itemize}
\item {fónica:bu-í}
\end{itemize}
\begin{itemize}
\item {Grp. gram.:m.}
\end{itemize}
\begin{itemize}
\item {Utilização:Bras}
\end{itemize}
Planta myrtácea.
\section{Cambulhada}
\begin{itemize}
\item {Grp. gram.:f.}
\end{itemize}
\begin{itemize}
\item {Proveniência:(De \textunderscore cambulho\textunderscore )}
\end{itemize}
Porção de cambulhos.
Cambada.
Confusão, desordem.
\section{Cambulho}
\begin{itemize}
\item {Grp. gram.:m.}
\end{itemize}
\begin{itemize}
\item {Utilização:T. do Fundão}
\end{itemize}
Rodelazinha de barro, com um buraco ao meio, usada por pescadores para fundearem as redes no mar.
Pessôa disforme e mal vestida; estafermo.
(Cp. cast. \textunderscore cambujo\textunderscore )
\section{Cambulubo}
\begin{itemize}
\item {Grp. gram.:m.}
\end{itemize}
Insecto angolense, que habita nos troncos das árvores e que, esmagado, exhala aroma agradável.
\section{Cambungo}
\begin{itemize}
\item {Grp. gram.:m.}
\end{itemize}
(V.cacongo)
\section{Cambuquira}
\begin{itemize}
\item {Grp. gram.:f.}
\end{itemize}
\begin{itemize}
\item {Utilização:Bras}
\end{itemize}
Grelos da aboboreira, que se guisam com outras ervas.
\section{Cambuta}
\begin{itemize}
\item {Grp. gram.:m.  e  f.}
\end{itemize}
Pessôa, que é cambaia das pernas.
\section{Cambuto}
\begin{itemize}
\item {Grp. gram.:adj.}
\end{itemize}
O mesmo que \textunderscore cambado\textunderscore .
\section{Camchadal}
\begin{itemize}
\item {Grp. gram.:m.}
\end{itemize}
\begin{itemize}
\item {Proveniência:(De \textunderscore Camchatca\textunderscore , n. p.)}
\end{itemize}
Língua hiperbórea, agglutinativa, do Nordéste da Ásia.
\section{Cameação}
\begin{itemize}
\item {Grp. gram.:f.}
\end{itemize}
Acto de \textunderscore camear\textunderscore .
\section{Camear}
\begin{itemize}
\item {Grp. gram.:v. t.}
\end{itemize}
\begin{itemize}
\item {Utilização:Prov.}
\end{itemize}
\begin{itemize}
\item {Utilização:dur.}
\end{itemize}
\begin{itemize}
\item {Proveniência:(De \textunderscore cama\textunderscore )}
\end{itemize}
Fazer a cama ou mergulhia de (videiras).
\section{Camearans}
\begin{itemize}
\item {Grp. gram.:m. pl.}
\end{itemize}
Designação genérica de cinco tríbos de Índios do Brasil, entre Goiás e o Pará.
\section{Camédrios}
\begin{itemize}
\item {Grp. gram.:m.}
\end{itemize}
Planta da serra de Sintra, também conhecida por \textunderscore erva-carvalhinha\textunderscore . Cf. \textunderscore Pharmac. Port.\textunderscore 
\section{Camelão}
\begin{itemize}
\item {Grp. gram.:m.}
\end{itemize}
\begin{itemize}
\item {Proveniência:(De \textunderscore camelo\textunderscore )}
\end{itemize}
Pano de pêlo de cabra.
\section{Camelaria}
\begin{itemize}
\item {Grp. gram.:f.}
\end{itemize}
\begin{itemize}
\item {Utilização:Fig.}
\end{itemize}
Porção de homens estúpidos.
Camelice. Cf. Castilho, \textunderscore Tartufo\textunderscore , 25.
\section{Cameleão}
\begin{itemize}
\item {Grp. gram.:m.}
\end{itemize}
O mesmo ou melhor que \textunderscore camaleão\textunderscore .
\section{Cameleia}
\begin{itemize}
\item {Grp. gram.:f.}
\end{itemize}
\begin{itemize}
\item {Proveniência:(Gr. \textunderscore kamelaia\textunderscore )}
\end{itemize}
Planta rutácea.
\section{Cameleira}
\begin{itemize}
\item {Grp. gram.:f.}
\end{itemize}
\begin{itemize}
\item {Utilização:Bras}
\end{itemize}
\begin{itemize}
\item {Utilização:Neol.}
\end{itemize}
Camélia, planta.
\section{Cameleiro}
\begin{itemize}
\item {Grp. gram.:m.}
\end{itemize}
Condutor de camelos.
\section{Camelete}
\begin{itemize}
\item {fónica:lê}
\end{itemize}
\begin{itemize}
\item {Grp. gram.:m.}
\end{itemize}
\begin{itemize}
\item {Utilização:Ant.}
\end{itemize}
\begin{itemize}
\item {Proveniência:(De \textunderscore camelo\textunderscore )}
\end{itemize}
Pequena peça de artilharia.
\section{Camélia}
\begin{itemize}
\item {Grp. gram.:f.}
\end{itemize}
\begin{itemize}
\item {Proveniência:(De \textunderscore Camelli\textunderscore , n. p.)}
\end{itemize}
Arbusto ornamental, da fam. das theáceas.
A flôr dêsse arbusto.
\section{Cameliáceas}
\begin{itemize}
\item {Grp. gram.:f. pl.}
\end{itemize}
Família de plantas, que têm por typo a camélia.
\section{Camelice}
\begin{itemize}
\item {Grp. gram.:f.}
\end{itemize}
\begin{itemize}
\item {Utilização:Pop.}
\end{itemize}
\begin{itemize}
\item {Proveniência:(De \textunderscore camelo\textunderscore )}
\end{itemize}
Tolice, estupidez.
\section{Camelídeos}
\begin{itemize}
\item {Grp. gram.:m. pl.}
\end{itemize}
\begin{itemize}
\item {Proveniência:(Do gr. \textunderscore kamelos\textunderscore  + \textunderscore eidos\textunderscore )}
\end{itemize}
Família de quadrúpedes ruminantes, em que está comprehendido o camelo.
\section{Cameliforme}
\begin{itemize}
\item {Grp. gram.:adj.}
\end{itemize}
\begin{itemize}
\item {Proveniência:(De \textunderscore camelo\textunderscore  + \textunderscore fórma\textunderscore )}
\end{itemize}
Semelhante ao camelo.
\section{Camelina}
\begin{itemize}
\item {Grp. gram.:f.}
\end{itemize}
\begin{itemize}
\item {Proveniência:(Fr. \textunderscore cameline\textunderscore )}
\end{itemize}
Planta crucífera.
\section{Camelino}
\begin{itemize}
\item {Grp. gram.:adj.}
\end{itemize}
\begin{itemize}
\item {Proveniência:(Lat. \textunderscore camelinus\textunderscore )}
\end{itemize}
Relativo ao camelo: \textunderscore a giba camelina\textunderscore .
\section{Camelo}
\begin{itemize}
\item {fónica:mê}
\end{itemize}
\begin{itemize}
\item {Grp. gram.:m.}
\end{itemize}
\begin{itemize}
\item {Utilização:Fig.}
\end{itemize}
\begin{itemize}
\item {Proveniência:(Lat. \textunderscore camelus\textunderscore )}
\end{itemize}
Quadrúpede, que tem gibas sôbre o dorso.
Homem estúpido, brutal.
Calabre.
Antiga peça de artilharia de pequeno calibre.
\section{Camelopárdale}
\begin{itemize}
\item {Grp. gram.:m.}
\end{itemize}
\begin{itemize}
\item {Proveniência:(Lat. \textunderscore camelopárdalis\textunderscore )}
\end{itemize}
Antigo nome da girafa.
Constellação boreal.
\section{Camelório}
\begin{itemize}
\item {Grp. gram.:m.}
\end{itemize}
\begin{itemize}
\item {Utilização:Fam.}
\end{itemize}
\begin{itemize}
\item {Proveniência:(De \textunderscore camelo\textunderscore )}
\end{itemize}
Parvalhão.
\section{Camelornitho}
\begin{itemize}
\item {Grp. gram.:m.}
\end{itemize}
\begin{itemize}
\item {Proveniência:(Do gr. \textunderscore kamelos\textunderscore  + \textunderscore ornis\textunderscore )}
\end{itemize}
Nome das aves semelhantes ao avestruz.
\section{Camelornito}
\begin{itemize}
\item {Grp. gram.:m.}
\end{itemize}
\begin{itemize}
\item {Proveniência:(Do gr. \textunderscore kamelos\textunderscore  + \textunderscore ornis\textunderscore )}
\end{itemize}
Nome das aves semelhantes ao avestruz.
\section{Camelote}
\begin{itemize}
\item {Grp. gram.:m.}
\end{itemize}
\begin{itemize}
\item {Utilização:Gír.}
\end{itemize}
Espólio.
\section{Camena}
\begin{itemize}
\item {Grp. gram.:f.}
\end{itemize}
\begin{itemize}
\item {Proveniência:(Lat. \textunderscore camena\textunderscore )}
\end{itemize}
O mesmo que \textunderscore musa\textunderscore ^1.
\section{Camengamenha}
\begin{itemize}
\item {Grp. gram.:f.}
\end{itemize}
Ave pernalta africana.
\section{Câmera}
\begin{itemize}
\item {Grp. gram.:f.}
\end{itemize}
O mesmo ou melhor que \textunderscore câmara\textunderscore .
\section{Cameral}
\begin{itemize}
\item {Grp. gram.:adj.}
\end{itemize}
\begin{itemize}
\item {Utilização:Bot.}
\end{itemize}
\begin{itemize}
\item {Proveniência:(De \textunderscore câmera\textunderscore )}
\end{itemize}
Relativo á cavidade anterior de um vegetal. Cf. Latino, \textunderscore Humboldt\textunderscore , 71, 86 e 90.
\section{Cameritela}
\begin{itemize}
\item {Grp. gram.:f.}
\end{itemize}
\begin{itemize}
\item {Proveniência:(Do lat. \textunderscore camera\textunderscore  + \textunderscore tela\textunderscore )}
\end{itemize}
Espécie de aranha, cuja teia fechada lhe serve de habitação.
\section{Camerlengo}
\begin{itemize}
\item {Grp. gram.:adj.}
\end{itemize}
(V.camarlengo)
\section{Cameróstomo}
\begin{itemize}
\item {Grp. gram.:m.}
\end{itemize}
\begin{itemize}
\item {Proveniência:(Do gr. \textunderscore kamera\textunderscore  + \textunderscore stoma\textunderscore )}
\end{itemize}
Parte exterior do corpo das aranhas.
\section{Camérula}
\begin{itemize}
\item {Grp. gram.:f.}
\end{itemize}
\begin{itemize}
\item {Proveniência:(Lat. \textunderscore camerula\textunderscore )}
\end{itemize}
Pequena cavidade, dentro de um vegetal.
\section{Camião}
\begin{itemize}
\item {Grp. gram.:m.}
\end{itemize}
\begin{itemize}
\item {Proveniência:(Fr. \textunderscore camion\textunderscore )}
\end{itemize}
Carreta de três rodas, em que os carregadores de estações de caminhos de ferro transportam bagagens ou mercadorias, nas estações e nos caes.
\section{Camiche}
\begin{itemize}
\item {Grp. gram.:m.}
\end{itemize}
Gênero de aves carnívoras da América do Sul.
\section{Camichi}
\begin{itemize}
\item {Grp. gram.:m.}
\end{itemize}
Nome de duas espécies de aves pernaltas.
O mesmo que \textunderscore camiche\textunderscore ?
\section{Camichim}
\begin{itemize}
\item {Grp. gram.:m.}
\end{itemize}
Espécie de figo silvestre do México.
\section{Camilha}
\begin{itemize}
\item {Grp. gram.:f.}
\end{itemize}
\begin{itemize}
\item {Proveniência:(De \textunderscore cama\textunderscore )}
\end{itemize}
Cama pequena.
Canapé ou encôsto, para nelle se dormir a sesta ou descansar.
\section{Camiliana}
\begin{itemize}
\item {Grp. gram.:f.}
\end{itemize}
Colecção das obras de Camilo Castelo-Branco, ou de escritos relativos a êsse escritor.
\section{Camilliana}
\begin{itemize}
\item {Grp. gram.:f.}
\end{itemize}
Collecção das obras de Camillo Castello-Branco, ou de escritos relativos a êsse escritor.
\section{Camillo}
\begin{itemize}
\item {Grp. gram.:adj.}
\end{itemize}
Dizia-se do frade da Ordem de San-Camillo.
\section{Camilo}
\begin{itemize}
\item {Grp. gram.:adj.}
\end{itemize}
Dizia-se do frade da Ordem de San-Camilo.
\section{Camina}
\begin{itemize}
\item {Grp. gram.:f.}
\end{itemize}
\begin{itemize}
\item {Utilização:Bras. do N}
\end{itemize}
Armadilha de pesca.
(Talvez t. tupi)
\section{Caminhada}
\begin{itemize}
\item {Grp. gram.:f.}
\end{itemize}
Acção de \textunderscore caminhar\textunderscore .
Passeio, jornada.
Grande extensão de caminho a percorrer.
\section{Caminhador}
\begin{itemize}
\item {Grp. gram.:m.  e  adj.}
\end{itemize}
\begin{itemize}
\item {Proveniência:(De \textunderscore caminhar\textunderscore )}
\end{itemize}
O que anda muito, sem se fatigar.
Andarilho.
\section{Caminhante}
\begin{itemize}
\item {Grp. gram.:m.  e  f.}
\end{itemize}
\begin{itemize}
\item {Grp. gram.:Adj.}
\end{itemize}
Pessôa que caminha.
Viandante, transeunte.
Que caminha. Cf. Filinto, XVII, 108.
\section{Caminhão}
\begin{itemize}
\item {Grp. gram.:m.}
\end{itemize}
\begin{itemize}
\item {Utilização:Bras}
\end{itemize}
Carro de carga, com quatro rodas e almofada.
(Cp. \textunderscore camião\textunderscore )
\section{Caminhar}
\begin{itemize}
\item {Grp. gram.:v. t.}
\end{itemize}
\begin{itemize}
\item {Grp. gram.:V. i.}
\end{itemize}
Percorrer caminho a pé; andar.
Percorrer, andando.
\section{Caminheira}
\begin{itemize}
\item {Grp. gram.:f.}
\end{itemize}
Espécie de locomotiva. Cf. G. Vianna, \textunderscore Apostilas\textunderscore .
\section{Caminheiro}
\begin{itemize}
\item {Grp. gram.:m.  e  adj.}
\end{itemize}
\begin{itemize}
\item {Proveniência:(De \textunderscore caminhar\textunderscore )}
\end{itemize}
O que anda muito, a pé.
Correio, recoveiro.
Viandante.
\section{Caminho}
\begin{itemize}
\item {Grp. gram.:m.}
\end{itemize}
\begin{itemize}
\item {Grp. gram.:Loc. adv.}
\end{itemize}
Faixa de terreno, por onde se transita, de um ponto para outro.
Estrada.
Atalho.
Direcção.
Espaço que se percorre, andando.
Distância.
Norma de proceder.
Destino.
Tendência.
\textunderscore De caminho\textunderscore , seguidamente, logo. Cf. Castilho, \textunderscore Fausto\textunderscore , 162.
\textunderscore Caminho de pé pôsto\textunderscore , atalho, carreiro. Cf. Filinto, I, 51.
\textunderscore Caminho de cabras\textunderscore , caminho estreito, íngreme e accidentado.
\textunderscore Caminho de ferro\textunderscore , systema de viação por meio de vehículos que se movem a vapor, sobre carris de ferro.
(B. lat. \textunderscore caminus\textunderscore )
\section{Caminologia}
\begin{itemize}
\item {Grp. gram.:f.}
\end{itemize}
\begin{itemize}
\item {Proveniência:(Do gr. \textunderscore kaminos\textunderscore  + \textunderscore logos\textunderscore )}
\end{itemize}
Tratado da construcção das chaminés.
\section{Camionagem}
\begin{itemize}
\item {Grp. gram.:f.}
\end{itemize}
\begin{itemize}
\item {Proveniência:(De \textunderscore camião\textunderscore )}
\end{itemize}
Acto de entregar nos domicílios mercadorias que foram transportadas em combóio.
\section{Camira}
\begin{itemize}
\item {Grp. gram.:f.}
\end{itemize}
Gênero de plantas crucíferas.
\section{Camiranga}
\begin{itemize}
\item {Grp. gram.:m.}
\end{itemize}
\begin{itemize}
\item {Utilização:Bras. do N}
\end{itemize}
Urubu de bico vermelho.
\section{Camisa}
\begin{itemize}
\item {Grp. gram.:f.}
\end{itemize}
\begin{itemize}
\item {Utilização:Sal.}
\end{itemize}
\begin{itemize}
\item {Utilização:Ant.}
\end{itemize}
\begin{itemize}
\item {Utilização:Bras}
\end{itemize}
\begin{itemize}
\item {Utilização:Constr.}
\end{itemize}
Vestuário de linho, algodão ou de outro tecido fino, com mangas, que se usa por baixo do outro fato e vai do pescoço ás côxas.
Argamassa, com que se reboca uma construcção.
Pellícula, que envolve a espiga do milho.
Envoltório.
Ligeira cobertura de sal, no fundo dos meios das marinhas.
Feltro, que vem da fula para se apropriar ou concluir o chapéu.
Muralha de recinto fortificado.
Envoltório de fôlha de cobre, com que se forra o tronco das árvores, na parte que fica no subsolo, para que os rebentos não prejudiquem o calçamento das ruas.
\textunderscore Fôrro de camisa e saia\textunderscore , fôrro dobrado ou sobreposto.
(B. lat. \textunderscore camisia\textunderscore )
\section{Camisa-do-pano}
\begin{itemize}
\item {Grp. gram.:f.}
\end{itemize}
\begin{itemize}
\item {Utilização:Náut.}
\end{itemize}
Parte da vela ferrada, que resai do centro, em fórma aproximada de triângulo.
\section{Camisão}
\begin{itemize}
\item {Grp. gram.:m.}
\end{itemize}
\begin{itemize}
\item {Utilização:Açor}
\end{itemize}
\begin{itemize}
\item {Utilização:ant.}
\end{itemize}
\begin{itemize}
\item {Utilização:Fam.}
\end{itemize}
\begin{itemize}
\item {Proveniência:(De \textunderscore camisa\textunderscore )}
\end{itemize}
Camisa grande.
Vestuário antigo, semelhante ás alvas dos padres.
Plebeu, que desempenha os serviços mais grosseiros.
Homem lisonjeiro, que approva quanto outro diz.
\section{Camisaria}
\begin{itemize}
\item {Grp. gram.:f.}
\end{itemize}
Estabelecimento, onde se fabricam ou se vendem camisas.
\section{Camiseira}
\begin{itemize}
\item {Grp. gram.:f.}
\end{itemize}
Costureira de camisas.
\section{Camiseiro}
\begin{itemize}
\item {Grp. gram.:m.}
\end{itemize}
\begin{itemize}
\item {Grp. gram.:Adj.}
\end{itemize}
Fabricante ou vendedor de camisas.
Próprio para camisas:«\textunderscore deixo uma peça de linho camiseiro...\textunderscore »(De um testamento de 1691)
\section{Camiseta}
\begin{itemize}
\item {fónica:misê}
\end{itemize}
\begin{itemize}
\item {Grp. gram.:f.}
\end{itemize}
\begin{itemize}
\item {Utilização:Prov.}
\end{itemize}
\begin{itemize}
\item {Utilização:alg.}
\end{itemize}
Camisa de pano, mais ou menos transparente, usada por algumas mulheres sôbre outra camisa.
O mesmo que \textunderscore camisola\textunderscore .
\section{Camisola}
\begin{itemize}
\item {Grp. gram.:f.}
\end{itemize}
Espécie de camisa curta, que se usa ordinariamente sôbre a pelle.
Blusa de operário ou marinheiro.
(\textunderscore De camisa\textunderscore )
\section{Camisoleira}
\begin{itemize}
\item {Grp. gram.:f.}
\end{itemize}
Mulher, que trabalha no fabrico de camisolas.
\section{Camisoleiro}
\begin{itemize}
\item {Grp. gram.:m.}
\end{itemize}
Fabricante ou vendedor de camisolas.
\section{Camisote}
\begin{itemize}
\item {Grp. gram.:m.}
\end{itemize}
\begin{itemize}
\item {Proveniência:(De \textunderscore camisa\textunderscore )}
\end{itemize}
Antiga armadura.
Camisa fina de seda, cassa, etc.
\section{Camita}
\begin{itemize}
\item {Grp. gram.:adj.}
\end{itemize}
\begin{itemize}
\item {Grp. gram.:M.}
\end{itemize}
Relativo a Cam, filho de Noé.
Descendente de Noé.
\section{Camítico}
\begin{itemize}
\item {Grp. gram.:adj.}
\end{itemize}
Relativo aos camitas.
\section{Camoéca}
\begin{itemize}
\item {Grp. gram.:f.}
\end{itemize}
\begin{itemize}
\item {Utilização:Pop.}
\end{itemize}
Embriaguez.
Torpor.
\section{Camões}
\begin{itemize}
\item {Grp. gram.:m.}
\end{itemize}
\begin{itemize}
\item {Utilização:Prov.}
\end{itemize}
\begin{itemize}
\item {Utilização:beir.}
\end{itemize}
\begin{itemize}
\item {Utilização:pop.}
\end{itemize}
\begin{itemize}
\item {Proveniência:(De \textunderscore Camões\textunderscore , n. p.)}
\end{itemize}
Homem vesgo ou cego de um ôlho.
\section{Camoês}
\begin{itemize}
\item {Grp. gram.:adj.}
\end{itemize}
Diz-se de uma casta de peros e maçans.
\section{Camoêsa}
\begin{itemize}
\item {Grp. gram.:f.}
\end{itemize}
\begin{itemize}
\item {Utilização:Prov.}
\end{itemize}
O mesmo que \textunderscore camoéca\textunderscore .
\section{Camoísta}
\begin{itemize}
\item {Grp. gram.:m.}
\end{itemize}
\begin{itemize}
\item {Utilização:Ant.}
\end{itemize}
O mesmo que \textunderscore camonianista\textunderscore .
\section{Camol}
\begin{itemize}
\item {Grp. gram.:m.}
\end{itemize}
(V.camalaçana)
\section{Camolenga}
\begin{itemize}
\item {Grp. gram.:f.}
\end{itemize}
Abóbora, o mesmo que \textunderscore jirimu\textunderscore .
\section{Camomila}
\begin{itemize}
\item {Grp. gram.:f.}
\end{itemize}
\begin{itemize}
\item {Proveniência:(Do gr. \textunderscore kamai\textunderscore  + \textunderscore melon\textunderscore )}
\end{itemize}
O mesmo que \textunderscore macela\textunderscore .
Designação de várias plantas da fam. das compostas.
\section{Camomilha}
\begin{itemize}
\item {Grp. gram.:f.}
\end{itemize}
(V.camomila)
\section{Camondongo}
\begin{itemize}
\item {Grp. gram.:m.}
\end{itemize}
\begin{itemize}
\item {Utilização:Bras}
\end{itemize}
O mesmo que \textunderscore camundongo\textunderscore .
\section{Camoniana}
\begin{itemize}
\item {Grp. gram.:f.}
\end{itemize}
\begin{itemize}
\item {Proveniência:(De \textunderscore camoniano\textunderscore )}
\end{itemize}
Collecção das obras de Camões ou dos escritos relativos a essas obras ou a Camões.
\section{Camonianista}
\begin{itemize}
\item {Grp. gram.:m.}
\end{itemize}
Aquelle que faz camonianas.
\section{Camoniano}
\begin{itemize}
\item {Grp. gram.:adj.}
\end{itemize}
\begin{itemize}
\item {Grp. gram.:M.}
\end{itemize}
Relativo a Camões.
Que reflecte o estilo de Camões.
Admirador de Camões; colleccionador das obras dêste poèta e de tudo que lhe diga respeito.
\section{Camonista}
\begin{itemize}
\item {Grp. gram.:m.  e  adj.}
\end{itemize}
O mesmo que \textunderscore camonianista\textunderscore .
\section{Camopim}
\begin{itemize}
\item {Grp. gram.:m.}
\end{itemize}
\begin{itemize}
\item {Utilização:Bras}
\end{itemize}
Planta medicinal do norte do Brasil.
\section{Camoquenque}
\begin{itemize}
\item {Grp. gram.:m.}
\end{itemize}
\begin{itemize}
\item {Utilização:Bras}
\end{itemize}
Espécie de mandioca de talo e raíz brancos.
\section{Camorupim}
\begin{itemize}
\item {Grp. gram.:m.}
\end{itemize}
\begin{itemize}
\item {Utilização:Bras. do N}
\end{itemize}
Nome de um peixe brasileiro.
Cavallo velho e magro.
\section{Camote}
\begin{itemize}
\item {Grp. gram.:m.}
\end{itemize}
Espécie de batata grande da América do Sul, (\textunderscore batata indica\textunderscore ).
\section{Camouco}
\begin{itemize}
\item {Grp. gram.:m.}
\end{itemize}
Crosta ou côdea pedregosa.
\section{Campa}
\begin{itemize}
\item {Grp. gram.:f.}
\end{itemize}
\begin{itemize}
\item {Proveniência:(Do rad. de \textunderscore campo\textunderscore ?)}
\end{itemize}
Pedra, que cobre a sepultura; sepultura.
\section{Campa}
\begin{itemize}
\item {Grp. gram.:f.}
\end{itemize}
Sineta de igreja ou de communidade.
(Cp. \textunderscore campaínha\textunderscore )
\section{Campação}
\begin{itemize}
\item {Grp. gram.:f.}
\end{itemize}
Acto de \textunderscore campar\textunderscore .
\section{Campador}
\begin{itemize}
\item {Grp. gram.:m.}
\end{itemize}
\begin{itemize}
\item {Utilização:Bras}
\end{itemize}
\begin{itemize}
\item {Proveniência:(De \textunderscore campar\textunderscore )}
\end{itemize}
Aquelle que passeia de noite em cata de amores.
\section{Campaínha}
\begin{itemize}
\item {Grp. gram.:f.}
\end{itemize}
\begin{itemize}
\item {Grp. gram.:Pl.}
\end{itemize}
\begin{itemize}
\item {Proveniência:(Do med. lat. \textunderscore campana\textunderscore )}
\end{itemize}
Pequeno utensílio, da mesma substância que os sinos, e de fórma semelhante á dêstes.
Flôr das plantas campanuláceas.
Antigo instrumento de percussão, usado nas grandes bandas militares e composto de uma haste que sustentava uma espécie de chapéu chinês de metal, guarnecido de guisos e campaínhas.
O mesmo que [[amýgdalas|amýgdala]].
\section{Campainhada}
\begin{itemize}
\item {fónica:pa-i}
\end{itemize}
\begin{itemize}
\item {Grp. gram.:f.}
\end{itemize}
Som de campaínha; toque de campaínha.
\section{Campainhão}
\begin{itemize}
\item {fónica:pa-i}
\end{itemize}
\begin{itemize}
\item {Grp. gram.:m.}
\end{itemize}
Aquelle que leva a campaínha nas procissões.
Andador de irmandade.
\section{Campainheiro}
\begin{itemize}
\item {fónica:pa-i}
\end{itemize}
\begin{itemize}
\item {Grp. gram.:m.}
\end{itemize}
Aquelle que leva a campaínha nas procissões.
Andador de irmandade.
\section{Campaiões}
\begin{itemize}
\item {fónica:pa-i}
\end{itemize}
\begin{itemize}
\item {Grp. gram.:m. pl.}
\end{itemize}
\begin{itemize}
\item {Utilização:Prov.}
\end{itemize}
\begin{itemize}
\item {Utilização:trasm.}
\end{itemize}
Flôres amarelas e mimosas, de corolla campanulada.
(Cp. \textunderscore campaínha\textunderscore )
\section{Campal}
\begin{itemize}
\item {Grp. gram.:adj.}
\end{itemize}
Relativo a campo.
Que se realiza no campo: \textunderscore Missa campal\textunderscore .
\section{Campana}
\begin{itemize}
\item {Grp. gram.:f.}
\end{itemize}
O mesmo que \textunderscore campaínha\textunderscore .
Variedade de pêra, talvez desconhecida hoje.
Abertura larga, em fórma de sino, que alguns instrumentos músicos têm na extremidade opposta á embocadura.
Parte superior de uma manilha, destinada a fazer juncção com outra manilha.
(Med. lat. \textunderscore campana\textunderscore )
\section{Campanado}
\begin{itemize}
\item {Grp. gram.:adj.}
\end{itemize}
\begin{itemize}
\item {Proveniência:(De \textunderscore campana\textunderscore )}
\end{itemize}
Que tem fórma de campaínha.
\section{Campanário}
\begin{itemize}
\item {Grp. gram.:m.}
\end{itemize}
\begin{itemize}
\item {Utilização:Ext.}
\end{itemize}
\begin{itemize}
\item {Proveniência:(De \textunderscore campana\textunderscore )}
\end{itemize}
Tôrre com sinos.
Parte da tôrre, em que se suspende o sino.
Freguesia.
Interesses locaes: \textunderscore questões de campanário\textunderscore .
\section{Campanear}
\begin{itemize}
\item {Grp. gram.:v. i.}
\end{itemize}
\begin{itemize}
\item {Utilização:Prov.}
\end{itemize}
\begin{itemize}
\item {Utilização:trasm.}
\end{itemize}
Caír, despenhando-se.
\section{Campanha}
\begin{itemize}
\item {Grp. gram.:f.}
\end{itemize}
\begin{itemize}
\item {Proveniência:(Lat. \textunderscore campania\textunderscore )}
\end{itemize}
Acampamento de tropas.
Campo de batalha.
Batalha.
Operações militares.
Conjunto de esforços para um fim determinado: \textunderscore a imprensa fez uma campanha contra o Ministro\textunderscore .
Campo, região deshabitada:«\textunderscore penhascos, que assoberbam os valles e a campanha.\textunderscore »M. Bernardes, \textunderscore Luz e Calor\textunderscore .«\textunderscore ...no mar, ou no rio, ou no monte, ou na campanha.\textunderscore »Vieira.
\section{Cam}
\begin{itemize}
\item {Grp. gram.:m.}
\end{itemize}
Chefe supremo de alguns paises asiáticos.
Governador de província, na Pérsia.
(Do \textunderscore mongol\textunderscore )
\section{Campanhista}
\begin{itemize}
\item {Grp. gram.:m.}
\end{itemize}
Soldado, que já tem entrado em campanhas.
\section{Campanhol}
\begin{itemize}
\item {Grp. gram.:m.}
\end{itemize}
Gênero de mammíferos roedores.
(Cast. \textunderscore campañol\textunderscore )
\section{Campaniço}
\begin{itemize}
\item {Grp. gram.:m.}
\end{itemize}
\begin{itemize}
\item {Utilização:Prov.}
\end{itemize}
\begin{itemize}
\item {Utilização:alent.}
\end{itemize}
Habitante de uma região denominada \textunderscore Campo Branco\textunderscore , pertencente aos concelhos de Mértola e Castro-Verde.
(Cp. lat. \textunderscore campaneus\textunderscore )
\section{Campaniforme}
\begin{itemize}
\item {Grp. gram.:adj.}
\end{itemize}
(V.campanulado)
\section{Campanil}
\begin{itemize}
\item {Grp. gram.:m.}
\end{itemize}
\begin{itemize}
\item {Proveniência:(Lat. \textunderscore campanilis\textunderscore )}
\end{itemize}
Metal para sinos.
\section{Campanina}
\begin{itemize}
\item {Grp. gram.:f.}
\end{itemize}
\begin{itemize}
\item {Utilização:Prov.}
\end{itemize}
\begin{itemize}
\item {Grp. gram.:Pl.}
\end{itemize}
\begin{itemize}
\item {Utilização:Prov.}
\end{itemize}
\begin{itemize}
\item {Utilização:trasm.}
\end{itemize}
\begin{itemize}
\item {Proveniência:(De \textunderscore campana\textunderscore )}
\end{itemize}
Sineta que, sôbre o telhado do corpo da igreja, se tange á elevação da hóstia e do cálix.
O mesmo que \textunderscore campaiões\textunderscore .
\section{Campanólogo}
\begin{itemize}
\item {Grp. gram.:m.}
\end{itemize}
\begin{itemize}
\item {Proveniência:(De \textunderscore campana\textunderscore  + gr. \textunderscore logos\textunderscore )}
\end{itemize}
Aquelle que toca música em sinos ou copos.
\section{Campanudo}
\begin{itemize}
\item {Grp. gram.:adj.}
\end{itemize}
\begin{itemize}
\item {Utilização:Fig.}
\end{itemize}
\begin{itemize}
\item {Proveniência:(De \textunderscore campana\textunderscore )}
\end{itemize}
Que tem fórma de sino.
Pomposo; bombástico: \textunderscore discurso campanudo\textunderscore .
\section{Campânula}
\begin{itemize}
\item {Grp. gram.:f.}
\end{itemize}
\begin{itemize}
\item {Proveniência:(De \textunderscore campana\textunderscore )}
\end{itemize}
Flôr das plantas campanuláceas.
Pequeno vaso, em fórma de sino.
\section{Campanuláceas}
\begin{itemize}
\item {Grp. gram.:f. pl.}
\end{itemize}
\begin{itemize}
\item {Proveniência:(De \textunderscore campanula\textunderscore )}
\end{itemize}
Família de plantas, cuja corolla tem fórma de campaínha.
\section{Campanulado}
\begin{itemize}
\item {Grp. gram.:adj.}
\end{itemize}
\begin{itemize}
\item {Proveniência:(De \textunderscore campânula\textunderscore )}
\end{itemize}
Que tem fórma de campaínha.
\section{Campanular}
\begin{itemize}
\item {Grp. gram.:v. i.}
\end{itemize}
\begin{itemize}
\item {Proveniência:(De \textunderscore campânula\textunderscore )}
\end{itemize}
Soar como uma campaínha.
\section{Campão}
\begin{itemize}
\item {Grp. gram.:m.}
\end{itemize}
\begin{itemize}
\item {Proveniência:(De \textunderscore Campan\textunderscore , n. p.)}
\end{itemize}
Mármore variegado dos Pyrenéus.
\section{Campão}
\begin{itemize}
\item {Grp. gram.:m.}
\end{itemize}
\begin{itemize}
\item {Utilização:Bras}
\end{itemize}
Campo muito extenso.
\section{Campar}
\begin{itemize}
\item {Grp. gram.:v. i.}
\end{itemize}
\begin{itemize}
\item {Utilização:Bras}
\end{itemize}
\begin{itemize}
\item {Proveniência:(De \textunderscore campo\textunderscore )}
\end{itemize}
Brilhar.
Ufanar-se; fazer ostentação.
Sair-se bem.
Sair de noite, para vêr ou procurar amantes.
O mesmo que \textunderscore acampar\textunderscore .
\section{Camparesco}
\begin{itemize}
\item {fónica:pa-rês}
\end{itemize}
\begin{itemize}
\item {Grp. gram.:adj.}
\end{itemize}
\begin{itemize}
\item {Utilização:Des.}
\end{itemize}
O mesmo que \textunderscore campestre\textunderscore .
\section{Campeação}
\begin{itemize}
\item {Grp. gram.:f.}
\end{itemize}
\begin{itemize}
\item {Utilização:Bras}
\end{itemize}
Acto de campear ou andar a cavallo pelo campo, em procura ou tratamento de gado.
\section{Campeador}
\begin{itemize}
\item {Grp. gram.:adj.}
\end{itemize}
\begin{itemize}
\item {Grp. gram.:M.}
\end{itemize}
\begin{itemize}
\item {Proveniência:(De \textunderscore campear\textunderscore )}
\end{itemize}
Que campeia.
Campeiro.
Campeão.
\section{Campeão}
\begin{itemize}
\item {Grp. gram.:m.}
\end{itemize}
\begin{itemize}
\item {Utilização:Bras}
\end{itemize}
Aquelle que combatia em campo fechado, em honra ou defesa de outrem.
Defensor: \textunderscore os campeões da moralidade\textunderscore .
Combatente.
Cavallo, em que se campeia.
(Infl. de \textunderscore campear\textunderscore ; cp. \textunderscore campião\textunderscore )
\section{Campear}
\begin{itemize}
\item {Grp. gram.:v. i.}
\end{itemize}
\begin{itemize}
\item {Utilização:Bras}
\end{itemize}
\begin{itemize}
\item {Utilização:Bras. do S}
\end{itemize}
\begin{itemize}
\item {Proveniência:(De \textunderscore campo\textunderscore )}
\end{itemize}
Acampar.
Estar em campanha.
Marchar garbosamente.
Sobresair.
Fazer ostentação.
Andar no campo a cavallo, em procura ou tratamento de gado.
Fazer explorações pelo campo, bater campo.
\section{Campeche}
\begin{itemize}
\item {Grp. gram.:m.}
\end{itemize}
\begin{itemize}
\item {Proveniência:(De \textunderscore Campeche\textunderscore , n. p.)}
\end{itemize}
Árvore leguminosa, cuja madeira avermelhada se emprega em tinturaria.
\section{Campecheiro}
\begin{itemize}
\item {Grp. gram.:m.}
\end{itemize}
O mesmo que \textunderscore campeche\textunderscore .
\section{Campéfago}
\begin{itemize}
\item {Grp. gram.:adj.}
\end{itemize}
\begin{itemize}
\item {Proveniência:(Do gr. \textunderscore kampe\textunderscore  + \textunderscore phagein\textunderscore )}
\end{itemize}
Que se alimenta de lagartos.
\section{Campeira}
\begin{itemize}
\item {Grp. gram.:f.}
\end{itemize}
\begin{itemize}
\item {Utilização:Bras}
\end{itemize}
Espécie de mandioca.
(Cp. \textunderscore campeiro\textunderscore ^1)
\section{Campeiro}
\begin{itemize}
\item {Grp. gram.:adj.}
\end{itemize}
\begin{itemize}
\item {Utilização:Bras}
\end{itemize}
\begin{itemize}
\item {Utilização:Prov.}
\end{itemize}
\begin{itemize}
\item {Grp. gram.:M.}
\end{itemize}
\begin{itemize}
\item {Utilização:Bras}
\end{itemize}
\begin{itemize}
\item {Proveniência:(De \textunderscore campo\textunderscore )}
\end{itemize}
Que trabalha no campo.
Que serve em usos campestres.
Largo: \textunderscore telha campeira\textunderscore .
Aquelle que tem a seu cargo arrebanhar as reses perdidas, ou que tem a seu cargo, no campo, o tratamento de gado.
\section{Campeiro}
\begin{itemize}
\item {Grp. gram.:m.}
\end{itemize}
\begin{itemize}
\item {Utilização:Ant.}
\end{itemize}
\begin{itemize}
\item {Proveniência:(De \textunderscore campa\textunderscore ^2)}
\end{itemize}
Tangedor de campa; chamador de irmandade.
\section{Campeiro}
\begin{itemize}
\item {Grp. gram.:adj.}
\end{itemize}
\begin{itemize}
\item {Utilização:Pop.}
\end{itemize}
Diz-se de uma habitação ampla, desafogada, sem que outras lhe tirem as vistas ou o ar livre: \textunderscore palácio campeiro\textunderscore .
(Cp. \textunderscore campeiro\textunderscore ^1)
\section{Campélia}
\begin{itemize}
\item {Grp. gram.:f.}
\end{itemize}
\begin{itemize}
\item {Proveniência:(Do gr. \textunderscore kampe\textunderscore  + \textunderscore helios\textunderscore )}
\end{itemize}
Planta herbácea, vivaz e lanceolada, das regiões tropicaes.
\section{Campeonato}
\begin{itemize}
\item {Grp. gram.:m.}
\end{itemize}
\begin{itemize}
\item {Proveniência:(De \textunderscore campeão\textunderscore )}
\end{itemize}
Exercício de corridas a cavallo, de velocípede, etc.
\section{Campéphago}
\begin{itemize}
\item {Grp. gram.:adj.}
\end{itemize}
\begin{itemize}
\item {Proveniência:(Do gr. \textunderscore kampe\textunderscore  + \textunderscore phagein\textunderscore )}
\end{itemize}
Que se alimenta de lagartos.
\section{Campestre}
\begin{itemize}
\item {Grp. gram.:adj.}
\end{itemize}
\begin{itemize}
\item {Grp. gram.:M.}
\end{itemize}
\begin{itemize}
\item {Proveniência:(Lat. \textunderscore campester\textunderscore )}
\end{itemize}
Relativo a campo; rústico.
O mesmo que \textunderscore campo\textunderscore :«\textunderscore quando passei o campestre, vi uma rês lá deitada\textunderscore ». Sílv. Romero, \textunderscore Contos Pop.\textunderscore , I, 90.
\section{Campezinho}
\begin{itemize}
\item {Grp. gram.:adj.}
\end{itemize}
O mesmo que \textunderscore campestre\textunderscore .
\section{Campezino}
\begin{itemize}
\item {Grp. gram.:adj.}
\end{itemize}
O mesmo que \textunderscore campestre\textunderscore .
\section{Campião}
\begin{itemize}
\item {Grp. gram.:m.}
\end{itemize}
O mesmo que \textunderscore campeão\textunderscore , mas orthogr. menos usada.
(Cp. b. lat. \textunderscore campiones\textunderscore , do lat. \textunderscore campus\textunderscore )
\section{Campichano}
\begin{itemize}
\item {Grp. gram.:adj.}
\end{itemize}
\begin{itemize}
\item {Utilização:Prov.}
\end{itemize}
\begin{itemize}
\item {Utilização:trasm.}
\end{itemize}
Affável, lhano.
Bem disposto.
Que está á sua vontade em toda a parte.
(Cp. \textunderscore chão\textunderscore )
\section{Campícolas}
\begin{itemize}
\item {Grp. gram.:m. pl.}
\end{itemize}
\begin{itemize}
\item {Proveniência:(Do lat. \textunderscore campus\textunderscore  + \textunderscore colere\textunderscore )}
\end{itemize}
Gênero de aves.
\section{Campido}
\begin{itemize}
\item {Grp. gram.:m.}
\end{itemize}
\begin{itemize}
\item {Utilização:Pint.}
\end{itemize}
\begin{itemize}
\item {Proveniência:(De \textunderscore campir\textunderscore )}
\end{itemize}
Confusa distribuição dos elementos que entram num quadro,(longes, céus, horizontes).
\section{Campilograma}
\begin{itemize}
\item {Grp. gram.:m.}
\end{itemize}
\begin{itemize}
\item {Proveniência:(Do gr. \textunderscore kampulos\textunderscore  + \textunderscore gramma\textunderscore )}
\end{itemize}
Instrumento, para facilitar a construcção das linhas curvas no traçado de planos dos navios.
\section{Campilotropia}
\begin{itemize}
\item {Grp. gram.:f.}
\end{itemize}
Qualidade ou estado de campilótropo.
\section{Campilótropo}
\begin{itemize}
\item {Grp. gram.:adj.}
\end{itemize}
\begin{itemize}
\item {Utilização:Bot.}
\end{itemize}
\begin{itemize}
\item {Proveniência:(Do gr. \textunderscore kampulos\textunderscore , curvo, e \textunderscore trepein\textunderscore , voltar)}
\end{itemize}
Diz-se do óvulo vegetal que, sendo anátropo, tem a fórma de um rim, sem nada que indique a direcção da raphe.
\section{Campímetro}
\begin{itemize}
\item {Grp. gram.:m.}
\end{itemize}
\begin{itemize}
\item {Proveniência:(Do lat. \textunderscore campus\textunderscore  + gr. \textunderscore metron\textunderscore )}
\end{itemize}
Instrumento, para medir o campo visual.
\section{Campina}
\begin{itemize}
\item {Grp. gram.:f.}
\end{itemize}
\begin{itemize}
\item {Proveniência:(De \textunderscore campo\textunderscore )}
\end{itemize}
Campo extenso, sem árvores.
Planície.
\section{Campineiro}
\begin{itemize}
\item {Grp. gram.:adj.}
\end{itemize}
\begin{itemize}
\item {Grp. gram.:M.}
\end{itemize}
Relativo á cidade de Campinas, no Brasil.
Habitante de Campinas.
\section{Campino}
\begin{itemize}
\item {Grp. gram.:m.}
\end{itemize}
\begin{itemize}
\item {Grp. gram.:Adj.}
\end{itemize}
\begin{itemize}
\item {Utilização:Des.}
\end{itemize}
\begin{itemize}
\item {Proveniência:(De \textunderscore campo\textunderscore )}
\end{itemize}
Camponês.
Guardador de toiros.
Pastor.
Relativo ao campo: \textunderscore a verdura campina\textunderscore .
\section{Campir}
\begin{itemize}
\item {Grp. gram.:v.}
\end{itemize}
\begin{itemize}
\item {Utilização:t. Pint.}
\end{itemize}
\begin{itemize}
\item {Proveniência:(It. \textunderscore campire\textunderscore )}
\end{itemize}
Fazer a perspectiva do horizonte em (um quadro).
\section{Campista}
\begin{itemize}
\item {Grp. gram.:m.}
\end{itemize}
Indivíduo natural de Campos, no Brasil.
\section{Campista}
\begin{itemize}
\item {Grp. gram.:m.}
\end{itemize}
\begin{itemize}
\item {Utilização:Bras}
\end{itemize}
Aquelle que campeia a cavallo, procurando o gado ou tratando delle.
\section{Campo}
\begin{itemize}
\item {Grp. gram.:m.}
\end{itemize}
\begin{itemize}
\item {Utilização:Pint.}
\end{itemize}
\begin{itemize}
\item {Utilização:Heráld.}
\end{itemize}
\begin{itemize}
\item {Utilização:Ant.}
\end{itemize}
\begin{itemize}
\item {Utilização:Prov.}
\end{itemize}
\begin{itemize}
\item {Proveniência:(Lat. \textunderscore campus\textunderscore )}
\end{itemize}
Terreno extenso e plano; planície.
Terreno, fóra dos povoados.
Terreiro, dentro de povoação.
Acampamento.
Liça.
Lugar de combate.
O fundo de um quadro.
Aso, ensejo.
Espaço do escudo, em que se pintam ou lavram as peças.
Fundo liso de qualquer estôfo, de que resaem os matizes.
Exército que acampou:«\textunderscore o campo se abalou daqui\textunderscore ». \textunderscore Peregrinação\textunderscore , CXX.
Espaço, lugar.
\textunderscore Campo santo\textunderscore , o mesmo que \textunderscore cemitério\textunderscore .
\section{Campomanésia}
\begin{itemize}
\item {Grp. gram.:f.}
\end{itemize}
\begin{itemize}
\item {Proveniência:(De \textunderscore Campomanes\textunderscore , n. p.)}
\end{itemize}
Planta myrtácea da América tropical.
\section{Camponês}
\begin{itemize}
\item {Grp. gram.:m.}
\end{itemize}
\begin{itemize}
\item {Grp. gram.:Adj.}
\end{itemize}
\begin{itemize}
\item {Proveniência:(De \textunderscore campo\textunderscore )}
\end{itemize}
Aquelle que habita ou trabalha no campo.
Próprio do campo; rústico.
\section{Camponesas}
\begin{itemize}
\item {Grp. gram.:f. pl.}
\end{itemize}
\begin{itemize}
\item {Utilização:Prov.}
\end{itemize}
\begin{itemize}
\item {Utilização:alent.}
\end{itemize}
Dança de roda.
\section{Campónio}
\begin{itemize}
\item {Grp. gram.:m.  e  adj.}
\end{itemize}
(Fórma depreciativa, por \textunderscore camponês\textunderscore )
\section{Camporda-paz}
\begin{itemize}
\item {Grp. gram.:f.}
\end{itemize}
Videira do Brasil.
\section{Campóscia}
\begin{itemize}
\item {Grp. gram.:f.}
\end{itemize}
Gênero de crustáceos decápodes.
\section{Campsim}
\begin{itemize}
\item {Grp. gram.:m.}
\end{itemize}
Vento, mais abrasador que o simum.
\section{Campsiúro}
\begin{itemize}
\item {Grp. gram.:m.}
\end{itemize}
Gênero de insectos coleópteros.
\section{Campodactilia}
\begin{itemize}
\item {Grp. gram.:f.}
\end{itemize}
Estado de \textunderscore camptodáctilo\textunderscore .
\section{Campodáctilo}
\begin{itemize}
\item {Grp. gram.:adj.}
\end{itemize}
\begin{itemize}
\item {Proveniência:(Do gr. \textunderscore kamptein\textunderscore  + \textunderscore daktulos\textunderscore )}
\end{itemize}
Que tem um ou mais dedos encurvados, por má conformação da mão.
\section{Camptodactylia}
\begin{itemize}
\item {Grp. gram.:f.}
\end{itemize}
Estado de \textunderscore camptodáctylo\textunderscore .
\section{Camptodáctylo}
\begin{itemize}
\item {Grp. gram.:adj.}
\end{itemize}
\begin{itemize}
\item {Proveniência:(Do gr. \textunderscore kamptein\textunderscore  + \textunderscore daktulos\textunderscore )}
\end{itemize}
Que tem um ou mais dedos encurvados, por má conformação da mão.
\section{Camptologia}
\begin{itemize}
\item {Grp. gram.:f.}
\end{itemize}
\begin{itemize}
\item {Utilização:Gram.}
\end{itemize}
\begin{itemize}
\item {Proveniência:(Do gr. \textunderscore kamptein\textunderscore  + \textunderscore logos\textunderscore )}
\end{itemize}
Parte da morphologia, que trata das variações de fórma, que as palavras flexivas podem apresentar no discurso, segundo a diversidade das relações que exprimem e das modificações que experimentam na sua significação.
\section{Camptológico}
\begin{itemize}
\item {Grp. gram.:adj.}
\end{itemize}
Relativo á camptologia.
\section{Campuda}
\begin{itemize}
\item {Grp. gram.:f.}
\end{itemize}
Variedade de maçan.
\section{Campylogramma}
\begin{itemize}
\item {Grp. gram.:m.}
\end{itemize}
\begin{itemize}
\item {Proveniência:(Do gr. \textunderscore kampulos\textunderscore  + \textunderscore gramma\textunderscore )}
\end{itemize}
Instrumento, para facilitar a construcção das linhas curvas no traçado de planos dos navios.
\section{Campylotropia}
\begin{itemize}
\item {Grp. gram.:f.}
\end{itemize}
Qualidade ou estado de campylótropo.
\section{Campylótropo}
\begin{itemize}
\item {Grp. gram.:adj.}
\end{itemize}
\begin{itemize}
\item {Utilização:Bot.}
\end{itemize}
\begin{itemize}
\item {Proveniência:(Do gr. \textunderscore kampulos\textunderscore , curvo, e \textunderscore trepein\textunderscore , voltar)}
\end{itemize}
Diz-se do óvulo vegetal que, sendo anátropo, tem a fórma de um rim, sem nada que indique a direcção da raphe.
\section{Camuacá}
\begin{itemize}
\item {Grp. gram.:m.}
\end{itemize}
\begin{itemize}
\item {Utilização:Bras}
\end{itemize}
Espécie de cipó medicinal.
\section{Camuça}
\begin{itemize}
\item {Grp. gram.:f.}
\end{itemize}
\begin{itemize}
\item {Utilização:Des.}
\end{itemize}
O mesmo que \textunderscore camurça\textunderscore .
\section{Camuci}
\begin{itemize}
\item {Grp. gram.:m.}
\end{itemize}
O mesmo que \textunderscore camucim\textunderscore .
\section{Camucim}
\begin{itemize}
\item {Grp. gram.:m.}
\end{itemize}
\begin{itemize}
\item {Utilização:Bras}
\end{itemize}
Boião de barro preto.
Talha de barro, em que os Índios sepultam os cadáveres da sua gente.
(Do tupi \textunderscore camuci\textunderscore , sorver)
\section{Camumbembe}
\begin{itemize}
\item {Grp. gram.:m.}
\end{itemize}
\begin{itemize}
\item {Utilização:Bras}
\end{itemize}
Vadio.
Homem de baixa condição.
\section{Camundongo}
\begin{itemize}
\item {Grp. gram.:m.}
\end{itemize}
\begin{itemize}
\item {Utilização:Bras}
\end{itemize}
Espécie de rato pequeno.
(Do quimb.)
\section{Camúnia}
\begin{itemize}
\item {Grp. gram.:f.}
\end{itemize}
\begin{itemize}
\item {Utilização:Prov.}
\end{itemize}
\begin{itemize}
\item {Utilização:trasm.}
\end{itemize}
Corja, súcia.
Agrupamento de rapazes e raparigas.
(Provavelmente, corr. de \textunderscore communa\textunderscore )
\section{Camurapim}
\begin{itemize}
\item {Grp. gram.:m.}
\end{itemize}
Peixe do Brasil.
\section{Camurça}
\begin{itemize}
\item {Grp. gram.:f.}
\end{itemize}
\begin{itemize}
\item {Proveniência:(Do ant. alt. al. \textunderscore gamuz\textunderscore )}
\end{itemize}
Espécie de cabra montês.
A pelle dêsse animal, usada no fabrico de calçado, luvas, etc.
Variedade de pêra, a mesma que \textunderscore providência\textunderscore .
\section{Camurçado}
\begin{itemize}
\item {Grp. gram.:adj.}
\end{itemize}
O mesmo que \textunderscore acamurçado\textunderscore .
\section{Camurim}
\begin{itemize}
\item {Grp. gram.:m.}
\end{itemize}
\begin{itemize}
\item {Utilização:Bras. do N}
\end{itemize}
\begin{itemize}
\item {Utilização:Bras}
\end{itemize}
O mesmo que \textunderscore robalo\textunderscore .
Pequena bóia de cortiça com linhas e anzol, para a pesca de tartarugas.
\section{Camurro}
\begin{itemize}
\item {Grp. gram.:adj.}
\end{itemize}
\begin{itemize}
\item {Utilização:Prov.}
\end{itemize}
\begin{itemize}
\item {Utilização:trasm.}
\end{itemize}
O mesmo que \textunderscore casmurro\textunderscore .
\section{Camuz}
\begin{itemize}
\item {Grp. gram.:adj.}
\end{itemize}
(?):«\textunderscore não soys camuz de entender damas.\textunderscore »\textunderscore Aulegrafia\textunderscore , 115.
\section{Can}
\begin{itemize}
\item {Grp. gram.:f.}
\end{itemize}
\begin{itemize}
\item {Proveniência:(Do lat. \textunderscore canus\textunderscore )}
\end{itemize}
Cabello branco.
(Des. no sing.)
\section{Can}
\begin{itemize}
\item {Grp. gram.:m.}
\end{itemize}
Chefe supremo de alguns paises asiáticos.
Governador de província, na Pérsia.
(Do \textunderscore mongol\textunderscore )
\section{Cana}
\begin{itemize}
\item {Grp. gram.:f.}
\end{itemize}
\begin{itemize}
\item {Proveniência:(Lat. \textunderscore canna\textunderscore )}
\end{itemize}
Planta gramínea, de haste ôca nos entrenós.
Caule de várias plantas gramíneas.
A parte superior e lisa do caule do milho, desde o último nó á bandeira.
Osso, mais ou menos alongado, de certas partes do corpo humano: \textunderscore cana da perna\textunderscore ; \textunderscore cana do nariz.\textunderscore 
Designação de vários objectos alongados e cylíndricos, que dão ideia de uma cana.
Alavanca de pau, com que se governa o leme.
\section{Canabrás}
\begin{itemize}
\item {Grp. gram.:f.}
\end{itemize}
Planta umbellífera, medicinal.
\section{Canaca}
\begin{itemize}
\item {Grp. gram.:m.}
\end{itemize}
\begin{itemize}
\item {Grp. gram.:Pl.}
\end{itemize}
Artífice, na Índia portuguesa.
Indígenas de Sandwich.
\section{Cana-caiana}
\begin{itemize}
\item {Grp. gram.:f.}
\end{itemize}
\begin{itemize}
\item {Utilização:Bras}
\end{itemize}
Uma das espécies de cana do açúcar.
\section{Canáceas}
\begin{itemize}
\item {Grp. gram.:f. pl.}
\end{itemize}
Família de plantas, a que serve de typo a cana da Índia.
\section{Cana-cheirosa}
\begin{itemize}
\item {Grp. gram.:f.}
\end{itemize}
\begin{itemize}
\item {Utilização:Bras}
\end{itemize}
O mesmo que \textunderscore cálamo\textunderscore .
\section{Canada}
\begin{itemize}
\item {Grp. gram.:f.}
\end{itemize}
\begin{itemize}
\item {Utilização:Prov.}
\end{itemize}
\begin{itemize}
\item {Utilização:alent.}
\end{itemize}
\begin{itemize}
\item {Utilização:Prov.}
\end{itemize}
\begin{itemize}
\item {Utilização:alent.}
\end{itemize}
\begin{itemize}
\item {Proveniência:(De \textunderscore cana\textunderscore )}
\end{itemize}
Pancada com cana.
Antiga medida portuguesa, igual a 4 quartilhos.
Azinhaga.
Atalho.
Fila de estacas, através de um rio, para indicar o vau.
Sulco, formado pelo rodar dos vehículos.
Faixa de terreno, que se deixa inculta dentro de uma propriedade ou entre duas propriedades, para passagem de gado.
Depressão de terreno ondulado, olga, valleiro.
\section{Canadá}
\begin{itemize}
\item {Grp. gram.:f.}
\end{itemize}
\begin{itemize}
\item {Proveniência:(De \textunderscore Canadá\textunderscore , n. p.)}
\end{itemize}
Variedade de videira americana, de pequenas fôlhas e cachos pretos.
\section{Cana-da-índia}
\begin{itemize}
\item {Grp. gram.:f.}
\end{itemize}
Planta canácea (\textunderscore canna-indica\textunderscore ), de que se fazem bengalas, muito conhecidas.
\section{Canadela}
\begin{itemize}
\item {Grp. gram.:f.}
\end{itemize}
\begin{itemize}
\item {Proveniência:(De \textunderscore canada\textunderscore )}
\end{itemize}
Antiga medida portuguesa para sólidos.
\section{Cana-de-macaco}
\begin{itemize}
\item {Grp. gram.:f.}
\end{itemize}
Espécie de plantas amómeas.
(\textunderscore costus Pisonis\textunderscore )
\section{Canadense}
\begin{itemize}
\item {Grp. gram.:m.  e  adj.}
\end{itemize}
O mesmo que \textunderscore canadiano\textunderscore .
\section{Canadiano}
\begin{itemize}
\item {Grp. gram.:adj.}
\end{itemize}
\begin{itemize}
\item {Grp. gram.:M.}
\end{itemize}
Relativo ao Canadá.
Habitante do Canadá.
\section{Canadiense}
\begin{itemize}
\item {Grp. gram.:m.  e  adj.}
\end{itemize}
(V.canadiano)
\section{Canadilho}
\begin{itemize}
\item {Grp. gram.:m.}
\end{itemize}
\begin{itemize}
\item {Utilização:T. de Pinhel}
\end{itemize}
\begin{itemize}
\item {Proveniência:(De \textunderscore canado\textunderscore ^2)}
\end{itemize}
Pequeno cesto para a vindima.
\section{Canado}
\begin{itemize}
\item {Grp. gram.:m.}
\end{itemize}
\begin{itemize}
\item {Proveniência:(De \textunderscore can\textunderscore ^2)}
\end{itemize}
A dignidade de can.
País, governado por um can.
\section{Canado}
\begin{itemize}
\item {Grp. gram.:m.}
\end{itemize}
\begin{itemize}
\item {Utilização:Prov.}
\end{itemize}
\begin{itemize}
\item {Utilização:dur.}
\end{itemize}
\begin{itemize}
\item {Utilização:Prov.}
\end{itemize}
\begin{itemize}
\item {Utilização:trasm.}
\end{itemize}
\begin{itemize}
\item {Utilização:Prov.}
\end{itemize}
\begin{itemize}
\item {Utilização:minh.}
\end{itemize}
Vasilha de lata ou de cobre, em que se media o vinho no Doiro e correspondia proximamente ao cântaro.
O mesmo que \textunderscore canada\textunderscore , medida.
Medida de cinco almudes, feita de madeira ou de lata.
(Cp. \textunderscore canada\textunderscore )
\section{Canafístula}
\begin{itemize}
\item {Grp. gram.:f.}
\end{itemize}
\begin{itemize}
\item {Proveniência:(De \textunderscore cana\textunderscore  + \textunderscore fístula\textunderscore )}
\end{itemize}
Planta leguminosa, medicinal.
\section{Canajeira}
\begin{itemize}
\item {Grp. gram.:f.}
\end{itemize}
\begin{itemize}
\item {Utilização:Marn.}
\end{itemize}
Espécie de pá.
\section{Canafrecha}
\begin{itemize}
\item {Grp. gram.:f.}
\end{itemize}
Planta umbellífera.
\section{Canal}
\begin{itemize}
\item {Grp. gram.:m.}
\end{itemize}
\begin{itemize}
\item {Utilização:Fig.}
\end{itemize}
\begin{itemize}
\item {Proveniência:(Lat. \textunderscore canalis\textunderscore )}
\end{itemize}
Escavação, fôsso, que leva águas.
Estreito, porção de água que, por entre costas, liga dois mares: \textunderscore o canal de Suez\textunderscore .
Leito de rio.
Braço de rio ou de mar, por onde se desviam águas para usos agrícolas ou industriáes.
Cavidade, que dá passagem a gases ou liquidos nos corpos organizados.
Aquillo ou aquelle que serve de intermédio para um fim determinado.
\section{Canalete}
\begin{itemize}
\item {fónica:lê}
\end{itemize}
\begin{itemize}
\item {Grp. gram.:m.}
\end{itemize}
Canal pequeno.
\section{Canalha}
\begin{itemize}
\item {Grp. gram.:f.}
\end{itemize}
\begin{itemize}
\item {Utilização:Prov.}
\end{itemize}
\begin{itemize}
\item {Grp. gram.:M.  e  adj.}
\end{itemize}
\begin{itemize}
\item {Proveniência:(It. \textunderscore canaglie\textunderscore )}
\end{itemize}
Gente vil.
Crianças pequenas.
Pessôa vil e infame.
\section{Canalhada}
\begin{itemize}
\item {Grp. gram.:f.}
\end{itemize}
Acção do canalha. Cf. Júl. Dinis, \textunderscore Morgadinha\textunderscore , 38.
\section{Canalhice}
\begin{itemize}
\item {Grp. gram.:f.}
\end{itemize}
Acção própria de canalha.
\section{Canalhismo}
\begin{itemize}
\item {Grp. gram.:m.}
\end{itemize}
Procedimento ou palavras do canalha. Cf. Arn. Gama, \textunderscore Última Dona\textunderscore , 84.
\section{Canalhocracia}
\begin{itemize}
\item {Grp. gram.:f.}
\end{itemize}
Designação sarcástica das pretensões da canalha.
Os defensores dos direitos sociaes da canalha.
\section{Canalícula}
\begin{itemize}
\item {Grp. gram.:f.}
\end{itemize}
(V.canalículo)
\section{Canaliculado}
\begin{itemize}
\item {Grp. gram.:adj.}
\end{itemize}
Que tem canaliculo.
\section{Canalículo}
\begin{itemize}
\item {Grp. gram.:m.}
\end{itemize}
\begin{itemize}
\item {Proveniência:(Lat. \textunderscore canaliculus\textunderscore )}
\end{itemize}
Pequeno canal.
Pequeno rêgo nas hastes, pecíolos, ou fôlhas de alguns vegetaes.
\section{Canaliforme}
\begin{itemize}
\item {Grp. gram.:adj.}
\end{itemize}
\begin{itemize}
\item {Proveniência:(Do lat. \textunderscore canalis\textunderscore  + \textunderscore forma\textunderscore )}
\end{itemize}
Que tem fórma de canal ou de calha.
\section{Canalização}
\begin{itemize}
\item {Grp. gram.:f.}
\end{itemize}
Acto de \textunderscore canalizar\textunderscore .
Conjunto de canos ou canaes.
\section{Canalizador}
\begin{itemize}
\item {Grp. gram.:m.}
\end{itemize}
Aquelle que trabalha em canalizações de gás ou de água.
\section{Canalizar}
\begin{itemize}
\item {Grp. gram.:v. t.}
\end{itemize}
\begin{itemize}
\item {Proveniência:(De \textunderscore canal\textunderscore )}
\end{itemize}
Collocar, abrir, canos em.
Cortar com canaes.
Dirigir por canos, canaes ou vallas: \textunderscore canalizar água\textunderscore .
\section{Canalizável}
\begin{itemize}
\item {Grp. gram.:adj.}
\end{itemize}
Que póde sêr canalizado.
\section{Canamão}
\begin{itemize}
\item {Grp. gram.:m.}
\end{itemize}
\begin{itemize}
\item {Utilização:Prov.}
\end{itemize}
\begin{itemize}
\item {Utilização:trasm.}
\end{itemize}
\begin{itemize}
\item {Proveniência:(De \textunderscore cana\textunderscore  + \textunderscore mão\textunderscore .)}
\end{itemize}
Pau, a que se apoiam os que andam a trilhar cereaes na eira.
\section{Canambaia}
\begin{itemize}
\item {Grp. gram.:f.}
\end{itemize}
Espécie de cacto.
\section{Canamilha}
\begin{itemize}
\item {Grp. gram.:f.}
\end{itemize}
Planta gramínea, (\textunderscore arundo donax\textunderscore , Lin.), também conhecida por \textunderscore caninha\textunderscore .
\section{Cânamo}
\begin{itemize}
\item {Grp. gram.:m.}
\end{itemize}
O mesmo que \textunderscore cânhamo\textunderscore .
\section{Canamões}
\begin{itemize}
\item {Grp. gram.:m. pl.}
\end{itemize}
\begin{itemize}
\item {Utilização:Prov.}
\end{itemize}
\begin{itemize}
\item {Utilização:trasm.}
\end{itemize}
\begin{itemize}
\item {Proveniência:(De \textunderscore canamão\textunderscore )}
\end{itemize}
Pulsos grossos, fortes.
\section{Canana}
\begin{itemize}
\item {Grp. gram.:f.}
\end{itemize}
Cartucheira de coiro, que os militares usam a tiracollo.
\section{Canango}
\begin{itemize}
\item {Grp. gram.:m.}
\end{itemize}
Árvore aromática da Ásia e da América.
\section{Canangundo}
\begin{itemize}
\item {Grp. gram.:m.}
\end{itemize}
Pássaro conirostro da África occidental.
\section{Canapé}
\begin{itemize}
\item {Grp. gram.:m.}
\end{itemize}
\begin{itemize}
\item {Proveniência:(Lat. \textunderscore conopeum\textunderscore , por intermédio do fr.)}
\end{itemize}
Assento comprido, com braços e costas, dando lugar a duas ou mais pessôas.
\section{Canapu}
\begin{itemize}
\item {Grp. gram.:m.}
\end{itemize}
\begin{itemize}
\item {Utilização:Bras}
\end{itemize}
Planta herbácea, da fam. das solâneas.
\section{Canará}
\begin{itemize}
\item {Grp. gram.:m.}
\end{itemize}
(V.canarim)
\section{Canarana}
\begin{itemize}
\item {Grp. gram.:f.}
\end{itemize}
\begin{itemize}
\item {Utilização:Bras. do N}
\end{itemize}
\begin{itemize}
\item {Proveniência:(De \textunderscore cana\textunderscore  + guar. \textunderscore rana\textunderscore )}
\end{itemize}
Cana brava, planta gramínea.
\section{Canareira}
\begin{itemize}
\item {Grp. gram.:f.}
\end{itemize}
Gaiola grande, em que se criam canários.
\section{Canaría}
\begin{itemize}
\item {Grp. gram.:f.}
\end{itemize}
\begin{itemize}
\item {Proveniência:(De \textunderscore cano\textunderscore )}
\end{itemize}
Conjunto dos canudos ou tubos, de que se compõe o órgão.
\section{Canária}
\begin{itemize}
\item {Grp. gram.:f.}
\end{itemize}
\begin{itemize}
\item {Proveniência:(De \textunderscore Canárias\textunderscore , n. p.)}
\end{itemize}
Dança antiga, em que o cavalheiro e a dama, separando-se, dançavam alternadamente um adeante do outro, com movimentos extravagantes, em que procuravam imitar, segundo se dizia, os selvagens das ilhas Canárias.
\section{Canária}
\begin{itemize}
\item {Grp. gram.:f.}
\end{itemize}
\begin{itemize}
\item {Utilização:Mad}
\end{itemize}
Fêmea do canário.
O mesmo que \textunderscore canário\textunderscore ^1.
\section{Canarim}
\begin{itemize}
\item {Grp. gram.:m.}
\end{itemize}
\begin{itemize}
\item {Utilização:Pop.}
\end{itemize}
\begin{itemize}
\item {Grp. gram.:Adj.}
\end{itemize}
\begin{itemize}
\item {Utilização:Bras}
\end{itemize}
\begin{itemize}
\item {Proveniência:(De \textunderscore Canará\textunderscore , n. p.)}
\end{itemize}
Língua do grupo decânico.
Homem natural da Índia portuguesa.
Diz-se do homem de côr amarela ou escura ou trigueira.
\section{Canarina}
\begin{itemize}
\item {Grp. gram.:f.}
\end{itemize}
\begin{itemize}
\item {Utilização:Bot.}
\end{itemize}
Trepadeira vivaz do Brasil.
\section{Canarinho}
\begin{itemize}
\item {Grp. gram.:m.}
\end{itemize}
\begin{itemize}
\item {Utilização:Mad}
\end{itemize}
Espécie de canário.
\section{Canarino}
\begin{itemize}
\item {Grp. gram.:adj.}
\end{itemize}
Relativo aos canarás; canarim.
\section{Canário}
\begin{itemize}
\item {Grp. gram.:m.}
\end{itemize}
\begin{itemize}
\item {Grp. gram.:M.  e  adj.}
\end{itemize}
\begin{itemize}
\item {Proveniência:(De \textunderscore Canárias\textunderscore , n. p.)}
\end{itemize}
Pequeno pássaro conirostro, de plumagem geralmente amarela e canto melodioso.
Dizia-se do indivíduo natural das Canárias.
\section{Canário}
\begin{itemize}
\item {Grp. gram.:adj.}
\end{itemize}
\begin{itemize}
\item {Grp. gram.:M.}
\end{itemize}
Diz-se de uma espécie de feijão minhoto.
Peixe de Portugal.
(Talvez da côr do \textunderscore canário\textunderscore , ave)
\section{Canário}
\begin{itemize}
\item {Grp. gram.:m.}
\end{itemize}
Dança antiga, o mesmo que \textunderscore canária\textunderscore .
\section{Canário-da-terra}
\begin{itemize}
\item {Grp. gram.:m.}
\end{itemize}
\begin{itemize}
\item {Utilização:Mad}
\end{itemize}
Espécie de canário.
\section{Canário-de-frança}
\begin{itemize}
\item {Grp. gram.:m.}
\end{itemize}
Nome, que na Foz-do-Doiro se dá ao lugre ou ao pintassilgo verde.
\section{Canas}
\begin{itemize}
\item {Grp. gram.:f. pl.}
\end{itemize}
Jôgo antigo, em que se parodiavam os torneios, servindo-se de canas os jogadores.
(Pl. de \textunderscore cana\textunderscore )
\section{Canas}
\begin{itemize}
\item {Grp. gram.:f. pl.}
\end{itemize}
\textunderscore Fazenda de canas\textunderscore , tecido antigo, talvez de padrão canelado.--Um \textunderscore Ms\textunderscore . de 1562, que está na Tôrre-do-Tombo, arm. 26, allude a essa fazenda, indicando que era prohibido exportá-la, sob pena de morte.--
\section{Canastra}
\begin{itemize}
\item {Grp. gram.:f.}
\end{itemize}
\begin{itemize}
\item {Utilização:Bras. do N}
\end{itemize}
\begin{itemize}
\item {Proveniência:(Do gr. \textunderscore kanastra\textunderscore . Cp. lat. \textunderscore canistra\textunderscore )}
\end{itemize}
Cesta larga e pouco alta, entretecida de vêrga ou corras de castinçal.
Espécie de tatu.
\section{Canastrada}
\begin{itemize}
\item {Grp. gram.:f.}
\end{itemize}
Objectos, que enchem uma canastra.
Quantidade de canastras.
\section{Canastrado}
\begin{itemize}
\item {Grp. gram.:adj.}
\end{itemize}
\begin{itemize}
\item {Proveniência:(De \textunderscore canastra\textunderscore )}
\end{itemize}
Diz-se de certos panos estampados, cujos desenhos imitam a tecedura da canastra.
\section{Canastrão}
\begin{itemize}
\item {Grp. gram.:m.}
\end{itemize}
Canastra grande.
\section{Canastreiro}
\begin{itemize}
\item {Grp. gram.:m.}
\end{itemize}
\begin{itemize}
\item {Utilização:Prov.}
\end{itemize}
Aquelle que faz canastras.
Aquelle que as vende.
Canastra, para transporte de peixe.
\section{Canastrel}
\begin{itemize}
\item {Grp. gram.:m.}
\end{itemize}
Espécie de pequena canastra com asa.
(B. lat. \textunderscore canistrellum\textunderscore )
\section{Canastrinha}
\begin{itemize}
\item {Grp. gram.:f.}
\end{itemize}
Espécie de jôgo popular.
\section{Canastro}
\begin{itemize}
\item {Grp. gram.:m.}
\end{itemize}
\begin{itemize}
\item {Utilização:Pop.}
\end{itemize}
\begin{itemize}
\item {Utilização:Prov.}
\end{itemize}
\begin{itemize}
\item {Utilização:dur.}
\end{itemize}
\begin{itemize}
\item {Utilização:minh.}
\end{itemize}
Espécie de canastra, de bordos altos.
O tronco humano; corpo: \textunderscore dou-te cabo do canastro\textunderscore .
O mesmo \textunderscore espigueiro\textunderscore .
(Cp. \textunderscore canastra\textunderscore )
\section{Canavã}
\begin{itemize}
\item {Grp. gram.:f.}
\end{itemize}
\begin{itemize}
\item {Proveniência:(De \textunderscore cana\textunderscore  + \textunderscore van\textunderscore , fem. de \textunderscore vão\textunderscore )}
\end{itemize}
Planta aquática.
\section{Canaval}
\begin{itemize}
\item {Grp. gram.:m.}
\end{itemize}
O mesmo que \textunderscore canavial\textunderscore :«\textunderscore canaval está florido, canas do amor\textunderscore ». Gil Vicente, \textunderscore Inês Pereira\textunderscore .
\section{Canavan}
\begin{itemize}
\item {Grp. gram.:f.}
\end{itemize}
\begin{itemize}
\item {Proveniência:(De \textunderscore cana\textunderscore  + \textunderscore van\textunderscore , fem. de \textunderscore vão\textunderscore )}
\end{itemize}
Planta aquática.
\section{Cânave}
\begin{itemize}
\item {Grp. gram.:m.}
\end{itemize}
\begin{itemize}
\item {Proveniência:(Lat. \textunderscore cannabis\textunderscore )}
\end{itemize}
O mesmo que \textunderscore cânhamo\textunderscore .
\section{Canaveal}
\begin{itemize}
\item {Grp. gram.:m.}
\end{itemize}
O mesmo ou melhor que \textunderscore canavial\textunderscore .
\section{Canaveira}
\begin{itemize}
\item {Grp. gram.:f.}
\end{itemize}
\begin{itemize}
\item {Utilização:Ant.}
\end{itemize}
\begin{itemize}
\item {Proveniência:(De \textunderscore cânave\textunderscore )}
\end{itemize}
Lugar, onde cresce o cânhamo. Cf. Sousa, \textunderscore Ann. de D. João III\textunderscore .
\section{Cana-verde}
\begin{itemize}
\item {Grp. gram.:f.}
\end{itemize}
Canção popular do Minho.
\section{Canavez}
\begin{itemize}
\item {Grp. gram.:m.}
\end{itemize}
\begin{itemize}
\item {Proveniência:(De \textunderscore cânave\textunderscore )}
\end{itemize}
Plantação do cânhamo.
\section{Canavial}
\begin{itemize}
\item {Grp. gram.:m.}
\end{itemize}
\begin{itemize}
\item {Proveniência:(De \textunderscore cânave\textunderscore )}
\end{itemize}
Lugar, onde crescem canas.
\section{Cana-vieira}
\begin{itemize}
\item {Grp. gram.:f.}
\end{itemize}
Planta gramínea, (\textunderscore sorgum saccharatum\textunderscore , Pers.).
\section{Canavoira}
\begin{itemize}
\item {Grp. gram.:f.}
\end{itemize}
\begin{itemize}
\item {Utilização:Prov.}
\end{itemize}
\begin{itemize}
\item {Utilização:alg.}
\end{itemize}
Planta, de fôlhas semelhantes ás da espadana.
Caule sêco da faveira.
\section{Canavoura}
\begin{itemize}
\item {Grp. gram.:f.}
\end{itemize}
Planta, de fôlhas semelhantes ás da espadana.
\section{Canaz}
\begin{itemize}
\item {Grp. gram.:m.}
\end{itemize}
\begin{itemize}
\item {Utilização:Des.}
\end{itemize}
\begin{itemize}
\item {Proveniência:(Do lat. \textunderscore canis\textunderscore )}
\end{itemize}
Canzarrão.
Homem vil.
\section{Cancaborrada}
\begin{itemize}
\item {Grp. gram.:f.}
\end{itemize}
(V.cacaborrada)
\section{Cancan}
\begin{itemize}
\item {Grp. gram.:m.}
\end{itemize}
\begin{itemize}
\item {Proveniência:(Fr. \textunderscore cancan\textunderscore )}
\end{itemize}
Dança affectada, impudica.
\section{Cancan}
\begin{itemize}
\item {Grp. gram.:m.}
\end{itemize}
\begin{itemize}
\item {Utilização:Bras}
\end{itemize}
Pássaro, espécie de folião, que come frutos e lárvas. Cf. G. Dias, \textunderscore Diccion. da Língua Tup\textunderscore .
\section{Cancanan}
\begin{itemize}
\item {Grp. gram.:m.}
\end{itemize}
\begin{itemize}
\item {Utilização:T. da Índia port}
\end{itemize}
Manilha, com que as bailadeiras enfeitam os pulsos.
\section{Cancanar}
\begin{itemize}
\item {Grp. gram.:v. i.}
\end{itemize}
\begin{itemize}
\item {Utilização:Neol.}
\end{itemize}
Dançar o cancan.
Exaggerar meneios em qualquer dança.
\section{Cancanista}
\begin{itemize}
\item {Grp. gram.:adj.}
\end{itemize}
\begin{itemize}
\item {Grp. gram.:M.  e  f.}
\end{itemize}
Relativo ou semelhante ao cancan.
Pessôa, que dança o cancan. Cf. Camillo, \textunderscore Volcões\textunderscore , 164.
\section{Cancanizar}
\begin{itemize}
\item {Grp. gram.:v. t.  e  i.}
\end{itemize}
Cancanar.
Dar feição de cancan a. Cf. Camillo, \textunderscore Corja\textunderscore , 201.
\section{Cancão}
\begin{itemize}
\item {Grp. gram.:m.}
\end{itemize}
Ave do Brasil; o mesmo que \textunderscore cancan\textunderscore ^2?
\section{Canção}
\begin{itemize}
\item {Grp. gram.:f.}
\end{itemize}
\begin{itemize}
\item {Proveniência:(Lat. \textunderscore cantio\textunderscore )}
\end{itemize}
Canto.
Composição poética, destinada geralmente a sêr cantada.
\section{Cancar}
\begin{itemize}
\item {Grp. gram.:m.}
\end{itemize}
Árvore de Damão.
\section{Cancatá}
\begin{itemize}
\item {Grp. gram.:m.}
\end{itemize}
\begin{itemize}
\item {Utilização:Bras}
\end{itemize}
Ave das regiões do Amazonas.
\section{Cancela}
\begin{itemize}
\item {Grp. gram.:f.}
\end{itemize}
Porta gradeada de madeira, mais tôsca que o cancêlo.
(Cp. \textunderscore cancêllo\textunderscore )
\section{Cancelação}
\begin{itemize}
\item {Grp. gram.:f.}
\end{itemize}
\begin{itemize}
\item {Proveniência:(Lat. \textunderscore cancellatio\textunderscore )}
\end{itemize}
O mesmo que \textunderscore cancelamento\textunderscore .
\section{Cancelada}
\begin{itemize}
\item {Grp. gram.:f.}
\end{itemize}
\begin{itemize}
\item {Utilização:Prov.}
\end{itemize}
\begin{itemize}
\item {Utilização:trasm.}
\end{itemize}
Encêrro, feito de cancelas, onde pernoita o alavão.
\section{Canceladura}
\begin{itemize}
\item {Grp. gram.:f.}
\end{itemize}
O mesmo que \textunderscore cancelamento\textunderscore .
\section{Cancelamento}
\begin{itemize}
\item {Grp. gram.:m.}
\end{itemize}
Acto de \textunderscore cancelar\textunderscore .
Traço ou nota de que um registo está cancelado.
\section{Cancelar}
\begin{itemize}
\item {Grp. gram.:v. t.}
\end{itemize}
\begin{itemize}
\item {Proveniência:(Lat. \textunderscore cancellare\textunderscore )}
\end{itemize}
Riscar, inutilizar com traços, (uma escrita, um registo, etc.).
Tornar sem efeito, por meio de declaração junta.
Concluir, fechar, (um processo).
\section{Cancelário}
\begin{itemize}
\item {Grp. gram.:m.}
\end{itemize}
\begin{itemize}
\item {Proveniência:(Lat. \textunderscore cancellarius\textunderscore )}
\end{itemize}
Antiga dignidade universitária.
\section{Cancella}
\begin{itemize}
\item {Grp. gram.:f.}
\end{itemize}
Porta gradeada de madeira, mais tôsca que o cancêllo.
(Cp. \textunderscore cancêllo\textunderscore )
\section{Cancellação}
\begin{itemize}
\item {Grp. gram.:f.}
\end{itemize}
\begin{itemize}
\item {Proveniência:(Lat. \textunderscore cancellatio\textunderscore )}
\end{itemize}
O mesmo que \textunderscore cancellamento\textunderscore .
\section{Cancellada}
\begin{itemize}
\item {Grp. gram.:f.}
\end{itemize}
\begin{itemize}
\item {Utilização:Prov.}
\end{itemize}
\begin{itemize}
\item {Utilização:trasm.}
\end{itemize}
Encêrro, feito de cancellas, onde pernoita o alavão.
\section{Cancelladura}
\begin{itemize}
\item {Grp. gram.:f.}
\end{itemize}
O mesmo que \textunderscore cancellamento\textunderscore .
\section{Cancellamento}
\begin{itemize}
\item {Grp. gram.:m.}
\end{itemize}
Acto de \textunderscore cancellar\textunderscore .
Traço ou nota de que um registo está cancellado.
\section{Cancellar}
\begin{itemize}
\item {Grp. gram.:v. t.}
\end{itemize}
\begin{itemize}
\item {Proveniência:(Lat. \textunderscore cancellare\textunderscore )}
\end{itemize}
Riscar, inutilizar com traços, (uma escrita, um registo, etc.).
Tornar sem effeito, por meio de declaração junta.
Concluir, fechar, (um processo).
\section{Cancellário}
\begin{itemize}
\item {Grp. gram.:m.}
\end{itemize}
\begin{itemize}
\item {Proveniência:(Lat. \textunderscore cancellarius\textunderscore )}
\end{itemize}
Antiga dignidade universitária.
\section{Cancêllo}
\begin{itemize}
\item {Grp. gram.:m.}
\end{itemize}
\begin{itemize}
\item {Utilização:Ant.}
\end{itemize}
\begin{itemize}
\item {Proveniência:(Lat. \textunderscore cancellus\textunderscore )}
\end{itemize}
Pequena porta gradeada.
Reunião de sebes, que fórmam curral transitório nos campos, para que o gado os esterque.
Grade nobre de capella; pórtico
\section{Cancêlo}
\begin{itemize}
\item {Grp. gram.:m.}
\end{itemize}
\begin{itemize}
\item {Utilização:Ant.}
\end{itemize}
\begin{itemize}
\item {Proveniência:(Lat. \textunderscore cancellus\textunderscore )}
\end{itemize}
Pequena porta gradeada.
Reunião de sebes, que fórmam curral transitório nos campos, para que o gado os esterque.
Grade nobre de capela; pórtico
\section{Câncer}
\begin{itemize}
\item {Grp. gram.:m.}
\end{itemize}
\begin{itemize}
\item {Proveniência:(Lat. \textunderscore cancer\textunderscore )}
\end{itemize}
Uma das constellações do Zodíaco.
Cancro.
\section{Canceração}
\begin{itemize}
\item {Grp. gram.:f.}
\end{itemize}
Acto de \textunderscore cancerar\textunderscore .
\section{Cancerar}
\begin{itemize}
\item {Grp. gram.:v. t.}
\end{itemize}
\begin{itemize}
\item {Proveniência:(Lat. \textunderscore cancerare\textunderscore )}
\end{itemize}
Converter em cancro.
\section{Canceriforme}
\begin{itemize}
\item {Grp. gram.:adj.}
\end{itemize}
\begin{itemize}
\item {Proveniência:(Do lat. \textunderscore cancer\textunderscore  + \textunderscore forma\textunderscore )}
\end{itemize}
Que tem fórma de cancro.
\section{Canceroso}
\begin{itemize}
\item {Grp. gram.:adj.}
\end{itemize}
\begin{itemize}
\item {Proveniência:(De \textunderscore câncer\textunderscore )}
\end{itemize}
Que tem natureza de cancro.
\section{Cancha}
\begin{itemize}
\item {Grp. gram.:m.}
\end{itemize}
\begin{itemize}
\item {Utilização:Bras}
\end{itemize}
\begin{itemize}
\item {Utilização:Fig.}
\end{itemize}
Lugar, onde se matam os bois.
Lugar, onde o cavallo está acostumado a correr.
Lugar, onde se fazem corridas de cavallos.
Lugar, onde se joga a pela.
Commodidade, bel-prazer, situação favorável: \textunderscore está na sua cancha\textunderscore .
\section{Canchal}
\begin{itemize}
\item {Grp. gram.:m.}
\end{itemize}
\begin{itemize}
\item {Utilização:Prov.}
\end{itemize}
\begin{itemize}
\item {Utilização:trasm.}
\end{itemize}
Grande porção; abundância.
(Talvez cast. \textunderscore canchal\textunderscore )
\section{Cancheiro}
\begin{itemize}
\item {Grp. gram.:adj.}
\end{itemize}
\begin{itemize}
\item {Utilização:Bras}
\end{itemize}
Perito nas canchas, acostumado a ellas.
\section{Canchim}
\begin{itemize}
\item {Grp. gram.:m.}
\end{itemize}
Árvore brasileira, de fôlhas espinhosas.
\section{Cancho}
\begin{itemize}
\item {Grp. gram.:m.}
\end{itemize}
\begin{itemize}
\item {Utilização:Prov.}
\end{itemize}
\begin{itemize}
\item {Utilização:alent.}
\end{itemize}
Penedo.
Penhasco.
(Cast. \textunderscore cancho\textunderscore )
\section{Cancioneiro}
\begin{itemize}
\item {Grp. gram.:m.}
\end{itemize}
\begin{itemize}
\item {Proveniência:(Do lat. \textunderscore cantio\textunderscore , \textunderscore cantionis\textunderscore )}
\end{itemize}
Reunião de canções.
Cada uma das collecções da antiga poesia lýrica portuguesa.
\section{Cancionista}
\begin{itemize}
\item {Grp. gram.:m.  e  f.}
\end{itemize}
Pessôa, que faz canções.
(Cp. \textunderscore cancioneiro\textunderscore )
\section{Çanco}
\begin{itemize}
\item {Grp. gram.:m.}
\end{itemize}
\begin{itemize}
\item {Utilização:Fig.}
\end{itemize}
\begin{itemize}
\item {Utilização:T. de Viana}
\end{itemize}
(Fórma exacta, em vez da usual \textunderscore sanco\textunderscore )
Perna da ave, desde a garra até á junta da coxa.
Perna delgada.
Perna de qualquer animal de açougue.
(Cast. \textunderscore zanco\textunderscore )
\section{Cancom}
\begin{itemize}
\item {Grp. gram.:m.}
\end{itemize}
Planta hortense da China. Cf. \textunderscore Ásia Sínica\textunderscore , 59.
\section{Cançoneta}
\begin{itemize}
\item {fónica:nê}
\end{itemize}
\begin{itemize}
\item {Grp. gram.:f.}
\end{itemize}
\begin{itemize}
\item {Proveniência:(De \textunderscore canção\textunderscore )}
\end{itemize}
Pequena canção, posta em música.
\section{Cançoneteiro}
\begin{itemize}
\item {Grp. gram.:m.}
\end{itemize}
\begin{itemize}
\item {Utilização:Deprec.}
\end{itemize}
O mesmo que \textunderscore cançonetista\textunderscore .
\section{Cançonetista}
\begin{itemize}
\item {Grp. gram.:adj.}
\end{itemize}
\begin{itemize}
\item {Grp. gram.:M.}
\end{itemize}
Relativo a cançoneta.
Próprio de cançoneta.
Usado em cançonetas.
Autor de cançonetas.
\section{Cancrejo}
\begin{itemize}
\item {Grp. gram.:m.}
\end{itemize}
\begin{itemize}
\item {Utilização:Ant.}
\end{itemize}
O mesmo que \textunderscore caranguejo\textunderscore .
(Cast. \textunderscore cangrejo\textunderscore )
\section{Cancrinita}
\begin{itemize}
\item {Grp. gram.:f.}
\end{itemize}
\begin{itemize}
\item {Proveniência:(De \textunderscore Cancrin\textunderscore , n. p.)}
\end{itemize}
Nome de um silicato natural de alumina, potassa e soda.
\section{Cancrizante}
\begin{itemize}
\item {Grp. gram.:adj.}
\end{itemize}
\begin{itemize}
\item {Utilização:Mús.}
\end{itemize}
\begin{itemize}
\item {Utilização:Ant.}
\end{itemize}
\begin{itemize}
\item {Proveniência:(Lat. barb. \textunderscore cancrizans\textunderscore , do lat. \textunderscore cancer\textunderscore .)}
\end{itemize}
Dizia-se da composição em contraponto retrógado, a qual podia sêr executada de deante para trás, ou de trás para deante.
\section{Cancro}
\begin{itemize}
\item {Grp. gram.:m.}
\end{itemize}
\begin{itemize}
\item {Utilização:Carp.}
\end{itemize}
\begin{itemize}
\item {Utilização:Fig.}
\end{itemize}
Tumor, que dilacera as partes onde se desenvolve.
Cirro.
Ulcera, de procedência venérea.
O mesmo que \textunderscore câncer\textunderscore , constellação e trópico. Cf. \textunderscore Conquista de Pegu.\textunderscore  I.
Peça de ferro, para fixar numa parede ou cantaria qualquer trabalho de carpinteiro.
Mal, que vai arruinando.
Utensilio de ferro, com que os carpinteiros seguram nos bancos a madeira em que trabalham.
(Alter. de \textunderscore cancer\textunderscore )
\section{Cancrocida}
\begin{itemize}
\item {Grp. gram.:m.}
\end{itemize}
\begin{itemize}
\item {Utilização:Bras}
\end{itemize}
\begin{itemize}
\item {Proveniência:(Do lat. \textunderscore cancer\textunderscore  + \textunderscore caedere\textunderscore )}
\end{itemize}
Medicamento contra os cancros.
\section{Cancroide}
\begin{itemize}
\item {Grp. gram.:adj.}
\end{itemize}
\begin{itemize}
\item {Proveniência:(De \textunderscore cancro\textunderscore  + gr. \textunderscore eidos\textunderscore )}
\end{itemize}
Semelhante ao cancro.
\section{Cancroma}
\begin{itemize}
\item {Grp. gram.:f.}
\end{itemize}
\begin{itemize}
\item {Proveniência:(Lat. \textunderscore cancroma\textunderscore )}
\end{itemize}
Ave pernalta da América.
\section{Cancroso}
\begin{itemize}
\item {Grp. gram.:adj.}
\end{itemize}
O mesmo que \textunderscore canceroso\textunderscore .
\section{Canda}
\begin{itemize}
\item {Grp. gram.:f.}
\end{itemize}
Árvore angolense, no Duque-de-Bragança.
\section{Candado}
\begin{itemize}
\item {Grp. gram.:m.}
\end{itemize}
(V. \textunderscore cando\textunderscore ^1)
\section{Candango}
\begin{itemize}
\item {Grp. gram.:m.}
\end{itemize}
Nome, com que os africanos designam o português.
(Do \textunderscore quimb.\textunderscore )
\section{Candanim}
\begin{itemize}
\item {Grp. gram.:m.}
\end{itemize}
\begin{itemize}
\item {Utilização:Ant.}
\end{itemize}
Espécie de taficira.
\section{Cande}
\begin{itemize}
\item {Grp. gram.:adj.}
\end{itemize}
\begin{itemize}
\item {Proveniência:(Do ár. \textunderscore kand\textunderscore )}
\end{itemize}
Diz-se do açúcar refinado, crystallizado e meio transparente.
\section{Candeada}
\begin{itemize}
\item {Grp. gram.:f.}
\end{itemize}
\begin{itemize}
\item {Proveniência:(De \textunderscore candeia\textunderscore ^1)}
\end{itemize}
Porção de óleo, que uma candeia comporta.
\section{Candearia}
\begin{itemize}
\item {Grp. gram.:f.}
\end{itemize}
\begin{itemize}
\item {Proveniência:(De \textunderscore candeia\textunderscore ^1)}
\end{itemize}
Conjunto de candeeiros, ou outros objectos que servem para alumiar.
\section{Candeeirada}
\begin{itemize}
\item {Grp. gram.:f.}
\end{itemize}
O mesmo que \textunderscore candeada\textunderscore .
Porção de óleo que um candeeiro comporta.
\section{Candeeireiro}
\begin{itemize}
\item {Grp. gram.:m.}
\end{itemize}
Aquelle que faz ou vende candeeiros.
\section{Candeeiro}
\begin{itemize}
\item {Grp. gram.:m.}
\end{itemize}
\begin{itemize}
\item {Utilização:Ant.}
\end{itemize}
\begin{itemize}
\item {Utilização:Mad}
\end{itemize}
\begin{itemize}
\item {Utilização:Bras. do S}
\end{itemize}
\begin{itemize}
\item {Utilização:Ant.}
\end{itemize}
\begin{itemize}
\item {Proveniência:(De \textunderscore candeia\textunderscore ^1)}
\end{itemize}
Vaso de várias fórmas, que, sustentado por um pé ou suspenso, é destinado a dar luz, alimentada por óleo ou gás inflammável.
Parapeito, que, nas minas, abriga os operários.
Archote, em assaltos nocturnos ás fortalezas.
Homem, que guia uma corsa ou corsão.
Homem, que, armado de aguilhada, vai adeante dos bois, guiando-os.
Bailarico, espécie de fandango.
Fabricante de velas.
\section{Candeia}
\begin{itemize}
\item {Grp. gram.:f.}
\end{itemize}
\begin{itemize}
\item {Utilização:Prov.}
\end{itemize}
\begin{itemize}
\item {Utilização:minh.}
\end{itemize}
\begin{itemize}
\item {Utilização:Ant.}
\end{itemize}
\begin{itemize}
\item {Grp. gram.:F. pl.}
\end{itemize}
\begin{itemize}
\item {Proveniência:(Lat. \textunderscore candela\textunderscore )}
\end{itemize}
Vaso de fôlha ou de barro, que se usa suspenso de parede ou velador, e em que se deita óleo, para alimentar a luz na torcida que sai por um bico do mesmo vaso.
Caruma sêca.
Amentilho, florescência em cachos.
Nome de várias plantas.
O mesmo que \textunderscore vela\textunderscore ^1 de cera.
Candelária, festa religiosa.
\section{Candeia}
\begin{itemize}
\item {Grp. gram.:adj.}
\end{itemize}
\begin{itemize}
\item {Utilização:Bras}
\end{itemize}
Elegante; formoso; gracioso.
(Talvez do guarani)
\section{Candeio}
\begin{itemize}
\item {Grp. gram.:m.}
\end{itemize}
\begin{itemize}
\item {Proveniência:(De \textunderscore candeia\textunderscore ^1)}
\end{itemize}
Luzeiro, que se usa de noite na caça ou pesca.
\section{Candela}
\begin{itemize}
\item {Grp. gram.:f.}
\end{itemize}
\begin{itemize}
\item {Utilização:Prov.}
\end{itemize}
\begin{itemize}
\item {Utilização:trasm.}
\end{itemize}
O mesmo que \textunderscore candeia\textunderscore ^1.
(Cast. \textunderscore candela\textunderscore )
\section{Candelabro}
\begin{itemize}
\item {Grp. gram.:m.}
\end{itemize}
\begin{itemize}
\item {Proveniência:(Lat. \textunderscore candelabrum\textunderscore )}
\end{itemize}
Grande castiçal, com ramificações, cada uma das quaes corresponde a uma luz.
Serpentina; lustre.
\section{Candelária}
\begin{itemize}
\item {Grp. gram.:f.}
\end{itemize}
\begin{itemize}
\item {Proveniência:(Do lat. \textunderscore candela\textunderscore )}
\end{itemize}
Festa da Purificação de Nossa Senhora, em 2 de Fevereiro; festa das \textunderscore candeias\textunderscore .
Nome de várias plantas.
\section{Candeleja}
\begin{itemize}
\item {Grp. gram.:f.}
\end{itemize}
\begin{itemize}
\item {Utilização:Ant.}
\end{itemize}
\begin{itemize}
\item {Proveniência:(De \textunderscore candela\textunderscore )}
\end{itemize}
O mesmo que \textunderscore candeia\textunderscore ^1.
\section{Candeliça}
\begin{itemize}
\item {Grp. gram.:f.}
\end{itemize}
\begin{itemize}
\item {Utilização:Náut.}
\end{itemize}
Cabo singelo, para içar velas ou bandeiras.
\section{Candelinha}
\begin{itemize}
\item {Grp. gram.:f.}
\end{itemize}
\begin{itemize}
\item {Proveniência:(Do lat. \textunderscore candela\textunderscore )}
\end{itemize}
O mesmo que \textunderscore algália\textunderscore ^1.
\section{Candena}
\begin{itemize}
\item {Grp. gram.:f.}
\end{itemize}
\begin{itemize}
\item {Utilização:Prov.}
\end{itemize}
\begin{itemize}
\item {Utilização:trasm.}
\end{itemize}
(V.candela)
\section{Candência}
\begin{itemize}
\item {Grp. gram.:f.}
\end{itemize}
\begin{itemize}
\item {Proveniência:(Lat. \textunderscore candentia\textunderscore )}
\end{itemize}
Qualidade daquillo que está candente.
\section{Candente}
\begin{itemize}
\item {Grp. gram.:adj.}
\end{itemize}
\begin{itemize}
\item {Proveniência:(Lat. \textunderscore candens\textunderscore )}
\end{itemize}
Que aqueceu a ponto de estar rubro-claro; em brasa.
\section{Candeólo}
\begin{itemize}
\item {Grp. gram.:m.}
\end{itemize}
\begin{itemize}
\item {Utilização:Prov.}
\end{itemize}
\begin{itemize}
\item {Proveniência:(De \textunderscore candeia\textunderscore ^1)}
\end{itemize}
O mesmo que \textunderscore sincelo\textunderscore ^1.
\section{Candeólos}
\begin{itemize}
\item {Grp. gram.:m. pl.}
\end{itemize}
\begin{itemize}
\item {Utilização:Prov.}
\end{itemize}
\begin{itemize}
\item {Utilização:trasm.}
\end{itemize}
O mesmo que \textunderscore carambina\textunderscore .
\section{Candeu}
\begin{itemize}
\item {Grp. gram.:adj.}
\end{itemize}
Semelhante ao cando, ou ao casco da bêsta:«\textunderscore has de pôr o pé candeu, como o põe o mariola\textunderscore ». Serrão de Castro, in \textunderscore Acad. dos Sing.\textunderscore , 204.
\section{Cândi}
\begin{itemize}
\item {Grp. gram.:adj.}
\end{itemize}
\begin{itemize}
\item {Proveniência:(Do ár. \textunderscore kand\textunderscore )}
\end{itemize}
Diz-se do açúcar refinado, crystallizado e meio transparente.
\section{Candial}
\begin{itemize}
\item {Grp. gram.:adj.}
\end{itemize}
Diz-se de uma espécie de trigo, cuja farinha é muito alva.
(Por \textunderscore candidal\textunderscore , de \textunderscore cândido\textunderscore )
\section{Candicante}
\begin{itemize}
\item {Grp. gram.:f.}
\end{itemize}
\begin{itemize}
\item {Proveniência:(Lat. \textunderscore candicans\textunderscore )}
\end{itemize}
Espécie de videira, cujas uvas brancas são pequenos bagos.
\section{Candidamente}
\begin{itemize}
\item {Grp. gram.:adv.}
\end{itemize}
\begin{itemize}
\item {Proveniência:(De \textunderscore cândido\textunderscore )}
\end{itemize}
Com ingenuidade, com candidez.
\section{Cândaro}
\begin{itemize}
\item {Grp. gram.:m.}
\end{itemize}
\begin{itemize}
\item {Utilização:Prov.}
\end{itemize}
\begin{itemize}
\item {Utilização:trasm.}
\end{itemize}
Pernada sêca de árvore.
\section{Candidato}
\begin{itemize}
\item {Grp. gram.:m.}
\end{itemize}
\begin{itemize}
\item {Utilização:Ant.}
\end{itemize}
\begin{itemize}
\item {Proveniência:(Lat. \textunderscore candidatus\textunderscore )}
\end{itemize}
Aquelle que aspira a emprêgo ou dignidade.
Aquelle que solicita votos, a fim de sêr eleito para um cargo.
Aquelle que em Roma pretendia cargo público, trajando de branco, para indicar a sinceridade da sua pretensão.
\section{Candidatura}
\begin{itemize}
\item {Grp. gram.:f.}
\end{itemize}
Qualidade de quem é candidato.
Apresentação de candidato aos votos dos cidadãos.
\section{Candidez}
\begin{itemize}
\item {Grp. gram.:f.}
\end{itemize}
\begin{itemize}
\item {Proveniência:(De \textunderscore cândido\textunderscore )}
\end{itemize}
Alvura.
O mesmo que \textunderscore candura\textunderscore .
\section{Candideza}
\begin{itemize}
\item {Grp. gram.:f.}
\end{itemize}
(V.candidez)
\section{Candidizar}
\begin{itemize}
\item {Grp. gram.:v. t.}
\end{itemize}
Tornar cândido, puro. Cf. F. Manuel, \textunderscore Fidalgo Apr\textunderscore .
\section{Cândido}
\begin{itemize}
\item {Grp. gram.:adj.}
\end{itemize}
\begin{itemize}
\item {Utilização:Fig.}
\end{itemize}
\begin{itemize}
\item {Proveniência:(Lat. \textunderscore candidus\textunderscore )}
\end{itemize}
Muito branco.
Sincero.
Ingênuo.
Puro; que tem candura.
\section{Candieiro}
\textunderscore m.\textunderscore  (e der.)
(V. \textunderscore candeeiro\textunderscore , etc.)
\section{Candil}
\begin{itemize}
\item {Grp. gram.:adj.}
\end{itemize}
O mesmo que \textunderscore candial\textunderscore .
\section{Candil}
\begin{itemize}
\item {Grp. gram.:m.}
\end{itemize}
\begin{itemize}
\item {Proveniência:(Do ár. \textunderscore kandil\textunderscore )}
\end{itemize}
Candeia, lampada.
Phosphorescência das águas.
\section{Candil}
\begin{itemize}
\item {Grp. gram.:m.}
\end{itemize}
Medida de capacidade na Índia.
Antiga moéda asiática.
(Do malaialim \textunderscore kandi\textunderscore )
\section{Candilar}
\begin{itemize}
\item {Grp. gram.:v. t.}
\end{itemize}
\begin{itemize}
\item {Proveniência:(De \textunderscore cândi\textunderscore )}
\end{itemize}
Tornar crystallizado.
Cobrir de açúcar cândi.
\section{Candim}
\begin{itemize}
\item {Grp. gram.:m.}
\end{itemize}
O mesmo que \textunderscore candil\textunderscore ^3. Cf. \textunderscore Peregrinação\textunderscore , CLVIII.
\section{Cândio}
\begin{itemize}
\item {Grp. gram.:m.}
\end{itemize}
Túnica persa.
(Cp. gr. \textunderscore kandus\textunderscore )
\section{Candiota}
\begin{itemize}
\item {Grp. gram.:m.  e  f.}
\end{itemize}
Indivíduo, natural de Cândia ou Creta.
\section{Candiru}
\begin{itemize}
\item {Grp. gram.:m.}
\end{itemize}
Peixe marítimo do Brasil, (\textunderscore cetopsis candiru\textunderscore ).
\section{Cando}
\begin{itemize}
\item {Grp. gram.:m.}
\end{itemize}
Parte do casco da bêsta, entre as ranilhas e o mais delgado da tapa.
\section{Cando}
\begin{itemize}
\item {Grp. gram.:m.}
\end{itemize}
\begin{itemize}
\item {Utilização:Prov.}
\end{itemize}
\begin{itemize}
\item {Utilização:trasm.}
\end{itemize}
Pernada sêca de árvore.
\section{Candólea}
\begin{itemize}
\item {Grp. gram.:f.}
\end{itemize}
\begin{itemize}
\item {Proveniência:(De \textunderscore Candolle\textunderscore , n. p.)}
\end{itemize}
Gênero de arbustos adstringentes.
\section{Candóllea}
\begin{itemize}
\item {Grp. gram.:f.}
\end{itemize}
\begin{itemize}
\item {Proveniência:(De \textunderscore Candolle\textunderscore , n. p.)}
\end{itemize}
Gênero de arbustos adstringentes.
\section{Candom}
\begin{itemize}
\item {Grp. gram.:m.}
\end{itemize}
Árvore da Índia portuguesa.
\section{Candombe}
\begin{itemize}
\item {Grp. gram.:m.}
\end{itemize}
\begin{itemize}
\item {Utilização:Bras}
\end{itemize}
Rede de pescar camarões.
Espécie de batuque, usado pelos negros.
(Cp. \textunderscore candomblé\textunderscore )
\section{Candombeiro}
\begin{itemize}
\item {Grp. gram.:m.}
\end{itemize}
\begin{itemize}
\item {Utilização:Bras}
\end{itemize}
Lançador de candombe.
\section{Candomblé}
\begin{itemize}
\item {Grp. gram.:m.}
\end{itemize}
\begin{itemize}
\item {Utilização:Bras}
\end{itemize}
Espécie de batuque, que os negros acompanham com exercícios de feitiçaria.
\section{Candonga}
\begin{itemize}
\item {Grp. gram.:f.}
\end{itemize}
Contrabando de gêneros alimentícios.
\section{Candonga}
\begin{itemize}
\item {Grp. gram.:f.}
\end{itemize}
Lisonja, afagos fingidos.
(Cp. cast. \textunderscore candonguero\textunderscore , que faz meiguices)
\section{Candongar}
\begin{itemize}
\item {Grp. gram.:v. i.}
\end{itemize}
\begin{itemize}
\item {Proveniência:(De \textunderscore candonga\textunderscore ^1)}
\end{itemize}
Fazer contrabando.
\section{Candongueiro}
\begin{itemize}
\item {Grp. gram.:m.}
\end{itemize}
\begin{itemize}
\item {Utilização:Prov.}
\end{itemize}
\begin{itemize}
\item {Utilização:dur.}
\end{itemize}
Aquelle que faz \textunderscore candonga\textunderscore ^1.
Mentiroso.
Impostor.
\section{Candonguice}
\begin{itemize}
\item {Grp. gram.:f.}
\end{itemize}
(V. \textunderscore candonga\textunderscore ^1)
\section{Candor}
\begin{itemize}
\item {Grp. gram.:m.}
\end{itemize}
\begin{itemize}
\item {Proveniência:(Lat. \textunderscore candor\textunderscore )}
\end{itemize}
O mesmo que \textunderscore candura\textunderscore .
\section{Candorça}
\begin{itemize}
\item {Grp. gram.:f.}
\end{itemize}
\begin{itemize}
\item {Utilização:Prov.}
\end{itemize}
\begin{itemize}
\item {Utilização:trasm.}
\end{itemize}
Égua ou mula velha e escanzelada.
Mulher muito magra.
(Cp. \textunderscore comborça\textunderscore )
\section{Candorim}
\begin{itemize}
\item {Grp. gram.:m.}
\end{itemize}
\begin{itemize}
\item {Proveniência:(Do mal. \textunderscore kanduri\textunderscore )}
\end{itemize}
Peso chinês, centésima parte da onça.
\section{Candro}
\begin{itemize}
\item {Grp. gram.:m.}
\end{itemize}
\begin{itemize}
\item {Utilização:Prov.}
\end{itemize}
\begin{itemize}
\item {Utilização:trasm.}
\end{itemize}
Pernada sêca de árvore.
\section{Candua}
\begin{itemize}
\item {Grp. gram.:f.}
\end{itemize}
Planta angolense monocotyledónea.
\section{Candundobala}
\begin{itemize}
\item {Grp. gram.:f.}
\end{itemize}
Ave trepadora da África.
\section{Candura}
\begin{itemize}
\item {Grp. gram.:f.}
\end{itemize}
Qualidade daquillo ou daquelle que é cândido.
Alvura.
Simplicidade; pureza.
(Por \textunderscore candidura\textunderscore , de \textunderscore cândido\textunderscore )
\section{Candurim}
\begin{itemize}
\item {Grp. gram.:m.}
\end{itemize}
O mesmo ou melhor que candorim.
\section{Canear}
\begin{itemize}
\item {Grp. gram.:v. i.}
\end{itemize}
\begin{itemize}
\item {Utilização:Prov.}
\end{itemize}
\begin{itemize}
\item {Utilização:trasm.}
\end{itemize}
Cabecear com somno.
\section{Canebas}
\begin{itemize}
\item {Grp. gram.:m.}
\end{itemize}
\begin{itemize}
\item {Proveniência:(Fr. \textunderscore canevas\textunderscore )}
\end{itemize}
Antigo estôfo de cânhamo.
Peça dêsse estôfo, nas antigas armaduras.
\section{Caneca}
\begin{itemize}
\item {Grp. gram.:f.}
\end{itemize}
\begin{itemize}
\item {Proveniência:(De \textunderscore cano\textunderscore ?)}
\end{itemize}
Pequeno vaso cylíndrico com asa.
\section{Canecada}
\begin{itemize}
\item {Grp. gram.:f.}
\end{itemize}
\begin{itemize}
\item {Proveniência:(De \textunderscore caneca\textunderscore )}
\end{itemize}
Porção de líquido, que uma caneca comporta.
\section{Caneco}
\begin{itemize}
\item {Grp. gram.:m.}
\end{itemize}
\begin{itemize}
\item {Utilização:Prov.}
\end{itemize}
\begin{itemize}
\item {Grp. gram.:Adj.}
\end{itemize}
\begin{itemize}
\item {Utilização:Prov.}
\end{itemize}
\begin{itemize}
\item {Utilização:trasm.}
\end{itemize}
Caneca alta e estreita.
Espécie de barril.
O mesmo que penante, chapéu alto.
Um tanto ébrio; ébrio.
(Cp. \textunderscore caneca\textunderscore )
\section{Çanefa}
\begin{itemize}
\item {Grp. gram.:f.}
\end{itemize}
\begin{itemize}
\item {Proveniência:(Do ár. \textunderscore aç-çanifa\textunderscore )}
\end{itemize}
(Fórma exacta em vez da usual, \textunderscore sanefa\textunderscore )
Larga tira de fazenda, que se atravessa como ornato na extremidade superior de uma cortina, nas vêrgas das janelas, etc.
Tábua atravessada, a que se segura uma série de outras, que são verticaes áquella.
\section{Canéfora}
\begin{itemize}
\item {Grp. gram.:f.}
\end{itemize}
\begin{itemize}
\item {Proveniência:(Do gr. \textunderscore kane\textunderscore  + \textunderscore phoros\textunderscore )}
\end{itemize}
Estátua decorativa, que tem um açafate na cabeça.
\section{Canefóreas}
\begin{itemize}
\item {Grp. gram.:f. pl.}
\end{itemize}
\begin{itemize}
\item {Proveniência:(De \textunderscore canéphora\textunderscore )}
\end{itemize}
Festas antigas, em que as donzellas rogavam á deusa Diana que as desobrigasse do voto de virgindade.
\section{Canéforo}
\begin{itemize}
\item {Grp. gram.:m.}
\end{itemize}
O mesmo que \textunderscore canéfora\textunderscore . Cf. Filinto, XVI, 203.
\section{Caneira}
\begin{itemize}
\item {Grp. gram.:f.}
\end{itemize}
\begin{itemize}
\item {Proveniência:(De \textunderscore cana\textunderscore )}
\end{itemize}
Caule de certas plantas, especialmente da faveira.
\section{Caneiro}
\begin{itemize}
\item {Grp. gram.:m.}
\end{itemize}
\begin{itemize}
\item {Proveniência:(De \textunderscore cano\textunderscore )}
\end{itemize}
Pequeno canal.
Passagem entre estacadas, no leito de um rio.
Parte mais funda do leito do rio.
Caniçada para pesca.
Braço de mar entre rochedos.
\section{Caneja}
\begin{itemize}
\item {Grp. gram.:f.}
\end{itemize}
\begin{itemize}
\item {Utilização:Marn.}
\end{itemize}
\begin{itemize}
\item {Proveniência:(Do rad. de \textunderscore cano\textunderscore )}
\end{itemize}
Rêgo, que, de dois em dois compartimentos, se abre na andaina de cima, nas salinas.
\section{Caneja}
\begin{itemize}
\item {Grp. gram.:f.}
\end{itemize}
Espécie de cação.
(Fem. de \textunderscore canejo\textunderscore )
\section{Canejo}
\begin{itemize}
\item {Grp. gram.:adj.}
\end{itemize}
\begin{itemize}
\item {Proveniência:(Do rad. do lat. \textunderscore canis\textunderscore )}
\end{itemize}
Relativo a cão.
Que tem apparência de cão.
\section{Canejo}
\begin{itemize}
\item {Grp. gram.:m.}
\end{itemize}
\begin{itemize}
\item {Utilização:Prov.}
\end{itemize}
\begin{itemize}
\item {Utilização:minh.}
\end{itemize}
Indivíduo cambaio, que tem as pernas tortas.
\section{Canela}
\begin{itemize}
\item {Grp. gram.:f.}
\end{itemize}
\begin{itemize}
\item {Grp. gram.:Loc.}
\end{itemize}
\begin{itemize}
\item {Utilização:fam.}
\end{itemize}
\begin{itemize}
\item {Proveniência:(Lat. \textunderscore canella\textunderscore , ou \textunderscore cannella\textunderscore )}
\end{itemize}
Casca odorífera de uma planta de Ceilão.
Árvore, que produz canela; caneleira.
Designação de algumas árvores, semelhantes á caneleira.
Parte da perna, entre o pé e o joêlho.
Pequeno canudo, em que se enrola o fio para a tecelagem.
\textunderscore Dar á canela\textunderscore , andar depressa, fugir.
\section{Canelada}
\begin{itemize}
\item {Grp. gram.:f.}
\end{itemize}
\begin{itemize}
\item {Proveniência:(De \textunderscore canela\textunderscore )}
\end{itemize}
Pancada na canela da perna.
\section{Canelado}
\begin{itemize}
\item {Grp. gram.:adj.}
\end{itemize}
Que tem duas caneluras.
\section{Caneladura}
\begin{itemize}
\item {Grp. gram.:f.}
\end{itemize}
O mesmo que \textunderscore canelura\textunderscore .
\section{Canelagem}
\begin{itemize}
\item {Grp. gram.:f.}
\end{itemize}
Acto de \textunderscore canelar\textunderscore .
\section{Canelão}
\begin{itemize}
\item {Grp. gram.:m.}
\end{itemize}
\begin{itemize}
\item {Utilização:T. de Coimbra}
\end{itemize}
\begin{itemize}
\item {Proveniência:(De \textunderscore canela\textunderscore )}
\end{itemize}
O mesmo que \textunderscore canelada\textunderscore .
Fio da teia, mais grosso que os outros.
Confeito, que finge amêndoa confeccionada e cujo núcleo é um pedaço de canela, revestido de uma cobertura granulada de açúcar.
Pontapé, com que os novatos da Universidade eram recebidos á porta do palácio das aulas.
\section{Canelar}
\begin{itemize}
\item {Grp. gram.:v. t.}
\end{itemize}
\begin{itemize}
\item {Grp. gram.:V. i.}
\end{itemize}
Lavrar caneluras em.
Encher as canelas, com que se tece.
\section{Caneleira}
\begin{itemize}
\item {Grp. gram.:f.}
\end{itemize}
Árvore laurínea, que dá canela.
Peça de armadura, que cobria a canela da perna.
Máquina, o mesmo que \textunderscore caneleiro\textunderscore .
Arbusto de jardim, de flôres aromáticas, da côr da canela.
\section{Caneleiro}
\begin{itemize}
\item {Grp. gram.:m.}
\end{itemize}
\begin{itemize}
\item {Proveniência:(De \textunderscore canela\textunderscore )}
\end{itemize}
Utensílio de tecelagem, em que se fixa a canela para enrolar o fio.
Operário, que enche as canelas nas fábricas de fiação e tecidos.
O mesmo que \textunderscore caneleira\textunderscore , árvore.
\section{Canelha}
\begin{itemize}
\item {fónica:nê}
\end{itemize}
\begin{itemize}
\item {Grp. gram.:f.}
\end{itemize}
\begin{itemize}
\item {Utilização:Prov.}
\end{itemize}
\begin{itemize}
\item {Utilização:trasm.}
\end{itemize}
Calleja, quelha.
Quelho.
(Cp. \textunderscore canada\textunderscore )
\section{Canelina}
\begin{itemize}
\item {Grp. gram.:f.}
\end{itemize}
\begin{itemize}
\item {Proveniência:(De \textunderscore canela\textunderscore )}
\end{itemize}
Princípio estimulante do cynnamomo.
\section{Canelo}
\begin{itemize}
\item {fónica:nê}
\end{itemize}
\begin{itemize}
\item {Grp. gram.:m.}
\end{itemize}
\begin{itemize}
\item {Utilização:Fam.}
\end{itemize}
\begin{itemize}
\item {Utilização:Pop.}
\end{itemize}
Ferradura curta, própria para bois.
Ferradura, gasta em parte.
Osso comprido.
A canela da perna.
O mesmo que \textunderscore calcanhar\textunderscore :«\textunderscore algumas mulheres, com os canelos callosos e encodeados\textunderscore ». Camillo, \textunderscore Brasileira\textunderscore , 339.
\section{Canelura}
\begin{itemize}
\item {Grp. gram.:f.}
\end{itemize}
\begin{itemize}
\item {Utilização:Bot.}
\end{itemize}
\begin{itemize}
\item {Proveniência:(De \textunderscore canela\textunderscore )}
\end{itemize}
Estria, sulco aberto como meia cana verticalmente, em columnas ou outras partes de construcção.
Estria nos caules.
\section{Canema}
\begin{itemize}
\item {Grp. gram.:f.}
\end{itemize}
Árvore do Brasil.
\section{Canena}
\begin{itemize}
\item {Grp. gram.:f.}
\end{itemize}
\begin{itemize}
\item {Utilização:Prov.}
\end{itemize}
\begin{itemize}
\item {Utilização:trasm.}
\end{itemize}
Mulher somítica, avarenta.
\section{Canepeteira}
\begin{itemize}
\item {Grp. gram.:f.}
\end{itemize}
Pequena ave pernalta de arribação.
\section{Canéphora}
\begin{itemize}
\item {Grp. gram.:f.}
\end{itemize}
\begin{itemize}
\item {Proveniência:(Do gr. \textunderscore kane\textunderscore  + \textunderscore phoros\textunderscore )}
\end{itemize}
Estátua decorativa, que tem um açafate na cabeça.
\section{Canephóreas}
\begin{itemize}
\item {Grp. gram.:f. pl.}
\end{itemize}
\begin{itemize}
\item {Proveniência:(De \textunderscore canéphora\textunderscore )}
\end{itemize}
Festas antigas, em que as donzellas rogavam á deusa Diana que as desobrigasse do voto de virgindade.
\section{Canéphoro}
\begin{itemize}
\item {Grp. gram.:m.}
\end{itemize}
O mesmo que \textunderscore canéphora\textunderscore . Cf. Filinto, XVI, 203.
\section{Canequi}
\begin{itemize}
\item {Grp. gram.:m.}
\end{itemize}
\begin{itemize}
\item {Utilização:Ant.}
\end{itemize}
Tecido de algodão da Índia.
\section{Canequim}
\begin{itemize}
\item {Grp. gram.:m.}
\end{itemize}
Tecido de algodão da Índia.
\section{Caneta}
\begin{itemize}
\item {fónica:nê}
\end{itemize}
\begin{itemize}
\item {Grp. gram.:f.}
\end{itemize}
\begin{itemize}
\item {Proveniência:(De \textunderscore cana\textunderscore )}
\end{itemize}
Pequeno tubo, em que se encaixa ou a que se adapta um lápis ou uma penna metállica, para que se possa escrever.
Cabo, com que os operadores cirúrgicos seguram o cautério.
\section{Caneta}
\begin{itemize}
\item {fónica:nê}
\end{itemize}
\begin{itemize}
\item {Grp. gram.:m.}
\end{itemize}
\begin{itemize}
\item {Utilização:Açor}
\end{itemize}
Diabo, cão tinhoso: \textunderscore vai-te para o caneta\textunderscore .
\section{Canevão}
\begin{itemize}
\item {Grp. gram.:m.}
\end{itemize}
Pequeno cylindro de metal, nas chaves das flautas, dos clarinetes e de instrumentos análogos.
\section{Canez}
\begin{itemize}
\item {Grp. gram.:f.}
\end{itemize}
\begin{itemize}
\item {Proveniência:(De \textunderscore cano\textunderscore ^2)}
\end{itemize}
Brancura. Cf. Usque, \textunderscore Tribulações\textunderscore , 37 V.^o
\section{Canfênico}
\begin{itemize}
\item {Grp. gram.:adj.}
\end{itemize}
\begin{itemize}
\item {Utilização:Chím.}
\end{itemize}
Diz-se de uma variedade de carbonetos.
\section{Çanfona}
\begin{itemize}
\item {Grp. gram.:f.}
\end{itemize}
(Graphia exacta, em vez da usual, \textunderscore sanfona\textunderscore )
\section{Cânfora}
\begin{itemize}
\item {Grp. gram.:f.}
\end{itemize}
Substância aromática, que se extrai do canforeiro.
Resina de várias plantas.
Árvore, que produz a cânfora; canforeiro.
(Ár. \textunderscore cafur\textunderscore )
\section{Canforado}
\begin{itemize}
\item {Grp. gram.:adj.}
\end{itemize}
Que tem cânfora, misturada ou dissolvida.
\section{Canforar}
\begin{itemize}
\item {Grp. gram.:v. t.}
\end{itemize}
Misturar com cânfora.
Dissolver cânfora em.
Cobrir de cânfora.
\section{Canforato}
\begin{itemize}
\item {Grp. gram.:m.}
\end{itemize}
\begin{itemize}
\item {Proveniência:(De \textunderscore cânfora\textunderscore )}
\end{itemize}
Sal, formado pela combinação do ácido canfórico com uma base.
\section{Canforeira}
\begin{itemize}
\item {Grp. gram.:f.}
\end{itemize}
\begin{itemize}
\item {Proveniência:(De \textunderscore cânfora\textunderscore )}
\end{itemize}
Árvore laurínea oriental, de que se extrái a cânfora por meio de destillação.
\section{Canforeiro}
\begin{itemize}
\item {Grp. gram.:m.}
\end{itemize}
\begin{itemize}
\item {Proveniência:(De \textunderscore cânfora\textunderscore )}
\end{itemize}
Árvore laurínea oriental, de que se extrái a cânfora por meio de destillação.
\section{Canfórico}
\begin{itemize}
\item {Grp. gram.:adj.}
\end{itemize}
\begin{itemize}
\item {Proveniência:(De \textunderscore cânfora\textunderscore )}
\end{itemize}
Diz-se de um ácido, produzido pela destillação do ácido azótico sôbre a cânfora.
\section{Canforífero}
\begin{itemize}
\item {Grp. gram.:adj.}
\end{itemize}
\begin{itemize}
\item {Proveniência:(De \textunderscore cânfora\textunderscore  + lat. \textunderscore ferre\textunderscore )}
\end{itemize}
Que produz cânfora.
\section{Canforina}
\begin{itemize}
\item {Grp. gram.:f.}
\end{itemize}
\begin{itemize}
\item {Proveniência:(De \textunderscore cânfora\textunderscore )}
\end{itemize}
Combinação neutra do ácido canfórico com a glycerina.
\section{Canforoide}
\begin{itemize}
\item {Grp. gram.:adj.}
\end{itemize}
\begin{itemize}
\item {Proveniência:(De \textunderscore cânfora\textunderscore  + gr. \textunderscore eidos\textunderscore )}
\end{itemize}
Que é semelhante á cânfora.
\section{Canforomania}
\begin{itemize}
\item {Grp. gram.:f.}
\end{itemize}
\begin{itemize}
\item {Proveniência:(De \textunderscore cânfora\textunderscore  + \textunderscore mania\textunderscore )}
\end{itemize}
Abuso da cânfora.
\section{Canforosma}
\begin{itemize}
\item {Grp. gram.:f.}
\end{itemize}
Planta medicinal, sudorífica e excitante.
\section{Canforosol}
\begin{itemize}
\item {Grp. gram.:m.}
\end{itemize}
\begin{itemize}
\item {Utilização:Pharm.}
\end{itemize}
Medicamento canforado e chlofórmico.
\section{Canfovínico}
\begin{itemize}
\item {Grp. gram.:adj.}
\end{itemize}
\begin{itemize}
\item {Proveniência:(De \textunderscore cânfora\textunderscore  + \textunderscore vínico\textunderscore )}
\end{itemize}
Diz-se de um ácido, que se obtém pela acção do ácido canfórico em álcool misturado com ácido sulfúrico ou chlorhýdrico.
\section{Canga}
\begin{itemize}
\item {Grp. gram.:f.}
\end{itemize}
\begin{itemize}
\item {Utilização:Pop.}
\end{itemize}
\begin{itemize}
\item {Utilização:Fig.}
\end{itemize}
\begin{itemize}
\item {Utilização:Prov.}
\end{itemize}
\begin{itemize}
\item {Utilização:minh.}
\end{itemize}
\begin{itemize}
\item {Utilização:Prov.}
\end{itemize}
\begin{itemize}
\item {Utilização:minh.}
\end{itemize}
\begin{itemize}
\item {Utilização:Gír.}
\end{itemize}
Jugo de madeira, com que se unem os bois para o trabalho.
Pau, que os moços de fretes apoiam no chinguiço e donde suspendem a corda com que transportam objectos pesados.
O mesmo que \textunderscore chinguiço\textunderscore .
Oppressão, domínio.
Pau, vergado ao lume, e que apanha a barbela dos animaes contra o jugo.
Engenhoca de madeira que se põe ao cachaço dos porcos, para os impedir, quando soltos pelos caminhos, de entrar nos bueiros e cancellos.
Instrumento de supplício, na China.
Igreja.
(Cp. \textunderscore cangar\textunderscore )
\section{Canga}
\begin{itemize}
\item {Grp. gram.:f.}
\end{itemize}
\begin{itemize}
\item {Utilização:Bras}
\end{itemize}
Mineral de ferro argiloso e pardacento, que se encontra no Estado de San-Paulo.
\section{Cangabicha}
\begin{itemize}
\item {Grp. gram.:f.}
\end{itemize}
Árvore brasileira.
\section{Cangaçaes}
\begin{itemize}
\item {Grp. gram.:m. pl.}
\end{itemize}
\begin{itemize}
\item {Utilização:Bras}
\end{itemize}
\begin{itemize}
\item {Utilização:burl.}
\end{itemize}
\begin{itemize}
\item {Proveniência:(De \textunderscore cangaço\textunderscore )}
\end{itemize}
Mobilia de pessôa pobre; cangalhada.
\section{Cangaceiro}
\begin{itemize}
\item {Grp. gram.:m.}
\end{itemize}
\begin{itemize}
\item {Utilização:Bras}
\end{itemize}
\begin{itemize}
\item {Proveniência:(De \textunderscore cangaço\textunderscore )}
\end{itemize}
Aquelle que usa muitas armas, ostentando valentia.
\section{Cangaço}
\begin{itemize}
\item {Grp. gram.:m.}
\end{itemize}
\begin{itemize}
\item {Utilização:Bras}
\end{itemize}
\begin{itemize}
\item {Utilização:Prov.}
\end{itemize}
\begin{itemize}
\item {Utilização:beir.}
\end{itemize}
\begin{itemize}
\item {Proveniência:(De \textunderscore cango\textunderscore )}
\end{itemize}
Engaço; resíduo das uvas, depois de pisadas e de extrahido o líquido.
Pedúnculo do coqueiro.
Utensílios de casa pobre.
Conjunto das armas do cangaceiro.
Carolo da espiga de milho.
\section{Cangalha}
\begin{itemize}
\item {Grp. gram.:f.}
\end{itemize}
\begin{itemize}
\item {Utilização:T. da Bairrada}
\end{itemize}
Carro, puxado por um só boi.
(Cp. \textunderscore cangalho\textunderscore )
\section{Cangalhada}
\begin{itemize}
\item {Grp. gram.:f.}
\end{itemize}
\begin{itemize}
\item {Proveniência:(De \textunderscore cangalho\textunderscore )}
\end{itemize}
Acervo de trastes velhos ou coisas reles.
\section{Cangalhão}
\begin{itemize}
\item {Grp. gram.:m.}
\end{itemize}
\begin{itemize}
\item {Utilização:Prov.}
\end{itemize}
\begin{itemize}
\item {Utilização:dur.}
\end{itemize}
\begin{itemize}
\item {Proveniência:(De \textunderscore cangalho\textunderscore )}
\end{itemize}
O mesmo que \textunderscore cangalho\textunderscore .
Homem precocemente avelhentado.
\section{Cangalhas}
\begin{itemize}
\item {Grp. gram.:f. pl.}
\end{itemize}
\begin{itemize}
\item {Utilização:Pop.}
\end{itemize}
\begin{itemize}
\item {Grp. gram.:Loc. adv.}
\end{itemize}
\begin{itemize}
\item {Proveniência:(Do rad. de \textunderscore canga\textunderscore )}
\end{itemize}
Armação, ordinariamente de madeira, em que se sustenta e equilibra carga de bêstas, de um e outro lado dellas.
Óculos.
Peças, em que descança a moéga.
\textunderscore De cangalhas\textunderscore , de pernas para o ar, estateladamente.
\section{Cangalheiro}
\begin{itemize}
\item {Grp. gram.:m.}
\end{itemize}
\begin{itemize}
\item {Grp. gram.:Adj.}
\end{itemize}
Aquelle que conduz bêstas com cangalhas, recoveiro.
Aquelle que aluga ou prepara os aprestos de entêrro.
Relativo a cangalhas.
\section{Cangalho}
\begin{itemize}
\item {Grp. gram.:m.}
\end{itemize}
\begin{itemize}
\item {Utilização:Prov.}
\end{itemize}
\begin{itemize}
\item {Utilização:dur.}
\end{itemize}
\begin{itemize}
\item {Utilização:Fam.}
\end{itemize}
\begin{itemize}
\item {Utilização:Prov.}
\end{itemize}
\begin{itemize}
\item {Utilização:Prov.}
\end{itemize}
\begin{itemize}
\item {Utilização:alg.}
\end{itemize}
\begin{itemize}
\item {Proveniência:(De \textunderscore canga\textunderscore )}
\end{itemize}
Cada um dos paus ou canzis que seguram a canga no pescoço dos bois, de muares ou equídeos.
O mesmo que \textunderscore cangalha\textunderscore .
O mesmo que \textunderscore canganho\textunderscore .
Pessôa ou coisa inútil ou velha.
Traste velho e inútil.
Pequena canga, em carro puxado por uma só bêsta.
\section{Cangalulo}
\begin{itemize}
\item {Grp. gram.:m.}
\end{itemize}
Árvore angolense, da fam. das solâneas.
\section{Cangambá}
\begin{itemize}
\item {Grp. gram.:m.}
\end{itemize}
\begin{itemize}
\item {Utilização:Bras. do N}
\end{itemize}
O mesmo que \textunderscore maritacaca\textunderscore .
\section{Cangancha}
\begin{itemize}
\item {Grp. gram.:f.}
\end{itemize}
\begin{itemize}
\item {Utilização:Bras}
\end{itemize}
Trapaça, velhacada, sobretudo ao jôgo.
\section{Canganho}
\begin{itemize}
\item {Grp. gram.:m.}
\end{itemize}
O mesmo que \textunderscore engaço\textunderscore ; esqueleto de cacho do uvas.
(Cp. \textunderscore cango\textunderscore )
\section{Cangapé}
\begin{itemize}
\item {Grp. gram.:m.}
\end{itemize}
\begin{itemize}
\item {Utilização:Bras}
\end{itemize}
O mesmo que \textunderscore cambapé\textunderscore .
\section{Cangapora}
\begin{itemize}
\item {Grp. gram.:m.}
\end{itemize}
\begin{itemize}
\item {Utilização:Bras}
\end{itemize}
Espécie de cágado.
\section{Cangar}
\begin{itemize}
\item {Grp. gram.:v. t.}
\end{itemize}
\begin{itemize}
\item {Utilização:Des.}
\end{itemize}
Unir com canga; junjir.
Subjugar, opprimir.
Segurar com paus (os tectos do colmo).
Fazer pirraça a.
(Por \textunderscore congar\textunderscore , contr. de \textunderscore conjugar\textunderscore )
\section{Cangarilhada}
\begin{itemize}
\item {Grp. gram.:f.}
\end{itemize}
\begin{itemize}
\item {Proveniência:(De \textunderscore cangar\textunderscore )}
\end{itemize}
Tramóia; engano.
\section{Cangarina}
\begin{itemize}
\item {Grp. gram.:f.}
\end{itemize}
\begin{itemize}
\item {Utilização:Gír.}
\end{itemize}
O mesmo que \textunderscore gangarina\textunderscore .
\section{Cangarinha}
\begin{itemize}
\item {Grp. gram.:f.}
\end{itemize}
Planta composta, de grandes flôres amarelas.
\section{Cangatá}
\begin{itemize}
\item {Grp. gram.:m.}
\end{itemize}
\begin{itemize}
\item {Utilização:Bras}
\end{itemize}
Cordão, feito de pennas.
\section{Cangemoiro}
\begin{itemize}
\item {Grp. gram.:m.}
\end{itemize}
\begin{itemize}
\item {Utilização:Prov.}
\end{itemize}
\begin{itemize}
\item {Utilização:trasm.}
\end{itemize}
Planta silvestre, espinhosa, com que se bardam paredes.
\section{Cangerana}
\begin{itemize}
\item {Grp. gram.:f.}
\end{itemize}
Árvore meliácea do Brasil.
\section{Cângi}
\begin{itemize}
\item {Grp. gram.:m.}
\end{itemize}
Decocção de arroz, sem tempêro algum, mas que os Índios consideram alimentícia e refrigerante.
\section{Cangica}
\begin{itemize}
\item {Grp. gram.:f.}
\end{itemize}
(V.canjica)
\section{Cangirão}
\begin{itemize}
\item {Grp. gram.:m.}
\end{itemize}
\begin{itemize}
\item {Proveniência:(Do lat. \textunderscore cangius\textunderscore )}
\end{itemize}
Grande vaso, de bôca larga, para vinho.
\section{Canabina}
\begin{itemize}
\item {Grp. gram.:f.}
\end{itemize}
\begin{itemize}
\item {Proveniência:(Lat. \textunderscore cannalina\textunderscore )}
\end{itemize}
Medicamento, applicado contra as perdas de sangue, em seguida ao mênstruo.
\section{Canabíneas}
\begin{itemize}
\item {Grp. gram.:f. pl.}
\end{itemize}
Família de plantas, que comprehende o cânhamo e o lúpulo.
(Cp. lat. \textunderscore cannabis\textunderscore )
\section{Cango}
\begin{itemize}
\item {Grp. gram.:m.}
\end{itemize}
\begin{itemize}
\item {Utilização:Prov.}
\end{itemize}
\begin{itemize}
\item {Utilização:dur.}
\end{itemize}
\begin{itemize}
\item {Utilização:Prov.}
\end{itemize}
\begin{itemize}
\item {Utilização:trasm.}
\end{itemize}
\begin{itemize}
\item {Utilização:Prov.}
\end{itemize}
\begin{itemize}
\item {Utilização:minh.}
\end{itemize}
Crosta, que as uvas, depois da primeira pisa, formam á superfície do lagar, em quanto o vinho ferve por baixo.
A flôr da oliveira.
Barrote.
\section{Cangoéra}
\begin{itemize}
\item {Grp. gram.:f.}
\end{itemize}
(V. \textunderscore cangueira\textunderscore ^1)
\section{Cangonha}
\begin{itemize}
\item {Grp. gram.:f.}
\end{itemize}
O mesmo que \textunderscore bangue\textunderscore .
\section{Cangosta}
\begin{itemize}
\item {Grp. gram.:f.}
\end{itemize}
(V.congosta)
\section{Cangote}
\begin{itemize}
\item {Grp. gram.:m.}
\end{itemize}
\begin{itemize}
\item {Utilização:Anat.}
\end{itemize}
Nome que, no Brasil, se dá á região occipital.
(Cp. \textunderscore cogote\textunderscore )
\section{Cangrejo}
\begin{itemize}
\item {Grp. gram.:m.}
\end{itemize}
\begin{itemize}
\item {Utilização:Ant.}
\end{itemize}
O mesmo que \textunderscore cancrejo\textunderscore . Cf. \textunderscore Lusiadas\textunderscore , VI, 18.
\section{Cangrena}
\begin{itemize}
\item {Grp. gram.:f.}
\end{itemize}
(Dissimilação de \textunderscore gangrena\textunderscore )
\section{Cangúa}
\begin{itemize}
\item {Grp. gram.:m.}
\end{itemize}
Pássaro conirostro da África.
\section{Canguari}
\begin{itemize}
\item {Grp. gram.:m.}
\end{itemize}
Ave pernalta da África.
\section{Canguçu}
\begin{itemize}
\item {Grp. gram.:m.}
\end{itemize}
\begin{itemize}
\item {Utilização:Bras}
\end{itemize}
Espécie de onça.
(Do tupi \textunderscore acanga\textunderscore  + \textunderscore uçu\textunderscore )
\section{Cangue}
\begin{itemize}
\item {Grp. gram.:m.}
\end{itemize}
Árvore da Índia portuguesa.
\section{Cangueira}
\begin{itemize}
\item {Grp. gram.:f.}
\end{itemize}
\begin{itemize}
\item {Utilização:Prov.}
\end{itemize}
\begin{itemize}
\item {Utilização:alent.}
\end{itemize}
\begin{itemize}
\item {Utilização:Fig.}
\end{itemize}
\begin{itemize}
\item {Proveniência:(De \textunderscore canga\textunderscore )}
\end{itemize}
Callosidade no pescoço de animaes, resultante do uso da canga.
Doença do pescoço.
\section{Cangueira}
\begin{itemize}
\item {fónica:gu-ei}
\end{itemize}
\begin{itemize}
\item {Grp. gram.:f.}
\end{itemize}
\begin{itemize}
\item {Utilização:Bras}
\end{itemize}
Frauta, feita de ossos de defunto.
\section{Cangueiro}
\begin{itemize}
\item {Grp. gram.:adj.}
\end{itemize}
\begin{itemize}
\item {Utilização:Bras}
\end{itemize}
\begin{itemize}
\item {Utilização:Bras. do N}
\end{itemize}
\begin{itemize}
\item {Grp. gram.:M.}
\end{itemize}
\begin{itemize}
\item {Proveniência:(De \textunderscore canga\textunderscore )}
\end{itemize}
Que tem canga, ou que está no caso de a supportar.
Preguiçoso; negligente.
Curvado a um pêso.
Obediente, submísso.
Espécie de barco chato, usado no Tejo.
\section{Canguelo}
\begin{itemize}
\item {fónica:guê}
\end{itemize}
\begin{itemize}
\item {Grp. gram.:m.}
\end{itemize}
\begin{itemize}
\item {Utilização:Prov.}
\end{itemize}
\begin{itemize}
\item {Utilização:beir.}
\end{itemize}
Ódio, rancor.
\section{Canguiço}
\begin{itemize}
\item {Grp. gram.:m.}
\end{itemize}
\begin{itemize}
\item {Utilização:T. de Barcelos}
\end{itemize}
Lenha miúda; gravetos; cavacos.
(Cp. \textunderscore cango\textunderscore )
\section{Canguinha}
\begin{itemize}
\item {Grp. gram.:m.}
\end{itemize}
\begin{itemize}
\item {Utilização:Fam.}
\end{itemize}
Homem apoucado; fraca figura.
Sovina.
\section{Canguinhar}
\begin{itemize}
\item {Grp. gram.:v. i.}
\end{itemize}
\begin{itemize}
\item {Utilização:Fam.}
\end{itemize}
\begin{itemize}
\item {Proveniência:(De \textunderscore canguinha\textunderscore )}
\end{itemize}
Sêr canguinhas.
Estar irresoluto; demorar-se em tomar decisão.
\section{Canguinhas}
\begin{itemize}
\item {Grp. gram.:m.}
\end{itemize}
\begin{itemize}
\item {Utilização:Fam.}
\end{itemize}
Homem apoucado; fraca figura.
Sovina.
\section{Cangulo}
\begin{itemize}
\item {Grp. gram.:m.}
\end{itemize}
Peixe de Portugal.
\section{Canguru}
\begin{itemize}
\item {Grp. gram.:m.}
\end{itemize}
\begin{itemize}
\item {Utilização:Zool.}
\end{itemize}
O maior mammífero da ordem dos marsupiaes.
Designação genérica dos marsupiaes.
\section{Canha}
\begin{itemize}
\item {Grp. gram.:f.}
\end{itemize}
\begin{itemize}
\item {Utilização:Pop.}
\end{itemize}
Mão esquerda.
(Fam. de \textunderscore canho\textunderscore )
\section{Canhada}
\begin{itemize}
\item {Grp. gram.:f.}
\end{itemize}
\begin{itemize}
\item {Utilização:Bras}
\end{itemize}
\begin{itemize}
\item {Utilização:Prov.}
\end{itemize}
\begin{itemize}
\item {Utilização:trasm.}
\end{itemize}
Planície estreita entre montanhas.
Canada, azinhaga.
(Cast. \textunderscore cañada\textunderscore , cp. \textunderscore canada\textunderscore )
\section{Canhamaço}
\begin{itemize}
\item {Grp. gram.:m.}
\end{itemize}
\begin{itemize}
\item {Proveniência:(Do b. lat. \textunderscore cannabaceus\textunderscore )}
\end{itemize}
Estôpa de cânhamo.
Tecido grosso de cânhamo.
\section{Canhameira}
\begin{itemize}
\item {Grp. gram.:f.}
\end{itemize}
\begin{itemize}
\item {Proveniência:(De \textunderscore cânhamo\textunderscore )}
\end{itemize}
Planta malvácea.
\section{Canhameiral}
\begin{itemize}
\item {Grp. gram.:m.}
\end{itemize}
Lugar, onde cresce cânhamo.
\section{Canhameiro}
\begin{itemize}
\item {Grp. gram.:m.}
\end{itemize}
O mesmo que \textunderscore canhameiral\textunderscore .
\section{Canhamiço}
\begin{itemize}
\item {Grp. gram.:adj.}
\end{itemize}
Relativo ao cânhamo.
\section{Cânhamo}
\begin{itemize}
\item {Grp. gram.:m.}
\end{itemize}
\begin{itemize}
\item {Proveniência:(Lat. \textunderscore cannabum\textunderscore )}
\end{itemize}
Planta cannabínea, cujos abundantes filamentos servem para tecidos.
Fios ou pano de cânhamo.
Designação de várias outras plantas filamentosas.
\section{Canhanho}
\begin{itemize}
\item {Grp. gram.:m.}
\end{itemize}
O mesmo ou melhor que \textunderscore canhenho\textunderscore ^1.
\section{Canhantes}
\begin{itemize}
\item {Grp. gram.:m. pl.}
\end{itemize}
\begin{itemize}
\item {Utilização:Gír.}
\end{itemize}
\begin{itemize}
\item {Proveniência:(De \textunderscore canhão\textunderscore )}
\end{itemize}
Botins.
\section{Canhão}
\begin{itemize}
\item {Grp. gram.:m.}
\end{itemize}
Peça de artilharia.
Peça de metal, que há em certas fechaduras, denominadas por isso \textunderscore fechaduras de canhão\textunderscore .
Cano de pennas grossas de asas de aves.
Extremidade inferior da manga do vestuário, quando é sobreposta ou finge sê-lo.
Extremidade superior do cano da bota.
(Cast. \textunderscore cañón\textunderscore )
\section{Canhas}
\begin{itemize}
\item {Grp. gram.:f. pl.}
\end{itemize}
(Us. só na loc. \textunderscore ás canhas\textunderscore , e significando \textunderscore ás avessas\textunderscore , \textunderscore á maneira de canhoto\textunderscore , \textunderscore ao contrário do que se usa\textunderscore )
(Pl. de \textunderscore canha\textunderscore )
\section{Canhas}
\begin{itemize}
\item {Grp. gram.:adj. f. pl.}
\end{itemize}
\begin{itemize}
\item {Utilização:Prov.}
\end{itemize}
\begin{itemize}
\item {Utilização:alent.}
\end{itemize}
Diz-se das migas que, depois de feitas, se comem com leite.
\section{Canhembora}
\begin{itemize}
\item {Grp. gram.:m.  e  f.}
\end{itemize}
\begin{itemize}
\item {Utilização:Bras}
\end{itemize}
Escravo fugidío, que se esconde nos quilombos.
(Do tupi \textunderscore acanhem\textunderscore )
\section{Canhengila}
\begin{itemize}
\item {Grp. gram.:f.}
\end{itemize}
Árvore angolense, de tronco tortuoso e copa farta.
\section{Canhenha}
\begin{itemize}
\item {Grp. gram.:f.}
\end{itemize}
\begin{itemize}
\item {Utilização:Bras}
\end{itemize}
Peixe do mar.
\section{Canhenho}
\begin{itemize}
\item {Grp. gram.:m.}
\end{itemize}
\begin{itemize}
\item {Utilização:Fig.}
\end{itemize}
Registo de lembranças; caderno de apontamentos.
A memória.
(Talvez de \textunderscore canhão\textunderscore )
\section{Canhenho}
\begin{itemize}
\item {Grp. gram.:m.  e  adj.}
\end{itemize}
\begin{itemize}
\item {Utilização:Prov.}
\end{itemize}
\begin{itemize}
\item {Utilização:dur.}
\end{itemize}
O mesmo que \textunderscore canhoto\textunderscore .
\section{Canhestramente}
\begin{itemize}
\item {Grp. gram.:adv.}
\end{itemize}
\begin{itemize}
\item {Utilização:Pop.}
\end{itemize}
De modo canhestro; desajeitadamente.
Com acanhamento.
\section{Canhestro}
\begin{itemize}
\item {fónica:nhês}
\end{itemize}
\begin{itemize}
\item {Grp. gram.:adj.}
\end{itemize}
\begin{itemize}
\item {Utilização:Pop.}
\end{itemize}
\begin{itemize}
\item {Proveniência:(De \textunderscore canho\textunderscore ^1)}
\end{itemize}
Feito ás canhas.
Desajeitado.
Acanhado.
\section{Canho}
\begin{itemize}
\item {Grp. gram.:adj.}
\end{itemize}
\begin{itemize}
\item {Proveniência:(Do lat. hyp. \textunderscore caneus\textunderscore , de \textunderscore canis\textunderscore , cão?)}
\end{itemize}
O mesmo que \textunderscore canhoto\textunderscore .
\section{Canho}
\begin{itemize}
\item {Grp. gram.:m.}
\end{itemize}
\begin{itemize}
\item {Utilização:Pop.}
\end{itemize}
Lucro caviloso.
\section{Canhoeira}
\begin{itemize}
\item {Grp. gram.:f.}
\end{itemize}
(V.canhoneira)
\section{Canhol}
\begin{itemize}
\item {Grp. gram.:m.}
\end{itemize}
\begin{itemize}
\item {Utilização:Prov.}
\end{itemize}
\begin{itemize}
\item {Utilização:dur.}
\end{itemize}
Cão pequeno, cachorro.
(Cp. toscano \textunderscore cagnolo\textunderscore )
\section{Canhona}
\begin{itemize}
\item {Grp. gram.:f.}
\end{itemize}
\begin{itemize}
\item {Utilização:Prov.}
\end{itemize}
\begin{itemize}
\item {Utilização:trasm.}
\end{itemize}
Ovelha.
\section{Canhonaço}
\begin{itemize}
\item {Grp. gram.:m.}
\end{itemize}
Tiro de canhão.
\section{Canhonada}
\begin{itemize}
\item {Grp. gram.:f.}
\end{itemize}
Tiroteio de canhão.
\section{Canhonar}
\begin{itemize}
\item {Grp. gram.:v. t.}
\end{itemize}
Guarnecer de canhões.
\section{Canhonear}
\begin{itemize}
\item {Grp. gram.:v. t.}
\end{itemize}
Atacar com tiros de canhão; bombardear.
\section{Canhoneio}
\begin{itemize}
\item {Grp. gram.:m.}
\end{itemize}
Canhonada.
Acto de \textunderscore canhonear\textunderscore .
\section{Canhoneira}
\begin{itemize}
\item {Grp. gram.:f.}
\end{itemize}
\begin{itemize}
\item {Proveniência:(De \textunderscore canhão\textunderscore )}
\end{itemize}
Abertura na muralha ou nos flancos de um navio, para assestar e disparar canhões.
Embarcação guarnecida de artilharia.
\section{Canhoneiro}
\begin{itemize}
\item {Grp. gram.:adj.}
\end{itemize}
\begin{itemize}
\item {Proveniência:(De \textunderscore canhão\textunderscore )}
\end{itemize}
Que tem artilharia.
\section{Canhonha}
\begin{itemize}
\item {Grp. gram.:f.}
\end{itemize}
\begin{itemize}
\item {Utilização:Prov.}
\end{itemize}
\begin{itemize}
\item {Utilização:trasm.}
\end{itemize}
O mesmo que \textunderscore canhona\textunderscore .
\section{Canhorra}
\begin{itemize}
\item {Grp. gram.:m.}
\end{itemize}
\begin{itemize}
\item {Utilização:Prov.}
\end{itemize}
\begin{itemize}
\item {Utilização:beir.}
\end{itemize}
Farricoco, que costumava ir na procissão dos Passos, tocando buzina.
\section{Canhos}
\begin{itemize}
\item {Grp. gram.:m. pl.}
\end{itemize}
\begin{itemize}
\item {Utilização:T. de Barcelos}
\end{itemize}
\begin{itemize}
\item {Utilização:Prov.}
\end{itemize}
\begin{itemize}
\item {Utilização:minh.}
\end{itemize}
Palhiço miúdo, que fica do centeio, depois de malhado.
\textunderscore Andar aos canhos\textunderscore , andar aos sobejos, ás migalhas, em cata do que lhe queiram dar.
\section{Canhota}
\begin{itemize}
\item {Grp. gram.:f.}
\end{itemize}
\begin{itemize}
\item {Utilização:Pop.}
\end{itemize}
\begin{itemize}
\item {Utilização:T. da Bairrada}
\end{itemize}
\begin{itemize}
\item {Grp. gram.:Loc. adv.}
\end{itemize}
A mão esquerda. Cf. Filinto, IV, 80.
Espécie de moinho de água, em que a roda grande gira contra a cale, recebendo a corrente por trás.
A roda grande dêsse moinho.
\textunderscore Á canhota\textunderscore , á maneira de canhoto.
(Cp. \textunderscore canhoto\textunderscore )
\section{Canhoteiro}
\begin{itemize}
\item {Grp. gram.:adj.}
\end{itemize}
\begin{itemize}
\item {Utilização:Bras. do S}
\end{itemize}
O mesmo que \textunderscore canhoto\textunderscore .
\section{Canhoto}
\begin{itemize}
\item {fónica:nhô}
\end{itemize}
\begin{itemize}
\item {Grp. gram.:adj.}
\end{itemize}
\begin{itemize}
\item {Grp. gram.:M.}
\end{itemize}
\begin{itemize}
\item {Utilização:Pop.}
\end{itemize}
\begin{itemize}
\item {Utilização:Prov.}
\end{itemize}
\begin{itemize}
\item {Utilização:minh.}
\end{itemize}
\begin{itemize}
\item {Utilização:Bras}
\end{itemize}
\begin{itemize}
\item {Utilização:Bras}
\end{itemize}
\begin{itemize}
\item {Proveniência:(De \textunderscore canho\textunderscore )}
\end{itemize}
Que executa com a mão esquerda serviços que geralmente se fazem com a direita.
Esquerdo.
Que não é destro, que não é hábil.
Indivíduo, que, em trabalho de mãos, se serve, com preferência, da esquerda.
Demónio: \textunderscore cruzes, canhoto\textunderscore !
Pedaço de lenha, toscamente partido.
Crustáceo de coiraça calcária.
Quitação ou documento official, que se entrega a quem pagou uma contribuição e que corresponde ao talão português.
\section{Caniana}
\begin{itemize}
\item {Grp. gram.:f.}
\end{itemize}
\begin{itemize}
\item {Utilização:Bras}
\end{itemize}
O mesmo que \textunderscore caninana\textunderscore .
\section{Canibal}
\begin{itemize}
\item {Grp. gram.:m.}
\end{itemize}
\begin{itemize}
\item {Proveniência:(De \textunderscore canniba\textunderscore , t. americano)}
\end{itemize}
Selvagem, que come carne humana.
Homem feroz.
\section{Canibalesco}
\begin{itemize}
\item {Grp. gram.:adj.}
\end{itemize}
Próprio de canibal.
\section{Canibalismo}
\begin{itemize}
\item {Grp. gram.:m.}
\end{itemize}
Qualidade de canibal.
\section{Caniça}
\begin{itemize}
\item {Grp. gram.:f.}
\end{itemize}
\begin{itemize}
\item {Utilização:Prov.}
\end{itemize}
\begin{itemize}
\item {Proveniência:(De \textunderscore cana\textunderscore )}
\end{itemize}
Tecido de vimes, que se crava aos lados, no leito do carro de lavoira; sebe.
\section{Caniçada}
\begin{itemize}
\item {Grp. gram.:f.}
\end{itemize}
\begin{itemize}
\item {Utilização:Prov.}
\end{itemize}
\begin{itemize}
\item {Proveniência:(De \textunderscore caniço\textunderscore )}
\end{itemize}
Latada, ou sebe, feita de canas ou caniços.
Carga, que se leva dentro da caniça.
\section{Caniçado}
\begin{itemize}
\item {Grp. gram.:m.}
\end{itemize}
O mesmo que \textunderscore caniçada\textunderscore .
\section{Caniçal}
\begin{itemize}
\item {Grp. gram.:m.}
\end{itemize}
Moita de caniços.
\section{Caniçalha}
\begin{itemize}
\item {Grp. gram.:f.}
\end{itemize}
O mesmo que \textunderscore cainçada\textunderscore .
\section{Canicaru}
\begin{itemize}
\item {Grp. gram.:m.}
\end{itemize}
\begin{itemize}
\item {Utilização:Bras}
\end{itemize}
Índio civilizado.
\section{Canicho}
\begin{itemize}
\item {Grp. gram.:m.}
\end{itemize}
\begin{itemize}
\item {Proveniência:(Do lat. \textunderscore canis\textunderscore )}
\end{itemize}
Pequeno cão, cãozinho.
\section{Canícia}
\begin{itemize}
\item {Grp. gram.:f.}
\end{itemize}
O mesmo que \textunderscore canície\textunderscore .
\section{Canicida}
\begin{itemize}
\item {Grp. gram.:m.}
\end{itemize}
Aquelle que mata um cão ou cães.
(Cp. \textunderscore canicídio\textunderscore )
\section{Canicídio}
\begin{itemize}
\item {Grp. gram.:m.}
\end{itemize}
\begin{itemize}
\item {Proveniência:(Do lat. \textunderscore canis\textunderscore  + \textunderscore caedere\textunderscore )}
\end{itemize}
Morte violenta de um cão.
\section{Canície}
\begin{itemize}
\item {Grp. gram.:f.}
\end{itemize}
\begin{itemize}
\item {Proveniência:(Lat. \textunderscore canities\textunderscore )}
\end{itemize}
Alvura dos cabellos.
Idade, em que os cabellos se tornam brancos.
\section{Canicinho}
\begin{itemize}
\item {Grp. gram.:m.}
\end{itemize}
\begin{itemize}
\item {Utilização:Açor}
\end{itemize}
Motejo, escárneo.
\section{Caniço}
\begin{itemize}
\item {Grp. gram.:m.}
\end{itemize}
\begin{itemize}
\item {Utilização:Prov.}
\end{itemize}
\begin{itemize}
\item {Utilização:Prov.}
\end{itemize}
\begin{itemize}
\item {Utilização:alent.}
\end{itemize}
Cana delgada.
Grade de canas para o fumeiro.
Canaveal.
Jangada.
Cana, para pescar ao anzol.
Armadilha para caçar, feita de vime, o mesmo que \textunderscore nassa\textunderscore .
Rede de canas, que se suspende do tecto, e em que se secam queijos.
\section{Caniçoso}
\begin{itemize}
\item {Grp. gram.:adj.}
\end{itemize}
\begin{itemize}
\item {Proveniência:(De \textunderscore caniço\textunderscore )}
\end{itemize}
Que tem muitas canas ou caniços.
\section{Caniçote}
\begin{itemize}
\item {Grp. gram.:m.}
\end{itemize}
\begin{itemize}
\item {Utilização:Prov.}
\end{itemize}
\begin{itemize}
\item {Utilização:trasm.}
\end{itemize}
O mesmo que \textunderscore arcabém\textunderscore .
\section{Canícula}
\begin{itemize}
\item {Grp. gram.:f.}
\end{itemize}
\begin{itemize}
\item {Proveniência:(Do lat. \textunderscore canicula\textunderscore )}
\end{itemize}
Estrêlla Sirio.
Estação calmosa, em que esta estrêlla e o Sol estão em conjunção.
Calor.
\section{Canícula}
\begin{itemize}
\item {Grp. gram.:f.}
\end{itemize}
\begin{itemize}
\item {Utilização:Pop.}
\end{itemize}
\begin{itemize}
\item {Proveniência:(De \textunderscore cana\textunderscore )}
\end{itemize}
Pequena cana.
Perna delgada.
\section{Canicular}
\begin{itemize}
\item {Grp. gram.:adj.}
\end{itemize}
\begin{itemize}
\item {Proveniência:(Lat. \textunderscore canicularis\textunderscore )}
\end{itemize}
Relativo ao tempo da canícula; calmoso.
\section{Canículo}
\begin{itemize}
\item {Grp. gram.:m.}
\end{itemize}
Espécie de pano brasileiro.
\section{Canida}
\begin{itemize}
\item {Grp. gram.:f.}
\end{itemize}
O mesmo que \textunderscore caraipe\textunderscore .
\section{Canifraz}
\begin{itemize}
\item {Grp. gram.:m.  e  adj.}
\end{itemize}
\begin{itemize}
\item {Proveniência:(Do rad. do lat. \textunderscore canis\textunderscore )}
\end{itemize}
Homem magro como cão esfomeado.
\section{Canijo}
\begin{itemize}
\item {Grp. gram.:m.}
\end{itemize}
\begin{itemize}
\item {Utilização:Prov.}
\end{itemize}
\begin{itemize}
\item {Utilização:beir.}
\end{itemize}
O mesmo que \textunderscore canejo\textunderscore ^2.
\section{Canil}
\begin{itemize}
\item {Grp. gram.:m.}
\end{itemize}
\begin{itemize}
\item {Proveniência:(Do lat. \textunderscore canis\textunderscore )}
\end{itemize}
O mesmo que \textunderscore cangalho\textunderscore .
\section{Canil}
\begin{itemize}
\item {Grp. gram.:m.}
\end{itemize}
\begin{itemize}
\item {Proveniência:(De \textunderscore cana\textunderscore )}
\end{itemize}
Canela de perna do gado cavallar.
\section{Canil}
\begin{itemize}
\item {Grp. gram.:m.}
\end{itemize}
\begin{itemize}
\item {Proveniência:(Do lat. \textunderscore canis\textunderscore )}
\end{itemize}
Lugar ou casa, onde se guardam ou se criam cães domésticos ou cães de caça.
Casinhoto, construido para abrigo de cão.
\section{Canilha}
\begin{itemize}
\item {Grp. gram.:f.}
\end{itemize}
\begin{itemize}
\item {Proveniência:(De \textunderscore cana\textunderscore )}
\end{itemize}
Pequena haste metállica, em que se enrola o fio, com que trabalha a lançadeira.
\section{Canilho}
\begin{itemize}
\item {Grp. gram.:m.}
\end{itemize}
\begin{itemize}
\item {Utilização:Ant.}
\end{itemize}
(?):«\textunderscore canilhos frescos de empada\textunderscore ». G. Resende, \textunderscore Cancion.\textunderscore 
\section{Canimos}
\begin{itemize}
\item {Grp. gram.:m. pl.}
\end{itemize}
Tríbo africana, mencionada na \textunderscore Etiópia Or.\textunderscore , l. I, c. 1.
\section{Canina}
\begin{itemize}
\item {Grp. gram.:f.}
\end{itemize}
O mesmo que \textunderscore caninana\textunderscore .
\section{Caninamente}
\begin{itemize}
\item {Grp. gram.:adv.}
\end{itemize}
À maneira dos cães; de modo canino.
\section{Caninana}
\begin{itemize}
\item {Grp. gram.:f.}
\end{itemize}
\begin{itemize}
\item {Proveniência:(Do rad. lat. \textunderscore canis\textunderscore )}
\end{itemize}
Serpente do Brasil, (\textunderscore coluber paecillostoma\textunderscore ), que acommete, dando saltos e erguendo-se sobre a cauda.
Planta rubiácea do Brasil.
\section{Canindé}
\begin{itemize}
\item {Grp. gram.:m.}
\end{itemize}
\begin{itemize}
\item {Utilização:Bras}
\end{itemize}
Espécie de arara.
\section{Caninha}
\begin{itemize}
\item {Grp. gram.:f.}
\end{itemize}
\begin{itemize}
\item {Utilização:Bras}
\end{itemize}
\begin{itemize}
\item {Proveniência:(De \textunderscore cana\textunderscore )}
\end{itemize}
Aguardente de cana de açúcar.
Cana do açúcar.
\section{Caninha-verde}
\begin{itemize}
\item {Grp. gram.:f.}
\end{itemize}
Canção e dança popular do Minho.
\section{Canino}
\begin{itemize}
\item {Grp. gram.:adj.}
\end{itemize}
\begin{itemize}
\item {Proveniência:(Lat. \textunderscore caninus\textunderscore )}
\end{itemize}
Relativo a cão.
\textunderscore Dentes caninos\textunderscore , os que, no homem, estão collocados entre os incisivos e os molares.
\textunderscore Fome canina\textunderscore , fome insaciável, bulímia.
\section{Canipreto}
\begin{itemize}
\item {Grp. gram.:adj.}
\end{itemize}
\begin{itemize}
\item {Proveniência:(De \textunderscore cana\textunderscore  + \textunderscore preto\textunderscore )}
\end{itemize}
Que tem pretas as canelas das pernas.
\section{Canistrel}
\begin{itemize}
\item {Grp. gram.:m.}
\end{itemize}
\begin{itemize}
\item {Proveniência:(Do lat. \textunderscore canistrum\textunderscore )}
\end{itemize}
O mesmo que \textunderscore canastrel\textunderscore .
\section{Canistrel}
\begin{itemize}
\item {Grp. gram.:m.}
\end{itemize}
Espécie de pequena canastra com asa.
(B. lat. \textunderscore canistrellum\textunderscore )
\section{Canitar}
\begin{itemize}
\item {Grp. gram.:m.}
\end{itemize}
Pennacho de guerreiros tupis.
\section{Canito}
\begin{itemize}
\item {Grp. gram.:m.}
\end{itemize}
\begin{itemize}
\item {Utilização:Fam.}
\end{itemize}
Cão pequeno.
(Cp. lat. \textunderscore canis\textunderscore )
\section{Canito}
\begin{itemize}
\item {Grp. gram.:m.}
\end{itemize}
\begin{itemize}
\item {Proveniência:(De \textunderscore cana\textunderscore )}
\end{itemize}
Planta brasileira.
\section{Canivete}
\begin{itemize}
\item {Grp. gram.:m.}
\end{itemize}
Navalhinha, para aguçar lápis, cortar unhas, etc.
(Dem. do ant. al. \textunderscore knî\textunderscore )
\section{Canivete}
\begin{itemize}
\item {Grp. gram.:m.}
\end{itemize}
Espécie de papagaio das Antílhas.
\section{Canja}
\begin{itemize}
\item {Grp. gram.:f.}
\end{itemize}
\begin{itemize}
\item {Proveniência:(Do conc. \textunderscore kangi\textunderscore )}
\end{itemize}
Caldo de gallinha com arroz.
\section{Canjante}
\begin{itemize}
\item {Grp. gram.:adj.}
\end{itemize}
\begin{itemize}
\item {Utilização:Ant.}
\end{itemize}
O mesmo que \textunderscore cambiante\textunderscore .
(Cp. \textunderscore cambiante\textunderscore )
\section{Canjar}
\begin{itemize}
\item {Grp. gram.:v. i.}
\end{itemize}
\begin{itemize}
\item {Utilização:Ant.}
\end{itemize}
Cambiar; mudar de côr ou de rumo.
(Cp. \textunderscore cambiar\textunderscore )
\section{Cânjar}
\begin{itemize}
\item {Grp. gram.:m.}
\end{itemize}
\begin{itemize}
\item {Proveniência:(T. ár.)}
\end{itemize}
Espécie de punhal de comprida lâmina, afiada dos dois lados.
\section{Canjerana}
\begin{itemize}
\item {Grp. gram.:f.}
\end{itemize}
Planta meliácea, medicinal, do Brasil.
\section{Canjerê}
\begin{itemize}
\item {Grp. gram.:m.}
\end{itemize}
\begin{itemize}
\item {Utilização:Bras}
\end{itemize}
\begin{itemize}
\item {Proveniência:(T. afr.?)}
\end{itemize}
Conluio de escravos, para illudirem ingênuos, ganhando-lhes dinheiro por meio de práticas de feitiçaria.
\section{Canjiar}
\begin{itemize}
\item {Grp. gram.:m.}
\end{itemize}
(V.cânjar)
\section{Canjica}
\begin{itemize}
\item {Grp. gram.:f.}
\end{itemize}
\begin{itemize}
\item {Utilização:Bras}
\end{itemize}
\begin{itemize}
\item {Grp. gram.:Adj.}
\end{itemize}
\begin{itemize}
\item {Utilização:Bras. de San-Paulo}
\end{itemize}
Papas de milho moído, espécie de mingau.
O mesmo que \textunderscore mungunzá\textunderscore .
Espécie de rapé.
Saibro grosso, misturado com areia.
Aguardente de cana; cachaça.
Bebedeira.
Bêbedo. Camillo, \textunderscore Corja\textunderscore , 148.
Milho sêco, pilado ligeiramente, para que se lhe despegue o tegumento, e cozido em água, para sêr comido com açúcar, leite ou melado.
(Do quimb.)
\section{Canjurupi}
\begin{itemize}
\item {Grp. gram.:m.}
\end{itemize}
\begin{itemize}
\item {Utilização:Bras}
\end{itemize}
Peixe marítimo, ordinário.
\section{Canna}
\begin{itemize}
\item {Grp. gram.:f.}
\end{itemize}
\begin{itemize}
\item {Proveniência:(Lat. \textunderscore canna\textunderscore )}
\end{itemize}
Planta gramínea, de haste ôca nos entrenós.
Caule de várias plantas gramíneas.
A parte superior e lisa do caule do milho, desde o último nó á bandeira.
Osso, mais ou menos alongado, de certas partes do corpo humano: \textunderscore canna da perna\textunderscore ; \textunderscore canna do nariz.\textunderscore 
Designação de vários objectos alongados e cylíndricos, que dão ideia de uma cana.
Alavanca de pau, com que se governa o leme.
\section{Cannabina}
\begin{itemize}
\item {Grp. gram.:f.}
\end{itemize}
\begin{itemize}
\item {Proveniência:(Lat. \textunderscore cannalina\textunderscore )}
\end{itemize}
Medicamento, applicado contra as perdas de sangue, em seguida ao mênstruo.
\section{Cannabíneas}
\begin{itemize}
\item {Grp. gram.:f. pl.}
\end{itemize}
Família de plantas, que comprehende o cânhamo e o lúpulo.
(Cp. lat. \textunderscore cannabis\textunderscore )
\section{Cannáceas}
\begin{itemize}
\item {Grp. gram.:f. pl.}
\end{itemize}
Família de plantas, a que serve de typo a canna da Índia.
\section{Cannibal}
\begin{itemize}
\item {Grp. gram.:m.}
\end{itemize}
\begin{itemize}
\item {Proveniência:(De \textunderscore canniba\textunderscore , t. americano)}
\end{itemize}
Selvagem, que come carne humana.
Homem feroz.
\section{Cannibalesco}
\begin{itemize}
\item {Grp. gram.:adj.}
\end{itemize}
Próprio de cannibal.
\section{Cannibalismo}
\begin{itemize}
\item {Grp. gram.:m.}
\end{itemize}
Qualidade de cannibal.
\section{Cano}
\begin{itemize}
\item {Grp. gram.:m.}
\end{itemize}
\begin{itemize}
\item {Utilização:Prov.}
\end{itemize}
\begin{itemize}
\item {Utilização:minh.}
\end{itemize}
\begin{itemize}
\item {Utilização:T. de Aveiro}
\end{itemize}
\begin{itemize}
\item {Proveniência:(Do rad. de \textunderscore cana\textunderscore )}
\end{itemize}
Tubo, próprio para conduzir líquidos ou gases.
Construcção tubular subterrânea, para conducção de água, de gás, ou de dejectos.
Tubo cylíndrico da espingarda, onde se encerra a carga e donde ella se expelle.
Tubo circular ou angular, para ventilação e tiragem do fumo das chaminés.
Nome de variadissimos objectos de fórma tubular.
Ramo horizontal de uma árvore:«\textunderscore no cano de alguma figueira.\textunderscore »\textunderscore Arco de Sant'Anna\textunderscore , II, 117.
Punho do remo.
\section{Cano}
\begin{itemize}
\item {Grp. gram.:adj.}
\end{itemize}
\begin{itemize}
\item {Utilização:P. us.}
\end{itemize}
\begin{itemize}
\item {Proveniência:(Lat. \textunderscore canus\textunderscore )}
\end{itemize}
Branco.
\section{Cano}
\begin{itemize}
\item {Grp. gram.:m.}
\end{itemize}
\begin{itemize}
\item {Proveniência:(Lat. \textunderscore canon\textunderscore )}
\end{itemize}
Nome, que na Idade-Média, se deu ao saltério.
\section{Canôa}
\begin{itemize}
\item {Grp. gram.:f.}
\end{itemize}
\begin{itemize}
\item {Utilização:Bras}
\end{itemize}
Pequena embarcação.
Frigideira, em fórma de canôa.
Banheira comprida.
Antigo pente de ornato para senhoras.
Conductos, abertos e inclinados, nos trabalhos da mineração do oiro.
(Cast. \textunderscore canoa\textunderscore , do germ.)
\section{Canôa-da-picada}
\begin{itemize}
\item {Grp. gram.:f.}
\end{itemize}
Embarcação, o mesmo que \textunderscore enviada\textunderscore .
\section{Canoão}
\begin{itemize}
\item {Grp. gram.:m.}
\end{itemize}
\begin{itemize}
\item {Utilização:Bras}
\end{itemize}
Canôa grande.
\section{Canoata}
\begin{itemize}
\item {Grp. gram.:m.}
\end{itemize}
\begin{itemize}
\item {Utilização:Ant.}
\end{itemize}
Imposto, que se pagava ao catual de Baçaim.
\section{Canoco}
\begin{itemize}
\item {fónica:nô}
\end{itemize}
\begin{itemize}
\item {Grp. gram.:adj.}
\end{itemize}
\begin{itemize}
\item {Proveniência:(De \textunderscore cano\textunderscore ^1)}
\end{itemize}
Diz-se de uma variedade de trigo.
\section{Canoco}
\begin{itemize}
\item {fónica:nô}
\end{itemize}
\begin{itemize}
\item {Grp. gram.:m.}
\end{itemize}
\begin{itemize}
\item {Utilização:Prov.}
\end{itemize}
Grande pedaço de pão.
\section{Canóculo}
\begin{itemize}
\item {Grp. gram.:m.}
\end{itemize}
\begin{itemize}
\item {Proveniência:(De \textunderscore cano\textunderscore ^1 + \textunderscore óculo\textunderscore )}
\end{itemize}
Óculo de vêr ao longe.
\section{Canoeiro}
\begin{itemize}
\item {Grp. gram.:m.}
\end{itemize}
Aquelle que dirige uma canôa.
\section{Canoeiros}
\begin{itemize}
\item {Grp. gram.:m. pl.}
\end{itemize}
Tríbo crudelíssima de Goiás.
\section{Canoila}
\begin{itemize}
\item {Grp. gram.:f.}
\end{itemize}
\begin{itemize}
\item {Proveniência:(De \textunderscore cana\textunderscore )}
\end{itemize}
Haste do milho.
Emblema heráldico, que tem a fórma dessa haste.
\section{Canoilo}
\begin{itemize}
\item {Grp. gram.:m.}
\end{itemize}
O mesmo que \textunderscore canoila\textunderscore .
\section{Canoira}
\begin{itemize}
\item {Grp. gram.:f.}
\end{itemize}
\begin{itemize}
\item {Proveniência:(De \textunderscore cano\textunderscore )}
\end{itemize}
Vaso quadrado, collocado acima da mó do moínho, e donde cai o grão que vai moêr-se; moéga; tremonha.
\section{Cânon}
\begin{itemize}
\item {Grp. gram.:m.}
\end{itemize}
\begin{itemize}
\item {Utilização:Mús.}
\end{itemize}
\begin{itemize}
\item {Utilização:ant.}
\end{itemize}
\begin{itemize}
\item {Grp. gram.:M. pl.}
\end{itemize}
\begin{itemize}
\item {Proveniência:(Lat. \textunderscore canon\textunderscore )}
\end{itemize}
Regra.
Decisão de concílio.
Catálogo, relação.
Fórmula de orações.
Quadro, tabella.
Fôro.
Composição musical, cujo thema, iniciado por uma voz, era rigorosamente imitado por outra ou outras vozes até o fim.
Antiga faculdade na Universidade de Coímbra.
\section{Cânone}
\begin{itemize}
\item {Grp. gram.:m.}
\end{itemize}
\begin{itemize}
\item {Utilização:Mús.}
\end{itemize}
\begin{itemize}
\item {Utilização:ant.}
\end{itemize}
\begin{itemize}
\item {Grp. gram.:M. pl.}
\end{itemize}
\begin{itemize}
\item {Proveniência:(Lat. \textunderscore canon\textunderscore )}
\end{itemize}
Regra.
Decisão de concílio.
Catálogo, relação.
Fórmula de orações.
Quadro, tabella.
Fôro.
Composição musical, cujo thema, iniciado por uma voz, era rigorosamente imitado por outra ou outras vozes até o fim.
Antiga faculdade na Universidade de Coímbra.
\section{Canonical}
\begin{itemize}
\item {Grp. gram.:adj.}
\end{itemize}
\begin{itemize}
\item {Proveniência:(De \textunderscore canónico\textunderscore )}
\end{itemize}
Relativo a cónegos.
\section{Canonicamente}
\begin{itemize}
\item {Grp. gram.:adv.}
\end{itemize}
De modo \textunderscore canónico\textunderscore ; segundo os cânones.
\section{Canonicato}
\begin{itemize}
\item {Grp. gram.:m.}
\end{itemize}
Dignidade de cónego.
(B. lat. \textunderscore canonicatus\textunderscore )
\section{Canonicidade}
\begin{itemize}
\item {Grp. gram.:f.}
\end{itemize}
Qualidade do que é canónico.
\section{Canónico}
\begin{itemize}
\item {Grp. gram.:adj.}
\end{itemize}
\begin{itemize}
\item {Grp. gram.:M.}
\end{itemize}
\begin{itemize}
\item {Utilização:Bras}
\end{itemize}
\begin{itemize}
\item {Proveniência:(Lat. \textunderscore canonicus\textunderscore )}
\end{itemize}
Relativo a cânones.
Que é conforme aos cânones da Igreja ou approvado por ella.
Ave, o mesmo que \textunderscore dom-fafe\textunderscore .
\section{Canonista}
\begin{itemize}
\item {Grp. gram.:m.}
\end{itemize}
\begin{itemize}
\item {Proveniência:(De \textunderscore cânon\textunderscore )}
\end{itemize}
Aquelle que é versado nos cânones da Igreja.
\section{Canoniza}
\begin{itemize}
\item {Grp. gram.:f.}
\end{itemize}
\begin{itemize}
\item {Proveniência:(Do lat. \textunderscore canonicus\textunderscore )}
\end{itemize}
Mulher, com dignidade correspondente á de cónego.
\section{Canonização}
\begin{itemize}
\item {Grp. gram.:f.}
\end{itemize}
Acto de \textunderscore canonizar\textunderscore .
\section{Canonizador}
\begin{itemize}
\item {Grp. gram.:m.}
\end{itemize}
Aquelle que canoniza.
\section{Canonizar}
\begin{itemize}
\item {Grp. gram.:v. t.}
\end{itemize}
\begin{itemize}
\item {Utilização:Fig.}
\end{itemize}
\begin{itemize}
\item {Proveniência:(Do gr. \textunderscore kanonizein\textunderscore )}
\end{itemize}
Declarar santo, inscrever no rol dos santos.
Elogiar excessivamente.
\section{Canonizável}
\begin{itemize}
\item {Grp. gram.:adj.}
\end{itemize}
Que merece sêr canonizado.
\section{Canopeia}
\begin{itemize}
\item {Grp. gram.:f.}
\end{itemize}
O mesmo que \textunderscore canopo\textunderscore .
\section{Canopi}
\begin{itemize}
\item {Grp. gram.:m.}
\end{itemize}
Árvore sapindácea do Brasil.
\section{Canopo}
\begin{itemize}
\item {fónica:nô}
\end{itemize}
\begin{itemize}
\item {Grp. gram.:m.}
\end{itemize}
\begin{itemize}
\item {Proveniência:(Gr. \textunderscore Kanopos\textunderscore , n. p.)}
\end{itemize}
Grande estrêlla da constellação Argos.
\section{Canoramente}
\begin{itemize}
\item {Grp. gram.:adv.}
\end{itemize}
De modo canoro.
Harmoniosamente.
\section{Canorço}
\begin{itemize}
\item {Grp. gram.:adj.}
\end{itemize}
\begin{itemize}
\item {Utilização:Prov.}
\end{itemize}
\begin{itemize}
\item {Utilização:alent.}
\end{itemize}
\begin{itemize}
\item {Proveniência:(Do lat. \textunderscore canis\textunderscore , cão)}
\end{itemize}
Velho e escanzelado.
\section{Canório}
\begin{itemize}
\item {Grp. gram.:m.}
\end{itemize}
\begin{itemize}
\item {Utilização:Prov.}
\end{itemize}
\begin{itemize}
\item {Utilização:trasm.}
\end{itemize}
\begin{itemize}
\item {Proveniência:(De \textunderscore cana\textunderscore )}
\end{itemize}
A cana do milho, que fica na terra depois de tirada a bandeira e a espiga.
\section{Canoro}
\begin{itemize}
\item {Grp. gram.:adj.}
\end{itemize}
\begin{itemize}
\item {Proveniência:(Lat. \textunderscore canorus\textunderscore )}
\end{itemize}
Que canta harmoniosamente; harmonioso.
\section{Canossiano}
\begin{itemize}
\item {Grp. gram.:adj.}
\end{itemize}
\begin{itemize}
\item {Proveniência:(De \textunderscore Canossa\textunderscore , n. p.)}
\end{itemize}
Diz-se de uma communidade, a que pertencem certas religiosas, que fazem parte do quadro da missão em Timor.
\section{Canotaria}
\begin{itemize}
\item {Grp. gram.:f.}
\end{itemize}
\begin{itemize}
\item {Utilização:Mús.}
\end{itemize}
O mesmo que \textunderscore canaría\textunderscore .
\section{Canotilho}
\begin{itemize}
\item {Grp. gram.:m.}
\end{itemize}
(V.canutilho)
\section{Canoura}
\begin{itemize}
\item {Grp. gram.:f.}
\end{itemize}
\begin{itemize}
\item {Proveniência:(De \textunderscore cano\textunderscore )}
\end{itemize}
Vaso quadrado, collocado acima da mó do moínho, e donde cai o grão que vai moêr-se; moéga; tremonha.
\section{Cansaço}
\begin{itemize}
\item {Grp. gram.:m.}
\end{itemize}
\begin{itemize}
\item {Utilização:Bras}
\end{itemize}
\begin{itemize}
\item {Proveniência:(De \textunderscore cansar\textunderscore )}
\end{itemize}
Fadiga, fraqueza, produzida por exercício demasiado ou doença.
Hydropisia.
\section{Cansadamente}
\begin{itemize}
\item {Grp. gram.:adv.}
\end{itemize}
\begin{itemize}
\item {Proveniência:(De \textunderscore cansado\textunderscore )}
\end{itemize}
Com cansaço.
\section{Cansadez}
\begin{itemize}
\item {Grp. gram.:f.}
\end{itemize}
\begin{itemize}
\item {Utilização:Ant.}
\end{itemize}
O mesmo que \textunderscore cansaço\textunderscore .
\section{Cansado}
\begin{itemize}
\item {Grp. gram.:adj.}
\end{itemize}
\begin{itemize}
\item {Utilização:Bras. de Minas}
\end{itemize}
Fatigado.
Enfraquecido, extenuado.
Vago, indefinido: \textunderscore uma dôr cansada\textunderscore .
\section{Cansamento}
\begin{itemize}
\item {Grp. gram.:m.}
\end{itemize}
O mesmo que \textunderscore cansaço\textunderscore . Cf. D. Bernardes, \textunderscore Lima\textunderscore , 107.
\section{Cansancão}
\begin{itemize}
\item {Grp. gram.:m.}
\end{itemize}
Planta urticácea do Brasil.
\section{Cansar}
\begin{itemize}
\item {Grp. gram.:v. t.}
\end{itemize}
\begin{itemize}
\item {Grp. gram.:V. t.}
\end{itemize}
\begin{itemize}
\item {Proveniência:(Lat. \textunderscore quassare\textunderscore )}
\end{itemize}
Causar cansaço a.
Enfastiar, importunar.
Estar cansado.
\section{Cansativo}
\begin{itemize}
\item {Grp. gram.:adj.}
\end{itemize}
\begin{itemize}
\item {Proveniência:(De \textunderscore cansar\textunderscore )}
\end{itemize}
Que fatiga.
\section{Cansável}
\begin{itemize}
\item {Grp. gram.:adj.}
\end{itemize}
Susceptível de cansar-se.
\section{Canseira}
\begin{itemize}
\item {Grp. gram.:f.}
\end{itemize}
O mesmo que \textunderscore cansaço\textunderscore .
\section{Canseirosamente}
\begin{itemize}
\item {Grp. gram.:adv.}
\end{itemize}
Com canseira.
\section{Canseiroso}
\begin{itemize}
\item {Grp. gram.:adj.}
\end{itemize}
Que tem canseira.
Que se dedica afanosamente a certos trabalhos. Cf. \textunderscore Techn. Rur.\textunderscore , 536.
\section{Canseirudo}
\begin{itemize}
\item {Grp. gram.:adj.}
\end{itemize}
\begin{itemize}
\item {Utilização:Prov.}
\end{itemize}
O mesmo que \textunderscore canseiroso\textunderscore .
\section{Cantá}
\begin{itemize}
\item {Grp. gram.:adv.}
\end{itemize}
\begin{itemize}
\item {Utilização:Ant.}
\end{itemize}
\begin{itemize}
\item {Proveniência:(De \textunderscore ca\textunderscore  == \textunderscore que\textunderscore , porquê, e \textunderscore antá\textunderscore , == \textunderscore até\textunderscore )}
\end{itemize}
Até; cantés:«\textunderscore cantá pela secada hum manco fará isso\textunderscore ». G. Vicente, \textunderscore Auto da Hist. de Deus\textunderscore .
\section{Cantabanco}
\begin{itemize}
\item {Grp. gram.:m.}
\end{itemize}
\begin{itemize}
\item {Utilização:Ant.}
\end{itemize}
\begin{itemize}
\item {Proveniência:(De \textunderscore cantar\textunderscore  + \textunderscore banco\textunderscore )}
\end{itemize}
Cantor de feira ou de praça pública.
\section{Cantábrico}
\begin{itemize}
\item {Grp. gram.:adj.}
\end{itemize}
Relativo á Cantábria.
Vasconço; basco; biscaínho.
\section{Cantábrio}
\begin{itemize}
\item {Grp. gram.:adj.}
\end{itemize}
O mesmo que \textunderscore cantábrico\textunderscore .
\section{Cántabro}
\begin{itemize}
\item {Grp. gram.:adj.}
\end{itemize}
O mesmo que \textunderscore cantábrico\textunderscore .
\section{Cantadeira}
\begin{itemize}
\item {Grp. gram.:adj. f.}
\end{itemize}
\begin{itemize}
\item {Grp. gram.:F.}
\end{itemize}
\begin{itemize}
\item {Utilização:Prov.}
\end{itemize}
\begin{itemize}
\item {Utilização:beir.}
\end{itemize}
Que canta muito.
Mulher, que canta.
O mesmo que \textunderscore cantadoira\textunderscore .
O mesmo que \textunderscore rangedeira\textunderscore , ave.
Espécie de chapuz, sôbre o cocão do carro e no qual se cravam as treitoeiras, atravessando-o, juntamente com o cocão e o taboleiro.
\section{Cantadeiro}
\begin{itemize}
\item {Grp. gram.:m.}
\end{itemize}
\begin{itemize}
\item {Proveniência:(De \textunderscore cantar\textunderscore )}
\end{itemize}
Cantador popular; cantador de arraiaes ou de bailaricos.
\section{Cantadela}
\begin{itemize}
\item {Grp. gram.:f.}
\end{itemize}
\begin{itemize}
\item {Utilização:Pop.}
\end{itemize}
(V.cantiga)
\section{Cantado}
\begin{itemize}
\item {Grp. gram.:adj.}
\end{itemize}
\begin{itemize}
\item {Proveniência:(Lat. \textunderscore cantatus\textunderscore )}
\end{itemize}
Expresso por música ou por cantochão: \textunderscore Missa cantada\textunderscore .
\section{Cantadoira}
\begin{itemize}
\item {Grp. gram.:f.}
\end{itemize}
\begin{itemize}
\item {Proveniência:(De \textunderscore cantar\textunderscore )}
\end{itemize}
Cada um dos paus verticaes que, atravessando o chedeiro, abraçam o eixo do carro, aos lados do cocão.
\section{Cantador}
\begin{itemize}
\item {Grp. gram.:m.  e  adj.}
\end{itemize}
Aquelle que canta.
\section{Cantadoura}
\begin{itemize}
\item {Grp. gram.:f.}
\end{itemize}
\begin{itemize}
\item {Proveniência:(De \textunderscore cantar\textunderscore )}
\end{itemize}
Cada um dos paus verticaes que, atravessando o chedeiro, abraçam o eixo do carro, aos lados do cocão.
\section{Cantante}
\begin{itemize}
\item {Grp. gram.:adj.}
\end{itemize}
\begin{itemize}
\item {Grp. gram.:M.}
\end{itemize}
\begin{itemize}
\item {Utilização:Gír.}
\end{itemize}
\begin{itemize}
\item {Proveniência:(Lat. \textunderscore cantans\textunderscore )}
\end{itemize}
Que canta.
Que é próprio para se cantar: \textunderscore a parte cantante de um hymno\textunderscore .
Relógio.
\section{Cantão}
\begin{itemize}
\item {Grp. gram.:m.}
\end{itemize}
\begin{itemize}
\item {Utilização:Heráld.}
\end{itemize}
\begin{itemize}
\item {Proveniência:(Fr. \textunderscore canton\textunderscore . Cp. \textunderscore canto\textunderscore ^1)}
\end{itemize}
Cada uma das divisões territoriaes de vários países.
Secção de estrada, cuja conservação e guarda está a cargo de um operário.
Cada um dos quatro cantos do escudo, separados pelos braços da cruz firmada.
\section{Cantar}
\begin{itemize}
\item {Grp. gram.:v. t.}
\end{itemize}
\begin{itemize}
\item {Grp. gram.:V. i.}
\end{itemize}
\begin{itemize}
\item {Utilização:Pop.}
\end{itemize}
\begin{itemize}
\item {Utilização:Gír.}
\end{itemize}
\begin{itemize}
\item {Grp. gram.:M.}
\end{itemize}
\begin{itemize}
\item {Proveniência:(Lat. \textunderscore cantare\textunderscore )}
\end{itemize}
Exprimir por meio de canto.
Celebrar em verso: \textunderscore cantar as armas e os barões assinalados\textunderscore .
Cadenciar.
Soltar canto, sons musicaes, com a voz.
Produzir sons cadenciados.
Retrucar, replicar com energia: \textunderscore accusou-me de imbecil, mas eu cantei-lhe\textunderscore .
Padecer.
O mesmo que \textunderscore cântico\textunderscore .
Trova; canção popular: \textunderscore os cantares do Minho\textunderscore .
\section{Cântara}
\begin{itemize}
\item {Grp. gram.:f.}
\end{itemize}
Pequeno cântaro.
\section{Cantareira}
\begin{itemize}
\item {Grp. gram.:f.}
\end{itemize}
\begin{itemize}
\item {Utilização:Prov.}
\end{itemize}
\begin{itemize}
\item {Utilização:Bras}
\end{itemize}
\begin{itemize}
\item {Proveniência:(De \textunderscore cântaro\textunderscore )}
\end{itemize}
Poial para cântaros.
O mesmo que \textunderscore prateleira\textunderscore .
O mesmo que \textunderscore clavícula\textunderscore , (osso).
\section{Cantarejar}
\begin{itemize}
\item {Grp. gram.:v. i.}
\end{itemize}
\begin{itemize}
\item {Utilização:Pop.}
\end{itemize}
(V.cantarolar)
\section{Cantarejo}
\begin{itemize}
\item {Grp. gram.:m.}
\end{itemize}
(V.cantarola)
\section{Cantaria}
\begin{itemize}
\item {Grp. gram.:f.}
\end{itemize}
\begin{itemize}
\item {Proveniência:(De \textunderscore canto\textunderscore ^1)}
\end{itemize}
Pedra rija, esquadrada para construcções.
\section{Cantárida}
\begin{itemize}
\item {Grp. gram.:f.}
\end{itemize}
\begin{itemize}
\item {Proveniência:(Gr. \textunderscore kantharis\textunderscore )}
\end{itemize}
Insecto trachelíneo, que, reduzido a pó, tem numerosas applicações medicinaes.
\section{Cantaridal}
\begin{itemize}
\item {Grp. gram.:adj.}
\end{itemize}
Relativo a \textunderscore cantárida\textunderscore .
\section{Cantaridar}
\begin{itemize}
\item {Grp. gram.:v. t.}
\end{itemize}
Cobrir com pó de cantáridas.
\section{Cantaridina}
\begin{itemize}
\item {Grp. gram.:f.}
\end{itemize}
\begin{itemize}
\item {Proveniência:(De \textunderscore canthárida\textunderscore )}
\end{itemize}
Princípio, a que a cantárida deve as suas propriedades epispáticas.
\section{Cantarilho}
\begin{itemize}
\item {Grp. gram.:m.}
\end{itemize}
Pequeno peixe vermelho, com espinhos nas barbatanas e cabeça óssea parecida á do ruivo.
\section{Cantarilho}
\begin{itemize}
\item {Grp. gram.:m.}
\end{itemize}
\begin{itemize}
\item {Proveniência:(De \textunderscore cantar\textunderscore )}
\end{itemize}
Nome de certas canções amorosas, entre os antigos trovadores portugueses.
\section{Cantarina}
\begin{itemize}
\item {Grp. gram.:f.}
\end{itemize}
\begin{itemize}
\item {Utilização:Des.}
\end{itemize}
\begin{itemize}
\item {Proveniência:(De \textunderscore cantar\textunderscore )}
\end{itemize}
Cantora.
\section{Cantarinha}
\begin{itemize}
\item {Grp. gram.:f.}
\end{itemize}
\begin{itemize}
\item {Utilização:Prov.}
\end{itemize}
\begin{itemize}
\item {Utilização:trasm.}
\end{itemize}
\begin{itemize}
\item {Proveniência:(De \textunderscore cântara\textunderscore )}
\end{itemize}
Bôlha de agua.
\section{Cantarinhas}
\begin{itemize}
\item {Grp. gram.:f. pl.}
\end{itemize}
\begin{itemize}
\item {Utilização:Prov.}
\end{itemize}
\begin{itemize}
\item {Utilização:alent.}
\end{itemize}
\begin{itemize}
\item {Proveniência:(De \textunderscore cantarinha\textunderscore ? ou de \textunderscore cantar\textunderscore ? Cp. \textunderscore cantarina\textunderscore )}
\end{itemize}
Instrumentos de barro, que, nos moínhos de vento, assobiam, em-quanto giram as velas.
\section{Cantarino}
\begin{itemize}
\item {Grp. gram.:m.}
\end{itemize}
\begin{itemize}
\item {Utilização:Des.}
\end{itemize}
Cantor. Cf. Filinto, XIII, 25.
\section{Cântaro}
\begin{itemize}
\item {Grp. gram.:m.}
\end{itemize}
\begin{itemize}
\item {Utilização:Prov.}
\end{itemize}
\begin{itemize}
\item {Utilização:Pop.}
\end{itemize}
\begin{itemize}
\item {Utilização:Prov.}
\end{itemize}
\begin{itemize}
\item {Utilização:alent.}
\end{itemize}
\begin{itemize}
\item {Proveniência:(Lat. \textunderscore cantharus\textunderscore )}
\end{itemize}
Vaso grande e bojudo, de barro ou fôlha, para líquidos.
O mesmo que \textunderscore almude\textunderscore .
Meio almude.
Peixe dos Açores.
\textunderscore Alma de cântaro\textunderscore  ou \textunderscore alma de canto\textunderscore , alma empedernida, insensível; pessôa ruím.
Jôgo de rapazes.
\section{Cantarola}
\begin{itemize}
\item {Grp. gram.:f.}
\end{itemize}
\begin{itemize}
\item {Proveniência:(De \textunderscore cantarolar\textunderscore )}
\end{itemize}
Canto em voz baixa.
Cantiga desentoada.
\section{Cantarolar}
\begin{itemize}
\item {Grp. gram.:v. t.  e  i.}
\end{itemize}
\begin{itemize}
\item {Proveniência:(De \textunderscore cantar\textunderscore )}
\end{itemize}
Cantar desafinadamente.
Cantar em voz baixa; trautear.
\section{Cantarolável}
\begin{itemize}
\item {Grp. gram.:adj.}
\end{itemize}
Que se póde cantarolar.
\section{Cantata}
\begin{itemize}
\item {Grp. gram.:f.}
\end{itemize}
\begin{itemize}
\item {Utilização:Pop.}
\end{itemize}
\begin{itemize}
\item {Proveniência:(It. \textunderscore cantata\textunderscore )}
\end{itemize}
Antiga fórma de poema lýrico.
Composição poética, posta em música.
Opereta.
Lábia, palavreado astucioso.
\section{Cantate}
\begin{itemize}
\item {Grp. gram.:m.}
\end{itemize}
\begin{itemize}
\item {Utilização:Fam.}
\end{itemize}
\begin{itemize}
\item {Proveniência:(Lat. \textunderscore cantate\textunderscore , imper. de \textunderscore cantare\textunderscore )}
\end{itemize}
Pechincha.
Grande prazer; festança. Cf. Castilho, \textunderscore Fausto\textunderscore , 347.
\section{Cantatriz}
\begin{itemize}
\item {Grp. gram.:f.}
\end{itemize}
\begin{itemize}
\item {Utilização:Des.}
\end{itemize}
\begin{itemize}
\item {Proveniência:(Lat. \textunderscore cantatrix\textunderscore )}
\end{itemize}
Cantora.
\section{Cantável}
\begin{itemize}
\item {Grp. gram.:adj.}
\end{itemize}
Que se póde cantar.
\section{Cante}
\begin{itemize}
\item {Grp. gram.:m.}
\end{itemize}
\begin{itemize}
\item {Utilização:T. da Nazaré}
\end{itemize}
Acto de \textunderscore cantar\textunderscore ; canto.
(Cp. \textunderscore descante\textunderscore , e cast. \textunderscore cante\textunderscore )
\section{Canté!}
\begin{itemize}
\item {Grp. gram.:interj.}
\end{itemize}
\begin{itemize}
\item {Utilização:Prov.}
\end{itemize}
O mesmo que \textunderscore cantés!\textunderscore 
(Cp. \textunderscore cantá\textunderscore )
\section{Canteira}
\begin{itemize}
\item {Grp. gram.:f.}
\end{itemize}
\begin{itemize}
\item {Proveniência:(De \textunderscore canto\textunderscore ^1)}
\end{itemize}
Pedreira, donde se tira pedra para cantaria.
\section{Canteira}
\begin{itemize}
\item {Grp. gram.:f.}
\end{itemize}
\begin{itemize}
\item {Utilização:Prov.}
\end{itemize}
\begin{itemize}
\item {Utilização:trasm.}
\end{itemize}
Gato de ferro, para não deixar abrir a junta de duas tábuas.
\section{Canteiro}
\begin{itemize}
\item {Grp. gram.:m.}
\end{itemize}
\begin{itemize}
\item {Proveniência:(De \textunderscore canto\textunderscore ^1)}
\end{itemize}
Homem, que trabalha em cantaria; marmoreiro.
Esculptor em pedra.
O mesmo que \textunderscore alegrete\textunderscore ^1; peça de terreno ajardinado.
O mesmo que \textunderscore baixete\textunderscore .
\section{Cantés!}
\begin{itemize}
\item {Grp. gram.:interj.}
\end{itemize}
\begin{itemize}
\item {Utilização:Prov.}
\end{itemize}
\begin{itemize}
\item {Utilização:Ant.}
\end{itemize}
Certamente! forte dúvida! quem déra!
(Cp. \textunderscore cantá\textunderscore )
\section{Canteu!}
\begin{itemize}
\item {Grp. gram.:interj.}
\end{itemize}
O mesmo que \textunderscore cantés!\textunderscore  Cf. \textunderscore Eufrosina\textunderscore , 160 e 208.
\section{Canthárida}
\begin{itemize}
\item {Grp. gram.:f.}
\end{itemize}
\begin{itemize}
\item {Proveniência:(Gr. \textunderscore kantharis\textunderscore )}
\end{itemize}
Insecto trachelíneo, que, reduzido a pó, tem numerosas applicações medicinaes.
\section{Cantharidal}
\begin{itemize}
\item {Grp. gram.:adj.}
\end{itemize}
Relativo a \textunderscore canthárida\textunderscore .
\section{Cantharidar}
\begin{itemize}
\item {Grp. gram.:v. t.}
\end{itemize}
Cobrir com pó de cantháridas.
\section{Cantharidina}
\begin{itemize}
\item {Grp. gram.:f.}
\end{itemize}
\begin{itemize}
\item {Proveniência:(De \textunderscore canthárida\textunderscore )}
\end{itemize}
Princípio, a que a canthárida deve as suas propriedades epispáticas.
\section{Canthoplastia}
\begin{itemize}
\item {Grp. gram.:f.}
\end{itemize}
\begin{itemize}
\item {Proveniência:(Do gr. \textunderscore kanthos\textunderscore  + \textunderscore plassein\textunderscore )}
\end{itemize}
Operação cirúrgica, com que se reforma o canto do ôlho.
\section{Cântica}
\begin{itemize}
\item {Grp. gram.:f.}
\end{itemize}
\begin{itemize}
\item {Utilização:Ant.}
\end{itemize}
O mesmo que \textunderscore canto\textunderscore ^2 ou \textunderscore cântico\textunderscore .
\section{Cântico}
\begin{itemize}
\item {Grp. gram.:m.}
\end{itemize}
\begin{itemize}
\item {Proveniência:(Lat. \textunderscore canticum\textunderscore )}
\end{itemize}
Canto em honra da Divindade.
Hymno; canto.
\section{Cantiga}
\begin{itemize}
\item {Grp. gram.:f.}
\end{itemize}
\begin{itemize}
\item {Utilização:Pop.}
\end{itemize}
Poesia cantada, formada de redondilhas ou de versos mais pequenos que a redondilha, e dividida geralmente em estrophes iguaes.
Narração ou conversa astuciosa, para illudir.
(Se não do lat. \textunderscore cantica\textunderscore , pl. de \textunderscore canticum\textunderscore , com deslocação de acento, viria de \textunderscore cantigo\textunderscore , subst. verb. do hyp. \textunderscore cantigar\textunderscore , ou do lat. hyp. \textunderscore canticula\textunderscore , pl. de \textunderscore canticulum\textunderscore )
\section{Cântigo}
\begin{itemize}
\item {Grp. gram.:m.}
\end{itemize}
\begin{itemize}
\item {Utilização:T. de Carregosa}
\end{itemize}
\begin{itemize}
\item {Proveniência:(Lat. \textunderscore canticum\textunderscore )}
\end{itemize}
O mesmo que \textunderscore cântico\textunderscore .
\section{Cantil}
\begin{itemize}
\item {Grp. gram.:m.}
\end{itemize}
\begin{itemize}
\item {Proveniência:(De \textunderscore canto\textunderscore ^1)}
\end{itemize}
Instrumento de carpintaria, com que se esquadram as tábuas para se ajustarem os seus lados.
Instrumento de esculptor, para alisar pedras.
Frasco, pequeno vaso de madeira ou vidro empalhado, para transporte de líquidos em viagem.
\section{Cantilena}
\begin{itemize}
\item {Grp. gram.:f.}
\end{itemize}
\begin{itemize}
\item {Proveniência:(Lat. \textunderscore cantilena\textunderscore )}
\end{itemize}
Pequena canção; cantiga.
\section{Cantimarão}
\begin{itemize}
\item {Grp. gram.:m.}
\end{itemize}
Barco de pesca, na costa de Coromandel.
\section{Cantimplora}
\begin{itemize}
\item {Grp. gram.:f.}
\end{itemize}
\begin{itemize}
\item {Utilização:Prov.}
\end{itemize}
\begin{itemize}
\item {Utilização:Baía.}
\end{itemize}
\begin{itemize}
\item {Proveniência:(It. \textunderscore cantimplora\textunderscore )}
\end{itemize}
Siphão, para trasfegar líquidos.
Vaso de metal, para resfriar água.
Almotolia, que deita azeite por um canudo estreito e comprido.
Regador de jardim.
Bueiro.
Sorveteira.
\section{Cantina}
\begin{itemize}
\item {Grp. gram.:f.}
\end{itemize}
Taberna em quartéis, acampamentos, arraiaes, etc.
Instituição popular, que ministra certas refeições e outros auxílios a crianças pobres.
(Cast. \textunderscore cantina\textunderscore )
\section{Cantineiro}
\begin{itemize}
\item {Grp. gram.:m.}
\end{itemize}
Aquelle que tem cantina ou vende nella.
\section{Cantinhos}
\begin{itemize}
\item {Grp. gram.:m. pl.}
\end{itemize}
\begin{itemize}
\item {Proveniência:(De \textunderscore canto\textunderscore ^1)}
\end{itemize}
Espécie de jôgo de crianças.
\section{Cantiplora}
\begin{itemize}
\item {Grp. gram.:f.}
\end{itemize}
(Vid. \textunderscore cantimplora\textunderscore )
\section{Canto}
\begin{itemize}
\item {Grp. gram.:m.}
\end{itemize}
\begin{itemize}
\item {Utilização:Carp.}
\end{itemize}
\begin{itemize}
\item {Proveniência:(De um rad. \textunderscore cant\textunderscore , muito espalhado em quási todas as línguas cultas)}
\end{itemize}
Ângulo saliente ou reentrante, formado pelo encontro de linhas ou superfícies.
Sítio esconso.
Lugar afastado e pouco frequentado.
Lado do pão.
Ângulo (do ôlho, da bôca, etc.).
Pedra; esquadria de pedra.
O mesmo que junta ou aresta de uma tábua.
\section{Canto}
\begin{itemize}
\item {Grp. gram.:m.}
\end{itemize}
\begin{itemize}
\item {Proveniência:(Lat. \textunderscore cantus\textunderscore )}
\end{itemize}
Série de sons musicaes, formados pela voz.
Acção de cantar.
Série de sons cadenciados.
Música.
Poesia cantável.
Divisão de certos poêmas, especialmente dos épicos.
\textunderscore Canto feito\textunderscore , harmonia, escrita sôbre cantochão.
\section{Cantochanista}
\begin{itemize}
\item {Grp. gram.:m.}
\end{itemize}
Cantor de cantochão.
\section{Cantochão}
\begin{itemize}
\item {Grp. gram.:m.}
\end{itemize}
\begin{itemize}
\item {Proveniência:(De \textunderscore canto\textunderscore  + \textunderscore chão\textunderscore )}
\end{itemize}
Canto ordinário da Igreja.
\section{Cantoeira}
\begin{itemize}
\item {Grp. gram.:f.}
\end{itemize}
\begin{itemize}
\item {Proveniência:(De \textunderscore canto\textunderscore ^1)}
\end{itemize}
Peça de ferro, com que se firmam as pedras de cantaria nos edifícios.
\section{Cantonado}
\begin{itemize}
\item {Grp. gram.:adj.}
\end{itemize}
\begin{itemize}
\item {Utilização:Heráld.}
\end{itemize}
\begin{itemize}
\item {Proveniência:(De \textunderscore canto\textunderscore ^1)}
\end{itemize}
Diz-se do escudo, que tem peça nos cantos.
\section{Cantonal}
\begin{itemize}
\item {Grp. gram.:adj.}
\end{itemize}
Relativo a cantão.
\section{Cantonalista}
\begin{itemize}
\item {Grp. gram.:m.  e  adj.}
\end{itemize}
Partidário, em Espanha, da politica chamada \textunderscore cantonal\textunderscore .
\section{Cantoneira}
\begin{itemize}
\item {Grp. gram.:f.}
\end{itemize}
\begin{itemize}
\item {Utilização:Ant.}
\end{itemize}
\begin{itemize}
\item {Utilização:Constr.}
\end{itemize}
Prateleira, que se adapta ao canto da casa, e que serve para nella se collocarem utensílios e objectos differentes.
Prostituta; mulher vagabunda, que anda pelos cantos.
Peça de ferro, espécie de barra cuja secção transversal, póde sêr L ou T.
Neste último caso, também se chama ferro em \textunderscore T\textunderscore .
\section{Cantoneiro}
\begin{itemize}
\item {Grp. gram.:m.}
\end{itemize}
\begin{itemize}
\item {Proveniência:(De \textunderscore cantão\textunderscore )}
\end{itemize}
Trabalhador, encarregado da conservação e guarda de um cantão de estrada.
\section{Cantoplastia}
\begin{itemize}
\item {Grp. gram.:f.}
\end{itemize}
\begin{itemize}
\item {Proveniência:(Do gr. \textunderscore kanthos\textunderscore  + \textunderscore plassein\textunderscore )}
\end{itemize}
Operação cirúrgica, com que se reforma o canto do ôlho.
\section{Cantor}
\begin{itemize}
\item {Grp. gram.:m.}
\end{itemize}
\begin{itemize}
\item {Proveniência:(Lat. \textunderscore cantor\textunderscore , de \textunderscore canere\textunderscore )}
\end{itemize}
Aquelle que canta.
Aquelle que canta por profissão.
Poéta: \textunderscore o cantor dos«Lusíadas»\textunderscore .
\section{Canto-redondo}
\begin{itemize}
\item {Grp. gram.:m.}
\end{itemize}
Espécie de lima de espingardeiro e ferreiro.
\section{Cantoria}
\begin{itemize}
\item {Grp. gram.:f.}
\end{itemize}
\begin{itemize}
\item {Proveniência:(De \textunderscore cantor\textunderscore )}
\end{itemize}
Reunião de vozes, cantando.
Acto de cantar.
\section{Cantorina}
\begin{itemize}
\item {Grp. gram.:f.}
\end{itemize}
O mesmo que \textunderscore cantatriz\textunderscore .
\section{Cantorinhar}
\begin{itemize}
\item {Grp. gram.:v. i.}
\end{itemize}
\begin{itemize}
\item {Utilização:Des.}
\end{itemize}
Cantar por muito tempo, em voz pouco elevada.
\section{Cantroço}
\begin{itemize}
\item {fónica:trô}
\end{itemize}
\begin{itemize}
\item {Grp. gram.:m.}
\end{itemize}
\begin{itemize}
\item {Utilização:Prov.}
\end{itemize}
\begin{itemize}
\item {Utilização:trasm.}
\end{itemize}
\begin{itemize}
\item {Proveniência:(De \textunderscore canto\textunderscore ^2)}
\end{itemize}
Grande pedaço de qualquer coisa comestível.
Canoco.
\section{Cantua}
\begin{itemize}
\item {Grp. gram.:f.}
\end{itemize}
Gênero de plantas herbáceas do Peru.
\section{Cantufa}
\begin{itemize}
\item {Grp. gram.:f.}
\end{itemize}
Acácia espinhosa, da Abyssínia.
\section{Cantunlia}
\begin{itemize}
\item {Grp. gram.:f.}
\end{itemize}
Imposto antigo, que se pagava em Gôa.
\section{Canudo}
\begin{itemize}
\item {Grp. gram.:m.}
\end{itemize}
\begin{itemize}
\item {Utilização:Pop.}
\end{itemize}
\begin{itemize}
\item {Utilização:Constr.}
\end{itemize}
Tubo, geralmente comprido.
Prega engomada nos folhos da roupa.
Cabellos em anel.
Lôgro.
\textunderscore Abóbada de canudo\textunderscore , abóbada cylíndrica.
(Cp. cast. \textunderscore canuto\textunderscore )
\section{Canudo-amargoso}
\begin{itemize}
\item {Grp. gram.:m.}
\end{itemize}
\begin{itemize}
\item {Utilização:Bras}
\end{itemize}
O mesmo que \textunderscore pau-pereiro\textunderscore .
\section{Canudo-de-pita}
\begin{itemize}
\item {Grp. gram.:m.}
\end{itemize}
\begin{itemize}
\item {Utilização:Bras}
\end{itemize}
Planta euphorbiácea medicinal.
\section{Cânula}
\begin{itemize}
\item {Grp. gram.:f.}
\end{itemize}
\begin{itemize}
\item {Proveniência:(Lat. \textunderscore cannula\textunderscore )}
\end{itemize}
Tubo de vários instrumentos cirúrgicos.
\section{Canumboto}
\begin{itemize}
\item {fónica:bô}
\end{itemize}
\begin{itemize}
\item {Grp. gram.:m.}
\end{itemize}
Serpente africana.
\section{Canungloquira}
\begin{itemize}
\item {Grp. gram.:f.}
\end{itemize}
Réptil sáurio da África.
\section{Canutilho}
\begin{itemize}
\item {Grp. gram.:m.}
\end{itemize}
Canudinho de vidro, para enfeites e guarnições de vestuário feminino.
(Cast. \textunderscore cañutillo\textunderscore )
\section{Canza}
\begin{itemize}
\item {Grp. gram.:m.}
\end{itemize}
Grosseiro instrumento musical, no Brasil.
\section{Canzana}
\begin{itemize}
\item {Grp. gram.:f.}
\end{itemize}
\begin{itemize}
\item {Utilização:T. da Bairrada}
\end{itemize}
\begin{itemize}
\item {Grp. gram.:Loc. adv.}
\end{itemize}
Vadiagem; mandriice.
\textunderscore Á canzana\textunderscore , á maneira dos cães: \textunderscore comer á canzana\textunderscore . (Colhido em Turquel)
\section{Canzarrão}
\begin{itemize}
\item {Grp. gram.:m.}
\end{itemize}
Cão grande.
\section{Canzenze}
\begin{itemize}
\item {Grp. gram.:m.}
\end{itemize}
Arbusto angolense, de fôlhas glaucas e compostas, e flôres papilionáceas, vermelhas.
\section{Canzil}
\begin{itemize}
\item {Grp. gram.:m.}
\end{itemize}
O mesmo que \textunderscore canil\textunderscore ^1 ou \textunderscore cangalho\textunderscore .
\section{Canzoada}
\begin{itemize}
\item {Grp. gram.:f.}
\end{itemize}
\begin{itemize}
\item {Utilização:Fig.}
\end{itemize}
Ajuntamento de cães.
Canalha.
\section{Canzoal}
\begin{itemize}
\item {Grp. gram.:adj.}
\end{itemize}
\begin{itemize}
\item {Utilização:Fig.}
\end{itemize}
\begin{itemize}
\item {Proveniência:(Do rad. de \textunderscore canzoada\textunderscore )}
\end{itemize}
Relativo a cães.
Vil.
\section{Canzoeira}
\begin{itemize}
\item {Grp. gram.:f.}
\end{itemize}
Canzoada; barulho de cães, ladrando.
\section{Canzurrada}
\begin{itemize}
\item {Grp. gram.:f.}
\end{itemize}
\begin{itemize}
\item {Utilização:Prov.}
\end{itemize}
Porção de paveias de trigo ou centeio, que se alastra na eira, para se debulhar de cada eirada.
(Cp. \textunderscore canzurral\textunderscore )
\section{Canzurral}
\begin{itemize}
\item {Grp. gram.:m.}
\end{itemize}
\begin{itemize}
\item {Utilização:Bras}
\end{itemize}
Mato de arbúsculos, prejudicial ao desenvolvimento das pastagens.
\section{Cão}
\begin{itemize}
\item {Grp. gram.:m.}
\end{itemize}
\begin{itemize}
\item {Utilização:Pop.}
\end{itemize}
\begin{itemize}
\item {Utilização:Fig.}
\end{itemize}
\begin{itemize}
\item {Utilização:Bras. do N}
\end{itemize}
\begin{itemize}
\item {Proveniência:(Do lat. \textunderscore canis\textunderscore )}
\end{itemize}
Quadrúpede doméstico, carnívoro.
Constellação austral.
Peça de espingarda, que percute a cápsula.
Peça de madeira, que pende da calha que deita cereaes no ôlho da mó e assenta sôbre a mesma mó.
Antiga peça de artilharia.
Dívida, que se não paga.
Homem mau, vil.
O mesmo que \textunderscore diabo\textunderscore .
\textunderscore Cão da chaminé\textunderscore , utensílio de ferro, para amparar a lenha que arde na lareira.
\section{Cão}
\begin{itemize}
\item {Grp. gram.:adj.}
\end{itemize}
\begin{itemize}
\item {Utilização:Ant.}
\end{itemize}
\begin{itemize}
\item {Proveniência:(Lat. \textunderscore canus\textunderscore )}
\end{itemize}
Branco; que tem cabellos brancos. \textunderscore Pl. Cãos.\textunderscore 
\section{Cão}
\begin{itemize}
\item {Grp. gram.:m.}
\end{itemize}
Peixe dos Açores.
\section{Caóba}
\begin{itemize}
\item {Grp. gram.:f.}
\end{itemize}
(V.acaju)
\section{Cão-do-mar}
\begin{itemize}
\item {Grp. gram.:m.}
\end{itemize}
Gênero de peixes, a que pertence o tubarão e a raia.
\section{Cão-do-mato}
\begin{itemize}
\item {Grp. gram.:m.}
\end{itemize}
\begin{itemize}
\item {Utilização:Bras}
\end{itemize}
Pequeno quadrúpede, escuro ou cinzento.
\section{Caolho}
\begin{itemize}
\item {fónica:ô}
\end{itemize}
\begin{itemize}
\item {Grp. gram.:m.  e  adj.}
\end{itemize}
\begin{itemize}
\item {Utilização:Bras}
\end{itemize}
O mesmo que \textunderscore zarolho\textunderscore .
\section{Cão-marinho}
\begin{itemize}
\item {Grp. gram.:m.}
\end{itemize}
O mesmo que \textunderscore peixe-cão\textunderscore .
\section{Cão-tinhoso}
\begin{itemize}
\item {Grp. gram.:m.}
\end{itemize}
\begin{itemize}
\item {Utilização:Pop.}
\end{itemize}
O diabo.
\section{Cãozeiro}
\begin{itemize}
\item {Grp. gram.:m.  e  adj.}
\end{itemize}
\begin{itemize}
\item {Utilização:Fam.}
\end{itemize}
O mesmo que \textunderscore caloteiro\textunderscore .
\section{Cãozinho}
\begin{itemize}
\item {Grp. gram.:m.}
\end{itemize}
Cão pequeno.
\section{Capa}
\begin{itemize}
\item {Grp. gram.:f.}
\end{itemize}
\begin{itemize}
\item {Utilização:Fig.}
\end{itemize}
\begin{itemize}
\item {Utilização:Prov.}
\end{itemize}
\begin{itemize}
\item {Utilização:beir.}
\end{itemize}
\begin{itemize}
\item {Utilização:Náut.}
\end{itemize}
Vestuário largo e sem mangas, que, pendente dos ombros, se usa sôbre a outra roupa.
Aquillo que envolve ou cobre alguma coisa.
A vela grande dos navios.
Acolhimento, protecção.
Pretexto, apparência: \textunderscore capa de santidade\textunderscore .
Pedra chata e sem apparelho, com que se cobre o alvanel ou aqueducto.
Manobra, empregada em occasião de mau tempo, para proteger o navio contra a violência das vagas: \textunderscore navegar de capa\textunderscore .
(B. lat. \textunderscore cappa\textunderscore )
\section{Capaça}
\begin{itemize}
\item {Grp. gram.:f.}
\end{itemize}
Árvore da Zambézia, (\textunderscore leuchocarpus laxiflorus\textunderscore , Guill.).
\section{Capação}
\begin{itemize}
\item {Grp. gram.:f.}
\end{itemize}
\begin{itemize}
\item {Utilização:Agr.}
\end{itemize}
Acto de \textunderscore capar ou\textunderscore  cortar (rebentos).
\section{Capa-capote}
\begin{itemize}
\item {Grp. gram.:f.}
\end{itemize}
Capote sem mangas, com aberturas por onde passam os braços, e com cabeção pendente, que encobre as mesmas aberturas.
\section{Capacete}
\begin{itemize}
\item {fónica:cê}
\end{itemize}
\begin{itemize}
\item {Grp. gram.:m.}
\end{itemize}
\begin{itemize}
\item {Proveniência:(Do rad. do lat. \textunderscore caput\textunderscore , donde o b. lat. \textunderscore capitium\textunderscore , que daria o dem. \textunderscore capicete\textunderscore  == \textunderscore capacete\textunderscore )}
\end{itemize}
Armadura, de copa oval, para a cabeça.
Peça côncava de metal, que cobre a caldeira do alambique.
Tecto móvel de moínho de vento.
\textunderscore Capacete de gêlo\textunderscore , camada de neve que, em certos casos, se applica á cabeça dos doentes.
\section{Capacheiro}
\begin{itemize}
\item {Grp. gram.:m.}
\end{itemize}
Fabricante ou vendedor de capachos.
\section{Capachice}
\begin{itemize}
\item {Grp. gram.:f.}
\end{itemize}
\begin{itemize}
\item {Utilização:Fam.}
\end{itemize}
O mesmo que \textunderscore capachismo\textunderscore .
\section{Capachismo}
\begin{itemize}
\item {Grp. gram.:m.}
\end{itemize}
\begin{itemize}
\item {Utilização:Fam.}
\end{itemize}
\begin{itemize}
\item {Proveniência:(De \textunderscore capacho\textunderscore )}
\end{itemize}
Servilismo.
\section{Capacho}
\begin{itemize}
\item {Grp. gram.:m.}
\end{itemize}
\begin{itemize}
\item {Utilização:Fig.}
\end{itemize}
Utensílio cylíndrico de esparto ou de outra substância, para aquecer os pés.
Tecido rectangular ou redondo, de esparto ou borracha, em que se limpa calçado.
Pessôa sem pundonor, que se curva servilmente àquelles de quem depende.
(Cp. \textunderscore cabaz\textunderscore  e cast. \textunderscore capazo\textunderscore . Talvez do lat. \textunderscore capax\textunderscore )
\section{Capacidade}
\begin{itemize}
\item {Grp. gram.:f.}
\end{itemize}
\begin{itemize}
\item {Proveniência:(Lat. \textunderscore capacitas\textunderscore )}
\end{itemize}
Âmbito interior de um corpo vazio.
Qualidade, que uma pessôa ou coisa tem, de satisfazer a certo fim.
Aptidão.
Pessôa de bom carácter.
Possibilidade legal.
Pessôa conspícua, de grandes aptidões.
\section{Capacitado}
\begin{itemize}
\item {Grp. gram.:adj.}
\end{itemize}
\begin{itemize}
\item {Proveniência:(De \textunderscore capacitar\textunderscore )}
\end{itemize}
Persuadido.
\section{Capacitar}
\begin{itemize}
\item {Grp. gram.:v. t.}
\end{itemize}
\begin{itemize}
\item {Utilização:Des.}
\end{itemize}
\begin{itemize}
\item {Grp. gram.:V. p.}
\end{itemize}
\begin{itemize}
\item {Proveniência:(Do rad. de \textunderscore capaz\textunderscore )}
\end{itemize}
Tornar capaz.
Persuadir.
Entender, comprehender.
Persuadir-se.
\section{Capada}
\begin{itemize}
\item {Grp. gram.:f.}
\end{itemize}
\begin{itemize}
\item {Proveniência:(Fr. \textunderscore capade\textunderscore )}
\end{itemize}
Camada de pêlo, nos chapéus de feltro.
\section{Capadaria}
\begin{itemize}
\item {Grp. gram.:f.}
\end{itemize}
\begin{itemize}
\item {Utilização:Bras. de Minas}
\end{itemize}
Conjunto de \textunderscore capados\textunderscore  ou porcos de engorda.
\section{Capadeira}
\begin{itemize}
\item {Grp. gram.:f.}
\end{itemize}
\begin{itemize}
\item {Proveniência:(De \textunderscore capar\textunderscore )}
\end{itemize}
Faca ou navalha, própria para capar.
Navalha pequena.
\section{Capadeiro}
\begin{itemize}
\item {Grp. gram.:m.}
\end{itemize}
(V.capador)
\section{Capadete}
\begin{itemize}
\item {fónica:dê}
\end{itemize}
\begin{itemize}
\item {Grp. gram.:m.}
\end{itemize}
\begin{itemize}
\item {Utilização:Bras. do S}
\end{itemize}
\begin{itemize}
\item {Proveniência:(De \textunderscore capar\textunderscore )}
\end{itemize}
Porco castrado, mas ainda não cevado.
\section{Capado}
\begin{itemize}
\item {Grp. gram.:m.}
\end{itemize}
\begin{itemize}
\item {Utilização:Bras}
\end{itemize}
\begin{itemize}
\item {Proveniência:(De \textunderscore capar\textunderscore )}
\end{itemize}
Bode ou carneiro castrado.
Porco grande, cevado.
\section{Capadoçada}
\begin{itemize}
\item {Grp. gram.:f.}
\end{itemize}
\begin{itemize}
\item {Utilização:Bras}
\end{itemize}
Acção de capadócio.
\section{Capadoçagem}
\begin{itemize}
\item {Grp. gram.:f.}
\end{itemize}
O mesmo que \textunderscore capadoçada\textunderscore .
\section{Capadoçal}
\begin{itemize}
\item {Grp. gram.:adj.}
\end{itemize}
\begin{itemize}
\item {Utilização:Bras}
\end{itemize}
Feito á maneira de capadócio.
\section{Capadócio}
\begin{itemize}
\item {Grp. gram.:m.  e  adj.}
\end{itemize}
\begin{itemize}
\item {Utilização:Bras. do N}
\end{itemize}
\begin{itemize}
\item {Utilização:Bras. do S}
\end{itemize}
Trapaceiro; charlatão.
Parlapatão.
Indivíduo que, de noite, vai tocar e descantar, sob as janelas de mulher namorada.
\section{Capador}
\begin{itemize}
\item {Grp. gram.:m.}
\end{itemize}
\begin{itemize}
\item {Utilização:Prov.}
\end{itemize}
\begin{itemize}
\item {Proveniência:(De \textunderscore capar\textunderscore )}
\end{itemize}
Aquelle que capa.
Instrumento músico de pastores, semelhante ao com que nalgumas terras se annunciam os capadores de porcos, amoladores ambulantes, etc.
\section{Capadura}
\begin{itemize}
\item {Grp. gram.:f.}
\end{itemize}
Acto de capar.
\section{Capagorja}
\begin{itemize}
\item {fónica:cá}
\end{itemize}
\begin{itemize}
\item {Grp. gram.:f.}
\end{itemize}
\begin{itemize}
\item {Proveniência:(De \textunderscore capa\textunderscore  + \textunderscore gorja\textunderscore )}
\end{itemize}
Vestuário antigo.
\section{Capa-homem}
\begin{itemize}
\item {Grp. gram.:m.}
\end{itemize}
\begin{itemize}
\item {Utilização:Bras}
\end{itemize}
Espécie de cipó medicinal.
\section{Capaia}
\begin{itemize}
\item {Grp. gram.:f.}
\end{itemize}
Vasilha africana para líquidos.
\section{Capala-mazeu}
\begin{itemize}
\item {Grp. gram.:f.}
\end{itemize}
Pequena árvore angolense, de fôlhas alternas, compostas, e inflorescência em pequenas espigas axillares.
\section{Capambo}
\begin{itemize}
\item {Grp. gram.:m.}
\end{itemize}
\begin{itemize}
\item {Utilização:T. da África port}
\end{itemize}
Peste dos bosques. Cf. Capello e Ivens, II, 19.
\section{Capanda}
\begin{itemize}
\item {Grp. gram.:f.}
\end{itemize}
Pequena árvore angolense, de hastes herbáceas, fôlhas simples, e flôres dispostas em cachos.
\section{Capandua}
\begin{itemize}
\item {Grp. gram.:f.}
\end{itemize}
Espécie de maçan vermelha.
\section{Capanga}
\begin{itemize}
\item {Grp. gram.:m.}
\end{itemize}
\begin{itemize}
\item {Utilização:Bras}
\end{itemize}
\begin{itemize}
\item {Grp. gram.:F.}
\end{itemize}
Valentão, ao serviço de alguém, para o defender ou vingar.
Pequena bôlsa, que os viajantes levam a tiracollo e que também se chama \textunderscore mocó\textunderscore .
\section{Capangada}
\begin{itemize}
\item {Grp. gram.:f.}
\end{itemize}
Grupo de capangas.
\section{Capangueiro}
\begin{itemize}
\item {Grp. gram.:m.}
\end{itemize}
\begin{itemize}
\item {Utilização:Bras}
\end{itemize}
Comprador de diamantes, em pequenas porções, aos mineiros que os extráem.
\section{Capanguinha}
\begin{itemize}
\item {Grp. gram.:f.}
\end{itemize}
\begin{itemize}
\item {Utilização:Bras. de Minas}
\end{itemize}
Pequena bôlsa de coiro, usada por pretos e caboclos, que a trazem presa ao correão.
\section{Capão}
\begin{itemize}
\item {Grp. gram.:m.}
\end{itemize}
\begin{itemize}
\item {Proveniência:(Gr. \textunderscore kapon\textunderscore )}
\end{itemize}
Gallo capado.
\section{Capão}
\begin{itemize}
\item {Grp. gram.:m.}
\end{itemize}
\begin{itemize}
\item {Utilização:Bras}
\end{itemize}
\begin{itemize}
\item {Proveniência:(Do tupi-guar. \textunderscore caápaún\textunderscore )}
\end{itemize}
Bosque insulado em meio de um descampado.
\section{Capão}
\begin{itemize}
\item {Grp. gram.:m.}
\end{itemize}
\begin{itemize}
\item {Utilização:Prov.}
\end{itemize}
\begin{itemize}
\item {Grp. gram.:Adj.}
\end{itemize}
\begin{itemize}
\item {Utilização:Prov.}
\end{itemize}
\begin{itemize}
\item {Utilização:trasm.}
\end{itemize}
Mólho ou feixe de vides, que se cortaram na poda.
Diz-se de uma variedade de feijão rasteiro.
\section{Capapele}
\begin{itemize}
\item {Grp. gram.:f.}
\end{itemize}
\begin{itemize}
\item {Proveniência:(De \textunderscore capa\textunderscore  + \textunderscore pelle\textunderscore )}
\end{itemize}
Antigo vestuário, forrado de pelles.
\section{Capapelle}
\begin{itemize}
\item {Grp. gram.:f.}
\end{itemize}
\begin{itemize}
\item {Proveniência:(De \textunderscore capa\textunderscore  + \textunderscore pelle\textunderscore )}
\end{itemize}
Antigo vestuário, forrado de pelles.
\section{Capar}
\begin{itemize}
\item {Grp. gram.:v. t.}
\end{itemize}
\begin{itemize}
\item {Proveniência:(Do rad. de \textunderscore capão\textunderscore ^1)}
\end{itemize}
Cortar, inutilizar, os órgãos da reproducção animal.
Castrar.
Cortar rebentos de (uma planta): \textunderscore capar o meloal\textunderscore .
\section{Caparala}
\begin{itemize}
\item {Grp. gram.:f.}
\end{itemize}
Nome de dois pássaros dentirostros da África.
\section{Caparão}
\begin{itemize}
\item {Grp. gram.:m.}
\end{itemize}
\begin{itemize}
\item {Utilização:Prov.}
\end{itemize}
\begin{itemize}
\item {Utilização:alg.}
\end{itemize}
Cobertura para a cabeça das aves que serviam na caça.
Espécie de alcofa, com que se peneira a farinha.
Alcofa, em que se põem pequenas mós, usadas para moer em casa pequenas porções de cereaes.
(B. lat. \textunderscore caparo\textunderscore )
\section{Caparazão}
\begin{itemize}
\item {Grp. gram.:m.}
\end{itemize}
\begin{itemize}
\item {Utilização:Ant.}
\end{itemize}
Espécie de chairel.
Antiga armadura dos cavallos.
(Cast. \textunderscore caparazón\textunderscore )
\section{Caparoeiro}
\begin{itemize}
\item {Grp. gram.:adj.}
\end{itemize}
\begin{itemize}
\item {Proveniência:(De \textunderscore caparão\textunderscore )}
\end{itemize}
Diz-se das aves que, empregadas em caça de altanaria, recebem bem o caparão.
\section{Caparoroca}
\begin{itemize}
\item {Grp. gram.:f.}
\end{itemize}
\begin{itemize}
\item {Utilização:Bras}
\end{itemize}
Árvore silvestre de bôa madeira para construcções.
\section{Caparrosa}
\begin{itemize}
\item {Grp. gram.:f.}
\end{itemize}
Designação vulgar de vários sulfatos.
(Cast. \textunderscore caparosa\textunderscore )
\section{Capa-saia}
\begin{itemize}
\item {Grp. gram.:f.}
\end{itemize}
\begin{itemize}
\item {Utilização:Ant.}
\end{itemize}
Capa fechada e redonda, a modo de batina.
O mesmo que \textunderscore capi-saio\textunderscore .
\section{Capassa}
\begin{itemize}
\item {Grp. gram.:f.}
\end{itemize}
Árvore da Zambézia, (\textunderscore leuchocarpus laxiflorus\textunderscore , Guill.).
\section{Capata}
\begin{itemize}
\item {Grp. gram.:f.}
\end{itemize}
O mesmo que \textunderscore quimbombo\textunderscore .
\section{Capataço}
\begin{itemize}
\item {Grp. gram.:m.}
\end{itemize}
\begin{itemize}
\item {Proveniência:(De \textunderscore pata\textunderscore  e um pref. obscuro)}
\end{itemize}
Pancada com a pata.
\section{Capatão}
\begin{itemize}
\item {Grp. gram.:m.}
\end{itemize}
Peixe pristipomátida, espécie de pargo.
(Cast. \textunderscore capitón\textunderscore )
\section{Capataz}
\begin{itemize}
\item {Grp. gram.:m.}
\end{itemize}
Chefe de um grupo de trabalhadores, aguadeiros, moços de fretes, etc.
(B. lat. \textunderscore capitacium\textunderscore )
\section{Capatazar}
\begin{itemize}
\item {Grp. gram.:v. t.}
\end{itemize}
\begin{itemize}
\item {Utilização:Neol.}
\end{itemize}
Dirigir como capataz.
\section{Capatazia}
\begin{itemize}
\item {Grp. gram.:f.}
\end{itemize}
Funcções de capataz.
Grupo de indivíduos, dirigidos por um capataz.
\section{Capatázio}
\begin{itemize}
\item {Grp. gram.:m.  e  adj.}
\end{itemize}
\begin{itemize}
\item {Utilização:Pop.}
\end{itemize}
\begin{itemize}
\item {Proveniência:(De \textunderscore capataz\textunderscore )}
\end{itemize}
Aquelle que pertence a um grupo de indivíduos dirigidos por um capataz.
Sócio, camarada.
\section{Çapato}
\begin{itemize}
\item {Grp. gram.:m.}
\end{itemize}
\begin{itemize}
\item {Grp. gram.:Loc.}
\end{itemize}
\begin{itemize}
\item {Utilização:fam.}
\end{itemize}
Peça de calçado, que cobre só o pé.
\textunderscore Sapatos de defunto\textunderscore , promessas ou esperanças, cuja realização é muito demorada ou muito incerta.
(Cp. \textunderscore çapata\textunderscore )
\section{Capaz}
\begin{itemize}
\item {Grp. gram.:adj.}
\end{itemize}
\begin{itemize}
\item {Utilização:Fam.}
\end{itemize}
\begin{itemize}
\item {Proveniência:(Lat. \textunderscore capax\textunderscore )}
\end{itemize}
Que tem capacidade.
Amplo.
Bom; prestadio.
Quási resolvido: \textunderscore estou capaz de lhe ir ás ventas\textunderscore .
\section{Capazmente}
\begin{itemize}
\item {Grp. gram.:adv.}
\end{itemize}
De modo capaz.
\section{Capazócio}
\begin{itemize}
\item {Grp. gram.:adj.}
\end{itemize}
\begin{itemize}
\item {Utilização:Pop.}
\end{itemize}
O mesmo que \textunderscore capazório\textunderscore .
\section{Capazório}
\begin{itemize}
\item {Grp. gram.:adj.}
\end{itemize}
\begin{itemize}
\item {Utilização:Pop.}
\end{itemize}
Capaz.
* \textunderscore Adj. T. da Bairrada\textunderscore , irón.
Que é pessôa desacreditada; que é um bandalho.
\section{Capcioso}
\begin{itemize}
\item {Grp. gram.:adj.}
\end{itemize}
\begin{itemize}
\item {Proveniência:(Lat. \textunderscore captiosus\textunderscore )}
\end{itemize}
Ardiloso, manhoso.
Argucioso, para induzir em êrro.
\section{Capeadamente}
\begin{itemize}
\item {Grp. gram.:adv.}
\end{itemize}
\begin{itemize}
\item {Proveniência:(De \textunderscore capear\textunderscore )}
\end{itemize}
Occultamente.
\section{Capeador}
\begin{itemize}
\item {Grp. gram.:m.}
\end{itemize}
Aquelle que capeia; toireiro.
\section{Capeamento}
\begin{itemize}
\item {Grp. gram.:m.}
\end{itemize}
Acto de \textunderscore capear\textunderscore  ou revestir superiormente com pedras soltas (uma parede).
\section{Capear}
\begin{itemize}
\item {Grp. gram.:v. t.}
\end{itemize}
\begin{itemize}
\item {Grp. gram.:V. i.}
\end{itemize}
\begin{itemize}
\item {Utilização:T. de marinheiros}
\end{itemize}
\begin{itemize}
\item {Utilização:Náut.}
\end{itemize}
\begin{itemize}
\item {Grp. gram.:V. t.  e  i.}
\end{itemize}
\begin{itemize}
\item {Proveniência:(De \textunderscore capa\textunderscore )}
\end{itemize}
Esconder com capa.
Cobrir, revestir.
Disfarçar.
Attrahir ardilosamente; illudir.
Chamar, provocar com capa (os toiros).
Fazer sinal com capa, bandeira, etc.
Brilhar, (falando-se de certas estrellas, que annunciam bom ou mau tempo, segundo brilham mais dêste ou daquelle dos pontos cardiaes).
Pôr de capa um navio.
Cobrir ou revestir de capeia.
\section{Capeba}
\begin{itemize}
\item {Grp. gram.:f.}
\end{itemize}
\begin{itemize}
\item {Grp. gram.:M.}
\end{itemize}
\begin{itemize}
\item {Utilização:Bras}
\end{itemize}
Arbusto americano, de raiz amarga e medicinal.
Amigo.
Sócio, camarada.
(Do tupi \textunderscore caá-peba\textunderscore )
\section{Capecha-apalu}
\begin{itemize}
\item {Grp. gram.:m.}
\end{itemize}
Arbusto de Moçambique.
\section{Capeia}
\begin{itemize}
\item {Grp. gram.:f.}
\end{itemize}
\begin{itemize}
\item {Utilização:Prov.}
\end{itemize}
\begin{itemize}
\item {Utilização:minh.}
\end{itemize}
\begin{itemize}
\item {Utilização:trasm.}
\end{itemize}
\begin{itemize}
\item {Proveniência:(De \textunderscore capear\textunderscore )}
\end{itemize}
Pedra grande, para revestimento da parte superior de um cano, de uma parede, etc.
\section{Capeirão}
\begin{itemize}
\item {Grp. gram.:m.}
\end{itemize}
Capa grande.
(Cast. \textunderscore capirón\textunderscore )
\section{Capeirete}
\begin{itemize}
\item {fónica:peirê}
\end{itemize}
\begin{itemize}
\item {Grp. gram.:m.}
\end{itemize}
\begin{itemize}
\item {Utilização:Ant.}
\end{itemize}
\begin{itemize}
\item {Proveniência:(De \textunderscore capeirão\textunderscore )}
\end{itemize}
Pequena capa.
\section{Capeiro}
\begin{itemize}
\item {Grp. gram.:m.}
\end{itemize}
\begin{itemize}
\item {Proveniência:(De \textunderscore capa\textunderscore )}
\end{itemize}
Aquelle que leva capa em procissões de igreja.
Cabide, guarda-fato.
Moço de guarda-roupa.
\section{Capeirote}
\begin{itemize}
\item {Grp. gram.:m.}
\end{itemize}
O mesmo que \textunderscore capeirete\textunderscore . Cf. Herculano, \textunderscore Lendas\textunderscore , I, 54 e 55.
\section{Capela}
\begin{itemize}
\item {Grp. gram.:f.}
\end{itemize}
\begin{itemize}
\item {Utilização:Prov.}
\end{itemize}
\begin{itemize}
\item {Utilização:Ant.}
\end{itemize}
\begin{itemize}
\item {Proveniência:(Lat. \textunderscore capella\textunderscore )}
\end{itemize}
Pequena igreja.
Santuário.
Parte ou dependência de palácio, colégio, etc.
Cada uma das divisões de um templo, com um altar.
Grupo de músicos, que cantam ou tocam em igreja ou capela.
Música.
Quinquilharias á venda em pequenos estabelecimentos.
Loja ou estabelecimento, em que se vendem quinquilharias ou miudezas de applicação vária: agulhas, linhas, pentes, etc.
Grinalda de flôres ou fôlhas.
Abóbada de fôrno.
Invólucro da espiga do milho, folhelho.
Vínculo, que tinha o encargo de occorrer ás despesas do culto de uma capela.
\section{Capelada}
\begin{itemize}
\item {Grp. gram.:f.}
\end{itemize}
\begin{itemize}
\item {Utilização:Ant.}
\end{itemize}
Peça que cobre a bôca dos coldres.
Correia, com que se atava o chapim.
(Cast. \textunderscore capellada\textunderscore )
\section{Capelada}
\begin{itemize}
\item {Grp. gram.:f.}
\end{itemize}
Porção de capellas ou grinaldas.
\section{Capelana}
\begin{itemize}
\item {Grp. gram.:f.}
\end{itemize}
\begin{itemize}
\item {Utilização:T. de Moçambique}
\end{itemize}
Pano, que os pretos trazem pendente do cínto, dobrando sôbre êste uma ponta, em que ás vezes guardam dinheiro.
\section{Capelania}
\begin{itemize}
\item {Grp. gram.:f.}
\end{itemize}
Cargo de capelão.
(B. lat. \textunderscore capellanea\textunderscore )
\section{Capelão}
\begin{itemize}
\item {Grp. gram.:m.}
\end{itemize}
Padre, que tem a seu cargo o dizer Missa em capela.
Aquelle que diz Missa e presta auxílios espirituaes aos regimentos militares.
(B. lat. \textunderscore capellanus\textunderscore )
\section{Capeleio}
\begin{itemize}
\item {Grp. gram.:m.}
\end{itemize}
\begin{itemize}
\item {Utilização:Ant.}
\end{itemize}
\begin{itemize}
\item {Proveniência:(De \textunderscore capello\textunderscore  ou \textunderscore capella\textunderscore )}
\end{itemize}
Espécie de toucado ou ornato da cabeça.
\section{Capeleira}
\begin{itemize}
\item {Grp. gram.:f.}
\end{itemize}
\begin{itemize}
\item {Utilização:Des.}
\end{itemize}
\begin{itemize}
\item {Proveniência:(De \textunderscore capella\textunderscore )}
\end{itemize}
Vendedora de flôres.
\section{Capelengau}
\begin{itemize}
\item {Grp. gram.:m.}
\end{itemize}
Árvore angolense de Caconda.
\section{Capeles}
\begin{itemize}
\item {Grp. gram.:m. pl.}
\end{itemize}
Uma das categorias, em que se divide o séquito do soba dos Gingas.
\section{Capeliço}
\begin{itemize}
\item {Grp. gram.:m.}
\end{itemize}
\begin{itemize}
\item {Utilização:Ant.}
\end{itemize}
\begin{itemize}
\item {Proveniência:(De \textunderscore capello\textunderscore )}
\end{itemize}
Roupão com capuz.
\section{Capelina}
\begin{itemize}
\item {Grp. gram.:f.}
\end{itemize}
\begin{itemize}
\item {Proveniência:(De \textunderscore capello\textunderscore )}
\end{itemize}
Elmo, ligeira armadura para a cabeça.
\section{Capelista}
\begin{itemize}
\item {Grp. gram.:m. f.}
\end{itemize}
Pessôa que vende em loja de capela.
\section{Capella}
\begin{itemize}
\item {Grp. gram.:f.}
\end{itemize}
\begin{itemize}
\item {Utilização:Prov.}
\end{itemize}
\begin{itemize}
\item {Utilização:Ant.}
\end{itemize}
\begin{itemize}
\item {Proveniência:(Lat. \textunderscore capella\textunderscore )}
\end{itemize}
Pequena igreja.
Santuário.
Parte ou dependência de palácio, collégio, etc.
Cada uma das divisões de um templo, com um altar.
Grupo de músicos, que cantam ou tocam em igreja ou capella.
Música.
Quinquilharias á venda em pequenos estabelecimentos.
Loja ou estabelecimento, em que se vendem quinquilharias ou miudezas de applicação vária: agulhas, linhas, pentes, etc.
Grinalda de flôres ou fôlhas.
Abóbada de fôrno.
Invólucro da espiga do milho, folhelho.
Vínculo, que tinha o encargo de occorrer ás despesas do culto de uma capella.
\section{Capellada}
\begin{itemize}
\item {Grp. gram.:f.}
\end{itemize}
\begin{itemize}
\item {Utilização:Ant.}
\end{itemize}
Peça que cobre a bôca dos coldres.
Correia, com que se atava o chapim.
(Cast. \textunderscore capellada\textunderscore )
\section{Capellada}
\begin{itemize}
\item {Grp. gram.:f.}
\end{itemize}
Porção de capellas ou grinaldas.
\section{Capellania}
\begin{itemize}
\item {Grp. gram.:f.}
\end{itemize}
Cargo de capellão.
(B. lat. \textunderscore capellanea\textunderscore )
\section{Capellão}
\begin{itemize}
\item {Grp. gram.:m.}
\end{itemize}
Padre, que tem a seu cargo o dizer Missa em capella.
Aquelle que diz Missa e presta auxílios espirituaes aos regimentos militares.
(B. lat. \textunderscore capellanus\textunderscore )
\section{Capelleio}
\begin{itemize}
\item {Grp. gram.:m.}
\end{itemize}
\begin{itemize}
\item {Utilização:Ant.}
\end{itemize}
\begin{itemize}
\item {Proveniência:(De \textunderscore capello\textunderscore  ou \textunderscore capella\textunderscore )}
\end{itemize}
Espécie de toucado ou ornato da cabeça.
\section{Capelleira}
\begin{itemize}
\item {Grp. gram.:f.}
\end{itemize}
\begin{itemize}
\item {Utilização:Des.}
\end{itemize}
\begin{itemize}
\item {Proveniência:(De \textunderscore capella\textunderscore )}
\end{itemize}
Vendedora de flôres.
\section{Capelliço}
\begin{itemize}
\item {Grp. gram.:m.}
\end{itemize}
\begin{itemize}
\item {Utilização:Ant.}
\end{itemize}
\begin{itemize}
\item {Proveniência:(De \textunderscore capello\textunderscore )}
\end{itemize}
Roupão com capuz.
\section{Capellina}
\begin{itemize}
\item {Grp. gram.:f.}
\end{itemize}
\begin{itemize}
\item {Proveniência:(De \textunderscore capello\textunderscore )}
\end{itemize}
Elmo, ligeira armadura para a cabeça.
\section{Capellista}
\begin{itemize}
\item {Grp. gram.:m. f.}
\end{itemize}
Pessôa que vende em loja de capella.
\section{Capello}
\begin{itemize}
\item {fónica:pê}
\end{itemize}
\begin{itemize}
\item {Grp. gram.:m.}
\end{itemize}
\begin{itemize}
\item {Utilização:Náut.}
\end{itemize}
\begin{itemize}
\item {Utilização:Constr.}
\end{itemize}
Capuz de frades.
Antiga touca de viúvas e freiras.
Espécie de murça, que os doutores põem sobre os ombros em acto solenne.
Chapéu cardinalício.
Dossel.
Volta, que se dá com a amarra nas abitas.
Parte superior da roda da prôa.
Parte superior das chaminés, formada em ângulo agudo, estreitando successivamente a fuga.
O mesmo que \textunderscore capeamento\textunderscore . Cf. Assis Teixeira, \textunderscore Aguas\textunderscore , 342.
O mesmo que \textunderscore capacete\textunderscore . Cf. Rebello, \textunderscore Ódio Velho\textunderscore , 11.
(B. lat. \textunderscore capellus\textunderscore )
\section{Capelludo}
\begin{itemize}
\item {Grp. gram.:Adj.}
\end{itemize}
Que usa capello.
\section{Capelo}
\begin{itemize}
\item {fónica:pê}
\end{itemize}
\begin{itemize}
\item {Grp. gram.:m.}
\end{itemize}
\begin{itemize}
\item {Utilização:Náut.}
\end{itemize}
\begin{itemize}
\item {Utilização:Constr.}
\end{itemize}
Capuz de frades.
Antiga touca de viúvas e freiras.
Espécie de murça, que os doutores põem sobre os ombros em acto solene.
Chapéu cardinalício.
Dossel.
Volta, que se dá com a amarra nas abitas.
Parte superior da roda da prôa.
Parte superior das chaminés, formada em ângulo agudo, estreitando successivamente a fuga.
O mesmo que \textunderscore capeamento\textunderscore . Cf. Assis Teixeira, \textunderscore Aguas\textunderscore , 342.
O mesmo que \textunderscore capacete\textunderscore . Cf. Rebello, \textunderscore Ódio Velho\textunderscore , 11.
(B. lat. \textunderscore capellus\textunderscore )
\section{Capeludo}
\begin{itemize}
\item {Grp. gram.:Adj.}
\end{itemize}
Que usa capelo.
\section{Capemba}
\begin{itemize}
\item {Grp. gram.:f.}
\end{itemize}
\begin{itemize}
\item {Utilização:Bras}
\end{itemize}
Fôlha larga e consistente, que se desprende do mangará de algumas palmeiras.
\section{Capendua}
\begin{itemize}
\item {Grp. gram.:f.}
\end{itemize}
(V.capandua)
\section{Capenga}
\begin{itemize}
\item {Grp. gram.:m.  e  adj.}
\end{itemize}
\begin{itemize}
\item {Utilização:Bras}
\end{itemize}
Pessôa coxa, indivíduo torto.
\section{Capengar}
\begin{itemize}
\item {Grp. gram.:v. i.}
\end{itemize}
\begin{itemize}
\item {Utilização:Bras}
\end{itemize}
\begin{itemize}
\item {Proveniência:(De \textunderscore capenga\textunderscore )}
\end{itemize}
Coxear.
\section{Capepuxis}
\begin{itemize}
\item {Grp. gram.:m. pl.}
\end{itemize}
Povo selvagem das margens do Araguaia, no Brasil.
\section{Capera}
\begin{itemize}
\item {Grp. gram.:f.}
\end{itemize}
Espécie de plantas, da fam. das euphorbiáceas e originárias da América.
\section{Caperotada}
\begin{itemize}
\item {Grp. gram.:f.}
\end{itemize}
\begin{itemize}
\item {Proveniência:(It. \textunderscore capperottata\textunderscore )}
\end{itemize}
Guisado de aves, assadas antes.
\section{Capeta}
\begin{itemize}
\item {Grp. gram.:m.}
\end{itemize}
\begin{itemize}
\item {Utilização:Bras}
\end{itemize}
Diabo.
Traquinas.
\section{Capetagem}
\begin{itemize}
\item {Grp. gram.:f.}
\end{itemize}
\begin{itemize}
\item {Proveniência:(De \textunderscore capeta\textunderscore )}
\end{itemize}
Diabrura.
\section{Capetinança}
\begin{itemize}
\item {Grp. gram.:f.}
\end{itemize}
\begin{itemize}
\item {Utilização:Ant.}
\end{itemize}
O mesmo que \textunderscore capitania\textunderscore .
(Por \textunderscore capitanança\textunderscore , de \textunderscore capitão\textunderscore )
\section{Capetinha}
\begin{itemize}
\item {Grp. gram.:f.}
\end{itemize}
\begin{itemize}
\item {Utilização:Bras}
\end{itemize}
\begin{itemize}
\item {Proveniência:(De \textunderscore capeta\textunderscore )}
\end{itemize}
Criança traquinas.
\section{Capiacanca}
\begin{itemize}
\item {Grp. gram.:f.}
\end{itemize}
Planta malvácea da África portuguesa, espécie de abutilão.
\section{Capialçado}
\begin{itemize}
\item {Grp. gram.:m.}
\end{itemize}
\begin{itemize}
\item {Utilização:Constr.}
\end{itemize}
\begin{itemize}
\item {Grp. gram.:Adj.}
\end{itemize}
Curvatura de abóbada.
Diz-se do sôbre-arco ou do córte oblíquo na parte superior das portas e janelas, para dar mais luz ás casas.
(Cast. \textunderscore capialzado\textunderscore )
\section{Capialço}
\begin{itemize}
\item {Grp. gram.:m.}
\end{itemize}
\begin{itemize}
\item {Utilização:Constr.}
\end{itemize}
Córte oblíquo na parte superior das portas ou janelas, para dar mais luz ao interior das casas.
(Cp. \textunderscore capialçado\textunderscore )
\section{Capiana}
\begin{itemize}
\item {Grp. gram.:f.}
\end{itemize}
Pequena planta aromática de Angola.
\section{Capianga}
\begin{itemize}
\item {Grp. gram.:f.}
\end{itemize}
\begin{itemize}
\item {Utilização:Bras}
\end{itemize}
Árvore silvestre.
\section{Capiangagem}
\begin{itemize}
\item {Grp. gram.:f.}
\end{itemize}
\begin{itemize}
\item {Utilização:Bras}
\end{itemize}
Acto de capiango.
\section{Capiangar}
\begin{itemize}
\item {Grp. gram.:v. i.}
\end{itemize}
\begin{itemize}
\item {Utilização:Bras}
\end{itemize}
\begin{itemize}
\item {Proveniência:(De \textunderscore capiango\textunderscore )}
\end{itemize}
Surripiar; furtar com destreza.
\section{Capiango}
\begin{itemize}
\item {Grp. gram.:m.}
\end{itemize}
\begin{itemize}
\item {Utilização:Bras}
\end{itemize}
Ladrão astuto.
Ladrão.
\section{Capiapia}
\begin{itemize}
\item {Grp. gram.:f.}
\end{itemize}
Pássaro dentirostro da África occidental.
\section{Capiar}
\begin{itemize}
\item {Grp. gram.:m.}
\end{itemize}
Espécie de sacristão, no Malabar.
\section{Capi-catinga}
\begin{itemize}
\item {Grp. gram.:f.}
\end{itemize}
\begin{itemize}
\item {Utilização:Bras}
\end{itemize}
Nome de uma planta.
\section{Capiçova}
\begin{itemize}
\item {Grp. gram.:f.}
\end{itemize}
Planta do Brasil.
Nome que, em Alagôas, se dá á erva-do-bicho.
\section{Capicua}
\begin{itemize}
\item {Grp. gram.:f.}
\end{itemize}
Nome que, no jôgo do dominó, se dá á pedra que póde fazer dominó, ou ganhar a partida, deslocando-a para um e outro lado.
\section{Cápide}
\begin{itemize}
\item {Grp. gram.:m.}
\end{itemize}
Taça ou copo de duas asas, usado nos sacrifícios antigos.
\section{Capídulo}
\begin{itemize}
\item {Grp. gram.:m.}
\end{itemize}
Espécie de vestuário, com que os Romanos cobriam a cabeça.
\section{Capigorrão}
\begin{itemize}
\item {Grp. gram.:m.}
\end{itemize}
\begin{itemize}
\item {Utilização:Des.}
\end{itemize}
Nome, que se dava aos estudantes que usam capa e gorro.
(Cast. \textunderscore capigorrón\textunderscore )
\section{Capiguará}
\begin{itemize}
\item {Grp. gram.:m.}
\end{itemize}
\begin{itemize}
\item {Utilização:Bras}
\end{itemize}
Espécie de lontra.
(Do guar.)
\section{Capiláceo}
\begin{itemize}
\item {Grp. gram.:adj.}
\end{itemize}
\begin{itemize}
\item {Proveniência:(De \textunderscore capillar\textunderscore )}
\end{itemize}
Que tem filamentos capilares.
\section{Capilamento}
\begin{itemize}
\item {Grp. gram.:m.}
\end{itemize}
\begin{itemize}
\item {Proveniência:(Lat. \textunderscore capillamentum\textunderscore )}
\end{itemize}
Fibra tenuíssima.
Cabeladura.
\section{Capilar}
\begin{itemize}
\item {Grp. gram.:adj.}
\end{itemize}
\begin{itemize}
\item {Proveniência:(Lat. \textunderscore capillaris\textunderscore )}
\end{itemize}
Relativo a cabelo.
Delgado como um cabelo.
Que se produz em tubos muito estreitos.
Relativo ás ramificações vasculares, por onde o sangue passa das artérias para as veias.
\section{Capilária}
\begin{itemize}
\item {Grp. gram.:f.}
\end{itemize}
O mesmo que \textunderscore avenca\textunderscore .
\section{Capilaridade}
\begin{itemize}
\item {Grp. gram.:f.}
\end{itemize}
\begin{itemize}
\item {Proveniência:(De \textunderscore capillar\textunderscore )}
\end{itemize}
Qualidade do que é delgado como um cabelo.
Phenómenos, resultantes do contacto dos líquidos com os sólidos, mormente quando nestes há interstícios capilares.
\section{Capilé}
\begin{itemize}
\item {Grp. gram.:m.}
\end{itemize}
\begin{itemize}
\item {Proveniência:(Do fr. \textunderscore capillaine\textunderscore , avenca?)}
\end{itemize}
Bebida, formada de água com xarope.
\section{Capilha}
\begin{itemize}
\item {Grp. gram.:f.}
\end{itemize}
Exemplar, que se dá de uma obra aos typógraphos e impressores, que nella trabalharam.
(Cast. \textunderscore capilla\textunderscore )
\section{Capilício}
\begin{itemize}
\item {Grp. gram.:m.}
\end{itemize}
\begin{itemize}
\item {Utilização:Ant.}
\end{itemize}
\begin{itemize}
\item {Proveniência:(Do lat. \textunderscore capillus\textunderscore )}
\end{itemize}
Trança da cabeleira.
\section{Capilifoliado}
\begin{itemize}
\item {Grp. gram.:adj.}
\end{itemize}
\begin{itemize}
\item {Proveniência:(Do lat. \textunderscore capillus\textunderscore  + \textunderscore folium\textunderscore )}
\end{itemize}
Que tem fôlhas capilares.
\section{Capiliforme}
\begin{itemize}
\item {Grp. gram.:adj.}
\end{itemize}
\begin{itemize}
\item {Proveniência:(Do lat. \textunderscore capillus\textunderscore  + \textunderscore forma\textunderscore )}
\end{itemize}
Que tem fórma de cabelo.
\section{Capilláceo}
\begin{itemize}
\item {Grp. gram.:adj.}
\end{itemize}
\begin{itemize}
\item {Proveniência:(De \textunderscore capillar\textunderscore )}
\end{itemize}
Que tem filamentos capillares.
\section{Capillamento}
\begin{itemize}
\item {Grp. gram.:m.}
\end{itemize}
\begin{itemize}
\item {Proveniência:(Lat. \textunderscore capillamentum\textunderscore )}
\end{itemize}
Fibra tenuíssima.
Cabelladura.
\section{Capillar}
\begin{itemize}
\item {Grp. gram.:adj.}
\end{itemize}
\begin{itemize}
\item {Proveniência:(Lat. \textunderscore capillaris\textunderscore )}
\end{itemize}
Relativo a cabello.
Delgado como um cabello.
Que se produz em tubos muito estreitos.
Relativo ás ramificações vasculares, por onde o sangue passa das artérias para as veias.
\section{Capillária}
\begin{itemize}
\item {Grp. gram.:f.}
\end{itemize}
O mesmo que \textunderscore avenca\textunderscore .
\section{Capillaridade}
\begin{itemize}
\item {Grp. gram.:f.}
\end{itemize}
\begin{itemize}
\item {Proveniência:(De \textunderscore capillar\textunderscore )}
\end{itemize}
Qualidade do que é delgado como um cabello.
Phenómenos, resultantes do contacto dos líquidos com os sólidos, mormente quando nestes há interstícios capillares.
\section{Capillé}
\begin{itemize}
\item {Grp. gram.:m.}
\end{itemize}
\begin{itemize}
\item {Proveniência:(Do fr. \textunderscore capillaine\textunderscore , avenca?)}
\end{itemize}
Bebida, formada de água com xarope.
\section{Capillício}
\begin{itemize}
\item {Grp. gram.:m.}
\end{itemize}
\begin{itemize}
\item {Utilização:Ant.}
\end{itemize}
\begin{itemize}
\item {Proveniência:(Do lat. \textunderscore capillus\textunderscore )}
\end{itemize}
Trança da cabelleira.
\section{Capillifoliado}
\begin{itemize}
\item {Grp. gram.:adj.}
\end{itemize}
\begin{itemize}
\item {Proveniência:(Do lat. \textunderscore capillus\textunderscore  + \textunderscore folium\textunderscore )}
\end{itemize}
Que tem fôlhas capillares.
\section{Capilliforme}
\begin{itemize}
\item {Grp. gram.:adj.}
\end{itemize}
\begin{itemize}
\item {Proveniência:(Do lat. \textunderscore capillus\textunderscore  + \textunderscore forma\textunderscore )}
\end{itemize}
Que tem fórma de cabello.
\section{Capilossada}
\begin{itemize}
\item {Grp. gram.:f.}
\end{itemize}
\begin{itemize}
\item {Utilização:Bras. do N}
\end{itemize}
Empresa arrojada.
\section{Capilota}
\begin{itemize}
\item {Grp. gram.:f.}
\end{itemize}
\begin{itemize}
\item {Utilização:Prov.}
\end{itemize}
\begin{itemize}
\item {Utilização:trasm.}
\end{itemize}
Pilota, sova, tareia.
\section{Capim}
\begin{itemize}
\item {Grp. gram.:m.}
\end{itemize}
\begin{itemize}
\item {Utilização:Bras}
\end{itemize}
Nome de várias plantas gramíneas e cyperáceas, na maior parte forraginosas.
(Do tupi \textunderscore caapiim\textunderscore )
\section{Capim}
\begin{itemize}
\item {Grp. gram.:m.}
\end{itemize}
\begin{itemize}
\item {Proveniência:(De \textunderscore capa\textunderscore )}
\end{itemize}
Rebôco áspero e de pouca consistência, feito de areia e cimento.
\section{Capimbeba}
\begin{itemize}
\item {Grp. gram.:f.}
\end{itemize}
\begin{itemize}
\item {Utilização:Bras}
\end{itemize}
Espécie de sapé medicinal, (\textunderscore anatherum bicorne\textunderscore ).
\section{Capina}
\begin{itemize}
\item {Grp. gram.:f.}
\end{itemize}
\begin{itemize}
\item {Utilização:Bras}
\end{itemize}
\begin{itemize}
\item {Utilização:Fig.}
\end{itemize}
\begin{itemize}
\item {Proveniência:(De \textunderscore capinar\textunderscore )}
\end{itemize}
Mondadura.
Reprehensão.
\section{Capinação}
\begin{itemize}
\item {Grp. gram.:f.}
\end{itemize}
\begin{itemize}
\item {Utilização:Bras}
\end{itemize}
O mesmo que \textunderscore capina\textunderscore .
\section{Capinador}
\begin{itemize}
\item {Grp. gram.:m.}
\end{itemize}
\begin{itemize}
\item {Utilização:Bras}
\end{itemize}
Aquelle que capina.
\section{Capinal}
\begin{itemize}
\item {Grp. gram.:m.}
\end{itemize}
\begin{itemize}
\item {Utilização:Bras}
\end{itemize}
O mesmo que \textunderscore capinzal\textunderscore .
\section{Capinan}
\begin{itemize}
\item {Grp. gram.:f.}
\end{itemize}
\begin{itemize}
\item {Utilização:Bras}
\end{itemize}
Árvore myrtácea, de fruto comestível.
\section{Capinar}
\begin{itemize}
\item {Grp. gram.:v. t.}
\end{itemize}
\begin{itemize}
\item {Utilização:Bras}
\end{itemize}
Mondar, segar, o capim em.
\section{Capindó}
\begin{itemize}
\item {Grp. gram.:m.}
\end{itemize}
\begin{itemize}
\item {Utilização:deprec.}
\end{itemize}
\begin{itemize}
\item {Utilização:Pop.}
\end{itemize}
Capinha curta, especialmente de mulher.
\section{Capineiro}
\begin{itemize}
\item {Grp. gram.:m.}
\end{itemize}
\begin{itemize}
\item {Utilização:Bras}
\end{itemize}
O mesmo que \textunderscore capinador\textunderscore .
\section{Capinha}
\begin{itemize}
\item {Grp. gram.:f.}
\end{itemize}
\begin{itemize}
\item {Grp. gram.:M.}
\end{itemize}
\begin{itemize}
\item {Proveniência:(De \textunderscore capa\textunderscore )}
\end{itemize}
Capa, com que o toireiro provoca ou distrai o toiro.
Toireiro, que capeia o toiro.
\section{Capinzal}
\begin{itemize}
\item {Grp. gram.:m.}
\end{itemize}
Terreno coberto de capim.
\section{Capipreto}
\begin{itemize}
\item {fónica:cá}
\end{itemize}
\begin{itemize}
\item {Grp. gram.:adj.}
\end{itemize}
Que tem cabeça preta. Cf. \textunderscore Viriato Trág.\textunderscore , XI, 104.
\section{Capirotada}
\begin{itemize}
\item {Grp. gram.:f.}
\end{itemize}
\begin{itemize}
\item {Utilização:Ant.}
\end{itemize}
\begin{itemize}
\item {Proveniência:(De \textunderscore capirote\textunderscore )}
\end{itemize}
Vestido com capello.
\section{Capirote}
\begin{itemize}
\item {Grp. gram.:m.}
\end{itemize}
\begin{itemize}
\item {Grp. gram.:Adj.}
\end{itemize}
\begin{itemize}
\item {Proveniência:(De \textunderscore capa\textunderscore )}
\end{itemize}
Capuz antigo.
Diz-se do toiro, que tem a cabeça e pescoço da mesma côr e pintas differentes no resto do corpo.
\section{Capi-saio}
\begin{itemize}
\item {Grp. gram.:m.}
\end{itemize}
\begin{itemize}
\item {Proveniência:(De \textunderscore capa\textunderscore  + \textunderscore saio\textunderscore )}
\end{itemize}
Vestidura antiga.
\section{Capiscar}
\begin{itemize}
\item {Grp. gram.:v. t.}
\end{itemize}
Perceber a intriga ou armadilha de:«\textunderscore apanha! capiscou-te! é fino o meco!\textunderscore »Castilho, \textunderscore Fausto\textunderscore , 161.
\section{Capiscol}
\begin{itemize}
\item {Grp. gram.:m.}
\end{itemize}
O mesmo que \textunderscore cabíscol\textunderscore .
\section{Capisondo}
\begin{itemize}
\item {Grp. gram.:m.}
\end{itemize}
Antigo empregado aduaneiro na China. Cf. \textunderscore Peregrinação\textunderscore , XLIX.
\section{Capistro}
\begin{itemize}
\item {Grp. gram.:m.}
\end{itemize}
\begin{itemize}
\item {Utilização:Cir.}
\end{itemize}
\begin{itemize}
\item {Utilização:Zool.}
\end{itemize}
\begin{itemize}
\item {Proveniência:(Lat. \textunderscore capistrum\textunderscore )}
\end{itemize}
Faixa ou atadura para a cabeça.
Parte da cabeça das aves, em volta do bico.
\section{Capitação}
\begin{itemize}
\item {Grp. gram.:f.}
\end{itemize}
\begin{itemize}
\item {Proveniência:(Lat. \textunderscore capitatio\textunderscore )}
\end{itemize}
Imposto, que se paga por cabeça.
\section{Capitaina}
\begin{itemize}
\item {Grp. gram.:f.  e  adj.}
\end{itemize}
\begin{itemize}
\item {Proveniência:(Do lat. \textunderscore capitaneus\textunderscore )}
\end{itemize}
Nau, em que ia o commandante, o chefe, o capitão de uma esquadra.
\section{Capital}
\begin{itemize}
\item {Grp. gram.:adj.}
\end{itemize}
\begin{itemize}
\item {Utilização:Typ.}
\end{itemize}
\begin{itemize}
\item {Grp. gram.:F.}
\end{itemize}
\begin{itemize}
\item {Grp. gram.:M.}
\end{itemize}
\begin{itemize}
\item {Proveniência:(Lat. \textunderscore capitalis\textunderscore )}
\end{itemize}
Relativo á cabeça, á vida de alguém.
Que diz respeito ao último supplício: \textunderscore pena capital\textunderscore .
Que é cabeça ou parte principal de alguma coisa; essencial: \textunderscore ideia capital\textunderscore .
Diz-se de certos caractéres typográphicos de corpo grande.
Povoação, que é a séde do govêrno de uma nação.
Povoação principal de qualquer circunscripção territorial: \textunderscore Viseu, capital de districto\textunderscore .
A parte de uma dívida, excluindo o juro.
Valor pecuniário, que constitue a base de uma indústria.
Acumulação de productos do trabalho ou da indústria, destinados á fomentação de novos productos.
Riqueza; numerário.
Meios de acção.
\section{Capitalismo}
\begin{itemize}
\item {Grp. gram.:m.}
\end{itemize}
Influência ou predomínio do capital.
\section{Capitalista}
\begin{itemize}
\item {Grp. gram.:m.}
\end{itemize}
\begin{itemize}
\item {Grp. gram.:M.  e  adj.}
\end{itemize}
\begin{itemize}
\item {Proveniência:(De \textunderscore capital\textunderscore )}
\end{itemize}
Aquelle que vive do rendimento de um capital.
Pessôa muito rica.
O que fornece capital a empresas: \textunderscore sócio capitalista\textunderscore .
\section{Capitalização}
\begin{itemize}
\item {Grp. gram.:f.}
\end{itemize}
Acto de capitalizar.
\section{Capitalizar}
\begin{itemize}
\item {Grp. gram.:v. t.}
\end{itemize}
\begin{itemize}
\item {Proveniência:(De \textunderscore capital\textunderscore )}
\end{itemize}
Addicionar ao capital: \textunderscore capitalizar os juros\textunderscore .
\section{Capitalizável}
\begin{itemize}
\item {Grp. gram.:adj.}
\end{itemize}
Que se póde capitalizar.
\section{Capitalmente}
\begin{itemize}
\item {Grp. gram.:adv.}
\end{itemize}
De modo capital.
\section{Capitana}
\begin{itemize}
\item {Grp. gram.:f.  e  adj.}
\end{itemize}
\begin{itemize}
\item {Proveniência:(Do lat. \textunderscore capitaneus\textunderscore )}
\end{itemize}
Nau, em que ia o commandante, o chefe, o capitão de uma esquadra.
\section{Capitânea}
\begin{itemize}
\item {Grp. gram.:f.  e  adj.}
\end{itemize}
\begin{itemize}
\item {Proveniência:(Do lat. \textunderscore capitaneus\textunderscore )}
\end{itemize}
Nau, em que ia o commandante, o chefe, o capitão de uma esquadra.
\section{Capitanear}
\begin{itemize}
\item {Grp. gram.:v. t.}
\end{itemize}
Dirigir como capitão; commandar.
Governar.
\section{Capitanete}
\begin{itemize}
\item {fónica:nê}
\end{itemize}
\begin{itemize}
\item {Grp. gram.:m.}
\end{itemize}
\begin{itemize}
\item {Utilização:Deprec.}
\end{itemize}
Capitão ridículo ou insignificante. Cf. Arn. Gama, \textunderscore Segr. do Ab.\textunderscore , 62.
\section{Capitango}
\begin{itemize}
\item {Grp. gram.:m.}
\end{itemize}
Espécie de juiz, que resolve as demandas, nalgumas tríbos de Angola. Cf. Capello e Ivens, I, 173.
\section{Capitânia}
\begin{itemize}
\item {Grp. gram.:f.  e  adj.}
\end{itemize}
\begin{itemize}
\item {Proveniência:(Do lat. \textunderscore capitaneus\textunderscore )}
\end{itemize}
Nau, em que ia o commandante, o chefe, o capitão de uma esquadra.
\section{Capitania}
\begin{itemize}
\item {Grp. gram.:f.}
\end{itemize}
Qualidade, dignidade de capitão.
Commando.
Antiga circunscripção territorial das colónias portuguesas.
Circunscripção militar em Espanha.
Circunscripção administrativa das águas territoriaes.
\section{Capitania-mór}
\begin{itemize}
\item {Grp. gram.:f.}
\end{itemize}
Cargo de capitão-mór. Cf. Herculano, \textunderscore Quest. Públ.\textunderscore  II, 334.
\section{Capitão}
\begin{itemize}
\item {Grp. gram.:m.}
\end{itemize}
\begin{itemize}
\item {Utilização:T. de Moçambique}
\end{itemize}
\begin{itemize}
\item {Utilização:Burl.}
\end{itemize}
Chefe militar.
O que commandava uma expedição, um exército, uma armada.
Chefe de uma companhia regimental, graduado entre tenente e major.
Commandante de navio mercante.
Antigo commandante de milícias locaes.
Caudilho, chefe.
Capataz ou feitor indígena.
Designação de várias plantas umbellíferas.
Autoridade administrativa das águas territoriaes.
Grumete, encarregado das vassoiras, a bordo dos navios de guerra.
\textunderscore Capitão de fragata\textunderscore , \textunderscore Capitão de mar e guerra\textunderscore , postos militares na armada.
(B. lat. \textunderscore capitanus\textunderscore )
\section{Capitari}
\begin{itemize}
\item {Grp. gram.:m.}
\end{itemize}
\begin{itemize}
\item {Utilização:Bras}
\end{itemize}
Macho da tartaruga.
\section{Capitato}
\begin{itemize}
\item {Grp. gram.:adj.}
\end{itemize}
\begin{itemize}
\item {Proveniência:(Lat. \textunderscore capitatus\textunderscore )}
\end{itemize}
Que tem fórma de cabeça.
Que termina em cabeça.
\section{Capitel}
\begin{itemize}
\item {Grp. gram.:m.}
\end{itemize}
Parte mais elevada de uma columna.
Parte superior de pilastra, balaústre, etc.
Capacete de alambique.
Cabeça do foguete.
Resguardo do ouvido das peças de artilharia.
(B. lat. \textunderscore capitellum\textunderscore )
\section{Capitilúvio}
\begin{itemize}
\item {fónica:cá}
\end{itemize}
\begin{itemize}
\item {Grp. gram.:m.}
\end{itemize}
\begin{itemize}
\item {Proveniência:(Lat. \textunderscore capitiluvium\textunderscore )}
\end{itemize}
Acção de banhar a cabeça.
\section{Cápito}
\begin{itemize}
\item {Grp. gram.:m.}
\end{itemize}
\begin{itemize}
\item {Utilização:Gír.}
\end{itemize}
Capitão de ladrões.
\section{Capitôa}
\begin{itemize}
\item {Grp. gram.:f.}
\end{itemize}
\begin{itemize}
\item {Grp. gram.:Adj. f.}
\end{itemize}
Mulher de capitão.
Mulher, que commanda outras.
O mesmo que \textunderscore capitânea\textunderscore .
(Fem. de \textunderscore capitão\textunderscore )
\section{Capitol}
\begin{itemize}
\item {Grp. gram.:m.}
\end{itemize}
Medicamento, descoberto pelo Dr. Eichoff, contra todos os parasitos de coiro cabelludo.
\section{Capitolina}
\begin{itemize}
\item {Grp. gram.:f.}
\end{itemize}
\begin{itemize}
\item {Utilização:Bras}
\end{itemize}
Loção de capitol, contra as doenças do cabello.
\section{Capitolino}
\begin{itemize}
\item {Grp. gram.:adj.}
\end{itemize}
\begin{itemize}
\item {Proveniência:(Lat. \textunderscore capitolinus\textunderscore )}
\end{itemize}
Relativo ao capitólio.
\section{Capitólio}
\begin{itemize}
\item {Grp. gram.:m.}
\end{itemize}
\begin{itemize}
\item {Utilização:Fig.}
\end{itemize}
\begin{itemize}
\item {Proveniência:(Do lat. \textunderscore Capitolium\textunderscore , n. p.)}
\end{itemize}
Templo pagão.
Triumpho; glória, esplendor.
\section{Capitoso}
\begin{itemize}
\item {Grp. gram.:adj.}
\end{itemize}
Que tem cabeça grande.
Cabeçudo; teimoso.
Que sobe á cabeça, que embriaga, que estonteia: \textunderscore vinho capitoso\textunderscore .
(Cast. \textunderscore capitoso\textunderscore )
\section{Capítula}
\begin{itemize}
\item {Grp. gram.:f.}
\end{itemize}
\begin{itemize}
\item {Proveniência:(Do lat. \textunderscore capitulum\textunderscore )}
\end{itemize}
Cada uma das lições curtas do \textunderscore Breviario\textunderscore , extrahidas da \textunderscore Bíblia\textunderscore .
\section{Capitulação}
\begin{itemize}
\item {Grp. gram.:f.}
\end{itemize}
\begin{itemize}
\item {Proveniência:(De \textunderscore capitular\textunderscore ^2)}
\end{itemize}
Acto de capitular: convenção, em que se estabelecem as condições, com que um chefe militar entrega ao inimigo o pôsto ou as fôrças que commanda.
Transacção, acôrdo, entre litigantes.
Transigência, sujeição á fôrça das circunstâncias.
\section{Capitulador}
\begin{itemize}
\item {Grp. gram.:m.}
\end{itemize}
\begin{itemize}
\item {Utilização:Des.}
\end{itemize}
\begin{itemize}
\item {Proveniência:(De \textunderscore capitular\textunderscore )}
\end{itemize}
Aquelle que accusa.
\section{Capitulante}
\begin{itemize}
\item {Grp. gram.:adj.}
\end{itemize}
\begin{itemize}
\item {Proveniência:(De \textunderscore capitular\textunderscore ^2)}
\end{itemize}
Que capitúla.
\section{Capitular}
\begin{itemize}
\item {Grp. gram.:adj.}
\end{itemize}
\begin{itemize}
\item {Proveniência:(Lat. \textunderscore capitularis\textunderscore )}
\end{itemize}
Relativo a capítulo, assembleia de dignidades ecclesiásticas.
Relativo a cabido.
Maiúsculo: \textunderscore letra capitular\textunderscore .
\section{Capitular}
\begin{itemize}
\item {Grp. gram.:v. t.}
\end{itemize}
\begin{itemize}
\item {Grp. gram.:V. i.}
\end{itemize}
\begin{itemize}
\item {Proveniência:(De \textunderscore capitulo\textunderscore )}
\end{itemize}
Combinar, contratar, mediante condições.
Classificar, qualificar; accusar, expondo a accusação em capítulos; reduzir a capítulos.
Entregar-se por capitulação.
Transigir, ceder.
\section{Capitulares}
\begin{itemize}
\item {Grp. gram.:f. pl.}
\end{itemize}
\begin{itemize}
\item {Proveniência:(Do lat. \textunderscore capitularis\textunderscore )}
\end{itemize}
Decretos reaes, e ordenanças prescritas pelas assembleias nacionaes, na França medieval.
\section{Capitularmente}
\begin{itemize}
\item {Grp. gram.:adv.}
\end{itemize}
\begin{itemize}
\item {Proveniência:(De \textunderscore capitular\textunderscore ^1)}
\end{itemize}
Em fórma de capítulo.
Á maneira de cabido, ou em reunião de cabido.
\section{Capituleiro}
\begin{itemize}
\item {Grp. gram.:m.}
\end{itemize}
Livro ecclesiástico, que contém as capítulas.
(B. lat. \textunderscore capitularium\textunderscore )
\section{Capítulo}
\begin{itemize}
\item {Grp. gram.:m.}
\end{itemize}
\begin{itemize}
\item {Utilização:Bot.}
\end{itemize}
\begin{itemize}
\item {Proveniência:(Lat. \textunderscore capitulum\textunderscore )}
\end{itemize}
Cada uma das divisões de um livro, contrato, etc.
Artigo de contrato, de accusação, etc.
Assembleia de frades, reunião de cónegos para tratar de certos assumptos.
Assumpto, objecto.
Lugar, em que se reunem os religiosos ou os cónegos.
Collegiada, ou corporação de cónegos.
Qualquer assembleia.
Inflorescência, em que muitas flôres reunidas, sustentadas por um pedúnculo, dão apparência de uma só flôr.
\section{Capituva}
\begin{itemize}
\item {Grp. gram.:f.}
\end{itemize}
\begin{itemize}
\item {Utilização:Bras}
\end{itemize}
Planta gramínea, que cresce á beira dos rios.
(Do tupi-guar.)
\section{Capivara}
\begin{itemize}
\item {Grp. gram.:f.}
\end{itemize}
\begin{itemize}
\item {Utilização:Bras}
\end{itemize}
\begin{itemize}
\item {Proveniência:(T. tupi)}
\end{itemize}
Mammífero roëdor.
\section{Capixaba}
\begin{itemize}
\item {Grp. gram.:m.}
\end{itemize}
\begin{itemize}
\item {Utilização:Bras}
\end{itemize}
Homem natural do Estado do Espirito-Santo.
Pequeno estabelecimento agrícola.
(Do tupi)
\section{Capnófugo}
\begin{itemize}
\item {Grp. gram.:adj.}
\end{itemize}
\begin{itemize}
\item {Proveniência:(Do gr. \textunderscore kapnos\textunderscore  + lat. \textunderscore fugere\textunderscore )}
\end{itemize}
Que livra do fumo.
\section{Capnomancia}
\begin{itemize}
\item {Grp. gram.:f.}
\end{itemize}
\begin{itemize}
\item {Proveniência:(Do gr. \textunderscore kapnos\textunderscore  + \textunderscore manteia\textunderscore )}
\end{itemize}
Adivinhação por meio do fumo.
\section{Capnomante}
\begin{itemize}
\item {Grp. gram.:m.}
\end{itemize}
Aquelle que pratíca a capnomancia.
\section{Capnomantecía}
\begin{itemize}
\item {Grp. gram.:f.}
\end{itemize}
\begin{itemize}
\item {Proveniência:(T. mal derivado de \textunderscore capnomante\textunderscore )}
\end{itemize}
O mesmo que \textunderscore capnomancia\textunderscore . Cf. Castilho, \textunderscore Fastos\textunderscore , III, 317.
\section{Capnomantico}
\begin{itemize}
\item {Grp. gram.:adj.}
\end{itemize}
Relativo á capnomancia.
\section{Capoão}
\begin{itemize}
\item {Grp. gram.:m.}
\end{itemize}
\begin{itemize}
\item {Utilização:Bras. de Minas}
\end{itemize}
Mata, o mesmo que \textunderscore capoeirão\textunderscore ^2.
\section{Capochos}
\begin{itemize}
\item {Grp. gram.:m. pl.}
\end{itemize}
Selvagens, que habitavam em Mato-Grosso.
\section{Capoeira}
\begin{itemize}
\item {Grp. gram.:f.}
\end{itemize}
\begin{itemize}
\item {Utilização:Bras}
\end{itemize}
\begin{itemize}
\item {Utilização:Pop.}
\end{itemize}
\begin{itemize}
\item {Proveniência:(De \textunderscore capão\textunderscore )}
\end{itemize}
Cêsto grande ou qualquer compartímento, ordinariamente gradeado, onde se guardam e criam capões ou outras aves.
Gaiola.
Espécie de cêsto, com que resguardam a cabeça os defensores de uma fortaleza.
Escavação, que se guarnece de seteiras.
Ave, semelhante á perdiz.
Sege velha.
\section{Capoeira}
\begin{itemize}
\item {Grp. gram.:f.}
\end{itemize}
\begin{itemize}
\item {Utilização:Bras}
\end{itemize}
\begin{itemize}
\item {Grp. gram.:M.}
\end{itemize}
Mata, que se roça ou é destinada a roçar-se.
Negro sertanejo, que assalta os viandantes.
Capanga.
Jôgo athlético dos crioulos brasileiros.
(Do tupi \textunderscore capuêra\textunderscore )
\section{Capoeiragem}
\begin{itemize}
\item {Grp. gram.:f.}
\end{itemize}
\begin{itemize}
\item {Utilização:Bras}
\end{itemize}
Vida de capoeira, de desordeiro.
\section{Capoeirar}
\begin{itemize}
\item {Grp. gram.:v. t.}
\end{itemize}
\begin{itemize}
\item {Utilização:Bras}
\end{itemize}
Têr vida de capoeira, de velhaco.
\section{Capoeirão}
\begin{itemize}
\item {Grp. gram.:m.  e  adj.}
\end{itemize}
\begin{itemize}
\item {Proveniência:(De \textunderscore capoeira\textunderscore ^1)}
\end{itemize}
Homem velho, e pacato pela idade.
\section{Capoeirão}
\begin{itemize}
\item {Grp. gram.:m.}
\end{itemize}
\begin{itemize}
\item {Utilização:Bras}
\end{itemize}
\begin{itemize}
\item {Proveniência:(De \textunderscore capoeira\textunderscore ^2)}
\end{itemize}
Mata muito densa.
\section{Capoeireiro}
\begin{itemize}
\item {Grp. gram.:m.}
\end{itemize}
\begin{itemize}
\item {Utilização:Bras. do N}
\end{itemize}
\begin{itemize}
\item {Proveniência:(De \textunderscore capoeira\textunderscore ^2)}
\end{itemize}
Espécie de veado.
\section{Capoeiro}
\begin{itemize}
\item {Grp. gram.:m.}
\end{itemize}
\begin{itemize}
\item {Utilização:Des.}
\end{itemize}
\begin{itemize}
\item {Utilização:Prov.}
\end{itemize}
\begin{itemize}
\item {Utilização:minh.}
\end{itemize}
\begin{itemize}
\item {Proveniência:(De \textunderscore capão\textunderscore )}
\end{itemize}
Aquelle que rouba aves de capoeira.
Larápio.
O mesmo que \textunderscore capoeira\textunderscore ^1.
\section{Capoeiro}
\begin{itemize}
\item {Grp. gram.:adj.}
\end{itemize}
\begin{itemize}
\item {Grp. gram.:M.}
\end{itemize}
\begin{itemize}
\item {Utilização:Bras. do N}
\end{itemize}
Relativo a matas que se roçam.
Aquelle que vive nessas matas.
Veado das matas.
(Cp. \textunderscore capoeira\textunderscore ^2)
\section{Capádoce}
\begin{itemize}
\item {Grp. gram.:m.}
\end{itemize}
Habitante da Capadócia.
Fórma preferível a \textunderscore capadócio\textunderscore . Cf. \textunderscore Lusiadas\textunderscore , III, 72.
\section{Capadócio}
\begin{itemize}
\item {Grp. gram.:adj.}
\end{itemize}
\begin{itemize}
\item {Grp. gram.:M.}
\end{itemize}
Relativo á Capadócia.
Habitante da Capadócia.
\section{Caparidáceas}
\begin{itemize}
\item {Grp. gram.:f. pl.}
\end{itemize}
O mesmo ou melhor que \textunderscore caparídeas\textunderscore .
\section{Caparídeas}
\begin{itemize}
\item {Grp. gram.:f. pl.}
\end{itemize}
\begin{itemize}
\item {Proveniência:(Do gr. \textunderscore kaparis\textunderscore  + \textunderscore eidos\textunderscore )}
\end{itemize}
Família de plantas, a que serve de typo a alcaparra.
\section{Capolacaxixe}
\begin{itemize}
\item {Grp. gram.:m.}
\end{itemize}
Arbusto angolense, de cujas fôlhas se servem os indígenas no tratamento de feridas contusas.
\section{Capolim}
\begin{itemize}
\item {Grp. gram.:m.}
\end{itemize}
Espécie de ameixa do México, (\textunderscore prunus capolli\textunderscore ).
\section{Capóna}
\begin{itemize}
\item {Grp. gram.:f.}
\end{itemize}
\begin{itemize}
\item {Utilização:Prov.}
\end{itemize}
\begin{itemize}
\item {Utilização:trasm.}
\end{itemize}
Égua pequena, mas forte.
\section{Caponga}
\begin{itemize}
\item {Grp. gram.:f.}
\end{itemize}
\begin{itemize}
\item {Utilização:Bras}
\end{itemize}
Pequeno lago de água doce, que se fórma naturalmente nos areaes do litoral.
\section{Capongui}
\begin{itemize}
\item {Grp. gram.:f.}
\end{itemize}
A fêmea do caloqueio.
\section{Caporal}
\begin{itemize}
\item {Grp. gram.:m.}
\end{itemize}
\begin{itemize}
\item {Proveniência:(Fr. \textunderscore caporal\textunderscore )}
\end{itemize}
Antiga graduação militar, entre cabo e sargento.
Qualidade de tabaco picado.
\section{Caporro}
\begin{itemize}
\item {fónica:pô}
\end{itemize}
\begin{itemize}
\item {Grp. gram.:m.}
\end{itemize}
\begin{itemize}
\item {Utilização:T. de Moçambique}
\end{itemize}
Mulato.
\section{Capota}
\begin{itemize}
\item {Grp. gram.:f.}
\end{itemize}
\begin{itemize}
\item {Proveniência:(T. cast.)}
\end{itemize}
Espécie de touca que, cobrindo a cabeça, cái sôbre os ombros.
Coberta de caleche, victoria e outros vehículos, a qual, quando se afroixam as molas, descai para o lado detrás, deixando descoberto o vehículo.
\section{Capotasto}
\begin{itemize}
\item {Grp. gram.:m.}
\end{itemize}
\begin{itemize}
\item {Utilização:Prov.}
\end{itemize}
\begin{itemize}
\item {Utilização:alent.}
\end{itemize}
\begin{itemize}
\item {Utilização:Bras}
\end{itemize}
\begin{itemize}
\item {Proveniência:(Do it. \textunderscore capo\textunderscore , cabeça, e \textunderscore tasto\textunderscore , tecla)}
\end{itemize}
Pequena barra de madeira ou metal, que se colloca sôbre as cordas do violão e da guitarra, e que se aperta com um parafuso contra o braço do instrumento, para fazer subir o diapasão meio ponto ou mais, desviando o risco de se partirem as cordas por demasiada tensão.
Sôbre-cobertura de palha na cumeeira e caibros da tacaníça.
\section{Capote}
\begin{itemize}
\item {Grp. gram.:m.}
\end{itemize}
\begin{itemize}
\item {Utilização:Fig.}
\end{itemize}
\begin{itemize}
\item {Proveniência:(De \textunderscore capa\textunderscore )}
\end{itemize}
Capa comprida e larga, com cabeção ou capuz.
Casaco comprido, usado por soldados.
Não fazer vasas ao jôgo ou fazer menos de trinta tentos, no jôgo da bisca.
Capinha de toireiro.
Gallinha de Angola.
Disfarce.
\section{Capoteira}
\begin{itemize}
\item {Grp. gram.:f.}
\end{itemize}
\begin{itemize}
\item {Utilização:Prov.}
\end{itemize}
\begin{itemize}
\item {Proveniência:(De \textunderscore capote\textunderscore )}
\end{itemize}
Capote curto, para mulheres.
\section{Capotilha}
\begin{itemize}
\item {Grp. gram.:f.}
\end{itemize}
\begin{itemize}
\item {Utilização:Prov.}
\end{itemize}
\begin{itemize}
\item {Utilização:minh.}
\end{itemize}
\begin{itemize}
\item {Proveniência:(De \textunderscore capota\textunderscore )}
\end{itemize}
Pequena cobertura, que as camponesas põem aos ombros.
\section{Capotilho}
\begin{itemize}
\item {Grp. gram.:m.}
\end{itemize}
Capote pequeno.
\section{Cappádoce}
\begin{itemize}
\item {Grp. gram.:m.}
\end{itemize}
Habitante da Cappadócia.
Fórma preferível a \textunderscore cappadócio\textunderscore . Cf. \textunderscore Lusiadas\textunderscore , III, 72.
\section{Cappadócio}
\begin{itemize}
\item {Grp. gram.:adj.}
\end{itemize}
\begin{itemize}
\item {Grp. gram.:M.}
\end{itemize}
Relativo á Cappadócia.
Habitante da Cappadócia.
\section{Capparidáceas}
\begin{itemize}
\item {Grp. gram.:f. pl.}
\end{itemize}
O mesmo ou melhor que \textunderscore capparídeas\textunderscore .
\section{Capparídeas}
\begin{itemize}
\item {Grp. gram.:f. pl.}
\end{itemize}
\begin{itemize}
\item {Proveniência:(Do gr. \textunderscore kaparis\textunderscore  + \textunderscore eidos\textunderscore )}
\end{itemize}
Família de plantas, a que serve de typo a alcaparra.
\section{Caprária}
\begin{itemize}
\item {Grp. gram.:f.}
\end{itemize}
\begin{itemize}
\item {Proveniência:(Do lat. \textunderscore caprarius\textunderscore )}
\end{itemize}
O mesmo que \textunderscore couve-gallega\textunderscore .
\section{Caprato}
\begin{itemize}
\item {Grp. gram.:m.}
\end{itemize}
\begin{itemize}
\item {Proveniência:(De \textunderscore capro\textunderscore )}
\end{itemize}
Sal, resultante da combinação do ácido cáprico com uma base.
\section{Cápreo}
\begin{itemize}
\item {Grp. gram.:adj.}
\end{itemize}
\begin{itemize}
\item {Proveniência:(Lat. \textunderscore capreus\textunderscore )}
\end{itemize}
O mesmo que \textunderscore caprino\textunderscore . Cf. Castilho, \textunderscore Fastos\textunderscore , II, 91.
\section{Capréolo}
\begin{itemize}
\item {Grp. gram.:m.}
\end{itemize}
\begin{itemize}
\item {Utilização:Des.}
\end{itemize}
\begin{itemize}
\item {Proveniência:(Lat. \textunderscore capreolus\textunderscore )}
\end{itemize}
Espécie de cabra montesinha.
\section{Capreúva}
\begin{itemize}
\item {Grp. gram.:f.}
\end{itemize}
(V.cabureira)
\section{Capribarbudo}
\begin{itemize}
\item {Grp. gram.:adj.}
\end{itemize}
\begin{itemize}
\item {Proveniência:(Do lat. \textunderscore caper\textunderscore  + \textunderscore barba\textunderscore )}
\end{itemize}
Que tem barbas como as do bode.
\section{Caprichar}
\begin{itemize}
\item {Grp. gram.:v. i.}
\end{itemize}
Têr capricho; timbrar.
\section{Capricho}
\begin{itemize}
\item {Grp. gram.:m.}
\end{itemize}
\begin{itemize}
\item {Utilização:Heráld.}
\end{itemize}
Vontade, que sobrevém de repente e sem fundamento razoável.
Variabilidade de modas ou de ideias; inconstância.
Obstinação arrazoada.
Sentimento de dignidade, pundonor; brio.
Extravagância em obras de arte.
\textunderscore Figuras de capricho\textunderscore , as figuras heráldicas de terceira ordem ou interpoladas.
(Cast. \textunderscore capricho\textunderscore )
\section{Caprichosamente}
\begin{itemize}
\item {Grp. gram.:adv.}
\end{itemize}
De modo caprichoso.
\section{Caprichoso}
\begin{itemize}
\item {Grp. gram.:adj.}
\end{itemize}
\begin{itemize}
\item {Proveniência:(De \textunderscore capricho\textunderscore )}
\end{itemize}
Que capricha.
Feito por capricho.
Variável.
Excêntrico, extravagante.
\section{Cáprico}
\begin{itemize}
\item {Grp. gram.:adj.}
\end{itemize}
\begin{itemize}
\item {Proveniência:(De \textunderscore capro\textunderscore )}
\end{itemize}
Diz-se de um ácido, cujo cheiro lembra o bodum.
\section{Capricórnio}
\begin{itemize}
\item {Grp. gram.:adj.}
\end{itemize}
\begin{itemize}
\item {Proveniência:(Lat. \textunderscore capricornius\textunderscore )}
\end{itemize}
Constellação do Zodíaco.
Signo, que o Sol attinge no solstício do inverno.
\section{Capricórnios}
\begin{itemize}
\item {Grp. gram.:m. pl.}
\end{itemize}
Insectos lignivoros, da ordem dos coleópteros.
(Cp. \textunderscore capricórnio\textunderscore )
\section{Caprídeo}
\begin{itemize}
\item {Grp. gram.:adj.}
\end{itemize}
\begin{itemize}
\item {Grp. gram.:M. pl.}
\end{itemize}
\begin{itemize}
\item {Proveniência:(Do lat. \textunderscore capra\textunderscore  + gr. \textunderscore eidos\textunderscore )}
\end{itemize}
Relativo ou semelhante a cabra.
Classe de animaes, que comprehende a cabra e o bode, grandes e pequenos.
\section{Caprificação}
\begin{itemize}
\item {Grp. gram.:f.}
\end{itemize}
\begin{itemize}
\item {Proveniência:(Lat. \textunderscore caprificatio\textunderscore )}
\end{itemize}
Acto de caprificar.
\section{Caprificar}
\begin{itemize}
\item {Grp. gram.:v. t.}
\end{itemize}
\begin{itemize}
\item {Proveniência:(Lat. \textunderscore caprificare\textunderscore , de \textunderscore caprificus\textunderscore )}
\end{itemize}
Tocar ou picar (os figos), para lhes apressar a maturação.
\section{Caprifoliáceas}
\begin{itemize}
\item {Grp. gram.:f. pl.}
\end{itemize}
\begin{itemize}
\item {Proveniência:(Do lat. \textunderscore caprifolium\textunderscore )}
\end{itemize}
Família de plantas, que tem por typo a madresilva.
\section{Caprígeno}
\begin{itemize}
\item {Grp. gram.:adj.}
\end{itemize}
\begin{itemize}
\item {Utilização:Des.}
\end{itemize}
\begin{itemize}
\item {Proveniência:(Lat. \textunderscore caprigenus\textunderscore )}
\end{itemize}
Procedente de cabras.
\section{Caprim}
\begin{itemize}
\item {Grp. gram.:m.}
\end{itemize}
Nome, que, no Brasil, se dá á térmite ou formiga branca.
\section{Caprino}
\begin{itemize}
\item {Grp. gram.:adj.}
\end{itemize}
\begin{itemize}
\item {Proveniência:(Lat. \textunderscore caprinus\textunderscore )}
\end{itemize}
Relativo ou semelhante á cabra ou ao bode.
\section{Caprípede}
\begin{itemize}
\item {Grp. gram.:adj.}
\end{itemize}
\begin{itemize}
\item {Proveniência:(Lat. \textunderscore capripes\textunderscore )}
\end{itemize}
Que tem pés de cabra ou de bode.
\section{Caprisaltante}
\begin{itemize}
\item {fónica:sal}
\end{itemize}
\begin{itemize}
\item {Grp. gram.:adj.}
\end{itemize}
\begin{itemize}
\item {Proveniência:(De \textunderscore capra\textunderscore  lat. + \textunderscore saltante\textunderscore )}
\end{itemize}
Que salta como as cabras.
\section{Caprissaltante}
\begin{itemize}
\item {Grp. gram.:adj.}
\end{itemize}
\begin{itemize}
\item {Proveniência:(De \textunderscore capra\textunderscore  lat. + \textunderscore saltante\textunderscore )}
\end{itemize}
Que salta como as cabras.
\section{Capro}
\begin{itemize}
\item {Grp. gram.:m.}
\end{itemize}
\begin{itemize}
\item {Utilização:Des.}
\end{itemize}
\begin{itemize}
\item {Proveniência:(Lat. \textunderscore caper\textunderscore )}
\end{itemize}
O mesmo que \textunderscore bode\textunderscore ^1.
\section{Caproico}
\begin{itemize}
\item {Grp. gram.:adj.}
\end{itemize}
O mesmo que \textunderscore hexýlico\textunderscore .
\section{Caprotinas}
\begin{itemize}
\item {Grp. gram.:f. pl.}
\end{itemize}
\begin{itemize}
\item {Proveniência:(Lat. \textunderscore caprotinae\textunderscore )}
\end{itemize}
Festas, que se celebravam em honra de Juno, no mês de Julho, entre os antigos Romanos.
\section{Caprum}
\begin{itemize}
\item {Grp. gram.:adj.}
\end{itemize}
O mesmo que \textunderscore caprino\textunderscore .
\section{Capsela}
\begin{itemize}
\item {Grp. gram.:f.}
\end{itemize}
\begin{itemize}
\item {Proveniência:(Lat. \textunderscore capsella\textunderscore )}
\end{itemize}
Pequena cápsula.
Gênero de plantas crucíferas.
\section{Capsella}
\begin{itemize}
\item {Grp. gram.:f.}
\end{itemize}
\begin{itemize}
\item {Proveniência:(Lat. \textunderscore capsella\textunderscore )}
\end{itemize}
Pequena cápsula.
Gênero de plantas crucíferas.
\section{Cápsico}
\begin{itemize}
\item {Grp. gram.:m.}
\end{itemize}
\begin{itemize}
\item {Proveniência:(Do lat. \textunderscore capsum\textunderscore )}
\end{itemize}
Gênero de plantas solâneas.
\section{Cápsula}
\begin{itemize}
\item {Grp. gram.:f.}
\end{itemize}
\begin{itemize}
\item {Proveniência:(Lat. \textunderscore capsula\textunderscore )}
\end{itemize}
Designação de várias coisas, análogas a uma pequena caixa ou a um pequeno invólucro.
\section{Capsular}
\begin{itemize}
\item {Grp. gram.:adj.}
\end{itemize}
\begin{itemize}
\item {Proveniência:(Lat. \textunderscore capsularis\textunderscore )}
\end{itemize}
Semelhante á cápsula.
\section{Capsular}
\begin{itemize}
\item {Grp. gram.:v. t.}
\end{itemize}
Encerrar em cápsulas.
\section{Capsulífero}
\begin{itemize}
\item {Grp. gram.:adj.}
\end{itemize}
\begin{itemize}
\item {Proveniência:(Do lat. \textunderscore capsula\textunderscore  + \textunderscore ferre\textunderscore )}
\end{itemize}
Que tem cápsulas.
\section{Captação}
\begin{itemize}
\item {Grp. gram.:f.}
\end{itemize}
Acto de captar.
\section{Captador}
\begin{itemize}
\item {Grp. gram.:m.}
\end{itemize}
Aquelle que capta.
\section{Captagem}
\begin{itemize}
\item {Grp. gram.:f.}
\end{itemize}
\begin{itemize}
\item {Proveniência:(De \textunderscore captar\textunderscore )}
\end{itemize}
Acto de apanhar ou recolher.
Acto de aproveitar (águas correntes).
\section{Captar}
\begin{itemize}
\item {Grp. gram.:v. t.}
\end{itemize}
\begin{itemize}
\item {Proveniência:(Lat. \textunderscore captare\textunderscore )}
\end{itemize}
Attrahir, dominar, empregando meios capciosos, ou fazendo valer o próprio mérito.
Aproveitar (águas correntes).
\section{Captivar}
\textunderscore v. t.\textunderscore  (e der.)
O mesmo que \textunderscore cativar\textunderscore , etc.
\section{Captor}
\begin{itemize}
\item {Grp. gram.:m.}
\end{itemize}
\begin{itemize}
\item {Proveniência:(Lat. \textunderscore captor\textunderscore )}
\end{itemize}
Aquelle que captura.
Aquelle que arresta.
\section{Captura}
\begin{itemize}
\item {Grp. gram.:f.}
\end{itemize}
\begin{itemize}
\item {Proveniência:(Lat. \textunderscore captura\textunderscore )}
\end{itemize}
Acção de capturar.
\section{Capturador}
\begin{itemize}
\item {Grp. gram.:m.}
\end{itemize}
Aquelle que captura.
\section{Capturar}
\begin{itemize}
\item {Grp. gram.:v. t.}
\end{itemize}
\begin{itemize}
\item {Proveniência:(De \textunderscore captura\textunderscore )}
\end{itemize}
Prender.
Arrestar.
\section{Capuaba}
\begin{itemize}
\item {Grp. gram.:f.}
\end{itemize}
\begin{itemize}
\item {Utilização:Bras. do N}
\end{itemize}
\begin{itemize}
\item {Proveniência:(T. tupi-guar.)}
\end{itemize}
Cabana, choça.
Terreno para roças.
\section{Capuava}
\begin{itemize}
\item {Grp. gram.:f.}
\end{itemize}
\begin{itemize}
\item {Utilização:Bras. do N}
\end{itemize}
\begin{itemize}
\item {Proveniência:(T. tupi-guar.)}
\end{itemize}
Cabana, choça.
Terreno para roças.
\section{Capucha}
\begin{itemize}
\item {Grp. gram.:f.}
\end{itemize}
\begin{itemize}
\item {Proveniência:(De \textunderscore capuz\textunderscore )}
\end{itemize}
Capa, que cobre cabeça e ombros e é usada por mulheres do povo em alguns pontos da Beira.
\section{Capucha}
\begin{itemize}
\item {Grp. gram.:f.}
\end{itemize}
\begin{itemize}
\item {Utilização:Ant.}
\end{itemize}
\begin{itemize}
\item {Proveniência:(De \textunderscore capucho\textunderscore )}
\end{itemize}
Harmonia improvisada por cantochanistas; fabordão.
Ordem religiosa, da regra de San-Francisco.
\section{Capuchar}
\begin{itemize}
\item {Grp. gram.:v. t.}
\end{itemize}
\begin{itemize}
\item {Utilização:Fig.}
\end{itemize}
Pôr capuz ou capucha em; cobrir com capuz.
Dissimular. Cf. Filinto, IV, 222.
\section{Capucheira}
\begin{itemize}
\item {Grp. gram.:f.}
\end{itemize}
\begin{itemize}
\item {Utilização:Prov.}
\end{itemize}
\begin{itemize}
\item {Utilização:beir.}
\end{itemize}
Mulher, que usa capucha.
\section{Capuchinha}
\begin{itemize}
\item {Grp. gram.:f.}
\end{itemize}
Planta hortense, (\textunderscore aropeolum\textunderscore ).
\section{Capuchinho}
\begin{itemize}
\item {Grp. gram.:m.}
\end{itemize}
\begin{itemize}
\item {Grp. gram.:Adj.}
\end{itemize}
\begin{itemize}
\item {Proveniência:(De \textunderscore capucho\textunderscore  e \textunderscore capuz\textunderscore )}
\end{itemize}
Frade da Capucha, ramificação da Ordem franciscana, fundada em 1525.
Capuz pequeno.
Diz-se do toiro que, desde a fronte á parte superior do pescoço, tem côr differente da do resto do corpo.
\section{Capucho}
\begin{itemize}
\item {Grp. gram.:m.  e  adj.}
\end{itemize}
\begin{itemize}
\item {Grp. gram.:M.}
\end{itemize}
\begin{itemize}
\item {Utilização:Prov.}
\end{itemize}
\begin{itemize}
\item {Utilização:minh.}
\end{itemize}
\begin{itemize}
\item {Grp. gram.:Loc. adv.}
\end{itemize}
\begin{itemize}
\item {Proveniência:(It. \textunderscore cappuccio\textunderscore )}
\end{itemize}
Frade franciscano.
Penitente austero.
Solitário.
\textunderscore Canto capucho\textunderscore , cantochão harmonizado a quatro vozes, e em que se distinguiram os frades capuchos da Arrábida.
Pequena mêda de centeio.
\textunderscore Á capucha\textunderscore , occultamente, modestamente.
\section{Capuco}
\begin{itemize}
\item {Grp. gram.:m.}
\end{itemize}
\begin{itemize}
\item {Utilização:Bras}
\end{itemize}
O mesmo que \textunderscore batuera\textunderscore .
\section{Capueira}
\begin{itemize}
\item {Grp. gram.:f.}
\end{itemize}
\begin{itemize}
\item {Utilização:Bras}
\end{itemize}
O mesmo ou melhor que \textunderscore capoeira\textunderscore ^2.
\section{Capulana}
\begin{itemize}
\item {Grp. gram.:f.}
\end{itemize}
\begin{itemize}
\item {Utilização:T. de Moçambique}
\end{itemize}
\begin{itemize}
\item {Grp. gram.:f.}
\end{itemize}
Pano, que os pretos trazem pendente do cínto, dobrando sôbre êste uma ponta, em que ás vezes guardam dinheiro.
Pano, com que os indígenas do sul de Moçambique, homens e mulheres, cobrem o corpo, desde a cintura até abaixo dos joêlhos.
\section{Capulho}
\begin{itemize}
\item {Grp. gram.:m.}
\end{itemize}
\begin{itemize}
\item {Proveniência:(De \textunderscore capa\textunderscore )}
\end{itemize}
Invólucro da flôr.
Cápsula, dentro da qual se fórma o algodão.
\section{Capungo-pungo}
\begin{itemize}
\item {Grp. gram.:m.}
\end{itemize}
Arbusto angolense, de apparência malvácea.
\section{Capuricenas}
\begin{itemize}
\item {Grp. gram.:m. pl.}
\end{itemize}
Indígenas da Guiana brasileira.
\section{Capuz}
\begin{itemize}
\item {Grp. gram.:m.}
\end{itemize}
\begin{itemize}
\item {Proveniência:(It. \textunderscore cappuccio\textunderscore )}
\end{itemize}
Cobertura de pano para a cabeça, e geralmente preso á capa, ao hábito ou ao casaco.
\section{Caquear}
\begin{itemize}
\item {Grp. gram.:v. i.}
\end{itemize}
\begin{itemize}
\item {Utilização:Prov.}
\end{itemize}
\begin{itemize}
\item {Utilização:alg.}
\end{itemize}
\begin{itemize}
\item {Grp. gram.:V. t.}
\end{itemize}
\begin{itemize}
\item {Utilização:Bras. do N}
\end{itemize}
\begin{itemize}
\item {Proveniência:(De \textunderscore caco\textunderscore )}
\end{itemize}
Matutar.
Procurar ás cegas; tactear.
\section{Caqueirada}
\begin{itemize}
\item {Grp. gram.:f.}
\end{itemize}
Grande porção de caqueiros ou cacos.
Reunião de objectos velhos ou inúteis.
Arremêsso de caqueiros; pancada com caqueiros.
\section{Caqueiro}
\begin{itemize}
\item {Grp. gram.:m.}
\end{itemize}
\begin{itemize}
\item {Proveniência:(De \textunderscore caco\textunderscore )}
\end{itemize}
Vaso de barro, muito usado, ou inútil, ou partido.
Chapéu velho.
\section{Caquelha}
\begin{itemize}
\item {fónica:quê}
\end{itemize}
\begin{itemize}
\item {Grp. gram.:m.  e  f.}
\end{itemize}
\begin{itemize}
\item {Utilização:Prov.}
\end{itemize}
\begin{itemize}
\item {Utilização:minh.}
\end{itemize}
Pessôa desbocada.
\section{Caquesseitão}
\begin{itemize}
\item {Grp. gram.:m.}
\end{itemize}
Animal disforme, de que fala a \textunderscore Peregrinação\textunderscore , c. XIV.
\section{Caquexia}
\begin{itemize}
\item {Grp. gram.:f.}
\end{itemize}
\begin{itemize}
\item {Proveniência:(Gr. \textunderscore kakhexia\textunderscore )}
\end{itemize}
Fraqueza geral do organismo.
Abatimento senil.
\section{Cáqui}
\begin{itemize}
\item {Grp. gram.:m.}
\end{itemize}
\begin{itemize}
\item {Utilização:Bras}
\end{itemize}
Fruto amarelo e comestível.
A arvoreta que o produz, (\textunderscore diospiros costata\textunderscore ).
É uma espécie de alperce.
A árvore, que o produz, diz-se \textunderscore cáqui\textunderscore  ou \textunderscore alperceiro-do-Japão\textunderscore .--A fórma \textunderscore kaki\textunderscore , incorrecta em português, provém do n. bot. \textunderscore diospyros kaki\textunderscore , Lin.
(Nos jornaes vejo \textunderscore kaki\textunderscore , cuja razão ou pretexto desconheço)
\section{Caquibosa}
\begin{itemize}
\item {Grp. gram.:f.}
\end{itemize}
Planta malvácea, (\textunderscore urena lofata\textunderscore , Lin.) da África portuguesa.
\section{Car}
\begin{itemize}
\item {Grp. gram.:conj.}
\end{itemize}
\begin{itemize}
\item {Utilização:Ant.}
\end{itemize}
Porquê; o mesmo que \textunderscore ca\textunderscore .
\section{Cara}
\begin{itemize}
\item {Grp. gram.:f.}
\end{itemize}
\begin{itemize}
\item {Utilização:Gír.}
\end{itemize}
\begin{itemize}
\item {Grp. gram.:Loc. adv.}
\end{itemize}
\begin{itemize}
\item {Proveniência:(Gr. \textunderscore kara\textunderscore )}
\end{itemize}
Parte anterior da cabeça, desde a testa ao queixo.
Rosto.
Semblante; apparência.
Ousadia: \textunderscore não teve cara para me apparecer\textunderscore .
Dois mil reis.
\textunderscore Cara a cara\textunderscore , frente a frente, na própria presença.
\section{Cará}
\begin{itemize}
\item {Grp. gram.:m.}
\end{itemize}
\begin{itemize}
\item {Utilização:Bras}
\end{itemize}
Peixe de água dôce.
Inhame.
Baile campestre, espécie de fandango.
\section{Caraá}
\begin{itemize}
\item {Grp. gram.:m.}
\end{itemize}
Planta gramínea do Brasil.
\section{Caraaçu}
\begin{itemize}
\item {Grp. gram.:m.}
\end{itemize}
Planta do Brasil, de raiz alimentícia.
\section{Caraba}
\begin{itemize}
\item {Grp. gram.:f.}
\end{itemize}
\begin{itemize}
\item {Utilização:Prov.}
\end{itemize}
O mesmo que \textunderscore carava\textunderscore .
\section{Caraba}
\begin{itemize}
\item {Grp. gram.:f.}
\end{itemize}
Espécie de cravo engastado de rubis e pérolas, com que as bailadeiras indianas enfeitam as orelhas.
(Conc. \textunderscore karaba\textunderscore )
\section{Carabana}
\begin{itemize}
\item {Grp. gram.:f.}
\end{itemize}
O mesmo que \textunderscore caravana\textunderscore . Cf. \textunderscore Luz e Calor\textunderscore , 366.
\section{Cará-barbado}
\begin{itemize}
\item {Grp. gram.:m.}
\end{itemize}
\begin{itemize}
\item {Utilização:Bras}
\end{itemize}
Espécie de batata.
\section{Carabelina}
\begin{itemize}
\item {Grp. gram.:f.}
\end{itemize}
\begin{itemize}
\item {Utilização:Prov.}
\end{itemize}
\begin{itemize}
\item {Utilização:trasm.}
\end{itemize}
O mesmo que \textunderscore cravelina\textunderscore .
\section{Carábicos}
\begin{itemize}
\item {Grp. gram.:m. pl.}
\end{itemize}
\begin{itemize}
\item {Proveniência:(De \textunderscore cárabo\textunderscore )}
\end{itemize}
Insectos carnívoros, da ordem dos coleópteros pentâmeros.
\section{Carabina}
\begin{itemize}
\item {Grp. gram.:f.}
\end{itemize}
Espingarda curta e estriada.
Espingarda.
(Cast. \textunderscore carabina\textunderscore )
\section{Carabinada}
\begin{itemize}
\item {Grp. gram.:f.}
\end{itemize}
Tiro de carabina.
\section{Carabineiro}
\begin{itemize}
\item {Grp. gram.:m.}
\end{itemize}
Soldado armado de carabina.
Fabricante de carabinas.
\section{Cárabo}
\begin{itemize}
\item {Grp. gram.:m.}
\end{itemize}
\begin{itemize}
\item {Proveniência:(Gr. \textunderscore karabos\textunderscore )}
\end{itemize}
Insecto coleóptero, que dá o nome á tríbo dos carábicos.
\section{Caraca}
\begin{itemize}
\item {Grp. gram.:f.}
\end{itemize}
Antigo navio português de 200 toneladas.
\section{Caraça}
\begin{itemize}
\item {Grp. gram.:f.}
\end{itemize}
\begin{itemize}
\item {Utilização:Pop.}
\end{itemize}
\begin{itemize}
\item {Grp. gram.:M.}
\end{itemize}
\begin{itemize}
\item {Proveniência:(De \textunderscore cara\textunderscore )}
\end{itemize}
O mesmo que \textunderscore máscara\textunderscore .
Boi ou cavallo, que tem malha branca no focinho.
\section{Çaraça}
\begin{itemize}
\item {Grp. gram.:f.}
\end{itemize}
\begin{itemize}
\item {Utilização:T. da Índia port}
\end{itemize}
Tecido ralo de algodão, saraça:«\textunderscore ...o mandou socorrer... com uma corja de çaraças, &amp; de panos Malayos pera sua molher e filhos\textunderscore ». \textunderscore Peregrinação\textunderscore , XXI.
O mesmo que \textunderscore cobertor\textunderscore .
(Cast. \textunderscore zaraza\textunderscore )
\section{Caracal}
\begin{itemize}
\item {Grp. gram.:m.}
\end{itemize}
Espécie de lynce ou gato selvagem.
\section{Caracará}
\begin{itemize}
\item {Grp. gram.:m.}
\end{itemize}
\begin{itemize}
\item {Utilização:Bras}
\end{itemize}
\begin{itemize}
\item {Proveniência:(T. tupi)}
\end{itemize}
Espécie de falcão americano.
Nome de varias aves de rapina.
\section{Caracas}
\begin{itemize}
\item {Grp. gram.:m.}
\end{itemize}
\begin{itemize}
\item {Proveniência:(De \textunderscore Caracas\textunderscore , n. p.)}
\end{itemize}
Vinho afamado de Venezuela.
\section{Caracaxá}
\begin{itemize}
\item {Grp. gram.:m.}
\end{itemize}
\begin{itemize}
\item {Utilização:Bras}
\end{itemize}
\begin{itemize}
\item {Proveniência:(T. onom.?)}
\end{itemize}
Espécie de chocalho, para entretenimento de crianças.
\section{Carachesco}
\begin{itemize}
\item {fónica:xês}
\end{itemize}
\begin{itemize}
\item {Grp. gram.:adj.}
\end{itemize}
Á maneira dos Carachos, pintores bolonheses, que fundaram uma escola do seu nome.
\section{Carachué}
\begin{itemize}
\item {Grp. gram.:m.}
\end{itemize}
\begin{itemize}
\item {Utilização:Bras. do N}
\end{itemize}
\begin{itemize}
\item {Utilização:Deprec.}
\end{itemize}
O mesmo que \textunderscore sabiá\textunderscore .
Indivíduo, que vive á custa de rameiras; rufião.
\section{Caracol}
\begin{itemize}
\item {Grp. gram.:m.}
\end{itemize}
Mollusco terrestre do gênero dos hélices.
Espiral; ziguezague.
Trança de cabello, enrolada em espiral.
Flôr do caracoleiro.
(Cast. \textunderscore caracol\textunderscore , talvez do lat. hyp. \textunderscore cochleolus\textunderscore , de \textunderscore cochleola\textunderscore )
\section{Caracolar}
\begin{itemize}
\item {Grp. gram.:v. i.}
\end{itemize}
\begin{itemize}
\item {Proveniência:(De \textunderscore caracol\textunderscore )}
\end{itemize}
Mover-se em espiral; andar, dando meias voltas, ora para a direita, ora para a esquerda.
\section{Caracolear}
\begin{itemize}
\item {Grp. gram.:v. i.}
\end{itemize}
O mesmo que \textunderscore caracolar\textunderscore : \textunderscore o cavallo caracoleia\textunderscore . Cf. Camillo, \textunderscore Canc. Alegre\textunderscore , 424.
\section{Caracoleiro}
\begin{itemize}
\item {Grp. gram.:m.}
\end{itemize}
\begin{itemize}
\item {Proveniência:(De \textunderscore caracol\textunderscore )}
\end{itemize}
Planta leguminosa e trepadeira, cuja flôr tem as pétalas em espiral.
\section{Carácter}
\begin{itemize}
\item {Proveniência:(Gr. \textunderscore karakter\textunderscore )}
\end{itemize}
\textunderscore m.\textunderscore  (\textunderscore Pl.\textunderscore  \textunderscore caractéres\textunderscore )
Cunho, marca, impressão traçada.
Cada uma das letras metállicas ou typos de imprensa.
Distintivo, especialidade.
Sinal de abreviatura, em algumas sciências.
Índole: \textunderscore pessôa de mau carácter\textunderscore .
Resolução, firmeza.
Expressão ajustada, propriedade.
Qualidade inherente a certos modos de sêr ou estados: \textunderscore a franqueza é carácter dos homens de bem\textunderscore .
\section{Característica}
\begin{itemize}
\item {Grp. gram.:f.}
\end{itemize}
\begin{itemize}
\item {Proveniência:(De \textunderscore característico\textunderscore )}
\end{itemize}
Aquillo que caracteriza.
\section{Caracteristicamente}
\begin{itemize}
\item {Grp. gram.:adv.}
\end{itemize}
De modo característico.
\section{Característico}
\begin{itemize}
\item {Grp. gram.:adj.}
\end{itemize}
\begin{itemize}
\item {Grp. gram.:M.}
\end{itemize}
\begin{itemize}
\item {Proveniência:(De \textunderscore carácter\textunderscore )}
\end{itemize}
Que caracteriza ou que distingue.
Aquillo que caracteriza; distintivo.
\section{Caracterização}
\begin{itemize}
\item {Grp. gram.:f.}
\end{itemize}
Acto ou effeito de caracterizar.
\section{Caracterizador}
\begin{itemize}
\item {Grp. gram.:m.  e  adj.}
\end{itemize}
O que caracteriza.
\section{Caracterizante}
\begin{itemize}
\item {Grp. gram.:adj.}
\end{itemize}
Que caracteriza.
\section{Caracterizar}
\begin{itemize}
\item {Grp. gram.:v. t.}
\end{itemize}
\begin{itemize}
\item {Proveniência:(De \textunderscore carácter\textunderscore )}
\end{itemize}
Tornar saliente, pôr em evidência, o carácter de; assignalar: \textunderscore aquella patifaria caracteriza o biltre\textunderscore .
Pintar e entrajar (o actor) para representar a personagem, a que corresponde em scena.
\section{Caracterologia}
\begin{itemize}
\item {Grp. gram.:f.}
\end{itemize}
\begin{itemize}
\item {Utilização:Bras}
\end{itemize}
\begin{itemize}
\item {Utilização:Neol.}
\end{itemize}
Sciência do carácter ou das relações da vontade com os motivos.
\section{Caracu}
\begin{itemize}
\item {Grp. gram.:m.}
\end{itemize}
\begin{itemize}
\item {Utilização:Bras}
\end{itemize}
\begin{itemize}
\item {Utilização:Bras}
\end{itemize}
\begin{itemize}
\item {Proveniência:(T. guar.)}
\end{itemize}
Medulla dos ossos do boi.
Gado vacum, de pêlo curto e liso.
\section{Cara-de-nó-cego}
\begin{itemize}
\item {Grp. gram.:m.}
\end{itemize}
\begin{itemize}
\item {Utilização:Pop.}
\end{itemize}
Indivíduo de má catadura, antipáthico.
\section{Cará-do-ar}
\begin{itemize}
\item {Grp. gram.:m.}
\end{itemize}
Planta trepadeira do Brasil.
\section{Caradrina}
\begin{itemize}
\item {Grp. gram.:f.}
\end{itemize}
\begin{itemize}
\item {Proveniência:(De \textunderscore Caradrino\textunderscore , n. p.)}
\end{itemize}
Insecto lepidóptero nocturno.
\section{Caradura}
\begin{itemize}
\item {Grp. gram.:f.}
\end{itemize}
\begin{itemize}
\item {Utilização:Bras. do Rio}
\end{itemize}
\begin{itemize}
\item {Grp. gram.:M.}
\end{itemize}
Bonde de terceira classe.
Homem desavergonhado.
\section{Caraeté}
\begin{itemize}
\item {fónica:cara-e}
\end{itemize}
\begin{itemize}
\item {Grp. gram.:m.}
\end{itemize}
Planta do Brasil.
\section{Carafate}
\begin{itemize}
\item {Grp. gram.:m.}
\end{itemize}
\begin{itemize}
\item {Utilização:Ant.}
\end{itemize}
O mesmo que \textunderscore calafate\textunderscore .
\section{Carafuzo}
\begin{itemize}
\item {Grp. gram.:m.}
\end{itemize}
\begin{itemize}
\item {Utilização:Bras}
\end{itemize}
Mestiço de negro e índio; caboré.
\section{Carago}
\begin{itemize}
\item {Grp. gram.:m.}
\end{itemize}
\begin{itemize}
\item {Grp. gram.:Interj.}
\end{itemize}
\begin{itemize}
\item {Utilização:Pleb.}
\end{itemize}
O mesmo que \textunderscore gallego\textunderscore :«\textunderscore o corpo colossal dos vis caragos\textunderscore ». Macedo, \textunderscore Burros\textunderscore , V, 30.
Peixe de Portugal.
(designativa de admiração, espanto, etc.)
(Do cast.)
\section{Caraguala}
\begin{itemize}
\item {Grp. gram.:f.}
\end{itemize}
Planta bromeliácea do Brasil.
\section{Caraguatá}
\begin{itemize}
\item {Grp. gram.:m.}
\end{itemize}
Planta filamentosa do Brasil.
\section{Caraíba}
\begin{itemize}
\item {Grp. gram.:m.}
\end{itemize}
\begin{itemize}
\item {Grp. gram.:Adj.}
\end{itemize}
Antiga língua das Antilhas.
Aquelle que era natural das Antilhas.
Relativo aos Caraíbas.
\section{Caraíba}
\begin{itemize}
\item {Grp. gram.:f.}
\end{itemize}
O mesmo que \textunderscore caraúba\textunderscore .
\section{Caraipe}
\begin{itemize}
\item {Grp. gram.:m.}
\end{itemize}
Árvore fructífera do Brasil, (\textunderscore licania microcarpa\textunderscore , Hook.).
\section{Caraísmo}
\begin{itemize}
\item {Grp. gram.:m.}
\end{itemize}
Seita dos caraítas.
\section{Caraítas}
\begin{itemize}
\item {Grp. gram.:m. pl.}
\end{itemize}
Sectários judeus, que só admittem os livros sagrados do antigo cânon, rejeitando o \textunderscore Talmud\textunderscore , as tradições, etc.
\section{Carajais}
\begin{itemize}
\item {Grp. gram.:m. pl.}
\end{itemize}
O mesmo que \textunderscore carajás\textunderscore .
\section{Carajás}
\begin{itemize}
\item {Grp. gram.:m. pl.}
\end{itemize}
Nome de várias tríbos índias das margens do Araguaia, no Brasil.
\section{Carajé}
\begin{itemize}
\item {Grp. gram.:m.}
\end{itemize}
\begin{itemize}
\item {Utilização:Bras}
\end{itemize}
Granjeia, com que se enfeita o pão de ló e doces.
\section{Caraju}
\begin{itemize}
\item {Grp. gram.:m.}
\end{itemize}
\begin{itemize}
\item {Utilização:Bras}
\end{itemize}
Espécie de batata.
\section{Carajuá}
\begin{itemize}
\item {Grp. gram.:m.}
\end{itemize}
Espécie de ave azul.
\section{Carajura}
\begin{itemize}
\item {Grp. gram.:m.}
\end{itemize}
\begin{itemize}
\item {Utilização:Bras. do N}
\end{itemize}
\begin{itemize}
\item {Proveniência:(T. tupi)}
\end{itemize}
Tinta vermelha, extrahida de um cipó e empregada nas artes.
Cipó, de que se extrai essa tinta.
\section{Carajuru}
\begin{itemize}
\item {Grp. gram.:m.}
\end{itemize}
Planta bignoniácea do Brasil.
\section{Caramanchão}
\begin{itemize}
\item {Grp. gram.:m.}
\end{itemize}
Edificação ligeira, formada de ripas, canas ou estacas, e revestida de trepadeiras, nos jardins.
Almenara.
(Metáth. de \textunderscore camaranchão\textunderscore )
\section{Caramanchel}
\begin{itemize}
\item {Grp. gram.:m.}
\end{itemize}
O mesmo que \textunderscore caramanchão\textunderscore .
\section{Caramba!}
\begin{itemize}
\item {Grp. gram.:interj.}
\end{itemize}
\begin{itemize}
\item {Utilização:Pop.}
\end{itemize}
\begin{itemize}
\item {Proveniência:(T. cast., euphemismo de um voc. obsceno)}
\end{itemize}
(indicativa de admiração ou ironia)
\section{Carambano}
\begin{itemize}
\item {Grp. gram.:m.}
\end{itemize}
\begin{itemize}
\item {Proveniência:(T. cast.)}
\end{itemize}
Bola de neve.
Ornato, com que se finge caramelo, e se guarnecem arcos rústicos de jardins, fontes, etc.
\section{Carambelo}
\begin{itemize}
\item {Grp. gram.:m.}
\end{itemize}
\begin{itemize}
\item {Utilização:T. de Barroso}
\end{itemize}
Carambina.
O mesmo que \textunderscore caramelo\textunderscore .
\section{Carambina}
\begin{itemize}
\item {Grp. gram.:f.}
\end{itemize}
\begin{itemize}
\item {Utilização:Prov.}
\end{itemize}
Gêlo pendente das árvores, dos penhascos, etc.; sincelo.
(Cp. \textunderscore carambano\textunderscore )
\section{Carambola}
\begin{itemize}
\item {Grp. gram.:f.}
\end{itemize}
\begin{itemize}
\item {Utilização:Prov.}
\end{itemize}
\begin{itemize}
\item {Utilização:alent.}
\end{itemize}
\begin{itemize}
\item {Proveniência:(T. cast.)}
\end{itemize}
Bóla vermelha do bilhar.
Acto de carambolar.
Tramóia, embuste.
Acto de matar duas perdizes com um só tiro.
\section{Carambola}
\begin{itemize}
\item {Grp. gram.:f.}
\end{itemize}
Fruto do caramboleiro.
Caramboleiro.
\section{Carambola}
\begin{itemize}
\item {Grp. gram.:f.}
\end{itemize}
Ave de arribação, que vem do Norte.
\section{Carambolar}
\begin{itemize}
\item {Grp. gram.:v. i.}
\end{itemize}
\begin{itemize}
\item {Proveniência:(De \textunderscore carambola\textunderscore ^1)}
\end{itemize}
Bater successivamente com uma bóla nas outras duas, ao bilhar.
Enganar, intrigar.
\section{Caramboleira}
\begin{itemize}
\item {Grp. gram.:f.}
\end{itemize}
O mesmo que \textunderscore caramboleiro\textunderscore .
\section{Caramboleiro}
\begin{itemize}
\item {Grp. gram.:m.}
\end{itemize}
\begin{itemize}
\item {Grp. gram.:Adj.}
\end{itemize}
\begin{itemize}
\item {Proveniência:(De \textunderscore carambola\textunderscore ^1)}
\end{itemize}
Planta oxalidácia, que comprehende duas espécies de árvores.
Embusteiro; intriguista.
\section{Carambolice}
\begin{itemize}
\item {Grp. gram.:f.}
\end{itemize}
\begin{itemize}
\item {Proveniência:(De \textunderscore carambola\textunderscore ^1)}
\end{itemize}
Lôgro, trapaça, embuste.
\section{Carambolim}
\begin{itemize}
\item {Grp. gram.:m.}
\end{itemize}
\begin{itemize}
\item {Utilização:Pop.}
\end{itemize}
\begin{itemize}
\item {Proveniência:(De \textunderscore carambola\textunderscore ^1)}
\end{itemize}
Perda simultânea de três paradas, no jôgo do monte.
\section{Caramburu}
\begin{itemize}
\item {Grp. gram.:m.}
\end{itemize}
\begin{itemize}
\item {Utilização:Bras}
\end{itemize}
Bebida refrigerante, o mesmo que \textunderscore aluá\textunderscore .
\section{Caramelejo}
\begin{itemize}
\item {Grp. gram.:m.}
\end{itemize}
\begin{itemize}
\item {Utilização:Prov.}
\end{itemize}
\begin{itemize}
\item {Utilização:alent.}
\end{itemize}
Oscillação ou tremor luminoso das camadas atmosphéricas, em dias de sol ardente.
\section{Caramelga}
\begin{itemize}
\item {Grp. gram.:f.}
\end{itemize}
Peixe selácio, espécie de arraia.
\section{Caramelo}
\begin{itemize}
\item {Grp. gram.:m.}
\end{itemize}
Gêlo.
Confeição de açúcar, coagulada e porosa.
Planta cucurbitácea.
(Talvez do lat. hyp. \textunderscore calamellus\textunderscore , dem. do lat. \textunderscore calamus\textunderscore )
\section{Caramiar}
\textunderscore v. i.\textunderscore  (e der.)
(Us. no Alentejo, por \textunderscore carmiar\textunderscore , etc.)
\section{Caramilho}
\begin{itemize}
\item {Grp. gram.:m.}
\end{itemize}
\begin{itemize}
\item {Utilização:Des.}
\end{itemize}
Questão de pouca monta.
Intriga.
(Cast. \textunderscore caramillo\textunderscore )
\section{Caraminguás}
\begin{itemize}
\item {Grp. gram.:m. pl.}
\end{itemize}
\begin{itemize}
\item {Utilização:Bras}
\end{itemize}
\begin{itemize}
\item {Proveniência:(Do guar. \textunderscore caramenguá\textunderscore )}
\end{itemize}
Objectos de pouco valor, que se levam em viagem.
Tarecos, designação modesta da mobília de uma casa.
\section{Caraminhola}
\begin{itemize}
\item {Grp. gram.:f.}
\end{itemize}
\begin{itemize}
\item {Utilização:Des.}
\end{itemize}
\begin{itemize}
\item {Proveniência:(De \textunderscore caramilho\textunderscore ?)}
\end{itemize}
Cabello entrançado no alto da cabeça; pôpa.
Guedelha.
Intriga; mentira.
\section{Caramôço}
\begin{itemize}
\item {Grp. gram.:m.}
\end{itemize}
\begin{itemize}
\item {Utilização:Prov.}
\end{itemize}
O mesmo que \textunderscore cramoiço\textunderscore .
\section{Caramoiço}
\begin{itemize}
\item {Grp. gram.:m.}
\end{itemize}
\begin{itemize}
\item {Utilização:Prov.}
\end{itemize}
O mesmo que \textunderscore cramoiço\textunderscore .
\section{Caramolo}
\begin{itemize}
\item {fónica:mô}
\end{itemize}
\begin{itemize}
\item {Grp. gram.:m.}
\end{itemize}
\begin{itemize}
\item {Utilização:T. de Lamego}
\end{itemize}
O mesmo que \textunderscore cramor\textunderscore .
(Por \textunderscore cramol\textunderscore , metáth. de \textunderscore clamor\textunderscore )
\section{Caramomom}
\begin{itemize}
\item {Grp. gram.:m.}
\end{itemize}
\begin{itemize}
\item {Utilização:Bras. do N}
\end{itemize}
Troixa, que se addiciona á carga regular de um animal.
\section{Caramono}
\begin{itemize}
\item {Grp. gram.:m.}
\end{itemize}
\begin{itemize}
\item {Utilização:Prov.}
\end{itemize}
\begin{itemize}
\item {Utilização:trasm.}
\end{itemize}
\begin{itemize}
\item {Proveniência:(De \textunderscore cara\textunderscore  + \textunderscore mono\textunderscore )}
\end{itemize}
Desenho tôsco de uma figura ou só de uma cabeça humana.
Cara feia de uma pessôa ou de uma imagem.
\section{Caramouço}
\begin{itemize}
\item {Grp. gram.:m.}
\end{itemize}
\begin{itemize}
\item {Utilização:Prov.}
\end{itemize}
O mesmo que \textunderscore cramoiço\textunderscore .
\section{Carampão}
\begin{itemize}
\item {Grp. gram.:m.}
\end{itemize}
\begin{itemize}
\item {Proveniência:(Fr. \textunderscore grampon\textunderscore )}
\end{itemize}
Uma das peças do prelo; grampo.
\section{Caramuçal}
\begin{itemize}
\item {Grp. gram.:m.}
\end{itemize}
Grande navio mercante, entre os Turcos.
\section{Caramucho}
\begin{itemize}
\item {Grp. gram.:m.}
\end{itemize}
\begin{itemize}
\item {Utilização:Prov.}
\end{itemize}
\begin{itemize}
\item {Utilização:minh.}
\end{itemize}
Maçaroca incompleta, meia maçaroca ou menos de meia.
\section{Caramujo}
\begin{itemize}
\item {Grp. gram.:m.}
\end{itemize}
Mollusco marítimo, univalve.
Espécie de couve.
Doença das salinas, produzida pela presença daquelle mollusco, e manifestada em fórma de petrificações brancas, vermelhas ou pretas.
\section{Caramuleiro}
\begin{itemize}
\item {Grp. gram.:adj.}
\end{itemize}
Relativo á região do Caramulo: \textunderscore bois caramuleiros\textunderscore .
\section{Caramunha}
\begin{itemize}
\item {Grp. gram.:f.}
\end{itemize}
\begin{itemize}
\item {Proveniência:(Do lat. \textunderscore querimonia\textunderscore )}
\end{itemize}
Chôro de crianças.
Queixa, lamúria: \textunderscore fazer o mal e a caramunha\textunderscore .
\section{Caramunhado}
\begin{itemize}
\item {Grp. gram.:adj.}
\end{itemize}
Em que há caramunha; acompanhado de caramunha: \textunderscore um pedido caramunhado\textunderscore .
\section{Caramunhar}
\begin{itemize}
\item {Grp. gram.:v. i.}
\end{itemize}
Fazer caramunha; lastimar-se.
\section{Caramuri}
\begin{itemize}
\item {Grp. gram.:m.}
\end{itemize}
Fruta silvestre das matas do Amazonas.
\section{Caramuru}
\begin{itemize}
\item {Grp. gram.:m.}
\end{itemize}
\begin{itemize}
\item {Utilização:Bras}
\end{itemize}
\begin{itemize}
\item {Utilização:ant.}
\end{itemize}
\begin{itemize}
\item {Utilização:Bras}
\end{itemize}
\begin{itemize}
\item {Utilização:mod.}
\end{itemize}
Europeu.
Espécie de peixe.
(Do tupi)
\section{Caraná}
\begin{itemize}
\item {Grp. gram.:f.}
\end{itemize}
Espécie de palmeira do Brasil.
\section{Caranchona}
\begin{itemize}
\item {Grp. gram.:f.}
\end{itemize}
\begin{itemize}
\item {Utilização:Prov.}
\end{itemize}
\begin{itemize}
\item {Utilização:trasm.}
\end{itemize}
Máscara horrenda.
O mesmo que \textunderscore carantonha\textunderscore .
\section{Carandá}
\begin{itemize}
\item {Grp. gram.:f.}
\end{itemize}
Espécie de palmeira do Brasil.
\section{Carandaí}
\begin{itemize}
\item {Grp. gram.:m.}
\end{itemize}
Espécie de palmeira do Brasil.
\section{Caranga}
\begin{itemize}
\item {Grp. gram.:f.}
\end{itemize}
Espécie de peixe marítimo.
Carango^1.
\section{Caranganho}
\begin{itemize}
\item {Grp. gram.:m.}
\end{itemize}
O mesmo que \textunderscore engaço\textunderscore .
\section{Carango}
\begin{itemize}
\item {Grp. gram.:m.}
\end{itemize}
Arbusto de Moçambique.
\section{Carango}
\begin{itemize}
\item {Grp. gram.:m.}
\end{itemize}
\begin{itemize}
\item {Utilização:Chul.}
\end{itemize}
\begin{itemize}
\item {Proveniência:(Do rad. de \textunderscore caranguejo\textunderscore ?)}
\end{itemize}
Piolho.
\section{Carangueja}
\begin{itemize}
\item {Grp. gram.:f.}
\end{itemize}
\begin{itemize}
\item {Utilização:Náut.}
\end{itemize}
\begin{itemize}
\item {Utilização:Des.}
\end{itemize}
\begin{itemize}
\item {Proveniência:(De \textunderscore caranguejo\textunderscore )}
\end{itemize}
Vêrga da vela grande, latina, em navios de dois mastros.
Vêrga de mezena, em navios de três mastros.
Espécie de ponte, em que assenta o combóio, que vem de uma linha, para entrar noutra, e que também se chama \textunderscore caranguejo\textunderscore .
Cancro.
\section{Caranguejar}
\begin{itemize}
\item {Grp. gram.:v. i.}
\end{itemize}
\begin{itemize}
\item {Utilização:Pop.}
\end{itemize}
Andar como o caranguejo, vagarosamente.
Hesitar.
\section{Caranguejeira}
\begin{itemize}
\item {Grp. gram.:f.}
\end{itemize}
\begin{itemize}
\item {Proveniência:(De \textunderscore caranguejo\textunderscore )}
\end{itemize}
Grande aranha do Brasil.
\section{Caranguejeira}
\begin{itemize}
\item {Grp. gram.:f.}
\end{itemize}
\begin{itemize}
\item {Proveniência:(Do ingl. \textunderscore green\textunderscore  + \textunderscore Gage\textunderscore , n. p.)}
\end{itemize}
Variedade de ameixa, também conhecida pelo nome de \textunderscore rainha-cláudia\textunderscore .
\section{Caranguejeiro}
\begin{itemize}
\item {Grp. gram.:m.}
\end{itemize}
Aquelle que apanha caranguejos.
\section{Caranguejo}
\begin{itemize}
\item {Grp. gram.:m.}
\end{itemize}
\begin{itemize}
\item {Utilização:Prov.}
\end{itemize}
\begin{itemize}
\item {Utilização:minh.}
\end{itemize}
\begin{itemize}
\item {Utilização:Des.}
\end{itemize}
\begin{itemize}
\item {Utilização:Fig.}
\end{itemize}
\begin{itemize}
\item {Utilização:T. de Lamego}
\end{itemize}
Animal crustáceo, de que há espécies comestíveis.
Abrunho grande.
Signo de Câncer.
Plataforma, que se move sôbre rodas, para deslocar vagões.
Variedade de ameixa, o mesmo que \textunderscore caranguejeira\textunderscore ^2.
(Corr. de \textunderscore cancrejo\textunderscore )
\section{Caranguejola}
\begin{itemize}
\item {Grp. gram.:f.}
\end{itemize}
\begin{itemize}
\item {Proveniência:(De \textunderscore caranguejo\textunderscore )}
\end{itemize}
Grande crustáceo, semelhante ao caranguejo.
Armação de madeira, que tem pouca solidez.
Acervo de coisas sobrepostas e mal seguras.
\section{Caranguejolar}
\begin{itemize}
\item {Grp. gram.:v. i.}
\end{itemize}
\begin{itemize}
\item {Utilização:Neol.}
\end{itemize}
\begin{itemize}
\item {Proveniência:(De \textunderscore caranguejola\textunderscore )}
\end{itemize}
Oscillar, como uma armação, pouco sólida, de madeira, ou como um acervo de coisas mal sobrepostas ou mal seguras.
\section{Caranha}
\begin{itemize}
\item {Grp. gram.:f.}
\end{itemize}
O mesmo que \textunderscore caraná\textunderscore .
Peixe marítimo do Brasil.
\section{Carantonha}
\begin{itemize}
\item {Grp. gram.:f.}
\end{itemize}
Cara feia; carranca.
Caraça.
Esgar.
(Cast. \textunderscore carantoña\textunderscore )
\section{Carântulas}
\begin{itemize}
\item {Grp. gram.:f. pl.}
\end{itemize}
\begin{itemize}
\item {Utilização:Ant.}
\end{itemize}
Cifras, caracteres ou imagens, de que se serviam os feiticeiros.
\section{Carão}
\begin{itemize}
\item {Grp. gram.:m.}
\end{itemize}
\begin{itemize}
\item {Utilização:Bras}
\end{itemize}
\begin{itemize}
\item {Utilização:Bras}
\end{itemize}
\begin{itemize}
\item {Proveniência:(De \textunderscore cara\textunderscore )}
\end{itemize}
Cara grande.
Epiderme do rosto:«\textunderscore carão moreno\textunderscore ». Bocage.
Espécie de macaco das regiões do Amazonas.
Ave dos paúes.
\section{Carão}
\begin{itemize}
\item {Grp. gram.:m.}
\end{itemize}
\begin{itemize}
\item {Utilização:Bras}
\end{itemize}
Reprehensão, dada em público a uma criança.
\section{Carão}
\begin{itemize}
\item {Grp. gram.:f.}
\end{itemize}
\begin{itemize}
\item {Utilização:Prov.}
\end{itemize}
\begin{itemize}
\item {Utilização:alg.}
\end{itemize}
Us. na loc. prep. \textunderscore á carão de\textunderscore , o mesmo que \textunderscore á carel de\textunderscore .(V.carel)
\section{Carão}
\begin{itemize}
\item {Grp. gram.:m.}
\end{itemize}
\begin{itemize}
\item {Utilização:Prov.}
\end{itemize}
\begin{itemize}
\item {Utilização:trasm.}
\end{itemize}
Parte comestível da noz.
\section{Carapaça}
\begin{itemize}
\item {Grp. gram.:f.}
\end{itemize}
\begin{itemize}
\item {Utilização:Gal}
\end{itemize}
\begin{itemize}
\item {Proveniência:(Fr. \textunderscore carapace\textunderscore )}
\end{itemize}
A cobertura córnea das tartarugas.--Em bom português, diz-se \textunderscore casca\textunderscore .
\section{Carapaná}
\begin{itemize}
\item {Grp. gram.:m.}
\end{itemize}
O mesmo que \textunderscore carapanan\textunderscore .
\section{Carapanan}
\begin{itemize}
\item {Grp. gram.:m.}
\end{itemize}
\begin{itemize}
\item {Utilização:Bras. do N}
\end{itemize}
\begin{itemize}
\item {Proveniência:(T. tupi)}
\end{itemize}
Espécie de mosquito de longas pernas.
\section{Carapanta}
\begin{itemize}
\item {Grp. gram.:f.}
\end{itemize}
\begin{itemize}
\item {Utilização:Pleb.}
\end{itemize}
O mesmo que \textunderscore carraspana\textunderscore .
\section{Carapatento}
\begin{itemize}
\item {Grp. gram.:adj.}
\end{itemize}
\begin{itemize}
\item {Utilização:Ant.}
\end{itemize}
Chagoso, repugnante, por \textunderscore carrapatento\textunderscore , de \textunderscore carrapata\textunderscore ?
Ou bêbedo, por \textunderscore carapantento\textunderscore , de \textunderscore carapanta\textunderscore ? Cf. G. Vicente, \textunderscore Inês Pereira\textunderscore .
\section{Carapau}
\begin{itemize}
\item {Grp. gram.:m.}
\end{itemize}
\begin{itemize}
\item {Utilização:Pleb.}
\end{itemize}
Pequeno peixe marítimo; pequeno chicharro.
Pessôa muito magra.
\section{Carapeba}
\begin{itemize}
\item {fónica:pê}
\end{itemize}
\begin{itemize}
\item {Grp. gram.:f.}
\end{itemize}
Peixe do Brasil.
\section{Carapeirana}
\begin{itemize}
\item {Grp. gram.:f.}
\end{itemize}
Planta rosácea da América.
\section{Carapela}
\begin{itemize}
\item {Grp. gram.:f.}
\end{itemize}
Folhelho, pellícula, que envolve a espiga do milho.
(Alter. de \textunderscore carpela\textunderscore )
\section{Carapêta}
\begin{itemize}
\item {Grp. gram.:f.}
\end{itemize}
\begin{itemize}
\item {Utilização:Pleb.}
\end{itemize}
Bolota ou espécie de pião, que os rapazes fazem girar, movendo-o com os dedos.
Maçaneta.
Mentira leve.
Porção de muco coagulado e adherente ás paredes das fossas nasaes.
\section{Carapéta}
\begin{itemize}
\item {Grp. gram.:f.}
\end{itemize}
\begin{itemize}
\item {Utilização:Bras}
\end{itemize}
Árvore meliácea dos sertões.
\section{Carapetal}
\begin{itemize}
\item {Grp. gram.:m.}
\end{itemize}
Saco, em que levam mantimentos os pretos que viajam no sertão.
\section{Carapetão}
\begin{itemize}
\item {Grp. gram.:m.}
\end{itemize}
\begin{itemize}
\item {Proveniência:(De \textunderscore carapêta\textunderscore )}
\end{itemize}
Mentira grande.
\section{Carapetar}
\begin{itemize}
\item {Grp. gram.:v. i.}
\end{itemize}
Dizer carapetões.
\section{Carapeteiro}
\begin{itemize}
\item {Grp. gram.:adj.}
\end{itemize}
\begin{itemize}
\item {Grp. gram.:M.}
\end{itemize}
\begin{itemize}
\item {Proveniência:(De \textunderscore carapêta\textunderscore )}
\end{itemize}
Mentiroso.
Aquelle que mente.
Espécie de pereira brava.
\section{Carapetento}
\begin{itemize}
\item {Grp. gram.:adj.}
\end{itemize}
\begin{itemize}
\item {Utilização:Ant.}
\end{itemize}
O mesmo que \textunderscore carapeteiro\textunderscore .
\section{Carapeto}
\begin{itemize}
\item {fónica:pê}
\end{itemize}
\begin{itemize}
\item {Grp. gram.:m.}
\end{itemize}
\begin{itemize}
\item {Utilização:T. do Fundão}
\end{itemize}
(V.contra-erva)
O mesmo que \textunderscore espinho\textunderscore .
\section{Carapiá}
\begin{itemize}
\item {Grp. gram.:f.}
\end{itemize}
Planta medicinal do Brasil.
\section{Carapicu}
\begin{itemize}
\item {Grp. gram.:m.}
\end{itemize}
Peixe do Brasil.
\section{Carapim}
\begin{itemize}
\item {Grp. gram.:m.}
\end{itemize}
Pequeno saco de liga, com que se começa a fabricação dos sapatos de liga. Cf. \textunderscore Inquér. Industr.\textunderscore , 2.^a p., l. I, 169.
\section{Carapina}
\begin{itemize}
\item {Grp. gram.:m.}
\end{itemize}
\begin{itemize}
\item {Utilização:Bras}
\end{itemize}
\begin{itemize}
\item {Proveniência:(T. tupi)}
\end{itemize}
Carpinteiro.
\section{Carapinha}
\begin{itemize}
\item {Grp. gram.:f.}
\end{itemize}
\begin{itemize}
\item {Utilização:Prov.}
\end{itemize}
\begin{itemize}
\item {Utilização:alent.}
\end{itemize}
Cabello crespo e lanoso.
Ovário da esteva, depois de caídas as pétalas.
\section{Carapinhada}
\begin{itemize}
\item {Grp. gram.:f.}
\end{itemize}
\begin{itemize}
\item {Proveniência:(De \textunderscore carapinha\textunderscore )}
\end{itemize}
Bebida congelada, que fórma frocos.
\section{Carapinheira}
\begin{itemize}
\item {Grp. gram.:f.}
\end{itemize}
\begin{itemize}
\item {Proveniência:(De \textunderscore Carapinheira\textunderscore , n. p. de localidade)}
\end{itemize}
Variedade de pêra sumarenta.
\section{Carapinho}
\begin{itemize}
\item {Grp. gram.:adj.}
\end{itemize}
Crespo, encrespado:«\textunderscore cabellos carapinhos\textunderscore ». Castilho, \textunderscore Avarento\textunderscore , II, sc. 7.
\section{Carapinhudo}
\begin{itemize}
\item {Grp. gram.:adj.}
\end{itemize}
Que tem carapinha.
\section{Carapinima}
\begin{itemize}
\item {Grp. gram.:f.}
\end{itemize}
Árvore do Brasil.
\section{Carapintina}
\begin{itemize}
\item {Grp. gram.:f.}
\end{itemize}
\begin{itemize}
\item {Utilização:Prov.}
\end{itemize}
\begin{itemize}
\item {Utilização:alg.}
\end{itemize}
O mesmo que \textunderscore carpintina\textunderscore .
\section{Carapira}
\begin{itemize}
\item {Grp. gram.:f.}
\end{itemize}
\begin{itemize}
\item {Utilização:Bras}
\end{itemize}
Ave ribeirinha das regiões do Amazonas.
\section{Carapitaia}
\begin{itemize}
\item {Grp. gram.:f.}
\end{itemize}
Planta liliácea do Brasil.
\section{Carapitanga}
\begin{itemize}
\item {Grp. gram.:f.}
\end{itemize}
Bom peixe marítimo de Pernambuco.
\section{Carapito}
\begin{itemize}
\item {Grp. gram.:m.}
\end{itemize}
\begin{itemize}
\item {Utilização:Prov.}
\end{itemize}
\begin{itemize}
\item {Utilização:trasm.}
\end{itemize}
Espécie de picapau ou trepadeira, ave.
\section{Carapó}
\begin{itemize}
\item {Grp. gram.:m.}
\end{itemize}
Enguia eléctrica do Brasil.
(Relaciona-se com \textunderscore carapau\textunderscore ?)
\section{Carapobeba}
\begin{itemize}
\item {fónica:bê}
\end{itemize}
\begin{itemize}
\item {Grp. gram.:f.}
\end{itemize}
Lagarto venenoso do Brasil.
\section{Carapotós}
\begin{itemize}
\item {Grp. gram.:m. pl.}
\end{itemize}
Tríbo de Índios Cairiris, nas Alagôas, (Brasil).
\section{Carapuça}
\begin{itemize}
\item {Grp. gram.:f.}
\end{itemize}
\begin{itemize}
\item {Proveniência:(Do cast. \textunderscore caperuza\textunderscore )}
\end{itemize}
Barrete de fórma cónica.
Designação de vários objectos de fórma semelhante á da carapuça.
\section{Carapuçada}
\begin{itemize}
\item {Grp. gram.:f.}
\end{itemize}
Porção de objectos que enchem uma carapuça.
\section{Carapução}
\begin{itemize}
\item {Grp. gram.:m.}
\end{itemize}
Carapuça grande.
\section{Carapuceiro}
\begin{itemize}
\item {Grp. gram.:m.}
\end{itemize}
Vendedor ou fabricante de carapuças.
\section{Carapuço}
\begin{itemize}
\item {Grp. gram.:m.}
\end{itemize}
\begin{itemize}
\item {Proveniência:(De \textunderscore carapuça\textunderscore )}
\end{itemize}
Carapuça.
Saco pequeno, em que se filtra a infusão do café.
\section{Carapulo}
\begin{itemize}
\item {Grp. gram.:m.}
\end{itemize}
Cálice da bolota e de frutos semelhantes.
\section{Carará}
\begin{itemize}
\item {Grp. gram.:m.}
\end{itemize}
\begin{itemize}
\item {Utilização:Bras}
\end{itemize}
Ave ribeirinha.
\section{Cara-raiada}
\begin{itemize}
\item {Grp. gram.:m.}
\end{itemize}
Espécie de macaco.
\section{Carato}
\begin{itemize}
\item {Grp. gram.:m.}
\end{itemize}
Árvore da América, de que se extrai uma espécie de cânfora.
\section{Carátules}
\begin{itemize}
\item {Grp. gram.:m. pl.}
\end{itemize}
(corr. ant. de \textunderscore carácteres\textunderscore , no sentido de typos ou letras de imprensa)
\section{Carauá}
\begin{itemize}
\item {Grp. gram.:m.}
\end{itemize}
\begin{itemize}
\item {Utilização:Bras}
\end{itemize}
O mesmo que \textunderscore caroá\textunderscore .
\section{Caraúba}
\begin{itemize}
\item {Grp. gram.:f.}
\end{itemize}
\begin{itemize}
\item {Utilização:Bras}
\end{itemize}
\begin{itemize}
\item {Proveniência:(T. tupi)}
\end{itemize}
Árvore, de casca amargosa e fôlhas amarelas.
\section{Caraúna}
\begin{itemize}
\item {Grp. gram.:f.}
\end{itemize}
Ave do Brasil.
\section{Caraúno}
\begin{itemize}
\item {Grp. gram.:adj.}
\end{itemize}
\begin{itemize}
\item {Utilização:Bras}
\end{itemize}
Diz-se do boi muito preto.
\section{Carava}
\begin{itemize}
\item {Grp. gram.:f.}
\end{itemize}
\begin{itemize}
\item {Utilização:Prov.}
\end{itemize}
\begin{itemize}
\item {Utilização:trasm.}
\end{itemize}
\begin{itemize}
\item {Utilização:beir.}
\end{itemize}
Companhia, reúnião, súcia.
(Cp. \textunderscore caravana\textunderscore )
\section{Caravana}
\begin{itemize}
\item {Grp. gram.:f.}
\end{itemize}
\begin{itemize}
\item {Utilização:Ext.}
\end{itemize}
Multidão de peregrinos, mercadores ou viajantes, que se juntam, para atravessar os desertos com segurança.
Reúnião de pessôas, que viajam ou passeiam juntas.
Guerra marítima, que os Cavalleiros de Malta eram obrigados a fazer contra os Turcos ou corsários.
(Ár. \textunderscore cairauan\textunderscore , do pers. \textunderscore caruan\textunderscore )
\section{Caravaneiro}
\begin{itemize}
\item {Grp. gram.:m.}
\end{itemize}
Guía de caravanas.
\section{Caravansará}
\begin{itemize}
\item {Grp. gram.:m.}
\end{itemize}
\begin{itemize}
\item {Proveniência:(Do pers. \textunderscore karuan\textunderscore  + \textunderscore sarai\textunderscore )}
\end{itemize}
Grande edifício, para hospedagem gratuita das caravanas.
\section{Caravansarai}
\begin{itemize}
\item {Grp. gram.:m.}
\end{itemize}
\begin{itemize}
\item {Proveniência:(Do pers. \textunderscore karuan\textunderscore  + \textunderscore sarai\textunderscore )}
\end{itemize}
Grande edifício, para hospedagem gratuita das caravanas.
\section{Caravanseralho}
\begin{itemize}
\item {Grp. gram.:m.}
\end{itemize}
(V.caravansará)
\section{Caravatá}
\begin{itemize}
\item {Grp. gram.:m.}
\end{itemize}
(V.caraguatá)
\section{Caraveiro}
\begin{itemize}
\item {Grp. gram.:m.}
\end{itemize}
\begin{itemize}
\item {Utilização:Prov.}
\end{itemize}
\begin{itemize}
\item {Utilização:trasm.}
\end{itemize}
Aquelle que gosta ou faz parte de caravas.
\section{Caravela}
\begin{itemize}
\item {Grp. gram.:f.}
\end{itemize}
\begin{itemize}
\item {Utilização:Fig.}
\end{itemize}
\begin{itemize}
\item {Utilização:Prov.}
\end{itemize}
\begin{itemize}
\item {Utilização:minh.}
\end{itemize}
\begin{itemize}
\item {Proveniência:(De \textunderscore cáravo\textunderscore )}
\end{itemize}
Pequena embarcação de velas latinas.
Moéda antiga de prata, equivalente a doze vintens.
Gorgeta, gratificação em dinheiro.
\textunderscore Sardinha de caravela\textunderscore , sardinha salgada.
Nome de um animálculo celenterado, (\textunderscore physalia pelágica\textunderscore ), marinho e fluctuante.
\section{Caravela}
\begin{itemize}
\item {Grp. gram.:f.}
\end{itemize}
\begin{itemize}
\item {Utilização:Prov.}
\end{itemize}
\begin{itemize}
\item {Utilização:T. do Fundão}
\end{itemize}
Catavento, munido de um vaso de fôlha, em que bate uma peça movida pelo vento, e destinado a espantar das searas, hortas e pomares, as aves nocivas.
Brinquedo de crianças, formado por uma canafrecha ou qualquer haste, com um papel na extremidade, em fórma de velas de moínho, o qual se agita e gira com o vento.
\section{Caravelão}
\begin{itemize}
\item {Grp. gram.:m.}
\end{itemize}
\begin{itemize}
\item {Utilização:Ant.}
\end{itemize}
\begin{itemize}
\item {Utilização:Des.}
\end{itemize}
\begin{itemize}
\item {Proveniência:(De \textunderscore caravela\textunderscore ^1)}
\end{itemize}
Embarcação semelhante aos navios de pesca de Lisbôa.
Mulher alta e magra.
\section{Caraveleiro}
\begin{itemize}
\item {Grp. gram.:m.}
\end{itemize}
Tripulante de caravelas.
\section{Caravelha}
\begin{itemize}
\item {fónica:vê}
\end{itemize}
\begin{itemize}
\item {Grp. gram.:f.}
\end{itemize}
(V.cravelha)
\section{Caravelho}
\begin{itemize}
\item {fónica:vê}
\end{itemize}
\begin{itemize}
\item {Grp. gram.:m.}
\end{itemize}
(V.cravelho)
\section{Caravenho}
\begin{itemize}
\item {Grp. gram.:adj.}
\end{itemize}
O mesmo que \textunderscore estevado\textunderscore .
\section{Cáravo}
\begin{itemize}
\item {Grp. gram.:m.}
\end{itemize}
\begin{itemize}
\item {Utilização:Ant.}
\end{itemize}
\begin{itemize}
\item {Proveniência:(Lat. \textunderscore carabus\textunderscore )}
\end{itemize}
Embarcação asiática de vela latina.
\section{Caravonada}
\begin{itemize}
\item {Grp. gram.:f.}
\end{itemize}
\begin{itemize}
\item {Utilização:Des.}
\end{itemize}
Talhadinhas de carne, assadas em grelha.
(Por \textunderscore carvonada\textunderscore , de \textunderscore carvão\textunderscore )
\section{Carbólico}
\begin{itemize}
\item {Grp. gram.:adj.}
\end{itemize}
Diz-se do ácido, mais vulgarmente chamado \textunderscore phênico\textunderscore .
\section{Carbonado}
\begin{itemize}
\item {Grp. gram.:adj.}
\end{itemize}
Que contém \textunderscore carbóne\textunderscore .
\section{Carbonaria}
\begin{itemize}
\item {Grp. gram.:f.}
\end{itemize}
Sociedade dos carbonários.
\section{Carbonário}
\begin{itemize}
\item {Grp. gram.:m.}
\end{itemize}
\begin{itemize}
\item {Proveniência:(It. \textunderscore carbonaro\textunderscore , carvoeiro)}
\end{itemize}
Aquelle que pertencia a certa sociedade secreta e revolucionária, na Italia.
Membro de qualquer sociedade secreta revolucionária.
\section{Carbonarismo}
\begin{itemize}
\item {Grp. gram.:m.}
\end{itemize}
Doutrina dos carbonários.
\section{Carbonatado}
\begin{itemize}
\item {Grp. gram.:adj.}
\end{itemize}
\begin{itemize}
\item {Proveniência:(De \textunderscore carbonar\textunderscore )}
\end{itemize}
Combinado com ácido carbónico.
\section{Carbonatar}
\begin{itemize}
\item {Grp. gram.:v. i.}
\end{itemize}
\begin{itemize}
\item {Proveniência:(De \textunderscore carbonato\textunderscore )}
\end{itemize}
Combinar com ácido carbónico.
Converter em carbonato.
\section{Carbonato}
\begin{itemize}
\item {Grp. gram.:m.}
\end{itemize}
\begin{itemize}
\item {Proveniência:(De \textunderscore carbóne\textunderscore )}
\end{itemize}
Combinação do ácido carbónico com uma base.
\section{Carbóne}
\begin{itemize}
\item {Grp. gram.:m.}
\end{itemize}
\begin{itemize}
\item {Utilização:Chím.}
\end{itemize}
\begin{itemize}
\item {Proveniência:(Do lat. \textunderscore carbo\textunderscore , \textunderscore carbonis\textunderscore )}
\end{itemize}
Corpo simples, que se não póde decompor, e constitue o carvão, o diamante, a madeira, etc.
\section{Carboneto}
\begin{itemize}
\item {Grp. gram.:m.}
\end{itemize}
\begin{itemize}
\item {Proveniência:(De \textunderscore carbóne\textunderscore )}
\end{itemize}
(Palavra, que deve substituír o vulgar \textunderscore carbureto\textunderscore )
\section{Carbónico}
\begin{itemize}
\item {Grp. gram.:adj.}
\end{itemize}
\begin{itemize}
\item {Utilização:Gír.}
\end{itemize}
\begin{itemize}
\item {Proveniência:(De \textunderscore carbóne\textunderscore )}
\end{itemize}
Diz-se de um ácido, formado pelo carbóne e pelo oxigênio.
Diz-se de um dos cinco períodos da época primária, entre o devónico e o pérmico.
\section{Carbonífero}
\begin{itemize}
\item {Grp. gram.:adj.}
\end{itemize}
\begin{itemize}
\item {Proveniência:(Do lat. \textunderscore carbo\textunderscore  + \textunderscore ferre\textunderscore )}
\end{itemize}
Que produz carvão.
\section{Carbónio}
\begin{itemize}
\item {Grp. gram.:m.}
\end{itemize}
(V.carbóne)
\section{Carbonito}
\begin{itemize}
\item {Grp. gram.:m.}
\end{itemize}
\begin{itemize}
\item {Proveniência:(De \textunderscore carbóne\textunderscore )}
\end{itemize}
Combinação do ácido carbonoso com uma base.
\section{Carbonização}
\begin{itemize}
\item {Grp. gram.:f.}
\end{itemize}
Acto de carbonizar.
\section{Carbonizado}
\begin{itemize}
\item {Grp. gram.:adj.}
\end{itemize}
\begin{itemize}
\item {Proveniência:(De \textunderscore carbonizar\textunderscore )}
\end{itemize}
Reduzido a carvão.
\section{Carbonizador}
\begin{itemize}
\item {Grp. gram.:adj.}
\end{itemize}
\begin{itemize}
\item {Grp. gram.:M.}
\end{itemize}
Que carboniza.
Apparelho ou retorta para fazer da madeira carvão.
\section{Carbonizar}
\begin{itemize}
\item {Grp. gram.:v. t.}
\end{itemize}
\begin{itemize}
\item {Proveniência:(Do lat. \textunderscore carbo\textunderscore , \textunderscore carbonis\textunderscore )}
\end{itemize}
Reduzir a carvão.
\section{Carbonizável}
\begin{itemize}
\item {Grp. gram.:adj.}
\end{itemize}
\begin{itemize}
\item {Proveniência:(De \textunderscore carbonizar\textunderscore )}
\end{itemize}
Que póde reduzir-se a carvão.
\section{Carbóno}
\begin{itemize}
\item {Grp. gram.:m.}
\end{itemize}
(V.carbóne)
\section{Carbonometria}
\begin{itemize}
\item {Grp. gram.:f.}
\end{itemize}
\begin{itemize}
\item {Proveniência:(Do lat. \textunderscore carbo\textunderscore  + gr. \textunderscore metron\textunderscore )}
\end{itemize}
Determinação da quantidade de ácido carbónico, expellido pelos pulmões, quando se respira.
\section{Carbonométrico}
\begin{itemize}
\item {Grp. gram.:adj.}
\end{itemize}
Relativo á carbonometria.
\section{Carbonoso}
\begin{itemize}
\item {Grp. gram.:adj.}
\end{itemize}
\begin{itemize}
\item {Proveniência:(De \textunderscore carbóne\textunderscore )}
\end{itemize}
Diz-se do ácido, que também se chama oxálico.
E diz-se de uma variedade de rochas, que constituem o carvão natural.
\section{Carbossulfureto}
\begin{itemize}
\item {fónica:furê}
\end{itemize}
\begin{itemize}
\item {Grp. gram.:m.}
\end{itemize}
\begin{itemize}
\item {Proveniência:(De \textunderscore carbóne\textunderscore  + \textunderscore sulfureto\textunderscore )}
\end{itemize}
Corpo, composto de carbóne e enxofre.
\section{Carbosulfureto}
\begin{itemize}
\item {fónica:sul}
\end{itemize}
\begin{itemize}
\item {Grp. gram.:m.}
\end{itemize}
\begin{itemize}
\item {Proveniência:(De \textunderscore carbóne\textunderscore  + \textunderscore sulfureto\textunderscore )}
\end{itemize}
Corpo, composto de carbóne e enxofre.
\section{Carboxilato}
\begin{itemize}
\item {Grp. gram.:m.}
\end{itemize}
Uma das ordens dos mineraes carbonados, que comprehende os saes de ácidos orgânicos.
\section{Carboxylato}
\begin{itemize}
\item {Grp. gram.:m.}
\end{itemize}
Uma das ordens dos mineraes carbonados, que comprehende os saes de ácidos orgânicos.
\section{Carbúnculo}
\begin{itemize}
\item {Grp. gram.:m.}
\end{itemize}
\begin{itemize}
\item {Proveniência:(Lat. \textunderscore carbunculus\textunderscore )}
\end{itemize}
Anthraz.
Rubim muito brilhante.
\section{Carbunculoso}
\begin{itemize}
\item {Grp. gram.:adj.}
\end{itemize}
Que tem natureza de carbúnculo.
\section{Carburação}
\begin{itemize}
\item {Grp. gram.:f.}
\end{itemize}
Acto de submeter o ferro á acção do carbóne.
Mistura do ar atmosphérico com o líquido inflamável, no carburador.
\section{Carburador}
\begin{itemize}
\item {Grp. gram.:m.}
\end{itemize}
Apparelho, onde se mistura o ar atmosphérico com o líquido combustível, que se incendia, para fazer explosão.
\section{Carbureto}
\begin{itemize}
\item {fónica:burê}
\end{itemize}
\begin{itemize}
\item {Grp. gram.:m.}
\end{itemize}
\begin{itemize}
\item {Utilização:Chím.}
\end{itemize}
Corpo produzido pela união do carbóne com metalloides ou metaes.--Palavra afrancesada, do fr. \textunderscore carbure\textunderscore .
Sería preferível \textunderscore carboneto\textunderscore , de \textunderscore carbóne\textunderscore .
\section{Çarça}
\begin{itemize}
\item {Grp. gram.:f.}
\end{itemize}
(Fórma exacta, em vez de \textunderscore sarça\textunderscore )(V.sarça)
\section{Çarça}
\begin{itemize}
\item {Grp. gram.:f.}
\end{itemize}
(Como escreveu Herculano)
O mesmo que \textunderscore silva\textunderscore .
Silvado, matagal.
(Cast. \textunderscore zarza\textunderscore )
\section{Carcacola}
\begin{itemize}
\item {Grp. gram.:f.}
\end{itemize}
Espécie de resina, usada pelos pharmacêuticos.
\section{Carcacolla}
\begin{itemize}
\item {Grp. gram.:f.}
\end{itemize}
Espécie de resina, usada pelos pharmacêuticos.
\section{Carcaio}
\begin{itemize}
\item {Grp. gram.:m.}
\end{itemize}
\begin{itemize}
\item {Utilização:Prov.}
\end{itemize}
\begin{itemize}
\item {Utilização:minh.}
\end{itemize}
O mesmo que \textunderscore meretriz\textunderscore .
\section{Carcaju}
\begin{itemize}
\item {Grp. gram.:m.}
\end{itemize}
Espécie de teixugo americano.
\section{Carcalhada}
\begin{itemize}
\item {Grp. gram.:f.}
\end{itemize}
\begin{itemize}
\item {Utilização:Prov.}
\end{itemize}
\begin{itemize}
\item {Utilização:alg.}
\end{itemize}
O mesmo que \textunderscore gargalhada\textunderscore .
\section{Carcalhota}
\begin{itemize}
\item {Grp. gram.:f.}
\end{itemize}
Nome, que, nas vizinhanças de Coimbra, se dá á codorniz.
(Cp. \textunderscore parpalhó\textunderscore )
\section{Carcamano}
\begin{itemize}
\item {Grp. gram.:m.}
\end{itemize}
\begin{itemize}
\item {Utilização:T. da Figueira da Foz}
\end{itemize}
\begin{itemize}
\item {Grp. gram.:M.}
\end{itemize}
\begin{itemize}
\item {Utilização:Bras. do N}
\end{itemize}
Rapazola; garoto.
Árvore de Angola.
Vendedor ambulante de armarinho.
Nome depreciativo, que no Brasil se dá aos Italianos.
\section{Carcanhola}
\begin{itemize}
\item {Grp. gram.:f.}
\end{itemize}
\begin{itemize}
\item {Utilização:Prov.}
\end{itemize}
\begin{itemize}
\item {Utilização:alg.}
\end{itemize}
O mesmo que \textunderscore ostra\textunderscore .
\section{Carcão}
\begin{itemize}
\item {Grp. gram.:m.}
\end{itemize}
Rocha, que tem veios de oiro.
\section{Çarçaparrilha}
\begin{itemize}
\item {Grp. gram.:f.}
\end{itemize}
(Fórma exacta, em vez da mais usada, \textunderscore salsa-parrilha\textunderscore )
(Cast. \textunderscore zarzaparrilla\textunderscore )
\section{Carcarear}
\begin{itemize}
\item {Grp. gram.:v. i.}
\end{itemize}
(V.cacarejar)
\section{Carcarejar}
\begin{itemize}
\item {Grp. gram.:v. i.}
\end{itemize}
(V.cacarejar)
\section{Carcás}
\begin{itemize}
\item {Grp. gram.:m.}
\end{itemize}
\begin{itemize}
\item {Proveniência:(Do lat. \textunderscore carchesium\textunderscore , segundo Körting)}
\end{itemize}
O mesmo que \textunderscore aljava\textunderscore .
\section{Carcás}
\begin{itemize}
\item {Grp. gram.:m.}
\end{itemize}
\begin{itemize}
\item {Utilização:Des.}
\end{itemize}
Projéctil, mais conhecido por \textunderscore carcassa\textunderscore .(V.carcassa)
\section{Carcassa}
\begin{itemize}
\item {Grp. gram.:f.}
\end{itemize}
\begin{itemize}
\item {Utilização:Fig.}
\end{itemize}
Esqueleto; arcaboiço.
Casco velho de navio.
Navio sem aprestos.
Urdidura do navio em construcção.
Armação, para chapéu de mulher.
Projéctil, que constava de três granadas e outras matérias inflammáveis, e era envolvido num saco alcatroado.
Mulher velha e feia.
(Cast. \textunderscore carcasa\textunderscore )
\section{Cárcava}
\begin{itemize}
\item {Grp. gram.:f.}
\end{itemize}
\begin{itemize}
\item {Utilização:Des.}
\end{itemize}
\begin{itemize}
\item {Proveniência:(Do ár. \textunderscore carcab\textunderscore , ventre)}
\end{itemize}
Fôsso, em volta de uma praça.
\section{Carcavão}
\begin{itemize}
\item {Grp. gram.:m.}
\end{itemize}
\begin{itemize}
\item {Utilização:Prov.}
\end{itemize}
\begin{itemize}
\item {Utilização:trasm.}
\end{itemize}
\begin{itemize}
\item {Proveniência:(De \textunderscore cárcava\textunderscore )}
\end{itemize}
Barranco.
\section{Carcavar}
\begin{itemize}
\item {Grp. gram.:v. t.}
\end{itemize}
\begin{itemize}
\item {Grp. gram.:V. i.}
\end{itemize}
\begin{itemize}
\item {Grp. gram.:V. p.}
\end{itemize}
Rodear com cárcava.
Tornar ôco.
Abrir cárcava.
Rasgar-se, abrir-se:«\textunderscore carcavava-se um algar.\textunderscore »Camillo, \textunderscore Mulher Fatal\textunderscore , 19.
\section{Carcavear}
\begin{itemize}
\item {Grp. gram.:v. i.}
\end{itemize}
\begin{itemize}
\item {Utilização:Des.}
\end{itemize}
O mesmo que \textunderscore carcavar\textunderscore .
\section{Carcaveira}
\begin{itemize}
\item {Grp. gram.:f.}
\end{itemize}
\begin{itemize}
\item {Utilização:Prov.}
\end{itemize}
\begin{itemize}
\item {Utilização:trasm.}
\end{itemize}
\begin{itemize}
\item {Proveniência:(De \textunderscore cárcava\textunderscore )}
\end{itemize}
Cavouco do moínho.
\section{Carcavelos}
\begin{itemize}
\item {Grp. gram.:m.}
\end{itemize}
Vinho, fabricado em Carcavelos.
\section{Carcel}
\begin{itemize}
\item {Grp. gram.:m.}
\end{itemize}
\begin{itemize}
\item {Proveniência:(Fr. \textunderscore carcel\textunderscore )}
\end{itemize}
Candeeiro suspenso, que se eleva ou se abaixa por meio de uma corda ou corrente, que passa por uma roldana. Cf. Garrett, \textunderscore Escritos Div.\textunderscore , 257.
\section{Carcela}
\begin{itemize}
\item {Grp. gram.:f.}
\end{itemize}
\begin{itemize}
\item {Proveniência:(Do rad. de \textunderscore cárcere\textunderscore )}
\end{itemize}
Tira de pano, com casas, que se cose a uma das bandas da farda ou do casaco, para se abotoar sôbre a outra banda, em que estão os botões.
\section{Carcélia}
\begin{itemize}
\item {Grp. gram.:f.}
\end{itemize}
\begin{itemize}
\item {Proveniência:(De \textunderscore Carcélia\textunderscore , n. p.)}
\end{itemize}
Insecto díptero.
\section{Cárcer}
\begin{itemize}
\item {Grp. gram.:m.}
\end{itemize}
\begin{itemize}
\item {Utilização:Ant.}
\end{itemize}
O mesmo que \textunderscore cárcere\textunderscore .
\section{Carceragem}
\begin{itemize}
\item {Grp. gram.:f.}
\end{itemize}
\begin{itemize}
\item {Proveniência:(De \textunderscore cárcere\textunderscore )}
\end{itemize}
Acto de encarcerar.
Quantia, que o preso paga ao carcereiro.
\section{Carcerar}
\begin{itemize}
\item {Grp. gram.:v. t.}
\end{itemize}
O mesmo que \textunderscore encarcerar\textunderscore . Cf. Filinto, II, 295.
\section{Carcerário}
\begin{itemize}
\item {Grp. gram.:adj.}
\end{itemize}
Relativo ao cárcere.
(B. lat. \textunderscore carcerarius\textunderscore )
\section{Carcerática}
\begin{itemize}
\item {Grp. gram.:f.}
\end{itemize}
\begin{itemize}
\item {Utilização:Ant.}
\end{itemize}
Quantia ou multa, que se pagava ao entrar na cadeia; carceragem.
\section{Cárcere}
\begin{itemize}
\item {Grp. gram.:m.}
\end{itemize}
\begin{itemize}
\item {Utilização:Fig.}
\end{itemize}
\begin{itemize}
\item {Proveniência:(Lat. \textunderscore carcer\textunderscore )}
\end{itemize}
Lugar, em que alguém está preso, ou que é destinado a prisão; cadeia.
Lugar, nos circos, donde saíam os cavallos para o espectáculo.
Obstácalo.
\section{Carcereiro}
\begin{itemize}
\item {Grp. gram.:m.}
\end{itemize}
\begin{itemize}
\item {Proveniência:(Do b. lat. \textunderscore carcerarius\textunderscore )}
\end{itemize}
Guarda do cárcere.
\section{Carcérula}
\begin{itemize}
\item {Grp. gram.:f.}
\end{itemize}
\begin{itemize}
\item {Proveniência:(De \textunderscore cárcere\textunderscore )}
\end{itemize}
Cada uma das cavidades de vários frutos indehiscentes.
\section{Carcerular}
\begin{itemize}
\item {Grp. gram.:adj.}
\end{itemize}
Relativo a carcérula.
Que tem carcérula.
\section{Carcha}
\begin{itemize}
\item {Grp. gram.:f.}
\end{itemize}
\begin{itemize}
\item {Utilização:Prov.}
\end{itemize}
\begin{itemize}
\item {Utilização:beir.}
\end{itemize}
Cada uma das duas rodelas grossas, em que se divide uma batata grande e crua, para que facilmente se coza.
\section{Carcinoide}
\begin{itemize}
\item {Grp. gram.:adj.}
\end{itemize}
\begin{itemize}
\item {Proveniência:(Do gr. \textunderscore karkinos\textunderscore  + \textunderscore eidos\textunderscore )}
\end{itemize}
Semelhante a crustáceos.
\section{Carcinoma}
\begin{itemize}
\item {Grp. gram.:m.}
\end{itemize}
\begin{itemize}
\item {Proveniência:(Gr. \textunderscore karkinoma\textunderscore )}
\end{itemize}
O mesmo que \textunderscore cancro\textunderscore .
\section{Carcinomatoso}
\begin{itemize}
\item {Grp. gram.:adj.}
\end{itemize}
Que tem natureza de carcinoma.
\section{Carcinose}
\begin{itemize}
\item {Grp. gram.:f.}
\end{itemize}
\begin{itemize}
\item {Utilização:Med.}
\end{itemize}
\begin{itemize}
\item {Proveniência:(Do gr. \textunderscore karkinos\textunderscore , cancro)}
\end{itemize}
Doença cancerosa; carcinoma.
Producção de tecidos cancerosos, nos órgãos internos.
\section{Cárcola}
\begin{itemize}
\item {Grp. gram.:f.}
\end{itemize}
Pedal dos teares. Cf. \textunderscore Inquér. Industr.\textunderscore , P. II, l. 3.^o, 147.
(Cp. cast. \textunderscore cárcova\textunderscore )
\section{Carcoma}
\begin{itemize}
\item {Grp. gram.:f.}
\end{itemize}
\begin{itemize}
\item {Proveniência:(De \textunderscore carcomer\textunderscore )}
\end{itemize}
Caruncho, insecto que rói a madeira.
Pó de madeira carcomida.
Podridão.
Aquillo que consome, que devora ou arruina.
\section{Carcome}
\begin{itemize}
\item {Grp. gram.:m.}
\end{itemize}
Acto de \textunderscore carcomer\textunderscore .
O mesmo que \textunderscore carcoma\textunderscore . Cf. Filinto, III, 106.
\section{Carcomer}
\begin{itemize}
\item {Grp. gram.:v. t.}
\end{itemize}
\begin{itemize}
\item {Utilização:Fig.}
\end{itemize}
\begin{itemize}
\item {Proveniência:(De \textunderscore comer\textunderscore , com um pref. incerto)}
\end{itemize}
Roer (madeira)
Escavar.
Arruinar.
\section{Carcomido}
\begin{itemize}
\item {Grp. gram.:adj.}
\end{itemize}
\begin{itemize}
\item {Proveniência:(De \textunderscore carcomer\textunderscore )}
\end{itemize}
Corroído.
Gasto.
\section{Carçonista}
\begin{itemize}
\item {Grp. gram.:m.}
\end{itemize}
Indivíduo natural de Carção, no concelho de Vimioso, e, geralmente, de origem judaica.
\section{Cárcova}
\begin{itemize}
\item {Grp. gram.:f.}
\end{itemize}
\begin{itemize}
\item {Utilização:Des.}
\end{itemize}
Porta falsa.
Caminho coberto.
Fôsso.
(Cp. \textunderscore cárcava\textunderscore )
\section{Carcunda}
\begin{itemize}
\item {Grp. gram.:f.}
\end{itemize}
\begin{itemize}
\item {Grp. gram.:M.  e  adj.}
\end{itemize}
Corcova, protuberância, nas costas ou no peito de uma pessôa.
Mulher, que tem carcunda.
Homem, que tem carcunda.
(Do quimb.)
\section{Carda}
\begin{itemize}
\item {Grp. gram.:f.}
\end{itemize}
Instrumento, com que se carda.
Acto de cardar.
Pregos miúdos.
Antigo instrumento de tortura.
Pequenas pastas de immundicie, que se agarram á lan dos animaes.
Sujidade na pelle das pessôas.
(Cp. \textunderscore cardo\textunderscore )
\section{Cardação}
\begin{itemize}
\item {Grp. gram.:f.}
\end{itemize}
Acto ou effeito de \textunderscore cardar\textunderscore .
\section{Cardaço}
\begin{itemize}
\item {Grp. gram.:m.}
\end{itemize}
\begin{itemize}
\item {Utilização:Prov.}
\end{itemize}
\begin{itemize}
\item {Utilização:beir.}
\end{itemize}
Bagaço de uvas.
\section{Cardada}
\begin{itemize}
\item {Grp. gram.:f.}
\end{itemize}
\begin{itemize}
\item {Proveniência:(De \textunderscore carda\textunderscore )}
\end{itemize}
Pancada com a carda.
Porção de lan, que se carda de uma vez.
\textunderscore Dar ao demo a cardada\textunderscore , arrepender-se do que fez, têr mau êxito.
\section{Cardadeira}
\begin{itemize}
\item {Grp. gram.:f.}
\end{itemize}
Mulher que carda.
\section{Cardador}
\begin{itemize}
\item {Grp. gram.:m.}
\end{itemize}
Aquelle que carda ou tem por offício cardar.
\section{Cardadura}
\begin{itemize}
\item {Grp. gram.:f.}
\end{itemize}
Acção ou effeito de \textunderscore cardar\textunderscore .
\section{Cardagem}
\begin{itemize}
\item {Grp. gram.:f.}
\end{itemize}
\begin{itemize}
\item {Proveniência:(De \textunderscore cardar\textunderscore )}
\end{itemize}
Arte ou officina de cardador.
\section{Cardal}
\begin{itemize}
\item {Grp. gram.:m.}
\end{itemize}
\begin{itemize}
\item {Utilização:Prov.}
\end{itemize}
\begin{itemize}
\item {Utilização:alent.}
\end{itemize}
\begin{itemize}
\item {Proveniência:(De \textunderscore cardo\textunderscore )}
\end{itemize}
Lugar, que produz cardos.
Cemitério.
\section{Cardamina}
\begin{itemize}
\item {Grp. gram.:f.}
\end{itemize}
\begin{itemize}
\item {Proveniência:(Gr. \textunderscore kardamin\textunderscore )}
\end{itemize}
Planta crucífera, espécie de agrião, (\textunderscore cardamina pratensis\textunderscore ).
\section{Cardamomo}
\begin{itemize}
\item {Grp. gram.:m.}
\end{itemize}
\begin{itemize}
\item {Proveniência:(Lat. \textunderscore cardamomum\textunderscore )}
\end{itemize}
Nome de várias plantas amómeas.
O mesmo que \textunderscore badiana\textunderscore .
\section{Cardanha}
\begin{itemize}
\item {Grp. gram.:f.}
\end{itemize}
\begin{itemize}
\item {Utilização:Prov.}
\end{itemize}
\begin{itemize}
\item {Utilização:dur.}
\end{itemize}
Pequena casa térrea, onde dormem jornaleiros.
\section{Cardanho}
\begin{itemize}
\item {Grp. gram.:m.}
\end{itemize}
\begin{itemize}
\item {Utilização:Prov.}
\end{itemize}
\begin{itemize}
\item {Utilização:trasm.}
\end{itemize}
\begin{itemize}
\item {Utilização:Pop.}
\end{itemize}
O mesmo que \textunderscore cardanha\textunderscore .
Roubo.
\section{Cardão}
\begin{itemize}
\item {Grp. gram.:adj.}
\end{itemize}
\begin{itemize}
\item {Proveniência:(De \textunderscore cardo\textunderscore )}
\end{itemize}
Azul-violáceo.
Que tem a côr da flôr do cardo.
\section{Cardar}
\begin{itemize}
\item {Grp. gram.:v. t.}
\end{itemize}
\begin{itemize}
\item {Utilização:Gír.}
\end{itemize}
\begin{itemize}
\item {Proveniência:(De \textunderscore carda\textunderscore )}
\end{itemize}
Desenredar, pentear com carda, (lan ou qualquer filaça).
Extorquir astuciosamente alguma coisa a; furtar.
\section{Cardazola}
\begin{itemize}
\item {Grp. gram.:f.}
\end{itemize}
\begin{itemize}
\item {Utilização:Prov.}
\end{itemize}
\begin{itemize}
\item {Utilização:alent.}
\end{itemize}
Planta, (\textunderscore centaurea Prolongi\textunderscore , Bois).
\section{Cardeal}
\begin{itemize}
\item {Grp. gram.:m.}
\end{itemize}
\begin{itemize}
\item {Grp. gram.:Adj.}
\end{itemize}
\begin{itemize}
\item {Proveniência:(Lat. \textunderscore cardinalis\textunderscore )}
\end{itemize}
Prelado do Sagrado Collégio pontifício.
Designação de várias aves, em cujas côres predomina a vermelha.
Planta lobeliácea.
Nome que, em Penafiel, se dá ao \textunderscore dom-fafe\textunderscore .
Principal.
\section{Cardealado}
\begin{itemize}
\item {Grp. gram.:m.}
\end{itemize}
\begin{itemize}
\item {Proveniência:(De \textunderscore cardeal\textunderscore )}
\end{itemize}
O mesmo que \textunderscore cardanilado\textunderscore .
\section{Cardeal-americano}
\begin{itemize}
\item {Grp. gram.:m.}
\end{itemize}
O mesmo que \textunderscore paroara\textunderscore .
\section{Cardeal-de-popa}
\begin{itemize}
\item {Grp. gram.:m}
\end{itemize}
O mesmo que \textunderscore bico-grossudo\textunderscore .
\section{Cardeal-do-japão}
\begin{itemize}
\item {Grp. gram.:m.}
\end{itemize}
\begin{itemize}
\item {Utilização:Bras}
\end{itemize}
Pássaro canoro, de bico vermelho e alongado.
\section{Cardear}
\begin{itemize}
\item {Grp. gram.:v. t.}
\end{itemize}
\begin{itemize}
\item {Utilização:Prov.}
\end{itemize}
\begin{itemize}
\item {Utilização:beir.}
\end{itemize}
\begin{itemize}
\item {Proveniência:(De \textunderscore cárdeo\textunderscore )}
\end{itemize}
Avergoar a pelle de (cavallo) com o chicote.
\section{Cardeiro}
\begin{itemize}
\item {Grp. gram.:m.}
\end{itemize}
Aquelle que faz ou vende cardas.
\section{Cardenho}
\begin{itemize}
\item {Grp. gram.:m.}
\end{itemize}
O mesmo ou melhor que \textunderscore cardanho\textunderscore . Cf. Camillo, \textunderscore Sereia\textunderscore , 113; \textunderscore Amor de Salvação\textunderscore , 265.
\section{Cardenilho}
\begin{itemize}
\item {Grp. gram.:m.}
\end{itemize}
O mesmo que \textunderscore verdete\textunderscore .
\section{Cárdeno}
\begin{itemize}
\item {Grp. gram.:adj.}
\end{itemize}
\begin{itemize}
\item {Utilização:Ant.}
\end{itemize}
O mesmo que \textunderscore cárdeo\textunderscore .
\section{Cárdeo}
\begin{itemize}
\item {Grp. gram.:adj.}
\end{itemize}
O mesmo que \textunderscore cardão\textunderscore .
(Cast. \textunderscore cárdeno\textunderscore )
\section{Cardete}
\begin{itemize}
\item {fónica:dê}
\end{itemize}
\begin{itemize}
\item {Grp. gram.:m.}
\end{itemize}
\begin{itemize}
\item {Utilização:Prov.}
\end{itemize}
\begin{itemize}
\item {Utilização:beir.}
\end{itemize}
Planta, (\textunderscore eryngium tenue\textunderscore , Lamk.)
\section{Cárdia}
\begin{itemize}
\item {Grp. gram.:f.}
\end{itemize}
\begin{itemize}
\item {Proveniência:(Gr. \textunderscore kardia\textunderscore )}
\end{itemize}
Abertura superior do estômago.
\section{Cardíaca}
\begin{itemize}
\item {Grp. gram.:f.}
\end{itemize}
\begin{itemize}
\item {Utilização:Bot.}
\end{itemize}
O mesmo que \textunderscore agripalma\textunderscore .
\section{Cardiáceas}
\begin{itemize}
\item {Grp. gram.:f. pl.}
\end{itemize}
\begin{itemize}
\item {Proveniência:(De \textunderscore cardia\textunderscore )}
\end{itemize}
Família de molluscos acéphalos, que têm por typo a bucárdia.
\section{Cardíaco}
\begin{itemize}
\item {Grp. gram.:adj.}
\end{itemize}
\begin{itemize}
\item {Grp. gram.:M.}
\end{itemize}
\begin{itemize}
\item {Proveniência:(Gr. \textunderscore kardiakos\textunderscore )}
\end{itemize}
Relativo ao coração ou á cárdia: \textunderscore doença cardíaca\textunderscore .
Aquelle que soffre do coração.
\section{Cardialgia}
\begin{itemize}
\item {Grp. gram.:f.}
\end{itemize}
\begin{itemize}
\item {Proveniência:(Gr. \textunderscore kardialgia\textunderscore )}
\end{itemize}
Soffrimento na cárdia.
\section{Cardiálgico}
\begin{itemize}
\item {Grp. gram.:adj.}
\end{itemize}
Relativo á cardialgia.
\section{Cardialina}
\begin{itemize}
\item {Grp. gram.:f.}
\end{itemize}
O mesmo que \textunderscore cardeal\textunderscore , planta.
\section{Cárdice}
\begin{itemize}
\item {Grp. gram.:f.}
\end{itemize}
\begin{itemize}
\item {Proveniência:(De \textunderscore cárdia\textunderscore )}
\end{itemize}
Camafeu, que apresenta um coração em relêvo.
\section{Cardiço}
\begin{itemize}
\item {Grp. gram.:m.}
\end{itemize}
\begin{itemize}
\item {Proveniência:(De \textunderscore carda\textunderscore )}
\end{itemize}
Pequena carda de chapeleiro.
\section{Cardido}
\begin{itemize}
\item {Grp. gram.:adj.}
\end{itemize}
Diz-se da madeira, que esteve muito tempo debaixo de água e apodreceu.
\section{Cardiguera}
\begin{itemize}
\item {Grp. gram.:f.}
\end{itemize}
\begin{itemize}
\item {Utilização:Bras}
\end{itemize}
Espécie de rôla.
\section{Cardim}
\begin{itemize}
\item {Grp. gram.:adj.}
\end{itemize}
\begin{itemize}
\item {Proveniência:(De \textunderscore cardo\textunderscore )}
\end{itemize}
Diz-se do toiro, que tem o pêlo branco e preto.
\section{Cardina}
\begin{itemize}
\item {Grp. gram.:f.}
\end{itemize}
\begin{itemize}
\item {Utilização:Ext.}
\end{itemize}
\begin{itemize}
\item {Utilização:Pop.}
\end{itemize}
Pequenas pastas de immundície, que adherem á lan ou pêlo dos animaes.
Sujidade na pelle das pessôas.
Bebedeira.
\section{Cardina}
\begin{itemize}
\item {Grp. gram.:f.}
\end{itemize}
\begin{itemize}
\item {Proveniência:(Do gr. \textunderscore kardia\textunderscore )}
\end{itemize}
Medicamento tónico para o coração.
\section{Cardinal}
\begin{itemize}
\item {Grp. gram.:adj.}
\end{itemize}
\begin{itemize}
\item {Utilização:Arith.}
\end{itemize}
\begin{itemize}
\item {Proveniência:(Lat. \textunderscore cardinalis\textunderscore )}
\end{itemize}
Relativo a gonzos.
Principal, importante.
Diz-se do \textunderscore adj.\textunderscore  que representa um número absoluto, em contraposição a \textunderscore ordinal\textunderscore : \textunderscore 4 é número cardinal\textunderscore .
\section{Cardinala}
\begin{itemize}
\item {Grp. gram.:f.}
\end{itemize}
(V.cardialina)
\section{Cardinalado}
\begin{itemize}
\item {Grp. gram.:m.}
\end{itemize}
\begin{itemize}
\item {Proveniência:(Do lat. \textunderscore cardinalis\textunderscore )}
\end{itemize}
Dignidade de cardeal.
\section{Cardinalato}
\begin{itemize}
\item {Grp. gram.:m.}
\end{itemize}
O mesmo que \textunderscore cardinalado\textunderscore .
\section{Cardinalício}
\begin{itemize}
\item {Grp. gram.:adj.}
\end{itemize}
\begin{itemize}
\item {Proveniência:(Do lat. \textunderscore cardinalis\textunderscore )}
\end{itemize}
Relativo a cardeal.
\section{Cardinha}
\begin{itemize}
\item {Grp. gram.:f.}
\end{itemize}
\begin{itemize}
\item {Utilização:T. da Bairrada}
\end{itemize}
O mesmo que \textunderscore tanjasno\textunderscore .
\section{Cardinho}
\begin{itemize}
\item {Grp. gram.:m.}
\end{itemize}
\begin{itemize}
\item {Proveniência:(De \textunderscore cardo\textunderscore )}
\end{itemize}
Planta medicinal.
\section{Cardinífero}
\begin{itemize}
\item {Grp. gram.:adj.}
\end{itemize}
\begin{itemize}
\item {Proveniência:(Do lat. \textunderscore cardo\textunderscore , \textunderscore cardinis\textunderscore  + \textunderscore ferre\textunderscore )}
\end{itemize}
Que tem charneira ou coiceira.
\section{Cardiocele}
\begin{itemize}
\item {Grp. gram.:f.}
\end{itemize}
\begin{itemize}
\item {Utilização:Med.}
\end{itemize}
\begin{itemize}
\item {Proveniência:(Do gr. \textunderscore kardia\textunderscore  + \textunderscore kele\textunderscore )}
\end{itemize}
Hérnia do coração.
\section{Cardiografia}
\begin{itemize}
\item {Grp. gram.:f.}
\end{itemize}
\begin{itemize}
\item {Proveniência:(Do gr. \textunderscore kardia\textunderscore  + \textunderscore graphein\textunderscore )}
\end{itemize}
Parte da Anatomia, em que se descreve o coração.
\section{Cardiógrafo}
\begin{itemize}
\item {Grp. gram.:m.}
\end{itemize}
\begin{itemize}
\item {Proveniência:(Do gr. \textunderscore kardia\textunderscore  + \textunderscore graphein\textunderscore )}
\end{itemize}
Instrumento, para calcular os movimentos do coração.
\section{Cardiographia}
\begin{itemize}
\item {Grp. gram.:f.}
\end{itemize}
\begin{itemize}
\item {Proveniência:(Do gr. \textunderscore kardia\textunderscore  + \textunderscore graphein\textunderscore )}
\end{itemize}
Parte da Anatomia, em que se descreve o coração.
\section{Cardiógrapho}
\begin{itemize}
\item {Grp. gram.:m.}
\end{itemize}
\begin{itemize}
\item {Proveniência:(Do gr. \textunderscore kardia\textunderscore  + \textunderscore graphein\textunderscore )}
\end{itemize}
Instrumento, para calcular os movimentos do coração.
\section{Cardiola}
\begin{itemize}
\item {Grp. gram.:f.}
\end{itemize}
\begin{itemize}
\item {Utilização:Prov.}
\end{itemize}
\begin{itemize}
\item {Utilização:trasm.}
\end{itemize}
O mesmo que \textunderscore cardina\textunderscore ^1.
\section{Cardiologia}
\begin{itemize}
\item {Grp. gram.:f.}
\end{itemize}
\begin{itemize}
\item {Proveniência:(Do gr. \textunderscore kardia\textunderscore  + \textunderscore logos\textunderscore )}
\end{itemize}
Tratado do coração.
\section{Cardiopathia}
\begin{itemize}
\item {Grp. gram.:f.}
\end{itemize}
\begin{itemize}
\item {Utilização:Med.}
\end{itemize}
\begin{itemize}
\item {Proveniência:(Do gr. \textunderscore kardia\textunderscore  + \textunderscore pathos\textunderscore )}
\end{itemize}
Moléstia do coração em geral.
\section{Cardiopatia}
\begin{itemize}
\item {Grp. gram.:f.}
\end{itemize}
\begin{itemize}
\item {Utilização:Med.}
\end{itemize}
\begin{itemize}
\item {Proveniência:(Do gr. \textunderscore kardia\textunderscore  + \textunderscore pathos\textunderscore )}
\end{itemize}
Moléstia do coração em geral.
\section{Cardiopétalo}
\begin{itemize}
\item {Grp. gram.:adj.}
\end{itemize}
\begin{itemize}
\item {Utilização:Bot.}
\end{itemize}
\begin{itemize}
\item {Proveniência:(Do gr. \textunderscore kardia\textunderscore  + \textunderscore petalon\textunderscore )}
\end{itemize}
Cujas pétalas têm fórma de coração.
\section{Cardioplegia}
\begin{itemize}
\item {Grp. gram.:f.}
\end{itemize}
\begin{itemize}
\item {Utilização:Med.}
\end{itemize}
\begin{itemize}
\item {Proveniência:(Do gr. \textunderscore kardia\textunderscore  + \textunderscore plessein\textunderscore )}
\end{itemize}
Paralisia do coração.
\section{Cardióptero}
\begin{itemize}
\item {Grp. gram.:adj.}
\end{itemize}
\begin{itemize}
\item {Proveniência:(Do gr. \textunderscore kardia\textunderscore  + \textunderscore pteron\textunderscore )}
\end{itemize}
Que tem asas ou barbatanas em fórma de coração.
\section{Cardiospermo}
\begin{itemize}
\item {Grp. gram.:m.}
\end{itemize}
\begin{itemize}
\item {Proveniência:(Do gr. \textunderscore kardia\textunderscore  + \textunderscore sperma\textunderscore )}
\end{itemize}
Planta herbácea, cuja semente tem forma de coração.
\section{Cardita}
\begin{itemize}
\item {Grp. gram.:f.}
\end{itemize}
Mollusco, espécie de mexilhão.
\section{Cardite}
\begin{itemize}
\item {Grp. gram.:f.}
\end{itemize}
\begin{itemize}
\item {Proveniência:(De \textunderscore cárdia\textunderscore )}
\end{itemize}
Inflammação do coração.
\section{Cardítico}
\begin{itemize}
\item {Grp. gram.:adj.}
\end{itemize}
Relativo á cardite.
\section{Cardo}
\begin{itemize}
\item {Grp. gram.:m.}
\end{itemize}
\begin{itemize}
\item {Grp. gram.:Adj.}
\end{itemize}
\begin{itemize}
\item {Utilização:ant.}
\end{itemize}
\begin{itemize}
\item {Utilização:Fig.}
\end{itemize}
\begin{itemize}
\item {Proveniência:(Lat. \textunderscore carduus\textunderscore )}
\end{itemize}
Planta espinhosa, espécie de alcachofra.
Espécie de cacto, no Brasil.
Corda.
Áspero, crespo.
\section{Cardoça}
\begin{itemize}
\item {Grp. gram.:f.}
\end{itemize}
\begin{itemize}
\item {Utilização:Prov.}
\end{itemize}
O mesmo que \textunderscore cardoçada\textunderscore .
\section{Cardoçada}
\begin{itemize}
\item {Grp. gram.:f.}
\end{itemize}
\begin{itemize}
\item {Utilização:Prov.}
\end{itemize}
Bátega de água.
(Colhido em Turquel)
\section{Cardo-de-santa-maria}
\begin{itemize}
\item {Grp. gram.:m.}
\end{itemize}
O mesmo que \textunderscore cardo-mariano\textunderscore .
\section{Cardoeira}
\begin{itemize}
\item {Grp. gram.:f.}
\end{itemize}
Planta euphorbiácea da Índia portuguesa, (\textunderscore stilago bunius\textunderscore , Roxb.).
\section{Cardo-mariano}
\begin{itemize}
\item {Grp. gram.:m.}
\end{itemize}
Planta medicinal, da fam. das compostas, (\textunderscore silybum marianum\textunderscore , Gaertner).
\section{Cardómetro}
\begin{itemize}
\item {Grp. gram.:m.}
\end{itemize}
\begin{itemize}
\item {Proveniência:(Do gr. \textunderscore kardia\textunderscore  + \textunderscore metron\textunderscore )}
\end{itemize}
Apparelho, para medir a fôrça, com que o coração põe o sangue em movimento.
\section{Cardo-palmatória}
\begin{itemize}
\item {Grp. gram.:m.}
\end{itemize}
\begin{itemize}
\item {Utilização:Bras}
\end{itemize}
Planta lactácea, de frutos comestíveis.
\section{Cardo-penteador}
\begin{itemize}
\item {Grp. gram.:m.}
\end{itemize}
Planta dipsácea, cuja cabeça ouriçada serve, depois de sêca, para cardar ou pentear baêtas, cobertores, etc.
\section{Cardosa}
\begin{itemize}
\item {Grp. gram.:f.}
\end{itemize}
Nome de um peixe, (\textunderscore gobius jozo\textunderscore , Lin.).
\section{Cardosa}
\begin{itemize}
\item {Grp. gram.:f.}
\end{itemize}
\begin{itemize}
\item {Utilização:T. de Almada}
\end{itemize}
Casa de malta.
\section{Cardo-santo}
\begin{itemize}
\item {Grp. gram.:m.}
\end{itemize}
Planta medicinal, (\textunderscore centaura benedicta\textunderscore , Lin.).
\section{Carduça}
\begin{itemize}
\item {Grp. gram.:f.}
\end{itemize}
\begin{itemize}
\item {Proveniência:(De \textunderscore carda\textunderscore )}
\end{itemize}
Carda grosseira, para começar a cardadura.
\section{Carduçador}
\begin{itemize}
\item {Grp. gram.:m.}
\end{itemize}
Aquelle que carduça.
\section{Carduçar}
\begin{itemize}
\item {Grp. gram.:v. t.}
\end{itemize}
Passar (a lan) pela carduça.
\section{Cardume}
\begin{itemize}
\item {Grp. gram.:m.}
\end{itemize}
\begin{itemize}
\item {Proveniência:(De \textunderscore carda\textunderscore )}
\end{itemize}
Bando de peixes.
Bando, multidão, ajuntamento.
\section{Carduncello}
\begin{itemize}
\item {Grp. gram.:m.}
\end{itemize}
\begin{itemize}
\item {Proveniência:(Do lat. \textunderscore carduus\textunderscore )}
\end{itemize}
Gênero de plantas compostas.
\section{Carduncelo}
\begin{itemize}
\item {Grp. gram.:m.}
\end{itemize}
\begin{itemize}
\item {Proveniência:(Do lat. \textunderscore carduus\textunderscore )}
\end{itemize}
Gênero de plantas compostas.
\section{Careação}
\begin{itemize}
\item {Grp. gram.:f.}
\end{itemize}
(V.acareação)
\section{Careador}
\begin{itemize}
\item {Grp. gram.:m.  e  adj.}
\end{itemize}
O que careia.
\section{Carear}
\begin{itemize}
\item {Grp. gram.:v. t.}
\end{itemize}
O mesmo que \textunderscore acarear\textunderscore ^1.
\section{Carear}
\begin{itemize}
\item {Grp. gram.:v. t.}
\end{itemize}
(V.carrear)
\section{Careca}
\begin{itemize}
\item {Grp. gram.:m.  e  f.}
\end{itemize}
\begin{itemize}
\item {Grp. gram.:F.}
\end{itemize}
\begin{itemize}
\item {Grp. gram.:Adj.}
\end{itemize}
\begin{itemize}
\item {Utilização:Prov.}
\end{itemize}
\begin{itemize}
\item {Utilização:dur.}
\end{itemize}
\begin{itemize}
\item {Grp. gram.:M.}
\end{itemize}
\begin{itemize}
\item {Utilização:Gír. de Lisbôa.}
\end{itemize}
Pessôa calva.
Calva; calvície.
Calvo.
Transparente, mal disfarçado, (falando-se de um lôgro ou de uma mentira): \textunderscore essa é muito calva\textunderscore .
Diz-se do pêssego liso, sem cotão.
Queijo.
\section{Careca}
\begin{itemize}
\item {Grp. gram.:m.}
\end{itemize}
\begin{itemize}
\item {Utilização:Prov.}
\end{itemize}
Moço de praça de toiros, encarregado de abrir a gaiola aos toiros que vão sêr lidados na arena.
Aquelle que deita fogo ás peças de artifício.
\section{Carecente}
\begin{itemize}
\item {Grp. gram.:adj.}
\end{itemize}
Que carece.
\section{Carecer}
\begin{itemize}
\item {Grp. gram.:v. i.}
\end{itemize}
\begin{itemize}
\item {Proveniência:(Lat. \textunderscore carescere\textunderscore )}
\end{itemize}
Têr falta; estar necessitado.
Não têr.
\section{Carecido}
\begin{itemize}
\item {Grp. gram.:adj.}
\end{itemize}
\begin{itemize}
\item {Proveniência:(De \textunderscore carecer\textunderscore )}
\end{itemize}
Que não tem:«\textunderscore muito enfermo e carecido da vista\textunderscore ». \textunderscore Alvará\textunderscore  de D. Sebast., in \textunderscore Rev. Lus.\textunderscore , XV, 134.
\section{Carecimento}
\begin{itemize}
\item {Grp. gram.:m.}
\end{itemize}
(V.carência)
\section{Careio}
\begin{itemize}
\item {Grp. gram.:m.}
\end{itemize}
\begin{itemize}
\item {Utilização:T. do Fundão}
\end{itemize}
Acção de \textunderscore carear\textunderscore ^1.
Juízo, tino, senso commum.
\section{Careiro}
\begin{itemize}
\item {Grp. gram.:adj.}
\end{itemize}
Que vende caro: \textunderscore lojista careiro\textunderscore .
\section{Carel}
\begin{itemize}
\item {Grp. gram.:f.}
\end{itemize}
\begin{itemize}
\item {Utilização:Prov.}
\end{itemize}
\begin{itemize}
\item {Utilização:alg.}
\end{itemize}
\begin{itemize}
\item {Proveniência:(De \textunderscore cara\textunderscore )}
\end{itemize}
Us. na loc. prep. \textunderscore á carel de\textunderscore , ao rés de.
\section{Carélio}
\begin{itemize}
\item {Grp. gram.:m.}
\end{itemize}
Língua uralo-altaica, do grupo ugro-finlandês.
\section{Carena}
\begin{itemize}
\item {Grp. gram.:f.}
\end{itemize}
\begin{itemize}
\item {Utilização:Prov.}
\end{itemize}
\begin{itemize}
\item {Utilização:trasm.}
\end{itemize}
\begin{itemize}
\item {Proveniência:(Lat. \textunderscore carina\textunderscore )}
\end{itemize}
Pétala inferior das flôres papilionáceas, pela analogia que ella tem com a carena ou querena do navio.
O mesmo que \textunderscore querena\textunderscore .
\textunderscore Dar carena\textunderscore , dizimar, desfalcar.
\section{Carenar}
\begin{itemize}
\item {Grp. gram.:v. t.}
\end{itemize}
O mesmo que \textunderscore querenar\textunderscore .
\section{Carência}
\begin{itemize}
\item {Grp. gram.:f.}
\end{itemize}
\begin{itemize}
\item {Proveniência:(Lat. \textunderscore carentia\textunderscore )}
\end{itemize}
Effeito de carecer.
Falta; privação.
\section{Carente}
\begin{itemize}
\item {Grp. gram.:adj.}
\end{itemize}
O mesmo que \textunderscore carecente\textunderscore .
\section{Carepa}
\begin{itemize}
\item {Grp. gram.:f.}
\end{itemize}
\begin{itemize}
\item {Utilização:Prov.}
\end{itemize}
\begin{itemize}
\item {Utilização:alg.}
\end{itemize}
Caspa, aspereza cutânea.
Lanugem de alguns frutos.
Superfície de madeira, cortada com enxó.
Folhelho do milho, fino e macio, com que se enchem colchões.
Cotão ou fuligem, produzida por explosão interior nos cylindros dos automóveis, motocycletas, aeroplanos, etc.
\section{Carapento}
\begin{itemize}
\item {Grp. gram.:adj.}
\end{itemize}
Que tem carepa.
\section{Carestia}
\begin{itemize}
\item {Grp. gram.:f.}
\end{itemize}
\begin{itemize}
\item {Proveniência:(De \textunderscore caro\textunderscore ? ou, antes, do rad. do lat. \textunderscore carescere\textunderscore ?)}
\end{itemize}
Preço alto.
Qualidade do que é caro.
Escassêz; carência.
\section{Carestioso}
\begin{itemize}
\item {Grp. gram.:adj.}
\end{itemize}
\begin{itemize}
\item {Utilização:Des.}
\end{itemize}
Em que há carestia.
\section{Careta}
\begin{itemize}
\item {fónica:carê}
\end{itemize}
\begin{itemize}
\item {Grp. gram.:f.}
\end{itemize}
\begin{itemize}
\item {Utilização:Gír.}
\end{itemize}
\begin{itemize}
\item {Grp. gram.:M.  e  f.}
\end{itemize}
\begin{itemize}
\item {Utilização:Bras. do N}
\end{itemize}
\begin{itemize}
\item {Utilização:Bras. do N}
\end{itemize}
\begin{itemize}
\item {Proveniência:(De \textunderscore cara\textunderscore )}
\end{itemize}
Trejeito do rosto; visagem.
Momice.
Caraça.
Moeda de 500 reis em prata, carinha.
Cada um dos comparsas das mascaradas, que se realizam nas festas dos Reis.
Diz-se do boi ou vaca, cuja cara é de côr differente da do resto do corpo.
\section{Caratear}
\begin{itemize}
\item {Grp. gram.:v. i.}
\end{itemize}
Fazer caretas.
\section{Careteira}
\begin{itemize}
\item {Grp. gram.:f.}
\end{itemize}
Planta cucurbitácea da Índia portuguesa, (\textunderscore momordica charantia\textunderscore , Lin.), de fruto amargo, mas comestível, depois de cozido.
\section{Caretete}
\begin{itemize}
\item {Grp. gram.:m.}
\end{itemize}
Arvore euphorbiácea de Angola.
\section{Careto}
\begin{itemize}
\item {fónica:carê}
\end{itemize}
\begin{itemize}
\item {Grp. gram.:m.}
\end{itemize}
\begin{itemize}
\item {Utilização:Prov.}
\end{itemize}
\begin{itemize}
\item {Utilização:trasm.}
\end{itemize}
\begin{itemize}
\item {Utilização:T. de Mira}
\end{itemize}
\begin{itemize}
\item {Grp. gram.:Adj.}
\end{itemize}
Homem, que anda de caranchona, fazendo de diabo á roda da povoação.
Indivíduo mascarado.
Diz-se do burro, que tem o focinho todo negro.
(Cp. \textunderscore careta\textunderscore )
\section{Cárevo}
\begin{itemize}
\item {Grp. gram.:m.}
\end{itemize}
\begin{itemize}
\item {Utilização:Ant.}
\end{itemize}
O mesmo que \textunderscore cáravo\textunderscore .
\section{Careza}
\begin{itemize}
\item {Grp. gram.:f.}
\end{itemize}
\begin{itemize}
\item {Proveniência:(Do b. lat. \textunderscore caritia\textunderscore )}
\end{itemize}
Qualidade do que é caro; carestia.
\section{Carga}
\begin{itemize}
\item {Grp. gram.:f.}
\end{itemize}
\begin{itemize}
\item {Utilização:Fam.}
\end{itemize}
\begin{itemize}
\item {Utilização:Bras. do N}
\end{itemize}
\begin{itemize}
\item {Utilização:Fam.}
\end{itemize}
\begin{itemize}
\item {Utilização:Pop.}
\end{itemize}
\begin{itemize}
\item {Utilização:Ant.}
\end{itemize}
\begin{itemize}
\item {Proveniência:(De \textunderscore cargar\textunderscore )}
\end{itemize}
Aquillo que é ou póde sêr transportado por homem, por animal, carro, navio, etc.
Acto de carregar.
Fardo; aquillo que pesa sobre alguma coisa ou pessôa.
Grande quantidade.
Oppressão, embaraço.
Encargo, responsabilidade.
Accusação; investida impetuosa.
Pancadaria.
Pólvora e projécteis, que se metem de uma vez numa arma de fogo.
Acumulação de electricidade.
Porção de minério ou carvão, que se lança de uma vez nos fornos metallúrgicos.
Medicamento cáustico, que se applica a um animal.
Grande quantidade.
\textunderscore Carga de ovos\textunderscore , pessôa muito delicada, em que se não deve tocar.
\textunderscore Carga de ossos\textunderscore , pessôa muito magra.
\textunderscore Carga de água\textunderscore , chuvada forte, bátega de água.
\textunderscore Carga cerrada\textunderscore , descarga simultânea de muitas armas.
Tropel.
O mesmo que \textunderscore bebedeira\textunderscore .
\textunderscore Carga maiór\textunderscore , carga de bêsta muar ou cavallar.
\textunderscore Carga menór\textunderscore , carga de bêsta asinina.
\section{Cargar}
\begin{itemize}
\item {Grp. gram.:v. t.}
\end{itemize}
O mesmo que \textunderscore carregar\textunderscore . Cf. Filinto, VI, 106; VII, 20 e 175; XIII, 447.
(Contr. de \textunderscore carregar\textunderscore )
\section{Çargaço}
\begin{itemize}
\item {Grp. gram.:m.}
\end{itemize}
(Fórma talvez preferível a \textunderscore sargaço\textunderscore )
\section{Cargo}
\begin{itemize}
\item {Grp. gram.:m.}
\end{itemize}
\begin{itemize}
\item {Proveniência:(De \textunderscore cargar\textunderscore )}
\end{itemize}
Carga.
Encargo; emprêgo público.
Responsabilidade.
Despesa.
Obrigação.
\section{Çarguça}
\begin{itemize}
\item {Grp. gram.:f.}
\end{itemize}
\begin{itemize}
\item {Utilização:Ant.}
\end{itemize}
Certo pano da Índia portuguesa.
(Cp. \textunderscore saragoça\textunderscore ^1)
\section{Cargueiro}
\begin{itemize}
\item {Grp. gram.:m.  e  adj.}
\end{itemize}
\begin{itemize}
\item {Utilização:Bras. do S}
\end{itemize}
\begin{itemize}
\item {Proveniência:(De \textunderscore carga\textunderscore )}
\end{itemize}
O que leva ou guia bêstas de carga.
Mau cavalleiro.
\section{Carguejar}
\begin{itemize}
\item {Grp. gram.:v. i.}
\end{itemize}
\begin{itemize}
\item {Grp. gram.:V. t.}
\end{itemize}
\begin{itemize}
\item {Proveniência:(De \textunderscore carga\textunderscore )}
\end{itemize}
Empregar-se em transportes de fardos.
Conduzir bêstas de carga.
Transportar.
\section{Cariacu}
\begin{itemize}
\item {Grp. gram.:m.}
\end{itemize}
\begin{itemize}
\item {Utilização:Bras. do N}
\end{itemize}
Pequeno veado, sarapintado de branco.
\section{Cariado}
\begin{itemize}
\item {Grp. gram.:adj.}
\end{itemize}
Que tem cárie: \textunderscore dentes cariados\textunderscore .
\section{Cariama}
\begin{itemize}
\item {Grp. gram.:f.}
\end{itemize}
Ave pernalta do Brasil.
\section{Cariar}
\begin{itemize}
\item {Grp. gram.:v. t.}
\end{itemize}
\begin{itemize}
\item {Grp. gram.:V. i.}
\end{itemize}
Encher de cárie.
Corromper.
Criar cárie.
Corromper-se.
\section{Caribé}
\begin{itemize}
\item {Grp. gram.:m.}
\end{itemize}
\begin{itemize}
\item {Utilização:Bras}
\end{itemize}
Iguaria, preparada com polpa de abacate.
Qualquer farinha fina.
\section{Caribes}
\begin{itemize}
\item {Grp. gram.:m. pl.}
\end{itemize}
Tríbo de Índios da Guiana brasileira.
\section{Cariboca}
\begin{itemize}
\item {Grp. gram.:m.  e  f.}
\end{itemize}
\begin{itemize}
\item {Utilização:Bras}
\end{itemize}
\begin{itemize}
\item {Proveniência:(T. tupi)}
\end{itemize}
Pessôa, que procede de europeus e caboclos.
Mestiço.
\section{Cárica}
\begin{itemize}
\item {Grp. gram.:f.}
\end{itemize}
Gênero de plantas arbóreas da América tropical.
\section{Caricacho}
\begin{itemize}
\item {Grp. gram.:m.}
\end{itemize}
\begin{itemize}
\item {Utilização:Prov.}
\end{itemize}
\begin{itemize}
\item {Utilização:trasm.}
\end{itemize}
Pedaço de terreno.
\section{Caricar}
\begin{itemize}
\item {Grp. gram.:v. t.}
\end{itemize}
\begin{itemize}
\item {Utilização:P. us.}
\end{itemize}
\begin{itemize}
\item {Proveniência:(It. \textunderscore caricare\textunderscore )}
\end{itemize}
O mesmo que \textunderscore caricaturar\textunderscore .
\section{Caricato}
\begin{itemize}
\item {Grp. gram.:adj.}
\end{itemize}
\begin{itemize}
\item {Grp. gram.:M.}
\end{itemize}
\begin{itemize}
\item {Proveniência:(It. \textunderscore caricato\textunderscore )}
\end{itemize}
Ridículo, burlesco: \textunderscore modos caricatos\textunderscore .
Actor, que, nos dramas, tem o papel de satirizar ou ridiculizar.
\section{Caricatura}
\begin{itemize}
\item {Grp. gram.:f.}
\end{itemize}
\begin{itemize}
\item {Proveniência:(It. \textunderscore caricatura\textunderscore )}
\end{itemize}
Representação burlesca de pessôas ou acontecimentos.
Imitação cómica.
Pessôa ridícula, pelos seus modos ou pelo seu aspecto.
\section{Caricatural}
\begin{itemize}
\item {Grp. gram.:adj.}
\end{itemize}
Relativo a caricatura.
\section{Caricaturar}
\begin{itemize}
\item {Grp. gram.:v. t.}
\end{itemize}
Representar em caricatura.
\section{Caricaturista}
\begin{itemize}
\item {Grp. gram.:m.}
\end{itemize}
Artista, que faz caricaturas.
\section{Carícea}
\begin{itemize}
\item {Grp. gram.:f.}
\end{itemize}
\begin{itemize}
\item {Proveniência:(Do lat. \textunderscore carex\textunderscore , \textunderscore caricis\textunderscore )}
\end{itemize}
Insecto díptero, que vive nas plantas e nos pântanos.
\section{Carícia}
\begin{itemize}
\item {Grp. gram.:f.}
\end{itemize}
Manifestação de affecto; carinho, afago.
(B. lat. \textunderscore caritia\textunderscore , do lat. \textunderscore carus\textunderscore )
\section{Cariciar}
\begin{itemize}
\item {Grp. gram.:v. t.}
\end{itemize}
(V.acariciar)
\section{Cariciativo}
\begin{itemize}
\item {Grp. gram.:m.}
\end{itemize}
O mesmo que \textunderscore acariciador\textunderscore . Cf. Camillo, \textunderscore Bruxa\textunderscore , 42.
\section{Cariciável}
\begin{itemize}
\item {Grp. gram.:adj.}
\end{itemize}
Caricioso, agradável, lisonjeiro:«\textunderscore cujo presente foi mui cariciável a Meliqueaz\textunderscore ». Filinto, \textunderscore D. Man.\textunderscore , III, 12. Cf. Camillo, \textunderscore Mulher Fatal\textunderscore , 107.
\section{Cariciosamente}
\begin{itemize}
\item {Grp. gram.:adv.}
\end{itemize}
De modo caricioso.
\section{Caricioso}
\begin{itemize}
\item {Grp. gram.:adj.}
\end{itemize}
Que faz carícias.
Carinhoso.
\section{Cariçó}
\begin{itemize}
\item {Grp. gram.:m.}
\end{itemize}
Planta gramínea de Cabo Verde.
\section{Caridade}
\begin{itemize}
\item {Grp. gram.:f.}
\end{itemize}
\begin{itemize}
\item {Utilização:Bras. do N}
\end{itemize}
\begin{itemize}
\item {Utilização:Irón.}
\end{itemize}
\begin{itemize}
\item {Utilização:Ant.}
\end{itemize}
\begin{itemize}
\item {Utilização:Ant.}
\end{itemize}
\begin{itemize}
\item {Utilização:Ant.}
\end{itemize}
\begin{itemize}
\item {Proveniência:(Lat. \textunderscore caritas\textunderscore )}
\end{itemize}
Amor ao próximo.
Benevolência.
Beneficência, esmola.
Bolo de farinha de trigo, manteiga, açúcar e ovos.
Offensa, damno.
Banquete ou bodo aos pobres.
Pitança, que se dava ás communidades dos monges ou cónegos.
Hospital para enfermos pobres.
Nome de uma antiga Ordem religiosa.
Solennidade religiosa, por intenção das almas do Purgatório.
\textunderscore Vossa caridade\textunderscore , tratamento, que os Prelados davam aos seus diocesanos.
\section{Caridosa}
\begin{itemize}
\item {Grp. gram.:f.}
\end{itemize}
\begin{itemize}
\item {Utilização:Prov.}
\end{itemize}
\begin{itemize}
\item {Utilização:alent.}
\end{itemize}
Espécie de dança de roda.
(Cp. \textunderscore carinhosa\textunderscore )
\section{Caridosamente}
\begin{itemize}
\item {Grp. gram.:adv.}
\end{itemize}
De modo caridoso.
\section{Caridoso}
\begin{itemize}
\item {Grp. gram.:adj.}
\end{itemize}
Em que há caridade.
Que tem caridade.
\section{Cárie}
\begin{itemize}
\item {Grp. gram.:f.}
\end{itemize}
\begin{itemize}
\item {Proveniência:(Lat. \textunderscore caries\textunderscore )}
\end{itemize}
Ulceração nos dentes e ossos, que os destrói progressivamente.
Caruncho.
Doença dos vegetaes, semelhante á cárie.
Destruição progressiva.
\section{Cáries}
\begin{itemize}
\item {Grp. gram.:f.}
\end{itemize}
(V.cárie)
\section{Carifranzido}
\begin{itemize}
\item {fónica:cá}
\end{itemize}
\begin{itemize}
\item {Grp. gram.:adj.}
\end{itemize}
\begin{itemize}
\item {Utilização:P. us.}
\end{itemize}
\begin{itemize}
\item {Proveniência:(De \textunderscore cara\textunderscore  + \textunderscore franzido\textunderscore )}
\end{itemize}
Que tem cara enrugada.
Que está de mau humor; carrancudo.
\section{Çarigueia}
\begin{itemize}
\item {Grp. gram.:f.}
\end{itemize}
O mesmo ou melhor que \textunderscore sarigueia\textunderscore .
\section{Carijo}
\begin{itemize}
\item {Grp. gram.:m.}
\end{itemize}
\begin{itemize}
\item {Utilização:Bras}
\end{itemize}
Armação de varas, em que se suspendem os ramos da congonha.
\section{Carijó}
\begin{itemize}
\item {Grp. gram.:f.  e  adj.}
\end{itemize}
\begin{itemize}
\item {Utilização:Bras}
\end{itemize}
\begin{itemize}
\item {Grp. gram.:M.}
\end{itemize}
\begin{itemize}
\item {Utilização:Bras}
\end{itemize}
\begin{itemize}
\item {Grp. gram.:M. pl.}
\end{itemize}
Diz-se da gallinha pedrês, talvez porque os Carijós, na guerra, se cobriam de pelles de onça, que têm o aspecto das pennas da carijó.
Espécie de cipó.
Índios do Brasil, que formavam nação numerosa, ao norte da lagôa dos Patos.
\section{Caril}
\begin{itemize}
\item {Grp. gram.:m.}
\end{itemize}
Pó indiano, composto de várias especiarias, para adubos culinários.
Môlho, em que entra êsse pó e sumo de tamarindo, coco, carne ou peixe, malagueta, e que depois se mistura com arroz cozido.
(Do canarim \textunderscore karil\textunderscore )
\section{Carilha}
\begin{itemize}
\item {Grp. gram.:f.}
\end{itemize}
Planta verbenácea da Índia portuguesa, (\textunderscore vitex alata\textunderscore , Heyne), de madeira útil.
\section{Carilho}
\begin{itemize}
\item {Grp. gram.:m.}
\end{itemize}
\begin{itemize}
\item {Utilização:Ant.}
\end{itemize}
Máquina, com que se fiava e dobava a seda.
(Por \textunderscore carrilho\textunderscore )
\section{Carimá}
\begin{itemize}
\item {Grp. gram.:f.}
\end{itemize}
\begin{itemize}
\item {Utilização:Bras}
\end{itemize}
\begin{itemize}
\item {Proveniência:(T. tupi)}
\end{itemize}
Bolo, feito da massa grossa da mandioca.
Farinha de mandioca, de que se fazem caldos para crianças.
\section{Cariman}
\begin{itemize}
\item {Grp. gram.:m.}
\end{itemize}
\begin{itemize}
\item {Utilização:Bras. do N}
\end{itemize}
\begin{itemize}
\item {Proveniência:(T. tupi)}
\end{itemize}
Bolo, feito da massa grossa da mandioca.
Farinha de mandioca, de que se fazem caldos para crianças.
\section{Carimbado}
\begin{itemize}
\item {Grp. gram.:adj.}
\end{itemize}
\begin{itemize}
\item {Utilização:Chul.}
\end{itemize}
Marcado com carimbo.
Reprovado, chumbado.
\section{Carimbador}
\begin{itemize}
\item {Grp. gram.:m.}
\end{itemize}
Aquelle que carimba a correspondência postal.
\section{Carimbagem}
\begin{itemize}
\item {Grp. gram.:f.}
\end{itemize}
Acto ou effeito de carimbar.
\section{Carimbamba}
\begin{itemize}
\item {Grp. gram.:m.}
\end{itemize}
\begin{itemize}
\item {Utilização:Bras. de Minas}
\end{itemize}
O mesmo que \textunderscore curandeiro\textunderscore .
\section{Carimbar}
\begin{itemize}
\item {Grp. gram.:v. t.}
\end{itemize}
\begin{itemize}
\item {Utilização:Chul.}
\end{itemize}
Marcar com carimbo.
Reprovar.
\section{Carimbo}
\begin{itemize}
\item {Grp. gram.:m.}
\end{itemize}
\begin{itemize}
\item {Proveniência:(Do quimb. \textunderscore karimba\textunderscore )}
\end{itemize}
Instrumento de metal, madeira ou borracha, que serve para marcar papéis de uso official ou particular.
Sêllo; sinete.
\section{Carina}
\begin{itemize}
\item {Grp. gram.:f.}
\end{itemize}
\begin{itemize}
\item {Proveniência:(Lat. \textunderscore carina\textunderscore )}
\end{itemize}
(V.carena)
\section{Carinado}
\begin{itemize}
\item {Grp. gram.:adj.}
\end{itemize}
\begin{itemize}
\item {Proveniência:(De \textunderscore carina\textunderscore )}
\end{itemize}
Que tem carena ou é semelhante á carena.
\section{Carinão}
\begin{itemize}
\item {Grp. gram.:m.}
\end{itemize}
Nome, que na Índia portuguesa se dá á árvore que produz a noz-vómica, (\textunderscore strychnos nux-vomica\textunderscore , Lin.).
\section{Carinária}
\begin{itemize}
\item {Grp. gram.:f.}
\end{itemize}
\begin{itemize}
\item {Proveniência:(Do lat. \textunderscore carina\textunderscore )}
\end{itemize}
Mollúsco gasterópode, gelatinoso e transparente.
\section{Carinegro}
\begin{itemize}
\item {fónica:cá}
\end{itemize}
\begin{itemize}
\item {Grp. gram.:adj.}
\end{itemize}
Que tem cara negra.
\section{Carinha}
\begin{itemize}
\item {Grp. gram.:f.}
\end{itemize}
\begin{itemize}
\item {Utilização:Gír.}
\end{itemize}
\begin{itemize}
\item {Proveniência:(De \textunderscore cara\textunderscore )}
\end{itemize}
Moéda de 500 reis em prata.
\section{Carinho}
\begin{itemize}
\item {Grp. gram.:m.}
\end{itemize}
\begin{itemize}
\item {Proveniência:(De \textunderscore caro\textunderscore )}
\end{itemize}
Carícia, afago.
Cuidado.
\section{Carinhosa}
\begin{itemize}
\item {Grp. gram.:f.}
\end{itemize}
\begin{itemize}
\item {Utilização:Prov.}
\end{itemize}
\begin{itemize}
\item {Utilização:alg.}
\end{itemize}
\begin{itemize}
\item {Utilização:Prov.}
\end{itemize}
\begin{itemize}
\item {Utilização:beir.}
\end{itemize}
Capuz de senhora.
Espécie de dança de roda. (Da primeira palavra da canção que acompanha essa dança):«\textunderscore carinhosa, minha carinhosa...\textunderscore »
\section{Carinhosamente}
\begin{itemize}
\item {Grp. gram.:adv.}
\end{itemize}
De modo \textunderscore carinhoso\textunderscore .
Com carinho.
\section{Carinhoso}
\begin{itemize}
\item {Grp. gram.:adj.}
\end{itemize}
Que tem carinho.
Em que há carinho.
\section{Carinífero}
\begin{itemize}
\item {Grp. gram.:adj.}
\end{itemize}
\begin{itemize}
\item {Proveniência:(Do lat. \textunderscore carina\textunderscore  + \textunderscore ferre\textunderscore )}
\end{itemize}
Que tem carina ou carena, (falando-se de vegetaes).
\section{Carioca}
\begin{itemize}
\item {Grp. gram.:m.  e  f.}
\end{itemize}
\begin{itemize}
\item {Utilização:Bras}
\end{itemize}
\begin{itemize}
\item {Utilização:Bras. de Minas}
\end{itemize}
Pessôa do Rio-de-Janeiro. Cf. Camillo, \textunderscore Serões\textunderscore , I, 14.
Que tem pintas na pelle: \textunderscore porco carioca\textunderscore .
(Do tupi \textunderscore cari\textunderscore , branco e \textunderscore oca\textunderscore , casa)
\section{Carioso}
\begin{itemize}
\item {Grp. gram.:adj.}
\end{itemize}
\begin{itemize}
\item {Proveniência:(Lat. \textunderscore cariosus\textunderscore )}
\end{itemize}
Relativo á cárie.
Cariado.
\section{Cariota}
\begin{itemize}
\item {Grp. gram.:f.}
\end{itemize}
\begin{itemize}
\item {Proveniência:(Gr. \textunderscore karuotos\textunderscore )}
\end{itemize}
Espécie de palmeira da Índia.
Gênero das palmeiras que dão tâmaras.
\section{Caripunas}
\begin{itemize}
\item {Grp. gram.:m. pl.}
\end{itemize}
Tríbo de indios da Guiana brasileira, junto das possessões holandesas.
\section{Cariri}
\begin{itemize}
\item {Grp. gram.:m.}
\end{itemize}
\begin{itemize}
\item {Utilização:Bras. do N}
\end{itemize}
Fôrça; esfôrço.
\section{Cariris}
\begin{itemize}
\item {Grp. gram.:m. pl.}
\end{itemize}
Selvagens, que habitaram na região de Paraíba, no Brasil.
\section{Carístias}
\begin{itemize}
\item {Grp. gram.:f. pl.}
\end{itemize}
\begin{itemize}
\item {Proveniência:(Lat. \textunderscore caristia\textunderscore )}
\end{itemize}
Banquetes de família, em que os antigos Romanos não admittiam gente estranha. Cf. Castilho, \textunderscore Fastos\textunderscore , I, 145 e 609.
\section{Caritativamente}
\begin{itemize}
\item {Grp. gram.:adv.}
\end{itemize}
De modo \textunderscore caritativo\textunderscore .
Com caridade: \textunderscore tratar caritativamente\textunderscore .
\section{Caritativo}
\begin{itemize}
\item {Grp. gram.:adj.}
\end{itemize}
O mesmo que \textunderscore caridoso\textunderscore .
\section{Caritel}
\begin{itemize}
\item {Grp. gram.:m.}
\end{itemize}
\begin{itemize}
\item {Utilização:Ant.}
\end{itemize}
Soccorro? alarma?--Em terras do Duque de Bragança e nas de outros Duques, \textunderscore aqui do duque\textunderscore ! era a voz de \textunderscore caritel\textunderscore .
\section{Caritenho}
\begin{itemize}
\item {Grp. gram.:m.}
\end{itemize}
\begin{itemize}
\item {Utilização:Ant.}
\end{itemize}
\begin{itemize}
\item {Proveniência:(Do lat. \textunderscore caritas\textunderscore )}
\end{itemize}
Pequeno breviário ou livro de ladaínhas.
\section{Carito}
\begin{itemize}
\item {Grp. gram.:m.}
\end{itemize}
\begin{itemize}
\item {Utilização:Prov.}
\end{itemize}
\begin{itemize}
\item {Utilização:dur.}
\end{itemize}
\begin{itemize}
\item {Utilização:minh.}
\end{itemize}
Orifício na parte supero-lateral de uma vasilha, para medir líquidos, até chegarem a êsse orifício; ou ponteiro soldado na parte interna da vasilha, á mesma altura, e que substitue o orifício, para o effeito da medida.
\section{Caritó}
\begin{itemize}
\item {Grp. gram.:m.}
\end{itemize}
\begin{itemize}
\item {Utilização:Bras. do N}
\end{itemize}
Casa pobre.
Pequena prateleira a um canto.
\section{Cariz}
\begin{itemize}
\item {Grp. gram.:m.}
\end{itemize}
\begin{itemize}
\item {Proveniência:(T. cast.)}
\end{itemize}
Semblante; aspecto.
Apparência atmosphérica.
Alcaravia.
\section{Carlá}
\begin{itemize}
\item {Grp. gram.:m.}
\end{itemize}
Antigo estôfo indiano.
\section{Carlagani}
\begin{itemize}
\item {Grp. gram.:m.}
\end{itemize}
Espécie de tecido indiano.
\section{Carlequim}
\begin{itemize}
\item {Grp. gram.:m.}
\end{itemize}
\begin{itemize}
\item {Utilização:Ant.}
\end{itemize}
Pequeno instrumento, semelhante ao bate-estacas, com que se introduzia a espoleta no ouvido das bombas ou granadas.
\section{Carlina}
\begin{itemize}
\item {Grp. gram.:f.}
\end{itemize}
\begin{itemize}
\item {Proveniência:(De \textunderscore Carlos\textunderscore , n. p.)}
\end{itemize}
Nome de várias plantas medicinaes.
\section{Carlina}
\begin{itemize}
\item {Grp. gram.:f.}
\end{itemize}
Cada uma das travessas, que prendem as longarinas, na construcção das pontes.
(Alter. de \textunderscore carlinga\textunderscore ?)
\section{Carlinga}
\begin{itemize}
\item {Grp. gram.:f.}
\end{itemize}
\begin{itemize}
\item {Utilização:Náut.}
\end{itemize}
\begin{itemize}
\item {Proveniência:(It. \textunderscore carlinga\textunderscore )}
\end{itemize}
Peça de madeira, em que assenta o mastro grande.
Sobrequilha.
\section{Carlíngio}
\begin{itemize}
\item {Grp. gram.:adj.}
\end{itemize}
\begin{itemize}
\item {Proveniência:(De \textunderscore Carlos\textunderscore , n. p.)}
\end{itemize}
O mesmo que \textunderscore carlovíngio\textunderscore .
\section{Carlino}
\begin{itemize}
\item {Grp. gram.:m.}
\end{itemize}
\begin{itemize}
\item {Proveniência:(It. \textunderscore carlino\textunderscore )}
\end{itemize}
Moéda italiana.
\section{Carlismo}
\begin{itemize}
\item {Grp. gram.:m.}
\end{itemize}
\begin{itemize}
\item {Proveniência:(De \textunderscore Carlos\textunderscore , n. p.)}
\end{itemize}
Partido dos Carlistas.
\section{Carlista}
\begin{itemize}
\item {Grp. gram.:m.}
\end{itemize}
\begin{itemize}
\item {Proveniência:(De \textunderscore Carlos\textunderscore , n. p.)}
\end{itemize}
Partidário de Carlos de Borbon, pretendente do throno espanhol.
\section{Carlos-quinto}
\begin{itemize}
\item {Grp. gram.:m.}
\end{itemize}
Espécie de capa curta, hoje quási desusada e semelhante á de alguns officiaes da armada.
\section{Carlota}
\begin{itemize}
\item {Grp. gram.:f.}
\end{itemize}
Espécie de oliveira.
Azeitona, também conhecida por \textunderscore cerieira\textunderscore  ou \textunderscore ceriosa\textunderscore .
\section{Carlovingiano}
\begin{itemize}
\item {Grp. gram.:adj.}
\end{itemize}
\begin{itemize}
\item {Proveniência:(De \textunderscore Carlos\textunderscore , n. p.)}
\end{itemize}
Relativo á dynastia de Carlos-Magno.
\section{Carlovíngio}
\begin{itemize}
\item {Grp. gram.:adj.}
\end{itemize}
\begin{itemize}
\item {Proveniência:(De \textunderscore Carlos\textunderscore , n. p.)}
\end{itemize}
Relativo á dynastia de Carlos-Magno.
\section{Carludovica}
\begin{itemize}
\item {Grp. gram.:f.}
\end{itemize}
Gênero de plantas vivazes, intertropicaes, de cujas espécies uma produz a palha dos chamados chapéus do Chile.
(Do nome de Carlos IV de Espanha, e do de sua mulher Maria Luisa, alatinado, \textunderscore Ludovica\textunderscore )
\section{Carmanhola}
\begin{itemize}
\item {Grp. gram.:f.}
\end{itemize}
\begin{itemize}
\item {Proveniência:(De \textunderscore Carmagnole\textunderscore , n. p.)}
\end{itemize}
Canção e dança dos revolucionários de 1793.
\section{Carmático}
\begin{itemize}
\item {Grp. gram.:adj.}
\end{itemize}
Diz-se de uma escrita arábica, em que se não empregam sinaes diacríticos.
\section{Carme}
\begin{itemize}
\item {Grp. gram.:m.}
\end{itemize}
\begin{itemize}
\item {Proveniência:(Lat. \textunderscore carmen\textunderscore )}
\end{itemize}
Canto.
Poema.
Versos.
\section{Carmear}
\textunderscore v. t.\textunderscore  (e der.)
O mesmo que \textunderscore carmiar\textunderscore , etc., que é talvez orthogr. mais correcta.
\section{Carmelina}
\begin{itemize}
\item {Grp. gram.:f.}
\end{itemize}
Lan inferior de vicunha.
(Cast. \textunderscore carmelina\textunderscore )
\section{Carmelita}
\begin{itemize}
\item {Grp. gram.:m.  e  f.}
\end{itemize}
\begin{itemize}
\item {Proveniência:(De \textunderscore Carmelo\textunderscore , n. p.)}
\end{itemize}
Frade ou freira de qualquer das Ordens religiosas de N. S. do Carmo ou do Monte-Carmelo, com várias ramificações, como os \textunderscore observantinos\textunderscore , os \textunderscore conventuaes\textunderscore  e os \textunderscore carmelitas descalços\textunderscore .
\section{Carmelitano}
\begin{itemize}
\item {Grp. gram.:adj.}
\end{itemize}
\begin{itemize}
\item {Proveniência:(De \textunderscore Carmelita\textunderscore )}
\end{itemize}
Relativo aos Carmelitas.
\section{Carmelo}
\begin{itemize}
\item {Grp. gram.:m.}
\end{itemize}
\begin{itemize}
\item {Utilização:Pop.}
\end{itemize}
O mesmo que \textunderscore caramelo\textunderscore . Cf. Camillo, \textunderscore Freira no Subterran.\textunderscore , 16.
\section{Carmental}
\begin{itemize}
\item {Grp. gram.:adj.}
\end{itemize}
\begin{itemize}
\item {Proveniência:(Lat. \textunderscore carmentalis\textunderscore )}
\end{itemize}
Dizia-se de uma porta de Roma. Cf. Castilho, \textunderscore Fastos\textunderscore , I, 521.
\section{Carmesim}
\begin{itemize}
\item {Grp. gram.:m.  e  adj.}
\end{itemize}
\begin{itemize}
\item {Proveniência:(Do ár. \textunderscore kirmizi\textunderscore )}
\end{itemize}
Côr vermelha muito viva.
Vermelho.
\section{Carmesinado}
\begin{itemize}
\item {Grp. gram.:adj.}
\end{itemize}
Revestido de carmesim:«\textunderscore tamborete carmesinado\textunderscore ». Camillo, \textunderscore Ninães\textunderscore , 26.
\section{Carmiadeira}
\begin{itemize}
\item {Grp. gram.:f.}
\end{itemize}
Mulher que carmia.
\section{Carmiador}
\begin{itemize}
\item {Grp. gram.:m.}
\end{itemize}
\begin{itemize}
\item {Proveniência:(Lat. \textunderscore carminator\textunderscore )}
\end{itemize}
Aquelle que carmia.
\section{Carmiar}
\begin{itemize}
\item {Grp. gram.:v. t.}
\end{itemize}
\begin{itemize}
\item {Proveniência:(Do lat. \textunderscore carminare\textunderscore )}
\end{itemize}
Desenredar, desfazer os nós de (lan, antes de carduçada).
\section{Carmim}
\begin{itemize}
\item {Grp. gram.:m.}
\end{itemize}
\begin{itemize}
\item {Proveniência:(Do ár. \textunderscore quirmiz\textunderscore )}
\end{itemize}
Tinta vermelha muito viva, que se extrai de várias plantas.
\section{Carmina}
\begin{itemize}
\item {Grp. gram.:f.}
\end{itemize}
\begin{itemize}
\item {Proveniência:(De \textunderscore carmim\textunderscore )}
\end{itemize}
Essência colorante da cochonilha.
\section{Carminar}
\begin{itemize}
\item {Grp. gram.:v. t.}
\end{itemize}
Tingir de carmim.
\section{Carminar}
\textunderscore v. t.\textunderscore  (e der.)
O mesmo que \textunderscore carmiar\textunderscore , etc.
\section{Carminativo}
\begin{itemize}
\item {Grp. gram.:adj.}
\end{itemize}
\begin{itemize}
\item {Grp. gram.:M.}
\end{itemize}
Anti-flatulento.
Medicamento contra gases intestinaes.
(B. lat. \textunderscore carminativum\textunderscore )
\section{Carmíneo}
\begin{itemize}
\item {Grp. gram.:adj.}
\end{itemize}
Que tem côr de carmim.
\section{Carminol}
\begin{itemize}
\item {Grp. gram.:m.}
\end{itemize}
Substância chímica, empregada como reagente em viticultura.
\section{Carmona}
\begin{itemize}
\item {Grp. gram.:f.}
\end{itemize}
\begin{itemize}
\item {Proveniência:(Fr. \textunderscore cremone\textunderscore )}
\end{itemize}
Ferrolho que, posto em toda a altura da janela ou porta, se embebe ao mesmo tempo em cima e em baixo.
\section{Carmona}
\begin{itemize}
\item {Grp. gram.:f.}
\end{itemize}
\begin{itemize}
\item {Proveniência:(De \textunderscore Carmona\textunderscore , n. p.)}
\end{itemize}
Casaquinho curto de senhora, um pouco semelhante á jaqueta de toireiro.
\section{Carmoso}
\begin{itemize}
\item {Grp. gram.:m.}
\end{itemize}
\begin{itemize}
\item {Utilização:Gír. de Lisbôa.}
\end{itemize}
Tostão.
\section{Carmufélico}
\begin{itemize}
\item {Grp. gram.:adj.}
\end{itemize}
Diz-se de um ácido extrahido do cravo da Índia.
\section{Carnaça}
\begin{itemize}
\item {Grp. gram.:f.}
\end{itemize}
\begin{itemize}
\item {Utilização:Pop.}
\end{itemize}
Grande porção de carne.
Excrescência carnosa.
\section{Carnação}
\begin{itemize}
\item {Grp. gram.:f.}
\end{itemize}
\begin{itemize}
\item {Proveniência:(Lat. \textunderscore carnatio\textunderscore )}
\end{itemize}
Côr da carne.
\section{Carnada}
\begin{itemize}
\item {Grp. gram.:f.}
\end{itemize}
\begin{itemize}
\item {Proveniência:(De \textunderscore carne\textunderscore )}
\end{itemize}
Isca de cabeças e tripas de sardinha, empregada em alguns apparelhos de pesca.
\section{Carnadura}
\begin{itemize}
\item {Grp. gram.:f.}
\end{itemize}
\begin{itemize}
\item {Proveniência:(Do lat. \textunderscore carnatus\textunderscore )}
\end{itemize}
Qualidade da carne.
Compleição, natureza; musculatura: \textunderscore tem má carnadura o rapaz\textunderscore .
\section{Carnaes}
\begin{itemize}
\item {Grp. gram.:m. pl.}
\end{itemize}
\begin{itemize}
\item {Proveniência:(De \textunderscore carnal\textunderscore )}
\end{itemize}
Primos em primeiro grau.
\section{Carnagem}
\begin{itemize}
\item {Grp. gram.:f.}
\end{itemize}
\begin{itemize}
\item {Proveniência:(De \textunderscore carne\textunderscore )}
\end{itemize}
Morticínio de animaes, para alimentação do homem.
Abastecimento de carne.
Mortandade (de gente)
\section{Carnais}
\begin{itemize}
\item {Grp. gram.:m. pl.}
\end{itemize}
\begin{itemize}
\item {Proveniência:(De \textunderscore carnal\textunderscore )}
\end{itemize}
Primos em primeiro grau.
\section{Carnal}
\begin{itemize}
\item {Grp. gram.:adj.}
\end{itemize}
\begin{itemize}
\item {Grp. gram.:M.}
\end{itemize}
\begin{itemize}
\item {Grp. gram.:F.}
\end{itemize}
\begin{itemize}
\item {Proveniência:(Lat. \textunderscore carnalis\textunderscore )}
\end{itemize}
Relativo á carne.
Que é de carne.
Lascivo: \textunderscore appetites carnaes\textunderscore .
Consanguíneo: \textunderscore primo carnal\textunderscore .
Tempo, em que a Igreja permitte comer carne; carnário.
Uva branca da região do Doiro.
\section{Carnalidade}
\begin{itemize}
\item {Grp. gram.:f.}
\end{itemize}
\begin{itemize}
\item {Proveniência:(Lat. \textunderscore carnalitas\textunderscore )}
\end{itemize}
Sensualidade.
\section{Carnalmente}
\begin{itemize}
\item {Grp. gram.:adv.}
\end{itemize}
De modo carnal.
Sensualmente; com lascívia.
\section{Carnar}
\begin{itemize}
\item {Grp. gram.:v. t.}
\end{itemize}
\begin{itemize}
\item {Proveniência:(De \textunderscore carne\textunderscore )}
\end{itemize}
Unir por parentesco. Cf. Camillo, \textunderscore Demónio do Oiro\textunderscore .
\section{Carnário}
\begin{itemize}
\item {Grp. gram.:m.}
\end{itemize}
\begin{itemize}
\item {Proveniência:(De \textunderscore carne\textunderscore )}
\end{itemize}
Tempo alheio á Quaresma, e em que a Igreja não veda o uso da carne.
\section{Carnaúba}
\begin{itemize}
\item {Grp. gram.:f.}
\end{itemize}
Espécie de sebo, que se extrai da carnaubeira.
O mesmo que \textunderscore carnaubeira\textunderscore .
(Do tupi)
\section{Carnaubeira}
\begin{itemize}
\item {fónica:na-u}
\end{itemize}
\begin{itemize}
\item {Grp. gram.:f.}
\end{itemize}
\begin{itemize}
\item {Proveniência:(De \textunderscore carnaúba\textunderscore )}
\end{itemize}
Espécie de palmeira do Brasil.
\section{Carnaval}
\begin{itemize}
\item {Grp. gram.:m.}
\end{itemize}
\begin{itemize}
\item {Proveniência:(Do lat. \textunderscore carrus\textunderscore  + \textunderscore navalis\textunderscore , segundo \textunderscore Körting\textunderscore )}
\end{itemize}
Época immediatamente anterior á Quaresma.
Tempo de folia, que precede a quarta-feira de cinza.
Folguedo, orgia.
Entrudo.
\section{Carnavalesco}
\begin{itemize}
\item {Grp. gram.:adj.}
\end{itemize}
Relativo ao carnaval.
Grutesco.
\section{Carnaz}
\begin{itemize}
\item {Grp. gram.:m.}
\end{itemize}
\begin{itemize}
\item {Utilização:Prov.}
\end{itemize}
\begin{itemize}
\item {Utilização:trasm.}
\end{itemize}
\begin{itemize}
\item {Proveniência:(De \textunderscore carne\textunderscore )}
\end{itemize}
O lado da pelle, opposto á cútis ou ao pêlo.
O mesmo que \textunderscore carnário\textunderscore .
Avêsso.
\section{Carne}
\begin{itemize}
\item {Grp. gram.:f.}
\end{itemize}
\begin{itemize}
\item {Proveniência:(Lat. \textunderscore caro\textunderscore , \textunderscore carnis\textunderscore )}
\end{itemize}
Tecido muscular do corpo humano e do corpo dos animaes.
Parte vermelha dos músculos.
Tecido muscular dos animaes que servem para alimentação do homem.
Consanguinidade.
Natureza animal: \textunderscore a carne é fraca\textunderscore .
Mesocarpo.
Sensualidade: \textunderscore as tentações da carne\textunderscore .
\section{Carneação}
\begin{itemize}
\item {Grp. gram.:f.}
\end{itemize}
\begin{itemize}
\item {Utilização:Bras}
\end{itemize}
Acto de \textunderscore carnear\textunderscore .
\section{Carnear}
\begin{itemize}
\item {Grp. gram.:v. t.  e  i.}
\end{itemize}
\begin{itemize}
\item {Utilização:Bras}
\end{itemize}
\begin{itemize}
\item {Proveniência:(De \textunderscore carne\textunderscore )}
\end{itemize}
Matar gado.
\section{Carnecoita}
\begin{itemize}
\item {Grp. gram.:adj.}
\end{itemize}
\begin{itemize}
\item {Proveniência:(Do lat. \textunderscore caro\textunderscore  + \textunderscore coctus\textunderscore )}
\end{itemize}
Diz-se da espécie de ameixa, que também se chama \textunderscore reinol\textunderscore .
\section{Carne-de-vaca}
\begin{itemize}
\item {Grp. gram.:f.}
\end{itemize}
Casta de uva de Tôrres-Vedras.
\section{Carneeiro}
\begin{itemize}
\item {Grp. gram.:adj.}
\end{itemize}
\begin{itemize}
\item {Utilização:Bras}
\end{itemize}
Que serve para cortar carne ou matar rêses:«\textunderscore tira a faca carneeira para o veado estripar.\textunderscore »Araújo Porto-Alegre.
\section{Carnegão}
\begin{itemize}
\item {Grp. gram.:m.}
\end{itemize}
O mesmo que \textunderscore carnicão\textunderscore .
\section{Carneira}
\begin{itemize}
\item {Grp. gram.:f.}
\end{itemize}
\begin{itemize}
\item {Proveniência:(De \textunderscore carneiro\textunderscore )}
\end{itemize}
Pelle de carneiro, preparada para differentes usos: \textunderscore sapatos de carneira\textunderscore .
\section{Carneira}
\begin{itemize}
\item {Grp. gram.:f.  e  adj.}
\end{itemize}
Uma espécie de abóbora.
\section{Carneirada}
\begin{itemize}
\item {Grp. gram.:f.}
\end{itemize}
\begin{itemize}
\item {Utilização:Bras}
\end{itemize}
\begin{itemize}
\item {Proveniência:(De \textunderscore carneiro\textunderscore ^1)}
\end{itemize}
Rebanho de carneiros.
Pequenas ondas espumosas.
Febres, peculiares á costa da África tropical.
Febre palústre. Cf. Goes, \textunderscore Govern. das Esmeraldas\textunderscore .
\section{Carneireiro}
\begin{itemize}
\item {Grp. gram.:m.}
\end{itemize}
\begin{itemize}
\item {Proveniência:(De \textunderscore carneiro\textunderscore ^1)}
\end{itemize}
Aquelle que tem carneiros; aquelle que os guarda.
\section{Carneiro}
\begin{itemize}
\item {Grp. gram.:m.}
\end{itemize}
\begin{itemize}
\item {Utilização:Bras}
\end{itemize}
\begin{itemize}
\item {Utilização:Prov.}
\end{itemize}
\begin{itemize}
\item {Utilização:minh.}
\end{itemize}
Quadrúpede ruminante e lanígero.
Animálculo, que se cria nas sementes leguminosas.
Ariete.
Pequena onda espumosa, seguida ou precedida de outras.
Constellação zodiacal, também chamada \textunderscore Áries\textunderscore .
Espécie de bomba.
Flôr do salgueiro.
(Cast. \textunderscore carnero\textunderscore )
\section{Carneiro}
\begin{itemize}
\item {Grp. gram.:m.}
\end{itemize}
Ossuário, jazigo; sepulcro.
Cemitério.
(B. lat. \textunderscore carnarium\textunderscore )
\section{Carneiró}
\begin{itemize}
\item {Grp. gram.:m.}
\end{itemize}
\begin{itemize}
\item {Utilização:Mad}
\end{itemize}
O mesmo que \textunderscore carreiró\textunderscore .
\section{Carneiro-almiscarado}
\begin{itemize}
\item {Grp. gram.:m.}
\end{itemize}
Mammífero fóssil, (\textunderscore ocibes moschatus\textunderscore , Blainv.), procedente das regiões árcticas.
\section{Carneirum}
\begin{itemize}
\item {Grp. gram.:adj.}
\end{itemize}
Relativo a carneiro: \textunderscore gado carneirum\textunderscore .
\section{Cárneo}
\begin{itemize}
\item {Grp. gram.:adj.}
\end{itemize}
\begin{itemize}
\item {Proveniência:(Lat. \textunderscore carneus\textunderscore )}
\end{itemize}
Relativo a carne: \textunderscore alimentação cárnea\textunderscore .
Que tem côr de carne.
\section{Carnéola}
\begin{itemize}
\item {Grp. gram.:f.}
\end{itemize}
\begin{itemize}
\item {Proveniência:(De \textunderscore carne\textunderscore )}
\end{itemize}
Variedade de calcedónia, da côr da carne ou arruivada.
Cornalina.
\section{Carnerina}
\begin{itemize}
\item {Grp. gram.:f.}
\end{itemize}
\begin{itemize}
\item {Proveniência:(De \textunderscore carne\textunderscore )}
\end{itemize}
Pedra preciosa, o mesmo que \textunderscore sárdio\textunderscore .
\section{Carnestolendas}
\begin{itemize}
\item {Grp. gram.:f. pl.}
\end{itemize}
\begin{itemize}
\item {Proveniência:(Do lat. \textunderscore caro\textunderscore , \textunderscore carnis\textunderscore , e \textunderscore tolendus\textunderscore , de \textunderscore ferre\textunderscore )}
\end{itemize}
O mesmo que \textunderscore carnaval\textunderscore . Cf. Latino, \textunderscore Camões\textunderscore , 136.
\section{Carniça}
\begin{itemize}
\item {Grp. gram.:f.}
\end{itemize}
\begin{itemize}
\item {Utilização:Bras. do N}
\end{itemize}
\begin{itemize}
\item {Proveniência:(De \textunderscore carne\textunderscore )}
\end{itemize}
Carne comestível.
Carnificina.
Pião, sôbre que se atiram outros.
Pessôa, que é objecto de motejos.
Despojos de um animal apodrecido.
\section{Carniçal}
\begin{itemize}
\item {Grp. gram.:adj.}
\end{itemize}
\begin{itemize}
\item {Proveniência:(De \textunderscore carniça\textunderscore )}
\end{itemize}
Carniceiro.
\section{Carnicão}
\begin{itemize}
\item {Grp. gram.:m.}
\end{itemize}
\begin{itemize}
\item {Proveniência:(De \textunderscore carne\textunderscore )}
\end{itemize}
Parte purulenta e dura de certos tumores.
\section{Carniçaria}
\begin{itemize}
\item {Grp. gram.:f.}
\end{itemize}
\begin{itemize}
\item {Proveniência:(De \textunderscore carniça\textunderscore )}
\end{itemize}
Carnificina.
Açougue.
Acto de preparar carne para a venda.
\section{Carniceiramente}
\begin{itemize}
\item {Grp. gram.:adv.}
\end{itemize}
\begin{itemize}
\item {Proveniência:(De \textunderscore carniceiro\textunderscore )}
\end{itemize}
Com crueldade; sanguinariamente.
\section{Carniceiro}
\begin{itemize}
\item {Grp. gram.:adj.}
\end{itemize}
\begin{itemize}
\item {Grp. gram.:M.}
\end{itemize}
\begin{itemize}
\item {Grp. gram.:Pl.}
\end{itemize}
\begin{itemize}
\item {Proveniência:(De \textunderscore carniça\textunderscore )}
\end{itemize}
Que se alimenta de carne; que prefere a carne como alimento: \textunderscore o lobo é animal carniceiro\textunderscore .
Sanguinário.
Aquelle que mata rêses para as vender a retalho; magarefe.
Ordem de mammíferos, cujo alimento principal são animaes vivos.
Família de coleópteros, que se alimentam de animálculos vivos.
\section{Carniçós}
\begin{itemize}
\item {Grp. gram.:m.}
\end{itemize}
\begin{itemize}
\item {Utilização:Prov.}
\end{itemize}
\begin{itemize}
\item {Utilização:trasm.}
\end{itemize}
\begin{itemize}
\item {Proveniência:(De \textunderscore carne\textunderscore ?)}
\end{itemize}
Designação vulgar da cravagem do centeio.
\section{Carnifazer}
\begin{itemize}
\item {Grp. gram.:v. i.}
\end{itemize}
Fazer carnagem ou prover-se de carne de animaes.
Fazer em postas ou pedaços os animaes caçados ou mortos, para alimentação do homem:«\textunderscore permittindo aos demais que carnifizessem também.\textunderscore »Filinto, \textunderscore D. Man.\textunderscore  III, 175.
\section{Carnificação}
\begin{itemize}
\item {Grp. gram.:f.}
\end{itemize}
Acto de \textunderscore carnificar-se\textunderscore .
\section{Carnificar-se}
\begin{itemize}
\item {Grp. gram.:v. p.}
\end{itemize}
\begin{itemize}
\item {Proveniência:(Do lat. \textunderscore caro\textunderscore  + \textunderscore facere\textunderscore )}
\end{itemize}
Alterar-se (o tecido muscular), tomando a apparência e a consistência de carne.
\section{Carnífice}
\begin{itemize}
\item {Grp. gram.:m.}
\end{itemize}
\begin{itemize}
\item {Grp. gram.:Adj.}
\end{itemize}
\begin{itemize}
\item {Proveniência:(Lat. \textunderscore carnifex\textunderscore )}
\end{itemize}
Verdugo, carrasco.
Sanguinário, cruel.
\section{Carnificina}
\begin{itemize}
\item {Grp. gram.:f.}
\end{itemize}
\begin{itemize}
\item {Proveniência:(Lat. \textunderscore carnificina\textunderscore )}
\end{itemize}
Mortandade; exterminio.
\section{Carniforme}
\begin{itemize}
\item {Grp. gram.:adj.}
\end{itemize}
\begin{itemize}
\item {Proveniência:(Do lat. \textunderscore caro\textunderscore  + \textunderscore forma\textunderscore )}
\end{itemize}
Que tem a apparência de carne.
\section{Carnigão}
\begin{itemize}
\item {Grp. gram.:m.}
\end{itemize}
O mesmo que \textunderscore carnicão\textunderscore .
\section{Carníncula}
\begin{itemize}
\item {Grp. gram.:f.}
\end{itemize}
Planta leguminosa do Brasil.
\section{Carniola}
\begin{itemize}
\item {Grp. gram.:f.}
\end{itemize}
\begin{itemize}
\item {Proveniência:(De \textunderscore Carniole\textunderscore , n. p.)}
\end{itemize}
Gênero de cogumelos.
\section{Carnismo}
\begin{itemize}
\item {Grp. gram.:m.}
\end{itemize}
\begin{itemize}
\item {Utilização:med.}
\end{itemize}
\begin{itemize}
\item {Utilização:Neol.}
\end{itemize}
\begin{itemize}
\item {Proveniência:(De \textunderscore carne\textunderscore )}
\end{itemize}
Abuso da alimentação cárnea.
\section{Carnista}
\begin{itemize}
\item {Grp. gram.:m.  e  f.}
\end{itemize}
Pessôa dada ao carnismo.
\section{Carnita}
\begin{itemize}
\item {Grp. gram.:f.}
\end{itemize}
\begin{itemize}
\item {Utilização:Pop.}
\end{itemize}
\begin{itemize}
\item {Proveniência:(De \textunderscore carne\textunderscore )}
\end{itemize}
Osso do pé do boi, que se usa em certo jôgo de rapazes.
\section{Carnivoridade}
\begin{itemize}
\item {Grp. gram.:f.}
\end{itemize}
Qualidade de carnivoro.
\section{Carnívoro}
\begin{itemize}
\item {Grp. gram.:adj.}
\end{itemize}
\begin{itemize}
\item {Grp. gram.:Pl.}
\end{itemize}
\begin{itemize}
\item {Proveniência:(Lat. \textunderscore carnivorus\textunderscore )}
\end{itemize}
Que se alimenta de carne: \textunderscore animaes carnivoros\textunderscore .
Ordem de mammíferos plantígrados e digitígrados.
\section{Carnosidade}
\begin{itemize}
\item {Grp. gram.:f.}
\end{itemize}
Qualidade do que é carnoso.
Excrescência carnosa.
\section{Carnoso}
\begin{itemize}
\item {Grp. gram.:adj.}
\end{itemize}
\begin{itemize}
\item {Proveniência:(Lat. \textunderscore carnosus\textunderscore )}
\end{itemize}
Cheio ou coberto de carne.
Que tem a apparência de carne.
Que tem mesocarpo suculento.
Que tem polpa espêssa; carnudo.
\section{Carnudo}
\begin{itemize}
\item {Grp. gram.:adj.}
\end{itemize}
\begin{itemize}
\item {Proveniência:(De \textunderscore carne\textunderscore )}
\end{itemize}
Que tem muito tecido muscular; carnoso.
Musculoso.
Gordo.
\section{Caro}
\begin{itemize}
\item {Grp. gram.:adj.}
\end{itemize}
\begin{itemize}
\item {Grp. gram.:Adv.}
\end{itemize}
\begin{itemize}
\item {Proveniência:(Lat. \textunderscore carus\textunderscore )}
\end{itemize}
Que se vende por alto preço.
Cujo preço excede o seu valor real.
Que custa sacrifícios ou grandes despesas: \textunderscore saiu-lhe cara a tentativa\textunderscore .
Querido; que se estima; que é tido em grande valor: \textunderscore meu caro amigo\textunderscore .
Por alto preço: \textunderscore vendeu caro as propriedades\textunderscore .
Com grande trabalho ou sacrifício.
\section{Caro}
\begin{itemize}
\item {Grp. gram.:m.}
\end{itemize}
\begin{itemize}
\item {Proveniência:(De \textunderscore Cária\textunderscore , n. p.)}
\end{itemize}
Gênero de plantas umbellíferas.
\section{Carfologia}
\begin{itemize}
\item {Grp. gram.:f.}
\end{itemize}
\begin{itemize}
\item {Proveniência:(Do gr. \textunderscore karphos\textunderscore  + \textunderscore legein\textunderscore )}
\end{itemize}
Agitação contínua e automática das mãos e dos dedos, que parecem procurar aprehender pequenos objectos.
\section{Carfológico}
\begin{itemize}
\item {Grp. gram.:adj.}
\end{itemize}
Relativo á \textunderscore carfologia\textunderscore .
\section{Caró}
\begin{itemize}
\item {Grp. gram.:m.}
\end{itemize}
Árvore indiana, (\textunderscore strychnos nux vomica\textunderscore ).
O mesmo que \textunderscore carinão\textunderscore .
\section{Caroá}
\begin{itemize}
\item {Grp. gram.:m.}
\end{itemize}
Planta bromeliácea do Brasil.
\section{Caroal}
\begin{itemize}
\item {Grp. gram.:adj.}
\end{itemize}
\begin{itemize}
\item {Utilização:Prov.}
\end{itemize}
\begin{itemize}
\item {Utilização:trasm.}
\end{itemize}
O mesmo que \textunderscore caroável\textunderscore .
\section{Caroatá}
\begin{itemize}
\item {Grp. gram.:m.}
\end{itemize}
Espécie de ananás silvestre.
\section{Caroável}
\begin{itemize}
\item {Grp. gram.:adj.}
\end{itemize}
\begin{itemize}
\item {Utilização:Des.}
\end{itemize}
\begin{itemize}
\item {Proveniência:(De \textunderscore caro\textunderscore ^1)}
\end{itemize}
Carinhoso; amável.
Propício, criador.
\section{Caroba}
\begin{itemize}
\item {fónica:carô}
\end{itemize}
\begin{itemize}
\item {Grp. gram.:f.}
\end{itemize}
Designação de duas plantas bignoniáceas do Brasil.
\section{Carobinha}
\begin{itemize}
\item {Grp. gram.:f.}
\end{itemize}
\begin{itemize}
\item {Utilização:Bras}
\end{itemize}
Planta bignoniácea medicinal, (\textunderscore jacaranda lancifoliata\textunderscore ).
\section{Caroca}
\begin{itemize}
\item {Grp. gram.:f.}
\end{itemize}
\begin{itemize}
\item {Utilização:Prov.}
\end{itemize}
\begin{itemize}
\item {Utilização:trasm.}
\end{itemize}
\begin{itemize}
\item {Utilização:T. de operários}
\end{itemize}
Pêta ou patranha, mais ou menos engenhosa.
Fantasia ou ideia falsa, que se nos apodera do espirito.
Trabalho de pouca monta, biscate, gancho.
\section{Caroça}
\begin{itemize}
\item {Grp. gram.:f.}
\end{itemize}
\begin{itemize}
\item {Utilização:Prov.}
\end{itemize}
A cabeça do linho, em que se contém a semente.
(Cp. \textunderscore caroço\textunderscore )
\section{Caroça}
\begin{itemize}
\item {Grp. gram.:f.}
\end{itemize}
\begin{itemize}
\item {Utilização:Prov.}
\end{itemize}
Capa de palha.
\section{Carocar}
\begin{itemize}
\item {Grp. gram.:v. i.}
\end{itemize}
\begin{itemize}
\item {Utilização:T. de operários}
\end{itemize}
Fazer caroca ou pequeno trabalho.
\section{Caroceiro}
\begin{itemize}
\item {Grp. gram.:m.}
\end{itemize}
\begin{itemize}
\item {Proveniência:(De \textunderscore caroço\textunderscore )}
\end{itemize}
Espécie de palmeira africana.
\section{Carócha}
\begin{itemize}
\item {Grp. gram.:f.}
\end{itemize}
\begin{itemize}
\item {Utilização:Escol.}
\end{itemize}
\begin{itemize}
\item {Utilização:Prov.}
\end{itemize}
\begin{itemize}
\item {Utilização:minh.}
\end{itemize}
\begin{itemize}
\item {Grp. gram.:Interj.}
\end{itemize}
\begin{itemize}
\item {Utilização:T. de Barcelos}
\end{itemize}
Insecto coleóptero, (\textunderscore lucanus servus\textunderscore , Lin.), cujo macho tem as antennas tão desenvolvidas, que lembram os cornos de um veado. Cp. fr. \textunderscore cerf velant\textunderscore .
Bruxa.
Mitra extravagante dos condemnados aos autos de fé.
Carapuça ridícula de papel, que por castigo se põe na cabeça das crianças que não sabem a lição ou que se portam mal.
Bandeira do milho.
Pequeno fogão, em que os funileiros aquecem os ferros.
Arreda! Fóra! Some-te!
(Cast. \textunderscore carocha\textunderscore )
\section{Carôcha}
\begin{itemize}
\item {Grp. gram.:f.}
\end{itemize}
Casta de uva preta algarvia.
\section{Carocho}
\begin{itemize}
\item {fónica:carô}
\end{itemize}
\begin{itemize}
\item {Grp. gram.:m.}
\end{itemize}
\begin{itemize}
\item {Utilização:Pop.}
\end{itemize}
\begin{itemize}
\item {Utilização:Prov.}
\end{itemize}
\begin{itemize}
\item {Utilização:minh.}
\end{itemize}
\begin{itemize}
\item {Utilização:Prov.}
\end{itemize}
\begin{itemize}
\item {Utilização:minh.}
\end{itemize}
\begin{itemize}
\item {Grp. gram.:Adj.}
\end{itemize}
Carócha pequena.
Peixe de Portugal.
O diabo.
Pequena mêda de centeio.
Boi, cujas pontas quási se juntam em cima.
Que tem côr de carócha; escuro, trigueiro.
(Cp. \textunderscore carócha\textunderscore )
\section{Caroço}
\begin{itemize}
\item {fónica:carô}
\end{itemize}
\begin{itemize}
\item {Grp. gram.:m.}
\end{itemize}
\begin{itemize}
\item {Utilização:Pop.}
\end{itemize}
\begin{itemize}
\item {Utilização:Prov.}
\end{itemize}
\begin{itemize}
\item {Utilização:minh.}
\end{itemize}
Parte dura de alguns frutos, que envolve a amêndoa ou semente dêlles.
Semente de vários frutos.
Cylindro, usado na fundição das peças de fogo, para lhes formar a alma.
Glândula enfartada, ingua.
Dinheiro.
Carolo (do milho).
(Cast. \textunderscore carozo\textunderscore )
\section{Caroço-de-alarce}
\begin{itemize}
\item {Grp. gram.:m.}
\end{itemize}
Espécie de doce em Beja.
\section{Carocoroca}
\begin{itemize}
\item {Grp. gram.:m.}
\end{itemize}
Pequeno peixe marítimo do Brasil.
\section{Caroçuda}
\begin{itemize}
\item {Grp. gram.:f.}
\end{itemize}
\begin{itemize}
\item {Utilização:Prov.}
\end{itemize}
\begin{itemize}
\item {Utilização:alent.}
\end{itemize}
\begin{itemize}
\item {Proveniência:(De \textunderscore caroçudo\textunderscore )}
\end{itemize}
Variedade de roman.
\section{Caroçudo}
\begin{itemize}
\item {Grp. gram.:adj.}
\end{itemize}
\begin{itemize}
\item {Utilização:Bras}
\end{itemize}
Que tem caroços; encaroçado: \textunderscore farinha caroçuda\textunderscore .
Que tem empolas: \textunderscore pelle caroçuda\textunderscore .
\section{Caroé}
\begin{itemize}
\item {Grp. gram.:m.}
\end{itemize}
Ave brasileira.
\section{Carola}
\begin{itemize}
\item {Grp. gram.:m.}
\end{itemize}
\begin{itemize}
\item {Grp. gram.:Adj.}
\end{itemize}
\begin{itemize}
\item {Grp. gram.:F.}
\end{itemize}
\begin{itemize}
\item {Utilização:Des.}
\end{itemize}
\begin{itemize}
\item {Utilização:Ant.}
\end{itemize}
\begin{itemize}
\item {Proveniência:(Do lat. \textunderscore corolla\textunderscore , dem. de \textunderscore corona\textunderscore )}
\end{itemize}
Indivíduo, que tem corôa aberta.
Promotor enthusiasta de festas de igreja.
Indivíduo apaixonado por qualquer ideia ou systema.
Fanático.
Apaixonado por uma ideia ou systema.
Cabeça.
Cara? rosto?«\textunderscore voltão a carola a esperanças.\textunderscore »\textunderscore Aulegrafia\textunderscore , 142.
\section{Carola}
\begin{itemize}
\item {Grp. gram.:f.}
\end{itemize}
\begin{itemize}
\item {Proveniência:(It. \textunderscore carola\textunderscore )}
\end{itemize}
Dança de roda.
\section{Carolice}
\begin{itemize}
\item {Grp. gram.:f.}
\end{itemize}
Qualidade do que é \textunderscore carola\textunderscore ^1.
\section{Carolim}
\begin{itemize}
\item {Grp. gram.:m.}
\end{itemize}
\begin{itemize}
\item {Proveniência:(De \textunderscore carolo\textunderscore )}
\end{itemize}
Receptáculo commum dos flósculos de uma espiga.
\section{Carolina}
\begin{itemize}
\item {Grp. gram.:f.}
\end{itemize}
Árvore leguminosa, originária da Índia.
\section{Carolínea}
\begin{itemize}
\item {Grp. gram.:f.}
\end{itemize}
Gênero de árvores esterculiáceas.
\section{Carolino}
\begin{itemize}
\item {Grp. gram.:adj.}
\end{itemize}
\begin{itemize}
\item {Proveniência:(Do b. lat. \textunderscore Carolus\textunderscore , n. p.)}
\end{itemize}
O mesmo que \textunderscore carlovíngio\textunderscore .
Diz-se de uma espécie de arroz exótico.
\section{Carolino}
\begin{itemize}
\item {Grp. gram.:adj.}
\end{itemize}
\begin{itemize}
\item {Grp. gram.:M.}
\end{itemize}
Relativo ás ilhas Carolinas.
Habitante das Carolinas.
\section{Carolismo}
\begin{itemize}
\item {Grp. gram.:m.}
\end{itemize}
Systema ou actos de carola; fanatismo.
\section{Carolo}
\begin{itemize}
\item {fónica:carô}
\end{itemize}
\begin{itemize}
\item {Grp. gram.:m.}
\end{itemize}
\begin{itemize}
\item {Utilização:Prov.}
\end{itemize}
\begin{itemize}
\item {Utilização:Prov.}
\end{itemize}
\begin{itemize}
\item {Utilização:minh.}
\end{itemize}
\begin{itemize}
\item {Grp. gram.:Pl.}
\end{itemize}
Maçaroca do milho, depois de debulhada.
Pancada na cabeça com vara ou cana.
Fatia de pão; pão de trigo, feito de farinha mais grossa.
Caroço.
Papas de farinha grossa de milho.
Farinha de milho, moída grosseiramente.
(Cp. \textunderscore carola\textunderscore ^1, cabeça)
\section{Carona}
\begin{itemize}
\item {Grp. gram.:f.}
\end{itemize}
\begin{itemize}
\item {Utilização:Prov.}
\end{itemize}
\begin{itemize}
\item {Utilização:alent.}
\end{itemize}
\begin{itemize}
\item {Utilização:Bras}
\end{itemize}
Cabeça do pião.
Peça dos arreios, que se colloca por baixo do lombilho, e que tem as abas mais largas que as dêste; o mesmo que \textunderscore baiana\textunderscore .
\textunderscore Levar ou tomar carona\textunderscore , diz-se do official militar, preterido por outro na promoção.
(Cast. \textunderscore carona\textunderscore )
\section{Caronada}
\begin{itemize}
\item {Grp. gram.:f.}
\end{itemize}
\begin{itemize}
\item {Proveniência:(Fr. \textunderscore caronade\textunderscore )}
\end{itemize}
Peça curta, mas de grande calibre, usada na artilharia marítima.
\section{Caronha}
\begin{itemize}
\item {Grp. gram.:f.}
\end{itemize}
\begin{itemize}
\item {Utilização:Prov.}
\end{itemize}
\begin{itemize}
\item {Utilização:trasm.}
\end{itemize}
O mesmo que \textunderscore caroço\textunderscore .
\section{Carónica}
\begin{itemize}
\item {Grp. gram.:f.}
\end{itemize}
\begin{itemize}
\item {Utilização:Ant.}
\end{itemize}
O mesmo que \textunderscore chrónica\textunderscore .
\section{Carôpa}
\begin{itemize}
\item {Grp. gram.:f.}
\end{itemize}
\begin{itemize}
\item {Utilização:Prov.}
\end{itemize}
A bandeira do milho.
\section{Carópa}
\begin{itemize}
\item {Grp. gram.:f.}
\end{itemize}
\begin{itemize}
\item {Utilização:Prov.}
\end{itemize}
A bandeira do milho.
\section{Carópa}
\begin{itemize}
\item {Grp. gram.:f.}
\end{itemize}
\begin{itemize}
\item {Utilização:Prov.}
\end{itemize}
\begin{itemize}
\item {Utilização:minh.}
\end{itemize}
Chuva miúda.
\section{Caropar}
\begin{itemize}
\item {Grp. gram.:v. i.}
\end{itemize}
\begin{itemize}
\item {Utilização:Prov.}
\end{itemize}
\begin{itemize}
\item {Utilização:minh.}
\end{itemize}
Cair carópa^2.
\section{Caroquinha}
\begin{itemize}
\item {fónica:caró}
\end{itemize}
\begin{itemize}
\item {Grp. gram.:f.}
\end{itemize}
\begin{itemize}
\item {Utilização:Prov.}
\end{itemize}
\begin{itemize}
\item {Utilização:trasm.}
\end{itemize}
O mesmo que \textunderscore caroca\textunderscore .
\section{Carosol}
\begin{itemize}
\item {Grp. gram.:m.}
\end{itemize}
Arbusto anonáceo de Timor.
\section{Carótico}
\begin{itemize}
\item {Grp. gram.:adj.}
\end{itemize}
\begin{itemize}
\item {Proveniência:(Do gr. \textunderscore karos\textunderscore )}
\end{itemize}
Relativo ao cárus.
\section{Carótida}
\begin{itemize}
\item {Grp. gram.:f.}
\end{itemize}
\begin{itemize}
\item {Proveniência:(Gr. \textunderscore karotides\textunderscore )}
\end{itemize}
Cada uma das duas grandes artérias, que da aorta levam o sangue á cabeça.
\section{Carotídeo}
\begin{itemize}
\item {Grp. gram.:adj.}
\end{itemize}
Relativo ás carótidas.
\section{Carotidiano}
\begin{itemize}
\item {Grp. gram.:adj.}
\end{itemize}
Relativo ás carótidas.
\section{Caroupo}
\begin{itemize}
\item {Grp. gram.:adj.}
\end{itemize}
\begin{itemize}
\item {Utilização:Prov.}
\end{itemize}
\begin{itemize}
\item {Utilização:dur.}
\end{itemize}
Diz-se do boi, que tem as pontas viradas para cima e para os lados. Cf. Araújo e Mello, \textunderscore Ling. Pop.\textunderscore 
\section{Carpa}
\begin{itemize}
\item {Grp. gram.:f.}
\end{itemize}
Corpulenta árvore amentácea, (\textunderscore carpinus betulus\textunderscore ).
Peixe cyprinoide, (\textunderscore cyprinus carpio\textunderscore ).
\section{Carpa}
\begin{itemize}
\item {Grp. gram.:f.}
\end{itemize}
Acto de \textunderscore carpir\textunderscore  a cana do açúcar.
\section{Carpal}
\begin{itemize}
\item {Grp. gram.:adj.}
\end{itemize}
\begin{itemize}
\item {Utilização:Anat.}
\end{itemize}
Relativo ao carpo.
\section{Carpanel}
\begin{itemize}
\item {Grp. gram.:m.}
\end{itemize}
\begin{itemize}
\item {Utilização:Ant.}
\end{itemize}
\begin{itemize}
\item {Proveniência:(T. cast.)}
\end{itemize}
Parede ou tecto almofadado ou encaixilhado.
\section{Carpanta}
\begin{itemize}
\item {Grp. gram.:f.}
\end{itemize}
\begin{itemize}
\item {Utilização:Prov.}
\end{itemize}
\begin{itemize}
\item {Utilização:trasm.}
\end{itemize}
\begin{itemize}
\item {Proveniência:(T. cast.)}
\end{itemize}
O mesmo que \textunderscore carapanta\textunderscore .
\section{Carpar}
\begin{itemize}
\item {Grp. gram.:v. t.}
\end{itemize}
\begin{itemize}
\item {Utilização:Bras}
\end{itemize}
Capinar, limpar.
\section{Carpear}
\textunderscore v. t.\textunderscore  (e der.)
O mesmo que \textunderscore carpiar\textunderscore , etc.
\section{Carpela}
\begin{itemize}
\item {Grp. gram.:f.}
\end{itemize}
\begin{itemize}
\item {Utilização:Bot.}
\end{itemize}
\begin{itemize}
\item {Proveniência:(Do fr. \textunderscore carpelle\textunderscore )}
\end{itemize}
Fôlha dobrada, que constitue o elemento essencial do ovário das plantas.
Cada uma das divisões foliáceas, que concorrem para a formação do fruto.
\section{Carpelar}
\begin{itemize}
\item {Grp. gram.:adj.}
\end{itemize}
Relativo á carpela.
\section{Carpelo}
\begin{itemize}
\item {Grp. gram.:m.}
\end{itemize}
O mesmo que \textunderscore pistillo\textunderscore .
O mesmo que \textunderscore folhelho\textunderscore  ou \textunderscore capella\textunderscore  da espiga do milho.
(Cp. \textunderscore carpela\textunderscore )
\section{Carpentária}
\begin{itemize}
\item {Grp. gram.:f.}
\end{itemize}
\begin{itemize}
\item {Utilização:Ant.}
\end{itemize}
\begin{itemize}
\item {Proveniência:(Do lat. \textunderscore carpentum\textunderscore )}
\end{itemize}
Certo número de carradas de lenha ou de madeira, que um casal ou prazo tinha de pagar ao senhorio.
\section{Carpento}
\begin{itemize}
\item {Grp. gram.:m.}
\end{itemize}
\begin{itemize}
\item {Proveniência:(Lat. \textunderscore carpentum\textunderscore )}
\end{itemize}
Coche antigo; carrão.
\section{Carpetanos}
\begin{itemize}
\item {Grp. gram.:m. pl.}
\end{itemize}
Antigo povo da Espanha.
\section{Carphologia}
\begin{itemize}
\item {Grp. gram.:f.}
\end{itemize}
\begin{itemize}
\item {Proveniência:(Do gr. \textunderscore karphos\textunderscore  + \textunderscore legein\textunderscore )}
\end{itemize}
Agitação contínua e automática das mãos e dos dedos, que parecem procurar aprehender pequenos objectos.
\section{Carphológico}
\begin{itemize}
\item {Grp. gram.:adj.}
\end{itemize}
Relativo á \textunderscore carphologia\textunderscore .
\section{Carpiadeira}
\begin{itemize}
\item {Grp. gram.:f.}
\end{itemize}
\begin{itemize}
\item {Proveniência:(De \textunderscore carpiar\textunderscore )}
\end{itemize}
O mesmo que \textunderscore carmiadeira\textunderscore . Cf. Castilho, \textunderscore Fastos\textunderscore , II, 346.
\section{Carpiar}
\textunderscore v. t.\textunderscore  (e der.)
(E fórma pop. de \textunderscore carmiar\textunderscore , etc.)
(Indica-se como étym. o rad. do lat. \textunderscore carpere\textunderscore : mas inclino-me a que é corr. de \textunderscore carmiar\textunderscore )
\section{Carpideira}
\begin{itemize}
\item {Grp. gram.:f.}
\end{itemize}
\begin{itemize}
\item {Proveniência:(De \textunderscore carpir\textunderscore )}
\end{itemize}
Mulher, a quem se pagava, para ir com outras prantear os mortos.
Mulher, que anda sempre a lastimar-se.
\section{Carpido}
\begin{itemize}
\item {Grp. gram.:m.}
\end{itemize}
\begin{itemize}
\item {Grp. gram.:Adj.}
\end{itemize}
\begin{itemize}
\item {Proveniência:(De \textunderscore carpir\textunderscore )}
\end{itemize}
Acto de carpir.
O mesmo que \textunderscore borrelho\textunderscore .
Lamentoso.
\section{Carpidor}
\begin{itemize}
\item {Grp. gram.:m.  e  adj.}
\end{itemize}
\begin{itemize}
\item {Proveniência:(De \textunderscore carpir\textunderscore )}
\end{itemize}
O que carpe ou que se carpe.
\section{Carpidura}
\begin{itemize}
\item {Grp. gram.:f.}
\end{itemize}
Acto de carpir; o mesmo que \textunderscore carpimento\textunderscore :«\textunderscore com muitas lágrimas e carpiduras\textunderscore ». Filinto, \textunderscore D. Man.\textunderscore , I, 155.
\section{Carpimento}
\begin{itemize}
\item {Grp. gram.:m.}
\end{itemize}
Acção de carpir.
\section{Carpina}
\begin{itemize}
\item {Grp. gram.:m.}
\end{itemize}
O mesmo que \textunderscore carapina\textunderscore .
\section{Carpino}
\begin{itemize}
\item {Grp. gram.:m.}
\end{itemize}
\begin{itemize}
\item {Proveniência:(Lat. \textunderscore carpinus\textunderscore )}
\end{itemize}
Gênero de árvores cupulíferas.
\section{Carpins}
\begin{itemize}
\item {Grp. gram.:m. pl.}
\end{itemize}
\begin{itemize}
\item {Utilização:Prov.}
\end{itemize}
\begin{itemize}
\item {Utilização:Prov.}
\end{itemize}
\begin{itemize}
\item {Utilização:dur.}
\end{itemize}
Peúgas.
Sapatos de liga.
\section{Carpinseira}
\begin{itemize}
\item {Grp. gram.:f.}
\end{itemize}
\begin{itemize}
\item {Utilização:Prov.}
\end{itemize}
\begin{itemize}
\item {Utilização:dur.}
\end{itemize}
Mulher que faz carpins, ou que faz o entrançado para sapatos de liga ou ourelo.
\section{Carpintaria}
\begin{itemize}
\item {Grp. gram.:f.}
\end{itemize}
Trabalho ou offício de carpinteiro.
\section{Carpinteirar}
\begin{itemize}
\item {Grp. gram.:v. t.  e  i.}
\end{itemize}
O mesmo que \textunderscore carpintejar\textunderscore .
\section{Carpinteiro}
\begin{itemize}
\item {Grp. gram.:m.}
\end{itemize}
\begin{itemize}
\item {Utilização:T. de theatro}
\end{itemize}
\begin{itemize}
\item {Proveniência:(Do lat. \textunderscore carpentarius\textunderscore )}
\end{itemize}
Artífice, que trabalha em construcções de madeira, ou que lavra e apparelha a madeira para qualquer obra.
Primitivamente, fabricante de carros.
O mesmo que \textunderscore carcoma\textunderscore .
Aquelle que arma o scenário no palco.
\section{Carpintejar}
\begin{itemize}
\item {Grp. gram.:v. i.}
\end{itemize}
\begin{itemize}
\item {Grp. gram.:V. t.}
\end{itemize}
\begin{itemize}
\item {Proveniência:(Do lat. \textunderscore carpentum\textunderscore )}
\end{itemize}
Exercer o offício de carpinteiro.
Aparelhar (madeira) para obras:«\textunderscore já estavam carpintejando as peças do cadafalso\textunderscore ». Camillo, \textunderscore Regicida\textunderscore , 101.
\section{Carpintina}
\begin{itemize}
\item {Grp. gram.:f.}
\end{itemize}
\begin{itemize}
\item {Utilização:Prov.}
\end{itemize}
\begin{itemize}
\item {Utilização:alg.}
\end{itemize}
\begin{itemize}
\item {Proveniência:(De \textunderscore carpir\textunderscore )}
\end{itemize}
Choradeira.
Espalhafato.
\section{Carpir}
\begin{itemize}
\item {Grp. gram.:v. t.}
\end{itemize}
\begin{itemize}
\item {Grp. gram.:V. i.}
\end{itemize}
\begin{itemize}
\item {Utilização:Bras}
\end{itemize}
\begin{itemize}
\item {Proveniência:(Lat. \textunderscore carpere\textunderscore )}
\end{itemize}
Apanhar, colhêr.
Mondar.
Arrancar (o cabello) em sinal de dor.
Prantear.
Chorar, lamentando-se.
Tratar de (uma roça), desmoitando-a.
\section{Carpo}
\begin{itemize}
\item {Grp. gram.:m.}
\end{itemize}
\begin{itemize}
\item {Proveniência:(Gr. \textunderscore karpos\textunderscore )}
\end{itemize}
Pulso.
Punho.
Qualquer fruto.
\section{Carpobálsamo}
\begin{itemize}
\item {Grp. gram.:m.}
\end{itemize}
\begin{itemize}
\item {Proveniência:(Do gr. \textunderscore karpos\textunderscore  + \textunderscore balsamon\textunderscore )}
\end{itemize}
Fruto do balsameiro de Meca.
\section{Carpóbolo}
\begin{itemize}
\item {Grp. gram.:m.}
\end{itemize}
\begin{itemize}
\item {Proveniência:(Do gr. \textunderscore karpos\textunderscore  + \textunderscore bolos\textunderscore )}
\end{itemize}
Gênero de cogumelos.
\section{Carpócero}
\begin{itemize}
\item {Grp. gram.:m.}
\end{itemize}
\begin{itemize}
\item {Proveniência:(Do gr. \textunderscore karpos\textunderscore  + \textunderscore keras\textunderscore )}
\end{itemize}
Planta herbácea do Cabo da Bôa-Esperança.
\section{Carpofagia}
\begin{itemize}
\item {Grp. gram.:f.}
\end{itemize}
Qualidade de carpófago.
\section{Carpófago}
\begin{itemize}
\item {Grp. gram.:adj.}
\end{itemize}
\begin{itemize}
\item {Proveniência:(Do gr. \textunderscore karpos\textunderscore  + \textunderscore phagein\textunderscore )}
\end{itemize}
Que se alimenta de frutos.
\section{Carpofilo}
\begin{itemize}
\item {Grp. gram.:m.}
\end{itemize}
\begin{itemize}
\item {Proveniência:(Do gr. \textunderscore karpos\textunderscore  + \textunderscore phullon\textunderscore )}
\end{itemize}
Fôlha, que tem fórma de fruto.
\section{Carpólitho}
\begin{itemize}
\item {Grp. gram.:m.}
\end{itemize}
\begin{itemize}
\item {Proveniência:(Do gr. \textunderscore karpos\textunderscore  + \textunderscore lithos\textunderscore )}
\end{itemize}
Concreção dura na polpa de certos frutos.
Fruto fóssil ou petrificado.
\section{Carpólito}
\begin{itemize}
\item {Grp. gram.:m.}
\end{itemize}
\begin{itemize}
\item {Proveniência:(Do gr. \textunderscore karpos\textunderscore  + \textunderscore lithos\textunderscore )}
\end{itemize}
Concreção dura na polpa de certos frutos.
Fruto fóssil ou petrificado.
\section{Carpologia}
\begin{itemize}
\item {Grp. gram.:f.}
\end{itemize}
\begin{itemize}
\item {Proveniência:(Do gr. \textunderscore karpos\textunderscore  + \textunderscore logos\textunderscore )}
\end{itemize}
Tratado dos frutos.
\section{Carpológico}
\begin{itemize}
\item {Grp. gram.:adj.}
\end{itemize}
Relativo á carpologia.
\section{Carpomania}
\begin{itemize}
\item {Grp. gram.:f.}
\end{itemize}
Moléstia vegetal, caracterizada por superabundância de nutrição nos órgãos da reproducção, com falta muito sensível nas fôlhas.
\section{Carpomorfo}
\begin{itemize}
\item {Grp. gram.:adj.}
\end{itemize}
\begin{itemize}
\item {Proveniência:(Do gr. \textunderscore karpos\textunderscore  + \textunderscore morphè\textunderscore )}
\end{itemize}
Que tem fórma de fruto.
\section{Carpomorpho}
\begin{itemize}
\item {Grp. gram.:adj.}
\end{itemize}
\begin{itemize}
\item {Proveniência:(Do gr. \textunderscore karpos\textunderscore  + \textunderscore morphè\textunderscore )}
\end{itemize}
Que tem fórma de fruto.
\section{Carpophagia}
\begin{itemize}
\item {Grp. gram.:f.}
\end{itemize}
Qualidade de carpóphago.
\section{Carpóphago}
\begin{itemize}
\item {Grp. gram.:adj.}
\end{itemize}
\begin{itemize}
\item {Proveniência:(Do gr. \textunderscore karpos\textunderscore  + \textunderscore phagein\textunderscore )}
\end{itemize}
Que se alimenta de frutos.
\section{Carpophyllo}
\begin{itemize}
\item {Grp. gram.:m.}
\end{itemize}
\begin{itemize}
\item {Proveniência:(Do gr. \textunderscore karpos\textunderscore  + \textunderscore phullon\textunderscore )}
\end{itemize}
Fôlha, que tem fórma de fruto.
\section{Carptor}
\begin{itemize}
\item {Grp. gram.:m.}
\end{itemize}
\begin{itemize}
\item {Proveniência:(De \textunderscore carptor\textunderscore )}
\end{itemize}
Escravo, incumbido de trinchar a carne, nos antigos banquetes romanos.
\section{Carque}
\begin{itemize}
\item {Grp. gram.:m.}
\end{itemize}
\begin{itemize}
\item {Utilização:Des.}
\end{itemize}
O mesmo que \textunderscore carqueja\textunderscore .
\section{Carqueja}
\begin{itemize}
\item {Grp. gram.:f.}
\end{itemize}
Planta silvestre, que serve de acendalha.
(Cp. o cast. \textunderscore carquexia\textunderscore )
\section{Carquejal}
\begin{itemize}
\item {Grp. gram.:m.}
\end{itemize}
Terreno, onde cresce carqueja.
\section{Carquejeira}
\begin{itemize}
\item {Grp. gram.:f.}
\end{itemize}
O mesmo que \textunderscore carquejal\textunderscore .
Mulher, que apanha ou vende carqueja.
Mulher rústica.
\section{Carquejeiro}
\begin{itemize}
\item {Grp. gram.:m.}
\end{itemize}
Aquelle que colhe carqueja e a vende pelas portas.
\section{Carquilha}
\begin{itemize}
\item {Grp. gram.:f.}
\end{itemize}
Ruga; dobra; prega.
(Relaciona-se com \textunderscore cárcava\textunderscore ?)
\section{Carraboiçal}
\begin{itemize}
\item {Grp. gram.:m.}
\end{itemize}
\begin{itemize}
\item {Utilização:Prov.}
\end{itemize}
\begin{itemize}
\item {Utilização:trasm.}
\end{itemize}
Barrocal.
Ladeira penhascosa, especialmente a que é coberta de silvas ou outras plantas bravas.
\section{Carraca}
\begin{itemize}
\item {Grp. gram.:f.}
\end{itemize}
\begin{itemize}
\item {Proveniência:(Do neerl. \textunderscore kraecke\textunderscore , segundo Körting)}
\end{itemize}
Grande embarcação antiga.
\section{Carraça}
\begin{itemize}
\item {Grp. gram.:f.}
\end{itemize}
\begin{itemize}
\item {Utilização:Fig.}
\end{itemize}
Pequeno crustáceo, que se prende á pelle de certos animaes.
Pessôa impertinente que não larga outrem.
(Por \textunderscore agarraça\textunderscore , de \textunderscore agarrar\textunderscore ?)
\section{Carracão}
\begin{itemize}
\item {Grp. gram.:m.}
\end{itemize}
Grande carraca.
\section{Carraçaria}
\begin{itemize}
\item {Grp. gram.:f.}
\end{itemize}
Grande porção de carraças.
\section{Carraceno}
\begin{itemize}
\item {Grp. gram.:adj.}
\end{itemize}
\begin{itemize}
\item {Utilização:Prov.}
\end{itemize}
Pequeno; anão; miúdo: \textunderscore milho carraceno\textunderscore ; \textunderscore ervilhas carracenas\textunderscore . (Colhido em Turquel)
\section{Carrachola}
\begin{itemize}
\item {Grp. gram.:f.}
\end{itemize}
\begin{itemize}
\item {Utilização:Prov.}
\end{itemize}
Pequena carrada; carrada mean.
\section{Carraço}
\begin{itemize}
\item {Grp. gram.:m.}
\end{itemize}
O mesmo que \textunderscore carraça\textunderscore .
Tojeiro.
\section{Carrada}
\begin{itemize}
\item {Grp. gram.:f.}
\end{itemize}
\begin{itemize}
\item {Utilização:Fig.}
\end{itemize}
\begin{itemize}
\item {Proveniência:(De \textunderscore carro\textunderscore )}
\end{itemize}
Objectos que um carro transporta de uma vez.
Quantidade, que póde carregar um carro.
Grande porção de qualquer coisa.
\section{Carrafagem}
\begin{itemize}
\item {Grp. gram.:f.}
\end{itemize}
\begin{itemize}
\item {Utilização:Ant.}
\end{itemize}
Ágio ou câmbio de moédas, na Índia portuguesa.
\section{Carrajó}
\begin{itemize}
\item {Grp. gram.:m.}
\end{itemize}
O mesmo que \textunderscore calraxo\textunderscore .
\section{Carral}
\begin{itemize}
\item {Grp. gram.:adj.}
\end{itemize}
Relativo a carros.
\section{Carramelo}
\begin{itemize}
\item {Grp. gram.:m.}
\end{itemize}
\begin{itemize}
\item {Utilização:Prov.}
\end{itemize}
\begin{itemize}
\item {Utilização:trasm.}
\end{itemize}
Montão de qualquer coisa.
\section{Carramoiço}
\begin{itemize}
\item {Grp. gram.:m.}
\end{itemize}
(V.cramoiço)
\section{Carranca}
\begin{itemize}
\item {Grp. gram.:f.}
\end{itemize}
\begin{itemize}
\item {Utilização:Ant.}
\end{itemize}
Rosto sombrio, carregado.
Aspecto, que indica mau humor.
Cara feia.
Maus modos.
Cara disforme, de pedra, madeira ou metal, com que se adornam differentes construcções.
Máscara.
O mesmo que \textunderscore açamo\textunderscore .
\section{Carrancada}
\begin{itemize}
\item {Grp. gram.:f.}
\end{itemize}
Série de carrancas.
\section{Carranchadas}
\begin{itemize}
\item {Grp. gram.:f. pl.}
\end{itemize}
\begin{itemize}
\item {Utilização:Prov.}
\end{itemize}
\begin{itemize}
\item {Utilização:trasm.}
\end{itemize}
O mesmo que \textunderscore carranchinhas\textunderscore .
\section{Carranchinhas}
\begin{itemize}
\item {Grp. gram.:f. pl.}
\end{itemize}
\begin{itemize}
\item {Utilização:Prov.}
\end{itemize}
\begin{itemize}
\item {Utilização:trasm.}
\end{itemize}
O mesmo que \textunderscore carrancholas\textunderscore .
\section{Carrancholas}
\begin{itemize}
\item {Grp. gram.:f. pl.}
\end{itemize}
\begin{itemize}
\item {Utilização:Prov.}
\end{itemize}
\begin{itemize}
\item {Utilização:trasm.}
\end{itemize}
O mesmo que \textunderscore cavallitas\textunderscore .
\textunderscore Levar ás carrancholas\textunderscore , levar ás cavallitas, levar ás costas.
(Cp. \textunderscore escarranchar\textunderscore )
\section{Carrancudo}
\begin{itemize}
\item {Grp. gram.:adj.}
\end{itemize}
Que faz carranca.
Que tem mau humor.
\section{Carranha}
\begin{itemize}
\item {Grp. gram.:f.}
\end{itemize}
\begin{itemize}
\item {Utilização:Prov.}
\end{itemize}
\begin{itemize}
\item {Utilização:trasm.}
\end{itemize}
\begin{itemize}
\item {Utilização:Prov.}
\end{itemize}
Muco empastado no nariz.
Remela.
Certa carne callosa, que os porcos têm no ceu da bôca.
(Relaciona-se com \textunderscore ranho\textunderscore ?)
\section{Carranhoso}
\begin{itemize}
\item {Grp. gram.:adj.}
\end{itemize}
Que tem carranha.
\section{Carranquear}
\begin{itemize}
\item {Grp. gram.:v. i.}
\end{itemize}
\begin{itemize}
\item {Grp. gram.:V. t.}
\end{itemize}
Fazer carranca, estar carrancudo. Cf. Th. Ribeiro, \textunderscore Jornadas\textunderscore , 192.
Tornar carrancudo:«\textunderscore atalhou o padre, carranqueando um daquelles gestos...\textunderscore »Camillo, \textunderscore Caveira\textunderscore , 169.
\section{Carranquinha}
\begin{itemize}
\item {Grp. gram.:f.}
\end{itemize}
\begin{itemize}
\item {Utilização:Prov.}
\end{itemize}
\begin{itemize}
\item {Utilização:minh.}
\end{itemize}
\begin{itemize}
\item {Proveniência:(De \textunderscore carranca\textunderscore )}
\end{itemize}
O mesmo que \textunderscore amúo\textunderscore . (Colhido em Barcelos)
\section{Carrão}
\begin{itemize}
\item {Grp. gram.:m.}
\end{itemize}
Carro grande.
Instrumento de pescadores, para puxar o barco para terra.
\section{Carrapata}
\begin{itemize}
\item {Grp. gram.:f.}
\end{itemize}
\begin{itemize}
\item {Utilização:Prov.}
\end{itemize}
\begin{itemize}
\item {Proveniência:(De \textunderscore carrapato\textunderscore )}
\end{itemize}
Ferida, de cura demorada.
O mesmo que \textunderscore carraça\textunderscore .
Difficuldade, embaraço.
\section{Carrapateiro}
\begin{itemize}
\item {Grp. gram.:m.}
\end{itemize}
\begin{itemize}
\item {Grp. gram.:m.}
\end{itemize}
\begin{itemize}
\item {Utilização:Prov.}
\end{itemize}
O mesmo que \textunderscore rícino\textunderscore .
Espécie de pereira brava, o mesmo que \textunderscore carapeteiro\textunderscore .
\section{Carrapato}
\begin{itemize}
\item {Grp. gram.:m.}
\end{itemize}
O mesmo que \textunderscore carraça\textunderscore .
O mesmo que \textunderscore carrapateiro\textunderscore .
Espécie de feijão.
Homem atarracado.
(Cast. \textunderscore garrapato\textunderscore )
\section{Carrapeta}
\begin{itemize}
\item {fónica:pê}
\end{itemize}
\begin{itemize}
\item {Grp. gram.:f.}
\end{itemize}
O mesmo que \textunderscore carapéta\textunderscore .
Peça circular, ao lado das máquinas de escrever.
\section{Carrapiceiro}
\begin{itemize}
\item {Grp. gram.:m.}
\end{itemize}
\begin{itemize}
\item {Utilização:Prov.}
\end{itemize}
\begin{itemize}
\item {Utilização:beir.}
\end{itemize}
Tronco de pinheiro, com galhos, collocado na cozinha, para nos mesmos galhos, se pendurarem panelas, caçoilas, etc.; loiceiro.
(Cp. \textunderscore carrapito\textunderscore , chifre)
\section{Carrapichinho}
\begin{itemize}
\item {Grp. gram.:m.}
\end{itemize}
\begin{itemize}
\item {Proveniência:(De \textunderscore carrapicho\textunderscore )}
\end{itemize}
Planta malvácea do Brasil.
\section{Carrapicho}
\begin{itemize}
\item {Grp. gram.:m.}
\end{itemize}
\begin{itemize}
\item {Utilização:Fam.}
\end{itemize}
\begin{itemize}
\item {Utilização:Prov.}
\end{itemize}
\begin{itemize}
\item {Utilização:beir.}
\end{itemize}
Cabello atado no alto ou na parte posterior da cabeça.
Nome de várias plantas do Brasil.
Capricho.
Rebentão ou vergôntea muito nova de carvalho.
\section{Carrapicho-de-beiço-de-boi}
\begin{itemize}
\item {Grp. gram.:m.}
\end{itemize}
\begin{itemize}
\item {Utilização:Bras}
\end{itemize}
O mesmo que \textunderscore desmódio\textunderscore .
\section{Carrapiço}
\begin{itemize}
\item {Grp. gram.:m.}
\end{itemize}
\begin{itemize}
\item {Utilização:Prov.}
\end{itemize}
Espécie de pequenino ouriço, que encerra as sementes de certas ervas e que se agarra facilmente ao fato da gente e á lan do gado lanígero.
\section{Carrapito}
\begin{itemize}
\item {Grp. gram.:m.}
\end{itemize}
\begin{itemize}
\item {Utilização:Ant.}
\end{itemize}
\begin{itemize}
\item {Utilização:Prov.}
\end{itemize}
\begin{itemize}
\item {Utilização:minh.}
\end{itemize}
\begin{itemize}
\item {Utilização:T. do Ribatejo}
\end{itemize}
\begin{itemize}
\item {Utilização:Prov.}
\end{itemize}
\begin{itemize}
\item {Utilização:trasm.}
\end{itemize}
\begin{itemize}
\item {Utilização:Prov.}
\end{itemize}
Porção de cabello atado sôbre a cabeça; caracol de cabello; carrapicho.
Chifre.
Corna.
Roseira brava.
Nome de uma ave.
Cocuruto.
\section{Carrar}
\begin{itemize}
\item {Grp. gram.:v. t.}
\end{itemize}
\begin{itemize}
\item {Utilização:Prov.}
\end{itemize}
\begin{itemize}
\item {Utilização:trasm.}
\end{itemize}
O mesmo que \textunderscore acarretar\textunderscore .
\section{Çarrar}
\textunderscore v. t.\textunderscore  (e der.)
(Fórma antiga e popular de \textunderscore serrar\textunderscore :«\textunderscore estas coisas se çarram com esta chave\textunderscore ». \textunderscore Comment. de Af. Albuquerque\textunderscore , p. II, c. 21)
\section{Carrara}
\begin{itemize}
\item {Grp. gram.:m.}
\end{itemize}
Mármore de Carrara, na Itália.
\section{Carrasca}
\begin{itemize}
\item {Grp. gram.:f.}
\end{itemize}
Espécie de oliveira.
\section{Carrascal}
\begin{itemize}
\item {Grp. gram.:m.}
\end{itemize}
Moita de carrascos.
\section{Carrascão}
\begin{itemize}
\item {Grp. gram.:m.  e  adj.}
\end{itemize}
\begin{itemize}
\item {Proveniência:(De \textunderscore carrasco\textunderscore )}
\end{itemize}
Vinho forte e taninoso.
Vinho adulterado ou aguardentado, para se tornar forte.
\section{Carrascaria}
\begin{itemize}
\item {Grp. gram.:f.}
\end{itemize}
O mesmo que \textunderscore carrascal\textunderscore .
\section{Carrasco}
\begin{itemize}
\item {Grp. gram.:m.}
\end{itemize}
\begin{itemize}
\item {Utilização:Prov.}
\end{itemize}
\begin{itemize}
\item {Utilização:beir.}
\end{itemize}
\begin{itemize}
\item {Utilização:Bras. de Minas}
\end{itemize}
\begin{itemize}
\item {Utilização:Bras. de Goiás}
\end{itemize}
Arbusto silvestre, espécie de carvalho.
Abrunheiro bravo.
Executor da pena de morte.
Algoz.
Homem cruel.
Aquelle que afflige alguém.
Azeitona ou oliveira, o mesmo que \textunderscore carrasca\textunderscore .
Caminho pedregoso.
Charneca, gândara inculta.
\section{Carrascoso}
\begin{itemize}
\item {Grp. gram.:adj.}
\end{itemize}
O mesmo que \textunderscore carrascão\textunderscore .
E diz-se do terreno, em que crescem carrascos.
\section{Carraspana}
\begin{itemize}
\item {Grp. gram.:f.}
\end{itemize}
\begin{itemize}
\item {Utilização:Pop.}
\end{itemize}
Bebedeira.
\section{Carraspeira}
\begin{itemize}
\item {Grp. gram.:f.}
\end{itemize}
\begin{itemize}
\item {Utilização:Prov.}
\end{itemize}
\begin{itemize}
\item {Utilização:trasm.}
\end{itemize}
Aspereza, que se sente na garganta, especialmente determinada por constipação.
\section{Carraspuda}
\begin{itemize}
\item {Grp. gram.:adj. f.}
\end{itemize}
\begin{itemize}
\item {Utilização:Prov.}
\end{itemize}
\begin{itemize}
\item {Utilização:trasm.}
\end{itemize}
Diz-se da língua, quando está muito suja ou áspera.
\section{Carrasqueiral}
\begin{itemize}
\item {Grp. gram.:m.}
\end{itemize}
(V.carrascal)
\section{Carrasqueiro}
\begin{itemize}
\item {Grp. gram.:m.}
\end{itemize}
O mesmo que \textunderscore carrasco\textunderscore , arbusto.
Pau dêsse arbusto.
\section{Carrasquenha}
\begin{itemize}
\item {Grp. gram.:f.}
\end{itemize}
Azeitona, o mesmo que \textunderscore carrasca\textunderscore .
Diz-se de uma variedade, muito productiva e estimada, de oliveira. Cf. \textunderscore Port. au Point de Vue Agr.\textunderscore , 455.
\section{Carrasquenho}
\begin{itemize}
\item {Grp. gram.:adj.}
\end{itemize}
\begin{itemize}
\item {Proveniência:(De \textunderscore carrasco\textunderscore )}
\end{itemize}
Em que crescem os carrasqueiros e outros arbustos baixos e silvestres.
\section{Carrasquenho-branco}
\begin{itemize}
\item {Grp. gram.:m.}
\end{itemize}
Casta de uva de Tôrres-Vedras.
\section{Carrasquenho-tinto}
\begin{itemize}
\item {Grp. gram.:m.}
\end{itemize}
Casta de uva de Leiria.
\section{Carrasquento}
\begin{itemize}
\item {Grp. gram.:adj.}
\end{itemize}
O mesmo que \textunderscore carrasquenho\textunderscore .
\section{Carrasquinha}
\begin{itemize}
\item {Grp. gram.:f.}
\end{itemize}
\begin{itemize}
\item {Utilização:Prov.}
\end{itemize}
\begin{itemize}
\item {Utilização:alent.}
\end{itemize}
\begin{itemize}
\item {Utilização:Prov.}
\end{itemize}
Espécie de uva branca, talvez a mesma que o agudenho.
Espécie de cardo comestivel.
Espécie de dança de roda.
\section{Carreadoiro}
\begin{itemize}
\item {Grp. gram.:m.}
\end{itemize}
\begin{itemize}
\item {Utilização:Bras}
\end{itemize}
\begin{itemize}
\item {Proveniência:(De \textunderscore carrear\textunderscore )}
\end{itemize}
Caminho, por onde passam carros.
\section{Carreador}
\begin{itemize}
\item {Grp. gram.:m.}
\end{itemize}
\begin{itemize}
\item {Utilização:Bras}
\end{itemize}
\begin{itemize}
\item {Proveniência:(De \textunderscore carrear\textunderscore )}
\end{itemize}
Caminho de carro, no campo.
\section{Carreadouro}
\begin{itemize}
\item {Grp. gram.:m.}
\end{itemize}
\begin{itemize}
\item {Utilização:Bras}
\end{itemize}
\begin{itemize}
\item {Proveniência:(De \textunderscore carrear\textunderscore )}
\end{itemize}
Caminho, por onde passam carros.
\section{Carrear}
\begin{itemize}
\item {Grp. gram.:v. t.}
\end{itemize}
\begin{itemize}
\item {Grp. gram.:V. i.}
\end{itemize}
Acarretar, levar em carro.
Conduzir, arrastar.
Guiar carros.
\section{Cárrebo}
\begin{itemize}
\item {Grp. gram.:m.}
\end{itemize}
\begin{itemize}
\item {Utilização:Ant.}
\end{itemize}
O mesmo que \textunderscore cáravo\textunderscore .
\section{Cárrega}
\begin{itemize}
\item {Grp. gram.:f.}
\end{itemize}
\begin{itemize}
\item {Utilização:Ant.}
\end{itemize}
O mesmo que \textunderscore carga\textunderscore .
\section{Carréga}
\begin{itemize}
\item {Grp. gram.:f.}
\end{itemize}
\begin{itemize}
\item {Utilização:Prov.}
\end{itemize}
\begin{itemize}
\item {Utilização:minh.}
\end{itemize}
\begin{itemize}
\item {Utilização:Ant.}
\end{itemize}
\begin{itemize}
\item {Proveniência:(De \textunderscore carregar\textunderscore )}
\end{itemize}
Bebedeira.
O mesmo que \textunderscore carga\textunderscore .
\section{Carréga}
\begin{itemize}
\item {Grp. gram.:f.}
\end{itemize}
\begin{itemize}
\item {Utilização:Prov.}
\end{itemize}
Planta gramínea de terrenos pantanosos.
\section{Carrega-bêsta}
\begin{itemize}
\item {Grp. gram.:f.}
\end{itemize}
Espécie de uva branca de bagos graúdos, a mesma que \textunderscore camarate\textunderscore .
\section{Carrega-burro}
\begin{itemize}
\item {Grp. gram.:f.}
\end{itemize}
Casta de uva de Tôrres-Vedras.
\section{Carregação}
\begin{itemize}
\item {Grp. gram.:f.}
\end{itemize}
\begin{itemize}
\item {Utilização:Pop.}
\end{itemize}
Acto de \textunderscore carregar\textunderscore .
Carga.
Grande quantidade.
Defluxeira.
\section{Carregadeira}
\begin{itemize}
\item {Grp. gram.:f.}
\end{itemize}
\begin{itemize}
\item {Utilização:Prov.}
\end{itemize}
\begin{itemize}
\item {Utilização:pop.}
\end{itemize}
\begin{itemize}
\item {Utilização:T. das margens do Sado}
\end{itemize}
\begin{itemize}
\item {Utilização:Bras}
\end{itemize}
\begin{itemize}
\item {Proveniência:(De \textunderscore carregar\textunderscore )}
\end{itemize}
Cabo, com que se carregam ou colhem as velas dos navios.
Mulher, que se occupa em transportar fardos á cabeça.
Sova, tunda.
Forquilha de muitos dentes, para encher as redes em que se transporta palha.
Espécie de formiga.
\section{Carregado}
\begin{itemize}
\item {Grp. gram.:adj.}
\end{itemize}
\begin{itemize}
\item {Utilização:Heráld.}
\end{itemize}
Diz-se da figura, em que assenta outra, de esmalte diverso.
\section{Carregador}
\begin{itemize}
\item {Grp. gram.:m.}
\end{itemize}
\begin{itemize}
\item {Proveniência:(De \textunderscore carregar\textunderscore )}
\end{itemize}
Aquelle que faz fretes.
Aquelle que conduz carga ou passageiros.
Negociante, que manda fazendas em navio.
Aquelle que carrega as bôcas de fogo.
Espécie de barco em Cezimbra.
\section{Carregal}
\begin{itemize}
\item {Grp. gram.:m.}
\end{itemize}
\begin{itemize}
\item {Proveniência:(De \textunderscore carrega\textunderscore ^2)}
\end{itemize}
Terreno, onde abundam carregas.
\section{Carrega-madeira}
\begin{itemize}
\item {Grp. gram.:m.}
\end{itemize}
\begin{itemize}
\item {Utilização:Bras}
\end{itemize}
Passarinho, que constrói o ninho com pauzinhos e gravetos.
\section{Carregamento}
\begin{itemize}
\item {Grp. gram.:m.}
\end{itemize}
Acto de carregar.
Objectos que formam a carga.
Oppressão, pêso.
\section{Carregar}
\begin{itemize}
\item {Grp. gram.:v. t.}
\end{itemize}
\begin{itemize}
\item {Grp. gram.:V. i.}
\end{itemize}
\begin{itemize}
\item {Proveniência:(Do b. lat. \textunderscore carricare\textunderscore )}
\end{itemize}
Pôr carga sôbre: \textunderscore carregar um macho\textunderscore .
Pesar sôbre.
Encher.
Vexar.
Perturbar, tornar sombrio, torvo, triste.
Meter pólvora e projécteis em: \textunderscore carregar a espingarda\textunderscore .
Acumular electricidade em.
Investir contra: \textunderscore carregar o inimigo\textunderscore .
Imputar.
Aggravar; exaggerar: \textunderscore carregar responsabilidades\textunderscore .
Colhêr (velas do navio).
Attribuir.
Basear-se, firmar-se.
Tomar carga; sentir pêso.
Insistir.
Adquirir mais fôrça.
Avançar com ímpeto.
Dirigir a marcha.
\section{Cárrego}
\begin{itemize}
\item {Grp. gram.:m.}
\end{itemize}
\begin{itemize}
\item {Utilização:Ant.}
\end{itemize}
(V.cargo)
\section{Carrêgo}
\begin{itemize}
\item {Grp. gram.:m.}
\end{itemize}
\begin{itemize}
\item {Utilização:Bras}
\end{itemize}
Acto de carregar.
Carga ou fardo, que se leva á cabeça, ou aos ombros, ou ás costas.
Carga de peça de artilharia.
\section{Carregonceira}
\begin{itemize}
\item {Grp. gram.:f.}
\end{itemize}
\begin{itemize}
\item {Proveniência:(De \textunderscore carregar\textunderscore )}
\end{itemize}
Rolha de junco e lama, para tapar a alcatruzada, nas salinas.
\section{Carregoso}
\begin{itemize}
\item {Grp. gram.:adj.}
\end{itemize}
\begin{itemize}
\item {Proveniência:(De \textunderscore carrêgo\textunderscore )}
\end{itemize}
Pesado, incômmodo.
\section{Carreira}
\begin{itemize}
\item {Grp. gram.:f.}
\end{itemize}
\begin{itemize}
\item {Grp. gram.:Loc. adv.}
\end{itemize}
Caminho de carro.
Estrada pouco larga, carril, carreiro.
Trilho, rasto.
Corrida veloz.
Caminho fechado entre barreiras, para corridas de cavallos.
Curso; profissão em que póde haver melhoria ou retrocesso.
Esphéra de actividade pessoal.
Qualquer percurso: \textunderscore a carreira de advogado\textunderscore .
Meio ou modo de proceder.
O decurso da existencia.
\textunderscore Em carreira\textunderscore , em linha; um atrás do outro.
\section{Carreiro}
\begin{itemize}
\item {Grp. gram.:m.}
\end{itemize}
Aquelle que dirige carro de bois.
Caminho estreito; atalho, vereda.
Espaço entre fileiras de árvores.
Caminho, que as formigas seguem em bando.
\section{Carreiró}
\begin{itemize}
\item {Grp. gram.:m.}
\end{itemize}
\begin{itemize}
\item {Utilização:Mad}
\end{itemize}
Ave, (\textunderscore anthus Bertheloti\textunderscore , Bolle).
\section{Carreirola}
\begin{itemize}
\item {Grp. gram.:f.}
\end{itemize}
Espécie de calhandra, (\textunderscore alauda brachydactyla\textunderscore , Lin.).
\section{Carreirote}
\begin{itemize}
\item {Grp. gram.:m.}
\end{itemize}
O mesmo que \textunderscore carreiro\textunderscore .
\section{Carrejão}
\begin{itemize}
\item {Grp. gram.:m.}
\end{itemize}
O mesmo que \textunderscore carregador\textunderscore .
\section{Carrejar}
\begin{itemize}
\item {Grp. gram.:v. i.}
\end{itemize}
O mesmo que \textunderscore carrear\textunderscore .
\section{Carrejo}
\begin{itemize}
\item {Grp. gram.:f.}
\end{itemize}
\begin{itemize}
\item {Utilização:Prov.}
\end{itemize}
\begin{itemize}
\item {Utilização:minh.}
\end{itemize}
Acto de carrejar.
Transporte de cereaes para os domicilios.
\section{Carrela}
\begin{itemize}
\item {Grp. gram.:f.}
\end{itemize}
\begin{itemize}
\item {Utilização:Prov.}
\end{itemize}
\begin{itemize}
\item {Utilização:minh.}
\end{itemize}
O mesmo que \textunderscore padiola\textunderscore .
\section{Carrelo}
\begin{itemize}
\item {fónica:ré}
\end{itemize}
\begin{itemize}
\item {Grp. gram.:m.}
\end{itemize}
\begin{itemize}
\item {Utilização:Prov.}
\end{itemize}
\begin{itemize}
\item {Utilização:trasm.}
\end{itemize}
Carramolo.
Montão de castanhas.
Cacho grande de frutos.
(Cp. \textunderscore carro\textunderscore ^1)
\section{Carrêta}
\begin{itemize}
\item {Grp. gram.:f.}
\end{itemize}
Pequeno carro; carroça.
Jôgo deanteiro da charrua e de outros instrumentos agricolas.
Viatura de artilharia.
Designação popular da Ursa-Maior.
\section{Carretada}
\begin{itemize}
\item {Grp. gram.:f.}
\end{itemize}
(V.carrada)
\section{Carretagem}
\begin{itemize}
\item {Grp. gram.:f.}
\end{itemize}
Acção de carretar.
Preço de um carrêto.
\section{Carretão}
\begin{itemize}
\item {Grp. gram.:m.}
\end{itemize}
O mesmo que \textunderscore carreteiro\textunderscore .
\section{Carretar}
\begin{itemize}
\item {Grp. gram.:v. t.}
\end{itemize}
O mesmo que \textunderscore acarretar\textunderscore .
\section{Carretas}
\begin{itemize}
\item {fónica:rê}
\end{itemize}
\begin{itemize}
\item {Grp. gram.:m.}
\end{itemize}
\begin{itemize}
\item {Utilização:Prov.}
\end{itemize}
\begin{itemize}
\item {Utilização:trasm.}
\end{itemize}
Homem cambaio, torto das pernas.
\section{Carrête}
\begin{itemize}
\item {Grp. gram.:m.}
\end{itemize}
\begin{itemize}
\item {Proveniência:(De \textunderscore carro\textunderscore )}
\end{itemize}
O mesmo que \textunderscore carretel\textunderscore .
Carrinho.
Pequena roda, ou peça cylíndrica, em vários maquinismos.
\section{Carretear}
\begin{itemize}
\item {Grp. gram.:v. t.}
\end{itemize}
(V.acarretar)
\section{Carreteira}
\begin{itemize}
\item {Grp. gram.:f.}
\end{itemize}
\begin{itemize}
\item {Utilização:Prov.}
\end{itemize}
\begin{itemize}
\item {Utilização:trasm.}
\end{itemize}
\begin{itemize}
\item {Utilização:Prov.}
\end{itemize}
\begin{itemize}
\item {Utilização:beir.}
\end{itemize}
\begin{itemize}
\item {Proveniência:(De \textunderscore carrêta\textunderscore )}
\end{itemize}
Rodeira do carro.
Caminho por entre matos.
\section{Carreteiro}
\begin{itemize}
\item {Grp. gram.:m.}
\end{itemize}
\begin{itemize}
\item {Grp. gram.:Adj.}
\end{itemize}
Aquelle que dirige carrêta ou faz carretos.
Diz-se do barco, empregado na carga e descarga de navios.
\section{Carretel}
\begin{itemize}
\item {Grp. gram.:m.}
\end{itemize}
\begin{itemize}
\item {Utilização:Náut.}
\end{itemize}
\begin{itemize}
\item {Proveniência:(De \textunderscore carrête\textunderscore )}
\end{itemize}
Pequeno cylindro de madeira com rebordos, para nelle se enrolar fio para costura ou lâminas para cordas de instrumentos.
Carrinho.
Molinete.
Pau roliço, que se atravessa por baixo de corpos pesados para os fazer mover, rodando.
Cylindro, onde se enrola a linha da barquinha, que mede a velocidade do navio.
\section{Carretilha}
\begin{itemize}
\item {Grp. gram.:f.}
\end{itemize}
\begin{itemize}
\item {Utilização:Prov.}
\end{itemize}
\begin{itemize}
\item {Utilização:alg.}
\end{itemize}
\begin{itemize}
\item {Proveniência:(De \textunderscore carrêta\textunderscore )}
\end{itemize}
Pequeno instrumento circular, com que se corta a massa de pastéis e bolos.
Broca de ferreiro, embebida numa pequena roda, que se move com a corda de um arco.
Peça de fogo de artifício, espécie de busca-pés.
\section{Carretilho}
\begin{itemize}
\item {Grp. gram.:m.}
\end{itemize}
\begin{itemize}
\item {Utilização:Prov.}
\end{itemize}
\begin{itemize}
\item {Utilização:beir.}
\end{itemize}
\begin{itemize}
\item {Proveniência:(De \textunderscore carrête\textunderscore )}
\end{itemize}
Carrinho de mão.
\section{Carrêto}
\begin{itemize}
\item {Grp. gram.:m.}
\end{itemize}
Acção de carretar.
Frete.
Preço do frete.
\section{Carriagem}
\begin{itemize}
\item {Grp. gram.:f.}
\end{itemize}
\begin{itemize}
\item {Utilização:Ant.}
\end{itemize}
\begin{itemize}
\item {Proveniência:(De \textunderscore carriar\textunderscore  por \textunderscore carrear\textunderscore )}
\end{itemize}
Série de carros.
Carretagem.
\section{Carrião}
\begin{itemize}
\item {Grp. gram.:m.}
\end{itemize}
\begin{itemize}
\item {Proveniência:(De \textunderscore carro\textunderscore )}
\end{itemize}
Instrumento de pisoeiro, formado de um eixo e duas rodas.
\section{Carriça}
\begin{itemize}
\item {Grp. gram.:f.}
\end{itemize}
Passarinho dentirostro, de côr castanho-escura, (\textunderscore troglodytes parvulus\textunderscore , Roxb.).
\section{Carriçal}
\begin{itemize}
\item {Grp. gram.:m.}
\end{itemize}
Moita de carriços.
\section{Carricinha}
\begin{itemize}
\item {Grp. gram.:f.}
\end{itemize}
\begin{itemize}
\item {Utilização:Prov.}
\end{itemize}
O mesmo que \textunderscore carriça\textunderscore .
\section{Carriço}
\begin{itemize}
\item {Grp. gram.:m.}
\end{itemize}
\begin{itemize}
\item {Utilização:T. da Bairrada}
\end{itemize}
Planta cyperácea, (\textunderscore carrex ambigua\textunderscore ).
O mesmo que \textunderscore carriça\textunderscore .
O mesmo que \textunderscore carrapiço\textunderscore .
\section{Carrijó}
\begin{itemize}
\item {Grp. gram.:m.}
\end{itemize}
\begin{itemize}
\item {Utilização:Prov.}
\end{itemize}
O mesmo que \textunderscore cabraxó\textunderscore .
\section{Carril}
\begin{itemize}
\item {Grp. gram.:m.}
\end{itemize}
\begin{itemize}
\item {Utilização:Prov.}
\end{itemize}
\begin{itemize}
\item {Utilização:trasm.}
\end{itemize}
\begin{itemize}
\item {Utilização:beir.}
\end{itemize}
Rasto, que deixam as rodas do carro; rodeira.
Barra de ferro, fixa geralmente em travéssas de madeira, e sôbre que se movem as rodas de differentes vehículos.
Carro de charrua.
Caminho estreito, carreiro.
\section{Carril}
\begin{itemize}
\item {Grp. gram.:m.}
\end{itemize}
\begin{itemize}
\item {Proveniência:(De \textunderscore Carril\textunderscore , n. p.)}
\end{itemize}
Variedade de pêra, cultivada no Minho.
\section{Carrilamento}
\begin{itemize}
\item {Grp. gram.:m.}
\end{itemize}
Acto de carrilar.
\section{Carrilano}
\begin{itemize}
\item {Grp. gram.:m.}
\end{itemize}
\begin{itemize}
\item {Utilização:T. de Albergaria}
\end{itemize}
\begin{itemize}
\item {Proveniência:(De \textunderscore carril\textunderscore )}
\end{itemize}
Aquelle que trabalha em construcção de linhas férreas.
\section{Carrilar}
\begin{itemize}
\item {Grp. gram.:v. t.}
\end{itemize}
\begin{itemize}
\item {Proveniência:(De \textunderscore carril\textunderscore )}
\end{itemize}
Collocar (vehículos) nos respectivos carris.
\section{Carrilhador}
\begin{itemize}
\item {Grp. gram.:m.}
\end{itemize}
(V.carrilhanor)
\section{Carrilhanor}
\begin{itemize}
\item {Grp. gram.:m.}
\end{itemize}
Tocador de carrilhão.
\section{Carrilhão}
\begin{itemize}
\item {Grp. gram.:m.}
\end{itemize}
\begin{itemize}
\item {Proveniência:(Fr. \textunderscore carrillon\textunderscore )}
\end{itemize}
Conjunto de sinos, com que se tocam peças de música.
Instrumento de Phýsica, composto de bólas metállicas e campaínhas, e que repica sôb a acção da electricidade.
\section{Carrilheira}
\begin{itemize}
\item {Grp. gram.:f.}
\end{itemize}
\begin{itemize}
\item {Utilização:T. de Almeida}
\end{itemize}
Maxilla inferior do porco, a qual, depois de despojada dos tecidos, sêca e partida, apresenta interiormente uma substância gordurosa, que se applica em fricções.
\section{Carrilho}
\begin{itemize}
\item {Grp. gram.:m.}
\end{itemize}
\begin{itemize}
\item {Utilização:Des.}
\end{itemize}
\begin{itemize}
\item {Utilização:Prolóq.}
\end{itemize}
Espiga de milho, depois de esbagoada.
Bochecha, face.
\textunderscore Comer a dois carrilhos\textunderscore , servir com lucro a dois senhores ou a dois partidos; acumular interesses.
(Cast. \textunderscore carrillo\textunderscore , bochecha)
\section{Carrilho}
\begin{itemize}
\item {Grp. gram.:m.}
\end{itemize}
\begin{itemize}
\item {Utilização:Prov.}
\end{itemize}
\begin{itemize}
\item {Utilização:trasm.}
\end{itemize}
\begin{itemize}
\item {Proveniência:(De \textunderscore carro\textunderscore )}
\end{itemize}
Espécie de dobadoira ou sarilho para seda.
\section{Carrimónia}
\begin{itemize}
\item {Grp. gram.:f.}
\end{itemize}
\begin{itemize}
\item {Utilização:Deprec.}
\end{itemize}
Carruagem velha ou ordinária.
\section{Carrimpana}
\begin{itemize}
\item {Grp. gram.:f.}
\end{itemize}
\begin{itemize}
\item {Utilização:T. da Bairrada}
\end{itemize}
O mesmo que \textunderscore carripana\textunderscore .
\section{Carrinha}
\begin{itemize}
\item {Grp. gram.:f.}
\end{itemize}
Pequena carroça alentejana e algarvia.
\section{Carrinho}
\begin{itemize}
\item {Grp. gram.:m.}
\end{itemize}
O mesmo que \textunderscore carretel\textunderscore .
Pequeno carro.
Antiga argola de ferro, que se adaptava ás pernas dos soldados por castigo.
\textunderscore Carrinho de mão\textunderscore , o mesmo que carro de mão.
\textunderscore Carrinho do monte\textunderscore , espécie de zorra ou carro sem rodas, que, na Madeira, os condutores arrastam, puxando-os por cordas ou correias.
\section{Carriofa}
\begin{itemize}
\item {Grp. gram.:f.}
\end{itemize}
Carro ordinário; carroça.
\section{Carripada}
\begin{itemize}
\item {Grp. gram.:f.}
\end{itemize}
\begin{itemize}
\item {Utilização:T. da Bairrada}
\end{itemize}
Carrada pequena.
\section{Carripana}
\begin{itemize}
\item {Grp. gram.:f.}
\end{itemize}
\begin{itemize}
\item {Utilização:Prov.}
\end{itemize}
\begin{itemize}
\item {Proveniência:(De \textunderscore carro\textunderscore )}
\end{itemize}
Vehículo ordinário ou diligência reles, para transporte de passageiros.
\section{Carripoila}
\begin{itemize}
\item {Grp. gram.:f.}
\end{itemize}
\begin{itemize}
\item {Utilização:Prov.}
\end{itemize}
O mesmo que \textunderscore carripana\textunderscore .
\section{Carro}
\begin{itemize}
\item {Grp. gram.:m.}
\end{itemize}
\begin{itemize}
\item {Proveniência:(Lat. \textunderscore carrus\textunderscore )}
\end{itemize}
Vehículo de rodas, para transporte de coisas ou pessôas.
Carro de mão, pequeno vehículo de uma só roda, para transporte de entulho, pedras, etc.
\textunderscore Carro de Vênus\textunderscore , o mesmo que \textunderscore acónito\textunderscore .
\section{Carro}
\begin{itemize}
\item {Grp. gram.:m.}
\end{itemize}
\begin{itemize}
\item {Utilização:Náut.}
\end{itemize}
\begin{itemize}
\item {Proveniência:(Do fr. \textunderscore carre\textunderscore )}
\end{itemize}
A parte intero-inferior e mais grossa da antena das velas bastardas.
\section{Carroá}
\begin{itemize}
\item {Grp. gram.:f.}
\end{itemize}
Gênero de palmeiras, procedente das Antilhas.
Filamento têxtil, tirado dessa árvore. Cf. a \textunderscore Pauta das Alfândegas\textunderscore .
O mesmo que \textunderscore caroá\textunderscore ?
\section{Carroça}
\begin{itemize}
\item {Grp. gram.:f.}
\end{itemize}
\begin{itemize}
\item {Proveniência:(Do b. lat. \textunderscore carrocia\textunderscore )}
\end{itemize}
Carro grosseiro, com resguardo de grades ou taipaes.
\section{Carroçada}
\begin{itemize}
\item {Grp. gram.:f.}
\end{itemize}
Carga de uma carroça; o que uma carroça póde transportar.
\section{Carroção}
\begin{itemize}
\item {Grp. gram.:m.}
\end{itemize}
\begin{itemize}
\item {Proveniência:(De \textunderscore carroça\textunderscore )}
\end{itemize}
Grande carro de bois, coberto, que se usava para transporte de pessôas.
\section{Carroceiro}
\begin{itemize}
\item {Grp. gram.:m.}
\end{itemize}
Conductor de carroça; aquelle que faz fretes com carroça.
\section{Carrocel}
\begin{itemize}
\item {Grp. gram.:m.}
\end{itemize}
Espécie de rodízio, movido com manivela e que tem suspensos á roda pequenos cavallos e vehículos, em que os rapazes se collocam, seguindo o movimento do rodízio.
(Palavra malfeita, do fr. \textunderscore carrousel\textunderscore , sôb a infl. de \textunderscore carroça\textunderscore )
\section{Carrocho}
\begin{itemize}
\item {fónica:rô}
\end{itemize}
\begin{itemize}
\item {Grp. gram.:m.}
\end{itemize}
\begin{itemize}
\item {Utilização:Prov.}
\end{itemize}
\begin{itemize}
\item {Utilização:trasm.}
\end{itemize}
O mesmo que \textunderscore mocho\textunderscore .
(Provavelmente, alt. de \textunderscore carôcho\textunderscore )
\section{Carrocho}
\begin{itemize}
\item {fónica:rô}
\end{itemize}
\begin{itemize}
\item {Grp. gram.:m.}
\end{itemize}
\begin{itemize}
\item {Utilização:Prov.}
\end{itemize}
Caminho estreito, atalho.
(Cp. \textunderscore carril\textunderscore ^1)
\section{Caro}
\begin{itemize}
\item {Grp. gram.:m.}
\end{itemize}
\begin{itemize}
\item {Utilização:Med.}
\end{itemize}
\begin{itemize}
\item {Proveniência:(Gr. \textunderscore karos\textunderscore )}
\end{itemize}
Somnolência, no último grau do estado comatoso.
\section{Carrocim}
\begin{itemize}
\item {Grp. gram.:m.}
\end{itemize}
Pequena carroça; pequeno coche.
\section{Carromato}
\begin{itemize}
\item {fónica:cá}
\end{itemize}
\begin{itemize}
\item {Grp. gram.:m.}
\end{itemize}
\begin{itemize}
\item {Proveniência:(It. \textunderscore carro-matto\textunderscore )}
\end{itemize}
Carro de rodas grandes, cujo taboleiro é formado de cordas entrançadas.
\section{Carroucho}
\begin{itemize}
\item {Grp. gram.:m.}
\end{itemize}
\begin{itemize}
\item {Utilização:Prov.}
\end{itemize}
\begin{itemize}
\item {Utilização:minh.}
\end{itemize}
(V. \textunderscore carrocho\textunderscore ^2)
\section{Carruagem}
\begin{itemize}
\item {Grp. gram.:f.}
\end{itemize}
\begin{itemize}
\item {Proveniência:(De \textunderscore carro\textunderscore )}
\end{itemize}
Carro de caixa, sôbre molas, destinado a transporte de pessôas.
Vagão.
\section{Carruagem}
\begin{itemize}
\item {Grp. gram.:f.}
\end{itemize}
\begin{itemize}
\item {Utilização:Prov.}
\end{itemize}
Porção de carros.
(Corr. de \textunderscore carriagem\textunderscore )
\section{Carruajar}
\begin{itemize}
\item {Grp. gram.:v. i.}
\end{itemize}
\begin{itemize}
\item {Utilização:Neol.}
\end{itemize}
Andar de carruagem. Cf. Ortigão, \textunderscore Hollanda\textunderscore , 136.
\section{Carruca}
\begin{itemize}
\item {Grp. gram.:f.}
\end{itemize}
\begin{itemize}
\item {Proveniência:(Lat. \textunderscore carruca\textunderscore )}
\end{itemize}
Carroça antiga.
\section{Carrulo}
\begin{itemize}
\item {Grp. gram.:m.}
\end{itemize}
\begin{itemize}
\item {Utilização:Prov.}
\end{itemize}
\begin{itemize}
\item {Utilização:minh.}
\end{itemize}
O alto das costas, o espaço posterior entre os ombros.
(Colhido em Barcelos)
\section{Carstenite}
\begin{itemize}
\item {Grp. gram.:f.}
\end{itemize}
\begin{itemize}
\item {Utilização:Miner.}
\end{itemize}
\begin{itemize}
\item {Proveniência:(De \textunderscore Karsten\textunderscore , n. p.)}
\end{itemize}
Espécie de sulfato, que crystalliza geralmente em prismas octogonaes e rectangulares.
\section{Carta}
\begin{itemize}
\item {Grp. gram.:f.}
\end{itemize}
\begin{itemize}
\item {Utilização:Ant.}
\end{itemize}
\begin{itemize}
\item {Proveniência:(Lat. \textunderscore charta\textunderscore )}
\end{itemize}
Fôlha, ou fôlhas, de papel escrito, que se dobra ou fecha noutro papel, e se dirige a pessôas ausentes, dando-lhes notícias, ou fazendo-lhes cumprimentos, pedidos, etc.; missiva.
Mappa: \textunderscore estudar a carta da Europa\textunderscore .
Cada um dos pedaços de cartão, que formam o baralho.
Designação de diversos títulos ou documentos officiaes: \textunderscore a carta de bacharel\textunderscore .
\textunderscore Carta branca\textunderscore , plenos poderes.
\textunderscore Carta mandadeira\textunderscore , carta de ordem, com poderes para algum negócio.
\section{Carta}
\begin{itemize}
\item {Grp. gram.:f.}
\end{itemize}
Peixe de Portugal.
\section{Cartabaque}
\begin{itemize}
\item {Grp. gram.:m.}
\end{itemize}
Peixe da Guiana inglesa, (\textunderscore tetragonopterus latus\textunderscore ).
\section{Cartabuxa}
\begin{itemize}
\item {Grp. gram.:f.}
\end{itemize}
Escôva de arame.
\section{Cartabuxar}
\begin{itemize}
\item {Grp. gram.:v. t.}
\end{itemize}
Limpar com cartabuxa.
\section{Cartáceo}
\begin{itemize}
\item {Grp. gram.:adj.}
\end{itemize}
\begin{itemize}
\item {Utilização:Bot.}
\end{itemize}
\begin{itemize}
\item {Proveniência:(De \textunderscore carta\textunderscore )}
\end{itemize}
Sêco, flexível e tenaz, como o pergaminho, (falando-se do episperma e do pericarpo).
\section{Cartada}
\begin{itemize}
\item {Grp. gram.:f.}
\end{itemize}
\begin{itemize}
\item {Proveniência:(De \textunderscore carta\textunderscore )}
\end{itemize}
Acto de jogar uma carta, nos jogos de vasa.
As duas cartas que, no jôgo do monte, o banqueiro tira do baralho, collocando-as a par sôbre a banca.
\section{Cartafede}
\begin{itemize}
\item {Grp. gram.:m.}
\end{itemize}
Planta gramínea da Guiana.
\section{Cartaginês}
\begin{itemize}
\item {Grp. gram.:adj.}
\end{itemize}
\begin{itemize}
\item {Grp. gram.:M.}
\end{itemize}
\begin{itemize}
\item {Proveniência:(Do lat. \textunderscore Carthaginensis\textunderscore )}
\end{itemize}
Relativo a Cartago.
Habitante de Cartago.
\section{Carta-lege}
\begin{itemize}
\item {Grp. gram.:f.}
\end{itemize}
Carta familiar? Cf. Soropita, \textunderscore Prosas\textunderscore , 87.
\section{Cartaloxo}
\begin{itemize}
\item {Grp. gram.:m.}
\end{itemize}
\begin{itemize}
\item {Utilização:Prov.}
\end{itemize}
\begin{itemize}
\item {Utilização:trasm.}
\end{itemize}
\begin{itemize}
\item {Proveniência:(De \textunderscore carta\textunderscore  + \textunderscore ?\textunderscore )}
\end{itemize}
Objecto cylíndrico, feito de cartas de jogar, e com que se ampara a estriga na roca.
\section{Cartamina}
\begin{itemize}
\item {Grp. gram.:f.}
\end{itemize}
Substância còrante, extraída do cártamo.
\section{Cártamo}
\begin{itemize}
\item {Grp. gram.:m.}
\end{itemize}
\begin{itemize}
\item {Proveniência:(Do ár. \textunderscore kartum\textunderscore )}
\end{itemize}
Planta herbácea, da fam. das compostas.
\section{Cartão}
\begin{itemize}
\item {Grp. gram.:m.}
\end{itemize}
\begin{itemize}
\item {Proveniência:(De \textunderscore carta\textunderscore )}
\end{itemize}
Papel muito encorpado.
Papelão.
Representação artística de um papel enrolado na extremidade, exhibindo uma legenda.
Bilhete de visita.
\section{Cartão-pedra}
\begin{itemize}
\item {Grp. gram.:m.}
\end{itemize}
\begin{itemize}
\item {Utilização:Constr.}
\end{itemize}
Papelão que, pisado, reduzido a massa e misturado com colla ou farinha triga e alúmen, recebe qualquer fórma e a conserva, tornando-se duro e resistente.
\section{Cartapácio}
\begin{itemize}
\item {Grp. gram.:m.}
\end{itemize}
Livro de lembranças.
Collecção de documentos, em fórma de livro.
Calhamaço.
(B. lat. \textunderscore chartapacia\textunderscore )
\section{Cartapé}
\begin{itemize}
\item {Grp. gram.:m.}
\end{itemize}
Papel, com que se envolve a estriga na roca; cartaloxo.
O mesmo que \textunderscore cartapelle\textunderscore .
\section{Cartapele}
\begin{itemize}
\item {Grp. gram.:f.}
\end{itemize}
\begin{itemize}
\item {Utilização:Prov.}
\end{itemize}
\begin{itemize}
\item {Utilização:beir.}
\end{itemize}
\begin{itemize}
\item {Proveniência:(De \textunderscore carta\textunderscore  + \textunderscore pelle\textunderscore )}
\end{itemize}
Invólucro de papel ou de pele, com que as fiandeiras amparam a estriga na roca.
\section{Cartapelle}
\begin{itemize}
\item {Grp. gram.:f.}
\end{itemize}
\begin{itemize}
\item {Utilização:Prov.}
\end{itemize}
\begin{itemize}
\item {Utilização:beir.}
\end{itemize}
\begin{itemize}
\item {Proveniência:(De \textunderscore carta\textunderscore  + \textunderscore pelle\textunderscore )}
\end{itemize}
Invólucro de papel ou de pelle, com que as fiandeiras amparam a estriga na roca.
\section{Cartapolinho}
\begin{itemize}
\item {Grp. gram.:m.}
\end{itemize}
\begin{itemize}
\item {Utilização:Ant.}
\end{itemize}
Papel escrito por escrivão público. Cf. Filinto, III, 101.
\section{Cartário}
\begin{itemize}
\item {Grp. gram.:m.}
\end{itemize}
\begin{itemize}
\item {Utilização:Ant.}
\end{itemize}
\begin{itemize}
\item {Proveniência:(De \textunderscore carta\textunderscore )}
\end{itemize}
Tombo ou livro, em que se guardavam títulos de doação e outros.
\section{Cartasana}
\begin{itemize}
\item {Grp. gram.:f.}
\end{itemize}
\begin{itemize}
\item {Utilização:Des.}
\end{itemize}
\begin{itemize}
\item {Proveniência:(De \textunderscore carta\textunderscore )}
\end{itemize}
Pergaminho ou cartão, com fios de oiro, de prata ou retrós, para bordados ou guarnições.
\section{Cartaxinho}
\begin{itemize}
\item {Grp. gram.:m.}
\end{itemize}
\begin{itemize}
\item {Utilização:Prov.}
\end{itemize}
\begin{itemize}
\item {Utilização:alent.}
\end{itemize}
Homem de baixa estatura.
\section{Cartaxo}
\begin{itemize}
\item {Grp. gram.:m.}
\end{itemize}
\begin{itemize}
\item {Utilização:T. de San-Pedro-do-Sul}
\end{itemize}
Pássaro dentirostro, de cabeça e asas pretas e peito amarelo, (\textunderscore pratincola rubíncola\textunderscore , Lin.).
O mesmo que \textunderscore gyrino\textunderscore .
\section{Cartaz}
\begin{itemize}
\item {Grp. gram.:m.}
\end{itemize}
\begin{itemize}
\item {Utilização:Des.}
\end{itemize}
\begin{itemize}
\item {Proveniência:(De \textunderscore carta\textunderscore )}
\end{itemize}
Papel grande, que contém um ou mais annúncios, e destinado a affixar-se em lugar público.
Passaporte, que os conquistadores portugueses davam aos commerciantes, para cruzar o mar das Índias.
\section{Cartazeiro}
\begin{itemize}
\item {Grp. gram.:m.}
\end{itemize}
\begin{itemize}
\item {Proveniência:(De \textunderscore cartaz\textunderscore )}
\end{itemize}
Aquelle que, por offício ou por estipêndio, se emprega em affixar cartazes.
\section{Carteamento}
\begin{itemize}
\item {Grp. gram.:m.}
\end{itemize}
Acto de se cartear.
\section{Cartear}
\begin{itemize}
\item {Grp. gram.:v. t.}
\end{itemize}
\begin{itemize}
\item {Grp. gram.:V. i.}
\end{itemize}
\begin{itemize}
\item {Grp. gram.:V. p.}
\end{itemize}
Jogar com cartas.
Calcular na carta geográphica o ponto em que se acha o navio.
Têr correspondencia por cartas.
\section{Carteio}
\begin{itemize}
\item {Grp. gram.:m.}
\end{itemize}
O mesmo que \textunderscore carteamento\textunderscore .
\section{Carteira}
\begin{itemize}
\item {Grp. gram.:f.}
\end{itemize}
\begin{itemize}
\item {Proveniência:(De \textunderscore carta\textunderscore )}
\end{itemize}
Bôlsa de coiro, para guardar ou enviar cartas ou outros papéis.
Pequena bôlsa de coiro, de metal ou de outra substância, para trazer na algibeira, tendo nella papéis de valor ou lembranças escritas.
Livrinho de lembranças.
Banca de escrever, secretária.
\section{Carteiro}
\begin{itemize}
\item {Grp. gram.:m.}
\end{itemize}
\begin{itemize}
\item {Proveniência:(Do lat. \textunderscore chartarius\textunderscore )}
\end{itemize}
Distribuïdor de cartas.
Condutor de malas postaes.
Fabricante de cartas de jogar.
\section{Cartéis}
\begin{itemize}
\item {Grp. gram.:m. pl.}
\end{itemize}
Certas redes da Póvoa de Varzim.
\section{Cartel}
\begin{itemize}
\item {Grp. gram.:m.}
\end{itemize}
\begin{itemize}
\item {Utilização:Ant.}
\end{itemize}
\begin{itemize}
\item {Utilização:Prov.}
\end{itemize}
\begin{itemize}
\item {Utilização:minh.}
\end{itemize}
Carta para desafio.
Annúncio público. Cf. \textunderscore Viriato Trág.\textunderscore  XI, 11.
Brincadeira, pândega.
(Cast. \textunderscore cartel\textunderscore )
\section{Cartela}
\begin{itemize}
\item {Grp. gram.:f.}
\end{itemize}
\begin{itemize}
\item {Utilização:Eccles.}
\end{itemize}
\begin{itemize}
\item {Proveniência:(De \textunderscore carta\textunderscore )}
\end{itemize}
Espaço liso, num pedestal ou num ornato architectural, e destinado á inscripção de uma legenda.
O mesmo que \textunderscore sacra\textunderscore  ou \textunderscore cánon\textunderscore .
\section{Carteleta}
\begin{itemize}
\item {fónica:lê}
\end{itemize}
\begin{itemize}
\item {Grp. gram.:f.}
\end{itemize}
\begin{itemize}
\item {Utilização:Des.}
\end{itemize}
\begin{itemize}
\item {Proveniência:(T. cast.)}
\end{itemize}
Estofo ligeiro de lan.
\section{Cartérico}
\begin{itemize}
\item {Grp. gram.:m.}
\end{itemize}
Insecto coleóptero, da fam. dos longicórneos.
\section{Cartesianismo}
\begin{itemize}
\item {Grp. gram.:m.}
\end{itemize}
\begin{itemize}
\item {Proveniência:(De \textunderscore cartesiano\textunderscore )}
\end{itemize}
Systema philosóphico de Descartes.
\section{Cartesiano}
\begin{itemize}
\item {Grp. gram.:adj.}
\end{itemize}
\begin{itemize}
\item {Proveniência:(De \textunderscore Cartesius\textunderscore , n. p. lat. de \textunderscore Descartes\textunderscore )}
\end{itemize}
Relativo ao systema de Descartes.
\section{Carteta}
\begin{itemize}
\item {fónica:tê}
\end{itemize}
\begin{itemize}
\item {Grp. gram.:f.}
\end{itemize}
\begin{itemize}
\item {Proveniência:(De \textunderscore carta\textunderscore )}
\end{itemize}
Antigo jôgo de parar.
\section{Carthaginês}
\begin{itemize}
\item {Grp. gram.:adj.}
\end{itemize}
\begin{itemize}
\item {Grp. gram.:M.}
\end{itemize}
\begin{itemize}
\item {Proveniência:(Do lat. \textunderscore Carthaginensis\textunderscore )}
\end{itemize}
Relativo a Carthago.
Habitante de Carthago.
\section{Carthamina}
\begin{itemize}
\item {Grp. gram.:f.}
\end{itemize}
Substância còrante, extrahída do cárthamo.
\section{Cárthamo}
\begin{itemize}
\item {Grp. gram.:m.}
\end{itemize}
\begin{itemize}
\item {Proveniência:(Do ár. \textunderscore kartum\textunderscore )}
\end{itemize}
Planta herbácea, da fam. das compostas.
\section{Cartilagem}
\begin{itemize}
\item {Grp. gram.:f.}
\end{itemize}
\begin{itemize}
\item {Utilização:Anat.}
\end{itemize}
\begin{itemize}
\item {Proveniência:(Lat. \textunderscore cartilago\textunderscore )}
\end{itemize}
Tecido branco e elástico, que se acha especialmente na extremidade dos ossos.
\section{Cartilagíneo}
\begin{itemize}
\item {Grp. gram.:adj.}
\end{itemize}
\begin{itemize}
\item {Proveniência:(Lat. \textunderscore cartilagineus\textunderscore )}
\end{itemize}
Que tem natureza de cartilagem.
\section{Cartilaginoso}
\begin{itemize}
\item {Grp. gram.:adj.}
\end{itemize}
\begin{itemize}
\item {Proveniência:(Lat. \textunderscore cartilaginosus\textunderscore )}
\end{itemize}
Cheio de cartilagens.
Que tem cartilagens.
Cartilagíneo.
\section{Cartilha}
\begin{itemize}
\item {Grp. gram.:f.}
\end{itemize}
\begin{itemize}
\item {Proveniência:(De \textunderscore carta\textunderscore )}
\end{itemize}
Livrinho, em que se aprende a lêr.
Compêndio de doutrina christan.
Tratado elementar.
\section{Cartismo}
\begin{itemize}
\item {Grp. gram.:m.}
\end{itemize}
Partido político dos sectários da \textunderscore Carta Constitucional\textunderscore . Cf. Herculano, \textunderscore Voz do Propheta\textunderscore , introd.
\section{Cartista}
\begin{itemize}
\item {Grp. gram.:m.  e  adj.}
\end{itemize}
\begin{itemize}
\item {Proveniência:(De \textunderscore Carta\textunderscore , n. p.)}
\end{itemize}
Partidário da \textunderscore Carta Constitucional\textunderscore .
Relativo aos partidários da \textunderscore Carta\textunderscore .
\section{Cartografia}
\begin{itemize}
\item {Grp. gram.:f.}
\end{itemize}
\begin{itemize}
\item {Proveniência:(De \textunderscore cartógrapho\textunderscore )}
\end{itemize}
Arte de compor cartas geográficas.
\section{Cartográfico}
\begin{itemize}
\item {Grp. gram.:adj.}
\end{itemize}
Relativo á cartografia.
\section{Cartógrafo}
\begin{itemize}
\item {Grp. gram.:m.}
\end{itemize}
\begin{itemize}
\item {Proveniência:(Do gr. \textunderscore khartes\textunderscore  + \textunderscore graphein\textunderscore )}
\end{itemize}
Aquelle que trabalha ou é versado em cartografia.
\section{Cartographia}
\begin{itemize}
\item {Grp. gram.:f.}
\end{itemize}
\begin{itemize}
\item {Proveniência:(De \textunderscore cartógrapho\textunderscore )}
\end{itemize}
Arte de compor cartas geográphicas.
\section{Cartográphico}
\begin{itemize}
\item {Grp. gram.:adj.}
\end{itemize}
Relativo á cartographia.
\section{Cartógrapho}
\begin{itemize}
\item {Grp. gram.:m.}
\end{itemize}
\begin{itemize}
\item {Proveniência:(Do gr. \textunderscore khartes\textunderscore  + \textunderscore graphein\textunderscore )}
\end{itemize}
Aquelle que trabalha ou é versado em cartographia.
\section{Cartola}
\begin{itemize}
\item {Grp. gram.:f.}
\end{itemize}
\begin{itemize}
\item {Utilização:Pop.}
\end{itemize}
\begin{itemize}
\item {Utilização:Prov.}
\end{itemize}
\begin{itemize}
\item {Utilização:trasm.}
\end{itemize}
Chapéu alto.
Chapéu ridículo pela fórma ou pelo tamanho.
O mesmo que \textunderscore quartola\textunderscore .
Bebedeira.
\section{Cartolada}
\begin{itemize}
\item {Grp. gram.:f.}
\end{itemize}
\begin{itemize}
\item {Utilização:Pop.}
\end{itemize}
Achatadela de chapéu com cascudo; gebada.
\section{Cartomancia}
\begin{itemize}
\item {Grp. gram.:f.}
\end{itemize}
\begin{itemize}
\item {Proveniência:(Do gr. \textunderscore khartes\textunderscore  + \textunderscore manteia\textunderscore )}
\end{itemize}
Arte de deitar cartas, para adivinhar.
\section{Cartomante}
\begin{itemize}
\item {Grp. gram.:m.  e  adj.}
\end{itemize}
O que pratíca a cartomancia.
\section{Cartonado}
\begin{itemize}
\item {Grp. gram.:adj.}
\end{itemize}
Encadernado em cartão.
\section{Cartonagem}
\begin{itemize}
\item {Grp. gram.:f.}
\end{itemize}
\begin{itemize}
\item {Proveniência:(De \textunderscore cartonar\textunderscore )}
\end{itemize}
Artefacto de cartão.
Encadernação em cartão.
Volume cartonado.
\section{Cartonar}
\begin{itemize}
\item {Grp. gram.:v. t.}
\end{itemize}
Encadernar em cartão.
\section{Cartorário}
\begin{itemize}
\item {Grp. gram.:m.}
\end{itemize}
\begin{itemize}
\item {Proveniência:(De \textunderscore cartório\textunderscore )}
\end{itemize}
Aquelle que guarda um cartório, ou é escrevente nelle.
\section{Cartório}
\begin{itemize}
\item {Grp. gram.:m.}
\end{itemize}
\begin{itemize}
\item {Proveniência:(De \textunderscore carta\textunderscore )}
\end{itemize}
Lugar, em que se guardam cartas ou quaesquer documentos importantes; archivo.
Escritório de escrivães ou tabelliães.
\section{Cartuchame}
\begin{itemize}
\item {Grp. gram.:m.}
\end{itemize}
Porção de cartuchos para armas de fogo.
\section{Cartucheira}
\begin{itemize}
\item {Grp. gram.:f.}
\end{itemize}
\begin{itemize}
\item {Utilização:Pop.}
\end{itemize}
\begin{itemize}
\item {Proveniência:(De \textunderscore cartucho\textunderscore )}
\end{itemize}
Patrona ou bôlsa para cartuchos.
Os dentes: \textunderscore não mostres a cartucheira\textunderscore , não te rias.
\section{Cartucheiro}
\begin{itemize}
\item {Grp. gram.:m.}
\end{itemize}
Fabricante de cartuchos para espingarda.
\section{Cartucho}
\begin{itemize}
\item {Grp. gram.:m.}
\end{itemize}
\begin{itemize}
\item {Grp. gram.:Adj.}
\end{itemize}
\begin{itemize}
\item {Proveniência:(Do it. \textunderscore cartuccio\textunderscore )}
\end{itemize}
Papel, em que se embrulha, nas mercearias e outras lojas, arroz, açúcar e outros géneros.
Carga para espingarda ou peça.
Embrulho.
Diz-se de uma qualidade de papel ordinário. Cf. \textunderscore Inquér. Industr.\textunderscore , 1.^a p., 230.
\section{Cártula}
\begin{itemize}
\item {Grp. gram.:f.}
\end{itemize}
Parte de um monumento, que simula uma fôlha de papel ou pergaminho, com um letreiro ou dístico; cartela.
(B. lat. \textunderscore chartula\textunderscore , de \textunderscore charta\textunderscore )
\section{Cartulário}
\begin{itemize}
\item {Grp. gram.:m.}
\end{itemize}
Registo dos títulos ou antiguidades de uma corporação, convento ou igreja.
(B. lat. \textunderscore chartularium\textunderscore , de \textunderscore chartula\textunderscore )
\section{Cartulinho}
\begin{itemize}
\item {Grp. gram.:m.}
\end{itemize}
Pequeno escudo.
(Cp. \textunderscore cártula\textunderscore )
\section{Cartusiano}
\begin{itemize}
\item {Grp. gram.:adj.}
\end{itemize}
\begin{itemize}
\item {Proveniência:(De \textunderscore Carthusia\textunderscore , n. p. lat. de Chartreux, onde se recolheu San-Bruno, fundador da ordem dos Cartuxos)}
\end{itemize}
Relativo aos Cartuxos.
\section{Cartuxa}
\begin{itemize}
\item {Grp. gram.:f.}
\end{itemize}
\begin{itemize}
\item {Proveniência:(De \textunderscore Carthusia\textunderscore , n. p.)}
\end{itemize}
Ordem religiosa, muito austera, fundada por San-Bruno.
Conjunto de construcções, que formavam um mosteiro da Ordem de San Bruno.
\section{Cartuxo}
\begin{itemize}
\item {Grp. gram.:m.}
\end{itemize}
Frade da cartuxa.
\section{Caruara}
\begin{itemize}
\item {Grp. gram.:f.}
\end{itemize}
\begin{itemize}
\item {Utilização:Bras. do N}
\end{itemize}
Dôr reumática.
Mau olhado.
Achaque.
Abelha pequenina.
(Do tupi guar. \textunderscore cara\textunderscore  + \textunderscore uara\textunderscore )
\section{Caruata}
\begin{itemize}
\item {Grp. gram.:f.}
\end{itemize}
\begin{itemize}
\item {Proveniência:(T. cast.)}
\end{itemize}
Espécie de pita da Guiana, de que se fazem cordas.
\section{Caruca}
\begin{itemize}
\item {Grp. gram.:f.}
\end{itemize}
Antigo imposto, que se lançava sôbre os criadores de gado, na Índia portuguesa.
\section{Carugem}
\begin{itemize}
\item {Grp. gram.:f.}
\end{itemize}
(V.caruncho)
\section{Caruja}
\begin{itemize}
\item {Grp. gram.:f.}
\end{itemize}
O mesmo que \textunderscore carujeira\textunderscore .
\section{Carujar}
\begin{itemize}
\item {Grp. gram.:v. i.}
\end{itemize}
\begin{itemize}
\item {Utilização:Prov.}
\end{itemize}
Chuviscar.
Cair orvalho.
\section{Carujeira}
\begin{itemize}
\item {Grp. gram.:f.}
\end{itemize}
\begin{itemize}
\item {Proveniência:(De \textunderscore carujar\textunderscore )}
\end{itemize}
Orvalho.
\section{Carujeiro}
\begin{itemize}
\item {Grp. gram.:m.}
\end{itemize}
\begin{itemize}
\item {Utilização:T. de Lamego}
\end{itemize}
Nevoeiro, nebrina espêssa.
\section{Carujo}
\begin{itemize}
\item {Grp. gram.:m.}
\end{itemize}
\begin{itemize}
\item {Utilização:Prov.}
\end{itemize}
\begin{itemize}
\item {Utilização:dur.}
\end{itemize}
Tempo de nevoeiro espêsso.
\section{Carula}
\begin{itemize}
\item {Grp. gram.:f.}
\end{itemize}
\begin{itemize}
\item {Utilização:Ant.}
\end{itemize}
Carocha; escaravelho; vaca-loira.
\section{Caruma}
\begin{itemize}
\item {Grp. gram.:f.}
\end{itemize}
\begin{itemize}
\item {Utilização:Prov.}
\end{itemize}
\begin{itemize}
\item {Utilização:Prov.}
\end{itemize}
\begin{itemize}
\item {Utilização:minh.}
\end{itemize}
Fôlha de pinheiro.
A pellícula, que reveste as castanhas ainda verdes e tenras.
O mesmo que \textunderscore faúlha\textunderscore . (Colhido em Barcelos)
(Or. afr.?)
\section{Caruma-cacuéme}
\begin{itemize}
\item {Grp. gram.:f.}
\end{itemize}
\begin{itemize}
\item {Proveniência:(T. lund.)}
\end{itemize}
Arbusto angolense, de fôlhas verde-escuras e pequenas flôres, com quatro pétalas brancas.
\section{Carumbé}
\begin{itemize}
\item {Grp. gram.:m.}
\end{itemize}
\begin{itemize}
\item {Utilização:Bras}
\end{itemize}
Espécie de gamela cónica, em que se transportam minérios para o lugar da lavagem.
\section{Carunchar}
\begin{itemize}
\item {Grp. gram.:v. i.}
\end{itemize}
Encher-se de caruncho.
\section{Carunchento}
\begin{itemize}
\item {Grp. gram.:adj.}
\end{itemize}
Que tem caruncho.
\section{Caruncho}
\begin{itemize}
\item {Grp. gram.:m.}
\end{itemize}
\begin{itemize}
\item {Utilização:Fig.}
\end{itemize}
Insecto, que corrói a madeira; carcoma.
Podridão.
Velhice.
(Cp. lat. \textunderscore caruncula\textunderscore )
\section{Carunchoso}
\begin{itemize}
\item {Grp. gram.:adj.}
\end{itemize}
\begin{itemize}
\item {Utilização:Fig.}
\end{itemize}
Que tem caruncho.
Carcomido.
Abatido, velho.
\section{Carúncula}
\begin{itemize}
\item {Grp. gram.:f.}
\end{itemize}
\begin{itemize}
\item {Utilização:Zool.}
\end{itemize}
\begin{itemize}
\item {Proveniência:(Lat. \textunderscore caruncula\textunderscore )}
\end{itemize}
Saliência carnuda.
Tecido da crista das aves.
Excrescência mamilar no ponto em que algumas sementes adheriram á placenta.
\section{Carundorundo}
\begin{itemize}
\item {Grp. gram.:m.}
\end{itemize}
Árvore africana das margens do Lovo.
\section{Carunfeiro}
\begin{itemize}
\item {Grp. gram.:m.}
\end{itemize}
\begin{itemize}
\item {Utilização:Gír.}
\end{itemize}
Fadista traiçoeiro.
\section{Carunha}
\begin{itemize}
\item {Grp. gram.:f.}
\end{itemize}
\begin{itemize}
\item {Utilização:Prov.}
\end{itemize}
\begin{itemize}
\item {Utilização:minh.}
\end{itemize}
\begin{itemize}
\item {Utilização:trasm.}
\end{itemize}
Caroço de fruto.
\section{Caruru}
\begin{itemize}
\item {Grp. gram.:m.}
\end{itemize}
Nome de duas plantas do Brasil.
Vinagreira, planta.
Espécie de esparregado (na Baía e em San-Thomé).
Appetitosa iguaria de San-Thomé, preparada com peixe, carne de gallinha, azeite, malaguetas, etc.
\section{Caruru-de-pomba}
\begin{itemize}
\item {Grp. gram.:m.}
\end{itemize}
\begin{itemize}
\item {Utilização:Bras}
\end{itemize}
Planta medicinal, (\textunderscore phytolocea decandra\textunderscore ).
\section{Caruru-guaçu}
\begin{itemize}
\item {Grp. gram.:m.}
\end{itemize}
\begin{itemize}
\item {Utilização:Bras}
\end{itemize}
O mesmo que \textunderscore caruru-de-pomba\textunderscore .
\section{Cárus}
\begin{itemize}
\item {Grp. gram.:m.}
\end{itemize}
\begin{itemize}
\item {Utilização:Med.}
\end{itemize}
\begin{itemize}
\item {Proveniência:(Gr. \textunderscore karos\textunderscore )}
\end{itemize}
Somnolência, no último grau do estado comatoso.
\section{Caruto}
\begin{itemize}
\item {Grp. gram.:m.}
\end{itemize}
Planta rubiácea do Brasil.
\section{Carva}
\begin{itemize}
\item {Grp. gram.:f.}
\end{itemize}
\begin{itemize}
\item {Utilização:T. de Alcanena}
\end{itemize}
Barranco, em caminho ou estrada.
\section{Carvalha}
\begin{itemize}
\item {Grp. gram.:f.}
\end{itemize}
\begin{itemize}
\item {Utilização:Prov.}
\end{itemize}
\begin{itemize}
\item {Utilização:minh.}
\end{itemize}
\begin{itemize}
\item {Grp. gram.:Adj.}
\end{itemize}
Pequeno carvalho; carvalheira.
Carvalho alto e esguio.
Diz-se de uma espécie de batata.
\section{Carvalhádega}
\begin{itemize}
\item {Grp. gram.:f.}
\end{itemize}
\begin{itemize}
\item {Utilização:Ant.}
\end{itemize}
O mesmo que \textunderscore carvalhal\textunderscore .
\section{Carvalhaes}
\begin{itemize}
\item {Grp. gram.:f.}
\end{itemize}
\begin{itemize}
\item {Proveniência:(De \textunderscore Carvalhaes\textunderscore , n. p.)}
\end{itemize}
Variedade de pêra.
\section{Carvalhais}
\begin{itemize}
\item {Grp. gram.:f.}
\end{itemize}
\begin{itemize}
\item {Proveniência:(De \textunderscore Carvalhaes\textunderscore , n. p.)}
\end{itemize}
Variedade de pêra.
\section{Carvalhal}
\begin{itemize}
\item {Grp. gram.:m.}
\end{itemize}
\begin{itemize}
\item {Grp. gram.:F.  e  adj.}
\end{itemize}
Arvoredo de carvalhos.
Diz-se de uma variedade de pêra.
Casta de uva preta minhota.
Variedade de figueira algarvia.
\section{Carvalheira}
\begin{itemize}
\item {Grp. gram.:f.}
\end{itemize}
Carvalhal.
Pequeno carvalho.
Moita de carvalhos silvestres.
\section{Carvalheiro}
\begin{itemize}
\item {Grp. gram.:m.}
\end{itemize}
Carvalho novo.
Bordão de carvalho.
\section{Carvalhiça}
\begin{itemize}
\item {Grp. gram.:f.}
\end{itemize}
Espécie de carvalho rasteiro.
\section{Carvalhiço}
\begin{itemize}
\item {Grp. gram.:m.}
\end{itemize}
O mesmo que \textunderscore carvalhiça\textunderscore .
\section{Carvalhinha}
\begin{itemize}
\item {Grp. gram.:f.}
\end{itemize}
Planta labiada, aquática, (\textunderscore veronica officinalis\textunderscore , Lin.).
\section{Carvalho}
\begin{itemize}
\item {Grp. gram.:m.}
\end{itemize}
Grande árvore amentácea, que produz bolotas.
(Or. desconhecida; mas não poderá alvitrar-se a alter. de \textunderscore curvalho\textunderscore , de \textunderscore curvo\textunderscore , visto que o tronco e os ramos do carvalho são geralmente tortuosos? Ou será o lat. hyp. \textunderscore quercalium\textunderscore , de \textunderscore quercus\textunderscore ?)
\section{Carvalhoto}
\begin{itemize}
\item {fónica:lhô}
\end{itemize}
\begin{itemize}
\item {Grp. gram.:m.}
\end{itemize}
\begin{itemize}
\item {Utilização:Prov.}
\end{itemize}
\begin{itemize}
\item {Utilização:trasm.}
\end{itemize}
Carvalho pequeno ou delgado.
\section{Carvão}
\begin{itemize}
\item {Grp. gram.:m.}
\end{itemize}
\begin{itemize}
\item {Proveniência:(Lat. \textunderscore carbo\textunderscore )}
\end{itemize}
Substância vegetal, mineral ou animal, obtida por combustão.
Brasa, cujo fogo se extinguiu.
Pedaço de madeira carbonizada, tição.
Desenho a carvão.
\section{Carviz}
\begin{itemize}
\item {Grp. gram.:m.}
\end{itemize}
\begin{itemize}
\item {Utilização:Ant.}
\end{itemize}
Pescador na Índia portuguesa.
\section{Carvoaria}
\begin{itemize}
\item {Grp. gram.:f.}
\end{itemize}
Lugar, em que se fabríca ou se vende carvão.
\section{Carvoeira}
\begin{itemize}
\item {Grp. gram.:f.}
\end{itemize}
\begin{itemize}
\item {Proveniência:(De \textunderscore carvão\textunderscore )}
\end{itemize}
Lugar, onde se guarda carvão.
Carvoaria.
Mulher de carvoeiro.
Mulher que vende carvão.
\section{Carvoeiras}
\begin{itemize}
\item {Grp. gram.:f. pl.}
\end{itemize}
\begin{itemize}
\item {Utilização:Prov.}
\end{itemize}
Dança de roda.
\section{Carvoeiro}
\begin{itemize}
\item {Grp. gram.:m.}
\end{itemize}
\begin{itemize}
\item {Utilização:Pop.}
\end{itemize}
\begin{itemize}
\item {Grp. gram.:Adj.}
\end{itemize}
Aquelle que faz, transporta ou vende carvão.
\textunderscore Maré do carvoeiro\textunderscore , opportunidade.
Relativo a carvão.
Negro.
\section{Carvoejar}
\begin{itemize}
\item {Grp. gram.:v. i.}
\end{itemize}
Fazer carvão vegetal.
\section{Carvoíço}
\begin{itemize}
\item {Grp. gram.:m.}
\end{itemize}
\begin{itemize}
\item {Utilização:T. da Bairrada}
\end{itemize}
\begin{itemize}
\item {Proveniência:(De \textunderscore carvão\textunderscore )}
\end{itemize}
Cinza dos fornos de cal, misturada com fragmentos de cal, e que serve para adubar terras.
\section{Caryátide}
\begin{itemize}
\item {Grp. gram.:f.}
\end{itemize}
\begin{itemize}
\item {Proveniência:(Gr. \textunderscore karuatides\textunderscore )}
\end{itemize}
Figura de mulher, sôbre que assenta uma cornija ou archítrave.
\section{Caryochromo}
\begin{itemize}
\item {Grp. gram.:adj.}
\end{itemize}
\begin{itemize}
\item {Proveniência:(Do gr. \textunderscore karuon\textunderscore  + \textunderscore khroma\textunderscore )}
\end{itemize}
Diz-se das partes de um núcleo, que tomam côr pela acção de certas substâncias còrantes.
\section{Caryocinese}
\begin{itemize}
\item {Grp. gram.:f.}
\end{itemize}
\begin{itemize}
\item {Utilização:Biol.}
\end{itemize}
\begin{itemize}
\item {Proveniência:(Do gr. \textunderscore karuon\textunderscore  + \textunderscore kinesis\textunderscore )}
\end{itemize}
Modo de multiplicação das céllulas, por divisão indirecta.
\section{Caryocostino}
\begin{itemize}
\item {Grp. gram.:m.}
\end{itemize}
Espécie de electuário purgativo.
\section{Caryophylláceas}
\begin{itemize}
\item {Grp. gram.:f. pl.}
\end{itemize}
\begin{itemize}
\item {Proveniência:(Do lat. \textunderscore caryophyllus\textunderscore )}
\end{itemize}
Família de plantas, a que serve de typo o craveiro.
\section{Caryopse}
\begin{itemize}
\item {Grp. gram.:m.}
\end{itemize}
\begin{itemize}
\item {Proveniência:(Do gr. \textunderscore karuon\textunderscore  + \textunderscore opsis\textunderscore )}
\end{itemize}
Fruto, cujo pericarpo é soldado aos tegumentos, como o fruto do trigo, cevada, etc.
\section{Caryota}
\begin{itemize}
\item {Grp. gram.:f.}
\end{itemize}
\begin{itemize}
\item {Proveniência:(Gr. \textunderscore karuotos\textunderscore )}
\end{itemize}
Gênero das palmeiras que dão tâmaras.
\section{Casa}
\begin{itemize}
\item {Grp. gram.:f.}
\end{itemize}
\begin{itemize}
\item {Proveniência:(Lat. \textunderscore casa\textunderscore )}
\end{itemize}
Edifício para habitação: \textunderscore uma casa moderna\textunderscore .
Morada, moradía, vivenda: \textunderscore vou para minha casa\textunderscore .
Cada uma das divisões de uma habitação; quarto: \textunderscore um andar com oito casas\textunderscore .
Estabelecimento.
Família: \textunderscore é da casa de Bragança\textunderscore .
Mobiliário.
Bens: \textunderscore fez bôa casa\textunderscore .
Subdivisão de uma caixa, taboleiro, etc.
Repartição pública: \textunderscore Casa da Moéda\textunderscore .
Abertura, em que entram os botões do fato.
Cada um dos espaços, separados por traços numa tabella ou mappa.
Lugar, occupado por um algarismo, em relação a outros que fórmam com elle o mesmo número: \textunderscore a casa das centenas\textunderscore .
\textunderscore Casa de saúde\textunderscore , hospital particular, em que os doentes pagam o tratamento.
\textunderscore A santa casa\textunderscore , ou \textunderscore a casa de misericórdia\textunderscore , instituição pia, para tratamento de enfermos pobres e outras obras de beneficência.
\section{Casabeque}
\begin{itemize}
\item {Grp. gram.:m.}
\end{itemize}
Casaco muito curto, para senhoras.
\section{Casaca}
\begin{itemize}
\item {Grp. gram.:f.}
\end{itemize}
\begin{itemize}
\item {Utilização:Fam.}
\end{itemize}
\begin{itemize}
\item {Grp. gram.:M.}
\end{itemize}
\begin{itemize}
\item {Utilização:Pop.}
\end{itemize}
\begin{itemize}
\item {Grp. gram.:Loc.}
\end{itemize}
\begin{itemize}
\item {Utilização:pop.}
\end{itemize}
\begin{itemize}
\item {Grp. gram.:M.}
\end{itemize}
\begin{itemize}
\item {Utilização:Pop.}
\end{itemize}
\begin{itemize}
\item {Proveniência:(Fr. \textunderscore casaque\textunderscore , do russo)}
\end{itemize}
Vestuário ceremonioso, para homem, e com abas que não chegam á frente.
Descompostura; reprehensão.
Homem encasacado, burguês asseado.
\textunderscore Cortar na casaca\textunderscore , dizer mal, murmurar.
\textunderscore Apanhar uma casaca de água\textunderscore , apanhar uma molhadela, uma pancada de água.
Patrão, dono de estabelecimento: \textunderscore o meu casaca vem cedo para a loja\textunderscore .
\section{Casaca}
\begin{itemize}
\item {Grp. gram.:m.}
\end{itemize}
\begin{itemize}
\item {Utilização:Bras}
\end{itemize}
O mesmo que \textunderscore caipira\textunderscore .
\section{Casaca-de-coiro}
\begin{itemize}
\item {Grp. gram.:m.}
\end{itemize}
\begin{itemize}
\item {Utilização:Bras}
\end{itemize}
Pássaro, amarelado por cima e pardo por baixo.
\section{Casacão}
\begin{itemize}
\item {Grp. gram.:m.}
\end{itemize}
Casaco largo de pano forte; sobretudo.
\section{Casacasa}
\begin{itemize}
\item {Grp. gram.:f.}
\end{itemize}
Árvore do Congo.
\section{Casaco}
\begin{itemize}
\item {Grp. gram.:m.}
\end{itemize}
Vestuário de homem, com mangas e abas.
Fraque.
(Cp. \textunderscore casaca\textunderscore ^1)
\section{Casada}
\begin{itemize}
\item {Grp. gram.:f.}
\end{itemize}
Jôgo de cartas, o mesmo que \textunderscore guimbarda\textunderscore .
\section{Casadeiro}
\begin{itemize}
\item {Grp. gram.:adj.}
\end{itemize}
O mesmo que \textunderscore casadoiro\textunderscore .
\section{Casado}
\begin{itemize}
\item {Grp. gram.:adj.}
\end{itemize}
Que casou; que está ligado por casamento.
\section{Casado}
\begin{itemize}
\item {Grp. gram.:m.}
\end{itemize}
\begin{itemize}
\item {Utilização:Ant.}
\end{itemize}
\begin{itemize}
\item {Proveniência:(De \textunderscore casa\textunderscore )}
\end{itemize}
Aquelle que vivia em casa sua.
Emphyteuta, que morava em casa do respectivo senhorio.
\section{Casadoiro}
\begin{itemize}
\item {Grp. gram.:adj.}
\end{itemize}
Que tem idade para casar.
Que tem tendência para o casamento.
\section{Casadouro}
\begin{itemize}
\item {Grp. gram.:adj.}
\end{itemize}
Que tem idade para casar.
Que tem tendência para o casamento.
\section{Casal}
\begin{itemize}
\item {Grp. gram.:m.}
\end{itemize}
\begin{itemize}
\item {Utilização:Techn.}
\end{itemize}
\begin{itemize}
\item {Utilização:Prov.}
\end{itemize}
\begin{itemize}
\item {Utilização:trasm.}
\end{itemize}
Pequeno povoado; lugarejo.
Conjunto de pequenas propriedades rústicas.
Par, composto de macho e fêmea, de marido e mulher.
O mesmo que \textunderscore urdidor\textunderscore .
Pequena propriedade cerrada, próxima, mas não annexa á residência do dono.
(B. lat. \textunderscore casalis\textunderscore )
\section{Casalar}
\begin{itemize}
\item {Grp. gram.:v. t.}
\end{itemize}
(V.acasalar)
\section{Casaleiro}
\begin{itemize}
\item {Grp. gram.:m.}
\end{itemize}
\begin{itemize}
\item {Grp. gram.:Adj.}
\end{itemize}
Aquelle que habita um casal.
Relativo a casal.
\section{Casalejo}
\begin{itemize}
\item {Grp. gram.:m.}
\end{itemize}
Casal pequeno.
Casa rústica e miserável. Cf. Castilho, \textunderscore Fastos\textunderscore , II, 484.
\section{Casália}
\begin{itemize}
\item {Grp. gram.:f.}
\end{itemize}
Gênero de plantas rubiáceas.
\section{Casamata}
\begin{itemize}
\item {fónica:cá}
\end{itemize}
\begin{itemize}
\item {Grp. gram.:f.}
\end{itemize}
\begin{itemize}
\item {Utilização:Fort.}
\end{itemize}
\begin{itemize}
\item {Proveniência:(It. \textunderscore casamatta\textunderscore )}
\end{itemize}
Casa ou subterrâneo com abóbada.
Bateria, que defende o fôsso.
\section{Casamatado}
\begin{itemize}
\item {Grp. gram.:adj.}
\end{itemize}
Que tem casamatas ou fórma de casamata.
\section{Casamenteiro}
\begin{itemize}
\item {Grp. gram.:adj.}
\end{itemize}
Que trata de casamentos, que faz casamentos.
Relativo a casamento; matrimonial. Cf. Rebello, \textunderscore Mocidade\textunderscore , I, 182.
\section{Casamento}
\begin{itemize}
\item {Grp. gram.:m.}
\end{itemize}
\begin{itemize}
\item {Proveniência:(De \textunderscore casar\textunderscore )}
\end{itemize}
União legítima entre homem e mulher.
Pensão annual, que certos mosteiros pagavam ás senhoras que eram descendentes dos fundadores dos mesmos mosteiros, ou que tinham comprado o padroado ou parte delle.
\section{Casanção}
\begin{itemize}
\item {Grp. gram.:m.}
\end{itemize}
(V.risanza)
\section{Casante}
\begin{itemize}
\item {Grp. gram.:m.  e  f.}
\end{itemize}
(V.nubente)
\section{Casão}
\begin{itemize}
\item {Grp. gram.:m.}
\end{itemize}
\begin{itemize}
\item {Utilização:Prov.}
\end{itemize}
Casa opulenta.
Alfaiataria, em um regimento.
Casamento rico.
\section{Casapo}
\begin{itemize}
\item {Grp. gram.:m.}
\end{itemize}
Antiga peça de artilharia.
\section{Casaquinha}
\begin{itemize}
\item {Grp. gram.:f.}
\end{itemize}
Casaco curto para senhoras.
\section{Casar}
\begin{itemize}
\item {Grp. gram.:v. t.}
\end{itemize}
\begin{itemize}
\item {Grp. gram.:V. i.}
\end{itemize}
\begin{itemize}
\item {Grp. gram.:V. p.}
\end{itemize}
\begin{itemize}
\item {Proveniência:(De \textunderscore casa\textunderscore )}
\end{itemize}
Ligar por meio de casamento; unir.
Unir-se por casamento.
Juntar-se em casamento.
Combinar-se; adaptar-se: \textunderscore o amarelo e o vermelho casam-se mal\textunderscore .
\section{Casar}
\begin{itemize}
\item {Grp. gram.:v. t.}
\end{itemize}
\begin{itemize}
\item {Utilização:Prov.}
\end{itemize}
\begin{itemize}
\item {Utilização:minh.}
\end{itemize}
Partir, quebrar. (Colhido em Barcelos)
(Por \textunderscore cassar\textunderscore . V. \textunderscore cassar\textunderscore )
\section{Casarão}
\begin{itemize}
\item {Grp. gram.:m.}
\end{itemize}
Casa grande.
Grande edifício com um só pavimento e sem divisões ou mal dividido.
\section{Casaréu}
\begin{itemize}
\item {Grp. gram.:m.}
\end{itemize}
\begin{itemize}
\item {Utilização:Prov.}
\end{itemize}
Casa grande e velha, sem condições de confôrto. Cf. Garrett, \textunderscore Camões\textunderscore , 243.
\section{Casaria}
\begin{itemize}
\item {Grp. gram.:f.}
\end{itemize}
Série de casas.
Direito real ou jugada, que correspondia proximamente á moderna contribuição de renda de casas.
\section{Casario}
\begin{itemize}
\item {Grp. gram.:m.}
\end{itemize}
O mesmo que \textunderscore casaria\textunderscore .
\section{Casarupa}
\begin{itemize}
\item {Grp. gram.:f.}
\end{itemize}
\begin{itemize}
\item {Utilização:Prov.}
\end{itemize}
\begin{itemize}
\item {Utilização:trasm.}
\end{itemize}
Casa pequena, ordinária.
\section{Casata}
\begin{itemize}
\item {Grp. gram.:f.}
\end{itemize}
\begin{itemize}
\item {Utilização:Ant.}
\end{itemize}
\begin{itemize}
\item {Proveniência:(De \textunderscore casa\textunderscore )}
\end{itemize}
Família, que vivia em certa choupana ou num grupo de choupanas. Cf. Herculano, \textunderscore Opúsc.\textunderscore , III, 306.
\section{Casaveque}
\begin{itemize}
\item {Grp. gram.:m.}
\end{itemize}
(V.casabeque)
\section{Casca}
\begin{itemize}
\item {Grp. gram.:f.}
\end{itemize}
\begin{itemize}
\item {Utilização:Fig.}
\end{itemize}
\begin{itemize}
\item {Utilização:Pop.}
\end{itemize}
\begin{itemize}
\item {Utilização:Gír.}
\end{itemize}
\begin{itemize}
\item {Utilização:Pop.}
\end{itemize}
Invólucro exterior das plantas, dos frutos, dos ovos, dos crustáceos, dos tubérculos, das sementes, etc.
Exterioridade, apparencia.
No voltarete, jôgo com as cartas que se não distribuiram.
Amuo, zanga, causada por zombaria.
Japona.
\textunderscore Dar a casca\textunderscore , morrer.
\textunderscore Dar á casca\textunderscore , dar em pantana, arruinar-se.
(Cp. cast. \textunderscore cáscara\textunderscore )
\section{Cascabulho}
\begin{itemize}
\item {Grp. gram.:m.}
\end{itemize}
\begin{itemize}
\item {Utilização:Bras}
\end{itemize}
\begin{itemize}
\item {Utilização:chul.}
\end{itemize}
\begin{itemize}
\item {Utilização:Bras. da Baía}
\end{itemize}
\begin{itemize}
\item {Utilização:Prov.}
\end{itemize}
\begin{itemize}
\item {Utilização:alg.}
\end{itemize}
Casca da glande e de várias sementes.
Monte de cascas.
Estudante de preparatórios.
Maçaroca de milho.
Mollusco bivalve, semelhante á ostra.
\section{Casca-de-anta}
\begin{itemize}
\item {Grp. gram.:f.}
\end{itemize}
Planta magnoliácea do Brasil.
\section{Casca-de-carvalho}
\begin{itemize}
\item {Grp. gram.:f.}
\end{itemize}
O mesmo que \textunderscore correia-de-inverno\textunderscore .
\section{Casca-de-caubi}
\begin{itemize}
\item {Grp. gram.:m.}
\end{itemize}
Árvore silvestre do Brasil.
\section{Casca-de-jacaré}
\begin{itemize}
\item {Grp. gram.:f.}
\end{itemize}
Árvore silvestre do Brasil.
\section{Cascal}
\begin{itemize}
\item {Grp. gram.:m.}
\end{itemize}
\begin{itemize}
\item {Proveniência:(De \textunderscore casca\textunderscore )}
\end{itemize}
Espécie de uva minhôta.
\section{Cascalense}
\begin{itemize}
\item {Grp. gram.:m.}
\end{itemize}
\begin{itemize}
\item {Proveniência:(De \textunderscore Cascal\textunderscore , supposto sing. de \textunderscore Cascaes\textunderscore , n. p.)}
\end{itemize}
Habitante de Cascaes, cascarejo.
\section{Cascalhada}
\begin{itemize}
\item {Grp. gram.:f.}
\end{itemize}
O mesmo que \textunderscore cascalheira\textunderscore .
\section{Cascalhada}
\begin{itemize}
\item {Grp. gram.:f.}
\end{itemize}
O mesmo que \textunderscore cachinada\textunderscore . Cf. Camillo, \textunderscore Corja\textunderscore , 305.
\section{Cascalhar}
\begin{itemize}
\item {Grp. gram.:v. i.}
\end{itemize}
O mesmo que \textunderscore cachinar\textunderscore . Cf. Camillo, \textunderscore Retr. de Ricard.\textunderscore , 232; \textunderscore Cavar em Ruínas\textunderscore , 223.
\section{Cascalheira}
\begin{itemize}
\item {Grp. gram.:f.}
\end{itemize}
\begin{itemize}
\item {Utilização:Prov.}
\end{itemize}
\begin{itemize}
\item {Utilização:alent.}
\end{itemize}
Lugar, em que há muito cascalho.
Terreno, formado por alluvião.
Ruído, causado pelo movimento do cascalho ou de muitos objectos miúdos.
Respiração diffícil e ruidosa; estertor.
Sítio, no leito das ribeiras, coberto de calhaus rolados e com pouca água, onde o peixe vai desovar.
Quéda de água, no Tejo.
\section{Cascalho}
\begin{itemize}
\item {Grp. gram.:m.}
\end{itemize}
\begin{itemize}
\item {Utilização:Bras}
\end{itemize}
Lascas de pedra; pedra britada.
Mistura de areia, seixos e cascas de crustáceos.
Escórias de ferro.
Alluviões auríferas ou diamantinas.
(Cp. cast. \textunderscore cascajo\textunderscore )
\section{Cascalhoso}
\begin{itemize}
\item {Grp. gram.:adj.}
\end{itemize}
O mesmo ou melhor que \textunderscore cascalhudo\textunderscore .
\section{Cascalhudo}
\begin{itemize}
\item {Grp. gram.:adj.}
\end{itemize}
Em que há muito cascalho; que tem cascalho.
\section{Cascalvo}
\begin{itemize}
\item {Grp. gram.:adj.}
\end{itemize}
\begin{itemize}
\item {Proveniência:(De \textunderscore casco\textunderscore  + \textunderscore alvo\textunderscore )}
\end{itemize}
Que tem cascos brancos.
\section{Cascalvo}
\begin{itemize}
\item {Grp. gram.:adj.}
\end{itemize}
\begin{itemize}
\item {Proveniência:(De \textunderscore casca\textunderscore  + \textunderscore alvo\textunderscore )}
\end{itemize}
Diz-se de uma variedade de trigo.
\section{Cascamurro}
\begin{itemize}
\item {Grp. gram.:adj.}
\end{itemize}
\begin{itemize}
\item {Utilização:Prov.}
\end{itemize}
\begin{itemize}
\item {Utilização:beir.}
\end{itemize}
O mesmo que \textunderscore casmurro\textunderscore .
\section{Cascão}
\begin{itemize}
\item {Grp. gram.:m.}
\end{itemize}
Casca grande ou crosta.
Camada pedregosa, ainda não petrificada.
Crosta de sujidade, na pelle de alguém.
Crosta de ferida, bostela.
Lagem tôsca, mais ou menos quadrangular.
Bôa casca, no jôgo do voltarete; o mesmo que \textunderscore cascarrão\textunderscore .
\section{Cascar}
\begin{itemize}
\item {Grp. gram.:v. t.  e  i.}
\end{itemize}
Bater, dar pancadas.
Retorquir com azedume.
(Cast. \textunderscore cascar\textunderscore )
\section{Cascar}
\begin{itemize}
\item {Grp. gram.:v. t.}
\end{itemize}
Tirar a casca a, descascar.
\section{Cáscara}
\begin{itemize}
\item {Grp. gram.:f.}
\end{itemize}
Cobre em bruto.
(Cast. \textunderscore cáscara\textunderscore )
\section{Cascarada}
\begin{itemize}
\item {Grp. gram.:f.}
\end{itemize}
\begin{itemize}
\item {Utilização:Prov.}
\end{itemize}
\begin{itemize}
\item {Utilização:trasm.}
\end{itemize}
O mesmo que \textunderscore cascalhada\textunderscore ^1.
\section{Cáscara-sagrada}
\begin{itemize}
\item {Grp. gram.:f.}
\end{itemize}
Medicamento laxatico.
\section{Cascarejo}
\begin{itemize}
\item {Grp. gram.:m.}
\end{itemize}
Habitante de Cascaes.
\section{Cascareta}
\begin{itemize}
\item {fónica:carê}
\end{itemize}
\begin{itemize}
\item {Grp. gram.:f.}
\end{itemize}
\begin{itemize}
\item {Utilização:T. de Turquel}
\end{itemize}
Camada corticál da casca do pinheiro; corcódea.
\section{Cascaria}
\begin{itemize}
\item {Grp. gram.:f.}
\end{itemize}
Conjunto de vasilhas para vinho.
Os cascos dos pés dos animaes:«\textunderscore ah! elle roía as unhas? é preciso que tenha bôa cascaria, para estar sempre abastecido de tal vitualha\textunderscore ». Camillo, \textunderscore Mulher Fatal\textunderscore , 35; \textunderscore Críticos do Canc.\textunderscore , 33.
\section{Cascariate}
\begin{itemize}
\item {Grp. gram.:m.}
\end{itemize}
\begin{itemize}
\item {Utilização:Ant.}
\end{itemize}
Fôro de casas e hortas, que se pagava em Baçaim.
\section{Cascarilha}
\begin{itemize}
\item {Grp. gram.:f.}
\end{itemize}
Planta euphorbiácea do Brasil, applicada contra doenças venéreas.
(Cp. \textunderscore cascarrilha\textunderscore ^1)
\section{Cascarina}
\begin{itemize}
\item {Grp. gram.:f.}
\end{itemize}
Princípio activo, extrahido da cáscara.
\section{Cascarna}
\begin{itemize}
\item {Grp. gram.:f.}
\end{itemize}
\begin{itemize}
\item {Utilização:Prov.}
\end{itemize}
\begin{itemize}
\item {Utilização:trasm.}
\end{itemize}
O mesmo que \textunderscore carranha\textunderscore .
\section{Cascarnoso}
\begin{itemize}
\item {Grp. gram.:adj.}
\end{itemize}
\begin{itemize}
\item {Utilização:Prov.}
\end{itemize}
\begin{itemize}
\item {Utilização:trasm.}
\end{itemize}
Que tem \textunderscore cascarna\textunderscore .
\section{Cascaroleta}
\begin{itemize}
\item {fónica:lê}
\end{itemize}
\begin{itemize}
\item {Grp. gram.:f.}
\end{itemize}
\begin{itemize}
\item {Utilização:Prov.}
\end{itemize}
\begin{itemize}
\item {Utilização:trasm.}
\end{itemize}
Rapariga, que está sempre a rir-se, ainda que não tenha de quê.
\section{Cascaroso}
\begin{itemize}
\item {Grp. gram.:adj.}
\end{itemize}
Que tem casca grossa.
\section{Cascarra}
\begin{itemize}
\item {Grp. gram.:f.}
\end{itemize}
Casca, ou as 13 cartas que ficam por distribuir, no voltarete e noutros jogos.
Peixe maritimo, semelhante ao cação.
\section{Cascarrão}
\begin{itemize}
\item {Grp. gram.:m.}
\end{itemize}
\begin{itemize}
\item {Utilização:Pop.}
\end{itemize}
\begin{itemize}
\item {Utilização:Prov.}
\end{itemize}
Grande casca, amuo, zanga.
Variedade de nozes de casca grossa.
\section{Cascarrão}
\begin{itemize}
\item {Grp. gram.:m.}
\end{itemize}
(Corr. de \textunderscore carrascão\textunderscore )
\section{Cascarrear}
\begin{itemize}
\item {Grp. gram.:v. i.}
\end{itemize}
\begin{itemize}
\item {Utilização:Prov.}
\end{itemize}
\begin{itemize}
\item {Utilização:beir.}
\end{itemize}
Proceder incorrectamente.
\section{Cascarreia}
\begin{itemize}
\item {Grp. gram.:f.}
\end{itemize}
\begin{itemize}
\item {Utilização:ant.}
\end{itemize}
\begin{itemize}
\item {Utilização:Chul.}
\end{itemize}
O mesmo que \textunderscore raça\textunderscore ^1.
\section{Cascarrilha}
\begin{itemize}
\item {Grp. gram.:f.}
\end{itemize}
Casca de várias árvores euphorbiáceas.
Espécie de quina.
(Cast. \textunderscore cascarilla\textunderscore )
\section{Cascarrilha}
\begin{itemize}
\item {Grp. gram.:f.}
\end{itemize}
O mesmo que \textunderscore cascarra\textunderscore , no jôgo.
\section{Cascas-de-honduras}
\begin{itemize}
\item {Grp. gram.:f. pl.}
\end{itemize}
Planta rutácea medicinal, (\textunderscore cascara amara\textunderscore ).
\section{Cascata}
\begin{itemize}
\item {Grp. gram.:f.}
\end{itemize}
\begin{itemize}
\item {Utilização:Pop.}
\end{itemize}
\begin{itemize}
\item {Proveniência:(It. \textunderscore cascata\textunderscore , de \textunderscore cascare\textunderscore , cair)}
\end{itemize}
Quéda de águas por entre pedras em escalão.
Construcção tôsca de penedos e conchas, ou só de penedos, imitando fragas, em que a água se despenha.
Mulher velha, feia e pretensiosa.
\section{Cascatear}
\begin{itemize}
\item {Grp. gram.:v. t.}
\end{itemize}
\begin{itemize}
\item {Utilização:bras}
\end{itemize}
\begin{itemize}
\item {Utilização:Neol.}
\end{itemize}
Formar cascata.
\section{Cascavel}
\begin{itemize}
\item {Grp. gram.:m.}
\end{itemize}
\begin{itemize}
\item {Utilização:Fig.}
\end{itemize}
\begin{itemize}
\item {Utilização:Bras}
\end{itemize}
\begin{itemize}
\item {Grp. gram.:Adj.}
\end{itemize}
Guizo.
Cabeça leve, pouco juizo.
Bagatela.
Serpente venenosa, cuja cauda, agitando-se, produz um som especial.
Espécie de rôla que, ao soltar as asas, produz certo ruído, como que um estalo.
Volúvel.
\section{Cascaveleira}
\begin{itemize}
\item {Grp. gram.:f.}
\end{itemize}
\begin{itemize}
\item {Utilização:Bras}
\end{itemize}
Planta solânea americana, de uma só fôlha.
\section{Casceta}
\begin{itemize}
\item {fónica:cê}
\end{itemize}
\begin{itemize}
\item {Grp. gram.:f.}
\end{itemize}
Planta da serra de Sintra.
\section{Casco}
\begin{itemize}
\item {Grp. gram.:m.}
\end{itemize}
\begin{itemize}
\item {Utilização:Bras}
\end{itemize}
\begin{itemize}
\item {Grp. gram.:Pl.}
\end{itemize}
\begin{itemize}
\item {Utilização:Fig.}
\end{itemize}
Casca.
Crânio.
Pelle da cabeça.
Quilha e costado da embarcação.
Vasilha de aduelas.
Paredes para construcção.
Unha de vários pachidermes.
Antiga armadura para a cabeça.
Armação (de chapéu de senhora, etc.).
Núcleo.
Barquinho, feito de peça inteiriça de madeira, cavada a fogo.
Massa dos ingredientes de pólvora, moída e tornada depois compacta, por meio de pressão.
Intelligência.
(Cp. \textunderscore casca\textunderscore )
\section{Casco}
\begin{itemize}
\item {Grp. gram.:m.}
\end{itemize}
\begin{itemize}
\item {Utilização:Prov.}
\end{itemize}
\begin{itemize}
\item {Utilização:trasm.}
\end{itemize}
Entremez, comédia ligeira.
\section{Cascol}
\begin{itemize}
\item {Grp. gram.:m.}
\end{itemize}
Resina de uma árvore americana, com que se faz uma espécie de lacre preto.
(Cast. \textunderscore cascol\textunderscore )
\section{Cascoso}
\begin{itemize}
\item {Grp. gram.:adj.}
\end{itemize}
\begin{itemize}
\item {Proveniência:(De \textunderscore casca\textunderscore )}
\end{itemize}
Que tem casca grossa; o mesmo que \textunderscore cascudo\textunderscore ^1.
\section{Cascoso}
\begin{itemize}
\item {Grp. gram.:adj.}
\end{itemize}
Relativo aos cascos dos animaes.
Que tem grandes cascos: \textunderscore cavallo cascoso\textunderscore .
\section{Cascudo}
\begin{itemize}
\item {Grp. gram.:adj.}
\end{itemize}
\begin{itemize}
\item {Grp. gram.:M.}
\end{itemize}
\begin{itemize}
\item {Utilização:Bras}
\end{itemize}
\begin{itemize}
\item {Utilização:Bras}
\end{itemize}
\begin{itemize}
\item {Utilização:Fam.}
\end{itemize}
\begin{itemize}
\item {Utilização:Bras. do N}
\end{itemize}
\begin{itemize}
\item {Proveniência:(De \textunderscore casca\textunderscore )}
\end{itemize}
Que tem casca grossa ou pelle dura.
Árvore silvestre.
Eleitor conservador, ou retrógrado.
Nome de um peixe.
Variedade de uva da Bairrada.
Designação genérica de varios insectos coleópteros.
\section{Cascudo}
\begin{itemize}
\item {Grp. gram.:m.}
\end{itemize}
\begin{itemize}
\item {Proveniência:(De \textunderscore casco\textunderscore )}
\end{itemize}
Carolo, pancada na cabeça, especialmente com as costas da mão.
\section{Casculho}
\begin{itemize}
\item {Grp. gram.:m.}
\end{itemize}
(V.cascabulho)
\section{Casculho}
\begin{itemize}
\item {Grp. gram.:m.}
\end{itemize}
\begin{itemize}
\item {Proveniência:(De \textunderscore casco\textunderscore )}
\end{itemize}
Casta de uva preta do Doiro.
\section{Cascunhar}
\begin{itemize}
\item {Grp. gram.:v. t.  e  i.}
\end{itemize}
\begin{itemize}
\item {Utilização:ant.}
\end{itemize}
\begin{itemize}
\item {Utilização:Gír.}
\end{itemize}
Vêr, olhar.
\section{Caseação}
\begin{itemize}
\item {Grp. gram.:f.}
\end{itemize}
\begin{itemize}
\item {Proveniência:(Do lat. \textunderscore caseus\textunderscore )}
\end{itemize}
Transformação do leite em queijo.
\section{Caseadeira}
\begin{itemize}
\item {Grp. gram.:f.}
\end{itemize}
Mulher que caseia.
\section{Caseado}
\begin{itemize}
\item {Grp. gram.:m.}
\end{itemize}
Acto ou effeito de \textunderscore casear\textunderscore .
\section{Casear}
\begin{itemize}
\item {Grp. gram.:v. t.  e  i.}
\end{itemize}
\begin{itemize}
\item {Utilização:Ant.}
\end{itemize}
Abrir casas e ponteá-las para os botões do vestuário.
Fazer moradas de casas em. Cf. G. Viana, \textunderscore Apostilas\textunderscore , vb. \textunderscore casa\textunderscore .
\section{Casebeque}
\begin{itemize}
\item {Grp. gram.:m.}
\end{itemize}
(V.casabeque)
\section{Casebre}
\begin{itemize}
\item {Grp. gram.:m.}
\end{itemize}
\begin{itemize}
\item {Proveniência:(De \textunderscore casa\textunderscore )}
\end{itemize}
Casa pequena e velha ou em ruínas; pardieiro; tugúrio.
\section{Caseco}
\begin{itemize}
\item {Grp. gram.:m.}
\end{itemize}
O mesmo que \textunderscore caseque\textunderscore .
\section{Caseiforme}
\begin{itemize}
\item {Grp. gram.:adj.}
\end{itemize}
\begin{itemize}
\item {Proveniência:(Do lat. \textunderscore caseus\textunderscore  + \textunderscore forma\textunderscore )}
\end{itemize}
Que tem aspecto de queijo.
\section{Caseína}
\begin{itemize}
\item {Grp. gram.:f.}
\end{itemize}
\begin{itemize}
\item {Proveniência:(Do lat. \textunderscore caseus\textunderscore )}
\end{itemize}
Princípio alcalino, que se encontra no leite e em algumas plantas.
\section{Caçala}
\begin{itemize}
\item {Grp. gram.:f.}
\end{itemize}
Arbusto angolense, de frutos ovoides e carminados.
\section{Caçanges}
\begin{itemize}
\item {Grp. gram.:m. pl.}
\end{itemize}
Tríbo independente, a léste de Malange, em África.
\section{Caçapuile}
\begin{itemize}
\item {Grp. gram.:m.}
\end{itemize}
Pássaro angolense, que se alimenta de peixes, (\textunderscore halcion semicaerulea\textunderscore ).
\section{Caçatinga}
\begin{itemize}
\item {Grp. gram.:f.}
\end{itemize}
Planta solânea do Brasil.
\section{Caceço}
\begin{itemize}
\item {Grp. gram.:m.}
\end{itemize}
\begin{itemize}
\item {Utilização:T. de Angola}
\end{itemize}
O mesmo que \textunderscore machado\textunderscore . Cf. Capello e Ivens, I, 135.
\section{Caseira}
\begin{itemize}
\item {Grp. gram.:f.}
\end{itemize}
Mulher, que dirige a cultura de uma quinta ou herdade.
Mulher de caseiro.
(Cp. \textunderscore caseiro\textunderscore )
\section{Caseira}
\begin{itemize}
\item {Grp. gram.:f.}
\end{itemize}
\begin{itemize}
\item {Utilização:Bras}
\end{itemize}
\begin{itemize}
\item {Proveniência:(De \textunderscore casa\textunderscore )}
\end{itemize}
Concubina.
\section{Caseiro}
\begin{itemize}
\item {Grp. gram.:adj.}
\end{itemize}
\begin{itemize}
\item {Utilização:Fig.}
\end{itemize}
\begin{itemize}
\item {Grp. gram.:M.}
\end{itemize}
\begin{itemize}
\item {Proveniência:(Lat. \textunderscore casarius\textunderscore )}
\end{itemize}
Relativo a casa.
Que se usa em família ou dentro de casa: \textunderscore fato caseiro\textunderscore .
Que vive quási sempre dentro de casa.
Modesto, simples.
Inquilino, arrendatário de um casal.
Aquelle que dirige a cultura de quinta ou herdade.
\section{Casela}
\begin{itemize}
\item {Grp. gram.:f.}
\end{itemize}
\begin{itemize}
\item {Utilização:Pop.}
\end{itemize}
\begin{itemize}
\item {Proveniência:(De \textunderscore casa\textunderscore )}
\end{itemize}
Cada um dos quadrados, no jôgo do homem.
\section{Caseoso}
\begin{itemize}
\item {Grp. gram.:adj.}
\end{itemize}
\begin{itemize}
\item {Proveniência:(Do lat. \textunderscore caseus\textunderscore )}
\end{itemize}
Que tem a natureza do queijo.
\section{Caseque}
\begin{itemize}
\item {Grp. gram.:m.}
\end{itemize}
Árvore angolense, no Duque-de-Bragança.
\section{Caseria}
\begin{itemize}
\item {Grp. gram.:f.}
\end{itemize}
\begin{itemize}
\item {Utilização:Ant.}
\end{itemize}
Hospedaria na Terra-Santa.
(Por \textunderscore casaria\textunderscore , de \textunderscore casa\textunderscore )
\section{Caserna}
\begin{itemize}
\item {Grp. gram.:f.}
\end{itemize}
\begin{itemize}
\item {Proveniência:(Fr. \textunderscore caserne\textunderscore )}
\end{itemize}
Habitação de soldados, dentro de um quartel ou de uma praça.
Quartel.
\section{Caserneiro}
\begin{itemize}
\item {Grp. gram.:m.}
\end{itemize}
Aquelle que trata das casernas.
\section{Caseró}
\begin{itemize}
\item {Grp. gram.:m.}
\end{itemize}
O mesmo que \textunderscore caró\textunderscore .
\section{Cásia}
\begin{itemize}
\item {Grp. gram.:f.}
\end{itemize}
\begin{itemize}
\item {Proveniência:(Lat. \textunderscore casia\textunderscore )}
\end{itemize}
Casca aromática, muito semelhante á canela.
Canela silvestre.
\section{Casibeque}
\begin{itemize}
\item {Grp. gram.:m.}
\end{itemize}
O mesmo que \textunderscore casabeque\textunderscore . Cf. Camillo, \textunderscore Eus. Macário\textunderscore .
\section{Casimira}
\begin{itemize}
\item {Grp. gram.:f.}
\end{itemize}
Pano de lan, fino e leve.
(Talvez corr. de \textunderscore cachemira\textunderscore )
\section{Casimireta}
\begin{itemize}
\item {fónica:mirê}
\end{itemize}
\begin{itemize}
\item {Grp. gram.:f.}
\end{itemize}
Casimira, de qualidade inferior. Cf. \textunderscore Inquér. Ind.\textunderscore , p. III, 164.
\section{Casinha}
\begin{itemize}
\item {Grp. gram.:f.}
\end{itemize}
\begin{itemize}
\item {Utilização:Pop.}
\end{itemize}
\begin{itemize}
\item {Utilização:Ant.}
\end{itemize}
\begin{itemize}
\item {Utilização:Prov.}
\end{itemize}
\begin{itemize}
\item {Utilização:alent.}
\end{itemize}
\begin{itemize}
\item {Proveniência:(De \textunderscore casa\textunderscore )}
\end{itemize}
Casa pequena.
Latrina.
Pôsto fiscal, junto ás barreiras.
Casa de almotacel.
Cárcere inquisitorial.
Casa, em que se fazia o despacho de mercês.
Casa, onde se reúnem e dormem os ganhões, como soldados em caserna.
\section{Casinhola}
\begin{itemize}
\item {Grp. gram.:f.}
\end{itemize}
\begin{itemize}
\item {Utilização:Pop.}
\end{itemize}
\begin{itemize}
\item {Utilização:Prov.}
\end{itemize}
\begin{itemize}
\item {Utilização:alent.}
\end{itemize}
\begin{itemize}
\item {Proveniência:(De \textunderscore casinha\textunderscore )}
\end{itemize}
Casa pequena e pobre.
Cêsto, para postura de ovos.
\section{Casinholo}
\begin{itemize}
\item {fónica:nhô}
\end{itemize}
\begin{itemize}
\item {Grp. gram.:m.}
\end{itemize}
O mesmo que \textunderscore casinhola\textunderscore .
\section{Casinhota}
\begin{itemize}
\item {Grp. gram.:f.}
\end{itemize}
O mesmo que \textunderscore casinhoto\textunderscore .
\section{Casinhoto}
\begin{itemize}
\item {fónica:nhô}
\end{itemize}
\begin{itemize}
\item {Grp. gram.:m.}
\end{itemize}
O mesmo que \textunderscore casinhola\textunderscore .
\section{Casino}
\begin{itemize}
\item {Grp. gram.:m.}
\end{itemize}
\begin{itemize}
\item {Utilização:Neol.}
\end{itemize}
\begin{itemize}
\item {Proveniência:(It. \textunderscore casino\textunderscore , de \textunderscore casa\textunderscore )}
\end{itemize}
Casa, ou lugar de reunião, para jogar, dançar, lêr, etc.
\section{Casmanhato}
\begin{itemize}
\item {Grp. gram.:m.}
\end{itemize}
Gênero de crustáceos decápodes.
(Cast. \textunderscore casmañato\textunderscore )
\section{Casmurrada}
\begin{itemize}
\item {Grp. gram.:f.}
\end{itemize}
Acto ou dito de casmurro. Cf. Filinto, II, 4.
\section{Casmurral}
\begin{itemize}
\item {Grp. gram.:adj.}
\end{itemize}
Próprio de casmurro. Cf. Filinto, VIII, 55 e 205.
\section{Casmurrice}
\begin{itemize}
\item {Grp. gram.:f.}
\end{itemize}
Qualidade de quem é casmurro.
\section{Casmurro}
\begin{itemize}
\item {Grp. gram.:m.  e  adj.}
\end{itemize}
Aquelle que é teimoso, cabeçudo.
Triste, sorumbático.
\section{Caso}
\begin{itemize}
\item {Grp. gram.:m.}
\end{itemize}
\begin{itemize}
\item {Utilização:Gram.}
\end{itemize}
\begin{itemize}
\item {Proveniência:(Lat. \textunderscore casus\textunderscore )}
\end{itemize}
Acontecimento, facto: \textunderscore os casos do dia\textunderscore .
Hypóthese, circunstância: \textunderscore no caso de se portar bem...\textunderscore 
Difficuldade.
Importância.
Acaso.
Estima: \textunderscore faz caso da pobreza\textunderscore .
Manifestação individual de uma doença: \textunderscore houve hoje quatro casos de peste\textunderscore .
Desinência de nomes e pronomes, para lhes designar a relação syntáctica, em algumas línguas: \textunderscore o nominativo é caso directo\textunderscore .
\section{Casoar}
\begin{itemize}
\item {Grp. gram.:m.}
\end{itemize}
\begin{itemize}
\item {Proveniência:(Fr. \textunderscore casoar\textunderscore )}
\end{itemize}
Ave pernalta.
\section{Casoarina}
\begin{itemize}
\item {Grp. gram.:f.}
\end{itemize}
\begin{itemize}
\item {Proveniência:(De \textunderscore casoar\textunderscore , pela semelhança das pennas desta ave com as fôlhas daquella planta)}
\end{itemize}
Árvore americana, de aspecto triste, e que parece gemer com o vento.
\section{Casola}
\begin{itemize}
\item {Grp. gram.:f.}
\end{itemize}
Aselha ou presilha para botão. Cf. \textunderscore Hyssope\textunderscore , V.
\section{Casona}
\begin{itemize}
\item {Grp. gram.:f.}
\end{itemize}
\begin{itemize}
\item {Utilização:Ant.}
\end{itemize}
Grande várzea, semeada de arroz.
(Provavelmente, de \textunderscore casa\textunderscore , por allusão á quantidade de arroz que enchesse uma casa)
\section{Casório}
\begin{itemize}
\item {Grp. gram.:m.}
\end{itemize}
\begin{itemize}
\item {Utilização:Chul.}
\end{itemize}
\begin{itemize}
\item {Proveniência:(De \textunderscore casar\textunderscore )}
\end{itemize}
Casamento.
\section{Casota}
\begin{itemize}
\item {Grp. gram.:f.}
\end{itemize}
O mesmo que \textunderscore casinhola\textunderscore .
Guarita de cão.
\section{Caspa}
\begin{itemize}
\item {Grp. gram.:f.}
\end{itemize}
\begin{itemize}
\item {Proveniência:(T. cast.)}
\end{itemize}
Escamas, que se criam na pelle da cabeça ou em qualquer outra parte da epiderme; carepa.
\section{Caspacho}
\begin{itemize}
\item {Grp. gram.:m.}
\end{itemize}
\begin{itemize}
\item {Utilização:Prov.}
\end{itemize}
Iguaria, composta de pedaços de pão em água fria, com azeite, vinagre, salsa e outros temperos.
(Cast. \textunderscore gaspacho\textunderscore )
\section{Caspento}
\begin{itemize}
\item {Grp. gram.:adj.}
\end{itemize}
O mesmo que \textunderscore casposo\textunderscore .
\section{Caspilra}
\begin{itemize}
\item {Grp. gram.:f.}
\end{itemize}
\begin{itemize}
\item {Utilização:Prov.}
\end{itemize}
\begin{itemize}
\item {Utilização:trasm.}
\end{itemize}
Mulher magra, ordinária e mal humorada.
Animal pequeno e reles.
\section{Cáspitè!}
\begin{itemize}
\item {Grp. gram.:interj.}
\end{itemize}
(indicativa de admiração, com um pouco de ironia)
(Cp. cast. \textunderscore caspita!\textunderscore )
\section{Casposo}
\begin{itemize}
\item {Grp. gram.:adj.}
\end{itemize}
Que cria caspa, que tem caspa.
\section{Casqueira}
\begin{itemize}
\item {Grp. gram.:f.}
\end{itemize}
Fasquia de madeira, o mesmo que \textunderscore costaneira\textunderscore .
\section{Casqueiro}
\begin{itemize}
\item {Grp. gram.:m.}
\end{itemize}
\begin{itemize}
\item {Utilização:Prov.}
\end{itemize}
\begin{itemize}
\item {Utilização:trasm.}
\end{itemize}
\begin{itemize}
\item {Utilização:Prov.}
\end{itemize}
\begin{itemize}
\item {Utilização:minh.}
\end{itemize}
\begin{itemize}
\item {Utilização:alent.}
\end{itemize}
\begin{itemize}
\item {Utilização:Prov.}
\end{itemize}
\begin{itemize}
\item {Utilização:trasm.}
\end{itemize}
\begin{itemize}
\item {Proveniência:(De \textunderscore casca\textunderscore )}
\end{itemize}
Lugar, em que se descasca e falqueja a madeira, para sêr serrada.
Aquelle que falqueja madeira.
Tanque, em que se tingem redes de pesca.
Tabuão da borda de um tronco que se serra em tábuas.
Pão de trigo, que se distribue aos soldados.
Côdea de pão.
\section{Casqueiro}
\begin{itemize}
\item {Grp. gram.:m.}
\end{itemize}
\begin{itemize}
\item {Utilização:Bras. do S}
\end{itemize}
\begin{itemize}
\item {Proveniência:(De \textunderscore casco\textunderscore )}
\end{itemize}
Aquelle que nivela os cascos dos animaes, para a ferragem.
\section{Casquejar}
\begin{itemize}
\item {Grp. gram.:v. i.}
\end{itemize}
Cicatrizar, criar novo casco.
\section{Casquento}
\begin{itemize}
\item {Grp. gram.:adj.}
\end{itemize}
O mesmo que \textunderscore cascudo\textunderscore ^1.
\section{Casquete}
\begin{itemize}
\item {fónica:quê}
\end{itemize}
\begin{itemize}
\item {Grp. gram.:m.}
\end{itemize}
Pequena cobertura de cabeça.
Barrete, carapuça.
Chapéu velho.
(Talvez do fr. \textunderscore casquette\textunderscore , boné)
\section{Casquibrando}
\begin{itemize}
\item {fónica:cás}
\end{itemize}
\begin{itemize}
\item {Grp. gram.:adj.}
\end{itemize}
\begin{itemize}
\item {Proveniência:(De \textunderscore casco\textunderscore  + \textunderscore brando\textunderscore )}
\end{itemize}
Que tem os cascos macios.
\section{Casquicheio}
\begin{itemize}
\item {fónica:cás}
\end{itemize}
\begin{itemize}
\item {Grp. gram.:adj.}
\end{itemize}
\begin{itemize}
\item {Utilização:Fig.}
\end{itemize}
Diz-se do cavallo que tem casco cheio.
Presumpçoso. Cf. \textunderscore Aulegrafia\textunderscore , 1.
\section{Casquilha}
\begin{itemize}
\item {Grp. gram.:f.}
\end{itemize}
Pequena casca; pedaço de casca.
\section{Casquilhada}
\begin{itemize}
\item {Grp. gram.:f.}
\end{itemize}
Porção de casquilhos; os casquilhos.
\section{Casquilhagem}
\begin{itemize}
\item {Grp. gram.:f.}
\end{itemize}
Casquilhice.
Os casquilhos.
\section{Casquilhar}
\begin{itemize}
\item {Grp. gram.:v. i.}
\end{itemize}
Andar casquilho.
\section{Casquilharia}
\begin{itemize}
\item {Grp. gram.:f.}
\end{itemize}
Enfeites, traje de casquilho.
\section{Casquilhice}
\begin{itemize}
\item {Grp. gram.:f.}
\end{itemize}
O mesmo que \textunderscore casquilharia\textunderscore .
\section{Casquilho}
\begin{itemize}
\item {Grp. gram.:adj.}
\end{itemize}
\begin{itemize}
\item {Grp. gram.:M.}
\end{itemize}
\begin{itemize}
\item {Proveniência:(De \textunderscore casca\textunderscore )}
\end{itemize}
Que se enfeita exaggeradamente; taful.
Aquelle que se enfeita, tratando muito da sua apparência; janota, peralta.
Cylindro ôco e metállico, que remata as lanças dos carros e outros objectos.
Ave aquática, (\textunderscore oceanites oceanica\textunderscore , Kuhli.).
\section{Casquilhório}
\begin{itemize}
\item {Grp. gram.:adj.}
\end{itemize}
Ridiculamente casquilho. Cf. Filinto, VIII, 187.
\section{Casquimole}
\begin{itemize}
\item {fónica:cás}
\end{itemize}
\begin{itemize}
\item {Grp. gram.:adj.}
\end{itemize}
Que tem os cascos moles.
\section{Casquimolle}
\begin{itemize}
\item {fónica:cás}
\end{itemize}
\begin{itemize}
\item {Grp. gram.:adj.}
\end{itemize}
Que tem os cascos molles.
\section{Casquinada}
\begin{itemize}
\item {Grp. gram.:f.}
\end{itemize}
\begin{itemize}
\item {Proveniência:(De \textunderscore casquinar\textunderscore )}
\end{itemize}
Gargalhada infantil; o mesmo que \textunderscore cachinada\textunderscore . Cf. Camillo, \textunderscore Corja\textunderscore , 208; \textunderscore Cancion.\textunderscore , Al., VIII; \textunderscore Doze Casamentos\textunderscore , 164.
\section{Casquinar}
\begin{itemize}
\item {Grp. gram.:v. i.}
\end{itemize}
Soltar pequenas risadas successivas.
Rir, escarnecendo. Cf. Camillo, \textunderscore Mulher Fatal\textunderscore , 35; \textunderscore Brasileira\textunderscore , 169.
(Cp. \textunderscore cachinar\textunderscore )
\section{Casquinha}
\begin{itemize}
\item {Grp. gram.:f.}
\end{itemize}
\begin{itemize}
\item {Utilização:Des.}
\end{itemize}
Pequena casca.
Madeira de pinho de Flandres.
Fôlha delgada de metal precioso, que reveste obra de metal ordinário.
Pequeno barco de pesca, na costa de Cabo-Delgado.
Cidrão; doce, feito de talhadas de cidra:«\textunderscore trazem madeira\textunderscore », \textunderscore em salva a casquinha gulosa e delicada da selvosa\textunderscore . Garrett, \textunderscore Camões\textunderscore .
\section{Casquinheiro}
\begin{itemize}
\item {Grp. gram.:m.}
\end{itemize}
Official, que reveste de casquinha de metal precioso metaes de somenos valor.
\section{Casquinho}
\begin{itemize}
\item {Grp. gram.:adj.}
\end{itemize}
\begin{itemize}
\item {Proveniência:(De \textunderscore casco\textunderscore )}
\end{itemize}
Diz-se do cavallo, cujo casco facilmente se encrava.
\section{Casquiseco}
\begin{itemize}
\item {fónica:sê}
\end{itemize}
\begin{itemize}
\item {Grp. gram.:adj.}
\end{itemize}
Diz-se do cavallo, que tem os cascos secos.
\section{Casquisseco}
\begin{itemize}
\item {Grp. gram.:adj.}
\end{itemize}
Diz-se do cavallo, que tem os cascos secos.
\section{Cassa}
\begin{itemize}
\item {Grp. gram.:f.}
\end{itemize}
Tecido transparente de algodão ou linho.
\section{Cassa-caniza}
\begin{itemize}
\item {Grp. gram.:f.}
\end{itemize}
Árvore angolense, de fôlhas coriáceas e pequenas flôres inodoras, em espigas.
\section{Cassa-co-ripata}
\begin{itemize}
\item {Grp. gram.:f.}
\end{itemize}
Elegante arbusto angolense, cujos ramos, depois de macerados, são pelos indigenas applicados contra os herpes.
\section{Cassala}
\begin{itemize}
\item {Grp. gram.:f.}
\end{itemize}
Arbusto angolense, de frutos ovoides e carminados.
\section{Cassanges}
\begin{itemize}
\item {Grp. gram.:m. pl.}
\end{itemize}
Tríbo independente, a léste de Malange, em África.
\section{Cassapuile}
\begin{itemize}
\item {Grp. gram.:m.}
\end{itemize}
Pássaro angolense, que se alimenta de peixes, (\textunderscore halcion semicaerulea\textunderscore ).
\section{Cassar}
\begin{itemize}
\item {Grp. gram.:v. t.}
\end{itemize}
\begin{itemize}
\item {Utilização:Ant.}
\end{itemize}
\begin{itemize}
\item {Proveniência:(Lat. \textunderscore quassare\textunderscore )}
\end{itemize}
Tornar nullo, sem effeito.
Quebrar.
\section{Cassatinga}
\begin{itemize}
\item {Grp. gram.:f.}
\end{itemize}
Planta solânea do Brasil.
\section{Casse}
\begin{itemize}
\item {Grp. gram.:m.}
\end{itemize}
Peça de madeira, mais ou menos curva, que vai da sobrecadeia ao tendal do carro.
\section{Cassear}
\begin{itemize}
\item {Grp. gram.:v. i.}
\end{itemize}
\begin{itemize}
\item {Utilização:Náut.}
\end{itemize}
Mudar de rumo; garrar. Cf. Filinto, \textunderscore D. Man.\textunderscore , II, 279.
\section{Cassesso}
\begin{itemize}
\item {Grp. gram.:m.}
\end{itemize}
\begin{itemize}
\item {Utilização:T. de Angola}
\end{itemize}
O mesmo que \textunderscore machado\textunderscore . Cf. Capello e Ivens, I, 135.
\section{Casse-tete}
\begin{itemize}
\item {fónica:té}
\end{itemize}
\begin{itemize}
\item {Grp. gram.:m.}
\end{itemize}
\begin{itemize}
\item {Proveniência:(T. fr.)}
\end{itemize}
Cacete.
\section{Cássia}
\begin{itemize}
\item {Grp. gram.:f.}
\end{itemize}
O mesmo que \textunderscore cásia\textunderscore .
\section{Cassiabeira}
\begin{itemize}
\item {Grp. gram.:f.}
\end{itemize}
Gênero de fêtos.
\section{Cassiáceas}
\begin{itemize}
\item {Grp. gram.:f. pl.}
\end{itemize}
\begin{itemize}
\item {Proveniência:(De \textunderscore cássia\textunderscore )}
\end{itemize}
Família de plantas leguminosas.
\section{Cássia-fístula}
\begin{itemize}
\item {Grp. gram.:f.}
\end{itemize}
O mesmo que \textunderscore canafístula\textunderscore .
\section{Cassico}
\begin{itemize}
\item {Grp. gram.:m.}
\end{itemize}
Pássaro conirostro da América, (\textunderscore cassicus\textunderscore ).
\section{Cássida}
\begin{itemize}
\item {Grp. gram.:f.}
\end{itemize}
\begin{itemize}
\item {Proveniência:(Lat. \textunderscore cassida\textunderscore )}
\end{itemize}
Insecto coleóptero, que se encontra sôbre a hortelan.
\section{Cassidária}
\begin{itemize}
\item {Grp. gram.:f.}
\end{itemize}
Gênero de molluscos.
\section{Cassidários}
\begin{itemize}
\item {Grp. gram.:m. pl.}
\end{itemize}
\begin{itemize}
\item {Proveniência:(De \textunderscore cássida\textunderscore )}
\end{itemize}
Tríbo de insectos coleópteros tetrâmeros, a que pertence a cássida.
\section{Cassídeas}
\begin{itemize}
\item {Grp. gram.:f. pl.}
\end{itemize}
O mesmo que \textunderscore cassidários\textunderscore .
\section{Cassidónia}
\begin{itemize}
\item {Grp. gram.:f.}
\end{itemize}
Variedade de pedra preciosa.
\section{Cassidulina}
\begin{itemize}
\item {Grp. gram.:f.}
\end{itemize}
Gênero de conchas microscópicas.
\section{Cassídulo}
\begin{itemize}
\item {Grp. gram.:m.}
\end{itemize}
Gênero de echinodermes.
\section{Cassina}
\begin{itemize}
\item {Grp. gram.:f.}
\end{itemize}
Espécie de azevinho.
\section{Cassineta}
\begin{itemize}
\item {fónica:nê}
\end{itemize}
\begin{itemize}
\item {Grp. gram.:f.}
\end{itemize}
Tecido fino de lan.
(Cp. \textunderscore cassa\textunderscore )
\section{Cassínia}
\begin{itemize}
\item {Grp. gram.:f.}
\end{itemize}
Gênero de plantas synanthéreas.
\section{Cassino}
\begin{itemize}
\item {Grp. gram.:m.}
\end{itemize}
Certo jôgo de cartas.(V.casino)
\section{Cassiopeia}
\begin{itemize}
\item {Grp. gram.:f.}
\end{itemize}
\begin{itemize}
\item {Proveniência:(Gr. \textunderscore Kasiopeia\textunderscore , n. p.)}
\end{itemize}
Constellação setentrional.
\section{Cassique}
\begin{itemize}
\item {Grp. gram.:m.}
\end{itemize}
Gênero de aves, oriundas da América, (\textunderscore cassicus\textunderscore ), do tamanho e fórma do melro; o mesmo que \textunderscore cassico\textunderscore .
\section{Cássis}
\begin{itemize}
\item {Grp. gram.:m.}
\end{itemize}
\begin{itemize}
\item {Proveniência:(Lat. \textunderscore cassis\textunderscore )}
\end{itemize}
Espécie de groselheira.
Espécie de groselha; licôr de groselha.
Insecto, o mesmo que \textunderscore cássida\textunderscore .
\section{Cassiterita}
\begin{itemize}
\item {Grp. gram.:f.}
\end{itemize}
\begin{itemize}
\item {Proveniência:(Do gr. \textunderscore cassiteros\textunderscore , estanho)}
\end{itemize}
Mineral, de que se extrai o estanho.
Estanho oxydado.
Bióxydo de estanho.
\section{Cassiterite}
\begin{itemize}
\item {Grp. gram.:f.}
\end{itemize}
\begin{itemize}
\item {Proveniência:(Do gr. \textunderscore cassiteros\textunderscore , estanho)}
\end{itemize}
Mineral, de que se extrai o estanho.
Estanho oxydado.
Bióxydo de estanho.
\section{Cassoco}
\begin{itemize}
\item {Grp. gram.:m.}
\end{itemize}
Qualquer moéda de prata.
(Cp. \textunderscore cassoquim\textunderscore )
\section{Cassoilo}
\begin{itemize}
\item {Grp. gram.:m.}
\end{itemize}
(V.caçoilo)
\section{Cassoiro}
\begin{itemize}
\item {Grp. gram.:m.}
\end{itemize}
\begin{itemize}
\item {Utilização:Náut.}
\end{itemize}
(V.caçoilo)
\section{Cassolete}
\begin{itemize}
\item {fónica:lê}
\end{itemize}
\begin{itemize}
\item {Grp. gram.:m.}
\end{itemize}
O mesmo que \textunderscore corselete\textunderscore . Cf. Garção, II, 22.
\section{Cassonete}
\begin{itemize}
\item {fónica:nê}
\end{itemize}
\begin{itemize}
\item {Grp. gram.:m.}
\end{itemize}
\begin{itemize}
\item {Utilização:Serralh.}
\end{itemize}
Peça para fazer roscas de parafusos.
\section{Cassoquim}
\begin{itemize}
\item {Grp. gram.:m.}
\end{itemize}
\begin{itemize}
\item {Utilização:Pop.}
\end{itemize}
Meio tostão em prata.
Prata miúda.
\section{Cassoquinho}
\begin{itemize}
\item {Grp. gram.:m.}
\end{itemize}
O mesmo que \textunderscore cassoquim\textunderscore .
\section{Cassungo}
\begin{itemize}
\item {Grp. gram.:m.}
\end{itemize}
\begin{itemize}
\item {Grp. gram.:Pl.}
\end{itemize}
Conta de bordado entre os negros da África. Cf. Serpa Pinto, II, 37.
Um dos povos da Guiné.
\section{Casta}
\begin{itemize}
\item {Grp. gram.:f.}
\end{itemize}
Espécie vegetal ou animal: \textunderscore uvas de bôa casta\textunderscore .
Raça; geração.
Qualidade; natureza: \textunderscore homem de má casta\textunderscore .
(Cp. \textunderscore casto\textunderscore )
\section{Castalho}
\begin{itemize}
\item {Grp. gram.:m.}
\end{itemize}
O mesmo que \textunderscore gastalho\textunderscore .
\section{Castamente}
\begin{itemize}
\item {Grp. gram.:adv.}
\end{itemize}
De modo casto.
Com castidade: \textunderscore viver castamente\textunderscore .
\section{Castâneas}
\begin{itemize}
\item {Grp. gram.:f. pl.}
\end{itemize}
\begin{itemize}
\item {Proveniência:(De \textunderscore castâneo\textunderscore )}
\end{itemize}
Família de plantas, separada das cupulíferas, e cujo typo é o castanheiro.
\section{Castâneo}
\begin{itemize}
\item {Grp. gram.:adj.}
\end{itemize}
\begin{itemize}
\item {Proveniência:(Do lat. \textunderscore castanea\textunderscore )}
\end{itemize}
Relativo ou semelhante ao castanheiro.
\section{Castanha}
\begin{itemize}
\item {Grp. gram.:f.}
\end{itemize}
\begin{itemize}
\item {Utilização:Prov.}
\end{itemize}
\begin{itemize}
\item {Utilização:alent.}
\end{itemize}
\begin{itemize}
\item {Utilização:Chul.}
\end{itemize}
\begin{itemize}
\item {Utilização:Veter.}
\end{itemize}
\begin{itemize}
\item {Grp. gram.:Pl.}
\end{itemize}
\begin{itemize}
\item {Proveniência:(Do lat. \textunderscore castanea\textunderscore )}
\end{itemize}
Fruto do castanheiro.
Fruto do caju.
Rôlo de cabello.
A cruzeta das azenhas.
Carolo.
Excremento de burro.
Engrossamento da epiderme, formando várias camadas, na parte interna do antebraço e metatarso de alguns quadrúpedes.
Peças de madeira ou ferro, pregadas no navio, e por onde passam os cabos.
\section{Castanhada}
\begin{itemize}
\item {Grp. gram.:f.}
\end{itemize}
\begin{itemize}
\item {Utilização:Prov.}
\end{itemize}
\begin{itemize}
\item {Utilização:alent.}
\end{itemize}
Doce de castanhas.
\section{Castanha-da-Índia}
\begin{itemize}
\item {Grp. gram.:f.}
\end{itemize}
\begin{itemize}
\item {Utilização:Prov.}
\end{itemize}
\begin{itemize}
\item {Utilização:beir.}
\end{itemize}
O mesmo que \textunderscore batata\textunderscore .
\section{Castanha-de-macaco}
\begin{itemize}
\item {Grp. gram.:f.}
\end{itemize}
Árvore brasileira das regiões do Amazonas.
\section{Castanhal}
\begin{itemize}
\item {Grp. gram.:m.}
\end{itemize}
\begin{itemize}
\item {Proveniência:(De \textunderscore castanha\textunderscore )}
\end{itemize}
Mata de castanheiros.
\section{Castanhedo}
\begin{itemize}
\item {fónica:nhê}
\end{itemize}
\begin{itemize}
\item {Grp. gram.:m.}
\end{itemize}
O mesmo que \textunderscore castanhal\textunderscore .
\section{Castanheira}
\begin{itemize}
\item {Grp. gram.:f.}
\end{itemize}
\begin{itemize}
\item {Proveniência:(De \textunderscore castanha\textunderscore )}
\end{itemize}
O mesmo que \textunderscore castinceira\textunderscore .
Mulher, que assa e vende castanhas.
\section{Castanheiro}
\begin{itemize}
\item {Grp. gram.:m.}
\end{itemize}
\begin{itemize}
\item {Proveniência:(De \textunderscore castanha\textunderscore )}
\end{itemize}
Árvore amentácea, (\textunderscore fagus castanea\textunderscore ).
Nome de outras árvores da mesma família.
\section{Castanheiro-da-Índia}
\begin{itemize}
\item {Grp. gram.:m.}
\end{itemize}
Grande árvore ornamental, vulgar em Lisbôa e Sintra, (\textunderscore aesculus hippocastanum\textunderscore , Lin.).
\section{Castanheiro-de-flôres-vermelhas}
\begin{itemize}
\item {Grp. gram.:m.}
\end{itemize}
Árvore, semelhante ao castanheiro-da-Índia, mas mais pequena e de flôres escarlates, (\textunderscore aesculus rubicunda\textunderscore , Loddi).
\section{Castanheta}
\begin{itemize}
\item {fónica:nhê}
\end{itemize}
\begin{itemize}
\item {Grp. gram.:f.}
\end{itemize}
\begin{itemize}
\item {Grp. gram.:Pl.}
\end{itemize}
\begin{itemize}
\item {Proveniência:(De \textunderscore castanha\textunderscore )}
\end{itemize}
Peixe.
O mesmo que \textunderscore castanholas\textunderscore .
\section{Castanhetear}
\begin{itemize}
\item {Grp. gram.:v. i.}
\end{itemize}
Tocar castanhetas.
\section{Castanho}
\begin{itemize}
\item {Grp. gram.:adj.}
\end{itemize}
\begin{itemize}
\item {Grp. gram.:M.}
\end{itemize}
\begin{itemize}
\item {Utilização:Prov.}
\end{itemize}
\begin{itemize}
\item {Utilização:trasm.}
\end{itemize}
Que tem côr de castanha: \textunderscore cabello castanho\textunderscore .
Madeira de castanheiro; castanheiro: \textunderscore uma viga de castanho\textunderscore .
Boi, cuja côr se aproxima da castanha.
O tempo das castanhas.
\section{Castanhol}
\begin{itemize}
\item {Grp. gram.:m.}
\end{itemize}
\begin{itemize}
\item {Utilização:Ant.}
\end{itemize}
Planta cyperácea, (\textunderscore scirpus mucronatus\textunderscore ).
Palha de centeio; colmo.
\section{Castanholar}
\begin{itemize}
\item {Grp. gram.:v. t.}
\end{itemize}
Fazer soar á maneira de castanholas:«\textunderscore castanholar os dedos\textunderscore ». Eça.
\section{Castanholas}
\begin{itemize}
\item {Grp. gram.:f. pl.}
\end{itemize}
\begin{itemize}
\item {Utilização:Prov.}
\end{itemize}
\begin{itemize}
\item {Utilização:minh.}
\end{itemize}
\begin{itemize}
\item {Utilização:Prov.}
\end{itemize}
\begin{itemize}
\item {Proveniência:(De \textunderscore castanha\textunderscore )}
\end{itemize}
Instrumento, composto de duas peças, geralmente de madeira, que, ligadas por um cordel aos dedos ou punhos, batem uma contra a outra.
Estalido, produzido pelas cabeças dos dedos pollegar e maior.
Batatas.
Castanhas.
\section{Castanhoso}
\begin{itemize}
\item {Grp. gram.:adj.}
\end{itemize}
Que tem castanhaes.
\section{Castanita}
\begin{itemize}
\item {Grp. gram.:f.}
\end{itemize}
\begin{itemize}
\item {Proveniência:(Do lat. \textunderscore castanea\textunderscore )}
\end{itemize}
Pedra, de côr e fórma semelhantes ás da castanha.
\section{Castão}
\begin{itemize}
\item {Grp. gram.:m.}
\end{itemize}
\begin{itemize}
\item {Proveniência:(Al. \textunderscore kasten\textunderscore )}
\end{itemize}
Ornato de metal, de osso ou de marfim, na parte superior de uma bengala.
\section{Castela}
\begin{itemize}
\item {Grp. gram.:f.}
\end{itemize}
\begin{itemize}
\item {Proveniência:(De \textunderscore Castella\textunderscore , n. p.)}
\end{itemize}
Antiga moéda castelhana, que correu em Portugal no tempo de D. João I.
\section{Castelã}
\begin{itemize}
\item {Grp. gram.:f.  e  adj.}
\end{itemize}
\begin{itemize}
\item {Utilização:Prov.}
\end{itemize}
\begin{itemize}
\item {Utilização:alent.}
\end{itemize}
Casta de uva, o mesmo que \textunderscore trincadeira\textunderscore .
\section{Castelada}
\begin{itemize}
\item {Grp. gram.:f.}
\end{itemize}
\begin{itemize}
\item {Utilização:Heráld.}
\end{itemize}
Bordadura, cruz, banda e outras peças ornadas de castelo.
\section{Castelan}
(fem. de \textunderscore castelão\textunderscore ^1)
\section{Castelan}
\begin{itemize}
\item {Grp. gram.:f.  e  adj.}
\end{itemize}
\begin{itemize}
\item {Utilização:Prov.}
\end{itemize}
\begin{itemize}
\item {Utilização:alent.}
\end{itemize}
Casta de uva, o mesmo que \textunderscore trincadeira\textunderscore .
\section{Castelania}
\begin{itemize}
\item {Grp. gram.:f.}
\end{itemize}
Jurisdicção de castelão.
\section{Castelão}
\begin{itemize}
\item {Grp. gram.:m.}
\end{itemize}
\begin{itemize}
\item {Proveniência:(Lat. \textunderscore castellanus\textunderscore )}
\end{itemize}
Senhor feudal, que, fortificando a sua residência, exercia jurisdicção própria em certa área.
Governador de castelo; alcaide.
\section{Castelão}
\begin{itemize}
\item {Grp. gram.:m.}
\end{itemize}
\begin{itemize}
\item {Utilização:Ant.}
\end{itemize}
\begin{itemize}
\item {Grp. gram.:M.  e  adj.}
\end{itemize}
\begin{itemize}
\item {Utilização:Des.}
\end{itemize}
\begin{itemize}
\item {Proveniência:(De \textunderscore Castella\textunderscore , n. p.)}
\end{itemize}
Variedade de uva preta, na Extremadura, Beira, Doiro e Alentejo, também conhecida por \textunderscore trincadeira\textunderscore .
Cavalo de raça castelhana.
O mesmo que \textunderscore castelhano\textunderscore . Cf. F. Manuel, \textunderscore Hospital das Letras\textunderscore .
\section{Castelaria}
\begin{itemize}
\item {Grp. gram.:f.}
\end{itemize}
Encargo de dirigir ou inspeccionar obras de castelo ou fortaleza.
(Cp. b. lat. \textunderscore castellare\textunderscore )
\section{Castelário}
\begin{itemize}
\item {Grp. gram.:m.}
\end{itemize}
\begin{itemize}
\item {Utilização:Des.}
\end{itemize}
Senhor de castelo; casteleiro.
\section{Castelático}
\begin{itemize}
\item {Grp. gram.:adj.}
\end{itemize}
\begin{itemize}
\item {Proveniência:(De \textunderscore castello\textunderscore )}
\end{itemize}
Dizia-se do tributo, que os vassalos pagavam ao senhor do castelo, para reparações dêste.
\section{Casteleiras}
\begin{itemize}
\item {fónica:ás}
\end{itemize}
\begin{itemize}
\item {Grp. gram.:loc. adv.}
\end{itemize}
Ás cavalitas.
(Us. na Bairrada)
\section{Casteleiro}
\begin{itemize}
\item {Grp. gram.:adj.}
\end{itemize}
\begin{itemize}
\item {Grp. gram.:M.}
\end{itemize}
Relativo a castelo. Cf. Herculano, \textunderscore Bobo\textunderscore , 254.
Senhor de castelo.
\section{Castelejo}
\begin{itemize}
\item {Grp. gram.:m.}
\end{itemize}
A parte mais elevada do castelo.
\section{Casteleta}
\begin{itemize}
\item {fónica:lê}
\end{itemize}
\begin{itemize}
\item {Grp. gram.:f.}
\end{itemize}
Espécie de pano nacional.
\section{Castelete}
\begin{itemize}
\item {fónica:lê}
\end{itemize}
\begin{itemize}
\item {Grp. gram.:m.}
\end{itemize}
Pequeno castelo. Cf. Tenreiro, XXIX.
\section{Castelhana}
\begin{itemize}
\item {Grp. gram.:f.}
\end{itemize}
\begin{itemize}
\item {Utilização:Prov.}
\end{itemize}
\begin{itemize}
\item {Utilização:alent.}
\end{itemize}
Faúlha, que o carvão solta, quando arde, estalando.
Variedade de figueira algarvia.
\section{Castelhana}
\begin{itemize}
\item {Grp. gram.:f.}
\end{itemize}
\begin{itemize}
\item {Utilização:Bras}
\end{itemize}
Árvore tinctória das regiões do Amazonas.
\section{Castelhanada}
\begin{itemize}
\item {Grp. gram.:f.}
\end{itemize}
\begin{itemize}
\item {Utilização:Prov.}
\end{itemize}
O mesmo que \textunderscore espanholada\textunderscore . Cf. Garrett, \textunderscore Retr. de Vénus\textunderscore , 218.
O mesmo que \textunderscore algaravia\textunderscore . (Colhido em Turquel)
\section{Castelhanamente}
\begin{itemize}
\item {Grp. gram.:adv.}
\end{itemize}
\begin{itemize}
\item {Proveniência:(De \textunderscore castelhano\textunderscore )}
\end{itemize}
Á maneira dos Castelhanos.
\section{Castelhanismo}
\begin{itemize}
\item {Grp. gram.:m.}
\end{itemize}
\begin{itemize}
\item {Proveniência:(De \textunderscore castelhano\textunderscore )}
\end{itemize}
Locução própria da língua castelhana.
\section{Castelhanizar}
\begin{itemize}
\item {Grp. gram.:v. t.}
\end{itemize}
O mesmo que \textunderscore espanholizar\textunderscore .
\section{Castelhano}
\begin{itemize}
\item {Grp. gram.:adj.}
\end{itemize}
\begin{itemize}
\item {Grp. gram.:M.}
\end{itemize}
Relativo a Castella.
Dialecto de Castella; língua espanhola.
Aquelle que é natural de Castella.
Antiga moéda espanhola, do valor de 25 reales.
Bôa casta de figo algarvio.
(Por \textunderscore castilhano\textunderscore , de \textunderscore Castilla\textunderscore , n. p. cast.)
\section{Castella}
\begin{itemize}
\item {Grp. gram.:f.}
\end{itemize}
\begin{itemize}
\item {Proveniência:(De \textunderscore Castella\textunderscore , n. p.)}
\end{itemize}
Antiga moéda castelhana, que correu em Portugal no tempo de D. João I.
\section{Castellada}
\begin{itemize}
\item {Grp. gram.:f.}
\end{itemize}
\begin{itemize}
\item {Utilização:Heráld.}
\end{itemize}
Bordadura, cruz, banda e outras peças ornadas de castello.
\section{Castellan}
(fem. de \textunderscore castellão\textunderscore ^1)
\section{Castellan}
\begin{itemize}
\item {Grp. gram.:f.  e  adj.}
\end{itemize}
\begin{itemize}
\item {Utilização:Prov.}
\end{itemize}
\begin{itemize}
\item {Utilização:alent.}
\end{itemize}
Casta de uva, o mesmo que \textunderscore trincadeira\textunderscore .
\section{Castellania}
\begin{itemize}
\item {Grp. gram.:f.}
\end{itemize}
Jurisdicção de castellão.
\section{Castellão}
\begin{itemize}
\item {Grp. gram.:m.}
\end{itemize}
\begin{itemize}
\item {Proveniência:(Lat. \textunderscore castellanus\textunderscore )}
\end{itemize}
Senhor feudal, que, fortificando a sua residência, exercia jurisdicção própria em certa área.
Governador de castello; alcaide.
\section{Castellão}
\begin{itemize}
\item {Grp. gram.:m.}
\end{itemize}
\begin{itemize}
\item {Utilização:Ant.}
\end{itemize}
\begin{itemize}
\item {Grp. gram.:M.  e  adj.}
\end{itemize}
\begin{itemize}
\item {Utilização:Des.}
\end{itemize}
\begin{itemize}
\item {Proveniência:(De \textunderscore Castella\textunderscore , n. p.)}
\end{itemize}
Variedade de uva preta, na Extremadura, Beira, Doiro e Alentejo, também conhecida por \textunderscore trincadeira\textunderscore .
Cavallo de raça castelhana.
O mesmo que \textunderscore castelhano\textunderscore . Cf. F. Manuel, \textunderscore Hospital das Letras\textunderscore .
\section{Castellão-branco}
\begin{itemize}
\item {Grp. gram.:m.}
\end{itemize}
Casta de uva de Collares.
\section{Castellão-francês}
\begin{itemize}
\item {Grp. gram.:m.}
\end{itemize}
Casta de uva do districto de Leiria.
\section{Castellaria}
\begin{itemize}
\item {Grp. gram.:f.}
\end{itemize}
Encargo de dirigir ou inspeccionar obras de castello ou fortaleza.
(Cp. b. lat. \textunderscore castellare\textunderscore )
\section{Castellário}
\begin{itemize}
\item {Grp. gram.:m.}
\end{itemize}
\begin{itemize}
\item {Utilização:Des.}
\end{itemize}
Senhor de castello; castelleiro.
\section{Castellático}
\begin{itemize}
\item {Grp. gram.:adj.}
\end{itemize}
\begin{itemize}
\item {Proveniência:(De \textunderscore castello\textunderscore )}
\end{itemize}
Dizia-se do tributo, que os vassallos pagavam ao senhor do castello, para reparações dêste.
\section{Castelleiras}
\begin{itemize}
\item {fónica:ás}
\end{itemize}
\begin{itemize}
\item {Grp. gram.:loc. adv.}
\end{itemize}
Ás cavallitas.
(Us. na Bairrada)
\section{Castelleiro}
\begin{itemize}
\item {Grp. gram.:adj.}
\end{itemize}
\begin{itemize}
\item {Grp. gram.:M.}
\end{itemize}
Relativo a castello. Cf. Herculano, \textunderscore Bobo\textunderscore , 254.
Senhor de castello.
\section{Castellejo}
\begin{itemize}
\item {Grp. gram.:m.}
\end{itemize}
A parte mais elevada do castello.
\section{Castelleta}
\begin{itemize}
\item {fónica:lê}
\end{itemize}
\begin{itemize}
\item {Grp. gram.:f.}
\end{itemize}
Espécie de pano nacional.
\section{Castellete}
\begin{itemize}
\item {fónica:lê}
\end{itemize}
\begin{itemize}
\item {Grp. gram.:m.}
\end{itemize}
Pequeno castello. Cf. Tenreiro, XXIX.
\section{Castello}
\begin{itemize}
\item {Grp. gram.:m.}
\end{itemize}
\begin{itemize}
\item {Utilização:Ant.}
\end{itemize}
\begin{itemize}
\item {Grp. gram.:Pl.}
\end{itemize}
\begin{itemize}
\item {Proveniência:(Lat. \textunderscore castellum\textunderscore )}
\end{itemize}
Residência fortificada.
Fortaleza.
Praça forte, com muralhas, barbacan, fôsso, etc.
A parte mais elevada no convés do navio.
Lugar de defesa.
Grande acumulação de objectos.
Casta de uva tinta.
Mastro, com muitos enfeites e galhardetes.
Espécie de jôgo popular.
\section{Castellôa}
\begin{itemize}
\item {Grp. gram.:f.}
\end{itemize}
Casta de uva, também chamada \textunderscore castellão\textunderscore .
\section{Castelo}
\begin{itemize}
\item {Grp. gram.:m.}
\end{itemize}
\begin{itemize}
\item {Utilização:Ant.}
\end{itemize}
\begin{itemize}
\item {Grp. gram.:Pl.}
\end{itemize}
\begin{itemize}
\item {Proveniência:(Lat. \textunderscore castellum\textunderscore )}
\end{itemize}
Residência fortificada.
Fortaleza.
Praça forte, com muralhas, barbacan, fôsso, etc.
A parte mais elevada no convés do navio.
Lugar de defesa.
Grande acumulação de objectos.
Casta de uva tinta.
Mastro, com muitos enfeites e galhardetes.
Espécie de jôgo popular.
\section{Castelôa}
\begin{itemize}
\item {Grp. gram.:f.}
\end{itemize}
Casta de uva, também chamada \textunderscore castelão\textunderscore .
\section{Casteval}
\begin{itemize}
\item {Grp. gram.:m.}
\end{itemize}
\begin{itemize}
\item {Utilização:Ant.}
\end{itemize}
O mesmo que \textunderscore alcaide\textunderscore . Cf. Faria, \textunderscore Europa Port.\textunderscore , III, 378; Leitão, \textunderscore Miscell.\textunderscore , 456.
\section{Castiçal}
\begin{itemize}
\item {Grp. gram.:m.}
\end{itemize}
Utensílio, que tem na parte superior um orifício circular, em que se segura a \textunderscore vela\textunderscore ^1.
(Segundo Michaëlis, do lat. barb. \textunderscore canicistalis\textunderscore , de \textunderscore cana\textunderscore  e do germ. \textunderscore stall\textunderscore )
\section{Castiçar}
\begin{itemize}
\item {Grp. gram.:v. t.}
\end{itemize}
\begin{itemize}
\item {Proveniência:(De \textunderscore castiço\textunderscore )}
\end{itemize}
Juntar (macho e fêmea).
Têr o macho cópula com (a fêmea).
\section{Castiço}
\begin{itemize}
\item {Grp. gram.:adj.}
\end{itemize}
\begin{itemize}
\item {Utilização:Prov.}
\end{itemize}
\begin{itemize}
\item {Utilização:alent.}
\end{itemize}
\begin{itemize}
\item {Grp. gram.:M.}
\end{itemize}
\begin{itemize}
\item {Utilização:Gír. de Lisbôa.}
\end{itemize}
\begin{itemize}
\item {Proveniência:(De \textunderscore casta\textunderscore )}
\end{itemize}
Que é de bôa casta.
Puro; vernáculo: \textunderscore linguagem castiça\textunderscore .
Próprio para reproducção de casta.
Que não presta.
Casta de uva preta.
Castelhano.
\section{Castidade}
\begin{itemize}
\item {Grp. gram.:f.}
\end{itemize}
\begin{itemize}
\item {Proveniência:(Lat. \textunderscore castitas\textunderscore )}
\end{itemize}
Qualidade do que é casto; pureza.
\section{Castificar}
\begin{itemize}
\item {Grp. gram.:v. t.}
\end{itemize}
\begin{itemize}
\item {Proveniência:(Lat. \textunderscore castificare\textunderscore )}
\end{itemize}
Tornar casto.
\section{Castigação}
\begin{itemize}
\item {Grp. gram.:f.}
\end{itemize}
(V.castigo)
\section{Castigador}
\begin{itemize}
\item {Grp. gram.:m.  e  adj.}
\end{itemize}
O que castiga.
\section{Castigamento}
\begin{itemize}
\item {Grp. gram.:m.}
\end{itemize}
\begin{itemize}
\item {Utilização:Ant.}
\end{itemize}
Acto de castigar.
\section{Castigar}
\begin{itemize}
\item {Grp. gram.:v. t.}
\end{itemize}
\begin{itemize}
\item {Proveniência:(Lat. \textunderscore casti-gare\textunderscore )}
\end{itemize}
Infligir castigo a; punir.
Emendar.
Admoestar.
Corrigir.
Tornar puro: \textunderscore castigar a linguagem\textunderscore .
\section{Castigável}
\begin{itemize}
\item {Grp. gram.:adj.}
\end{itemize}
\begin{itemize}
\item {Proveniência:(De \textunderscore castigar\textunderscore )}
\end{itemize}
Que merece, póde ou deve sêr castigado.
\section{Castigo}
\begin{itemize}
\item {Grp. gram.:m.}
\end{itemize}
\begin{itemize}
\item {Utilização:Taur.}
\end{itemize}
\begin{itemize}
\item {Proveniência:(De \textunderscore castigar\textunderscore )}
\end{itemize}
Soffrimento, que se applica a quem delinquiu; punição.
Admoestação.
Emenda.
Mortificação.
Acto de meter os ferros no toiro.
\section{Castileja}
\begin{itemize}
\item {Grp. gram.:f.}
\end{itemize}
Gênero de plantas escrofularíneas.
\section{Castilha}
\begin{itemize}
\item {Grp. gram.:f.}
\end{itemize}
O mesmo que \textunderscore castina\textunderscore .
\section{Castilho}
\begin{itemize}
\item {Grp. gram.:m.}
\end{itemize}
\begin{itemize}
\item {Utilização:Açor}
\end{itemize}
O mesmo que \textunderscore castello\textunderscore .
(Cast. \textunderscore castillo\textunderscore )
\section{Castiliano}
\begin{itemize}
\item {Grp. gram.:adj.}
\end{itemize}
Relativo a Castilho, (escritor):«\textunderscore églogas castilianas.\textunderscore »\textunderscore Mem. de Castilho\textunderscore , I, 299.
\section{Castilleja}
\begin{itemize}
\item {Grp. gram.:f.}
\end{itemize}
Gênero de plantas escrofularíneas.
\section{Castina}
\begin{itemize}
\item {Grp. gram.:f.}
\end{itemize}
\begin{itemize}
\item {Proveniência:(Fr. \textunderscore castine\textunderscore )}
\end{itemize}
Pedra, que se junta ao minério de ferro, para lhe facilitar a fusão.
\section{Castinça}
\begin{itemize}
\item {Grp. gram.:f.}
\end{itemize}
\begin{itemize}
\item {Utilização:Prov.}
\end{itemize}
\begin{itemize}
\item {Utilização:trasm.}
\end{itemize}
O mesmo que \textunderscore castinceira\textunderscore .
\section{Castinçal}
\begin{itemize}
\item {Grp. gram.:m.}
\end{itemize}
\begin{itemize}
\item {Proveniência:(De \textunderscore castinça\textunderscore )}
\end{itemize}
Mata de castinceiras.
\section{Castinceira}
\begin{itemize}
\item {Grp. gram.:f.}
\end{itemize}
Castanheiro bravo.
\section{Castinceiro}
\begin{itemize}
\item {Grp. gram.:m.}
\end{itemize}
O mesmo que \textunderscore castinceira\textunderscore .
\section{Castinha}
\begin{itemize}
\item {Grp. gram.:f.}
\end{itemize}
Variedade de uva da Bairrada.
\section{Castinheiro}
\begin{itemize}
\item {Grp. gram.:m.}
\end{itemize}
(Alter. pop. de \textunderscore castanheiro\textunderscore )
\section{Castival}
\begin{itemize}
\item {Grp. gram.:m.}
\end{itemize}
\begin{itemize}
\item {Utilização:Ant.}
\end{itemize}
O mesmo que \textunderscore alcaide\textunderscore . Cf. Faria, \textunderscore Europa Port.\textunderscore , III, 378; Leitão, \textunderscore Miscell.\textunderscore , 456.
\section{Casto}
\begin{itemize}
\item {Grp. gram.:adj.}
\end{itemize}
\begin{itemize}
\item {Proveniência:(Lat. \textunderscore castus\textunderscore )}
\end{itemize}
Que se abstém de prazeres sensuaes.
Innocente, puro.
Sem mescla.
\section{Castor}
\begin{itemize}
\item {Grp. gram.:m.}
\end{itemize}
\begin{itemize}
\item {Proveniência:(Lat. \textunderscore castor\textunderscore )}
\end{itemize}
Animal mammífero roedor.
Pêlo dêste animal.
Estrella dupla da constellação dos Gêmeos.
Chapéu de pêlo fino, preto e luzídio. Cf. Filinto, IX, 29.
Variedade de pano.
\section{Castorato}
\begin{itemize}
\item {Grp. gram.:m.}
\end{itemize}
Sal, produzido pela combinação do ácido castórico com uma base.
\section{Castorenho}
\begin{itemize}
\item {Grp. gram.:m.}
\end{itemize}
Chapéu de picador de toiros.
(Cast. \textunderscore castoreño\textunderscore )
\section{Castóreo}
\begin{itemize}
\item {Grp. gram.:m.}
\end{itemize}
\begin{itemize}
\item {Proveniência:(Lat. \textunderscore castoreum\textunderscore )}
\end{itemize}
Substância medicinal, segregada por glândulas do ventre do castor.
\section{Castórico}
\begin{itemize}
\item {Grp. gram.:adj.}
\end{itemize}
Diz-se de um ácido, que se obtém do princípio amargo, chamado castorina, por meio do ácido nítrico.
\section{Castorina}
\begin{itemize}
\item {Grp. gram.:f.}
\end{itemize}
\begin{itemize}
\item {Utilização:Chím.}
\end{itemize}
\begin{itemize}
\item {Proveniência:(Lat. \textunderscore castorina\textunderscore )}
\end{itemize}
Tecido de lan, macio e lustroso.
Princípio branco, crystallino e amargo, que se obtém do castóreo, por meio do alcool quente.
\section{Castração}
\begin{itemize}
\item {Grp. gram.:f.}
\end{itemize}
\begin{itemize}
\item {Proveniência:(Lat. \textunderscore castratio\textunderscore )}
\end{itemize}
Acto de castrar.
\section{Castrado}
\begin{itemize}
\item {Grp. gram.:m.}
\end{itemize}
O mesmo que \textunderscore capado\textunderscore .
\section{Castrador}
\begin{itemize}
\item {Grp. gram.:m.}
\end{itemize}
Aquelle que castra.
\section{Castrametação}
\begin{itemize}
\item {Grp. gram.:f.}
\end{itemize}
\begin{itemize}
\item {Proveniência:(Do lat. \textunderscore castra\textunderscore  + \textunderscore metatio\textunderscore )}
\end{itemize}
Acto de castramento.
\section{Castrametar}
\begin{itemize}
\item {Grp. gram.:v. t.}
\end{itemize}
\begin{itemize}
\item {Grp. gram.:V. i.}
\end{itemize}
\begin{itemize}
\item {Proveniência:(Do lat. \textunderscore castra\textunderscore  + \textunderscore metari\textunderscore )}
\end{itemize}
Acampar.
Fortificar.
Escolher e medir terreno para assentar arraial.
\section{Castrar}
\begin{itemize}
\item {Grp. gram.:v. t.}
\end{itemize}
\begin{itemize}
\item {Proveniência:(Lat. \textunderscore castrare\textunderscore )}
\end{itemize}
Cortar os órgãos reproductores a; capar.
\section{Castreja}
\textunderscore fem.\textunderscore  de \textunderscore castrejo\textunderscore ^1.
\section{Castrejo}
\begin{itemize}
\item {Grp. gram.:m.}
\end{itemize}
\begin{itemize}
\item {Utilização:Prov.}
\end{itemize}
\begin{itemize}
\item {Utilização:minh.}
\end{itemize}
\begin{itemize}
\item {Proveniência:(De \textunderscore Castro\textunderscore , n. p.)}
\end{itemize}
Aquelle que é natural de Castro-Laboreiro.
\section{Castrejo}
\begin{itemize}
\item {Grp. gram.:m.}
\end{itemize}
O mesmo que \textunderscore castrello\textunderscore .
\section{Castrello}
\begin{itemize}
\item {fónica:trê}
\end{itemize}
\begin{itemize}
\item {Grp. gram.:m.}
\end{itemize}
\begin{itemize}
\item {Utilização:Ant.}
\end{itemize}
Pequeno castro.
Lugar alto, defensável por natureza ou pela arte.
(B. lat. \textunderscore castrellum\textunderscore )
\section{Castrelo}
\begin{itemize}
\item {fónica:trê}
\end{itemize}
\begin{itemize}
\item {Grp. gram.:m.}
\end{itemize}
\begin{itemize}
\item {Utilização:Ant.}
\end{itemize}
Pequeno castro.
Lugar alto, defensável por natureza ou pela arte.
(B. lat. \textunderscore castrellum\textunderscore )
\section{Castrense}
\begin{itemize}
\item {Grp. gram.:adj.}
\end{itemize}
\begin{itemize}
\item {Proveniência:(Lat. \textunderscore castrensis\textunderscore )}
\end{itemize}
Relativo a acampamento militar.
Que respeita ao serviço militar.
\section{Castro}
\begin{itemize}
\item {Grp. gram.:m.}
\end{itemize}
\begin{itemize}
\item {Proveniência:(Lat. \textunderscore castrum\textunderscore )}
\end{itemize}
Castello, de origem romana ou pre-romana.
\section{Castrodaire}
\begin{itemize}
\item {fónica:cás}
\end{itemize}
\begin{itemize}
\item {Grp. gram.:f.}
\end{itemize}
\begin{itemize}
\item {Proveniência:(De \textunderscore Castro-Daire\textunderscore , n. p.)}
\end{itemize}
Variedade de pêra.
\section{Castrolomancia}
\begin{itemize}
\item {Grp. gram.:f.}
\end{itemize}
Arte de adivinhar, por meio de garrafa ou redoma, cheia de água. Cf. Castilho, \textunderscore Fastos\textunderscore , III, 324.
\section{Casual}
\begin{itemize}
\item {Grp. gram.:adj.}
\end{itemize}
\begin{itemize}
\item {Proveniência:(Lat. \textunderscore casualis\textunderscore )}
\end{itemize}
Dependente do acaso; accidental; fortuito.
\section{Casualidade}
\begin{itemize}
\item {Grp. gram.:f.}
\end{itemize}
Qualidade do que é casual.
Acaso.
\section{Casualmente}
\begin{itemize}
\item {Grp. gram.:adv.}
\end{itemize}
De modo casual.
\section{Casucha}
\begin{itemize}
\item {Grp. gram.:f.}
\end{itemize}
Casa pequena.
\section{Casuco}
\begin{itemize}
\item {Grp. gram.:m.}
\end{itemize}
\begin{itemize}
\item {Utilização:Prov.}
\end{itemize}
\begin{itemize}
\item {Utilização:alent.}
\end{itemize}
O mesmo que \textunderscore casinhola\textunderscore .
\section{Casuísta}
\begin{itemize}
\item {Grp. gram.:m.}
\end{itemize}
\begin{itemize}
\item {Proveniência:(De \textunderscore caso\textunderscore )}
\end{itemize}
Aquelle que resolve casos de consciência.
Aquelle que explica a moral, por meio de casos.
\section{Casuística}
\begin{itemize}
\item {Grp. gram.:f.}
\end{itemize}
\begin{itemize}
\item {Proveniência:(De \textunderscore casuístico\textunderscore )}
\end{itemize}
Systema dos casuístas.
\section{Casuístico}
\begin{itemize}
\item {Grp. gram.:adj.}
\end{itemize}
Relativo ao systema dos casuístas.
\section{Casula}
\begin{itemize}
\item {Grp. gram.:f.}
\end{itemize}
\begin{itemize}
\item {Utilização:Prov.}
\end{itemize}
\begin{itemize}
\item {Utilização:trasm.}
\end{itemize}
Cadeirinha de marnoto.
Vagem verde de feijão.
(Cp. \textunderscore casulo\textunderscore )
\section{Casula}
\begin{itemize}
\item {Grp. gram.:f.}
\end{itemize}
\begin{itemize}
\item {Proveniência:(Lat. \textunderscore casula\textunderscore )}
\end{itemize}
Vestimenta sacerdotal, que se põe sôbre a alva e a estola.
\section{Casulo}
\begin{itemize}
\item {Grp. gram.:m.}
\end{itemize}
\begin{itemize}
\item {Utilização:T. do Fundão}
\end{itemize}
Invólucro de sementes.
Invólucro filamentoso, construído pela larva do bicho da seda ou por outras larvas.
O mesmo que \textunderscore biscalongo\textunderscore .
Tubo metállico, por onde a chave entra na fechadura.
\section{Casuloso}
\begin{itemize}
\item {Grp. gram.:adj.}
\end{itemize}
Semelhante a casulo.
Que tem casulo.
\section{Casunguel}
\begin{itemize}
\item {Grp. gram.:m.}
\end{itemize}
\begin{itemize}
\item {Utilização:Náut.}
\end{itemize}
Caixa, para arrecadação do pão dos ranchos.
\section{Casuso}
\begin{itemize}
\item {Grp. gram.:adv.}
\end{itemize}
O mesmo que \textunderscore acasuso\textunderscore .
\section{Cata}
\begin{itemize}
\item {Grp. gram.:f.}
\end{itemize}
\begin{itemize}
\item {Utilização:Bras}
\end{itemize}
\begin{itemize}
\item {Proveniência:(De \textunderscore catar\textunderscore )}
\end{itemize}
Pesquisa, busca: \textunderscore andar em cata de alguém\textunderscore .
Lugar cavado, para extrahir oiro da terra.
\section{Cata}
\begin{itemize}
\item {Grp. gram.:f.}
\end{itemize}
Nome de duas árvores medicinaes, na ilha de San-Thomé.
\section{Catabolismo}
\begin{itemize}
\item {Grp. gram.:m.}
\end{itemize}
\begin{itemize}
\item {Utilização:Med.}
\end{itemize}
Conjunto dos actos de desassimilação.
(Cp. \textunderscore anabolismo\textunderscore )
\section{Catacáustica}
\begin{itemize}
\item {Grp. gram.:f.}
\end{itemize}
\begin{itemize}
\item {Utilização:Phýs.}
\end{itemize}
\begin{itemize}
\item {Proveniência:(Do gr. \textunderscore kata\textunderscore  + \textunderscore kaio\textunderscore )}
\end{itemize}
Curva dos raios luminosos, refractados por uma superfície curva.
\section{Catacego}
\begin{itemize}
\item {fónica:cá}
\end{itemize}
\begin{itemize}
\item {Grp. gram.:adj.}
\end{itemize}
\begin{itemize}
\item {Utilização:Pop.}
\end{itemize}
Que tem pouca vista; que tem a vista curta.
Pouco atilado.
\section{Cataceto}
\begin{itemize}
\item {Grp. gram.:m.}
\end{itemize}
\begin{itemize}
\item {Utilização:Bras}
\end{itemize}
Planta ornamental.
\section{Catachrese}
\begin{itemize}
\item {Grp. gram.:f.}
\end{itemize}
\begin{itemize}
\item {Proveniência:(Gr. \textunderscore katakhresis\textunderscore )}
\end{itemize}
Tropo, que consiste em desviar palavras da sua significação natural para outra, que tenha analogia com o sentido primitivo.
\section{Cataclisma}
\begin{itemize}
\item {Grp. gram.:f.}
\end{itemize}
\begin{itemize}
\item {Utilização:P. us.}
\end{itemize}
O mesmo que \textunderscore cataclismo\textunderscore .
\section{Cataclísmico}
\begin{itemize}
\item {Grp. gram.:adj.}
\end{itemize}
Relativo a cataclismo.
\section{Cataclismo}
\begin{itemize}
\item {Grp. gram.:m.}
\end{itemize}
\begin{itemize}
\item {Proveniência:(Gr. \textunderscore kataklusmos\textunderscore )}
\end{itemize}
Inundação.
Transformação geológica.
Desastre social.
\section{Cataclismologia}
\begin{itemize}
\item {Grp. gram.:f.}
\end{itemize}
\begin{itemize}
\item {Proveniência:(Do gr. \textunderscore kataklusmos\textunderscore  + \textunderscore logos\textunderscore )}
\end{itemize}
História dos cataclismos.
\section{Cataclysma}
\begin{itemize}
\item {Grp. gram.:f.}
\end{itemize}
\begin{itemize}
\item {Utilização:P. us.}
\end{itemize}
O mesmo que \textunderscore cataclysmo\textunderscore .
\section{Cataclýsmico}
\begin{itemize}
\item {Grp. gram.:adj.}
\end{itemize}
Relativo a cataclysmo.
\section{Cataclysmo}
\begin{itemize}
\item {Grp. gram.:m.}
\end{itemize}
\begin{itemize}
\item {Proveniência:(Gr. \textunderscore kataklusmos\textunderscore )}
\end{itemize}
Inundação.
Transformação geológica.
Desastre social.
\section{Cataclysmologia}
\begin{itemize}
\item {Grp. gram.:f.}
\end{itemize}
\begin{itemize}
\item {Proveniência:(Do gr. \textunderscore kataklusmos\textunderscore  + \textunderscore logos\textunderscore )}
\end{itemize}
História dos cataclysmos.
\section{Catacrese}
\begin{itemize}
\item {Grp. gram.:f.}
\end{itemize}
\begin{itemize}
\item {Proveniência:(Gr. \textunderscore katakhresis\textunderscore )}
\end{itemize}
Tropo, que consiste em desviar palavras da sua significação natural para outra, que tenha analogia com o sentido primitivo.
\section{Catacumbas}
\begin{itemize}
\item {Grp. gram.:f. pl.}
\end{itemize}
Subterrâneos, em que se escondiam Christãos primitivos, e que serviam principalmente para sepulturas.
Gruta com ossuário.
(B. lat. \textunderscore catacumba\textunderscore )
\section{Catacus}
\begin{itemize}
\item {Grp. gram.:m.}
\end{itemize}
\begin{itemize}
\item {Utilização:Prov.}
\end{itemize}
\begin{itemize}
\item {Utilização:alent.}
\end{itemize}
Planta herbácea, que se cose com legumes e de que se faz esparregado.
\section{Catacústica}
\begin{itemize}
\item {Grp. gram.:f.}
\end{itemize}
\begin{itemize}
\item {Utilização:Phýs.}
\end{itemize}
\begin{itemize}
\item {Proveniência:(Do gr. \textunderscore kata\textunderscore  + \textunderscore akoustikos\textunderscore )}
\end{itemize}
Estudo da reflexão dos sons.
\section{Catacústico}
\begin{itemize}
\item {Grp. gram.:adj.}
\end{itemize}
Relativo á catacústica.
\section{Catadióptrica}
\begin{itemize}
\item {Grp. gram.:f.}
\end{itemize}
\begin{itemize}
\item {Utilização:Phýs.}
\end{itemize}
\begin{itemize}
\item {Proveniência:(Do gr. \textunderscore kata\textunderscore  + \textunderscore dioptrikos\textunderscore )}
\end{itemize}
Estudo de reflexão e refracção da luz.
\section{Catadióptrico}
\begin{itemize}
\item {Grp. gram.:adj.}
\end{itemize}
Que diz respeito á catadióptrica.
\section{Catador}
\begin{itemize}
\item {Grp. gram.:m.}
\end{itemize}
\begin{itemize}
\item {Utilização:Bras}
\end{itemize}
\begin{itemize}
\item {Proveniência:(De \textunderscore catar\textunderscore )}
\end{itemize}
Instrumento, para limpar café.
\section{Catadupa}
\begin{itemize}
\item {Grp. gram.:f.}
\end{itemize}
\begin{itemize}
\item {Proveniência:(Gr. \textunderscore katadoupe\textunderscore )}
\end{itemize}
Quéda de grande porção de água corrente.
Cataracta.
\section{Catadupejar}
\begin{itemize}
\item {Grp. gram.:v. i.}
\end{itemize}
Caír em catadupa.
\section{Catadura}
\begin{itemize}
\item {Grp. gram.:f.}
\end{itemize}
\begin{itemize}
\item {Proveniência:(De \textunderscore catar\textunderscore )}
\end{itemize}
Semblante; apparência.
Estado de ânimo.
\section{Catafalco}
\begin{itemize}
\item {Grp. gram.:m.}
\end{itemize}
Estrado alto, essa, em que se colloca o féretro.
(B. lat. \textunderscore catafaltus\textunderscore )
\section{Catafeder}
\begin{itemize}
\item {Grp. gram.:v. i.}
\end{itemize}
\begin{itemize}
\item {Utilização:Prov.}
\end{itemize}
\begin{itemize}
\item {Utilização:dur.}
\end{itemize}
Cheirar mal, por effeito de mêdo.
\section{Cataglotismo}
\begin{itemize}
\item {Grp. gram.:m.}
\end{itemize}
\begin{itemize}
\item {Proveniência:(Gr. \textunderscore kataglottismos\textunderscore )}
\end{itemize}
Emprêgo de palavras esquisitas, extravagantes ou pretensiosas.
\section{Cataglottismo}
\begin{itemize}
\item {Grp. gram.:m.}
\end{itemize}
\begin{itemize}
\item {Proveniência:(Gr. \textunderscore kataglottismos\textunderscore )}
\end{itemize}
Emprêgo de palavras esquisitas, extravagantes ou pretensiosas.
\section{Catagmático}
\begin{itemize}
\item {Grp. gram.:adj.}
\end{itemize}
\begin{itemize}
\item {Utilização:Cir.}
\end{itemize}
\begin{itemize}
\item {Proveniência:(Do gr. \textunderscore katagma\textunderscore )}
\end{itemize}
Que auxilia a consolidação das fracturas.
\section{Catagrama}
\begin{itemize}
\item {Grp. gram.:f.}
\end{itemize}
\begin{itemize}
\item {Proveniência:(Do gr. \textunderscore kata\textunderscore  + \textunderscore gramma\textunderscore )}
\end{itemize}
Insecto lepidóptero diurno.
\section{Catagramma}
\begin{itemize}
\item {Grp. gram.:f.}
\end{itemize}
\begin{itemize}
\item {Proveniência:(Do gr. \textunderscore kata\textunderscore  + \textunderscore gramma\textunderscore )}
\end{itemize}
Insecto lepidóptero diurno.
\section{Cata-grande}
\begin{itemize}
\item {Grp. gram.:f.}
\end{itemize}
Árvore da ilha de San-Thomé.
\section{Cataia}
\begin{itemize}
\item {Grp. gram.:f.}
\end{itemize}
Erva medicinal do Brasil.
\section{Catalana}
\begin{itemize}
\item {Grp. gram.:f.}
\end{itemize}
\begin{itemize}
\item {Utilização:Prov.}
\end{itemize}
\begin{itemize}
\item {Utilização:minh.}
\end{itemize}
Panela de barro, oblonga.
\section{Catalânico}
\begin{itemize}
\item {Grp. gram.:adj.}
\end{itemize}
Relativo aos Catalães. Cf. Herculano, \textunderscore Eurico\textunderscore , 25 e 26.
\section{Catalão}
\begin{itemize}
\item {Grp. gram.:Adj.}
\end{itemize}
\textunderscore m.\textunderscore  (\textunderscore Pl. catalães\textunderscore )
Dialecto da Catalunha.
Indivíduo, que nasceu na Catalunha.
Relativo á Catalunha.
(Cast. \textunderscore catalán\textunderscore )
\section{Catalão}
\begin{itemize}
\item {Grp. gram.:m.}
\end{itemize}
\begin{itemize}
\item {Utilização:T. de Alcanena}
\end{itemize}
\begin{itemize}
\item {Utilização:Prov.}
\end{itemize}
\begin{itemize}
\item {Utilização:alent.}
\end{itemize}
Jumento grande.
Variedade de pimentão.
\section{Catalectico}
\begin{itemize}
\item {Grp. gram.:m.  e  adj.}
\end{itemize}
\begin{itemize}
\item {Proveniência:(Do gr. \textunderscore katalektikos\textunderscore )}
\end{itemize}
Verso grego ou latino a que, propositadamente, falta uma sýllaba.
\section{Catalecto}
\begin{itemize}
\item {Grp. gram.:m.}
\end{itemize}
\begin{itemize}
\item {Proveniência:(Gr. \textunderscore katalekta\textunderscore )}
\end{itemize}
Anthologia clássica.
\section{Catalefo}
\begin{itemize}
\item {Grp. gram.:m.}
\end{itemize}
\begin{itemize}
\item {Utilização:Ant.}
\end{itemize}
Essa, tumba.
\section{Catalepsia}
\begin{itemize}
\item {Grp. gram.:f.}
\end{itemize}
\begin{itemize}
\item {Proveniência:(Gr. \textunderscore katalepsis\textunderscore )}
\end{itemize}
Doença nervosa, caracterizada principalmente pela immobilidade do corpo e rigidez dos músculos.
\section{Cataléptico}
\begin{itemize}
\item {Grp. gram.:adj.}
\end{itemize}
\begin{itemize}
\item {Grp. gram.:M.}
\end{itemize}
\begin{itemize}
\item {Proveniência:(Gr. \textunderscore kataleptikos\textunderscore )}
\end{itemize}
Relativo á catalepsia.
Que padece catalepsia.
O enfermo de catalepsia.
\section{Catalogação}
\begin{itemize}
\item {Grp. gram.:f.}
\end{itemize}
Acto ou effeito de catalogar.
\section{Catalogador}
\begin{itemize}
\item {Grp. gram.:m.}
\end{itemize}
Aquelle que cataloga.
\section{Catalogar}
\begin{itemize}
\item {Grp. gram.:v. t.}
\end{itemize}
Ordenar ou inscrever em catálogo.
\section{Catalogizar}
\begin{itemize}
\item {Grp. gram.:v. t.}
\end{itemize}
O mesmo que \textunderscore catalogar\textunderscore . Cf. João Ribeiro, \textunderscore Estética\textunderscore .
\section{Catálogo}
\begin{itemize}
\item {Grp. gram.:m.}
\end{itemize}
\begin{itemize}
\item {Proveniência:(Gr. \textunderscore katalogos\textunderscore )}
\end{itemize}
Relação summária, methódica e ás vêzes alphabética, de coisas numerosas ou de pessôas.
\section{Catafonia}
\begin{itemize}
\item {Grp. gram.:f.}
\end{itemize}
\begin{itemize}
\item {Proveniência:(Do gr. \textunderscore kata\textunderscore  + \textunderscore phone\textunderscore )}
\end{itemize}
Música, produzida pelo echo.
\section{Catafónica}
\begin{itemize}
\item {Grp. gram.:f.}
\end{itemize}
\begin{itemize}
\item {Proveniência:(Do gr. \textunderscore kata\textunderscore  + \textunderscore phone\textunderscore )}
\end{itemize}
O mesmo que \textunderscore catacústica\textunderscore .
\section{Catáfora}
\begin{itemize}
\item {Grp. gram.:f.}
\end{itemize}
\begin{itemize}
\item {Proveniência:(Gr. \textunderscore kataphora\textunderscore )}
\end{itemize}
Somnolência mórbida, sem febre nem delirio.
\section{Cataforese}
\begin{itemize}
\item {Grp. gram.:f.}
\end{itemize}
\begin{itemize}
\item {Utilização:Med.}
\end{itemize}
Modo de introducção de substâncias medicamentosas no organismo, através da pelle, por meio da electricidade.
\section{Catafracta}
\begin{itemize}
\item {Grp. gram.:f.}
\end{itemize}
\begin{itemize}
\item {Utilização:Ant.}
\end{itemize}
\begin{itemize}
\item {Proveniência:(Lat. \textunderscore cataphracta\textunderscore )}
\end{itemize}
Coiraça; cota de armas, o mesmo que \textunderscore catafracto\textunderscore .
\section{Catafractário}
\begin{itemize}
\item {Grp. gram.:m.}
\end{itemize}
\begin{itemize}
\item {Proveniência:(Lat. \textunderscore cataphractarius\textunderscore )}
\end{itemize}
Aquelle que se resguardava com catafracto.
\section{Catafracto}
\begin{itemize}
\item {Grp. gram.:m.}
\end{itemize}
\begin{itemize}
\item {Proveniência:(Lat. \textunderscore cataphractus\textunderscore )}
\end{itemize}
Antiga armadura, espécie de coiraça, revestida de escamas de ferro.
Antigo navio de coberta, longo e estreito.
\section{Catalipia}
\begin{itemize}
\item {Grp. gram.:f.}
\end{itemize}
Processo de obter cópias ou imagens, por meio da catálise. Cf. \textunderscore Jornal do Comm.\textunderscore , do Rio, 26-VII-903.
\section{Catalisação}
\begin{itemize}
\item {Grp. gram.:f.}
\end{itemize}
Acto de catalisar.
\section{Catalisador}
\begin{itemize}
\item {Grp. gram.:m.}
\end{itemize}
Substância que catalisa.
\section{Catalisar}
\begin{itemize}
\item {Grp. gram.:v. t.}
\end{itemize}
Decompor pela catálise.
\section{Catálise}
\begin{itemize}
\item {Grp. gram.:f.}
\end{itemize}
\begin{itemize}
\item {Proveniência:(Gr. \textunderscore katalusis\textunderscore )}
\end{itemize}
Destruição de combinações chímicas, pela simples presença de certos corpos.
\section{Cataliticamente}
\begin{itemize}
\item {Grp. gram.:adv.}
\end{itemize}
\begin{itemize}
\item {Proveniência:(De \textunderscore catalýtico\textunderscore )}
\end{itemize}
Por meio da catálise.
\section{Catalítico}
\begin{itemize}
\item {Grp. gram.:adj.}
\end{itemize}
Relativo á catálise.
\section{Catalpa}
\begin{itemize}
\item {Grp. gram.:f.}
\end{itemize}
Formosa árvore ornamental, de flôres brancas, salpicadas de vermelho, (\textunderscore bignonia catalpa\textunderscore , Lin.).
\section{Catalpo}
\begin{itemize}
\item {Grp. gram.:m.}
\end{itemize}
O mesmo que \textunderscore catalpa\textunderscore .
\section{Catalufa}
\begin{itemize}
\item {Grp. gram.:f.}
\end{itemize}
\begin{itemize}
\item {Utilização:Des.}
\end{itemize}
Tecido de prata. Cf. Corvo, \textunderscore Anno na Côrte\textunderscore , I, 34.
\section{Catalypia}
\begin{itemize}
\item {Grp. gram.:f.}
\end{itemize}
Processo de obter cópias ou imagens, por meio da catályse. Cf. \textunderscore Jornal do Comm.\textunderscore , do Rio, 26-VII-903.
\section{Catalysação}
\begin{itemize}
\item {Grp. gram.:f.}
\end{itemize}
Acto de catalysar.
\section{Catalysador}
\begin{itemize}
\item {Grp. gram.:m.}
\end{itemize}
Substância que catalysa.
\section{Catalysar}
\begin{itemize}
\item {Grp. gram.:v. t.}
\end{itemize}
Decompor pela catályse.
\section{Catályse}
\begin{itemize}
\item {Grp. gram.:f.}
\end{itemize}
\begin{itemize}
\item {Proveniência:(Gr. \textunderscore katalusis\textunderscore )}
\end{itemize}
Destruição de combinações chímicas, pela simples presença de certos corpos.
\section{Catalyticamente}
\begin{itemize}
\item {Grp. gram.:adv.}
\end{itemize}
\begin{itemize}
\item {Proveniência:(De \textunderscore catalýtico\textunderscore )}
\end{itemize}
Por meio da catályse.
\section{Catalýtico}
\begin{itemize}
\item {Grp. gram.:adj.}
\end{itemize}
Relativo á catályse.
\section{Catamaran}
\begin{itemize}
\item {Grp. gram.:m.}
\end{itemize}
\begin{itemize}
\item {Utilização:Bras}
\end{itemize}
Primitivo salva-vidas, que foi talvez ponto de partida para a construcção do salva-vidas actual.
\section{Catambuera}
\begin{itemize}
\item {Grp. gram.:f.}
\end{itemize}
O mesmo que \textunderscore catanguera\textunderscore .
\section{Catambruera}
\begin{itemize}
\item {Grp. gram.:f.}
\end{itemize}
O mesmo que \textunderscore catanguera\textunderscore .
\section{Catamenial}
\begin{itemize}
\item {Grp. gram.:adj.}
\end{itemize}
Relativo ao catamênio.
\section{Catamênio}
\begin{itemize}
\item {Grp. gram.:m.}
\end{itemize}
\begin{itemize}
\item {Proveniência:(Gr. \textunderscore katamenia\textunderscore )}
\end{itemize}
O mesmo que \textunderscore mênstruo\textunderscore .
\section{Catamito}
\begin{itemize}
\item {Grp. gram.:m.}
\end{itemize}
\begin{itemize}
\item {Utilização:Fig.}
\end{itemize}
Homem effeminado. Cf. C. Lobo, \textunderscore Sát.\textunderscore , I, 119 e 215.
\section{Catana}
\begin{itemize}
\item {Grp. gram.:f.}
\end{itemize}
Alfange asiático.
Pequena espada curva.
Espada, com bainha de madeira, em uso entre os Timores.
(Japon. \textunderscore katana\textunderscore )
\section{Catanada}
\begin{itemize}
\item {Grp. gram.:f.}
\end{itemize}
\begin{itemize}
\item {Utilização:Pop.}
\end{itemize}
Pancada ou golpe com catana.
Espadeirada.
Reprehensão severa.
\section{Catanar}
\begin{itemize}
\item {Grp. gram.:v. t.}
\end{itemize}
\begin{itemize}
\item {Utilização:T. da Chamusca}
\end{itemize}
O mesmo que \textunderscore ceifar\textunderscore .
\section{Catanari}
\begin{itemize}
\item {Grp. gram.:m.}
\end{itemize}
Árvore brasileira das regiões do Amazonas.
\section{Catandice}
\begin{itemize}
\item {Grp. gram.:f.}
\end{itemize}
Planta angolense, da fam. das leguminosas.
\section{Catanduba}
\begin{itemize}
\item {Grp. gram.:f.}
\end{itemize}
O mesmo que \textunderscore catanduva\textunderscore .
\section{Catandur}
\begin{itemize}
\item {Grp. gram.:m.}
\end{itemize}
\begin{itemize}
\item {Utilização:Zool.}
\end{itemize}
Espécie de arganaz da Índia portuguesa.
\section{Catanduva}
\begin{itemize}
\item {Grp. gram.:f.}
\end{itemize}
\begin{itemize}
\item {Utilização:Bras}
\end{itemize}
Espécie de mato espinhoso.
Grande árvore, de madeira branca.
\section{Catanguera}
\begin{itemize}
\item {fónica:gu-é}
\end{itemize}
\begin{itemize}
\item {Grp. gram.:f.}
\end{itemize}
(V.catimpuera)
\section{Catano}
\begin{itemize}
\item {Grp. gram.:m.}
\end{itemize}
\begin{itemize}
\item {Utilização:Pleb.}
\end{itemize}
Pênis.
(Cp. \textunderscore catana\textunderscore )
\section{Catanta}
\begin{itemize}
\item {Grp. gram.:f.}
\end{itemize}
Arbusto trepador e sarmentoso de Angola.
\section{Catão}
\begin{itemize}
\item {Grp. gram.:m.}
\end{itemize}
\begin{itemize}
\item {Utilização:Fig.}
\end{itemize}
\begin{itemize}
\item {Proveniência:(De \textunderscore Catão\textunderscore , n. p.)}
\end{itemize}
Homem austero.
Aquelle que apparenta austeridade ou virtude.
\section{Catapasmo}
\begin{itemize}
\item {Grp. gram.:m.}
\end{itemize}
\begin{itemize}
\item {Proveniência:(Gr. \textunderscore katapasmos\textunderscore )}
\end{itemize}
Pó, com que se polvilha uma parte do corpo, por indicação médica.
\section{Catapereiro}
\begin{itemize}
\item {Grp. gram.:m.}
\end{itemize}
Árvore pomácea, (\textunderscore pirus communis\textunderscore , Lin.).
\section{Catapétalo}
\begin{itemize}
\item {Grp. gram.:adj.}
\end{itemize}
\begin{itemize}
\item {Utilização:Bot.}
\end{itemize}
\begin{itemize}
\item {Proveniência:(Do gr. \textunderscore kata\textunderscore  + \textunderscore petalon\textunderscore )}
\end{itemize}
Que tem as pétalas juntas com os estames.
\section{Cataphonia}
\begin{itemize}
\item {Grp. gram.:f.}
\end{itemize}
\begin{itemize}
\item {Proveniência:(Do gr. \textunderscore kata\textunderscore  + \textunderscore phone\textunderscore )}
\end{itemize}
Música, produzida pelo echo.
\section{Cataphónica}
\begin{itemize}
\item {Grp. gram.:f.}
\end{itemize}
\begin{itemize}
\item {Proveniência:(Do gr. \textunderscore kata\textunderscore  + \textunderscore phone\textunderscore )}
\end{itemize}
O mesmo que \textunderscore catacústica\textunderscore .
\section{Catáphora}
\begin{itemize}
\item {Grp. gram.:f.}
\end{itemize}
\begin{itemize}
\item {Proveniência:(Gr. \textunderscore kataphora\textunderscore )}
\end{itemize}
Somnolência mórbida, sem febre nem delirio.
\section{Cataphorese}
\begin{itemize}
\item {Grp. gram.:f.}
\end{itemize}
\begin{itemize}
\item {Utilização:Med.}
\end{itemize}
Modo de introducção de substâncias medicamentosas no organismo, através da pelle, por meio da electricidade.
\section{Cataphracta}
\begin{itemize}
\item {Grp. gram.:f.}
\end{itemize}
\begin{itemize}
\item {Utilização:Ant.}
\end{itemize}
\begin{itemize}
\item {Proveniência:(Lat. \textunderscore cataphracta\textunderscore )}
\end{itemize}
Coiraça; cota de armas, o mesmo que \textunderscore cataphracto\textunderscore .
\section{Cataphractário}
\begin{itemize}
\item {Grp. gram.:m.}
\end{itemize}
\begin{itemize}
\item {Proveniência:(Lat. \textunderscore cataphractarius\textunderscore )}
\end{itemize}
Aquelle que se resguardava com cataphracto.
\section{Cataphracto}
\begin{itemize}
\item {Grp. gram.:m.}
\end{itemize}
\begin{itemize}
\item {Proveniência:(Lat. \textunderscore cataphractus\textunderscore )}
\end{itemize}
Antiga armadura, espécie de coiraça, revestida de escamas de ferro.
Antigo navio de coberta, longo e estreito.
\section{Cataphraltos}
\begin{itemize}
\item {Grp. gram.:m. pl.}
\end{itemize}
Cavallaria de linha, na phalange macedónica.
\section{Cataplasma}
\begin{itemize}
\item {Grp. gram.:f.}
\end{itemize}
\begin{itemize}
\item {Utilização:Fig.}
\end{itemize}
\begin{itemize}
\item {Proveniência:(Gr. \textunderscore kataplasma\textunderscore )}
\end{itemize}
Papas medicamentosas.
Peça dos arreios, em que se prendem as argolas, por onde passam as guias das cavalgaduras.
Pessôa fraca, branda, doente.
\section{Cataplasmado}
\begin{itemize}
\item {Grp. gram.:adj.}
\end{itemize}
\begin{itemize}
\item {Utilização:Fam.}
\end{itemize}
Que tem cataplasma.
Adoentado, achacado.
\section{Cataplasmar}
\begin{itemize}
\item {Grp. gram.:v. t.}
\end{itemize}
(V.encataplasmar)
\section{Catapléctico}
\begin{itemize}
\item {Grp. gram.:adj.}
\end{itemize}
Relativo á cataplexia.
\section{Cataplexia}
\begin{itemize}
\item {Grp. gram.:f.}
\end{itemize}
\begin{itemize}
\item {Proveniência:(Gr. \textunderscore kataplexis\textunderscore )}
\end{itemize}
Perda súbita dos sentidos.
\section{Cataplónia}
\begin{itemize}
\item {Grp. gram.:f.}
\end{itemize}
Espécie de panela, hoje pouco usada, em que os marceneiros fazem verniz.
\section{Cataporas}
\begin{itemize}
\item {Grp. gram.:f. pl.}
\end{itemize}
\begin{itemize}
\item {Utilização:Bras}
\end{itemize}
Erupção cutânea, vulgarmente conhecida por \textunderscore bexigas doidas\textunderscore .
\section{Cataptose}
\begin{itemize}
\item {Grp. gram.:f.}
\end{itemize}
\begin{itemize}
\item {Proveniência:(Gr. \textunderscore kataptosis\textunderscore )}
\end{itemize}
Quéda súbita de um corpo, por doença.
\section{Catapu}
\begin{itemize}
\item {Grp. gram.:m.}
\end{itemize}
Planta solanácea.
\section{Catapúcia}
\begin{itemize}
\item {Grp. gram.:f.}
\end{itemize}
\begin{itemize}
\item {Proveniência:(Fr. \textunderscore catapuce\textunderscore )}
\end{itemize}
O mesmo que \textunderscore carrapateiro\textunderscore .
Nome de algumas plantas euphorbiáceas.
\section{Catapuias}
\begin{itemize}
\item {Grp. gram.:f. pl.}
\end{itemize}
\begin{itemize}
\item {Utilização:Bras}
\end{itemize}
Aborígenes, que habitaram no Pará.
\section{Catapulta}
\begin{itemize}
\item {Grp. gram.:f.}
\end{itemize}
\begin{itemize}
\item {Proveniência:(Lat. \textunderscore catapulta\textunderscore )}
\end{itemize}
Antiga máquina de guerra, para arremessar projécteis.
\section{Catar}
\begin{itemize}
\item {Grp. gram.:v. t.}
\end{itemize}
\begin{itemize}
\item {Proveniência:(Lat. \textunderscore captare\textunderscore )}
\end{itemize}
Pesquisar, buscar; espiolhar.
Buscar e matar os parasitos capillares a.
Examinar attentamente.
\section{Catar}
\begin{itemize}
\item {Grp. gram.:v. t.}
\end{itemize}
O mesmo que \textunderscore acatar\textunderscore . Cf. Garrett, \textunderscore Dona Branca\textunderscore , 50.
Dedicar, sentir:«\textunderscore eu cá só cato respeito a meu amo\textunderscore ». Camillo, \textunderscore Retr. de Ricardina\textunderscore , 58.
\section{Cataracta}
\begin{itemize}
\item {Grp. gram.:f.}
\end{itemize}
\begin{itemize}
\item {Utilização:Med.}
\end{itemize}
\begin{itemize}
\item {Grp. gram.:Pl.}
\end{itemize}
\begin{itemize}
\item {Proveniência:(Lat. \textunderscore cataracta\textunderscore )}
\end{itemize}
O mesmo que \textunderscore catadupa\textunderscore .
Opacidade, que impede a chegada dos raios luminosos á retina.
Portas ou represas, de que, em estilo bíblico, se diz que retêm as águas do céu.
\section{Catarina}
\begin{itemize}
\item {Grp. gram.:adj. f.}
\end{itemize}
\begin{itemize}
\item {Proveniência:(De \textunderscore Catharina\textunderscore , n. p.?)}
\end{itemize}
Diz-se de uma roda pequena dos relógios.
\section{Catarinas}
\begin{itemize}
\item {Grp. gram.:f. pl.}
\end{itemize}
\begin{itemize}
\item {Utilização:Pleb.}
\end{itemize}
Mamas grandes de mulher.
\section{Catarlan}
\begin{itemize}
\item {Grp. gram.:f.}
\end{itemize}
\begin{itemize}
\item {Utilização:T. da Bairrada}
\end{itemize}
\begin{itemize}
\item {Proveniência:(T. onom.)}
\end{itemize}
Espingarda de pederneira; espingarda velha ou reles.
\section{Catarral}
\begin{itemize}
\item {Grp. gram.:adj.}
\end{itemize}
\begin{itemize}
\item {Grp. gram.:F.}
\end{itemize}
\begin{itemize}
\item {Proveniência:(De \textunderscore catarrho\textunderscore )}
\end{itemize}
Relativo a catarrho.
Bronchite aguda.
\section{Catarrão}
\begin{itemize}
\item {Grp. gram.:m.}
\end{itemize}
\begin{itemize}
\item {Utilização:Des.}
\end{itemize}
O mesmo que \textunderscore catarrheira\textunderscore . Cf. G. Vicente, \textunderscore Inês Pereira\textunderscore .
\section{Catarrento}
\begin{itemize}
\item {Grp. gram.:adj.}
\end{itemize}
Que tem catarrho; propenso ao catarrho.
\section{Catarreira}
\begin{itemize}
\item {Grp. gram.:f.}
\end{itemize}
\begin{itemize}
\item {Utilização:Fam.}
\end{itemize}
\begin{itemize}
\item {Proveniência:(De \textunderscore catarrho\textunderscore )}
\end{itemize}
Defluxo; constipação.
\section{Catarrhal}
\begin{itemize}
\item {Grp. gram.:adj.}
\end{itemize}
\begin{itemize}
\item {Grp. gram.:F.}
\end{itemize}
\begin{itemize}
\item {Proveniência:(De \textunderscore catarrho\textunderscore )}
\end{itemize}
Relativo a catarrho.
Bronchite aguda.
\section{Catarrhão}
\begin{itemize}
\item {Grp. gram.:m.}
\end{itemize}
\begin{itemize}
\item {Utilização:Des.}
\end{itemize}
O mesmo que \textunderscore catarrheira\textunderscore . Cf. G. Vicente, \textunderscore Inês Pereira\textunderscore .
\section{Catarrético}
\begin{itemize}
\item {Grp. gram.:adj.}
\end{itemize}
\begin{itemize}
\item {Proveniência:(Gr. \textunderscore katarrhetikos\textunderscore )}
\end{itemize}
Dizia-se de algumas substâncias, que tinham a virtude de dissolver outras.
\section{Catarrheira}
\begin{itemize}
\item {Grp. gram.:f.}
\end{itemize}
\begin{itemize}
\item {Utilização:Fam.}
\end{itemize}
\begin{itemize}
\item {Proveniência:(De \textunderscore catarrho\textunderscore )}
\end{itemize}
Defluxo; constipação.
\section{Catarrhento}
\begin{itemize}
\item {Grp. gram.:adj.}
\end{itemize}
Que tem catarrho; propenso ao catarrho.
\section{Catarrhético}
\begin{itemize}
\item {Grp. gram.:adj.}
\end{itemize}
\begin{itemize}
\item {Proveniência:(Gr. \textunderscore katarrhetikos\textunderscore )}
\end{itemize}
Dizia-se de algumas substâncias, que tinham a virtude de dissolver outras.
\section{Catarrhineanos}
\begin{itemize}
\item {Grp. gram.:m. pl.}
\end{itemize}
\begin{itemize}
\item {Proveniência:(Do gr. \textunderscore kata\textunderscore  + \textunderscore rhin\textunderscore )}
\end{itemize}
Designação dos macacos do antigo continente, por terem as ventas muito aproximadas.
\section{Catarrhíneos}
\begin{itemize}
\item {Grp. gram.:m. pl.}
\end{itemize}
\begin{itemize}
\item {Proveniência:(Do gr. \textunderscore kata\textunderscore  + \textunderscore rhin\textunderscore )}
\end{itemize}
Designação dos macacos do antigo continente, por terem as ventas muito aproximadas.
\section{Catarrho}
\begin{itemize}
\item {Grp. gram.:m.}
\end{itemize}
\begin{itemize}
\item {Proveniência:(Gr. \textunderscore katarrhos\textunderscore )}
\end{itemize}
Fluxão nas membranas mucosas.
Bronchite.
Constipação, acompanhada de tosse.
\section{Catarrhoso}
\begin{itemize}
\item {Grp. gram.:adj.}
\end{itemize}
O mesmo que [[encatarrhoado|encatarrhoar-se]].
\section{Catarrineanos}
\begin{itemize}
\item {Grp. gram.:m. pl.}
\end{itemize}
\begin{itemize}
\item {Proveniência:(Do gr. \textunderscore kata\textunderscore  + \textunderscore rhin\textunderscore )}
\end{itemize}
Designação dos macacos do antigo continente, por terem as ventas muito aproximadas.
\section{Catarríneanos}
\begin{itemize}
\item {Grp. gram.:m. pl.}
\end{itemize}
\begin{itemize}
\item {Proveniência:(Do gr. \textunderscore kata\textunderscore  + \textunderscore rhin\textunderscore )}
\end{itemize}
Designação dos macacos do antigo continente, por terem as ventas muito aproximadas.
\section{Catarro}
\begin{itemize}
\item {Grp. gram.:m.}
\end{itemize}
\begin{itemize}
\item {Proveniência:(Gr. \textunderscore katarrhos\textunderscore )}
\end{itemize}
Fluxão nas membranas mucosas.
Bronchite.
Constipação, acompanhada de tosse.
\section{Catarroso}
\begin{itemize}
\item {Grp. gram.:adj.}
\end{itemize}
O mesmo que [[encatarroado|encatarroar-se]].
\section{Cataseto}
\begin{itemize}
\item {fónica:tassê}
\end{itemize}
\begin{itemize}
\item {Grp. gram.:m.}
\end{itemize}
\begin{itemize}
\item {Proveniência:(T. hybr., do gr. \textunderscore kata\textunderscore  + lat. \textunderscore seta\textunderscore )}
\end{itemize}
Gênero de orchídeas.
\section{Catasol}
\begin{itemize}
\item {fónica:tassol}
\end{itemize}
\begin{itemize}
\item {Grp. gram.:m.}
\end{itemize}
\begin{itemize}
\item {Utilização:Prov.}
\end{itemize}
\begin{itemize}
\item {Utilização:beir.}
\end{itemize}
\begin{itemize}
\item {Utilização:Bras}
\end{itemize}
\begin{itemize}
\item {Proveniência:(De \textunderscore catar\textunderscore  + \textunderscore sol\textunderscore )}
\end{itemize}
Cambiante.
Antigo tecido, fino e lustroso.
Erva, de que se faz um caldo laxante.
O mesmo que \textunderscore caracol\textunderscore .
\section{Catassol}
\begin{itemize}
\item {Grp. gram.:m.}
\end{itemize}
\begin{itemize}
\item {Utilização:Prov.}
\end{itemize}
\begin{itemize}
\item {Utilização:beir.}
\end{itemize}
\begin{itemize}
\item {Utilização:Bras}
\end{itemize}
\begin{itemize}
\item {Proveniência:(De \textunderscore catar\textunderscore  + \textunderscore sol\textunderscore )}
\end{itemize}
Cambiante.
Antigo tecido, fino e lustroso.
Erva, de que se faz um caldo laxante.
O mesmo que \textunderscore caracol\textunderscore .
\section{Catasta}
\begin{itemize}
\item {Grp. gram.:f.}
\end{itemize}
\begin{itemize}
\item {Proveniência:(Lat. \textunderscore catasta\textunderscore )}
\end{itemize}
Lugar, em que os escravos eram expostos á venda, entre os Romanos.
Antigo instrumento e lugar de tortura.
\section{Catástase}
\begin{itemize}
\item {Grp. gram.:f.}
\end{itemize}
\begin{itemize}
\item {Proveniência:(Gr. \textunderscore katastasis\textunderscore )}
\end{itemize}
Estado sanitário geral.
\section{Catastático}
\begin{itemize}
\item {Grp. gram.:adj.}
\end{itemize}
\begin{itemize}
\item {Proveniência:(Gr. \textunderscore katastatikos\textunderscore )}
\end{itemize}
Relativo a doenças, que predominam em determinadas circunstâncias atmosphéricas.
\section{Catástrofe}
\begin{itemize}
\item {Grp. gram.:f.}
\end{itemize}
\begin{itemize}
\item {Proveniência:(Gr. \textunderscore katastrophe\textunderscore )}
\end{itemize}
Desfecho de uma tragédia.
Fim lastimoso.
Grande desgraça.
\section{Catástrophe}
\begin{itemize}
\item {Grp. gram.:f.}
\end{itemize}
\begin{itemize}
\item {Proveniência:(Gr. \textunderscore katastrophe\textunderscore )}
\end{itemize}
Desfecho de uma tragédia.
Fim lastimoso.
Grande desgraça.
\section{Catatagambe}
\begin{itemize}
\item {Grp. gram.:m.}
\end{itemize}
Arvoreta angolense.
\section{Catatau}
\begin{itemize}
\item {Grp. gram.:m.}
\end{itemize}
\begin{itemize}
\item {Utilização:Prov.}
\end{itemize}
\begin{itemize}
\item {Utilização:trasm.}
\end{itemize}
\begin{itemize}
\item {Utilização:Ext.}
\end{itemize}
\begin{itemize}
\item {Utilização:Fam.}
\end{itemize}
Bêsta grande e velha.
Pessôa velha e magra.
Castigo, pancada.
\section{Catatraz!}
\begin{itemize}
\item {Grp. gram.:interj.}
\end{itemize}
(imitativa do estrondo, produzido por quéda ou pancadaria)
\section{Catatua}
\begin{itemize}
\item {Grp. gram.:f.}
\end{itemize}
(V.cacatua)
\section{Catatua}
\begin{itemize}
\item {Grp. gram.:f.}
\end{itemize}
\begin{itemize}
\item {Utilização:Prov.}
\end{itemize}
Espécie de carrocel.
\textunderscore Enfiar na catatua\textunderscore , achar-se num círculo vicioso, malhar em ferro frio, não saír da cepa torta.
\section{Catau}
\begin{itemize}
\item {Grp. gram.:m.}
\end{itemize}
\begin{itemize}
\item {Utilização:Náut.}
\end{itemize}
Dobra ou nó de um cabo, para o tornar mais curto.
\section{Catauari}
\begin{itemize}
\item {Grp. gram.:m.}
\end{itemize}
\begin{itemize}
\item {Utilização:Bras}
\end{itemize}
Espécie de palmeira.
\section{Cataúxis}
\begin{itemize}
\item {Grp. gram.:m. pl.}
\end{itemize}
Nome de algumas tribos indígenas do Brasil, que vivem nas margens dos rios Madeira, Cuari e Purus.
\section{Catavento}
\begin{itemize}
\item {Grp. gram.:m.}
\end{itemize}
\begin{itemize}
\item {Utilização:Fig.}
\end{itemize}
\begin{itemize}
\item {Utilização:Prov.}
\end{itemize}
\begin{itemize}
\item {Proveniência:(De \textunderscore catar\textunderscore  + \textunderscore vento\textunderscore )}
\end{itemize}
Bandeirinha, geralmente de ferro ou lata, enfiada numa haste, e collocada no alto dos edifícios, para indicar a direcção dos ventos que a movem; grimpa.
Lugar a bordo, occupado por quem dirige a manobra.
Ventilador.
Pessôa inconstante.
O mesmo que \textunderscore gaivão\textunderscore ^1.
\section{Cataxu}
\begin{itemize}
\item {Grp. gram.:m.}
\end{itemize}
Planta brasileira.
\section{Catazola}
\begin{itemize}
\item {Grp. gram.:f.}
\end{itemize}
\begin{itemize}
\item {Utilização:Açor}
\end{itemize}
Variedade de jôgo de pião.
\section{Cate}
\begin{itemize}
\item {Grp. gram.:m.}
\end{itemize}
\begin{itemize}
\item {Proveniência:(Do mal. \textunderscore kati\textunderscore )}
\end{itemize}
Antigo pêso indiano, correspondente a pouco mais de três kilogrammas.--Outros lhe attribuem seiscentas e vinte e cinco grammas, como Yule &amp; Burnell; e outros ainda, como Richard, \textunderscore Diction. Malab.\textunderscore , 650 grammas.
\section{Catébi}
\begin{itemize}
\item {Grp. gram.:m.}
\end{itemize}
Nome de duas espécies de falcão africano.
\section{Catechese}
\begin{itemize}
\item {fónica:qué}
\end{itemize}
\begin{itemize}
\item {Grp. gram.:f.}
\end{itemize}
\begin{itemize}
\item {Proveniência:(Gr. \textunderscore katekhesis\textunderscore )}
\end{itemize}
Instrucção methódica e oral, sôbre coisas de religião.
Acto de doutrinar.
\section{Catechético}
\begin{itemize}
\item {fónica:qué}
\end{itemize}
\begin{itemize}
\item {Grp. gram.:adj.}
\end{itemize}
Relativo à catechese.
Diz-se especialmente de um processo pedagógico, em que o ensino se ministra por prelecção, independentemente do interrogatório.
\section{Catechina}
\begin{itemize}
\item {fónica:qui}
\end{itemize}
\begin{itemize}
\item {Grp. gram.:f.}
\end{itemize}
\begin{itemize}
\item {Utilização:Chím.}
\end{itemize}
Substância, resultante da acção do ar sôbre a solução do ácido cachútico.
\section{Catechismo}
\begin{itemize}
\item {fónica:quis}
\end{itemize}
\begin{itemize}
\item {Grp. gram.:m.}
\end{itemize}
(V.catecismo)
\section{Catechista}
\begin{itemize}
\item {fónica:quis}
\end{itemize}
\begin{itemize}
\item {Grp. gram.:m.}
\end{itemize}
\begin{itemize}
\item {Proveniência:(Lat. \textunderscore catechista\textunderscore )}
\end{itemize}
O mesmo que \textunderscore catechizador\textunderscore .
\section{Catechístico}
\begin{itemize}
\item {fónica:quis}
\end{itemize}
\begin{itemize}
\item {Grp. gram.:adj.}
\end{itemize}
Que tem fórma de catecismo; relativo a catechista.
\section{Catechização}
\begin{itemize}
\item {fónica:qui}
\end{itemize}
\begin{itemize}
\item {Grp. gram.:f.}
\end{itemize}
Acto de catechizar.
\section{Catechizador}
\begin{itemize}
\item {fónica:qui}
\end{itemize}
\begin{itemize}
\item {Grp. gram.:m.  e  adj.}
\end{itemize}
O que catechiza.
\section{Catechizante}
\begin{itemize}
\item {Grp. gram.:m.  e  adj.}
\end{itemize}
O mesmo que \textunderscore catechizador\textunderscore .
\section{Catechizar}
\begin{itemize}
\item {fónica:qui}
\end{itemize}
\begin{itemize}
\item {Grp. gram.:v. t.}
\end{itemize}
\begin{itemize}
\item {Proveniência:(Gr. \textunderscore khatekhizein\textunderscore . Cp. \textunderscore catechese\textunderscore )}
\end{itemize}
Instruir em matéria religiosa.
Doutrinar sôbre questões sociaes.
Procurar convencer.
Ensinar.
\section{Catechu}
\begin{itemize}
\item {Grp. gram.:m.}
\end{itemize}
Substância vegetal, extrahida de várias espécies de acácia.
O mesmo que \textunderscore catechueiro\textunderscore . Cf. \textunderscore Techn. Rur.\textunderscore , 24 e 26.
(Mal. \textunderscore kachu\textunderscore )
\section{Catechueiro}
\begin{itemize}
\item {Grp. gram.:m.}
\end{itemize}
O mesmo que \textunderscore pau-ferro\textunderscore .
\section{Catechumenado}
\begin{itemize}
\item {fónica:cu}
\end{itemize}
\begin{itemize}
\item {Grp. gram.:m.}
\end{itemize}
Estado de catechúmeno.
\section{Catechumenato}
\begin{itemize}
\item {fónica:cu}
\end{itemize}
\begin{itemize}
\item {Grp. gram.:m.}
\end{itemize}
Estado de catechúmeno.
\section{Catechúmeno}
\begin{itemize}
\item {fónica:cu}
\end{itemize}
\begin{itemize}
\item {Grp. gram.:m.}
\end{itemize}
\begin{itemize}
\item {Proveniência:(Gr. \textunderscore katekoumenos\textunderscore )}
\end{itemize}
Aquelle que se prepara e instrue para receber o baptismo.
\section{Catecismo}
\begin{itemize}
\item {Grp. gram.:m.}
\end{itemize}
\begin{itemize}
\item {Utilização:Ext.}
\end{itemize}
\begin{itemize}
\item {Proveniência:(Gr. \textunderscore katekhismos\textunderscore . Cp. \textunderscore catechese\textunderscore )}
\end{itemize}
Livro elementar de instrucção religiosa por perguntas e respostas.
Doutrinação elementar sôbre qualquer sciência ou arte.
\section{Catecos}
\begin{itemize}
\item {Grp. gram.:m. pl.}
\end{itemize}
Uma das categorias, em que se divide o séquito do soba dos Zingas. Cf. \textunderscore Capello e Ivens\textunderscore , II, 52 e 62.
\section{Catecumenado}
\begin{itemize}
\item {Grp. gram.:m.}
\end{itemize}
Estado de catecúmeno.
\section{Catecumenato}
\begin{itemize}
\item {Grp. gram.:m.}
\end{itemize}
Estado de catecúmeno.
\section{Catecúmeno}
\begin{itemize}
\item {Grp. gram.:m.}
\end{itemize}
\begin{itemize}
\item {Proveniência:(Gr. \textunderscore katekoumenos\textunderscore )}
\end{itemize}
Aquelle que se prepara e instrue para receber o baptismo.
\section{Categorema}
\begin{itemize}
\item {Grp. gram.:m.}
\end{itemize}
\begin{itemize}
\item {Proveniência:(Gr. \textunderscore kategorema\textunderscore . Cp. \textunderscore categoria\textunderscore )}
\end{itemize}
Qualidade que, segundo a philosophia aristotélica, colloca um objecto em tal ou tal categoria.
\section{Categoremático}
\begin{itemize}
\item {Grp. gram.:adj.}
\end{itemize}
Relativo a categorema.
\section{Categoria}
\begin{itemize}
\item {Grp. gram.:f.}
\end{itemize}
\begin{itemize}
\item {Proveniência:(Gr. \textunderscore kategoria\textunderscore )}
\end{itemize}
Classe.
Série.
Espécie, natureza.
Classe de ideias, em Philosophia.
\section{Categoricamente}
\begin{itemize}
\item {Grp. gram.:adv.}
\end{itemize}
De modo categórico.
\section{Categórico}
\begin{itemize}
\item {Grp. gram.:adj.}
\end{itemize}
\begin{itemize}
\item {Proveniência:(Lat. \textunderscore categoricus\textunderscore )}
\end{itemize}
Relativo a categoria.
Claro, positivo, explícito.
\section{Categorizador}
\begin{itemize}
\item {Grp. gram.:m.}
\end{itemize}
Aquelle que categoriza.
\section{Categorizar}
\begin{itemize}
\item {Grp. gram.:v. t.}
\end{itemize}
Dispôr em categorias.
\section{Cateia}
\begin{itemize}
\item {Grp. gram.:f.}
\end{itemize}
\begin{itemize}
\item {Proveniência:(Lat. \textunderscore cateja\textunderscore )}
\end{itemize}
Espécie de lança ou seta, usada por Gállios e Teutões. Cf. Herculano, \textunderscore Eurico\textunderscore , 153.
Espécie de clava, guarnecida de pregos ou puas, o mesmo que \textunderscore borda\textunderscore .
\section{Catejuá}
\begin{itemize}
\item {Grp. gram.:m.}
\end{itemize}
Árvore silvestre do Brasil.
\section{Cátele}
\begin{itemize}
\item {Grp. gram.:m.}
\end{itemize}
O mesmo que catre:«\textunderscore ...está de joelhos chorando de bruços sobre o catele...\textunderscore »\textunderscore Peregrin.\textunderscore , f. 285.
\section{Catena}
\begin{itemize}
\item {Grp. gram.:f.}
\end{itemize}
Pássaro dentirostro da África.
\section{Catenação}
\begin{itemize}
\item {Grp. gram.:f.}
\end{itemize}
\begin{itemize}
\item {Proveniência:(Lat. \textunderscore catenatio\textunderscore )}
\end{itemize}
O mesmo que \textunderscore concatenação\textunderscore .
\section{Catenária}
\begin{itemize}
\item {Grp. gram.:f.}
\end{itemize}
\begin{itemize}
\item {Proveniência:(Lat. \textunderscore catenaria\textunderscore )}
\end{itemize}
Curva, formada por corda flexível, tendo fixos os pontos extremos.
\section{Catenários}
\begin{itemize}
\item {Grp. gram.:m. pl.}
\end{itemize}
\begin{itemize}
\item {Proveniência:(Lat. \textunderscore catenarius\textunderscore )}
\end{itemize}
Gênero de pólypos bryzoários.
\section{Catenatício}
\begin{itemize}
\item {Grp. gram.:m.}
\end{itemize}
O mesmo que \textunderscore carcerática\textunderscore .
(Cp. lat. \textunderscore catena\textunderscore )
\section{Catenela}
\begin{itemize}
\item {Grp. gram.:f.}
\end{itemize}
Gênero de plantas marinhas.
\section{Catenífero}
\begin{itemize}
\item {Grp. gram.:adj.}
\end{itemize}
\begin{itemize}
\item {Proveniência:(Do lat. \textunderscore catena\textunderscore  + \textunderscore ferre\textunderscore )}
\end{itemize}
Que tem cadeias, ou traços em fórma de cadeia.
\section{Catenípora}
\begin{itemize}
\item {Grp. gram.:f.}
\end{itemize}
Gênero de madréporas fósseis.
\section{Catênula}
\begin{itemize}
\item {Grp. gram.:f.}
\end{itemize}
\begin{itemize}
\item {Proveniência:(Lat. \textunderscore catenula\textunderscore )}
\end{itemize}
Pequena cadeia.
Traço, em fórma de cadeia.
\section{Catenulado}
\begin{itemize}
\item {Grp. gram.:adj.}
\end{itemize}
Que tem fórma de catênula.
\section{Catequese}
\begin{itemize}
\item {Grp. gram.:f.}
\end{itemize}
\begin{itemize}
\item {Proveniência:(Gr. \textunderscore katekhesis\textunderscore )}
\end{itemize}
Instrucção metódica e oral, sôbre coisas de religião.
Acto de doutrinar.
\section{Catequético}
\begin{itemize}
\item {Grp. gram.:adj.}
\end{itemize}
Relativo à catequese.
Diz-se especialmente de um processo pedagógico, em que o ensino se ministra por prelecção, independentemente do interrogatório.
\section{Catequina}
\begin{itemize}
\item {Grp. gram.:f.}
\end{itemize}
\begin{itemize}
\item {Utilização:Chím.}
\end{itemize}
Substância, resultante da acção do ar sôbre a solução do ácido cacútico.
\section{Catequista}
\begin{itemize}
\item {Grp. gram.:m.}
\end{itemize}
\begin{itemize}
\item {Proveniência:(Lat. \textunderscore catechista\textunderscore )}
\end{itemize}
O mesmo que \textunderscore catequizador\textunderscore .
\section{Catequístico}
\begin{itemize}
\item {Grp. gram.:adj.}
\end{itemize}
Que tem fórma de catecismo; relativo a catequista.
\section{Catequização}
\begin{itemize}
\item {Grp. gram.:f.}
\end{itemize}
Acto de catequizar.
\section{Catequizador}
\begin{itemize}
\item {Grp. gram.:m.  e  adj.}
\end{itemize}
O que catequiza.
\section{Catequizante}
\begin{itemize}
\item {Grp. gram.:m.  e  adj.}
\end{itemize}
O mesmo que \textunderscore catequizador\textunderscore .
\section{Catequizar}
\begin{itemize}
\item {Grp. gram.:v. t.}
\end{itemize}
\begin{itemize}
\item {Proveniência:(Gr. \textunderscore khatekhizein\textunderscore . Cp. \textunderscore catechese\textunderscore )}
\end{itemize}
Instruir em matéria religiosa.
Doutrinar sôbre questões sociaes.
Procurar convencer.
Ensinar.
\section{Cateretê}
\begin{itemize}
\item {Grp. gram.:m.}
\end{itemize}
\begin{itemize}
\item {Utilização:Bras. do S}
\end{itemize}
Espécie de batuque.
\section{Caterineta}
\begin{itemize}
\item {fónica:nê}
\end{itemize}
\begin{itemize}
\item {Grp. gram.:f.}
\end{itemize}
\begin{itemize}
\item {Utilização:Bras}
\end{itemize}
\begin{itemize}
\item {Proveniência:(De \textunderscore Caterina\textunderscore , n. p.)}
\end{itemize}
Boneca de trapos.
\section{Caterva}
\begin{itemize}
\item {Grp. gram.:f.}
\end{itemize}
\begin{itemize}
\item {Proveniência:(Lat. \textunderscore caterva\textunderscore )}
\end{itemize}
Multidão de tropas.
Multidão de gente ou de animaes.
\section{Cata}
\begin{itemize}
\item {Grp. gram.:f.}
\end{itemize}
Gênero de plantas celastríneas.
\section{Catárase}
\begin{itemize}
\item {Grp. gram.:f.}
\end{itemize}
\begin{itemize}
\item {Utilização:P. us.}
\end{itemize}
\begin{itemize}
\item {Proveniência:(Do gr. \textunderscore kathairein\textunderscore )}
\end{itemize}
O mesmo que \textunderscore diarreia\textunderscore .
\section{Catarina}
\begin{itemize}
\item {Grp. gram.:f.}
\end{itemize}
Casta de uva branca de Ourém.
\section{Catarinaconga}
\begin{itemize}
\item {Grp. gram.:f.}
\end{itemize}
Árvore silvestre do Brasil.
\section{Catarinense}
\begin{itemize}
\item {Grp. gram.:adj.}
\end{itemize}
\begin{itemize}
\item {Grp. gram.:M.}
\end{itemize}
\begin{itemize}
\item {Proveniência:(De \textunderscore Catharina\textunderscore , n. p.)}
\end{itemize}
Relativo ao Estado de Santa-Catarina, no Brasil.
Habitante dêsse Estado.
\section{Catarma}
\begin{itemize}
\item {Grp. gram.:m.}
\end{itemize}
\begin{itemize}
\item {Utilização:Des.}
\end{itemize}
\begin{itemize}
\item {Proveniência:(Gr. \textunderscore katharma\textunderscore , lixo)}
\end{itemize}
Objecto de desprêzo?«\textunderscore Quis o Senhor fazer-se catarma dos homens, para lhes dar remedio\textunderscore ». Amador Arráiz.
\section{Catártico}
\begin{itemize}
\item {Grp. gram.:adj.}
\end{itemize}
\begin{itemize}
\item {Grp. gram.:M.}
\end{itemize}
\begin{itemize}
\item {Proveniência:(Gr. \textunderscore kathartikos\textunderscore )}
\end{itemize}
Que tem qualidades purgativas mais enérgicas que os laxantes e menos que os drásticos.
Medicamento, que tem essas qualidades.
\section{Catartina}
\begin{itemize}
\item {Grp. gram.:f.}
\end{itemize}
Substancia acre e nauseabunda, que se encontra no sene.
(Cp. \textunderscore cathártico\textunderscore )
\section{Catartínico}
\begin{itemize}
\item {Grp. gram.:adj.}
\end{itemize}
\begin{itemize}
\item {Proveniência:(De \textunderscore cathartina\textunderscore )}
\end{itemize}
Diz-se de um ácido, extrahido dos folículos do sene e que se emprega em Medicina, como purgante.
\section{Catarto}
\begin{itemize}
\item {Grp. gram.:m.}
\end{itemize}
\begin{itemize}
\item {Proveniência:(Gr. \textunderscore kathartes\textunderscore )}
\end{itemize}
Abutre americano, que as populações respeitam, porque lhes limpa as ruas, devorando os resíduos e detritos orgânicos.
\section{Cátedra}
\begin{itemize}
\item {Grp. gram.:f.}
\end{itemize}
\begin{itemize}
\item {Proveniência:(Lat. \textunderscore cathedra\textunderscore )}
\end{itemize}
Cadeira de quem ensina.
Cadeira pontifícia.
\section{Catedral}
\begin{itemize}
\item {Grp. gram.:adj.}
\end{itemize}
\begin{itemize}
\item {Grp. gram.:F.}
\end{itemize}
Diz-se da igreja da séde de uma diocese.
Sé.
(B. lat. \textunderscore cathedralis\textunderscore )
\section{Catedrátigo}
\begin{itemize}
\item {Grp. gram.:m.}
\end{itemize}
\begin{itemize}
\item {Utilização:Ant.}
\end{itemize}
Censo ou pensão annual que as igrejas seculares pagavam ao seu Bispo, em testemunho de sujeição.
(Cp. \textunderscore cathedrático\textunderscore )
\section{Catedráticamente}
\begin{itemize}
\item {Grp. gram.:adv.}
\end{itemize}
De modo catedrático.
Autoritariamente.
\section{Catedrático}
\begin{itemize}
\item {Grp. gram.:m.  e  adj.}
\end{itemize}
\begin{itemize}
\item {Grp. gram.:M.}
\end{itemize}
\begin{itemize}
\item {Utilização:Ant.}
\end{itemize}
\begin{itemize}
\item {Utilização:Ant.}
\end{itemize}
Professor effectivo de escolas superiores.
O mesmo que \textunderscore catedrátigo\textunderscore .
Propinas, que o Bispo dava, quando tomava posse do seu lugar.
(B. lat. \textunderscore cathedráticus\textunderscore )
\section{Catedrilha}
\begin{itemize}
\item {Grp. gram.:f.}
\end{itemize}
(dem. de \textunderscore cátedra\textunderscore )
\section{Catérese}
\begin{itemize}
\item {Grp. gram.:f.}
\end{itemize}
\begin{itemize}
\item {Proveniência:(Do gr. \textunderscore kathairesis\textunderscore )}
\end{itemize}
Evacuação ou hemorragia natural.
\section{Caterético}
\begin{itemize}
\item {Grp. gram.:m.  e  adj.}
\end{itemize}
\begin{itemize}
\item {Proveniência:(Gr. \textunderscore cathairetikos\textunderscore )}
\end{itemize}
Medicamento levemente cáustico, ou empregado em pequena quantidade.
\section{Catesbeia}
\begin{itemize}
\item {Grp. gram.:f.}
\end{itemize}
Nome de varios arbustos rubiáceos, (\textunderscore catesboea\textunderscore ).
\section{Cateta-silvestre}
\begin{itemize}
\item {Grp. gram.:f.}
\end{itemize}
Arbusto angolense, de frutos amarelos.
\section{Cateta-umito}
\begin{itemize}
\item {Grp. gram.:f.}
\end{itemize}
Arbusto angolense, trepador e de flôres hermaphroditas e inodoras.
\section{Catete}
\begin{itemize}
\item {fónica:tê}
\end{itemize}
\begin{itemize}
\item {Grp. gram.:m.}
\end{itemize}
\begin{itemize}
\item {Utilização:Bras}
\end{itemize}
\begin{itemize}
\item {Grp. gram.:Adj.}
\end{itemize}
\begin{itemize}
\item {Utilização:Bras}
\end{itemize}
Variedade de milho.
Diz-se das gallinhas pequenas, de pennugem lisa e pernas nuas.
\section{Catetê}
\begin{itemize}
\item {Grp. gram.:m.}
\end{itemize}
Pássaro dentirostro africano.
\section{Catetê}
\begin{itemize}
\item {Grp. gram.:m.}
\end{itemize}
Tecelão indiano.
\section{Cateter}
\begin{itemize}
\item {Grp. gram.:m.}
\end{itemize}
\begin{itemize}
\item {Proveniência:(Gr. \textunderscore katheter\textunderscore )}
\end{itemize}
Sonda, que se emprega na operação da talha.
\section{Cateterismo}
\begin{itemize}
\item {Grp. gram.:m.}
\end{itemize}
\begin{itemize}
\item {Proveniência:(Gr. \textunderscore katheterismos\textunderscore )}
\end{itemize}
Sondagem, por meio do cateter.
Qualquer sondagem cirúrgica.
\section{Cateterizar}
\begin{itemize}
\item {Grp. gram.:v. t.}
\end{itemize}
Sondar com o cateter.
\section{Cateto}
\begin{itemize}
\item {Grp. gram.:m.}
\end{itemize}
\begin{itemize}
\item {Utilização:Mathem.}
\end{itemize}
\begin{itemize}
\item {Utilização:Phýs.}
\end{itemize}
\begin{itemize}
\item {Proveniência:(Gr. \textunderscore kathetos\textunderscore )}
\end{itemize}
Linha perpendicular a outra ou a uma superfície.
Própriamente, cada um dos lados do ângulo recto no triângulo rectangular.
Raio luminoso, que incide ou é reflectido perpendicularmente.
\section{Catetómetro}
\begin{itemize}
\item {Grp. gram.:m.}
\end{itemize}
\begin{itemize}
\item {Proveniência:(Do gr. \textunderscore kathete\textunderscore  + \textunderscore metron\textunderscore )}
\end{itemize}
Instrumento, para medir extensões verticaes.
\section{Catha}
\begin{itemize}
\item {Grp. gram.:f.}
\end{itemize}
Gênero de plantas celastríneas.
\section{Cathárase}
\begin{itemize}
\item {Grp. gram.:f.}
\end{itemize}
\begin{itemize}
\item {Utilização:P. us.}
\end{itemize}
\begin{itemize}
\item {Proveniência:(Do gr. \textunderscore kathairein\textunderscore )}
\end{itemize}
O mesmo que \textunderscore diarreia\textunderscore .
\section{Catharina}
\begin{itemize}
\item {Grp. gram.:f.}
\end{itemize}
Casta de uva branca de Ourém.
\section{Catharinaconga}
\begin{itemize}
\item {Grp. gram.:f.}
\end{itemize}
Árvore silvestre do Brasil.
\section{Catharinense}
\begin{itemize}
\item {Grp. gram.:adj.}
\end{itemize}
\begin{itemize}
\item {Grp. gram.:M.}
\end{itemize}
\begin{itemize}
\item {Proveniência:(De \textunderscore Catharina\textunderscore , n. p.)}
\end{itemize}
Relativo ao Estado de Santa-Catharina, no Brasil.
Habitante dêsse Estado.
\section{Catharma}
\begin{itemize}
\item {Grp. gram.:m.}
\end{itemize}
\begin{itemize}
\item {Utilização:Des.}
\end{itemize}
\begin{itemize}
\item {Proveniência:(Gr. \textunderscore katharma\textunderscore , lixo)}
\end{itemize}
Objecto de desprêzo?«\textunderscore Quis o Senhor fazer-se catharma dos homens, para lhes dar remedio\textunderscore ». Amador Arráiz.
\section{Cathártico}
\begin{itemize}
\item {Grp. gram.:adj.}
\end{itemize}
\begin{itemize}
\item {Grp. gram.:M.}
\end{itemize}
\begin{itemize}
\item {Proveniência:(Gr. \textunderscore kathartikos\textunderscore )}
\end{itemize}
Que tem qualidades purgativas mais enérgicas que os laxantes e menos que os drásticos.
Medicamento, que tem essas qualidades.
\section{Cathartina}
\begin{itemize}
\item {Grp. gram.:f.}
\end{itemize}
Substancia acre e nauseabunda, que se encontra no sene.
(Cp. \textunderscore cathártico\textunderscore )
\section{Cathartínico}
\begin{itemize}
\item {Grp. gram.:adj.}
\end{itemize}
\begin{itemize}
\item {Proveniência:(De \textunderscore cathartina\textunderscore )}
\end{itemize}
Diz-se de um ácido, extrahido dos folículos do sene e que se emprega em Medicina, como purgante.
\section{Catharto}
\begin{itemize}
\item {Grp. gram.:m.}
\end{itemize}
\begin{itemize}
\item {Proveniência:(Gr. \textunderscore kathartes\textunderscore )}
\end{itemize}
Abutre americano, que as populações respeitam, porque lhes limpa as ruas, devorando os resíduos e detritos orgânicos.
\section{Cáthedra}
\begin{itemize}
\item {Grp. gram.:f.}
\end{itemize}
\begin{itemize}
\item {Proveniência:(Lat. \textunderscore cathedra\textunderscore )}
\end{itemize}
Cadeira de quem ensina.
Cadeira pontifícia.
\section{Cathedral}
\begin{itemize}
\item {Grp. gram.:adj.}
\end{itemize}
\begin{itemize}
\item {Grp. gram.:F.}
\end{itemize}
Diz-se da igreja da séde de uma diocese.
Sé.
(B. lat. \textunderscore cathedralis\textunderscore )
\section{Cathedrátego}
\begin{itemize}
\item {Grp. gram.:m.}
\end{itemize}
\begin{itemize}
\item {Utilização:Ant.}
\end{itemize}
Censo ou pensão annual que as igrejas seculares pagavam ao seu Bispo, em testemunho de sujeição.
(Cp. \textunderscore cathedrático\textunderscore )
\section{Cathedráticamente}
\begin{itemize}
\item {Grp. gram.:adv.}
\end{itemize}
De modo cathedrático.
Autoritariamente.
\section{Cathedrático}
\begin{itemize}
\item {Grp. gram.:m.  e  adj.}
\end{itemize}
\begin{itemize}
\item {Grp. gram.:M.}
\end{itemize}
\begin{itemize}
\item {Utilização:Ant.}
\end{itemize}
\begin{itemize}
\item {Utilização:Ant.}
\end{itemize}
Professor effectivo de escolas superiores.
O mesmo que \textunderscore cathedrátego\textunderscore .
Propinas, que o Bispo dava, quando tomava posse do seu lugar.
(B. lat. \textunderscore cathedráticus\textunderscore )
\section{Cathedrilha}
\begin{itemize}
\item {Grp. gram.:f.}
\end{itemize}
(dem. de \textunderscore cáthedra\textunderscore )
\section{Cathérese}
\begin{itemize}
\item {Grp. gram.:f.}
\end{itemize}
\begin{itemize}
\item {Proveniência:(Do gr. \textunderscore kathairesis\textunderscore )}
\end{itemize}
Evacuação ou hemorragia natural.
\section{Catheretico}
\begin{itemize}
\item {Grp. gram.:m.  e  adj.}
\end{itemize}
\begin{itemize}
\item {Proveniência:(Gr. \textunderscore cathairetikos\textunderscore )}
\end{itemize}
Medicamento levemente cáustico, ou empregado em pequena quantidade.
\section{Catheter}
\begin{itemize}
\item {Grp. gram.:m.}
\end{itemize}
\begin{itemize}
\item {Proveniência:(Gr. \textunderscore katheter\textunderscore )}
\end{itemize}
Sonda, que se emprega na operação da talha.
\section{Catheterismo}
\begin{itemize}
\item {Grp. gram.:m.}
\end{itemize}
\begin{itemize}
\item {Proveniência:(Gr. \textunderscore katheterismos\textunderscore )}
\end{itemize}
Sondagem, por meio do catheter.
Qualquer sondagem cirúrgica.
\section{Catheterizar}
\begin{itemize}
\item {Grp. gram.:v. t.}
\end{itemize}
Sondar com o catheter.
\section{Catheto}
\begin{itemize}
\item {Grp. gram.:m.}
\end{itemize}
\begin{itemize}
\item {Utilização:Mathem.}
\end{itemize}
\begin{itemize}
\item {Utilização:Phýs.}
\end{itemize}
\begin{itemize}
\item {Proveniência:(Gr. \textunderscore kathetos\textunderscore )}
\end{itemize}
Linha perpendicular a outra ou a uma superfície.
Própriamente, cada um dos lados do ângulo recto no triângulo rectangular.
Raio luminoso, que incide ou é reflectido perpendicularmente.
\section{Cathetómetro}
\begin{itemize}
\item {Grp. gram.:m.}
\end{itemize}
\begin{itemize}
\item {Proveniência:(Do gr. \textunderscore kathete\textunderscore  + \textunderscore metron\textunderscore )}
\end{itemize}
Instrumento, para medir extensões verticaes.
\section{Catholicamente}
\begin{itemize}
\item {Grp. gram.:adv.}
\end{itemize}
De modo cathólico.
Segundo as leis do Catholicismo.
\section{Catholicão}
\begin{itemize}
\item {Grp. gram.:m.}
\end{itemize}
\begin{itemize}
\item {Utilização:Deprec.}
\end{itemize}
\begin{itemize}
\item {Proveniência:(De \textunderscore cathólico\textunderscore )}
\end{itemize}
Antiga panaceia purgativa.
Indivíduo, exaggeradamente cathólico.
\section{Catholicidade}
\begin{itemize}
\item {Grp. gram.:f.}
\end{itemize}
\begin{itemize}
\item {Proveniência:(De \textunderscore cathólico\textunderscore )}
\end{itemize}
Universalidade, carácter da Igreja Catholica.
Qualidade do que é cathólico.
Conformidade com o Catholicismo.
\section{Catholicismo}
\begin{itemize}
\item {Grp. gram.:m.}
\end{itemize}
\begin{itemize}
\item {Proveniência:(De \textunderscore cathólico\textunderscore )}
\end{itemize}
Religião cathólica; o mundo cathólico.
\section{Cathólico}
\begin{itemize}
\item {Grp. gram.:adj.}
\end{itemize}
\begin{itemize}
\item {Utilização:Prov.}
\end{itemize}
\begin{itemize}
\item {Utilização:trasm.}
\end{itemize}
\begin{itemize}
\item {Grp. gram.:M.}
\end{itemize}
\begin{itemize}
\item {Proveniência:(Lat. \textunderscore catholicus\textunderscore )}
\end{itemize}
Universal.
Relativo á religião romana.
Que fala muito, que é tagarela.
Aquelle que segue a religião, de que é chefe o Pontífice romano.
Antiga moéda da Índia portuguesa.
O primaz da Igreja armênia e de outras Igrejas asiáticas.
Cobrador de impostos, no Império Romano do Oriente.
\section{Catholização}
\begin{itemize}
\item {Grp. gram.:f.}
\end{itemize}
Acto de \textunderscore catholizar\textunderscore .
\section{Catholizar}
\begin{itemize}
\item {Utilização:Neol.}
\end{itemize}
Tornar cathólico. Cf. Th. Ribeiro, I, 227 e 254.
(Palavra mal derivada de \textunderscore cathólico\textunderscore )
\section{Catilinária}
\begin{itemize}
\item {Grp. gram.:f.}
\end{itemize}
\begin{itemize}
\item {Proveniência:(Lat. \textunderscore catilinaria\textunderscore , fem. de \textunderscore catilinarius\textunderscore , de \textunderscore Catilina\textunderscore , n. p.)}
\end{itemize}
Accusação vehemente; censura acerba.
\section{Catimbau}
\begin{itemize}
\item {Grp. gram.:m.}
\end{itemize}
\begin{itemize}
\item {Utilização:Bras}
\end{itemize}
\begin{itemize}
\item {Grp. gram.:M. e adj.}
\end{itemize}
\begin{itemize}
\item {Utilização:Des.}
\end{itemize}
Cachimbo reles.
Aquelle que faz momices para divertir o vulgo; chocarreiro; gracioso.
(Do tupi)
\section{Catimpuera}
\begin{itemize}
\item {Grp. gram.:f.}
\end{itemize}
\begin{itemize}
\item {Utilização:Bras}
\end{itemize}
Espécie de bebida fermentada, feita de mandioca.
(Do tupi)
\section{Catinga}
\begin{itemize}
\item {Grp. gram.:f.}
\end{itemize}
\begin{itemize}
\item {Grp. gram.:M.}
\end{itemize}
Cheiro desagradável da pelle dos Negros.
Transpiração mal cheirosa.
Nome de várias plantas brasileiras.
Homem avarento.
(Do tupi)
\section{Catinga}
\begin{itemize}
\item {Grp. gram.:f.}
\end{itemize}
\begin{itemize}
\item {Utilização:Bras}
\end{itemize}
Mata enfezada de arvoretas tortuosas e um tanto raras.
(Do tupi \textunderscore caá\textunderscore  + \textunderscore tininga\textunderscore , arvoredo sêco)
\section{Catinga-branca}
\begin{itemize}
\item {Grp. gram.:f.}
\end{itemize}
Arbusto medicinal do Brasil.
\section{Catinga-de-bode}
\begin{itemize}
\item {Grp. gram.:f.}
\end{itemize}
\begin{itemize}
\item {Utilização:Bras. do N}
\end{itemize}
Planta medicinal.
\section{Catinga-de-porco}
\begin{itemize}
\item {Grp. gram.:f.}
\end{itemize}
Árvore silvestre do Brasil.
\section{Catingante}
\begin{itemize}
\item {Grp. gram.:adj.}
\end{itemize}
O mesmo que \textunderscore catingoso\textunderscore .
\section{Catingar}
\begin{itemize}
\item {Grp. gram.:v. i.}
\end{itemize}
\begin{itemize}
\item {Utilização:Bras}
\end{itemize}
\begin{itemize}
\item {Proveniência:(De \textunderscore catinga\textunderscore ^1)}
\end{itemize}
Regatear, mostrar-se avaro.
Cheirar mal.
\section{Catingoso}
\begin{itemize}
\item {Grp. gram.:adj.}
\end{itemize}
\begin{itemize}
\item {Utilização:Bras}
\end{itemize}
\begin{itemize}
\item {Proveniência:(De \textunderscore catinga\textunderscore ^1)}
\end{itemize}
Que exhala mau cheiro.
\section{Catinguá}
\begin{itemize}
\item {Grp. gram.:m.}
\end{itemize}
\begin{itemize}
\item {Utilização:Bras}
\end{itemize}
Árvore meliácea.
\section{Catingueira}
\begin{itemize}
\item {Grp. gram.:f.}
\end{itemize}
\begin{itemize}
\item {Utilização:Bras}
\end{itemize}
O mesmo que \textunderscore cretone\textunderscore .
\section{Catingueiro}
\begin{itemize}
\item {Grp. gram.:adj.}
\end{itemize}
\begin{itemize}
\item {Utilização:Fam.}
\end{itemize}
\begin{itemize}
\item {Grp. gram.:M.}
\end{itemize}
\begin{itemize}
\item {Utilização:Bras}
\end{itemize}
Que tem catinga^1.
Que cheira mal.
Avarento.
Espécie de veado.
\section{Catingueiro}
\begin{itemize}
\item {Grp. gram.:adj.}
\end{itemize}
\begin{itemize}
\item {Utilização:Bras. do N}
\end{itemize}
\begin{itemize}
\item {Proveniência:(De \textunderscore catinga\textunderscore ^2)}
\end{itemize}
Que habita as matas, chamadas catingas.
Espécie de veado.
\section{Catinguento}
\begin{itemize}
\item {Grp. gram.:adj.}
\end{itemize}
\begin{itemize}
\item {Utilização:Bras}
\end{itemize}
O mesmo que \textunderscore catingoso\textunderscore .
\section{Catininga}
\begin{itemize}
\item {Grp. gram.:f.}
\end{itemize}
\begin{itemize}
\item {Utilização:Bras}
\end{itemize}
O mesmo que \textunderscore pixirica\textunderscore .
\section{Catita}
\begin{itemize}
\item {Grp. gram.:f.}
\end{itemize}
\begin{itemize}
\item {Utilização:Bras. do N}
\end{itemize}
Calaboiço, prisão.
\section{Catita}
\begin{itemize}
\item {Grp. gram.:f.}
\end{itemize}
\begin{itemize}
\item {Utilização:Prov.}
\end{itemize}
\begin{itemize}
\item {Utilização:alg.}
\end{itemize}
\begin{itemize}
\item {Utilização:Náut.}
\end{itemize}
\begin{itemize}
\item {Utilização:T. de Alcácer}
\end{itemize}
\begin{itemize}
\item {Grp. gram.:M.  e  f.}
\end{itemize}
\begin{itemize}
\item {Grp. gram.:Adj.}
\end{itemize}
\begin{itemize}
\item {Utilização:Fam.}
\end{itemize}
Espécie de giga, usada por peixeiros.
Vela pequena de popa, usada em chalupas e outros pequenos barcos.
Cada uma das ilhotas e ínsuas do Sado.
Pessôa elegante ou bem vestida.
Garrido, casquilho.
Formoso.
\section{Catitar}
\begin{itemize}
\item {Grp. gram.:v. i.}
\end{itemize}
Mostrar-se catita, janotar.
\section{Catitice}
\begin{itemize}
\item {Grp. gram.:f.}
\end{itemize}
O mesmo que \textunderscore catitismo\textunderscore .
\section{Catitismo}
\begin{itemize}
\item {Grp. gram.:m.}
\end{itemize}
\begin{itemize}
\item {Utilização:Fam.}
\end{itemize}
Elegância.
Janotismo.
\section{Catitu}
\begin{itemize}
\item {Grp. gram.:m.}
\end{itemize}
(V.caititu)
\section{Cativa}
\begin{itemize}
\item {Grp. gram.:f.}
\end{itemize}
\begin{itemize}
\item {Proveniência:(De \textunderscore cativo\textunderscore )}
\end{itemize}
Dança moirisca, que se usou na península hispânica, e em que entravam oito moiros agrilhoados. Cf. \textunderscore Panorama\textunderscore , V, 309.
\section{Cativa!}
\begin{itemize}
\item {Grp. gram.:interj.}
\end{itemize}
\begin{itemize}
\item {Utilização:Prov.}
\end{itemize}
\begin{itemize}
\item {Utilização:minh.}
\end{itemize}
(Designa \textunderscore asco\textunderscore , \textunderscore nojo\textunderscore , etc.)
(Cp. \textunderscore catixa!\textunderscore )
\section{Cativação}
\begin{itemize}
\item {Grp. gram.:f.}
\end{itemize}
Acto de cativar.
\section{Cativante}
\begin{itemize}
\item {Grp. gram.:adj.}
\end{itemize}
Que cativa.
Que desperta muita gratidão.
\section{Cativar}
\begin{itemize}
\item {Grp. gram.:v. t.}
\end{itemize}
\begin{itemize}
\item {Proveniência:(Lat. \textunderscore captivare\textunderscore )}
\end{itemize}
Tornar cativo.
Alliciar; seduzir.
Dominar.
Tornar muito agradecido: \textunderscore é favor que muito me cativa\textunderscore .
\section{Cativeiro}
\begin{itemize}
\item {Grp. gram.:m.}
\end{itemize}
\begin{itemize}
\item {Utilização:Ext.}
\end{itemize}
\begin{itemize}
\item {Proveniência:(De \textunderscore cativo\textunderscore )}
\end{itemize}
Estado de quem é cativo.
Lugar, onde se está cativo.
Clausura.
Perda da liberdade.
\section{Cativeza}
\begin{itemize}
\item {Grp. gram.:f.}
\end{itemize}
\begin{itemize}
\item {Utilização:Ant.}
\end{itemize}
\begin{itemize}
\item {Proveniência:(De \textunderscore cativo\textunderscore )}
\end{itemize}
Acanhamento, timidez.
\section{Catividade}
\begin{itemize}
\item {Grp. gram.:f.}
\end{itemize}
\begin{itemize}
\item {Utilização:Des.}
\end{itemize}
O mesmo que \textunderscore cativeiro\textunderscore .
\section{Cativo}
\begin{itemize}
\item {Grp. gram.:adj.}
\end{itemize}
\begin{itemize}
\item {Grp. gram.:M.}
\end{itemize}
\begin{itemize}
\item {Proveniência:(Lat. \textunderscore captivus\textunderscore )}
\end{itemize}
Prisioneiro de guerra.
Obrigado á escravidão.
Aprehendido, preso, encarcerado.
Dominado, seduzido, attrahido: \textunderscore sentiu-se cativo dos encantos della\textunderscore .
Onerado: \textunderscore prédio cativo de um foro\textunderscore .
Que desbota facilmente.
Aquelle que é prisioneiro.
Escravo.
\textunderscore Bacelleiros de cativo\textunderscore , aquelles que não têm raízes, ao contrário dos bacelleiros barbados.
\section{Catixa!}
\begin{itemize}
\item {Grp. gram.:interj.}
\end{itemize}
\begin{itemize}
\item {Utilização:Prov.}
\end{itemize}
\begin{itemize}
\item {Utilização:minh.}
\end{itemize}
O mesmo que \textunderscore cativa!\textunderscore 
\section{Catle}
\begin{itemize}
\item {Grp. gram.:m.}
\end{itemize}
\begin{itemize}
\item {Utilização:Ant.}
\end{itemize}
O mesmo que \textunderscore catre\textunderscore .
Camilha dobradiça.
\section{Catoblepa}
\begin{itemize}
\item {Grp. gram.:f.}
\end{itemize}
Antílope dos sertões do Congo.
\section{Catocar}
\begin{itemize}
\item {Grp. gram.:v. t.}
\end{itemize}
\begin{itemize}
\item {Utilização:Bras}
\end{itemize}
Dar sinal a, tocando com o cotovelo, com o pé ou com a mão.
Chamar a attenção de.
(Talvez corr. de \textunderscore cutucar\textunderscore )
\section{Catodal}
\begin{itemize}
\item {Grp. gram.:adj.}
\end{itemize}
O mesmo que \textunderscore catódico\textunderscore . Cf. Verg. Machado, \textunderscore Raios\textunderscore , X.
\section{Catódico}
\begin{itemize}
\item {Grp. gram.:adj.}
\end{itemize}
Relativo ao catodo.
O mesmo que \textunderscore catodo\textunderscore , adj.
\section{Catodo}
\begin{itemize}
\item {Grp. gram.:m.  e  adj.}
\end{itemize}
\begin{itemize}
\item {Utilização:Phýs.}
\end{itemize}
\begin{itemize}
\item {Proveniência:(Do gr. \textunderscore kata\textunderscore  + \textunderscore odos\textunderscore )}
\end{itemize}
Dá-se este nome ao electródio negativo, (cp. \textunderscore electródio\textunderscore ); e diz-se do raio invisível, que penetra os corpos opacos e que determinou o recente processo photográphico de R[oe]ntgen.
\section{Catódio}
\begin{itemize}
\item {Grp. gram.:m.  e  adj.}
\end{itemize}
O mesmo ou melhor que \textunderscore catodo\textunderscore .
\section{Catodonte}
\begin{itemize}
\item {Grp. gram.:m.}
\end{itemize}
Espécie de baleia, que só tem dentes na maxilla inferior.
\section{Catolé}
\begin{itemize}
\item {Grp. gram.:m.}
\end{itemize}
Palmeira silvestre do Brasil.
Fruto dessa árvore.
\section{Catolereiro}
\begin{itemize}
\item {Grp. gram.:m.}
\end{itemize}
Palmeira, o mesmo que \textunderscore catolé\textunderscore .
\section{Católi}
\begin{itemize}
\item {Grp. gram.:m.}
\end{itemize}
Arbusto angolense, de fôlhas muito verdes e flôres amarelas.
\section{Catolicamente}
\begin{itemize}
\item {Grp. gram.:adv.}
\end{itemize}
De modo católico.
Segundo as leis do Catolicismo.
\section{Catolicão}
\begin{itemize}
\item {Grp. gram.:m.}
\end{itemize}
\begin{itemize}
\item {Utilização:Deprec.}
\end{itemize}
\begin{itemize}
\item {Proveniência:(De \textunderscore cathólico\textunderscore )}
\end{itemize}
Antiga panaceia purgativa.
Indivíduo, exaggeradamente católico.
\section{Catolicidade}
\begin{itemize}
\item {Grp. gram.:f.}
\end{itemize}
\begin{itemize}
\item {Proveniência:(De \textunderscore cathólico\textunderscore )}
\end{itemize}
Universalidade, carácter da Igreja Catolica.
Qualidade do que é católico.
Conformidade com o Catolicismo.
\section{Catolicismo}
\begin{itemize}
\item {Grp. gram.:m.}
\end{itemize}
\begin{itemize}
\item {Proveniência:(De \textunderscore cathólico\textunderscore )}
\end{itemize}
Religião católica; o mundo católico.
\section{Católico}
\begin{itemize}
\item {Grp. gram.:adj.}
\end{itemize}
\begin{itemize}
\item {Utilização:Prov.}
\end{itemize}
\begin{itemize}
\item {Utilização:trasm.}
\end{itemize}
\begin{itemize}
\item {Grp. gram.:M.}
\end{itemize}
\begin{itemize}
\item {Proveniência:(Lat. \textunderscore catholicus\textunderscore )}
\end{itemize}
Universal.
Relativo á religião romana.
Que fala muito, que é tagarela.
Aquelle que segue a religião, de que é chefe o Pontífice romano.
Antiga moéda da Índia portuguesa.
O primaz da Igreja armênia e de outras Igrejas asiáticas.
Cobrador de impostos, no Império Romano do Oriente.
\section{Católi-silvestre}
\begin{itemize}
\item {Grp. gram.:m.}
\end{itemize}
Arbusto annual de Angola.
\section{Catolização}
\begin{itemize}
\item {Grp. gram.:f.}
\end{itemize}
Acto de \textunderscore catolizar\textunderscore .
\section{Catolizar}
\begin{itemize}
\item {Utilização:Neol.}
\end{itemize}
Tornar católico. Cf. Th. Ribeiro, I, 227 e 254.
(Palavra mal derivada de \textunderscore cathólico\textunderscore )
\section{Catombe}
\begin{itemize}
\item {Grp. gram.:m.}
\end{itemize}
Arbusto angolense.
\section{Catombo}
\begin{itemize}
\item {Grp. gram.:m.}
\end{itemize}
\begin{itemize}
\item {Utilização:Bras. do N}
\end{itemize}
Inchaço; tumor.
\section{Catoniano}
\begin{itemize}
\item {Grp. gram.:adj.}
\end{itemize}
\begin{itemize}
\item {Proveniência:(Lat. \textunderscore catonianus\textunderscore )}
\end{itemize}
Próprio de catão.
\section{Catónico}
\begin{itemize}
\item {Grp. gram.:adj.}
\end{itemize}
O mesmo que \textunderscore catoniano\textunderscore .
\section{Catonismo}
\begin{itemize}
\item {Grp. gram.:m.}
\end{itemize}
Qualidade de quem é catão.
\section{Catópodes}
\begin{itemize}
\item {Grp. gram.:m. pl.}
\end{itemize}
\begin{itemize}
\item {Proveniência:(Do gr. \textunderscore kata\textunderscore  + \textunderscore pous\textunderscore , \textunderscore podos\textunderscore )}
\end{itemize}
Ordem de peixes, que têm barbatanas no ventre.
\section{Catóptrica}
\begin{itemize}
\item {Grp. gram.:f.}
\end{itemize}
\begin{itemize}
\item {Utilização:Phýs.}
\end{itemize}
\begin{itemize}
\item {Proveniência:(De \textunderscore catóptrico\textunderscore )}
\end{itemize}
Tratado da reflexão dos raios luminosos.
\section{Catóptrico}
\begin{itemize}
\item {Grp. gram.:adj.}
\end{itemize}
\begin{itemize}
\item {Proveniência:(Gr. \textunderscore katoptrikos\textunderscore )}
\end{itemize}
Relativo á reflexão da luz.
\section{Catoptromancia}
\begin{itemize}
\item {Grp. gram.:f.}
\end{itemize}
\begin{itemize}
\item {Proveniência:(Do gr. \textunderscore katoptron\textunderscore  + \textunderscore manteia\textunderscore )}
\end{itemize}
Adivinhação por meio de um espelho.
\section{Catoptromântico}
\begin{itemize}
\item {Grp. gram.:adj.}
\end{itemize}
Relativo á catoptromancia.
\section{Catóri}
\begin{itemize}
\item {Grp. gram.:m.}
\end{itemize}
Arbusto angolense, annual e rasteiro, cujas fôlhas cozidas são applicadas pelos indígenas contra as dores de estômago.
\section{Catorrita}
\begin{itemize}
\item {Grp. gram.:f.}
\end{itemize}
\begin{itemize}
\item {Utilização:Bras}
\end{itemize}
Espécie de periquito.
\section{Catorze}
\begin{itemize}
\item {Grp. gram.:adj.}
\end{itemize}
O mesmo ou melhor que \textunderscore quatorze\textunderscore .
\section{Catoscópio}
\begin{itemize}
\item {Grp. gram.:m.}
\end{itemize}
\begin{itemize}
\item {Proveniência:(Do gr. \textunderscore kata\textunderscore  + \textunderscore skopein\textunderscore )}
\end{itemize}
Gênero de musgos, que crescem em pontos elevados.
\section{Catota}
\begin{itemize}
\item {Grp. gram.:f.}
\end{itemize}
Planta solanácea do Brasil.
\section{Catotol}
\begin{itemize}
\item {Grp. gram.:m.}
\end{itemize}
Pequena ave do Brasil.
\section{Catraeiro}
\begin{itemize}
\item {Grp. gram.:m.}
\end{itemize}
\begin{itemize}
\item {Proveniência:(De \textunderscore catraia\textunderscore )}
\end{itemize}
Aquelle que tripula uma catraia; barqueiro.
\section{Catrafiar}
\begin{itemize}
\item {Grp. gram.:v. t.}
\end{itemize}
\begin{itemize}
\item {Utilização:Pop.}
\end{itemize}
O mesmo que \textunderscore catrafilar\textunderscore .
\section{Catrafilar}
\begin{itemize}
\item {Grp. gram.:v. t.}
\end{itemize}
\begin{itemize}
\item {Utilização:Pop.}
\end{itemize}
\begin{itemize}
\item {Proveniência:(De um pref. caprichoso e \textunderscore filar\textunderscore )}
\end{itemize}
Agarrar; prender; encarcerar.
\section{Catraia}
\begin{itemize}
\item {Grp. gram.:f.}
\end{itemize}
\begin{itemize}
\item {Utilização:Prov.}
\end{itemize}
Pequeno barco, tripulado por um homem.
Pequena construcção.
Casinhola, baiuca, taberna.
\section{Catraiar}
\begin{itemize}
\item {Grp. gram.:v. i.}
\end{itemize}
Fazer transportes em catraio.
\section{Catrâimbas}
\begin{itemize}
\item {Grp. gram.:f. pl.}
\end{itemize}
O mesmo que \textunderscore catrâmbias!\textunderscore 
\section{Catraio}
\begin{itemize}
\item {Grp. gram.:m.}
\end{itemize}
\begin{itemize}
\item {Utilização:Gír.}
\end{itemize}
O mesmo que \textunderscore catraia\textunderscore , (barco).
Gaiato; criança traquinas.
\section{Catrâmbias!}
\begin{itemize}
\item {Grp. gram.:interj.}
\end{itemize}
\begin{itemize}
\item {Utilização:fam.}
\end{itemize}
\begin{itemize}
\item {Grp. gram.:Loc. adv.}
\end{itemize}
\begin{itemize}
\item {Utilização:Prov.}
\end{itemize}
\begin{itemize}
\item {Utilização:trasm.}
\end{itemize}
\begin{itemize}
\item {Grp. gram.:F. pl.}
\end{itemize}
Bólas!
Cebolório!«\textunderscore Catrâmbias para gente de semelhante feitio!\textunderscore »Castilho, in \textunderscore Rev. da Acad. Bras.\textunderscore , III, 24.
\textunderscore De catrâmbias\textunderscore , de pernas para o ar.
De bruços.
Cambalhotas.
Carranca: \textunderscore andar de catrâmbias\textunderscore .
\section{Catrame}
\begin{itemize}
\item {Grp. gram.:m.}
\end{itemize}
\begin{itemize}
\item {Proveniência:(It. \textunderscore catrame\textunderscore )}
\end{itemize}
Espécie de pez, alcatrão.
\section{Catrameço}
\begin{itemize}
\item {fónica:mê}
\end{itemize}
\begin{itemize}
\item {Grp. gram.:m.}
\end{itemize}
\begin{itemize}
\item {Utilização:Prov.}
\end{itemize}
\begin{itemize}
\item {Utilização:trasm.}
\end{itemize}
Grande pedaço; tracanaz.
\section{Catramonho}
\begin{itemize}
\item {Grp. gram.:m.}
\end{itemize}
\begin{itemize}
\item {Utilização:Prov.}
\end{itemize}
\begin{itemize}
\item {Utilização:alent.}
\end{itemize}
Mólho, mal atado.
\section{Catrapão}
\begin{itemize}
\item {Grp. gram.:m.}
\end{itemize}
\begin{itemize}
\item {Utilização:T. do Funchal}
\end{itemize}
Cavalgadura pesada, feia e de mau passo.
\section{Catrapeanha}
\begin{itemize}
\item {Grp. gram.:f.}
\end{itemize}
\begin{itemize}
\item {Proveniência:(De \textunderscore quatro\textunderscore  + \textunderscore peanha\textunderscore , porque o respectivo tear tem quatro peanhas ou pedaes)}
\end{itemize}
Saragoça entrançada, tecido grosseiro, usado em saias por mulher do campo. Cf. \textunderscore Inquér. Industr.\textunderscore , P. II, l. 3.^o, 147.
\section{Catrapeço}
\begin{itemize}
\item {fónica:pê}
\end{itemize}
\begin{itemize}
\item {Grp. gram.:m.}
\end{itemize}
\begin{itemize}
\item {Utilização:Prov.}
\end{itemize}
\begin{itemize}
\item {Utilização:trasm.}
\end{itemize}
O mesmo que \textunderscore catrameço\textunderscore .
\section{Catrapéu}
\begin{itemize}
\item {Grp. gram.:m.}
\end{itemize}
\begin{itemize}
\item {Utilização:ant.}
\end{itemize}
\begin{itemize}
\item {Utilização:Gír.}
\end{itemize}
Cavallo.
(Cp. \textunderscore catrapós!\textunderscore )
\section{Catrapiscar}
\begin{itemize}
\item {Grp. gram.:v. t.  e  i.}
\end{itemize}
\begin{itemize}
\item {Utilização:Pop.}
\end{itemize}
\begin{itemize}
\item {Utilização:Prov.}
\end{itemize}
\begin{itemize}
\item {Utilização:beir.}
\end{itemize}
Namorar, piscando o ôlho.
Perceber, comprehender.
\section{Catrapizonga}
\begin{itemize}
\item {Grp. gram.:f.}
\end{itemize}
\begin{itemize}
\item {Utilização:T. da Bairrada}
\end{itemize}
Homem pesado e gordo, que, andando, arrasta os pés.
\section{Catrapós!}
\begin{itemize}
\item {Grp. gram.:m.  e  interj.}
\end{itemize}
\begin{itemize}
\item {Proveniência:(T. onom.)}
\end{itemize}
O galopar do cavallo.
Voz imitativa do galopar.
Voz imitativa de quéda ruidosa e repentina.
\section{Catrapus!}
\begin{itemize}
\item {Grp. gram.:m.  e  interj.}
\end{itemize}
\begin{itemize}
\item {Proveniência:(T. onom.)}
\end{itemize}
O galopar do cavallo.
Voz imitativa do galopar.
Voz imitativa de quéda ruidosa e repentina.
\section{Catre}
\begin{itemize}
\item {Grp. gram.:m.}
\end{itemize}
Camilha dobradiça.
Cama de viagem.
Leito tôsco e pobre.
(Cast. \textunderscore catre\textunderscore )
\section{Catrefa}
\begin{itemize}
\item {Grp. gram.:f.}
\end{itemize}
\begin{itemize}
\item {Utilização:Pop.}
\end{itemize}
(Corr. de \textunderscore caterva\textunderscore )
\section{Catrinas}
\begin{itemize}
\item {Grp. gram.:f. pl.}
\end{itemize}
O mesmo que [[catharinas|catharina]].
\section{Catrofa}
\begin{itemize}
\item {Grp. gram.:f.}
\end{itemize}
\begin{itemize}
\item {Utilização:Prov.}
\end{itemize}
\begin{itemize}
\item {Utilização:trasm.}
\end{itemize}
Parte posterior da cabeça; nuca.
\section{Catróio}
\begin{itemize}
\item {Grp. gram.:m.}
\end{itemize}
\begin{itemize}
\item {Utilização:Gír.}
\end{itemize}
Cavalgadura.
\section{Catropomancia}
\begin{itemize}
\item {Grp. gram.:f.}
\end{itemize}
Arte de adivinhar, por meio de espelhos encantados, para se conhecer o termo de uma doença. Cf. Castilho, \textunderscore Fastos\textunderscore , III, 323.
(Corr. de \textunderscore catoptromancia\textunderscore ?)
\section{Catrozada}
\begin{itemize}
\item {Grp. gram.:f.}
\end{itemize}
\begin{itemize}
\item {Utilização:Pop.}
\end{itemize}
Grande porção.
(Por \textunderscore catorzada\textunderscore , de \textunderscore catorze\textunderscore )
\section{Catruchas}
\begin{itemize}
\item {Grp. gram.:f. pl.}
\end{itemize}
\begin{itemize}
\item {Utilização:Gír.}
\end{itemize}
Botas altas, contra a chuva.
(Refl. de \textunderscore cartucho\textunderscore , com metáth.?)
\section{Catrunha}
\begin{itemize}
\item {Grp. gram.:f.}
\end{itemize}
\begin{itemize}
\item {Utilização:trasm}
\end{itemize}
\begin{itemize}
\item {Utilização:Gír.}
\end{itemize}
Cabeça da gente.
(Cp. \textunderscore catrofa\textunderscore )
\section{Catrus}
\begin{itemize}
\item {Grp. gram.:m.}
\end{itemize}
\begin{itemize}
\item {Utilização:Bras}
\end{itemize}
Madeira de construcção.
\section{Catuaba}
\begin{itemize}
\item {Grp. gram.:f.}
\end{itemize}
Árvore do Pará, de cujas fôlhas se faz um chá aphrodisíaco.
\section{Catual}
\begin{itemize}
\item {Grp. gram.:m.}
\end{itemize}
\begin{itemize}
\item {Proveniência:(Do pers. \textunderscore katual\textunderscore )}
\end{itemize}
Funccionário público em alguns povos do Oriente; intendente de negócios com os estrangeiros.
\section{Catualia}
\begin{itemize}
\item {Grp. gram.:f.}
\end{itemize}
Jurisdição de catual.
\section{Catucar}
\begin{itemize}
\item {Grp. gram.:v. t.}
\end{itemize}
O mesmo ou melhor que \textunderscore catocar\textunderscore .
(Cp. \textunderscore cutucar\textunderscore )
\section{Catullo}
\begin{itemize}
\item {Grp. gram.:m.}
\end{itemize}
O mesmo que \textunderscore tarrantana\textunderscore .
\section{Catulo}
\begin{itemize}
\item {Grp. gram.:m.}
\end{itemize}
O mesmo que \textunderscore tarrantana\textunderscore .
\section{Catulo}
\begin{itemize}
\item {Grp. gram.:m.}
\end{itemize}
\begin{itemize}
\item {Utilização:Prov.}
\end{itemize}
O mesmo que \textunderscore cogulo\textunderscore : \textunderscore medida de catulo\textunderscore .
\section{Cátulo}
\begin{itemize}
\item {Grp. gram.:m.}
\end{itemize}
\begin{itemize}
\item {Utilização:Poét.}
\end{itemize}
\begin{itemize}
\item {Proveniência:(Lat. \textunderscore catulus\textunderscore )}
\end{itemize}
Cãozinho, cachorro. Cf. \textunderscore Ulysseia\textunderscore , VI, 51.
\section{Catumbas}
\begin{itemize}
\item {Grp. gram.:f. pl.}
\end{itemize}
\begin{itemize}
\item {Utilização:Des.}
\end{itemize}
O mesmo que \textunderscore catacumbas\textunderscore .
\section{Catulu}
\begin{itemize}
\item {Grp. gram.:m.}
\end{itemize}
Arbusto angolense, de fôlhas medicinaes.
\section{Catupé}
\begin{itemize}
\item {Grp. gram.:m.}
\end{itemize}
Antiga dança brasileira.
\section{Catuquinas}
\begin{itemize}
\item {Grp. gram.:m. pl.}
\end{itemize}
Indígenas do Brasil, nas margens do Iuruá.
\section{Catur}
\begin{itemize}
\item {Grp. gram.:m.}
\end{itemize}
\begin{itemize}
\item {Proveniência:(Do ár. \textunderscore katireh\textunderscore ?)}
\end{itemize}
Pequena embarcação indiana.
\section{Catureiro}
\begin{itemize}
\item {Grp. gram.:m.}
\end{itemize}
Tripulante de um catur.
\section{Caturna}
\begin{itemize}
\item {Grp. gram.:f.}
\end{itemize}
\begin{itemize}
\item {Utilização:T. do Porto}
\end{itemize}
O mesmo que \textunderscore caturno\textunderscore .
\section{Caturno}
\begin{itemize}
\item {Grp. gram.:m.}
\end{itemize}
\begin{itemize}
\item {Utilização:Prov.}
\end{itemize}
\begin{itemize}
\item {Utilização:beir.}
\end{itemize}
O mesmo que \textunderscore peúga\textunderscore  ou meia para homem.
(Cp. \textunderscore cothurno\textunderscore )
\section{Caturra}
\begin{itemize}
\item {Grp. gram.:m.  e  f.}
\end{itemize}
Pessôa teimosa, aferrada a ideias antigas, diffícil de se satisfazer, achando defeitos amiúde e comprazendo-se em discutir.
\section{Caturrar}
\begin{itemize}
\item {Grp. gram.:v. i.}
\end{itemize}
\begin{itemize}
\item {Utilização:Náut.}
\end{itemize}
Mostrar-se caturra; discutir com insistência; teimar.
Baloiçar-se o navio em marcha, andando muito pouco; arfar.
\section{Caturreira}
\begin{itemize}
\item {Grp. gram.:f.}
\end{itemize}
Discussão, qualidade ou acto de caturra.
\section{Caturrice}
\begin{itemize}
\item {Grp. gram.:f.}
\end{itemize}
Discussão, qualidade ou acto de caturra.
\section{Caturrismo}
\begin{itemize}
\item {Grp. gram.:m.}
\end{itemize}
Palavras ou ideias de caturra.
\section{Caturro}
\begin{itemize}
\item {Grp. gram.:m.}
\end{itemize}
\begin{itemize}
\item {Utilização:Prov.}
\end{itemize}
\begin{itemize}
\item {Utilização:alent.}
\end{itemize}
Pequeno cachimbo, grosso e curto.
\section{Catututu}
\begin{itemize}
\item {Grp. gram.:m.}
\end{itemize}
O mesmo que \textunderscore mututu\textunderscore .
\section{Cauaba}
\begin{itemize}
\item {Grp. gram.:f.}
\end{itemize}
\begin{itemize}
\item {Utilização:Bras}
\end{itemize}
Vasilha, que contém o cauim.
\section{Cauaçu}
\begin{itemize}
\item {Grp. gram.:m.}
\end{itemize}
\begin{itemize}
\item {Utilização:Bras. do N}
\end{itemize}
Espécie de palmeira.
\section{Cauan}
\begin{itemize}
\item {Grp. gram.:m.}
\end{itemize}
\begin{itemize}
\item {Utilização:Bras}
\end{itemize}
Espécie de gavião.
\section{Cauana}
\begin{itemize}
\item {fónica:ca-u}
\end{itemize}
\begin{itemize}
\item {Grp. gram.:f.}
\end{itemize}
Espécie de tartaruga.
\section{Cauanas}
\begin{itemize}
\item {Grp. gram.:m. pl.}
\end{itemize}
Indígenas do Brasil, nas margens do Juruá.
\section{Cauaracaá}
\begin{itemize}
\item {Grp. gram.:m.}
\end{itemize}
Árvore medicinal do Alto-Amazonas.
\section{Cauaris}
\begin{itemize}
\item {Grp. gram.:m. pl.}
\end{itemize}
Indígenas do Brasil, nas margens do Juruá.
\section{Cauaxis}
\begin{itemize}
\item {Grp. gram.:m. pl.}
\end{itemize}
Indígenas do Brasil, nas margens do Juruá.
\section{Caubi}
\begin{itemize}
\item {Grp. gram.:m.}
\end{itemize}
\begin{itemize}
\item {Utilização:Bras}
\end{itemize}
O mesmo que \textunderscore pau-carga\textunderscore .
\section{Caubila}
\begin{itemize}
\item {Grp. gram.:m.  e  adj.}
\end{itemize}
\begin{itemize}
\item {Utilização:Bras}
\end{itemize}
Pessôa avarenta, sovina.
\section{Caucaá}
\begin{itemize}
\item {Grp. gram.:m.}
\end{itemize}
Erva medicinal do Alto-Amazonas.
\section{Caução}
\begin{itemize}
\item {Grp. gram.:f.}
\end{itemize}
\begin{itemize}
\item {Proveniência:(Lat. \textunderscore cautio\textunderscore )}
\end{itemize}
Cautela.
Garantia; segurança; responsabilidade.
Valores depositados ou acceitos para garantia de um contrato ou para tornar effectiva a responsabilidade de um encargo.
\section{Caucasiano}
\begin{itemize}
\item {Grp. gram.:adj.}
\end{itemize}
O mesmo que \textunderscore caucásico\textunderscore .
\section{Caucásico}
\begin{itemize}
\item {Grp. gram.:adj.}
\end{itemize}
Relativo ao Cáucaso.
\section{Caucásio}
\begin{itemize}
\item {Grp. gram.:adj.}
\end{itemize}
O mesmo que \textunderscore caucásico\textunderscore . Cf. Castilho, \textunderscore Amor e Melanc.\textunderscore , 328.
\section{Caucela}
\begin{itemize}
\item {Grp. gram.:f.}
\end{itemize}
\begin{itemize}
\item {Utilização:Ant.}
\end{itemize}
\begin{itemize}
\item {Proveniência:(Do lat. \textunderscore capsella\textunderscore )}
\end{itemize}
Vaso sagrado, o mesmo que \textunderscore pýxide\textunderscore . Cp. \textunderscore causela\textunderscore .
\section{Caucheiro}
\begin{itemize}
\item {fónica:ca-u}
\end{itemize}
\begin{itemize}
\item {Grp. gram.:m.}
\end{itemize}
\begin{itemize}
\item {Utilização:Bras}
\end{itemize}
O mesmo que \textunderscore seringueiro\textunderscore .
\section{Cauchins}
\begin{itemize}
\item {Grp. gram.:m. pl.}
\end{itemize}
\begin{itemize}
\item {Utilização:Ant.}
\end{itemize}
Habitantes da Cochinchina. Cp. \textunderscore Peregrinação\textunderscore , XLV.
\section{Cauchu}
\begin{itemize}
\item {Grp. gram.:m.}
\end{itemize}
Árvore da borracha; gomma elástica.
(Or. amer.)
\section{Caucionante}
\begin{itemize}
\item {Grp. gram.:m.  e  adj.}
\end{itemize}
O que cauciona.
\section{Caucionar}
\begin{itemize}
\item {Grp. gram.:v. t.}
\end{itemize}
\begin{itemize}
\item {Proveniência:(Do lat. \textunderscore cautio\textunderscore , \textunderscore cautionis\textunderscore )}
\end{itemize}
Assegurar com caução; afiançar.
\section{Caucionário}
\begin{itemize}
\item {Grp. gram.:adj.}
\end{itemize}
\begin{itemize}
\item {Grp. gram.:M.}
\end{itemize}
\begin{itemize}
\item {Proveniência:(Do lat. \textunderscore cautio\textunderscore , \textunderscore cautionis\textunderscore )}
\end{itemize}
Relativo a caução.
Aquelle que dá caução.
\section{Cauda}
\begin{itemize}
\item {Grp. gram.:f.}
\end{itemize}
\begin{itemize}
\item {Utilização:Mús.}
\end{itemize}
\begin{itemize}
\item {Utilização:Veter.}
\end{itemize}
\begin{itemize}
\item {Proveniência:(Lat. \textunderscore cauda\textunderscore )}
\end{itemize}
Appêndice posterior, mais ou menos longo, do corpo dos animaes; rabo.
Rètaguarda.
Rasto luminoso (dos cometas).
Parte do vestido, que se arrasta posteriormente.
Linha perpendicular, que tem todas as notas, excepto a semibreve.
\textunderscore Cauda de rato\textunderscore , impigem ao longo do tendão, desde a ranilha ao meio da perna do cavallo.
\section{Cauda-de-San-Francisco}
\begin{itemize}
\item {Grp. gram.:f.}
\end{itemize}
Planta lycopodiácea da Índia portuguesa, (\textunderscore lycopodium phlegmaria\textunderscore , Willd.).
\section{Cauda-de-zorro}
\begin{itemize}
\item {Grp. gram.:f.}
\end{itemize}
Designação vulgar da \textunderscore alopecura\textunderscore .
\section{Caudal}
\begin{itemize}
\item {Grp. gram.:adj.}
\end{itemize}
\begin{itemize}
\item {Grp. gram.:M.}
\end{itemize}
Relativo a cauda.
Torrencial.
Torrente.
Cachoeira.
\section{Caudalosamente}
\begin{itemize}
\item {Grp. gram.:adv.}
\end{itemize}
De modo caudaloso.
\section{Caudalosidade}
\begin{itemize}
\item {Grp. gram.:f.}
\end{itemize}
Qualidade de caudaloso.
\section{Caudaloso}
\begin{itemize}
\item {Grp. gram.:adj.}
\end{itemize}
\begin{itemize}
\item {Proveniência:(De \textunderscore caudal\textunderscore )}
\end{itemize}
Que leva água em abundancia; caudal.
Abundante.
\section{Caudatário}
\begin{itemize}
\item {Grp. gram.:m.}
\end{itemize}
\begin{itemize}
\item {Utilização:Fig.}
\end{itemize}
\begin{itemize}
\item {Utilização:T. de Turquel}
\end{itemize}
\begin{itemize}
\item {Proveniência:(De \textunderscore caudato\textunderscore )}
\end{itemize}
Aquelle que, nas solennidades, levanta e leva a cauda das vestes das autoridades ecclesiásticas ou altos dignitários.
Homem servil.
Bisbilhoteiro ao serviço de alguém.
\section{Caudato}
\begin{itemize}
\item {Grp. gram.:adj.}
\end{itemize}
Que tem cauda.
\section{Cauda-vermelha}
\begin{itemize}
\item {Grp. gram.:m.}
\end{itemize}
Espécie de ave, (\textunderscore phoenicurus\textunderscore , Lin.).
\section{Caudeiro}
\begin{itemize}
\item {Grp. gram.:m.}
\end{itemize}
\begin{itemize}
\item {Utilização:Ant.}
\end{itemize}
Caudatário. Cf. Cortesão, \textunderscore Subs\textunderscore .
\section{Caudel}
\textunderscore m.\textunderscore  (e der.)
O mesmo ou melhor que \textunderscore coudel\textunderscore , etc.
\section{Caudelar}
\begin{itemize}
\item {Grp. gram.:v. i.}
\end{itemize}
\begin{itemize}
\item {Utilização:Ant.}
\end{itemize}
\begin{itemize}
\item {Proveniência:(De \textunderscore caudel\textunderscore )}
\end{itemize}
Acaudilhar.
\section{Caudelaria}
\begin{itemize}
\item {Grp. gram.:f.}
\end{itemize}
(V.coudelaria)
\section{Cáudex}
\begin{itemize}
\item {Grp. gram.:m.}
\end{itemize}
O mesmo que \textunderscore cáudice\textunderscore .
\section{Caudicária}
\begin{itemize}
\item {Grp. gram.:f.}
\end{itemize}
Espécie de canôa antiga.
(Cp. \textunderscore caudicário\textunderscore )
\section{Caudicário}
\begin{itemize}
\item {Grp. gram.:m.}
\end{itemize}
\begin{itemize}
\item {Proveniência:(Lat. \textunderscore caudicarius\textunderscore )}
\end{itemize}
Aquelle que guiava a caudicária.
\section{Cáudice}
\begin{itemize}
\item {Grp. gram.:m.}
\end{itemize}
\begin{itemize}
\item {Proveniência:(Do lat. \textunderscore caudex\textunderscore )}
\end{itemize}
Parte da árvore, que não tem rama.
Tronco.
Porção subterrânea de um tronco.
\section{Caudiciforme}
\begin{itemize}
\item {Grp. gram.:adj.}
\end{itemize}
\begin{itemize}
\item {Utilização:Bot.}
\end{itemize}
\begin{itemize}
\item {Proveniência:(Do lat. \textunderscore caudex\textunderscore  + \textunderscore forma\textunderscore )}
\end{itemize}
Diz-se da haste que não tem ramificações.
\section{Caudículo}
\begin{itemize}
\item {Grp. gram.:m.}
\end{itemize}
Pequeno cáudice.
\section{Caudífero}
\begin{itemize}
\item {Grp. gram.:adj.}
\end{itemize}
\begin{itemize}
\item {Proveniência:(Do lat. \textunderscore cauda\textunderscore  + \textunderscore ferre\textunderscore )}
\end{itemize}
Que tem cauda.
\section{Caudilhamento}
\begin{itemize}
\item {Grp. gram.:m.}
\end{itemize}
\begin{itemize}
\item {Utilização:Ant.}
\end{itemize}
\begin{itemize}
\item {Proveniência:(De \textunderscore caudilhar\textunderscore )}
\end{itemize}
Offício ou dignidade de chefe militar.
\section{Caudilhar}
\begin{itemize}
\item {Grp. gram.:v. t.}
\end{itemize}
(V.acaudilhar)
\section{Caudilhismo}
\begin{itemize}
\item {Grp. gram.:m.}
\end{itemize}
\begin{itemize}
\item {Utilização:Bras}
\end{itemize}
Processos de caudilho; galopinagem; caciquismo.
\section{Caudilho}
\begin{itemize}
\item {Grp. gram.:m.}
\end{itemize}
\begin{itemize}
\item {Proveniência:(Do ant. cast. \textunderscore cabdillo\textunderscore , do lat. \textunderscore caput\textunderscore )}
\end{itemize}
Chefe militar.
Chefe; aquelle que dirige uma facção ou bando.
\section{Caudímano}
\begin{itemize}
\item {Grp. gram.:adj.}
\end{itemize}
\begin{itemize}
\item {Proveniência:(Do lat. \textunderscore cauda\textunderscore  + \textunderscore manus\textunderscore )}
\end{itemize}
Que aprehende os objectos com a cauda.
\section{Caudinas}
\begin{itemize}
\item {Grp. gram.:adj. f. pl.}
\end{itemize}
\begin{itemize}
\item {Utilização:Fig.}
\end{itemize}
\begin{itemize}
\item {Proveniência:(Lat. \textunderscore caudinae\textunderscore , fem. pl. de \textunderscore caudinus\textunderscore , de \textunderscore Caudium\textunderscore , n. p.)}
\end{itemize}
Diz-se das fôrcas, ou desfiladeiro, por onde os Samnitas fizeram passar as legiões romanas.
\textunderscore Fôrcas caudinas\textunderscore , humilhação.
\section{Çaugate}
\begin{itemize}
\item {Grp. gram.:m.}
\end{itemize}
\begin{itemize}
\item {Utilização:Ant.}
\end{itemize}
Espécie de embarcação indiana. Cf. \textunderscore Peregrinação\textunderscore , XI.
\section{Cauim}
\begin{itemize}
\item {Grp. gram.:m.}
\end{itemize}
\begin{itemize}
\item {Utilização:Bras}
\end{itemize}
Espécie de bebida, preparada com mandioca cozida, que se deixa fermentar em certa porção de água.
(Do tupi)
\section{Cauintau}
\begin{itemize}
\item {Grp. gram.:m.}
\end{itemize}
\begin{itemize}
\item {Utilização:Bras}
\end{itemize}
Ave ribeirinha das regiões do Amazonas.
\section{Cauíra}
\begin{itemize}
\item {Grp. gram.:adj.}
\end{itemize}
\begin{itemize}
\item {Utilização:Bras. do N}
\end{itemize}
Sovina, avarento.
\section{Cauixi}
\begin{itemize}
\item {Grp. gram.:m.}
\end{itemize}
\begin{itemize}
\item {Utilização:Bras. do N}
\end{itemize}
Substância que, em fórma de esponja, se agglomera nas raízes das árvores, á beira de alguns rios.
\section{Caulange}
\begin{itemize}
\item {Grp. gram.:m.}
\end{itemize}
(?)«\textunderscore ...lhes mandava do Céo da Lua aquelles caulanges pera com elles adormentarem os peixes\textunderscore ». \textunderscore Peregrinação\textunderscore , CLXIV.
\section{Caule}
\begin{itemize}
\item {Grp. gram.:m.}
\end{itemize}
\begin{itemize}
\item {Proveniência:(Lat. \textunderscore caulis\textunderscore )}
\end{itemize}
Haste das plantas.
\section{Cauleoso}
\begin{itemize}
\item {Grp. gram.:adj.}
\end{itemize}
Que tem caule.
\section{Caulescência}
\begin{itemize}
\item {Grp. gram.:f.}
\end{itemize}
Estado ou qualidade do que é caulescente.
\section{Caulescente}
\begin{itemize}
\item {Grp. gram.:adj.}
\end{itemize}
Que tem caule.
\section{Caulícola}
\begin{itemize}
\item {Grp. gram.:m.  e  adj.}
\end{itemize}
\begin{itemize}
\item {Proveniência:(Do lat. \textunderscore caulis\textunderscore  + \textunderscore colere\textunderscore )}
\end{itemize}
Planta parasita, que vive na haste de outros vegetaes.
\section{Caulículo}
\begin{itemize}
\item {Grp. gram.:adj.}
\end{itemize}
\begin{itemize}
\item {Grp. gram.:M. pl.}
\end{itemize}
\begin{itemize}
\item {Proveniência:(Lat. \textunderscore cauliculus\textunderscore )}
\end{itemize}
Pequeno caule.
Hastes que, em ornamentação architectónica, sáem de entre fôlhas de acantho, formando volutas, por baixo da parte superior de um capitel corínthio.
\section{Caulífero}
\begin{itemize}
\item {Grp. gram.:adj.}
\end{itemize}
\begin{itemize}
\item {Proveniência:(Do lat. \textunderscore caulis\textunderscore  + \textunderscore ferre\textunderscore )}
\end{itemize}
Que tem caule.
\section{Caulificação}
\begin{itemize}
\item {Grp. gram.:f.}
\end{itemize}
\begin{itemize}
\item {Proveniência:(Do lat. \textunderscore caulis\textunderscore  + \textunderscore facere\textunderscore )}
\end{itemize}
Formação do caule.
\section{Caulifloro}
\begin{itemize}
\item {Grp. gram.:adj.}
\end{itemize}
\begin{itemize}
\item {Proveniência:(De \textunderscore caule\textunderscore  + \textunderscore flôr\textunderscore )}
\end{itemize}
Diz-se das plantas, que têm flôres no caule.
\section{Caulim}
\begin{itemize}
\item {Grp. gram.:m.}
\end{itemize}
\begin{itemize}
\item {Proveniência:(Do chin. \textunderscore ka-lim\textunderscore )}
\end{itemize}
Substância argilosa, que serve para o fabrico da porcellana.
\section{Caulinar}
\begin{itemize}
\item {Grp. gram.:adj.}
\end{itemize}
\begin{itemize}
\item {Utilização:Bot.}
\end{itemize}
\begin{itemize}
\item {Proveniência:(De \textunderscore caulino\textunderscore )}
\end{itemize}
Relativo ao caule.
Que nasce sôbre o caule.
\section{Caulínia}
\begin{itemize}
\item {Grp. gram.:f.}
\end{itemize}
Gênero de plantas nayadáceas.
\section{Caulinita}
\begin{itemize}
\item {Grp. gram.:f.}
\end{itemize}
\begin{itemize}
\item {Proveniência:(De \textunderscore caulino\textunderscore )}
\end{itemize}
Vestígio de caules fósseis, em certos terrenos calcários.
\section{Caulinização}
\begin{itemize}
\item {Grp. gram.:f.}
\end{itemize}
Acto de caulinizar.
\section{Caulinizar}
\begin{itemize}
\item {Grp. gram.:v. t.}
\end{itemize}
Transformar em caulim.
\section{Caulino}
\begin{itemize}
\item {Grp. gram.:adj.}
\end{itemize}
\begin{itemize}
\item {Utilização:Bot.}
\end{itemize}
Relativo ao caule.
Que nasce no caule.
\section{Caulino}
\begin{itemize}
\item {Grp. gram.:m.}
\end{itemize}
\begin{itemize}
\item {Proveniência:(Do chin. \textunderscore ka-lim\textunderscore )}
\end{itemize}
Substância argilosa, que serve para o fabrico da porcellana.
\section{Cauliodontes}
\begin{itemize}
\item {Grp. gram.:m. pl.}
\end{itemize}
\begin{itemize}
\item {Proveniência:(Do lat. \textunderscore caulis\textunderscore  + gr. \textunderscore odous\textunderscore , \textunderscore odontos\textunderscore )}
\end{itemize}
Gênero de peixes, cujos dentes superiores se cruzam com os inferiores.
Gênero de aves palmípedes, semelhantes aos patos.
Gênero de insectos nocturnos.
\section{Caulóbio}
\begin{itemize}
\item {Grp. gram.:m.}
\end{itemize}
\begin{itemize}
\item {Proveniência:(Do gr. \textunderscore kaulos\textunderscore  + \textunderscore bios\textunderscore )}
\end{itemize}
Insecto lepidóptero nocturno, que vive no interior das plantas aquáticas.
\section{Caulocárpico}
\begin{itemize}
\item {Grp. gram.:adj.}
\end{itemize}
Que tem caulocarpo.
\section{Caulocarpo}
\begin{itemize}
\item {Grp. gram.:m.}
\end{itemize}
\begin{itemize}
\item {Proveniência:(Do gr. \textunderscore kaulos\textunderscore  + \textunderscore carpos\textunderscore )}
\end{itemize}
Caule, que dá fruto differentes vezes.
\section{Caulogastro}
\begin{itemize}
\item {Grp. gram.:m.}
\end{itemize}
\begin{itemize}
\item {Proveniência:(Do gr. \textunderscore kaulos\textunderscore  + \textunderscore gaster\textunderscore )}
\end{itemize}
Cogumelo microscópico, que vive nos frutos do bôrdo.
\section{Cauloglossos}
\begin{itemize}
\item {Grp. gram.:m. pl.}
\end{itemize}
\begin{itemize}
\item {Proveniência:(Do gr. \textunderscore kaulos\textunderscore  + \textunderscore glossa\textunderscore )}
\end{itemize}
Gênero de cogumelos gasteromycetos.
\section{Cauman}
\begin{itemize}
\item {fónica:ca-u}
\end{itemize}
\begin{itemize}
\item {Grp. gram.:m.}
\end{itemize}
\begin{itemize}
\item {Utilização:Bras}
\end{itemize}
Grande ave de rapina.
\section{Caumba}
\begin{itemize}
\item {Grp. gram.:f.}
\end{itemize}
Ave pernalta da África.
\section{Cauna}
\begin{itemize}
\item {Grp. gram.:f.}
\end{itemize}
\begin{itemize}
\item {Utilização:Bras}
\end{itemize}
Erva, que se toma de infusão com o mate.
\section{Caúnho}
\begin{itemize}
\item {Grp. gram.:m.}
\end{itemize}
O mesmo que \textunderscore cônho\textunderscore .
\section{Caúno}
\begin{itemize}
\item {Grp. gram.:m.}
\end{itemize}
Ave pernalta da América do Sul.
\section{Cauré}
\begin{itemize}
\item {Grp. gram.:m.}
\end{itemize}
\begin{itemize}
\item {Utilização:Bras}
\end{itemize}
Erva aromática das regiões do Amazonas.
\section{Cauri}
\begin{itemize}
\item {Grp. gram.:m.}
\end{itemize}
Mollúsco e concha, o mesmo que \textunderscore caurim\textunderscore .
\section{Cauril}
\begin{itemize}
\item {Grp. gram.:m.}
\end{itemize}
\begin{itemize}
\item {Utilização:Pop.}
\end{itemize}
\begin{itemize}
\item {Proveniência:(Do indost. \textunderscore kauri\textunderscore )}
\end{itemize}
Mollúsco gasterópode (\textunderscore cyprea moneta\textunderscore ).
Pequena concha, que serve de moéda, em alguns pontos do Oriente.
Calote.
Pirraça.
\section{Caurim}
\begin{itemize}
\item {Grp. gram.:m.}
\end{itemize}
\begin{itemize}
\item {Utilização:Pop.}
\end{itemize}
\begin{itemize}
\item {Proveniência:(Do indost. \textunderscore kauri\textunderscore )}
\end{itemize}
Mollúsco gasterópode (\textunderscore cyprea moneta\textunderscore ).
Pequena concha, que serve de moéda, em alguns pontos do Oriente.
Calote.
Pirraça.
\section{Caurinar}
\begin{itemize}
\item {Grp. gram.:v. t.}
\end{itemize}
\begin{itemize}
\item {Utilização:Pop.}
\end{itemize}
Pregar caurim a; lograr.
\section{Caurineiro}
\begin{itemize}
\item {Grp. gram.:m.}
\end{itemize}
\begin{itemize}
\item {Utilização:Pop.}
\end{itemize}
\begin{itemize}
\item {Proveniência:(De \textunderscore caurim\textunderscore )}
\end{itemize}
Caloteiro.
Biltre.
\section{Causa}
\begin{itemize}
\item {Grp. gram.:f.}
\end{itemize}
\begin{itemize}
\item {Proveniência:(Lat. \textunderscore causa\textunderscore )}
\end{itemize}
Aquillo ou aquelle que faz que uma coisa exista.
Aquillo que determina um acontecimento: \textunderscore as causas da revolução\textunderscore .
Aquillo que produz.
Motivo, razão: \textunderscore a causa das minhas mágoas\textunderscore .
Origem.
Acção judicial.
Partido, facção.
\section{Causação}
\begin{itemize}
\item {Grp. gram.:f.}
\end{itemize}
Acto de causar; causa.
\section{Causador}
\begin{itemize}
\item {Grp. gram.:m.  e  adj.}
\end{itemize}
Aquillo ou aquelle que causa.
\section{Causal}
\begin{itemize}
\item {Grp. gram.:adj.}
\end{itemize}
\begin{itemize}
\item {Grp. gram.:F.}
\end{itemize}
\begin{itemize}
\item {Proveniência:(Lat. \textunderscore causalis\textunderscore )}
\end{itemize}
Relativo a causa.
Causa, motivo, origem.
\section{Causalidade}
\begin{itemize}
\item {Grp. gram.:f.}
\end{itemize}
\begin{itemize}
\item {Proveniência:(De \textunderscore causal\textunderscore )}
\end{itemize}
Qualidade, que uma coisa tem, de produzir effeito.
Princípio, em virtude do qual os effeitos se ligam ás causas.
\section{Causante}
\begin{itemize}
\item {Grp. gram.:adj.}
\end{itemize}
O mesmo que \textunderscore causador\textunderscore .
\section{Causar}
\begin{itemize}
\item {Grp. gram.:v. t.}
\end{itemize}
\begin{itemize}
\item {Proveniência:(Lat. \textunderscore causari\textunderscore )}
\end{itemize}
Sêr causa de; motivar, produzir.
\section{Causativo}
\begin{itemize}
\item {Grp. gram.:adj.}
\end{itemize}
\begin{itemize}
\item {Proveniência:(Lat. \textunderscore causativus\textunderscore )}
\end{itemize}
Relativo a causa.
Causador.
\section{Causela}
\begin{itemize}
\item {Grp. gram.:f.}
\end{itemize}
\begin{itemize}
\item {Utilização:Ant.}
\end{itemize}
\begin{itemize}
\item {Utilização:Prov.}
\end{itemize}
\begin{itemize}
\item {Utilização:beir.}
\end{itemize}
\begin{itemize}
\item {Proveniência:(Do lat. \textunderscore capsella\textunderscore )}
\end{itemize}
Caixinha.
Vaso sagrado, o mesmo que \textunderscore pýxide\textunderscore .
Caixinha para hóstias.
\section{Causídico}
\begin{itemize}
\item {Grp. gram.:m.}
\end{itemize}
\begin{itemize}
\item {Proveniência:(Lat. \textunderscore causidicus\textunderscore )}
\end{itemize}
Defensor de causas, advogado.
Rábula.
\section{Cáustica}
\begin{itemize}
\item {Grp. gram.:f.}
\end{itemize}
\begin{itemize}
\item {Proveniência:(De \textunderscore cáustico\textunderscore )}
\end{itemize}
Curva, formada pelo cruzamento dos raios luminosos, que uma superfície curva reflecte ou refrange.
\section{Causticação}
\begin{itemize}
\item {Grp. gram.:f.}
\end{itemize}
\begin{itemize}
\item {Utilização:Fig.}
\end{itemize}
Acto de causticar.
Acto de importunar, de molestar.
\section{Causticamente}
\begin{itemize}
\item {Grp. gram.:adv.}
\end{itemize}
De modo cáustico.
\section{Causticante}
\begin{itemize}
\item {Grp. gram.:adj.}
\end{itemize}
Que caustíca.
\section{Causticar}
\begin{itemize}
\item {Grp. gram.:v. t.}
\end{itemize}
\begin{itemize}
\item {Utilização:Fig.}
\end{itemize}
Applicar cáustico a.
Importunar, molestar.
\section{Causticidade}
\begin{itemize}
\item {Grp. gram.:f.}
\end{itemize}
Qualidade do que é cáustico.
\section{Cáustico}
\begin{itemize}
\item {Grp. gram.:adj.}
\end{itemize}
\begin{itemize}
\item {Utilização:Fig.}
\end{itemize}
\begin{itemize}
\item {Grp. gram.:M.}
\end{itemize}
\begin{itemize}
\item {Utilização:Fig.}
\end{itemize}
\begin{itemize}
\item {Proveniência:(Lat. \textunderscore causticus\textunderscore )}
\end{itemize}
Que queima.
Que cauteriza.
Que carboniza os tecidos orgánicos.
Vesicatório.
Mordaz.
Que fere.
Emplastro epispástico; vesicatório.
Pessôa importuna, molesta.
\section{Cautamente}
\begin{itemize}
\item {Grp. gram.:adv.}
\end{itemize}
De modo cauto.
Com cautela.
\section{Cautela}
\begin{itemize}
\item {Grp. gram.:f.}
\end{itemize}
\begin{itemize}
\item {Utilização:Ant.}
\end{itemize}
\begin{itemize}
\item {Proveniência:(Lat. \textunderscore cautela\textunderscore )}
\end{itemize}
Cuidado, para evitar um mal; precaução.
Senha, documento provisório: \textunderscore cautela de casa de penhores\textunderscore .
Subdivisão dos bilhetes de lotaria.
Fraude.
\section{Cautelar}
\begin{itemize}
\item {Grp. gram.:v. t.}
\end{itemize}
\begin{itemize}
\item {Utilização:Des.}
\end{itemize}
O mesmo que \textunderscore acautelar\textunderscore .
\section{Cauteleiro}
\begin{itemize}
\item {Grp. gram.:m.}
\end{itemize}
Vendedor de cautelas ou bilhetes de lotaria.
\section{Cautelosamente}
\begin{itemize}
\item {Grp. gram.:adv.}
\end{itemize}
De modo cauteloso.
Com cautela.
\section{Cauteloso}
\begin{itemize}
\item {Grp. gram.:adj.}
\end{itemize}
Que procede com cautela; que tem prudência.
\section{Cautério}
\begin{itemize}
\item {Grp. gram.:m.}
\end{itemize}
\begin{itemize}
\item {Utilização:Fig.}
\end{itemize}
\begin{itemize}
\item {Proveniência:(Lat. \textunderscore cauterium\textunderscore )}
\end{itemize}
Aquillo que se emprega em Medicina para queimar ou desorganizar uma porção de tecidos orgânicos.
Pequena úlcera, resultante da applicação do cautério.
\textunderscore Cautério actual\textunderscore , instrumento com que se cauterisa.
Castigo, correcção enérgica.
\section{Cauterização}
\begin{itemize}
\item {Grp. gram.:f.}
\end{itemize}
Acto de \textunderscore cauterizar\textunderscore .
\section{Cauterizar}
\begin{itemize}
\item {Grp. gram.:v. t.}
\end{itemize}
\begin{itemize}
\item {Utilização:Fig.}
\end{itemize}
\begin{itemize}
\item {Proveniência:(Lat. \textunderscore cauterizare\textunderscore )}
\end{itemize}
Applicar cautério ou cáustico a.
Corrigir energicamente; castigar.
Extinguir um mal de; sanificar.
\section{Cautivo}
\textunderscore m.\textunderscore  e \textunderscore adj.\textunderscore  (e der.)
Fórma ant. de \textunderscore catívo\textunderscore , etc.
\section{Cauto}
\begin{itemize}
\item {Grp. gram.:adj.}
\end{itemize}
\begin{itemize}
\item {Proveniência:(Lat. \textunderscore cautus\textunderscore )}
\end{itemize}
Que tem cautela, acautelado.
\section{Cava}
\begin{itemize}
\item {Grp. gram.:f.}
\end{itemize}
Acto de cavar.
Lugar cavado; fôsso; cavidade.
Abertura de vestuário, em que se pregam as mangas e aquella a que se adapta o collar.
Adega ou frasqueira subterrânea.
Pavimento inferior de uma casa, abaixo do nivel do arruamento.
\section{Çava}
\begin{itemize}
\item {Grp. gram.:f.}
\end{itemize}
\begin{itemize}
\item {Proveniência:(Do b. lat. \textunderscore zava\textunderscore )}
\end{itemize}
Corpete ou jaquetão acolchoado, que os Godos usavam por baixo da coiraça, para suavizar a dureza d'esta.
\section{Cavaca}
\begin{itemize}
\item {Grp. gram.:f.}
\end{itemize}
\begin{itemize}
\item {Utilização:T. do Fundão}
\end{itemize}
Acha, pedaço de lenha.
Biscoito leve e duro.
Jôgo de crianças, com botões ou moédas, sôbre uma pequena tábua.
(Cp. \textunderscore cavaco\textunderscore )
\section{Cavacar}
\textunderscore v. t.\textunderscore  (e der.)
(V. \textunderscore escavacar\textunderscore , etc.)
\section{Cavaco}
\begin{itemize}
\item {Grp. gram.:m.}
\end{itemize}
\begin{itemize}
\item {Utilização:T. da Índia portuguesa}
\end{itemize}
\begin{itemize}
\item {Utilização:Fam.}
\end{itemize}
\begin{itemize}
\item {Utilização:Fam.}
\end{itemize}
Estilha, pequena lasca de madeira.
Pedacinho de madeira, para lenha.
O mesmo que \textunderscore caló\textunderscore ^1.
Peixe do mar dos Açores. Cf. Flaviense, \textunderscore Diccion. Geográph.\textunderscore 
Mostras de enfado ou zanga, da parte de quem é troçado ou ridiculizado.
Conversação amigavel, simples e despretensiosa.
\textunderscore Dar o cavaco\textunderscore , gostar muito: \textunderscore dar o cavaco por lampreia\textunderscore .
Mostrar-se enfadado ou zangado, por sêr objecto de motejo ou de pirraça: \textunderscore dei o cavaco com aquella patifaria\textunderscore .
(Cast. \textunderscore cabaco\textunderscore )
\section{Cavada}
\begin{itemize}
\item {Grp. gram.:f.}
\end{itemize}
\begin{itemize}
\item {Utilização:Prov.}
\end{itemize}
\begin{itemize}
\item {Utilização:minh.}
\end{itemize}
Acto de cavar.
Lavra.
\section{Cavadeira}
\begin{itemize}
\item {Grp. gram.:f.}
\end{itemize}
\begin{itemize}
\item {Utilização:Bras}
\end{itemize}
\begin{itemize}
\item {Utilização:Pesc.}
\end{itemize}
\begin{itemize}
\item {Proveniência:(De \textunderscore cavar\textunderscore )}
\end{itemize}
Instrumento agrícola, para cavar terra ou juntar ervas que se cortam.
Enxada, usada em pesca fluvial.
\section{Cavadela}
\begin{itemize}
\item {Grp. gram.:f.}
\end{itemize}
Acto de cavar.
Enxadada.
\section{Cavadiço}
\begin{itemize}
\item {Grp. gram.:adj.}
\end{itemize}
\begin{itemize}
\item {Proveniência:(De \textunderscore cavado\textunderscore )}
\end{itemize}
Que se tira da terra, escavando-a.
\section{Cavado}
\begin{itemize}
\item {Grp. gram.:m.}
\end{itemize}
\begin{itemize}
\item {Utilização:Mús.}
\end{itemize}
\begin{itemize}
\item {Proveniência:(De \textunderscore cavar\textunderscore )}
\end{itemize}
Lugar que se escavou.
Buraco.
Cada peça, que se separa da partitura para cada instrumento.
\section{Cavador}
\begin{itemize}
\item {Grp. gram.:m.}
\end{itemize}
\begin{itemize}
\item {Proveniência:(Lat. \textunderscore cavator\textunderscore )}
\end{itemize}
Aquelle que cava.
Trabalhador rural.
\section{Cavadora}
\begin{itemize}
\item {Grp. gram.:f.}
\end{itemize}
\begin{itemize}
\item {Proveniência:(De \textunderscore cavar\textunderscore )}
\end{itemize}
Máquina agrícola, para desterroar a camada terrosa, sotoposta á terra vegetal.
\section{Cavadura}
\begin{itemize}
\item {Grp. gram.:f.}
\end{itemize}
\begin{itemize}
\item {Utilização:Ant.}
\end{itemize}
(V.cavadela)
O mesmo que \textunderscore cova\textunderscore . Cf. \textunderscore Rev. Lus.\textunderscore , XVI, 2.
\section{Cá-vai}
\begin{itemize}
\item {Grp. gram.:m.}
\end{itemize}
Nome que, em Abrantes, se dá ao \textunderscore noitibó\textunderscore .
\section{Cavala}
\begin{itemize}
\item {Grp. gram.:f.}
\end{itemize}
Peixe, da fam. dos escômbridas, espécie de sarda.
\section{Cavalada}
\begin{itemize}
\item {Grp. gram.:f.}
\end{itemize}
\begin{itemize}
\item {Proveniência:(De \textunderscore cavallo\textunderscore )}
\end{itemize}
Grande asneira.
Acto bestial.
\section{Cavalagem}
\begin{itemize}
\item {Grp. gram.:f.}
\end{itemize}
\begin{itemize}
\item {Proveniência:(De \textunderscore cavallo\textunderscore )}
\end{itemize}
O mesmo que \textunderscore padreação\textunderscore .
Preço da padreação.
\section{Cavalão}
\begin{itemize}
\item {Grp. gram.:m.}
\end{itemize}
\begin{itemize}
\item {Utilização:Fig.}
\end{itemize}
\begin{itemize}
\item {Proveniência:(De \textunderscore cavallo\textunderscore )}
\end{itemize}
Cavallo grande.
Peixe, da fam. dos escômbridas.
Pessôa desenvolta, que anda aos saltos e em correrias.
\section{Cavalar}
\begin{itemize}
\item {Grp. gram.:adj.}
\end{itemize}
Relativo a cavalo.
Pertencente á espécie cavalo.
\section{Cavalar}
\begin{itemize}
\item {Grp. gram.:v. i.}
\end{itemize}
\begin{itemize}
\item {Utilização:Fam.}
\end{itemize}
O mesmo que \textunderscore cavaloar\textunderscore .
\section{Cavalaria}
\begin{itemize}
\item {Grp. gram.:f.}
\end{itemize}
Reunião de cavalos.
Multidão de gente a cavalo.
Regimento, ou regimentos militares, que servem a cavalo.
Arte de montar a cavalo, equitação.
Façanha de cavaleiro andante.
Proêza.
Pensão annual, que os mosteiros pagavam aos herdeiros dos seus fundadores, quando êsses herdeiros eram do sexo masculino; aos do sexo feminino pagava-se o casamento.
(Cp. \textunderscore casamento\textunderscore )
Multa, que era paga pelos que, devendo apresentar cavalo de marca, nas revistas do mês de Maio, o não apresentavam.
Terra, casal, ou herdade, que se concedia com a obrigação de o concessiónário apresentar depois certo número de cavalos, para determinada expedição.
Bens, que, por consenso dos herdeiros, ficavam indivisos, como se fôssem morgado ou vínculo.
\section{Cavalariano}
\begin{itemize}
\item {Grp. gram.:m.}
\end{itemize}
\begin{itemize}
\item {Utilização:Bras. do N}
\end{itemize}
\begin{itemize}
\item {Utilização:Bras. do S}
\end{itemize}
\begin{itemize}
\item {Proveniência:(De \textunderscore cavallaria\textunderscore )}
\end{itemize}
Mercador de cavalos.
Soldado de cavalaria.
\section{Cavalariça}
\begin{itemize}
\item {Grp. gram.:f.}
\end{itemize}
\begin{itemize}
\item {Proveniência:(De \textunderscore cavallaria\textunderscore )}
\end{itemize}
Casa térrea, destinada a habitação de cavalos.
Cocheira.
\section{Cavalariço}
\begin{itemize}
\item {Grp. gram.:m.}
\end{itemize}
Moço de cavalariça.
(Cp. \textunderscore cavallariça\textunderscore )
\section{Cavalear}
\begin{itemize}
\item {Grp. gram.:v. t.}
\end{itemize}
O mesmo que \textunderscore cavalgar\textunderscore ^1. Cf. Camillo, \textunderscore Amor de Perd.\textunderscore  e \textunderscore Regicida\textunderscore , 42.
\section{Cavaleira}
\begin{itemize}
\item {Grp. gram.:f.}
\end{itemize}
\begin{itemize}
\item {Grp. gram.:Adj. f.}
\end{itemize}
\begin{itemize}
\item {Utilização:Mathem.}
\end{itemize}
\begin{itemize}
\item {Proveniência:(De \textunderscore cavalleiro\textunderscore )}
\end{itemize}
Mulher, que costuma e sabe andar a cavalo; amazona.
Diz-se da perspectiva, que tem por fim determinar a projecção oblíqua de um objecto, sôbre um plano de frente.
\section{Cavaleirado}
\begin{itemize}
\item {Grp. gram.:m.}
\end{itemize}
Dignidade de cavaleiro.
Tença de cavaleiro, na Idade-Média.
\section{Cavaleiramente}
\begin{itemize}
\item {Grp. gram.:adv.}
\end{itemize}
Á maneira de cavaleiro.
Jactanciosamente.
\section{Cavaleirão}
\begin{itemize}
\item {Grp. gram.:m.  e  adj.}
\end{itemize}
\begin{itemize}
\item {Utilização:Ant.}
\end{itemize}
\begin{itemize}
\item {Proveniência:(De \textunderscore cavalleiro\textunderscore )}
\end{itemize}
Homem jactancioso, arrogante. Cf. G. Vicente, \textunderscore Inês Pereira\textunderscore .
\section{Cavaleirar}
\begin{itemize}
\item {Grp. gram.:v. i.}
\end{itemize}
\begin{itemize}
\item {Utilização:Ant.}
\end{itemize}
Exercer funcção de cavaleiro; marchar a cavalo.
\section{Cavaleiras}
\begin{itemize}
\item {Grp. gram.:f. pl.}
\end{itemize}
(V.ás-cavalleiras)
\section{Cavaleirato}
\begin{itemize}
\item {Grp. gram.:m.}
\end{itemize}
Dignidade de cavaleiro.
Tença de cavaleiro, na Idade-Média.
\section{Cavaleiro}
\begin{itemize}
\item {Grp. gram.:m.}
\end{itemize}
\begin{itemize}
\item {Grp. gram.:Loc. adv.}
\end{itemize}
\begin{itemize}
\item {Grp. gram.:Adj.}
\end{itemize}
Homem, que anda a cavalo.
Aquelle que sabe andar a cavalo.
Soldado de cavalaria.
Aquelle que pertence a uma instituição religiosa e militar de cavalaria.
Primeira graduação das actuaes Ordens militares, honorificas.
Homem nobre.
Aquelle que, nas justas antigas, tomava a defesa de uma dama ou de outra pessôa ou de uma ideia.
Cavalheiro.
Ponto elevado, em que se coloca uma bateria.
\textunderscore A cavaleiro\textunderscore , em lugar eminente, sobranceiro.
Que anda a cavalo.
Alto.
Denodado.
Relativo a cavalaria:«\textunderscore para os diferençar das mais Ordens cavaleiras\textunderscore ».
Filinto, \textunderscore D. Man.\textunderscore , I, 35.
\section{Cavaleirosamente}
\begin{itemize}
\item {Grp. gram.:adv.}
\end{itemize}
De modo cavaleiroso.
\section{Cavaleiroso}
\begin{itemize}
\item {Grp. gram.:adj.}
\end{itemize}
Próprio de cavaleiro.
\section{Cavaleta}
\begin{itemize}
\item {fónica:lê}
\end{itemize}
\begin{itemize}
\item {Grp. gram.:f.}
\end{itemize}
\begin{itemize}
\item {Utilização:Prov.}
\end{itemize}
\begin{itemize}
\item {Grp. gram.:Pl. Loc. adv.}
\end{itemize}
Égua ordinária.
Alimária reles; azêmola.
\textunderscore Ás-cavaletas\textunderscore , o mesmo que \textunderscore ás-cavalitas\textunderscore .
\section{Cavalete}
\begin{itemize}
\item {fónica:lê}
\end{itemize}
\begin{itemize}
\item {Grp. gram.:m.}
\end{itemize}
\begin{itemize}
\item {Utilização:Bras. do N}
\end{itemize}
\begin{itemize}
\item {Utilização:Bras. do N}
\end{itemize}
\begin{itemize}
\item {Proveniência:(De \textunderscore cavallo\textunderscore )}
\end{itemize}
Armação triangular, em que os pintores colocam a tela em que trabalham.
Peça de madeira, em que se coloca o quadro preto ou ardósia grande, nas casas de estudo.
Antigo instrumento de tortura, ecúleo.
Peça, que sustenta as xalmas.
Mesa, que sustenta os caixotins typográphicos.
Qualquer banqueta ou peça semelhante, em que os oficiaes mechânicos colocam a obra em que trabalham.
Peça, com que se transportam cabos, a bordo.
Arqueamento (do naris).
Toro de madeira, de que se servem nadadores, para atravessar rios.
Trave, em que os vaqueiros penduram sellas e outros arreios.
\section{Cavalgada}
\begin{itemize}
\item {Grp. gram.:f.}
\end{itemize}
\begin{itemize}
\item {Proveniência:(De \textunderscore cavalgar\textunderscore ^2)}
\end{itemize}
Reunião de pessôas a cavallo.
Marcha de um trôço de cavalleiros.
\section{Cavalgador}
\begin{itemize}
\item {Grp. gram.:m.  e  adj.}
\end{itemize}
(V.cavalgante)
\section{Cavalgadura}
\begin{itemize}
\item {Grp. gram.:f.}
\end{itemize}
\begin{itemize}
\item {Utilização:Fig.}
\end{itemize}
\begin{itemize}
\item {Proveniência:(De \textunderscore cavalgar\textunderscore ^1)}
\end{itemize}
Bêsta cavallar, muar, ou asinina, que se póde cavalgar.
Pessôa grosseira e estúpida.
\section{Cavalgante}
\begin{itemize}
\item {Grp. gram.:m.  e  adj.}
\end{itemize}
Aquelle que cavalga.
\section{Cavalgar}
\begin{itemize}
\item {Grp. gram.:v. i.}
\end{itemize}
\begin{itemize}
\item {Grp. gram.:V. t.}
\end{itemize}
Montar a cavallo; andar a cavallo.
Sentar-se, como se montasse a cavallo: \textunderscore cavalgar num muro\textunderscore .
Passar por cima.
Montar.
Galgar.
Ir acima de.
(B. lat. \textunderscore caballicare\textunderscore , do lat. \textunderscore caballus\textunderscore )
\section{Cavalgar}
\begin{itemize}
\item {Grp. gram.:adj.}
\end{itemize}
\begin{itemize}
\item {Utilização:Ant.}
\end{itemize}
O mesmo que \textunderscore cavallar\textunderscore ^1.
\section{Cavalgata}
\begin{itemize}
\item {Grp. gram.:f.}
\end{itemize}
O mesmo que \textunderscore cavalgada\textunderscore .
\section{Cavalhada}
\begin{itemize}
\item {Grp. gram.:f.}
\end{itemize}
\begin{itemize}
\item {Utilização:Bras}
\end{itemize}
Porção de cavallos.
Gado cavallar.
(Cast. \textunderscore caballada\textunderscore , de \textunderscore caballo\textunderscore )
\section{Cavalhadas}
\begin{itemize}
\item {Grp. gram.:f. pl.}
\end{itemize}
Diversão popular, espécie de torneio, em que os lidadores, geralmente montados em burros, e ás vezes a pé, pleiteiam com canas ou varas prêmios enfiados numa corda.
(Pl. de \textunderscore cavalhada\textunderscore )
\section{Cavalhariça}
\begin{itemize}
\item {Grp. gram.:f.}
\end{itemize}
(V.cavallariça)(Cp. cast. \textunderscore caballariza\textunderscore )
\section{Cavalharice}
\begin{itemize}
\item {Grp. gram.:f.}
\end{itemize}
(V.cavallariça)
\section{Cavalheira}
\textunderscore f. T. de Montalto\textunderscore  e de outros pontos de além Tejo.
O mesmo que \textunderscore cavallariça\textunderscore .
\section{Cavalheiramente}
\begin{itemize}
\item {Grp. gram.:adv.}
\end{itemize}
(V.cavalheirosamente)
\section{Cavalheiresco}
\begin{itemize}
\item {Grp. gram.:adj.}
\end{itemize}
(V.cavalheiroso)
\section{Cavalheirismo}
\begin{itemize}
\item {Grp. gram.:m.}
\end{itemize}
\begin{itemize}
\item {Proveniência:(De \textunderscore cavalheiro\textunderscore )}
\end{itemize}
Acção ou qualidade própria de cavalheiro.
Acto nobre, bizarro.
\section{Cavalheiritas}
\begin{itemize}
\item {Grp. gram.:f. pl.}
\end{itemize}
Espécie de jôgo de rapazes.
\section{Cavalheiro}
\begin{itemize}
\item {Grp. gram.:m.}
\end{itemize}
\begin{itemize}
\item {Grp. gram.:Adj.}
\end{itemize}
Homem de sentimentos e acções nobres.
Homem delicado; homem bem educado.
Homem que baila com uma dama.
O mesmo que \textunderscore cavalheiroso\textunderscore .
(Cast. \textunderscore caballero\textunderscore , de \textunderscore caballo\textunderscore )
\section{Cavalheiroso}
\begin{itemize}
\item {Grp. gram.:adj.}
\end{itemize}
Próprio de cavalheiro.
Distinto.
Delicado.
\section{Cavalheirote}
\begin{itemize}
\item {Grp. gram.:m.}
\end{itemize}
\begin{itemize}
\item {Utilização:Pop.}
\end{itemize}
\begin{itemize}
\item {Proveniência:(De \textunderscore cavalheiro\textunderscore )}
\end{itemize}
Designação depreciativa de um indivíduo de pouco cavalheirismo.
\section{Cavalheriço}
\begin{itemize}
\item {Grp. gram.:m.}
\end{itemize}
O mesmo que \textunderscore cavallariço\textunderscore . Cf. Filinto, VII, 140.
\section{Cavalinha}
\begin{itemize}
\item {Grp. gram.:f.}
\end{itemize}
\begin{itemize}
\item {Proveniência:(De \textunderscore cavallo\textunderscore )}
\end{itemize}
Pequena cavala.
Designação vulgar do equiseto.
\section{Cavalinhas}
\begin{itemize}
\item {Grp. gram.:f. pl.}
\end{itemize}
(V.ás-cavalinhas)
\section{Cavalinho}
\begin{itemize}
\item {Grp. gram.:m.}
\end{itemize}
\begin{itemize}
\item {Utilização:Prov.}
\end{itemize}
\begin{itemize}
\item {Utilização:alent.}
\end{itemize}
\begin{itemize}
\item {Utilização:Gír.}
\end{itemize}
\begin{itemize}
\item {Utilização:Prov.}
\end{itemize}
\begin{itemize}
\item {Utilização:alg.}
\end{itemize}
\begin{itemize}
\item {Grp. gram.:Pl.}
\end{itemize}
\begin{itemize}
\item {Utilização:Pop.}
\end{itemize}
\begin{itemize}
\item {Proveniência:(De \textunderscore cavallo\textunderscore )}
\end{itemize}
Pequeno cavalo.
Utensílio de ferro com quatro pés, sôbre o qual descansa a ponta do espêto, na cozinha.
Libra esterlina.
Rinchão, ave.
Companhia equestre, que se apresenta nos circos.
\section{Cavaliquoque}
\begin{itemize}
\item {Grp. gram.:m.}
\end{itemize}
\begin{itemize}
\item {Proveniência:(De \textunderscore cavallo\textunderscore )}
\end{itemize}
Bêsta reles; pileca.
\section{Cavalitas}
\begin{itemize}
\item {Grp. gram.:f. pl.}
\end{itemize}
(V.ás-cavalitas)
\section{Cavalla}
\begin{itemize}
\item {Grp. gram.:f.}
\end{itemize}
Peixe, da fam. dos escômbridas, espécie de sarda.
\section{Cavallada}
\begin{itemize}
\item {Grp. gram.:f.}
\end{itemize}
\begin{itemize}
\item {Proveniência:(De \textunderscore cavallo\textunderscore )}
\end{itemize}
Grande asneira.
Acto bestial.
\section{Cavallagem}
\begin{itemize}
\item {Grp. gram.:f.}
\end{itemize}
\begin{itemize}
\item {Proveniência:(De \textunderscore cavallo\textunderscore )}
\end{itemize}
O mesmo que \textunderscore padreação\textunderscore .
Preço da padreação.
\section{Cavallão}
\begin{itemize}
\item {Grp. gram.:m.}
\end{itemize}
\begin{itemize}
\item {Utilização:Fig.}
\end{itemize}
\begin{itemize}
\item {Proveniência:(De \textunderscore cavallo\textunderscore )}
\end{itemize}
Cavallo grande.
Peixe, da fam. dos escômbridas.
Pessôa desenvolta, que anda aos saltos e em correrias.
\section{Cavallar}
\begin{itemize}
\item {Grp. gram.:adj.}
\end{itemize}
Relativo a cavallo.
Pertencente á espécie cavallo.
\section{Cavallar}
\begin{itemize}
\item {Grp. gram.:v. i.}
\end{itemize}
\begin{itemize}
\item {Utilização:Fam.}
\end{itemize}
O mesmo que \textunderscore cavaloar\textunderscore .
\section{Cavallaria}
\begin{itemize}
\item {Grp. gram.:f.}
\end{itemize}
Reunião de cavallos.
Multidão de gente a cavallo.
Regimento, ou regimentos militares, que servem a cavallo.
Arte de montar a cavallo, equitação.
Façanha de cavalleiro andante.
Proêza.
Pensão annual, que os mosteiros pagavam aos herdeiros dos seus fundadores, quando êsses herdeiros eram do sexo masculino; aos do sexo feminino pagava-se o casamento.
(Cp. \textunderscore casamento\textunderscore )
Multa, que era paga pelos que, devendo apresentar cavallo de marca, nas revistas do mês de Maio, o não apresentavam.
Terra, casal, ou herdade, que se concedia com a obrigação de o concessiónário apresentar depois certo número de cavallos, para determinada expedição.
Bens, que, por consenso dos herdeiros, ficavam indivisos, como se fôssem morgado ou vínculo.
\section{Cavallariano}
\begin{itemize}
\item {Grp. gram.:m.}
\end{itemize}
\begin{itemize}
\item {Utilização:Bras. do N}
\end{itemize}
\begin{itemize}
\item {Utilização:Bras. do S}
\end{itemize}
\begin{itemize}
\item {Proveniência:(De \textunderscore cavallaria\textunderscore )}
\end{itemize}
Mercador de cavallos.
Soldado de cavallaria.
\section{Cavallariça}
\begin{itemize}
\item {Grp. gram.:f.}
\end{itemize}
\begin{itemize}
\item {Proveniência:(De \textunderscore cavallaria\textunderscore )}
\end{itemize}
Casa térrea, destinada a habitação de cavallos.
Cocheira.
\section{Cavallariço}
\begin{itemize}
\item {Grp. gram.:m.}
\end{itemize}
Moço de cavallariça.
(Cp. \textunderscore cavallariça\textunderscore )
\section{Cavallear}
\begin{itemize}
\item {Grp. gram.:v. t.}
\end{itemize}
O mesmo que \textunderscore cavalgar\textunderscore ^1. Cf. Camillo, \textunderscore Amor de Perd.\textunderscore  e \textunderscore Regicida\textunderscore , 42.
\section{Cavalleira}
\begin{itemize}
\item {Grp. gram.:f.}
\end{itemize}
\begin{itemize}
\item {Grp. gram.:Adj. f.}
\end{itemize}
\begin{itemize}
\item {Utilização:Mathem.}
\end{itemize}
\begin{itemize}
\item {Proveniência:(De \textunderscore cavalleiro\textunderscore )}
\end{itemize}
Mulher, que costuma e sabe andar a cavallo; amazona.
Diz-se da perspectiva, que tem por fim determinar a projecção oblíqua de um objecto, sôbre um plano de frente.
\section{Cavalleiramente}
\begin{itemize}
\item {Grp. gram.:adv.}
\end{itemize}
Á maneira de cavalleiro.
Jactanciosamente.
\section{Cavalleirão}
\begin{itemize}
\item {Grp. gram.:m.  e  adj.}
\end{itemize}
\begin{itemize}
\item {Utilização:Ant.}
\end{itemize}
\begin{itemize}
\item {Proveniência:(De \textunderscore cavalleiro\textunderscore )}
\end{itemize}
Homem jactancioso, arrogante. Cf. G. Vicente, \textunderscore Inês Pereira\textunderscore .
\section{Cavalleirar}
\begin{itemize}
\item {Grp. gram.:v. i.}
\end{itemize}
\begin{itemize}
\item {Utilização:Ant.}
\end{itemize}
Exercer funcção de cavalleiro; marchar a cavallo.
\section{Cavalleiras}
\begin{itemize}
\item {Grp. gram.:f. pl.}
\end{itemize}
(V.ás-cavalleiras)
\section{Cavalleirato}
\begin{itemize}
\item {Grp. gram.:m.}
\end{itemize}
Dignidade de cavalleiro.
Tença de cavalleiro, na Idade-Média.
\section{Cavalleiro}
\begin{itemize}
\item {Grp. gram.:m.}
\end{itemize}
\begin{itemize}
\item {Grp. gram.:Loc. adv.}
\end{itemize}
\begin{itemize}
\item {Grp. gram.:Adj.}
\end{itemize}
Homem, que anda a cavallo.
Aquelle que sabe andar a cavallo.
Soldado de cavallaria.
Aquelle que pertence a uma instituição religiosa e militar de cavallaria.
Primeira graduação das actuaes Ordens militares, honorificas.
Homem nobre.
Aquelle que, nas justas antigas, tomava a defesa de uma dama ou de outra pessôa ou de uma ideia.
Cavalheiro.
Ponto elevado, em que se colloca uma bateria.
\textunderscore A cavalleiro\textunderscore , em lugar eminente, sobranceiro.
Que anda a cavallo.
Alto.
Denodado.
Relativo a cavallaria:«\textunderscore para os differençar das mais Ordens cavalleiras\textunderscore ».
Filinto, \textunderscore D. Man.\textunderscore , I, 35.
\section{Cavalleirosamente}
\begin{itemize}
\item {Grp. gram.:adv.}
\end{itemize}
De modo cavalleiroso.
\section{Cavalleiroso}
\begin{itemize}
\item {Grp. gram.:adj.}
\end{itemize}
Próprio de cavalleiro.
\section{Cavalleta}
\begin{itemize}
\item {fónica:lê}
\end{itemize}
\begin{itemize}
\item {Grp. gram.:f.}
\end{itemize}
\begin{itemize}
\item {Utilização:Prov.}
\end{itemize}
\begin{itemize}
\item {Grp. gram.:Pl. Loc. adv.}
\end{itemize}
Égua ordinária.
Alimária reles; azêmola.
\textunderscore Ás-cavalletas\textunderscore , o mesmo que \textunderscore ás-cavallitas\textunderscore .
\section{Cavallete}
\begin{itemize}
\item {fónica:lê}
\end{itemize}
\begin{itemize}
\item {Grp. gram.:m.}
\end{itemize}
\begin{itemize}
\item {Utilização:Bras. do N}
\end{itemize}
\begin{itemize}
\item {Utilização:Bras. do N}
\end{itemize}
\begin{itemize}
\item {Proveniência:(De \textunderscore cavallo\textunderscore )}
\end{itemize}
Armação triangular, em que os pintores collocam a tela em que trabalham.
Peça de madeira, em que se colloca o quadro preto ou ardósia grande, nas casas de estudo.
Antigo instrumento de tortura, ecúleo.
Peça, que sustenta as xalmas.
Mesa, que sustenta os caixotins typográphicos.
Qualquer banqueta ou peça semelhante, em que os officiaes mechânicos collocam a obra em que trabalham.
Peça, com que se transportam cabos, a bordo.
Arqueamento (do naris).
Toro de madeira, de que se servem nadadores, para atravessar rios.
Trave, em que os vaqueiros penduram sellas e outros arreios.
\section{Cavallicoque}
\begin{itemize}
\item {Grp. gram.:m.}
\end{itemize}
\begin{itemize}
\item {Proveniência:(De \textunderscore cavallo\textunderscore )}
\end{itemize}
Bêsta reles; pileca.
\section{Cavallinha}
\begin{itemize}
\item {Grp. gram.:f.}
\end{itemize}
\begin{itemize}
\item {Proveniência:(De \textunderscore cavallo\textunderscore )}
\end{itemize}
Pequena cavalla.
Designação vulgar do equiseto.
\section{Cavallinhas}
\begin{itemize}
\item {Grp. gram.:f. pl.}
\end{itemize}
(V.ás-cavallinhas)
\section{Cavallinho}
\begin{itemize}
\item {Grp. gram.:m.}
\end{itemize}
\begin{itemize}
\item {Utilização:Prov.}
\end{itemize}
\begin{itemize}
\item {Utilização:alent.}
\end{itemize}
\begin{itemize}
\item {Utilização:Gír.}
\end{itemize}
\begin{itemize}
\item {Utilização:Prov.}
\end{itemize}
\begin{itemize}
\item {Utilização:alg.}
\end{itemize}
\begin{itemize}
\item {Grp. gram.:Pl.}
\end{itemize}
\begin{itemize}
\item {Utilização:Pop.}
\end{itemize}
\begin{itemize}
\item {Proveniência:(De \textunderscore cavallo\textunderscore )}
\end{itemize}
Pequeno cavallo.
Utensílio de ferro com quatro pés, sôbre o qual descansa a ponta do espêto, na cozinha.
Libra esterlina.
Rinchão, ave.
Companhia equestre, que se apresenta nos circos.
\section{Cavallinho-de-judeu}
\begin{itemize}
\item {Grp. gram.:m.}
\end{itemize}
\begin{itemize}
\item {Utilização:Bras}
\end{itemize}
Insecto, o mesmo que \textunderscore libellinha\textunderscore .
\section{Cavallinhos-fuscos}
\begin{itemize}
\item {Grp. gram.:m. pl.}
\end{itemize}
\begin{itemize}
\item {Utilização:Prov.}
\end{itemize}
\begin{itemize}
\item {Utilização:alent.}
\end{itemize}
O mesmo que \textunderscore toirinha\textunderscore ^1.
\section{Cavallitas}
\begin{itemize}
\item {Grp. gram.:f. pl.}
\end{itemize}
(V.ás-cavallitas)
\section{Cavallo}
\begin{itemize}
\item {Grp. gram.:m.}
\end{itemize}
\begin{itemize}
\item {Utilização:Gír.}
\end{itemize}
\begin{itemize}
\item {Proveniência:(Lat. \textunderscore caballus\textunderscore )}
\end{itemize}
Quadrúpede doméstico, solípede.
Unidade convencional, em Mechânica, equivalente á força necessaria para elevar 75 kilogrammas a 1 metro de altura em um segundo.
Banco de tanoaria.
Ramo ou tronco, em que se enxerta.
Cancro siphylítico.
Peça de xadrez.
Nome de uma carta de jôgo.
Unidade de um corpo de cavallaria.
Tenaz de fogão.
Nome de alguns peixes.
\textunderscore Cavallo vapor\textunderscore , medida dynâmica, igual ao trabalho de 75 quilogrâmmetros por segundo.
\textunderscore Cavallo hora\textunderscore , medida dynâmica, igual ao trabalho de um cavallo vapor em uma hora.
Libra esterlina, cavallinho.
\section{Cavalloar}
\begin{itemize}
\item {Grp. gram.:v. i.}
\end{itemize}
\begin{itemize}
\item {Utilização:Fam.}
\end{itemize}
\begin{itemize}
\item {Proveniência:(De \textunderscore cavallão\textunderscore )}
\end{itemize}
Saltar como os cavallos; traquinar muito; sêr estúrdio.
\section{Cavallo-de-Maio}
\begin{itemize}
\item {Grp. gram.:m.}
\end{itemize}
Multa ou pena, que pagavam os chefes de família que, no dia 1 de Maio, não apresentavam cavallo de marca.
\section{Cavallo-marinho}
\begin{itemize}
\item {Grp. gram.:m.}
\end{itemize}
O mesmo que \textunderscore hippópotamo\textunderscore .
Designação vulgar do hippocampo, dos syngnatos e de outros peixes ainda.
\section{Cavallo-rinchão}
\begin{itemize}
\item {Grp. gram.:m.}
\end{itemize}
Nome vulgar do peto-real ou picapau verde, (\textunderscore gecinus viridis\textunderscore ).
\section{Cavallório}
\begin{itemize}
\item {Grp. gram.:m.}
\end{itemize}
Cavalgadura grande mas ordinária e de pouco préstimo.
\section{Cavalo}
\begin{itemize}
\item {Grp. gram.:m.}
\end{itemize}
\begin{itemize}
\item {Utilização:Gír.}
\end{itemize}
\begin{itemize}
\item {Proveniência:(Lat. \textunderscore caballus\textunderscore )}
\end{itemize}
Quadrúpede doméstico, solípede.
Unidade convencional, em Mechânica, equivalente á força necessaria para elevar 75 kilogrammas a 1 metro de altura em um segundo.
Banco de tanoaria.
Ramo ou tronco, em que se enxerta.
Cancro siphylítico.
Peça de xadrez.
Nome de uma carta de jôgo.
Unidade de um corpo de cavalaria.
Tenaz de fogão.
Nome de alguns peixes.
\textunderscore Cavalo vapor\textunderscore , medida dinâmica, igual ao trabalho de 75 quilogrâmetros por segundo.
\textunderscore Cavalo hora\textunderscore , medida dinâmica, igual ao trabalho de um cavalo vapor em uma hora.
Libra esterlina, cavalinho.
\section{Cavaloar}
\begin{itemize}
\item {Grp. gram.:v. i.}
\end{itemize}
\begin{itemize}
\item {Utilização:Fam.}
\end{itemize}
\begin{itemize}
\item {Proveniência:(De \textunderscore cavallão\textunderscore )}
\end{itemize}
Saltar como os cavalos; traquinar muito; sêr estúrdio.
\section{Cavalório}
\begin{itemize}
\item {Grp. gram.:m.}
\end{itemize}
Cavalgadura grande mas ordinária e de pouco préstimo.
\section{Cavanejo}
\begin{itemize}
\item {Grp. gram.:m.}
\end{itemize}
(V.cabanejo)
\section{Cavão}
\begin{itemize}
\item {Grp. gram.:m.}
\end{itemize}
O mesmo que \textunderscore cavador\textunderscore .
\section{Cavaqueador}
\begin{itemize}
\item {Grp. gram.:m.  e  adj.}
\end{itemize}
O que cavaqueia.
\section{Cavaquear}
\begin{itemize}
\item {Grp. gram.:v. i.}
\end{itemize}
\begin{itemize}
\item {Utilização:Fam.}
\end{itemize}
\begin{itemize}
\item {Proveniência:(De \textunderscore cavaco\textunderscore )}
\end{itemize}
Estar ao cavaco, conversar singelamente, ao acaso.
\section{Cavaqueira}
\begin{itemize}
\item {Grp. gram.:f.}
\end{itemize}
\begin{itemize}
\item {Utilização:Fam.}
\end{itemize}
\begin{itemize}
\item {Proveniência:(De \textunderscore cavaco\textunderscore )}
\end{itemize}
Acto de cavaquear, demoradamente.
\section{Cavaqueira}
\begin{itemize}
\item {Grp. gram.:f.}
\end{itemize}
\begin{itemize}
\item {Proveniência:(De \textunderscore cavaca\textunderscore )}
\end{itemize}
Mulher que fabríca ou vende biscoitos, chamados cavacas.
\section{Cavaquinho}
\begin{itemize}
\item {Grp. gram.:m.}
\end{itemize}
\begin{itemize}
\item {Grp. gram.:Loc.}
\end{itemize}
\begin{itemize}
\item {Utilização:fam.}
\end{itemize}
\begin{itemize}
\item {Proveniência:(De \textunderscore cavaco\textunderscore )}
\end{itemize}
Pequeno instrumento de quatro cordas.
\textunderscore Dar o cavaquinho\textunderscore , gostar muito.
\section{Cavar}
\begin{itemize}
\item {Grp. gram.:v. t.}
\end{itemize}
\begin{itemize}
\item {Utilização:Fig.}
\end{itemize}
\begin{itemize}
\item {Grp. gram.:V. i.}
\end{itemize}
\begin{itemize}
\item {Proveniência:(Lat. \textunderscore cavare\textunderscore )}
\end{itemize}
Abrir com enxada, com sacho, sachola ou picareta (a terra).
Fazer escavação em volta de ou em.
Escavar.
Tornar côncavo.
Tirar da terra, cavando.
Procurar, investigar.
Abrir cava em (vestuário).
Trabalhar, cavando.
\section{Cavaterra}
\begin{itemize}
\item {fónica:cá}
\end{itemize}
\begin{itemize}
\item {Grp. gram.:m.}
\end{itemize}
Espécie de caranguejo, que se encontra á beira dos rios, onde fórma tocas, (\textunderscore cancer paragus\textunderscore , Lin.).
\section{Cavatina}
\begin{itemize}
\item {Grp. gram.:f.}
\end{itemize}
\begin{itemize}
\item {Proveniência:(It. \textunderscore cavatina\textunderscore )}
\end{itemize}
Pequena ária, composta ordinariamente num recitativo, com andamento ora vivo ora lento.
\section{Cavatura}
\begin{itemize}
\item {Grp. gram.:f.}
\end{itemize}
(V.cavadura)
\section{Cávea}
\begin{itemize}
\item {Grp. gram.:f.}
\end{itemize}
\begin{itemize}
\item {Proveniência:(Lat. \textunderscore cavea\textunderscore , de \textunderscore cavus\textunderscore )}
\end{itemize}
Jaula.
Gaiola.
Covil.
Lugar do povo nos espectáculos antigos.
\section{Cavedal}
\begin{itemize}
\item {Grp. gram.:m.}
\end{itemize}
Instrumento de espingardeiro.
\section{Caveira}
\begin{itemize}
\item {fónica:cá}
\end{itemize}
\begin{itemize}
\item {Grp. gram.:f.}
\end{itemize}
\begin{itemize}
\item {Utilização:Fig.}
\end{itemize}
\begin{itemize}
\item {Proveniência:(Do lat. \textunderscore calvaria\textunderscore )}
\end{itemize}
Crânio descarnado.
Rosto magro.
\section{Caveirado}
\begin{itemize}
\item {Grp. gram.:adj.}
\end{itemize}
Diz-se do solho, em que as tábuas não seguem todas a mesma direcção, sendo algumas dispostas de modo, que formam á roda do compartimento uma faixa ou dividem a superfície em differentes rectangulos.
(Vi o termo nos \textunderscore Pavimentos\textunderscore  de Liberato Telles, mas não entrevejo a razão do vocábulo)
\section{Caveiroso}
\begin{itemize}
\item {fónica:cá}
\end{itemize}
\begin{itemize}
\item {Grp. gram.:adj.}
\end{itemize}
Semelhante á càveira.
Muito magro.
\section{Cavendísia}
\begin{itemize}
\item {Grp. gram.:f.}
\end{itemize}
\begin{itemize}
\item {Proveniência:(De \textunderscore Cavendish\textunderscore , n. p.)}
\end{itemize}
Arbusto peruano, sempre verde e de fôlhas encarnadas.
\section{Caverna}
\begin{itemize}
\item {Grp. gram.:f.}
\end{itemize}
\begin{itemize}
\item {Utilização:Bras}
\end{itemize}
\begin{itemize}
\item {Proveniência:(Lat. \textunderscore caverna\textunderscore )}
\end{itemize}
Cavidade subterrânea; gruta; furna.
Cada uma das peças, que fórmam o arcaboiço do navio.
Cavidade anormal nos pulmões, que produz certo ruído ou fervores, revelando moléstia naquelle órgão.
\textunderscore Prego de caverna\textunderscore , o mesmo que \textunderscore prego caibral\textunderscore .
\section{Cavernal}
\begin{itemize}
\item {Grp. gram.:adj.}
\end{itemize}
\begin{itemize}
\item {Proveniência:(De \textunderscore caverna\textunderscore )}
\end{itemize}
Relativo a caverna, (falando-se especialmente de plantas, que se dão nos subterrâneos).
\section{Cavername}
\begin{itemize}
\item {Grp. gram.:m.}
\end{itemize}
\begin{itemize}
\item {Utilização:Fam.}
\end{itemize}
Conjunto das cavernas de uma embarcação.
Ossada, esqueleto.
(B. lat. \textunderscore cavernamen\textunderscore )
\section{Cavernosamente}
\begin{itemize}
\item {Grp. gram.:adv.}
\end{itemize}
De modo cavernoso.
\section{Cavernosidade}
\begin{itemize}
\item {Grp. gram.:f.}
\end{itemize}
Qualidade de um lugar ou corpo que tem cavernas.
Qualidade do que é cavernoso.
\section{Cavernoso}
\begin{itemize}
\item {Grp. gram.:adj.}
\end{itemize}
\begin{itemize}
\item {Proveniência:(Lat. \textunderscore cavernosus\textunderscore )}
\end{itemize}
Que tem cavernas.
Semelhante a cavernas.
Cavo, rouco, que parece sair de uma caverna: \textunderscore voz cavernosa\textunderscore .
\section{Caveto}
\begin{itemize}
\item {fónica:vê}
\end{itemize}
\begin{itemize}
\item {Grp. gram.:m.}
\end{itemize}
\begin{itemize}
\item {Proveniência:(De \textunderscore cava\textunderscore )}
\end{itemize}
Parte reentrante da cornija.
\section{Caiapónia}
\begin{itemize}
\item {Grp. gram.:f.}
\end{itemize}
Planta cucurbitácea do Brasil.
\section{Caiena}
\begin{itemize}
\item {Grp. gram.:f.}
\end{itemize}
Espécie de banana do Brasil.
\section{Caixa}
\begin{itemize}
\item {Grp. gram.:f.}
\end{itemize}
Moéda de deminuto valor, na Índia e noutras partes do Oriente.
(Do tamul \textunderscore kásu\textunderscore )
\section{Cávia}
\begin{itemize}
\item {Grp. gram.:m.}
\end{itemize}
Porquinho da Índia, (\textunderscore cavia oenoema\textunderscore ).
\section{Caviano}
\begin{itemize}
\item {Grp. gram.:adj.}
\end{itemize}
Relativo ao cávia.
\section{Caviar}
\begin{itemize}
\item {Grp. gram.:m.}
\end{itemize}
\begin{itemize}
\item {Proveniência:(Fr. \textunderscore caviar\textunderscore )}
\end{itemize}
Igúaria, composta de ovos salgados de esturjão.
\section{Cavicórneo}
\begin{itemize}
\item {Grp. gram.:adj.}
\end{itemize}
\begin{itemize}
\item {Proveniência:(Do lat. \textunderscore cavus\textunderscore  + \textunderscore cornu\textunderscore )}
\end{itemize}
Que tem cornos ocos.
\section{Cavidade}
\begin{itemize}
\item {Grp. gram.:f.}
\end{itemize}
\begin{itemize}
\item {Proveniência:(Lat. hyp. \textunderscore cavitas\textunderscore , de \textunderscore cavus\textunderscore )}
\end{itemize}
Espaço cavado de um côrpo sólido.
Cova; buraco; caverna.
\section{Cavidadoiro}
\begin{itemize}
\item {Grp. gram.:adj.}
\end{itemize}
\begin{itemize}
\item {Proveniência:(De \textunderscore cavidar\textunderscore )}
\end{itemize}
O mesmo que \textunderscore cauteloso\textunderscore .
\section{Cavidadouro}
\begin{itemize}
\item {Grp. gram.:adj.}
\end{itemize}
\begin{itemize}
\item {Proveniência:(De \textunderscore cavidar\textunderscore )}
\end{itemize}
O mesmo que \textunderscore cauteloso\textunderscore .
\section{Cavidar}
\begin{itemize}
\item {Grp. gram.:v. t.}
\end{itemize}
\begin{itemize}
\item {Utilização:Ant.}
\end{itemize}
Acautelar, prever.
(Cp. lat. \textunderscore cavere\textunderscore )
\section{Cavilação}
\begin{itemize}
\item {Grp. gram.:f.}
\end{itemize}
\begin{itemize}
\item {Proveniência:(Lat. \textunderscore cavillatio\textunderscore )}
\end{itemize}
Sophisma.
Ardil; astúcia.
Ironia maliciosa.
\section{Cavilador}
\begin{itemize}
\item {Grp. gram.:m.}
\end{itemize}
\begin{itemize}
\item {Proveniência:(De \textunderscore cavillar\textunderscore )}
\end{itemize}
Aquelle que emprega cavilações.
\section{Cavilar}
\begin{itemize}
\item {Grp. gram.:v. i.}
\end{itemize}
\begin{itemize}
\item {Utilização:Des.}
\end{itemize}
\begin{itemize}
\item {Proveniência:(Lat. \textunderscore cavillari\textunderscore )}
\end{itemize}
Usar de cavilações.
\section{Cavilha}
\begin{itemize}
\item {Grp. gram.:f.}
\end{itemize}
\begin{itemize}
\item {Proveniência:(Lat. \textunderscore clavicula\textunderscore )}
\end{itemize}
Peça de madeira ou metal, para segurar ou juntar madeiros ou chapas.
Appendice ósseo de cada lado do temporal dos animaes cavicórneos, e no qual estão como inseridos os cornos.
\section{Cavilhador}
\begin{itemize}
\item {Grp. gram.:m.}
\end{itemize}
Aquelle que cavilha.
\section{Cavilhar}
\begin{itemize}
\item {Grp. gram.:v. t.}
\end{itemize}
Segurar com cavilha.
Maquinar com astúcia, cavillar. Cf. Filinto, \textunderscore D. Man.\textunderscore , II, 70.
\section{Cavillação}
\begin{itemize}
\item {Grp. gram.:f.}
\end{itemize}
\begin{itemize}
\item {Proveniência:(Lat. \textunderscore cavillatio\textunderscore )}
\end{itemize}
Sophisma.
Ardil; astúcia.
Ironia maliciosa.
\section{Cavillador}
\begin{itemize}
\item {Grp. gram.:m.}
\end{itemize}
\begin{itemize}
\item {Proveniência:(De \textunderscore cavillar\textunderscore )}
\end{itemize}
Aquelle que emprega cavillações.
\section{Cavillar}
\begin{itemize}
\item {Grp. gram.:v. i.}
\end{itemize}
\begin{itemize}
\item {Utilização:Des.}
\end{itemize}
\begin{itemize}
\item {Proveniência:(Lat. \textunderscore cavillari\textunderscore )}
\end{itemize}
Usar de cavillações.
\section{Cavillosamente}
\begin{itemize}
\item {Grp. gram.:adv.}
\end{itemize}
De modo cavilloso.
\section{Cavilloso}
\begin{itemize}
\item {Grp. gram.:adj.}
\end{itemize}
\begin{itemize}
\item {Proveniência:(Lat. \textunderscore cavillosus\textunderscore )}
\end{itemize}
Que emprega cavillações.
Em que há cavillação.
\section{Cavil-maril}
\begin{itemize}
\item {Grp. gram.:m.}
\end{itemize}
Arbusto silvestre de Angola, de casca fendida e flôres vermelhas.
\section{Cavilosamente}
\begin{itemize}
\item {Grp. gram.:adv.}
\end{itemize}
De modo caviloso.
\section{Caviloso}
\begin{itemize}
\item {Grp. gram.:adj.}
\end{itemize}
\begin{itemize}
\item {Proveniência:(Lat. \textunderscore cavillosus\textunderscore )}
\end{itemize}
Que emprega cavilações.
Em que há cavilação.
\section{Cavirostro}
\begin{itemize}
\item {fónica:virrós}
\end{itemize}
\begin{itemize}
\item {Grp. gram.:adj.}
\end{itemize}
\begin{itemize}
\item {Utilização:Zool.}
\end{itemize}
\begin{itemize}
\item {Proveniência:(Do lat. \textunderscore cavus\textunderscore  + \textunderscore rostrum\textunderscore )}
\end{itemize}
Que tem bico oco.
\section{Cavirrostro}
\begin{itemize}
\item {Grp. gram.:adj.}
\end{itemize}
\begin{itemize}
\item {Utilização:Zool.}
\end{itemize}
\begin{itemize}
\item {Proveniência:(Do lat. \textunderscore cavus\textunderscore  + \textunderscore rostrum\textunderscore )}
\end{itemize}
Que tem bico oco.
\section{Cavitário}
\begin{itemize}
\item {Grp. gram.:adj.}
\end{itemize}
\begin{itemize}
\item {Utilização:Anat.}
\end{itemize}
Diz-se de certos órgãos, que estão numa cavidade.
Que tem cavidade.
Relativo a cavidade.
(Cp. \textunderscore cavidade\textunderscore )
\section{Cavo}
\begin{itemize}
\item {Grp. gram.:adj.}
\end{itemize}
\begin{itemize}
\item {Proveniência:(Lat. \textunderscore cavus\textunderscore )}
\end{itemize}
Côncavo.
Oco.
Rouco, cavernoso: \textunderscore voz cava\textunderscore .
\section{Cavolina}
\begin{itemize}
\item {Grp. gram.:f.}
\end{itemize}
Gênero de molluscos, que andam de costas.
\section{Cavoucar}
\begin{itemize}
\item {Grp. gram.:v. t.}
\end{itemize}
Abrir cavouco em.
\section{Cavouco}
\textunderscore m.\textunderscore  (e der.)
O mesmo ou melhor que \textunderscore cabouco\textunderscore , etc.
\section{Caxa}
\begin{itemize}
\item {Grp. gram.:f.}
\end{itemize}
Moéda de deminuto valor, na Índia e noutras partes do Oriente.
(Do tamul \textunderscore kásu\textunderscore )
\section{Caxambu}
\begin{itemize}
\item {Grp. gram.:m.}
\end{itemize}
\begin{itemize}
\item {Utilização:Bras}
\end{itemize}
Espécie de batuque, que os Negros dançam ao som do tambor.
\section{Caxarela}
\begin{itemize}
\item {Grp. gram.:m.}
\end{itemize}
\begin{itemize}
\item {Utilização:Bras}
\end{itemize}
O macho da baleia.
\section{Caxarelo}
\begin{itemize}
\item {Grp. gram.:m.}
\end{itemize}
O mesmo que \textunderscore caxarela\textunderscore .
\section{Caxerenguengue}
\begin{itemize}
\item {Grp. gram.:m.}
\end{itemize}
\begin{itemize}
\item {Utilização:Bras. de Minas}
\end{itemize}
Faca velha e inútil.
\section{Caxexe}
\begin{itemize}
\item {Grp. gram.:m.}
\end{itemize}
Pássaro conirostro africano.
\section{Caxianguelé}
\begin{itemize}
\item {Grp. gram.:m.}
\end{itemize}
(V.caxinguelê)
\section{Caxicante}
\begin{itemize}
\item {Grp. gram.:m.}
\end{itemize}
Ave africana (\textunderscore pratincola torquata\textunderscore ).
\section{Caxim}
\begin{itemize}
\item {Grp. gram.:m.}
\end{itemize}
Planta euphorbiácea do Brasil.
\section{Caximbeque}
\begin{itemize}
\item {Grp. gram.:m.}
\end{itemize}
O mesmo que \textunderscore calhambeque\textunderscore .
\section{Caxinduba}
\begin{itemize}
\item {Grp. gram.:f.}
\end{itemize}
Árvore medicinal do Alto-Amazonas.
\section{Caxinganguelê}
\begin{itemize}
\item {Grp. gram.:m.}
\end{itemize}
(V.caxinguelê)
\section{Caxingar}
\begin{itemize}
\item {Grp. gram.:v. i.}
\end{itemize}
\begin{itemize}
\item {Utilização:Bras}
\end{itemize}
Coxear.
(Por \textunderscore coxingar\textunderscore , de \textunderscore coxo\textunderscore )
\section{Caxinglé}
\begin{itemize}
\item {Grp. gram.:m.}
\end{itemize}
(V.caxinguelê)
\section{Caxingó}
\begin{itemize}
\item {Grp. gram.:adj.}
\end{itemize}
\begin{itemize}
\item {Utilização:Bras. do Ceará}
\end{itemize}
\begin{itemize}
\item {Utilização:Bras. do Amaz}
\end{itemize}
Que caxinga, que coxeia.
Ruim, detestável, (falando-se de gado ou de carne).
\section{Caxingo-anguluve}
\begin{itemize}
\item {Grp. gram.:m.}
\end{itemize}
Ave laniádea da África.
\section{Caxinguelê}
\begin{itemize}
\item {Grp. gram.:m.}
\end{itemize}
\begin{itemize}
\item {Utilização:Bras}
\end{itemize}
Mammífero roedor.
(Talvez t. afr.)
\section{Caxirenguengue}
\begin{itemize}
\item {Grp. gram.:m.}
\end{itemize}
\begin{itemize}
\item {Utilização:Bras}
\end{itemize}
(V.caxerenguengue)
\section{Caxiri}
\begin{itemize}
\item {Grp. gram.:m.}
\end{itemize}
\begin{itemize}
\item {Utilização:Bras}
\end{itemize}
Caxirenguengue.
Iguaría, feita de beiju diluído em água.
\section{Caxitico}
\begin{itemize}
\item {Grp. gram.:m.}
\end{itemize}
Ave africana, (\textunderscore parus afer\textunderscore ).
\section{Caxixi}
\begin{itemize}
\item {Grp. gram.:adj.}
\end{itemize}
\begin{itemize}
\item {Utilização:Bras}
\end{itemize}
\begin{itemize}
\item {Grp. gram.:M.}
\end{itemize}
Diz-se da aguardente de qualidade inferior.
Aguardente ordinária.
\section{Caxumba}
\begin{itemize}
\item {Grp. gram.:f.}
\end{itemize}
\begin{itemize}
\item {Utilização:Bras}
\end{itemize}
Inflammação da parótida, trasorelho, papeira.
\section{Caxundé}
\begin{itemize}
\item {Grp. gram.:m.}
\end{itemize}
\begin{itemize}
\item {Utilização:Ant.}
\end{itemize}
\begin{itemize}
\item {Proveniência:(T. afr.)}
\end{itemize}
Confeito odorífero e estomacal.
\section{Cayapónia}
\begin{itemize}
\item {Grp. gram.:f.}
\end{itemize}
Planta cucurbitácea do Brasil.
\section{Cayena}
\begin{itemize}
\item {Grp. gram.:f.}
\end{itemize}
Espécie de banana do Brasil.
\section{Cazembe}
\begin{itemize}
\item {Grp. gram.:m.}
\end{itemize}
\begin{itemize}
\item {Utilização:T. da Afr. Or. port}
\end{itemize}
Commandante de ensaca.
\section{Cazembi}
\begin{itemize}
\item {Grp. gram.:m.}
\end{itemize}
Espécie de acácia de Angola e da Zambézia.
\section{Cazol}
\begin{itemize}
\item {Grp. gram.:m.}
\end{itemize}
Cosmético oriental.
\section{Cazonzonzo}
\begin{itemize}
\item {Grp. gram.:m.}
\end{itemize}
Árvore silvestre da África intertropical.
\section{Cazoró}
\begin{itemize}
\item {Grp. gram.:m.}
\end{itemize}
Árvore da Índia portuguesa.
\section{Cazu}
\begin{itemize}
\item {Grp. gram.:m.}
\end{itemize}
Árvore africana, de frutos comestíveis. Mammífero da Guiné.
\section{Cear}
\begin{itemize}
\item {Grp. gram.:v. t.}
\end{itemize}
\begin{itemize}
\item {Grp. gram.:V. i.}
\end{itemize}
\begin{itemize}
\item {Proveniência:(Lat. \textunderscore coenare\textunderscore , de \textunderscore coena\textunderscore )}
\end{itemize}
Comer, na occasião da ceia.
Comer a ceia.
\section{Cearense}
\begin{itemize}
\item {Grp. gram.:adj.}
\end{itemize}
\begin{itemize}
\item {Grp. gram.:M.}
\end{itemize}
Relativo ao Estado do Ceará, no Brasil.
Habitante do Ceará.
\section{Ceba-do-rio}
\begin{itemize}
\item {Grp. gram.:f.}
\end{itemize}
\begin{itemize}
\item {Utilização:T. de Aveiro}
\end{itemize}
Planta marinha, monocotyledónea, (\textunderscore rostera marina\textunderscore . Lin.).
\section{Cebiano}
\begin{itemize}
\item {Grp. gram.:m.}
\end{itemize}
O mesmo que \textunderscore cébus\textunderscore .
\section{Cebino}
\begin{itemize}
\item {Grp. gram.:m.}
\end{itemize}
O mesmo que \textunderscore cébus\textunderscore .
\section{Cebipira}
\begin{itemize}
\item {Grp. gram.:f.}
\end{itemize}
Árvore do Brasil.
\section{Cebo}
\begin{itemize}
\item {Grp. gram.:m.}
\end{itemize}
\begin{itemize}
\item {Proveniência:(Gr. \textunderscore kebos\textunderscore , macaco)}
\end{itemize}
Designação scientifica de uma espécie de pequenos quadrúmanos, procedentes da América.
\section{Cebola}
\begin{itemize}
\item {fónica:bô}
\end{itemize}
\begin{itemize}
\item {Grp. gram.:f.}
\end{itemize}
\begin{itemize}
\item {Utilização:Pop.}
\end{itemize}
\begin{itemize}
\item {Proveniência:(Do lat. \textunderscore caepulla\textunderscore )}
\end{itemize}
Planta bulbosa, hortense, (\textunderscore allium caepa\textunderscore ).
Bolbo da cebola.
Bolbo.
Relógio antigo de prata, para algibeira.
Pessôa muito indolente, enfraquecida, fatigada.
\section{Cebolada}
\begin{itemize}
\item {Grp. gram.:f.}
\end{itemize}
\begin{itemize}
\item {Proveniência:(De \textunderscore cebôla\textunderscore )}
\end{itemize}
Iguaria, guisada ou frita com cebolas.
Guisado, môlho, de cebolas: \textunderscore bife de cebolada\textunderscore .
\section{Cebola-de-lobo}
\begin{itemize}
\item {Grp. gram.:f.}
\end{itemize}
Planta medicinal da Guiné.
\section{Cebolal}
\begin{itemize}
\item {Grp. gram.:m.}
\end{itemize}
\begin{itemize}
\item {Grp. gram.:Adj.}
\end{itemize}
\begin{itemize}
\item {Proveniência:(De \textunderscore cebola\textunderscore )}
\end{itemize}
Plantação de cebolas.
Diz-se de uma variedade de maçan.
\section{Cebolão}
\begin{itemize}
\item {Grp. gram.:m.}
\end{itemize}
\begin{itemize}
\item {Utilização:Fam.}
\end{itemize}
Relógio grande de prata. Cf. Eça, \textunderscore Padre Amaro\textunderscore , 183.
\section{Ceboleira}
\begin{itemize}
\item {Grp. gram.:adj. f.}
\end{itemize}
\begin{itemize}
\item {Utilização:Prov.}
\end{itemize}
\begin{itemize}
\item {Utilização:trasm.}
\end{itemize}
Diz-se da maçan, também chamada \textunderscore baionesa\textunderscore .
\section{Cebolinha}
\begin{itemize}
\item {Grp. gram.:f.}
\end{itemize}
Espécie de cebola pequena.
\section{Cebolinho}
\begin{itemize}
\item {Grp. gram.:m.}
\end{itemize}
\begin{itemize}
\item {Utilização:Prov.}
\end{itemize}
\begin{itemize}
\item {Utilização:trasm.}
\end{itemize}
\begin{itemize}
\item {Proveniência:(De \textunderscore cebola\textunderscore )}
\end{itemize}
Semente de cebola.
A planta tenra da cebola, antes de formado o bolbo.
Canteiro, em que há alfobre de cebolas.
\section{Cebolo}
\begin{itemize}
\item {fónica:bô}
\end{itemize}
\begin{itemize}
\item {Grp. gram.:m.}
\end{itemize}
Semente de cebolas.
Pé de cebola, antes de formado o bolbo.
Alfobre de cebolas, antes de plantadas.
(Cp. \textunderscore cebola\textunderscore )
\section{Cebolório!}
\begin{itemize}
\item {Grp. gram.:interj.}
\end{itemize}
\begin{itemize}
\item {Utilização:Pop.}
\end{itemize}
\begin{itemize}
\item {Proveniência:(De \textunderscore cebola\textunderscore )}
\end{itemize}
(para indicar desdém ou descontentamento)
\section{Cebuano}
\begin{itemize}
\item {Grp. gram.:m.}
\end{itemize}
Um dos dialecto das Filippinas.
\section{Cébus}
\begin{itemize}
\item {Grp. gram.:m.}
\end{itemize}
\begin{itemize}
\item {Proveniência:(Gr. \textunderscore kebos\textunderscore , macaco)}
\end{itemize}
Designação scientifica de uma espécie de pequenos quadrúmanos, procedentes da América.
\section{Cecal}
\begin{itemize}
\item {Grp. gram.:adj.}
\end{itemize}
\begin{itemize}
\item {Utilização:Anat.}
\end{itemize}
Relativo ao ceco: \textunderscore appêndice cecal\textunderscore .
\section{Cecear}
\begin{itemize}
\item {Grp. gram.:v. i.}
\end{itemize}
\begin{itemize}
\item {Utilização:Prov.}
\end{itemize}
Falar affectadamente, á lisboêta, pronunciando o \textunderscore s\textunderscore  intervocálico como \textunderscore z\textunderscore , ou os \textunderscore ss\textunderscore  como \textunderscore ç\textunderscore .
\section{Ceceio}
\begin{itemize}
\item {Grp. gram.:m.}
\end{itemize}
Acto de \textunderscore cecear\textunderscore .
\section{Cecém}
\begin{itemize}
\item {Grp. gram.:f.}
\end{itemize}
O mesmo que \textunderscore açucena\textunderscore .
\section{Ceceoso}
\begin{itemize}
\item {Grp. gram.:adj.}
\end{itemize}
Que ceceia.
\section{Cecídia}
\begin{itemize}
\item {Grp. gram.:f.}
\end{itemize}
\begin{itemize}
\item {Utilização:Bot.}
\end{itemize}
Engrossamento, mais ou menos unilateral, do colmo, a pequena altura da terra.
\section{Cecidómia}
\begin{itemize}
\item {Grp. gram.:f.}
\end{itemize}
Insecto díptero, que ataca principalmente as gramíneas.
\section{Cecidomía}
\begin{itemize}
\item {Grp. gram.:f.}
\end{itemize}
Doença, prozida pela \textunderscore cecidómia\textunderscore .
\section{Cecília}
\begin{itemize}
\item {Grp. gram.:f.}
\end{itemize}
\begin{itemize}
\item {Proveniência:(Lat. \textunderscore caecilia\textunderscore , de \textunderscore caecus\textunderscore )}
\end{itemize}
Reptil amphíbio da América do Sul, sem membros e sem olhos apparentes.
\section{Ceco}
\begin{itemize}
\item {Grp. gram.:m.}
\end{itemize}
O mesmo ou melhor que \textunderscore cécum\textunderscore .
\section{Cecrópia}
\begin{itemize}
\item {Grp. gram.:f.}
\end{itemize}
Espécie de planta brasileira.
\section{Cécuba}
\begin{itemize}
\item {Grp. gram.:f.}
\end{itemize}
Vinho do monte Cécubo, na Campânia.
\section{Cécubo}
\begin{itemize}
\item {Grp. gram.:m.}
\end{itemize}
O mesmo que \textunderscore cécuba\textunderscore .
\section{Cécum}
\begin{itemize}
\item {Grp. gram.:m.}
\end{itemize}
\begin{itemize}
\item {Proveniência:(Lat. \textunderscore caecum\textunderscore )}
\end{itemize}
A parte mais larga do intestino grosso.
\section{Cecydómia}
\begin{itemize}
\item {Grp. gram.:f.}
\end{itemize}
Insecto díptero, que ataca principalmente as gramíneas.
\section{Cecydomía}
\begin{itemize}
\item {Grp. gram.:f.}
\end{itemize}
Doença, prozida pela \textunderscore cecydómia\textunderscore .
\section{Cedência}
\begin{itemize}
\item {Grp. gram.:f.}
\end{itemize}
\begin{itemize}
\item {Proveniência:(De \textunderscore ceder\textunderscore )}
\end{itemize}
O mesmo que \textunderscore cessão\textunderscore .
\section{Cedente}
\begin{itemize}
\item {Grp. gram.:adj.}
\end{itemize}
\begin{itemize}
\item {Proveniência:(Lat. \textunderscore cedens\textunderscore )}
\end{itemize}
Que cede.
\section{Ceder}
\begin{itemize}
\item {Grp. gram.:v. t.}
\end{itemize}
\begin{itemize}
\item {Grp. gram.:V. i.}
\end{itemize}
\begin{itemize}
\item {Proveniência:(Lat. \textunderscore cedere\textunderscore )}
\end{itemize}
Transferir o direito ou a posse de.
Pôr á disposição de alguém: \textunderscore ceder a casa a um amigo\textunderscore .
Curvar-se ao pêso.
Sujeitar-se; transigir: \textunderscore ceder a imposições\textunderscore .
Abalar-se.
\section{Cediço}
\begin{itemize}
\item {Grp. gram.:adj.}
\end{itemize}
(Fórma preferïvel a \textunderscore sediço\textunderscore , seg. G. Viana, \textunderscore Apostilas\textunderscore )
\section{Cedilha}
\begin{itemize}
\item {Grp. gram.:f.}
\end{itemize}
\begin{itemize}
\item {Proveniência:(Do it. \textunderscore zediglia\textunderscore )}
\end{itemize}
Sinal gráphico, indicativo de que, sotopôsto ao \textunderscore c\textunderscore  que precede \textunderscore a\textunderscore , \textunderscore o\textunderscore  ou \textunderscore u\textunderscore , dá ao mesmo o valor de \textunderscore ss\textunderscore .
\section{Cedilhado}
\begin{itemize}
\item {Grp. gram.:adj.}
\end{itemize}
Que tem cedilha: \textunderscore um C cedilhado\textunderscore .
\section{Cedilhar}
\begin{itemize}
\item {Grp. gram.:v. t.}
\end{itemize}
Pôr cedilha em.
\section{Cedimento}
\begin{itemize}
\item {Grp. gram.:m.}
\end{itemize}
\begin{itemize}
\item {Proveniência:(De \textunderscore ceder\textunderscore )}
\end{itemize}
O mesmo que \textunderscore cessão\textunderscore .
\section{Cedinho}
\begin{itemize}
\item {Grp. gram.:adv.}
\end{itemize}
Muito cedo; logo de manhan.
\section{Cedível}
\begin{itemize}
\item {Grp. gram.:adj.}
\end{itemize}
Que se póde ceder.
\section{Cedo}
\begin{itemize}
\item {fónica:cê}
\end{itemize}
\begin{itemize}
\item {Grp. gram.:adv.}
\end{itemize}
\begin{itemize}
\item {Grp. gram.:M.}
\end{itemize}
\begin{itemize}
\item {Proveniência:(Lat. \textunderscore cito\textunderscore )}
\end{itemize}
Antes da occasião própria.
Temporanmente; prematuramente: \textunderscore êste anno o calor veio cedo\textunderscore .
De madrugada: \textunderscore levantar-se cedo\textunderscore .
Depressa.
De prompto.
\textunderscore No cedo\textunderscore , antes da plena estação:«\textunderscore herdade, bôa de pastagem no cedo\textunderscore ». Fialho de Almeida.
\section{Cedoiro}
\begin{itemize}
\item {Grp. gram.:m.}
\end{itemize}
\begin{itemize}
\item {Proveniência:(De \textunderscore cedo\textunderscore )}
\end{itemize}
Casta de uva.
\section{Cedouro}
\begin{itemize}
\item {Grp. gram.:m.}
\end{itemize}
\begin{itemize}
\item {Proveniência:(De \textunderscore cedo\textunderscore )}
\end{itemize}
Casta de uva.
\section{Cedovém}
\begin{itemize}
\item {Grp. gram.:m.}
\end{itemize}
\begin{itemize}
\item {Utilização:Prov.}
\end{itemize}
O mesmo que \textunderscore megengra\textunderscore .
\section{Cedrela}
\begin{itemize}
\item {Grp. gram.:f.}
\end{itemize}
Árvore da América, de madeira vermelha e odorífera, (\textunderscore cedrela odorata\textunderscore , Lin.).
\section{Cedreláceas}
\begin{itemize}
\item {Grp. gram.:f. pl.}
\end{itemize}
Família de plantas, que tem por typo a cedrela.
\section{Cédria}
\begin{itemize}
\item {Grp. gram.:f.}
\end{itemize}
\begin{itemize}
\item {Proveniência:(Lat. \textunderscore cedria\textunderscore )}
\end{itemize}
Resina de cedro.
\section{Cedrino}
\begin{itemize}
\item {Grp. gram.:adj.}
\end{itemize}
Relativo ao cedro.
\section{Cedríntio}
\begin{itemize}
\item {Grp. gram.:m.}
\end{itemize}
Árvore do Brasil.
\section{Cedrita}
\begin{itemize}
\item {Grp. gram.:f.}
\end{itemize}
\begin{itemize}
\item {Proveniência:(De \textunderscore cedro\textunderscore )}
\end{itemize}
Medicamento vermífugo, preparado com vinho e resina de cedro.
\section{Cedro}
\begin{itemize}
\item {Grp. gram.:m.}
\end{itemize}
\begin{itemize}
\item {Proveniência:(Lat. \textunderscore cedrus\textunderscore )}
\end{itemize}
Árvore conífera.
\section{Cedronela}
\begin{itemize}
\item {Grp. gram.:f.}
\end{itemize}
Planta labiáda do México e das Canárias.
\section{Cedrosta}
\begin{itemize}
\item {Grp. gram.:f.}
\end{itemize}
Cobra branca.
\section{Cédula}
\begin{itemize}
\item {Grp. gram.:f.}
\end{itemize}
\begin{itemize}
\item {Proveniência:(Do lat. \textunderscore schedula\textunderscore )}
\end{itemize}
Designação de documentos escritos de vária natureza.
Apontamento.
Simples confissão de divida, escrita mas não legalizada.
Apólice.
Papel, representativo de moéda de curso legal.
\section{Ceeiro}
\begin{itemize}
\item {Grp. gram.:m.}
\end{itemize}
\begin{itemize}
\item {Utilização:Ant.}
\end{itemize}
\begin{itemize}
\item {Proveniência:(De \textunderscore ceia\textunderscore )}
\end{itemize}
Trabalhador que, occupando-se em casa ou fazenda alheia, recebia como ceia um pão caseiro.
\section{Cefo}
\begin{itemize}
\item {Grp. gram.:m.}
\end{itemize}
Ruminante de Angola.
\section{Cega}
\begin{itemize}
\item {Grp. gram.:f.}
\end{itemize}
\begin{itemize}
\item {Grp. gram.:Pl. Loc. adv.}
\end{itemize}
\begin{itemize}
\item {Utilização:Fig.}
\end{itemize}
\begin{itemize}
\item {Proveniência:(De \textunderscore cego\textunderscore )}
\end{itemize}
Mulher que não vê.
O mesmo que \textunderscore cecília\textunderscore .
\textunderscore Ás cegas\textunderscore , cegamente; na escuridão.
Inconscientemente.
\section{Cègada}
\begin{itemize}
\item {Grp. gram.:f.}
\end{itemize}
\begin{itemize}
\item {Proveniência:(De \textunderscore cego\textunderscore )}
\end{itemize}
Ajuntamento de figuras mascaradas, que pelo Carnaval percorrem as ruas de Lisbôa, cantando e pedindo esmola, á imitação de cegos.
\section{Cegagem}
\begin{itemize}
\item {Grp. gram.:f.}
\end{itemize}
\begin{itemize}
\item {Utilização:Agr.}
\end{itemize}
\begin{itemize}
\item {Proveniência:(De \textunderscore cegar\textunderscore )}
\end{itemize}
Suppressão dos olhos ou gemas das árvores.
\section{Cegamente}
\begin{itemize}
\item {fónica:cé}
\end{itemize}
\begin{itemize}
\item {Grp. gram.:adv.}
\end{itemize}
Á maneira de cego; com cegueira.
Inconscientemente.
\section{Cegamento}
\begin{itemize}
\item {Grp. gram.:m.}
\end{itemize}
O mesmo que \textunderscore cegueira\textunderscore .
\section{Ceganucho}
\begin{itemize}
\item {Grp. gram.:m.  e  adj.}
\end{itemize}
\begin{itemize}
\item {Utilização:Prov.}
\end{itemize}
\begin{itemize}
\item {Utilização:trasm.}
\end{itemize}
O que é cego de um só ôlho.
\section{Cega-ôlho}
\begin{itemize}
\item {Grp. gram.:m.}
\end{itemize}
\begin{itemize}
\item {Utilização:Bras}
\end{itemize}
O mesmo que \textunderscore official-da-sala\textunderscore .
\section{Cegar}
\begin{itemize}
\item {Grp. gram.:v. t.}
\end{itemize}
\begin{itemize}
\item {Utilização:Fig.}
\end{itemize}
\begin{itemize}
\item {Grp. gram.:V. i.}
\end{itemize}
\begin{itemize}
\item {Proveniência:(Do lat. \textunderscore caecare\textunderscore )}
\end{itemize}
Tornar cego, tirar a vista a.
Alucinar: \textunderscore cegou-o a paixão\textunderscore .
Deslumbrar.
Illudir.
Perder a vista; deixar de vêr.
\section{Cegar}
\begin{itemize}
\item {Grp. gram.:v. t.}
\end{itemize}
Obstruir; tapar, entupindo.
\section{Cègar}
\begin{itemize}
\item {Grp. gram.:v. t.}
\end{itemize}
\begin{itemize}
\item {Utilização:Bras. do N}
\end{itemize}
\begin{itemize}
\item {Proveniência:(De \textunderscore cego\textunderscore )}
\end{itemize}
Embotar, tirar o fio ou gume de (facas ou outros instrumentos).
\section{Cegarrega}
\begin{itemize}
\item {fónica:cé}
\end{itemize}
\begin{itemize}
\item {Grp. gram.:f.}
\end{itemize}
\begin{itemize}
\item {Utilização:Fam.}
\end{itemize}
Cigarra.
Instrumento, que imita o retinir da cigarra.
Pessôa muito faladora, de voz desagradável e impertinente.
(Por \textunderscore cigarrega\textunderscore , de \textunderscore cigarra\textunderscore )
\section{Cego}
\begin{itemize}
\item {Grp. gram.:adj.}
\end{itemize}
\begin{itemize}
\item {Utilização:Fig.}
\end{itemize}
\begin{itemize}
\item {Utilização:Bras. do N}
\end{itemize}
\begin{itemize}
\item {Grp. gram.:M.}
\end{itemize}
\begin{itemize}
\item {Utilização:P. us.}
\end{itemize}
\begin{itemize}
\item {Proveniência:(Do lat. \textunderscore caecus\textunderscore )}
\end{itemize}
Privado do sentido da vista; que não vê.
Alucinado.
Inconsciente.
Desvairado.
Obscuro.
Obliterado.
Entulhado.
Diz-se do instrumento, que perdeu o fio ou que corta mal.
Diz-se do intestino, chamado \textunderscore ceco\textunderscore .
\textunderscore Nó cego\textunderscore , o nó que difficilmente se desata, ou que se não póde desatar.
Homem, que não vê.
Cécum, ceco, ou intestino cego.
Baixio.
\section{Cegonha}
\begin{itemize}
\item {Grp. gram.:f.}
\end{itemize}
\begin{itemize}
\item {Utilização:Prov.}
\end{itemize}
\begin{itemize}
\item {Proveniência:(Do lat. \textunderscore ciconia\textunderscore )}
\end{itemize}
Ave pernalta de arribação.
Engenho tôsco, para extrahir água, o mesmo que \textunderscore burra\textunderscore .
O mesmo que \textunderscore bebedeira\textunderscore .
\section{Cegonho}
\begin{itemize}
\item {Grp. gram.:m.}
\end{itemize}
Engenho tôsco para extrahir água, o mesmo que \textunderscore cegonha\textunderscore . (Colhido em Miranda e na Bairrada)
\section{Cegude}
\begin{itemize}
\item {Grp. gram.:f.}
\end{itemize}
Designação pop. da \textunderscore cicuta\textunderscore .
\section{Cegudo}
\begin{itemize}
\item {Grp. gram.:m.}
\end{itemize}
\begin{itemize}
\item {Utilização:Ant.}
\end{itemize}
O mesmo que \textunderscore cegude\textunderscore .
\section{Cegueira}
\begin{itemize}
\item {Grp. gram.:f.}
\end{itemize}
\begin{itemize}
\item {Utilização:Fig.}
\end{itemize}
Estado do que é cego.
Extrema affeição a alguém ou a alguma coisa.
\section{Cègueta}
\begin{itemize}
\item {fónica:guê}
\end{itemize}
\begin{itemize}
\item {Grp. gram.:m.  e  f.}
\end{itemize}
Pessôa que é pitosga ou pisca.
\section{Ceguidade}
\begin{itemize}
\item {Grp. gram.:f.}
\end{itemize}
(V.cegueira)
\section{Ceguidão}
\begin{itemize}
\item {Grp. gram.:f.}
\end{itemize}
O mesmo que \textunderscore cegueira\textunderscore .
\section{Ceia}
\begin{itemize}
\item {Grp. gram.:f.}
\end{itemize}
\begin{itemize}
\item {Proveniência:(Do lat. \textunderscore coena\textunderscore )}
\end{itemize}
Refeição, que se toma á noite, e geralmente a última que se toma em cada dia.
\section{Ceiba}
\begin{itemize}
\item {Grp. gram.:f.}
\end{itemize}
Árvore malvácea.
\section{Ceiçal}
\begin{itemize}
\item {Grp. gram.:m.}
\end{itemize}
O mesmo que \textunderscore sinceiral\textunderscore .
\section{Ceiceiral}
\begin{itemize}
\item {Grp. gram.:m.}
\end{itemize}
O mesmo que \textunderscore sinceiral\textunderscore . Cf. Garrett, \textunderscore Arco de Sant'Anna\textunderscore , I, 204.
\section{Ceiceiro}
\begin{itemize}
\item {Grp. gram.:m.}
\end{itemize}
O mesmo que \textunderscore sinceiro\textunderscore :«\textunderscore ao pé dum gran ceiceiro\textunderscore ». Ferreira, \textunderscore Éclogas\textunderscore .
\section{Ceifa}
\begin{itemize}
\item {Grp. gram.:f.}
\end{itemize}
\begin{itemize}
\item {Utilização:Fig.}
\end{itemize}
\begin{itemize}
\item {Proveniência:(Do ár. \textunderscore aç-ceif\textunderscore )}
\end{itemize}
Acto de ceifar.
Época de ceifar.
Colheita de cereaes.
Mortandade.
\section{Ceifão}
\begin{itemize}
\item {Grp. gram.:m.}
\end{itemize}
(V.ceifeiro)
\section{Ceifar}
\begin{itemize}
\item {Grp. gram.:v. t.}
\end{itemize}
\begin{itemize}
\item {Proveniência:(De \textunderscore ceifa\textunderscore )}
\end{itemize}
Cortar, segar.
Abater (searas maduras) com foice ou outro instrumento apropriado.
\section{Ceifeira}
\begin{itemize}
\item {Grp. gram.:f.}
\end{itemize}
Mulher, que ceifa.
Instrumento ou máquina para ceifar.
\section{Ceifeiro}
\begin{itemize}
\item {Grp. gram.:adj.}
\end{itemize}
\begin{itemize}
\item {Grp. gram.:M.}
\end{itemize}
\begin{itemize}
\item {Proveniência:(De \textunderscore ceifar\textunderscore )}
\end{itemize}
Relativo á ceifa.
Homem, que ceifa.
\section{Ceifões}
\begin{itemize}
\item {Grp. gram.:m. pl.}
\end{itemize}
(V.çafões)
\section{Ceijupira}
\begin{itemize}
\item {Grp. gram.:f.}
\end{itemize}
Grande peixe dos mares do Brasil.
\section{Ceilanita}
\begin{itemize}
\item {Grp. gram.:f.}
\end{itemize}
\begin{itemize}
\item {Proveniência:(De \textunderscore Ceilão\textunderscore , n. p.)}
\end{itemize}
Substância pedregosa e negra de Ceilão.
Espinella.
\section{Ceilanite}
\begin{itemize}
\item {Grp. gram.:f.}
\end{itemize}
\begin{itemize}
\item {Proveniência:(De \textunderscore Ceilão\textunderscore , n. p.)}
\end{itemize}
Substância pedregosa e negra de Ceilão.
Espinella.
\section{Ceira}
\textunderscore f.\textunderscore  (e der.)
(V. \textunderscore seira\textunderscore , etc.)
\section{Ceita}
\begin{itemize}
\item {Grp. gram.:f.}
\end{itemize}
Tributo de 10 reis por família, que antigamente se pagava, para se ficar isento de ir pessoalmente servir na praça de \textunderscore Ceuta\textunderscore  ou \textunderscore Ceita\textunderscore .
\section{Ceitan}
\begin{itemize}
\item {Grp. gram.:f.}
\end{itemize}
Uva, o mesmo que \textunderscore aceitan\textunderscore .
\section{Ceitão}
\begin{itemize}
\item {Grp. gram.:m.}
\end{itemize}
O mesmo que \textunderscore ceitan\textunderscore .
\section{Ceitil}
\begin{itemize}
\item {Grp. gram.:m.}
\end{itemize}
\begin{itemize}
\item {Utilização:Fig.}
\end{itemize}
Antiga moéda portuguesa, de pouco valor.
Insignificância.
(Ár. \textunderscore cebtil\textunderscore , de \textunderscore Cebta\textunderscore , Ceuta)
\section{Ceiva}
\begin{itemize}
\item {Grp. gram.:f.}
\end{itemize}
\begin{itemize}
\item {Utilização:T. do Ribatejo}
\end{itemize}
Acto de ceivar.
\section{Ceivar}
\begin{itemize}
\item {Grp. gram.:v. i.}
\end{itemize}
\begin{itemize}
\item {Utilização:Prov.}
\end{itemize}
\begin{itemize}
\item {Utilização:minh.}
\end{itemize}
Desprender do jugo (os bois).
Abrir, destapar: \textunderscore ceivar a água da poça\textunderscore .
\section{Ceive}
\begin{itemize}
\item {Grp. gram.:m.}
\end{itemize}
\begin{itemize}
\item {Utilização:Prov.}
\end{itemize}
\begin{itemize}
\item {Utilização:minh.}
\end{itemize}
Acto de ceivar (águas).
\section{Cela}
\begin{itemize}
\item {Grp. gram.:f.}
\end{itemize}
\begin{itemize}
\item {Proveniência:(Lat. \textunderscore cella\textunderscore )}
\end{itemize}
Pequeno quarto de dormir; alcova.
Aposento de frades ou de freiras, nos conventos.
Aposento de condemnado, em cadeias penitenciárias.
Cada uma das cavidades dos favos.
\section{Celada}
\begin{itemize}
\item {Grp. gram.:f.}
\end{itemize}
\begin{itemize}
\item {Proveniência:(Do lat. \textunderscore caelatus\textunderscore )}
\end{itemize}
Antiga armadura de ferro, para a cabeça.
O mesmo que \textunderscore barbuda\textunderscore , moéda do tempo de D. Fernando.
\section{Celadolo}
\begin{itemize}
\item {Grp. gram.:m.}
\end{itemize}
Planta do Malabar.
\section{Celadonita}
\begin{itemize}
\item {Grp. gram.:f.}
\end{itemize}
\begin{itemize}
\item {Proveniência:(De \textunderscore Celadon\textunderscore , n. p.)}
\end{itemize}
Terra verde aluminosa.
\section{Celafobia}
\begin{itemize}
\item {Grp. gram.:f.}
\end{itemize}
\begin{itemize}
\item {Utilização:Med.}
\end{itemize}
O mesmo que \textunderscore hyperacusia\textunderscore , considerada como phobia mental. Cf. Sousa Martins, \textunderscore Nosographia\textunderscore .
\section{Celagem}
\begin{itemize}
\item {Grp. gram.:f.}
\end{itemize}
\begin{itemize}
\item {Proveniência:(Do lat. \textunderscore caelum\textunderscore )}
\end{itemize}
Aspecto do céu.
\section{Celaio}
\begin{itemize}
\item {Grp. gram.:m.}
\end{itemize}
\begin{itemize}
\item {Utilização:Ant.}
\end{itemize}
Imposto sôbre o pão, que se cozia na vila de Alenquer.
(Cp. lat. \textunderscore cellarium\textunderscore )
\section{Celale}
\begin{itemize}
\item {Grp. gram.:m.}
\end{itemize}
Insecto de Angola.
\section{Celamim}
\begin{itemize}
\item {Grp. gram.:m.}
\end{itemize}
Décima sexta parte de um alqueire.
(Cp. cast. \textunderscore celemin\textunderscore )
\section{Celaphobia}
\begin{itemize}
\item {Grp. gram.:f.}
\end{itemize}
\begin{itemize}
\item {Utilização:Med.}
\end{itemize}
O mesmo que \textunderscore hyperacusia\textunderscore , considerada como phobia mental. Cf. Sousa Martins, \textunderscore Nosographia\textunderscore .
\section{Celareiro}
\begin{itemize}
\item {Grp. gram.:m.}
\end{itemize}
\begin{itemize}
\item {Utilização:Ant.}
\end{itemize}
\begin{itemize}
\item {Proveniência:(Lat. \textunderscore cellararius\textunderscore )}
\end{itemize}
O mesmo que \textunderscore celeireiro\textunderscore .
\section{Celária}
\begin{itemize}
\item {Grp. gram.:f.}
\end{itemize}
\begin{itemize}
\item {Proveniência:(De \textunderscore cella\textunderscore )}
\end{itemize}
Gênero de polypeiros.
\section{Celáricos}
\begin{itemize}
\item {Grp. gram.:m. pl.}
\end{itemize}
Familia de pólypos, que tem por typo a \textunderscore celária\textunderscore .
\section{Celastríneas}
\begin{itemize}
\item {Grp. gram.:f. pl.}
\end{itemize}
\begin{itemize}
\item {Proveniência:(De \textunderscore celastro\textunderscore )}
\end{itemize}
Fam. de plantas arbustivas, frequentes nas regiões tropicaes.
\section{Celastro}
\begin{itemize}
\item {Grp. gram.:m.}
\end{itemize}
\begin{itemize}
\item {Proveniência:(Gr. \textunderscore kelastros\textunderscore )}
\end{itemize}
Gênero de plantas, que serve de typo ás celastrineas.
\section{Celatura}
\begin{itemize}
\item {Grp. gram.:f.}
\end{itemize}
\begin{itemize}
\item {Utilização:Des.}
\end{itemize}
\begin{itemize}
\item {Proveniência:(Lat. \textunderscore cælatura\textunderscore )}
\end{itemize}
Arte de gravador.
\section{Celé-alé}
\begin{itemize}
\item {Grp. gram.:m.}
\end{itemize}
Arbusto santhomense, (\textunderscore leea tinctoria\textunderscore , Lindl.), de que se extrai uma tinta roxa.
\section{Celebérrimo}
\begin{itemize}
\item {Grp. gram.:adj.}
\end{itemize}
Muito célebre.
(Sup. de \textunderscore célebre\textunderscore )
\section{Celebração}
\begin{itemize}
\item {Grp. gram.:f.}
\end{itemize}
Acto de celebrar.
\section{Celebrador}
\begin{itemize}
\item {Grp. gram.:m.}
\end{itemize}
\begin{itemize}
\item {Proveniência:(Lat. \textunderscore celebrator\textunderscore )}
\end{itemize}
Aquelle que celebra.
\section{Celebrante}
\begin{itemize}
\item {Grp. gram.:adj.}
\end{itemize}
\begin{itemize}
\item {Grp. gram.:M.}
\end{itemize}
\begin{itemize}
\item {Proveniência:(Lat. \textunderscore celebrans\textunderscore )}
\end{itemize}
Que celebra.
O padre que celebra Missa.
\section{Celebrar}
\begin{itemize}
\item {Grp. gram.:v. t.}
\end{itemize}
\begin{itemize}
\item {Grp. gram.:V. i.}
\end{itemize}
\begin{itemize}
\item {Proveniência:(Lat. \textunderscore celebrare\textunderscore )}
\end{itemize}
Realizar com solennidade.
Commemorar: \textunderscore celebrar uma victória\textunderscore .
Publicar com louvor: \textunderscore celebrar os méritos de alguém\textunderscore .
Exaltar com ironia.
Dizer Missa.
\section{Celebrável}
\begin{itemize}
\item {Grp. gram.:adj.}
\end{itemize}
\begin{itemize}
\item {Proveniência:(Lat. \textunderscore celebrabilis\textunderscore )}
\end{itemize}
Digno de sêr celebrado.
\section{Célebre}
\begin{itemize}
\item {Grp. gram.:adj.}
\end{itemize}
\begin{itemize}
\item {Utilização:Fam.}
\end{itemize}
\begin{itemize}
\item {Proveniência:(Lat. \textunderscore celeber\textunderscore )}
\end{itemize}
Que tem grande nomeada; muito notório; notável: \textunderscore escritor célebre\textunderscore .
Extravagante, esquisito.
\section{Celebreira}
\begin{itemize}
\item {Grp. gram.:f.}
\end{itemize}
\begin{itemize}
\item {Utilização:Fam.}
\end{itemize}
\begin{itemize}
\item {Proveniência:(De \textunderscore célebre\textunderscore ; ou por \textunderscore cerebreira\textunderscore , de \textunderscore cérebro\textunderscore )}
\end{itemize}
Extravagância; manía; esquisitice.
\section{Celebremente}
\begin{itemize}
\item {fónica:cé}
\end{itemize}
\begin{itemize}
\item {Grp. gram.:adv.}
\end{itemize}
De modo célebre.
Com celebridade.
\section{Celebridade}
\begin{itemize}
\item {Grp. gram.:f.}
\end{itemize}
\begin{itemize}
\item {Proveniência:(Lat. \textunderscore celebritas\textunderscore )}
\end{itemize}
Qualidade do que é célebre.
Pessôa célebre.
Coisa célebre.
\section{Celebrização}
\begin{itemize}
\item {Grp. gram.:f.}
\end{itemize}
Acto ou effeito de celebrizar.
\section{Celebrizar}
\begin{itemize}
\item {Grp. gram.:v. t.}
\end{itemize}
Tornar célebre.
\section{Celebrório}
\begin{itemize}
\item {Grp. gram.:adj.}
\end{itemize}
\begin{itemize}
\item {Utilização:Fam.}
\end{itemize}
\begin{itemize}
\item {Proveniência:(De \textunderscore célebre\textunderscore )}
\end{itemize}
Esquisito, excêntrico.
\section{Celeireiro}
\begin{itemize}
\item {Grp. gram.:m.}
\end{itemize}
Guarda ou administrador de celeiro.
\section{Celeiro}
\begin{itemize}
\item {Grp. gram.:m.}
\end{itemize}
\begin{itemize}
\item {Proveniência:(Lat. \textunderscore cellarium\textunderscore )}
\end{itemize}
Casa, em que se ajuntam e guardam cereaes.
Depósito de provisões.
\section{Celena}
\begin{itemize}
\item {Grp. gram.:f.}
\end{itemize}
Insecto lepidóptero nocturno.
\section{Celenterado}
\begin{itemize}
\item {Grp. gram.:m.}
\end{itemize}
\begin{itemize}
\item {Grp. gram.:Pl.}
\end{itemize}
\begin{itemize}
\item {Proveniência:(Do gr. \textunderscore koilenteron\textunderscore , cavidade)}
\end{itemize}
Animal de symetria radíada e consistência gelatinosa, com uma cavidade commum para a digestão e circulação, como a hydra de água doce: \textunderscore a vinagreira e a caravela são celenterados\textunderscore .
Animaes marinhos de ordem inferior, que foram conhecidos por \textunderscore acalephos\textunderscore .
\section{Celêntero}
\begin{itemize}
\item {Grp. gram.:m.}
\end{itemize}
\begin{itemize}
\item {Proveniência:(Gr. \textunderscore koilenteron\textunderscore )}
\end{itemize}
Cavidade da gástrula.
\section{Celeradamente}
\begin{itemize}
\item {Grp. gram.:adv.}
\end{itemize}
O mesmo que \textunderscore acceleradamente\textunderscore . Cf. \textunderscore Inéd. da Hist. Port.\textunderscore , I, 362.
\section{Célere}
\begin{itemize}
\item {Grp. gram.:adj.}
\end{itemize}
\begin{itemize}
\item {Proveniência:(Lat. \textunderscore celer\textunderscore )}
\end{itemize}
Veloz.
\section{Celeri}
\begin{itemize}
\item {Grp. gram.:m.}
\end{itemize}
Aipo hortense. Cf. \textunderscore Pharmac. Port.\textunderscore 
\section{Celeridade}
\begin{itemize}
\item {Grp. gram.:f.}
\end{itemize}
\begin{itemize}
\item {Proveniência:(Lat. \textunderscore celeritas\textunderscore )}
\end{itemize}
Qualidade do que é célere.
Rapidez.
\section{Celerifero}
\begin{itemize}
\item {Grp. gram.:m.}
\end{itemize}
\begin{itemize}
\item {Proveniência:(Do lat. \textunderscore celer\textunderscore  + \textunderscore ferre\textunderscore )}
\end{itemize}
Antigo apparelho de locomoção, inventado em França em 1790, e precursor do velocípede.
\section{Celerígrado}
\begin{itemize}
\item {Grp. gram.:adj.}
\end{itemize}
\begin{itemize}
\item {Proveniência:(Do lat. \textunderscore celer\textunderscore  + \textunderscore gradi\textunderscore )}
\end{itemize}
Que anda rapidamente.
*\textunderscore  M. pl.\textunderscore 
Classe de animaes roedores.
\section{Celerímetro}
\begin{itemize}
\item {Grp. gram.:m.}
\end{itemize}
\begin{itemize}
\item {Proveniência:(Do lat. \textunderscore celer\textunderscore  + gr. \textunderscore metron\textunderscore )}
\end{itemize}
Instrumento, que mede o caminho percorrido por uma carruagem; taxímetro.
\section{Celerípede}
\begin{itemize}
\item {Grp. gram.:adj.}
\end{itemize}
\begin{itemize}
\item {Proveniência:(Do lat. \textunderscore celer\textunderscore  + \textunderscore pes\textunderscore )}
\end{itemize}
Que caminha com rapidez.
\section{Celérrimo}
\begin{itemize}
\item {Grp. gram.:adj.}
\end{itemize}
(Sup. de \textunderscore célere\textunderscore )
\section{Celeste}
\begin{itemize}
\item {Grp. gram.:adj.}
\end{itemize}
\begin{itemize}
\item {Utilização:Fig.}
\end{itemize}
\begin{itemize}
\item {Proveniência:(Do lat. \textunderscore caelestis\textunderscore )}
\end{itemize}
Relativo ao céu: \textunderscore a abobada celeste\textunderscore .
Que se avista no céu.
Que está no céu.
Concorrente á divindade; sobrenatural.
Perfeito, delicioso: \textunderscore formosura celeste\textunderscore .
Epitheto do Império chinês: \textunderscore o celeste Império\textunderscore .
\section{Celestial}
\begin{itemize}
\item {Grp. gram.:adj.}
\end{itemize}
O mesmo que \textunderscore celeste\textunderscore .
\section{Celestialmente}
\begin{itemize}
\item {Grp. gram.:adv.}
\end{itemize}
De modo celestial.
\section{Celestina}
\begin{itemize}
\item {Grp. gram.:f.}
\end{itemize}
\begin{itemize}
\item {Grp. gram.:M.}
\end{itemize}
\begin{itemize}
\item {Proveniência:(De \textunderscore celeste\textunderscore )}
\end{itemize}
Planta corymbífera, de bellas flôres azues.
Variedade azul de sulfato de estrôncio.
\section{Celestino}
\begin{itemize}
\item {Grp. gram.:adj.}
\end{itemize}
\begin{itemize}
\item {Proveniência:(De \textunderscore celeste\textunderscore )}
\end{itemize}
Que tem a côr do céu.
\section{Celestino}
\begin{itemize}
\item {Grp. gram.:m.}
\end{itemize}
\begin{itemize}
\item {Utilização:Mús.}
\end{itemize}
Espécie de cravo, inventado no século XVIII, o qual prolongava os sons, por meio de um cordão de seda, movido por uma roda.
\section{Celestrina}
\begin{itemize}
\item {Grp. gram.:f.}
\end{itemize}
\begin{itemize}
\item {Utilização:Prov.}
\end{itemize}
Mulher muito remexida. (Colhido em Turquel)
\section{Celeto}
\begin{itemize}
\item {fónica:lé}
\end{itemize}
\begin{itemize}
\item {Grp. gram.:m.}
\end{itemize}
Insecto coleóptero do Brasil.
\section{Celeuma}
\begin{itemize}
\item {Grp. gram.:f.}
\end{itemize}
\begin{itemize}
\item {Proveniência:(Do lat. \textunderscore celeusma\textunderscore )}
\end{itemize}
Vozearia de homens que trabalham.
Canto ou vozes de barqueiros.
Barulho, algazarra.
Alarma.
\section{Celeumear}
\begin{itemize}
\item {Grp. gram.:v. i.}
\end{itemize}
Fazer celeuma.
\section{Celga}
\begin{itemize}
\item {Grp. gram.:f.}
\end{itemize}
\begin{itemize}
\item {Proveniência:(Do ár. \textunderscore celca\textunderscore )}
\end{itemize}
Planta hortense, (\textunderscore beta vulgaris\textunderscore , Lin.).
\section{Celha}
\begin{itemize}
\item {fónica:cê}
\end{itemize}
\begin{itemize}
\item {Grp. gram.:f.}
\end{itemize}
(V.selha)
\section{Celha}
\begin{itemize}
\item {fónica:cê}
\end{itemize}
\begin{itemize}
\item {Grp. gram.:f.}
\end{itemize}
\begin{itemize}
\item {Proveniência:(Lat. \textunderscore cilia\textunderscore , pl. de \textunderscore cilium\textunderscore )}
\end{itemize}
Pêlos, que garnecem as pálpebras; pestanas; cílios.
Pêlos ou sedas, que se criam no fio marginal das folhas de certas plantas.
\section{Celhas}
\begin{itemize}
\item {fónica:cê}
\end{itemize}
\begin{itemize}
\item {Grp. gram.:f. pl.}
\end{itemize}
\begin{itemize}
\item {Proveniência:(Lat. cilia, pl. de cilium)}
\end{itemize}
Pêlos, que garnecem as pálpebras; pestanas; cílios.
Pêlos ou sedas, que se criam no fio marginal das folhas de certas plantas.
\section{Celheado}
\begin{itemize}
\item {Grp. gram.:adj.}
\end{itemize}
\begin{itemize}
\item {Proveniência:(De \textunderscore celha\textunderscore ^2)}
\end{itemize}
Que tem celhas.
Diz-se principalmente do cavallo que tem sobrancelhas brancas.
\section{Celíaco}
\begin{itemize}
\item {Grp. gram.:adj.}
\end{itemize}
\begin{itemize}
\item {Proveniência:(Lat. \textunderscore coeliacus\textunderscore )}
\end{itemize}
Relativo aos intestinos.
\section{Celibatário}
\begin{itemize}
\item {Grp. gram.:m.  e  adj.}
\end{itemize}
O que vive em celibato.
\section{Celibatarismo}
\begin{itemize}
\item {Grp. gram.:m.}
\end{itemize}
\begin{itemize}
\item {Utilização:Neol.}
\end{itemize}
Systema ou vida de celibatário.
\section{Celibato}
\begin{itemize}
\item {Grp. gram.:m.}
\end{itemize}
\begin{itemize}
\item {Proveniência:(Lat. \textunderscore caelibatus\textunderscore )}
\end{itemize}
Estado da pessôa que é solteira e que não tem tenção de casar, ou não póde casar: \textunderscore o celibato dos padres\textunderscore .
\section{Célibe}
\begin{itemize}
\item {Grp. gram.:m.}
\end{itemize}
\begin{itemize}
\item {Utilização:Des.}
\end{itemize}
\begin{itemize}
\item {Proveniência:(Lat. \textunderscore caelebs\textunderscore )}
\end{itemize}
Celibatário.
\section{Célico}
\begin{itemize}
\item {Grp. gram.:adj.}
\end{itemize}
\begin{itemize}
\item {Proveniência:(Lat. \textunderscore caelicus\textunderscore )}
\end{itemize}
O mesmo que \textunderscore celeste\textunderscore .
\section{Celícola}
\begin{itemize}
\item {Grp. gram.:m.}
\end{itemize}
\begin{itemize}
\item {Proveniência:(Lat. \textunderscore coelicola\textunderscore )}
\end{itemize}
Aquelle que habita no céu.
\section{Celideia}
\begin{itemize}
\item {Grp. gram.:f.}
\end{itemize}
\begin{itemize}
\item {Proveniência:(Lat. hyp. \textunderscore coeli-dea\textunderscore , deusa do céu)}
\end{itemize}
Anémona côr de rosa.
\section{Celidografia}
\begin{itemize}
\item {Grp. gram.:f.}
\end{itemize}
\begin{itemize}
\item {Proveniência:(Do gr. \textunderscore kelis\textunderscore  + \textunderscore graphein\textunderscore )}
\end{itemize}
Descripção das manchas, que se observam em alguns astros.
\section{Celidógrafo}
\begin{itemize}
\item {Grp. gram.:m.}
\end{itemize}
Aquelle que se occupa de \textunderscore celidografia\textunderscore .
\section{Celidographia}
\begin{itemize}
\item {Grp. gram.:f.}
\end{itemize}
\begin{itemize}
\item {Proveniência:(Do gr. \textunderscore kelis\textunderscore  + \textunderscore graphein\textunderscore )}
\end{itemize}
Descripção das manchas, que se observam em alguns astros.
\section{Celidógrapho}
\begin{itemize}
\item {Grp. gram.:m.}
\end{itemize}
Aquelle que se occupa de \textunderscore celidographia\textunderscore .
\section{Celidónia}
\begin{itemize}
\item {Grp. gram.:f.}
\end{itemize}
\begin{itemize}
\item {Proveniência:(Gr. \textunderscore khelidonion\textunderscore )}
\end{itemize}
Planta papaverácea, vulgarmente chamada \textunderscore erva-andorinha\textunderscore .
Planta ranunculácea.
Pedra preciosa.
\section{Celífero}
\begin{itemize}
\item {Grp. gram.:adj.}
\end{itemize}
\begin{itemize}
\item {Proveniência:(Lat. \textunderscore caelifer\textunderscore )}
\end{itemize}
O mesmo que \textunderscore celígero\textunderscore .
\section{Celificar}
\begin{itemize}
\item {Grp. gram.:v. t.}
\end{itemize}
\begin{itemize}
\item {Utilização:Ant.}
\end{itemize}
\begin{itemize}
\item {Proveniência:(Do lat. \textunderscore caelum\textunderscore  + \textunderscore facere\textunderscore )}
\end{itemize}
Collocar no céu, no firmamento, entre os astros.
\section{Celífluo}
\begin{itemize}
\item {Grp. gram.:adj.}
\end{itemize}
\begin{itemize}
\item {Proveniência:(Lat. \textunderscore coelifluus\textunderscore )}
\end{itemize}
Que dimana do céu.
\section{Celígena}
\begin{itemize}
\item {Grp. gram.:m.  e  adj.}
\end{itemize}
\begin{itemize}
\item {Proveniência:(Lat. \textunderscore coeligenus\textunderscore )}
\end{itemize}
Que procede do céu.
\section{Celígero}
\begin{itemize}
\item {Grp. gram.:adj.}
\end{itemize}
\begin{itemize}
\item {Proveniência:(Lat. \textunderscore coeliger\textunderscore )}
\end{itemize}
Que sustenta o céu sôbre si, (epítheto poético de Hércules e de Atlas).
\section{Celina}
\begin{itemize}
\item {Grp. gram.:f.}
\end{itemize}
Insecto lepidóptero nocturno.
Insecto coleóptero da América do Sul.
\section{Celipotente}
\begin{itemize}
\item {Grp. gram.:adj.}
\end{itemize}
\begin{itemize}
\item {Proveniência:(Lat. \textunderscore coelipotens\textunderscore )}
\end{itemize}
Poderoso no céu.
\section{Cella}
\begin{itemize}
\item {Grp. gram.:f.}
\end{itemize}
\begin{itemize}
\item {Proveniência:(Lat. \textunderscore cella\textunderscore )}
\end{itemize}
Pequeno quarto de dormir; alcova.
Aposento de frades ou de freiras, nos conventos.
Aposento de condemnado, em cadeias penitenciárias.
Cada uma das cavidades dos favos.
\section{Cellaio}
\begin{itemize}
\item {Grp. gram.:m.}
\end{itemize}
\begin{itemize}
\item {Utilização:Ant.}
\end{itemize}
Imposto sôbre o pão, que se cozia na villa de Alenquer.
(Cp. lat. \textunderscore cellarium\textunderscore )
\section{Cellareiro}
\begin{itemize}
\item {Grp. gram.:m.}
\end{itemize}
\begin{itemize}
\item {Utilização:Ant.}
\end{itemize}
\begin{itemize}
\item {Proveniência:(Lat. \textunderscore cellararius\textunderscore )}
\end{itemize}
O mesmo que \textunderscore celleireiro\textunderscore .
\section{Cellária}
\begin{itemize}
\item {Grp. gram.:f.}
\end{itemize}
\begin{itemize}
\item {Proveniência:(De \textunderscore cella\textunderscore )}
\end{itemize}
Gênero de polypeiros.
\section{Celláricos}
\begin{itemize}
\item {Grp. gram.:m. pl.}
\end{itemize}
Familia de pólypos, que tem por typo a \textunderscore cellária\textunderscore .
\section{Celleireiro}
\begin{itemize}
\item {Grp. gram.:m.}
\end{itemize}
Guarda ou administrador de celleiro.
\section{Celleiro}
\begin{itemize}
\item {Grp. gram.:m.}
\end{itemize}
\begin{itemize}
\item {Proveniência:(Lat. \textunderscore cellarium\textunderscore )}
\end{itemize}
Casa, em que se ajuntam e guardam cereaes.
Depósito de provisões.
\section{Cellépora}
\begin{itemize}
\item {Grp. gram.:f.}
\end{itemize}
O mesmo que \textunderscore celléporo\textunderscore .
\section{Celleporáceos}
\begin{itemize}
\item {Grp. gram.:m. pl.}
\end{itemize}
\begin{itemize}
\item {Proveniência:(De \textunderscore celléporo\textunderscore )}
\end{itemize}
Ordem de pólypos, que têm por typo o celléporo, e que constituem coraes cellulíferos.
\section{Cellepóreos}
\begin{itemize}
\item {Grp. gram.:m. pl.}
\end{itemize}
O mesmo que \textunderscore celleporáceos\textunderscore .
\section{Celleporinos}
\begin{itemize}
\item {Grp. gram.:m. pl.}
\end{itemize}
O mesmo que \textunderscore celleporáceos\textunderscore .
\section{Celléporo}
\begin{itemize}
\item {Grp. gram.:m.}
\end{itemize}
\begin{itemize}
\item {Proveniência:(Do lat. \textunderscore cella\textunderscore  + \textunderscore porus\textunderscore )}
\end{itemize}
Gênero de pólypos bryozoários.
\section{Céllula}
\begin{itemize}
\item {Grp. gram.:f.}
\end{itemize}
\begin{itemize}
\item {Utilização:Physiol.}
\end{itemize}
\begin{itemize}
\item {Proveniência:(Lat. \textunderscore cellula\textunderscore )}
\end{itemize}
Pequena cella.
Casulo de semente.
Cada um dos elementos plásticos dos tecidos orgânicos.
Pequeno interstício no tecido esponjoso dos ossos, etc.
\section{Cellular}
\begin{itemize}
\item {Grp. gram.:adj.}
\end{itemize}
Que tem céllulas; que é formado de céllulas: \textunderscore carro cellular\textunderscore .
Relativo ás cadeias penitenciárias: \textunderscore prisão cellular\textunderscore .
\section{Cellulífero}
\begin{itemize}
\item {Grp. gram.:adj.}
\end{itemize}
\begin{itemize}
\item {Proveniência:(Do lat. \textunderscore cellula\textunderscore  + \textunderscore ferre\textunderscore )}
\end{itemize}
Que tem céllulas.
\section{Celluliforme}
\begin{itemize}
\item {Grp. gram.:adj.}
\end{itemize}
\begin{itemize}
\item {Proveniência:(Do lat. \textunderscore cellula\textunderscore  + \textunderscore forma\textunderscore )}
\end{itemize}
Que tem fórma de céllula.
\section{Cellulítelo}
\begin{itemize}
\item {Grp. gram.:adj.}
\end{itemize}
\begin{itemize}
\item {Proveniência:(Do lat. \textunderscore cellula\textunderscore  + \textunderscore tela\textunderscore )}
\end{itemize}
Que faz teias cellulosas.
\section{Celluloide}
\begin{itemize}
\item {Grp. gram.:f.}
\end{itemize}
Substância sólida, transparente, elástica, que, depois de aquecida, se torna malleável, e de que se fabrícam vários objectos, como collarinhos, pentes, objectos de fantasia, etc.
\section{Cellulose}
\begin{itemize}
\item {Grp. gram.:f.}
\end{itemize}
\begin{itemize}
\item {Proveniência:(De \textunderscore céllula\textunderscore )}
\end{itemize}
Princípio dos corpos organizados, que constitue a parte sólida dos vegetaes.
\section{Cellulósico}
\begin{itemize}
\item {Grp. gram.:adj.}
\end{itemize}
Relativo á cellulose. Cf. \textunderscore Techn. Rur.\textunderscore , 120.
\section{Celépora}
\begin{itemize}
\item {Grp. gram.:f.}
\end{itemize}
O mesmo que \textunderscore celéporo\textunderscore .
\section{Celeporáceos}
\begin{itemize}
\item {Grp. gram.:m. pl.}
\end{itemize}
\begin{itemize}
\item {Proveniência:(De \textunderscore celléporo\textunderscore )}
\end{itemize}
Ordem de pólypos, que têm por typo o celéporo, e que constituem coraes celulíferos.
\section{Celepóreos}
\begin{itemize}
\item {Grp. gram.:m. pl.}
\end{itemize}
O mesmo que \textunderscore celeporáceos\textunderscore .
\section{Celeporinos}
\begin{itemize}
\item {Grp. gram.:m. pl.}
\end{itemize}
O mesmo que \textunderscore celeporáceos\textunderscore .
\section{Celéporo}
\begin{itemize}
\item {Grp. gram.:m.}
\end{itemize}
\begin{itemize}
\item {Proveniência:(Do lat. \textunderscore cella\textunderscore  + \textunderscore porus\textunderscore )}
\end{itemize}
Gênero de pólypos bryozoários.
\section{Cellulosidade}
\begin{itemize}
\item {Grp. gram.:f.}
\end{itemize}
Estado do que é celluloso.
\section{Cellulosina}
\begin{itemize}
\item {Grp. gram.:f.}
\end{itemize}
\begin{itemize}
\item {Proveniência:(De \textunderscore cellulose\textunderscore )}
\end{itemize}
Substância incombustível, que tem por base a cellulose e foi inventada em 1892, com applicação a tecidos, imitações de seda, etc.
\section{Celluloso}
\begin{itemize}
\item {Grp. gram.:adj.}
\end{itemize}
Que tem céllulas, que está dividido em céllulas.
\section{Cellulotypia}
\begin{itemize}
\item {Grp. gram.:f.}
\end{itemize}
Processo novo, (1902), que, na gravura a talho doce, substitue o ácido pela água-forte e substitue a placa metállica ou a pedra por uma lâmina ou placa de celluloide transparente.
\section{Celópodo}
\begin{itemize}
\item {Grp. gram.:m.}
\end{itemize}
\begin{itemize}
\item {Proveniência:(Do gr. \textunderscore kelos\textunderscore  + \textunderscore pous\textunderscore , \textunderscore podos\textunderscore )}
\end{itemize}
Insecto díptero.
\section{Celósia}
\begin{itemize}
\item {Grp. gram.:f.}
\end{itemize}
Gênero de plantas amarantháceas.
\section{Celóstato}
\begin{itemize}
\item {Grp. gram.:m.}
\end{itemize}
\begin{itemize}
\item {Proveniência:(Do lat. \textunderscore caelum\textunderscore  + \textunderscore stare\textunderscore )}
\end{itemize}
Instrumento, inventado recentemente, em 1895, pelo astrónomo Lippmann, e que immobiliza a imagem de todo o céu.
\section{Celotomia}
\begin{itemize}
\item {Grp. gram.:f.}
\end{itemize}
\begin{itemize}
\item {Proveniência:(De \textunderscore celótomo\textunderscore )}
\end{itemize}
Operação cirúrgica, que consiste em desapertar a hérnia.
\section{Celótomo}
\begin{itemize}
\item {Grp. gram.:m.}
\end{itemize}
\begin{itemize}
\item {Proveniência:(Do gr. \textunderscore kele\textunderscore  + \textunderscore temnein\textunderscore )}
\end{itemize}
Instrumento, para a celotomia.
\section{Célsia}
\begin{itemize}
\item {Grp. gram.:f.}
\end{itemize}
\begin{itemize}
\item {Proveniência:(De \textunderscore Célsio\textunderscore , n. p.)}
\end{itemize}
Gênero de plantas escrofularíneas.
\section{Celsitude}
\begin{itemize}
\item {Grp. gram.:f.}
\end{itemize}
\begin{itemize}
\item {Proveniência:(Lat. \textunderscore celsitudo\textunderscore )}
\end{itemize}
Qualidade do que é celso.
\section{Celso}
\begin{itemize}
\item {Grp. gram.:adj.}
\end{itemize}
\begin{itemize}
\item {Proveniência:(Lat. \textunderscore celsus\textunderscore )}
\end{itemize}
Alto; sublime.
\section{Celta}
\begin{itemize}
\item {Grp. gram.:m.}
\end{itemize}
\begin{itemize}
\item {Grp. gram.:Adj.}
\end{itemize}
Idiomas dos Celtas.
Indivíduo, pertencente á raça céltica.
Relativo aos Celtas.
\section{Celtas}
\begin{itemize}
\item {Grp. gram.:m. pl.}
\end{itemize}
\begin{itemize}
\item {Proveniência:(Lat. \textunderscore Celtae\textunderscore )}
\end{itemize}
Povo notável, de origem indo europeia, que constituiu parte da antiga população da Gállia, estendendo-se á Espanha e a outros pontos da Europa.
\section{Celte}
\begin{itemize}
\item {Grp. gram.:f.}
\end{itemize}
\begin{itemize}
\item {Grp. gram.:m.}
\end{itemize}
\begin{itemize}
\item {Proveniência:(Lat. \textunderscore celtis\textunderscore )}
\end{itemize}
Gênero de lódão.
Designação genérica dos instrumentos cortantes de pedra, das eras prehistóricas.
\section{Celtibérico}
\begin{itemize}
\item {Grp. gram.:adj.}
\end{itemize}
Relativo aos Celtiberos.
\section{Celtibéros}
\begin{itemize}
\item {Grp. gram.:m. pl.}
\end{itemize}
\begin{itemize}
\item {Proveniência:(Lat. \textunderscore celtiberi\textunderscore )}
\end{itemize}
Povo da antiga Espanha, constituido pela juncção de Celtas com Iberos.
\section{Celticismo}
\begin{itemize}
\item {Grp. gram.:m.}
\end{itemize}
\begin{itemize}
\item {Proveniência:(De \textunderscore céltico\textunderscore )}
\end{itemize}
Idiotismo das línguas célticas.
Celtomania.
\section{Céltico}
\begin{itemize}
\item {Grp. gram.:adj.}
\end{itemize}
\begin{itemize}
\item {Grp. gram.:M.}
\end{itemize}
\begin{itemize}
\item {Proveniência:(Lat. \textunderscore celticus\textunderscore )}
\end{itemize}
Relativo aos Celtas.
As línguas dos Celtas.
\section{Celtídeas}
\begin{itemize}
\item {Grp. gram.:f. pl.}
\end{itemize}
\begin{itemize}
\item {Proveniência:(De \textunderscore celte\textunderscore )}
\end{itemize}
Fam. de plantas arbóreas ou arbustivas, próprias das regiões quentes.
\section{Celtismo}
\begin{itemize}
\item {Grp. gram.:m.}
\end{itemize}
Systema ou estudos próprios de celtista. Cf. Herculano, \textunderscore Hist. de Port.\textunderscore , I, 23, 30 e 34.
\section{Celtista}
\begin{itemize}
\item {Grp. gram.:m.}
\end{itemize}
Aquelle que se occupa da linguagem e costumes dos Celtas.
\section{Celtólogo}
\begin{itemize}
\item {Grp. gram.:m.}
\end{itemize}
Homem períto na história e língua dos Celtas. Cf. Latino, \textunderscore Elogios\textunderscore , 73.
\section{Celtomania}
\begin{itemize}
\item {Grp. gram.:f.}
\end{itemize}
\begin{itemize}
\item {Proveniência:(De \textunderscore celta\textunderscore  + \textunderscore mania\textunderscore )}
\end{itemize}
Mania de alguns sábios, que viam no celta a origem da maior parte das línguas europeias, nomeadamente da francesa.
\section{Celtomaníaco}
\begin{itemize}
\item {Grp. gram.:m.}
\end{itemize}
Aquelle que tem celtomania.
\section{Celula}
\begin{itemize}
\item {Grp. gram.:f.}
\end{itemize}
\begin{itemize}
\item {Utilização:Physiol.}
\end{itemize}
\begin{itemize}
\item {Proveniência:(Lat. \textunderscore cellula\textunderscore )}
\end{itemize}
Pequena cela.
Casulo de semente.
Cada um dos elementos plásticos dos tecidos orgânicos.
Pequeno interstício no tecido esponjoso dos ossos, etc.
\section{Celular}
\begin{itemize}
\item {Grp. gram.:adj.}
\end{itemize}
Que tem células; que é formado de células: \textunderscore carro celular\textunderscore .
Relativo ás cadeias penitenciárias: \textunderscore prisão celular\textunderscore .
\section{Celulífero}
\begin{itemize}
\item {Grp. gram.:adj.}
\end{itemize}
\begin{itemize}
\item {Proveniência:(Do lat. \textunderscore cellula\textunderscore  + \textunderscore ferre\textunderscore )}
\end{itemize}
Que tem células.
\section{Celuliforme}
\begin{itemize}
\item {Grp. gram.:adj.}
\end{itemize}
\begin{itemize}
\item {Proveniência:(Do lat. \textunderscore cellula\textunderscore  + \textunderscore forma\textunderscore )}
\end{itemize}
Que tem fórma de célula.
\section{Celulítelo}
\begin{itemize}
\item {Grp. gram.:adj.}
\end{itemize}
\begin{itemize}
\item {Proveniência:(Do lat. \textunderscore cellula\textunderscore  + \textunderscore tela\textunderscore )}
\end{itemize}
Que faz teias celulosas.
\section{Celuloide}
\begin{itemize}
\item {Grp. gram.:f.}
\end{itemize}
Substância sólida, transparente, elástica, que, depois de aquecida, se torna maleável, e de que se fabrícam vários objectos, como colarinhos, pentes, objectos de fantasia, etc.
\section{Celulose}
\begin{itemize}
\item {Grp. gram.:f.}
\end{itemize}
\begin{itemize}
\item {Proveniência:(De \textunderscore céllula\textunderscore )}
\end{itemize}
Princípio dos corpos organizados, que constitue a parte sólida dos vegetaes.
\section{Celulósico}
\begin{itemize}
\item {Grp. gram.:adj.}
\end{itemize}
Relativo á celulose. Cf. \textunderscore Techn. Rur.\textunderscore , 120.
\section{Celulosidade}
\begin{itemize}
\item {Grp. gram.:f.}
\end{itemize}
Estado do que é celuloso.
\section{Celulosina}
\begin{itemize}
\item {Grp. gram.:f.}
\end{itemize}
\begin{itemize}
\item {Proveniência:(De \textunderscore cellulose\textunderscore )}
\end{itemize}
Substância incombustível, que tem por base a celulose e foi inventada em 1892, com aplicação a tecidos, imitações de seda, etc.
\section{Celuloso}
\begin{itemize}
\item {Grp. gram.:adj.}
\end{itemize}
Que tem células, que está dividido em células.
\section{Celulotipia}
\begin{itemize}
\item {Grp. gram.:f.}
\end{itemize}
Processo novo, (1902), que, na gravura a talho doce, substitue o ácido pela água-forte e substitue a placa metállica ou a pedra por uma lâmina ou placa de celluloide transparente.
\section{Cem}
\begin{itemize}
\item {Grp. gram.:adj.}
\end{itemize}
\begin{itemize}
\item {Grp. gram.:M.}
\end{itemize}
\begin{itemize}
\item {Proveniência:(Do lat. \textunderscore centum\textunderscore )}
\end{itemize}
Déz vezes déz.
O mesmo que \textunderscore cento\textunderscore .
\section{Cemba}
\begin{itemize}
\item {Grp. gram.:f.}
\end{itemize}
\begin{itemize}
\item {Utilização:Prov.}
\end{itemize}
\begin{itemize}
\item {Utilização:trasm.}
\end{itemize}
O mesmo que \textunderscore barranca\textunderscore  ou montículo de palha trilhada.
\section{Cêmbalo}
\begin{itemize}
\item {Grp. gram.:m.}
\end{itemize}
O mesmo que \textunderscore címbalo\textunderscore .
\section{Cembro}
\begin{itemize}
\item {Grp. gram.:m.}
\end{itemize}
Espécie de pinheiro alpino.
\section{Cem-dobrado}
\begin{itemize}
\item {Grp. gram.:adj.}
\end{itemize}
\begin{itemize}
\item {Proveniência:(De \textunderscore cem-dobrar\textunderscore )}
\end{itemize}
Centuplicado.
\section{Cem-dobrar}
\begin{itemize}
\item {Grp. gram.:v. t.}
\end{itemize}
Centuplicar. Cf. M. Bernárdez, \textunderscore N. Floresta\textunderscore .
\section{Cementação}
\begin{itemize}
\item {Grp. gram.:f.}
\end{itemize}
Acto de cementar.
\section{Cementador}
\begin{itemize}
\item {Grp. gram.:m.}
\end{itemize}
Aquelle que cementa.
\section{Cementar}
\begin{itemize}
\item {Grp. gram.:v. t.}
\end{itemize}
\begin{itemize}
\item {Proveniência:(De \textunderscore cemento\textunderscore )}
\end{itemize}
Modificar as propriedades de (um metal, combinado com outra substância sob a acção do calor)
\section{Cementatório}
\begin{itemize}
\item {Grp. gram.:adj.}
\end{itemize}
\begin{itemize}
\item {Proveniência:(De \textunderscore cementar\textunderscore )}
\end{itemize}
Relativo á cementação.
\section{Cemento}
\begin{itemize}
\item {Grp. gram.:m.}
\end{itemize}
\begin{itemize}
\item {Proveniência:(Lat. \textunderscore caementum\textunderscore )}
\end{itemize}
Substância, com que se rodeia um corpo para o cementar.
Substância, que entra na composição dos dentes de alguns mammíferos.
\section{Cementoso}
\begin{itemize}
\item {Grp. gram.:adj.}
\end{itemize}
Que tem os caracteres do cemento.
\section{Cemiterial}
\begin{itemize}
\item {Grp. gram.:adj.}
\end{itemize}
\begin{itemize}
\item {Utilização:Des.}
\end{itemize}
Relativo a cemitério.
\section{Cemitério}
\begin{itemize}
\item {Grp. gram.:m.}
\end{itemize}
\begin{itemize}
\item {Proveniência:(Lat. eccles. \textunderscore caemiterium\textunderscore )}
\end{itemize}
Terreno descoberto, em que se enterram ou guardam os defuntos.
\section{Cempasso}
\begin{itemize}
\item {Grp. gram.:m.}
\end{itemize}
\begin{itemize}
\item {Utilização:Bras. do Ceará}
\end{itemize}
\begin{itemize}
\item {Proveniência:(De \textunderscore cem\textunderscore  + \textunderscore passo\textunderscore )}
\end{itemize}
Medida de superfície, com cem passos em quadro.
\section{Cena}
\begin{itemize}
\item {Grp. gram.:f.}
\end{itemize}
\begin{itemize}
\item {Utilização:Fig.}
\end{itemize}
\begin{itemize}
\item {Proveniência:(Lat. \textunderscore scena\textunderscore )}
\end{itemize}
\textunderscore f.\textunderscore  (e der.)
O mesmo ou melhor que \textunderscore scena\textunderscore , etc.
Parte do teatro, em que os actores representam os seus papéis.
Palco.
Decoração teatral.
Parte de um acto de uma peça teatral, durante a qual as vistas do palco são as mesmas e os mesmos os actores que representam.
Lugar, onde se realiza algum facto.
Acontecimento dramático ou susceptível de representação teatral.
Acto mais ou menos censurável.
Perspectiva; coisa ou coisas, que se abrangem com a vista; panorama.
Arte dramatica: \textunderscore dedicar-se á cena\textunderscore .
\section{Cenáculo}
\begin{itemize}
\item {Grp. gram.:m.}
\end{itemize}
\begin{itemize}
\item {Utilização:Fig.}
\end{itemize}
\begin{itemize}
\item {Proveniência:(Lat. \textunderscore coenaculum\textunderscore )}
\end{itemize}
Antiga designação da sala em que se comia a ceia ou o jantar.
Refeitório.
Lugar, em que Christo teve a última ceia com os seus discípulos.
Convivência.
Ajuntamento de indivíduos, que professam as mesmas ideias ou miram o mesmo fim.
\section{Cenagal}
\begin{itemize}
\item {Grp. gram.:m.}
\end{itemize}
O mesmo que \textunderscore ceno\textunderscore ^1.
(Cast. \textunderscore cenaga\textunderscore )
\section{Cenagoso}
\begin{itemize}
\item {Grp. gram.:adj.}
\end{itemize}
Immundo; torpe; cenoso. Cf. Latino, \textunderscore Elogios\textunderscore , 42; \textunderscore Camões\textunderscore , 318.
(Cast. \textunderscore cenagoso\textunderscore )
\section{Cenantho}
\begin{itemize}
\item {Grp. gram.:m.}
\end{itemize}
O mesmo que \textunderscore amphanto\textunderscore .
\section{Cenanto}
\begin{itemize}
\item {Grp. gram.:m.}
\end{itemize}
O mesmo que \textunderscore anfanto\textunderscore .
\section{Cenário}
\begin{itemize}
\item {Grp. gram.:adj.}
\end{itemize}
\begin{itemize}
\item {Proveniência:(Do lat. \textunderscore coena\textunderscore )}
\end{itemize}
Relativo a ceia.
\section{Cenarreno}
\begin{itemize}
\item {Grp. gram.:m.}
\end{itemize}
\begin{itemize}
\item {Proveniência:(Do gr. \textunderscore kenos\textunderscore  + \textunderscore arrhen\textunderscore )}
\end{itemize}
Árvore da Tasmânia.
\section{Cenatório}
\begin{itemize}
\item {Grp. gram.:adj.}
\end{itemize}
\begin{itemize}
\item {Proveniência:(Lat. \textunderscore coenatorius\textunderscore )}
\end{itemize}
O mesmo que \textunderscore cenário\textunderscore ^1.
\section{Cencê}
\begin{itemize}
\item {Grp. gram.:m.}
\end{itemize}
Tubérculo medicinal da ilha de San-Thomé.
\section{Cencerra}
\begin{itemize}
\item {Grp. gram.:f.}
\end{itemize}
\begin{itemize}
\item {Utilização:Prov.}
\end{itemize}
\begin{itemize}
\item {Utilização:alent.}
\end{itemize}
Espécie de chocalho.
(Cp. \textunderscore cencerro\textunderscore )
\section{Cencerro}
\begin{itemize}
\item {fónica:cê}
\end{itemize}
\begin{itemize}
\item {Grp. gram.:m.}
\end{itemize}
O mesmo que \textunderscore cencerra\textunderscore .
(Cast. \textunderscore cencerro\textunderscore )
\section{Cenchrame}
\begin{itemize}
\item {Grp. gram.:m.}
\end{itemize}
\begin{itemize}
\item {Proveniência:(Lat. \textunderscore cenchramis\textunderscore )}
\end{itemize}
Ave de arribação.
\section{Cencrame}
\begin{itemize}
\item {Grp. gram.:m.}
\end{itemize}
\begin{itemize}
\item {Proveniência:(Lat. \textunderscore cenchramis\textunderscore )}
\end{itemize}
Ave de arribação.
\section{Cendal}
\begin{itemize}
\item {Grp. gram.:m.}
\end{itemize}
Tecido transparente e fino.
Véu para o rosto ou para todo o corpo.
(Cp. cast. \textunderscore cendal\textunderscore )
\section{Cendrado}
\begin{itemize}
\item {Grp. gram.:adj.}
\end{itemize}
\begin{itemize}
\item {Proveniência:(Do cast. \textunderscore cendra\textunderscore )}
\end{itemize}
Que tem côr de cinza.
E o mesmo que \textunderscore acendrado\textunderscore .
\section{Cendrar}
\begin{itemize}
\item {Grp. gram.:v. t.}
\end{itemize}
O mesmo que \textunderscore acendrar\textunderscore . Cf. Garrett, \textunderscore Helena\textunderscore , 11.
\section{Cendrisco}
\begin{itemize}
\item {Grp. gram.:m.}
\end{itemize}
(V.bicançudo)
\section{Cenefa}
\begin{itemize}
\item {Grp. gram.:f.}
\end{itemize}
O mesmo que \textunderscore sanefa\textunderscore . Cf. \textunderscore Viriato Trág.\textunderscore , XVIII, 23.
(Cast. \textunderscore cenefa\textunderscore )
\section{Cenestesia}
\begin{itemize}
\item {Grp. gram.:f.}
\end{itemize}
\begin{itemize}
\item {Proveniência:(Do gr. \textunderscore koinos\textunderscore  + \textunderscore aisthesis\textunderscore )}
\end{itemize}
Espécie de vago sentimento, que temos em nosso sêr, independentemente da indicação dos sentidos.
\section{Cenesthesia}
\begin{itemize}
\item {Grp. gram.:f.}
\end{itemize}
\begin{itemize}
\item {Proveniência:(Do gr. \textunderscore koinos\textunderscore  + \textunderscore aisthesis\textunderscore )}
\end{itemize}
Espécie de vago sentimento, que temos em nosso sêr, independentemente da indicação dos sentidos.
\section{Cengo}
\begin{itemize}
\item {Grp. gram.:m.}
\end{itemize}
Árvore do Congo.
\section{Cenhir}
\begin{itemize}
\item {Grp. gram.:v.}
\end{itemize}
\begin{itemize}
\item {Utilização:t. Pesc.}
\end{itemize}
Deitar do mesmo local (muitos apparelhos de xávega).
\section{Cenho}
\begin{itemize}
\item {Grp. gram.:m.}
\end{itemize}
Rosto carrancudo; aspecto severo.
Doença, entre o pêlo e o casco da bêsta.
(Cast. \textunderscore ceño\textunderscore )
\section{Cenhoso}
\begin{itemize}
\item {Grp. gram.:adj.}
\end{itemize}
Que tem cenho.
\section{Cênia}
\begin{itemize}
\item {Grp. gram.:f.}
\end{itemize}
Planta do Cabo da Bôa-Esperança, da fam. das compostas.
\section{Cenismo}
\begin{itemize}
\item {Grp. gram.:m.}
\end{itemize}
\begin{itemize}
\item {Proveniência:(Gr. \textunderscore koinismos\textunderscore )}
\end{itemize}
Emprêgo de vocábulos de várias línguas, na mesma obra ou discurso.
\section{Ceno}
\begin{itemize}
\item {Grp. gram.:m.}
\end{itemize}
\begin{itemize}
\item {Utilização:Des.}
\end{itemize}
\begin{itemize}
\item {Proveniência:(Lat. \textunderscore coenum\textunderscore )}
\end{itemize}
Lodaçal, atoleiro.
\section{Ceno}
\begin{itemize}
\item {Grp. gram.:m.}
\end{itemize}
\begin{itemize}
\item {Utilização:Prov.}
\end{itemize}
\begin{itemize}
\item {Utilização:trasm.}
\end{itemize}
O mesmo que \textunderscore cenho\textunderscore  e \textunderscore sobrecenho\textunderscore .
\section{Cenobial}
\begin{itemize}
\item {Grp. gram.:adj.}
\end{itemize}
O mesmo que \textunderscore cenobítico\textunderscore .
\section{Cenobialmente}
\begin{itemize}
\item {Grp. gram.:adv.}
\end{itemize}
(V.cenobiticamente)
\section{Cenobiarca}
\begin{itemize}
\item {Grp. gram.:m.}
\end{itemize}
\begin{itemize}
\item {Proveniência:(Do gr. \textunderscore koinobion\textunderscore  + \textunderscore arkhein\textunderscore )}
\end{itemize}
Prelado de um convento de cenobitas.
\section{Cenobiarcha}
\begin{itemize}
\item {fónica:ca}
\end{itemize}
\begin{itemize}
\item {Grp. gram.:m.}
\end{itemize}
\begin{itemize}
\item {Proveniência:(Do gr. \textunderscore koinobion\textunderscore  + \textunderscore arkhein\textunderscore )}
\end{itemize}
Prelado de um convento de cenobitas.
\section{Cenóbio}
\begin{itemize}
\item {Grp. gram.:m.}
\end{itemize}
\begin{itemize}
\item {Proveniência:(Lat. \textunderscore coenobium\textunderscore )}
\end{itemize}
Habitação de cenobitas.
\section{Cenobismo}
\begin{itemize}
\item {Grp. gram.:m.}
\end{itemize}
\begin{itemize}
\item {Proveniência:(De \textunderscore cenóbio\textunderscore )}
\end{itemize}
Vida de cenobita.
\section{Cenobita}
\begin{itemize}
\item {Grp. gram.:m.  e  f.}
\end{itemize}
\begin{itemize}
\item {Proveniência:(De \textunderscore cenóbio\textunderscore )}
\end{itemize}
Monge ou monja, que vivia em communidade.
\section{Cenobiticamente}
\begin{itemize}
\item {Grp. gram.:adv.}
\end{itemize}
\begin{itemize}
\item {Proveniência:(De \textunderscore cenobítico\textunderscore )}
\end{itemize}
Á maneira dos cenobitas.
\section{Cenobítico}
\begin{itemize}
\item {Grp. gram.:adj.}
\end{itemize}
Relativo a cenobitas.
\section{Cenobitismo}
\begin{itemize}
\item {Grp. gram.:m.}
\end{itemize}
Estado de quem é cenobita.
Modo de vida, praticado por cenobitas.
\section{Cenococo}
\begin{itemize}
\item {Grp. gram.:m.}
\end{itemize}
\begin{itemize}
\item {Proveniência:(Do gr. \textunderscore kenos\textunderscore  + \textunderscore kokkos\textunderscore )}
\end{itemize}
Cogumelo microscópico.
\section{Cenogastro}
\begin{itemize}
\item {Grp. gram.:m.}
\end{itemize}
\begin{itemize}
\item {Proveniência:(Do gr. \textunderscore kenos\textunderscore  + \textunderscore gaster\textunderscore )}
\end{itemize}
Insecto díptero.
\section{Cenogênese}
\begin{itemize}
\item {Grp. gram.:f.}
\end{itemize}
Producção do vácuo. Cf. Beviláqua, \textunderscore Direito da Família\textunderscore .
\section{Cenoira}
\begin{itemize}
\item {Grp. gram.:f.}
\end{itemize}
Planta umbellífera, hortense.
(Cp. cast. \textunderscore cenoria\textunderscore )
\section{Cenologia}
\begin{itemize}
\item {Grp. gram.:f.}
\end{itemize}
\begin{itemize}
\item {Proveniência:(Do gr. \textunderscore koinos\textunderscore  + \textunderscore logos\textunderscore )}
\end{itemize}
Conferência entre médicos.
\section{Cenologia}
\begin{itemize}
\item {Grp. gram.:f.}
\end{itemize}
\begin{itemize}
\item {Proveniência:(Do gr. \textunderscore kenos\textunderscore  + \textunderscore logos\textunderscore )}
\end{itemize}
Parte da Phýsica, que trata do vácuo.
\section{Cenológico}
\begin{itemize}
\item {Grp. gram.:adj.}
\end{itemize}
Relativo á cenologia^2.
\section{Cenópode}
\begin{itemize}
\item {Grp. gram.:m.}
\end{itemize}
\begin{itemize}
\item {Proveniência:(Do gr. \textunderscore koinos\textunderscore , + \textunderscore pous\textunderscore , \textunderscore podos\textunderscore )}
\end{itemize}
Embryão das plantas monocotyledóneas.
\section{Cenórias!}
\begin{itemize}
\item {Grp. gram.:interj.}
\end{itemize}
\begin{itemize}
\item {Utilização:Prov.}
\end{itemize}
\begin{itemize}
\item {Utilização:trasm.}
\end{itemize}
\begin{itemize}
\item {Proveniência:(De \textunderscore ceno\textunderscore ^1)}
\end{itemize}
Trampa!
Cebolório!
\section{Cenoscópico}
\begin{itemize}
\item {Grp. gram.:adj.}
\end{itemize}
\begin{itemize}
\item {Proveniência:(Do gr. \textunderscore koinos\textunderscore  + \textunderscore skopein\textunderscore )}
\end{itemize}
Que tem por objecto as propriedades geraes dos seres.
\section{Cenósia}
\begin{itemize}
\item {Grp. gram.:f.}
\end{itemize}
Gênero de insectos dípteros.
\section{Cenosidade}
\begin{itemize}
\item {Grp. gram.:f.}
\end{itemize}
\begin{itemize}
\item {Proveniência:(Lat. \textunderscore coenositas\textunderscore )}
\end{itemize}
Lodaçal; immundície.
\section{Cenoso}
\begin{itemize}
\item {Grp. gram.:adj.}
\end{itemize}
\begin{itemize}
\item {Proveniência:(Lat. \textunderscore coenosus\textunderscore )}
\end{itemize}
Lamacento, immundo.
Torpe.
\section{Cenotáfio}
\begin{itemize}
\item {Grp. gram.:m.}
\end{itemize}
\begin{itemize}
\item {Proveniência:(Do gr. \textunderscore kenos\textunderscore  + \textunderscore taphos\textunderscore )}
\end{itemize}
Monumento fúnebre, erigido á memória de alguém, mas que lhe não contém o corpo.
\section{Cenotáphio}
\begin{itemize}
\item {Grp. gram.:m.}
\end{itemize}
\begin{itemize}
\item {Proveniência:(Do gr. \textunderscore kenos\textunderscore  + \textunderscore taphos\textunderscore )}
\end{itemize}
Monumento fúnebre, erigido á memória de alguém, mas que lhe não contém o corpo.
\section{Cenoura}
\begin{itemize}
\item {Grp. gram.:f.}
\end{itemize}
Planta umbellífera, hortense.
(Cp. cast. \textunderscore cenoria\textunderscore )
\section{Cenozoico}
\begin{itemize}
\item {Grp. gram.:adj.}
\end{itemize}
\begin{itemize}
\item {Utilização:Zool.}
\end{itemize}
\begin{itemize}
\item {Proveniência:(Do gr. \textunderscore kainos\textunderscore , recente e \textunderscore zoon\textunderscore , animal)}
\end{itemize}
Diz-se do período geológico, a cujos fósseis pertencem muitas espécies, que ainda hoje vivem.
\section{Cenra}
\begin{itemize}
\item {Grp. gram.:f.}
\end{itemize}
\begin{itemize}
\item {Utilização:Ant.}
\end{itemize}
O mesmo que \textunderscore seara\textunderscore , ou terreno em que crescem cereaes.
\section{Cenrada}
\begin{itemize}
\item {Grp. gram.:f.}
\end{itemize}
\begin{itemize}
\item {Utilização:Pesc.}
\end{itemize}
\begin{itemize}
\item {Proveniência:(Lat. hyp. \textunderscore cinerata\textunderscore , do lat. \textunderscore cinis\textunderscore , \textunderscore cineris\textunderscore , cinza)}
\end{itemize}
Barrela.
O mesmo que \textunderscore biscalongo\textunderscore .
\section{Cenradeiro}
\begin{itemize}
\item {Grp. gram.:m.}
\end{itemize}
\begin{itemize}
\item {Utilização:Prov.}
\end{itemize}
\begin{itemize}
\item {Utilização:alent.}
\end{itemize}
\begin{itemize}
\item {Proveniência:(De \textunderscore cenrada\textunderscore )}
\end{itemize}
Pano, em que se faz a barrela.
\section{Cenreira}
\begin{itemize}
\item {Grp. gram.:f.}
\end{itemize}
\begin{itemize}
\item {Utilização:Pop.}
\end{itemize}
Teimosia; birra. Cf. Garrett, \textunderscore Camões\textunderscore , 227.
\section{Censatário}
\begin{itemize}
\item {Grp. gram.:m.  e  adj.}
\end{itemize}
O que paga censo.
\section{Censionário}
\begin{itemize}
\item {Grp. gram.:m.  e  adj.}
\end{itemize}
(V.censatário)
\section{Censitário}
\begin{itemize}
\item {Grp. gram.:adj.}
\end{itemize}
Relativo ao censo; censatário.
\section{Censítico}
\begin{itemize}
\item {Grp. gram.:adj.}
\end{itemize}
O mesmo que \textunderscore emphytêutico\textunderscore . Cf. Latino, \textunderscore Elogios\textunderscore , 241.
\section{Censo}
\begin{itemize}
\item {Grp. gram.:m.}
\end{itemize}
\begin{itemize}
\item {Proveniência:(Lat. \textunderscore census\textunderscore )}
\end{itemize}
Recenseamento da população.
Rendimento, que serve de base ao exercício de certos direitos.
Pensão annual, pela posse de uma terra, ou em virtude de um contrato.
\section{Censor}
\begin{itemize}
\item {Grp. gram.:m.}
\end{itemize}
\begin{itemize}
\item {Proveniência:(Lat. \textunderscore censor\textunderscore )}
\end{itemize}
Magistrado, que, entre os Romanos, recenseava a população e velava pelos bons costumes.
Aquelle que censura.
Crítico.
Empregado público, encarregado da revísão e censura de obras literárias ou artísticas.
\section{Censório}
\begin{itemize}
\item {Grp. gram.:adj.}
\end{itemize}
\begin{itemize}
\item {Proveniência:(Lat. \textunderscore censorius\textunderscore )}
\end{itemize}
Relativo a censor ou á censura.
\section{Censual}
\begin{itemize}
\item {Grp. gram.:adj.}
\end{itemize}
\begin{itemize}
\item {Proveniência:(Lat. \textunderscore censualis\textunderscore )}
\end{itemize}
Relativo ao censo.
\section{Censualista}
\begin{itemize}
\item {Grp. gram.:m.}
\end{itemize}
\begin{itemize}
\item {Proveniência:(De \textunderscore censual\textunderscore )}
\end{itemize}
Recebedor de censos.
\section{Censualmente}
\begin{itemize}
\item {Grp. gram.:adv.}
\end{itemize}
\begin{itemize}
\item {Proveniência:(De \textunderscore censual\textunderscore )}
\end{itemize}
Com direito de censo.
\section{Censuário}
\begin{itemize}
\item {Grp. gram.:adj.}
\end{itemize}
(V.censual)
\section{Censuísta}
\begin{itemize}
\item {Grp. gram.:m.}
\end{itemize}
(V.censualista)
\section{Censura}
\begin{itemize}
\item {Grp. gram.:f.}
\end{itemize}
\begin{itemize}
\item {Proveniência:(Lat. \textunderscore censura\textunderscore )}
\end{itemize}
Cargo, dignidade, de censor.
Condemnação ecclesiástica de certas obras.
Exame crítico de obras literárias ou artísticas.
Corporação, encarregada dêsse exame.
Reprehensão.
\section{Censurador}
\begin{itemize}
\item {Grp. gram.:m.  e  adj.}
\end{itemize}
O que censura.
\section{Censurar}
\begin{itemize}
\item {Grp. gram.:v. t.}
\end{itemize}
Exercer censura sôbre.
Criticar.
Condemnar.
Reprehender.
\section{Censurável}
\begin{itemize}
\item {Grp. gram.:adj.}
\end{itemize}
\begin{itemize}
\item {Proveniência:(De \textunderscore censurar\textunderscore )}
\end{itemize}
Que merece censura.
\section{Censuria}
\begin{itemize}
\item {Grp. gram.:f.}
\end{itemize}
\begin{itemize}
\item {Utilização:Ant.}
\end{itemize}
Renda determinada, a que se obrigava quem tomava conta de casaes dos povoados. Cf. Herculano, \textunderscore Hist. de Port.\textunderscore , III, 359.
(Cp. \textunderscore censo\textunderscore )
\section{Centafolho}
\begin{itemize}
\item {fónica:fô}
\end{itemize}
\begin{itemize}
\item {Grp. gram.:m.}
\end{itemize}
Mesentério do boi. Cf. \textunderscore Eufrosina\textunderscore , 314; \textunderscore Aulegrafia\textunderscore , 4.
(Por \textunderscore centifólio\textunderscore )
\section{Centanário}
\begin{itemize}
\item {Grp. gram.:adj.}
\end{itemize}
Que tem cem anos.
Que tem séculos; secular:«\textunderscore ruge o canhão centanario\textunderscore », Júl. Castilho, \textunderscore Prim. Versos\textunderscore , 21.
\section{Centannário}
\begin{itemize}
\item {Grp. gram.:adj.}
\end{itemize}
Que tem cem annos.
Que tem séculos; secular:«\textunderscore ruge o canhão centannario\textunderscore », Júl. Castilho, \textunderscore Prim. Versos\textunderscore , 21.
\section{Centão}
\begin{itemize}
\item {Grp. gram.:m.}
\end{itemize}
\begin{itemize}
\item {Proveniência:(Lat. \textunderscore cento\textunderscore )}
\end{itemize}
Manta esfarrapada.
Cobertura grosseira de peças de artilharia.
Composição poética, formada de differentes versos de outrem, ou de fragmentos de versos alheios.
\section{Centáurea}
\begin{itemize}
\item {Grp. gram.:f.}
\end{itemize}
\begin{itemize}
\item {Proveniência:(Lat. \textunderscore centaurea\textunderscore )}
\end{itemize}
Planta medicinal, da fam. das compostas.
\section{Centáureo}
\begin{itemize}
\item {Grp. gram.:adj.}
\end{itemize}
Relativo ao centauro.
\section{Centauro}
\begin{itemize}
\item {Grp. gram.:m.}
\end{itemize}
\begin{itemize}
\item {Proveniência:(De \textunderscore Centauro\textunderscore , n. p. de um monstro fabuloso)}
\end{itemize}
Constellação austral.
\section{Centavo}
\begin{itemize}
\item {Grp. gram.:m.}
\end{itemize}
\begin{itemize}
\item {Proveniência:(T. cast.)}
\end{itemize}
Centésima parte: centésimo.
Moéda portuguesa, que é a centésima parte de um escudo e correspondente a 10 reis do anterior systema monetário.
\section{Centeal}
\begin{itemize}
\item {Grp. gram.:m.}
\end{itemize}
Seara de centeio.
\section{Centeia}
\begin{itemize}
\item {Grp. gram.:adj. f.}
\end{itemize}
Diz-se da palha e da farinha de centeio.
\section{Centeio}
\begin{itemize}
\item {Grp. gram.:m.}
\end{itemize}
\begin{itemize}
\item {Proveniência:(Do lat. \textunderscore centenus\textunderscore )}
\end{itemize}
Planta gramínea cerealífera.
\section{Centelha}
\begin{itemize}
\item {fónica:tê}
\end{itemize}
\begin{itemize}
\item {Grp. gram.:f.}
\end{itemize}
\begin{itemize}
\item {Proveniência:(Do lat. \textunderscore scintilla\textunderscore )}
\end{itemize}
Partícula luminosa, que resalta de um corpo encandescente.
Faísca.
Brilho momentâneo.
Revérbero.
\section{Centelhar}
\begin{itemize}
\item {Grp. gram.:v. i.}
\end{itemize}
O mesmo ou melhor que \textunderscore scintillar\textunderscore . Cf. Herculano, \textunderscore Lendas\textunderscore , I, 76; \textunderscore Bobo\textunderscore , 289; Filinto, VI, 181; Gonçalves Dias, \textunderscore Poes.\textunderscore , 109.
\section{Centena}
\begin{itemize}
\item {Grp. gram.:f.}
\end{itemize}
\begin{itemize}
\item {Proveniência:(Do lat. \textunderscore centeni\textunderscore )}
\end{itemize}
Quantidade de cem.
Unidade numérica, entre dezena e milhar.
\section{Centenar}
\begin{itemize}
\item {Grp. gram.:m.}
\end{itemize}
O mesmo que \textunderscore centena\textunderscore .
\section{Centenário}
\begin{itemize}
\item {Grp. gram.:adj.}
\end{itemize}
\begin{itemize}
\item {Grp. gram.:M.}
\end{itemize}
\begin{itemize}
\item {Proveniência:(Lat. \textunderscore centenarius\textunderscore )}
\end{itemize}
Que encerra o número cem.
Relativo a cem.
Cêntuplo; centuplicado.
Homem de cem ou mais anos.
Centurião.
Espaço de cem anos.
Comemoração secular.
\section{Centenarista}
\begin{itemize}
\item {Grp. gram.:m.}
\end{itemize}
Aquelle que celebra centenários, ou que é apologista de centenários. Cf. Camillo, \textunderscore Perfil\textunderscore , 127 e 164.
\section{Centenico}
\begin{itemize}
\item {Grp. gram.:m.}
\end{itemize}
\begin{itemize}
\item {Utilização:Prov.}
\end{itemize}
\begin{itemize}
\item {Utilização:trasm.}
\end{itemize}
\begin{itemize}
\item {Proveniência:(Do lat. \textunderscore centenus\textunderscore )}
\end{itemize}
Centeio temporão.
\section{Centenilha}
\begin{itemize}
\item {Grp. gram.:f.}
\end{itemize}
\begin{itemize}
\item {Proveniência:(Do rad. de \textunderscore centão\textunderscore )}
\end{itemize}
Planta primulácea, de que há uma espécie em o norte da Europa, e outra na América.
\section{Centenilho}
\begin{itemize}
\item {Grp. gram.:m.}
\end{itemize}
\begin{itemize}
\item {Proveniência:(Do rad. de \textunderscore centão\textunderscore )}
\end{itemize}
Planta primulácea, de que há uma espécie em o norte da Europa, e outra na América.
\section{Centeninho}
\begin{itemize}
\item {Grp. gram.:m.}
\end{itemize}
\begin{itemize}
\item {Utilização:Prov.}
\end{itemize}
\begin{itemize}
\item {Utilização:trasm.}
\end{itemize}
\begin{itemize}
\item {Proveniência:(Do lat. \textunderscore centenus\textunderscore )}
\end{itemize}
Centeio lavado e moído no moínho alveiro.
\section{Centenoso}
\begin{itemize}
\item {Grp. gram.:adj.}
\end{itemize}
\begin{itemize}
\item {Proveniência:(Do lat. \textunderscore centenus\textunderscore )}
\end{itemize}
Que produz centeio.
Semelhante ao centeio.
\section{Centesimal}
\begin{itemize}
\item {Grp. gram.:adj.}
\end{itemize}
Diz-se da fracção, que tem por denominador 100.
Diz-se da divisão, que contém 100 partes ou um múltiplo de cem.
Relativo a \textunderscore centésimo\textunderscore .
\section{Centésimo}
\begin{itemize}
\item {Grp. gram.:adj.}
\end{itemize}
\begin{itemize}
\item {Grp. gram.:M.}
\end{itemize}
\begin{itemize}
\item {Proveniência:(Lat. \textunderscore centesimus\textunderscore )}
\end{itemize}
Que numa série occupa o lugar de cem.
Centésima parte.
\section{Centi...}
\begin{itemize}
\item {Grp. gram.:pref.}
\end{itemize}
\begin{itemize}
\item {Proveniência:(Lat. \textunderscore centum\textunderscore )}
\end{itemize}
(indic. de \textunderscore cem\textunderscore , ou de que uma unidade é cem vezes menor que a unidade fundamental)
\section{Centiare}
\begin{itemize}
\item {Grp. gram.:m.}
\end{itemize}
\begin{itemize}
\item {Proveniência:(De \textunderscore centi...\textunderscore  + \textunderscore are\textunderscore )}
\end{itemize}
Centésima parte de um are; metro quadrado.
\section{Centieiro}
\begin{itemize}
\item {Grp. gram.:m.}
\end{itemize}
Nome que, em Castello-de-Paiva, se dá á \textunderscore escrevedeira\textunderscore .
\section{Centifólio}
\begin{itemize}
\item {Grp. gram.:adj.}
\end{itemize}
\begin{itemize}
\item {Proveniência:(Do lat. \textunderscore centum\textunderscore  + \textunderscore folium\textunderscore )}
\end{itemize}
Que tem cem fôlhas.
\section{Centígrado}
\begin{itemize}
\item {Grp. gram.:adj.}
\end{itemize}
\begin{itemize}
\item {Proveniência:(Do lat. \textunderscore centum\textunderscore  + \textunderscore gradus\textunderscore )}
\end{itemize}
Que tem cem graus: \textunderscore thermómetro centígrado\textunderscore .
\section{Centigrama}
\begin{itemize}
\item {Grp. gram.:m.}
\end{itemize}
\begin{itemize}
\item {Proveniência:(De \textunderscore centi...\textunderscore  + \textunderscore gramma\textunderscore )}
\end{itemize}
Centésima parte do grama.
\section{Centigramma}
\begin{itemize}
\item {Grp. gram.:m.}
\end{itemize}
\begin{itemize}
\item {Proveniência:(De \textunderscore centi...\textunderscore  + \textunderscore gramma\textunderscore )}
\end{itemize}
Centésima parte do gramma.
\section{Centilíngue}
\begin{itemize}
\item {Grp. gram.:adj.}
\end{itemize}
\begin{itemize}
\item {Proveniência:(De \textunderscore centi...\textunderscore  + \textunderscore língua\textunderscore )}
\end{itemize}
Que tem cem línguas.
Relativo a muitas línguas.
\section{Centilitro}
\begin{itemize}
\item {Grp. gram.:m.}
\end{itemize}
\begin{itemize}
\item {Proveniência:(De \textunderscore centi...\textunderscore  + \textunderscore litro\textunderscore )}
\end{itemize}
Centésima parte do litro.
\section{Centímano}
\begin{itemize}
\item {Grp. gram.:adj.}
\end{itemize}
\begin{itemize}
\item {Proveniência:(Lat. \textunderscore centimanus\textunderscore )}
\end{itemize}
Que tem cem mãos.
\section{Centímetro}
\begin{itemize}
\item {Grp. gram.:m.}
\end{itemize}
\begin{itemize}
\item {Proveniência:(De \textunderscore centi...\textunderscore  + \textunderscore metro\textunderscore )}
\end{itemize}
Centésima parte de um metro.
\section{Centimo}
\begin{itemize}
\item {Grp. gram.:m.}
\end{itemize}
\begin{itemize}
\item {Utilização:Gal}
\end{itemize}
\begin{itemize}
\item {Proveniência:(Fr. \textunderscore centime\textunderscore )}
\end{itemize}
Centésima parte do franco, moéda francesa.
\section{Centineto}
\begin{itemize}
\item {Grp. gram.:m.}
\end{itemize}
\begin{itemize}
\item {Utilização:Neol.}
\end{itemize}
\begin{itemize}
\item {Proveniência:(De \textunderscore centi...\textunderscore  + \textunderscore neto\textunderscore )}
\end{itemize}
Descendente muito afastado.
\section{Centinódia}
\begin{itemize}
\item {Grp. gram.:f.}
\end{itemize}
\begin{itemize}
\item {Proveniência:(Do lat. \textunderscore centum\textunderscore  + \textunderscore nodus\textunderscore )}
\end{itemize}
Planta, conhecida também por \textunderscore sempre-noiva\textunderscore .
\section{Centípeda}
\begin{itemize}
\item {Grp. gram.:f.}
\end{itemize}
(V.centopeia)
\section{Centípede}
\begin{itemize}
\item {Grp. gram.:adj.}
\end{itemize}
\begin{itemize}
\item {Proveniência:(Lat. \textunderscore centipes\textunderscore )}
\end{itemize}
Que tem cem pés.
\section{Cento}
\begin{itemize}
\item {Grp. gram.:m.  e  adj.}
\end{itemize}
\begin{itemize}
\item {Proveniência:(Lat. \textunderscore centum\textunderscore )}
\end{itemize}
O número cem; centena; cem.
\section{Centóculo}
\begin{itemize}
\item {Grp. gram.:adj.}
\end{itemize}
\begin{itemize}
\item {Proveniência:(Lat. \textunderscore centoculus\textunderscore )}
\end{itemize}
Que tem cem olhos.
\section{Centonização}
\begin{itemize}
\item {Grp. gram.:f.}
\end{itemize}
Acto ou effeito de centonizar.
\section{Centonizar}
\begin{itemize}
\item {Grp. gram.:v. t.}
\end{itemize}
\begin{itemize}
\item {Utilização:Neol.}
\end{itemize}
\begin{itemize}
\item {Proveniência:(De \textunderscore centão\textunderscore )}
\end{itemize}
Converter em centões ou numa composição poética (versos alheios), alterando-os, interpolando-os ou alternando-os com versos próprios. Cf. P. Caldas, \textunderscore Álvaro de Braga\textunderscore .
\section{Centopeia}
\begin{itemize}
\item {Grp. gram.:f.}
\end{itemize}
\begin{itemize}
\item {Proveniência:(Lat. hyp. \textunderscore centipedia\textunderscore , de \textunderscore centipes\textunderscore )}
\end{itemize}
Animal myriápode.
\section{Centoteca}
\begin{itemize}
\item {Grp. gram.:f.}
\end{itemize}
\begin{itemize}
\item {Proveniência:(Do gr. \textunderscore kentema\textunderscore  + \textunderscore theke\textunderscore )}
\end{itemize}
Planta gramínea da Ásia e da Austrália.
\section{Centotheca}
\begin{itemize}
\item {Grp. gram.:f.}
\end{itemize}
\begin{itemize}
\item {Proveniência:(Do gr. \textunderscore kentema\textunderscore  + \textunderscore theke\textunderscore )}
\end{itemize}
Planta gramínea da Ásia e da Austrália.
\section{Centradênia}
\begin{itemize}
\item {Grp. gram.:f.}
\end{itemize}
\begin{itemize}
\item {Proveniência:(Do gr. \textunderscore kentron\textunderscore  + \textunderscore aden\textunderscore )}
\end{itemize}
Planta do México.
\section{Central}
\begin{itemize}
\item {Grp. gram.:adj.}
\end{itemize}
\begin{itemize}
\item {Proveniência:(Lat. \textunderscore centralis\textunderscore )}
\end{itemize}
Relativo a centro.
Que está no centro: \textunderscore estação central\textunderscore .
\section{Centralidade}
\begin{itemize}
\item {Grp. gram.:f.}
\end{itemize}
\begin{itemize}
\item {Proveniência:(De \textunderscore central\textunderscore )}
\end{itemize}
Qualidade dos phenómenos, que se dão nos centros nervosos.
\section{Centralista}
\begin{itemize}
\item {Grp. gram.:m.  e  adj.}
\end{itemize}
\begin{itemize}
\item {Proveniência:(De \textunderscore central\textunderscore )}
\end{itemize}
Sectário da centralização dos poderes públicos.
\section{Centralização}
\begin{itemize}
\item {Grp. gram.:f.}
\end{itemize}
Acto de centralizar.
Systema político, que defere ao Gôverno ou ao poder central a resolução dos negócios mais importantes da administração pública, com exclusão da interferência das localidades ou fracções do país.
\section{Centralizador}
\begin{itemize}
\item {Grp. gram.:m.  e  adj.}
\end{itemize}
\begin{itemize}
\item {Proveniência:(De \textunderscore centralizar\textunderscore )}
\end{itemize}
O que centraliza.
Sectário da centralização administrativa; centralista.
\section{Céfala}
\begin{itemize}
\item {Grp. gram.:f.}
\end{itemize}
\begin{itemize}
\item {Proveniência:(Gr. \textunderscore kephale\textunderscore )}
\end{itemize}
Pequena borboleta diurna.
\section{Cefalado}
\begin{itemize}
\item {Grp. gram.:adj.}
\end{itemize}
\begin{itemize}
\item {Utilização:Hist. Nat.}
\end{itemize}
\begin{itemize}
\item {Proveniência:(Do gr. \textunderscore kephale\textunderscore , cabeça)}
\end{itemize}
Diz-se dos moluscos que têm cabeça, por oposição aos acéfalos.
\section{Cefalagra}
\begin{itemize}
\item {Grp. gram.:f.}
\end{itemize}
\begin{itemize}
\item {Utilização:Med.}
\end{itemize}
Doença gotosa na cabeça.
\section{Cefalalgia}
\begin{itemize}
\item {Grp. gram.:f.}
\end{itemize}
\begin{itemize}
\item {Proveniência:(Gr. \textunderscore kephalalgia\textunderscore )}
\end{itemize}
Dôr de cabeça.
\section{Cefalálgico}
\begin{itemize}
\item {Grp. gram.:adj.}
\end{itemize}
Relativo á cefalalgia.
\section{Cefalandra}
\begin{itemize}
\item {Grp. gram.:f.}
\end{itemize}
\begin{itemize}
\item {Proveniência:(Do gr. \textunderscore kephale\textunderscore  + \textunderscore andros\textunderscore )}
\end{itemize}
Planta cucurbitácea do Cabo da Bôa-Esperança.
\section{Cefalanto}
\begin{itemize}
\item {Grp. gram.:m.}
\end{itemize}
\begin{itemize}
\item {Proveniência:(Do gr. \textunderscore kephale\textunderscore  + \textunderscore anthos\textunderscore )}
\end{itemize}
Formoso arbusto da America.
\section{Cefalapagia}
\begin{itemize}
\item {Grp. gram.:f.}
\end{itemize}
Qualidade de \textunderscore cefalápagos\textunderscore .
\section{Cefalápagos}
\begin{itemize}
\item {Grp. gram.:m. pl.}
\end{itemize}
Monstros humanos, ligados pela cabeça.
\section{Cefalária}
\begin{itemize}
\item {Grp. gram.:f.}
\end{itemize}
\begin{itemize}
\item {Proveniência:(Do gr. \textunderscore kephale\textunderscore )}
\end{itemize}
Gênero de plantas dipsáceas.
\section{Cefaleia}
\begin{itemize}
\item {Grp. gram.:f.}
\end{itemize}
\begin{itemize}
\item {Proveniência:(Do gr. \textunderscore kephale\textunderscore )}
\end{itemize}
Dôr violenta de cabeça.
\section{Cefalematoma}
\begin{itemize}
\item {Grp. gram.:m.}
\end{itemize}
Tumor resistente e fluctuante, no crânio das crianças.
\section{Cefalia}
\begin{itemize}
\item {Grp. gram.:f.}
\end{itemize}
O mesmo ou melhor que \textunderscore cefaleia\textunderscore .
\section{Cefálico}
\begin{itemize}
\item {Grp. gram.:adj.}
\end{itemize}
\begin{itemize}
\item {Proveniência:(Gr. \textunderscore kephalikos\textunderscore )}
\end{itemize}
Relativo á cabeça ou ao cérebro.
\section{Cefalídeos}
\begin{itemize}
\item {Grp. gram.:m. pl.}
\end{itemize}
Moluscos, que têm cérebro rudimental.
\section{Cefalite}
\begin{itemize}
\item {Grp. gram.:f.}
\end{itemize}
\begin{itemize}
\item {Proveniência:(Do gr. \textunderscore kephale\textunderscore )}
\end{itemize}
Inflamação cerebral.
\section{Cefalóbaros}
\begin{itemize}
\item {Grp. gram.:m. pl.}
\end{itemize}
\begin{itemize}
\item {Proveniência:(Do gr. \textunderscore kephale\textunderscore  + \textunderscore baros\textunderscore )}
\end{itemize}
Insectos coleópteros pentámeros.
\section{Cefalodela}
\begin{itemize}
\item {Grp. gram.:f.}
\end{itemize}
\begin{itemize}
\item {Proveniência:(Do gr. \textunderscore kephale\textunderscore  + \textunderscore delos\textunderscore )}
\end{itemize}
Animálculo, cujo corpo termina por cabeça sem bôca visível.
\section{Cefalodial}
\begin{itemize}
\item {Grp. gram.:adj.}
\end{itemize}
\begin{itemize}
\item {Utilização:Bot.}
\end{itemize}
Diz-se da frutificação de certos líchens.
\section{Cefalodiano}
\begin{itemize}
\item {Grp. gram.:adj.}
\end{itemize}
\begin{itemize}
\item {Utilização:Bot.}
\end{itemize}
\begin{itemize}
\item {Grp. gram.:M. pl.}
\end{itemize}
Que tem cefalódios.
Ordem de líchens, que comprehende os que têm os conceptáculos quási globulosos.
\section{Cefalódio}
\begin{itemize}
\item {Grp. gram.:m.}
\end{itemize}
\begin{itemize}
\item {Utilização:Bot.}
\end{itemize}
Apoteca globulosa dos líchens.
\section{Cefalodontes}
\begin{itemize}
\item {Grp. gram.:m. pl.}
\end{itemize}
\begin{itemize}
\item {Proveniência:(Do gr. \textunderscore kephale\textunderscore  + \textunderscore odous\textunderscore , \textunderscore odontos\textunderscore )}
\end{itemize}
Gênero de insectos coleópteros da América.
\section{Cefalografia}
\begin{itemize}
\item {Grp. gram.:f.}
\end{itemize}
\begin{itemize}
\item {Proveniência:(Do gr. \textunderscore kephale\textunderscore  + \textunderscore graphein\textunderscore )}
\end{itemize}
Descripção anatómica da cabeça.
\section{Cefalográfico}
\begin{itemize}
\item {Grp. gram.:adj.}
\end{itemize}
Relativo á cephalografia.
\section{Cefaloide}
\begin{itemize}
\item {Grp. gram.:adj.}
\end{itemize}
\begin{itemize}
\item {Proveniência:(Do gr. \textunderscore kephale\textunderscore  + \textunderscore eidos\textunderscore )}
\end{itemize}
Que tem fórma de cabeça.
\section{Cefaloídeos}
\begin{itemize}
\item {Grp. gram.:m. pl.}
\end{itemize}
Divisão da fam. dos líchens, caracterizada pela fórma da frutificação.
(Cp. \textunderscore cephaloide\textunderscore )
\section{Cefaloleia}
\begin{itemize}
\item {Grp. gram.:f.}
\end{itemize}
\begin{itemize}
\item {Proveniência:(Do gr. \textunderscore kephale\textunderscore  + \textunderscore leia\textunderscore )}
\end{itemize}
Insecto coleóptero tetrâmero.
\section{Cefalomancia}
\begin{itemize}
\item {Grp. gram.:f.}
\end{itemize}
\begin{itemize}
\item {Proveniência:(Do gr. \textunderscore kephale\textunderscore  + \textunderscore manteia\textunderscore )}
\end{itemize}
Adivinhação, por meio da cabeça de um burro, collocada sôbre o fôgo.
\section{Cefalomântico}
\begin{itemize}
\item {Grp. gram.:adj.}
\end{itemize}
Relativo á cefalomância.
\section{Cefalometria}
\begin{itemize}
\item {Grp. gram.:f.}
\end{itemize}
\begin{itemize}
\item {Proveniência:(De \textunderscore cephalómetro\textunderscore )}
\end{itemize}
Medição da cabeça, no estudo das raças humanas.
\section{Cefalómetro}
\begin{itemize}
\item {Grp. gram.:m.}
\end{itemize}
\begin{itemize}
\item {Proveniência:(Do gr. \textunderscore kephale\textunderscore  + \textunderscore metron\textunderscore )}
\end{itemize}
Instrumento, para medir as dimensões da cabeça.
\section{Cefalófora}
\begin{itemize}
\item {Grp. gram.:f.}
\end{itemize}
\begin{itemize}
\item {Proveniência:(De \textunderscore cephalóphoro\textunderscore )}
\end{itemize}
Gênero de plantas compostas.
\section{Cefalóforo}
\begin{itemize}
\item {Grp. gram.:adj.}
\end{itemize}
\begin{itemize}
\item {Utilização:Bot.}
\end{itemize}
\begin{itemize}
\item {Proveniência:(Do gr. \textunderscore kephale\textunderscore  + \textunderscore phoros\textunderscore )}
\end{itemize}
Que tem flôr em fórma de cabeça.
\section{Cefalópodes}
\begin{itemize}
\item {Grp. gram.:m. pl.}
\end{itemize}
\begin{itemize}
\item {Proveniência:(Do gr. \textunderscore kephale\textunderscore  + \textunderscore pous\textunderscore , \textunderscore podos\textunderscore )}
\end{itemize}
Moluscos, que têm os tentáculos á roda da bôca.
\section{Cefalóptero}
\begin{itemize}
\item {Grp. gram.:adj.}
\end{itemize}
\begin{itemize}
\item {Utilização:Zool.}
\end{itemize}
\begin{itemize}
\item {Proveniência:(Do gr. \textunderscore kephale\textunderscore  + \textunderscore pteron\textunderscore )}
\end{itemize}
Que na cabeça tem penas, que aparentam a fórma de uma asa.
\section{Cefaloscopia}
\begin{itemize}
\item {Grp. gram.:f.}
\end{itemize}
\begin{itemize}
\item {Proveniência:(Do gr. \textunderscore kephale\textunderscore  + \textunderscore skopein\textunderscore )}
\end{itemize}
Exame da cabeça, para se conhecer o estado das faculdades intelectuaes.
\section{Cefalossomo}
\begin{itemize}
\item {Grp. gram.:adj.}
\end{itemize}
\begin{itemize}
\item {Proveniência:(Do gr. \textunderscore kephale\textunderscore  + \textunderscore soma\textunderscore )}
\end{itemize}
Diz-se do peixe que tem o corpo grosso na parte anterior.
\section{Cefalostigma}
\begin{itemize}
\item {Grp. gram.:f.}
\end{itemize}
\begin{itemize}
\item {Proveniência:(Do gr. \textunderscore kephale\textunderscore  + \textunderscore stigma\textunderscore )}
\end{itemize}
Planta campanulácea da Birmânia e de Senegâmbia.
\section{Cefalote}
\begin{itemize}
\item {Grp. gram.:m.  e  adj.}
\end{itemize}
O mesmo que \textunderscore cefaloto\textunderscore .
\section{Cefaloteca}
\begin{itemize}
\item {Grp. gram.:f.}
\end{itemize}
\begin{itemize}
\item {Proveniência:(Do gr. \textunderscore kephale\textunderscore  + \textunderscore theke\textunderscore )}
\end{itemize}
Invólucro da cabeça das crisálidas.
\section{Cefalotórax}
\begin{itemize}
\item {Grp. gram.:m.}
\end{itemize}
\begin{itemize}
\item {Proveniência:(Do gr. \textunderscore kephale\textunderscore  + \textunderscore thorax\textunderscore )}
\end{itemize}
A cabeça e o tórax de certos insectos.
\section{Cefaloto}
\begin{itemize}
\item {Grp. gram.:m.  e  adj.}
\end{itemize}
\begin{itemize}
\item {Proveniência:(Do gr. \textunderscore kephalotos\textunderscore )}
\end{itemize}
Nome de varios peixes, morcegos e insectos, que têm cabeça grande.
\section{Cefalotomia}
\begin{itemize}
\item {Grp. gram.:f.}
\end{itemize}
Operação, com que se parte a cabeça de um féto, para facilitar a saída da bacia.
(Cp. \textunderscore cephalótomo\textunderscore )
\section{Cefalótomo}
\begin{itemize}
\item {Grp. gram.:m.}
\end{itemize}
\begin{itemize}
\item {Proveniência:(Do gr. \textunderscore kephale\textunderscore  + \textunderscore tome\textunderscore )}
\end{itemize}
Instrumento, próprio para a cefalotomia.
\section{Cefalótribo}
\begin{itemize}
\item {Grp. gram.:m.}
\end{itemize}
\begin{itemize}
\item {Proveniência:(Do gr. \textunderscore kephale\textunderscore  + \textunderscore tribein\textunderscore )}
\end{itemize}
Instrumento, para esmagar a cabeça do féto e facilitar-lhe a saída da bacia da parturiente.
\section{Cefélia}
\begin{itemize}
\item {Grp. gram.:f.}
\end{itemize}
\begin{itemize}
\item {Proveniência:(Do gr. \textunderscore kephele\textunderscore )}
\end{itemize}
Planta rubiácea da América.
\section{Cefeu}
\begin{itemize}
\item {Grp. gram.:m.}
\end{itemize}
\begin{itemize}
\item {Proveniência:(Do gr. \textunderscore Kepheus\textunderscore , n. p.)}
\end{itemize}
Constelação setentrional.
\section{Cefísio}
\begin{itemize}
\item {Grp. gram.:adj.}
\end{itemize}
Relativo a Cefiso. Cf. \textunderscore Lusíadas\textunderscore , IX, 60.
\section{Centia}
\begin{itemize}
\item {Grp. gram.:f.}
\end{itemize}
(?)«\textunderscore ...vejome remeyro preso--em centia de gualee\textunderscore ». Resende, \textunderscore Cancion.\textunderscore , f. 27, V.^o
\section{Centralizar}
\begin{itemize}
\item {Grp. gram.:v. t.}
\end{itemize}
Tornar central.
Reunir num centro.
Fazer convergir para um centro.
\section{Centralmente}
\begin{itemize}
\item {Grp. gram.:adv.}
\end{itemize}
\begin{itemize}
\item {Proveniência:(De \textunderscore central\textunderscore )}
\end{itemize}
No centro; pelo centro.
\section{Centrantera}
\begin{itemize}
\item {Grp. gram.:f.}
\end{itemize}
\begin{itemize}
\item {Proveniência:(Do gr. \textunderscore kentron\textunderscore  + \textunderscore anthere\textunderscore )}
\end{itemize}
Planta escrofularínea da Ásia e da Austrália.
\section{Centranthera}
\begin{itemize}
\item {Grp. gram.:f.}
\end{itemize}
\begin{itemize}
\item {Proveniência:(Do gr. \textunderscore kentron\textunderscore  + \textunderscore anthere\textunderscore )}
\end{itemize}
Planta escrofularínea da Ásia e da Austrália.
\section{Centrantho}
\begin{itemize}
\item {Grp. gram.:m.}
\end{itemize}
\begin{itemize}
\item {Proveniência:(Do gr. \textunderscore kentron\textunderscore  + \textunderscore anthos\textunderscore )}
\end{itemize}
Planta valerianácea, que abrange seis espécies herbáceas.
\section{Centranto}
\begin{itemize}
\item {Grp. gram.:m.}
\end{itemize}
\begin{itemize}
\item {Proveniência:(Do gr. \textunderscore kentron\textunderscore  + \textunderscore anthos\textunderscore )}
\end{itemize}
Planta valerianácea, que abrange seis espécies herbáceas.
\section{Centrarco}
\begin{itemize}
\item {Grp. gram.:m.}
\end{itemize}
\begin{itemize}
\item {Proveniência:(Do gr. \textunderscore kentron\textunderscore  + \textunderscore arkhos\textunderscore )}
\end{itemize}
Peixe de água dôce, na América do Norte.
\section{Centricipital}
\begin{itemize}
\item {Grp. gram.:adj.}
\end{itemize}
\begin{itemize}
\item {Proveniência:(Do rad. de \textunderscore centricipúcio\textunderscore )}
\end{itemize}
Relativo ao centricipúcio.
\section{Centricipúcio}
\begin{itemize}
\item {Grp. gram.:m.}
\end{itemize}
\begin{itemize}
\item {Proveniência:(Do lat. \textunderscore centrum\textunderscore  + \textunderscore caput\textunderscore )}
\end{itemize}
Parte média do crânio.
\section{Centrífugo}
\begin{itemize}
\item {Grp. gram.:adj.}
\end{itemize}
\begin{itemize}
\item {Grp. gram.:M.}
\end{itemize}
\begin{itemize}
\item {Proveniência:(Do lat. \textunderscore centrum\textunderscore  + \textunderscore fugere\textunderscore )}
\end{itemize}
Que se afasta do centro; que procura desviar-se do centro: \textunderscore fôrças centrífugas\textunderscore .
O mesmo que \textunderscore centrífuga\textunderscore .
\section{Centrino}
\begin{itemize}
\item {Grp. gram.:m.}
\end{itemize}
Designação scientífica do \textunderscore peixe-porco\textunderscore .
\section{Centrípeto}
\begin{itemize}
\item {Grp. gram.:adj.}
\end{itemize}
\begin{itemize}
\item {Proveniência:(Do lat. \textunderscore centrum\textunderscore  + \textunderscore petere\textunderscore )}
\end{itemize}
Que se dirige ao centro; que procura aproximar-se do centro: \textunderscore fôrça centrípeta\textunderscore .
\section{Centrisco}
\begin{itemize}
\item {Grp. gram.:m.}
\end{itemize}
Gênero de peixes marítimos.
\section{Centro}
\begin{itemize}
\item {Grp. gram.:m.}
\end{itemize}
\begin{itemize}
\item {Proveniência:(Lat. \textunderscore centrum\textunderscore )}
\end{itemize}
Ponto, que está a igual distância de todos os pontos de uma circumferencia, ou de todos os pontos da superfície de uma esphera.
Meio de uma linha recta, que divide uma figura ou um espaço em duas partes iguaes.
Ponto que, numa superfície curva, divide em duas partes iguaes o arco traçado sôbre ella.
Meio de qualquer espaço.
Fundo, profundeza: \textunderscore no centro da terra\textunderscore .
Ponto, para onde as coisas convergem, como para uma natural posição de repoiso.
Lugar, onde habitualmente se procuram certas coisas ou se tratam certos negócios: \textunderscore um grande centro commercial\textunderscore .
Papel theatral, concernente a uma personagem secundária de idade madura ou avançada. Legisladores, que, nas respectivas assembleias, tomam lugar entre os amigos do Govêrno e os opposicionistas: \textunderscore o centro applaudiu o discurso do Ministro\textunderscore .
Assembleia, lugar, em que se reúnem partidários de uma facção política: \textunderscore encontravam-no todas as noites no Centro Republicano\textunderscore .
Casino, club.
\section{Centrobárico}
\begin{itemize}
\item {Grp. gram.:adj.}
\end{itemize}
\begin{itemize}
\item {Proveniência:(Do gr. \textunderscore kentron\textunderscore  + \textunderscore baros\textunderscore )}
\end{itemize}
Que depende do centro de gravidade.
\section{Centrocerco}
\begin{itemize}
\item {Grp. gram.:m.}
\end{itemize}
\begin{itemize}
\item {Proveniência:(Do gr. \textunderscore kentron\textunderscore  + \textunderscore kerkos\textunderscore )}
\end{itemize}
Ave da Califórnia.
\section{Centrodonte}
\begin{itemize}
\item {Grp. gram.:adj.}
\end{itemize}
\begin{itemize}
\item {Utilização:Zool.}
\end{itemize}
\begin{itemize}
\item {Proveniência:(Do gr. \textunderscore kentron\textunderscore  + \textunderscore odous\textunderscore , \textunderscore odontos\textunderscore )}
\end{itemize}
Que tem dentes agudos.
\section{Centrofilo}
\begin{itemize}
\item {Grp. gram.:m.}
\end{itemize}
\begin{itemize}
\item {Proveniência:(Do gr. \textunderscore kentron\textunderscore  + \textunderscore phullon\textunderscore )}
\end{itemize}
Gênero de plantas synanthéreas.
\section{Centrogastro}
\begin{itemize}
\item {Grp. gram.:m.}
\end{itemize}
\begin{itemize}
\item {Proveniência:(Do gr. \textunderscore kentron\textunderscore  + \textunderscore gaster\textunderscore )}
\end{itemize}
Nome de um peixe.
\section{Centrolépida}
\begin{itemize}
\item {Grp. gram.:f.}
\end{itemize}
\begin{itemize}
\item {Proveniência:(Do gr. \textunderscore kentron\textunderscore  + \textunderscore lepis\textunderscore )}
\end{itemize}
Planta da Tasmânia.
\section{Centropétala}
\begin{itemize}
\item {Grp. gram.:f. pl.}
\end{itemize}
\begin{itemize}
\item {Proveniência:(Do gr. \textunderscore kentron\textunderscore  + \textunderscore petalon\textunderscore )}
\end{itemize}
Planta peruana, da fam. das orchídeas.
\section{Centrophyllo}
\begin{itemize}
\item {Grp. gram.:m.}
\end{itemize}
\begin{itemize}
\item {Proveniência:(Do gr. \textunderscore kentron\textunderscore  + \textunderscore phullon\textunderscore )}
\end{itemize}
Gênero de plantas synanthéreas.
\section{Centrópode}
\begin{itemize}
\item {Grp. gram.:m.}
\end{itemize}
\begin{itemize}
\item {Proveniência:(Do gr. \textunderscore kentron\textunderscore  + \textunderscore pous\textunderscore , \textunderscore podos\textunderscore )}
\end{itemize}
Nome de um peixe.
\section{Centroscopia}
\begin{itemize}
\item {Grp. gram.:f.}
\end{itemize}
\begin{itemize}
\item {Proveniência:(Do gr. \textunderscore kentron\textunderscore  + \textunderscore skopein\textunderscore )}
\end{itemize}
Parte da Geometria, que trata do centro das grandezas.
\section{Centroscópico}
\begin{itemize}
\item {Grp. gram.:adj.}
\end{itemize}
Relativo á centroscopia.
\section{Centrosemo}
\begin{itemize}
\item {fónica:sê}
\end{itemize}
\begin{itemize}
\item {Grp. gram.:m.}
\end{itemize}
\begin{itemize}
\item {Proveniência:(Do gr. \textunderscore kentron\textunderscore  + \textunderscore sema\textunderscore )}
\end{itemize}
Planta leguminosa do Brasil.
\section{Centrossemo}
\begin{itemize}
\item {Grp. gram.:m.}
\end{itemize}
\begin{itemize}
\item {Proveniência:(Do gr. \textunderscore kentron\textunderscore  + \textunderscore sema\textunderscore )}
\end{itemize}
Planta leguminosa do Brasil.
\section{Centrostemmo}
\begin{itemize}
\item {Grp. gram.:m.}
\end{itemize}
\begin{itemize}
\item {Proveniência:(Do gr. \textunderscore kentron\textunderscore  + \textunderscore stemma\textunderscore )}
\end{itemize}
Planta do Japão.
\section{Centrostemo}
\begin{itemize}
\item {Grp. gram.:m.}
\end{itemize}
\begin{itemize}
\item {Proveniência:(Do gr. \textunderscore kentron\textunderscore  + \textunderscore stemma\textunderscore )}
\end{itemize}
Planta do Japão.
\section{Centroto}
\begin{itemize}
\item {Grp. gram.:m.}
\end{itemize}
\begin{itemize}
\item {Proveniência:(Do gr. \textunderscore kentron\textunderscore )}
\end{itemize}
Insecto hemíptero.
\section{Centrúridos}
\begin{itemize}
\item {Grp. gram.:m. pl.}
\end{itemize}
Família de escorpiões, que têm por typo o \textunderscore centruro\textunderscore .
\section{Centruro}
\begin{itemize}
\item {Grp. gram.:m.}
\end{itemize}
\begin{itemize}
\item {Proveniência:(Do gr. \textunderscore kentron\textunderscore  + \textunderscore oura\textunderscore )}
\end{itemize}
Gênero de escorpiões, que têm déz olhos e vive na América.
\section{Centumvirado}
\begin{itemize}
\item {Grp. gram.:m.}
\end{itemize}
Magistratura dos centúmviros.
\section{Centumviral}
\begin{itemize}
\item {Grp. gram.:adj.}
\end{itemize}
\begin{itemize}
\item {Proveniência:(Lat. \textunderscore centumviralis\textunderscore )}
\end{itemize}
Relativo aos centúmviros.
\section{Centumvirato}
\begin{itemize}
\item {Grp. gram.:m.}
\end{itemize}
O mesmo que \textunderscore centumvirado\textunderscore .
\section{Centúmviros}
\begin{itemize}
\item {Grp. gram.:m. pl.}
\end{itemize}
\begin{itemize}
\item {Proveniência:(Lat. \textunderscore centumviri\textunderscore )}
\end{itemize}
Cem magistrados, que constituíam um tribunal em Roma.
\section{Centunvirado}
\begin{itemize}
\item {Grp. gram.:m.}
\end{itemize}
Magistratura dos centúnviros.
\section{Centunviral}
\begin{itemize}
\item {Grp. gram.:adj.}
\end{itemize}
\begin{itemize}
\item {Proveniência:(Lat. \textunderscore centumviralis\textunderscore )}
\end{itemize}
Relativo aos centúnviros.
\section{Centunvirato}
\begin{itemize}
\item {Grp. gram.:m.}
\end{itemize}
O mesmo que \textunderscore centunvirado\textunderscore .
\section{Centúnviros}
\begin{itemize}
\item {Grp. gram.:m. pl.}
\end{itemize}
\begin{itemize}
\item {Proveniência:(Lat. \textunderscore centumviri\textunderscore )}
\end{itemize}
Cem magistrados, que constituíam um tribunal em Roma.
\section{Centuplicadamente}
\begin{itemize}
\item {Grp. gram.:adv.}
\end{itemize}
De modo centuplicado.
\section{Centuplicado}
\begin{itemize}
\item {Grp. gram.:adj.}
\end{itemize}
\begin{itemize}
\item {Utilização:Fig.}
\end{itemize}
Multiplicado por cem.
Muito aumentado.
\section{Centuplicar}
\begin{itemize}
\item {Grp. gram.:v. t.}
\end{itemize}
\begin{itemize}
\item {Utilização:Fig.}
\end{itemize}
\begin{itemize}
\item {Proveniência:(Do lat. \textunderscore centuplex\textunderscore )}
\end{itemize}
Dobrar cem vezes; multiplicar por cem.
Avolumar, aumentar muito.
\section{Cêntuplo}
\begin{itemize}
\item {Grp. gram.:adj.}
\end{itemize}
\begin{itemize}
\item {Grp. gram.:M.}
\end{itemize}
\begin{itemize}
\item {Proveniência:(Lat. \textunderscore centuplus\textunderscore )}
\end{itemize}
Centuplicado.
Resultado da multiplicação por cem.
\section{Centúria}
\begin{itemize}
\item {Grp. gram.:f.}
\end{itemize}
\begin{itemize}
\item {Proveniência:(Lat. \textunderscore centuria\textunderscore )}
\end{itemize}
Centena; grupo de cem objectos da mesma espécie.
Uma das divisões políticas dos Romanos.
Companhia de cem homens de guerra.
Centenário.
Narração histórica, dividida em períodos seculares.
\section{Centurial}
\begin{itemize}
\item {Grp. gram.:adj.}
\end{itemize}
\begin{itemize}
\item {Proveniência:(Lat. \textunderscore centurialis\textunderscore )}
\end{itemize}
Relativo a centúria.
\section{Centurião}
\begin{itemize}
\item {Grp. gram.:m.}
\end{itemize}
\begin{itemize}
\item {Proveniência:(Lat. \textunderscore centurio\textunderscore )}
\end{itemize}
Chefe de cem homens ou de uma centúria, no exército romano.
\section{Centuriato}
\begin{itemize}
\item {Grp. gram.:m.}
\end{itemize}
Cargo de centurião. Cf. Castilho, \textunderscore Fastos\textunderscore , II, 223.
\section{Centúrio}
\begin{itemize}
\item {Grp. gram.:m.}
\end{itemize}
(V.centurião)
\section{Centurionado}
\begin{itemize}
\item {Grp. gram.:m.}
\end{itemize}
\begin{itemize}
\item {Proveniência:(Lat. \textunderscore centurionatus\textunderscore )}
\end{itemize}
Dignidade, cargo, de centurião.
\section{Centuriónico}
\begin{itemize}
\item {Grp. gram.:adj.}
\end{itemize}
Relativo ao centurião.
\section{Centya}
\begin{itemize}
\item {Grp. gram.:f.}
\end{itemize}
(?)«\textunderscore ...vejome remeyro preso--em centya de gualee\textunderscore ». Resende, \textunderscore Cancion.\textunderscore , f. 27, V.^o
\section{Cenudo}
\begin{itemize}
\item {Grp. gram.:adj.}
\end{itemize}
\begin{itemize}
\item {Utilização:Prov.}
\end{itemize}
\begin{itemize}
\item {Utilização:trasm.}
\end{itemize}
\begin{itemize}
\item {Proveniência:(De \textunderscore ceno\textunderscore ^1)}
\end{itemize}
Carrancudo.
\section{Cenuro}
\begin{itemize}
\item {Grp. gram.:m.}
\end{itemize}
\begin{itemize}
\item {Proveniência:(Do gr. \textunderscore koinos\textunderscore  + \textunderscore oura\textunderscore )}
\end{itemize}
Gênero de helminthos, que têm uma vesícula comum a muitos corpos.
\section{Céomo}
\begin{itemize}
\item {Grp. gram.:m.}
\end{itemize}
\begin{itemize}
\item {Proveniência:(Do gr. \textunderscore khaio\textunderscore )}
\end{itemize}
Cogumelo microscópico e parasito.
\section{Ceóte}
\begin{itemize}
\item {Grp. gram.:m.}
\end{itemize}
\begin{itemize}
\item {Utilização:Fam.}
\end{itemize}
Pequena ceia; ceia.
\section{Cepa}
\begin{itemize}
\item {fónica:cê}
\end{itemize}
\begin{itemize}
\item {Grp. gram.:f.}
\end{itemize}
\begin{itemize}
\item {Proveniência:(De \textunderscore cepo\textunderscore )}
\end{itemize}
Tronco de videira.
Parte inferior das árvores, incluindo as raízes, de que se faz carvão.
\textunderscore Estar sempre na cepa torta\textunderscore , ou \textunderscore não passar da cepa torta\textunderscore , não melhorar de posição, não progredir, não apprender.
\section{Cepáceo}
\begin{itemize}
\item {Grp. gram.:adj.}
\end{itemize}
\begin{itemize}
\item {Proveniência:(Do lat. \textunderscore caepa\textunderscore )}
\end{itemize}
Que tem cheiro ou fórma de cebola.
\section{Cepeira}
\begin{itemize}
\item {Grp. gram.:f.}
\end{itemize}
O mesmo que \textunderscore cepa\textunderscore .
\section{Cepelho}
\begin{itemize}
\item {fónica:pê}
\end{itemize}
\begin{itemize}
\item {Grp. gram.:m.}
\end{itemize}
\begin{itemize}
\item {Utilização:Des.}
\end{itemize}
Armadilha para caçar, mais conhecida hoje por \textunderscore cepo\textunderscore .
\section{Céphala}
\begin{itemize}
\item {Grp. gram.:f.}
\end{itemize}
\begin{itemize}
\item {Proveniência:(Gr. \textunderscore kephale\textunderscore )}
\end{itemize}
Pequena borboleta diurna.
\section{Cephalado}
\begin{itemize}
\item {Grp. gram.:adj.}
\end{itemize}
\begin{itemize}
\item {Utilização:Hist. Nat.}
\end{itemize}
\begin{itemize}
\item {Proveniência:(Do gr. \textunderscore kephale\textunderscore , cabeça)}
\end{itemize}
Diz-se dos molluscos que têm cabeça, por opposição aos acéphalos.
\section{Cephalagra}
\begin{itemize}
\item {Grp. gram.:f.}
\end{itemize}
\begin{itemize}
\item {Utilização:Med.}
\end{itemize}
Doença gotosa na cabeça.
\section{Cephalalgia}
\begin{itemize}
\item {Grp. gram.:f.}
\end{itemize}
\begin{itemize}
\item {Proveniência:(Gr. \textunderscore kephalalgia\textunderscore )}
\end{itemize}
Dôr de cabeça.
\section{Cephalálgico}
\begin{itemize}
\item {Grp. gram.:adj.}
\end{itemize}
Relativo á cephalalgia.
\section{Cephalandra}
\begin{itemize}
\item {Grp. gram.:f.}
\end{itemize}
\begin{itemize}
\item {Proveniência:(Do gr. \textunderscore kephale\textunderscore  + \textunderscore andros\textunderscore )}
\end{itemize}
Planta cucurbitácea do Cabo da Bôa-Esperança.
\section{Cephalanto}
\begin{itemize}
\item {Grp. gram.:m.}
\end{itemize}
\begin{itemize}
\item {Proveniência:(Do gr. \textunderscore kephale\textunderscore  + \textunderscore anthos\textunderscore )}
\end{itemize}
Formoso arbusto da America.
\section{Cephalapagia}
\begin{itemize}
\item {Grp. gram.:f.}
\end{itemize}
Qualidade de \textunderscore cephalápagos\textunderscore .
\section{Cephalápagos}
\begin{itemize}
\item {Grp. gram.:m. pl.}
\end{itemize}
Monstros humanos, ligados pela cabeça.
\section{Cephalária}
\begin{itemize}
\item {Grp. gram.:f.}
\end{itemize}
\begin{itemize}
\item {Proveniência:(Do gr. \textunderscore kephale\textunderscore )}
\end{itemize}
Gênero de plantas dipsáceas.
\section{Cephaleia}
\begin{itemize}
\item {Grp. gram.:f.}
\end{itemize}
\begin{itemize}
\item {Proveniência:(Do gr. \textunderscore kephale\textunderscore )}
\end{itemize}
Dôr violenta de cabeça.
\section{Cephalematoma}
\begin{itemize}
\item {Grp. gram.:m.}
\end{itemize}
Tumor resistente e fluctuante, no crânio das crianças.
\section{Cephalia}
\begin{itemize}
\item {Grp. gram.:f.}
\end{itemize}
O mesmo ou melhor que \textunderscore cephaleia\textunderscore .
\section{Cephálico}
\begin{itemize}
\item {Grp. gram.:adj.}
\end{itemize}
\begin{itemize}
\item {Proveniência:(Gr. \textunderscore kephalikos\textunderscore )}
\end{itemize}
Relativo á cabeça ou ao cérebro.
\section{Cephalídeos}
\begin{itemize}
\item {Grp. gram.:m. pl.}
\end{itemize}
Molluscos, que têm cérebro rudimental.
\section{Cephalite}
\begin{itemize}
\item {Grp. gram.:f.}
\end{itemize}
\begin{itemize}
\item {Proveniência:(Do gr. \textunderscore kephale\textunderscore )}
\end{itemize}
Inflammação cerebral.
\section{Cephalóbaros}
\begin{itemize}
\item {Grp. gram.:m. pl.}
\end{itemize}
\begin{itemize}
\item {Proveniência:(Do gr. \textunderscore kephale\textunderscore  + \textunderscore baros\textunderscore )}
\end{itemize}
Insectos coleópteros pentámeros.
\section{Cephalo-cranialgia}
\begin{itemize}
\item {Grp. gram.:f.}
\end{itemize}
\begin{itemize}
\item {Utilização:Med.}
\end{itemize}
Dôr, que abrange o crânio e o cérebro.
\section{Cephalodela}
\begin{itemize}
\item {Grp. gram.:f.}
\end{itemize}
\begin{itemize}
\item {Proveniência:(Do gr. \textunderscore kephale\textunderscore  + \textunderscore delos\textunderscore )}
\end{itemize}
Animálculo, cujo corpo termina por cabeça sem bôca visível.
\section{Cephalodial}
\begin{itemize}
\item {Grp. gram.:adj.}
\end{itemize}
\begin{itemize}
\item {Utilização:Bot.}
\end{itemize}
Diz-se da frutificação de certos líchens.
\section{Cephalodiano}
\begin{itemize}
\item {Grp. gram.:adj.}
\end{itemize}
\begin{itemize}
\item {Utilização:Bot.}
\end{itemize}
\begin{itemize}
\item {Grp. gram.:M. pl.}
\end{itemize}
Que tem cephalódios.
Ordem de líchens, que comprehende os que têm os conceptáculos quási globulosos.
\section{Cephalódio}
\begin{itemize}
\item {Grp. gram.:m.}
\end{itemize}
\begin{itemize}
\item {Utilização:Bot.}
\end{itemize}
Apotheca globulosa dos líchens.
\section{Cephalodontes}
\begin{itemize}
\item {Grp. gram.:m. pl.}
\end{itemize}
\begin{itemize}
\item {Proveniência:(Do gr. \textunderscore kephale\textunderscore  + \textunderscore odous\textunderscore , \textunderscore odontos\textunderscore )}
\end{itemize}
Gênero de insectos coleópteros da América.
\section{Cephalographia}
\begin{itemize}
\item {Grp. gram.:f.}
\end{itemize}
\begin{itemize}
\item {Proveniência:(Do gr. \textunderscore kephale\textunderscore  + \textunderscore graphein\textunderscore )}
\end{itemize}
Descripção anatómica da cabeça.
\section{Cephalográphico}
\begin{itemize}
\item {Grp. gram.:adj.}
\end{itemize}
Relativo á cephalographia.
\section{Cephaloide}
\begin{itemize}
\item {Grp. gram.:adj.}
\end{itemize}
\begin{itemize}
\item {Proveniência:(Do gr. \textunderscore kephale\textunderscore  + \textunderscore eidos\textunderscore )}
\end{itemize}
Que tem fórma de cabeça.
\section{Cephaloídeos}
\begin{itemize}
\item {Grp. gram.:m. pl.}
\end{itemize}
Divisão da fam. dos líchens, caracterizada pela fórma da frutificação.
(Cp. \textunderscore cephaloide\textunderscore )
\section{Cephaloleia}
\begin{itemize}
\item {Grp. gram.:f.}
\end{itemize}
\begin{itemize}
\item {Proveniência:(Do gr. \textunderscore kephale\textunderscore  + \textunderscore leia\textunderscore )}
\end{itemize}
Insecto coleóptero tetrâmero.
\section{Cephalomancia}
\begin{itemize}
\item {Grp. gram.:f.}
\end{itemize}
\begin{itemize}
\item {Proveniência:(Do gr. \textunderscore kephale\textunderscore  + \textunderscore manteia\textunderscore )}
\end{itemize}
Adivinhação, por meio da cabeça de um burro, collocada sôbre o fôgo.
\section{Cephalomântico}
\begin{itemize}
\item {Grp. gram.:adj.}
\end{itemize}
Relativo á cephalomância.
\section{Cephalometria}
\begin{itemize}
\item {Grp. gram.:f.}
\end{itemize}
\begin{itemize}
\item {Proveniência:(De \textunderscore cephalómetro\textunderscore )}
\end{itemize}
Medição da cabeça, no estudo das raças humanas.
\section{Cephalómetro}
\begin{itemize}
\item {Grp. gram.:m.}
\end{itemize}
\begin{itemize}
\item {Proveniência:(Do gr. \textunderscore kephale\textunderscore  + \textunderscore metron\textunderscore )}
\end{itemize}
Instrumento, para medir as dimensões da cabeça.
\section{Cephalóphora}
\begin{itemize}
\item {Grp. gram.:f.}
\end{itemize}
\begin{itemize}
\item {Proveniência:(De \textunderscore cephalóphoro\textunderscore )}
\end{itemize}
Gênero de plantas compostas.
\section{Cephalóphoro}
\begin{itemize}
\item {Grp. gram.:adj.}
\end{itemize}
\begin{itemize}
\item {Utilização:Bot.}
\end{itemize}
\begin{itemize}
\item {Proveniência:(Do gr. \textunderscore kephale\textunderscore  + \textunderscore phoros\textunderscore )}
\end{itemize}
Que tem flôr em fórma de cabeça.
\section{Cephalópodes}
\begin{itemize}
\item {Grp. gram.:m. pl.}
\end{itemize}
\begin{itemize}
\item {Proveniência:(Do gr. \textunderscore kephale\textunderscore  + \textunderscore pous\textunderscore , \textunderscore podos\textunderscore )}
\end{itemize}
Molluscos, que têm os tentáculos á roda da bôca.
\section{Cephalóptero}
\begin{itemize}
\item {Grp. gram.:adj.}
\end{itemize}
\begin{itemize}
\item {Utilização:Zool.}
\end{itemize}
\begin{itemize}
\item {Proveniência:(Do gr. \textunderscore kephale\textunderscore  + \textunderscore pteron\textunderscore )}
\end{itemize}
Que na cabeça tem pennas, que apparentam a fórma de uma asa.
\section{Cephaloscopia}
\begin{itemize}
\item {Grp. gram.:f.}
\end{itemize}
\begin{itemize}
\item {Proveniência:(Do gr. \textunderscore kephale\textunderscore  + \textunderscore skopein\textunderscore )}
\end{itemize}
Exame da cabeça, para se conhecer o estado das faculdades intellectuaes.
\section{Cephalosomo}
\begin{itemize}
\item {fónica:so}
\end{itemize}
\begin{itemize}
\item {Grp. gram.:adj.}
\end{itemize}
\begin{itemize}
\item {Proveniência:(Do gr. \textunderscore kephale\textunderscore  + \textunderscore soma\textunderscore )}
\end{itemize}
Diz-se do peixe que tem o corpo grosso na parte anterior.
\section{Cephalostigma}
\begin{itemize}
\item {Grp. gram.:f.}
\end{itemize}
\begin{itemize}
\item {Proveniência:(Do gr. \textunderscore kephale\textunderscore  + \textunderscore stigma\textunderscore )}
\end{itemize}
Planta campanulácea da Birmânia e de Senegâmbia.
\section{Cephalote}
\begin{itemize}
\item {Grp. gram.:m.  e  adj.}
\end{itemize}
O mesmo que \textunderscore cephaloto\textunderscore .
\section{Cephalotheca}
\begin{itemize}
\item {Grp. gram.:f.}
\end{itemize}
\begin{itemize}
\item {Proveniência:(Do gr. \textunderscore kephale\textunderscore  + \textunderscore theke\textunderscore )}
\end{itemize}
Invólucro da cabeça das chrysálidas.
\section{Cephalothórax}
\begin{itemize}
\item {Grp. gram.:m.}
\end{itemize}
\begin{itemize}
\item {Proveniência:(Do gr. \textunderscore kephale\textunderscore  + \textunderscore thorax\textunderscore )}
\end{itemize}
A cabeça e o thórax de certos insectos.
\section{Cephaloto}
\begin{itemize}
\item {Grp. gram.:m.  e  adj.}
\end{itemize}
\begin{itemize}
\item {Proveniência:(Do gr. \textunderscore kephalotos\textunderscore )}
\end{itemize}
Nome de varios peixes, morcegos e insectos, que têm cabeça grande.
\section{Cephalotomia}
\begin{itemize}
\item {Grp. gram.:f.}
\end{itemize}
Operação, com que se parte a cabeça de um féto, para facilitar a saída da bacia.
(Cp. \textunderscore cephalótomo\textunderscore )
\section{Cephalótomo}
\begin{itemize}
\item {Grp. gram.:m.}
\end{itemize}
\begin{itemize}
\item {Proveniência:(Do gr. \textunderscore kephale\textunderscore  + \textunderscore tome\textunderscore )}
\end{itemize}
Instrumento, próprio para a cephalotomia.
\section{Cephalótribo}
\begin{itemize}
\item {Grp. gram.:m.}
\end{itemize}
\begin{itemize}
\item {Proveniência:(Do gr. \textunderscore kephale\textunderscore  + \textunderscore tribein\textunderscore )}
\end{itemize}
Instrumento, para esmagar a cabeça do féto e facilitar-lhe a saída da bacia da parturiente.
\section{Cephélia}
\begin{itemize}
\item {Grp. gram.:f.}
\end{itemize}
\begin{itemize}
\item {Proveniência:(Do gr. \textunderscore kephele\textunderscore )}
\end{itemize}
Planta rubiácea da América.
\section{Cepheu}
\begin{itemize}
\item {Grp. gram.:m.}
\end{itemize}
\begin{itemize}
\item {Proveniência:(Do gr. \textunderscore Kepheus\textunderscore , n. p.)}
\end{itemize}
Constellação setentrional.
\section{Cephíseo}
\begin{itemize}
\item {Grp. gram.:adj.}
\end{itemize}
Relativo a Cephiso. Cf. \textunderscore Lusíadas\textunderscore , IX, 60.
\section{Cepilhar}
\textunderscore v. t.\textunderscore  (e der.)
(V. \textunderscore acepilhar\textunderscore , etc.)
\section{Cepilho}
\begin{itemize}
\item {Grp. gram.:m.}
\end{itemize}
\begin{itemize}
\item {Proveniência:(De \textunderscore cepo\textunderscore )}
\end{itemize}
Plaina, com que os carpinteiros alisam a madeira.
Lima de espingardeiro.
Parte anterior e elevada da sella.
\section{Cepipa}
\begin{itemize}
\item {Grp. gram.:f.}
\end{itemize}
Fécula da mandioca. Cf. \textunderscore Pharmac. Port.\textunderscore 
\section{Cepirrão}
\begin{itemize}
\item {Grp. gram.:m.}
\end{itemize}
\begin{itemize}
\item {Utilização:Agr.}
\end{itemize}
Rebento ou ladrão no pé da cepa. (Colhido na Bairrada)
\section{Cepo}
\begin{itemize}
\item {fónica:cê}
\end{itemize}
\begin{itemize}
\item {Grp. gram.:m.}
\end{itemize}
\begin{itemize}
\item {Utilização:Fig.}
\end{itemize}
\begin{itemize}
\item {Utilização:Fig.}
\end{itemize}
\begin{itemize}
\item {Utilização:Ant.}
\end{itemize}
\begin{itemize}
\item {Utilização:Carp.}
\end{itemize}
\begin{itemize}
\item {Utilização:Carp.}
\end{itemize}
\begin{itemize}
\item {Utilização:Carp.}
\end{itemize}
\begin{itemize}
\item {Utilização:Carp.}
\end{itemize}
\begin{itemize}
\item {Proveniência:(Do lat. \textunderscore cippus\textunderscore )}
\end{itemize}
Tôro, pedaço de um tôro, cortado transversalmente.
Parte inferior de uma árvore, incluidas as raízes.
Pedaço de madeira ou trambolho, que se prende ás pernas dos animaes, para não fugirem.
Armadilha para caçar.
Pessôa indolente.
Instrumento análogo á plaina, com o rasto convexo ou côncavo, segundo é destinado a formar cordões salientes ou a abrir meias canas.
Grossa prancha de madeira, nos pianos que não são armados em ferro, e na qual estão embutidas as cravelhas.
Parte inferior do braço dos instrumentos de corda, que se liga á caixa de resonância.
A parte do arado, que entra na terra, isto é, a relha, as orelhas e o teiró.
Pessôa, que anda com difficuldade ou que quási se não póde mover.
O mesmo que \textunderscore tronco\textunderscore ^1 (de família).
\textunderscore Cêpo de corôa\textunderscore , utensílio, com que se moldam caixilhos e que tem a fórma de um quarto de círculo entre dois filetes.
\textunderscore Cepo maroto\textunderscore , ferramenta, que produz um moldado semelhante ao \textunderscore cepo da corôa\textunderscore .
\textunderscore Cepo de colla\textunderscore , ferramenta, com que se fazem os ganzepes de algumas portas.
\textunderscore Cepo de gula\textunderscore , ferramenta, que faz a moldura chamada gula.
\section{Cépola}
\begin{itemize}
\item {Grp. gram.:f.}
\end{itemize}
Espécie de peixe, cuja carne se separa em fórma de fôlhas.
\section{Ceptro}
\begin{itemize}
\item {Grp. gram.:m.}
\end{itemize}
\begin{itemize}
\item {Utilização:Fig.}
\end{itemize}
\begin{itemize}
\item {Proveniência:(Lat. \textunderscore sceptrum\textunderscore )}
\end{itemize}
O mesmo ou melhor que \textunderscore sceptro\textunderscore . Cf. Usque, \textunderscore Tribulações\textunderscore , 27 v.^o
Bastão, que antigamente designava autoridade real.
Pequeno bastão, encimado por uma flôr, uma esphera ou outro qualquer ornato, usado antigamente pelos Consules e Imperadores romanos e modernamente pelos Soberanos da Europa.
Autoridade soberana.
Poder real.
O rei.
Preeminência.
Despotismo.
\section{Cepudo}
\begin{itemize}
\item {Grp. gram.:adj.}
\end{itemize}
Grosso, mal feito:«\textunderscore pernas cepudas\textunderscore ». Camillo, \textunderscore Eusébio\textunderscore .
\section{Ceque}
\begin{itemize}
\item {Grp. gram.:m.}
\end{itemize}
Ave africana, (\textunderscore crateropus kartlaubi\textunderscore , Bocage).
\section{Cequim}
\begin{itemize}
\item {Grp. gram.:m.}
\end{itemize}
\begin{itemize}
\item {Proveniência:(Do it. \textunderscore zecchino\textunderscore )}
\end{itemize}
Antiga moéda de oiro, italiana, que valia proximamente dois mil reis.
Pequeno disco de metal amarelo, que serve para enfeites, em vestuário de ciganas.
\section{Cér}
\begin{itemize}
\item {Grp. gram.:m.}
\end{itemize}
\begin{itemize}
\item {Proveniência:(Do indost. \textunderscore cer\textunderscore )}
\end{itemize}
Antigo pêso indiano.
\section{Cera}
\begin{itemize}
\item {fónica:cê}
\end{itemize}
\begin{itemize}
\item {Grp. gram.:f.}
\end{itemize}
\begin{itemize}
\item {Utilização:Fig.}
\end{itemize}
\begin{itemize}
\item {Utilização:Ant.}
\end{itemize}
\begin{itemize}
\item {Utilização:Zool.}
\end{itemize}
\begin{itemize}
\item {Proveniência:(Lat. \textunderscore cera\textunderscore )}
\end{itemize}
Substância, que as abelhas produzem, e com que ellas fabricam os favos.
Substância vegetal, semelhante á cera das abelhas.
Velas de cera, brandões, tochas.
Humor untuoso, que se fórma nos ouvidos.
Pessôa branda, indolente.
Coisa branda.
Carácter froixo, muito dócil.
Trabalho negligente, serviço de mandrião: \textunderscore aquelle faz muita cera\textunderscore .
Pensão annual, o mesmo que \textunderscore cathedrátego\textunderscore .
Membrana molle, que cobre a base da parte superior do bico de algumas aves. Cf. P. Moraes, \textunderscore Zool. Elem.\textunderscore , 277.
\section{Ceráceo}
\begin{itemize}
\item {Grp. gram.:adj.}
\end{itemize}
Que tem o aspecto ou a brandura da cera.
\section{Ceraferário}
\begin{itemize}
\item {Grp. gram.:m.}
\end{itemize}
\begin{itemize}
\item {Utilização:Ant.}
\end{itemize}
O mesmo que \textunderscore ceroferário\textunderscore .
\section{Cerafilocele}
\begin{itemize}
\item {Grp. gram.:m.}
\end{itemize}
\begin{itemize}
\item {Proveniência:(Do gr. \textunderscore keras\textunderscore  + \textunderscore phullon\textunderscore  + \textunderscore kele\textunderscore )}
\end{itemize}
Tumor córneo, entre a parede do casco do cavalo e os tecidos subjacentes.
\section{Cerafiloso}
\begin{itemize}
\item {Grp. gram.:adj.}
\end{itemize}
\begin{itemize}
\item {Proveniência:(Do gr. \textunderscore keras\textunderscore  + \textunderscore phullon\textunderscore )}
\end{itemize}
Diz-se do tecido orgânico da taipa do pé do cavalo. Cf. Leon, \textunderscore Arte de Ferrar\textunderscore , 32.
\section{Cerafro}
\begin{itemize}
\item {Grp. gram.:m.}
\end{itemize}
Insecto hymenóptero.
\section{Cerafrontitos}
\begin{itemize}
\item {Grp. gram.:m. pl.}
\end{itemize}
Grupo de insectos, que têm por typo o \textunderscore cerafro\textunderscore .
\section{Cerambicinos}
\begin{itemize}
\item {Grp. gram.:m. pl.}
\end{itemize}
O mesmo que [[longicórneos|longicórneo]].
\section{Cerame}
\begin{itemize}
\item {Grp. gram.:m.}
\end{itemize}
Pequena habitação asiática e africana, cujo sobrado se firma em quatro troncos de árvores, e cujo tecto é formado de fôlhas de palmeira.
\section{Ceramiárias}
\begin{itemize}
\item {Grp. gram.:f. pl.}
\end{itemize}
Fam. de plantas acotyledóneas, formada á custa das algas.
(Cp. \textunderscore céramo\textunderscore )
\section{Cerâmica}
\begin{itemize}
\item {Grp. gram.:f.}
\end{itemize}
\begin{itemize}
\item {Proveniência:(Do gr. \textunderscore keramos\textunderscore )}
\end{itemize}
Arte de fabricar loiça de barro ou de outra substância congênere; olaria.
\section{Cerâmico}
\begin{itemize}
\item {Grp. gram.:adj.}
\end{itemize}
Relativo á cerâmica.
\section{Cerâmio}
\begin{itemize}
\item {Grp. gram.:m.}
\end{itemize}
Planta submarina, escarlate.
\section{Ceramista}
\begin{itemize}
\item {Grp. gram.:m.}
\end{itemize}
\begin{itemize}
\item {Proveniência:(De \textunderscore céramo\textunderscore )}
\end{itemize}
Pintor de loiça fina de barro.
\section{Céramo}
\begin{itemize}
\item {Grp. gram.:m.}
\end{itemize}
\begin{itemize}
\item {Proveniência:(Gr. \textunderscore keramos\textunderscore , argilla ou vaso de argilla)}
\end{itemize}
Vaso de barro, de que os Gregos se serviam á mesa.
\section{Ceramografia}
\begin{itemize}
\item {Grp. gram.:f.}
\end{itemize}
\begin{itemize}
\item {Proveniência:(Do gr. \textunderscore keramos\textunderscore  + \textunderscore graphein\textunderscore )}
\end{itemize}
Descripção de loiças antigas.
\section{Ceramográfico}
\begin{itemize}
\item {Grp. gram.:adj.}
\end{itemize}
Relativo á ceramografia.
\section{Ceramographia}
\begin{itemize}
\item {Grp. gram.:f.}
\end{itemize}
\begin{itemize}
\item {Proveniência:(Do gr. \textunderscore keramos\textunderscore  + \textunderscore graphein\textunderscore )}
\end{itemize}
Descripção de loiças antigas.
\section{Ceramográphico}
\begin{itemize}
\item {Grp. gram.:adj.}
\end{itemize}
Relativo á ceramographia.
\section{Cerândria}
\begin{itemize}
\item {Grp. gram.:f.}
\end{itemize}
\begin{itemize}
\item {Proveniência:(Do gr. \textunderscore keras\textunderscore  + \textunderscore andria\textunderscore )}
\end{itemize}
Insecto coleóptero heterómero.
\section{Ceraphyllocele}
\begin{itemize}
\item {Grp. gram.:m.}
\end{itemize}
\begin{itemize}
\item {Proveniência:(Do gr. \textunderscore keras\textunderscore  + \textunderscore phullon\textunderscore  + \textunderscore kele\textunderscore )}
\end{itemize}
Tumor córneo, entre a parede do casco do cavallo e os tecidos subjacentes.
\section{Ceraphylloso}
\begin{itemize}
\item {Grp. gram.:adj.}
\end{itemize}
\begin{itemize}
\item {Proveniência:(Do gr. \textunderscore keras\textunderscore  + \textunderscore phullon\textunderscore )}
\end{itemize}
Diz-se do tecido orgânico da taipa do pé do cavallo. Cf. Leon, \textunderscore Arte de Ferrar\textunderscore , 32.
\section{Cerapo}
\begin{itemize}
\item {Grp. gram.:m.}
\end{itemize}
Gênero de crustáceos isópodes.
\section{Cerapodina}
\begin{itemize}
\item {Grp. gram.:f.}
\end{itemize}
Gênero de pequenos crustáceos amphípodes.
\section{Cerar}
\begin{itemize}
\item {Grp. gram.:v. t.}
\end{itemize}
\begin{itemize}
\item {Utilização:Ant.}
\end{itemize}
Fechar (carta ou outro escrito), com cera.
Lacrar.
\section{Cerasina}
\begin{itemize}
\item {Grp. gram.:f.}
\end{itemize}
Substância, extrahida das velas de cera, e que é a osocerite purificada.
\section{Cerasina}
\begin{itemize}
\item {Grp. gram.:f.}
\end{itemize}
\begin{itemize}
\item {Proveniência:(Do lat. \textunderscore cerasus\textunderscore )}
\end{itemize}
Resina da cerejeira e de outras árvores fructíferas.
Antiga bebida, feita com cerejas.
\section{Cerasita}
\begin{itemize}
\item {Grp. gram.:f.}
\end{itemize}
\begin{itemize}
\item {Proveniência:(Do lat. \textunderscore cerasus\textunderscore )}
\end{itemize}
Fóssil, que semelha uma cereja petrificada.
\section{Cerasta}
\begin{itemize}
\item {Grp. gram.:f.}
\end{itemize}
\begin{itemize}
\item {Proveniência:(Gr. \textunderscore kerastes\textunderscore )}
\end{itemize}
Vibora, que tem na cabeça duas protuberâncias escamosas.
\section{Cerástio}
\begin{itemize}
\item {Grp. gram.:m.}
\end{itemize}
\begin{itemize}
\item {Proveniência:(Do gr. \textunderscore keras\textunderscore )}
\end{itemize}
Gênero de plantas portuláceas.
\section{Ceratandra}
\begin{itemize}
\item {Grp. gram.:f.}
\end{itemize}
\begin{itemize}
\item {Proveniência:(Do gr. \textunderscore keras\textunderscore  + \textunderscore aner\textunderscore , \textunderscore andros\textunderscore )}
\end{itemize}
Planta do Cabo da Bôa-Esperança, da fam. das orchídeas.
\section{Ceratectomia}
\begin{itemize}
\item {Grp. gram.:f.}
\end{itemize}
Operação cirúrgica de uma pupílla artificial, por excisão de uma parte da córnea.
\section{Ceratina}
\begin{itemize}
\item {Grp. gram.:adj. f.}
\end{itemize}
\begin{itemize}
\item {Proveniência:(Do gr. \textunderscore keras\textunderscore , \textunderscore keratos\textunderscore )}
\end{itemize}
Dizia-se, na Escolástica, de uma questão capciosa ou sophística.
Substância orgânica, que se encontra nos cornos, nas unhas, etc.
\section{Ceratite}
\begin{itemize}
\item {Grp. gram.:f.}
\end{itemize}
\begin{itemize}
\item {Utilização:Med.}
\end{itemize}
\begin{itemize}
\item {Proveniência:(Do gr. \textunderscore keras\textunderscore )}
\end{itemize}
Inflammação da córnea.
\section{Cerato}
\begin{itemize}
\item {Grp. gram.:m.}
\end{itemize}
\begin{itemize}
\item {Proveniência:(Lat. \textunderscore ceratum\textunderscore )}
\end{itemize}
Medicamento, em que entra principalmente a cera e um óleo.
\section{Ceratocarpo}
\begin{itemize}
\item {Grp. gram.:adj.}
\end{itemize}
\begin{itemize}
\item {Utilização:Bot.}
\end{itemize}
\begin{itemize}
\item {Proveniência:(Do gr. \textunderscore keras\textunderscore  + \textunderscore karpos\textunderscore )}
\end{itemize}
Cujo fruto é semelhante a um corno.
\section{Ceratocéfalo}
\begin{itemize}
\item {Grp. gram.:m.}
\end{itemize}
\begin{itemize}
\item {Proveniência:(Do gr. \textunderscore keras\textunderscore  + \textunderscore kephale\textunderscore )}
\end{itemize}
Gênero de plantas ranunculáceas.
\section{Ceratocéphalo}
\begin{itemize}
\item {Grp. gram.:m.}
\end{itemize}
\begin{itemize}
\item {Proveniência:(Do gr. \textunderscore keras\textunderscore  + \textunderscore kephale\textunderscore )}
\end{itemize}
Gênero de plantas ranunculáceas.
\section{Ceratocone}
\begin{itemize}
\item {Grp. gram.:m.}
\end{itemize}
\begin{itemize}
\item {Utilização:Anat.}
\end{itemize}
\begin{itemize}
\item {Proveniência:(Do gr. \textunderscore keras\textunderscore  + \textunderscore konos\textunderscore )}
\end{itemize}
Córnea cónica.
\section{Ceratodo}
\begin{itemize}
\item {Grp. gram.:m.}
\end{itemize}
Gênero de peixes.
\section{Ceratofíleas}
\begin{itemize}
\item {Grp. gram.:f. pl.}
\end{itemize}
Fam. de plantas, a que serve de typo o ceratofilo.
\section{Ceratofilo}
\begin{itemize}
\item {Grp. gram.:m.}
\end{itemize}
\begin{itemize}
\item {Proveniência:(Do gr. \textunderscore keras\textunderscore  + \textunderscore phullon\textunderscore )}
\end{itemize}
Gênero de plantas medicinaes.
\section{Ceratoglobo}
\begin{itemize}
\item {Grp. gram.:m.}
\end{itemize}
\begin{itemize}
\item {Utilização:Anat.}
\end{itemize}
\begin{itemize}
\item {Proveniência:(Do gr. \textunderscore keras\textunderscore  + lat. \textunderscore globus\textunderscore )}
\end{itemize}
Córnea globosa.
\section{Ceratoglosso}
\begin{itemize}
\item {Grp. gram.:adj.}
\end{itemize}
\begin{itemize}
\item {Utilização:Anat.}
\end{itemize}
\begin{itemize}
\item {Proveniência:(Do gr. \textunderscore keras\textunderscore  + \textunderscore glossa\textunderscore )}
\end{itemize}
Relativo á lingua e á ponta do osso hyoide.
\section{Ceratohial}
\begin{itemize}
\item {Grp. gram.:m.}
\end{itemize}
\begin{itemize}
\item {Utilização:Anat.}
\end{itemize}
\begin{itemize}
\item {Proveniência:(Do gr. \textunderscore keras\textunderscore , \textunderscore keratos\textunderscore , e \textunderscore hyal\textunderscore )}
\end{itemize}
Peça média do meio arco hioídeo.
\section{Ceratohyal}
\begin{itemize}
\item {Grp. gram.:m.}
\end{itemize}
\begin{itemize}
\item {Utilização:Anat.}
\end{itemize}
\begin{itemize}
\item {Proveniência:(Do gr. \textunderscore keras\textunderscore , \textunderscore keratos\textunderscore , e \textunderscore hyal\textunderscore )}
\end{itemize}
Peça média do meio arco hyoídeo.
\section{Ceratolenos}
\begin{itemize}
\item {Grp. gram.:m. pl.}
\end{itemize}
\begin{itemize}
\item {Proveniência:(Do gr. \textunderscore keras\textunderscore  + \textunderscore olene\textunderscore )}
\end{itemize}
Família de animaes acéphalos, que têm braços articulados e próximos da bôca.
\section{Ceratólitho}
\begin{itemize}
\item {Grp. gram.:m.}
\end{itemize}
\begin{itemize}
\item {Proveniência:(Do gr. \textunderscore keras\textunderscore  + \textunderscore lithos\textunderscore )}
\end{itemize}
Corno petrificado.
\section{Ceratólito}
\begin{itemize}
\item {Grp. gram.:m.}
\end{itemize}
\begin{itemize}
\item {Proveniência:(Do gr. \textunderscore keras\textunderscore  + \textunderscore lithos\textunderscore )}
\end{itemize}
Corno petrificado.
\section{Ceratomalacia}
\begin{itemize}
\item {Grp. gram.:f.}
\end{itemize}
\begin{itemize}
\item {Utilização:Med.}
\end{itemize}
\begin{itemize}
\item {Proveniência:(Do gr. \textunderscore keras\textunderscore  + \textunderscore malakos\textunderscore )}
\end{itemize}
Amollecimento da córnea.
\section{Ceratónia}
\begin{itemize}
\item {Grp. gram.:f.}
\end{itemize}
\begin{itemize}
\item {Proveniência:(Do gr. \textunderscore keras\textunderscore , \textunderscore keratos\textunderscore )}
\end{itemize}
Nome scientífico da alfarrobeira.
\section{Ceratoniáceas}
\begin{itemize}
\item {Grp. gram.:f. pl.}
\end{itemize}
Família de plantas, que têm por typo a ceratónia, e que alguns botânicos incluem nas papilionáceas.
\section{Ceratopétalo}
\begin{itemize}
\item {Grp. gram.:adj.}
\end{itemize}
\begin{itemize}
\item {Utilização:Bot.}
\end{itemize}
\begin{itemize}
\item {Proveniência:(Do gr. \textunderscore keras\textunderscore  + \textunderscore petalon\textunderscore )}
\end{itemize}
Que tem as pétalas em fórma de corno.
\section{Ceratophýlleas}
\begin{itemize}
\item {Grp. gram.:f. pl.}
\end{itemize}
Fam. de plantas, a que serve de typo o ceratophyllo.
\section{Ceratophyllo}
\begin{itemize}
\item {Grp. gram.:m.}
\end{itemize}
\begin{itemize}
\item {Proveniência:(Do gr. \textunderscore keras\textunderscore  + \textunderscore phullon\textunderscore )}
\end{itemize}
Gênero de plantas medicinaes.
\section{Ceratoplato}
\begin{itemize}
\item {Grp. gram.:m.}
\end{itemize}
Insecto díptero.
\section{Ceratostigma}
\begin{itemize}
\item {Grp. gram.:m.}
\end{itemize}
\begin{itemize}
\item {Proveniência:(Do gr. \textunderscore keras\textunderscore  + \textunderscore stigma\textunderscore )}
\end{itemize}
Planta plumbagínea da China.
\section{Ceratoscópio}
\begin{itemize}
\item {Grp. gram.:m.}
\end{itemize}
\begin{itemize}
\item {Proveniência:(Do gr. \textunderscore keras\textunderscore  + \textunderscore skopein\textunderscore )}
\end{itemize}
Instrumento, para observar a curvatura da córnea.
\section{Ceratoteca}
\begin{itemize}
\item {Grp. gram.:f.}
\end{itemize}
\begin{itemize}
\item {Proveniência:(Do gr. \textunderscore keras\textunderscore  + \textunderscore theke\textunderscore )}
\end{itemize}
Invólucro das antenas das crisálidas.
\section{Ceratotheca}
\begin{itemize}
\item {Grp. gram.:f.}
\end{itemize}
\begin{itemize}
\item {Proveniência:(Do gr. \textunderscore keras\textunderscore  + \textunderscore theke\textunderscore )}
\end{itemize}
Invólucro das antennas das chrysálidas.
\section{Ceratotomia}
\begin{itemize}
\item {Grp. gram.:f.}
\end{itemize}
Incisão da córnea transparente.
(Cp. \textunderscore ceratótomo\textunderscore )
\section{Ceratotómico}
\begin{itemize}
\item {Grp. gram.:adj.}
\end{itemize}
Relativo a ceratotomia.
\section{Ceratótomo}
\begin{itemize}
\item {Grp. gram.:m.}
\end{itemize}
\begin{itemize}
\item {Proveniência:(Do gr. \textunderscore keras\textunderscore  + \textunderscore tome\textunderscore )}
\end{itemize}
Espécie de escalpêlo, com que se faz a incisão da córnea transparente, na operação da cataracta.
\section{Ceraulofone}
\begin{itemize}
\item {Grp. gram.:m.}
\end{itemize}
\begin{itemize}
\item {Proveniência:(Do gr. \textunderscore keras\textunderscore , + \textunderscore aulos\textunderscore  + \textunderscore phone\textunderscore )}
\end{itemize}
Registo de órgão, de tubos de zinco.
\section{Ceraulofónio}
\begin{itemize}
\item {Grp. gram.:m.}
\end{itemize}
\begin{itemize}
\item {Proveniência:(Do gr. \textunderscore keras\textunderscore , + \textunderscore aulos\textunderscore  + \textunderscore phone\textunderscore )}
\end{itemize}
Registo de órgão, de tubos de zinco.
\section{Ceraulophone}
\begin{itemize}
\item {Grp. gram.:m.}
\end{itemize}
\begin{itemize}
\item {Proveniência:(Do gr. \textunderscore keras\textunderscore , + \textunderscore aulos\textunderscore  + \textunderscore phone\textunderscore )}
\end{itemize}
Registo de órgão, de tubos de zinco.
\section{Ceráunia}
\begin{itemize}
\item {Grp. gram.:f.}
\end{itemize}
\begin{itemize}
\item {Proveniência:(Lat. \textunderscore ceraunia\textunderscore )}
\end{itemize}
Raio.
Pedra preciosa, que se julgava ter caído com o raio.
\section{Ceráunio}
\begin{itemize}
\item {Grp. gram.:m.}
\end{itemize}
\begin{itemize}
\item {Proveniência:(Lat. \textunderscore ceraunius\textunderscore )}
\end{itemize}
Antiga sigla paleográphica, com que se marcavam os versos defeituosos.
\section{Ceraunita}
\begin{itemize}
\item {Grp. gram.:f.}
\end{itemize}
\begin{itemize}
\item {Proveniência:(De \textunderscore ceráunia\textunderscore )}
\end{itemize}
Pedra meteórica.
Ceráunia.
Fulgorite.
\section{Ceraunite}
\begin{itemize}
\item {Grp. gram.:f.}
\end{itemize}
(V.ceraunita)
\section{Ceraunómetro}
\begin{itemize}
\item {Grp. gram.:m.}
\end{itemize}
\begin{itemize}
\item {Proveniência:(Do gr. \textunderscore keraunos\textunderscore  + \textunderscore metron\textunderscore )}
\end{itemize}
Instrumento de Phýsica, para medir a faísca eléctrica.
\section{Ceraunoscopia}
\begin{itemize}
\item {Grp. gram.:f.}
\end{itemize}
\begin{itemize}
\item {Proveniência:(Do gr. \textunderscore keraunos\textunderscore  + \textunderscore skopein\textunderscore )}
\end{itemize}
Supposta arte de adivinhar, por meio dos phenómenos do raio.
\section{Cerbera}
\begin{itemize}
\item {Grp. gram.:f.}
\end{itemize}
Nome scientífico de uma planta mexicana.
\section{Cerberina}
\begin{itemize}
\item {Grp. gram.:f.}
\end{itemize}
Substância açucarada e venenosa, extraída de cerbera.
\section{Cerberite}
\begin{itemize}
\item {Grp. gram.:f.}
\end{itemize}
Novo explosivo, com os mesmos elementos da dynamite e mais dois óleos mineraes, adicionados á nitro-glycerina.
\section{Cérbero}
\begin{itemize}
\item {Grp. gram.:m.}
\end{itemize}
\begin{itemize}
\item {Proveniência:(Lat. \textunderscore cerberus\textunderscore )}
\end{itemize}
Porteiro ou guarda intratável, brutal.
Constellação septentrional.
\section{Cêrca}
\begin{itemize}
\item {Grp. gram.:f.}
\end{itemize}
\begin{itemize}
\item {Proveniência:(De \textunderscore cercar\textunderscore )}
\end{itemize}
Obra, com que se rodeia ou fecha um terreno.
Terreno fechado por muro, sebe ou vallado.
Quinta ou quintal murado.
\section{Cêrca}
\begin{itemize}
\item {Grp. gram.:Loc. prop.}
\end{itemize}
\begin{itemize}
\item {Proveniência:(Lat. \textunderscore circa\textunderscore )}
\end{itemize}
\textunderscore prep.\textunderscore , (geralmente seguida de \textunderscore de\textunderscore ), e \textunderscore adv.\textunderscore 
Quási; perto, próximo: \textunderscore gastou cêrca de um conto de reis\textunderscore .
\textunderscore Á cêrca de\textunderscore , a respeito de, relativamente a: \textunderscore dissertação á cêrca dos terremotos\textunderscore .
\section{Cercadeira}
\begin{itemize}
\item {Grp. gram.:f.}
\end{itemize}
\begin{itemize}
\item {Proveniência:(De \textunderscore cercar\textunderscore )}
\end{itemize}
Maquinismo empregado em construcções hydráulicas.
\section{Cercado}
\begin{itemize}
\item {Grp. gram.:m.}
\end{itemize}
\begin{itemize}
\item {Proveniência:(De \textunderscore cercar\textunderscore )}
\end{itemize}
Terreno, que tem cêrca, que é murado ou tapado com sebes, estacaria, etc.
\section{Cercador}
\begin{itemize}
\item {Grp. gram.:m.}
\end{itemize}
Aquelle que cerca.
\section{Cercadura}
\begin{itemize}
\item {Grp. gram.:f.}
\end{itemize}
\begin{itemize}
\item {Proveniência:(De \textunderscore cercar\textunderscore )}
\end{itemize}
Orla.
Guarnição na orla.
\section{Cercal}
\begin{itemize}
\item {Grp. gram.:m.}
\end{itemize}
\begin{itemize}
\item {Proveniência:(Lat. hyp. \textunderscore quercalis\textunderscore , do lat. \textunderscore quercus\textunderscore )}
\end{itemize}
Mata de carvalhos cerquinhos.
\section{Cercaleiro}
\begin{itemize}
\item {Grp. gram.:m.}
\end{itemize}
\begin{itemize}
\item {Utilização:T. da Bairrada}
\end{itemize}
\begin{itemize}
\item {Proveniência:(De \textunderscore Cercal\textunderscore , n. p. top.)}
\end{itemize}
Homem grosseiro, boçal, porcalhão.
\section{Cercamento}
\begin{itemize}
\item {Grp. gram.:m.}
\end{itemize}
\begin{itemize}
\item {Utilização:Ant.}
\end{itemize}
\begin{itemize}
\item {Proveniência:(De \textunderscore cercar\textunderscore )}
\end{itemize}
Colgadura.
\section{Cercania}
\begin{itemize}
\item {Grp. gram.:f.}
\end{itemize}
\begin{itemize}
\item {Proveniência:(T. cast.)}
\end{itemize}
Vizinhança.
Arredores; proximidade; aros.
\section{Cercanias}
\begin{itemize}
\item {Grp. gram.:f. pl.}
\end{itemize}
\begin{itemize}
\item {Proveniência:(T. cast.)}
\end{itemize}
Vizinhança.
Arredores; proximidade; aros.
\section{Cercante}
\begin{itemize}
\item {Grp. gram.:m.  e  adj.}
\end{itemize}
\begin{itemize}
\item {Proveniência:(De \textunderscore cercar\textunderscore )}
\end{itemize}
O que cerca.
\section{Cercão}
\begin{itemize}
\item {Grp. gram.:adj.}
\end{itemize}
Que é das cercanias; vizinho, próximo.
(Cast. \textunderscore cercano\textunderscore )
\section{Cercar}
\begin{itemize}
\item {Grp. gram.:v. t.}
\end{itemize}
\begin{itemize}
\item {Proveniência:(Do lat. \textunderscore circare\textunderscore )}
\end{itemize}
Fazer cêrca a; fechar com muro, sebe, etc.
Pôr cêrco militar a: \textunderscore cercar o inimigo\textunderscore .
Abranger, rodear.
Estender uma coisa em volta de.
Apertar, constranger.
\section{Cercariados}
\begin{itemize}
\item {Grp. gram.:m. pl.}
\end{itemize}
Família de infusórios, que têm por typo o cercário.
\section{Cercário}
\begin{itemize}
\item {Grp. gram.:m.}
\end{itemize}
\begin{itemize}
\item {Proveniência:(Do gr. \textunderscore kerkos\textunderscore )}
\end{itemize}
Infusório dos pântanos.
\section{Cerce}
\begin{itemize}
\item {Grp. gram.:adv.}
\end{itemize}
\begin{itemize}
\item {Grp. gram.:Adj.}
\end{itemize}
\begin{itemize}
\item {Utilização:Náut.}
\end{itemize}
\begin{itemize}
\item {Proveniência:(De \textunderscore cercear\textunderscore )}
\end{itemize}
Pela raiz, pela parte mais inferior.
Diz-se da prôa, quando talhada a pique.
\section{Cércea}
\begin{itemize}
\item {Grp. gram.:f.}
\end{itemize}
Chapa, que se empregava na verificação das bôcas de fogo.
Apparelho nas estações de caminhos de ferro, para determinar o máximo volume que póde attingir a carga de um combóio.
Molde para o córte das pedras.
Curva, recortada em madeira, para auxiliar o desenho.
(Fem. de \textunderscore cérceo\textunderscore )
\section{Cerceador}
\begin{itemize}
\item {Grp. gram.:m.}
\end{itemize}
O que cerceia.
\section{Cerceadura}
\begin{itemize}
\item {Grp. gram.:f.}
\end{itemize}
O mesmo que \textunderscore cerceamento\textunderscore .
\section{Cerceal}
\begin{itemize}
\item {Grp. gram.:m.}
\end{itemize}
Casta de uva branca.
Vinho privativo da Madeira.
Azeitona madural ou negral.
\section{Cerceamente}
\begin{itemize}
\item {Grp. gram.:adv.}
\end{itemize}
De modo cérceo, cerce.
\section{Cerceamento}
\begin{itemize}
\item {Grp. gram.:m.}
\end{itemize}
Acto ou effeito de cercear.
\section{Cercear}
\begin{itemize}
\item {Grp. gram.:v. t.}
\end{itemize}
\begin{itemize}
\item {Proveniência:(Do lat. \textunderscore circinare\textunderscore )}
\end{itemize}
Aparar, cortar em roda.
Cortar pela raiz, pela base.
Deminuir: \textunderscore cercear despesas\textunderscore .
Cortar, desfazer.
\section{Cercefi}
\begin{itemize}
\item {Grp. gram.:m.}
\end{itemize}
Planta, cultivada pelos jardineiros, (\textunderscore tragopogon porrifolíum\textunderscore , Lin.).
(Do norm.)
\section{Cerceio}
\begin{itemize}
\item {Grp. gram.:m.}
\end{itemize}
Acção de cercear.
\section{Cérceo}
\begin{itemize}
\item {Grp. gram.:adj.}
\end{itemize}
\begin{itemize}
\item {Proveniência:(De \textunderscore cercear\textunderscore )}
\end{itemize}
Cortado pela raiz, pela base.
\section{Cerceta}
\begin{itemize}
\item {fónica:cê}
\end{itemize}
\begin{itemize}
\item {Grp. gram.:f.}
\end{itemize}
\begin{itemize}
\item {Proveniência:(Do lat. \textunderscore querquedula\textunderscore )}
\end{itemize}
Ave palmipede.
\section{Cercilhar}
\begin{itemize}
\item {Grp. gram.:v. t.}
\end{itemize}
Abrir cercilho em.
\section{Cercilho}
\begin{itemize}
\item {Grp. gram.:m.}
\end{itemize}
\begin{itemize}
\item {Proveniência:(De \textunderscore cerce\textunderscore )}
\end{itemize}
Corôa, tonsura larga e redonda, de que usavam frades.
Extremidades ásperas do pergaminho.
\section{Cercílio}
\begin{itemize}
\item {Grp. gram.:m.}
\end{itemize}
(V.cercilho)
\section{Cêrco}
\begin{itemize}
\item {Grp. gram.:m.}
\end{itemize}
\begin{itemize}
\item {Utilização:Ant.}
\end{itemize}
\begin{itemize}
\item {Proveniência:(Lat. \textunderscore circus\textunderscore )}
\end{itemize}
Acto de cercar.
Coisa ou coisas, que se collocam em volta.
Roda, círculo.
Assédio militar: \textunderscore durante o cêrco do Pôrto\textunderscore .
Lugar cercado.
Fileira de caçadores, que formam circulo, para colhêr a caça.
Circuito, circo.
Acto de apontar numa carta contra as outras três, no jôgo do monte.
Procissão, que, depois da Páscoa, se fazia, percorrendo os limites da paróchia.
\section{Cercódea}
\begin{itemize}
\item {Grp. gram.:f.}
\end{itemize}
\begin{itemize}
\item {Proveniência:(Do gr. \textunderscore kerkos\textunderscore  + \textunderscore eidos\textunderscore )}
\end{itemize}
Planta da Nova-Zelândia.
\section{Cercomónada}
\begin{itemize}
\item {Grp. gram.:m.}
\end{itemize}
\begin{itemize}
\item {Proveniência:(Do gr. \textunderscore kerkos\textunderscore  + \textunderscore monas\textunderscore )}
\end{itemize}
Infusório, da fam. dos mónadas.
\section{Cercope}
\begin{itemize}
\item {Grp. gram.:m.}
\end{itemize}
\begin{itemize}
\item {Proveniência:(Do gr. \textunderscore kerkope\textunderscore )}
\end{itemize}
Insecto hemíptero.
\section{Cercopiteco}
\begin{itemize}
\item {Grp. gram.:m.}
\end{itemize}
\begin{itemize}
\item {Proveniência:(Do gr. \textunderscore kerkos\textunderscore  + \textunderscore pithekos\textunderscore )}
\end{itemize}
Espécie de macaco de longa cauda.
\section{Cercopitheco}
\begin{itemize}
\item {Grp. gram.:m.}
\end{itemize}
\begin{itemize}
\item {Proveniência:(Do gr. \textunderscore kerkos\textunderscore  + \textunderscore pithekos\textunderscore )}
\end{itemize}
Espécie de macaco de longa cauda.
\section{Cercostilo}
\begin{itemize}
\item {Grp. gram.:m.}
\end{itemize}
\begin{itemize}
\item {Proveniência:(Do gr. \textunderscore kerkos\textunderscore  + \textunderscore stule\textunderscore )}
\end{itemize}
Planta do Brasil, da fam. das compostas.
\section{Cercostylo}
\begin{itemize}
\item {Grp. gram.:m.}
\end{itemize}
\begin{itemize}
\item {Proveniência:(Do gr. \textunderscore kerkos\textunderscore  + \textunderscore stule\textunderscore )}
\end{itemize}
Planta do Brasil, da fam. das compostas.
\section{Cerdana}
\begin{itemize}
\item {Grp. gram.:f.}
\end{itemize}
Espécie de abrunheiro do Peru.
(Cast. \textunderscore cerdana\textunderscore )
\section{Cerdão}
\begin{itemize}
\item {Grp. gram.:m.}
\end{itemize}
\begin{itemize}
\item {Utilização:Bras}
\end{itemize}
Espécie de cotovia.
\section{Cerdas}
\begin{itemize}
\item {Grp. gram.:f. pl.}
\end{itemize}
Sedas de javali e de outros animaes.
(Cast. \textunderscore cerda\textunderscore )
\section{Cerdeira}
\begin{itemize}
\item {Grp. gram.:f.}
\end{itemize}
\begin{itemize}
\item {Utilização:Prov.}
\end{itemize}
O mesmo que \textunderscore cerejeira\textunderscore .
\section{Cerdeiro}
\begin{itemize}
\item {Grp. gram.:m.}
\end{itemize}
\begin{itemize}
\item {Utilização:Prov.}
\end{itemize}
\begin{itemize}
\item {Utilização:trasm.}
\end{itemize}
O mesmo que \textunderscore cerejeira\textunderscore .
\section{Cerdo}
\begin{itemize}
\item {Grp. gram.:m.}
\end{itemize}
Porco.
(Cast. \textunderscore cerdo\textunderscore )
\section{Cerdoeira}
\begin{itemize}
\item {Grp. gram.:f.}
\end{itemize}
O mesmo que \textunderscore cerdoeiro\textunderscore .
\section{Cerdoeiro}
\begin{itemize}
\item {Grp. gram.:m.}
\end{itemize}
\begin{itemize}
\item {Utilização:Prov.}
\end{itemize}
\begin{itemize}
\item {Utilização:minh.}
\end{itemize}
\begin{itemize}
\item {Proveniência:(De \textunderscore cerdo\textunderscore )}
\end{itemize}
Quintal murado.
\section{Cerdorística}
\begin{itemize}
\item {Grp. gram.:f.}
\end{itemize}
\begin{itemize}
\item {Utilização:Des.}
\end{itemize}
\begin{itemize}
\item {Proveniência:(Do gr. \textunderscore kerdos\textunderscore  + \textunderscore orizo\textunderscore )}
\end{itemize}
Sciência, que ensina a calcular os lucros e perdas de uma empresa.
\section{Cerdoso}
\begin{itemize}
\item {Grp. gram.:adj.}
\end{itemize}
Que tem cerdas.
Áspero como as cerdas.
\section{Cereal}
\begin{itemize}
\item {Grp. gram.:adj.}
\end{itemize}
\begin{itemize}
\item {Grp. gram.:M. pl.}
\end{itemize}
\begin{itemize}
\item {Proveniência:(Lat. \textunderscore cerealis\textunderscore , de \textunderscore Ceres\textunderscore , n. p.)}
\end{itemize}
Que produz pão.
Relativo a pão:«\textunderscore por entre loiros chãos de ceral cultura.\textunderscore »Castilho, \textunderscore Geórgicas\textunderscore .
Searas, messes.
Fruto das searas: \textunderscore moer cereaes\textunderscore .
\section{Cereal}
\begin{itemize}
\item {Grp. gram.:m.}
\end{itemize}
(V.cirial)
\section{Cerealífero}
\begin{itemize}
\item {Grp. gram.:adj.}
\end{itemize}
\begin{itemize}
\item {Proveniência:(Do lat. \textunderscore cerealis\textunderscore  + \textunderscore ferre\textunderscore )}
\end{itemize}
Relativo a cereaes.
Que produz cereaes.
\section{Cerealina}
\begin{itemize}
\item {Grp. gram.:f.}
\end{itemize}
\begin{itemize}
\item {Proveniência:(De \textunderscore cereal\textunderscore )}
\end{itemize}
Alcaloide, que reside na entrecasca do trigo e é nocivo ao pão. Cf. Inquér. Indust., P. II, l, 3.^o, 241.
\section{Cerebelar}
\begin{itemize}
\item {Grp. gram.:adj.}
\end{itemize}
Relativo ao cerebelo.
\section{Cerebilite}
\begin{itemize}
\item {Grp. gram.:f.}
\end{itemize}
Inflamação do cerebelo.
\section{Cerebellar}
\begin{itemize}
\item {Grp. gram.:adj.}
\end{itemize}
Relativo ao cerebello.
\section{Cerebillite}
\begin{itemize}
\item {Grp. gram.:f.}
\end{itemize}
Inflammação do cerebello.
\section{Cerebello}
\begin{itemize}
\item {Grp. gram.:m.}
\end{itemize}
\begin{itemize}
\item {Proveniência:(Lat. \textunderscore cerebellum\textunderscore )}
\end{itemize}
Parte posterior de encéphalo.
\section{Cerebelloso}
\begin{itemize}
\item {Grp. gram.:adj.}
\end{itemize}
Relativo ao cerebello.
\section{Cerebelo}
\begin{itemize}
\item {Grp. gram.:m.}
\end{itemize}
\begin{itemize}
\item {Proveniência:(Lat. \textunderscore cerebellum\textunderscore )}
\end{itemize}
Parte posterior de encéfalo.
\section{Cerebeloso}
\begin{itemize}
\item {Grp. gram.:adj.}
\end{itemize}
Relativo ao cerebelo.
\section{Cerebração}
\begin{itemize}
\item {Grp. gram.:f.}
\end{itemize}
\begin{itemize}
\item {Utilização:Neol.}
\end{itemize}
\begin{itemize}
\item {Proveniência:(De \textunderscore cérebro\textunderscore )}
\end{itemize}
Actividade intellectual.
\section{Cerebral}
\begin{itemize}
\item {Grp. gram.:adj.}
\end{itemize}
Relativo ao cérebro: \textunderscore amollecimento cerebral\textunderscore .
\section{Cerebrastenia}
\begin{itemize}
\item {Grp. gram.:f.}
\end{itemize}
\begin{itemize}
\item {Utilização:Med.}
\end{itemize}
Esgotamento cerebral.
\section{Cerebrasthenia}
\begin{itemize}
\item {Grp. gram.:f.}
\end{itemize}
\begin{itemize}
\item {Utilização:Med.}
\end{itemize}
Esgotamento cerebral.
\section{Cerebrátulo}
\begin{itemize}
\item {Grp. gram.:m.}
\end{itemize}
\begin{itemize}
\item {Proveniência:(De \textunderscore cérebro\textunderscore )}
\end{itemize}
Helmintho do Adriático.
\section{Cerébrico}
\begin{itemize}
\item {Grp. gram.:adj.}
\end{itemize}
Diz-se de um ácido, que é a substância branca, descoberta por Vauquelin no cérebro.
\section{Cerebriforme}
\begin{itemize}
\item {Grp. gram.:adj.}
\end{itemize}
\begin{itemize}
\item {Proveniência:(Do lat. \textunderscore cerebrum\textunderscore  + \textunderscore forma\textunderscore )}
\end{itemize}
Que tem a fórma e apparência da substância do cérebro.
\section{Cerebrina}
\begin{itemize}
\item {Grp. gram.:f.}
\end{itemize}
\begin{itemize}
\item {Utilização:Pharm.}
\end{itemize}
Nome de várias substâncias, encontradas no cérebro.
Solução alcoólica de antipyrina, cafeína e cocaína.
\section{Cerebrino}
\begin{itemize}
\item {Grp. gram.:adj.}
\end{itemize}
\begin{itemize}
\item {Proveniência:(De \textunderscore cérebro\textunderscore )}
\end{itemize}
Cerebral.
Imaginoso, phantástico.
Extravagante: \textunderscore ideas cerebrinas\textunderscore .
\section{Cerebrite}
\begin{itemize}
\item {Grp. gram.:f.}
\end{itemize}
Inflammação do cérebro.
\section{Cérebro}
\begin{itemize}
\item {Grp. gram.:m.}
\end{itemize}
\begin{itemize}
\item {Utilização:Fig.}
\end{itemize}
\begin{itemize}
\item {Proveniência:(Lat. \textunderscore cerebrum\textunderscore )}
\end{itemize}
Massa de substância nervosa, que occupa a cavidade do crânio.
Parte do encéphalo, separada do cerebello.
Intelligência; razão.
\section{Cerebropata}
\begin{itemize}
\item {Grp. gram.:m.}
\end{itemize}
\begin{itemize}
\item {Proveniência:(De \textunderscore cérebro\textunderscore  + gr. \textunderscore pathos\textunderscore )}
\end{itemize}
Aquelle que padece cerebropatia.
\section{Cerebropatha}
\begin{itemize}
\item {Grp. gram.:m.}
\end{itemize}
\begin{itemize}
\item {Proveniência:(De \textunderscore cérebro\textunderscore  + gr. \textunderscore pathos\textunderscore )}
\end{itemize}
Aquelle que padece cerebropathia.
\section{Cerebropathia}
\begin{itemize}
\item {Grp. gram.:f.}
\end{itemize}
Doença do cérebro.
(Cp. \textunderscore cerebropatha\textunderscore )
\section{Cerebropatia}
\begin{itemize}
\item {Grp. gram.:f.}
\end{itemize}
Doença do cérebro.
(Cp. \textunderscore cerebropatha\textunderscore )
\section{Cerectaria}
\begin{itemize}
\item {Grp. gram.:f.}
\end{itemize}
\begin{itemize}
\item {Utilização:Med.}
\end{itemize}
\begin{itemize}
\item {Proveniência:(Do gr. \textunderscore keras\textunderscore  + \textunderscore ektaris\textunderscore )}
\end{itemize}
Dilatação da córnea.
\section{Cerefolho}
\begin{itemize}
\item {fónica:fô}
\end{itemize}
\begin{itemize}
\item {Grp. gram.:m.}
\end{itemize}
(V.cerefólio)
\section{Cerefólio}
\begin{itemize}
\item {Grp. gram.:m.}
\end{itemize}
\begin{itemize}
\item {Proveniência:(Lat. \textunderscore caerefolium\textunderscore )}
\end{itemize}
Planta umbellífera, que se cultiva nas hortas.
\section{Cereja}
\begin{itemize}
\item {Grp. gram.:f.}
\end{itemize}
\begin{itemize}
\item {Proveniência:(Do b. lat. \textunderscore cerasea\textunderscore )}
\end{itemize}
Fruto vermelho ou escuro da cerejeira.
\section{Cerejal}
\begin{itemize}
\item {Grp. gram.:m.}
\end{itemize}
\begin{itemize}
\item {Proveniência:(De \textunderscore cereja\textunderscore )}
\end{itemize}
Lugar, onde crescem cerejeiras.
\section{Cerejeira}
\begin{itemize}
\item {Grp. gram.:f.}
\end{itemize}
Árvore fructífera, da fam. das rosáceas.
Madeira desta árvore.
(B. lat. hyp. \textunderscore cerasearia\textunderscore )
\section{Cerejo}
\begin{itemize}
\item {Grp. gram.:m.}
\end{itemize}
\begin{itemize}
\item {Utilização:Prov.}
\end{itemize}
\begin{itemize}
\item {Utilização:trasm.}
\end{itemize}
O tempo das cerejas.
\section{Ceremónia}
\begin{itemize}
\item {Grp. gram.:f.}
\end{itemize}
\begin{itemize}
\item {Proveniência:(Lat. \textunderscore caeremonia\textunderscore )}
\end{itemize}
Fórma exterior do culto religioso.
Solennidade, pompas de uma festa pública.
Formalidades, que a civilidade preceitua entre pessôas bem educadas, que se não tratam familiarmente.
Embaraço, resultante da necessidade de sêr polido entre particulares.
\section{Ceremonial}
\begin{itemize}
\item {Grp. gram.:adj.}
\end{itemize}
\begin{itemize}
\item {Grp. gram.:M.}
\end{itemize}
\begin{itemize}
\item {Proveniência:(Lat. \textunderscore caeremonialis\textunderscore )}
\end{itemize}
Relativo a ceremónias.
Conjunto de formalidades, que se devem observar numa solennidade pública.
Regra, que estabelece essas formalidades.
Livro, que as contém.
Ceremónia.
\section{Ceremoniar}
\begin{itemize}
\item {Grp. gram.:v. t.}
\end{itemize}
\begin{itemize}
\item {Utilização:Des.}
\end{itemize}
\begin{itemize}
\item {Proveniência:(Lat. \textunderscore caeremoniari\textunderscore )}
\end{itemize}
Tratar ceremoniosamente.
Celebrar.
\section{Ceremoniaticamente}
\begin{itemize}
\item {Grp. gram.:adv.}
\end{itemize}
De modo ceremoniático.
\section{Ceremoniático}
\begin{itemize}
\item {Grp. gram.:adj.}
\end{itemize}
\begin{itemize}
\item {Utilização:Fam.}
\end{itemize}
\begin{itemize}
\item {Proveniência:(De \textunderscore ceremónia\textunderscore )}
\end{itemize}
Muito escrupuloso ou excessivo em ceremónias.
\section{Ceremoniosamente}
\begin{itemize}
\item {Grp. gram.:adv.}
\end{itemize}
De modo ceremoniôso.
\section{Ceremonioso}
\begin{itemize}
\item {Grp. gram.:adj.}
\end{itemize}
\begin{itemize}
\item {Proveniência:(Lat. \textunderscore caeremoniosus\textunderscore )}
\end{itemize}
Que tem ceremónias.
Em que há ceremónias.
Que usa de ceremónias.
\section{Céreo}
\begin{itemize}
\item {Grp. gram.:adj.}
\end{itemize}
\begin{itemize}
\item {Grp. gram.:M.}
\end{itemize}
\begin{itemize}
\item {Utilização:Ant.}
\end{itemize}
\begin{itemize}
\item {Proveniência:(Lat. \textunderscore cereus\textunderscore )}
\end{itemize}
Feito de cera.
Semelhante á cera; que é da côr da cera.
O mesmo que \textunderscore círio\textunderscore .
\section{Ceres}
\begin{itemize}
\item {Grp. gram.:f.}
\end{itemize}
\begin{itemize}
\item {Utilização:Fig.}
\end{itemize}
\begin{itemize}
\item {Proveniência:(Lat. \textunderscore Ceres\textunderscore , n. p.)}
\end{itemize}
Cereaes; agricultura.
Nome de um planeta.
\section{Cerésia}
\begin{itemize}
\item {Grp. gram.:f.}
\end{itemize}
\begin{itemize}
\item {Proveniência:(De \textunderscore Ceres\textunderscore , n. p.)}
\end{itemize}
Gênero de plantas gramíneas.
\section{Ceresina}
\begin{itemize}
\item {Grp. gram.:f.}
\end{itemize}
\begin{itemize}
\item {Proveniência:(De \textunderscore cera\textunderscore )}
\end{itemize}
Cera mineral ou fóssil, composta de carbone e hydrogênio, e semelhante á cera commum, até no cheiro.
\section{Cérica}
\begin{itemize}
\item {Grp. gram.:f.}
\end{itemize}
O mesmo que \textunderscore cerato\textunderscore .
\section{Cérico}
\begin{itemize}
\item {Grp. gram.:adj.}
\end{itemize}
\begin{itemize}
\item {Utilização:Chím.}
\end{itemize}
\begin{itemize}
\item {Proveniência:(De \textunderscore cera\textunderscore )}
\end{itemize}
Diz-se de um ácido, que resulta da acção do ácido nítrico sôbre a cera.
\section{Cericumás}
\begin{itemize}
\item {Grp. gram.:m. pl.}
\end{itemize}
Indígenas da Guiana brasileira.
\section{Cerieira}
\begin{itemize}
\item {Grp. gram.:f.}
\end{itemize}
\begin{itemize}
\item {Proveniência:(De \textunderscore cera\textunderscore )}
\end{itemize}
Planta, que produz a cera vegetal.
Azeitona carlota.
\section{Cerieiro}
\begin{itemize}
\item {Grp. gram.:m.}
\end{itemize}
Aquelle que trabalha em cera.
Aquelle que vende obras de cera.
\section{Cerieiro}
\begin{itemize}
\item {Grp. gram.:m.}
\end{itemize}
(V.cirieiro)
\section{Cerífero}
\begin{itemize}
\item {Grp. gram.:adj.}
\end{itemize}
\begin{itemize}
\item {Proveniência:(Do lat. \textunderscore cera\textunderscore  + \textunderscore ferre\textunderscore )}
\end{itemize}
Que produz cera.
\section{Cerífico}
\begin{itemize}
\item {Grp. gram.:adj.}
\end{itemize}
\begin{itemize}
\item {Proveniência:(Do lat. \textunderscore cera\textunderscore  + \textunderscore facere\textunderscore )}
\end{itemize}
Que produz cera.
\section{Cerilhoto}
\begin{itemize}
\item {fónica:lhô}
\end{itemize}
\begin{itemize}
\item {Grp. gram.:m.}
\end{itemize}
\begin{itemize}
\item {Utilização:Prov.}
\end{itemize}
\begin{itemize}
\item {Utilização:minh.}
\end{itemize}
Deminuta porção de excrementos sólidos humanos, recentemente expellidos. (Colhido em Guimarães)
\section{Cerimónia}
\textunderscore f.\textunderscore  (e der.)
(V. \textunderscore ceremónia\textunderscore , etc.)
\section{Cerina}
\begin{itemize}
\item {Grp. gram.:f.}
\end{itemize}
\begin{itemize}
\item {Proveniência:(De \textunderscore cera\textunderscore )}
\end{itemize}
Um dos princípios que constituem a cera.
\section{Ceringonhar}
\begin{itemize}
\item {Grp. gram.:v. i.}
\end{itemize}
\begin{itemize}
\item {Utilização:Prov.}
\end{itemize}
\begin{itemize}
\item {Utilização:trasm.}
\end{itemize}
Pedir impertinentemente; maçar com instâncias.
\section{Cerintho}
\begin{itemize}
\item {Grp. gram.:m.}
\end{itemize}
\begin{itemize}
\item {Proveniência:(Lat. \textunderscore cerintha\textunderscore )}
\end{itemize}
Planta borragínea.
\section{Cerinto}
\begin{itemize}
\item {Grp. gram.:m.}
\end{itemize}
\begin{itemize}
\item {Proveniência:(Lat. \textunderscore cerintha\textunderscore )}
\end{itemize}
Planta borragínea.
\section{Cério}
\begin{itemize}
\item {Grp. gram.:m.}
\end{itemize}
Nome de um metal, descoberto na cerita.
(Cp. \textunderscore cerita\textunderscore )
\section{Cério}
\begin{itemize}
\item {Grp. gram.:m.}
\end{itemize}
\begin{itemize}
\item {Proveniência:(Do gr. \textunderscore kerion\textunderscore , céllula)}
\end{itemize}
Nome, que alguns botânicos deram ao fruto das gramíneas.
\section{Ceriosa}
\begin{itemize}
\item {Grp. gram.:f.}
\end{itemize}
Nome de uma flôr:«\textunderscore vaso enfeitado de ceriosas fragrantes\textunderscore ». Castilho, \textunderscore Mil e um Myst.\textunderscore , 228.
Variedade de azeitona, também conhecida por \textunderscore carlota\textunderscore  ou \textunderscore cerieira\textunderscore .
\section{Cerirostro}
\begin{itemize}
\item {fónica:rôs}
\end{itemize}
\begin{itemize}
\item {Grp. gram.:adj.}
\end{itemize}
\begin{itemize}
\item {Utilização:Zool.}
\end{itemize}
\begin{itemize}
\item {Proveniência:(Do lat. \textunderscore cera\textunderscore  + \textunderscore rostrum\textunderscore )}
\end{itemize}
Que tem no bico uma membrana cerosa.
\section{Cerirrostro}
\begin{itemize}
\item {Grp. gram.:adj.}
\end{itemize}
\begin{itemize}
\item {Utilização:Zool.}
\end{itemize}
\begin{itemize}
\item {Proveniência:(Do lat. \textunderscore cera\textunderscore  + \textunderscore rostrum\textunderscore )}
\end{itemize}
Que tem no bico uma membrana cerosa.
\section{Cerita}
\begin{itemize}
\item {Grp. gram.:f.}
\end{itemize}
\begin{itemize}
\item {Proveniência:(Gr. \textunderscore kerites\textunderscore )}
\end{itemize}
Minério côr da cera.
\section{Cerite}
\begin{itemize}
\item {Grp. gram.:f.}
\end{itemize}
(V.cerita)
\section{Cerna}
\begin{itemize}
\item {Grp. gram.:f.}
\end{itemize}
\begin{itemize}
\item {Utilização:Ant.}
\end{itemize}
O mesmo que \textunderscore cenra\textunderscore .
\section{Cernada}
\begin{itemize}
\item {Grp. gram.:f.}
\end{itemize}
\begin{itemize}
\item {Utilização:Ant.}
\end{itemize}
Cerna extensa; conjunto de cernas.
\section{Cernada}
\begin{itemize}
\item {Grp. gram.:f.}
\end{itemize}
Acção de cernar.
\section{Cernandi}
\begin{itemize}
\item {Grp. gram.:m.}
\end{itemize}
\begin{itemize}
\item {Utilização:Bras. do Pará}
\end{itemize}
A borracha mais grosseira.
\section{Cernar}
\begin{itemize}
\item {Grp. gram.:v. t.}
\end{itemize}
Descobrir o cerne de.
Cortar até ao cerne.
Extrahir o cerne de.
\section{Cerne}
\begin{itemize}
\item {Grp. gram.:m.}
\end{itemize}
\begin{itemize}
\item {Proveniência:(Fr. \textunderscore cerne\textunderscore )}
\end{itemize}
A parte interior e mais dura das árvores.
\section{Cerneira}
\begin{itemize}
\item {Grp. gram.:f.}
\end{itemize}
\begin{itemize}
\item {Proveniência:(De \textunderscore cerne\textunderscore )}
\end{itemize}
Parte lenhosa dos troncos ou ramos que, apodrecendo, largam a casca e alburno.
Tábua de cerne, ou tábua sem alburno.
\section{Cerneiro}
\begin{itemize}
\item {Grp. gram.:adj.}
\end{itemize}
Que tem cerne.
\section{Cernelha}
\begin{itemize}
\item {fónica:nê}
\end{itemize}
\begin{itemize}
\item {Grp. gram.:f.}
\end{itemize}
\begin{itemize}
\item {Utilização:Prov.}
\end{itemize}
\begin{itemize}
\item {Utilização:trasm.}
\end{itemize}
\begin{itemize}
\item {Proveniência:(Do lat. \textunderscore cernícla\textunderscore , segundo Meyer-Lübke)}
\end{itemize}
Parte do corpo de alguns animaes, em que se juntam as espáduas.
Fio do lombo.
Rima de mólhos de centeio, trigo ou milho, nas terras ceifadas ou nas eiras, e de fórma especial.
\section{Cernir}
\begin{itemize}
\item {Grp. gram.:v. t.}
\end{itemize}
\begin{itemize}
\item {Utilização:Des.}
\end{itemize}
\begin{itemize}
\item {Grp. gram.:V. i.}
\end{itemize}
Peneirar.
Saracotear-se.
(Cast. \textunderscore cernir\textunderscore )
\section{Cerofala}
\begin{itemize}
\item {Grp. gram.:f.}
\end{itemize}
\begin{itemize}
\item {Utilização:Ant.}
\end{itemize}
\begin{itemize}
\item {Proveniência:(Do b. lat. \textunderscore ceroferale\textunderscore )}
\end{itemize}
O mesmo que \textunderscore castiçal\textunderscore .
Tocheiro.
\section{Ceroferário}
\begin{itemize}
\item {Grp. gram.:m.}
\end{itemize}
Aquelle que leva tocheira ou círio.
Acólyto.
(B. lat. \textunderscore ceroferarius\textunderscore )
\section{Ceroide}
\begin{itemize}
\item {Grp. gram.:adj.}
\end{itemize}
\begin{itemize}
\item {Proveniência:(Do gr. \textunderscore keros\textunderscore  + \textunderscore eidos\textunderscore )}
\end{itemize}
Que tem apparência de cera.
\section{Ceroilas}
\begin{itemize}
\item {Grp. gram.:f. pl.}
\end{itemize}
\begin{itemize}
\item {Proveniência:(Do ár. \textunderscore çarauil\textunderscore )}
\end{itemize}
Vestuário, que os homens usam por baixo das calças.
\section{Cerol}
\begin{itemize}
\item {Grp. gram.:m.}
\end{itemize}
\begin{itemize}
\item {Utilização:T. da Bairrada}
\end{itemize}
\begin{itemize}
\item {Proveniência:(De \textunderscore cera\textunderscore )}
\end{itemize}
Massa de cera, pez e sebo, com que se enceram as linhas para as obras de sola ou coiro.
Mêdo.
\section{Cerolha}
\begin{itemize}
\item {fónica:cerô}
\end{itemize}
\begin{itemize}
\item {Grp. gram.:adj. f.}
\end{itemize}
\begin{itemize}
\item {Utilização:Prov.}
\end{itemize}
\begin{itemize}
\item {Utilização:minh.}
\end{itemize}
Diz-se da roupa mal enxuta. (Colhido em Barcelos)
\section{Ceromancia}
\begin{itemize}
\item {Grp. gram.:f.}
\end{itemize}
\begin{itemize}
\item {Proveniência:(Do gr. \textunderscore keros\textunderscore  + \textunderscore manteia\textunderscore )}
\end{itemize}
Systêma de adivinhação, por meio de cera derretida.
\section{Ceromântico}
\begin{itemize}
\item {Grp. gram.:adj.}
\end{itemize}
Relativo á \textunderscore ceromancia\textunderscore .
\section{Cerome}
\begin{itemize}
\item {Grp. gram.:m.}
\end{itemize}
Capa antiga de mulher.
(Ár. \textunderscore selham\textunderscore )
\section{Ceromel}
\begin{itemize}
\item {Grp. gram.:m.}
\end{itemize}
\begin{itemize}
\item {Proveniência:(De \textunderscore cera\textunderscore  + \textunderscore mel\textunderscore )}
\end{itemize}
Unguento de cera e mel.
\section{Ceroplástica}
\begin{itemize}
\item {Grp. gram.:f.}
\end{itemize}
\begin{itemize}
\item {Proveniência:(De \textunderscore cera\textunderscore  + \textunderscore plástica\textunderscore )}
\end{itemize}
Arte de fazer figuras de cera.
\section{Ceroscopia}
\begin{itemize}
\item {Grp. gram.:f.}
\end{itemize}
\begin{itemize}
\item {Proveniência:(Do gr. \textunderscore keros\textunderscore  + \textunderscore skopein\textunderscore )}
\end{itemize}
O mesmo que \textunderscore ceromancia\textunderscore .
\section{Cerosene}
\begin{itemize}
\item {Grp. gram.:m.}
\end{itemize}
Nome, que alguns chímicos dão ao petróleo.
\section{Ceroso}
\begin{itemize}
\item {Grp. gram.:adj.}
\end{itemize}
\begin{itemize}
\item {Proveniência:(Lat. \textunderscore cerosus\textunderscore )}
\end{itemize}
(V.céreo)
\section{Ceroto}
\begin{itemize}
\item {fónica:cerô}
\end{itemize}
\begin{itemize}
\item {Grp. gram.:m.}
\end{itemize}
O mesmo que \textunderscore cerato\textunderscore .
\section{Ceroulas}
\begin{itemize}
\item {Grp. gram.:f. pl.}
\end{itemize}
\begin{itemize}
\item {Proveniência:(Do ár. \textunderscore çarauil\textunderscore )}
\end{itemize}
Vestuário, que os homens usam por baixo das calças.
\section{Ceroxilo}
\begin{itemize}
\item {Grp. gram.:m.}
\end{itemize}
\begin{itemize}
\item {Proveniência:(Do gr. \textunderscore keros\textunderscore  + \textunderscore xulon\textunderscore )}
\end{itemize}
Palmeira, que produz uma espécie de cera.
\section{Ceroxylo}
\begin{itemize}
\item {Grp. gram.:m.}
\end{itemize}
\begin{itemize}
\item {Proveniência:(Do gr. \textunderscore keros\textunderscore  + \textunderscore xulon\textunderscore )}
\end{itemize}
Palmeira, que produz uma espécie de cera.
\section{Cerqueiro}
\begin{itemize}
\item {Grp. gram.:adj.}
\end{itemize}
\begin{itemize}
\item {Grp. gram.:M.}
\end{itemize}
\begin{itemize}
\item {Proveniência:(De \textunderscore cêrca\textunderscore  e \textunderscore cercar\textunderscore )}
\end{itemize}
Que cérca, que rodeia ou envolve: \textunderscore parede cerqueira\textunderscore .
Cultivador de uma cêrca.
\section{Cerquido}
\begin{itemize}
\item {Grp. gram.:m.}
\end{itemize}
\begin{itemize}
\item {Utilização:Prov.}
\end{itemize}
\begin{itemize}
\item {Utilização:minh.}
\end{itemize}
Soito de carvalhos cerquinhos.
\section{Cerquinho}
\begin{itemize}
\item {Grp. gram.:adj.}
\end{itemize}
\begin{itemize}
\item {Proveniência:(Lat. \textunderscore quercinus\textunderscore , de \textunderscore quercus\textunderscore )}
\end{itemize}
Diz-se de uma espécie de carvalho.
\section{Cerra-bôca}
\begin{itemize}
\item {Grp. gram.:m.}
\end{itemize}
\begin{itemize}
\item {Utilização:Náut.}
\end{itemize}
Um dos cabos usados a bordo das baleeiras.
\section{Cerra-cabos}
\begin{itemize}
\item {Grp. gram.:m.}
\end{itemize}
\begin{itemize}
\item {Proveniência:(De \textunderscore cerrar\textunderscore  + \textunderscore cabo\textunderscore )}
\end{itemize}
Systema especial de deitar dois apparelhos de xávega, um a par do outro e um terceiro no meio delles.
\section{Cerra-cancella}
\begin{itemize}
\item {Grp. gram.:f.}
\end{itemize}
\begin{itemize}
\item {Utilização:T. da Bairrada}
\end{itemize}
Nome de um bichinho escuro e longo, com muitas pernas.
\section{Cerração}
\begin{itemize}
\item {Grp. gram.:f.}
\end{itemize}
\begin{itemize}
\item {Utilização:Fig.}
\end{itemize}
\begin{itemize}
\item {Proveniência:(De \textunderscore cerrar\textunderscore )}
\end{itemize}
Nevoeiro espêsso.
Escuridão.
Rouquidão, difficuldade em falar.
\section{Cerradamente}
\begin{itemize}
\item {Grp. gram.:adv.}
\end{itemize}
\begin{itemize}
\item {Utilização:Ant.}
\end{itemize}
\begin{itemize}
\item {Proveniência:(De \textunderscore cerrado\textunderscore )}
\end{itemize}
Dissimuladamente.
Com pertinácia.
Em tôrno, á roda.
\section{Cerradinha}
\begin{itemize}
\item {Grp. gram.:f.}
\end{itemize}
Acto de cerrar-se (a noite); o anoitecer:«\textunderscore á cerradinha da noite chegávamos...\textunderscore »Th. Ribeiro, \textunderscore Jornadas\textunderscore , II, 224.
\section{Cerrado}
\begin{itemize}
\item {Grp. gram.:adj.}
\end{itemize}
\begin{itemize}
\item {Utilização:Prov.}
\end{itemize}
\begin{itemize}
\item {Utilização:minh.}
\end{itemize}
\begin{itemize}
\item {Grp. gram.:M.}
\end{itemize}
\begin{itemize}
\item {Utilização:Bras}
\end{itemize}
Diz-se do arado, que lavra á flôr da terra, ao contrário do arado bicheiro.
Cêrca, terreno tapado ou murado.
Mata xeróphyta dos planaltos.
\section{Cerradoiro}
\begin{itemize}
\item {Grp. gram.:m.}
\end{itemize}
\begin{itemize}
\item {Proveniência:(De \textunderscore cerrar\textunderscore )}
\end{itemize}
Cordão, com que se cerram bôlsas, sacos, etc.
\section{Cerradouro}
\begin{itemize}
\item {Grp. gram.:m.}
\end{itemize}
\begin{itemize}
\item {Proveniência:(De \textunderscore cerrar\textunderscore )}
\end{itemize}
Cordão, com que se cerram bôlsas, sacos, etc.
\section{Cerradura}
\begin{itemize}
\item {Grp. gram.:f.}
\end{itemize}
\begin{itemize}
\item {Proveniência:(De \textunderscore cerrar\textunderscore )}
\end{itemize}
Cêrca, muro.
\section{Cerra-fila}
\begin{itemize}
\item {Grp. gram.:m.}
\end{itemize}
\begin{itemize}
\item {Proveniência:(De \textunderscore cerrar\textunderscore  + \textunderscore fila\textunderscore )}
\end{itemize}
Soldado, que fica atrás do chefe de uma fila.
Navio, que vai na retaguarda de outros.
\section{Cerramento}
\begin{itemize}
\item {Grp. gram.:m.}
\end{itemize}
Acção de cerrar.
\section{Cerrar}
\begin{itemize}
\item {Grp. gram.:v. t.}
\end{itemize}
\begin{itemize}
\item {Grp. gram.:V. i.}
\end{itemize}
\begin{itemize}
\item {Grp. gram.:V. p.}
\end{itemize}
Fechar: \textunderscore cerrar a janela\textunderscore .
Ajuntar, unir.
Encerrar.
Encobrir; tapar.
Terminar: \textunderscore cerrar um discurso\textunderscore .
Têr (a bêsta) a idade, em que os dentes estão completamente desenvolvidos.
Acabar de falar ou de escrever: \textunderscore e aqui me cérro\textunderscore .
(Cast. \textunderscore cerrar\textunderscore )
\section{Cêrro}
\begin{itemize}
\item {Grp. gram.:m.}
\end{itemize}
Oiteiro; pequeno monte penhascoso.
(Cp. cast. \textunderscore cerro\textunderscore )
\section{Cêrro}
\begin{itemize}
\item {Grp. gram.:m.}
\end{itemize}
\begin{itemize}
\item {Utilização:Prov.}
\end{itemize}
\begin{itemize}
\item {Utilização:trasm.}
\end{itemize}
Carne do lombo do porco, pegada ao coiro.
\section{Cerrucho}
\begin{itemize}
\item {Grp. gram.:m.}
\end{itemize}
\begin{itemize}
\item {Utilização:Prov.}
\end{itemize}
\begin{itemize}
\item {Utilização:trasm.}
\end{itemize}
Pequeníssima porção de líquido, no fundo de uma vasilha.
\section{Certa}
\begin{itemize}
\item {Grp. gram.:f.}
\end{itemize}
\begin{itemize}
\item {Utilização:Pop.}
\end{itemize}
\begin{itemize}
\item {Proveniência:(De \textunderscore certo\textunderscore )}
\end{itemize}
Certeza; o que é certo. (Us. principalmente na loc. \textunderscore pela certa\textunderscore )
\section{Certame}
\begin{itemize}
\item {Grp. gram.:m.}
\end{itemize}
\begin{itemize}
\item {Proveniência:(Lat. \textunderscore certamen\textunderscore )}
\end{itemize}
Peleja; luta; combate.
Debate; discussão.
Concurso literário, scientifico ou industrial.
\section{Certâmen}
\begin{itemize}
\item {Grp. gram.:m.}
\end{itemize}
(V.certame)
\section{Certamente}
\begin{itemize}
\item {Grp. gram.:adv.}
\end{itemize}
\begin{itemize}
\item {Proveniência:(De \textunderscore certo\textunderscore )}
\end{itemize}
Com certeza; em verdade.
\section{Certão}
\begin{itemize}
\item {Grp. gram.:adj.}
\end{itemize}
\begin{itemize}
\item {Utilização:Ant.}
\end{itemize}
O mesmo que \textunderscore certo\textunderscore .
(Cp. fr. \textunderscore certain\textunderscore )
\section{Certão}
\begin{itemize}
\item {Grp. gram.:m.}
\end{itemize}
A maior fôrça?«\textunderscore ...pelo certão da calma\textunderscore ». Lobo, \textunderscore Côrte na Aldeia\textunderscore , I, 78.
\section{Certar}
\begin{itemize}
\item {Grp. gram.:v. i.}
\end{itemize}
\begin{itemize}
\item {Proveniência:(Lat. \textunderscore certare\textunderscore )}
\end{itemize}
Combater; pleitear.
Discutir.
Ir a concurso.
\section{Certeiramente}
\begin{itemize}
\item {Grp. gram.:adv.}
\end{itemize}
De modo certeiro.
Com acêrto.
\section{Certeiro}
\begin{itemize}
\item {Grp. gram.:adj.}
\end{itemize}
\begin{itemize}
\item {Proveniência:(De \textunderscore certo\textunderscore )}
\end{itemize}
Que acerta bem: \textunderscore espingarda certeira\textunderscore .
Bem dirigido; acertado: \textunderscore tiro certeiro\textunderscore .
\section{Certela}
\begin{itemize}
\item {Grp. gram.:f.}
\end{itemize}
\begin{itemize}
\item {Utilização:Prov.}
\end{itemize}
Acto de pescar enguias, com anzol, provido de minhocas.
\section{Certeza}
\begin{itemize}
\item {Grp. gram.:f.}
\end{itemize}
\begin{itemize}
\item {Proveniência:(De \textunderscore certo\textunderscore )}
\end{itemize}
Qualidade do que é certo.
Conhecimento exacto; convicção.
Coisa certa.
Estabilidade; firmêza.
\section{Cérthia}
\begin{itemize}
\item {Grp. gram.:f.}
\end{itemize}
Gênero de aves trepadoras.
\section{Cértia}
\begin{itemize}
\item {Grp. gram.:f.}
\end{itemize}
Gênero de aves trepadoras.
\section{Certidão}
\begin{itemize}
\item {Grp. gram.:f.}
\end{itemize}
\begin{itemize}
\item {Utilização:Ant.}
\end{itemize}
\begin{itemize}
\item {Proveniência:(Lat. \textunderscore certitudo\textunderscore )}
\end{itemize}
Documento, com que se certifica alguma coisa.
Attestação.
Certeza.
\section{Certificação}
\begin{itemize}
\item {Grp. gram.:f.}
\end{itemize}
Acto de certificar.
(B. lat. \textunderscore certificatio\textunderscore )
\section{Certificado}
\begin{itemize}
\item {Grp. gram.:m.}
\end{itemize}
\begin{itemize}
\item {Proveniência:(De \textunderscore certificar\textunderscore )}
\end{itemize}
Certidão; documento, em que se certifica alguma coisa.
\section{Certificador}
\begin{itemize}
\item {Grp. gram.:m. e adj.}
\end{itemize}
O mesmo que \textunderscore certificante\textunderscore .
\section{Certificante}
\begin{itemize}
\item {Grp. gram.:m.  e  adj.}
\end{itemize}
O que certifica.
(B. lat. \textunderscore certificans\textunderscore )
\section{Certificar}
\begin{itemize}
\item {Grp. gram.:v. t.}
\end{itemize}
Asseverar a certeza de; attestar.
Tornar alguém sciente de.
Passar certidão de.
(B. lat. \textunderscore certificare\textunderscore )
\section{Certificativo}
\begin{itemize}
\item {Grp. gram.:adj.}
\end{itemize}
Que certifica; próprio para certificar.
\section{Certificatório}
\begin{itemize}
\item {Grp. gram.:adj.}
\end{itemize}
Que certifica; próprio para certificar.
\section{Certilha}
\begin{itemize}
\item {Grp. gram.:f.}
\end{itemize}
\begin{itemize}
\item {Utilização:Prov.}
\end{itemize}
Espécie de armadilha para caça.
\section{Certo}
\begin{itemize}
\item {Grp. gram.:adj.}
\end{itemize}
\begin{itemize}
\item {Grp. gram.:M.}
\end{itemize}
\begin{itemize}
\item {Grp. gram.:Adv.}
\end{itemize}
\begin{itemize}
\item {Proveniência:(Lat. \textunderscore certus\textunderscore )}
\end{itemize}
Verdadeiro; evidente.
Infallível; em que não há erro: \textunderscore é certo que 2 e 2 são 4\textunderscore .
Previamente determinado.
Convencido: \textunderscore estou certo de que te enganas\textunderscore .
Exacto; preciso.
Qualquer.
Ajustado.
Um, algum: \textunderscore certo homem\textunderscore .
Coisa certa.
Certamente.
\textunderscore Ao certo\textunderscore , com exactidão.
\textunderscore De certo\textunderscore , com certeza.
\section{Ceruana}
\begin{itemize}
\item {Grp. gram.:f.}
\end{itemize}
Planta do Egypto, da fam. das compostas.
\section{Ceruda}
\begin{itemize}
\item {Grp. gram.:f.}
\end{itemize}
(V.celidónia)
\section{Cerúleo}
\begin{itemize}
\item {Grp. gram.:adj.}
\end{itemize}
O mesmo que \textunderscore cérulo\textunderscore .
\section{Cerulicrinito}
\begin{itemize}
\item {fónica:cé}
\end{itemize}
\begin{itemize}
\item {Grp. gram.:adj.}
\end{itemize}
\begin{itemize}
\item {Utilização:Poét.}
\end{itemize}
\begin{itemize}
\item {Proveniência:(Do lat. \textunderscore caerulus\textunderscore  + \textunderscore crinitus\textunderscore )}
\end{itemize}
Que tem os cabellos azues.
\section{Cerulina}
\begin{itemize}
\item {Grp. gram.:f.}
\end{itemize}
\begin{itemize}
\item {Proveniência:(De \textunderscore cérulo\textunderscore )}
\end{itemize}
Anil solúvel.
\section{Cérulo}
\begin{itemize}
\item {Grp. gram.:adj.}
\end{itemize}
\begin{itemize}
\item {Proveniência:(Lat. \textunderscore caerulus\textunderscore )}
\end{itemize}
Azul-escuro.
Azulado.
Da côr verde-mar.
Da côr do céu.
\section{Cerume}
\begin{itemize}
\item {Grp. gram.:m.}
\end{itemize}
\begin{itemize}
\item {Proveniência:(Lat. \textunderscore cerumen\textunderscore )}
\end{itemize}
Humor untuoso, espêsso e amarelado, que se fórma nos ouvidos, e que vulgarmente se chama cera das orelhas.
\section{Cerumen}
\begin{itemize}
\item {Grp. gram.:m.}
\end{itemize}
(V.cerume)
\section{Ceruminoso}
\begin{itemize}
\item {Grp. gram.:adj.}
\end{itemize}
\begin{itemize}
\item {Proveniência:(Do lat. \textunderscore cerumen\textunderscore )}
\end{itemize}
Que tem as qualidades do cerume.
Relativo ao cerume.
\section{Cerusa}
\begin{itemize}
\item {Grp. gram.:f.}
\end{itemize}
(V.alvaiade)
\section{Cerussite}
\begin{itemize}
\item {Grp. gram.:f.}
\end{itemize}
Sulfureto natural, producto da alteração da galenite.
\section{Cerva}
\begin{itemize}
\item {Grp. gram.:f.}
\end{itemize}
\begin{itemize}
\item {Proveniência:(Lat. \textunderscore cerva\textunderscore )}
\end{itemize}
Fêmea do veado.
Não é o mesmo que \textunderscore corça\textunderscore , não obstante o que dizem os diccionários.
\section{Cabázia}
\begin{itemize}
\item {Grp. gram.:f.}
\end{itemize}
\begin{itemize}
\item {Proveniência:(Do gr. \textunderscore khabazios\textunderscore )}
\end{itemize}
Variedade de silicato aluminoso.
\section{Cabazite}
\begin{itemize}
\item {Grp. gram.:f.}
\end{itemize}
\begin{itemize}
\item {Proveniência:(Do gr. \textunderscore khabazios\textunderscore )}
\end{itemize}
Espécie de zoólito.
\section{Céo}
\begin{itemize}
\item {Grp. gram.:m.}
\end{itemize}
\begin{itemize}
\item {Utilização:Fig.}
\end{itemize}
\begin{itemize}
\item {Grp. gram.:Pl. interj.}
\end{itemize}
\begin{itemize}
\item {Proveniência:(Do lat. \textunderscore caelum\textunderscore )}
\end{itemize}
Espaço illimitado, em que se movem os astros.
Parte dêsse espaço, limitado pelo horizonte: \textunderscore o céo está nublado\textunderscore .
Atmosphera: \textunderscore a serenidade do céo\textunderscore .
Lugar, onde, segundo as crenças religiosas, estão as almas dos justos: \textunderscore minha mãe está no céo\textunderscore .
Deus.
Providência: \textunderscore demos graças ao céo\textunderscore .
\textunderscore Fogo do céo\textunderscore , o raio.
\textunderscore Um céo aberto\textunderscore , grande ventura.
\textunderscore Céo da bôca\textunderscore , o palato.
(designativa de surpresa ou dôr): \textunderscore céos! que vejo!\textunderscore 
\section{Cerval}
\begin{itemize}
\item {Grp. gram.:adj.}
\end{itemize}
\begin{itemize}
\item {Utilização:Fig.}
\end{itemize}
\begin{itemize}
\item {Proveniência:(Lat. \textunderscore cervarius\textunderscore )}
\end{itemize}
Relativo ao cervo.
Feroz.
\textunderscore Lobo cerval\textunderscore , o mesmo que \textunderscore lynce\textunderscore .
\section{Cerval}
\begin{itemize}
\item {Grp. gram.:m.  e  adj.}
\end{itemize}
Casta de uva trasmontana.
\section{Cervantesco}
\begin{itemize}
\item {fónica:tês}
\end{itemize}
\begin{itemize}
\item {Grp. gram.:adj.}
\end{itemize}
\begin{itemize}
\item {Proveniência:(De \textunderscore Cervantes\textunderscore , n. p.)}
\end{itemize}
Relativo a Cervantes, ao seu estilo, ou aos seus heróis.
\section{Cervantésia}
\begin{itemize}
\item {Grp. gram.:f.}
\end{itemize}
\begin{itemize}
\item {Proveniência:(De \textunderscore Cervantes\textunderscore , n. p.)}
\end{itemize}
Planta santalácea do Peru.
\section{Cervantina}
\begin{itemize}
\item {Grp. gram.:f.}
\end{itemize}
Variedade de figos.
\section{Cervato}
\begin{itemize}
\item {Grp. gram.:m.}
\end{itemize}
\begin{itemize}
\item {Proveniência:(De \textunderscore cervo\textunderscore )}
\end{itemize}
Cervo pequeno.
\section{Cerveiro}
\begin{itemize}
\item {Grp. gram.:m.}
\end{itemize}
\begin{itemize}
\item {Utilização:Ant.}
\end{itemize}
O mesmo que \textunderscore cérbero\textunderscore , como se a pronúncia fôsse \textunderscore cerbéro\textunderscore . Cf. \textunderscore Eufrosina\textunderscore , 142.
\section{Cerveja}
\begin{itemize}
\item {Grp. gram.:f.}
\end{itemize}
\begin{itemize}
\item {Proveniência:(Do lat. \textunderscore cervicia\textunderscore )}
\end{itemize}
Bebida alcoólica, feita com lúpulo e cevada ou outros cereaes.
\section{Cervejada}
\begin{itemize}
\item {Grp. gram.:f.}
\end{itemize}
\begin{itemize}
\item {Utilização:Fam.}
\end{itemize}
Copo de cerveja.
\section{Cervejaria}
\begin{itemize}
\item {Grp. gram.:f.}
\end{itemize}
Casa, onde se fabríca ou se vende cerveja.
\section{Cervejeiro}
\begin{itemize}
\item {Grp. gram.:m.}
\end{itemize}
Aquelle que fabríca ou vende cerveja.
\section{Cervello}
\begin{itemize}
\item {Grp. gram.:m.}
\end{itemize}
\begin{itemize}
\item {Utilização:Ant.}
\end{itemize}
O mesmo que \textunderscore cerviz\textunderscore . Cf. D. Bernardes, \textunderscore Lima\textunderscore , 210.
(Cp. \textunderscore cerebello\textunderscore )
\section{Cervelo}
\begin{itemize}
\item {Grp. gram.:m.}
\end{itemize}
\begin{itemize}
\item {Utilização:Ant.}
\end{itemize}
O mesmo que \textunderscore cerviz\textunderscore . Cf. D. Bernardes, \textunderscore Lima\textunderscore , 210.
(Cp. \textunderscore cerebello\textunderscore )
\section{Cérvia}
\begin{itemize}
\item {Grp. gram.:f.}
\end{itemize}
Gênero de plantas convolvuláceas.
\section{Cervicabra}
\begin{itemize}
\item {Grp. gram.:f.}
\end{itemize}
Espécie de cabra montês.
\section{Cervical}
\begin{itemize}
\item {Grp. gram.:adj.}
\end{itemize}
\begin{itemize}
\item {Proveniência:(Lat. \textunderscore cervicalis\textunderscore )}
\end{itemize}
Relativo a cerviz.
\section{Cervicina}
\begin{itemize}
\item {Grp. gram.:f.}
\end{itemize}
Gênero de plantas campanuláceas.
\section{Cervicórneo}
\begin{itemize}
\item {Grp. gram.:adj.}
\end{itemize}
\begin{itemize}
\item {Utilização:Zool.}
\end{itemize}
\begin{itemize}
\item {Proveniência:(Do lat. \textunderscore cervus\textunderscore  + \textunderscore cornu\textunderscore )}
\end{itemize}
Que tem antennas semelhantes a cornos de veado.
\section{Cerviculado}
\begin{itemize}
\item {Grp. gram.:adj.}
\end{itemize}
\begin{itemize}
\item {Proveniência:(Do lat. \textunderscore cervicula\textunderscore )}
\end{itemize}
Semelhante a um pequeno pescoço.
\section{Cervídeos}
\begin{itemize}
\item {Grp. gram.:m. pl.}
\end{itemize}
\begin{itemize}
\item {Proveniência:(Do lat. \textunderscore cervus\textunderscore  + gr. \textunderscore eidos\textunderscore )}
\end{itemize}
Família de animaes, a que pertence o cervo.
\section{Cervigueira}
\begin{itemize}
\item {Grp. gram.:f.}
\end{itemize}
\begin{itemize}
\item {Proveniência:(Lat. hyp. \textunderscore cervicaria\textunderscore , do lat. \textunderscore cervix\textunderscore )}
\end{itemize}
Doença, nas gengivas dos porcos, a qual lhes difficulta o comer.
\section{Cervilha}
\begin{itemize}
\item {Grp. gram.:f.}
\end{itemize}
\begin{itemize}
\item {Utilização:Ant.}
\end{itemize}
\begin{itemize}
\item {Proveniência:(Do b. lat. \textunderscore cervella\textunderscore )}
\end{itemize}
Cabeça do barco ou a parte elevada da prôa.
\section{Cervilheira}
\begin{itemize}
\item {Grp. gram.:f.}
\end{itemize}
\begin{itemize}
\item {Proveniência:(Do b. lat. \textunderscore cervelleria\textunderscore )}
\end{itemize}
Espécie de capacete antigo.
Cervigueira.
\section{Cervino}
\begin{itemize}
\item {Grp. gram.:adj.}
\end{itemize}
\begin{itemize}
\item {Proveniência:(Lat. \textunderscore cervinus\textunderscore )}
\end{itemize}
Relativo ao cervo.
\section{Cerviola}
\begin{itemize}
\item {Grp. gram.:f.}
\end{itemize}
O mesmo que \textunderscore serviola\textunderscore . Cf. \textunderscore Eufrosina\textunderscore , 115.
\section{Cerviz}
\begin{itemize}
\item {Grp. gram.:f.}
\end{itemize}
\begin{itemize}
\item {Proveniência:(Lat. \textunderscore cervix\textunderscore )}
\end{itemize}
Cachaço.
A nuca, comprehendendo a parte posterior do pescoço.
Pescoço.
Cabeça.
\section{Cervo}
\begin{itemize}
\item {Grp. gram.:m.}
\end{itemize}
\begin{itemize}
\item {Proveniência:(Lat. \textunderscore cervus\textunderscore )}
\end{itemize}
O mesmo que \textunderscore veado\textunderscore ^1.
\section{Cérvulo}
\begin{itemize}
\item {Grp. gram.:m.}
\end{itemize}
\begin{itemize}
\item {Utilização:Zool.}
\end{itemize}
Divisão do gênero cervo, a qual comprehende as espécies, cujos cornos se sustentam num pedículo ósseo, dependente do osso coronal.
\section{Cervum}
\begin{itemize}
\item {Grp. gram.:m.  e  adj.}
\end{itemize}
\begin{itemize}
\item {Proveniência:(De \textunderscore cervo\textunderscore )}
\end{itemize}
Diz-se de certo pasto, que era dilecto dos veados, (cervos). (Colhido na Serra da Estrêlla)
\section{Cesalpíneas}
\begin{itemize}
\item {Grp. gram.:f. pl.}
\end{itemize}
Tríbo de plantas leguminosas, que têm por typo a cesalpínia.
\section{Cesalpínia}
\begin{itemize}
\item {Grp. gram.:f.}
\end{itemize}
\begin{itemize}
\item {Proveniência:(De \textunderscore Cesalpin\textunderscore , n. p.)}
\end{itemize}
Árvore leguminosa das regiões tropicaes.
\section{César}
\begin{itemize}
\item {Grp. gram.:m.}
\end{itemize}
\begin{itemize}
\item {Proveniência:(Lat. \textunderscore Caesar\textunderscore )}
\end{itemize}
Designação commum ao general e ditador romano Júlio César e aos primeiros onze soberanos que, depois de César, governaram o Império Romano.
\section{Cesáreo}
\begin{itemize}
\item {Grp. gram.:adj.}
\end{itemize}
\begin{itemize}
\item {Proveniência:(Lat. \textunderscore caesareus\textunderscore )}
\end{itemize}
Relativo aos Césares romanos.
\section{Cesariano}
\begin{itemize}
\item {Grp. gram.:m.}
\end{itemize}
\begin{itemize}
\item {Proveniência:(Lat. \textunderscore caesarianus\textunderscore )}
\end{itemize}
Relativo ao cesarismo.
\section{Cesariano}
\begin{itemize}
\item {Grp. gram.:adj.}
\end{itemize}
\begin{itemize}
\item {Proveniência:(Fr. \textunderscore cesarienne\textunderscore , do lat. \textunderscore caesus\textunderscore )}
\end{itemize}
Diz-se de uma operação cirúrgica, em certos partos.
\section{Cesarino}
\begin{itemize}
\item {Grp. gram.:adj.}
\end{itemize}
\begin{itemize}
\item {Grp. gram.:M.}
\end{itemize}
\begin{itemize}
\item {Proveniência:(Lat. \textunderscore caesarinus\textunderscore )}
\end{itemize}
O mesmo que \textunderscore cesariano\textunderscore ^1.
Soldado dos Césares romanos.
\section{Cesarismo}
\begin{itemize}
\item {Grp. gram.:m.}
\end{itemize}
\begin{itemize}
\item {Proveniência:(De \textunderscore César\textunderscore , n. p.)}
\end{itemize}
Govêrno despótico; autocracia.
Poder pessoal.
\section{Cesarista}
\begin{itemize}
\item {Grp. gram.:m.}
\end{itemize}
\begin{itemize}
\item {Proveniência:(De \textunderscore César\textunderscore , n. p.)}
\end{itemize}
Partidário do cesarismo.
\section{Césio}
\begin{itemize}
\item {Grp. gram.:m.}
\end{itemize}
\begin{itemize}
\item {Proveniência:(Lat. \textunderscore caesius\textunderscore )}
\end{itemize}
Variedade de metal azul.
\section{Céspede}
\begin{itemize}
\item {Grp. gram.:m.}
\end{itemize}
\begin{itemize}
\item {Proveniência:(Lat. \textunderscore caespes\textunderscore , \textunderscore caespitis\textunderscore )}
\end{itemize}
Torrão, com erva curta e basta.
Pedaço de relva, adherente ao torrão separado do solo.
\section{Cespitoso}
\begin{itemize}
\item {Grp. gram.:adj.}
\end{itemize}
\begin{itemize}
\item {Proveniência:(Do lat. \textunderscore caespes\textunderscore , \textunderscore caespitis\textunderscore )}
\end{itemize}
Diz-se do vegetal, que da mesma raiz lança vários troncos.
\section{Cessação}
\begin{itemize}
\item {Grp. gram.:f.}
\end{itemize}
Acto de cessar.
\section{Cessamento}
\begin{itemize}
\item {Grp. gram.:m.}
\end{itemize}
O mesmo que \textunderscore cessação\textunderscore .
\section{Cessante}
\begin{itemize}
\item {Grp. gram.:adj.}
\end{itemize}
\begin{itemize}
\item {Proveniência:(Lat. \textunderscore cessans\textunderscore )}
\end{itemize}
Que cessa.
\section{Cessão}
\begin{itemize}
\item {Grp. gram.:f.}
\end{itemize}
\begin{itemize}
\item {Proveniência:(Lat. \textunderscore cessio\textunderscore )}
\end{itemize}
Acto de ceder.
\section{Cessar}
\begin{itemize}
\item {Grp. gram.:v. i.}
\end{itemize}
\begin{itemize}
\item {Grp. gram.:V. t.}
\end{itemize}
\begin{itemize}
\item {Proveniência:(Lat. \textunderscore cessare\textunderscore )}
\end{itemize}
Parar; suspender-se: \textunderscore cessou a chuva\textunderscore .
Desistir.
Acabar.
Deixar de:«\textunderscore não cessava dar louvores\textunderscore ». Bern. da Cruz, \textunderscore Chrón. de D. Sebast.\textunderscore , c. II.
\section{Cessionário}
\begin{itemize}
\item {Grp. gram.:m.}
\end{itemize}
\begin{itemize}
\item {Proveniência:(Do lat. \textunderscore cessio\textunderscore , \textunderscore cessionis\textunderscore )}
\end{itemize}
Aquelle, a quem se faz cessão; aquelle que a acceita.
\section{Cessível}
\begin{itemize}
\item {Grp. gram.:adj.}
\end{itemize}
(V.cedível)
\section{Cesso}
\begin{itemize}
\item {Grp. gram.:m.}
\end{itemize}
Ave africana.
\section{Cêsta}
\begin{itemize}
\item {Grp. gram.:f.}
\end{itemize}
\begin{itemize}
\item {Utilização:Bras}
\end{itemize}
\begin{itemize}
\item {Proveniência:(Do lat. \textunderscore cista\textunderscore )}
\end{itemize}
Utensílio, feito geralmente de vêrga e com asa, para guardar ou transportar frutas, roupas, pequenas mercadorias, etc.
Apparelho de vime, em fórma de luva, que se calça, para jogar a péla ou pelota.
\section{Cestada}
\begin{itemize}
\item {Grp. gram.:f.}
\end{itemize}
O que se póde conter numa cêsta.
\section{Cestão}
\begin{itemize}
\item {Grp. gram.:m.}
\end{itemize}
\begin{itemize}
\item {Utilização:Náut.}
\end{itemize}
Cêsto grande, cheio de terra, empregado em fortificação.
Jangada para passagem de rios.
Parte accessória do mastro de gávea.
\section{Cestaria}
\begin{itemize}
\item {Grp. gram.:f.}
\end{itemize}
\begin{itemize}
\item {Proveniência:(De \textunderscore cêsto\textunderscore )}
\end{itemize}
Indústria de cesteiro.
Estabelecimento onde se vendem cêstos.
\section{Cêsta-rôta}
\begin{itemize}
\item {Grp. gram.:m.  e  f.}
\end{itemize}
Pessôa chocalheira, que não sabe guardar segredos.
\section{Cesteiro}
\begin{itemize}
\item {Grp. gram.:m.}
\end{itemize}
\begin{itemize}
\item {Proveniência:(Do lat. \textunderscore cistarius\textunderscore )}
\end{itemize}
Aquelle que faz ou vende cêstos.
\section{Cêsto}
\begin{itemize}
\item {Grp. gram.:m.}
\end{itemize}
\begin{itemize}
\item {Utilização:Náut.}
\end{itemize}
\begin{itemize}
\item {Proveniência:(De \textunderscore cêsta\textunderscore )}
\end{itemize}
Cêsta pequena.
Utensílio semelhante á cêsta, mas mais fundo e ás vezes com tampa.
\textunderscore Cêsto da gávea\textunderscore , plataforma horizontal, no alto de um mastro que a atravessa.
\section{Césto}
\begin{itemize}
\item {Grp. gram.:m.}
\end{itemize}
\begin{itemize}
\item {Proveniência:(Lat. \textunderscore caestus\textunderscore )}
\end{itemize}
Antiga manopla.
\section{Césto}
\begin{itemize}
\item {Grp. gram.:m.}
\end{itemize}
\begin{itemize}
\item {Utilização:Ant.}
\end{itemize}
\begin{itemize}
\item {Proveniência:(Gr. \textunderscore kestos\textunderscore )}
\end{itemize}
Cinto.
\section{Cestoide}
\begin{itemize}
\item {Grp. gram.:adj.}
\end{itemize}
\begin{itemize}
\item {Grp. gram.:Pl.}
\end{itemize}
\begin{itemize}
\item {Proveniência:(Do gr. \textunderscore kestos\textunderscore  + \textunderscore eidos\textunderscore )}
\end{itemize}
Semelhante a um cinto, a uma fita.
Vermes, da classe dos helminthos.
\section{Cestríneas}
\begin{itemize}
\item {Grp. gram.:f. pl.}
\end{itemize}
\begin{itemize}
\item {Utilização:Bot.}
\end{itemize}
Tríbo de solanáceas, que têm por tipo o cestro.
\section{Cestro}
\begin{itemize}
\item {Grp. gram.:m.}
\end{itemize}
\begin{itemize}
\item {Utilização:Bot.}
\end{itemize}
\begin{itemize}
\item {Proveniência:(Do gr. \textunderscore kestron\textunderscore )}
\end{itemize}
O mesmo que \textunderscore betónica\textunderscore .
\section{Cesúlia}
\begin{itemize}
\item {Grp. gram.:f.}
\end{itemize}
Gênero de plantas synanthéreas.
\section{Cesura}
\begin{itemize}
\item {Grp. gram.:f.}
\end{itemize}
\begin{itemize}
\item {Proveniência:(Lat. \textunderscore caesura\textunderscore )}
\end{itemize}
Acto de cortar.
Golpe de lanceta.
Cicatriz.
Primeira parte do verso hexâmetro.
Pausa, no fim do primeiro hemistíchio do verso alexandrino.
Última sýllaba de uma palavra, começando o pé de um verso latino ou grego.
\section{Cesurar}
\begin{itemize}
\item {Grp. gram.:v. t.}
\end{itemize}
\begin{itemize}
\item {Proveniência:(De \textunderscore cesura\textunderscore )}
\end{itemize}
Golpear.
\section{Ceta}
\begin{itemize}
\item {fónica:cê}
\end{itemize}
\begin{itemize}
\item {Grp. gram.:f.}
\end{itemize}
\begin{itemize}
\item {Utilização:Prov.}
\end{itemize}
\begin{itemize}
\item {Utilização:alent.}
\end{itemize}
Casta de uva encarnada.
\section{Cetáceo}
\begin{itemize}
\item {Grp. gram.:adj.}
\end{itemize}
\begin{itemize}
\item {Grp. gram.:M.}
\end{itemize}
\begin{itemize}
\item {Grp. gram.:Pl.}
\end{itemize}
\begin{itemize}
\item {Proveniência:(Do gr. \textunderscore ketos\textunderscore , grande peixe do mar)}
\end{itemize}
Relativo aos grandes mammíferos que têm fórma de peixe.
Qualquer dêsses mammíferos.
Ordem de mammíferos marítimos, a que pertence a baleia, o golfinho, etc.
\section{Ceteraque}
\begin{itemize}
\item {Grp. gram.:m.}
\end{itemize}
Fêto medicinal, (\textunderscore asplenium ceterach\textunderscore , Lin.).
(Ár. \textunderscore xetraque\textunderscore )
\section{Cetil}
\begin{itemize}
\item {Grp. gram.:m.}
\end{itemize}
Casta de uva.
\section{Cetim}
\begin{itemize}
\item {Grp. gram.:m.}
\end{itemize}
\begin{itemize}
\item {Grp. gram.:m.}
\end{itemize}
\begin{itemize}
\item {Utilização:Fig.}
\end{itemize}
\begin{itemize}
\item {Proveniência:(Do ár. \textunderscore zeituni\textunderscore )}
\end{itemize}
\begin{itemize}
\item {Proveniência:(Do ár. \textunderscore zaitunie\textunderscore ?)}
\end{itemize}
Fórma exacta, em vez de \textunderscore setim\textunderscore . Cf. \textunderscore Lusíadas\textunderscore , II, 97; Góes, \textunderscore D. Man.\textunderscore , I, 38.
Pano lustroso e fino de sêda ou lan.
Coisa macia ou suave.
O mesmo que \textunderscore pau-setim\textunderscore .
\section{Cetina}
\begin{itemize}
\item {Grp. gram.:f.}
\end{itemize}
\begin{itemize}
\item {Proveniência:(Do gr. \textunderscore ketos\textunderscore )}
\end{itemize}
Espermacete.
\section{Cetrina}
\begin{itemize}
\item {Grp. gram.:f.}
\end{itemize}
(?)«\textunderscore Para ti perdem musas a cetrina\textunderscore ». Filinto, VI, 111.
\section{Cetrino}
\begin{itemize}
\item {Grp. gram.:adj.}
\end{itemize}
O mesmo que \textunderscore vermelho\textunderscore . Cp. Moraes, \textunderscore Diccion.\textunderscore 
\section{Ceto}
\begin{itemize}
\item {Grp. gram.:m.}
\end{itemize}
\begin{itemize}
\item {Utilização:Des.}
\end{itemize}
\begin{itemize}
\item {Proveniência:(Do gr. \textunderscore ketos\textunderscore )}
\end{itemize}
Monstro marinho, cetáceo.
\section{Cetografia}
\begin{itemize}
\item {Grp. gram.:f.}
\end{itemize}
\begin{itemize}
\item {Proveniência:(Do gr. \textunderscore ketos\textunderscore  + \textunderscore graphein\textunderscore )}
\end{itemize}
Descripção dos cetáceos.
\section{Cetográfico}
\begin{itemize}
\item {Grp. gram.:adj.}
\end{itemize}
Relativo á cetografia.
\section{Cetographia}
\begin{itemize}
\item {Grp. gram.:f.}
\end{itemize}
\begin{itemize}
\item {Proveniência:(Do gr. \textunderscore ketos\textunderscore  + \textunderscore graphein\textunderscore )}
\end{itemize}
Descripção dos cetáceos.
\section{Cetográphico}
\begin{itemize}
\item {Grp. gram.:adj.}
\end{itemize}
Relativo á cetographia.
\section{Cetologia}
\begin{itemize}
\item {Grp. gram.:f.}
\end{itemize}
\begin{itemize}
\item {Proveniência:(Do gr. \textunderscore ketos\textunderscore  + \textunderscore logos\textunderscore )}
\end{itemize}
Tratado dos cetáceos.
\section{Cetológico}
\begin{itemize}
\item {Grp. gram.:adj.}
\end{itemize}
Relativo á cetologia.
\section{Cetra}
\begin{itemize}
\item {Grp. gram.:f.}
\end{itemize}
\begin{itemize}
\item {Proveniência:(Lat. \textunderscore cetra\textunderscore )}
\end{itemize}
Antigo escudo, coberto de coiro.
\section{Cetraria}
\begin{itemize}
\item {Grp. gram.:f.}
\end{itemize}
Lavores em fórma de cetras.
\section{Cetrária}
\begin{itemize}
\item {Grp. gram.:f.}
\end{itemize}
Gênero de líchens.
\section{Cetras}
\begin{itemize}
\item {Grp. gram.:f. pl.}
\end{itemize}
\begin{itemize}
\item {Proveniência:(Do lat. \textunderscore cetera\textunderscore , de \textunderscore ceterus\textunderscore )}
\end{itemize}
Traços, sinal, que imita a sigla, que representava abreviadamente um \textunderscore et cetera\textunderscore .
\section{Céu}
\begin{itemize}
\item {Grp. gram.:m.}
\end{itemize}
\begin{itemize}
\item {Utilização:Fig.}
\end{itemize}
\begin{itemize}
\item {Grp. gram.:Pl. interj.}
\end{itemize}
\begin{itemize}
\item {Proveniência:(Do lat. \textunderscore caelum\textunderscore )}
\end{itemize}
Espaço illimitado, em que se movem os astros.
Parte dêsse espaço, limitado pelo horizonte: \textunderscore o céu está nublado\textunderscore .
Atmosphera: \textunderscore a serenidade do céu\textunderscore .
Lugar, onde, segundo as crenças religiosas, estão as almas dos justos: \textunderscore minha mãe está no céu\textunderscore .
Deus.
Providência: \textunderscore demos graças ao céu\textunderscore .
\textunderscore Fogo do céu\textunderscore , o raio.
\textunderscore Um céu aberto\textunderscore , grande ventura.
\textunderscore Céu da bôca\textunderscore , o palato.
(designativa de surpresa ou dôr): \textunderscore céus! que vejo!\textunderscore 
\section{Ceva}
\begin{itemize}
\item {Grp. gram.:f.}
\end{itemize}
\begin{itemize}
\item {Utilização:Prov.}
\end{itemize}
\begin{itemize}
\item {Utilização:minh.}
\end{itemize}
\begin{itemize}
\item {Utilização:Bras}
\end{itemize}
Acção de cevar.
Alimento, com que se engordam animaes.
Porco cevado.
Lugar, onde se deitam grãos ou isca, para attrahir animaes e caçá-los.
\section{Cevada}
\begin{itemize}
\item {Grp. gram.:f.}
\end{itemize}
\begin{itemize}
\item {Proveniência:(De \textunderscore cevar\textunderscore )}
\end{itemize}
Planta cerealífera.
\section{Cevadal}
\begin{itemize}
\item {Grp. gram.:f.}
\end{itemize}
Campo de cevada.
\section{Cevadaria}
\begin{itemize}
\item {Grp. gram.:f.}
\end{itemize}
\begin{itemize}
\item {Proveniência:(De \textunderscore cevada\textunderscore )}
\end{itemize}
Antigo depósito de forragens, para as cavallariças da Casa Real.
\section{Cevadeira}
\begin{itemize}
\item {Grp. gram.:f.}
\end{itemize}
\begin{itemize}
\item {Utilização:Náut.}
\end{itemize}
\begin{itemize}
\item {Utilização:Fig.}
\end{itemize}
\begin{itemize}
\item {Utilização:Ant.}
\end{itemize}
\begin{itemize}
\item {Utilização:Bras}
\end{itemize}
\begin{itemize}
\item {Proveniência:(De \textunderscore ceva\textunderscore )}
\end{itemize}
Saco, em que se dá cevada ou outro alimento ás cavalgaduras.
Pequena vela, suspensa de uma vêrga, á prôa.
Designação depreciativa do emprêgo ou lugar, de que alguém aufere grandes e indevidos proventos.
Alforge, farnel.
Maquinismo, para ralar mandioca.
\section{Cevadeira}
\begin{itemize}
\item {Grp. gram.:f.}
\end{itemize}
\begin{itemize}
\item {Utilização:Prov.}
\end{itemize}
\begin{itemize}
\item {Utilização:alent.}
\end{itemize}
O mesmo que \textunderscore cevadal\textunderscore .
\section{Cevadeiro}
\begin{itemize}
\item {Grp. gram.:m.}
\end{itemize}
\begin{itemize}
\item {Proveniência:(De \textunderscore cevar\textunderscore )}
\end{itemize}
Empregado de cevadaria.
Lugar, onde os porcos se cevam.
Aquelle que cevava falcões para caça de altanaria.
\section{Cevadiço}
\begin{itemize}
\item {Grp. gram.:adj.}
\end{itemize}
Que se ceva.
\section{Cevadilha}
\begin{itemize}
\item {Grp. gram.:f.}
\end{itemize}
Planta melanthácea.
Semente dessa planta.
O mesmo que \textunderscore espirradeira\textunderscore  ou \textunderscore loendro\textunderscore .
(Cast. \textunderscore cebadilla\textunderscore )
\section{Cevadilheira}
\begin{itemize}
\item {Grp. gram.:f.}
\end{itemize}
O mesmo que \textunderscore cevadilha\textunderscore .
\section{Cevadinha}
\begin{itemize}
\item {Grp. gram.:f.}
\end{itemize}
\begin{itemize}
\item {Proveniência:(De \textunderscore cevada\textunderscore )}
\end{itemize}
Cevada pilada, de que se faz sopa.
\section{Cevado}
\begin{itemize}
\item {Grp. gram.:m.}
\end{itemize}
Porco, que se cevou.
\section{Cevadoiro}
\begin{itemize}
\item {Grp. gram.:m.}
\end{itemize}
\begin{itemize}
\item {Proveniência:(De \textunderscore cevar\textunderscore )}
\end{itemize}
Lugar, em que se cevam animaes.
Lugar em que se põe isca, para attrahir e caçar aves.
\section{Cevador}
\begin{itemize}
\item {Grp. gram.:m.}
\end{itemize}
\begin{itemize}
\item {Proveniência:(De \textunderscore cevar\textunderscore )}
\end{itemize}
Aquelle que trata da ceva de animaes.
\section{Cevadouro}
\begin{itemize}
\item {Grp. gram.:m.}
\end{itemize}
\begin{itemize}
\item {Proveniência:(De \textunderscore cevar\textunderscore )}
\end{itemize}
Lugar, em que se cevam animaes.
Lugar em que se põe isca, para attrahir e caçar aves.
\section{Cevadura}
\begin{itemize}
\item {Grp. gram.:f.}
\end{itemize}
\begin{itemize}
\item {Proveniência:(De \textunderscore cevar\textunderscore )}
\end{itemize}
O mesmo que \textunderscore ceva\textunderscore .
Barro, com que se cobre o açúcar, para o limpar com a água que se filtra.
Restos do alimento, destinado á ceva do falcão.
Carnificina.
\section{Cevandija}
\textunderscore f.\textunderscore  (e der.)
(V. \textunderscore sevandija\textunderscore , etc.)
\section{Cevão}
\begin{itemize}
\item {Grp. gram.:m.}
\end{itemize}
(V.cevado)
\section{Cevar}
\begin{itemize}
\item {Grp. gram.:v. t.}
\end{itemize}
\begin{itemize}
\item {Proveniência:(Do lat. \textunderscore cibare\textunderscore )}
\end{itemize}
Tornar gordo.
Nutrir.
Pôr isca em.
Engordar.
Saciar.
Enriquecer.
Fomentar.
\section{Cevar}
\begin{itemize}
\item {Grp. gram.:f.  e  adj.}
\end{itemize}
Diz-se de uma pedrinha fina e lisa, que, sobposta ás palpebras, tem a qualidade de expulsar dali quaesquer argueiros, arrastando-os consigo.
Pedra argueirinha.
Pedra de andorinha:«\textunderscore esta mulher tem pedra de cevar\textunderscore ». Camillo, \textunderscore Santo da Mont.\textunderscore , 165.
\section{Cevatício}
\begin{itemize}
\item {Grp. gram.:adj.}
\end{itemize}
Que é bom para cevar ou engordar animaes: \textunderscore plantas cevatícias\textunderscore . Cf. Barganha, \textunderscore Hyg. Pec.\textunderscore , 195 e 232.
\section{Ceveira}
\begin{itemize}
\item {Grp. gram.:f.}
\end{itemize}
\begin{itemize}
\item {Utilização:Ant.}
\end{itemize}
\begin{itemize}
\item {Proveniência:(De \textunderscore cevar\textunderscore )}
\end{itemize}
Cereaes.
\section{Cêvo}
\begin{itemize}
\item {Grp. gram.:m.}
\end{itemize}
\begin{itemize}
\item {Proveniência:(Do lat. \textunderscore cibus\textunderscore )}
\end{itemize}
Ceva.
Isca.
\section{Cf.}
(abrev. de \textunderscore conferir\textunderscore  ou \textunderscore confira\textunderscore )
\section{Chá}
\begin{itemize}
\item {Grp. gram.:m.}
\end{itemize}
\begin{itemize}
\item {Utilização:Fig.}
\end{itemize}
\begin{itemize}
\item {Proveniência:(Do chin. \textunderscore cha\textunderscore )}
\end{itemize}
Planta, da fam. das theáceas.
As fôlhas dessa planta, depois de sêcas: \textunderscore 500 grammas de chá\textunderscore .
Infusão dessas fôlhas: \textunderscore tomar chá\textunderscore .
Infusão medicinal de várias plantas: \textunderscore chá de tília\textunderscore ; \textunderscore chá de borragens\textunderscore .
Refeição, em que se serve chá: \textunderscore convidou-me para o chá\textunderscore .
Motejo indirecto; reparo reprehensivo.
\section{Chã}
\begin{itemize}
\item {Grp. gram.:f.}
\end{itemize}
Terreno plano, planície.
Carne de coxa, no talho.
(Fem. de \textunderscore chão\textunderscore )
\section{Chãmente}
\begin{itemize}
\item {Grp. gram.:adv.}
\end{itemize}
De modo chão, lhano.
\section{Chabacano}
\begin{itemize}
\item {Grp. gram.:m.}
\end{itemize}
Fruto do México, espécie de damasco.
\section{Chabaçar}
\begin{itemize}
\item {Grp. gram.:v. t.}
\end{itemize}
O mesmo que \textunderscore achabaçar\textunderscore .
\section{Chabázia}
\begin{itemize}
\item {fónica:ca}
\end{itemize}
\begin{itemize}
\item {Grp. gram.:f.}
\end{itemize}
\begin{itemize}
\item {Proveniência:(Do gr. \textunderscore khabazios\textunderscore )}
\end{itemize}
Variedade de silicato aluminoso.
\section{Chabazite}
\begin{itemize}
\item {fónica:ca}
\end{itemize}
\begin{itemize}
\item {Grp. gram.:f.}
\end{itemize}
\begin{itemize}
\item {Proveniência:(Do gr. \textunderscore khabazios\textunderscore )}
\end{itemize}
Espécie de zoólitho.
\section{Chabiana}
\begin{itemize}
\item {Grp. gram.:f.}
\end{itemize}
(V.chaviana)
\section{Chabiano}
\begin{itemize}
\item {Grp. gram.:m.  e  adj.}
\end{itemize}
(V.chaviano)
\section{Chabouco}
\begin{itemize}
\item {Grp. gram.:m.}
\end{itemize}
\begin{itemize}
\item {Utilização:Prov.}
\end{itemize}
\begin{itemize}
\item {Utilização:alent.}
\end{itemize}
\begin{itemize}
\item {Utilização:Prov.}
\end{itemize}
O mesmo que \textunderscore cabouco\textunderscore .
Grande charco; poça de água estagnada. (Colhido em Turquel)
\section{Chabouqueiro}
\begin{itemize}
\item {Grp. gram.:adj.}
\end{itemize}
\begin{itemize}
\item {Utilização:Bras. do N}
\end{itemize}
\begin{itemize}
\item {Proveniência:(De \textunderscore chabouco\textunderscore )}
\end{itemize}
Grosseiro; tôsco; mal feito.
\section{Chabraque}
\begin{itemize}
\item {Grp. gram.:m.}
\end{itemize}
Espécie de xairel, para cobrir a anca do cavallo e os coldres.
(Turco \textunderscore chabrak\textunderscore )
\section{Chabu}
\begin{itemize}
\item {Grp. gram.:m.}
\end{itemize}
\begin{itemize}
\item {Utilização:Bras}
\end{itemize}
Estoiro imprevisto, que dá o foguete, ao tocar-se-lhe.
\section{Chabuco}
\begin{itemize}
\item {Grp. gram.:m.}
\end{itemize}
\begin{itemize}
\item {Utilização:Ant.}
\end{itemize}
Chicote; acoite.
(Persa \textunderscore chabuk\textunderscore )
\section{Chaça}
\begin{itemize}
\item {Grp. gram.:f.}
\end{itemize}
\begin{itemize}
\item {Utilização:Fig.}
\end{itemize}
\begin{itemize}
\item {Utilização:Prov.}
\end{itemize}
\begin{itemize}
\item {Proveniência:(T. biscaínho)}
\end{itemize}
Lugar, onde pára a bóla, no jôgo dêste nome; sinal que marca êsse lugar.
Abalo moral.
Acto de empinar-se o cavallo.
Pequena discussão.
Contenda, briga:«\textunderscore andar em chaças\textunderscore ». Heit. Pinto.
O mesmo que \textunderscore chaço\textunderscore ^2 e \textunderscore chaço\textunderscore ^3.
\section{Chaça}
\begin{itemize}
\item {Grp. gram.:f.}
\end{itemize}
\begin{itemize}
\item {Utilização:Prov.}
\end{itemize}
\begin{itemize}
\item {Utilização:trasm.}
\end{itemize}
O mesmo que \textunderscore chaço\textunderscore ^1.
\section{Chacal}
\begin{itemize}
\item {Grp. gram.:m.}
\end{itemize}
Quadrúpede feroz, da família dos cães.
(Turc. \textunderscore chakal\textunderscore )
\section{Chaçar}
\begin{itemize}
\item {Grp. gram.:v. i.}
\end{itemize}
Dar chaça.
Têr vantagem.
\section{Chácara}
\begin{itemize}
\item {Grp. gram.:f.}
\end{itemize}
\begin{itemize}
\item {Utilização:Bras}
\end{itemize}
Quinta.
Habitação campestre, perto da cidade.
\section{Cháçara}
\begin{itemize}
\item {Grp. gram.:f.}
\end{itemize}
\begin{itemize}
\item {Utilização:Pop.}
\end{itemize}
Chalaça grosseira; pulha.
(Cast. \textunderscore cháchara\textunderscore )
\section{Chacareiro}
\begin{itemize}
\item {Grp. gram.:m.}
\end{itemize}
\begin{itemize}
\item {Utilização:Bras}
\end{itemize}
\begin{itemize}
\item {Proveniência:(De \textunderscore chácara\textunderscore )}
\end{itemize}
Administrador ou feitor de chácara.
Pequeno criador de gado.
\section{Chacarola}
\begin{itemize}
\item {Grp. gram.:f.}
\end{itemize}
\begin{itemize}
\item {Utilização:Bras}
\end{itemize}
Pequena chácara.
\section{Chacatais}
\begin{itemize}
\item {Grp. gram.:m. pl.}
\end{itemize}
Casta nobre de Mogores. Cp. Barros, \textunderscore Déc.\textunderscore  IV, l. VI, c. 1.
\section{Chacatuala}
\begin{itemize}
\item {Grp. gram.:f.}
\end{itemize}
\begin{itemize}
\item {Proveniência:(T. lund.)}
\end{itemize}
Árvore angolense, de fôlhas verde-amareladas e frutos monospermos.
\section{Chaceamento}
\begin{itemize}
\item {Grp. gram.:m.}
\end{itemize}
\begin{itemize}
\item {Utilização:Constr.}
\end{itemize}
\begin{itemize}
\item {Proveniência:(De \textunderscore chaço\textunderscore ^1)}
\end{itemize}
Certa disposição de barrotes, em alguns tectos.
\section{Chacho}
\begin{itemize}
\item {Grp. gram.:m.}
\end{itemize}
(Corr. alg. de \textunderscore sacho\textunderscore )
\section{Chacim}
\begin{itemize}
\item {Grp. gram.:m.}
\end{itemize}
\begin{itemize}
\item {Utilização:Ant.}
\end{itemize}
O mesmo que \textunderscore porco\textunderscore .
\section{Calásia}
\begin{itemize}
\item {Grp. gram.:f.}
\end{itemize}
\begin{itemize}
\item {Utilização:Med.}
\end{itemize}
\begin{itemize}
\item {Proveniência:(Do gr. \textunderscore khalasis\textunderscore )}
\end{itemize}
Separação parcial entre a córnea e a esclerótica.
\section{Calástico}
\begin{itemize}
\item {Grp. gram.:adj.}
\end{itemize}
\begin{itemize}
\item {Proveniência:(Gr. \textunderscore khalástikos\textunderscore )}
\end{itemize}
Laxativo, (falando-se de medicamentos).
\section{Calaza}
\begin{itemize}
\item {Grp. gram.:f.}
\end{itemize}
\begin{itemize}
\item {Proveniência:(Gr. \textunderscore khalaza\textunderscore )}
\end{itemize}
Ponto interior de um grão, por onde o embryão recebe o alimento.
Ponto embryonário na superfície da gema do ovo fecundado.
Cordões gelatinosos, que ligam a gema aos dois polos do ovo.
Terçol.
\section{Calazião}
\begin{itemize}
\item {Grp. gram.:m.}
\end{itemize}
\begin{itemize}
\item {Proveniência:(Gr. \textunderscore khalazion\textunderscore )}
\end{itemize}
Tumor, na borda da palpebra; terçol, chalaza.
\section{Calazóforo}
\begin{itemize}
\item {Grp. gram.:adj.}
\end{itemize}
\begin{itemize}
\item {Proveniência:(Gr. \textunderscore khalazophoros\textunderscore )}
\end{itemize}
Diz-se de uma membrana sem vasos, produzida pela primeira camada do branco do ovo, condensando-se na superfície da gema.
\section{Calcofóno}
\begin{itemize}
\item {Grp. gram.:m.}
\end{itemize}
\begin{itemize}
\item {Proveniência:(Lat. \textunderscore chalcophonos\textunderscore )}
\end{itemize}
Pedra preciosa escura, conhecida dos antigos e que tinha o som do bronze.
\section{Calcografar}
\begin{itemize}
\item {Grp. gram.:v. t.}
\end{itemize}
Gravar em metal.
(Cp. \textunderscore calcógrafo\textunderscore )
\section{Calcografia}
\begin{itemize}
\item {Grp. gram.:f.}
\end{itemize}
\begin{itemize}
\item {Proveniência:(Do gr. \textunderscore khalkos\textunderscore  + \textunderscore graphein\textunderscore )}
\end{itemize}
Arte de gravar em metal.
\section{Calcografico}
\begin{itemize}
\item {Grp. gram.:adj.}
\end{itemize}
Relativo á calcografia.
\section{Calcógrafo}
\begin{itemize}
\item {Grp. gram.:m.}
\end{itemize}
Aquele que exerce a calcografia.
\section{Calcopirite}
\begin{itemize}
\item {Grp. gram.:f.}
\end{itemize}
\begin{itemize}
\item {Proveniência:(Do gr. \textunderscore khalkos\textunderscore  + \textunderscore pur\textunderscore )}
\end{itemize}
Cobre piritoso.
Pirite de cobre.
Metal amarelo, levemente irisado.
\section{Calcotipia}
\begin{itemize}
\item {Grp. gram.:f.}
\end{itemize}
\begin{itemize}
\item {Proveniência:(Do gr. \textunderscore khalkos\textunderscore  + \textunderscore tupos\textunderscore )}
\end{itemize}
Processo de gravar em relêvo sôbre cobre.
\section{Caldaico}
\begin{itemize}
\item {Grp. gram.:adj.}
\end{itemize}
\begin{itemize}
\item {Grp. gram.:M.}
\end{itemize}
\begin{itemize}
\item {Proveniência:(Do gr. \textunderscore khaldaikos\textunderscore )}
\end{itemize}
Relativo á Caldeia.
Língua dos Caldeus.
\section{Caldaísmo}
\begin{itemize}
\item {Grp. gram.:m.}
\end{itemize}
O mesmo que \textunderscore caldeísmo\textunderscore .
\section{Caldeísmo}
\begin{itemize}
\item {Grp. gram.:m.}
\end{itemize}
Locução própria de caldeu.
\section{Caldeu}
\begin{itemize}
\item {Grp. gram.:adj.}
\end{itemize}
\begin{itemize}
\item {Grp. gram.:M.}
\end{itemize}
\begin{itemize}
\item {Proveniência:(Gr. \textunderscore khaldaios\textunderscore )}
\end{itemize}
Relativo á Caldeia, caldaico.
Habitante da Caldeia.
Língua dos Caldeus.
\section{Calibeado}
\begin{itemize}
\item {Grp. gram.:adj.}
\end{itemize}
\begin{itemize}
\item {Proveniência:(Do gr. \textunderscore khalups\textunderscore , ferro temperado)}
\end{itemize}
Diz-se dos medicamentos que contêm ferro.
\section{Calinópteros}
\begin{itemize}
\item {Grp. gram.:m. pl.}
\end{itemize}
Uma das divisões dos lepidópteros, no systema de Blanchard, na qual se comprehendem aquelles, cujas asas, durante o repoiso, tomam a posição horizontal.
\section{Camaia}
\begin{itemize}
\item {Grp. gram.:f.}
\end{itemize}
\begin{itemize}
\item {Proveniência:(Do lat. \textunderscore chama\textunderscore )}
\end{itemize}
Molusco acéfalo.
\section{Chacina}
\begin{itemize}
\item {Grp. gram.:f.}
\end{itemize}
Acto de chacinar.
Carne de porco em postas.
\section{Chacinador}
\begin{itemize}
\item {Grp. gram.:m.}
\end{itemize}
Aquelle que chacina.
\section{Chacinar}
\begin{itemize}
\item {Grp. gram.:v. t.}
\end{itemize}
\begin{itemize}
\item {Proveniência:(De \textunderscore chacim\textunderscore ?)}
\end{itemize}
Partir em postas.
Preparar e salgar (postas de carne).
Matar.
\section{Chacineiro}
\begin{itemize}
\item {Grp. gram.:m.}
\end{itemize}
Vendedor de carne de porco.
\section{Chacman}
\begin{itemize}
\item {Grp. gram.:m.}
\end{itemize}
Espécie de macaco, de cabeça parecida á do porco.
\section{Chaço}
\begin{itemize}
\item {Grp. gram.:m.}
\end{itemize}
\begin{itemize}
\item {Utilização:Náut.}
\end{itemize}
\begin{itemize}
\item {Utilização:Constr.}
\end{itemize}
\begin{itemize}
\item {Utilização:Prov.}
\end{itemize}
\begin{itemize}
\item {Utilização:dur.}
\end{itemize}
\begin{itemize}
\item {Proveniência:(Do lat. hyp. \textunderscore plateus\textunderscore ?)}
\end{itemize}
Pedaço de madeira, com que o tanoeiro aperta os arcos, apoiando-o nelles e batendo-lhe com um maço.
Uma das peças da roda do carro.
Peça, que consolída a ligação dos vaus reaes, de encontro ao calcês do mastro real.
Os barrotes que compõem o chaceamento.
Certa peça com uma depressão em duplo ângulo recto e que serve, por exemplo, para apertar os tampos de uma viola, quando esta se está fazendo.
\section{Chaço}
\begin{itemize}
\item {Grp. gram.:m.}
\end{itemize}
\begin{itemize}
\item {Utilização:Prov.}
\end{itemize}
\begin{itemize}
\item {Utilização:trasm.}
\end{itemize}
\begin{itemize}
\item {Utilização:T. do Porto}
\end{itemize}
Pechincha, conveviencia.
Calote, cão.
\section{Chaço}
\begin{itemize}
\item {Grp. gram.:m.}
\end{itemize}
\begin{itemize}
\item {Utilização:Prov.}
\end{itemize}
\begin{itemize}
\item {Utilização:trasm.}
\end{itemize}
Remendo no calcanhar das meias.
\section{Chacoina}
\begin{itemize}
\item {Grp. gram.:f.}
\end{itemize}
O mesmo que \textunderscore chacona\textunderscore . Cf. Filinto, VII, 110.
\section{Chacoli}
\begin{itemize}
\item {Grp. gram.:m.}
\end{itemize}
Aguapé de Biscaia.
(Cast. \textunderscore chacoli\textunderscore )
\section{Chacona}
\begin{itemize}
\item {Grp. gram.:f.}
\end{itemize}
\begin{itemize}
\item {Proveniência:(T. cast.)}
\end{itemize}
Ária e bailado antigo.
\section{Chacorreiro}
\begin{itemize}
\item {Grp. gram.:m.}
\end{itemize}
\begin{itemize}
\item {Utilização:Des.}
\end{itemize}
O mesmo que \textunderscore chocarreiro\textunderscore .
\section{Chacota}
\begin{itemize}
\item {Grp. gram.:f.}
\end{itemize}
Antiga canção popular.
Trovas satíricas.
Zombaria, escárneo, troça.
Antiga dança, acompanhada de canto.
(Cast. \textunderscore chacota\textunderscore )
\section{Chacoteação}
\begin{itemize}
\item {Grp. gram.:f.}
\end{itemize}
Acto de chacotear.
\section{Chacoteado}
\begin{itemize}
\item {Grp. gram.:adj.}
\end{itemize}
Que é objecto de chacota.
\section{Chacoteador}
\begin{itemize}
\item {Grp. gram.:m.}
\end{itemize}
Aquelle que chacoteia.
\section{Chacotear}
\begin{itemize}
\item {Grp. gram.:v. i.}
\end{itemize}
\begin{itemize}
\item {Proveniência:(De \textunderscore chacota\textunderscore )}
\end{itemize}
Fazer trovas burlescas e satíricas.
Fazer chacota ou zombaria.
\section{Chacoteiro}
\begin{itemize}
\item {Grp. gram.:m.}
\end{itemize}
Aquelle que chacoteia ou escarnece.
\section{Chacotice}
\begin{itemize}
\item {Grp. gram.:f.}
\end{itemize}
\begin{itemize}
\item {Utilização:Des.}
\end{itemize}
Chacota, zombaria.
\section{Chacriabás}
\begin{itemize}
\item {Grp. gram.:m. pl.}
\end{itemize}
Índios valorosos, que dominavam em Goiás, (Brasil).
\section{Chactas}
\begin{itemize}
\item {Grp. gram.:m. pl.}
\end{itemize}
Povo indígena dos Estados-Unidos da América do Norte.
\section{Chadeiro}
\begin{itemize}
\item {Grp. gram.:m.}
\end{itemize}
\begin{itemize}
\item {Utilização:Prov.}
\end{itemize}
\begin{itemize}
\item {Utilização:minh.}
\end{itemize}
O mesmo que \textunderscore chedeiro\textunderscore .
\section{Chadér}
\begin{itemize}
\item {Grp. gram.:m.}
\end{itemize}
\begin{itemize}
\item {Utilização:Ant.}
\end{itemize}
Espécie de pano indiano.
\section{Chaém}
\begin{itemize}
\item {Grp. gram.:m.}
\end{itemize}
Antigo e supremo magistrado judicial em Nanquim.
\section{Chafalhão}
\begin{itemize}
\item {Grp. gram.:m.}
\end{itemize}
Chafalho grande.
\section{Chafalheiro}
\begin{itemize}
\item {Grp. gram.:adj.}
\end{itemize}
Garrído, casquilho:«\textunderscore essa mulher tão chafalheira que está nesse camarote\textunderscore ». Filinto, XIX, 96.
\section{Chafalho}
\begin{itemize}
\item {Grp. gram.:m.}
\end{itemize}
(V.chanfalho)
\section{Chafardel}
\begin{itemize}
\item {Grp. gram.:m.}
\end{itemize}
\begin{itemize}
\item {Utilização:Prov.}
\end{itemize}
\begin{itemize}
\item {Utilização:trasm.}
\end{itemize}
\begin{itemize}
\item {Utilização:Prov.}
\end{itemize}
\begin{itemize}
\item {Utilização:alent.}
\end{itemize}
O mesmo que \textunderscore safardana\textunderscore .
O mesmo que \textunderscore rebanho\textunderscore ^1.
\section{Chafarica}
\begin{itemize}
\item {Grp. gram.:f.}
\end{itemize}
\begin{itemize}
\item {Utilização:Pop.}
\end{itemize}
Loja maçónica.
Baiuca; taberna.
\section{Chafariqueiro}
\begin{itemize}
\item {Grp. gram.:m.}
\end{itemize}
\begin{itemize}
\item {Utilização:Deprec.}
\end{itemize}
Aquelle que tem chafarica.
Mação.
\section{Chafariz}
\begin{itemize}
\item {Grp. gram.:m.}
\end{itemize}
\begin{itemize}
\item {Grp. gram.:Adj. f.}
\end{itemize}
\begin{itemize}
\item {Utilização:Prov.}
\end{itemize}
\begin{itemize}
\item {Utilização:alent.}
\end{itemize}
\begin{itemize}
\item {Proveniência:(Do ár. \textunderscore sahrij\textunderscore )}
\end{itemize}
Construcção de alvenaria, que apresenta várias bicas, por onde corre água potável.
Diz-se de uma variedade de roman.
\section{Chafarrica}
\begin{itemize}
\item {Grp. gram.:f.}
\end{itemize}
\begin{itemize}
\item {Utilização:Prov.}
\end{itemize}
\begin{itemize}
\item {Utilização:trasm.}
\end{itemize}
O mesmo que \textunderscore chafarica\textunderscore .
\section{Chafarruz}
\begin{itemize}
\item {Grp. gram.:m.}
\end{itemize}
\begin{itemize}
\item {Utilização:Ant.}
\end{itemize}
Antigo jogo de tábolas.
\section{Chafeira}
\begin{itemize}
\item {Grp. gram.:f.}
\end{itemize}
\begin{itemize}
\item {Utilização:Prov.}
\end{itemize}
\begin{itemize}
\item {Utilização:alent.}
\end{itemize}
\begin{itemize}
\item {Utilização:fam.}
\end{itemize}
Estado mórbido, durante uma convalescença.
\section{Chafreira}
\begin{itemize}
\item {Grp. gram.:f.}
\end{itemize}
\begin{itemize}
\item {Utilização:Prov.}
\end{itemize}
\begin{itemize}
\item {Utilização:alg.}
\end{itemize}
O mesmo que \textunderscore chaveira\textunderscore .
\section{Chafundar}
\begin{itemize}
\item {Grp. gram.:v. t.}
\end{itemize}
\begin{itemize}
\item {Utilização:Pop.}
\end{itemize}
Enterrar no lodo; meter no fundo da água.
\section{Chafurda}
\begin{itemize}
\item {Grp. gram.:f.}
\end{itemize}
\begin{itemize}
\item {Proveniência:(De \textunderscore chafurdar\textunderscore )}
\end{itemize}
Lamaçal, em que se atolam os porcos.
Casa immunda.
Chiqueiro.
Immundície.
\section{Chafurdar}
\begin{itemize}
\item {Grp. gram.:v. i.}
\end{itemize}
Revolver-se em lamaçal.
Tornar-se immundo; perverter-se.
\section{Chafurdeiro}
\begin{itemize}
\item {Grp. gram.:m.}
\end{itemize}
O mesmo que \textunderscore chafurda\textunderscore .
\section{Chafurdice}
\begin{itemize}
\item {Grp. gram.:f.}
\end{itemize}
Acto de chafurdar.
\section{Chafurdo}
\begin{itemize}
\item {Grp. gram.:m.}
\end{itemize}
O mesmo que \textunderscore chafurda\textunderscore .
\section{Chafurrão}
\begin{itemize}
\item {Grp. gram.:m.}
\end{itemize}
\begin{itemize}
\item {Utilização:Prov.}
\end{itemize}
\begin{itemize}
\item {Utilização:trasm.}
\end{itemize}
Grande cicatriz.
\section{Chaga}
\begin{itemize}
\item {Grp. gram.:f.}
\end{itemize}
\begin{itemize}
\item {Utilização:Fam.}
\end{itemize}
\begin{itemize}
\item {Grp. gram.:Pl.}
\end{itemize}
\begin{itemize}
\item {Utilização:Gír.}
\end{itemize}
\begin{itemize}
\item {Proveniência:(Lat. \textunderscore plaga\textunderscore )}
\end{itemize}
Ferida aberta.
Incisão na casca das árvores.
Coisa que penaliza; desgraça.
Pessôa importuna.
Planta trepadeira.
Flôr dessa planta.
Annos de degrêdo.
\section{Chagado}
\begin{itemize}
\item {Grp. gram.:adj.}
\end{itemize}
\begin{itemize}
\item {Proveniência:(Do lat. \textunderscore plagatus\textunderscore )}
\end{itemize}
Que tem chagas: \textunderscore corpo chagado\textunderscore .
\section{Chagador}
\begin{itemize}
\item {Grp. gram.:m.}
\end{itemize}
Aquelle que faz chagas ou ferimentos.
\section{Chagar}
\begin{itemize}
\item {Grp. gram.:v. t.}
\end{itemize}
Fazer chagas em.
Ferir.
Molestar; torturar.
\section{Chagaz}
\begin{itemize}
\item {Grp. gram.:m.}
\end{itemize}
Espécie de gaivina, (\textunderscore sterna angelica\textunderscore , Mont.).
\section{Chagrém}
\begin{itemize}
\item {Grp. gram.:m.}
\end{itemize}
\begin{itemize}
\item {Proveniência:(Do fr. \textunderscore chagrin\textunderscore )}
\end{itemize}
Coiro granuloso, que se prepara ordinariamente com pelles de jumento ou de macho.
\section{Chaguarçal}
\begin{itemize}
\item {Grp. gram.:m.}
\end{itemize}
\begin{itemize}
\item {Utilização:Prov.}
\end{itemize}
\begin{itemize}
\item {Utilização:trasm.}
\end{itemize}
Campo de chaguarços.
\section{Chaguarço}
\begin{itemize}
\item {Grp. gram.:m.}
\end{itemize}
\begin{itemize}
\item {Utilização:Prov.}
\end{itemize}
\begin{itemize}
\item {Utilização:trasm.}
\end{itemize}
Arbusto, que se cria espontaneamente nas faldas das serras e ao pé dos castanheiros.
\section{Chagueira}
\begin{itemize}
\item {Grp. gram.:f.}
\end{itemize}
\begin{itemize}
\item {Proveniência:(De \textunderscore chaga\textunderscore )}
\end{itemize}
O mesmo que [[chagas|chaga]], planta.
\section{Chaguento}
\begin{itemize}
\item {Grp. gram.:adj.}
\end{itemize}
O mesmo que \textunderscore chagado\textunderscore .
\section{Chaguiriçá}
\begin{itemize}
\item {Grp. gram.:m.}
\end{itemize}
Peixe marítimo do Brasil.
\section{Chaile}
\begin{itemize}
\item {Grp. gram.:m.}
\end{itemize}
O mesmo que \textunderscore chale\textunderscore ^1.
\section{Chaínha}
\begin{itemize}
\item {Grp. gram.:f.}
\end{itemize}
Espécie de maçan.
Variedade de pêra, hoje desconhecida. Cf. M. Leitão de Andrade, \textunderscore Miscellânea\textunderscore .
(Por \textunderscore cheiínha\textunderscore , de \textunderscore cheio\textunderscore ?)
\section{Chaira}
\begin{itemize}
\item {Grp. gram.:adj. f.}
\end{itemize}
\begin{itemize}
\item {Utilização:Prov.}
\end{itemize}
\begin{itemize}
\item {Utilização:trasm.}
\end{itemize}
Diz-se de uma terra fraca ou muito solta.
\section{Chala}
\begin{itemize}
\item {Grp. gram.:f.}
\end{itemize}
\begin{itemize}
\item {Utilização:Gír.}
\end{itemize}
\begin{itemize}
\item {Proveniência:(De \textunderscore chalar\textunderscore )}
\end{itemize}
Absolvição.
\section{Chalaça}
\begin{itemize}
\item {Grp. gram.:f.}
\end{itemize}
Dito de zombaria.
Phrase graciosa e satírica.
(Talvez de \textunderscore chalar\textunderscore , por \textunderscore chalrar\textunderscore )
\section{Chalaçar}
\begin{itemize}
\item {Grp. gram.:v. i.}
\end{itemize}
O mesmo que \textunderscore chalacear\textunderscore .
\section{Chalaceador}
\begin{itemize}
\item {Grp. gram.:m.}
\end{itemize}
Aquelle que chalaceia.
\section{Chalacear}
\begin{itemize}
\item {Grp. gram.:v. i.}
\end{itemize}
Dizer chalaças.
\section{Chalaceiro}
\begin{itemize}
\item {Grp. gram.:m.}
\end{itemize}
Aquelle que diz chalaças.
\section{Chala-chala}
\begin{itemize}
\item {Grp. gram.:f.}
\end{itemize}
Árvore angolense, de raiz fusiforme e aquosa, fôlhas carnosas e flôres perpendiculares ao centro das fôlhas.
\section{Chalacista}
\begin{itemize}
\item {Grp. gram.:m.}
\end{itemize}
O mesmo que \textunderscore chalaceador\textunderscore .
\section{Chalada}
\begin{itemize}
\item {Grp. gram.:adj. f.}
\end{itemize}
\begin{itemize}
\item {Proveniência:(De \textunderscore chá\textunderscore )}
\end{itemize}
Diz-se da água, misturada com infusão de chá.
\section{Chalado}
\begin{itemize}
\item {Grp. gram.:adj.}
\end{itemize}
\begin{itemize}
\item {Utilização:Gír.}
\end{itemize}
\begin{itemize}
\item {Proveniência:(De \textunderscore chalar\textunderscore )}
\end{itemize}
Amalucado.
\section{Chalana}
\begin{itemize}
\item {Grp. gram.:f.}
\end{itemize}
Barco espanhol, para transporte de mercadorias.
\section{Chalão}
\begin{itemize}
\item {Grp. gram.:m.}
\end{itemize}
Espécie de barco de serviço em obras fluviaes ou marítimas.
\section{Chalar}
\begin{itemize}
\item {Grp. gram.:v. i.}
\end{itemize}
\begin{itemize}
\item {Utilização:Gír.}
\end{itemize}
\begin{itemize}
\item {Grp. gram.:V. p.}
\end{itemize}
\begin{itemize}
\item {Proveniência:(T. caló)}
\end{itemize}
Andar, fugir.
A mesma significação.
\section{Chalásia}
\begin{itemize}
\item {fónica:ca}
\end{itemize}
\begin{itemize}
\item {Grp. gram.:f.}
\end{itemize}
\begin{itemize}
\item {Utilização:Med.}
\end{itemize}
\begin{itemize}
\item {Proveniência:(Do gr. \textunderscore khalasis\textunderscore )}
\end{itemize}
Separação parcial entre a córnea e a esclerótica.
\section{Chalástico}
\begin{itemize}
\item {fónica:ca}
\end{itemize}
\begin{itemize}
\item {Grp. gram.:adj.}
\end{itemize}
\begin{itemize}
\item {Proveniência:(Gr. \textunderscore khalástikos\textunderscore )}
\end{itemize}
Laxativo, (falando-se de medicamentos).
\section{Chalaza}
\begin{itemize}
\item {fónica:ca}
\end{itemize}
\begin{itemize}
\item {Grp. gram.:f.}
\end{itemize}
\begin{itemize}
\item {Proveniência:(Gr. \textunderscore khalaza\textunderscore )}
\end{itemize}
Ponto interior de um grão, por onde o embryão recebe o alimento.
Ponto embryonário na superfície da gema do ovo fecundado.
Cordões gelatinosos, que ligam a gema aos dois polos do ovo.
Terçol.
\section{Chalazião}
\begin{itemize}
\item {fónica:ca}
\end{itemize}
\begin{itemize}
\item {Grp. gram.:m.}
\end{itemize}
\begin{itemize}
\item {Proveniência:(Gr. \textunderscore khalazion\textunderscore )}
\end{itemize}
Tumor, na borda da palpebra; terçol, chalaza.
\section{Chalazóphoro}
\begin{itemize}
\item {fónica:ca}
\end{itemize}
\begin{itemize}
\item {Grp. gram.:adj.}
\end{itemize}
\begin{itemize}
\item {Proveniência:(Gr. \textunderscore khalazophoros\textunderscore )}
\end{itemize}
Diz-se de uma membrana sem vasos, produzida pela primeira camada do branco do ovo, condensando-se na superfície da gema.
\section{Chalcographar}
\begin{itemize}
\item {fónica:cal}
\end{itemize}
Gravar em metal.
(Cp. \textunderscore chalcógrapho\textunderscore )
\section{Chalcographia}
\begin{itemize}
\item {fónica:cal}
\end{itemize}
\begin{itemize}
\item {Grp. gram.:f.}
\end{itemize}
\begin{itemize}
\item {Proveniência:(Do gr. \textunderscore khalkos\textunderscore  + \textunderscore graphein\textunderscore )}
\end{itemize}
Arte de gravar em metal.
\section{Chalcographico}
\begin{itemize}
\item {fónica:cal}
\end{itemize}
\begin{itemize}
\item {Grp. gram.:adj.}
\end{itemize}
Relativo á chalcographia.
\section{Chalcógrapho}
\begin{itemize}
\item {fónica:cal}
\end{itemize}
\begin{itemize}
\item {Grp. gram.:m.}
\end{itemize}
Aquelle que exerce a chalcographia.
\section{Chalcophóno}
\begin{itemize}
\item {fónica:cal}
\end{itemize}
\begin{itemize}
\item {Grp. gram.:m.}
\end{itemize}
\begin{itemize}
\item {Proveniência:(Lat. \textunderscore chalcophonos\textunderscore )}
\end{itemize}
Pedra preciosa escura, conhecida dos antigos e que tinha o som do bronze.
\section{Chalcopyrite}
\begin{itemize}
\item {fónica:cal}
\end{itemize}
\begin{itemize}
\item {Grp. gram.:f.}
\end{itemize}
\begin{itemize}
\item {Proveniência:(Do gr. \textunderscore khalkos\textunderscore  + \textunderscore pur\textunderscore )}
\end{itemize}
Cobre pyritoso.
Pyrite de cobre.
Metal amarelo, levemente irisado.
\section{Chalcotypia}
\begin{itemize}
\item {fónica:cal}
\end{itemize}
\begin{itemize}
\item {Grp. gram.:f.}
\end{itemize}
\begin{itemize}
\item {Proveniência:(Do gr. \textunderscore khalkos\textunderscore  + \textunderscore tupos\textunderscore )}
\end{itemize}
Processo de gravar em relêvo sôbre cobre.
\section{Chaldaico}
\begin{itemize}
\item {fónica:cal}
\end{itemize}
\begin{itemize}
\item {Grp. gram.:adj.}
\end{itemize}
\begin{itemize}
\item {Grp. gram.:M.}
\end{itemize}
\begin{itemize}
\item {Proveniência:(Do gr. \textunderscore khaldaikos\textunderscore )}
\end{itemize}
Relativo á Chaldeia.
Língua dos Chaldeus.
\section{Chaldaísmo}
\begin{itemize}
\item {fónica:cal}
\end{itemize}
\begin{itemize}
\item {Grp. gram.:m.}
\end{itemize}
O mesmo que \textunderscore chaldeísmo\textunderscore .
\section{Chaldeísmo}
\begin{itemize}
\item {fónica:cal}
\end{itemize}
\begin{itemize}
\item {Grp. gram.:m.}
\end{itemize}
Locução própria de chaldeu.
\section{Chaldeu}
\begin{itemize}
\item {fónica:cál}
\end{itemize}
\begin{itemize}
\item {Grp. gram.:adj.}
\end{itemize}
\begin{itemize}
\item {Grp. gram.:M.}
\end{itemize}
\begin{itemize}
\item {Proveniência:(Gr. \textunderscore khaldaios\textunderscore )}
\end{itemize}
Relativo á Chaldeia, chaldaico.
Habitante da Chaldeia.
Língua dos Chaldeus.
\section{Chaldrar}
\begin{itemize}
\item {Grp. gram.:v. i.}
\end{itemize}
\begin{itemize}
\item {Utilização:Prov.}
\end{itemize}
\begin{itemize}
\item {Utilização:trasm.}
\end{itemize}
Convir, quadrar, aprazer: \textunderscore essa condição não me chaldra\textunderscore . (Colhido em Bragança)
\section{Chale}
\begin{itemize}
\item {Grp. gram.:m.}
\end{itemize}
Peça de vestuário, que no Oriente tem várias applicações, e que na Europa é usado pelas mulheres como ornato ou como agasalho dos ombros e tronco.
(Do persa)
\section{Chale}
\begin{itemize}
\item {Grp. gram.:m.}
\end{itemize}
Peixe de Portugal.
\section{Chalé}
\begin{itemize}
\item {Grp. gram.:m.}
\end{itemize}
\begin{itemize}
\item {Proveniência:(Fr. \textunderscore chalet\textunderscore )}
\end{itemize}
Casa campestre, como a usam aldeões Suíços.
Construcção caprichosa e ligeira, para sêr habitada especialmente na estação calmosa.
Casa rústica.
\section{Chalé}
\begin{itemize}
\item {Grp. gram.:m.}
\end{itemize}
Moradía de artífices, em palmar indiano.
(Indost. \textunderscore chalé\textunderscore )
\section{Chaleira}
\begin{itemize}
\item {Grp. gram.:f.}
\end{itemize}
\begin{itemize}
\item {Utilização:Gír.}
\end{itemize}
\begin{itemize}
\item {Proveniência:(De \textunderscore chá\textunderscore )}
\end{itemize}
Vaso de metal, em que se aquece água, ordinariamente para o cha.
Nádegas.
\section{Chaleira}
\begin{itemize}
\item {Grp. gram.:f.}
\end{itemize}
(V.cheleira)
\section{Chaleira}
\begin{itemize}
\item {Grp. gram.:f.}
\end{itemize}
\begin{itemize}
\item {Utilização:T. do Ribatejo}
\end{itemize}
Mulher, que da Beira vem trabalhar no Alentejo.
(Por \textunderscore chelleira\textunderscore , de \textunderscore Chellas\textunderscore , n. p. top.?)
\section{Chaliço}
\begin{itemize}
\item {Grp. gram.:m.}
\end{itemize}
\begin{itemize}
\item {Utilização:Prov.}
\end{itemize}
Pequeno robalo.
\section{Chalinópteros}
\begin{itemize}
\item {fónica:ca}
\end{itemize}
\begin{itemize}
\item {Grp. gram.:m. pl.}
\end{itemize}
Uma das divisões dos lepidópteros, no systema de Blanchard, na qual se comprehendem aquelles, cujas asas, durante o repoiso, tomam a posição horizontal.
\section{Chalo}
\begin{itemize}
\item {Grp. gram.:m.}
\end{itemize}
\begin{itemize}
\item {Utilização:Bras}
\end{itemize}
Leito de varas.
\section{Chalocas}
\begin{itemize}
\item {Grp. gram.:f. pl.}
\end{itemize}
\begin{itemize}
\item {Utilização:Prov.}
\end{itemize}
\begin{itemize}
\item {Utilização:alg.}
\end{itemize}
Sapatos de ourelo, com rastos de pau.
\section{Chalorda}
\begin{itemize}
\item {Grp. gram.:f.}
\end{itemize}
\begin{itemize}
\item {Utilização:Prov.}
\end{itemize}
\begin{itemize}
\item {Utilização:beir.}
\end{itemize}
Pequena terra cultivada, horta, leira. (Colhido em Mortágua)
\section{Chalota}
\begin{itemize}
\item {Grp. gram.:f.}
\end{itemize}
\begin{itemize}
\item {Proveniência:(Fr. \textunderscore echalotte\textunderscore )}
\end{itemize}
Planta hortense, da fam. das liliáceas.
\section{Chalotas}
\begin{itemize}
\item {Grp. gram.:f. pl.}
\end{itemize}
\begin{itemize}
\item {Utilização:Prov.}
\end{itemize}
O mesmo que \textunderscore chalocas\textunderscore .
\section{Chalotinha}
\begin{itemize}
\item {Grp. gram.:f.}
\end{itemize}
(V.chalota)
\section{Chalrar}
\textunderscore v. i.\textunderscore  (e der.)
O mesmo que \textunderscore chalrear\textunderscore , etc.
\section{Chalreada}
\begin{itemize}
\item {Grp. gram.:f.}
\end{itemize}
\begin{itemize}
\item {Proveniência:(De \textunderscore chalrear\textunderscore )}
\end{itemize}
Ruído simultâneo de muitas vozes.
Falario.
\section{Chalreador}
\begin{itemize}
\item {Grp. gram.:adj.}
\end{itemize}
Que chalreia.
\section{Chalrear}
\begin{itemize}
\item {Grp. gram.:v. i.}
\end{itemize}
Falar á tôa, alegremente, com voz estrídula, juntamente com outras pessôas.
Chilrear.
Soltar vozes inarticuladas, (falando-se de crianças).
(Variante de \textunderscore chilrear\textunderscore ?)
\section{Chalreio}
\begin{itemize}
\item {Grp. gram.:m.}
\end{itemize}
O mesmo que \textunderscore chalreada\textunderscore .
\section{Chalreta}
\begin{itemize}
\item {fónica:rê}
\end{itemize}
\begin{itemize}
\item {Grp. gram.:f.}
\end{itemize}
Ave pernalta, o mesmo que \textunderscore fuselo\textunderscore .
\section{Chalrote}
\begin{itemize}
\item {Grp. gram.:m.}
\end{itemize}
\begin{itemize}
\item {Utilização:Prov.}
\end{itemize}
Casca de pinheiro, corcódea.
\section{Chalupa}
\begin{itemize}
\item {Grp. gram.:f.}
\end{itemize}
\begin{itemize}
\item {Grp. gram.:Pl.}
\end{itemize}
\begin{itemize}
\item {Utilização:Gír.}
\end{itemize}
Pequena embarcação de um só mastro, para navegação de cabotagem.
Barco de vela e remos.
No voltarete, as três cartas de mais valor.
Botas.
(Hol. \textunderscore sloep\textunderscore )
\section{Chalybeado}
\begin{itemize}
\item {fónica:ca}
\end{itemize}
\begin{itemize}
\item {Grp. gram.:adj.}
\end{itemize}
\begin{itemize}
\item {Proveniência:(Do gr. \textunderscore khalups\textunderscore , ferro temperado)}
\end{itemize}
Diz-se dos medicamentos que contêm ferro.
\section{Chama}
\begin{itemize}
\item {Grp. gram.:f.}
\end{itemize}
\begin{itemize}
\item {Utilização:T. de Cezimbra}
\end{itemize}
O mesmo que \textunderscore chamada\textunderscore .
Negaça; chamariz.
Pequeno pau ou cacete.
\section{Chamada}
\begin{itemize}
\item {Grp. gram.:f.}
\end{itemize}
Acto de chamar.
Sinal para chamar.
Braçado de lenha.
\section{Chamadeira}
\begin{itemize}
\item {Grp. gram.:f.}
\end{itemize}
\begin{itemize}
\item {Proveniência:(De \textunderscore chamar\textunderscore )}
\end{itemize}
Bagalhão de linho, quando, maduro, começa a abrir-se.
O mesmo que \textunderscore chamariz\textunderscore . Cf. \textunderscore Bibl. da G. do Campo\textunderscore , 298.
\section{Chamado}
\begin{itemize}
\item {Grp. gram.:m.}
\end{itemize}
\begin{itemize}
\item {Utilização:T. do Marão}
\end{itemize}
Chamamento, chamada.
Assembleia local.
\section{Chamadoiro}
\begin{itemize}
\item {Grp. gram.:m.}
\end{itemize}
\begin{itemize}
\item {Utilização:Prov.}
\end{itemize}
\begin{itemize}
\item {Utilização:dur.}
\end{itemize}
\begin{itemize}
\item {Utilização:minh.}
\end{itemize}
Taramela do moínho.
\section{Chamador}
\begin{itemize}
\item {Grp. gram.:m.}
\end{itemize}
Aquelle que chama.
\section{Chamadouro}
\begin{itemize}
\item {Grp. gram.:m.}
\end{itemize}
\begin{itemize}
\item {Utilização:Prov.}
\end{itemize}
\begin{itemize}
\item {Utilização:dur.}
\end{itemize}
\begin{itemize}
\item {Utilização:minh.}
\end{itemize}
Taramela do moínho.
\section{Chamadura}
\begin{itemize}
\item {Grp. gram.:f.}
\end{itemize}
(V.chamamento)
\section{Chamaia}
\begin{itemize}
\item {fónica:ca}
\end{itemize}
\begin{itemize}
\item {Grp. gram.:f.}
\end{itemize}
\begin{itemize}
\item {Proveniência:(Do lat. \textunderscore chama\textunderscore )}
\end{itemize}
Mollusco acéphalo.
\section{Chamalote}
\begin{itemize}
\item {Grp. gram.:m.}
\end{itemize}
Tecido de pêlo ou lan, geralmente com seda.
(B. lat. \textunderscore camelotum\textunderscore )
\section{Chamamento}
\begin{itemize}
\item {Grp. gram.:m.}
\end{itemize}
O mesmo que \textunderscore chamada\textunderscore .
\section{Chaman}
\begin{itemize}
\item {Grp. gram.:m.}
\end{itemize}
\begin{itemize}
\item {Proveniência:(Fr. \textunderscore chaman\textunderscore )}
\end{itemize}
Nome do mágico, que pratíca o chamanismo.
\section{Chamancada}
\begin{itemize}
\item {Grp. gram.:f.}
\end{itemize}
\begin{itemize}
\item {Utilização:Prov.}
\end{itemize}
\begin{itemize}
\item {Utilização:trasm.}
\end{itemize}
\begin{itemize}
\item {Utilização:fam.}
\end{itemize}
Patetice, cabeçada.
\section{Chamanismo}
\begin{itemize}
\item {Grp. gram.:m.}
\end{itemize}
\begin{itemize}
\item {Proveniência:(De \textunderscore chaman\textunderscore )}
\end{itemize}
Prática de exorcismos e evocações, systema de magia, usado por selvagens, principalmente entre os Samoiedos, Turcomanos, etc.
\section{Chamar}
\begin{itemize}
\item {Grp. gram.:v. t.}
\end{itemize}
\begin{itemize}
\item {Grp. gram.:V. i.}
\end{itemize}
\begin{itemize}
\item {Grp. gram.:V. p.}
\end{itemize}
\begin{itemize}
\item {Proveniência:(Do lat. \textunderscore clamare\textunderscore )}
\end{itemize}
Dar sinal, chamar, para que (alguém) venha ou se aproxime.
Convocar.
Attrahir por engano.
Escolher, para desempenhar um cargo.
Invocar.
Appelidar.
Attrahir, impellir para si.
Clamar, para que alguém venha.
Têr o nome de: \textunderscore meu pai chama-se Joaquim\textunderscore .
Dar-se o nome de:«\textunderscore a esta sýnthese chama-se leitura auricular, como escrita auricular se chamara áquella anályse\textunderscore ». Castilho.--Há quem duvide da legitimidade desta última construcção syntáctica.
\section{Chamarilho}
\begin{itemize}
\item {Grp. gram.:m.}
\end{itemize}
\begin{itemize}
\item {Utilização:Prov.}
\end{itemize}
\begin{itemize}
\item {Utilização:alent.}
\end{itemize}
O mesmo que \textunderscore chamariz\textunderscore .
\section{Chama-rita}
\begin{itemize}
\item {Grp. gram.:f.}
\end{itemize}
\begin{itemize}
\item {Proveniência:(De \textunderscore chamar\textunderscore  + \textunderscore Rita\textunderscore , n. p.)}
\end{itemize}
Música popular nos Açores.
\section{Chamariz}
\begin{itemize}
\item {Grp. gram.:m.}
\end{itemize}
\begin{itemize}
\item {Grp. gram.:F.}
\end{itemize}
\begin{itemize}
\item {Proveniência:(De \textunderscore chamar\textunderscore )}
\end{itemize}
Coisa que chama; reclamo.
Aquillo que attrai.
Pequena ave, o mesmo que \textunderscore milheira\textunderscore .
\section{Chamarra}
\begin{itemize}
\item {Grp. gram.:f.}
\end{itemize}
Espécie de batina, sem mangas e de estôfo ordinário; o mesmo que \textunderscore chimarra\textunderscore .
(Cast. \textunderscore chamarra\textunderscore )
\section{Chamás}
\begin{itemize}
\item {Grp. gram.:m.}
\end{itemize}
Aquelle que tem ordens ecclesiásticas, abaixo de presbýtero, no Malabar.
\section{Chamatão}
\begin{itemize}
\item {Grp. gram.:m.}
\end{itemize}
\begin{itemize}
\item {Utilização:Ant.}
\end{itemize}
Acto de chamar a attenção com clamores ou alaridos. Cf. Campos Junior, \textunderscore Camões\textunderscore , c. XII.
\section{Camecéfalo}
\begin{itemize}
\item {Grp. gram.:adj.}
\end{itemize}
\begin{itemize}
\item {Proveniência:(Do gr. \textunderscore khamai\textunderscore  + \textunderscore kephale\textunderscore )}
\end{itemize}
Que tem a cabeça abatida, humilde.
\section{Chama}
\begin{itemize}
\item {Grp. gram.:f.}
\end{itemize}
\begin{itemize}
\item {Utilização:Fig.}
\end{itemize}
\begin{itemize}
\item {Proveniência:(Lat. \textunderscore flamma\textunderscore )}
\end{itemize}
Porção de luz, ou espécie de auréola luminosa, que se eleva de matérias incendiadas, e é resultante da combustão dos gases produzidos por essas matérias.
Luz.
Labareda.
Ardor, paixões: \textunderscore a chamma do amor\textunderscore .
\section{Chamarela}
\begin{itemize}
\item {Grp. gram.:f.}
\end{itemize}
\begin{itemize}
\item {Utilização:Pop.}
\end{itemize}
\begin{itemize}
\item {Proveniência:(De \textunderscore chamma\textunderscore , sob a infl. de \textunderscore labareda\textunderscore )}
\end{itemize}
Incêndio.
\section{Chambaçal}
\begin{itemize}
\item {Grp. gram.:m.}
\end{itemize}
Variedade de arroz asiático.
\section{Chambandela}
\begin{itemize}
\item {Grp. gram.:f.}
\end{itemize}
\begin{itemize}
\item {Utilização:Prov.}
\end{itemize}
\begin{itemize}
\item {Utilização:alg.}
\end{itemize}
Cambalhota.
(Cp. \textunderscore chambão\textunderscore )
\section{Chambão}
\begin{itemize}
\item {Grp. gram.:m.}
\end{itemize}
\begin{itemize}
\item {Grp. gram.:Adj.}
\end{itemize}
\begin{itemize}
\item {Utilização:Pop.}
\end{itemize}
\begin{itemize}
\item {Proveniência:(Fr. \textunderscore jambon\textunderscore ?)}
\end{itemize}
Carne de má qualidade.
Contrapêso na venda da carne.
Grosseiro, mal educado.
\section{Chambaril}
\begin{itemize}
\item {Grp. gram.:m.}
\end{itemize}
\begin{itemize}
\item {Proveniência:(De \textunderscore chambão\textunderscore )}
\end{itemize}
Pau curvo, que se enfia nos jarretes do porco, para o pendurar e para sêr aberto.
\section{Chambas}
\begin{itemize}
\item {Grp. gram.:m.}
\end{itemize}
\begin{itemize}
\item {Utilização:Prov.}
\end{itemize}
\begin{itemize}
\item {Utilização:trasm.}
\end{itemize}
Homem lorpa e desajeitado; labrego.
(Cp. \textunderscore chambão\textunderscore )
\section{Chambetas}
\begin{itemize}
\item {fónica:bê}
\end{itemize}
\begin{itemize}
\item {Grp. gram.:f. pl.}
\end{itemize}
\begin{itemize}
\item {Utilização:Prov.}
\end{itemize}
\begin{itemize}
\item {Utilização:minh.}
\end{itemize}
O mesmo que \textunderscore gambetas\textunderscore .
\section{Chambo}
\begin{itemize}
\item {Grp. gram.:m.}
\end{itemize}
\begin{itemize}
\item {Utilização:T. da Afr. Or. port}
\end{itemize}
O mesmo que \textunderscore cânhamo\textunderscore .
\section{Chamboíce}
\begin{itemize}
\item {Grp. gram.:f.}
\end{itemize}
Qualidade do que é chambão, grosseiro.
Bordado grosseiro.
\section{Chamboqueiro}
\begin{itemize}
\item {Grp. gram.:adj.}
\end{itemize}
\begin{itemize}
\item {Utilização:Bras}
\end{itemize}
Grosseiro; tôsco.
(Cp. \textunderscore chambão\textunderscore )
\section{Chambordista}
\begin{itemize}
\item {Grp. gram.:m.  e  adj.}
\end{itemize}
Partidário político do Conde de Chambord, em França. Cf. Th. Ribeiro, I, \textunderscore Jornadas\textunderscore , 242.
\section{Chamborgas}
\begin{itemize}
\item {Grp. gram.:m.}
\end{itemize}
Fanfarrão.
\section{Chamborreirão}
\begin{itemize}
\item {Grp. gram.:m.  e  adj.}
\end{itemize}
\begin{itemize}
\item {Utilização:Prov.}
\end{itemize}
\begin{itemize}
\item {Utilização:alent.}
\end{itemize}
Diz-se do artífice ou official, que só produz obras grosseiras ou mal acabadas.
(Cp. \textunderscore chambão\textunderscore )
\section{Chambrana}
\begin{itemize}
\item {Grp. gram.:f.}
\end{itemize}
\begin{itemize}
\item {Utilização:Ant.}
\end{itemize}
Moldura, em volta de uma porta ou janela.
(Cp. fr. \textunderscore chambranle\textunderscore )
\section{Chambre}
\begin{itemize}
\item {Grp. gram.:m.}
\end{itemize}
\begin{itemize}
\item {Proveniência:(Fr. \textunderscore chambre\textunderscore )}
\end{itemize}
Roupão caseiro, para homem ou mulher.
Casaco para mulher, branco e leve, de uso doméstico.
\section{Chambrié}
\begin{itemize}
\item {Grp. gram.:m.}
\end{itemize}
\begin{itemize}
\item {Proveniência:(Do fr. \textunderscore chambrière\textunderscore )}
\end{itemize}
Chicote comprido e leve, usado por picadores.
\section{Chambuco}
\begin{itemize}
\item {Grp. gram.:m.}
\end{itemize}
\begin{itemize}
\item {Utilização:Des.}
\end{itemize}
\begin{itemize}
\item {Proveniência:(T. ind.)}
\end{itemize}
Chicote.
\section{Chameante}
\begin{itemize}
\item {Grp. gram.:adj.}
\end{itemize}
Que chameia.
\section{Chamear}
\begin{itemize}
\item {Grp. gram.:v. i.}
\end{itemize}
O mesmo que \textunderscore chamejar\textunderscore .
\section{Chamecéphalo}
\begin{itemize}
\item {fónica:ca}
\end{itemize}
\begin{itemize}
\item {Grp. gram.:adj.}
\end{itemize}
\begin{itemize}
\item {Proveniência:(Do gr. \textunderscore khamai\textunderscore  + \textunderscore kephale\textunderscore )}
\end{itemize}
Que tem a cabeça abatida, humilde.
\section{Chamedris}
\begin{itemize}
\item {Grp. gram.:f.}
\end{itemize}
O mesmo que \textunderscore carvalhinha\textunderscore .
\section{Chamego}
\begin{itemize}
\item {fónica:mê}
\end{itemize}
\begin{itemize}
\item {Grp. gram.:m.}
\end{itemize}
\begin{itemize}
\item {Utilização:Bras. do N}
\end{itemize}
\begin{itemize}
\item {Proveniência:(De \textunderscore chamar\textunderscore ?)}
\end{itemize}
Namôro.
Amizade muito estreita.
\section{Chameira}
\begin{itemize}
\item {Grp. gram.:f.}
\end{itemize}
\begin{itemize}
\item {Utilização:Ant.}
\end{itemize}
Mulher que leva o pão ao forno e, depois de cozido, o traz de lá.
(Por \textunderscore chammeira\textunderscore , de \textunderscore chamma\textunderscore ?)
\section{Chamejamento}
\begin{itemize}
\item {Grp. gram.:m.}
\end{itemize}
Acto de chamejar ou de passar pelas chamas qualquer objecto para o desinfectar.
\section{Chamejante}
\begin{itemize}
\item {Grp. gram.:adj.}
\end{itemize}
Que chameja.
\section{Chamejar}
\begin{itemize}
\item {Grp. gram.:v. i.}
\end{itemize}
\begin{itemize}
\item {Grp. gram.:V. t.}
\end{itemize}
\begin{itemize}
\item {Proveniência:(De \textunderscore chamma\textunderscore )}
\end{itemize}
Deitar chamas.
Arder.
Brilhar.
Dardejar.
Expedir como chamas.
Passar pelas chamas (um objecto), para desinfectar.
\section{Chamejo}
\begin{itemize}
\item {Grp. gram.:m.}
\end{itemize}
Acto de chamejar.
\section{Chameli}
\begin{itemize}
\item {Grp. gram.:m.}
\end{itemize}
Árvore de Damão.
\section{Chamelote}
\begin{itemize}
\item {Grp. gram.:m.}
\end{itemize}
(V.chamalote)
\section{Chamepite}
\begin{itemize}
\item {Grp. gram.:m.}
\end{itemize}
Variedade de uva; bastardo.
\section{Chamiça}
\begin{itemize}
\item {Grp. gram.:f.}
\end{itemize}
\begin{itemize}
\item {Utilização:Prov.}
\end{itemize}
Variedade de junco.
Corda, com que se ligam alcatruzes.
Carqueja.
Chamiço.
(Cast. \textunderscore chamiza\textunderscore )
\section{Chamiceiro}
\begin{itemize}
\item {Grp. gram.:m.}
\end{itemize}
\begin{itemize}
\item {Utilização:Prov.}
\end{itemize}
\begin{itemize}
\item {Utilização:beir.}
\end{itemize}
Aquelle que apanha e vende chamiço.
Aquelle que mete a lenha no forno.
\section{Chamiço}
\begin{itemize}
\item {Grp. gram.:m.}
\end{itemize}
\begin{itemize}
\item {Utilização:Prov.}
\end{itemize}
\begin{itemize}
\item {Utilização:alent.}
\end{itemize}
Accendalhas.
Lenha miúda.
Ramos sêcos.
Tição.
Porco magro.
(Cast. \textunderscore chamizo\textunderscore )
\section{Chaminé}
\begin{itemize}
\item {Grp. gram.:f.}
\end{itemize}
\begin{itemize}
\item {Proveniência:(Fr. \textunderscore cheminé\textunderscore )}
\end{itemize}
Tubo redondo ou de outra fórma, que, dando tiragem ao ar, dá saída ao fumo.
Parte inclinada da parede, por onde o fumo da cozinha vai ao tubo que o leva acima do edifício.
Lugar, onde se accende lume.
Lareira.
Fogão de sala.
Tubo, que estabelece a tiragem do ar nos candeeiros.
Ventilador.
Parte do cachimbo, onde arde o tabaco.
\section{Chamíssoa}
\begin{itemize}
\item {Grp. gram.:f.}
\end{itemize}
\begin{itemize}
\item {Proveniência:(De \textunderscore Chamisso\textunderscore , n. p.)}
\end{itemize}
Gênero de plantas amarantháceas das regiões tropicaes.
\section{Chamma}
\begin{itemize}
\item {Grp. gram.:f.}
\end{itemize}
\begin{itemize}
\item {Utilização:Fig.}
\end{itemize}
\begin{itemize}
\item {Proveniência:(Lat. \textunderscore flamma\textunderscore )}
\end{itemize}
Porção de luz, ou espécie de auréola luminosa, que se eleva de matérias incendiadas, e é resultante da combustão dos gases produzidos por essas matérias.
Luz.
Labareda.
Ardor, paixões: \textunderscore a chamma do amor\textunderscore .
\section{Chammarela}
\begin{itemize}
\item {Grp. gram.:f.}
\end{itemize}
\begin{itemize}
\item {Utilização:Pop.}
\end{itemize}
\begin{itemize}
\item {Proveniência:(De \textunderscore chamma\textunderscore , sob a infl. de \textunderscore labareda\textunderscore )}
\end{itemize}
Incêndio.
\section{Chammeante}
\begin{itemize}
\item {Grp. gram.:adj.}
\end{itemize}
Que chammeia.
\section{Chammear}
\begin{itemize}
\item {Grp. gram.:v. i.}
\end{itemize}
O mesmo que \textunderscore chammejar\textunderscore .
\section{Chammejamento}
\begin{itemize}
\item {Grp. gram.:m.}
\end{itemize}
Acto de chammejar ou de passar pelas chammas qualquer objecto para o desinfectar.
\section{Chammejante}
\begin{itemize}
\item {Grp. gram.:adj.}
\end{itemize}
Que chammeja.
\section{Chammejar}
\begin{itemize}
\item {Grp. gram.:v. i.}
\end{itemize}
\begin{itemize}
\item {Grp. gram.:V. t.}
\end{itemize}
\begin{itemize}
\item {Proveniência:(De \textunderscore chamma\textunderscore )}
\end{itemize}
Deitar chammas.
Arder.
Brilhar.
Dardejar.
Expedir como chammas.
Passar pelas chammas (um objecto), para desinfectar.
\section{Chammejo}
\begin{itemize}
\item {Grp. gram.:m.}
\end{itemize}
Acto de chammejar.
\section{Chamo}
\begin{itemize}
\item {Grp. gram.:m.}
\end{itemize}
\begin{itemize}
\item {Utilização:Prov.}
\end{itemize}
Chamada, chamamento: \textunderscore acudir ao chamo de alguém\textunderscore .
Chamariz.
\section{Chamorro}
\begin{itemize}
\item {fónica:mô}
\end{itemize}
\begin{itemize}
\item {Grp. gram.:m.  e  adj.}
\end{itemize}
Tosquiado.
Designação injuriosa, que os Espanhóis deram aos Portugueses.
Epítheto depreciativo, que os Realistas de 1828 deram aos Constitucionaes.
(Cast. \textunderscore chamorro\textunderscore )
\section{Chamotim}
\begin{itemize}
\item {Grp. gram.:m.}
\end{itemize}
O mesmo que \textunderscore cafuné\textunderscore .
\section{Champa}
\begin{itemize}
\item {Grp. gram.:f.}
\end{itemize}
Prancha da espada.
(Corr. de \textunderscore chapa\textunderscore ?)
\section{Champaca}
\begin{itemize}
\item {Grp. gram.:f.}
\end{itemize}
Planta brasileira ornamental.
\section{Champana}
\begin{itemize}
\item {Grp. gram.:f.}
\end{itemize}
Embarcação da Índia.
\section{Champanha}
\begin{itemize}
\item {Grp. gram.:m.}
\end{itemize}
Vinho espumoso de Champagne, em França. Cf. Filinto, II, 139; III, 303; VIII, 128; Castilho, \textunderscore Fausto\textunderscore , 169.
\section{Champanhe}
\begin{itemize}
\item {Grp. gram.:m.}
\end{itemize}
O mesmo que \textunderscore champanha\textunderscore ^1. Cf. Castilho, \textunderscore Avarento\textunderscore , 183.
\section{Champanhizar}
\begin{itemize}
\item {Grp. gram.:v. t.}
\end{itemize}
Dar aspectos ou qualidades de champanha a: \textunderscore vinho champanhizado\textunderscore .
\section{Champão}
\begin{itemize}
\item {Grp. gram.:m.}
\end{itemize}
Champana grande. Cf. \textunderscore Peregrinação\textunderscore , c. CLIII.
\section{Champás}
\begin{itemize}
\item {Grp. gram.:m. pl.}
\end{itemize}
Antigo povo da Malásia.
\section{Champil}
\begin{itemize}
\item {Grp. gram.:m.}
\end{itemize}
\begin{itemize}
\item {Utilização:Prov.}
\end{itemize}
\begin{itemize}
\item {Utilização:alent.}
\end{itemize}
\begin{itemize}
\item {Proveniência:(De \textunderscore champa\textunderscore )}
\end{itemize}
Pedaço de cortiça, em que se poisa o pombo que serve de negaça, na caça aos pombos bravos.
\section{Champilha}
\begin{itemize}
\item {Grp. gram.:m.}
\end{itemize}
\begin{itemize}
\item {Utilização:Prov.}
\end{itemize}
\begin{itemize}
\item {Utilização:trasm.}
\end{itemize}
Homem fraco e pobre.
\section{Champló}
\begin{itemize}
\item {Grp. gram.:m.}
\end{itemize}
Árvore de Timor.
O mesmo que \textunderscore champó\textunderscore ?
\section{Champó}
\begin{itemize}
\item {Grp. gram.:m.}
\end{itemize}
Árvore da Guiné e da Índia, (\textunderscore michelia champaca\textunderscore ).
\section{Champor}
\begin{itemize}
\item {Grp. gram.:m.}
\end{itemize}
Árvore da Índia portuguesa.
O mesmo que \textunderscore champó\textunderscore ?
\section{Champorreirão}
\begin{itemize}
\item {Grp. gram.:m.  e  adj.}
\end{itemize}
\begin{itemize}
\item {Utilização:Prov.}
\end{itemize}
\begin{itemize}
\item {Utilização:alent.}
\end{itemize}
O mesmo que \textunderscore chamborreirão\textunderscore .
\section{Champurrião}
\begin{itemize}
\item {Grp. gram.:m.}
\end{itemize}
Acto de dar juntas todas as cartas, que pertencem a cada parceiro.
\section{Champúrrio}
\begin{itemize}
\item {Grp. gram.:m.}
\end{itemize}
\begin{itemize}
\item {Utilização:T. de jôgo}
\end{itemize}
Acto de dar juntas todas as cartas, que pertencem a cada parceiro.
\section{Chamuna}
\begin{itemize}
\item {Grp. gram.:f.}
\end{itemize}
Arbusto de Angola.
\section{Chamusano}
\begin{itemize}
\item {Grp. gram.:m.}
\end{itemize}
\begin{itemize}
\item {Utilização:Prov.}
\end{itemize}
\begin{itemize}
\item {Utilização:minh.}
\end{itemize}
Homem velhaco, tratante. (Colhido em Barcelos)
\section{Chamusca}
\begin{itemize}
\item {Grp. gram.:f.}
\end{itemize}
Acto de chamuscar.
\section{Chamuscada}
\begin{itemize}
\item {Grp. gram.:f.}
\end{itemize}
\begin{itemize}
\item {Utilização:Prov.}
\end{itemize}
\begin{itemize}
\item {Utilização:trasm.}
\end{itemize}
\begin{itemize}
\item {Proveniência:(De \textunderscore chamuscar\textunderscore )}
\end{itemize}
Bolo, feito de massa não bem levedada, e assado ou antes crestado á porta do forno, em quanto este arde.
\section{Chamuscador}
\begin{itemize}
\item {Grp. gram.:m.  e  adj.}
\end{itemize}
Aquelle que chamusca.
\section{Chamuscadura}
\begin{itemize}
\item {Grp. gram.:f.}
\end{itemize}
O mesmo que \textunderscore chamusca\textunderscore .
\section{Chamuscar}
\begin{itemize}
\item {Grp. gram.:v. t.}
\end{itemize}
\begin{itemize}
\item {Proveniência:(De \textunderscore chamma\textunderscore )}
\end{itemize}
Queimar ligeiramente; crestar; passar pela chamma.
\section{Chamusco}
\begin{itemize}
\item {Grp. gram.:m.}
\end{itemize}
\begin{itemize}
\item {Utilização:Fig.}
\end{itemize}
\begin{itemize}
\item {Utilização:Prov.}
\end{itemize}
\begin{itemize}
\item {Proveniência:(De \textunderscore chamuscar\textunderscore )}
\end{itemize}
Chamusca.
Cheiro de coisa queimada.
Suspeita.
Espécie de urze, (\textunderscore ulex genestoides\textunderscore ).
\section{Chan}
\begin{itemize}
\item {Grp. gram.:f.}
\end{itemize}
Terreno plano, planície.
Carne de coxa, no talho.
(Fem. de \textunderscore chão\textunderscore )
\section{Chanambo}
\begin{itemize}
\item {Grp. gram.:m.}
\end{itemize}
Espécie de cal, resultante da calcinação de cascas de ostras, na Índia portuguesa.
\section{Chanato}
\begin{itemize}
\item {Grp. gram.:m.}
\end{itemize}
\begin{itemize}
\item {Utilização:T. de Lanhoso}
\end{itemize}
Cigarro ordinário.
\section{Chanca}
\begin{itemize}
\item {Grp. gram.:f.}
\end{itemize}
\begin{itemize}
\item {Utilização:Pop.}
\end{itemize}
Pé grande.
Calçado grande e tôsco.
Perna alta e delgada de homem.
(Cp. cast. \textunderscore chanclo\textunderscore )
\section{Chança}
\begin{itemize}
\item {Grp. gram.:f.}
\end{itemize}
\begin{itemize}
\item {Utilização:Prov.}
\end{itemize}
\begin{itemize}
\item {Utilização:trasm.}
\end{itemize}
Troça; dito zombeteiro.
Vaidade; modo pretencioso.
Réplica.
(Cast. \textunderscore chanza\textunderscore )
\section{Chancada}
\begin{itemize}
\item {Grp. gram.:f.}
\end{itemize}
\begin{itemize}
\item {Utilização:Prov.}
\end{itemize}
\begin{itemize}
\item {Utilização:beir.}
\end{itemize}
\begin{itemize}
\item {Proveniência:(De \textunderscore chanca\textunderscore )}
\end{itemize}
Passo largo e pesado.
\section{Chancar}
\begin{itemize}
\item {Grp. gram.:v. i.}
\end{itemize}
\begin{itemize}
\item {Utilização:Prov.}
\end{itemize}
\begin{itemize}
\item {Utilização:beir.}
\end{itemize}
\begin{itemize}
\item {Proveniência:(De \textunderscore chanca\textunderscore )}
\end{itemize}
Fazer estrondo com as chancas ou tamancos, andando.
\section{Chançarel}
\begin{itemize}
\item {Grp. gram.:m.}
\end{itemize}
\begin{itemize}
\item {Utilização:Ant.}
\end{itemize}
O mesmo que \textunderscore chanceller\textunderscore . Cf. \textunderscore Peregrinação\textunderscore , LXXXV, e CIII.
\section{Chancarina}
\begin{itemize}
\item {Grp. gram.:f.}
\end{itemize}
(V.chancarona)
\section{Chancarona}
\begin{itemize}
\item {Grp. gram.:f.}
\end{itemize}
Peixe marinho, semelhante ao pargo.
\section{Chancear}
\begin{itemize}
\item {Grp. gram.:v. i.}
\end{itemize}
\begin{itemize}
\item {Utilização:Des.}
\end{itemize}
Dizer chanças.
\section{Chanceiro}
\begin{itemize}
\item {Grp. gram.:adj.}
\end{itemize}
Que diz chanças.
\section{Chancela}
\begin{itemize}
\item {Grp. gram.:f.}
\end{itemize}
Sêlo.
Sinal gravado, representativo de uma assinatura oficial ou do título de uma Repartição pública.
Acto de \textunderscore chancelar\textunderscore .
\section{Chancelar}
\begin{itemize}
\item {Grp. gram.:v. t.}
\end{itemize}
\begin{itemize}
\item {Proveniência:(Lat. \textunderscore cancellare\textunderscore )}
\end{itemize}
Selar.
Assinar com chancela.
\section{Chancelaria}
\begin{itemize}
\item {Grp. gram.:f.}
\end{itemize}
\begin{itemize}
\item {Proveniência:(Fr. \textunderscore chancellerie\textunderscore )}
\end{itemize}
Lugar público, Repartição, em que se põe chancela nos documentos que dela precisam.
Colecção de documentos ou diplomas oficiaes.
Cargo de chanceler.
\section{Chanceler}
\begin{itemize}
\item {Grp. gram.:m.}
\end{itemize}
\begin{itemize}
\item {Proveniência:(Fr. \textunderscore chancelier\textunderscore )}
\end{itemize}
Antigo magistrado, que tinha a seu cargo a guarda do sêlo real.
Funcionário público, que põe chancela ou sêlo em documentos ou diplomas; guarda-sêlos.
Administrador dos bens de uma Ordem militar.
Chefe da justiça, em alguns Estados alemães.
\section{Chancella}
\begin{itemize}
\item {Grp. gram.:f.}
\end{itemize}
Sêllo.
Sinal gravado, representativo de uma assinatura official ou do título de uma Repartição pública.
Acto de \textunderscore chancellar\textunderscore .
\section{Chancellar}
\begin{itemize}
\item {Grp. gram.:v. t.}
\end{itemize}
\begin{itemize}
\item {Proveniência:(Lat. \textunderscore cancellare\textunderscore )}
\end{itemize}
Sellar.
Assinar com chancella.
\section{Chancellaria}
\begin{itemize}
\item {Grp. gram.:f.}
\end{itemize}
\begin{itemize}
\item {Proveniência:(Fr. \textunderscore chancellerie\textunderscore )}
\end{itemize}
Lugar público, Repartição, em que se põe chancella nos documentos que della precisam.
Collecção de documentos ou diplomas officiaes.
Cargo de chanceller.
\section{Chanceller}
\begin{itemize}
\item {Grp. gram.:m.}
\end{itemize}
\begin{itemize}
\item {Proveniência:(Fr. \textunderscore chancelier\textunderscore )}
\end{itemize}
Antigo magistrado, que tinha a seu cargo a guarda do sêllo real.
Funccionário público, que põe chancella ou sêllo em documentos ou diplomas; guarda-sêllos.
Administrador dos bens de uma Ordem militar.
Chefe da justiça, em alguns Estados alemães.
\section{Chanço}
\begin{itemize}
\item {Grp. gram.:m.}
\end{itemize}
\begin{itemize}
\item {Utilização:Prov.}
\end{itemize}
Pequeno pau curvo, cujas extremidades são ligadas por um cordão de crina, e que faz parte de uma armadilha para apanhar pássaros.
\section{Chancra}
\begin{itemize}
\item {Grp. gram.:f.}
\end{itemize}
\begin{itemize}
\item {Utilização:Bras. de Minas}
\end{itemize}
O mesmo que \textunderscore chanca\textunderscore .
\section{Chancudo}
\begin{itemize}
\item {Grp. gram.:adj.}
\end{itemize}
\begin{itemize}
\item {Proveniência:(De \textunderscore chanca\textunderscore )}
\end{itemize}
Que tem pés grandes.
\section{Chandeirola}
\begin{itemize}
\item {Grp. gram.:f.}
\end{itemize}
\begin{itemize}
\item {Utilização:T. de Manteigas}
\end{itemize}
O mesmo que \textunderscore horta\textunderscore .
\section{Chandeu}
\begin{itemize}
\item {Grp. gram.:m.}
\end{itemize}
Espécie de arraial ou romaria, na China. Cf. \textunderscore Peregrinação\textunderscore , XCVIII.
\section{Chanduló}
\begin{itemize}
\item {Grp. gram.:m.}
\end{itemize}
Arvoreta da Índia portuguesa.
\section{Chane!}
\begin{itemize}
\item {Grp. gram.:Interj.}
\end{itemize}
\begin{itemize}
\item {Utilização:Bras. do N}
\end{itemize}
Voz, com que se chama o bichano, o gato.
\section{Chanesco}
\begin{itemize}
\item {Grp. gram.:adj.}
\end{itemize}
\begin{itemize}
\item {Utilização:Pop.}
\end{itemize}
\begin{itemize}
\item {Proveniência:(De \textunderscore chão\textunderscore )}
\end{itemize}
Mal trabalhado, mal acabado, grosseiro.
\section{Chaneza}
\begin{itemize}
\item {Grp. gram.:f.}
\end{itemize}
\begin{itemize}
\item {Utilização:Des.}
\end{itemize}
\begin{itemize}
\item {Proveniência:(De \textunderscore chão\textunderscore )}
\end{itemize}
O mesmo que \textunderscore lhaneza\textunderscore .
\section{Chanfalhão}
\begin{itemize}
\item {Grp. gram.:adj.}
\end{itemize}
\begin{itemize}
\item {Utilização:Pop.}
\end{itemize}
\begin{itemize}
\item {Grp. gram.:M.}
\end{itemize}
\begin{itemize}
\item {Proveniência:(De \textunderscore chanfalho\textunderscore )}
\end{itemize}
Brincalhão, jovial.
Espada grande e velha.
\section{Chanfalhar}
\begin{itemize}
\item {Grp. gram.:v. i.}
\end{itemize}
\begin{itemize}
\item {Utilização:Pop.}
\end{itemize}
\begin{itemize}
\item {Utilização:Fig.}
\end{itemize}
\begin{itemize}
\item {Proveniência:(De \textunderscore chanfalho\textunderscore )}
\end{itemize}
Tocar um mau instrumento; tocar desafinadamente.
Folgar, mostrar-se alegre.
\section{Chanfalheiro}
\begin{itemize}
\item {Grp. gram.:m.}
\end{itemize}
Aquelle que brande ridiculamente um chanfalho.
Aquelle que diz graçolas.
\section{Chanfalhice}
\begin{itemize}
\item {Grp. gram.:f.}
\end{itemize}
Qualidade ou acto de chanfalhão.
\section{Chanfalho}
\begin{itemize}
\item {Grp. gram.:m.}
\end{itemize}
Instrumento desafinado.
Espada velha e ferrugenta.
Utensílio deteriorado.
(Cast. \textunderscore chafallo\textunderscore )
\section{Chanfana}
\begin{itemize}
\item {Grp. gram.:f.}
\end{itemize}
\begin{itemize}
\item {Utilização:Prov.}
\end{itemize}
\begin{itemize}
\item {Utilização:Burl.}
\end{itemize}
\begin{itemize}
\item {Utilização:Prov.}
\end{itemize}
Espécie de guisado de fígado.
Sarapatel, sarrabulho.
Comida mal feita.
Carne magra de carneiro, pellanga.
Espada.
Aguardente ordinária. (Colhido no Fundão)
(Cast. \textunderscore chanfaina\textunderscore )
\section{Chanfana}
\begin{itemize}
\item {Grp. gram.:f.}
\end{itemize}
\begin{itemize}
\item {Utilização:Prov.}
\end{itemize}
O mesmo que \textunderscore borralheira\textunderscore . (Colhido em Turquel)
\section{Chanfaneiro}
\begin{itemize}
\item {Grp. gram.:m.}
\end{itemize}
Aquelle que prepara ou vende chanfana.
Vendedor de miúdos de carneiro, de vaca, etc.
\section{Chanfaranho}
\begin{itemize}
\item {Grp. gram.:m.}
\end{itemize}
Antigo dignitário chinês. Cf. \textunderscore Peregrinação\textunderscore , CLX.
\section{Chanfeniteiro}
\begin{itemize}
\item {Grp. gram.:m.}
\end{itemize}
\begin{itemize}
\item {Utilização:Prov.}
\end{itemize}
\begin{itemize}
\item {Utilização:trasm.}
\end{itemize}
Quinquilheiro; tendeiro ambulante; bufarinheiro.
\section{Chanfrador}
\begin{itemize}
\item {Grp. gram.:m.}
\end{itemize}
Instrumento de chanfrar.
\section{Chanfradura}
\begin{itemize}
\item {Grp. gram.:f.}
\end{itemize}
Effeito de chanfrar.
Recorte curvilíneo das extremidades de um terreno ou de outro objecto.
\section{Chanfrar}
\begin{itemize}
\item {Grp. gram.:v. t.}
\end{itemize}
\begin{itemize}
\item {Utilização:Carp.}
\end{itemize}
\begin{itemize}
\item {Proveniência:(Fr. \textunderscore chanfrer\textunderscore )}
\end{itemize}
Cortar em semi-círculo.
Cortar com plaina ou garlopa as arestas de; fazer chanfros em.
\section{Chanfreta}
\begin{itemize}
\item {fónica:frê}
\end{itemize}
\begin{itemize}
\item {Grp. gram.:f.}
\end{itemize}
\begin{itemize}
\item {Utilização:Bras}
\end{itemize}
Facécia, galhofa.
Motejo.
\section{Chanfro}
\begin{itemize}
\item {Grp. gram.:m.}
\end{itemize}
(V.chanfradura)
\section{Changarço}
\textunderscore m.\textunderscore  (e der.)
O mesmo que \textunderscore chaguarço\textunderscore , etc.
\section{Changueiro}
\begin{itemize}
\item {Grp. gram.:m.}
\end{itemize}
\begin{itemize}
\item {Utilização:Bras. do S}
\end{itemize}
\begin{itemize}
\item {Proveniência:(De \textunderscore changui\textunderscore )}
\end{itemize}
Cavallo para pequenas corridas.
\section{Changui}
\begin{itemize}
\item {Grp. gram.:m.}
\end{itemize}
\begin{itemize}
\item {Utilização:Bras. do S}
\end{itemize}
Concessão, que se faz, ou partido que se dá, em jogos de corridas.
\section{Chaníssimo}
\begin{itemize}
\item {Grp. gram.:adj.}
\end{itemize}
\begin{itemize}
\item {Proveniência:(De \textunderscore chão\textunderscore )}
\end{itemize}
Muito chão, muito plano.
Muito lhano, muito franco.
\section{Chanmente}
\begin{itemize}
\item {Grp. gram.:adv.}
\end{itemize}
De modo chão, lhano.
\section{Chanquear}
\begin{itemize}
\item {Grp. gram.:v. t.}
\end{itemize}
\begin{itemize}
\item {Utilização:Prov.}
\end{itemize}
\begin{itemize}
\item {Utilização:minh.}
\end{itemize}
Baptizar á pressa (uma criança), quando nasce quási morta e não há tempo para chamar o padre, podendo qualquer pessôa administrar o baptismo.
\section{Chanqueta}
\begin{itemize}
\item {fónica:quê}
\end{itemize}
\begin{itemize}
\item {Grp. gram.:f.}
\end{itemize}
\begin{itemize}
\item {Utilização:Pop.}
\end{itemize}
\begin{itemize}
\item {Grp. gram.:M.}
\end{itemize}
\begin{itemize}
\item {Utilização:Prov.}
\end{itemize}
\begin{itemize}
\item {Utilização:beir.}
\end{itemize}
\begin{itemize}
\item {Utilização:Prov.}
\end{itemize}
\begin{itemize}
\item {Utilização:trasm.}
\end{itemize}
\begin{itemize}
\item {Proveniência:(De chantar)}
\end{itemize}
Calçado, sem contraforte no calcanhar ou com o contraforte acalcanhado.
Homem coxo.
\textunderscore Andar de chanqueta\textunderscore , andar servilmente ás ordens de alguém, adivinhando-lhe as vontades e bajulando-o.
\section{Chanta}
\begin{itemize}
\item {Grp. gram.:f.}
\end{itemize}
\begin{itemize}
\item {Utilização:Ant.}
\end{itemize}
(V.chantão)
Lugar, em que se plantaram tanchões ou estacas; bacellada. Cf. \textunderscore Port. Ant. e Mod.\textunderscore , vb. \textunderscore Santarém\textunderscore .
\section{Chantado}
\begin{itemize}
\item {Grp. gram.:m.}
\end{itemize}
O mesmo que \textunderscore chantoeira\textunderscore .
\section{Chantadoria}
\begin{itemize}
\item {Grp. gram.:f.}
\end{itemize}
\begin{itemize}
\item {Utilização:Ant.}
\end{itemize}
\begin{itemize}
\item {Proveniência:(De \textunderscore chantar\textunderscore )}
\end{itemize}
Plantação de arvoredo.
\section{Chantadura}
\begin{itemize}
\item {Grp. gram.:f.}
\end{itemize}
Acção de \textunderscore chantar\textunderscore .
\section{Chantage}
\begin{itemize}
\item {Grp. gram.:f.}
\end{itemize}
\begin{itemize}
\item {Proveniência:(T. fr., de \textunderscore chanter\textunderscore )}
\end{itemize}
Acto de extorquir dinheiro a alguém, ameaçando-o de revelar qualquer coisa escandalosa, de o diffamar, etc.
\section{Chantagem}
\begin{itemize}
\item {Grp. gram.:f.}
\end{itemize}
\begin{itemize}
\item {Utilização:Ant.}
\end{itemize}
Acto de \textunderscore chantar\textunderscore .
\section{Chantão}
\begin{itemize}
\item {Grp. gram.:m.}
\end{itemize}
\begin{itemize}
\item {Proveniência:(De \textunderscore chantar\textunderscore )}
\end{itemize}
Estaca ou ramo, que se planta sem raíz, para reproducção; tanchão.
\section{Chantar}
\begin{itemize}
\item {Grp. gram.:v. t.}
\end{itemize}
\begin{itemize}
\item {Utilização:Ant.}
\end{itemize}
\begin{itemize}
\item {Proveniência:(Do lat. \textunderscore plantare\textunderscore )}
\end{itemize}
Plantar de estaca.
Collocar, assentar.
Pregar. Cf. \textunderscore Port. Mon. Hist.\textunderscore , \textunderscore Script.\textunderscore , 260.
\section{Chantel}
\begin{itemize}
\item {Grp. gram.:m.}
\end{itemize}
Peça, que fórma o fundo ou parte do fundo, em vasilha de tanoeiro.
\section{Chanto}
\begin{itemize}
\item {Grp. gram.:m.}
\end{itemize}
\begin{itemize}
\item {Utilização:Ant.}
\end{itemize}
\begin{itemize}
\item {Proveniência:(Lat. \textunderscore planctus\textunderscore )}
\end{itemize}
O mesmo que \textunderscore pranto\textunderscore . Cf. G. Vicente, I, 347.
\section{Chantoal}
\begin{itemize}
\item {Grp. gram.:m.}
\end{itemize}
O mesmo que \textunderscore chantoeira\textunderscore .
\section{Chantoeira}
\begin{itemize}
\item {Grp. gram.:f.}
\end{itemize}
\begin{itemize}
\item {Proveniência:(De \textunderscore chantão\textunderscore )}
\end{itemize}
Lugar plantado de estacas ou ramos, para reproducção.
O mesmo que \textunderscore tanchoeira\textunderscore .
\section{Cáos}
\begin{itemize}
\item {Grp. gram.:m.}
\end{itemize}
\begin{itemize}
\item {Utilização:Fig.}
\end{itemize}
\begin{itemize}
\item {Proveniência:(Gr. \textunderscore khaos\textunderscore )}
\end{itemize}
Confusão de todos os elementos, antes de se formar o mundo.
Grande desordem, confusão.
\section{Caótico}
\begin{itemize}
\item {Grp. gram.:adj.}
\end{itemize}
\begin{itemize}
\item {Proveniência:(Do gr. \textunderscore khaos\textunderscore )}
\end{itemize}
Confuso, desordenado.
\section{Caotizar}
\begin{itemize}
\item {Grp. gram.:v. t.}
\end{itemize}
\begin{itemize}
\item {Utilização:Neol.}
\end{itemize}
Tornar caótico.
\section{Cara}
\begin{itemize}
\item {Grp. gram.:m.}
\end{itemize}
\begin{itemize}
\item {Proveniência:(Lat. \textunderscore chara\textunderscore )}
\end{itemize}
Gênero de algas submergidas.
\section{Caráceas}
\begin{itemize}
\item {Grp. gram.:f. pl.}
\end{itemize}
Familia de plantas criptogâmicas, que têm por tipo o \textunderscore cara\textunderscore .
\section{Chantrado}
\begin{itemize}
\item {Grp. gram.:m.}
\end{itemize}
O mesmo que \textunderscore chantria\textunderscore .
\section{Chantre}
\begin{itemize}
\item {Grp. gram.:m.}
\end{itemize}
\begin{itemize}
\item {Proveniência:(Fr. \textunderscore chantre\textunderscore )}
\end{itemize}
Funccionário ecclesiástico, que dirige o côro.
Aquelle que entôa os salmos nos templos protestantes.
\section{Chantria}
\begin{itemize}
\item {Grp. gram.:f.}
\end{itemize}
Cargo de chantre.
\section{Chão}
\begin{itemize}
\item {Grp. gram.:adj.}
\end{itemize}
\begin{itemize}
\item {Grp. gram.:M.}
\end{itemize}
\begin{itemize}
\item {Utilização:Prov.}
\end{itemize}
\begin{itemize}
\item {Utilização:beir.}
\end{itemize}
\begin{itemize}
\item {Utilização:Ant.}
\end{itemize}
\begin{itemize}
\item {Proveniência:(Do lat. \textunderscore planus\textunderscore )}
\end{itemize}
O mesmo que \textunderscore plano\textunderscore ; liso.
Tranquillo.
Franco; lhano; singelo.
Acostumado.
Terra chan.
Pequena terra arborizada e regadia.
Solo.
A superfície da terra.
Pavimento.
Fundo de um quadro, de um tecido, de um escudo, de qualquer superfície colorida.
Medida agrária de 60 palmos de comprido e 30 de largo.
\section{Cháos}
\begin{itemize}
\item {fónica:cá}
\end{itemize}
\begin{itemize}
\item {Grp. gram.:m.}
\end{itemize}
\begin{itemize}
\item {Utilização:Fig.}
\end{itemize}
\begin{itemize}
\item {Proveniência:(Gr. \textunderscore khaos\textunderscore )}
\end{itemize}
Confusão de todos os elementos, antes de se formar o mundo.
Grande desordem, confusão.
\section{Chaótico}
\begin{itemize}
\item {fónica:ca}
\end{itemize}
\begin{itemize}
\item {Grp. gram.:adj.}
\end{itemize}
\begin{itemize}
\item {Proveniência:(Do gr. \textunderscore khaos\textunderscore )}
\end{itemize}
Confuso, desordenado.
\section{Chaotizar}
\begin{itemize}
\item {fónica:ca}
\end{itemize}
\begin{itemize}
\item {Grp. gram.:v. t.}
\end{itemize}
\begin{itemize}
\item {Utilização:Neol.}
\end{itemize}
Tornar chaótico.
\section{Chapa}
\begin{itemize}
\item {Grp. gram.:f.}
\end{itemize}
\begin{itemize}
\item {Grp. gram.:Loc. adv.}
\end{itemize}
\begin{itemize}
\item {Grp. gram.:Pl.}
\end{itemize}
Peça chata e plana de matéria consistente, como metal, madeira, etc.
Lâmina.
Desenho, aberto em metal, para se transportar para a tela.
Peça de metal ou madeira, com trabalho de gravura, que se destina a imprimir-se.
Insígnia honorífica, aberta em metal.
Distinctivo de aguadeiros, moços de fretes, etc., em Lisbôa.
Antigo instrumento militar.
Planície, chapada.
\textunderscore De chapa\textunderscore , de frente, em cheio, de cara a cara.
Segundo um modêlo único; pelo mesmo teor: \textunderscore appareceu nos jornaes uma notícia de chapa\textunderscore .
Espécie de jôgo de rapazes.
\section{Chapa}
\begin{itemize}
\item {Grp. gram.:f.}
\end{itemize}
\begin{itemize}
\item {Utilização:Ant.}
\end{itemize}
\begin{itemize}
\item {Proveniência:(T. as.)}
\end{itemize}
Ordenança; permissão.
\section{Chapaçal}
\begin{itemize}
\item {Grp. gram.:m.}
\end{itemize}
\begin{itemize}
\item {Utilização:Prov.}
\end{itemize}
\begin{itemize}
\item {Utilização:trasm.}
\end{itemize}
Atoleiro, paul.
\section{Chapaceiro}
\begin{itemize}
\item {Grp. gram.:m.}
\end{itemize}
\begin{itemize}
\item {Utilização:Prov.}
\end{itemize}
\begin{itemize}
\item {Utilização:trasm.}
\end{itemize}
Lameiro.
\section{Chapada}
\begin{itemize}
\item {Grp. gram.:f.}
\end{itemize}
Planura.
Planalto.
Clareira.
Pancada em cheio.
Porção de líquido, que cai ou que se atira de uma vez.
\section{Chapadamente}
\begin{itemize}
\item {Grp. gram.:adv.}
\end{itemize}
\begin{itemize}
\item {Utilização:Ant.}
\end{itemize}
\begin{itemize}
\item {Proveniência:(De \textunderscore chapado\textunderscore )}
\end{itemize}
Claramente, distintamente.
\section{Chapadão}
\begin{itemize}
\item {Grp. gram.:m.}
\end{itemize}
\begin{itemize}
\item {Utilização:Bras}
\end{itemize}
Chapada grande.
\section{Chapado}
\begin{itemize}
\item {Grp. gram.:adj.}
\end{itemize}
\begin{itemize}
\item {Utilização:Pop.}
\end{itemize}
Rematado, completo, perfeito: \textunderscore um asno chapado\textunderscore .
\section{Chapalheta}
\begin{itemize}
\item {fónica:lhê}
\end{itemize}
\begin{itemize}
\item {Grp. gram.:f.}
\end{itemize}
Espécie de gaivota, (\textunderscore larus ridibundus\textunderscore , Lin.).
\section{Chapar}
\begin{itemize}
\item {Grp. gram.:v. t.}
\end{itemize}
\begin{itemize}
\item {Utilização:Gír.}
\end{itemize}
\begin{itemize}
\item {Grp. gram.:V. p.}
\end{itemize}
Pôr chapa em.
Segurar com chapa.
Cunhar.
Dar fórma de chapa a.
Têr cóito.
Estender-se no chão, caíndo; estatelar-se; caír de chapa: \textunderscore chapar-se o cavallo\textunderscore .
\section{Chaparia}
\begin{itemize}
\item {Grp. gram.:f.}
\end{itemize}
Conjunto de chapas.
\section{Chaparral}
\begin{itemize}
\item {Grp. gram.:m.}
\end{itemize}
Mata de chaparros.
\section{Chaparreiro}
\begin{itemize}
\item {Grp. gram.:m.}
\end{itemize}
O mesmo que \textunderscore chaparro\textunderscore .
\section{Chaparrinho}
\begin{itemize}
\item {Grp. gram.:m.}
\end{itemize}
\begin{itemize}
\item {Utilização:Prov.}
\end{itemize}
\begin{itemize}
\item {Utilização:trasm.}
\end{itemize}
Indivíduo muito estúpido.
(Por \textunderscore chapadinho\textunderscore , de \textunderscore chapado\textunderscore ?)
\section{Chaparro}
\begin{itemize}
\item {Grp. gram.:m.}
\end{itemize}
Arvore pequena e tortuosa.
Sobreiro pequeno.
(Talvez do vasc.)
\section{Chaparus!}
\begin{itemize}
\item {Grp. gram.:Interj.}
\end{itemize}
Voz imitativa da quéda de um corpo pesado na água. (Colhido em Turquel)
\section{Chapatesta}
\begin{itemize}
\item {fónica:chá}
\end{itemize}
\begin{itemize}
\item {Grp. gram.:f.}
\end{itemize}
\begin{itemize}
\item {Proveniência:(De \textunderscore chapa\textunderscore  + \textunderscore testa\textunderscore )}
\end{itemize}
Chapa ou lâmina, em que entra a lingueta da fechadura sob o impulso da chave.
\section{Chape}
\begin{itemize}
\item {Grp. gram.:m.}
\end{itemize}
\begin{itemize}
\item {Grp. gram.:Interj.}
\end{itemize}
\begin{itemize}
\item {Proveniência:(T. onom.)}
\end{itemize}
Pancada na água, som de qualquer coisa que bate ou cái na água.
\textunderscore Fazer chape\textunderscore , produzir êsse som.
(Serve para exprimir o mesmo som, ou o som do cão da espingarda, batendo em falso sôbre o ouvido da espoleta)
\section{Chapeado}
\begin{itemize}
\item {Grp. gram.:adj.}
\end{itemize}
\begin{itemize}
\item {Grp. gram.:M.}
\end{itemize}
\begin{itemize}
\item {Utilização:Bras. do S}
\end{itemize}
Revestido de chapa ou chapas.
Cabeçada, guarnecida de prata.
\section{Chapear}
\begin{itemize}
\item {Grp. gram.:v. t.}
\end{itemize}
\begin{itemize}
\item {Utilização:Marn.}
\end{itemize}
Revestir de chapas.
Achatar.
Tapar e segurar com torrões (a cobertura vegetal dos montes de sal).
\section{Chapear}
\begin{itemize}
\item {Grp. gram.:v. i.}
\end{itemize}
\begin{itemize}
\item {Utilização:Prov.}
\end{itemize}
\begin{itemize}
\item {Utilização:trasm.}
\end{itemize}
\begin{itemize}
\item {Proveniência:(De \textunderscore chape\textunderscore )}
\end{itemize}
Errar fogo (a arma).
\section{Chapeca}
\begin{itemize}
\item {Grp. gram.:f.}
\end{itemize}
\begin{itemize}
\item {Utilização:Gír.}
\end{itemize}
Moeda de déz reis.
(Talvez corr. de \textunderscore sapeca\textunderscore ^1)
\section{Chapeirada}
\begin{itemize}
\item {Grp. gram.:f.}
\end{itemize}
Caldeirada.
Quantidade, que póde contêr-se num chapéu.
(Cp. \textunderscore chapeirão\textunderscore )
\section{Chapeirão}
\begin{itemize}
\item {Grp. gram.:m.}
\end{itemize}
\begin{itemize}
\item {Utilização:Bras}
\end{itemize}
\begin{itemize}
\item {Proveniência:(Fr. \textunderscore chaperon\textunderscore )}
\end{itemize}
Capuz. Cf. Bernardim, Gil Vicente, etc.
Recife á flôr da água.
\section{Chapejar}
\begin{itemize}
\item {Grp. gram.:v. t.  e  i.}
\end{itemize}
\begin{itemize}
\item {Grp. gram.:V. i.}
\end{itemize}
\begin{itemize}
\item {Utilização:Prov.}
\end{itemize}
\begin{itemize}
\item {Utilização:trasm.}
\end{itemize}
\begin{itemize}
\item {Proveniência:(De \textunderscore chape\textunderscore )}
\end{itemize}
O mesmo que \textunderscore chapinhar\textunderscore .
Errar fogo a espingarda, estoirando só o fulminante e ficando dentro o tiro.
\section{Chapel}
\begin{itemize}
\item {Grp. gram.:m.}
\end{itemize}
\begin{itemize}
\item {Utilização:Gal}
\end{itemize}
\begin{itemize}
\item {Utilização:ant.}
\end{itemize}
\begin{itemize}
\item {Proveniência:(Fr. ant. \textunderscore chapels\textunderscore  = fr. mod. \textunderscore chapeau\textunderscore )}
\end{itemize}
O mesmo que \textunderscore elmo\textunderscore . Cf. \textunderscore Cancion. da Vaticana\textunderscore .
\section{Chapelada}
\begin{itemize}
\item {Grp. gram.:f.}
\end{itemize}
Porção que póde caber num chapéu.
\section{Chapelaria}
\begin{itemize}
\item {Grp. gram.:f.}
\end{itemize}
\begin{itemize}
\item {Proveniência:(Fr. \textunderscore chapellerie\textunderscore )}
\end{itemize}
Commércio, offício, estabelecimento, de chapeleiro.
\section{Chapeleira}
\begin{itemize}
\item {Grp. gram.:f.}
\end{itemize}
\begin{itemize}
\item {Proveniência:(Fr. \textunderscore chápelière\textunderscore )}
\end{itemize}
Caixa para chapéus.
Pequeno cabide portátil, para chapéus.
Mulher de chapeleiro.
\section{Chapeleirão}
\begin{itemize}
\item {Grp. gram.:m.}
\end{itemize}
Chapéu grande.
\section{Chapeleiro}
\begin{itemize}
\item {Grp. gram.:m.}
\end{itemize}
\begin{itemize}
\item {Proveniência:(Fr. \textunderscore chapelier\textunderscore )}
\end{itemize}
Aquelle que faz ou vende chapéus.
\section{Chapeleta}
\begin{itemize}
\item {fónica:lê}
\end{itemize}
\begin{itemize}
\item {Grp. gram.:f.}
\end{itemize}
\begin{itemize}
\item {Utilização:Pop.}
\end{itemize}
Chapelinho.
Válvula, nas bombas que se usam a bórdo.
Ricochete.
Círculos concêntricos, formados na água tranquilla, por um objecto que cái nella.
Roseta das faces.
Carolo, pancada na cabeça, com vara, cana, etc.
(Cast. \textunderscore chapelete\textunderscore )
\section{Chapelete}
\begin{itemize}
\item {Grp. gram.:m.}
\end{itemize}
(dem. de \textunderscore chapéu\textunderscore )
\section{Chapelina}
\begin{itemize}
\item {Grp. gram.:f.}
\end{itemize}
\begin{itemize}
\item {Utilização:Bras. do Ceará}
\end{itemize}
Chapéu, usado pelas mulheres do sertão.
\section{Chapelinha}
\begin{itemize}
\item {Grp. gram.:f.}
\end{itemize}
\begin{itemize}
\item {Utilização:Prov.}
\end{itemize}
\begin{itemize}
\item {Utilização:dur.}
\end{itemize}
Chapéu baixo, de abas largas, enfeitado de flôres ou plumas vistosas, para uso de senhoras em dias de festa. Cf. Ar. e Melo, \textunderscore Ling. Pop.\textunderscore 
\section{Chapelinho}
\begin{itemize}
\item {Grp. gram.:m.}
\end{itemize}
(dem. de \textunderscore chapéu\textunderscore )
\section{Chapelório}
\begin{itemize}
\item {Grp. gram.:m.}
\end{itemize}
\begin{itemize}
\item {Utilização:Fam.}
\end{itemize}
Chapéu ordinário, de abas grandes.
\section{Chapéo}
\begin{itemize}
\item {Grp. gram.:m.}
\end{itemize}
\begin{itemize}
\item {Utilização:Mús.}
\end{itemize}
\begin{itemize}
\item {Utilização:Fig.}
\end{itemize}
\begin{itemize}
\item {Proveniência:(Fr. \textunderscore chapeau\textunderscore )}
\end{itemize}
Cobertura, formada de copa e abas, para a cabeça de homem.
Cobertura, de variadissimos feitios, para cabeça de mulher.
Guarda-sol, guarda-chuva.
Pequena peça espheroidal, que tapa a parte superior da frauta.
Nome vulgar do \textunderscore agárico\textunderscore .
Dignidade (cardinalicia).
Nome de alguns objectos, que têm configuração semelhante á de um chapéu ou guarda-chuva.
\section{Chapetão}
\begin{itemize}
\item {Grp. gram.:m.}
\end{itemize}
\begin{itemize}
\item {Utilização:Bras. do S}
\end{itemize}
Indivíduo sonso.
Pacóvio.
(Cast. \textunderscore chapetón\textunderscore )
\section{Chapetonada}
\begin{itemize}
\item {Grp. gram.:f.}
\end{itemize}
\begin{itemize}
\item {Utilização:Bras. do S}
\end{itemize}
\begin{itemize}
\item {Proveniência:(De \textunderscore chapetão\textunderscore )}
\end{itemize}
Engano.
\section{Chapéu}
\begin{itemize}
\item {Grp. gram.:m.}
\end{itemize}
\begin{itemize}
\item {Utilização:Mús.}
\end{itemize}
\begin{itemize}
\item {Utilização:Fig.}
\end{itemize}
\begin{itemize}
\item {Utilização:Miner.}
\end{itemize}
\begin{itemize}
\item {Proveniência:(Fr. \textunderscore chapeau\textunderscore )}
\end{itemize}
Cobertura, formada de copa e abas, para a cabeça de homem.
Cobertura, de variadissimos feitios, para cabeça de mulher.
Guarda-sol, guarda-chuva.
Pequena peça espheroidal, que tapa a parte superior da frauta.
Nome vulgar do \textunderscore agárico\textunderscore .
Dignidade (cardinalicia).
Nome de alguns objectos, que têm configuração semelhante á de um chapéu ou guarda-chuva.
Camada delgada de um mineral, que cobre o jazigo de outro, mais possante.
\section{Chapéu!}
\begin{itemize}
\item {Grp. gram.:interj.}
\end{itemize}
\begin{itemize}
\item {Proveniência:(T. onom. Cp. \textunderscore chapa\textunderscore ^1)}
\end{itemize}
Voz imitativa do som produzido pelo cão da espingarda, ao bater na espoleta, sem que produza a explosão que se espera.
\section{Chapéu-de-coiro}
\begin{itemize}
\item {Grp. gram.:m.}
\end{itemize}
\begin{itemize}
\item {Utilização:Bras. de Minas}
\end{itemize}
Planta medicinal.
\section{Chapéu-de-feiticeira}
\begin{itemize}
\item {Grp. gram.:m.}
\end{itemize}
\begin{itemize}
\item {Utilização:Açor}
\end{itemize}
O mesmo que \textunderscore agárico\textunderscore . (Colhido em San-Jorge)
\section{Chapéu-de-sol}
\begin{itemize}
\item {Grp. gram.:m.}
\end{itemize}
Planta ornamental do Brasil.
\section{Chapeuzinho}
\begin{itemize}
\item {Grp. gram.:m.}
\end{itemize}
(dem. de \textunderscore chapéu\textunderscore )
\section{Chapiçada}
\begin{itemize}
\item {Grp. gram.:f.}
\end{itemize}
\begin{itemize}
\item {Utilização:Prov.}
\end{itemize}
\begin{itemize}
\item {Utilização:trasm.}
\end{itemize}
Borrifo de água.
\section{Chapiceiro}
\begin{itemize}
\item {Grp. gram.:m.}
\end{itemize}
\begin{itemize}
\item {Utilização:Prov.}
\end{itemize}
\begin{itemize}
\item {Utilização:trasm.}
\end{itemize}
O mesmo que \textunderscore chapaceiro\textunderscore .
\section{Chapilhar}
\begin{itemize}
\item {Grp. gram.:v. i.}
\end{itemize}
\begin{itemize}
\item {Utilização:Prov.}
\end{itemize}
\begin{itemize}
\item {Utilização:trasm.}
\end{itemize}
O mesmo que \textunderscore chapinhar\textunderscore .
\section{Chapim}
\begin{itemize}
\item {Grp. gram.:m.}
\end{itemize}
\begin{itemize}
\item {Utilização:Fig.}
\end{itemize}
Antigo calçado, de sola alta, para mulheres.
Antigo cothurno, usado na representação das tragédias.
Patim.
Chapa, que liga os carris de ferro ás travéssas.
Base.
Sapatinho elegante.
(Cast. \textunderscore chapin\textunderscore )
\section{Chapim}
\begin{itemize}
\item {Grp. gram.:m.}
\end{itemize}
\begin{itemize}
\item {Proveniência:(T. onom. Cp. \textunderscore patachim\textunderscore )}
\end{itemize}
O mesmo que \textunderscore megengra\textunderscore .
\section{Chapinada}
\begin{itemize}
\item {Grp. gram.:f.}
\end{itemize}
\begin{itemize}
\item {Utilização:Ant.}
\end{itemize}
Briga com chapins. Cf. \textunderscore Fenix\textunderscore , IV, 11.
\section{Chapinar}
\begin{itemize}
\item {Grp. gram.:v. t.  e  i.}
\end{itemize}
O mesmo que \textunderscore chapinhar\textunderscore . Cf. Camillo, \textunderscore Brasileira\textunderscore , 61; Arn. Gama, \textunderscore Ult. Dona\textunderscore , 284.
\section{Chapineiro}
\begin{itemize}
\item {Grp. gram.:m.}
\end{itemize}
\begin{itemize}
\item {Utilização:Des.}
\end{itemize}
Aquelle que faz ou vende chapins.
\section{Chapinhar}
\begin{itemize}
\item {Grp. gram.:v. t.}
\end{itemize}
\begin{itemize}
\item {Grp. gram.:V. i.}
\end{itemize}
\begin{itemize}
\item {Proveniência:(De \textunderscore chape\textunderscore )}
\end{itemize}
Banhar com a mão ou com um trapo embebido em líquido, repetidas e frequentes vezes.
Bater de chapa com as mãos na água, agitando-a.
Agitar a água com as mãos ou com os pés.
\section{Chapinheira}
\begin{itemize}
\item {Grp. gram.:f.}
\end{itemize}
\begin{itemize}
\item {Utilização:Prov.}
\end{itemize}
\begin{itemize}
\item {Utilização:beir.}
\end{itemize}
O mesmo que \textunderscore nassa\textunderscore .
\section{Chapinheiro}
\begin{itemize}
\item {Grp. gram.:m.}
\end{itemize}
\begin{itemize}
\item {Proveniência:(De \textunderscore chapinhar\textunderscore )}
\end{itemize}
Lugar encharcado, em que se chapinha.
Água entornada ou empoçada no chão.
\section{Chapitel}
\begin{itemize}
\item {Grp. gram.:m.}
\end{itemize}
\begin{itemize}
\item {Utilização:Des.}
\end{itemize}
O mesmo que \textunderscore capitel\textunderscore . Cf. \textunderscore Archeólogo Português\textunderscore , V, 3.
\section{Chapitéo}
\begin{itemize}
\item {Grp. gram.:m.}
\end{itemize}
\begin{itemize}
\item {Proveniência:(Fr. \textunderscore chapiteau\textunderscore )}
\end{itemize}
A parte mais elevada da prôa e da popa do navio.
\section{Chapitéu}
\begin{itemize}
\item {Grp. gram.:m.}
\end{itemize}
\begin{itemize}
\item {Proveniência:(Fr. \textunderscore chapiteau\textunderscore )}
\end{itemize}
A parte mais elevada da prôa e da popa do navio.
\section{Chapoda}
\begin{itemize}
\item {Grp. gram.:f.}
\end{itemize}
\begin{itemize}
\item {Utilização:Prov.}
\end{itemize}
\begin{itemize}
\item {Utilização:trasm.}
\end{itemize}
\begin{itemize}
\item {Proveniência:(De \textunderscore chapodar\textunderscore )}
\end{itemize}
Ramúsculo, de que se despojam certas árvores.
\section{Chapodar}
\textunderscore v. t.\textunderscore  (e der.)
O mesmo que \textunderscore chapotar\textunderscore .
\section{Chapoeirada}
\begin{itemize}
\item {Grp. gram.:f.}
\end{itemize}
\begin{itemize}
\item {Utilização:Prov.}
\end{itemize}
O mesmo que \textunderscore chapoirada\textunderscore .
\section{Chapoirada}
\begin{itemize}
\item {Grp. gram.:f.}
\end{itemize}
\begin{itemize}
\item {Utilização:Pop.}
\end{itemize}
Grande quantidade, o mesmo que \textunderscore chapeirada\textunderscore .
\section{Chaporrada}
\begin{itemize}
\item {Grp. gram.:f.}
\end{itemize}
\begin{itemize}
\item {Utilização:Prov.}
\end{itemize}
O mesmo que \textunderscore chapoirada\textunderscore .
\section{Chapota}
\begin{itemize}
\item {Grp. gram.:f.}
\end{itemize}
Acto de chapotar.
\section{Chapotar}
\begin{itemize}
\item {Grp. gram.:v. t.}
\end{itemize}
\begin{itemize}
\item {Proveniência:(Fr. \textunderscore chapoter\textunderscore )}
\end{itemize}
Aparar, cortar, os ramos inúteis a.
\section{Chaprão}
\begin{itemize}
\item {Grp. gram.:m.}
\end{itemize}
\begin{itemize}
\item {Utilização:Açor}
\end{itemize}
\begin{itemize}
\item {Utilização:Prov.}
\end{itemize}
\begin{itemize}
\item {Utilização:minh.}
\end{itemize}
\begin{itemize}
\item {Grp. gram.:Pl.}
\end{itemize}
\begin{itemize}
\item {Utilização:Bras}
\end{itemize}
Pessôa mal feita de corpo, desgraciosa.
Prancha grande; pau tôsco, que serve de malhal.
Barrote ou madeiro a prumo, sôbre que assentam os eixos de um engenho.
(Por \textunderscore chãprão\textunderscore , metáth. de \textunderscore pranchão\textunderscore )
\section{Chaptália}
\begin{itemize}
\item {Grp. gram.:f.}
\end{itemize}
\begin{itemize}
\item {Proveniência:(De \textunderscore Chaptal\textunderscore , n. p.)}
\end{itemize}
Género de plantas, da fam. das compostas.
\section{Chapuçada}
\begin{itemize}
\item {Grp. gram.:f.}
\end{itemize}
\begin{itemize}
\item {Utilização:Prov.}
\end{itemize}
\begin{itemize}
\item {Utilização:beir.}
\end{itemize}
\begin{itemize}
\item {Proveniência:(De \textunderscore chapuçar\textunderscore )}
\end{itemize}
Porção (de água), que se deita fóra ou que se atira a alguém.
\section{Chapuçar}
\begin{itemize}
\item {Grp. gram.:v. t.}
\end{itemize}
\begin{itemize}
\item {Utilização:Prov.}
\end{itemize}
\begin{itemize}
\item {Utilização:beir.}
\end{itemize}
O mesmo que \textunderscore atolar\textunderscore .
(Talvez alter. de \textunderscore chapuzar\textunderscore )
\section{Chapuço}
\begin{itemize}
\item {Grp. gram.:m.}
\end{itemize}
\begin{itemize}
\item {Utilização:Prov.}
\end{itemize}
\begin{itemize}
\item {Proveniência:(De \textunderscore chapuçar\textunderscore )}
\end{itemize}
Poça de lama.
\section{Chaputa}
\begin{itemize}
\item {Grp. gram.:f.}
\end{itemize}
Peixe marítimo, de corpo azul-escuro e em fórma de rabo de andorinha.
\section{Chapuz}
\begin{itemize}
\item {Grp. gram.:m.}
\end{itemize}
\begin{itemize}
\item {Grp. gram.:Loc. adv.}
\end{itemize}
\begin{itemize}
\item {Proveniência:(De \textunderscore chapa\textunderscore )}
\end{itemize}
Pedaço de madeira, embebido na parede, para nelle se pregar qualquer objecto.
Gancho, que segura o varal no cilhão.
Cunha.
Peça de madeira, em que assenta a culatra da peça de artilharia.
Chumaço, por baixo das curvas do beque do navio.
\textunderscore De chapuz\textunderscore , de chapa; repentinamente; imprevistamente.
De cabeça para baixo.
\section{Chapuzar}
\begin{itemize}
\item {Grp. gram.:v. t.}
\end{itemize}
\begin{itemize}
\item {Utilização:Des.}
\end{itemize}
\begin{itemize}
\item {Proveniência:(De \textunderscore chapuz\textunderscore )}
\end{itemize}
Lançar de cabeça para baixo.
\section{Chaquéo}
\begin{itemize}
\item {Grp. gram.:m.}
\end{itemize}
Certa maneira de esporear o cavallo.
\section{Chaquér}
\begin{itemize}
\item {Grp. gram.:m.}
\end{itemize}
\begin{itemize}
\item {Utilização:Ant.}
\end{itemize}
Vaso de coiro curtido, em que se refrescava a água.
\section{Chaquéu}
\begin{itemize}
\item {Grp. gram.:m.}
\end{itemize}
Certa maneira de esporear o cavallo.
\section{Chaquiça}
\begin{itemize}
\item {Grp. gram.:f.}
\end{itemize}
\begin{itemize}
\item {Utilização:Prov.}
\end{itemize}
\begin{itemize}
\item {Utilização:trasm.}
\end{itemize}
Lenha de urgueira.
O mesmo que \textunderscore chaquiço\textunderscore .
\section{Chaquiçar}
\begin{itemize}
\item {Grp. gram.:v. t.}
\end{itemize}
\begin{itemize}
\item {Utilização:Prov.}
\end{itemize}
\begin{itemize}
\item {Utilização:trasm.}
\end{itemize}
\begin{itemize}
\item {Proveniência:(De \textunderscore chaquiça\textunderscore )}
\end{itemize}
Aguçar, (varas, tanchões).
\section{Chaquiço}
\begin{itemize}
\item {Grp. gram.:m.}
\end{itemize}
\begin{itemize}
\item {Utilização:Prov.}
\end{itemize}
\begin{itemize}
\item {Utilização:trasm.}
\end{itemize}
Parte inferior das estevas, giestas e outras plantas, comprehendendo a raiz e a parte do caule.
Accendalhas.
\section{Chara}
\begin{itemize}
\item {fónica:ca}
\end{itemize}
\begin{itemize}
\item {Grp. gram.:m.}
\end{itemize}
\begin{itemize}
\item {Proveniência:(Lat. \textunderscore chara\textunderscore )}
\end{itemize}
Gênero de algas submergidas.
\section{Chará}
\begin{itemize}
\item {Grp. gram.:m.  e  f.}
\end{itemize}
\begin{itemize}
\item {Utilização:Bras. do N}
\end{itemize}
Pessôa, que tem o mesmo nome que outra; homónymo.
\section{Charaban}
\begin{itemize}
\item {Grp. gram.:m.}
\end{itemize}
Fórma, proposta, como aportuguesamento do fr. \textunderscore char-à-bancs\textunderscore .
\section{Charabasca}
\begin{itemize}
\item {Grp. gram.:f.}
\end{itemize}
\begin{itemize}
\item {Utilização:Prov.}
\end{itemize}
\begin{itemize}
\item {Utilização:trasm.}
\end{itemize}
Terra de pouco valor, ou estéril.
\section{Charabasco}
\begin{itemize}
\item {Grp. gram.:m.}
\end{itemize}
\begin{itemize}
\item {Utilização:Prov.}
\end{itemize}
\begin{itemize}
\item {Utilização:trasm.}
\end{itemize}
O mesmo que \textunderscore charabasca\textunderscore .
\section{Charabasqueira}
\begin{itemize}
\item {Grp. gram.:f.}
\end{itemize}
O mesmo que \textunderscore charabasca\textunderscore .
\section{Charabilhano}
\begin{itemize}
\item {Grp. gram.:m.}
\end{itemize}
\begin{itemize}
\item {Utilização:Prov.}
\end{itemize}
\begin{itemize}
\item {Utilização:trasm.}
\end{itemize}
Chouriço, o mesmo que \textunderscore chaviano\textunderscore .
\section{Charáceas}
\begin{itemize}
\item {fónica:ca}
\end{itemize}
\begin{itemize}
\item {Grp. gram.:f. pl.}
\end{itemize}
Familia de plantas cryptogâmicas, que têm por typo o \textunderscore chara\textunderscore .
\section{Charachina}
\begin{itemize}
\item {Grp. gram.:f.}
\end{itemize}
\begin{itemize}
\item {Proveniência:(Do mal. \textunderscore tara\textunderscore , modo, e \textunderscore China\textunderscore , n. p.)}
\end{itemize}
Costume chinês.
Modo da China. Cf. \textunderscore Peregrinação\textunderscore , XLVII, LXII e LXXVII.
\section{Characina}
\begin{itemize}
\item {Grp. gram.:f.}
\end{itemize}
\begin{itemize}
\item {Proveniência:(Do mal. \textunderscore tara\textunderscore , modo, e \textunderscore China\textunderscore , n. p.)}
\end{itemize}
Costume chinês.
Modo da China. Cf. \textunderscore Peregrinação\textunderscore , XLVII, LXII e LXXVII.
\section{Charada}
\begin{itemize}
\item {Grp. gram.:f.}
\end{itemize}
\begin{itemize}
\item {Utilização:Fig.}
\end{itemize}
\begin{itemize}
\item {Proveniência:(Fr. \textunderscore charade\textunderscore )}
\end{itemize}
Espécie de problema ou adivinha, cuja solução depende da decifração de cada uma das partes, em que a adivinha se decompõe.
Linguagem obscura.
\section{Charadista}
\begin{itemize}
\item {Grp. gram.:m.  e  f.}
\end{itemize}
Pessôa, que faz ou adivinha charadas.
\section{Charamba}
\begin{itemize}
\item {Grp. gram.:f.}
\end{itemize}
Dança popular nos Açores.
\section{Charamega}
\begin{itemize}
\item {Grp. gram.:f.}
\end{itemize}
\begin{itemize}
\item {Utilização:T. da Bairrada}
\end{itemize}
Urze, o mesmo que \textunderscore queiró\textunderscore .
\section{Charamela}
\begin{itemize}
\item {Grp. gram.:f.}
\end{itemize}
\begin{itemize}
\item {Grp. gram.:M.}
\end{itemize}
\begin{itemize}
\item {Grp. gram.:F.}
\end{itemize}
\begin{itemize}
\item {Utilização:ant.}
\end{itemize}
\begin{itemize}
\item {Utilização:Pop.}
\end{itemize}
\begin{itemize}
\item {Proveniência:(Do lat. \textunderscore calamellus\textunderscore )}
\end{itemize}
Antigo instrumento, com uma palheta metida em cápsula ou barrilete, onde se soprava com fôrça, como nas buzinas.--Havia charamelas de três dimensões: \textunderscore bastarda\textunderscore , \textunderscore média\textunderscore , e \textunderscore charamelinha\textunderscore . O \textunderscore oboé\textunderscore  e o \textunderscore clarinete\textunderscore  são a charamela, aperfeiçoada.--
Tocador de charamela.
Banda de música, composta de instrumentos de sôpro.
Registo dos órgãos antigos, com tubos de estanho.
O mesmo que \textunderscore charanga\textunderscore .
\section{Charameleiro}
\begin{itemize}
\item {Grp. gram.:m.}
\end{itemize}
Aquelle que toca charamela.
\section{Charamelinha}
\begin{itemize}
\item {Grp. gram.:f.}
\end{itemize}
Pequena charamela, que se afinava numa quinta acima da charamela média.
\section{Charanga}
\begin{itemize}
\item {Grp. gram.:f.}
\end{itemize}
Conjunto, corporação, de músicos, que tocam principalmente instrumentos de sôpro.
(Cast. \textunderscore charanga\textunderscore )
\section{Charangueiro}
\begin{itemize}
\item {Grp. gram.:m.}
\end{itemize}
\begin{itemize}
\item {Utilização:Pop.}
\end{itemize}
Músico de charanga.
\section{Charão}
\begin{itemize}
\item {Grp. gram.:m.}
\end{itemize}
Verniz especial da China e do Japão.
(Cp. cast. \textunderscore charo\textunderscore )
\section{Charapa}
\begin{itemize}
\item {Grp. gram.:f.}
\end{itemize}
Pequena tartaruga do Maranhão.
\section{Charapim}
\begin{itemize}
\item {Grp. gram.:m.}
\end{itemize}
\begin{itemize}
\item {Utilização:Bras. do N}
\end{itemize}
O mesmo que \textunderscore chará\textunderscore .
\section{Charatone}
\begin{itemize}
\item {Grp. gram.:m.}
\end{itemize}
Antiga embarcação indiana.
(Cp. \textunderscore tone\textunderscore )
\section{Charavascal}
\begin{itemize}
\item {Grp. gram.:m.}
\end{itemize}
\begin{itemize}
\item {Utilização:Prov.}
\end{itemize}
O mesmo que \textunderscore chavascal\textunderscore .
(Cp. \textunderscore charabasca\textunderscore )
\section{Charca}
\begin{itemize}
\item {Grp. gram.:f.}
\end{itemize}
\begin{itemize}
\item {Utilização:Prov.}
\end{itemize}
\begin{itemize}
\item {Utilização:trasm.}
\end{itemize}
O mesmo que \textunderscore charco\textunderscore .
Espécie de passarinho.
\section{Charco}
\begin{itemize}
\item {Grp. gram.:m.}
\end{itemize}
\begin{itemize}
\item {Utilização:T. do Fundão}
\end{itemize}
Água immunda e estagnada.
Lodaçal; atoleiro.
Tanque de chafariz.
(Cast. \textunderscore charco\textunderscore )
\section{Chardó}
\begin{itemize}
\item {Grp. gram.:m.}
\end{itemize}
Guerreiro, homem da segunda casta indiana, segundo a organização brahmânica; o mesmo que \textunderscore xátria\textunderscore .
(Conc. \textunderscore charado\textunderscore )
\section{Charela}
\begin{itemize}
\item {Grp. gram.:f.}
\end{itemize}
Espécie de perdiz, (\textunderscore perdix cinerea\textunderscore , Br.).
\section{Carisma}
\begin{itemize}
\item {Grp. gram.:m.}
\end{itemize}
\begin{itemize}
\item {Proveniência:(Gr. \textunderscore kharisma\textunderscore )}
\end{itemize}
Graça divina.
\section{Charepe}
\begin{itemize}
\item {Grp. gram.:m.}
\end{itemize}
\begin{itemize}
\item {Utilização:Prov.}
\end{itemize}
Sujeito desavergonhado; bisbórria; garoto. (Colhido na Bairrada)
\section{Chareta}
\begin{itemize}
\item {fónica:charê}
\end{itemize}
\begin{itemize}
\item {Grp. gram.:f.}
\end{itemize}
\begin{itemize}
\item {Utilização:Prov.}
\end{itemize}
\begin{itemize}
\item {Utilização:alg.}
\end{itemize}
Dobra de pano, franzida, que contém cordão ou fita corredia.
\section{Charéu}
\begin{itemize}
\item {Grp. gram.:m.}
\end{itemize}
\begin{itemize}
\item {Utilização:Prov.}
\end{itemize}
\begin{itemize}
\item {Utilização:dur.}
\end{itemize}
Repreensão, descompostura.
\section{Charguça}
\begin{itemize}
\item {Grp. gram.:f.}
\end{itemize}
\begin{itemize}
\item {Utilização:Ant.}
\end{itemize}
Provavelmente o mesmo que \textunderscore çarguça\textunderscore . Cf. \textunderscore Lembranças das cousas da Índia\textunderscore , nos \textunderscore Subsídios\textunderscore  de Felner.
\section{Charisma}
\begin{itemize}
\item {fónica:ca}
\end{itemize}
\begin{itemize}
\item {Grp. gram.:m.}
\end{itemize}
\begin{itemize}
\item {Proveniência:(Gr. \textunderscore kharisma\textunderscore )}
\end{itemize}
Graça divina.
\section{Charites}
\begin{itemize}
\item {Grp. gram.:f. pl.}
\end{itemize}
\begin{itemize}
\item {Utilização:Ant.}
\end{itemize}
Graças. Cf. \textunderscore Laura de Anfriso\textunderscore , 126.
\section{Charivari}
\begin{itemize}
\item {Grp. gram.:m.}
\end{itemize}
\begin{itemize}
\item {Proveniência:(T. fr.)}
\end{itemize}
Chinfrim, desordem, balbúrdia.
\section{Charla}
\begin{itemize}
\item {Grp. gram.:f.}
\end{itemize}
\begin{itemize}
\item {Proveniência:(De \textunderscore charlar\textunderscore )}
\end{itemize}
Conversa á tôa.
\section{Charlador}
\begin{itemize}
\item {Grp. gram.:m.}
\end{itemize}
Aquelle que charla.
\section{Charlar}
\begin{itemize}
\item {Grp. gram.:v. i.}
\end{itemize}
\begin{itemize}
\item {Proveniência:(It. \textunderscore ciarlare\textunderscore , cast. \textunderscore charlar\textunderscore )}
\end{itemize}
Falar á tôa; palrar.
\section{Charlatanaria}
\begin{itemize}
\item {Grp. gram.:f.}
\end{itemize}
Modos, linguagem, de charlatão.
\section{Charlatanear}
\begin{itemize}
\item {Grp. gram.:v. i.}
\end{itemize}
Têr modos de charlatão; falar como charlatão.
\section{Charlatanesco}
\begin{itemize}
\item {Grp. gram.:adj.}
\end{itemize}
Relativo a charlatão; próprio de charlatão.
\section{Charlatanice}
\begin{itemize}
\item {Grp. gram.:f.}
\end{itemize}
O mesmo que \textunderscore charlatanismo\textunderscore .
\section{Charlatânico}
\begin{itemize}
\item {Grp. gram.:adj.}
\end{itemize}
\begin{itemize}
\item {Utilização:Bras}
\end{itemize}
O mesmo que \textunderscore charlatanesco\textunderscore .
\section{Charlatanismo}
\begin{itemize}
\item {Grp. gram.:m.}
\end{itemize}
Qualidade de quem é charlatão; charlatanaria.
\section{Charlatão}
\begin{itemize}
\item {Grp. gram.:m.}
\end{itemize}
\begin{itemize}
\item {Proveniência:(It. \textunderscore ciarlatano\textunderscore , de \textunderscore ciarlare\textunderscore , charlar)}
\end{itemize}
Aquelle que publicamente vende drogas, apregoando exaggeradamente a virtude dellas.
Aquelle que explora a bôa fé do público; impostor.
\section{Charlateira}
\begin{itemize}
\item {Grp. gram.:f.}
\end{itemize}
Espécie de dragona de metal doirado, usada por officiaes militares.
\section{Charlota}
\begin{itemize}
\item {Grp. gram.:f.}
\end{itemize}
Pastelão de creme, cercado de lascas de pão de ló.
\section{Charneca}
\begin{itemize}
\item {Grp. gram.:f.}
\end{itemize}
Terreno inculto, em que só crescem rasteiras plantas silvestres.
(Cast. \textunderscore charneca\textunderscore )
\section{Charneco}
\begin{itemize}
\item {Grp. gram.:m.}
\end{itemize}
\begin{itemize}
\item {Utilização:Prov.}
\end{itemize}
\begin{itemize}
\item {Utilização:alg.}
\end{itemize}
\begin{itemize}
\item {Utilização:alent.}
\end{itemize}
O mesmo que \textunderscore rabilongo\textunderscore .
\section{Charneira}
\begin{itemize}
\item {Grp. gram.:f.}
\end{itemize}
Reunião de peças, que giram num eixo commum.
União das valvas da concha.
Extremidade de uma correia com fivela.
(B. lat. \textunderscore cardinaria\textunderscore , do lat. \textunderscore cardo\textunderscore , \textunderscore cardinis\textunderscore )
\section{Charnequeiro}
\begin{itemize}
\item {Grp. gram.:adj.}
\end{itemize}
Relativo a charneca.
Que vive em charneca: \textunderscore cabras charnequeiras\textunderscore .
\section{Charnequenho}
\begin{itemize}
\item {Grp. gram.:adj.}
\end{itemize}
O mesmo que \textunderscore charnequeiro\textunderscore .
\section{Charoar}
\begin{itemize}
\item {Grp. gram.:v. t.}
\end{itemize}
(V.acharoar)
\section{Charodó}
\begin{itemize}
\item {Grp. gram.:m.}
\end{itemize}
(V.chardó)
\section{Charola}
\begin{itemize}
\item {Grp. gram.:f.}
\end{itemize}
\begin{itemize}
\item {Utilização:Prov.}
\end{itemize}
\begin{itemize}
\item {Utilização:trasm.}
\end{itemize}
\begin{itemize}
\item {Utilização:T. de Alcanena}
\end{itemize}
Andor para imagens religiosas.
Nicho.
O mesmo que \textunderscore procissão\textunderscore .
Pequena fábrica de curtumes.
(Cp. cast. \textunderscore charol\textunderscore )
\section{Charoleiro}
\begin{itemize}
\item {Grp. gram.:m.}
\end{itemize}
\begin{itemize}
\item {Utilização:T. de Alcanena}
\end{itemize}
Aquelle que faz charolas ou andores.
Dono de fábrica, chamada \textunderscore charola\textunderscore .
\section{Charpa}
\begin{itemize}
\item {Grp. gram.:f.}
\end{itemize}
\begin{itemize}
\item {Proveniência:(Fr. \textunderscore écharpe\textunderscore )}
\end{itemize}
Banda de pano.
Cinta.
Suspensório, em que se apoia o braço doente.
\section{Charque}
\begin{itemize}
\item {Grp. gram.:m.}
\end{itemize}
\begin{itemize}
\item {Utilização:Bras}
\end{itemize}
Carne salgada e sêca.
Preparação dessa carne.
(Do quichua \textunderscore chharque\textunderscore )
\section{Charqueação}
\begin{itemize}
\item {Grp. gram.:f.}
\end{itemize}
\begin{itemize}
\item {Utilização:Bras}
\end{itemize}
Acto de \textunderscore charquear\textunderscore .
\section{Charqueada}
\begin{itemize}
\item {Grp. gram.:f.}
\end{itemize}
\begin{itemize}
\item {Utilização:Bras}
\end{itemize}
\begin{itemize}
\item {Proveniência:(De \textunderscore charquear\textunderscore )}
\end{itemize}
Grande estabelecimento, em que se prepara o charque.
\section{Charqueador}
\begin{itemize}
\item {Grp. gram.:m.}
\end{itemize}
\begin{itemize}
\item {Utilização:Bras}
\end{itemize}
\begin{itemize}
\item {Proveniência:(De \textunderscore charquear\textunderscore )}
\end{itemize}
Fabricante de charque.
Proprietário de charqueada.
\section{Charquear}
\begin{itemize}
\item {Grp. gram.:v. t.}
\end{itemize}
\begin{itemize}
\item {Utilização:Bras}
\end{itemize}
Preparar o charque.
\section{Charqueio}
\begin{itemize}
\item {Grp. gram.:m.}
\end{itemize}
O mesmo que \textunderscore charqueação\textunderscore .
\section{Charqueiró}
\begin{itemize}
\item {Grp. gram.:m.}
\end{itemize}
\begin{itemize}
\item {Grp. gram.:Adj.}
\end{itemize}
(V.charco)
Lodacento. Cf. \textunderscore Viriato\textunderscore , \textunderscore Trág.\textunderscore , XIV, 87.
\section{Charrafusca}
\begin{itemize}
\item {Grp. gram.:f.}
\end{itemize}
\begin{itemize}
\item {Utilização:Prov.}
\end{itemize}
\begin{itemize}
\item {Utilização:beir.}
\end{itemize}
(V.sarrafusca)
\section{Charramente}
\begin{itemize}
\item {fónica:chá}
\end{itemize}
\begin{itemize}
\item {Grp. gram.:adv.}
\end{itemize}
De modo charro; grosseiramente.
\section{Charrano}
\begin{itemize}
\item {Grp. gram.:m.}
\end{itemize}
\begin{itemize}
\item {Utilização:Prov.}
\end{itemize}
\begin{itemize}
\item {Utilização:alg.}
\end{itemize}
O mesmo que \textunderscore gaivina\textunderscore .
\section{Charrasca}
\begin{itemize}
\item {Grp. gram.:f.}
\end{itemize}
\begin{itemize}
\item {Utilização:Prov.}
\end{itemize}
O mesmo que \textunderscore papa-amoras\textunderscore .
\section{Charrasqueira}
\begin{itemize}
\item {Grp. gram.:f.}
\end{itemize}
\begin{itemize}
\item {Utilização:Prov.}
\end{itemize}
\begin{itemize}
\item {Utilização:beir.}
\end{itemize}
Castanheiro bravo, próprio para madeira de construcção.
(Cp. \textunderscore carrasqueiro\textunderscore )
\section{Charrela}
\begin{itemize}
\item {Grp. gram.:f.}
\end{itemize}
Perdiz parda da Beira-Alta.
\section{Charreu}
\begin{itemize}
\item {Grp. gram.:m.}
\end{itemize}
Peixe de Portugal.
\section{Charro}
\begin{itemize}
\item {Grp. gram.:adj.}
\end{itemize}
Grosseiro, bronco, rude.
(Cast. \textunderscore charro\textunderscore )
\section{Charro}
\begin{itemize}
\item {Grp. gram.:m.}
\end{itemize}
\begin{itemize}
\item {Utilização:Prov.}
\end{itemize}
\begin{itemize}
\item {Utilização:alg.}
\end{itemize}
O mesmo que \textunderscore chicharro\textunderscore .
\section{Charroco}
\begin{itemize}
\item {fónica:rô}
\end{itemize}
\begin{itemize}
\item {Grp. gram.:m.}
\end{itemize}
\begin{itemize}
\item {Utilização:Prov.}
\end{itemize}
Espécie de picanço, o mesmo que \textunderscore pica-porco\textunderscore . (Colhido na Bairrada)
\section{Charrua}
\begin{itemize}
\item {Grp. gram.:f.}
\end{itemize}
\begin{itemize}
\item {Utilização:Fig.}
\end{itemize}
\begin{itemize}
\item {Proveniência:(Fr. \textunderscore charrue\textunderscore )}
\end{itemize}
Espécie de arado, com uma só aiveca.
A agricultura.
\section{Charruada}
\begin{itemize}
\item {Grp. gram.:f.}
\end{itemize}
\begin{itemize}
\item {Proveniência:(De \textunderscore charruar\textunderscore )}
\end{itemize}
Terreno lavrado.
\section{Charruadela}
\begin{itemize}
\item {Grp. gram.:f.}
\end{itemize}
Acto de charruar.
\section{Charruar}
\begin{itemize}
\item {Grp. gram.:v. t.}
\end{itemize}
Lavrar com charrua. Cf. \textunderscore Museu Techn.\textunderscore , 78.
\section{Charruas}
\begin{itemize}
\item {Grp. gram.:m. pl.}
\end{itemize}
Índios aguerridos e cruéis, que dominavam entre o rio da Prata e o Uruguai, e de que ainda há famílias dispersas nas matas do Brasil.
\section{Charrueco}
\begin{itemize}
\item {Grp. gram.:m.}
\end{itemize}
\begin{itemize}
\item {Proveniência:(De \textunderscore charrua\textunderscore )}
\end{itemize}
Charrua grosseira do Alentejo.
\section{Chartamina}
\begin{itemize}
\item {Grp. gram.:f.}
\end{itemize}
\begin{itemize}
\item {Utilização:Bras}
\end{itemize}
Matéria corante, carmim de açafrão.
\section{Charutaria}
\begin{itemize}
\item {Grp. gram.:f.}
\end{itemize}
\begin{itemize}
\item {Utilização:Bras}
\end{itemize}
Lugar, onde se vendem charutos; tabacaria.
\section{Charuteira}
\begin{itemize}
\item {Grp. gram.:f.}
\end{itemize}
Caixinha de algibeira, para charutos.
\section{Charuteiro}
\begin{itemize}
\item {Grp. gram.:m.}
\end{itemize}
Aquelle que, nas fábricas de tabaco, manipula charutos.
Pequeno insecto que, atacando as fôlhas da videira, as enrola, em fórma de charuto.
Espécie de tabaco.
\section{Charuto}
\begin{itemize}
\item {Grp. gram.:m.}
\end{itemize}
\begin{itemize}
\item {Utilização:Bras}
\end{itemize}
\begin{itemize}
\item {Proveniência:(Do ingl. \textunderscore sheroot\textunderscore )}
\end{itemize}
Rôlo de fôlhas sêcas de tabaco, preparado para se fumar.
Bolo em fórma de charuto.
Pequeno barco de recreio, governado por um só remo, terminado em pá de ambos os lados e com que, alternadamente, se rasga a água, de uma e outra banda.
\section{Chasca}
\begin{itemize}
\item {Grp. gram.:f.}
\end{itemize}
\begin{itemize}
\item {Utilização:Prov.}
\end{itemize}
\begin{itemize}
\item {Utilização:trasm.}
\end{itemize}
\begin{itemize}
\item {Utilização:Prov.}
\end{itemize}
\begin{itemize}
\item {Utilização:trasm.}
\end{itemize}
Pássaro muito pequeno e muito vivo.
Rapariga leviana.
Pião pequeno.
O mesmo que \textunderscore chasco\textunderscore ^2?
\section{Chás-chás}
\begin{itemize}
\item {Grp. gram.:m.}
\end{itemize}
\begin{itemize}
\item {Utilização:Prov.}
\end{itemize}
Design. onom. do \textunderscore cartaxo\textunderscore .
\section{Chasco}
\begin{itemize}
\item {Grp. gram.:m.}
\end{itemize}
Gracejo satírico, motejo.
(Do al., segundo \textunderscore Körting\textunderscore )
\section{Chasco}
\begin{itemize}
\item {Grp. gram.:m.}
\end{itemize}
Espécie de cartaxo, (\textunderscore pratincola rubetra\textunderscore , Lin.).
(Talvez t. onom. Cp. \textunderscore chás-chás\textunderscore )
\section{Chasco-branco}
\begin{itemize}
\item {Grp. gram.:m.}
\end{itemize}
\begin{itemize}
\item {Utilização:Prov.}
\end{itemize}
O mesmo que \textunderscore tanjasno\textunderscore .
\section{Chasco-de-leque}
\begin{itemize}
\item {Grp. gram.:m.}
\end{itemize}
O mesmo que \textunderscore rabo-branco\textunderscore .
\section{Chasco-do-rêgo}
\begin{itemize}
\item {Grp. gram.:m.}
\end{itemize}
\begin{itemize}
\item {Utilização:Prov.}
\end{itemize}
O mesmo que \textunderscore pica-peixe\textunderscore , ave.
\section{Chaspa}
\begin{itemize}
\item {Grp. gram.:f.}
\end{itemize}
\begin{itemize}
\item {Utilização:Prov.}
\end{itemize}
\begin{itemize}
\item {Utilização:trasm.}
\end{itemize}
Panela ou tacho, com tampa, largo e baixo.
\section{Chaspulho}
\begin{itemize}
\item {Grp. gram.:m.}
\end{itemize}
Nome, com que, nos bastidores, se designa o reprego ou trainel, que se colloca insuladamente em scena, representando plantas, flôres, etc.
\section{Chasqueador}
\begin{itemize}
\item {Grp. gram.:m.  e  adj.}
\end{itemize}
O que chasqueia.
\section{Chasquear}
\begin{itemize}
\item {Grp. gram.:v. t.}
\end{itemize}
\begin{itemize}
\item {Grp. gram.:V. i.}
\end{itemize}
\begin{itemize}
\item {Proveniência:(De \textunderscore chasco\textunderscore ^1)}
\end{itemize}
Dirigir chascos a.
Dizer chascos.
\section{Chasqueiro}
\begin{itemize}
\item {Grp. gram.:adj.}
\end{itemize}
\begin{itemize}
\item {Utilização:Bras. do S}
\end{itemize}
Diz-se do trote largo e incômmodo.
\section{Chasqueta}
\begin{itemize}
\item {fónica:quê}
\end{itemize}
\begin{itemize}
\item {Grp. gram.:f.}
\end{itemize}
\begin{itemize}
\item {Utilização:Prov.}
\end{itemize}
\begin{itemize}
\item {Utilização:trasm.}
\end{itemize}
\begin{itemize}
\item {Proveniência:(De \textunderscore chasca\textunderscore )}
\end{itemize}
Rapariga leviana, de pouco juizo.
\section{Chasquete}
\begin{itemize}
\item {fónica:quê}
\end{itemize}
\begin{itemize}
\item {Grp. gram.:m.}
\end{itemize}
\begin{itemize}
\item {Utilização:Ant.}
\end{itemize}
Rapaz leviano, doidivanas:«\textunderscore um chasquete de primeira tonsura\textunderscore ». \textunderscore Anat. Joc.\textunderscore , 37.
(Cp. \textunderscore chasqueta\textunderscore )
\section{Chasselas}
\begin{itemize}
\item {Grp. gram.:f.}
\end{itemize}
Variedade de videira americana.
\section{Chassellas}
\begin{itemize}
\item {Grp. gram.:f.}
\end{itemize}
Variedade de videira americana.
\section{Chata}
\begin{itemize}
\item {Grp. gram.:f.}
\end{itemize}
\begin{itemize}
\item {Utilização:Bras}
\end{itemize}
\begin{itemize}
\item {Proveniência:(De \textunderscore chato\textunderscore )}
\end{itemize}
Barcaça larga e pouco funda.
Embarcação de duas prôas, fortemente construida e de fundo chato.
\section{Chata}
\begin{itemize}
\item {Grp. gram.:f.}
\end{itemize}
\begin{itemize}
\item {Proveniência:(T. as.)}
\end{itemize}
Jantar, que em dia de entêrro se dava entre christãos.
\section{Chatamente}
\begin{itemize}
\item {Grp. gram.:adj.}
\end{itemize}
De modo chato; com chateza.
\section{Chaté}
\begin{itemize}
\item {Grp. gram.:m.}
\end{itemize}
Espécie de dança antiga?«\textunderscore ...danças altas, chatés, quartas e outavas.\textunderscore »Filinto, IX, 142.
\section{Chatear}
\begin{itemize}
\item {Grp. gram.:v. t.}
\end{itemize}
\begin{itemize}
\item {Utilização:Fam.}
\end{itemize}
\begin{itemize}
\item {Proveniência:(De \textunderscore chato\textunderscore )}
\end{itemize}
Importunar, maçar.
\section{Chateza}
\begin{itemize}
\item {Grp. gram.:f.}
\end{itemize}
Qualidade do que é chato.
\section{Chatice}
\begin{itemize}
\item {Grp. gram.:f.}
\end{itemize}
\begin{itemize}
\item {Utilização:Fam.}
\end{itemize}
\begin{itemize}
\item {Proveniência:(De \textunderscore chato\textunderscore )}
\end{itemize}
Qualidade reles; baixeza.
Qualidade de maçador; maçada.
\section{Chatifa}
\begin{itemize}
\item {Grp. gram.:m.}
\end{itemize}
Um dos quatro ritos orthodoxos dos Muçulmanos. Cf. Benalcanfor, \textunderscore C. de Viagem\textunderscore , XXX.
\section{Chatim}
\begin{itemize}
\item {Grp. gram.:m.}
\end{itemize}
Traficante; velhaco.
(Do conc.)
\section{Chatinador}
\begin{itemize}
\item {Grp. gram.:m.}
\end{itemize}
O mesmo que \textunderscore chatim\textunderscore .
\section{Chatinar}
\begin{itemize}
\item {Grp. gram.:v. i.}
\end{itemize}
\begin{itemize}
\item {Proveniência:(De \textunderscore chatim\textunderscore )}
\end{itemize}
Negociar sem escrupulo; traficar.
\section{Chatinaria}
\begin{itemize}
\item {Grp. gram.:f.}
\end{itemize}
Negócio de chatins.
\section{Chato}
\begin{itemize}
\item {Grp. gram.:adj.}
\end{itemize}
\begin{itemize}
\item {Grp. gram.:M.  e  adj.}
\end{itemize}
\begin{itemize}
\item {Utilização:Fam.}
\end{itemize}
\begin{itemize}
\item {Proveniência:(Gr. \textunderscore platus\textunderscore )}
\end{itemize}
Plano, sem relêvo, sem saliências.
Magro.
Acanhado.
Trivial, baixo, vulgar.
Maçador, importuno.
\section{Chatos}
\begin{itemize}
\item {Grp. gram.:m. pl.}
\end{itemize}
Espécie de piôlhos.
(Cp. \textunderscore chato\textunderscore )
\section{Chau}
\begin{itemize}
\item {Grp. gram.:interj.}
\end{itemize}
\begin{itemize}
\item {Utilização:Prov.}
\end{itemize}
Serve para recommendar segrêdo ou silêncio: \textunderscore a respeito do que eu te disse, chau!\textunderscore  (Colhido em Turquel)
\section{Chaudel}
\begin{itemize}
\item {Grp. gram.:m.}
\end{itemize}
Pano de Bengala, que servia para cobertas; o mesmo que \textunderscore chadér\textunderscore .
\section{Chauitas}
\begin{itemize}
\item {Grp. gram.:m. pl.}
\end{itemize}
Indígenas do Brasil, nas margens do Javari.
\section{Chauvin}
\begin{itemize}
\item {fónica:xôvén}
\end{itemize}
\begin{itemize}
\item {Grp. gram.:m.}
\end{itemize}
\begin{itemize}
\item {Proveniência:(De \textunderscore Chauvin\textunderscore , n. p. fr.)}
\end{itemize}
Aquelle que tem sentimentos exaggerados e ridículos de patriotismo e de guerra.
\section{Chauvinismo}
\begin{itemize}
\item {fónica:xô}
\end{itemize}
\begin{itemize}
\item {Grp. gram.:m.}
\end{itemize}
Sentimentos de \textunderscore chauvin\textunderscore .
\section{Chavádego}
\begin{itemize}
\item {Grp. gram.:m.}
\end{itemize}
\begin{itemize}
\item {Proveniência:(De \textunderscore chave\textunderscore ?)}
\end{itemize}
Antiga pensão, que os foreiros pagavam aos senhorios.
Refeição ou molhadura, que se dava por occasião do conchavo ou ajuste entre os mosteiros e novos emphyteutas.
\section{Chavantes}
\begin{itemize}
\item {Grp. gram.:m. pl.}
\end{itemize}
Índios valorosos, que dominavam nas matas do Tocantins, (Brasil).
\section{Chavão}
\begin{itemize}
\item {Grp. gram.:m.}
\end{itemize}
\begin{itemize}
\item {Utilização:Fam.}
\end{itemize}
\begin{itemize}
\item {Proveniência:(De \textunderscore chave\textunderscore )}
\end{itemize}
Chave grande.
Fórma, molde ou marca, para bolos.
Modêlo, typo.
Autor ou obra, que tem grande autoridade.
Fórmula, muito repetida, de falar, de escrever ou de proceder.
\section{Chavaria}
\begin{itemize}
\item {Grp. gram.:f.}
\end{itemize}
Porção de chaves.
\section{Chavaria}
\begin{itemize}
\item {Grp. gram.:f.}
\end{itemize}
Ave americana.
\section{Chavascado}
\begin{itemize}
\item {Grp. gram.:adj.}
\end{itemize}
Achavascado; chavasco; tôsco.
\section{Chavascal}
\begin{itemize}
\item {Grp. gram.:m.}
\end{itemize}
\begin{itemize}
\item {Proveniência:(De \textunderscore chavasco\textunderscore )}
\end{itemize}
Baiuca immunda; pocilga.
Moitedo.
Terra pouco productiva.
\section{Chavascar}
\begin{itemize}
\item {Grp. gram.:v. t.}
\end{itemize}
\begin{itemize}
\item {Proveniência:(De \textunderscore chavasco\textunderscore )}
\end{itemize}
O mesmo que \textunderscore achavascar\textunderscore ; fazer toscamente.
\section{Chavasco}
\begin{itemize}
\item {Grp. gram.:adj.}
\end{itemize}
Grosseiro; bronco; mal feito.
\section{Chavasqueira}
\begin{itemize}
\item {Grp. gram.:f.}
\end{itemize}
\begin{itemize}
\item {Utilização:Prov.}
\end{itemize}
\begin{itemize}
\item {Utilização:trasm.}
\end{itemize}
\begin{itemize}
\item {Proveniência:(De \textunderscore chavasco\textunderscore )}
\end{itemize}
Terra ruím.
\section{Chavasqueiro}
\begin{itemize}
\item {Grp. gram.:adj.}
\end{itemize}
\begin{itemize}
\item {Grp. gram.:M.}
\end{itemize}
O mesmo que \textunderscore chavasco\textunderscore .
O mesmo que \textunderscore chavascal\textunderscore .
\section{Chavasquice}
\begin{itemize}
\item {Grp. gram.:f.}
\end{itemize}
Qualidade daquillo que é \textunderscore chavasco\textunderscore  ou chavascado.
\section{Chave}
\begin{itemize}
\item {Grp. gram.:f.}
\end{itemize}
\begin{itemize}
\item {Utilização:Prov.}
\end{itemize}
\begin{itemize}
\item {Utilização:alent.}
\end{itemize}
\begin{itemize}
\item {Utilização:Prov.}
\end{itemize}
\begin{itemize}
\item {Proveniência:(Do lat. \textunderscore clavis\textunderscore )}
\end{itemize}
Instrumento que, embebendo-se na abertura de uma fechadura, serve para abrir ou fechar portas, gavetas, caixas, etc.
Instrumento análogo, com que se dá corda a relógios.
Objecto, que serve para apertar, aparafusar, estender, fixar, etc.
Lugar, que fecha um território e póde sêr ponto estratégico contra inimigos.
Aquillo que facilita ou explica: \textunderscore a chave do enigma\textunderscore .
Insignia de posse ou de auctoridade: \textunderscore entregar as chaves da cidade\textunderscore .
A parte central e superior de uma construcção.
O principio ou o fim de um soneto ou de qualquer trabalho: \textunderscore o soneto deve abrir-se com chave de prata e fechar-se com chave de oiro.\textunderscore 
Cavilha de ferro, que atravessa a parte superior do fuso do lagar, prendendo-lhe o pêso ou a pedra, pelo veio.
Corno de boi, preparado como vasilha, para transportar azeite ou outros líquidos.
A palma (da mão).
A largura inferior (do pé).
Sinal orthográphico, para abranger com uma só designação differentes objectos ou termos.
Recanto ou cotovelo, que uma belga ou um terreno faz, para algum dos lados.
\section{Chaveco}
\begin{itemize}
\item {Grp. gram.:m.}
\end{itemize}
\begin{itemize}
\item {Utilização:Fam.}
\end{itemize}
\begin{itemize}
\item {Proveniência:(Do ár. \textunderscore xabeca\textunderscore )}
\end{itemize}
Pequena embarcação.
Barco pequeno e mal construído ou velho; embarcação ordinária.
\section{Chávega}
\begin{itemize}
\item {Grp. gram.:f.}
\end{itemize}
Rede para a pesca de peixe miúdo.
Barco, em que os pescadores levam essa rede.
(Da mesma or. que \textunderscore chaveco\textunderscore )
\section{Chaveira}
\begin{itemize}
\item {Grp. gram.:f.}
\end{itemize}
\begin{itemize}
\item {Proveniência:(Do b. lat. \textunderscore clavelus\textunderscore ?)}
\end{itemize}
Doença que o cysticerco, durante o seu estado de larva, produz nos porcos; cysticereose.
\section{Chaveirão}
\begin{itemize}
\item {Grp. gram.:m.}
\end{itemize}
\begin{itemize}
\item {Utilização:Heráld.}
\end{itemize}
\begin{itemize}
\item {Proveniência:(De \textunderscore chave\textunderscore )}
\end{itemize}
Barras em ângulo, nos escudos.
\section{Chaveirento}
\begin{itemize}
\item {Grp. gram.:m.}
\end{itemize}
Que tem chaveira.
\section{Chaveiro}
\begin{itemize}
\item {Grp. gram.:m.}
\end{itemize}
Aquelle que guarda chaves; claviculário.
Carcereiro.
\section{Chaveiroado}
\begin{itemize}
\item {Grp. gram.:m.}
\end{itemize}
\begin{itemize}
\item {Utilização:Heráld.}
\end{itemize}
\begin{itemize}
\item {Proveniência:(De \textunderscore chaveirão\textunderscore )}
\end{itemize}
Campo coberto de chaveirões de metal de côr. Cf. L. Ribeiro, \textunderscore Trat. de Armaria\textunderscore .
\section{Chaveiroso}
\begin{itemize}
\item {Grp. gram.:adj.}
\end{itemize}
Que tem chaveira.
\section{Chavelha}
\begin{itemize}
\item {fónica:vê}
\end{itemize}
\begin{itemize}
\item {Grp. gram.:f.}
\end{itemize}
\begin{itemize}
\item {Proveniência:(Do lat. \textunderscore clavicula\textunderscore )}
\end{itemize}
Peça de pau, que se mete no cabeçalho do carro, junto á canga, e também conhecida por \textunderscore mata-boi\textunderscore .
\section{Chavelhal}
\begin{itemize}
\item {Grp. gram.:m.}
\end{itemize}
\begin{itemize}
\item {Utilização:Prov.}
\end{itemize}
\begin{itemize}
\item {Utilização:trasm.}
\end{itemize}
Baraço, na extremidade do cabeçalho, para entrar a chavelha.
\section{Chavelhão}
\begin{itemize}
\item {Grp. gram.:m.}
\end{itemize}
\begin{itemize}
\item {Proveniência:(De \textunderscore chavelha\textunderscore )}
\end{itemize}
Peça de ferro, a que se atrela segunda junta de bois, para tirarem o carro ou o arado.
\section{Chavelho}
\begin{itemize}
\item {fónica:vê}
\end{itemize}
\begin{itemize}
\item {Grp. gram.:m.}
\end{itemize}
\begin{itemize}
\item {Utilização:Gír.}
\end{itemize}
Corno.
Copo.
(Cp. \textunderscore chavelha\textunderscore )
\section{Chávena}
\begin{itemize}
\item {Grp. gram.:f.}
\end{itemize}
\begin{itemize}
\item {Proveniência:(Do chin. \textunderscore cha-van\textunderscore )}
\end{itemize}
O mesmo que \textunderscore chícara\textunderscore .
\section{Chaveta}
\begin{itemize}
\item {fónica:vê}
\end{itemize}
\begin{itemize}
\item {Grp. gram.:f.}
\end{itemize}
\begin{itemize}
\item {Utilização:Prov.}
\end{itemize}
\begin{itemize}
\item {Proveniência:(De \textunderscore chave\textunderscore )}
\end{itemize}
Peça de ferro, na extremidade de um eixo, para não deixar sair as rodas, ou peça que segura uma cavilha.
Cavilha.
Pequena chave, como sinal orthográphico.
Haste, em que jogam as dobradiças.
O mesmo que \textunderscore chavelha\textunderscore .
\section{Chavetar}
\begin{itemize}
\item {Grp. gram.:v. t.}
\end{itemize}
Segurar com chaveta.
\section{Chaviana}
\begin{itemize}
\item {Grp. gram.:f.}
\end{itemize}
\begin{itemize}
\item {Utilização:Prov.}
\end{itemize}
\begin{itemize}
\item {Utilização:trasm.}
\end{itemize}
\begin{itemize}
\item {Proveniência:(De \textunderscore chaviano\textunderscore )}
\end{itemize}
Espécie de linguiça.
\section{Chaviano}
\begin{itemize}
\item {Grp. gram.:adj.}
\end{itemize}
\begin{itemize}
\item {Utilização:P. us.}
\end{itemize}
\begin{itemize}
\item {Grp. gram.:M.  e  adj.}
\end{itemize}
\begin{itemize}
\item {Utilização:Prov.}
\end{itemize}
\begin{itemize}
\item {Utilização:trasm.}
\end{itemize}
Relativo á villa de Chaves.
Diz-se de uma espécie de chouriço, feito de gorduras e carnes ensanguentadas, com mistura de sêmeas ou de pão ralado.
\section{Chavo}
\begin{itemize}
\item {Grp. gram.:m.}
\end{itemize}
Insignificância monetária.
Pouco valor:«\textunderscore não ter de meu nem sequer dois chavos\textunderscore ». Castilho, \textunderscore Avarento\textunderscore , act. II, sc. 7.
(Cast. \textunderscore chavo\textunderscore  = \textunderscore ochavo\textunderscore )
\section{Chàzeiro}
\begin{itemize}
\item {Grp. gram.:m.}
\end{itemize}
(V.cheda)
\section{Chàzeiro}
\begin{itemize}
\item {Grp. gram.:adj.}
\end{itemize}
O mesmo que \textunderscore chàzista\textunderscore .
\section{Chàzista}
\begin{itemize}
\item {Grp. gram.:adj.}
\end{itemize}
Diz-se de quem gosta muito de chá.
\section{Ché}
\begin{itemize}
\item {Grp. gram.:m.}
\end{itemize}
Lyra chinesa, de muitas cordas. Cf. Castilho, \textunderscore Fastos\textunderscore , III, 207.
\section{Chebê}
\begin{itemize}
\item {Grp. gram.:m.}
\end{itemize}
\begin{itemize}
\item {Utilização:Bras}
\end{itemize}
Toicinho salgado.
\section{Checha}
\begin{itemize}
\item {Grp. gram.:f.}
\end{itemize}
\begin{itemize}
\item {Utilização:Prov.}
\end{itemize}
\begin{itemize}
\item {Utilização:trasm.}
\end{itemize}
\begin{itemize}
\item {Utilização:fam.}
\end{itemize}
Trela, parlenga.
\section{Chèché}
\begin{itemize}
\item {Grp. gram.:m.}
\end{itemize}
\begin{itemize}
\item {Utilização:Prov.}
\end{itemize}
\begin{itemize}
\item {Utilização:trasm.}
\end{itemize}
Bocadinho de qualquer coisa.
O mesmo que \textunderscore xèxé\textunderscore .
\section{Cheche}
\begin{itemize}
\item {Grp. gram.:m.}
\end{itemize}
\begin{itemize}
\item {Utilização:T. de Lamego}
\end{itemize}
Pequena bofetada.
\section{Chèchéu}
\begin{itemize}
\item {Grp. gram.:m.}
\end{itemize}
Nome que, em Pernambuco, se dá ao japim.
\section{Cheda}
\begin{itemize}
\item {fónica:chê}
\end{itemize}
\begin{itemize}
\item {Grp. gram.:f.}
\end{itemize}
\begin{itemize}
\item {Utilização:Prov.}
\end{itemize}
\begin{itemize}
\item {Utilização:minh.}
\end{itemize}
Cada uma das pranchas lateraes do leito do carro, nas quaes se encaixam os fueiros.
Plataforma do carro de lavoira.
\section{Chede}
\begin{itemize}
\item {fónica:chê}
\end{itemize}
\begin{itemize}
\item {Grp. gram.:m.}
\end{itemize}
\begin{itemize}
\item {Utilização:Prov.}
\end{itemize}
\begin{itemize}
\item {Utilização:trasm.}
\end{itemize}
Nome de uma avezinha, o mesmo que \textunderscore cheide\textunderscore .
\section{Chedeiro}
\begin{itemize}
\item {Grp. gram.:m.}
\end{itemize}
\begin{itemize}
\item {Proveniência:(De \textunderscore cheda\textunderscore )}
\end{itemize}
Leito do carro de bois.
\section{Chefado}
\begin{itemize}
\item {Grp. gram.:m.}
\end{itemize}
Dignidade de chefe.
\section{Chefatura}
\begin{itemize}
\item {Grp. gram.:f.}
\end{itemize}
(V.chefado)
\section{Chefe}
\begin{itemize}
\item {Grp. gram.:m.}
\end{itemize}
\begin{itemize}
\item {Utilização:Heráld.}
\end{itemize}
\begin{itemize}
\item {Proveniência:(Fr. \textunderscore chef\textunderscore )}
\end{itemize}
Indivíduo, que, entre outros, é cabeça ou o principal.
Aquelle que commanda.
Aquelle que governa.
Capitão; caudilho.
Cimo ou parte superior interna do escudo.
\section{Chefia}
\begin{itemize}
\item {Grp. gram.:f.}
\end{itemize}
O mesmo que \textunderscore chefado\textunderscore .
\section{Chefre}
\begin{itemize}
\item {Grp. gram.:m.}
\end{itemize}
\begin{itemize}
\item {Utilização:pop.}
\end{itemize}
\begin{itemize}
\item {Utilização:Ant.}
\end{itemize}
O mesmo que \textunderscore chefe\textunderscore .
\section{Chega}
\begin{itemize}
\item {fónica:chê}
\end{itemize}
\begin{itemize}
\item {Grp. gram.:f.}
\end{itemize}
\begin{itemize}
\item {Utilização:Ant.}
\end{itemize}
\begin{itemize}
\item {Utilização:Fam.}
\end{itemize}
\begin{itemize}
\item {Proveniência:(De \textunderscore chegar\textunderscore )}
\end{itemize}
Citação para juízo, por causa de dívidas.
Censura, crítica.
\section{Chegada}
\begin{itemize}
\item {Grp. gram.:f.}
\end{itemize}
Acto de chegar.
\section{Chegadeira}
\begin{itemize}
\item {Grp. gram.:f.}
\end{itemize}
\begin{itemize}
\item {Proveniência:(De \textunderscore chegar\textunderscore )}
\end{itemize}
Utensílio de ferreiro, para chegar carvão á forja.
\section{Chegadiço}
\begin{itemize}
\item {Grp. gram.:adj.}
\end{itemize}
\begin{itemize}
\item {Utilização:Des.}
\end{itemize}
\begin{itemize}
\item {Proveniência:(De \textunderscore chegar\textunderscore )}
\end{itemize}
O mesmo que \textunderscore adventicio\textunderscore .
O mesmo que \textunderscore metediço\textunderscore .
\section{Chegadinha}
\begin{itemize}
\item {Grp. gram.:f.}
\end{itemize}
\begin{itemize}
\item {Utilização:Gír.}
\end{itemize}
\begin{itemize}
\item {Proveniência:(De \textunderscore chegar\textunderscore )}
\end{itemize}
Bofetada.
\section{Chegador}
\begin{itemize}
\item {Grp. gram.:m.}
\end{itemize}
\begin{itemize}
\item {Utilização:Ant.}
\end{itemize}
Aquelle que chega.
Aquelle que mete lenha ou carvão em fornalhas.
Espécie de feitor, que fazia comparecer em certo dia os foreiros remissos e os que impediam o recebimento dos foros e rendas de seus amos.
\section{Chegamento}
\begin{itemize}
\item {Grp. gram.:m.}
\end{itemize}
\begin{itemize}
\item {Utilização:Ant.}
\end{itemize}
\begin{itemize}
\item {Proveniência:(De \textunderscore chegar\textunderscore )}
\end{itemize}
Chegada.
Citação judicial.
\section{Chegança}
\begin{itemize}
\item {Grp. gram.:f.}
\end{itemize}
\begin{itemize}
\item {Utilização:Ant.}
\end{itemize}
Chegamento.
Citação official.
Dança lasciva do séc. XVIII.
\section{Cheganço}
\begin{itemize}
\item {Grp. gram.:m.}
\end{itemize}
\begin{itemize}
\item {Utilização:Chul.}
\end{itemize}
\begin{itemize}
\item {Utilização:T. de bilharista}
\end{itemize}
Censura, reprehensão, chega.
Tacada, que obriga a bola a recuar ou a formar ângulo com a posição do taco.
\section{Chegar}
\begin{itemize}
\item {Grp. gram.:v. t.}
\end{itemize}
\begin{itemize}
\item {Utilização:Pop.}
\end{itemize}
\begin{itemize}
\item {Grp. gram.:V. i.}
\end{itemize}
\begin{itemize}
\item {Proveniência:(Do lat. \textunderscore plicare\textunderscore )}
\end{itemize}
Aproximar, mover para perto.
Levar (uma fêmea) á padreação.
Aproximar-se.
Vir.
Sêr bastante, sufficiente: \textunderscore cinco tostões chegam para um almôço\textunderscore .
Raiar, orçar.
Bater: \textunderscore o pai chegou-lhe\textunderscore .
Elevar-se: \textunderscore chegou a ministro\textunderscore .
\section{Chego}
\begin{itemize}
\item {Grp. gram.:m.}
\end{itemize}
\begin{itemize}
\item {Proveniência:(T. as.)}
\end{itemize}
Certo pêso ou avaliação de pérolas.
\section{Cheia}
\begin{itemize}
\item {Grp. gram.:f.}
\end{itemize}
\begin{itemize}
\item {Utilização:Fig.}
\end{itemize}
\begin{itemize}
\item {Proveniência:(De \textunderscore cheio\textunderscore )}
\end{itemize}
Enchente fluvial.
Inundação.
Porção, quantidade enorme.
\section{Cheide}
\begin{itemize}
\item {Grp. gram.:m.}
\end{itemize}
\begin{itemize}
\item {Utilização:Prov.}
\end{itemize}
\begin{itemize}
\item {Utilização:trasm.}
\end{itemize}
Espécie de tutinegra.
\section{Cheina}
\begin{itemize}
\item {Grp. gram.:adj. f.}
\end{itemize}
\begin{itemize}
\item {Utilização:Prov.}
\end{itemize}
\begin{itemize}
\item {Utilização:trasm.}
\end{itemize}
O mesmo que \textunderscore chaira\textunderscore .
\section{Cheio}
\begin{itemize}
\item {Grp. gram.:adj.}
\end{itemize}
\begin{itemize}
\item {Grp. gram.:M.}
\end{itemize}
\begin{itemize}
\item {Utilização:Mús.}
\end{itemize}
\begin{itemize}
\item {Utilização:Mús.}
\end{itemize}
\begin{itemize}
\item {Utilização:Carp.}
\end{itemize}
\begin{itemize}
\item {Grp. gram.:Loc. adv.}
\end{itemize}
\begin{itemize}
\item {Utilização:Náut.}
\end{itemize}
\begin{itemize}
\item {Proveniência:(Lat. \textunderscore plenus\textunderscore )}
\end{itemize}
Que encerra quanto póde encerrar: \textunderscore um copo cheio\textunderscore .
Massiço, compacto.
Que tem grande porção de coisas: \textunderscore uma casa cheia\textunderscore .
Rico.
Completo.
Feliz.
Atarefado.
Nutrido; gordo.
Amplo.
Aquillo que está inteiramente cheio.
Reunião de todos os registos no órgão ou no harmónio.
Conjunto de todas as vozes de um côro ou de todos os instrumentos da orchestra.
Almofada ou qualquer parte saliente de porta ou janela.
\textunderscore Em cheio\textunderscore , de fronte, de chapa; plenamente: \textunderscore o sol bate em cheio na casa\textunderscore .
\textunderscore Meter em cheio\textunderscore , arribar.
\section{Cheira}
\begin{itemize}
\item {Grp. gram.:m., f.  e  adj.}
\end{itemize}
\begin{itemize}
\item {Utilização:Gír.}
\end{itemize}
\begin{itemize}
\item {Grp. gram.:F.}
\end{itemize}
\begin{itemize}
\item {Proveniência:(De \textunderscore cheirar\textunderscore )}
\end{itemize}
Pessôa metediça.
Pólvora.
\section{Cheiradeira}
\begin{itemize}
\item {Grp. gram.:f.}
\end{itemize}
\begin{itemize}
\item {Utilização:Des.}
\end{itemize}
\begin{itemize}
\item {Proveniência:(De \textunderscore cheirar\textunderscore )}
\end{itemize}
Caixa, para rapé ou tabaco, com orifício.
\section{Cheirador}
\begin{itemize}
\item {Grp. gram.:m.}
\end{itemize}
Aquelle que cheira.
\section{Cheira-fraldas}
\begin{itemize}
\item {Grp. gram.:m.}
\end{itemize}
\begin{itemize}
\item {Utilização:Prov.}
\end{itemize}
\begin{itemize}
\item {Utilização:alent.}
\end{itemize}
O mesmo que \textunderscore maricas\textunderscore .
\section{Cheirante}
\begin{itemize}
\item {Grp. gram.:adj.}
\end{itemize}
O mesmo que \textunderscore cheiroso\textunderscore .
\section{Cheirar}
\begin{itemize}
\item {Grp. gram.:v. t.}
\end{itemize}
\begin{itemize}
\item {Grp. gram.:V. i.}
\end{itemize}
\begin{itemize}
\item {Utilização:Pop.}
\end{itemize}
\begin{itemize}
\item {Proveniência:(Do lat. \textunderscore fragrare\textunderscore )}
\end{itemize}
Applicar o olfacto a.
Introduzir no nariz (rapé, cânfora, etc.).
Indagar; pesquizar.
Exhalar cheiro.
Têr semelhança.
Agradar: \textunderscore isso não me cheira\textunderscore .
\section{Cheireta}
\begin{itemize}
\item {fónica:cheirê}
\end{itemize}
\begin{itemize}
\item {Grp. gram.:m.}
\end{itemize}
\begin{itemize}
\item {Utilização:Bras. de Minas}
\end{itemize}
Indivíduo metediço, intrometido.
\section{Cheirete}
\begin{itemize}
\item {fónica:cheirê}
\end{itemize}
\begin{itemize}
\item {Grp. gram.:m.}
\end{itemize}
\begin{itemize}
\item {Utilização:Fam.}
\end{itemize}
Mau cheiro, fedor.
\section{Cheiro}
\begin{itemize}
\item {Grp. gram.:m.}
\end{itemize}
\begin{itemize}
\item {Utilização:Fig.}
\end{itemize}
\begin{itemize}
\item {Utilização:Fig.}
\end{itemize}
\begin{itemize}
\item {Proveniência:(De \textunderscore cheirar\textunderscore )}
\end{itemize}
Impressão, produzida no sentido do olfacto pelas partículas que se evolam dos corpos.
Aroma, perfume.
Faro, attracção.
Impressão moral.
Substância aromatica, essência.
Salsa, hortelan, ou qualquer outra erva aromática, de applicação culinária.
\textunderscore Cheiro de santidade\textunderscore , cheiro agradável que, segundo a crença popular, se exhala da sepultura de pessôas virtuosas.
Bôa reputação, bôa fama.
\section{Cheiro}
\begin{itemize}
\item {Grp. gram.:m.}
\end{itemize}
\begin{itemize}
\item {Utilização:Bras. do N}
\end{itemize}
(V.chará)
\section{Cheiroga}
\begin{itemize}
\item {Grp. gram.:f.}
\end{itemize}
\begin{itemize}
\item {Utilização:Prov.}
\end{itemize}
\begin{itemize}
\item {Utilização:trasm.}
\end{itemize}
Espécie de urze rasteira.
Provavelmente, o mesmo que \textunderscore queiró\textunderscore .
\section{Cheirópteros}
\begin{itemize}
\item {fónica:quei}
\end{itemize}
\begin{itemize}
\item {Grp. gram.:m. pl.}
\end{itemize}
(V.chirópteros)
\section{Cheirosa}
\begin{itemize}
\item {Grp. gram.:f.}
\end{itemize}
Árvore silvestre do Brasil.
\section{Cheiroso}
\begin{itemize}
\item {Grp. gram.:adj.}
\end{itemize}
Que exhala bom cheiro.
\section{Cheirum}
\begin{itemize}
\item {Grp. gram.:m.}
\end{itemize}
\begin{itemize}
\item {Utilização:Prov.}
\end{itemize}
\begin{itemize}
\item {Utilização:alg.}
\end{itemize}
\begin{itemize}
\item {Proveniência:(De \textunderscore cheiro\textunderscore )}
\end{itemize}
Mau cheiro, fedor.
\section{Chela}
\begin{itemize}
\item {Grp. gram.:f.}
\end{itemize}
\begin{itemize}
\item {Utilização:T. da Áfr. Or. port}
\end{itemize}
Tecido, fazenda.
\section{Cheldra}
\begin{itemize}
\item {Grp. gram.:f.}
\end{itemize}
\begin{itemize}
\item {Utilização:Prov.}
\end{itemize}
O mesmo que \textunderscore papa-amoras\textunderscore .
\section{Cheleira}
\begin{itemize}
\item {Grp. gram.:f.}
\end{itemize}
Lugar, em que se empilham as balas, na bataria de um navio.
\section{Cheleme}
\begin{itemize}
\item {Grp. gram.:m.}
\end{itemize}
\begin{itemize}
\item {Proveniência:(Fr. \textunderscore chelem\textunderscore )}
\end{itemize}
Lance, em que, ao jôgo do \textunderscore whist\textunderscore  ou do bóston, dois parceiros fazem todas as vasas, contra os outros dois.
\section{Chelim}
\begin{itemize}
\item {Grp. gram.:m.}
\end{itemize}
\begin{itemize}
\item {Utilização:Prov.}
\end{itemize}
\begin{itemize}
\item {Utilização:trasm.}
\end{itemize}
A pedra maior, no jôgo das nécaras. (Colhido em Lagoaça)
\section{Chelique}
\begin{itemize}
\item {Grp. gram.:m.}
\end{itemize}
O mesmo que \textunderscore chilique\textunderscore . Cf. Camillo, \textunderscore Corja\textunderscore , 178.
\section{Chelónios}
\begin{itemize}
\item {fónica:que}
\end{itemize}
\begin{itemize}
\item {Grp. gram.:m. pl.}
\end{itemize}
\begin{itemize}
\item {Proveniência:(Do gr. \textunderscore khelone\textunderscore )}
\end{itemize}
Ordem da classe dos reptis, que têm por typo a tartaruga.
\section{Chelonita}
\begin{itemize}
\item {fónica:que}
\end{itemize}
\begin{itemize}
\item {Grp. gram.:f.}
\end{itemize}
\begin{itemize}
\item {Proveniência:(Do gr. \textunderscore khelone\textunderscore )}
\end{itemize}
Tartaruga petrificada.
\section{Chelonographia}
\begin{itemize}
\item {fónica:que}
\end{itemize}
\begin{itemize}
\item {Grp. gram.:f.}
\end{itemize}
\begin{itemize}
\item {Proveniência:(Do gr. \textunderscore khelone\textunderscore  + \textunderscore graphein\textunderscore )}
\end{itemize}
Descripção das tartarugas.
\section{Chelonógrapho}
\begin{itemize}
\item {fónica:que}
\end{itemize}
\begin{itemize}
\item {Grp. gram.:m.}
\end{itemize}
Naturalista, que se occupa especialmente das tartarugas.
(Cp. \textunderscore chelonographia\textunderscore )
\section{Chelonóphago}
\begin{itemize}
\item {fónica:que}
\end{itemize}
\begin{itemize}
\item {Grp. gram.:adj.}
\end{itemize}
\begin{itemize}
\item {Proveniência:(Do gr. \textunderscore khelone\textunderscore  + \textunderscore phagein\textunderscore )}
\end{itemize}
Que se alimenta de tartarugas; que come tartarugas.
\section{Chelpa}
\begin{itemize}
\item {Grp. gram.:f.}
\end{itemize}
\begin{itemize}
\item {Utilização:Gír.}
\end{itemize}
Dinheiro.
\section{Chelro}
\begin{itemize}
\item {Grp. gram.:m.}
\end{itemize}
\begin{itemize}
\item {Utilização:Gír.}
\end{itemize}
As galés.
\section{Chelydro}
\begin{itemize}
\item {fónica:que}
\end{itemize}
\begin{itemize}
\item {Grp. gram.:m.}
\end{itemize}
\begin{itemize}
\item {Proveniência:(Lat. \textunderscore chelidrus\textunderscore )}
\end{itemize}
Nome de uma serpente amphíbia e venenosa. Cf. Filinto, VI, 235.
\section{Chemela}
\begin{itemize}
\item {Grp. gram.:f.}
\end{itemize}
\begin{itemize}
\item {Utilização:Prov.}
\end{itemize}
Travesseirinha da cama.
\section{Chemose}
\begin{itemize}
\item {fónica:que}
\end{itemize}
\begin{itemize}
\item {Grp. gram.:f.}
\end{itemize}
\begin{itemize}
\item {Utilização:Med.}
\end{itemize}
\begin{itemize}
\item {Proveniência:(Gr. \textunderscore khemosis\textunderscore )}
\end{itemize}
Espécie de conjuntivite.
\section{Chena}
\begin{itemize}
\item {Grp. gram.:f.}
\end{itemize}
\begin{itemize}
\item {Utilização:Prov.}
\end{itemize}
\begin{itemize}
\item {Utilização:Gír.}
\end{itemize}
\begin{itemize}
\item {Proveniência:(Do fr. \textunderscore chaîne\textunderscore )}
\end{itemize}
Cadeia para malfeitores.
Calaboiço.
\section{Chenchico}
\begin{itemize}
\item {Grp. gram.:m.}
\end{itemize}
O mesmo que \textunderscore chenchicogim\textunderscore . Cf. \textunderscore Peregrinação\textunderscore , CCXXIII.
\section{Chenchicogim}
\begin{itemize}
\item {Grp. gram.:m.}
\end{itemize}
Nome, que na China se dava a qualquer português. Cf. \textunderscore Peregrinação\textunderscore , CXXXV.
\section{Chenita}
\begin{itemize}
\item {Grp. gram.:f.}
\end{itemize}
\begin{itemize}
\item {Utilização:Prov.}
\end{itemize}
Pequena porção de vinho, que se bebe de um trago.
\section{Chenopodiáceas}
\begin{itemize}
\item {fónica:que}
\end{itemize}
\begin{itemize}
\item {Grp. gram.:f. pl.}
\end{itemize}
Família de plantas, que têm por typo o \textunderscore chenopódio\textunderscore .
\section{Chenopódio}
\begin{itemize}
\item {fónica:que}
\end{itemize}
\begin{itemize}
\item {Grp. gram.:m.}
\end{itemize}
\begin{itemize}
\item {Proveniência:(Do gr. \textunderscore khen\textunderscore  + \textunderscore pous\textunderscore , \textunderscore podos\textunderscore )}
\end{itemize}
O mesmo que \textunderscore anserina\textunderscore .
\section{Cheque}
\begin{itemize}
\item {Grp. gram.:m.}
\end{itemize}
\begin{itemize}
\item {Proveniência:(Ingl. \textunderscore check\textunderscore )}
\end{itemize}
Ordem de pagamento, para êste sêr feito ao portador.
No jôgo do xadrez, posição em que uma peça inutiliza outra das principaes.
Successo parlamentar, que envolve perigo para o ministério.
Perigo; contratempo.
\section{Cheque}
\begin{itemize}
\item {Grp. gram.:m.}
\end{itemize}
Dialecto do ramo esclavónico.
O mesmo que \textunderscore bohêmio\textunderscore ^1.
\section{Cheramela}
\begin{itemize}
\item {Grp. gram.:f.}
\end{itemize}
Espécie de mirobálano. Cf. G. Orta, \textunderscore Coll.\textunderscore , 37.
\section{Cherelo}
\begin{itemize}
\item {fónica:cherê}
\end{itemize}
\begin{itemize}
\item {Grp. gram.:m.}
\end{itemize}
\begin{itemize}
\item {Utilização:Prov.}
\end{itemize}
\begin{itemize}
\item {Utilização:minh.}
\end{itemize}
Peixe pequeno, talvez o mesmo que \textunderscore carapau\textunderscore .
\section{Cheremisso}
\begin{itemize}
\item {Grp. gram.:m.}
\end{itemize}
\begin{itemize}
\item {Grp. gram.:Pl.}
\end{itemize}
Língua uralo-altaica.
Uma das tríbos da Sibéria.
\section{Cherico}
\begin{itemize}
\item {Grp. gram.:m.}
\end{itemize}
Espécie de canário de Angola.
\section{Cherimólia}
\begin{itemize}
\item {Grp. gram.:f.}
\end{itemize}
Planta anonácea, (\textunderscore anona cherimolia\textunderscore , Lam.).
Fruto dessa planta, o melhor dos anonáceos.
\section{Cheringalho}
\begin{itemize}
\item {Grp. gram.:m.}
\end{itemize}
\begin{itemize}
\item {Utilização:Prov.}
\end{itemize}
\begin{itemize}
\item {Utilização:trasm.}
\end{itemize}
Maltrapilho; bigorrilhas; troca-tintas.
\section{Cherinola}
\begin{itemize}
\item {Grp. gram.:f.}
\end{itemize}
Léria.
Palavreado:«\textunderscore que os exorcismos eram cherinolas.\textunderscore »Camillo, \textunderscore Brasileira\textunderscore , 365.
\section{Cheripá}
\begin{itemize}
\item {Grp. gram.:m.}
\end{itemize}
O mesmo que \textunderscore chiripá\textunderscore .
\section{Cherivia}
\begin{itemize}
\item {Grp. gram.:f.}
\end{itemize}
O mesmo que \textunderscore cherovia\textunderscore .
\section{Cherna}
\begin{itemize}
\item {Grp. gram.:f.}
\end{itemize}
Peixe, muito semelhante ao \textunderscore cherne\textunderscore .
\section{Cherne}
\begin{itemize}
\item {Grp. gram.:m.}
\end{itemize}
Peixe de água salgada.
\section{Chernite}
\begin{itemize}
\item {fónica:quer}
\end{itemize}
\begin{itemize}
\item {Grp. gram.:f.}
\end{itemize}
\begin{itemize}
\item {Proveniência:(Gr. \textunderscore khernites\textunderscore )}
\end{itemize}
Pedra branca, alabastro fino.
\section{Cherovia}
\begin{itemize}
\item {Grp. gram.:f.}
\end{itemize}
\begin{itemize}
\item {Utilização:Prov.}
\end{itemize}
\begin{itemize}
\item {Utilização:beir.}
\end{itemize}
\begin{itemize}
\item {Proveniência:(Do ár. \textunderscore carivia\textunderscore )}
\end{itemize}
Planta hortense, (\textunderscore sium sisarum\textunderscore , Lin.), cujo bolbo branco, partido em tiras, se come depois de frito.
\section{Cherrafusca}
\begin{itemize}
\item {Grp. gram.:f.}
\end{itemize}
(V.sarrafusca)
\section{Chersoneso}
\begin{itemize}
\item {fónica:quer}
\end{itemize}
\begin{itemize}
\item {Grp. gram.:m.}
\end{itemize}
\begin{itemize}
\item {Utilização:Ant.}
\end{itemize}
\begin{itemize}
\item {Proveniência:(Do gr. \textunderscore khersos\textunderscore , terra, e \textunderscore nesos\textunderscore , ilha)}
\end{itemize}
O mesmo que \textunderscore península\textunderscore .
\section{Cherub}
\begin{itemize}
\item {fónica:que}
\end{itemize}
\begin{itemize}
\item {Grp. gram.:m.}
\end{itemize}
\begin{itemize}
\item {Proveniência:(Lat. \textunderscore cherub\textunderscore )}
\end{itemize}
O mesmo que \textunderscore cherubim\textunderscore . Cf. Filinto, XVI, 317.
\section{Cherúbico}
\begin{itemize}
\item {fónica:que}
\end{itemize}
\begin{itemize}
\item {Grp. gram.:adj.}
\end{itemize}
O mesmo que \textunderscore cherubínico\textunderscore .
\section{Cherubim}
\begin{itemize}
\item {fónica:que}
\end{itemize}
\begin{itemize}
\item {Grp. gram.:m.}
\end{itemize}
\begin{itemize}
\item {Proveniência:(Lat. eccl. \textunderscore cherubim\textunderscore )}
\end{itemize}
Anjo da primeira jerarchia, segundo a Theologia.
Anjo.
Pintura ou escultura de uma cabeça de criança com asas, representando um cherubim.
\section{Cherubínico}
\begin{itemize}
\item {fónica:que}
\end{itemize}
\begin{itemize}
\item {Grp. gram.:adj.}
\end{itemize}
Relativo a cherubim.
\section{Cherumba}
\begin{itemize}
\item {Grp. gram.:f.}
\end{itemize}
\begin{itemize}
\item {Utilização:Prov.}
\end{itemize}
\begin{itemize}
\item {Utilização:trasm.}
\end{itemize}
O jôgo do batuque.
\section{Cherva}
\begin{itemize}
\item {Grp. gram.:f.}
\end{itemize}
Fibra têxtil, própria para tapêtes. Cf. \textunderscore Inquér. Ind.\textunderscore , T. II, l. 2.^o, 310.
(Cast. \textunderscore cherva\textunderscore )
\section{Chesmininés!}
\begin{itemize}
\item {Grp. gram.:interj.}
\end{itemize}
\begin{itemize}
\item {Utilização:Des.}
\end{itemize}
\begin{itemize}
\item {Grp. gram.:M. pl.}
\end{itemize}
Atinei; dei no vinte; já sei a razão.
Atavios, adornos:«\textunderscore são poéticos chesmininés.\textunderscore »Filinto, IV, 73.
\section{Cheta}
\begin{itemize}
\item {fónica:chê}
\end{itemize}
\begin{itemize}
\item {Grp. gram.:f.}
\end{itemize}
\begin{itemize}
\item {Utilização:Gír.}
\end{itemize}
Pequena moéda de cobre.
Pouco dinheiro.
Vintem.
\section{Chetá}
\begin{itemize}
\item {Grp. gram.:m.}
\end{itemize}
\begin{itemize}
\item {Utilização:Prov.}
\end{itemize}
\begin{itemize}
\item {Utilização:trasm.}
\end{itemize}
\begin{itemize}
\item {Grp. gram.:Interj.}
\end{itemize}
Qualquer bêsta, (falando-se della ou mostrando-a a crianças).
Voz, para mandar parar as bêstas.
\section{Chetodonte}
\begin{itemize}
\item {fónica:que}
\end{itemize}
\begin{itemize}
\item {Grp. gram.:m.}
\end{itemize}
\begin{itemize}
\item {Proveniência:(Do gr. \textunderscore khaite\textunderscore  + \textunderscore odous\textunderscore , \textunderscore odontos\textunderscore )}
\end{itemize}
Gênero de peixes, que têm os dentes muito finos.
\section{Chetópode}
\begin{itemize}
\item {fónica:que}
\end{itemize}
\begin{itemize}
\item {Grp. gram.:m.  e  adj.}
\end{itemize}
\begin{itemize}
\item {Proveniência:(Do gr. \textunderscore khaite\textunderscore  + \textunderscore pous\textunderscore , \textunderscore podos\textunderscore )}
\end{itemize}
Animal, que tem sedas em lugar de patas.
\section{Chetóptero}
\begin{itemize}
\item {fónica:que}
\end{itemize}
\begin{itemize}
\item {Grp. gram.:m.}
\end{itemize}
\begin{itemize}
\item {Proveniência:(Do gr. \textunderscore khaite\textunderscore  + \textunderscore pteron\textunderscore )}
\end{itemize}
Insecto chetópode nadador, proveniente das Antilhas.
\section{Cheúra}
\begin{itemize}
\item {Grp. gram.:f.}
\end{itemize}
\begin{itemize}
\item {Utilização:Prov.}
\end{itemize}
\begin{itemize}
\item {Utilização:trasm.}
\end{itemize}
\begin{itemize}
\item {Proveniência:(De \textunderscore cheio\textunderscore )}
\end{itemize}
Estado do que é cheio; abundância, fartura.
\section{Cheveca}
\begin{itemize}
\item {Grp. gram.:f.}
\end{itemize}
Operária, que, nas fábricas de caixas de papelão, trabalha pelo systema cheveco.
\section{Cheveco}
\begin{itemize}
\item {Grp. gram.:adj.}
\end{itemize}
\begin{itemize}
\item {Proveniência:(De \textunderscore Schweickardt\textunderscore , n. p. do alemão que inventou o systema)}
\end{itemize}
Diz-se de um systema de armar o papel e o papelão para fazer caixinhas.
\section{Cheviote}
\begin{itemize}
\item {Grp. gram.:m.}
\end{itemize}
\begin{itemize}
\item {Proveniência:(De \textunderscore Cheviot\textunderscore , n. p.)}
\end{itemize}
Tecido inglês de lan.
\section{Cheviotina}
\begin{itemize}
\item {Grp. gram.:f.}
\end{itemize}
Espécie de pano nacional, semelhante ao cheviote.
\section{Chi}
\begin{itemize}
\item {Grp. gram.:m.}
\end{itemize}
\begin{itemize}
\item {Utilização:Infant.}
\end{itemize}
Abraço.
\section{Chiada}
\begin{itemize}
\item {Grp. gram.:f.}
\end{itemize}
Acto de chiar.
Conjunto de vozes agudas e desagradáveis.
\section{Chiadeira}
\begin{itemize}
\item {Grp. gram.:f.}
\end{itemize}
O mesmo que \textunderscore chiada\textunderscore .
\section{Chiado}
\begin{itemize}
\item {Grp. gram.:m.}
\end{itemize}
(V.chiada)
\section{Chiado}
\begin{itemize}
\item {Grp. gram.:adj.}
\end{itemize}
\begin{itemize}
\item {Proveniência:(T. as.)}
\end{itemize}
Em que há malícia.
\section{Chiador}
\begin{itemize}
\item {Grp. gram.:m.  e  adj.}
\end{itemize}
O que chia.
\section{Chiadura}
\begin{itemize}
\item {Grp. gram.:f.}
\end{itemize}
O mesmo que \textunderscore chiada\textunderscore .
\section{Chião}
\begin{itemize}
\item {Grp. gram.:m.}
\end{itemize}
\begin{itemize}
\item {Utilização:Prov.}
\end{itemize}
\begin{itemize}
\item {Utilização:minh.}
\end{itemize}
\begin{itemize}
\item {Utilização:Ext.}
\end{itemize}
Boneca ou boneco.
Criança de mama.
\section{Chiar}
\begin{itemize}
\item {Grp. gram.:v. i.}
\end{itemize}
Fazêr chio.
\section{Chiasco}
\begin{itemize}
\item {Grp. gram.:m.}
\end{itemize}
\begin{itemize}
\item {Utilização:T. de Chaves}
\end{itemize}
O mesmo que \textunderscore rexio\textunderscore .
\section{Chiasco}
\begin{itemize}
\item {Grp. gram.:m.}
\end{itemize}
\begin{itemize}
\item {Utilização:Prov.}
\end{itemize}
\begin{itemize}
\item {Utilização:trasm.}
\end{itemize}
Vento frio e cortante.
\section{Chiasma}
\begin{itemize}
\item {fónica:qui}
\end{itemize}
\begin{itemize}
\item {Grp. gram.:m.}
\end{itemize}
\begin{itemize}
\item {Utilização:Anat.}
\end{itemize}
\begin{itemize}
\item {Proveniência:(Gr. \textunderscore khiasma\textunderscore )}
\end{itemize}
Cruzamento de nervos ópticos sôbre o esphenoide.
\section{Chiastolífero}
\begin{itemize}
\item {fónica:qui}
\end{itemize}
\begin{itemize}
\item {Grp. gram.:adj.}
\end{itemize}
\begin{itemize}
\item {Proveniência:(Do gr. \textunderscore khiastos\textunderscore  + lat. \textunderscore ferre\textunderscore )}
\end{itemize}
Em que há chiastólithos, (falando-se de terrenos ou piçarras).
\section{Chiastólitho}
\begin{itemize}
\item {fónica:qui}
\end{itemize}
\begin{itemize}
\item {Grp. gram.:m.}
\end{itemize}
\begin{itemize}
\item {Proveniência:(Do gr. \textunderscore khiastos\textunderscore  + \textunderscore lithos\textunderscore )}
\end{itemize}
Variedade de andaluzite, que se apresenta em prismas rectangulares e quási quadrados.
\section{Chiastro}
\begin{itemize}
\item {fónica:qui}
\end{itemize}
\begin{itemize}
\item {Grp. gram.:m.}
\end{itemize}
\begin{itemize}
\item {Proveniência:(Do gr. \textunderscore khiazein\textunderscore )}
\end{itemize}
Ligadura em fórma de X, que se usava nas fracturas das pernas.
\section{Chiata}
\begin{itemize}
\item {Grp. gram.:f.}
\end{itemize}
\begin{itemize}
\item {Utilização:Bras. do N}
\end{itemize}
Pilhéria; gracejo.
\section{Chiatar}
\begin{itemize}
\item {Grp. gram.:v. i.}
\end{itemize}
\begin{itemize}
\item {Utilização:Bras}
\end{itemize}
Dizer chiatas.
\section{Chiba}
\begin{itemize}
\item {Grp. gram.:f.}
\end{itemize}
O mesmo que \textunderscore cabra\textunderscore ^1.
\section{Chiba}
\begin{itemize}
\item {Grp. gram.:f.}
\end{itemize}
\begin{itemize}
\item {Utilização:Prov.}
\end{itemize}
\begin{itemize}
\item {Utilização:alent.}
\end{itemize}
Empola, que se fórma, em mãos não callejadas, pelo attrito de corpo duro, como o cabo de uma ferramenta, cordas de um instrumento, etc.
O mesmo que \textunderscore indigestão\textunderscore .
\section{Chiba}
\begin{itemize}
\item {Grp. gram.:f.}
\end{itemize}
(Corr. pop. de \textunderscore gibba\textunderscore )
\section{Chiba}
\begin{itemize}
\item {Grp. gram.:m.}
\end{itemize}
\begin{itemize}
\item {Utilização:Bras}
\end{itemize}
Espécie de batuque.
\section{Chiba}
\begin{itemize}
\item {Grp. gram.:f.}
\end{itemize}
\begin{itemize}
\item {Utilização:Prov.}
\end{itemize}
\begin{itemize}
\item {Utilização:trasm.}
\end{itemize}
O mesmo que \textunderscore chibança\textunderscore .
\section{Chibaça}
\begin{itemize}
\item {Grp. gram.:f.}
\end{itemize}
\begin{itemize}
\item {Utilização:Prov.}
\end{itemize}
\begin{itemize}
\item {Utilização:trasm.}
\end{itemize}
O mesmo que \textunderscore chibança\textunderscore .
\section{Chibaço}
\begin{itemize}
\item {Grp. gram.:m.}
\end{itemize}
Espécie de cabaça, geralmente ornada de desenhos e lavores, e com a qual os indígenas do sul de Moçambique cobrem, por decência, a glande do pênis.
\section{Chibalé}
\begin{itemize}
\item {Grp. gram.:m.}
\end{itemize}
\begin{itemize}
\item {Utilização:Gír.}
\end{itemize}
Adversário.
(Or. ind.)
\section{Chibança}
\begin{itemize}
\item {Grp. gram.:f.}
\end{itemize}
O mesmo que \textunderscore chibantice\textunderscore .
\section{Chibantaria}
\begin{itemize}
\item {Grp. gram.:f.}
\end{itemize}
O mesmo que \textunderscore chibantice\textunderscore .
\section{Chibante}
\begin{itemize}
\item {Grp. gram.:m.  e  adj.}
\end{itemize}
\begin{itemize}
\item {Proveniência:(De \textunderscore chibar\textunderscore )}
\end{itemize}
Valentão, orgulhoso; fanfarrão.
Janota, casquilho.
\section{Chibantear}
\begin{itemize}
\item {Grp. gram.:v. i.}
\end{itemize}
Mostrar-se chibante.
\section{Chibantesco}
\begin{itemize}
\item {Grp. gram.:adj.}
\end{itemize}
Em que há orgulho ou fanfarronada.
\section{Chibantice}
\begin{itemize}
\item {Grp. gram.:f.}
\end{itemize}
Qualidade do que é chibante.
\section{Chibantismo}
\begin{itemize}
\item {Grp. gram.:m.}
\end{itemize}
(V.chibantice)
\section{Chibanze}
\begin{itemize}
\item {Grp. gram.:m.}
\end{itemize}
Árvore do Congo.
\section{Chibar}
\begin{itemize}
\item {Grp. gram.:v. i.}
\end{itemize}
\begin{itemize}
\item {Proveniência:(De \textunderscore chibo\textunderscore ?)}
\end{itemize}
O mesmo que \textunderscore chibantear\textunderscore .
\section{Chibarás}
\begin{itemize}
\item {Grp. gram.:m. pl.}
\end{itemize}
Indígenas brasileiros das margens do Juruá.
\section{Chibarrada}
\begin{itemize}
\item {Grp. gram.:f.}
\end{itemize}
\begin{itemize}
\item {Proveniência:(De \textunderscore chibarro\textunderscore )}
\end{itemize}
Rebanho caprino.
\section{Chibarreiro}
\begin{itemize}
\item {Grp. gram.:m.}
\end{itemize}
Guarda de chibarros; cabreiro.
\section{Chibarro}
\begin{itemize}
\item {Grp. gram.:m.}
\end{itemize}
\begin{itemize}
\item {Proveniência:(De \textunderscore chibo\textunderscore )}
\end{itemize}
Pequeno bode castrado.
\section{Chibata}
\begin{itemize}
\item {Grp. gram.:f.}
\end{itemize}
Junco; vara delgada e comprida, para fustigar.
\section{Chibatada}
\begin{itemize}
\item {Grp. gram.:f.}
\end{itemize}
Pancada de chibata.
\section{Chibatan}
\begin{itemize}
\item {Grp. gram.:f.}
\end{itemize}
Árvore therebintácea do Brasil.
\section{Chibatar}
\begin{itemize}
\item {Grp. gram.:v. t.}
\end{itemize}
Bater com chibata.
\section{Chibateamento}
\begin{itemize}
\item {Grp. gram.:m.}
\end{itemize}
Acto de \textunderscore chibatear\textunderscore .
\section{Chibatear}
\begin{itemize}
\item {Grp. gram.:v. t.}
\end{itemize}
O mesmo que \textunderscore chibatar\textunderscore .
\section{Chibato}
\begin{itemize}
\item {Grp. gram.:m.}
\end{itemize}
\begin{itemize}
\item {Proveniência:(De \textunderscore chibo\textunderscore )}
\end{itemize}
Pequeno bode, que tem mais de 6 meses e menos de um anno.
\section{Chibato}
\begin{itemize}
\item {Grp. gram.:m.}
\end{itemize}
(?)«\textunderscore Froilão Dias, chibato da Ordem de Malta...\textunderscore »Garrett, \textunderscore Viagens\textunderscore , I, 130.
\section{Chibcha}
\begin{itemize}
\item {Grp. gram.:m.}
\end{itemize}
Idioma dos indígenas da Colúmbia.
\section{Chibé}
\begin{itemize}
\item {Grp. gram.:m.}
\end{itemize}
\begin{itemize}
\item {Utilização:Bras}
\end{itemize}
Bolo de farinha de mandioca.
Bebida refrigerante, feita de água, mel e farinha de mandioca.
\section{Chibé}
\begin{itemize}
\item {Grp. gram.:m.}
\end{itemize}
Ave aquática.
\section{Chibembe}
\begin{itemize}
\item {Grp. gram.:m.}
\end{itemize}
Pequeno peixe africano.
\section{Chibo}
\begin{itemize}
\item {Grp. gram.:m.}
\end{itemize}
\begin{itemize}
\item {Utilização:Gír.}
\end{itemize}
\begin{itemize}
\item {Proveniência:(Do alt. al. \textunderscore zibbe\textunderscore )}
\end{itemize}
O mesmo que \textunderscore cabrito\textunderscore .
Alavanca.
\section{Chibuque}
\begin{itemize}
\item {Grp. gram.:m.}
\end{itemize}
Longo cachimbo oriental.
\section{Chica}
\begin{itemize}
\item {Grp. gram.:f.}
\end{itemize}
Dança dos Negros.
Bebida alcoólica da América do Sul.
\section{Chiça!}
\begin{itemize}
\item {Grp. gram.:interj.}
\end{itemize}
\begin{itemize}
\item {Utilização:chul.}
\end{itemize}
\begin{itemize}
\item {Proveniência:(De \textunderscore chiçar\textunderscore )}
\end{itemize}
(designativa de grande desprêzo)
\section{Chicabequelababa}
\begin{itemize}
\item {Grp. gram.:f.}
\end{itemize}
Ave africana, da ordem das pernaltas.
\section{Chicada}
\begin{itemize}
\item {Grp. gram.:f.}
\end{itemize}
\begin{itemize}
\item {Utilização:Prov.}
\end{itemize}
\begin{itemize}
\item {Utilização:alent.}
\end{itemize}
\begin{itemize}
\item {Proveniência:(Do cast. \textunderscore chico\textunderscore )}
\end{itemize}
Pequeno grupo de ovelhas, com borregos muito novos.
\section{Chicadeiro}
\begin{itemize}
\item {Grp. gram.:m.  e  adj.}
\end{itemize}
Guardador ou pastor de chicada.
\section{Chica-la-fava}
\begin{itemize}
\item {Grp. gram.:f.}
\end{itemize}
Espécie de jôgo popular.
\section{Chicana}
\begin{itemize}
\item {Grp. gram.:f.}
\end{itemize}
\begin{itemize}
\item {Proveniência:(Fr. \textunderscore chicane\textunderscore )}
\end{itemize}
Tramóia, enrêdo, em questões judiciáes.
Ardil; sophisma.
Contestação capciosa.
\section{Chicanar}
\begin{itemize}
\item {Grp. gram.:v. i.}
\end{itemize}
Fazer chicana.
\section{Chicaneiro}
\begin{itemize}
\item {Grp. gram.:m.}
\end{itemize}
Aquelle que é dado a chicanas forenses.
Trapaceiro. Cf. Castilho, \textunderscore Fastos\textunderscore , I, 294.
\section{Chicanga}
\begin{itemize}
\item {Grp. gram.:f.}
\end{itemize}
Árvore do Congo.
\section{Chicanista}
\begin{itemize}
\item {Grp. gram.:m.  e  adj.}
\end{itemize}
\begin{itemize}
\item {Utilização:Bras}
\end{itemize}
O mesmo que \textunderscore chicaneiro\textunderscore .
\section{Chiçar}
\begin{itemize}
\item {Grp. gram.:v. i.}
\end{itemize}
\begin{itemize}
\item {Utilização:Chul.}
\end{itemize}
Bater, dar sova.
Têr cópula carnal (o homem).
\section{Chícara}
\begin{itemize}
\item {Grp. gram.:f.}
\end{itemize}
Taça, vaso pequeno, para tomar chá ou outra infusão.
(Cp. it. \textunderscore chicchera\textunderscore )
\section{Chicarada}
\begin{itemize}
\item {Grp. gram.:f.}
\end{itemize}
Líquido, contido numa chícara.
\section{Chicarola}
\begin{itemize}
\item {Grp. gram.:f.}
\end{itemize}
Espécie de chicória.
\section{Chicha}
\begin{itemize}
\item {Grp. gram.:f.}
\end{itemize}
\begin{itemize}
\item {Utilização:Infant.}
\end{itemize}
\begin{itemize}
\item {Utilização:Escol.}
\end{itemize}
\begin{itemize}
\item {Utilização:Bras}
\end{itemize}
Comida.
Gulodice.
Sardinha.
Commentário, ou traducção interlinear, apontamento, sebenta.
Bebida alcoólica, com mel e água.
\section{Chichá}
\begin{itemize}
\item {Grp. gram.:m.}
\end{itemize}
Planta esterculiácea da região do Amazonas, (\textunderscore sterculea chicha\textunderscore ).
(Cp. \textunderscore chiche\textunderscore )
\section{Chícharo}
\begin{itemize}
\item {Grp. gram.:m.}
\end{itemize}
\begin{itemize}
\item {Proveniência:(Do it. \textunderscore cicero\textunderscore )}
\end{itemize}
Planta leguminosa.
Variedade de feijão, fruto daquella planta.
\section{Chicharro}
\begin{itemize}
\item {Grp. gram.:m.}
\end{itemize}
Carapau grande.
\section{Chiche}
\begin{itemize}
\item {Grp. gram.:m.}
\end{itemize}
Árvore angolense, espécie de estercúlia, (\textunderscore sterculia tomentosa\textunderscore , Guill.).
\section{Chichelada}
\begin{itemize}
\item {Grp. gram.:f.}
\end{itemize}
Pancada com chichelo.
\section{Chicheleiro}
\begin{itemize}
\item {Grp. gram.:adj.}
\end{itemize}
\begin{itemize}
\item {Utilização:ant.}
\end{itemize}
\begin{itemize}
\item {Utilização:Pop.}
\end{itemize}
\begin{itemize}
\item {Proveniência:(De \textunderscore chichelo\textunderscore )}
\end{itemize}
Ridículo, vil, insignificante.
\section{Chichelo}
\begin{itemize}
\item {Grp. gram.:m.}
\end{itemize}
\begin{itemize}
\item {Utilização:Ant.}
\end{itemize}
Sapato velho, que se traz acalcanhado; chinelo.
\section{Chichi}
\begin{itemize}
\item {Grp. gram.:m.}
\end{itemize}
\begin{itemize}
\item {Utilização:Infant.}
\end{itemize}
\begin{itemize}
\item {Proveniência:(T. onom.)}
\end{itemize}
Acto de urinar: \textunderscore fazer chichi\textunderscore .
\section{Chichiar}
\begin{itemize}
\item {Grp. gram.:v. i.}
\end{itemize}
\begin{itemize}
\item {Utilização:Bras}
\end{itemize}
\begin{itemize}
\item {Utilização:Neol.}
\end{itemize}
Chiar muito.
\section{Chichicuilote}
\begin{itemize}
\item {Grp. gram.:m.}
\end{itemize}
Ave aquática, que se alimenta dos mosquitos que cobrem a superfície dos lagos, no valle do México, (\textunderscore gambetta melancolica\textunderscore , Wils.).
\section{Chichimeco}
\begin{itemize}
\item {Grp. gram.:m.}
\end{itemize}
\begin{itemize}
\item {Utilização:Bras}
\end{itemize}
Bisbórria, bigorrilhas.
\section{Chichinha}
\begin{itemize}
\item {Grp. gram.:f.}
\end{itemize}
\begin{itemize}
\item {Utilização:Prov.}
\end{itemize}
\begin{itemize}
\item {Utilização:infant.}
\end{itemize}
\begin{itemize}
\item {Proveniência:(De \textunderscore chicha\textunderscore )}
\end{itemize}
Carne. (Colhido em Turquel)
\section{Chichisbéu}
\begin{itemize}
\item {Grp. gram.:m.}
\end{itemize}
\begin{itemize}
\item {Proveniência:(Do it. \textunderscore cicisbeo\textunderscore )}
\end{itemize}
Aquelle que corteja ou galanteia, com assiduidade importuna, uma senhora.
\section{Chichizinho}
\begin{itemize}
\item {Grp. gram.:m.}
\end{itemize}
\begin{itemize}
\item {Utilização:Prov.}
\end{itemize}
\begin{itemize}
\item {Utilização:trasm.}
\end{itemize}
\begin{itemize}
\item {Proveniência:(De \textunderscore chicho\textunderscore )}
\end{itemize}
Pedacinho de qualquer coisa: \textunderscore um chichizinho de pão\textunderscore .
\section{Chicho}
\begin{itemize}
\item {Grp. gram.:m.}
\end{itemize}
\begin{itemize}
\item {Utilização:Prov.}
\end{itemize}
\begin{itemize}
\item {Utilização:trasm.}
\end{itemize}
Bocadinho de carne, da que está para se ensacar, e que se separa para se assar nas brasas e comer-se logo.
\section{Chichorro}
\begin{itemize}
\item {Grp. gram.:m.}
\end{itemize}
\begin{itemize}
\item {Utilização:Ant.}
\end{itemize}
\begin{itemize}
\item {Utilização:Ant.}
\end{itemize}
\begin{itemize}
\item {Utilização:Prov.}
\end{itemize}
\begin{itemize}
\item {Utilização:beir.}
\end{itemize}
O mesmo que \textunderscore cachorro\textunderscore .
Peça de artilharia.
Escudella de ferro.
\section{Chichorrobiar}
\begin{itemize}
\item {Grp. gram.:v. i.}
\end{itemize}
\begin{itemize}
\item {Utilização:Prov.}
\end{itemize}
\begin{itemize}
\item {Proveniência:(De \textunderscore chichorrobio\textunderscore )}
\end{itemize}
Assobiar.
\section{Chichorrobio}
\begin{itemize}
\item {Grp. gram.:adj.}
\end{itemize}
\begin{itemize}
\item {Grp. gram.:M.}
\end{itemize}
\begin{itemize}
\item {Utilização:Prov.}
\end{itemize}
Que termina em bico.
Assobio.
\section{Chiclopé}
\begin{itemize}
\item {Grp. gram.:m.}
\end{itemize}
Espécie de jôgo popular, o mesmo que \textunderscore chicolapé\textunderscore .
\section{Chico}
\begin{itemize}
\item {Grp. gram.:m.}
\end{itemize}
\begin{itemize}
\item {Utilização:T. de Barcelos}
\end{itemize}
O mesmo que \textunderscore porco\textunderscore .
\section{Chiço}
\begin{itemize}
\item {Grp. gram.:m.}
\end{itemize}
\begin{itemize}
\item {Utilização:Prov.}
\end{itemize}
\begin{itemize}
\item {Utilização:chul.}
\end{itemize}
Rapariga, que aprende a costurar em casa de modistas, ou em estabelecimento de modas. (Colhido no Porto)
\section{Chico-chico}
\begin{itemize}
\item {Grp. gram.:m.}
\end{itemize}
Reptil angolense, (\textunderscore onychocephalus angolensis\textunderscore , Bocage).
\section{Chico-da-ronda}
\begin{itemize}
\item {Grp. gram.:m.}
\end{itemize}
\begin{itemize}
\item {Utilização:Bras}
\end{itemize}
Bailarico, espécie de fandango.
\section{Chicolapé}
\begin{itemize}
\item {Grp. gram.:m.}
\end{itemize}
Jôgo de rapazes, também chamado \textunderscore jôgo do homem\textunderscore .
(Cp. \textunderscore chinclopé\textunderscore )
\section{Chicolatado}
\begin{itemize}
\item {Grp. gram.:adj.}
\end{itemize}
Misturado com chicolate. Cf. Camillo, \textunderscore Volcões\textunderscore , 153.
\section{Chicolate}
\textunderscore m. Pop.\textunderscore  (e der.)
O mesmo que \textunderscore chocolate\textunderscore , etc.
(O \textunderscore i\textunderscore  da fórma pop. proveio de que o \textunderscore ch\textunderscore  é palatal, como o \textunderscore i\textunderscore )
\section{Chico-preto}
\begin{itemize}
\item {Grp. gram.:m.}
\end{itemize}
\begin{itemize}
\item {Utilização:Bras. do N}
\end{itemize}
Pássaro preto, semelhante ao japi.
\section{Chico-puxado}
\begin{itemize}
\item {Grp. gram.:m.}
\end{itemize}
\begin{itemize}
\item {Utilização:Bras}
\end{itemize}
Variedade de baile campestre.
\section{Chi-coração}
\begin{itemize}
\item {Grp. gram.:m.}
\end{itemize}
\begin{itemize}
\item {Utilização:fam.}
\end{itemize}
\begin{itemize}
\item {Utilização:Infant.}
\end{itemize}
Abraço.
\section{Chicoráceas}
\begin{itemize}
\item {Grp. gram.:f. pl.}
\end{itemize}
(V.chicoriáceas)
\section{Chicória}
\begin{itemize}
\item {Grp. gram.:f.}
\end{itemize}
\begin{itemize}
\item {Proveniência:(Do lat. \textunderscore cichorium\textunderscore )}
\end{itemize}
Planta hortense.
\section{Chicória}
\begin{itemize}
\item {Grp. gram.:m.}
\end{itemize}
\begin{itemize}
\item {Utilização:T. de Turquel}
\end{itemize}
Homem impertinente e egoista.
\section{Chicoriáceas}
\begin{itemize}
\item {Grp. gram.:f. pl.}
\end{itemize}
Fam. de plantas que têm por typo a chicória.
\section{Chicorrio}
\begin{itemize}
\item {Grp. gram.:m.}
\end{itemize}
\begin{itemize}
\item {Utilização:Prov.}
\end{itemize}
\begin{itemize}
\item {Utilização:alg.}
\end{itemize}
O mesmo que \textunderscore trigueirão\textunderscore .
\section{Chicoso}
\begin{itemize}
\item {Grp. gram.:m.}
\end{itemize}
\begin{itemize}
\item {Utilização:Prov.}
\end{itemize}
\begin{itemize}
\item {Utilização:trasm.}
\end{itemize}
Como vocativo ou interjeição, é fórma negativa e irónica; \textunderscore Ê, chicoso!\textunderscore  isso, sim! conta com isso!
\section{Chicotada}
\begin{itemize}
\item {Grp. gram.:f.}
\end{itemize}
Pancada de chicote.
\section{Chicotar}
\begin{itemize}
\item {Grp. gram.:v. t.}
\end{itemize}
O mesmo que \textunderscore chicotear\textunderscore .
\section{Chicote}
\begin{itemize}
\item {Grp. gram.:m.}
\end{itemize}
\begin{itemize}
\item {Utilização:Des.}
\end{itemize}
Cordel entrançado, ou correia de coiro, ligada a um cabo de madeira, geralmente para castigar animaes.
Extremidade de cabo náutico.
Trança de cabello, apertada com fita.
Movimento de lacete, rápido e sacudido, da máquina de um combóio.
\section{Chicotear}
\begin{itemize}
\item {Grp. gram.:v. t.}
\end{itemize}
Bater com chicote.
\section{Chicote-queimado}
\begin{itemize}
\item {Grp. gram.:m.}
\end{itemize}
Jôgo de rapazes, que consiste em se esconder uma coisa, para que aquelle que a ache corra os outros, azorragando-os.
\section{Chicuala}
\begin{itemize}
\item {Grp. gram.:f.}
\end{itemize}
\begin{itemize}
\item {Proveniência:(T. lund.)}
\end{itemize}
Árvore angolense, cujas flôres, dispostas em espigas, têm o limbo dividido em lóbos raiados de amarelo e roxo.
\section{Chicunco}
\begin{itemize}
\item {Grp. gram.:m.}
\end{itemize}
Pequena mucanda.
\section{Chideiro}
\begin{itemize}
\item {Grp. gram.:m.}
\end{itemize}
\begin{itemize}
\item {Utilização:Prov.}
\end{itemize}
\begin{itemize}
\item {Utilização:minh.}
\end{itemize}
Espécie de casaco de rabo curto, golla alta e botões amarelos, que os lavradores de Refoios do Lima usavam há meio século. Cf. Rev. \textunderscore Tradição\textunderscore , IV, 154.
(Cp. \textunderscore chedeiro\textunderscore )
\section{Chidura}
\begin{itemize}
\item {Grp. gram.:f.}
\end{itemize}
Vaso, em que os indígenas de Moçambique guardam mantimentos.
\section{Chieira}
\begin{itemize}
\item {Grp. gram.:f.}
\end{itemize}
\begin{itemize}
\item {Utilização:Prov.}
\end{itemize}
\begin{itemize}
\item {Utilização:dur.}
\end{itemize}
\begin{itemize}
\item {Utilização:minh.}
\end{itemize}
Vaidade; bazófia.
(Cp. \textunderscore chião\textunderscore )
\section{Chieirento}
\begin{itemize}
\item {Grp. gram.:adj.}
\end{itemize}
\begin{itemize}
\item {Utilização:Prov.}
\end{itemize}
\begin{itemize}
\item {Utilização:dur.}
\end{itemize}
\begin{itemize}
\item {Utilização:minh.}
\end{itemize}
\begin{itemize}
\item {Proveniência:(De \textunderscore chieira\textunderscore )}
\end{itemize}
Vaidoso.
Casquilho, peralta.
\section{Chifanga}
\begin{itemize}
\item {Grp. gram.:m.}
\end{itemize}
Antigo cárcere chinês. Cf. \textunderscore Peregrinação\textunderscore , LXIII e LXV.
\section{Chifarate}
\begin{itemize}
\item {Grp. gram.:m.}
\end{itemize}
\begin{itemize}
\item {Utilização:T. de Barcelos}
\end{itemize}
Cântaro.
Pichel.
\section{Chifarote}
\begin{itemize}
\item {Grp. gram.:m.}
\end{itemize}
\begin{itemize}
\item {Utilização:Des.}
\end{itemize}
\begin{itemize}
\item {Proveniência:(De \textunderscore chifra\textunderscore )}
\end{itemize}
Pequena espada.
\section{Chifra}
\begin{itemize}
\item {Grp. gram.:f.}
\end{itemize}
\begin{itemize}
\item {Proveniência:(Do ár. \textunderscore xafra\textunderscore )}
\end{itemize}
Instrumento de ferro, para raspar e adelgaçar o coiro.
\section{Chifrar}
\begin{itemize}
\item {Grp. gram.:v. t.}
\end{itemize}
Adelgaçar com a chifra.
\section{Chifre}
\begin{itemize}
\item {Grp. gram.:m.}
\end{itemize}
O mesmo que \textunderscore chavelho\textunderscore .
\section{Chifu}
\begin{itemize}
\item {Grp. gram.:m.}
\end{itemize}
Espécie de alcaide, entre os Chineses. Cf. \textunderscore Peregrinação\textunderscore , LXXXVII.
\section{Chila}
\begin{itemize}
\item {Grp. gram.:f.}
\end{itemize}
Espécie de abóbora pequena, de que se faz doce.
(Contr. da \textunderscore chilacaiota\textunderscore )
\section{Chila}
\begin{itemize}
\item {Grp. gram.:f.}
\end{itemize}
\begin{itemize}
\item {Utilização:Bras}
\end{itemize}
Fazenda de algodão, fabricada em Inglaterra e que se reexporta do Brasil para a costa da África.
\section{Chilacaiota}
\begin{itemize}
\item {Grp. gram.:f.}
\end{itemize}
\begin{itemize}
\item {Proveniência:(Do it. \textunderscore scilacaiota\textunderscore )}
\end{itemize}
Planta cucurbitácea, de que se faz o chamado dôce de chila.
\section{Chile}
\begin{itemize}
\item {Grp. gram.:m.}
\end{itemize}
\begin{itemize}
\item {Utilização:T. de Macau}
\end{itemize}
Espécie de pimento, (\textunderscore capsicum\textunderscore ).
\section{Chileira}
\begin{itemize}
\item {Grp. gram.:f.}
\end{itemize}
\begin{itemize}
\item {Utilização:Ant.}
\end{itemize}
\begin{itemize}
\item {Utilização:Prov.}
\end{itemize}
\begin{itemize}
\item {Utilização:dur.}
\end{itemize}
O mesmo que \textunderscore cheleira\textunderscore .
Pequeno sobrado no barco rabelo, junto á prôa.
\textunderscore Chileira da ré\textunderscore , pequeno sobrado, á ré dos barcos rabelos, onde se deitam e dormem os tripulantes.
\section{Chileireiro}
\begin{itemize}
\item {Grp. gram.:m.}
\end{itemize}
\begin{itemize}
\item {Utilização:Prov.}
\end{itemize}
\begin{itemize}
\item {Utilização:dur.}
\end{itemize}
Tripulante, que vai na chileira, á prôa.
\section{Chilena}
\begin{itemize}
\item {Grp. gram.:f.}
\end{itemize}
\begin{itemize}
\item {Utilização:Bras}
\end{itemize}
\begin{itemize}
\item {Proveniência:(De \textunderscore chileno\textunderscore )}
\end{itemize}
Espóra grande, de haste virada e grande roseta.
\section{Chilenismo}
\begin{itemize}
\item {Grp. gram.:m.}
\end{itemize}
Vocábulo ou expressão privativa do Chile.
\section{Chileno}
\begin{itemize}
\item {Grp. gram.:adj.}
\end{itemize}
\begin{itemize}
\item {Grp. gram.:M.}
\end{itemize}
Relativo ao Chile.
Habitante do Chile.
O mesmo que \textunderscore araucano\textunderscore .
\section{Chilíada}
\begin{itemize}
\item {fónica:qui}
\end{itemize}
\begin{itemize}
\item {Grp. gram.:f.}
\end{itemize}
\begin{itemize}
\item {Proveniência:(Do gr. \textunderscore khilias\textunderscore )}
\end{itemize}
Um milhar.
\section{Chiliarcha}
\begin{itemize}
\item {fónica:qui}
\end{itemize}
\begin{itemize}
\item {Grp. gram.:m.}
\end{itemize}
\begin{itemize}
\item {Proveniência:(Gr. \textunderscore khiliarkhos\textunderscore )}
\end{itemize}
Commandante de chiliarchia.
\section{Chiliarchia}
\begin{itemize}
\item {fónica:qui}
\end{itemize}
\begin{itemize}
\item {Grp. gram.:f.}
\end{itemize}
Formatura de 1024 homens ou duas pentacosiarchias, na phalange macedónica.
\section{Chiliare}
\begin{itemize}
\item {fónica:qui}
\end{itemize}
\begin{itemize}
\item {Grp. gram.:m.}
\end{itemize}
\begin{itemize}
\item {Proveniência:(Do gr. \textunderscore chilioi\textunderscore , e de \textunderscore are\textunderscore )}
\end{itemize}
Medida de superfície, equivalente a mil ares.
\section{Chiliasta}
\begin{itemize}
\item {fónica:qui}
\end{itemize}
\begin{itemize}
\item {Grp. gram.:m.}
\end{itemize}
\begin{itemize}
\item {Proveniência:(Do gr. \textunderscore khilioi\textunderscore , mil)}
\end{itemize}
O mesmo que \textunderscore millenário\textunderscore , sectário christão, que affirmava que a resurreição dos santos se havia de antecipar mil annos á geral.
\section{Chilido}
\begin{itemize}
\item {Grp. gram.:m.}
\end{itemize}
O mesmo que \textunderscore chilro\textunderscore ^1.
\section{Chilindrão}
\begin{itemize}
\item {Grp. gram.:m.}
\end{itemize}
Espécie de jôgo de cartas.
\section{Chiliógono}
\begin{itemize}
\item {fónica:qui}
\end{itemize}
\begin{itemize}
\item {Grp. gram.:m.}
\end{itemize}
\begin{itemize}
\item {Utilização:Mathem.}
\end{itemize}
\begin{itemize}
\item {Proveniência:(Do gr. \textunderscore khilioi\textunderscore  + \textunderscore gonos\textunderscore )}
\end{itemize}
Polýgno regular de mil lados.
\section{Chilindró}
\begin{itemize}
\item {Grp. gram.:m.}
\end{itemize}
\begin{itemize}
\item {Utilização:Pop.}
\end{itemize}
Estação policial; calaboiço.
\section{Chilique}
\begin{itemize}
\item {Grp. gram.:m.}
\end{itemize}
\begin{itemize}
\item {Utilização:Pop.}
\end{itemize}
Desmaio.
\section{Chilognathos}
\begin{itemize}
\item {fónica:qui}
\end{itemize}
\begin{itemize}
\item {Grp. gram.:m. pl}
\end{itemize}
\begin{itemize}
\item {Proveniência:(Do gr. \textunderscore kheilos\textunderscore  + \textunderscore gnathos\textunderscore )}
\end{itemize}
Ordem de animaes myriápodes.
\section{Chilondra}
\begin{itemize}
\item {Grp. gram.:f.}
\end{itemize}
\begin{itemize}
\item {Utilização:Prov.}
\end{itemize}
\begin{itemize}
\item {Utilização:trasm.}
\end{itemize}
Porca magra e grande.
(Cp. \textunderscore gironda\textunderscore ^1)
\section{Chiloplastia}
\begin{itemize}
\item {fónica:qui}
\end{itemize}
\begin{itemize}
\item {Grp. gram.:f.}
\end{itemize}
\begin{itemize}
\item {Proveniência:(Do gr. \textunderscore kheilos\textunderscore  + \textunderscore plassein\textunderscore )}
\end{itemize}
Operação cirúrgica, com que se restaura um ou ambos os lábios.
\section{Chiloplástico}
\begin{itemize}
\item {fónica:qui}
\end{itemize}
\begin{itemize}
\item {Grp. gram.:adj.}
\end{itemize}
Relativo á \textunderscore chiloplastia\textunderscore .
\section{Chilrada}
\begin{itemize}
\item {Grp. gram.:f.}
\end{itemize}
Acto de chilrar.
\section{Chilrão}
\begin{itemize}
\item {Grp. gram.:m.}
\end{itemize}
Rede, para pescar camarões.
\section{Chilrar}
\begin{itemize}
\item {Grp. gram.:v. i.}
\end{itemize}
(V.chilrear)(Cast. \textunderscore chilrar\textunderscore )
\section{Chilre}
\begin{itemize}
\item {Grp. gram.:adj.}
\end{itemize}
O mesmo que \textunderscore chilro\textunderscore ^1. Cf. Filinto, II, 85; V, 252.
\section{Chilreada}
\begin{itemize}
\item {Grp. gram.:f.}
\end{itemize}
O mesmo que \textunderscore chilro\textunderscore ^1.
\section{Chilreador}
\begin{itemize}
\item {Grp. gram.:m.  e  adj.}
\end{itemize}
O que chilreia.
\section{Chilreante}
\begin{itemize}
\item {Grp. gram.:adj.}
\end{itemize}
Que chilreia.
\section{Chilrear}
\begin{itemize}
\item {Grp. gram.:v. i.}
\end{itemize}
Pipilar.
Tagarelar.
(Refl. de \textunderscore chalrar\textunderscore )
\section{Chilreio}
\begin{itemize}
\item {Grp. gram.:m.}
\end{itemize}
Acto de chilrear.
\section{Chilreiro}
\begin{itemize}
\item {Grp. gram.:adj.}
\end{itemize}
(V.chilreador)
\section{Chilreta}
\begin{itemize}
\item {fónica:rê}
\end{itemize}
\begin{itemize}
\item {Grp. gram.:f.}
\end{itemize}
\begin{itemize}
\item {Utilização:Prov.}
\end{itemize}
\begin{itemize}
\item {Utilização:dur.}
\end{itemize}
O mesmo que \textunderscore gaivina\textunderscore .
\section{Chilro}
\begin{itemize}
\item {Grp. gram.:m.}
\end{itemize}
\begin{itemize}
\item {Proveniência:(De \textunderscore chilrar\textunderscore )}
\end{itemize}
Voz das aves.
\section{Chilro}
\begin{itemize}
\item {Grp. gram.:adj.}
\end{itemize}
Diz-se da água ou do caldo, sem substância oleosa, sem tempêro.
Insípido.
Insignificante, sem valor: \textunderscore prova chilra\textunderscore .
\section{Chim}
\begin{itemize}
\item {Grp. gram.:m.  e  adj.}
\end{itemize}
O mesmo que \textunderscore chinês\textunderscore .
\section{Chimaanas}
\begin{itemize}
\item {Grp. gram.:m. pl.}
\end{itemize}
Indígenas brasileiros, nas margens do Javari.
\section{Chimando}
\begin{itemize}
\item {Grp. gram.:m.}
\end{itemize}
\begin{itemize}
\item {Utilização:Bras}
\end{itemize}
Espécie de rato.
\section{Chimangata}
\begin{itemize}
\item {Grp. gram.:m.}
\end{itemize}
Cervo especial, em que alguns potentados e dignitários africanos se transportam cavalgados.
\section{Chimango}
\begin{itemize}
\item {Grp. gram.:m.}
\end{itemize}
\begin{itemize}
\item {Utilização:Bras. de Minas}
\end{itemize}
\begin{itemize}
\item {Utilização:Des.}
\end{itemize}
Nome, que se dava ao \textunderscore liberal\textunderscore , adversário dos \textunderscore caramurus\textunderscore , (Portugueses).
\section{Chimanos}
\begin{itemize}
\item {Grp. gram.:m. pl.}
\end{itemize}
Tríbo de índios do Brasil, nas margens do Javari.
\section{Chimarona}
\begin{itemize}
\item {Grp. gram.:f.}
\end{itemize}
Planta borragínea do Brasil.
\section{Chimarra}
\begin{itemize}
\item {Grp. gram.:f.}
\end{itemize}
Batina leve e simples de padre ou sacristão; sotaina, loba.
(Cp. \textunderscore samarra\textunderscore )
\section{Chimarrão}
\begin{itemize}
\item {Grp. gram.:m.}
\end{itemize}
\begin{itemize}
\item {Utilização:Bras. do S}
\end{itemize}
Gado bovino, que foge para os matos, vivendo ali sem sujeição alguma.
Mate sem açúcar ou mate amargo.
\section{Chimbé}
\begin{itemize}
\item {Grp. gram.:adj.}
\end{itemize}
\begin{itemize}
\item {Utilização:Bras. do S}
\end{itemize}
Diz-se do animal que tem o focinho chato.
(Do guar.)
\section{Chimbeva}
\begin{itemize}
\item {Grp. gram.:adj.}
\end{itemize}
\begin{itemize}
\item {Utilização:Bras}
\end{itemize}
O mesmo que \textunderscore chimbé\textunderscore .
\section{Chimbicar}
\begin{itemize}
\item {Grp. gram.:v. t.}
\end{itemize}
\begin{itemize}
\item {Utilização:T. de Angola}
\end{itemize}
Impellir (barco), fincando no chão a ponta da vara.
\section{Chimbile}
\begin{itemize}
\item {Grp. gram.:m.}
\end{itemize}
Árvore do Congo.
\section{Chimbombo}
\begin{itemize}
\item {Grp. gram.:m.}
\end{itemize}
O mesmo que \textunderscore quimbombo\textunderscore .
\section{Chimborgas}
\begin{itemize}
\item {Grp. gram.:m.}
\end{itemize}
\begin{itemize}
\item {Utilização:Prov.}
\end{itemize}
\begin{itemize}
\item {Utilização:alent.}
\end{itemize}
Indivíduo desprezível, bigorrilha.
\section{Chimboto}
\begin{itemize}
\item {Grp. gram.:m.}
\end{itemize}
Reptil angolense.
\section{Chimbuia}
\begin{itemize}
\item {Grp. gram.:f.}
\end{itemize}
Machadinha de luxo, entre os povos da Lunda, (África occidental).
\section{Chimela}
\begin{itemize}
\item {Grp. gram.:f.}
\end{itemize}
\begin{itemize}
\item {Utilização:Prov.}
\end{itemize}
\begin{itemize}
\item {Utilização:alg.}
\end{itemize}
O mesmo que \textunderscore chumela\textunderscore .
\section{Chimera}
\begin{itemize}
\item {fónica:qui}
\end{itemize}
\begin{itemize}
\item {Grp. gram.:f.}
\end{itemize}
\begin{itemize}
\item {Proveniência:(Do lat. \textunderscore chimaira\textunderscore )}
\end{itemize}
Monstro fabuloso.
Fantasia; producto da imaginação.
Absurdo.
Gênero de peixes.
Peixe dos mares do Sul, de pelle prateada e lisa, com três barbatanas no dorso.
\section{Chimericamente}
\begin{itemize}
\item {fónica:qui}
\end{itemize}
\begin{itemize}
\item {Grp. gram.:adv.}
\end{itemize}
De modo \textunderscore chimerico\textunderscore .
\section{Chimérico}
\begin{itemize}
\item {fónica:qui}
\end{itemize}
\begin{itemize}
\item {Grp. gram.:adj.}
\end{itemize}
\begin{itemize}
\item {Proveniência:(De \textunderscore chimera\textunderscore )}
\end{itemize}
Que não existe realmente; fantástico.
Que toma a fantasia como realidade.
\section{Chimerista}
\begin{itemize}
\item {fónica:qui}
\end{itemize}
\begin{itemize}
\item {Grp. gram.:m.}
\end{itemize}
Inventor de chimeras.
\section{Chimerizar}
\begin{itemize}
\item {fónica:qui}
\end{itemize}
\begin{itemize}
\item {Grp. gram.:v. i.}
\end{itemize}
\begin{itemize}
\item {Utilização:Neol.}
\end{itemize}
\begin{itemize}
\item {Grp. gram.:V. t.}
\end{itemize}
\begin{itemize}
\item {Proveniência:(De \textunderscore chimera\textunderscore )}
\end{itemize}
Inventar chimeras.
Imaginar, suppor chimericamente.
\section{Chimiatra}
\begin{itemize}
\item {fónica:qui}
\end{itemize}
\begin{itemize}
\item {Grp. gram.:m.}
\end{itemize}
\begin{itemize}
\item {Proveniência:(Do gr. \textunderscore khumia\textunderscore  + \textunderscore iatros\textunderscore )}
\end{itemize}
Médico, partidário da chimiatria.
\section{Chimiatria}
\begin{itemize}
\item {fónica:qui}
\end{itemize}
\begin{itemize}
\item {Grp. gram.:f.}
\end{itemize}
Systema daquelles que, no fim da Idade-Média, pretendiam explicar pela Chímica todos os phenómenos da economia animal.
(Cp. \textunderscore chimiatra\textunderscore )
\section{Chímica}
\begin{itemize}
\item {fónica:qui}
\end{itemize}
\begin{itemize}
\item {Grp. gram.:f.}
\end{itemize}
\begin{itemize}
\item {Proveniência:(Gr. \textunderscore khúmia\textunderscore , de \textunderscore khumos\textunderscore , suco)}
\end{itemize}
Sciência, que estuda a natureza e propriedade dos corpos, e as leis das suas combinações e decomposições.
\section{Chimicamente}
\begin{itemize}
\item {fónica:qui}
\end{itemize}
\begin{itemize}
\item {Grp. gram.:adv.}
\end{itemize}
De modo chímico.
\section{Chímico}
\begin{itemize}
\item {fónica:qui}
\end{itemize}
\begin{itemize}
\item {Grp. gram.:adj.}
\end{itemize}
\begin{itemize}
\item {Grp. gram.:M.}
\end{itemize}
Relativo á \textunderscore Chímica\textunderscore .
Aquelle que é versado em Chímica.
\section{Chimicotherapia}
\begin{itemize}
\item {fónica:qui}
\end{itemize}
\begin{itemize}
\item {Grp. gram.:f.}
\end{itemize}
Systema médico, que emprega de preferência os agentes chímicos.
\section{Chimiotaxia}
\begin{itemize}
\item {fónica:qui}
\end{itemize}
\begin{itemize}
\item {Grp. gram.:f.}
\end{itemize}
Acção attractiva ou repulsiva, exercida por certas substâncias sobre os microorganismos.
\section{Chimismo}
\begin{itemize}
\item {fónica:qui}
\end{itemize}
\begin{itemize}
\item {Grp. gram.:m.}
\end{itemize}
\begin{itemize}
\item {Proveniência:(De \textunderscore Chímica\textunderscore )}
\end{itemize}
Conjunto de combinações ou de composições de um organismo.
Abuso da Chímica.
\section{Chimista}
\begin{itemize}
\item {fónica:qui}
\end{itemize}
\begin{itemize}
\item {Grp. gram.:m.}
\end{itemize}
\begin{itemize}
\item {Proveniência:(Fr. \textunderscore chimiste\textunderscore )}
\end{itemize}
Aquelle que se dedica á prática da Chímica.
\section{Chimitypia}
\begin{itemize}
\item {fónica:qui}
\end{itemize}
\begin{itemize}
\item {Grp. gram.:f.}
\end{itemize}
Processo de gravura chímica, que transforma em lâmina de alto relêvo outra, gravada em baixo relêvo, acommodando-a á impressão.
\section{Chimonantho}
\begin{itemize}
\item {fónica:qui}
\end{itemize}
\begin{itemize}
\item {Grp. gram.:m.}
\end{itemize}
Planta calycanthácea.
\section{Chimophobia}
\begin{itemize}
\item {fónica:qui}
\end{itemize}
\begin{itemize}
\item {Grp. gram.:f.}
\end{itemize}
Temor mórbido das tempestades.
\section{Chimose}
\begin{itemize}
\item {fónica:qui}
\end{itemize}
\begin{itemize}
\item {Grp. gram.:f.}
\end{itemize}
\begin{itemize}
\item {Utilização:Med.}
\end{itemize}
\begin{itemize}
\item {Proveniência:(Gr. \textunderscore khimosis\textunderscore )}
\end{itemize}
Inchação na conjuntiva.
\section{Chimpanzé}
\begin{itemize}
\item {Grp. gram.:m.}
\end{itemize}
Grande macaco, sem cauda; pongo.
\textunderscore Fig.\textunderscore *
Pessôa feia e desajeitada.
\section{Chimpar}
\begin{itemize}
\item {Grp. gram.:v. t.}
\end{itemize}
\begin{itemize}
\item {Utilização:Pop.}
\end{itemize}
\begin{itemize}
\item {Utilização:Prov.}
\end{itemize}
\begin{itemize}
\item {Utilização:Minh.}
\end{itemize}
Pespegar.
Causar.
Meter, pôr, assentar.
Verter; entornar.
Derrubar.
\section{Chimporrada}
\begin{itemize}
\item {Grp. gram.:f.}
\end{itemize}
\begin{itemize}
\item {Utilização:Prov.}
\end{itemize}
\begin{itemize}
\item {Utilização:minh.}
\end{itemize}
O mesmo que \textunderscore chapoirada\textunderscore .
\section{China}
\begin{itemize}
\item {Grp. gram.:m.  e  f.}
\end{itemize}
Pessôa natural da China.
\textunderscore Laranja da china\textunderscore , variedade apreciada de laranjas.
\textunderscore Pomar da china\textunderscore , pomar que produz aquella variedade de laranja. Cf. P. Carvalho, \textunderscore Corogr. Port.\textunderscore , III, 325.
\section{China}
\begin{itemize}
\item {Grp. gram.:f.}
\end{itemize}
\begin{itemize}
\item {Utilização:Bras. do S}
\end{itemize}
Mulher de raça aborígene.
Espécie de raça bovina.
\section{China}
\begin{itemize}
\item {Grp. gram.:f.}
\end{itemize}
\begin{itemize}
\item {Utilização:ant.}
\end{itemize}
\begin{itemize}
\item {Utilização:Gír.}
\end{itemize}
Dinheiro.
\section{China}
\begin{itemize}
\item {Grp. gram.:f.}
\end{itemize}
\begin{itemize}
\item {Utilização:Prov.}
\end{itemize}
\begin{itemize}
\item {Utilização:trasm.}
\end{itemize}
Pedrinha, ou pedaço de caco, sôbre que se doba para fazer novelo. (Cast. \textunderscore china\textunderscore )
\section{Chinaphenina}
\begin{itemize}
\item {Grp. gram.:f.}
\end{itemize}
\begin{itemize}
\item {Utilização:Pharm.}
\end{itemize}
Medicamento contra a coqueluche.
\section{Chinar}
\begin{itemize}
\item {Grp. gram.:v. t.}
\end{itemize}
\begin{itemize}
\item {Utilização:Prov.}
\end{itemize}
\begin{itemize}
\item {Utilização:trasm.}
\end{itemize}
\begin{itemize}
\item {Proveniência:(De \textunderscore china\textunderscore ^4)}
\end{itemize}
Tapar com pedrinhas ou chinas os buracos de (uma parede, depois de se lhe deitar argamassa ou barro).
\section{Chinar}
\begin{itemize}
\item {Grp. gram.:v. t.}
\end{itemize}
\begin{itemize}
\item {Utilização:Prov.}
\end{itemize}
\begin{itemize}
\item {Utilização:trasm.}
\end{itemize}
\begin{itemize}
\item {Proveniência:(De \textunderscore chino\textunderscore ^1)}
\end{itemize}
Demarcar (campos).
\section{Chinar}
\begin{itemize}
\item {Grp. gram.:v. i.}
\end{itemize}
\begin{itemize}
\item {Utilização:Gír. de gatunos.}
\end{itemize}
Golpear, por fóra, um casaco vestido, para subtrahir carteira, que está numa algibeira interior.
\section{Chincada}
\begin{itemize}
\item {Grp. gram.:f.}
\end{itemize}
Acção de chincar.
\section{Chincalhada}
\begin{itemize}
\item {Grp. gram.:f.}
\end{itemize}
\begin{itemize}
\item {Utilização:Prov.}
\end{itemize}
\begin{itemize}
\item {Utilização:dur.}
\end{itemize}
Caçoada; zombaria.
(Cp. \textunderscore achincalhar\textunderscore )
\section{Chincalhão}
\begin{itemize}
\item {Grp. gram.:m.}
\end{itemize}
Jôgo de cartas, popular.
\section{Chincalhão}
\begin{itemize}
\item {Grp. gram.:m.}
\end{itemize}
O mesmo que \textunderscore tentilhão\textunderscore , em Penafiel e no Gerez.
\section{Chincar}
\begin{itemize}
\item {Grp. gram.:v. t.}
\end{itemize}
\begin{itemize}
\item {Utilização:Prov.}
\end{itemize}
\begin{itemize}
\item {Utilização:Fam.}
\end{itemize}
\begin{itemize}
\item {Grp. gram.:V. i.}
\end{itemize}
\begin{itemize}
\item {Utilização:Prov.}
\end{itemize}
\begin{itemize}
\item {Utilização:beir.}
\end{itemize}
Provar.
Gozar.
Fazer cambalear.
Desarmar casualmente (uma costella, uma armadilha).
Abichar.
Cair (num lôgro).
\section{Chincar}
\begin{itemize}
\item {Grp. gram.:v. i.}
\end{itemize}
(V. \textunderscore cincar\textunderscore ^1)
\section{Chincha}
\begin{itemize}
\item {Grp. gram.:f.}
\end{itemize}
Barco de pesca.
Rede pequena de arrastar.
\section{Chinchafóes}
\begin{itemize}
\item {Grp. gram.:m.}
\end{itemize}
O mesmo que \textunderscore chinchafoles\textunderscore .
\section{Chinchafóis}
\begin{itemize}
\item {Grp. gram.:m.}
\end{itemize}
O mesmo que \textunderscore chinchafoles\textunderscore .
\section{Chinchafol}
\begin{itemize}
\item {Grp. gram.:m.}
\end{itemize}
\begin{itemize}
\item {Utilização:Prov.}
\end{itemize}
Indivíduo presumido, ridiculamente janota.
\section{Chinchafoles}
\begin{itemize}
\item {Grp. gram.:f.}
\end{itemize}
O mesmo que \textunderscore chincra\textunderscore .
\section{Chincha-la-raiz}
\begin{itemize}
\item {Grp. gram.:m.}
\end{itemize}
\begin{itemize}
\item {Utilização:Prov.}
\end{itemize}
\begin{itemize}
\item {Utilização:trasm.}
\end{itemize}
Pequeno pássaro que, ao cantar, parece dizer \textunderscore chincha-la-raíz\textunderscore .
Chamam-lhe também \textunderscore tem-te-na-raiz\textunderscore .
\section{Chinchão}
\begin{itemize}
\item {Grp. gram.:m.}
\end{itemize}
\begin{itemize}
\item {Utilização:Prov.}
\end{itemize}
\begin{itemize}
\item {Utilização:trasm.}
\end{itemize}
Passarinho que, cantando, parece dizer só: \textunderscore chim! chim!\textunderscore 
\section{Chincharavelha}
\begin{itemize}
\item {fónica:vê}
\end{itemize}
\begin{itemize}
\item {Grp. gram.:f.}
\end{itemize}
\begin{itemize}
\item {Utilização:Prov.}
\end{itemize}
O mesmo que \textunderscore chincharavelho\textunderscore .
\section{Chincharavelho}
\begin{itemize}
\item {fónica:vê}
\end{itemize}
\begin{itemize}
\item {Grp. gram.:m.}
\end{itemize}
O mesmo que \textunderscore chinchavarelho\textunderscore . Cf. Júl. Dinís, \textunderscore Fam. Inglesa\textunderscore .
\section{Chincharel}
\begin{itemize}
\item {Grp. gram.:m.}
\end{itemize}
Peça de madeira, que se prega no tecto, entre dois barrotes, para fixar a suspensão de um candeeiro.
Peça de madeira, que se colloca diagonalmente sôbre o vigamento, para assentamento do parquete.
\section{Chincharravelho}
\begin{itemize}
\item {fónica:vê}
\end{itemize}
\begin{itemize}
\item {Grp. gram.:m.}
\end{itemize}
\begin{itemize}
\item {Utilização:Prov.}
\end{itemize}
\begin{itemize}
\item {Utilização:trasm.}
\end{itemize}
Pássaro dentirostro, o mesmo que \textunderscore megengra\textunderscore .
Traquinas.
\section{Chinchavarelho}
\begin{itemize}
\item {fónica:varê}
\end{itemize}
\begin{itemize}
\item {Grp. gram.:m.}
\end{itemize}
\begin{itemize}
\item {Utilização:Prov.}
\end{itemize}
\begin{itemize}
\item {Utilização:trasm.}
\end{itemize}
Pássaro dentirostro, o mesmo que \textunderscore megengra\textunderscore .
Traquinas.
\section{Chinche}
\begin{itemize}
\item {Grp. gram.:m.}
\end{itemize}
\begin{itemize}
\item {Utilização:Prov.}
\end{itemize}
\begin{itemize}
\item {Utilização:trasm.}
\end{itemize}
\begin{itemize}
\item {Proveniência:(Do lat. \textunderscore cimex\textunderscore )}
\end{itemize}
Insecto hemíptero.
Percevejo.
\section{Chincheiro}
\begin{itemize}
\item {Grp. gram.:m.}
\end{itemize}
Pescador, que se serve dos insectos chamados chinches, para isca.
\section{Chincherineta}
\begin{itemize}
\item {Grp. gram.:m.}
\end{itemize}
\begin{itemize}
\item {Utilização:Prov.}
\end{itemize}
\begin{itemize}
\item {Utilização:minh.}
\end{itemize}
Espécie de água-pé.
\section{Chinchila}
\begin{itemize}
\item {Grp. gram.:m.}
\end{itemize}
Animal peruano, da fam. dos roedores.
(Cast. \textunderscore chinchilla\textunderscore )
\section{Chinchilla}
\begin{itemize}
\item {Grp. gram.:m.}
\end{itemize}
Animal peruano, da fam. dos roedores.
(Cast. \textunderscore chinchilla\textunderscore )
\section{Chinchinho}
\begin{itemize}
\item {Grp. gram.:adj.}
\end{itemize}
\begin{itemize}
\item {Utilização:Açor}
\end{itemize}
O mesmo que \textunderscore pequerrucho\textunderscore .
\section{Chinchinim}
\begin{itemize}
\item {Grp. gram.:m.}
\end{itemize}
\begin{itemize}
\item {Proveniência:(T. onom.)}
\end{itemize}
O mesmo que \textunderscore megengra\textunderscore .
\section{Chincho}
\begin{itemize}
\item {Grp. gram.:m.}
\end{itemize}
\begin{itemize}
\item {Utilização:Açor}
\end{itemize}
O mesmo que \textunderscore menino\textunderscore .
\section{Chinchona}
\textunderscore f.\textunderscore  (e der.)
O mesmo ou melhor que \textunderscore cinchona\textunderscore , etc.
\section{Chinchona}
\begin{itemize}
\item {Grp. gram.:f.}
\end{itemize}
\begin{itemize}
\item {Utilização:Prov.}
\end{itemize}
\begin{itemize}
\item {Utilização:trasm.}
\end{itemize}
A fêmea do \textunderscore chinchão\textunderscore .
\section{Chinchorra}
\begin{itemize}
\item {fónica:chô}
\end{itemize}
\begin{itemize}
\item {Grp. gram.:adj. f.}
\end{itemize}
\begin{itemize}
\item {Utilização:T. de Aveiro}
\end{itemize}
Diz-se da bateira, em que se faz o lançamento da \textunderscore chincha\textunderscore .
\section{Chinchorro}
\begin{itemize}
\item {fónica:chô}
\end{itemize}
\begin{itemize}
\item {Grp. gram.:m.}
\end{itemize}
Espécie de chincha.
\section{Chinchoxo}
\begin{itemize}
\item {fónica:chô}
\end{itemize}
\begin{itemize}
\item {Grp. gram.:m.}
\end{itemize}
(V.chinchorro)
\section{Chinchoso}
\begin{itemize}
\item {Grp. gram.:adj.}
\end{itemize}
Que tem chinches.
\section{Chinclopé}
\begin{itemize}
\item {Grp. gram.:m.}
\end{itemize}
\begin{itemize}
\item {Utilização:Pop.}
\end{itemize}
Acto de andar com um pé no chão e outro no ar, como no \textunderscore jogo do homem\textunderscore ; pé-coxinho.
(Cp. \textunderscore chincar\textunderscore ^1, e \textunderscore pé\textunderscore )
\section{Chincoca}
\begin{itemize}
\item {Grp. gram.:f.}
\end{itemize}
\begin{itemize}
\item {Utilização:Prov.}
\end{itemize}
\begin{itemize}
\item {Utilização:alg.}
\end{itemize}
Coisa desagradável.
\section{Chincra}
\begin{itemize}
\item {Grp. gram.:f.}
\end{itemize}
Avezinha, (\textunderscore cysticola cursitans\textunderscore , Frankl.).
\section{Chindim}
\begin{itemize}
\item {Grp. gram.:m.}
\end{itemize}
\begin{itemize}
\item {Utilização:T. da Índia port}
\end{itemize}
Madeixa de cabello, que os gentios deixam crescer no alto da cabeça.
\section{Chineiro}
\begin{itemize}
\item {Grp. gram.:adj.}
\end{itemize}
\begin{itemize}
\item {Utilização:des.}
\end{itemize}
\begin{itemize}
\item {Utilização:Chul.}
\end{itemize}
Que anda endinheirado.
\section{Chinela}
\begin{itemize}
\item {Grp. gram.:f.}
\end{itemize}
\begin{itemize}
\item {Proveniência:(Do it. \textunderscore pianella\textunderscore )}
\end{itemize}
Calçado sem talão.
\section{Chinelada}
\begin{itemize}
\item {Grp. gram.:f.}
\end{itemize}
\begin{itemize}
\item {Proveniência:(De \textunderscore chinelo\textunderscore , ou \textunderscore chinela\textunderscore )}
\end{itemize}
Pancada com chinelo ou chinela.
\section{Chinelar}
\begin{itemize}
\item {Grp. gram.:v. i.}
\end{itemize}
\begin{itemize}
\item {Utilização:P. us.}
\end{itemize}
Fazer o ruído de quem anda com chinelas.
\section{Chineleiro}
\begin{itemize}
\item {Grp. gram.:m.}
\end{itemize}
\begin{itemize}
\item {Utilização:Fig.}
\end{itemize}
Aquelle que faz chinelas ou chinelos.
Indivíduo reles.
\section{Chinélo}
\begin{itemize}
\item {Grp. gram.:m.}
\end{itemize}
(V.chinela)
\section{Chinêlo}
\begin{itemize}
\item {Grp. gram.:m.}
\end{itemize}
(V.chinela)
\section{Chinês}
\begin{itemize}
\item {Grp. gram.:m.  e  adj.}
\end{itemize}
\begin{itemize}
\item {Grp. gram.:Adj.}
\end{itemize}
Aquelle que é natural da China.
Língua dos Chineses.
Relativo á China.
\section{Chinesaria}
\begin{itemize}
\item {Grp. gram.:f.}
\end{itemize}
O mesmo que \textunderscore chinesice\textunderscore .
\section{Chinesice}
\begin{itemize}
\item {Grp. gram.:f.}
\end{itemize}
\begin{itemize}
\item {Proveniência:(De \textunderscore chinês\textunderscore )}
\end{itemize}
Modos de chinês.
Bugiganga; pequeno artefacto que revela grande paciência.
\section{Chineta}
\begin{itemize}
\item {fónica:nê}
\end{itemize}
\begin{itemize}
\item {Grp. gram.:f.}
\end{itemize}
\begin{itemize}
\item {Utilização:Prov.}
\end{itemize}
O mesmo que \textunderscore chinita\textunderscore . (Colhido na Bairrada)
\section{Chinfrão}
\begin{itemize}
\item {Grp. gram.:m.}
\end{itemize}
Moéda antiga, equivalente a 14 reis.
\section{Chinfrar}
\begin{itemize}
\item {Grp. gram.:v. t.}
\end{itemize}
\begin{itemize}
\item {Utilização:Prov.}
\end{itemize}
\begin{itemize}
\item {Utilização:dur.}
\end{itemize}
Traçar ou partir, durante a vindima ou antes disso, as varas de (videiras goreiras), para prevenção dos podadores.
(Relaciona-se com \textunderscore chanfrar\textunderscore ?)
\section{Chinfre}
\begin{itemize}
\item {Grp. gram.:m.}
\end{itemize}
\begin{itemize}
\item {Utilização:Prov.}
\end{itemize}
\begin{itemize}
\item {Utilização:trasm.}
\end{itemize}
\begin{itemize}
\item {Proveniência:(De \textunderscore chinfrar\textunderscore )}
\end{itemize}
Fasquia de um canastro, de um caniço, de um tapamento.
\section{Chinfrim}
\begin{itemize}
\item {Grp. gram.:m.}
\end{itemize}
\begin{itemize}
\item {Utilização:Pop.}
\end{itemize}
\begin{itemize}
\item {Utilização:Bras}
\end{itemize}
\begin{itemize}
\item {Grp. gram.:Adj.}
\end{itemize}
Desordem, algazarra.
Baile popular.
Insignificante, reles.
\section{Chinfrinada}
\begin{itemize}
\item {Grp. gram.:f.}
\end{itemize}
\begin{itemize}
\item {Utilização:Pop.}
\end{itemize}
\begin{itemize}
\item {Proveniência:(De \textunderscore chinfrim\textunderscore )}
\end{itemize}
Algazarra.
Coisa ridícula.
\section{Chinfrinar}
\begin{itemize}
\item {Grp. gram.:v. i.}
\end{itemize}
Fazer chinfrim.
\section{Chinfrineira}
\begin{itemize}
\item {Grp. gram.:f.}
\end{itemize}
O mesmo que \textunderscore chinfrinada\textunderscore .
\section{Chinganja}
\begin{itemize}
\item {Grp. gram.:f.}
\end{itemize}
Grande peixe africano.
\section{Chingo}
\begin{itemize}
\item {Grp. gram.:m.}
\end{itemize}
\begin{itemize}
\item {Utilização:Prov.}
\end{itemize}
\begin{itemize}
\item {Utilização:beir.}
\end{itemize}
Pequena porção de líquido: \textunderscore bebeu de manhan um chingo de aguardente\textunderscore ; \textunderscore dá um chingo de água a essa criança\textunderscore .
\section{Chinguene}
\begin{itemize}
\item {Grp. gram.:m.}
\end{itemize}
Pequeno peixe africano.
\section{Chinguiço}
\begin{itemize}
\item {Grp. gram.:m.}
\end{itemize}
Rodoiça semi-circular, em que os mariolas assentam o pau, no cachaço.
\section{Chinguvo}
\begin{itemize}
\item {Grp. gram.:m.}
\end{itemize}
Instrumento musical, em fórma de caixa, usado na África.
\section{Chininha}
\begin{itemize}
\item {Grp. gram.:f.}
\end{itemize}
\begin{itemize}
\item {Utilização:Bras. do S}
\end{itemize}
Rapariga cabocla.
\section{Chinita}
\begin{itemize}
\item {Grp. gram.:f.}
\end{itemize}
\begin{itemize}
\item {Utilização:Prov.}
\end{itemize}
\begin{itemize}
\item {Utilização:alg.}
\end{itemize}
\begin{itemize}
\item {Utilização:Prov.}
\end{itemize}
\begin{itemize}
\item {Utilização:dur.}
\end{itemize}
\begin{itemize}
\item {Utilização:beir.}
\end{itemize}
Aguardente de figo, usada especialmente em feiras e arraiaes, acompanhada de batata doce, cozida.
Pequeno copo de qualquer bebida.
Pequena porção.
Pouca coisa.
\section{Chino}
\begin{itemize}
\item {Grp. gram.:m.}
\end{itemize}
\begin{itemize}
\item {Utilização:Prov.}
\end{itemize}
\begin{itemize}
\item {Utilização:trasm.}
\end{itemize}
\begin{itemize}
\item {Utilização:Prov.}
\end{itemize}
Pedrinha, para o mesmo fim que a \textunderscore china\textunderscore ^4 e para \textunderscore achinar\textunderscore  o tiro no jôgo da barra.
Marco dos campos.
Jôgo popular, espécie de \textunderscore fito\textunderscore .
\section{Chino}
\begin{itemize}
\item {Grp. gram.:m.  e  adj.}
\end{itemize}
(V.chinês)
\section{Chinó}
\begin{itemize}
\item {Grp. gram.:m.}
\end{itemize}
Cabelleira postiça.
\section{Chinoca}
\begin{itemize}
\item {Grp. gram.:f.}
\end{itemize}
\begin{itemize}
\item {Utilização:Bras. do S}
\end{itemize}
O mesmo que \textunderscore chininha\textunderscore .
\section{Chinoca}
\begin{itemize}
\item {Grp. gram.:adj. f.}
\end{itemize}
\begin{itemize}
\item {Utilização:Gír.}
\end{itemize}
Muito bôa.
\section{Chinonugo}
\begin{itemize}
\item {Grp. gram.:m.}
\end{itemize}
Árvore da Guiné portuguesa.
\section{Chinque}
\begin{itemize}
\item {Grp. gram.:m.}
\end{itemize}
Rêde, o mesmo que \textunderscore chincha\textunderscore .
Isca, engôdo, que se põe no anzol:«\textunderscore qual peixe ao chinque, todos acodem.\textunderscore »\textunderscore Vir. Trág.\textunderscore , XI, 54.
\section{Chinquilho}
\begin{itemize}
\item {Grp. gram.:m.}
\end{itemize}
Jôgo, também conhecido por \textunderscore malha\textunderscore .
(Cast. \textunderscore cinquillo\textunderscore )
\section{Chio}
\begin{itemize}
\item {Grp. gram.:m.}
\end{itemize}
\begin{itemize}
\item {Proveniência:(T. onom.)}
\end{itemize}
Voz aguda de alguns animaes.
Som agudo, produzido pela fricção de duas superfícies polidas.
\section{Chiococco}
\begin{itemize}
\item {Grp. gram.:m.}
\end{itemize}
\begin{itemize}
\item {Proveniência:(Do gr. \textunderscore khion\textunderscore  + \textunderscore kokko\textunderscore )}
\end{itemize}
Planta rubiácea.
\section{Chiococo}
\begin{itemize}
\item {fónica:cô}
\end{itemize}
\begin{itemize}
\item {Grp. gram.:m.}
\end{itemize}
\begin{itemize}
\item {Proveniência:(Do gr. \textunderscore khion\textunderscore  + \textunderscore kokko\textunderscore )}
\end{itemize}
Planta rubiácea.
\section{Chiola}
\begin{itemize}
\item {Grp. gram.:f.}
\end{itemize}
\begin{itemize}
\item {Utilização:Prov.}
\end{itemize}
\begin{itemize}
\item {Utilização:trasm.}
\end{itemize}
Carro de bois, velho, prestes a desconjuntar-se.
\section{Chiolas}
\begin{itemize}
\item {Grp. gram.:f. pl.}
\end{itemize}
\begin{itemize}
\item {Utilização:Prov.}
\end{itemize}
\begin{itemize}
\item {Utilização:beir.}
\end{itemize}
\begin{itemize}
\item {Utilização:Prov.}
\end{itemize}
\begin{itemize}
\item {Utilização:trasm.}
\end{itemize}
\begin{itemize}
\item {Proveniência:(De \textunderscore chiar\textunderscore ?)}
\end{itemize}
O mesmo que \textunderscore andas\textunderscore .
Botas com lastro de madeira, bipartido e ligado por sola.
\section{Chiote}
\begin{itemize}
\item {Grp. gram.:m.}
\end{itemize}
\begin{itemize}
\item {Utilização:Ant.}
\end{itemize}
Vestuário de burel, com capello, usado por pastores e camponeses.
\section{Chipa}
\begin{itemize}
\item {Grp. gram.:f.}
\end{itemize}
Designação vulgar, na África central, da \textunderscore tetrapleura andongensis\textunderscore , Welw. Cf. Serpa Pinto, \textunderscore Como eu atravess. a Áfr.\textunderscore 
\section{Chipaia}
\begin{itemize}
\item {Grp. gram.:f.}
\end{itemize}
Cêsto africano, em fórma de vaso.
\section{Chipalada}
\begin{itemize}
\item {Grp. gram.:f.}
\end{itemize}
Quadrúpede da África Oriental portuguesa.
\section{Chipante}
\begin{itemize}
\item {Grp. gram.:m.}
\end{itemize}
\begin{itemize}
\item {Proveniência:(De \textunderscore chipo\textunderscore )}
\end{itemize}
Barco, com que se pesca aljofre.
\section{Chipo}
\begin{itemize}
\item {Grp. gram.:m.}
\end{itemize}
\begin{itemize}
\item {Proveniência:(T. as.)}
\end{itemize}
Ostra, que produz aljofre.
\section{Chipolim}
\begin{itemize}
\item {Grp. gram.:m.}
\end{itemize}
\begin{itemize}
\item {Proveniência:(Fr. \textunderscore chipolin\textunderscore )}
\end{itemize}
Processo antigo de pintura a colla, com muitas camadas de verniz, alisando-se tudo a pedra pomes.
\section{Chique}
\begin{itemize}
\item {Grp. gram.:adj.}
\end{itemize}
\begin{itemize}
\item {Utilização:Neol.}
\end{itemize}
\begin{itemize}
\item {Proveniência:(Do fr. \textunderscore chic\textunderscore )}
\end{itemize}
Elegante; bonito; catita.
\section{Chique}
\begin{itemize}
\item {Grp. gram.:m.  ou  pron.}
\end{itemize}
\begin{itemize}
\item {Utilização:Ant.}
\end{itemize}
Pouca coisa?«\textunderscore Nem chique nem mique nem nada dão a elle nem a mi.\textunderscore »G. Vicente.
\section{Chiquechique}
\begin{itemize}
\item {Grp. gram.:m.}
\end{itemize}
Espécie de cacto do Brasil.
\section{Chiqueirá}
\begin{itemize}
\item {Grp. gram.:m.}
\end{itemize}
\begin{itemize}
\item {Utilização:Bras}
\end{itemize}
\begin{itemize}
\item {Proveniência:(De \textunderscore chiqueirar\textunderscore )}
\end{itemize}
Espécie de chicote, composto de um cacete, e que tem numa das extremidades uma tira de coiro, torcida ou chata.
\section{Chiqueirador}
\begin{itemize}
\item {Grp. gram.:m.}
\end{itemize}
\begin{itemize}
\item {Utilização:Bras. do N}
\end{itemize}
\begin{itemize}
\item {Proveniência:(De \textunderscore chiqueirar\textunderscore )}
\end{itemize}
Espécie de chicote, composto de um cacete, e que tem numa das extremidades uma tira de coiro, torcida ou chata.
\section{Chiqueirar}
\begin{itemize}
\item {Grp. gram.:v. t.}
\end{itemize}
\begin{itemize}
\item {Utilização:Bras. do N}
\end{itemize}
Meter em chiqueiro ou separar (bezerros) das mães, para que não mamem ou se possam desmamar.
\section{Chiqueiro}
\begin{itemize}
\item {Grp. gram.:m.}
\end{itemize}
\begin{itemize}
\item {Utilização:Bras}
\end{itemize}
Curral de porcos.
Lodaçal.
Lugar immundo.
Immundície.
Um dos compartimentos do curral de peixe.
Estacaría, com que nos rios se detém o peixe.
Curral de bezerros.
\section{Chiquel}
\begin{itemize}
\item {Grp. gram.:m.}
\end{itemize}
\begin{itemize}
\item {Proveniência:(T. as.)}
\end{itemize}
Vasilha de coiro, ou borracha, para água, em jornada.
\section{Chiquilha}
\begin{itemize}
\item {Grp. gram.:f.}
\end{itemize}
\begin{itemize}
\item {Utilização:Prov.}
\end{itemize}
\begin{itemize}
\item {Utilização:alent.}
\end{itemize}
Sardinha pequena; petinga.
(Cast. \textunderscore chiquilla\textunderscore , fem. de \textunderscore chiquillo\textunderscore , de \textunderscore chico\textunderscore , pequeno)
\section{Chiquismo}
\begin{itemize}
\item {Grp. gram.:m.}
\end{itemize}
\begin{itemize}
\item {Utilização:Neol.}
\end{itemize}
Qualidade do que é \textunderscore chique\textunderscore ^1.
\section{Chiquita}
\begin{itemize}
\item {Grp. gram.:f.}
\end{itemize}
Pelle, que os Lundeses suspendem sôbre as nádegas.
\section{Chiquita}
\begin{itemize}
\item {Grp. gram.:f.}
\end{itemize}
\begin{itemize}
\item {Utilização:Prov.}
\end{itemize}
\begin{itemize}
\item {Utilização:trasm.}
\end{itemize}
Borracheira.
\section{Chiquiteira}
\begin{itemize}
\item {Grp. gram.:f.}
\end{itemize}
Mulher, que fabríca chiquitos.
\section{Chiquitinho}
\begin{itemize}
\item {Grp. gram.:adj.}
\end{itemize}
\begin{itemize}
\item {Utilização:Prov.}
\end{itemize}
\begin{itemize}
\item {Utilização:minh.}
\end{itemize}
Pequenino.
Estreito ou apertado e curto: \textunderscore casaco chiquitinho\textunderscore .
\section{Chiquito}
\begin{itemize}
\item {Grp. gram.:m.}
\end{itemize}
\begin{itemize}
\item {Utilização:Prov.}
\end{itemize}
\begin{itemize}
\item {Utilização:bras. do N}
\end{itemize}
\begin{itemize}
\item {Utilização:alg.}
\end{itemize}
Sapatinho de criança.
(Cast. \textunderscore chiquito\textunderscore , de \textunderscore chico\textunderscore )
\section{Chiquitos}
\begin{itemize}
\item {Grp. gram.:m.}
\end{itemize}
\begin{itemize}
\item {Grp. gram.:Pl.}
\end{itemize}
Língua peruana.
Índios selvagens do Peru, na fronteira da Bolívia.
(Cp. \textunderscore chiquito\textunderscore )
\section{Chira}
\begin{itemize}
\item {Grp. gram.:f.}
\end{itemize}
Alimentação. Cf. Filinto, IV, 168. Us. especialmente na expressão \textunderscore bona-chira\textunderscore , ou \textunderscore boa-chira\textunderscore .
\section{Chiragra}
\begin{itemize}
\item {fónica:qui}
\end{itemize}
\begin{itemize}
\item {Grp. gram.:f.}
\end{itemize}
\begin{itemize}
\item {Utilização:Med.}
\end{itemize}
\begin{itemize}
\item {Proveniência:(Do gr. \textunderscore kheir\textunderscore  + \textunderscore agra\textunderscore )}
\end{itemize}
Doença de gota nas mãos.
\section{Chiraita}
\begin{itemize}
\item {Grp. gram.:f.}
\end{itemize}
Planta gencianácea medicinal, do norte da Índia, (\textunderscore ophelia chirata\textunderscore , Grisebach). Cf. \textunderscore Pharmacopeia Port.\textunderscore 
\section{Chirantho}
\begin{itemize}
\item {fónica:qui}
\end{itemize}
\begin{itemize}
\item {Grp. gram.:m.}
\end{itemize}
\begin{itemize}
\item {Proveniência:(Do gr. \textunderscore kheir\textunderscore , mão, e \textunderscore anthos\textunderscore , flôr)}
\end{itemize}
Gênero de plantas crucíferas.
\section{Chirata}
\begin{itemize}
\item {Grp. gram.:f.}
\end{itemize}
Planta gencianácea medicinal, do norte da Índia, (\textunderscore ophelia chirata\textunderscore , Grisebach). Cf. \textunderscore Pharmacopeia Port.\textunderscore 
\section{Chircá}
\begin{itemize}
\item {Grp. gram.:m.}
\end{itemize}
Antigo magistrado na Indo-China.
\section{Chirelo}
\begin{itemize}
\item {Grp. gram.:m.}
\end{itemize}
\begin{itemize}
\item {Utilização:Prov.}
\end{itemize}
\begin{itemize}
\item {Utilização:minh.}
\end{itemize}
O mesmo que \textunderscore carapau\textunderscore .
\section{Chicio}
\begin{itemize}
\item {Grp. gram.:m.}
\end{itemize}
Arbusto de Moçambique, cujas vagens dão uma tinta preta, usada pelos indígenas.
\section{Chireta}
\begin{itemize}
\item {fónica:chirê}
\end{itemize}
\begin{itemize}
\item {Grp. gram.:f.}
\end{itemize}
O mesmo que \textunderscore furabuxo\textunderscore .
\section{Chirimbote}
\begin{itemize}
\item {Grp. gram.:m.}
\end{itemize}
\begin{itemize}
\item {Utilização:Prov.}
\end{itemize}
\begin{itemize}
\item {Utilização:trasm.}
\end{itemize}
Traços caprichosos de luz, que se fazem no ar, agitando-se um tição aceso, quási sempre para entreter crianças.
\section{Chirimia}
\begin{itemize}
\item {Grp. gram.:f.}
\end{itemize}
\begin{itemize}
\item {Utilização:Ant.}
\end{itemize}
O mesmo que \textunderscore charamela\textunderscore .
\section{Chirimóia}
\begin{itemize}
\item {Grp. gram.:f.}
\end{itemize}
\begin{itemize}
\item {Proveniência:(T. cast.)}
\end{itemize}
O mesmo que \textunderscore cherimólia\textunderscore .
\section{Chirimóia}
\begin{itemize}
\item {Grp. gram.:m.}
\end{itemize}
\begin{itemize}
\item {Utilização:Prov.}
\end{itemize}
\begin{itemize}
\item {Utilização:trasm.}
\end{itemize}
Homem de nenhuma importância, um jan-ninguém.
\section{Chirinola}
\begin{itemize}
\item {Grp. gram.:f.}
\end{itemize}
\begin{itemize}
\item {Utilização:Pop.}
\end{itemize}
Confusão, trapalhada.
Armadilha.
(Cast. \textunderscore chirinola\textunderscore )
\section{Chiripá}
\begin{itemize}
\item {Grp. gram.:m.}
\end{itemize}
\begin{itemize}
\item {Utilização:Bras. do S}
\end{itemize}
Baêta encarnada, que os peões usam em redor da cintura.
\section{Chirita}
\begin{itemize}
\item {fónica:qui}
\end{itemize}
\begin{itemize}
\item {Grp. gram.:f.}
\end{itemize}
\begin{itemize}
\item {Proveniência:(Do gr. \textunderscore kheir\textunderscore )}
\end{itemize}
Estalactite em fórma de mão.
\section{Chiriúba}
\begin{itemize}
\item {Grp. gram.:f.}
\end{itemize}
Árvore, cujas raízes se aproveitam no fabrico do sabão.
\section{Chiriubeira}
\begin{itemize}
\item {Grp. gram.:f.}
\end{itemize}
\begin{itemize}
\item {Utilização:Bras}
\end{itemize}
Árvore, cujas raízes se aproveitam no fabrico do sabão.
\section{Chirographário}
\begin{itemize}
\item {fónica:qui}
\end{itemize}
\begin{itemize}
\item {Grp. gram.:adj.}
\end{itemize}
\begin{itemize}
\item {Proveniência:(Lat. \textunderscore chirographarius\textunderscore )}
\end{itemize}
Relativo a documentos particulares, não authenticados.
\section{Chirógrapho}
\begin{itemize}
\item {fónica:qui}
\end{itemize}
\begin{itemize}
\item {Grp. gram.:m.}
\end{itemize}
\begin{itemize}
\item {Proveniência:(Lat. \textunderscore chirographum\textunderscore )}
\end{itemize}
O mesmo que \textunderscore autógrapho\textunderscore .
Diploma.
Breve pontifício, não publicado.
\section{Chirogymnasta}
\begin{itemize}
\item {fónica:qui}
\end{itemize}
\begin{itemize}
\item {Grp. gram.:m.}
\end{itemize}
O mesmo que \textunderscore chiroplasto\textunderscore .
\section{Chirologia}
\begin{itemize}
\item {fónica:qui}
\end{itemize}
\begin{itemize}
\item {Grp. gram.:f.}
\end{itemize}
O mesmo que \textunderscore dactylologia\textunderscore .
\section{Chirológico}
\begin{itemize}
\item {fónica:qui}
\end{itemize}
\begin{itemize}
\item {Grp. gram.:adj.}
\end{itemize}
Relativo á chirologia.
\section{Chiromancia}
\begin{itemize}
\item {fónica:qui}
\end{itemize}
\begin{itemize}
\item {Grp. gram.:f.}
\end{itemize}
\begin{itemize}
\item {Proveniência:(Do gr. \textunderscore kheir\textunderscore  + \textunderscore manteia\textunderscore )}
\end{itemize}
Systema de adivinhação, pela inspecção das linhas da palma da mão.
\section{Chiromante}
\begin{itemize}
\item {fónica:qui}
\end{itemize}
\begin{itemize}
\item {Grp. gram.:m.}
\end{itemize}
Aquelle que pratíca a chiromancia.
\section{Chiromântico}
\begin{itemize}
\item {fónica:qui}
\end{itemize}
\begin{itemize}
\item {Grp. gram.:adj.}
\end{itemize}
Relativo á chiromancia.
\section{Chironecto}
\begin{itemize}
\item {fónica:qui}
\end{itemize}
\begin{itemize}
\item {Grp. gram.:m.}
\end{itemize}
\begin{itemize}
\item {Proveniência:(Do gr. \textunderscore kheir\textunderscore  + \textunderscore nektes\textunderscore )}
\end{itemize}
Mamífero aquático, do gênero das sarigueias.
\section{Chironomia}
\begin{itemize}
\item {fónica:qui}
\end{itemize}
\begin{itemize}
\item {Grp. gram.:f.}
\end{itemize}
Arte de apropriar os gestos ao discurso.
(Cp. \textunderscore chirónomo\textunderscore )
\section{Chironómico}
\begin{itemize}
\item {fónica:qui}
\end{itemize}
\begin{itemize}
\item {Grp. gram.:adj.}
\end{itemize}
Relativo á chironomia.
\section{Chirónomo}
\begin{itemize}
\item {fónica:qui}
\end{itemize}
\begin{itemize}
\item {Grp. gram.:m.}
\end{itemize}
\begin{itemize}
\item {Proveniência:(Do gr. \textunderscore kheir\textunderscore  + \textunderscore nomos\textunderscore )}
\end{itemize}
Aquelle que pratíca ou ensina a chironomia.
\section{Chiroplasto}
\begin{itemize}
\item {fónica:qui}
\end{itemize}
\begin{itemize}
\item {Grp. gram.:m.}
\end{itemize}
\begin{itemize}
\item {Proveniência:(Do gr. \textunderscore kheir\textunderscore  + \textunderscore plassein\textunderscore )}
\end{itemize}
Apparelho, para facilitar o estudo do piano, adaptando-se ao teclado e guiando o movimento dos dedos.
\section{Chiropótamo}
\begin{itemize}
\item {fónica:qui}
\end{itemize}
\begin{itemize}
\item {Grp. gram.:m.}
\end{itemize}
Animal fóssil, aquático.
\section{Chirópteros}
\begin{itemize}
\item {fónica:qui}
\end{itemize}
\begin{itemize}
\item {Grp. gram.:m. pl.}
\end{itemize}
\begin{itemize}
\item {Proveniência:(Do gr. \textunderscore kheir\textunderscore , mão, e \textunderscore pteron\textunderscore , asa)}
\end{itemize}
Ordem de mamíferos, que têm por typo o morcego.
\section{Chiroqui}
\begin{itemize}
\item {Grp. gram.:m.}
\end{itemize}
Uma das línguas da América do Norte.
\section{Chiroscopia}
\begin{itemize}
\item {fónica:qui}
\end{itemize}
\begin{itemize}
\item {Grp. gram.:f.}
\end{itemize}
\begin{itemize}
\item {Proveniência:(Do gr. \textunderscore kheir\textunderscore  + \textunderscore skopein\textunderscore )}
\end{itemize}
O mesmo que \textunderscore chiromancia\textunderscore .
\section{Chirothecas}
\begin{itemize}
\item {fónica:qui}
\end{itemize}
\begin{itemize}
\item {Grp. gram.:f. pl.}
\end{itemize}
\begin{itemize}
\item {Proveniência:(Do gr. \textunderscore kheir\textunderscore  + \textunderscore theke\textunderscore )}
\end{itemize}
Luvas, que os Bispos ou certos Abbades usavam em certas solennidades.
\section{Chirotonia}
\begin{itemize}
\item {fónica:qui}
\end{itemize}
\begin{itemize}
\item {Grp. gram.:f.}
\end{itemize}
\begin{itemize}
\item {Proveniência:(Do gr. \textunderscore kheir\textunderscore  + \textunderscore teinein\textunderscore )}
\end{itemize}
Imposição das mãos.
Entre os Gregos, acto de votar, levantando a mão.
\section{Chirreante}
\begin{itemize}
\item {Grp. gram.:adj.}
\end{itemize}
Que chirreia.
\section{Chirrear}
\begin{itemize}
\item {Grp. gram.:v. i.}
\end{itemize}
Produzir som estrídulo e prolongado, como a voz da coruja.
Cantar (a coruja).
(Talvez corr. de \textunderscore chilrear\textunderscore )
\section{Chiruça}
\begin{itemize}
\item {Grp. gram.:f.}
\end{itemize}
Árvore de Moçambique.
\section{Chirurgia}
\begin{itemize}
\item {fónica:qui}
\end{itemize}
\textunderscore f.\textunderscore  (e der.)
O mesmo que \textunderscore cirurgia\textunderscore , etc.
\section{Chirussa}
\begin{itemize}
\item {Grp. gram.:f.}
\end{itemize}
Árvore de Moçambique.
\section{Chirúvia}
\begin{itemize}
\item {fónica:qui}
\end{itemize}
\begin{itemize}
\item {Grp. gram.:f.}
\end{itemize}
\begin{itemize}
\item {Utilização:Ant.}
\end{itemize}
Planta, o mesmo que \textunderscore bisnaga\textunderscore . Cf. \textunderscore Desengano da Med.\textunderscore , 193.
\section{Chisca}
\begin{itemize}
\item {Grp. gram.:f.}
\end{itemize}
\begin{itemize}
\item {Utilização:Prov.}
\end{itemize}
\begin{itemize}
\item {Utilização:beir.}
\end{itemize}
\begin{itemize}
\item {Utilização:T. de Turquel}
\end{itemize}
Pequenina porção.
Jôgo infantil.
\section{Chiscar}
\begin{itemize}
\item {Grp. gram.:v. i.}
\end{itemize}
\begin{itemize}
\item {Utilização:Prov.}
\end{itemize}
\begin{itemize}
\item {Utilização:trasm.}
\end{itemize}
\begin{itemize}
\item {Utilização:T. de Murça}
\end{itemize}
Questionar (entre rapazes).
Picar o lanço, em leilões.
Brincar.
\section{Chiscarás}
(?)«\textunderscore ...não sabeis quantos fazem chiscarás\textunderscore ». \textunderscore Auto de Santo António\textunderscore , cit. por Castilho.
\section{Chisco}
\begin{itemize}
\item {Grp. gram.:m.}
\end{itemize}
\begin{itemize}
\item {Utilização:Prov.}
\end{itemize}
\begin{itemize}
\item {Utilização:beir.}
\end{itemize}
O mesmo que \textunderscore chisca\textunderscore , bocadinho.
(Cp. \textunderscore cisco\textunderscore )
\section{Chismes}
\begin{itemize}
\item {Grp. gram.:m. pl.}
\end{itemize}
\begin{itemize}
\item {Utilização:Prov.}
\end{itemize}
\begin{itemize}
\item {Utilização:trasm.}
\end{itemize}
\begin{itemize}
\item {Proveniência:(Do lat. \textunderscore cimex\textunderscore )}
\end{itemize}
Petrechos de caça.
Conjunto dos petrechos precisos para se petiscar lume.
Percevejos.
\section{Chisnar}
\begin{itemize}
\item {Grp. gram.:v. t.}
\end{itemize}
\begin{itemize}
\item {Utilização:Prov.}
\end{itemize}
\begin{itemize}
\item {Utilização:trasm.}
\end{itemize}
\begin{itemize}
\item {Utilização:minh.}
\end{itemize}
Esturrar.
Queimar.
(Por \textunderscore tisnar\textunderscore )
\section{Chispa}
\begin{itemize}
\item {Grp. gram.:f.}
\end{itemize}
Centelha, faísca, fagulha, que resalta de uma substância candente ou ferida por outro corpo.
Lampejo; fulgor passageiro.
Talento, gênio.
(Cast. \textunderscore chispa\textunderscore )
\section{Chispante}
\begin{itemize}
\item {Grp. gram.:adj.}
\end{itemize}
Que chispa. Cf. Castilho, \textunderscore Fausto\textunderscore , 335.
\section{Chispar}
\begin{itemize}
\item {Grp. gram.:v. i.}
\end{itemize}
\begin{itemize}
\item {Utilização:Fig.}
\end{itemize}
Lançar chispas.
Encolerizar-se.
\section{Chispe}
\begin{itemize}
\item {Grp. gram.:m.}
\end{itemize}
\begin{itemize}
\item {Utilização:Ant.}
\end{itemize}
Pé de porco.
Sapatinho polido e alto, de que usavam mulheres dissolutas.
\section{Chispo}
\begin{itemize}
\item {Grp. gram.:m.}
\end{itemize}
\begin{itemize}
\item {Utilização:Ant.}
\end{itemize}
Sapato alto e bicudo, usado por mulheres.
O mesmo que \textunderscore chispe\textunderscore .
\section{Chisquinho}
\begin{itemize}
\item {Grp. gram.:m.}
\end{itemize}
\begin{itemize}
\item {Utilização:Prov.}
\end{itemize}
\begin{itemize}
\item {Utilização:beir.}
\end{itemize}
O mesmo que \textunderscore chisquito\textunderscore .
\section{Chisquito}
\begin{itemize}
\item {Grp. gram.:m.}
\end{itemize}
\begin{itemize}
\item {Utilização:Prov.}
\end{itemize}
\begin{itemize}
\item {Utilização:beir.}
\end{itemize}
Pequena porção de qualquer coisa de pouco valor.
Graveto.
(Por \textunderscore cisquito\textunderscore , de \textunderscore cisco\textunderscore ?)
\section{Chissio}
\begin{itemize}
\item {Grp. gram.:m.}
\end{itemize}
Arbusto de Moçambique, cujas vagens dão uma tinta preta, usada pelos indígenas.
\section{Chiste}
\begin{itemize}
\item {Grp. gram.:m.}
\end{itemize}
Dito gracioso; graça; facécia.
(Cast. \textunderscore chiste\textunderscore )
\section{Chistira}
\begin{itemize}
\item {Grp. gram.:f.}
\end{itemize}
\begin{itemize}
\item {Utilização:Ant.}
\end{itemize}
Espécie de esteira, fabricada na China.
\section{Chistoso}
\begin{itemize}
\item {Grp. gram.:adj.}
\end{itemize}
Que tem chiste.
Em que há chiste.
Que revela chiste.
Gracioso.
\section{Chita}
\begin{itemize}
\item {Grp. gram.:f.}
\end{itemize}
Tecido ordinário de algodão, estampado a côres.
(Do marata \textunderscore chit\textunderscore )
\section{Chitada}
\begin{itemize}
\item {Grp. gram.:f.}
\end{itemize}
Perda de um jôgo de cartas, sem se têr feito uma vasa.
\section{Chitão!}
\begin{itemize}
\item {Grp. gram.:interj.}
\end{itemize}
O mesmo que \textunderscore chiton!\textunderscore .
\section{Chitaria}
\begin{itemize}
\item {Grp. gram.:f.}
\end{itemize}
Estabelecimento ou fábrica de chitas.
\section{Chitata}
\begin{itemize}
\item {Grp. gram.:f.}
\end{itemize}
Pequena aringa.
\section{Chite!}
\begin{itemize}
\item {Grp. gram.:interj.}
\end{itemize}
\begin{itemize}
\item {Utilização:Prov.}
\end{itemize}
\begin{itemize}
\item {Utilização:trasm.}
\end{itemize}
O mesmo que \textunderscore chíton\textunderscore .
\section{Chitela}
\begin{itemize}
\item {Grp. gram.:f.}
\end{itemize}
Espécie de antílope? Cf. Th. Ribeiro, \textunderscore Jornadas\textunderscore , II, 259.
\section{Chitelha}
\begin{itemize}
\item {fónica:tê}
\end{itemize}
\begin{itemize}
\item {Grp. gram.:f.}
\end{itemize}
Chita de qualidade inferior.
\section{Chitina}
\begin{itemize}
\item {fónica:xitoukit}
\end{itemize}
\begin{itemize}
\item {Grp. gram.:f.}
\end{itemize}
Substância exsudada pela epiderme e que, fluida ao princípio, depois endurece, formando a parte dura do esqueleto dos arthrópodes.
\section{Chitó}
\begin{itemize}
\item {Grp. gram.:m.}
\end{itemize}
\begin{itemize}
\item {Utilização:T. da Bairrada}
\end{itemize}
O mesmo que \textunderscore chitelha\textunderscore .
\section{Chitom!}
\begin{itemize}
\item {Grp. gram.:interj.}
\end{itemize}
\begin{itemize}
\item {Proveniência:(Fr. \textunderscore chut donc\textunderscore )}
\end{itemize}
Silêncio! caluda!
\section{Chiton!}
\begin{itemize}
\item {Grp. gram.:interj.}
\end{itemize}
\begin{itemize}
\item {Proveniência:(Fr. \textunderscore chut donc\textunderscore )}
\end{itemize}
Silêncio! caluda!
\section{Chíton}
\begin{itemize}
\item {fónica:qui}
\end{itemize}
\begin{itemize}
\item {Grp. gram.:m.}
\end{itemize}
\begin{itemize}
\item {Proveniência:(Gr. \textunderscore khiton\textunderscore )}
\end{itemize}
Túnica jónica e dórica.
Gênero de molluscos gasterópodes.
\section{Chitoto}
\begin{itemize}
\item {Grp. gram.:m.}
\end{itemize}
Pequena lontra africana.
\section{Chitungulo}
\begin{itemize}
\item {Grp. gram.:m.}
\end{itemize}
Pequeno peixe africano.
\section{Chiu}
\begin{itemize}
\item {Grp. gram.:m.}
\end{itemize}
Árvore de Moçambique, de que se fazem arcos.
\section{Chizinho}
\begin{itemize}
\item {Grp. gram.:m.}
\end{itemize}
\begin{itemize}
\item {Utilização:Prov.}
\end{itemize}
\begin{itemize}
\item {Utilização:beir.}
\end{itemize}
Pequeníssima porção; chingo.
Tudo-nada.
\section{Chlâmyde}
\begin{itemize}
\item {Grp. gram.:f.}
\end{itemize}
\begin{itemize}
\item {Proveniência:(Do gr. \textunderscore khlamus\textunderscore )}
\end{itemize}
Manto rico dos antigos, seguro por um broche ao pescoço ou sôbre o ombro direito.
\section{Chlamýphoro}
\begin{itemize}
\item {Grp. gram.:m.}
\end{itemize}
\begin{itemize}
\item {Proveniência:(Do gr. \textunderscore khlamus\textunderscore  + \textunderscore phoros\textunderscore )}
\end{itemize}
Gênero de fêtos.
\section{Chloasma}
\begin{itemize}
\item {Grp. gram.:m.}
\end{itemize}
\begin{itemize}
\item {Proveniência:(Gr. \textunderscore khloasma\textunderscore )}
\end{itemize}
Mancha na pelle, em resultado de doença hepáthica.
\section{Chlora}
\begin{itemize}
\item {Grp. gram.:f.}
\end{itemize}
\begin{itemize}
\item {Proveniência:(Do gr. \textunderscore khloros\textunderscore , verde amarelado)}
\end{itemize}
Gênero de plantas gencianáceas.
\section{Chlorácido}
\begin{itemize}
\item {Grp. gram.:m.}
\end{itemize}
\begin{itemize}
\item {Proveniência:(De \textunderscore chloro\textunderscore  + \textunderscore ácido\textunderscore )}
\end{itemize}
Acido, em que o chloro faz o papel de princípio acidificante.
\section{Chloral}
\begin{itemize}
\item {Grp. gram.:m.}
\end{itemize}
\begin{itemize}
\item {Proveniência:(De \textunderscore chloro\textunderscore  + \textunderscore álcool\textunderscore )}
\end{itemize}
Mistura de chloro e álcool.
\section{Chloralamida}
\begin{itemize}
\item {Grp. gram.:f.}
\end{itemize}
Medicamento hypnótico.
\section{Chloralose}
\begin{itemize}
\item {Grp. gram.:f.}
\end{itemize}
Medicamento, que provoca o somno.
\section{Chloranthia}
\begin{itemize}
\item {Grp. gram.:f.}
\end{itemize}
\begin{itemize}
\item {Proveniência:(De \textunderscore chlorantho\textunderscore )}
\end{itemize}
Degenerescência vegetal, em que os órgãos floraes apresentam a côr, a consistência e, ás vezes, a fórma das fôlhas.
\section{Chlorantho}
\begin{itemize}
\item {Grp. gram.:adj.}
\end{itemize}
\begin{itemize}
\item {Proveniência:(Do gr. \textunderscore khloros\textunderscore  + \textunderscore anthos\textunderscore )}
\end{itemize}
Que tem côr verde.
Que tem chloranthia.
\section{Chlorato}
\begin{itemize}
\item {Grp. gram.:m.}
\end{itemize}
\begin{itemize}
\item {Proveniência:(De \textunderscore chloro\textunderscore )}
\end{itemize}
Combinação do ácido chlórico com uma base.
\section{Chloretado}
\begin{itemize}
\item {Grp. gram.:adj.}
\end{itemize}
Que tem chloro ou chloreto.
\section{Chlorete}
\begin{itemize}
\item {fónica:clorê}
\end{itemize}
\begin{itemize}
\item {Grp. gram.:m.}
\end{itemize}
\begin{itemize}
\item {Utilização:Pop.}
\end{itemize}
O mesmo que \textunderscore chloreto\textunderscore .
\section{Chloreto}
\begin{itemize}
\item {fónica:clorê}
\end{itemize}
\begin{itemize}
\item {Grp. gram.:m.}
\end{itemize}
\begin{itemize}
\item {Proveniência:(De \textunderscore chloro\textunderscore )}
\end{itemize}
Combinação do chloro com um corpo simples, exceptuado o oxygênio e hydrogênio.
\section{Chlorethýlico}
\begin{itemize}
\item {Grp. gram.:adj.}
\end{itemize}
Relativo ao chlorethylo.
\section{Chlorethylização}
\begin{itemize}
\item {Grp. gram.:f.}
\end{itemize}
Acto de chlorethylizar.
\section{Chlorethylizar}
\begin{itemize}
\item {Grp. gram.:v. t.}
\end{itemize}
Anesthesiar pelo chlorethylo.
\section{Chlorethylo}
\begin{itemize}
\item {Grp. gram.:m.}
\end{itemize}
\begin{itemize}
\item {Utilização:Chím.}
\end{itemize}
Chloreto de ethyla.
\section{Chlorhydrato}
\begin{itemize}
\item {Grp. gram.:m.}
\end{itemize}
\begin{itemize}
\item {Proveniência:(De \textunderscore chlorhýdrico\textunderscore )}
\end{itemize}
Sal, formado pela combinação do ácido chlorhýdrico com uma base.
\section{Chlorhýdrico}
\begin{itemize}
\item {Grp. gram.:adj.}
\end{itemize}
\begin{itemize}
\item {Proveniência:(De \textunderscore chloro\textunderscore  e \textunderscore hydrogênio\textunderscore )}
\end{itemize}
Diz-se do ácido, composto de volumes iguaes de hydrogênio e de chloro.
\section{Chloribase}
\begin{itemize}
\item {Grp. gram.:f.}
\end{itemize}
\begin{itemize}
\item {Utilização:Chím.}
\end{itemize}
\begin{itemize}
\item {Proveniência:(De \textunderscore chloro\textunderscore  + \textunderscore base\textunderscore )}
\end{itemize}
Composto binário de chloro, que opéra como uma base.
\section{Chlórico}
\begin{itemize}
\item {Grp. gram.:adj.}
\end{itemize}
Relativo ao chloro.
\section{Chlórido}
\begin{itemize}
\item {Grp. gram.:m.}
\end{itemize}
\begin{itemize}
\item {Utilização:Fam.}
\end{itemize}
\begin{itemize}
\item {Proveniência:(De \textunderscore chloro\textunderscore )}
\end{itemize}
Combinação electronegativa de chloro com corpos metállicos ou metalloides.
Combinação de corpos simples, em que entra o chloro.
\section{Chlorino}
\begin{itemize}
\item {Grp. gram.:m.}
\end{itemize}
Mineral haloide, a que pertence o sal-gemma.
\section{Chlorístico}
\begin{itemize}
\item {Grp. gram.:adj.}
\end{itemize}
Relativo ao chloro.
\section{Chlorite}
\begin{itemize}
\item {Grp. gram.:f.}
\end{itemize}
\begin{itemize}
\item {Proveniência:(Do gr. \textunderscore khloros\textunderscore , verde)}
\end{itemize}
Mineral, de côr geralmente verde, e análogo á mica.
\section{Chlorito}
\begin{itemize}
\item {Grp. gram.:m.}
\end{itemize}
\begin{itemize}
\item {Proveniência:(De \textunderscore chloro\textunderscore )}
\end{itemize}
Sal, formado pela combinação do ácido chloroso com uma base.
\section{Chloritoso}
\begin{itemize}
\item {Grp. gram.:adj.}
\end{itemize}
Que tem chlorito.
\section{Chloro}
\begin{itemize}
\item {Grp. gram.:m.}
\end{itemize}
\begin{itemize}
\item {Proveniência:(Gr. \textunderscore khloros\textunderscore , esverdeado)}
\end{itemize}
Corpo simples gasoso, de um sabor cáustico e cheiro activo.
\section{Chloro-áurico}
\begin{itemize}
\item {Grp. gram.:adj.}
\end{itemize}
Diz-se de um ácido. Cf. F. Lapa, \textunderscore Phýs. e Chím.\textunderscore , II, 70.
\section{Chlorodyna}
\begin{itemize}
\item {Grp. gram.:f.}
\end{itemize}
Medicamento antineurálgico.
\section{Chlorofórmico}
\begin{itemize}
\item {Grp. gram.:adj.}
\end{itemize}
Relativo ao chlorofórmio.
\section{Chlorofórmio}
\begin{itemize}
\item {Grp. gram.:m.}
\end{itemize}
Substância líquida, incolor e aromática, de propriedades anesthésicas.
(Contr. de \textunderscore chlorofórmico\textunderscore )
\section{Chloroformização}
\begin{itemize}
\item {Grp. gram.:f.}
\end{itemize}
Acto de chloroformizar.
\section{Chloroformizar}
\begin{itemize}
\item {Grp. gram.:v. t.}
\end{itemize}
Ministrar chlorofórmio a.
Anesthesiar com chlorofórmio.
\section{Chlorolina}
\begin{itemize}
\item {Grp. gram.:f.}
\end{itemize}
Líquido anti-séptico e desinfectante.
\section{Chlorometria}
\begin{itemize}
\item {Grp. gram.:f.}
\end{itemize}
\begin{itemize}
\item {Proveniência:(De \textunderscore chlorómetro\textunderscore )}
\end{itemize}
Processo chímico, para determinar a quantidade de chloro, contida numa combinação.
\section{Chlorómetro}
\begin{itemize}
\item {Grp. gram.:m.}
\end{itemize}
\begin{itemize}
\item {Proveniência:(Do gr. \textunderscore khloros\textunderscore  + \textunderscore metron\textunderscore )}
\end{itemize}
Apparelho, para se praticar a chlorometria.
\section{Chloróphana}
\begin{itemize}
\item {Grp. gram.:f.}
\end{itemize}
\begin{itemize}
\item {Proveniência:(De \textunderscore chloróphano\textunderscore )}
\end{itemize}
Variedade de fluorina da Sibéria, de côr violácea, e que, depois de aquecida, se torna phosphorescente, de uma bella côr verde.
\section{Chloróphano}
\begin{itemize}
\item {Grp. gram.:adj.}
\end{itemize}
\begin{itemize}
\item {Grp. gram.:M.}
\end{itemize}
\begin{itemize}
\item {Proveniência:(Do gr. \textunderscore khloros\textunderscore  + \textunderscore phainein\textunderscore )}
\end{itemize}
Que parece verde.
Gênero de coleópteros de suave côr verde.
\section{Chlorophylla}
\begin{itemize}
\item {Grp. gram.:f.}
\end{itemize}
\begin{itemize}
\item {Proveniência:(Do gr. \textunderscore khloros\textunderscore  + \textunderscore phullon\textunderscore )}
\end{itemize}
Substância, que existe nas céllulas vegetaes e que dá ás plantas a côr verde.
\section{Chlorophylleano}
\begin{itemize}
\item {Grp. gram.:adj.}
\end{itemize}
Relativo á chloróphylla.
\section{Chlorophyllino}
\begin{itemize}
\item {Grp. gram.:adj.}
\end{itemize}
Relativo á chloróphylla.
\section{Chloróphyto}
\begin{itemize}
\item {Grp. gram.:m.}
\end{itemize}
\begin{itemize}
\item {Proveniência:(Do gr. \textunderscore khloros\textunderscore  + \textunderscore phuton\textunderscore )}
\end{itemize}
Gênero de plantas liliáceas.
\section{Chlorose}
\begin{itemize}
\item {Grp. gram.:m.}
\end{itemize}
\begin{itemize}
\item {Proveniência:(Do gr. \textunderscore khloros\textunderscore )}
\end{itemize}
Doença do sexo feminino, determinada geralmente pela ausência do catamênio, e caracterizada por fraqueza e pallidez.
Definhamento das plantas.
\section{Chloroso}
\begin{itemize}
\item {Grp. gram.:adj.}
\end{itemize}
\begin{itemize}
\item {Proveniência:(De \textunderscore chloro\textunderscore )}
\end{itemize}
Diz-se de um ácido, corpo gasoso, solúvel na água, e de cheiro análogo ao do chloro.
\section{Chlorótico}
\begin{itemize}
\item {Grp. gram.:adj.}
\end{itemize}
Que padece chlorose.
Relativo á chlorose.
\section{Chlorureto}
\begin{itemize}
\item {fónica:êto}
\end{itemize}
\begin{itemize}
\item {Grp. gram.:m.}
\end{itemize}
(V.chloreto)
\section{Chó}
\begin{itemize}
\item {Grp. gram.:m.}
\end{itemize}
Grande peixe africano.
\section{Choanoide}
\begin{itemize}
\item {fónica:co}
\end{itemize}
\begin{itemize}
\item {Grp. gram.:adj.}
\end{itemize}
\begin{itemize}
\item {Proveniência:(Do gr. \textunderscore khoane\textunderscore  + \textunderscore eidos\textunderscore )}
\end{itemize}
Que tem fórma de funil.
\section{Chôas}
\begin{itemize}
\item {Grp. gram.:m. pl.}
\end{itemize}
Tríbo da Abyssínia.
\section{Chobia}
\begin{itemize}
\item {Grp. gram.:f.}
\end{itemize}
Ave corvídea da África occidental.
\section{Choca}
\begin{itemize}
\item {Grp. gram.:f.}
\end{itemize}
\begin{itemize}
\item {Proveniência:(Do ár. \textunderscore jocan\textunderscore )}
\end{itemize}
Pau, de que os rapazes se servem no jôgo da bola.
A bola dêsse jôgo.
\section{Choca}
\begin{itemize}
\item {Grp. gram.:f.}
\end{itemize}
\begin{itemize}
\item {Proveniência:(Do b. lat. \textunderscore cloca\textunderscore )}
\end{itemize}
Chocalho grande.
Vaca que guia os toiros.
\section{Choca}
\begin{itemize}
\item {Grp. gram.:f.}
\end{itemize}
\begin{itemize}
\item {Utilização:Fam.}
\end{itemize}
Mancha de lama na barra de um vestido.
\section{Choca}
\begin{itemize}
\item {Utilização:ant.}
\end{itemize}
O mesmo que \textunderscore namôro\textunderscore .
\section{Choça}
\begin{itemize}
\item {Grp. gram.:f.}
\end{itemize}
\begin{itemize}
\item {Utilização:Prov.}
\end{itemize}
Raíz queimada e carbonizada de urze ou torga.
\section{Choça}
\begin{itemize}
\item {Grp. gram.:f.}
\end{itemize}
\begin{itemize}
\item {Proveniência:(Do lat. \textunderscore plutea\textunderscore )}
\end{itemize}
Choupana.
Casa rústica, humilde.
\section{Chocadeira}
\begin{itemize}
\item {Grp. gram.:f.}
\end{itemize}
Apparelho, para chocar ovos.
\section{Chocagem}
\begin{itemize}
\item {Grp. gram.:f.}
\end{itemize}
Acto ou effeito de chocar. Cf. Ortigão, \textunderscore Hollanda\textunderscore , 110.
\section{Chocalhada}
\begin{itemize}
\item {Grp. gram.:f.}
\end{itemize}
Som de chocalhos.
Acção de chocalhar.
\section{Chocalhar}
\begin{itemize}
\item {Grp. gram.:v. t.}
\end{itemize}
\begin{itemize}
\item {Grp. gram.:V. i.}
\end{itemize}
\begin{itemize}
\item {Utilização:Fig.}
\end{itemize}
\begin{itemize}
\item {Proveniência:(De \textunderscore chocalho\textunderscore )}
\end{itemize}
Vascolejar, agitar, produzindo som semelhante ao do chocalho.
Agitar dentro de um vaso ou de uma caixa.
Tocar, agitar, chocalhos.
Acompanhar ao som de chocalhos.
Dar gargalhadas.
Divulgar segredos.
\section{Chocalheira}
\begin{itemize}
\item {Grp. gram.:f.  e  adj.}
\end{itemize}
\begin{itemize}
\item {Grp. gram.:F.}
\end{itemize}
\begin{itemize}
\item {Utilização:Gír.}
\end{itemize}
\begin{itemize}
\item {Utilização:Bot.}
\end{itemize}
Mulher mexeriqueira, indiscreta.
Secretária ou mesa com dinheiro na gaveta.
Planta gramínea, (\textunderscore briza maxima\textunderscore , Lin.).
\section{Chocalheirinha}
\begin{itemize}
\item {Grp. gram.:f.}
\end{itemize}
\begin{itemize}
\item {Utilização:Bot.}
\end{itemize}
O mesmo que \textunderscore pandeirinha\textunderscore .
\section{Chocalheiro}
\begin{itemize}
\item {Grp. gram.:adj.}
\end{itemize}
\begin{itemize}
\item {Utilização:Fig.}
\end{itemize}
\begin{itemize}
\item {Grp. gram.:M.}
\end{itemize}
\begin{itemize}
\item {Proveniência:(De \textunderscore chocalhar\textunderscore )}
\end{itemize}
Que chocalha.
Que traz chocalho.
Mexeriqueiro, que revela indiscretamente o que ouviu.
Aquelle que mexerica.
Aquelle que fala muito e indiscretamente.
\section{Chocalhice}
\begin{itemize}
\item {Grp. gram.:f.}
\end{itemize}
\begin{itemize}
\item {Proveniência:(De \textunderscore chocalhar\textunderscore )}
\end{itemize}
Qualidade de quem é chocalheiro, mexeriqueiro e indiscretamente falador.
\section{Chocalho}
\begin{itemize}
\item {Grp. gram.:m.}
\end{itemize}
\begin{itemize}
\item {Utilização:Pop.}
\end{itemize}
\begin{itemize}
\item {Proveniência:(De \textunderscore choca\textunderscore ^2)}
\end{itemize}
Instrumento de metal, com badalo, e mais ou menos semelhante a uma campaínha, para se pôr ao pescoço de animaes.
Cabaça que, com pedras dentro, se agita, produzindo som semelhante ao do chocalho.
Pessôa chocalheira.
\section{Chocar}
\begin{itemize}
\item {Grp. gram.:v. i.}
\end{itemize}
\begin{itemize}
\item {Grp. gram.:V. t.}
\end{itemize}
\begin{itemize}
\item {Grp. gram.:V. p.}
\end{itemize}
\begin{itemize}
\item {Proveniência:(De \textunderscore choque\textunderscore )}
\end{itemize}
Dar choque.
Ir de encontro.
Melindrar, offender.
Diz-se de coisas, que esbarram reciprocamente ou batem de encontro, umas contra outras.
\section{Chocar}
\begin{itemize}
\item {Grp. gram.:v. t.  e  i.}
\end{itemize}
\begin{itemize}
\item {Utilização:Fig.}
\end{itemize}
\begin{itemize}
\item {Utilização:Gír.}
\end{itemize}
Cobrir e aquecer ovos, para lhes desenvolver o germe; incubar.
Planear; preparar em segredo.
Apodrecer.
\textunderscore Chocar os ovos\textunderscore , preparar o roubo.
(Cast. \textunderscore cloquear\textunderscore )
\section{Chocarrear}
\begin{itemize}
\item {Grp. gram.:v. i.}
\end{itemize}
\begin{itemize}
\item {Proveniência:(De \textunderscore chocarreiro\textunderscore )}
\end{itemize}
Dizer chocarrices.
\section{Chocarreiramente}
\begin{itemize}
\item {Grp. gram.:adv.}
\end{itemize}
Á maneira de chocarreiro.
\section{Chocarreiro}
\begin{itemize}
\item {Grp. gram.:m.  e  adj.}
\end{itemize}
O que diz chocarrices.
Em que há chocarrice.
\section{Chocarrice}
\begin{itemize}
\item {Grp. gram.:f.}
\end{itemize}
\begin{itemize}
\item {Proveniência:(De \textunderscore chocarrear\textunderscore )}
\end{itemize}
Chalaça grosseira.
Truanice; gracejo atrevido.
\section{Chocas}
\begin{itemize}
\item {Grp. gram.:f. pl.}
\end{itemize}
\begin{itemize}
\item {Utilização:Prov.}
\end{itemize}
\begin{itemize}
\item {Utilização:alg.}
\end{itemize}
O mesmo que \textunderscore chalocas\textunderscore .
\section{Chocha}
\begin{itemize}
\item {fónica:chô}
\end{itemize}
\begin{itemize}
\item {Grp. gram.:f.}
\end{itemize}
\begin{itemize}
\item {Utilização:alg.}
\end{itemize}
\begin{itemize}
\item {Utilização:Gír.}
\end{itemize}
Partes pudendas da mulher.
\section{Chochice}
\begin{itemize}
\item {Grp. gram.:f.}
\end{itemize}
Qualidade de chocho.
Insipidez.
Insignificância. Cf. Júl. Diniz, \textunderscore Morgadinha\textunderscore , 162 e 163.
\section{Chòchim}
\begin{itemize}
\item {Grp. gram.:m.}
\end{itemize}
O mesmo que \textunderscore chòchinha\textunderscore .
\section{Chòchinha}
\begin{itemize}
\item {Grp. gram.:m.  e  f.}
\end{itemize}
\begin{itemize}
\item {Proveniência:(De \textunderscore chocho\textunderscore )}
\end{itemize}
Pessôa pequena e magra.
Palerma.
\section{Chocho}
\begin{itemize}
\item {fónica:chô}
\end{itemize}
\begin{itemize}
\item {Grp. gram.:adj.}
\end{itemize}
\begin{itemize}
\item {Utilização:Fig.}
\end{itemize}
\begin{itemize}
\item {Proveniência:(Do lat. \textunderscore suctus\textunderscore )}
\end{itemize}
Que não tem suco.
Sem miolo.
Sêco.
Gôro (falando-se do ovo).
Oco.
Fútil.
Tolo.
Sem préstimo.
Enfraquecido, doente.
\section{Chocho}
\begin{itemize}
\item {fónica:chô}
\end{itemize}
\begin{itemize}
\item {Grp. gram.:m.}
\end{itemize}
\begin{itemize}
\item {Utilização:Pop.}
\end{itemize}
\begin{itemize}
\item {Proveniência:(T. onom.)}
\end{itemize}
Beijoca.
\section{Chocho}
\begin{itemize}
\item {fónica:chô}
\end{itemize}
\begin{itemize}
\item {Grp. gram.:m.}
\end{itemize}
\begin{itemize}
\item {Utilização:Prov.}
\end{itemize}
\begin{itemize}
\item {Utilização:trasm.}
\end{itemize}
O mesmo que \textunderscore tremoço\textunderscore .
\section{Chochô}
\begin{itemize}
\item {Grp. gram.:m.}
\end{itemize}
\begin{itemize}
\item {Utilização:Bras}
\end{itemize}
Banha, que os pretos fazem com leite de coco.
\section{Chôco}
\begin{itemize}
\item {Grp. gram.:m.}
\end{itemize}
Peixe, o mesmo que \textunderscore siba\textunderscore .
\section{Chôco}
\begin{itemize}
\item {Grp. gram.:adj.}
\end{itemize}
\begin{itemize}
\item {Utilização:Fig.}
\end{itemize}
\begin{itemize}
\item {Grp. gram.:M.}
\end{itemize}
\begin{itemize}
\item {Utilização:Pop.}
\end{itemize}
\begin{itemize}
\item {Proveniência:(De \textunderscore chocar\textunderscore ^2)}
\end{itemize}
Diz-se do ovo, em que se está desenvolvendo o germe.
E diz-se da gallinha, que está incubando.
Podre; estragado; gôro.
Incubação; acto de chocar (ovos): \textunderscore a gallinha está no chôco\textunderscore .
\textunderscore Estar de chôco\textunderscore , estar de cama.
\section{Choço}
\begin{itemize}
\item {fónica:chô}
\end{itemize}
\begin{itemize}
\item {Grp. gram.:m.}
\end{itemize}
\begin{itemize}
\item {Utilização:Prov.}
\end{itemize}
\begin{itemize}
\item {Utilização:alent.}
\end{itemize}
Alpendrada, onde se abrigam os porcos, que se cevam com sobejos de comidas.
(Cp. \textunderscore choça\textunderscore ^2)
\section{Chocolataria}
\begin{itemize}
\item {Grp. gram.:f.}
\end{itemize}
Lugar, em que se fabríca a pasta do chocolate, ou onde se vende a bebida de chocolate.
\section{Chocolate}
\begin{itemize}
\item {Grp. gram.:m.}
\end{itemize}
\begin{itemize}
\item {Proveniência:(Do mex. \textunderscore chocolatl\textunderscore )}
\end{itemize}
Pasta alimentar, feita de cacau, açúcar e várias substâncias aromáticas.
Bebida, preparada com essa pasta.
\section{Chocolateira}
\begin{itemize}
\item {Grp. gram.:f.}
\end{itemize}
\begin{itemize}
\item {Utilização:Ext.}
\end{itemize}
\begin{itemize}
\item {Proveniência:(De \textunderscore chocolate\textunderscore )}
\end{itemize}
Vaso, em que se prepara a bebida do chocolate.
Vaso de metal, em que se aquece água; cafeteira; chaleira.
\section{Chocolateiro}
\begin{itemize}
\item {Grp. gram.:m.}
\end{itemize}
\begin{itemize}
\item {Utilização:Neol.}
\end{itemize}
\begin{itemize}
\item {Proveniência:(De \textunderscore chocolate\textunderscore )}
\end{itemize}
Aquelle que faz ou vende chocolate.
Negociante ou cultivador de cacau.
\section{Chocolejar}
\begin{itemize}
\item {Grp. gram.:v. i.}
\end{itemize}
\begin{itemize}
\item {Utilização:Prov.}
\end{itemize}
\begin{itemize}
\item {Utilização:trasm.}
\end{itemize}
\begin{itemize}
\item {Grp. gram.:V. t.}
\end{itemize}
Não estar seguro, abanar: \textunderscore a ferradura chocoleja\textunderscore .
Agitar, vascolejar.
\section{Chocrão}
\begin{itemize}
\item {Grp. gram.:m.}
\end{itemize}
\begin{itemize}
\item {Utilização:Ant.}
\end{itemize}
Fanão de oiro baixo, na Índia portuguesa.
\section{Chodene}
\begin{itemize}
\item {Grp. gram.:m.}
\end{itemize}
Antiga medida de Cochim, para azeite e manteiga.
\section{Chodó}
\begin{itemize}
\item {Grp. gram.:m.}
\end{itemize}
\begin{itemize}
\item {Utilização:Bras. do N}
\end{itemize}
Namôro descarado.
Baile de gente ordinária.
\section{Choéphora}
\begin{itemize}
\item {fónica:co}
\end{itemize}
\begin{itemize}
\item {Grp. gram.:f.}
\end{itemize}
\begin{itemize}
\item {Proveniência:(Do gr. \textunderscore khoe\textunderscore  + \textunderscore phoros\textunderscore )}
\end{itemize}
Mulher, que, entre os antigos Gregos, levava offerendas, destinadas aos mortos.
\section{Chofrada}
\begin{itemize}
\item {Grp. gram.:f.}
\end{itemize}
\begin{itemize}
\item {Proveniência:(De \textunderscore chofrar\textunderscore )}
\end{itemize}
Pancada de chofre.
\section{Chofrado}
\begin{itemize}
\item {Grp. gram.:adj.}
\end{itemize}
\begin{itemize}
\item {Proveniência:(De \textunderscore chofrar\textunderscore )}
\end{itemize}
Estimulado, escandalizado.
\section{Chofrão}
\begin{itemize}
\item {Grp. gram.:m.}
\end{itemize}
Abutre do Egypto.
\section{Chofrar}
\begin{itemize}
\item {Grp. gram.:v. t.}
\end{itemize}
\begin{itemize}
\item {Utilização:Fig.}
\end{itemize}
\begin{itemize}
\item {Grp. gram.:V. i.}
\end{itemize}
\begin{itemize}
\item {Utilização:Bras}
\end{itemize}
Dar de chofre em.
Ferir de súbito.
Vexar com um motejo imprevisto.
Atirar de chofre, (falando-se do caçador).
Rumorejar.
\section{Chofre}
\begin{itemize}
\item {Grp. gram.:m.}
\end{itemize}
\begin{itemize}
\item {Grp. gram.:Loc. adv.}
\end{itemize}
Choque repentino.
Tiro contra a ave, que se levanta.
Pancada de taco na bola de bilhar.
\textunderscore De chofre\textunderscore , repentinamente.
\section{Chofreiro}
\begin{itemize}
\item {Grp. gram.:m.  e  adj.}
\end{itemize}
\begin{itemize}
\item {Utilização:Des.}
\end{itemize}
Aquelle que procede de chofre.
\section{Chofrudo}
\begin{itemize}
\item {Grp. gram.:adj.}
\end{itemize}
(V.chofreiro)
\section{Chogó}
\begin{itemize}
\item {Grp. gram.:m.}
\end{itemize}
Espécie de vestuário gentílico, na Índia portuguesa. Cf. Th. Ribeiro, \textunderscore Jornadas\textunderscore , 101.
\section{Choina}
\begin{itemize}
\item {Grp. gram.:f.}
\end{itemize}
\begin{itemize}
\item {Utilização:Prov.}
\end{itemize}
\begin{itemize}
\item {Utilização:trasm.}
\end{itemize}
Faúlha; chispa; centelha.
\section{Choina}
\begin{itemize}
\item {Grp. gram.:m.  e  f.}
\end{itemize}
O mesmo que \textunderscore chona\textunderscore .
\section{Choinar}
\begin{itemize}
\item {Grp. gram.:v. i.}
\end{itemize}
O mesmo que \textunderscore chonar\textunderscore .
\section{Choisinha}
\begin{itemize}
\item {Grp. gram.:m.}
\end{itemize}
\begin{itemize}
\item {Utilização:Prov.}
\end{itemize}
\begin{itemize}
\item {Utilização:trasm.}
\end{itemize}
Parvo; bisonho; bacoco.
(Alter. de \textunderscore chochinha\textunderscore )
\section{Chola}
\begin{itemize}
\item {Grp. gram.:f.}
\end{itemize}
\begin{itemize}
\item {Utilização:Pop.}
\end{itemize}
\begin{itemize}
\item {Proveniência:(T. cast.)}
\end{itemize}
Cabeça.
\section{Cholá}
\begin{itemize}
\item {Grp. gram.:m.}
\end{itemize}
Planta leguminosa de Dio.
\section{Cholagogo}
\begin{itemize}
\item {fónica:co}
\end{itemize}
\begin{itemize}
\item {Grp. gram.:adj.}
\end{itemize}
\begin{itemize}
\item {Grp. gram.:M.}
\end{itemize}
\begin{itemize}
\item {Proveniência:(Do gr. \textunderscore khole\textunderscore  + \textunderscore agein\textunderscore )}
\end{itemize}
Que faz segregar a bílis do fígado.
Que actua sôbre o apparelho biliário.
Medicamento cholagogo.
\section{Choldra}
\begin{itemize}
\item {Grp. gram.:f.}
\end{itemize}
\begin{itemize}
\item {Utilização:Pop.}
\end{itemize}
\begin{itemize}
\item {Utilização:Prov.}
\end{itemize}
\begin{itemize}
\item {Utilização:trasm.}
\end{itemize}
Canalha; ralé.
Confusão de gente ordinária.
Salgalhada.
Caldo mal feito ou mal temperado.
\section{Choldraboldra}
\begin{itemize}
\item {Grp. gram.:f.}
\end{itemize}
\begin{itemize}
\item {Utilização:Pop.}
\end{itemize}
O mesmo que \textunderscore choldra\textunderscore .
\section{Cholédoco}
\begin{itemize}
\item {fónica:co}
\end{itemize}
\begin{itemize}
\item {Grp. gram.:adj.}
\end{itemize}
\begin{itemize}
\item {Proveniência:(Gr. \textunderscore kholedokos\textunderscore )}
\end{itemize}
Diz-se do canal, que leva a bílis ao duodeno.
\section{Cholélitho}
\begin{itemize}
\item {fónica:co}
\end{itemize}
\begin{itemize}
\item {Grp. gram.:m.}
\end{itemize}
\begin{itemize}
\item {Utilização:Med.}
\end{itemize}
\begin{itemize}
\item {Proveniência:(Do gr. \textunderscore khole\textunderscore  + \textunderscore lithos\textunderscore )}
\end{itemize}
Cálculo biliário.
\section{Cholelogia}
\begin{itemize}
\item {fónica:co}
\end{itemize}
\begin{itemize}
\item {Grp. gram.:f.}
\end{itemize}
\begin{itemize}
\item {Proveniência:(Do gr. \textunderscore khole\textunderscore  + \textunderscore logos\textunderscore )}
\end{itemize}
Tratado á cêrca da bílis.
\section{Chólera}
\begin{itemize}
\item {fónica:có}
\end{itemize}
\begin{itemize}
\item {Grp. gram.:f.}
\end{itemize}
\begin{itemize}
\item {Proveniência:(Lat. \textunderscore cholera\textunderscore , gr. \textunderscore kholera\textunderscore , goteira)}
\end{itemize}
Doença, caracterizada por grandes evacuações, fraqueza e resfriamento.
O mesmo que \textunderscore cólera\textunderscore .(V.cólera)
\section{Chólera-mórbus}
\begin{itemize}
\item {Grp. gram.:f.}
\end{itemize}
O mesmo que \textunderscore chólera\textunderscore , doença.
\section{Cholérico}
\begin{itemize}
\item {fónica:co}
\end{itemize}
\begin{itemize}
\item {Grp. gram.:adj.}
\end{itemize}
\begin{itemize}
\item {Grp. gram.:M.}
\end{itemize}
Relativo á \textunderscore chólera\textunderscore .
Pessôa atacada de chólera.(V.colérico)
\section{Choleriforme}
\begin{itemize}
\item {fónica:co}
\end{itemize}
\begin{itemize}
\item {Grp. gram.:adj.}
\end{itemize}
\begin{itemize}
\item {Proveniência:(Do lat. \textunderscore cholera\textunderscore  + \textunderscore forma\textunderscore )}
\end{itemize}
Semelhante á chólera.
\section{Cholerígeno}
\begin{itemize}
\item {fónica:co}
\end{itemize}
\begin{itemize}
\item {Grp. gram.:adj.}
\end{itemize}
\begin{itemize}
\item {Proveniência:(Do gr. \textunderscore kholera\textunderscore  + \textunderscore genos\textunderscore )}
\end{itemize}
Que produz chólera-mórbo.
\section{Cholerina}
\begin{itemize}
\item {fónica:co}
\end{itemize}
\begin{itemize}
\item {Grp. gram.:f.}
\end{itemize}
Chólera benigna, attenuada.
\section{Cholerínico}
\begin{itemize}
\item {fónica:co}
\end{itemize}
\begin{itemize}
\item {Grp. gram.:adj.}
\end{itemize}
\begin{itemize}
\item {Grp. gram.:M.}
\end{itemize}
Relativo á cholerina.
Aquelle que padece cholerina.
\section{Cholerogênico}
\begin{itemize}
\item {fónica:co}
\end{itemize}
\begin{itemize}
\item {Grp. gram.:adj.}
\end{itemize}
O mesmo que \textunderscore cholerígeno\textunderscore .
\section{Cholestearina}
\begin{itemize}
\item {fónica:co}
\end{itemize}
\begin{itemize}
\item {Grp. gram.:f.}
\end{itemize}
\begin{itemize}
\item {Proveniência:(Do gr. \textunderscore khole\textunderscore  + \textunderscore stear\textunderscore )}
\end{itemize}
A gordura da bílis.
\section{Cholesteatoma}
\begin{itemize}
\item {Grp. gram.:m.}
\end{itemize}
\begin{itemize}
\item {Utilização:Med.}
\end{itemize}
\begin{itemize}
\item {Proveniência:(Do gr. \textunderscore khole\textunderscore  + \textunderscore stedo\textunderscore )}
\end{itemize}
Lipoma, formado pela sobreposição de vesículas adiposas, entre as quaes há uma substância, composta de cholesterina e estearina.
\section{Cholesterato}
\begin{itemize}
\item {fónica:co}
\end{itemize}
\begin{itemize}
\item {Grp. gram.:m.}
\end{itemize}
\begin{itemize}
\item {Proveniência:(De \textunderscore cholestérico\textunderscore )}
\end{itemize}
Gênero de saes, formados pelo ácido cholestérico.
\section{Cholestérico}
\begin{itemize}
\item {fónica:co}
\end{itemize}
\begin{itemize}
\item {Grp. gram.:adj.}
\end{itemize}
Diz-se de um ácido, formado pela reacção do ácido azótico sôbre a cholesterina.
(Cp. \textunderscore cholesterina\textunderscore )
\section{Cholesterina}
\begin{itemize}
\item {fónica:co}
\end{itemize}
\begin{itemize}
\item {Grp. gram.:f.}
\end{itemize}
\begin{itemize}
\item {Proveniência:(Do gr. \textunderscore khole\textunderscore  + \textunderscore steao\textunderscore )}
\end{itemize}
Substância crystallizada dos cálculos biliários humanos.
\section{Choli}
\begin{itemize}
\item {Grp. gram.:m.}
\end{itemize}
Espécie de saia, usada por bailadeiras indianas. Cf. Th. Ribeiro, \textunderscore Jornadas\textunderscore , II, 104.
\section{Cholihemia}
\begin{itemize}
\item {fónica:co}
\end{itemize}
\begin{itemize}
\item {Grp. gram.:f.}
\end{itemize}
\begin{itemize}
\item {Proveniência:(Do gr. \textunderscore khole\textunderscore  + \textunderscore haima\textunderscore )}
\end{itemize}
Penetração da bílis no sangue.
\section{Choluria}
\begin{itemize}
\item {Grp. gram.:f.}
\end{itemize}
\begin{itemize}
\item {Proveniência:(Do gr. \textunderscore khole\textunderscore  + \textunderscore ouron\textunderscore )}
\end{itemize}
Passagem, pela urina, dos princípios corantes da bílis.
\section{Chomélia}
\begin{itemize}
\item {Grp. gram.:f.}
\end{itemize}
\begin{itemize}
\item {Proveniência:(De \textunderscore Chomel\textunderscore , n. p.)}
\end{itemize}
Planta rubiácea.
\section{Chona}
\begin{itemize}
\item {Grp. gram.:f.}
\end{itemize}
\begin{itemize}
\item {Utilização:Gír.}
\end{itemize}
\begin{itemize}
\item {Proveniência:(De \textunderscore chonar\textunderscore )}
\end{itemize}
Noite.
Homem que está dormindo.
\section{Chonar}
\begin{itemize}
\item {Grp. gram.:v. i.}
\end{itemize}
\begin{itemize}
\item {Utilização:Gír.}
\end{itemize}
Dormir.
(Por \textunderscore somnar\textunderscore , de \textunderscore somno\textunderscore )
\section{Chonas}
\begin{itemize}
\item {Grp. gram.:f. pl.}
\end{itemize}
\begin{itemize}
\item {Utilização:Fam.}
\end{itemize}
Lábia, paleio, cantigas.
\section{Chondracor}
\begin{itemize}
\item {Grp. gram.:m.}
\end{itemize}
Espécie de toucado das bailadeiras indianas.
\section{Chondral}
\begin{itemize}
\item {fónica:con}
\end{itemize}
\begin{itemize}
\item {Grp. gram.:adj.}
\end{itemize}
Relativo ao chondro.
\section{Chondrificação}
\begin{itemize}
\item {fónica:con}
\end{itemize}
\begin{itemize}
\item {Grp. gram.:f.}
\end{itemize}
Acto ou effeito de chondrificar-se.
\section{Chondrificar-se}
\begin{itemize}
\item {fónica:con}
\end{itemize}
\begin{itemize}
\item {Grp. gram.:v. p.}
\end{itemize}
\begin{itemize}
\item {Proveniência:(T. hybr., do gr. \textunderscore khondros\textunderscore  + lat. \textunderscore facere\textunderscore )}
\end{itemize}
Tornar-se cartilaginoso.
\section{Chondrilha}
\begin{itemize}
\item {fónica:con}
\end{itemize}
\begin{itemize}
\item {Grp. gram.:f.}
\end{itemize}
\begin{itemize}
\item {Proveniência:(Do gr. \textunderscore khondros\textunderscore , grão)}
\end{itemize}
Gênero de plantas chicoriáceas.
\section{Chondrina}
\begin{itemize}
\item {fónica:con}
\end{itemize}
\begin{itemize}
\item {Grp. gram.:f.}
\end{itemize}
\begin{itemize}
\item {Proveniência:(Do gr. \textunderscore khondros\textunderscore )}
\end{itemize}
Substância, que se extrai de certas cartilagens.
\section{Chondro}
\begin{itemize}
\item {fónica:con}
\end{itemize}
\begin{itemize}
\item {Grp. gram.:m.}
\end{itemize}
\begin{itemize}
\item {Proveniência:(Gr. \textunderscore khondros\textunderscore )}
\end{itemize}
Designação scientífica da cartilagem.
\section{Chondrographia}
\begin{itemize}
\item {fónica:con}
\end{itemize}
\begin{itemize}
\item {Grp. gram.:f.}
\end{itemize}
\begin{itemize}
\item {Proveniência:(Do gr. \textunderscore khondros\textunderscore  + \textunderscore graphein\textunderscore )}
\end{itemize}
Descripção das cartilagens.
\section{Chondroide}
\begin{itemize}
\item {fónica:con}
\end{itemize}
\begin{itemize}
\item {Grp. gram.:adj.}
\end{itemize}
\begin{itemize}
\item {Proveniência:(Do gr. \textunderscore khondros\textunderscore  + \textunderscore eidos\textunderscore )}
\end{itemize}
Semelhante a cartilagens.
\section{Chondrologia}
\begin{itemize}
\item {fónica:con}
\end{itemize}
\begin{itemize}
\item {Grp. gram.:f.}
\end{itemize}
\begin{itemize}
\item {Proveniência:(Do gr. \textunderscore khondros\textunderscore  + \textunderscore logos\textunderscore )}
\end{itemize}
Tratado das cartilagens.
\section{Chondroma}
\begin{itemize}
\item {fónica:con}
\end{itemize}
\begin{itemize}
\item {Grp. gram.:m.}
\end{itemize}
\begin{itemize}
\item {Proveniência:(Do gr. \textunderscore khondros\textunderscore )}
\end{itemize}
Tumor cartilaginoso.
\section{Chondropterýgios}
\begin{itemize}
\item {fónica:con}
\end{itemize}
\begin{itemize}
\item {Grp. gram.:m. pl.}
\end{itemize}
\begin{itemize}
\item {Proveniência:(Do gr. \textunderscore khondros\textunderscore  + \textunderscore pterux\textunderscore )}
\end{itemize}
Peixes, caracterizados por terem esqueleto cartilaginoso.
\section{Chondrotomia}
\begin{itemize}
\item {fónica:con}
\end{itemize}
\begin{itemize}
\item {Grp. gram.:f.}
\end{itemize}
\begin{itemize}
\item {Proveniência:(Do gr. \textunderscore khondros\textunderscore  + \textunderscore tome\textunderscore )}
\end{itemize}
Dissecação das cartilagens.
\section{Choninha}
\begin{itemize}
\item {fónica:chó}
\end{itemize}
\begin{itemize}
\item {Grp. gram.:m.  e  f.}
\end{itemize}
\begin{itemize}
\item {Utilização:Prov.}
\end{itemize}
\begin{itemize}
\item {Utilização:dur.}
\end{itemize}
Pessôa magra, enfezada.
Pessôa inútil.
\section{Choninhas}
\begin{itemize}
\item {fónica:chó}
\end{itemize}
\begin{itemize}
\item {Grp. gram.:m.  e  f.}
\end{itemize}
\begin{itemize}
\item {Utilização:Prov.}
\end{itemize}
\begin{itemize}
\item {Utilização:dur.}
\end{itemize}
Pessôa magra, enfezada.
Pessôa inútil.
\section{Chopim}
\begin{itemize}
\item {Grp. gram.:m.}
\end{itemize}
\begin{itemize}
\item {Utilização:Bras}
\end{itemize}
Nome, que, na Foz-do-Doiro, se dá ao tentilhão.
Pássaro, notável pelo seu canto.
\section{Choque}
\begin{itemize}
\item {Grp. gram.:m.}
\end{itemize}
Embate violento de dois corpos.
Encontrão.
Commoção; grande impressão moral.
Encontro violento de fôrças militares.
Luta.
(Talvez do fr. \textunderscore choc\textunderscore )
\section{Choqueiro}
\begin{itemize}
\item {Grp. gram.:m.}
\end{itemize}
\begin{itemize}
\item {Proveniência:(De \textunderscore chocar\textunderscore )}
\end{itemize}
Lugar, onde as gallinhas chocam os ovos.
\section{Choquel}
\begin{itemize}
\item {Grp. gram.:m.}
\end{itemize}
\begin{itemize}
\item {Utilização:Ant.}
\end{itemize}
Frete do cravo de Maluco, pago em gêneros.
\section{Choquento}
\begin{itemize}
\item {Grp. gram.:adj.}
\end{itemize}
\begin{itemize}
\item {Proveniência:(De \textunderscore choco\textunderscore ^2)}
\end{itemize}
Choco.
Enfraquecido, adoentado.
\section{Choquento}
\begin{itemize}
\item {Grp. gram.:adj.}
\end{itemize}
\begin{itemize}
\item {Proveniência:(De \textunderscore choca\textunderscore ^3)}
\end{itemize}
Que tem chocas, lama; sujo.
\section{Choquice}
\begin{itemize}
\item {Grp. gram.:f.}
\end{itemize}
\begin{itemize}
\item {Utilização:Prov.}
\end{itemize}
\begin{itemize}
\item {Utilização:Fig.}
\end{itemize}
Estada no chôco.
Torpor de ânimo; abatimento.
\section{Choquilha}
\begin{itemize}
\item {Grp. gram.:f.}
\end{itemize}
\begin{itemize}
\item {Utilização:Prov.}
\end{itemize}
\begin{itemize}
\item {Utilização:alg.}
\end{itemize}
O mesmo que \textunderscore chocalho\textunderscore :«\textunderscore repicando a choquilha\textunderscore ». J. de Deus.
\section{Chora}
\begin{itemize}
\item {Grp. gram.:f.}
\end{itemize}
\begin{itemize}
\item {Utilização:Prov.}
\end{itemize}
\begin{itemize}
\item {Utilização:trasm.}
\end{itemize}
Acto de chorar.
\section{Choradamente}
\begin{itemize}
\item {Grp. gram.:adv.}
\end{itemize}
\begin{itemize}
\item {Proveniência:(De \textunderscore chorar\textunderscore )}
\end{itemize}
De modo choroso.
\section{Choradeira}
\begin{itemize}
\item {Grp. gram.:f.}
\end{itemize}
\begin{itemize}
\item {Utilização:Fam.}
\end{itemize}
\begin{itemize}
\item {Utilização:Prov.}
\end{itemize}
\begin{itemize}
\item {Utilização:Prov.}
\end{itemize}
Carpideira.
Acção de chorar muito e impertinentemente.
Pedido lamuriento.
O mesmo que \textunderscore abibe\textunderscore .
Planta, também conhecida por \textunderscore orvalheira\textunderscore .
\section{Choradinho}
\begin{itemize}
\item {Grp. gram.:m.}
\end{itemize}
\begin{itemize}
\item {Utilização:Bras}
\end{itemize}
\begin{itemize}
\item {Proveniência:(De \textunderscore chorar\textunderscore )}
\end{itemize}
Toada musical, espécie de fado.
Bailado popular.
\section{Chorado}
\begin{itemize}
\item {Grp. gram.:m.}
\end{itemize}
\begin{itemize}
\item {Utilização:Bras. do N}
\end{itemize}
Espécie de baile popular.
\section{Choradoiro}
\begin{itemize}
\item {Grp. gram.:m.}
\end{itemize}
\begin{itemize}
\item {Utilização:Agr.}
\end{itemize}
Fio de água, que escorre das represas. Cf. Assis, \textunderscore Águas\textunderscore , 169.
\section{Chorador}
\begin{itemize}
\item {Grp. gram.:m.  e  adj.}
\end{itemize}
\begin{itemize}
\item {Utilização:Des.}
\end{itemize}
\begin{itemize}
\item {Proveniência:(De \textunderscore chorar\textunderscore )}
\end{itemize}
Que chora facilmente.
\section{Choradouro}
\begin{itemize}
\item {Grp. gram.:m.}
\end{itemize}
\begin{itemize}
\item {Utilização:Agr.}
\end{itemize}
Fio de água, que escorre das represas. Cf. Assis, \textunderscore Águas\textunderscore , 169.
\section{Chorágicos}
\begin{itemize}
\item {fónica:co}
\end{itemize}
\begin{itemize}
\item {Grp. gram.:m. pl.}
\end{itemize}
\begin{itemize}
\item {Proveniência:(Do lat. \textunderscore choragium\textunderscore )}
\end{itemize}
Monumentos que, entre os Gregos, se erguiam aos mestres de côro, premiados.
\section{Choral}
\begin{itemize}
\item {fónica:co}
\end{itemize}
\begin{itemize}
\item {Grp. gram.:adj.}
\end{itemize}
O mesmo que \textunderscore coral\textunderscore ^2.
\section{Choramigador}
\begin{itemize}
\item {Grp. gram.:m.}
\end{itemize}
Aquelle que choramiga.
\section{Choramigar}
\begin{itemize}
\item {Grp. gram.:v. i.}
\end{itemize}
\begin{itemize}
\item {Proveniência:(De \textunderscore chora-migas\textunderscore )}
\end{itemize}
Chorar por motivos fúteis, e amiúde.
\section{Chora-migas}
\begin{itemize}
\item {Grp. gram.:m.  e  f.}
\end{itemize}
Pessôa que choramiga.
\section{Choramigueiro}
\begin{itemize}
\item {Grp. gram.:adj.}
\end{itemize}
Que choramiga.
\section{Chora-mingas}
\textunderscore m.\textunderscore  e \textunderscore f.\textunderscore  (e der.)
O mesmo que \textunderscore chora-migas\textunderscore , etc.
\section{Chora-mínguas}
\begin{itemize}
\item {Grp. gram.:m.  e  f.}
\end{itemize}
O mesmo ou melhor que \textunderscore chora-migas\textunderscore .
\section{Chorão}
\begin{itemize}
\item {Grp. gram.:m.}
\end{itemize}
\begin{itemize}
\item {Grp. gram.:M.  e  adj.}
\end{itemize}
\begin{itemize}
\item {Utilização:Fam.}
\end{itemize}
\begin{itemize}
\item {Proveniência:(De \textunderscore chorar\textunderscore )}
\end{itemize}
Espécie de salgueiro, de ramos pendentes, (\textunderscore salix babylonica\textunderscore ).
Designação de várias plantas, cujas hastes se inclinam e pendem de vasos ou paredes.
Macaco do Brasil.
Passarinho conirostro.
Aquelle que chora muito; lamecha.
\section{Chorar}
\begin{itemize}
\item {Grp. gram.:v. t.}
\end{itemize}
\begin{itemize}
\item {Grp. gram.:V. i.}
\end{itemize}
\begin{itemize}
\item {Proveniência:(Do lat. \textunderscore plorare\textunderscore )}
\end{itemize}
Deplorar, prantear.
Sentir a perda de: \textunderscore ainda hoje chora os pais\textunderscore .
Arrepender-se de.
Significar, chorando: \textunderscore chorava a sua desgraça\textunderscore .
Derramar dos olhos: \textunderscore chorar lagrimas em fio\textunderscore .
Derramar lágrimas.
Têr som análogo á voz dos que pranteiam: \textunderscore chorava o bronze do campanário\textunderscore .
Constranger-se de dor.
\section{Chorecer}
\begin{itemize}
\item {Grp. gram.:v. i.}
\end{itemize}
\begin{itemize}
\item {Utilização:Ant.}
\end{itemize}
\begin{itemize}
\item {Proveniência:(Do lat. \textunderscore florescere\textunderscore )}
\end{itemize}
Deitar gomos ou rebentos (a planta).
\section{Choregia}
\begin{itemize}
\item {fónica:co}
\end{itemize}
\begin{itemize}
\item {Grp. gram.:f.}
\end{itemize}
Cargo ou funcções de \textunderscore chorego\textunderscore .
\section{Chorego}
\begin{itemize}
\item {fónica:co}
\end{itemize}
\begin{itemize}
\item {Grp. gram.:m.}
\end{itemize}
\begin{itemize}
\item {Proveniência:(Gr. \textunderscore khoregos\textunderscore )}
\end{itemize}
Aquelle que entre os Gregos custeava a despesa dos espectáculos.
Mestre de côro, entre os Gregos.
\section{Chopa}
\begin{itemize}
\item {fónica:chô}
\end{itemize}
\begin{itemize}
\item {Grp. gram.:f.}
\end{itemize}
\begin{itemize}
\item {Proveniência:(Lat. \textunderscore clupea\textunderscore )}
\end{itemize}
Ponta de ferro ou de aço, com que se armam as garrochas, chuços, etc.
Peixe esparoide.
\section{Choréa}
\begin{itemize}
\item {fónica:co}
\end{itemize}
\begin{itemize}
\item {Grp. gram.:f.}
\end{itemize}
\begin{itemize}
\item {Proveniência:(Gr. \textunderscore khoreia\textunderscore )}
\end{itemize}
Doença, que obriga a movimentos contínuos de certos órgãos.
Nome de uma dança grega.
\section{Choregraphia}
\begin{itemize}
\item {fónica:co}
\end{itemize}
\begin{itemize}
\item {Grp. gram.:f.}
\end{itemize}
\begin{itemize}
\item {Proveniência:(De \textunderscore corégrapho\textunderscore )}
\end{itemize}
Arte de compor bailados.
Arte de dançar.
\section{Chorégrapho}
\begin{itemize}
\item {fónica:co}
\end{itemize}
\begin{itemize}
\item {Grp. gram.:m.}
\end{itemize}
\begin{itemize}
\item {Proveniência:(Do gr. \textunderscore khoros\textunderscore  + \textunderscore graphein\textunderscore )}
\end{itemize}
Aquelle que é versado em coregraphia.
\section{Choreia}
\begin{itemize}
\item {fónica:co}
\end{itemize}
\begin{itemize}
\item {Grp. gram.:f.}
\end{itemize}
\begin{itemize}
\item {Proveniência:(Gr. \textunderscore khoreia\textunderscore )}
\end{itemize}
Doença, que obriga a movimentos contínuos de certos órgãos.
Nome de uma dança grega.
\section{Choreico}
\begin{itemize}
\item {fónica:co}
\end{itemize}
\begin{itemize}
\item {Grp. gram.:adj.}
\end{itemize}
Relativo á choreia.
\section{Choreographia}
\begin{itemize}
\item {Grp. gram.:f.}
\end{itemize}
O mesmo que \textunderscore choregraphia\textunderscore .
\section{Choreográphico}
\begin{itemize}
\item {Grp. gram.:adj.}
\end{itemize}
Relativo á choreographia.
\section{Choreógrapho}
\begin{itemize}
\item {fónica:co}
\end{itemize}
\textunderscore m.\textunderscore  (e der.)
O mesmo que \textunderscore chorégrapho\textunderscore , etc.
\section{Choreu}
\begin{itemize}
\item {fónica:co}
\end{itemize}
\begin{itemize}
\item {Grp. gram.:m.}
\end{itemize}
\begin{itemize}
\item {Proveniência:(Lat. \textunderscore choraeus\textunderscore )}
\end{itemize}
Pé de um verso latino ou grego, composto de uma sýllaba longa, seguida de outra breve.
\section{Choriambo}
\begin{itemize}
\item {fónica:co}
\end{itemize}
\begin{itemize}
\item {Grp. gram.:m.}
\end{itemize}
\begin{itemize}
\item {Proveniência:(Do gr. \textunderscore khoreios\textunderscore  + \textunderscore iambos\textunderscore )}
\end{itemize}
Pé de verso grego ou latino, formado de duas sýllabas breves entre duas longas.
\section{Choricas}
\begin{itemize}
\item {Grp. gram.:m.  e  f.}
\end{itemize}
O mesmo que \textunderscore chora-migas\textunderscore .
\section{Chorina}
\begin{itemize}
\item {Grp. gram.:f.}
\end{itemize}
\begin{itemize}
\item {Utilização:Fam.}
\end{itemize}
Chinó.
\section{Chorina}
\begin{itemize}
\item {Grp. gram.:f.}
\end{itemize}
Planta umbellífera, semelhante ao chorão, e com a qual se bordam alegretes e canteiros.
(Cp. \textunderscore chorão\textunderscore )
\section{Chorinca}
\begin{itemize}
\item {Grp. gram.:m.  e  f.}
\end{itemize}
O mesmo que \textunderscore chorincas\textunderscore .
\section{Chorincar}
\begin{itemize}
\item {Grp. gram.:v. i.}
\end{itemize}
Chorar como as crianças, repetidamente.
\section{Chorincas}
\begin{itemize}
\item {Grp. gram.:m.}
\end{itemize}
\begin{itemize}
\item {Utilização:Pop.}
\end{itemize}
O mesmo que \textunderscore chora-migas\textunderscore .
\section{Chorinco}
\begin{itemize}
\item {Grp. gram.:m.}
\end{itemize}
O mesmo que \textunderscore chôro\textunderscore , no prolóquio \textunderscore os brincos dão em chorincos\textunderscore . Cf. Castilho, \textunderscore Avarento\textunderscore , act. III, sc. 6.
\section{Chorinola}
\begin{itemize}
\item {Grp. gram.:f.}
\end{itemize}
\begin{itemize}
\item {Utilização:Prov.}
\end{itemize}
Mania.
Prosápia, presumpção.
\section{Chorioide}
\begin{itemize}
\item {fónica:co}
\end{itemize}
\begin{itemize}
\item {Grp. gram.:f.}
\end{itemize}
O mesmo ou melhor que \textunderscore choroide\textunderscore .
\section{Chórion}
\begin{itemize}
\item {fónica:có}
\end{itemize}
\begin{itemize}
\item {Grp. gram.:m.}
\end{itemize}
\begin{itemize}
\item {Proveniência:(Gr. \textunderscore khorion\textunderscore , coiro)}
\end{itemize}
Membrana exterior do féto.
\section{Chorionina}
\begin{itemize}
\item {fónica:co}
\end{itemize}
\begin{itemize}
\item {Grp. gram.:f.}
\end{itemize}
\begin{itemize}
\item {Proveniência:(De \textunderscore chórion\textunderscore )}
\end{itemize}
Medicamento, que é extracto de placenta e se applica contra a falta de leite.
\section{Chorizonte}
\begin{itemize}
\item {fónica:co}
\end{itemize}
\begin{itemize}
\item {Grp. gram.:m.}
\end{itemize}
\begin{itemize}
\item {Proveniência:(Do gr. \textunderscore khorizein\textunderscore , de \textunderscore khoris\textunderscore , separação)}
\end{itemize}
Aquelle que, entre os Gregos, attribuía a Ilíada e a Odysseia a autores differentes.
\section{Chorlo}
\begin{itemize}
\item {Grp. gram.:m.}
\end{itemize}
\begin{itemize}
\item {Proveniência:(Al. \textunderscore schorl\textunderscore )}
\end{itemize}
Mineral, espécie de basalto.
\section{Chôro}
\begin{itemize}
\item {Grp. gram.:m.}
\end{itemize}
\begin{itemize}
\item {Utilização:Bras. do N}
\end{itemize}
Acto de chorar.
Música popular, executada á viola.
\section{Choró}
\begin{itemize}
\item {Grp. gram.:m.}
\end{itemize}
Avezinha do norte do Brasil.
\section{Chorographia}
\begin{itemize}
\item {fónica:co}
\end{itemize}
\begin{itemize}
\item {Grp. gram.:f.}
\end{itemize}
\begin{itemize}
\item {Proveniência:(Do gr. \textunderscore khora\textunderscore  + \textunderscore graphein\textunderscore )}
\end{itemize}
Descripção de uma região ou de uma parte importante de um território.
\section{Chorográphico}
\begin{itemize}
\item {fónica:co}
\end{itemize}
\begin{itemize}
\item {Grp. gram.:adj.}
\end{itemize}
Relativo á chorographia.
\section{Chorógrapho}
\begin{itemize}
\item {fónica:co}
\end{itemize}
\begin{itemize}
\item {Grp. gram.:m.}
\end{itemize}
Aquelle que escreve sôbre chorographia.
\section{Choroide}
\begin{itemize}
\item {fónica:co}
\end{itemize}
\begin{itemize}
\item {Grp. gram.:f.}
\end{itemize}
\begin{itemize}
\item {Proveniência:(Gr. \textunderscore khoroeides\textunderscore )}
\end{itemize}
Membrana da parte posterior do ôlho.
Membrana da pia-máter.
\section{Chorona}
\begin{itemize}
\item {Grp. gram.:f.  e  adj.}
\end{itemize}
(fem. de \textunderscore chorão\textunderscore )
\section{Chorosamente}
\begin{itemize}
\item {Grp. gram.:adv.}
\end{itemize}
\begin{itemize}
\item {Proveniência:(De \textunderscore choroso\textunderscore )}
\end{itemize}
Com chôro.
\section{Choroso}
\begin{itemize}
\item {Grp. gram.:adj.}
\end{itemize}
\begin{itemize}
\item {Proveniência:(De \textunderscore chorar\textunderscore )}
\end{itemize}
Que chora; que causa lástima ou dôr.
\section{Chorrar}
\begin{itemize}
\item {Grp. gram.:v. i.}
\end{itemize}
(V. \textunderscore jorrar\textunderscore ^2)
\section{Chorreado}
\begin{itemize}
\item {Grp. gram.:adj.}
\end{itemize}
Diz-se do toiro que tem linhas escuras e verticaes no pêlo.
\section{Chorreira}
\begin{itemize}
\item {Grp. gram.:f.}
\end{itemize}
\begin{itemize}
\item {Utilização:Prov.}
\end{itemize}
\begin{itemize}
\item {Utilização:minh.}
\end{itemize}
\begin{itemize}
\item {Proveniência:(De \textunderscore chôrro\textunderscore )}
\end{itemize}
O mesmo que \textunderscore enxurrada\textunderscore .
\section{Chorreiro}
\begin{itemize}
\item {Grp. gram.:m.}
\end{itemize}
\begin{itemize}
\item {Utilização:Pop.}
\end{itemize}
\begin{itemize}
\item {Proveniência:(De \textunderscore chôrro\textunderscore )}
\end{itemize}
Grande porção (de asneiras, de mentiras, etc.).
\section{Chorrião}
\begin{itemize}
\item {Grp. gram.:m.}
\end{itemize}
Carruagem pesada; carrão.
(Relaciona-se com \textunderscore zorra\textunderscore ?)
\section{Chorrilho}
\begin{itemize}
\item {Grp. gram.:m.}
\end{itemize}
\begin{itemize}
\item {Proveniência:(De \textunderscore chôrro\textunderscore )}
\end{itemize}
Série, successão, conjunto de coisas ou pessôas, mais ou menos semelhantes: \textunderscore um chorrilho de disparates\textunderscore .
\section{Chôrro}
\begin{itemize}
\item {Grp. gram.:m.}
\end{itemize}
\begin{itemize}
\item {Utilização:Des.}
\end{itemize}
O mesmo que \textunderscore jôrro\textunderscore .
\section{Chortonomia}
\begin{itemize}
\item {fónica:cor}
\end{itemize}
\begin{itemize}
\item {Grp. gram.:f.}
\end{itemize}
\begin{itemize}
\item {Proveniência:(Do gr. \textunderscore khorton\textunderscore  + \textunderscore nomos\textunderscore )}
\end{itemize}
Arte de fazer herbários.
\section{Chorudo}
\begin{itemize}
\item {Grp. gram.:adj.}
\end{itemize}
\begin{itemize}
\item {Utilização:Pop.}
\end{itemize}
\begin{itemize}
\item {Proveniência:(Lat. hyp. \textunderscore florutus\textunderscore , de \textunderscore flos\textunderscore , \textunderscore floris\textunderscore )}
\end{itemize}
Gôrdo.
Rendoso: \textunderscore emprêgo chorudo\textunderscore .
\section{Chorume}
\begin{itemize}
\item {Grp. gram.:m.}
\end{itemize}
\begin{itemize}
\item {Utilização:Fig.}
\end{itemize}
\begin{itemize}
\item {Proveniência:(Lat. hyp. \textunderscore florumen\textunderscore , de \textunderscore flos\textunderscore , \textunderscore floris\textunderscore )}
\end{itemize}
Banha, pingo.
Abundância.
Opulência.
\section{Chorumento}
\begin{itemize}
\item {Grp. gram.:adj.}
\end{itemize}
Que tem chorume.
\section{Chostra}
\begin{itemize}
\item {fónica:chôs}
\end{itemize}
\begin{itemize}
\item {Grp. gram.:f.}
\end{itemize}
\begin{itemize}
\item {Utilização:Prov.}
\end{itemize}
Porcaria ou sujidade na roupa; badalhocas.
Qualquer coisa, feita sem asseio.
Coisa mal feita.
O mesmo que \textunderscore lôstra\textunderscore .
\section{Chotba}
\begin{itemize}
\item {Grp. gram.:f.}
\end{itemize}
\begin{itemize}
\item {Proveniência:(T. ár.)}
\end{itemize}
Solennidade moirisca, espécie de Missa. Cf. Herculano, \textunderscore Hist. de Port.\textunderscore , liv. VII, parte I.
\section{Chote!}
\begin{itemize}
\item {Grp. gram.:interj.}
\end{itemize}
\begin{itemize}
\item {Proveniência:(T. onom.)}
\end{itemize}
(para afugentar aves)
\section{Choto}
\begin{itemize}
\item {Grp. gram.:m.}
\end{itemize}
\begin{itemize}
\item {Utilização:Prov.}
\end{itemize}
\begin{itemize}
\item {Utilização:dur.}
\end{itemize}
Vão, que fica por baixo do soqueiro, nos barcos rabelos, e serve geralmente para dispensa, arrumação de ferramentas, etc.
(Alter. de \textunderscore sótão\textunderscore ?)
\section{Choupa}
\begin{itemize}
\item {Grp. gram.:f.}
\end{itemize}
\begin{itemize}
\item {Proveniência:(Lat. \textunderscore clupea\textunderscore )}
\end{itemize}
Ponta de ferro ou de aço, com que se armam as garrochas, chuços, etc.
Peixe esparoide.
Ferro de dois gumes e cabo curto, com que se abatem reses no matadoiro. Cf. B. Pereira, \textunderscore Prosodia\textunderscore , vb. \textunderscore cultrus\textunderscore .
\section{Choupa}
\begin{itemize}
\item {Grp. gram.:f.}
\end{itemize}
Árvore, semelhante ao choupo, mas mais còpada e de fôlhas mais largas.
(Cp. \textunderscore choupo\textunderscore ^1)
\section{Choupal}
\begin{itemize}
\item {Grp. gram.:m.}
\end{itemize}
Lugar, onde crescem choupos.
\section{Choupana}
\begin{itemize}
\item {Grp. gram.:f.}
\end{itemize}
Cabana, casa rústica de madeira, coberta de ramos ou de colmo.
\section{Choupaneiro}
\begin{itemize}
\item {Grp. gram.:m.}
\end{itemize}
Morador de choupana.
\section{Choupelo}
\begin{itemize}
\item {fónica:pê}
\end{itemize}
\begin{itemize}
\item {Grp. gram.:m.}
\end{itemize}
\begin{itemize}
\item {Utilização:Prov.}
\end{itemize}
Choupo, delgado e alto.
Mata de feno grosseiro?«\textunderscore Não vê no alto um choupelo, a modo de uma toiça?\textunderscore »Camillo, \textunderscore Myst. de Lisb.\textunderscore , I, 355.
\section{Choupilo}
\begin{itemize}
\item {Grp. gram.:m.}
\end{itemize}
\begin{itemize}
\item {Utilização:Prov.}
\end{itemize}
O mesmo que \textunderscore conchelo\textunderscore .
\section{Choupo}
\begin{itemize}
\item {Grp. gram.:m.}
\end{itemize}
\begin{itemize}
\item {Proveniência:(De \textunderscore plopus\textunderscore , metáth. de \textunderscore pop'lus\textunderscore  = lat. \textunderscore populus\textunderscore )}
\end{itemize}
Árvore salicínea.
\section{Choura}
\begin{itemize}
\item {Grp. gram.:f.}
\end{itemize}
\begin{itemize}
\item {Utilização:Gír. lisb.}
\end{itemize}
Cada um dos cabazes, em que os peixeiros trazem o peixe, e que se suspendem de um pau roliço atravessado sôbre os ombros.
Chouriço.
\section{Chouri}
\begin{itemize}
\item {Grp. gram.:m.}
\end{itemize}
Espécie de pennacho, com cabo de prata, usado em solennidades dos pagodes indianos. Cf. Th. Ribeiro, \textunderscore Jornadas\textunderscore , II, 113.
\section{Chouriça}
\begin{itemize}
\item {Grp. gram.:f.}
\end{itemize}
O mesmo que \textunderscore chouriço\textunderscore .
\section{Chouriçada}
\begin{itemize}
\item {Grp. gram.:f.}
\end{itemize}
Grande porção de chouriços.
Pancada com chouriço.
\section{Chouriceiro}
\begin{itemize}
\item {Grp. gram.:m.}
\end{itemize}
Aquelle que faz ou vende chouriços.
\section{Chouriço}
\begin{itemize}
\item {Grp. gram.:m.}
\end{itemize}
\begin{itemize}
\item {Utilização:Bras. do S}
\end{itemize}
\begin{itemize}
\item {Utilização:Bras. do N}
\end{itemize}
Pedaço de tripa, cheio de carne com gordura e temperos, e sêco ao fumo.
Saco longo e cylíndrico, cheio de areia ou serradura, para tapar as fisgas inferiores de portas e janelas.
Chinguiço.
Rôlo de cabello, para altear o penteado.
Parte acolchoada do rabicho, que passa por baixo da raíz da cauda do cavallo.
Iguaria, feita de sangue de porco e açúcar.
(Cast. \textunderscore chorizo\textunderscore )
\section{Chousa}
\begin{itemize}
\item {Grp. gram.:f.}
\end{itemize}
\begin{itemize}
\item {Utilização:Ant.}
\end{itemize}
O mesmo que \textunderscore chouso\textunderscore .
\section{Chousal}
\begin{itemize}
\item {Grp. gram.:m.}
\end{itemize}
O mesmo que \textunderscore chouso\textunderscore .
\section{Chouso}
\begin{itemize}
\item {Grp. gram.:m.}
\end{itemize}
\begin{itemize}
\item {Utilização:Ant.}
\end{itemize}
\begin{itemize}
\item {Proveniência:(Do lat. \textunderscore clausus\textunderscore )}
\end{itemize}
Redil ou sebe, que os pastores armam no campo, de verão, para ali recolherem o gado.
Pequena fazenda cerrada sôbre si; tapado; cerrado.
\section{Chousura}
\begin{itemize}
\item {Grp. gram.:f.}
\end{itemize}
\begin{itemize}
\item {Utilização:Ant.}
\end{itemize}
\begin{itemize}
\item {Proveniência:(De \textunderscore chouso\textunderscore )}
\end{itemize}
Tapume ou parede, que separa uma terra da outra.
\section{Choutador}
\begin{itemize}
\item {Grp. gram.:adj.}
\end{itemize}
\begin{itemize}
\item {Proveniência:(De \textunderscore choutar\textunderscore )}
\end{itemize}
Que anda a chouto.
\section{Choutão}
\begin{itemize}
\item {Grp. gram.:adj.}
\end{itemize}
(V.choutador)
\section{Choutar}
\begin{itemize}
\item {Grp. gram.:v. i.}
\end{itemize}
Andar a chouto.
(Cp. lat. \textunderscore tolutaris\textunderscore )
\section{Choutear}
\begin{itemize}
\item {Grp. gram.:v. i.}
\end{itemize}
O mesmo que \textunderscore choutar\textunderscore . Cf. Herculano, \textunderscore M. de Cister\textunderscore , 235.
\section{Chouteiro}
\begin{itemize}
\item {Grp. gram.:adj.}
\end{itemize}
(V.choutador)
\section{Chouto}
\begin{itemize}
\item {Grp. gram.:m.}
\end{itemize}
\begin{itemize}
\item {Proveniência:(De \textunderscore choutar\textunderscore )}
\end{itemize}
Trote miúdo e incômmodo.
\section{Chouvir}
\begin{itemize}
\item {Grp. gram.:v. t.}
\end{itemize}
\begin{itemize}
\item {Utilização:Ant.}
\end{itemize}
\begin{itemize}
\item {Proveniência:(Do lat. \textunderscore claudere\textunderscore )}
\end{itemize}
Fechar, encerrar.
\section{Chovediço}
\begin{itemize}
\item {Grp. gram.:adj.}
\end{itemize}
\begin{itemize}
\item {Proveniência:(De \textunderscore chover\textunderscore )}
\end{itemize}
Proveniente da chuva; pluvial.
Que ameaça chuva.
\section{Chovedio}
\begin{itemize}
\item {Grp. gram.:adj.}
\end{itemize}
O mesmo que \textunderscore chovediço\textunderscore . Cf. \textunderscore Techn. Rur.\textunderscore , 78.
\section{Chover}
\begin{itemize}
\item {Grp. gram.:v. i.}
\end{itemize}
\begin{itemize}
\item {Utilização:Fig.}
\end{itemize}
\begin{itemize}
\item {Grp. gram.:V. t.}
\end{itemize}
\begin{itemize}
\item {Proveniência:(Do lat. \textunderscore pluere\textunderscore )}
\end{itemize}
Caír água da atmosphera.
Caír da atmosphera: \textunderscore chovem raios e coriscos\textunderscore .
Vir em abundancia: \textunderscore no tempo daquelle Ministro, choveram os decretos\textunderscore .
Fazer caír em gotas ou jorros, como a chuva.
Causar.
\section{Chovisco}
\textunderscore m.\textunderscore  (e der.)
(V. \textunderscore chuvisco\textunderscore , etc.)
\section{Chrematística}
\begin{itemize}
\item {Grp. gram.:f.}
\end{itemize}
\begin{itemize}
\item {Proveniência:(Gr. \textunderscore krematistike\textunderscore )}
\end{itemize}
Arte de produzir riqueza.
Tratado das riquezas.
\section{Chrematístico}
\begin{itemize}
\item {Grp. gram.:adj.}
\end{itemize}
Relativo á chrematística.
\section{Chrematologia}
\begin{itemize}
\item {Grp. gram.:f.}
\end{itemize}
\begin{itemize}
\item {Proveniência:(Do gr. \textunderscore khrema\textunderscore  + \textunderscore logos\textunderscore )}
\end{itemize}
Tratado da riqueza.
\section{Chrematológico}
\begin{itemize}
\item {Grp. gram.:adj.}
\end{itemize}
Relativo á chrematologia.
\section{Chrematonomia}
\begin{itemize}
\item {Grp. gram.:f.}
\end{itemize}
\begin{itemize}
\item {Proveniência:(Do gr. \textunderscore khrema\textunderscore  + \textunderscore nomos\textunderscore )}
\end{itemize}
Conjunto das leis naturaes, que regulam a producção e repartição da riqueza.
\section{Chrematonómico}
\begin{itemize}
\item {Grp. gram.:adj.}
\end{itemize}
Relativo á crematonomia.
\section{Chrestomathia}
\begin{itemize}
\item {Grp. gram.:f.}
\end{itemize}
\begin{itemize}
\item {Proveniência:(Gr. \textunderscore khrestomatheia\textunderscore )}
\end{itemize}
O mesmo que \textunderscore anthologia\textunderscore .
\section{Chrisma}
\begin{itemize}
\item {Grp. gram.:m.}
\end{itemize}
\begin{itemize}
\item {Proveniência:(Gr. \textunderscore khrisma\textunderscore )}
\end{itemize}
Óleo perfumado, que serve na ministração de alguns sacramentos e em outras ceremónias religiosas.
Sacramento da confirmação.
\section{Chrismar}
\begin{itemize}
\item {Grp. gram.:v. t.}
\end{itemize}
\begin{itemize}
\item {Utilização:Fig.}
\end{itemize}
Conferir a chrisma a.
Mudar o nome a; alcunhar.
\section{Christã}
\begin{itemize}
\item {Grp. gram.:f.  e  adj.}
\end{itemize}
(fem. de \textunderscore christão\textunderscore )
\section{Christan}
\begin{itemize}
\item {Grp. gram.:f.  e  adj.}
\end{itemize}
(fem. de \textunderscore christão\textunderscore )
\section{Christandade}
\begin{itemize}
\item {Grp. gram.:f.}
\end{itemize}
\begin{itemize}
\item {Proveniência:(Lat. \textunderscore christianitas\textunderscore )}
\end{itemize}
Qualidade do que é christão.
Conjunto dos povos christãos.
\section{Christandia}
\begin{itemize}
\item {Grp. gram.:f.}
\end{itemize}
\begin{itemize}
\item {Utilização:Des.}
\end{itemize}
Christandade.
Grande número de christãos. Cf. o romance popular \textunderscore D. João da Armada\textunderscore .
\section{Christanmente}
\begin{itemize}
\item {Grp. gram.:adv.}
\end{itemize}
De modo christão.
\section{Christanvelhice}
\begin{itemize}
\item {Grp. gram.:f.}
\end{itemize}
\begin{itemize}
\item {Utilização:P. us.}
\end{itemize}
\begin{itemize}
\item {Proveniência:(De \textunderscore christão\textunderscore  + \textunderscore velho\textunderscore )}
\end{itemize}
Qualidade de christão velho, em opposição a christão novo ou a judeu baptizado.
\section{Christão}
\begin{itemize}
\item {Grp. gram.:adj.}
\end{itemize}
\begin{itemize}
\item {Grp. gram.:M.}
\end{itemize}
\begin{itemize}
\item {Proveniência:(Do lat. \textunderscore christianus\textunderscore )}
\end{itemize}
Que professa o christianismo.
Relativo ao christianismo.
Sectário da religião de Christo.
\section{Christengo}
\begin{itemize}
\item {Grp. gram.:adj.}
\end{itemize}
\begin{itemize}
\item {Utilização:Ant.}
\end{itemize}
\begin{itemize}
\item {Proveniência:(Do lat. \textunderscore christianicus\textunderscore )}
\end{itemize}
Christão, relativo a christãos.
E dizia-se dos caracteres latinos.
\section{Christianicida}
\begin{itemize}
\item {Grp. gram.:m.}
\end{itemize}
\begin{itemize}
\item {Proveniência:(Do lat. \textunderscore christianus\textunderscore  + \textunderscore caedere\textunderscore )}
\end{itemize}
Matador de christãos.
\section{Christianicídio}
\begin{itemize}
\item {Grp. gram.:m.}
\end{itemize}
Matança de christãos.
(Cp. \textunderscore christianicida\textunderscore )
\section{Christianismo}
\begin{itemize}
\item {Grp. gram.:m.}
\end{itemize}
\begin{itemize}
\item {Proveniência:(Do lat. \textunderscore christianus\textunderscore )}
\end{itemize}
Religião de Christo.
\section{Christianíssimo}
\begin{itemize}
\item {Grp. gram.:adj.}
\end{itemize}
(sup. de \textunderscore christão\textunderscore )
\section{Christianização}
\begin{itemize}
\item {Grp. gram.:f.}
\end{itemize}
Acto de christianizar.
\section{Christianizador}
\begin{itemize}
\item {Grp. gram.:m.  e  adj.}
\end{itemize}
O que christianiza. Cf. Th. Ribeiro, \textunderscore Jornadas\textunderscore , II, 71.
\section{Christianizar}
\begin{itemize}
\item {Grp. gram.:v. t.}
\end{itemize}
\begin{itemize}
\item {Proveniência:(Do lat. \textunderscore christianus\textunderscore )}
\end{itemize}
Tornar christão.
Incluir na disciplina ou na prátíca dos christãos.
\section{Christiano}
\begin{itemize}
\item {Grp. gram.:m.}
\end{itemize}
\begin{itemize}
\item {Proveniência:(De \textunderscore Christiano\textunderscore , n. p.)}
\end{itemize}
Moéda de oiro, na Dinamarca.
\section{Christicida}
\begin{itemize}
\item {Grp. gram.:m.}
\end{itemize}
\begin{itemize}
\item {Proveniência:(Do lat. \textunderscore Christus\textunderscore , n. p. + \textunderscore caedere\textunderscore )}
\end{itemize}
Quem matou Christo.
\section{Christicídio}
\begin{itemize}
\item {Grp. gram.:m.}
\end{itemize}
Morte de Christo.
(Cp. \textunderscore christicida\textunderscore )
\section{Christícola}
\begin{itemize}
\item {Proveniência:(Do lat. \textunderscore Christus\textunderscore , n. p. + \textunderscore colere\textunderscore )}
\end{itemize}
Aquelle que adora Chrísto.
\section{Christífero}
\begin{itemize}
\item {Grp. gram.:adj.}
\end{itemize}
\begin{itemize}
\item {Proveniência:(Do lat. \textunderscore Christus\textunderscore , n. p. + \textunderscore ferre\textunderscore )}
\end{itemize}
Que sustenta ou leva uma imagem de Christo.
\section{Christino}
\begin{itemize}
\item {Grp. gram.:m.}
\end{itemize}
Partidário da rainha Christina, em Espanha.
\section{Christípara}
\begin{itemize}
\item {Grp. gram.:f.}
\end{itemize}
\begin{itemize}
\item {Proveniência:(Do lat. \textunderscore Christus\textunderscore , n. p. + \textunderscore parere\textunderscore )}
\end{itemize}
Mãe de Christo.
\section{Christo}
\begin{itemize}
\item {Grp. gram.:m.}
\end{itemize}
\begin{itemize}
\item {Proveniência:(Lat. \textunderscore Christus\textunderscore , n. p.)}
\end{itemize}
Imagem de Christo crucificado: \textunderscore tinha á cabeceira um Christo de marfim\textunderscore .
\section{Christofle}
\begin{itemize}
\item {Grp. gram.:m.}
\end{itemize}
\begin{itemize}
\item {Proveniência:(De \textunderscore Christofle\textunderscore , n. p.)}
\end{itemize}
Metal, de composição análoga á do argentão.
\section{Christologia}
\begin{itemize}
\item {Grp. gram.:f.}
\end{itemize}
\begin{itemize}
\item {Proveniência:(Do gr. \textunderscore Khristos\textunderscore , n. p. + \textunderscore logos\textunderscore )}
\end{itemize}
Tratado á cêrca da pessôa e doutrina de Christo.
\section{Christológico}
\begin{itemize}
\item {Grp. gram.:adj.}
\end{itemize}
Que diz respeito á christologia.
\section{Christómacho}
\begin{itemize}
\item {Grp. gram.:m.}
\end{itemize}
\begin{itemize}
\item {Proveniência:(Gr. \textunderscore khristomakos\textunderscore )}
\end{itemize}
Aquelle que sustenta doutrina falsa sôbre a natureza ou pessôa de Christo.
\section{Christophania}
\begin{itemize}
\item {Grp. gram.:f.}
\end{itemize}
\begin{itemize}
\item {Proveniência:(Do gr. \textunderscore Khristos\textunderscore  + \textunderscore phainestai\textunderscore , apparecer)}
\end{itemize}
Apparição de Christo.
\section{Chrithomancia}
\begin{itemize}
\item {Grp. gram.:f.}
\end{itemize}
Supposta arte de adivinhar, por meio do sal com farinha de cevada. Cf. Castilho, \textunderscore Fastos\textunderscore , III, 315.
\section{Chroma}
\begin{itemize}
\item {Grp. gram.:f.}
\end{itemize}
\begin{itemize}
\item {Utilização:Mús.}
\end{itemize}
Escala chromática.
Melodia, que procede por semi-tons. Cf. Camillo, \textunderscore Cav. em Ruínas\textunderscore , 50.
\section{Chromado}
\begin{itemize}
\item {Grp. gram.:adj.}
\end{itemize}
Que tem chromo.
\section{Chromâmetro}
\begin{itemize}
\item {Grp. gram.:m.}
\end{itemize}
\begin{itemize}
\item {Utilização:Mús.}
\end{itemize}
\begin{itemize}
\item {Proveniência:(Do gr. \textunderscore khroma\textunderscore  + \textunderscore metron\textunderscore )}
\end{itemize}
Apparelho, hoje desusado, para exercício de quem aprende a afinar pianos, e que é uma espécie de monocórdio, com um braço em que estão marcadas todas as divisões da escala chromática.
\section{Chromática}
\begin{itemize}
\item {Grp. gram.:f.}
\end{itemize}
\begin{itemize}
\item {Proveniência:(Do gr. \textunderscore khroma\textunderscore )}
\end{itemize}
Arte de combinar as côres.
\section{Chromaticamente}
\begin{itemize}
\item {Grp. gram.:adv.}
\end{itemize}
Por semi-tons; de modo chromático.
\section{Chromático}
\begin{itemize}
\item {Grp. gram.:adj.}
\end{itemize}
\begin{itemize}
\item {Utilização:Mús.}
\end{itemize}
\begin{itemize}
\item {Proveniência:(Do gr. \textunderscore khroma\textunderscore )}
\end{itemize}
Relativo a côres, em Phýsica.
Composto de uma série de semi-tons.
\section{Chromatina}
\begin{itemize}
\item {Grp. gram.:f.}
\end{itemize}
\begin{itemize}
\item {Utilização:Physiol.}
\end{itemize}
\begin{itemize}
\item {Proveniência:(Do gr. \textunderscore khroma\textunderscore )}
\end{itemize}
Substância, que entra na composição do núcleo cellular, assim chamada pela sua affinidade com as matérias corantes.
\section{Chromatismo}
\begin{itemize}
\item {Grp. gram.:m.}
\end{itemize}
\begin{itemize}
\item {Utilização:Phýs.}
\end{itemize}
\begin{itemize}
\item {Proveniência:(Gr. \textunderscore khromatismos\textunderscore )}
\end{itemize}
Dispersão da luz.
Recomposição da luz, que atravessou corpos diáphanos.
\section{Chromato}
\begin{itemize}
\item {Grp. gram.:m.}
\end{itemize}
\begin{itemize}
\item {Proveniência:(De \textunderscore chromo\textunderscore )}
\end{itemize}
Combinação do ácido chrómico com uma base.
\section{Chromatogênico}
\begin{itemize}
\item {Grp. gram.:adj.}
\end{itemize}
\begin{itemize}
\item {Proveniência:(Do gr. \textunderscore khroma\textunderscore  + \textunderscore genea\textunderscore )}
\end{itemize}
Diz-se de certos micróbios, que se revelam pela producção de côres.
\section{Chrómico}
\begin{itemize}
\item {Grp. gram.:adj.}
\end{itemize}
\begin{itemize}
\item {Proveniência:(De \textunderscore chromo\textunderscore )}
\end{itemize}
Diz-se de um ácido, em que entra o chromo e o oxygênio.
Relativo a côres.
\section{Chrómio}
\begin{itemize}
\item {Grp. gram.:m.}
\end{itemize}
O mesmo que \textunderscore chromo\textunderscore .
\section{Chromismo}
\begin{itemize}
\item {Grp. gram.:m.}
\end{itemize}
\begin{itemize}
\item {Utilização:Bot.}
\end{itemize}
\begin{itemize}
\item {Proveniência:(Do gr. \textunderscore khroma\textunderscore )}
\end{itemize}
Excesso anómalo de coloração.
\section{Chromite}
\begin{itemize}
\item {Grp. gram.:f.}
\end{itemize}
\begin{itemize}
\item {Utilização:Geol.}
\end{itemize}
\begin{itemize}
\item {Proveniência:(De \textunderscore chromo\textunderscore )}
\end{itemize}
Espécie de espinella.
\section{Chromo}
\begin{itemize}
\item {Grp. gram.:m.}
\end{itemize}
\begin{itemize}
\item {Proveniência:(Do gr. \textunderscore khroma\textunderscore )}
\end{itemize}
Metal cinzento, que se encontra no ferro e noutros corpos.
Desenho impresso a côres.
\section{Chromogênio}
\begin{itemize}
\item {Grp. gram.:m.}
\end{itemize}
\begin{itemize}
\item {Proveniência:(Do gr. \textunderscore khroma\textunderscore  + \textunderscore genea\textunderscore )}
\end{itemize}
Diz-se de certo micróbio, que dá coloração verde á neve, sôbre que vive.
\section{Chromógrapho}
\begin{itemize}
\item {Grp. gram.:m.}
\end{itemize}
\begin{itemize}
\item {Proveniência:(Do gr. \textunderscore khroma\textunderscore  + \textunderscore graphein\textunderscore )}
\end{itemize}
Apparelho de balística, para medir a velocidade dos projécteis e o tempo que gastam no seu percurso.
\section{Chromolithographia}
\begin{itemize}
\item {Grp. gram.:f.}
\end{itemize}
\begin{itemize}
\item {Proveniência:(De \textunderscore chromo\textunderscore  + \textunderscore lithographia\textunderscore )}
\end{itemize}
Lithographia a côres.
\section{Chromolithográphico}
\begin{itemize}
\item {Grp. gram.:adj.}
\end{itemize}
Relativo á chromo-litographia.
\section{Chromophilia}
\begin{itemize}
\item {Grp. gram.:f.}
\end{itemize}
\begin{itemize}
\item {Utilização:Neol.}
\end{itemize}
\begin{itemize}
\item {Proveniência:(De \textunderscore chromóphilo\textunderscore )}
\end{itemize}
Grande affeição ás côres vivas.
\section{Chromóphilo}
\begin{itemize}
\item {Grp. gram.:adj.}
\end{itemize}
\begin{itemize}
\item {Proveniência:(Do gr. \textunderscore khroma\textunderscore  + \textunderscore philos\textunderscore )}
\end{itemize}
Que gosta de côres vivas.
\section{Chromóphoro}
\begin{itemize}
\item {Grp. gram.:m.}
\end{itemize}
\begin{itemize}
\item {Utilização:Zool.}
\end{itemize}
\begin{itemize}
\item {Proveniência:(Do gr. \textunderscore khroma\textunderscore  + \textunderscore phoros\textunderscore )}
\end{itemize}
Folículo colorido, que guarnece o corpo dos cephalópodes.
\section{Chromophytose}
\begin{itemize}
\item {Grp. gram.:f.}
\end{itemize}
\begin{itemize}
\item {Utilização:Med.}
\end{itemize}
\begin{itemize}
\item {Proveniência:(Do gr. \textunderscore khroma\textunderscore  + \textunderscore phuton\textunderscore )}
\end{itemize}
Doença cutânea, conhecida vulgarmente por \textunderscore pano\textunderscore , (\textunderscore pannus hepaticus\textunderscore ), e que, sendo affecção parasitária, póde atacar o tronco e as extremidades superiores do corpo.
\section{Chromotherapia}
\begin{itemize}
\item {Grp. gram.:f.}
\end{itemize}
Tratamento médico pela acção das côres.
\section{Chromotypographia}
\begin{itemize}
\item {Grp. gram.:f.}
\end{itemize}
Processo de impressão a côres.
\section{Chrómula}
\begin{itemize}
\item {Grp. gram.:f.}
\end{itemize}
\begin{itemize}
\item {Proveniência:(Do gr. \textunderscore khroma\textunderscore  + \textunderscore ule\textunderscore )}
\end{itemize}
O mesmo que \textunderscore chlorophylla\textunderscore .
\section{Chromurgia}
\begin{itemize}
\item {Grp. gram.:f.}
\end{itemize}
\begin{itemize}
\item {Proveniência:(Do gr. \textunderscore khroma\textunderscore  + \textunderscore ergon\textunderscore )}
\end{itemize}
Parte da Chímica, que trata das côres e das tintas.
\section{Chromúrgico}
\begin{itemize}
\item {Grp. gram.:adj.}
\end{itemize}
Relativo á chromurgia.
\section{Chrónica}
\begin{itemize}
\item {Grp. gram.:f.}
\end{itemize}
\begin{itemize}
\item {Proveniência:(Lat. \textunderscore chronica\textunderscore , pl. de \textunderscore chronicum\textunderscore )}
\end{itemize}
Narração histórica, segundo a ordem dos tempos.
Noticiário dos periódicos.
Revista scientifica ou literária, que preenche periodicamente uma secção de jornal.
\section{Chronicamente}
\begin{itemize}
\item {Grp. gram.:adv.}
\end{itemize}
De modo chrónico.
\section{Chronicão}
\begin{itemize}
\item {Grp. gram.:m.}
\end{itemize}
Volumosa chrónica medieval. Cf. Camillo, \textunderscore Quéda\textunderscore , 9.
(B. lat. \textunderscore chronicon\textunderscore )
\section{Chronicidade}
\begin{itemize}
\item {Grp. gram.:f.}
\end{itemize}
\begin{itemize}
\item {Proveniência:(De \textunderscore chrónico\textunderscore )}
\end{itemize}
Qualidade das doenças chrónicas.
\section{Chrónico}
\begin{itemize}
\item {Grp. gram.:adj.}
\end{itemize}
\begin{itemize}
\item {Utilização:Fig.}
\end{itemize}
\begin{itemize}
\item {Proveniência:(Lat. \textunderscore chronicus\textunderscore )}
\end{itemize}
Que dura há muito tempo.
Inveterado: \textunderscore doenças chrónicas\textunderscore .
\section{Chrónicon}
\begin{itemize}
\item {Grp. gram.:m.}
\end{itemize}
(V.chronicão)Cf. Herculano, \textunderscore Hist. de Port.\textunderscore , I, 2, 182, 483.
\section{Chroniqueiro}
\begin{itemize}
\item {Grp. gram.:m.}
\end{itemize}
\begin{itemize}
\item {Utilização:Fam.}
\end{itemize}
\begin{itemize}
\item {Proveniência:(De \textunderscore chrónica\textunderscore )}
\end{itemize}
Noticiarista na imprensa.
\section{Chroniquizar}
\begin{itemize}
\item {Grp. gram.:v. t.}
\end{itemize}
Reduzir a uma chrónica; narrar em chrónica.
\section{Chronista}
\begin{itemize}
\item {Grp. gram.:m.}
\end{itemize}
\begin{itemize}
\item {Proveniência:(Do gr. \textunderscore khronos\textunderscore , tempo)}
\end{itemize}
Aquelle que escreve chrónicas.
\section{Chronizoico}
\begin{itemize}
\item {Grp. gram.:adj.}
\end{itemize}
\begin{itemize}
\item {Utilização:Pharm.}
\end{itemize}
Diz-se do medicamento officinal, já preparado; officinal.
\section{Chrono}
\begin{itemize}
\item {Grp. gram.:m.}
\end{itemize}
\begin{itemize}
\item {Utilização:Geol.}
\end{itemize}
\begin{itemize}
\item {Proveniência:(Do gr. \textunderscore khronos\textunderscore )}
\end{itemize}
Lapso de tempo, correspondente a um \textunderscore andar\textunderscore , uma das subdivisões das séries em que se divide o conjunto dos terrenos sedimentares.
\section{Chronogramma}
\begin{itemize}
\item {Grp. gram.:m.}
\end{itemize}
\begin{itemize}
\item {Grp. gram.:m.}
\end{itemize}
\begin{itemize}
\item {Proveniência:(Do gr. \textunderscore khronos\textunderscore  + \textunderscore gramma\textunderscore )}
\end{itemize}
Data enigmática, formada de letras numeraes romanas, espalhadas por differentes palavras de que fazem parte.
Inscripção, em que as letras numeraes, em cifras romanas, indicam a data de um acontecimento.
\section{Chronogrammático}
\begin{itemize}
\item {Grp. gram.:adj.}
\end{itemize}
Que contém chronogramma.
\section{Chronographia}
\begin{itemize}
\item {Proveniência:(Do gr. \textunderscore khronos\textunderscore  + \textunderscore graphein\textunderscore )}
\end{itemize}
\textunderscore f.\textunderscore  (e der.)
O mesmo que \textunderscore chronologia\textunderscore , etc.
\section{Chronográphico}
\begin{itemize}
\item {Grp. gram.:adj.}
\end{itemize}
Relativo á chronographia.
\section{Chronologia}
\begin{itemize}
\item {Grp. gram.:f.}
\end{itemize}
\begin{itemize}
\item {Proveniência:(Do gr. \textunderscore khronos\textunderscore  + \textunderscore logos\textunderscore )}
\end{itemize}
Tratado das divisões do tempo.
Tratado das datas hístóricas.
\section{Chronologicamente}
\begin{itemize}
\item {Grp. gram.:adv.}
\end{itemize}
\begin{itemize}
\item {Proveniência:(De \textunderscore chronológico\textunderscore )}
\end{itemize}
Segundo a ordem dos tempos.
\section{Chronológico}
\begin{itemize}
\item {Grp. gram.:adj.}
\end{itemize}
Relativo a chronologia.
\section{Chronologista}
\begin{itemize}
\item {Grp. gram.:m.}
\end{itemize}
Aquelle que é versado em chronologia.
\section{Chronólogo}
\begin{itemize}
\item {Grp. gram.:m.}
\end{itemize}
O mesmo que chronologista.
\section{Chronometria}
\begin{itemize}
\item {Grp. gram.:f.}
\end{itemize}
\begin{itemize}
\item {Proveniência:(De \textunderscore chronómetro\textunderscore )}
\end{itemize}
Medida do tempo.
\section{Chronometricamente}
\begin{itemize}
\item {Grp. gram.:adv.}
\end{itemize}
De modo chronométrico; á maneira de chronómetro.
\section{Chronométrico}
\begin{itemize}
\item {Grp. gram.:adj.}
\end{itemize}
Relativo á chronometria.
\section{Chronometrista}
\begin{itemize}
\item {Grp. gram.:m.}
\end{itemize}
Aquelle que fabríca chronómetros.
\section{Chronómetro}
\begin{itemize}
\item {Grp. gram.:m.}
\end{itemize}
\begin{itemize}
\item {Proveniência:(Do gr. \textunderscore khronos\textunderscore  + \textunderscore metron\textunderscore )}
\end{itemize}
Instrumento, com que se mede o tempo.
Relógio perfeito.
\section{Chronophotographia}
\begin{itemize}
\item {Grp. gram.:f.}
\end{itemize}
\begin{itemize}
\item {Proveniência:(Do gr. \textunderscore khronos\textunderscore  + \textunderscore photos\textunderscore  + \textunderscore graphein\textunderscore )}
\end{itemize}
Processo photográphico, para analysar os movimentos de um objecto móvel, tirando photographias instantâneas, com intervallos regularmente espaçados.
\section{Chronophotográphico}
\begin{itemize}
\item {Grp. gram.:adj.}
\end{itemize}
Relativo á chronophotographia.
\section{Chronoscópio}
\begin{itemize}
\item {Grp. gram.:m.}
\end{itemize}
O mesmo que \textunderscore chronómetro\textunderscore .
\section{Chrysállida}
\begin{itemize}
\item {Grp. gram.:f.}
\end{itemize}
\begin{itemize}
\item {Proveniência:(Gr. \textunderscore khrusallis\textunderscore , de \textunderscore khrusos\textunderscore , oiro)}
\end{itemize}
Fórma que os lepidópteros tomam, para passar do estado de lagarta para o de borboleta; casulo.
\section{Chrysallidar}
\begin{itemize}
\item {Grp. gram.:v. i.}
\end{itemize}
\begin{itemize}
\item {Proveniência:(De \textunderscore crysállida\textunderscore )}
\end{itemize}
Converter-se (a lagarta) em chrysállida ou nympha.
\section{Chrysânthemo}
\begin{itemize}
\item {Grp. gram.:m.}
\end{itemize}
\begin{itemize}
\item {Proveniência:(Lat. \textunderscore crysanthemum\textunderscore )}
\end{itemize}
Gênero de plantas de fôlhas alternas e flôres brancas, amarelas ou rosadas; vulgarmente, \textunderscore despedidas do verão\textunderscore .
\section{Chrysantho}
\begin{itemize}
\item {Grp. gram.:m.}
\end{itemize}
\begin{itemize}
\item {Proveniência:(Do gr. \textunderscore khrusos\textunderscore  + \textunderscore anthos\textunderscore )}
\end{itemize}
O mesmo que \textunderscore chrysânthemo\textunderscore . Cf. M. Bernárdez, \textunderscore Floresta\textunderscore , II, 244.
\section{Chryselephantina}
\begin{itemize}
\item {Grp. gram.:f.}
\end{itemize}
\begin{itemize}
\item {Proveniência:(Do gr. \textunderscore khrusos\textunderscore  + \textunderscore elephas\textunderscore )}
\end{itemize}
Dizia-se a esculptura, em que entrava oiro e marfim.
\section{Chrýseo}
\begin{itemize}
\item {Grp. gram.:adj.}
\end{itemize}
\begin{itemize}
\item {Proveniência:(Lat. \textunderscore chryseus\textunderscore )}
\end{itemize}
Feito de oiro; doirado.
\section{Chrýside}
\begin{itemize}
\item {Grp. gram.:f.}
\end{itemize}
Espécie de vespa amarela, que serve de typo aos chrysídidos.
\section{Chrysídidas}
\begin{itemize}
\item {Grp. gram.:f. pl.}
\end{itemize}
\begin{itemize}
\item {Proveniência:(Do gr. \textunderscore khrusos\textunderscore  + \textunderscore eidos\textunderscore )}
\end{itemize}
Família de insectos hymenópteros, que tem por typo a crýside.
\section{Chrysídidos}
\begin{itemize}
\item {Grp. gram.:f. pl.}
\end{itemize}
\begin{itemize}
\item {Proveniência:(Do gr. \textunderscore khrusos\textunderscore  + \textunderscore eidos\textunderscore )}
\end{itemize}
Família de insectos hymenópteros, que tem por typo a crýside.
\section{Chrýsis}
\begin{itemize}
\item {Grp. gram.:f.}
\end{itemize}
\begin{itemize}
\item {Proveniência:(Do gr. \textunderscore khrusos\textunderscore , oiro)}
\end{itemize}
Designação scientífica da vespa doirada.
\section{Chrysobullo}
\begin{itemize}
\item {Grp. gram.:m.}
\end{itemize}
Diploma com sêllo doirado.
(B. gr. \textunderscore khrusobullon\textunderscore )
\section{Chrysócalo}
\begin{itemize}
\item {Grp. gram.:m.}
\end{itemize}
\begin{itemize}
\item {Utilização:Fig.}
\end{itemize}
\begin{itemize}
\item {Proveniência:(Do gr. \textunderscore khrusos\textunderscore , oiro + \textunderscore kalos\textunderscore , bello)}
\end{itemize}
Aquillo que imita oiro.
Aquillo que só é bom na apparência.
\section{Chrysocarpo}
\begin{itemize}
\item {Grp. gram.:adj.}
\end{itemize}
\begin{itemize}
\item {Proveniência:(Do gr. \textunderscore khrusos\textunderscore  + \textunderscore karpos\textunderscore )}
\end{itemize}
Que tem frutos côr de oiro.
\section{Chrysocéphalo}
\begin{itemize}
\item {Grp. gram.:adj.}
\end{itemize}
\begin{itemize}
\item {Proveniência:(Do gr. \textunderscore khrusos\textunderscore , oiro, e \textunderscore kephale\textunderscore , cabeça)}
\end{itemize}
Que tem a cabeça ou o cimo da côr de oiro.
\section{Chrysochalo}
\begin{itemize}
\item {fónica:cá}
\end{itemize}
\begin{itemize}
\item {Grp. gram.:m.}
\end{itemize}
\begin{itemize}
\item {Proveniência:(Do gr. \textunderscore khrusos\textunderscore  + \textunderscore khalos\textunderscore )}
\end{itemize}
Liga de cobre e oiro.
\section{Chrysochloro}
\begin{itemize}
\item {Grp. gram.:adj.}
\end{itemize}
\begin{itemize}
\item {Proveniência:(Do gr. \textunderscore khrusos\textunderscore  + \textunderscore khloros\textunderscore )}
\end{itemize}
Auri-verde.
\section{Chrysocolla}
\begin{itemize}
\item {Grp. gram.:f.}
\end{itemize}
\begin{itemize}
\item {Proveniência:(Do gr. \textunderscore khrusos\textunderscore  + \textunderscore kholla\textunderscore )}
\end{itemize}
Designação antiga do bórax.
\section{Chrysócoma}
\begin{itemize}
\item {Grp. gram.:adj.}
\end{itemize}
\begin{itemize}
\item {Proveniência:(Do gr. \textunderscore khrusos\textunderscore  + \textunderscore kome\textunderscore )}
\end{itemize}
Planta exótica, de flôres amarelas.
\section{Chrysogastro}
\begin{itemize}
\item {Grp. gram.:adj.}
\end{itemize}
\begin{itemize}
\item {Proveniência:(Do gr. \textunderscore khrusos\textunderscore  + \textunderscore gaster\textunderscore )}
\end{itemize}
Que tem o ventre da côr do oiro, (falando-se de certos animaes).
\section{Chrysoglyphia}
\begin{itemize}
\item {Grp. gram.:f.}
\end{itemize}
Processo da gravura em relêvo sôbre cobre, que se executa por meio do oiro e agentes chímicos.
\section{Chrysographia}
\begin{itemize}
\item {Grp. gram.:f.}
\end{itemize}
Arte de escrever em letras de oiro.
(Cp. \textunderscore chrysógrapho\textunderscore )
\section{Chrysógrapho}
\begin{itemize}
\item {Grp. gram.:m.}
\end{itemize}
\begin{itemize}
\item {Proveniência:(Do gr. \textunderscore khrusos\textunderscore  + \textunderscore graphein\textunderscore )}
\end{itemize}
Aquelle que escreve em letras de oiro.
\section{Chrysólitha}
\begin{itemize}
\item {Grp. gram.:f.}
\end{itemize}
\begin{itemize}
\item {Proveniência:(Gr. \textunderscore chrusolithos\textunderscore )}
\end{itemize}
Pedra preciosa, da côr do oiro.
\section{Chrysólitho}
\begin{itemize}
\item {Grp. gram.:m.}
\end{itemize}
O mesmo ou melhor que \textunderscore chrysólitha\textunderscore .
\section{Chrysologia}
\begin{itemize}
\item {Grp. gram.:f.}
\end{itemize}
\begin{itemize}
\item {Utilização:P. us.}
\end{itemize}
O mesmo que \textunderscore chrysonomia\textunderscore .
\section{Chrysólogo}
\begin{itemize}
\item {Grp. gram.:adj.}
\end{itemize}
\begin{itemize}
\item {Proveniência:(Do gr. \textunderscore khrusos\textunderscore  + \textunderscore logos\textunderscore )}
\end{itemize}
Que tem palavras de oiro, (como se dizia de alguns Padres da Igreja).
\section{Chrysómela}
\begin{itemize}
\item {Grp. gram.:f.}
\end{itemize}
\begin{itemize}
\item {Proveniência:(Do gr. \textunderscore khrusos\textunderscore  + \textunderscore melos\textunderscore )}
\end{itemize}
Insecto herbívoro, espécie de escaravelho.
\section{Chrysómelo}
\begin{itemize}
\item {Grp. gram.:m.}
\end{itemize}
\begin{itemize}
\item {Proveniência:(Do gr. \textunderscore khrusos\textunderscore  + \textunderscore melos\textunderscore )}
\end{itemize}
Insecto herbívoro, espécie de escaravelho.
\section{Chrysopeia}
\begin{itemize}
\item {Grp. gram.:f.}
\end{itemize}
\begin{itemize}
\item {Proveniência:(Do gr. \textunderscore khrusos\textunderscore  + \textunderscore poiein\textunderscore )}
\end{itemize}
Supposta arte de fazer oiro.
\section{Chrysophânico}
\begin{itemize}
\item {Grp. gram.:adj.}
\end{itemize}
Diz-se de um ácido, extrahido do rhuibarbo.
\section{Chrysophtalmo}
\begin{itemize}
\item {Grp. gram.:adj.}
\end{itemize}
\begin{itemize}
\item {Proveniência:(Do gr. \textunderscore khrusos\textunderscore  + \textunderscore ophthalmos\textunderscore )}
\end{itemize}
Que tem olhos doirados, (falando-se de certos animaes).
\section{Chrysophyllo}
\begin{itemize}
\item {Grp. gram.:adj.}
\end{itemize}
\begin{itemize}
\item {Proveniência:(Do gr. \textunderscore khrusos\textunderscore  + \textunderscore phullon\textunderscore )}
\end{itemize}
Que tem fôlhas doiradas.
Árvore fructífera do Brasil.
\section{Chrysopraso}
\begin{itemize}
\item {Grp. gram.:m.}
\end{itemize}
\begin{itemize}
\item {Proveniência:(Do gr. \textunderscore khrusos\textunderscore  + \textunderscore prasos\textunderscore )}
\end{itemize}
Variedade de ágata.
\section{Chrysóptero}
\begin{itemize}
\item {Grp. gram.:adj.}
\end{itemize}
\begin{itemize}
\item {Proveniência:(Do gr. \textunderscore khrusos\textunderscore  + \textunderscore pteron\textunderscore )}
\end{itemize}
Que tem asas doiradas.
\section{Chrysorhamnina}
\begin{itemize}
\item {Grp. gram.:f.}
\end{itemize}
Substância còrante de uma espécie de rhamno, (\textunderscore rhamnus amygdalinus\textunderscore ).
\section{Chrysóstomo}
\begin{itemize}
\item {Grp. gram.:adj.}
\end{itemize}
\begin{itemize}
\item {Utilização:Fig.}
\end{itemize}
\begin{itemize}
\item {Proveniência:(Gr. \textunderscore krusostomos\textunderscore )}
\end{itemize}
Que tem bôca doirada.
Eloquente.
\section{Chthoniano}
\begin{itemize}
\item {Grp. gram.:adj.}
\end{itemize}
\begin{itemize}
\item {Proveniência:(Do gr. \textunderscore khthon\textunderscore , terra)}
\end{itemize}
Diz-se, em Mythologia, dos deuses que residem nas cavidades da terra.
Relativo ao culto dêsses deuses.
\section{Chthónico}
\begin{itemize}
\item {Grp. gram.:adj.}
\end{itemize}
O mesmo que \textunderscore chthoniano\textunderscore .
\section{Chuanga}
\begin{itemize}
\item {Grp. gram.:m.}
\end{itemize}
Escravo de emphyteuta, nos prazos de Moçambique. Cf. Gamito, \textunderscore Muata\textunderscore .
\section{Chubé}
\begin{itemize}
\item {Grp. gram.:m.}
\end{itemize}
\begin{itemize}
\item {Utilização:Bras}
\end{itemize}
Bebida, o mesmo que \textunderscore chibé\textunderscore ^1.
\section{Chuca}
\begin{itemize}
\item {Grp. gram.:f.}
\end{itemize}
\begin{itemize}
\item {Proveniência:(Al. \textunderscore chouc\textunderscore )}
\end{itemize}
Espécie de gralha.
\section{Chuça}
\begin{itemize}
\item {Grp. gram.:f.}
\end{itemize}
\begin{itemize}
\item {Utilização:ant.}
\end{itemize}
\begin{itemize}
\item {Utilização:Fam.}
\end{itemize}
(V.chuço)
Lanceta.
\section{Chuçada}
\begin{itemize}
\item {Grp. gram.:f.}
\end{itemize}
\begin{itemize}
\item {Proveniência:(De \textunderscore chuçar\textunderscore )}
\end{itemize}
Golpe de chuço ou de instrumento semelhante.
\section{Chuçar}
\begin{itemize}
\item {Grp. gram.:v. t.}
\end{itemize}
Ferir ou impellir com chuço.
\section{Chuceiro}
\begin{itemize}
\item {Grp. gram.:m.}
\end{itemize}
\begin{itemize}
\item {Utilização:Des.}
\end{itemize}
Homem, armado de chuço.
\section{Chucha}
\begin{itemize}
\item {Grp. gram.:f.}
\end{itemize}
\begin{itemize}
\item {Utilização:Infant.}
\end{itemize}
\begin{itemize}
\item {Grp. gram.:Loc.}
\end{itemize}
\begin{itemize}
\item {Utilização:fam.}
\end{itemize}
\begin{itemize}
\item {Proveniência:(De \textunderscore chuchar\textunderscore )}
\end{itemize}
Acto de \textunderscore chuchar\textunderscore .
Mama.
Alimento.
Boneca ou trapo embebido em leite ou em água açucarada, e que as crianças chucham.
\textunderscore A chucha calada\textunderscore , levando pancada e calando-se.
\section{Chuchadeira}
\begin{itemize}
\item {Grp. gram.:f.}
\end{itemize}
\begin{itemize}
\item {Utilização:Prov.}
\end{itemize}
\begin{itemize}
\item {Utilização:chul.}
\end{itemize}
\begin{itemize}
\item {Proveniência:(De \textunderscore chuchar\textunderscore )}
\end{itemize}
Chucha.
Mangação; desfrute.
\section{Chuchamel}
\begin{itemize}
\item {Grp. gram.:m.}
\end{itemize}
(V.chupamel)
\section{Chuchapitos}
\begin{itemize}
\item {Grp. gram.:m.}
\end{itemize}
\begin{itemize}
\item {Utilização:Prov.}
\end{itemize}
\begin{itemize}
\item {Utilização:dur.}
\end{itemize}
\begin{itemize}
\item {Grp. gram.:M. pl.}
\end{itemize}
\begin{itemize}
\item {Utilização:Prov.}
\end{itemize}
\begin{itemize}
\item {Utilização:minh.}
\end{itemize}
Insecto de longas patas, que habita sítios escuros, e a que o povo attribue a morte de pintaínhos.
Espécie de planta, (\textunderscore lamium maculatum\textunderscore , Lin.).
\section{Chuchar}
\begin{itemize}
\item {Grp. gram.:v. t.}
\end{itemize}
\begin{itemize}
\item {Utilização:Fig.}
\end{itemize}
\begin{itemize}
\item {Grp. gram.:V. i.}
\end{itemize}
\begin{itemize}
\item {Utilização:Chul.}
\end{itemize}
\begin{itemize}
\item {Proveniência:(Lat. \textunderscore hyp. suctiare\textunderscore )}
\end{itemize}
Sugar.
Mamar.
Receber, adquirir.
Fazer caçoada, mangar.
\section{Chuchas}
\begin{itemize}
\item {Grp. gram.:f. pl.}
\end{itemize}
\begin{itemize}
\item {Utilização:T. de Amarante}
\end{itemize}
O mesmo que \textunderscore chuchapitos\textunderscore .
\section{Chuchicala}
\begin{itemize}
\item {Grp. gram.:f.}
\end{itemize}
Chucha calada.
(Cp. \textunderscore chucha\textunderscore )
\section{Chuchicalha}
\begin{itemize}
\item {Grp. gram.:f.}
\end{itemize}
\begin{itemize}
\item {Utilização:Prov.}
\end{itemize}
\begin{itemize}
\item {Utilização:trasm.}
\end{itemize}
Chucha calada.
(Cp. \textunderscore chucha\textunderscore )
\section{Chuchu}
\begin{itemize}
\item {Grp. gram.:m.}
\end{itemize}
\begin{itemize}
\item {Proveniência:(Fr. \textunderscore chouchou\textunderscore )}
\end{itemize}
Planta cucurbitácea hortense.
\section{Chuchurreado}
\begin{itemize}
\item {Grp. gram.:adj.}
\end{itemize}
Ruidoso e demorado, (falando-se do beijo).
\section{Chuchurrear}
\begin{itemize}
\item {Grp. gram.:v. i.}
\end{itemize}
O mesmo que \textunderscore gorgolejar\textunderscore :«\textunderscore chuchurreou o porto adocicado pelas cavacas\textunderscore ». Camillo, \textunderscore Myst. de Fafe\textunderscore .
\section{Chuchurrubiu}
\begin{itemize}
\item {Grp. gram.:m.}
\end{itemize}
\begin{itemize}
\item {Utilização:des.}
\end{itemize}
\begin{itemize}
\item {Utilização:Fam.}
\end{itemize}
Maroto; tunante.
Aquelle que traz chapéu formando cantos ou amachucado.
(Cp. \textunderscore chichorrobio\textunderscore )
\section{Chuço}
\begin{itemize}
\item {Grp. gram.:m.}
\end{itemize}
\begin{itemize}
\item {Utilização:Prov.}
\end{itemize}
\begin{itemize}
\item {Utilização:Des.}
\end{itemize}
\begin{itemize}
\item {Grp. gram.:M.  e  adj.}
\end{itemize}
\begin{itemize}
\item {Utilização:Prov.}
\end{itemize}
\begin{itemize}
\item {Utilização:trasm.}
\end{itemize}
Vara ou pau, armado de aguilhão ou choupa.
Peixe de Portugal.
O mesmo que \textunderscore bofetada\textunderscore . (Colhido na Bairrada)
Soldado miliciano, (nos princípios do séc. XIX). Cf. Macedo, \textunderscore Burros\textunderscore .
Aquelle que não é judeu ou que não pertence á descendência hebraica.
\section{Chuços}
\begin{itemize}
\item {Grp. gram.:m. pl.}
\end{itemize}
\begin{itemize}
\item {Utilização:Gír. lisb.}
\end{itemize}
Tamancos.
\section{Chucro}
\begin{itemize}
\item {Grp. gram.:adj.}
\end{itemize}
\begin{itemize}
\item {Utilização:Bras. do S}
\end{itemize}
Selvagem; bravio.
Intratável.
(Do peruano \textunderscore chicaro\textunderscore )
\section{Chucurus}
\begin{itemize}
\item {Grp. gram.:m. pl.}
\end{itemize}
Arborígens de Pernambuco.
\section{Chué}
\begin{itemize}
\item {Grp. gram.:adj.}
\end{itemize}
Mal preparado, reles.
Apoucado.
(Corr. de \textunderscore soez\textunderscore ?)
\section{Chufa}
\begin{itemize}
\item {Grp. gram.:f.}
\end{itemize}
Motejo, zombaria; dito picante.
(Cast. \textunderscore chufa\textunderscore )
\section{Chufador}
\begin{itemize}
\item {Grp. gram.:m.  e  adj.}
\end{itemize}
\begin{itemize}
\item {Utilização:Des.}
\end{itemize}
\begin{itemize}
\item {Proveniência:(De \textunderscore chufar\textunderscore )}
\end{itemize}
O que diz chufas.
\section{Chufar}
\begin{itemize}
\item {Grp. gram.:v. t.}
\end{itemize}
Dirigir chufas a.
\section{Chula}
\begin{itemize}
\item {Grp. gram.:f.}
\end{itemize}
\begin{itemize}
\item {Proveniência:(De \textunderscore chulo\textunderscore )}
\end{itemize}
Variedade de dança e música popular.
Rapariga de vida airada, em Espanha.
\section{Chula}
\begin{itemize}
\item {Grp. gram.:f.}
\end{itemize}
\begin{itemize}
\item {Utilização:Prov.}
\end{itemize}
\begin{itemize}
\item {Utilização:trasm.}
\end{itemize}
O mesmo que \textunderscore enxó\textunderscore ^1. Cf. \textunderscore sula\textunderscore ^2.
\section{Chilúrico}
\begin{itemize}
\item {fónica:qui}
\end{itemize}
\begin{itemize}
\item {Grp. gram.:adj.}
\end{itemize}
\begin{itemize}
\item {Grp. gram.:M.}
\end{itemize}
Relativo á chyluria.
Aquelle que padece chyluria.
\section{Chipriota}
\begin{itemize}
\item {Grp. gram.:m.  e  adj.}
\end{itemize}
O que é de Chipre.
\section{Chularia}
\begin{itemize}
\item {Grp. gram.:f.}
\end{itemize}
(V.chulice)
\section{Chulata}
\begin{itemize}
\item {Grp. gram.:f.}
\end{itemize}
Dança chula. Cf. Cortesão, \textunderscore Subs.\textunderscore 
\section{Chulé}
\begin{itemize}
\item {Grp. gram.:m.}
\end{itemize}
\begin{itemize}
\item {Utilização:Pleb.}
\end{itemize}
Mau cheiro dos pés.
Bodum.
\section{Chulear}
\begin{itemize}
\item {Grp. gram.:v. t.}
\end{itemize}
\begin{itemize}
\item {Proveniência:(Do lat. \textunderscore subligare\textunderscore )}
\end{itemize}
Coser ligeiramente a orla de (qualquer tecido), para se não desfiar.
\section{Chuleio}
\begin{itemize}
\item {Grp. gram.:m.}
\end{itemize}
Acto ou effeito de chulear.
\section{Chuleiro}
\begin{itemize}
\item {Grp. gram.:adj.}
\end{itemize}
Que toca ou dança a chula.
Diz-se também do instrumento, em que se toca a chula: \textunderscore viola chuleira\textunderscore .
\section{Chuleta}
\begin{itemize}
\item {fónica:lê}
\end{itemize}
\begin{itemize}
\item {Grp. gram.:f.}
\end{itemize}
\begin{itemize}
\item {Proveniência:(T. cast.)}
\end{itemize}
Posta de carne. Cf. Rebello, \textunderscore Mocidade\textunderscore , III, 44.
\section{Chulice}
\begin{itemize}
\item {Grp. gram.:f.}
\end{itemize}
Dito, coisa, acção chula.
\section{Chulipa}
\begin{itemize}
\item {Grp. gram.:f.}
\end{itemize}
\begin{itemize}
\item {Proveniência:(Do ingl. \textunderscore sleeper\textunderscore )}
\end{itemize}
Nome, que se dá vulgarmente a cada uma das travessas, em que assentam os carris, nos caminhos de ferro.
\section{Chulipa}
\begin{itemize}
\item {Grp. gram.:f.}
\end{itemize}
\begin{itemize}
\item {Utilização:Pop.}
\end{itemize}
Pancada com o lado exterior do pé nas nádegas de outrem.
\section{Chulismo}
\begin{itemize}
\item {Grp. gram.:m.}
\end{itemize}
Expressão chula.
\section{Chulista}
\begin{itemize}
\item {Grp. gram.:m.}
\end{itemize}
\begin{itemize}
\item {Proveniência:(De \textunderscore chulo\textunderscore  e de \textunderscore chula\textunderscore )}
\end{itemize}
Aquelle que canta ou toca a chula.
Aquelle que faz ou diz chulices.
\section{Chulo}
\begin{itemize}
\item {Grp. gram.:adj.}
\end{itemize}
Grosseiro.
Usado na conversação da ralé.
\textunderscore Viola chula\textunderscore , violão de cordas de arame, usado pelos populares das provincias do norte.
(Cast. \textunderscore chulo\textunderscore , do ár. \textunderscore xul\textunderscore )
\section{Chulpo}
\begin{itemize}
\item {Grp. gram.:m.}
\end{itemize}
Espécie de milho da América do Sul, que tem a propriedade de se dilatar sôb a acção do fogo.
\section{Chumaçar}
\begin{itemize}
\item {Grp. gram.:v. t.}
\end{itemize}
(V.enchumaçar)
\section{Chumaceira}
\begin{itemize}
\item {Grp. gram.:f.}
\end{itemize}
\begin{itemize}
\item {Utilização:Prov.}
\end{itemize}
\begin{itemize}
\item {Utilização:dur.}
\end{itemize}
Peça, que se mete nas empolgadeiras ou noutras cavidades em que giram eixos, para abrandar o attrito.
Pedaço de madeira ou de coiro, sôbre que se move o remo, nas bordas da embarcação.
Peça de ferro, que se põe ao lado do dente da charrua ou do arado, quando êsse dente se tem adelgaçado com o uso.
Moitão, feito de um tronco de árvore e no qual assenta e se move a espadela das embarcações do Doiro.
\section{Chumacete}
\begin{itemize}
\item {fónica:cê}
\end{itemize}
\begin{itemize}
\item {Grp. gram.:m.}
\end{itemize}
(dem. de \textunderscore chumaço\textunderscore )
\section{Chumaço}
\begin{itemize}
\item {Grp. gram.:m.}
\end{itemize}
\begin{itemize}
\item {Utilização:Prov.}
\end{itemize}
\begin{itemize}
\item {Utilização:minh.}
\end{itemize}
\begin{itemize}
\item {Proveniência:(De um hypoth. \textunderscore plumaceus\textunderscore )}
\end{itemize}
Pasta de algodão em rama, entre os forros e o pano de um vestuário, para lhe altear a fórma.
Porção de pennas ou de outros objectos flexíveis, para o mesmo effeito.
Substância, com que se almofada qualquer objecto.
Pequena almofada.
Caruma sêca.
\section{Chumarra}
\begin{itemize}
\item {Grp. gram.:f.}
\end{itemize}
(V.chamarra)
\section{Chumbada}
\begin{itemize}
\item {Grp. gram.:f.}
\end{itemize}
\begin{itemize}
\item {Proveniência:(De \textunderscore chumbar\textunderscore )}
\end{itemize}
Peças de chumbo nas redes de pesca.
Tiro de chumbo.
Chumbo, que se gasta num tiro.
Ferimento com tiro de chumbo miúdo.
\section{Chumbagem}
\begin{itemize}
\item {Grp. gram.:f.}
\end{itemize}
Acto ou effeito de chumbar.
\section{Chumbar}
\begin{itemize}
\item {Grp. gram.:v. t.}
\end{itemize}
\begin{itemize}
\item {Utilização:Des.}
\end{itemize}
\begin{itemize}
\item {Utilização:Escol.}
\end{itemize}
\begin{itemize}
\item {Utilização:Pop.}
\end{itemize}
\begin{itemize}
\item {Utilização:Fig.}
\end{itemize}
\begin{itemize}
\item {Proveniência:(De \textunderscore chumbo\textunderscore )}
\end{itemize}
Prender, ligar, tapar com chumbo ou outro metal fusível.
Ferir com chumbo.
Guarnecer com pesos de chumbo.
Pôr sêllo de chumbo em.
Dar côr de chumbo a.
Reprovar.
Embriagar.
Fixar, prender muito.
\section{Chumbear}
\begin{itemize}
\item {Grp. gram.:v. t.}
\end{itemize}
Segurar com chúmbeas.
\section{Chúmbeas}
\begin{itemize}
\item {Grp. gram.:f. pl.}
\end{itemize}
\begin{itemize}
\item {Proveniência:(Do ár. \textunderscore jama'a\textunderscore )}
\end{itemize}
Peças com que se ligam os mastros estalados, para se não partirem.
(É errada a prosódia \textunderscore chuméas\textunderscore  ou \textunderscore chumbéas\textunderscore  do \textunderscore Diccion.\textunderscore  de Frei D. Vieira. Cf. Dozv.)
\section{Chumbeira}
\begin{itemize}
\item {Grp. gram.:f.}
\end{itemize}
\begin{itemize}
\item {Proveniência:(De \textunderscore chumbo\textunderscore )}
\end{itemize}
Rêde de pesca, guarnecida de chumbo; tarrafa.
Pedaço de chumbo, de espaço a espaço, na tralha inferior da rêde, para a obrigar a ir ao fundo.
\section{Chumbeiro}
\begin{itemize}
\item {Grp. gram.:m.}
\end{itemize}
Estojo de coiro, para chumbo de caça.
Bago de chumbo.
\section{Chumbim}
\begin{itemize}
\item {Grp. gram.:m.}
\end{itemize}
Official de guarda, entre os Chineses?«\textunderscore Cada um tem o seu chumbim e 20 homens de guarda\textunderscore ». \textunderscore Peregrinação\textunderscore .
\section{Chumbo}
\begin{itemize}
\item {Grp. gram.:m.}
\end{itemize}
\begin{itemize}
\item {Utilização:Fig.}
\end{itemize}
\begin{itemize}
\item {Utilização:Fam.}
\end{itemize}
\begin{itemize}
\item {Utilização:Escol.}
\end{itemize}
\begin{itemize}
\item {Proveniência:(Do lat. \textunderscore plumbum\textunderscore )}
\end{itemize}
Metal azulado, flexível e muito pesado.
Grãos dêsse metal, para caça miúda e outros usos.
Pedaços de chumbo, que guarnecem as rêdes.
Aquillo que pesa muito.
Juizo, tino.
Reprovação.
\section{Chumear}
\begin{itemize}
\item {Grp. gram.:v. t.}
\end{itemize}
O mesmo que \textunderscore chumbear\textunderscore .
\section{Chúmeas}
\begin{itemize}
\item {Grp. gram.:f. pl.}
\end{itemize}
(V.chúmbeas)
\section{Chumeco}
\begin{itemize}
\item {Grp. gram.:m.}
\end{itemize}
\begin{itemize}
\item {Utilização:Gír.}
\end{itemize}
\begin{itemize}
\item {Proveniência:(Do ingl. \textunderscore shoemaker\textunderscore )}
\end{itemize}
Sapateiro.
\section{Chumela}
\begin{itemize}
\item {Grp. gram.:f.}
\end{itemize}
\begin{itemize}
\item {Utilização:Prov.}
\end{itemize}
\begin{itemize}
\item {Utilização:alg.}
\end{itemize}
\begin{itemize}
\item {Utilização:Prov.}
\end{itemize}
\begin{itemize}
\item {Utilização:alent.}
\end{itemize}
\begin{itemize}
\item {Utilização:Prov.}
\end{itemize}
\begin{itemize}
\item {Utilização:alg.}
\end{itemize}
\begin{itemize}
\item {Proveniência:(Do rad. de \textunderscore chumaço\textunderscore )}
\end{itemize}
Pequeno travesseiro; almofada.
Compressa, que se põe na cesura de uma sangria, sotopondo-se a uma ligadura.
Faixa de recémnascido.
\section{Chumieira}
\begin{itemize}
\item {Grp. gram.:f.}
\end{itemize}
\begin{itemize}
\item {Utilização:Prov.}
\end{itemize}
\begin{itemize}
\item {Utilização:minh.}
\end{itemize}
Lumeeira, feita geralmente com um feixe de colmo.
(Corr. de \textunderscore chamieira\textunderscore , de \textunderscore chamma\textunderscore ? ou de \textunderscore lumeeira\textunderscore ?)
\section{Chuna}
\begin{itemize}
\item {Grp. gram.:f.}
\end{itemize}
Planície africana, coberta de gramíneas.
\section{Chuna}
\begin{itemize}
\item {Grp. gram.:f.}
\end{itemize}
Cal de ostras, com que se rebocam edifícios na China. Cf. Orta, \textunderscore Collóquios\textunderscore .
(Do prácrito \textunderscore chuna\textunderscore )
\section{Chunambo}
\begin{itemize}
\item {Grp. gram.:m.}
\end{itemize}
\begin{itemize}
\item {Utilização:T. de Macau e Ceilão}
\end{itemize}
O mesmo que \textunderscore chuna\textunderscore ^2.
\section{Chuname}
\begin{itemize}
\item {Grp. gram.:m.}
\end{itemize}
\begin{itemize}
\item {Utilização:T. de Gôa}
\end{itemize}
O mesmo que \textunderscore chuna\textunderscore ^2.
\section{Chupa}
\begin{itemize}
\item {Grp. gram.:f.}
\end{itemize}
\begin{itemize}
\item {Utilização:Bras. do N}
\end{itemize}
\begin{itemize}
\item {Proveniência:(De \textunderscore chupar\textunderscore )}
\end{itemize}
Laranja descascada e não partida, para que se lhe sorva o suco.
\section{Chupa-caldo}
\begin{itemize}
\item {Grp. gram.:m.}
\end{itemize}
\begin{itemize}
\item {Utilização:Bras}
\end{itemize}
Adulador, alcoviteiro.
\section{Chupadela}
\begin{itemize}
\item {Grp. gram.:f.}
\end{itemize}
Acção de chupar.
\section{Chupado}
\begin{itemize}
\item {Grp. gram.:adj.}
\end{itemize}
\begin{itemize}
\item {Utilização:Fam.}
\end{itemize}
Muito magro, esquelético.
\section{Chupadoiro}
\begin{itemize}
\item {Grp. gram.:m.}
\end{itemize}
O mesmo que \textunderscore chupeta\textunderscore .
\section{Chupador}
\begin{itemize}
\item {Grp. gram.:m.  e  adj.}
\end{itemize}
O que chupa.
\section{Chupadouro}
\begin{itemize}
\item {Grp. gram.:m.}
\end{itemize}
O mesmo que \textunderscore chupeta\textunderscore .
\section{Chupadura}
\begin{itemize}
\item {Grp. gram.:f.}
\end{itemize}
Acto de chupar.
Aquillo que se chupa de uma vez.
\section{Chupaflor}
\begin{itemize}
\item {Grp. gram.:m.}
\end{itemize}
(V.colibri)
\section{Chupa-jantares}
\begin{itemize}
\item {Grp. gram.:m.}
\end{itemize}
O mesmo que \textunderscore papa-jantares\textunderscore .
\section{Chupamel}
\begin{itemize}
\item {Grp. gram.:m.}
\end{itemize}
\begin{itemize}
\item {Utilização:Prov.}
\end{itemize}
\begin{itemize}
\item {Utilização:beir.}
\end{itemize}
\begin{itemize}
\item {Proveniência:(De \textunderscore chupar\textunderscore  + \textunderscore mel\textunderscore )}
\end{itemize}
Planta, da fam. das caprifoliáceas.
Colibri.
Flôr do sargaço.
\section{Chupamento}
\begin{itemize}
\item {Grp. gram.:m.}
\end{itemize}
Acto de chupar.
\section{Chupa-môlho}
\begin{itemize}
\item {Grp. gram.:m.}
\end{itemize}
\begin{itemize}
\item {Utilização:T. da Baía}
\end{itemize}
Espécie de pêso de carne.
\section{Chupante}
\begin{itemize}
\item {Grp. gram.:m.}
\end{itemize}
\begin{itemize}
\item {Utilização:Fam.}
\end{itemize}
O mesmo que \textunderscore chupista\textunderscore .
\section{Chupão}
\begin{itemize}
\item {Grp. gram.:m.}
\end{itemize}
\begin{itemize}
\item {Utilização:Prov.}
\end{itemize}
\begin{itemize}
\item {Utilização:trasm.}
\end{itemize}
Chaminé de cozinha.
Orifício, junto á lareira, para tiragem do fumo.
\section{Chupão}
\begin{itemize}
\item {Grp. gram.:m.}
\end{itemize}
\begin{itemize}
\item {Utilização:Pop.}
\end{itemize}
\begin{itemize}
\item {Grp. gram.:Adj.}
\end{itemize}
\begin{itemize}
\item {Proveniência:(De \textunderscore chupar\textunderscore )}
\end{itemize}
Beijo ruidoso.
Mancha, resultante da compressão dos lábios sôbre a pelle.
Que chupa.
\section{Chupar}
\begin{itemize}
\item {Grp. gram.:v. t.}
\end{itemize}
\begin{itemize}
\item {Utilização:Fig.}
\end{itemize}
Sorver; absorver.
Sugar.
Lucrar.
Comer.
Gastar, consumir.
(Cast. \textunderscore chupar\textunderscore )
\section{Chupeta}
\begin{itemize}
\item {fónica:pê}
\end{itemize}
\begin{itemize}
\item {Grp. gram.:f.}
\end{itemize}
\begin{itemize}
\item {Grp. gram.:Loc.}
\end{itemize}
\begin{itemize}
\item {Utilização:pop.}
\end{itemize}
\begin{itemize}
\item {Proveniência:(De \textunderscore chupar\textunderscore )}
\end{itemize}
Tubo, com que se chupa um líquido.
\textunderscore De chupeta\textunderscore , que é appetitoso, excellente.
\section{Chupista}
\begin{itemize}
\item {Grp. gram.:m.  e  f.}
\end{itemize}
\begin{itemize}
\item {Proveniência:(De \textunderscore chupar\textunderscore )}
\end{itemize}
Pessôa dada a bebidas alcoolicas.
Papa-jantares.
Pessôa, que explora ardilosamente a bondade alheia.
\section{Chupitar}
\begin{itemize}
\item {Grp. gram.:v. t.}
\end{itemize}
Chupar devagarinho, repetidas vezes.
\section{Churdo}
\begin{itemize}
\item {Grp. gram.:adj.}
\end{itemize}
\begin{itemize}
\item {Grp. gram.:M.}
\end{itemize}
\begin{itemize}
\item {Proveniência:(Do lat. \textunderscore surdidos\textunderscore ?)}
\end{itemize}
Diz-se de lan, antes de preparada.
Homem ruím, vil.
\section{Chureta}
\begin{itemize}
\item {fónica:churê}
\end{itemize}
\begin{itemize}
\item {Grp. gram.:m.}
\end{itemize}
\begin{itemize}
\item {Utilização:Prov.}
\end{itemize}
\begin{itemize}
\item {Utilização:minh.}
\end{itemize}
O mesmo que \textunderscore gaivina\textunderscore .
\section{Churinar}
\begin{itemize}
\item {Grp. gram.:v. t.}
\end{itemize}
\begin{itemize}
\item {Utilização:Gír.}
\end{itemize}
Esfaquear.
(Caló espanhol \textunderscore churinar\textunderscore , de \textunderscore churi\textunderscore , faca)
\section{Churmigueira}
\begin{itemize}
\item {Grp. gram.:f.}
\end{itemize}
Arbusto do Doiro.
\section{Churra, churrinha!}
\begin{itemize}
\item {Grp. gram.:interj.}
\end{itemize}
\begin{itemize}
\item {Utilização:Prov.}
\end{itemize}
\begin{itemize}
\item {Utilização:minh.}
\end{itemize}
(Serve para chamar gallinhas)
\section{Churrasco}
\begin{itemize}
\item {Grp. gram.:m.}
\end{itemize}
\begin{itemize}
\item {Utilização:Bras}
\end{itemize}
Pedaço de carne, assado em espêto ou nas brasas.
(Talvez de or. afr.)
\section{Churrasquear}
\begin{itemize}
\item {Grp. gram.:v. i.}
\end{itemize}
\begin{itemize}
\item {Utilização:Bras. do S}
\end{itemize}
\begin{itemize}
\item {Utilização:Ext.}
\end{itemize}
\begin{itemize}
\item {Proveniência:(De \textunderscore churrasco\textunderscore )}
\end{itemize}
Preparar churrasco e comê-lo.
Preparar qualquer comida.
\section{Churré}
\begin{itemize}
\item {Grp. gram.:adj.}
\end{itemize}
\begin{itemize}
\item {Utilização:Gír.}
\end{itemize}
Jovem.
\section{Churreu}
\begin{itemize}
\item {Grp. gram.:m.}
\end{itemize}
(V.charreu)
\section{Churrião}
\begin{itemize}
\item {Grp. gram.:m.}
\end{itemize}
(V.chorrião)
\section{Churro}
\begin{itemize}
\item {Grp. gram.:adj.}
\end{itemize}
(V.churdo)
\section{Churro}
\begin{itemize}
\item {Grp. gram.:m.}
\end{itemize}
Sujidade da pelle.
(Cast. \textunderscore churro\textunderscore )
\section{Chus}
\begin{itemize}
\item {Grp. gram.:m.}
\end{itemize}
(Cp. \textunderscore bus\textunderscore )
\section{Chus}
\begin{itemize}
\item {Grp. gram.:adv.}
\end{itemize}
\begin{itemize}
\item {Utilização:Ant.}
\end{itemize}
Debaixo.
(Cp. \textunderscore juso\textunderscore )
\section{Chus}
\begin{itemize}
\item {Grp. gram.:adv.}
\end{itemize}
\begin{itemize}
\item {Utilização:Ant.}
\end{itemize}
\begin{itemize}
\item {Proveniência:(Lat. \textunderscore plus\textunderscore )}
\end{itemize}
O mesmo que \textunderscore mais\textunderscore ^1. Cf. Fr. Fortun., \textunderscore Inéd.\textunderscore , I, 302.
\section{Chusma}
\begin{itemize}
\item {Grp. gram.:f.}
\end{itemize}
\begin{itemize}
\item {Utilização:Mús.}
\end{itemize}
\begin{itemize}
\item {Proveniência:(Do lat. \textunderscore celeusma\textunderscore )}
\end{itemize}
Tripulação.
Grande quantidade.
Montão.
Rancho.
Conjunto das vozes de um côro.
\section{Chusmar}
\begin{itemize}
\item {Grp. gram.:v. t.}
\end{itemize}
\begin{itemize}
\item {Utilização:Des.}
\end{itemize}
\begin{itemize}
\item {Proveniência:(De \textunderscore chusma\textunderscore )}
\end{itemize}
Guarnecer de marinhagem.
\section{Chuta!}
\begin{itemize}
\item {Grp. gram.:interj.}
\end{itemize}
O mesmo que \textunderscore caluda!\textunderscore 
\section{Chuva}
\begin{itemize}
\item {Grp. gram.:f.}
\end{itemize}
\begin{itemize}
\item {Utilização:Fig.}
\end{itemize}
\begin{itemize}
\item {Proveniência:(Lat. \textunderscore pluvia\textunderscore )}
\end{itemize}
Água, que cái da atmosphera.
Aquillo que cái do ar em abundância.
Grande quantidade, abundância.
\section{Chuvaceira}
\begin{itemize}
\item {Grp. gram.:f.}
\end{itemize}
\begin{itemize}
\item {Utilização:Ant.}
\end{itemize}
O mesmo que \textunderscore chuveiro\textunderscore .
\section{Chuvada}
\begin{itemize}
\item {Grp. gram.:f.}
\end{itemize}
Chuva forte.
\section{Chuva-de-oiro}
\begin{itemize}
\item {Grp. gram.:f.}
\end{itemize}
\begin{itemize}
\item {Utilização:Bras}
\end{itemize}
Planta, o mesmo que \textunderscore cytiso\textunderscore .
\section{Chuvarada}
\begin{itemize}
\item {Grp. gram.:f.}
\end{itemize}
\begin{itemize}
\item {Utilização:Bras}
\end{itemize}
O mesmo que \textunderscore chuvada\textunderscore .
\section{Chuvedice}
\begin{itemize}
\item {Grp. gram.:f.}
\end{itemize}
\begin{itemize}
\item {Utilização:Prov.}
\end{itemize}
\begin{itemize}
\item {Utilização:minh.}
\end{itemize}
Água da chuva. (Colhido em Barcelos)
\section{Chuvediço}
\begin{itemize}
\item {Grp. gram.:adj.}
\end{itemize}
(V.chovediço)
\section{Chuveirão}
\begin{itemize}
\item {Grp. gram.:m.}
\end{itemize}
Grande bátega de água. Cf. Rebello, \textunderscore Contos e lendas\textunderscore , 83.
\section{Chuveiro}
\begin{itemize}
\item {Grp. gram.:m.}
\end{itemize}
\begin{itemize}
\item {Utilização:Fig.}
\end{itemize}
\begin{itemize}
\item {Utilização:Ant.}
\end{itemize}
\begin{itemize}
\item {Proveniência:(De \textunderscore chuva\textunderscore )}
\end{itemize}
Chuva abundante, mas passageira.
Grande porção de coisas, que cáem ou que se succedem com rapidez.
Crivo, por onde sai água dos regadores.
Pulverizador de líquido aromático. Cf. M. Bernárdez, \textunderscore Floresta\textunderscore .
\section{Chuvenisca}
\begin{itemize}
\item {Grp. gram.:m.  e  f.}
\end{itemize}
\begin{itemize}
\item {Utilização:Prov.}
\end{itemize}
\begin{itemize}
\item {Utilização:alg.}
\end{itemize}
Criança brincalhona, travêssa.
\section{Chuveniscar}
\begin{itemize}
\item {Grp. gram.:v. i.}
\end{itemize}
\begin{itemize}
\item {Utilização:Prov.}
\end{itemize}
\begin{itemize}
\item {Utilização:alg.}
\end{itemize}
O mesmo que \textunderscore chuviscar\textunderscore .
\section{Chuvenisco}
\begin{itemize}
\item {Grp. gram.:m.}
\end{itemize}
\begin{itemize}
\item {Utilização:Prov.}
\end{itemize}
\begin{itemize}
\item {Utilização:alg.}
\end{itemize}
O mesmo que \textunderscore chuvisco\textunderscore .
\section{Chuvinhar}
\begin{itemize}
\item {Grp. gram.:v. i.}
\end{itemize}
\begin{itemize}
\item {Utilização:Prov.}
\end{itemize}
\begin{itemize}
\item {Utilização:alg.}
\end{itemize}
\begin{itemize}
\item {Proveniência:(De \textunderscore chuvinha\textunderscore , de \textunderscore chuva\textunderscore )}
\end{itemize}
O mesmo que \textunderscore chuviscar\textunderscore .
\section{Chuviscar}
\begin{itemize}
\item {Grp. gram.:v. i.}
\end{itemize}
Cair chuvisco.
\section{Chuvisco}
\begin{itemize}
\item {Grp. gram.:m.}
\end{itemize}
\begin{itemize}
\item {Proveniência:(De \textunderscore chuva\textunderscore )}
\end{itemize}
Chuva miúda.
\section{Chuvisqueiro}
\begin{itemize}
\item {Grp. gram.:m.}
\end{itemize}
\begin{itemize}
\item {Utilização:Bras. do S}
\end{itemize}
O mesmo que \textunderscore chuvisco\textunderscore .
\section{Chuvoso}
\begin{itemize}
\item {Grp. gram.:adj.}
\end{itemize}
\begin{itemize}
\item {Proveniência:(Lat. \textunderscore pluviosus\textunderscore )}
\end{itemize}
Em que há chuva: \textunderscore dia chuvoso\textunderscore .
\section{Chuxo}
\begin{itemize}
\item {Grp. gram.:m.}
\end{itemize}
\begin{itemize}
\item {Utilização:Prov.}
\end{itemize}
\begin{itemize}
\item {Utilização:alg.}
\end{itemize}
Peixe, espécie de raia.
\section{Chylífero}
\begin{itemize}
\item {fónica:qui}
\end{itemize}
\begin{itemize}
\item {Grp. gram.:adj.}
\end{itemize}
\begin{itemize}
\item {Utilização:Physiol.}
\end{itemize}
\begin{itemize}
\item {Proveniência:(De \textunderscore chylo\textunderscore  + lat. \textunderscore ferre\textunderscore )}
\end{itemize}
Por onde passa o chylo.
\section{Chylificação}
\begin{itemize}
\item {fónica:qui}
\end{itemize}
\begin{itemize}
\item {Grp. gram.:f.}
\end{itemize}
Acto de chylificar.
\section{Chylificar}
\begin{itemize}
\item {fónica:qui}
\end{itemize}
\begin{itemize}
\item {Grp. gram.:v.}
\end{itemize}
\begin{itemize}
\item {Utilização:t. Physiol.}
\end{itemize}
\begin{itemize}
\item {Proveniência:(De \textunderscore chylo\textunderscore  + lat. \textunderscore facere\textunderscore )}
\end{itemize}
Converter em chylo.
\section{Chylo}
\begin{itemize}
\item {fónica:qui}
\end{itemize}
\begin{itemize}
\item {Grp. gram.:m.}
\end{itemize}
\begin{itemize}
\item {Proveniência:(Gr. \textunderscore khulos\textunderscore )}
\end{itemize}
Parte líquida da digestão.
\section{Chylologia}
\begin{itemize}
\item {fónica:qui}
\end{itemize}
\begin{itemize}
\item {Grp. gram.:f.}
\end{itemize}
\begin{itemize}
\item {Proveniência:(Do gr. \textunderscore khulos\textunderscore  + \textunderscore logos\textunderscore )}
\end{itemize}
Tratado sôbre o chylo.
\section{Chylose}
\begin{itemize}
\item {fónica:qui}
\end{itemize}
\begin{itemize}
\item {Grp. gram.:f.}
\end{itemize}
O mesmo que \textunderscore chylificação\textunderscore .
\section{Chyloso}
\begin{itemize}
\item {fónica:qui}
\end{itemize}
\begin{itemize}
\item {Grp. gram.:adj.}
\end{itemize}
Relativo ao chylo.
\section{Chyluria}
\begin{itemize}
\item {fónica:qui}
\end{itemize}
\begin{itemize}
\item {Grp. gram.:f.}
\end{itemize}
\begin{itemize}
\item {Proveniência:(De \textunderscore chylo\textunderscore  + gr. \textunderscore ouron\textunderscore )}
\end{itemize}
Estado mórbido, determinado pela presença de chylo na urina.
\section{Chymificação}
\begin{itemize}
\item {fónica:qui}
\end{itemize}
\begin{itemize}
\item {Grp. gram.:f.}
\end{itemize}
Acto de chymificar.
\section{Chymificar}
\begin{itemize}
\item {fónica:qui}
\end{itemize}
\begin{itemize}
\item {Grp. gram.:v. t.}
\end{itemize}
\begin{itemize}
\item {Proveniência:(De \textunderscore chymo\textunderscore  + lat. \textunderscore facere\textunderscore )}
\end{itemize}
Converter em chymo.
\section{Chymo}
\begin{itemize}
\item {fónica:qui}
\end{itemize}
\begin{itemize}
\item {Grp. gram.:m.}
\end{itemize}
\begin{itemize}
\item {Proveniência:(Gr. \textunderscore khumos\textunderscore , suco)}
\end{itemize}
Alimentos reduzidos a uma pasta pela digestão estomacal.
\section{Chymophylla}
\begin{itemize}
\item {fónica:qui}
\end{itemize}
\begin{itemize}
\item {Grp. gram.:m.}
\end{itemize}
\begin{itemize}
\item {Proveniência:(Do gr. \textunderscore khumos\textunderscore  + \textunderscore phullon\textunderscore )}
\end{itemize}
Producto pharmacêutico, contra as manifestações mórbidas da primeira dentição.
\section{Chymosina}
\begin{itemize}
\item {fónica:qui}
\end{itemize}
\begin{itemize}
\item {Grp. gram.:f.}
\end{itemize}
Producto pharmacêutico, segregado na mucosa gástrica dos mammíferos e das aves.
\section{Chypriota}
\begin{itemize}
\item {Grp. gram.:m.  e  adj.}
\end{itemize}
O que é de Chypre.
\section{Cia}
\begin{itemize}
\item {Grp. gram.:f.}
\end{itemize}
O mesmo que \textunderscore cicia\textunderscore .
Nome, que se dá a uma espécie de cotovia, (\textunderscore anthus arboreus\textunderscore , Bechst), também conhecida por \textunderscore sombria\textunderscore ; e a outra espécie, (\textunderscore anthus pratensis\textunderscore ), também conhecida por \textunderscore petinha\textunderscore  e \textunderscore cioto\textunderscore .
\section{Ciado}
\begin{itemize}
\item {Grp. gram.:adj.}
\end{itemize}
\begin{itemize}
\item {Utilização:Ant.}
\end{itemize}
\begin{itemize}
\item {Proveniência:(De \textunderscore ciar\textunderscore ^1)}
\end{itemize}
Que causa ciumes. Cf. \textunderscore Vir. Trág.\textunderscore , IX, 104.
\section{Ciar}
\begin{itemize}
\item {Grp. gram.:v. t.}
\end{itemize}
\begin{itemize}
\item {Utilização:Ant.}
\end{itemize}
\begin{itemize}
\item {Grp. gram.:V. i.}
\end{itemize}
Têr ciumes de.
Têr ciumes:«\textunderscore ella tinha cujo, e ciava a outra irmãa.\textunderscore »\textunderscore Eufrosina\textunderscore , 85.
\section{Ciar}
\begin{itemize}
\item {Grp. gram.:v. i.}
\end{itemize}
Remar para trás.
\section{Ciavoga}
\begin{itemize}
\item {Grp. gram.:f.}
\end{itemize}
\begin{itemize}
\item {Proveniência:(De \textunderscore ciar\textunderscore ^2 + \textunderscore vogar\textunderscore )}
\end{itemize}
Movimento do barco, vogando os remeiros de um lado e ciando os do outro.
\section{Cibalho}
\begin{itemize}
\item {Grp. gram.:m.}
\end{itemize}
\begin{itemize}
\item {Proveniência:(De \textunderscore cibo\textunderscore )}
\end{itemize}
Alimento das aves bravas.
\section{Cibana}
\begin{itemize}
\item {Grp. gram.:f.}
\end{itemize}
\begin{itemize}
\item {Utilização:Prov.}
\end{itemize}
\begin{itemize}
\item {Utilização:beir.}
\end{itemize}
Carga de lenha miúda, formada por três feixes.
\section{Cibando}
\begin{itemize}
\item {Grp. gram.:m.}
\end{itemize}
Ave de rapina.
\section{Cibaque}
\begin{itemize}
\item {Grp. gram.:m.}
\end{itemize}
\begin{itemize}
\item {Utilização:Prov.}
\end{itemize}
\begin{itemize}
\item {Utilização:beir.}
\end{itemize}
Cibo, que a ave leva á sua ninhada.
\section{Cibário}
\begin{itemize}
\item {Grp. gram.:m.}
\end{itemize}
\begin{itemize}
\item {Proveniência:(Lat. \textunderscore cibarium\textunderscore )}
\end{itemize}
Regulamento, sôbre provisões alimentícias, entre os Romanos.
Farinha ordinária, ou a que se separa da mais fina pela joeira.
\section{Cibarrada}
\begin{itemize}
\item {Grp. gram.:f.}
\end{itemize}
\begin{itemize}
\item {Utilização:Prov.}
\end{itemize}
Cibalho, cibana. (Colhido na Bairrada)
\section{Cibato}
\begin{itemize}
\item {Grp. gram.:m.}
\end{itemize}
\begin{itemize}
\item {Proveniência:(Lat. \textunderscore cibatus\textunderscore )}
\end{itemize}
O mesmo que \textunderscore cibalho\textunderscore .
\section{Cibo}
\begin{itemize}
\item {Grp. gram.:m.}
\end{itemize}
\begin{itemize}
\item {Utilização:Fam.}
\end{itemize}
\begin{itemize}
\item {Proveniência:(Lat. \textunderscore cibus\textunderscore )}
\end{itemize}
Comida, alimento, (especialmente das aves).
Pequena porção de qualquer alimento ou de qualquer coisa.
\section{Cibôa}
\begin{itemize}
\item {Grp. gram.:f.}
\end{itemize}
Espécie de palmeira africana.
\section{Cíbolo}
\begin{itemize}
\item {Grp. gram.:m.}
\end{itemize}
Toiro do México.
\section{Cibório}
\begin{itemize}
\item {Grp. gram.:m.}
\end{itemize}
\begin{itemize}
\item {Proveniência:(Lat. \textunderscore ciborium\textunderscore )}
\end{itemize}
Vaso, em que se guardam as hóstias ou partículas consagradas.
Vaso, em que os antigos navegantes levavam seu alimento.
\section{Cica}
\begin{itemize}
\item {Grp. gram.:f.}
\end{itemize}
\begin{itemize}
\item {Utilização:Bras}
\end{itemize}
Adstringência peculiar a certas frutas, especialmente ás que não estão bem sazonadas.
Pequena e bonita palmeira, cultivada especialmente em jardins.
(Talvez do tupi)
\section{Cica}
\begin{itemize}
\item {Grp. gram.:m.}
\end{itemize}
\begin{itemize}
\item {Utilização:Ant.}
\end{itemize}
T. injurioso?«\textunderscore Ó caca, ó cica!\textunderscore »G. Vicente, I, 216.
\section{Cicadárias}
\begin{itemize}
\item {Grp. gram.:f. pl.}
\end{itemize}
\begin{itemize}
\item {Proveniência:(Do lat. \textunderscore cicada\textunderscore )}
\end{itemize}
Família de insectos, que têm por typo a cigarra.
\section{Cicateiro}
\begin{itemize}
\item {Grp. gram.:adj.}
\end{itemize}
\begin{itemize}
\item {Utilização:Prov.}
\end{itemize}
\begin{itemize}
\item {Utilização:trasm.}
\end{itemize}
\begin{itemize}
\item {Utilização:beir.}
\end{itemize}
Rabugento; que em tudo vê motivo de questão.
\section{Cicatice}
\begin{itemize}
\item {Grp. gram.:f.}
\end{itemize}
\begin{itemize}
\item {Utilização:Prov.}
\end{itemize}
\begin{itemize}
\item {Utilização:trasm.}
\end{itemize}
Caturrice; dito ou acto de \textunderscore cicateiro\textunderscore .
\section{Cicatricial}
\begin{itemize}
\item {Grp. gram.:adj.}
\end{itemize}
\begin{itemize}
\item {Utilização:Neol.}
\end{itemize}
Relativo a cicatriz.
\section{Cicatrícula}
\begin{itemize}
\item {Grp. gram.:f.}
\end{itemize}
\begin{itemize}
\item {Proveniência:(Lat. \textunderscore cicatricula\textunderscore )}
\end{itemize}
Mancha, que, na superfície da gema do ovo, indica o germe.
Ponto, em que se revela a germinação, na superfície das sementes.
\section{Cicatriz}
\begin{itemize}
\item {Grp. gram.:f.}
\end{itemize}
\begin{itemize}
\item {Utilização:Fig.}
\end{itemize}
\begin{itemize}
\item {Proveniência:(Lat. \textunderscore cicatrix\textunderscore )}
\end{itemize}
Vestígio, que a ferida deixa depois de curada.
Impressão duradoira de uma offensa ou de uma desgraça.
Vestígio, que deixam na haste as fôlhas ou ramos que cáem.
\section{Cicatrização}
\begin{itemize}
\item {Grp. gram.:f.}
\end{itemize}
Acto de cicatrizar.
\section{Cicatrizante}
\begin{itemize}
\item {Grp. gram.:adj.}
\end{itemize}
Que cicatriza.
\section{Cicatrizar}
\begin{itemize}
\item {Grp. gram.:v. t.}
\end{itemize}
\begin{itemize}
\item {Grp. gram.:V. i.}
\end{itemize}
\begin{itemize}
\item {Proveniência:(De \textunderscore cicatriz\textunderscore )}
\end{itemize}
Favorecer a cicatrização de.
Determinar o cerramento de (uma ferida) por cicatrização.
Fechar-se (a ferida) por cicatriz.
\section{Cicatrizável}
\begin{itemize}
\item {Grp. gram.:adj.}
\end{itemize}
\begin{itemize}
\item {Proveniência:(De \textunderscore cicatrizar\textunderscore )}
\end{itemize}
Que facilmente se cicatriza.
\section{Cicêndia}
\begin{itemize}
\item {Grp. gram.:f.}
\end{itemize}
Planta herbácea, da fam. das gencianáceas.
\section{Cicérico}
\begin{itemize}
\item {Grp. gram.:adj.}
\end{itemize}
\begin{itemize}
\item {Utilização:Chím.}
\end{itemize}
\begin{itemize}
\item {Proveniência:(Do lat. \textunderscore cicer\textunderscore )}
\end{itemize}
Procedente do grão de bico.
\section{Cicerone}
\begin{itemize}
\item {fónica:chicheróne}
\end{itemize}
\begin{itemize}
\item {Grp. gram.:m.}
\end{itemize}
\begin{itemize}
\item {Proveniência:(T. it.)}
\end{itemize}
Pessôa, que guia estrangeiros ou viajantes, mostrando-lhes o que há de importante numa localidade, ou dando-lhes informações que lhes interessem.
\section{Ciceronianismo}
\begin{itemize}
\item {Grp. gram.:m.}
\end{itemize}
\begin{itemize}
\item {Utilização:P. us.}
\end{itemize}
Imitação do estilo de Cícero.
\section{Ciceroniano}
\begin{itemize}
\item {Grp. gram.:adj.}
\end{itemize}
\begin{itemize}
\item {Proveniência:(Do lat. \textunderscore Cicero\textunderscore , n. p.)}
\end{itemize}
Relativo a Cícero.
Eloquente.
Elevado, como o estílo de Cícero.
\section{Cicerónico}
\begin{itemize}
\item {Grp. gram.:adj.}
\end{itemize}
O mesmo que \textunderscore ciceroniano\textunderscore . Cf. Macedo, \textunderscore Burros\textunderscore , 308.
\section{Cicia}
\begin{itemize}
\item {Grp. gram.:f.}
\end{itemize}
Pássaro conirostro, espécie de verdelhão, (\textunderscore emberiza cirius\textunderscore , Lin.).
\section{Ciciamento}
\begin{itemize}
\item {Grp. gram.:m.}
\end{itemize}
Acto de ciciar.
\section{Ciciar}
\begin{itemize}
\item {Grp. gram.:v. i.}
\end{itemize}
\begin{itemize}
\item {Grp. gram.:V. t.}
\end{itemize}
\begin{itemize}
\item {Proveniência:(De \textunderscore cicio\textunderscore )}
\end{itemize}
Rumorejar levemente.
Dizer em voz baixa.
\section{Cicica}
\begin{itemize}
\item {Grp. gram.:f.}
\end{itemize}
\begin{itemize}
\item {Utilização:Bras. do N}
\end{itemize}
O mesmo que \textunderscore caxirenguengue\textunderscore .
\section{Cicindela}
\begin{itemize}
\item {Grp. gram.:f.}
\end{itemize}
\begin{itemize}
\item {Proveniência:(Lat. \textunderscore cicindela\textunderscore )}
\end{itemize}
Gênero de insectos carnívoros.
\section{Cicindelianos}
\begin{itemize}
\item {Grp. gram.:m. pl.}
\end{itemize}
Tríbo de insectos, que têm por typo a cicindela.
\section{Cicio}
\begin{itemize}
\item {Grp. gram.:m.}
\end{itemize}
\begin{itemize}
\item {Proveniência:(T. onom.)}
\end{itemize}
Rumor brando, como o da aragem nos ramos das árvores.
Murmúrio de palavras em voz baixa.
\section{Cicioso}
\begin{itemize}
\item {Grp. gram.:adj.}
\end{itemize}
\begin{itemize}
\item {Proveniência:(De \textunderscore cicio\textunderscore )}
\end{itemize}
Que cicia.
\section{Cicisbeia}
\begin{itemize}
\item {Grp. gram.:f.}
\end{itemize}
\begin{itemize}
\item {Utilização:Prov.}
\end{itemize}
\begin{itemize}
\item {Utilização:trasm.}
\end{itemize}
Rapariga pobre, mas muito presumida e affectada.
(Por \textunderscore chichisbeia\textunderscore , de \textunderscore chichisbéu\textunderscore )
\section{Ciclamor}
\begin{itemize}
\item {Grp. gram.:m.}
\end{itemize}
Árvore leguminosa, de flôres encarnadas e fôlhas cordiformes.
\section{Ciclatão}
\begin{itemize}
\item {Grp. gram.:m.}
\end{itemize}
Tela de seda fina e preciosa, de que se faziam roçagantes e ricos trajes, para homens e mulheres.
(Cast. \textunderscore ciclaton\textunderscore )
\section{Ciclaton}
\begin{itemize}
\item {Grp. gram.:m.}
\end{itemize}
O mesmo que \textunderscore ciclatão\textunderscore . Cf. Herculano, \textunderscore Bobo\textunderscore , 151.
\section{Ciçó}
\begin{itemize}
\item {Grp. gram.:m.}
\end{itemize}
Árvore indiana, cuja madeira avermelhada se emprega em obras de marcenaria.
(Conc. \textunderscore xixó\textunderscore )
\section{Ciconídeas}
\begin{itemize}
\item {Grp. gram.:f. pl.}
\end{itemize}
\begin{itemize}
\item {Utilização:Zool.}
\end{itemize}
\begin{itemize}
\item {Proveniência:(T. hybr., do lat. \textunderscore ciconia\textunderscore  + gr. \textunderscore eidos\textunderscore )}
\end{itemize}
Ordem de aves, que têm por typo a cegonha.
\section{Cicuta}
\begin{itemize}
\item {Grp. gram.:f.}
\end{itemize}
\begin{itemize}
\item {Proveniência:(Lat. \textunderscore cicuta\textunderscore )}
\end{itemize}
Planta umbellífera, venenosa.
\section{Cicutado}
\begin{itemize}
\item {Grp. gram.:adj.}
\end{itemize}
Que tem cicuta.
Impregnado de cicuta.
\section{Cicutária}
\begin{itemize}
\item {Grp. gram.:f.}
\end{itemize}
\begin{itemize}
\item {Proveniência:(De \textunderscore cicuta\textunderscore )}
\end{itemize}
Nome de várias plantas umbellíferas, venenosas.
\section{Cicutina}
\begin{itemize}
\item {Grp. gram.:f.}
\end{itemize}
Alcali da cicuta.
\section{Cicútio}
\begin{itemize}
\item {Grp. gram.:m.}
\end{itemize}
O mesmo que \textunderscore cicútis\textunderscore .
\section{Cicútis}
\begin{itemize}
\item {Grp. gram.:m.}
\end{itemize}
Extracto de cicuta. Cf. \textunderscore Pharm. Port.\textunderscore 
\section{Cidadã}
(fem. de \textunderscore cidadão\textunderscore )
\section{Cidadan}
(fem. de \textunderscore cidadão\textunderscore )
\section{Cidadania}
\begin{itemize}
\item {Grp. gram.:f.}
\end{itemize}
\begin{itemize}
\item {Utilização:bras}
\end{itemize}
\begin{itemize}
\item {Utilização:Neol.}
\end{itemize}
Qualidade de cidadão.
\section{Cidadão}
\begin{itemize}
\item {Grp. gram.:m.}
\end{itemize}
Morador de uma cidade.
Aquelle que está no gôzo dos direitos civis e políticos de um Estado.
\section{Cidade}
\begin{itemize}
\item {Grp. gram.:f.}
\end{itemize}
\begin{itemize}
\item {Utilização:Ant.}
\end{itemize}
\begin{itemize}
\item {Proveniência:(Lat. \textunderscore civitas\textunderscore , \textunderscore civitatis\textunderscore )}
\end{itemize}
Nome commum, legalmente reconhecido, das povoações de certa categoria num país.
Os moradores de uma cidade.
Julgado ou concelho, que tinha por cabeça uma villa acastellada.
No tempo dos Godos, reunião de lugares ou povoações, situadas em planície, ao passo que as situadas em lugares altos e defendíveis por natureza e arte, se chamavam castros ou castellos.--Quando o vocábulo precede um nome da localidade, ínterpõe-se-lhes a partícula \textunderscore de\textunderscore : \textunderscore cidade de Lisbôa\textunderscore ; \textunderscore cidade de Coímbra]\textunderscore . Há porém excepções nos clássicos, sob a infl. talvez da pronúncia vulgar: \textunderscore cidade Ceuta\textunderscore ; \textunderscore cidade Tânger\textunderscore . Cf. Filinto, \textunderscore D. Man.\textunderscore , I, 22 e 25. Algumas vezes, o nome próprio precede o commum: \textunderscore Évora cidade\textunderscore ; \textunderscore Tânger cidade\textunderscore . Cf. Filinto, \textunderscore D. Man.\textunderscore , I, 23.
\section{Cidade}
\begin{itemize}
\item {Grp. gram.:f.}
\end{itemize}
\begin{itemize}
\item {Utilização:Bras}
\end{itemize}
Vasto formigueiro de saúbas.
\section{Cidadela}
\begin{itemize}
\item {Grp. gram.:f.}
\end{itemize}
\begin{itemize}
\item {Utilização:Fig.}
\end{itemize}
Castello forte, que defende uma cidade.
Lugar, em que se reúnem, ou argumento com que se defendem, os partidários de uma doutrina ou de um systema.
(B. lat. \textunderscore civitatella\textunderscore , do lat. \textunderscore civitas\textunderscore , \textunderscore civitatis\textunderscore )
\section{Cidadelha}
\begin{itemize}
\item {fónica:dê}
\end{itemize}
\begin{itemize}
\item {Grp. gram.:f.}
\end{itemize}
Cidade pequena; cidade pouco importante.
\section{Cidadella}
\begin{itemize}
\item {Grp. gram.:f.}
\end{itemize}
\begin{itemize}
\item {Utilização:Fig.}
\end{itemize}
Castello forte, que defende uma cidade.
Lugar, em que se reúnem, ou argumento com que se defendem, os partidários de uma doutrina ou de um systema.
(B. lat. \textunderscore civitatella\textunderscore , do lat. \textunderscore civitas\textunderscore , \textunderscore civitatis\textunderscore )
\section{Cidadôa}
\begin{itemize}
\item {Grp. gram.:f.}
\end{itemize}
(V.cidadan)
\section{Cidão}
\begin{itemize}
\item {Grp. gram.:m.}
\end{itemize}
Espécie de fôro, na Índia portuguesa.
\section{Cidaritas}
\begin{itemize}
\item {Grp. gram.:m. pl.}
\end{itemize}
\begin{itemize}
\item {Proveniência:(Do gr. \textunderscore kidaris\textunderscore )}
\end{itemize}
Família de molluscos.
\section{Cídasis}
\begin{itemize}
\item {Grp. gram.:m.}
\end{itemize}
Boné coroado ou tiara dos xás da Pérsia.
(Cp. gr. \textunderscore kidasus\textunderscore )
\section{Cidra}
\begin{itemize}
\item {Grp. gram.:f.}
\end{itemize}
\begin{itemize}
\item {Proveniência:(Lat. \textunderscore citrea\textunderscore , fem. de \textunderscore citreus\textunderscore )}
\end{itemize}
Fruto de cidreira.
\section{Cidra}
\begin{itemize}
\item {Grp. gram.:f.}
\end{itemize}
(V.sidra)
\section{Cidrada}
\begin{itemize}
\item {Grp. gram.:f.}
\end{itemize}
Doce, feito da casca de cidra.
\section{Cidral}
\begin{itemize}
\item {Grp. gram.:m.}
\end{itemize}
\begin{itemize}
\item {Proveniência:(De \textunderscore cidra\textunderscore )}
\end{itemize}
Pomar de cidreiras.
\section{Cidrão}
\begin{itemize}
\item {Grp. gram.:m.}
\end{itemize}
\begin{itemize}
\item {Proveniência:(De \textunderscore cidra\textunderscore )}
\end{itemize}
Espécie de cidra de casca grossa.
Doce, feito da casca de cidrão.
\section{Cidreira}
\begin{itemize}
\item {Grp. gram.:f.}
\end{itemize}
\begin{itemize}
\item {Proveniência:(De \textunderscore cidra\textunderscore )}
\end{itemize}
Árvore auranciácea, fructífera.
\textunderscore Erva cidreira\textunderscore , nome vulgar da \textunderscore citronella-menor\textunderscore .
\section{Cidreirinha}
\begin{itemize}
\item {Grp. gram.:f.}
\end{itemize}
\begin{itemize}
\item {Proveniência:(De \textunderscore cidreira\textunderscore )}
\end{itemize}
Casta de uva preta da Bairrada.
\section{Cieiro}
\begin{itemize}
\item {Grp. gram.:m.}
\end{itemize}
Pequenas fendas ou feridas, produzidas na epiderme pelo frio ou pelos ácidos.
(Talvez do lat. \textunderscore cerium\textunderscore , úlcera)
\section{Ciência}
\begin{itemize}
\item {Grp. gram.:f.}
\end{itemize}
\begin{itemize}
\item {Proveniência:(Lat. \textunderscore scientia\textunderscore )}
\end{itemize}
\textunderscore f.\textunderscore  (e der.)
O mesmo ou melhor que \textunderscore sciência\textunderscore , etc.
Conhecimento de qualquer coisa.
Instrução.
Conjunto ou sistema de certos conhecimentos.
Saber, que se adquire pela leitura e pela meditação.
Conjunto sistemático de princípios ou leis, que dizem respeito a objectos correlacionados.
Tudo que é susceptível de formar preceitos ou regras.
\section{Cifa}
\begin{itemize}
\item {Grp. gram.:f.}
\end{itemize}
\begin{itemize}
\item {Utilização:Ant.}
\end{itemize}
Areia, que se emprega nos frascos, em que se moldam as peças, que os ourives hão de lavrar.
Untura, que se dava aos navios com gordura ou azeite de peixe.
(Cp. ár. \textunderscore çaifa\textunderscore , areia fina, e fr. \textunderscore suif\textunderscore , sebo)
\section{Cifada}
\begin{itemize}
\item {Grp. gram.:f.}
\end{itemize}
\begin{itemize}
\item {Utilização:Ant.}
\end{itemize}
\begin{itemize}
\item {Proveniência:(De \textunderscore cifar\textunderscore )}
\end{itemize}
Embarcação, apparelhada ou abastecida para navegar.
\section{Cifado}
\begin{itemize}
\item {Grp. gram.:m.}
\end{itemize}
Navio, apparelhado ou abastecido para navegar.
\section{Cifar}
\begin{itemize}
\item {Grp. gram.:v. t.}
\end{itemize}
\begin{itemize}
\item {Proveniência:(De \textunderscore cifa\textunderscore )}
\end{itemize}
Untar com cifa.
Apparelhar ou abastecer (uma embarcação), para se lançar á agua.
\section{Cifra}
\begin{itemize}
\item {Grp. gram.:f.}
\end{itemize}
\begin{itemize}
\item {Utilização:Des.}
\end{itemize}
\begin{itemize}
\item {Utilização:Gír.}
\end{itemize}
\begin{itemize}
\item {Grp. gram.:Pl.}
\end{itemize}
\begin{itemize}
\item {Proveniência:(Do ár. \textunderscore cifr\textunderscore )}
\end{itemize}
O mesmo que \textunderscore zero\textunderscore .
Explicação ou chave de uma escrita enigmática ou secreta.
Sinaes convencionaes de uma escrita, que não deve sêr comprehendida por todos.
Monogramma de um nome.
Cálculo total.
O ponto capital, a questão, a summa:«\textunderscore a cifra é que quem quizer saber a vida do Arcebispo...\textunderscore »Sousa, \textunderscore V. do. Arceb.\textunderscore , II, 299.
Ânus.
Contabilidade.
\section{Cifra}
\begin{itemize}
\item {Grp. gram.:f.}
\end{itemize}
\begin{itemize}
\item {Utilização:Prov.}
\end{itemize}
\begin{itemize}
\item {Utilização:trasm.}
\end{itemize}
O mesmo que \textunderscore cicatice\textunderscore .
\section{Cifrante}
\begin{itemize}
\item {Grp. gram.:m.}
\end{itemize}
\begin{itemize}
\item {Proveniência:(De \textunderscore cifrar\textunderscore )}
\end{itemize}
Livro, que contém os sinaes de uma escritura enigmática ou secreta e a respectiva chave ou explicação.
\section{Cifrão}
\begin{itemize}
\item {Grp. gram.:m.}
\end{itemize}
\begin{itemize}
\item {Proveniência:(De \textunderscore cifra\textunderscore )}
\end{itemize}
Sinal que, em a numeração usual, separa dos milhares a casa das centenas, assim 123$456.
\section{Cifrar}
\begin{itemize}
\item {Grp. gram.:v. t.}
\end{itemize}
\begin{itemize}
\item {Grp. gram.:V. p.}
\end{itemize}
\begin{itemize}
\item {Proveniência:(De \textunderscore cifra\textunderscore )}
\end{itemize}
Escrever em cifra.
Resumir, reduzir.
Resumir-se: \textunderscore em duas palavras se cifra tudo\textunderscore .
\section{Cifreiro}
\begin{itemize}
\item {Grp. gram.:adj.}
\end{itemize}
\begin{itemize}
\item {Utilização:Prov.}
\end{itemize}
\begin{itemize}
\item {Utilização:trasm.}
\end{itemize}
\begin{itemize}
\item {Proveniência:(De \textunderscore cifra\textunderscore ^2)}
\end{itemize}
O mesmo que \textunderscore cicateiro\textunderscore .
\section{Cigalheiro}
\begin{itemize}
\item {Grp. gram.:adj.}
\end{itemize}
\begin{itemize}
\item {Utilização:Des.}
\end{itemize}
\begin{itemize}
\item {Proveniência:(De \textunderscore cigalho\textunderscore )}
\end{itemize}
Avarento, sovina, interesseiro.
\section{Cigalho}
\begin{itemize}
\item {Grp. gram.:m.}
\end{itemize}
Bocadinho; parte mínima de uma coisa.
Cibalho.
\section{Cigana}
\begin{itemize}
\item {Grp. gram.:f.}
\end{itemize}
\begin{itemize}
\item {Utilização:Bras}
\end{itemize}
Ave ribeirinha do Purus.
\section{Ciganagem}
\begin{itemize}
\item {Grp. gram.:f.}
\end{itemize}
\begin{itemize}
\item {Utilização:Bras. do Amaz}
\end{itemize}
Reunião de ciganos.
Acto de cigano, ciganice.
Factura de mercadorias, ou acto de as vender, entre officiaes da marinha mercante.
\section{Ciganar}
\begin{itemize}
\item {Grp. gram.:v. i.}
\end{itemize}
\begin{itemize}
\item {Utilização:Prov.}
\end{itemize}
Proceder como cigano.
Viver de intrujices.
\section{Ciganaria}
\begin{itemize}
\item {Grp. gram.:f.}
\end{itemize}
Multidão de ciganos.
Acção própria de ciganos.
\section{Ciganas}
\begin{itemize}
\item {Grp. gram.:f. pl.}
\end{itemize}
\begin{itemize}
\item {Proveniência:(De \textunderscore cigano\textunderscore ^1)}
\end{itemize}
Arrecadas de um só pingente.
\section{Ciganeiro}
\begin{itemize}
\item {Grp. gram.:m.  e  adj.}
\end{itemize}
O mesmo que \textunderscore cigalheiro\textunderscore .
\section{Ciganice}
\begin{itemize}
\item {Grp. gram.:f.}
\end{itemize}
\begin{itemize}
\item {Proveniência:(De \textunderscore cigano\textunderscore ^1)}
\end{itemize}
Traficância.
Lisonja ardilosa.
Trapaça em compras ou vendas.
\section{Cigano}
\begin{itemize}
\item {Grp. gram.:m.}
\end{itemize}
\begin{itemize}
\item {Grp. gram.:Adj.}
\end{itemize}
\begin{itemize}
\item {Grp. gram.:M. pl.}
\end{itemize}
\begin{itemize}
\item {Proveniência:(Al. \textunderscore zigeuner\textunderscore , russo \textunderscore tzigane\textunderscore )}
\end{itemize}
Aquelle que pertence á raça dos ciganos.
Trapaceiro; ladino.
Povo errante e miserável, de procedencia indiana, que, fugindo á invasão mongólica, se distribuiu por todo o mundo, falando dialectos que são prácritos corrompidos, e empregando-se ora em enganar vendedores ou compradores de gados nas feiras, ora na pirataria, no acrobatismo, etc.
\section{Cigano}
\begin{itemize}
\item {Grp. gram.:m.}
\end{itemize}
Ave de plumagem pardacenta do norte do Brasil.
\section{Cigarra}
\begin{itemize}
\item {Grp. gram.:f.}
\end{itemize}
\begin{itemize}
\item {Proveniência:(Do lat. \textunderscore cicada\textunderscore , por intermédio de um dem. \textunderscore cicad'la\textunderscore  = \textunderscore cicadula\textunderscore )}
\end{itemize}
Insecto hemíptero, que, durante o tempo calmoso, faz ouvir nos campos um ruído estridente e monótono.
Avezinha canora do Brasil.
\section{Cigarrar}
\begin{itemize}
\item {Grp. gram.:v. i.}
\end{itemize}
Fumar cigarros.
\section{Cigarreira}
\begin{itemize}
\item {Grp. gram.:f.}
\end{itemize}
\begin{itemize}
\item {Proveniência:(De \textunderscore cigarro\textunderscore )}
\end{itemize}
Mulher, que trabalha em fábrica de cigarros.
Caixinha ou estojo, em que se trazem cigarros.
\section{Cigarreiro}
\begin{itemize}
\item {Grp. gram.:m.}
\end{itemize}
\begin{itemize}
\item {Proveniência:(De \textunderscore cigarro\textunderscore )}
\end{itemize}
Homem, que trabalha em fábrica de cigarros.
Espécie de tabaco. Cf. \textunderscore Inquér. Industr.\textunderscore , p. II, v. I, 320.
\section{Cigarrilha}
\begin{itemize}
\item {Grp. gram.:f.}
\end{itemize}
\begin{itemize}
\item {Proveniência:(De \textunderscore cigarro\textunderscore )}
\end{itemize}
Pequeno cigarro, de capa de tabaco.
Pequeno charuto.
Tubozinho, que contem uma substância medicinal, para se aspirar.
\section{Cigarrinho}
\begin{itemize}
\item {Grp. gram.:m.}
\end{itemize}
\begin{itemize}
\item {Proveniência:(De \textunderscore cigarra\textunderscore )}
\end{itemize}
Ave madeirense, (\textunderscore sylvia compicillata\textunderscore ).
\section{Cigarrista}
\begin{itemize}
\item {Grp. gram.:m.}
\end{itemize}
Fumador de cigarros.
\section{Cigarro}
\begin{itemize}
\item {Grp. gram.:m.}
\end{itemize}
\begin{itemize}
\item {Utilização:Des.}
\end{itemize}
Pequena porção de tabaco, envolta em papel, para se fumar.
O mesmo que \textunderscore charuto\textunderscore .
(Cast. \textunderscore cigarro\textunderscore , or. incerta. Propôs-se já a etym. do al. \textunderscore saugen\textunderscore , chupar, e \textunderscore rauch\textunderscore , fumo; mas talvez se relacione com \textunderscore cigarra\textunderscore , por semelhança de fórma)
\section{Cigas}
\begin{itemize}
\item {Grp. gram.:f. pl.}
\end{itemize}
\begin{itemize}
\item {Utilização:Prov.}
\end{itemize}
\begin{itemize}
\item {Utilização:trasm.}
\end{itemize}
Miudezas, insignificâncias.
\section{Cígora}
\begin{itemize}
\item {Grp. gram.:f.}
\end{itemize}
\begin{itemize}
\item {Utilização:Prov.}
\end{itemize}
\begin{itemize}
\item {Utilização:trasm.}
\end{itemize}
Certo jôgo de pião.
\section{Cigorelha}
\begin{itemize}
\item {fónica:gorê}
\end{itemize}
\begin{itemize}
\item {Grp. gram.:f.}
\end{itemize}
\begin{itemize}
\item {Utilização:Prov.}
\end{itemize}
\begin{itemize}
\item {Utilização:trasm.}
\end{itemize}
\begin{itemize}
\item {Utilização:Pop.}
\end{itemize}
O mesmo que \textunderscore cígora\textunderscore .
Rapariga magra, esperta e ladina, sempre á espreita de assumpto para mexericos.
\section{Cilada}
\begin{itemize}
\item {Grp. gram.:f.}
\end{itemize}
Lugar occulto, onde se espera a caça, ou donde se acommete quem se espera.
Traição.
Armadilha; embuste.
(Do latim \textunderscore celatus\textunderscore , de \textunderscore celare\textunderscore , esconder)
\section{Cilha}
\begin{itemize}
\item {Grp. gram.:f.}
\end{itemize}
\begin{itemize}
\item {Proveniência:(Do lat. \textunderscore cingula\textunderscore )}
\end{itemize}
Cinto, com que se aperta a sella ou a carga das bêstas.
\section{Cilhão}
\begin{itemize}
\item {Grp. gram.:adj.}
\end{itemize}
\begin{itemize}
\item {Utilização:Bras. do S}
\end{itemize}
\begin{itemize}
\item {Grp. gram.:M.}
\end{itemize}
\begin{itemize}
\item {Utilização:Prov.}
\end{itemize}
\begin{itemize}
\item {Utilização:minh.}
\end{itemize}
\begin{itemize}
\item {Proveniência:(De \textunderscore cílha\textunderscore )}
\end{itemize}
Diz-se do cavallo, que tem o espinhaço curvado no meio.
Grande cilha.
\textunderscore Dar ao cilhão\textunderscore , recalcitrar, responder com maus modos aos amos, (falando-se de criados).
\section{Cilhar}
\begin{itemize}
\item {Grp. gram.:v. t.}
\end{itemize}
\begin{itemize}
\item {Utilização:Fig.}
\end{itemize}
\begin{itemize}
\item {Proveniência:(De \textunderscore cilha\textunderscore )}
\end{itemize}
Apertar com cilha.
Apertar.
\section{Ciliar}
\begin{itemize}
\item {Grp. gram.:adj.}
\end{itemize}
Relativo aos cílios.
\section{Ciliciar-se}
\begin{itemize}
\item {Grp. gram.:v. i.}
\end{itemize}
Usar cilícios.
Mortificar-se com cilícios.
\section{Cilício}
\begin{itemize}
\item {Grp. gram.:m.}
\end{itemize}
\begin{itemize}
\item {Utilização:Fig.}
\end{itemize}
\begin{itemize}
\item {Proveniência:(Lat. \textunderscore cilicium\textunderscore )}
\end{itemize}
Cinto ou cordão, de pêlo ou lan áspera, ou eriçado de pontas de arame, que se traz sôbre a pelle, por penitência.
Tormento; sacrifício voluntário.
\section{Cilífero}
\begin{itemize}
\item {Grp. gram.:adj.}
\end{itemize}
O mesmo que \textunderscore cilígero\textunderscore .
\section{Ciliforme}
\begin{itemize}
\item {Grp. gram.:adj.}
\end{itemize}
\begin{itemize}
\item {Utilização:Hist. Nat.}
\end{itemize}
\begin{itemize}
\item {Proveniência:(Do lat. \textunderscore cilium\textunderscore  + \textunderscore forma\textunderscore )}
\end{itemize}
Que tem fórma de cílio.
\section{Cilígero}
\begin{itemize}
\item {Grp. gram.:adj.}
\end{itemize}
\begin{itemize}
\item {Proveniência:(Do lat. \textunderscore cilium\textunderscore  + \textunderscore gerere\textunderscore )}
\end{itemize}
Que tem cílios.
\section{Cílio}
\begin{itemize}
\item {Grp. gram.:m.}
\end{itemize}
\begin{itemize}
\item {Proveniência:(Lat. \textunderscore cilium\textunderscore )}
\end{itemize}
Pêlo, que guarnece as pálpebras; celha.
Cada um dos pêlos, que guarnecem certos órgãos vegetaes.
Cada um dos filamentos finíssimos e vibráteis, que se notam em certos elementos anatómicos de alguns animaes invertebrados, de alguns embryões de vertebrados e de algumas algas.
\section{Ciliobrânchio}
\begin{itemize}
\item {fónica:qui}
\end{itemize}
\begin{itemize}
\item {Grp. gram.:adj.}
\end{itemize}
Diz-se dos molluscos que tem brânchias em fórma de cílios.
\section{Ciliobrânquio}
\begin{itemize}
\item {Grp. gram.:adj.}
\end{itemize}
Diz-se dos moluscos que tem brânquias em fórma de cílios.
\section{Ciliógrado}
\begin{itemize}
\item {Grp. gram.:adj.}
\end{itemize}
\begin{itemize}
\item {Utilização:Zool.}
\end{itemize}
\begin{itemize}
\item {Proveniência:(Do lat. \textunderscore cilium\textunderscore  + \textunderscore gradi\textunderscore )}
\end{itemize}
Que marcha com o auxilio dos cílios.
\section{Cilíolo}
\begin{itemize}
\item {Grp. gram.:m.}
\end{itemize}
\begin{itemize}
\item {Utilização:Bot.}
\end{itemize}
Pequeno cílio.
\section{Cima}
\begin{itemize}
\item {Grp. gram.:f.}
\end{itemize}
\begin{itemize}
\item {Utilização:Ant.}
\end{itemize}
\begin{itemize}
\item {Proveniência:(Lat. \textunderscore cyma\textunderscore , ou \textunderscore cuma\textunderscore , gr. \textunderscore kuma\textunderscore )}
\end{itemize}
A parte mais elevada.
Cume, cumeeira.
O mesmo que \textunderscore fim\textunderscore . Cf. \textunderscore Port. Mon. Hist.\textunderscore , Script., 256.--Entra principalmente nas loc. adv. \textunderscore em-cima\textunderscore , \textunderscore de-cima\textunderscore , \textunderscore por-cima\textunderscore , \textunderscore para-cima\textunderscore , que são loc. prep. quando as segue a partícula \textunderscore de\textunderscore .
\section{Cimáceo}
\begin{itemize}
\item {Grp. gram.:m.}
\end{itemize}
\begin{itemize}
\item {Proveniência:(Lat. \textunderscore cymatium\textunderscore )}
\end{itemize}
Moldura, que remata uma cornija.
\section{Cimalha}
\begin{itemize}
\item {Grp. gram.:f.}
\end{itemize}
\begin{itemize}
\item {Utilização:Gram.}
\end{itemize}
\begin{itemize}
\item {Utilização:Náut.}
\end{itemize}
\begin{itemize}
\item {Proveniência:(De \textunderscore cima\textunderscore )}
\end{itemize}
Cimo.
A parte superior da cornija.
Architrave.
Saliência na parte mais alta da parede, em que assentam os beiraes do telhado.
O mesmo que \textunderscore trema\textunderscore .
Gávea.
\section{Cimão}
\begin{itemize}
\item {Grp. gram.:m. Loc. adv.}
\end{itemize}
\begin{itemize}
\item {Utilização:Prov.}
\end{itemize}
\begin{itemize}
\item {Utilização:trasm.}
\end{itemize}
\begin{itemize}
\item {Proveniência:(De \textunderscore cima\textunderscore ?)}
\end{itemize}
\textunderscore De cimão\textunderscore , por baixo do braço, (quando se atiram pedras por essa banda).
\section{Cimba}
\begin{itemize}
\item {Grp. gram.:f.}
\end{itemize}
Formoso e raro animal da África oriental.
\section{Címbalo}
\begin{itemize}
\item {Grp. gram.:m.}
\end{itemize}
Antigo instrumento de cordas.
(Cp. cast. \textunderscore cimbalo\textunderscore )
\section{Cimbrar}
\begin{itemize}
\item {Grp. gram.:v. t.}
\end{itemize}
\begin{itemize}
\item {Proveniência:(De \textunderscore cimbre\textunderscore ?)}
\end{itemize}
Curvar, dobrar.
O mesmo que \textunderscore azumbrar\textunderscore .
\section{Cimbre}
\begin{itemize}
\item {Grp. gram.:m.}
\end{itemize}
Armação de madeira, que serve de molde para abóbada ou arco; cambota.
(Cast. \textunderscore címbra\textunderscore )
\section{Címbrico}
\begin{itemize}
\item {Grp. gram.:adj.}
\end{itemize}
Relativo aos cimbros.
\section{Címbrio}
\begin{itemize}
\item {Grp. gram.:m.}
\end{itemize}
O mesmo que \textunderscore cimbre\textunderscore .
\section{Cimbro}
\begin{itemize}
\item {Grp. gram.:m.}
\end{itemize}
Espécie de mollúsco fluvial.
\section{Cimbros}
\begin{itemize}
\item {Grp. gram.:m. pl.}
\end{itemize}
\begin{itemize}
\item {Proveniência:(Lat. \textunderscore Cimbri\textunderscore )}
\end{itemize}
Antigo povo do norte da Germânia.
\section{Cimeira}
\begin{itemize}
\item {Grp. gram.:f.}
\end{itemize}
\begin{itemize}
\item {Proveniência:(De \textunderscore cima\textunderscore )}
\end{itemize}
Elmo; ornato no cimo do capacete.
O mesmo que \textunderscore cume\textunderscore , \textunderscore cumeeira\textunderscore .
\section{Cimeiro}
\begin{itemize}
\item {Grp. gram.:adj.}
\end{itemize}
Que está ao cimo: \textunderscore o prédio cimeiro de uma calçada\textunderscore .
\section{Cimeliarca}
\begin{itemize}
\item {Grp. gram.:f.}
\end{itemize}
\begin{itemize}
\item {Proveniência:(Gr. \textunderscore kaimeliarkhes\textunderscore )}
\end{itemize}
Designação antiga do thesoireiro ou guarda das preciosidades de uma igreja.
\section{Cimélio}
\begin{itemize}
\item {Grp. gram.:m.}
\end{itemize}
\begin{itemize}
\item {Utilização:Fig.}
\end{itemize}
\begin{itemize}
\item {Proveniência:(Do gr. \textunderscore kaimelion\textunderscore )}
\end{itemize}
Alfaia preciosa de uma igreja.
Thesoiro de uma igreja.
Thesoiro, preciosidades.
\section{Cimentação}
\begin{itemize}
\item {Grp. gram.:f.}
\end{itemize}
Acto de cimentar.
\section{Cimentar}
\begin{itemize}
\item {Grp. gram.:v. t.}
\end{itemize}
\begin{itemize}
\item {Proveniência:(De \textunderscore cimento\textunderscore )}
\end{itemize}
Ligar com cimento.
Fazer os alicerces de.
Alicerçar.
Consolidar.
\section{Cimento}
\begin{itemize}
\item {Grp. gram.:m.}
\end{itemize}
\begin{itemize}
\item {Utilização:Fig.}
\end{itemize}
\begin{itemize}
\item {Proveniência:(Do lat. \textunderscore caementum\textunderscore )}
\end{itemize}
Cascalho.
Pedra solta.
Espécie de argamassa.
Alicerce; fundamento.
\section{Cimífuga}
\begin{itemize}
\item {Grp. gram.:f.}
\end{itemize}
\begin{itemize}
\item {Proveniência:(Do lat. \textunderscore cimex\textunderscore  + \textunderscore fugere\textunderscore )}
\end{itemize}
Gênero de plantas ranunculáceas.
\section{Cimita}
\begin{itemize}
\item {Grp. gram.:f.}
\end{itemize}
\begin{itemize}
\item {Utilização:Des.}
\end{itemize}
\begin{itemize}
\item {Proveniência:(De \textunderscore cima\textunderscore )}
\end{itemize}
Remate, o ponto mais elevado.
\section{Cimitarra}
\begin{itemize}
\item {Grp. gram.:f.}
\end{itemize}
\begin{itemize}
\item {Proveniência:(Do pers. \textunderscore chimchor\textunderscore )}
\end{itemize}
Espada, de lâmina larga e curva.
\section{Cimo}
\begin{itemize}
\item {Grp. gram.:m.}
\end{itemize}
O mesmo que \textunderscore cume\textunderscore ; cima.
(Cp. \textunderscore cima\textunderscore ^1)
\section{Cimólia}
\begin{itemize}
\item {Grp. gram.:f.}
\end{itemize}
\begin{itemize}
\item {Proveniência:(Lat. \textunderscore cimolia\textunderscore )}
\end{itemize}
Espécie de barro adstringente.
\section{Cimoliano}
\begin{itemize}
\item {Grp. gram.:adj.}
\end{itemize}
Dizia-se do barco ou pedra, chamada \textunderscore cimólia\textunderscore . Cf. Castilho, \textunderscore Fastos\textunderscore , I, 322.
\section{Cimolithia}
\begin{itemize}
\item {Grp. gram.:f.}
\end{itemize}
O mesmo que \textunderscore cimólia\textunderscore . Cf. Castilho, \textunderscore Fastos\textunderscore , I, 322.
\section{Cimolitia}
\begin{itemize}
\item {Grp. gram.:f.}
\end{itemize}
O mesmo que \textunderscore cimólia\textunderscore . Cf. Castilho, \textunderscore Fastos\textunderscore , I, 322.
\section{Cinabre}
\begin{itemize}
\item {Grp. gram.:m.}
\end{itemize}
O mesmo que \textunderscore cinábrio\textunderscore .
\section{Cinabre}
\begin{itemize}
\item {Grp. gram.:m.}
\end{itemize}
\begin{itemize}
\item {Utilização:Prov.}
\end{itemize}
\begin{itemize}
\item {Utilização:beir.}
\end{itemize}
Vigota, que cruza o madeiramento dos telhados, para o fortificar.
\section{Cinabrino}
\begin{itemize}
\item {Grp. gram.:adj.}
\end{itemize}
Semelhante ao cinábrio.
Preparado com cinábrio.
\section{Cinábrio}
\begin{itemize}
\item {Grp. gram.:m.}
\end{itemize}
\begin{itemize}
\item {Proveniência:(Lat. \textunderscore cinnabaris\textunderscore )}
\end{itemize}
Sulfureto vermelho de mercúrio.
Espécie de galenite.
\section{Cinamato}
\begin{itemize}
\item {Grp. gram.:m.}
\end{itemize}
Combinação do ácido cinâmico com uma base.
\section{Cinâmico}
\begin{itemize}
\item {Grp. gram.:adj.}
\end{itemize}
\begin{itemize}
\item {Proveniência:(De \textunderscore cinnamo\textunderscore )}
\end{itemize}
Diz-se de um ácido extrahido do bálsamo do Perú.
\section{Cinamo}
\begin{itemize}
\item {Grp. gram.:m.}
\end{itemize}
\begin{itemize}
\item {Utilização:Ant.}
\end{itemize}
\begin{itemize}
\item {Proveniência:(Gr. \textunderscore kinnamomon\textunderscore )}
\end{itemize}
Nome científico da caneleira.
Substância aromática, que uns supõem têr sido a canela, outros a mirra.
\section{Cinamomo}
\begin{itemize}
\item {Grp. gram.:m.}
\end{itemize}
\begin{itemize}
\item {Utilização:Ant.}
\end{itemize}
\begin{itemize}
\item {Proveniência:(Gr. \textunderscore kinnamomon\textunderscore )}
\end{itemize}
Nome científico da caneleira.
Substância aromática, que uns supõem têr sido a canela, outros a mirra.
\section{Cinara}
\begin{itemize}
\item {Grp. gram.:f.}
\end{itemize}
\begin{itemize}
\item {Proveniência:(Gr. \textunderscore kinara\textunderscore )}
\end{itemize}
Alcachofra comestível.
\section{Cináreo}
\begin{itemize}
\item {Grp. gram.:adj.}
\end{itemize}
\begin{itemize}
\item {Proveniência:(De \textunderscore cinara\textunderscore )}
\end{itemize}
Diz-se das plantas, que têm a flôr em fórma de cabeça, como a alcachofra.
\section{Cinarina}
\begin{itemize}
\item {Grp. gram.:f.}
\end{itemize}
\begin{itemize}
\item {Utilização:Med.}
\end{itemize}
\begin{itemize}
\item {Proveniência:(De \textunderscore cinara\textunderscore )}
\end{itemize}
Princípio activo e drástico, extrahido da alcachofra.
\section{Cinarocéphalo}
\begin{itemize}
\item {Grp. gram.:adj.}
\end{itemize}
\begin{itemize}
\item {Utilização:Bot.}
\end{itemize}
\begin{itemize}
\item {Proveniência:(Do gr. \textunderscore kinara\textunderscore  + \textunderscore kephalè\textunderscore )}
\end{itemize}
Que tem flôres semelhantes ás da alcachofra.
\section{Cinasco}
\begin{itemize}
\item {Grp. gram.:m.}
\end{itemize}
\begin{itemize}
\item {Utilização:Prov.}
\end{itemize}
\begin{itemize}
\item {Utilização:trasm.}
\end{itemize}
Migalha, estilha, estilhaço.
\section{Cinca}
\begin{itemize}
\item {Grp. gram.:f.}
\end{itemize}
\begin{itemize}
\item {Utilização:Fig.}
\end{itemize}
\begin{itemize}
\item {Proveniência:(De \textunderscore cinco\textunderscore )}
\end{itemize}
Perda de cinco pontos, no jôgo da bola.
Êrro, acto de cincar^1.
\section{Cincada}
\begin{itemize}
\item {Grp. gram.:f.}
\end{itemize}
Acto de cincar ou de errar.
\section{Cincar}
\begin{itemize}
\item {Grp. gram.:v. i.}
\end{itemize}
Dar cincas; errar; falhar.
\section{Cincar}
\begin{itemize}
\item {Grp. gram.:v. t.}
\end{itemize}
\begin{itemize}
\item {Utilização:Prov.}
\end{itemize}
Despejar, esgotar. (Colhido na Bairrada)
\section{Cincerro}
\begin{itemize}
\item {fónica:cê}
\end{itemize}
\begin{itemize}
\item {Grp. gram.:m.}
\end{itemize}
\begin{itemize}
\item {Utilização:Bras}
\end{itemize}
Campaínha grande, que se pendura ao pescoço da bêsta que serve de guia ás outras.
(Cast. \textunderscore cencerro\textunderscore )
\section{Cincha}
\begin{itemize}
\item {Grp. gram.:f.}
\end{itemize}
\begin{itemize}
\item {Utilização:Bras. do S}
\end{itemize}
Espécie de cinta ou cilha, com que se apertam os arreios de um cavallo.
(Cp. \textunderscore cincho\textunderscore ^1)
\section{Cinchador}
\begin{itemize}
\item {Grp. gram.:m.}
\end{itemize}
\begin{itemize}
\item {Utilização:Bras. do S}
\end{itemize}
\begin{itemize}
\item {Proveniência:(De \textunderscore cinchar\textunderscore )}
\end{itemize}
Peça de ferro ou coiro, presa á cincha, com uma argola em que se prende uma extremidade do laço, opposta á outra extremidade em que há outra argola.
\section{Cinchão}
\begin{itemize}
\item {Grp. gram.:m.}
\end{itemize}
\begin{itemize}
\item {Utilização:Bras. do S}
\end{itemize}
\begin{itemize}
\item {Proveniência:(De \textunderscore cincha\textunderscore )}
\end{itemize}
Cinta larga de tecido e franja, que substitue a sobrecincha nos arreios mais decentes.
\section{Cinchar}
\begin{itemize}
\item {Grp. gram.:v. t.}
\end{itemize}
\begin{itemize}
\item {Utilização:Bras. do Sul}
\end{itemize}
\begin{itemize}
\item {Proveniência:(De \textunderscore cincha\textunderscore )}
\end{itemize}
Segurar por um laço preso á cincha (um animal).
\section{Cinchar}
\begin{itemize}
\item {Grp. gram.:v. t.}
\end{itemize}
\begin{itemize}
\item {Proveniência:(De \textunderscore cincho\textunderscore )}
\end{itemize}
Apertar com o cincho.
\section{Cincho}
\begin{itemize}
\item {Grp. gram.:m.}
\end{itemize}
\begin{itemize}
\item {Proveniência:(Do lat. \textunderscore cingulum\textunderscore )}
\end{itemize}
Aro, em que se aperta o queijo, para lhe dar fórma e espremer o soro.
Molde, em que se faz o queijo.
Caixa cylíndrica, crivada de orifícios, dentro da qual se colloca a azeitona moída, para se sujeitar á pressão.
\section{Cincho}
\begin{itemize}
\item {Grp. gram.:m.}
\end{itemize}
\begin{itemize}
\item {Utilização:Prov.}
\end{itemize}
\begin{itemize}
\item {Utilização:trasm.}
\end{itemize}
Planta, que nasce nas hortas e milharaes, de haste avermelhada e fôlha pequena e verde-negra.
\section{Cinchona}
\begin{itemize}
\item {Grp. gram.:f.}
\end{itemize}
\begin{itemize}
\item {Proveniência:(De \textunderscore Chinchon\textunderscore , n. p.)}
\end{itemize}
Planta, que produz a quina.
Seria preferível \textunderscore chinchona\textunderscore , como alguns escrevem.
\section{Cinchonáceas}
\begin{itemize}
\item {Grp. gram.:f. pl.}
\end{itemize}
Família de plantas, que têm por typo a cinchona.
\section{Cinchonina}
\begin{itemize}
\item {Grp. gram.:f.}
\end{itemize}
\begin{itemize}
\item {Proveniência:(De \textunderscore cinchona\textunderscore )}
\end{itemize}
Alcaloide, que se encontra em várias cinchonáceas.
\section{Cinchonino}
\begin{itemize}
\item {Grp. gram.:m.}
\end{itemize}
Princípio vegetal, descoberto na quina; o mesmo que \textunderscore cinchonina\textunderscore .
\section{Cinclo}
\begin{itemize}
\item {Grp. gram.:m.}
\end{itemize}
\begin{itemize}
\item {Proveniência:(Lat. \textunderscore cinclus\textunderscore )}
\end{itemize}
Ave, da fam. dos melros, espécie de calhandra aquática.
\section{Cinclosomo}
\begin{itemize}
\item {fónica:só}
\end{itemize}
\begin{itemize}
\item {Grp. gram.:m.}
\end{itemize}
\begin{itemize}
\item {Proveniência:(De \textunderscore cinclo\textunderscore  + gr. \textunderscore soma\textunderscore )}
\end{itemize}
Espécie de melro australiano.
\section{Cinclossomo}
\begin{itemize}
\item {Grp. gram.:m.}
\end{itemize}
\begin{itemize}
\item {Proveniência:(De \textunderscore cinclo\textunderscore  + gr. \textunderscore soma\textunderscore )}
\end{itemize}
Espécie de melro australiano.
\section{Cinco}
\begin{itemize}
\item {Grp. gram.:adj.}
\end{itemize}
\begin{itemize}
\item {Grp. gram.:M.}
\end{itemize}
\begin{itemize}
\item {Proveniência:(Do lat. \textunderscore quinque\textunderscore )}
\end{itemize}
Diz-se do número cardinal, formado de quatro e mais um; quinto.
O algarismo representativo dêsse número.
Carta de jogar ou peça do dominó, que tem cinco pontos.
Aquelle ou aquillo que numa série de cinco occupa o último lugar.
\section{Cinco-em-rama}
\begin{itemize}
\item {Grp. gram.:f.}
\end{itemize}
Planta rosácea, que tem cinco fôlhas em cada ramo.
\section{Cinco-em-ramo}
\begin{itemize}
\item {Grp. gram.:m.}
\end{itemize}
Planta rosácea, que tem cinco fôlhas em cada ramo.
\section{Cincoenta}
\begin{itemize}
\item {Grp. gram.:adj.}
\end{itemize}
\begin{itemize}
\item {Proveniência:(Do lat. \textunderscore quinquaginta\textunderscore , sôb a infl. de \textunderscore cinco\textunderscore )}
\end{itemize}
Cinco vezes déz.
\section{Cincoentenário}
\begin{itemize}
\item {Grp. gram.:m.}
\end{itemize}
Celebração do quinquagésimo anniversário: \textunderscore o Papa tenciona celebrar o cincoentenário da Immaculada Conceição\textunderscore .
\section{Cincoentona}
\begin{itemize}
\item {Grp. gram.:f.  e  adj. f.}
\end{itemize}
Mulher solteira, que tem proximamente cincoenta annos. Cf. Camillo, \textunderscore Corja\textunderscore , 12 e 140.
\section{Cinco-fôlhas}
\begin{itemize}
\item {Grp. gram.:f.}
\end{itemize}
Árvore bignoniácea, medicinal, do Brasil, (\textunderscore bignonia longiflora\textunderscore , Well.).
\section{Cincos}
\begin{itemize}
\item {Grp. gram.:m. pl.}
\end{itemize}
Espécie de cisa, que as fazendas pagavam, ao entrar nas alfândegas. Cf. F. Borges, \textunderscore Diccion. Jur.\textunderscore 
\section{Cinctípedes}
\begin{itemize}
\item {Grp. gram.:m. pl.}
\end{itemize}
\begin{itemize}
\item {Proveniência:(Do lat. \textunderscore cinctus\textunderscore  + \textunderscore pes\textunderscore )}
\end{itemize}
Animaes, cujos pés são rodeados por um círculo colorido.
\section{Cindros}
\begin{itemize}
\item {Grp. gram.:m.}
\end{itemize}
(?):«\textunderscore ...o grã crocodilo mata ousadamente o pequenino cindros\textunderscore ». \textunderscore Vir. Trág.\textunderscore , XII, 10.
\section{Cinemática}
\begin{itemize}
\item {Grp. gram.:f.}
\end{itemize}
\begin{itemize}
\item {Proveniência:(De \textunderscore cinemático\textunderscore )}
\end{itemize}
Sciência dos movimentos.
\section{Cinemático}
\begin{itemize}
\item {Grp. gram.:adj.}
\end{itemize}
\begin{itemize}
\item {Proveniência:(Gr. \textunderscore kinematikos\textunderscore )}
\end{itemize}
Relativo ao movimento mechânico.
\section{Cinematografar}
\begin{itemize}
\item {Grp. gram.:v. t.}
\end{itemize}
Expor á vista, por meio de cinematógrafo.
\section{Cinematografia}
\begin{itemize}
\item {Grp. gram.:f.}
\end{itemize}
Processo ou prática de cinematógrafo.
\section{Cinematografiar}
\begin{itemize}
\item {Grp. gram.:v. t.}
\end{itemize}
O mesmo que \textunderscore cinematografar\textunderscore .
\section{Cinematográfico}
\begin{itemize}
\item {Grp. gram.:adj.}
\end{itemize}
Relativo á cinematografia.
\section{Cinematógrafo}
\begin{itemize}
\item {Grp. gram.:m.}
\end{itemize}
\begin{itemize}
\item {Proveniência:(Do gr. \textunderscore kinema\textunderscore  + \textunderscore graphein\textunderscore )}
\end{itemize}
Aparelho crono-fotográfico, inventado em 1895, e que permite, durante um minuto, a projecção de scenas animadas ou em movimento.
\section{Cinematographar}
\begin{itemize}
\item {Grp. gram.:v. t.}
\end{itemize}
Expor á vista, por meio de cinematógrapho.
\section{Cinematographia}
\begin{itemize}
\item {Grp. gram.:f.}
\end{itemize}
Processo ou prática de cinematógrapho.
\section{Cinematographiar}
\begin{itemize}
\item {Grp. gram.:v. t.}
\end{itemize}
O mesmo que \textunderscore cinematographar\textunderscore .
\section{Cinematográphico}
\begin{itemize}
\item {Grp. gram.:adj.}
\end{itemize}
Relativo á cinematographia.
\section{Cinematógrapho}
\begin{itemize}
\item {Grp. gram.:m.}
\end{itemize}
\begin{itemize}
\item {Proveniência:(Do gr. \textunderscore kinema\textunderscore  + \textunderscore graphein\textunderscore )}
\end{itemize}
Apparelho chrono-photográphico, inventado em 1895, e que permitte, durante um minuto, a projecção de scenas animadas ou em movimento.
\section{Cineral}
\begin{itemize}
\item {Grp. gram.:m.}
\end{itemize}
\begin{itemize}
\item {Utilização:Neol.}
\end{itemize}
\begin{itemize}
\item {Proveniência:(Do lat. \textunderscore cinis, cineris\textunderscore )}
\end{itemize}
Montão de cinzas.
\section{Cinerar}
\textunderscore v. t.\textunderscore  (e der.)(V.incinerar)
\section{Cinerária}
\begin{itemize}
\item {Grp. gram.:f.}
\end{itemize}
\begin{itemize}
\item {Proveniência:(De \textunderscore cinerário\textunderscore )}
\end{itemize}
Gênero de plantas ornamentaes.
\section{Cinerário}
\begin{itemize}
\item {Grp. gram.:adj.}
\end{itemize}
\begin{itemize}
\item {Grp. gram.:M.}
\end{itemize}
\begin{itemize}
\item {Proveniência:(Lat. \textunderscore cinerarius\textunderscore )}
\end{itemize}
Relativo a cinzas.
Que contém os restos mortaes de um defunto.
Urna cinerária.
\section{Cinérea}
\begin{itemize}
\item {Grp. gram.:f.}
\end{itemize}
\begin{itemize}
\item {Proveniência:(De \textunderscore cinéreo\textunderscore )}
\end{itemize}
Variedade de videira americana, vigorosa e resistente.
\section{Cinéreo}
\begin{itemize}
\item {Grp. gram.:adj.}
\end{itemize}
\begin{itemize}
\item {Proveniência:(Lat. \textunderscore cinereus\textunderscore )}
\end{itemize}
Cinzento.
\section{Cinerício}
\begin{itemize}
\item {Grp. gram.:adj.}
\end{itemize}
O mesmo que \textunderscore cinéreo\textunderscore .
\section{Cineriforme}
\begin{itemize}
\item {Grp. gram.:adj.}
\end{itemize}
\begin{itemize}
\item {Proveniência:(Do lat. \textunderscore cinis\textunderscore  + \textunderscore forma\textunderscore )}
\end{itemize}
Semelhante á cinza.
\section{Cinesia}
\begin{itemize}
\item {Grp. gram.:f.}
\end{itemize}
\begin{itemize}
\item {Proveniência:(Gr. \textunderscore kinesis\textunderscore )}
\end{itemize}
Faculdade, que a alma tem, de imprimir movimento aos membros.
\section{Cinesiologia}
\begin{itemize}
\item {Grp. gram.:f.}
\end{itemize}
\begin{itemize}
\item {Proveniência:(Do gr. \textunderscore kinesis\textunderscore  + \textunderscore logos\textunderscore )}
\end{itemize}
Sciência do movimento; o mesmo que \textunderscore cinética\textunderscore  e \textunderscore cinemática\textunderscore .
\section{Cinesiterapia}
\begin{itemize}
\item {Grp. gram.:f.}
\end{itemize}
\begin{itemize}
\item {Proveniência:(Do gr. \textunderscore kinesis\textunderscore  + \textunderscore therapeia\textunderscore )}
\end{itemize}
Tratamento das doenças por meio da ginástica.
\section{Cinesiterápico}
\begin{itemize}
\item {Grp. gram.:adj.}
\end{itemize}
Relativo á cinesiterapia.
\section{Cinesitherapia}
\begin{itemize}
\item {Grp. gram.:f.}
\end{itemize}
\begin{itemize}
\item {Proveniência:(Do gr. \textunderscore kinesis\textunderscore  + \textunderscore therapeia\textunderscore )}
\end{itemize}
Tratamento das doenças por meio da gymnástica.
\section{Cinesitherápico}
\begin{itemize}
\item {Grp. gram.:adj.}
\end{itemize}
Relativo á cinesitherapia.
\section{Cinética}
\begin{itemize}
\item {Grp. gram.:f.}
\end{itemize}
\begin{itemize}
\item {Proveniência:(De \textunderscore cinético\textunderscore )}
\end{itemize}
Sciência, que tem por objecto a extensão das fôrças, consideradas nos movimentos variados que ellas produzem.
\section{Cinético}
\begin{itemize}
\item {Grp. gram.:adj.}
\end{itemize}
\begin{itemize}
\item {Proveniência:(Gr. \textunderscore kinetikos\textunderscore )}
\end{itemize}
Relativo a movimento.
Que tem fôrça e movimento.
\section{Cinetofónio}
\begin{itemize}
\item {Grp. gram.:m.}
\end{itemize}
\begin{itemize}
\item {Proveniência:(Do gr. \textunderscore kinein\textunderscore  + \textunderscore phone\textunderscore )}
\end{itemize}
Aparelho, resultante da combinação do cinematógrafo com o fonógrafo.
\section{Cinetógrafo}
\begin{itemize}
\item {Grp. gram.:m.}
\end{itemize}
O mesmo que \textunderscore cinematógrafo\textunderscore .
\section{Cinetógrapho}
\begin{itemize}
\item {Grp. gram.:m.}
\end{itemize}
O mesmo que \textunderscore cinematógrapho\textunderscore .
\section{Cinetophónio}
\begin{itemize}
\item {Grp. gram.:m.}
\end{itemize}
\begin{itemize}
\item {Proveniência:(Do gr. \textunderscore kinein\textunderscore  + \textunderscore phone\textunderscore )}
\end{itemize}
Apparelho, resultante da combinação do cinematógrapho com o phonógrapho.
\section{Cinetoscópio}
\begin{itemize}
\item {Grp. gram.:m.}
\end{itemize}
\begin{itemize}
\item {Proveniência:(Do gr. \textunderscore kinein\textunderscore  + \textunderscore skopein\textunderscore )}
\end{itemize}
Designação, dada por Edison ao apparelho que reproduz a photographia animada. Cp. \textunderscore cinematógrapho\textunderscore .
\section{Cingalá}
\begin{itemize}
\item {Grp. gram.:m.}
\end{itemize}
Língua de Ceilão; o mesmo ou melhor que cingalês.
\section{Cingalês}
\begin{itemize}
\item {Grp. gram.:adj.}
\end{itemize}
\begin{itemize}
\item {Grp. gram.:M.}
\end{itemize}
\begin{itemize}
\item {Proveniência:(Do n. p. indígena de Ceilão)}
\end{itemize}
Relativo a Ceilão.
Habitante de Ceilão.
Língua de Ceilão.
\section{Cingel}
\begin{itemize}
\item {Grp. gram.:m.}
\end{itemize}
\begin{itemize}
\item {Proveniência:(Lat. hyp. \textunderscore cingellus\textunderscore , dem. de \textunderscore cingulus\textunderscore )}
\end{itemize}
Junta de bois.
\section{Cingelada}
\begin{itemize}
\item {Grp. gram.:f.}
\end{itemize}
(V.cingel)
\section{Cingeleiro}
\begin{itemize}
\item {Grp. gram.:m.}
\end{itemize}
\begin{itemize}
\item {Utilização:Prov.}
\end{itemize}
\begin{itemize}
\item {Proveniência:(De \textunderscore cingel\textunderscore )}
\end{itemize}
Aquelle que tem e aluga ou guia uma junta de bois ou vacas.
Boieiro que, em Espinho, ajuda o arrasto das redes de pesca.
\section{Cingente}
\begin{itemize}
\item {Grp. gram.:m.}
\end{itemize}
O mesmo que \textunderscore cingento\textunderscore .
\section{Cingento}
\begin{itemize}
\item {Grp. gram.:m.}
\end{itemize}
\begin{itemize}
\item {Proveniência:(De \textunderscore cingir\textunderscore )}
\end{itemize}
Espécie de grampo grande, com que os marceneiros e carpinteiros cingem certas peças para as grudar ou lavrar.
\section{Cingideiras}
\begin{itemize}
\item {Grp. gram.:f. pl.}
\end{itemize}
\begin{itemize}
\item {Proveniência:(De \textunderscore cingir\textunderscore )}
\end{itemize}
Dedos do meio, nos pés das aves de rapina.
\section{Cingidoiro}
\begin{itemize}
\item {Grp. gram.:m.}
\end{itemize}
O mesmo que \textunderscore cinto\textunderscore .
\section{Cingidouro}
\begin{itemize}
\item {Grp. gram.:m.}
\end{itemize}
O mesmo que \textunderscore cinto\textunderscore .
\section{Cingidor}
\begin{itemize}
\item {Grp. gram.:adj.}
\end{itemize}
Que cinge. Cf. Filinto, II, 302.
\section{Cingigola}
\begin{itemize}
\item {Grp. gram.:f.}
\end{itemize}
\begin{itemize}
\item {Proveniência:(De \textunderscore cingir\textunderscore  + \textunderscore golla\textunderscore )}
\end{itemize}
Correia, que faz parte da cabeçada.
\section{Cingigolla}
\begin{itemize}
\item {Grp. gram.:f.}
\end{itemize}
\begin{itemize}
\item {Proveniência:(De \textunderscore cingir\textunderscore  + \textunderscore golla\textunderscore )}
\end{itemize}
Correia, que faz parte da cabeçada.
\section{Cingir}
\begin{itemize}
\item {Grp. gram.:v. t.}
\end{itemize}
\begin{itemize}
\item {Proveniência:(Lat. \textunderscore cingere\textunderscore )}
\end{itemize}
Rodear, apertar em roda.
Pôr á cinta: \textunderscore cingir a espada\textunderscore .
Ligar com cinta.
Pôr em volta da cabeça: \textunderscore cingir uma corôa\textunderscore .
\section{Cíngulo}
\begin{itemize}
\item {Grp. gram.:m.}
\end{itemize}
\begin{itemize}
\item {Proveniência:(Lat. \textunderscore cingulus\textunderscore )}
\end{itemize}
Cordão, com que o sacerdote aperta a alva, na cintura.
Cinto.
\section{Cinho}
\begin{itemize}
\item {Grp. gram.:m.}
\end{itemize}
(V. \textunderscore cincho\textunderscore ^1)
\section{Cinifólio}
\begin{itemize}
\item {Grp. gram.:m.}
\end{itemize}
\begin{itemize}
\item {Utilização:Bot.}
\end{itemize}
O mesmo que \textunderscore gracíola\textunderscore . Cf. \textunderscore Pharm. Port.\textunderscore 
\section{Cinira}
\begin{itemize}
\item {Grp. gram.:f.}
\end{itemize}
\begin{itemize}
\item {Proveniência:(Gr. \textunderscore kinura\textunderscore )}
\end{itemize}
Espécie de lira ou cítara, entre os Hebreus, Sírios e Fenícios.
\section{Cinisga}
\begin{itemize}
\item {Grp. gram.:f.}
\end{itemize}
\begin{itemize}
\item {Utilização:Prov.}
\end{itemize}
\begin{itemize}
\item {Utilização:trasm.}
\end{itemize}
Rapariga magra, cigorelha.
\section{Cinnamato}
\begin{itemize}
\item {Grp. gram.:m.}
\end{itemize}
Combinação do ácido cinnâmico com uma base.
\section{Cinnâmico}
\begin{itemize}
\item {Grp. gram.:adj.}
\end{itemize}
\begin{itemize}
\item {Proveniência:(De \textunderscore cinnamo\textunderscore )}
\end{itemize}
Diz-se de um ácido extrahido do bálsamo do Perú.
\section{Cinnamo}
\begin{itemize}
\item {Grp. gram.:m.}
\end{itemize}
\begin{itemize}
\item {Utilização:Ant.}
\end{itemize}
\begin{itemize}
\item {Proveniência:(Gr. \textunderscore kinnamomon\textunderscore )}
\end{itemize}
Nome scientífico da caneleira.
Substância aromática, que uns suppõem têr sido a canela, outros a myrrha.
\section{Cinnamomo}
\begin{itemize}
\item {Grp. gram.:m.}
\end{itemize}
\begin{itemize}
\item {Utilização:Ant.}
\end{itemize}
\begin{itemize}
\item {Proveniência:(Gr. \textunderscore kinnamomon\textunderscore )}
\end{itemize}
Nome scientífico da caneleira.
Substância aromática, que uns suppõem têr sido a canela, outros a myrrha.
\section{Cinnor}
\begin{itemize}
\item {Grp. gram.:m.}
\end{itemize}
\begin{itemize}
\item {Proveniência:(Hebr. \textunderscore kinnor\textunderscore )}
\end{itemize}
Instrumento músico dos Hebreus.
\section{Cinor}
\begin{itemize}
\item {Grp. gram.:m.}
\end{itemize}
\begin{itemize}
\item {Proveniência:(Hebr. \textunderscore kinnor\textunderscore )}
\end{itemize}
Instrumento músico dos Hebreus.
\section{Cinque}
\begin{itemize}
\item {Grp. gram.:adj.}
\end{itemize}
\begin{itemize}
\item {Utilização:Ant.}
\end{itemize}
O mesmo que \textunderscore cinco\textunderscore .
\section{Cinquena}
\begin{itemize}
\item {Grp. gram.:f.}
\end{itemize}
\begin{itemize}
\item {Utilização:Ant.}
\end{itemize}
\begin{itemize}
\item {Proveniência:(De \textunderscore cinco\textunderscore )}
\end{itemize}
Devoção ou reza, que durava cinco dias.
Espaço de cinco dias.
\section{Cinquenta}
\begin{itemize}
\item {fónica:cu-en}
\end{itemize}
\begin{itemize}
\item {Grp. gram.:adj.}
\end{itemize}
Outra fórma de \textunderscore cincoenta\textunderscore , usada por Garrett, mas pouco diffundida, embora exacta.
\section{Cinquete}
\begin{itemize}
\item {fónica:quê}
\end{itemize}
\begin{itemize}
\item {Grp. gram.:m.}
\end{itemize}
Aquelle que cinca ou commete erros?:«\textunderscore pois vedesme aqui mais refinado cinquete que um cartaxo\textunderscore ». \textunderscore Eufrosina\textunderscore , 7.
\section{Cinquinho}
\begin{itemize}
\item {Grp. gram.:m.}
\end{itemize}
\begin{itemize}
\item {Proveniência:(De \textunderscore cinco\textunderscore )}
\end{itemize}
Antiga moéda portuguesa.
\section{Cinta}
\begin{itemize}
\item {Grp. gram.:f.}
\end{itemize}
Faixa, com que se cinge o corpo na cintura.
Cintura.
Tira de pano ou de coiro, para apertar ou cingir.
Disposição circular de objectos semelhantes.
Tira de papel, com que se cingem os jornaes ou livros enviados pelo correio.
Filete architectónico.
Peças de madeira, que cingem a embarcação, da prôa á popa, exteriormente.
Cós.
(Cp. \textunderscore cinto\textunderscore )
\section{Cintado}
\begin{itemize}
\item {Grp. gram.:m.}
\end{itemize}
\begin{itemize}
\item {Utilização:Náut.}
\end{itemize}
\begin{itemize}
\item {Grp. gram.:Adj.}
\end{itemize}
\begin{itemize}
\item {Proveniência:(De \textunderscore cintar\textunderscore )}
\end{itemize}
Série de pranchas grossas, que cavilham exteriormente para o cavername, em todo o comprimento do navio.
Diz-se da peça de vestuário, que representa próximamente a depressão da cintura.
\section{Cintador}
\begin{itemize}
\item {Grp. gram.:m.}
\end{itemize}
Aquelle que cinta jornaes para a expedição postal.
\section{Cintar}
\begin{itemize}
\item {Grp. gram.:v. t.}
\end{itemize}
\begin{itemize}
\item {Proveniência:(De \textunderscore cinta\textunderscore )}
\end{itemize}
Pôr cinta em: \textunderscore cintar um jornal\textunderscore .
Apertar com arcos de madeira ou ferro: \textunderscore cintar o tonel\textunderscore .
Dar a depressão da cintura a (peça de vestuário): \textunderscore cintar um casaco\textunderscore .
\section{Cinteiro}
\begin{itemize}
\item {Grp. gram.:m.}
\end{itemize}
\begin{itemize}
\item {Proveniência:(De \textunderscore cinto\textunderscore )}
\end{itemize}
Aquelle que faz ou vende cintos.
Faixa, com que se ligam os cueiros das crianças.
Fita, que abraça a copa do chapéu, junto á aba.
\section{Cintel}
\begin{itemize}
\item {Grp. gram.:m.}
\end{itemize}
\begin{itemize}
\item {Proveniência:(De \textunderscore cinto\textunderscore )}
\end{itemize}
Espaço circular, em que se movem os animaes, que fazem andar um engenho.
Espécie de compasso, com que se traçam grandes círculos.
Cincho.
Peça da roda do carro, ao lado do meão.
\section{Cintilho}
\begin{itemize}
\item {Grp. gram.:m.}
\end{itemize}
Cinto pequeno:«\textunderscore as roupas recamadas de oiro, e tomadas airosamente em um cintilho\textunderscore ». Vieira.
\section{Cinto}
\begin{itemize}
\item {Grp. gram.:m.}
\end{itemize}
\begin{itemize}
\item {Utilização:Ant.}
\end{itemize}
\begin{itemize}
\item {Proveniência:(Lat. \textunderscore cinctus\textunderscore )}
\end{itemize}
Correia, que cerca a cintura com uma só volta.
Boldrié.
Cós.
Zona.
Cerrado, cêrca.
\textunderscore part. irr.\textunderscore  de \textunderscore cingir\textunderscore : \textunderscore trazia espada cinta\textunderscore .
\section{Cintrã}
\begin{itemize}
\item {Grp. gram.:f.}
\end{itemize}
\begin{itemize}
\item {Proveniência:(De \textunderscore cintrão\textunderscore )}
\end{itemize}
Mulher de Cintra.
\section{Cintradora}
\begin{itemize}
\item {Grp. gram.:f.}
\end{itemize}
Maquinismo, usado no fabríco das seges. Cf. \textunderscore Inquér. Industr.\textunderscore , P. II, l. 2.^o, 254.
(Cp. fr. \textunderscore cintrer\textunderscore )
\section{Cintran}
\begin{itemize}
\item {Grp. gram.:f.}
\end{itemize}
\begin{itemize}
\item {Proveniência:(De \textunderscore cintrão\textunderscore )}
\end{itemize}
Mulher de Cintra.
\section{Cintrão}
\begin{itemize}
\item {Grp. gram.:m.}
\end{itemize}
Habitante de Cintra.
\section{Cintura}
\begin{itemize}
\item {Grp. gram.:f.}
\end{itemize}
\begin{itemize}
\item {Proveniência:(Lat. \textunderscore cinctura\textunderscore )}
\end{itemize}
A parte média do tronco humano.
Cinta.
A parte do vestuário, que rodeia o meio do tronco.
\section{Cinturado}
\begin{itemize}
\item {Grp. gram.:adj.}
\end{itemize}
Apertado na cintura.
\section{Cinturão}
\begin{itemize}
\item {Grp. gram.:m.}
\end{itemize}
\begin{itemize}
\item {Proveniência:(De \textunderscore cintura\textunderscore )}
\end{itemize}
Grande cinta, geralmente de coiro, em que se suspendem armas, em que se traz dinheiro, etc.
\section{Cinyra}
\begin{itemize}
\item {Grp. gram.:f.}
\end{itemize}
\begin{itemize}
\item {Proveniência:(Gr. \textunderscore kinura\textunderscore )}
\end{itemize}
Espécie de lyra ou cíthara, entre os Hebreus, Sýrios e Phenícios.
\section{Cinza}
\begin{itemize}
\item {Grp. gram.:f.}
\end{itemize}
\begin{itemize}
\item {Utilização:Fig.}
\end{itemize}
\begin{itemize}
\item {Proveniência:(Lat. hyp. \textunderscore cinitia\textunderscore , de \textunderscore cinis\textunderscore )}
\end{itemize}
Pó ou resíduos da combustão de certas substâncias.
Aniquilamento.
Luto.
Humilhação.
Restos mortaes.
Memória dos finados.
\section{Cinzal}
\begin{itemize}
\item {Grp. gram.:m.}
\end{itemize}
\begin{itemize}
\item {Proveniência:(De \textunderscore cinza\textunderscore )}
\end{itemize}
Espécie de uva preta minhota.
\section{Cinzão}
\begin{itemize}
\item {Grp. gram.:m.}
\end{itemize}
\begin{itemize}
\item {Proveniência:(De \textunderscore cinza\textunderscore )}
\end{itemize}
Espécie de uva minhota.
\section{Cinzar}
\begin{itemize}
\item {Grp. gram.:v. t.}
\end{itemize}
\begin{itemize}
\item {Utilização:Bras. de Minas}
\end{itemize}
\begin{itemize}
\item {Grp. gram.:V. i.}
\end{itemize}
Enganar, illudir.
Uma das operações, nas fábricas de papel:«\textunderscore cylindros de cinzar\textunderscore ». \textunderscore Inquér. Industr.\textunderscore , P. II, l. 3.^o, 223.
\section{Cinzeiro}
\begin{itemize}
\item {Grp. gram.:m.}
\end{itemize}
\begin{itemize}
\item {Proveniência:(De \textunderscore cinza\textunderscore )}
\end{itemize}
Monte de cinzas.
Lugar, em que cai a cinza do fogão.
Pequeno objecto de loiça ou metal, em que os fumadores deitam a cinza do tabaco.
O mesmo que \textunderscore oídio\textunderscore .
\section{Cinzel}
\begin{itemize}
\item {Grp. gram.:m.}
\end{itemize}
Instrumento cortante em uma das extremidades, usado principalmente por esculptores e gravadores.
(Cast. \textunderscore cincel\textunderscore )
\section{Cinzelado}
\begin{itemize}
\item {Grp. gram.:adj.}
\end{itemize}
\begin{itemize}
\item {Proveniência:(De \textunderscore cinzelar\textunderscore )}
\end{itemize}
Lavrado com cinzel; esculpido.
\section{Cinzelador}
\begin{itemize}
\item {Grp. gram.:m.  e  adj.}
\end{itemize}
O que cinzela.
\section{Cinzeladura}
\begin{itemize}
\item {Grp. gram.:f.}
\end{itemize}
Effeito de cinzelar.
\section{Cinzelamento}
\begin{itemize}
\item {Grp. gram.:m.}
\end{itemize}
Acto ou effeito de \textunderscore cinzelar\textunderscore . Cf. Arn. Gama, \textunderscore Segr. do Ab.\textunderscore , 360.
\section{Cinzelar}
\begin{itemize}
\item {Grp. gram.:v. t.}
\end{itemize}
\begin{itemize}
\item {Utilização:Fig.}
\end{itemize}
Lavrar com cinzel; esculpir.
Aprimorar; fazer delicadamente: \textunderscore cinzelar estrophes\textunderscore .
\section{Cinzento}
\begin{itemize}
\item {Grp. gram.:adj.}
\end{itemize}
Que tem côr de cinza.
\section{Cio}
\begin{itemize}
\item {Grp. gram.:m.}
\end{itemize}
\begin{itemize}
\item {Utilização:Prov.}
\end{itemize}
\begin{itemize}
\item {Utilização:trasm.}
\end{itemize}
\begin{itemize}
\item {Proveniência:(Do lat. \textunderscore zelus\textunderscore )}
\end{itemize}
Appetite sexual dos animaes em certos períodos.
Viço, vigor (de plantas).
\section{Cioba}
\begin{itemize}
\item {Grp. gram.:f.}
\end{itemize}
Peixe de Pernambuco.
\section{Ciocho}
\begin{itemize}
\item {Grp. gram.:m.}
\end{itemize}
O mesmo que \textunderscore cicia\textunderscore .
\section{Ciocoto}
\begin{itemize}
\item {Grp. gram.:m.}
\end{itemize}
Árvore da ilha de San-Thomé.
\section{Cionite}
\begin{itemize}
\item {Grp. gram.:f.}
\end{itemize}
\begin{itemize}
\item {Utilização:Med.}
\end{itemize}
\begin{itemize}
\item {Proveniência:(Do gr. \textunderscore kion\textunderscore )}
\end{itemize}
Inflammação da úvula.
\section{Cíono}
\begin{itemize}
\item {Grp. gram.:m.}
\end{itemize}
\begin{itemize}
\item {Proveniência:(Do gr. \textunderscore kion\textunderscore )}
\end{itemize}
Insecto coleóptero tetrâmero.
\section{Ciosamente}
\begin{itemize}
\item {Grp. gram.:adv.}
\end{itemize}
De modo cioso.
Com ciúme.
\section{Cioso}
\begin{itemize}
\item {Grp. gram.:adj.}
\end{itemize}
Que tem ciumes; ciumento.
Invejoso.
Interessado, por affeição extrema.
Procedente de ciumes.
(Cp. \textunderscore cio\textunderscore )
\section{Cioto}
\begin{itemize}
\item {fónica:ô}
\end{itemize}
\begin{itemize}
\item {Grp. gram.:m.}
\end{itemize}
O mesmo que \textunderscore cicia\textunderscore .
\section{Ciótomo}
\begin{itemize}
\item {Grp. gram.:m.}
\end{itemize}
\begin{itemize}
\item {Proveniência:(Do gr. \textunderscore kion\textunderscore  + \textunderscore tome\textunderscore )}
\end{itemize}
Instrumento cirúrgico, que se emprega nos apertos accidentaes do intestino recto e da bexiga.
\section{Cipaios}
\begin{itemize}
\item {Grp. gram.:m. pl.}
\end{itemize}
\begin{itemize}
\item {Proveniência:(Do pers. \textunderscore sipahi\textunderscore )}
\end{itemize}
Soldados indígenas da Índia, ao serviço de Ingleses.
\section{Cipais}
\begin{itemize}
\item {Grp. gram.:m. pl.}
\end{itemize}
\begin{itemize}
\item {Proveniência:(Do pers. \textunderscore sipahi\textunderscore )}
\end{itemize}
Soldados indígenas da Índia, ao serviço de Ingleses.
\section{Ciparaba}
\begin{itemize}
\item {Grp. gram.:f.}
\end{itemize}
Planta trepadeira do Brasil.
\section{Cipó}
\begin{itemize}
\item {Grp. gram.:m.}
\end{itemize}
\begin{itemize}
\item {Utilização:Bras}
\end{itemize}
\begin{itemize}
\item {Grp. gram.:Adj.}
\end{itemize}
Nome commum das plantas sarmentosas do sertão.
Trepadeira convolvulácea.
Diz-se de uma cobra, pela sua semelhança com o tronco do cipó.
(Do tupi)
\section{Cipoada}
\begin{itemize}
\item {Grp. gram.:f.}
\end{itemize}
Pancada com cipó.
\section{Cipoal}
\begin{itemize}
\item {Grp. gram.:m.}
\end{itemize}
\begin{itemize}
\item {Utilização:Bras}
\end{itemize}
Mata de cipós.
Difficuldade.
\section{Cipoar}
\begin{itemize}
\item {Grp. gram.:v. t.}
\end{itemize}
Bater com cipó.
\section{Cipó-carneiro}
\begin{itemize}
\item {Grp. gram.:m.}
\end{itemize}
\begin{itemize}
\item {Utilização:Bras}
\end{itemize}
Planta apocýnea medicinal, (\textunderscore echites suberosa\textunderscore ).
\section{Cipó-chumbo}
\begin{itemize}
\item {Grp. gram.:m.}
\end{itemize}
\begin{itemize}
\item {Utilização:Bras}
\end{itemize}
Planta medicinal, (\textunderscore cuscuta umbellata\textunderscore ).
\section{Cipó-cravo}
\begin{itemize}
\item {Grp. gram.:m.}
\end{itemize}
Planta brasileira e medicinal.
\section{Cipó-cruz}
\begin{itemize}
\item {Grp. gram.:m.}
\end{itemize}
\begin{itemize}
\item {Utilização:Bras}
\end{itemize}
O mesmo que \textunderscore cainca\textunderscore .
\section{Cipó-do-imbê}
\begin{itemize}
\item {Grp. gram.:m.}
\end{itemize}
\begin{itemize}
\item {Utilização:Bras}
\end{itemize}
Planta arácea, medicinal, (\textunderscore arum arborescens\textunderscore ).
\section{Cipolino}
\begin{itemize}
\item {Grp. gram.:m.}
\end{itemize}
\begin{itemize}
\item {Proveniência:(It. \textunderscore cipollino\textunderscore )}
\end{itemize}
Mármore verde e branco.
\section{Cipó-sumá}
\begin{itemize}
\item {Grp. gram.:m.}
\end{itemize}
\begin{itemize}
\item {Utilização:Bras}
\end{itemize}
Planta violácea, medicinal, (\textunderscore anchietea salutaris\textunderscore ).
\section{Cipotada}
\begin{itemize}
\item {Grp. gram.:f.}
\end{itemize}
\begin{itemize}
\item {Utilização:Prov.}
\end{itemize}
\begin{itemize}
\item {Utilização:beir.}
\end{itemize}
\begin{itemize}
\item {Utilização:trasm.}
\end{itemize}
Pancada com cipote.
\section{Cipote}
\begin{itemize}
\item {Grp. gram.:m.}
\end{itemize}
\begin{itemize}
\item {Utilização:Prov.}
\end{itemize}
\begin{itemize}
\item {Utilização:beir.}
\end{itemize}
\begin{itemize}
\item {Utilização:trasm.}
\end{itemize}
Cacete grande.
(Cp. \textunderscore cipó\textunderscore )
\section{Cippo}
\begin{itemize}
\item {Grp. gram.:m.}
\end{itemize}
\begin{itemize}
\item {Proveniência:(Lat. \textunderscore cippus\textunderscore )}
\end{itemize}
Pequena columna sem capitel.
Antigo marco milliário; marco.
Columna, em que se affixavam as decisões do senado romano e outras instrucções de interesse público.
Estrado, em que se collocava o esquife.
Pedra tumular, em Roma.
\section{Cipo}
\begin{itemize}
\item {Grp. gram.:m.}
\end{itemize}
\begin{itemize}
\item {Proveniência:(Lat. \textunderscore cippus\textunderscore )}
\end{itemize}
Pequena columna sem capitel.
Antigo marco miliário; marco.
Coluna, em que se afixavam as decisões do senado romano e outras instrucções de interesse público.
Estrado, em que se colocava o esquife.
Pedra tumular, em Roma.
\section{Cipre}
\textunderscore m.\textunderscore  (ou \textunderscore f.\textunderscore ?)
Perfume de Chypre?«\textunderscore ...afora saias, cotas e cipres de dona...\textunderscore »Fernam Lopes, \textunderscore Chrón. de D. Fernando\textunderscore , cap. XLIX.
\section{Cipriota}
\begin{itemize}
\item {Grp. gram.:m.  e  adj.}
\end{itemize}
\begin{itemize}
\item {Grp. gram.:adj.}
\end{itemize}
\begin{itemize}
\item {Grp. gram.:M.}
\end{itemize}
O que é de Chipre.
O mesmo que \textunderscore cíprio\textunderscore .
Habitante de Cipro.
\section{Cipura}
\begin{itemize}
\item {Grp. gram.:f.}
\end{itemize}
\begin{itemize}
\item {Proveniência:(T. ind.)}
\end{itemize}
Gênero de plantas irídeas.
\section{Cira}
\begin{itemize}
\item {Grp. gram.:f.}
\end{itemize}
\begin{itemize}
\item {Utilização:Ant.}
\end{itemize}
\begin{itemize}
\item {Proveniência:(Do lat. \textunderscore xyris\textunderscore ?)}
\end{itemize}
Mata; brenha.
Matagal.
Chavascal.
\section{Ciranda}
\begin{itemize}
\item {Grp. gram.:f.}
\end{itemize}
\begin{itemize}
\item {Proveniência:(Do cast. \textunderscore zaranda\textunderscore )}
\end{itemize}
Peneira grossa, com que se joeira areia, grãos, etc.
Dança popular.
\section{Cirandagem}
\begin{itemize}
\item {Grp. gram.:f.}
\end{itemize}
Acto de cirandar.
Porção joeirada, limpa.
Palhas, que vôam da ciranda.
\section{Cirandão}
\begin{itemize}
\item {Grp. gram.:m.}
\end{itemize}
Peneiro, ciranda grande.
\section{Cirandar}
\begin{itemize}
\item {Grp. gram.:v. t.}
\end{itemize}
\begin{itemize}
\item {Grp. gram.:V. i.}
\end{itemize}
\begin{itemize}
\item {Utilização:Fam.}
\end{itemize}
Joeirar com ciranda.
Dar voltas.
Dançar de roda:«\textunderscore vamos nós a cirandar\textunderscore ». (De uma canção pop.)
\section{Cirandinha}
\begin{itemize}
\item {Grp. gram.:f.}
\end{itemize}
Dança, o mesmo que \textunderscore ciranda\textunderscore .
\section{Cirata}
\begin{itemize}
\item {Grp. gram.:f.}
\end{itemize}
\begin{itemize}
\item {Utilização:Des.}
\end{itemize}
Espécie de xairel.
\section{Circassiano}
\begin{itemize}
\item {Grp. gram.:m.  e  adj.}
\end{itemize}
O que é da Circássia.
\section{Circatejano}
\begin{itemize}
\item {Grp. gram.:adj.}
\end{itemize}
\begin{itemize}
\item {Proveniência:(De \textunderscore circa\textunderscore , lat. + \textunderscore Tejo\textunderscore , n. p.)}
\end{itemize}
Que é das margens do Tejo. Cf. Garrett, \textunderscore Romanceiro\textunderscore , II, 171.
\section{Circéa}
\begin{itemize}
\item {Grp. gram.:f.}
\end{itemize}
\begin{itemize}
\item {Proveniência:(De \textunderscore Círce\textunderscore , n. p.)}
\end{itemize}
Planta vivaz, também conhecida por \textunderscore erva-de-santo-Estêvão\textunderscore .
\section{Círceas}
\begin{itemize}
\item {Grp. gram.:f. pl.}
\end{itemize}
\begin{itemize}
\item {Proveniência:(De \textunderscore circeia\textunderscore )}
\end{itemize}
Tríbo de plantas onagrárias, na classificação de De-Candolle.
\section{Circeia}
\begin{itemize}
\item {Grp. gram.:f.}
\end{itemize}
\begin{itemize}
\item {Proveniência:(De \textunderscore Círce\textunderscore , n. p.)}
\end{itemize}
Planta vivaz, também conhecida por \textunderscore erva-de-santo-Estêvão\textunderscore .
\section{Circense}
\begin{itemize}
\item {Grp. gram.:adj.}
\end{itemize}
\begin{itemize}
\item {Grp. gram.:M. pl.}
\end{itemize}
\begin{itemize}
\item {Proveniência:(Lat. \textunderscore circensis\textunderscore )}
\end{itemize}
Relativo ao circo.
Espectáculos de circo.
\section{Circiadela}
\begin{itemize}
\item {Grp. gram.:f.}
\end{itemize}
Acto de circiar.
\section{Circiar}
\begin{itemize}
\item {Grp. gram.:v. t.}
\end{itemize}
\begin{itemize}
\item {Proveniência:(De \textunderscore circio\textunderscore )}
\end{itemize}
Passar com o circio.
\section{Circinal}
\begin{itemize}
\item {Grp. gram.:adj.}
\end{itemize}
\begin{itemize}
\item {Utilização:Bot.}
\end{itemize}
\begin{itemize}
\item {Proveniência:(Do lat. \textunderscore circinus\textunderscore )}
\end{itemize}
Diz-se dos cotylédones, quando enrolados em espiral sôbre si mesmos.
Diz-se do verticillo, quando disposto como as frondes dos fêtos.
Enrolado em espiral.
\section{Círcio}
\begin{itemize}
\item {Grp. gram.:m.}
\end{itemize}
Cylindro de madeira, com que os marnotos curam o solo das marinhas.
(Refl. de \textunderscore cercear\textunderscore ?)
\section{Circo}
\begin{itemize}
\item {Grp. gram.:m.}
\end{itemize}
\begin{itemize}
\item {Proveniência:(Lat. \textunderscore circus\textunderscore )}
\end{itemize}
Antigo e grande recinto, para jogos públicos.
Amphitheatro, recinto circular para espectáculos de gymnástica, acrobatismo, equïtação, etc.
Cincho.
Círculo.
\section{Circuição}
\begin{itemize}
\item {fónica:cu-i}
\end{itemize}
\begin{itemize}
\item {Grp. gram.:f.}
\end{itemize}
\begin{itemize}
\item {Proveniência:(Lat. \textunderscore circuitio\textunderscore )}
\end{itemize}
Acto de andar á roda.
\section{Circuitar}
\begin{itemize}
\item {Grp. gram.:v. i.}
\end{itemize}
\begin{itemize}
\item {Proveniência:(De \textunderscore circuito\textunderscore )}
\end{itemize}
Andar á roda.
\section{Circuito}
\begin{itemize}
\item {Grp. gram.:m.}
\end{itemize}
\begin{itemize}
\item {Proveniência:(Lat. \textunderscore circuitus\textunderscore )}
\end{itemize}
O mesmo que \textunderscore circumferência\textunderscore .
Volta.
Cêrca.
Rodeio.
Successão de phenómenos periódicos.
\section{Circulação}
\begin{itemize}
\item {Grp. gram.:f.}
\end{itemize}
Acto de circular.
\section{Circulante}
\begin{itemize}
\item {Grp. gram.:adj.}
\end{itemize}
Que circula.
\section{Circular}
\begin{itemize}
\item {Grp. gram.:adj.}
\end{itemize}
\begin{itemize}
\item {Grp. gram.:F.}
\end{itemize}
\begin{itemize}
\item {Proveniência:(Lat. \textunderscore circularis\textunderscore )}
\end{itemize}
Que tem fórma de círculo.
Que volta ao ponto donde partiu.
Diz-se de uma carta, manifesto ou offício que, reproduzido em muitos exemplares, é dirigido a muitas pessoas.
Carta, manifesto ou offício circular: \textunderscore distribuiu-se uma circular do inspector\textunderscore .
\section{Circular}
\begin{itemize}
\item {Grp. gram.:v. t.}
\end{itemize}
\begin{itemize}
\item {Grp. gram.:V. i.}
\end{itemize}
\begin{itemize}
\item {Proveniência:(Lat. \textunderscore circulare\textunderscore )}
\end{itemize}
Rodear.
Guarnecer em volta.
Mover-se circularmente, voltando ao ponto da partida.
Renovar-se (o ar).
Passar de mão em mão, propagar-se: \textunderscore circulam vários jornaes\textunderscore .
Sêr recebido em commércio, como representativo de valor: \textunderscore os patacos já não circulam\textunderscore .
Transitar facilmente pelas ruas.
Viajar.
\section{Circularmente}
\begin{itemize}
\item {Grp. gram.:adv.}
\end{itemize}
Em volta; de modo circular.
\section{Circulatório}
\begin{itemize}
\item {Grp. gram.:adj.}
\end{itemize}
\begin{itemize}
\item {Proveniência:(Lat. \textunderscore circulatorius\textunderscore )}
\end{itemize}
Relativo a circulação.
\section{Círculo}
\begin{itemize}
\item {Grp. gram.:m.}
\end{itemize}
\begin{itemize}
\item {Utilização:Fig.}
\end{itemize}
\begin{itemize}
\item {Proveniência:(Lat. \textunderscore circulus\textunderscore )}
\end{itemize}
Figura plana, limitada por uma circunferência; circunferência.
Circo.
Cinto.
Arco.
Anel.
Giro.
Área.
Circunscripção territorial: \textunderscore círculos eleitoraes\textunderscore .
Assembleia, ponto de reunião.
\section{Circum...}
\begin{itemize}
\item {Grp. gram.:pref.}
\end{itemize}
Que significa \textunderscore em roda\textunderscore , e que, juntando-se a palavras começadas por consoante, que não seja labial, póde ou deve escrever-se \textunderscore circun...\textunderscore 
\section{Circum-ambiente}
\begin{itemize}
\item {Grp. gram.:adj.}
\end{itemize}
\begin{itemize}
\item {Proveniência:(De \textunderscore circum...\textunderscore  + \textunderscore ambiente\textunderscore )}
\end{itemize}
Que está em volta.
\section{Circumcidado}
\begin{itemize}
\item {Grp. gram.:adj.}
\end{itemize}
Em quem se fez a circumcisão.
\section{Circumcidar}
\begin{itemize}
\item {Grp. gram.:v. t.}
\end{itemize}
\begin{itemize}
\item {Proveniência:(Lat. \textunderscore circumcidere\textunderscore )}
\end{itemize}
Fazer a circuncisão em.
\section{Circumcisão}
\begin{itemize}
\item {Grp. gram.:f.}
\end{itemize}
\begin{itemize}
\item {Proveniência:(Lat. \textunderscore circumcisio\textunderscore )}
\end{itemize}
Acto de cortar o prepúcio.
Celebração da circumcisão de Christo.
\section{Circumciso}
\begin{itemize}
\item {Grp. gram.:m.}
\end{itemize}
\begin{itemize}
\item {Utilização:Deprec.}
\end{itemize}
\begin{itemize}
\item {Proveniência:(Lat. \textunderscore circumcisus\textunderscore )}
\end{itemize}
Homem circumcidado.
Judeu.
\section{Circumdamento}
\begin{itemize}
\item {Grp. gram.:m.}
\end{itemize}
\begin{itemize}
\item {Proveniência:(De \textunderscore circumdar\textunderscore )}
\end{itemize}
Circuito; barreira.
Divisa.
\section{Circumdante}
\begin{itemize}
\item {Grp. gram.:adj.}
\end{itemize}
Que circunda.
\section{Circumdar}
\begin{itemize}
\item {Grp. gram.:v. t.}
\end{itemize}
\begin{itemize}
\item {Proveniência:(Lat. \textunderscore circumdare\textunderscore )}
\end{itemize}
Rodear; andar á volta de.
\section{Circumducção}
\begin{itemize}
\item {Grp. gram.:f.}
\end{itemize}
\begin{itemize}
\item {Proveniência:(Lat. \textunderscore circumductio\textunderscore )}
\end{itemize}
Rotação em volta de um centro ou eixo.
\section{Circumductar}
\begin{itemize}
\item {Grp. gram.:v. t.}
\end{itemize}
\begin{itemize}
\item {Proveniência:(De \textunderscore circumducto\textunderscore )}
\end{itemize}
Julgar nullo.
\section{Circumferência}
\begin{itemize}
\item {Grp. gram.:f.}
\end{itemize}
\begin{itemize}
\item {Proveniência:(Lat. \textunderscore circumferentia\textunderscore )}
\end{itemize}
Linha que fêcha um circulo; peripheria.
Circuito, linha que fêcha qualquer área.
\section{Circumferente}
\begin{itemize}
\item {Grp. gram.:adj.}
\end{itemize}
\begin{itemize}
\item {Proveniência:(Lat. \textunderscore circumferens\textunderscore )}
\end{itemize}
Que gira, que anda á volta.
\section{Circumflexamente}
\begin{itemize}
\item {Grp. gram.:adv.}
\end{itemize}
Por meio do assento circumflexo.
\section{Circumflexão}
\begin{itemize}
\item {Grp. gram.:f.}
\end{itemize}
\begin{itemize}
\item {Proveniência:(Lat. \textunderscore circumflexio\textunderscore )}
\end{itemize}
Acto de dobrar em roda.
\section{Circumflexo}
\begin{itemize}
\item {Grp. gram.:adj.}
\end{itemize}
\begin{itemize}
\item {Proveniência:(Lat. \textunderscore circumflexus\textunderscore )}
\end{itemize}
Recurvado em roda.
\textunderscore Accento circunflexo\textunderscore , sinal que, entre nós, dá ás vogaes \textunderscore e\textunderscore  e \textunderscore o\textunderscore  um som médio entre o agudo e o tênue e tira á vogal \textunderscore a\textunderscore  o som agudo; isto é, dá ao \textunderscore a\textunderscore , ao \textunderscore e\textunderscore , e ao \textunderscore o\textunderscore  o valor de vogaes fechadas.
\section{Circumfluência}
\begin{itemize}
\item {Grp. gram.:f.}
\end{itemize}
\begin{itemize}
\item {Proveniência:(Do lat. \textunderscore circumfluens\textunderscore )}
\end{itemize}
Movimento circular de um líquido ou de um fluido.
\section{Circumfluente}
\begin{itemize}
\item {Grp. gram.:adj.}
\end{itemize}
\begin{itemize}
\item {Proveniência:(Lat. \textunderscore circumfluens\textunderscore )}
\end{itemize}
Que corre em volta.
\section{Circumfluir}
\begin{itemize}
\item {Grp. gram.:v. t.}
\end{itemize}
\begin{itemize}
\item {Proveniência:(Lat. \textunderscore circumfluere\textunderscore )}
\end{itemize}
Fluir em roda.
\section{Circumforâneo}
\begin{itemize}
\item {Grp. gram.:adj.}
\end{itemize}
\begin{itemize}
\item {Utilização:Des.}
\end{itemize}
\begin{itemize}
\item {Proveniência:(Lat. \textunderscore circumforâneus\textunderscore )}
\end{itemize}
Que anda pelas praças ou á roda dellas; ambulante.
Próprio de charlatão de praça.
\section{Circumfundir}
\begin{itemize}
\item {Grp. gram.:v. t.}
\end{itemize}
\begin{itemize}
\item {Proveniência:(Lat. \textunderscore circumfundere\textunderscore )}
\end{itemize}
Espalhar em volta.
Entornar, derramar em volta.
\section{Circumfusão}
\begin{itemize}
\item {Grp. gram.:f.}
\end{itemize}
Acto de circumfundir.
\section{Circumgirar}
\begin{itemize}
\item {Grp. gram.:v. i.}
\end{itemize}
\begin{itemize}
\item {Proveniência:(De \textunderscore circum...\textunderscore  + \textunderscore girar\textunderscore )}
\end{itemize}
Girar em volta.
\section{Circumjacente}
\begin{itemize}
\item {Grp. gram.:adj.}
\end{itemize}
\begin{itemize}
\item {Proveniência:(Lat. \textunderscore circumjacens\textunderscore )}
\end{itemize}
Que está situado em roda; circumvizinho.
\section{Circumjazer}
\begin{itemize}
\item {Grp. gram.:v. i.}
\end{itemize}
Está em volta; sêr circumvizinho.
\section{Circumlocução}
\begin{itemize}
\item {Grp. gram.:f.}
\end{itemize}
O mesmo que \textunderscore circumlóquio\textunderscore .
\section{Circumlóquio}
\begin{itemize}
\item {Grp. gram.:m.}
\end{itemize}
\begin{itemize}
\item {Proveniência:(Do lat. \textunderscore circumloqui\textunderscore )}
\end{itemize}
Rodeio de palávras; períphrase.
\section{Circummurado}
\begin{itemize}
\item {Grp. gram.:adj.}
\end{itemize}
\begin{itemize}
\item {Proveniência:(De \textunderscore circum...\textunderscore  + \textunderscore muro\textunderscore )}
\end{itemize}
Que tem muro em volta.
\section{Circum-navegação}
\begin{itemize}
\item {Grp. gram.:f.}
\end{itemize}
Acto de circum-navegar.
\section{Circum-navegador}
\begin{itemize}
\item {Grp. gram.:m.}
\end{itemize}
Aquelle que circum-navega.
\section{Circum-navegar}
\begin{itemize}
\item {Grp. gram.:v. t.  e  i.}
\end{itemize}
\begin{itemize}
\item {Proveniência:(Lat. \textunderscore circumnavigare\textunderscore )}
\end{itemize}
Rodear, navegando.
Navegar á volta do globo, de uma ilha, de um continente.
\section{Circumpatente}
\begin{itemize}
\item {Grp. gram.:adj.}
\end{itemize}
Aberto em roda; patente por todos os lados:«\textunderscore região circumpatente\textunderscore ». Castilho, \textunderscore Metam.\textunderscore , 282.
\section{Circumpercorrer}
\begin{itemize}
\item {Grp. gram.:v. t.}
\end{itemize}
Percorrer em tôrno.
\section{Circumpolar}
\begin{itemize}
\item {Grp. gram.:adj.}
\end{itemize}
\begin{itemize}
\item {Proveniência:(De \textunderscore circum...\textunderscore  + \textunderscore polar\textunderscore )}
\end{itemize}
Que está perto do polo, em volta do polo.
\section{Circumposto}
\begin{itemize}
\item {Grp. gram.:adj.}
\end{itemize}
Posto á roda:«\textunderscore as almofadas circumpostas no camarim\textunderscore ». Camillo, \textunderscore Cav. em Ruínas\textunderscore , 105.
\section{Circumrevoluto}
\begin{itemize}
\item {Grp. gram.:adj.}
\end{itemize}
Enrolado em volta de alguma coisa:«\textunderscore o lenço circumrevoluto á feição de turbante\textunderscore ». Castilho, \textunderscore Mil e Um Myst.\textunderscore 
\section{Circumscrever}
\begin{itemize}
\item {Grp. gram.:v. t.}
\end{itemize}
\begin{itemize}
\item {Proveniência:(Lat. \textunderscore circumscribere\textunderscore )}
\end{itemize}
Limitar com uma linha ou com um círculo.
Limitar.
Traçar á roda.
Abranger.
\section{Circumscripção}
\begin{itemize}
\item {Grp. gram.:f.}
\end{itemize}
\begin{itemize}
\item {Proveniência:(Lat. \textunderscore circumscriptio\textunderscore )}
\end{itemize}
Acto de circumscrever.
Linha, que limita de todos os lados uma área.
Divisão territorial.
\section{Circumscriptivo}
\begin{itemize}
\item {Grp. gram.:adj.}
\end{itemize}
Que circumscreve ou limita.
\section{Circumscripto}
\begin{itemize}
\item {Grp. gram.:adj.}
\end{itemize}
Limitado, de todos os lados, por uma linha.
\section{Circumsoante}
\begin{itemize}
\item {Grp. gram.:adj.}
\end{itemize}
\begin{itemize}
\item {Proveniência:(Lat. \textunderscore circumsonans\textunderscore )}
\end{itemize}
Que sôa em roda.
\section{Circumsonante}
\begin{itemize}
\item {Grp. gram.:adj.}
\end{itemize}
\begin{itemize}
\item {Proveniência:(Lat. \textunderscore circumsonans\textunderscore )}
\end{itemize}
Que sôa em roda.
\section{Circumspecção}
\begin{itemize}
\item {Grp. gram.:f.}
\end{itemize}
\begin{itemize}
\item {Proveniência:(Lat. \textunderscore circumspectio\textunderscore )}
\end{itemize}
Qualidade de quem é circumspecto.
\section{Circumspectamente}
\begin{itemize}
\item {Grp. gram.:adv.}
\end{itemize}
\begin{itemize}
\item {Proveniência:(De \textunderscore circumspecto\textunderscore )}
\end{itemize}
Com circumspecção.
\section{Circumspecto}
\begin{itemize}
\item {Grp. gram.:adj.}
\end{itemize}
\begin{itemize}
\item {Proveniência:(Lat. \textunderscore circumspectus\textunderscore )}
\end{itemize}
Que olha á volta de si.
Cauteloso; prudente.
Em que há circumspecção: \textunderscore procedimento circumspecto\textunderscore .
\section{Circumstância}
\begin{itemize}
\item {Grp. gram.:f.}
\end{itemize}
\begin{itemize}
\item {Proveniência:(Lat. \textunderscore circumstantia\textunderscore )}
\end{itemize}
Estado; particularidade que acompanha um facto.
Motivo ou facto, que, acompanhando, seguindo ou precedendo outro facto, o aggrava ou attenua: \textunderscore circumstâncias atenuantes\textunderscore .
\section{Circumstanciadamente}
\begin{itemize}
\item {Grp. gram.:adv.}
\end{itemize}
\begin{itemize}
\item {Proveniência:(De \textunderscore circumstanciado\textunderscore )}
\end{itemize}
Pormenorizadamente.
\section{Circumstanciado}
\begin{itemize}
\item {Grp. gram.:adj.}
\end{itemize}
Exposto minuciosamente.
\section{Circumstanciador}
\begin{itemize}
\item {Grp. gram.:adj.}
\end{itemize}
Que circumstancía.
\section{Circumstancial}
\begin{itemize}
\item {Grp. gram.:adj.}
\end{itemize}
Que exprime uma circumstância.
\section{Circumstanciar}
\begin{itemize}
\item {Grp. gram.:v. t.}
\end{itemize}
\begin{itemize}
\item {Proveniência:(De \textunderscore circumstancia\textunderscore )}
\end{itemize}
Descrever minuciosamente, com todas as circumstâncias.
\section{Circumstante}
\begin{itemize}
\item {Grp. gram.:adj.}
\end{itemize}
\begin{itemize}
\item {Grp. gram.:M.  e  f.}
\end{itemize}
\begin{itemize}
\item {Grp. gram.:Pl.}
\end{itemize}
\begin{itemize}
\item {Proveniência:(De \textunderscore circunstar\textunderscore )}
\end{itemize}
Que está á volta, circumjacente.
Pessôa que está presente.
Auditório.
\section{Circumstar}
\begin{itemize}
\item {Grp. gram.:v. t.  e  i.}
\end{itemize}
\begin{itemize}
\item {Proveniência:(Do lat. \textunderscore circum\textunderscore  + \textunderscore stare\textunderscore )}
\end{itemize}
Estar em roda; estar perto, á vista.
\section{Circumstoso}
\begin{itemize}
\item {Grp. gram.:adj.}
\end{itemize}
Diffícil.
Pouco provável. (Colhido em Turquel)
(Cp. \textunderscore circumstar\textunderscore )
\section{Circumvagante}
\begin{itemize}
\item {Grp. gram.:adj.}
\end{itemize}
Que circumvaga.
\section{Circumvagar}
\begin{itemize}
\item {Grp. gram.:v. i.}
\end{itemize}
\begin{itemize}
\item {Proveniência:(Lat. \textunderscore circumvagari\textunderscore )}
\end{itemize}
Andar em volta; divagar.
\section{Circúmvago}
\begin{itemize}
\item {Grp. gram.:adj.}
\end{itemize}
O mesmo que \textunderscore circumvagante\textunderscore .
\section{Circumvallação}
\begin{itemize}
\item {Grp. gram.:f.}
\end{itemize}
\begin{itemize}
\item {Proveniência:(De \textunderscore circumvallar\textunderscore )}
\end{itemize}
Fôsso, valla com parapeito, que corta as communicações de uma praça com o exterior.
Barreiras, em volta de uma povoação.
\section{Circumvallar}
\begin{itemize}
\item {Grp. gram.:v. t.}
\end{itemize}
\begin{itemize}
\item {Proveniência:(Lat. \textunderscore circumvallare\textunderscore )}
\end{itemize}
Cingir de fossos ou barreiras.
\section{Circumvizinhança}
\begin{itemize}
\item {Grp. gram.:f.}
\end{itemize}
\begin{itemize}
\item {Proveniência:(De \textunderscore circum...\textunderscore  + \textunderscore vizinhança\textunderscore )}
\end{itemize}
Subúrbio.
População vizinha; arredores.
\section{Circumvizinhar}
\begin{itemize}
\item {Grp. gram.:v. i.}
\end{itemize}
\begin{itemize}
\item {Proveniência:(De \textunderscore circumvizinho\textunderscore )}
\end{itemize}
Estar na vizinhança.
\section{Circumvizinho}
\begin{itemize}
\item {Grp. gram.:adj.}
\end{itemize}
\begin{itemize}
\item {Proveniência:(De \textunderscore circum...\textunderscore  + \textunderscore vizinho\textunderscore )}
\end{itemize}
Que está próximo ou em volta; confinante.
\section{Circumvolução}
\begin{itemize}
\item {Grp. gram.:f.}
\end{itemize}
\begin{itemize}
\item {Proveniência:(Do lat. \textunderscore circumvolutus\textunderscore )}
\end{itemize}
Movimento em volta de um centro.
Contôrno, sinuosidade.
\section{Circumvolucionário}
\begin{itemize}
\item {Grp. gram.:adj.}
\end{itemize}
\begin{itemize}
\item {Proveniência:(De \textunderscore circumvolução\textunderscore )}
\end{itemize}
Relativo ás circumvoluções do cérebro.
\section{Circumvolver}
\begin{itemize}
\item {Grp. gram.:v. t.}
\end{itemize}
\begin{itemize}
\item {Proveniência:(Lat. \textunderscore circumvolvere\textunderscore )}
\end{itemize}
Volvêr em roda.
\section{Circuncidado}
\begin{itemize}
\item {Grp. gram.:adj.}
\end{itemize}
Em quem se fez a circuncisão.
\section{Circuncidar}
\begin{itemize}
\item {Grp. gram.:v. t.}
\end{itemize}
\begin{itemize}
\item {Proveniência:(Lat. \textunderscore circumcidere\textunderscore )}
\end{itemize}
Fazer a circuncisão em.
\section{Circuncisão}
\begin{itemize}
\item {Grp. gram.:f.}
\end{itemize}
\begin{itemize}
\item {Proveniência:(Lat. \textunderscore circumcisio\textunderscore )}
\end{itemize}
Acto de cortar o prepúcio.
Celebração da circuncisão de Cristo.
\section{Circunciso}
\begin{itemize}
\item {Grp. gram.:m.}
\end{itemize}
\begin{itemize}
\item {Utilização:Deprec.}
\end{itemize}
\begin{itemize}
\item {Proveniência:(Lat. \textunderscore circumcisus\textunderscore )}
\end{itemize}
Homem circuncidado.
Judeu.
\section{Circundamento}
\begin{itemize}
\item {Grp. gram.:m.}
\end{itemize}
\begin{itemize}
\item {Proveniência:(De \textunderscore circumdar\textunderscore )}
\end{itemize}
Circuito; barreira.
Divisa.
\section{Circundante}
\begin{itemize}
\item {Grp. gram.:adj.}
\end{itemize}
Que circunda.
\section{Circundar}
\begin{itemize}
\item {Grp. gram.:v. t.}
\end{itemize}
\begin{itemize}
\item {Proveniência:(Lat. \textunderscore circumdare\textunderscore )}
\end{itemize}
Rodear; andar á volta de.
\section{Circundução}
\begin{itemize}
\item {Grp. gram.:f.}
\end{itemize}
\begin{itemize}
\item {Proveniência:(Lat. \textunderscore circumductio\textunderscore )}
\end{itemize}
Rotação em volta de um centro ou eixo.
\section{Circundutar}
\begin{itemize}
\item {Grp. gram.:v. t.}
\end{itemize}
\begin{itemize}
\item {Proveniência:(De \textunderscore circumducto\textunderscore )}
\end{itemize}
Julgar nulo.
\section{Circunferência}
\begin{itemize}
\item {Grp. gram.:f.}
\end{itemize}
\begin{itemize}
\item {Proveniência:(Lat. \textunderscore circumferentia\textunderscore )}
\end{itemize}
Linha que fêcha um circulo; periferia.
Circuito, linha que fêcha qualquer área.
\section{Circunferente}
\begin{itemize}
\item {Grp. gram.:adj.}
\end{itemize}
\begin{itemize}
\item {Proveniência:(Lat. \textunderscore circumferens\textunderscore )}
\end{itemize}
Que gira, que anda á volta.
\section{Circunflexamente}
\begin{itemize}
\item {Grp. gram.:adv.}
\end{itemize}
Por meio do assento circunflexo.
\section{Circunflexão}
\begin{itemize}
\item {Grp. gram.:f.}
\end{itemize}
\begin{itemize}
\item {Proveniência:(Lat. \textunderscore circumflexio\textunderscore )}
\end{itemize}
Acto de dobrar em roda.
\section{Circunflexo}
\begin{itemize}
\item {Grp. gram.:adj.}
\end{itemize}
\begin{itemize}
\item {Proveniência:(Lat. \textunderscore circumflexus\textunderscore )}
\end{itemize}
Recurvado em roda.
\textunderscore Accento circunflexo\textunderscore , sinal que, entre nós, dá ás vogaes \textunderscore e\textunderscore  e \textunderscore o\textunderscore  um som médio entre o agudo e o tênue e tira á vogal \textunderscore a\textunderscore  o som agudo; isto é, dá ao \textunderscore a\textunderscore , ao \textunderscore e\textunderscore , e ao \textunderscore o\textunderscore  o valor de vogaes fechadas.
\section{Circunfluência}
\begin{itemize}
\item {Grp. gram.:f.}
\end{itemize}
\begin{itemize}
\item {Proveniência:(Do lat. \textunderscore circumfluens\textunderscore )}
\end{itemize}
Movimento circular de um líquido ou de um fluido.
\section{Circunfluente}
\begin{itemize}
\item {Grp. gram.:adj.}
\end{itemize}
\begin{itemize}
\item {Proveniência:(Lat. \textunderscore circumfluens\textunderscore )}
\end{itemize}
Que corre em volta.
\section{Circunfluir}
\begin{itemize}
\item {Grp. gram.:v. t.}
\end{itemize}
\begin{itemize}
\item {Proveniência:(Lat. \textunderscore circumfluere\textunderscore )}
\end{itemize}
Fluir em roda.
\section{Circunforâneo}
\begin{itemize}
\item {Grp. gram.:adj.}
\end{itemize}
\begin{itemize}
\item {Utilização:Des.}
\end{itemize}
\begin{itemize}
\item {Proveniência:(Lat. \textunderscore circumforâneus\textunderscore )}
\end{itemize}
Que anda pelas praças ou á roda delas; ambulante.
Próprio de charlatão de praça.
\section{Circunfundir}
\begin{itemize}
\item {Grp. gram.:v. t.}
\end{itemize}
\begin{itemize}
\item {Proveniência:(Lat. \textunderscore circumfundere\textunderscore )}
\end{itemize}
Espalhar em volta.
Entornar, derramar em volta.
\section{Circunfusão}
\begin{itemize}
\item {Grp. gram.:f.}
\end{itemize}
Acto de circunfundir.
\section{Circungirar}
\begin{itemize}
\item {Grp. gram.:v. i.}
\end{itemize}
\begin{itemize}
\item {Proveniência:(De \textunderscore circum...\textunderscore  + \textunderscore girar\textunderscore )}
\end{itemize}
Girar em volta.
\section{Circunjacente}
\begin{itemize}
\item {Grp. gram.:adj.}
\end{itemize}
\begin{itemize}
\item {Proveniência:(Lat. \textunderscore circumjacens\textunderscore )}
\end{itemize}
Que está situado em roda; circunvizinho.
\section{Circunjazer}
\begin{itemize}
\item {Grp. gram.:v. i.}
\end{itemize}
Está em volta; sêr circunvizinho.
\section{Circunlocução}
\begin{itemize}
\item {Grp. gram.:f.}
\end{itemize}
O mesmo que \textunderscore circumlóquio\textunderscore .
\section{Circunlóquio}
\begin{itemize}
\item {Grp. gram.:m.}
\end{itemize}
\begin{itemize}
\item {Proveniência:(Do lat. \textunderscore circumloqui\textunderscore )}
\end{itemize}
Rodeio de palávras; perífrase.
\section{Circunpatente}
Aberto em roda; patente por todos os lados:«\textunderscore região circunpatente\textunderscore ». Castilho, \textunderscore Metam.\textunderscore , 282.
\section{Circunpolar}
\begin{itemize}
\item {Grp. gram.:adj.}
\end{itemize}
\begin{itemize}
\item {Proveniência:(De \textunderscore circum...\textunderscore  + \textunderscore polar\textunderscore )}
\end{itemize}
Que está perto do polo, em volta do polo.
\section{Circunrevoluto}
\begin{itemize}
\item {Grp. gram.:adj.}
\end{itemize}
Enrolado em volta de alguma coisa:«\textunderscore o lenço circunrevoluto á feição de turbante\textunderscore ». Castilho, \textunderscore Mil e Um Myst.\textunderscore 
\section{Circunscrever}
\begin{itemize}
\item {Grp. gram.:v. t.}
\end{itemize}
\begin{itemize}
\item {Proveniência:(Lat. \textunderscore circumscribere\textunderscore )}
\end{itemize}
Limitar com uma linha ou com um círculo.
Limitar.
Traçar á roda.
Abranger.
\section{Circunscrição}
\begin{itemize}
\item {Grp. gram.:f.}
\end{itemize}
\begin{itemize}
\item {Proveniência:(Lat. \textunderscore circumscriptio\textunderscore )}
\end{itemize}
Acto de circunscrever.
Linha, que limita de todos os lados uma área.
Divisão territorial.
\section{Circunscritivo}
\begin{itemize}
\item {Grp. gram.:adj.}
\end{itemize}
Que circunscreve ou limita.
\section{Circunscrito}
\begin{itemize}
\item {Grp. gram.:adj.}
\end{itemize}
Limitado, de todos os lados, por uma linha.
\section{Circunsoante}
\begin{itemize}
\item {Grp. gram.:adj.}
\end{itemize}
\begin{itemize}
\item {Proveniência:(Lat. \textunderscore circumsonans\textunderscore )}
\end{itemize}
Que sôa em roda.
\section{Circunsonante}
\begin{itemize}
\item {Grp. gram.:adj.}
\end{itemize}
\begin{itemize}
\item {Proveniência:(Lat. \textunderscore circumsonans\textunderscore )}
\end{itemize}
Que sôa em roda.
\section{Circunspecção}
\begin{itemize}
\item {Grp. gram.:f.}
\end{itemize}
\begin{itemize}
\item {Proveniência:(Lat. \textunderscore circumspectio\textunderscore )}
\end{itemize}
Qualidade de quem é circunspecto.
\section{Circunspectamente}
\begin{itemize}
\item {Grp. gram.:adv.}
\end{itemize}
\begin{itemize}
\item {Proveniência:(De \textunderscore circumspecto\textunderscore )}
\end{itemize}
Com circunspecção.
\section{Circunspecto}
\begin{itemize}
\item {Grp. gram.:adj.}
\end{itemize}
\begin{itemize}
\item {Proveniência:(Lat. \textunderscore circumspectus\textunderscore )}
\end{itemize}
Que olha á volta de si.
Cauteloso; prudente.
Em que há circunspecção: \textunderscore procedimento circunspecto\textunderscore .
\section{Circunstância}
\begin{itemize}
\item {Grp. gram.:f.}
\end{itemize}
\begin{itemize}
\item {Proveniência:(Lat. \textunderscore circumstantia\textunderscore )}
\end{itemize}
Estado; particularidade que acompanha um facto.
Motivo ou facto, que, acompanhando, seguindo ou precedendo outro facto, o agrava ou atenua: \textunderscore circunstâncias atenuantes\textunderscore .
\section{Circunstanciadamente}
\begin{itemize}
\item {Grp. gram.:adv.}
\end{itemize}
\begin{itemize}
\item {Proveniência:(De \textunderscore circumstanciado\textunderscore )}
\end{itemize}
Pormenorizadamente.
\section{Circunstanciado}
\begin{itemize}
\item {Grp. gram.:adj.}
\end{itemize}
Exposto minuciosamente.
\section{Circunstanciador}
\begin{itemize}
\item {Grp. gram.:adj.}
\end{itemize}
Que circunstancía.
\section{Circunstancial}
\begin{itemize}
\item {Grp. gram.:adj.}
\end{itemize}
Que exprime uma circunstância.
\section{Circunstanciar}
\begin{itemize}
\item {Grp. gram.:v. t.}
\end{itemize}
\begin{itemize}
\item {Proveniência:(De \textunderscore circumstancia\textunderscore )}
\end{itemize}
Descrever minuciosamente, com todas as circunstâncias.
\section{Circunstante}
\begin{itemize}
\item {Grp. gram.:adj.}
\end{itemize}
\begin{itemize}
\item {Grp. gram.:M.  e  f.}
\end{itemize}
\begin{itemize}
\item {Grp. gram.:Pl.}
\end{itemize}
\begin{itemize}
\item {Proveniência:(De \textunderscore circunstar\textunderscore )}
\end{itemize}
Que está á volta, circunjacente.
Pessôa que está presente.
Auditório.
\section{Circunstar}
\begin{itemize}
\item {Grp. gram.:v. t.  e  i.}
\end{itemize}
\begin{itemize}
\item {Proveniência:(Do lat. \textunderscore circum\textunderscore  + \textunderscore stare\textunderscore )}
\end{itemize}
Estar em roda; estar perto, á vista.
\section{Circunstoso}
\begin{itemize}
\item {Grp. gram.:adj.}
\end{itemize}
Difícil.
Pouco provável. (Colhido em Turquel)
(Cp. \textunderscore circumstar\textunderscore )
\section{Circunvagante}
\begin{itemize}
\item {Grp. gram.:adj.}
\end{itemize}
Que circunvaga.
\section{Circunvagar}
\begin{itemize}
\item {Grp. gram.:v. i.}
\end{itemize}
\begin{itemize}
\item {Proveniência:(Lat. \textunderscore circumvagari\textunderscore )}
\end{itemize}
Andar em volta; divagar.
\section{Circúnvago}
\begin{itemize}
\item {Grp. gram.:adj.}
\end{itemize}
O mesmo que \textunderscore circunvagante\textunderscore .
\section{Circunvalação}
\begin{itemize}
\item {Grp. gram.:f.}
\end{itemize}
\begin{itemize}
\item {Proveniência:(De \textunderscore circumvallar\textunderscore )}
\end{itemize}
Fôsso, vala com parapeito, que corta as comunicações de uma praça com o exterior.
Barreiras, em volta de uma povoação.
\section{Circunvalar}
\begin{itemize}
\item {Grp. gram.:v. t.}
\end{itemize}
\begin{itemize}
\item {Proveniência:(Lat. \textunderscore circumvallare\textunderscore )}
\end{itemize}
Cingir de fossos ou barreiras.
\section{Circunvizinhança}
\begin{itemize}
\item {Grp. gram.:f.}
\end{itemize}
\begin{itemize}
\item {Proveniência:(De \textunderscore circum...\textunderscore  + \textunderscore vizinhança\textunderscore )}
\end{itemize}
Subúrbio.
População vizinha; arredores.
\section{Circunvizinhar}
\begin{itemize}
\item {Grp. gram.:v. i.}
\end{itemize}
\begin{itemize}
\item {Proveniência:(De \textunderscore circumvizinho\textunderscore )}
\end{itemize}
Estar na vizinhança.
\section{Circunvizinho}
\begin{itemize}
\item {Grp. gram.:adj.}
\end{itemize}
\begin{itemize}
\item {Proveniência:(De \textunderscore circum...\textunderscore  + \textunderscore vizinho\textunderscore )}
\end{itemize}
Que está próximo ou em volta; confinante.
\section{Circunvolução}
\begin{itemize}
\item {Grp. gram.:f.}
\end{itemize}
\begin{itemize}
\item {Proveniência:(Do lat. \textunderscore circumvolutus\textunderscore )}
\end{itemize}
Movimento em volta de um centro.
Contôrno, sinuosidade.
\section{Circunvolucionário}
\begin{itemize}
\item {Grp. gram.:adj.}
\end{itemize}
\begin{itemize}
\item {Proveniência:(De \textunderscore circumvolução\textunderscore )}
\end{itemize}
Relativo ás circunvoluções do cérebro.
\section{Circunvolver}
\begin{itemize}
\item {Grp. gram.:v. t.}
\end{itemize}
\begin{itemize}
\item {Proveniência:(Lat. \textunderscore circumvolvere\textunderscore )}
\end{itemize}
Volvêr em roda.
\section{Círia}
\begin{itemize}
\item {Grp. gram.:f.}
\end{itemize}
\begin{itemize}
\item {Utilização:Prov.}
\end{itemize}
Fôrça muscular.
\section{Cirial}
\begin{itemize}
\item {Grp. gram.:m.}
\end{itemize}
\begin{itemize}
\item {Proveniência:(De \textunderscore círio\textunderscore )}
\end{itemize}
Castiçal comprido, que termina superiormente em lanterna e se leva com vela accesa, ao lado da cruz alçada.
\section{Cirieiro}
\begin{itemize}
\item {Grp. gram.:m.}
\end{itemize}
\begin{itemize}
\item {Proveniência:(De \textunderscore círio\textunderscore )}
\end{itemize}
Vendedor ou fabricante de círios ou velas.
Tem-se escrito \textunderscore cerieiro\textunderscore , por infl. de \textunderscore cera\textunderscore , mas é fórma errónea.
\section{Cirigo}
\begin{itemize}
\item {Grp. gram.:m.}
\end{itemize}
\begin{itemize}
\item {Utilização:T. de Aveiro}
\end{itemize}
Espécie de alga, (\textunderscore zostera marina angustifolia\textunderscore , Lin.).
\section{Círio}
\begin{itemize}
\item {Grp. gram.:m.}
\end{itemize}
\begin{itemize}
\item {Utilização:Bot.}
\end{itemize}
\begin{itemize}
\item {Proveniência:(Do lat. \textunderscore cereus\textunderscore )}
\end{itemize}
Vela grande de cera.
Procissão que, partindo de uma localidade, vai levar a outra um círio.
Espécie de cacto.
\section{Círio-do-rei}
\begin{itemize}
\item {Grp. gram.:m.}
\end{itemize}
\begin{itemize}
\item {Utilização:Bras}
\end{itemize}
O mesmo que \textunderscore verbasco\textunderscore .
\section{Cirita}
\begin{itemize}
\item {Grp. gram.:m.}
\end{itemize}
\begin{itemize}
\item {Utilização:Ant.}
\end{itemize}
\begin{itemize}
\item {Proveniência:(De \textunderscore cira\textunderscore )}
\end{itemize}
Habitador de charnecas ou brenhas.
\section{Cirolano}
\begin{itemize}
\item {Grp. gram.:m.}
\end{itemize}
Crustáceo isopode, nadador.
\section{Cirrar}
\begin{itemize}
\item {Grp. gram.:v. i.}
\end{itemize}
Pescar a corripo.
\section{Cirrhito}
\begin{itemize}
\item {Grp. gram.:m.}
\end{itemize}
\begin{itemize}
\item {Proveniência:(Do gr. \textunderscore kirrhis\textunderscore )}
\end{itemize}
Peixe, de côres vivas, no mar das Índias, e semelhante á perca.
\section{Cirrífero}
\begin{itemize}
\item {Grp. gram.:adj.}
\end{itemize}
\begin{itemize}
\item {Utilização:Zool.}
\end{itemize}
Que tem cirros.
\section{Cirriforme}
\begin{itemize}
\item {Grp. gram.:adj.}
\end{itemize}
\begin{itemize}
\item {Proveniência:(Do lat. \textunderscore cirrus\textunderscore  + \textunderscore forma\textunderscore )}
\end{itemize}
Que tem fórma de verruma.
\section{Cirrípedes}
\begin{itemize}
\item {Grp. gram.:m. pl.}
\end{itemize}
\begin{itemize}
\item {Proveniência:(Do lat. \textunderscore cirrus\textunderscore  + \textunderscore pes\textunderscore )}
\end{itemize}
Classe de animaes articulados, cujos pés são uns appêndices chamados cirros.
\section{Cirrito}
\begin{itemize}
\item {Grp. gram.:m.}
\end{itemize}
\begin{itemize}
\item {Proveniência:(Do gr. \textunderscore kirrhis\textunderscore )}
\end{itemize}
Peixe, de côres vivas, no mar das Índias, e semelhante á perca.
\section{Cirro}
\begin{itemize}
\item {Grp. gram.:m.}
\end{itemize}
Appêndice filiforme de algumas plantas.
Gavinha.
Pennas em tôrno das ventas de algumas aves.
Tentáculos labiaes de alguns peixes.
Appêndice de alguns anélidos e de outras famílias de animaes.
Nuvem branca e muito alta, que parece formada de filamentos cruzados. (Lat. \textunderscore cirrus\textunderscore )
\section{Cirro}
\begin{itemize}
\item {Grp. gram.:m.}
\end{itemize}
\begin{itemize}
\item {Proveniência:(Do lat. \textunderscore scirrhos\textunderscore )}
\end{itemize}
Tumor canceroso ou que se torna canceroso.
\section{Cirrópodes}
\begin{itemize}
\item {Grp. gram.:m. pl.}
\end{itemize}
O mesmo que \textunderscore cirrípedes\textunderscore .
\section{Cirrosidade}
\begin{itemize}
\item {Grp. gram.:f.}
\end{itemize}
Tumor cirroso.
Qualidade do que é cirroso.
\section{Cirroso}
\begin{itemize}
\item {Grp. gram.:adj.}
\end{itemize}
\begin{itemize}
\item {Proveniência:(De \textunderscore cirro\textunderscore ^2)}
\end{itemize}
Semelhante ao cirro; que tem a natureza do cirro.
\section{Cirroso}
\begin{itemize}
\item {Grp. gram.:adj.}
\end{itemize}
\begin{itemize}
\item {Utilização:Hist. Nat.}
\end{itemize}
\begin{itemize}
\item {Proveniência:(De \textunderscore cirro\textunderscore ^1)}
\end{itemize}
Que tem appêndices ou gavinhas, chamadas cirros.
\section{Círsio}
\begin{itemize}
\item {Grp. gram.:m.}
\end{itemize}
\begin{itemize}
\item {Proveniência:(Do gr. \textunderscore kirsos\textunderscore )}
\end{itemize}
Gênero de plantas compostas.
\section{Cirsocele}
\begin{itemize}
\item {Grp. gram.:m.  e  f.}
\end{itemize}
\begin{itemize}
\item {Proveniência:(Do gr. \textunderscore kirsos\textunderscore  + \textunderscore kele\textunderscore )}
\end{itemize}
Dilatação varicosa do escroto.
Dilatação varicosa das veias espermáticas.
\section{Cirsoftalmia}
\begin{itemize}
\item {Grp. gram.:f.}
\end{itemize}
\begin{itemize}
\item {Proveniência:(Do gr. \textunderscore kirsos\textunderscore  + \textunderscore ophtalmos\textunderscore )}
\end{itemize}
Oftalmia varicosa.
\section{Cirsômphalo}
\begin{itemize}
\item {Grp. gram.:m.}
\end{itemize}
\begin{itemize}
\item {Proveniência:(Do gr. \textunderscore kirsos\textunderscore  + \textunderscore omphalos\textunderscore )}
\end{itemize}
Dilatação varicosa das veias do umbigo.
\section{Cirsônfalo}
\begin{itemize}
\item {Grp. gram.:m.}
\end{itemize}
\begin{itemize}
\item {Proveniência:(Do gr. \textunderscore kirsos\textunderscore  + \textunderscore omphalos\textunderscore )}
\end{itemize}
Dilatação varicosa das veias do umbigo.
\section{Cirsophtalmia}
\begin{itemize}
\item {Grp. gram.:f.}
\end{itemize}
\begin{itemize}
\item {Proveniência:(Do gr. \textunderscore kirsos\textunderscore  + \textunderscore ophtalmos\textunderscore )}
\end{itemize}
Ophtalmia varicosa.
\section{Cirurgia}
\begin{itemize}
\item {Grp. gram.:f.}
\end{itemize}
\begin{itemize}
\item {Proveniência:(Gr. \textunderscore kheirourgia\textunderscore )}
\end{itemize}
Parte da Medicina, que trata principalmente de lesões externas, e dos processos manuaes ou operações com que ellas se curam, bem como de operações que facilitam ou tornam possível o tratamento de lesões internas.
\section{Cirurgião}
\begin{itemize}
\item {Grp. gram.:m.}
\end{itemize}
Aquelle que professa a cirurgia.
\section{Cirúrgico}
\begin{itemize}
\item {Grp. gram.:adj.}
\end{itemize}
Relativo á cirurgia.
\section{Cirus}
\begin{itemize}
\item {Grp. gram.:m. pl.}
\end{itemize}
Indígenas brasileiros, das regiões do Amazonas.
\section{Cirzeta}
\begin{itemize}
\item {fónica:zê}
\end{itemize}
\begin{itemize}
\item {Grp. gram.:f.}
\end{itemize}
Ave palmípede, semelhante ao pato.
\section{Cis...}
\begin{itemize}
\item {Grp. gram.:pref.}
\end{itemize}
(sign. de \textunderscore cá\textunderscore , \textunderscore aquém\textunderscore , como em cis\textunderscore alpino\textunderscore , cis\textunderscore montano\textunderscore , etc.)
\section{Cisalhas}
\begin{itemize}
\item {Grp. gram.:f. pl.}
\end{itemize}
\begin{itemize}
\item {Proveniência:(Do rad. do lat. \textunderscore accisus\textunderscore )}
\end{itemize}
Aparas, pequenos fragmentos de metal.
\section{Cisalpino}
\begin{itemize}
\item {Grp. gram.:adj.}
\end{itemize}
\begin{itemize}
\item {Proveniência:(De \textunderscore cis...\textunderscore  + \textunderscore alpino\textunderscore )}
\end{itemize}
Que está da banda de cá dos Alpes, (em relação a Roma).
\section{Cisandino}
\begin{itemize}
\item {Grp. gram.:adj.}
\end{itemize}
\begin{itemize}
\item {Utilização:Bras}
\end{itemize}
De àquém dos Andes.
\section{Cisão}
\begin{itemize}
\item {Grp. gram.:m.}
\end{itemize}
\begin{itemize}
\item {Proveniência:(Lat. \textunderscore scissío\textunderscore )}
\end{itemize}
Córte, numa parte insulada de um projecto architectónico.
\section{Cisatlântico}
\begin{itemize}
\item {Grp. gram.:adj.}
\end{itemize}
De aquém do Atlântico. Cf. Camillo, \textunderscore Cancion. Al.\textunderscore , 283; Garrett, \textunderscore Port. na Balança\textunderscore , 116.
\section{Cisca}
\begin{itemize}
\item {Grp. gram.:f.}
\end{itemize}
\begin{itemize}
\item {Utilização:Prov.}
\end{itemize}
\begin{itemize}
\item {Utilização:minh.}
\end{itemize}
Caruma sêca.
(Cp. \textunderscore cisco\textunderscore )
\section{Ciscada}
\begin{itemize}
\item {Grp. gram.:f.}
\end{itemize}
\begin{itemize}
\item {Utilização:Des.}
\end{itemize}
\begin{itemize}
\item {Proveniência:(De \textunderscore cisco\textunderscore )}
\end{itemize}
Porção de cisco ou detritos vegetaes, que as enchentes deixam nas margens dos rios.
Cascalho.
\section{Ciscalhagem}
\begin{itemize}
\item {Grp. gram.:f.}
\end{itemize}
Porção de ciscalho.
\section{Ciscalho}
\begin{itemize}
\item {Grp. gram.:m.}
\end{itemize}
Varredura.
Porção de cisco.
Miudezas de carvão.
\section{Ciscar}
\begin{itemize}
\item {Grp. gram.:v. t.}
\end{itemize}
\begin{itemize}
\item {Grp. gram.:V. i.}
\end{itemize}
\begin{itemize}
\item {Utilização:Prov.}
\end{itemize}
\begin{itemize}
\item {Utilização:beir.}
\end{itemize}
\begin{itemize}
\item {Utilização:Bras}
\end{itemize}
\begin{itemize}
\item {Grp. gram.:V. p.}
\end{itemize}
\begin{itemize}
\item {Proveniência:(De \textunderscore cisco\textunderscore )}
\end{itemize}
Tirar ciscos, gravetos, etc., a.
Limpar.
Defecar aos poucos, sujando aqui e alli, (falando-se de animaes ou de crianças).
Estorcer-se no chão, depois de batido ou agonizante.
Revolver o cisco.
Safar-se, escapulir-se.
\section{Cisco}
\begin{itemize}
\item {Grp. gram.:m.}
\end{itemize}
\begin{itemize}
\item {Utilização:Prov.}
\end{itemize}
\begin{itemize}
\item {Utilização:minh.}
\end{itemize}
\begin{itemize}
\item {Proveniência:(Do lat. \textunderscore cinisculus\textunderscore ?)}
\end{itemize}
Pó de carvão ou miudezas de carvão.
Lixo.
Ramos, gravetos, etc., arrastados pelas enxurradas ou pelas ondas.
Cisca.
Aparas miúdas.
\section{Cisdanubiano}
\begin{itemize}
\item {Grp. gram.:adj.}
\end{itemize}
De aquém do Danúbio. Cf. C. Lobo, \textunderscore Sát. de Jur.\textunderscore , I, 20.
\section{Cisel}
\begin{itemize}
\item {Grp. gram.:m.}
\end{itemize}
O mesmo que \textunderscore cintel\textunderscore .
\section{Cisgangético}
\begin{itemize}
\item {Grp. gram.:adj.}
\end{itemize}
De aquém do Ganges.
\section{Cisgola}
\begin{itemize}
\item {Grp. gram.:f.}
\end{itemize}
Correia, que faz parte da cabeçada e se prende com fivela àquém da gorja, em relação ao cavaleiro.
(Cp. fr. \textunderscore cisgorge\textunderscore )
\section{Cisgolla}
\begin{itemize}
\item {Grp. gram.:f.}
\end{itemize}
Correia, que faz parte da cabeçada e se prende com fivela àquém da gorja, em relação ao cavalleiro.
(Cp. fr. \textunderscore cisgorge\textunderscore )
\section{Císio}
\begin{itemize}
\item {Grp. gram.:m.}
\end{itemize}
\begin{itemize}
\item {Proveniência:(Lat. \textunderscore cisium\textunderscore )}
\end{itemize}
Carruagem de duas rodas, entre os antigos Romanos.
\section{Cisjurano}
\begin{itemize}
\item {Grp. gram.:adj.}
\end{itemize}
\begin{itemize}
\item {Proveniência:(De \textunderscore cis...\textunderscore  + \textunderscore Jura\textunderscore , n. p.)}
\end{itemize}
Que está aquém do Jura.
\section{Cisma}
\begin{itemize}
\item {Grp. gram.:f.}
\end{itemize}
\textunderscore f.\textunderscore  (e der.)
O mesmo ou melhor que \textunderscore scisma\textunderscore , etc.
Acto de cismar.
Mania.
Devaneio.
\section{Cismontano}
\begin{itemize}
\item {Grp. gram.:adj.}
\end{itemize}
\begin{itemize}
\item {Proveniência:(De \textunderscore cis...\textunderscore  + \textunderscore monte\textunderscore )}
\end{itemize}
Situado aquém dos montes.
Relativo á região que fica do lado de cá dos montes.
Que não é ultramontano.
\section{Cisoiro}
\begin{itemize}
\item {Grp. gram.:m.}
\end{itemize}
Tira larga de coiro, numa das extremidades do pírtigo.
\section{Cisouro}
\begin{itemize}
\item {Grp. gram.:m.}
\end{itemize}
Tira larga de coiro, numa das extremidades do pírtigo.
\section{Cispadano}
\begin{itemize}
\item {Grp. gram.:m.}
\end{itemize}
\begin{itemize}
\item {Proveniência:(Do lat. \textunderscore cis\textunderscore  + \textunderscore Padus\textunderscore , n. p.)}
\end{itemize}
Situado aquém do rio Pó.
\section{Cispar}
\begin{itemize}
\item {Grp. gram.:v. t.}
\end{itemize}
\begin{itemize}
\item {Utilização:Prov.}
\end{itemize}
\begin{itemize}
\item {Utilização:beir.}
\end{itemize}
Fechar bem, fechar hermeticamente (porta, janela, etc.).
\section{Cispelho}
\begin{itemize}
\item {fónica:pê}
\end{itemize}
\begin{itemize}
\item {Grp. gram.:m.}
\end{itemize}
\begin{itemize}
\item {Utilização:Prov.}
\end{itemize}
\begin{itemize}
\item {Utilização:trasm.}
\end{itemize}
Homem fraco.
\section{Cisque!}
\begin{itemize}
\item {Grp. gram.:interj.}
\end{itemize}
\begin{itemize}
\item {Utilização:Prov.}
\end{itemize}
\begin{itemize}
\item {Proveniência:(De \textunderscore ciscar\textunderscore )}
\end{itemize}
Fóra de aqui! ala! gire!
\section{Cisqueiro}
\begin{itemize}
\item {Grp. gram.:m.}
\end{itemize}
\begin{itemize}
\item {Utilização:Bras}
\end{itemize}
Lugar, onde se junta cisco.
Ciscalhagem.
\section{Cisrenano}
\begin{itemize}
\item {Grp. gram.:adj.}
\end{itemize}
\begin{itemize}
\item {Proveniência:(De \textunderscore cis...\textunderscore  + \textunderscore Rheno\textunderscore , n. p.)}
\end{itemize}
Que fica do lado de cá do Reno.
\section{Cisrhenano}
\begin{itemize}
\item {Grp. gram.:adj.}
\end{itemize}
\begin{itemize}
\item {Proveniência:(De \textunderscore cis...\textunderscore  + \textunderscore Rheno\textunderscore , n. p.)}
\end{itemize}
Que fica do lado de cá do Rheno.
\section{Cisso}
\begin{itemize}
\item {Grp. gram.:f.}
\end{itemize}
\begin{itemize}
\item {Proveniência:(Do gr. \textunderscore kissos\textunderscore )}
\end{itemize}
Um dos gêneros de videiras, em que se dividiu modernamente a fam. das ampelídeas.
\section{Cissoide}
\begin{itemize}
\item {Grp. gram.:adj.}
\end{itemize}
\begin{itemize}
\item {Proveniência:(Do gr. \textunderscore kissos\textunderscore  + \textunderscore eidos\textunderscore )}
\end{itemize}
Semelhante, na fórma, á fôlha da hera.
\section{Cista}
\begin{itemize}
\item {Grp. gram.:f.}
\end{itemize}
\begin{itemize}
\item {Proveniência:(Lat. \textunderscore cista\textunderscore )}
\end{itemize}
Caixinha, urna, cofre.
Cofre de bronze, usado pelos Etruscos em suas necrópoles.
Vaso funerário de pedra, em que os antigos guardavam cinzas humanas.
Cêsto, empregado pelos Gregos nas festas de Elêusis.
\section{Cistáceas}
\begin{itemize}
\item {Grp. gram.:f. pl.}
\end{itemize}
O mesmo ou melhor que \textunderscore cistíneas\textunderscore .
\section{Císteas}
\begin{itemize}
\item {Grp. gram.:f. pl.}
\end{itemize}
O mesmo que \textunderscore cistíneas\textunderscore .
\section{Cistela}
\begin{itemize}
\item {Grp. gram.:f.}
\end{itemize}
Gênero de insectos coleópteros.
\section{Cistercience}
\begin{itemize}
\item {Grp. gram.:adj.}
\end{itemize}
Relativo á Ordem de Cistér.
\section{Cisterna}
\begin{itemize}
\item {Grp. gram.:f.}
\end{itemize}
\begin{itemize}
\item {Proveniência:(Lat. \textunderscore cisterna\textunderscore )}
\end{itemize}
Poço, receptáculo, abaixo do nível da terra, para conservar águas pluviaes.
\section{Cistíneas}
\begin{itemize}
\item {Grp. gram.:f. pl.}
\end{itemize}
\begin{itemize}
\item {Proveniência:(De \textunderscore cistíneo\textunderscore )}
\end{itemize}
Plantas herbáceas ou arborescentes, que têm por typo a esteva.
\section{Cistíneo}
\begin{itemize}
\item {Grp. gram.:adj.}
\end{itemize}
\begin{itemize}
\item {Proveniência:(Do gr. \textunderscore kistos\textunderscore )}
\end{itemize}
Relativo ou semelhante ao cisto ou esteva.
\section{Cisto}
\begin{itemize}
\item {Grp. gram.:m.}
\end{itemize}
\begin{itemize}
\item {Proveniência:(Gr. \textunderscore kistos\textunderscore )}
\end{itemize}
Variedade de esteva.
\section{Cisto}
\begin{itemize}
\item {Grp. gram.:m.}
\end{itemize}
(V.cista)
\section{Cistófora}
\begin{itemize}
\item {Grp. gram.:f.}
\end{itemize}
Donzela, que levava corbelhas nas orgias ou festas de Baco.
(Cp. \textunderscore cistóphoro\textunderscore )
\section{Cistóforo}
\begin{itemize}
\item {Grp. gram.:m.}
\end{itemize}
\begin{itemize}
\item {Proveniência:(Gr. \textunderscore kistophoros\textunderscore )}
\end{itemize}
Medalha antiga, em que se vêem açafates.
\section{Cistóphora}
\begin{itemize}
\item {Grp. gram.:f.}
\end{itemize}
Donzella, que levava corbelhas nas orgias ou festas de Baccho.
(Cp. \textunderscore cistóphoro\textunderscore )
\section{Cistóphoro}
\begin{itemize}
\item {Grp. gram.:m.}
\end{itemize}
\begin{itemize}
\item {Proveniência:(Gr. \textunderscore kistophoros\textunderscore )}
\end{itemize}
Medalha antiga, em que se vêem açafates.
\section{Cístula}
\begin{itemize}
\item {Grp. gram.:f.}
\end{itemize}
\begin{itemize}
\item {Utilização:Bot.}
\end{itemize}
\begin{itemize}
\item {Proveniência:(Lat. \textunderscore cistula\textunderscore )}
\end{itemize}
Conceptáculo orbicular de certos líchens.
\section{Cisura}
\begin{itemize}
\item {Grp. gram.:m.}
\end{itemize}
(V.cesura)
\section{Cita}
\begin{itemize}
\item {Grp. gram.:f.}
\end{itemize}
\begin{itemize}
\item {Proveniência:(De \textunderscore citar\textunderscore )}
\end{itemize}
Referência a um trecho ou a uma opinião autorizada.
\section{Citação}
\begin{itemize}
\item {Grp. gram.:f.}
\end{itemize}
Acto ou effeito de citar.
Texto que se cita.
Intimação judicial ou em nome de qualquer autoridade.
\section{Cita-Christos}
\begin{itemize}
\item {Grp. gram.:m.}
\end{itemize}
\begin{itemize}
\item {Utilização:ant.}
\end{itemize}
\begin{itemize}
\item {Utilização:Pop.}
\end{itemize}
Esbirro; official de diligências; beleguim.
\section{Cita-cristos}
\begin{itemize}
\item {Grp. gram.:m.}
\end{itemize}
\begin{itemize}
\item {Utilização:ant.}
\end{itemize}
\begin{itemize}
\item {Utilização:Pop.}
\end{itemize}
Esbirro; oficial de diligências; beleguim.
\section{Citadino}
\begin{itemize}
\item {Grp. gram.:m.}
\end{itemize}
\begin{itemize}
\item {Proveniência:(It. \textunderscore cittadino\textunderscore )}
\end{itemize}
Habitante de cidade.
\section{Citador}
\begin{itemize}
\item {Grp. gram.:m.  e  adj.}
\end{itemize}
O que cita ou faz citações.
\section{Citamento}
\begin{itemize}
\item {Grp. gram.:m.}
\end{itemize}
\begin{itemize}
\item {Utilização:Ant.}
\end{itemize}
O mesmo que \textunderscore citação\textunderscore .
\section{Citamíneas}
\begin{itemize}
\item {Grp. gram.:f. pl.}
\end{itemize}
Ordem de plantas, a que pertence o gengivre, o açafrão, etc.
\section{Citânia}
\begin{itemize}
\item {Grp. gram.:f.}
\end{itemize}
Nome commum a várias povoações acastelladas, pre-romanas, da península hispânica.
(Talvez do lat. hyp. \textunderscore civitatania\textunderscore )
\section{Citaniense}
\begin{itemize}
\item {Grp. gram.:m.}
\end{itemize}
\begin{itemize}
\item {Grp. gram.:Adj.}
\end{itemize}
Habitante de alguma citânia.
Relativo a citânia.
\section{Citante}
\begin{itemize}
\item {Grp. gram.:m.  e  adj.}
\end{itemize}
\begin{itemize}
\item {Proveniência:(Lat. \textunderscore citans\textunderscore )}
\end{itemize}
O que cita.
\section{Citar}
\begin{itemize}
\item {Grp. gram.:v. t.}
\end{itemize}
\begin{itemize}
\item {Proveniência:(Lat. \textunderscore citare\textunderscore )}
\end{itemize}
Avisar, intimar, aprazar, para comparecer em juízo ou cumprir qualquer ordem judicial.
Mencionar ou transcrever como autoridade ou exemplo: \textunderscore citar Herculano\textunderscore .
Provocar (o toiro), para realizar qualquer sorte tauromáchica.
\section{Cítara}
\begin{itemize}
\item {Grp. gram.:f.}
\end{itemize}
\begin{itemize}
\item {Proveniência:(Gr. \textunderscore kithera\textunderscore )}
\end{itemize}
Antigo instrumento de cordas, semelhante á lira, mas maior.
Na Idade-Média, guitarra de quatro ou seis pares de cordas.
Modernamente, espécie de psaltério, com trinta a trinta e nove cordas.
\section{Citaredo}
\begin{itemize}
\item {fónica:tarê}
\end{itemize}
\begin{itemize}
\item {Grp. gram.:m.}
\end{itemize}
\begin{itemize}
\item {Utilização:Des.}
\end{itemize}
\begin{itemize}
\item {Proveniência:(Gr. \textunderscore kitharedos\textunderscore )}
\end{itemize}
Aquele que toca cítara.
\section{Citarina}
\begin{itemize}
\item {Grp. gram.:f.}
\end{itemize}
Medicamento contra a doença da gota.
\section{Citarista}
\begin{itemize}
\item {Grp. gram.:m.}
\end{itemize}
\begin{itemize}
\item {Proveniência:(Gr. \textunderscore kitharistes\textunderscore )}
\end{itemize}
O mesmo que \textunderscore citaredo\textunderscore .
\section{Citarística}
\begin{itemize}
\item {Grp. gram.:f.}
\end{itemize}
\begin{itemize}
\item {Proveniência:(De \textunderscore cítharista\textunderscore )}
\end{itemize}
Arte de tocar cítara.
\section{Citarizar}
\begin{itemize}
\item {Grp. gram.:v. i.}
\end{itemize}
Tocar cítara. Cf. Filinto, IX, 302.
\section{Citaródia}
\begin{itemize}
\item {Grp. gram.:f.}
\end{itemize}
Canto, acompanhado pela cítara, entre os antigos Gregos.
\section{Citatório}
\begin{itemize}
\item {Grp. gram.:adj.}
\end{itemize}
Relativo a citação.
\section{Citável}
\begin{itemize}
\item {Grp. gram.:adj.}
\end{itemize}
Que se póde citar.
Que é digno de sêr citado.
\section{Cite}
\begin{itemize}
\item {Grp. gram.:m.}
\end{itemize}
\begin{itemize}
\item {Utilização:Taur.}
\end{itemize}
\begin{itemize}
\item {Proveniência:(De \textunderscore citar\textunderscore )}
\end{itemize}
Acto de citar o toiro.
\section{Citerior}
\begin{itemize}
\item {Grp. gram.:adj.}
\end{itemize}
\begin{itemize}
\item {Proveniência:(Lat. \textunderscore citerior\textunderscore )}
\end{itemize}
Que está da nossa banda, do lado de cá.
\section{Cíthara}
\begin{itemize}
\item {Grp. gram.:f.}
\end{itemize}
\begin{itemize}
\item {Proveniência:(Gr. \textunderscore kithera\textunderscore )}
\end{itemize}
Antigo instrumento de cordas, semelhante á lyra, mas maior.
Na Idade-Média, guitarra de quatro ou seis pares de cordas.
Modernamente, espécie de psaltério, com trinta a trinta e nove cordas.
\section{Citharedo}
\begin{itemize}
\item {Grp. gram.:m.}
\end{itemize}
\begin{itemize}
\item {Utilização:Des.}
\end{itemize}
\begin{itemize}
\item {Proveniência:(Gr. \textunderscore kitharedos\textunderscore )}
\end{itemize}
Aquelle que toca cíthara.
\section{Citharista}
\begin{itemize}
\item {Grp. gram.:m.}
\end{itemize}
\begin{itemize}
\item {Proveniência:(Gr. \textunderscore kitharistes\textunderscore )}
\end{itemize}
O mesmo que \textunderscore citaredo\textunderscore .
\section{Citharística}
\begin{itemize}
\item {Grp. gram.:f.}
\end{itemize}
\begin{itemize}
\item {Proveniência:(De \textunderscore cítharista\textunderscore )}
\end{itemize}
Arte de tocar cíthara.
\section{Citharizar}
\begin{itemize}
\item {Grp. gram.:v. i.}
\end{itemize}
Tocar cíthara. Cf. Filinto, IX, 302.
\section{Citharódia}
\begin{itemize}
\item {Grp. gram.:f.}
\end{itemize}
Canto, acompanhado pela cíthara, entre os antigos Gregos.
\section{Citígrado}
\begin{itemize}
\item {Grp. gram.:adj.}
\end{itemize}
\begin{itemize}
\item {Proveniência:(Do lat. \textunderscore citus\textunderscore  + \textunderscore gradi\textunderscore )}
\end{itemize}
Que anda depressa.
\section{Citila}
\begin{itemize}
\item {Grp. gram.:f.}
\end{itemize}
Animal mamífero do norte da Europa.
\section{Citilla}
\begin{itemize}
\item {Grp. gram.:f.}
\end{itemize}
Animal mammífero do norte da Europa.
\section{Citima}
\begin{itemize}
\item {Grp. gram.:f.}
\end{itemize}
Casta de uva preta algarvia.
\section{Cito}
\begin{itemize}
\item {Grp. gram.:m.}
\end{itemize}
\begin{itemize}
\item {Utilização:Prov.}
\end{itemize}
\begin{itemize}
\item {Utilização:beir.}
\end{itemize}
Jôgo de crianças, também conhecido por \textunderscore escondidas\textunderscore .
\section{Citocácio}
\begin{itemize}
\item {Grp. gram.:m.}
\end{itemize}
\begin{itemize}
\item {Proveniência:(Lat. \textunderscore citocacium\textunderscore )}
\end{itemize}
Nome de uma planta mal determinada.
\section{Cito-cito}
\begin{itemize}
\item {Grp. gram.:m.}
\end{itemize}
\begin{itemize}
\item {Utilização:T. da Bairrada}
\end{itemize}
O mesmo que \textunderscore citote\textunderscore .
\section{Cítola}
\begin{itemize}
\item {Grp. gram.:f.}
\end{itemize}
O mesmo que \textunderscore taramela\textunderscore  de moínho.
Nome antigo da cíthara.
\section{Citom}
\begin{itemize}
\item {Grp. gram.:m.}
\end{itemize}
Cidadão? No \textunderscore Auto do Dia de Juizo\textunderscore , cit. por Castilho, fala um villão:«\textunderscore --Que sei mais que um citom.\textunderscore »
\section{Citote}
\begin{itemize}
\item {Grp. gram.:m.}
\end{itemize}
\begin{itemize}
\item {Utilização:Pop.}
\end{itemize}
\begin{itemize}
\item {Proveniência:(De \textunderscore citar\textunderscore )}
\end{itemize}
Official de diligências, ou qualquer empregado judicial, que faz citações.
O mesmo que citação. Cf. Filinto, III, 92; V, 107; VIII, 27.
\section{Citra}
\begin{itemize}
\item {Grp. gram.:f.}
\end{itemize}
\begin{itemize}
\item {Utilização:Des.}
\end{itemize}
O mesmo que \textunderscore cíthara\textunderscore .
\section{Citráceas}
\begin{itemize}
\item {Grp. gram.:f.}
\end{itemize}
\begin{itemize}
\item {Proveniência:(De \textunderscore citráceo\textunderscore )}
\end{itemize}
Família de plantas, que têm por typo a cidra.
\section{Citráceo}
\begin{itemize}
\item {Grp. gram.:adj.}
\end{itemize}
\begin{itemize}
\item {Proveniência:(Do lat. \textunderscore citrus\textunderscore )}
\end{itemize}
Relativo ou semelhante á cidra.
\section{Citrária}
\begin{itemize}
\item {Grp. gram.:adj. f.}
\end{itemize}
\begin{itemize}
\item {Utilização:Ant.}
\end{itemize}
Dizia-se da arte de caçar com falcões, ou da caça de altanaria.
\section{Citrato}
\begin{itemize}
\item {Grp. gram.:m.}
\end{itemize}
\begin{itemize}
\item {Proveniência:(Do lat. \textunderscore citrus\textunderscore )}
\end{itemize}
Combinação do ácido cítrico com uma base.
\section{Cítreo}
\begin{itemize}
\item {Grp. gram.:adj.}
\end{itemize}
\begin{itemize}
\item {Proveniência:(Lat. \textunderscore citreus\textunderscore )}
\end{itemize}
Relativo á cidreira.
\section{Citreiro}
\begin{itemize}
\item {Grp. gram.:m.}
\end{itemize}
\begin{itemize}
\item {Utilização:Ant.}
\end{itemize}
Caçador perito em altanaria.
(Cp. \textunderscore citrária\textunderscore )
\section{Cítrico}
\begin{itemize}
\item {Grp. gram.:adj.}
\end{itemize}
\begin{itemize}
\item {Proveniência:(Do lat. \textunderscore citrus\textunderscore )}
\end{itemize}
Diz-se do ácido, que se extráe do limão e de outros frutos ácidos.
\section{Citrilo}
\begin{itemize}
\item {Grp. gram.:m.}
\end{itemize}
Uma das essências, que se extrahem da casca do limão.
\section{Citrina}
\begin{itemize}
\item {Grp. gram.:f.}
\end{itemize}
\begin{itemize}
\item {Proveniência:(De \textunderscore citrino\textunderscore )}
\end{itemize}
Pedra preciosa, amarela.
Essência do limão.
\section{Citrinela}
\begin{itemize}
\item {Grp. gram.:f.}
\end{itemize}
\begin{itemize}
\item {Proveniência:(De \textunderscore citrino\textunderscore )}
\end{itemize}
Gênero de aves canoras.
\section{Citrinino}
\begin{itemize}
\item {Grp. gram.:adj.}
\end{itemize}
Relativo á citrina.
\section{Citrino}
\begin{itemize}
\item {Grp. gram.:adj.}
\end{itemize}
\begin{itemize}
\item {Proveniência:(Lat. \textunderscore citrinus\textunderscore )}
\end{itemize}
Que tem côr de cidra.
\section{Citro}
\begin{itemize}
\item {Grp. gram.:m.}
\end{itemize}
\begin{itemize}
\item {Proveniência:(Lat. \textunderscore citrus\textunderscore )}
\end{itemize}
Certa árvore africana, ornamental.
O mesmo que limoeiro. Cf. C. Lobo. \textunderscore Sát. de Juv.\textunderscore , I, 211.
\section{Citrofeno}
\begin{itemize}
\item {Grp. gram.:m.}
\end{itemize}
Uma das essências que se extrahem da casca do limão.
\section{Citronato}
\begin{itemize}
\item {Grp. gram.:m.}
\end{itemize}
\begin{itemize}
\item {Proveniência:(Fr. \textunderscore citronnat\textunderscore )}
\end{itemize}
Conserva de cidrão.
\section{Citronela}
\begin{itemize}
\item {Grp. gram.:f.}
\end{itemize}
Nome scientífico da erva-cidreira e de algumas outras plantas, que têm aroma semelhante ao limão.
(Cp. fr. \textunderscore citron\textunderscore )
\section{Citropheno}
\begin{itemize}
\item {Grp. gram.:m.}
\end{itemize}
Uma das essências que se extrahem da casca do limão.
\section{Ciumar}
\begin{itemize}
\item {fónica:ci-u}
\end{itemize}
\begin{itemize}
\item {Grp. gram.:v. i.}
\end{itemize}
Têr ciumes, Cf. Camillo, \textunderscore Cancion. Al.\textunderscore , 443.
\section{Ciumaria}
\begin{itemize}
\item {fónica:ci-u}
\end{itemize}
\begin{itemize}
\item {Grp. gram.:f.}
\end{itemize}
Grande ciúme.
\section{Ciúme}
\begin{itemize}
\item {Grp. gram.:m.}
\end{itemize}
\begin{itemize}
\item {Proveniência:(De \textunderscore cio\textunderscore )}
\end{itemize}
Zelos amorosos.
Emulação; inveja.
\section{Ciumeira}
\begin{itemize}
\item {fónica:ci-u}
\end{itemize}
\begin{itemize}
\item {Grp. gram.:f.}
\end{itemize}
\begin{itemize}
\item {Utilização:Pop.}
\end{itemize}
Ciúme exaggerado.
\section{Ciumento}
\begin{itemize}
\item {fónica:ci-u}
\end{itemize}
\begin{itemize}
\item {Grp. gram.:adj.}
\end{itemize}
\begin{itemize}
\item {Grp. gram.:M.}
\end{itemize}
Que tem ciúmes.
Aquelle que tem ciúmes.
\section{Ciumoso}
\begin{itemize}
\item {Grp. gram.:adj.}
\end{itemize}
O mesmo que \textunderscore ciumento\textunderscore . Cf. Camillo, \textunderscore Myst. de Lisb.\textunderscore , II, 109.
\section{Cível}
\begin{itemize}
\item {Grp. gram.:adj.}
\end{itemize}
\begin{itemize}
\item {Grp. gram.:M.}
\end{itemize}
\begin{itemize}
\item {Proveniência:(Lat. \textunderscore civilis\textunderscore )}
\end{itemize}
Relativo ao Direito Civil.
Jurisdicção dos tribunaes civis.
\section{Civelmente}
\begin{itemize}
\item {Grp. gram.:adv.}
\end{itemize}
\begin{itemize}
\item {Proveniência:(De \textunderscore cível\textunderscore )}
\end{itemize}
Segundo a jurisdicção cível.
\section{Civeta}
\begin{itemize}
\item {fónica:vê}
\end{itemize}
\begin{itemize}
\item {Grp. gram.:f.}
\end{itemize}
\begin{itemize}
\item {Proveniência:(Fr. \textunderscore civette\textunderscore )}
\end{itemize}
Gênero de quadrúpedes carnívoros, cuja espécie principal é conhecida por \textunderscore gato-de-algália\textunderscore .
\section{Cívico}
\begin{itemize}
\item {Grp. gram.:adj.}
\end{itemize}
\begin{itemize}
\item {Grp. gram.:M.}
\end{itemize}
\begin{itemize}
\item {Utilização:Neol.}
\end{itemize}
\begin{itemize}
\item {Proveniência:(Lat. \textunderscore civicus\textunderscore )}
\end{itemize}
Relativo aos cidadãos, como membros do Estado.
Patriótico.
Guarda de polícia de segurança: \textunderscore o faquista foi preso pelo cívico n.^o 410\textunderscore .
\section{Cividade}
\begin{itemize}
\item {Grp. gram.:f.}
\end{itemize}
\begin{itemize}
\item {Utilização:Ant.}
\end{itemize}
\begin{itemize}
\item {Proveniência:(Lat. \textunderscore civitas\textunderscore )}
\end{itemize}
Cidade.
\section{Civil}
\begin{itemize}
\item {Grp. gram.:adj.}
\end{itemize}
\begin{itemize}
\item {Grp. gram.:M.}
\end{itemize}
\begin{itemize}
\item {Proveniência:(Lat. \textunderscore civilis\textunderscore )}
\end{itemize}
Que diz respeito ás relações dos cidadãos entre si.
Que não tem carácter militar nem ecclesiástico.
Civilizado.
Delicado, cortês.
Jurisdicção dos tribunaes civis.
\section{Civilidade}
\begin{itemize}
\item {Grp. gram.:f.}
\end{itemize}
\begin{itemize}
\item {Proveniência:(Lat. \textunderscore civilitas\textunderscore )}
\end{itemize}
Conjunto de formalidades, observadas pelos cidadãos entre si, quando bem educados.
Delicadeza; cortesia.
\section{Civilista}
\begin{itemize}
\item {Grp. gram.:m.}
\end{itemize}
Tratadista de direito civil. Cf. Herculano, \textunderscore Opúsc.\textunderscore , IV, 281.
\section{Civilização}
\begin{itemize}
\item {Grp. gram.:f.}
\end{itemize}
Estado de adeantamento e cultura social.
Acto de civilizar.
\section{Civilizado}
\begin{itemize}
\item {Grp. gram.:adj.}
\end{itemize}
Que tem civilização.
Bem educado; cortês.
\section{Civilizador}
\begin{itemize}
\item {Grp. gram.:adj.}
\end{itemize}
Que civiliza.
\section{Civilizante}
\begin{itemize}
\item {Grp. gram.:m.}
\end{itemize}
\begin{itemize}
\item {Utilização:Deprec.}
\end{itemize}
Aquelle que se inculca como civilizador:«\textunderscore uns civilizantes de má morte\textunderscore ». Camillo.
\section{Civilizar}
\begin{itemize}
\item {Grp. gram.:v. t.}
\end{itemize}
\begin{itemize}
\item {Proveniência:(De \textunderscore civil\textunderscore )}
\end{itemize}
Dar civilização a: \textunderscore civilizar um povo\textunderscore .
Tornar delicado, cortês, instruído.
\section{Civilizável}
\begin{itemize}
\item {Grp. gram.:adj.}
\end{itemize}
Que se póde civilizar.
\section{Civilmente}
\begin{itemize}
\item {Grp. gram.:adv.}
\end{itemize}
Segundo o Direito Civil.
Delicadamente, de modo civil.
\section{Civismo}
\begin{itemize}
\item {Grp. gram.:m.}
\end{itemize}
\begin{itemize}
\item {Proveniência:(Do lat. \textunderscore civis\textunderscore )}
\end{itemize}
Dedicação ao interesse público; patriotismo.
\section{Cizânia}
\begin{itemize}
\item {Grp. gram.:f.}
\end{itemize}
\begin{itemize}
\item {Utilização:Fig.}
\end{itemize}
\begin{itemize}
\item {Proveniência:(Do gr. \textunderscore zizanion\textunderscore )}
\end{itemize}
Joio.
Rixa, desharmonia: \textunderscore espalhar cizânia numa família\textunderscore .
\section{Cizicenses}
\begin{itemize}
\item {Grp. gram.:f. pl.}
\end{itemize}
\begin{itemize}
\item {Proveniência:(De \textunderscore Cizico\textunderscore , n. p.)}
\end{itemize}
Salas luxuosas, onde se banqueteavam os Gregos opulentos.
\section{Cizirão}
\begin{itemize}
\item {Grp. gram.:m.}
\end{itemize}
Planta leguminosa e trepadeira, (\textunderscore lathyrus latifolius\textunderscore ).
\section{Clá-clá}
\begin{itemize}
\item {Grp. gram.:m.}
\end{itemize}
Árvore medicinal da ilha de San-Thomé.
\section{Cladantho}
\begin{itemize}
\item {Grp. gram.:m.}
\end{itemize}
\begin{itemize}
\item {Proveniência:(Do gr. \textunderscore klados\textunderscore  + \textunderscore anthos\textunderscore )}
\end{itemize}
Gênero de plantas compostas.
\section{Cladanto}
\begin{itemize}
\item {Grp. gram.:m.}
\end{itemize}
\begin{itemize}
\item {Proveniência:(Do gr. \textunderscore klados\textunderscore  + \textunderscore anthos\textunderscore )}
\end{itemize}
Gênero de plantas compostas.
\section{Clade}
\begin{itemize}
\item {Grp. gram.:f.}
\end{itemize}
\begin{itemize}
\item {Proveniência:(Lat. \textunderscore clades\textunderscore )}
\end{itemize}
Morticínio.
\section{Cladode}
\begin{itemize}
\item {Grp. gram.:m.}
\end{itemize}
\begin{itemize}
\item {Utilização:Bot.}
\end{itemize}
(É fórma incorrecta, em vez de \textunderscore cladódio\textunderscore )
\section{Cladódio}
\begin{itemize}
\item {Grp. gram.:m.}
\end{itemize}
\begin{itemize}
\item {Utilização:Bot.}
\end{itemize}
\begin{itemize}
\item {Proveniência:(Lat. \textunderscore cladodium\textunderscore )}
\end{itemize}
Gomo, que se dilatou, apresentando o aspecto de fôlha.
\section{Cladótrico}
\begin{itemize}
\item {Grp. gram.:adj.}
\end{itemize}
\begin{itemize}
\item {Utilização:Med.}
\end{itemize}
\begin{itemize}
\item {Proveniência:(Do gr. \textunderscore klados\textunderscore  + \textunderscore trix\textunderscore )}
\end{itemize}
Gênero de arbustos leguminosos do Brasil.
Bactéria, cujos elementos têm a fórma de longos filamentos, mais ou menos ramificados.
\section{Clamador}
\begin{itemize}
\item {Grp. gram.:m.  e  adj.}
\end{itemize}
O que clama.
\section{Clamante}
\begin{itemize}
\item {Grp. gram.:adj.}
\end{itemize}
Que clama.
\section{Clamar}
\begin{itemize}
\item {Grp. gram.:v. t.}
\end{itemize}
\begin{itemize}
\item {Grp. gram.:V. i.}
\end{itemize}
\begin{itemize}
\item {Proveniência:(Lat. \textunderscore clamare\textunderscore )}
\end{itemize}
Proferir em altas vozes.
Pedir em voz alta: \textunderscore clamar soccorro\textunderscore .
Bradar, queixar-se em voz alta.
Gritar.
\section{Clâmide}
\begin{itemize}
\item {Grp. gram.:f.}
\end{itemize}
\begin{itemize}
\item {Proveniência:(Do gr. \textunderscore khlamus\textunderscore )}
\end{itemize}
Manto rico dos antigos, seguro por um broche ao pescoço ou sôbre o ombro direito.
\section{Clamíforo}
\begin{itemize}
\item {Grp. gram.:m.}
\end{itemize}
\begin{itemize}
\item {Proveniência:(Do gr. \textunderscore khlamus\textunderscore  + \textunderscore phoros\textunderscore )}
\end{itemize}
Gênero de fêtos.
\section{Clamor}
\begin{itemize}
\item {Grp. gram.:m.}
\end{itemize}
\begin{itemize}
\item {Utilização:Prov.}
\end{itemize}
\begin{itemize}
\item {Proveniência:(Lat. \textunderscore clamor\textunderscore )}
\end{itemize}
Acção de clamar.
Súpplica em voz alta.
Queixa.
Procissão de penitência.
\section{Clamorosamente}
\begin{itemize}
\item {Grp. gram.:adv.}
\end{itemize}
De modo clamoroso.
Com clamor.
\section{Clamoroso}
\begin{itemize}
\item {Grp. gram.:adj.}
\end{itemize}
Feito com clamor.
Em que há clamor: \textunderscore queixas clamorosas\textunderscore .
\section{Clan}
\begin{itemize}
\item {Grp. gram.:m.}
\end{itemize}
\begin{itemize}
\item {Proveniência:(Do ingl. \textunderscore clan\textunderscore )}
\end{itemize}
Tríbo, formada por certo número de famílias, regida por um chefe hereditário e por determinados costumes, entre os antigos Gállios, Escoceses e Irlandeses. Cf. Herculano, \textunderscore Bobo\textunderscore , 8.
\section{Clandestina}
\begin{itemize}
\item {Grp. gram.:f.}
\end{itemize}
\begin{itemize}
\item {Proveniência:(De \textunderscore clandestino\textunderscore )}
\end{itemize}
Planta herbácea, cujas flôres estão, em parte, encobertas pela terra.
\section{Clandestinamente}
\begin{itemize}
\item {Grp. gram.:adv.}
\end{itemize}
De modo clandestino.
Ás occultas.
\section{Clandestinidade}
\begin{itemize}
\item {Grp. gram.:f.}
\end{itemize}
\begin{itemize}
\item {Utilização:eccles.}
\end{itemize}
\begin{itemize}
\item {Utilização:Jur.}
\end{itemize}
Qualidade de clandestino, um dos quatorze impedimentos dirimentes do matrimónio.
\section{Clandestino}
\begin{itemize}
\item {Grp. gram.:adj.}
\end{itemize}
\begin{itemize}
\item {Proveniência:(Lat. \textunderscore clandestinus\textunderscore )}
\end{itemize}
Feito ás occultas: \textunderscore casamento clandestino\textunderscore .
\section{Clangor}
\begin{itemize}
\item {Grp. gram.:m.}
\end{itemize}
\begin{itemize}
\item {Proveniência:(Lat. \textunderscore clangor\textunderscore )}
\end{itemize}
Som de trombeta.
\section{Clangorar}
\begin{itemize}
\item {Grp. gram.:v. i.}
\end{itemize}
O mesmo que \textunderscore clangorejar\textunderscore . Cf. C. Neto, \textunderscore Saldunes\textunderscore .
\section{Clangorejar}
\begin{itemize}
\item {Grp. gram.:v. i.}
\end{itemize}
\begin{itemize}
\item {Utilização:Neol.}
\end{itemize}
Soltar clangor.
Apregoar um acontecimento.
\section{Clangoroso}
\begin{itemize}
\item {Grp. gram.:adj.}
\end{itemize}
\begin{itemize}
\item {Proveniência:(De \textunderscore clangor\textunderscore )}
\end{itemize}
Estridente como o som de trombeta.
\section{Claque}
\begin{itemize}
\item {Grp. gram.:f.}
\end{itemize}
\begin{itemize}
\item {Proveniência:(T. fr.)}
\end{itemize}
Chapéu de pasta, de molas.
Reunião de pessoas, conluiadas para applaudir ou patear.
\section{Clara}
\begin{itemize}
\item {Grp. gram.:f.}
\end{itemize}
\begin{itemize}
\item {Proveniência:(De \textunderscore claro\textunderscore )}
\end{itemize}
Albumina, que envolve a gema do ovo.
Esclerótica: \textunderscore a clara do ôlho\textunderscore .
Clareira.
Abertura em algumas peças do navio.
\section{Clarabela}
\begin{itemize}
\item {Grp. gram.:f.}
\end{itemize}
\begin{itemize}
\item {Utilização:Mús.}
\end{itemize}
Instrumento de manivela, recentemente inventado, cujos sons são produzidos por uma colecção de timbres, cujos martelos são impelidos por um cilindro, que a manivela faz girar.
\section{Clarabella}
\begin{itemize}
\item {Grp. gram.:f.}
\end{itemize}
\begin{itemize}
\item {Utilização:Mús.}
\end{itemize}
Instrumento de manivela, recentemente inventado, cujos sons são produzidos por uma collecção de timbres, cujos martelos são impellidos por um cylindro, que a manivela faz girar.
\section{Clarabóia}
\begin{itemize}
\item {Grp. gram.:f.}
\end{itemize}
\begin{itemize}
\item {Proveniência:(Fr. \textunderscore claire-voie\textunderscore )}
\end{itemize}
Abertura, geralmente envidraçada, no alto de um edifício.
Janela redonda, ou fresta, por onde entra luz numa casa.
Abertura perpendicular, por onde entra a luz numa mina.
\section{Claraíba}
\begin{itemize}
\item {Grp. gram.:f.}
\end{itemize}
Árvore silvestre do Brasil.
\section{Claramente}
\begin{itemize}
\item {Grp. gram.:adv.}
\end{itemize}
De modo claro, sem dúvida.
\section{Clarão}
\begin{itemize}
\item {Grp. gram.:m.}
\end{itemize}
\begin{itemize}
\item {Utilização:Fig.}
\end{itemize}
\begin{itemize}
\item {Proveniência:(De \textunderscore claro\textunderscore )}
\end{itemize}
Luz viva; claridade intensa.
Luz intellectual.
\section{Clarão}
\begin{itemize}
\item {Grp. gram.:m.}
\end{itemize}
\begin{itemize}
\item {Utilização:Mús.}
\end{itemize}
Registo de órgão, cujos tubos são de zinco, com bocal de palheta. (Fr. \textunderscore clairon\textunderscore )
\section{Clareação}
\begin{itemize}
\item {Grp. gram.:f.}
\end{itemize}
Acto ou effeito de clarear.
\section{Clarear}
\begin{itemize}
\item {Grp. gram.:v. t.}
\end{itemize}
\begin{itemize}
\item {Grp. gram.:V. i.}
\end{itemize}
\begin{itemize}
\item {Utilização:Fig.}
\end{itemize}
Tornar claro, aclarar.
Tornar raro, abrir espaços em: \textunderscore clarear um alfobre\textunderscore .
Fazer-se claro, aclarar-se: \textunderscore o dia clareou\textunderscore .
Tornar-se lúcido.
\section{Clareia}
\begin{itemize}
\item {Grp. gram.:f.}
\end{itemize}
\begin{itemize}
\item {Utilização:Ant.}
\end{itemize}
\begin{itemize}
\item {Proveniência:(De \textunderscore claro\textunderscore )}
\end{itemize}
Vinho branco com mel.
\section{Clareira}
\begin{itemize}
\item {Grp. gram.:f.}
\end{itemize}
\begin{itemize}
\item {Proveniência:(De \textunderscore claro\textunderscore )}
\end{itemize}
Espaço sem árvores, dentro de um bosque.
Terreno desmoitado ou arroteado, em meio de brenhas ou matas.
\section{Clarejar}
\begin{itemize}
\item {Grp. gram.:v. t.}
\end{itemize}
Tornar claro. Cf. Castilho, \textunderscore Fastos\textunderscore , II, 101.
\section{Clareta}
\begin{itemize}
\item {fónica:clarê}
\end{itemize}
\begin{itemize}
\item {Grp. gram.:f.}
\end{itemize}
\begin{itemize}
\item {Utilização:Ant.}
\end{itemize}
Espécie de licôr de canela. Cf. \textunderscore Âncora Méd.\textunderscore , 193.
\section{Clarete}
\begin{itemize}
\item {fónica:clarê}
\end{itemize}
\begin{itemize}
\item {Grp. gram.:m.}
\end{itemize}
\begin{itemize}
\item {Proveniência:(De \textunderscore claro\textunderscore )}
\end{itemize}
O mesmo que vinho palhete.
\section{Clareza}
\begin{itemize}
\item {Grp. gram.:f.}
\end{itemize}
\begin{itemize}
\item {Utilização:Fig.}
\end{itemize}
\begin{itemize}
\item {Proveniência:(De \textunderscore claro\textunderscore )}
\end{itemize}
Qualidade do que é claro.
Limpidez.
Qualidade da vista que distingue bem os objectos: \textunderscore vêr com clareza\textunderscore .
Bom timbre.
Qualidade do que se percebe bem: \textunderscore clareza do estilo\textunderscore .
Declaração escrita de um contrato ou de um encargo.
\section{Claridade}
\begin{itemize}
\item {Grp. gram.:f.}
\end{itemize}
\begin{itemize}
\item {Proveniência:(Lat. \textunderscore claritas\textunderscore )}
\end{itemize}
Qualidade do que é claro.
Luz viva: \textunderscore a claridade do Sol\textunderscore .
\section{Clarificação}
\begin{itemize}
\item {Grp. gram.:f.}
\end{itemize}
\begin{itemize}
\item {Proveniência:(Lat. \textunderscore clarificatio\textunderscore )}
\end{itemize}
Acto de clarificar.
\section{Clarificador}
\begin{itemize}
\item {Grp. gram.:m.  e  adj.}
\end{itemize}
O que clarifica.
\section{Clarificar}
\begin{itemize}
\item {Grp. gram.:v. t.}
\end{itemize}
\begin{itemize}
\item {Proveniência:(Lat. \textunderscore clarificare\textunderscore )}
\end{itemize}
Tornar claro.
Purificar (um líquido), fazendo precipitar as substâncias, que nelle há em suspensão: \textunderscore clarificar o vinho\textunderscore .
\section{Clarificativo}
\begin{itemize}
\item {Grp. gram.:adj.}
\end{itemize}
\begin{itemize}
\item {Proveniência:(De \textunderscore clarificar\textunderscore )}
\end{itemize}
Que clarifica.
\section{Clarim}
\begin{itemize}
\item {Grp. gram.:m.}
\end{itemize}
\begin{itemize}
\item {Proveniência:(De \textunderscore claro\textunderscore )}
\end{itemize}
Espécie de trombeta de som claro e estridente.
Aquelle que toca clarim.
Registo de órgãos, com que se imita o som do clarim.
\section{Clarina}
\begin{itemize}
\item {Grp. gram.:f.}
\end{itemize}
Espécie de clarinete de metal, para bandas militares, inventado em 1891.
(Cp. \textunderscore clarim\textunderscore )
\section{Clarineta}
\begin{itemize}
\item {fónica:nê}
\end{itemize}
\begin{itemize}
\item {Grp. gram.:f.}
\end{itemize}
\begin{itemize}
\item {Utilização:Ant.}
\end{itemize}
O mesmo que \textunderscore clarinete\textunderscore .
\section{Clarinete}
\begin{itemize}
\item {fónica:nê}
\end{itemize}
\begin{itemize}
\item {Grp. gram.:m.}
\end{itemize}
\begin{itemize}
\item {Proveniência:(De \textunderscore clarim\textunderscore )}
\end{itemize}
Instrumento de sopro, com bocal de palheta e orifícios como os da frauta.
Aquelle que toca clarinete.
Registo de órgão, com tubos de estanho e bocal de palheta.
\section{Clariofone}
\begin{itemize}
\item {Grp. gram.:m.}
\end{itemize}
Instrumento de manivela, espécie de realejo.
\section{Clariofónio}
\begin{itemize}
\item {Grp. gram.:m.}
\end{itemize}
Instrumento de manivela, espécie de realejo.
\section{Clariophone}
\begin{itemize}
\item {Grp. gram.:m.}
\end{itemize}
Instrumento de manivela, espécie de realejo.
\section{Clarirrubro}
\begin{itemize}
\item {Grp. gram.:adj.}
\end{itemize}
Vermelho-claro. Cf. Filinto, V, 194.
\section{Clarirubro}
\begin{itemize}
\item {fónica:ru}
\end{itemize}
\begin{itemize}
\item {Grp. gram.:adj.}
\end{itemize}
Vermelho-claro. Cf. Filinto, V, 194.
\section{Clarisono}
\begin{itemize}
\item {fónica:so}
\end{itemize}
\begin{itemize}
\item {Grp. gram.:adj.}
\end{itemize}
\begin{itemize}
\item {Utilização:Poét.}
\end{itemize}
\begin{itemize}
\item {Proveniência:(Lat. \textunderscore clarisonus\textunderscore )}
\end{itemize}
Que tem som claro.
\section{Clarissa}
\begin{itemize}
\item {Grp. gram.:f.}
\end{itemize}
O mesmo que \textunderscore clarista\textunderscore .
\section{Clarissono}
\begin{itemize}
\item {Grp. gram.:adj.}
\end{itemize}
\begin{itemize}
\item {Utilização:Poét.}
\end{itemize}
\begin{itemize}
\item {Proveniência:(Lat. \textunderscore clarisonus\textunderscore )}
\end{itemize}
Que tem som claro.
\section{Clarista}
\begin{itemize}
\item {Grp. gram.:m. ,  f.  e  adj.}
\end{itemize}
\begin{itemize}
\item {Proveniência:(De \textunderscore Clara\textunderscore , n. p.)}
\end{itemize}
Pessôa que pertence á Ordem religiosa de Santa-Clara.
\section{Clarividencia}
\begin{itemize}
\item {Grp. gram.:f.}
\end{itemize}
\begin{itemize}
\item {Utilização:Neol.}
\end{itemize}
Qualidade de clarividente.
\section{Clarividente}
\begin{itemize}
\item {Grp. gram.:adj.}
\end{itemize}
\begin{itemize}
\item {Utilização:Neol.}
\end{itemize}
\begin{itemize}
\item {Proveniência:(Do lat. \textunderscore clarus\textunderscore  + \textunderscore videns\textunderscore )}
\end{itemize}
Que vê bem, que vê com clareza.
Atilado, prudente.
\section{Claro}
\begin{itemize}
\item {Grp. gram.:adj.}
\end{itemize}
\begin{itemize}
\item {Utilização:Fig.}
\end{itemize}
\begin{itemize}
\item {Grp. gram.:M.}
\end{itemize}
\begin{itemize}
\item {Utilização:T. de Aveiro}
\end{itemize}
\begin{itemize}
\item {Grp. gram.:Loc. adv.}
\end{itemize}
\begin{itemize}
\item {Proveniência:(Lat. \textunderscore clarus\textunderscore )}
\end{itemize}
Que dá luz.
Brilhante.
Em que há luz.
Que não é escuro, (falando-se de côres).
Límpido; que reflecte bem os corpos, (falando-se de espelhos).
Que distingue bem, (falando-se da vista).
Que se distingue bem com a vista.
Quási branco: \textunderscore um vestido claro\textunderscore .
Vibrante, que tem bom timbre.
Facilmente intelligível.
Que comprehende bem, (falando-se da intelligencia).
Que se evidencia, que não offerece dúvidas; certo.
Illustre.
O mesmo que \textunderscore boiante\textunderscore , (falando-se do toiro).
Espaço em branco.
Clareira.
A parte mais clara de um objecto.
A quartela da rede dos calões, que fica mais distante do saco e tem a malha mais larga que as outras quartelas, (\textunderscore regalo\textunderscore , \textunderscore caçarote\textunderscore  e \textunderscore alcanela\textunderscore ).
\textunderscore Em claro\textunderscore , sem dormir.
Por cima, por alto; fazendo omissão.
\section{Claro-escuro}
\begin{itemize}
\item {Grp. gram.:m.}
\end{itemize}
Transição do claro para o escuro.
Impressão, que produz no observador o contraste dos claros com os escuros, em desenho, pintura ou gravura.
Combinação, misto, de sombras e luz.
\section{Clarom}
\begin{itemize}
\item {Grp. gram.:m.}
\end{itemize}
\begin{itemize}
\item {Utilização:Ant.}
\end{itemize}
O mesmo que \textunderscore clarinete\textunderscore .
\section{Claror}
\begin{itemize}
\item {Grp. gram.:m.}
\end{itemize}
\begin{itemize}
\item {Utilização:Prov.}
\end{itemize}
\begin{itemize}
\item {Utilização:alg.}
\end{itemize}
O mesmo que \textunderscore clarão\textunderscore ^1.
\section{Classar}
\begin{itemize}
\item {Grp. gram.:v.}
\end{itemize}
\begin{itemize}
\item {Utilização:t. Hist. Nat.}
\end{itemize}
\begin{itemize}
\item {Proveniência:(De \textunderscore classe\textunderscore )}
\end{itemize}
O mesmo que \textunderscore classificar\textunderscore .
\section{Classe}
\begin{itemize}
\item {Grp. gram.:f.}
\end{itemize}
\begin{itemize}
\item {Utilização:Bras. da Baía}
\end{itemize}
\begin{itemize}
\item {Proveniência:(Lat. \textunderscore classis\textunderscore )}
\end{itemize}
Categoria social: \textunderscore a classe dos funccionários\textunderscore .
Categoria, grupo: \textunderscore classe de animaes\textunderscore .
Cada uma das divisões de um conjunto.
Aula; alumnos de uma aula.
Carruagem de estrada de ferro.
\section{Classicismo}
\begin{itemize}
\item {Grp. gram.:m.}
\end{itemize}
Systema dos que admiram os clássicos.
Phrase ou estílo de clássicos.
A literatura clássica.
\section{Clássico}
\begin{itemize}
\item {Grp. gram.:adj.}
\end{itemize}
\begin{itemize}
\item {Grp. gram.:M.}
\end{itemize}
\begin{itemize}
\item {Proveniência:(Lat. \textunderscore classicus\textunderscore )}
\end{itemize}
Usado nas aulas.
Que é modêlo em bellas-letras: \textunderscore escritores clássicos\textunderscore .
Relativo á literatura grega e latina.
Autorizado por escritores clássicos: \textunderscore phrase clássica\textunderscore .
Antigo, inveterado.
Autor de obra clássica, e especialmente escritor grego ou latino.
\section{Classificação}
\begin{itemize}
\item {Grp. gram.:f.}
\end{itemize}
Acto ou effeito de classificar.
\section{Classificador}
\begin{itemize}
\item {Grp. gram.:m.}
\end{itemize}
Aquelle que classifica.
\section{Classificar}
\begin{itemize}
\item {Grp. gram.:v. t.}
\end{itemize}
\begin{itemize}
\item {Proveniência:(Do lat. \textunderscore classis\textunderscore  + \textunderscore facere\textunderscore )}
\end{itemize}
Distribuir em classes.
Pôr em ordem.
Determinar as categorias em que se divide e subdivide (um conjunto).
Qualificar.
\section{Classificável}
\begin{itemize}
\item {Grp. gram.:adj.}
\end{itemize}
Que se póde classificar.
\section{Clástica}
\begin{itemize}
\item {Grp. gram.:f.}
\end{itemize}
Cada uma das peças, que se inventaram para representar, em seu conjunto, todos os órgãos do corpo humano, e servir a estudos anatómicos.
(Cp. \textunderscore clástico\textunderscore )
\section{Clástico}
\begin{itemize}
\item {Grp. gram.:adj.}
\end{itemize}
\begin{itemize}
\item {Proveniência:(Do gr. \textunderscore klastos\textunderscore , quebrado)}
\end{itemize}
Diz-se das rochas formadas pela reunião de fragmentos de rochas de outros grupos.
\section{Clastra}
\begin{itemize}
\item {Grp. gram.:f.}
\end{itemize}
\begin{itemize}
\item {Utilização:Ant.}
\end{itemize}
O mesmo que \textunderscore crasta\textunderscore .
\section{Clathráceas}
\begin{itemize}
\item {Grp. gram.:f. pl.}
\end{itemize}
\begin{itemize}
\item {Proveniência:(Do lat. \textunderscore clathrare\textunderscore )}
\end{itemize}
Tríbo de cogumelos, na classificação de Broguiart.
\section{Clatráceas}
\begin{itemize}
\item {Grp. gram.:f. pl.}
\end{itemize}
\begin{itemize}
\item {Proveniência:(Do lat. \textunderscore clathrare\textunderscore )}
\end{itemize}
Tríbo de cogumelos, na classificação de Broguiart.
\section{Claudicação}
\begin{itemize}
\item {Grp. gram.:f.}
\end{itemize}
\begin{itemize}
\item {Proveniência:(Lat. \textunderscore claudicatio\textunderscore )}
\end{itemize}
Acto ou effeito de claudicar.
\section{Claudicante}
\begin{itemize}
\item {Grp. gram.:adj.}
\end{itemize}
\begin{itemize}
\item {Proveniência:(Lat. \textunderscore claudicans\textunderscore )}
\end{itemize}
Que claudica.
\section{Claudicar}
\begin{itemize}
\item {Grp. gram.:v. i.}
\end{itemize}
\begin{itemize}
\item {Utilização:Fig.}
\end{itemize}
\begin{itemize}
\item {Proveniência:(Lat. \textunderscore claudicare\textunderscore )}
\end{itemize}
Não têr firmeza em um dos pés.
Coxear.
Fraquejar intellectualmente.
Cometer falta.
Têr imperfeição, deficiência.
\section{Claudicar}
\begin{itemize}
\item {Grp. gram.:v. t.}
\end{itemize}
\begin{itemize}
\item {Utilização:Prov.}
\end{itemize}
\begin{itemize}
\item {Proveniência:(De \textunderscore cláudio\textunderscore )}
\end{itemize}
Pregar; segurar com alfinetes.
\section{Cláudio}
\begin{itemize}
\item {Grp. gram.:m.}
\end{itemize}
\begin{itemize}
\item {Utilização:Prov.}
\end{itemize}
\begin{itemize}
\item {Proveniência:(Do b. lat. \textunderscore claudus\textunderscore )}
\end{itemize}
Alfinete.
\section{Claustra}
\begin{itemize}
\item {Grp. gram.:f.}
\end{itemize}
\begin{itemize}
\item {Utilização:Des.}
\end{itemize}
O mesmo que \textunderscore claustro\textunderscore .
\section{Claustral}
\begin{itemize}
\item {Grp. gram.:adj.}
\end{itemize}
\begin{itemize}
\item {Proveniência:(Lat. \textunderscore claustralis\textunderscore )}
\end{itemize}
Relativo a claustro.
\section{Claustralidade}
\begin{itemize}
\item {Grp. gram.:f.}
\end{itemize}
\begin{itemize}
\item {Utilização:Des.}
\end{itemize}
\begin{itemize}
\item {Proveniência:(De \textunderscore claustral\textunderscore )}
\end{itemize}
Maus costumes de pessôas que vivem nos conventos.
\section{Claustrar}
\begin{itemize}
\item {Grp. gram.:v. t.}
\end{itemize}
Converter em claustro. Cf. Filinto, I, 323.
\section{Claustro}
\begin{itemize}
\item {Grp. gram.:m.}
\end{itemize}
\begin{itemize}
\item {Utilização:Ant.}
\end{itemize}
\begin{itemize}
\item {Utilização:Fig.}
\end{itemize}
\begin{itemize}
\item {Proveniência:(Lat. \textunderscore claustrum\textunderscore )}
\end{itemize}
Pátio interior, descoberto e rodeado geralmente de arcarias, nos conventos, ou nos edifícios que o foram.
Convento.
Vida monástica.
Assembleia de professores universitários.
Terreno vedado com parede ou tapume; cêrca, cerrado.
Concentração moral.
Restricção, limite. Cf. Bernardes, \textunderscore Luz e calor\textunderscore , 205.
\section{Claustrofobia}
\begin{itemize}
\item {Grp. gram.:f.}
\end{itemize}
\begin{itemize}
\item {Proveniência:(T. hybr., do lat. \textunderscore claustrum\textunderscore  + gr. \textunderscore phobein\textunderscore )}
\end{itemize}
Medo mórbido da clausura ou dos pequenos espaços.
\section{Claustrófobo}
\begin{itemize}
\item {Grp. gram.:m.}
\end{itemize}
Aquele que sofre claustrofobia.
\section{Claustrophobia}
\begin{itemize}
\item {Grp. gram.:f.}
\end{itemize}
\begin{itemize}
\item {Proveniência:(T. hybr., do lat. \textunderscore claustrum\textunderscore  + gr. \textunderscore phobein\textunderscore )}
\end{itemize}
Medo mórbido da clausura ou dos pequenos espaços.
\section{Claustróphobo}
\begin{itemize}
\item {Grp. gram.:m.}
\end{itemize}
Aquelle que soffre claustrophobia.
\section{Cláusula}
\begin{itemize}
\item {Grp. gram.:f.}
\end{itemize}
\begin{itemize}
\item {Proveniência:(Lat. \textunderscore clausula\textunderscore )}
\end{itemize}
Preceito, que faz parte de um contrato público ou particular.
Artigo; condição.
\section{Claitónia}
\begin{itemize}
\item {Grp. gram.:f.}
\end{itemize}
Planta hortense no Brasil, também conhecida por \textunderscore espinafre de Cuba\textunderscore .
\section{Clausular}
\begin{itemize}
\item {Grp. gram.:v. t.}
\end{itemize}
Dividir em cláusulas.
Estabelecer cláusulas em.
\section{Clausura}
\begin{itemize}
\item {Grp. gram.:f.}
\end{itemize}
\begin{itemize}
\item {Proveniência:(Lat. \textunderscore clausura\textunderscore )}
\end{itemize}
Recinto fechado.
Reclusão.
Vida conventual.
\section{Clausural}
\begin{itemize}
\item {Grp. gram.:adj.}
\end{itemize}
Relativo a clausura. Cf. Filinto, X, 111.
\section{Clausurar}
\begin{itemize}
\item {Grp. gram.:v. t.}
\end{itemize}
O mesmo que \textunderscore enclausurar\textunderscore .
\section{Clausuras}
\begin{itemize}
\item {Grp. gram.:f. pl.}
\end{itemize}
\begin{itemize}
\item {Utilização:Prov.}
\end{itemize}
Considerações diffusas de quem se desculpa ou pretende alguma coisa.
(Por \textunderscore cláusulas\textunderscore )
\section{Clava}
\begin{itemize}
\item {Grp. gram.:f.}
\end{itemize}
\begin{itemize}
\item {Proveniência:(Lat. \textunderscore clava\textunderscore )}
\end{itemize}
Moça, maça: \textunderscore a clava de Hércules\textunderscore .
\section{Clavaria}
\begin{itemize}
\item {Grp. gram.:f.}
\end{itemize}
Dignidade de claveiro.
(B. lat. \textunderscore clavaria\textunderscore )
\section{Clavário}
\begin{itemize}
\item {Grp. gram.:m.}
\end{itemize}
O mesmo que \textunderscore claveiro\textunderscore .
\section{Clavásia}
\begin{itemize}
\item {Grp. gram.:f.}
\end{itemize}
Gênero de cogumelos.
\section{Clave}
\begin{itemize}
\item {Grp. gram.:f.}
\end{itemize}
\begin{itemize}
\item {Proveniência:(Lat. \textunderscore clavis\textunderscore )}
\end{itemize}
Sinal, no princípio de uma pauta de música, para indicar o nome das notas e o grau, mais ou menos elevado, do som que ellas representam.
\section{Clavecinista}
\begin{itemize}
\item {Grp. gram.:m.}
\end{itemize}
Tocador de clavecino.
\section{Clavecino}
\begin{itemize}
\item {Grp. gram.:m.}
\end{itemize}
O mesmo que \textunderscore clavezingo\textunderscore .
\section{Claveiro}
\begin{itemize}
\item {Grp. gram.:m.}
\end{itemize}
\begin{itemize}
\item {Proveniência:(Do b. lat. \textunderscore clavarius\textunderscore )}
\end{itemize}
Chaveiro, em algumas Ordens religiosas ou militares.
\section{Clavelina}
\begin{itemize}
\item {Grp. gram.:f.}
\end{itemize}
O mesmo que \textunderscore cravina\textunderscore .
\section{Clávena}
\begin{itemize}
\item {Grp. gram.:f.}
\end{itemize}
Gênero de plantas compostas.
\section{Clavezingo}
\begin{itemize}
\item {Grp. gram.:m.}
\end{itemize}
\begin{itemize}
\item {Utilização:Des.}
\end{itemize}
\begin{itemize}
\item {Proveniência:(Fr. \textunderscore clavecin\textunderscore )}
\end{itemize}
Espécie de cravo, instrumento musical.
\section{Claviarpa}
\begin{itemize}
\item {Grp. gram.:f.}
\end{itemize}
\begin{itemize}
\item {Proveniência:(De \textunderscore clave\textunderscore  + \textunderscore harpa\textunderscore )}
\end{itemize}
Espécie de piano, cujos martelos ferem cordas como as da harpa.
\section{Clavicilindro}
\begin{itemize}
\item {Grp. gram.:m.}
\end{itemize}
\begin{itemize}
\item {Proveniência:(De \textunderscore clave\textunderscore  + \textunderscore cylindro\textunderscore )}
\end{itemize}
Instrumento de cordas, em que estas produzem o som, roçando por um cilindro de vidro em movimento giratório.
\section{Clavicímbalo}
\begin{itemize}
\item {Grp. gram.:m.}
\end{itemize}
O mesmo que \textunderscore clavicórdio\textunderscore .
\section{Clavicítara}
\begin{itemize}
\item {Grp. gram.:f.}
\end{itemize}
\begin{itemize}
\item {Utilização:Mús.}
\end{itemize}
\begin{itemize}
\item {Utilização:ant.}
\end{itemize}
Espécie de cravo, que se usou nos séculos XVI e XVII, e cuja caixa sonora, armada de cordas de tripa, estava collocada verticalmente, acima do teclado.
\section{Clavicíthara}
\begin{itemize}
\item {Grp. gram.:f.}
\end{itemize}
\begin{itemize}
\item {Utilização:Mús.}
\end{itemize}
\begin{itemize}
\item {Utilização:ant.}
\end{itemize}
Espécie de cravo, que se usou nos séculos XVI e XVII, e cuja caixa sonora, armada de cordas de tripa, estava collocada verticalmente, acima do teclado.
\section{Clavicórdio}
\begin{itemize}
\item {Grp. gram.:m.}
\end{itemize}
\begin{itemize}
\item {Proveniência:(Do lat. \textunderscore clavis\textunderscore  + \textunderscore chorda\textunderscore )}
\end{itemize}
Antigo instrumento músico.
\section{Clavicórneos}
\begin{itemize}
\item {Grp. gram.:m. pl.}
\end{itemize}
\begin{itemize}
\item {Proveniência:(De \textunderscore clava\textunderscore  + \textunderscore córneo\textunderscore )}
\end{itemize}
Uma das famílias dos insectos coleópteros pentâmeros, caracterizada por têr as antennas em fórma de clava.
\section{Clavícula}
\begin{itemize}
\item {Grp. gram.:f.}
\end{itemize}
\begin{itemize}
\item {Utilização:Anat.}
\end{itemize}
\begin{itemize}
\item {Proveniência:(Lat. \textunderscore clavicula\textunderscore )}
\end{itemize}
Pequena chave.
Osso, na parte anterior do ombro.
\section{Claviculado}
\begin{itemize}
\item {Grp. gram.:adj.}
\end{itemize}
\begin{itemize}
\item {Grp. gram.:M. pl.}
\end{itemize}
\begin{itemize}
\item {Proveniência:(De \textunderscore clavícula\textunderscore )}
\end{itemize}
Que tem clavículas.
Mamíferos roedores, que têm clavículas perfeitas.
\section{Clavicular}
\begin{itemize}
\item {Grp. gram.:adj.}
\end{itemize}
Relativo á clavícula.
\section{Claviculário}
\begin{itemize}
\item {Grp. gram.:m.}
\end{itemize}
O mesmo que \textunderscore chaveiro\textunderscore .
\section{Clavicylindro}
\begin{itemize}
\item {Grp. gram.:m.}
\end{itemize}
\begin{itemize}
\item {Proveniência:(De \textunderscore clave\textunderscore  + \textunderscore cylindro\textunderscore )}
\end{itemize}
Instrumento de cordas, em que estas produzem o som, roçando por um cylindro de vidro em movimento giratório.
\section{Clavifoliado}
\begin{itemize}
\item {Grp. gram.:adj.}
\end{itemize}
\begin{itemize}
\item {Utilização:Bot.}
\end{itemize}
\begin{itemize}
\item {Proveniência:(Do lat. \textunderscore clava\textunderscore  + \textunderscore folium\textunderscore )}
\end{itemize}
Que têm as fôlhas em fórma de clava.
\section{Claviforme}
\begin{itemize}
\item {Grp. gram.:adj.}
\end{itemize}
\begin{itemize}
\item {Proveniência:(De \textunderscore clava\textunderscore  + \textunderscore forma\textunderscore )}
\end{itemize}
Semelhante a uma clava.
\section{Clavígero}
\begin{itemize}
\item {Grp. gram.:adj.}
\end{itemize}
\begin{itemize}
\item {Proveniência:(Do lat. \textunderscore clavis\textunderscore  + \textunderscore gerere\textunderscore )}
\end{itemize}
Que leva as chaves, (epítheto de Jano).
\section{Clavígeros}
\begin{itemize}
\item {Grp. gram.:m. pl.}
\end{itemize}
\begin{itemize}
\item {Proveniência:(Do lat. \textunderscore clava\textunderscore  + \textunderscore gerere\textunderscore )}
\end{itemize}
Insectos coleópteros trímeros, que são armados de uma espécie de pequena clava.
\section{Claviharpa}
\begin{itemize}
\item {Grp. gram.:f.}
\end{itemize}
\begin{itemize}
\item {Proveniência:(De \textunderscore clave\textunderscore  + \textunderscore harpa\textunderscore )}
\end{itemize}
Espécie de piano, cujos martelos ferem cordas como as da harpa.
\section{Clavija}
\begin{itemize}
\item {Grp. gram.:f.}
\end{itemize}
Escápula, em que o tintureiro pendura as meadas para secarem.
Peça do tear, em que se enrola a meada, que se vai tecer.
Cravelha, com que se liga o jôgo dianteiro ao jôgo traseiro dos carros.
(Cast. \textunderscore clavija\textunderscore )
\section{Clavilâmina}
\begin{itemize}
\item {Grp. gram.:f.}
\end{itemize}
Instrumento de teclado, cujos sons são produzidos por lâminas de aço, postas em vibração.
\section{Clavilha}
\begin{itemize}
\item {Grp. gram.:f.}
\end{itemize}
(Dem. de \textunderscore clave\textunderscore )
\section{Clavilira}
\begin{itemize}
\item {Grp. gram.:f.}
\end{itemize}
\begin{itemize}
\item {Proveniência:(De \textunderscore clave\textunderscore  + \textunderscore lyra\textunderscore )}
\end{itemize}
O mesmo que \textunderscore claviharpa\textunderscore .
\section{Clavilyra}
\begin{itemize}
\item {Grp. gram.:f.}
\end{itemize}
\begin{itemize}
\item {Proveniência:(De \textunderscore clave\textunderscore  + \textunderscore lyra\textunderscore )}
\end{itemize}
O mesmo que \textunderscore claviharpa\textunderscore .
\section{Clavina}
\begin{itemize}
\item {Grp. gram.:f.}
\end{itemize}
(Corr. de \textunderscore carabina\textunderscore )
\section{Clavina}
\begin{itemize}
\item {Grp. gram.:f.}
\end{itemize}
\begin{itemize}
\item {Utilização:Pharm.}
\end{itemize}
Substância, extrahida do centeio e que é medicamento para provocar contracções uterinas.
\section{Clavinaço}
\begin{itemize}
\item {Grp. gram.:m.}
\end{itemize}
\begin{itemize}
\item {Utilização:Des.}
\end{itemize}
Tiro de clavina.
\section{Clavineiro}
\begin{itemize}
\item {Grp. gram.:m.}
\end{itemize}
\begin{itemize}
\item {Utilização:Ant.}
\end{itemize}
Soldado, armado de clavina.
\section{Clavinote}
\begin{itemize}
\item {Grp. gram.:m.}
\end{itemize}
\begin{itemize}
\item {Utilização:Bras. do N}
\end{itemize}
Pequena clavina^1.
\section{Clavinoteiro}
\begin{itemize}
\item {Grp. gram.:m.  e  adj.}
\end{itemize}
\begin{itemize}
\item {Utilização:Bras}
\end{itemize}
\begin{itemize}
\item {Proveniência:(De \textunderscore clavinote\textunderscore )}
\end{itemize}
Bandido sertanejo, armado de clavina.
\section{Claviórgão}
\begin{itemize}
\item {Grp. gram.:m.}
\end{itemize}
\begin{itemize}
\item {Proveniência:(De \textunderscore clave\textunderscore  + \textunderscore órgão\textunderscore )}
\end{itemize}
Instrumento musical, grande cravo antigo, com um ou mais registos de órgão.
\section{Clavisignato}
\begin{itemize}
\item {fónica:si}
\end{itemize}
\begin{itemize}
\item {Grp. gram.:m.}
\end{itemize}
\begin{itemize}
\item {Proveniência:(Do lat. \textunderscore clavis\textunderscore  + \textunderscore signatus\textunderscore )}
\end{itemize}
Antigo soldado pontifício, que tinha por insígnia as chaves da bandeira papal.
\section{Clavissignato}
\begin{itemize}
\item {Grp. gram.:m.}
\end{itemize}
\begin{itemize}
\item {Proveniência:(Do lat. \textunderscore clavis\textunderscore  + \textunderscore signatus\textunderscore )}
\end{itemize}
Antigo soldado pontifício, que tinha por insígnia as chaves da bandeira papal.
\section{Claytónia}
\begin{itemize}
\item {Grp. gram.:f.}
\end{itemize}
Planta hortense no Brasil, também conhecida por \textunderscore espinafre de Cuba\textunderscore .
\section{Clazomênio}
\begin{itemize}
\item {Grp. gram.:adj.}
\end{itemize}
\begin{itemize}
\item {Proveniência:(Lat. \textunderscore clazomenius\textunderscore )}
\end{itemize}
Natural de Clazomena; relativo a Clazomena.
\section{Cleftomania}
\begin{itemize}
\item {Grp. gram.:f.}
\end{itemize}
O mesmo que \textunderscore clopemania\textunderscore .
\section{Clélia}
\begin{itemize}
\item {Grp. gram.:f.}
\end{itemize}
Gênero de reptis ophídeos.
\section{Clematicissos}
\begin{itemize}
\item {Grp. gram.:m. pl.}
\end{itemize}
\begin{itemize}
\item {Proveniência:(Do lat. \textunderscore clematis\textunderscore  + \textunderscore cissus\textunderscore )}
\end{itemize}
Um dos gêneros de videiras, em que se dividiu a família das ampelídeas.
\section{Clematídeas}
\begin{itemize}
\item {Grp. gram.:f. pl.}
\end{itemize}
Tríbo de plantas ranunculáceas, na classificação de De-Candolle.
(Cp. \textunderscore clematite\textunderscore )
\section{Clematite}
\begin{itemize}
\item {Grp. gram.:f.}
\end{itemize}
\begin{itemize}
\item {Proveniência:(Gr. \textunderscore klematitis\textunderscore )}
\end{itemize}
Planta trepadeira, ranunculácea.
\section{Clemência}
\begin{itemize}
\item {Grp. gram.:f.}
\end{itemize}
\begin{itemize}
\item {Utilização:Fig.}
\end{itemize}
\begin{itemize}
\item {Proveniência:(Lat. \textunderscore clementia\textunderscore )}
\end{itemize}
Indulgência, bondade.
Amenidade: \textunderscore a clemência do clima\textunderscore .
\section{Clemenciar}
\begin{itemize}
\item {Grp. gram.:v. t.}
\end{itemize}
Tratar com clemência. Cf. Filinto, IX, 132.
\section{Clemente}
\begin{itemize}
\item {Grp. gram.:adj.}
\end{itemize}
\begin{itemize}
\item {Proveniência:(Lat. \textunderscore clemens\textunderscore )}
\end{itemize}
Que tem clemência.
Indulgente.
Bondoso.
\section{Clementemente}
\begin{itemize}
\item {Grp. gram.:adv.}
\end{itemize}
De modo clemente.
Com clemência.
\section{Clementinas}
\begin{itemize}
\item {Grp. gram.:f. pl.}
\end{itemize}
\begin{itemize}
\item {Proveniência:(De \textunderscore Clemente\textunderscore , n. p.)}
\end{itemize}
Decretaes de Clemente V, publicadas em 1317.
\section{Clementino}
\begin{itemize}
\item {Grp. gram.:adj.}
\end{itemize}
Diz-se do alphabeto, usado pelos Russos e outros povos esclavónicos do ramo oriental.
\section{Clenoides}
\begin{itemize}
\item {Grp. gram.:m. pl.}
\end{itemize}
Peixes que, na classificação de Agassiz, correspondem aos acanthopterýgios.
\section{Cleónia}
\begin{itemize}
\item {Grp. gram.:f.}
\end{itemize}
Gênero de plantas labiadas.
\section{Cleopatrino}
\begin{itemize}
\item {Grp. gram.:adj.}
\end{itemize}
\begin{itemize}
\item {Utilização:Neol.}
\end{itemize}
Relativo a Cleópatra.
\section{Cleopatrizar}
\begin{itemize}
\item {Grp. gram.:v. t.}
\end{itemize}
\begin{itemize}
\item {Utilização:Neol.}
\end{itemize}
\begin{itemize}
\item {Proveniência:(De \textunderscore Cleópatra\textunderscore , n. p.)}
\end{itemize}
Tornar ardente, luxurioso.
\section{Clephtomania}
\begin{itemize}
\item {Grp. gram.:f.}
\end{itemize}
O mesmo que \textunderscore clopemania\textunderscore .
\section{Clepsidra}
\begin{itemize}
\item {Grp. gram.:f.}
\end{itemize}
\begin{itemize}
\item {Proveniência:(Gr. \textunderscore klepsudra\textunderscore )}
\end{itemize}
Relógio de água.
\section{Clepsidro}
\begin{itemize}
\item {Grp. gram.:m.}
\end{itemize}
O mesmo que \textunderscore clepsidra\textunderscore .
\section{Clepsydra}
\begin{itemize}
\item {Grp. gram.:f.}
\end{itemize}
\begin{itemize}
\item {Proveniência:(Gr. \textunderscore klepsudra\textunderscore )}
\end{itemize}
Relógio de água.
\section{Clepsydro}
\begin{itemize}
\item {Grp. gram.:m.}
\end{itemize}
O mesmo que \textunderscore clepsydra\textunderscore .
\section{Clerestório}
\begin{itemize}
\item {Grp. gram.:m.}
\end{itemize}
\begin{itemize}
\item {Proveniência:(Do ingl. \textunderscore clerc\textunderscore  + \textunderscore story\textunderscore )}
\end{itemize}
Galeria superior ao trifório nas igrejas ogivaes.
\section{Clerezia}
\begin{itemize}
\item {Grp. gram.:f.}
\end{itemize}
\begin{itemize}
\item {Proveniência:(Do lat. \textunderscore clericus\textunderscore )}
\end{itemize}
Classe clerical; clero.
\section{Clerical}
\begin{itemize}
\item {Grp. gram.:adj.}
\end{itemize}
\begin{itemize}
\item {Proveniência:(Lat. \textunderscore clericalis\textunderscore )}
\end{itemize}
Relativo aos clérigos, ao clero: \textunderscore a influência clerical\textunderscore .
\section{Clericalismo}
\begin{itemize}
\item {Grp. gram.:m.}
\end{itemize}
\begin{itemize}
\item {Proveniência:(De \textunderscore clerical\textunderscore )}
\end{itemize}
Partido clerical.
Systema dos que apoiam incondicionalmente o clero.
\section{Clericalmente}
\begin{itemize}
\item {Grp. gram.:adv.}
\end{itemize}
Á maneira do clero; de modo clerical.
\section{Clericato}
\begin{itemize}
\item {Grp. gram.:m.}
\end{itemize}
\begin{itemize}
\item {Proveniência:(Lat. \textunderscore clericatus\textunderscore )}
\end{itemize}
Estado, dignidade, do clero.
\section{Clériga}
\begin{itemize}
\item {Grp. gram.:f.}
\end{itemize}
\begin{itemize}
\item {Utilização:Ant.}
\end{itemize}
A monja, que rezava no côro o offício divino; corista.
(B. lat. \textunderscore clerica\textunderscore )
\section{Clérigo}
\begin{itemize}
\item {Grp. gram.:m.}
\end{itemize}
\begin{itemize}
\item {Utilização:Ichthyol.}
\end{itemize}
\begin{itemize}
\item {Grp. gram.:Loc.}
\end{itemize}
\begin{itemize}
\item {Utilização:pop.}
\end{itemize}
\begin{itemize}
\item {Proveniência:(Lat. \textunderscore clericus\textunderscore )}
\end{itemize}
Aquelle que tem algumas ou todas as Ordens Sacras da Igreja Cathólica; padre.
Nome de um peixe.
\textunderscore Cantar de clérigo\textunderscore , fanfarrear, alanzoar.
\section{Clero}
\begin{itemize}
\item {Grp. gram.:m.}
\end{itemize}
\begin{itemize}
\item {Proveniência:(Lat. \textunderscore clerus\textunderscore )}
\end{itemize}
Classe clerical.
Corporação de sacerdotes.
\section{Clerodendro}
\begin{itemize}
\item {Grp. gram.:m.}
\end{itemize}
\begin{itemize}
\item {Proveniência:(Do gr. \textunderscore kleros\textunderscore  + \textunderscore dendron\textunderscore )}
\end{itemize}
Planta ornamental, da fam. das verbenáceas.
\section{Clerões}
\begin{itemize}
\item {Grp. gram.:m. pl.}
\end{itemize}
Gênero de insectos coleópteros.
\section{Cleromancia}
\begin{itemize}
\item {Grp. gram.:f.}
\end{itemize}
\begin{itemize}
\item {Proveniência:(Do gr. \textunderscore kleros\textunderscore  + \textunderscore manteia\textunderscore )}
\end{itemize}
Supposta arte de adivinhar por meio de dados.
\section{Cleromante}
\begin{itemize}
\item {Grp. gram.:m.}
\end{itemize}
Aquelle que pratíca a cleromancia.
\section{Cleromântico}
\begin{itemize}
\item {Grp. gram.:adj.}
\end{itemize}
Relativo á cleromancia.
\section{Clethra}
\begin{itemize}
\item {Grp. gram.:f.}
\end{itemize}
\begin{itemize}
\item {Proveniência:(Gr. \textunderscore klethra\textunderscore , amieiro)}
\end{itemize}
Gênero de árvores ericáceas.
\section{Cletra}
\begin{itemize}
\item {Grp. gram.:f.}
\end{itemize}
\begin{itemize}
\item {Proveniência:(Gr. \textunderscore klethra\textunderscore , amieiro)}
\end{itemize}
Gênero de árvores ericáceas.
\section{Cliché}
\begin{itemize}
\item {Grp. gram.:m.}
\end{itemize}
Francesismo inútil, porque, em photographia, temos o voc. correspondente, que é \textunderscore matriz\textunderscore ; e, noutras accepções, devemos representá-lo por \textunderscore molde\textunderscore , \textunderscore chavão\textunderscore , etc.
\section{Clícia}
\begin{itemize}
\item {Grp. gram.:f.}
\end{itemize}
Gênero de dípteros.
\section{Clido-costal}
\begin{itemize}
\item {Grp. gram.:adj.}
\end{itemize}
\begin{itemize}
\item {Utilização:Anat.}
\end{itemize}
Relativo á clavícula e ás costas. Cf. J. A. Serrano, \textunderscore Osteologia\textunderscore , I, 189.
\section{Clido-mammário}
\begin{itemize}
\item {Grp. gram.:adj.}
\end{itemize}
\begin{itemize}
\item {Proveniência:(Do gr. \textunderscore kleis\textunderscore , \textunderscore kleidos\textunderscore , e \textunderscore mammário\textunderscore )}
\end{itemize}
Relativo á clavícula e á região mammária.
\section{Clidomancia}
\begin{itemize}
\item {Grp. gram.:f.}
\end{itemize}
\begin{itemize}
\item {Proveniência:(Do gr. \textunderscore kleis\textunderscore , \textunderscore kleidos\textunderscore  + \textunderscore manteia\textunderscore )}
\end{itemize}
Adivinhação, por meio de uma chave presa por um fio á \textunderscore Bíblia\textunderscore . Cp. Castilho, \textunderscore Fastos\textunderscore , III, 322.
\section{Clidomântico}
\begin{itemize}
\item {Grp. gram.:adj.}
\end{itemize}
Relativo á clidomancia.
\section{Clidoscopia}
\begin{itemize}
\item {Grp. gram.:f.}
\end{itemize}
\begin{itemize}
\item {Proveniência:(Do gr. \textunderscore kleis\textunderscore  + \textunderscore skopein\textunderscore )}
\end{itemize}
O mesmo que \textunderscore clidomancia\textunderscore .
\section{Cliente}
\begin{itemize}
\item {Grp. gram.:m.  e  f.}
\end{itemize}
\begin{itemize}
\item {Proveniência:(Lat. \textunderscore cliens\textunderscore )}
\end{itemize}
Pessôa protegida.
Constituinte, pessôa, que confia a defesa de negócios seus a procurador ou advogado.
Aquelle que é tratado habitualmente por um médico.
Freguês.
\section{Clientela}
\begin{itemize}
\item {Grp. gram.:f.}
\end{itemize}
\begin{itemize}
\item {Proveniência:(Lat. \textunderscore clientela\textunderscore )}
\end{itemize}
Conjunto de clientes.
\section{Clima}
\begin{itemize}
\item {Grp. gram.:m.}
\end{itemize}
\begin{itemize}
\item {Proveniência:(Do gr. \textunderscore klima\textunderscore )}
\end{itemize}
Temperatura e outras condições atmosphéricas de determinada região.
Região, em que a temperatura e mais condições atmosphéricas são geralmente as mesmas.
Zona terrestre, entre círculos parallelos.
\section{Climactérico}
\begin{itemize}
\item {Grp. gram.:adj.}
\end{itemize}
\begin{itemize}
\item {Proveniência:(Gr. \textunderscore klimakterikos\textunderscore )}
\end{itemize}
Relativo a qualquer das épocas da vida, consideradas antigamente como críticas.
\section{Climático}
\begin{itemize}
\item {Grp. gram.:adj.}
\end{itemize}
\begin{itemize}
\item {Proveniência:(Do gr. \textunderscore klima\textunderscore , \textunderscore klimatos\textunderscore )}
\end{itemize}
Relativo ao clima.
Climatológico.
\section{Climatização}
\begin{itemize}
\item {Grp. gram.:f.}
\end{itemize}
O mesmo que \textunderscore aclimação\textunderscore .
\section{Climatizar}
\begin{itemize}
\item {Grp. gram.:v. t.}
\end{itemize}
O mesmo que \textunderscore aclimar\textunderscore .
\section{Climatologia}
\begin{itemize}
\item {Grp. gram.:f.}
\end{itemize}
\begin{itemize}
\item {Proveniência:(Do gr. \textunderscore klima\textunderscore  + \textunderscore logos\textunderscore )}
\end{itemize}
Tratado dos climas ou da sua influência sôbre a economia animal.
\section{Climatológico}
\begin{itemize}
\item {Grp. gram.:adj.}
\end{itemize}
Relativo á climatologia.
Relativo ao clima.
\section{Climatoterapia}
\begin{itemize}
\item {Grp. gram.:f.}
\end{itemize}
\begin{itemize}
\item {Proveniência:(Do gr. \textunderscore klima\textunderscore , \textunderscore klimatos\textunderscore  + \textunderscore therapeia\textunderscore )}
\end{itemize}
Terapêutica, que tem por base a procura do bom ar, mormente o ar das regiões elevadas.
\section{Climatoterápico}
\begin{itemize}
\item {Grp. gram.:adj.}
\end{itemize}
Relativo á climatoterapia.
\section{Climatotherapia}
\begin{itemize}
\item {Grp. gram.:f.}
\end{itemize}
\begin{itemize}
\item {Proveniência:(Do gr. \textunderscore klima\textunderscore , \textunderscore klimatos\textunderscore  + \textunderscore therapeia\textunderscore )}
\end{itemize}
Therapêutica, que tem por base a procura do bom ar, mormente o ar das regiões elevadas.
\section{Climatotherápico}
\begin{itemize}
\item {Grp. gram.:adj.}
\end{itemize}
Relativo á climatotherapia.
\section{Clímax}
\begin{itemize}
\item {Grp. gram.:m.}
\end{itemize}
\begin{itemize}
\item {Proveniência:(Gr. \textunderscore klimax\textunderscore , escala)}
\end{itemize}
O mesmo que \textunderscore gradação\textunderscore .
\section{Clina}
\begin{itemize}
\item {Grp. gram.:f.}
\end{itemize}
O mesmo que \textunderscore crina\textunderscore .
(Cast. \textunderscore clin\textunderscore )
\section{Cliname}
\begin{itemize}
\item {Grp. gram.:m.}
\end{itemize}
\begin{itemize}
\item {Proveniência:(Lat. \textunderscore clinamen\textunderscore )}
\end{itemize}
A declinação dos átomos, no systema de Epicuro.
\section{Clinantho}
\begin{itemize}
\item {Grp. gram.:m.}
\end{itemize}
\begin{itemize}
\item {Utilização:Bot.}
\end{itemize}
\begin{itemize}
\item {Proveniência:(Do gr. \textunderscore kline\textunderscore  + \textunderscore anthos\textunderscore )}
\end{itemize}
Superfície plana, que limita um pedúnculo commum.
\section{Clinanto}
\begin{itemize}
\item {Grp. gram.:m.}
\end{itemize}
\begin{itemize}
\item {Utilização:Bot.}
\end{itemize}
\begin{itemize}
\item {Proveniência:(Do gr. \textunderscore kline\textunderscore  + \textunderscore anthos\textunderscore )}
\end{itemize}
Superfície plana, que limita um pedúnculo comum.
\section{Clínica}
\begin{itemize}
\item {Grp. gram.:f.}
\end{itemize}
Prática da Medicina.
Clientela de um médico.
(Cp. \textunderscore clínico\textunderscore )
\section{Clínico}
\begin{itemize}
\item {Grp. gram.:m.}
\end{itemize}
\begin{itemize}
\item {Grp. gram.:Adj.}
\end{itemize}
\begin{itemize}
\item {Proveniência:(Lat. \textunderscore clinicus\textunderscore )}
\end{itemize}
Médico, que visita doentes, que exerce a clínica.
Relativo ao tratamento médico dos doentes.
\section{Clinocéfalo}
\begin{itemize}
\item {Grp. gram.:adj.}
\end{itemize}
\begin{itemize}
\item {Utilização:Bot.}
\end{itemize}
\begin{itemize}
\item {Proveniência:(Do gr. \textunderscore klinè\textunderscore  + \textunderscore kephalé\textunderscore )}
\end{itemize}
Que tem a parte superior acamada ou em fórma de sela.
\section{Clinocéphalo}
\begin{itemize}
\item {Grp. gram.:adj.}
\end{itemize}
\begin{itemize}
\item {Utilização:Bot.}
\end{itemize}
\begin{itemize}
\item {Proveniência:(Do gr. \textunderscore klinè\textunderscore  + \textunderscore kephalé\textunderscore )}
\end{itemize}
Que tem a parte superior acamada ou em fórma de sella.
\section{Clinochloro}
\begin{itemize}
\item {Grp. gram.:m.}
\end{itemize}
\begin{itemize}
\item {Utilização:Geol.}
\end{itemize}
Uma das principaes espécies de chlorite.
\section{Clinocloro}
\begin{itemize}
\item {Grp. gram.:m.}
\end{itemize}
\begin{itemize}
\item {Utilização:Geol.}
\end{itemize}
Uma das principaes espécies de clorite.
\section{Clinodiagonal}
\begin{itemize}
\item {Grp. gram.:adj.}
\end{itemize}
\begin{itemize}
\item {Utilização:Geol.}
\end{itemize}
\begin{itemize}
\item {Proveniência:(Do gr. \textunderscore klinein\textunderscore )}
\end{itemize}
Diz-se de um dos eixos oblíquos dos mineraes do systema monoclínico.
\section{Clinodoma}
\begin{itemize}
\item {Grp. gram.:m.}
\end{itemize}
\begin{itemize}
\item {Utilização:Geol.}
\end{itemize}
Uma das fórmas holoédricas dos mineraes monoclínicos.
\section{Clinoide}
\begin{itemize}
\item {Grp. gram.:adj.}
\end{itemize}
\begin{itemize}
\item {Utilização:Anat.}
\end{itemize}
\begin{itemize}
\item {Proveniência:(Do gr. \textunderscore kline\textunderscore  + \textunderscore eidos\textunderscore )}
\end{itemize}
Que tem forma de leito, (falando-se do espaço entre certas apóphyses).
\section{Clinómetro}
\begin{itemize}
\item {Grp. gram.:m.}
\end{itemize}
\begin{itemize}
\item {Proveniência:(Do gr. \textunderscore klinein\textunderscore  + \textunderscore metron\textunderscore )}
\end{itemize}
Instrumento, para medir inclinações do terreno ou outras.
\section{Clinopinacoide}
\begin{itemize}
\item {Grp. gram.:m.}
\end{itemize}
\begin{itemize}
\item {Utilização:Geol.}
\end{itemize}
Uma das fórmas holoédricas dos mineraes monoclínicos, limitada por dois planos parallelos entre si.
\section{Clinoterapia}
\begin{itemize}
\item {Grp. gram.:f.}
\end{itemize}
\begin{itemize}
\item {Proveniência:(Do gr. \textunderscore kline\textunderscore  + \textunderscore therapeuein\textunderscore )}
\end{itemize}
Tratamento médico, por meio do repoiso.
\section{Clinotherapia}
\begin{itemize}
\item {Grp. gram.:f.}
\end{itemize}
\begin{itemize}
\item {Proveniência:(Do gr. \textunderscore kline\textunderscore  + \textunderscore therapeuein\textunderscore )}
\end{itemize}
Tratamento médico, por meio do repoiso.
\section{Clínton}
\begin{itemize}
\item {Grp. gram.:f.}
\end{itemize}
\begin{itemize}
\item {Proveniência:(De \textunderscore Clinton\textunderscore , n. p.)}
\end{itemize}
Variedade de cepas americanas.
\section{Clintónia}
\begin{itemize}
\item {Grp. gram.:f.}
\end{itemize}
\begin{itemize}
\item {Proveniência:(De \textunderscore Clinton\textunderscore , n. p.)}
\end{itemize}
Gênero de plantas lobeliáceas da Colômbia.
\section{Clinudo}
\begin{itemize}
\item {Grp. gram.:adj.}
\end{itemize}
\begin{itemize}
\item {Utilização:Bras. do S}
\end{itemize}
Que tem grande clina.
\section{Clis}
\begin{itemize}
\item {Grp. gram.:m.}
\end{itemize}
\begin{itemize}
\item {Utilização:Prov.}
\end{itemize}
\begin{itemize}
\item {Utilização:dur.}
\end{itemize}
O mesmo que \textunderscore eclípse\textunderscore .
(Cp. \textunderscore cris\textunderscore )
\section{Clisar}
\begin{itemize}
\item {Grp. gram.:v. t.}
\end{itemize}
\begin{itemize}
\item {Utilização:Gír.}
\end{itemize}
\begin{itemize}
\item {Proveniência:(De \textunderscore clises\textunderscore )}
\end{itemize}
Vêr.
\section{Clise}
\begin{itemize}
\item {Grp. gram.:adj.}
\end{itemize}
\begin{itemize}
\item {Utilização:Prov.}
\end{itemize}
Diz-se do sol ou da lua, quando há eclipse: \textunderscore o sol está clise\textunderscore . (Colhido em Odemira)
(Cp. \textunderscore clis\textunderscore )
\section{Clises}
\begin{itemize}
\item {Grp. gram.:m. pl.}
\end{itemize}
\begin{itemize}
\item {Utilização:Gír.}
\end{itemize}
Olhos.
(Do caló \textunderscore clisé\textunderscore , ôlho)
\section{Clitória}
\begin{itemize}
\item {Grp. gram.:f.}
\end{itemize}
\begin{itemize}
\item {Utilização:Bras}
\end{itemize}
Trepadeira annual.
\section{Clitóride}
\begin{itemize}
\item {Grp. gram.:f.}
\end{itemize}
O mesmo ou melhor que \textunderscore clitóris\textunderscore .
\section{Clitoridectomia}
\begin{itemize}
\item {Grp. gram.:f.}
\end{itemize}
\begin{itemize}
\item {Proveniência:(Do gr. \textunderscore kleitoris\textunderscore  + \textunderscore ektome\textunderscore )}
\end{itemize}
Extracção cirúrgica do clitóris.
\section{Clitoridiano}
\begin{itemize}
\item {Grp. gram.:adj.}
\end{itemize}
Relativo á clitóride.
\section{Clitóris}
\begin{itemize}
\item {Grp. gram.:m.}
\end{itemize}
\begin{itemize}
\item {Utilização:Anat.}
\end{itemize}
\begin{itemize}
\item {Proveniência:(Gr. \textunderscore kleitoris\textunderscore )}
\end{itemize}
Protuberância carnuda, na parte superior da vulva.
\section{Clítoris}
\begin{itemize}
\item {Grp. gram.:m.}
\end{itemize}
\begin{itemize}
\item {Utilização:Anat.}
\end{itemize}
\begin{itemize}
\item {Proveniência:(Gr. \textunderscore kleitoris\textunderscore )}
\end{itemize}
Protuberância carnuda, na parte superior da vulva.
\section{Clivagem}
\begin{itemize}
\item {Grp. gram.:f.}
\end{itemize}
\begin{itemize}
\item {Proveniência:(Fr. \textunderscore clivage\textunderscore )}
\end{itemize}
Propriedade, que têm certos corpos mineraes, de se dividirem mais facilmente segundo certos planos, do que segundo quaesquer outras direcções.
\section{Clívia}
\begin{itemize}
\item {Grp. gram.:f.}
\end{itemize}
Planta herbácea do Cabo da Bôa-Esperança.
\section{Clivina}
\begin{itemize}
\item {Grp. gram.:f.}
\end{itemize}
Gênero de insectos coleópteros pentâmeros.
\section{Clivo}
\begin{itemize}
\item {Grp. gram.:m.}
\end{itemize}
\begin{itemize}
\item {Proveniência:(Lat. \textunderscore clivus\textunderscore )}
\end{itemize}
Ladeira.
Encosta de monte.
Oiteiro.
\section{Clivoso}
\begin{itemize}
\item {Grp. gram.:adj.}
\end{itemize}
\begin{itemize}
\item {Proveniência:(Lat. \textunderscore clivosus\textunderscore )}
\end{itemize}
Escarpado; em declive.
\section{Cloaca}
\begin{itemize}
\item {Grp. gram.:m.}
\end{itemize}
\begin{itemize}
\item {Utilização:Fig.}
\end{itemize}
\begin{itemize}
\item {Proveniência:(Lat. \textunderscore cloaca\textunderscore )}
\end{itemize}
Cano, fossa, que recebe immundícies.
Aquillo que é immundo, que cheira mal.
\section{Cloacal}
\begin{itemize}
\item {Grp. gram.:adj.}
\end{itemize}
\begin{itemize}
\item {Proveniência:(Lat. \textunderscore cloacalis\textunderscore )}
\end{itemize}
Relativo a cloaca.
\section{Cloacário}
\begin{itemize}
\item {Grp. gram.:m.}
\end{itemize}
\begin{itemize}
\item {Proveniência:(Lat. \textunderscore cloacarius\textunderscore )}
\end{itemize}
Aquelle que tratava das cloacas ou canos de esgôto, entre os Romanos.
\section{Cloacino}
\begin{itemize}
\item {Grp. gram.:adj.}
\end{itemize}
\begin{itemize}
\item {Proveniência:(Lat. \textunderscore cloacinus\textunderscore )}
\end{itemize}
Relativo a cloaca.
Latrinário; Indecente. Cf. Camillo, \textunderscore Vinho do Porto\textunderscore , 9.
\section{Cloaqueiro}
\begin{itemize}
\item {Grp. gram.:m.}
\end{itemize}
O mesmo que \textunderscore cloacário\textunderscore .
\section{Cloasma}
\begin{itemize}
\item {Grp. gram.:m.}
\end{itemize}
\begin{itemize}
\item {Proveniência:(Gr. \textunderscore khloasma\textunderscore )}
\end{itemize}
Mancha na pele, em resultado de doença hepática.
\section{Cloasonado}
\begin{itemize}
\item {Grp. gram.:adj.}
\end{itemize}
\begin{itemize}
\item {Proveniência:(Fr. \textunderscore cloisonné\textunderscore )}
\end{itemize}
Diz-se do esmalte, em que um aro metállico divide as côres e os tons.
\section{Clipeásteros}
\begin{itemize}
\item {Grp. gram.:m. pl.}
\end{itemize}
\begin{itemize}
\item {Proveniência:(Do lat. \textunderscore clypeus\textunderscore )}
\end{itemize}
Gênero de insectos coleópteros tetrâmeros.
\section{Clipeiforme}
\begin{itemize}
\item {Grp. gram.:adj.}
\end{itemize}
\begin{itemize}
\item {Proveniência:(Do lat. \textunderscore clypeus\textunderscore  + \textunderscore forma\textunderscore )}
\end{itemize}
Que tem fórma de escudo.
\section{Clísmeo}
\begin{itemize}
\item {Grp. gram.:adj.}
\end{itemize}
\begin{itemize}
\item {Utilização:Med.}
\end{itemize}
\begin{itemize}
\item {Proveniência:(Do lat. \textunderscore clysmus\textunderscore )}
\end{itemize}
Relativo a dejecções; fecal.
\section{Clistér}
\begin{itemize}
\item {Grp. gram.:m.}
\end{itemize}
\begin{itemize}
\item {Proveniência:(Gr. \textunderscore kluster\textunderscore )}
\end{itemize}
Injecção de água ou de outro líquido medicamentoso nos intestinos, por meio de seringa ou aparelho análogo.
\section{Clisterização}
\begin{itemize}
\item {Grp. gram.:f.}
\end{itemize}
Acto de clisterizar.
\section{Clisterizar}
\begin{itemize}
\item {Grp. gram.:v. t.}
\end{itemize}
Dar clisteres a.
\section{Clocotó}
\begin{itemize}
\item {Grp. gram.:m.}
\end{itemize}
Árvore santhomense, de raízes medicinaes.
\section{Clompão}
\begin{itemize}
\item {Grp. gram.:m.}
\end{itemize}
Arbusto sarmentoso da fam. das leguminosas.
\section{Clónico}
\begin{itemize}
\item {Grp. gram.:adj.}
\end{itemize}
\begin{itemize}
\item {Utilização:Med.}
\end{itemize}
\begin{itemize}
\item {Proveniência:(Do gr. \textunderscore klonos\textunderscore )}
\end{itemize}
Diz-se do espasmo ou das contracções espasmódicas, em que há movimentos irregulares e involuntários.
\section{Clonismo}
\begin{itemize}
\item {Grp. gram.:m.}
\end{itemize}
\begin{itemize}
\item {Utilização:Med.}
\end{itemize}
\begin{itemize}
\item {Proveniência:(Do gr. \textunderscore klonos\textunderscore )}
\end{itemize}
Contracções espasmódicas nas epilepsias hystéricas.
\section{Clopemania}
\begin{itemize}
\item {Grp. gram.:f.}
\end{itemize}
\begin{itemize}
\item {Proveniência:(Do gr. \textunderscore klope\textunderscore  + \textunderscore mania\textunderscore )}
\end{itemize}
Tendência irresistível para o roubo.
\section{Cloportídeos}
\begin{itemize}
\item {Grp. gram.:m. pl.}
\end{itemize}
\begin{itemize}
\item {Proveniência:(Do fr. \textunderscore cloporte\textunderscore  + gr. \textunderscore eidos\textunderscore )}
\end{itemize}
Família de crustáceos, a que pertence o bicho-de-conta.
\section{Cloquaire}
\begin{itemize}
\item {Grp. gram.:m.}
\end{itemize}
\begin{itemize}
\item {Utilização:Obsol.}
\end{itemize}
Colhér.
(Metáth. de \textunderscore cochlear\textunderscore , do lat. \textunderscore cochlea\textunderscore )
\section{Cloques}
\begin{itemize}
\item {Grp. gram.:m. pl.}
\end{itemize}
\begin{itemize}
\item {Utilização:Prov.}
\end{itemize}
\begin{itemize}
\item {Utilização:alg.}
\end{itemize}
O mesmo que \textunderscore chalocas\textunderscore .
\section{Clora}
\begin{itemize}
\item {Grp. gram.:f.}
\end{itemize}
\begin{itemize}
\item {Proveniência:(Do gr. \textunderscore khloros\textunderscore , verde amarelado)}
\end{itemize}
Gênero de plantas gencianáceas.
\section{Clorácido}
\begin{itemize}
\item {Grp. gram.:m.}
\end{itemize}
\begin{itemize}
\item {Proveniência:(De \textunderscore chloro\textunderscore  + \textunderscore ácido\textunderscore )}
\end{itemize}
Acido, em que o chloro faz o papel de princípio acidificante.
\section{Cloral}
\begin{itemize}
\item {Grp. gram.:m.}
\end{itemize}
\begin{itemize}
\item {Proveniência:(De \textunderscore chloro\textunderscore  + \textunderscore álcool\textunderscore )}
\end{itemize}
Mistura de cloro e álcool.
\section{Cloralamida}
\begin{itemize}
\item {Grp. gram.:f.}
\end{itemize}
Medicamento hipnótico.
\section{Cloralose}
\begin{itemize}
\item {Grp. gram.:f.}
\end{itemize}
Medicamento, que provoca o sono.
\section{Clorantia}
\begin{itemize}
\item {Grp. gram.:f.}
\end{itemize}
\begin{itemize}
\item {Proveniência:(De \textunderscore chlorantho\textunderscore )}
\end{itemize}
Degenerescência vegetal, em que os órgãos floraes apresentam a côr, a consistência e, ás vezes, a fórma das fôlhas.
\section{Cloranto}
\begin{itemize}
\item {Grp. gram.:adj.}
\end{itemize}
\begin{itemize}
\item {Proveniência:(Do gr. \textunderscore khloros\textunderscore  + \textunderscore anthos\textunderscore )}
\end{itemize}
Que tem côr verde.
Que tem clorantia.
\section{Clorato}
\begin{itemize}
\item {Grp. gram.:m.}
\end{itemize}
\begin{itemize}
\item {Proveniência:(De \textunderscore chloro\textunderscore )}
\end{itemize}
Combinação do ácido clórico com uma base.
\section{Cloretado}
\begin{itemize}
\item {Grp. gram.:adj.}
\end{itemize}
Que tem cloro ou cloreto.
\section{Clorete}
\begin{itemize}
\item {Grp. gram.:m.}
\end{itemize}
\begin{itemize}
\item {Utilização:Pop.}
\end{itemize}
O mesmo que \textunderscore cloreto\textunderscore .
\section{Cloretílico}
\begin{itemize}
\item {Grp. gram.:adj.}
\end{itemize}
Relativo ao cloretilo.
\section{Cloretilização}
\begin{itemize}
\item {Grp. gram.:f.}
\end{itemize}
Acto de cloretilizar.
\section{Cloretilizar}
\begin{itemize}
\item {Grp. gram.:v. t.}
\end{itemize}
Anestesiar pelo cloretilo.
\section{Cloretilo}
\begin{itemize}
\item {Grp. gram.:m.}
\end{itemize}
\begin{itemize}
\item {Utilização:Chím.}
\end{itemize}
Cloreto de etila.
\section{Cloreto}
\begin{itemize}
\item {fónica:clorê}
\end{itemize}
\begin{itemize}
\item {Grp. gram.:m.}
\end{itemize}
\begin{itemize}
\item {Proveniência:(De \textunderscore chloro\textunderscore )}
\end{itemize}
Combinação do cloro com um corpo simples, exceptuado o oxigênio e hidrogênio.
\section{Clori}
\begin{itemize}
\item {Grp. gram.:f.}
\end{itemize}
\begin{itemize}
\item {Utilização:ant.}
\end{itemize}
\begin{itemize}
\item {Utilização:Gír.}
\end{itemize}
Amante.
Meretriz.
\section{Cloribase}
\begin{itemize}
\item {Grp. gram.:f.}
\end{itemize}
\begin{itemize}
\item {Utilização:Chím.}
\end{itemize}
\begin{itemize}
\item {Proveniência:(De \textunderscore chloro\textunderscore  + \textunderscore base\textunderscore )}
\end{itemize}
Composto binário de cloro, que opéra como uma base.
\section{Clórico}
\begin{itemize}
\item {Grp. gram.:adj.}
\end{itemize}
Relativo ao cloro.
\section{Clórido}
\begin{itemize}
\item {Grp. gram.:m.}
\end{itemize}
\begin{itemize}
\item {Utilização:Fam.}
\end{itemize}
\begin{itemize}
\item {Proveniência:(De \textunderscore chloro\textunderscore )}
\end{itemize}
Combinação electronegativa de cloro com corpos metálicos ou metaloides.
Combinação de corpos simples, em que entra o cloro.
\section{Cloridrato}
\begin{itemize}
\item {Grp. gram.:m.}
\end{itemize}
\begin{itemize}
\item {Proveniência:(De \textunderscore chlorhýdrico\textunderscore )}
\end{itemize}
Sal, formado pela combinação do ácido clorídrico com uma base.
\section{Clorídrico}
\begin{itemize}
\item {Grp. gram.:adj.}
\end{itemize}
\begin{itemize}
\item {Proveniência:(De \textunderscore chloro\textunderscore  e \textunderscore hydrogênio\textunderscore )}
\end{itemize}
Diz-se do ácido, composto de volumes iguaes de hidrogênio e de cloro.
\section{Clorino}
\begin{itemize}
\item {Grp. gram.:m.}
\end{itemize}
Mineral haloide, a que pertence o sal-gema.
\section{Clórion}
\begin{itemize}
\item {Grp. gram.:m.}
\end{itemize}
\begin{itemize}
\item {Utilização:Med.}
\end{itemize}
\begin{itemize}
\item {Proveniência:(Do gr. \textunderscore klorion\textunderscore , coiro, pelle)}
\end{itemize}
Membrana villosa, que envolve externamente o óvulo e o liga á parede do útero.
\section{Clorístico}
\begin{itemize}
\item {Grp. gram.:adj.}
\end{itemize}
Relativo ao cloro.
\section{Clorite}
\begin{itemize}
\item {Grp. gram.:f.}
\end{itemize}
\begin{itemize}
\item {Proveniência:(Do gr. \textunderscore khloros\textunderscore , verde)}
\end{itemize}
Mineral, de côr geralmente verde, e análogo á mica.
\section{Clorito}
\begin{itemize}
\item {Grp. gram.:m.}
\end{itemize}
\begin{itemize}
\item {Proveniência:(De \textunderscore chloro\textunderscore )}
\end{itemize}
Sal, formado pela combinação do ácido cloroso com uma base.
\section{Cloritoso}
\begin{itemize}
\item {Grp. gram.:adj.}
\end{itemize}
Que tem clorito.
\section{Cloro}
\begin{itemize}
\item {Grp. gram.:m.}
\end{itemize}
\begin{itemize}
\item {Proveniência:(Gr. \textunderscore khloros\textunderscore , esverdeado)}
\end{itemize}
Corpo simples gasoso, de um sabor cáustico e cheiro activo.
\section{Clorodina}
\begin{itemize}
\item {Grp. gram.:f.}
\end{itemize}
Medicamento antineurálgico.
\section{Clorófana}
\begin{itemize}
\item {Grp. gram.:f.}
\end{itemize}
\begin{itemize}
\item {Proveniência:(De \textunderscore chloróphano\textunderscore )}
\end{itemize}
Variedade de fluorina da Sibéria, de côr violácea, e que, depois de aquecida, se torna fosforescente, de uma bela côr verde.
\section{Clorófano}
\begin{itemize}
\item {Grp. gram.:adj.}
\end{itemize}
\begin{itemize}
\item {Grp. gram.:M.}
\end{itemize}
\begin{itemize}
\item {Proveniência:(Do gr. \textunderscore khloros\textunderscore  + \textunderscore phainein\textunderscore )}
\end{itemize}
Que parece verde.
Gênero de coleópteros de suave côr verde.
\section{Clorofila}
\begin{itemize}
\item {Grp. gram.:f.}
\end{itemize}
\begin{itemize}
\item {Proveniência:(Do gr. \textunderscore khloros\textunderscore  + \textunderscore phullon\textunderscore )}
\end{itemize}
Substância, que existe nas células vegetaes e que dá ás plantas a côr verde.
\section{Clorofileano}
\begin{itemize}
\item {Grp. gram.:adj.}
\end{itemize}
Relativo á clorófila.
\section{Clorofilino}
\begin{itemize}
\item {Grp. gram.:adj.}
\end{itemize}
Relativo á clorófila.
\section{Clorófito}
\begin{itemize}
\item {Grp. gram.:m.}
\end{itemize}
\begin{itemize}
\item {Proveniência:(Do gr. \textunderscore khloros\textunderscore  + \textunderscore phuton\textunderscore )}
\end{itemize}
Gênero de plantas liliáceas.
\section{Clorofórmico}
\begin{itemize}
\item {Grp. gram.:adj.}
\end{itemize}
Relativo ao clorofórmio.
\section{Clorofórmio}
\begin{itemize}
\item {Grp. gram.:m.}
\end{itemize}
Substância líquida, incolor e aromática, de propriedades anestésicas.
(Contr. de \textunderscore chlorofórmico\textunderscore )
\section{Cloroformização}
\begin{itemize}
\item {Grp. gram.:f.}
\end{itemize}
Acto de cloroformizar.
\section{Cloroformizar}
\begin{itemize}
\item {Grp. gram.:v. t.}
\end{itemize}
Ministrar clorofórmio a.
Anestesiar com clorofórmio.
\section{Clorolina}
\begin{itemize}
\item {Grp. gram.:f.}
\end{itemize}
Líquido anti-séptico e desinfectante.
\section{Clorometria}
\begin{itemize}
\item {Grp. gram.:f.}
\end{itemize}
\begin{itemize}
\item {Proveniência:(De \textunderscore chlorómetro\textunderscore )}
\end{itemize}
Processo químico, para determinar a quantidade de cloro, contida numa combinação.
\section{Clorómetro}
\begin{itemize}
\item {Grp. gram.:m.}
\end{itemize}
\begin{itemize}
\item {Proveniência:(Do gr. \textunderscore khloros\textunderscore  + \textunderscore metron\textunderscore )}
\end{itemize}
Aparelho, para se praticar a clorometria.
\section{Clorose}
\begin{itemize}
\item {Grp. gram.:m.}
\end{itemize}
\begin{itemize}
\item {Proveniência:(Do gr. \textunderscore khloros\textunderscore )}
\end{itemize}
Doença do sexo feminino, determinada geralmente pela ausência do catamênio, e caracterizada por fraqueza e palidez.
Definhamento das plantas.
\section{Cloroso}
\begin{itemize}
\item {Grp. gram.:adj.}
\end{itemize}
\begin{itemize}
\item {Proveniência:(De \textunderscore chloro\textunderscore )}
\end{itemize}
Diz-se de um ácido, corpo gasoso, solúvel na água, e de cheiro análogo ao do cloro.
\section{Clorótico}
\begin{itemize}
\item {Grp. gram.:adj.}
\end{itemize}
Que padece clorose.
Relativo á clorose.
\section{Clorureto}
\begin{itemize}
\item {fónica:êto}
\end{itemize}
\begin{itemize}
\item {Grp. gram.:m.}
\end{itemize}
(V.cloreto)
\section{Clóstera}
\begin{itemize}
\item {Grp. gram.:f.}
\end{itemize}
Gênero de insectos lepidópteros nocturnos.
\section{Cluantho}
\begin{itemize}
\item {Grp. gram.:m.}
\end{itemize}
\begin{itemize}
\item {Proveniência:(Do gr. \textunderscore kluanthes\textunderscore )}
\end{itemize}
Gênero de plantas verbenáceas.
\section{Cluanto}
\begin{itemize}
\item {Grp. gram.:m.}
\end{itemize}
\begin{itemize}
\item {Proveniência:(Do gr. \textunderscore kluanthes\textunderscore )}
\end{itemize}
Gênero de plantas verbenáceas.
\section{Club}
\begin{itemize}
\item {Grp. gram.:m.}
\end{itemize}
\begin{itemize}
\item {Proveniência:(Ingl. \textunderscore club\textunderscore )}
\end{itemize}
Casa, em que habitualmente se reunem pessôas, para jogar, dançar, conversar, discutir, etc.
Sociedade de pessôas para um fim commum.
Associação política.
\section{Clube}
\begin{itemize}
\item {Grp. gram.:m.}
\end{itemize}
\begin{itemize}
\item {Proveniência:(Ingl. \textunderscore club\textunderscore )}
\end{itemize}
Casa, em que habitualmente se reunem pessôas, para jogar, dançar, conversar, discutir, etc.
Sociedade de pessôas para um fim commum.
Associação política.
\section{Clubista}
\begin{itemize}
\item {Grp. gram.:m.}
\end{itemize}
Membro, frequentador, de um club.
\section{Clundo}
\begin{itemize}
\item {Grp. gram.:m.}
\end{itemize}
Fruto do mulondo.
\section{Cluniacense}
\begin{itemize}
\item {Grp. gram.:adj.}
\end{itemize}
Relativo aos frades benedictinos de Cluny. Cf. Herculano, \textunderscore Hist. de Port.\textunderscore , II, 233.
\section{Clunípedes}
\begin{itemize}
\item {Grp. gram.:m. pl.}
\end{itemize}
\begin{itemize}
\item {Utilização:Ornit.}
\end{itemize}
\begin{itemize}
\item {Proveniência:(Do lat. \textunderscore clunis\textunderscore  + \textunderscore pes\textunderscore )}
\end{itemize}
Aves, que, como os mergulhões, têm os pés atrás do corpo.
\section{Clúpeos}
\begin{itemize}
\item {Grp. gram.:m. pl.}
\end{itemize}
\begin{itemize}
\item {Proveniência:(Do lat. \textunderscore clupea\textunderscore )}
\end{itemize}
Família de peixes, que têm por typo o harenque.
\section{Clúsia}
\begin{itemize}
\item {Grp. gram.:f.}
\end{itemize}
\begin{itemize}
\item {Proveniência:(De \textunderscore Clusius\textunderscore , n. p. lat. de \textunderscore L'Ecluse\textunderscore , prof. de Leyde)}
\end{itemize}
Gênero de plantas intertropicaes, de casca adstringente e vermífuga.
\section{Clusiáceas}
\begin{itemize}
\item {Grp. gram.:f. pl.}
\end{itemize}
\begin{itemize}
\item {Proveniência:(De \textunderscore clúsia\textunderscore )}
\end{itemize}
Família de plantas, que têm por typo a figueira-maldita de San-Domingos.
\section{Cluva}
\begin{itemize}
\item {Grp. gram.:f.}
\end{itemize}
Espécie de corvo marinho da China.
\section{Clypeásteros}
\begin{itemize}
\item {Grp. gram.:m. pl.}
\end{itemize}
\begin{itemize}
\item {Proveniência:(Do lat. \textunderscore clypeus\textunderscore )}
\end{itemize}
Gênero de insectos coleópteros tetrâmeros.
\section{Clypeiforme}
\begin{itemize}
\item {Grp. gram.:adj.}
\end{itemize}
\begin{itemize}
\item {Proveniência:(Do lat. \textunderscore clypeus\textunderscore  + \textunderscore forma\textunderscore )}
\end{itemize}
Que tem fórma de escudo.
\section{Clýsmeo}
\begin{itemize}
\item {Grp. gram.:adj.}
\end{itemize}
\begin{itemize}
\item {Utilização:Med.}
\end{itemize}
\begin{itemize}
\item {Proveniência:(Do lat. \textunderscore clysmus\textunderscore )}
\end{itemize}
Relativo a dejecções; fecal.
\section{Clystér}
\begin{itemize}
\item {Grp. gram.:m.}
\end{itemize}
\begin{itemize}
\item {Proveniência:(Gr. \textunderscore kluster\textunderscore )}
\end{itemize}
Injecção de água ou de outro líquido medicamentoso nos intestinos, por meio de seringa ou aparelho análogo.
\section{Clysterização}
\begin{itemize}
\item {Grp. gram.:f.}
\end{itemize}
Acto de clysterizar.
\section{Clysterizar}
\begin{itemize}
\item {Grp. gram.:v. t.}
\end{itemize}
Dar clysteres a.
\section{Cnêmida}
\begin{itemize}
\item {Grp. gram.:f.}
\end{itemize}
\begin{itemize}
\item {Proveniência:(Gr. \textunderscore knemis\textunderscore )}
\end{itemize}
Bota defensiva, usada pelos soldados gregos.
\section{Cnico}
\begin{itemize}
\item {Grp. gram.:m.}
\end{itemize}
\begin{itemize}
\item {Proveniência:(Gr. \textunderscore knukos\textunderscore )}
\end{itemize}
Gênero de plantas, que abrange uma espécie medicinal chamada cardobento.
\section{Cnídio}
\begin{itemize}
\item {Grp. gram.:f.}
\end{itemize}
\begin{itemize}
\item {Proveniência:(Do gr. \textunderscore knidion\textunderscore )}
\end{itemize}
Gênero de plantas, da fam. das umbellíferas.
\section{Cnidoblasto}
\begin{itemize}
\item {Grp. gram.:m.}
\end{itemize}
\begin{itemize}
\item {Utilização:Zool.}
\end{itemize}
Céllula urticante, na oxoderme dos celenterados.
\section{Cnidose}
\begin{itemize}
\item {Grp. gram.:f.}
\end{itemize}
\begin{itemize}
\item {Proveniência:(Gr. \textunderscore knidosis\textunderscore )}
\end{itemize}
Comichão ardente, semelhante á que produzem as urtigas.
\section{Cnu-cnu}
\begin{itemize}
\item {Grp. gram.:m.}
\end{itemize}
Arbusto medicinal de Moçambique.
\section{Cnute}
\begin{itemize}
\item {Grp. gram.:m.}
\end{itemize}
Azorrague, formado de ramaes de coiro.
(Russo \textunderscore knut\textunderscore )
\section{Cnyco}
\begin{itemize}
\item {Grp. gram.:m.}
\end{itemize}
\begin{itemize}
\item {Proveniência:(Gr. \textunderscore knukos\textunderscore )}
\end{itemize}
Gênero de plantas, que abrange uma espécie medicinal chamada cardobento.
\section{Co}
(contr. pop. e ant. de \textunderscore com\textunderscore  + \textunderscore o\textunderscore )
\section{Co...}
\begin{itemize}
\item {Grp. gram.:pref.}
\end{itemize}
O mesmo que \textunderscore com\textunderscore . Cp. \textunderscore coexistir\textunderscore , \textunderscore coincidir\textunderscore , etc.
\section{Côa}
\begin{itemize}
\item {Grp. gram.:f.}
\end{itemize}
\begin{itemize}
\item {Utilização:Prov.}
\end{itemize}
\begin{itemize}
\item {Utilização:trasm.}
\end{itemize}
Thesoiro público, erário.
Grandes riquezas.
\section{Côa}
\begin{itemize}
\item {Grp. gram.:f.}
\end{itemize}
\begin{itemize}
\item {Utilização:Des.}
\end{itemize}
\begin{itemize}
\item {Utilização:Bras. do N}
\end{itemize}
Acto de coar.
Líquido coado.
Nata, que coalha á superficie do leite tépido ou quente.
\section{Coação}
\begin{itemize}
\item {Grp. gram.:f.}
\end{itemize}
Acto de coar.
\section{Coacção}
\begin{itemize}
\item {Grp. gram.:f.}
\end{itemize}
\begin{itemize}
\item {Proveniência:(Lat. \textunderscore coactio\textunderscore )}
\end{itemize}
Acto de coagir.
Estado de quem está coacto.
\section{Coaccusado}
\begin{itemize}
\item {Grp. gram.:m.}
\end{itemize}
O mesmo que \textunderscore corréu\textunderscore .
\section{Coacervar}
\begin{itemize}
\item {Grp. gram.:v. t.}
\end{itemize}
\begin{itemize}
\item {Utilização:Des.}
\end{itemize}
\begin{itemize}
\item {Proveniência:(Lat. \textunderscore coacervare\textunderscore )}
\end{itemize}
Amontoar.
\section{Coaco}
\begin{itemize}
\item {Grp. gram.:adj.}
\end{itemize}
Relativo á ilha de Cós ou Côa.
\section{Coaco-branco}
\begin{itemize}
\item {Grp. gram.:m.}
\end{itemize}
Árvore da ilha de San-Thomé.
\section{Coacquisição}
\begin{itemize}
\item {Grp. gram.:f.}
\end{itemize}
Acto ou effeito de coadquirir.
\section{Coactar}
\begin{itemize}
\item {Grp. gram.:v. t.}
\end{itemize}
\begin{itemize}
\item {Utilização:Neol.}
\end{itemize}
Tornar coacto, coagir.
\section{Coactivo}
\begin{itemize}
\item {Grp. gram.:adj.}
\end{itemize}
\begin{itemize}
\item {Proveniência:(Do lat. \textunderscore coactus\textunderscore )}
\end{itemize}
Que coage.
\section{Coacto}
\begin{itemize}
\item {Grp. gram.:adj.}
\end{itemize}
\begin{itemize}
\item {Proveniência:(Lat. \textunderscore coactus\textunderscore )}
\end{itemize}
Constrangido.
Obrigado á fôrça.
\section{Coactor}
\begin{itemize}
\item {Grp. gram.:m.}
\end{itemize}
\begin{itemize}
\item {Proveniência:(Lat. \textunderscore coactor\textunderscore )}
\end{itemize}
Antigo recebedor de impostos, entre os Romanos.
\section{Coacusado}
\begin{itemize}
\item {Grp. gram.:m.}
\end{itemize}
O mesmo que \textunderscore corréu\textunderscore .
\section{Coada}
\begin{itemize}
\item {Grp. gram.:f.}
\end{itemize}
\begin{itemize}
\item {Proveniência:(De \textunderscore coado\textunderscore )}
\end{itemize}
Porção de líquido coado.
Barrela.
\section{Côa-das-pichas}
\begin{itemize}
\item {Grp. gram.:f.}
\end{itemize}
Uma das redes usadas pelos pescadores do Mondego.
\section{Coadeira}
\begin{itemize}
\item {Grp. gram.:f.}
\end{itemize}
O mesmo que \textunderscore coador\textunderscore .
\section{Coadeiro}
\begin{itemize}
\item {Grp. gram.:m.}
\end{itemize}
\begin{itemize}
\item {Utilização:Prov.}
\end{itemize}
\begin{itemize}
\item {Utilização:alent.}
\end{itemize}
\begin{itemize}
\item {Proveniência:(De \textunderscore coar\textunderscore ^1)}
\end{itemize}
Pano, por onde se côa o leite, que cai dentro do asado, para ali coalhar sob a influência do cardo e formar depois o queijo. Cf. Ficalho, \textunderscore in\textunderscore  Rev. \textunderscore Tradição\textunderscore , IX, 132.
\section{Coadela}
\begin{itemize}
\item {Grp. gram.:f.}
\end{itemize}
\begin{itemize}
\item {Utilização:Prov.}
\end{itemize}
\begin{itemize}
\item {Utilização:beir.}
\end{itemize}
\begin{itemize}
\item {Proveniência:(De \textunderscore coar\textunderscore ^2)}
\end{itemize}
Apuro, no aplainar de uma tábua.
\section{Coadjutor}
\begin{itemize}
\item {Grp. gram.:m.  e  adj.}
\end{itemize}
\begin{itemize}
\item {Proveniência:(Lat. \textunderscore coadjutor\textunderscore )}
\end{itemize}
O que coadjuva.
Adjunto de um párocho.
\section{Coadjutoria}
\begin{itemize}
\item {Grp. gram.:f.}
\end{itemize}
Serviço de coadjutor.
\section{Coadjuvação}
\begin{itemize}
\item {Grp. gram.:f.}
\end{itemize}
Acto de coadjuvar.
\section{Coadjuvador}
\begin{itemize}
\item {Grp. gram.:m.  e  adj.}
\end{itemize}
O mesmo que \textunderscore coadjuvante\textunderscore .
\section{Coadjuvante}
\begin{itemize}
\item {Grp. gram.:m.  e  adj.}
\end{itemize}
\begin{itemize}
\item {Proveniência:(Lat. \textunderscore coadjuvans\textunderscore )}
\end{itemize}
O que coadjuva.
\section{Coadjuvar}
\begin{itemize}
\item {Grp. gram.:v. t.}
\end{itemize}
\begin{itemize}
\item {Proveniência:(Lat. \textunderscore coadjuvare\textunderscore )}
\end{itemize}
Ajudar.
Trabalhar com.
\section{Coadministração}
\begin{itemize}
\item {Grp. gram.:f.}
\end{itemize}
Acto de coadministrar.
\section{Coadministrador}
\begin{itemize}
\item {Grp. gram.:m.}
\end{itemize}
Aquelle que coadministra.
\section{Coadministrar}
\begin{itemize}
\item {Grp. gram.:v. t.}
\end{itemize}
\begin{itemize}
\item {Proveniência:(De \textunderscore co...\textunderscore  + \textunderscore administrar\textunderscore )}
\end{itemize}
Administrar, juntamente com outrem.
\section{Coado}
\begin{itemize}
\item {Grp. gram.:adj.}
\end{itemize}
\begin{itemize}
\item {Utilização:Prov.}
\end{itemize}
\begin{itemize}
\item {Proveniência:(De \textunderscore coar\textunderscore )}
\end{itemize}
Que passou por filtro; filtrado: \textunderscore água coada\textunderscore .
Lívido, de susto, sem pinga de sangue:«\textunderscore aqui vy com o ferro vossa figura coada\textunderscore ». Usque, \textunderscore Tribulações\textunderscore , 50, v.^o
Diz-se do toiro, que soffreu castração.
Diz-se do pão, fabricado de farinhas, muito apuradas, de trigo, milho, centeio e cevada. (Colhido em Arganil)
\section{Coadoiro}
\begin{itemize}
\item {Grp. gram.:m.  e  adj.}
\end{itemize}
\begin{itemize}
\item {Utilização:Prov.}
\end{itemize}
\begin{itemize}
\item {Utilização:minh.}
\end{itemize}
O mesmo que \textunderscore coador\textunderscore .
Pano grosseiro, que serve para coar a lixívia.
\section{Coador}
\begin{itemize}
\item {Grp. gram.:m.  e  adj.}
\end{itemize}
O que côa, ou serve para coar.
\section{Coadouro}
\begin{itemize}
\item {Grp. gram.:m.  e  adj.}
\end{itemize}
\begin{itemize}
\item {Utilização:Prov.}
\end{itemize}
\begin{itemize}
\item {Utilização:minh.}
\end{itemize}
O mesmo que \textunderscore coador\textunderscore .
Pano grosseiro, que serve para coar a lixívia.
\section{Coadquirição}
\begin{itemize}
\item {Grp. gram.:f.}
\end{itemize}
O mesmo que \textunderscore coacquisição\textunderscore .
\section{Coadquirir}
\begin{itemize}
\item {Grp. gram.:v. t.}
\end{itemize}
\begin{itemize}
\item {Proveniência:(De \textunderscore co...\textunderscore  + \textunderscore adquirir\textunderscore )}
\end{itemize}
Adquirir em commum.
\section{Coadunação}
\begin{itemize}
\item {Grp. gram.:f.}
\end{itemize}
\begin{itemize}
\item {Proveniência:(Lat. \textunderscore coadunatio\textunderscore )}
\end{itemize}
Acto de coadunar.
\section{Coadunar}
\begin{itemize}
\item {Grp. gram.:v. t.}
\end{itemize}
\begin{itemize}
\item {Proveniência:(Lat. \textunderscore coadunare\textunderscore )}
\end{itemize}
Juntar em um.
Ligar.
Combinar; harmonizar.
\section{Coadunável}
\begin{itemize}
\item {Grp. gram.:adj.}
\end{itemize}
Que se póde coadunar.
\section{Coadura}
\begin{itemize}
\item {Grp. gram.:f.}
\end{itemize}
O mesmo que \textunderscore coada\textunderscore .
\section{Coagente}
\begin{itemize}
\item {Grp. gram.:adj.}
\end{itemize}
Que coage, que obriga.
\section{Coagir}
\begin{itemize}
\item {Grp. gram.:v. t.}
\end{itemize}
\begin{itemize}
\item {Proveniência:(Lat. \textunderscore cogere\textunderscore , contr. de \textunderscore co-agere\textunderscore )}
\end{itemize}
Constranger, forçar.
\section{Coagmentação}
\begin{itemize}
\item {Grp. gram.:f.}
\end{itemize}
O mesmo que \textunderscore coagmento\textunderscore .
\section{Coagmentar}
\begin{itemize}
\item {Grp. gram.:v. t.}
\end{itemize}
\begin{itemize}
\item {Utilização:Des.}
\end{itemize}
\begin{itemize}
\item {Proveniência:(Lat. \textunderscore coagmentare\textunderscore )}
\end{itemize}
Amassar, ligar.
\section{Coagmento}
\begin{itemize}
\item {Grp. gram.:m.}
\end{itemize}
\begin{itemize}
\item {Proveniência:(Lat. \textunderscore coagmentum\textunderscore )}
\end{itemize}
Acção de coagmentar.
\section{Coagulação}
\begin{itemize}
\item {Grp. gram.:f.}
\end{itemize}
Acto de coagular.
\section{Coagulador}
\begin{itemize}
\item {Grp. gram.:adj.}
\end{itemize}
\begin{itemize}
\item {Grp. gram.:M.}
\end{itemize}
\begin{itemize}
\item {Proveniência:(De \textunderscore coagular\textunderscore )}
\end{itemize}
Que produz coagulação.
Última cavidade do estômago dos ruminantes.
\section{Coagulante}
\begin{itemize}
\item {Grp. gram.:adj.}
\end{itemize}
Que coagula.
\section{Coagular}
\begin{itemize}
\item {Grp. gram.:v. t.}
\end{itemize}
\begin{itemize}
\item {Proveniência:(Lat. \textunderscore coagulare\textunderscore )}
\end{itemize}
(Fórma erudita ou artificial, em vez de \textunderscore coalhar\textunderscore . V. \textunderscore coalhar\textunderscore )
\section{Coagulável}
\begin{itemize}
\item {Grp. gram.:adj.}
\end{itemize}
Que se póde coagular.
\section{Coágulo}
\begin{itemize}
\item {Grp. gram.:m.}
\end{itemize}
\begin{itemize}
\item {Proveniência:(Lat. \textunderscore coagulum\textunderscore )}
\end{itemize}
Effeito de coagular.
Parte coagulada de um líquido.
Substância que coagula.
\section{Coaitá}
\begin{itemize}
\item {Grp. gram.:m.}
\end{itemize}
Espécie de macaco.
\section{Coajinguva}
\begin{itemize}
\item {Grp. gram.:f.}
\end{itemize}
\begin{itemize}
\item {Utilização:Bras}
\end{itemize}
Planta urticácea, de suco vermifugo.
\section{Coajuba}
\begin{itemize}
\item {Grp. gram.:f.}
\end{itemize}
\begin{itemize}
\item {Utilização:Bras}
\end{itemize}
Espécie de abelha das regiões do Amazonas.
\section{Coalescência}
\begin{itemize}
\item {Grp. gram.:f.}
\end{itemize}
\begin{itemize}
\item {Proveniência:(De \textunderscore coalescente\textunderscore )}
\end{itemize}
Juncção de partes, que se achavam separadas.
Agglutinação.
\section{Coalescente}
\begin{itemize}
\item {Grp. gram.:adj.}
\end{itemize}
\begin{itemize}
\item {Proveniência:(Lat. \textunderscore coalescens\textunderscore )}
\end{itemize}
Unido, adherente.
Agglutinante.
\section{Coalescer}
\begin{itemize}
\item {Grp. gram.:v. t.}
\end{itemize}
\begin{itemize}
\item {Proveniência:(Lat. \textunderscore coalescere\textunderscore )}
\end{itemize}
Juntar, unir.
\section{Coalhada}
\begin{itemize}
\item {Grp. gram.:f.}
\end{itemize}
Leite coalhado.
\section{Coalhado}
\begin{itemize}
\item {Grp. gram.:adj.}
\end{itemize}
\begin{itemize}
\item {Proveniência:(De \textunderscore coalhar\textunderscore )}
\end{itemize}
Solidificado: \textunderscore azeite coalhado\textunderscore .
\section{Coalhadura}
\begin{itemize}
\item {Grp. gram.:f.}
\end{itemize}
Effeito de coalhar.
\section{Coalha-leite}
\begin{itemize}
\item {Grp. gram.:f.}
\end{itemize}
\begin{itemize}
\item {Utilização:Bras}
\end{itemize}
Planta Synanthérea alimentar, espécie de cardo, (\textunderscore cynara cardunculus\textunderscore , Lin.).
\section{Coalhamento}
\begin{itemize}
\item {Grp. gram.:m.}
\end{itemize}
O mesmo que \textunderscore coalhadura\textunderscore .
\section{Coalhar}
\begin{itemize}
\item {Grp. gram.:v. t.}
\end{itemize}
\begin{itemize}
\item {Utilização:Fig.}
\end{itemize}
\begin{itemize}
\item {Grp. gram.:V. i.  e  p.}
\end{itemize}
\begin{itemize}
\item {Proveniência:(Do lat. \textunderscore coagulare\textunderscore )}
\end{itemize}
Solidificar.
Encher; obstruir.
Solidificar-se (um líquido, como leite, azeite, etc.).
\section{Coalheira}
\begin{itemize}
\item {Grp. gram.:f.}
\end{itemize}
\begin{itemize}
\item {Proveniência:(De \textunderscore coalhar\textunderscore )}
\end{itemize}
Víscera de animaes, que se emprega nas queijarias, para coalhar o leite.
O mesmo que \textunderscore coagulador\textunderscore , última cavidade do estômago dos bovídeos, lanígeros e caprídeos.
\section{Coalheira}
\begin{itemize}
\item {Grp. gram.:f.}
\end{itemize}
\begin{itemize}
\item {Utilização:Prov.}
\end{itemize}
\begin{itemize}
\item {Utilização:alent.}
\end{itemize}
(V. \textunderscore coelheira\textunderscore ^2)
\section{Coalho}
\begin{itemize}
\item {Grp. gram.:m.}
\end{itemize}
\begin{itemize}
\item {Utilização:Fig.}
\end{itemize}
O mesmo que \textunderscore coágulo\textunderscore .
O mesmo que \textunderscore coalheira\textunderscore ^1.
A flôr do cardo.
Solidificação de um líquido.
Dinheiro, que se economizou. Cf. Filinto, IX, 62 e 63.
\section{Coalizão}
\begin{itemize}
\item {Grp. gram.:f.}
\end{itemize}
Acôrdo político, para um fim commum.
(Por \textunderscore coalição\textunderscore , do lat. hyp. \textunderscore coalitio\textunderscore , de \textunderscore coalescere\textunderscore )
\section{Coandu}
\begin{itemize}
\item {Grp. gram.:m.}
\end{itemize}
Animal roedor do Brasil.
\section{Coanha}
\begin{itemize}
\item {Grp. gram.:f.}
\end{itemize}
\begin{itemize}
\item {Proveniência:(De \textunderscore coanhar\textunderscore )}
\end{itemize}
Espécie de vassoira, com que nas eiras se separa dos grãos o palhiço ou rabeiras.
\section{Coanhar}
\begin{itemize}
\item {Grp. gram.:v. t.}
\end{itemize}
\begin{itemize}
\item {Proveniência:(De \textunderscore coar\textunderscore ? Cp. \textunderscore còinar\textunderscore )}
\end{itemize}
Separar (dos grãos) o palhiço ou as rabeiras, na eira.
\section{Coanhe}
\begin{itemize}
\item {Grp. gram.:m.}
\end{itemize}
Árvore africana, de frutos semelhantes ás cerejas.
\section{Coanhos}
\begin{itemize}
\item {Grp. gram.:m. pl.}
\end{itemize}
\begin{itemize}
\item {Utilização:Prov.}
\end{itemize}
\begin{itemize}
\item {Utilização:trasm.}
\end{itemize}
\begin{itemize}
\item {Proveniência:(De \textunderscore coanhar\textunderscore )}
\end{itemize}
Palhiço ou rabeiras, que ficam misturadas no centeio, quando se malha.
\section{Coanoide}
\begin{itemize}
\item {Grp. gram.:adj.}
\end{itemize}
\begin{itemize}
\item {Proveniência:(Do gr. \textunderscore khoane\textunderscore  + \textunderscore eidos\textunderscore )}
\end{itemize}
Que tem fórma de funil.
\section{Coapóstolo}
\begin{itemize}
\item {Grp. gram.:m.}
\end{itemize}
\begin{itemize}
\item {Proveniência:(De \textunderscore co...\textunderscore  + \textunderscore apóstolo\textunderscore )}
\end{itemize}
O que apostola juntamente com outrem:«\textunderscore disse S. Pedro que nas cartas deste seu coapóstolo havia algumas coisas...\textunderscore »\textunderscore Luz e Calor\textunderscore , 166.
\section{Coaptação}
\begin{itemize}
\item {Grp. gram.:f.}
\end{itemize}
\begin{itemize}
\item {Utilização:Cir.}
\end{itemize}
\begin{itemize}
\item {Proveniência:(Lat. \textunderscore coaptatio\textunderscore )}
\end{itemize}
Acto de adaptar reciprocamente as extremidades de ossos fracturados.
\section{Coar}
\begin{itemize}
\item {Grp. gram.:v. t.}
\end{itemize}
\begin{itemize}
\item {Grp. gram.:V. i.}
\end{itemize}
\begin{itemize}
\item {Proveniência:(Lat. \textunderscore colare\textunderscore )}
\end{itemize}
Fazer passar por filtro; filtrar.
Passar através de fendas, de orifícios.
Fundir.
Entrar suavemente, a pouco e pouco.
\section{Coarctação}
\begin{itemize}
\item {Grp. gram.:f.}
\end{itemize}
Acto de coarctar.
\section{Coarctada}
\begin{itemize}
\item {Grp. gram.:f.}
\end{itemize}
\begin{itemize}
\item {Proveniência:(De \textunderscore coarctar\textunderscore )}
\end{itemize}
Coarctação.
Resposta categórica, réplica convincente.
\section{Coarctado}
\begin{itemize}
\item {Grp. gram.:adj.}
\end{itemize}
\begin{itemize}
\item {Proveniência:(De \textunderscore coarctar\textunderscore )}
\end{itemize}
Restringido, circunscrito.
\section{Coarctar}
\begin{itemize}
\item {Grp. gram.:v. t.}
\end{itemize}
\begin{itemize}
\item {Proveniência:(Lat. \textunderscore coarctare\textunderscore )}
\end{itemize}
Circunscrever estreitamente; restringir: \textunderscore coarctar os direitos de alguém\textunderscore .
\section{Coarcto}
\begin{itemize}
\item {Grp. gram.:adj.}
\end{itemize}
O mesmo que \textunderscore coarctado\textunderscore .
\section{Coarrendador}
\begin{itemize}
\item {Grp. gram.:m.}
\end{itemize}
Aquelle que coarrenda.
\section{Coarrendar}
\begin{itemize}
\item {Grp. gram.:v. t.}
\end{itemize}
\begin{itemize}
\item {Proveniência:(De \textunderscore co...\textunderscore  + \textunderscore arrendar\textunderscore )}
\end{itemize}
Arrendar juntamente com outrem.
\section{Coatá}
\begin{itemize}
\item {Grp. gram.:m.}
\end{itemize}
O mesmo que \textunderscore coaitá\textunderscore .
\section{Coati}
\begin{itemize}
\item {Grp. gram.:m.}
\end{itemize}
Mammífero carniceiro da América.
\section{Coatiara}
\begin{itemize}
\item {Grp. gram.:f.}
\end{itemize}
(V.cotiara)
\section{Coautor}
\begin{itemize}
\item {Grp. gram.:m.}
\end{itemize}
\begin{itemize}
\item {Proveniência:(De \textunderscore co...\textunderscore  + \textunderscore autor\textunderscore )}
\end{itemize}
Aquelle que com outrem produz qualquer coisa, especialmente obra literária.
Aquelle que com outrem demanda alguém em juízo.
\section{Coautoria}
\begin{itemize}
\item {Grp. gram.:f.}
\end{itemize}
Estado ou qualidade de coautor.
\section{Coaxação}
\begin{itemize}
\item {Grp. gram.:f.}
\end{itemize}
O mesmo que \textunderscore coaxo\textunderscore .
\section{Coaxante}
\begin{itemize}
\item {Grp. gram.:adj.}
\end{itemize}
Que coaxa.
\section{Coaxar}
\begin{itemize}
\item {Grp. gram.:v. i.}
\end{itemize}
Gritar (a ran).
(B. lat. \textunderscore coaxare\textunderscore )
\section{Coaxo}
\begin{itemize}
\item {Grp. gram.:m.}
\end{itemize}
A voz das rans.
Acto de coaxar.
\section{Çoba!}
\begin{itemize}
\item {Grp. gram.:interj.}
\end{itemize}
\begin{itemize}
\item {Utilização:Prov.}
\end{itemize}
\begin{itemize}
\item {Utilização:trasm.}
\end{itemize}
(para açular cães)
\section{Cobaia}
\begin{itemize}
\item {Grp. gram.:f.}
\end{itemize}
\begin{itemize}
\item {Proveniência:(Fr. \textunderscore cobaye\textunderscore )}
\end{itemize}
Pequeno mammífero, vulgarmente conhecido por \textunderscore porquinho da Índia\textunderscore , (\textunderscore cavia cobaya\textunderscore ).
\section{Cobaio}
\begin{itemize}
\item {Grp. gram.:m.}
\end{itemize}
\begin{itemize}
\item {Proveniência:(Fr. \textunderscore cobaye\textunderscore )}
\end{itemize}
Pequeno mammífero, vulgarmente conhecido por \textunderscore porquinho da Índia\textunderscore , (\textunderscore cavia cobaya\textunderscore ).
\section{Cobáltico}
\begin{itemize}
\item {Grp. gram.:adj.}
\end{itemize}
Relativo ao cobalto.
\section{Cobaltífero}
\begin{itemize}
\item {Grp. gram.:adj.}
\end{itemize}
\begin{itemize}
\item {Proveniência:(De \textunderscore cobalto\textunderscore  + lat. \textunderscore ferre\textunderscore )}
\end{itemize}
Em que há cobalto.
\section{Cobaltizado}
\begin{itemize}
\item {Grp. gram.:adj.}
\end{itemize}
Que tem côr de cobalto.
\section{Cobaltizagem}
\begin{itemize}
\item {Grp. gram.:f.}
\end{itemize}
Acto ou effeito de cobaltizar.
\section{Cobaltizar}
\begin{itemize}
\item {Grp. gram.:v. t.}
\end{itemize}
Dar côr de cobalto a.
\section{Cobalto}
\begin{itemize}
\item {Grp. gram.:m.}
\end{itemize}
\begin{itemize}
\item {Proveniência:(Al. \textunderscore kobalt\textunderscore )}
\end{itemize}
Metal arroxado, pouco fusível.
\section{Çobar}
\begin{itemize}
\item {Grp. gram.:v. t.}
\end{itemize}
\begin{itemize}
\item {Utilização:Prov.}
\end{itemize}
\begin{itemize}
\item {Utilização:trasm.}
\end{itemize}
\begin{itemize}
\item {Proveniência:(De \textunderscore çoba\textunderscore )}
\end{itemize}
Açular (cães).
\section{Cobarde}
\begin{itemize}
\item {Grp. gram.:m.  e  adj.}
\end{itemize}
Pessôa medrosa; pusillânime.
Indivíduo tímido.
Traiçoeiro.
(Cast. \textunderscore cobarde\textunderscore )
\section{Cobardemente}
\begin{itemize}
\item {Grp. gram.:adv.}
\end{itemize}
De modo cobarde.
Com cobardia.
\section{Cobardia}
\begin{itemize}
\item {Grp. gram.:f.}
\end{itemize}
\begin{itemize}
\item {Proveniência:(De \textunderscore cobarde\textunderscore )}
\end{itemize}
Pusillanimidade; timidez.
Animo traiçoeiro.
Acção, que denota medo e perversidade.
\section{Cobardice}
\begin{itemize}
\item {Grp. gram.:f.}
\end{itemize}
\begin{itemize}
\item {Utilização:Ant.}
\end{itemize}
O mesmo que \textunderscore cobardia\textunderscore .
\section{Cobéa}
\begin{itemize}
\item {Grp. gram.:f.}
\end{itemize}
Planta trepadeira da América tropical, (\textunderscore cobaea escandens\textunderscore ).
\section{Cobeia}
\begin{itemize}
\item {Grp. gram.:f.}
\end{itemize}
Planta trepadeira da América tropical, (\textunderscore cobaea escandens\textunderscore ).
\section{Cobela}
\begin{itemize}
\item {Grp. gram.:f.}
\end{itemize}
\begin{itemize}
\item {Utilização:Prov.}
\end{itemize}
\begin{itemize}
\item {Utilização:dur.}
\end{itemize}
Insecto coleóptero heterómero, espécie de carocha, semelhante á vaca-loira, mas sem o desenvolvimento da cornamenta desta.
\section{Coberta}
\begin{itemize}
\item {Grp. gram.:f.}
\end{itemize}
\begin{itemize}
\item {Utilização:Constr.}
\end{itemize}
\begin{itemize}
\item {Utilização:Fig.}
\end{itemize}
\begin{itemize}
\item {Utilização:Serralh.}
\end{itemize}
\begin{itemize}
\item {Utilização:Bras}
\end{itemize}
\begin{itemize}
\item {Grp. gram.:Adj. f.}
\end{itemize}
\begin{itemize}
\item {Utilização:Prov.}
\end{itemize}
\begin{itemize}
\item {Proveniência:(De \textunderscore coberto\textunderscore )}
\end{itemize}
Estôfo ou objecto, que cobre alguma coisa; cobertor.
O tampo de degrau, numa escada.
Conjunto de iguarias, que se servem ao mesmo tempo.
Pavimento do navio.
Abrigo, protecção.
Peça que cobre a fechadura.
Embarcação de toldos de madeira.
O mesmo que \textunderscore prenhe\textunderscore , (falando-se de animaes).
\section{Cobertal}
\begin{itemize}
\item {Grp. gram.:m.}
\end{itemize}
\begin{itemize}
\item {Utilização:Ant.}
\end{itemize}
O mesmo que \textunderscore cobertor\textunderscore .
\section{Cobertamente}
\begin{itemize}
\item {Grp. gram.:adv.}
\end{itemize}
De modo coberto; occultamente.
\section{Coberteiras}
\begin{itemize}
\item {Grp. gram.:f. pl.}
\end{itemize}
\begin{itemize}
\item {Utilização:Prov.}
\end{itemize}
\begin{itemize}
\item {Utilização:trasm.}
\end{itemize}
\begin{itemize}
\item {Proveniência:(De \textunderscore coberto\textunderscore )}
\end{itemize}
Pennas que, na cauda do falcão, cobrem as chamadas pennas reaes.
Pelle, de cão ordinariamente, que cobre a fronte dos bois, quando vão puxando o carro.
\section{Coberto}
\begin{itemize}
\item {Grp. gram.:adj.}
\end{itemize}
\begin{itemize}
\item {Grp. gram.:M.}
\end{itemize}
\begin{itemize}
\item {Proveniência:(Do lat. \textunderscore coopertus\textunderscore )}
\end{itemize}
Tapado com algum objecto; resguardado por alguma coisa.
\textunderscore Abóbora coberta\textunderscore , doce de casca de abóbora branca, cortada em tiras e cozida em calda de açúcar.
\textunderscore Doce coberto\textunderscore , doce, polvilhado de açúcar.
Lugar coberto, alpendre, telheiro.
\section{Cobertor}
\begin{itemize}
\item {Grp. gram.:m.}
\end{itemize}
\begin{itemize}
\item {Utilização:Ant.}
\end{itemize}
\begin{itemize}
\item {Utilização:Carp.}
\end{itemize}
\begin{itemize}
\item {Proveniência:(De \textunderscore coberto\textunderscore )}
\end{itemize}
Peça encorpada e felpuda, de lan ou algodão, que se estende na cama sôbre os lençóes.
Colcha, colgadura.
Qualquer objecto, que cobre.
A peça ou parte superior de um degrau, também chamada \textunderscore coberta\textunderscore .
\section{Cobertura}
\begin{itemize}
\item {Grp. gram.:f.}
\end{itemize}
\begin{itemize}
\item {Proveniência:(De \textunderscore coberto\textunderscore )}
\end{itemize}
Aquillo que cobre.
Acto de cobrir.
Densidade ou corpo (do vinho). Cf. \textunderscore Techn. Rur.\textunderscore , 43.
\section{Cobião}
\begin{itemize}
\item {Grp. gram.:m.}
\end{itemize}
\begin{itemize}
\item {Proveniência:(Lat. \textunderscore cobion\textunderscore )}
\end{itemize}
Planta euphorbiácea, mais conhecida por \textunderscore maleiteira\textunderscore .
\section{Cobiça}
\begin{itemize}
\item {Grp. gram.:f.}
\end{itemize}
\begin{itemize}
\item {Proveniência:(Do lat. hypoth. \textunderscore cupiditia\textunderscore , de \textunderscore cupidus\textunderscore )}
\end{itemize}
\textunderscore f.\textunderscore  (e der.)
O mesmo ou melhor que \textunderscore cubiça\textunderscore , etc.
Desejo forte.
Avidez; ambição.
\section{Cóbio}
\begin{itemize}
\item {Grp. gram.:m.}
\end{itemize}
(V.cobião)
\section{Cobra}
\begin{itemize}
\item {Grp. gram.:f.}
\end{itemize}
\begin{itemize}
\item {Utilização:Fig.}
\end{itemize}
\begin{itemize}
\item {Utilização:Prov.}
\end{itemize}
\begin{itemize}
\item {Utilização:alent.}
\end{itemize}
\begin{itemize}
\item {Utilização:Prov.}
\end{itemize}
\begin{itemize}
\item {Utilização:alent.}
\end{itemize}
\begin{itemize}
\item {Proveniência:(Lat. \textunderscore colubra\textunderscore )}
\end{itemize}
Reptil, da fam. das serpentes.
Serpente, que não é venenosa.
Objecto, que tem fórma semelhante á da cobra.
Pessôa de má índole.
Bolo, em fórma de cobra, feito de farinha, ovos e açúcar, e que se serve com calda.
Espécie de jôgo infantil.
\section{Cobra-capello}
\begin{itemize}
\item {Grp. gram.:f.}
\end{itemize}
Serpente venenosa, o mesmo que \textunderscore naja\textunderscore .
\section{Cobrada}
\begin{itemize}
\item {Grp. gram.:f.}
\end{itemize}
\begin{itemize}
\item {Utilização:Prov.}
\end{itemize}
\begin{itemize}
\item {Utilização:minh.}
\end{itemize}
\begin{itemize}
\item {Proveniência:(De \textunderscore cobrar\textunderscore ?)}
\end{itemize}
Grupo de pescadores, munidos de uma só rede, que alternam os lances com outro grupo.
\section{Cobra-de-vidro}
\begin{itemize}
\item {Grp. gram.:f.}
\end{itemize}
\begin{itemize}
\item {Utilização:Pop.}
\end{itemize}
O mesmo que \textunderscore licranço\textunderscore .
\section{Cobradoiro}
\begin{itemize}
\item {Grp. gram.:m.}
\end{itemize}
\begin{itemize}
\item {Utilização:Prov.}
\end{itemize}
\begin{itemize}
\item {Utilização:trasm.}
\end{itemize}
O mesmo que \textunderscore talhadoiro\textunderscore  (de águas).
(Por \textunderscore quebradoiro\textunderscore , de \textunderscore quebrar\textunderscore )
\section{Cobra-do-mar}
\begin{itemize}
\item {Grp. gram.:f.}
\end{itemize}
Peixe de Portugal.
\section{Cobrador}
\begin{itemize}
\item {Grp. gram.:m.}
\end{itemize}
\begin{itemize}
\item {Grp. gram.:Adj.}
\end{itemize}
\begin{itemize}
\item {Proveniência:(De \textunderscore cobrar\textunderscore )}
\end{itemize}
Aquelle que cobra, que recebe.
Aquelle que faz cobranças.

Que apanha bem a caça ferida, (falando-se de um cão).
\section{Cobradouro}
\begin{itemize}
\item {Grp. gram.:m.}
\end{itemize}
\begin{itemize}
\item {Utilização:Prov.}
\end{itemize}
\begin{itemize}
\item {Utilização:trasm.}
\end{itemize}
O mesmo que \textunderscore talhadouro\textunderscore  (de águas).
(Por \textunderscore quebradoiro\textunderscore , de \textunderscore quebrar\textunderscore )
\section{Cobrança}
\begin{itemize}
\item {Grp. gram.:f.}
\end{itemize}
Acto de cobrar.
\section{Cobrancista}
\begin{itemize}
\item {Grp. gram.:m.  e  adj.}
\end{itemize}
\begin{itemize}
\item {Utilização:Prov.}
\end{itemize}
\begin{itemize}
\item {Utilização:minh.}
\end{itemize}
O que faz cobranças.
Cobrador.
\section{Cobrançosa}
\begin{itemize}
\item {Grp. gram.:adj.}
\end{itemize}
\begin{itemize}
\item {Utilização:Prov.}
\end{itemize}
\begin{itemize}
\item {Utilização:trasm.}
\end{itemize}
Diz-se de uma variedade de azeitona.
\section{Cobranto}
\begin{itemize}
\item {Grp. gram.:m.}
\end{itemize}
\begin{itemize}
\item {Utilização:Pop.}
\end{itemize}
Encantamento, quebranto. Cf. \textunderscore Museu Techn.\textunderscore , 68.
(Alter. de \textunderscore quebranto\textunderscore )
\section{Cobrão}
\begin{itemize}
\item {Grp. gram.:m.}
\end{itemize}
\begin{itemize}
\item {Utilização:Prov.}
\end{itemize}
\begin{itemize}
\item {Utilização:trasm.}
\end{itemize}
O mesmo que \textunderscore cobrelo\textunderscore .
O macho da cobra.
\section{Cobrar}
\begin{itemize}
\item {Grp. gram.:v. t.}
\end{itemize}
\begin{itemize}
\item {Grp. gram.:V. i.}
\end{itemize}
\begin{itemize}
\item {Proveniência:(De \textunderscore cuperare\textunderscore , aphér. do lat. \textunderscore recuperare\textunderscore )}
\end{itemize}
Receber: \textunderscore cobrar impostos\textunderscore .
Readquirir.
Deixar-se possuir de: \textunderscore cobrar ânimo\textunderscore .
\textunderscore Cobrar de ferido\textunderscore , diz-se do cão que vai apanhar a caça ferida.
\section{Cobra-veado}
\begin{itemize}
\item {Grp. gram.:f.}
\end{itemize}
Cobra venenosa do Brasil.
\section{Cobrável}
\begin{itemize}
\item {Grp. gram.:adj.}
\end{itemize}
Que se póde cobrar.
\section{Cobra-verde}
\begin{itemize}
\item {Grp. gram.:f.}
\end{itemize}
Cobra venenosa do Brasil.
\section{Cobre}
\begin{itemize}
\item {Grp. gram.:m.}
\end{itemize}
\begin{itemize}
\item {Utilização:Fig.}
\end{itemize}
\begin{itemize}
\item {Proveniência:(Do lat. \textunderscore cuprum\textunderscore )}
\end{itemize}
Metal avermelhado.
Moédas de cobre: \textunderscore não trago cobre comigo\textunderscore .
Dinheiro. Cf. Filinto, III, 209.
\section{Cobrear}
\textunderscore v. t.\textunderscore  (e der.)
(V. \textunderscore acobrear\textunderscore , etc.)
\section{Cobrejão}
\begin{itemize}
\item {Grp. gram.:m.}
\end{itemize}
Manta alentejana.
Chaile-manta.
Manta, com que se cobre a cavalgadura, quando desarreada. Cf. Camillo, \textunderscore Myst. de Lisb.\textunderscore , 135.
\section{Cobrejar}
\begin{itemize}
\item {Grp. gram.:v. i.}
\end{itemize}
O mesmo que \textunderscore serpear\textunderscore  ou \textunderscore serpentear\textunderscore :«\textunderscore correios dos montes a cobrejar nos prados\textunderscore ». Garrett, \textunderscore Flôres sem Fruto\textunderscore , 103.
\section{Cobrelo}
\begin{itemize}
\item {fónica:brê}
\end{itemize}
\begin{itemize}
\item {Grp. gram.:m.}
\end{itemize}
\begin{itemize}
\item {Proveniência:(De \textunderscore cobra\textunderscore )}
\end{itemize}
Pequena cobra.
Erupção na pelle, attribuida pelo vulgo á passagem de cobra pelo fato que se vestiu.
\section{Cobre-nuca}
\begin{itemize}
\item {Grp. gram.:m.}
\end{itemize}
\begin{itemize}
\item {Proveniência:(De \textunderscore cobrir\textunderscore  + \textunderscore nuca\textunderscore )}
\end{itemize}
Cobertura branca das barretinas dos militares.
\section{Cobricama}
\begin{itemize}
\item {Grp. gram.:f.}
\end{itemize}
\begin{itemize}
\item {Utilização:Des.}
\end{itemize}
\begin{itemize}
\item {Proveniência:(De \textunderscore cobrir\textunderscore  + \textunderscore cama\textunderscore )}
\end{itemize}
O mesmo que coberta de cama.
Dossel, sobrecéu.
\section{Cobrição}
\begin{itemize}
\item {Grp. gram.:f.}
\end{itemize}
Cópula de quadrupedes.
Acção de cobrir.
\section{Cobricunha}
\begin{itemize}
\item {Grp. gram.:f.}
\end{itemize}
Peixe do Brasil.
\section{Cobridor}
\begin{itemize}
\item {Grp. gram.:m.}
\end{itemize}
\begin{itemize}
\item {Utilização:Prov.}
\end{itemize}
\begin{itemize}
\item {Utilização:alent.}
\end{itemize}
\begin{itemize}
\item {Proveniência:(De \textunderscore cobrir\textunderscore )}
\end{itemize}
Tampa de barro, de fórma cónico-achatada, que termina numa asa em que há um orifício, e que serve para cobrir as caçarolas que se levam ao lume.
\section{Cobril}
\begin{itemize}
\item {Grp. gram.:m.}
\end{itemize}
\begin{itemize}
\item {Utilização:Neol.}
\end{itemize}
\begin{itemize}
\item {Proveniência:(De \textunderscore cobra\textunderscore )}
\end{itemize}
Lugar, onde se guardam e se criam cobras, para estudo de ophidismo.
\section{Cobrilha}
\begin{itemize}
\item {Grp. gram.:f.}
\end{itemize}
\begin{itemize}
\item {Utilização:Prov.}
\end{itemize}
\begin{itemize}
\item {Utilização:alent.}
\end{itemize}
\begin{itemize}
\item {Proveniência:(De \textunderscore cobra\textunderscore )}
\end{itemize}
Larva, ou bichinho, que se cria sob a casca do sobreiro.
\section{Cobrimento}
\begin{itemize}
\item {Grp. gram.:m.}
\end{itemize}
Coisa que cobre.
Acto de \textunderscore cobrir\textunderscore .
\section{Cobrinha}
\begin{itemize}
\item {Grp. gram.:f.}
\end{itemize}
\begin{itemize}
\item {Utilização:Prov.}
\end{itemize}
\begin{itemize}
\item {Utilização:extrem.}
\end{itemize}
O mesmo que \textunderscore parietária\textunderscore . (Colhido em Ourém)
\section{Cobrir}
\begin{itemize}
\item {Grp. gram.:v. t.}
\end{itemize}
\begin{itemize}
\item {Utilização:Prov.}
\end{itemize}
\begin{itemize}
\item {Utilização:trasm.}
\end{itemize}
\begin{itemize}
\item {Grp. gram.:V. p.}
\end{itemize}
\begin{itemize}
\item {Proveniência:(Lat. \textunderscore cooperire\textunderscore )}
\end{itemize}
Tapar, occultar, com algum objecto pôsto em cima.
Resguardar, collocando-se alguém ou alguma coisa adeante ou em volta: \textunderscore cobrir a retirada\textunderscore .
Estar, alargar-se por cima de: \textunderscore o céu cobre a humanidade\textunderscore .
Proteger.
Encher.
Vestir.
Fecundar: \textunderscore o boi cobriu a vaca\textunderscore .
Disfarçar: \textunderscore cobrir os próprios defeitos\textunderscore .
Exceder.
Abafar (o som).
Pôr na cabeça: \textunderscore cobrir o chapéu\textunderscore .
Vestir-se com: \textunderscore cobrir o capote\textunderscore .
Pôr na cabeça o chapéu, barrete, etc.
Em caminhos de ferro, avisar com sinaes (o conductor de um combóio), ou pôr a salvo por meio de sinaes (um combóio).
\section{Côbro}
\begin{itemize}
\item {Grp. gram.:m.}
\end{itemize}
\begin{itemize}
\item {Utilização:Ant.}
\end{itemize}
Termo, fim.
Repressão (de acção má): \textunderscore pôr cobro ao desafôro\textunderscore .
Acção de cobrar.
Fôro, que certos reguengueiros pagavam ao rei.
\section{Côbro}
\begin{itemize}
\item {Grp. gram.:m.}
\end{itemize}
\begin{itemize}
\item {Proveniência:(Do rad. de \textunderscore cobra\textunderscore )}
\end{itemize}
O mesmo que \textunderscore cobrelo\textunderscore .
Cada volta, dada pela amarra no convés, quando se tem de largar âncora em sitio fundo.
\section{Coca}
\begin{itemize}
\item {Grp. gram.:f.}
\end{itemize}
\begin{itemize}
\item {Proveniência:(Lat. \textunderscore coccum\textunderscore )}
\end{itemize}
Cada uma das células ocas de um pericarpo.
\section{Cóca}
\begin{itemize}
\item {Grp. gram.:f.}
\end{itemize}
\begin{itemize}
\item {Grp. gram.:Loc. adv.}
\end{itemize}
Planta narcótica e nutritiva, (\textunderscore erythroxylon coca\textunderscore , Lin.).
Substância vegetal, com que se narcotizam os peixes, para se apanharem á mão, (\textunderscore coculus indica\textunderscore ).
\textunderscore Estar á cóca\textunderscore , cocar.
\section{Cóca}
\begin{itemize}
\item {Grp. gram.:f.}
\end{itemize}
\begin{itemize}
\item {Utilização:Prov.}
\end{itemize}
\begin{itemize}
\item {Utilização:minh.}
\end{itemize}
O mesmo que \textunderscore abóbora\textunderscore .
\section{Cóca}
\begin{itemize}
\item {Grp. gram.:f.}
\end{itemize}
\begin{itemize}
\item {Utilização:Prov.}
\end{itemize}
\begin{itemize}
\item {Utilização:minh.}
\end{itemize}
O mesmo que \textunderscore côca\textunderscore ^1, papão.
\section{Côca}
\begin{itemize}
\item {Grp. gram.:f.}
\end{itemize}
\begin{itemize}
\item {Utilização:Pop.}
\end{itemize}
\begin{itemize}
\item {Utilização:Infant.}
\end{itemize}
Bioco; capuz.
Papão.
Panela ou abóbora ôca, em que se fazem três ou quatro buracos, a fingir olhos, naríz e boca, e em que, de noite e em lugar escuro, se põe dentro uma luz, para meter medo a crianças ou gente tímida.
(Cast. \textunderscore coca\textunderscore )
\section{Côca}
\begin{itemize}
\item {Grp. gram.:f.}
\end{itemize}
\begin{itemize}
\item {Utilização:Prov.}
\end{itemize}
\begin{itemize}
\item {Utilização:trasm.}
\end{itemize}
Ferimento em crianças, axe.
\section{Coça}
\begin{itemize}
\item {Grp. gram.:f.}
\end{itemize}
\begin{itemize}
\item {Utilização:Pop.}
\end{itemize}
Acto de coçar; tosa, tareia: \textunderscore dei-lhe uma coça\textunderscore .
\section{Cocada}
\begin{itemize}
\item {Grp. gram.:f.}
\end{itemize}
Dôce de côco.
\section{Coçado}
\begin{itemize}
\item {Grp. gram.:adj.}
\end{itemize}
Roçado, cotiado, muito usado, (falando-se de vestuário).
\section{Coçadoiro}
\begin{itemize}
\item {Grp. gram.:m.}
\end{itemize}
Objecto, em que os animaes coçam o corpo:«\textunderscore servir de coçadoiro ás vacas.\textunderscore »Ortigão, \textunderscore Holl.\textunderscore , 100.
\section{Coçadouro}
\begin{itemize}
\item {Grp. gram.:m.}
\end{itemize}
Objecto, em que os animaes coçam o corpo:«\textunderscore servir de coçadouro ás vacas.\textunderscore »Ortigão, \textunderscore Holl.\textunderscore , 100.
\section{Cocadriz}
\begin{itemize}
\item {Grp. gram.:f.}
\end{itemize}
\begin{itemize}
\item {Utilização:Ant.}
\end{itemize}
Crocodilo.
\section{Coçadura}
\begin{itemize}
\item {Grp. gram.:f.}
\end{itemize}
Acto de coçar.
\section{Cocaína}
\begin{itemize}
\item {Grp. gram.:f.}
\end{itemize}
\begin{itemize}
\item {Proveniência:(De \textunderscore cóca\textunderscore ^1)}
\end{itemize}
Novo alcaloide natural, que se descobriu nas fôlhas da cóca.
\section{Cocainização}
\begin{itemize}
\item {fónica:ca-i}
\end{itemize}
\begin{itemize}
\item {Grp. gram.:f.}
\end{itemize}
Acto ou effeito de cocainizar.
\section{Cocainizar}
\begin{itemize}
\item {fónica:ca-i}
\end{itemize}
\begin{itemize}
\item {Grp. gram.:v. t.}
\end{itemize}
Anesthesiar com cocaína.
\section{Cocainomania}
\begin{itemize}
\item {fónica:ca-i}
\end{itemize}
\begin{itemize}
\item {Grp. gram.:f.}
\end{itemize}
Hábito mórbido e imperioso de tomar injecções subcutâneas de cocaína.
\section{Cocal}
\begin{itemize}
\item {Grp. gram.:m.}
\end{itemize}
\begin{itemize}
\item {Utilização:Bras}
\end{itemize}
O mesmo que \textunderscore coqueiral\textunderscore .
\section{Cocamas}
\begin{itemize}
\item {Grp. gram.:m. pl.}
\end{itemize}
Tríbo de indígenas do Peru.
\section{Cocamilas}
\begin{itemize}
\item {Grp. gram.:m. pl.}
\end{itemize}
Tríbo de indígenas do Peru.
\section{Cocanha}
\begin{itemize}
\item {Grp. gram.:f.}
\end{itemize}
\begin{itemize}
\item {Proveniência:(Fr. \textunderscore cocagne\textunderscore )}
\end{itemize}
\textunderscore Mastro de cocanha\textunderscore , mastro untado de sebo, em cujo tôpo se prendem aves ou outros objectos, para quem ouse ir lá buscá-los.
\section{Cocante}
\begin{itemize}
\item {Grp. gram.:m.}
\end{itemize}
\begin{itemize}
\item {Utilização:Gír.}
\end{itemize}
Ôlho, lúzio.
\section{Cocão}
\begin{itemize}
\item {Grp. gram.:m.}
\end{itemize}
\begin{itemize}
\item {Utilização:Prov.}
\end{itemize}
\begin{itemize}
\item {Proveniência:(Do b. lat. \textunderscore cocha\textunderscore )}
\end{itemize}
Cada um dos paus verticaes, sob o taboleiro do carro de bois, e entre os quaes gira o eixo.
Vasadura, por baixo do chedeiro, entre aquelles paus verticaes, a que também se chamam cantadeiras, e contra a qual gira o eixo do carro.
\section{Cocão}
\begin{itemize}
\item {Grp. gram.:m.}
\end{itemize}
\begin{itemize}
\item {Utilização:Bras. do N}
\end{itemize}
Árvore brasileira, que se encontra principalmente em Alagôas e Pernambuco, e cuja madeira é empregada em construcções civis. Aínda não está classificada. Cf. Del-Vecchio, \textunderscore Materiaes de Construcção\textunderscore , 79.
Bengala grossa; cacete.
\section{Còcão}
\begin{itemize}
\item {Grp. gram.:m.}
\end{itemize}
\begin{itemize}
\item {Utilização:Prov.}
\end{itemize}
\begin{itemize}
\item {Utilização:trasm.}
\end{itemize}
Primeiro ovo de perdiz, pôsto fóra do ninho.
\section{Cocar}
\begin{itemize}
\item {Grp. gram.:m.}
\end{itemize}
\begin{itemize}
\item {Proveniência:(Fr. \textunderscore cocarde\textunderscore )}
\end{itemize}
Pennacho, laço, distintivo, no chapéu ou capacete.
Distintivo de partido.
Roseta, para enfeite de cavallos.
\section{Cocar}
\begin{itemize}
\item {Grp. gram.:v. i.}
\end{itemize}
\begin{itemize}
\item {Utilização:Gír.}
\end{itemize}
\begin{itemize}
\item {Grp. gram.:V. t.}
\end{itemize}
\begin{itemize}
\item {Utilização:Prov.}
\end{itemize}
\begin{itemize}
\item {Utilização:minh.}
\end{itemize}
\begin{itemize}
\item {Proveniência:(De \textunderscore cóca\textunderscore ^1)}
\end{itemize}
Estar á espreita, estar de atalaia.
Mondar pela segunda vez (linho).
\section{Coçar}
\begin{itemize}
\item {Grp. gram.:v. t.}
\end{itemize}
\begin{itemize}
\item {Utilização:Fig.}
\end{itemize}
\begin{itemize}
\item {Utilização:Marn.}
\end{itemize}
\begin{itemize}
\item {Proveniência:(Do lat. hyp. \textunderscore coctiare\textunderscore , do lat. \textunderscore coctus\textunderscore )}
\end{itemize}
Esfregar com as unhas ou com outro objecto (qualquer parte do corpo, em que há prurido).
Bater.
Aplanar (o chão).
\section{Çocar}
\begin{itemize}
\item {Grp. gram.:v. t.}
\end{itemize}
\begin{itemize}
\item {Utilização:Bras}
\end{itemize}
O mesmo ou melhor que \textunderscore socar\textunderscore ^2.
\section{Cócaras}
\begin{itemize}
\item {Grp. gram.:f. pl.}
\end{itemize}
(V.cócoras)
\section{Cocarda}
\begin{itemize}
\item {Grp. gram.:f.}
\end{itemize}
O mesmo que \textunderscore cocar\textunderscore ^1. Cf. San-Luís, \textunderscore Glossário\textunderscore .
\section{Cocaria}
\begin{itemize}
\item {Grp. gram.:f.}
\end{itemize}
\begin{itemize}
\item {Utilização:Prov.}
\end{itemize}
\begin{itemize}
\item {Utilização:alent.}
\end{itemize}
\begin{itemize}
\item {Proveniência:(De \textunderscore coque\textunderscore ^2)}
\end{itemize}
Rancho de trabalhadores, que se juntam para fazer comida.
\section{Cocarinhas}
\begin{itemize}
\item {fónica:De}
\end{itemize}
\begin{itemize}
\item {Grp. gram.:loc. adv.}
\end{itemize}
\begin{itemize}
\item {Utilização:Prov.}
\end{itemize}
Muito de cócoras. Cf. Camillo, \textunderscore Volcões\textunderscore , 103.
\section{Cocca}
\begin{itemize}
\item {Grp. gram.:f.}
\end{itemize}
\begin{itemize}
\item {Proveniência:(Lat. \textunderscore coccum\textunderscore )}
\end{itemize}
Cada uma das céllulas ocas de um pericarpo.
\section{Cocção}
\begin{itemize}
\item {Grp. gram.:f.}
\end{itemize}
\begin{itemize}
\item {Proveniência:(Lat. \textunderscore coctio\textunderscore )}
\end{itemize}
Acção de cozer.
Digestão dos alimentos, no estômago.
\section{Coccinela}
\begin{itemize}
\item {Grp. gram.:f.}
\end{itemize}
\begin{itemize}
\item {Proveniência:(Do rad. do lat. \textunderscore coccineus\textunderscore )}
\end{itemize}
Gênero de insectos coleópteros trímeros.
\section{Coccíneo}
\begin{itemize}
\item {Grp. gram.:adj.}
\end{itemize}
\begin{itemize}
\item {Proveniência:(Lat. \textunderscore coccineus\textunderscore )}
\end{itemize}
De côr escarlate, granadino.
\section{Côcco}
\begin{itemize}
\item {Grp. gram.:m.}
\end{itemize}
\begin{itemize}
\item {Utilização:Med.}
\end{itemize}
\begin{itemize}
\item {Proveniência:(Lat. \textunderscore coccum\textunderscore )}
\end{itemize}
Segundo uns, fruto vermelho de uma espécie de carvalho, empregado em tinturaria.
Segundo outros, o mesmo que \textunderscore cochinilha\textunderscore .
Bactéria, de fórma arredondada.
\section{Cócculo}
\begin{itemize}
\item {Grp. gram.:m.}
\end{itemize}
Planta menispermácea.
\section{Coccýgeo}
\begin{itemize}
\item {Grp. gram.:adj.}
\end{itemize}
\begin{itemize}
\item {Utilização:Anat.}
\end{itemize}
Relativo ao \textunderscore cóccyx\textunderscore .
\section{Coccygomorphas}
\begin{itemize}
\item {Grp. gram.:f. pl.}
\end{itemize}
\begin{itemize}
\item {Utilização:Zool.}
\end{itemize}
\begin{itemize}
\item {Proveniência:(Do gr. \textunderscore kokkux\textunderscore  + \textunderscore morphe\textunderscore )}
\end{itemize}
Ordem de aves, que têm por typo o cuco.
\section{Cóccyx}
\begin{itemize}
\item {Grp. gram.:m.}
\end{itemize}
\begin{itemize}
\item {Utilização:Anat.}
\end{itemize}
\begin{itemize}
\item {Proveniência:(Gr. \textunderscore kokkux\textunderscore )}
\end{itemize}
Pequeno osso, que termina inferiormente a columna vertebral do homem.
\section{Cócedra}
\begin{itemize}
\item {Grp. gram.:f.}
\end{itemize}
\begin{itemize}
\item {Utilização:Ant.}
\end{itemize}
\begin{itemize}
\item {Proveniência:(Do lat. \textunderscore culcitra\textunderscore )}
\end{itemize}
Colchão de pennas.
\section{Cócegas}
\begin{itemize}
\item {Grp. gram.:f. pl.}
\end{itemize}
\begin{itemize}
\item {Utilização:Fig.}
\end{itemize}
\begin{itemize}
\item {Proveniência:(Do rad. de \textunderscore coçar\textunderscore )}
\end{itemize}
Sensação especial, acompanhada de riso convulsivo, e produzida pela fricção em alguns pontos da pelle ou das mucosas.
Tentação.
Impaciência.
\section{Cocégas}
\begin{itemize}
\item {Grp. gram.:f. pl.}
\end{itemize}
\begin{itemize}
\item {Utilização:Prov.}
\end{itemize}
\begin{itemize}
\item {Utilização:alg.}
\end{itemize}
O mesmo que \textunderscore cócegas\textunderscore .
\section{Coceguento}
\begin{itemize}
\item {Grp. gram.:adj.}
\end{itemize}
\begin{itemize}
\item {Utilização:Prov.}
\end{itemize}
\begin{itemize}
\item {Utilização:Fig.}
\end{itemize}
Que facilmente sente cócegas.
Rabugento, impertinente.
Muito desejoso.
\section{Coceira}
\begin{itemize}
\item {Grp. gram.:f.}
\end{itemize}
\begin{itemize}
\item {Proveniência:(De \textunderscore coçar\textunderscore )}
\end{itemize}
Comichão.
\section{Cocemegas}
\begin{itemize}
\item {Grp. gram.:f. pl.}
\end{itemize}
\begin{itemize}
\item {Utilização:Prov.}
\end{itemize}
\begin{itemize}
\item {Utilização:chul.}
\end{itemize}
O mesmo que \textunderscore cócegas\textunderscore .
\section{Cócha}
\begin{itemize}
\item {Grp. gram.:f.}
\end{itemize}
Cada um dos ramos que, torcidos, formam um cabo de embarcação.
Torcedura de cabo.
\section{Côcha}
\begin{itemize}
\item {Grp. gram.:f.}
\end{itemize}
\begin{itemize}
\item {Utilização:Prov.}
\end{itemize}
Gamela ou vaso, o mesmo que \textunderscore cocho\textunderscore . (Colhido em Turquel)
\section{Cochada}
\begin{itemize}
\item {Grp. gram.:f.}
\end{itemize}
\begin{itemize}
\item {Utilização:Des.}
\end{itemize}
Coche, cheio de gente.
\section{Cochado}
\begin{itemize}
\item {Grp. gram.:adj.}
\end{itemize}
\begin{itemize}
\item {Utilização:Ant.}
\end{itemize}
\begin{itemize}
\item {Proveniência:(De \textunderscore cochar\textunderscore )}
\end{itemize}
Apertado.
Cerrado.
\section{Cochar}
\begin{itemize}
\item {Grp. gram.:v. t.}
\end{itemize}
\begin{itemize}
\item {Utilização:Ext.}
\end{itemize}
\begin{itemize}
\item {Utilização:Prov.}
\end{itemize}
\begin{itemize}
\item {Utilização:beir.}
\end{itemize}
\begin{itemize}
\item {Proveniência:(De \textunderscore cocha\textunderscore )}
\end{itemize}
Torcer (cabos náuticos).
Apertar. Cf. Bernardes, \textunderscore Luz e Calor\textunderscore , 383.
Encher, fartar.
\section{Cochar}
\begin{itemize}
\item {Grp. gram.:v. t.}
\end{itemize}
\begin{itemize}
\item {Utilização:Prov.}
\end{itemize}
\begin{itemize}
\item {Utilização:beir.}
\end{itemize}
\begin{itemize}
\item {Proveniência:(De \textunderscore coche\textunderscore ^3)}
\end{itemize}
Tirar das poças ou presas (água), com cocho ou cabaço.
\section{Cocharra}
\begin{itemize}
\item {Grp. gram.:f.}
\end{itemize}
\begin{itemize}
\item {Utilização:Prov.}
\end{itemize}
\begin{itemize}
\item {Utilização:trasm.}
\end{itemize}
O mesmo que \textunderscore colhér\textunderscore .
(Cp. \textunderscore cocharro\textunderscore )
\section{Cocharro}
\begin{itemize}
\item {Grp. gram.:m.}
\end{itemize}
\begin{itemize}
\item {Utilização:Prov.}
\end{itemize}
\begin{itemize}
\item {Utilização:alent.}
\end{itemize}
\begin{itemize}
\item {Utilização:T. de Serpa}
\end{itemize}
\begin{itemize}
\item {Proveniência:(De \textunderscore cocho\textunderscore ^1)}
\end{itemize}
Vaso de cortiça, cuja cavidade é natural, por corresponder a um nó da árvore respectiva.
Vasilha de barro, em que a água se conserva fresca.
\section{Coche}
\begin{itemize}
\item {fónica:cô}
\end{itemize}
\begin{itemize}
\item {Grp. gram.:m.}
\end{itemize}
Carruagem antiga e rica.
Sege.
Pequena embarcação da África oriental.
(Provavelmente do hung. \textunderscore kocsi\textunderscore )
\section{Coche!}
\begin{itemize}
\item {fónica:cô}
\end{itemize}
\begin{itemize}
\item {Grp. gram.:interj.}
\end{itemize}
\begin{itemize}
\item {Utilização:Pop.}
\end{itemize}
Voz, com que se enxotam ou se chamam porcos.
\section{Coche}
\begin{itemize}
\item {fónica:cô}
\end{itemize}
\begin{itemize}
\item {Grp. gram.:m.}
\end{itemize}
\begin{itemize}
\item {Utilização:T. de Lisbôa}
\end{itemize}
\begin{itemize}
\item {Utilização:Prov.}
\end{itemize}
\begin{itemize}
\item {Utilização:beir.}
\end{itemize}
Taboleiro com rebordos, para conduzir cal amassada; o mesmo que \textunderscore cocho\textunderscore ^1.
Caixa do rebôlo de carpinteiros e marceneiros.
Vasilha de lata, segura por um cabo de madeira, e com a qual se extrai água das poças ou presas.
(O mesmo que \textunderscore cocho\textunderscore ^1 e \textunderscore corcho\textunderscore )
\section{Cocheira}
\begin{itemize}
\item {Grp. gram.:f.}
\end{itemize}
\begin{itemize}
\item {Proveniência:(De \textunderscore coche\textunderscore ^1)}
\end{itemize}
Casa, onde se guardam carruagens, ou onde se alojam cavallos.
\section{Cocheiro}
\begin{itemize}
\item {Grp. gram.:m.}
\end{itemize}
\begin{itemize}
\item {Proveniência:(De \textunderscore coche\textunderscore ^1)}
\end{itemize}
Aquelle que dirige os cavallos de uma carruagem.
Constellação setentrional.
\section{Cochela}
\begin{itemize}
\item {Grp. gram.:f.}
\end{itemize}
Pequeno cocho^1.
\section{Cochenilha}
\begin{itemize}
\item {Grp. gram.:f.}
\end{itemize}
(V.cochinilha)
\section{Cochicha}
\begin{itemize}
\item {Grp. gram.:f.}
\end{itemize}
Nome de um pequeno pássaro.
(Cp. \textunderscore cochicho\textunderscore ^1)
\section{Cochichador}
\begin{itemize}
\item {Grp. gram.:m.}
\end{itemize}
Aquelle que cochicha.
\section{Cochichar}
\begin{itemize}
\item {Grp. gram.:v. i.}
\end{itemize}
\begin{itemize}
\item {Utilização:Fam.}
\end{itemize}
\begin{itemize}
\item {Proveniência:(De \textunderscore cochicho\textunderscore ^1)}
\end{itemize}
Falar em voz baixa.
\section{Cochicho}
\begin{itemize}
\item {Grp. gram.:m.}
\end{itemize}
\begin{itemize}
\item {Utilização:Gír.}
\end{itemize}
\begin{itemize}
\item {Utilização:Gír.}
\end{itemize}
\begin{itemize}
\item {Proveniência:(T. onom.)}
\end{itemize}
Pássaro, da fam. dos conirostros.
Instrumento infantil, cujo som imita a voz do cochicho.
Pequena casa, aposento muito estreito.
Chapéu velho.
Moéda de 50 reis.
Chapéu pequeno.
\section{Cochicho}
\begin{itemize}
\item {Grp. gram.:m.}
\end{itemize}
Acto de cochichar. Cf. Camillo, \textunderscore Volcões\textunderscore , 88.
\section{Cochicholo}
\begin{itemize}
\item {Grp. gram.:m.}
\end{itemize}
\begin{itemize}
\item {Utilização:Pop.}
\end{itemize}
\begin{itemize}
\item {Proveniência:(De \textunderscore cochicho\textunderscore ^1)}
\end{itemize}
Casa muito pequena; compartimento acanhadissimo.
\section{Cochila}
\begin{itemize}
\item {Grp. gram.:f.}
\end{itemize}
\begin{itemize}
\item {Utilização:Bras}
\end{itemize}
Cordilheira escalvada.
\section{Cochilar}
\begin{itemize}
\item {Grp. gram.:v. i.}
\end{itemize}
\begin{itemize}
\item {Utilização:Bras}
\end{itemize}
\begin{itemize}
\item {Proveniência:(T. afr.)}
\end{itemize}
Cabecear com somno; toscanejar.
Oscillar.
\section{Cochilha}
\begin{itemize}
\item {Grp. gram.:f.}
\end{itemize}
O mesmo que \textunderscore cochila\textunderscore .
\section{Cochílis}
\begin{itemize}
\item {fónica:qui}
\end{itemize}
\begin{itemize}
\item {Grp. gram.:f.}
\end{itemize}
Pequena borboleta, espécie de traça, que come as flôres da videira e lhe corrói os bagos.
Doença das vinhas, occasionada por aquelle insecto.
\section{Cochilo}
\begin{itemize}
\item {Grp. gram.:m.}
\end{itemize}
\begin{itemize}
\item {Utilização:Bras}
\end{itemize}
Acto de cochilar.
\section{Cochinada}
\begin{itemize}
\item {Grp. gram.:f.}
\end{itemize}
\begin{itemize}
\item {Utilização:Prov.}
\end{itemize}
\begin{itemize}
\item {Utilização:trasm.}
\end{itemize}
\begin{itemize}
\item {Proveniência:(De \textunderscore cochino\textunderscore )}
\end{itemize}
Porcaria.
Cancaborrada.
Vara de pequenos porcos.
Ruído de malta de gallegos.
Acção indecorosa.
\section{Cochinar}
\begin{itemize}
\item {Grp. gram.:v. i.}
\end{itemize}
\begin{itemize}
\item {Utilização:Fig.}
\end{itemize}
\begin{itemize}
\item {Proveniência:(De \textunderscore cochino\textunderscore )}
\end{itemize}
Grunhir.
Fazer ruído de vozes, como malta de gallegos.
\section{Cochinchina}
\begin{itemize}
\item {Grp. gram.:f.}
\end{itemize}
Espécie de ave gallinácea.
\section{Cochinchino}
\begin{itemize}
\item {Grp. gram.:m.}
\end{itemize}
Habitante da Cochinchina.
Língua da Cochinchina.
\section{Cochinês}
\begin{itemize}
\item {Grp. gram.:m.}
\end{itemize}
Habitante de Cochim, na Índia. Cf. Filinto, \textunderscore D. Man.\textunderscore , I, 319.
\section{Cochinilha}
\begin{itemize}
\item {Grp. gram.:f.}
\end{itemize}
Insecto hemíptero, de que se extrai uma substância, com que se fabricam tintas vermelhas.
A substância còrante da cochinilha.
\textunderscore Cochinilha da vinha\textunderscore , parasito, que ataca as videiras.
(Cast. \textunderscore cochinilla\textunderscore )
\section{Cochino}
\begin{itemize}
\item {Grp. gram.:m.}
\end{itemize}
\begin{itemize}
\item {Utilização:Pop.}
\end{itemize}
\begin{itemize}
\item {Utilização:Fig.}
\end{itemize}
\begin{itemize}
\item {Grp. gram.:Adj.}
\end{itemize}
\begin{itemize}
\item {Utilização:Prov.}
\end{itemize}
\begin{itemize}
\item {Utilização:trasm.}
\end{itemize}
\begin{itemize}
\item {Proveniência:(T. cast.)}
\end{itemize}
Porco, não cevado.
Indivíduo immundo e resmungão.
Sujo, immundo.
\section{Cóchlea}
\begin{itemize}
\item {Grp. gram.:f.}
\end{itemize}
\begin{itemize}
\item {Utilização:Fig.}
\end{itemize}
\begin{itemize}
\item {Proveniência:(Lat. \textunderscore cochlea\textunderscore )}
\end{itemize}
Caracol.
Canal auditivo.
Parafuso de Archimedes.
\section{Cochlear}
\begin{itemize}
\item {Grp. gram.:adj.}
\end{itemize}
\begin{itemize}
\item {Proveniência:(Lat. \textunderscore cochlearis\textunderscore )}
\end{itemize}
Que tem fórma de espiral.
\section{Cochleária}
\begin{itemize}
\item {Grp. gram.:f.}
\end{itemize}
\begin{itemize}
\item {Proveniência:(Lat. \textunderscore cochlearia\textunderscore )}
\end{itemize}
Planta crucífera, medicinal.
\section{Cochleariforme}
\begin{itemize}
\item {Grp. gram.:adj.}
\end{itemize}
\begin{itemize}
\item {Proveniência:(Do lat. \textunderscore cochlearium\textunderscore  + \textunderscore forma\textunderscore )}
\end{itemize}
Que tem fórma de colhér.
\section{Cochleiforme}
\begin{itemize}
\item {Grp. gram.:adj.}
\end{itemize}
O mesmo que \textunderscore cochlear\textunderscore .
\section{Cocho}
\begin{itemize}
\item {fónica:cô}
\end{itemize}
\begin{itemize}
\item {Grp. gram.:m.}
\end{itemize}
\begin{itemize}
\item {Utilização:Prov.}
\end{itemize}
\begin{itemize}
\item {Utilização:alent.}
\end{itemize}
\begin{itemize}
\item {Utilização:Bras}
\end{itemize}
Taboleiro, para transportar cal amassada.
Caixa, onde gira a mó dos amoladores.
Pedaço côncavo de cortiça, por onde se bebe água nas fontes.
Espécie de vasilha oblonga, onde se põe água ou comida para o gado.
(Cp. \textunderscore corcho\textunderscore )
\section{Cocho}
\begin{itemize}
\item {fónica:cô}
\end{itemize}
\begin{itemize}
\item {Grp. gram.:m.}
\end{itemize}
\begin{itemize}
\item {Utilização:Prov.}
\end{itemize}
\begin{itemize}
\item {Utilização:minh.}
\end{itemize}
O mesmo que \textunderscore porco\textunderscore .
(Cp. \textunderscore cochino\textunderscore )
\section{Cochom}
\begin{itemize}
\item {Grp. gram.:m.}
\end{itemize}
\begin{itemize}
\item {Utilização:Ant.}
\end{itemize}
\begin{itemize}
\item {Proveniência:(Fr. \textunderscore cochon\textunderscore )}
\end{itemize}
Porco.
\section{Cochonilha}
\begin{itemize}
\item {Grp. gram.:f.}
\end{itemize}
(V.cochinilha)
\section{Cochumiacos}
\begin{itemize}
\item {Grp. gram.:m.}
\end{itemize}
Espécie de letra de câmbio, que os bonzos do Japão passavam a quem, saindo desta vida, queria entrar no céu. Cf. \textunderscore Peregrinação\textunderscore , CCXII.
\section{Cocígeo}
\begin{itemize}
\item {Grp. gram.:adj.}
\end{itemize}
\begin{itemize}
\item {Utilização:Anat.}
\end{itemize}
Relativo ao \textunderscore cócix\textunderscore .
\section{Cocigomorfas}
\begin{itemize}
\item {Grp. gram.:f. pl.}
\end{itemize}
\begin{itemize}
\item {Utilização:Zool.}
\end{itemize}
\begin{itemize}
\item {Proveniência:(Do gr. \textunderscore kokkux\textunderscore  + \textunderscore morphe\textunderscore )}
\end{itemize}
Ordem de aves, que têm por tipo o cuco.
\section{Cocinela}
\begin{itemize}
\item {Grp. gram.:f.}
\end{itemize}
\begin{itemize}
\item {Proveniência:(Do rad. do lat. \textunderscore coccineus\textunderscore )}
\end{itemize}
Gênero de insectos coleópteros trímeros.
\section{Cocíneo}
\begin{itemize}
\item {Grp. gram.:adj.}
\end{itemize}
\begin{itemize}
\item {Proveniência:(Lat. \textunderscore coccineus\textunderscore )}
\end{itemize}
De côr escarlate, granadino.
\section{Cocivarado}
\begin{itemize}
\item {Grp. gram.:m.}
\end{itemize}
Pensão ou foro territorial, que os lavradores pagavam ás autoridades portuguesas nas faldas dos Gates e nas tanadarias de Gôa.
\section{Cócix}
\begin{itemize}
\item {Grp. gram.:m.}
\end{itemize}
\begin{itemize}
\item {Utilização:Anat.}
\end{itemize}
\begin{itemize}
\item {Proveniência:(Gr. \textunderscore kokkux\textunderscore )}
\end{itemize}
Pequeno osso, que termina inferiormente a columna vertebral do homem.
\section{Cóclea}
\begin{itemize}
\item {Grp. gram.:f.}
\end{itemize}
\begin{itemize}
\item {Utilização:Fig.}
\end{itemize}
\begin{itemize}
\item {Proveniência:(Lat. \textunderscore cochlea\textunderscore )}
\end{itemize}
Caracol.
Canal auditivo.
Parafuso de Archimedes.
\section{Coclear}
\begin{itemize}
\item {Grp. gram.:adj.}
\end{itemize}
\begin{itemize}
\item {Proveniência:(Lat. \textunderscore cochlearis\textunderscore )}
\end{itemize}
Que tem fórma de espiral.
\section{Cocleária}
\begin{itemize}
\item {Grp. gram.:f.}
\end{itemize}
\begin{itemize}
\item {Proveniência:(Lat. \textunderscore cochlearia\textunderscore )}
\end{itemize}
Planta crucífera, medicinal.
\section{Cocleariforme}
\begin{itemize}
\item {Grp. gram.:adj.}
\end{itemize}
\begin{itemize}
\item {Proveniência:(Do lat. \textunderscore cochlearium\textunderscore  + \textunderscore forma\textunderscore )}
\end{itemize}
Que tem fórma de colhér.
\section{Cocleiforme}
\begin{itemize}
\item {Grp. gram.:adj.}
\end{itemize}
O mesmo que \textunderscore coclear\textunderscore .
\section{Côco}
\begin{itemize}
\item {Grp. gram.:m.}
\end{itemize}
\begin{itemize}
\item {Utilização:Prov.}
\end{itemize}
\begin{itemize}
\item {Utilização:Gír.}
\end{itemize}
\begin{itemize}
\item {Utilização:Ant.}
\end{itemize}
Fruto de coqueiro.
Vasilha grosseira, feita de casca dêsse fruto.
Metade dêsse fruto, empregada em esfregar as casas.
O mesmo que cabaço.
O mesmo que \textunderscore coqueiro\textunderscore ^1.
Copo.
O mesmo que \textunderscore côca\textunderscore ^1, abantesma, papão.
\textunderscore Chapéu de coco\textunderscore , chapéu de aba estreita e copa pequena e arredondada, mas não de feltro molle.
(Cast. \textunderscore coco\textunderscore )
\section{Côco}
\begin{itemize}
\item {Grp. gram.:adj.}
\end{itemize}
Diz-se da amêndoa, cuja casca se desfaz entre os dedos, por opposição á \textunderscore mollar\textunderscore , que se parte com os dentes, e á \textunderscore durázia\textunderscore , que se quebra a martelo.
\section{Côco}
\begin{itemize}
\item {Grp. gram.:m.}
\end{itemize}
\begin{itemize}
\item {Utilização:Bras}
\end{itemize}
Dança popular no Estado das Alagôas, ao som de palmas e canto, sem música instrumental.
\section{Côco}
\begin{itemize}
\item {Grp. gram.:m.}
\end{itemize}
\begin{itemize}
\item {Utilização:Med.}
\end{itemize}
\begin{itemize}
\item {Proveniência:(Lat. \textunderscore coccum\textunderscore )}
\end{itemize}
Segundo uns, fruto vermelho de uma espécie de carvalho, empregado em tinturaria.
Segundo outros, o mesmo que \textunderscore cochinilha\textunderscore .
Bactéria, de fórma arredondada.
\section{Cóco}
\begin{itemize}
\item {Grp. gram.:m.}
\end{itemize}
Ave da África occidental, (\textunderscore strix flammea\textunderscore ).
\section{Có-có}
\begin{itemize}
\item {Grp. gram.:m.}
\end{itemize}
\begin{itemize}
\item {Utilização:Bras. do N}
\end{itemize}
Caracol de cabello, no alto da cabeça; carrapito.
\section{Có-có}
\begin{itemize}
\item {Grp. gram.:m.}
\end{itemize}
\begin{itemize}
\item {Utilização:Prov.}
\end{itemize}
\begin{itemize}
\item {Utilização:trasm.}
\end{itemize}
O mesmo que \textunderscore ânus\textunderscore .
\section{Çóco}
\begin{itemize}
\item {Grp. gram.:m.}
\end{itemize}
Plintho; quadro, em que termina a moldura inferior de um pedestal.
Base do pedestal.
(Cast. \textunderscore zueco\textunderscore )
\section{Cocoíneas}
\begin{itemize}
\item {Grp. gram.:f. pl.}
\end{itemize}
\begin{itemize}
\item {Proveniência:(De \textunderscore côco\textunderscore ^1)}
\end{itemize}
Secção da fam. das palmeiras, segundo Martius.
\section{Cocololombua}
\begin{itemize}
\item {Grp. gram.:f.}
\end{itemize}
Ave africana, (\textunderscore streptopelia damarensis\textunderscore ).
\section{Cocomílio}
\begin{itemize}
\item {Grp. gram.:m.}
\end{itemize}
\begin{itemize}
\item {Proveniência:(T. it.)}
\end{itemize}
Variedade de abrunheiro da Calábria.
\section{Coconote}
\begin{itemize}
\item {Grp. gram.:m.}
\end{itemize}
\begin{itemize}
\item {Proveniência:(Fr. \textunderscore coconotte\textunderscore )}
\end{itemize}
Semente de uma espécie de palmeira, de que se extrai um óleo commercial.
\section{Cócoras}
\begin{itemize}
\item {fónica:De}
\end{itemize}
\begin{itemize}
\item {Grp. gram.:loc. adv.}
\end{itemize}
Diz-se que \textunderscore está de cócoras\textunderscore  quem está quási sentado no chão ou sentado sôbre os calcanhares.
\section{Có-có-ró-có!}
(Voz imitativa do canto da gallinha, em seguida á postura do ovo)
\section{Cocorote}
\begin{itemize}
\item {Grp. gram.:m.}
\end{itemize}
\begin{itemize}
\item {Utilização:Bras}
\end{itemize}
Carolo, pancada com os nós dos dedos na cabeça de outrem.
(Talvez de \textunderscore cocoruto\textunderscore . Cp. \textunderscore piparote\textunderscore )
\section{Cocoruto}
\begin{itemize}
\item {Grp. gram.:m.}
\end{itemize}
\begin{itemize}
\item {Proveniência:(De \textunderscore coruto\textunderscore , com um prefixo arbitrário)}
\end{itemize}
A parte mais alta.
O alto da cabeça.
Vértice.
\section{Cocota}
\begin{itemize}
\item {Grp. gram.:f.}
\end{itemize}
\begin{itemize}
\item {Utilização:Prov.}
\end{itemize}
\begin{itemize}
\item {Utilização:trasm.}
\end{itemize}
Nuca.
(Cp. \textunderscore cocuruta\textunderscore ^2)
\section{Cocote}
\begin{itemize}
\item {Grp. gram.:f.}
\end{itemize}
\begin{itemize}
\item {Utilização:Neol.}
\end{itemize}
Papel de côr, atado em fórma de boneca de envernizador, e que, contendo papelinhos e areia, ou outros objectos, se emprega como projéctil em folguedos do Carnaval.
\section{Cocrèdor}
\begin{itemize}
\item {Grp. gram.:m.}
\end{itemize}
\begin{itemize}
\item {Proveniência:(De \textunderscore co...\textunderscore  + \textunderscore crèdor\textunderscore )}
\end{itemize}
Aquelle que é crèdor, juntamente com outrem.
\section{Cocto}
\begin{itemize}
\item {Grp. gram.:adj.}
\end{itemize}
\begin{itemize}
\item {Proveniência:(Lat. \textunderscore coctus\textunderscore )}
\end{itemize}
Cozido, repassado pelo calor do fogo.--Us. principalmente em cerâmica.
\section{Coculo}
\begin{itemize}
\item {Grp. gram.:m.}
\end{itemize}
\begin{itemize}
\item {Utilização:Bras. do N}
\end{itemize}
O mesmo que \textunderscore cogulo\textunderscore .
\section{Cóculo}
\begin{itemize}
\item {Grp. gram.:m.}
\end{itemize}
Planta menispermácea.
\section{Cocumbi}
\begin{itemize}
\item {Grp. gram.:m.}
\end{itemize}
\begin{itemize}
\item {Utilização:Bras. do S}
\end{itemize}
Dança festiva, própria de Africanos.
\section{Cocuruta}
\begin{itemize}
\item {Grp. gram.:f.}
\end{itemize}
Peixe de Portugal.
\section{Cocuruta}
\begin{itemize}
\item {Grp. gram.:f.}
\end{itemize}
\begin{itemize}
\item {Proveniência:(De \textunderscore coruto\textunderscore , com um prefixo arbitrário)}
\end{itemize}
A parte mais alta.
O alto da cabeça.
Vértice.
\section{Coda}
\begin{itemize}
\item {Grp. gram.:f.}
\end{itemize}
\begin{itemize}
\item {Utilização:Ant.}
\end{itemize}
\begin{itemize}
\item {Proveniência:(It. \textunderscore coda\textunderscore , cauda)}
\end{itemize}
Período musical, vivo e brilhante, que termina a execução de um trecho.
O mesmo que \textunderscore cauda\textunderscore .
\section{Codagem}
\begin{itemize}
\item {Grp. gram.:f.}
\end{itemize}
Planta medicinal do Brasil, também conhecida por \textunderscore pé-de-cavallo\textunderscore .
\section{Códão}
\begin{itemize}
\item {Grp. gram.:m.}
\end{itemize}
Congelação da humidade infiltrada no solo.
Sincelo.
Geada.
\section{Codaste}
\begin{itemize}
\item {Grp. gram.:m.}
\end{itemize}
(V.cadaste)
\section{Côdea}
\begin{itemize}
\item {Grp. gram.:f.}
\end{itemize}
\begin{itemize}
\item {Utilização:Prov.}
\end{itemize}
\begin{itemize}
\item {Utilização:Ext.}
\end{itemize}
\begin{itemize}
\item {Grp. gram.:M.}
\end{itemize}
\begin{itemize}
\item {Utilização:T. do Pôrto}
\end{itemize}
\begin{itemize}
\item {Proveniência:(Lat. hyp. \textunderscore cutina\textunderscore , do lat. \textunderscore cutis\textunderscore . Cp. cast. \textunderscore códena\textunderscore )}
\end{itemize}
Casca; crosta.
Sujidade no fato.
Pequena refeição dos trabalhadores do campo, entre o almôço e o jantar.
Pequeno jantar, jantarela.
Crosta do pão.
Servente de pedreiro.
\section{Codeão}
\begin{itemize}
\item {Grp. gram.:m.}
\end{itemize}
\begin{itemize}
\item {Utilização:Prov.}
\end{itemize}
\begin{itemize}
\item {Utilização:alent.}
\end{itemize}
\begin{itemize}
\item {Proveniência:(De \textunderscore côdea\textunderscore )}
\end{itemize}
Terra endurecida pela geada.
\section{Codear}
\begin{itemize}
\item {Grp. gram.:v. t.}
\end{itemize}
\begin{itemize}
\item {Utilização:Prov.}
\end{itemize}
\begin{itemize}
\item {Grp. gram.:V. i.}
\end{itemize}
\begin{itemize}
\item {Proveniência:(De \textunderscore côdea\textunderscore )}
\end{itemize}
O mesmo que \textunderscore escodear\textunderscore .
Lanchar, comer a piqueta; comer côdea.
Debicar, petiscar. Cf. \textunderscore Anat. Joc.\textunderscore , I, 123.
\section{Côdeas}
\begin{itemize}
\item {Grp. gram.:m.}
\end{itemize}
\begin{itemize}
\item {Utilização:Fam.}
\end{itemize}
\begin{itemize}
\item {Proveniência:(De \textunderscore côdea\textunderscore )}
\end{itemize}
Aquelle que usa o fato muito sujo ou immundo; besuntão.
\section{Codeçal}
\begin{itemize}
\item {Grp. gram.:m.}
\end{itemize}
Lugar, onde crescem codeços.
\section{Codeceira}
\begin{itemize}
\item {Grp. gram.:f.}
\end{itemize}
O mesmo que \textunderscore codeçal\textunderscore .
\section{Codeço}
\begin{itemize}
\item {fónica:dê}
\end{itemize}
\begin{itemize}
\item {Grp. gram.:m.}
\end{itemize}
Arbusto, da fam. das leguminosas, (\textunderscore cytisus hirsutus\textunderscore ).
\section{Côdega}
\begin{itemize}
\item {Grp. gram.:f.}
\end{itemize}
Casta de uva branca trasmontana e duriense.
\section{Codeína}
\begin{itemize}
\item {Grp. gram.:f.}
\end{itemize}
\begin{itemize}
\item {Proveniência:(Do gr. \textunderscore kode\textunderscore , cabeça de papoila)}
\end{itemize}
Alcaloide, descoberto no ópio.
\section{Codejado}
\begin{itemize}
\item {Grp. gram.:adj.}
\end{itemize}
\begin{itemize}
\item {Proveniência:(De \textunderscore codejar\textunderscore )}
\end{itemize}
Em que há codo.
\section{Codejar}
\begin{itemize}
\item {Grp. gram.:v. i.}
\end{itemize}
Formar-se o codo: \textunderscore caminho codejado\textunderscore .
\section{Codejo}
\begin{itemize}
\item {Grp. gram.:m.}
\end{itemize}
\begin{itemize}
\item {Utilização:Marn.}
\end{itemize}
\begin{itemize}
\item {Proveniência:(De \textunderscore côdea\textunderscore ?)}
\end{itemize}
Sulfato de cal.
\section{Codelinquência}
\begin{itemize}
\item {Grp. gram.:f.}
\end{itemize}
Estado de \textunderscore codelinquente\textunderscore .
\section{Codelinquente}
\begin{itemize}
\item {Grp. gram.:m.}
\end{itemize}
Aquelle que é delinquente, juntamente com outrem. Cf. T. Barreto, \textunderscore Est. de Dir.\textunderscore , 515.
\section{Codemandante}
\begin{itemize}
\item {Grp. gram.:m.}
\end{itemize}
\begin{itemize}
\item {Proveniência:(De \textunderscore co...\textunderscore  + \textunderscore demandante\textunderscore )}
\end{itemize}
Aquelle que, juntamente com outrem, demanda alguém em juizo.
\section{Códeo}
\begin{itemize}
\item {Grp. gram.:m.}
\end{itemize}
\begin{itemize}
\item {Utilização:Prov.}
\end{itemize}
\begin{itemize}
\item {Utilização:trasm.}
\end{itemize}
Terreno endurecido pelo códão.
O mesmo que \textunderscore códo\textunderscore .
\section{Codesso}
\begin{itemize}
\item {fónica:dê}
\end{itemize}
\begin{itemize}
\item {Grp. gram.:m.}
\end{itemize}
(Fórma exacta em vez da usual, \textunderscore codeço\textunderscore )
\textunderscore m.\textunderscore  (e der.)
Fórma preferível a \textunderscore codeço\textunderscore , etc., em vista da etymologia.
\section{Codetentor}
\begin{itemize}
\item {Grp. gram.:m.}
\end{itemize}
\begin{itemize}
\item {Proveniência:(De \textunderscore co...\textunderscore  + \textunderscore detentor\textunderscore )}
\end{itemize}
Aquelle que, juntamente com outrem, detém em seu poder uma quantia, uma propriedade.
\section{Codeúdo}
\begin{itemize}
\item {Grp. gram.:adj.}
\end{itemize}
Que tem côdea grossa: \textunderscore pão codeúdo\textunderscore .
\section{Codevedor}
\begin{itemize}
\item {Grp. gram.:m.}
\end{itemize}
\begin{itemize}
\item {Proveniência:(De \textunderscore co...\textunderscore  + \textunderscore devedor\textunderscore )}
\end{itemize}
Aquelle que, juntamente com outrem, é responsável por uma divida.
\section{Códex}
\begin{itemize}
\item {Grp. gram.:m.}
\end{itemize}
O mesmo que \textunderscore códice\textunderscore .
\section{Co-dialecto}
\begin{itemize}
\item {Grp. gram.:m.}
\end{itemize}
Dialecto, que proveio de uma língua, juntamente com outra língua ou dialecto: \textunderscore o mirandês, o português e o austríaco são co-dialectos entre si, porque provêm do latim, se é que o mirandês não é dialecto asturiano. Neste caso, são co-dialectos o português e o asturiano\textunderscore .
\section{Códice}
\begin{itemize}
\item {Grp. gram.:m.}
\end{itemize}
\begin{itemize}
\item {Proveniência:(Lat. \textunderscore codex\textunderscore )}
\end{itemize}
Volume antigo e manuscrito.
Código antigo.
\section{Codicilar}
\begin{itemize}
\item {Grp. gram.:adj.}
\end{itemize}
Relativo a codicilo.
\section{Codicillar}
\begin{itemize}
\item {Grp. gram.:adj.}
\end{itemize}
Relativo a codicillo.
\section{Codicillo}
\begin{itemize}
\item {Grp. gram.:m.}
\end{itemize}
\begin{itemize}
\item {Proveniência:(Lat. \textunderscore codicilli\textunderscore , de \textunderscore codex\textunderscore )}
\end{itemize}
Alteração ou aditamento de um testamento.
\section{Codicilo}
\begin{itemize}
\item {Grp. gram.:m.}
\end{itemize}
\begin{itemize}
\item {Proveniência:(Lat. \textunderscore codicilli\textunderscore , de \textunderscore codex\textunderscore )}
\end{itemize}
Alteração ou aditamento de um testamento.
\section{Codicioso}
\begin{itemize}
\item {Grp. gram.:adj.}
\end{itemize}
\begin{itemize}
\item {Proveniência:(T. cast.)}
\end{itemize}
Diz-se do toiro que procura com afan o vulto do toireiro.
\section{Codificação}
\begin{itemize}
\item {Grp. gram.:f.}
\end{itemize}
Acto ou effeito de codificar.
\section{Codificador}
\begin{itemize}
\item {Grp. gram.:m.}
\end{itemize}
Aquelle que codifica.
\section{Codificar}
\begin{itemize}
\item {Grp. gram.:v. t.}
\end{itemize}
\begin{itemize}
\item {Utilização:Des.}
\end{itemize}
\begin{itemize}
\item {Proveniência:(Do lat. \textunderscore codex\textunderscore  + \textunderscore facere\textunderscore )}
\end{itemize}
Reduzir a código; reunir em código.
Encadernar. Cf. Castilho, \textunderscore Fastos\textunderscore , I, 323.
\section{Código}
\begin{itemize}
\item {Grp. gram.:m.}
\end{itemize}
\begin{itemize}
\item {Proveniência:(Lat. \textunderscore codex\textunderscore )}
\end{itemize}
Collecção de leis.
Compilação methódica e articulada de disposições legaes, relativas a um assumpto ou a um ramo de Direito.
Collecção autorizada de fórmulas médicas ou pharmacêuticas.
Reunião de preceitos de qualquer gênero.
Nórma.
\section{Codilhar}
\begin{itemize}
\item {Grp. gram.:v. t.}
\end{itemize}
\begin{itemize}
\item {Utilização:Fig.}
\end{itemize}
Dar codilho a.
Lograr.
Desbancar.
\section{Codilho}
\begin{itemize}
\item {Grp. gram.:m.}
\end{itemize}
\begin{itemize}
\item {Utilização:Fig.}
\end{itemize}
Incidente no jôgo do voltarete, em que o feito teve menos vasas que um dos parceiros.
Saliência na articulação superior da mão do cavallo.
Lôgro, engano.
(Or. cast.)
\section{Codinhar}
\begin{itemize}
\item {Grp. gram.:v. i.}
\end{itemize}
\begin{itemize}
\item {Utilização:Prov.}
\end{itemize}
\begin{itemize}
\item {Utilização:trasm.}
\end{itemize}
Comer o codinho.
\section{Codinho}
\begin{itemize}
\item {Grp. gram.:m.}
\end{itemize}
\begin{itemize}
\item {Utilização:Prov.}
\end{itemize}
\begin{itemize}
\item {Utilização:trasm.}
\end{itemize}
\begin{itemize}
\item {Proveniência:(De \textunderscore côdea\textunderscore )}
\end{itemize}
Pequena refeição ou parva, que os trabalhadores comem, no verão, pelas 10 horas.
\section{Codirector}
\begin{itemize}
\item {Grp. gram.:m.}
\end{itemize}
\begin{itemize}
\item {Proveniência:(De \textunderscore co...\textunderscore  + \textunderscore director\textunderscore )}
\end{itemize}
Aquelle que dirige, juntamente com outrem.
\section{Codito}
\begin{itemize}
\item {Grp. gram.:m.}
\end{itemize}
\begin{itemize}
\item {Utilização:Prov.}
\end{itemize}
\begin{itemize}
\item {Utilização:beir.}
\end{itemize}
\begin{itemize}
\item {Proveniência:(De \textunderscore côdea\textunderscore )}
\end{itemize}
Pequena côdea (de pão).
Pedacinho.
\section{Códo}
\begin{itemize}
\item {Grp. gram.:m.}
\end{itemize}
O mesmo que \textunderscore códão\textunderscore .
\section{Codonantho}
\begin{itemize}
\item {Grp. gram.:m.}
\end{itemize}
Gênero de plantas loganiáceas.
\section{Codonanto}
\begin{itemize}
\item {Grp. gram.:m.}
\end{itemize}
Gênero de plantas loganiáceas.
\section{Codonatário}
\begin{itemize}
\item {Grp. gram.:m.}
\end{itemize}
\begin{itemize}
\item {Proveniência:(De \textunderscore co...\textunderscore  + \textunderscore donatário\textunderscore )}
\end{itemize}
Aquelle que é associado a outrem numa doação que se lhes faz.
\section{Codonocarpo}
\begin{itemize}
\item {Grp. gram.:m.}
\end{itemize}
Gênero de plantas phytoláceas.
\section{Codonofone}
\begin{itemize}
\item {Grp. gram.:m.}
\end{itemize}
Instrumento, destinado a imitar os sons dos sinos, nos theatros, e composto de vinte e cinco tubos, que são percutidos por martelos, postos em movimento por meio de teclas.
\section{Codonofónio}
\begin{itemize}
\item {Grp. gram.:m.}
\end{itemize}
Instrumento, destinado a imitar os sons dos sinos, nos theatros, e composto de vinte e cinco tubos, que são percutidos por martelos, postos em movimento por meio de teclas.
\section{Codonophone}
\begin{itemize}
\item {Grp. gram.:m.}
\end{itemize}
Instrumento, destinado a imitar os sons dos sinos, nos theatros, e composto de vinte e cinco tubos, que são percutidos por martelos, postos em movimento por meio de teclas.
\section{Codó-qué}
\begin{itemize}
\item {Grp. gram.:m.}
\end{itemize}
Planta trepadeira da ilha de San-Thomé.
\section{Còdório}
\begin{itemize}
\item {Grp. gram.:m.}
\end{itemize}
\begin{itemize}
\item {Utilização:bras}
\end{itemize}
\begin{itemize}
\item {Utilização:Pop.}
\end{itemize}
Gole de vinho ou aguardente.
Pequena porção de alimento.
(Corr. burl. do lat. \textunderscore quod\textunderscore  + \textunderscore ore\textunderscore , palavras de uma locução, empregada no sacrifício da Missa)
\section{Codorna}
\begin{itemize}
\item {Grp. gram.:f.}
\end{itemize}
Ave do Brasil.
\section{Codorniz}
\begin{itemize}
\item {Grp. gram.:m.}
\end{itemize}
\begin{itemize}
\item {Proveniência:(Lat. \textunderscore codurnix\textunderscore )}
\end{itemize}
Ave gallinácea.
\section{Codornizão}
\begin{itemize}
\item {Grp. gram.:m.}
\end{itemize}
\begin{itemize}
\item {Proveniência:(De \textunderscore codorniz\textunderscore )}
\end{itemize}
Ave pernalta, (\textunderscore crex pratensis\textunderscore ).
\section{Codórno}
\begin{itemize}
\item {Grp. gram.:m.}
\end{itemize}
\begin{itemize}
\item {Utilização:Bras. do N}
\end{itemize}
Somneca.
Cochilo.
Sesta.
\section{Codôrno}
\begin{itemize}
\item {Grp. gram.:m.}
\end{itemize}
\begin{itemize}
\item {Utilização:Prov.}
\end{itemize}
\begin{itemize}
\item {Utilização:trasm.}
\end{itemize}
Espécie de pero grande.
Variedade de pêra, de que há mais de uma espécie.
Pedaço de pão, tirado da borda; canto de pão.
\section{Coeducação}
\begin{itemize}
\item {fónica:co-e}
\end{itemize}
\begin{itemize}
\item {Grp. gram.:f.}
\end{itemize}
\begin{itemize}
\item {Utilização:Neol.}
\end{itemize}
\begin{itemize}
\item {Proveniência:(De \textunderscore co...\textunderscore  + \textunderscore educação\textunderscore )}
\end{itemize}
Educação simultânea; educação em commum.
\section{Coefficiente}
\begin{itemize}
\item {Grp. gram.:m.}
\end{itemize}
\begin{itemize}
\item {Utilização:Arith.}
\end{itemize}
\begin{itemize}
\item {Proveniência:(De \textunderscore co...\textunderscore  + \textunderscore efficiente\textunderscore )}
\end{itemize}
Algarismo, que mostra quantas vezes se multiplica um termo.
Multiplicador.
\section{Coeficiente}
\begin{itemize}
\item {fónica:co-e}
\end{itemize}
\begin{itemize}
\item {Grp. gram.:m.}
\end{itemize}
\begin{itemize}
\item {Utilização:Arith.}
\end{itemize}
\begin{itemize}
\item {Proveniência:(De \textunderscore co...\textunderscore  + \textunderscore efficiente\textunderscore )}
\end{itemize}
Algarismo, que mostra quantas vezes se multiplica um termo.
Multiplicador.
\section{Coéfora}
\begin{itemize}
\item {Grp. gram.:f.}
\end{itemize}
\begin{itemize}
\item {Proveniência:(Do gr. \textunderscore khoe\textunderscore  + \textunderscore phoros\textunderscore )}
\end{itemize}
Mulher, que, entre os antigos Gregos, levava oferendas, destinadas aos mortos.
\section{Coeleitor}
\begin{itemize}
\item {fónica:co-e}
\end{itemize}
\begin{itemize}
\item {Grp. gram.:m.}
\end{itemize}
\begin{itemize}
\item {Proveniência:(De \textunderscore co...\textunderscore  + \textunderscore eleitor\textunderscore )}
\end{itemize}
Aquelle que com outrem goza do direito de eleitor.
\section{Coelhal}
\begin{itemize}
\item {Grp. gram.:adj.}
\end{itemize}
Relativo a coêlho, próprio de coêlho. Cf. Filinto, XIII, 210.
\section{Coelheira}
\begin{itemize}
\item {fónica:co-e}
\end{itemize}
\begin{itemize}
\item {Grp. gram.:f.}
\end{itemize}
Recinto, em que se criam coêlhos.
Vaso de barro, lura, em que se criam coêlhos.
\section{Coelheira}
\begin{itemize}
\item {fónica:co-e}
\end{itemize}
\begin{itemize}
\item {Grp. gram.:f.}
\end{itemize}
Parte dos arreios dos cavallos de tiro, que lhes cinge o pescoço.
(Cast. \textunderscore cuellera\textunderscore , de \textunderscore cuello\textunderscore , pescoço)
\section{Coelheiro}
\begin{itemize}
\item {fónica:co-e}
\end{itemize}
\begin{itemize}
\item {Grp. gram.:m.}
\end{itemize}
\begin{itemize}
\item {Grp. gram.:Adj.}
\end{itemize}
Caçador de coêlhos.
Que é bom caçador de coêlhos: \textunderscore cão coelheiro\textunderscore .
\section{Coêlho}
\begin{itemize}
\item {Grp. gram.:m.}
\end{itemize}
\begin{itemize}
\item {Proveniência:(Do lat. \textunderscore cuniculus\textunderscore )}
\end{itemize}
Animal mamífero, da ordem dos roedores.
Nome de um peixe.
\section{Coelva}
\begin{itemize}
\item {Grp. gram.:f.}
\end{itemize}
O mesmo que \textunderscore tanjasno\textunderscore .
\section{Coempção}
\begin{itemize}
\item {Grp. gram.:f.}
\end{itemize}
\begin{itemize}
\item {Proveniência:(Lat. \textunderscore coemptio\textunderscore )}
\end{itemize}
Compra em commum.
Compra recíproca.
\section{Coenha}
\begin{itemize}
\item {Grp. gram.:f.}
\end{itemize}
O mesmo que \textunderscore labaça\textunderscore ^1, (\textunderscore rumex pulcher\textunderscore , Lin.).
\section{Coentrada}
\begin{itemize}
\item {Grp. gram.:f.}
\end{itemize}
\begin{itemize}
\item {Proveniência:(De \textunderscore coentro\textunderscore )}
\end{itemize}
Môlho, adubado com coentros.
\section{Coentrela}
\begin{itemize}
\item {Grp. gram.:f.}
\end{itemize}
(V.pimpinela)
\section{Coentrilho}
\begin{itemize}
\item {Grp. gram.:m.}
\end{itemize}
Árvore rutácea do Brasil.
\section{Coentro}
\begin{itemize}
\item {Grp. gram.:m.}
\end{itemize}
\begin{itemize}
\item {Proveniência:(Do lat. \textunderscore coriandrum\textunderscore )}
\end{itemize}
Planta aromática, umbellífera, que se cultiva nas hortas.
\section{Coequação}
\begin{itemize}
\item {fónica:co-e}
\end{itemize}
\begin{itemize}
\item {Grp. gram.:f.}
\end{itemize}
\begin{itemize}
\item {Grp. gram.:F. pl.}
\end{itemize}
\begin{itemize}
\item {Proveniência:(De \textunderscore co...\textunderscore  + \textunderscore equação\textunderscore )}
\end{itemize}
Distribuição da parte proporcional a câda contribuinte.
Equações simultâneas em Álgebra.
\section{Coerana}
\begin{itemize}
\item {fónica:co-e}
\end{itemize}
\begin{itemize}
\item {Grp. gram.:f.}
\end{itemize}
Nome de várias plantas solâneas da América.
\section{Coerção}
\begin{itemize}
\item {fónica:co-er}
\end{itemize}
\begin{itemize}
\item {Grp. gram.:f.}
\end{itemize}
\begin{itemize}
\item {Proveniência:(Lat. \textunderscore coertio\textunderscore )}
\end{itemize}
Acto de coagir.
\section{Coercibilidade}
\begin{itemize}
\item {fónica:co-er}
\end{itemize}
\begin{itemize}
\item {Grp. gram.:f.}
\end{itemize}
Qualidade do que é coercível.
\section{Coercitivo}
\begin{itemize}
\item {fónica:co-er}
\end{itemize}
\begin{itemize}
\item {Grp. gram.:adj.}
\end{itemize}
(V.coercivo)
\section{Coercível}
\begin{itemize}
\item {fónica:co-er}
\end{itemize}
\begin{itemize}
\item {Grp. gram.:adj.}
\end{itemize}
\begin{itemize}
\item {Proveniência:(Do rad. do lat. \textunderscore coercere\textunderscore )}
\end{itemize}
Que póde sêr coagido, reprimido, encerrado em menor espaço.
\section{Coercivo}
\begin{itemize}
\item {fónica:co-er}
\end{itemize}
\begin{itemize}
\item {Grp. gram.:adj.}
\end{itemize}
\begin{itemize}
\item {Proveniência:(Do rad. do lat. \textunderscore coercere\textunderscore )}
\end{itemize}
Que coage.
Que reprime; que impõe pena.
\section{Coesposa}
\begin{itemize}
\item {fónica:co-es}
\end{itemize}
\begin{itemize}
\item {Grp. gram.:f.}
\end{itemize}
Freira que, juntamente com outras, se considera esposa de Christo:«\textunderscore Paula, que desejava estender uma bofetada de mão real a todas as caras de suas irmans em S. Bernardo e coesposas de Jesus...\textunderscore »Camillo, \textunderscore Caveira\textunderscore , 59.
\section{Coabitar}
\begin{itemize}
\item {Grp. gram.:v. t.}
\end{itemize}
\begin{itemize}
\item {Grp. gram.:V. i.}
\end{itemize}
Habitar em commum.
Viver em commum.
Viver intimamente.
Têr relações habituaes, licitas ou illicitas, com pessôa de outro sexo.
(B. lat. \textunderscore cohabitare\textunderscore )
\section{Coerdar}
\begin{itemize}
\item {Grp. gram.:v. t.  e  i.}
\end{itemize}
\begin{itemize}
\item {Proveniência:(De \textunderscore co...\textunderscore  + \textunderscore herdar\textunderscore )}
\end{itemize}
Herdar em commum.
\section{Coerdeiro}
\begin{itemize}
\item {Grp. gram.:m.}
\end{itemize}
\begin{itemize}
\item {Proveniência:(De \textunderscore co...\textunderscore  + \textunderscore herdeiro\textunderscore )}
\end{itemize}
Aquelle que herda com outrem.
\section{Coerência}
\begin{itemize}
\item {Grp. gram.:f.}
\end{itemize}
Estado do que é coerente.
\section{Coerente}
\begin{itemize}
\item {Grp. gram.:adj.}
\end{itemize}
\begin{itemize}
\item {Proveniência:(Lat. \textunderscore cohaerens\textunderscore )}
\end{itemize}
Em que há coesão, ligação recíproca.
Confórme, lógico, procedente.
\section{Coerentemente}
\begin{itemize}
\item {Grp. gram.:adv.}
\end{itemize}
De modo coerente.
\section{Coerido}
\begin{itemize}
\item {fónica:co-e}
\end{itemize}
\begin{itemize}
\item {Grp. gram.:adj.}
\end{itemize}
\begin{itemize}
\item {Proveniência:(De \textunderscore coherir\textunderscore )}
\end{itemize}
Ligado, aderente.
\section{Coerir}
\begin{itemize}
\item {fónica:co-e}
\end{itemize}
\begin{itemize}
\item {Grp. gram.:v. i.}
\end{itemize}
\begin{itemize}
\item {Proveniência:(Lat. \textunderscore cohaerere\textunderscore )}
\end{itemize}
Fazer coesão; aderir reciprocamente.
\section{Coesão}
\begin{itemize}
\item {Grp. gram.:f.}
\end{itemize}
\begin{itemize}
\item {Utilização:Fig.}
\end{itemize}
\begin{itemize}
\item {Proveniência:(Do lat. \textunderscore cohaesus\textunderscore )}
\end{itemize}
Ligação recíproca das moléculas dos corpos.
Harmonia, associação íntima.
\section{Coesivo}
\begin{itemize}
\item {Grp. gram.:adj.}
\end{itemize}
\begin{itemize}
\item {Proveniência:(Do lat. \textunderscore cohaesus\textunderscore )}
\end{itemize}
Que liga.
Em que há ligação, coesão.
\section{Coessência}
\begin{itemize}
\item {fónica:co-e}
\end{itemize}
\begin{itemize}
\item {Grp. gram.:f.}
\end{itemize}
\begin{itemize}
\item {Proveniência:(De \textunderscore co...\textunderscore  + \textunderscore essência\textunderscore )}
\end{itemize}
Essência commum.
\section{Coessencial}
\begin{itemize}
\item {fónica:co-e}
\end{itemize}
\begin{itemize}
\item {Grp. gram.:adj.}
\end{itemize}
\begin{itemize}
\item {Proveniência:(De \textunderscore coessência\textunderscore )}
\end{itemize}
Que tem essência commum.
\section{Coessencialmente}
\begin{itemize}
\item {fónica:co-e}
\end{itemize}
\begin{itemize}
\item {Grp. gram.:adv.}
\end{itemize}
Á maneira de coessencial.
\section{Coestender}
\begin{itemize}
\item {fónica:co-es}
\end{itemize}
\begin{itemize}
\item {Grp. gram.:v. t.}
\end{itemize}
\begin{itemize}
\item {Proveniência:(De \textunderscore co...\textunderscore  + \textunderscore estender\textunderscore )}
\end{itemize}
Estender juntamente com outrem.
\section{Coetâneo}
\begin{itemize}
\item {fónica:co-e}
\end{itemize}
\begin{itemize}
\item {Grp. gram.:adj.}
\end{itemize}
\begin{itemize}
\item {Proveniência:(Lat. \textunderscore coaetaneus\textunderscore )}
\end{itemize}
Contemporâneo.
\section{Coeternidade}
\begin{itemize}
\item {fónica:co-e}
\end{itemize}
\begin{itemize}
\item {Grp. gram.:f.}
\end{itemize}
Qualidade do que é \textunderscore coeterno\textunderscore .
\section{Coeterno}
\begin{itemize}
\item {fónica:co-e}
\end{itemize}
\begin{itemize}
\item {Grp. gram.:adj.}
\end{itemize}
\begin{itemize}
\item {Proveniência:(Lat. eccles. \textunderscore coaeternus\textunderscore )}
\end{itemize}
Que existe com outro desde sempre.
\section{Coévo}
\begin{itemize}
\item {Grp. gram.:adj.}
\end{itemize}
\begin{itemize}
\item {Proveniência:(Lat. \textunderscore coaevus\textunderscore )}
\end{itemize}
O mesmo que \textunderscore coetâneo\textunderscore .
\section{Coexistência}
\begin{itemize}
\item {fónica:co-e}
\end{itemize}
\begin{itemize}
\item {Grp. gram.:f.}
\end{itemize}
\begin{itemize}
\item {Proveniência:(De \textunderscore co...\textunderscore  + \textunderscore existência\textunderscore )}
\end{itemize}
Existência simultânea.
\section{Coexistente}
\begin{itemize}
\item {fónica:co-e}
\end{itemize}
\begin{itemize}
\item {Grp. gram.:adj.}
\end{itemize}
\begin{itemize}
\item {Proveniência:(De \textunderscore co...\textunderscore  + \textunderscore existente\textunderscore )}
\end{itemize}
Que coexiste.
\section{Coexistir}
\begin{itemize}
\item {fónica:co-e}
\end{itemize}
\begin{itemize}
\item {Grp. gram.:v. i.}
\end{itemize}
\begin{itemize}
\item {Proveniência:(De \textunderscore co...\textunderscore  + \textunderscore existir\textunderscore )}
\end{itemize}
Existir simultaneamente.
\section{Cofiador}
\begin{itemize}
\item {Grp. gram.:m.}
\end{itemize}
\begin{itemize}
\item {Proveniência:(De \textunderscore co...\textunderscore  + \textunderscore fiador\textunderscore )}
\end{itemize}
Aquelle que, com outrem, se tornou fiador do mesmo devedor e pela mesma dívida.
\section{Cofiar}
\begin{itemize}
\item {Grp. gram.:v. t.}
\end{itemize}
\begin{itemize}
\item {Proveniência:(Fr. \textunderscore coiffer\textunderscore )}
\end{itemize}
Afagar, alisar (cabello ou barba), passando a mão por ella.
\section{Cofinho}
\begin{itemize}
\item {Grp. gram.:m.}
\end{itemize}
\begin{itemize}
\item {Utilização:Prov.}
\end{itemize}
\begin{itemize}
\item {Utilização:beir.}
\end{itemize}
\begin{itemize}
\item {Utilização:Ant.}
\end{itemize}
\begin{itemize}
\item {Proveniência:(De \textunderscore côfo\textunderscore ^1)}
\end{itemize}
Cestinho de esparto ou vêrga, que, atado ao focinho dos animaes, lhes serve de açamo, e que, nos bois, os impede de lançar a bôca á verdura que encontram quando lavram.
\section{Cofino}
\begin{itemize}
\item {Grp. gram.:m.}
\end{itemize}
O mesmo que \textunderscore cofinho\textunderscore .
\section{Côfo}
\begin{itemize}
\item {Grp. gram.:m.}
\end{itemize}
\begin{itemize}
\item {Utilização:Bras}
\end{itemize}
\begin{itemize}
\item {Utilização:Prov.}
\end{itemize}
\begin{itemize}
\item {Utilização:minh.}
\end{itemize}
Espécie de cêsto oblongo de bôca estreita, em que os pescadores arrecadam o peixe.
O mesmo que \textunderscore cofinho\textunderscore . (Colhido em Barcelos)
(Cp. \textunderscore alcofa\textunderscore ^1)
\section{Côfo}
\begin{itemize}
\item {Grp. gram.:m.}
\end{itemize}
\begin{itemize}
\item {Utilização:Ant.}
\end{itemize}
O mesmo que \textunderscore pantufo\textunderscore . Cf. \textunderscore Livro das Vereações\textunderscore  de Braga, ms. de 1505.
\section{Côfo}
\begin{itemize}
\item {Grp. gram.:m.}
\end{itemize}
\begin{itemize}
\item {Utilização:Ant.}
\end{itemize}
Espécie de escudo. Cf. \textunderscore Peregrinação\textunderscore , CXIX; Tenreiro, I.
\section{Côfo}
\begin{itemize}
\item {Grp. gram.:m.}
\end{itemize}
\begin{itemize}
\item {Utilização:Bras}
\end{itemize}
O mesmo que \textunderscore abacá\textunderscore , ou \textunderscore cânhamo de Maníla\textunderscore .
\section{Cofre}
\begin{itemize}
\item {Grp. gram.:m.}
\end{itemize}
\begin{itemize}
\item {Proveniência:(Do gr. \textunderscore kophinos\textunderscore )}
\end{itemize}
Caixa de madeira ou metal, em que se guarda dinheiro ou outros objectos de valor.
\section{Cogelo}
\begin{itemize}
\item {Grp. gram.:m.}
\end{itemize}
Reptil africano.
\section{Cogiar}
\begin{itemize}
\item {Grp. gram.:v. i.}
\end{itemize}
\begin{itemize}
\item {Utilização:Prov.}
\end{itemize}
Observar, espreitar, analysar.
\section{Cogitabundo}
\begin{itemize}
\item {Grp. gram.:adj.}
\end{itemize}
\begin{itemize}
\item {Proveniência:(Lat. \textunderscore cogitabundus\textunderscore )}
\end{itemize}
Pensativo.
\section{Cogitação}
\begin{itemize}
\item {Grp. gram.:f.}
\end{itemize}
\begin{itemize}
\item {Proveniência:(Lat. \textunderscore cogitatio\textunderscore )}
\end{itemize}
Acto de cogitar.
\section{Cogitar}
\begin{itemize}
\item {Grp. gram.:v. t.  e  i.}
\end{itemize}
\begin{itemize}
\item {Proveniência:(Lat. \textunderscore cogitare\textunderscore )}
\end{itemize}
Fazer reflexão.
Imaginar.
Pensar muito.
\section{Cogitativo}
\begin{itemize}
\item {Grp. gram.:adj.}
\end{itemize}
O mesmo que \textunderscore cogitabundo\textunderscore .
\section{Cognação}
\begin{itemize}
\item {Grp. gram.:f.}
\end{itemize}
\begin{itemize}
\item {Proveniência:(Lat. \textunderscore cognatio\textunderscore )}
\end{itemize}
Parentesco pelo lado das mulheres.
\section{Cognado}
\begin{itemize}
\item {Grp. gram.:m.  e  adj.}
\end{itemize}
\begin{itemize}
\item {Proveniência:(Lat. \textunderscore cognatus\textunderscore )}
\end{itemize}
O que é parente por cognação.
\section{Cognato}
\begin{itemize}
\item {Grp. gram.:m.  e  adj.}
\end{itemize}
(V.cognado)
\section{Cognição}
\begin{itemize}
\item {Grp. gram.:f.}
\end{itemize}
\begin{itemize}
\item {Proveniência:(Lat. \textunderscore cognitio\textunderscore )}
\end{itemize}
Acto de adquirir um conhecimento.
\section{Cognitivo}
\begin{itemize}
\item {Grp. gram.:adj.}
\end{itemize}
\begin{itemize}
\item {Proveniência:(Do lat. \textunderscore cognitus\textunderscore )}
\end{itemize}
Relativo á cognição.
\section{Cógnito}
\begin{itemize}
\item {Grp. gram.:adj.}
\end{itemize}
\begin{itemize}
\item {Proveniência:(Lat. \textunderscore cognitus\textunderscore )}
\end{itemize}
O mesmo que \textunderscore conhecido\textunderscore . Cf. \textunderscore Lusíadas\textunderscore , I, 72.
\section{Cognome}
\begin{itemize}
\item {Grp. gram.:m.}
\end{itemize}
\begin{itemize}
\item {Proveniência:(Lat. \textunderscore cognomen\textunderscore )}
\end{itemize}
Appellido.
Alcunha.
\section{Cognomento}
\begin{itemize}
\item {Grp. gram.:m.}
\end{itemize}
(V.cognome)
\section{Cognominação}
\begin{itemize}
\item {Grp. gram.:f.}
\end{itemize}
Acto de cognominar.
\section{Cognominar}
\begin{itemize}
\item {Grp. gram.:v. i.}
\end{itemize}
\begin{itemize}
\item {Proveniência:(Lat. \textunderscore cognominare\textunderscore )}
\end{itemize}
Designar por cognome.
\section{Cognoscer}
\begin{itemize}
\item {Grp. gram.:v. i.}
\end{itemize}
\begin{itemize}
\item {Utilização:Ant.}
\end{itemize}
O mesmo que \textunderscore conhecer\textunderscore .
\section{Cognoscibilidade}
\begin{itemize}
\item {Grp. gram.:f.}
\end{itemize}
Qualidade do que é cognoscível.
\section{Cognoscitivo}
\begin{itemize}
\item {Grp. gram.:adj.}
\end{itemize}
\begin{itemize}
\item {Proveniência:(Do lat. \textunderscore cognoscitum\textunderscore )}
\end{itemize}
Que tem a faculdade de conhecer.
\section{Cognoscível}
\begin{itemize}
\item {Grp. gram.:adj.}
\end{itemize}
\begin{itemize}
\item {Proveniência:(Lat. \textunderscore cognoscibilis\textunderscore )}
\end{itemize}
Que se póde conhecer.
\section{Cogoilo}
\begin{itemize}
\item {Grp. gram.:m.}
\end{itemize}
Espécie de paquife, com que se decoram cornijas.
\section{Cogombral}
\begin{itemize}
\item {Grp. gram.:m.}
\end{itemize}
Terreno, onde crescem cogombros.
\section{Cogombro}
\begin{itemize}
\item {Grp. gram.:m.}
\end{itemize}
\begin{itemize}
\item {Proveniência:(Do lat. \textunderscore cucumer\textunderscore )}
\end{itemize}
O mesmo que \textunderscore pepino\textunderscore . Cf. Rezende, \textunderscore Cancion.\textunderscore 
\section{Cogote}
\begin{itemize}
\item {Grp. gram.:m.}
\end{itemize}
\begin{itemize}
\item {Utilização:Pop.}
\end{itemize}
\begin{itemize}
\item {Utilização:Bras}
\end{itemize}
Região occipital.
Cachaço.
\section{Cogotilho}
\begin{itemize}
\item {Grp. gram.:m.}
\end{itemize}
\begin{itemize}
\item {Utilização:Bras. do S}
\end{itemize}
\begin{itemize}
\item {Proveniência:(De \textunderscore cogote\textunderscore )}
\end{itemize}
Crinas do cavallo, cortadas de maneira, que ficam mais curtas entre as orelhas, do que ao longo do cachaço.
\section{Cogula}
\begin{itemize}
\item {Grp. gram.:f.}
\end{itemize}
\begin{itemize}
\item {Proveniência:(Do lat. eccles. \textunderscore cuculla\textunderscore )}
\end{itemize}
Túnica larga de alguns frades.
Casula.
\section{Cogular}
\textunderscore v. t.\textunderscore  (e der.)
(V. \textunderscore acogular\textunderscore , etc.)
\section{Cogulho}
\begin{itemize}
\item {Grp. gram.:m.}
\end{itemize}
O mesmo que \textunderscore cogoilo\textunderscore .
\section{Cogulo}
\begin{itemize}
\item {Grp. gram.:m.}
\end{itemize}
\begin{itemize}
\item {Utilização:Ant.}
\end{itemize}
\begin{itemize}
\item {Proveniência:(Do lat. \textunderscore cucullus\textunderscore )}
\end{itemize}
A parte dos cereaes, legumes secos ou outros gêneros, que fica acima das bordas de uma medida.
Acervo.
Demasia.
Medida quadrada, correspondente a um alqueire acogulado.
\section{Cogumelaria}
\begin{itemize}
\item {Grp. gram.:f.}
\end{itemize}
Subterrâneo, onde se faz cultura de cogumelos, usada principalmente em França.
\section{Cogumelo}
\begin{itemize}
\item {fónica:mê}
\end{itemize}
\begin{itemize}
\item {Grp. gram.:m.}
\end{itemize}
\begin{itemize}
\item {Proveniência:(Do gr. \textunderscore kokkumelon\textunderscore ?)}
\end{itemize}
Classe de plantas cryptogâmicas, designadas vulgarmente por \textunderscore tortulhos\textunderscore .
\section{Cohabitação}
\begin{itemize}
\item {Grp. gram.:f.}
\end{itemize}
Acto de cohabitar.
\section{Cohabitar}
\begin{itemize}
\item {Grp. gram.:v. t.}
\end{itemize}
\begin{itemize}
\item {Grp. gram.:V. i.}
\end{itemize}
Habitar em commum.
Viver em commum.
Viver intimamente.
Têr relações habituaes, licitas ou illicitas, com pessôa de outro sexo.
(B. lat. \textunderscore cohabitare\textunderscore )
\section{Coherdar}
\begin{itemize}
\item {Grp. gram.:v. t.  e  i.}
\end{itemize}
\begin{itemize}
\item {Proveniência:(De \textunderscore co...\textunderscore  + \textunderscore herdar\textunderscore )}
\end{itemize}
Herdar em commum.
\section{Coherdeiro}
\begin{itemize}
\item {Grp. gram.:m.}
\end{itemize}
\begin{itemize}
\item {Proveniência:(De \textunderscore co...\textunderscore  + \textunderscore herdeiro\textunderscore )}
\end{itemize}
Aquelle que herda com outrem.
\section{Coherência}
\begin{itemize}
\item {Grp. gram.:f.}
\end{itemize}
Estado do que é coherente.
\section{Coherente}
\begin{itemize}
\item {Grp. gram.:adj.}
\end{itemize}
\begin{itemize}
\item {Proveniência:(Lat. \textunderscore cohaerens\textunderscore )}
\end{itemize}
Em que há cohesão, ligação recíproca.
Confórme, lógico, procedente.
\section{Coherentemente}
\begin{itemize}
\item {Grp. gram.:adv.}
\end{itemize}
De modo coherente.
\section{Coherido}
\begin{itemize}
\item {Grp. gram.:adj.}
\end{itemize}
\begin{itemize}
\item {Proveniência:(De \textunderscore coherir\textunderscore )}
\end{itemize}
Ligado, adherente.
\section{Coherir}
\begin{itemize}
\item {Grp. gram.:v. i.}
\end{itemize}
\begin{itemize}
\item {Proveniência:(Lat. \textunderscore cohaerere\textunderscore )}
\end{itemize}
Fazer cohesão; adherir reciprocamente.
\section{Cohesão}
\begin{itemize}
\item {Grp. gram.:f.}
\end{itemize}
\begin{itemize}
\item {Utilização:Fig.}
\end{itemize}
\begin{itemize}
\item {Proveniência:(Do lat. \textunderscore cohaesus\textunderscore )}
\end{itemize}
Ligação recíproca das moléculas dos corpos.
Harmonia, associação íntima.
\section{Cohesivo}
\begin{itemize}
\item {Grp. gram.:adj.}
\end{itemize}
\begin{itemize}
\item {Proveniência:(Do lat. \textunderscore cohaesus\textunderscore )}
\end{itemize}
Que liga.
Em que há ligação, cohesão.
\section{Cohibição}
\begin{itemize}
\item {Grp. gram.:f.}
\end{itemize}
Acto de cohibir.
\section{Cohibir}
\begin{itemize}
\item {Grp. gram.:v. t.}
\end{itemize}
\begin{itemize}
\item {Proveniência:(Lat. \textunderscore cohibere\textunderscore )}
\end{itemize}
Reprimir, obstar a que continue.
Impedir (alguém) de que faça alguma coisa.
\section{Cohonestação}
\begin{itemize}
\item {Grp. gram.:f.}
\end{itemize}
Acto de cohonestar.
\section{Cohonestador}
\begin{itemize}
\item {Grp. gram.:adj.}
\end{itemize}
Que cohonesta.
\section{Cohonestar}
\begin{itemize}
\item {Grp. gram.:v. t.}
\end{itemize}
\begin{itemize}
\item {Proveniência:(Lat. \textunderscore cohonestare\textunderscore )}
\end{itemize}
Dar apparência de honesto a.
Rehabilitar.
\section{Cohorte}
\begin{itemize}
\item {Grp. gram.:f.}
\end{itemize}
\begin{itemize}
\item {Proveniência:(Lat. \textunderscore cohors\textunderscore )}
\end{itemize}
Parte de uma legião, entre os Romanos.
Porção de gente armada.
Magote.
\section{Cói}
\begin{itemize}
\item {Grp. gram.:m.}
\end{itemize}
\begin{itemize}
\item {Utilização:Pop.}
\end{itemize}
\begin{itemize}
\item {Utilização:Marit.}
\end{itemize}
\begin{itemize}
\item {Utilização:Des.}
\end{itemize}
O mesmo que \textunderscore coio\textunderscore ^2.
Pequeno lugar de agasalho, que nos navios se dá a cada marinheiro. Cf. Filinto, X, 138.
(Cp. holl. \textunderscore kooi\textunderscore , cama de bordo)
\section{Côi}
\begin{itemize}
\item {Grp. gram.:m.}
\end{itemize}
\begin{itemize}
\item {Utilização:Pop.}
\end{itemize}
\begin{itemize}
\item {Utilização:Marit.}
\end{itemize}
\begin{itemize}
\item {Utilização:Des.}
\end{itemize}
O mesmo que \textunderscore coio\textunderscore ^2.
Pequeno lugar de agasalho, que nos navios se dá a cada marinheiro. Cf. Filinto, X, 138.
(Cp. holl. \textunderscore kooi\textunderscore , cama de bordo)
\section{Cóia}
\begin{itemize}
\item {Grp. gram.:f.}
\end{itemize}
\begin{itemize}
\item {Utilização:Prov.}
\end{itemize}
\begin{itemize}
\item {Utilização:alg.}
\end{itemize}
\begin{itemize}
\item {Utilização:Prov.}
\end{itemize}
Mulher esperta e maliciosa.
Concubina.
Meretriz.
(Cp. \textunderscore cróia\textunderscore )
\section{Coibição}
\begin{itemize}
\item {fónica:co-i}
\end{itemize}
\begin{itemize}
\item {Grp. gram.:f.}
\end{itemize}
Acto de coibir.
\section{Coibir}
\begin{itemize}
\item {fónica:co-i}
\end{itemize}
\begin{itemize}
\item {Grp. gram.:v. t.}
\end{itemize}
\begin{itemize}
\item {Proveniência:(Lat. \textunderscore cohibere\textunderscore )}
\end{itemize}
Reprimir, obstar a que continue.
Impedir (alguém) de que faça alguma coisa.
\section{Coicão}
\begin{itemize}
\item {Grp. gram.:m.}
\end{itemize}
\begin{itemize}
\item {Utilização:Prov.}
\end{itemize}
\begin{itemize}
\item {Utilização:trasm.}
\end{itemize}
Cova, em que a perdiz faz o ninho.
(Talvez alter. \textunderscore concão\textunderscore , de \textunderscore conca\textunderscore )
\section{Coição}
\begin{itemize}
\item {Grp. gram.:m.}
\end{itemize}
\begin{itemize}
\item {Utilização:Prov.}
\end{itemize}
\begin{itemize}
\item {Proveniência:(De \textunderscore coice\textunderscore )}
\end{itemize}
O mesmo que \textunderscore coiceira\textunderscore ; coiceira grande.
\section{Coice}
\begin{itemize}
\item {Grp. gram.:m.}
\end{itemize}
\begin{itemize}
\item {Utilização:pop.}
\end{itemize}
\begin{itemize}
\item {Utilização:Fig.}
\end{itemize}
\begin{itemize}
\item {Proveniência:(Lat. \textunderscore calx\textunderscore )}
\end{itemize}
Rètaguarda, traseira, parte posterior de alguma coisa.
Dente da rabiça.
Calcanhar.
Pancada com o calcanhar, com o pé, com a pata.
Coiceira.
Brutalidade.
Ingratidão.
\section{Coicear}
\begin{itemize}
\item {Grp. gram.:v. t.  e  i.}
\end{itemize}
Dar coices.
\section{Coiceira}
\begin{itemize}
\item {Grp. gram.:f.}
\end{itemize}
\begin{itemize}
\item {Proveniência:(De \textunderscore coice\textunderscore )}
\end{itemize}
Coice da porta.
Parte da porta, em que se pregam os gonzos ou dobradiças.
Soleira da porta.
Variedade de uva preta da região do Doiro.
\section{Coiceiro}
\begin{itemize}
\item {Grp. gram.:m.}
\end{itemize}
\begin{itemize}
\item {Utilização:T. da Nazaré}
\end{itemize}
\begin{itemize}
\item {Grp. gram.:Adj.}
\end{itemize}
\begin{itemize}
\item {Utilização:Bras}
\end{itemize}
\begin{itemize}
\item {Proveniência:(De \textunderscore coice\textunderscore )}
\end{itemize}
Um dos homens que levantam a rede, e que trabalha atrás dos outros.
Que costuma dar coices.
\section{Coicieira}
\begin{itemize}
\item {Grp. gram.:f.}
\end{itemize}
Casta de uva do Doiro, também chamada \textunderscore coiceira\textunderscore .
\section{Coicil}
\begin{itemize}
\item {Grp. gram.:m.}
\end{itemize}
\begin{itemize}
\item {Utilização:Prov.}
\end{itemize}
\begin{itemize}
\item {Utilização:trasm.}
\end{itemize}
\begin{itemize}
\item {Proveniência:(De \textunderscore coice\textunderscore )}
\end{itemize}
Espigão de madeira na coiceira das portas, o qual gira sôbre o lado côncavo de um fundo de garrafa ou sôbre um tacão de sapato velho.
\section{Coicilhão}
\begin{itemize}
\item {Grp. gram.:m.}
\end{itemize}
\begin{itemize}
\item {Utilização:Prov.}
\end{itemize}
\begin{itemize}
\item {Utilização:trasm.}
\end{itemize}
\begin{itemize}
\item {Proveniência:(De \textunderscore coice\textunderscore )}
\end{itemize}
Peça, em que se embebem as entriteiras do carro.
\section{Coicilho}
\begin{itemize}
\item {Grp. gram.:m.}
\end{itemize}
\begin{itemize}
\item {Utilização:Prov.}
\end{itemize}
\begin{itemize}
\item {Utilização:trasm.}
\end{itemize}
O mesmo que \textunderscore coicil\textunderscore .
\section{Coiçoeira}
\begin{itemize}
\item {Grp. gram.:f.}
\end{itemize}
\begin{itemize}
\item {Utilização:Prov.}
\end{itemize}
\begin{itemize}
\item {Utilização:Bras}
\end{itemize}
\begin{itemize}
\item {Utilização:Prov.}
\end{itemize}
O mesmo que \textunderscore coiceira\textunderscore .
Pessôa estúpida.
\section{Coifa}
\begin{itemize}
\item {Grp. gram.:f.}
\end{itemize}
Rêde, com que se amparam as tranças das mulheres.
Cobertura da escorva, nas peças de artilharia.
Membrana, que envolve a cabeça do féto, ao nascer.
(Talvez do ant. alt. al. \textunderscore kupphja\textunderscore )
\section{Coigual}
\begin{itemize}
\item {fónica:co-i}
\end{itemize}
\begin{itemize}
\item {Grp. gram.:adj.}
\end{itemize}
\begin{itemize}
\item {Proveniência:(De \textunderscore co...\textunderscore  + \textunderscore igual\textunderscore )}
\end{itemize}
Diz-se, em Theologia, de cada uma das pessôas da Trindade, em relação ás outras.
\section{Coim}
\begin{itemize}
\item {Grp. gram.:m.}
\end{itemize}
\begin{itemize}
\item {Utilização:Prov.}
\end{itemize}
O mesmo que \textunderscore abibe\textunderscore . Cf. M. Paulino, \textunderscore Aves da Peninsula\textunderscore .
\section{Cóima}
\begin{itemize}
\item {Grp. gram.:f.}
\end{itemize}
\begin{itemize}
\item {Proveniência:(Do lat. \textunderscore calumnia\textunderscore )}
\end{itemize}
Multa.
Pena, que se impõe especialmente ao dono de gados que damnificam propriedade alheia.
\section{Coimar}
\begin{itemize}
\item {Grp. gram.:v. t.}
\end{itemize}
Impor cóima a.
\section{Coimável}
\begin{itemize}
\item {Grp. gram.:adj.}
\end{itemize}
\begin{itemize}
\item {Proveniência:(De \textunderscore coimar\textunderscore )}
\end{itemize}
Sujeito a cóima.
\section{Coimbrão}
\begin{itemize}
\item {Grp. gram.:m.}
\end{itemize}
\begin{itemize}
\item {Grp. gram.:Adj.}
\end{itemize}
\begin{itemize}
\item {Utilização:Fig.}
\end{itemize}
Indivíduo, natural de Coimbra.
Relativo a Coimbra.
\textunderscore Caminho coimbrão\textunderscore , ou \textunderscore estrada coimbran\textunderscore , rotina, costume geral, ramerrão. Cf. R. Lobo, \textunderscore Côrte na Aldeia\textunderscore , I, 39.
\section{Coimbrês}
\begin{itemize}
\item {Grp. gram.:adj.}
\end{itemize}
\begin{itemize}
\item {Utilização:Prov.}
\end{itemize}
\begin{itemize}
\item {Utilização:trasm.}
\end{itemize}
\begin{itemize}
\item {Proveniência:(De \textunderscore Coimbra\textunderscore , n. p.)}
\end{itemize}
Diz-se de uma variedade de feijão.
\section{Coimeiro}
\begin{itemize}
\item {Grp. gram.:adj.}
\end{itemize}
\begin{itemize}
\item {Grp. gram.:M.}
\end{itemize}
O mesmo que \textunderscore coimável\textunderscore .
Cobrador de cóimas.
\section{Cóina}
\begin{itemize}
\item {Grp. gram.:f.}
\end{itemize}
\begin{itemize}
\item {Utilização:Prov.}
\end{itemize}
\begin{itemize}
\item {Proveniência:(Do lat. \textunderscore cuna\textunderscore )}
\end{itemize}
Sertan.
\section{Còinar}
\begin{itemize}
\item {Grp. gram.:v. t.}
\end{itemize}
\begin{itemize}
\item {Utilização:Prov.}
\end{itemize}
\begin{itemize}
\item {Utilização:alent.}
\end{itemize}
Limpar com o cóino (o trigo na eira).
\section{Coincidência}
\begin{itemize}
\item {Grp. gram.:f.}
\end{itemize}
Acto de coincidir.
\section{Coincidente}
\begin{itemize}
\item {Grp. gram.:adj.}
\end{itemize}
Que coincide.
\section{Coincidir}
\begin{itemize}
\item {Grp. gram.:v. t.}
\end{itemize}
\begin{itemize}
\item {Proveniência:(Do lat. \textunderscore cum\textunderscore  + \textunderscore incidere\textunderscore )}
\end{itemize}
Ajustar-se exactamente, (falando-se de linhas ou superfícies, com dimensões e fórmas idênticas).
Succeder ao mesmo tempo.
Concordar.
\section{Coincidível}
\begin{itemize}
\item {Grp. gram.:adj.}
\end{itemize}
Que póde coincidir.
\section{Coindicação}
\begin{itemize}
\item {Grp. gram.:f.}
\end{itemize}
Acto de coindicar.
\section{Coindicante}
\begin{itemize}
\item {Grp. gram.:adj.}
\end{itemize}
Que coindica.
\section{Coindicar}
\begin{itemize}
\item {Grp. gram.:v. t.}
\end{itemize}
\begin{itemize}
\item {Proveniência:(De \textunderscore co...\textunderscore  + \textunderscore indicar\textunderscore )}
\end{itemize}
Indicar ao mesmo tempo.
\section{Cóino}
\begin{itemize}
\item {Grp. gram.:m.}
\end{itemize}
\begin{itemize}
\item {Utilização:Prov.}
\end{itemize}
\begin{itemize}
\item {Utilização:alent.}
\end{itemize}
Vassoira, feita das hastes sêcas de certas plantas, e com que nas eiras se separa do trigo o casulo e algum palhiço.
(Cp. \textunderscore coanha\textunderscore )
\section{Coinquinar}
\begin{itemize}
\item {Grp. gram.:v. t.}
\end{itemize}
\begin{itemize}
\item {Proveniência:(Lat. \textunderscore coinquinare\textunderscore )}
\end{itemize}
Inquinar de todos os lados; manchar completamente.
\section{Cointeressado}
\begin{itemize}
\item {Grp. gram.:adj.}
\end{itemize}
\begin{itemize}
\item {Proveniência:(De \textunderscore co...\textunderscore  + \textunderscore interessado\textunderscore )}
\end{itemize}
Que é interessado com outro.
\section{Coio}
\begin{itemize}
\item {Grp. gram.:adj.}
\end{itemize}
\begin{itemize}
\item {Utilização:T. de Gaia}
\end{itemize}
Ordinário, reles, que não presta para nada.
\section{Coio}
\begin{itemize}
\item {fónica:côiooucóio}
\end{itemize}
\begin{itemize}
\item {Grp. gram.:m.}
\end{itemize}
\begin{itemize}
\item {Utilização:Pop.}
\end{itemize}
Esconderijo.
Abrigo de malfeitores ou de gente suspeita.
Valhacoito.
(Alter. de \textunderscore cói\textunderscore )
\section{Coio}
\begin{itemize}
\item {Grp. gram.:m.}
\end{itemize}
Pequeno peixe do Brasil.
\section{Coiol}
\begin{itemize}
\item {Grp. gram.:m.}
\end{itemize}
Fruto de uma espécie de palmeira do México.
\section{Coiquinho}
\begin{itemize}
\item {Grp. gram.:m.}
\end{itemize}
\begin{itemize}
\item {Utilização:Prov.}
\end{itemize}
\begin{itemize}
\item {Utilização:trasm.}
\end{itemize}
Lugar de reunião, em que se murmura das vidas alheias.
(Cp. \textunderscore coio\textunderscore ^2)
\section{Coira}
\begin{itemize}
\item {Grp. gram.:f.}
\end{itemize}
\begin{itemize}
\item {Utilização:Ant.}
\end{itemize}
Antigo gibão de coiro, para guerreiros.
O mesmo que \textunderscore coiraça\textunderscore .
\section{Cóira}
\begin{itemize}
\item {Grp. gram.:f.}
\end{itemize}
\begin{itemize}
\item {Utilização:Prov.}
\end{itemize}
\begin{itemize}
\item {Utilização:trasm.}
\end{itemize}
O mesmo que \textunderscore cróia\textunderscore .
\section{Coiraça}
\begin{itemize}
\item {Grp. gram.:f.}
\end{itemize}
\begin{itemize}
\item {Utilização:Fig.}
\end{itemize}
\begin{itemize}
\item {Proveniência:(De \textunderscore coiro\textunderscore )}
\end{itemize}
Armadura para o peito.
Revestimento de navios com ferro ou outro metal.
Aquillo que serve de resguardo contra a maledicência ou contra a má sorte.
\section{Coiraçado}
\begin{itemize}
\item {Grp. gram.:adj.}
\end{itemize}
\begin{itemize}
\item {Grp. gram.:M.}
\end{itemize}
Revestido de metal: \textunderscore navio coiraçado\textunderscore .
Navio coiraçado.
\section{Coiraçar}
\begin{itemize}
\item {Grp. gram.:v. t.}
\end{itemize}
\begin{itemize}
\item {Proveniência:(De \textunderscore coiraça\textunderscore )}
\end{itemize}
Armar de coiraça.
Revestir de aço ou de outro metal (navios).
Abroquelar, proteger.
Tornar impassível.
\section{Coiraceiro}
\begin{itemize}
\item {Grp. gram.:m.}
\end{itemize}
Soldado, que tem coiraça.
\section{Coirama}
\begin{itemize}
\item {Grp. gram.:f.}
\end{itemize}
\begin{itemize}
\item {Utilização:Bras. do N}
\end{itemize}
\begin{itemize}
\item {Proveniência:(De \textunderscore coiro\textunderscore )}
\end{itemize}
Porção de coiros.
Vestuário de coiro, para vaqueiro.
\section{Coirana}
\begin{itemize}
\item {Grp. gram.:m.}
\end{itemize}
Variedade de cestro do Brasil.
\section{Coirão}
\begin{itemize}
\item {Grp. gram.:m.}
\end{itemize}
\begin{itemize}
\item {Utilização:Chul.}
\end{itemize}
\begin{itemize}
\item {Proveniência:(De \textunderscore coiro\textunderscore )}
\end{itemize}
Rameira velha.
Variedade de uva.
\section{Coirato}
\begin{itemize}
\item {Grp. gram.:m.}
\end{itemize}
\begin{itemize}
\item {Utilização:Prov.}
\end{itemize}
\begin{itemize}
\item {Utilização:trasm.}
\end{itemize}
Coiro de porco.
Pedaço de coiro duro.
\section{Coirear}
\begin{itemize}
\item {Grp. gram.:v. t.}
\end{itemize}
\begin{itemize}
\item {Utilização:Bras. do S}
\end{itemize}
Extrahir o coiro de (um animal).
\section{Coireiro}
\begin{itemize}
\item {Grp. gram.:m.}
\end{itemize}
Vendedor de coiros; samarreiro.
\section{Coirela}
\begin{itemize}
\item {Grp. gram.:f.}
\end{itemize}
\begin{itemize}
\item {Utilização:Ant.}
\end{itemize}
\begin{itemize}
\item {Utilização:Des.}
\end{itemize}
\begin{itemize}
\item {Proveniência:(Do b. lat. \textunderscore quarellus\textunderscore , contr. de \textunderscore quadrellus\textunderscore , do lat. \textunderscore quadrum\textunderscore )}
\end{itemize}
Porção de terra cultivável, longa e estreita.
Medida agrária, equivalente a 100 braças de comprimento e 10 de largura.
Casal.
\section{Coireleiro}
\begin{itemize}
\item {Grp. gram.:m.}
\end{itemize}
\begin{itemize}
\item {Proveniência:(De \textunderscore coirela\textunderscore )}
\end{itemize}
Aquelle que antigamente repartia as terras incultas, ou as conquistadas, pelos povoadores que iam arroteá-las ou habitá-las.
\section{Coirmão}
\begin{itemize}
\item {Grp. gram.:adj.}
\end{itemize}
\begin{itemize}
\item {Proveniência:(De \textunderscore co...\textunderscore  + \textunderscore irmão\textunderscore )}
\end{itemize}
Diz-se dos primos que são filhos de irmãos.
\section{Coiro}
\begin{itemize}
\item {Grp. gram.:m.}
\end{itemize}
\begin{itemize}
\item {Utilização:Fig.}
\end{itemize}
\begin{itemize}
\item {Utilização:Prov.}
\end{itemize}
\begin{itemize}
\item {Utilização:beir.}
\end{itemize}
\begin{itemize}
\item {Proveniência:(Lat. \textunderscore corium\textunderscore )}
\end{itemize}
Pelle dura de alguns animaes.
Pelle da cabeça humana.
Pelle.
Rameira desprezível e de idade madura.
\textunderscore Estar em coiro\textunderscore , estar nu.
\section{Coirona}
\begin{itemize}
\item {Grp. gram.:f.}
\end{itemize}
Rameira. Cf. Macedo, \textunderscore Burrão\textunderscore , 5.
\section{Coisa}
\begin{itemize}
\item {Grp. gram.:f.}
\end{itemize}
\begin{itemize}
\item {Grp. gram.:Pl.}
\end{itemize}
\begin{itemize}
\item {Proveniência:(Do lat. \textunderscore causa\textunderscore )}
\end{itemize}
Qualquer objecto inanimado.
Aquilo que existe ou póde existir.
Realidade; facto: \textunderscore essa é que é a coisa\textunderscore .
Negócio: \textunderscore tenho uma coisa a tratar\textunderscore .
Acto.
Causa.
Espécie.
Mistério: \textunderscore aí há coisa\textunderscore .
Bens.
\section{Coisada}
\begin{itemize}
\item {Grp. gram.:f.}
\end{itemize}
\begin{itemize}
\item {Utilização:Prov.}
\end{itemize}
\begin{itemize}
\item {Utilização:burl.}
\end{itemize}
Coisa que se não quer declarar: \textunderscore é cá uma coisada\textunderscore . (Colhido em Turquel)
\section{Coisa-feita}
\begin{itemize}
\item {Grp. gram.:f.}
\end{itemize}
\begin{itemize}
\item {Utilização:Bras}
\end{itemize}
Veneno, preparado e applicado com fórmulas de feitiçaria; feitiçaria.
\section{Coisar}
\begin{itemize}
\item {Grp. gram.:v. i.}
\end{itemize}
\begin{itemize}
\item {Utilização:Prov.}
\end{itemize}
\begin{itemize}
\item {Utilização:trasm.}
\end{itemize}
Fazer alguma coisa.
\section{Coiseiro}
\begin{itemize}
\item {Grp. gram.:m.}
\end{itemize}
\begin{itemize}
\item {Proveniência:(De \textunderscore coisa\textunderscore )}
\end{itemize}
Livro de notas e apontamentos, usado na Inquisição.
\section{Coisíssima}
\begin{itemize}
\item {Grp. gram.:f.}
\end{itemize}
\begin{itemize}
\item {Utilização:Fam.}
\end{itemize}
Us. na loc. \textunderscore coisíssima nenhuma\textunderscore , absolutamente nada.
\section{Coiso}
\begin{itemize}
\item {Grp. gram.:m.}
\end{itemize}
\begin{itemize}
\item {Utilização:Chul.}
\end{itemize}
Qualquer sujeito; fulano: \textunderscore aquelle coiso saiu-me um traste\textunderscore !
(Cp. \textunderscore coisa\textunderscore )
\section{Coita}
\begin{itemize}
\item {Grp. gram.:f.}
\end{itemize}
\begin{itemize}
\item {Utilização:Ant.}
\end{itemize}
Desgraça.
(Cast. \textunderscore cuita\textunderscore )
\section{Coitada}
\begin{itemize}
\item {Grp. gram.:f.}
\end{itemize}
\begin{itemize}
\item {Proveniência:(De \textunderscore côito\textunderscore )}
\end{itemize}
Terra defesa; cerrado.
\section{Coitado}
\begin{itemize}
\item {Grp. gram.:adj.}
\end{itemize}
\begin{itemize}
\item {Proveniência:(De \textunderscore coitar\textunderscore ^2)}
\end{itemize}
Desgraçado, mísero.
\section{Coitamento}
\begin{itemize}
\item {Grp. gram.:m.}
\end{itemize}
Acto de \textunderscore coitar\textunderscore ^1. Cp. Herculano, \textunderscore Hist. de Port.\textunderscore , IV, 272, 274, 276 e 279.
\section{Coitar}
\begin{itemize}
\item {Grp. gram.:v. t.}
\end{itemize}
\begin{itemize}
\item {Utilização:Ant.}
\end{itemize}
\begin{itemize}
\item {Proveniência:(De \textunderscore côito\textunderscore )}
\end{itemize}
Tornar defesa (uma propriedade), prohibindo a entrada nella, ou dando-lhe certos privilégios.
O mesmo que \textunderscore acoitar\textunderscore .
\section{Coitar}
\begin{itemize}
\item {Grp. gram.:v. t.}
\end{itemize}
\begin{itemize}
\item {Utilização:Ant.}
\end{itemize}
\begin{itemize}
\item {Proveniência:(De \textunderscore coita\textunderscore )}
\end{itemize}
Magoar; desgraçar.
\section{Coitaria}
\begin{itemize}
\item {Grp. gram.:f.}
\end{itemize}
Offício de coiteiro.
\section{Coiteiro}
\begin{itemize}
\item {Grp. gram.:m.}
\end{itemize}
\begin{itemize}
\item {Proveniência:(De \textunderscore côito\textunderscore )}
\end{itemize}
Aquelle que guarda as coitadas, os côitos.
\section{Coitelho}
\begin{itemize}
\item {fónica:tê}
\end{itemize}
\begin{itemize}
\item {Grp. gram.:m.}
\end{itemize}
\begin{itemize}
\item {Proveniência:(De \textunderscore côito\textunderscore )}
\end{itemize}
Pomar cercado.
Quinchoso; cerrado; chouso.
\section{Cóito}
\begin{itemize}
\item {Grp. gram.:m.}
\end{itemize}
\begin{itemize}
\item {Proveniência:(Lat. \textunderscore coitus\textunderscore )}
\end{itemize}
Cópula carnal.
\section{Côito}
\begin{itemize}
\item {Grp. gram.:m.}
\end{itemize}
\begin{itemize}
\item {Proveniência:(Do lat. \textunderscore cautus\textunderscore )}
\end{itemize}
Terra coitada, defesa, privilegiada.
\section{Coivara}
\begin{itemize}
\item {Grp. gram.:f.}
\end{itemize}
\begin{itemize}
\item {Utilização:Bras. do N}
\end{itemize}
\begin{itemize}
\item {Proveniência:(Do guar. \textunderscore kó\textunderscore  + \textunderscore ibá\textunderscore )}
\end{itemize}
Montículo de galhos ou gravetos, que não foram inteiramente queimados na roça a que se deitou fogo, e que se juntam, para serem reduzidos a cinza.
Fogueira.
\section{Coivarar}
\begin{itemize}
\item {Grp. gram.:v. t.}
\end{itemize}
\begin{itemize}
\item {Utilização:Bras}
\end{itemize}
\begin{itemize}
\item {Proveniência:(De \textunderscore coivara\textunderscore )}
\end{itemize}
Reunir em pilhas ou coivaras.
\section{Coixão}
\begin{itemize}
\item {Grp. gram.:m.}
\end{itemize}
\begin{itemize}
\item {Utilização:Prov.}
\end{itemize}
\begin{itemize}
\item {Utilização:beir.}
\end{itemize}
Perna de carneiro, de vitella, etc.
(Por \textunderscore coxão\textunderscore , de \textunderscore coxa\textunderscore )
\section{Cojuba}
\begin{itemize}
\item {Grp. gram.:f.}
\end{itemize}
\begin{itemize}
\item {Utilização:Bras}
\end{itemize}
Pequena cabaça.
\section{Cojubi}
\begin{itemize}
\item {Grp. gram.:m.}
\end{itemize}
\begin{itemize}
\item {Utilização:Bras}
\end{itemize}
Ave gallinácea do Amazonas.
\section{Cola}
\begin{itemize}
\item {Grp. gram.:f.}
\end{itemize}
\begin{itemize}
\item {Utilização:Des.}
\end{itemize}
Peugada; rasto.
Cauda.
(Cast. \textunderscore cola\textunderscore )
\section{Cola}
\begin{itemize}
\item {Grp. gram.:f.}
\end{itemize}
Espécie de castanha medicinal.
Árvore intertropical, espécie de estercúlia.
O mesmo que \textunderscore coleira\textunderscore ^1.
\section{Cola}
\begin{itemize}
\item {Grp. gram.:f.}
\end{itemize}
\begin{itemize}
\item {Utilização:Prov.}
\end{itemize}
Rebo, que se mete por baixo dos calhaus, para os tombar melhor. (Colhido em Vouzela)
\section{Cola}
\begin{itemize}
\item {Grp. gram.:f.}
\end{itemize}
(?):«\textunderscore ...trazia cobertas, e testeiras, e colas, com que se ornão os cavallos.\textunderscore »Filinto, \textunderscore D. Man.\textunderscore , II, 296.
\section{Cola-amarga}
\begin{itemize}
\item {Grp. gram.:f.}
\end{itemize}
Árvore da Guiné, de cujas sementes se extrai um óleo amargo, medicinal.
\section{Colada}
\begin{itemize}
\item {Grp. gram.:f.}
\end{itemize}
\begin{itemize}
\item {Utilização:Prov.}
\end{itemize}
Intestinos da rês; fressura.
(Cp. \textunderscore colhada\textunderscore )
\section{Colafizamento}
\begin{itemize}
\item {Grp. gram.:m.}
\end{itemize}
Acto de colafizar.
\section{Colafizar}
\begin{itemize}
\item {Grp. gram.:v. t.}
\end{itemize}
\begin{itemize}
\item {Utilização:Des.}
\end{itemize}
\begin{itemize}
\item {Proveniência:(Gr. \textunderscore kolaphizein\textunderscore )}
\end{itemize}
Esbofetear.
Provocar.
\section{Colagogo}
\begin{itemize}
\item {Grp. gram.:adj.}
\end{itemize}
\begin{itemize}
\item {Grp. gram.:M.}
\end{itemize}
\begin{itemize}
\item {Proveniência:(Do gr. \textunderscore khole\textunderscore  + \textunderscore agein\textunderscore )}
\end{itemize}
Que faz segregar a bílis do fígado.
Que actua sôbre o aparelho biliário.
Medicamento colagogo.
\section{Colao}
\begin{itemize}
\item {Grp. gram.:m.}
\end{itemize}
\begin{itemize}
\item {Proveniência:(Do chin. \textunderscore ko lao\textunderscore , de \textunderscore ko\textunderscore , câmara, e \textunderscore lao\textunderscore , ancião)}
\end{itemize}
Espécie de Ministro de Estado, na China.
\section{Colaphizamento}
\begin{itemize}
\item {Grp. gram.:m.}
\end{itemize}
Acto de colaphizar.
\section{Colaphizar}
\begin{itemize}
\item {Grp. gram.:v. t.}
\end{itemize}
\begin{itemize}
\item {Utilização:Des.}
\end{itemize}
\begin{itemize}
\item {Proveniência:(Gr. \textunderscore kolaphizein\textunderscore )}
\end{itemize}
Esbofetear.
Provocar.
\section{Colários}
\begin{itemize}
\item {Grp. gram.:m. pl.}
\end{itemize}
O mesmo que \textunderscore mundas\textunderscore .
\section{Colcha}
\begin{itemize}
\item {fónica:côl}
\end{itemize}
\begin{itemize}
\item {Grp. gram.:f.}
\end{itemize}
\begin{itemize}
\item {Proveniência:(Do lat. \textunderscore culcita\textunderscore )}
\end{itemize}
Coberta de cama, estampada ou com lavores.
\section{Colchão}
\begin{itemize}
\item {Grp. gram.:m.}
\end{itemize}
\begin{itemize}
\item {Proveniência:(De \textunderscore colcha\textunderscore )}
\end{itemize}
Grande almofada ou coxim, cheio de substância flexível e segura com bastas, que se estende sôbre o enxergão ordinariamente.
\section{Colcheia}
\begin{itemize}
\item {Grp. gram.:f.}
\end{itemize}
\begin{itemize}
\item {Proveniência:(Fr. \textunderscore croche\textunderscore )}
\end{itemize}
Nota de música, que vale metade de uma semínima.
\section{Colcheta}
\begin{itemize}
\item {fónica:chê}
\end{itemize}
\begin{itemize}
\item {Grp. gram.:f.}
\end{itemize}
\begin{itemize}
\item {Utilização:Prov.}
\end{itemize}
\begin{itemize}
\item {Proveniência:(De \textunderscore colchete\textunderscore )}
\end{itemize}
Argolinha, mais ou menos em fórma de lyra, na qual se engancha o colchete.
\section{Colchete}
\begin{itemize}
\item {fónica:chê}
\end{itemize}
\begin{itemize}
\item {Grp. gram.:m.}
\end{itemize}
\begin{itemize}
\item {Utilização:Fam.}
\end{itemize}
\begin{itemize}
\item {Proveniência:(Fr. \textunderscore crochet\textunderscore )}
\end{itemize}
Pequena abotoadura de metal, de extremidade voltada, em fórma de gancho.
Gancho duplo, em que os açougueiros penduram a carne que expõem á venda.
Companheiro íntimo. Cf. Castilho, \textunderscore Fausto\textunderscore , 292.
Sinal orthográphico, o mesmo que \textunderscore chave\textunderscore .
O mesmo que \textunderscore cogoilo\textunderscore .
\section{Colchicáceas}
\begin{itemize}
\item {fónica:qui}
\end{itemize}
\begin{itemize}
\item {Grp. gram.:f. pl.}
\end{itemize}
Família de plantas herbáceas, que têm por typo o \textunderscore cólchico\textunderscore .
\section{Colchicina}
\begin{itemize}
\item {fónica:qui}
\end{itemize}
\begin{itemize}
\item {Grp. gram.:f.}
\end{itemize}
\begin{itemize}
\item {Proveniência:(De \textunderscore cólchico\textunderscore )}
\end{itemize}
Alcaloide, que se descobriu nas sementes do cólchico.
\section{Cólchico}
\begin{itemize}
\item {fónica:qui}
\end{itemize}
\begin{itemize}
\item {Grp. gram.:m.}
\end{itemize}
\begin{itemize}
\item {Proveniência:(De \textunderscore Colchor\textunderscore , n. p.)}
\end{itemize}
Lírio verde, (\textunderscore colchicum\textunderscore ).
\section{Colchoar}
\textunderscore v. t.\textunderscore  (e der.)
(V. \textunderscore acolchoar\textunderscore , etc.)
\section{Colchoaria}
\begin{itemize}
\item {Grp. gram.:f.}
\end{itemize}
Estabelecimento, onde se fabricam ou se vendem colchões.
\section{Colchoeiro}
\begin{itemize}
\item {Grp. gram.:m.}
\end{itemize}
Aquelle que faz ou vende colchões.
\section{Colcorrhiza}
\begin{itemize}
\item {Grp. gram.:f.}
\end{itemize}
\begin{itemize}
\item {Utilização:Bot.}
\end{itemize}
\begin{itemize}
\item {Proveniência:(Do gr. \textunderscore kolkos\textunderscore  + \textunderscore rhiza\textunderscore )}
\end{itemize}
Espécie de estojo, que cobre a extremidade das radículas.
\section{Colcorriza}
\begin{itemize}
\item {Grp. gram.:f.}
\end{itemize}
\begin{itemize}
\item {Utilização:Bot.}
\end{itemize}
\begin{itemize}
\item {Proveniência:(Do gr. \textunderscore kolkos\textunderscore  + \textunderscore rhiza\textunderscore )}
\end{itemize}
Espécie de estojo, que cobre a extremidade das radículas.
\section{Colcotar}
\begin{itemize}
\item {Grp. gram.:m.}
\end{itemize}
\begin{itemize}
\item {Proveniência:(Fr. \textunderscore colcotar\textunderscore )}
\end{itemize}
Peróxydo de ferro.
\section{Coldrado}
\begin{itemize}
\item {Grp. gram.:m.}
\end{itemize}
\begin{itemize}
\item {Proveniência:(De \textunderscore coldre\textunderscore ?)}
\end{itemize}
Imposto antigo. Cf. Herculano, \textunderscore Hist. de Port.\textunderscore , IV, 430.
\section{Coldre}
\begin{itemize}
\item {Grp. gram.:m.}
\end{itemize}
\begin{itemize}
\item {Utilização:Prov.}
\end{itemize}
\begin{itemize}
\item {Utilização:beir.}
\end{itemize}
\begin{itemize}
\item {Proveniência:(Do lat. \textunderscore corytus\textunderscore ?)}
\end{itemize}
Cada um dos dois estojos ou sacos de sola, pendentes do arção da sella, para trazer pistolas ou outras armas.
Rameira, mulher pública, coiro.
\section{Cole}
\begin{itemize}
\item {Grp. gram.:m.}
\end{itemize}
Colono, índio ou chinês, em colónias europeias.
(Do tamil \textunderscore kuli\textunderscore ?)
\section{Colédoco}
\begin{itemize}
\item {Grp. gram.:adj.}
\end{itemize}
\begin{itemize}
\item {Proveniência:(Gr. \textunderscore kholedokos\textunderscore )}
\end{itemize}
Diz-se do canal, que leva a bílis ao duodeno.
\section{Colegado}
\begin{itemize}
\item {Grp. gram.:m.}
\end{itemize}
\begin{itemize}
\item {Proveniência:(De \textunderscore co...\textunderscore  + \textunderscore legado\textunderscore ^2)}
\end{itemize}
Legado, que se transmitte com outro ou outros. Cf. Herculano, \textunderscore Hist. de Port.\textunderscore , II, 283.
\section{Colegatário}
\begin{itemize}
\item {Grp. gram.:m.}
\end{itemize}
\begin{itemize}
\item {Proveniência:(De \textunderscore co...\textunderscore  + \textunderscore legatário\textunderscore )}
\end{itemize}
Aquelle que é legatário com outrem.
\section{Colegislativo}
\begin{itemize}
\item {Grp. gram.:adj.}
\end{itemize}
\begin{itemize}
\item {Proveniência:(De \textunderscore co...\textunderscore  + \textunderscore legislativo\textunderscore )}
\end{itemize}
Diz-se das duas câmaras que constituem o Parlamento nalguns países.
\section{Coleira}
\begin{itemize}
\item {Grp. gram.:f.}
\end{itemize}
Árvore esterculiácea das possessões portuguesas da África occidental, (\textunderscore cola acuminata\textunderscore , R. Br.).
\section{Coleira}
\begin{itemize}
\item {Grp. gram.:m.}
\end{itemize}
\begin{itemize}
\item {Utilização:Bras. do N}
\end{itemize}
Espécie de carrapato.
Indivíduo velhaco; mau pagador.
\section{Colélito}
\begin{itemize}
\item {Grp. gram.:m.}
\end{itemize}
\begin{itemize}
\item {Utilização:Med.}
\end{itemize}
\begin{itemize}
\item {Proveniência:(Do gr. \textunderscore khole\textunderscore  + \textunderscore lithos\textunderscore )}
\end{itemize}
Cálculo biliário.
\section{Colelogia}
\begin{itemize}
\item {Grp. gram.:f.}
\end{itemize}
\begin{itemize}
\item {Proveniência:(Do gr. \textunderscore khole\textunderscore  + \textunderscore logos\textunderscore )}
\end{itemize}
Tratado á cêrca da bílis.
\section{Colendíssimo}
\begin{itemize}
\item {Grp. gram.:adj.}
\end{itemize}
\begin{itemize}
\item {Utilização:P. us.}
\end{itemize}
Muito respeitável.
(Sup. de \textunderscore colendo\textunderscore )
\section{Colendo}
\begin{itemize}
\item {Grp. gram.:adj.}
\end{itemize}
\begin{itemize}
\item {Proveniência:(Lat. \textunderscore colendus\textunderscore )}
\end{itemize}
Respeitável.
Venerando.
\section{Cóleo}
\begin{itemize}
\item {Grp. gram.:m.}
\end{itemize}
\begin{itemize}
\item {Proveniência:(Gr. \textunderscore koleos\textunderscore )}
\end{itemize}
Gênero de plantas ornamentaes, labiadas, procedente dos climas quentes.
\section{Coleoderme}
\begin{itemize}
\item {Grp. gram.:adj.}
\end{itemize}
\begin{itemize}
\item {Proveniência:(Do gr. \textunderscore koleos\textunderscore  + \textunderscore derma\textunderscore )}
\end{itemize}
Diz-se de certos animaes, cujo envoltório natural é uma espécie de saco.
\section{Coleofila}
\begin{itemize}
\item {Grp. gram.:f.}
\end{itemize}
\begin{itemize}
\item {Utilização:Bot.}
\end{itemize}
\begin{itemize}
\item {Proveniência:(Do gr. \textunderscore koleos\textunderscore  + \textunderscore phullon\textunderscore )}
\end{itemize}
Baínha membranosa na base da plúmula.
\section{Coleófita}
\begin{itemize}
\item {Grp. gram.:f.}
\end{itemize}
\begin{itemize}
\item {Proveniência:(Do gr. \textunderscore koleos\textunderscore  + \textunderscore phuton\textunderscore )}
\end{itemize}
Planta ornamental, de fôlhas lanceoladas e estriadas de branco.
\section{Coleophylla}
\begin{itemize}
\item {Grp. gram.:f.}
\end{itemize}
\begin{itemize}
\item {Utilização:Bot.}
\end{itemize}
\begin{itemize}
\item {Proveniência:(Do gr. \textunderscore koleos\textunderscore  + \textunderscore phullon\textunderscore )}
\end{itemize}
Baínha membranosa na base da plúmula.
\section{Coleóphyta}
\begin{itemize}
\item {Grp. gram.:f.}
\end{itemize}
\begin{itemize}
\item {Proveniência:(Do gr. \textunderscore koleos\textunderscore  + \textunderscore phuton\textunderscore )}
\end{itemize}
Planta ornamental, de fôlhas lanceoladas e estriadas de branco.
\section{Coleópode}
\begin{itemize}
\item {Grp. gram.:adj.}
\end{itemize}
\begin{itemize}
\item {Proveniência:(Do gr. \textunderscore koleos\textunderscore  + \textunderscore pous\textunderscore , \textunderscore podos\textunderscore )}
\end{itemize}
Diz-se de certos animaes, que têm os pés como occultos num estojo.
\section{Coleopterologia}
\begin{itemize}
\item {Grp. gram.:f.}
\end{itemize}
Tratado á cêrca dos coleópteros.
\section{Coleopterólogo}
\begin{itemize}
\item {Grp. gram.:m.}
\end{itemize}
Naturalista, que se occupa de coleopterologia.
\section{Coleópteros}
\begin{itemize}
\item {Grp. gram.:m. pl.}
\end{itemize}
\begin{itemize}
\item {Proveniência:(Do gr. \textunderscore koleos\textunderscore  + \textunderscore pteron\textunderscore )}
\end{itemize}
Ordem de insectos, cujas asas superiores abrigam as inferiores.
\section{Coleóptila}
\begin{itemize}
\item {Grp. gram.:f.}
\end{itemize}
\begin{itemize}
\item {Proveniência:(Do gr. \textunderscore koleos\textunderscore  + \textunderscore ptilon\textunderscore )}
\end{itemize}
O mesmo que \textunderscore coleophylla\textunderscore .
\section{Cólera}
\begin{itemize}
\item {Grp. gram.:f.}
\end{itemize}
\begin{itemize}
\item {Utilização:Fig.}
\end{itemize}
\begin{itemize}
\item {Proveniência:(Lat. \textunderscore cholera\textunderscore , gr. \textunderscore kholera\textunderscore , goteira)}
\end{itemize}
Paixão, irritação, produzida por offensa, ou por um facto que indigna.
Sentimento de justiça, que se attribue a Deus.
Ferocidade (de animaes).
Doença, caracterizada por grandes evacuações, fraqueza e resfriamento.
O mesmo que \textunderscore chólera\textunderscore .(V.chólera)
Ímpeto.
\section{Cólera-morbo}
\begin{itemize}
\item {Grp. gram.:f.}
\end{itemize}
O mesmo que \textunderscore cólera\textunderscore , doença.
\section{Colericamente}
\begin{itemize}
\item {Grp. gram.:adv.}
\end{itemize}
De modo colérico.
Com cólera.
\section{Colérico}
\begin{itemize}
\item {Grp. gram.:adj.}
\end{itemize}
\begin{itemize}
\item {Grp. gram.:M.}
\end{itemize}
\begin{itemize}
\item {Proveniência:(Lat. \textunderscore cholericus\textunderscore )}
\end{itemize}
Propenso a encolerizar-se.
Encolerizado.
Atacado da doença da cólera.
Relativo á \textunderscore cólera\textunderscore .
Aquelle que é atacado da doença chamada \textunderscore cólera\textunderscore .(V.cholérico)
\section{Coleriforme}
\begin{itemize}
\item {Grp. gram.:adj.}
\end{itemize}
\begin{itemize}
\item {Proveniência:(Do lat. \textunderscore cholera\textunderscore  + \textunderscore forma\textunderscore )}
\end{itemize}
Que tem semelhança com a cólera, doença.
\section{Colerígeno}
\begin{itemize}
\item {Grp. gram.:adj.}
\end{itemize}
\begin{itemize}
\item {Proveniência:(Do gr. \textunderscore kholera\textunderscore  + \textunderscore genos\textunderscore )}
\end{itemize}
Que produz cólera-morbo.
\section{Colerina}
\begin{itemize}
\item {Grp. gram.:f.}
\end{itemize}
\begin{itemize}
\item {Proveniência:(De \textunderscore cólera\textunderscore )}
\end{itemize}
Espécie de cólera-morbo, mas mais benigna ou atenuada.
\section{Colerínico}
\begin{itemize}
\item {Grp. gram.:adj.}
\end{itemize}
\begin{itemize}
\item {Grp. gram.:M.}
\end{itemize}
Relativo á colerina.
Aquelle que padece colerina.
\section{Colerogênico}
\begin{itemize}
\item {Grp. gram.:adj.}
\end{itemize}
O mesmo que \textunderscore colerígeno\textunderscore .
\section{Colestearina}
\begin{itemize}
\item {Grp. gram.:f.}
\end{itemize}
\begin{itemize}
\item {Proveniência:(Do gr. \textunderscore khole\textunderscore  + \textunderscore stear\textunderscore )}
\end{itemize}
A gordura da bílis.
\section{Colesteatoma}
\begin{itemize}
\item {Grp. gram.:m.}
\end{itemize}
\begin{itemize}
\item {Utilização:Med.}
\end{itemize}
\begin{itemize}
\item {Proveniência:(Do gr. \textunderscore khole\textunderscore  + \textunderscore stedo\textunderscore )}
\end{itemize}
Lipoma, formado pela sobreposição de vesículas adiposas, entre as quaes há uma substância, composta de colesterina e estearina.
\section{Colesterato}
\begin{itemize}
\item {Grp. gram.:m.}
\end{itemize}
\begin{itemize}
\item {Proveniência:(De \textunderscore cholestérico\textunderscore )}
\end{itemize}
Gênero de saes, formados pelo ácido colestérico.
\section{Colestérico}
\begin{itemize}
\item {Grp. gram.:adj.}
\end{itemize}
Diz-se de um ácido, formado pela reacção do ácido azótico sôbre a colesterina.
(Cp. \textunderscore colesterina\textunderscore )
\section{Colesterina}
\begin{itemize}
\item {Grp. gram.:f.}
\end{itemize}
\begin{itemize}
\item {Proveniência:(Do gr. \textunderscore khole\textunderscore  + \textunderscore steao\textunderscore )}
\end{itemize}
Substância cristalizada dos cálculos biliários humanos.
\section{Coleta}
\begin{itemize}
\item {fónica:lê}
\end{itemize}
\begin{itemize}
\item {Grp. gram.:f.}
\end{itemize}
Trança de cabello, que os toireiros espanhóes usam na parte posterior da cabeça.
(Cast. \textunderscore coleta\textunderscore )
\section{Colga}
\begin{itemize}
\item {Grp. gram.:adj. f.}
\end{itemize}
\begin{itemize}
\item {Utilização:Prov.}
\end{itemize}
\begin{itemize}
\item {Utilização:trasm.}
\end{itemize}
Diz-se da mulher preguiçosa.
\section{Colgadura}
\begin{itemize}
\item {Grp. gram.:f.}
\end{itemize}
\begin{itemize}
\item {Proveniência:(De \textunderscore colgar\textunderscore )}
\end{itemize}
Estôfo, que se pendura nas paredes ou janelas, para as cobrir e ornar.
\section{Colgalho}
\begin{itemize}
\item {Grp. gram.:m.}
\end{itemize}
\begin{itemize}
\item {Utilização:Prov.}
\end{itemize}
\begin{itemize}
\item {Utilização:trasm.}
\end{itemize}
\begin{itemize}
\item {Proveniência:(De \textunderscore colgar\textunderscore )}
\end{itemize}
Dependura de uvas ou de outros frutos.
\section{Colgar}
\begin{itemize}
\item {Grp. gram.:v. t.}
\end{itemize}
\begin{itemize}
\item {Utilização:fig.}
\end{itemize}
\begin{itemize}
\item {Utilização:Ant.}
\end{itemize}
\begin{itemize}
\item {Proveniência:(Do lat. \textunderscore collocare\textunderscore )}
\end{itemize}
Pendurar.
Ornar com colgaduras.
Enforcar.
\section{Colhada}
\begin{itemize}
\item {Grp. gram.:f.}
\end{itemize}
\begin{itemize}
\item {Utilização:Pop.}
\end{itemize}
Fressura ou intestinos de animaes.
(Cp. \textunderscore colada\textunderscore ^1)
\section{Colhareiro}
\begin{itemize}
\item {Grp. gram.:m.}
\end{itemize}
\begin{itemize}
\item {Utilização:Ant.}
\end{itemize}
Caixa, onde se guardavam colheres.
(Por \textunderscore colhereiro\textunderscore , de \textunderscore colhér\textunderscore )
\section{Colhedeira}
\begin{itemize}
\item {Grp. gram.:f.}
\end{itemize}
\begin{itemize}
\item {Proveniência:(De \textunderscore colhêr\textunderscore )}
\end{itemize}
Utensílio de pau, com que os pintores reúnem as tintas que móem.
\section{Colhedor}
\begin{itemize}
\item {Grp. gram.:m.  e  adj.}
\end{itemize}
\begin{itemize}
\item {Proveniência:(De \textunderscore colhêr\textunderscore )}
\end{itemize}
O que colhe; o que recebe.
\section{Colheiceiro}
\begin{itemize}
\item {Grp. gram.:m.}
\end{itemize}
\begin{itemize}
\item {Utilização:Ant.}
\end{itemize}
Rendeiro, que cobrava as colheitas do rei.
(Talvez por \textunderscore colhenceiro\textunderscore , de \textunderscore colhença\textunderscore )
\section{Colheira}
\begin{itemize}
\item {Grp. gram.:f.}
\end{itemize}
\begin{itemize}
\item {Utilização:Prov.}
\end{itemize}
Espécie de almofada de palha, em volta do pescoço das bêstas de tiro, sôbre a qual assenta o furcate.
O mesmo que \textunderscore coelheira\textunderscore ^2.--É castelhanismo, substituível por \textunderscore colleira\textunderscore .
(Cast. \textunderscore collera\textunderscore )
\section{Colheita}
\begin{itemize}
\item {Grp. gram.:f.}
\end{itemize}
\begin{itemize}
\item {Utilização:Ant.}
\end{itemize}
\begin{itemize}
\item {Proveniência:(Do lat. \textunderscore collecta\textunderscore )}
\end{itemize}
Acto de recolher (productos agrícolas): \textunderscore trabalhar na colheita\textunderscore .
Conjunto dos productos agrícolas de um anno: \textunderscore êste anno, foi má a colheita\textunderscore .
O que se colhe, o que se ajunta: \textunderscore aquelle mendigo faz bôa colheita\textunderscore .
Pensão que os vassallos pagavam ao senhorio ou príncipe, nos annos em que este visitava as terras dos vassallos.
\section{Colheiteiro}
\begin{itemize}
\item {Grp. gram.:m.}
\end{itemize}
Aquelle que faz colheitas.
Lavrador:«\textunderscore chegou a Lisbôa uma commissão de colheiteiros de vinho de Freixo\textunderscore ». \textunderscore Diário de Noticias\textunderscore , de 16-VII-908.
\section{Colheito}
\begin{itemize}
\item {Grp. gram.:adj.}
\end{itemize}
O mesmo que [[colhido|colhêr]] (part. de \textunderscore colhêr\textunderscore ). Cf. \textunderscore Port. Mon. Hist.\textunderscore , \textunderscore Script.\textunderscore , 284.
\section{Colhença}
\begin{itemize}
\item {Grp. gram.:f.}
\end{itemize}
\begin{itemize}
\item {Utilização:Des.}
\end{itemize}
O mesmo que \textunderscore colheita\textunderscore .
\section{Colhér}
\begin{itemize}
\item {Grp. gram.:f.}
\end{itemize}
\begin{itemize}
\item {Utilização:Prov.}
\end{itemize}
\begin{itemize}
\item {Utilização:Ant.}
\end{itemize}
\begin{itemize}
\item {Proveniência:(Do lat. \textunderscore cochlear\textunderscore )}
\end{itemize}
Instrumento, que é composto de um cabo e de uma parte côncava, e serve especialmente para levar alimentos á bôca.
Porção de líquido, que uma colhér póde conter.
Nome de vários utensílios, de feitio mais ou menos semelhante ao da colhér.
Bichinho, que é a larva ou embryão dos batrácios; o mesmo que \textunderscore gyrino\textunderscore .
Espécie de direito-real.--Quarenta colhéres perfaziam um alqueire.
\section{Colhêr}
\begin{itemize}
\item {Grp. gram.:v. t.}
\end{itemize}
\begin{itemize}
\item {Proveniência:(Do lat. \textunderscore colligere\textunderscore )}
\end{itemize}
Tirar das árvores ou das plantas: \textunderscore colhêr figos\textunderscore .
Apanhar, receber: \textunderscore colhêr o producto do estudo\textunderscore .
Surprehender, encontrar: \textunderscore o boi colheu o toireiro\textunderscore .
Tomar.
Adquirir.
Attingir: \textunderscore o toiro colheu o bandarilheiro\textunderscore .
Amainar: \textunderscore o barqueiro colhe a vela\textunderscore .
\section{Colhera}
\begin{itemize}
\item {Grp. gram.:f.}
\end{itemize}
\begin{itemize}
\item {Utilização:Bras. do S}
\end{itemize}
Ajoujo, com que se jungem dois animaes entre si.
(Cast. \textunderscore collera\textunderscore )
\section{Colherada}
\begin{itemize}
\item {Grp. gram.:f.}
\end{itemize}
\begin{itemize}
\item {Utilização:Fam.}
\end{itemize}
\begin{itemize}
\item {Proveniência:(De \textunderscore colhér\textunderscore )}
\end{itemize}
Aquillo que póde conter-se numa colhér.
Bedelho, acto de intrometer-se: \textunderscore meter a sua colherada\textunderscore .
\section{Colhereira}
\begin{itemize}
\item {Grp. gram.:f.}
\end{itemize}
Ave pernalta, (\textunderscore spatula clypeata\textunderscore , Lin.).
\section{Colhereiro}
\begin{itemize}
\item {Grp. gram.:m.}
\end{itemize}
Aquelle que faz ou vende colhéres.
Ave pernalta, de bico chato, (\textunderscore platalea lenorodia\textunderscore , Lin.).
\section{Colherete}
\begin{itemize}
\item {fónica:lherê}
\end{itemize}
\begin{itemize}
\item {Grp. gram.:m.}
\end{itemize}
\begin{itemize}
\item {Proveniência:(De \textunderscore colhêr\textunderscore )}
\end{itemize}
Pancada, com a pela, nos mirões do jôgo.
\section{Colheril}
\begin{itemize}
\item {Grp. gram.:m.}
\end{itemize}
Pequena colhér de estucador.
\section{Colherim}
\begin{itemize}
\item {Grp. gram.:m.}
\end{itemize}
\begin{itemize}
\item {Utilização:Prov.}
\end{itemize}
\begin{itemize}
\item {Utilização:alent.}
\end{itemize}
O mesmo que \textunderscore colheril\textunderscore .
\section{Colherinha}
\begin{itemize}
\item {Grp. gram.:f.}
\end{itemize}
\begin{itemize}
\item {Utilização:Prov.}
\end{itemize}
\begin{itemize}
\item {Utilização:alent.}
\end{itemize}
Variedade de castanha.
\section{Colherudo}
\begin{itemize}
\item {Grp. gram.:m.}
\end{itemize}
\begin{itemize}
\item {Utilização:Pop.}
\end{itemize}
O mesmo que \textunderscore gyrino\textunderscore .
(Cp. \textunderscore colhér\textunderscore )
\section{Colhimento}
\begin{itemize}
\item {Grp. gram.:m.}
\end{itemize}
Acto de colhêr.
\section{Colhoal}
\begin{itemize}
\item {Grp. gram.:adj.}
\end{itemize}
\begin{itemize}
\item {Utilização:Prov.}
\end{itemize}
\begin{itemize}
\item {Utilização:alent.}
\end{itemize}
Diz-se de uma variedade de ameixa.
\section{Cóli}
\begin{itemize}
\item {Grp. gram.:m.}
\end{itemize}
Colono, índio ou chinês, em colónias europeias.
(Do tamil \textunderscore kuli\textunderscore ?)
\section{Cólias}
\begin{itemize}
\item {Grp. gram.:f. pl.}
\end{itemize}
Gênero de lepidópteros.
\section{Coliangu}
\begin{itemize}
\item {Grp. gram.:m.}
\end{itemize}
\begin{itemize}
\item {Utilização:Bras}
\end{itemize}
O mesmo que \textunderscore bacurau\textunderscore .
\section{Colibacilar}
\begin{itemize}
\item {Grp. gram.:adj.}
\end{itemize}
Relativo ao colibacilo.
\section{Colibacillar}
\begin{itemize}
\item {Grp. gram.:adj.}
\end{itemize}
Relativo ao colibacillo.
\section{Colibacillo}
\begin{itemize}
\item {Grp. gram.:m.}
\end{itemize}
Bacillo, que se encontra normalmente no intestino do homem e dos animaes.
\section{Colibacillose}
\begin{itemize}
\item {Grp. gram.:f.}
\end{itemize}
\begin{itemize}
\item {Utilização:Med.}
\end{itemize}
Conjunto dos accidentes mórbidos, causados pelo colibacillo.
\section{Colibacilo}
\begin{itemize}
\item {Grp. gram.:m.}
\end{itemize}
Bacilo, que se encontra normalmente no intestino do homem e dos animaes.
\section{Colibacilose}
\begin{itemize}
\item {Grp. gram.:f.}
\end{itemize}
\begin{itemize}
\item {Utilização:Med.}
\end{itemize}
Conjunto dos accidentes mórbidos, causados pelo colibacilo.
\section{Colibri}
\begin{itemize}
\item {Grp. gram.:m.}
\end{itemize}
\begin{itemize}
\item {Proveniência:(Fr. \textunderscore colibri\textunderscore )}
\end{itemize}
Pássaro tenuirostro; beija-flôr; chupa-mel.
\section{Colíbrio}
\begin{itemize}
\item {Grp. gram.:m.}
\end{itemize}
O mesmo que \textunderscore colibri\textunderscore :«\textunderscore ao ruflar do colíbrio, amor das flores\textunderscore ». Ar. Porto-Alegre, \textunderscore Colombo\textunderscore , vol. II, canto XX.
\section{Cólica}
\begin{itemize}
\item {Grp. gram.:f.}
\end{itemize}
\begin{itemize}
\item {Proveniência:(De \textunderscore cólico\textunderscore )}
\end{itemize}
Dôr intensa na cavidade abdominal.
\section{Colicativo}
\begin{itemize}
\item {Grp. gram.:adj.}
\end{itemize}
Relativo á cólica.
\section{Cólico}
\begin{itemize}
\item {Grp. gram.:adj.}
\end{itemize}
\begin{itemize}
\item {Proveniência:(Gr. \textunderscore kolikos\textunderscore )}
\end{itemize}
Relativo ao cólon.
\section{Coliemia}
\begin{itemize}
\item {Grp. gram.:f.}
\end{itemize}
\begin{itemize}
\item {Proveniência:(Do gr. \textunderscore khole\textunderscore  + \textunderscore haima\textunderscore )}
\end{itemize}
Penetração da bílis no sangue.
\section{Coliflor}
\begin{itemize}
\item {Grp. gram.:f.}
\end{itemize}
O mesmo que \textunderscore couveflor\textunderscore .
\section{Colim}
\begin{itemize}
\item {Grp. gram.:m.}
\end{itemize}
Ave galinácea do México.
\section{Colimbo}
\begin{itemize}
\item {Grp. gram.:m.}
\end{itemize}
Ave aquática, palmípede, que, só arrojada pelas vagas ou por uma tempestade, é que poisa em terra, e a que os Franceses e Alemães chamam \textunderscore grèbe\textunderscore .
\section{Cola}
\begin{itemize}
\item {Grp. gram.:f.}
\end{itemize}
\begin{itemize}
\item {Utilização:Gír.}
\end{itemize}
\begin{itemize}
\item {Proveniência:(Gr. \textunderscore kolla\textunderscore )}
\end{itemize}
Preparação glutinosa, para fazer aderir papel, madeira ou outras substâncias.
Grude.
Fechadura.
\section{Cola}
\begin{itemize}
\item {Grp. gram.:f.}
\end{itemize}
\begin{itemize}
\item {Utilização:Bras}
\end{itemize}
\begin{itemize}
\item {Proveniência:(De \textunderscore collar\textunderscore ^3)}
\end{itemize}
Cópia clandestina de um ponto de exame, a que um estudante tem de responder.
\section{Cola}
\begin{itemize}
\item {Grp. gram.:m.}
\end{itemize}
Vento forte, que sopra nas costas das Filipinas.
\section{Colaboração}
\begin{itemize}
\item {Grp. gram.:f.}
\end{itemize}
Acto ou efeito de colaborar.
\section{Colaborador}
\begin{itemize}
\item {Grp. gram.:m.  e  adj.}
\end{itemize}
O que colabora.
Aquele que, sem pertencer á Redacção efectiva de um periódico, escreve para ele habitualmente ou alguma vez.
\section{Colaborar}
\begin{itemize}
\item {Grp. gram.:v. t.  e  i.}
\end{itemize}
Trabalhar em comum, especialmente em obra literária ou cientifica.
Escrever para (um periódico) uma ou outra vez, se não habitualmente, sem pertencer ao quadro efectivo dos redactores.
(B. lat. \textunderscore collaborare\textunderscore )
\section{Colação}
\begin{itemize}
\item {Grp. gram.:f.}
\end{itemize}
\begin{itemize}
\item {Utilização:Jur.}
\end{itemize}
\begin{itemize}
\item {Utilização:Ant.}
\end{itemize}
\begin{itemize}
\item {Proveniência:(Lat. \textunderscore collatio\textunderscore )}
\end{itemize}
Acto de conferir: \textunderscore colação de Ordens sacras\textunderscore .
Refeição ligeira.
Confronto.
Restituição, á massa da herança, dos valores recebidos pelos herdeiros, antes da partilha.
Freguesia.
\section{Colação}
\begin{itemize}
\item {Grp. gram.:f.}
\end{itemize}
Acto de colar^3.
\section{Colacia}
\begin{itemize}
\item {Grp. gram.:f.}
\end{itemize}
Relação entre colaços.
\section{Colacionar}
\begin{itemize}
\item {Grp. gram.:v. t.}
\end{itemize}
\begin{itemize}
\item {Proveniência:(Do lat. \textunderscore collatio\textunderscore )}
\end{itemize}
O mesmo que \textunderscore conferir\textunderscore .
\section{Colaço}
\begin{itemize}
\item {Grp. gram.:m.  e  adj.}
\end{itemize}
\begin{itemize}
\item {Proveniência:(Lat. \textunderscore collacicus\textunderscore )}
\end{itemize}
Diz-se dos indivíduos, que foram criados com leite da mesma mulher.
\section{Colada}
\begin{itemize}
\item {Grp. gram.:f.}
\end{itemize}
\begin{itemize}
\item {Proveniência:(De \textunderscore collo\textunderscore )}
\end{itemize}
Passagem larga entre montanhas.
\section{Colador}
\begin{itemize}
\item {Grp. gram.:m.}
\end{itemize}
\begin{itemize}
\item {Proveniência:(De \textunderscore colar\textunderscore ^3)}
\end{itemize}
Aquele que cola.
\section{Colagem}
\begin{itemize}
\item {Grp. gram.:f.}
\end{itemize}
Acto de \textunderscore colar\textunderscore ^3.
Preparação de vinhos, por meio de cola.
\section{Colaíte}
\begin{itemize}
\item {Grp. gram.:f.}
\end{itemize}
Espécie de mineral.
\section{Colandréu}
\begin{itemize}
\item {Grp. gram.:m.}
\end{itemize}
\begin{itemize}
\item {Utilização:Prov.}
\end{itemize}
\begin{itemize}
\item {Utilização:trasm.}
\end{itemize}
Gola da véstia, do casaco, etc.
(Relaciona-se com \textunderscore collo\textunderscore )
\section{Colapso}
\begin{itemize}
\item {Grp. gram.:m.}
\end{itemize}
\begin{itemize}
\item {Proveniência:(Lat. \textunderscore collapsus\textunderscore )}
\end{itemize}
Deminuição da excitabilidade nervosa.
\section{Colar}
\begin{itemize}
\item {Grp. gram.:m.}
\end{itemize}
\begin{itemize}
\item {Proveniência:(Lat. \textunderscore collare\textunderscore )}
\end{itemize}
Ornato para o pescoço.
Gola.
Colarinho.
Parte do boi, correspondente ao bôrdo anterior da espádua.
\section{Colar}
\begin{itemize}
\item {Grp. gram.:v. t.}
\end{itemize}
\begin{itemize}
\item {Proveniência:(Do rad. de \textunderscore colação\textunderscore )}
\end{itemize}
Conferir benefício eclesiastico e vitalício a.
Investir num emprêgo.
\section{Colar}
\begin{itemize}
\item {Grp. gram.:v. t.}
\end{itemize}
\begin{itemize}
\item {Proveniência:(De \textunderscore cola\textunderscore ^1)}
\end{itemize}
Fazer aderir com cola.
Juntar, pegar.
Clarificar com cola (vinho).
\section{Colareja}
\begin{itemize}
\item {Grp. gram.:f.}
\end{itemize}
\begin{itemize}
\item {Proveniência:(De \textunderscore Collares\textunderscore , n. p.)}
\end{itemize}
Vendedeira de frutas e legumes nos mercados de Lisbôa.
Regateira; mulher que discute grosseiramente.
\section{Colarejo}
\begin{itemize}
\item {Grp. gram.:m.}
\end{itemize}
Indivíduo habitante ou natural de Colares.
\section{Colares}
\begin{itemize}
\item {Grp. gram.:m.}
\end{itemize}
Vinho tinto e palhete, procedente das vinhas de Colares.
\section{Colarete}
\begin{itemize}
\item {fónica:larê}
\end{itemize}
\begin{itemize}
\item {Grp. gram.:m.}
\end{itemize}
\begin{itemize}
\item {Proveniência:(De \textunderscore collar\textunderscore )}
\end{itemize}
Moldura, composta de um astrágalo e filete.
\section{Colargol}
\begin{itemize}
\item {Grp. gram.:m.}
\end{itemize}
Prata coloidal, modificação alotrópica da prata metálica, descoberta em 1889 e introduzida na terapêutica em 1897.
\section{Colarinho}
\begin{itemize}
\item {Grp. gram.:m.}
\end{itemize}
\begin{itemize}
\item {Proveniência:(De \textunderscore colar\textunderscore ^1)}
\end{itemize}
Gola de pano, ligada ou cosida á camisa, em volta do pescoço.
Moldura chata, e estreita como uma fita, usada geralmente no alto das colunas.
\section{Colata}
\begin{itemize}
\item {Grp. gram.:f.}
\end{itemize}
\begin{itemize}
\item {Utilização:Ant.}
\end{itemize}
Garganta larga entre oiteiros ou montes; desfiladeiro; o mesmo que \textunderscore colada\textunderscore ^2.
\section{Colatário}
\begin{itemize}
\item {Grp. gram.:m.}
\end{itemize}
Aquele, em favor de quem se exerce o direito de colação.
(Cp. \textunderscore colação\textunderscore ^1)
\section{Colateral}
\begin{itemize}
\item {Grp. gram.:adj.}
\end{itemize}
\begin{itemize}
\item {Proveniência:(Lat. \textunderscore collateralis\textunderscore )}
\end{itemize}
Que está ao lado; paralelo.
Que é parente, mas não em linha recta.
\section{Colateralidade}
\begin{itemize}
\item {Grp. gram.:f.}
\end{itemize}
Qualidade do que é colateral.
\section{Colateralmente}
\begin{itemize}
\item {Grp. gram.:adv.}
\end{itemize}
Em linha colateral.
\section{Colativo}
\begin{itemize}
\item {Grp. gram.:adj.}
\end{itemize}
\begin{itemize}
\item {Proveniência:(Lat. \textunderscore collativus\textunderscore )}
\end{itemize}
Relativo a colação^1.
Que é conferido.
\section{Colator}
\begin{itemize}
\item {Grp. gram.:m.}
\end{itemize}
\begin{itemize}
\item {Proveniência:(Lat. \textunderscore collator\textunderscore )}
\end{itemize}
Aquele que confere benefício eclesiástico.
\section{Cole}
\begin{itemize}
\item {Grp. gram.:m.}
\end{itemize}
\begin{itemize}
\item {Utilização:Ant.}
\end{itemize}
\begin{itemize}
\item {Proveniência:(Lat. \textunderscore collis\textunderscore )}
\end{itemize}
Oiteiro.
\section{Coleado}
\begin{itemize}
\item {Grp. gram.:adj.}
\end{itemize}
\begin{itemize}
\item {Proveniência:(De \textunderscore collear\textunderscore )}
\end{itemize}
Que tem fórma de colo.
Que serpeia.
\section{Colear}
\begin{itemize}
\item {Grp. gram.:v. i.}
\end{itemize}
\begin{itemize}
\item {Grp. gram.:V. p.}
\end{itemize}
\begin{itemize}
\item {Utilização:Fig.}
\end{itemize}
Mover o colo.
Andar, fazendo zigue zague.
Serpear.
Inocular-se, introduzir-se sorrateiramente:«\textunderscore não sei por onde se coleou nas repúblicas tão bárbaro costume\textunderscore ». Filinto, \textunderscore D. Man.\textunderscore , III, 240.
\section{Colecção}
\begin{itemize}
\item {Grp. gram.:f.}
\end{itemize}
\begin{itemize}
\item {Proveniência:(Lat. \textunderscore collectio\textunderscore )}
\end{itemize}
Conjunto; reunião de objectos: \textunderscore colecção de jóias\textunderscore .
Compilação: \textunderscore colecção de aforismos\textunderscore .
Ajuntamento.
\section{Coleccionação}
\begin{itemize}
\item {Grp. gram.:f.}
\end{itemize}
Acto de coleccionar.
\section{Coleccionador}
\begin{itemize}
\item {Grp. gram.:m.}
\end{itemize}
Aquele que colecciona.
\section{Coleccionar}
\begin{itemize}
\item {Grp. gram.:v. t.}
\end{itemize}
\begin{itemize}
\item {Proveniência:(Do lat. \textunderscore collectio\textunderscore )}
\end{itemize}
Fazer colecção de; coligir: \textunderscore coleccionar trechos clássicos\textunderscore .
\section{Coleccionista}
\begin{itemize}
\item {Grp. gram.:m.}
\end{itemize}
O mesmo que \textunderscore coleccionador\textunderscore .
\section{Colecta}
\begin{itemize}
\item {Grp. gram.:f.}
\end{itemize}
\begin{itemize}
\item {Utilização:Ant.}
\end{itemize}
\begin{itemize}
\item {Proveniência:(Lat. \textunderscore collecta\textunderscore )}
\end{itemize}
Contribuição, imposto individual.
Quota, para obra de piedade ou para despesa commum.
Oração em nome de todo o povo.
O mesmo que \textunderscore colheita\textunderscore .
\section{Colectânea}
\begin{itemize}
\item {Grp. gram.:f.}
\end{itemize}
\begin{itemize}
\item {Proveniência:(De \textunderscore collectâneo\textunderscore )}
\end{itemize}
Excerptos selectos e reunidos de diversas obras.
\section{Colectâneo}
\begin{itemize}
\item {Grp. gram.:adj.}
\end{itemize}
\begin{itemize}
\item {Proveniência:(Lat. \textunderscore collectaneus\textunderscore )}
\end{itemize}
Extraido de várias obras; coligido.
\section{Colectano}
\begin{itemize}
\item {Grp. gram.:m.}
\end{itemize}
Livro, que continha as orações da Missa, chamadas \textunderscore colectas\textunderscore .
\section{Colectar}
\begin{itemize}
\item {Grp. gram.:v. t.}
\end{itemize}
\begin{itemize}
\item {Proveniência:(De \textunderscore collecta\textunderscore )}
\end{itemize}
Tributar, lançar contribuição sôbre: \textunderscore colectar uma indústria\textunderscore .
Designar quota a.
\section{Colectário}
\begin{itemize}
\item {Grp. gram.:m.}
\end{itemize}
\begin{itemize}
\item {Proveniência:(De \textunderscore collecta\textunderscore )}
\end{itemize}
Livro de orações, que contém todas as colectas do ano.
\section{Colectável}
\begin{itemize}
\item {Grp. gram.:adj.}
\end{itemize}
\begin{itemize}
\item {Proveniência:(De \textunderscore collectar\textunderscore )}
\end{itemize}
Que póde sêr colectado: \textunderscore rendimento colectável\textunderscore .
\section{Colectício}
\begin{itemize}
\item {Grp. gram.:adj.}
\end{itemize}
\begin{itemize}
\item {Utilização:Des.}
\end{itemize}
\begin{itemize}
\item {Proveniência:(Lat. \textunderscore collecticius\textunderscore )}
\end{itemize}
Dizia-se da gente reunída á pressa, sem escolha, para a guerra: \textunderscore tropas colectícias\textunderscore .
\section{Colectivamente}
\begin{itemize}
\item {Grp. gram.:adv.}
\end{itemize}
De modo colectivo.
Em globo.
\section{Colectividade}
\begin{itemize}
\item {Grp. gram.:f.}
\end{itemize}
Qualidade do que é colectivo.
Sociedade.
Conjunto.
\section{Colectivismo}
\begin{itemize}
\item {Grp. gram.:m.}
\end{itemize}
\begin{itemize}
\item {Proveniência:(De \textunderscore collectivo\textunderscore )}
\end{itemize}
Sistema sociológico, que procura tornar os meios de producção communs a todos os membros da sociedade.
\section{Colectivista}
\begin{itemize}
\item {Grp. gram.:m.}
\end{itemize}
Partidário do colectivismo.
\section{Colectivo}
\begin{itemize}
\item {Grp. gram.:adj.}
\end{itemize}
\begin{itemize}
\item {Utilização:Gram.}
\end{itemize}
\begin{itemize}
\item {Proveniência:(Lat. \textunderscore collectivus\textunderscore )}
\end{itemize}
Que abrange muitas coisas ou pessôas.
Relativo a muitas coisas ou pessôas.
Que no singular exprime o conjunto de indivíduos da mesma espécie.
\section{Colecto}
\begin{itemize}
\item {Grp. gram.:adj.}
\end{itemize}
\begin{itemize}
\item {Proveniência:(Lat. \textunderscore collectus\textunderscore )}
\end{itemize}
Coligido, escolhido.
\section{Colector}
\begin{itemize}
\item {Grp. gram.:m.}
\end{itemize}
\begin{itemize}
\item {Grp. gram.:Adj.}
\end{itemize}
\begin{itemize}
\item {Proveniência:(Do lat. \textunderscore collectus\textunderscore )}
\end{itemize}
Aquele que lança ou recebe colectas.
Que colige; que reúne.
\section{Colectoria}
\begin{itemize}
\item {Grp. gram.:f.}
\end{itemize}
\begin{itemize}
\item {Utilização:Bras}
\end{itemize}
\begin{itemize}
\item {Proveniência:(De \textunderscore collector\textunderscore )}
\end{itemize}
Lugar, onde se pagam os impostos.
Cargo de colector.
\section{Colega}
\begin{itemize}
\item {Grp. gram.:m.  e  f.}
\end{itemize}
\begin{itemize}
\item {Proveniência:(Lat. \textunderscore collega\textunderscore )}
\end{itemize}
Pessôa, que, em relação a outra, faz parte da mesma comunidade, corporação, profissão, etc.
Cada um dos que exercem a mesma profissão, especialmente na classe civil e eclesiástica.--Na classe militar, prefere-se geralmente o t. \textunderscore camarada\textunderscore .
\section{Colegiada}
\begin{itemize}
\item {Grp. gram.:f.}
\end{itemize}
\begin{itemize}
\item {Proveniência:(De \textunderscore collégio\textunderscore )}
\end{itemize}
Conjunto dos alunos de um colégio.
Corporação de sacerdotes, que têm funções de cónegos, em igreja que não é episcopal: \textunderscore a colegiada de Guimarães\textunderscore .
Igreja, onde há essa corporação.
\section{Colegial}
\begin{itemize}
\item {Grp. gram.:adj.}
\end{itemize}
\begin{itemize}
\item {Grp. gram.:M.}
\end{itemize}
Relativo a colégio: \textunderscore a vida colegial\textunderscore .
Aluno de colégio.
\section{Colegiatura}
\begin{itemize}
\item {Grp. gram.:f.}
\end{itemize}
\begin{itemize}
\item {Proveniência:(De \textunderscore collégio\textunderscore )}
\end{itemize}
Qualidade do que é colegial.
\section{Colégio}
\begin{itemize}
\item {Grp. gram.:m.}
\end{itemize}
\begin{itemize}
\item {Utilização:Ant.}
\end{itemize}
\begin{itemize}
\item {Utilização:Gír.}
\end{itemize}
\begin{itemize}
\item {Proveniência:(Lat. \textunderscore collegium\textunderscore )}
\end{itemize}
Corporação, cujos membros têm igual dignidade.
Estabelecimento de ensino primário ou secundário.
Reunião de gente para fins eleitoraes.
Convento de jesuitas, com ónus de ensino.
Cárcere.
\section{Coleira}
\begin{itemize}
\item {Grp. gram.:f.}
\end{itemize}
\begin{itemize}
\item {Utilização:Bras}
\end{itemize}
\begin{itemize}
\item {Proveniência:(De \textunderscore collo\textunderscore )}
\end{itemize}
Armadura, resguardo de coiro, para o pescoço.
Peça de coiro ou metal, que se põe no pescoço dos cães e de outros animaes.
Fêmea do coleiro.
\section{Coleirado}
\begin{itemize}
\item {Grp. gram.:adj.}
\end{itemize}
Que tem coleira.
Que tem no pescoço malha ou pêlos, que dão a aparência de coleira.
\section{Coleirinho}
\begin{itemize}
\item {Grp. gram.:adj.}
\end{itemize}
\begin{itemize}
\item {Utilização:Ant.}
\end{itemize}
\begin{itemize}
\item {Proveniência:(De \textunderscore collo\textunderscore )}
\end{itemize}
Que ainda anda ao colo, (falando-se do menino).
\section{Coleiro}
\begin{itemize}
\item {Grp. gram.:m.}
\end{itemize}
\begin{itemize}
\item {Proveniência:(De \textunderscore collo\textunderscore )}
\end{itemize}
Ave canora do Brasil, que parece têr ao pescoço uma gravata preta.
\section{Coleitor}
\begin{itemize}
\item {Grp. gram.:m.}
\end{itemize}
\begin{itemize}
\item {Proveniência:(Lat. \textunderscore collector\textunderscore )}
\end{itemize}
Antigo dignitário eclesiástico, encarregado pela Santa Sé de reivindicar e assegurar as rendas da Igreja, e de representar a côrte pontifícia, junto dos Governos, na ausência do Núncio.
\section{Colema}
\begin{itemize}
\item {Grp. gram.:m.}
\end{itemize}
\begin{itemize}
\item {Proveniência:(Do rad. do gr. \textunderscore kolla\textunderscore )}
\end{itemize}
Planta criptogâmica das zonas temperadas.
\section{Colênquima}
\begin{itemize}
\item {Grp. gram.:m.}
\end{itemize}
\begin{itemize}
\item {Proveniência:(Do gr. \textunderscore kolla\textunderscore  + \textunderscore egkhein\textunderscore )}
\end{itemize}
Tecido utricular, vegetal, caracterizado pela grande espessura dos utriculos constituintes.
\section{Colete}
\begin{itemize}
\item {fónica:lê}
\end{itemize}
\begin{itemize}
\item {Grp. gram.:m.}
\end{itemize}
\begin{itemize}
\item {Utilização:Bras}
\end{itemize}
\begin{itemize}
\item {Proveniência:(De \textunderscore collo\textunderscore )}
\end{itemize}
Peça de vestuário, curta e sem mangas, ajustada ao peito e abotoada na frente.
Espartilho.
Resguardo de madeira ou arame, nas hastes dos arbustos.
\section{Coleteiro}
\begin{itemize}
\item {Grp. gram.:m.}
\end{itemize}
\begin{itemize}
\item {Utilização:Bras}
\end{itemize}
Fabricante de coletes ou espartilhos.
\section{Colétia}
\begin{itemize}
\item {Grp. gram.:f.}
\end{itemize}
Gênero de plantas rhamnáceas.
\section{Colidir}
\begin{itemize}
\item {Grp. gram.:v. t.}
\end{itemize}
\begin{itemize}
\item {Grp. gram.:V. p.}
\end{itemize}
\begin{itemize}
\item {Proveniência:(Lat. \textunderscore collidere\textunderscore )}
\end{itemize}
Fazer ir (uma coisa) contra outra.
Ir de encontro; sêr oposto reciprocamente: \textunderscore isso são ideias que colidem\textunderscore .
\section{Colífero}
\begin{itemize}
\item {Grp. gram.:adj.}
\end{itemize}
\begin{itemize}
\item {Proveniência:(Do lat. \textunderscore collum\textunderscore  + \textunderscore ferre\textunderscore )}
\end{itemize}
Provido de colar, (falando-se do pedúnculo de certos cogumelos).
\section{Colífio}
\begin{itemize}
\item {Grp. gram.:m.}
\end{itemize}
\begin{itemize}
\item {Proveniência:(Lat. \textunderscore coliphium\textunderscore )}
\end{itemize}
Sistema de alimentação privativa dos atletas, entre os Romanos. Cf. C. Lobo, \textunderscore Sát. de Juv.\textunderscore , I, 216.
\section{Coligação}
\begin{itemize}
\item {Grp. gram.:f.}
\end{itemize}
\begin{itemize}
\item {Proveniência:(Lat. \textunderscore colligatio\textunderscore )}
\end{itemize}
Liga, aliança, de várias pessôas para um fim comum.
Assimilação.
Liga de quaesquer substâncias.
Confederação.
Trama.
\section{Coligar}
\begin{itemize}
\item {Grp. gram.:v. t.}
\end{itemize}
\begin{itemize}
\item {Proveniência:(Lat. \textunderscore colligare\textunderscore )}
\end{itemize}
Associar por coligação.
\section{Coligativo}
\begin{itemize}
\item {Grp. gram.:adj.}
\end{itemize}
\begin{itemize}
\item {Proveniência:(De \textunderscore colligar\textunderscore )}
\end{itemize}
Relativo a coligação.
Que coliga. Cf. \textunderscore Propriedades Coligativas das Soluções\textunderscore , por Ach. Machado.
\section{Coligir}
\begin{itemize}
\item {Grp. gram.:v. t.}
\end{itemize}
\begin{itemize}
\item {Proveniência:(Lat. \textunderscore colligere\textunderscore )}
\end{itemize}
Juntar; reünir em colecção: \textunderscore coligir apontamentos\textunderscore .
Inferir: \textunderscore eis o que colíjo dessas teorias\textunderscore .
\section{Colimação}
\begin{itemize}
\item {Grp. gram.:f.}
\end{itemize}
Acto de colimar.
\section{Colimar}
\begin{itemize}
\item {Grp. gram.:v. t.}
\end{itemize}
\begin{itemize}
\item {Proveniência:(Lat. \textunderscore collimare\textunderscore , palavra que se lê nalgumas edições, onde se deveria lêr \textunderscore collineare\textunderscore , seguir uma linha com a vista)}
\end{itemize}
Observar, por meio de instrumento.
\section{Colimbriense}
\begin{itemize}
\item {Grp. gram.:adj.}
\end{itemize}
O mesmo que \textunderscore conimbricense\textunderscore . Cf. Herculano, \textunderscore Hist. de Port.\textunderscore , I, 195, 200 e 477.
\section{Colimitar}
\begin{itemize}
\item {Grp. gram.:v. t.}
\end{itemize}
\begin{itemize}
\item {Proveniência:(Lat. \textunderscore collimitare\textunderscore )}
\end{itemize}
Pôr limite comum a.
\section{Colina}
\begin{itemize}
\item {Grp. gram.:f.}
\end{itemize}
\begin{itemize}
\item {Proveniência:(Lat. \textunderscore collina\textunderscore )}
\end{itemize}
Pequena montanha, oiteiro; encosta.
\section{Colinos}
\begin{itemize}
\item {Grp. gram.:m. pl.}
\end{itemize}
Aborígenes do Pará.
\section{Colinoso}
\begin{itemize}
\item {Grp. gram.:adj.}
\end{itemize}
\begin{itemize}
\item {Proveniência:(De \textunderscore collina\textunderscore )}
\end{itemize}
Que tem muitas colinas.
\section{Colinsónia}
\begin{itemize}
\item {Grp. gram.:f.}
\end{itemize}
\begin{itemize}
\item {Proveniência:(De \textunderscore Collinson\textunderscore , n. p.)}
\end{itemize}
Planta labiada da América do Norte.
\section{Cólio}
\begin{itemize}
\item {Grp. gram.:m.}
\end{itemize}
Pássaro conirostro, (\textunderscore colius\textunderscore ).
\section{Colipeu}
\begin{itemize}
\item {Grp. gram.:m.}
\end{itemize}
Bichinho fantástico:«\textunderscore descobriu um colipeu nas ervas do seu jardim..., bichinho singular... que tem virtudes para curar de pleurizes... e de omoplatas nas pernas...\textunderscore »Castilho, \textunderscore Méd. á Fôrça\textunderscore , 170.
\section{Colíphio}
\begin{itemize}
\item {Grp. gram.:m.}
\end{itemize}
\begin{itemize}
\item {Proveniência:(Lat. \textunderscore coliphium\textunderscore )}
\end{itemize}
Systema de alimentação privativa dos athletas, entre os Romanos. Cf. C. Lobo, \textunderscore Sát. de Juv.\textunderscore , I, 216.
\section{Coliquação}
\begin{itemize}
\item {Grp. gram.:f.}
\end{itemize}
\begin{itemize}
\item {Proveniência:(Lat. \textunderscore colliquatio\textunderscore )}
\end{itemize}
Dissolução orgânica, com excreções abundantes.
\section{Coliquante}
\begin{itemize}
\item {Grp. gram.:adj.}
\end{itemize}
\begin{itemize}
\item {Proveniência:(Do rad. do lat. \textunderscore colliquescere\textunderscore )}
\end{itemize}
Que dissolve; que derrete.
\section{Coliquar}
\begin{itemize}
\item {Grp. gram.:v. t.}
\end{itemize}
Fundir, derreter. Cf. \textunderscore Ancora Méd.\textunderscore , 7.
\section{Coliquativo}
\begin{itemize}
\item {Grp. gram.:adj.}
\end{itemize}
Que é produzido pela coliquação.
(Cp. \textunderscore coliquação\textunderscore )
\section{Colir}
\begin{itemize}
\item {Grp. gram.:m.}
\end{itemize}
Funccionário, que na China exercia as funcções de censor.
\section{Colirrostros}
\begin{itemize}
\item {Grp. gram.:m. pl.}
\end{itemize}
\begin{itemize}
\item {Proveniência:(Do lat. \textunderscore collum\textunderscore  + \textunderscore rostrum\textunderscore )}
\end{itemize}
Família de insectos hemípteros, cujo bico parece nascer do pescoço.
\section{Colisão}
\begin{itemize}
\item {Grp. gram.:f.}
\end{itemize}
\begin{itemize}
\item {Proveniência:(Lat. \textunderscore collisio\textunderscore )}
\end{itemize}
Embate recíproco de dois corpos.
Luta.
Contrariedade.
Alternativa, dificuldade de opção: \textunderscore na colisão de dois extremos\textunderscore .
\section{Coliseu}
\begin{itemize}
\item {Grp. gram.:m.}
\end{itemize}
O maior amphitheatro romano.
Circo.
(B. lat. \textunderscore coliseum\textunderscore , de \textunderscore colosseum\textunderscore , do lat. \textunderscore colossus\textunderscore )
\section{Colita}
\begin{itemize}
\item {Grp. gram.:f.}
\end{itemize}
Planta indiana, (\textunderscore dolichos uniflorus\textunderscore ).
(Do conc.)
\section{Colite}
\begin{itemize}
\item {Grp. gram.:f.}
\end{itemize}
Inflammação do cólon.
\section{Colitigante}
\begin{itemize}
\item {Grp. gram.:m.  e  adj.}
\end{itemize}
\begin{itemize}
\item {Proveniência:(De \textunderscore co...\textunderscore  + \textunderscore litigante\textunderscore )}
\end{itemize}
O que litiga juntamente com outrem.
\section{Colitigar}
\begin{itemize}
\item {Grp. gram.:v. t.}
\end{itemize}
\begin{itemize}
\item {Proveniência:(De \textunderscore co...\textunderscore  + \textunderscore litigar\textunderscore )}
\end{itemize}
Litigar juntamente com outrem.
\section{Colla}
\begin{itemize}
\item {Grp. gram.:f.}
\end{itemize}
\begin{itemize}
\item {Utilização:Gír.}
\end{itemize}
\begin{itemize}
\item {Proveniência:(Gr. \textunderscore kolla\textunderscore )}
\end{itemize}
Preparação glutinosa, para fazer adherir papel, madeira ou outras substâncias.
Grude.
Fechadura.
\section{Colla}
\begin{itemize}
\item {Grp. gram.:f.}
\end{itemize}
\begin{itemize}
\item {Utilização:Bras}
\end{itemize}
\begin{itemize}
\item {Proveniência:(De \textunderscore collar\textunderscore ^3)}
\end{itemize}
Cópia clandestina de um ponto de exame, a que um estudante tem de responder.
\section{Colla}
\begin{itemize}
\item {Grp. gram.:m.}
\end{itemize}
Vento forte, que sopra nas costas das Filippinas.
\section{Collaboração}
\begin{itemize}
\item {Grp. gram.:f.}
\end{itemize}
Acto ou effeito de collaborar.
\section{Collaborador}
\begin{itemize}
\item {Grp. gram.:m.  e  adj.}
\end{itemize}
O que collabora.
Aquelle que, sem pertencer á Redacção effectiva de um periódico, escreve para êlle habitualmente ou alguma vez.
\section{Collaborar}
\begin{itemize}
\item {Grp. gram.:v. t.  e  i.}
\end{itemize}
Trabalhar em commum, especialmente em obra literária ou scientifica.
Escrever para (um periódico) uma ou outra vez, se não habitualmente, sem pertencer ao quadro effectivo dos redactores.
(B. lat. \textunderscore collaborare\textunderscore )
\section{Collação}
\begin{itemize}
\item {Grp. gram.:f.}
\end{itemize}
\begin{itemize}
\item {Utilização:Jur.}
\end{itemize}
\begin{itemize}
\item {Utilização:Ant.}
\end{itemize}
\begin{itemize}
\item {Proveniência:(Lat. \textunderscore collatio\textunderscore )}
\end{itemize}
Acto de conferir: \textunderscore collação de Ordens sacras\textunderscore .
Refeição ligeira.
Confronto.
Restituição, á massa da herança, dos valores recebidos pelos herdeiros, antes da partilha.
Freguesia.
\section{Collação}
\begin{itemize}
\item {Grp. gram.:f.}
\end{itemize}
Acto de collar^3.
\section{Collacia}
\begin{itemize}
\item {Grp. gram.:f.}
\end{itemize}
Relação entre collaços.
\section{Collacionar}
\begin{itemize}
\item {Grp. gram.:v. t.}
\end{itemize}
\begin{itemize}
\item {Proveniência:(Do lat. \textunderscore collatio\textunderscore )}
\end{itemize}
O mesmo que \textunderscore conferir\textunderscore .
\section{Collaço}
\begin{itemize}
\item {Grp. gram.:m.  e  adj.}
\end{itemize}
\begin{itemize}
\item {Proveniência:(Lat. \textunderscore collacicus\textunderscore )}
\end{itemize}
Diz-se dos indivíduos, que foram criados com leite da mesma mulher.
\section{Collada}
\begin{itemize}
\item {Grp. gram.:f.}
\end{itemize}
\begin{itemize}
\item {Proveniência:(De \textunderscore collo\textunderscore )}
\end{itemize}
Passagem larga entre montanhas.
\section{Collador}
\begin{itemize}
\item {Grp. gram.:m.}
\end{itemize}
\begin{itemize}
\item {Proveniência:(De \textunderscore collar\textunderscore ^3)}
\end{itemize}
Aquelle que colla.
\section{Collagem}
\begin{itemize}
\item {Grp. gram.:f.}
\end{itemize}
Acto de \textunderscore collar\textunderscore ^3.
Preparação de vinhos, por meio de colla.
\section{Collaíte}
\begin{itemize}
\item {Grp. gram.:f.}
\end{itemize}
Espécie de mineral.
\section{Collandréu}
\begin{itemize}
\item {Grp. gram.:m.}
\end{itemize}
\begin{itemize}
\item {Utilização:Prov.}
\end{itemize}
\begin{itemize}
\item {Utilização:trasm.}
\end{itemize}
Golla da véstia, do casaco, etc.
(Relaciona-se com \textunderscore collo\textunderscore )
\section{Collapso}
\begin{itemize}
\item {Grp. gram.:m.}
\end{itemize}
\begin{itemize}
\item {Proveniência:(Lat. \textunderscore collapsus\textunderscore )}
\end{itemize}
Deminuição da excitabilidade nervosa.
\section{Collar}
\begin{itemize}
\item {Grp. gram.:m.}
\end{itemize}
\begin{itemize}
\item {Proveniência:(Lat. \textunderscore collare\textunderscore )}
\end{itemize}
Ornato para o pescoço.
Golla.
Collarinho.
Parte do boi, correspondente ao bôrdo anterior da espádua.
\section{Collar}
\begin{itemize}
\item {Grp. gram.:v. t.}
\end{itemize}
\begin{itemize}
\item {Proveniência:(Do rad. de \textunderscore collação\textunderscore )}
\end{itemize}
Conferir benefício ecclesiastico e vitalício a.
Investir num emprêgo.
\section{Collar}
\begin{itemize}
\item {Grp. gram.:v. t.}
\end{itemize}
\begin{itemize}
\item {Proveniência:(De \textunderscore colla\textunderscore ^1)}
\end{itemize}
Fazer adherir com colla.
Juntar, pegar.
Clarificar com colla (vinho).
\section{Collareja}
\begin{itemize}
\item {Grp. gram.:f.}
\end{itemize}
\begin{itemize}
\item {Proveniência:(De \textunderscore Collares\textunderscore , n. p.)}
\end{itemize}
Vendedeira de frutas e legumes nos mercados de Lisbôa.
Regateira; mulher que discute grosseiramente.
\section{Collarejo}
\begin{itemize}
\item {Grp. gram.:m.}
\end{itemize}
Indivíduo habitante ou natural de Collares.
\section{Collares}
\begin{itemize}
\item {Grp. gram.:m.}
\end{itemize}
Vinho tinto e palhete, procedente das vinhas de Collares.
\section{Collarete}
\begin{itemize}
\item {fónica:larê}
\end{itemize}
\begin{itemize}
\item {Grp. gram.:m.}
\end{itemize}
\begin{itemize}
\item {Proveniência:(De \textunderscore collar\textunderscore )}
\end{itemize}
Moldura, composta de um astrágalo e filete.
\section{Collargol}
\begin{itemize}
\item {Grp. gram.:m.}
\end{itemize}
Prata colloidal, modificação allotrópica da prata metállica, descoberta em 1889 e introduzida na therapêutica em 1897.
\section{Collarinho}
\begin{itemize}
\item {Grp. gram.:m.}
\end{itemize}
\begin{itemize}
\item {Proveniência:(De \textunderscore collar\textunderscore ^1)}
\end{itemize}
Golla de pano, ligada ou cosida á camisa, em volta do pescoço.
Moldura chata, e estreita como uma fita, usada geralmente no alto das columnas.
\section{Collata}
\begin{itemize}
\item {Grp. gram.:f.}
\end{itemize}
\begin{itemize}
\item {Utilização:Ant.}
\end{itemize}
Garganta larga entre oiteiros ou montes; desfiladeiro; o mesmo que \textunderscore collada\textunderscore .
\section{Collatário}
\begin{itemize}
\item {Grp. gram.:m.}
\end{itemize}
Aquelle, em favor de quem se exerce o direito de collação.
(Cp. \textunderscore collação\textunderscore ^1)
\section{Collateral}
\begin{itemize}
\item {Grp. gram.:adj.}
\end{itemize}
\begin{itemize}
\item {Proveniência:(Lat. \textunderscore collateralis\textunderscore )}
\end{itemize}
Que está ao lado; parallelo.
Que é parente, mas não em linha recta.
\section{Collateralidade}
\begin{itemize}
\item {Grp. gram.:f.}
\end{itemize}
Qualidade do que é collateral.
\section{Collateralmente}
\begin{itemize}
\item {Grp. gram.:adv.}
\end{itemize}
Em linha collateral.
\section{Collativo}
\begin{itemize}
\item {Grp. gram.:adj.}
\end{itemize}
\begin{itemize}
\item {Proveniência:(Lat. \textunderscore collativus\textunderscore )}
\end{itemize}
Relativo a collação^1.
Que é conferido.
\section{Collator}
\begin{itemize}
\item {Grp. gram.:m.}
\end{itemize}
\begin{itemize}
\item {Proveniência:(Lat. \textunderscore collator\textunderscore )}
\end{itemize}
Aquelle que confere benefício ecclesiástico.
\section{Colle}
\begin{itemize}
\item {Grp. gram.:m.}
\end{itemize}
\begin{itemize}
\item {Utilização:Ant.}
\end{itemize}
\begin{itemize}
\item {Proveniência:(Lat. \textunderscore collis\textunderscore )}
\end{itemize}
Oiteiro.
\section{Colleado}
\begin{itemize}
\item {Grp. gram.:adj.}
\end{itemize}
\begin{itemize}
\item {Proveniência:(De \textunderscore collear\textunderscore )}
\end{itemize}
Que tem fórma de collo.
Que serpeia.
\section{Collear}
\begin{itemize}
\item {Grp. gram.:v. i.}
\end{itemize}
\begin{itemize}
\item {Grp. gram.:V. p.}
\end{itemize}
\begin{itemize}
\item {Utilização:Fig.}
\end{itemize}
Mover o collo.
Andar, fazendo zigue zague.
Serpear.
Inocular-se, introduzir-se sorrateiramente:«\textunderscore não sei por onde se colleou nas repúblicas tão bárbaro costume\textunderscore ». Filinto, \textunderscore D. Man.\textunderscore , III, 240.
\section{Collecção}
\begin{itemize}
\item {Grp. gram.:f.}
\end{itemize}
\begin{itemize}
\item {Proveniência:(Lat. \textunderscore collectio\textunderscore )}
\end{itemize}
Conjunto; reunião de objectos: \textunderscore collecção de jóias\textunderscore .
Compilação: \textunderscore collecção de aphorismos\textunderscore .
Ajuntamento.
\section{Colleccionação}
\begin{itemize}
\item {Grp. gram.:f.}
\end{itemize}
Acto de colleccionar.
\section{Colleccionador}
\begin{itemize}
\item {Grp. gram.:m.}
\end{itemize}
Aquelle que collecciona.
\section{Colleccionar}
\begin{itemize}
\item {Grp. gram.:v. t.}
\end{itemize}
\begin{itemize}
\item {Proveniência:(Do lat. \textunderscore collectio\textunderscore )}
\end{itemize}
Fazer collecção de; colligir: \textunderscore colleccionar trechos clássicos\textunderscore .
\section{Colleccionista}
\begin{itemize}
\item {Grp. gram.:m.}
\end{itemize}
O mesmo que \textunderscore colleccionador\textunderscore .
\section{Collecta}
\begin{itemize}
\item {Grp. gram.:f.}
\end{itemize}
\begin{itemize}
\item {Utilização:Ant.}
\end{itemize}
\begin{itemize}
\item {Proveniência:(Lat. \textunderscore collecta\textunderscore )}
\end{itemize}
Contribuição, imposto individual.
Quota, para obra de piedade ou para despesa commum.
Oração em nome de todo o povo.
O mesmo que \textunderscore colheita\textunderscore .
\section{Collectânea}
\begin{itemize}
\item {Grp. gram.:f.}
\end{itemize}
\begin{itemize}
\item {Proveniência:(De \textunderscore collectâneo\textunderscore )}
\end{itemize}
Excerptos selectos e reunidos de diversas obras.
\section{Collectâneo}
\begin{itemize}
\item {Grp. gram.:adj.}
\end{itemize}
\begin{itemize}
\item {Proveniência:(Lat. \textunderscore collectaneus\textunderscore )}
\end{itemize}
Extrahido de várias obras; colligido.
\section{Collectano}
\begin{itemize}
\item {Grp. gram.:m.}
\end{itemize}
Livro, que continha as orações da Missa, chamadas \textunderscore collectas\textunderscore .
\section{Collectar}
\begin{itemize}
\item {Grp. gram.:v. t.}
\end{itemize}
\begin{itemize}
\item {Proveniência:(De \textunderscore collecta\textunderscore )}
\end{itemize}
Tributar, lançar contribuição sôbre: \textunderscore collectar uma indústria\textunderscore .
Designar quota a.
\section{Collectário}
\begin{itemize}
\item {Grp. gram.:m.}
\end{itemize}
\begin{itemize}
\item {Proveniência:(De \textunderscore collecta\textunderscore )}
\end{itemize}
Livro de orações, que contém todas as collectas do anno.
\section{Collectável}
\begin{itemize}
\item {Grp. gram.:adj.}
\end{itemize}
\begin{itemize}
\item {Proveniência:(De \textunderscore collectar\textunderscore )}
\end{itemize}
Que póde sêr collectado: \textunderscore rendimento collectável\textunderscore .
\section{Collectício}
\begin{itemize}
\item {Grp. gram.:adj.}
\end{itemize}
\begin{itemize}
\item {Utilização:Des.}
\end{itemize}
\begin{itemize}
\item {Proveniência:(Lat. \textunderscore collecticius\textunderscore )}
\end{itemize}
Dizia-se da gente reunída á pressa, sem escolha, para a guerra: \textunderscore tropas collectícias\textunderscore .
\section{Collectivamente}
\begin{itemize}
\item {Grp. gram.:adv.}
\end{itemize}
De modo collectivo.
Em globo.
\section{Collectividade}
\begin{itemize}
\item {Grp. gram.:f.}
\end{itemize}
Qualidade do que é collectivo.
Sociedade.
Conjunto.
\section{Collectivismo}
\begin{itemize}
\item {Grp. gram.:m.}
\end{itemize}
\begin{itemize}
\item {Proveniência:(De \textunderscore collectivo\textunderscore )}
\end{itemize}
Systema sociológico, que procura tornar os meios de producção communs a todos os membros da sociedade.
\section{Collectivista}
\begin{itemize}
\item {Grp. gram.:m.}
\end{itemize}
Partidário do collectivismo.
\section{Collectivo}
\begin{itemize}
\item {Grp. gram.:adj.}
\end{itemize}
\begin{itemize}
\item {Utilização:Gram.}
\end{itemize}
\begin{itemize}
\item {Proveniência:(Lat. \textunderscore collectivus\textunderscore )}
\end{itemize}
Que abrange muitas coisas ou pessôas.
Relativo a muitas coisas ou pessôas.
Que no singular exprime o conjunto de indivíduos da mesma espécie.
\section{Collecto}
\begin{itemize}
\item {Grp. gram.:adj.}
\end{itemize}
\begin{itemize}
\item {Proveniência:(Lat. \textunderscore collectus\textunderscore )}
\end{itemize}
Colligido, escolhido.
\section{Collector}
\begin{itemize}
\item {Grp. gram.:m.}
\end{itemize}
\begin{itemize}
\item {Grp. gram.:Adj.}
\end{itemize}
\begin{itemize}
\item {Proveniência:(Do lat. \textunderscore collectus\textunderscore )}
\end{itemize}
Aquelle que lança ou recebe collectas.
Que collige; que reúne.
\section{Collectoria}
\begin{itemize}
\item {Grp. gram.:f.}
\end{itemize}
\begin{itemize}
\item {Utilização:Bras}
\end{itemize}
\begin{itemize}
\item {Proveniência:(De \textunderscore collector\textunderscore )}
\end{itemize}
Lugar, onde se pagam os impostos.
Cargo de collector.
\section{Collega}
\begin{itemize}
\item {Grp. gram.:m.  e  f.}
\end{itemize}
\begin{itemize}
\item {Proveniência:(Lat. \textunderscore collega\textunderscore )}
\end{itemize}
Pessôa, que, em relação a outra, faz parte da mesma communidade, corporação, profissão, etc.
Cada um dos que exercem a mesma profissão, especialmente na classe civil e ecclesiástica.--Na classe militar, prefere-se geralmente o t. \textunderscore camarada\textunderscore .
\section{Collegiada}
\begin{itemize}
\item {Grp. gram.:f.}
\end{itemize}
\begin{itemize}
\item {Proveniência:(De \textunderscore collégio\textunderscore )}
\end{itemize}
Conjunto dos alumnos de um collégio.
Corporação de sacerdotes, que têm funcções de cónegos, em igreja que não é episcopal: \textunderscore a collegiada de Guimarães\textunderscore .
Igreja, onde há essa corporação.
\section{Collegial}
\begin{itemize}
\item {Grp. gram.:adj.}
\end{itemize}
\begin{itemize}
\item {Grp. gram.:M.}
\end{itemize}
Relativo a collégio: \textunderscore a vida collegial\textunderscore .
Alumno de collégio.
\section{Collegiatura}
\begin{itemize}
\item {Grp. gram.:f.}
\end{itemize}
\begin{itemize}
\item {Proveniência:(De \textunderscore collégio\textunderscore )}
\end{itemize}
Qualidade do que é collegial.
\section{Collégio}
\begin{itemize}
\item {Grp. gram.:m.}
\end{itemize}
\begin{itemize}
\item {Utilização:Ant.}
\end{itemize}
\begin{itemize}
\item {Utilização:Gír.}
\end{itemize}
\begin{itemize}
\item {Proveniência:(Lat. \textunderscore collegium\textunderscore )}
\end{itemize}
Corporação, cujos membros têm igual dignidade.
Estabelecimento de ensino primário ou secundário.
Reunião de gente para fins eleitoraes.
Convento de jesuitas, com ónus de ensino.
Cárcere.
\section{Colleira}
\begin{itemize}
\item {Grp. gram.:f.}
\end{itemize}
\begin{itemize}
\item {Utilização:Bras}
\end{itemize}
\begin{itemize}
\item {Proveniência:(De \textunderscore collo\textunderscore )}
\end{itemize}
Armadura, resguardo de coiro, para o pescoço.
Peça de coiro ou metal, que se põe no pescoço dos cães e de outros animaes.
Fêmea do colleiro.
\section{Colleirado}
\begin{itemize}
\item {Grp. gram.:adj.}
\end{itemize}
Que tem colleira.
Que tem no pescoço malha ou pêlos, que dão a apparência de colleira.
\section{Colleirinho}
\begin{itemize}
\item {Grp. gram.:adj.}
\end{itemize}
\begin{itemize}
\item {Utilização:Ant.}
\end{itemize}
\begin{itemize}
\item {Proveniência:(De \textunderscore collo\textunderscore )}
\end{itemize}
Que ainda anda ao collo, (falando-se do menino).
\section{Colleiro}
\begin{itemize}
\item {Grp. gram.:m.}
\end{itemize}
\begin{itemize}
\item {Proveniência:(De \textunderscore collo\textunderscore )}
\end{itemize}
Ave canora do Brasil, que parece têr ao pescoço uma gravata preta.
\section{Colleitor}
\begin{itemize}
\item {Grp. gram.:m.}
\end{itemize}
\begin{itemize}
\item {Proveniência:(Lat. \textunderscore collector\textunderscore )}
\end{itemize}
Antigo dignitário ecclesiástico, encarregado pela Santa Sé de reivindicar e assegurar as rendas da Igreja, e de representar a côrte pontifícia, junto dos Governos, na ausência do Núncio.
\section{Collema}
\begin{itemize}
\item {Grp. gram.:m.}
\end{itemize}
\begin{itemize}
\item {Proveniência:(Do rad. do gr. \textunderscore kolla\textunderscore )}
\end{itemize}
Planta cryptogâmica das zonas temperadas.
\section{Collênchyma}
\begin{itemize}
\item {fónica:qui}
\end{itemize}
\begin{itemize}
\item {Grp. gram.:m.}
\end{itemize}
\begin{itemize}
\item {Proveniência:(Do gr. \textunderscore kolla\textunderscore  + \textunderscore egkhein\textunderscore )}
\end{itemize}
Tecido utricular, vegetal, caracterizado pela grande espessura dos utriculos constituintes.
\section{Collete}
\begin{itemize}
\item {Grp. gram.:m.}
\end{itemize}
\begin{itemize}
\item {Utilização:Bras}
\end{itemize}
\begin{itemize}
\item {Proveniência:(De \textunderscore collo\textunderscore )}
\end{itemize}
Peça de vestuário, curta e sem mangas, ajustada ao peito e abotoada na frente.
Espartilho.
Resguardo de madeira ou arame, nas hastes dos arbustos.
\section{Colleteiro}
\begin{itemize}
\item {Grp. gram.:m.}
\end{itemize}
\begin{itemize}
\item {Utilização:Bras}
\end{itemize}
Fabricante de colletes ou espartilhos.
\section{Collétia}
\begin{itemize}
\item {Grp. gram.:f.}
\end{itemize}
Gênero de plantas rhamnáceas.
\section{Collidir}
\begin{itemize}
\item {Grp. gram.:v. t.}
\end{itemize}
\begin{itemize}
\item {Grp. gram.:V. p.}
\end{itemize}
\begin{itemize}
\item {Proveniência:(Lat. \textunderscore collidere\textunderscore )}
\end{itemize}
Fazer ir (uma coisa) contra outra.
Ir de encontro; sêr opposto reciprocamente: \textunderscore isso são ideias que collidem\textunderscore .
\section{Collífero}
\begin{itemize}
\item {Grp. gram.:adj.}
\end{itemize}
\begin{itemize}
\item {Proveniência:(Do lat. \textunderscore collum\textunderscore  + \textunderscore ferre\textunderscore )}
\end{itemize}
Provido de collar, (falando-se do pedúnculo de certos cogumelos).
\section{Colligação}
\begin{itemize}
\item {Grp. gram.:f.}
\end{itemize}
\begin{itemize}
\item {Proveniência:(Lat. \textunderscore colligatio\textunderscore )}
\end{itemize}
Liga, alliança, de várias pessôas para um fim commum.
Assimilação.
Liga de quaesquer substâncias.
Confederação.
Trama.
\section{Colligar}
\begin{itemize}
\item {Grp. gram.:v. t.}
\end{itemize}
\begin{itemize}
\item {Proveniência:(Lat. \textunderscore colligare\textunderscore )}
\end{itemize}
Associar por colligação.
\section{Colligativo}
\begin{itemize}
\item {Grp. gram.:adj.}
\end{itemize}
\begin{itemize}
\item {Proveniência:(De \textunderscore colligar\textunderscore )}
\end{itemize}
Relativo a colligação.
Que colliga. Cf. \textunderscore Propriedades Colligativas das Soluções\textunderscore , por Ach. Machado.
\section{Colligir}
\begin{itemize}
\item {Grp. gram.:v. t.}
\end{itemize}
\begin{itemize}
\item {Proveniência:(Lat. \textunderscore colligere\textunderscore )}
\end{itemize}
Juntar; reünir em collecção: \textunderscore colligir apontamentos\textunderscore .
Inferir: \textunderscore eis o que collíjo dessas theorias\textunderscore .
\section{Collimação}
\begin{itemize}
\item {Grp. gram.:f.}
\end{itemize}
Acto de collimar.
\section{Collimar}
\begin{itemize}
\item {Grp. gram.:v. t.}
\end{itemize}
\begin{itemize}
\item {Proveniência:(Lat. \textunderscore collimare\textunderscore , palavra que se lê nalgumas edições, onde se deveria lêr \textunderscore collineare\textunderscore , seguir uma linha com a vista)}
\end{itemize}
Observar, por meio de instrumento.
\section{Collimitar}
\begin{itemize}
\item {Grp. gram.:v. t.}
\end{itemize}
\begin{itemize}
\item {Proveniência:(Lat. \textunderscore collimitare\textunderscore )}
\end{itemize}
Pôr limite commum a.
\section{Collina}
\begin{itemize}
\item {Grp. gram.:f.}
\end{itemize}
\begin{itemize}
\item {Proveniência:(Lat. \textunderscore collina\textunderscore )}
\end{itemize}
Pequena montanha, oiteiro; encosta.
\section{Collinoso}
\begin{itemize}
\item {Grp. gram.:adj.}
\end{itemize}
\begin{itemize}
\item {Proveniência:(De \textunderscore collina\textunderscore )}
\end{itemize}
Que tem muitas collinas.
\section{Collinsónia}
\begin{itemize}
\item {Grp. gram.:f.}
\end{itemize}
\begin{itemize}
\item {Proveniência:(De \textunderscore Collinson\textunderscore , n. p.)}
\end{itemize}
Planta labiada da América do Norte.
\section{Collipeu}
\begin{itemize}
\item {Grp. gram.:m.}
\end{itemize}
Bichinho fantástico:«\textunderscore descobriu um collipeu nas ervas do seu jardim..., bichinho singular... que tem virtudes para curar de pleurizes... e de omoplatas nas pernas...\textunderscore »Castilho, \textunderscore Méd. á Fôrça\textunderscore , 170.
\section{Colliquação}
\begin{itemize}
\item {Grp. gram.:f.}
\end{itemize}
\begin{itemize}
\item {Proveniência:(Lat. \textunderscore colliquatio\textunderscore )}
\end{itemize}
Dissolução orgânica, com excreções abundantes.
\section{Colliquante}
\begin{itemize}
\item {Grp. gram.:adj.}
\end{itemize}
\begin{itemize}
\item {Proveniência:(Do rad. do lat. \textunderscore colliquescere\textunderscore )}
\end{itemize}
Que dissolve; que derrete.
\section{Colliquar}
\begin{itemize}
\item {Grp. gram.:v. t.}
\end{itemize}
Fundir, derreter. Cf. \textunderscore Ancora Méd.\textunderscore , 7.
\section{Colliquativo}
\begin{itemize}
\item {Grp. gram.:adj.}
\end{itemize}
Que é produzido pela colliquação.
(Cp. \textunderscore colliquação\textunderscore )
\section{Collirostros}
\begin{itemize}
\item {fónica:rós}
\end{itemize}
\begin{itemize}
\item {Grp. gram.:m. pl.}
\end{itemize}
\begin{itemize}
\item {Proveniência:(Do lat. \textunderscore collum\textunderscore  + \textunderscore rostrum\textunderscore )}
\end{itemize}
Família de insectos hemípteros, cujo bico parece nascer do pescoço.
\section{Collisão}
\begin{itemize}
\item {Grp. gram.:f.}
\end{itemize}
\begin{itemize}
\item {Proveniência:(Lat. \textunderscore collisio\textunderscore )}
\end{itemize}
Embate recíproco de dois corpos.
Luta.
Contrariedade.
Alternativa, difficuldade de opção: \textunderscore na collisão de dois extremos\textunderscore .
\section{Collitigar}
\begin{itemize}
\item {Grp. gram.:v. t.}
\end{itemize}
\begin{itemize}
\item {Proveniência:(De \textunderscore co...\textunderscore  + \textunderscore litigar\textunderscore )}
\end{itemize}
Litigar juntamente com outrem.
\section{Collo}
\begin{itemize}
\item {Grp. gram.:m.}
\end{itemize}
\begin{itemize}
\item {Utilização:Ant.}
\end{itemize}
\begin{itemize}
\item {Utilização:Ant.}
\end{itemize}
\begin{itemize}
\item {Proveniência:(Lat. \textunderscore collum\textunderscore )}
\end{itemize}
O mesmo que \textunderscore pescoço\textunderscore .
Nome de vários objectos, que têm analogia com a fórma do collo.
Regaço.
\textunderscore Collo do pé\textunderscore , peito do pé.
O mesmo que \textunderscore desfiladeiro\textunderscore .
O mesmo que \textunderscore cabeça\textunderscore .
\section{Collocação}
\begin{itemize}
\item {Grp. gram.:f.}
\end{itemize}
\begin{itemize}
\item {Proveniência:(Lat. \textunderscore collocatio\textunderscore )}
\end{itemize}
Acto de collocar.
Situação.
Emprêgo: \textunderscore collocação de fundos\textunderscore .
\section{Colimbo}
\begin{itemize}
\item {Grp. gram.:m.}
\end{itemize}
\begin{itemize}
\item {Proveniência:(Gr. \textunderscore kolumbos\textunderscore )}
\end{itemize}
O mesmo que \textunderscore mergulhão\textunderscore , ave.
\section{Colírio}
\begin{itemize}
\item {Grp. gram.:m.}
\end{itemize}
\begin{itemize}
\item {Proveniência:(Gr. \textunderscore kollurion\textunderscore )}
\end{itemize}
Medicamento, destinado especialmente á cura de inflamações na conjuntiva.
\section{Collocador}
\begin{itemize}
\item {Grp. gram.:m.}
\end{itemize}
\begin{itemize}
\item {Proveniência:(De \textunderscore collocar\textunderscore )}
\end{itemize}
Aquelle que colloca qualquer coisa: \textunderscore precisam-se collocadores de campaínhas e apparelhos eléctricos\textunderscore .
\section{Collocar}
\begin{itemize}
\item {Grp. gram.:v. t.}
\end{itemize}
\begin{itemize}
\item {Proveniência:(Lat. \textunderscore collocare\textunderscore )}
\end{itemize}
Pôr num logar: \textunderscore collocar um livro numa estante\textunderscore .
Dispor.
Empregar: \textunderscore collocar os afilhados\textunderscore .
\section{Collocutor}
\begin{itemize}
\item {Grp. gram.:m.}
\end{itemize}
\begin{itemize}
\item {Proveniência:(Lat. \textunderscore collocutor\textunderscore )}
\end{itemize}
Aquelle que fala com outro.
\section{Collódio}
\begin{itemize}
\item {Grp. gram.:m.}
\end{itemize}
\begin{itemize}
\item {Proveniência:(Do gr. \textunderscore kollodes\textunderscore )}
\end{itemize}
Substância chímica, obtida pela solução do algodão-pólvora, em éther e alcool.
\section{Collodionar}
\begin{itemize}
\item {Grp. gram.:v.}
\end{itemize}
\begin{itemize}
\item {Utilização:t. Phot.}
\end{itemize}
Cobrir com delgada camada de collódio (uma placa de crystal), para augmentar o brilho de certas provas photográphicas.
\section{Collodionito}
\begin{itemize}
\item {Grp. gram.:m.}
\end{itemize}
Medicamento, que tem por excipiente o collódio.
\section{Colloidal}
\begin{itemize}
\item {Grp. gram.:adj.}
\end{itemize}
Que tem a apparência e a transparência da colla^1.
(Cp. \textunderscore colloide\textunderscore )
\section{Colloide}
\begin{itemize}
\item {Grp. gram.:adj.}
\end{itemize}
\begin{itemize}
\item {Proveniência:(Do gr. \textunderscore kolla\textunderscore  + \textunderscore eidos\textunderscore )}
\end{itemize}
Semelhante á colla^2.
\section{Colloquial}
\begin{itemize}
\item {Grp. gram.:adj.}
\end{itemize}
Relativo a collóquio.
\section{Collóquio}
\begin{itemize}
\item {Grp. gram.:m.}
\end{itemize}
\begin{itemize}
\item {Proveniência:(Lat. \textunderscore colloquium\textunderscore )}
\end{itemize}
Conversação, palestra, entre duas ou mais pessôas.
\section{Colloxylina}
\begin{itemize}
\item {Grp. gram.:f.}
\end{itemize}
\begin{itemize}
\item {Proveniência:(De \textunderscore collódio\textunderscore  + gr. \textunderscore xulos\textunderscore )}
\end{itemize}
O mesmo que \textunderscore pyroxilina\textunderscore .
\section{Colluio}
\begin{itemize}
\item {Grp. gram.:m.}
\end{itemize}
(V.conluio)
\section{Collutório}
\begin{itemize}
\item {Grp. gram.:m.}
\end{itemize}
\begin{itemize}
\item {Proveniência:(Do lat. \textunderscore collutus\textunderscore )}
\end{itemize}
Qualquer líquido medicinal, para as mucosas da bôca.
\section{Colluvião}
\begin{itemize}
\item {Grp. gram.:f.}
\end{itemize}
\begin{itemize}
\item {Proveniência:(Lat. \textunderscore colluvio\textunderscore )}
\end{itemize}
O mesmo que \textunderscore inundação\textunderscore .
\section{Collýrio}
\begin{itemize}
\item {Grp. gram.:m.}
\end{itemize}
\begin{itemize}
\item {Proveniência:(Gr. \textunderscore kollurion\textunderscore )}
\end{itemize}
Medicamento, destinado especialmente á cura de inflammações na conjuntiva.
\section{Colma}
\begin{itemize}
\item {Grp. gram.:f.}
\end{itemize}
Nome de duas árvores medicinaes, na ilha de San-Thomé.
\section{Colmaça}
\begin{itemize}
\item {Grp. gram.:adj. f.}
\end{itemize}
\begin{itemize}
\item {Utilização:Prov.}
\end{itemize}
\begin{itemize}
\item {Utilização:minh.}
\end{itemize}
Diz-se da casa coberta de colmo.
\section{Colmaçar}
\begin{itemize}
\item {Grp. gram.:v. t.}
\end{itemize}
\begin{itemize}
\item {Proveniência:(De \textunderscore colmaço\textunderscore )}
\end{itemize}
Cobrir de colmo.
\section{Colmaço}
\begin{itemize}
\item {Grp. gram.:m.}
\end{itemize}
\begin{itemize}
\item {Utilização:Prov.}
\end{itemize}
\begin{itemize}
\item {Utilização:trasm.}
\end{itemize}
Cobertura de colmo de um palheiro, de uma cabana, etc.; colmado.
\section{Colmado}
\begin{itemize}
\item {Grp. gram.:m.}
\end{itemize}
\begin{itemize}
\item {Utilização:Prov.}
\end{itemize}
\begin{itemize}
\item {Utilização:minh.}
\end{itemize}
\begin{itemize}
\item {Proveniência:(De \textunderscore colmar\textunderscore ^1)}
\end{itemize}
Pequena casa, coberta de colmo.
Palhoça, cabana, choupana.
Casa rústica, que domina fazenda ou herdade.
\section{Colmar}
\begin{itemize}
\item {Grp. gram.:v. t.}
\end{itemize}
Cobrir de colmo.
\section{Colmar}
\begin{itemize}
\item {Grp. gram.:v. t.}
\end{itemize}
\begin{itemize}
\item {Proveniência:(Do it. \textunderscore colmare\textunderscore )}
\end{itemize}
Elevar ao ponto mais alto, sublimar.
Encher: \textunderscore colmar alguém de benefícios\textunderscore .
\section{Colmar}
\begin{itemize}
\item {Grp. gram.:adj.}
\end{itemize}
Diz-se de uma espécie de pêra serôdia. Cf. B. Pato, \textunderscore Livro do Monte\textunderscore .
\section{Colmata}
\begin{itemize}
\item {Grp. gram.:f.}
\end{itemize}
Acto de colmatar.
\section{Colmatagem}
\begin{itemize}
\item {Grp. gram.:f.}
\end{itemize}
\begin{itemize}
\item {Utilização:Neol.}
\end{itemize}
\begin{itemize}
\item {Proveniência:(Fr. \textunderscore colmatage\textunderscore )}
\end{itemize}
Depósito ou sobreposição de terras, resultante de obras de arte ou escavações para plantio de arvoredo.
\section{Colmatar}
\begin{itemize}
\item {Grp. gram.:v. t.}
\end{itemize}
\begin{itemize}
\item {Proveniência:(Do it. \textunderscore colmata\textunderscore )}
\end{itemize}
Fazer colmatagem a. Cf. Assis, \textunderscore Águas\textunderscore , 165 e 166.
\section{Colmeal}
\begin{itemize}
\item {Grp. gram.:m.}
\end{itemize}
Lugar, onde há colmeias.
Silhal; porção de colmeias.
\section{Colmeeiro}
\begin{itemize}
\item {Grp. gram.:m.}
\end{itemize}
Aquelle que trata de colmeias ou que negocía com colmeias.
\section{Colmeia}
\begin{itemize}
\item {Grp. gram.:f.}
\end{itemize}
\begin{itemize}
\item {Utilização:Fig.}
\end{itemize}
\begin{itemize}
\item {Proveniência:(Do cast. \textunderscore colmena\textunderscore )}
\end{itemize}
Cortiço de abelhas.
Enxame de abelhas.
Accumulação, grande porção de coisas ou pessôas.
\section{Colmeiro}
\begin{itemize}
\item {Grp. gram.:m.}
\end{itemize}
\begin{itemize}
\item {Utilização:Prov.}
\end{itemize}
\begin{itemize}
\item {Utilização:minh.}
\end{itemize}
\begin{itemize}
\item {Utilização:Ant.}
\end{itemize}
Mólho de colmo ou de palha.
\section{Colmeirôa}
\begin{itemize}
\item {Grp. gram.:f.}
\end{itemize}
Gênero de plantas euphorbiáceas.
\section{Colmífero}
\begin{itemize}
\item {Grp. gram.:adj.}
\end{itemize}
Diz-se dos cereaes, cuja haste é colmo.
O mesmo que \textunderscore cerealífero\textunderscore . Cf. Herculano, \textunderscore Opúsc.\textunderscore , IV, 119, 144 e 160.
\section{Colmilho}
\begin{itemize}
\item {Grp. gram.:m.}
\end{itemize}
Dente agudo, presa.
(Cast. \textunderscore colmillo\textunderscore )
\section{Colmilhoso}
\begin{itemize}
\item {Grp. gram.:adj.}
\end{itemize}
Que tem grandes colmilhos.
\section{Colmilhudo}
\begin{itemize}
\item {Grp. gram.:adj.}
\end{itemize}
Que tem grandes colmilhos.
\section{Colmo}
\begin{itemize}
\item {Grp. gram.:m.}
\end{itemize}
\begin{itemize}
\item {Utilização:Fig.}
\end{itemize}
\begin{itemize}
\item {Proveniência:(Do lat. \textunderscore culmus\textunderscore )}
\end{itemize}
Caule das plantas gramíneas, entre a raiz e a espiga.
Caule do junco e da junça.
Colmado.
\section{Colo}
\begin{itemize}
\item {Grp. gram.:m.}
\end{itemize}
\begin{itemize}
\item {Utilização:Ant.}
\end{itemize}
\begin{itemize}
\item {Utilização:Ant.}
\end{itemize}
\begin{itemize}
\item {Proveniência:(Lat. \textunderscore collum\textunderscore )}
\end{itemize}
O mesmo que \textunderscore pescoço\textunderscore .
Nome de vários objectos, que têm analogia com a fórma do colo.
Regaço.
\textunderscore Colo do pé\textunderscore , peito do pé.
O mesmo que \textunderscore desfiladeiro\textunderscore .
O mesmo que \textunderscore cabeça\textunderscore .
\section{Colóbio}
\begin{itemize}
\item {Grp. gram.:m.}
\end{itemize}
\begin{itemize}
\item {Utilização:Ant.}
\end{itemize}
Túnica sem mangas.
Dalmática.
\section{Coloboma}
\begin{itemize}
\item {Grp. gram.:m.}
\end{itemize}
\begin{itemize}
\item {Proveniência:(Do gr. \textunderscore kolobos\textunderscore )}
\end{itemize}
Em Ophthalmologia, chama-se assim a falta de uma parte de um órgão.
\section{Colocação}
\begin{itemize}
\item {Grp. gram.:f.}
\end{itemize}
\begin{itemize}
\item {Proveniência:(Lat. \textunderscore collocatio\textunderscore )}
\end{itemize}
Acto de colocar.
Situação.
Emprêgo: \textunderscore colocação de fundos\textunderscore .
\section{Colocador}
\begin{itemize}
\item {Grp. gram.:m.}
\end{itemize}
\begin{itemize}
\item {Proveniência:(De \textunderscore collocar\textunderscore )}
\end{itemize}
Aquele que coloca qualquer coisa: \textunderscore precisam-se colocadores de campaínhas e aparelhos eléctricos\textunderscore .
\section{Colocar}
\begin{itemize}
\item {Grp. gram.:v. t.}
\end{itemize}
\begin{itemize}
\item {Proveniência:(Lat. \textunderscore collocare\textunderscore )}
\end{itemize}
Pôr num logar: \textunderscore colocar um livro numa estante\textunderscore .
Dispor.
Empregar: \textunderscore colocar os afilhados\textunderscore .
\section{Colocásia}
\begin{itemize}
\item {Grp. gram.:f.}
\end{itemize}
\begin{itemize}
\item {Proveniência:(Lat. \textunderscore colocasia\textunderscore )}
\end{itemize}
Planta arácea.
Inhame do Egypto.
\section{Colocutor}
\begin{itemize}
\item {Grp. gram.:m.}
\end{itemize}
\begin{itemize}
\item {Proveniência:(Lat. \textunderscore collocutor\textunderscore )}
\end{itemize}
Aquele que fala com outro.
\section{Colódio}
\begin{itemize}
\item {Grp. gram.:m.}
\end{itemize}
\begin{itemize}
\item {Proveniência:(Do gr. \textunderscore kollodes\textunderscore )}
\end{itemize}
Substância química, obtida pela solução do algodão-pólvora, em éter e alcool.
\section{Colodionar}
\begin{itemize}
\item {Grp. gram.:v.}
\end{itemize}
\begin{itemize}
\item {Utilização:t. Phot.}
\end{itemize}
Cobrir com delgada camada de colódio (uma placa de cristal), para aumentar o brilho de certas provas fotográficas.
\section{Colodionito}
\begin{itemize}
\item {Grp. gram.:m.}
\end{itemize}
Medicamento, que tem por excipiente o colódio.
\section{Colodra}
\begin{itemize}
\item {fónica:lô}
\end{itemize}
\begin{itemize}
\item {Grp. gram.:f.}
\end{itemize}
O mesmo que \textunderscore colondro\textunderscore .
\section{Colodro}
\begin{itemize}
\item {fónica:lô}
\end{itemize}
\begin{itemize}
\item {Grp. gram.:m.}
\end{itemize}
O mesmo que \textunderscore colondro\textunderscore .
\section{Cólofon}
\begin{itemize}
\item {Grp. gram.:m.}
\end{itemize}
\begin{itemize}
\item {Proveniência:(T. gr.)}
\end{itemize}
Dístico final, em manuscritos madievaes, relativo ao autor ou escriba, ao lugar onde se escreveu a obra e á data dela.
\section{Colofónia}
\begin{itemize}
\item {Grp. gram.:f.}
\end{itemize}
\begin{itemize}
\item {Proveniência:(Gr. \textunderscore kolophonia\textunderscore )}
\end{itemize}
Espécie de pez ou resina, resíduo da destilação da terebentina.
\section{Coloide}
\begin{itemize}
\item {Grp. gram.:adj.}
\end{itemize}
\begin{itemize}
\item {Proveniência:(Do gr. \textunderscore kolla\textunderscore  + \textunderscore eidos\textunderscore )}
\end{itemize}
Semelhante á cola^6.
\section{Colomba}
\begin{itemize}
\item {Grp. gram.:f.}
\end{itemize}
\begin{itemize}
\item {Utilização:Ant.}
\end{itemize}
Feixe ou mólho, que um homem ou mulher póde levar ás costas.
\section{Colômbia}
\begin{itemize}
\item {Grp. gram.:f.}
\end{itemize}
Espécie de tabaco.
\section{Colombiano}
\begin{itemize}
\item {Grp. gram.:adj.}
\end{itemize}
Relativo a Colombo, aos seus tempos ou aos seus descobrimentos.
\section{Colômbio}
\begin{itemize}
\item {Grp. gram.:m.}
\end{itemize}
\begin{itemize}
\item {Utilização:Phýs.}
\end{itemize}
\begin{itemize}
\item {Proveniência:(De \textunderscore Coulomb\textunderscore , n. p.)}
\end{itemize}
Unidade prática de quantidade electrica.
\section{Colombo}
\begin{itemize}
\item {Grp. gram.:m.}
\end{itemize}
Grande árvore da Índia portuguesa.
\section{Colombo}
\begin{itemize}
\item {Grp. gram.:m.}
\end{itemize}
O mesmo que \textunderscore colomba\textunderscore .
\section{Colombolo}
\begin{itemize}
\item {Grp. gram.:m.}
\end{itemize}
Serpente de Angola, (\textunderscore rhagerrhis tritaeniatus\textunderscore ).
\section{Colombro}
\begin{itemize}
\item {Grp. gram.:m.}
\end{itemize}
O mesmo que \textunderscore colondro\textunderscore . Cf. \textunderscore Pharm. Port.\textunderscore 
\section{Colomi}
\begin{itemize}
\item {Grp. gram.:m.}
\end{itemize}
Rapaz.
Criado.
\section{Colomim}
\begin{itemize}
\item {Grp. gram.:m.}
\end{itemize}
\begin{itemize}
\item {Utilização:Bras}
\end{itemize}
Rapaz.
Criado.
\section{Cólon}
\begin{itemize}
\item {Grp. gram.:m.}
\end{itemize}
\begin{itemize}
\item {Proveniência:(Gr. \textunderscore kolon\textunderscore )}
\end{itemize}
Parte do intestino grosso, em seguida ao ceco.
\section{Colonato}
\begin{itemize}
\item {Grp. gram.:m.}
\end{itemize}
Estado de colono.
Instituição de colonos. Cf. Herculano, \textunderscore Opúsc.\textunderscore , III, 250; IV, 89; \textunderscore Hist. de Port.\textunderscore , III, 228 e 248.
\section{Colondro}
\begin{itemize}
\item {Grp. gram.:m.}
\end{itemize}
\begin{itemize}
\item {Proveniência:(Do gr. \textunderscore kulindros\textunderscore )}
\end{itemize}
Fruto comprido de algumas cucurbitáceas.
Cabaço.
\section{Colonho}
\begin{itemize}
\item {Grp. gram.:m.}
\end{itemize}
\begin{itemize}
\item {Grp. gram.:m.}
\end{itemize}
\begin{itemize}
\item {Utilização:Prov.}
\end{itemize}
O mesmo que \textunderscore colomba\textunderscore .
Feixe ou carga, que se leva á cabeça.
\section{Colónia}
\begin{itemize}
\item {Grp. gram.:f.}
\end{itemize}
\begin{itemize}
\item {Proveniência:(Lat. \textunderscore colonia\textunderscore )}
\end{itemize}
Povoação de colonos.
Reunião de emigrantes, que se estabelece em país estranho.
Grupo de trabalhadores, que, saindo da sua provincia, vão estabelecer-se e trabalhar noutra, dentro do seu país.
\section{Colónía}
\begin{itemize}
\item {Grp. gram.:f.}
\end{itemize}
\begin{itemize}
\item {Proveniência:(De \textunderscore colono\textunderscore )}
\end{itemize}
Contrato entre colono e proprietário, na ilha da Madeira, pelo qual o colono perde o direito ás bem-feitorias prediaes.
\section{Colonial}
\begin{itemize}
\item {Grp. gram.:adj.}
\end{itemize}
\begin{itemize}
\item {Grp. gram.:M.}
\end{itemize}
Relativo a colónia, ou ás colónias: \textunderscore estudos coloniaes\textunderscore .
Aquelle que conhece bem as colónias ou que se dedica a assumptos coloniaes.
\section{Colónico}
\begin{itemize}
\item {Grp. gram.:adj.}
\end{itemize}
\begin{itemize}
\item {Utilização:Bras}
\end{itemize}
Relativo a colonos ou a colónias.
\section{Colonista}
\begin{itemize}
\item {Grp. gram.:m.}
\end{itemize}
\begin{itemize}
\item {Utilização:Bras}
\end{itemize}
Aquelle que se dedica a questões relacionadas com colonos ou colónias.
\section{Colonização}
\begin{itemize}
\item {Grp. gram.:f.}
\end{itemize}
Acto ou effeito de colonizar.
\section{Colonizador}
\begin{itemize}
\item {Grp. gram.:m.  e  adj.}
\end{itemize}
O que coloniza.
\section{Colonizar}
\begin{itemize}
\item {Grp. gram.:v. t.}
\end{itemize}
\begin{itemize}
\item {Proveniência:(De \textunderscore colono\textunderscore )}
\end{itemize}
Estabelecer colónia em: \textunderscore colonizar o planalto de Huíla\textunderscore . Habitar como colono.
\section{Colonizável}
\begin{itemize}
\item {Grp. gram.:adj.}
\end{itemize}
Que póde sêr colonizado.
\section{Colono}
\begin{itemize}
\item {Grp. gram.:m.}
\end{itemize}
\begin{itemize}
\item {Proveniência:(Lat. \textunderscore colonus\textunderscore )}
\end{itemize}
Membro de uma colónia.
Cultivador.
\section{Cólophon}
\begin{itemize}
\item {Grp. gram.:m.}
\end{itemize}
\begin{itemize}
\item {Proveniência:(T. gr.)}
\end{itemize}
Dístico final, em manuscritos madievaes, relativo ao autor ou escriba, ao lugar onde se escreveu a obra e á data della.
\section{Colophónia}
\begin{itemize}
\item {Grp. gram.:f.}
\end{itemize}
\begin{itemize}
\item {Proveniência:(Gr. \textunderscore kolophonia\textunderscore )}
\end{itemize}
Espécie de pez ou resina, resíduo da destillação da terebenthina.
\section{Coloquial}
\begin{itemize}
\item {Grp. gram.:adj.}
\end{itemize}
Relativo a colóquio.
\section{Coloquíntida}
\begin{itemize}
\item {Grp. gram.:f.}
\end{itemize}
\begin{itemize}
\item {Proveniência:(Do gr. \textunderscore kolokuntha\textunderscore )}
\end{itemize}
Espécie de pepino amargo.
\section{Colóquio}
\begin{itemize}
\item {Grp. gram.:m.}
\end{itemize}
\begin{itemize}
\item {Proveniência:(Lat. \textunderscore colloquium\textunderscore )}
\end{itemize}
Conversação, palestra, entre duas ou mais pessôas.
\section{Color}
\begin{itemize}
\item {Grp. gram.:f.}
\end{itemize}
\begin{itemize}
\item {Utilização:Ant.}
\end{itemize}
\begin{itemize}
\item {Grp. gram.:M.}
\end{itemize}
\begin{itemize}
\item {Grp. gram.:Loc. prep.}
\end{itemize}
\begin{itemize}
\item {Proveniência:(Lat. \textunderscore color\textunderscore )}
\end{itemize}
O mesmo que côr:«\textunderscore mestura das colores\textunderscore ». Usque, \textunderscore Tribulações\textunderscore , 8 v.^o
Ornato, adôrno. Cf. \textunderscore Eufrosina\textunderscore , 259.
\textunderscore Sob color de\textunderscore , a pretexto de, sob a apparência de.
\section{Coloração}
\begin{itemize}
\item {Grp. gram.:f.}
\end{itemize}
Acto de colorar.
\section{Colorado}
\begin{itemize}
\item {Grp. gram.:adj.}
\end{itemize}
\begin{itemize}
\item {Utilização:Bras. do S}
\end{itemize}
\begin{itemize}
\item {Proveniência:(T. cast.)}
\end{itemize}
Vermelho.
\section{Colorante}
\begin{itemize}
\item {Grp. gram.:adj.}
\end{itemize}
Que colora.
\section{Colorar}
\begin{itemize}
\item {Grp. gram.:v. t.}
\end{itemize}
\begin{itemize}
\item {Utilização:Ant.}
\end{itemize}
\begin{itemize}
\item {Proveniência:(Lat. \textunderscore colorare\textunderscore )}
\end{itemize}
O mesmo que \textunderscore colorir\textunderscore .
Pretextar.
\section{Colorau}
\begin{itemize}
\item {Grp. gram.:m.}
\end{itemize}
\begin{itemize}
\item {Proveniência:(Do cast. \textunderscore colorado\textunderscore )}
\end{itemize}
Pó vermelho e condimentoso, extrahido do pimentão. Cf. Camillo, \textunderscore Canc. Aleg.\textunderscore , 128.
\section{Colorear}
\textunderscore v. t.\textunderscore  (e der.)
O mesmo que \textunderscore colorir\textunderscore , etc.
\section{Colorido}
\begin{itemize}
\item {Grp. gram.:m.}
\end{itemize}
Côr, combinação de côres, em pintura, em vegetação, etc.
\section{Colorífico}
\begin{itemize}
\item {Grp. gram.:adj.}
\end{itemize}
\begin{itemize}
\item {Proveniência:(Do lat. \textunderscore color\textunderscore  + \textunderscore facere\textunderscore )}
\end{itemize}
Que produz côr.
\section{Colorígrado}
\begin{itemize}
\item {Grp. gram.:m.}
\end{itemize}
\begin{itemize}
\item {Proveniência:(Do lat. \textunderscore color\textunderscore  + \textunderscore gradus\textunderscore )}
\end{itemize}
Instrumento, que determina o grau da coloração dos corpos.
\section{Colorímetro}
\begin{itemize}
\item {Grp. gram.:m.}
\end{itemize}
\begin{itemize}
\item {Proveniência:(Do lat. \textunderscore color\textunderscore  + gr. \textunderscore metron\textunderscore )}
\end{itemize}
Instrumento, para determinar a fôrça còrante das substâncias empregadas nas indústrias.
\section{Colorina}
\begin{itemize}
\item {Grp. gram.:f.}
\end{itemize}
\begin{itemize}
\item {Proveniência:(Do lat. \textunderscore color\textunderscore )}
\end{itemize}
Substância còrante, extrahída da raiz da ruiva.
\section{Colorir}
\begin{itemize}
\item {Grp. gram.:v. t.}
\end{itemize}
\begin{itemize}
\item {Utilização:Fig.}
\end{itemize}
\begin{itemize}
\item {Proveniência:(De \textunderscore color\textunderscore )}
\end{itemize}
Dar côr ou côres a.
Matizar.
Tornar brilhante, descrever brilhantemente.
Disfarçar.
Cohonestar.
\section{Colorismo}
\begin{itemize}
\item {Grp. gram.:m.}
\end{itemize}
\begin{itemize}
\item {Utilização:Neol.}
\end{itemize}
Systema ou escola de colorista.
\section{Colorista}
\begin{itemize}
\item {Grp. gram.:m.}
\end{itemize}
\begin{itemize}
\item {Utilização:Fig.}
\end{itemize}
\begin{itemize}
\item {Proveniência:(De \textunderscore color\textunderscore )}
\end{itemize}
Aquelle que se occupa em colorir.
Pintor, que sobresai no emprêgo e combinações das côres.
Escritor, que sobresai pelo brilho e vivacidade do estilo.
\section{Colorização}
\begin{itemize}
\item {Grp. gram.:f.}
\end{itemize}
\begin{itemize}
\item {Proveniência:(De \textunderscore colorizar\textunderscore )}
\end{itemize}
Manifestação de uma côr.
Mudança de côr.
\section{Colorizar}
\textunderscore v. t.\textunderscore  (des.)(V.colorir)
\section{Colossal}
\begin{itemize}
\item {Grp. gram.:adj.}
\end{itemize}
\begin{itemize}
\item {Proveniência:(De \textunderscore colosso\textunderscore )}
\end{itemize}
Muito grande, enorme, vastíssimo: \textunderscore império colossal\textunderscore .
\section{Colosso}
\begin{itemize}
\item {fónica:lô}
\end{itemize}
\begin{itemize}
\item {Grp. gram.:m.}
\end{itemize}
\begin{itemize}
\item {Utilização:Fig.}
\end{itemize}
\begin{itemize}
\item {Proveniência:(Lat. \textunderscore colossus\textunderscore )}
\end{itemize}
Estátua enorme.
Pessôa corpulenta.
Objecto, com grandes dimensões.
Grande poderio.
Quem tem grande poderio ou valimento.
\section{Colostração}
\begin{itemize}
\item {Grp. gram.:f.}
\end{itemize}
\begin{itemize}
\item {Proveniência:(Lat. \textunderscore colostratio\textunderscore )}
\end{itemize}
Doença de crianças, produzida pelo colostro.
\section{Colostro}
\begin{itemize}
\item {fónica:lôs}
\end{itemize}
\begin{itemize}
\item {Grp. gram.:m.}
\end{itemize}
\begin{itemize}
\item {Proveniência:(Lat. \textunderscore colostrum\textunderscore )}
\end{itemize}
Primeiro leite da mulher, logo depois do parto.
Leite de animaes, produzido logo depois do parto, e que é aguado ou pouco nutritivo.
\section{Coloxilina}
\begin{itemize}
\item {Grp. gram.:f.}
\end{itemize}
\begin{itemize}
\item {Proveniência:(De \textunderscore collódio\textunderscore  + gr. \textunderscore xulos\textunderscore )}
\end{itemize}
O mesmo que \textunderscore piroxilina\textunderscore .
\section{Cólpodes}
\begin{itemize}
\item {Grp. gram.:m. pl.}
\end{itemize}
\begin{itemize}
\item {Proveniência:(Do gr. \textunderscore kolpos\textunderscore )}
\end{itemize}
Gênero de infusórios da água estagnada.
\section{Colpódio}
\begin{itemize}
\item {Grp. gram.:m.}
\end{itemize}
Gênero de plantas gramíneas.
\section{Colquicáceas}
\begin{itemize}
\item {Grp. gram.:f. pl.}
\end{itemize}
Família de plantas herbáceas, que têm por tipo o \textunderscore cólquico\textunderscore .
\section{Colquicina}
\begin{itemize}
\item {Grp. gram.:f.}
\end{itemize}
\begin{itemize}
\item {Proveniência:(De \textunderscore cólchico\textunderscore )}
\end{itemize}
Alcaloide, que se descobriu nas sementes do cólquico.
\section{Cólquico}
\begin{itemize}
\item {Grp. gram.:m.}
\end{itemize}
\begin{itemize}
\item {Proveniência:(De \textunderscore Colchor\textunderscore , n. p.)}
\end{itemize}
Lírio verde, (\textunderscore colchicum\textunderscore )
\section{Colra}
\begin{itemize}
\item {Grp. gram.:f.}
\end{itemize}
\begin{itemize}
\item {Utilização:Prov.}
\end{itemize}
\begin{itemize}
\item {Utilização:alg.}
\end{itemize}
\begin{itemize}
\item {Utilização:dur.}
\end{itemize}
O mesmo que \textunderscore cólera\textunderscore .
(Contr. de \textunderscore cólera\textunderscore )
\section{Coltar}
\begin{itemize}
\item {Grp. gram.:m.}
\end{itemize}
\begin{itemize}
\item {Proveniência:(Ingl. \textunderscore coaltar\textunderscore , de \textunderscore coal\textunderscore , carvão, e \textunderscore tar\textunderscore , alcatrão)}
\end{itemize}
Alcatrão, produzido pela destillação da hulha.
Fezes do gás, breu.
\section{Coltarização}
\begin{itemize}
\item {Grp. gram.:f.}
\end{itemize}
Acto de coltarizar.
\section{Coltarizar}
\begin{itemize}
\item {Grp. gram.:v. t.}
\end{itemize}
Revestir de coltar.
\section{Colubreado}
\begin{itemize}
\item {Grp. gram.:adj.}
\end{itemize}
\begin{itemize}
\item {Proveniência:(Do lat. \textunderscore coluber\textunderscore )}
\end{itemize}
Que tem fórma de cobra.
Traçado em fórma de cobra.
\section{Colubrejar}
\begin{itemize}
\item {Grp. gram.:v. i.}
\end{itemize}
\begin{itemize}
\item {Utilização:Ant.}
\end{itemize}
\begin{itemize}
\item {Proveniência:(Do lat. \textunderscore colubra\textunderscore , cobra)}
\end{itemize}
Collear, como a serpente.
\section{Colubreta}
\begin{itemize}
\item {fónica:brê}
\end{itemize}
\begin{itemize}
\item {Grp. gram.:f.}
\end{itemize}
\begin{itemize}
\item {Proveniência:(Do lat. \textunderscore colubra\textunderscore )}
\end{itemize}
Antiga peça de artilharia. Cf. Azurara, \textunderscore Chrón. de D. João I\textunderscore , c. LXXI.
\section{Colubrídeas}
\begin{itemize}
\item {Grp. gram.:f. pl.}
\end{itemize}
\begin{itemize}
\item {Proveniência:(Do lat. \textunderscore colubra\textunderscore  + gr. \textunderscore eidos\textunderscore )}
\end{itemize}
Família de reptis, que tem por typo a cobra.
\section{Colubrina}
\begin{itemize}
\item {Grp. gram.:f.}
\end{itemize}
\begin{itemize}
\item {Proveniência:(De \textunderscore colubrino\textunderscore )}
\end{itemize}
Planta rhamnacea.
Briónia.
Antiga peça de artilharia.
Espada antiga, de lâmina sinuosa.
\section{Colubrineiro}
\begin{itemize}
\item {Grp. gram.:m.}
\end{itemize}
Soldado, que manejava a colubrina.
\section{Colubrino}
\begin{itemize}
\item {Grp. gram.:adj.}
\end{itemize}
\begin{itemize}
\item {Proveniência:(Lat. \textunderscore colubrinus\textunderscore )}
\end{itemize}
Relativo ou semelhante á cobra.
\section{Colugli}
\begin{itemize}
\item {Grp. gram.:m.}
\end{itemize}
Argelino, filho de pai turco e mãe indígena.
(Do turco \textunderscore kul\textunderscore , escravo, e \textunderscore oghli\textunderscore , filho)
\section{Coluio}
\begin{itemize}
\item {Grp. gram.:m.}
\end{itemize}
(V.conluio)
\section{Columbário}
\begin{itemize}
\item {Grp. gram.:m.}
\end{itemize}
\begin{itemize}
\item {Utilização:Des.}
\end{itemize}
\begin{itemize}
\item {Proveniência:(Lat. \textunderscore columbarius\textunderscore )}
\end{itemize}
Cavidade subterrânea, em que alguns povos antigos collocavam as urnas funerárias.
O mesmo que \textunderscore pombal\textunderscore .
\section{Colúmbia}
\begin{itemize}
\item {Grp. gram.:f.}
\end{itemize}
Gênero de árvores tiliáceas.
Casta de uva preta americana.
\section{Columbino}
\begin{itemize}
\item {Grp. gram.:adj.}
\end{itemize}
\begin{itemize}
\item {Utilização:Fig.}
\end{itemize}
\begin{itemize}
\item {Grp. gram.:M. pl.}
\end{itemize}
\begin{itemize}
\item {Proveniência:(Lat. \textunderscore columbinus\textunderscore )}
\end{itemize}
Relativo a pomba ou pombo: \textunderscore asas columbinas\textunderscore .
Puro, innocente.
Família de aves, que tem por typo a pomba.
\section{Columbino}
\begin{itemize}
\item {Grp. gram.:adj.}
\end{itemize}
Relativo ao Estado da Colúmbia.
\section{Columela}
\begin{itemize}
\item {Grp. gram.:f.}
\end{itemize}
\begin{itemize}
\item {Proveniência:(Lat. \textunderscore columella\textunderscore )}
\end{itemize}
Pequena coluna.
Eixo vertical dos frutos.
Eixo interior das conchas.
\section{Columelado}
\begin{itemize}
\item {Grp. gram.:adj.}
\end{itemize}
Que tem columela.
\section{Columella}
\begin{itemize}
\item {Grp. gram.:f.}
\end{itemize}
\begin{itemize}
\item {Proveniência:(Lat. \textunderscore columella\textunderscore )}
\end{itemize}
Pequena columna.
Eixo vertical dos frutos.
Eixo interior das conchas.
\section{Columellado}
\begin{itemize}
\item {Grp. gram.:adj.}
\end{itemize}
Que tem columella.
\section{Columim}
\begin{itemize}
\item {Grp. gram.:m.}
\end{itemize}
\begin{itemize}
\item {Utilização:Bras. do N}
\end{itemize}
Caboclo ainda jovem.
Dente, que nasce nos animaes cavallares de cinco annos.
\section{Columna}
\begin{itemize}
\item {Grp. gram.:f.}
\end{itemize}
\begin{itemize}
\item {Utilização:Fig.}
\end{itemize}
\begin{itemize}
\item {Proveniência:(Lat. \textunderscore columna\textunderscore )}
\end{itemize}
Pilar cylíndrico, que serve de ornato, ou sustenta abóbada, estátua, etc., e consta de base, fuste e capitel.
Objecto cylíndrico, análogo a uma columna.
Esteio, sustentáculo.
Cada uma das subdivisões verticaes das páginas de um periodico e de alguns livros.
Posição vertical de uns objectos sôbre outros.
Troço de soldados em linha.
Espaço, occupado por uma porção de gás ou de líquido.
Reunião de vértebras sôbrepostas: \textunderscore a columna vertebral\textunderscore .
\section{Columnar}
\begin{itemize}
\item {Grp. gram.:adj.}
\end{itemize}
\begin{itemize}
\item {Proveniência:(Lat. \textunderscore columnaris\textunderscore )}
\end{itemize}
Relativo a columna; que tem fórma de columna.
\section{Columnário}
\begin{itemize}
\item {Grp. gram.:adj.}
\end{itemize}
\begin{itemize}
\item {Proveniência:(Lat. \textunderscore columnarius\textunderscore )}
\end{itemize}
Em que há columna ou columnas representadas.
\section{Columnata}
\begin{itemize}
\item {Grp. gram.:f.}
\end{itemize}
Série de columnas.
\section{Columnello}
\begin{itemize}
\item {Grp. gram.:m.}
\end{itemize}
\begin{itemize}
\item {Proveniência:(Lat. \textunderscore columnella\textunderscore )}
\end{itemize}
Pequena columna.
Marco.
\section{Columneta}
\begin{itemize}
\item {fónica:nê}
\end{itemize}
\begin{itemize}
\item {Grp. gram.:f.}
\end{itemize}
Pequena columna.
\section{Columníferas}
\begin{itemize}
\item {Grp. gram.:f. pl.}
\end{itemize}
\begin{itemize}
\item {Proveniência:(Do lat. \textunderscore columna\textunderscore  + \textunderscore ferre\textunderscore )}
\end{itemize}
Ordem de plantas, que abrange as malváceas e as tiliáceas.
\section{Coluna}
\begin{itemize}
\item {Grp. gram.:f.}
\end{itemize}
\begin{itemize}
\item {Utilização:Fig.}
\end{itemize}
\begin{itemize}
\item {Proveniência:(Lat. \textunderscore columna\textunderscore )}
\end{itemize}
Pilar cilíndrico, que serve de ornato, ou sustenta abóbada, estátua, etc., e consta de base, fuste e capitel.
Objecto cilíndrico, análogo a uma coluna.
Esteio, sustentáculo.
Cada uma das subdivisões verticaes das páginas de um periodico e de alguns livros.
Posição vertical de uns objectos sôbre outros.
Troço de soldados em linha.
Espaço, occupado por uma porção de gás ou de líquido.
Reunião de vértebras sôbrepostas: \textunderscore a coluna vertebral\textunderscore .
\section{Colunar}
\begin{itemize}
\item {Grp. gram.:adj.}
\end{itemize}
\begin{itemize}
\item {Proveniência:(Lat. \textunderscore columnaris\textunderscore )}
\end{itemize}
Relativo a coluna; que tem fórma de coluna.
\section{Colunário}
\begin{itemize}
\item {Grp. gram.:adj.}
\end{itemize}
\begin{itemize}
\item {Proveniência:(Lat. \textunderscore columnarius\textunderscore )}
\end{itemize}
Em que há coluna ou colunas representadas.
\section{Colunata}
\begin{itemize}
\item {Grp. gram.:f.}
\end{itemize}
Série de colunas.
\section{Colunelo}
\begin{itemize}
\item {Grp. gram.:m.}
\end{itemize}
\begin{itemize}
\item {Proveniência:(Lat. \textunderscore columnella\textunderscore )}
\end{itemize}
Pequena coluna.
Marco.
\section{Coluneta}
\begin{itemize}
\item {fónica:nê}
\end{itemize}
\begin{itemize}
\item {Grp. gram.:f.}
\end{itemize}
Pequena coluna.
\section{Coluníferas}
\begin{itemize}
\item {Grp. gram.:f. pl.}
\end{itemize}
\begin{itemize}
\item {Proveniência:(Do lat. \textunderscore columna\textunderscore  + \textunderscore ferre\textunderscore )}
\end{itemize}
Ordem de plantas, que abrange as malváceas e as tiliáceas.
\section{Coluria}
\begin{itemize}
\item {Grp. gram.:f.}
\end{itemize}
\begin{itemize}
\item {Proveniência:(Do gr. \textunderscore khole\textunderscore  + \textunderscore ouron\textunderscore )}
\end{itemize}
Passagem, pela urina, dos princípios corantes da bílis.
\section{Colurno}
\begin{itemize}
\item {Grp. gram.:m.}
\end{itemize}
\begin{itemize}
\item {Proveniência:(Lat. \textunderscore colurnus\textunderscore )}
\end{itemize}
Espécie de avelleira.
\section{Coluro}
\begin{itemize}
\item {Grp. gram.:m.}
\end{itemize}
\begin{itemize}
\item {Proveniência:(Gr. \textunderscore kolouros\textunderscore )}
\end{itemize}
Cada um dos dois grandes círculos imaginários, que cortam o Equador em quatro partes iguaes, passando um pelos pontos equinociaes e outro pelos solsticiaes.
\section{Colútea}
\begin{itemize}
\item {Grp. gram.:f.}
\end{itemize}
\begin{itemize}
\item {Proveniência:(Gr. \textunderscore kolutea\textunderscore )}
\end{itemize}
Arbusto leguminoso.
\section{Colutório}
\begin{itemize}
\item {Grp. gram.:m.}
\end{itemize}
\begin{itemize}
\item {Proveniência:(Do lat. \textunderscore collutus\textunderscore )}
\end{itemize}
Qualquer líquido medicinal, para as mucosas da bôca.
\section{Coluvião}
\begin{itemize}
\item {Grp. gram.:f.}
\end{itemize}
\begin{itemize}
\item {Proveniência:(Lat. \textunderscore colluvio\textunderscore )}
\end{itemize}
O mesmo que \textunderscore inundação\textunderscore .
\section{Colymbo}
\begin{itemize}
\item {Grp. gram.:m.}
\end{itemize}
\begin{itemize}
\item {Proveniência:(Gr. \textunderscore kolumbos\textunderscore )}
\end{itemize}
O mesmo que \textunderscore mergulhão\textunderscore , ave.
\section{Colza}
\begin{itemize}
\item {Grp. gram.:f.}
\end{itemize}
\begin{itemize}
\item {Proveniência:(Do fr. \textunderscore colza\textunderscore )}
\end{itemize}
Variedade de couve, que serve de forragem para o gado.
\section{Com}
\begin{itemize}
\item {Grp. gram.:prep.}
\end{itemize}
(indicativa de companhia, instrumento, ligação, etc.)
Apesar de:«\textunderscore esta máquina tão bem composta do mundo, com ser obra do braço omnipotente, que é o que a sustenta...\textunderscore »Vieira.
\section{Com...}
\begin{itemize}
\item {Grp. gram.:pref.}
\end{itemize}
(correspondente á prep. \textunderscore com\textunderscore , antes de palavras começadas por \textunderscore m\textunderscore , \textunderscore b\textunderscore  ou \textunderscore p\textunderscore . Nas demais, cai o \textunderscore m\textunderscore , ou transforma-se em \textunderscore n\textunderscore , ou é assimilado pela consoante que antecede. As excepções desta regra derivam principalmente de práticas muito discutiveis, como \textunderscore comtigo\textunderscore , \textunderscore comvosco\textunderscore , etc.)
\section{Coma}
\begin{itemize}
\item {Grp. gram.:f.}
\end{itemize}
\begin{itemize}
\item {Proveniência:(Lat. \textunderscore coma\textunderscore )}
\end{itemize}
Cabello abundante e crescido.
Juba.
Crinas.
Pennachos.
Franças: \textunderscore a coma do arvoredo\textunderscore .
\section{Coma}
\begin{itemize}
\item {Grp. gram.:f.}
\end{itemize}
\begin{itemize}
\item {Proveniência:(Gr. \textunderscore koma\textunderscore )}
\end{itemize}
Somnolência, que procede de certas doenças graves.
\section{Coma}
\begin{itemize}
\item {Grp. gram.:conj.}
\end{itemize}
\begin{itemize}
\item {Utilização:Ant.}
\end{itemize}
O mesmo que \textunderscore como\textunderscore ^1. Cf. G. Vicente, I, 262.--Na linguagem popular, ainda se nos deparam as locuções \textunderscore coma nós\textunderscore , \textunderscore coma êlle\textunderscore , \textunderscore coma ti\textunderscore , etc. em vez de \textunderscore como nós\textunderscore , \textunderscore como êlle\textunderscore , \textunderscore como tu\textunderscore , etc.
\section{Comado}
\begin{itemize}
\item {Grp. gram.:adj.}
\end{itemize}
\begin{itemize}
\item {Proveniência:(Lat. \textunderscore comatus\textunderscore )}
\end{itemize}
Que tem coma ou cabelleira.
\section{Comadre}
\begin{itemize}
\item {Grp. gram.:f.}
\end{itemize}
\begin{itemize}
\item {Utilização:Fam.}
\end{itemize}
\begin{itemize}
\item {Utilização:Fig.}
\end{itemize}
\begin{itemize}
\item {Proveniência:(De \textunderscore com\textunderscore  + \textunderscore madre\textunderscore )}
\end{itemize}
Madrinha de um neóphyto, em relação aos pais deste.
Mãe do neóphyto, em relação á madrinha e ao padrinho.
Parteira.
Botija ou outro vaso de água quente, para aquecimento de lençóes.
Mulher mexeriqueira.
\section{Comadresco}
\begin{itemize}
\item {Grp. gram.:adj.}
\end{itemize}
\begin{itemize}
\item {Proveniência:(De \textunderscore comadre\textunderscore )}
\end{itemize}
Que diz respeito a comadres. Cf. Filinto, I, 159; VI, 83.
\section{Com-alumno}
\begin{itemize}
\item {Grp. gram.:m.}
\end{itemize}
Aquelle que é alumno, juntamente com outros. Cf. Garrett, \textunderscore Retr. de Vênus\textunderscore , 117.
\section{Com-aluno}
\begin{itemize}
\item {Grp. gram.:m.}
\end{itemize}
Aquele que é aluno, juntamente com outros. Cf. Garrett, \textunderscore Retr. de Vênus\textunderscore , 117.
\section{Comanches}
\begin{itemize}
\item {Grp. gram.:m. pl.}
\end{itemize}
Tríbo de índios da América do Norte.
\section{Comandahiba}
\begin{itemize}
\item {Grp. gram.:f.}
\end{itemize}
Planta leguminosa do Brasil.
\section{Comandaíba}
\begin{itemize}
\item {Grp. gram.:f.}
\end{itemize}
Planta leguminosa do Brasil.
\section{Comanis}
\begin{itemize}
\item {Grp. gram.:m. pl.}
\end{itemize}
Aborígenes do Pará.
\section{Comão}
\begin{itemize}
\item {Grp. gram.:m.}
\end{itemize}
\begin{itemize}
\item {Utilização:Des.}
\end{itemize}
Comedor, comilão:«\textunderscore grande comão de vesugos\textunderscore ». G. Rezende, \textunderscore Cancion.\textunderscore 
\section{Comarca}
\begin{itemize}
\item {Grp. gram.:f.}
\end{itemize}
\begin{itemize}
\item {Utilização:Ant.}
\end{itemize}
Cada uma das subdivisões de um districto judicial.
Região.
Confins.
(Segundo uns de \textunderscore comarcar\textunderscore , de \textunderscore com\textunderscore  + \textunderscore marcar\textunderscore . Segundo outros, do mesmo thema que \textunderscore comarcho\textunderscore ; cp. êste t. No b. lat., temos \textunderscore commarca\textunderscore  e \textunderscore comarcha\textunderscore , em que a prep. \textunderscore cum\textunderscore  se junta a \textunderscore marca\textunderscore , fronteira, que é provavelmente o mesmo que o fr. \textunderscore marche\textunderscore , got. \textunderscore marka\textunderscore , ant. alt. al. \textunderscore marcha\textunderscore )
\section{Comarcão}
\begin{itemize}
\item {Grp. gram.:adj.}
\end{itemize}
Relativo a comarca.
\section{Comarcar}
\begin{itemize}
\item {Grp. gram.:v. i.}
\end{itemize}
\begin{itemize}
\item {Utilização:Des.}
\end{itemize}
\begin{itemize}
\item {Proveniência:(De \textunderscore co...\textunderscore  + \textunderscore marcar\textunderscore , ou de \textunderscore comarca\textunderscore ?)}
\end{itemize}
Confinar.
\section{Comarchia}
\begin{itemize}
\item {fónica:qui}
\end{itemize}
\begin{itemize}
\item {Grp. gram.:f.}
\end{itemize}
Cargo de comarcho.
\section{Comarcho}
\begin{itemize}
\item {fónica:co}
\end{itemize}
\begin{itemize}
\item {Grp. gram.:m.}
\end{itemize}
\begin{itemize}
\item {Proveniência:(Gr. \textunderscore komarkhos\textunderscore )}
\end{itemize}
Chefe ou governador de aldeia, entre os antigos Gregos.
\section{Comarco}
\begin{itemize}
\item {Grp. gram.:m.}
\end{itemize}
\begin{itemize}
\item {Proveniência:(Gr. \textunderscore komarkhos\textunderscore )}
\end{itemize}
Chefe ou governador de aldeia, entre os antigos Gregos.
\section{Comareiro}
\begin{itemize}
\item {Grp. gram.:m.}
\end{itemize}
\begin{itemize}
\item {Utilização:Prov.}
\end{itemize}
Muro, que sustenta terreno de socalco; arrêto; botaréu.
(Por \textunderscore comoreiro\textunderscore , de \textunderscore cômoro\textunderscore )
\section{Cômaro}
\begin{itemize}
\item {Grp. gram.:m.}
\end{itemize}
Planta rosácea do Brasil.
\section{Comarquia}
\begin{itemize}
\item {Grp. gram.:f.}
\end{itemize}
Cargo de comarcho.
\section{Comatiás}
\begin{itemize}
\item {Grp. gram.:m. pl.}
\end{itemize}
Indígenas brasileiros das margens do Javari.
\section{Comato}
\begin{itemize}
\item {Grp. gram.:adj.}
\end{itemize}
(V.comado)
\section{Comatoso}
\begin{itemize}
\item {Grp. gram.:adj.}
\end{itemize}
Relativo a coma, (somnolência): \textunderscore estado comatoso\textunderscore .
\section{Comátula}
\begin{itemize}
\item {Grp. gram.:f.}
\end{itemize}
\begin{itemize}
\item {Proveniência:(Lat. \textunderscore comatula\textunderscore )}
\end{itemize}
Gênero de animaes echinodermes, semelhantes á estrêlla-do-mar.
\section{Comazino}
\begin{itemize}
\item {Grp. gram.:m.}
\end{itemize}
\begin{itemize}
\item {Proveniência:(Do gr. \textunderscore kome\textunderscore )}
\end{itemize}
Gênero de insectos coleópteros.
\section{Comba}
\begin{itemize}
\item {Grp. gram.:f.}
\end{itemize}
\begin{itemize}
\item {Proveniência:(De \textunderscore combo\textunderscore )}
\end{itemize}
Valle (entre montanhas).
\section{Combalenga}
\begin{itemize}
\item {Grp. gram.:f.}
\end{itemize}
Espécie de abóbora, (\textunderscore cucurbita indica\textunderscore ).
\section{Combalido}
\begin{itemize}
\item {Grp. gram.:adj.}
\end{itemize}
\begin{itemize}
\item {Proveniência:(De \textunderscore combalir\textunderscore )}
\end{itemize}
Deteriorado, podre, (especialmente falando-se de fruta).
Abatido, adoentado, fraco.
\section{Combalir}
\begin{itemize}
\item {Grp. gram.:v. t.}
\end{itemize}
Abalar.
Tornar fraco, abatido.
Deteriorar, tornar podre.
\section{Combanir}
\begin{itemize}
\item {Grp. gram.:v. t.}
\end{itemize}
\begin{itemize}
\item {Utilização:Pop.}
\end{itemize}
O mesmo que \textunderscore combalir\textunderscore .
\section{Combarim}
\begin{itemize}
\item {Grp. gram.:m.}
\end{itemize}
Planta solânea, também conhecida por \textunderscore pimentinha\textunderscore .
\section{Combataria}
\begin{itemize}
\item {Grp. gram.:f.}
\end{itemize}
\begin{itemize}
\item {Utilização:Des.}
\end{itemize}
Grande numero ou successão de combates.
\section{Combate}
\begin{itemize}
\item {Grp. gram.:m.}
\end{itemize}
Acção de combater.
\section{Combatedor}
\begin{itemize}
\item {Grp. gram.:m.  e  adj.}
\end{itemize}
(V.combatente)
\section{Combatente}
\begin{itemize}
\item {Grp. gram.:adj.}
\end{itemize}
\begin{itemize}
\item {Grp. gram.:M.}
\end{itemize}
Que combate ou está prompto para combater: \textunderscore exército combatente\textunderscore .
Aquelle que combate ou póde combater.
Ave de arribação, (\textunderscore tringa pugnax\textunderscore , Lin.), que apparece nos países do norte da Europa.
\section{Combater}
\begin{itemize}
\item {Grp. gram.:v. t.}
\end{itemize}
\begin{itemize}
\item {Grp. gram.:V. i.}
\end{itemize}
\begin{itemize}
\item {Proveniência:(Do lat. \textunderscore combatuere\textunderscore )}
\end{itemize}
Bater-se com.
Contestar, discutir.
Fazer opposição a: \textunderscore combater o Govêrno\textunderscore .
Pelejar; lutar.
Tomar a defesa.
\section{Combatimento}
\begin{itemize}
\item {Grp. gram.:m.}
\end{itemize}
\begin{itemize}
\item {Utilização:Ant.}
\end{itemize}
Acto de combater.
Combate. Cf. \textunderscore Port. Mon. Hist.\textunderscore , \textunderscore Script.\textunderscore , 253.
\section{Combatível}
\begin{itemize}
\item {Grp. gram.:adj.}
\end{itemize}
\begin{itemize}
\item {Proveniência:(De \textunderscore combater\textunderscore )}
\end{itemize}
Sujeito a sêr combatido.
Discutível.
\section{Combatividade}
\begin{itemize}
\item {Grp. gram.:f.}
\end{itemize}
\begin{itemize}
\item {Proveniência:(De \textunderscore combater\textunderscore )}
\end{itemize}
Tendência, que os phrenologistas apontam no homem e nos animaes, para combater.
\section{Combativo}
\begin{itemize}
\item {Grp. gram.:adj.}
\end{itemize}
\begin{itemize}
\item {Utilização:Neol.}
\end{itemize}
Que tem tendência para combater, ou temperamento de combatente.
\section{Combato}
\begin{itemize}
\item {Grp. gram.:m.}
\end{itemize}
\begin{itemize}
\item {Utilização:Ant.}
\end{itemize}
O mesmo que \textunderscore combate\textunderscore .
\section{Combeiro}
\begin{itemize}
\item {Grp. gram.:m.}
\end{itemize}
\begin{itemize}
\item {Utilização:Prov.}
\end{itemize}
\begin{itemize}
\item {Utilização:alent.}
\end{itemize}
\begin{itemize}
\item {Proveniência:(De \textunderscore combo\textunderscore ?)}
\end{itemize}
Apparelho, para baldear água, formado por uma ou mais pás, suspensas de um ponto de apoio por uma corda.
\section{Combi}
\begin{itemize}
\item {Grp. gram.:m.}
\end{itemize}
Ave africana de rapina.
\section{Combinação}
\begin{itemize}
\item {Grp. gram.:f.}
\end{itemize}
\begin{itemize}
\item {Utilização:Chím.}
\end{itemize}
\begin{itemize}
\item {Grp. gram.:Pl.}
\end{itemize}
\begin{itemize}
\item {Utilização:Mathem.}
\end{itemize}
Acto ou effeito de combinar.
Disposição ordenada de quaesquer coisas ou objectos.
Reunião de várias coisas, em grupos de duas e duas, de três e três, etc.
Conformidade.
Contrato, acôrdo.
Fusão.
Consubstânciação.
Juncção de duas substâncias differentes, que produzem outra, inteiramente diversa daquella.
Grupos, formados por um número qualquer de objectos, a dois e dois, a três e três, etc., de modo que cada grupo diffira de cada um dos outros em menos um objecto, e que cada objecto não entre mais de uma vez em cada grupo.
\section{Combinadamente}
\begin{itemize}
\item {Grp. gram.:adv.}
\end{itemize}
\begin{itemize}
\item {Proveniência:(De \textunderscore combinar\textunderscore )}
\end{itemize}
Por combinação.
\section{Combinador}
\begin{itemize}
\item {Grp. gram.:m.  e  adj.}
\end{itemize}
O que combina.
\section{Combinar}
\begin{itemize}
\item {Grp. gram.:v. t.}
\end{itemize}
\begin{itemize}
\item {Utilização:Chím.}
\end{itemize}
\begin{itemize}
\item {Proveniência:(Lat. \textunderscore combinare\textunderscore )}
\end{itemize}
Agrupar, juntar em certa ordem.
Ajustar.
Misturar harmonicamente, consubstanciar.
Calcular.
Comparar.
Fazer que (dois corpos differentes) se unam e dêem origem a um corpo diverso.
Reunir em grupos.
\section{Combinatório}
\begin{itemize}
\item {Grp. gram.:adj.}
\end{itemize}
\begin{itemize}
\item {Proveniência:(De \textunderscore combinar\textunderscore )}
\end{itemize}
Relativo a combinações.
\section{Combinável}
\begin{itemize}
\item {Grp. gram.:adj.}
\end{itemize}
Que se póde combinar.
\section{Combió}
\begin{itemize}
\item {Grp. gram.:m.}
\end{itemize}
Árvore indiana, (\textunderscore coreya arborea\textunderscore ).
\section{Combo}
\begin{itemize}
\item {Grp. gram.:adj.}
\end{itemize}
Curvo.
(Cast. \textunderscore combo\textunderscore )
\section{Combocas}
\begin{itemize}
\item {Grp. gram.:f. pl.}
\end{itemize}
Aborígenes do Peru.
\section{Combói}
\begin{itemize}
\item {Grp. gram.:m.}
\end{itemize}
O mesmo que \textunderscore combóio\textunderscore . Cf. \textunderscore Luz e Calor\textunderscore , 65.
\section{Comboiar}
\begin{itemize}
\item {Grp. gram.:v. t.}
\end{itemize}
\begin{itemize}
\item {Proveniência:(De \textunderscore combóio\textunderscore )}
\end{itemize}
Escoltar, guiar, (combóio).
Auxiliar o transporte de.
\section{Comboieiro}
\begin{itemize}
\item {Grp. gram.:m.  e  adj.}
\end{itemize}
O que escolta ou guia combóio.
\section{Combóio}
\begin{itemize}
\item {Grp. gram.:m.}
\end{itemize}
\begin{itemize}
\item {Utilização:Bras. do N}
\end{itemize}
\begin{itemize}
\item {Proveniência:(Do fr. \textunderscore convoi\textunderscore )}
\end{itemize}
Porção de carros, que se dirigem ao mesmo destino.
Carros de munições e mantimentos, que acompanham fôrças militares.
Navio carregado, escoltado por embarcações de guerra.
Série de carruagens, engatadas e movidas sôbre carris de ferro por uma locomotiva ou por duas máquinas conjugadas.
Reunião de carregadores, que transportam mercadorias entre o sertão e as povoações, na África e América.
Grupo de animaes cavallares, que transportam carga.
\section{Combolcore}
\begin{itemize}
\item {Grp. gram.:m.}
\end{itemize}
Árvore de Moçambique.
\section{Combona}
\begin{itemize}
\item {Grp. gram.:f.}
\end{itemize}
(V. \textunderscore cambôa\textunderscore ^1)
\section{Comborça}
\begin{itemize}
\item {Grp. gram.:f.}
\end{itemize}
Concubina.
\section{Comborçaria}
\begin{itemize}
\item {Grp. gram.:f.}
\end{itemize}
\begin{itemize}
\item {Proveniência:(De \textunderscore comborça\textunderscore )}
\end{itemize}
Concubinato. Cf. Machado Assis, \textunderscore Hist. sem Data\textunderscore , 260.
\section{Comborço}
\begin{itemize}
\item {fónica:bôr}
\end{itemize}
\begin{itemize}
\item {Grp. gram.:m.}
\end{itemize}
Indivíduo amancebado, em relação a outro amante ou ao marido da mesma mulher, com quem se amancebou.
\section{Combrão}
\begin{itemize}
\item {Grp. gram.:m.}
\end{itemize}
\begin{itemize}
\item {Utilização:Prov.}
\end{itemize}
\begin{itemize}
\item {Utilização:alg.}
\end{itemize}
\begin{itemize}
\item {Proveniência:(De \textunderscore combro\textunderscore )}
\end{itemize}
Combro grande.
Pequena elevação longitudinal, que separa propriedades rústicas.
\section{Combretáceas}
\begin{itemize}
\item {Grp. gram.:f. pl.}
\end{itemize}
\begin{itemize}
\item {Proveniência:(De \textunderscore combreto\textunderscore )}
\end{itemize}
Família de plantas dicotyledóneas.
\section{Combreto}
\begin{itemize}
\item {Grp. gram.:m.}
\end{itemize}
\begin{itemize}
\item {Proveniência:(Lat. \textunderscore combretum\textunderscore )}
\end{itemize}
Gênero de plantas das regiões intertropicaes.
\section{Combro}
\begin{itemize}
\item {Grp. gram.:m.}
\end{itemize}
(Corr. de \textunderscore cômoro\textunderscore )
\section{Combua}
\begin{itemize}
\item {Grp. gram.:f.}
\end{itemize}
Pássaro tenuirostro da África.
\section{Combuca}
\begin{itemize}
\item {Grp. gram.:f.}
\end{itemize}
\begin{itemize}
\item {Utilização:Bras. do N}
\end{itemize}
Cabaça, de bôca grande.
\section{Combuco}
\begin{itemize}
\item {Grp. gram.:adj.}
\end{itemize}
\begin{itemize}
\item {Utilização:Bras. do N}
\end{itemize}
Diz-se das reses, que têm os chifres curvos para baixo.
\section{Comburente}
\begin{itemize}
\item {Grp. gram.:adj.}
\end{itemize}
\begin{itemize}
\item {Proveniência:(Lat. \textunderscore comburens\textunderscore )}
\end{itemize}
Que queima.
\section{Combustão}
\begin{itemize}
\item {Grp. gram.:f.}
\end{itemize}
\begin{itemize}
\item {Utilização:Fig.}
\end{itemize}
\begin{itemize}
\item {Proveniência:(Lat. \textunderscore combustio\textunderscore )}
\end{itemize}
Acto de queimar.
Estado de um corpo que arde, produzindo calor e luz.
Conflagração moral.
\section{Combustar}
\begin{itemize}
\item {Grp. gram.:v. t.}
\end{itemize}
\begin{itemize}
\item {Utilização:Neol.}
\end{itemize}
\begin{itemize}
\item {Proveniência:(De \textunderscore combustão\textunderscore )}
\end{itemize}
Queimar; abrasar.
\section{Combustibilidade}
\begin{itemize}
\item {Grp. gram.:f.}
\end{itemize}
Qualidade do que é combustível.
\section{Combustível}
\begin{itemize}
\item {Grp. gram.:adj.}
\end{itemize}
\begin{itemize}
\item {Grp. gram.:M.}
\end{itemize}
\begin{itemize}
\item {Proveniência:(De \textunderscore combusto\textunderscore )}
\end{itemize}
Que tem a propriedade de se consumir pelo fogo.
Lenha ou qualquer substancia, com que se faz lume.
\section{Combustivo}
\begin{itemize}
\item {Grp. gram.:adj.}
\end{itemize}
(V.combustível)
\section{Combusto}
\begin{itemize}
\item {Grp. gram.:adj.}
\end{itemize}
\begin{itemize}
\item {Utilização:Des.}
\end{itemize}
\begin{itemize}
\item {Proveniência:(Lat. \textunderscore combustus\textunderscore )}
\end{itemize}
Queimado.
\section{Combustor}
\begin{itemize}
\item {Grp. gram.:m.}
\end{itemize}
\begin{itemize}
\item {Utilização:bras}
\end{itemize}
\begin{itemize}
\item {Utilização:Neol.}
\end{itemize}
\begin{itemize}
\item {Proveniência:(De \textunderscore combusto\textunderscore )}
\end{itemize}
Poste, para illuminação pública.
\section{Come}
\begin{itemize}
\item {Grp. gram.:conj.}
\end{itemize}
\begin{itemize}
\item {Utilização:Ant.}
\end{itemize}
O mesmo que \textunderscore como\textunderscore ^1. Cf. \textunderscore Port. Mon. Hist. Script.\textunderscore , 260.
\section{Começador}
\begin{itemize}
\item {Grp. gram.:m.  e  adj.}
\end{itemize}
O que começa.
\section{Começante}
\begin{itemize}
\item {Grp. gram.:adj.}
\end{itemize}
Que começa, que está em princípio:«\textunderscore bronchite começante\textunderscore ». \textunderscore Jornal do Commércio\textunderscore , do Rio, de 11-V-903.
\section{Começar}
\begin{itemize}
\item {Grp. gram.:v. t.}
\end{itemize}
\begin{itemize}
\item {Grp. gram.:V. i.}
\end{itemize}
\begin{itemize}
\item {Proveniência:(Do lat. \textunderscore cum\textunderscore  + \textunderscore initiare\textunderscore )}
\end{itemize}
Dar comêço a.
Têr comêço.
\section{Comecilho}
\begin{itemize}
\item {Grp. gram.:m.}
\end{itemize}
\begin{itemize}
\item {Utilização:Pop.}
\end{itemize}
\begin{itemize}
\item {Proveniência:(De \textunderscore comêço\textunderscore )}
\end{itemize}
Aquillo ou aquelle que está em comêço.
Principiante: \textunderscore um comecilho de homem\textunderscore .
\section{Comêço}
\begin{itemize}
\item {Grp. gram.:m.}
\end{itemize}
\begin{itemize}
\item {Proveniência:(De \textunderscore começar\textunderscore )}
\end{itemize}
Primeira parte de uma acção, de uma época ou de coisa que tem extensão.
\section{Comedeiro}
\begin{itemize}
\item {Grp. gram.:adj.}
\end{itemize}
\begin{itemize}
\item {Utilização:Prov.}
\end{itemize}
\begin{itemize}
\item {Utilização:minh.}
\end{itemize}
O mesmo que \textunderscore comedor\textunderscore .
Interesseiro.
\section{Comedela}
\begin{itemize}
\item {Grp. gram.:f.}
\end{itemize}
\begin{itemize}
\item {Utilização:Pop.}
\end{itemize}
\begin{itemize}
\item {Proveniência:(De \textunderscore comer\textunderscore )}
\end{itemize}
Lôgro, fraude.
\section{Comedente}
\begin{itemize}
\item {Grp. gram.:m.  e  adj.}
\end{itemize}
\begin{itemize}
\item {Proveniência:(Lat. \textunderscore comedens\textunderscore )}
\end{itemize}
O que come.
\section{Comédia}
\begin{itemize}
\item {Grp. gram.:f.}
\end{itemize}
\begin{itemize}
\item {Utilização:Fig.}
\end{itemize}
\begin{itemize}
\item {Utilização:Pop.}
\end{itemize}
\begin{itemize}
\item {Proveniência:(Lat. \textunderscore comoedia\textunderscore )}
\end{itemize}
Peça de theatro, em que predomina a sátira ou a graça.
Facto ridículo.
Dissimulação.
Theatro.
\section{Comediador}
\begin{itemize}
\item {Grp. gram.:m.}
\end{itemize}
\begin{itemize}
\item {Proveniência:(De \textunderscore co...\textunderscore  + \textunderscore mediador\textunderscore )}
\end{itemize}
Aquelle que com outrem é mediador num negócio.
\section{Comedianta}
\begin{itemize}
\item {Grp. gram.:f.}
\end{itemize}
(fem. de \textunderscore comediante\textunderscore )
\section{Comediante}
\begin{itemize}
\item {Grp. gram.:m.}
\end{itemize}
Actor de comédias.
Actor.
\section{Comediar}
\begin{itemize}
\item {Grp. gram.:v. t.}
\end{itemize}
\begin{itemize}
\item {Proveniência:(De \textunderscore comédia\textunderscore )}
\end{itemize}
Tornar cómico; converter em comédia. Cf. Filinto, V, 115.
\section{Comedías}
\begin{itemize}
\item {Grp. gram.:f. pl.}
\end{itemize}
\begin{itemize}
\item {Proveniência:(Do lat. \textunderscore comedere\textunderscore )}
\end{itemize}
Pensão vitalícia, que os soberanos davam aos militares beneméritos.
Comedorias.
\section{Comediógrafo}
\begin{itemize}
\item {Grp. gram.:m.}
\end{itemize}
\begin{itemize}
\item {Proveniência:(Do gr. \textunderscore komoidia\textunderscore  + \textunderscore graphein\textunderscore )}
\end{itemize}
Aquelle que faz comédias.
\section{Comediógrapho}
\begin{itemize}
\item {Grp. gram.:m.}
\end{itemize}
\begin{itemize}
\item {Proveniência:(Do gr. \textunderscore komoidia\textunderscore  + \textunderscore graphein\textunderscore )}
\end{itemize}
Aquelle que faz comédias.
\section{Comedoiro}
\begin{itemize}
\item {Grp. gram.:m.}
\end{itemize}
\begin{itemize}
\item {Grp. gram.:Adj.}
\end{itemize}
\begin{itemize}
\item {Proveniência:(De \textunderscore comer\textunderscore )}
\end{itemize}
Lugar ou vaso, em que comem animaes.
Que é bom para sêr comido.
\section{Comedor}
\begin{itemize}
\item {Grp. gram.:m.  e  adj.}
\end{itemize}
\begin{itemize}
\item {Proveniência:(De \textunderscore comer\textunderscore )}
\end{itemize}
O que come.
O que come muito; comilão.
Perdulário.
Parasito.
\section{Comedorias}
\begin{itemize}
\item {Grp. gram.:f. pl.}
\end{itemize}
\begin{itemize}
\item {Utilização:Marn.}
\end{itemize}
\begin{itemize}
\item {Proveniência:(De \textunderscore comedor\textunderscore )}
\end{itemize}
Sustento.
Pensão para alimentos.
Pensão, que antigamente se pagava aos fundadores de conventos ou a seus descendentes.
Ração diária de militares a bordo.
Grupo de viveiros e algibés, nas salinas.
\section{Comedouro}
\begin{itemize}
\item {Grp. gram.:m.}
\end{itemize}
\begin{itemize}
\item {Grp. gram.:Adj.}
\end{itemize}
\begin{itemize}
\item {Proveniência:(De \textunderscore comer\textunderscore )}
\end{itemize}
Lugar ou vaso, em que comem animaes.
Que é bom para sêr comido.
\section{Comedura}
\begin{itemize}
\item {Grp. gram.:f.}
\end{itemize}
O mesmo que \textunderscore comedorias\textunderscore .
\section{Come-e-cala}
\begin{itemize}
\item {Grp. gram.:f.}
\end{itemize}
Variedade de pêra de boa qualidade.
\section{Come-em-vão}
\begin{itemize}
\item {Grp. gram.:m.}
\end{itemize}
\begin{itemize}
\item {Utilização:fam.}
\end{itemize}
\begin{itemize}
\item {Utilização:Ant.}
\end{itemize}
Avarento, homem sovina.
\section{Come-gente}
\begin{itemize}
\item {Grp. gram.:m.}
\end{itemize}
Pequeno rebote, para desbastar madeira.
\section{Comeios}
\begin{itemize}
\item {Grp. gram.:m.}
\end{itemize}
O mesmo que \textunderscore comenos\textunderscore .
\section{Comenos}
\begin{itemize}
\item {Grp. gram.:m.}
\end{itemize}
\begin{itemize}
\item {Proveniência:(De \textunderscore com\textunderscore  + \textunderscore menos\textunderscore ?)}
\end{itemize}
Instante, momento, occasião: \textunderscore naquelle comenos, appareceu o homem\textunderscore .
\section{Comer}
\begin{itemize}
\item {Grp. gram.:v. t.}
\end{itemize}
\begin{itemize}
\item {Utilização:Fig.}
\end{itemize}
\begin{itemize}
\item {Utilização:Pop.}
\end{itemize}
\begin{itemize}
\item {Utilização:Burl.}
\end{itemize}
\begin{itemize}
\item {Grp. gram.:V. i.}
\end{itemize}
\begin{itemize}
\item {Grp. gram.:V. p.}
\end{itemize}
\begin{itemize}
\item {Grp. gram.:M.}
\end{itemize}
\begin{itemize}
\item {Proveniência:(Do lat. \textunderscore comedere\textunderscore )}
\end{itemize}
Introduzir (alimento) no estomago pela bôca.
Mastigar e engolir.
Gastar em comida.
Dissipar.
Defraudar.
Attenuar.
Consumir.
Acreditar facilmente, ingenuamente: \textunderscore essa não como eu\textunderscore .
Lograr, enganar: \textunderscore não me comes\textunderscore .
Tomar alimento.
Têr comichão.
Causar comichão.
Tirar proveito.
Roubar.
Amofinar-se.
Comida.
\section{Comes}
\begin{itemize}
\item {Grp. gram.:m. pl.}
\end{itemize}
Us. na loc. \textunderscore comes e bebes\textunderscore , aquillo que se come e que se bebe.
\section{Comestível}
\begin{itemize}
\item {Grp. gram.:adj.}
\end{itemize}
\begin{itemize}
\item {Grp. gram.:M. pl.}
\end{itemize}
\begin{itemize}
\item {Proveniência:(De \textunderscore comesto\textunderscore )}
\end{itemize}
Próprio para sêr comido.
Aquillo que é próprio para alimento.
Gêneros alimentícios.
\section{Comestivo}
\begin{itemize}
\item {Grp. gram.:m.}
\end{itemize}
\begin{itemize}
\item {Proveniência:(De \textunderscore comesto\textunderscore )}
\end{itemize}
Aquillo que serve ou é próprio para se comer; comida:«\textunderscore contentão-se com mui tenue comestivo\textunderscore ». Filinto, \textunderscore D. Man.\textunderscore , I, 43; II, 100.
\section{Comesto}
\begin{itemize}
\item {Grp. gram.:adj.}
\end{itemize}
\begin{itemize}
\item {Utilização:Ant.}
\end{itemize}
\begin{itemize}
\item {Proveniência:(Lat. \textunderscore comestus\textunderscore )}
\end{itemize}
Comido.
\section{Cometa}
\begin{itemize}
\item {fónica:mê}
\end{itemize}
\begin{itemize}
\item {Grp. gram.:m.}
\end{itemize}
\begin{itemize}
\item {Utilização:Bras}
\end{itemize}
\begin{itemize}
\item {Proveniência:(Lat. \textunderscore cometa\textunderscore )}
\end{itemize}
Astro de cauda luminosa, que descreve órbitas muito alongadas á volta do sol.
Cobrador viajante.
\section{Cometa}
\begin{itemize}
\item {fónica:mê}
\end{itemize}
\begin{itemize}
\item {Grp. gram.:m.}
\end{itemize}
\begin{itemize}
\item {Utilização:Chul.}
\end{itemize}
\begin{itemize}
\item {Proveniência:(De \textunderscore comer\textunderscore )}
\end{itemize}
Comilão.
\section{Cometar}
\begin{itemize}
\item {Grp. gram.:adj.}
\end{itemize}
O mesmo que \textunderscore cometário\textunderscore .
\section{Cometário}
\begin{itemize}
\item {Grp. gram.:adj.}
\end{itemize}
Relativo a cometa^1.
\section{Cometodos}
\begin{itemize}
\item {fónica:có}
\end{itemize}
\begin{itemize}
\item {Grp. gram.:m.}
\end{itemize}
\begin{itemize}
\item {Utilização:Prov.}
\end{itemize}
O maior dos cinco dedos da mão.
(Colhido na Bairrada)
\section{Cometografia}
\begin{itemize}
\item {Grp. gram.:f.}
\end{itemize}
Descripção dos cometas.
\section{Cometógrafo}
\begin{itemize}
\item {Grp. gram.:m.}
\end{itemize}
\begin{itemize}
\item {Proveniência:(Do gr. \textunderscore kometes\textunderscore  + \textunderscore graphein\textunderscore )}
\end{itemize}
Aquelle que se dedica á cometografia.
\section{Cometographia}
\begin{itemize}
\item {Grp. gram.:f.}
\end{itemize}
Descripção dos cometas.
\section{Cometógrapho}
\begin{itemize}
\item {Grp. gram.:m.}
\end{itemize}
\begin{itemize}
\item {Proveniência:(Do gr. \textunderscore kometes\textunderscore  + \textunderscore graphein\textunderscore )}
\end{itemize}
Aquelle que se dedica á cometographia.
\section{Cometologia}
\begin{itemize}
\item {Grp. gram.:f.}
\end{itemize}
\begin{itemize}
\item {Proveniência:(Do gr. \textunderscore kometes\textunderscore  + \textunderscore logos\textunderscore )}
\end{itemize}
Tratado dos cometas.
\section{Cometomancia}
\begin{itemize}
\item {Grp. gram.:f.}
\end{itemize}
\begin{itemize}
\item {Proveniência:(Do gr. \textunderscore kometes\textunderscore  + \textunderscore manteia\textunderscore )}
\end{itemize}
Supposta arte de adivinhar, pela observação dos cometas.
\section{Coma}
\begin{itemize}
\item {Grp. gram.:f.}
\end{itemize}
\begin{itemize}
\item {Utilização:Des.}
\end{itemize}
\begin{itemize}
\item {Grp. gram.:Pl.}
\end{itemize}
\begin{itemize}
\item {Utilização:Mús.}
\end{itemize}
\begin{itemize}
\item {Proveniência:(Gr. \textunderscore komma\textunderscore )}
\end{itemize}
Vírgula.
Sinal ortográfico, composto de duas vírgulas naturaes ou invertidas, o mesmo que \textunderscore aspas\textunderscore .
Distância entre o semitom maior e o menor.
\section{Comandamento}
\begin{itemize}
\item {Grp. gram.:m.}
\end{itemize}
\begin{itemize}
\item {Utilização:Des.}
\end{itemize}
(V.comando)
\section{Comandância}
\begin{itemize}
\item {Grp. gram.:f.}
\end{itemize}
Cargo de comandante. Cf. Camillo, \textunderscore Perfil\textunderscore , 237 e 302.
\section{Comandanta}
\begin{itemize}
\item {Grp. gram.:f.}
\end{itemize}
\begin{itemize}
\item {Utilização:Fam.}
\end{itemize}
Mulher do comandante.
\section{Comandante}
\begin{itemize}
\item {Grp. gram.:adj.}
\end{itemize}
\begin{itemize}
\item {Grp. gram.:M.}
\end{itemize}
\begin{itemize}
\item {Proveniência:(De \textunderscore comandar\textunderscore )}
\end{itemize}
Que comanda.
Aquele que tem um comando militar.
\section{Comandar}
\begin{itemize}
\item {Grp. gram.:v. t.}
\end{itemize}
\begin{itemize}
\item {Proveniência:(De \textunderscore com\textunderscore  + \textunderscore mandar\textunderscore )}
\end{itemize}
Dirigir, como superior; mandar.
\section{Comandita}
\begin{itemize}
\item {Grp. gram.:f.}
\end{itemize}
\begin{itemize}
\item {Proveniência:(Fr. \textunderscore commandite\textunderscore )}
\end{itemize}
Fórma de sociedade comercial, em que há um ou mais associados, de responsabilidade solidária, e um ou mais sócios capitalistas, de responsabilidade que não excede o capital subscripto.
\section{Comanditar}
\begin{itemize}
\item {Grp. gram.:v. t.}
\end{itemize}
\begin{itemize}
\item {Proveniência:(De \textunderscore commandita\textunderscore )}
\end{itemize}
Encarregar da administração dos fundos, numa sociedade de comandita.
\section{Comanditário}
\begin{itemize}
\item {Grp. gram.:m.}
\end{itemize}
\begin{itemize}
\item {Proveniência:(De \textunderscore comandita\textunderscore )}
\end{itemize}
Sócio capitalista ou fornecedor de fundos, em uma sociedade de comandita.
\section{Comando}
\begin{itemize}
\item {Grp. gram.:m.}
\end{itemize}
Acção de comandar.
Govêrno de uma divisão de tropas.
\section{Comedidamente}
\begin{itemize}
\item {Grp. gram.:adv.}
\end{itemize}
De modo comedido.
Com comedimento.
\section{Comedido}
\begin{itemize}
\item {Grp. gram.:adj.}
\end{itemize}
\begin{itemize}
\item {Proveniência:(De \textunderscore comedir\textunderscore )}
\end{itemize}
Moderado.
Respeitoso.
\section{Comedimento}
\begin{itemize}
\item {Grp. gram.:m.}
\end{itemize}
\begin{itemize}
\item {Proveniência:(De \textunderscore comedir\textunderscore )}
\end{itemize}
Carácter daquele ou daquilo que é comedido.
\section{Comedir}
\begin{itemize}
\item {Grp. gram.:v. t.}
\end{itemize}
\begin{itemize}
\item {Proveniência:(De \textunderscore com...\textunderscore  + \textunderscore medir\textunderscore )}
\end{itemize}
Adequar.
Regular convenientemente.
Sujeitar ao dever.
Tornar respeitoso.
Moderar.
\section{Comego}
\begin{itemize}
\item {fónica:mê}
\end{itemize}
\begin{itemize}
\item {Grp. gram.:loc. pron.}
\end{itemize}
\begin{itemize}
\item {Utilização:Ant.}
\end{itemize}
O mesmo que \textunderscore comigo\textunderscore . Cf. Sim. Machado, f. 78, v., cit. por Castilho.
\section{Comelina}
\begin{itemize}
\item {Grp. gram.:f.}
\end{itemize}
\begin{itemize}
\item {Proveniência:(De \textunderscore Commelin\textunderscore , n. p.)}
\end{itemize}
Gênero de plantas, a que pertence a tradescância.
\section{Comelináceas}
\begin{itemize}
\item {Grp. gram.:f. pl.}
\end{itemize}
O mesmo ou melhor que \textunderscore comelíneas\textunderscore .
\section{Comelíneas}
\begin{itemize}
\item {Grp. gram.:f. pl.}
\end{itemize}
\begin{itemize}
\item {Proveniência:(De \textunderscore commelina\textunderscore )}
\end{itemize}
Família de plantas herbáceas, a que pertence a tradescância.
\section{Comemoração}
\begin{itemize}
\item {Grp. gram.:f.}
\end{itemize}
\begin{itemize}
\item {Proveniência:(Lat. \textunderscore commemoratio\textunderscore )}
\end{itemize}
Acto de comemorar.
\section{Comemorar}
\begin{itemize}
\item {Grp. gram.:v. t.}
\end{itemize}
\begin{itemize}
\item {Proveniência:(Lat. \textunderscore commemorare\textunderscore )}
\end{itemize}
Lembrar, trazer á memória.
Solenizar, recordando: \textunderscore comemorar uma victória\textunderscore .
\section{Comemorativo}
\begin{itemize}
\item {Grp. gram.:adj.}
\end{itemize}
\begin{itemize}
\item {Proveniência:(Do lat. \textunderscore commemoratus\textunderscore )}
\end{itemize}
Que comemora.
\section{Comemorável}
\begin{itemize}
\item {Grp. gram.:adj.}
\end{itemize}
\begin{itemize}
\item {Proveniência:(Lat. \textunderscore commemorabilis\textunderscore )}
\end{itemize}
Que merece sêr comemorado.
\section{Comenda}
\begin{itemize}
\item {Grp. gram.:f.}
\end{itemize}
Antigo benefício, que se dava a eclesiásticos ou a cavaleiros de Ordens militares.
Porção de terras, com que oficialmente se recompensavam serviços.
Distinção puramente honorifica, e correspondente a um grau de Ordem militar.
Insígnia de comendador.
(B. lat. \textunderscore commenda\textunderscore )
\section{Comendadeira}
(fem. de \textunderscore comendador\textunderscore )
\section{Comendador}
\begin{itemize}
\item {Grp. gram.:m.}
\end{itemize}
Aquele que tem comenda.
\section{Comendadoria}
\begin{itemize}
\item {Grp. gram.:f.}
\end{itemize}
\begin{itemize}
\item {Proveniência:(De \textunderscore comendador\textunderscore )}
\end{itemize}
Dignidade de comendador.
Benefício de comenda.
\section{Comendamento}
\begin{itemize}
\item {Grp. gram.:m.}
\end{itemize}
\begin{itemize}
\item {Utilização:Ant.}
\end{itemize}
\begin{itemize}
\item {Proveniência:(De \textunderscore comendar\textunderscore )}
\end{itemize}
O mesmo que \textunderscore recomendação\textunderscore . Cf. \textunderscore Port. Mon. Hist.\textunderscore , \textunderscore Script.\textunderscore , 286.
\section{Comendar}
\begin{itemize}
\item {Grp. gram.:v. t.}
\end{itemize}
(V.encomendar)
\section{Comendataria}
\begin{itemize}
\item {Grp. gram.:f.}
\end{itemize}
(V.comendadoria)
\section{Comendatário}
\begin{itemize}
\item {Grp. gram.:adj.}
\end{itemize}
Que frui benefício de comenda.
(B. lat. \textunderscore commendatarius\textunderscore )
\section{Comendatício}
\begin{itemize}
\item {Grp. gram.:adj.}
\end{itemize}
\begin{itemize}
\item {Proveniência:(Lat. \textunderscore commendaticius\textunderscore )}
\end{itemize}
Que se recomenda.
\section{Comendativo}
\begin{itemize}
\item {Grp. gram.:adj.}
\end{itemize}
\begin{itemize}
\item {Proveniência:(Do lat. \textunderscore commendativus\textunderscore )}
\end{itemize}
Que recomenda, que é próprio para recomendar.
\section{Comendatório}
\begin{itemize}
\item {Grp. gram.:adj.}
\end{itemize}
(V.comendatício)
\section{Comendavelmente}
\begin{itemize}
\item {Grp. gram.:adv.}
\end{itemize}
\begin{itemize}
\item {Utilização:Ant.}
\end{itemize}
De modo recomendável.
\section{Comendela}
\begin{itemize}
\item {Grp. gram.:f.}
\end{itemize}
\begin{itemize}
\item {Utilização:Ant.}
\end{itemize}
Pequena comenda.
\section{Comensal}
\begin{itemize}
\item {Grp. gram.:m.}
\end{itemize}
\begin{itemize}
\item {Proveniência:(Do lat. \textunderscore cum\textunderscore  + \textunderscore mensa\textunderscore )}
\end{itemize}
Cada um dos que comem juntos.
Aquele que come habitualmente em casa alheia; parasito.
\section{Comensalidade}
\begin{itemize}
\item {Grp. gram.:f.}
\end{itemize}
Qualidade de quem é comensal.
\section{Comensurabilidade}
\begin{itemize}
\item {Grp. gram.:f.}
\end{itemize}
Qualidade daquilo que é comensurável.
\section{Comensuração}
\begin{itemize}
\item {Grp. gram.:f.}
\end{itemize}
Acto de comensurar.
\section{Comensurar}
\begin{itemize}
\item {Grp. gram.:v. t.}
\end{itemize}
\begin{itemize}
\item {Proveniência:(Do lat. \textunderscore cum\textunderscore  + \textunderscore mensurare\textunderscore )}
\end{itemize}
Medir (duas ou mais quantidades) com a mesma unidade.
Proporcionar.
\section{Comensurável}
\begin{itemize}
\item {Grp. gram.:adj.}
\end{itemize}
\begin{itemize}
\item {Proveniência:(De \textunderscore commensurar\textunderscore )}
\end{itemize}
Que tem medida comum.
Que póde medir-se.
\section{Comentação}
\begin{itemize}
\item {Grp. gram.:f.}
\end{itemize}
Acto ou efeito de comentar; comentário.
\section{Comentador}
\begin{itemize}
\item {Grp. gram.:m.}
\end{itemize}
Aquele que comenta.
\section{Comentar}
\begin{itemize}
\item {Grp. gram.:v. t.}
\end{itemize}
\begin{itemize}
\item {Proveniência:(Lat. \textunderscore commentare\textunderscore )}
\end{itemize}
Explicar, interpretando ou anotando: \textunderscore comentar os«Lusiadas»\textunderscore .
Fazer comentário a: \textunderscore comentar uma notícia\textunderscore .
Criticar.
\section{Comentário}
\begin{itemize}
\item {Grp. gram.:m.}
\end{itemize}
\begin{itemize}
\item {Proveniência:(Lat. \textunderscore commentarius\textunderscore )}
\end{itemize}
Série de notas, com que se esclarece ou critíca uma producção literária ou científica.
Ponderações sôbre um facto.
Crítica maliciosa.
\section{Comentício}
\begin{itemize}
\item {Grp. gram.:adj.}
\end{itemize}
\begin{itemize}
\item {Utilização:Des.}
\end{itemize}
\begin{itemize}
\item {Proveniência:(Lat. \textunderscore commenticius\textunderscore )}
\end{itemize}
Fabuloso.
\section{Comentista}
\begin{itemize}
\item {Grp. gram.:m.}
\end{itemize}
\begin{itemize}
\item {Utilização:Des.}
\end{itemize}
\begin{itemize}
\item {Proveniência:(De \textunderscore comento\textunderscore )}
\end{itemize}
O mesmo que \textunderscore comentador\textunderscore .
\section{Comercial}
\begin{itemize}
\item {Grp. gram.:adj.}
\end{itemize}
Relativo ao comércio: \textunderscore associação comercial\textunderscore .
\section{Comercialista}
\begin{itemize}
\item {Grp. gram.:m.}
\end{itemize}
Indivíduo, versado em Direito Comercial.
\section{Comercialmente}
\begin{itemize}
\item {Grp. gram.:adv.}
\end{itemize}
De modo comercial.
\section{Comerciante}
\begin{itemize}
\item {Grp. gram.:m.  e  adj.}
\end{itemize}
\begin{itemize}
\item {Proveniência:(De \textunderscore comerciar\textunderscore )}
\end{itemize}
O que exerce comércio.
\section{Comerciar}
\begin{itemize}
\item {Grp. gram.:v. i.}
\end{itemize}
Exercer comércio, têr comércio.
\section{Comerciável}
\begin{itemize}
\item {Grp. gram.:adj.}
\end{itemize}
\begin{itemize}
\item {Proveniência:(De \textunderscore comerciar\textunderscore )}
\end{itemize}
Que póde sêr objecto de comércio.
\section{Comércio}
\begin{itemize}
\item {Grp. gram.:m.}
\end{itemize}
\begin{itemize}
\item {Proveniência:(Lat. \textunderscore commercium\textunderscore )}
\end{itemize}
Permutação de productos naturaes ou artificiáes.
Troca de valores.
A classe dos que comerciam: \textunderscore o comércio está satisfeito\textunderscore .
Relações, convivência, trato.
\section{Comersão}
\begin{itemize}
\item {Grp. gram.:m.}
\end{itemize}
Peixe negro das profundidades do Pacífico.
\section{Comêssea}
\begin{itemize}
\item {Grp. gram.:f.}
\end{itemize}
\begin{itemize}
\item {Utilização:Ant.}
\end{itemize}
O mesmo que \textunderscore comenda\textunderscore .
\section{Cometácula}
\begin{itemize}
\item {Grp. gram.:f.}
\end{itemize}
\begin{itemize}
\item {Proveniência:(Lat. \textunderscore commetacula\textunderscore )}
\end{itemize}
Varinha, que os flâmines usavam em Roma, quando iam para o sacrifício, e com a qual desviavam a multidão, para passar.
\section{Cometedor}
\begin{itemize}
\item {Grp. gram.:m.  e  adj.}
\end{itemize}
O que comete: \textunderscore cometedor de um crime\textunderscore .
\section{Cometente}
\begin{itemize}
\item {Grp. gram.:m.  e  f.}
\end{itemize}
(V.comitente)
\section{Cometer}
\begin{itemize}
\item {Grp. gram.:v. t.}
\end{itemize}
\begin{itemize}
\item {Proveniência:(Lat. \textunderscore committere\textunderscore )}
\end{itemize}
Fazer; praticar; perpetrar: \textunderscore cometer um êrro\textunderscore .
Confiar, encarregar: \textunderscore cometi-lhe aquela missão\textunderscore .
Propor.
Atacar: \textunderscore cometer a fortaleza\textunderscore .
Empreender.
\section{Cometida}
\begin{itemize}
\item {Grp. gram.:f.}
\end{itemize}
\begin{itemize}
\item {Proveniência:(De \textunderscore cometer\textunderscore )}
\end{itemize}
Ataque, investida.
\section{Cometimento}
\begin{itemize}
\item {Grp. gram.:m.}
\end{itemize}
Acto de cometer.
\section{Cometomante}
\begin{itemize}
\item {Grp. gram.:m.}
\end{itemize}
Aquelle que se dedica á cometomancia.
\section{Comezaina}
\begin{itemize}
\item {Grp. gram.:f.}
\end{itemize}
\begin{itemize}
\item {Utilização:Pop.}
\end{itemize}
\begin{itemize}
\item {Proveniência:(Do rad. de \textunderscore comer\textunderscore )}
\end{itemize}
Refeição abundante.
Patuscada.
\section{Comezana}
\begin{itemize}
\item {Grp. gram.:f.}
\end{itemize}
\begin{itemize}
\item {Utilização:Pop.}
\end{itemize}
\begin{itemize}
\item {Proveniência:(Do rad. de \textunderscore comer\textunderscore )}
\end{itemize}
Refeição abundante.
Patuscada.
\section{Comezinho}
\begin{itemize}
\item {Grp. gram.:adj.}
\end{itemize}
\begin{itemize}
\item {Utilização:Fig.}
\end{itemize}
\begin{itemize}
\item {Proveniência:(Do rad. de \textunderscore comer\textunderscore )}
\end{itemize}
Bom para se comer.
Fácil de entender: \textunderscore estilo comezinho\textunderscore .
Caseiro; simples: \textunderscore remédio comezinho\textunderscore .
\section{Cómica}
\begin{itemize}
\item {Grp. gram.:f.}
\end{itemize}
O mesmo que \textunderscore comedianta\textunderscore .
\section{Comicamente}
\begin{itemize}
\item {Grp. gram.:adv.}
\end{itemize}
De modo cómico.
\section{Comicha}
\begin{itemize}
\item {Grp. gram.:adj.}
\end{itemize}
\begin{itemize}
\item {Utilização:Prov.}
\end{itemize}
Importuno, maçador. (Colhido no Fundão)
(Cp. \textunderscore comichão\textunderscore )
\section{Comichão}
\begin{itemize}
\item {Grp. gram.:f.}
\end{itemize}
\begin{itemize}
\item {Utilização:Fig.}
\end{itemize}
\begin{itemize}
\item {Proveniência:(Do rad. de \textunderscore comer\textunderscore )}
\end{itemize}
Prurido.
Desejo ardente.
\section{Comichar}
\begin{itemize}
\item {Grp. gram.:v. t.}
\end{itemize}
\begin{itemize}
\item {Utilização:Des.}
\end{itemize}
\begin{itemize}
\item {Proveniência:(Do rad. de \textunderscore comer\textunderscore , ou do lat. \textunderscore comestio\textunderscore )}
\end{itemize}
Causar comichão a.
\section{Comichona}
\begin{itemize}
\item {Grp. gram.:adj. f.}
\end{itemize}
\begin{itemize}
\item {Proveniência:(De \textunderscore comichão\textunderscore )}
\end{itemize}
Que causa prurido ou comichão, (falando-se da tinha ou de outras doenças cutâneas).
\section{Comichoso}
\begin{itemize}
\item {Grp. gram.:adj.}
\end{itemize}
Sujeito a comichão.
Que tem comichão.
O mesmo que \textunderscore comicha\textunderscore .
\section{Comicial}
\begin{itemize}
\item {Grp. gram.:adj.}
\end{itemize}
\begin{itemize}
\item {Proveniência:(Lat. \textunderscore comitialis\textunderscore )}
\end{itemize}
Relativo a comicios.
\section{Comício}
\begin{itemize}
\item {Grp. gram.:m.}
\end{itemize}
\begin{itemize}
\item {Proveniência:(Lat. \textunderscore comitium\textunderscore )}
\end{itemize}
Reunião de cidadãos, para tratar assumptos de interesse público.
Assembleia popular, entre os Romanos.
\section{Cómico}
\begin{itemize}
\item {Grp. gram.:adj.}
\end{itemize}
\begin{itemize}
\item {Grp. gram.:M.}
\end{itemize}
\begin{itemize}
\item {Proveniência:(Lat. \textunderscore comicus\textunderscore )}
\end{itemize}
Relativo a comédia.
Que faz rir; ridículo.
Actor de comédias; actor.
\section{Comida}
\begin{itemize}
\item {Grp. gram.:f.}
\end{itemize}
\begin{itemize}
\item {Proveniência:(De \textunderscore comido\textunderscore )}
\end{itemize}
Aquillo que serve para se comer.
Aquillo que se come.
Acção de comer.
\section{Comidade}
\begin{itemize}
\item {Grp. gram.:f.}
\end{itemize}
\begin{itemize}
\item {Utilização:P. us.}
\end{itemize}
\begin{itemize}
\item {Proveniência:(Lat. \textunderscore comitas\textunderscore )}
\end{itemize}
Lhaneza; urbanidade.
\section{Comido}
\begin{itemize}
\item {Grp. gram.:adj.}
\end{itemize}
\begin{itemize}
\item {Utilização:Burl.}
\end{itemize}
\begin{itemize}
\item {Proveniência:(De \textunderscore comer\textunderscore )}
\end{itemize}
Mastigado e ingerido no estômago.
Enganado, logrado: \textunderscore fiquei comido\textunderscore .
\section{Comífora}
\begin{itemize}
\item {Grp. gram.:f.}
\end{itemize}
Árvore amarilídea, (\textunderscore commiphora madagascarensis\textunderscore ).
\section{Comigo}
\begin{itemize}
\item {Grp. gram.:loc. pron.}
\end{itemize}
\begin{itemize}
\item {Proveniência:(De \textunderscore com\textunderscore  + \textunderscore migo\textunderscore , flexão do pron. \textunderscore eu\textunderscore )}
\end{itemize}
Em companhia de mim: \textunderscore fugiu comigo\textunderscore .
De mim para mim: \textunderscore o que eu dizia comigo\textunderscore .
A respeito de mim: \textunderscore aquilo não era comigo\textunderscore .
\section{Comilagem}
\begin{itemize}
\item {Grp. gram.:f.}
\end{itemize}
\begin{itemize}
\item {Utilização:Fam.}
\end{itemize}
Acto de comilão.
Ladroagem.
(Cp. \textunderscore comilão\textunderscore )
\section{Comilão}
\begin{itemize}
\item {Grp. gram.:m.}
\end{itemize}
\begin{itemize}
\item {Proveniência:(Do rad. de \textunderscore comer\textunderscore )}
\end{itemize}
Aquelle que come muito; glotão.
\section{Comilitão}
\begin{itemize}
\item {Grp. gram.:m.}
\end{itemize}
\begin{itemize}
\item {Utilização:Des.}
\end{itemize}
\begin{itemize}
\item {Proveniência:(Lat. \textunderscore commilito\textunderscore )}
\end{itemize}
Companheiro de armas.
Camarada; sócio. Cf. \textunderscore Anat. Joc.\textunderscore , I, 102.
\section{Cominação}
\begin{itemize}
\item {Grp. gram.:f.}
\end{itemize}
\begin{itemize}
\item {Proveniência:(Lat. \textunderscore comminatio\textunderscore )}
\end{itemize}
Acto de cominar.
\section{Cominador}
\begin{itemize}
\item {Grp. gram.:adj.}
\end{itemize}
\begin{itemize}
\item {Proveniência:(Lat. \textunderscore comminator\textunderscore )}
\end{itemize}
Que comina.
\section{Cominar}
\begin{itemize}
\item {Grp. gram.:v. t.}
\end{itemize}
\begin{itemize}
\item {Proveniência:(Lat. \textunderscore comminari\textunderscore )}
\end{itemize}
Ameaçar com pena.
Impor, prescrever, (pena, castigo).
\section{Cominativo}
\begin{itemize}
\item {Grp. gram.:adj.}
\end{itemize}
O mesmo que \textunderscore cominatório\textunderscore .
\section{Cominatório}
\begin{itemize}
\item {Grp. gram.:adj.}
\end{itemize}
\begin{itemize}
\item {Proveniência:(Do lat. \textunderscore comminator\textunderscore )}
\end{itemize}
Que envolve cominação.
\section{Cominge}
\begin{itemize}
\item {Grp. gram.:m.}
\end{itemize}
\begin{itemize}
\item {Utilização:Ant.}
\end{itemize}
\begin{itemize}
\item {Proveniência:(De \textunderscore Comminge\textunderscore , n. p.)}
\end{itemize}
Morteiro, de 16 ou 18 pollegadas.
\section{Cominheiro}
\begin{itemize}
\item {Grp. gram.:m.}
\end{itemize}
\begin{itemize}
\item {Utilização:Fig.}
\end{itemize}
\begin{itemize}
\item {Proveniência:(De \textunderscore cominho\textunderscore )}
\end{itemize}
Vendedor de cominhos.
Aquelle que dá valor a insignificâncias.
\section{Cominho}
\begin{itemize}
\item {Grp. gram.:m.}
\end{itemize}
\begin{itemize}
\item {Grp. gram.:Pl.}
\end{itemize}
\begin{itemize}
\item {Proveniência:(Gr. \textunderscore kuminon\textunderscore )}
\end{itemize}
Planta umbellífera.
Os grãos dessa planta.
\section{Cominuir}
\begin{itemize}
\item {Grp. gram.:v. t.}
\end{itemize}
\begin{itemize}
\item {Proveniência:(Lat. \textunderscore comminuere\textunderscore )}
\end{itemize}
Partir em bocados, fragmentar.
\section{Cominutivo}
\begin{itemize}
\item {Grp. gram.:adj.}
\end{itemize}
\begin{itemize}
\item {Proveniência:(De \textunderscore cominuir\textunderscore )}
\end{itemize}
Em que houve divisão ou fragmentação.
\section{Comirante}
\begin{itemize}
\item {Grp. gram.:adj.}
\end{itemize}
Que comira.
\section{Comirar}
\begin{itemize}
\item {Grp. gram.:v. t.}
\end{itemize}
\begin{itemize}
\item {Utilização:Des.}
\end{itemize}
\begin{itemize}
\item {Proveniência:(De \textunderscore com...\textunderscore  + \textunderscore mirar\textunderscore )}
\end{itemize}
Atender.
Considerar por todos os lados.
\section{Comiscar}
\begin{itemize}
\item {Grp. gram.:v. i.}
\end{itemize}
\begin{itemize}
\item {Proveniência:(De \textunderscore comer\textunderscore )}
\end{itemize}
Comer aos poucos ou amiúde, por guloseima.
\section{Comiseração}
\begin{itemize}
\item {Grp. gram.:f.}
\end{itemize}
\begin{itemize}
\item {Proveniência:(Lat. \textunderscore commiseratio\textunderscore )}
\end{itemize}
Acto de comiserar-se.
\section{Comiserador}
\begin{itemize}
\item {Grp. gram.:adj.}
\end{itemize}
\begin{itemize}
\item {Proveniência:(De \textunderscore comiserar\textunderscore )}
\end{itemize}
Que inspira compaixão.
Que tem compaixão, que se comisera.
\section{Comiserar}
\begin{itemize}
\item {Grp. gram.:v. t.}
\end{itemize}
\begin{itemize}
\item {Grp. gram.:V. p.}
\end{itemize}
\begin{itemize}
\item {Proveniência:(Lat. \textunderscore commiserari\textunderscore )}
\end{itemize}
Inspirar dó, compaixão, pena a.
Têr compaixão, piedade.
\section{Comiserativo}
\begin{itemize}
\item {Grp. gram.:adj.}
\end{itemize}
Que produz comiseração. Cf. Júl. Castilho, \textunderscore Lisb. Ant.\textunderscore 
\section{Comissairaria}
\begin{itemize}
\item {Grp. gram.:f.}
\end{itemize}
\begin{itemize}
\item {Proveniência:(De \textunderscore comissairo\textunderscore , por \textunderscore comissário\textunderscore )}
\end{itemize}
Funções de comissário comercial.
\section{Comissão}
\begin{itemize}
\item {Grp. gram.:f.}
\end{itemize}
\begin{itemize}
\item {Proveniência:(Lat. \textunderscore commissio\textunderscore )}
\end{itemize}
Acto de encarregar, de cometer.
Encargo.
Pessôas, encarregadas de tratar conjuntamente um assunto.
Reunião dessas pessôas para êsse efeito.
Gratificação ou retribuição, paga pelo comitente ao comissionado.
Carta de corso.
\section{Comissariado}
\begin{itemize}
\item {Grp. gram.:m.}
\end{itemize}
Cargo do comissário.
Repartição, em que o comissário exerce suas funções.
\section{Comissário}
\begin{itemize}
\item {Grp. gram.:m.}
\end{itemize}
\begin{itemize}
\item {Proveniência:(Do lat. \textunderscore commissus\textunderscore )}
\end{itemize}
Aquele que exerce comissão.
Aquele que representa o Govêrno ou outra entidade, junto de uma Companhia ou em funções de administração.
\section{Comissionar}
\begin{itemize}
\item {Grp. gram.:v. t.}
\end{itemize}
\begin{itemize}
\item {Proveniência:(Do lat. \textunderscore commissio\textunderscore )}
\end{itemize}
Dar comissão a.
Encarregar provisoriamente.
\section{Comissionista}
\begin{itemize}
\item {Grp. gram.:m.}
\end{itemize}
\begin{itemize}
\item {Utilização:Bras}
\end{itemize}
\begin{itemize}
\item {Proveniência:(De \textunderscore comissão\textunderscore )}
\end{itemize}
Indivíduo, encarregado de comissão comercial ou industrial.
\section{Comitato}
\begin{itemize}
\item {Grp. gram.:m.}
\end{itemize}
\begin{itemize}
\item {Utilização:Ant.}
\end{itemize}
O mesmo que \textunderscore condado\textunderscore .
\section{Comitiva}
\begin{itemize}
\item {Grp. gram.:f.}
\end{itemize}
\begin{itemize}
\item {Proveniência:(Do lat. \textunderscore comes\textunderscore , \textunderscore comitis\textunderscore )}
\end{itemize}
Gente, que acompanha.
Séqüito.
\section{Comitre}
\begin{itemize}
\item {Grp. gram.:m.}
\end{itemize}
\begin{itemize}
\item {Utilização:Ant.}
\end{itemize}
\begin{itemize}
\item {Proveniência:(Do lat. \textunderscore comes\textunderscore , \textunderscore comitis\textunderscore )}
\end{itemize}
Official de galés, que superintendia nos forçados.
\section{Comível}
\begin{itemize}
\item {Grp. gram.:adj.}
\end{itemize}
\begin{itemize}
\item {Proveniência:(De \textunderscore comer\textunderscore )}
\end{itemize}
O mesmo que \textunderscore comestível\textunderscore .
\section{Comma}
\begin{itemize}
\item {Grp. gram.:f.}
\end{itemize}
\begin{itemize}
\item {Utilização:Des.}
\end{itemize}
\begin{itemize}
\item {Grp. gram.:Pl.}
\end{itemize}
\begin{itemize}
\item {Utilização:Mús.}
\end{itemize}
\begin{itemize}
\item {Proveniência:(Gr. \textunderscore komma\textunderscore )}
\end{itemize}
Vírgula.
Sinal ortográphico, composto de duas vírgulas naturaes ou invertidas, o mesmo que \textunderscore aspas\textunderscore .
Distância entre o semitom maior e o menor.
\section{Commandamento}
\begin{itemize}
\item {Grp. gram.:m.}
\end{itemize}
\begin{itemize}
\item {Utilização:Des.}
\end{itemize}
(V.commando)
\section{Commandância}
\begin{itemize}
\item {Grp. gram.:f.}
\end{itemize}
Cargo de commandante. Cf. Camillo, \textunderscore Perfil\textunderscore , 237 e 302.
\section{Commandanta}
\begin{itemize}
\item {Grp. gram.:f.}
\end{itemize}
\begin{itemize}
\item {Utilização:Fam.}
\end{itemize}
Mulher do commandante.
\section{Commandante}
\begin{itemize}
\item {Grp. gram.:adj.}
\end{itemize}
\begin{itemize}
\item {Grp. gram.:M.}
\end{itemize}
\begin{itemize}
\item {Proveniência:(De \textunderscore commandar\textunderscore )}
\end{itemize}
Que commanda.
Aquelle que tem um commando militar.
\section{Commandar}
\begin{itemize}
\item {Grp. gram.:v. t.}
\end{itemize}
\begin{itemize}
\item {Proveniência:(De \textunderscore com\textunderscore  + \textunderscore mandar\textunderscore )}
\end{itemize}
Dirigir, como superior; mandar.
\section{Commandita}
\begin{itemize}
\item {Grp. gram.:f.}
\end{itemize}
\begin{itemize}
\item {Proveniência:(Fr. \textunderscore commandite\textunderscore )}
\end{itemize}
Fórma de sociedade commercial, em que há um ou mais associados, de responsabilidade solidária, e um ou mais sócios capitalistas, de responsabilidade que não excede o capital subscripto.
\section{Commanditar}
\begin{itemize}
\item {Grp. gram.:v. t.}
\end{itemize}
\begin{itemize}
\item {Proveniência:(De \textunderscore commandita\textunderscore )}
\end{itemize}
Encarregar da administração dos fundos, numa sociedade de commandita.
\section{Commanditário}
\begin{itemize}
\item {Grp. gram.:m.}
\end{itemize}
\begin{itemize}
\item {Proveniência:(De \textunderscore commandita\textunderscore )}
\end{itemize}
Sócio capitalista ou fornecedor de fundos, em uma sociedade de commandita.
\section{Commando}
\begin{itemize}
\item {Grp. gram.:m.}
\end{itemize}
Acção de commandar.
Govêrno de uma divisão de tropas.
\section{Commedidamente}
\begin{itemize}
\item {Grp. gram.:adv.}
\end{itemize}
De modo commedido.
Com commedimento.
\section{Commedido}
\begin{itemize}
\item {Grp. gram.:adj.}
\end{itemize}
\begin{itemize}
\item {Proveniência:(De \textunderscore commedir\textunderscore )}
\end{itemize}
Moderado.
Respeitoso.
\section{Commedimento}
\begin{itemize}
\item {Grp. gram.:m.}
\end{itemize}
\begin{itemize}
\item {Proveniência:(De \textunderscore commedir\textunderscore )}
\end{itemize}
Carácter daquelle ou daquillo que é commedido.
\section{Commedir}
\begin{itemize}
\item {Grp. gram.:v. t.}
\end{itemize}
\begin{itemize}
\item {Proveniência:(De \textunderscore com...\textunderscore  + \textunderscore medir\textunderscore )}
\end{itemize}
Adequar.
Regular convenientemente.
Sujeitar ao dever.
Tornar respeitoso.
Moderar.
\section{Commego}
\begin{itemize}
\item {Grp. gram.:loc. pron.}
\end{itemize}
\begin{itemize}
\item {Utilização:Ant.}
\end{itemize}
O mesmo que \textunderscore commigo\textunderscore . Cf. Sim. Machado, f. 78, v., cit. por Castilho.
\section{Commelina}
\begin{itemize}
\item {Grp. gram.:f.}
\end{itemize}
\begin{itemize}
\item {Proveniência:(De \textunderscore Commelin\textunderscore , n. p.)}
\end{itemize}
Gênero de plantas, a que pertence a tradescância.
\section{Commelináceas}
\begin{itemize}
\item {Grp. gram.:f. pl.}
\end{itemize}
O mesmo ou melhor que \textunderscore commelíneas\textunderscore .
\section{Commelíneas}
\begin{itemize}
\item {Grp. gram.:f. pl.}
\end{itemize}
\begin{itemize}
\item {Proveniência:(De \textunderscore commelina\textunderscore )}
\end{itemize}
Família de plantas herbáceas, a que pertence a tradescância.
\section{Commemoração}
\begin{itemize}
\item {Grp. gram.:f.}
\end{itemize}
\begin{itemize}
\item {Proveniência:(Lat. \textunderscore commemoratio\textunderscore )}
\end{itemize}
Acto de commemorar.
\section{Commemorar}
\begin{itemize}
\item {Grp. gram.:v. t.}
\end{itemize}
\begin{itemize}
\item {Proveniência:(Lat. \textunderscore commemorare\textunderscore )}
\end{itemize}
Lembrar, trazer á memória.
Solennizar, recordando: \textunderscore commemorar uma victória\textunderscore .
\section{Commemorativo}
\begin{itemize}
\item {Grp. gram.:adj.}
\end{itemize}
\begin{itemize}
\item {Proveniência:(Do lat. \textunderscore commemoratus\textunderscore )}
\end{itemize}
Que commemora.
\section{Commemorável}
\begin{itemize}
\item {Grp. gram.:adj.}
\end{itemize}
\begin{itemize}
\item {Proveniência:(Lat. \textunderscore commemorabilis\textunderscore )}
\end{itemize}
Que merece sêr commemorado.
\section{Commenda}
\begin{itemize}
\item {Grp. gram.:f.}
\end{itemize}
Antigo benefício, que se dava a ecclesiásticos ou a cavalleiros de Ordens militares.
Porção de terras, com que officialmente se recompensavam serviços.
Distincção puramente honorifica, e correspondente a um grau de Ordem militar.
Insígnia de commendador.
(B. lat. \textunderscore commenda\textunderscore )
\section{Commendadeira}
(fem. de \textunderscore commendador\textunderscore )
\section{Commendador}
\begin{itemize}
\item {Grp. gram.:m.}
\end{itemize}
Aquelle que tem commenda.
\section{Commendadoria}
\begin{itemize}
\item {Grp. gram.:f.}
\end{itemize}
\begin{itemize}
\item {Proveniência:(De \textunderscore commendador\textunderscore )}
\end{itemize}
Dignidade de commendador.
Benefício de commenda.
\section{Commendamento}
\begin{itemize}
\item {Grp. gram.:m.}
\end{itemize}
\begin{itemize}
\item {Utilização:Ant.}
\end{itemize}
\begin{itemize}
\item {Proveniência:(De \textunderscore commendar\textunderscore )}
\end{itemize}
O mesmo que \textunderscore recommendação\textunderscore . Cf. \textunderscore Port. Mon. Hist.\textunderscore , \textunderscore Script.\textunderscore , 286.
\section{Commendar}
\begin{itemize}
\item {Grp. gram.:v. t.}
\end{itemize}
(V.encommendar)
\section{Commendataria}
\begin{itemize}
\item {Grp. gram.:f.}
\end{itemize}
(V.commendadoria)
\section{Commendatário}
\begin{itemize}
\item {Grp. gram.:adj.}
\end{itemize}
Que frui benefício de commenda.
(B. lat. \textunderscore commendatarius\textunderscore )
\section{Commendatício}
\begin{itemize}
\item {Grp. gram.:adj.}
\end{itemize}
\begin{itemize}
\item {Proveniência:(Lat. \textunderscore commendaticius\textunderscore )}
\end{itemize}
Que se recommenda.
\section{Commendativo}
\begin{itemize}
\item {Grp. gram.:adj.}
\end{itemize}
\begin{itemize}
\item {Proveniência:(Do lat. \textunderscore commendativus\textunderscore )}
\end{itemize}
Que recommenda, que é próprio para recommendar.
\section{Commendatório}
\begin{itemize}
\item {Grp. gram.:adj.}
\end{itemize}
(V.commendatício)
\section{Commendavelmente}
\begin{itemize}
\item {Grp. gram.:adv.}
\end{itemize}
\begin{itemize}
\item {Utilização:Ant.}
\end{itemize}
De modo recommendável.
\section{Commendella}
\begin{itemize}
\item {Grp. gram.:f.}
\end{itemize}
\begin{itemize}
\item {Utilização:Ant.}
\end{itemize}
Pequena commenda.
\section{Commensal}
\begin{itemize}
\item {Grp. gram.:m.}
\end{itemize}
\begin{itemize}
\item {Proveniência:(Do lat. \textunderscore cum\textunderscore  + \textunderscore mensa\textunderscore )}
\end{itemize}
Cada um dos que comem juntos.
Aquelle que come habitualmente em casa alheia; parasito.
\section{Commensalidade}
\begin{itemize}
\item {Grp. gram.:f.}
\end{itemize}
Qualidade de quem é commensal.
\section{Commensurabilidade}
\begin{itemize}
\item {Grp. gram.:f.}
\end{itemize}
Qualidade daquillo que é commensurável.
\section{Commensuração}
\begin{itemize}
\item {Grp. gram.:f.}
\end{itemize}
Acto de commensurar.
\section{Commensurar}
\begin{itemize}
\item {Grp. gram.:v. t.}
\end{itemize}
\begin{itemize}
\item {Proveniência:(Do lat. \textunderscore cum\textunderscore  + \textunderscore mensurare\textunderscore )}
\end{itemize}
Medir (duas ou mais quantidades) com a mesma unidade.
Proporcionar.
\section{Commensurável}
\begin{itemize}
\item {Grp. gram.:adj.}
\end{itemize}
\begin{itemize}
\item {Proveniência:(De \textunderscore commensurar\textunderscore )}
\end{itemize}
Que tem medida commum.
Que póde medir-se.
\section{Commentação}
\begin{itemize}
\item {Grp. gram.:f.}
\end{itemize}
Acto ou effeito de commentar; commentário.
\section{Commentador}
\begin{itemize}
\item {Grp. gram.:m.}
\end{itemize}
Aquelle que commenta.
\section{Commentar}
\begin{itemize}
\item {Grp. gram.:v. t.}
\end{itemize}
\begin{itemize}
\item {Proveniência:(Lat. \textunderscore commentare\textunderscore )}
\end{itemize}
Explicar, interpretando ou annotando: \textunderscore commentar os«Lusiadas»\textunderscore .
Fazer commentário a: \textunderscore commentar uma notícia\textunderscore .
Criticar.
\section{Commentário}
\begin{itemize}
\item {Grp. gram.:m.}
\end{itemize}
\begin{itemize}
\item {Proveniência:(Lat. \textunderscore commentarius\textunderscore )}
\end{itemize}
Série de notas, com que se esclarece ou critíca uma producção literária ou scientífica.
Ponderações sôbre um facto.
Crítica maliciosa.
\section{Commentício}
\begin{itemize}
\item {Grp. gram.:adj.}
\end{itemize}
\begin{itemize}
\item {Utilização:Des.}
\end{itemize}
\begin{itemize}
\item {Proveniência:(Lat. \textunderscore commenticius\textunderscore )}
\end{itemize}
Fabuloso.
\section{Commentista}
\begin{itemize}
\item {Grp. gram.:m.}
\end{itemize}
\begin{itemize}
\item {Utilização:Des.}
\end{itemize}
\begin{itemize}
\item {Proveniência:(De \textunderscore commento\textunderscore )}
\end{itemize}
O mesmo que \textunderscore commentador\textunderscore .
\section{Commento}
\begin{itemize}
\item {Grp. gram.:m.}
\end{itemize}
\begin{itemize}
\item {Proveniência:(Lat. \textunderscore commentus\textunderscore )}
\end{itemize}
O mesmo que \textunderscore commentário\textunderscore .
\section{Commercial}
\begin{itemize}
\item {Grp. gram.:adj.}
\end{itemize}
Relativo ao commércio: \textunderscore associação commercial\textunderscore .
\section{Commercialista}
\begin{itemize}
\item {Grp. gram.:m.}
\end{itemize}
Indivíduo, versado em Direito Commercial.
\section{Commercialmente}
\begin{itemize}
\item {Grp. gram.:adv.}
\end{itemize}
De modo commercial.
\section{Commerciante}
\begin{itemize}
\item {Grp. gram.:m.  e  adj.}
\end{itemize}
\begin{itemize}
\item {Proveniência:(De \textunderscore commerciar\textunderscore )}
\end{itemize}
O que exerce commércio.
\section{Commerciar}
\begin{itemize}
\item {Grp. gram.:v. i.}
\end{itemize}
Exercer commércio, têr commércio.
\section{Commerciável}
\begin{itemize}
\item {Grp. gram.:adj.}
\end{itemize}
\begin{itemize}
\item {Proveniência:(De \textunderscore commerciar\textunderscore )}
\end{itemize}
Que póde sêr objecto de commércio.
\section{Commércio}
\begin{itemize}
\item {Grp. gram.:m.}
\end{itemize}
\begin{itemize}
\item {Proveniência:(Lat. \textunderscore commercium\textunderscore )}
\end{itemize}
Permutação de productos naturaes ou artificiáes.
Troca de valores.
A classe dos que commerciam: \textunderscore o commércio está satisfeito\textunderscore .
Relações, convivência, trato.
\section{Commersão}
\begin{itemize}
\item {Grp. gram.:m.}
\end{itemize}
Peixe negro das profundidades do Pacífico.
\section{Commêssea}
\begin{itemize}
\item {Grp. gram.:f.}
\end{itemize}
\begin{itemize}
\item {Utilização:Ant.}
\end{itemize}
O mesmo que \textunderscore commenda\textunderscore .
\section{Commetácula}
\begin{itemize}
\item {Grp. gram.:f.}
\end{itemize}
\begin{itemize}
\item {Proveniência:(Lat. \textunderscore commetacula\textunderscore )}
\end{itemize}
Varinha, que os flâmines usavam em Roma, quando iam para o sacrifício, e com a qual desviavam a multidão, para passar.
\section{Commetedor}
\begin{itemize}
\item {Grp. gram.:m.  e  adj.}
\end{itemize}
O que commete: \textunderscore commetedor de um crime\textunderscore .
\section{Commetente}
\begin{itemize}
\item {Grp. gram.:m.  e  f.}
\end{itemize}
(V.committente)
\section{Commeter}
\begin{itemize}
\item {Grp. gram.:v. t.}
\end{itemize}
\begin{itemize}
\item {Proveniência:(Lat. \textunderscore committere\textunderscore )}
\end{itemize}
Fazer; praticar; perpetrar: \textunderscore commeter um êrro\textunderscore .
Confiar, encarregar: \textunderscore commeti-lhe aquella missão\textunderscore .
Propor.
Atacar: \textunderscore commeter a fortaleza\textunderscore .
Emprehender.
\section{Commetida}
\begin{itemize}
\item {Grp. gram.:f.}
\end{itemize}
\begin{itemize}
\item {Proveniência:(De \textunderscore commeter\textunderscore )}
\end{itemize}
Ataque, investida.
\section{Commetimento}
\begin{itemize}
\item {Grp. gram.:m.}
\end{itemize}
Acto de commeter.
\section{Commigo}
\begin{itemize}
\item {Grp. gram.:loc. pron.}
\end{itemize}
\begin{itemize}
\item {Proveniência:(De \textunderscore com\textunderscore  + \textunderscore migo\textunderscore , flexão do pron. \textunderscore eu\textunderscore )}
\end{itemize}
Em companhia de mim: \textunderscore fugiu commigo\textunderscore .
De mim para mim: \textunderscore o que eu dizia commigo\textunderscore .
A respeito de mim: \textunderscore aquillo não era commigo\textunderscore .
\section{Commilitão}
\begin{itemize}
\item {Grp. gram.:m.}
\end{itemize}
\begin{itemize}
\item {Utilização:Des.}
\end{itemize}
\begin{itemize}
\item {Proveniência:(Lat. \textunderscore commilito\textunderscore )}
\end{itemize}
Companheiro de armas.
Camarada; sócio. Cf. \textunderscore Anat. Joc.\textunderscore , I, 102.
\section{Comminação}
\begin{itemize}
\item {Grp. gram.:f.}
\end{itemize}
\begin{itemize}
\item {Proveniência:(Lat. \textunderscore comminatio\textunderscore )}
\end{itemize}
Acto de comminar.
\section{Comminador}
\begin{itemize}
\item {Grp. gram.:adj.}
\end{itemize}
\begin{itemize}
\item {Proveniência:(Lat. \textunderscore comminator\textunderscore )}
\end{itemize}
Que commina.
\section{Comminar}
\begin{itemize}
\item {Grp. gram.:v. t.}
\end{itemize}
\begin{itemize}
\item {Proveniência:(Lat. \textunderscore comminari\textunderscore )}
\end{itemize}
Ameaçar com pena.
Impor, prescrever, (pena, castigo).
\section{Comminativo}
\begin{itemize}
\item {Grp. gram.:adj.}
\end{itemize}
O mesmo que \textunderscore comminatório\textunderscore .
\section{Comminatório}
\begin{itemize}
\item {Grp. gram.:adj.}
\end{itemize}
\begin{itemize}
\item {Proveniência:(Do lat. \textunderscore comminator\textunderscore )}
\end{itemize}
Que envolve comminação.
\section{Comminuir}
\begin{itemize}
\item {Grp. gram.:v. t.}
\end{itemize}
\begin{itemize}
\item {Proveniência:(Lat. \textunderscore comminuere\textunderscore )}
\end{itemize}
Partir em bocados, fragmentar.
\section{Comminutivo}
\begin{itemize}
\item {Grp. gram.:adj.}
\end{itemize}
\begin{itemize}
\item {Proveniência:(De \textunderscore comminuir\textunderscore )}
\end{itemize}
Em que houve divisão ou fragmentação.
\section{Commíphora}
\begin{itemize}
\item {Grp. gram.:f.}
\end{itemize}
Árvore amaryllídea, (\textunderscore commiphora madagascarensis\textunderscore ).
\section{Commirante}
\begin{itemize}
\item {Grp. gram.:adj.}
\end{itemize}
Que commira.
\section{Commirar}
\begin{itemize}
\item {Grp. gram.:v. t.}
\end{itemize}
\begin{itemize}
\item {Utilização:Des.}
\end{itemize}
\begin{itemize}
\item {Proveniência:(De \textunderscore com...\textunderscore  + \textunderscore mirar\textunderscore )}
\end{itemize}
Attender.
Considerar por todos os lados.
\section{Commiseração}
\begin{itemize}
\item {Grp. gram.:f.}
\end{itemize}
\begin{itemize}
\item {Proveniência:(Lat. \textunderscore commiseratio\textunderscore )}
\end{itemize}
Acto de commiserar-se.
\section{Commiserador}
\begin{itemize}
\item {Grp. gram.:adj.}
\end{itemize}
\begin{itemize}
\item {Proveniência:(De \textunderscore commiserar\textunderscore )}
\end{itemize}
Que inspira compaixão.
Que tem compaixão, que se commisera.
\section{Commiserar}
\begin{itemize}
\item {Grp. gram.:v. t.}
\end{itemize}
\begin{itemize}
\item {Grp. gram.:V. p.}
\end{itemize}
\begin{itemize}
\item {Proveniência:(Lat. \textunderscore commiserari\textunderscore )}
\end{itemize}
Inspirar dó, compaixão, pena a.
Têr compaixão, piedade.
\section{Commiserativo}
\begin{itemize}
\item {Grp. gram.:adj.}
\end{itemize}
Que produz commiseração. Cf. Júl. Castilho, \textunderscore Lisb. Ant.\textunderscore 
\section{Commissairaria}
\begin{itemize}
\item {Grp. gram.:f.}
\end{itemize}
\begin{itemize}
\item {Proveniência:(De \textunderscore commissairo\textunderscore , por \textunderscore commissário\textunderscore )}
\end{itemize}
Funcções de commissário commercial.
\section{Commissão}
\begin{itemize}
\item {Grp. gram.:f.}
\end{itemize}
\begin{itemize}
\item {Proveniência:(Lat. \textunderscore commissio\textunderscore )}
\end{itemize}
Acto de encarregar, de commeter.
Encargo.
Pessôas, encarregadas de tratar conjuntamente um assumpto.
Reunião dessas pessôas para êsse effeito.
Gratificação ou retribuição, paga pelo committente ao commissionado.
Carta de corso.
\section{Commissariado}
\begin{itemize}
\item {Grp. gram.:m.}
\end{itemize}
Cargo do commissário.
Repartição, em que o commissário exerce suas funcções.
\section{Commissário}
\begin{itemize}
\item {Grp. gram.:m.}
\end{itemize}
\begin{itemize}
\item {Proveniência:(Do lat. \textunderscore commissus\textunderscore )}
\end{itemize}
Aquelle que exerce commissão.
Aquelle que representa o Govêrno ou outra entidade, junto de uma Companhia ou em funcções de administração.
\section{Commissionar}
\begin{itemize}
\item {Grp. gram.:v. t.}
\end{itemize}
\begin{itemize}
\item {Proveniência:(Do lat. \textunderscore commissio\textunderscore )}
\end{itemize}
Dar commissão a.
Encarregar provisoriamente.
\section{Commissionista}
\begin{itemize}
\item {Grp. gram.:m.}
\end{itemize}
\begin{itemize}
\item {Utilização:Bras}
\end{itemize}
\begin{itemize}
\item {Proveniência:(De \textunderscore commissão\textunderscore )}
\end{itemize}
Indivíduo, encarregado de commissão comercial ou industrial.
\section{Comisso}
\begin{itemize}
\item {Grp. gram.:m.}
\end{itemize}
\begin{itemize}
\item {Proveniência:(Lat. \textunderscore commissum\textunderscore )}
\end{itemize}
Multa, pena, em que incorre o que falta a certas condições, impostas por um contrato ou por uma lei.
\section{Comissório}
\begin{itemize}
\item {Grp. gram.:adj.}
\end{itemize}
\begin{itemize}
\item {Grp. gram.:M.}
\end{itemize}
\begin{itemize}
\item {Utilização:Ant.}
\end{itemize}
\begin{itemize}
\item {Proveniência:(Lat. \textunderscore commissorius\textunderscore )}
\end{itemize}
Cuja inexactidão determina a nulidade de um contrato.
Carta de coito ou indicação das penas que se aplicavam a quem quebrantasse os privilégios de uma terra.
\section{Comissura}
\begin{itemize}
\item {Grp. gram.:f.}
\end{itemize}
\begin{itemize}
\item {Proveniência:(Do lat. \textunderscore commissura\textunderscore )}
\end{itemize}
Linha de junção: \textunderscore a comissura dos lábios\textunderscore .
Sutura.
Abertura, fenda.
\section{Comistão}
\begin{itemize}
\item {Grp. gram.:f.}
\end{itemize}
\begin{itemize}
\item {Utilização:Des.}
\end{itemize}
\begin{itemize}
\item {Proveniência:(Lat. \textunderscore commixtio\textunderscore )}
\end{itemize}
O mesmo que \textunderscore mistura\textunderscore .
\section{Comistura}
\begin{itemize}
\item {Grp. gram.:f.}
\end{itemize}
\begin{itemize}
\item {Proveniência:(Lat. \textunderscore commixtura\textunderscore )}
\end{itemize}
O mesmo que \textunderscore comistão\textunderscore .
\section{Comisturar}
\begin{itemize}
\item {Grp. gram.:v. t.}
\end{itemize}
\begin{itemize}
\item {Proveniência:(De \textunderscore comistura\textunderscore )}
\end{itemize}
O mesmo que \textunderscore misturar\textunderscore .
\section{Comitente}
\begin{itemize}
\item {Grp. gram.:m. ,  f.  e  adj.}
\end{itemize}
\begin{itemize}
\item {Proveniência:(Lat. \textunderscore committens\textunderscore )}
\end{itemize}
Pessôa, que dá comissão, que encarrega; constituinte.
\section{Commisso}
\begin{itemize}
\item {Grp. gram.:m.}
\end{itemize}
\begin{itemize}
\item {Proveniência:(Lat. \textunderscore commissum\textunderscore )}
\end{itemize}
Multa, pena, em que incorre o que falta a certas condições, impostas por um contrato ou por uma lei.
\section{Commissório}
\begin{itemize}
\item {Grp. gram.:adj.}
\end{itemize}
\begin{itemize}
\item {Grp. gram.:M.}
\end{itemize}
\begin{itemize}
\item {Utilização:Ant.}
\end{itemize}
\begin{itemize}
\item {Proveniência:(Lat. \textunderscore commissorius\textunderscore )}
\end{itemize}
Cuja inexactidão determina a nullidade de um contrato.
Carta de coito ou indicação das penas que se applicavam a quem quebrantasse os privilégios de uma terra.
\section{Commissura}
\begin{itemize}
\item {Grp. gram.:f.}
\end{itemize}
\begin{itemize}
\item {Proveniência:(Do lat. \textunderscore commissura\textunderscore )}
\end{itemize}
Linha de juncção: \textunderscore a commissura dos lábios\textunderscore .
Sutura.
Abertura, fenda.
\section{Commistão}
\begin{itemize}
\item {Grp. gram.:f.}
\end{itemize}
\begin{itemize}
\item {Utilização:Des.}
\end{itemize}
\begin{itemize}
\item {Proveniência:(Lat. \textunderscore commixtio\textunderscore )}
\end{itemize}
O mesmo que \textunderscore mistura\textunderscore .
\section{Commistura}
\begin{itemize}
\item {Grp. gram.:f.}
\end{itemize}
\begin{itemize}
\item {Proveniência:(Lat. \textunderscore commixtura\textunderscore )}
\end{itemize}
O mesmo que \textunderscore comistão\textunderscore .
\section{Commisturar}
\begin{itemize}
\item {Grp. gram.:v. t.}
\end{itemize}
\begin{itemize}
\item {Proveniência:(De \textunderscore commistura\textunderscore )}
\end{itemize}
O mesmo que \textunderscore misturar\textunderscore .
\section{Committente}
\begin{itemize}
\item {Grp. gram.:m. ,  f.  e  adj.}
\end{itemize}
\begin{itemize}
\item {Proveniência:(Lat. \textunderscore committens\textunderscore )}
\end{itemize}
Pessôa, que dá commissão, que encarrega; constituinte.
\section{Commoção}
\begin{itemize}
\item {Grp. gram.:f.}
\end{itemize}
Acto ou effeito de commover.
Abalo.
Motim; revolta.
\section{Commocionar}
\begin{itemize}
\item {Proveniência:(De \textunderscore commoção\textunderscore )}
\end{itemize}
\textunderscore v. t.\textunderscore  (e der.)
O mesmo que \textunderscore commover\textunderscore , etc.
\section{Cômmoda}
\begin{itemize}
\item {Grp. gram.:f.}
\end{itemize}
\begin{itemize}
\item {Proveniência:(De \textunderscore cômmodo\textunderscore )}
\end{itemize}
Espécie de mesa, com gavetas desde a base até á face superior.
\section{Commodamente}
\begin{itemize}
\item {Grp. gram.:adv.}
\end{itemize}
De modo cômmodo.
Á vontade.
\section{Commodante}
\begin{itemize}
\item {Grp. gram.:m.  e  f.}
\end{itemize}
\begin{itemize}
\item {Proveniência:(Lat. \textunderscore commodans\textunderscore )}
\end{itemize}
Pessôa, que empresta gratuitamente objecto não fungível.
\section{Commodatário}
\begin{itemize}
\item {Grp. gram.:m.}
\end{itemize}
\begin{itemize}
\item {Proveniência:(Lat. \textunderscore commodatarius\textunderscore )}
\end{itemize}
Aquelle que contrai commodato.
\section{Commodato}
\begin{itemize}
\item {Grp. gram.:m.}
\end{itemize}
\begin{itemize}
\item {Proveniência:(Lat. \textunderscore commodatum\textunderscore )}
\end{itemize}
Empréstimo gratuito de coisa não fungível.
\section{Commodidade}
\begin{itemize}
\item {Grp. gram.:f.}
\end{itemize}
\begin{itemize}
\item {Proveniência:(Lat. \textunderscore commoditas\textunderscore )}
\end{itemize}
Qualidade do que é cômmodo.
Aquillo que é cômmodo.
Opportunidade.
Bem-estar.
\section{Commodista}
\begin{itemize}
\item {Grp. gram.:m. ,  f.  e  adj.}
\end{itemize}
\begin{itemize}
\item {Utilização:Fam.}
\end{itemize}
\begin{itemize}
\item {Proveniência:(De \textunderscore cômmodo\textunderscore )}
\end{itemize}
Pessôa, que attende principalmente ás suas commodidades; egoísta.
\section{Cômmodo}
\begin{itemize}
\item {Grp. gram.:adj.}
\end{itemize}
\begin{itemize}
\item {Grp. gram.:M.}
\end{itemize}
\begin{itemize}
\item {Utilização:Prov.}
\end{itemize}
\begin{itemize}
\item {Utilização:alent.}
\end{itemize}
\begin{itemize}
\item {Proveniência:(Lat. \textunderscore commodus\textunderscore )}
\end{itemize}
Útil.
Adequado.
Favorável.
Tranquillo.
O mesmo que \textunderscore commodidade\textunderscore : \textunderscore trabalha para seu cômmodo\textunderscore .
Conjunto das herdades, que constituem uma lavoira.
Accommodação, emprêgo, (de pessôas).
Agasalho, hospitalidade.
\section{Commodoro}
\begin{itemize}
\item {Grp. gram.:m.}
\end{itemize}
\begin{itemize}
\item {Proveniência:(Ingl. \textunderscore commodore\textunderscore , do cast. \textunderscore comendador\textunderscore )}
\end{itemize}
Commandante de esquadra hollandesa.
Official da marinha inglesa e americana, immediatamente superior ao capitão de mar e guerra.
Título honorífico em associações navaes, em Portugal e noutros países.
\section{Commoração}
\begin{itemize}
\item {Grp. gram.:f.}
\end{itemize}
\begin{itemize}
\item {Proveniência:(Lat. \textunderscore commoratio\textunderscore )}
\end{itemize}
Insistência de um orador em um ponto do seu discurso.
\section{Commorante}
\begin{itemize}
\item {Grp. gram.:adj.}
\end{itemize}
\begin{itemize}
\item {Proveniência:(Lat. \textunderscore commorans\textunderscore )}
\end{itemize}
Que commora.
\section{Commorar}
\begin{itemize}
\item {Grp. gram.:v. i.}
\end{itemize}
\begin{itemize}
\item {Proveniência:(Lat. \textunderscore commorari\textunderscore )}
\end{itemize}
O mesmo que \textunderscore cohabitar\textunderscore .
\section{Commoriente}
\begin{itemize}
\item {Grp. gram.:adj.}
\end{itemize}
\begin{itemize}
\item {Proveniência:(Lat. \textunderscore commoriens\textunderscore )}
\end{itemize}
Que morre juntamente com outrem.
\section{Commovedor}
\begin{itemize}
\item {Grp. gram.:adj.}
\end{itemize}
O mesmo que \textunderscore commovente\textunderscore .
\section{Commovente}
\begin{itemize}
\item {Grp. gram.:adj.}
\end{itemize}
\begin{itemize}
\item {Proveniência:(Lat. \textunderscore commovens\textunderscore )}
\end{itemize}
Que commove.
\section{Commover}
\begin{itemize}
\item {Grp. gram.:v. t.}
\end{itemize}
\begin{itemize}
\item {Grp. gram.:V. i.}
\end{itemize}
\begin{itemize}
\item {Proveniência:(Lat. \textunderscore commovere\textunderscore )}
\end{itemize}
Mover muito.
Deslocar.
Abalar; impressionar; enternecer.
Produzir impressão moral ou enternecimento.
\section{Commua}
\begin{itemize}
\item {Grp. gram.:adj.}
\end{itemize}
\begin{itemize}
\item {Utilização:Ant.}
\end{itemize}
\begin{itemize}
\item {Grp. gram.:F.}
\end{itemize}
\begin{itemize}
\item {Proveniência:(De \textunderscore commum\textunderscore )}
\end{itemize}
(Fem. de \textunderscore commum\textunderscore )
Latrina, retreta.
\section{Commudação}
\begin{itemize}
\item {Grp. gram.:f.}
\end{itemize}
Acto ou effeito de commudar.
\section{Commudar}
\begin{itemize}
\item {Grp. gram.:v. t.}
\end{itemize}
O mesmo que \textunderscore commutar\textunderscore .
\section{Commum}
\begin{itemize}
\item {Grp. gram.:adj.}
\end{itemize}
\begin{itemize}
\item {Grp. gram.:M.}
\end{itemize}
\begin{itemize}
\item {Proveniência:(Lat. \textunderscore communis\textunderscore )}
\end{itemize}
Relativo a muitos ou todos.
Vulgar, habitual.
Feito em communidade, em sociedade.
Insignificante.
Maioria: \textunderscore o commum dos mortaes\textunderscore .
Vulgaridade.
\section{Commummente}
\begin{itemize}
\item {Grp. gram.:adv.}
\end{itemize}
\begin{itemize}
\item {Proveniência:(De \textunderscore commum\textunderscore )}
\end{itemize}
Geralmente; de ordinário; vulgarmente.
\section{Communa}
\begin{itemize}
\item {Grp. gram.:f.}
\end{itemize}
\begin{itemize}
\item {Proveniência:(De \textunderscore commum\textunderscore )}
\end{itemize}
Antigo agrupamento de estrangeiros, especialmente Judeus e Moiros, que eram obrigados a viver em arruamentos determinados.
Povoação que, na Idade-Média, se emancipava do feudalismo, governando-se autonomicamente.
Subdivisão territorial em França.
Administração de concelho.
\section{Communal}
\begin{itemize}
\item {Grp. gram.:adj.}
\end{itemize}
\begin{itemize}
\item {Utilização:Ant.}
\end{itemize}
\begin{itemize}
\item {Grp. gram.:M.}
\end{itemize}
Relativo a communa.
O mesmo que \textunderscore commum\textunderscore .
Habitante de uma communa.
(B. lat. \textunderscore communalis\textunderscore )
\section{Communalismo}
\begin{itemize}
\item {Grp. gram.:m.}
\end{itemize}
Doutrina ou systema dos communalistas.
Municipalismo.
\section{Communalista}
\begin{itemize}
\item {Grp. gram.:m.}
\end{itemize}
\begin{itemize}
\item {Proveniência:(De \textunderscore communal\textunderscore )}
\end{itemize}
Propugnador dos privilégios communaes ou municipaes.
Partidário da descentralização administrativa.
\section{Communalmente}
\begin{itemize}
\item {Grp. gram.:adv.}
\end{itemize}
\begin{itemize}
\item {Utilização:Ant.}
\end{itemize}
\begin{itemize}
\item {Proveniência:(De \textunderscore communal\textunderscore )}
\end{itemize}
Commummente; ordinariamente.
\section{Communeiro}
\begin{itemize}
\item {Grp. gram.:m.}
\end{itemize}
Communal.
Membro das communidades, que em Espanha se revoltaram contra Carlos V.
(Cast. \textunderscore comunero\textunderscore )
\section{Communeza}
\begin{itemize}
\item {Grp. gram.:f.}
\end{itemize}
\begin{itemize}
\item {Utilização:Ant.}
\end{itemize}
O mesmo que \textunderscore communidade\textunderscore .
\section{Commungante}
\begin{itemize}
\item {Grp. gram.:adj.}
\end{itemize}
Que communga.
\section{Commungar}
\begin{itemize}
\item {Grp. gram.:v. t.}
\end{itemize}
\begin{itemize}
\item {Grp. gram.:V. i.}
\end{itemize}
\begin{itemize}
\item {Proveniência:(Lat. \textunderscore communicare\textunderscore )}
\end{itemize}
Administrar a communhão a.
Receber em communhão.
Receber a communhão.
Participar das crenças de uma seita, facção ou grupo de indivíduos: \textunderscore commungar nas ideias de outrem\textunderscore .
Estar de acôrdo.
Tomar parte.
\section{Commungatório}
\begin{itemize}
\item {Grp. gram.:adj.}
\end{itemize}
\begin{itemize}
\item {Grp. gram.:M.}
\end{itemize}
\begin{itemize}
\item {Proveniência:(Do lat. \textunderscore communicator\textunderscore )}
\end{itemize}
Relativo a communhão.
Local, onde se toma communhão.
\section{Communhão}
\begin{itemize}
\item {Grp. gram.:f.}
\end{itemize}
\begin{itemize}
\item {Proveniência:(Lat. \textunderscore communio\textunderscore )}
\end{itemize}
Acto ou effeito de commungar.
Communidade de opiniões.
Participação commum em crenças.
Sacramento da Eucharistia.
Recepção da Eucharistia.
\section{Communial}
\begin{itemize}
\item {Grp. gram.:adj.}
\end{itemize}
\begin{itemize}
\item {Proveniência:(Do lat. \textunderscore communio\textunderscore )}
\end{itemize}
Relativo a communhão.
\section{Communião}
\begin{itemize}
\item {Grp. gram.:f.}
\end{itemize}
\begin{itemize}
\item {Utilização:Des.}
\end{itemize}
O mesmo que \textunderscore communhão\textunderscore . Cf. Garrett, \textunderscore Port. na Balança\textunderscore , 203.
\section{Communicabilidade}
\begin{itemize}
\item {Grp. gram.:f.}
\end{itemize}
Qualidade do que é communicável.
\section{Communicação}
\begin{itemize}
\item {Grp. gram.:f.}
\end{itemize}
\begin{itemize}
\item {Proveniência:(Lat. \textunderscore communicatio\textunderscore )}
\end{itemize}
Acto, effeito ou meio de communicar.
Transmissão: \textunderscore communicação de um telegramma\textunderscore .
Convivência, trato.
Lugar, por onde se passa de um ponto para outro.
\section{Communicado}
\begin{itemize}
\item {Grp. gram.:m.}
\end{itemize}
\begin{itemize}
\item {Proveniência:(De \textunderscore communicar\textunderscore )}
\end{itemize}
Escrito ou artigo, geralmente de interesse particular, dirigido a jornal ou jornaes.
\section{Communicador}
\begin{itemize}
\item {Grp. gram.:m.  e  adj.}
\end{itemize}
\begin{itemize}
\item {Proveniência:(Lat. \textunderscore communicator\textunderscore )}
\end{itemize}
O que communica.
\section{Communicante}
\begin{itemize}
\item {Grp. gram.:adj.}
\end{itemize}
\begin{itemize}
\item {Proveniência:(Lat. \textunderscore communicans\textunderscore )}
\end{itemize}
Que communica.
\section{Communicar}
\begin{itemize}
\item {Grp. gram.:v. t.}
\end{itemize}
\begin{itemize}
\item {Grp. gram.:V. i.}
\end{itemize}
\begin{itemize}
\item {Proveniência:(Lat. \textunderscore communicare\textunderscore )}
\end{itemize}
Fazer commum.
Tornar conhecido, fazer saber, participar: \textunderscore communicar um casamento\textunderscore .
Ligar; pôr em contacto.
Transmittir.
Conviver com. Cf. Filinto, XVII, 197; XVIII, 75 e 168.
Estar ligado; têr contacto.
Têr correspondência.
Têr passagem commum.
\section{Communicativamente}
\begin{itemize}
\item {Grp. gram.:adv.}
\end{itemize}
De modo communicativo.
Expansivamente.
\section{Communicativo}
\begin{itemize}
\item {Grp. gram.:adj.}
\end{itemize}
\begin{itemize}
\item {Proveniência:(Lat. \textunderscore communicativus\textunderscore )}
\end{itemize}
Que se communica facilmente.
Que expõe francamente.
Expansivo.
\section{Communicável}
\begin{itemize}
\item {Grp. gram.:adj.}
\end{itemize}
\begin{itemize}
\item {Proveniência:(Lat. \textunderscore communicabilis\textunderscore )}
\end{itemize}
Que se póde communicar.
Franco, expansivo.
\section{Communidade}
\begin{itemize}
\item {Grp. gram.:f.}
\end{itemize}
\begin{itemize}
\item {Proveniência:(Lat. \textunderscore communitas\textunderscore )}
\end{itemize}
Qualidade daquillo que é commum; communhão.
Sociedade.
Nacionalidade.
Agremiação de indivíduos, que têm a mesma crença ou a mesma norma de vida.
Lugar, onde residem êsses indivíduos: \textunderscore as communidades indianas\textunderscore .
Communa.
\section{Communismo}
\begin{itemize}
\item {Grp. gram.:m.}
\end{itemize}
\begin{itemize}
\item {Proveniência:(De \textunderscore commum\textunderscore )}
\end{itemize}
Theoría social, que pretende a communhão dos bens naturaes e dos productos do trabalho.
\section{Communíssimo}
\begin{itemize}
\item {Grp. gram.:adj.}
\end{itemize}
Muito commum, vulgaríssimo, trivial.
(Sup. de \textunderscore commum\textunderscore )
\section{Communista}
\begin{itemize}
\item {Grp. gram.:adj.}
\end{itemize}
\begin{itemize}
\item {Grp. gram.:M.}
\end{itemize}
\begin{itemize}
\item {Proveniência:(De \textunderscore commum\textunderscore )}
\end{itemize}
Relativo ao communismo.
Sectário do communismo.
\section{Communitário}
\begin{itemize}
\item {Grp. gram.:m.}
\end{itemize}
\begin{itemize}
\item {Utilização:Neol.}
\end{itemize}
\begin{itemize}
\item {Grp. gram.:Adj.}
\end{itemize}
\begin{itemize}
\item {Proveniência:(Do lat. \textunderscore communitas\textunderscore )}
\end{itemize}
O mesmo que \textunderscore communista\textunderscore .
Aquelle que á iniciativa individual prefere viver á custa do Estado, sem responsabilidades nem cuidados. Cf. \textunderscore Diário de Notícias\textunderscore  de 2-XII-900.
Diz-se da formação dos povos, em que prepondera o sentimento de communidade, como nas tríbos orientaes, em opposição a \textunderscore particularista\textunderscore .
\section{Communs}
\begin{itemize}
\item {Grp. gram.:m. pl.}
\end{itemize}
Membros da camara baixa do parlamento inglês, eleitos pelas povoações do reino.
(Chamam-se assim, por fazerem parte da \textunderscore house of commons\textunderscore , casa ou camara das \textunderscore communas\textunderscore )
\section{Commutação}
\begin{itemize}
\item {Grp. gram.:f.}
\end{itemize}
\begin{itemize}
\item {Utilização:Gram.}
\end{itemize}
\begin{itemize}
\item {Proveniência:(Lat. \textunderscore commutatio\textunderscore )}
\end{itemize}
Acto de commutar.
O mesmo que \textunderscore metáthese\textunderscore .
Permutação, substituição.
\section{Commutador}
\begin{itemize}
\item {Grp. gram.:m.  e  adj.}
\end{itemize}
\begin{itemize}
\item {Grp. gram.:M.}
\end{itemize}
\begin{itemize}
\item {Utilização:Phýs.}
\end{itemize}
O que commuta.
Apparelho para alterar a direcção das correntes eléctricas.
\section{Commutar}
\begin{itemize}
\item {Grp. gram.:v. t.}
\end{itemize}
\begin{itemize}
\item {Proveniência:(Lat. \textunderscore commutare\textunderscore )}
\end{itemize}
Permutar.
Substituir.
\section{Commutativo}
\begin{itemize}
\item {Grp. gram.:adj.}
\end{itemize}
\begin{itemize}
\item {Proveniência:(De \textunderscore commutar\textunderscore )}
\end{itemize}
Que commuta.
Relativo á troca.
\section{Commutável}
\begin{itemize}
\item {Grp. gram.:adj.}
\end{itemize}
\begin{itemize}
\item {Proveniência:(Lat. \textunderscore commutabilis\textunderscore )}
\end{itemize}
Que se póde commutar.
\section{Comnosco}
\begin{itemize}
\item {Grp. gram.:loc. pron.}
\end{itemize}
Em companhia de nós.
A nosso respeito.
(Flexão do pronome \textunderscore nós\textunderscore , precedido da prep. \textunderscore com\textunderscore )
\section{Como}
\begin{itemize}
\item {Grp. gram.:conj.}
\end{itemize}
\begin{itemize}
\item {Proveniência:(Do lat. \textunderscore quomodo\textunderscore )}
\end{itemize}
Da mesma fórma que: \textunderscore estás como eu\textunderscore .
De que modo: \textunderscore como queres que eu te attenda\textunderscore ?
\section{Como}
\begin{itemize}
\item {Grp. gram.:conj.}
\end{itemize}
\begin{itemize}
\item {Proveniência:(Do lat. \textunderscore quum\textunderscore )}
\end{itemize}
Logo que: \textunderscore como eu o lobriguei, corri para elle\textunderscore .
Porque: \textunderscore como era uma senhora, attendi-a\textunderscore .
\section{Comoção}
\begin{itemize}
\item {Grp. gram.:f.}
\end{itemize}
Acto ou efeito de comover.
Abalo.
Motim; revolta.
\section{Comocionar}
\begin{itemize}
\item {Proveniência:(De \textunderscore comoção\textunderscore )}
\end{itemize}
\textunderscore v. t.\textunderscore  (e der.)
O mesmo que \textunderscore comover\textunderscore , etc.
\section{Comocládia}
\begin{itemize}
\item {Grp. gram.:f.}
\end{itemize}
Arbusto terebintháceo das Antilhas.
\section{Cômoda}
\begin{itemize}
\item {Grp. gram.:f.}
\end{itemize}
\begin{itemize}
\item {Proveniência:(De \textunderscore cômodo\textunderscore )}
\end{itemize}
Espécie de mesa, com gavetas desde a base até á face superior.
\section{Comodamente}
\begin{itemize}
\item {Grp. gram.:adv.}
\end{itemize}
De modo cômodo.
Á vontade.
\section{Comodante}
\begin{itemize}
\item {Grp. gram.:m.  e  f.}
\end{itemize}
\begin{itemize}
\item {Proveniência:(Lat. \textunderscore commodans\textunderscore )}
\end{itemize}
Pessôa, que empresta gratuitamente objecto não fungível.
\section{Comodatário}
\begin{itemize}
\item {Grp. gram.:m.}
\end{itemize}
\begin{itemize}
\item {Proveniência:(Lat. \textunderscore commodatarius\textunderscore )}
\end{itemize}
Aquele que contrai comodato.
\section{Comodato}
\begin{itemize}
\item {Grp. gram.:m.}
\end{itemize}
\begin{itemize}
\item {Proveniência:(Lat. \textunderscore commodatum\textunderscore )}
\end{itemize}
Empréstimo gratuito de coisa não fungível.
\section{Comodidade}
\begin{itemize}
\item {Grp. gram.:f.}
\end{itemize}
\begin{itemize}
\item {Proveniência:(Lat. \textunderscore commoditas\textunderscore )}
\end{itemize}
Qualidade do que é cômodo.
Aquilo que é cômodo.
Oportunidade.
Bem-estar.
\section{Comodista}
\begin{itemize}
\item {Grp. gram.:m. ,  f.  e  adj.}
\end{itemize}
\begin{itemize}
\item {Utilização:Fam.}
\end{itemize}
\begin{itemize}
\item {Proveniência:(De \textunderscore cômmodo\textunderscore )}
\end{itemize}
Pessôa, que atende principalmente ás suas comodidades; egoísta.
\section{Cômodo}
\begin{itemize}
\item {Grp. gram.:adj.}
\end{itemize}
\begin{itemize}
\item {Grp. gram.:M.}
\end{itemize}
\begin{itemize}
\item {Utilização:Prov.}
\end{itemize}
\begin{itemize}
\item {Utilização:alent.}
\end{itemize}
\begin{itemize}
\item {Proveniência:(Lat. \textunderscore commodus\textunderscore )}
\end{itemize}
Útil.
Adequado.
Favorável.
Tranquilo.
O mesmo que \textunderscore comodidade\textunderscore : \textunderscore trabalha para seu cômodo\textunderscore .
Conjunto das herdades, que constituem uma lavoira.
Acomodação, emprêgo, (de pessôas).
Agasalho, hospitalidade.
\section{Comodoro}
\begin{itemize}
\item {Grp. gram.:m.}
\end{itemize}
\begin{itemize}
\item {Proveniência:(Ingl. \textunderscore commodore\textunderscore , do cast. \textunderscore comendador\textunderscore )}
\end{itemize}
Comandante de esquadra holandesa.
Oficial da marinha inglesa e americana, imediatamente superior ao capitão de mar e guerra.
Título honorífico em associações navaes, em Portugal e noutros países.
\section{Comoração}
\begin{itemize}
\item {Grp. gram.:f.}
\end{itemize}
\begin{itemize}
\item {Proveniência:(Lat. \textunderscore commoratio\textunderscore )}
\end{itemize}
Insistência de um orador em um ponto do seu discurso.
\section{Comorante}
\begin{itemize}
\item {Grp. gram.:adj.}
\end{itemize}
\begin{itemize}
\item {Proveniência:(Lat. \textunderscore commorans\textunderscore )}
\end{itemize}
Que comora.
\section{Comorar}
\begin{itemize}
\item {Grp. gram.:v. i.}
\end{itemize}
\begin{itemize}
\item {Proveniência:(Lat. \textunderscore commorari\textunderscore )}
\end{itemize}
O mesmo que \textunderscore coabitar\textunderscore .
\section{Comoriente}
\begin{itemize}
\item {Grp. gram.:adj.}
\end{itemize}
\begin{itemize}
\item {Proveniência:(Lat. \textunderscore commoriens\textunderscore )}
\end{itemize}
Que morre juntamente com outrem.
\section{Cômoro}
\begin{itemize}
\item {Grp. gram.:m.}
\end{itemize}
\begin{itemize}
\item {Proveniência:(Do lat. \textunderscore cumulus\textunderscore )}
\end{itemize}
Pequena elevação de terreno.
Socalco, botaréu.
Canteiro, alegrete.
\section{Comorôço}
\begin{itemize}
\item {Grp. gram.:m.}
\end{itemize}
\begin{itemize}
\item {Proveniência:(De \textunderscore cômoro\textunderscore )}
\end{itemize}
Montão, effeito de encomoroçar.
\section{Comoroiço}
\begin{itemize}
\item {Grp. gram.:m.}
\end{itemize}
\begin{itemize}
\item {Proveniência:(De \textunderscore cômoro\textunderscore )}
\end{itemize}
Montão, effeito de encomoroçar.
\section{Comorouço}
\begin{itemize}
\item {Grp. gram.:m.}
\end{itemize}
\begin{itemize}
\item {Proveniência:(De \textunderscore cômoro\textunderscore )}
\end{itemize}
Montão, effeito de encomoroçar.
\section{Comoso}
\begin{itemize}
\item {Grp. gram.:adj.}
\end{itemize}
O mesmo que \textunderscore comado\textunderscore .
\section{Comovedor}
\begin{itemize}
\item {Grp. gram.:adj.}
\end{itemize}
O mesmo que \textunderscore comovente\textunderscore .
\section{Comovente}
\begin{itemize}
\item {Grp. gram.:adj.}
\end{itemize}
\begin{itemize}
\item {Proveniência:(Lat. \textunderscore commovens\textunderscore )}
\end{itemize}
Que comove.
\section{Comover}
\begin{itemize}
\item {Grp. gram.:v. t.}
\end{itemize}
\begin{itemize}
\item {Grp. gram.:V. i.}
\end{itemize}
\begin{itemize}
\item {Proveniência:(Lat. \textunderscore commovere\textunderscore )}
\end{itemize}
Mover muito.
Deslocar.
Abalar; impressionar; enternecer.
Produzir impressão moral ou enternecimento.
\section{Compactamente}
\begin{itemize}
\item {Grp. gram.:adv.}
\end{itemize}
De modo compacto; de modo denso.
\section{Compacto}
\begin{itemize}
\item {Grp. gram.:adj.}
\end{itemize}
\begin{itemize}
\item {Proveniência:(Lat. \textunderscore compactus\textunderscore )}
\end{itemize}
Cujas partes componentes estão muito juntas.
Denso, espêsso, comprimido: \textunderscore multidão compacta\textunderscore .
\section{Compadecedor}
\begin{itemize}
\item {Grp. gram.:adj.}
\end{itemize}
\begin{itemize}
\item {Proveniência:(De \textunderscore compadecer\textunderscore )}
\end{itemize}
Que desperta compaixão.
Que se compadece.
\section{Compadecer}
\begin{itemize}
\item {Grp. gram.:v. t.}
\end{itemize}
\begin{itemize}
\item {Grp. gram.:V. p.}
\end{itemize}
\begin{itemize}
\item {Proveniência:(De \textunderscore com...\textunderscore  + \textunderscore padecer\textunderscore )}
\end{itemize}
Têr compaixão de.
Supportar.
Têr compaixão; commiserar-se.
Sêr compatível, harmonizar-se: \textunderscore êsse acto não se compadece com a lei\textunderscore .
\section{Compadecidamente}
\begin{itemize}
\item {Grp. gram.:adv.}
\end{itemize}
De modo compadecido.
\section{Compadecido}
\begin{itemize}
\item {Grp. gram.:adj.}
\end{itemize}
Que se compadece, que tem compaixão.
\section{Compadecimento}
\begin{itemize}
\item {Grp. gram.:m.}
\end{itemize}
Acto de compadecer-se.
\section{Compadrado}
\begin{itemize}
\item {Grp. gram.:m.}
\end{itemize}
O mesmo que \textunderscore compadrio\textunderscore .
\section{Compadrar}
\begin{itemize}
\item {Grp. gram.:v. t.}
\end{itemize}
Tornar compadre.
Tomar relações íntimas com.
\section{Compadre}
\begin{itemize}
\item {Grp. gram.:m.}
\end{itemize}
\begin{itemize}
\item {Proveniência:(Lat. \textunderscore compater\textunderscore )}
\end{itemize}
Padrinho de um neóphyto, em relação aos pais dêste.
Pai de um neóphyto, em relação ao padrinho ou madrinha dêste.
Amigo íntimo.
Cada uma das pessôas que entram num conlúio.
\section{Compadrego}
\begin{itemize}
\item {fónica:drê}
\end{itemize}
\begin{itemize}
\item {Grp. gram.:m.}
\end{itemize}
O mesmo que \textunderscore compadrio\textunderscore . Cf. Garrett, discurso parl. em 8-II-1840.
\section{Compadresco}
\begin{itemize}
\item {fónica:drês}
\end{itemize}
\begin{itemize}
\item {Grp. gram.:adj.}
\end{itemize}
\begin{itemize}
\item {Grp. gram.:M.}
\end{itemize}
\begin{itemize}
\item {Proveniência:(De \textunderscore compadre\textunderscore )}
\end{itemize}
Relativo ao parentesco ou ás relações de compadre.
Parentesco de compadres.
\section{Compadria}
\begin{itemize}
\item {Grp. gram.:f.}
\end{itemize}
O mesmo que \textunderscore compadrice\textunderscore .
\section{Compadrice}
\begin{itemize}
\item {Grp. gram.:f.}
\end{itemize}
O mesmo que \textunderscore compadrio\textunderscore .
\section{Compadrio}
\begin{itemize}
\item {Grp. gram.:m.}
\end{itemize}
\begin{itemize}
\item {Proveniência:(De \textunderscore compadre\textunderscore )}
\end{itemize}
Relações entre compadres.
Intimidade.
Patronato exaggerado ou contrário á justiça.
\section{Compaginação}
\begin{itemize}
\item {Grp. gram.:f.}
\end{itemize}
Acto ou effeito de compaginar^1.
\section{Compaginar}
\begin{itemize}
\item {Grp. gram.:v. t.}
\end{itemize}
\begin{itemize}
\item {Proveniência:(Lat. \textunderscore compaginare\textunderscore )}
\end{itemize}
Ligar intimamente.
\section{Compaginar}
\begin{itemize}
\item {Grp. gram.:v. t.}
\end{itemize}
\begin{itemize}
\item {Proveniência:(De \textunderscore com...\textunderscore  + \textunderscore paginar\textunderscore )}
\end{itemize}
Meter em página (composição a granel).
\section{Compaixão}
\begin{itemize}
\item {Grp. gram.:f.}
\end{itemize}
\begin{itemize}
\item {Proveniência:(Lat. \textunderscore compassio\textunderscore )}
\end{itemize}
Pesar, dôr, que em nós desperta o mal de outrem.
Pena, dó: \textunderscore tenha compaixão de mim\textunderscore .
\section{Compamu}
\begin{itemize}
\item {Grp. gram.:m.}
\end{itemize}
Tratamento, que os indígenas de San-Thomé dão aos amigos e companheiros de trabalho.
\section{Companha}
\begin{itemize}
\item {Grp. gram.:f.}
\end{itemize}
\begin{itemize}
\item {Utilização:Des.}
\end{itemize}
Tripulação de barco.
Aggremiação de pescadores.
O mesmo que \textunderscore companhia\textunderscore .
(B. lat. \textunderscore compania\textunderscore )
\section{Companhão}
\begin{itemize}
\item {Grp. gram.:m.}
\end{itemize}
\begin{itemize}
\item {Utilização:Ant.}
\end{itemize}
O mesmo que \textunderscore testículo\textunderscore . Cf. Cortesão, \textunderscore Subs.\textunderscore 
\section{Companheira}
\begin{itemize}
\item {Grp. gram.:f.}
\end{itemize}
\begin{itemize}
\item {Proveniência:(De \textunderscore companheiro\textunderscore )}
\end{itemize}
Mulher, que faz companhia.
Cada uma das mulheres, que trabalham, estudam ou vivem em commum.
Consorte, espôsa.
Concubina.
A fêmea, em relação ao macho.
Coisa que acompanha outra.
\section{Companheiro}
\begin{itemize}
\item {Grp. gram.:adj.}
\end{itemize}
\begin{itemize}
\item {Grp. gram.:M.}
\end{itemize}
\begin{itemize}
\item {Proveniência:(Do b. lat. \textunderscore companarius\textunderscore )}
\end{itemize}
Que acompanha.
Aquelle que acompanha.
Collega, camarada.
Graduação inferior no rito maçónico.
\section{Companhia}
\begin{itemize}
\item {Grp. gram.:f.}
\end{itemize}
\begin{itemize}
\item {Proveniência:(De \textunderscore companha\textunderscore )}
\end{itemize}
Acção de acompanhar.
Aquillo ou aquelle que acompanha.
Reunião de pessôas.
Reunião de pessôas, para um fim commum.
Convivência: \textunderscore fuja das más companhias\textunderscore .
Sociedade commercial, formada por accionistas.
Subdivisão de um regimento, sob o commando de um capitão.
\section{Companhôa}
\begin{itemize}
\item {Grp. gram.:f.}
\end{itemize}
\begin{itemize}
\item {Utilização:Ant.}
\end{itemize}
\begin{itemize}
\item {Proveniência:(De \textunderscore companhom\textunderscore )}
\end{itemize}
O mesmo que \textunderscore companheira\textunderscore .
\section{Companhom}
\begin{itemize}
\item {Grp. gram.:m.}
\end{itemize}
\begin{itemize}
\item {Utilização:Ant.}
\end{itemize}
\begin{itemize}
\item {Proveniência:(Fr. \textunderscore compagnon\textunderscore )}
\end{itemize}
O mesmo que \textunderscore companheiro\textunderscore :«\textunderscore emprazamos a vos Esteuam Annes comigo, nosso companhom.\textunderscore »Mss. de 1531, no archivo da Collegiada de Guimarães.
\section{Cômpar}
\begin{itemize}
\item {Grp. gram.:adj.}
\end{itemize}
\begin{itemize}
\item {Proveniência:(Lat. \textunderscore compar\textunderscore )}
\end{itemize}
Igual; que está a par.
\section{Comparabilidade}
\begin{itemize}
\item {Grp. gram.:f.}
\end{itemize}
\begin{itemize}
\item {Proveniência:(De \textunderscore comparábil\textunderscore , por \textunderscore comparável\textunderscore )}
\end{itemize}
Qualidade das coisas comparáveis entre si.
\section{Comparação}
\begin{itemize}
\item {Grp. gram.:f.}
\end{itemize}
\begin{itemize}
\item {Proveniência:(Lat. \textunderscore comparatio\textunderscore )}
\end{itemize}
Acto ou effeito de comparar.
\section{Comparador}
\begin{itemize}
\item {Grp. gram.:m.}
\end{itemize}
Aquelle que compara.
\section{Comparança}
\begin{itemize}
\item {Grp. gram.:f.}
\end{itemize}
\begin{itemize}
\item {Utilização:Pop.}
\end{itemize}
O mesmo que \textunderscore comparação\textunderscore . Cf. B. Moreno, \textunderscore Com. do Campo\textunderscore , II, 167.
\section{Comparar}
\begin{itemize}
\item {Grp. gram.:v. t.}
\end{itemize}
\begin{itemize}
\item {Proveniência:(Lat. \textunderscore comparare\textunderscore )}
\end{itemize}
Examinar ao mesmo tempo, para achar as semelhanças ou differenças.
Cotejar, confrontar.
Achar semelhante, igual.
\section{Comparativamente}
\begin{itemize}
\item {Grp. gram.:adv.}
\end{itemize}
De modo comparativo.
Fazendo comparação.
\section{Comparativo}
\begin{itemize}
\item {Grp. gram.:adj.}
\end{itemize}
\begin{itemize}
\item {Proveniência:(Lat. \textunderscore comparativus\textunderscore )}
\end{itemize}
Que serve para comparar.
Que emprega comparação.
\section{Comparável}
\begin{itemize}
\item {Grp. gram.:adj.}
\end{itemize}
\begin{itemize}
\item {Proveniência:(Lat. \textunderscore comparabilis\textunderscore )}
\end{itemize}
Que póde sêr comparado.
\section{Comparecência}
\begin{itemize}
\item {Grp. gram.:f.}
\end{itemize}
O mesmo que \textunderscore comparecimento\textunderscore .
\section{Comparecente}
\begin{itemize}
\item {Grp. gram.:adj.}
\end{itemize}
Que comparece.
\section{Comparecer}
\begin{itemize}
\item {Grp. gram.:v. i.}
\end{itemize}
\begin{itemize}
\item {Proveniência:(Do lat. \textunderscore comparere\textunderscore )}
\end{itemize}
Apparecer pessoalmente ou por meio de procurador.
\section{Comparecimento}
\begin{itemize}
\item {Grp. gram.:m.}
\end{itemize}
Acto de comparecer.
\section{Comparência}
\begin{itemize}
\item {Grp. gram.:f.}
\end{itemize}
O mesmo que \textunderscore comparecimento\textunderscore .
\section{Comparétia}
\begin{itemize}
\item {Grp. gram.:f.}
\end{itemize}
\begin{itemize}
\item {Proveniência:(De \textunderscore Comparetti\textunderscore , n. p.)}
\end{itemize}
Planta, da fam. das orchídeas.
\section{Comparéttia}
\begin{itemize}
\item {Grp. gram.:f.}
\end{itemize}
\begin{itemize}
\item {Proveniência:(De \textunderscore Comparetti\textunderscore , n. p.)}
\end{itemize}
Planta, da fam. das orchídeas.
\section{Comparochiano}
\begin{itemize}
\item {fónica:qui}
\end{itemize}
\begin{itemize}
\item {Grp. gram.:m.}
\end{itemize}
\begin{itemize}
\item {Proveniência:(De \textunderscore com...\textunderscore  + \textunderscore parochiano\textunderscore )}
\end{itemize}
Aquelle que, com outrem, pertence a uma paróchia. Cf. Júl. Dinis, \textunderscore Morgadinha\textunderscore , 367.
\section{Comparoquiano}
\begin{itemize}
\item {Grp. gram.:m.}
\end{itemize}
\begin{itemize}
\item {Proveniência:(De \textunderscore com...\textunderscore  + \textunderscore parochiano\textunderscore )}
\end{itemize}
Aquelle que, com outrem, pertence a uma paróchia. Cf. Júl. Dinis, \textunderscore Morgadinha\textunderscore , 367.
\section{Comparsa}
\begin{itemize}
\item {Grp. gram.:m.  e  f.}
\end{itemize}
\begin{itemize}
\item {Proveniência:(It. \textunderscore comparsa\textunderscore )}
\end{itemize}
Pessôa, que, entrando numa representação dramática, pouco ou nada tem que dizer.
\section{Comparsaria}
\begin{itemize}
\item {Grp. gram.:f.}
\end{itemize}
Conjunto de comparsas; os comparsas.
\section{Comparte}
\begin{itemize}
\item {Grp. gram.:m. ,  f.  e  adj.}
\end{itemize}
\begin{itemize}
\item {Proveniência:(Lat. \textunderscore compars\textunderscore )}
\end{itemize}
Pessôa, que toma parte, que participa.
\section{Compartilha}
\begin{itemize}
\item {Grp. gram.:f.}
\end{itemize}
Acto ou effeito de compartilhar.
\section{Compartilhar}
\begin{itemize}
\item {Grp. gram.:v. t.}
\end{itemize}
\begin{itemize}
\item {Proveniência:(De \textunderscore com...\textunderscore  + \textunderscore partilhar\textunderscore )}
\end{itemize}
Participar de; tomar parte em.
Partilhar com alguém.
\section{Compartimentagem}
\begin{itemize}
\item {Grp. gram.:f.}
\end{itemize}
\begin{itemize}
\item {Utilização:bras}
\end{itemize}
\begin{itemize}
\item {Utilização:Neol.}
\end{itemize}
Conjunto de compartimentos. Cf. \textunderscore Notícia\textunderscore , de 9-IX-903.
\section{Compartimento}
\begin{itemize}
\item {Grp. gram.:m.}
\end{itemize}
\begin{itemize}
\item {Proveniência:(De \textunderscore compartir\textunderscore )}
\end{itemize}
Cada uma das divisões de casa, gaveta, etc.
\section{Compartir}
\begin{itemize}
\item {Grp. gram.:v. t.}
\end{itemize}
\begin{itemize}
\item {Proveniência:(Lat. \textunderscore cumpartiri\textunderscore )}
\end{itemize}
Compartilhar.
Repartir.
Dividir em compartimentos.
\section{Compáscuo}
\begin{itemize}
\item {Grp. gram.:m.}
\end{itemize}
\begin{itemize}
\item {Proveniência:(Lat. \textunderscore cumpascuus\textunderscore )}
\end{itemize}
Pastagem commum.
\section{Compassadamente}
\begin{itemize}
\item {Grp. gram.:adv.}
\end{itemize}
\begin{itemize}
\item {Proveniência:(De \textunderscore compassar\textunderscore )}
\end{itemize}
De modo lento, vagaroso.
\section{Compassageiro}
\begin{itemize}
\item {Grp. gram.:m.}
\end{itemize}
\begin{itemize}
\item {Proveniência:(De \textunderscore com...\textunderscore  + \textunderscore passageiro\textunderscore )}
\end{itemize}
Aquelle que jornadeia ou viaja com outra ou outras pessôas.
\section{Compassar}
\begin{itemize}
\item {Grp. gram.:v. t.}
\end{itemize}
\begin{itemize}
\item {Utilização:Náut.}
\end{itemize}
Medir a compasso.
Calcular.
Fazer que se mova lentamente, tornar vagaroso.
Proporcionar.
Moderar.
Collocar convenientemente (a vêrga); dispor em equilíbrio (o velame).
\section{Compassivamente}
\begin{itemize}
\item {Grp. gram.:adv.}
\end{itemize}
De modo compassivo.
\section{Compassível}
\begin{itemize}
\item {Grp. gram.:adj.}
\end{itemize}
\begin{itemize}
\item {Proveniência:(Lat. \textunderscore compassibilis\textunderscore )}
\end{itemize}
Que facilmente se compadece.
\section{Compassividade}
\begin{itemize}
\item {Grp. gram.:f.}
\end{itemize}
Qualidade de compassivo.
\section{Compassivo}
\begin{itemize}
\item {Grp. gram.:adj.}
\end{itemize}
\begin{itemize}
\item {Proveniência:(Do lat. \textunderscore compassus\textunderscore )}
\end{itemize}
Que tem compaixão.
Que revela compaixão: \textunderscore palavras compassivas\textunderscore .
\section{Compasso}
\begin{itemize}
\item {Grp. gram.:m.}
\end{itemize}
\begin{itemize}
\item {Proveniência:(De \textunderscore com...\textunderscore  + \textunderscore passo\textunderscore )}
\end{itemize}
Instrumento de metal ou madeira, ou de uma e outra coisa, composto de dois braços ou pernas, que se abrem e se fecham, servindo para traçar círculos ou tirar medidas.
Medida de tempo, na música.
Movimento regulado.
Medida, regra.
Constellação meridional.
\section{Compatibilidade}
\begin{itemize}
\item {Grp. gram.:f.}
\end{itemize}
Qualidade do que é compatível.
\section{Compatível}
\begin{itemize}
\item {Grp. gram.:adj.}
\end{itemize}
\begin{itemize}
\item {Proveniência:(Do lat. \textunderscore compati\textunderscore )}
\end{itemize}
Que póde coexistir.
Que é conciliável com outro ou com outros.
\section{Compativelmente}
\begin{itemize}
\item {Grp. gram.:adv.}
\end{itemize}
De modo compatível.
\section{Compatrício}
\begin{itemize}
\item {Grp. gram.:m.}
\end{itemize}
\begin{itemize}
\item {Proveniência:(De \textunderscore com...\textunderscore  + \textunderscore patrício\textunderscore )}
\end{itemize}
Aquelle que, em relação a outrem, é da mesma nação, da mesma província ou da mesma terra.
\section{Compatriota}
\begin{itemize}
\item {Grp. gram.:m.  e  adj.}
\end{itemize}
\begin{itemize}
\item {Proveniência:(Lat. \textunderscore compatriota\textunderscore )}
\end{itemize}
Diz-se dos indivíduos que têm a mesma pátria.
\section{Compeçar}
\begin{itemize}
\item {Grp. gram.:v. t.}
\end{itemize}
\begin{itemize}
\item {Utilização:Açor}
\end{itemize}
\begin{itemize}
\item {Utilização:ant.}
\end{itemize}
O mesmo que \textunderscore começar\textunderscore . Cf. \textunderscore Port. Mon. Hist., Script.\textunderscore , 234.
\section{Compecilho}
\begin{itemize}
\item {Grp. gram.:m.}
\end{itemize}
\begin{itemize}
\item {Utilização:Ant.}
\end{itemize}
\begin{itemize}
\item {Proveniência:(De \textunderscore compêço\textunderscore )}
\end{itemize}
Delineamento; plano rudimentar.
\section{Compêço}
\begin{itemize}
\item {Grp. gram.:m.}
\end{itemize}
\begin{itemize}
\item {Utilização:Ant.}
\end{itemize}
O mesmo que \textunderscore comêço\textunderscore .
\section{Compegar}
\begin{itemize}
\item {Grp. gram.:v. i.}
\end{itemize}
\begin{itemize}
\item {Utilização:Ant.}
\end{itemize}
Comer pão com (alguma coisa). Cf. Fernão Oliveira, \textunderscore Gram.\textunderscore , c. XXXVI.
(Cp. \textunderscore apeguilhar\textunderscore )
\section{Compelação}
\begin{itemize}
\item {Grp. gram.:f.}
\end{itemize}
\begin{itemize}
\item {Utilização:Jur.}
\end{itemize}
\begin{itemize}
\item {Proveniência:(Lat. \textunderscore compellatio\textunderscore )}
\end{itemize}
Acto de chamar a juizo, de accusar.
\section{Compelir}
\begin{itemize}
\item {Grp. gram.:v. t.}
\end{itemize}
\begin{itemize}
\item {Proveniência:(Lat. \textunderscore compellere\textunderscore )}
\end{itemize}
Constranger; forçar; obrigar.
Empurrar; impellir.
\section{Compellação}
\begin{itemize}
\item {Grp. gram.:f.}
\end{itemize}
\begin{itemize}
\item {Utilização:Jur.}
\end{itemize}
\begin{itemize}
\item {Proveniência:(Lat. \textunderscore compellatio\textunderscore )}
\end{itemize}
Acto de chamar a juizo, de accusar.
\section{Compellir}
\begin{itemize}
\item {Grp. gram.:v. t.}
\end{itemize}
\begin{itemize}
\item {Proveniência:(Lat. \textunderscore compellere\textunderscore )}
\end{itemize}
Constranger; forçar; obrigar.
Empurrar; impellir.
\section{Compendiador}
\begin{itemize}
\item {Grp. gram.:m.}
\end{itemize}
Aquelle que compendia.
\section{Compendiar}
\begin{itemize}
\item {Grp. gram.:v. t.}
\end{itemize}
\begin{itemize}
\item {Proveniência:(Lat. \textunderscore compendiare\textunderscore )}
\end{itemize}
Reduzir a compêndio.
Resumir; synthetizar.
\section{Compêndio}
\begin{itemize}
\item {Grp. gram.:m.}
\end{itemize}
\begin{itemize}
\item {Proveniência:(Lat. \textunderscore compendium\textunderscore )}
\end{itemize}
Resumo; sýnthese.
Obra, que serve de texto nas escolas.
\section{Compendíolo}
\begin{itemize}
\item {Grp. gram.:m.}
\end{itemize}
Pequeno compêndio.
\section{Compendiosamente}
\begin{itemize}
\item {Grp. gram.:adv.}
\end{itemize}
De modo compendioso.
\section{Compendioso}
\begin{itemize}
\item {Grp. gram.:adj.}
\end{itemize}
\begin{itemize}
\item {Proveniência:(Lat. \textunderscore compendiosus\textunderscore )}
\end{itemize}
Resumido.
Que tem a fórma de compêndio.
\section{Compenetração}
\begin{itemize}
\item {Grp. gram.:f.}
\end{itemize}
Acto de compenetrar.
\section{Compenetrar}
\begin{itemize}
\item {Grp. gram.:v. t.}
\end{itemize}
\begin{itemize}
\item {Proveniência:(De \textunderscore com...\textunderscore  + \textunderscore penetrar\textunderscore )}
\end{itemize}
Levar ao íntimo de.
Fazer convencer profundamente.
\section{Compensação}
\begin{itemize}
\item {Grp. gram.:f.}
\end{itemize}
Acto ou effeito de compensar.
\section{Compensador}
\begin{itemize}
\item {Grp. gram.:m.  e  adj.}
\end{itemize}
O que compensa.
\section{Compensar}
\begin{itemize}
\item {Grp. gram.:v. t.}
\end{itemize}
\begin{itemize}
\item {Proveniência:(Lat. \textunderscore compensare\textunderscore )}
\end{itemize}
Contrabalançar.
Estabelecer equilíbrio entre.
Indemnizar.
Substituir.
\section{Compensativo}
\begin{itemize}
\item {Grp. gram.:adj.}
\end{itemize}
\begin{itemize}
\item {Proveniência:(Lat. \textunderscore compensativus\textunderscore )}
\end{itemize}
Que serve para compensar.
\section{Compensatório}
\begin{itemize}
\item {Grp. gram.:adj.}
\end{itemize}
\begin{itemize}
\item {Proveniência:(Do lat. \textunderscore compensatus\textunderscore )}
\end{itemize}
Que envolve compensação.
\section{Compensável}
\begin{itemize}
\item {Grp. gram.:adj.}
\end{itemize}
\begin{itemize}
\item {Proveniência:(De \textunderscore compensar\textunderscore )}
\end{itemize}
Que póde ou deve sêr compensado.
\section{Comperto}
\begin{itemize}
\item {Grp. gram.:adj.}
\end{itemize}
Descoberto, patente. Cf. Cortesão, \textunderscore Subs.\textunderscore 
\section{Competência}
\begin{itemize}
\item {Grp. gram.:f.}
\end{itemize}
\begin{itemize}
\item {Proveniência:(Lat. \textunderscore competentia\textunderscore )}
\end{itemize}
Faculdade legal, que um funccionário ou um tribunal tem, de apreciar e julgar um pleito ou questão.
Qualidade de quem é capaz de apreciar e resolver qualquer assumpto.
Aptidão, idoneidade: \textunderscore homem de grande competência\textunderscore .
\section{Competente}
\begin{itemize}
\item {Grp. gram.:adj.}
\end{itemize}
\begin{itemize}
\item {Proveniência:(Lat. \textunderscore competens\textunderscore )}
\end{itemize}
Que tem competência.
Que tem habilidade, aptidão.
Sufficiente.
Legal: \textunderscore têr a idade competente para casar\textunderscore .
Que é devido.
Que é permittido por lei.
\section{Competentemente}
\begin{itemize}
\item {Grp. gram.:adv.}
\end{itemize}
De modo competente.
\section{Competição}
\begin{itemize}
\item {Grp. gram.:f.}
\end{itemize}
Acto de competir.
\section{Competidor}
\begin{itemize}
\item {Grp. gram.:m.  e  adj.}
\end{itemize}
\begin{itemize}
\item {Proveniência:(Lat. \textunderscore competitor\textunderscore )}
\end{itemize}
O que compete.
O que com outrem pretende ou pleiteia simultaneamente uma coisa.
Rival; antagonista.
\section{Competir}
\begin{itemize}
\item {Grp. gram.:v. i.}
\end{itemize}
\begin{itemize}
\item {Proveniência:(Lat. \textunderscore competere\textunderscore )}
\end{itemize}
Pretender alguma coisa, simultaneamente com outrem.
Rívalizar.
Impender, sêr próprio das attribuições de alguém.
\section{Compilação}
\begin{itemize}
\item {Grp. gram.:f.}
\end{itemize}
Acto ou effeito de compilar.
\section{Compilador}
\begin{itemize}
\item {Grp. gram.:m.}
\end{itemize}
Aquelle que compila.
\section{Compilar}
\begin{itemize}
\item {Grp. gram.:v. t.}
\end{itemize}
\begin{itemize}
\item {Proveniência:(Lat. \textunderscore compilare\textunderscore )}
\end{itemize}
Colligir, reunir, (falando-se de documentos, leis ou outros escritos de vária procedência ou natureza)
\section{Compilatório}
\begin{itemize}
\item {Grp. gram.:adj.}
\end{itemize}
Relativo a compilação.
\section{Compita}
\begin{itemize}
\item {Grp. gram.:f.}
\end{itemize}
\begin{itemize}
\item {Proveniência:(Do rad. de \textunderscore competir\textunderscore )}
\end{itemize}
Us. na loc. adv. \textunderscore á compita\textunderscore , que é o mesmo que \textunderscore á porfia\textunderscore , com rivalidade. Cf. Camillo, \textunderscore Noites de Insómnia\textunderscore , I, 8; II, 26.
\section{Compitaes}
\begin{itemize}
\item {Grp. gram.:m. pl.}
\end{itemize}
\begin{itemize}
\item {Proveniência:(Lat. \textunderscore compitalia\textunderscore )}
\end{itemize}
Festas romanas, em honra dos deuses Lares.
\section{Compitais}
\begin{itemize}
\item {Grp. gram.:m. pl.}
\end{itemize}
\begin{itemize}
\item {Proveniência:(Lat. \textunderscore compitalia\textunderscore )}
\end{itemize}
Festas romanas, em honra dos deuses Lares.
\section{Cômpito}
\begin{itemize}
\item {Grp. gram.:m.}
\end{itemize}
\begin{itemize}
\item {Utilização:Prov.}
\end{itemize}
Medida, padrão.
(Por \textunderscore cômputo\textunderscore )
\section{Cômpito}
\begin{itemize}
\item {Grp. gram.:m.}
\end{itemize}
Encruzilhada.
Ponto, onde desembocam vários caminhos. Cf. Castilho, \textunderscore Fastos\textunderscore , III, 570.
\section{Complacência}
\begin{itemize}
\item {Grp. gram.:f.}
\end{itemize}
\begin{itemize}
\item {Proveniência:(De \textunderscore complacente\textunderscore )}
\end{itemize}
Desejo ou acto de comprazer.
\section{Complacente}
\begin{itemize}
\item {Grp. gram.:adj.}
\end{itemize}
\begin{itemize}
\item {Proveniência:(Lat. \textunderscore complacens\textunderscore )}
\end{itemize}
Que tem complacência.
Em que há complacência.
\section{Complacentemente}
\begin{itemize}
\item {Grp. gram.:adv.}
\end{itemize}
De modo complacente.
\section{Complanar}
\begin{itemize}
\item {Grp. gram.:v. t.}
\end{itemize}
\begin{itemize}
\item {Proveniência:(Lat. \textunderscore complanare\textunderscore )}
\end{itemize}
Tornar plano; nivelar.
Estender planamente.
\section{Complectível}
\begin{itemize}
\item {Grp. gram.:adj.}
\end{itemize}
\begin{itemize}
\item {Proveniência:(Lat. \textunderscore complectibilis\textunderscore )}
\end{itemize}
Que póde sêr abrangido.
\section{Complectivo}
\begin{itemize}
\item {Grp. gram.:adj.}
\end{itemize}
\begin{itemize}
\item {Proveniência:(Do lat. \textunderscore complecti\textunderscore )}
\end{itemize}
Que abrange, cobre ou abraça.
\section{Compleicionado}
\begin{itemize}
\item {Grp. gram.:adj.}
\end{itemize}
Que tem certa compleição.
\section{Compleicional}
\begin{itemize}
\item {Grp. gram.:adj.}
\end{itemize}
Relativo a compleição.
\section{Compleição}
\begin{itemize}
\item {Grp. gram.:f.}
\end{itemize}
Organização phýsica de alguém.
Temperamento.
Disposição de espirito.
(Derivação irregular do lat. \textunderscore complexio\textunderscore )
\section{Compleiçoado}
\begin{itemize}
\item {Grp. gram.:adj.}
\end{itemize}
Que tem bôa ou má compleição:«\textunderscore no mal compleiçoado bisneto de...\textunderscore »Camillo, \textunderscore Caveira\textunderscore , 165.
\section{Complementar}
\begin{itemize}
\item {Grp. gram.:adj.}
\end{itemize}
Relativo a complemento.
Que serve de complemento.
\section{Complementar}
\begin{itemize}
\item {Grp. gram.:v. t.}
\end{itemize}
\begin{itemize}
\item {Proveniência:(De \textunderscore complemento\textunderscore )}
\end{itemize}
O mesmo que \textunderscore completar\textunderscore .
\section{Complemento}
\begin{itemize}
\item {Grp. gram.:m.}
\end{itemize}
\begin{itemize}
\item {Utilização:Mathem.}
\end{itemize}
\begin{itemize}
\item {Proveniência:(Lat. \textunderscore complementum\textunderscore )}
\end{itemize}
Aquillo que completa.
Acto de completar; remate.
Diz-se de qualquer parte que, junta a outra, fórma uma unidade natural ou artificial.
\textunderscore Complemento arithmético\textunderscore , número, que exprime a differença entre outro e a unidade de ordem immediatamente superior.--Assim, 4 é o complemento de 6, porque 10 é a unidade immediatamente superior.
\section{Complente}
\begin{itemize}
\item {Grp. gram.:adj.}
\end{itemize}
\begin{itemize}
\item {Utilização:Ant.}
\end{itemize}
\begin{itemize}
\item {Proveniência:(Lat. \textunderscore complens\textunderscore )}
\end{itemize}
Dizia-se da maré cheia.
\section{Completação}
\begin{itemize}
\item {Grp. gram.:f.}
\end{itemize}
Acto de completar. Cf. Castilho, \textunderscore Fastos\textunderscore , I, p. XLVII.
\section{Completador}
\begin{itemize}
\item {Grp. gram.:adj.}
\end{itemize}
Que completa. Cf. Eça, \textunderscore P. Basílio\textunderscore , 61.
\section{Completamente}
\begin{itemize}
\item {Grp. gram.:adv.}
\end{itemize}
De modo completo.
\section{Completamento}
\begin{itemize}
\item {Grp. gram.:m.}
\end{itemize}
\begin{itemize}
\item {Utilização:Des.}
\end{itemize}
Acto de completar.
\section{Completar}
\begin{itemize}
\item {Grp. gram.:v. t.}
\end{itemize}
Fazer completo; preencher: \textunderscore completar 50 annos\textunderscore .
Rematar; concluir.
\section{Completas}
\begin{itemize}
\item {Grp. gram.:f. pl.}
\end{itemize}
\begin{itemize}
\item {Proveniência:(De \textunderscore completo\textunderscore )}
\end{itemize}
Últimas horas canónicas dos officios litúrgicos.
\section{Completivo}
\begin{itemize}
\item {Grp. gram.:adj.}
\end{itemize}
\begin{itemize}
\item {Proveniência:(De \textunderscore completo\textunderscore )}
\end{itemize}
Que completa.
Que serve de complemento.
\section{Completo}
\begin{itemize}
\item {Grp. gram.:adj.}
\end{itemize}
\begin{itemize}
\item {Grp. gram.:M.}
\end{itemize}
\begin{itemize}
\item {Proveniência:(Lat. \textunderscore completus\textunderscore )}
\end{itemize}
A que não falta nada do que póde ou deve têr.
Preenchido.
Concluido.
Total.
Inteiro.
Perfeito.
Aquillo que está completo, perfeito.
\section{Completório}
\begin{itemize}
\item {Grp. gram.:m.}
\end{itemize}
O mesmo que \textunderscore completas\textunderscore .
\section{Complexão}
\begin{itemize}
\item {Grp. gram.:f.}
\end{itemize}
\begin{itemize}
\item {Proveniência:(Lat. \textunderscore complexio\textunderscore )}
\end{itemize}
Conjunto.
União.
\section{Complexidade}
\begin{itemize}
\item {Grp. gram.:f.}
\end{itemize}
Qualidade do que é complexo.
\section{Complexidão}
\begin{itemize}
\item {Grp. gram.:f.}
\end{itemize}
O mesmo que \textunderscore complexidade\textunderscore . Cf. A. Cândido, \textunderscore Phil. Pol.\textunderscore , 43 e 68.
\section{Complexo}
\begin{itemize}
\item {Grp. gram.:adj.}
\end{itemize}
\begin{itemize}
\item {Grp. gram.:M.}
\end{itemize}
\begin{itemize}
\item {Proveniência:(Lat. \textunderscore complexus\textunderscore )}
\end{itemize}
Que encerra ou abrange muitos elementos ou partes.
Que póde sêr observado sob vários pontos de vista.
Acção de abranger.
Conjunto de muitas coisas, circunstâncias ou actos, que entre si têm qualquer ligação.
\section{Complicação}
\begin{itemize}
\item {Grp. gram.:f.}
\end{itemize}
Acto e effeito de complicar.
\section{Complicadamente}
\begin{itemize}
\item {Grp. gram.:adv.}
\end{itemize}
De modo complicado.
\section{Complicado}
\begin{itemize}
\item {Grp. gram.:adj.}
\end{itemize}
Em que há complicação.
Embaraçado, diffícil.
\section{Complicador}
\begin{itemize}
\item {Grp. gram.:adj.}
\end{itemize}
Que complica.
\section{Complicar}
\begin{itemize}
\item {Grp. gram.:v. t.}
\end{itemize}
\begin{itemize}
\item {Proveniência:(Lat. \textunderscore complicare\textunderscore )}
\end{itemize}
Reunir (coisas de differente natureza).
Embaraçar, tornar intrincado, confuso, difficil de comprehender.
Difficultar a solução de: \textunderscore complicar um problema\textunderscore .
\section{Cômplice}
\begin{itemize}
\item {Grp. gram.:m. ,  f.  e  adj.}
\end{itemize}
O mesmo que \textunderscore cúmplice\textunderscore :«\textunderscore hão de entrar como cômplices do mesmo crime\textunderscore ». M. Bernárdez, \textunderscore Luz e Calor\textunderscore .
\section{Complutense}
\begin{itemize}
\item {Grp. gram.:adj.}
\end{itemize}
Relativo á antiga cidadede Compluto, (hoje Alcalá de Henarez). Cf. Herculano, \textunderscore Hist. de Port.\textunderscore , I, 457.
\section{Complúvio}
\begin{itemize}
\item {Grp. gram.:m.}
\end{itemize}
\begin{itemize}
\item {Utilização:Ant.}
\end{itemize}
\begin{itemize}
\item {Proveniência:(Lat. \textunderscore compluvium\textunderscore )}
\end{itemize}
Cisterna no pátio interno das casas, para receber a água da chuva.
\section{Compoedor}
\begin{itemize}
\item {fónica:po-e}
\end{itemize}
\begin{itemize}
\item {Grp. gram.:m.}
\end{itemize}
\begin{itemize}
\item {Utilização:Ant.}
\end{itemize}
\begin{itemize}
\item {Utilização:Ant.}
\end{itemize}
\begin{itemize}
\item {Proveniência:(De \textunderscore compoer\textunderscore )}
\end{itemize}
Árbitro; avindor.
Compositor typográphico.
\section{Compoer}
\begin{itemize}
\item {Grp. gram.:v. t.}
\end{itemize}
\begin{itemize}
\item {Utilização:Ant.}
\end{itemize}
\begin{itemize}
\item {Proveniência:(Do lat. \textunderscore componere\textunderscore )}
\end{itemize}
O mesmo que \textunderscore compor\textunderscore .
\section{Componedor}
\begin{itemize}
\item {Grp. gram.:m.}
\end{itemize}
\begin{itemize}
\item {Proveniência:(Do lat. \textunderscore componere\textunderscore )}
\end{itemize}
Utensílio, sôbre que o typógrapho vai alinhando os caracteres typográphicos, que tira dos compartimentos da caixa para formar as palavras.
\section{Componenda}
\begin{itemize}
\item {Grp. gram.:f.}
\end{itemize}
\begin{itemize}
\item {Proveniência:(Do lat. \textunderscore componendus\textunderscore )}
\end{itemize}
Convenção, que se faz na dataria, sôbre o que se há de pagar por certas concessões.
\section{Componente}
\begin{itemize}
\item {Grp. gram.:m.  e  adj.}
\end{itemize}
\begin{itemize}
\item {Proveniência:(Lat. \textunderscore componens\textunderscore )}
\end{itemize}
Aquillo que compõe ou entra na composição de alguma coisa.
\section{Componer}
\begin{itemize}
\item {Grp. gram.:v. t.}
\end{itemize}
\begin{itemize}
\item {Utilização:Ant.}
\end{itemize}
O mesmo que \textunderscore compor\textunderscore .
\section{Componista}
\begin{itemize}
\item {Grp. gram.:m.}
\end{itemize}
\begin{itemize}
\item {Utilização:bras}
\end{itemize}
\begin{itemize}
\item {Utilização:Neol.}
\end{itemize}
Aquelle que compõe música:«\textunderscore Carlos Gomes, o grande componista\textunderscore ». S. Romero, \textunderscore M. Assis\textunderscore , 7.
\section{Compónium}
\begin{itemize}
\item {Grp. gram.:m.}
\end{itemize}
\begin{itemize}
\item {Proveniência:(Do lat. \textunderscore componere\textunderscore )}
\end{itemize}
Espécie de órgão com cylindro, que, pelo seu próprio mecanismo, varia as peças que se desejam ouvir.
\section{Componível}
\begin{itemize}
\item {Grp. gram.:adj.}
\end{itemize}
Que se póde compor.
(Cp. lat. \textunderscore componere\textunderscore )
\section{Compor}
\begin{itemize}
\item {Grp. gram.:v. t.}
\end{itemize}
\begin{itemize}
\item {Proveniência:(Lat. \textunderscore componere\textunderscore )}
\end{itemize}
Formar de várias coisas: \textunderscore compor um ramalhete\textunderscore .
Entrar na composição de, fazer parte de.
Dispor os caracteres typográphicos, com que se imprime (um livro, um jornal, etc.).
Arranjar; consertar: \textunderscore compor umas botas\textunderscore .
Produzir; escrever: \textunderscore compor um poema\textunderscore .
Coordenar.
Alinhar.
Ajustar, aconchegar: \textunderscore compor o chaile ao pescoço\textunderscore .
Imaginar.
Harmonizar: \textunderscore compor as partes litigantes\textunderscore .
Melhorar.
Aparentar.
Tornar tranquillo.
\section{Comporta}
\begin{itemize}
\item {Grp. gram.:f.}
\end{itemize}
\begin{itemize}
\item {Utilização:Prov.}
\end{itemize}
\begin{itemize}
\item {Proveniência:(De \textunderscore com...\textunderscore  + \textunderscore porta\textunderscore )}
\end{itemize}
Porta, que sustém águas de dique ou açude, e que póde abrir-se para as deixar correr.
Portinhola do lagar de vinho. (Colhido na Bairrada)
\section{Comporta}
\begin{itemize}
\item {Grp. gram.:f.}
\end{itemize}
Dança popular do século XVIII.
\section{Comportado}
\begin{itemize}
\item {Grp. gram.:adj.}
\end{itemize}
\begin{itemize}
\item {Proveniência:(De \textunderscore comportar-se\textunderscore )}
\end{itemize}
Us. na loc. \textunderscore bem\textunderscore  ou \textunderscore mal comportado\textunderscore , que procede bem ou mal.
\section{Comportamento}
\begin{itemize}
\item {Grp. gram.:m.}
\end{itemize}
Modo de comportar-se.
\section{Comportar}
\begin{itemize}
\item {Grp. gram.:v. t.}
\end{itemize}
\begin{itemize}
\item {Grp. gram.:V. p.}
\end{itemize}
\begin{itemize}
\item {Proveniência:(Lat. \textunderscore comportare\textunderscore )}
\end{itemize}
Supportar, soffrer.
Proceder, portar-se.
\section{Comportas}
\begin{itemize}
\item {Grp. gram.:f. pl.}
\end{itemize}
\begin{itemize}
\item {Utilização:Bras}
\end{itemize}
\begin{itemize}
\item {Proveniência:(De \textunderscore comportar-se\textunderscore ?)}
\end{itemize}
Artifício, com que alguém se insinua no animo de outrem; lábia.
\section{Comportável}
\begin{itemize}
\item {Grp. gram.:adj.}
\end{itemize}
Que se póde comportar.
\section{Composição}
\begin{itemize}
\item {Grp. gram.:f.}
\end{itemize}
\begin{itemize}
\item {Proveniência:(Lat. \textunderscore compositio\textunderscore )}
\end{itemize}
Acto de compor: \textunderscore composição de um livro\textunderscore .
Organização.
Coisa composta.
Agrupamento de moléculas.
Proporção entre as partes, que constituem um todo.
Reunião de palavras em orações e destas em discurso.
Arte de escrever música original.
Acôrdo, conciliação: \textunderscore antes má composição, do que bôa demanda\textunderscore .
\section{Compósita}
\begin{itemize}
\item {Grp. gram.:adj.}
\end{itemize}
\begin{itemize}
\item {Proveniência:(Do lat. \textunderscore compositus\textunderscore )}
\end{itemize}
Diz-se de uma ordem de Architectura, em que entram elementos das ordens jónica e corínthia.
\section{Compositeiro}
\begin{itemize}
\item {Grp. gram.:m.}
\end{itemize}
\begin{itemize}
\item {Utilização:Deprec.}
\end{itemize}
Compositor. Cf. Filinto, I, 42.
\section{Compositivo}
\begin{itemize}
\item {Grp. gram.:adj.}
\end{itemize}
\begin{itemize}
\item {Proveniência:(Do lat. \textunderscore compositus\textunderscore )}
\end{itemize}
Relativo á composição.
\section{Compositor}
\begin{itemize}
\item {Grp. gram.:m.}
\end{itemize}
\begin{itemize}
\item {Utilização:Bras. do S}
\end{itemize}
\begin{itemize}
\item {Proveniência:(Do lat. \textunderscore compositus\textunderscore )}
\end{itemize}
Aquelle que compõe.
Typógrapho.
Aquelle que trata de cavallos, preparando-os para corridas.
\section{Compossessor}
\begin{itemize}
\item {Grp. gram.:m.}
\end{itemize}
\begin{itemize}
\item {Utilização:Jur.}
\end{itemize}
\begin{itemize}
\item {Proveniência:(De \textunderscore com...\textunderscore  + \textunderscore possessor\textunderscore )}
\end{itemize}
Aquelle que, com outrem, possue alguma coisa ou direito.
\section{Compossível}
\begin{itemize}
\item {Grp. gram.:adj.}
\end{itemize}
\begin{itemize}
\item {Utilização:Des.}
\end{itemize}
Compatível. Cf. Cortesão, \textunderscore Subs.\textunderscore 
\section{Compostamente}
\begin{itemize}
\item {Grp. gram.:adv.}
\end{itemize}
De modo composto.
Com discrição.
\section{Compostas}
\begin{itemize}
\item {Grp. gram.:f. pl.}
\end{itemize}
\begin{itemize}
\item {Utilização:Bot.}
\end{itemize}
\begin{itemize}
\item {Proveniência:(De \textunderscore composto\textunderscore )}
\end{itemize}
Família de plantas dicotyledóneas, que comprehende os gêneros mais vulgares em todo o mundo, e cujas flôres se reunem sôbre um receptáculo, cercadas de envoltório commum.
\section{Compostelano}
\begin{itemize}
\item {Grp. gram.:m.  e  adj.}
\end{itemize}
O que é de Compostela. Cf. Herculano, \textunderscore Hist. de Port.\textunderscore , I, 253 e 264.
\section{Composto}
\begin{itemize}
\item {Grp. gram.:adj.}
\end{itemize}
\begin{itemize}
\item {Utilização:Fig.}
\end{itemize}
\begin{itemize}
\item {Grp. gram.:M.}
\end{itemize}
\begin{itemize}
\item {Proveniência:(Lat. \textunderscore compositus\textunderscore )}
\end{itemize}
Modesto, sério.
Substância ou corpo composto.
Complexo de várias coisas combinadas.
\section{Compostura}
\begin{itemize}
\item {Grp. gram.:f.}
\end{itemize}
\begin{itemize}
\item {Utilização:Bras. do S}
\end{itemize}
\begin{itemize}
\item {Utilização:Mús.}
\end{itemize}
\begin{itemize}
\item {Utilização:ant.}
\end{itemize}
\begin{itemize}
\item {Grp. gram.:Pl.}
\end{itemize}
\begin{itemize}
\item {Proveniência:(De \textunderscore composto\textunderscore )}
\end{itemize}
Composição; consêrto.
Falsificação.
Acto ou effeito de preparar o cavallo para corridas; tempo que nisso se gasta.
Acto de harmonizar ou adaptar um canto para differentes vozes.
Artifícios, cosméticos.
\section{Compota}
\begin{itemize}
\item {Grp. gram.:f.}
\end{itemize}
\begin{itemize}
\item {Proveniência:(Fr. \textunderscore compote\textunderscore )}
\end{itemize}
Doce de fruta, cozida em água e açúcar.
\section{Compoteira}
\begin{itemize}
\item {Grp. gram.:f.}
\end{itemize}
\begin{itemize}
\item {Proveniência:(De \textunderscore compota\textunderscore )}
\end{itemize}
Vaso, em que se guarda a compota, ou que serve para isso.
\section{Compra}
\begin{itemize}
\item {Grp. gram.:f.}
\end{itemize}
Acto de comprar.
Coisa comprada.
Acção de tirar do baralho certo número de cartas, quando se joga.
\section{Compradia}
\begin{itemize}
\item {Grp. gram.:f.}
\end{itemize}
\begin{itemize}
\item {Utilização:Ant.}
\end{itemize}
O mesmo que \textunderscore compra\textunderscore .
\section{Compradiço}
\begin{itemize}
\item {Grp. gram.:adj.}
\end{itemize}
\begin{itemize}
\item {Utilização:P. us.}
\end{itemize}
O mesmo que \textunderscore comprável\textunderscore .
\section{Comprador}
\begin{itemize}
\item {Grp. gram.:m.}
\end{itemize}
Aquelle que compra.
\section{Comprar}
\begin{itemize}
\item {Grp. gram.:v. t.}
\end{itemize}
\begin{itemize}
\item {Proveniência:(Do lat. \textunderscore comparare\textunderscore )}
\end{itemize}
Adquirir por dinheiro.
Peitar; subornar: \textunderscore comprar um juiz\textunderscore .
Tirar do baralho certo número de cartas em certos jogos.
\section{Comprável}
\begin{itemize}
\item {Grp. gram.:adj.}
\end{itemize}
Que se póde comprar.
\section{Comprazedor}
\begin{itemize}
\item {Grp. gram.:m.  e  adj.}
\end{itemize}
O que gosta de comprazer; condescendente.
\section{Comprazente}
\begin{itemize}
\item {Grp. gram.:adj.}
\end{itemize}
O mesmo que \textunderscore complacente\textunderscore . Cf. Filinto, XIX, 242.
\section{Comprazer}
\begin{itemize}
\item {Grp. gram.:v. i.}
\end{itemize}
\begin{itemize}
\item {Grp. gram.:V. p.}
\end{itemize}
\begin{itemize}
\item {Proveniência:(Lat. \textunderscore complacere\textunderscore )}
\end{itemize}
Transigir com o gôsto de alguém.
Condescender.
Tornar-se agradável.
Sentir prazer; deleitar-se: \textunderscore comprazer-se em falar mal\textunderscore .
\section{Comprazimento}
\begin{itemize}
\item {Grp. gram.:m.}
\end{itemize}
Acto de comprazer.
\section{Compreender}
\begin{itemize}
\item {Grp. gram.:v. t.}
\end{itemize}
\begin{itemize}
\item {Proveniência:(Lat. \textunderscore comprehendere\textunderscore )}
\end{itemize}
Conter em si; abranger: \textunderscore êste livro compreende déz capítulos\textunderscore .
Incluir.
Perceber; entender: \textunderscore inda não compreendi o que desejas\textunderscore .
Conhecer as intenções de: \textunderscore bem te compreendo\textunderscore .
\section{Compreendido}
\begin{itemize}
\item {Grp. gram.:adj.}
\end{itemize}
Que incorreu, que está incurso:«\textunderscore compreendida em adultério\textunderscore ». \textunderscore Ethiópia Or.\textunderscore  Comprometido.
\section{Compreensão}
\begin{itemize}
\item {Grp. gram.:f.}
\end{itemize}
\begin{itemize}
\item {Proveniência:(Lat. \textunderscore comprehensio\textunderscore )}
\end{itemize}
Acção de compreender.
Faculdade de perceber.
\section{Compreensibilidade}
\begin{itemize}
\item {Grp. gram.:f.}
\end{itemize}
Qualidade do que é compreensivel.
\section{Compreensiva}
\begin{itemize}
\item {Grp. gram.:f.}
\end{itemize}
\begin{itemize}
\item {Utilização:Des.}
\end{itemize}
\begin{itemize}
\item {Proveniência:(De \textunderscore compreensivo\textunderscore )}
\end{itemize}
Compreensão, faculdade de compreender.
\section{Compreensivamente}
\begin{itemize}
\item {Grp. gram.:adv.}
\end{itemize}
De modo compreensivo.
\section{Compreensível}
\begin{itemize}
\item {Grp. gram.:adj.}
\end{itemize}
\begin{itemize}
\item {Proveniência:(Lat. \textunderscore comprehensibilis\textunderscore )}
\end{itemize}
Que póde sêr compreendido.
\section{Compreensivelmente}
\begin{itemize}
\item {Grp. gram.:adv.}
\end{itemize}
De modo compreensível.
\section{Compreensivo}
\begin{itemize}
\item {Grp. gram.:adj.}
\end{itemize}
\begin{itemize}
\item {Proveniência:(Lat. \textunderscore comprehensivus\textunderscore )}
\end{itemize}
Que compreende.
\section{Compreensor}
\begin{itemize}
\item {Grp. gram.:m.}
\end{itemize}
\begin{itemize}
\item {Proveniência:(Do lat. \textunderscore comprehensus\textunderscore )}
\end{itemize}
Aquele que compreende mistérios, que tem visões beatíficas:«\textunderscore vindes de muy longe, a fazer compreensores os caminhantes\textunderscore ». M. Bernárdez, \textunderscore Luz e Calor\textunderscore , 587.
\section{Comprehender}
\begin{itemize}
\item {Grp. gram.:v. t.}
\end{itemize}
\begin{itemize}
\item {Proveniência:(Lat. \textunderscore comprehendere\textunderscore )}
\end{itemize}
Conter em si; abranger: \textunderscore êste livro comprehende déz capítulos\textunderscore .
Incluir.
Perceber; entender: \textunderscore inda não comprehendi o que desejas\textunderscore .
Conhecer as intenções de: \textunderscore bem te comprehendo\textunderscore .
\section{Comprehendido}
\begin{itemize}
\item {Grp. gram.:adj.}
\end{itemize}
Que incorreu, que está incurso:«\textunderscore comprehendida em adultério\textunderscore ». \textunderscore Ethiópia Or.\textunderscore  Comprometido.
\section{Comprehensão}
\begin{itemize}
\item {Grp. gram.:f.}
\end{itemize}
\begin{itemize}
\item {Proveniência:(Lat. \textunderscore comprehensio\textunderscore )}
\end{itemize}
Acção de comprehender.
Faculdade de perceber.
\section{Comprehensibilidade}
\begin{itemize}
\item {Grp. gram.:f.}
\end{itemize}
Qualidade do que é comprehensivel.
\section{Comprehensiva}
\begin{itemize}
\item {Grp. gram.:f.}
\end{itemize}
\begin{itemize}
\item {Utilização:Des.}
\end{itemize}
\begin{itemize}
\item {Proveniência:(De \textunderscore comprehensivo\textunderscore )}
\end{itemize}
Comprehensão, faculdade de comprehender.
\section{Comprehensivamente}
\begin{itemize}
\item {Grp. gram.:adv.}
\end{itemize}
De modo comprehensivo.
\section{Comprehensível}
\begin{itemize}
\item {Grp. gram.:adj.}
\end{itemize}
\begin{itemize}
\item {Proveniência:(Lat. \textunderscore comprehensibilis\textunderscore )}
\end{itemize}
Que póde sêr comprehendido.
\section{Comprehensivelmente}
\begin{itemize}
\item {Grp. gram.:adv.}
\end{itemize}
De modo comprehensível.
\section{Comprehensivo}
\begin{itemize}
\item {Grp. gram.:adj.}
\end{itemize}
\begin{itemize}
\item {Proveniência:(Lat. \textunderscore comprehensivus\textunderscore )}
\end{itemize}
Que comprehende.
\section{Comprehensor}
\begin{itemize}
\item {Grp. gram.:m.}
\end{itemize}
\begin{itemize}
\item {Proveniência:(Do lat. \textunderscore comprehensus\textunderscore )}
\end{itemize}
Aquelle que comprehende mystérios, que tem visões beatíficas:«\textunderscore vindes de muy longe, a fazer comprehensores os caminhantes\textunderscore ». M. Bernárdez, \textunderscore Luz e Calor\textunderscore , 587.
\section{Compreição}
\begin{itemize}
\item {Grp. gram.:f.}
\end{itemize}
\begin{itemize}
\item {Utilização:Ant.}
\end{itemize}
O mesmo que \textunderscore compleição\textunderscore . Cf. Filodemo, act. V, sc. 4.
\section{Compressa}
\begin{itemize}
\item {Grp. gram.:f.}
\end{itemize}
\begin{itemize}
\item {Proveniência:(Do lat. \textunderscore compressus\textunderscore )}
\end{itemize}
Chumaço de pano, ou estôfo ordinariamente dobrado, que se applica sôbre ferida ou parte doente.
\section{Compressão}
\begin{itemize}
\item {Grp. gram.:f.}
\end{itemize}
\begin{itemize}
\item {Proveniência:(Lat. \textunderscore compressio\textunderscore )}
\end{itemize}
Acto ou effeito de comprimir.
\section{Compressibilidade}
\begin{itemize}
\item {Grp. gram.:f.}
\end{itemize}
Propriedade do que é compressível.
\section{Compressicaudo}
\begin{itemize}
\item {Grp. gram.:adj.}
\end{itemize}
\begin{itemize}
\item {Utilização:Zool.}
\end{itemize}
\begin{itemize}
\item {Proveniência:(Do lat. \textunderscore compressus\textunderscore  + \textunderscore cauda\textunderscore )}
\end{itemize}
Que tem cauda chata.
\section{Compressicaulo}
\begin{itemize}
\item {Grp. gram.:adj.}
\end{itemize}
\begin{itemize}
\item {Utilização:Bot.}
\end{itemize}
\begin{itemize}
\item {Proveniência:(Do lat. \textunderscore compressus\textunderscore  + \textunderscore caulis\textunderscore )}
\end{itemize}
Que tem o cáule comprimido.
\section{Compressicórneo}
\begin{itemize}
\item {Grp. gram.:adj.}
\end{itemize}
\begin{itemize}
\item {Utilização:Zool.}
\end{itemize}
\begin{itemize}
\item {Proveniência:(Do lat. \textunderscore compressus\textunderscore  + \textunderscore cornu\textunderscore )}
\end{itemize}
Que tem antennas comprimidas.
\section{Compressível}
\begin{itemize}
\item {Grp. gram.:adj.}
\end{itemize}
\begin{itemize}
\item {Proveniência:(Do lat. \textunderscore compressus\textunderscore )}
\end{itemize}
Que se póde comprimir.
\section{Compressivo}
\begin{itemize}
\item {Grp. gram.:adj.}
\end{itemize}
\begin{itemize}
\item {Proveniência:(Do lat. \textunderscore compressus\textunderscore )}
\end{itemize}
Que serve para comprimir.
Que reprime.
\section{Compressor}
\begin{itemize}
\item {Grp. gram.:adj.}
\end{itemize}
\begin{itemize}
\item {Grp. gram.:M.}
\end{itemize}
\begin{itemize}
\item {Proveniência:(Lat. \textunderscore compressor\textunderscore )}
\end{itemize}
Que comprime.
Aquelle ou aquillo que comprime.
Instrumento cirúrgico, para comprimir nervos, etc.
\section{Compressório}
\begin{itemize}
\item {Grp. gram.:adj.}
\end{itemize}
Próprio para compremir.
\section{Compridez}
\begin{itemize}
\item {Grp. gram.:f.}
\end{itemize}
O mesmo que \textunderscore comprimento\textunderscore .
\section{Comprido}
\begin{itemize}
\item {Grp. gram.:adj.}
\end{itemize}
\begin{itemize}
\item {Utilização:Ant.}
\end{itemize}
\begin{itemize}
\item {Proveniência:(De \textunderscore comprir\textunderscore )}
\end{itemize}
Extenso; longo: \textunderscore caminho comprido\textunderscore .
Completo, perfeito, realizado.
\section{Comprimentar}
\textunderscore v. t.\textunderscore  (e der.)
(V. \textunderscore cumprimentar\textunderscore , etc)
\section{Comprimente}
\begin{itemize}
\item {Grp. gram.:adj.}
\end{itemize}
\begin{itemize}
\item {Proveniência:(Lat. \textunderscore comprimens\textunderscore )}
\end{itemize}
Que comprime.
\section{Comprimento}
\begin{itemize}
\item {Grp. gram.:m.}
\end{itemize}
\begin{itemize}
\item {Proveniência:(De \textunderscore comprir\textunderscore )}
\end{itemize}
Extensão de um objecto, desde o princípio ao fim delle.
Extensão de um objecto, entre os dois lados mais distantes: \textunderscore comprimento de uma sala\textunderscore .
Distância.
Tamanho.
Saudação, cumprimento. Cp. \textunderscore cumprimento\textunderscore .
\section{Comprimidamente}
\begin{itemize}
\item {Grp. gram.:adv.}
\end{itemize}
Com compressão.
\section{Comprimido}
\begin{itemize}
\item {Grp. gram.:m.}
\end{itemize}
\begin{itemize}
\item {Utilização:Pharm.}
\end{itemize}
\begin{itemize}
\item {Proveniência:(De \textunderscore comprimir\textunderscore )}
\end{itemize}
Substância medicamentosa, comprimida, em fórma de pastilha:«\textunderscore os comprimidos de Bayer\textunderscore ».
\section{Comprimir}
\begin{itemize}
\item {Grp. gram.:v. t.}
\end{itemize}
\begin{itemize}
\item {Proveniência:(Lat. \textunderscore comprimere\textunderscore )}
\end{itemize}
Apertar (um corpo), deminuindo-lhe o volume.
Deminuir.
Reprimir.
Confranger.
\section{Comprir}
\textunderscore v. t.\textunderscore  (e der.)
P. us. em vez de \textunderscore cumprir\textunderscore , etc., mas tem a mesma etym.
\section{Comprobação}
\begin{itemize}
\item {Grp. gram.:f.}
\end{itemize}
\begin{itemize}
\item {Proveniência:(Do lat. \textunderscore comprobatio\textunderscore )}
\end{itemize}
Acto de comprovar.
\section{Comprobante}
\begin{itemize}
\item {Grp. gram.:adj.}
\end{itemize}
\begin{itemize}
\item {Proveniência:(Do lat. \textunderscore comprobans\textunderscore )}
\end{itemize}
Que comprova.
\section{Comprobativo}
\begin{itemize}
\item {Grp. gram.:adj.}
\end{itemize}
\begin{itemize}
\item {Proveniência:(Do lat. \textunderscore comprobatus\textunderscore )}
\end{itemize}
Que comprova.
\section{Comprobatório}
\begin{itemize}
\item {Grp. gram.:adj.}
\end{itemize}
(V.comprobativo)
\section{Comprometedor}
\begin{itemize}
\item {Grp. gram.:adj.}
\end{itemize}
Que compromete.
\section{Comprometer}
\begin{itemize}
\item {Grp. gram.:v. t.}
\end{itemize}
\begin{itemize}
\item {Grp. gram.:V. p.}
\end{itemize}
\begin{itemize}
\item {Proveniência:(Lat. \textunderscore compromittere\textunderscore )}
\end{itemize}
Sujeitar.
Empenhar: \textunderscore comprometer a sua palavra\textunderscore .
Obrigar-se simultaneamente com outrem a alguma coisa.
Obrigar-se: \textunderscore comprometo-me a ir lá\textunderscore .
Estabelecer compromisso.
Assumir responsabilidade grave.
\section{Comprometido}
\begin{itemize}
\item {Grp. gram.:adj.}
\end{itemize}
\begin{itemize}
\item {Utilização:Fam.}
\end{itemize}
Envergonhado de acção que praticou.
Vexado.
\section{Comprometimento}
\begin{itemize}
\item {Grp. gram.:m.}
\end{itemize}
Acção de comprometer ou de comprometer-se.
\section{Compromissal}
\begin{itemize}
\item {Grp. gram.:adj.}
\end{itemize}
\begin{itemize}
\item {Utilização:bras}
\end{itemize}
\begin{itemize}
\item {Utilização:Neol.}
\end{itemize}
Relativo a compromisso.
\section{Compromissário}
\begin{itemize}
\item {Grp. gram.:adj.}
\end{itemize}
Relativo a compromisso.
\section{Compromissivo}
\begin{itemize}
\item {Grp. gram.:adj.}
\end{itemize}
Que envolve compromisso.
\section{Compromisso}
\begin{itemize}
\item {Grp. gram.:m.}
\end{itemize}
\begin{itemize}
\item {Proveniência:(Lat. \textunderscore compromissum\textunderscore )}
\end{itemize}
Obrigação ou promessa, entre duas ou mais pessoas, de sujeitarem a um árbitro a decisão de um pleito.
Qualquer obrigação ou promessa, mais ou menos solenne.
Concordata de fallidos com credores.
Acôrdo político.
Convenção, contrato.
Comprometimento.
Estatutos de confraria.
Escritura vincular.
\section{Compromissório}
\begin{itemize}
\item {Grp. gram.:adj.}
\end{itemize}
Em que há compromisso.
\section{Compromitente}
\begin{itemize}
\item {Grp. gram.:m.  e  adj.}
\end{itemize}
\begin{itemize}
\item {Proveniência:(Lat. \textunderscore compromittens\textunderscore )}
\end{itemize}
O que toma compromisso.
\section{Compropriedade}
\begin{itemize}
\item {Grp. gram.:f.}
\end{itemize}
\begin{itemize}
\item {Proveniência:(De \textunderscore com...\textunderscore  + \textunderscore propriedade\textunderscore )}
\end{itemize}
Propriedade commum.
\section{Comproprietário}
\begin{itemize}
\item {Grp. gram.:m.  e  adj.}
\end{itemize}
\begin{itemize}
\item {Proveniência:(De \textunderscore com...\textunderscore  + \textunderscore proprietário\textunderscore )}
\end{itemize}
O que participa de uma propriedade com outrem.
\section{Comprotector}
\begin{itemize}
\item {Grp. gram.:m.}
\end{itemize}
\begin{itemize}
\item {Proveniência:(De \textunderscore com...\textunderscore  + \textunderscore protector\textunderscore )}
\end{itemize}
Aquelle que protege, juntamente com outrem.
\section{Comprovação}
\begin{itemize}
\item {Grp. gram.:f.}
\end{itemize}
\begin{itemize}
\item {Proveniência:(Lat. \textunderscore comprobatio\textunderscore )}
\end{itemize}
Acto de comprovar.
\section{Comprovador}
\begin{itemize}
\item {Grp. gram.:adj.}
\end{itemize}
Que comprova.
\section{Comprovante}
\begin{itemize}
\item {Grp. gram.:adj.}
\end{itemize}
\begin{itemize}
\item {Proveniência:(Lat. \textunderscore comprobans\textunderscore )}
\end{itemize}
Comprovador.
\section{Comprovar}
\begin{itemize}
\item {Grp. gram.:v. t.}
\end{itemize}
\begin{itemize}
\item {Proveniência:(Lat. \textunderscore comprobare\textunderscore )}
\end{itemize}
Concorrer para provar.
Demonstrar.
Examinar se foram observadas as emendas em provas typográphicas de.
\section{Comprovativo}
\begin{itemize}
\item {Grp. gram.:adj.}
\end{itemize}
\begin{itemize}
\item {Proveniência:(De \textunderscore comprovar\textunderscore )}
\end{itemize}
Comprovador.
\section{Compso}
\begin{itemize}
\item {Grp. gram.:m.}
\end{itemize}
\begin{itemize}
\item {Proveniência:(Gr. \textunderscore kompsos\textunderscore )}
\end{itemize}
Insecto coleóptero da Guiana.
\section{Compsócero}
\begin{itemize}
\item {Grp. gram.:m.}
\end{itemize}
\begin{itemize}
\item {Proveniência:(Do gr. \textunderscore kompsos\textunderscore  + \textunderscore keras\textunderscore )}
\end{itemize}
Insecto coleóptero, longicórneo, de corpo longo e encarnado.
\section{Compsosomo}
\begin{itemize}
\item {Grp. gram.:m.}
\end{itemize}
\begin{itemize}
\item {Proveniência:(Do gr. \textunderscore kompsos\textunderscore  + \textunderscore soma\textunderscore )}
\end{itemize}
Insecto longicórneo, de côres vivas e variadas.
\section{Comptonita}
\begin{itemize}
\item {Grp. gram.:f.}
\end{itemize}
\begin{itemize}
\item {Proveniência:(De \textunderscore Compton\textunderscore , n. p.)}
\end{itemize}
Substância branca ou parda, que se encontra, entre outros pontos, nas lavas do Vesúvio.
\section{Compugnar}
\begin{itemize}
\item {Grp. gram.:v. i.}
\end{itemize}
\begin{itemize}
\item {Proveniência:(Lat. \textunderscore compugnare\textunderscore )}
\end{itemize}
Pelejar em commum; combater.
\section{Compulsação}
\begin{itemize}
\item {Grp. gram.:f.}
\end{itemize}
Acto de compulsar.
\section{Compulsador}
\begin{itemize}
\item {Grp. gram.:m.  e  adj.}
\end{itemize}
O que compulsa.
\section{Compulsão}
\begin{itemize}
\item {Grp. gram.:m.}
\end{itemize}
Acto de compellir.
\section{Compulsar}
\begin{itemize}
\item {Grp. gram.:v. t.}
\end{itemize}
\begin{itemize}
\item {Utilização:Des.}
\end{itemize}
\begin{itemize}
\item {Proveniência:(Lat. \textunderscore compulsare\textunderscore )}
\end{itemize}
Examinar, lendo.
Percorrer, folhear, consultando, (livros ou documentos).
Compellir.
\section{Compulsivo}
\begin{itemize}
\item {Grp. gram.:adj.}
\end{itemize}
\begin{itemize}
\item {Proveniência:(Do lat. \textunderscore compulsus\textunderscore )}
\end{itemize}
Destinado a compellir.
\section{Compulso}
\begin{itemize}
\item {Proveniência:(Lat. \textunderscore compulsus\textunderscore )}
\end{itemize}
\textunderscore part. irr.\textunderscore  de \textunderscore compellir\textunderscore .
\section{Compulsório}
\begin{itemize}
\item {Grp. gram.:adj.}
\end{itemize}
\begin{itemize}
\item {Proveniência:(De \textunderscore compulso\textunderscore )}
\end{itemize}
Que compelle.
\section{Compume}
\begin{itemize}
\item {Grp. gram.:m.}
\end{itemize}
Pássaro tenuirostro da África.
\section{Compunção}
\begin{itemize}
\item {Grp. gram.:f.}
\end{itemize}
\begin{itemize}
\item {Proveniência:(Lat. \textunderscore compunctio\textunderscore )}
\end{itemize}
Pesar de haver cometido acção má ou peccaminosa.
Manifestação dêsse pesar.
Pesar profundo, afflicção.
\section{Compuncção}
\begin{itemize}
\item {Grp. gram.:f.}
\end{itemize}
\begin{itemize}
\item {Proveniência:(Lat. \textunderscore compunctio\textunderscore )}
\end{itemize}
Pesar de haver cometido acção má ou peccaminosa.
Manifestação dêsse pesar.
Pesar profundo, afflicção.
\section{Compungimento}
\begin{itemize}
\item {Grp. gram.:m.}
\end{itemize}
\begin{itemize}
\item {Proveniência:(De \textunderscore compungir\textunderscore )}
\end{itemize}
O mesmo que \textunderscore compuncção\textunderscore .
\section{Compungir}
\begin{itemize}
\item {Grp. gram.:v. t.}
\end{itemize}
\begin{itemize}
\item {Grp. gram.:V. p.}
\end{itemize}
\begin{itemize}
\item {Proveniência:(Lat. \textunderscore compungere\textunderscore )}
\end{itemize}
Pungir.
Despertar compuncção em.
Affligir.
Têr compuncção.
\section{Compungitivo}
\begin{itemize}
\item {Grp. gram.:adj.}
\end{itemize}
\begin{itemize}
\item {Proveniência:(De \textunderscore compungir\textunderscore )}
\end{itemize}
Que compunge.
Compungido.
\section{Compurgação}
\begin{itemize}
\item {Grp. gram.:f.}
\end{itemize}
Acto de compurgar. Cf. Herculano, \textunderscore Hist. de Port.\textunderscore , IV, 347, 359 e 362.
\section{Compurgador}
\begin{itemize}
\item {Grp. gram.:m.  e  adj.}
\end{itemize}
O que compurga. Cf. Herculano, \textunderscore Hist. de Port.\textunderscore , IV, 368 e 370.
\section{Compurgar}
\begin{itemize}
\item {Grp. gram.:v. t.}
\end{itemize}
\begin{itemize}
\item {Utilização:Ant.}
\end{itemize}
\begin{itemize}
\item {Proveniência:(Lat. \textunderscore compurgare\textunderscore )}
\end{itemize}
Purificar por meio do ordálio.
\section{Computação}
\begin{itemize}
\item {Grp. gram.:f.}
\end{itemize}
Acto de computar.
\section{Computador}
\begin{itemize}
\item {Grp. gram.:m.}
\end{itemize}
\begin{itemize}
\item {Proveniência:(De \textunderscore computar\textunderscore )}
\end{itemize}
Aquelle que faz cômputos, que calcula.
\section{Computar}
\begin{itemize}
\item {Grp. gram.:v. t.}
\end{itemize}
\begin{itemize}
\item {Proveniência:(Lat. \textunderscore computare\textunderscore )}
\end{itemize}
Fazer o cômputo de.
Contar.
Orçar; calcular.
\section{Computável}
\begin{itemize}
\item {Grp. gram.:adj.}
\end{itemize}
\begin{itemize}
\item {Proveniência:(Lat. \textunderscore computabilis\textunderscore )}
\end{itemize}
Que se póde computar.
\section{Computista}
\begin{itemize}
\item {Grp. gram.:m.}
\end{itemize}
\begin{itemize}
\item {Proveniência:(De \textunderscore cômputo\textunderscore )}
\end{itemize}
Calendarista.
\section{Cômputo}
\begin{itemize}
\item {Grp. gram.:m.}
\end{itemize}
\begin{itemize}
\item {Utilização:Des.}
\end{itemize}
\begin{itemize}
\item {Proveniência:(Lat. \textunderscore computus\textunderscore )}
\end{itemize}
Conta; cálculo.
Processo, com que os calendaristas determinam o dia, em que deve cair a Páschoa.
Lapso de tempo:«\textunderscore tal pressa se deram em construir a igreja, que em breve cômputo ficou concluida\textunderscore ». Filinto, \textunderscore D. Man.\textunderscore , I, 233.
\section{Com-quanto}
\begin{itemize}
\item {Grp. gram.:loc. conj.}
\end{itemize}
Se bem que, posto que.
\section{Com-sego}
\begin{itemize}
\item {fónica:sê}
\end{itemize}
\begin{itemize}
\item {Grp. gram.:loc. pron.}
\end{itemize}
\begin{itemize}
\item {Utilização:Ant.}
\end{itemize}
O mesmo que \textunderscore com-sigo\textunderscore .
\section{Com-sigo}
\begin{itemize}
\item {Grp. gram.:loc. pron.}
\end{itemize}
Em companhia de pessôa ou pessôas, de quem se fala: \textunderscore levou-me consigo\textunderscore .
De si para si: \textunderscore falava consigo\textunderscore .
(Flexão do pron. \textunderscore si\textunderscore , precedido da prep. \textunderscore com\textunderscore )
\section{Com-tanto-que}
\begin{itemize}
\item {Grp. gram.:loc. conj.}
\end{itemize}
Dado que; sob condição de que.
\section{Comteano}
\begin{itemize}
\item {Grp. gram.:adj.}
\end{itemize}
\begin{itemize}
\item {Grp. gram.:M.}
\end{itemize}
Relativo ao philósopho Comte.
Sectário de Comte.
\section{Com-tego}
\begin{itemize}
\item {fónica:tê}
\end{itemize}
\begin{itemize}
\item {Grp. gram.:loc. pron.}
\end{itemize}
\begin{itemize}
\item {Utilização:Ant.}
\end{itemize}
O mesmo que \textunderscore com-tigo\textunderscore .
\section{Com-tigo}
\begin{itemize}
\item {Grp. gram.:loc. pron.}
\end{itemize}
Na tua companhia: \textunderscore jantei contigo\textunderscore .
De ti para ti: \textunderscore estavas falando contigo\textunderscore .
(Flexão do pron. \textunderscore tu\textunderscore , precedido da prep. \textunderscore com\textunderscore )
\section{Comtista}
\begin{itemize}
\item {Grp. gram.:m.}
\end{itemize}
Sectário de Comte, comteano.
\section{Com-tudo}
\begin{itemize}
\item {Grp. gram.:adv.  e  conj.}
\end{itemize}
Todavia.
Não obstante.
\section{Comua}
\begin{itemize}
\item {Grp. gram.:adj.}
\end{itemize}
\begin{itemize}
\item {Utilização:Ant.}
\end{itemize}
\begin{itemize}
\item {Grp. gram.:F.}
\end{itemize}
\begin{itemize}
\item {Proveniência:(De \textunderscore comum\textunderscore )}
\end{itemize}
(Fem. de \textunderscore comum\textunderscore )
Latrina, retreta.
\section{Comudação}
\begin{itemize}
\item {Grp. gram.:f.}
\end{itemize}
Acto ou efeito de comudar.
\section{Comudar}
\begin{itemize}
\item {Grp. gram.:v. t.}
\end{itemize}
O mesmo que \textunderscore comutar\textunderscore .
\section{Comum}
\begin{itemize}
\item {Grp. gram.:adj.}
\end{itemize}
\begin{itemize}
\item {Grp. gram.:M.}
\end{itemize}
\begin{itemize}
\item {Proveniência:(Lat. \textunderscore communis\textunderscore )}
\end{itemize}
Relativo a muitos ou todos.
Vulgar, habitual.
Feito em comunidade, em sociedade.
Insignificante.
Maioria: \textunderscore o comum dos mortaes\textunderscore .
Vulgaridade.
\section{Comumbá}
\begin{itemize}
\item {Grp. gram.:m.}
\end{itemize}
\begin{itemize}
\item {Utilização:Bras}
\end{itemize}
Árvore dos sertões.
\section{Comummente}
\begin{itemize}
\item {Grp. gram.:adv.}
\end{itemize}
\begin{itemize}
\item {Proveniência:(De \textunderscore comum\textunderscore )}
\end{itemize}
Geralmente; de ordinário; vulgarmente.
\section{Comuna}
\begin{itemize}
\item {Grp. gram.:f.}
\end{itemize}
\begin{itemize}
\item {Proveniência:(De \textunderscore commum\textunderscore )}
\end{itemize}
Antigo agrupamento de estrangeiros, especialmente Judeus e Moiros, que eram obrigados a viver em arruamentos determinados.
Povoação que, na Idade-Média, se emancipava do feudalismo, governando-se autonomicamente.
Subdivisão territorial em França.
Administração de concelho.
\section{Comunal}
\begin{itemize}
\item {Grp. gram.:adj.}
\end{itemize}
\begin{itemize}
\item {Utilização:Ant.}
\end{itemize}
\begin{itemize}
\item {Grp. gram.:M.}
\end{itemize}
Relativo a comuna.
O mesmo que \textunderscore comum\textunderscore .
Habitante de uma comuna.
(B. lat. \textunderscore communalis\textunderscore )
\section{Comunalismo}
\begin{itemize}
\item {Grp. gram.:m.}
\end{itemize}
Doutrina ou sistema dos comunalistas.
Municipalismo.
\section{Comunalista}
\begin{itemize}
\item {Grp. gram.:m.}
\end{itemize}
\begin{itemize}
\item {Proveniência:(De \textunderscore comunal\textunderscore )}
\end{itemize}
Propugnador dos privilégios comunaes ou municipaes.
Partidário da descentralização administrativa.
\section{Comunalmente}
\begin{itemize}
\item {Grp. gram.:adv.}
\end{itemize}
\begin{itemize}
\item {Utilização:Ant.}
\end{itemize}
\begin{itemize}
\item {Proveniência:(De \textunderscore comunal\textunderscore )}
\end{itemize}
Comummente; ordinariamente.
\section{Comuneiro}
\begin{itemize}
\item {Grp. gram.:m.}
\end{itemize}
Comunal.
Membro das comunidades, que em Espanha se revoltaram contra Carlos V.
(Cast. \textunderscore comunero\textunderscore )
\section{Comuneza}
\begin{itemize}
\item {Grp. gram.:f.}
\end{itemize}
\begin{itemize}
\item {Utilização:Ant.}
\end{itemize}
O mesmo que \textunderscore comunidade\textunderscore .
\section{Comungante}
\begin{itemize}
\item {Grp. gram.:adj.}
\end{itemize}
Que comunga.
\section{Comungar}
\begin{itemize}
\item {Grp. gram.:v. t.}
\end{itemize}
\begin{itemize}
\item {Grp. gram.:V. i.}
\end{itemize}
\begin{itemize}
\item {Proveniência:(Lat. \textunderscore communicare\textunderscore )}
\end{itemize}
Administrar a comunhão a.
Receber em comunhão.
Receber a comunhão.
Participar das crenças de uma seita, facção ou grupo de indivíduos: \textunderscore comungar nas ideias de outrem\textunderscore .
Estar de acôrdo.
Tomar parte.
\section{Comungatório}
\begin{itemize}
\item {Grp. gram.:adj.}
\end{itemize}
\begin{itemize}
\item {Grp. gram.:M.}
\end{itemize}
\begin{itemize}
\item {Proveniência:(Do lat. \textunderscore communicator\textunderscore )}
\end{itemize}
Relativo a comunhão.
Local, onde se toma comunhão.
\section{Comunhão}
\begin{itemize}
\item {Grp. gram.:f.}
\end{itemize}
\begin{itemize}
\item {Proveniência:(Lat. \textunderscore communio\textunderscore )}
\end{itemize}
Acto ou efeito de comungar.
Comunidade de opiniões.
Participação comum em crenças.
Sacramento da Eucaristia.
Recepção da Eucaristia.
\section{Comunial}
\begin{itemize}
\item {Grp. gram.:adj.}
\end{itemize}
\begin{itemize}
\item {Proveniência:(Do lat. \textunderscore communio\textunderscore )}
\end{itemize}
Relativo a comunhão.
\section{Comunião}
\begin{itemize}
\item {Grp. gram.:f.}
\end{itemize}
\begin{itemize}
\item {Utilização:Des.}
\end{itemize}
O mesmo que \textunderscore comunhão\textunderscore . Cf. Garrett, \textunderscore Port. na Balança\textunderscore , 203.
\section{Comunicabilidade}
\begin{itemize}
\item {Grp. gram.:f.}
\end{itemize}
Qualidade do que é comunicável.
\section{Comunicação}
\begin{itemize}
\item {Grp. gram.:f.}
\end{itemize}
\begin{itemize}
\item {Proveniência:(Lat. \textunderscore communicatio\textunderscore )}
\end{itemize}
Acto, efeito ou meio de comunicar.
Transmissão: \textunderscore comunicação de um telegrama\textunderscore .
Convivência, trato.
Lugar, por onde se passa de um ponto para outro.
\section{Comunicado}
\begin{itemize}
\item {Grp. gram.:m.}
\end{itemize}
\begin{itemize}
\item {Proveniência:(De \textunderscore comunicar\textunderscore )}
\end{itemize}
Escrito ou artigo, geralmente de interesse particular, dirigido a jornal ou jornaes.
\section{Comunicador}
\begin{itemize}
\item {Grp. gram.:m.  e  adj.}
\end{itemize}
\begin{itemize}
\item {Proveniência:(Lat. \textunderscore communicator\textunderscore )}
\end{itemize}
O que comunica.
\section{Comunicante}
\begin{itemize}
\item {Grp. gram.:adj.}
\end{itemize}
\begin{itemize}
\item {Proveniência:(Lat. \textunderscore communicans\textunderscore )}
\end{itemize}
Que comunica.
\section{Comunicar}
\begin{itemize}
\item {Grp. gram.:v. t.}
\end{itemize}
\begin{itemize}
\item {Grp. gram.:V. i.}
\end{itemize}
\begin{itemize}
\item {Proveniência:(Lat. \textunderscore communicare\textunderscore )}
\end{itemize}
Fazer comum.
Tornar conhecido, fazer saber, participar: \textunderscore comunicar um casamento\textunderscore .
Ligar; pôr em contacto.
Transmitir.
Conviver com. Cf. Filinto, XVII, 197; XVIII, 75 e 168.
Estar ligado; têr contacto.
Têr correspondência.
Têr passagem comum.
\section{Comunicativamente}
\begin{itemize}
\item {Grp. gram.:adv.}
\end{itemize}
De modo comunicativo.
Expansivamente.
\section{Comunicativo}
\begin{itemize}
\item {Grp. gram.:adj.}
\end{itemize}
\begin{itemize}
\item {Proveniência:(Lat. \textunderscore communicativus\textunderscore )}
\end{itemize}
Que se comunica facilmente.
Que expõe francamente.
Expansivo.
\section{Comunicável}
\begin{itemize}
\item {Grp. gram.:adj.}
\end{itemize}
\begin{itemize}
\item {Proveniência:(Lat. \textunderscore communicabilis\textunderscore )}
\end{itemize}
Que se póde comunicar.
Franco, expansivo.
\section{Comunidade}
\begin{itemize}
\item {Grp. gram.:f.}
\end{itemize}
\begin{itemize}
\item {Proveniência:(Lat. \textunderscore communitas\textunderscore )}
\end{itemize}
Qualidade daquilo que é commum; comunhão.
Sociedade.
Nacionalidade.
Agremiação de indivíduos, que têm a mesma crença ou a mesma norma de vida.
Lugar, onde residem êsses indivíduos: \textunderscore as comunidades indianas\textunderscore .
Comuna.
\section{Comunismo}
\begin{itemize}
\item {Grp. gram.:m.}
\end{itemize}
\begin{itemize}
\item {Proveniência:(De \textunderscore comum\textunderscore )}
\end{itemize}
Theoría social, que pretende a comunhão dos bens naturaes e dos productos do trabalho.
\section{Comuníssimo}
\begin{itemize}
\item {Grp. gram.:adj.}
\end{itemize}
Muito comum, vulgaríssimo, trivial.
(Sup. de \textunderscore comum\textunderscore )
\section{Comunista}
\begin{itemize}
\item {Grp. gram.:adj.}
\end{itemize}
\begin{itemize}
\item {Grp. gram.:M.}
\end{itemize}
\begin{itemize}
\item {Proveniência:(De \textunderscore comum\textunderscore )}
\end{itemize}
Relativo ao comunismo.
Sectário do comunismo.
\section{Comunitário}
\begin{itemize}
\item {Grp. gram.:m.}
\end{itemize}
\begin{itemize}
\item {Utilização:Neol.}
\end{itemize}
\begin{itemize}
\item {Grp. gram.:Adj.}
\end{itemize}
\begin{itemize}
\item {Proveniência:(Do lat. \textunderscore communitas\textunderscore )}
\end{itemize}
O mesmo que \textunderscore comunista\textunderscore .
Aquele que á iniciativa individual prefere viver á custa do Estado, sem responsabilidades nem cuidados. Cf. \textunderscore Diário de Notícias\textunderscore  de 2-XII-900.
Diz-se da formação dos povos, em que prepondera o sentimento de comunidade, como nas tríbos orientaes, em oposição a \textunderscore particularista\textunderscore .
\section{Comuns}
\begin{itemize}
\item {Grp. gram.:m. pl.}
\end{itemize}
Membros da camara baixa do parlamento inglês, eleitos pelas povoações do reino.
(Chamam-se assim, por fazerem parte da \textunderscore house of commons\textunderscore , casa ou camara das \textunderscore comunas\textunderscore )
\section{Comutação}
\begin{itemize}
\item {Grp. gram.:f.}
\end{itemize}
\begin{itemize}
\item {Utilização:Gram.}
\end{itemize}
\begin{itemize}
\item {Proveniência:(Lat. \textunderscore commutatio\textunderscore )}
\end{itemize}
Acto de comutar.
O mesmo que \textunderscore metátese\textunderscore .
Permutação, substituição.
\section{Comutador}
\begin{itemize}
\item {Grp. gram.:m.  e  adj.}
\end{itemize}
\begin{itemize}
\item {Grp. gram.:M.}
\end{itemize}
\begin{itemize}
\item {Utilização:Phýs.}
\end{itemize}
O que comuta.
Aparelho para alterar a direcção das correntes eléctricas.
\section{Comutar}
\begin{itemize}
\item {Grp. gram.:v. t.}
\end{itemize}
\begin{itemize}
\item {Proveniência:(Lat. \textunderscore commutare\textunderscore )}
\end{itemize}
Permutar.
Substituir.
\section{Comutativo}
\begin{itemize}
\item {Grp. gram.:adj.}
\end{itemize}
\begin{itemize}
\item {Proveniência:(De \textunderscore comutar\textunderscore )}
\end{itemize}
Que comuta.
Relativo á troca.
\section{Comutável}
\begin{itemize}
\item {Grp. gram.:adj.}
\end{itemize}
\begin{itemize}
\item {Proveniência:(Lat. \textunderscore commutabilis\textunderscore )}
\end{itemize}
Que se póde comutar.
\section{Com-vosco}
\begin{itemize}
\item {fónica:vôs}
\end{itemize}
\begin{itemize}
\item {Grp. gram.:loc. pron.}
\end{itemize}
Em vossa companhia: \textunderscore levai-me convosco\textunderscore .
De vós para vós.
Entre vós: \textunderscore resolvei lá isso convosco\textunderscore .
(Flexão do pron. \textunderscore vós\textunderscore , precedido da prep. \textunderscore com\textunderscore )
\section{Con...}
\begin{itemize}
\item {Grp. gram.:pref.}
\end{itemize}
(Cp. \textunderscore com...\textunderscore )
\section{Conabi}
\begin{itemize}
\item {Grp. gram.:m.}
\end{itemize}
Planta euphorbiácea do Brasil, (\textunderscore phyllantus conami\textunderscore ).
\section{Conana}
\begin{itemize}
\item {Grp. gram.:m.}
\end{itemize}
\begin{itemize}
\item {Utilização:Pop.}
\end{itemize}
Homem mulherengo; maricas.
\section{Conanas}
\begin{itemize}
\item {Grp. gram.:m.}
\end{itemize}
\begin{itemize}
\item {Utilização:Pop.}
\end{itemize}
Homem mulherengo; maricas.
\section{Conantera}
\begin{itemize}
\item {Grp. gram.:f.}
\end{itemize}
\begin{itemize}
\item {Proveniência:(Do gr. \textunderscore khonos\textunderscore  + \textunderscore antheros\textunderscore )}
\end{itemize}
Gênero de plantas herbáceas do Chile.
\section{Conanthera}
\begin{itemize}
\item {Grp. gram.:f.}
\end{itemize}
\begin{itemize}
\item {Proveniência:(Do gr. \textunderscore khonos\textunderscore  + \textunderscore antheros\textunderscore )}
\end{itemize}
Gênero de plantas herbáceas do Chile.
\section{Conapa}
\begin{itemize}
\item {Grp. gram.:f.}
\end{itemize}
\begin{itemize}
\item {Utilização:Prov.}
\end{itemize}
\begin{itemize}
\item {Utilização:beir.}
\end{itemize}
Acto de conapar.
Cerzidura mal feita ou remendo mal deitado.
\section{Conapar}
\begin{itemize}
\item {Grp. gram.:v. t.}
\end{itemize}
\begin{itemize}
\item {Utilização:Prov.}
\end{itemize}
\begin{itemize}
\item {Utilização:beir.}
\end{itemize}
O mesmo que \textunderscore enconapar\textunderscore .
\section{Conca}
\begin{itemize}
\item {Grp. gram.:f.}
\end{itemize}
\begin{itemize}
\item {Utilização:Des.}
\end{itemize}
\begin{itemize}
\item {Proveniência:(Lat. \textunderscore concha\textunderscore )}
\end{itemize}
Tigela.
Pedra, ou tijolo, para o jôgo da malha.
Concha da orelha.
Variedade de maçan.
\section{Concameração}
\begin{itemize}
\item {Grp. gram.:f.}
\end{itemize}
\begin{itemize}
\item {Utilização:Phýs.}
\end{itemize}
\begin{itemize}
\item {Proveniência:(Lat. \textunderscore concameratio\textunderscore )}
\end{itemize}
Parte arqueada de uma edificação.
Abóbada.
Columna de ar entre duas ondas sonoras.
Lóculo ou cavidade, relacionada com outra, no organismo vegetal.
\section{Concani}
\begin{itemize}
\item {Grp. gram.:m.}
\end{itemize}
Língua vulgar no território de Gôa, língua do Concão.
\section{Concanim}
\begin{itemize}
\item {Grp. gram.:m.}
\end{itemize}
Língua vulgar no território de Gôa, língua do Concão.
\section{Concanjila}
\begin{itemize}
\item {Grp. gram.:m.}
\end{itemize}
Insecto africano, espécie de escaravelho.
\section{Concanó}
\begin{itemize}
\item {Grp. gram.:m.}
\end{itemize}
\begin{itemize}
\item {Utilização:T. da Índia port}
\end{itemize}
Gentio do Concão.
\section{Concão}
\begin{itemize}
\item {Grp. gram.:m.}
\end{itemize}
\begin{itemize}
\item {Utilização:Prov.}
\end{itemize}
\begin{itemize}
\item {Utilização:trasm.}
\end{itemize}
O mesmo que \textunderscore còcão\textunderscore .
\section{Concassor}
\begin{itemize}
\item {Grp. gram.:m.}
\end{itemize}
\begin{itemize}
\item {Utilização:bras}
\end{itemize}
\begin{itemize}
\item {Utilização:Neol.}
\end{itemize}
\begin{itemize}
\item {Proveniência:(Fr. \textunderscore concasseur\textunderscore )}
\end{itemize}
Utensílio, para partir e britar pedras.
\section{Concatenação}
\begin{itemize}
\item {Grp. gram.:f.}
\end{itemize}
\begin{itemize}
\item {Proveniência:(Lat. \textunderscore concatenatio\textunderscore )}
\end{itemize}
Acto ou effeito de concatenar.
\section{Concatenar}
\begin{itemize}
\item {Grp. gram.:v. t.}
\end{itemize}
\begin{itemize}
\item {Proveniência:(Lat. \textunderscore concatenare\textunderscore )}
\end{itemize}
Encadear, prender.
Relacionar: \textunderscore concatenar ideias\textunderscore .
\section{Concausa}
\begin{itemize}
\item {Grp. gram.:f.}
\end{itemize}
\begin{itemize}
\item {Utilização:P. us.}
\end{itemize}
\begin{itemize}
\item {Proveniência:(De \textunderscore com...\textunderscore  + \textunderscore causa\textunderscore )}
\end{itemize}
Causa, que concorre com outra para a producção do seu effeito.
Pessôa, que, com outra, concorre para um fim.
\section{Concavar}
\begin{itemize}
\item {Grp. gram.:v. t.}
\end{itemize}
\begin{itemize}
\item {Proveniência:(Lat. \textunderscore concavare\textunderscore )}
\end{itemize}
Tornar côncavo.
\section{Concavidade}
\begin{itemize}
\item {Grp. gram.:f.}
\end{itemize}
Fórma côncava de um objecto.
Qualidade daquillo que é côncavo.
\section{Concavifoliado}
\begin{itemize}
\item {Grp. gram.:adj.}
\end{itemize}
\begin{itemize}
\item {Utilização:Bot.}
\end{itemize}
\begin{itemize}
\item {Proveniência:(Do lat. \textunderscore concavus\textunderscore  + \textunderscore folïum\textunderscore )}
\end{itemize}
Que tem fôlhas côncavas.
\section{Côncavo}
\begin{itemize}
\item {Grp. gram.:adj.}
\end{itemize}
\begin{itemize}
\item {Grp. gram.:M.}
\end{itemize}
\begin{itemize}
\item {Proveniência:(Lat. \textunderscore concavus\textunderscore )}
\end{itemize}
Cavado, escavado.
Sinuoso.
Menos elevado no meio que nas bordas.
Concavidade; cavidade.
\section{Côncavò-côncavo}
\begin{itemize}
\item {Grp. gram.:adj.}
\end{itemize}
Que é côncavo nas duas faces.
\section{Côncavò-convexo}
\begin{itemize}
\item {Grp. gram.:adj.}
\end{itemize}
Que é côncavo de um lado e convexo do outro.
\section{Conceber}
\begin{itemize}
\item {Grp. gram.:v. i.  e  t.}
\end{itemize}
\begin{itemize}
\item {Proveniência:(Lat. \textunderscore concipere\textunderscore )}
\end{itemize}
Gerar.
Formar no espírito ou no coração: \textunderscore conceber planos\textunderscore .
Entender, perceber.
Assumir, tomar: \textunderscore conceber muito frio\textunderscore .
\section{Concebimento}
\begin{itemize}
\item {Grp. gram.:m.}
\end{itemize}
O mesmo que \textunderscore concepção\textunderscore .
\section{Concebível}
\begin{itemize}
\item {Grp. gram.:adj.}
\end{itemize}
Que se póde conceber.
\section{Concedente}
\begin{itemize}
\item {Grp. gram.:m.  e  adj.}
\end{itemize}
\begin{itemize}
\item {Proveniência:(Lat. \textunderscore concedens\textunderscore )}
\end{itemize}
O que concede.
\section{Conceder}
\begin{itemize}
\item {Grp. gram.:v. t.}
\end{itemize}
\begin{itemize}
\item {Grp. gram.:V. i.}
\end{itemize}
\begin{itemize}
\item {Proveniência:(Lat. \textunderscore concedere\textunderscore )}
\end{itemize}
Consentir.
Permittir.
Dar: \textunderscore conceder licença\textunderscore .
\textunderscore Conceder em\textunderscore , transigir com. Cf. \textunderscore Peregrinação\textunderscore , XXVIII.
\section{Conceição}
\begin{itemize}
\item {Grp. gram.:f.}
\end{itemize}
\begin{itemize}
\item {Proveniência:(Do lat. \textunderscore conceptio\textunderscore )}
\end{itemize}
Concepção da Virgem Maria no ventre de sua mãe.
Festa, com que se celebra essa concepção.
Ordem militar portuguesa.
Antiga moéda portuguesa, do valor de 12$000 reis, no tempo de D. João IV.
\section{Conceicionista}
\begin{itemize}
\item {Grp. gram.:f.}
\end{itemize}
Freira da Ordem da Conceição de Maria.
\section{Conceitarrão}
\begin{itemize}
\item {Grp. gram.:m.}
\end{itemize}
Grande conceito, juízo excellente. Cf. Filinto, XXII, 16.
\section{Conceitarraz}
\begin{itemize}
\item {Grp. gram.:m.}
\end{itemize}
O mesmo que \textunderscore conceitarrão\textunderscore . Cf. Filinto, VI, 13.
\section{Conceito}
\begin{itemize}
\item {Grp. gram.:m.}
\end{itemize}
\begin{itemize}
\item {Proveniência:(Do lat. \textunderscore conceptus\textunderscore )}
\end{itemize}
Aquillo que o espírito concebe.
Entendimento, opinião: \textunderscore formar mau conceito de alguém\textunderscore .
Reputação.
Bôa reputação.
Máxima, dito sentencioso: \textunderscore os conceitos de La-Bruyère\textunderscore .
Sýnthese, substância.
Parte da charada, em que se faz referência á palavra completa e respectiva.
\section{Conceituar}
\begin{itemize}
\item {Grp. gram.:v. t.}
\end{itemize}
\begin{itemize}
\item {Proveniência:(De \textunderscore conceito\textunderscore )}
\end{itemize}
Formar conceito á cêrca de.
Avaliar.
\section{Conceituosamente}
\begin{itemize}
\item {Grp. gram.:adv.}
\end{itemize}
De modo conceituoso.
\section{Conceituoso}
\begin{itemize}
\item {Grp. gram.:adj.}
\end{itemize}
Em que há conceito.
Espirituoso.
Sentencioso.
\section{Concela}
\begin{itemize}
\item {Grp. gram.:f.}
\end{itemize}
\begin{itemize}
\item {Utilização:Ant.}
\end{itemize}
\begin{itemize}
\item {Proveniência:(Do rad. do lat. \textunderscore concelare\textunderscore )}
\end{itemize}
Vaso sagrado, âmbula.
\section{Concelebração}
\begin{itemize}
\item {Grp. gram.:f.}
\end{itemize}
Acto ou effeito de concelebrar.
\section{Concelebrar}
\begin{itemize}
\item {Grp. gram.:v. t.}
\end{itemize}
\begin{itemize}
\item {Proveniência:(Lat. \textunderscore concelebrare\textunderscore )}
\end{itemize}
Celebrar em commum.
\section{Concelheiro}
\begin{itemize}
\item {Grp. gram.:adj.}
\end{itemize}
\begin{itemize}
\item {Grp. gram.:M.}
\end{itemize}
\begin{itemize}
\item {Utilização:Prov.}
\end{itemize}
\begin{itemize}
\item {Utilização:trasm.}
\end{itemize}
O mesmo que \textunderscore concelhio\textunderscore .
Terreno baldio.
\section{Concelhio}
\begin{itemize}
\item {Grp. gram.:adj.}
\end{itemize}
Relativo ou pertencente ao concelho: \textunderscore interesses concelhios\textunderscore .
\section{Concelho}
\begin{itemize}
\item {fónica:cê}
\end{itemize}
\begin{itemize}
\item {Grp. gram.:m.}
\end{itemize}
\begin{itemize}
\item {Proveniência:(Lat. \textunderscore concilium\textunderscore )}
\end{itemize}
Circunscripção territorial, que é uma das divisões do distrito.
Município.
\section{Concento}
\begin{itemize}
\item {Grp. gram.:m.}
\end{itemize}
\begin{itemize}
\item {Proveniência:(Lat. \textunderscore concentus\textunderscore )}
\end{itemize}
Canto simultâneo.
Consonância.
\section{Concentração}
\begin{itemize}
\item {Grp. gram.:f.}
\end{itemize}
Acto ou effeito de concentrar.
\section{Concentradamente}
\begin{itemize}
\item {Grp. gram.:adv.}
\end{itemize}
De modo concentrado.
\section{Concentrador}
\begin{itemize}
\item {Grp. gram.:m.  e  adj.}
\end{itemize}
O que concentra.
\section{Concentrar}
\begin{itemize}
\item {Grp. gram.:v. t.}
\end{itemize}
\begin{itemize}
\item {Utilização:Chím.}
\end{itemize}
\begin{itemize}
\item {Grp. gram.:V. p.}
\end{itemize}
\begin{itemize}
\item {Proveniência:(De \textunderscore con...\textunderscore  + \textunderscore centro\textunderscore )}
\end{itemize}
Fazer convergir para um centro; centralizar: \textunderscore concentrar em si attribuições dos outros\textunderscore .
Applicar em um objecto único.
Recolher.
Tirar de (um líquido a água que nelle há).
Occultar, dissimular.
Não dar expansão a.
Tornar mais denso ou mais activo (um sal), deminuindo o líquido em que está, ou augmentando o mesmo sal, sem alterar a quantidade do líquido.
Preoccupar-se de uma só coisa ou meditar nella, abstrahindo de tudo mais.
Meditar profundamente.
\section{Concentricidade}
\begin{itemize}
\item {Grp. gram.:f.}
\end{itemize}
Qualidade do que é concêntrico.
\section{Concêntrico}
\begin{itemize}
\item {Grp. gram.:adj.}
\end{itemize}
\begin{itemize}
\item {Utilização:Geom.}
\end{itemize}
\begin{itemize}
\item {Proveniência:(De \textunderscore con...\textunderscore  + \textunderscore centro\textunderscore )}
\end{itemize}
Diz-se das curvas, que têm o mesmo centro e raios differentes.
\section{Concepção}
\begin{itemize}
\item {Grp. gram.:f.}
\end{itemize}
\begin{itemize}
\item {Proveniência:(Lat. \textunderscore conceptio\textunderscore )}
\end{itemize}
Acto de sêr concebido, de sêr gerado.
Geração.
Faculdade de perceber, percepção.
Fantasia: \textunderscore as concepções dos poétas\textunderscore .
\section{Concepcionário}
\begin{itemize}
\item {Grp. gram.:f.}
\end{itemize}
\begin{itemize}
\item {Proveniência:(Do lat. \textunderscore conceptio\textunderscore )}
\end{itemize}
Defensor da conceição immaculada da Virgem Maria.
\section{Conceptáculo}
\begin{itemize}
\item {Grp. gram.:m.}
\end{itemize}
\begin{itemize}
\item {Proveniência:(Lat. \textunderscore conceptaculum\textunderscore )}
\end{itemize}
Receptáculo.
Parte do receptáculo de alguns cogumelos, em que se alojam os germes reproductores.
\section{Conceptibilidade}
\begin{itemize}
\item {Grp. gram.:f.}
\end{itemize}
Qualidade do que é conceptível.
\section{Conceptível}
\begin{itemize}
\item {Grp. gram.:adj.}
\end{itemize}
O mesmo que \textunderscore concebível\textunderscore .
\section{Conceptivo}
\begin{itemize}
\item {Grp. gram.:adj.}
\end{itemize}
\begin{itemize}
\item {Proveniência:(Lat. \textunderscore conceptivus\textunderscore )}
\end{itemize}
Próprio para sêr concebido.
\section{Conceptual}
\begin{itemize}
\item {Grp. gram.:adj.}
\end{itemize}
\begin{itemize}
\item {Proveniência:(Do lat. \textunderscore conceptus\textunderscore )}
\end{itemize}
Relativo á concepção.
\section{Conceptualismo}
\begin{itemize}
\item {Grp. gram.:m.}
\end{itemize}
\begin{itemize}
\item {Proveniência:(De \textunderscore conceptual\textunderscore )}
\end{itemize}
Systema philosóphico, que occupa o meio termo entre o nominalismo e o realismo.
Mau gôsto literário, que consiste em trocadilhos e conceitos subtís, e que, na Alemanha, correspondia ao gongorismo espanhol.
\section{Conceptualista}
\begin{itemize}
\item {Grp. gram.:m.}
\end{itemize}
Sectário do conceptualismo.
(Cp. \textunderscore conceptualismo\textunderscore )
\section{Concernência}
\begin{itemize}
\item {Grp. gram.:f.}
\end{itemize}
Qualidade de concernente.
Aquillo que diz respeito a alguma coisa:«\textunderscore decreta em todas as concernencias da religião\textunderscore ». Filinto, \textunderscore D. Man.\textunderscore , I, 322.
\section{Concernente}
\begin{itemize}
\item {Grp. gram.:adj.}
\end{itemize}
\begin{itemize}
\item {Proveniência:(Lat. \textunderscore concernens\textunderscore )}
\end{itemize}
Que concerne.
Relativo.
Respectivo.
\section{Concernir}
\begin{itemize}
\item {Grp. gram.:v. i.}
\end{itemize}
\begin{itemize}
\item {Proveniência:(Lat. \textunderscore concernere\textunderscore )}
\end{itemize}
Dizer respeito.
Têr relação, referir-se.
\section{Concertadamente}
\begin{itemize}
\item {Grp. gram.:adv.}
\end{itemize}
De modo concertado.
\section{Concertador}
\begin{itemize}
\item {Grp. gram.:m.}
\end{itemize}
Aquelle que concerta.
\section{Concertamento}
\begin{itemize}
\item {Grp. gram.:m.}
\end{itemize}
(V. \textunderscore concêrto\textunderscore ^1)
\section{Concertante}
\begin{itemize}
\item {Grp. gram.:adj.}
\end{itemize}
\begin{itemize}
\item {Utilização:Mús.}
\end{itemize}
\begin{itemize}
\item {Proveniência:(De \textunderscore concertar\textunderscore )}
\end{itemize}
Que peleja, que litiga com alguém.
Que entra num concêrto vocal ou instrumental.
\section{Concertar}
\begin{itemize}
\item {Grp. gram.:v. t.}
\end{itemize}
\begin{itemize}
\item {Grp. gram.:V. i.}
\end{itemize}
\begin{itemize}
\item {Proveniência:(Lat. \textunderscore concertare\textunderscore )}
\end{itemize}
Ajustar, combinar: \textunderscore concertar um plano\textunderscore .
Conferenciar.
Harmonizar; compor: \textunderscore concertar os desavindos\textunderscore .
Dispor em ordem.
Confrontar.
Concordar.
\section{Concertar}
\begin{itemize}
\item {Grp. gram.:v. t.}
\end{itemize}
(V.consertar)
\section{Concertina}
\begin{itemize}
\item {Grp. gram.:f.}
\end{itemize}
\begin{itemize}
\item {Proveniência:(De \textunderscore concêrto\textunderscore )}
\end{itemize}
Instrumento musical.
\section{Concertino}
\begin{itemize}
\item {Grp. gram.:m.}
\end{itemize}
\begin{itemize}
\item {Utilização:Mús.}
\end{itemize}
\begin{itemize}
\item {Proveniência:(De \textunderscore concêrto\textunderscore )}
\end{itemize}
Rabequista que, na orchestra, occupa o lugar immediato ao do primeiro violino principal.
\section{Concertista}
\begin{itemize}
\item {Grp. gram.:m.}
\end{itemize}
Músico de concertos.
\section{Concêrto}
\begin{itemize}
\item {Grp. gram.:m.}
\end{itemize}
Acto de concertar.
Consonância de instrumentos ou de vozes que cantam.
Sessão musical: \textunderscore há hoje concêrto no casino\textunderscore .
Harmonia; conjunto de sons.
Composição musical, extensa e desenvolvida, para um instrumento, com acompanhamento de orchestra, quarteto ou piano.
Assembleia, onde se executam vários trechos de música.
\section{Concêrto}
\begin{itemize}
\item {Grp. gram.:m.}
\end{itemize}
(V.consêrto)
\section{Concessão}
\begin{itemize}
\item {Grp. gram.:f.}
\end{itemize}
\begin{itemize}
\item {Proveniência:(Lat. \textunderscore concessio\textunderscore )}
\end{itemize}
Acto de conceder.
Permissão.
\section{Concessionário}
\begin{itemize}
\item {Grp. gram.:m.  e  adj.}
\end{itemize}
\begin{itemize}
\item {Proveniência:(Do lat. \textunderscore concessio\textunderscore )}
\end{itemize}
O que obtém uma concessão.
\section{Concessivo}
\begin{itemize}
\item {Grp. gram.:adj.}
\end{itemize}
Relativo a concessão.
\section{Concessor}
\begin{itemize}
\item {Grp. gram.:m.}
\end{itemize}
\begin{itemize}
\item {Proveniência:(Do lat. \textunderscore concessus\textunderscore )}
\end{itemize}
Aquelle que concede.
\section{Concessório}
\begin{itemize}
\item {Grp. gram.:adj.}
\end{itemize}
O mesmo que \textunderscore concessivo\textunderscore .
\section{Concha}
\begin{itemize}
\item {Grp. gram.:f.}
\end{itemize}
\begin{itemize}
\item {Proveniência:(Do lat. hyp. \textunderscore conchula\textunderscore , de \textunderscore concha\textunderscore )}
\end{itemize}
Invólucro calcário ou córneo de certos animaes.
Objecto, de feitio análogo ao da concha, como um colhér muito côncava, o caço, a entrada do canal auditivo, etc.
Prato de balança.
A parte côncava das chaves que, nos instrumentos de sopro, fecham os orifícios.
Peça de metal, que se emprega nas gavetas, em substituição dos puxadores.
Porca de madeira, em que volteia o fuso ou parafuso, nas antigas prensas de vara dos nossos lagares.
\section{Conchal}
\begin{itemize}
\item {Grp. gram.:m.}
\end{itemize}
O mesmo que \textunderscore conchalim\textunderscore . Cf. \textunderscore Peregrinação\textunderscore , LXXXVI.
\section{Conchali}
\begin{itemize}
\item {Grp. gram.:m.}
\end{itemize}
O mesmo que \textunderscore conchalim\textunderscore . Cf. \textunderscore Peregrinação\textunderscore , XCVII.
\section{Conchalim}
\begin{itemize}
\item {Grp. gram.:m.}
\end{itemize}
Magistrado judicial, entre os Chineses.
\section{Conchar}
\begin{itemize}
\item {Grp. gram.:v. t.}
\end{itemize}
(V.conchear)
\section{Conchar}
\begin{itemize}
\item {Grp. gram.:m.}
\end{itemize}
\begin{itemize}
\item {Utilização:Ant.}
\end{itemize}
Logista, vendedor de arroz, na Índia portuguesa.
\section{Concharia}
\begin{itemize}
\item {Grp. gram.:f.}
\end{itemize}
Grande quantidade de conchas.
\section{Conchavar}
\begin{itemize}
\item {Grp. gram.:v. t.}
\end{itemize}
\begin{itemize}
\item {Proveniência:(Lat. \textunderscore conclavare\textunderscore )}
\end{itemize}
Ligar.
Encaixar.
Combinar, ajustar.
Conluiar.
\section{Conchavo}
\begin{itemize}
\item {Grp. gram.:m.}
\end{itemize}
Acto de conchavar.
\section{Concheado}
\begin{itemize}
\item {Grp. gram.:adj.}
\end{itemize}
Que tem concha.
\section{Conchear}
\begin{itemize}
\item {Grp. gram.:v. t.}
\end{itemize}
Ornar ou revestir de conchas.
\section{Conchegar}
\begin{itemize}
\item {Grp. gram.:v. t.}
\end{itemize}
\begin{itemize}
\item {Proveniência:(De \textunderscore con...\textunderscore  + \textunderscore chegar\textunderscore )}
\end{itemize}
Aproximar, pôr em contacto.
Dispor bem.
\section{Conchegativo}
\begin{itemize}
\item {Grp. gram.:adj.}
\end{itemize}
\begin{itemize}
\item {Proveniência:(De \textunderscore conchegar\textunderscore )}
\end{itemize}
Que ministra conchego, confôrto ou commodidade ao corpo; confortável.
\section{Conchego}
\begin{itemize}
\item {fónica:chê}
\end{itemize}
\begin{itemize}
\item {Grp. gram.:m.}
\end{itemize}
Acto de conchegar.
Commodidade.
Pessôa que protege.
Amparo.
\section{Conchelo}
\begin{itemize}
\item {Grp. gram.:m.}
\end{itemize}
Planta crassulácea, conhecida vulgarmente por \textunderscore arroz-de-telhado\textunderscore , (\textunderscore umbilicus pendulinus\textunderscore , De-Cand.).
O mesmo que \textunderscore couxilgo\textunderscore  e \textunderscore coucelo\textunderscore .
\section{Conchícola}
\begin{itemize}
\item {fónica:qui}
\end{itemize}
\begin{itemize}
\item {Grp. gram.:adj.}
\end{itemize}
\begin{itemize}
\item {Utilização:Zool.}
\end{itemize}
\begin{itemize}
\item {Proveniência:(Do lat. \textunderscore concha\textunderscore  + \textunderscore colere\textunderscore )}
\end{itemize}
Que vive em concha bivalve.
\section{Conchífero}
\begin{itemize}
\item {Grp. gram.:adj.}
\end{itemize}
\begin{itemize}
\item {Proveniência:(Do lat. \textunderscore concha\textunderscore  + \textunderscore ferre\textunderscore )}
\end{itemize}
Que tem conchas.
\section{Conchite}
\begin{itemize}
\item {fónica:qui}
\end{itemize}
\begin{itemize}
\item {Grp. gram.:f.}
\end{itemize}
\begin{itemize}
\item {Utilização:Des.}
\end{itemize}
\begin{itemize}
\item {Proveniência:(Do lat. \textunderscore concha\textunderscore )}
\end{itemize}
Petrificação, semelhante a uma concha.
Concha fóssil.
\section{Concho}
\begin{itemize}
\item {Grp. gram.:adj.}
\end{itemize}
\begin{itemize}
\item {Utilização:Pop.}
\end{itemize}
\begin{itemize}
\item {Grp. gram.:M.}
\end{itemize}
\begin{itemize}
\item {Utilização:Prov.}
\end{itemize}
\begin{itemize}
\item {Utilização:trasm.}
\end{itemize}
Que tem confiança em si; vaidoso.
Vaso de fôlha ou de cortiça, com um cabo comprido, e que serve para tirar água dos poços, nas regas.
(Cp. \textunderscore concha\textunderscore )
\section{Concho!}
\begin{itemize}
\item {Grp. gram.:interj.}
\end{itemize}
\begin{itemize}
\item {Utilização:Prov.}
\end{itemize}
\begin{itemize}
\item {Utilização:trasm.}
\end{itemize}
Caramba!
Bólas!
\section{Conchoidal}
\begin{itemize}
\item {fónica:cói}
\end{itemize}
\begin{itemize}
\item {Grp. gram.:adj.}
\end{itemize}
\begin{itemize}
\item {Utilização:Geom.}
\end{itemize}
Semelhante a uma concha.
Relativo á conchoide.
\section{Conchoide}
\begin{itemize}
\item {fónica:cói}
\end{itemize}
\begin{itemize}
\item {Grp. gram.:adj.}
\end{itemize}
\begin{itemize}
\item {Grp. gram.:F.}
\end{itemize}
\begin{itemize}
\item {Proveniência:(Gr. \textunderscore konkhoiedes\textunderscore )}
\end{itemize}
Que é semelhante a uma concha.
Designação de uma curva geométrica.
\section{Conchoso}
\begin{itemize}
\item {Grp. gram.:adj.}
\end{itemize}
Em que há muitas conchas. Cf. Castilho, \textunderscore Primavera\textunderscore , 209.
\section{Conchouso}
\begin{itemize}
\item {Grp. gram.:m.}
\end{itemize}
\begin{itemize}
\item {Utilização:Des.}
\end{itemize}
O mesmo que \textunderscore quinchoso\textunderscore .
\section{Conchudo}
\begin{itemize}
\item {Grp. gram.:adj.}
\end{itemize}
O mesmo que \textunderscore concheado\textunderscore  e \textunderscore concho\textunderscore .
\section{Conchyliologia}
\begin{itemize}
\item {fónica:qui}
\end{itemize}
\begin{itemize}
\item {Grp. gram.:f.}
\end{itemize}
\begin{itemize}
\item {Proveniência:(Do gr. \textunderscore konkhulion\textunderscore  + \textunderscore logos\textunderscore )}
\end{itemize}
Tratado das conchas.
\section{Conchyliológico}
\begin{itemize}
\item {fónica:qui}
\end{itemize}
\begin{itemize}
\item {Grp. gram.:adj.}
\end{itemize}
Relativo á conchyliologia.
\section{Conchyliologista}
\begin{itemize}
\item {fónica:qui}
\end{itemize}
\begin{itemize}
\item {Grp. gram.:m.}
\end{itemize}
Aquelle que é versado em conchyliologia.
\section{Conchylióphoro}
\begin{itemize}
\item {fónica:qui}
\end{itemize}
\begin{itemize}
\item {Grp. gram.:adj.}
\end{itemize}
\begin{itemize}
\item {Proveniência:(Do gr. \textunderscore konkhulion\textunderscore  + \textunderscore phoros\textunderscore )}
\end{itemize}
Que tem concha.
\section{Concidadão}
\begin{itemize}
\item {Grp. gram.:m.}
\end{itemize}
\begin{itemize}
\item {Proveniência:(De \textunderscore con...\textunderscore  + \textunderscore cidadão\textunderscore )}
\end{itemize}
Indivíduo, que, em relação a outro ou outros, é da mesma cidade ou do mesmo país.
\section{Conciliábulo}
\begin{itemize}
\item {Grp. gram.:m.}
\end{itemize}
\begin{itemize}
\item {Proveniência:(Lat. \textunderscore conciliabulum\textunderscore )}
\end{itemize}
Pequeno concílio.
Conventículo.
Assembleia secreta; conluio.
\section{Conciliação}
\begin{itemize}
\item {Grp. gram.:f.}
\end{itemize}
\begin{itemize}
\item {Proveniência:(Lat. \textunderscore conciliatio\textunderscore )}
\end{itemize}
Acto e effeito de conciliar.
Acto de harmonizar litigantes ou pessôas divergentes.
\section{Conciliador}
\begin{itemize}
\item {Grp. gram.:m.  e  adj.}
\end{itemize}
O que concilía.
\section{Conciliante}
\begin{itemize}
\item {Grp. gram.:adj.}
\end{itemize}
\begin{itemize}
\item {Proveniência:(Lat. \textunderscore concilians\textunderscore )}
\end{itemize}
Que concilía.
\section{Conciliar}
\begin{itemize}
\item {Grp. gram.:adj.}
\end{itemize}
Relativo a concílio: \textunderscore resoluções conciliares\textunderscore .
\section{Conciliar}
\begin{itemize}
\item {Grp. gram.:v. t.}
\end{itemize}
\begin{itemize}
\item {Proveniência:(Lat. \textunderscore conciliare\textunderscore )}
\end{itemize}
Harmonizar; estabelecer acôrdo entre.
Combinar.
Unir.
Chamar a si, conseguir: \textunderscore conciliar a attenção dos ouvintes\textunderscore .
\section{Conciliário}
\begin{itemize}
\item {Grp. gram.:adj.}
\end{itemize}
O mesmo que \textunderscore conciliar\textunderscore ^1.
\section{Conciliativo}
\begin{itemize}
\item {Grp. gram.:adj.}
\end{itemize}
\begin{itemize}
\item {Proveniência:(De \textunderscore conciliar\textunderscore ^2)}
\end{itemize}
Conciliador.
\section{Conciliatório}
\begin{itemize}
\item {Grp. gram.:adj.}
\end{itemize}
Destinado a conciliar; que serve para conciliar.
\section{Conciliável}
\begin{itemize}
\item {Grp. gram.:adj.}
\end{itemize}
Que se póde conciliar.
\section{Concílio}
\begin{itemize}
\item {Grp. gram.:m.}
\end{itemize}
\begin{itemize}
\item {Utilização:Ant.}
\end{itemize}
\begin{itemize}
\item {Proveniência:(Lat. \textunderscore concilium\textunderscore )}
\end{itemize}
Assembleia de Prelados cathólicos, em que se tratam assumptos dogmáticos ou disciplinares.
Território ou jurisdição, que se separou de uma diocése.
\section{Concinidade}
\begin{itemize}
\item {Grp. gram.:f.}
\end{itemize}
\begin{itemize}
\item {Utilização:Ant.}
\end{itemize}
\begin{itemize}
\item {Proveniência:(Lat. \textunderscore concinnitas\textunderscore )}
\end{itemize}
Elegância; apuro.
\section{Concinnidade}
\begin{itemize}
\item {Grp. gram.:f.}
\end{itemize}
\begin{itemize}
\item {Utilização:Ant.}
\end{itemize}
\begin{itemize}
\item {Proveniência:(Lat. \textunderscore concinnitas\textunderscore )}
\end{itemize}
Elegância; apuro.
\section{Concional}
\begin{itemize}
\item {Grp. gram.:adj.}
\end{itemize}
\begin{itemize}
\item {Proveniência:(Lat. \textunderscore concionalis\textunderscore )}
\end{itemize}
Relativo a assembleias públicas: \textunderscore eloquência concional\textunderscore .
\section{Concionar}
\begin{itemize}
\item {Grp. gram.:v. i.}
\end{itemize}
\begin{itemize}
\item {Utilização:Des.}
\end{itemize}
\begin{itemize}
\item {Proveniência:(Lat. \textunderscore concionari\textunderscore )}
\end{itemize}
Falar em público.
\section{Concionário}
\begin{itemize}
\item {Grp. gram.:adj.}
\end{itemize}
\begin{itemize}
\item {Proveniência:(Lat. \textunderscore concionarius\textunderscore )}
\end{itemize}
O mesmo que \textunderscore concional\textunderscore .
\section{Concionatório}
\begin{itemize}
\item {Grp. gram.:adj.}
\end{itemize}
O mesmo que \textunderscore concional\textunderscore .
\section{Concisamente}
\begin{itemize}
\item {Grp. gram.:adv.}
\end{itemize}
De modo conciso.
Laconicamente.
\section{Concisão}
\begin{itemize}
\item {Grp. gram.:f.}
\end{itemize}
\begin{itemize}
\item {Proveniência:(Lat. \textunderscore concisio\textunderscore )}
\end{itemize}
Qualidade do que é conciso; brevidade, laconismo.
\section{Conciso}
\begin{itemize}
\item {Grp. gram.:adj.}
\end{itemize}
\begin{itemize}
\item {Proveniência:(Lat. \textunderscore concisus\textunderscore )}
\end{itemize}
Que expõe as ideias em poucas palavras.
Lacónico; resumido: \textunderscore exposição concisa\textunderscore .
\section{Concitação}
\begin{itemize}
\item {Grp. gram.:f.}
\end{itemize}
\begin{itemize}
\item {Proveniência:(Lat. \textunderscore concitatio\textunderscore )}
\end{itemize}
Acto ou effeito de concitar.
\section{Concitador}
\begin{itemize}
\item {Grp. gram.:m.  e  adj.}
\end{itemize}
\begin{itemize}
\item {Proveniência:(Lat. \textunderscore concitator\textunderscore )}
\end{itemize}
Que concita.
\section{Concitar}
\begin{itemize}
\item {Grp. gram.:v. t.}
\end{itemize}
\begin{itemize}
\item {Proveniência:(Lat. \textunderscore concitare\textunderscore )}
\end{itemize}
Agitar, commover.
Instigar; excitar.
Perturbar.
\section{Concitativo}
\begin{itemize}
\item {Grp. gram.:adj.}
\end{itemize}
Que concita.
\section{Conclamação}
\begin{itemize}
\item {Grp. gram.:f.}
\end{itemize}
\begin{itemize}
\item {Proveniência:(Lat. \textunderscore conclamatio\textunderscore )}
\end{itemize}
Acto de conclamar.
\section{Conclamar}
\begin{itemize}
\item {Grp. gram.:v. i.}
\end{itemize}
\begin{itemize}
\item {Grp. gram.:V. t.}
\end{itemize}
\begin{itemize}
\item {Proveniência:(Lat. \textunderscore conclamare\textunderscore )}
\end{itemize}
Clamar em commum.
Gritar em tumulto.
Aclamar em commum, aclamar com outros:«\textunderscore todas as aldeias do Minho conclamaram D. Miguel\textunderscore ». Camillo, \textunderscore Brasileira\textunderscore , 63.
\section{Conclave}
\begin{itemize}
\item {Grp. gram.:m.}
\end{itemize}
\begin{itemize}
\item {Proveniência:(Lat. \textunderscore conclave\textunderscore )}
\end{itemize}
Assembleia de Cardeaes, para a eleição do Papa.
Lugar, em que êlles se reúnem para êsse fim.--Filinto e António Dinís, sem razão, consideram proparoxýtono êste voc.
\section{Conclavista}
\begin{itemize}
\item {Grp. gram.:m.}
\end{itemize}
\begin{itemize}
\item {Proveniência:(De \textunderscore conclave\textunderscore )}
\end{itemize}
Membro de conclave.
Fâmulo de Cardeal, que com êste se encerra no conclave, até á eleição do Papa.
\section{Concludente}
\begin{itemize}
\item {Grp. gram.:adj.}
\end{itemize}
\begin{itemize}
\item {Proveniência:(Lat. \textunderscore concludens\textunderscore )}
\end{itemize}
Que conclue.
Procedente; que prova o que se allega: \textunderscore argumentos concludentes\textunderscore .
\section{Concludentemente}
\begin{itemize}
\item {Grp. gram.:adv.}
\end{itemize}
De modo concludente.
\section{Concludir}
\begin{itemize}
\item {Grp. gram.:v. t.}
\end{itemize}
\begin{itemize}
\item {Utilização:Ant.}
\end{itemize}
O mesmo que \textunderscore concluir\textunderscore . Cf. \textunderscore Inéd. da Hist. Port.\textunderscore , I, 400.
\section{Concluimento}
\begin{itemize}
\item {fónica:clu-i}
\end{itemize}
\begin{itemize}
\item {Grp. gram.:m.}
\end{itemize}
Acto de concluir; o mesmo que \textunderscore conclusão\textunderscore :«\textunderscore ventura no concluimento das batalhas\textunderscore ». Filinto, \textunderscore D. Man.\textunderscore , I, 337.
\section{Concluinte}
\begin{itemize}
\item {Grp. gram.:adj.}
\end{itemize}
\begin{itemize}
\item {Utilização:Des.}
\end{itemize}
O mesmo que \textunderscore concludente\textunderscore . Cf. Sousa, \textunderscore Vida do Arceb.\textunderscore , II, 272.
\section{Concluir}
\begin{itemize}
\item {Grp. gram.:v. t.}
\end{itemize}
\begin{itemize}
\item {Proveniência:(Lat. \textunderscore concludere\textunderscore )}
\end{itemize}
Pôr fim a; terminar.
Deduzir: \textunderscore eis o que se conclue do arrazoado\textunderscore .
Resolver.
\section{Conclusão}
\begin{itemize}
\item {Grp. gram.:f.}
\end{itemize}
\begin{itemize}
\item {Proveniência:(Lat. \textunderscore conclusio\textunderscore )}
\end{itemize}
Acto de concluir.
Illação, deducção.
Estado de um processo, que é mandado ao juiz, para que êste lavre despacho ou sentença.
These.
\section{Conclusionista}
\begin{itemize}
\item {Grp. gram.:m.}
\end{itemize}
\begin{itemize}
\item {Proveniência:(Do lat. \textunderscore conclusio\textunderscore )}
\end{itemize}
Aquelle que na Universidade defende conclusões ou theses finaes.
\section{Conclusivamente}
\begin{itemize}
\item {Grp. gram.:adv.}
\end{itemize}
De modo conclusivo.
\section{Conclusivo}
\begin{itemize}
\item {Grp. gram.:adj.}
\end{itemize}
\begin{itemize}
\item {Proveniência:(De \textunderscore concluso\textunderscore )}
\end{itemize}
Que contém conclusão.
Próprio para se concluir.
\section{Concluso}
\begin{itemize}
\item {Grp. gram.:adj.}
\end{itemize}
Diz-se principalmente do processo entregue ao juiz para despacho ou sentença.
(Part. irr. de \textunderscore concluir\textunderscore )
\section{Concocção}
\begin{itemize}
\item {Grp. gram.:f.}
\end{itemize}
\begin{itemize}
\item {Utilização:Des.}
\end{itemize}
\begin{itemize}
\item {Proveniência:(Lat. \textunderscore concoctio\textunderscore )}
\end{itemize}
Digestão.
\section{Concoctivo}
\begin{itemize}
\item {Grp. gram.:adj.}
\end{itemize}
\begin{itemize}
\item {Proveniência:(Do lat. \textunderscore concoctus\textunderscore )}
\end{itemize}
Relativo á concocção.
\section{Concoctor}
\begin{itemize}
\item {Grp. gram.:adj.}
\end{itemize}
Que facilita a digestão.
(Cp. \textunderscore concocção\textunderscore )
\section{Concoidal}
\begin{itemize}
\item {Grp. gram.:adj.}
\end{itemize}
\begin{itemize}
\item {Utilização:Geom.}
\end{itemize}
Semelhante a uma concha.
Relativo á concoide.
\section{Concoide}
\begin{itemize}
\item {Grp. gram.:adj.}
\end{itemize}
\begin{itemize}
\item {Grp. gram.:F.}
\end{itemize}
\begin{itemize}
\item {Proveniência:(Gr. \textunderscore konkhoiedes\textunderscore )}
\end{itemize}
Que é semelhante a uma concha.
Designação de uma curva geométrica.
\section{Concomitância}
\begin{itemize}
\item {Grp. gram.:f.}
\end{itemize}
Qualidade do que é concomitante.
\section{Concomitante}
\begin{itemize}
\item {Grp. gram.:m.}
\end{itemize}
\begin{itemize}
\item {Proveniência:(Lat. \textunderscore concomitans\textunderscore )}
\end{itemize}
Que acompanha.
Que se manifesta ao mesmo tempo que outro.
Accessório.
\section{Concomitantemente}
\begin{itemize}
\item {Grp. gram.:adv.}
\end{itemize}
Simultaneamente; de modo concomitante.
\section{Concordança}
\begin{itemize}
\item {Grp. gram.:f.}
\end{itemize}
\begin{itemize}
\item {Utilização:Ant.}
\end{itemize}
Acto ou effeito de concordar.
Concordância; concórdia.
\section{Concordância}
\begin{itemize}
\item {Grp. gram.:f.}
\end{itemize}
\begin{itemize}
\item {Proveniência:(De \textunderscore concordar\textunderscore )}
\end{itemize}
Acto de concordar.
Consonância; harmonia.
Identidade de gênero, número, etc., de certas palavras, com um número, gênero, etc., de outras, com que têm relações syntácticas.
\section{Concordante}
\begin{itemize}
\item {Grp. gram.:adj.}
\end{itemize}
\begin{itemize}
\item {Proveniência:(Lat. \textunderscore concordans\textunderscore )}
\end{itemize}
Que concorda.
\section{Concordantemente}
\begin{itemize}
\item {Grp. gram.:adv.}
\end{itemize}
De modo concordante.
\section{Concordar}
\begin{itemize}
\item {Grp. gram.:v. t.}
\end{itemize}
\begin{itemize}
\item {Grp. gram.:V. i.}
\end{itemize}
\begin{itemize}
\item {Proveniência:(Lat. \textunderscore concordare\textunderscore )}
\end{itemize}
Conciliar; pôr em concordância.
Têr concordância.
Harmonizar-se, estar de acôrdo.
\section{Concordata}
\begin{itemize}
\item {Grp. gram.:f.}
\end{itemize}
\begin{itemize}
\item {Proveniência:(Do lat. \textunderscore concordatus\textunderscore )}
\end{itemize}
Convenção entre o Estado e a Igreja, á cêrca de assumptos religiosos de uma nação.
Acôrdo entre um negociante fallido e os seus credores, prescindindo êstes da liquidação, e obrigando-se aquelle ao pagamento dos débitos em determinado prazo, e com reducção na quantia devida.
\section{Concordatário}
\begin{itemize}
\item {Grp. gram.:m.}
\end{itemize}
\begin{itemize}
\item {Grp. gram.:Adj.}
\end{itemize}
Aquelle que propôs ou acceitou concordata.
Relativo a concordata.
\section{Concordável}
\begin{itemize}
\item {Grp. gram.:adj.}
\end{itemize}
\begin{itemize}
\item {Proveniência:(De \textunderscore concordar\textunderscore )}
\end{itemize}
Sôbre o que póde haver acôrdo.
\section{Concorde}
\begin{itemize}
\item {Grp. gram.:adj.}
\end{itemize}
\begin{itemize}
\item {Proveniência:(Lat. \textunderscore concors\textunderscore )}
\end{itemize}
Concordante.
Que é da mesma opinião.
\section{Concordemente}
\begin{itemize}
\item {Grp. gram.:adv.}
\end{itemize}
De modo concorde.
\section{Concórdia}
\begin{itemize}
\item {Grp. gram.:f.}
\end{itemize}
\begin{itemize}
\item {Proveniência:(Lat. \textunderscore concordia\textunderscore )}
\end{itemize}
Concordância.
Harmonia de vontades.
Paz.
\section{Concorpóreo}
\begin{itemize}
\item {Grp. gram.:adj.}
\end{itemize}
\begin{itemize}
\item {Proveniência:(De \textunderscore con...\textunderscore  + \textunderscore corporeo\textunderscore )}
\end{itemize}
Que participa no corpo de Christo pela communhão.
\section{Concorrência}
\begin{itemize}
\item {Grp. gram.:f.}
\end{itemize}
\begin{itemize}
\item {Proveniência:(Lat. \textunderscore concurrentia\textunderscore )}
\end{itemize}
Acto de concorrer.
\section{Concorrente}
\begin{itemize}
\item {Grp. gram.:m.  e  adj.}
\end{itemize}
\begin{itemize}
\item {Proveniência:(Lat. \textunderscore concurrens\textunderscore )}
\end{itemize}
O que concorre.
\section{Concorrentemente}
\begin{itemize}
\item {Grp. gram.:adv.}
\end{itemize}
De modo concorrente.
\section{Concorrer}
\begin{itemize}
\item {Grp. gram.:v. i.}
\end{itemize}
\begin{itemize}
\item {Proveniência:(Lat. \textunderscore concurrere\textunderscore )}
\end{itemize}
Têr a mesma pretensão que outrem: \textunderscore concorrer ao lugar de escrivão\textunderscore .
Ir com outrem.
Affluir, ajuntar-se.
Contribuir: \textunderscore concorreu com 1O$OOO reis para aquella construcção\textunderscore .
Existir ao mesmo tempo: \textunderscore concorrem nelle qualidades recommendáveis\textunderscore .
Encontrar-se.
\section{Concrear}
\textunderscore v. t.\textunderscore  (e der.)(V.concriar)
\section{Concreção}
\begin{itemize}
\item {Grp. gram.:f.}
\end{itemize}
\begin{itemize}
\item {Proveniência:(Do lat. \textunderscore concretus\textunderscore )}
\end{itemize}
Acto de se condensar, de se solidificar.
Effeito de aggregação dos sólidos contidos num líquido.
Conjunto de partículas no interior dos tecidos vegetaes.
Aggregação de partículas de phosphato calcário dentro de certos órgãos.
Formação anormal de um osso.
\section{Concrecionado}
\begin{itemize}
\item {Grp. gram.:adj.}
\end{itemize}
\begin{itemize}
\item {Utilização:Miner.}
\end{itemize}
Em que há concreção; que fórma concreção.
\section{Concrescibilidade}
\begin{itemize}
\item {Grp. gram.:f.}
\end{itemize}
Qualidade do que é concrescível.
\section{Concrescível}
\begin{itemize}
\item {Grp. gram.:adj.}
\end{itemize}
\begin{itemize}
\item {Proveniência:(Do lat. \textunderscore concrescere\textunderscore )}
\end{itemize}
Susceptível de se tornar concreto.
\section{Concretização}
\begin{itemize}
\item {Grp. gram.:f.}
\end{itemize}
Acto de \textunderscore concretizar\textunderscore .
\section{Concretizar}
\begin{itemize}
\item {Grp. gram.:v. t.}
\end{itemize}
Tornar concreto.
\section{Concreto}
\begin{itemize}
\item {Grp. gram.:adj.}
\end{itemize}
\begin{itemize}
\item {Grp. gram.:M.}
\end{itemize}
\begin{itemize}
\item {Proveniência:(Lat. \textunderscore concretus\textunderscore )}
\end{itemize}
Condensado; espêsso; solidificado.
Determinado; particular: \textunderscore accusações concretas\textunderscore .
Concreção.
Aquillo que é concreto.
\section{Concriação}
\begin{itemize}
\item {Grp. gram.:f.}
\end{itemize}
Acto de concriar.
\section{Concriar}
\begin{itemize}
\item {Grp. gram.:v. t.}
\end{itemize}
\begin{itemize}
\item {Proveniência:(De \textunderscore con...\textunderscore  + \textunderscore criar\textunderscore )}
\end{itemize}
Criar simultaneamente.
\section{Concrudir}
\begin{itemize}
\item {Grp. gram.:v. t.}
\end{itemize}
\begin{itemize}
\item {Utilização:Ant.}
\end{itemize}
\begin{itemize}
\item {Proveniência:(Lat. \textunderscore concludere\textunderscore )}
\end{itemize}
O mesmo que \textunderscore concluir\textunderscore . Cf. G. Vicente, I, 231.
\section{Concruir}
\begin{itemize}
\item {Grp. gram.:v. t.}
\end{itemize}
\begin{itemize}
\item {Utilização:Ant.}
\end{itemize}
O mesmo que \textunderscore concluir\textunderscore . Cf. G. Vicente, \textunderscore Carta a D. João III\textunderscore .
\section{Concrusão}
\begin{itemize}
\item {Grp. gram.:f.}
\end{itemize}
\begin{itemize}
\item {Utilização:Ant.}
\end{itemize}
O mesmo que \textunderscore conclusão\textunderscore . Cf. \textunderscore Eufrosina\textunderscore , 9.
\section{Concubina}
\begin{itemize}
\item {Grp. gram.:f.}
\end{itemize}
\begin{itemize}
\item {Proveniência:(Lat. \textunderscore concubina\textunderscore )}
\end{itemize}
Mulher, que dorme habitual e illicitamente com um homem; amásia.
\section{Concubinagem}
\begin{itemize}
\item {Grp. gram.:f.}
\end{itemize}
O mesmo que \textunderscore concubinato\textunderscore .
\section{Concubinariamente}
\begin{itemize}
\item {Grp. gram.:adv.}
\end{itemize}
Á maneira de concubinário.
\section{Concubinário}
\begin{itemize}
\item {Grp. gram.:m.  e  adj.}
\end{itemize}
O que tem concubina.
\section{Concubinar-se}
\begin{itemize}
\item {Grp. gram.:v. p.}
\end{itemize}
\begin{itemize}
\item {Utilização:Fig.}
\end{itemize}
Amancebar-se.
Conluiar-se, conchavar-se. Cf. Cortesão, \textunderscore Subs\textunderscore .
\section{Concubinato}
\begin{itemize}
\item {Grp. gram.:m.}
\end{itemize}
\begin{itemize}
\item {Proveniência:(Lat. \textunderscore concubinatus\textunderscore )}
\end{itemize}
Estado daquelle que tem concubina, ou da mulher que é concubina.
Mancebia.
\section{Concubitata}
\begin{itemize}
\item {Grp. gram.:adj. f.}
\end{itemize}
\begin{itemize}
\item {Proveniência:(De \textunderscore concúbito\textunderscore )}
\end{itemize}
Forniziada, prostituída.
\section{Concúbito}
\begin{itemize}
\item {Grp. gram.:m.}
\end{itemize}
\begin{itemize}
\item {Proveniência:(Lat. \textunderscore concubitus\textunderscore )}
\end{itemize}
Cóito; cohabitação.
\section{Conculcador}
\begin{itemize}
\item {Grp. gram.:m.  e  adj.}
\end{itemize}
O que conculca.
\section{Conculcar}
\begin{itemize}
\item {Grp. gram.:v. t.}
\end{itemize}
\begin{itemize}
\item {Proveniência:(Lat. \textunderscore conculcare\textunderscore )}
\end{itemize}
Calcar muito com os pés; espesinhar.
Desprezar, postergar: \textunderscore conculcar deveres\textunderscore .
\section{Conculha}
\begin{itemize}
\item {Grp. gram.:f.}
\end{itemize}
\begin{itemize}
\item {Utilização:Prov.}
\end{itemize}
\begin{itemize}
\item {Utilização:trasm.}
\end{itemize}
\begin{itemize}
\item {Proveniência:(De \textunderscore conca\textunderscore )}
\end{itemize}
Pequena porção de cereaes ou de outras coisas, dentro de um saco, enchendo-lhe apenas um canto ou pouco mais.
\section{Concunhada}
\textunderscore fem.\textunderscore  de concunhado.
\section{Concunhado}
\begin{itemize}
\item {Grp. gram.:m.}
\end{itemize}
\begin{itemize}
\item {Proveniência:(De \textunderscore con...\textunderscore  + \textunderscore cunhado\textunderscore )}
\end{itemize}
Cunhado de um cônjuge, com relação a outro.
\section{Concupiscência}
\begin{itemize}
\item {Grp. gram.:f.}
\end{itemize}
\begin{itemize}
\item {Proveniência:(Lat. \textunderscore concupiscentia\textunderscore )}
\end{itemize}
Grande desejo de bens ou gozos materiáes.
Appetite sensual.
\section{Concupiscente}
\begin{itemize}
\item {Grp. gram.:adj.}
\end{itemize}
\begin{itemize}
\item {Proveniência:(Lat. \textunderscore concupiscens\textunderscore )}
\end{itemize}
Que tem concupiscência.
\section{Concupiscível}
\begin{itemize}
\item {Grp. gram.:adj.}
\end{itemize}
\begin{itemize}
\item {Proveniência:(Lat. \textunderscore concupiscibilis\textunderscore )}
\end{itemize}
Que póde despertar concupiscência.
\section{Concurso}
\begin{itemize}
\item {Grp. gram.:m.}
\end{itemize}
\begin{itemize}
\item {Proveniência:(Lat. \textunderscore concursus\textunderscore )}
\end{itemize}
Acto de concorrer.
Acto de se dirigirem muitas pessôas ao mesmo lugar ou fim; affluência.
Encontro.
Cooperação.
Certame.
Provas literárias, scientíficas ou artísticas, prestadas pelos que pretendem emprêgo ou certas concessões.
\section{Concussão}
\begin{itemize}
\item {Grp. gram.:f.}
\end{itemize}
\begin{itemize}
\item {Utilização:Fig.}
\end{itemize}
\begin{itemize}
\item {Proveniência:(Lat. \textunderscore concussio\textunderscore )}
\end{itemize}
Commoção forte.
Extorsão, peculato, commetido por empregado público, no exercicio das suas funcções.
\textunderscore Espoleta de concussão\textunderscore , espoleta, em que a inflammação é produzida pelo choque de uma substância, existente dentro da espoleta.
\section{Concussionário}
\begin{itemize}
\item {Grp. gram.:m.  e  adj.}
\end{itemize}
\begin{itemize}
\item {Proveniência:(Do lat. \textunderscore concussio\textunderscore )}
\end{itemize}
O que pratica concussão.
\section{Concutir}
\begin{itemize}
\item {Grp. gram.:v. t.}
\end{itemize}
\begin{itemize}
\item {Proveniência:(Lat. \textunderscore concutere\textunderscore )}
\end{itemize}
Abalar:«\textunderscore frade, cujo nome faz concutir as abóbadas do inferno\textunderscore ». Camillo, \textunderscore Bruxa\textunderscore , 196.
\section{Concutor}
\begin{itemize}
\item {Grp. gram.:m.}
\end{itemize}
\begin{itemize}
\item {Proveniência:(Do lat. \textunderscore concutere\textunderscore )}
\end{itemize}
Órgão das espoletas de concussão.
\section{Condado}
\begin{itemize}
\item {Grp. gram.:m.}
\end{itemize}
\begin{itemize}
\item {Proveniência:(Do lat. \textunderscore comitatus\textunderscore )}
\end{itemize}
Dignidade de Conde.
Antiga jurisdição ou território de Conde.
Antigo direito, que se pagava pela caça que se matava em terreno alheio.
\section{Condal}
\begin{itemize}
\item {Grp. gram.:adj.}
\end{itemize}
Relativo a Conde.
\section{Condão}
\begin{itemize}
\item {Grp. gram.:m.}
\end{itemize}
\begin{itemize}
\item {Proveniência:(Do rad. do lat. \textunderscore condonare\textunderscore , doar, fazer mercê de)}
\end{itemize}
Virtude especial, poder mysterioso, a que se attribue influência benéfica ou maléfica.
Dom; faculdade.
\textunderscore Vara\textunderscore  ou \textunderscore varinha de condão\textunderscore , vara mágica de feiticeiros e fadas.
\section{Condaria}
\begin{itemize}
\item {Grp. gram.:f.}
\end{itemize}
\begin{itemize}
\item {Utilização:Ant.}
\end{itemize}
O mesmo que \textunderscore condado\textunderscore .
\section{Condarim}
\begin{itemize}
\item {Grp. gram.:m.}
\end{itemize}
O mesmo que \textunderscore candorim\textunderscore .
\section{Conde}
\begin{itemize}
\item {Grp. gram.:m.}
\end{itemize}
\begin{itemize}
\item {Utilização:Pop.}
\end{itemize}
\begin{itemize}
\item {Proveniência:(Do lat. \textunderscore comes\textunderscore , \textunderscore comitis\textunderscore )}
\end{itemize}
Dignitário e commandante militar, no império Romano do Oriente.
Soberano de um condado na Idade-Média.
Título nobiliárchico, que em Portugal é inferior ao de Marquês e superior ao de Visconde.
Valete.
\textunderscore Pêra-do-conde\textunderscore , espécie de pêra grande.
\section{Condé}
\begin{itemize}
\item {Grp. gram.:m.}
\end{itemize}
Planta cucurbitácea do Brasil, também chamada \textunderscore guardião\textunderscore .
\section{Condeça}
\begin{itemize}
\item {fónica:dê}
\end{itemize}
\textunderscore f.\textunderscore  (e der.)
(V. \textunderscore condessa\textunderscore ^1, etc.)
\section{Condecoração}
\begin{itemize}
\item {Grp. gram.:f.}
\end{itemize}
Insígnia de Ordem militar.
Insígnia honorífica.
Acto de condecorar.
\section{Condecorado}
\begin{itemize}
\item {Grp. gram.:m.}
\end{itemize}
\begin{itemize}
\item {Proveniência:(De \textunderscore condecorar\textunderscore )}
\end{itemize}
Aquelle que usa ou recebeu condecoração.
\section{Condecorar}
\begin{itemize}
\item {Grp. gram.:v. t.}
\end{itemize}
\begin{itemize}
\item {Proveniência:(Lat. \textunderscore condecorare\textunderscore )}
\end{itemize}
Distinguir com condecoração.
Dar título honorífico a.
Nobilitar.
\section{Condeia}
\begin{itemize}
\item {Grp. gram.:f.}
\end{itemize}
Planta labiada da América.
O mesmo que \textunderscore condé\textunderscore ?
\section{Condeirelos}
\begin{itemize}
\item {Grp. gram.:m.}
\end{itemize}
\begin{itemize}
\item {Utilização:Prov.}
\end{itemize}
\begin{itemize}
\item {Utilização:trasm.}
\end{itemize}
Jôgo do \textunderscore pilha\textunderscore .
\section{Condemnação}
\begin{itemize}
\item {Grp. gram.:f.}
\end{itemize}
\begin{itemize}
\item {Proveniência:(Lat. \textunderscore condemnatio\textunderscore )}
\end{itemize}
Acto ou effeito de condemnar.
\section{Condemnado}
\begin{itemize}
\item {Grp. gram.:m.}
\end{itemize}
\begin{itemize}
\item {Proveniência:(De \textunderscore condemnar\textunderscore )}
\end{itemize}
Homem criminoso.
Aquelle que foi julgado criminoso.
\section{Condemnador}
\begin{itemize}
\item {Grp. gram.:m.  e  adj.}
\end{itemize}
\begin{itemize}
\item {Proveniência:(Lat. \textunderscore condemnator\textunderscore )}
\end{itemize}
O que condemna.
\section{Condemnar}
\begin{itemize}
\item {Grp. gram.:v. t.}
\end{itemize}
\begin{itemize}
\item {Proveniência:(Lat. \textunderscore condemnare\textunderscore )}
\end{itemize}
Pronunciar sentença contra.
Fixar a pena ou castigo de.
Mostrar que é criminoso.
Reprovar: \textunderscore condenmar uma ideia\textunderscore .
Forçar.
Julgar perdido.
\section{Condemnatório}
\begin{itemize}
\item {Grp. gram.:adj.}
\end{itemize}
Em que há condemnação: \textunderscore sentença condemnatória\textunderscore .
\section{Condemnável}
\begin{itemize}
\item {Grp. gram.:adj.}
\end{itemize}
\begin{itemize}
\item {Proveniência:(Lat. \textunderscore condemnabilis\textunderscore )}
\end{itemize}
Que merece condemnação.
\section{Condenação}
\begin{itemize}
\item {Grp. gram.:f.}
\end{itemize}
\begin{itemize}
\item {Proveniência:(Lat. \textunderscore condemnatio\textunderscore )}
\end{itemize}
Acto ou efeito de condenar.
\section{Condenado}
\begin{itemize}
\item {Grp. gram.:m.}
\end{itemize}
\begin{itemize}
\item {Proveniência:(De \textunderscore condenar\textunderscore )}
\end{itemize}
Homem criminoso.
Aquele que foi julgado criminoso.
\section{Condenador}
\begin{itemize}
\item {Grp. gram.:m.  e  adj.}
\end{itemize}
\begin{itemize}
\item {Proveniência:(Lat. \textunderscore condemnator\textunderscore )}
\end{itemize}
O que condena.
\section{Condenar}
\begin{itemize}
\item {Grp. gram.:v. t.}
\end{itemize}
\begin{itemize}
\item {Proveniência:(Lat. \textunderscore condemnare\textunderscore )}
\end{itemize}
Pronunciar sentença contra.
Fixar a pena ou castigo de.
Mostrar que é criminoso.
Reprovar: \textunderscore condenar uma ideia\textunderscore .
Forçar.
Julgar perdido.
\section{Condenatório}
\begin{itemize}
\item {Grp. gram.:adj.}
\end{itemize}
Em que há condenação: \textunderscore sentença condenatória\textunderscore .
\section{Condenável}
\begin{itemize}
\item {Grp. gram.:adj.}
\end{itemize}
\begin{itemize}
\item {Proveniência:(Lat. \textunderscore condemnabilis\textunderscore )}
\end{itemize}
Que merece condenação.
\section{Condensabilidade}
\begin{itemize}
\item {Grp. gram.:f.}
\end{itemize}
Propriedade de se condensar.
\section{Condensação}
\begin{itemize}
\item {Grp. gram.:f.}
\end{itemize}
\begin{itemize}
\item {Proveniência:(Lat. \textunderscore condensatio\textunderscore )}
\end{itemize}
Acto ou effeito de condensar.
\section{Condensador}
\begin{itemize}
\item {Grp. gram.:adj.}
\end{itemize}
\begin{itemize}
\item {Grp. gram.:M.}
\end{itemize}
Que condensa.
Instrumento que condensa.
Parte de certas máquinas em que se condensa o vapor.
\section{Condensar}
\begin{itemize}
\item {Grp. gram.:v. t.}
\end{itemize}
\begin{itemize}
\item {Utilização:Fig.}
\end{itemize}
\begin{itemize}
\item {Proveniência:(Lat. \textunderscore condensare\textunderscore )}
\end{itemize}
Tornar mais denso, fazer espêsso.
Resumir; exprimir em poucas palavras.
\section{Condensativo}
\begin{itemize}
\item {Grp. gram.:adj.}
\end{itemize}
Que condensa.
\section{Condensável}
\begin{itemize}
\item {Grp. gram.:adj.}
\end{itemize}
Que se póde condensar.
\section{Condensor}
\begin{itemize}
\item {Grp. gram.:m.}
\end{itemize}
(V.condensador)
\section{Condescendência}
\begin{itemize}
\item {Grp. gram.:f.}
\end{itemize}
Acção de condescender.
Qualidade de quem é condescendente.
\section{Condescendente}
\begin{itemize}
\item {Grp. gram.:adj.}
\end{itemize}
\begin{itemize}
\item {Proveniência:(Lat. \textunderscore condescendens\textunderscore )}
\end{itemize}
Que condescende.
\section{Condescender}
\begin{itemize}
\item {Grp. gram.:v. i.}
\end{itemize}
\begin{itemize}
\item {Proveniência:(Lat. \textunderscore condescendere\textunderscore )}
\end{itemize}
Transigir espontaneamente.
Annuir de vontade própria á vontade alheia.
Annuir.
\section{Condescendimento}
\begin{itemize}
\item {Grp. gram.:m.}
\end{itemize}
(V.condescendência)
\section{Condessa}
\begin{itemize}
\item {fónica:dé}
\end{itemize}
\begin{itemize}
\item {Grp. gram.:f.}
\end{itemize}
\begin{itemize}
\item {Proveniência:(Lat. \textunderscore condensa\textunderscore )}
\end{itemize}
Pequena cesta de vimes ou vêrga, com tampa.
\section{Condessa}
\begin{itemize}
\item {fónica:dê}
\end{itemize}
\begin{itemize}
\item {Grp. gram.:f.}
\end{itemize}
\begin{itemize}
\item {Utilização:Prov.}
\end{itemize}
\begin{itemize}
\item {Utilização:alent.}
\end{itemize}
\begin{itemize}
\item {Proveniência:(Do b. lat. \textunderscore comitissa\textunderscore )}
\end{itemize}
Mulher de Conde.
Aquella que recebeu título honorífico, correspondente ao de Conde.
Antiga senhora de condado.
Planta anonácea do Brasil.
Variedade de pêra.
\section{Condessar}
\begin{itemize}
\item {Grp. gram.:adj.}
\end{itemize}
\begin{itemize}
\item {Utilização:Ant.}
\end{itemize}
\begin{itemize}
\item {Proveniência:(De \textunderscore condessa\textunderscore ^1)}
\end{itemize}
Guardar; pôr em depósito.
\section{Condessilha}
\begin{itemize}
\item {Grp. gram.:f.}
\end{itemize}
\begin{itemize}
\item {Utilização:Ant.}
\end{itemize}
O mesmo que \textunderscore condessilho\textunderscore .
\section{Condessilho}
\begin{itemize}
\item {Grp. gram.:m.}
\end{itemize}
\begin{itemize}
\item {Utilização:Ant.}
\end{itemize}
\begin{itemize}
\item {Proveniência:(De \textunderscore condessa\textunderscore ^1)}
\end{itemize}
Guarda; depósito.
Recato, cuidado.
\section{Condessinha}
\begin{itemize}
\item {Grp. gram.:f.}
\end{itemize}
Jôgo popular.
O mesmo que \textunderscore la-condessa\textunderscore .
\section{Condesso}
\begin{itemize}
\item {fónica:dê}
\end{itemize}
\begin{itemize}
\item {Grp. gram.:m.}
\end{itemize}
\begin{itemize}
\item {Utilização:Chul.}
\end{itemize}
Marido, sem título, de uma mulher que é condessa.
(Cp. \textunderscore condessa\textunderscore ^2)
\section{Condestable}
\begin{itemize}
\item {Grp. gram.:m.}
\end{itemize}
\begin{itemize}
\item {Utilização:Ant.}
\end{itemize}
O mesmo que \textunderscore condestável\textunderscore .
\section{Condestablessa}
\begin{itemize}
\item {Grp. gram.:f.}
\end{itemize}
\begin{itemize}
\item {Utilização:Ant.}
\end{itemize}
O mesmo que \textunderscore condestabresa\textunderscore .
\section{Condestabre}
\begin{itemize}
\item {Grp. gram.:m.}
\end{itemize}
\begin{itemize}
\item {Utilização:Ant.}
\end{itemize}
O mesmo que \textunderscore condestável\textunderscore .
\section{Condestabresa}
\begin{itemize}
\item {fónica:brê}
\end{itemize}
\begin{itemize}
\item {Grp. gram.:f.}
\end{itemize}
\begin{itemize}
\item {Utilização:Ant.}
\end{itemize}
Mulher de condestabre.
\section{Condestável}
\begin{itemize}
\item {Grp. gram.:m.}
\end{itemize}
Antigamente, chefe superior do exército.
Chefe de artilheiros.
Antigo intendente das cavallariças reaes.
Título de infante, que nas grandes solennidades se collocava á direita do throno real.
(Da loc. lat. \textunderscore comes stabuli\textunderscore )
\section{Condezilha}
\begin{itemize}
\item {Grp. gram.:f.}
\end{itemize}
(V.condecilha)
\section{Condi}
\begin{itemize}
\item {Grp. gram.:m.}
\end{itemize}
Vara graduada, para medições, na Índia portuguêsa.
\section{Condição}
\begin{itemize}
\item {Grp. gram.:f.}
\end{itemize}
\begin{itemize}
\item {Proveniência:(Lat. \textunderscore conditio\textunderscore )}
\end{itemize}
Classe social: \textunderscore homem de baixa condição\textunderscore .
Circunstância, situação: \textunderscore em taes condições, não há que esperar\textunderscore .
Carácter.
Qualidade que se requere.
Cláusula: \textunderscore perdoou-lhe, sob a condição de não reíncidir\textunderscore .
Distincção, categoria elevada: \textunderscore pessôas de condição\textunderscore .
\section{Condicente}
\begin{itemize}
\item {Grp. gram.:adj.}
\end{itemize}
Que condiz, que está de acôrdo. Cf. Arn. Gama, \textunderscore Motim\textunderscore , 284.
\section{Condicional}
\begin{itemize}
\item {Grp. gram.:adj.}
\end{itemize}
\begin{itemize}
\item {Proveniência:(Lat. \textunderscore conditionalis\textunderscore )}
\end{itemize}
Dependente de condição.
Que envolve condição, que exprime circunstância de condição.
\section{Condicionalmente}
\begin{itemize}
\item {Grp. gram.:adv.}
\end{itemize}
De modo condicional.
\section{Condicionamento}
\begin{itemize}
\item {Grp. gram.:m.}
\end{itemize}
Conjunto das condições, em que se realiza um facto; circunstâncias. Cf. Sousa Martins, \textunderscore Nesographia\textunderscore .
\section{Condicionar}
\begin{itemize}
\item {Grp. gram.:v. t.}
\end{itemize}
\begin{itemize}
\item {Proveniência:(Do lat. \textunderscore conditio\textunderscore )}
\end{itemize}
Tornar dependente de condição.
Acondicionar.
\section{Condiçoar}
\begin{itemize}
\item {Grp. gram.:v. t.}
\end{itemize}
\begin{itemize}
\item {Utilização:Ant.}
\end{itemize}
Pôr por condição.
\section{Condignamente}
\begin{itemize}
\item {Grp. gram.:adv.}
\end{itemize}
De modo condigno.
\section{Condignidade}
\begin{itemize}
\item {Grp. gram.:f.}
\end{itemize}
Qualidade de condigno.
\section{Condigno}
\begin{itemize}
\item {Grp. gram.:adj.}
\end{itemize}
\begin{itemize}
\item {Proveniência:(Lat. \textunderscore condignus\textunderscore )}
\end{itemize}
Em que há a dignidade conveniente.
Proporcional ao mérito: \textunderscore prêmio condigno\textunderscore .
\section{Condiliano}
\begin{itemize}
\item {Grp. gram.:adj.}
\end{itemize}
Relativo a \textunderscore côndilo\textunderscore .
\section{Côndilo}
\begin{itemize}
\item {Grp. gram.:m.}
\end{itemize}
\begin{itemize}
\item {Utilização:Anat.}
\end{itemize}
\begin{itemize}
\item {Proveniência:(Gr. \textunderscore kondulos\textunderscore )}
\end{itemize}
Saliência articular de um osso, arredondado de um lado e achatado de outro.
\section{Condilóforo}
\begin{itemize}
\item {Grp. gram.:adj.}
\end{itemize}
\begin{itemize}
\item {Proveniência:(Do gr. \textunderscore kondulos\textunderscore  + \textunderscore phoros\textunderscore )}
\end{itemize}
Que tem nós, (falando-se de vegetaes).
\section{Condiloide}
\begin{itemize}
\item {Grp. gram.:adj.}
\end{itemize}
\begin{itemize}
\item {Proveniência:(Do gr. \textunderscore kondulos\textunderscore  + \textunderscore eidos\textunderscore )}
\end{itemize}
Que tem a fórma de côndilo.
\section{Condiloma}
\begin{itemize}
\item {Grp. gram.:f.}
\end{itemize}
\begin{itemize}
\item {Proveniência:(Gr. \textunderscore konduloma\textunderscore )}
\end{itemize}
Pequeno tumor no ânus, determinado pela hipertrofia da derme.
\section{Condimentação}
\begin{itemize}
\item {Grp. gram.:f.}
\end{itemize}
Acto ou effeito de condimentar.
\section{Condimentar}
\begin{itemize}
\item {Grp. gram.:v. t.}
\end{itemize}
Deitar condimento em; adubar; temperar.
\section{Condimento}
\begin{itemize}
\item {Grp. gram.:m.}
\end{itemize}
\begin{itemize}
\item {Proveniência:(Lat. \textunderscore condimentum\textunderscore )}
\end{itemize}
Aquillo que serve para temperar alguma coisa.
Tempêro, adubo.
\section{Condimentoso}
\begin{itemize}
\item {Grp. gram.:adj.}
\end{itemize}
\begin{itemize}
\item {Proveniência:(De \textunderscore condimento\textunderscore )}
\end{itemize}
Que condimenta.
\section{Condir}
\begin{itemize}
\item {Grp. gram.:v. t.}
\end{itemize}
\begin{itemize}
\item {Proveniência:(Lat. \textunderscore condire\textunderscore )}
\end{itemize}
Temperar (medicamentos).
\section{Condiscipulado}
\begin{itemize}
\item {Grp. gram.:m.}
\end{itemize}
\begin{itemize}
\item {Proveniência:(Lat. \textunderscore condiscipulatus\textunderscore )}
\end{itemize}
Qualidade de sêr condiscípulo.
Tempo, em que se é condiscípulo.
Sociedade escolar.
\section{Condiscípulo}
\begin{itemize}
\item {Grp. gram.:m.}
\end{itemize}
\begin{itemize}
\item {Proveniência:(Lat. \textunderscore condiscipulus\textunderscore )}
\end{itemize}
Aquelle que estuda ou apprende com outrem.
Companheiro de escola.
\section{Condizente}
\begin{itemize}
\item {Grp. gram.:adj.}
\end{itemize}
\begin{itemize}
\item {Proveniência:(Lat. \textunderscore condicens\textunderscore )}
\end{itemize}
Que condiz.
\section{Condizer}
\begin{itemize}
\item {Grp. gram.:v. i.}
\end{itemize}
\begin{itemize}
\item {Proveniência:(Lat. \textunderscore condicere\textunderscore )}
\end{itemize}
Estar em proporção, em harmonia: \textunderscore essas allegações não condizem\textunderscore .
\section{Condoer}
\begin{itemize}
\item {Grp. gram.:v. t.}
\end{itemize}
\begin{itemize}
\item {Grp. gram.:V. p.}
\end{itemize}
\begin{itemize}
\item {Proveniência:(Lat. \textunderscore condolere\textunderscore )}
\end{itemize}
Despertar compaixão em; excitar á dôr.
Compadecer-se.
\section{Condoimento}
\begin{itemize}
\item {fónica:do-i}
\end{itemize}
\begin{itemize}
\item {Grp. gram.:m.}
\end{itemize}
O mesmo que \textunderscore condolência\textunderscore .
\section{Condoível}
\begin{itemize}
\item {Grp. gram.:adj.}
\end{itemize}
Capaz de se condoer. Cf. Filinto, XX, 178.
\section{Condolência}
\begin{itemize}
\item {Grp. gram.:f.}
\end{itemize}
\begin{itemize}
\item {Proveniência:(De \textunderscore condolente\textunderscore )}
\end{itemize}
Estado de quem se condói; compaixão.
Sentimento de pesar.
\section{Condolente}
\begin{itemize}
\item {Grp. gram.:adj.}
\end{itemize}
\begin{itemize}
\item {Proveniência:(Lat. \textunderscore condolens\textunderscore )}
\end{itemize}
Que tem compaixão.
\section{Condoma}
\begin{itemize}
\item {Grp. gram.:m.}
\end{itemize}
Antílope do Cabo da Bôa-Esperança.
\section{Condomínio}
\begin{itemize}
\item {Grp. gram.:m.}
\end{itemize}
\begin{itemize}
\item {Proveniência:(De \textunderscore con...\textunderscore  + \textunderscore dominio\textunderscore )}
\end{itemize}
Domínio, que pertence a mais de uma pessôa juntamente.
\section{Condómino}
\begin{itemize}
\item {Grp. gram.:m.}
\end{itemize}
\begin{itemize}
\item {Utilização:Neol.}
\end{itemize}
O mesmo que \textunderscore comproprietário\textunderscore . Cf. \textunderscore Proj. de Cod. Civ. Bras.\textunderscore , art. 1140.
\section{Condonatário}
\begin{itemize}
\item {Grp. gram.:m.}
\end{itemize}
\begin{itemize}
\item {Proveniência:(De \textunderscore con...\textunderscore  + \textunderscore donatário\textunderscore )}
\end{itemize}
Aquelle que faz doação com outrem.
\section{Condor}
\begin{itemize}
\item {Grp. gram.:m.}
\end{itemize}
Corpulenta ave de rapina, (\textunderscore vultur gryphus\textunderscore ).
(Do quichua \textunderscore kuntur\textunderscore )
\section{Condoreiro}
\begin{itemize}
\item {Grp. gram.:m.}
\end{itemize}
\begin{itemize}
\item {Utilização:bras}
\end{itemize}
\begin{itemize}
\item {Utilização:Neol.}
\end{itemize}
\begin{itemize}
\item {Proveniência:(De \textunderscore condor\textunderscore )}
\end{itemize}
Diz-se do estilo elevado ou empolado.
\section{Condottiere}
\begin{itemize}
\item {Grp. gram.:m.}
\end{itemize}
\begin{itemize}
\item {Proveniência:(T. it.)}
\end{itemize}
Chefe de guerrilheiros de ladrões ou de bandidos. Cf. Herculano, \textunderscore Hist de Port.\textunderscore , III, 161; Latino, \textunderscore Elogios\textunderscore , 378.
\section{Condral}
\begin{itemize}
\item {Grp. gram.:adj.}
\end{itemize}
Relativo ao condro.
\section{Condrificação}
\begin{itemize}
\item {Grp. gram.:f.}
\end{itemize}
Acto ou efeito de condrificar-se.
\section{Condrificar-se}
\begin{itemize}
\item {Grp. gram.:v. p.}
\end{itemize}
\begin{itemize}
\item {Proveniência:(T. hybr., do gr. \textunderscore khondros\textunderscore  + lat. \textunderscore facere\textunderscore )}
\end{itemize}
Tornar-se cartilaginoso.
\section{Condrilha}
\begin{itemize}
\item {Grp. gram.:f.}
\end{itemize}
\begin{itemize}
\item {Proveniência:(Do gr. \textunderscore khondros\textunderscore , grão)}
\end{itemize}
Gênero de plantas chicoriáceas.
\section{Condrina}
\begin{itemize}
\item {Grp. gram.:f.}
\end{itemize}
\begin{itemize}
\item {Proveniência:(Do gr. \textunderscore khondros\textunderscore )}
\end{itemize}
Substância, que se extrai de certas cartilagens.
\section{Condro}
\begin{itemize}
\item {Grp. gram.:m.}
\end{itemize}
\begin{itemize}
\item {Proveniência:(Gr. \textunderscore khondros\textunderscore )}
\end{itemize}
Designação científica da cartilagem.
\section{Condrografia}
\begin{itemize}
\item {Grp. gram.:f.}
\end{itemize}
\begin{itemize}
\item {Proveniência:(Do gr. \textunderscore khondros\textunderscore  + \textunderscore graphein\textunderscore )}
\end{itemize}
Descripção das cartilagens.
\section{Condroide}
\begin{itemize}
\item {Grp. gram.:adj.}
\end{itemize}
\begin{itemize}
\item {Proveniência:(Do gr. \textunderscore khondros\textunderscore  + \textunderscore eidos\textunderscore )}
\end{itemize}
Semelhante a cartilagens.
\section{Condrologia}
\begin{itemize}
\item {Grp. gram.:f.}
\end{itemize}
\begin{itemize}
\item {Proveniência:(Do gr. \textunderscore khondros\textunderscore  + \textunderscore logos\textunderscore )}
\end{itemize}
Tratado das cartilagens.
\section{Condroma}
\begin{itemize}
\item {Grp. gram.:m.}
\end{itemize}
\begin{itemize}
\item {Proveniência:(Do gr. \textunderscore khondros\textunderscore )}
\end{itemize}
Tumor cartilaginoso.
\section{Condropterígios}
\begin{itemize}
\item {Grp. gram.:m. pl.}
\end{itemize}
\begin{itemize}
\item {Proveniência:(Do gr. \textunderscore khondros\textunderscore  + \textunderscore pterux\textunderscore )}
\end{itemize}
Peixes, caracterizados por terem esqueleto cartilaginoso.
\section{Condrotomia}
\begin{itemize}
\item {Grp. gram.:f.}
\end{itemize}
\begin{itemize}
\item {Proveniência:(Do gr. \textunderscore khondros\textunderscore  + \textunderscore tome\textunderscore )}
\end{itemize}
Dissecação das cartilagens.
\section{Condução}
\begin{itemize}
\item {Grp. gram.:f.}
\end{itemize}
\begin{itemize}
\item {Proveniência:(Lat. \textunderscore conductio\textunderscore )}
\end{itemize}
Acto, efeito ou meio de conduzir.
Transporte: \textunderscore condução das malas do correio\textunderscore .
\section{Conducção}
\begin{itemize}
\item {Grp. gram.:f.}
\end{itemize}
\begin{itemize}
\item {Proveniência:(Lat. \textunderscore conductio\textunderscore )}
\end{itemize}
Acto, effeito ou meio de conduzir.
Transporte: \textunderscore conducção das malas do correio\textunderscore .
\section{Conducente}
\begin{itemize}
\item {Grp. gram.:adj.}
\end{itemize}
\begin{itemize}
\item {Proveniência:(Lat. \textunderscore conducens\textunderscore )}
\end{itemize}
Que conduz (a um fim); tendente.
\section{Conducta}
\begin{itemize}
\item {Grp. gram.:f.}
\end{itemize}
\begin{itemize}
\item {Utilização:Ant.}
\end{itemize}
\begin{itemize}
\item {Utilização:Des.}
\end{itemize}
\begin{itemize}
\item {Proveniência:(Do lat. \textunderscore conductus\textunderscore )}
\end{itemize}
Conducção.
Conjunto de pessôas, conduzidas para algum lugar.
O mesmo que \textunderscore conducto\textunderscore , canal.
Superintendência.--Com a significação de \textunderscore procedimento\textunderscore , é gallicismo inútil.
\section{Conductar}
\begin{itemize}
\item {Grp. gram.:v. t.}
\end{itemize}
\begin{itemize}
\item {Utilização:Fig.}
\end{itemize}
\begin{itemize}
\item {Proveniência:(De \textunderscore conducto\textunderscore )}
\end{itemize}
Comer (pão) com algum conducto.
Gastar a pouco e pouco, poupar.
\section{Conducteiro}
\begin{itemize}
\item {Grp. gram.:m.}
\end{itemize}
\begin{itemize}
\item {Utilização:Ant.}
\end{itemize}
Serviçal assalariado.
\section{Conductibilidade}
\begin{itemize}
\item {Grp. gram.:f.}
\end{itemize}
\begin{itemize}
\item {Utilização:Phýs.}
\end{itemize}
\begin{itemize}
\item {Proveniência:(Do lat. \textunderscore conductus\textunderscore )}
\end{itemize}
Propriedade, que os corpos têm, de sêr conductores de calor, electricidade, etc.
\section{Conductício}
\begin{itemize}
\item {Grp. gram.:adj.}
\end{itemize}
\begin{itemize}
\item {Proveniência:(Lat. \textunderscore conducticius\textunderscore )}
\end{itemize}
Alugado.
Assoldadado.
\section{Conductível}
\begin{itemize}
\item {Grp. gram.:adj.}
\end{itemize}
\begin{itemize}
\item {Proveniência:(De \textunderscore conducto\textunderscore )}
\end{itemize}
Que póde sêr conduzido.
Que tem conductibilidade.
\section{Conductivo}
\begin{itemize}
\item {Grp. gram.:adj.}
\end{itemize}
\begin{itemize}
\item {Proveniência:(De \textunderscore conducto\textunderscore )}
\end{itemize}
Que conduz.
\section{Conducto}
\begin{itemize}
\item {Grp. gram.:m.}
\end{itemize}
\begin{itemize}
\item {Utilização:Pop.}
\end{itemize}
\begin{itemize}
\item {Proveniência:(Lat. \textunderscore conductus\textunderscore )}
\end{itemize}
Via; canal.
Aquillo que se come habitualmente com pão.
\section{Conductor}
\begin{itemize}
\item {Grp. gram.:m.}
\end{itemize}
\begin{itemize}
\item {Utilização:Prov.}
\end{itemize}
\begin{itemize}
\item {Utilização:minh.}
\end{itemize}
\begin{itemize}
\item {Proveniência:(Lat. \textunderscore conductor\textunderscore )}
\end{itemize}
Aquelle que conduz.
Guia.
Corpo, que é transmissor de calor, electricidade, etc.
Categoria do funccionários de obras públicas.
Jarro ou regador de lavatório.
\section{Conduplicação}
\begin{itemize}
\item {Grp. gram.:f.}
\end{itemize}
\begin{itemize}
\item {Proveniência:(Lat. \textunderscore conduplicatio\textunderscore )}
\end{itemize}
Repetição de palavra no príncípio ou meio da phrase.
\section{Conduplicado}
\begin{itemize}
\item {Grp. gram.:adj.}
\end{itemize}
\begin{itemize}
\item {Proveniência:(Lat. \textunderscore conduplicatus\textunderscore )}
\end{itemize}
Dobrado em duas partes longitudinalmente.
\section{Condurango}
\begin{itemize}
\item {Grp. gram.:m.}
\end{itemize}
\begin{itemize}
\item {Utilização:Bras}
\end{itemize}
Planta asclepiadácea, medicinal, (\textunderscore gonolobus condurango\textunderscore ).
\section{Conduru}
\begin{itemize}
\item {Grp. gram.:m.}
\end{itemize}
Árvore urticácea do Brasil.
\section{Conduta}
\begin{itemize}
\item {Grp. gram.:f.}
\end{itemize}
\begin{itemize}
\item {Utilização:Ant.}
\end{itemize}
\begin{itemize}
\item {Utilização:Des.}
\end{itemize}
\begin{itemize}
\item {Proveniência:(Do lat. \textunderscore conductus\textunderscore )}
\end{itemize}
Condução.
Conjunto de pessôas, conduzidas para algum lugar.
O mesmo que \textunderscore conduto\textunderscore , canal.
Superintendência.--Com a significação de \textunderscore procedimento\textunderscore , é galicismo inútil.
\section{Condutar}
\begin{itemize}
\item {Grp. gram.:v. t.}
\end{itemize}
\begin{itemize}
\item {Utilização:Fig.}
\end{itemize}
\begin{itemize}
\item {Proveniência:(De \textunderscore conducto\textunderscore )}
\end{itemize}
Comer (pão) com algum conduto.
Gastar a pouco e pouco, poupar.
\section{Conduteiro}
\begin{itemize}
\item {Grp. gram.:m.}
\end{itemize}
\begin{itemize}
\item {Utilização:Ant.}
\end{itemize}
Serviçal assalariado.
\section{Condutibilidade}
\begin{itemize}
\item {Grp. gram.:f.}
\end{itemize}
\begin{itemize}
\item {Utilização:Phýs.}
\end{itemize}
\begin{itemize}
\item {Proveniência:(Do lat. \textunderscore conductus\textunderscore )}
\end{itemize}
Propriedade, que os corpos têm, de sêr condutores de calor, electricidade, etc.
\section{Condutício}
\begin{itemize}
\item {Grp. gram.:adj.}
\end{itemize}
\begin{itemize}
\item {Proveniência:(Lat. \textunderscore conducticius\textunderscore )}
\end{itemize}
Alugado.
Assoldadado.
\section{Condutível}
\begin{itemize}
\item {Grp. gram.:adj.}
\end{itemize}
\begin{itemize}
\item {Proveniência:(De \textunderscore conduto\textunderscore )}
\end{itemize}
Que póde sêr conduzido.
Que tem condutibilidade.
\section{Condutivo}
\begin{itemize}
\item {Grp. gram.:adj.}
\end{itemize}
\begin{itemize}
\item {Proveniência:(De \textunderscore conduto\textunderscore )}
\end{itemize}
Que conduz.
\section{Conduto}
\begin{itemize}
\item {Grp. gram.:m.}
\end{itemize}
\begin{itemize}
\item {Utilização:Pop.}
\end{itemize}
\begin{itemize}
\item {Proveniência:(Lat. \textunderscore conductus\textunderscore )}
\end{itemize}
Via; canal.
Aquilo que se come habitualmente com pão.
\section{Condutor}
\begin{itemize}
\item {Grp. gram.:m.}
\end{itemize}
\begin{itemize}
\item {Utilização:Prov.}
\end{itemize}
\begin{itemize}
\item {Utilização:minh.}
\end{itemize}
\begin{itemize}
\item {Proveniência:(Lat. \textunderscore conductor\textunderscore )}
\end{itemize}
Aquele que conduz.
Guia.
Corpo, que é transmissor de calor, electricidade, etc.
Categoria do funcionários de obras públicas.
Jarro ou regador de lavatório.
\section{Conduzir}
\begin{itemize}
\item {Grp. gram.:v. t.}
\end{itemize}
\begin{itemize}
\item {Grp. gram.:V. i.}
\end{itemize}
\begin{itemize}
\item {Proveniência:(Lat. \textunderscore conducere\textunderscore )}
\end{itemize}
Levar ou trazer, dirigindo.
Guiar: \textunderscore conduzir uma junta de bois\textunderscore .
Transportar.
Trasmittir.
Ir têr a um fim.
Prolongar-se até certo ponto, (falando-se de um caminho ou estrada): \textunderscore aquella estrada conduz ao Pôrto\textunderscore .
\section{Condyliano}
\begin{itemize}
\item {Grp. gram.:adj.}
\end{itemize}
Relativo a \textunderscore côndylo\textunderscore .
\section{Côndylo}
\begin{itemize}
\item {Grp. gram.:m.}
\end{itemize}
\begin{itemize}
\item {Utilização:Anat.}
\end{itemize}
\begin{itemize}
\item {Proveniência:(Gr. \textunderscore kondulos\textunderscore )}
\end{itemize}
Saliência articular de um osso, arredondado de um lado e achatado de outro.
\section{Condyloide}
\begin{itemize}
\item {Grp. gram.:adj.}
\end{itemize}
\begin{itemize}
\item {Proveniência:(Do gr. \textunderscore kondulos\textunderscore  + \textunderscore eidos\textunderscore )}
\end{itemize}
Que tem a fórma de côndylo.
\section{Condyloma}
\begin{itemize}
\item {Grp. gram.:f.}
\end{itemize}
\begin{itemize}
\item {Proveniência:(Gr. \textunderscore konduloma\textunderscore )}
\end{itemize}
Pequeno tumor no ânus, determinado pela hyperthrophia da derme.
\section{Condylóphoro}
\begin{itemize}
\item {Grp. gram.:adj.}
\end{itemize}
\begin{itemize}
\item {Proveniência:(Do gr. \textunderscore kondulos\textunderscore  + \textunderscore phoros\textunderscore )}
\end{itemize}
Que tem nós, (falando-se de vegetaes).
\section{Cóne}
\begin{itemize}
\item {Grp. gram.:m.}
\end{itemize}
\begin{itemize}
\item {Utilização:Geom.}
\end{itemize}
\begin{itemize}
\item {Proveniência:(Gr. \textunderscore konos\textunderscore )}
\end{itemize}
Sólido, de base circular ou ellíptica, terminando em ponta.
Fruto que, como o do cypreste, é formado de sementes nuas em massas cónicas, ovoides ou globulosas, cobertas por brácteas escamosas e duras.
\section{Cónega}
\begin{itemize}
\item {Grp. gram.:f.}
\end{itemize}
\begin{itemize}
\item {Utilização:Ant.}
\end{itemize}
Mulher, que fazia parte de um cabido de religiosas.
(Fem. de \textunderscore cónego\textunderscore )
\section{Conéga}
\begin{itemize}
\item {Grp. gram.:f.}
\end{itemize}
Dialecto dos habitantes da ilha de Kodiak.
\section{Cónego}
\begin{itemize}
\item {Grp. gram.:m.}
\end{itemize}
\begin{itemize}
\item {Proveniência:(Do lat. \textunderscore canonicus\textunderscore )}
\end{itemize}
Clérigo secular, que faz parte de um cabido, e ao qual impendem obrigações religiosas numa sé ou collegiada.
\section{Conezia}
\begin{itemize}
\item {Grp. gram.:f.}
\end{itemize}
\begin{itemize}
\item {Utilização:Fig.}
\end{itemize}
\begin{itemize}
\item {Proveniência:(Do lat. hyp. \textunderscore canonicia\textunderscore , de \textunderscore canonicus\textunderscore )}
\end{itemize}
Canonicato.
Sinecura.
\section{Confabulação}
\begin{itemize}
\item {Grp. gram.:f.}
\end{itemize}
Acto de confabular.
\section{Confabulador}
\begin{itemize}
\item {Grp. gram.:m.}
\end{itemize}
Aquelle que confabula.
\section{Confabular}
\begin{itemize}
\item {Grp. gram.:v. t.  e  i.}
\end{itemize}
\begin{itemize}
\item {Utilização:Des.}
\end{itemize}
\begin{itemize}
\item {Proveniência:(Lat. \textunderscore confabulari\textunderscore )}
\end{itemize}
O mesmo que \textunderscore conversar\textunderscore .
\section{Confarreação}
\begin{itemize}
\item {Grp. gram.:f.}
\end{itemize}
\begin{itemize}
\item {Proveniência:(Lat. \textunderscore confarreatio\textunderscore )}
\end{itemize}
Fórma antiga e solenne de casamento entre os romanos, celebrado com offerta de pão e na presença de déz testemunhas.
\section{Confecalamei}
\begin{itemize}
\item {Grp. gram.:m.}
\end{itemize}
\begin{itemize}
\item {Utilização:Ant.}
\end{itemize}
Erva-doce. Cf. \textunderscore Lembrança das Cousas da Índia\textunderscore , nos \textunderscore Subsídios\textunderscore  de Felner.
\section{Confecção}
\begin{itemize}
\item {Grp. gram.:f.}
\end{itemize}
\begin{itemize}
\item {Grp. gram.:Pl.}
\end{itemize}
\begin{itemize}
\item {Proveniência:(Lat. \textunderscore confectio\textunderscore )}
\end{itemize}
Acto de confeccionar.
Na significação de trabalhos de modista, é gallicismo injustificável.
\section{Confeccionador}
\begin{itemize}
\item {Grp. gram.:m.}
\end{itemize}
Aquelle que confecciona.
\section{Confeccionar}
\begin{itemize}
\item {Grp. gram.:v. t.}
\end{itemize}
\begin{itemize}
\item {Proveniência:(Do lat. \textunderscore confectio\textunderscore )}
\end{itemize}
O mesmo que \textunderscore confeiçoar\textunderscore .
\section{Confederação}
\begin{itemize}
\item {Grp. gram.:f.}
\end{itemize}
\begin{itemize}
\item {Proveniência:(De \textunderscore con...\textunderscore  + \textunderscore federação\textunderscore )}
\end{itemize}
Acto ou effeito de confederar.
Reunião de Estados, que, em relação aos estrangeiros fórmam um só, reconhecendo um chefe commum.
Alliança de várias nações para um fim commum.
\section{Confederar}
\begin{itemize}
\item {Grp. gram.:v. t.}
\end{itemize}
\begin{itemize}
\item {Proveniência:(De \textunderscore con...\textunderscore  + \textunderscore federar\textunderscore )}
\end{itemize}
Unir em confederação.
\section{Confederativo}
\begin{itemize}
\item {Grp. gram.:adj.}
\end{itemize}
Relativo a confederação.
\section{Confeição}
\begin{itemize}
\item {Grp. gram.:f.}
\end{itemize}
\begin{itemize}
\item {Proveniência:(Lat. \textunderscore confectio\textunderscore )}
\end{itemize}
Acto ou effeito de confeiçoar.
\section{Confeiçoar}
\begin{itemize}
\item {Grp. gram.:v. t.}
\end{itemize}
\begin{itemize}
\item {Utilização:Des.}
\end{itemize}
\begin{itemize}
\item {Proveniência:(De \textunderscore confeição\textunderscore )}
\end{itemize}
Preparar (medicamentos) com várias drogas.
Manipular (bolos e outros objectos de confeitaria).
Juntar substância estranha a (vinhos).
Fazer completamente, concluir.
\section{Confeitada}
\begin{itemize}
\item {Grp. gram.:f.}
\end{itemize}
\begin{itemize}
\item {Utilização:Prov.}
\end{itemize}
\begin{itemize}
\item {Utilização:alent.}
\end{itemize}
\begin{itemize}
\item {Proveniência:(De \textunderscore confeito\textunderscore )}
\end{itemize}
Presente de amêndoas confeitadas, que é costume dar-se em Quinta-Feira Santa.
\section{Confeitado}
\begin{itemize}
\item {Grp. gram.:adj.}
\end{itemize}
\begin{itemize}
\item {Proveniência:(De \textunderscore confeitar\textunderscore )}
\end{itemize}
Diz-se da semente ou pevide, coberta com açúcar e convertido numa espécie de pastilha doce e mais ou menos esphérica.
E diz-se de certas pastilhas doces, em que não entra pevide ou semente.
\section{Confeitar}
\begin{itemize}
\item {Grp. gram.:v. t.}
\end{itemize}
\begin{itemize}
\item {Utilização:Fig.}
\end{itemize}
\begin{itemize}
\item {Proveniência:(De \textunderscore confeito\textunderscore )}
\end{itemize}
Cobrir com açúcar.
Dissimular.
Adoçar para illudir.
\section{Confeitaria}
\begin{itemize}
\item {Grp. gram.:f.}
\end{itemize}
\begin{itemize}
\item {Proveniência:(De \textunderscore confeito\textunderscore )}
\end{itemize}
Casa onde se fabricam ou vendem confeitos e outros doces.
\section{Confeiteira}
\textunderscore fem.\textunderscore  de \textunderscore confeiteiro\textunderscore .
\section{Confeiteiro}
\begin{itemize}
\item {Grp. gram.:m.}
\end{itemize}
Aquelle que fabríca ou vende confeitos ou doces.
\section{Confeito}
\begin{itemize}
\item {Grp. gram.:m.}
\end{itemize}
Pequena semente ou pevide, coberta de açúcar, preparada em xarope e sêca ao fogo.
(Provavelmente do it. \textunderscore confetto\textunderscore )
\section{Confeito}
\begin{itemize}
\item {Grp. gram.:adj.}
\end{itemize}
O mesmo que \textunderscore confeitado\textunderscore : \textunderscore amêndoas confeitas\textunderscore .
\section{Conferência}
\begin{itemize}
\item {Grp. gram.:f.}
\end{itemize}
\begin{itemize}
\item {Proveniência:(De \textunderscore conferente\textunderscore )}
\end{itemize}
Acto de conferir, de confrontar.
Conversação entre duas ou mais pessôas, sôbre negócios de interesse commum, público ou internacional.
Discurso literário ou scientífico em público.
Junta de médicos, que se consultam ou esclarecem mutuamente, sôbre o estado de um enfermo ou sôbre medidas sanitárias.
\section{Conferenciador}
\begin{itemize}
\item {Grp. gram.:m.}
\end{itemize}
\begin{itemize}
\item {Utilização:bras}
\end{itemize}
\begin{itemize}
\item {Utilização:Neol.}
\end{itemize}
\begin{itemize}
\item {Proveniência:(De \textunderscore conferenciar\textunderscore )}
\end{itemize}
Aquelle que faz conferências ou que discorre em público, sobre assumptos literários, scientíficos ou sociaes.
\section{Conferencial}
\begin{itemize}
\item {Grp. gram.:adj.}
\end{itemize}
Relativo a conferência.
Que tem fórma de conferência.
\section{Conferenciar}
\begin{itemize}
\item {Grp. gram.:v. i.}
\end{itemize}
Conversar.
Falar em conferência.
Têr conferência.
\section{Conferencionista}
\begin{itemize}
\item {Grp. gram.:m.}
\end{itemize}
\begin{itemize}
\item {Utilização:bras}
\end{itemize}
\begin{itemize}
\item {Utilização:Neol.}
\end{itemize}
\begin{itemize}
\item {Proveniência:(T. mal formado, em vez de \textunderscore conferenciador\textunderscore )}
\end{itemize}
\begin{itemize}
\item {Proveniência:(V. }
\end{itemize}
\begin{itemize}
\item {Proveniência:conferenciador}
\end{itemize}
\begin{itemize}
\item {Proveniência:)}
\end{itemize}

\section{Conferente}
\begin{itemize}
\item {Grp. gram.:adj.}
\end{itemize}
\begin{itemize}
\item {Grp. gram.:M.}
\end{itemize}
\begin{itemize}
\item {Proveniência:(Lat. \textunderscore conferens\textunderscore )}
\end{itemize}
Que confere.
Aquelle que faz uma conferência, discurso literário ou scientífico.
Aquelle que faz parte de uma conferência ou nella toma assento.
\section{Conferir}
\begin{itemize}
\item {Grp. gram.:v. t.}
\end{itemize}
\begin{itemize}
\item {Grp. gram.:V. i.}
\end{itemize}
\begin{itemize}
\item {Proveniência:(Lat. \textunderscore conferre\textunderscore )}
\end{itemize}
Comparar; cotejar.
Examinar para vêr se está exacto.
Conferenciar.
Dar, conceder: \textunderscore conferir um prémio\textunderscore .
Trazer á collação (bens).
Estar exacto, conforme.
\section{Conferva}
\begin{itemize}
\item {Grp. gram.:f.}
\end{itemize}
\begin{itemize}
\item {Proveniência:(Lat. \textunderscore conferva\textunderscore )}
\end{itemize}
Planta aquática, composta de filamentos verdes, vulgarmente chamados \textunderscore limos\textunderscore .
\section{Conferváceas}
\begin{itemize}
\item {Grp. gram.:f. pl.}
\end{itemize}
Classe da fam. das algas, que tem por typo a \textunderscore conferva\textunderscore .
\section{Conférveas}
\begin{itemize}
\item {Grp. gram.:f. pl.}
\end{itemize}
Fam. de plantas, que tem por typo a conferva, na classificação de Bory de Saint-Vincent.
\section{Confervícola}
\begin{itemize}
\item {Grp. gram.:adj.}
\end{itemize}
\begin{itemize}
\item {Proveniência:(Do lat. \textunderscore conferva\textunderscore  + \textunderscore colere\textunderscore )}
\end{itemize}
Que cresce ou habita entre confervas.
\section{Confessa}
\begin{itemize}
\item {Grp. gram.:f.}
\end{itemize}
\begin{itemize}
\item {Proveniência:(De \textunderscore confesso\textunderscore )}
\end{itemize}
Dizia-se a monja ou freira que vivia no mosteiro.
\section{Confessada}
\textunderscore fem.\textunderscore  de \textunderscore confessado\textunderscore .
\section{Confessado}
\begin{itemize}
\item {Grp. gram.:m.}
\end{itemize}
Aquelle que habitualmente se confessa a um padre.
Aquelle que se confessou.
\section{Confessador}
\begin{itemize}
\item {Grp. gram.:m.}
\end{itemize}
\begin{itemize}
\item {Utilização:Ant.}
\end{itemize}
O mesmo que \textunderscore confessor\textunderscore .
\section{Confessar}
\begin{itemize}
\item {Grp. gram.:v. t.}
\end{itemize}
\begin{itemize}
\item {Grp. gram.:V. p.}
\end{itemize}
\begin{itemize}
\item {Proveniência:(De \textunderscore confésso\textunderscore )}
\end{itemize}
Declarar (o que se fez ou o que se pensa): \textunderscore confessar um crime\textunderscore .
Declarar a um confessor (peccados e erros).
Ouvir a confissão de.
O mesmo que \textunderscore professar\textunderscore  ou \textunderscore seguir\textunderscore  (um systema, uma religião):«\textunderscore simulárão confessar a religião christã\textunderscore ». Filinto, \textunderscore D. Man.\textunderscore , I, 20.
Declarar peccados ao confessor.
\section{Confessional}
\begin{itemize}
\item {Grp. gram.:adj.}
\end{itemize}
\begin{itemize}
\item {Proveniência:(Do lat. \textunderscore confessio\textunderscore )}
\end{itemize}
Relativo a uma crença religiosa.
\section{Confessionário}
\begin{itemize}
\item {Grp. gram.:m.}
\end{itemize}
\begin{itemize}
\item {Proveniência:(Do lat. \textunderscore confessio\textunderscore )}
\end{itemize}
Lugar, onde o sacerdote ouve confissões.
Tribunal da penitência.
\section{Confésso}
\begin{itemize}
\item {Grp. gram.:adj.}
\end{itemize}
\begin{itemize}
\item {Grp. gram.:M.}
\end{itemize}
\begin{itemize}
\item {Proveniência:(Lat. \textunderscore confessus\textunderscore )}
\end{itemize}
Que confessou suas culpas: \textunderscore ladrão confésso\textunderscore .
Convertido.
Dizia-se o monge que vivia em mosteiro.
\section{Confêsso}
\begin{itemize}
\item {Grp. gram.:m.}
\end{itemize}
\begin{itemize}
\item {Utilização:Pop.}
\end{itemize}
\begin{itemize}
\item {Proveniência:(De \textunderscore confessar\textunderscore )}
\end{itemize}
Confissão de peccados ao sacerdote.
\section{Confessor}
\begin{itemize}
\item {Grp. gram.:m.}
\end{itemize}
\begin{itemize}
\item {Utilização:Ant.}
\end{itemize}
\begin{itemize}
\item {Utilização:Ant.}
\end{itemize}
\begin{itemize}
\item {Proveniência:(Lat. \textunderscore confessor\textunderscore )}
\end{itemize}
Sacerdote, que ouve confissões de penitentes.
Indivíduo, que confessa a fé christã.
O mesmo que \textunderscore mártyr\textunderscore .
O mesmo que \textunderscore confésso\textunderscore .
\section{Confessora}
\begin{itemize}
\item {Grp. gram.:f.}
\end{itemize}
Santa, que confessou a fé christã.
(Fem. de \textunderscore confessor\textunderscore )
\section{Confessório}
\begin{itemize}
\item {Grp. gram.:adj.}
\end{itemize}
\begin{itemize}
\item {Proveniência:(Lat. \textunderscore confessorius\textunderscore )}
\end{itemize}
Relativo a confissão.
\section{Confiadamente}
\begin{itemize}
\item {Grp. gram.:adv.}
\end{itemize}
De modo confiado.
Com confiança.
\section{Confiado}
\begin{itemize}
\item {Grp. gram.:adj.}
\end{itemize}
\begin{itemize}
\item {Utilização:Pop.}
\end{itemize}
\begin{itemize}
\item {Proveniência:(De \textunderscore confiar\textunderscore )}
\end{itemize}
Que tem confiança: \textunderscore fiquei confiado na tua palavra\textunderscore .
Atrevido: \textunderscore sempre és muito confiado\textunderscore !
\section{Confiança}
\begin{itemize}
\item {Grp. gram.:f.}
\end{itemize}
\begin{itemize}
\item {Utilização:Pop.}
\end{itemize}
\begin{itemize}
\item {Proveniência:(De \textunderscore confiar\textunderscore )}
\end{itemize}
Segurança íntima, com que se procede.
Crédito; bôa nomeada.
Segurança e bom conceito, que inspiram as pessôas probas, talentosas, discretas, etc.
Esperança firme.
Familiaridade: \textunderscore têr confiança com alguém\textunderscore .
Atrevimento.
\section{Confiante}
\begin{itemize}
\item {Grp. gram.:adj.}
\end{itemize}
Que confia.
\section{Confiar}
\begin{itemize}
\item {Grp. gram.:v. t.}
\end{itemize}
\begin{itemize}
\item {Grp. gram.:V. i.}
\end{itemize}
\begin{itemize}
\item {Proveniência:(De \textunderscore con...\textunderscore  + \textunderscore fiar\textunderscore )}
\end{itemize}
Transmittir.
Entregar, com segurança: \textunderscore confiei-lhe a educação dos pequenos\textunderscore .
Dar em depósito.
Dar a saber, communicar: \textunderscore confiar um segrêdo\textunderscore .
Têr confiança, esperança.
\section{Conficionar}
\begin{itemize}
\item {Grp. gram.:v. t.}
\end{itemize}
O mesmo que \textunderscore confeiçoar\textunderscore . Cf. R. Lobo, \textunderscore Côrte na Aldeia\textunderscore , II, 22.
\section{Confidência}
\begin{itemize}
\item {Grp. gram.:f.}
\end{itemize}
\begin{itemize}
\item {Proveniência:(Lat. \textunderscore confidentia\textunderscore )}
\end{itemize}
Communicação secreta; participação de um segrêdo.
Confiança.
\section{Confidencial}
\begin{itemize}
\item {Grp. gram.:adj.}
\end{itemize}
Secreto.
Que se diz ou que se escreve em confidência.
\section{Confidencialmente}
\begin{itemize}
\item {Grp. gram.:adv.}
\end{itemize}
De modo confidencial.
Em segrêdo.
\section{Confidenciar}
\begin{itemize}
\item {Grp. gram.:v. t.}
\end{itemize}
Dizer em segrêdo, em confidência.
\section{Confidencioso}
\begin{itemize}
\item {Grp. gram.:adj.}
\end{itemize}
Relativo a confidência.
Revelado em confidência.
Que tem modos de confidência. Cf. Camillo, \textunderscore Sc. da Foz\textunderscore , 85.
\section{Confidente}
\begin{itemize}
\item {Grp. gram.:m.  e  adj.}
\end{itemize}
\begin{itemize}
\item {Proveniência:(Lat. \textunderscore confidens\textunderscore )}
\end{itemize}
A quem se confia um segrêdo ou segredos.
\section{Configuração}
\begin{itemize}
\item {Grp. gram.:f.}
\end{itemize}
\begin{itemize}
\item {Proveniência:(Lat. \textunderscore configuratio\textunderscore )}
\end{itemize}
Fórma exterior de um corpo; aspecto; figura.
Fórma de um grupo de astros, ligados por linhas imaginárias.
\section{Configurar}
\begin{itemize}
\item {Grp. gram.:v. t.}
\end{itemize}
\begin{itemize}
\item {Proveniência:(Lat. \textunderscore configurare\textunderscore )}
\end{itemize}
Dar fórma a; representar.
\section{Confim}
\begin{itemize}
\item {Grp. gram.:adj.}
\end{itemize}
\begin{itemize}
\item {Grp. gram.:M. pl.}
\end{itemize}
\begin{itemize}
\item {Proveniência:(Lat. \textunderscore confinis\textunderscore )}
\end{itemize}
O mesmo que \textunderscore confinante\textunderscore .
Raias, fronteiras.
Extremo: \textunderscore até os confins do mundo\textunderscore .
\section{Confinante}
\begin{itemize}
\item {Grp. gram.:adj.}
\end{itemize}
Que confina.
\section{Confinar}
\begin{itemize}
\item {Grp. gram.:v. t.}
\end{itemize}
\begin{itemize}
\item {Grp. gram.:V. i.}
\end{itemize}
\begin{itemize}
\item {Proveniência:(De \textunderscore confim\textunderscore )}
\end{itemize}
Limitar, circumscrever.
Estar nos confins.
Têr limite commum: \textunderscore o meu jardim confina com o delle\textunderscore .
\section{Confinidade}
\begin{itemize}
\item {Grp. gram.:f.}
\end{itemize}
\begin{itemize}
\item {Proveniência:(Do lat. \textunderscore confinis\textunderscore )}
\end{itemize}
Qualidade daquillo que confina.
\section{Confioso}
\begin{itemize}
\item {Grp. gram.:adj.}
\end{itemize}
Cheio de confiança. Cf. Côrvo, \textunderscore Anno na Côrte\textunderscore , II, 139.
\section{Confirmação}
\begin{itemize}
\item {Grp. gram.:f.}
\end{itemize}
\begin{itemize}
\item {Utilização:Rhet.}
\end{itemize}
\begin{itemize}
\item {Proveniência:(Lat. \textunderscore confirmatio\textunderscore )}
\end{itemize}
Acto ou effeito de confirmar.
Chrisma, um dos sacramentos da Igreja.
Parte do discurso, em que o orador desenvolve as provas.
\section{Confirmadamente}
\begin{itemize}
\item {Grp. gram.:adv.}
\end{itemize}
Com confirmação.
\section{Confirmador}
\begin{itemize}
\item {Grp. gram.:m.}
\end{itemize}
Aquelle que confirma.
\section{Confirmante}
\begin{itemize}
\item {Grp. gram.:adj.}
\end{itemize}
\begin{itemize}
\item {Proveniência:(Lat. \textunderscore confirmans\textunderscore )}
\end{itemize}
Que confirma.
\section{Confirmar}
\begin{itemize}
\item {Grp. gram.:v. t.}
\end{itemize}
\begin{itemize}
\item {Proveniência:(Lat. \textunderscore confirmare\textunderscore )}
\end{itemize}
Tornar firme.
Affirmar categoricamente.
Certificar; ratificar: \textunderscore confirmo a accusação\textunderscore .
Comprovar.
Manter firme, sustentar.
Applicar a chrisma a.
\section{Confirmativo}
\begin{itemize}
\item {Grp. gram.:adj.}
\end{itemize}
\begin{itemize}
\item {Proveniência:(Lat. \textunderscore confirmativus\textunderscore )}
\end{itemize}
Que confirma.
\section{Confirmatório}
\begin{itemize}
\item {Grp. gram.:adj.}
\end{itemize}
\begin{itemize}
\item {Proveniência:(De \textunderscore confirmar\textunderscore )}
\end{itemize}
Que envolve confirmação.
\section{Confiscação}
\begin{itemize}
\item {Grp. gram.:f.}
\end{itemize}
\begin{itemize}
\item {Proveniência:(Lat. \textunderscore confiscatio\textunderscore )}
\end{itemize}
Acto ou effeito de confiscar.
\section{Confiscar}
\begin{itemize}
\item {Grp. gram.:v. t.}
\end{itemize}
\begin{itemize}
\item {Proveniência:(Lat. \textunderscore confiscare\textunderscore )}
\end{itemize}
Fazer entrar nos cofres públicos, em consequência de delicto ou crime.
Arrestar.
\section{Confiscável}
\begin{itemize}
\item {Grp. gram.:adj.}
\end{itemize}
\begin{itemize}
\item {Proveniência:(De \textunderscore confiscar\textunderscore )}
\end{itemize}
Que póde sêr confiscado.
\section{Confisco}
\begin{itemize}
\item {Grp. gram.:m.}
\end{itemize}
(V.confiscação)
\section{Confissão}
\begin{itemize}
\item {Grp. gram.:f.}
\end{itemize}
\begin{itemize}
\item {Utilização:Ant.}
\end{itemize}
\begin{itemize}
\item {Utilização:Ant.}
\end{itemize}
\begin{itemize}
\item {Utilização:Ant.}
\end{itemize}
\begin{itemize}
\item {Utilização:Ant.}
\end{itemize}
\begin{itemize}
\item {Utilização:Ant.}
\end{itemize}
\begin{itemize}
\item {Proveniência:(Lat. \textunderscore confessio\textunderscore )}
\end{itemize}
Acto de confessar ou de se confessar.
Cada uma das seitas christãs.
Uma das orações da Igreja.
Túmulo ou sepultura de um mártyr.
Altar ou oratório.
Qualquer lugar, onde se administrasse o sacramento da penitência.
Profissão monacal.
Qualquer profissão ou offício.
\section{Confita}
\begin{itemize}
\item {Grp. gram.:f. Loc. adv.}
\end{itemize}
\begin{itemize}
\item {Utilização:Prov.}
\end{itemize}
\begin{itemize}
\item {Utilização:minh.}
\end{itemize}
\begin{itemize}
\item {Utilização:Prov.}
\end{itemize}
\begin{itemize}
\item {Utilização:trasm.}
\end{itemize}
\textunderscore Á certa confita\textunderscore , com certeza, sem dúvida.
Na occasião aprazada; no tempo combinado.
Inesperadamente; quando menos se esperava.--A expressão é antiga na língua, como se vê na \textunderscore Eufrosina\textunderscore , act. I, sc. 2. Entretanto, parece que a locução provém de que alguns aventureiros de feiras jogam a vermelhinha, servindo-se de uma fita enrolada, em que o ponto mete o dedo, perdendo a partida, se o dedo fica solto; o que succede normalmente, pelo que se diz que o \textunderscore ponto\textunderscore  perde \textunderscore á certa\textunderscore , \textunderscore com fita\textunderscore  (\textunderscore confita\textunderscore ).
\section{Confitente}
\begin{itemize}
\item {Grp. gram.:m. f.  e  adj.}
\end{itemize}
\begin{itemize}
\item {Proveniência:(Lat. \textunderscore confitens\textunderscore )}
\end{itemize}
Pessôa, que confessa ou que se confessa.
\section{Conflagração}
\begin{itemize}
\item {Grp. gram.:f.}
\end{itemize}
\begin{itemize}
\item {Utilização:Fig.}
\end{itemize}
\begin{itemize}
\item {Proveniência:(Lat. \textunderscore conflagratio\textunderscore )}
\end{itemize}
Incêndio, que se alastrou.
Grande excitação de ânimo.
Revolução.
\section{Conflagrar}
\begin{itemize}
\item {Grp. gram.:v. t.}
\end{itemize}
\begin{itemize}
\item {Utilização:Fig.}
\end{itemize}
\begin{itemize}
\item {Proveniência:(Lat. \textunderscore conflagrare\textunderscore )}
\end{itemize}
Incendiar extensamente.
Excitar; pôr em convulsão.
\section{Conflicto}
\begin{itemize}
\item {Grp. gram.:m.}
\end{itemize}
\begin{itemize}
\item {Proveniência:(Lat. \textunderscore conflitus\textunderscore )}
\end{itemize}
Embate dos que lutam.
Discussão injuriosa.
Conjuntura.
Luta.
Pleito.
\section{Conflito}
\begin{itemize}
\item {Grp. gram.:m.}
\end{itemize}
\begin{itemize}
\item {Proveniência:(Lat. \textunderscore conflitus\textunderscore )}
\end{itemize}
Embate dos que lutam.
Discussão injuriosa.
Conjuntura.
Luta.
Pleito.
\section{Conflução}
\begin{itemize}
\item {Grp. gram.:f.}
\end{itemize}
\begin{itemize}
\item {Utilização:Des.}
\end{itemize}
Acto de confluir; confluência. Cf. \textunderscore Viriato Trág.\textunderscore , XV, 4.
(Má derivação de \textunderscore confluir\textunderscore . Preferivel seria \textunderscore confluição\textunderscore , á maneira de \textunderscore influição\textunderscore ; ou ainda \textunderscore confluxão\textunderscore , análogo a \textunderscore fluxão\textunderscore )
\section{Confluência}
\begin{itemize}
\item {Grp. gram.:f.}
\end{itemize}
\begin{itemize}
\item {Proveniência:(Lat. \textunderscore confluentia\textunderscore )}
\end{itemize}
Qualidade do que é confluente.
Lugar, aonde confluem rios.
\section{Confluente}
\begin{itemize}
\item {Grp. gram.:adj.}
\end{itemize}
\begin{itemize}
\item {Utilização:Med.}
\end{itemize}
\begin{itemize}
\item {Proveniência:(Lat. \textunderscore confluens\textunderscore )}
\end{itemize}
Que conflue.
Diz-se da varíola, em que são tantas as pústulas, que se confundem.
Diz-se do rio, que vai desaguar em outro; da veia, que emboca noutra; e de um órgão vegetal, que se reúne a outro por uma extremidade.
\section{Confluir}
\begin{itemize}
\item {Grp. gram.:v. i.}
\end{itemize}
\begin{itemize}
\item {Proveniência:(Lat. \textunderscore confluere\textunderscore )}
\end{itemize}
Correr para o mesmo ponto; convergir.
Diz-se especialmente dos rios que, juntando-se num ponto, correm depois em um leito commum.
\section{Conformação}
\begin{itemize}
\item {Grp. gram.:f.}
\end{itemize}
\begin{itemize}
\item {Proveniência:(Lat. \textunderscore conformatio\textunderscore )}
\end{itemize}
Configuração.
Conformidade, resignação.
\section{Conformador}
\begin{itemize}
\item {Grp. gram.:adj.}
\end{itemize}
\begin{itemize}
\item {Grp. gram.:M.}
\end{itemize}
\begin{itemize}
\item {Proveniência:(De \textunderscore conformar\textunderscore )}
\end{itemize}
Que conforma.
Aquelle que conforma.
Apparelho articulado de chapeleiro, para determinar ou accusar a conformação exacta de uma cabeça.
\section{Conformar}
\begin{itemize}
\item {Grp. gram.:v. t.}
\end{itemize}
\begin{itemize}
\item {Grp. gram.:V. p.}
\end{itemize}
\begin{itemize}
\item {Grp. gram.:V. i.}
\end{itemize}
\begin{itemize}
\item {Proveniência:(Lat. \textunderscore conformare\textunderscore )}
\end{itemize}
O mesmo que \textunderscore configurar\textunderscore .
Harmonizar.
Sêr conforme.
Concordar; condescender.
Resignar-se: \textunderscore conformar-se com a sorte\textunderscore .
Sêr conforme.
\section{Conformativo}
\begin{itemize}
\item {Grp. gram.:adj.}
\end{itemize}
\begin{itemize}
\item {Utilização:bras}
\end{itemize}
\begin{itemize}
\item {Utilização:Neol.}
\end{itemize}
Destinado a conformar.
\section{Conforme}
\begin{itemize}
\item {Grp. gram.:adj.}
\end{itemize}
\begin{itemize}
\item {Grp. gram.:Adv.}
\end{itemize}
\begin{itemize}
\item {Grp. gram.:Conj.}
\end{itemize}
\begin{itemize}
\item {Proveniência:(Lat. \textunderscore conformis\textunderscore )}
\end{itemize}
Que tem a mesma fórma; idêntico.
Concorde.
Resignado.
Que está nos devidos termos: \textunderscore a certidão está conforme\textunderscore .
Em conformidade.
Como; segundo as circumstâncias de: \textunderscore procedi conforme as tuas indicações\textunderscore .
\section{Conformemente}
\begin{itemize}
\item {Grp. gram.:adv.}
\end{itemize}
De modo conforme.
\section{Conformidade}
\begin{itemize}
\item {Grp. gram.:f.}
\end{itemize}
Qualidade do que é conforme, ou de quem se conforma.
Resignação.
\section{Confortabilidade}
\begin{itemize}
\item {Grp. gram.:f.}
\end{itemize}
Qualidade de confortável.
\section{Confortação}
\begin{itemize}
\item {Grp. gram.:f.}
\end{itemize}
(V.confôrto)
\section{Confortador}
\begin{itemize}
\item {Grp. gram.:adj.}
\end{itemize}
Que conforta.
\section{Confortamento}
\begin{itemize}
\item {Grp. gram.:m.}
\end{itemize}
O mesmo que \textunderscore confôrto\textunderscore .
\section{Confortante}
\begin{itemize}
\item {Grp. gram.:adj.}
\end{itemize}
Que conforta.
\section{Confortantes}
\begin{itemize}
\item {Grp. gram.:m. pl.}
\end{itemize}
\begin{itemize}
\item {Utilização:Prov.}
\end{itemize}
\begin{itemize}
\item {Utilização:alent.}
\end{itemize}
\begin{itemize}
\item {Proveniência:(De \textunderscore confortar\textunderscore )}
\end{itemize}
Luvas sem dedeiras, que só resguardam a chave da mão, deixando livres os dedos, para coser, bordar, etc.
\section{Confortar}
\begin{itemize}
\item {Grp. gram.:v. t.}
\end{itemize}
\begin{itemize}
\item {Utilização:Fig.}
\end{itemize}
\begin{itemize}
\item {Proveniência:(Lat. \textunderscore confortare\textunderscore )}
\end{itemize}
Dar fôrças a; fortificar: \textunderscore confortar o estômago\textunderscore .
Consolar: \textunderscore confortar os tristes\textunderscore .
\section{Confortativo}
\begin{itemize}
\item {Grp. gram.:adj.}
\end{itemize}
\begin{itemize}
\item {Grp. gram.:M.}
\end{itemize}
\begin{itemize}
\item {Proveniência:(De \textunderscore confortar\textunderscore )}
\end{itemize}
Próprio para confortar ou fortificar.
Medicamento fortificante.
\section{Confortável}
\begin{itemize}
\item {Grp. gram.:adj.}
\end{itemize}
\begin{itemize}
\item {Proveniência:(De \textunderscore confortar\textunderscore )}
\end{itemize}
Que conforta: \textunderscore alimento confortável\textunderscore .
Que dá commodidade: \textunderscore habitação confortável\textunderscore .
\section{Confôrto}
\begin{itemize}
\item {Grp. gram.:m.}
\end{itemize}
Acto ou effeito de confortar.
Estado de quem é confortado.
Commodidade material.
Consolação.
Variedade de pêra ordinária.
\section{Confrade}
\begin{itemize}
\item {Grp. gram.:m.}
\end{itemize}
\begin{itemize}
\item {Proveniência:(Do lat. \textunderscore confrater\textunderscore )}
\end{itemize}
Membro de confraria.
Camarada; collega.
Aquelle que exerce a mesma profissão ou pertence á mesma categoria que outrem.
\section{Confragoso}
\begin{itemize}
\item {Grp. gram.:adj.}
\end{itemize}
\begin{itemize}
\item {Utilização:P. us.}
\end{itemize}
\begin{itemize}
\item {Proveniência:(Lat. \textunderscore confragosus\textunderscore )}
\end{itemize}
Áspero; cheio de penedias.
\section{Confrangedor}
\begin{itemize}
\item {Grp. gram.:adj.}
\end{itemize}
Que confrange.
\section{Confranger}
\begin{itemize}
\item {Grp. gram.:v. t.}
\end{itemize}
\begin{itemize}
\item {Proveniência:(Do lat. \textunderscore cum\textunderscore  + \textunderscore frangere\textunderscore , partir)}
\end{itemize}
Partir.
Moer.
Apertar.
Atormentar; angustiar: \textunderscore confrangeu-me aquella desgraça\textunderscore .
\section{Confrangimento}
\begin{itemize}
\item {Grp. gram.:m.}
\end{itemize}
Acto ou efeito de confranger.
\section{Confraria}
\begin{itemize}
\item {Grp. gram.:f.}
\end{itemize}
Associação com fins religiosos.
Sociedade.
Conjunto das pessôas que exercem a mesma profissão ou tem o mesmo modo de vida.
(Por \textunderscore confradia\textunderscore , de \textunderscore confrade\textunderscore . Talvez infl. do fr. \textunderscore confrèrie\textunderscore , de \textunderscore confrère\textunderscore )
\section{Confraternal}
\begin{itemize}
\item {Grp. gram.:adj.}
\end{itemize}
\begin{itemize}
\item {Utilização:bras}
\end{itemize}
\begin{itemize}
\item {Utilização:Neol.}
\end{itemize}
\begin{itemize}
\item {Proveniência:(De \textunderscore con...\textunderscore  + \textunderscore fraternal\textunderscore )}
\end{itemize}
Que é reciprocamente fraternal: \textunderscore amizade confraternal\textunderscore .
\section{Confraternar}
\begin{itemize}
\item {Grp. gram.:v. t.}
\end{itemize}
\begin{itemize}
\item {Proveniência:(De \textunderscore con...\textunderscore  + \textunderscore fraterno\textunderscore )}
\end{itemize}
Ligar como irmãos.
\section{Confraternidade}
\begin{itemize}
\item {Grp. gram.:adj.}
\end{itemize}
\begin{itemize}
\item {Proveniência:(De \textunderscore con...\textunderscore  + \textunderscore fraternidade\textunderscore )}
\end{itemize}
Ligação fraterna.
Amizade, comparável á que deve haver entre irmãos.
\section{Confraternização}
\begin{itemize}
\item {Grp. gram.:f.}
\end{itemize}
Acto de confraternizar.
\section{Confraternizar}
\begin{itemize}
\item {Grp. gram.:v. t.}
\end{itemize}
\begin{itemize}
\item {Grp. gram.:V. i.}
\end{itemize}
\begin{itemize}
\item {Proveniência:(De \textunderscore con...\textunderscore  + \textunderscore fraternizar\textunderscore )}
\end{itemize}
Confraternar.
Conviver fraternalmente.
Têr sentimentos, opiniões ou crenças idênticas á de outrem.
\section{Confrontação}
\begin{itemize}
\item {Grp. gram.:f.}
\end{itemize}
\begin{itemize}
\item {Grp. gram.:Pl.}
\end{itemize}
Acto de confrontar, de acarear.
Limites de um prédio, estremas.
\section{Confrontador}
\begin{itemize}
\item {Grp. gram.:m.}
\end{itemize}
Aquelle que confronta.
\section{Confrontante}
\begin{itemize}
\item {Grp. gram.:adj.}
\end{itemize}
Que confronta; confinante.
\section{Confrontar}
\begin{itemize}
\item {Grp. gram.:v. t.}
\end{itemize}
\begin{itemize}
\item {Grp. gram.:V. i.}
\end{itemize}
\begin{itemize}
\item {Proveniência:(De \textunderscore con...\textunderscore  + \textunderscore fronte\textunderscore )}
\end{itemize}
Pôr defronte reciprocamente (duas ou mais coisas ou pessôas).
Comparar, acarear, conferir, cotejar.
Estar de fronte; defrontar; confinar: \textunderscore o meu prédio confronta com o teu\textunderscore .
\section{Confronto}
\begin{itemize}
\item {Grp. gram.:m.}
\end{itemize}
Acto ou effeito de confrontar.
\section{Confucianismo}
\begin{itemize}
\item {Grp. gram.:m.}
\end{itemize}
\begin{itemize}
\item {Proveniência:(De \textunderscore Confúcio\textunderscore , n. p.)}
\end{itemize}
Religião chinesa, que dá carácter moral á adoração das coisas phýsicas, e se baseia na benevolência.
\section{Confuciano}
\begin{itemize}
\item {Grp. gram.:adj.}
\end{itemize}
Relativo a Confúcio: \textunderscore moral confuciana\textunderscore .
\section{Confugir}
\begin{itemize}
\item {Grp. gram.:v. i.}
\end{itemize}
\begin{itemize}
\item {Proveniência:(Lat. \textunderscore confugere\textunderscore )}
\end{itemize}
Fugir simultaneamente.
Recorrer.
Solicitar auxílio.
\section{Confundas}
\begin{itemize}
\item {Grp. gram.:f. pl.}
\end{itemize}
\begin{itemize}
\item {Utilização:Bras}
\end{itemize}
\begin{itemize}
\item {Utilização:pop.}
\end{itemize}
O mesmo que \textunderscore profundas\textunderscore . Cf. S. Romero, \textunderscore Contos Pop.\textunderscore 
\section{Confundidamente}
\begin{itemize}
\item {Grp. gram.:adv.}
\end{itemize}
Com confusão.
\section{Confundir}
\begin{itemize}
\item {Grp. gram.:v. t.}
\end{itemize}
\begin{itemize}
\item {Proveniência:(Lat. \textunderscore confundere\textunderscore )}
\end{itemize}
Unir desordenadamente.
Misturar.
Identificar.
Não distinguir: \textunderscore confundiu-me com meu irmão\textunderscore .
Impossibilitar de responder.
Vexar; humilhar.
Envergonhar, ferir a modestia de.
Perturbar.
Deturpar.
\section{Confundível}
\begin{itemize}
\item {Grp. gram.:adj.}
\end{itemize}
Que se póde confundir.
\section{Confusamente}
\begin{itemize}
\item {Grp. gram.:adv.}
\end{itemize}
De modo confuso.
\section{Confusão}
\begin{itemize}
\item {Grp. gram.:f.}
\end{itemize}
\begin{itemize}
\item {Proveniência:(Lat. \textunderscore confusio\textunderscore )}
\end{itemize}
Acto ou effeito de confundir.
Perplexidade.
Barulho; multidão desordenada.
\section{Confuso}
\begin{itemize}
\item {Proveniência:(Lat. \textunderscore confusus\textunderscore )}
\end{itemize}
\textunderscore part. irr.\textunderscore  de \textunderscore confundir\textunderscore : \textunderscore ficar confuso\textunderscore .
\section{Confutação}
\begin{itemize}
\item {Grp. gram.:f.}
\end{itemize}
\begin{itemize}
\item {Proveniência:(Lat. \textunderscore confutatio\textunderscore )}
\end{itemize}
Acto ou effeito de confutar.
\section{Confutador}
\begin{itemize}
\item {Grp. gram.:m.}
\end{itemize}
\begin{itemize}
\item {Proveniência:(Lat. \textunderscore confutator\textunderscore )}
\end{itemize}
Aquelle que confuta.
\section{Confutar}
\begin{itemize}
\item {Grp. gram.:v. t.}
\end{itemize}
\begin{itemize}
\item {Proveniência:(Lat. \textunderscore confutare\textunderscore )}
\end{itemize}
Refutar; rebater.
Reprimir.
\section{Confutável}
\begin{itemize}
\item {Grp. gram.:adj.}
\end{itemize}
Que se póde confutar.
\section{Congelação}
\begin{itemize}
\item {Grp. gram.:f.}
\end{itemize}
\begin{itemize}
\item {Proveniência:(Lat. \textunderscore congelatio\textunderscore )}
\end{itemize}
Acto ou effeito de congelar.
\section{Congelar}
\begin{itemize}
\item {Grp. gram.:v. t.}
\end{itemize}
\begin{itemize}
\item {Utilização:Fig.}
\end{itemize}
\begin{itemize}
\item {Proveniência:(Lat. \textunderscore congelare\textunderscore )}
\end{itemize}
Tornar em gêlo.
Solidificar (o que era líquido).
Embargar, embaraçar.
\section{Congelativo}
\begin{itemize}
\item {Grp. gram.:adj.}
\end{itemize}
Que faz congelar.
Que se póde congelar.
\section{Congelável}
\begin{itemize}
\item {Grp. gram.:adj.}
\end{itemize}
Que se póde congelar.
\section{Congeminação}
\begin{itemize}
\item {Grp. gram.:f.}
\end{itemize}
Formação dupla e simultânea.
Acto de congeminar.
\section{Congeminar}
\begin{itemize}
\item {Grp. gram.:v. t.}
\end{itemize}
\begin{itemize}
\item {Utilização:Neol.}
\end{itemize}
\begin{itemize}
\item {Proveniência:(Lat. \textunderscore congeminare\textunderscore )}
\end{itemize}
Empregam este t. escritores novos, na accepção de \textunderscore irmanar\textunderscore  ou \textunderscore fraternîzar\textunderscore . Cf. Camillo, \textunderscore Myst. de Lisb.\textunderscore , I, 135 e 247. É accepção imprópria. A accepção legítima seria \textunderscore redobrar\textunderscore , \textunderscore multiplicar\textunderscore .
\section{Congeminar}
\begin{itemize}
\item {Grp. gram.:v. i.}
\end{itemize}
\begin{itemize}
\item {Utilização:Prov.}
\end{itemize}
\begin{itemize}
\item {Utilização:trasm.}
\end{itemize}
Meditar.
Scismar:«\textunderscore o homem esteve lá a congeminar com os botões\textunderscore ». Camillo, \textunderscore Myst. de Lisb.\textunderscore , I.
(Alter. de \textunderscore imaginar\textunderscore )
\section{Congeminência}
\begin{itemize}
\item {Grp. gram.:f.}
\end{itemize}
\begin{itemize}
\item {Utilização:Fam.}
\end{itemize}
Conjuntura; lance esquisito.
\section{Congênere}
\begin{itemize}
\item {Grp. gram.:adj.}
\end{itemize}
\begin{itemize}
\item {Proveniência:(Lat. \textunderscore congener\textunderscore )}
\end{itemize}
Que tem o mesmo gênero; idêntico: defeitos congêneres.
\section{Congenial}
\begin{itemize}
\item {Grp. gram.:adj.}
\end{itemize}
\begin{itemize}
\item {Proveniência:(De \textunderscore con...\textunderscore  + \textunderscore genial\textunderscore )}
\end{itemize}
Conforme á índole, ao génio de alguém.
Próprio por natureza.
\section{Congenialidade}
\begin{itemize}
\item {Grp. gram.:f.}
\end{itemize}
Qualidade do que é congenial.
\section{Congênito}
\begin{itemize}
\item {Grp. gram.:adj.}
\end{itemize}
\begin{itemize}
\item {Proveniência:(Lat. \textunderscore congenitus\textunderscore )}
\end{itemize}
Gerado simultaneamente.
Que nasceu com o indivíduo: \textunderscore tendências congênitas\textunderscore .
Apropriado.
\section{Congerar}
\begin{itemize}
\item {Grp. gram.:v. t.}
\end{itemize}
O mesmo que \textunderscore gerar\textunderscore . Cf. Mestre Giraldo, \textunderscore Enfermidades das Aves de Caça\textunderscore .
\section{Congérie}
\begin{itemize}
\item {Grp. gram.:f.}
\end{itemize}
\begin{itemize}
\item {Proveniência:(Lat. \textunderscore congeries\textunderscore )}
\end{itemize}
Reunião de muitas coisas differentes.
Montão, acumulação.
\section{Congestão}
\begin{itemize}
\item {Grp. gram.:f.}
\end{itemize}
\begin{itemize}
\item {Proveniência:(Lat. \textunderscore congestio\textunderscore )}
\end{itemize}
Affluência anormal do sangue aos vasos de um órgão: \textunderscore congestão pulmonar\textunderscore .
\section{Congestionar-se}
\begin{itemize}
\item {Grp. gram.:v. p.}
\end{itemize}
\begin{itemize}
\item {Utilização:Fig.}
\end{itemize}
\begin{itemize}
\item {Proveniência:(Do lat. \textunderscore congestio\textunderscore )}
\end{itemize}
Acumular-se (o sangue ou outro líquido) nos vasos de um órgão.
Ruborizar-se de cólera ou de indignação.
\section{Congesto}
\begin{itemize}
\item {Grp. gram.:M.}
\end{itemize}
\textunderscore part. irr.\textunderscore  de \textunderscore congestionar-se\textunderscore .
O mesmo que \textunderscore congestão\textunderscore . Cf. Castilho, \textunderscore Fastos\textunderscore , I, 196; Filinto, VII, 184.
\section{Côngio}
\begin{itemize}
\item {Grp. gram.:m.}
\end{itemize}
\begin{itemize}
\item {Proveniência:(Lat. \textunderscore congius\textunderscore )}
\end{itemize}
Antiga medida romana de capacidade.
\section{Conglobação}
\begin{itemize}
\item {Grp. gram.:f.}
\end{itemize}
\begin{itemize}
\item {Proveniência:(Lat. \textunderscore conglobatio\textunderscore )}
\end{itemize}
Acto ou effeito de conglobar.
\section{Conglobar}
\begin{itemize}
\item {Grp. gram.:v. t.}
\end{itemize}
\begin{itemize}
\item {Grp. gram.:V. p.}
\end{itemize}
\begin{itemize}
\item {Proveniência:(Lat. \textunderscore conglobare\textunderscore )}
\end{itemize}
Juntar em globo.
Amontoar.
Resumir, concentrar.
Tomar a fórma do globo; ennovelar-se.
\section{Conglomerados}
\begin{itemize}
\item {Grp. gram.:m. pl.}
\end{itemize}
\begin{itemize}
\item {Utilização:Miner.}
\end{itemize}
\begin{itemize}
\item {Proveniência:(De \textunderscore conglomerar\textunderscore )}
\end{itemize}
Fragmentos, que constituem uma rocha clástica, quando se lhes interpõe uma substância estranha, que representa o papel de cimento. Cf. Gonç. Guimarãs, \textunderscore Geologia\textunderscore .
\section{Conglomerar}
\begin{itemize}
\item {Proveniência:(Lat. \textunderscore conglomerare\textunderscore )}
\end{itemize}
\textunderscore v. t.\textunderscore  (e der.)
O mesmo que \textunderscore conglobar\textunderscore , etc.
\section{Conglutinação}
\begin{itemize}
\item {Grp. gram.:f.}
\end{itemize}
\begin{itemize}
\item {Proveniência:(Lat. \textunderscore conglutinatio\textunderscore )}
\end{itemize}
Acto ou effeito de conglutinar.
\section{Conglutinante}
\begin{itemize}
\item {Grp. gram.:adj.}
\end{itemize}
\begin{itemize}
\item {Proveniência:(Lat. \textunderscore conglutinans\textunderscore )}
\end{itemize}
Que conglutina.
\section{Conglutinar}
\begin{itemize}
\item {Grp. gram.:v. t.}
\end{itemize}
\begin{itemize}
\item {Proveniência:(Lat. \textunderscore conglutinare\textunderscore )}
\end{itemize}
Ligar com substância viscosa.
\section{Conglutinativo}
\begin{itemize}
\item {Grp. gram.:adj.}
\end{itemize}
(V.conglutinante)
\section{Conglutinoso}
\begin{itemize}
\item {Grp. gram.:adj.}
\end{itemize}
\begin{itemize}
\item {Proveniência:(Lat. \textunderscore conglutinosus\textunderscore )}
\end{itemize}
Viscoso; pegajoso.
\section{Congo}
\begin{itemize}
\item {Grp. gram.:m.}
\end{itemize}
O mesmo que \textunderscore conguês\textunderscore .
\section{Congo}
\begin{itemize}
\item {Grp. gram.:m.  e  adj.}
\end{itemize}
Diz-se de uma espécie de chá preto, que é a geralmente consumida na China.
\section{Congonha}
\begin{itemize}
\item {Grp. gram.:f.}
\end{itemize}
Planta illicínea da América, mais vulgarmente chamada \textunderscore mate\textunderscore .
(Do tupi)
\section{Congonhar}
\begin{itemize}
\item {Grp. gram.:v. i.}
\end{itemize}
\begin{itemize}
\item {Utilização:Bras}
\end{itemize}
\begin{itemize}
\item {Proveniência:(De \textunderscore congonha\textunderscore )}
\end{itemize}
Beber a infusão, feita de congonha ou mate.
\section{Congorsa}
\begin{itemize}
\item {Grp. gram.:f.}
\end{itemize}
Planta herbácea de flôres azues.
\section{Congossa}
\begin{itemize}
\item {Grp. gram.:f.}
\end{itemize}
Planta herbácea de flôres azues.
\section{Congosta}
\begin{itemize}
\item {Grp. gram.:f.}
\end{itemize}
Rua estreita e longa.
Caminho estreito, entre paredes, e mais ou menos em declive.
(Por \textunderscore congosta\textunderscore , do lat. \textunderscore canalis\textunderscore  + \textunderscore augustus\textunderscore , que por condensação deu \textunderscore cananguto\textunderscore  &lt; \textunderscore canangusta\textunderscore  &lt; \textunderscore canangosta\textunderscore  &lt; \textunderscore caangosta\textunderscore  &lt; \textunderscore cangosta\textunderscore  &lt; \textunderscore congosta\textunderscore )
\section{Congote}
\begin{itemize}
\item {Grp. gram.:m.}
\end{itemize}
\begin{itemize}
\item {Utilização:Bras}
\end{itemize}
A parte posterior do pescoço.
(Corr. de \textunderscore cangote\textunderscore , de \textunderscore canga\textunderscore ?)
\section{Congoxa}
\begin{itemize}
\item {fónica:gô}
\end{itemize}
\begin{itemize}
\item {Grp. gram.:f.}
\end{itemize}
\begin{itemize}
\item {Utilização:Ant.}
\end{itemize}
Afflicção; angústia.
(Cast. \textunderscore congoja\textunderscore )
\section{Congoxadamente}
\begin{itemize}
\item {Grp. gram.:adv.}
\end{itemize}
\begin{itemize}
\item {Utilização:Ant.}
\end{itemize}
\begin{itemize}
\item {Proveniência:(De \textunderscore congoxado\textunderscore )}
\end{itemize}
De modo afflictivo; ansiosamente.
\section{Congoxado}
\begin{itemize}
\item {Grp. gram.:adj.}
\end{itemize}
\begin{itemize}
\item {Utilização:Ant.}
\end{itemize}
\begin{itemize}
\item {Proveniência:(De \textunderscore congoxar\textunderscore )}
\end{itemize}
Afflicto, ansioso.
\section{Congoxar}
\begin{itemize}
\item {Grp. gram.:v. t.}
\end{itemize}
\begin{itemize}
\item {Utilização:Ant.}
\end{itemize}
Causar congoxa a; affligir.
\section{Congoxosamente}
\begin{itemize}
\item {Grp. gram.:adv.}
\end{itemize}
\begin{itemize}
\item {Utilização:Ant.}
\end{itemize}
O mesmo que \textunderscore congoxadamente\textunderscore .
\section{Congoxoso}
\begin{itemize}
\item {Grp. gram.:adj.}
\end{itemize}
\begin{itemize}
\item {Utilização:Ant.}
\end{itemize}
Que tem congoxa; afflicto.
\section{Congraçador}
\begin{itemize}
\item {Grp. gram.:m.  e  adj.}
\end{itemize}
O que congraça.
\section{Congraçar}
\begin{itemize}
\item {Grp. gram.:v. t.}
\end{itemize}
\begin{itemize}
\item {Proveniência:(De \textunderscore con...\textunderscore  + \textunderscore graça\textunderscore )}
\end{itemize}
Reconciliar; tornar amigo; pacificar.
\section{Congratulação}
\begin{itemize}
\item {Grp. gram.:f.}
\end{itemize}
\begin{itemize}
\item {Proveniência:(Lat. \textunderscore congratulatio\textunderscore )}
\end{itemize}
Acto de congratular-se.
\section{Congratulador}
\begin{itemize}
\item {Grp. gram.:m.}
\end{itemize}
Aquelle que se congratula.
\section{Congratulante}
\begin{itemize}
\item {Grp. gram.:adj.}
\end{itemize}
Que se congratula.
\section{Congratular}
\begin{itemize}
\item {Grp. gram.:v. i.}
\end{itemize}
\begin{itemize}
\item {Grp. gram.:V. p.}
\end{itemize}
\begin{itemize}
\item {Proveniência:(Lat. \textunderscore congratulari\textunderscore )}
\end{itemize}
Felicitar alguém.
Regozijar-se com o bem ou a satisfação de outrem.
Dirigir felicitações, parabens.
\section{Congratulatório}
\begin{itemize}
\item {Grp. gram.:adj.}
\end{itemize}
\begin{itemize}
\item {Proveniência:(De \textunderscore congratular\textunderscore )}
\end{itemize}
Que envolve congratulação.
\section{Congregação}
\begin{itemize}
\item {Grp. gram.:f.}
\end{itemize}
\begin{itemize}
\item {Proveniência:(Lat. \textunderscore congregatio\textunderscore )}
\end{itemize}
Acto ou effeito de congregar.
Assembleia.
Companhia de frades ou freiras.
Confraria.
\section{Congregacional}
\begin{itemize}
\item {Grp. gram.:adj.}
\end{itemize}
Relativo a congregações (religiosas).
\section{Congregacionalista}
\begin{itemize}
\item {Grp. gram.:m.}
\end{itemize}
\begin{itemize}
\item {Proveniência:(De \textunderscore congregacional\textunderscore )}
\end{itemize}
Sectário christão, nos Estados-Unidos.
\section{Congregado}
\begin{itemize}
\item {Grp. gram.:m.}
\end{itemize}
Membro de congregação religiosa.
\section{Congreganista}
\begin{itemize}
\item {Grp. gram.:adj.}
\end{itemize}
\begin{itemize}
\item {Utilização:Neol.}
\end{itemize}
\begin{itemize}
\item {Grp. gram.:M.}
\end{itemize}
\begin{itemize}
\item {Proveniência:(Fr. \textunderscore congreganiste\textunderscore )}
\end{itemize}
Relativo a congregações religiosas: \textunderscore propaganda congreganista\textunderscore .
Membro de congregação religiosa.
\section{Congregante}
\begin{itemize}
\item {Grp. gram.:adj.}
\end{itemize}
\begin{itemize}
\item {Grp. gram.:M.}
\end{itemize}
\begin{itemize}
\item {Proveniência:(Lat. \textunderscore congregans\textunderscore )}
\end{itemize}
Que congrega.
Membro de uma congregação.
\section{Congregar}
\begin{itemize}
\item {Grp. gram.:v. t.}
\end{itemize}
\begin{itemize}
\item {Proveniência:(Lat. \textunderscore congregare\textunderscore )}
\end{itemize}
Convocar; reunir.
Ligar.
\section{Congressional}
\begin{itemize}
\item {Grp. gram.:adj.}
\end{itemize}
\begin{itemize}
\item {Proveniência:(Do lat. \textunderscore congressio\textunderscore )}
\end{itemize}
Relativo a congresso.
\section{Congressista}
\begin{itemize}
\item {Grp. gram.:adj.}
\end{itemize}
\begin{itemize}
\item {Grp. gram.:M.}
\end{itemize}
Relativo a congresso.
Membro de um congresso.
\section{Congresso}
\begin{itemize}
\item {Grp. gram.:m.}
\end{itemize}
\begin{itemize}
\item {Proveniência:(Lat. \textunderscore congressus\textunderscore )}
\end{itemize}
Ajuntamento, reunião solenne, dos corpos legislativos, de sábios, de diplomatas, dos representantes do commércio, indústria, associações, etc., para negócios de interesse commum, público ou internacional.
\section{Congro}
\begin{itemize}
\item {Grp. gram.:m.}
\end{itemize}
\begin{itemize}
\item {Proveniência:(Lat. \textunderscore conger\textunderscore , \textunderscore congri\textunderscore )}
\end{itemize}
Espécie de peixe marinho; safio grande.
\section{Côngrua}
\begin{itemize}
\item {Grp. gram.:f.}
\end{itemize}
\begin{itemize}
\item {Proveniência:(De \textunderscore côngruo\textunderscore )}
\end{itemize}
Pagamento aos párochos, obtido, para sua conveniente sustentação, por meio de derrama parochial.
\section{Congruado}
\begin{itemize}
\item {Grp. gram.:adj.}
\end{itemize}
Que recebe côngrua.
\section{Congruamente}
\begin{itemize}
\item {Grp. gram.:adv.}
\end{itemize}
De modo côngruo.
\section{Congruário}
\begin{itemize}
\item {Grp. gram.:adj.}
\end{itemize}
(V.congruado)
\section{Congruência}
\begin{itemize}
\item {Grp. gram.:f.}
\end{itemize}
\begin{itemize}
\item {Proveniência:(Lat. \textunderscore congruentia\textunderscore )}
\end{itemize}
Harmonia de uma coisa ou facto com o fim a que se destina.
Conveniência.
Coherência.
Propriedade.
\section{Congruente}
\begin{itemize}
\item {Grp. gram.:adj.}
\end{itemize}
\begin{itemize}
\item {Proveniência:(Lat. \textunderscore congruens\textunderscore )}
\end{itemize}
Em que há congruência.
\section{Congruentemente}
\begin{itemize}
\item {Grp. gram.:adv.}
\end{itemize}
De modo congruente.
\section{Congruidade}
\begin{itemize}
\item {fónica:gru-i}
\end{itemize}
\begin{itemize}
\item {Grp. gram.:f.}
\end{itemize}
(V.congruência)
\section{Congruísmo}
\begin{itemize}
\item {Grp. gram.:m.}
\end{itemize}
\begin{itemize}
\item {Proveniência:(De \textunderscore côngrua\textunderscore )}
\end{itemize}
Systema theológico dos que sustentam que Deus dá aos homens graça côngrua, bastante.
\section{Congruísta}
\begin{itemize}
\item {Grp. gram.:m.}
\end{itemize}
Partidário do congruísmo.
\section{Côngruo}
\begin{itemize}
\item {Grp. gram.:adj.}
\end{itemize}
\begin{itemize}
\item {Grp. gram.:Loc. adv.}
\end{itemize}
\begin{itemize}
\item {Utilização:ant.}
\end{itemize}
\begin{itemize}
\item {Proveniência:(Lat. \textunderscore congruus\textunderscore )}
\end{itemize}
Adequado, apto.
Manifestado em termos precisos.
\textunderscore De côngruo\textunderscore , congruamente. Cf. \textunderscore Eufrosina\textunderscore , 99.
\section{Congueirão}
\begin{itemize}
\item {Grp. gram.:m.}
\end{itemize}
\begin{itemize}
\item {Utilização:Ant.}
\end{itemize}
\begin{itemize}
\item {Proveniência:(De \textunderscore congo\textunderscore , por \textunderscore congro\textunderscore )}
\end{itemize}
Pescador ou vendedor de congros. Cf. \textunderscore Anat. Joc.\textunderscore , I, 10.
\section{Conguês}
\begin{itemize}
\item {Grp. gram.:m.}
\end{itemize}
\begin{itemize}
\item {Grp. gram.:Adj.}
\end{itemize}
Homem natural do Congo.
Língua do Congo.
Relativo ao Congo.
\section{Conhaque}
\begin{itemize}
\item {Grp. gram.:m.}
\end{itemize}
\begin{itemize}
\item {Proveniência:(De \textunderscore Cognac\textunderscore , n. p.)}
\end{itemize}
Águardente, composta em Cognac, ou semelhante á que lá se fabríca.
\section{Cònhar}
\begin{itemize}
\item {Grp. gram.:v. t.}
\end{itemize}
\begin{itemize}
\item {Utilização:Prov.}
\end{itemize}
Varrer com o cónho. (Colhido em Turquel)
\section{Conhecedor}
\begin{itemize}
\item {Grp. gram.:m.  e  adj.}
\end{itemize}
O que conhece.
\section{Conhecença}
\begin{itemize}
\item {Grp. gram.:f.}
\end{itemize}
(V.conhecimento)
Prestação, que antigamente se pagava aos párochos, por certos rendimentos, em relação aos quaes não havia regra para se pagarem dízimos.
\section{Conhecente}
\begin{itemize}
\item {Grp. gram.:adj.}
\end{itemize}
\begin{itemize}
\item {Utilização:Ant.}
\end{itemize}
O mesmo que \textunderscore conhecedor\textunderscore .
\section{Conhecer}
\begin{itemize}
\item {Grp. gram.:v. t.}
\end{itemize}
\begin{itemize}
\item {Utilização:Euph.}
\end{itemize}
\begin{itemize}
\item {Grp. gram.:V. i.}
\end{itemize}
\begin{itemize}
\item {Proveniência:(Lat. \textunderscore cognoscere\textunderscore )}
\end{itemize}
Têr noção de; saber.
Têr relações com.
Têr ouvido.
Distinguir.
Julgar, avaliar.
Reconhecer.
Têr experimentado: \textunderscore conhecer desgostos\textunderscore .
Sentir a acção de.
Admittir.
Têr cópula carnal com.
Tomar conhecimento.
\section{Conhecidamente}
\begin{itemize}
\item {Grp. gram.:adv.}
\end{itemize}
De modo conhecido.
\section{Conhecido}
\begin{itemize}
\item {Grp. gram.:adj.}
\end{itemize}
\begin{itemize}
\item {Grp. gram.:M.}
\end{itemize}
Illustre.
Perito.
Indivíduo, de quem temos conhecimento, ou com quem temos ligeiras relações: \textunderscore os nossos conhecidos\textunderscore .
\section{Conhecimento}
\begin{itemize}
\item {Grp. gram.:m.}
\end{itemize}
Acto ou effeito de conhecer.
Direito de julgar.
Relações entre pessôas que, não sendo amigas nem inimigas, se encontram e falam: \textunderscore o João é do meu conhecimento\textunderscore .
Pessôa, com quem se tem relações.
Documento escrito; recibo de contribuïção.
\section{Conhecível}
\begin{itemize}
\item {Grp. gram.:adj.}
\end{itemize}
Que se póde conhecer.
\section{Conheçudo}
\begin{itemize}
\item {Utilização:Ant.}
\end{itemize}
\textunderscore Part.\textunderscore  de \textunderscore conhecer\textunderscore .
\section{Conhedo}
\begin{itemize}
\item {fónica:nhê}
\end{itemize}
\begin{itemize}
\item {Grp. gram.:m.}
\end{itemize}
Lugar, onde há muitos \textunderscore cônhos\textunderscore .
\section{Conhescer}
\begin{itemize}
\item {Grp. gram.:v. t.}
\end{itemize}
\begin{itemize}
\item {Utilização:Ant.}
\end{itemize}
O mesmo que \textunderscore conhecer\textunderscore .
\section{Conhinha}
\begin{itemize}
\item {Grp. gram.:f.}
\end{itemize}
Ave, o mesmo que \textunderscore abesconinha\textunderscore . (Colhido na Bairrada)
\section{Cônho}
\begin{itemize}
\item {Grp. gram.:m.}
\end{itemize}
\begin{itemize}
\item {Utilização:Des.}
\end{itemize}
\begin{itemize}
\item {Proveniência:(Lat. \textunderscore cuneus\textunderscore )}
\end{itemize}
Penedo insulado e redondo, em meio de um rio.
\section{Cónho}
\begin{itemize}
\item {Grp. gram.:m.}
\end{itemize}
\begin{itemize}
\item {Utilização:Prov.}
\end{itemize}
\begin{itemize}
\item {Grp. gram.:M. pl.}
\end{itemize}
\begin{itemize}
\item {Utilização:Prov.}
\end{itemize}
Vassoira espalmada, com que, nas eiras, ao padejar o grão, se vão retirando alguns fragmentos de palha ou carolo. (Colhido em Turquel)
Mistura de sementes de feno e de outras plantas, para sementeira de pastagens. (Colhido em Óbidos)
\section{Conhoscer}
\begin{itemize}
\item {Grp. gram.:v. t.}
\end{itemize}
\begin{itemize}
\item {Utilização:Ant.}
\end{itemize}
O mesmo quo \textunderscore conhecer\textunderscore . Cf. \textunderscore Liv. III\textunderscore  de D. Dinis, na Tôrre do Tombo.
\section{Conicar}
\begin{itemize}
\item {Grp. gram.:v. t.}
\end{itemize}
\begin{itemize}
\item {Utilização:Prov.}
\end{itemize}
\begin{itemize}
\item {Utilização:minh.}
\end{itemize}
O mesmo que \textunderscore enconapar\textunderscore ; refegar (uma costura), por imperícia.
\section{Conicidade}
\begin{itemize}
\item {Grp. gram.:f.}
\end{itemize}
\begin{itemize}
\item {Proveniência:(De \textunderscore cónico\textunderscore )}
\end{itemize}
Fórma cónica.
\section{Conicina}
\begin{itemize}
\item {Grp. gram.:f.}
\end{itemize}
\begin{itemize}
\item {Proveniência:(Do gr. \textunderscore koneion\textunderscore )}
\end{itemize}
Alcaloide, que se encontra especialmente na cicuta.
\section{Cónico}
\begin{itemize}
\item {Grp. gram.:adj.}
\end{itemize}
\begin{itemize}
\item {Proveniência:(Gr. \textunderscore konikus\textunderscore )}
\end{itemize}
Que tem a fórma de cóne.
\section{Coníco}
\begin{itemize}
\item {Grp. gram.:m.}
\end{itemize}
\begin{itemize}
\item {Utilização:Prov.}
\end{itemize}
\begin{itemize}
\item {Utilização:trasm.}
\end{itemize}
Acto ou effeito de conicar.
\section{Conídeos}
\begin{itemize}
\item {Grp. gram.:m. pl.}
\end{itemize}
\begin{itemize}
\item {Proveniência:(Do gr. \textunderscore konis\textunderscore  + \textunderscore eidos\textunderscore )}
\end{itemize}
Sementes do mildio.
\section{Conídia}
\begin{itemize}
\item {Grp. gram.:f.}
\end{itemize}
\begin{itemize}
\item {Proveniência:(Do gr. \textunderscore konis\textunderscore )}
\end{itemize}
Espécie de pó, que recobre os lichens.
\section{Coníferas}
\begin{itemize}
\item {Grp. gram.:f. pl.}
\end{itemize}
\begin{itemize}
\item {Proveniência:(De \textunderscore conífero\textunderscore )}
\end{itemize}
Ordem de árvores, que, como o pinheiro, produzem frutos em fórma de cóne.
\section{Conífero}
\begin{itemize}
\item {Grp. gram.:adj.}
\end{itemize}
\begin{itemize}
\item {Proveniência:(Lat. \textunderscore conifer\textunderscore )}
\end{itemize}
Cujo fruto tem fórma de cóne.
\section{Conifloro}
\begin{itemize}
\item {Grp. gram.:adj.}
\end{itemize}
\begin{itemize}
\item {Utilização:Bot.}
\end{itemize}
\begin{itemize}
\item {Proveniência:(De \textunderscore cóne\textunderscore  + \textunderscore flor\textunderscore )}
\end{itemize}
Que tem flôres em fórma cónica.
\section{Coniforme}
\begin{itemize}
\item {Grp. gram.:adj.}
\end{itemize}
\begin{itemize}
\item {Proveniência:(Do lat. \textunderscore conus\textunderscore  + \textunderscore forma\textunderscore )}
\end{itemize}
Que tem fórma de cóne.
\section{Cóniga}
\begin{itemize}
\item {Grp. gram.:f.}
\end{itemize}
Planta conífera.
\section{Cónigo}
\begin{itemize}
\item {Grp. gram.:m.}
\end{itemize}
\begin{itemize}
\item {Proveniência:(Do lat. \textunderscore canonicus\textunderscore )}
\end{itemize}
Clérigo secular, que faz parte de um cabido, e ao qual impendem obrigações religiosas numa sé ou collegiada.
\section{Conimbricense}
\begin{itemize}
\item {Grp. gram.:adj.}
\end{itemize}
\begin{itemize}
\item {Grp. gram.:M.}
\end{itemize}
\begin{itemize}
\item {Proveniência:(Do lat. \textunderscore Conimbrica\textunderscore , n. p.)}
\end{itemize}
Relativo a Coimbra.
Aquelle que é natural de Coimbra.
\section{Conio}
\begin{itemize}
\item {Grp. gram.:m.}
\end{itemize}
\begin{itemize}
\item {Proveniência:(Do gr. \textunderscore koneion\textunderscore )}
\end{itemize}
Gênero de plantas umbellíferas.
\section{Coniothýrio}
\begin{itemize}
\item {Grp. gram.:m.}
\end{itemize}
\begin{itemize}
\item {Proveniência:(Do gr. \textunderscore konis\textunderscore , poeira)}
\end{itemize}
Doença dos viveiros das videiras.
\section{Coniotírio}
\begin{itemize}
\item {Grp. gram.:m.}
\end{itemize}
\begin{itemize}
\item {Proveniência:(Do gr. \textunderscore konis\textunderscore , poeira)}
\end{itemize}
Doença dos viveiros das videiras.
\section{Conirostros}
\begin{itemize}
\item {fónica:rós}
\end{itemize}
\begin{itemize}
\item {Grp. gram.:m. pl.}
\end{itemize}
\begin{itemize}
\item {Proveniência:(Do lat. \textunderscore conus\textunderscore  + \textunderscore rostrum\textunderscore )}
\end{itemize}
Família de aves caracterizadas por um bico curto e cónico como o dos pardaes.
\section{Conirrostros}
\begin{itemize}
\item {fónica:rós}
\end{itemize}
\begin{itemize}
\item {Grp. gram.:m. pl.}
\end{itemize}
\begin{itemize}
\item {Proveniência:(Do lat. \textunderscore conus\textunderscore  + \textunderscore rostrum\textunderscore )}
\end{itemize}
Família de aves caracterizadas por um bico curto e cónico como o dos pardaes.
\section{Conivalve}
\begin{itemize}
\item {Grp. gram.:adj.}
\end{itemize}
\begin{itemize}
\item {Proveniência:(De \textunderscore cóne\textunderscore  + \textunderscore valva\textunderscore )}
\end{itemize}
Que tem concha cónica.
\section{Conja}
\begin{itemize}
\item {Grp. gram.:f.}
\end{itemize}
Antiga medida de Çofala, correspondente a pouco mais de meio litro.
\section{Conjecção}
\begin{itemize}
\item {Grp. gram.:f.}
\end{itemize}
\begin{itemize}
\item {Utilização:Ant.}
\end{itemize}
\begin{itemize}
\item {Proveniência:(Lat. \textunderscore conjectio\textunderscore )}
\end{itemize}
Condição.
Causa.
\section{Conjector}
\begin{itemize}
\item {Grp. gram.:m.}
\end{itemize}
\begin{itemize}
\item {Utilização:Des.}
\end{itemize}
Aquelle que faz conjecturas; aquelle que explica sonhos. Cf. Macedo, \textunderscore Burros\textunderscore , 130.
\section{Conjectura}
\begin{itemize}
\item {Grp. gram.:f.}
\end{itemize}
\begin{itemize}
\item {Proveniência:(Lat. \textunderscore conjectura\textunderscore )}
\end{itemize}
Opinião com fundamento incerto.
Supposição, hypóthese.
\section{Conjecturadamente}
\begin{itemize}
\item {Grp. gram.:adv.}
\end{itemize}
Por conjectura.
\section{Conjecturador}
\begin{itemize}
\item {Grp. gram.:m.}
\end{itemize}
Aquelle que conjectura.
\section{Conjectural}
\begin{itemize}
\item {Grp. gram.:adj.}
\end{itemize}
\begin{itemize}
\item {Proveniência:(Lat. \textunderscore conjecturalis\textunderscore )}
\end{itemize}
Baseado em conjectura.
\section{Conjecturalmente}
\begin{itemize}
\item {Grp. gram.:adv.}
\end{itemize}
De modo conjectural.
\section{Conjecturar}
\begin{itemize}
\item {Grp. gram.:v. t.}
\end{itemize}
\begin{itemize}
\item {Proveniência:(Lat. \textunderscore conjecturare\textunderscore )}
\end{itemize}
Julgar por conjectura; suppôr; presumir.
\section{Conjecturável}
\begin{itemize}
\item {Grp. gram.:adj.}
\end{itemize}
Que se póde conjecturar.
\section{Conjecturista}
\begin{itemize}
\item {Grp. gram.:m.}
\end{itemize}
Aquelle que faz conjecturas. Cf. Castilho, \textunderscore Fastos\textunderscore , I, 182.
\section{Conjeitura}
\begin{itemize}
\item {Grp. gram.:f.}
\end{itemize}
\begin{itemize}
\item {Utilização:Ant.}
\end{itemize}
Conjectura. Cf. G. Vicente, I, 154.
\section{Conjeiturar}
\begin{itemize}
\item {Grp. gram.:v. t.}
\end{itemize}
\begin{itemize}
\item {Utilização:Ant.}
\end{itemize}
O mesmo que \textunderscore conjecturar\textunderscore . Cf. \textunderscore Eufrosina\textunderscore , 338.
\section{Conjugação}
\begin{itemize}
\item {Grp. gram.:f.}
\end{itemize}
\begin{itemize}
\item {Proveniência:(Lat. \textunderscore conjugatio\textunderscore )}
\end{itemize}
Flexão dos verbos, por tempos e pessôas.
Acto de conjugar (verbos).
Juncção.
\section{Conjugal}
\begin{itemize}
\item {Grp. gram.:adj.}
\end{itemize}
\begin{itemize}
\item {Proveniência:(Lat. \textunderscore conjugalis\textunderscore )}
\end{itemize}
Relativo ao casamento, a cônjuges.
\section{Conjugalmente}
\begin{itemize}
\item {Grp. gram.:adv.}
\end{itemize}
De modo conjugal: \textunderscore viver conjugalmente\textunderscore .
\section{Conjugar}
\begin{itemize}
\item {Grp. gram.:v. t.}
\end{itemize}
\begin{itemize}
\item {Proveniência:(Lat. \textunderscore conjugare\textunderscore )}
\end{itemize}
Unir juntamente.
Expor as flexões de (um verbo).
\section{Conjugativo}
\begin{itemize}
\item {Grp. gram.:adj.}
\end{itemize}
\begin{itemize}
\item {Utilização:Philol.}
\end{itemize}
Que diz respeito a conjugação.
\section{Conjugável}
\begin{itemize}
\item {Grp. gram.:adj.}
\end{itemize}
Que se póde conjugar.
\section{Conjuge}
\begin{itemize}
\item {Grp. gram.:m.}
\end{itemize}
\begin{itemize}
\item {Grp. gram.:Adj.}
\end{itemize}
\begin{itemize}
\item {Proveniência:(Lat. \textunderscore conjux\textunderscore , de \textunderscore conjungere\textunderscore )}
\end{itemize}
Cada uma das pessôas, que estão recíprocamente ligadas pelo casamento.
Que é casado. Cf. Castilho, \textunderscore Fastos\textunderscore , I, 266.
\section{Conjungicida}
\begin{itemize}
\item {Grp. gram.:m.  e  f.}
\end{itemize}
\begin{itemize}
\item {Proveniência:(Do lat. \textunderscore conjux\textunderscore  + \textunderscore caedere\textunderscore )}
\end{itemize}
Pessôa casada, que mata o seu cônjuge. Cf. Camillo, \textunderscore Vinho do Porto\textunderscore  32.
\section{Conjugicídio}
\begin{itemize}
\item {Grp. gram.:m.}
\end{itemize}
Crime de quem mata o seu cônjuge. Cf. Camillo, \textunderscore Narcót.\textunderscore , I, 104.
(Cp. \textunderscore conjungicida\textunderscore )
\section{Conjúgio}
\begin{itemize}
\item {Grp. gram.:m.}
\end{itemize}
\begin{itemize}
\item {Proveniência:(Lat. \textunderscore conjugium\textunderscore )}
\end{itemize}
União conjugal; casamento.
\section{Conjuiz}
\begin{itemize}
\item {Grp. gram.:m.}
\end{itemize}
\begin{itemize}
\item {Utilização:Des.}
\end{itemize}
\begin{itemize}
\item {Proveniência:(Do lat. \textunderscore cum\textunderscore  + \textunderscore judex\textunderscore )}
\end{itemize}
Cada um dos juizes que intervêm no mesmo julgamento.
\section{Conjunção}
\begin{itemize}
\item {Grp. gram.:f.}
\end{itemize}
\begin{itemize}
\item {Utilização:Astron.}
\end{itemize}
\begin{itemize}
\item {Utilização:Gram.}
\end{itemize}
\begin{itemize}
\item {Proveniência:(Lat. \textunderscore conjunctio\textunderscore )}
\end{itemize}
União.
Conjuntura; oportunidade.
Encontro aparente dos astros no mesmo ponto, em relação á Terra.
Partícula, que liga duas palavras, duas frases ou duas orações.
\section{Conjuncção}
\begin{itemize}
\item {Grp. gram.:f.}
\end{itemize}
\begin{itemize}
\item {Utilização:Astron.}
\end{itemize}
\begin{itemize}
\item {Utilização:Gram.}
\end{itemize}
\begin{itemize}
\item {Proveniência:(Lat. \textunderscore conjunctio\textunderscore )}
\end{itemize}
União.
Conjuntura; opportunidade.
Encontro apparente dos astros no mesmo ponto, em relação á Terra.
Partícula, que liga duas palavras, duas phrases ou duas orações.
\section{Conjunctar}
\begin{itemize}
\item {Grp. gram.:v. t.}
\end{itemize}
\begin{itemize}
\item {Utilização:Des.}
\end{itemize}
Tornar conjuncto.
\section{Conjunctiva}
\begin{itemize}
\item {Grp. gram.:f.}
\end{itemize}
\begin{itemize}
\item {Proveniência:(De \textunderscore conjunctivo\textunderscore )}
\end{itemize}
Membrana mucosa, que fórra a parte externa do globo ocular, exceptuada a córnea, e que fórra a face posterior das pálpebras, ligando-as ao mesmo globo ocular.
\section{Conjunctivite}
\begin{itemize}
\item {Grp. gram.:f.}
\end{itemize}
Inflammação da conjunctiva.
\section{Conjunctivo}
\begin{itemize}
\item {Grp. gram.:adj.}
\end{itemize}
\begin{itemize}
\item {Utilização:Gram.}
\end{itemize}
\begin{itemize}
\item {Grp. gram.:M.}
\end{itemize}
\begin{itemize}
\item {Utilização:Gram.}
\end{itemize}
\begin{itemize}
\item {Proveniência:(Lat. \textunderscore conjunctivus\textunderscore )}
\end{itemize}
Que une.
Que liga palavras ou proposições, que tém o valor de uma conjuncção grammatical: \textunderscore locução conjunctiva\textunderscore .
Diz-se do modo dos verbos, em que se exprime uma acção ou relação dependente de outra.
Modo conjunctivo.
\section{Conjuncto}
\begin{itemize}
\item {Grp. gram.:adj.}
\end{itemize}
\begin{itemize}
\item {Grp. gram.:M.}
\end{itemize}
\begin{itemize}
\item {Proveniência:(Lat. \textunderscore conjunctus\textunderscore )}
\end{itemize}
Junto simultaneamente.
Ligado.
Próximo; annexo.
Complexo; reunião das partes que constituem um todo; totalidade.
\section{Conjunctório}
\begin{itemize}
\item {Grp. gram.:m.}
\end{itemize}
\begin{itemize}
\item {Utilização:Bot.}
\end{itemize}
\begin{itemize}
\item {Proveniência:(De \textunderscore conjuncto\textunderscore )}
\end{itemize}
Peça foliácea, que reveste a urna dos musgos.
\section{Conjunctura}
\begin{itemize}
\item {Grp. gram.:f.}
\end{itemize}
\begin{itemize}
\item {Proveniência:(De \textunderscore conjuncto\textunderscore )}
\end{itemize}
Encontro de acontecimentos.
Acontecimento.
Opportunidade; ensejo.
\section{Conjungir}
\begin{itemize}
\item {Grp. gram.:v. t.}
\end{itemize}
Ligar, casar. Cf. Filinto, \textunderscore D. Man.\textunderscore , I, 150.
\section{Conjungo}
\begin{itemize}
\item {Grp. gram.:m.}
\end{itemize}
\begin{itemize}
\item {Utilização:Pop.}
\end{itemize}
\begin{itemize}
\item {Proveniência:(Lat. \textunderscore conjungo\textunderscore , 1.^a pess. do pres. do indic. de \textunderscore conjungere\textunderscore )}
\end{itemize}
Casamento.
\section{Conjunta}
\begin{itemize}
\item {Grp. gram.:f.}
\end{itemize}
\begin{itemize}
\item {Utilização:Mús.}
\end{itemize}
\begin{itemize}
\item {Utilização:ant.}
\end{itemize}
Meio tom chromático ou accidental.
\section{Conjuntar}
\begin{itemize}
\item {Grp. gram.:v. t.}
\end{itemize}
\begin{itemize}
\item {Utilização:Des.}
\end{itemize}
Tornar conjunto.
\section{Conjuntiva}
\begin{itemize}
\item {Grp. gram.:f.}
\end{itemize}
\begin{itemize}
\item {Proveniência:(De \textunderscore conjuntivo\textunderscore )}
\end{itemize}
Membrana mucosa, que fórra a parte externa do globo ocular, exceptuada a córnea, e que fórra a face posterior das pálpebras, ligando-as ao mesmo globo ocular.
\section{Conjuntivite}
\begin{itemize}
\item {Grp. gram.:f.}
\end{itemize}
Inflammação da conjuntiva.
\section{Conjuntivo}
\begin{itemize}
\item {Grp. gram.:adj.}
\end{itemize}
\begin{itemize}
\item {Utilização:Gram.}
\end{itemize}
\begin{itemize}
\item {Grp. gram.:M.}
\end{itemize}
\begin{itemize}
\item {Utilização:Gram.}
\end{itemize}
\begin{itemize}
\item {Proveniência:(Lat. \textunderscore conjunctivus\textunderscore )}
\end{itemize}
Que une.
Que liga palavras ou proposições, que tém o valor de uma conjuncção grammatical: \textunderscore locução conjuntiva\textunderscore .
Diz-se do modo dos verbos, em que se exprime uma acção ou relação dependente de outra.
Modo conjuntivo.
\section{Conjunto}
\begin{itemize}
\item {Grp. gram.:adj.}
\end{itemize}
\begin{itemize}
\item {Grp. gram.:M.}
\end{itemize}
\begin{itemize}
\item {Proveniência:(Lat. \textunderscore conjunctus\textunderscore )}
\end{itemize}
Junto simultaneamente.
Ligado.
Próximo; annexo.
Complexo; reunião das partes que constituem um todo; totalidade.
\section{Conjuntório}
\begin{itemize}
\item {Grp. gram.:m.}
\end{itemize}
\begin{itemize}
\item {Utilização:Bot.}
\end{itemize}
\begin{itemize}
\item {Proveniência:(De \textunderscore conjunto\textunderscore )}
\end{itemize}
Peça foliácea, que reveste a urna dos musgos.
\section{Conjuntura}
\begin{itemize}
\item {Grp. gram.:f.}
\end{itemize}
\begin{itemize}
\item {Proveniência:(De \textunderscore conjunto\textunderscore )}
\end{itemize}
Encontro de acontecimentos.
Acontecimento.
Opportunidade; ensejo.
\section{Conjura}
\begin{itemize}
\item {Grp. gram.:f.}
\end{itemize}
\begin{itemize}
\item {Utilização:Fam.}
\end{itemize}
\begin{itemize}
\item {Proveniência:(De \textunderscore conjurar\textunderscore )}
\end{itemize}
Conjuro.
Conjuração.
\section{Conjuração}
\begin{itemize}
\item {Grp. gram.:f.}
\end{itemize}
\begin{itemize}
\item {Proveniência:(Lat. \textunderscore conjuratio\textunderscore )}
\end{itemize}
Acto de conjurar.
Ajuntamento de pessôas conjuradas.
Conjuro.
\section{Conjurado}
\begin{itemize}
\item {Grp. gram.:m.}
\end{itemize}
Aquelle que conjura.
\section{Conjurador}
\begin{itemize}
\item {Grp. gram.:m.}
\end{itemize}
\begin{itemize}
\item {Proveniência:(De \textunderscore conjurar\textunderscore )}
\end{itemize}
Aquelle que faz conjuros.
\section{Conjurante}
\begin{itemize}
\item {Grp. gram.:adj.}
\end{itemize}
Que conjura.
\section{Conjurar}
\begin{itemize}
\item {Grp. gram.:v. t.}
\end{itemize}
\begin{itemize}
\item {Utilização:Fig.}
\end{itemize}
\begin{itemize}
\item {Grp. gram.:V. i.}
\end{itemize}
\begin{itemize}
\item {Grp. gram.:V. p.}
\end{itemize}
\begin{itemize}
\item {Utilização:Fig.}
\end{itemize}
\begin{itemize}
\item {Proveniência:(Lat. \textunderscore conjurare\textunderscore )}
\end{itemize}
Ajuramentar, convocar para conspiração.
Maquinar.
Supplicar.
Desviar, evitar.
Fazer conjuros.
Conspirar.
Insurgir-se.
Armar tramas contra alguém ou contra alguma coisa.
Ligar-se em conjuração.
Lastimar-se.
\section{Conato}
\begin{itemize}
\item {Grp. gram.:adj.}
\end{itemize}
\begin{itemize}
\item {Proveniência:(Lat. \textunderscore connatus\textunderscore )}
\end{itemize}
Inato.
\section{Conatural}
\begin{itemize}
\item {Grp. gram.:adj.}
\end{itemize}
\begin{itemize}
\item {Proveniência:(Lat. \textunderscore connaturalis\textunderscore )}
\end{itemize}
Congênito.
Conforme á natureza.
Apropriado.
\section{Conaturalizado}
\begin{itemize}
\item {Grp. gram.:adj.}
\end{itemize}
\begin{itemize}
\item {Proveniência:(De \textunderscore con...\textunderscore  + \textunderscore naturalizado\textunderscore )}
\end{itemize}
Naturalizado com outrem.
\section{Conectivo}
\begin{itemize}
\item {Utilização:Bot.}
\end{itemize}
\begin{itemize}
\item {Grp. gram.:Adj.}
\end{itemize}
\begin{itemize}
\item {Proveniência:(Do lat. \textunderscore connectere\textunderscore )}
\end{itemize}
Membrana, que une as duas células da antera.
Que liga.
\section{Conexão}
\begin{itemize}
\item {Grp. gram.:f.}
\end{itemize}
\begin{itemize}
\item {Proveniência:(Lat. \textunderscore connexio\textunderscore )}
\end{itemize}
Ligação; nexo.
Dependência, analogia.
\section{Conexidade}
\begin{itemize}
\item {Grp. gram.:f.}
\end{itemize}
Qualidade daquilo que é conexo.
\section{Conexivo}
\begin{itemize}
\item {Grp. gram.:adj.}
\end{itemize}
\begin{itemize}
\item {Proveniência:(Lat. \textunderscore connexivus\textunderscore )}
\end{itemize}
Relativo á conexão.
Copulativo.
\section{Conexo}
\begin{itemize}
\item {Grp. gram.:adj.}
\end{itemize}
\begin{itemize}
\item {Proveniência:(Lat. \textunderscore connexus\textunderscore )}
\end{itemize}
Que tem ou em que há conexão.
\section{Conivência}
\begin{itemize}
\item {Grp. gram.:f.}
\end{itemize}
\begin{itemize}
\item {Proveniência:(Lat. \textunderscore conniventia\textunderscore )}
\end{itemize}
Acto de sêr conivente.
\section{Conivente}
\begin{itemize}
\item {Grp. gram.:adj.}
\end{itemize}
\begin{itemize}
\item {Proveniência:(Lat. \textunderscore connivens\textunderscore )}
\end{itemize}
Que finge não vêr o mal que outrem pratica.
Conluiado; cúmplice.
Que se aproxima, que se toca.
\section{Conjuratório}
\begin{itemize}
\item {Grp. gram.:adj.}
\end{itemize}
\begin{itemize}
\item {Proveniência:(De \textunderscore conjurar\textunderscore )}
\end{itemize}
Relativo ao conjuro.
\section{Conjuro}
\begin{itemize}
\item {Grp. gram.:m.}
\end{itemize}
\begin{itemize}
\item {Proveniência:(De \textunderscore conjurar\textunderscore )}
\end{itemize}
Invocação mágica.
Palavras imperativas, que se dirigem ao demónio ou ás almas do outro mundo; exorcismo.
\section{Conloiar}
\begin{itemize}
\item {Grp. gram.:v. t.}
\end{itemize}
\begin{itemize}
\item {Utilização:Prov.}
\end{itemize}
O mesmo que \textunderscore conluiar\textunderscore .
\section{Conloio}
\begin{itemize}
\item {Grp. gram.:m.}
\end{itemize}
\begin{itemize}
\item {Utilização:Prov.}
\end{itemize}
O mesmo que \textunderscore conluio\textunderscore :«\textunderscore eu diria quem dava o exemplo dos conloios e veniagas\textunderscore ». Camillo, \textunderscore Cav. em Ruinas\textunderscore , 61.
\section{Conluiadamente}
\begin{itemize}
\item {Grp. gram.:adv.}
\end{itemize}
Com conluio.
Por conluio.
\section{Conluiar}
\begin{itemize}
\item {Grp. gram.:v. i.}
\end{itemize}
Unir em conluio.
Tramar.
Fraudar, de combinação com outrem.
\section{Conluio}
\begin{itemize}
\item {Grp. gram.:m.}
\end{itemize}
\begin{itemize}
\item {Proveniência:(Lat. \textunderscore colludium\textunderscore )}
\end{itemize}
Maquinação, combinação, entre duas ou mais pessôas, para prejudicar outrem.
Conspiração.
\section{Connato}
\begin{itemize}
\item {Grp. gram.:adj.}
\end{itemize}
\begin{itemize}
\item {Proveniência:(Lat. \textunderscore connatus\textunderscore )}
\end{itemize}
Innato.
\section{Connatural}
\begin{itemize}
\item {Grp. gram.:adj.}
\end{itemize}
\begin{itemize}
\item {Proveniência:(Lat. \textunderscore connaturalis\textunderscore )}
\end{itemize}
Congênito.
Conforme á natureza.
Apropriado.
\section{Connaturalizado}
\begin{itemize}
\item {Grp. gram.:adj.}
\end{itemize}
\begin{itemize}
\item {Proveniência:(De \textunderscore con...\textunderscore  + \textunderscore naturalizado\textunderscore )}
\end{itemize}
Naturalizado com outrem.
\section{Connectivo}
\begin{itemize}
\item {Utilização:Bot.}
\end{itemize}
\begin{itemize}
\item {Grp. gram.:Adj.}
\end{itemize}
\begin{itemize}
\item {Proveniência:(Do lat. \textunderscore connectere\textunderscore )}
\end{itemize}
Membrana, que une as duas céllulas da anthera.
Que liga.
\section{Connexão}
\begin{itemize}
\item {Grp. gram.:f.}
\end{itemize}
\begin{itemize}
\item {Proveniência:(Lat. \textunderscore connexio\textunderscore )}
\end{itemize}
Ligação; nexo.
Dependência, analogia.
\section{Connexidade}
\begin{itemize}
\item {Grp. gram.:f.}
\end{itemize}
Qualidade daquillo que é connexo.
\section{Connexivo}
\begin{itemize}
\item {Grp. gram.:adj.}
\end{itemize}
\begin{itemize}
\item {Proveniência:(Lat. \textunderscore connexivus\textunderscore )}
\end{itemize}
Relativo á connexão.
Copulativo.
\section{Connexo}
\begin{itemize}
\item {Grp. gram.:adj.}
\end{itemize}
\begin{itemize}
\item {Proveniência:(Lat. \textunderscore connexus\textunderscore )}
\end{itemize}
Que tem ou em que há connexão.
\section{Connivência}
\begin{itemize}
\item {Grp. gram.:f.}
\end{itemize}
\begin{itemize}
\item {Proveniência:(Lat. \textunderscore conniventia\textunderscore )}
\end{itemize}
Acto de sêr connivente.
\section{Connivente}
\begin{itemize}
\item {Grp. gram.:adj.}
\end{itemize}
\begin{itemize}
\item {Proveniência:(Lat. \textunderscore connivens\textunderscore )}
\end{itemize}
Que finge não vêr o mal que outrem pratica.
Conluiado; cúmplice.
Que se aproxima, que se toca.
\section{Connotação}
\begin{itemize}
\item {Grp. gram.:f.}
\end{itemize}
\begin{itemize}
\item {Proveniência:(De \textunderscore con...\textunderscore  + \textunderscore notação\textunderscore )}
\end{itemize}
Dependência, que se nota entre duas ou mais coisas.
\section{Connubial}
\begin{itemize}
\item {Grp. gram.:adj.}
\end{itemize}
\begin{itemize}
\item {Proveniência:(Lat. \textunderscore connubialis\textunderscore )}
\end{itemize}
Relativo a connúbio.
\section{Connúbio}
\begin{itemize}
\item {Grp. gram.:m.}
\end{itemize}
\begin{itemize}
\item {Utilização:Fig.}
\end{itemize}
\begin{itemize}
\item {Proveniência:(Lat. \textunderscore connubium\textunderscore )}
\end{itemize}
Matrimónio.
União.
\section{Connumerar}
\begin{itemize}
\item {Grp. gram.:v. t.}
\end{itemize}
\begin{itemize}
\item {Proveniência:(Lat. \textunderscore connumerare\textunderscore )}
\end{itemize}
Contar juntamente.
Referir ao mesmo número.
\section{Conóbia}
\begin{itemize}
\item {Grp. gram.:f.}
\end{itemize}
Gênero de plantas primuláceas.
\section{Conocárpio}
\begin{itemize}
\item {Grp. gram.:adj.}
\end{itemize}
\begin{itemize}
\item {Utilização:Bot.}
\end{itemize}
Que tem frutos cónicos.
\section{Conocéfalo}
\begin{itemize}
\item {Grp. gram.:m.}
\end{itemize}
\begin{itemize}
\item {Proveniência:(Do gr. \textunderscore konos\textunderscore  + \textunderscore kephale\textunderscore )}
\end{itemize}
Gênero de plantas artocárpeas.
\section{Conocéphalo}
\begin{itemize}
\item {Grp. gram.:m.}
\end{itemize}
\begin{itemize}
\item {Proveniência:(Do gr. \textunderscore konos\textunderscore  + \textunderscore kephale\textunderscore )}
\end{itemize}
Gênero de plantas artocárpeas.
\section{Conocer}
\textunderscore v. t.\textunderscore  (e der.)
Fórma ant. de \textunderscore conhecer\textunderscore , etc.
\section{Conclínio}
\begin{itemize}
\item {Grp. gram.:m.}
\end{itemize}
Gênero de plantas synanthéreas.
\section{Conoçudo}
\begin{itemize}
\item {Utilização:Ant.}
\end{itemize}
O mesmo que \textunderscore conheçudo\textunderscore .
\section{Conoidal}
\begin{itemize}
\item {Grp. gram.:adj.}
\end{itemize}
\begin{itemize}
\item {Proveniência:(De \textunderscore conoide\textunderscore )}
\end{itemize}
Que tem fórma de cóne.
\section{Conoide}
\begin{itemize}
\item {Grp. gram.:m.}
\end{itemize}
\begin{itemize}
\item {Grp. gram.:Adj.}
\end{itemize}
\begin{itemize}
\item {Proveniência:(Do gr. \textunderscore konos\textunderscore  + \textunderscore eidos\textunderscore )}
\end{itemize}
Corpo semelhante a um cóne.
Que tem fórma de cóne.
\section{Conomear}
\begin{itemize}
\item {Grp. gram.:v. t.}
\end{itemize}
\begin{itemize}
\item {Utilização:Ant.}
\end{itemize}
\begin{itemize}
\item {Proveniência:(De \textunderscore co...\textunderscore  + \textunderscore nomear\textunderscore )}
\end{itemize}
Nomear (alguém) juntamente com outrem.
\section{Conopleia}
\begin{itemize}
\item {Grp. gram.:f.}
\end{itemize}
Gênero de cogumelos parasitos.
\section{Conoscer}
\begin{itemize}
\item {Grp. gram.:v. t.}
\end{itemize}
\begin{itemize}
\item {Utilização:Ant.}
\end{itemize}
O mesmo que \textunderscore conhecer\textunderscore .
\section{Conosco}
\begin{itemize}
\item {fónica:nôs}
\end{itemize}
\begin{itemize}
\item {Grp. gram.:loc. pron.}
\end{itemize}
Em companhia de nós.
A nosso respeito.
(Flexão do pronome \textunderscore nós\textunderscore , precedido da prep. \textunderscore com\textunderscore )
\section{Conospermo}
\begin{itemize}
\item {Grp. gram.:m.}
\end{itemize}
\begin{itemize}
\item {Proveniência:(Do gr. \textunderscore konos\textunderscore  + \textunderscore sperma\textunderscore )}
\end{itemize}
Planta proteácea da Austrália.
\section{Conóstomo}
\begin{itemize}
\item {Grp. gram.:m.}
\end{itemize}
\begin{itemize}
\item {Proveniência:(Do gr. \textunderscore konos\textunderscore  + \textunderscore stoma\textunderscore )}
\end{itemize}
Gênero de musgos.
\section{Conotação}
\begin{itemize}
\item {Grp. gram.:f.}
\end{itemize}
\begin{itemize}
\item {Proveniência:(De \textunderscore con...\textunderscore  + \textunderscore notação\textunderscore )}
\end{itemize}
Dependência, que se nota entre duas ou mais coisas.
\section{Conóvia}
\begin{itemize}
\item {Grp. gram.:f.}
\end{itemize}
(V.conóbia)
\section{Conqueiro}
\begin{itemize}
\item {Grp. gram.:m.}
\end{itemize}
\begin{itemize}
\item {Utilização:Ant.}
\end{itemize}
\begin{itemize}
\item {Utilização:Prov.}
\end{itemize}
\begin{itemize}
\item {Utilização:trasm.}
\end{itemize}
\begin{itemize}
\item {Proveniência:(De \textunderscore conca\textunderscore )}
\end{itemize}
Aquelle que faz tijelas de madeira.
Espécie de sapo.
\section{Conquém}
\begin{itemize}
\item {Grp. gram.:m.}
\end{itemize}
\begin{itemize}
\item {Utilização:Bras}
\end{itemize}
\begin{itemize}
\item {Proveniência:(T. onom.)}
\end{itemize}
Gallinha-da-Índia; pintada.
\section{Conquerer}
\begin{itemize}
\item {Grp. gram.:v. t.}
\end{itemize}
\begin{itemize}
\item {Utilização:Ant.}
\end{itemize}
O mesmo que \textunderscore conquerir\textunderscore .
\section{Conquerir}
\begin{itemize}
\item {Grp. gram.:v. t.}
\end{itemize}
\begin{itemize}
\item {Utilização:Ant.}
\end{itemize}
\begin{itemize}
\item {Proveniência:(Lat. \textunderscore conquirere\textunderscore )}
\end{itemize}
O mesmo que \textunderscore conquistar\textunderscore .
\section{Conquial}
\begin{itemize}
\item {Grp. gram.:m.}
\end{itemize}
Categoria de sacerdotes chinêses. Cf. \textunderscore Peregrinação\textunderscore , CVIII.
(Relaciona-se com \textunderscore conchal\textunderscore ?)
\section{Conquíbus}
\begin{itemize}
\item {Grp. gram.:m. pl.}
\end{itemize}
\begin{itemize}
\item {Utilização:Pop.}
\end{itemize}
O mesmo que \textunderscore cum-quibus\textunderscore . Cf. Camillo, \textunderscore Caveira\textunderscore , 468.
\section{Conquícola}
\begin{itemize}
\item {Grp. gram.:adj.}
\end{itemize}
\begin{itemize}
\item {Utilização:Zool.}
\end{itemize}
\begin{itemize}
\item {Proveniência:(Do lat. \textunderscore concha\textunderscore  + \textunderscore colere\textunderscore )}
\end{itemize}
Que vive em concha bivalve.
\section{Conquilha}
\begin{itemize}
\item {Grp. gram.:f.}
\end{itemize}
\begin{itemize}
\item {Proveniência:(Do lat. \textunderscore conchylia\textunderscore )}
\end{itemize}
Marisco saboroso, que se encontra na costa do Algarve.
\section{Conquilióforo}
\begin{itemize}
\item {Grp. gram.:adj.}
\end{itemize}
\begin{itemize}
\item {Proveniência:(Do gr. \textunderscore konkhulion\textunderscore  + \textunderscore phoros\textunderscore )}
\end{itemize}
Que tem concha.
\section{Conquiliologia}
\begin{itemize}
\item {Grp. gram.:f.}
\end{itemize}
\begin{itemize}
\item {Proveniência:(Do gr. \textunderscore konkhulion\textunderscore  + \textunderscore logos\textunderscore )}
\end{itemize}
Tratado das conchas.
\section{Conquiliológico}
\begin{itemize}
\item {Grp. gram.:adj.}
\end{itemize}
Relativo á conquiliologia.
\section{Conquiliologista}
\begin{itemize}
\item {Grp. gram.:m.}
\end{itemize}
Aquelle que é versado em conquiliologia.
\section{Conquista}
\begin{itemize}
\item {Grp. gram.:f.}
\end{itemize}
\begin{itemize}
\item {Utilização:Fam.}
\end{itemize}
Acto de conquistar.
Coisa conquistada.
Namôro.
\section{Conquistador}
\begin{itemize}
\item {Grp. gram.:m.}
\end{itemize}
Aquelle que conquista.
\section{Conquistar}
\begin{itemize}
\item {Grp. gram.:v. t.}
\end{itemize}
\begin{itemize}
\item {Utilização:Fam.}
\end{itemize}
\begin{itemize}
\item {Proveniência:(Do lat. \textunderscore conquisitus\textunderscore )}
\end{itemize}
Subjugar com armas.
Adquirir pela fôrça.
Conseguir, alcançar.
Conseguir amor, amizade de.
\section{Conquistável}
\begin{itemize}
\item {Grp. gram.:adj.}
\end{itemize}
Que se póde conquistar.
\section{Conquite}
\begin{itemize}
\item {Grp. gram.:f.}
\end{itemize}
\begin{itemize}
\item {Utilização:Des.}
\end{itemize}
\begin{itemize}
\item {Proveniência:(Do lat. \textunderscore concha\textunderscore )}
\end{itemize}
Petrificação, semelhante a uma concha.
Concha fóssil.
\section{Conresponder}
\textunderscore v. i.\textunderscore  (e der.) \textunderscore Pop.\textunderscore 
O mesmo que \textunderscore corresponder\textunderscore , etc.
\section{Conróbia}
\begin{itemize}
\item {Grp. gram.:f.}
\end{itemize}
\begin{itemize}
\item {Utilização:Prov.}
\end{itemize}
\begin{itemize}
\item {Utilização:chul.}
\end{itemize}
Súcia, malta; reunião de pessôas de fama duvidosa. (Colhido em Tondela)
\section{Conromper}
\textunderscore v. t.\textunderscore  (e der.) \textunderscore Pop.\textunderscore  e \textunderscore ant.\textunderscore 
O mesmo que \textunderscore corromper\textunderscore , etc.
\section{Consabedor}
\begin{itemize}
\item {Grp. gram.:m.}
\end{itemize}
\begin{itemize}
\item {Proveniência:(De \textunderscore con...\textunderscore  + \textunderscore sabedor\textunderscore )}
\end{itemize}
Aquelle que sabe, juntamente com outrem.
\section{Consagração}
\begin{itemize}
\item {Grp. gram.:f.}
\end{itemize}
\begin{itemize}
\item {Proveniência:(Do lat. \textunderscore consecratio\textunderscore )}
\end{itemize}
Acto de consagrar.
\section{Consagrador}
\begin{itemize}
\item {Grp. gram.:m.  e  adj.}
\end{itemize}
O mesmo que \textunderscore consagrante\textunderscore .
\section{Consagramento}
\begin{itemize}
\item {Grp. gram.:m.}
\end{itemize}
\begin{itemize}
\item {Utilização:Ant.}
\end{itemize}
\begin{itemize}
\item {Proveniência:(De \textunderscore consagrar\textunderscore )}
\end{itemize}
Juramento, que alguém fazia pela hóstia consagrada, de morrer quando outrem.
\section{Consagrante}
\begin{itemize}
\item {Grp. gram.:m.  e  adj.}
\end{itemize}
\begin{itemize}
\item {Proveniência:(Lat. \textunderscore consecrans\textunderscore )}
\end{itemize}
O que consagra.
\section{Consagrar}
\begin{itemize}
\item {Grp. gram.:v. t.}
\end{itemize}
\begin{itemize}
\item {Grp. gram.:V. i.}
\end{itemize}
\begin{itemize}
\item {Utilização:Ant.}
\end{itemize}
\begin{itemize}
\item {Proveniência:(Do lat. \textunderscore consecrare\textunderscore )}
\end{itemize}
Tornar sagrado, sagrar.
Dedicar a Deus.
Converter (pão e vinho) no corpo e sangue de Christo.
Offerecer em homenagem.
Tributar: \textunderscore consagrar admiração a alguém\textunderscore .
Destinar.
Votar.
Autorizar, sanccionar.
Jurar pela sagrada hóstia.
\section{Consanguíneo}
\begin{itemize}
\item {fónica:gu-i}
\end{itemize}
\begin{itemize}
\item {Grp. gram.:adj.}
\end{itemize}
\begin{itemize}
\item {Grp. gram.:M.}
\end{itemize}
\begin{itemize}
\item {Proveniência:(Lat. \textunderscore consanguineus\textunderscore )}
\end{itemize}
Que é do mesmo sangue.
Que é filho do mesmo pai.
Parente por consanguïnidade.
\section{Consanguinidade}
\begin{itemize}
\item {fónica:gu-i}
\end{itemize}
\begin{itemize}
\item {Grp. gram.:f.}
\end{itemize}
\begin{itemize}
\item {Proveniência:(Lat. \textunderscore consanguinitas\textunderscore )}
\end{itemize}
Relação, parentesco entre os que procedem do mesmo pai ou da mesma raça.
\section{Consciência}
\begin{itemize}
\item {Grp. gram.:f.}
\end{itemize}
\begin{itemize}
\item {Utilização:Prov.}
\end{itemize}
\begin{itemize}
\item {Proveniência:(Lat. \textunderscore conscientia\textunderscore )}
\end{itemize}
Sentimento ou percepção do que se passa em nós.
Voz secreta da alma, approvando ou reprovando os nossos actos.
Sinceridade: \textunderscore falar com consciência\textunderscore .
Cuidado extremo, com que se executa um trabalho.
Opinião.
Honradez: \textunderscore homem de consciência\textunderscore .
Facto, que produz remorsos; injustiça, sem razão: \textunderscore isso é uma consciência\textunderscore .
Pedra, suspensa por um cordel, para retesar a teia nos teares manuaes. (Colhido em Turquel)
\section{Consciente}
\begin{itemize}
\item {Grp. gram.:adj.}
\end{itemize}
\begin{itemize}
\item {Proveniência:(Lat. \textunderscore consciens\textunderscore )}
\end{itemize}
Que sabe que existe, que sabe o que faz.
Que procede da consciência: \textunderscore affirmação consciente\textunderscore .
Que é feito com consciência.
\section{Conscienciosamente}
\begin{itemize}
\item {Grp. gram.:adv.}
\end{itemize}
De modo consciencioso.
\section{Consciencioso}
\begin{itemize}
\item {Grp. gram.:adj.}
\end{itemize}
Que tem consciência.
Em que há consciência, escrúpulo, cuidado: \textunderscore trabalho consciencioso\textunderscore .
\section{Cônscio}
\begin{itemize}
\item {Grp. gram.:adj.}
\end{itemize}
\begin{itemize}
\item {Proveniência:(Lat. \textunderscore conscius\textunderscore )}
\end{itemize}
Que conhece bem o que faz ou o que lhe cumpre fazer.
\section{Conscripto}
\begin{itemize}
\item {Grp. gram.:adj.}
\end{itemize}
Alistado, recrutado. Cf. Rui Barbosa, \textunderscore Cartas de Ingl.\textunderscore , 17.
\section{Conscriptos}
\begin{itemize}
\item {Grp. gram.:adj. pl.}
\end{itemize}
\begin{itemize}
\item {Proveniência:(Lat. \textunderscore conscriptus\textunderscore )}
\end{itemize}
Dizia-se primitivamente dos senadores adjuntos, na antiga Roma; e depois dizia-se de todos os senadores romanos.
\section{Conscrito}
\begin{itemize}
\item {Grp. gram.:adj.}
\end{itemize}
Alistado, recrutado. Cf. Rui Barbosa, \textunderscore Cartas de Ingl.\textunderscore , 17.
\section{Conscritos}
\begin{itemize}
\item {Grp. gram.:adj. pl.}
\end{itemize}
\begin{itemize}
\item {Proveniência:(Lat. \textunderscore conscriptus\textunderscore )}
\end{itemize}
Dizia-se primitivamente dos senadores adjuntos, na antiga Roma; e depois dizia-se de todos os senadores romanos.
\section{Consecrante}
\begin{itemize}
\item {Grp. gram.:m.  e  adj.}
\end{itemize}
(V.consagrante)
\section{Consecratório}
\begin{itemize}
\item {Grp. gram.:adj.}
\end{itemize}
\begin{itemize}
\item {Proveniência:(Do lat. \textunderscore consecrare\textunderscore )}
\end{itemize}
Relativo á consagração.
\section{Consectário}
\begin{itemize}
\item {Grp. gram.:m.}
\end{itemize}
O mesmo que \textunderscore consequência\textunderscore . Cf. Latino, \textunderscore His. Pol.\textunderscore , 115.
\section{Consecução}
\begin{itemize}
\item {Grp. gram.:f.}
\end{itemize}
\begin{itemize}
\item {Proveniência:(Lat. \textunderscore consecutio\textunderscore )}
\end{itemize}
Acto ou effeito de conseguir.
\section{Consecutivamente}
\begin{itemize}
\item {Grp. gram.:adv.}
\end{itemize}
De modo consecutivo.
\section{Consecutivo}
\begin{itemize}
\item {Grp. gram.:adj.}
\end{itemize}
\begin{itemize}
\item {Proveniência:(Do lat. \textunderscore consecutus\textunderscore )}
\end{itemize}
Immediato; que segue outro.
\section{Consego}
\begin{itemize}
\item {fónica:sê}
\end{itemize}
\begin{itemize}
\item {Grp. gram.:loc. pron.}
\end{itemize}
\begin{itemize}
\item {Utilização:Ant.}
\end{itemize}
O mesmo que \textunderscore consigo\textunderscore .
\section{Consegrar}
\begin{itemize}
\item {Grp. gram.:v. t.}
\end{itemize}
\begin{itemize}
\item {Utilização:Ant.}
\end{itemize}
\begin{itemize}
\item {Proveniência:(Lat. \textunderscore consecrare\textunderscore )}
\end{itemize}
O mesmo que \textunderscore consagrar\textunderscore .
\section{Conseguidoiro}
\begin{itemize}
\item {Grp. gram.:adj.}
\end{itemize}
\begin{itemize}
\item {Utilização:Des.}
\end{itemize}
Que se póde conseguir.
\section{Conseguidor}
\begin{itemize}
\item {Grp. gram.:m.}
\end{itemize}
Aquelle que consegue.
\section{Conseguidouro}
\begin{itemize}
\item {Grp. gram.:adj.}
\end{itemize}
\begin{itemize}
\item {Utilização:Des.}
\end{itemize}
Que se póde conseguir.
\section{Conseguimento}
\begin{itemize}
\item {Grp. gram.:m.}
\end{itemize}
\begin{itemize}
\item {Proveniência:(De \textunderscore conseguir\textunderscore )}
\end{itemize}
O mesmo que \textunderscore consecução\textunderscore .
\section{Conseguinte}
\begin{itemize}
\item {Grp. gram.:adj.}
\end{itemize}
\begin{itemize}
\item {Grp. gram.:Loc. adv.}
\end{itemize}
\begin{itemize}
\item {Grp. gram.:M.}
\end{itemize}
\begin{itemize}
\item {Proveniência:(Lat. \textunderscore consequens\textunderscore )}
\end{itemize}
O mesmo que \textunderscore consecutivo\textunderscore .
\textunderscore Por conseguinte\textunderscore , por consequência, portanto.
O mesmo que \textunderscore consequência\textunderscore . Cf. Filinto, XIII, 85; XVII, 208; XXI, 48.
\section{Conseguintemente}
\begin{itemize}
\item {Grp. gram.:adv.}
\end{itemize}
\begin{itemize}
\item {Proveniência:(De \textunderscore conseguinte\textunderscore )}
\end{itemize}
Por consequência; por isso; portanto; logo.
\section{Conseguir}
\begin{itemize}
\item {Grp. gram.:v. t.}
\end{itemize}
\begin{itemize}
\item {Proveniência:(Do lat. \textunderscore consequi\textunderscore )}
\end{itemize}
Obter.
Entrar na posse de.
Têr como consequência.
\section{Conseguível}
\begin{itemize}
\item {Grp. gram.:adj.}
\end{itemize}
Que se póde conseguir.
\section{Conselha}
\begin{itemize}
\item {fónica:sê}
\end{itemize}
\begin{itemize}
\item {Grp. gram.:f.}
\end{itemize}
\begin{itemize}
\item {Utilização:Ant.}
\end{itemize}
O mesmo que \textunderscore conselho\textunderscore :«\textunderscore o lobo, &amp; a golpelha todos são de uma conselha\textunderscore ». \textunderscore Eufrosina\textunderscore , 84.
\section{Conselhar}
\begin{itemize}
\item {Grp. gram.:v. t.}
\end{itemize}
\begin{itemize}
\item {Proveniência:(Do lat. \textunderscore consiliare\textunderscore )}
\end{itemize}
O mesmo que \textunderscore aconselhar\textunderscore . Cf. \textunderscore Eufrosina\textunderscore , 70.
\section{Conselheiral}
\begin{itemize}
\item {Grp. gram.:adj.}
\end{itemize}
\begin{itemize}
\item {Utilização:Fam.}
\end{itemize}
O mesmo que \textunderscore conselheirático\textunderscore .
\section{Conselheiramente}
\begin{itemize}
\item {Grp. gram.:adv.}
\end{itemize}
\begin{itemize}
\item {Utilização:Fam.}
\end{itemize}
Com gravidade; á maneira de Conselheiro.
\section{Conselheirático}
\begin{itemize}
\item {Grp. gram.:adj.}
\end{itemize}
\begin{itemize}
\item {Utilização:Fam.}
\end{itemize}
Próprio de Conselheiro.
Que tem modos graves, modos de Conselheiro.
\section{Conselheirismo}
\begin{itemize}
\item {Grp. gram.:m.}
\end{itemize}
\begin{itemize}
\item {Utilização:Fam.}
\end{itemize}
Aprumo ou gravidade, própria de conselheiro.
\section{Conselheiro}
\begin{itemize}
\item {Grp. gram.:adj.}
\end{itemize}
\begin{itemize}
\item {Grp. gram.:M.}
\end{itemize}
\begin{itemize}
\item {Proveniência:(Lat. \textunderscore consiliarius\textunderscore )}
\end{itemize}
Que aconselha.
Aquelle que aconselha.
Membro de um conselho ou de certos tribunaes.
Aquelle que foi agraciado com a Carta de Conselho.
\section{Conselho}
\begin{itemize}
\item {fónica:sê}
\end{itemize}
\begin{itemize}
\item {Grp. gram.:m.}
\end{itemize}
\begin{itemize}
\item {Proveniência:(Do lat. \textunderscore consilium\textunderscore )}
\end{itemize}
Parecer, opinião que se emitte, para que possa sêr adoptada: \textunderscore não seguiste o meu conselho\textunderscore .
Admoestação, aviso.
Prudência.
Corporação, a que impende dar parecer ou conselho sôbre certos negócios públicos: \textunderscore Conselho Superior de Instrucção Pública\textunderscore .
Designação de certos tribunaes e de outras corporações superiores.
Reunião ou assembleia de ministros.
Reunião de pessôas, que intervêm na resolução de negócios de outrem, públicos ou particulares: \textunderscore conselho de família\textunderscore .
\textunderscore Carta de Conselho\textunderscore , diploma honorífico, que dava, a quem o recebia, o título de Conselheiro do Rei ou Rainha.
\section{Consenciente}
\begin{itemize}
\item {Grp. gram.:adj.}
\end{itemize}
\begin{itemize}
\item {Proveniência:(Lat. \textunderscore consentiens\textunderscore )}
\end{itemize}
Que consente.
\section{Consenhor}
\begin{itemize}
\item {Grp. gram.:m.}
\end{itemize}
\begin{itemize}
\item {Proveniência:(De \textunderscore con...\textunderscore  + \textunderscore senhor\textunderscore )}
\end{itemize}
Que é senhor, juntamente com outro.
\section{Consensial}
\begin{itemize}
\item {Grp. gram.:adj.}
\end{itemize}
Relativo a consenso.
\section{Consenso}
\begin{itemize}
\item {Grp. gram.:m.}
\end{itemize}
\begin{itemize}
\item {Proveniência:(Lat. \textunderscore consensus\textunderscore )}
\end{itemize}
O mesmo que \textunderscore consentimento\textunderscore .
\section{Consensual}
\begin{itemize}
\item {Grp. gram.:adj.}
\end{itemize}
Relativo a consenso; que depende de consenso. Cf. Latino, \textunderscore Elogios\textunderscore , 123.
\section{Consensualidade}
\begin{itemize}
\item {Grp. gram.:f.}
\end{itemize}
Qualidade de consensual.
\section{Consentaneamente}
\begin{itemize}
\item {Grp. gram.:adv.}
\end{itemize}
De modo consentâneo.
\section{Consentâneo}
\begin{itemize}
\item {Grp. gram.:adj.}
\end{itemize}
\begin{itemize}
\item {Proveniência:(Lat. \textunderscore consentaneus\textunderscore )}
\end{itemize}
Apropriado; congruente; adequado.
\section{Consentidor}
\begin{itemize}
\item {Grp. gram.:m.  e  adj.}
\end{itemize}
O que consente.
\section{Consentimento}
\begin{itemize}
\item {Grp. gram.:m.}
\end{itemize}
Acôrdo.
Annuência.
Tolerância.
Acto de consentir.
\section{Consentir}
\begin{itemize}
\item {Grp. gram.:v. t.}
\end{itemize}
\begin{itemize}
\item {Grp. gram.:V. i.}
\end{itemize}
\begin{itemize}
\item {Proveniência:(Lat. \textunderscore consentire\textunderscore )}
\end{itemize}
Permittir; tolerar.
Approvar.
Dar consentimento.
Annuir.
\section{Consequência}
\begin{itemize}
\item {Grp. gram.:f.}
\end{itemize}
\begin{itemize}
\item {Proveniência:(Lat. \textunderscore consequentia\textunderscore )}
\end{itemize}
Resultado.
Deducção, conclusão, illação.
Importância.
\section{Consequencial}
\begin{itemize}
\item {Grp. gram.:adj.}
\end{itemize}
Relativo a consequência.
\section{Consequente}
\begin{itemize}
\item {Grp. gram.:adj.}
\end{itemize}
\begin{itemize}
\item {Grp. gram.:M.}
\end{itemize}
\begin{itemize}
\item {Proveniência:(Lat. \textunderscore consequens\textunderscore )}
\end{itemize}
Que segue naturalmente.
Que se deduz, que se infere.
Que procede coherentemente.
Que raciocina bem.
Segunda proposição do enthymema.
Segundo termo do uma razão, em Mathemática.
\section{Consequentemente}
\begin{itemize}
\item {Grp. gram.:adv.}
\end{itemize}
De modo consequente.
Por conseguinte; portanto.
\section{Consertador}
\begin{itemize}
\item {Grp. gram.:m.}
\end{itemize}
\begin{itemize}
\item {Utilização:T. de Buarcos}
\end{itemize}
Aquelle que conserta.
Aquelle que conserta e encasca as redes.
\section{Consertamento}
\begin{itemize}
\item {Grp. gram.:m.}
\end{itemize}
O mesmo que \textunderscore consêrto\textunderscore .
\section{Consertar}
\begin{itemize}
\item {Grp. gram.:v. t.}
\end{itemize}
\begin{itemize}
\item {Proveniência:(De \textunderscore consêrto\textunderscore )}
\end{itemize}
Coser, reparando; remendar com costura; accrescentar, cosendo: \textunderscore consertar umas calças\textunderscore .
Reparar, arranjar: \textunderscore consertar um relógio; consertar uma parede\textunderscore .
\section{Consêrto}
\begin{itemize}
\item {Grp. gram.:m.}
\end{itemize}
\begin{itemize}
\item {Proveniência:(Lat. \textunderscore consertum\textunderscore , de \textunderscore conserere\textunderscore , coser, ligar)}
\end{itemize}
Acto ou effeito de consertar.
Remendo.
Reparação, arranjo.
\section{Conserva}
\begin{itemize}
\item {Grp. gram.:f.}
\end{itemize}
\begin{itemize}
\item {Proveniência:(De \textunderscore conservar\textunderscore )}
\end{itemize}
Líquido ou calda, em que se conservam substâncias alimentícias.
Substância, conservada nessa calda.
Preparação pharmacêutica com plantas e açúcar.
\textunderscore Navio de conserva\textunderscore , navio, que acompanha outro, para o socorrer, sendo preciso.
Variedade de azeitona, o mesmo que \textunderscore longal\textunderscore .
\section{Conservação}
\begin{itemize}
\item {Grp. gram.:f.}
\end{itemize}
\begin{itemize}
\item {Proveniência:(Lat. \textunderscore conservatio\textunderscore )}
\end{itemize}
Acto de conservar.
\section{Conservador}
\begin{itemize}
\item {Grp. gram.:adj.}
\end{itemize}
\begin{itemize}
\item {Grp. gram.:M.}
\end{itemize}
\begin{itemize}
\item {Proveniência:(Lat. \textunderscore conservator\textunderscore )}
\end{itemize}
Que conserva.
Aquelle que conserva.
Funccionário público, encarregado do registo predial ou do registo civil.
Aquelle que é encarregado da conservação de um archivo: \textunderscore conservador da Bibliotheca Nacional\textunderscore .
Indivíduo, que em política opina pela conservação do estado actual, oppondo-se a reformas essenciaes.
\section{Conservante}
\begin{itemize}
\item {Grp. gram.:adj.}
\end{itemize}
\begin{itemize}
\item {Proveniência:(Lat. \textunderscore conservans\textunderscore )}
\end{itemize}
Que conserva.
\section{Conservantismo}
\begin{itemize}
\item {Grp. gram.:m.}
\end{itemize}
\begin{itemize}
\item {Proveniência:(De \textunderscore conservante\textunderscore )}
\end{itemize}
Systema ou doutrina dos que, avessos a reformas, pugnam pela conservação das circunstâncias presentes.
\section{Conservar}
\begin{itemize}
\item {Grp. gram.:v. t.}
\end{itemize}
\begin{itemize}
\item {Proveniência:(Lat. \textunderscore conservare\textunderscore )}
\end{itemize}
Manter no estado actual.
Manter no seu lugar: \textunderscore o Govêrno conserva os antigos Governadores Civis\textunderscore .
Guardar cuidadosamente: \textunderscore conservar frutas para o inverno\textunderscore .
Preservar.
Continuar a têr: \textunderscore conservar antigos vícios\textunderscore .
Lembrar-se de.
Amparar, salvar.
Fazer durar.
\section{Conservaria}
\begin{itemize}
\item {Grp. gram.:f.}
\end{itemize}
Lugar, onde se fabrícam conservas.
Estabelecimento, onde ellas se vendem.
\section{Conservativo}
\begin{itemize}
\item {Grp. gram.:adj.}
\end{itemize}
\begin{itemize}
\item {Proveniência:(Lat. \textunderscore conservativus\textunderscore )}
\end{itemize}
Próprio para conservar alguma coisa.
\section{Conservatória}
\begin{itemize}
\item {Grp. gram.:f.}
\end{itemize}
\begin{itemize}
\item {Proveniência:(De \textunderscore conservatório\textunderscore )}
\end{itemize}
Repartição, em que se registam direitos prediaes, ou em que, nas capitaes de districto, se registam os nascimentos, casamentos e óbitos.
\section{Conservatório}
\begin{itemize}
\item {Grp. gram.:adj.}
\end{itemize}
\begin{itemize}
\item {Grp. gram.:M.}
\end{itemize}
\begin{itemize}
\item {Proveniência:(Lat. \textunderscore conservatorius\textunderscore )}
\end{itemize}
Que serve para conservar alguma coisa: \textunderscore providências conservatórias\textunderscore .
Estabelecimento público, destinado principalmente ao ensino de bellas-artes.
Estabelecimento, em que se recolhem donzellas pobres ou órfãs, para as preservar dos vícios sociaes.
\section{Conservável}
\begin{itemize}
\item {Grp. gram.:adj.}
\end{itemize}
\begin{itemize}
\item {Proveniência:(Lat. \textunderscore conservabilis\textunderscore )}
\end{itemize}
Que se póde conservar.
\section{Conserveiro}
\begin{itemize}
\item {Grp. gram.:m.}
\end{itemize}
Aquelle que faz ou vende conservas.
\section{Conservidor}
\begin{itemize}
\item {Grp. gram.:m.}
\end{itemize}
\begin{itemize}
\item {Proveniência:(De \textunderscore con...\textunderscore  + \textunderscore servidor\textunderscore )}
\end{itemize}
Aquelle que é servidor, juntamente com outrem.
\section{Conservo}
\begin{itemize}
\item {Grp. gram.:m.}
\end{itemize}
\begin{itemize}
\item {Utilização:Des.}
\end{itemize}
Aquelle que é servo, juntamente com outrem. Cf. Sousa, \textunderscore Vida do Arceb.\textunderscore , I, 54.
\section{Consesso}
\begin{itemize}
\item {Grp. gram.:m.}
\end{itemize}
\begin{itemize}
\item {Utilização:Des.}
\end{itemize}
\begin{itemize}
\item {Proveniência:(Lat. \textunderscore consessus\textunderscore )}
\end{itemize}
Assembleia; congresso; concílio.
\section{Consideração}
\begin{itemize}
\item {Grp. gram.:f.}
\end{itemize}
\begin{itemize}
\item {Proveniência:(Lat. \textunderscore consideratio\textunderscore )}
\end{itemize}
Acto de considerar.
Respeito, estima, que se dedica a alguém.
Valimento.
Attenção.
Reflexão.
Raciocinio.
\section{Consideradamente}
\begin{itemize}
\item {Grp. gram.:adv.}
\end{itemize}
Com consideração.
\section{Considerando}
\begin{itemize}
\item {Grp. gram.:m.}
\end{itemize}
\begin{itemize}
\item {Proveniência:(Lat. \textunderscore considerandus\textunderscore )}
\end{itemize}
Cada uma das considerações ou fundamentos, cuja exposição ordenada precede certos documentos: \textunderscore os considerandos de uma sentença\textunderscore .
Motivo, razão.
\section{Considerar}
\begin{itemize}
\item {Grp. gram.:v. t.}
\end{itemize}
\begin{itemize}
\item {Proveniência:(Lat. \textunderscore considerare\textunderscore )}
\end{itemize}
Examinar.
Dar attenção a.
Calcular.
Reputar; apreciar.
Têr em bôa conta.
\section{Considerável}
\begin{itemize}
\item {Grp. gram.:adj.}
\end{itemize}
Notável; que se deve considerar.
Importante: \textunderscore riqueza considerável\textunderscore .
\section{Consideravelmente}
\begin{itemize}
\item {Grp. gram.:adv.}
\end{itemize}
De modo considerável.
\section{Consignação}
\begin{itemize}
\item {Grp. gram.:f.}
\end{itemize}
\begin{itemize}
\item {Proveniência:(Lat. \textunderscore consignatio\textunderscore )}
\end{itemize}
Acto ou effeito de consignar.
\section{Consignador}
\begin{itemize}
\item {Grp. gram.:m.}
\end{itemize}
Aquelle que consigna.
\section{Consignante}
\begin{itemize}
\item {Grp. gram.:adj.}
\end{itemize}
\begin{itemize}
\item {Proveniência:(Lat. \textunderscore consignans\textunderscore )}
\end{itemize}
Que consigna.
\section{Consignar}
\begin{itemize}
\item {Grp. gram.:v. t.}
\end{itemize}
\begin{itemize}
\item {Proveniência:(Lat. \textunderscore consignare\textunderscore )}
\end{itemize}
Notar, assignalar.
Advertir.
Affirmar.
Depositar (valores, cuja propriedade depende de liquidação ou que têm de sêr applicados a determinadas despesas).
Dirigir ou entregar (navios ou mercadorias) a um correspondente ou commissário.
Destinar (rendimentos) para pagamento de crèdores, dando a êstes o usufruto de bens, cujo rendimento há de pagar os respectivos créditos.
Entregar em depósito ou em commissão.
\section{Consignatário}
\begin{itemize}
\item {Grp. gram.:m.}
\end{itemize}
\begin{itemize}
\item {Proveniência:(De \textunderscore consignar\textunderscore )}
\end{itemize}
Negociante ou commissário, a quem se dirigem navios ou mercadorias.
Crèdor, em favor de quem se consignam rendimentos.
Depositário de valores litigiosos ou destinados a determinadas despesas.
\section{Consignativo}
\begin{itemize}
\item {Grp. gram.:adj.}
\end{itemize}
\begin{itemize}
\item {Proveniência:(De \textunderscore consignar\textunderscore )}
\end{itemize}
Diz-se do censo ou quantia, que se entrega por uma vez a quem se obriga a pagar annualmente determinada pensão.
\section{Consignável}
\begin{itemize}
\item {Grp. gram.:adj.}
\end{itemize}
Que se póde consignar.
\section{Consigo}
\begin{itemize}
\item {Grp. gram.:loc. pron.}
\end{itemize}
Em companhia de pessôa ou pessôas, de quem se fala: \textunderscore levou-me consigo\textunderscore .
De si para si: \textunderscore falava consigo\textunderscore .
(Flexão do pron. \textunderscore si\textunderscore , precedido da prep. \textunderscore com\textunderscore )
\section{Consiliário}
\begin{itemize}
\item {Grp. gram.:m.}
\end{itemize}
\begin{itemize}
\item {Proveniência:(Do lat. \textunderscore consilium\textunderscore )}
\end{itemize}
O mesmo que \textunderscore conselheiro\textunderscore . Cf. Camillo, \textunderscore Coisas Leves e Pesadas\textunderscore , 39.
\section{Consinha}
\begin{itemize}
\item {Grp. gram.:f.}
\end{itemize}
(?):«\textunderscore ...quem amores tomasse de estudante, que são mais engraxados, que consinha\textunderscore ». \textunderscore Eufrosina\textunderscore , 249.
\section{Consirar}
\begin{itemize}
\item {Grp. gram.:v. i.}
\end{itemize}
(Fórma antiga de \textunderscore considerar\textunderscore . Cf. Azurara, \textunderscore Chrón. de D. João I\textunderscore , XLIX; Rui de Pina, \textunderscore D. Duarte\textunderscore , etc.)
\section{Consistência}
\begin{itemize}
\item {Grp. gram.:f.}
\end{itemize}
Estado daquillo que é consistente.
\section{Consistente}
\begin{itemize}
\item {Grp. gram.:adj.}
\end{itemize}
\begin{itemize}
\item {Proveniência:(Lat. \textunderscore consistens\textunderscore )}
\end{itemize}
Que consiste.
Que subsiste.
Sólido, espêsso, duro.
Estável, duradoiro.
Constante.
\section{Consistir}
\begin{itemize}
\item {Grp. gram.:v. i.}
\end{itemize}
\begin{itemize}
\item {Proveniência:(Lat. \textunderscore consistere\textunderscore )}
\end{itemize}
Subsistir.
Resumir-se.
Sêr constituido.
Basear-se.
Sêr formado, constar: \textunderscore os seus haveres consistem em prédios\textunderscore .
\section{Consistorial}
\begin{itemize}
\item {Grp. gram.:adj.}
\end{itemize}
Relativo a consistório.
\section{Consistório}
\begin{itemize}
\item {Grp. gram.:m.}
\end{itemize}
\begin{itemize}
\item {Proveniência:(Lat. \textunderscore consistorium\textunderscore )}
\end{itemize}
Assembleia de Cardeaes presidida pelo Papa.
Lugar, em que se celebra essa assembleia.
Qualquer assembleia, em que se tratam assumptos graves.
\section{Consoada}
\begin{itemize}
\item {Grp. gram.:f.}
\end{itemize}
\begin{itemize}
\item {Proveniência:(De \textunderscore consoar\textunderscore ^2)}
\end{itemize}
Refeição ligeira, que se toma á noite, nos dias de jejum.
Presente, que se dá pelo Natal.
Banquete familiar em a noite do Natal.
\section{Consoante}
\begin{itemize}
\item {Grp. gram.:adj.}
\end{itemize}
\begin{itemize}
\item {Grp. gram.:F.}
\end{itemize}
\begin{itemize}
\item {Grp. gram.:M.}
\end{itemize}
\begin{itemize}
\item {Utilização:T. de Coimbra}
\end{itemize}
\begin{itemize}
\item {Grp. gram.:Prep.}
\end{itemize}
\begin{itemize}
\item {Proveniência:(Lat. \textunderscore consonans\textunderscore )}
\end{itemize}
Que tem consonância.
Que não tem som próprio e que nas palavras representa os ruídos articulados, (falando-se de letras).
Letra consoante.
Palavra, que rima com outra.
O mesmo que \textunderscore namorado\textunderscore .
Conforme: \textunderscore vive-se consoante os tempos\textunderscore .
\section{Consoanteiro}
\begin{itemize}
\item {Grp. gram.:adj.}
\end{itemize}
\begin{itemize}
\item {Utilização:Deprec.}
\end{itemize}
Relativo a consoantes ou rimas:«\textunderscore desenxabidas prosas consoanteiras\textunderscore ». Filinto, VIII, 35.
\section{Consoantemente}
\begin{itemize}
\item {Grp. gram.:adv.}
\end{itemize}
De modo consoante.
\section{Consoar}
\begin{itemize}
\item {Grp. gram.:v. i.}
\end{itemize}
\begin{itemize}
\item {Proveniência:(Do lat. \textunderscore cum\textunderscore  + \textunderscore sub\textunderscore  + \textunderscore unare\textunderscore , seg. Car. Michaëlis)}
\end{itemize}
Soar juntamente.
Sêr consoante.
Rimar.
\section{Consoar}
\begin{itemize}
\item {Grp. gram.:v. i.}
\end{itemize}
\begin{itemize}
\item {Grp. gram.:V. t.}
\end{itemize}
\begin{itemize}
\item {Proveniência:(Do lat. \textunderscore consolari\textunderscore )}
\end{itemize}
Tomar a consoada.
Tomar como consoada, ou em consoada.
\section{Consociação}
\begin{itemize}
\item {Grp. gram.:f.}
\end{itemize}
Acto ou effeito de consociar.
\section{Consociar}
\begin{itemize}
\item {Grp. gram.:v. t.}
\end{itemize}
\begin{itemize}
\item {Proveniência:(Lat. \textunderscore consociare\textunderscore )}
\end{itemize}
Tornar sócio; associar.
Conciliar, unir.
\section{Consócio}
\begin{itemize}
\item {Grp. gram.:m.  e  adj.}
\end{itemize}
\begin{itemize}
\item {Proveniência:(Lat. \textunderscore consocius\textunderscore )}
\end{itemize}
O que é sócio, em relação a outro.
\section{Consola}
\begin{itemize}
\item {Grp. gram.:f.}
\end{itemize}
\begin{itemize}
\item {Proveniência:(Fr. \textunderscore console\textunderscore )}
\end{itemize}
Peça do architectura, saliente, para sustentar estátuas, vasos, etc.
Móvel de sala, espécie de mesa, em que se collocam jarras ou pequenos objectos de curiosidade e ornato.
A parte superior da harpa, de fórma recurvada, e também chamada \textunderscore modilhão\textunderscore .
\section{Consolação}
\begin{itemize}
\item {Grp. gram.:f.}
\end{itemize}
\begin{itemize}
\item {Utilização:Ant.}
\end{itemize}
\begin{itemize}
\item {Proveniência:(Lat. \textunderscore consolatio\textunderscore )}
\end{itemize}
Acto ou effeito de consolar.
Coisa ou pessôa, que consola: \textunderscore os filhos são a sua consolação\textunderscore .
O mesmo que \textunderscore consoada\textunderscore .
\section{Consoladamente}
\begin{itemize}
\item {Grp. gram.:adj.}
\end{itemize}
Com consolação.
\section{Consoladeza}
\begin{itemize}
\item {Grp. gram.:f.}
\end{itemize}
O mesmo que \textunderscore consolação\textunderscore . Cf. Filinto, III, 40.
\section{Consolador}
\begin{itemize}
\item {Grp. gram.:m.  e  adj.}
\end{itemize}
\begin{itemize}
\item {Proveniência:(Lat. \textunderscore consolator\textunderscore )}
\end{itemize}
O que consola.
\section{Consolando}
\begin{itemize}
\item {Grp. gram.:adj.}
\end{itemize}
Que se deve consolar; que se há de consolar. Cf. Filinto, XVI, 275.
\section{Consolar}
\begin{itemize}
\item {Grp. gram.:v. t.}
\end{itemize}
\begin{itemize}
\item {Proveniência:(Lat. \textunderscore consolari\textunderscore )}
\end{itemize}
Aliviar a pena, o soffrer de: \textunderscore consolar desgraçados\textunderscore .
Confortar.
Dar prazer a: \textunderscore consolam-no os bons jantares\textunderscore .
\section{Consolativo}
\begin{itemize}
\item {Grp. gram.:adj.}
\end{itemize}
\begin{itemize}
\item {Proveniência:(De \textunderscore consolar\textunderscore )}
\end{itemize}
Próprio para consolar; consolador:«\textunderscore não achava ali nada consolativo\textunderscore ». Camillo, \textunderscore Freira no Subter.\textunderscore , 96.
\section{Consolatório}
\begin{itemize}
\item {Grp. gram.:adj.}
\end{itemize}
\begin{itemize}
\item {Proveniência:(Lat. \textunderscore consolatorius\textunderscore )}
\end{itemize}
Que serve para consolar.
\section{Consolável}
\begin{itemize}
\item {Grp. gram.:adj.}
\end{itemize}
\begin{itemize}
\item {Proveniência:(Lat. \textunderscore consolabilis\textunderscore )}
\end{itemize}
Que se póde consolar.
\section{Consolda}
\begin{itemize}
\item {Grp. gram.:f.}
\end{itemize}
\begin{itemize}
\item {Proveniência:(Do lat. \textunderscore consolida\textunderscore )}
\end{itemize}
Solda.
Planta borragínea medicinal.
Búbula.
Planta ranunculácea, vulgarmente chamada \textunderscore espora\textunderscore .
\section{Consólida}
\begin{itemize}
\item {Grp. gram.:f.}
\end{itemize}
(V.consolda)
\section{Consolidação}
\begin{itemize}
\item {Grp. gram.:f.}
\end{itemize}
\begin{itemize}
\item {Proveniência:(Lat. \textunderscore consolidatio\textunderscore )}
\end{itemize}
Acto ou effeito de consolidar.
\section{Consolidante}
\begin{itemize}
\item {Grp. gram.:adj.}
\end{itemize}
Que consolida.
\section{Consolidar}
\begin{itemize}
\item {Grp. gram.:v. t.}
\end{itemize}
\begin{itemize}
\item {Proveniência:(Lat. \textunderscore consolidare\textunderscore )}
\end{itemize}
Tornar sólido, firme, estável: \textunderscore consolidar créditos\textunderscore .
Fazer adherir (os topos de um osso fracturado).
Tornar permanente (a divida pública).
\section{Consolidativo}
\begin{itemize}
\item {Grp. gram.:adj.}
\end{itemize}
Que póde consolidar.
\section{Consolista}
\begin{itemize}
\item {Grp. gram.:m.  e  f.}
\end{itemize}
\begin{itemize}
\item {Proveniência:(De \textunderscore consôlo\textunderscore )}
\end{itemize}
Pessôa que dá consolação. Cf. Filinto, XIII, 270.
\section{Consólo}
\begin{itemize}
\item {Grp. gram.:m.}
\end{itemize}
(V.consola)
\section{Consôlo}
\begin{itemize}
\item {Grp. gram.:m.}
\end{itemize}
O mesmo que \textunderscore consolação\textunderscore .
\section{Consonância}
\begin{itemize}
\item {Grp. gram.:f.}
\end{itemize}
\begin{itemize}
\item {Proveniência:(Lat. \textunderscore consonantia\textunderscore )}
\end{itemize}
Conjunto de sons agradáveis ao ouvido.
Harmonia.
Rima.
Acôrdo, conformidade.
\section{Consonantal}
\begin{itemize}
\item {Grp. gram.:adj.}
\end{itemize}
\begin{itemize}
\item {Proveniência:(De \textunderscore consonante\textunderscore )}
\end{itemize}
Relativo a letras consoantes.
\section{Consonante}
\begin{itemize}
\item {Grp. gram.:adj.}
\end{itemize}
\begin{itemize}
\item {Proveniência:(Lat. \textunderscore consonans\textunderscore )}
\end{itemize}
Que fórma ou tem consonância.
\section{Consonantemente}
\begin{itemize}
\item {Grp. gram.:adv.}
\end{itemize}
De modo consonante.
\section{Consonântico}
\begin{itemize}
\item {Grp. gram.:adj.}
\end{itemize}
O mesmo que \textunderscore consonantal\textunderscore .
\section{Consonantismo}
\begin{itemize}
\item {Grp. gram.:m.}
\end{itemize}
\begin{itemize}
\item {Utilização:Gram.}
\end{itemize}
\begin{itemize}
\item {Proveniência:(De \textunderscore consonante\textunderscore )}
\end{itemize}
Doutrina, relativa ás consoantes.
\section{Consonantização}
\begin{itemize}
\item {Grp. gram.:f.}
\end{itemize}
\begin{itemize}
\item {Utilização:Gram.}
\end{itemize}
Transformação de semi-vogal em consoante.
Acto ou effeito de consonantizar.
\section{Consonantizar}
\begin{itemize}
\item {Grp. gram.:v.}
\end{itemize}
\begin{itemize}
\item {Utilização:t. Gram.}
\end{itemize}
Transformar em consonante.
\section{Consonar}
\begin{itemize}
\item {Grp. gram.:v. i.}
\end{itemize}
(V. \textunderscore consoar\textunderscore ^1)
\section{Cônsono}
\begin{itemize}
\item {Grp. gram.:adj.}
\end{itemize}
\begin{itemize}
\item {Proveniência:(Lat. \textunderscore consonus\textunderscore )}
\end{itemize}
O mesmo que \textunderscore consonante\textunderscore .
\section{Consorciar}
\begin{itemize}
\item {Grp. gram.:v. t.}
\end{itemize}
\begin{itemize}
\item {Proveniência:(De \textunderscore consórcio\textunderscore )}
\end{itemize}
Associar.
Ligar por casamento.
\section{Consórcio}
\begin{itemize}
\item {Grp. gram.:m.}
\end{itemize}
\begin{itemize}
\item {Proveniência:(Lat. \textunderscore consortium\textunderscore )}
\end{itemize}
Associação.
Communhão de interesses.
Casamento.
\section{Consorte}
\begin{itemize}
\item {Grp. gram.:m.  e  f.}
\end{itemize}
\begin{itemize}
\item {Proveniência:(Lat. \textunderscore consors\textunderscore )}
\end{itemize}
Pessôa, que tem sorte ou estado commum.
Cônjuge.
Quem com outrem é participe de direitos ou coisas.
\section{Conspecto}
\begin{itemize}
\item {Grp. gram.:m.}
\end{itemize}
\begin{itemize}
\item {Proveniência:(Lat. \textunderscore conspectus\textunderscore )}
\end{itemize}
Acto de vêr, aspecto.
\section{Conspeito}
\begin{itemize}
\item {Grp. gram.:m.}
\end{itemize}
O mesmo que \textunderscore conspecto\textunderscore . Cf. D. Bernardes, \textunderscore Lima\textunderscore , 268.
\section{Conspicuidade}
\begin{itemize}
\item {Grp. gram.:f.}
\end{itemize}
Qualidade do que é conspícuo.
\section{Conspícuo}
\begin{itemize}
\item {Grp. gram.:adj.}
\end{itemize}
\begin{itemize}
\item {Proveniência:(Lat. \textunderscore conspicuus\textunderscore )}
\end{itemize}
Notável.
Illustre, distinto.
Respeitável, sério.
\section{Conspiração}
\begin{itemize}
\item {Grp. gram.:f.}
\end{itemize}
\begin{itemize}
\item {Proveniência:(Lat. \textunderscore conspiratio\textunderscore )}
\end{itemize}
Acto de conspirar.
\section{Conspirador}
\begin{itemize}
\item {Grp. gram.:m.  e  adj.}
\end{itemize}
O que conspira.
\section{Conspirante}
\begin{itemize}
\item {Grp. gram.:m.  e  adj.}
\end{itemize}
\begin{itemize}
\item {Proveniência:(Lat. \textunderscore conspirans\textunderscore )}
\end{itemize}
O mesmo que \textunderscore conspirador\textunderscore .
\section{Conspirar}
\begin{itemize}
\item {Grp. gram.:v. t.}
\end{itemize}
\begin{itemize}
\item {Proveniência:(Lat. \textunderscore conspirare\textunderscore )}
\end{itemize}
Concorrer para certo fim.
Tramar, maquinar, contra os poderes públicos.
Entrar em conlúio contra outrem.
\section{Conspirata}
\begin{itemize}
\item {Grp. gram.:f.}
\end{itemize}
\begin{itemize}
\item {Utilização:Fam.}
\end{itemize}
O mesmo que \textunderscore conspiração\textunderscore .
\section{Conspirativo}
\begin{itemize}
\item {Grp. gram.:adj.}
\end{itemize}
Que conspira ou concorre para certo effeito:«\textunderscore o bello, o bom, o verdadeiro e o útil eram simultaneamente conspirativos para a felicitação da humanidade\textunderscore ». Castilho.
\section{Conspurcação}
\begin{itemize}
\item {Grp. gram.:f.}
\end{itemize}
Acto ou effeito de conspurcar.
\section{Conspurcar}
\begin{itemize}
\item {Grp. gram.:v. t.}
\end{itemize}
\begin{itemize}
\item {Proveniência:(Lat. \textunderscore conspurcare\textunderscore )}
\end{itemize}
Sujar.
Pôr nódoas em.
Corromper.
Macular: \textunderscore conspurcar a reputação de alguém\textunderscore .
\section{Conspurcável}
\begin{itemize}
\item {Grp. gram.:adj.}
\end{itemize}
Que póde ser conspurcado.
\section{Constância}
\begin{itemize}
\item {Grp. gram.:f.}
\end{itemize}
\begin{itemize}
\item {Proveniência:(Lat. \textunderscore constantia\textunderscore )}
\end{itemize}
Qualidade daquillo ou daquelle que é constante.
Perseverança.
Persistência: \textunderscore a constância da chuva\textunderscore .
Coragem, firmeza de ânimo.
Duração.
\section{Constante}
\begin{itemize}
\item {Grp. gram.:adj.}
\end{itemize}
\begin{itemize}
\item {Proveniência:(Lat. \textunderscore constans\textunderscore )}
\end{itemize}
Que se não desloca.
Que se não altera, que não muda.
Incessante: \textunderscore amor constante\textunderscore .
Invariável.
Unânime: \textunderscore opinião constante\textunderscore .
Consignado, mencionado.
Que consiste, que é formado: \textunderscore livraria, constante de mil volumes\textunderscore .
\section{Constantemente}
\begin{itemize}
\item {Grp. gram.:adv.}
\end{itemize}
De modo constante; continuamente.
\section{Constantino}
\begin{itemize}
\item {Grp. gram.:adj.}
\end{itemize}
Relativo a Constantino ou feito por Constantino, fabricante de flôres, e conhecido por \textunderscore o rei dos floristas\textunderscore :«\textunderscore ...e flôres constantinas\textunderscore ». Castilho, \textunderscore Misanthropo\textunderscore , 24.
\section{Constantinopolitano}
\begin{itemize}
\item {Grp. gram.:m.}
\end{itemize}
\begin{itemize}
\item {Grp. gram.:Adj.}
\end{itemize}
\begin{itemize}
\item {Proveniência:(Lat. \textunderscore constantinopolitanus\textunderscore )}
\end{itemize}
Indivíduo natural de Constantinopla.
Relativo a Constantinopla.
\section{Constar}
\begin{itemize}
\item {Grp. gram.:v. i.}
\end{itemize}
\begin{itemize}
\item {Proveniência:(Lat. \textunderscore constare\textunderscore )}
\end{itemize}
Saber-se.
Passar por certo.
Contar-se como provável.
Deduzir-se.
Consistir, ser formado: \textunderscore esta casa consta de quatro andares\textunderscore .
\section{Constatar}
\begin{itemize}
\item {Grp. gram.:v. t.}
\end{itemize}
\begin{itemize}
\item {Proveniência:(Fr. \textunderscore constater\textunderscore )}
\end{itemize}
(É francesismo dispensável, com a significação de \textunderscore certificar\textunderscore , \textunderscore mostrar\textunderscore , \textunderscore provar\textunderscore )
\section{Constelação}
\begin{itemize}
\item {Grp. gram.:f.}
\end{itemize}
\begin{itemize}
\item {Utilização:Fig.}
\end{itemize}
\begin{itemize}
\item {Proveniência:(Lat. \textunderscore constellatio\textunderscore )}
\end{itemize}
Grupo de estrêlas, que, ligadas por linhas imaginárias, formam diferentes figuras nos mapas celestes.
Conjunto de ornatos ou outros objectos brilhantes.
\section{Constelar}
\begin{itemize}
\item {Grp. gram.:v. t.}
\end{itemize}
\begin{itemize}
\item {Grp. gram.:V. p.}
\end{itemize}
\begin{itemize}
\item {Utilização:Marn.}
\end{itemize}
\begin{itemize}
\item {Proveniência:(Do lat. \textunderscore cum\textunderscore  + \textunderscore stellare\textunderscore )}
\end{itemize}
Reunir em fórma de constelação.
Ornar de objectos brilhantes, semelhantes a estrêlas.
Iluminar superiormente.
Aureolar.
Mostrar (a água) cristaes de cloreto de sódio.
\section{Constellação}
\begin{itemize}
\item {Grp. gram.:f.}
\end{itemize}
\begin{itemize}
\item {Utilização:Fig.}
\end{itemize}
\begin{itemize}
\item {Proveniência:(Lat. \textunderscore constellatio\textunderscore )}
\end{itemize}
Grupo de estrêllas, que, ligadas por linhas imaginárias, formam differentes figuras nos mappas celestes.
Conjunto de ornatos ou outros objectos brilhantes.
\section{Constellar}
\begin{itemize}
\item {Grp. gram.:v. t.}
\end{itemize}
\begin{itemize}
\item {Grp. gram.:V. p.}
\end{itemize}
\begin{itemize}
\item {Utilização:Marn.}
\end{itemize}
\begin{itemize}
\item {Proveniência:(Do lat. \textunderscore cum\textunderscore  + \textunderscore stellare\textunderscore )}
\end{itemize}
Reunir em fórma de constellação.
Ornar de objectos brilhantes, semelhantes a estrêllas.
Illuminar superiormente.
Aureolar.
Mostrar (a água) crystaes de chloreto de sódio.
\section{Consternação}
\begin{itemize}
\item {Grp. gram.:f.}
\end{itemize}
\begin{itemize}
\item {Proveniência:(Lat. \textunderscore consternatio\textunderscore )}
\end{itemize}
Effeito de consternar.
\section{Consternador}
\begin{itemize}
\item {Grp. gram.:adj.}
\end{itemize}
Que consterna.
\section{Consternar}
\begin{itemize}
\item {Grp. gram.:v. i.}
\end{itemize}
\begin{itemize}
\item {Proveniência:(Lat. \textunderscore consternare\textunderscore )}
\end{itemize}
Desalentar.
Causar abatimento a.
Affligir.
Prostrar.
\section{Constipação}
\begin{itemize}
\item {Grp. gram.:f.}
\end{itemize}
\begin{itemize}
\item {Proveniência:(Lat. \textunderscore constipatio\textunderscore )}
\end{itemize}
Estado mórbido, com differentes caracteres, taes como: defluxão, calafrios, prisão de ventre, etc.
\section{Constipado}
\begin{itemize}
\item {Grp. gram.:adj.}
\end{itemize}
\begin{itemize}
\item {Proveniência:(De \textunderscore constipar\textunderscore )}
\end{itemize}
Que soffre constipação.
\section{Constipar}
\begin{itemize}
\item {Grp. gram.:v. t.}
\end{itemize}
\begin{itemize}
\item {Grp. gram.:V. p.}
\end{itemize}
\begin{itemize}
\item {Proveniência:(Lat. \textunderscore constipare\textunderscore )}
\end{itemize}
Causar constipação a.
Adquirir constipação.
\section{Constipativo}
\begin{itemize}
\item {Grp. gram.:adj.}
\end{itemize}
\begin{itemize}
\item {Proveniência:(De \textunderscore constipar\textunderscore )}
\end{itemize}
Que produz constipação.
\section{Constitucional}
\begin{itemize}
\item {Grp. gram.:adj.}
\end{itemize}
\begin{itemize}
\item {Grp. gram.:M.}
\end{itemize}
\begin{itemize}
\item {Proveniência:(Do lat. \textunderscore constitutio\textunderscore )}
\end{itemize}
Relativo á Constituição: \textunderscore reforma constitucional\textunderscore .
Que é conforme á Constituição de um Estado: \textunderscore procedimento constitucional\textunderscore .
Próprio do temperamento do indivíduo, inherente á sua organização: \textunderscore fraqueza constitucional\textunderscore .
Partidário da \textunderscore Carta Constitucional\textunderscore .
\section{Constitucionalidade}
\begin{itemize}
\item {Grp. gram.:f.}
\end{itemize}
Qualidade daquillo que é conforme á \textunderscore Carta Constitucional\textunderscore .
\section{Constitucionalismo}
\begin{itemize}
\item {Grp. gram.:m.}
\end{itemize}
Systema, partido, dos sectários da \textunderscore Carta Constitucional\textunderscore .
\section{Constitucionalista}
\begin{itemize}
\item {Grp. gram.:m.}
\end{itemize}
Partidário do constitucionalismo.
\section{Constitucionalizar}
\begin{itemize}
\item {Grp. gram.:v. t.}
\end{itemize}
Fazer constitucional.
\section{Constitucionalmente}
\begin{itemize}
\item {Grp. gram.:adv.}
\end{itemize}
De modo constitucional.
\section{Constituição}
\begin{itemize}
\item {fónica:tu-i}
\end{itemize}
\begin{itemize}
\item {Grp. gram.:f.}
\end{itemize}
\begin{itemize}
\item {Proveniência:(Do lat. \textunderscore constitutio\textunderscore )}
\end{itemize}
Acto de constituir, de firmar, de estabelecer: \textunderscore constituição de um syndicato\textunderscore .
Organização, natureza particular.
Compleição phýsica: \textunderscore o pequeno tem constituição robusta\textunderscore .
Formação.
Lei fundamental, que regula os direitos e deveres dos cidadãos, em relação ao Estado: \textunderscore a Constituição da República Portuguesa\textunderscore .
Conjunto de preceitos, por que se regula uma instituição, corporação, etc.
Estatuto, ordenação.
\section{Constituidor}
\begin{itemize}
\item {Grp. gram.:m.  e  adj.}
\end{itemize}
O mesmo que \textunderscore constituinte\textunderscore .
\section{Constituinte}
\begin{itemize}
\item {Grp. gram.:adj.}
\end{itemize}
\begin{itemize}
\item {Grp. gram.:M.}
\end{itemize}
\begin{itemize}
\item {Grp. gram.:F.}
\end{itemize}
\begin{itemize}
\item {Grp. gram.:Pl.}
\end{itemize}
\begin{itemize}
\item {Proveniência:(Lat. \textunderscore constituens\textunderscore )}
\end{itemize}
Que constitue.
Que faz parte de um organismo, de um todo.
Que pugna pela formação de uma nova constituição nacional.
Relativo a côrtes, que além das attribuições ordinárias, podem reformar a constituição do Estado, inteira ou parcialmente.
Pessôa, que faz de outra seu procurador ou representante: \textunderscore os constituintes do advogado Vaz\textunderscore .
Aquelle que faz parte de uma câmara legislativa constituinte.
Uma das primeiras assembleias legislativas da primeira república francesa.
Côrtes, convocadas para que, além das suas attribuições ordinárias, reformem a lei fundamental da nação.
\section{Constituir}
\begin{itemize}
\item {Grp. gram.:v. t.}
\end{itemize}
\begin{itemize}
\item {Proveniência:(Lat. \textunderscore constituere\textunderscore )}
\end{itemize}
Formar a essência de: \textunderscore a farda militar constitue a sua ambição\textunderscore .
Estabelecer.
Organizar: \textunderscore constituir uma associação\textunderscore .
Firmar.
Dar procuração a.
\section{Constitutivamente}
\begin{itemize}
\item {Grp. gram.:adv.}
\end{itemize}
De modo constitutivo.
\section{Constitutivo}
\begin{itemize}
\item {Grp. gram.:adj.}
\end{itemize}
\begin{itemize}
\item {Proveniência:(Lat. \textunderscore constitutivus\textunderscore )}
\end{itemize}
Que constitue, que estabelece.
Essencial.
Característico.
\section{Constório}
\begin{itemize}
\item {Grp. gram.:m.}
\end{itemize}
\begin{itemize}
\item {Utilização:Prov.}
\end{itemize}
\begin{itemize}
\item {Utilização:trasm.}
\end{itemize}
Commentário desfavorável.
Reunião de pessôas, para murmurar das vidas alheias.
(Contr. de \textunderscore consistório\textunderscore ?)
\section{Constrangedor}
\begin{itemize}
\item {Grp. gram.:adj.}
\end{itemize}
Que constrange.
\section{Constranger}
\begin{itemize}
\item {Grp. gram.:v. t.}
\end{itemize}
Apertar.
Compellir.
Obrigar á força.
Impedir os movimentos de.
(Cp. lat. \textunderscore constringere\textunderscore )
\section{Constrangidamente}
\begin{itemize}
\item {Grp. gram.:adv.}
\end{itemize}
De modo constrangido.
\section{Constrangimento}
\begin{itemize}
\item {Grp. gram.:m.}
\end{itemize}
Acto ou effeito de constranger.
\section{Constrição}
\begin{itemize}
\item {Grp. gram.:f.}
\end{itemize}
\begin{itemize}
\item {Proveniência:(Lat. \textunderscore constrictio\textunderscore )}
\end{itemize}
Pressão circular, que deminue o diâmetro de um objecto.
\section{Constricção}
\begin{itemize}
\item {Grp. gram.:f.}
\end{itemize}
\begin{itemize}
\item {Proveniência:(Lat. \textunderscore constrictio\textunderscore )}
\end{itemize}
Pressão circular, que deminue o diâmetro de um objecto.
\section{Constrictivo}
\begin{itemize}
\item {Grp. gram.:adj.}
\end{itemize}
\begin{itemize}
\item {Proveniência:(Lat. \textunderscore constrictivus\textunderscore )}
\end{itemize}
Que produz constricção.
\section{Constricto}
\begin{itemize}
\item {Grp. gram.:part. irr.  de  constringir}
\end{itemize}
\begin{itemize}
\item {Utilização:Gram.}
\end{itemize}
Diz-se das vozes, que geralmente se têm chamado \textunderscore consoantes\textunderscore .
\section{Constrictor}
\begin{itemize}
\item {Grp. gram.:adj.}
\end{itemize}
\begin{itemize}
\item {Grp. gram.:M.}
\end{itemize}
\begin{itemize}
\item {Proveniência:(Lat. \textunderscore constrictor\textunderscore )}
\end{itemize}
Que constringe.
Espécie de serpente da Guiana.
\section{Constringente}
\begin{itemize}
\item {Grp. gram.:adj.}
\end{itemize}
\begin{itemize}
\item {Proveniência:(Lat. \textunderscore constringens\textunderscore )}
\end{itemize}
Que constringe.
\section{Constringir}
\begin{itemize}
\item {Grp. gram.:v. t.}
\end{itemize}
\begin{itemize}
\item {Proveniência:(Lat. \textunderscore constringere\textunderscore )}
\end{itemize}
Cingir, apertando.
Apertar circularmente.
\section{Constritivo}
\begin{itemize}
\item {Grp. gram.:adj.}
\end{itemize}
\begin{itemize}
\item {Proveniência:(Lat. \textunderscore constrictivus\textunderscore )}
\end{itemize}
Que produz constrição.
\section{Constrito}
\begin{itemize}
\item {Grp. gram.:part. irr.  de  constringir}
\end{itemize}
\begin{itemize}
\item {Utilização:Gram.}
\end{itemize}
Diz-se das vozes, que geralmente se têm chamado \textunderscore consoantes\textunderscore .
\section{Constritor}
\begin{itemize}
\item {Grp. gram.:adj.}
\end{itemize}
\begin{itemize}
\item {Grp. gram.:M.}
\end{itemize}
\begin{itemize}
\item {Proveniência:(Lat. \textunderscore constrictor\textunderscore )}
\end{itemize}
Que constringe.
Espécie de serpente da Guiana.
\section{Construção}
\begin{itemize}
\item {Grp. gram.:f.}
\end{itemize}
\begin{itemize}
\item {Proveniência:(Lat. \textunderscore constructio\textunderscore )}
\end{itemize}
Acto, efeito, modo ou arte do construir.
Organismo.
\section{Construcção}
\begin{itemize}
\item {Grp. gram.:f.}
\end{itemize}
\begin{itemize}
\item {Proveniência:(Lat. \textunderscore constructio\textunderscore )}
\end{itemize}
Acto, effeito, modo ou arte do construir.
Organismo.
\section{Constructivamente}
\begin{itemize}
\item {Grp. gram.:adv.}
\end{itemize}
De modo constructivo.
\section{Constructivo}
\begin{itemize}
\item {Grp. gram.:adj.}
\end{itemize}
\begin{itemize}
\item {Proveniência:(Lat. \textunderscore constructivus\textunderscore )}
\end{itemize}
Que serve para construir.
\section{Constructor}
\begin{itemize}
\item {Grp. gram.:m.  e  adj.}
\end{itemize}
\begin{itemize}
\item {Proveniência:(Lat. \textunderscore constructor\textunderscore )}
\end{itemize}
O que constróe.
\section{Constructura}
\begin{itemize}
\item {Grp. gram.:f.}
\end{itemize}
Modo de construir.
\section{Construir}
\begin{itemize}
\item {Grp. gram.:v. t.}
\end{itemize}
\begin{itemize}
\item {Proveniência:(Lat. \textunderscore construere\textunderscore )}
\end{itemize}
Edificar: \textunderscore construir prédios\textunderscore .
Dar estructura a.
Formar, architectar: \textunderscore construir planos\textunderscore .
Dispor, segundo certas regras (as palavras).
Traçar (figuras geométrias).
\section{Construtivamente}
\begin{itemize}
\item {Grp. gram.:adv.}
\end{itemize}
De modo construtivo.
\section{Construtivo}
\begin{itemize}
\item {Grp. gram.:adj.}
\end{itemize}
\begin{itemize}
\item {Proveniência:(Lat. \textunderscore constructivus\textunderscore )}
\end{itemize}
Que serve para construir.
\section{Construtor}
\begin{itemize}
\item {Grp. gram.:m.  e  adj.}
\end{itemize}
\begin{itemize}
\item {Proveniência:(Lat. \textunderscore constructor\textunderscore )}
\end{itemize}
O que constróe.
\section{Construtura}
\begin{itemize}
\item {Grp. gram.:f.}
\end{itemize}
Modo de construir.
\section{Consubstanciação}
\begin{itemize}
\item {Grp. gram.:f.}
\end{itemize}
\begin{itemize}
\item {Proveniência:(De \textunderscore consubstanciar\textunderscore )}
\end{itemize}
União de dois ou mais corpos de uma só substância.
\section{Consubstancial}
\begin{itemize}
\item {Grp. gram.:adj.}
\end{itemize}
\begin{itemize}
\item {Proveniência:(Lat. \textunderscore consubstantialis\textunderscore )}
\end{itemize}
Que tem uma só substância.
\section{Consubstancialidade}
\begin{itemize}
\item {Grp. gram.:f.}
\end{itemize}
\begin{itemize}
\item {Proveniência:(Lat. \textunderscore consubstancialitas\textunderscore )}
\end{itemize}
Unidade de substância.
\section{Consubstancialmente}
\begin{itemize}
\item {Grp. gram.:adv.}
\end{itemize}
De modo consubstancial.
\section{Consubstanciar}
\begin{itemize}
\item {Grp. gram.:v. t.}
\end{itemize}
\begin{itemize}
\item {Proveniência:(Do lat. \textunderscore cum\textunderscore  + \textunderscore substantia\textunderscore )}
\end{itemize}
Juntar numa só substância.
Unificar.
\section{Consueto}
\begin{itemize}
\item {Grp. gram.:adj.}
\end{itemize}
\begin{itemize}
\item {Proveniência:(Lat. \textunderscore consuetus\textunderscore )}
\end{itemize}
Acostumado.
Usual.
\section{Consuetudinário}
\begin{itemize}
\item {Grp. gram.:adj.}
\end{itemize}
\begin{itemize}
\item {Proveniência:(Lat. \textunderscore consuetudinarius\textunderscore )}
\end{itemize}
Costumado.
Fundado nos costumes: \textunderscore o direito consuetudinário\textunderscore .
\section{Consuetudinarismo}
\begin{itemize}
\item {Grp. gram.:m.}
\end{itemize}
Apêgo ao que é consuetudinário.
\section{Cônsul}
\begin{itemize}
\item {Grp. gram.:m.}
\end{itemize}
\begin{itemize}
\item {Proveniência:(Lat. \textunderscore consul\textunderscore )}
\end{itemize}
Um dos magistrados supremos na república romana, e na primeira república francesa.
Agente de uma nação, encarregado, em país estrangeiro, de proteger os súbditos dessa nação, fomentar o respectivo commércio, etc.
\section{Consulado}
\begin{itemize}
\item {Grp. gram.:m.}
\end{itemize}
\begin{itemize}
\item {Proveniência:(Lat. \textunderscore consulatus\textunderscore )}
\end{itemize}
Dignidade, cargo de cônsul.
Tempo, em que um indivíduo exerce êsse cargo.
Residência de cônsul ou casa onde se exercem as funcções de cônsul.
\section{Consulagem}
\begin{itemize}
\item {Grp. gram.:f.}
\end{itemize}
\begin{itemize}
\item {Proveniência:(De \textunderscore cônsul\textunderscore )}
\end{itemize}
Emolumento, que se paga ao cônsul, pela sua intervenção na expedição de navios.
\section{Consular}
\begin{itemize}
\item {Grp. gram.:adj.}
\end{itemize}
\begin{itemize}
\item {Proveniência:(Lat. \textunderscore consularis\textunderscore )}
\end{itemize}
Relativo a cônsul.
\section{Consularmente}
\begin{itemize}
\item {Grp. gram.:adv.}
\end{itemize}
Pelos meios consulares.
Pela jurisdicção consular.
\section{Consulente}
\begin{itemize}
\item {Grp. gram.:m.  e  adj.}
\end{itemize}
\begin{itemize}
\item {Proveniência:(Lat. \textunderscore consulens\textunderscore )}
\end{itemize}
O que consulta.
\section{Consulêsa}
\begin{itemize}
\item {Grp. gram.:f.}
\end{itemize}
Mulher de cônsul.
\section{Consulta}
\begin{itemize}
\item {Grp. gram.:f.}
\end{itemize}
\begin{itemize}
\item {Proveniência:(Lat. \textunderscore consultus\textunderscore )}
\end{itemize}
Acto de consultar.
Conselho.
Parecer.
Conferência para deliberação.
Reflexão, prudência.
\section{Consultação}
\begin{itemize}
\item {Grp. gram.:f.}
\end{itemize}
\begin{itemize}
\item {Proveniência:(Lat. \textunderscore consultatio\textunderscore )}
\end{itemize}
Acto de consultar.
\section{Consultador}
\begin{itemize}
\item {Grp. gram.:m.  e  adj.}
\end{itemize}
O que consulta.
\section{Consultamente}
\begin{itemize}
\item {Grp. gram.:adv.}
\end{itemize}
\begin{itemize}
\item {Proveniência:(Do lat. \textunderscore consultus\textunderscore )}
\end{itemize}
Judiciosamente.
Sabiamente.
\section{Consultante}
\begin{itemize}
\item {Grp. gram.:m., f.  e  adj.}
\end{itemize}
\begin{itemize}
\item {Proveniência:(Lat. \textunderscore consultans\textunderscore )}
\end{itemize}
Pessôa, que consulta.
Quem pede conselho.
Quem dá conselho.
\section{Consultar}
\begin{itemize}
\item {Grp. gram.:v. t.}
\end{itemize}
\begin{itemize}
\item {Grp. gram.:V. i.}
\end{itemize}
\begin{itemize}
\item {Proveniência:(Lat. \textunderscore consultare\textunderscore )}
\end{itemize}
Pedir conselho, parecer, a.
Observar.
Procurar esclarecimentos em: \textunderscore consultar os clássicos\textunderscore .
Examinar.
Conferenciar.
Dar parecer.
\section{Consultivo}
\begin{itemize}
\item {Grp. gram.:adj.}
\end{itemize}
Relativo a consulta.
Que involve conselho.
E diz-se das corporações, que emittem parecer sem voto deliberativo.
\section{Consulto}
\begin{itemize}
\item {Grp. gram.:m.}
\end{itemize}
\begin{itemize}
\item {Utilização:Ant.}
\end{itemize}
\begin{itemize}
\item {Proveniência:(Lat. \textunderscore consultus\textunderscore )}
\end{itemize}
Conluio; conspiração:«\textunderscore ...fazião muitos ajuntamentos e consultos pera fazerem mal\textunderscore ». \textunderscore Alvará\textunderscore  de D. Sebast., in \textunderscore Rev. Lus.\textunderscore , XV, 105.
\section{Consultor}
\begin{itemize}
\item {Grp. gram.:m.}
\end{itemize}
\begin{itemize}
\item {Proveniência:(Lat. \textunderscore consultor\textunderscore )}
\end{itemize}
Aquelle que dá conselho.
Aquelle que o pede.
Quem consulta, examinando.
\section{Consultório}
\begin{itemize}
\item {Grp. gram.:m.}
\end{itemize}
\begin{itemize}
\item {Proveniência:(De \textunderscore consultor\textunderscore )}
\end{itemize}
Lugar, ou casa, onde se dão consultas de medicina ou de engenharia ou de outras matérias.
\section{Consumação}
\begin{itemize}
\item {Grp. gram.:f.}
\end{itemize}
Acto de consumar.
\section{Consumadamente}
\begin{itemize}
\item {Grp. gram.:adv.}
\end{itemize}
De modo consumado.
\section{Consumado}
\begin{itemize}
\item {Grp. gram.:adj.}
\end{itemize}
\begin{itemize}
\item {Proveniência:(De \textunderscore consumar\textunderscore )}
\end{itemize}
Que se consumou; acabado.
Perfeito: \textunderscore artista consumado\textunderscore .
\section{Consumador}
\begin{itemize}
\item {Grp. gram.:m.}
\end{itemize}
Aquele que consuma, acaba ou aperfeiçôa.
\section{Consumar}
\begin{itemize}
\item {Grp. gram.:v. t.}
\end{itemize}
\begin{itemize}
\item {Proveniência:(Lat. \textunderscore consummare\textunderscore )}
\end{itemize}
Terminar, completar.
Praticar: \textunderscore consumar um delicto\textunderscore .
Aperfeiçoar.
\section{Consumição}
\begin{itemize}
\item {Grp. gram.:f.}
\end{itemize}
Acto de consumir.
Effeito de mortificar.
\section{Consumidor}
\begin{itemize}
\item {Grp. gram.:adj.}
\end{itemize}
\begin{itemize}
\item {Grp. gram.:M.}
\end{itemize}
\begin{itemize}
\item {Proveniência:(De \textunderscore consumir\textunderscore )}
\end{itemize}
Que consome.
Quem compra, para gastar em uso próprio.
\section{Consumir}
\begin{itemize}
\item {Grp. gram.:v. t.}
\end{itemize}
\begin{itemize}
\item {Utilização:Fig.}
\end{itemize}
\begin{itemize}
\item {Grp. gram.:V. i.}
\end{itemize}
\begin{itemize}
\item {Proveniência:(Lat. \textunderscore consumere\textunderscore )}
\end{itemize}
Gastar: \textunderscore consumir riquezas\textunderscore .
Destruir: \textunderscore o incendio consumiu o palácio\textunderscore .
Enfraquecer.
Absorver.
Obliterar: \textunderscore o tempo consome inscripções lapidares\textunderscore .
Desfazer na boca (a hóstia).
Mortificar.
Commungar (o padre á Missa).
\section{Consumível}
\begin{itemize}
\item {Grp. gram.:adj.}
\end{itemize}
Que se póde consumir.
\section{Consummação}
\begin{itemize}
\item {Grp. gram.:f.}
\end{itemize}
Acto de consummar.
\section{Consummadamente}
\begin{itemize}
\item {Grp. gram.:adv.}
\end{itemize}
De modo consummado.
\section{Consummado}
\begin{itemize}
\item {Grp. gram.:adj.}
\end{itemize}
\begin{itemize}
\item {Proveniência:(De \textunderscore consummar\textunderscore )}
\end{itemize}
Que se consummou; acabado.
Perfeito: \textunderscore artista consummado\textunderscore .
\section{Consummador}
\begin{itemize}
\item {Grp. gram.:m.}
\end{itemize}
Aquelle que consumma, acaba ou aperfeiçôa.
\section{Consummar}
\begin{itemize}
\item {Grp. gram.:v. t.}
\end{itemize}
\begin{itemize}
\item {Proveniência:(Lat. \textunderscore consummare\textunderscore )}
\end{itemize}
Terminar, completar.
Praticar: \textunderscore consummar um delicto\textunderscore .
Aperfeiçoar.
\section{Consumo}
\begin{itemize}
\item {Grp. gram.:m.}
\end{itemize}
Acto e effeito de consumir.
Extracção de mercadorias.
\section{Consumpção}
\begin{itemize}
\item {Grp. gram.:f.}
\end{itemize}
\begin{itemize}
\item {Proveniência:(Lat. \textunderscore consumptio\textunderscore )}
\end{itemize}
Effeito de consumir.
Definhamento lento e progressivo, produzido por doença no organismo humano.
\section{Consumptibilidade}
\begin{itemize}
\item {Grp. gram.:f.}
\end{itemize}
Qualidade daquillo que é consumptível.
\section{Consumptível}
\begin{itemize}
\item {Grp. gram.:adj.}
\end{itemize}
\begin{itemize}
\item {Proveniência:(De \textunderscore consumpto\textunderscore )}
\end{itemize}
Que se póde consumir.
\section{Consumptivo}
\begin{itemize}
\item {Grp. gram.:adj.}
\end{itemize}
\begin{itemize}
\item {Proveniência:(De \textunderscore consumpto\textunderscore )}
\end{itemize}
Que consome.
\section{Consumpto}
\begin{itemize}
\item {Grp. gram.:part. irr. de consumir.}
\end{itemize}
\begin{itemize}
\item {Proveniência:(Lat. \textunderscore consumptus\textunderscore )}
\end{itemize}
Consumido.
\section{Consunção}
\begin{itemize}
\item {Grp. gram.:f.}
\end{itemize}
\begin{itemize}
\item {Proveniência:(Lat. \textunderscore consumptio\textunderscore )}
\end{itemize}
Efeito de consumir.
Definhamento lento e progressivo, produzido por doença no organismo humano.
\section{Consuntibilidade}
\begin{itemize}
\item {Grp. gram.:f.}
\end{itemize}
Qualidade daquilo que é consuntível.
\section{Consuntível}
\begin{itemize}
\item {Grp. gram.:adj.}
\end{itemize}
\begin{itemize}
\item {Proveniência:(De \textunderscore consunto\textunderscore )}
\end{itemize}
Que se póde consumir.
\section{Consuntivo}
\begin{itemize}
\item {Grp. gram.:adj.}
\end{itemize}
\begin{itemize}
\item {Proveniência:(De \textunderscore consunto\textunderscore )}
\end{itemize}
Que consome.
\section{Consunto}
\begin{itemize}
\item {Grp. gram.:part. irr. de consumir.}
\end{itemize}
\begin{itemize}
\item {Proveniência:(Lat. \textunderscore consumptus\textunderscore )}
\end{itemize}
Consumido.
\section{Conta}
\begin{itemize}
\item {Grp. gram.:f.}
\end{itemize}
\begin{itemize}
\item {Utilização:Prov.}
\end{itemize}
\begin{itemize}
\item {Utilização:alent.}
\end{itemize}
Acto e effeito de contar.
Cálculo.
Estado de créditos e débitos ou de receita e despesa.
Objecto ou somma de uma dívida.
Reputação: \textunderscore tenho-o na conta de honrado\textunderscore .
Attenção.
Cuidado, cautela.
Gasto.
Justificação.
Imputação.
Supposição.
Estimação.
Responsabilidade: \textunderscore proceder por conta alheia\textunderscore .
Informação: \textunderscore dar conta de successos\textunderscore .
Unidade de cereaes, correspondente a 40 alqueires.
Pequena bola furada, que serve para rosários, collares, etc.
\textunderscore Fazer conta que\textunderscore , suppor, imaginar:«\textunderscore fazei conta que um rei...\textunderscore »F. Manuel, \textunderscore Apól.\textunderscore 
\textunderscore Á conta de\textunderscore , por causa de:«\textunderscore execraram sempre Luciano á conta de uma biographia escandalosa que elle escreveu\textunderscore ». Camillo, \textunderscore Sebenta\textunderscore , VII, 23.
\textunderscore Levar a sua conta\textunderscore , apanhar sova. Cf. Camillo, \textunderscore Perfil\textunderscore , 256.
\textunderscore Tomar conta\textunderscore , encarregar-se: \textunderscore tomou conta do pequeno\textunderscore .
\textunderscore Dar conta\textunderscore , sêr responsável, sêr obrigado a apresentar: \textunderscore hás de me dar conta do relógio que te emprestei\textunderscore .
\section{Contabescência}
\begin{itemize}
\item {Grp. gram.:f.}
\end{itemize}
Estado de contabescente.
\section{Contabescente}
\begin{itemize}
\item {Grp. gram.:adj.}
\end{itemize}
\begin{itemize}
\item {Proveniência:(Do lat. \textunderscore contabescere\textunderscore )}
\end{itemize}
Que definha, que emmagrece extraordinariamente, por effeito de doença.
\section{Contabescer}
\begin{itemize}
\item {Grp. gram.:v. i.}
\end{itemize}
\begin{itemize}
\item {Utilização:Med.}
\end{itemize}
\begin{itemize}
\item {Proveniência:(Lat. \textunderscore contabescere\textunderscore )}
\end{itemize}
Definhar, consumir-se.
\section{Contabilidade}
\begin{itemize}
\item {Grp. gram.:f.}
\end{itemize}
\begin{itemize}
\item {Proveniência:(Do lat. \textunderscore computabilis\textunderscore )}
\end{itemize}
Arte de fazer contas commerciaes ou burocráticas.
Cálculo.
Repartição, onde se escrituram receitas e despesas.
Escrituração de receitas e despesas.
\section{Contacto}
\begin{itemize}
\item {Grp. gram.:m.}
\end{itemize}
\begin{itemize}
\item {Utilização:Fig.}
\end{itemize}
\begin{itemize}
\item {Proveniência:(Lat. \textunderscore contactus\textunderscore )}
\end{itemize}
Estado de dois ou mais corpos que se tocam.
Acto de exercer o sentido do tacto.
Influência, proximidade.
\section{Contadamente}
\begin{itemize}
\item {Grp. gram.:adv.}
\end{itemize}
\begin{itemize}
\item {Proveniência:(De \textunderscore contado\textunderscore )}
\end{itemize}
Por conta.
\section{Conta-de-pão}
\begin{itemize}
\item {Grp. gram.:f.  Loc.}
\end{itemize}
\begin{itemize}
\item {Utilização:Loc. de Leiria.}
\end{itemize}
Déz pães.
\section{Contado}
\begin{itemize}
\item {Grp. gram.:m.}
\end{itemize}
\begin{itemize}
\item {Utilização:Gír.}
\end{itemize}
O anno.
\section{Contador}
\begin{itemize}
\item {Grp. gram.:m.}
\end{itemize}
\begin{itemize}
\item {Grp. gram.:Adj.}
\end{itemize}
\begin{itemize}
\item {Proveniência:(Do lat. \textunderscore computator\textunderscore )}
\end{itemize}
Aquelle que conta.
Aquelle que refere.
Aquelle que verifica contas.
Funccionário judicial, que conta salários e custas.
Apparelho, para a contagem do gás ou da água.
Peça de mobiliário antigo, que consiste numa espécie de armário com pequeninas gavêtas, mais ou menos numerosas, firmado numa peanha ou sustentado por quatro pés.
Borla de rosário.
Que conta.
\section{Contadoria}
\begin{itemize}
\item {Grp. gram.:f.}
\end{itemize}
\begin{itemize}
\item {Proveniência:(De \textunderscore contador\textunderscore )}
\end{itemize}
Repartição, onde se verificam contas, ou onde se paga e se recebe dinheiro.
\section{Conta-fios}
\begin{itemize}
\item {Grp. gram.:m.}
\end{itemize}
\begin{itemize}
\item {Proveniência:(De \textunderscore contar\textunderscore  + \textunderscore fio\textunderscore )}
\end{itemize}
Espécie de microscópio, usado nas alfândegas, para a contagem dos fios de um tecido em que entram substâncias differentes.
\section{Contagem}
\begin{itemize}
\item {Grp. gram.:f.}
\end{itemize}
Acto ou effeito de contar.
Salário de contador de tribunal.
\section{Contagião}
\begin{itemize}
\item {Grp. gram.:f.}
\end{itemize}
\begin{itemize}
\item {Utilização:Des.}
\end{itemize}
\begin{itemize}
\item {Proveniência:(Lat. \textunderscore contagio\textunderscore )}
\end{itemize}
O mesmo que \textunderscore contágio\textunderscore . Cf. Lucena, \textunderscore S. Franc. Xav.\textunderscore ; Camillo, \textunderscore Coisas Espant.\textunderscore 
\section{Contagiar}
\begin{itemize}
\item {Grp. gram.:v. t.}
\end{itemize}
\begin{itemize}
\item {Proveniência:(De \textunderscore contágio\textunderscore )}
\end{itemize}
Communicar.
Propagar doença epidêmica a.
\section{Contágio}
\begin{itemize}
\item {Grp. gram.:m.}
\end{itemize}
\begin{itemize}
\item {Utilização:Ext.}
\end{itemize}
\begin{itemize}
\item {Proveniência:(Lat. \textunderscore contagium\textunderscore )}
\end{itemize}
Communicação, propagação, de doença, por contacto directo ou indirecto.
Doença contagiosa.
Transmissão de males ou vicios.
\section{Contagionista}
\begin{itemize}
\item {Grp. gram.:m.}
\end{itemize}
\begin{itemize}
\item {Proveniência:(De \textunderscore contagião\textunderscore )}
\end{itemize}
Sectário da doutrina pathológica, que affirma a propriedade de se transmittirem certas moléstias, de individuo para individuo.
\section{Contagiosidade}
\begin{itemize}
\item {Grp. gram.:f.}
\end{itemize}
Qualidade daquillo que é contagioso.
\section{Contagioso}
\begin{itemize}
\item {Grp. gram.:adj.}
\end{itemize}
\begin{itemize}
\item {Proveniência:(Lat. \textunderscore contagiosus\textunderscore )}
\end{itemize}
Que se communica por contágio.
\section{Conta-gotas}
\begin{itemize}
\item {Grp. gram.:m.}
\end{itemize}
\begin{itemize}
\item {Proveniência:(De \textunderscore contar\textunderscore  + \textunderscore gota\textunderscore )}
\end{itemize}
Instrumento, com que se contam as gotas de um medicamento ou de outro líquido.
\section{Contaminabilidade}
\begin{itemize}
\item {Grp. gram.:f.}
\end{itemize}
\begin{itemize}
\item {Proveniência:(Do lat. \textunderscore contaminabilis\textunderscore )}
\end{itemize}
Qualidade daquillo ou de quem é contaminável.
\section{Contaminação}
\begin{itemize}
\item {Grp. gram.:f.}
\end{itemize}
\begin{itemize}
\item {Proveniência:(Lat. \textunderscore contaminatio\textunderscore )}
\end{itemize}
Acto ou effeito de contaminar.
Infecção.
\section{Contaminador}
\begin{itemize}
\item {Grp. gram.:m.  e  adj.}
\end{itemize}
\begin{itemize}
\item {Proveniência:(Lat. \textunderscore contaminator\textunderscore )}
\end{itemize}
O que contamina.
\section{Contaminar}
\begin{itemize}
\item {Grp. gram.:v. t.}
\end{itemize}
\begin{itemize}
\item {Proveniência:(Lat. \textunderscore contaminare\textunderscore )}
\end{itemize}
Contagiar.
Infeccionar.
Manchar.
\section{Contaminável}
\begin{itemize}
\item {Grp. gram.:adj.}
\end{itemize}
\begin{itemize}
\item {Proveniência:(Lat. \textunderscore contaminabilis\textunderscore )}
\end{itemize}
Que póde sêr contaminado.
\section{Conta-passos}
\begin{itemize}
\item {Grp. gram.:m.}
\end{itemize}
(V.pedómetro)
\section{Contar}
\begin{itemize}
\item {Grp. gram.:v. t.}
\end{itemize}
\begin{itemize}
\item {Grp. gram.:V. i.}
\end{itemize}
\begin{itemize}
\item {Proveniência:(Lat. \textunderscore computare\textunderscore )}
\end{itemize}
Determinar o número de: \textunderscore contar as reses de um rebanho\textunderscore .
Enumerar.
Calcular.
Têr o numero de.
Têr: \textunderscore já conto 80 annos\textunderscore .
Têr na conta.
Levar em conta.
Planear, têr tenção de.
Narrar, descrever: \textunderscore contar histórias\textunderscore .
Fazer contas.
Têr esperança, confiar: \textunderscore contar com o auxílio de alguém\textunderscore .
\section{Contarênia}
\begin{itemize}
\item {Grp. gram.:f.}
\end{itemize}
Planta herbácea do Brasil.
\section{Contaria}
\begin{itemize}
\item {Grp. gram.:f.}
\end{itemize}
Estabelecimento, onde se fabricam ou vendem contas ou missanga.
Enfiadas de contas.
\section{Contável}
\begin{itemize}
\item {Grp. gram.:adj.}
\end{itemize}
Que se póde contar.
\section{Contego}
\begin{itemize}
\item {fónica:tê}
\end{itemize}
\begin{itemize}
\item {Grp. gram.:pron.}
\end{itemize}
\begin{itemize}
\item {Utilização:Ant.}
\end{itemize}
\begin{itemize}
\item {Grp. gram.:loc. pron.}
\end{itemize}
\begin{itemize}
\item {Utilização:Ant.}
\end{itemize}
O mesmo que \textunderscore contigo\textunderscore  ou \textunderscore comtigo\textunderscore .
O mesmo que \textunderscore contigo\textunderscore .
\section{Conteira}
\begin{itemize}
\item {Grp. gram.:f.}
\end{itemize}
\begin{itemize}
\item {Proveniência:(De \textunderscore conto\textunderscore ^2)}
\end{itemize}
Peça, com que se reforça a extremidade inferior da baínha das espadas.
Rasto do canhão.
Parte posterior do reparo da peça de artilharia.
\section{Conteirar}
\begin{itemize}
\item {Grp. gram.:v. t.}
\end{itemize}
Mover a conteira de.
\section{Conteiro}
\begin{itemize}
\item {Grp. gram.:m.}
\end{itemize}
Indivíduo que faz ou vende contas de rezar ou de enfeites.
\section{Contemplação}
\begin{itemize}
\item {Grp. gram.:f.}
\end{itemize}
\begin{itemize}
\item {Proveniência:(Lat. \textunderscore contemplatio\textunderscore )}
\end{itemize}
Acto de contemplar.
\section{Contemplador}
\begin{itemize}
\item {Grp. gram.:m.  e  adj.}
\end{itemize}
\begin{itemize}
\item {Proveniência:(Lat. \textunderscore contemplator\textunderscore )}
\end{itemize}
O que contempla.
\section{Contemplante}
\begin{itemize}
\item {Grp. gram.:adj.}
\end{itemize}
\begin{itemize}
\item {Proveniência:(Lat. \textunderscore contemplans\textunderscore )}
\end{itemize}
Que contempla.
\section{Contemplar}
\begin{itemize}
\item {Grp. gram.:v. t.}
\end{itemize}
\begin{itemize}
\item {Grp. gram.:V. i.}
\end{itemize}
\begin{itemize}
\item {Proveniência:(Lat. \textunderscore contemplari\textunderscore )}
\end{itemize}
Olhar com attenção: \textunderscore contemplar os astros\textunderscore .
Considerar com admiração ou com amor.
Meditar.
Imaginar.
Tratar com benevolência.
Attender: \textunderscore contemplar um pedido\textunderscore .
Remunerar.
Meditar profundamente.
Fitar os olhos:«\textunderscore contemplava na casta Lua\textunderscore ». \textunderscore Eufrosina\textunderscore , 27.
\section{Contemplativa}
\begin{itemize}
\item {Grp. gram.:f.}
\end{itemize}
\begin{itemize}
\item {Proveniência:(De \textunderscore contemplativo\textunderscore )}
\end{itemize}
Faculdade de contemplar.
\section{Contemplativamente}
\begin{itemize}
\item {Grp. gram.:adv.}
\end{itemize}
De modo contemplativo.
\section{Contemplativo}
\begin{itemize}
\item {Grp. gram.:adj.}
\end{itemize}
\begin{itemize}
\item {Proveniência:(Lat. \textunderscore contemplativus\textunderscore )}
\end{itemize}
Dado á contemplação.
Relativo á contemplação.
Que excita á contemplação.
\section{Contemporaneamente}
\begin{itemize}
\item {Grp. gram.:adv.}
\end{itemize}
\begin{itemize}
\item {Proveniência:(De \textunderscore contemporâneo\textunderscore )}
\end{itemize}
Ao mesmo tempo.
Na mesma época.
\section{Contemporaneidade}
\begin{itemize}
\item {Grp. gram.:f.}
\end{itemize}
Qualidade do que é contemporâneo.
\section{Contemporâneo}
\begin{itemize}
\item {Grp. gram.:adj.}
\end{itemize}
\begin{itemize}
\item {Grp. gram.:M.}
\end{itemize}
\begin{itemize}
\item {Proveniência:(Lat. \textunderscore contemporaneus\textunderscore )}
\end{itemize}
Que é do mesmo tempo.
Que é do nosso tempo.
Homem do mesmo tempo.
Homem do nosso tempo.
\section{Contemporão}
\begin{itemize}
\item {Grp. gram.:adj.}
\end{itemize}
O mesmo que \textunderscore contemporâneo\textunderscore . Cf. Garrett, \textunderscore Retr. de Vénus\textunderscore , 229.
\section{Contemporização}
\begin{itemize}
\item {Grp. gram.:f.}
\end{itemize}
Acto de contemporizar.
\section{Contemporizador}
\begin{itemize}
\item {Grp. gram.:m.  e  adj.}
\end{itemize}
O que contemporiza.
\section{Contemporizante}
\begin{itemize}
\item {Grp. gram.:adj.}
\end{itemize}
Que contemporiza.
\section{Contemporizar}
\begin{itemize}
\item {Grp. gram.:v. i.}
\end{itemize}
\begin{itemize}
\item {Grp. gram.:V. t.}
\end{itemize}
\begin{itemize}
\item {Proveniência:(De \textunderscore con...\textunderscore  + \textunderscore temporizar\textunderscore )}
\end{itemize}
Acommodar-se, transigir: \textunderscore contemporizar com exigências\textunderscore .
Dar tempo a.
Entreter.
\section{Contemprar}
\begin{itemize}
\item {Grp. gram.:v. t.}
\end{itemize}
\begin{itemize}
\item {Utilização:Ant.}
\end{itemize}
O mesmo que \textunderscore contemplar\textunderscore . Cf. \textunderscore Eufrosina\textunderscore , 155.
\section{Contemptamento}
\begin{itemize}
\item {Grp. gram.:m.}
\end{itemize}
\begin{itemize}
\item {Utilização:Ant.}
\end{itemize}
\begin{itemize}
\item {Proveniência:(Do lat. \textunderscore contemptus\textunderscore )}
\end{itemize}
O mesmo que \textunderscore desprêzo\textunderscore .
\section{Contemptível}
\begin{itemize}
\item {Grp. gram.:adj.}
\end{itemize}
\begin{itemize}
\item {Proveniência:(Lat. \textunderscore contemptibilis\textunderscore )}
\end{itemize}
Digno de desprêzo.
\section{Contempto}
\begin{itemize}
\item {Grp. gram.:m.}
\end{itemize}
\begin{itemize}
\item {Proveniência:(Lat. \textunderscore contemptus\textunderscore )}
\end{itemize}
Desprêzo.
\section{Contemptor}
\begin{itemize}
\item {Grp. gram.:adj.}
\end{itemize}
Desprezador:«\textunderscore algo contemptor dos homens.\textunderscore »Machado Assis, \textunderscore B. Cubas\textunderscore .
\section{Contenção}
\begin{itemize}
\item {Grp. gram.:f.}
\end{itemize}
\begin{itemize}
\item {Proveniência:(Lat. \textunderscore contentio\textunderscore )}
\end{itemize}
Acto de contender.
Esfôrço.
\section{Contenção}
\begin{itemize}
\item {Grp. gram.:f.}
\end{itemize}
\begin{itemize}
\item {Utilização:Bras}
\end{itemize}
Acto de conter.
\section{Contenças}
\begin{itemize}
\item {Grp. gram.:f. pl.}
\end{itemize}
\begin{itemize}
\item {Proveniência:(De \textunderscore conter\textunderscore )}
\end{itemize}
Móveis miúdos de casa.
\section{Contenciosamente}
\begin{itemize}
\item {Grp. gram.:adv.}
\end{itemize}
De modo contencioso.
\section{Contencioso}
\begin{itemize}
\item {Grp. gram.:adj.}
\end{itemize}
\begin{itemize}
\item {Grp. gram.:M.}
\end{itemize}
\begin{itemize}
\item {Proveniência:(Lat. \textunderscore contentiosus\textunderscore )}
\end{itemize}
Em que há contenção ou litigio.
Litigioso.
Duvidoso; incerto.
Jurisdicção contenciosa: \textunderscore o contencioso fiscal de 1.^a e de 2.^a instância\textunderscore .
\section{Contenda}
\begin{itemize}
\item {Grp. gram.:f.}
\end{itemize}
\begin{itemize}
\item {Proveniência:(De \textunderscore contender\textunderscore )}
\end{itemize}
Contenção.
Altercação.
Luta; combate.
Esfôrço.
\section{Contendedor}
\begin{itemize}
\item {Grp. gram.:m.  e  adj.}
\end{itemize}
Aquelle que contende.
\section{Contendente}
\begin{itemize}
\item {Grp. gram.:m.  e  adj.}
\end{itemize}
\begin{itemize}
\item {Proveniência:(Lat. \textunderscore contendens\textunderscore )}
\end{itemize}
O mesmo que \textunderscore contendedor\textunderscore .
\section{Contender}
\begin{itemize}
\item {Grp. gram.:v. i.}
\end{itemize}
\begin{itemize}
\item {Proveniência:(Lat. \textunderscore contendere\textunderscore )}
\end{itemize}
Brigar.
Pleitear, litigar.
Rivalizar.
Esforçar-se.
Oppor-se.
Dirigir provocação.
\section{Contendimento}
\begin{itemize}
\item {Grp. gram.:m.}
\end{itemize}
\begin{itemize}
\item {Utilização:Ant.}
\end{itemize}
Acto de contender.
Opposição; objecção.
\section{Contendor}
\begin{itemize}
\item {Grp. gram.:m.}
\end{itemize}
(Contr. de \textunderscore contendedor\textunderscore )
\section{Contenença}
\begin{itemize}
\item {Grp. gram.:f.}
\end{itemize}
\begin{itemize}
\item {Utilização:Ant.}
\end{itemize}
\begin{itemize}
\item {Proveniência:(Fr. \textunderscore contenance\textunderscore )}
\end{itemize}
Aspecto.
Semblante.
\section{Contenho}
\begin{itemize}
\item {Grp. gram.:m.}
\end{itemize}
\begin{itemize}
\item {Proveniência:(It. \textunderscore contegno\textunderscore )}
\end{itemize}
Aspecto, porte, garbo:«\textunderscore em postura e contenho de guerreira\textunderscore ». \textunderscore Caramuru\textunderscore , VI, 29.--Na 2.^a ed., os editores mudaram \textunderscore contenho\textunderscore  para \textunderscore contento\textunderscore !
\section{Contensão}
\begin{itemize}
\item {Grp. gram.:f.}
\end{itemize}
\begin{itemize}
\item {Proveniência:(De \textunderscore con...\textunderscore  + \textunderscore tensão\textunderscore )}
\end{itemize}
Grande applicação, grande esforço, para adquirir um conhecimento ou remover uma difficuldade.
\section{Contentadiço}
\begin{itemize}
\item {Grp. gram.:adj.}
\end{itemize}
Que se contenta facilmente. Cf. \textunderscore Panorama\textunderscore , VII, 1.
\section{Contentamento}
\begin{itemize}
\item {Grp. gram.:m.}
\end{itemize}
Acto ou effeito de contentar.
Alegria.
Satisfação.
\section{Contentar}
\begin{itemize}
\item {Grp. gram.:v. t.}
\end{itemize}
Tornar contente.
Dar prazer, satisfação, a.
Entreter, distrahir.
Tranquillizar.
\section{Contentável}
\begin{itemize}
\item {Grp. gram.:adj.}
\end{itemize}
Que se póde contentar.
\section{Contente}
\begin{itemize}
\item {Grp. gram.:adj.}
\end{itemize}
\begin{itemize}
\item {Proveniência:(Lat. \textunderscore contentus\textunderscore )}
\end{itemize}
Que está satisfeito; alegre.
Que causa satisfação.
\section{Contentemente}
\begin{itemize}
\item {Grp. gram.:adv.}
\end{itemize}
\begin{itemize}
\item {Proveniência:(De \textunderscore contente\textunderscore )}
\end{itemize}
Com contentamento.
\section{Contentivo}
\begin{itemize}
\item {Grp. gram.:adj.}
\end{itemize}
\begin{itemize}
\item {Utilização:Bras}
\end{itemize}
\begin{itemize}
\item {Proveniência:(Do lat. \textunderscore contentus\textunderscore )}
\end{itemize}
Diz-se de um apparelho para a reducção de fracturas.
\section{Contento}
\begin{itemize}
\item {Grp. gram.:m.}
\end{itemize}
\begin{itemize}
\item {Proveniência:(De \textunderscore contentar\textunderscore )}
\end{itemize}
O mesmo que \textunderscore contentamento\textunderscore .
\textunderscore A contento\textunderscore , para experiência: \textunderscore a criada ficou a contento\textunderscore .
\section{Contento}
\begin{itemize}
\item {Grp. gram.:adj.}
\end{itemize}
\begin{itemize}
\item {Utilização:Des.}
\end{itemize}
O mesmo que \textunderscore contente\textunderscore .
\section{Contento}
\begin{itemize}
\item {Grp. gram.:m.}
\end{itemize}
\begin{itemize}
\item {Proveniência:(Lat. \textunderscore contentus\textunderscore )}
\end{itemize}
O mesmo que \textunderscore conteúdo\textunderscore .
\section{Conter}
\begin{itemize}
\item {Grp. gram.:v. t.}
\end{itemize}
\begin{itemize}
\item {Proveniência:(Lat. \textunderscore continere\textunderscore )}
\end{itemize}
Abranger, incluir em si: \textunderscore Lisbôa contém 500.000 habitantes\textunderscore .
Refrear; reprimir: \textunderscore conter os impetos de alguém\textunderscore .
Reter unido.
\section{Contérmino}
\begin{itemize}
\item {Grp. gram.:adj.}
\end{itemize}
\begin{itemize}
\item {Grp. gram.:M.}
\end{itemize}
\begin{itemize}
\item {Proveniência:(Lat. \textunderscore conterminus\textunderscore )}
\end{itemize}
Que confina.
Adjacente.
Confim, raia.
\section{Conterrâneo}
\begin{itemize}
\item {Grp. gram.:m.  e  adj.}
\end{itemize}
\begin{itemize}
\item {Proveniência:(Lat. \textunderscore conterraneus\textunderscore )}
\end{itemize}
O que é da mesma terra.
Compatrício.
\section{Contérrito}
\begin{itemize}
\item {Grp. gram.:adj.}
\end{itemize}
\begin{itemize}
\item {Proveniência:(Lat. \textunderscore conterritus\textunderscore )}
\end{itemize}
Muito aterrado.
Espantado. Cf. Camillo, \textunderscore Sereia\textunderscore , 208.
\section{Contestabilidade}
\begin{itemize}
\item {Grp. gram.:f.}
\end{itemize}
Qualidade daquillo que é contestável.
\section{Contestação}
\begin{itemize}
\item {Grp. gram.:f.}
\end{itemize}
\begin{itemize}
\item {Proveniência:(Lat. \textunderscore contestatio\textunderscore )}
\end{itemize}
Acto de contestar.
\section{Contestador}
\begin{itemize}
\item {Grp. gram.:m.  e  adj.}
\end{itemize}
O mesmo que contestante.
\section{Contestante}
\begin{itemize}
\item {Grp. gram.:m.  e  adj.}
\end{itemize}
\begin{itemize}
\item {Proveniência:(Lat. \textunderscore contestans\textunderscore )}
\end{itemize}
O que contesta.
\section{Contestar}
\begin{itemize}
\item {Grp. gram.:v. t.}
\end{itemize}
\begin{itemize}
\item {Grp. gram.:V. i.}
\end{itemize}
\begin{itemize}
\item {Proveniência:(Lat. \textunderscore contestari\textunderscore )}
\end{itemize}
Provar com o testemunho de outrem.
Confirmar.
Contender.
Contradizer.
Negar: \textunderscore contestar uma affirmação\textunderscore .
Oppor-se.
Discutir.
Dizer como resposta, replicar.--Nesta accepção, parece castelhanismo inútil; entretanto, é abonado por este texto:«\textunderscore contestando ás interrogações do govêrno russo...\textunderscore »Latino,
\textunderscore Humboldt\textunderscore , 334.
\section{Contestável}
\begin{itemize}
\item {Grp. gram.:adj.}
\end{itemize}
Que se póde contestar.
\section{Conteste}
\begin{itemize}
\item {Grp. gram.:adj.}
\end{itemize}
\begin{itemize}
\item {Proveniência:(De \textunderscore contestar\textunderscore )}
\end{itemize}
Que testemunha ou que affirma o mesmo que outrem.
Que comprova.
\section{Contestemente}
\begin{itemize}
\item {Grp. gram.:adv.}
\end{itemize}
\begin{itemize}
\item {Proveniência:(De \textunderscore conteste\textunderscore )}
\end{itemize}
Identicamente.
Com depoimento igual.
\section{Conteúdo}
\begin{itemize}
\item {Grp. gram.:adj.}
\end{itemize}
\begin{itemize}
\item {Grp. gram.:M.}
\end{itemize}
\begin{itemize}
\item {Proveniência:(De \textunderscore conter\textunderscore )}
\end{itemize}
Contido: \textunderscore dinheiro, conteúdo na arca\textunderscore .
Aquillo que está contido ou encerrado em alguma coisa: \textunderscore o conteúdo de uma bôlsa\textunderscore .
\section{Contexto}
\begin{itemize}
\item {Grp. gram.:m.}
\end{itemize}
\begin{itemize}
\item {Proveniência:(Lat. \textunderscore contextus\textunderscore )}
\end{itemize}
Encadeamento das ideias de um escrito.
Contextura, composição.
\section{Contextuação}
\begin{itemize}
\item {Grp. gram.:f.}
\end{itemize}
Acto de contextuar.
\section{Contextuar}
\begin{itemize}
\item {Grp. gram.:v. t.}
\end{itemize}
\begin{itemize}
\item {Proveniência:(De \textunderscore contexto\textunderscore )}
\end{itemize}
Incluir ou intercalar num texto.
\section{Contextura}
\begin{itemize}
\item {Grp. gram.:f.}
\end{itemize}
\begin{itemize}
\item {Proveniência:(De \textunderscore contexto\textunderscore )}
\end{itemize}
Encadeamento, trama.
Ligação entre as partes de um todo; contexto.
\section{Contia}
\begin{itemize}
\item {Grp. gram.:f.}
\end{itemize}
\begin{itemize}
\item {Utilização:Ant.}
\end{itemize}
Retribuição, que os reis davam a certos cavalleiros, por serviços no paço ou na guerra.
(Fórma ant. de \textunderscore quantia\textunderscore )
\section{Contiado}
\begin{itemize}
\item {Grp. gram.:adj.}
\end{itemize}
\begin{itemize}
\item {Utilização:Ant.}
\end{itemize}
Que recebia contia.
\section{Contiçar}
\begin{itemize}
\item {Grp. gram.:v. i.}
\end{itemize}
\begin{itemize}
\item {Utilização:Prov.}
\end{itemize}
O mesmo que \textunderscore atiçar\textunderscore  ou \textunderscore açular\textunderscore .
\section{Contigo}
\begin{itemize}
\item {Grp. gram.:loc. pron.}
\end{itemize}
Na tua companhia: \textunderscore jantei contigo\textunderscore .
De ti para ti: \textunderscore estavas falando contigo\textunderscore .
(Flexão do pron. \textunderscore tu\textunderscore , precedido da prep. \textunderscore com\textunderscore )
\section{Contiguamente}
\begin{itemize}
\item {Grp. gram.:adv.}
\end{itemize}
De modo contiguo.
\section{Contiguar}
\begin{itemize}
\item {Grp. gram.:v. t.}
\end{itemize}
\begin{itemize}
\item {Proveniência:(De \textunderscore contíguo\textunderscore )}
\end{itemize}
Tornar contíguo, avizinhar.
\section{Contiguidade}
\begin{itemize}
\item {fónica:gu-i}
\end{itemize}
\begin{itemize}
\item {Grp. gram.:f.}
\end{itemize}
Estado daquillo que é contíguo.
\section{Contíguo}
\begin{itemize}
\item {Grp. gram.:adj.}
\end{itemize}
\begin{itemize}
\item {Proveniência:(Lat. \textunderscore contiguus\textunderscore )}
\end{itemize}
Que está em contacto.
Junto, próximo: \textunderscore dois prédios contíguos\textunderscore .
\section{Contilheira}
\begin{itemize}
\item {Grp. gram.:f.}
\end{itemize}
\begin{itemize}
\item {Proveniência:(De \textunderscore contilho\textunderscore , dem. hyp. de \textunderscore conto\textunderscore )}
\end{itemize}
Mulher que conta histórias; mulher mexeriqueira.
\section{Contina}
\begin{itemize}
\item {Grp. gram.:f.}
\end{itemize}
\begin{itemize}
\item {Utilização:Ant.}
\end{itemize}
(?):«\textunderscore não ha tornar atrás, Catrina, de tua contina\textunderscore ».
Sim. Machado, p. 86.
\section{Continão}
\begin{itemize}
\item {Grp. gram.:m.}
\end{itemize}
Espécie de antigo promotor de justiça, nos tribunaes chineses. Cf. \textunderscore Peregrinação\textunderscore , CIII.
\section{Continência}
\begin{itemize}
\item {Grp. gram.:f.}
\end{itemize}
\begin{itemize}
\item {Utilização:Ant.}
\end{itemize}
\begin{itemize}
\item {Proveniência:(Lat. \textunderscore continentia\textunderscore )}
\end{itemize}
Abstenção de prazeres sensuais.
Moderação.
Capacidade.
Cortesia militar.
Catadura, aspecto, o mesmo que \textunderscore contenença\textunderscore . Cf. Sousa, \textunderscore Vida do Arceb.\textunderscore , I, 296.
Preparativo, disposição. Cf. Filinto, \textunderscore D. Man.\textunderscore , III, 52.
\section{Continental}
\begin{itemize}
\item {Grp. gram.:adj.}
\end{itemize}
Relativo a continente.
\section{Continente}
\begin{itemize}
\item {Grp. gram.:m.}
\end{itemize}
\begin{itemize}
\item {Grp. gram.:Adj.}
\end{itemize}
\begin{itemize}
\item {Proveniência:(Lat. \textunderscore continens\textunderscore )}
\end{itemize}
Grande extensão de terra, sem interrupção de continuidade: \textunderscore o continente africano\textunderscore .
Aquillo que contém alguma coisa.
Que tem continência.
Moderado.
Que contém alguma coisa.
\section{Contingência}
\begin{itemize}
\item {Grp. gram.:f.}
\end{itemize}
\begin{itemize}
\item {Proveniência:(Lat. \textunderscore contingentia\textunderscore )}
\end{itemize}
Eventualidade, qualidade daquillo que é contingente.
\section{Contingente}
\begin{itemize}
\item {Grp. gram.:adj.}
\end{itemize}
\begin{itemize}
\item {Grp. gram.:M.}
\end{itemize}
\begin{itemize}
\item {Proveniência:(Lat. \textunderscore contingens\textunderscore )}
\end{itemize}
Eventual.
Que póde ou não succeder.
Duvidoso, incerto.
Que, entre muitos indivíduos, compete a cada um.
Quota.
Porção de homens, que certa circunscripção territorial tem que dar para o sorteio militar ou para alguns serviços públicos.
Aquillo que é eventual.
\section{Contingentemente}
\begin{itemize}
\item {Grp. gram.:adv.}
\end{itemize}
Eventualmente, de modo contingente.
\section{Contino}
\begin{itemize}
\item {Grp. gram.:adj.}
\end{itemize}
\begin{itemize}
\item {Utilização:Des.}
\end{itemize}
\begin{itemize}
\item {Grp. gram.:Loc. adv.}
\end{itemize}
O mesmo que \textunderscore contínuo\textunderscore .
\textunderscore De contino\textunderscore , continuamente. Cf. \textunderscore Lusíadas\textunderscore .
\section{Continuação}
\begin{itemize}
\item {Grp. gram.:f.}
\end{itemize}
\begin{itemize}
\item {Proveniência:(Lat. \textunderscore continuatio\textunderscore )}
\end{itemize}
Acto de continuar.
Duração; prolongamento.
\section{Continuadamente}
\begin{itemize}
\item {Grp. gram.:adv.}
\end{itemize}
\begin{itemize}
\item {Proveniência:(De \textunderscore continuado\textunderscore )}
\end{itemize}
O mesmo que \textunderscore continuamente\textunderscore .
\section{Continuador}
\begin{itemize}
\item {Grp. gram.:m.  e  adj.}
\end{itemize}
O que continua.
\section{Continuamente}
\begin{itemize}
\item {Grp. gram.:adv.}
\end{itemize}
De modo contínuo.
\section{Continuamento}
\begin{itemize}
\item {Grp. gram.:m.}
\end{itemize}
O mesmo que \textunderscore continuação\textunderscore .
\section{Continuar}
\begin{itemize}
\item {Grp. gram.:v. t.}
\end{itemize}
\begin{itemize}
\item {Grp. gram.:V. i.}
\end{itemize}
\begin{itemize}
\item {Proveniência:(Lat. \textunderscore continuare\textunderscore )}
\end{itemize}
Prolongar.
Não interromper: \textunderscore continuar a jornada\textunderscore .
Vir depois de.
Durar, proseguir, prolongar-se.
Não soffrer mudança: \textunderscore a noite continuava serena\textunderscore .
\section{Continuidade}
\begin{itemize}
\item {Grp. gram.:f.}
\end{itemize}
\begin{itemize}
\item {Proveniência:(Lat. \textunderscore continuitas\textunderscore )}
\end{itemize}
Qualidade daquillo que é contínuo.
\section{Contínuo}
\begin{itemize}
\item {Grp. gram.:adj.}
\end{itemize}
\begin{itemize}
\item {Grp. gram.:M.}
\end{itemize}
\begin{itemize}
\item {Grp. gram.:Loc. adv.}
\end{itemize}
\begin{itemize}
\item {Proveniência:(Lat. \textunderscore continuus\textunderscore )}
\end{itemize}
Em que não há interrupção: \textunderscore trabalho contínuo\textunderscore .
Empregado subalterno, que serve a uma repartição pública ou particular: \textunderscore os contínuos da minha Repartição\textunderscore .
\textunderscore De contínuo\textunderscore , immediatamente.
Continuamente, constantemente.
\section{Contista}
\begin{itemize}
\item {Grp. gram.:m.}
\end{itemize}
Autor de contos.
\section{Conto}
\begin{itemize}
\item {Grp. gram.:m.}
\end{itemize}
\begin{itemize}
\item {Utilização:Ant.}
\end{itemize}
\begin{itemize}
\item {Grp. gram.:Pl.}
\end{itemize}
\begin{itemize}
\item {Proveniência:(Lat. \textunderscore computus\textunderscore , de \textunderscore computare\textunderscore . Com a significação de narrativa, vem do lat. \textunderscore comentum\textunderscore , segundo Castilho)}
\end{itemize}
Número.
Dez vezes cem mil (reis).
Vinte dúzias de (ovos).
Narrativa.
Historieta.
Fábula.
Mil vezes mil coisas ou pessôas:«\textunderscore concorre a ella tanta gente, que se affirma que passa de tres contos de pessôas\textunderscore ». \textunderscore Peregrinação\textunderscore , CVIII.
Intrigas, embustes.
Antigos bens, doados por graça real, mas em que não entravam os mordomos reaes.
\section{Conto}
\begin{itemize}
\item {Grp. gram.:m.}
\end{itemize}
\begin{itemize}
\item {Proveniência:(Lat. \textunderscore contus\textunderscore )}
\end{itemize}
Extremidade inferior da lança ou do bastão.
Remate globular do canhão.
\section{Conto}
\begin{itemize}
\item {Grp. gram.:m.}
\end{itemize}
(?):«\textunderscore os de fora acabaram sua cava e puseram grande parte do muro em contos... derom fogo á cava... Durando por bem espaço (o combate), arderam os contos... e cairom delle (muro) bem dezoito braços em torrões...\textunderscore »Fernam Lopes, \textunderscore Chrón. de D. Fern.\textunderscore , cap. XL.
\section{Contoada}
\begin{itemize}
\item {Grp. gram.:f.}
\end{itemize}
\begin{itemize}
\item {Proveniência:(De \textunderscore conto\textunderscore ^2)}
\end{itemize}
Pancada com o conto de bastão ou lança.
\section{Conto-de-pão}
\begin{itemize}
\item {Grp. gram.:m.}
\end{itemize}
\begin{itemize}
\item {Utilização:Prov.}
\end{itemize}
\begin{itemize}
\item {Utilização:alent.}
\end{itemize}
Vinte pães.
\section{Contorção}
\begin{itemize}
\item {Grp. gram.:f.}
\end{itemize}
\begin{itemize}
\item {Proveniência:(Lat. \textunderscore contortio\textunderscore )}
\end{itemize}
Acto de contorcer.
Contracção de músculos.
Movimento irregular e violento.
Posição forçada, desagradável, incômmoda.
\section{Contorcer}
\begin{itemize}
\item {Grp. gram.:v. t.}
\end{itemize}
\begin{itemize}
\item {Proveniência:(Do lat. \textunderscore contorquere\textunderscore )}
\end{itemize}
Torcer muito.
Dobrar.
\section{Contornamento}
\begin{itemize}
\item {Grp. gram.:m.}
\end{itemize}
Acto de contornar.
\section{Contornar}
\begin{itemize}
\item {Grp. gram.:v. t.}
\end{itemize}
\begin{itemize}
\item {Proveniência:(De \textunderscore contôrno\textunderscore )}
\end{itemize}
Fazer o contôrno de.
Andar em volta de: \textunderscore contornar uma praça\textunderscore .
Estender-se em roda de.
Tornar redondo.
Ladear: \textunderscore contornar uma estrada\textunderscore .
\section{Contornear}
\begin{itemize}
\item {Grp. gram.:v. t.}
\end{itemize}
(V.contornar)
\section{Contôrno}
\begin{itemize}
\item {Grp. gram.:m.}
\end{itemize}
\begin{itemize}
\item {Utilização:Fig.}
\end{itemize}
\begin{itemize}
\item {Proveniência:(De \textunderscore com...\textunderscore  + \textunderscore tôrno\textunderscore )}
\end{itemize}
Linha, que fecha ou limita um corpo.
Circuito.
Peripheria.
Linha, que determina os relevos.
O arredondado de certas fórmas.
Arredondamento, elegância, relevo, no estilo.
\section{Contorsão}
\begin{itemize}
\item {Grp. gram.:f.}
\end{itemize}
(V.contorção)
\section{Contortas}
\begin{itemize}
\item {Grp. gram.:f. pl.}
\end{itemize}
\begin{itemize}
\item {Proveniência:(Lat. \textunderscore contortus\textunderscore )}
\end{itemize}
Ordem de plantas, de corollas monopétalas, torcidas na orla.
\section{Contra}
\begin{itemize}
\item {Grp. gram.:prep.}
\end{itemize}
\begin{itemize}
\item {Grp. gram.:Adv.}
\end{itemize}
\begin{itemize}
\item {Grp. gram.:M.}
\end{itemize}
\begin{itemize}
\item {Proveniência:(Lat. \textunderscore contra\textunderscore )}
\end{itemize}
Em opposição a: \textunderscore votação contra o Govêrno\textunderscore .
Em direcção opposta á de.
Defronte, em frente de:«\textunderscore Trento é uma cidade na raia da Allemanha contra Italia\textunderscore ». Sousa, \textunderscore Vida do Arceb.\textunderscore , I, 209.
Em contacto com: \textunderscore encostou-se contra o muro\textunderscore .
Apesar de.
Em contradicção com: \textunderscore proceder contra o seu próprio systema\textunderscore .
Em troca de.
Em desfavor de: \textunderscore questão resolvida contra mim\textunderscore .
Contrariamente: \textunderscore uns votaram a favor, outros votaram contra\textunderscore .
Objecção, réplica.
Obstáculo: \textunderscore isso tem um contra\textunderscore .
\section{Contra}
\begin{itemize}
\item {Grp. gram.:m.}
\end{itemize}
\begin{itemize}
\item {Utilização:Bras}
\end{itemize}
Contraveneno.
\section{Contra}
\begin{itemize}
\item {Grp. gram.:f.}
\end{itemize}
Baracha ou travessão, nos talhos das marinhas do Guadiana.
(Provavelmente relaciona-se com \textunderscore contra\textunderscore ^1)
\section{Contra...}
\begin{itemize}
\item {Grp. gram.:pref.}
\end{itemize}
(que indica opposição, refôrço, proximidade, etc.)
\section{Contra-abertura}
\begin{itemize}
\item {Grp. gram.:f.}
\end{itemize}
Abertura, em ponto ou sentido opposto a outra.
\section{Contra-almeida}
\begin{itemize}
\item {Grp. gram.:f.}
\end{itemize}
Parte da embarcação, entre a barra de almeida e o parapeito das janelas da câmara.
\section{Contra-almirante}
\begin{itemize}
\item {Grp. gram.:m.}
\end{itemize}
Official da armada, de patente inferior immediatamente á de vice-almirante.
\section{Contra-amura}
\begin{itemize}
\item {Grp. gram.:f.}
\end{itemize}
\begin{itemize}
\item {Utilização:Náut.}
\end{itemize}
Cabo, com que se facilitam as manobras da amura.
\section{Contra-arcada}
\begin{itemize}
\item {Grp. gram.:f.}
\end{itemize}
Conjunto de enxilhares de uma arcada fingida.
\section{Contra-arco}
\begin{itemize}
\item {Grp. gram.:m.}
\end{itemize}
Parte da quilha de um navio, debaixo da mastreação.
\section{Contra-arco-pendural}
\begin{itemize}
\item {Grp. gram.:m.}
\end{itemize}
Recorte ao longo de um arco, pela parte inferior, usado dantes em architectura.
\section{Contra-arminhos}
\begin{itemize}
\item {Grp. gram.:m.}
\end{itemize}
Nome, que, em Heráldica, se dá ao campo negro com salpicos brancos.
\section{Contra-asa}
\begin{itemize}
\item {Grp. gram.:f.}
\end{itemize}
Peça, que se colloca na parte superior da asa de um regador, para o reforçar.
\section{Contra-assembleia}
\begin{itemize}
\item {Grp. gram.:f.}
\end{itemize}
Assembleia, formada em opposição a outra.
\section{Contra-ataque}
\begin{itemize}
\item {Grp. gram.:m.}
\end{itemize}
Trincheira de fortificação.
\section{Contra-aviso}
\begin{itemize}
\item {Grp. gram.:m.}
\end{itemize}
Aviso que se dá, em sentido contrário ao de outro.
\section{Contrabaixista}
\begin{itemize}
\item {Grp. gram.:m.}
\end{itemize}
Tocador de contrabaixo.
\section{Contrabaixo}
\begin{itemize}
\item {Grp. gram.:m.}
\end{itemize}
\begin{itemize}
\item {Proveniência:(De \textunderscore contra...\textunderscore  + \textunderscore baixo\textunderscore )}
\end{itemize}
Voz mais grave que a do baixo.
Cantor, que tem essa voz.
Rabecão de três ou quatro cordas, que substitue ou acompanha a voz do contrabaixo.
O mesmo que \textunderscore contrabaixista\textunderscore .
Registo de órgão, no gênero mais grave das frautas.
\section{Contrabalançar}
\begin{itemize}
\item {Grp. gram.:v. t.}
\end{itemize}
\begin{itemize}
\item {Proveniência:(De \textunderscore contra...\textunderscore  + \textunderscore balançar\textunderscore )}
\end{itemize}
Equilibrar.
Contrapesar.
Compensar.
\section{Contrabaluarte}
\begin{itemize}
\item {Grp. gram.:f.}
\end{itemize}
\begin{itemize}
\item {Proveniência:(De \textunderscore contra...\textunderscore  + \textunderscore baluarte\textunderscore )}
\end{itemize}
Baluarte de refôrço, atrás de outro.
\section{Contrabanda}
\begin{itemize}
\item {Grp. gram.:f.}
\end{itemize}
\begin{itemize}
\item {Utilização:Heráld.}
\end{itemize}
\begin{itemize}
\item {Proveniência:(De \textunderscore contra...\textunderscore  + \textunderscore banda\textunderscore )}
\end{itemize}
Peça no escudo, da direita para a esquerda.
\section{Contrabandado}
\begin{itemize}
\item {Grp. gram.:m.}
\end{itemize}
\begin{itemize}
\item {Utilização:Heráld.}
\end{itemize}
O mesmo que \textunderscore barrado\textunderscore ^1.
\section{Contrabandear}
\begin{itemize}
\item {Grp. gram.:v. i.}
\end{itemize}
Fazer contrabando.
\section{Contrabandista}
\begin{itemize}
\item {Grp. gram.:m.  e  f.}
\end{itemize}
\begin{itemize}
\item {Utilização:Prov.}
\end{itemize}
\begin{itemize}
\item {Utilização:beir.}
\end{itemize}
\begin{itemize}
\item {Grp. gram.:F. pl.}
\end{itemize}
\begin{itemize}
\item {Utilização:Prov.}
\end{itemize}
\begin{itemize}
\item {Proveniência:(De \textunderscore contrabando\textunderscore )}
\end{itemize}
Pessôa, que faz contrabando.
Pessôa, que vende quinquilharias pelas ruas; bufarinheiro.
Vendedor ambulante de fazendas e lençaria.
Jôgo de rapazes.
\section{Contrabando}
\begin{itemize}
\item {Grp. gram.:m.}
\end{itemize}
\begin{itemize}
\item {Utilização:Fam.}
\end{itemize}
\begin{itemize}
\item {Proveniência:(De \textunderscore contra...\textunderscore  + \textunderscore bando\textunderscore )}
\end{itemize}
Commércio prohibido.
Introducção clandestina de mercadorias, sem que estas paguem os direitos a que estão sujeitas.
Acto máu, praticado occultamente.
Gente de má nota.
\section{Contrabataria}
\begin{itemize}
\item {Grp. gram.:f.}
\end{itemize}
\begin{itemize}
\item {Proveniência:(De \textunderscore contra...\textunderscore  + \textunderscore bataria\textunderscore )}
\end{itemize}
Bataria oposta a outra.
\section{Contrabater}
\begin{itemize}
\item {Grp. gram.:v. t.}
\end{itemize}
\begin{itemize}
\item {Proveniência:(De \textunderscore contra...\textunderscore  + \textunderscore bater\textunderscore )}
\end{itemize}
Atacar com a contrabataria.
\section{Contrabateria}
\begin{itemize}
\item {Grp. gram.:f.}
\end{itemize}
\begin{itemize}
\item {Proveniência:(De \textunderscore contra...\textunderscore  + \textunderscore bateria\textunderscore )}
\end{itemize}
Bateria opposta a outra.
\section{Contrabico}
\begin{itemize}
\item {Grp. gram.:m.}
\end{itemize}
\begin{itemize}
\item {Proveniência:(De \textunderscore contra...\textunderscore  + \textunderscore bico\textunderscore )}
\end{itemize}
Extremidade superior do bico de certos vasos de latão, a qual fórma ângulo com a outra extremidade.
\section{Contraboça}
\begin{itemize}
\item {Grp. gram.:f.}
\end{itemize}
Corrente, com que se reforça a bóça.
\section{Contrabordo}
\begin{itemize}
\item {Grp. gram.:m.}
\end{itemize}
\begin{itemize}
\item {Proveniência:(De \textunderscore contra...\textunderscore  + \textunderscore bordo\textunderscore )}
\end{itemize}
Resguardo ou fôrro da querena do navio.
\section{Contra-braço}
\begin{itemize}
\item {Grp. gram.:M.}
\end{itemize}
\begin{itemize}
\item {Utilização:Náut.}
\end{itemize}
Cabo, que reforça um braço.
\section{Contracadaste}
\begin{itemize}
\item {Grp. gram.:m.}
\end{itemize}
Peça de navio, com que se cobre o cadaste.
\section{Contracaixilho}
\begin{itemize}
\item {Grp. gram.:m.}
\end{itemize}
Caixilho com pano, collocado por fóra de outro, para o proteger da neve.
Caixilho com vidro, collocado por dentro de outro, para attenuar a luz.
\section{Contracalimba}
\begin{itemize}
\item {Grp. gram.:f.}
\end{itemize}
\begin{itemize}
\item {Proveniência:(De \textunderscore contra...\textunderscore  + \textunderscore calimba\textunderscore )}
\end{itemize}
Segunda rêde do saco, no apparelho da xávega.
\section{Contracambiar}
\begin{itemize}
\item {Grp. gram.:v. t.}
\end{itemize}
\begin{itemize}
\item {Proveniência:(De \textunderscore contra...\textunderscore  + \textunderscore cambiar\textunderscore )}
\end{itemize}
Pagar mal.
Corresponder mal a um favor.
Trocar.
\section{Contracâmbio}
\begin{itemize}
\item {Grp. gram.:m.}
\end{itemize}
Acto de contracambiar.
\section{Contracanto}
\begin{itemize}
\item {Grp. gram.:m.}
\end{itemize}
\begin{itemize}
\item {Utilização:Mús.}
\end{itemize}
Melodia accessória, que serve de acompanhamento a outra, principal.
\section{Contracarril}
\begin{itemize}
\item {Grp. gram.:m.}
\end{itemize}
Carril, que, nas vias férréas, se assenta ao lado dos carris ordinários, para os resguardar e para evitar descarrilamentos.
\section{Contracção}
\begin{itemize}
\item {Grp. gram.:f.}
\end{itemize}
\begin{itemize}
\item {Utilização:Gram.}
\end{itemize}
\begin{itemize}
\item {Proveniência:(Lat. \textunderscore contractio\textunderscore )}
\end{itemize}
Acto ou effeito de contrahir.
Compressão das moléculas de um corpo.
Retrahimento (de músculos).
Reducção de mais de uma sýllaba a uma só.
\section{Contracédula}
\begin{itemize}
\item {Grp. gram.:f.}
\end{itemize}
Cédula, que revoga outra.
\section{Contrachefe}
\begin{itemize}
\item {Grp. gram.:m.}
\end{itemize}
\begin{itemize}
\item {Utilização:Heráld.}
\end{itemize}
\begin{itemize}
\item {Proveniência:(De \textunderscore contra...\textunderscore  + \textunderscore chefe\textunderscore )}
\end{itemize}
Nona peça honrosa ordinária, na parte inferior do escudo, de que, ordinariamente, occupa um terço.
\section{Contracifra}
\begin{itemize}
\item {Grp. gram.:f.}
\end{itemize}
\begin{itemize}
\item {Proveniência:(De \textunderscore contra...\textunderscore  + \textunderscore cifra\textunderscore )}
\end{itemize}
Chave, com que se decifra um escrito enigmático.
\section{Contracoiceiro}
\begin{itemize}
\item {Grp. gram.:m.}
\end{itemize}
\begin{itemize}
\item {Utilização:T. da Nazareth}
\end{itemize}
Um dos homens que trabalham no levantamento das redes, e substitue ou ajuda o coiceiro.
\section{Contracorrente}
\begin{itemize}
\item {Grp. gram.:f.}
\end{itemize}
Corrente, opposta a outra.
\section{Contracosta}
\begin{itemize}
\item {Grp. gram.:f.}
\end{itemize}
\begin{itemize}
\item {Proveniência:(De \textunderscore contra...\textunderscore  + \textunderscore costa\textunderscore )}
\end{itemize}
Costa de mar, opposta a outra, no mesmo continente ou na mesma ilha.
\section{Contracoticado}
\begin{itemize}
\item {Grp. gram.:adj.}
\end{itemize}
\begin{itemize}
\item {Proveniência:(De \textunderscore contra...\textunderscore  + \textunderscore coticado\textunderscore )}
\end{itemize}
Que tem a cotica lançada da esquerda para a direita.
\section{Contráctil}
\begin{itemize}
\item {Grp. gram.:adj.}
\end{itemize}
\begin{itemize}
\item {Proveniência:(Do lat. \textunderscore contractus\textunderscore )}
\end{itemize}
Susceptível de contracção.
Que se contrai facilmente.
\section{Contractilidade}
\begin{itemize}
\item {Grp. gram.:f.}
\end{itemize}
Qualidade do que é contráctil.
\section{Contractivo}
\begin{itemize}
\item {Grp. gram.:adj.}
\end{itemize}
\begin{itemize}
\item {Proveniência:(Do lat. \textunderscore contractus\textunderscore )}
\end{itemize}
Que faz encolher.
\section{Contractura}
\begin{itemize}
\item {Grp. gram.:f.}
\end{itemize}
\begin{itemize}
\item {Utilização:Neol.}
\end{itemize}
\begin{itemize}
\item {Proveniência:(Lat. \textunderscore contractura\textunderscore )}
\end{itemize}
Qualidade do que está contrahido.
Effeito de contrahir.
\section{Contracunhar}
\begin{itemize}
\item {Grp. gram.:v. t.}
\end{itemize}
\begin{itemize}
\item {Proveniência:(De \textunderscore contra...\textunderscore  + \textunderscore cunhar\textunderscore )}
\end{itemize}
Cunhar novamente.
\section{Contracurva}
\begin{itemize}
\item {Grp. gram.:f.}
\end{itemize}
\begin{itemize}
\item {Utilização:Mathem.}
\end{itemize}
\begin{itemize}
\item {Proveniência:(De \textunderscore contra...\textunderscore  + \textunderscore curva\textunderscore )}
\end{itemize}
Curva, que termina um arco, tomando direcção opposta á dêste.
\section{Contradança}
\begin{itemize}
\item {Grp. gram.:f.}
\end{itemize}
\begin{itemize}
\item {Proveniência:(De \textunderscore contra...\textunderscore  + \textunderscore dança\textunderscore )}
\end{itemize}
Dança de quatro ou mais pares, defrontando uns com os outros.
Música, com que se acompanha essa dança.
Mudança frequente de lugar.
Alterações successivas.
\section{Contradançar}
\begin{itemize}
\item {Grp. gram.:v. i.}
\end{itemize}
Dançar contradanças.
\section{Contradeclaração}
\begin{itemize}
\item {Grp. gram.:f.}
\end{itemize}
Declaração, em sentido opposto ao de outra.
\section{Contradescarga}
\begin{itemize}
\item {Grp. gram.:f.}
\end{itemize}
\begin{itemize}
\item {Utilização:Phýs.}
\end{itemize}
Fulminação indirecta, quando uma nuvem descarrega a electricidade noutra nuvem, interrompendo-se a sua acção e havendo recombinação violenta das electricidades dos objectos terrestres, que por isso são despedaçados. Cf. F. Lapa, \textunderscore Phýs. e Chím.\textunderscore , I. 107.
\section{Contradição}
\begin{itemize}
\item {Grp. gram.:f.}
\end{itemize}
\begin{itemize}
\item {Proveniência:(Lat. \textunderscore contradictio\textunderscore )}
\end{itemize}
Acto de contradizer.
Oposíção.
Objecção.
\section{Contradicção}
\begin{itemize}
\item {Grp. gram.:f.}
\end{itemize}
\begin{itemize}
\item {Proveniência:(Lat. \textunderscore contradictio\textunderscore )}
\end{itemize}
Acto de contradizer.
Opposíção.
Objecção.
\section{Contra-dique}
\begin{itemize}
\item {Grp. gram.:m.}
\end{itemize}
Dique, que reforça outro.
Construcção, que reforça um dique.
\section{Contradistinguir}
\begin{itemize}
\item {Grp. gram.:v. t.}
\end{itemize}
\begin{itemize}
\item {Proveniência:(De \textunderscore contra...\textunderscore  + \textunderscore distinguir\textunderscore )}
\end{itemize}
Mostrar a differença entre (duas coisas):«\textunderscore para se contradistinguir a quiete, de que falamos...\textunderscore »\textunderscore Luz e Calor\textunderscore , 182.
\section{Contradita}
\begin{itemize}
\item {Grp. gram.:f.}
\end{itemize}
\begin{itemize}
\item {Proveniência:(De \textunderscore contradito\textunderscore )}
\end{itemize}
Allegação forense, apresentada por um dos litigantes contra outro.
Opposição, por meio de uma testemunha, ao depoimento de outra.
Testemunha, que contradiz outra.
Contestação.
\section{Contraditar}
\begin{itemize}
\item {Grp. gram.:v. t.}
\end{itemize}
\begin{itemize}
\item {Proveniência:(De \textunderscore contradita\textunderscore )}
\end{itemize}
Oppor contradita a.
Contestar.
\section{Contraditável}
\begin{itemize}
\item {Grp. gram.:adj.}
\end{itemize}
Que se póde contraditar:«\textunderscore argumentos contraditáveis\textunderscore ». Camillo, \textunderscore Hist. e Sentim.\textunderscore , 63.
\section{Contradito}
\textunderscore part. irr.\textunderscore  de \textunderscore contradizer\textunderscore .
\section{Contraditor}
\begin{itemize}
\item {Grp. gram.:m.  e  adj.}
\end{itemize}
\begin{itemize}
\item {Proveniência:(Lat. \textunderscore contradictor\textunderscore )}
\end{itemize}
O que contradiz.
Que oppõe contradita.
\section{Contraditória}
\begin{itemize}
\item {Grp. gram.:f.}
\end{itemize}
\begin{itemize}
\item {Proveniência:(De \textunderscore contraditório\textunderscore )}
\end{itemize}
Proposição, opposta a outra.
\section{Contraditoriamente}
\begin{itemize}
\item {Grp. gram.:adv.}
\end{itemize}
Em sentido contrário.
\section{Contraditório}
\begin{itemize}
\item {Grp. gram.:adj.}
\end{itemize}
\begin{itemize}
\item {Proveniência:(Lat. \textunderscore contradictorius\textunderscore )}
\end{itemize}
Que envolve contradicção: \textunderscore allegações contraditórias\textunderscore .
\section{Contradizer}
\begin{itemize}
\item {Grp. gram.:v. t.}
\end{itemize}
\begin{itemize}
\item {Grp. gram.:V. i.}
\end{itemize}
\begin{itemize}
\item {Proveniência:(Lat. \textunderscore contradicere\textunderscore )}
\end{itemize}
Dizer o contrário de.
Contestar.
Allegar o contrário.
Fazer opposição.
\section{Contradizimento}
\begin{itemize}
\item {Grp. gram.:m.}
\end{itemize}
\begin{itemize}
\item {Utilização:Des.}
\end{itemize}
\begin{itemize}
\item {Proveniência:(De \textunderscore contradizer\textunderscore )}
\end{itemize}
O mesmo que \textunderscore contradicção\textunderscore .
\section{Contradormentes}
\begin{itemize}
\item {Grp. gram.:m. pl.}
\end{itemize}
Pranchões ou dormentes, que reforçam outros.
\section{Contra-édito}
\begin{itemize}
\item {Grp. gram.:m.}
\end{itemize}
Édito, contrário a outro.
\section{Contra-embuscada}
\begin{itemize}
\item {Grp. gram.:f.}
\end{itemize}
Embuscada, que se faz contra outra.
\section{Contraente}
\begin{itemize}
\item {Grp. gram.:m.  e  adj.}
\end{itemize}
\begin{itemize}
\item {Proveniência:(Lat. \textunderscore contrahens\textunderscore )}
\end{itemize}
O que contrai.
\section{Contra-erva}
\begin{itemize}
\item {Grp. gram.:f.}
\end{itemize}
Espécie de raíz, que serve de antídoto.
\section{Contra-escarpa}
\begin{itemize}
\item {Grp. gram.:f.}
\end{itemize}
Talude de fôsso fronteiro á escarpa.
\section{Contra-escota}
\begin{itemize}
\item {Grp. gram.:f.}
\end{itemize}
\begin{itemize}
\item {Utilização:Náut.}
\end{itemize}
Cabo, com que se facilitam as manobras da escota.
\section{Contra-escritura}
\begin{itemize}
\item {Grp. gram.:f.}
\end{itemize}
Revogação secreta de uma escritura pública.
\section{Contra-estais}
\begin{itemize}
\item {Grp. gram.:m. pl.}
\end{itemize}
\begin{itemize}
\item {Utilização:Náut.}
\end{itemize}
Cabos, que reforçam os estais.
\section{Contra-estimulante}
\begin{itemize}
\item {Grp. gram.:m.  e  adj.}
\end{itemize}
O que contra-estimula.
\section{Contra-estimular}
\begin{itemize}
\item {Grp. gram.:v. t.}
\end{itemize}
Combater o estado ou excesso de estimulação em.
\section{Contra-estímulo}
\begin{itemize}
\item {Grp. gram.:m.}
\end{itemize}
Estado opposto ao do estímulo.
\section{Contrafacção}
\begin{itemize}
\item {Grp. gram.:f.}
\end{itemize}
\begin{itemize}
\item {Proveniência:(Lat. \textunderscore contrafactio\textunderscore )}
\end{itemize}
Acto de contrafazer.
\section{Contrafactor}
\begin{itemize}
\item {Grp. gram.:m.}
\end{itemize}
\begin{itemize}
\item {Proveniência:(Do lat. \textunderscore contra\textunderscore  + \textunderscore factor\textunderscore )}
\end{itemize}
Aquelle que contrafaz.
\section{Contrafaixa}
\begin{itemize}
\item {Grp. gram.:f.}
\end{itemize}
\begin{itemize}
\item {Utilização:Heráld.}
\end{itemize}
\begin{itemize}
\item {Proveniência:(De \textunderscore contra...\textunderscore  + \textunderscore faixa\textunderscore )}
\end{itemize}
Faixa, dividida em duas, de differente esmalte, nos brasões.
\section{Contrafaixado}
\begin{itemize}
\item {Grp. gram.:adj.}
\end{itemize}
Que tem contrafaixa.
\section{Contrafazedor}
\begin{itemize}
\item {Grp. gram.:m.}
\end{itemize}
\begin{itemize}
\item {Proveniência:(De \textunderscore contrafazer\textunderscore )}
\end{itemize}
Aquelle que imita ou arremeda.
\section{Contrafazer}
\begin{itemize}
\item {Grp. gram.:v. t.}
\end{itemize}
\begin{itemize}
\item {Proveniência:(Lat. \textunderscore contrafacere\textunderscore )}
\end{itemize}
Imitar.
Disfarçar.
Reproduzir, imitando.
Imitar, falsificando.
Constranger, reprimir.
Mascarar.
\section{Contrafé}
\begin{itemize}
\item {Grp. gram.:f.}
\end{itemize}
\begin{itemize}
\item {Proveniência:(De \textunderscore contra...\textunderscore  + \textunderscore fé\textunderscore )}
\end{itemize}
Cópia authêntica de intimação ou citação judicial, que é entregue á pessôa intimada ou citada.
\section{Contrafeição}
\begin{itemize}
\item {Grp. gram.:f.}
\end{itemize}
Acto de contrafazer; estado de contrafeito; contrafacção. Cf. Garrett, \textunderscore Camões\textunderscore , p. IV.
\section{Contrafeitiço}
\begin{itemize}
\item {Grp. gram.:m.}
\end{itemize}
Feitiço, opposto a outro.
\section{Contrafeito}
\begin{itemize}
\item {Grp. gram.:adj.}
\end{itemize}
\begin{itemize}
\item {Proveniência:(Do lat. \textunderscore contrafactus\textunderscore )}
\end{itemize}
Constrangido.
Que não está á vontade: \textunderscore neste lugar, sinto-me contrafeito\textunderscore .
\section{Contra-feito}
\begin{itemize}
\item {Grp. gram.:m.}
\end{itemize}
\begin{itemize}
\item {Utilização:Carp.}
\end{itemize}
Viga, pregada na extremidade mais baixa dos caibros, para suavizar a inclinação do telhado sobre a sanca.
Moldura, que se não faz na própria peça, mas á parte, pregando-se depois aquella peça.
\section{Contrafileira}
\begin{itemize}
\item {Grp. gram.:f.}
\end{itemize}
Fileira, atrás de outra.
Peça de madeira, que escora obliquamente o madeiramento do telhado.
\section{Contrafio}
\begin{itemize}
\item {Grp. gram.:m.}
\end{itemize}
\begin{itemize}
\item {Utilização:Ant.}
\end{itemize}
Armadilha, ou talvez parte de uma armadilha. Cf. Bernardes, \textunderscore Luz e Calor\textunderscore , 20.
\section{Contrafixa}
\begin{itemize}
\item {Grp. gram.:f.}
\end{itemize}
\begin{itemize}
\item {Proveniência:(De \textunderscore contra...\textunderscore  + \textunderscore fixa\textunderscore )}
\end{itemize}
Peça de madeira, collocada obliquamente, para escorar o madeiramento do telhado.
Contrafileira.
\section{Contrafixo}
\begin{itemize}
\item {Grp. gram.:m.}
\end{itemize}
\begin{itemize}
\item {Proveniência:(De \textunderscore contra...\textunderscore  + \textunderscore fixo\textunderscore )}
\end{itemize}
Pequena chapa metállica, que forra o orifício, em que gira um eixo de ferro.
\section{Contrafloreado}
\begin{itemize}
\item {Grp. gram.:adj.}
\end{itemize}
\begin{itemize}
\item {Utilização:Heráld.}
\end{itemize}
\begin{itemize}
\item {Proveniência:(De \textunderscore contra...\textunderscore  + \textunderscore floreado\textunderscore )}
\end{itemize}
Diz-se do escudo, cujos florões são oppostos e alternos.
\section{Contraforte}
\begin{itemize}
\item {Grp. gram.:m.}
\end{itemize}
\begin{itemize}
\item {Proveniência:(De \textunderscore contra...\textunderscore  + \textunderscore forte\textunderscore )}
\end{itemize}
Fôrro, que reforça o calçado, na parte em que se assenta o calcanhar.
Peça de estôfo, com que se reforça outra.
Reparo, edificação, com que se reforça uma muralha.
Ligação das alhetas com o cadaste, nos navios.
Pilar, que reforça uma parede, para sustentar uma abóbada, um terraço, etc.
\section{Contrafosso}
\begin{itemize}
\item {fónica:fô}
\end{itemize}
\begin{itemize}
\item {Grp. gram.:m.}
\end{itemize}
\begin{itemize}
\item {Proveniência:(De \textunderscore contra...\textunderscore  + \textunderscore fôsso\textunderscore )}
\end{itemize}
Fôsso, a par de outro.
\section{Contrafuga}
\begin{itemize}
\item {Grp. gram.:f.}
\end{itemize}
\begin{itemize}
\item {Proveniência:(De \textunderscore contra...\textunderscore  + \textunderscore fuga\textunderscore )}
\end{itemize}
Fuga musical, em sentido contrário á natural.
\section{Contrafundo}
\begin{itemize}
\item {Grp. gram.:adv.}
\end{itemize}
\begin{itemize}
\item {Proveniência:(De \textunderscore contra...\textunderscore  + \textunderscore fundo\textunderscore )}
\end{itemize}
Para baixo.
\section{Contraguarda}
\begin{itemize}
\item {Grp. gram.:f.}
\end{itemize}
\begin{itemize}
\item {Proveniência:(De \textunderscore contra\textunderscore  + \textunderscore guarda\textunderscore )}
\end{itemize}
Edificação angular, que reforça um baluarte.
\section{Contrahente}
\begin{itemize}
\item {Grp. gram.:m.  e  adj.}
\end{itemize}
\begin{itemize}
\item {Proveniência:(Lat. \textunderscore contrahens\textunderscore )}
\end{itemize}
O que contrahi.
\section{Contrahir}
\begin{itemize}
\item {Grp. gram.:v. t.}
\end{itemize}
\begin{itemize}
\item {Proveniência:(Lat. \textunderscore contrahere\textunderscore )}
\end{itemize}
Tornar apertado, estreito.
Contratar.
Tomar sôbre si: \textunderscore contrahir obrigações\textunderscore .
Adquirir.
Assumir.
Fazer.
\section{Contrahivel}
\begin{itemize}
\item {Grp. gram.:adj.}
\end{itemize}
Que se póde contrahir.
\section{Contra-indicação}
\begin{itemize}
\item {Grp. gram.:f.}
\end{itemize}
Acto de contra-indicar.
\section{Contra-indicar}
\begin{itemize}
\item {Grp. gram.:v. t.}
\end{itemize}
Indicar, ao contrário de.
\section{Contrair}
\begin{itemize}
\item {Grp. gram.:v. t.}
\end{itemize}
\begin{itemize}
\item {Proveniência:(Lat. \textunderscore contrahere\textunderscore )}
\end{itemize}
Tornar apertado, estreito.
Contratar.
Tomar sôbre si: \textunderscore contrair obrigações\textunderscore .
Adquirir.
Assumir.
Fazer.
\section{Contrairo}
\begin{itemize}
\item {Grp. gram.:adj.}
\end{itemize}
\begin{itemize}
\item {Utilização:Des.}
\end{itemize}
O mesmo que \textunderscore contrário\textunderscore .
\section{Contraível}
\begin{itemize}
\item {Grp. gram.:adj.}
\end{itemize}
Que se póde contrair.
\section{Contralais}
\begin{itemize}
\item {Grp. gram.:m. pl.}
\end{itemize}
\begin{itemize}
\item {Utilização:Náut.}
\end{itemize}
\begin{itemize}
\item {Proveniência:(De \textunderscore contra...\textunderscore  + \textunderscore lais\textunderscore )}
\end{itemize}
Cabo, com que se reforçam lais.
\section{Contraliga}
\begin{itemize}
\item {Grp. gram.:f.}
\end{itemize}
\begin{itemize}
\item {Proveniência:(De \textunderscore contra...\textunderscore  + \textunderscore liga\textunderscore )}
\end{itemize}
Liga, que se fórma contra outra.
\section{Contralto}
\begin{itemize}
\item {Grp. gram.:m.}
\end{itemize}
\begin{itemize}
\item {Proveniência:(T. it.)}
\end{itemize}
A mais grave voz feminina, opposta ao soprano.
Mulher que tem essa voz.
\section{Contraluz}
\begin{itemize}
\item {Grp. gram.:m.}
\end{itemize}
Lugar, opposto àquelle em que a luz dá em cheio.
Luz, que dá num quadro, em sentido opposto àquelle em que foi pintado.
\section{Contramalha}
\begin{itemize}
\item {Grp. gram.:f.}
\end{itemize}
\begin{itemize}
\item {Proveniência:(De \textunderscore contra...\textunderscore  + \textunderscore malha\textunderscore )}
\end{itemize}
Malha, que reforça outra.
\section{Contramalhado}
\begin{itemize}
\item {Grp. gram.:adj.}
\end{itemize}
Que tem contramalha.
\section{Contramandado}
\begin{itemize}
\item {Grp. gram.:m.}
\end{itemize}
\begin{itemize}
\item {Proveniência:(De \textunderscore contra...\textunderscore  + \textunderscore mandado\textunderscore )}
\end{itemize}
Mandado, opposto a outro já dado.
\section{Contramandar}
\begin{itemize}
\item {Grp. gram.:v. i.}
\end{itemize}
\begin{itemize}
\item {Proveniência:(De \textunderscore contra...\textunderscore  + \textunderscore mandar\textunderscore )}
\end{itemize}
Dar ordens oppostas a (outras que se tinham dado).
\section{Contramangas}
\begin{itemize}
\item {Grp. gram.:f. pl.}
\end{itemize}
Segundas mangas, no mesmo vestuário, largas e compridas.
\section{Contramarca}
\begin{itemize}
\item {Grp. gram.:f.}
\end{itemize}
\begin{itemize}
\item {Proveniência:(De \textunderscore contra...\textunderscore  + \textunderscore marca\textunderscore )}
\end{itemize}
Marca, para substituir ou para authenticar outra.
Senha de theatro.
\section{Contramarcar}
\begin{itemize}
\item {Grp. gram.:v. t.}
\end{itemize}
Pôr contramarca em.
\section{Contramarcha}
\begin{itemize}
\item {Grp. gram.:f.}
\end{itemize}
\begin{itemize}
\item {Proveniência:(De \textunderscore contra...\textunderscore  + \textunderscore marcha\textunderscore )}
\end{itemize}
Marcha, em sentido opposto ao da que se fazia.
\section{Contramarchar}
\begin{itemize}
\item {Grp. gram.:v. i.}
\end{itemize}
Fazer contramarcha.
\section{Contramaré}
\begin{itemize}
\item {Grp. gram.:m.}
\end{itemize}
\begin{itemize}
\item {Proveniência:(De \textunderscore contra...\textunderscore  + \textunderscore maré\textunderscore )}
\end{itemize}
Maré, contrária á maré ordinária.
\section{Contramargem}
\begin{itemize}
\item {Grp. gram.:f.}
\end{itemize}
Faixa de terreno, annexa á margem: \textunderscore «areias que se espalhavão por uma contramargem.»\textunderscore  F. Manuel.
\section{Contramartelos}
\begin{itemize}
\item {Grp. gram.:m. pl.}
\end{itemize}
Pequenas peças de madeira, no maquinismo dos pianos, para conservar suspensos os martelos, emquanto as teclas estão sob a pressão dos dedos do tocador.
\section{Contramérico}
\begin{itemize}
\item {Grp. gram.:adj.}
\end{itemize}
Relativo ao contrâmero.
\section{Contrâmero}
\begin{itemize}
\item {Grp. gram.:m.}
\end{itemize}
\begin{itemize}
\item {Proveniência:(De \textunderscore contra...\textunderscore  + gr. \textunderscore meros\textunderscore )}
\end{itemize}
Cada uma das metades do corpo, considerando-se êste dividido por um plano vertical transversal.
\section{Contramestre}
\begin{itemize}
\item {Grp. gram.:m.}
\end{itemize}
\begin{itemize}
\item {Proveniência:(De \textunderscore contra...\textunderscore  + \textunderscore mestre\textunderscore )}
\end{itemize}
Empregado de navio, immediatamente inferior ao mestre.
Artífice, que substitue o mestre.
\section{Contramezena}
\begin{itemize}
\item {Grp. gram.:f.}
\end{itemize}
Mastro, opposto ao de mezena.
\section{Contramina}
\begin{itemize}
\item {Grp. gram.:f.}
\end{itemize}
\begin{itemize}
\item {Utilização:Fig.}
\end{itemize}
\begin{itemize}
\item {Proveniência:(De \textunderscore contra...\textunderscore  + \textunderscore mina\textunderscore )}
\end{itemize}
Mina, por onde se procura a do inimigo.
Artifício, para inutilizar uma intriga, um ardil.
\section{Contraminar}
\begin{itemize}
\item {Grp. gram.:v. t.}
\end{itemize}
Frustrar, inutilizar por meio de contramina.
\section{Contramoldagem}
\begin{itemize}
\item {Grp. gram.:f.}
\end{itemize}
\begin{itemize}
\item {Proveniência:(De \textunderscore contra...\textunderscore  + \textunderscore moldagem\textunderscore )}
\end{itemize}
Acto de reproduzir pela moldagem.
\section{Contramoldar}
\begin{itemize}
\item {Grp. gram.:v. t.}
\end{itemize}
\begin{itemize}
\item {Proveniência:(De \textunderscore contra...\textunderscore  + \textunderscore moldar\textunderscore )}
\end{itemize}
Reproduzir por moldagem.
\section{Contramolde}
\begin{itemize}
\item {Grp. gram.:m.}
\end{itemize}
\begin{itemize}
\item {Proveniência:(De \textunderscore contra...\textunderscore  + \textunderscore molde\textunderscore )}
\end{itemize}
Desenho ou fórma invertida do objecto que se procura reproduzir.
\section{Contramuralha}
\begin{itemize}
\item {Grp. gram.:f.}
\end{itemize}
(V.contramuro)
\section{Contramurar}
\begin{itemize}
\item {Grp. gram.:v. t.}
\end{itemize}
Guarnecer com contramuro.
\section{Contramuro}
\begin{itemize}
\item {Grp. gram.:m.}
\end{itemize}
\begin{itemize}
\item {Proveniência:(De \textunderscore contra...\textunderscore  + \textunderscore muro\textunderscore )}
\end{itemize}
Muro, que reforça outro e o póde substituir.
\section{Contranatura}
\begin{itemize}
\item {Grp. gram.:adv.}
\end{itemize}
Contra as leis da natureza. Cf. Garrett, \textunderscore Port. na Balança\textunderscore , 238.
\section{Contranatural}
\begin{itemize}
\item {Grp. gram.:adj.}
\end{itemize}
\begin{itemize}
\item {Proveniência:(De \textunderscore contra...\textunderscore  + \textunderscore natural\textunderscore )}
\end{itemize}
Opposto á natureza.
\section{Contra-oitava}
\begin{itemize}
\item {Grp. gram.:f.}
\end{itemize}
\begin{itemize}
\item {Utilização:Mús.}
\end{itemize}
A oitava, inferior a outra.
\section{Contra-ordem}
\begin{itemize}
\item {Grp. gram.:f.}
\end{itemize}
O mesmo que \textunderscore contramandado\textunderscore .
\section{Contra-ordenar}
\begin{itemize}
\item {Grp. gram.:v. t.}
\end{itemize}
Ordenar, em sentido contrário àquelle em que já se tinham dado ordens.
\section{Contrapantes}
\begin{itemize}
\item {Grp. gram.:adj. pl.}
\end{itemize}
(V.contra-rapantes)
\section{Contraparente}
\begin{itemize}
\item {Grp. gram.:m.}
\end{itemize}
Parente remoto.
Affim.
\section{Contraparentesco}
\begin{itemize}
\item {Grp. gram.:m.}
\end{itemize}
Estado do contraparente.
\section{Contrapassantes}
\begin{itemize}
\item {Grp. gram.:adj.}
\end{itemize}
\begin{itemize}
\item {Utilização:Heráld.}
\end{itemize}
\begin{itemize}
\item {Proveniência:(De \textunderscore contra...\textunderscore  + \textunderscore passante\textunderscore )}
\end{itemize}
Diz-se de dois animaes, representados um sôbre o outro, em direcção opposta.
\section{Contrapasso}
\begin{itemize}
\item {Grp. gram.:m.}
\end{itemize}
Passo, opposto a outro.
\section{Contrapatarraz}
\begin{itemize}
\item {Grp. gram.:m.}
\end{itemize}
\begin{itemize}
\item {Proveniência:(De \textunderscore contra...\textunderscore  + \textunderscore patarraz\textunderscore )}
\end{itemize}
Corrente de ferro, semelhante ao patarraz, fixa nas amuras e dirigida á respectiva chapa do gurupés.
\section{Contrapé}
\begin{itemize}
\item {Grp. gram.:m.}
\end{itemize}
Apoio, esteio:«\textunderscore correu..., toma um penhasco, faz delle contrapé\textunderscore ». Castilho, \textunderscore Metam.\textunderscore , 214.
\section{Contra-peçonha}
\begin{itemize}
\item {Grp. gram.:f.}
\end{itemize}
\begin{itemize}
\item {Utilização:Pop.}
\end{itemize}
\begin{itemize}
\item {Utilização:Ant.}
\end{itemize}
\begin{itemize}
\item {Proveniência:(De \textunderscore contra...\textunderscore  + \textunderscore peçonha\textunderscore )}
\end{itemize}
Contraveneno.
Antídoto.
Designação de uma planta medicinal da fam. das asclepiadáceas. Cf. \textunderscore Desengano da Med.\textunderscore , 35.
\section{Contrapêlo}
\begin{itemize}
\item {Grp. gram.:m.}
\end{itemize}
\begin{itemize}
\item {Proveniência:(De \textunderscore contra...\textunderscore  + \textunderscore pêlo\textunderscore )}
\end{itemize}
Revés do pêlo.
Direcção, opposta á inclinação natural do pêlo.
\section{Contrapesar}
\begin{itemize}
\item {Grp. gram.:v. t.}
\end{itemize}
Contrabalançar.
Equilibrar com contrapeso.
\section{Contrapêso}
\begin{itemize}
\item {Grp. gram.:m.}
\end{itemize}
\begin{itemize}
\item {Utilização:Fig.}
\end{itemize}
\begin{itemize}
\item {Proveniência:(De \textunderscore contra...\textunderscore  + \textunderscore pêso\textunderscore )}
\end{itemize}
Pêso addicional que, collocado num dos pratos da balança, equilibra este com o outro prato.
Pequena porção de uma mercância, com que se perfaz o pêso da que se pretende.
Maromba.
Aquillo, que recompensa ou contrabalança alguma coisa.
\section{Contrapilastra}
\begin{itemize}
\item {Grp. gram.:f.}
\end{itemize}
\begin{itemize}
\item {Proveniência:(De \textunderscore contra...\textunderscore  + \textunderscore pilastra\textunderscore )}
\end{itemize}
Pilastra, em frente de outra.
\section{Contrapinásio}
\begin{itemize}
\item {Grp. gram.:m.}
\end{itemize}
\begin{itemize}
\item {Utilização:Carp.}
\end{itemize}
\begin{itemize}
\item {Proveniência:(De \textunderscore contra...\textunderscore  + \textunderscore pinásio\textunderscore )}
\end{itemize}
Travessa, no alto e no fundo da porta, parallela e igual ao pinásio.
\section{Contrapisa}
\begin{itemize}
\item {Grp. gram.:m.}
\end{itemize}
\begin{itemize}
\item {Utilização:Prov.}
\end{itemize}
\begin{itemize}
\item {Utilização:alg.}
\end{itemize}
\begin{itemize}
\item {Proveniência:(De \textunderscore contra...\textunderscore  + \textunderscore pisar\textunderscore )}
\end{itemize}
Guarda-lama nos vestidos.
\section{Contrapontado}
\begin{itemize}
\item {Grp. gram.:adj.}
\end{itemize}
\begin{itemize}
\item {Proveniência:(De \textunderscore contra...\textunderscore  + \textunderscore ponta\textunderscore )}
\end{itemize}
Diz-se do escudo que tem as pontas oppostas, umas ás outras.
\section{Contrapontear}
\begin{itemize}
\item {Grp. gram.:v. t.}
\end{itemize}
\begin{itemize}
\item {Utilização:Bras. do S}
\end{itemize}
\begin{itemize}
\item {Proveniência:(De \textunderscore contraponto\textunderscore )}
\end{itemize}
Instrumentar.
Pôr em contraponto.
Contrariar.
Contraditar.
\section{Contrapontista}
\begin{itemize}
\item {Grp. gram.:m.  e  f.}
\end{itemize}
Pessôa, versada em regras de contraponto.
\section{Contrapontístico}
\begin{itemize}
\item {Grp. gram.:adj.}
\end{itemize}
Relativo a contrapontista ou a contraponto.
\section{Contraponto}
\begin{itemize}
\item {Grp. gram.:m.}
\end{itemize}
Arte de juntar uma ou mais partes melódicas a um canto dado.
Composição musical, em que se observaram as regras daquella arte.
(B. lat. \textunderscore contrapunctus\textunderscore )
\section{Contrapor}
\begin{itemize}
\item {Grp. gram.:v. t.}
\end{itemize}
\begin{itemize}
\item {Proveniência:(Lat. \textunderscore contraponere\textunderscore )}
\end{itemize}
Pôr em frente.
Oppor.
Confrontar.
Expor parallelamente ou em sentido contrário.
\section{Contraposição}
\begin{itemize}
\item {Grp. gram.:f.}
\end{itemize}
\begin{itemize}
\item {Proveniência:(Lat. \textunderscore contrapositio\textunderscore )}
\end{itemize}
Acto ou effeito de contrapor.
\section{Contraposto}
\textunderscore part.\textunderscore  de \textunderscore contrapor\textunderscore .
\section{Contraproducente}
\begin{itemize}
\item {Grp. gram.:adj.}
\end{itemize}
\begin{itemize}
\item {Proveniência:(Do lat. \textunderscore contra\textunderscore  + \textunderscore producens\textunderscore )}
\end{itemize}
Que prova o contrário do que se pretende demonstrar.
\section{Contraprova}
\begin{itemize}
\item {Grp. gram.:f.}
\end{itemize}
\begin{itemize}
\item {Proveniência:(De \textunderscore contra...\textunderscore  + \textunderscore prova\textunderscore )}
\end{itemize}
Impugnação jurídica de um libello.
Desenho ou estampa, que se obtém, assentando um papel sôbre outro desenho a lápis ou sôbre uma prova fresca.
Segunda prova typográphica.
\section{Contraprovar}
\begin{itemize}
\item {Grp. gram.:v. t.}
\end{itemize}
Fazer a contraprova de.
Comparar (nova prova typográphica, com a anterior).
\section{Contrapunho}
\begin{itemize}
\item {Grp. gram.:m.}
\end{itemize}
\begin{itemize}
\item {Utilização:Náut.}
\end{itemize}
\begin{itemize}
\item {Proveniência:(De \textunderscore contra...\textunderscore  + \textunderscore punho\textunderscore )}
\end{itemize}
Cabo, ligado á ponta da vela grande e do traquete, para auxiliar a manobra.
\section{Contraquarteado}
\begin{itemize}
\item {Grp. gram.:adj.}
\end{itemize}
\begin{itemize}
\item {Utilização:Heráld.}
\end{itemize}
\begin{itemize}
\item {Proveniência:(De \textunderscore contra...\textunderscore  + \textunderscore quarteado\textunderscore )}
\end{itemize}
Diz-se do brasão, que tem os quartéis divididos em quatro partes.
\section{Contraquartel}
\begin{itemize}
\item {Grp. gram.:m.}
\end{itemize}
\begin{itemize}
\item {Utilização:Heráld.}
\end{itemize}
\begin{itemize}
\item {Proveniência:(De \textunderscore contra...\textunderscore  + \textunderscore quartel\textunderscore )}
\end{itemize}
Cada uma das quatro divisões de cada quartel do escudo.
\section{Contraquilha}
\begin{itemize}
\item {Grp. gram.:f.}
\end{itemize}
Peça de madeira, que reveste a quilha por dentro do navio.
\section{Contra-rapantes}
\begin{itemize}
\item {Grp. gram.:adj.}
\end{itemize}
Diz-se, em Heráldica, dos animaes rapantes, voltados um contra o outro.
\section{Contra-regra}
\begin{itemize}
\item {Grp. gram.:m.}
\end{itemize}
Aquelle que marca a entrada dos actores em scena.
\section{Contra-reparo}
\begin{itemize}
\item {Grp. gram.:m.}
\end{itemize}
Segunda trincheira em volta de uma praça de guerra.
\section{Contra-réplica}
\begin{itemize}
\item {Grp. gram.:f.}
\end{itemize}
O mesmo que \textunderscore tréplica\textunderscore .
\section{Contra-resposta}
\begin{itemize}
\item {Grp. gram.:f.}
\end{itemize}
\begin{itemize}
\item {Utilização:Mús.}
\end{itemize}
Resposta ao contra-sujeito de uma fuga.
\section{Contra-retábulo}
\begin{itemize}
\item {Grp. gram.:m.}
\end{itemize}
Fundo, na decoração de um altar, para quadro ou baixo relêvo.
\section{Contra-revolução}
\begin{itemize}
\item {Grp. gram.:f.}
\end{itemize}
Revolução contrária a outra.
\section{Contrari}
\begin{itemize}
\item {Grp. gram.:m.}
\end{itemize}
Espécie de tecido antigo.
\section{Contrariador}
\begin{itemize}
\item {Grp. gram.:m.  e  adj.}
\end{itemize}
O que contraría.
\section{Contráriamente}
\begin{itemize}
\item {Grp. gram.:adv.}
\end{itemize}
De modo \textunderscore contrário\textunderscore .
\section{Contrariante}
\begin{itemize}
\item {Grp. gram.:adj.}
\end{itemize}
Que contraría.
\section{Contrariar}
\begin{itemize}
\item {Grp. gram.:v. t.}
\end{itemize}
\begin{itemize}
\item {Proveniência:(De \textunderscore contrário\textunderscore )}
\end{itemize}
Fazer opposição a.
Dizer, fazer, querer o contrário de: \textunderscore não fazes senão contrariar-me\textunderscore .
Embaraçar.
Contestar.
\section{Contrariável}
\begin{itemize}
\item {Grp. gram.:adj.}
\end{itemize}
Que se póde contrariar; discutivel.
\section{Contrariedade}
\begin{itemize}
\item {Grp. gram.:f.}
\end{itemize}
\begin{itemize}
\item {Proveniência:(Lat. \textunderscore contrarietas\textunderscore )}
\end{itemize}
Estado de coisas, reciprocamente contrárias.
Obstáculo.
Contratempo.
Contestação jurídica de um libello.
\section{Contrário}
\begin{itemize}
\item {Grp. gram.:adj.}
\end{itemize}
\begin{itemize}
\item {Grp. gram.:M.}
\end{itemize}
\begin{itemize}
\item {Grp. gram.:Loc. adv.}
\end{itemize}
\begin{itemize}
\item {Proveniência:(Lat. \textunderscore contrarius\textunderscore )}
\end{itemize}
Opposto.
Inverso: \textunderscore virar um tecido do lado contrário\textunderscore .
Que tem direcção opposta: \textunderscore navegar com vento contrário\textunderscore .
Essencialmente diverso.
Prejudicial: \textunderscore correr contrário o tempo\textunderscore .
Desfavorável.
Aquelle que é opposto.
Adversário: \textunderscore os contrários patearam-no\textunderscore .
\textunderscore Do contrário\textunderscore , \textunderscore pelo contrário\textunderscore , ou \textunderscore ao contrário\textunderscore , contrariamente, de modo diverso.
\section{Contra-roda}
\begin{itemize}
\item {Grp. gram.:f.}
\end{itemize}
\begin{itemize}
\item {Utilização:Náut.}
\end{itemize}
Roda interna (da prôa), \textunderscore contra-roda da pôpa\textunderscore , cadaste falso.
\section{Contrarrestar}
\begin{itemize}
\item {Grp. gram.:v. t.}
\end{itemize}
\begin{itemize}
\item {Proveniência:(De \textunderscore contra...\textunderscore  + \textunderscore arrestar\textunderscore )}
\end{itemize}
Decidir em contrário de. Cf. Latino, \textunderscore Elogios\textunderscore , 149; A. Cândido, \textunderscore Philos. Pol.\textunderscore , 113.
\section{Contrarrotura}
\begin{itemize}
\item {Grp. gram.:f.}
\end{itemize}
Rotura, em sentido ou lugar opposto ao de outra.
\section{Contraruptura}
\begin{itemize}
\item {Grp. gram.:f.}
\end{itemize}
Ruptura, em sentido ou lugar opposto ao de outra.
\section{Contrasellar}
\begin{itemize}
\item {Grp. gram.:v. t.}
\end{itemize}
Pôr contra-sello em.
\section{Contrasêllo}
\begin{itemize}
\item {Grp. gram.:m.}
\end{itemize}
Sêllo, que se põe ao lado ou em cima de outro.
\section{Contrasenha}
\begin{itemize}
\item {Grp. gram.:f.}
\end{itemize}
Palavra ou sinal, que se junta a uma senha ou a outro sinal.
\section{Contrasenso}
\begin{itemize}
\item {fónica:sen}
\end{itemize}
\begin{itemize}
\item {Grp. gram.:m.}
\end{itemize}
\begin{itemize}
\item {Proveniência:(De \textunderscore contra...\textunderscore  + \textunderscore senso\textunderscore )}
\end{itemize}
Dito ou acto contrário ao bom senso.
Disparate.
\section{Contra-senso}
\begin{itemize}
\item {Grp. gram.:m.}
\end{itemize}
\begin{itemize}
\item {Proveniência:(De \textunderscore contra...\textunderscore  + \textunderscore senso\textunderscore )}
\end{itemize}
Dito ou acto contrário ao bom senso.
Disparate.
\section{Contrassinal}
\begin{itemize}
\item {Grp. gram.:m.}
\end{itemize}
Contra-senha; disfarce.
\section{Contrastar}
\begin{itemize}
\item {Grp. gram.:v. t.}
\end{itemize}
\begin{itemize}
\item {Grp. gram.:V. i.}
\end{itemize}
\begin{itemize}
\item {Proveniência:(Do lat. \textunderscore contra\textunderscore  + \textunderscore stare\textunderscore )}
\end{itemize}
Fazer opposição a.
Lutar com.
Arrostar.
Examinar, conhecer os quilates de.
Estar em opposição, divergir essencialmente, fazer contraste.
\section{Contrastaria}
\begin{itemize}
\item {Grp. gram.:f.}
\end{itemize}
\begin{itemize}
\item {Proveniência:(De \textunderscore contraste\textunderscore )}
\end{itemize}
Profissão de quem contrasta metaes preciosos.
Estabelecimento, em que se exerce essa profissão.
\section{Contrastável}
\begin{itemize}
\item {Grp. gram.:adj.}
\end{itemize}
Que se póde contrastar.
\section{Contraste}
\begin{itemize}
\item {Grp. gram.:m.}
\end{itemize}
\begin{itemize}
\item {Proveniência:(De \textunderscore contrastar\textunderscore )}
\end{itemize}
Opposição entre coisas ou pessôas, uma das quaes faz sobresair a outra.
Verificação.
Aquelle, que avalia os quilates dos metaes preciosos e o preço das jóias.
Embate, esgarrão:«\textunderscore em altura de vinte e seis graos lhe deo hum rijo contraste de Noroeste\textunderscore ». \textunderscore Peregrinação\textunderscore , LXI.
\section{Contrasteação}
\begin{itemize}
\item {Grp. gram.:f.}
\end{itemize}
Acto de contrastear:«\textunderscore sincera contrasteação dos quilates do julgado\textunderscore ». Castilho, \textunderscore Livr. Cláss.\textunderscore , VII, 93.
\section{Contrastear}
\begin{itemize}
\item {Grp. gram.:v. t.}
\end{itemize}
\begin{itemize}
\item {Proveniência:(De \textunderscore contraste\textunderscore )}
\end{itemize}
Avaliar (os quilates de metaes preciosos).
\section{Contra-sujeito}
\begin{itemize}
\item {Grp. gram.:m.}
\end{itemize}
\begin{itemize}
\item {Utilização:Mús.}
\end{itemize}
Contraponto, dobrado á oitava, com que muitas vezes se acompanha o sujeito de uma fuga.
\section{Contrata}
\begin{itemize}
\item {Grp. gram.:f.}
\end{itemize}
\begin{itemize}
\item {Utilização:Pop.}
\end{itemize}
\begin{itemize}
\item {Proveniência:(De \textunderscore contratar\textunderscore )}
\end{itemize}
Ajuste, que alguém faz de serviços temporários.
Contrato.
\section{Contratação}
\begin{itemize}
\item {Grp. gram.:f.}
\end{itemize}
\begin{itemize}
\item {Utilização:Des.}
\end{itemize}
Acto de contratar.
\section{Contratador}
\begin{itemize}
\item {Grp. gram.:m.}
\end{itemize}
\begin{itemize}
\item {Proveniência:(De \textunderscore contratar\textunderscore )}
\end{itemize}
Aquelle que contrata.
\section{Contratalho}
\begin{itemize}
\item {Grp. gram.:m.}
\end{itemize}
\begin{itemize}
\item {Proveniência:(De \textunderscore contra...\textunderscore  + \textunderscore talho\textunderscore )}
\end{itemize}
Em gravura, talho cruzado com outro ou outros.
\section{Contratante}
\begin{itemize}
\item {Grp. gram.:adj.}
\end{itemize}
\begin{itemize}
\item {Grp. gram.:M.}
\end{itemize}
\begin{itemize}
\item {Proveniência:(De \textunderscore contratar\textunderscore )}
\end{itemize}
Que contrata.
Que faz um tratado.
Contratador.
\section{Contratar}
\begin{itemize}
\item {Grp. gram.:v. t.}
\end{itemize}
\begin{itemize}
\item {Grp. gram.:V. i.}
\end{itemize}
\begin{itemize}
\item {Proveniência:(De \textunderscore contrato\textunderscore )}
\end{itemize}
Combinar, ajustar, convencionar.
Adquirir por contrato: \textunderscore contratar um serviçal\textunderscore .
Fazer negócio.
\section{Contratável}
\begin{itemize}
\item {Grp. gram.:adj.}
\end{itemize}
Que se póde contratar.
\section{Contratejo}
\begin{itemize}
\item {Grp. gram.:m.}
\end{itemize}
\begin{itemize}
\item {Utilização:Marn.}
\end{itemize}
Um dos compartimentos das marinhas de Faro. Cf. \textunderscore Museu Techn.\textunderscore , 107.
\section{Contratelar}
\begin{itemize}
\item {Grp. gram.:v. t.}
\end{itemize}
\begin{itemize}
\item {Proveniência:(De \textunderscore contra...\textunderscore  + \textunderscore tela\textunderscore )}
\end{itemize}
Reforçar, forrar, com pano (uma tela).
\section{Contratema}
\begin{itemize}
\item {Grp. gram.:m.}
\end{itemize}
\begin{itemize}
\item {Utilização:Mús.}
\end{itemize}
\begin{itemize}
\item {Proveniência:(De \textunderscore contra...\textunderscore  + \textunderscore thema\textunderscore )}
\end{itemize}
Contraponto, que se escreve sôbre um tema qualquer.
O mesmo que \textunderscore contra-sujeito\textunderscore .
\section{Contratempo}
\begin{itemize}
\item {Grp. gram.:m.}
\end{itemize}
\begin{itemize}
\item {Proveniência:(De \textunderscore contra...\textunderscore  + \textunderscore tempo\textunderscore )}
\end{itemize}
Accidente imprevisto.
Obstáculo.
Contrariedade.
Antigo passo de dança.
Compasso musical, apoiado em tempos fracos.
\section{Contraterraço}
\begin{itemize}
\item {Grp. gram.:m.}
\end{itemize}
Terraço construído ao lado de outro.
\section{Contrathema}
\begin{itemize}
\item {Grp. gram.:m.}
\end{itemize}
\begin{itemize}
\item {Utilização:Mús.}
\end{itemize}
\begin{itemize}
\item {Proveniência:(De \textunderscore contra...\textunderscore  + \textunderscore thema\textunderscore )}
\end{itemize}
Contraponto, que se escreve sôbre um thema qualquer.
O mesmo que \textunderscore contra-sujeito\textunderscore .
\section{Contrato}
\begin{itemize}
\item {Grp. gram.:m.}
\end{itemize}
\begin{itemize}
\item {Grp. gram.:Adj.}
\end{itemize}
\begin{itemize}
\item {Utilização:Des.}
\end{itemize}
\begin{itemize}
\item {Proveniência:(Lat. \textunderscore contractus\textunderscore )}
\end{itemize}
Acto ou effeito de contratar.
Convenção.
Acôrdo, em que duas ou mais pessôas transferem entre si algum direito ou se sujeitam a alguma obrigação.
Promessa acceita.
Contrahido.
\section{Contratorpedeiro}
\begin{itemize}
\item {Grp. gram.:m.}
\end{itemize}
Navio destruidor de torpedos.
\section{Contratual}
\begin{itemize}
\item {Grp. gram.:adj.}
\end{itemize}
\begin{itemize}
\item {Utilização:Jur.}
\end{itemize}
Relativo a contrato; que tem as formalidades de contrato.
\section{Contravalação}
\begin{itemize}
\item {Grp. gram.:f.}
\end{itemize}
\begin{itemize}
\item {Proveniência:(De \textunderscore contravalar\textunderscore )}
\end{itemize}
Fôsso, com parapeito, para impedir as sortidas dos sitiados.
\section{Contravalar}
\begin{itemize}
\item {Grp. gram.:v. t.}
\end{itemize}
\begin{itemize}
\item {Proveniência:(De \textunderscore contra...\textunderscore  + \textunderscore vallar\textunderscore )}
\end{itemize}
Guarnecer, fortificar com contravalação.
\section{Contravallação}
\begin{itemize}
\item {Grp. gram.:f.}
\end{itemize}
\begin{itemize}
\item {Proveniência:(De \textunderscore contravallar\textunderscore )}
\end{itemize}
Fôsso, com parapeito, para impedir as sortidas dos sitiados.
\section{Contravallar}
\begin{itemize}
\item {Grp. gram.:v. t.}
\end{itemize}
\begin{itemize}
\item {Proveniência:(De \textunderscore contra...\textunderscore  + \textunderscore vallar\textunderscore )}
\end{itemize}
Guarnecer, fortificar com contravallação.
\section{Contraveia}
\begin{itemize}
\item {Grp. gram.:adv.}
\end{itemize}
Ao invés; de arrepio.
Contra a maré. Cf. Filinto, I, 283.
\section{Contraveiro}
\begin{itemize}
\item {Grp. gram.:m.}
\end{itemize}
\begin{itemize}
\item {Utilização:Heráld.}
\end{itemize}
\begin{itemize}
\item {Proveniência:(De \textunderscore contra...\textunderscore  + \textunderscore veiro\textunderscore )}
\end{itemize}
Guarnição de escudo, em que se oppõe metal a metal e côr a côr.
\section{Contravenção}
\begin{itemize}
\item {Grp. gram.:f.}
\end{itemize}
Acto de contravir.
\section{Contraveneno}
\begin{itemize}
\item {Grp. gram.:m.}
\end{itemize}
\begin{itemize}
\item {Proveniência:(De \textunderscore contra...\textunderscore  + \textunderscore veneno\textunderscore )}
\end{itemize}
Substância, medicamento, que frustra a acção do veneno.
Antídoto.
\section{Contraveniente}
\begin{itemize}
\item {Grp. gram.:m.  e  adj.}
\end{itemize}
O mesmo que \textunderscore contraventor\textunderscore .
\section{Contravento}
\begin{itemize}
\item {Grp. gram.:m.}
\end{itemize}
\begin{itemize}
\item {Proveniência:(De \textunderscore contra...\textunderscore  + \textunderscore vento\textunderscore )}
\end{itemize}
Vento contrário.
Guarda-vento.
\section{Contraventor}
\begin{itemize}
\item {Grp. gram.:m.  e  adj.}
\end{itemize}
\begin{itemize}
\item {Proveniência:(Do lat. \textunderscore contraventus\textunderscore )}
\end{itemize}
O que contravém, que transgride.
\section{Contravergueiro}
\begin{itemize}
\item {Grp. gram.:m.}
\end{itemize}
\begin{itemize}
\item {Utilização:Náut.}
\end{itemize}
\begin{itemize}
\item {Proveniência:(De \textunderscore contra...\textunderscore  + \textunderscore vergueiro\textunderscore )}
\end{itemize}
Cabo, com que se atraca o vergueiro, de uma a outra parte, junto á amurada.
\section{Contraversão}
\begin{itemize}
\item {Grp. gram.:f.}
\end{itemize}
\begin{itemize}
\item {Proveniência:(Do lat. \textunderscore contraversus\textunderscore )}
\end{itemize}
Contravenção.
Inversão.
\section{Contraverter}
\begin{itemize}
\item {Grp. gram.:v. t.}
\end{itemize}
\begin{itemize}
\item {Proveniência:(Do lat. \textunderscore contra...\textunderscore  + \textunderscore vertere\textunderscore )}
\end{itemize}
Voltar para o lado contrário.
Inverter.
\section{Contra-vidraça}
\begin{itemize}
\item {Grp. gram.:f.}
\end{itemize}
O mesmo que \textunderscore contracaixilho\textunderscore .
\section{Contravir}
\begin{itemize}
\item {Grp. gram.:v. t.}
\end{itemize}
\begin{itemize}
\item {Grp. gram.:V. i.}
\end{itemize}
\begin{itemize}
\item {Proveniência:(Lat. \textunderscore contravenire\textunderscore )}
\end{itemize}
Transgredir.
Infringir lei ou regulamento.
Retorquir; responder.
\section{Contrectação}
\begin{itemize}
\item {Grp. gram.:f.}
\end{itemize}
\begin{itemize}
\item {Proveniência:(Lat. \textunderscore contrectatio\textunderscore )}
\end{itemize}
Acto de tirar alguma coisa da posse ou domínio de alguém.
\section{Contribar}
\begin{itemize}
\item {Grp. gram.:v. t.}
\end{itemize}
\begin{itemize}
\item {Utilização:T. de Ilhavo}
\end{itemize}
Impedir; contrariar.
\section{Contribuição}
\begin{itemize}
\item {fónica:bu-i}
\end{itemize}
\begin{itemize}
\item {Grp. gram.:f.}
\end{itemize}
\begin{itemize}
\item {Proveniência:(Lat. \textunderscore contributio\textunderscore )}
\end{itemize}
Acção de contribuir.
Quota parte, com que cada indivíduo entra em uma despesa commum ou nas despesas do Estado.
Imposto.
Subsídio moral, literário ou scientífico para algum fim útil.
\section{Contribuidor}
\begin{itemize}
\item {fónica:bu-i}
\end{itemize}
\begin{itemize}
\item {Grp. gram.:m.}
\end{itemize}
(V.contribuinte)
\section{Contribuinte}
\begin{itemize}
\item {Grp. gram.:adj.}
\end{itemize}
\begin{itemize}
\item {Grp. gram.:M.}
\end{itemize}
\begin{itemize}
\item {Proveniência:(Lat. \textunderscore contribuens\textunderscore )}
\end{itemize}
Que contribue.
Que está no caso de sêr tributado.
Aquelle que paga contribuição.
\section{Contribuir}
\begin{itemize}
\item {Grp. gram.:v. i.}
\end{itemize}
\begin{itemize}
\item {Proveniência:(Lat. \textunderscore contribuere\textunderscore )}
\end{itemize}
Têr parte nos meios, para a realização de uma coisa.
Cooperar.
Têr parte em um resultado.
Têr parte em uma despesa commum.
Pagar contribuições.
\section{Contribuitivo}
\begin{itemize}
\item {fónica:bu-i}
\end{itemize}
\begin{itemize}
\item {Grp. gram.:adj.}
\end{itemize}
\begin{itemize}
\item {Utilização:Neol.}
\end{itemize}
\begin{itemize}
\item {Proveniência:(De \textunderscore contribuir\textunderscore )}
\end{itemize}
Relativo a contribuição.
\section{Contributário}
\begin{itemize}
\item {Grp. gram.:m.  e  adj.}
\end{itemize}
\begin{itemize}
\item {Proveniência:(De \textunderscore con...\textunderscore  + \textunderscore tributário\textunderscore )}
\end{itemize}
O que é tributário com outro.
\section{Contrição}
\begin{itemize}
\item {Grp. gram.:f.}
\end{itemize}
\begin{itemize}
\item {Proveniência:(Lat. \textunderscore contritio\textunderscore )}
\end{itemize}
Arrependimento, dôr profunda, por se têr offendido a Deus.
\section{Contristação}
\begin{itemize}
\item {Grp. gram.:f.}
\end{itemize}
\begin{itemize}
\item {Proveniência:(Lat. \textunderscore contristatio\textunderscore )}
\end{itemize}
Acto ou effeito de contristar.
\section{Contristador}
\begin{itemize}
\item {Grp. gram.:adj.}
\end{itemize}
Que contrista.
\section{Contristar}
\begin{itemize}
\item {Grp. gram.:v. t.}
\end{itemize}
\begin{itemize}
\item {Proveniência:(Lat. \textunderscore contristare\textunderscore )}
\end{itemize}
Tornar triste.
Affligir.
\section{Contritamente}
\begin{itemize}
\item {Grp. gram.:adv.}
\end{itemize}
De modo contrito.
\section{Contrito}
\begin{itemize}
\item {Grp. gram.:adj.}
\end{itemize}
\begin{itemize}
\item {Proveniência:(Lat. \textunderscore contritus\textunderscore )}
\end{itemize}
Em que há contrição.
Que tem contrição.
Pesaroso, triste.
\section{Contro}
\begin{itemize}
\item {Grp. gram.:m.  e  interj.}
\end{itemize}
\begin{itemize}
\item {Proveniência:(De \textunderscore contra\textunderscore )}
\end{itemize}
Termo náutico de commando, para arribar.
\section{Contro...}
\begin{itemize}
\item {Proveniência:(Do lat. \textunderscore contra\textunderscore )}
\end{itemize}
\textunderscore pref.\textunderscore  de opposição, em \textunderscore controverter\textunderscore  e seus derivados.
\section{Controvérsia}
\begin{itemize}
\item {Grp. gram.:f.}
\end{itemize}
\begin{itemize}
\item {Proveniência:(Lat. \textunderscore controversia\textunderscore )}
\end{itemize}
Discussão regular, em matéria literária, scientífica ou religiosa.
Contestação.
\section{Controversista}
\begin{itemize}
\item {Grp. gram.:m.  e  f.}
\end{itemize}
\begin{itemize}
\item {Proveniência:(De \textunderscore controvérsia\textunderscore )}
\end{itemize}
Pessôa, que controverte, que põe dúvidas, que objecta.
Polemista.
\section{Controverso}
\begin{itemize}
\item {Grp. gram.:adj.}
\end{itemize}
\begin{itemize}
\item {Proveniência:(Lat. \textunderscore controversus\textunderscore )}
\end{itemize}
Que se controverteu ou se controverte.
Controvertido.
\section{Controverter}
\begin{itemize}
\item {Grp. gram.:v. t.}
\end{itemize}
\begin{itemize}
\item {Proveniência:(Do lat. \textunderscore contra\textunderscore  + \textunderscore vertere\textunderscore )}
\end{itemize}
Discutir.
Pôr em dúvida.
Fazer objecções sôbre: \textunderscore controverter uma theoria\textunderscore .
\section{Controvertido}
\begin{itemize}
\item {Grp. gram.:adj.}
\end{itemize}
\begin{itemize}
\item {Proveniência:(De \textunderscore controverter\textunderscore )}
\end{itemize}
Pôsto em dúvida.
Impugnado.
\section{Controvertível}
\begin{itemize}
\item {Grp. gram.:adj.}
\end{itemize}
Que se póde controverter.
Discutível.
Duvidoso.
\section{Contubernal}
\begin{itemize}
\item {Grp. gram.:adj.}
\end{itemize}
\begin{itemize}
\item {Grp. gram.:M. e adj.}
\end{itemize}
\begin{itemize}
\item {Proveniência:(Lat. \textunderscore contubernalis\textunderscore )}
\end{itemize}
Em que há contubérnio.
O que vive em contubérnio.
\section{Contubernar-se}
\begin{itemize}
\item {Grp. gram.:v. p.}
\end{itemize}
Viver em commum; amancebar-se:«\textunderscore contubernara-se com uma cigana\textunderscore ». Camillo, \textunderscore Judeu\textunderscore , 139.
(Por \textunderscore contuberniar-se\textunderscore , de \textunderscore contubérnio\textunderscore )
\section{Contubérnio}
\begin{itemize}
\item {Grp. gram.:m.}
\end{itemize}
\begin{itemize}
\item {Proveniência:(Lat. \textunderscore contuberlium\textunderscore )}
\end{itemize}
Vida em commum.
Convivência.
Familiaridade; camaradagem.
Concubinato.
Tenda de campanha. Cf. Castilho, \textunderscore Fastos\textunderscore , 584.
\section{Contudo}
\begin{itemize}
\item {Grp. gram.:adv.  e  conj.}
\end{itemize}
Todavia.
Não obstante.
\section{Contumácia}
\begin{itemize}
\item {Grp. gram.:f.}
\end{itemize}
\begin{itemize}
\item {Proveniência:(Lat. \textunderscore contumacia\textunderscore )}
\end{itemize}
Grande teimosia, obstinação.
\section{Contumaz}
\begin{itemize}
\item {Grp. gram.:adj.}
\end{itemize}
\begin{itemize}
\item {Proveniência:(Lat. \textunderscore contumax\textunderscore )}
\end{itemize}
Que tem contumácia.
\section{Contumélia}
\begin{itemize}
\item {Grp. gram.:f.}
\end{itemize}
\begin{itemize}
\item {Utilização:Chul.}
\end{itemize}
\begin{itemize}
\item {Proveniência:(Lat. \textunderscore contumelia\textunderscore )}
\end{itemize}
Invectiva.
Injúria.
Cumprimento, ràpapé.
\section{Contumeliosamente}
\begin{itemize}
\item {Grp. gram.:adv.}
\end{itemize}
De modo contumelioso.
\section{Contumelioso}
\begin{itemize}
\item {Grp. gram.:adj.}
\end{itemize}
\begin{itemize}
\item {Grp. gram.:M.}
\end{itemize}
\begin{itemize}
\item {Proveniência:(Lat. \textunderscore contumeliosus\textunderscore )}
\end{itemize}
Injurioso.
Aquelle que injuria.
\section{Contundente}
\begin{itemize}
\item {Grp. gram.:adj.}
\end{itemize}
\begin{itemize}
\item {Proveniência:(Lat. \textunderscore contundens\textunderscore )}
\end{itemize}
Que contunde.
\section{Contundido}
\begin{itemize}
\item {Grp. gram.:adj.}
\end{itemize}
\begin{itemize}
\item {Proveniência:(De \textunderscore contundir\textunderscore )}
\end{itemize}
Que soffreu contusão.
\section{Contundir}
\begin{itemize}
\item {Grp. gram.:v. i.}
\end{itemize}
\begin{itemize}
\item {Proveniência:(Lat. \textunderscore contundere\textunderscore )}
\end{itemize}
Pisar.
Moer; conculcar.
\section{Conturbação}
\begin{itemize}
\item {Grp. gram.:f.}
\end{itemize}
\begin{itemize}
\item {Proveniência:(Lat. \textunderscore conturbatio\textunderscore )}
\end{itemize}
Acto ou effeito de conturbar.
\section{Conturbador}
\begin{itemize}
\item {Grp. gram.:m.  e  adj.}
\end{itemize}
\begin{itemize}
\item {Proveniência:(Lat. \textunderscore conturbator\textunderscore )}
\end{itemize}
O que conturba.
\section{Conturbar}
\begin{itemize}
\item {Grp. gram.:v. t.}
\end{itemize}
\begin{itemize}
\item {Proveniência:(Lat. \textunderscore conturbare\textunderscore )}
\end{itemize}
Perturbar.
Alvorotar.
Amotinar.
\section{Conturbativo}
\begin{itemize}
\item {Grp. gram.:adj.}
\end{itemize}
\begin{itemize}
\item {Proveniência:(De \textunderscore conturbar\textunderscore )}
\end{itemize}
Que conturba.
\section{Contusamente}
\begin{itemize}
\item {Grp. gram.:adv.}
\end{itemize}
Com contusão.
\section{Contusão}
\begin{itemize}
\item {Grp. gram.:f.}
\end{itemize}
\begin{itemize}
\item {Proveniência:(Lat. \textunderscore contusio\textunderscore )}
\end{itemize}
Effeito de contundir.
Pisadura.
Lesão em tecidos vivos, produzida por pancada, sem rompimento da pelle.
Impressão, resentimento.
\section{Contuso}
\begin{itemize}
\item {Grp. gram.:adj.}
\end{itemize}
\begin{itemize}
\item {Proveniência:(Lat. \textunderscore contusus\textunderscore )}
\end{itemize}
O mesmo que \textunderscore contundido\textunderscore .
\section{Conubial}
\begin{itemize}
\item {Grp. gram.:adj.}
\end{itemize}
\begin{itemize}
\item {Proveniência:(Lat. \textunderscore connubialis\textunderscore )}
\end{itemize}
Relativo a conúbio.
\section{Conúbio}
\begin{itemize}
\item {Grp. gram.:m.}
\end{itemize}
\begin{itemize}
\item {Utilização:Fig.}
\end{itemize}
\begin{itemize}
\item {Proveniência:(Lat. \textunderscore connubium\textunderscore )}
\end{itemize}
Matrimónio.
União.
\section{Conulário}
\begin{itemize}
\item {Grp. gram.:m.}
\end{itemize}
\begin{itemize}
\item {Proveniência:(De \textunderscore cóne\textunderscore )}
\end{itemize}
Mollusco fóssil, de corpo cónico quadrangular.
\section{Conumerar}
\begin{itemize}
\item {Grp. gram.:v. t.}
\end{itemize}
\begin{itemize}
\item {Proveniência:(Lat. \textunderscore connumerare\textunderscore )}
\end{itemize}
Contar juntamente.
Referir ao mesmo número.
\section{Convalamarina}
\begin{itemize}
\item {Grp. gram.:f.}
\end{itemize}
Substância medicinal, extraida do lírio convale.
\section{Convalária}
\begin{itemize}
\item {Grp. gram.:f.}
\end{itemize}
\begin{itemize}
\item {Proveniência:(Do lat. \textunderscore convallis\textunderscore )}
\end{itemize}
Gênero de plantas esmiláceas.
\section{Convale}
\begin{itemize}
\item {Grp. gram.:adj.}
\end{itemize}
\begin{itemize}
\item {Proveniência:(Lat. \textunderscore convallis\textunderscore )}
\end{itemize}
Diz-se de um lírio branco.--O lírio convale floresce em Maio e tem aroma suave.
\section{Convalecer}
\textunderscore v. i.\textunderscore  (e der.)
(V. \textunderscore convalescer\textunderscore , etc.)
\section{Convales}
\begin{itemize}
\item {Grp. gram.:m. pl.}
\end{itemize}
\begin{itemize}
\item {Proveniência:(Do lat. \textunderscore convallis\textunderscore )}
\end{itemize}
Planícies entre colinas.
\section{Convalescença}
\begin{itemize}
\item {Grp. gram.:f.}
\end{itemize}
\begin{itemize}
\item {Proveniência:(Lat. \textunderscore convalescentia\textunderscore )}
\end{itemize}
Acto de convalescer.
\section{Convalescência}
\begin{itemize}
\item {Grp. gram.:f.}
\end{itemize}
(V.convalescença)
\section{Convalescente}
\begin{itemize}
\item {Grp. gram.:m.  e  adj.}
\end{itemize}
\begin{itemize}
\item {Proveniência:(Lat. \textunderscore convalescens\textunderscore )}
\end{itemize}
O que convalesce.
\section{Convalescer}
\begin{itemize}
\item {Grp. gram.:v. i.}
\end{itemize}
\begin{itemize}
\item {Proveniência:(Lat. \textunderscore convalescere\textunderscore )}
\end{itemize}
Adquirir fôrças, tornar-se mais forte.
Recuperar saúde.
Passar, mais ou menos lentamente, do estado de doente para o estado de são.
\section{Convallamarina}
\begin{itemize}
\item {Grp. gram.:f.}
\end{itemize}
Substância medicinal, extrahida do lírio convalle.
\section{Convallária}
\begin{itemize}
\item {Grp. gram.:f.}
\end{itemize}
\begin{itemize}
\item {Proveniência:(Do lat. \textunderscore convallis\textunderscore )}
\end{itemize}
Gênero de plantas esmiláceas.
\section{Convalle}
\begin{itemize}
\item {Grp. gram.:adj.}
\end{itemize}
\begin{itemize}
\item {Proveniência:(Lat. \textunderscore convallis\textunderscore )}
\end{itemize}
Diz-se de um lírio branco.--O lírio convalle floresce em Maio e tem aroma suave.
\section{Convalles}
\begin{itemize}
\item {Grp. gram.:m. pl.}
\end{itemize}
\begin{itemize}
\item {Proveniência:(Do lat. \textunderscore convallis\textunderscore )}
\end{itemize}
Planícies entre collinas.
\section{Convelir}
\begin{itemize}
\item {Grp. gram.:v. t.}
\end{itemize}
\begin{itemize}
\item {Grp. gram.:V. i.}
\end{itemize}
\begin{itemize}
\item {Utilização:Med.}
\end{itemize}
\begin{itemize}
\item {Proveniência:(Do lat. \textunderscore convellere\textunderscore )}
\end{itemize}
Deslocar (o que estava firme).
Abalar.
Subverter.
Têr convulsões ou espasmos.
\section{Convellir}
\begin{itemize}
\item {Grp. gram.:v. t.}
\end{itemize}
\begin{itemize}
\item {Grp. gram.:V. i.}
\end{itemize}
\begin{itemize}
\item {Utilização:Med.}
\end{itemize}
\begin{itemize}
\item {Proveniência:(Do lat. \textunderscore convellere\textunderscore )}
\end{itemize}
Deslocar (o que estava firme).
Abalar.
Subverter.
Têr convulsões ou espasmos.
\section{Convença}
\begin{itemize}
\item {Grp. gram.:f.}
\end{itemize}
\begin{itemize}
\item {Utilização:Prov.}
\end{itemize}
\begin{itemize}
\item {Utilização:Ant.}
\end{itemize}
O mesmo que \textunderscore convenção\textunderscore :«\textunderscore avenças, convenças, factos, compras e vendas.\textunderscore »\textunderscore Orden. do Reino\textunderscore , l. III, t. 59.
\section{Convenção}
\begin{itemize}
\item {Grp. gram.:f.}
\end{itemize}
\begin{itemize}
\item {Proveniência:(Lat. \textunderscore conventio\textunderscore )}
\end{itemize}
Ajuste, verbal ou escrito, entre duas ou mais pessôas.
Aquillo que tacitamente se acha convencionado nas relações sociaes, ou geralmente admittido e praticado.
Acôrdo, pacto, entre partidos que se hostilizavam: \textunderscore a Convenção de Évora-Monte\textunderscore .
Uma das assembleias legislativas da primeira república francesa.
\section{Convencedor}
\begin{itemize}
\item {Grp. gram.:adj.}
\end{itemize}
O mesmo que \textunderscore convincente\textunderscore . Cf. Arn. Gama, \textunderscore Ult. Dona\textunderscore , 392.
\section{Convencer}
\begin{itemize}
\item {Grp. gram.:v. t.}
\end{itemize}
\begin{itemize}
\item {Proveniência:(Lat. \textunderscore convincere\textunderscore )}
\end{itemize}
Persuadir com razões ou factos, sem que haja lugar para resposta ou objecção.
Deduzir, persuadindo.
Obrigar (alguém) a reconhecer que tem culpa.
\section{Convencido}
\begin{itemize}
\item {Grp. gram.:adj.}
\end{itemize}
\begin{itemize}
\item {Proveniência:(De \textunderscore convencer\textunderscore )}
\end{itemize}
Que adquiriu convicção; que não tem dúvidas.
\section{Convencimento}
\begin{itemize}
\item {Grp. gram.:m.}
\end{itemize}
\begin{itemize}
\item {Proveniência:(De \textunderscore convencer\textunderscore )}
\end{itemize}
O mesmo que \textunderscore convicção\textunderscore .
\section{Convencionado}
\begin{itemize}
\item {Grp. gram.:m.}
\end{itemize}
Nome, que se dá especialmente ao militar, que foi amnistiado pela Convenção política de Évora-Monte, em 1834.
\section{Convencional}
\begin{itemize}
\item {Grp. gram.:adj.}
\end{itemize}
\begin{itemize}
\item {Grp. gram.:M.}
\end{itemize}
\begin{itemize}
\item {Proveniência:(Lat. \textunderscore conventionalis\textunderscore )}
\end{itemize}
Relativo a convenção.
Resultante de convenção: \textunderscore a graphia«kilo»é puramente convencional\textunderscore .
Partidário da assembleia política, que se chamou Convenção.
\section{Convencionalismo}
\begin{itemize}
\item {Grp. gram.:m.}
\end{itemize}
\begin{itemize}
\item {Proveniência:(De \textunderscore convencional\textunderscore )}
\end{itemize}
Conjunto de convenções; systema de convenções.
\section{Convencionalmente}
\begin{itemize}
\item {Grp. gram.:adv.}
\end{itemize}
De modo convencional.
\section{Convencionar}
\begin{itemize}
\item {Grp. gram.:v. t.}
\end{itemize}
\begin{itemize}
\item {Proveniência:(Do lat. \textunderscore conventio\textunderscore )}
\end{itemize}
Pactuar.
Estabelecer por convenção.
\section{Convencível}
\begin{itemize}
\item {Grp. gram.:adj.}
\end{itemize}
Que se póde convencer.
\section{Convenente}
\begin{itemize}
\item {Grp. gram.:adj.}
\end{itemize}
\begin{itemize}
\item {Utilização:Ant.}
\end{itemize}
\begin{itemize}
\item {Proveniência:(Do lat. \textunderscore conveniens\textunderscore )}
\end{itemize}
Que ajusta, que contrata.
Contratante.
\section{Conveniência}
\begin{itemize}
\item {Grp. gram.:f.}
\end{itemize}
\begin{itemize}
\item {Grp. gram.:Pl.}
\end{itemize}
\begin{itemize}
\item {Proveniência:(Lat. \textunderscore convenientia\textunderscore )}
\end{itemize}
Qualidade daquillo que é conveniente.
Vantagem.
Interesse: \textunderscore por conveniência própria\textunderscore .
Decência.
Convenções, usos de sociedade.
\section{Conveniencioso}
\begin{itemize}
\item {Grp. gram.:adj.}
\end{itemize}
\begin{itemize}
\item {Proveniência:(De \textunderscore conveniência\textunderscore )}
\end{itemize}
O mesmo que \textunderscore interesseiro\textunderscore .
\section{Conveniente}
\begin{itemize}
\item {Grp. gram.:adj.}
\end{itemize}
\begin{itemize}
\item {Proveniência:(Lat. \textunderscore conveniens\textunderscore )}
\end{itemize}
Que convém, que se conforma.
Vantajoso, útil.
Decente.
\section{Convenientemente}
\begin{itemize}
\item {Grp. gram.:adv.}
\end{itemize}
De modo conveniente.
\section{Convênio}
\begin{itemize}
\item {Grp. gram.:m.}
\end{itemize}
\begin{itemize}
\item {Proveniência:(Do lat. \textunderscore convenire\textunderscore )}
\end{itemize}
Convenção, pacto internacional.
\section{Conventicular}
\begin{itemize}
\item {Grp. gram.:adj.}
\end{itemize}
Secreto.
Relativo a conventículo.
\section{Conventículo}
\begin{itemize}
\item {Grp. gram.:m.}
\end{itemize}
\begin{itemize}
\item {Proveniência:(Lat. \textunderscore conventiculum\textunderscore )}
\end{itemize}
Assembleia secreta de pessôas, que conspiram ou premeditam algum mal.
Conluio.
Ajuntamento de pessôas, sem publicidade.
\section{Conventilho}
\begin{itemize}
\item {Grp. gram.:m.}
\end{itemize}
\begin{itemize}
\item {Utilização:Bras. do S}
\end{itemize}
\begin{itemize}
\item {Proveniência:(De \textunderscore convento\textunderscore )}
\end{itemize}
Alcoice, lupanar.
\section{Convento}
\begin{itemize}
\item {Grp. gram.:m.}
\end{itemize}
\begin{itemize}
\item {Utilização:Ant.}
\end{itemize}
\begin{itemize}
\item {Utilização:Fig.}
\end{itemize}
\begin{itemize}
\item {Proveniência:(Lat. \textunderscore conventus\textunderscore )}
\end{itemize}
Habitação de communidade religiosa.
Communidade religiosa.
Communidade.
Divisão judicial do Império Romano.
Casa, em que se vive com recolhimento e sem diversões.
\section{Conventual}
\begin{itemize}
\item {Grp. gram.:adj.}
\end{itemize}
\begin{itemize}
\item {Grp. gram.:M.  e  f.}
\end{itemize}
\begin{itemize}
\item {Grp. gram.:Pl.}
\end{itemize}
\begin{itemize}
\item {Proveniência:(Do lat. \textunderscore conventus\textunderscore )}
\end{itemize}
Relativo a convento: \textunderscore a vida conventual\textunderscore .
Diz-se da Missa, rezada pelo párocho nos Domingos e dias santificados, e que também se chama \textunderscore Missa do dia\textunderscore .
Pessôa, residente num convento.
Ramificação da Ordem dos Carmelitas.
\section{Conventualidade}
\begin{itemize}
\item {Grp. gram.:f.}
\end{itemize}
\begin{itemize}
\item {Proveniência:(De \textunderscore conventual\textunderscore )}
\end{itemize}
Morada fixa num convento.
\section{Conventualmente}
\begin{itemize}
\item {Grp. gram.:adv.}
\end{itemize}
De modo conventual: \textunderscore viver conventualmemte\textunderscore .
\section{Convergência}
\begin{itemize}
\item {Grp. gram.:f.}
\end{itemize}
Acto de convergir.
\section{Convergente}
\begin{itemize}
\item {Grp. gram.:adj.}
\end{itemize}
\begin{itemize}
\item {Proveniência:(Lat. \textunderscore convergens\textunderscore )}
\end{itemize}
Que converge.
\section{Convergir}
\begin{itemize}
\item {Grp. gram.:v. i.}
\end{itemize}
\begin{itemize}
\item {Proveniência:(Lat. \textunderscore convergere\textunderscore )}
\end{itemize}
Tender, inclinar-se, dirigir-se, para um ponto commum.
Concorrer ao mesmo ponto.
Tender para o mesmo fim.
\section{Conversa}
\begin{itemize}
\item {Grp. gram.:f.}
\end{itemize}
\begin{itemize}
\item {Utilização:Pop.}
\end{itemize}
\begin{itemize}
\item {Proveniência:(De \textunderscore conversar\textunderscore )}
\end{itemize}
O mesmo que \textunderscore conversação\textunderscore .
Palavreado, cavaqueira.
Intrujice.
Peta.
\section{Conversa}
\begin{itemize}
\item {Grp. gram.:f.}
\end{itemize}
\begin{itemize}
\item {Proveniência:(De \textunderscore converso\textunderscore ^1)}
\end{itemize}
Mulher, recolhida num convento, sem professar.
\section{Conversação}
\begin{itemize}
\item {Grp. gram.:f.}
\end{itemize}
\begin{itemize}
\item {Proveniência:(Lat. \textunderscore conversatio\textunderscore )}
\end{itemize}
Acto de conversar.
Convivência, familiaridade.
\section{Conversada}
\begin{itemize}
\item {Grp. gram.:f.}
\end{itemize}
\begin{itemize}
\item {Utilização:Pop.}
\end{itemize}
\begin{itemize}
\item {Proveniência:(De \textunderscore conversado\textunderscore )}
\end{itemize}
Namorada.
\section{Conversadeira}
\begin{itemize}
\item {Grp. gram.:f.}
\end{itemize}
\begin{itemize}
\item {Utilização:Bras}
\end{itemize}
\begin{itemize}
\item {Proveniência:(De \textunderscore conversar\textunderscore )}
\end{itemize}
Cadeira dupla, com assentos oppostos.
\section{Conversado}
\begin{itemize}
\item {Grp. gram.:m.}
\end{itemize}
\begin{itemize}
\item {Utilização:Pop.}
\end{itemize}
Namorado.
\section{Conversador}
\begin{itemize}
\item {Grp. gram.:m.}
\end{itemize}
Aquelle que conversa.
Aquelle que gosta de conversar.
\section{Conversalhar}
\begin{itemize}
\item {Grp. gram.:v. i.}
\end{itemize}
\begin{itemize}
\item {Utilização:Prov.}
\end{itemize}
\begin{itemize}
\item {Proveniência:(De \textunderscore conversar\textunderscore )}
\end{itemize}
Conversar por mero passatempo; cavaquear.
\section{Conversante}
\begin{itemize}
\item {Grp. gram.:adj.}
\end{itemize}
\begin{itemize}
\item {Proveniência:(De \textunderscore conversar\textunderscore )}
\end{itemize}
Que conversa. Cf. Garrett, \textunderscore Viagens\textunderscore .
\section{Conversão}
\begin{itemize}
\item {Grp. gram.:f.}
\end{itemize}
\begin{itemize}
\item {Proveniência:(Lat. \textunderscore conversio\textunderscore )}
\end{itemize}
Acto ou effeito de converter.
\section{Conversar}
\begin{itemize}
\item {Grp. gram.:v. i.}
\end{itemize}
\begin{itemize}
\item {Utilização:Fig.}
\end{itemize}
\begin{itemize}
\item {Utilização:Pop.}
\end{itemize}
\begin{itemize}
\item {Grp. gram.:V. t.}
\end{itemize}
\begin{itemize}
\item {Proveniência:(Lat. \textunderscore conversari\textunderscore )}
\end{itemize}
Conviver.
Falar com alguém.
Palestrar.
Cavaquear.
Tomar conselho, ensinamento.
Namorar.
Tratar intimamente.
Sondar o pensamento de.
\section{Conversável}
\begin{itemize}
\item {Grp. gram.:adj.}
\end{itemize}
\begin{itemize}
\item {Proveniência:(De \textunderscore conversar\textunderscore )}
\end{itemize}
Que tem bom trato, que é sociável.
\section{Conversibilidade}
\begin{itemize}
\item {Grp. gram.:f.}
\end{itemize}
Qualidade de conversível.
\section{Conversível}
\begin{itemize}
\item {Grp. gram.:adj.}
\end{itemize}
O mesmo que \textunderscore convertível\textunderscore .
\section{Converso}
\begin{itemize}
\item {Grp. gram.:m.}
\end{itemize}
\begin{itemize}
\item {Proveniência:(Lat. \textunderscore conversus\textunderscore )}
\end{itemize}
Homem leigo, que servia em convento.
\section{Converso}
\begin{itemize}
\item {Grp. gram.:m.}
\end{itemize}
\begin{itemize}
\item {Utilização:Pop.}
\end{itemize}
\begin{itemize}
\item {Proveniência:(De \textunderscore conversar\textunderscore )}
\end{itemize}
Conversação.
\section{Conversor}
\begin{itemize}
\item {Grp. gram.:m.}
\end{itemize}
Apparelho metallúrgico.
(Cp. lat. \textunderscore conversus\textunderscore )
\section{Convertedor}
\begin{itemize}
\item {Grp. gram.:m.  e  adj.}
\end{itemize}
O que converte.
\section{Converter}
\begin{itemize}
\item {Grp. gram.:v. t.}
\end{itemize}
\begin{itemize}
\item {Proveniência:(Lat. \textunderscore convertere\textunderscore )}
\end{itemize}
Voltar.
Transformar: \textunderscore converter o dinheiro em rosas\textunderscore .
Substituir.
Fazer mudar de crença, opinião ou partido: \textunderscore converter os infiéis\textunderscore .
\section{Convertibilidade}
\begin{itemize}
\item {Grp. gram.:f.}
\end{itemize}
Qualidade daquillo ou de quem é convertível.
\section{Convertido}
\begin{itemize}
\item {Grp. gram.:m.}
\end{itemize}
Aquelle, que se converteu.
\section{Convertimento}
\begin{itemize}
\item {Grp. gram.:m.}
\end{itemize}
(V.conversão)
\section{Convertível}
\begin{itemize}
\item {Grp. gram.:adj.}
\end{itemize}
Que se póde converter.
\section{Convés}
\begin{itemize}
\item {Grp. gram.:m.}
\end{itemize}
\begin{itemize}
\item {Utilização:Náut.}
\end{itemize}
\begin{itemize}
\item {Proveniência:(Do lat. \textunderscore conversus\textunderscore )}
\end{itemize}
Espaço, entre o mastro grande e o do traquete, na coberta superior do navio.
Área da primeira coberta do navio.
\section{Convexidade}
\begin{itemize}
\item {Grp. gram.:f.}
\end{itemize}
\begin{itemize}
\item {Proveniência:(Lat. \textunderscore convexitas\textunderscore )}
\end{itemize}
Qualidade daquillo que é convexo.
Curvatura exterior.
\section{Convexirostro}
\begin{itemize}
\item {fónica:rós}
\end{itemize}
\begin{itemize}
\item {Grp. gram.:adj.}
\end{itemize}
\begin{itemize}
\item {Utilização:Zool.}
\end{itemize}
\begin{itemize}
\item {Proveniência:(Do lat. \textunderscore convexus\textunderscore  + \textunderscore rostrum\textunderscore )}
\end{itemize}
Cujo bico é convexo.
\section{Convexirrostro}
\begin{itemize}
\item {Grp. gram.:adj.}
\end{itemize}
\begin{itemize}
\item {Utilização:Zool.}
\end{itemize}
\begin{itemize}
\item {Proveniência:(Do lat. \textunderscore convexus\textunderscore  + \textunderscore rostrum\textunderscore )}
\end{itemize}
Cujo bico é convexo.
\section{Convexo}
\begin{itemize}
\item {Grp. gram.:adj.}
\end{itemize}
\begin{itemize}
\item {Proveniência:(Lat. \textunderscore convexus\textunderscore )}
\end{itemize}
Que tem saliencia curva.
Arredondado exteriormente.
Bojudo.
\section{Convicção}
\begin{itemize}
\item {Grp. gram.:f.}
\end{itemize}
\begin{itemize}
\item {Proveniência:(Lat. \textunderscore convictio\textunderscore )}
\end{itemize}
Effeito de convencer.
Certeza, obtida por factos ou razões, que não deixam dúvida, nem dão logar a objecção.
Persuasão.
Reconhecimento da própria culpa.
\section{Convício}
\begin{itemize}
\item {Grp. gram.:m.}
\end{itemize}
\begin{itemize}
\item {Proveniência:(Lat. \textunderscore convicium\textunderscore )}
\end{itemize}
Palavras injuriosas.
Injúria.
\section{Convicto}
\begin{itemize}
\item {Grp. gram.:adj.}
\end{itemize}
\begin{itemize}
\item {Proveniência:(Lat. \textunderscore convictus\textunderscore )}
\end{itemize}
O mesmo que \textunderscore convencido\textunderscore .
\section{Convidado}
\begin{itemize}
\item {Grp. gram.:m.}
\end{itemize}
\begin{itemize}
\item {Proveniência:(De \textunderscore convidar\textunderscore )}
\end{itemize}
Indivíduo, a quem se faz convite.
\section{Convidador}
\begin{itemize}
\item {Grp. gram.:m.}
\end{itemize}
Aquelle que gosta de convidar.
\section{Convidar}
\begin{itemize}
\item {Grp. gram.:v. t.}
\end{itemize}
\begin{itemize}
\item {Proveniência:(Do lat. hyp. \textunderscore convitare\textunderscore )}
\end{itemize}
Pedir que compareça, que tome parte em algum facto: \textunderscore convidar para um passeio\textunderscore .
Convocar.
Solicitar.
Excitar.
Atrahir, provocar: \textunderscore convidar para uma conspiração\textunderscore .
Obsequiar: \textunderscore convidar com um bello relógio\textunderscore .
Remunerar.
\section{Convidativo}
\begin{itemize}
\item {Grp. gram.:adj.}
\end{itemize}
\begin{itemize}
\item {Proveniência:(De \textunderscore convidar\textunderscore )}
\end{itemize}
Que convida.
Atrahente.
Appetitoso.
\section{Convidoso}
\begin{itemize}
\item {Grp. gram.:adj.}
\end{itemize}
O mesmo que \textunderscore convidativo\textunderscore .
\section{Convincente}
\begin{itemize}
\item {Grp. gram.:adj.}
\end{itemize}
\begin{itemize}
\item {Proveniência:(Lat. \textunderscore convincens\textunderscore )}
\end{itemize}
Que convence: \textunderscore provas convincentes\textunderscore .
\section{Convindo}
\begin{itemize}
\item {Grp. gram.:adj.}
\end{itemize}
\begin{itemize}
\item {Proveniência:(De \textunderscore convir\textunderscore )}
\end{itemize}
Conveniente.
Aceito, grato:«\textunderscore não era de estranhar não fossem elles bem convindos a Sua Magestade.\textunderscore »Filinto, \textunderscore D. Man.\textunderscore , I, 107.
\section{Convinhável}
\begin{itemize}
\item {Grp. gram.:adj.}
\end{itemize}
\begin{itemize}
\item {Utilização:Ant.}
\end{itemize}
\begin{itemize}
\item {Proveniência:(De \textunderscore convir\textunderscore )}
\end{itemize}
Conveniente.
\section{Convir}
\begin{itemize}
\item {Grp. gram.:v. i.}
\end{itemize}
\begin{itemize}
\item {Proveniência:(Lat. \textunderscore convenir\textunderscore )}
\end{itemize}
Vir juntamente.
Fazer ajuste.
Concordar: \textunderscore convieram em procurar o outro\textunderscore .
Sêr útil.
Sêr decoroso, decente: \textunderscore tal parcialidade não convém a juízes\textunderscore .
Conformar-se.
\section{Convite}
\begin{itemize}
\item {Grp. gram.:m.}
\end{itemize}
\begin{itemize}
\item {Utilização:Des.}
\end{itemize}
Acto do convidar.
Meio de convidar.
Dádiva, presente.
O mesmo que \textunderscore banquete\textunderscore .
(Cp. lat. \textunderscore convictus\textunderscore )
\section{Conviva}
\begin{itemize}
\item {Grp. gram.:m.}
\end{itemize}
\begin{itemize}
\item {Proveniência:(Lat. \textunderscore conviva\textunderscore )}
\end{itemize}
Aquelle que toma parte num banquete.
Commensal.
\section{Convival}
\begin{itemize}
\item {Grp. gram.:adj.}
\end{itemize}
\begin{itemize}
\item {Proveniência:(Lat. \textunderscore convivalis\textunderscore )}
\end{itemize}
Relativo a banquete.
\section{Convivência}
\begin{itemize}
\item {Grp. gram.:f.}
\end{itemize}
Acto ou effeito de conviver.
Familiaridade.
Relações íntimas.
\section{Convivente}
\begin{itemize}
\item {Grp. gram.:m.  e  adj.}
\end{itemize}
\begin{itemize}
\item {Proveniência:(Lat. \textunderscore convivens\textunderscore )}
\end{itemize}
O que convive.
\section{Conviver}
\begin{itemize}
\item {Grp. gram.:v. i.}
\end{itemize}
\begin{itemize}
\item {Proveniência:(Lat. \textunderscore convivere\textunderscore )}
\end{itemize}
Viver com outrem.
Têr intimidade.
\section{Convivial}
\begin{itemize}
\item {Grp. gram.:adj.}
\end{itemize}
O mesmo que \textunderscore convival\textunderscore .
\section{Convívio}
\begin{itemize}
\item {Grp. gram.:m.}
\end{itemize}
\begin{itemize}
\item {Utilização:Fig.}
\end{itemize}
\begin{itemize}
\item {Proveniência:(Lat. \textunderscore convivium\textunderscore )}
\end{itemize}
Banquete.
Camaradagem.
Convivência.
\section{Convizinhança}
\begin{itemize}
\item {Grp. gram.:m.}
\end{itemize}
\begin{itemize}
\item {Proveniência:(De \textunderscore convizinhar\textunderscore )}
\end{itemize}
Estado daquillo ou de quem convizinha.
\section{Convizinhar}
\begin{itemize}
\item {Grp. gram.:v. i.}
\end{itemize}
\begin{itemize}
\item {Utilização:Fig.}
\end{itemize}
Sêr convizinho.
Têr semelhança.
\section{Convizinho}
\begin{itemize}
\item {Grp. gram.:m.  e  adj.}
\end{itemize}
\begin{itemize}
\item {Utilização:Fig.}
\end{itemize}
\begin{itemize}
\item {Proveniência:(De \textunderscore con...\textunderscore  + \textunderscore vizinho\textunderscore )}
\end{itemize}
Vizinho com outrem.
Próximo.
Semelhante.
\section{Convocação}
\begin{itemize}
\item {Grp. gram.:f.}
\end{itemize}
\begin{itemize}
\item {Proveniência:(Lat. \textunderscore convocatio\textunderscore )}
\end{itemize}
Acto de convocar.
Convite.
\section{Convocador}
\begin{itemize}
\item {Grp. gram.:m.  e  adj.}
\end{itemize}
\begin{itemize}
\item {Proveniência:(De \textunderscore convocar\textunderscore )}
\end{itemize}
O que convoca.
\section{Convocar}
\begin{itemize}
\item {Grp. gram.:v. t.}
\end{itemize}
\begin{itemize}
\item {Proveniência:(Lat. \textunderscore convocare\textunderscore )}
\end{itemize}
Chamar para uma reunião.
Mandar reunir; constituir: \textunderscore convocar uma assembleia\textunderscore .
\section{Convocatória}
\begin{itemize}
\item {Grp. gram.:f.}
\end{itemize}
\begin{itemize}
\item {Proveniência:(De \textunderscore convocatório\textunderscore )}
\end{itemize}
Carta circular de convocação.
Convocação.
\section{Convocatório}
\begin{itemize}
\item {Grp. gram.:adj.}
\end{itemize}
Que serve para convocar.
\section{Convolar}
\begin{itemize}
\item {Grp. gram.:V. i.}
\end{itemize}
Passar rapidamente (de um estado a outro). Cf. Rui Rarbosa, \textunderscore Conf. na Baía\textunderscore .
\section{Convoluto}
\begin{itemize}
\item {Grp. gram.:adj.}
\end{itemize}
\begin{itemize}
\item {Proveniência:(Lat. \textunderscore convolutus\textunderscore )}
\end{itemize}
Enrolado.
Dobrado em fórma cylíndrica.
\section{Convolvuláceas}
\begin{itemize}
\item {Grp. gram.:f. pl.}
\end{itemize}
Família de plantas, que tem por typo o convólvulo.
(F. pl. de \textunderscore convólvuláceo\textunderscore )
\section{Convolvuláceo}
\begin{itemize}
\item {Grp. gram.:adj.}
\end{itemize}
Relativo ou semelhante ao \textunderscore convólvulo\textunderscore .
\section{Convolvulifoliado}
\begin{itemize}
\item {Grp. gram.:adj.}
\end{itemize}
\begin{itemize}
\item {Utilização:Bot.}
\end{itemize}
\begin{itemize}
\item {Proveniência:(Do lat. \textunderscore convolvulus\textunderscore  + \textunderscore folium\textunderscore )}
\end{itemize}
Que tem fôlhas semelhantes ás do convólvulo.
\section{Convólvulo}
\begin{itemize}
\item {Grp. gram.:m.}
\end{itemize}
\begin{itemize}
\item {Proveniência:(Lat. \textunderscore convolvulus\textunderscore )}
\end{itemize}
Planta trepadeira, de flôres semelhantes ás do lírio, e conhecida vulgarmente por \textunderscore bons-dias\textunderscore .
Corriola.
\section{Convulsamente}
\begin{itemize}
\item {Grp. gram.:adv.}
\end{itemize}
Em convulsão.
\section{Convulsão}
\begin{itemize}
\item {Grp. gram.:f.}
\end{itemize}
\begin{itemize}
\item {Proveniência:(Lat. \textunderscore convulsio\textunderscore )}
\end{itemize}
Acto de convellir.
Contracção muscullar, produzida por doença.
Cataclismo.
Grande agitação.
Grande transformação.
Revolução.
\section{Cofose}
\begin{itemize}
\item {Grp. gram.:f.}
\end{itemize}
\begin{itemize}
\item {Proveniência:(Gr. \textunderscore kophosis\textunderscore )}
\end{itemize}
Surdez completa.
\section{Conistra}
\begin{itemize}
\item {Grp. gram.:f.}
\end{itemize}
Recinto, onde se guardava a areia, com que se polvilhavam os atletas gregos, depois de besuntados com azeite.
Arena, em que se exercitavam os atletas.
\section{Coniza}
\begin{itemize}
\item {Grp. gram.:f.}
\end{itemize}
\begin{itemize}
\item {Proveniência:(Lat. \textunderscore conyza\textunderscore )}
\end{itemize}
Planta hermafrodita, de que há duas espécies.
\section{Convosco}
\begin{itemize}
\item {fónica:vôs}
\end{itemize}
\begin{itemize}
\item {Grp. gram.:loc. pron.}
\end{itemize}
Em vossa companhia: \textunderscore levai-me convosco\textunderscore .
De vós para vós.
Entre vós: \textunderscore resolvei lá isso convosco\textunderscore .
(Flexão do pron. \textunderscore vós\textunderscore , precedido da prep. \textunderscore com\textunderscore )
\section{Convulsar}
\begin{itemize}
\item {Grp. gram.:v. i.}
\end{itemize}
\begin{itemize}
\item {Proveniência:(De \textunderscore convulso\textunderscore )}
\end{itemize}
Diz-se, em veterinária, dos nervos que se contráhem.
\section{Convulsibilidade}
\begin{itemize}
\item {Grp. gram.:f.}
\end{itemize}
\begin{itemize}
\item {Proveniência:(Do lat. hypoth. \textunderscore convulsibilis\textunderscore )}
\end{itemize}
Disposição para convulsões mórbidas.
\section{Convulsionar}
\begin{itemize}
\item {Grp. gram.:v. t.}
\end{itemize}
\begin{itemize}
\item {Proveniência:(Do lat. \textunderscore convulsio\textunderscore )}
\end{itemize}
Pôr em convulsão.
Excitar.
Revolucionar.
\section{Convulsionário}
\begin{itemize}
\item {Grp. gram.:m.  e  adj.}
\end{itemize}
\begin{itemize}
\item {Proveniência:(Do lat. \textunderscore convulsio\textunderscore )}
\end{itemize}
O que soffre, ou finge soffrer convulsões.
\section{Convulsivamente}
\begin{itemize}
\item {Grp. gram.:adv.}
\end{itemize}
De modo convulsivo.
\section{Convulsivo}
\begin{itemize}
\item {Grp. gram.:adj.}
\end{itemize}
\begin{itemize}
\item {Proveniência:(De \textunderscore convulso\textunderscore )}
\end{itemize}
Relativo a convulsão.
Em que há convulsão.
Que se parece á convulsão.
\section{Convulso}
\begin{itemize}
\item {Grp. gram.:adj.}
\end{itemize}
\begin{itemize}
\item {Grp. gram.:M.}
\end{itemize}
\begin{itemize}
\item {Proveniência:(Lat. \textunderscore convulsus\textunderscore )}
\end{itemize}
Em que há convulsão.
Trémulo.
O mesmo que \textunderscore convulsão\textunderscore . Cf. Rebello, \textunderscore Mocidade\textunderscore , III, 42.
\section{Conystra}
\begin{itemize}
\item {Grp. gram.:f.}
\end{itemize}
Recinto, onde se guardava a areia, com que se polvilhavam os athletas gregos, depois de besuntados com azeite.
Arena, em que se exercitavam os athletas.
\section{Conyza}
\begin{itemize}
\item {Grp. gram.:f.}
\end{itemize}
\begin{itemize}
\item {Proveniência:(Lat. \textunderscore conyza\textunderscore )}
\end{itemize}
Planta hermaphrodita, de que há duas espécies.
\section{Coobrigado}
\begin{itemize}
\item {Grp. gram.:adj.}
\end{itemize}
\begin{itemize}
\item {Proveniência:(De \textunderscore co...\textunderscore  + \textunderscore obrigado\textunderscore )}
\end{itemize}
Obrigado juntamente com outrem. Cf. Rui Barbosa, \textunderscore Réplica\textunderscore , II, 157.
\section{Coonestação}
\begin{itemize}
\item {Grp. gram.:f.}
\end{itemize}
Acto de coonestar.
\section{Coonestador}
\begin{itemize}
\item {Grp. gram.:adj.}
\end{itemize}
Que coonesta.
\section{Coonestar}
\begin{itemize}
\item {Grp. gram.:v. t.}
\end{itemize}
\begin{itemize}
\item {Proveniência:(Lat. \textunderscore cohonestare\textunderscore )}
\end{itemize}
Dar aparência de honesto a.
Reabilitar.
\section{Cooperação}
\begin{itemize}
\item {Grp. gram.:f.}
\end{itemize}
Acto de cooperar.
\section{Cooperador}
\begin{itemize}
\item {Grp. gram.:m.  e  adj.}
\end{itemize}
\begin{itemize}
\item {Proveniência:(Lat. \textunderscore cooperator\textunderscore )}
\end{itemize}
O que coopera.
\section{Cooperante}
\begin{itemize}
\item {Grp. gram.:adj.}
\end{itemize}
\begin{itemize}
\item {Proveniência:(Lat. \textunderscore cooperans\textunderscore )}
\end{itemize}
Que coopera.
\section{Cooperar}
\begin{itemize}
\item {Grp. gram.:v. i.}
\end{itemize}
\begin{itemize}
\item {Proveniência:(Lat. \textunderscore cooperari\textunderscore )}
\end{itemize}
Operar simultaneamente.
Trabalhar em commum.
Collaborar.
\section{Cooperário}
\begin{itemize}
\item {Grp. gram.:m.}
\end{itemize}
(V.cooperador)
\section{Cooperativa}
\begin{itemize}
\item {Grp. gram.:f.}
\end{itemize}
\begin{itemize}
\item {Proveniência:(De \textunderscore cooperativo\textunderscore )}
\end{itemize}
Sociedade, em que são capitalistas os associados, e que tem por fim o benefício de todos elles, reduzindo-se-lhes os preços dos objectos de consumo, ou facilitando-se-lhes empréstimos, ou proporcionando-se-lhes trabalho lucrativo.
\section{Cooperativismo}
\begin{itemize}
\item {Grp. gram.:m.}
\end{itemize}
\begin{itemize}
\item {Proveniência:(De \textunderscore cooperativo\textunderscore )}
\end{itemize}
Systema dos que advogam o princípio cooperativo, como meio de progresso socialista e decadência capitalista.
\section{Cooperativista}
\begin{itemize}
\item {Grp. gram.:adj.}
\end{itemize}
\begin{itemize}
\item {Proveniência:(De \textunderscore cooperativa\textunderscore )}
\end{itemize}
Relativo ás sociedades cooperativas: \textunderscore progresso cooperativista\textunderscore .
\section{Cooperativo}
\begin{itemize}
\item {Grp. gram.:adj.}
\end{itemize}
\begin{itemize}
\item {Proveniência:(Lat. \textunderscore cooperativus\textunderscore )}
\end{itemize}
Em que há cooperação.
Que coopera.
\section{Coopositor}
\begin{itemize}
\item {Grp. gram.:m.}
\end{itemize}
\begin{itemize}
\item {Proveniência:(De \textunderscore co...\textunderscore  + \textunderscore oppositor\textunderscore )}
\end{itemize}
Aquele que é opositor com outrem.
\section{Cooppositor}
\begin{itemize}
\item {Grp. gram.:m.}
\end{itemize}
\begin{itemize}
\item {Proveniência:(De \textunderscore co...\textunderscore  + \textunderscore oppositor\textunderscore )}
\end{itemize}
Aquelle que é oppositor com outrem.
\section{Cooptação}
\begin{itemize}
\item {Grp. gram.:f.}
\end{itemize}
\begin{itemize}
\item {Proveniência:(Lat. \textunderscore cooptatio\textunderscore )}
\end{itemize}
Acto de cooptar.
\section{Cooptar}
\begin{itemize}
\item {Grp. gram.:v. t.}
\end{itemize}
\begin{itemize}
\item {Proveniência:(Lat. \textunderscore cooptare\textunderscore )}
\end{itemize}
Aggregar.
Admittir numa cooperação, com dispensa das formalidades ordinariamente exigidas.
\section{Coordenação}
\begin{itemize}
\item {Grp. gram.:f.}
\end{itemize}
\begin{itemize}
\item {Proveniência:(Lat. \textunderscore coordenatio\textunderscore )}
\end{itemize}
Acto ou effeito de coordenar.
\section{Coordenadas}
\begin{itemize}
\item {Grp. gram.:f. pl.}
\end{itemize}
\begin{itemize}
\item {Utilização:Mathem.}
\end{itemize}
\begin{itemize}
\item {Proveniência:(De \textunderscore coordenar\textunderscore )}
\end{itemize}
As abscissas e ordenadas de um ponto, de uma linha ou de uma superfície.
\section{Coordenado}
\begin{itemize}
\item {Grp. gram.:adj.}
\end{itemize}
\begin{itemize}
\item {Proveniência:(De \textunderscore coordenar\textunderscore )}
\end{itemize}
Disposto, segundo certas normas e méthodos.
\section{Coordenador}
\begin{itemize}
\item {Grp. gram.:adj.}
\end{itemize}
Que coordena.
\section{Coordenar}
\begin{itemize}
\item {Grp. gram.:v. t.}
\end{itemize}
\begin{itemize}
\item {Proveniência:(De \textunderscore co...\textunderscore  + \textunderscore ordenar\textunderscore )}
\end{itemize}
Dispor, em certa ordem ou segundo certas relações.
Arranjar, organizar: \textunderscore coordenar um recenseamento\textunderscore .
\section{Coordenativo}
\begin{itemize}
\item {Grp. gram.:adj.}
\end{itemize}
Que produz coordenação.
\section{Coorte}
\begin{itemize}
\item {Grp. gram.:f.}
\end{itemize}
\begin{itemize}
\item {Proveniência:(Lat. \textunderscore cohors\textunderscore )}
\end{itemize}
Parte de uma legião, entre os Romanos.
Porção de gente armada.
Magote.
\section{Copa}
\begin{itemize}
\item {Grp. gram.:f.}
\end{itemize}
\begin{itemize}
\item {Utilização:Ant.}
\end{itemize}
\begin{itemize}
\item {Utilização:Prov.}
\end{itemize}
\begin{itemize}
\item {Utilização:alent.}
\end{itemize}
\begin{itemize}
\item {Utilização:Prov.}
\end{itemize}
\begin{itemize}
\item {Utilização:minh.}
\end{itemize}
\begin{itemize}
\item {Utilização:Prov.}
\end{itemize}
\begin{itemize}
\item {Utilização:alent.}
\end{itemize}
\begin{itemize}
\item {Proveniência:(Lat. \textunderscore cupa\textunderscore )}
\end{itemize}
Grande vaso de madeira, mais conhecido por \textunderscore dorna\textunderscore  ou \textunderscore balseiro\textunderscore , (accepção des.).
Espécie de armário ou compartimento, em que se guardam loiças, gêneros alimentícios, etc.
A parte superior e convexa da ramagem da árvore.
Parte superior do chapéu, que cobre a cabeça.
Loiça, para serviço de mesa; baixella.
Vaso covo.
Copo.
Conjunto de peças do vestuário.
\textunderscore Copa de palha\textunderscore , feixe de palha.
Fato, roupas.
\section{Copada}
\begin{itemize}
\item {Grp. gram.:f.}
\end{itemize}
\begin{itemize}
\item {Utilização:Bras}
\end{itemize}
\begin{itemize}
\item {Proveniência:(De \textunderscore copa\textunderscore )}
\end{itemize}
Grande copa de árvore.
\section{Copada}
\begin{itemize}
\item {Grp. gram.:f.}
\end{itemize}
\begin{itemize}
\item {Utilização:Gír.}
\end{itemize}
\begin{itemize}
\item {Proveniência:(De \textunderscore copo\textunderscore )}
\end{itemize}
Porção de líquido, que um copo póde comportar.
Parte saliente e arredondada da base de uma columna.
Café.
\section{Copado}
\begin{itemize}
\item {Grp. gram.:adj.}
\end{itemize}
Que tem grande copa: \textunderscore árvore copada\textunderscore .
Convexo.
Enfunado.
\section{Copahiba}
\begin{itemize}
\item {Grp. gram.:f.}
\end{itemize}
Substância medicinal, extrahida de algumas árvores leguminosas.
\section{Copahibeira}
\begin{itemize}
\item {Grp. gram.:f.}
\end{itemize}
O mesmo que \textunderscore copahibeiro\textunderscore .
\section{Copahibeiro}
\begin{itemize}
\item {Grp. gram.:m.}
\end{itemize}
Uma das árvores leguminosas, de que se extrai a copahiba.
\section{Copahina}
\begin{itemize}
\item {Grp. gram.:f.}
\end{itemize}
Princípio estimulante das mucosas, contido na copahiba.
\section{Copaíba}
\begin{itemize}
\item {Grp. gram.:f.}
\end{itemize}
Substância medicinal, extraida de algumas árvores leguminosas.
\section{Copaibeiro}
\begin{itemize}
\item {fónica:pa-i}
\end{itemize}
\begin{itemize}
\item {Grp. gram.:m.}
\end{itemize}
Uma das árvores leguminosas, de que se extrai a copaiba.
\section{Copaibeira}
\begin{itemize}
\item {fónica:pa-i}
\end{itemize}
\begin{itemize}
\item {Grp. gram.:f.}
\end{itemize}
O mesmo que \textunderscore copaibeiro\textunderscore .
\section{Copaína}
\begin{itemize}
\item {Grp. gram.:f.}
\end{itemize}
Princípio estimulante das mucosas, contido na copaiba.
\section{Copal}
\begin{itemize}
\item {Grp. gram.:adj.}
\end{itemize}
\begin{itemize}
\item {Grp. gram.:M.}
\end{itemize}
\begin{itemize}
\item {Grp. gram.:F.}
\end{itemize}
\begin{itemize}
\item {Proveniência:(T. mex.)}
\end{itemize}
Diz-se de um suco resinoso e aromático, que se extrai de algumas árvores leguminosas das regiões tropicaes.
Goma copal.
Grande árvore leguminosa, (\textunderscore trachylobium\textunderscore ), de que se extrai aquelle suco.
\section{Copalcocote}
\begin{itemize}
\item {Grp. gram.:m.}
\end{itemize}
Fruto silvestre do México.
\section{Copalina}
\begin{itemize}
\item {Grp. gram.:f.}
\end{itemize}
Essência da goma copal.
\section{Copalmo}
\begin{itemize}
\item {Grp. gram.:m.}
\end{itemize}
Substância líquida, espécie de âmbar, que, por incisão, se extrai de uma árvore, (\textunderscore liquidambar styraciflua\textunderscore , Lin.).
\section{Copano}
\begin{itemize}
\item {Grp. gram.:m.}
\end{itemize}
Grande hotel indiano.
\section{Copar}
\begin{itemize}
\item {Grp. gram.:v. t.}
\end{itemize}
\begin{itemize}
\item {Grp. gram.:V. i.}
\end{itemize}
\begin{itemize}
\item {Proveniência:(De \textunderscore copa\textunderscore )}
\end{itemize}
Tornar copado.
Tosquiar a rama de (uma árvore), para a tornar copada.
Enfunar.
Dar fórma convexa a.
Tornar-se copado.
Têr fórma de copa.
\section{Coparrão}
\begin{itemize}
\item {Grp. gram.:m.}
\end{itemize}
O mesmo que \textunderscore copázio\textunderscore . Cf. Castilho, \textunderscore Tartufo\textunderscore , 19.
\section{Coparticipação}
\begin{itemize}
\item {Grp. gram.:f.}
\end{itemize}
Acto de coparticipar.
\section{Coparticipar}
\begin{itemize}
\item {Grp. gram.:v. i.}
\end{itemize}
\begin{itemize}
\item {Proveniência:(De \textunderscore co...\textunderscore  + \textunderscore participar\textunderscore )}
\end{itemize}
Participar, juntamente com outrem.
\section{Copas}
\begin{itemize}
\item {Grp. gram.:f. pl.}
\end{itemize}
\begin{itemize}
\item {Proveniência:(De \textunderscore copa\textunderscore )}
\end{itemize}
Um dos quatro naipes das cartas de jogar, no qual cada um dos pontos tem a figura de um coração encarnado.
\section{Copázio}
\begin{itemize}
\item {Grp. gram.:m.}
\end{itemize}
\begin{itemize}
\item {Utilização:Pop.}
\end{itemize}
\begin{itemize}
\item {Proveniência:(De \textunderscore copo\textunderscore )}
\end{itemize}
Copo grande.
Líquido que enche um copo: \textunderscore beber um copázio\textunderscore .
\section{Copé}
\begin{itemize}
\item {Grp. gram.:m.}
\end{itemize}
\begin{itemize}
\item {Utilização:Bras}
\end{itemize}
\begin{itemize}
\item {Utilização:Ant.}
\end{itemize}
Choupana; palhoça.
Rede, de malhas muito estreitas.
\section{Copeica}
\begin{itemize}
\item {Grp. gram.:f.}
\end{itemize}
Unidade de moéda russa.
\section{Copeira}
\begin{itemize}
\item {Grp. gram.:f.}
\end{itemize}
\begin{itemize}
\item {Proveniência:(De \textunderscore copa\textunderscore )}
\end{itemize}
Lugar, em que se guardam loiças de mesa.
Copa.
\section{Copeiro}
\begin{itemize}
\item {Grp. gram.:m.}
\end{itemize}
\begin{itemize}
\item {Grp. gram.:Adj.}
\end{itemize}
\begin{itemize}
\item {Proveniência:(Do b. lat. \textunderscore copparius\textunderscore )}
\end{itemize}
Aquelle, que trata da copa ou das loiças de mesa.
Aquelle, que prepara doces ou licores.
Aparador para copos e garrafas.
Intervallo cónico das rodas da carruagem.
Diz-se do engenho de açúcar, que é movido por águas que vêm de grande altura.
\section{Copejada}
\begin{itemize}
\item {Grp. gram.:f.}
\end{itemize}
\begin{itemize}
\item {Proveniência:(De \textunderscore copejar\textunderscore )}
\end{itemize}
Parte da rede de galeão, onde se reúne o peixe, para sêr copejado.
\section{Copejador}
\begin{itemize}
\item {Grp. gram.:m.}
\end{itemize}
Aquelle que copeja.
\section{Copejadura}
\begin{itemize}
\item {Grp. gram.:f.}
\end{itemize}
Acto de copejar.
\section{Copejar}
\begin{itemize}
\item {Grp. gram.:v. t.}
\end{itemize}
Arpoar, pescar com arpão.
Tirar das armações ou da rede de galeão (o peixe que nellas caiu).
\section{Copela}
\begin{itemize}
\item {Grp. gram.:f.}
\end{itemize}
\begin{itemize}
\item {Proveniência:(Lat. \textunderscore cupella\textunderscore )}
\end{itemize}
Vaso, que serve para a copelação.
\section{Copelação}
\begin{itemize}
\item {Grp. gram.:f.}
\end{itemize}
\begin{itemize}
\item {Proveniência:(De \textunderscore copelar\textunderscore )}
\end{itemize}
Operação, com que se separa a prata de outros metaes, pela acção do fogo na copela.
\section{Copelar}
\begin{itemize}
\item {Grp. gram.:v. t.}
\end{itemize}
Apurar ou passar pela copela.
\section{Copelha}
\begin{itemize}
\item {fónica:pê}
\end{itemize}
\begin{itemize}
\item {Grp. gram.:f.}
\end{itemize}
(V.copella)
\section{Copelhar}
\begin{itemize}
\item {Grp. gram.:m.}
\end{itemize}
\begin{itemize}
\item {Utilização:Ant.}
\end{itemize}
Dossel? :«\textunderscore hum pavilhão com copelhar de velludo\textunderscore ». \textunderscore Doc. do séc. XVI\textunderscore .--Em Barros, \textunderscore Déc.\textunderscore  I, c. VI, leio:«\textunderscore um capelhar do mesmo pau\textunderscore ». Qual das duas formas é exacta, se alguma o é?
\section{Copella}
\begin{itemize}
\item {Grp. gram.:f.}
\end{itemize}
\begin{itemize}
\item {Proveniência:(Lat. \textunderscore cupella\textunderscore )}
\end{itemize}
Vaso, que serve para a copellação.
\section{Copellação}
\begin{itemize}
\item {Grp. gram.:f.}
\end{itemize}
\begin{itemize}
\item {Proveniência:(De \textunderscore copellar\textunderscore )}
\end{itemize}
Operação, com que se separa a prata de outros metaes, pela acção do fogo na copella.
\section{Copellar}
\begin{itemize}
\item {Grp. gram.:v. t.}
\end{itemize}
Apurar ou passar pela copella.
\section{Copépodes}
\begin{itemize}
\item {Grp. gram.:m. pl.}
\end{itemize}
\begin{itemize}
\item {Proveniência:(Do gr. \textunderscore kope\textunderscore  + \textunderscore pous\textunderscore )}
\end{itemize}
Ordem de crustáceos, cujo corpo é dividido em anéis.
\section{Copérnico}
\begin{itemize}
\item {Grp. gram.:m.}
\end{itemize}
Instrumento, que representa o movimento dos corpos celestes, segundo o systema de Copérnico.
Uma das manchas da Lua.
\section{Copete}
\begin{itemize}
\item {fónica:pê}
\end{itemize}
\begin{itemize}
\item {Grp. gram.:m.}
\end{itemize}
O mesmo que \textunderscore topête\textunderscore .
(Cast. \textunderscore copete\textunderscore )
\section{Cophose}
\begin{itemize}
\item {Grp. gram.:f.}
\end{itemize}
\begin{itemize}
\item {Proveniência:(Gr. \textunderscore kophosis\textunderscore )}
\end{itemize}
Surdez completa.
\section{Cópia}
\begin{itemize}
\item {Grp. gram.:f.}
\end{itemize}
\begin{itemize}
\item {Proveniência:(Lat. \textunderscore copia\textunderscore )}
\end{itemize}
Grande quantidade, abundância.
Reproducção mecânica.
Traslado: \textunderscore cópia de um manuscrito\textunderscore .
Imitação.
\section{Copiá}
\begin{itemize}
\item {Grp. gram.:m.}
\end{itemize}
\begin{itemize}
\item {Utilização:Bras. do N}
\end{itemize}
O mesmo que copiar^2.
\section{Copiador}
\begin{itemize}
\item {Grp. gram.:m.}
\end{itemize}
Aquelle que copía.
Apparelho, para copiar escritos.
Livro, em que se copiam cartas ou outros documentos.
\section{Copiar}
\begin{itemize}
\item {Grp. gram.:v. t.}
\end{itemize}
Fazer a cópia de.
Reproduzir, imitando.
Imitar.
Plagiar.
\section{Copiar}
\begin{itemize}
\item {Grp. gram.:m.}
\end{itemize}
\begin{itemize}
\item {Utilização:Bras}
\end{itemize}
Alpendre.
Deanteira das casas pequenas ou palhoças.
(Do tupi)
\section{Copiara}
\begin{itemize}
\item {Grp. gram.:f.}
\end{itemize}
\begin{itemize}
\item {Utilização:Bras}
\end{itemize}
O mesmo que \textunderscore copiar\textunderscore ^2.
\section{Copilar}
\textunderscore v. t.\textunderscore  (e der.)
O mesmo que \textunderscore compilar\textunderscore , etc.
\section{Copilo}
\begin{itemize}
\item {Grp. gram.:m.}
\end{itemize}
\begin{itemize}
\item {Utilização:Prov.}
\end{itemize}
\begin{itemize}
\item {Utilização:beir.}
\end{itemize}
Planta parietária, também conhecida por \textunderscore sombreirinha-dos-telhados\textunderscore .
\section{Copio}
\begin{itemize}
\item {Grp. gram.:m.}
\end{itemize}
Rede miúda de arrastar.
\section{Copiografar}
\begin{itemize}
\item {Grp. gram.:v. t.}
\end{itemize}
Reproduzir com o copiógrafo (uma escrita).
\section{Copiógrafo}
\begin{itemize}
\item {Grp. gram.:m.}
\end{itemize}
\begin{itemize}
\item {Proveniência:(De \textunderscore copia\textunderscore  + gr. \textunderscore graphein\textunderscore )}
\end{itemize}
Utensílio, cuja parte principal é uma pasta de gelatina ou substância análoga, sôbre a qual se estampa um manuscrito, de que se extraem mecanicamente muitos exemplares.
\section{Copiographar}
\begin{itemize}
\item {Grp. gram.:v. t.}
\end{itemize}
Reproduzir com o copiógrapho (uma escrita).
\section{Copiógrapho}
\begin{itemize}
\item {Grp. gram.:m.}
\end{itemize}
\begin{itemize}
\item {Proveniência:(De \textunderscore copia\textunderscore  + gr. \textunderscore graphein\textunderscore )}
\end{itemize}
Utensílio, cuja parte principal é uma pasta de gelatina ou substância análoga, sôbre a qual se estampa um manuscrito, de que se extrahem mecanicamente muitos exemplares.
\section{Copiosamente}
\begin{itemize}
\item {Grp. gram.:adv.}
\end{itemize}
De modo copioso.
\section{Copiosidade}
\begin{itemize}
\item {Grp. gram.:f.}
\end{itemize}
Qualidade daquillo que é copioso.
\section{Copioso}
\begin{itemize}
\item {Grp. gram.:adj.}
\end{itemize}
\begin{itemize}
\item {Proveniência:(Lat. \textunderscore copiosus\textunderscore )}
\end{itemize}
De que há cópia.
Abundante.
Grande.
\section{Copista}
\begin{itemize}
\item {Grp. gram.:m.}
\end{itemize}
Aquelle que copía.
\section{Copista}
\begin{itemize}
\item {Grp. gram.:m.}
\end{itemize}
\begin{itemize}
\item {Utilização:Pop.}
\end{itemize}
\begin{itemize}
\item {Proveniência:(De \textunderscore copo\textunderscore )}
\end{itemize}
Beberrão.
\section{Copistaria}
\begin{itemize}
\item {Grp. gram.:f.}
\end{itemize}
\begin{itemize}
\item {Utilização:Ant.}
\end{itemize}
\begin{itemize}
\item {Proveniência:(De \textunderscore copista\textunderscore )}
\end{itemize}
Officina, onde trabalham copistas de música.
\section{Copla}
\begin{itemize}
\item {Grp. gram.:f.}
\end{itemize}
\begin{itemize}
\item {Proveniência:(Do lat. \textunderscore copula\textunderscore )}
\end{itemize}
Estrophe, pequeno grupo de versos.
Quadra.
\section{Copo}
\begin{itemize}
\item {Grp. gram.:m.}
\end{itemize}
\begin{itemize}
\item {Grp. gram.:Pl.}
\end{itemize}
Pequeno vaso, proximamente cylíndrico, para se beber por elle.
Conteúdo de um copo.
Objecto semelhante ao copo.
Guarda da mão na espada.
(Da mesma or. que \textunderscore copa\textunderscore )
\section{Çopo}
\begin{itemize}
\item {fónica:çô}
\end{itemize}
\begin{itemize}
\item {Grp. gram.:m.}
\end{itemize}
O mesmo que \textunderscore zopo\textunderscore . Cf. Herculano, \textunderscore Lendas\textunderscore , I, 266.
\section{Copofone}
\begin{itemize}
\item {Grp. gram.:m.}
\end{itemize}
\begin{itemize}
\item {Proveniência:(De \textunderscore copo\textunderscore  + gr. \textunderscore phone\textunderscore )}
\end{itemize}
Designação híbrida, mas vulgar, de uma série de copos graduados que, friccionados pelos dedos, emittem sons mais ou menos agradáveis.
\section{Copofónio}
\begin{itemize}
\item {Grp. gram.:m.}
\end{itemize}
\begin{itemize}
\item {Proveniência:(De \textunderscore copo\textunderscore  + gr. \textunderscore phone\textunderscore )}
\end{itemize}
Designação híbrida, mas vulgar, de uma série de copos graduados que, friccionados pelos dedos, emittem sons mais ou menos agradáveis.
\section{Copophone}
\begin{itemize}
\item {Grp. gram.:m.}
\end{itemize}
\begin{itemize}
\item {Proveniência:(De \textunderscore copo\textunderscore  + gr. \textunderscore phone\textunderscore )}
\end{itemize}
Designação hýbrida, mas vulgar, de uma série de copos graduados que, friccionados pelos dedos, emittem sons mais ou menos agradáveis.
\section{Coppano}
\begin{itemize}
\item {Grp. gram.:m.}
\end{itemize}
Grande hotel indiano.
\section{Copra}
\begin{itemize}
\item {Grp. gram.:f.}
\end{itemize}
\begin{itemize}
\item {Proveniência:(Do sanscr. \textunderscore kharpara\textunderscore )}
\end{itemize}
Amêndoa de côco, sêca e preparada, para della se extrahir o copraol.
\section{Copra}
\begin{itemize}
\item {Grp. gram.:f.}
\end{itemize}
\begin{itemize}
\item {Utilização:Ant.}
\end{itemize}
O mesmo que \textunderscore copla\textunderscore . Cf. \textunderscore Eufrosina\textunderscore , 181.
\section{Coprão}
\begin{itemize}
\item {Grp. gram.:m.}
\end{itemize}
(?):«\textunderscore ...hũa maneira de coprão de cuberta de armar\textunderscore ». Barros, \textunderscore Déc.\textunderscore  II, l. 7, c. 3.^o.
(Por \textunderscore cofrão\textunderscore , de \textunderscore cofre\textunderscore ?)
\section{Copraol}
\begin{itemize}
\item {Grp. gram.:m.}
\end{itemize}
\begin{itemize}
\item {Proveniência:(De copra)}
\end{itemize}
Substância gorda, própria para suppositórios, velas, etc.
\section{Coprocurador}
\begin{itemize}
\item {Grp. gram.:m.}
\end{itemize}
\begin{itemize}
\item {Utilização:Bras}
\end{itemize}
Aquelle que é procurador, juntamente com outro.
\section{Coprófago}
\begin{itemize}
\item {Grp. gram.:adj.}
\end{itemize}
\begin{itemize}
\item {Proveniência:(Do gr. \textunderscore kopros\textunderscore  + \textunderscore phagein\textunderscore )}
\end{itemize}
Diz-se dos animaes que vivem de excrementos.
\section{Coprolalia}
\begin{itemize}
\item {Grp. gram.:f.}
\end{itemize}
\begin{itemize}
\item {Utilização:Neol.}
\end{itemize}
\begin{itemize}
\item {Proveniência:(T. hybr., do gr. \textunderscore kopros\textunderscore  + lat. \textunderscore lallare\textunderscore )}
\end{itemize}
Doença, caracterizada pela necessidade inconsciente de dizer obscenidades ou plebeísmos.
\section{Coprolallia}
\begin{itemize}
\item {Grp. gram.:f.}
\end{itemize}
\begin{itemize}
\item {Utilização:Neol.}
\end{itemize}
\begin{itemize}
\item {Proveniência:(T. hybr., do gr. \textunderscore kopros\textunderscore  + lat. \textunderscore lallare\textunderscore )}
\end{itemize}
Doença, caracterizada pela necessidade inconsciente de dizer obscenidades ou plebeísmos.
\section{Coprólitho}
\begin{itemize}
\item {Grp. gram.:m.}
\end{itemize}
\begin{itemize}
\item {Proveniência:(Do gr. \textunderscore kopros\textunderscore  + \textunderscore lithos\textunderscore )}
\end{itemize}
Excremento fóssil.
\section{Coprólito}
\begin{itemize}
\item {Grp. gram.:m.}
\end{itemize}
\begin{itemize}
\item {Proveniência:(Do gr. \textunderscore kopros\textunderscore  + \textunderscore lithos\textunderscore )}
\end{itemize}
Excremento fóssil.
\section{Copróphago}
\begin{itemize}
\item {Grp. gram.:adj.}
\end{itemize}
\begin{itemize}
\item {Proveniência:(Do gr. \textunderscore kopros\textunderscore  + \textunderscore phagein\textunderscore )}
\end{itemize}
Diz-se dos animaes que vivem de excrementos.
\section{Copropriedade}
\begin{itemize}
\item {Grp. gram.:f.}
\end{itemize}
(V.compropriedade)
\section{Coproprietário}
\begin{itemize}
\item {Grp. gram.:m.}
\end{itemize}
(V.comproprietário)
\section{Coprosclerose}
\begin{itemize}
\item {Grp. gram.:f.}
\end{itemize}
\begin{itemize}
\item {Proveniência:(Do gr. \textunderscore kropos\textunderscore  + \textunderscore sklerosis\textunderscore )}
\end{itemize}
Endurecimento dos excrementos, nos intestinos.
\section{Coprosma}
\begin{itemize}
\item {Grp. gram.:f.}
\end{itemize}
Gênero de plantas rubiáceas.
\section{Copróstase}
\begin{itemize}
\item {Grp. gram.:f.}
\end{itemize}
O mesmo que \textunderscore coprostasia\textunderscore .
\section{Coprostasia}
\begin{itemize}
\item {Grp. gram.:f.}
\end{itemize}
\begin{itemize}
\item {Proveniência:(Do gr. \textunderscore kopros\textunderscore  + \textunderscore stasïas\textunderscore )}
\end{itemize}
Retenção dos excrementos.
Constipação de ventre.
\section{Copta}
\begin{itemize}
\item {Grp. gram.:m.  e  adj.}
\end{itemize}
(V.copto)
\section{Cóptico}
\begin{itemize}
\item {Grp. gram.:adj.}
\end{itemize}
Relativo aos coptos.
\section{Cóptido}
\begin{itemize}
\item {Grp. gram.:m.}
\end{itemize}
\begin{itemize}
\item {Proveniência:(Do gr. \textunderscore kopto\textunderscore )}
\end{itemize}
Planta ranunculácea das regiões árcticas.
\section{Copto}
\begin{itemize}
\item {Grp. gram.:adj.}
\end{itemize}
\begin{itemize}
\item {Grp. gram.:M.}
\end{itemize}
\begin{itemize}
\item {Grp. gram.:Pl.}
\end{itemize}
Relativo aos Coptos.
Língua dos Coptos.
Christãos, que habitáram no Egypto e na Abyssínia.
\section{Coptografia}
\begin{itemize}
\item {Grp. gram.:f.}
\end{itemize}
\begin{itemize}
\item {Proveniência:(Do gr. \textunderscore koptein\textunderscore  + \textunderscore graphein\textunderscore )}
\end{itemize}
Arte de recortar pedaços de cartão, de maneira que se desenham figuras pela sombra deles, projectada numa parede.
\section{Coptographia}
\begin{itemize}
\item {Grp. gram.:f.}
\end{itemize}
\begin{itemize}
\item {Proveniência:(Do gr. \textunderscore koptein\textunderscore  + \textunderscore graphein\textunderscore )}
\end{itemize}
Arte de recortar pedaços de cartão, de maneira que se desenham figuras pela sombra delles, projectada numa parede.
\section{Copu}
\begin{itemize}
\item {Grp. gram.:m.}
\end{itemize}
\begin{itemize}
\item {Utilização:Bras. do N}
\end{itemize}
Bebida refrigerante de cacau.
\section{Cópula}
\begin{itemize}
\item {Grp. gram.:f.}
\end{itemize}
\begin{itemize}
\item {Utilização:Gram.}
\end{itemize}
\begin{itemize}
\item {Utilização:Mús.}
\end{itemize}
\begin{itemize}
\item {Utilização:Mús.}
\end{itemize}
\begin{itemize}
\item {Utilização:ant.}
\end{itemize}
\begin{itemize}
\item {Proveniência:(Lat. \textunderscore copula\textunderscore )}
\end{itemize}
União ou ligação sexual.
Verbo, que une o predicado ao nome.
Registo de órgão, que une um teclado com outro.
Série do notas rápidas, que acompanham uma nota longa.
\section{Copulação}
\begin{itemize}
\item {Grp. gram.:f.}
\end{itemize}
\begin{itemize}
\item {Proveniência:(De \textunderscore copular\textunderscore )}
\end{itemize}
Ligação chímica.
Cópula.
\section{Copulador}
\begin{itemize}
\item {Grp. gram.:m.}
\end{itemize}
Aquelle que copula.
\section{Copular}
\begin{itemize}
\item {Grp. gram.:v. t.}
\end{itemize}
\begin{itemize}
\item {Grp. gram.:V. i.}
\end{itemize}
\begin{itemize}
\item {Proveniência:(Lat. \textunderscore copulare\textunderscore )}
\end{itemize}
Ligar.
Acasalar.
Têr cópula.
\section{Copulativo}
\begin{itemize}
\item {Grp. gram.:adj.}
\end{itemize}
\begin{itemize}
\item {Proveniência:(Lat. \textunderscore copulativus\textunderscore )}
\end{itemize}
Que liga, que serve para ligar: \textunderscore conjuncção copulativa\textunderscore .
\section{Coque}
\begin{itemize}
\item {Grp. gram.:m.}
\end{itemize}
\begin{itemize}
\item {Proveniência:(T. onom.)}
\end{itemize}
Pancada na cabeça com os nós dos dedos, com vara, cana, etc.
Carolo.
\section{Coque}
\begin{itemize}
\item {Grp. gram.:m.}
\end{itemize}
\begin{itemize}
\item {Proveniência:(Do lat. \textunderscore coqua\textunderscore , se não do ingl. \textunderscore cook\textunderscore )}
\end{itemize}
Cozinheiro de marnotos, nas margens do Sado.
\section{Coque}
\begin{itemize}
\item {Grp. gram.:m.}
\end{itemize}
\begin{itemize}
\item {Proveniência:(Ingl. \textunderscore coke\textunderscore )}
\end{itemize}
Espécie de carvão, que se obtém, destillando a hulha.
\section{Coqueiral}
\begin{itemize}
\item {Grp. gram.:m.}
\end{itemize}
Lugar, onde crescem coqueiros.
\section{Coqueiro}
\begin{itemize}
\item {Grp. gram.:m.}
\end{itemize}
Variedade de palmeira, que produz cocos.
\section{Coqueiro}
\begin{itemize}
\item {Grp. gram.:m.}
\end{itemize}
\begin{itemize}
\item {Utilização:des.}
\end{itemize}
\begin{itemize}
\item {Utilização:Pop.}
\end{itemize}
\begin{itemize}
\item {Proveniência:(De \textunderscore cocar\textunderscore )}
\end{itemize}
Intrujão.
Especulador.
\section{Coqueluche}
\begin{itemize}
\item {Grp. gram.:f.}
\end{itemize}
\begin{itemize}
\item {Proveniência:(Fr. \textunderscore coqueluche\textunderscore )}
\end{itemize}
Tosse convulsa.
\section{Coqueta}
\begin{itemize}
\item {fónica:quê}
\end{itemize}
\begin{itemize}
\item {Grp. gram.:f.}
\end{itemize}
Aportuguezamento do fr. \textunderscore coquette\textunderscore . Cf. Filinto, XI, 54.
\section{Coquettismo}
\begin{itemize}
\item {Grp. gram.:m.}
\end{itemize}
\begin{itemize}
\item {Utilização:Gal}
\end{itemize}
\begin{itemize}
\item {Proveniência:(De \textunderscore coquette\textunderscore )}
\end{itemize}
Garridice.
Ar galanteador, pretensioso.
\section{Coqui}
\begin{itemize}
\item {Grp. gram.:m.}
\end{itemize}
Bella árvore fructifera, que no Japão produz o ébano do commércio e que começou há pouco (1898) a aclimatar-se em Portugal.
\section{Coquilho}
\begin{itemize}
\item {Grp. gram.:m.}
\end{itemize}
\begin{itemize}
\item {Proveniência:(De \textunderscore côco\textunderscore )}
\end{itemize}
Pequeno côco ou substância vegetal e dura, de que se fazem contas de rosários, etc.
Árvore dos Açores, (\textunderscore platanus orientalis\textunderscore ).
\section{Coquílis}
\begin{itemize}
\item {Grp. gram.:f.}
\end{itemize}
Pequena borboleta, espécie de traça, que come as flôres da videira e lhe corrói os bagos.
Doença das vinhas, occasionada por aquelle insecto.
\section{Coquinha}
\begin{itemize}
\item {Grp. gram.:f.}
\end{itemize}
Planta cyperácea da Índia portuguesa, (\textunderscore hyllinga brevifolia\textunderscore , Rottb).
\section{Côr}
\begin{itemize}
\item {Grp. gram.:f.}
\end{itemize}
\begin{itemize}
\item {Grp. gram.:Loc. adv.}
\end{itemize}
\begin{itemize}
\item {Proveniência:(Do lat. \textunderscore color\textunderscore )}
\end{itemize}
Impressão, que a luz, reflectida pelos corpos, produz no órgão da vista.
Matéria colorante, que se applica em tintas.
Rubor das faces: \textunderscore está convalescente e já tem côr\textunderscore .
Qualquer côr, excepto o branco e o preto: \textunderscore fazenda de côr\textunderscore .
Colorido.
Côr escura: \textunderscore homem de côr\textunderscore .
Vermelhão.
Carácter: \textunderscore esta descripção tem côr local\textunderscore .
Bandeira.
Partido.
Pretexto, apparência: \textunderscore um velhaquete sob côr de santarrão\textunderscore .
\textunderscore Com côr\textunderscore , com vergonha; còrando.
\section{Cór}
\begin{itemize}
\item {Grp. gram.:m.}
\end{itemize}
(V.de-cór)
\section{Cora}
\begin{itemize}
\item {Grp. gram.:f.}
\end{itemize}
\begin{itemize}
\item {Utilização:Prov.}
\end{itemize}
\begin{itemize}
\item {Utilização:minh.}
\end{itemize}
\begin{itemize}
\item {Proveniência:(De \textunderscore córar\textunderscore )}
\end{itemize}
Acto de còrar (roupa, cera, etc.).
Brasído, que se conserva á boca do forno, em quanto neste se metem as broas.
\section{Corá}
\begin{itemize}
\item {Grp. gram.:f.}
\end{itemize}
Iguaria brasileira, feita de milho verde.
\section{Coração}
\begin{itemize}
\item {Grp. gram.:m.}
\end{itemize}
\begin{itemize}
\item {Utilização:Prov.}
\end{itemize}
\begin{itemize}
\item {Utilização:Prov.}
\end{itemize}
\begin{itemize}
\item {Utilização:alent.}
\end{itemize}
Órgão musculoso, que é o centro da circulação do sangue.
Sentimento moral: \textunderscore têr bom coração\textunderscore .
Complexo de faculdades affectivas: \textunderscore saiu-lhe do coração aquelle grito\textunderscore .
Consciência.
Memória.
Peito: \textunderscore apertar alguém ao coração\textunderscore .
Affeição, amor.
Pessôa, que alimenta sentimentos moraes e exerce faculdades affectivas.
Generosidade.
Presentimento: \textunderscore não sei que me diz o coração\textunderscore .
Centro: \textunderscore no coração da Europa\textunderscore .
Objecto, que tem a fórma de coração: \textunderscore presenteou-a com um coração de oiro\textunderscore .
Animo, coragem:«\textunderscore eram as palavras de Calon, tão cheias de certeza, que dariam novos corações.\textunderscore »Simão de Vasc., \textunderscore Hist. Insulana\textunderscore .
Vontade.
Designação de várias plantas.
Nome de várias estrêllas.
Cada um dos pesos do tear, em fórma de coração.
Peça da forquilha, em que entram os dentes e o cabo.
Peça fundida, ou formada de troços de carris, angular, na intersecção de via férrea.
(Cast. \textunderscore corazón\textunderscore )
\section{Coração}
\begin{itemize}
\item {Grp. gram.:m.}
\end{itemize}
\begin{itemize}
\item {Utilização:Bras. do Rio}
\end{itemize}
Varanda.
\section{Còração}
\begin{itemize}
\item {Grp. gram.:f.}
\end{itemize}
\begin{itemize}
\item {Utilização:Neol.}
\end{itemize}
Acto ou effeito de còrar.
\section{Coração-de-gallo}
\begin{itemize}
\item {Grp. gram.:m.}
\end{itemize}
\begin{itemize}
\item {Utilização:Prov.}
\end{itemize}
\begin{itemize}
\item {Utilização:alent.}
\end{itemize}
Casta de uva preta.--\textunderscore Coração\textunderscore  é talvez euphemismo, em substituição de um t. obsceno, por que é geralmente designada esta uva.
Variedade de azeitona.
\section{Coração-de-lamarinau}
\begin{itemize}
\item {Grp. gram.:m.}
\end{itemize}
Árvore de Timor.
\section{Coração-de-nossa-senhora}
\begin{itemize}
\item {Grp. gram.:m.}
\end{itemize}
\begin{itemize}
\item {Utilização:Bras}
\end{itemize}
Planta dos jardins.
\section{Coráceos}
\begin{itemize}
\item {Grp. gram.:m. pl.}
\end{itemize}
\begin{itemize}
\item {Proveniência:(Do gr. \textunderscore korax\textunderscore )}
\end{itemize}
Nome zoológico da família dos corvos.
\section{Coracoidal}
\begin{itemize}
\item {Grp. gram.:adj.}
\end{itemize}
\begin{itemize}
\item {Utilização:Anat.}
\end{itemize}
\begin{itemize}
\item {Proveniência:(De \textunderscore coracoide\textunderscore )}
\end{itemize}
Diz-se do ligamento que, aproximando-se da apóphyse coracoide, converte em orifício a borda superior da omoplata.
\section{Coracoide}
\begin{itemize}
\item {Grp. gram.:adj.}
\end{itemize}
\begin{itemize}
\item {Utilização:Anat.}
\end{itemize}
\begin{itemize}
\item {Grp. gram.:M.}
\end{itemize}
\begin{itemize}
\item {Proveniência:(Do gr. \textunderscore korax\textunderscore  + \textunderscore eidos\textunderscore )}
\end{itemize}
Recurvo.
Diz-se da apóphyse, que termina exteriormente a borda superior da omoplata.
Nome, dado por alguns anatómicos ao acrómio.
\section{Coracoídeo}
\begin{itemize}
\item {Grp. gram.:adj.}
\end{itemize}
\begin{itemize}
\item {Grp. gram.:M.}
\end{itemize}
\begin{itemize}
\item {Proveniência:(Do gr. \textunderscore korax\textunderscore  + \textunderscore eidos\textunderscore )}
\end{itemize}
O mesmo que \textunderscore coracoidal\textunderscore .
Diz-se de um osso independente, na espádua de alguns animaes.
Apóphyse da omoplata.
\section{Coracoideu}
\begin{itemize}
\item {Grp. gram.:adj.}
\end{itemize}
(V.coracoídeo)
\section{Coraçones}
\begin{itemize}
\item {Grp. gram.:m. pl.}
\end{itemize}
\begin{itemize}
\item {Utilização:Ant.}
\end{itemize}
Habitantes de Coraçan, na Pérsia. Cf. \textunderscore Tombo do Estado da Índia\textunderscore , 225.
\section{Coracora}
\begin{itemize}
\item {Grp. gram.:f.}
\end{itemize}
\begin{itemize}
\item {Proveniência:(Do ár. \textunderscore corcor\textunderscore )}
\end{itemize}
Embarcação asiática, o mesmo que \textunderscore corocora\textunderscore .
\section{Coraçudo}
\begin{itemize}
\item {Grp. gram.:adj.}
\end{itemize}
\begin{itemize}
\item {Utilização:Ant.}
\end{itemize}
\begin{itemize}
\item {Utilização:Prov.}
\end{itemize}
\begin{itemize}
\item {Proveniência:(De \textunderscore coração\textunderscore )}
\end{itemize}
Corajoso, valente. Cf. G. Vicente, I, 169.
Que tem coração duro, insensível.
\section{Còradamente}
\begin{itemize}
\item {Grp. gram.:adv.}
\end{itemize}
\begin{itemize}
\item {Proveniência:(De \textunderscore còrado\textunderscore )}
\end{itemize}
Com côr.
\section{Còradinha}
\begin{itemize}
\item {Grp. gram.:f.}
\end{itemize}
\begin{itemize}
\item {Utilização:Prov.}
\end{itemize}
\begin{itemize}
\item {Utilização:alent.}
\end{itemize}
Dança de roda.
\section{Còrado}
\begin{itemize}
\item {Grp. gram.:adj.}
\end{itemize}
\begin{itemize}
\item {Utilização:Fig.}
\end{itemize}
\begin{itemize}
\item {Proveniência:(De \textunderscore còrar\textunderscore )}
\end{itemize}
Que têm côr viva; rubicundo.
Vermelho.
Cheio de pudor; envergonhado.
\section{Còradoiro}
\begin{itemize}
\item {Grp. gram.:m.}
\end{itemize}
Acto de còrar (roupa, cera, etc.).
Lugar, em que se estende, para se còrar, (a roupa, cera, etc.).
\section{Còradouro}
\begin{itemize}
\item {Grp. gram.:m.}
\end{itemize}
Acto de còrar (roupa, cera, etc.).]
Lugar, em que se estende, para se còrar, (a roupa, cera, etc.).
\section{Coragem}
\begin{itemize}
\item {Grp. gram.:f.}
\end{itemize}
\begin{itemize}
\item {Utilização:Gír.}
\end{itemize}
\begin{itemize}
\item {Proveniência:(Do lat. \textunderscore cor\textunderscore )}
\end{itemize}
Firmeza, energia, deante do perigo.
Intrepidez.
Perseverança.
Ousadia, desembaraço.
Dinheiro.
\section{Còragem}
\begin{itemize}
\item {Grp. gram.:f.}
\end{itemize}
\begin{itemize}
\item {Utilização:Prov.}
\end{itemize}
(V.còradoiro)
As côres do rosto. (Colhido em Turquel)
\section{Coragento}
\begin{itemize}
\item {Grp. gram.:adj.}
\end{itemize}
O mesmo que \textunderscore corajoso\textunderscore .
\section{Coragiás}
\begin{itemize}
\item {Grp. gram.:m. pl.}
\end{itemize}
Tríbo de Índios do Brasil entre os rios Tocantins e Araguaia.
\section{Corágicos}
\begin{itemize}
\item {Grp. gram.:m. pl.}
\end{itemize}
\begin{itemize}
\item {Proveniência:(Do lat. \textunderscore choragium\textunderscore )}
\end{itemize}
Monumentos que, entre os Gregos, se erguiam aos mestres de côro, premiados.
\section{Coraixitas}
\begin{itemize}
\item {Grp. gram.:m. pl.}
\end{itemize}
Tríbo árabe, a que pertencia Mafoma.
\section{Coraixítico}
\begin{itemize}
\item {Grp. gram.:adj.}
\end{itemize}
\begin{itemize}
\item {Grp. gram.:M.}
\end{itemize}
Relativo aos Coraixitas.
Dialecto, que se tornou a língua geral dos Árabes.
\section{Corajado}
\begin{itemize}
\item {Grp. gram.:adj.}
\end{itemize}
\begin{itemize}
\item {Utilização:Bras. de Minas}
\end{itemize}
Corajoso.
\section{Corajosamente}
\begin{itemize}
\item {Grp. gram.:adv.}
\end{itemize}
De modo corajoso.
Com coragem.
\section{Corajoso}
\begin{itemize}
\item {Grp. gram.:adj.}
\end{itemize}
Que tem coragem.
Em que há coragem.
\section{Coral}
\begin{itemize}
\item {Grp. gram.:m.}
\end{itemize}
\begin{itemize}
\item {Utilização:Fig.}
\end{itemize}
\begin{itemize}
\item {Utilização:Pop.}
\end{itemize}
\begin{itemize}
\item {Grp. gram.:Pl.}
\end{itemize}
\begin{itemize}
\item {Proveniência:(Lat. \textunderscore coralium\textunderscore )}
\end{itemize}
Produção calcária, ramosa e geralmente vermelha, que fórma o éixo de certos pólypos.
Côr vermelha.
Vivacidade, esperteza: \textunderscore aquella pequena é um coral\textunderscore .
Excrescências carnosas e vermelhas da cabeça de certos animais.
\section{Coral}
\begin{itemize}
\item {Grp. gram.:adj.}
\end{itemize}
\begin{itemize}
\item {Grp. gram.:M.}
\end{itemize}
Relativo a côro: \textunderscore canto coral\textunderscore .
Canto em côro; canto coral.
\section{Coral}
\begin{itemize}
\item {Grp. gram.:f.}
\end{itemize}
Pequena cobra venenosa da América.
\section{Coral-bóla}
\begin{itemize}
\item {Grp. gram.:f.}
\end{itemize}
Serpente do Brasil.
\section{Coraleira}
\begin{itemize}
\item {Grp. gram.:f.}
\end{itemize}
Árvore, cujas flôres imitam os coraes.
Embarcação, para a pesca do coral.
\section{Coraleiro}
\begin{itemize}
\item {Grp. gram.:adj.}
\end{itemize}
\begin{itemize}
\item {Grp. gram.:M.}
\end{itemize}
Relativo á pesca do coral.
Pescador de coral.
O mesmo que \textunderscore coraleira\textunderscore .
\section{Coraliários}
\begin{itemize}
\item {Grp. gram.:m. pl.}
\end{itemize}
\begin{itemize}
\item {Proveniência:(Do lat. \textunderscore coralium\textunderscore )}
\end{itemize}
Classe de pólypos, a que pertencem os coraes.
\section{Coralim}
\begin{itemize}
\item {Grp. gram.:m.}
\end{itemize}
Espécie de serpente do Brasil.
\section{Coralina}
\begin{itemize}
\item {Grp. gram.:f.}
\end{itemize}
\begin{itemize}
\item {Proveniência:(De \textunderscore coralino\textunderscore )}
\end{itemize}
Encrustação calcária e variegada de uma espécie de alga.
\section{Coralina}
\begin{itemize}
\item {Grp. gram.:f.}
\end{itemize}
(V.cornalina)
\section{Coralino}
\begin{itemize}
\item {Grp. gram.:adj.}
\end{itemize}
Que tem côr de coral.
\section{Çorame}
\begin{itemize}
\item {Grp. gram.:m.}
\end{itemize}
O mesmo que \textunderscore cerome\textunderscore .
\section{Còrante}
\begin{itemize}
\item {Grp. gram.:adj.}
\end{itemize}
Que cora ou torna vermelho: \textunderscore matéria corante\textunderscore . Cf. Castilho, \textunderscore Fastos\textunderscore , II, 364.
\section{Còrar}
\begin{itemize}
\item {Grp. gram.:v. t.}
\end{itemize}
\begin{itemize}
\item {Utilização:Fig.}
\end{itemize}
\begin{itemize}
\item {Grp. gram.:V. i.}
\end{itemize}
\begin{itemize}
\item {Proveniência:(Do lat. \textunderscore colorare\textunderscore )}
\end{itemize}
Dar côr a.
Tingir.
Branquear (roupa, cera, etc.).
Disfarçar, attenuar.
Enrubescer nas faces.
Envergonhar-se.
\section{Corás}
\begin{itemize}
\item {Grp. gram.:m.}
\end{itemize}
Peixe do Brasil.
\section{Coraulo}
\begin{itemize}
\item {Grp. gram.:m.}
\end{itemize}
Aulete, que, entre os Gregos, acompanhava o côro.
\section{Corazil}
\begin{itemize}
\item {Grp. gram.:m.}
\end{itemize}
\begin{itemize}
\item {Utilização:Ant.}
\end{itemize}
Parte do porco, que se pagava em pensão, nos antigos prazos e foraes.
Manta de toicinho.
\section{Corbelha}
\begin{itemize}
\item {fónica:bê}
\end{itemize}
\begin{itemize}
\item {Grp. gram.:f.}
\end{itemize}
\begin{itemize}
\item {Proveniência:(Do lat. \textunderscore corbicula\textunderscore )}
\end{itemize}
Cestinho de vimes, vidro ou madeira, para doces, frutas ou brindes:«\textunderscore mil frutos, mil corbelhas, mil compotas\textunderscore ». Dinis, \textunderscore Hyssope\textunderscore , XXXIV.«\textunderscore Entretecem-se corbelhas\textunderscore ». Castilho, \textunderscore Geórg.\textunderscore , 33.
Lugar, em que se collocam os brindes de núpcias: \textunderscore viam-se muitos brindes na corbelha da noiva\textunderscore .
\section{Corbol}
\begin{itemize}
\item {Grp. gram.:m.}
\end{itemize}
Árvore da Índia portuguesa.
\section{Corca}
\begin{itemize}
\item {Grp. gram.:f.}
\end{itemize}
\begin{itemize}
\item {Utilização:Prov.}
\end{itemize}
\begin{itemize}
\item {Utilização:beir.}
\end{itemize}
O mesmo que \textunderscore alcorca\textunderscore .
Funda depressão natural de terreno, formada pelas águas pluviaes ou pelo trânsito de carros.
Córrego, caminho estreito entre montes.
\section{Corça}
\begin{itemize}
\item {fónica:côr}
\end{itemize}
\begin{itemize}
\item {Grp. gram.:f.}
\end{itemize}
\begin{itemize}
\item {Proveniência:(De \textunderscore côrço\textunderscore )}
\end{itemize}
Espécie de antílope, mais pequeno que a cerva, com que alguns o confundiram.--Attinge apenas o tamanho de uma cabra, e por isso a consideraram também cabra silvestre.
\section{Corça}
\begin{itemize}
\item {fónica:côr}
\end{itemize}
\begin{itemize}
\item {Grp. gram.:f.}
\end{itemize}
\begin{itemize}
\item {Utilização:Prov.}
\end{itemize}
\begin{itemize}
\item {Utilização:trasm.}
\end{itemize}
Zorra rudimentar, grosseira, para transporte de cantaria.
(Relaciona-se com o lat. \textunderscore cursus\textunderscore , como \textunderscore côrço\textunderscore ?)
\section{Çorça}
\begin{itemize}
\item {Grp. gram.:f.}
\end{itemize}
\begin{itemize}
\item {Utilização:Ant.}
\end{itemize}
Espécie de gaiola? capoeira?«\textunderscore ...passarinhos trazidos em gaiolas e çorças\textunderscore ». Fern. Mendes Pinto, \textunderscore Peregrin.\textunderscore , LXXVIII.
(Cp. \textunderscore sorça\textunderscore  e \textunderscore surça\textunderscore )
\section{Corcalhé}
\begin{itemize}
\item {Grp. gram.:m.}
\end{itemize}
\begin{itemize}
\item {Utilização:Prov.}
\end{itemize}
\begin{itemize}
\item {Utilização:trasm.}
\end{itemize}
\begin{itemize}
\item {Proveniência:(T. onom.)}
\end{itemize}
O mesmo que \textunderscore codorniz\textunderscore .
\section{Corcel}
\begin{itemize}
\item {Grp. gram.:m.}
\end{itemize}
Cavallo, que corre muito.
Cavallo de campanha.
(Cast. \textunderscore corcel\textunderscore )
\section{Corcha}
\begin{itemize}
\item {fónica:côr}
\end{itemize}
\begin{itemize}
\item {Grp. gram.:f.}
\end{itemize}
\begin{itemize}
\item {Utilização:Prov.}
\end{itemize}
\begin{itemize}
\item {Utilização:beir.}
\end{itemize}
(Corr. de \textunderscore colcha\textunderscore )
\section{Corcha}
\begin{itemize}
\item {fónica:côr}
\end{itemize}
\begin{itemize}
\item {Grp. gram.:f.}
\end{itemize}
Cortiça.
Casca de árvore.
Rolha de madeira, com que se tapam as bocas das peças de artilharia, para que nellas se não introduza água.
(Cast. \textunderscore corcha\textunderscore )
\section{Corchete}
\begin{itemize}
\item {fónica:chê}
\end{itemize}
\begin{itemize}
\item {Grp. gram.:m.}
\end{itemize}
O mesmo ou melhor que \textunderscore colchete\textunderscore .
\section{Corchingo}
\begin{itemize}
\item {Grp. gram.:m.}
\end{itemize}
Árvore da Índia portuguesa.
\section{Corcho}
\begin{itemize}
\item {fónica:côr}
\end{itemize}
\begin{itemize}
\item {Grp. gram.:m.}
\end{itemize}
\begin{itemize}
\item {Utilização:Prov.}
\end{itemize}
\begin{itemize}
\item {Utilização:Prov.}
\end{itemize}
\begin{itemize}
\item {Utilização:dur.}
\end{itemize}
\begin{itemize}
\item {Utilização:Prov.}
\end{itemize}
\begin{itemize}
\item {Utilização:alg.}
\end{itemize}
\begin{itemize}
\item {Utilização:Prov.}
\end{itemize}
\begin{itemize}
\item {Utilização:alg.}
\end{itemize}
\begin{itemize}
\item {Utilização:Prov.}
\end{itemize}
\begin{itemize}
\item {Utilização:trasm.}
\end{itemize}
Vaso de cortiça, usado no Alentejo.
Enfiada de peças de cortiça, com que se fazem fluctuar apparelhos de pesca.
Cardume, peixes em bando.
Cortiço das abelhas.
Pequena tábua, em que os serventes de pedreiro levam a argamassa.
Espécie de caixote, em que se põem as crianças, quando já pódem sentar-se.
(Cast. \textunderscore corcho\textunderscore )
\section{Córchoro}
\begin{itemize}
\item {Grp. gram.:m.}
\end{itemize}
Planta tiliácea.
\section{Corço}
\begin{itemize}
\item {fónica:côr}
\end{itemize}
\begin{itemize}
\item {Grp. gram.:m.}
\end{itemize}
\begin{itemize}
\item {Proveniência:(Do lat. \textunderscore cursus\textunderscore , de \textunderscore currere\textunderscore ? Em tal caso, deveríamos escrever \textunderscore corso\textunderscore )}
\end{itemize}
Filho da corça.
\section{Corcódea}
\begin{itemize}
\item {Grp. gram.:f.}
\end{itemize}
\begin{itemize}
\item {Utilização:Prov.}
\end{itemize}
Casca de pinheiro.
\section{Corcoroca}
\begin{itemize}
\item {Grp. gram.:f.}
\end{itemize}
\begin{itemize}
\item {Utilização:Bras}
\end{itemize}
Pequeno peixe achatado e saboroso da baía do Rio-de-Janeiro.
\section{Corcova}
\begin{itemize}
\item {Grp. gram.:f.}
\end{itemize}
\begin{itemize}
\item {Proveniência:(De \textunderscore corcovar\textunderscore )}
\end{itemize}
Curva saliente.
Carcunda.
Caminho, que dá volta.
\section{Corcovado}
\begin{itemize}
\item {Grp. gram.:m.}
\end{itemize}
\begin{itemize}
\item {Utilização:Bras}
\end{itemize}
Ave das regiões do Amazonas.
\section{Corcovado-uru}
\begin{itemize}
\item {Grp. gram.:m.}
\end{itemize}
Ave do Brasil.
\section{Corcovar}
\begin{itemize}
\item {Grp. gram.:v. t.}
\end{itemize}
\begin{itemize}
\item {Proveniência:(Do lat. \textunderscore concurvare\textunderscore ?)}
\end{itemize}
Curvar.
Dar fórma arqueada a.
\section{Corcovear}
\begin{itemize}
\item {Grp. gram.:v. i.}
\end{itemize}
Dar corcovos.
\section{Corcovo}
\begin{itemize}
\item {fónica:cô}
\end{itemize}
\begin{itemize}
\item {Grp. gram.:m.}
\end{itemize}
Salto do cavallo, arqueando o dorso.
Montículo ou elevação de terreno.
(Da mesma or. que \textunderscore corcova\textunderscore )
\section{Corculhér}
\begin{itemize}
\item {Grp. gram.:f.}
\end{itemize}
Espécie de cotovia.
(Cp. \textunderscore corcalhé\textunderscore  e \textunderscore carcalhota\textunderscore )
\section{Córculo}
\begin{itemize}
\item {Grp. gram.:m.}
\end{itemize}
\begin{itemize}
\item {Proveniência:(Lat. \textunderscore corculum\textunderscore )}
\end{itemize}
O mesmo que \textunderscore embryão\textunderscore .
\section{Corcunda}
\begin{itemize}
\item {Grp. gram.:f.}
\end{itemize}
(V.carcunda)
\section{Corcunda}
\begin{itemize}
\item {Grp. gram.:f.}
\end{itemize}
Nome, que em Lisbôa, se dá ao \textunderscore capatão\textunderscore .
\section{Corda}
\begin{itemize}
\item {Grp. gram.:f.}
\end{itemize}
\begin{itemize}
\item {Utilização:Fig.}
\end{itemize}
\begin{itemize}
\item {Utilização:Geom.}
\end{itemize}
\begin{itemize}
\item {Utilização:Anat.}
\end{itemize}
\begin{itemize}
\item {Utilização:Gír.}
\end{itemize}
\begin{itemize}
\item {Utilização:Escol.}
\end{itemize}
\begin{itemize}
\item {Grp. gram.:Pl.}
\end{itemize}
\begin{itemize}
\item {Proveniência:(Lat. \textunderscore chorda\textunderscore )}
\end{itemize}
Peça de fios, unidos e torcidos uns sôbre os outros.
Fio de tripa, usado em certos instrumentos.
Arame, liso ou torcido, que, vibrado, produz o som de certos instrumentos.
Lâmina de aço, que dá movimento ás rodas dos relógios.
Série: \textunderscore os rapazes formaram corda\textunderscore .
Linha recta, que liga dois pontos da circunferência, não passando pelo centro.
Prega membranosa (da glote).
Cordão de oiro.
\textunderscore Corda de vento\textunderscore , vento forte, que sopra por muito tempo, numa só direcção.
\textunderscore Estar com a corda na garganta\textunderscore , estar em apertos, em circunstâncias graves.
\textunderscore Corda sensível\textunderscore , o lado fraco do carácter de alguém; predilecção.
\textunderscore Roer a corda\textunderscore , não cumprir o que prometeu.
\textunderscore Dar corda a um relógio\textunderscore , enrolar a lâmina, chamada corda de relógio, para que o maquinismo dêste continue ou comece a mover-se.
\textunderscore Dar corda a alguém\textunderscore , dar-lhe pretexto, para que fale muito; provocá-lo.
\textunderscore Andar á corda\textunderscore , andar ao arbítrio de alguém.
Preparar-se todos os dias para dar lição e não lha pedir o professor.
\textunderscore Mosquitos por cordas\textunderscore , tumulto, motim, balbúrdia.
\section{Corda-cadeira}
\begin{itemize}
\item {Grp. gram.:f.}
\end{itemize}
Planta trepadeira da ilha de San-Thomé.
\section{Cordada}
\begin{itemize}
\item {Grp. gram.:f.}
\end{itemize}
\begin{itemize}
\item {Utilização:Prov.}
\end{itemize}
\begin{itemize}
\item {Utilização:minh.}
\end{itemize}
\begin{itemize}
\item {Proveniência:(De \textunderscore corda\textunderscore )}
\end{itemize}
Grande feixe de linho, ainda não maçado.
\section{Corda-de-água}
\begin{itemize}
\item {Grp. gram.:f.}
\end{itemize}
Planta trepadeira santhomense, cujo suco mata a sêde.
O mesmo que \textunderscore aguaceiro\textunderscore .
\section{Corda-dorsal}
\begin{itemize}
\item {Grp. gram.:f.}
\end{itemize}
O mesmo que \textunderscore notocórdio\textunderscore .
\section{Cordagem}
\begin{itemize}
\item {Grp. gram.:f.}
\end{itemize}
(V.cordame)
\section{Cordal}
\begin{itemize}
\item {Grp. gram.:adj.}
\end{itemize}
Relativo á corda-dorsal.
\section{Cordame}
\begin{itemize}
\item {Grp. gram.:m.}
\end{itemize}
\begin{itemize}
\item {Proveniência:(De \textunderscore corda\textunderscore )}
\end{itemize}
Reunião de cordas.
Conjunto dos cabos de um navio.
\section{Cordante}
\begin{itemize}
\item {Grp. gram.:f.}
\end{itemize}
\begin{itemize}
\item {Utilização:Gír.}
\end{itemize}
\begin{itemize}
\item {Proveniência:(De \textunderscore corda\textunderscore )}
\end{itemize}
Fôrca.
\section{Cordão}
\begin{itemize}
\item {Grp. gram.:m.}
\end{itemize}
\begin{itemize}
\item {Utilização:Bras. do Rio}
\end{itemize}
\begin{itemize}
\item {Proveniência:(Fr. \textunderscore cordon\textunderscore )}
\end{itemize}
Corda delgada.
Fileira.
Série de postos militares, para evitar um contágio.
Objecto ou órgão, que tem semelhança com uma pequena corda.
Certo cortejo ou grupo carnavalesco.
Parte dos animaes bovídeos, ao longo do perineu.
\section{Cordão-de-frade}
\begin{itemize}
\item {Grp. gram.:m.}
\end{itemize}
\begin{itemize}
\item {Utilização:Bras}
\end{itemize}
Planta labiada, medicinal, (\textunderscore leonotis nepetifolia\textunderscore ).
\section{Cordão-de-san-francisco}
\begin{itemize}
\item {Grp. gram.:m.}
\end{itemize}
\begin{itemize}
\item {Utilização:Bras}
\end{itemize}
Planta; o mesmo que \textunderscore cordão-de-frade\textunderscore ?
\section{Corda-pau}
\begin{itemize}
\item {Grp. gram.:f.}
\end{itemize}
Planta trepadeira da ilha de San-Thomé.
\section{Cordapso}
\begin{itemize}
\item {Grp. gram.:m.}
\end{itemize}
\begin{itemize}
\item {Utilização:Med.}
\end{itemize}
Cólica violenta.
\section{Cordato}
\begin{itemize}
\item {Grp. gram.:adj.}
\end{itemize}
\begin{itemize}
\item {Proveniência:(Lat. \textunderscore cordatus\textunderscore )}
\end{itemize}
Prudente.
Que tem bom senso.
\section{Corda-ubuá}
\begin{itemize}
\item {Grp. gram.:f.}
\end{itemize}
Árvore da ilha de San-Thomé.
\section{Cordeação}
\begin{itemize}
\item {Grp. gram.:f.}
\end{itemize}
Acto de cordear.
\section{Cordeador}
\begin{itemize}
\item {Grp. gram.:m.}
\end{itemize}
\begin{itemize}
\item {Utilização:Bras. do N}
\end{itemize}
\begin{itemize}
\item {Proveniência:(De \textunderscore corda\textunderscore )}
\end{itemize}
O mesmo que \textunderscore arruador\textunderscore .
\section{Cordear}
\begin{itemize}
\item {Grp. gram.:v. t.}
\end{itemize}
\begin{itemize}
\item {Utilização:Náut.}
\end{itemize}
\begin{itemize}
\item {Proveniência:(De \textunderscore corda\textunderscore )}
\end{itemize}
Medir com corda.
O mesmo que \textunderscore alinhar\textunderscore .
Bracear (vêrgas), de modo que offereçam pouca resistência ao vento.
\section{Cordeca}
\begin{itemize}
\item {Grp. gram.:f.}
\end{itemize}
\begin{itemize}
\item {Utilização:Prov.}
\end{itemize}
\begin{itemize}
\item {Proveniência:(Do lat. \textunderscore cortex\textunderscore )}
\end{itemize}
Casca de pinheiro, corcódea.
\section{Cordeira}
\begin{itemize}
\item {Grp. gram.:f.}
\end{itemize}
\begin{itemize}
\item {Proveniência:(De \textunderscore cordeiro\textunderscore )}
\end{itemize}
Filha, ainda nova, da ovelha.
Pelle da cordeira:«\textunderscore ...beca forrada de cordeiras brancas\textunderscore ». Z. Brandão, \textunderscore Pêro da Covilhan\textunderscore .
\section{Cordeirinho}
\begin{itemize}
\item {Grp. gram.:m.}
\end{itemize}
\begin{itemize}
\item {Utilização:Prov.}
\end{itemize}
\begin{itemize}
\item {Utilização:dur.}
\end{itemize}
\begin{itemize}
\item {Proveniência:(De \textunderscore cordeiro\textunderscore )}
\end{itemize}
Planta de fôlhas brancas, curtas e macias, que vegeta nas arribas do mar.
\section{Cordeiro}
\begin{itemize}
\item {Grp. gram.:m.}
\end{itemize}
\begin{itemize}
\item {Utilização:Fig.}
\end{itemize}
Filho da ovelha, ainda novo e tenro.
Indivíduo manso e innocente.
\section{Cordel}
\begin{itemize}
\item {Grp. gram.:m.}
\end{itemize}
\begin{itemize}
\item {Proveniência:(De \textunderscore corda\textunderscore )}
\end{itemize}
Cordão, guita, barbante.
\textunderscore Livraria de cordel\textunderscore , dizia-se a livraria, que expunha os seus folhetos pendurados á porta em cordel.
\textunderscore Literatura de cordel\textunderscore , conjunto de publicações, de pouco ou nenhum valor.
\section{Cordelejo}
\begin{itemize}
\item {Grp. gram.:m.}
\end{itemize}
\begin{itemize}
\item {Utilização:ant.}
\end{itemize}
\begin{itemize}
\item {Utilização:Fam.}
\end{itemize}
\begin{itemize}
\item {Proveniência:(De \textunderscore cordel\textunderscore )}
\end{itemize}
Reprehensão áspera.
\section{Cordelinhos}
\begin{itemize}
\item {Grp. gram.:m. pl.}
\end{itemize}
\begin{itemize}
\item {Utilização:Fig.}
\end{itemize}
\begin{itemize}
\item {Proveniência:(De \textunderscore cordel\textunderscore , por anal. com os cordelinhos occultos, que fazem mover fantoches)}
\end{itemize}
Artes ou meios occultos, com que se encaminham certos negócios.
\section{Córdia}
\begin{itemize}
\item {Grp. gram.:f.}
\end{itemize}
\begin{itemize}
\item {Proveniência:(De \textunderscore Cordus\textunderscore , n. p. de um bot. al.)}
\end{itemize}
Gênero de plantas, que serve de typo ás cordiáceas.
\section{Cordíaca}
\begin{itemize}
\item {Grp. gram.:f.}
\end{itemize}
\begin{itemize}
\item {Proveniência:(Do lat. \textunderscore cordiacus\textunderscore )}
\end{itemize}
Doença no coração dos cavallos.
\section{Cordiáceas}
\begin{itemize}
\item {Grp. gram.:f. pl.}
\end{itemize}
\begin{itemize}
\item {Proveniência:(De \textunderscore córdia\textunderscore )}
\end{itemize}
Fam. de plantas intertropicaes, arbóreas e arbustivas.
\section{Cordial}
\begin{itemize}
\item {Grp. gram.:adj.}
\end{itemize}
\begin{itemize}
\item {Grp. gram.:M.}
\end{itemize}
\begin{itemize}
\item {Proveniência:(Do lat. \textunderscore cor\textunderscore , \textunderscore cordis\textunderscore )}
\end{itemize}
Relativo ao coração.
Affectuoso.
Sincero, franco: \textunderscore agradecimento cordial\textunderscore .
Bebida ou medicamento, que conforta ou fortalece.
\section{Cordialidade}
\begin{itemize}
\item {Grp. gram.:f.}
\end{itemize}
\begin{itemize}
\item {Proveniência:(De \textunderscore cordial\textunderscore )}
\end{itemize}
Affeição sincera.
Amenidade e franqueza de trato.
Sinceridade.
\section{Cordialmente}
\begin{itemize}
\item {Grp. gram.:adv.}
\end{itemize}
De modo cordial.
\section{Cordifólia}
\begin{itemize}
\item {Grp. gram.:f.}
\end{itemize}
\begin{itemize}
\item {Proveniência:(Do lat. \textunderscore cor\textunderscore , \textunderscore cordis\textunderscore  + \textunderscore folium\textunderscore )}
\end{itemize}
Variedade de videira americana.
\section{Cordifoliado}
\begin{itemize}
\item {Grp. gram.:adj.}
\end{itemize}
\begin{itemize}
\item {Utilização:Bot.}
\end{itemize}
\begin{itemize}
\item {Proveniência:(Do lat. \textunderscore cor\textunderscore , \textunderscore cordis\textunderscore  + \textunderscore folium\textunderscore )}
\end{itemize}
Que tem fôlhas em fórma de coração.
\section{Cordifólio}
\begin{itemize}
\item {Grp. gram.:adj.}
\end{itemize}
O mesmo que \textunderscore cordifoliado\textunderscore .
\section{Cordiforme}
\begin{itemize}
\item {Grp. gram.:adj.}
\end{itemize}
\begin{itemize}
\item {Proveniência:(Do lat. \textunderscore cor\textunderscore  + \textunderscore forma\textunderscore )}
\end{itemize}
Que tem fórma de coração.
\section{Cordilha}
\begin{itemize}
\item {Grp. gram.:f.}
\end{itemize}
\begin{itemize}
\item {Proveniência:(De \textunderscore corda\textunderscore )}
\end{itemize}
Chama-se assim o atum quando sái do ovo, por têr o feitio de uma pequena corda.
\section{Cordilheira}
\begin{itemize}
\item {Grp. gram.:f.}
\end{itemize}
\begin{itemize}
\item {Proveniência:(De \textunderscore corda\textunderscore )}
\end{itemize}
Série, cadeia, de montes.
\section{Cordilheiro}
\begin{itemize}
\item {Grp. gram.:adj.}
\end{itemize}
\begin{itemize}
\item {Utilização:Açor}
\end{itemize}
Velhaco.
Que tem más qualidades.
(Corr. de \textunderscore quadrilheiro\textunderscore ?)
\section{Cordina}
\begin{itemize}
\item {Grp. gram.:f.}
\end{itemize}
\begin{itemize}
\item {Utilização:Prov.}
\end{itemize}
Baraço, com que se joga o pião. (Colhido em Turquel)
(Por \textunderscore cordinha\textunderscore , de \textunderscore corda\textunderscore )
\section{Cordite}
\begin{itemize}
\item {Grp. gram.:f.}
\end{itemize}
Explosivo inglês.
\section{Cordo}
\begin{itemize}
\item {Grp. gram.:adj.}
\end{itemize}
\begin{itemize}
\item {Utilização:Des.}
\end{itemize}
(Contr. de \textunderscore cordato\textunderscore . Cf. \textunderscore Port. Mon. Hist., Script.\textunderscore , 241)
\section{Cordoada}
\begin{itemize}
\item {Grp. gram.:f.}
\end{itemize}
\begin{itemize}
\item {Proveniência:(De \textunderscore cordão\textunderscore )}
\end{itemize}
Pancada com cordão.
Cordoalha.
Aguaceiro, corda-de-água.
\section{Cordoajamento}
\begin{itemize}
\item {Grp. gram.:m.}
\end{itemize}
\begin{itemize}
\item {Utilização:Ant.}
\end{itemize}
O mesmo que \textunderscore cordame\textunderscore .
\section{Cordoalha}
\begin{itemize}
\item {Grp. gram.:f.}
\end{itemize}
\begin{itemize}
\item {Proveniência:(De \textunderscore cordão\textunderscore )}
\end{itemize}
Conjunto de cordas de vária espécie.
Cordame.
\section{Cordoaria}
\begin{itemize}
\item {Grp. gram.:f.}
\end{itemize}
\begin{itemize}
\item {Proveniência:(De \textunderscore cordão\textunderscore )}
\end{itemize}
Fábrica de cordas.
Lugar, onde se vendem cordas.
Commércio de cordas.
\section{Cordoeira}
\begin{itemize}
\item {Grp. gram.:f.}
\end{itemize}
Nome vulgar da antidesma.
\section{Cordoeiro}
\begin{itemize}
\item {Grp. gram.:m.}
\end{itemize}
\begin{itemize}
\item {Proveniência:(De \textunderscore cordão\textunderscore )}
\end{itemize}
Fabricante ou vendedor de cordas.
\section{Cordómetro}
\begin{itemize}
\item {Grp. gram.:m.}
\end{itemize}
\begin{itemize}
\item {Proveniência:(Do gr. \textunderscore khorda\textunderscore  + \textunderscore metron\textunderscore )}
\end{itemize}
Instrumento, com que se mede a grossura das cordas.
\section{Cordovaneiro}
\begin{itemize}
\item {Grp. gram.:m.}
\end{itemize}
Aquelle, que fabríca ou vende cordovão.
\section{Cordovão}
\begin{itemize}
\item {Grp. gram.:m.}
\end{itemize}
Coiro de cabra, curtido, e preparado especialmente para calçado.
(Cast. \textunderscore cordobán\textunderscore )
\section{Cordoveias}
\begin{itemize}
\item {Grp. gram.:f. pl.}
\end{itemize}
\begin{itemize}
\item {Proveniência:(De \textunderscore corda\textunderscore  + \textunderscore veia\textunderscore )}
\end{itemize}
Designação vulgar das veias e tendões do pescoço.
\section{Cordovês}
\begin{itemize}
\item {Grp. gram.:adj.}
\end{itemize}
\begin{itemize}
\item {Grp. gram.:M.}
\end{itemize}
Relativo a Córdova.
Indivíduo natural de Córdova.
\section{Cordovesa}
\begin{itemize}
\item {Grp. gram.:adj. f.}
\end{itemize}
\begin{itemize}
\item {Proveniência:(De \textunderscore Córdova\textunderscore , n. p.)}
\end{itemize}
Diz-se de uma espécie de azeitona, grande e carnuda.
\section{Cordovil}
\begin{itemize}
\item {Grp. gram.:adj.}
\end{itemize}
\begin{itemize}
\item {Proveniência:(De \textunderscore Córdova\textunderscore , n. p.)}
\end{itemize}
Diz-se de uma espécie de oliveira minhota e alentejana.
\section{Cordura}
\begin{itemize}
\item {Grp. gram.:f.}
\end{itemize}
Qualidade daquillo ou de quem é côrdo.
Sensatez.
Gravidade.
\section{Coré}
\begin{itemize}
\item {Grp. gram.:m.}
\end{itemize}
Árvore brasileira.
\section{Coreano}
\begin{itemize}
\item {Grp. gram.:adj.}
\end{itemize}
\begin{itemize}
\item {Grp. gram.:M.}
\end{itemize}
Relativo a Coreia.
Habitante da Coreia.
Língua agglutinativa e bárbara, vernácula na Coreia.
\section{Co-redactor}
\begin{itemize}
\item {Grp. gram.:m.}
\end{itemize}
\begin{itemize}
\item {Proveniência:(De \textunderscore co...\textunderscore  + \textunderscore redactor\textunderscore )}
\end{itemize}
Aquelle, que redige com outrem.
\section{Coregia}
\begin{itemize}
\item {Grp. gram.:f.}
\end{itemize}
Cargo ou funções de \textunderscore corego\textunderscore .
\section{Corego}
\begin{itemize}
\item {Grp. gram.:m.}
\end{itemize}
\begin{itemize}
\item {Proveniência:(Gr. \textunderscore khoregos\textunderscore )}
\end{itemize}
Aquele que entre os Gregos custeava a despesa dos espectáculos.
Mestre de côro, entre os Gregos.
\section{Coregrafia}
\begin{itemize}
\item {Grp. gram.:f.}
\end{itemize}
\begin{itemize}
\item {Proveniência:(De \textunderscore corégrapho\textunderscore )}
\end{itemize}
Arte de compor bailados.
Arte de dançar.
\section{Corégrafo}
\begin{itemize}
\item {Grp. gram.:m.}
\end{itemize}
\begin{itemize}
\item {Proveniência:(Do gr. \textunderscore khoros\textunderscore  + \textunderscore graphein\textunderscore )}
\end{itemize}
Aquele que é versado em coregrafia.
\section{Coreia}
\begin{itemize}
\item {Grp. gram.:f.}
\end{itemize}
\begin{itemize}
\item {Proveniência:(Gr. \textunderscore khoreia\textunderscore )}
\end{itemize}
Doença, que obriga a movimentos contínuos de certos órgãos.
Nome de uma dança grega.
\section{Coreia}
\begin{itemize}
\item {Grp. gram.:f.}
\end{itemize}
Insecto hemíptero, que vive nas plantas, e que é também conhecido por \textunderscore percevejo da terra\textunderscore .
Planta primulácea.
\section{Coreico}
\begin{itemize}
\item {Grp. gram.:adj.}
\end{itemize}
Relativo á coreia^1.
\section{Coreiro}
\begin{itemize}
\item {Grp. gram.:m.}
\end{itemize}
Aquelle, que reza num côro ou é empregado num côro.
\section{Coreixa}
\begin{itemize}
\item {Grp. gram.:f.}
\end{itemize}
Ave pernalta, espécie de grou.
\section{Corema}
\begin{itemize}
\item {Grp. gram.:f.}
\end{itemize}
\begin{itemize}
\item {Proveniência:(Do gr. \textunderscore korema\textunderscore , vassoira)}
\end{itemize}
Gênero de pequenos arbustos, cujos frutos são acídulos levemente e medicinaes.
\section{Corenta}
\begin{itemize}
\item {Grp. gram.:adj.}
\end{itemize}
(Fórma pop. e ant. de \textunderscore quarenta\textunderscore )
\section{Coreografia}
\begin{itemize}
\item {Grp. gram.:f.}
\end{itemize}
O mesmo que \textunderscore coregrafia\textunderscore .
\section{Coreográfico}
\begin{itemize}
\item {Grp. gram.:adj.}
\end{itemize}
Relativo á coreografia.
\section{Coreógrafo}
\textunderscore m.\textunderscore  (e der.)
O mesmo que \textunderscore corégrafo\textunderscore , etc.
\section{Coresma}
\begin{itemize}
\item {Grp. gram.:f.}
\end{itemize}
\begin{itemize}
\item {Utilização:ant.}
\end{itemize}
\begin{itemize}
\item {Utilização:Pop.}
\end{itemize}
O mesmo que \textunderscore quaresma\textunderscore .
\section{Coreto}
\begin{itemize}
\item {fónica:corê}
\end{itemize}
\begin{itemize}
\item {Grp. gram.:m.}
\end{itemize}
\begin{itemize}
\item {Proveniência:(De \textunderscore côro\textunderscore )}
\end{itemize}
Espécie de côro, construido ao ar livre, para concertos musicaes.
\section{Coretus}
\begin{itemize}
\item {Grp. gram.:m. pl.}
\end{itemize}
Tríbo do alto Amazonas.
\section{Coreu}
\begin{itemize}
\item {Grp. gram.:m.}
\end{itemize}
\begin{itemize}
\item {Proveniência:(Lat. \textunderscore choraeus\textunderscore )}
\end{itemize}
Pé de um verso latino ou grego, composto de uma sílaba longa, seguida de outra breve.
\section{Corga}
\begin{itemize}
\item {Grp. gram.:f.}
\end{itemize}
O mesmo que \textunderscore corgo\textunderscore .
\section{Córgão}
\begin{itemize}
\item {Grp. gram.:m.}
\end{itemize}
\begin{itemize}
\item {Utilização:Mad}
\end{itemize}
O mesmo que \textunderscore corgo\textunderscore .
\section{Corgo}
\begin{itemize}
\item {Grp. gram.:m.}
\end{itemize}
\begin{itemize}
\item {Utilização:Pop.}
\end{itemize}
\begin{itemize}
\item {Utilização:Prov.}
\end{itemize}
\begin{itemize}
\item {Utilização:alent.}
\end{itemize}
\begin{itemize}
\item {Utilização:Prov.}
\end{itemize}
\begin{itemize}
\item {Utilização:alg.}
\end{itemize}
O mesmo que \textunderscore córrego\textunderscore .
Caminho apertado entre montes; corca.
Terra grossa e baixa, no sopé de encostas.
\section{Corgul}
\begin{itemize}
\item {Grp. gram.:m.}
\end{itemize}
Árvore da Índia portuguesa.
\section{Corgulhada}
\begin{itemize}
\item {Grp. gram.:f.}
\end{itemize}
\begin{itemize}
\item {Utilização:Prov.}
\end{itemize}
\begin{itemize}
\item {Utilização:beir.}
\end{itemize}
Cambada de chouriços ou de frutas, que se reúnem, para se pendurarem conjuntamente.
(Por \textunderscore cogulada\textunderscore , de \textunderscore cogular\textunderscore ?)
\section{Cori}
\begin{itemize}
\item {Grp. gram.:m.}
\end{itemize}
Mammífero roedor da ilha de Cuba.
\section{Coriáceo}
\begin{itemize}
\item {Grp. gram.:adj.}
\end{itemize}
\begin{itemize}
\item {Proveniência:(Lat. \textunderscore coriaceus\textunderscore )}
\end{itemize}
Semelhante a coiro.
Duro como coiro cru.
\section{Coriambo}
\begin{itemize}
\item {Grp. gram.:m.}
\end{itemize}
\begin{itemize}
\item {Proveniência:(Do gr. \textunderscore khoreios\textunderscore  + \textunderscore iambos\textunderscore )}
\end{itemize}
Pé de verso grego ou latino, formado de duas sílabas breves entre duas longas.
\section{Coriandro}
\begin{itemize}
\item {Grp. gram.:m.}
\end{itemize}
\begin{itemize}
\item {Proveniência:(Lat. \textunderscore coriandrum\textunderscore )}
\end{itemize}
Designação scientífica do coentro.
\section{Coriária}
\begin{itemize}
\item {Grp. gram.:f.}
\end{itemize}
\begin{itemize}
\item {Proveniência:(Do lat. \textunderscore coriarius\textunderscore )}
\end{itemize}
Substância, que se emprega no curtume dos coiros.
Planta, que produz essa substância.
\section{Coriariáceas}
\begin{itemize}
\item {Grp. gram.:f. pl.}
\end{itemize}
O mesmo ou melhor que \textunderscore coriaríadas\textunderscore .
\section{Coriaríadas}
\begin{itemize}
\item {Grp. gram.:f. pl.}
\end{itemize}
Fam. de plantas, que tem por typo a coriária.
\section{Coriarina}
\begin{itemize}
\item {Grp. gram.:f.}
\end{itemize}
Alcaloide, encontrado na coriária.
\section{Corica}
\begin{itemize}
\item {Grp. gram.:f.}
\end{itemize}
Espécie de papagaio.
\section{Coricida}
\begin{itemize}
\item {Grp. gram.:m.}
\end{itemize}
\begin{itemize}
\item {Proveniência:(Do lat. \textunderscore corium\textunderscore  + \textunderscore caedere\textunderscore )}
\end{itemize}
Substância, que embrandece ou corrói o coiro ou pelle endurecida.
Medicamento, que destrói os callos.
\section{Córico}
\begin{itemize}
\item {Grp. gram.:adj.}
\end{itemize}
\begin{itemize}
\item {Proveniência:(De \textunderscore côro\textunderscore )}
\end{itemize}
Dizia-se dos versos, que o côro cantava nas peças theatraes.
\section{Corículo}
\begin{itemize}
\item {Grp. gram.:m.}
\end{itemize}
\begin{itemize}
\item {Proveniência:(Do lat. \textunderscore corium\textunderscore )}
\end{itemize}
Correia, tira de coiro.
\section{Corídon}
\begin{itemize}
\item {Grp. gram.:m.}
\end{itemize}
\begin{itemize}
\item {Utilização:Miner.}
\end{itemize}
Alumina crystallizada, de que são espécies a safira, o rubi, o topázio, a esmeralda, a amethysta.
\section{Corifemo}
\begin{itemize}
\item {Grp. gram.:m.}
\end{itemize}
Gênero de peixes do Atlântico, de lombo escuro, listado de azul.
\section{Corima}
\begin{itemize}
\item {Grp. gram.:f.}
\end{itemize}
Peixe comestível do Brasil.
\section{Corimá}
\begin{itemize}
\item {Grp. gram.:f.}
\end{itemize}
Peixe comestível do Brasil.
\section{Coriman}
\begin{itemize}
\item {Grp. gram.:m.}
\end{itemize}
Peixe do Tocantins.
(O mesmo que \textunderscore corima\textunderscore ?)
\section{Corimatan}
\begin{itemize}
\item {Grp. gram.:m.}
\end{itemize}
Saboroso peixe do Tocantins.
\section{Corimbó}
\begin{itemize}
\item {Grp. gram.:m.}
\end{itemize}
Tambor, de madeira ôca, entre os selvagens do norte do Brasil.
\section{Corina}
\begin{itemize}
\item {Grp. gram.:f.}
\end{itemize}
Espécie de antílope.
\section{Corinda}
\begin{itemize}
\item {Grp. gram.:f.}
\end{itemize}
\begin{itemize}
\item {Proveniência:(Do lat. \textunderscore cor\textunderscore  + \textunderscore indicus\textunderscore )}
\end{itemize}
Planta tropical.
\section{Corinthíaco}
\begin{itemize}
\item {Grp. gram.:adj.}
\end{itemize}
Relativo a Corintho ou aos Corínthios.
\section{Corínthio}
\begin{itemize}
\item {Grp. gram.:m.}
\end{itemize}
\begin{itemize}
\item {Grp. gram.:Adj.}
\end{itemize}
Indivíduo natural de Corintho.
Relativo a Corintho.
\section{Corintho}
\begin{itemize}
\item {Grp. gram.:adj.}
\end{itemize}
\begin{itemize}
\item {Proveniência:(De \textunderscore Corintho\textunderscore , n. p.)}
\end{itemize}
Variedade de uvas.
Casta de videiras.
\section{Corintíaco}
\begin{itemize}
\item {Grp. gram.:adj.}
\end{itemize}
Relativo a Corinto ou aos Coríntios.
\section{Coríntio}
\begin{itemize}
\item {Grp. gram.:m.}
\end{itemize}
\begin{itemize}
\item {Grp. gram.:Adj.}
\end{itemize}
Indivíduo natural de Corinto.
Relativo a Corinto.
\section{Corinto}
\begin{itemize}
\item {Grp. gram.:adj.}
\end{itemize}
\begin{itemize}
\item {Proveniência:(De \textunderscore Corintho\textunderscore , n. p.)}
\end{itemize}
Variedade de uvas.
Casta de videiras.
\section{Corioide}
\begin{itemize}
\item {Grp. gram.:f.}
\end{itemize}
O mesmo ou melhor que \textunderscore coroide\textunderscore .
\section{Córion}
\begin{itemize}
\item {Grp. gram.:m.}
\end{itemize}
\begin{itemize}
\item {Proveniência:(Gr. \textunderscore khorion\textunderscore , coiro)}
\end{itemize}
Membrana exterior do féto.
\section{Corionina}
\begin{itemize}
\item {Grp. gram.:f.}
\end{itemize}
\begin{itemize}
\item {Proveniência:(De \textunderscore chórion\textunderscore )}
\end{itemize}
Medicamento, que é extracto de placenta e se aplica contra a falta de leite.
\section{Coriphemo}
\begin{itemize}
\item {Grp. gram.:m.}
\end{itemize}
Gênero de peixes do Atlântico, de lombo escuro, listado de azul.
\section{Corisa}
\begin{itemize}
\item {Grp. gram.:f.}
\end{itemize}
\begin{itemize}
\item {Proveniência:(Do gr. \textunderscore koris\textunderscore )}
\end{itemize}
Insecto hemíptero.
\section{Corisantho}
\begin{itemize}
\item {Grp. gram.:m.}
\end{itemize}
Gênero de orchídeas.
\section{Corisanto}
\begin{itemize}
\item {Grp. gram.:m.}
\end{itemize}
Gênero de orquídeas.
\section{Coriscação}
\begin{itemize}
\item {Grp. gram.:f.}
\end{itemize}
Acto de coriscar.
\section{Coriscada}
\begin{itemize}
\item {Grp. gram.:f.}
\end{itemize}
Grande porção de coriscos.
\section{Coriscado}
\begin{itemize}
\item {Grp. gram.:adj.}
\end{itemize}
\begin{itemize}
\item {Utilização:Fig.}
\end{itemize}
\begin{itemize}
\item {Proveniência:(De \textunderscore coriscar\textunderscore )}
\end{itemize}
Ferido por qualquer coisa estimulante ou ardente. Cf. Camillo, \textunderscore Volcões\textunderscore , 157.
\section{Coriscante}
\begin{itemize}
\item {Grp. gram.:adj.}
\end{itemize}
Que corisca.
\section{Coriscar}
\begin{itemize}
\item {Grp. gram.:v. i.}
\end{itemize}
\begin{itemize}
\item {Proveniência:(Do lat. \textunderscore coruscare\textunderscore )}
\end{itemize}
Brilhar em coriscos.
Brilhar como corisco.
Relampaguear; faiscar.
\section{Corisco}
\begin{itemize}
\item {Grp. gram.:m.}
\end{itemize}
\begin{itemize}
\item {Utilização:Prov.}
\end{itemize}
\begin{itemize}
\item {Proveniência:(De \textunderscore coriscar\textunderscore )}
\end{itemize}
Faísca eléctrica.
Centelha, que rasga as nuvens electrizadas, sem que se oiçam trovões.
O mesmo que \textunderscore pedra-de-raio\textunderscore .
\section{Corispermo}
\begin{itemize}
\item {Grp. gram.:m.}
\end{itemize}
\begin{itemize}
\item {Proveniência:(Do gr. \textunderscore koris\textunderscore  + \textunderscore sperma\textunderscore )}
\end{itemize}
Gênero de plantas, cujas sementes parecem percevejos.
\section{Corista}
\begin{itemize}
\item {Grp. gram.:m.  e  f.}
\end{itemize}
\begin{itemize}
\item {Proveniência:(De \textunderscore côro\textunderscore )}
\end{itemize}
Pessôa, que faz parte dos coros theatraes.
\section{Corixo}
\begin{itemize}
\item {Grp. gram.:m.}
\end{itemize}
\begin{itemize}
\item {Utilização:Bras. de Mato-Grosso}
\end{itemize}
Atoleiro, charco.
\section{Corizonte}
\begin{itemize}
\item {Grp. gram.:m.}
\end{itemize}
\begin{itemize}
\item {Proveniência:(Do gr. \textunderscore khorizein\textunderscore , de \textunderscore khoris\textunderscore , separação)}
\end{itemize}
Aquele que, entre os Gregos, atribuía a Ilíada e a Odisseia a autores diferentes.
\section{Corja}
\begin{itemize}
\item {Grp. gram.:f.}
\end{itemize}
\begin{itemize}
\item {Utilização:Ant.}
\end{itemize}
Multidão, (em sentido depreciativo).
Canalha; súcia.
O mesmo que \textunderscore vinte\textunderscore ^1: \textunderscore comprar uma corja de camisas\textunderscore .
\section{Corla}
\begin{itemize}
\item {Grp. gram.:f.}
\end{itemize}
\begin{itemize}
\item {Utilização:Prov.}
\end{itemize}
O mesmo que \textunderscore cólera\textunderscore , doença.
(Metát. de \textunderscore colra\textunderscore , contr. de \textunderscore cólera\textunderscore )
\section{Cormedillo-papalvo}
\begin{itemize}
\item {Grp. gram.:m.}
\end{itemize}
Planta da serra de Cintra.
\section{Cormenhos}
\begin{itemize}
\item {Grp. gram.:f.}
\end{itemize}
\begin{itemize}
\item {Utilização:Ant.}
\end{itemize}
Variedade de pera, hoje desconhecida.
\section{Cormorão}
\begin{itemize}
\item {Grp. gram.:m.}
\end{itemize}
Ave marítima.
\section{Corna}
\begin{itemize}
\item {Grp. gram.:f.}
\end{itemize}
\begin{itemize}
\item {Utilização:Ant.}
\end{itemize}
\begin{itemize}
\item {Utilização:Prov.}
\end{itemize}
\begin{itemize}
\item {Utilização:alent.}
\end{itemize}
\begin{itemize}
\item {Utilização:Prov.}
\end{itemize}
\begin{itemize}
\item {Utilização:Prov.}
\end{itemize}
\begin{itemize}
\item {Utilização:minh.}
\end{itemize}
\begin{itemize}
\item {Utilização:Ant.}
\end{itemize}
\begin{itemize}
\item {Proveniência:(De \textunderscore corno\textunderscore )}
\end{itemize}
Espécie de meio bastião, nas fortificações.
O mesmo que \textunderscore corno\textunderscore .
Chavelho de boi, applicado a recipiente de líquidos ou comestíveis e a outros usos.
Colhér de ponta de cabra.
Buzina, para convocar assembleia local.
Espécie de meia lua, feita de tecido consistente, e com que se conservavam erguidos os penteados.
\section{Cornaca}
\begin{itemize}
\item {Grp. gram.:m.}
\end{itemize}
Aquelle, que conduz elephantes e trata delles.
(Talvez do cingalês \textunderscore kurrawa nayaka\textunderscore , maioral de elephantes)
\section{Cornaça}
\begin{itemize}
\item {Grp. gram.:m.}
\end{itemize}
\begin{itemize}
\item {Utilização:Pleb.}
\end{itemize}
Marido, a quem a mulher é infiel. Cf. Camillo, \textunderscore Corja\textunderscore , 139.
\section{Cornada}
\begin{itemize}
\item {Grp. gram.:f.}
\end{itemize}
Pancada com os cornos.
Marrada.
\section{Cornado}
\begin{itemize}
\item {Grp. gram.:adj.}
\end{itemize}
\begin{itemize}
\item {Utilização:Ant.}
\end{itemize}
Que tem cornos.
Cornígero. Cf. G. Vicente, I, 251.
\section{Cornadura}
\begin{itemize}
\item {Grp. gram.:f.}
\end{itemize}
\begin{itemize}
\item {Proveniência:(De \textunderscore corno\textunderscore )}
\end{itemize}
Pontas dos animaes cornígeros; chifres.
\section{Cornal}
\begin{itemize}
\item {Grp. gram.:m.}
\end{itemize}
\begin{itemize}
\item {Utilização:Prov.}
\end{itemize}
\begin{itemize}
\item {Utilização:trasm.}
\end{itemize}
Correia, com que se prendem os cornos do boi ao jugo.
\section{Cornalão}
\begin{itemize}
\item {Grp. gram.:adj.}
\end{itemize}
Diz-se do toiro, que tem hastes muito grandes.
(Cast. \textunderscore cornalón\textunderscore )
\section{Cornalheira}
\begin{itemize}
\item {Grp. gram.:f.}
\end{itemize}
\begin{itemize}
\item {Proveniência:(De \textunderscore corno\textunderscore )}
\end{itemize}
Planta terebinthácea.
\section{Cornalhuda}
\begin{itemize}
\item {Grp. gram.:adj. f.}
\end{itemize}
\begin{itemize}
\item {Utilização:Prov.}
\end{itemize}
\begin{itemize}
\item {Utilização:trasm.}
\end{itemize}
\begin{itemize}
\item {Proveniência:(De \textunderscore corno\textunderscore )}
\end{itemize}
Diz-se de uma variedade de azeitona, também chamada \textunderscore bical\textunderscore , \textunderscore bicuda\textunderscore , \textunderscore cornicabra\textunderscore , \textunderscore cornogela\textunderscore .
\section{Cornalina}
\begin{itemize}
\item {Grp. gram.:f.}
\end{itemize}
\begin{itemize}
\item {Proveniência:(T. cast.)}
\end{itemize}
Espécie de ágatha, meio transparente e de cores differentes.
\section{Cornamenta}
\begin{itemize}
\item {Grp. gram.:f.}
\end{itemize}
O mesmo que \textunderscore cornadura\textunderscore .
\section{Cornamusa}
\begin{itemize}
\item {Grp. gram.:f.}
\end{itemize}
\begin{itemize}
\item {Proveniência:(It. \textunderscore cornamusa\textunderscore )}
\end{itemize}
Gaita de folles.
\section{Cornante}
\begin{itemize}
\item {Grp. gram.:m.}
\end{itemize}
\begin{itemize}
\item {Utilização:Gír.}
\end{itemize}
Boi. (De \textunderscore corno\textunderscore )
\section{Cornar}
\begin{itemize}
\item {Grp. gram.:v. t.}
\end{itemize}
Bater ou ferir com os cornos: \textunderscore bois em terra alheia, as vacas os cornam\textunderscore . (Colhido em Vouzela)
\section{Cornaria}
\begin{itemize}
\item {Grp. gram.:f.}
\end{itemize}
\begin{itemize}
\item {Proveniência:(De \textunderscore corno\textunderscore )}
\end{itemize}
Antigo imposto, que pagavam os possuidores de gado vaccum.
\section{Corne}
\begin{itemize}
\item {Grp. gram.:m.}
\end{itemize}
\begin{itemize}
\item {Proveniência:(Do Ingl. \textunderscore korn\textunderscore , corno)}
\end{itemize}
(V. \textunderscore trompa\textunderscore ^1)
\section{Córnea}
\begin{itemize}
\item {Grp. gram.:f.}
\end{itemize}
\begin{itemize}
\item {Utilização:Anat.}
\end{itemize}
\begin{itemize}
\item {Proveniência:(De \textunderscore córneo\textunderscore )}
\end{itemize}
Membrana transparente, que deixa passar os raios luminosos para a pupilla do ôlho.
\section{Corneação}
\begin{itemize}
\item {Grp. gram.:f.}
\end{itemize}
Acto de cornear.
\section{Cornear}
\begin{itemize}
\item {Grp. gram.:v. t.}
\end{itemize}
\begin{itemize}
\item {Utilização:Pop.}
\end{itemize}
\begin{itemize}
\item {Proveniência:(De \textunderscore corno\textunderscore )}
\end{itemize}
Sêr infiel a mulher a (o marido).
\section{Córneas}
\begin{itemize}
\item {Grp. gram.:f. pl.}
\end{itemize}
O mesmo que \textunderscore cornuáceas\textunderscore .
\section{Córneas}
\begin{itemize}
\item {Grp. gram.:f. pl.}
\end{itemize}
\begin{itemize}
\item {Proveniência:(De \textunderscore córneo\textunderscore )}
\end{itemize}
Fam. de plantas dicotyledóneas, cuja madeira é muito resistente.
\section{Cornecha}
\begin{itemize}
\item {fónica:nê}
\end{itemize}
\begin{itemize}
\item {Grp. gram.:f.}
\end{itemize}
Casta de uva.
\section{Corneíba}
\begin{itemize}
\item {Grp. gram.:f.}
\end{itemize}
Árvore terebinthácea do Brasil.
\section{Corne-inglês}
\begin{itemize}
\item {Grp. gram.:m.}
\end{itemize}
\begin{itemize}
\item {Utilização:Mús.}
\end{itemize}
\begin{itemize}
\item {Proveniência:(Fr. \textunderscore cor-anglais\textunderscore )}
\end{itemize}
Instrumento, semelhante ao oboé.
\section{Corneira}
\begin{itemize}
\item {Grp. gram.:f.}
\end{itemize}
\begin{itemize}
\item {Proveniência:(De \textunderscore corno\textunderscore )}
\end{itemize}
Correia, que prende o boi pelos cornos á canga ou aos cornos de outro boi; cornal.
\section{Cornejar}
\begin{itemize}
\item {Grp. gram.:v. i.}
\end{itemize}
\begin{itemize}
\item {Proveniência:(De \textunderscore corno\textunderscore )}
\end{itemize}
Dilatar os cornos (o caracol).
\section{Cornel}
\begin{itemize}
\item {Grp. gram.:m.}
\end{itemize}
(Forma pop. de \textunderscore coronel\textunderscore )
\section{Cornela}
\begin{itemize}
\item {Grp. gram.:f.}
\end{itemize}
\begin{itemize}
\item {Utilização:T. de Penafiel}
\end{itemize}
O mesmo que \textunderscore vaca-loira\textunderscore .
\section{Cornelhas}
\begin{itemize}
\item {fónica:nê}
\end{itemize}
\begin{itemize}
\item {Grp. gram.:f. pl.}
\end{itemize}
\begin{itemize}
\item {Utilização:Prov.}
\end{itemize}
\begin{itemize}
\item {Utilização:dur.}
\end{itemize}
Anéis de pano, que se enfiam nos cornos do boi, para que a base dêstes se não fira com a correia, que prende o jugo e as molhelhas aos mesmos cornos.
\section{Cornelho}
\begin{itemize}
\item {fónica:nê}
\end{itemize}
\begin{itemize}
\item {Grp. gram.:m.}
\end{itemize}
\begin{itemize}
\item {Proveniência:(De \textunderscore corno\textunderscore )}
\end{itemize}
O mesmo que fungão ou cravagem, nas plantas cerealíferas.
\section{Córneo}
\begin{itemize}
\item {Grp. gram.:adj.}
\end{itemize}
\begin{itemize}
\item {Proveniência:(Lat. \textunderscore corneus\textunderscore )}
\end{itemize}
Relativo a corno.
Feito de corno.
Resistente como corno.
\section{Corneta}
\begin{itemize}
\item {fónica:nê}
\end{itemize}
\begin{itemize}
\item {Grp. gram.:f.}
\end{itemize}
\begin{itemize}
\item {Utilização:Prov.}
\end{itemize}
\begin{itemize}
\item {Utilização:alent.}
\end{itemize}
\begin{itemize}
\item {Utilização:Mús.}
\end{itemize}
\begin{itemize}
\item {Utilização:Gír.}
\end{itemize}
\begin{itemize}
\item {Grp. gram.:M.}
\end{itemize}
\begin{itemize}
\item {Grp. gram.:Adj.}
\end{itemize}
\begin{itemize}
\item {Utilização:Bras}
\end{itemize}
\begin{itemize}
\item {Grp. gram.:F. pl.}
\end{itemize}
\begin{itemize}
\item {Utilização:Ant.}
\end{itemize}
\begin{itemize}
\item {Proveniência:(De \textunderscore corno\textunderscore )}
\end{itemize}
Instrumento de sopro, feito de corno, de metal, etc.
Buzina.
Jôgo popular, em que se emprega um chavelho de carneiro.
Registo do órgão.
O mesmo que \textunderscore cara\textunderscore ^1: \textunderscore parto-lhe a corneta\textunderscore .
Corneteiro.
Diz-se do toiro ou da vaca, que tem um só corno.
Espécie de meia lua, feita de bocaxim ou de outro tecido consistente, e com que dantes se levantavam os penteados.
\section{Cornetada}
\begin{itemize}
\item {Grp. gram.:f.}
\end{itemize}
Toque de corneta.
\section{Corneteiro}
\begin{itemize}
\item {Grp. gram.:m.}
\end{itemize}
\begin{itemize}
\item {Proveniência:(De \textunderscore corneta\textunderscore )}
\end{itemize}
Aquelle que toca corneta num batalhão.
\section{Cornetilha}
\begin{itemize}
\item {Grp. gram.:f.}
\end{itemize}
\begin{itemize}
\item {Utilização:Mús.}
\end{itemize}
\begin{itemize}
\item {Proveniência:(De \textunderscore corneta\textunderscore )}
\end{itemize}
Registo dos órgãos antigos.
\section{Cornetim}
\begin{itemize}
\item {Grp. gram.:m.}
\end{itemize}
\begin{itemize}
\item {Proveniência:(De \textunderscore corneta\textunderscore )}
\end{itemize}
Instrumento de sopro, feito de metal e com três chaves ou êmbolos.
Aquelle, que toca esse instrumento.
\section{Corneto}
\begin{itemize}
\item {fónica:nê}
\end{itemize}
\begin{itemize}
\item {Grp. gram.:m.}
\end{itemize}
\begin{itemize}
\item {Proveniência:(De \textunderscore corno\textunderscore )}
\end{itemize}
Lâmina óssea, dobrada sôbre si, dentro das fossas nasaes.
\section{Cornialto}
\begin{itemize}
\item {Grp. gram.:adj.}
\end{itemize}
\begin{itemize}
\item {Proveniência:(De \textunderscore corno\textunderscore  + \textunderscore alto\textunderscore )}
\end{itemize}
Diz-se do toiro, cujas pontas se elevam mais do que é vulgar.
\section{Corniavacado}
\begin{itemize}
\item {Grp. gram.:adj.}
\end{itemize}
\begin{itemize}
\item {Proveniência:(De \textunderscore corno\textunderscore  + \textunderscore vaca\textunderscore )}
\end{itemize}
Diz-se do toiro, cujas hastes nascem muito atrás do testo ou da fronte.
\section{Cornibaixo}
\begin{itemize}
\item {Grp. gram.:adj.}
\end{itemize}
\begin{itemize}
\item {Proveniência:(De \textunderscore corno\textunderscore  + \textunderscore baixo\textunderscore )}
\end{itemize}
Diz-se do toiro que tem hastes baixas ou inclinadas para baixo.
\section{Cornicabra}
\begin{itemize}
\item {Grp. gram.:f.}
\end{itemize}
\begin{itemize}
\item {Proveniência:(Do lat. \textunderscore cornu\textunderscore  + \textunderscore capra\textunderscore )}
\end{itemize}
Planta solânea.
Variedade de pêra temporan, que se cultiva na região do Doiro.
O mesmo que \textunderscore cornalhuda\textunderscore .
\section{Cornicabra-dos-algarvios}
\begin{itemize}
\item {Grp. gram.:f.}
\end{itemize}
Designação vulgar de uma planta gnetácea, (\textunderscore ephedra fragilis\textunderscore , Desf.)
\section{Cornicesto}
\begin{itemize}
\item {fónica:cê}
\end{itemize}
\begin{itemize}
\item {Grp. gram.:m.}
\end{itemize}
\begin{itemize}
\item {Utilização:Prov.}
\end{itemize}
\begin{itemize}
\item {Utilização:trasm.}
\end{itemize}
Variedade de uva, ordinariamente preta e de má qualidade.
\section{Cornichão}
\begin{itemize}
\item {Grp. gram.:m.}
\end{itemize}
Planta leguminosa, (\textunderscore lotus corniculatus\textunderscore , Lin.).
\section{Cornichela}
\begin{itemize}
\item {Grp. gram.:f.}
\end{itemize}
Casta de uva de Azeitão.
\section{Cornicho}
\begin{itemize}
\item {Grp. gram.:m.}
\end{itemize}
\begin{itemize}
\item {Utilização:Prov.}
\end{itemize}
\begin{itemize}
\item {Utilização:Prov.}
\end{itemize}
\begin{itemize}
\item {Utilização:trasm.}
\end{itemize}
\begin{itemize}
\item {Utilização:Prov.}
\end{itemize}
\begin{itemize}
\item {Utilização:trasm.}
\end{itemize}
\begin{itemize}
\item {Utilização:Prov.}
\end{itemize}
\begin{itemize}
\item {Utilização:trasm.}
\end{itemize}
\begin{itemize}
\item {Proveniência:(De \textunderscore corno\textunderscore )}
\end{itemize}
Pequeno chifre.
Antenna.
Tentáculo de caracol.
Pão, de feitio aproximado ao do chifre.
Cada uma das pontas ou cantos do saco.
Prega de malhas, em que se começa o calcanhar da meia.
Rabicho da albarda.
\section{Córnico}
\begin{itemize}
\item {Grp. gram.:m.}
\end{itemize}
Lingua local de Cornualha, em Inglaterra.
\section{Corniculáceas}
\begin{itemize}
\item {Grp. gram.:f. pl.}
\end{itemize}
\begin{itemize}
\item {Proveniência:(Do lat. \textunderscore corniculum\textunderscore )}
\end{itemize}
Ordem de plantas, que comprehende as saxifragáceas e as ribesiáceas.
\section{Corniculário}
\begin{itemize}
\item {Grp. gram.:m.}
\end{itemize}
\begin{itemize}
\item {Proveniência:(Lat. \textunderscore cornicularius\textunderscore )}
\end{itemize}
Soldado romano, que ia á frente de uma pequena divisão militar.
Official inferior, adjunto a um centurião, entre os Romanos.
\section{Cornículo}
\begin{itemize}
\item {Grp. gram.:m.}
\end{itemize}
\begin{itemize}
\item {Proveniência:(Lat. \textunderscore corniculum\textunderscore )}
\end{itemize}
Insígnia que, em fórma de pequeno corno, se concedia, como galardão, ao soldado romano, que tivesse dado provas do seu valor.
\section{Cornicurto}
\begin{itemize}
\item {Grp. gram.:adj.}
\end{itemize}
\begin{itemize}
\item {Proveniência:(De \textunderscore corno\textunderscore  + \textunderscore curto\textunderscore )}
\end{itemize}
Que tem cornos curtos.
\section{Cornideanteiro}
\begin{itemize}
\item {Grp. gram.:adj.}
\end{itemize}
\begin{itemize}
\item {Proveniência:(De \textunderscore corno\textunderscore  + \textunderscore deanteiro\textunderscore )}
\end{itemize}
Diz-se do toiro, cujas hastes sáem da parte mais deanteira do testo.
\section{Cornídia}
\begin{itemize}
\item {Grp. gram.:f.}
\end{itemize}
Gênero de plantas saxifragáceas do Peru e do Chile.
\section{Cornífero}
\begin{itemize}
\item {Grp. gram.:adj.}
\end{itemize}
\begin{itemize}
\item {Proveniência:(Do lat. \textunderscore cornu\textunderscore  + \textunderscore ferre\textunderscore )}
\end{itemize}
Que tem cornos ou excrescência em fórma de cornos.
\section{Cornifesto}
\begin{itemize}
\item {Grp. gram.:m.}
\end{itemize}
Casta de uva tinta, trasmontana.
\section{Corniforme}
\begin{itemize}
\item {Grp. gram.:adj.}
\end{itemize}
\begin{itemize}
\item {Proveniência:(Do lat. \textunderscore cornu\textunderscore  + \textunderscore forma\textunderscore )}
\end{itemize}
Que tem fórma de corno.
\section{Cornifrente}
\begin{itemize}
\item {Grp. gram.:adj.}
\end{itemize}
O mesmo que \textunderscore cornudo\textunderscore . Cf. Macedo, \textunderscore Burros\textunderscore , 250.
\section{Cornígero}
\begin{itemize}
\item {Grp. gram.:adj.}
\end{itemize}
\begin{itemize}
\item {Proveniência:(Do lat. \textunderscore cornu\textunderscore  + \textunderscore gerere\textunderscore )}
\end{itemize}
O mesmo que \textunderscore cornífero\textunderscore .
\section{Cornija}
\begin{itemize}
\item {Grp. gram.:f.}
\end{itemize}
\begin{itemize}
\item {Proveniência:(Do it. \textunderscore cornice\textunderscore )}
\end{itemize}
Molduras sobrepostas, que formam saliência na parte superior de uma parede, de uma porta, de um móvel, etc.
\section{Cornil}
\begin{itemize}
\item {Grp. gram.:m.}
\end{itemize}
\begin{itemize}
\item {Utilização:T. de Miranda}
\end{itemize}
\begin{itemize}
\item {Proveniência:(De \textunderscore corno\textunderscore )}
\end{itemize}
Correia, o mesmo que \textunderscore soga\textunderscore .
\section{Cornilargo}
\begin{itemize}
\item {Grp. gram.:adj.}
\end{itemize}
\begin{itemize}
\item {Proveniência:(De \textunderscore corno\textunderscore  + \textunderscore largo\textunderscore )}
\end{itemize}
Diz-se do toiro, cujas hastes tem as extremidades muito afastadas uma da outra.
\section{Cornilhal}
\begin{itemize}
\item {Grp. gram.:m.}
\end{itemize}
\begin{itemize}
\item {Utilização:Prov.}
\end{itemize}
\begin{itemize}
\item {Utilização:trasm.}
\end{itemize}
\begin{itemize}
\item {Proveniência:(De \textunderscore corno\textunderscore )}
\end{itemize}
Cada uma das pontas ou cantos de um saco.
Cp. \textunderscore cornicho\textunderscore .
\section{Cornilhão}
\begin{itemize}
\item {Grp. gram.:m.}
\end{itemize}
\begin{itemize}
\item {Proveniência:(Fr. \textunderscore cornillon\textunderscore )}
\end{itemize}
Planta herbácea e leguminosa.
\section{Corniluzente}
\begin{itemize}
\item {Grp. gram.:adj.}
\end{itemize}
Que tem cornos luzidios. Cf. Filinto, IX, 116.
\section{Cornimboque}
\begin{itemize}
\item {Grp. gram.:m.}
\end{itemize}
\begin{itemize}
\item {Utilização:Bras}
\end{itemize}
\begin{itemize}
\item {Proveniência:(De \textunderscore corno\textunderscore )}
\end{itemize}
Caixa de corno, para rapé.
\section{Corninglês}
\begin{itemize}
\item {Grp. gram.:m.}
\end{itemize}
O mesmo que \textunderscore corne-inglês\textunderscore .
\section{Corníola}
\begin{itemize}
\item {Grp. gram.:f.}
\end{itemize}
\begin{itemize}
\item {Proveniência:(De \textunderscore corno\textunderscore )}
\end{itemize}
Variedade de pedra transparente.
Cornalina.
\section{Corníolo}
\begin{itemize}
\item {Grp. gram.:m.}
\end{itemize}
(V.pilriteiro)
\section{Corninho}
\begin{itemize}
\item {Grp. gram.:m.}
\end{itemize}
\begin{itemize}
\item {Utilização:Prov.}
\end{itemize}
\begin{itemize}
\item {Utilização:minh.}
\end{itemize}
Robalo pequeno, de meio palmo, pouco mais ou menos.
\section{Cornipaso}
\begin{itemize}
\item {Grp. gram.:adj.}
\end{itemize}
Que tem as pontas dos cornos voltadas para os lados, (falando-se do toiro).
\section{Cornípede}
\begin{itemize}
\item {Grp. gram.:adj.}
\end{itemize}
\begin{itemize}
\item {Utilização:Zool.}
\end{itemize}
\begin{itemize}
\item {Proveniência:(Lat. \textunderscore cornipes\textunderscore )}
\end{itemize}
Cujas patas são de substância córnea.
\section{Cornípeto}
\begin{itemize}
\item {Grp. gram.:m.  e  adj.}
\end{itemize}
O mesmo que \textunderscore cornúpeto\textunderscore .
\section{Cornipo}
\begin{itemize}
\item {Grp. gram.:m.}
\end{itemize}
\begin{itemize}
\item {Utilização:Prov.}
\end{itemize}
Corno pequeno, cornicho.
\section{Cornisal}
\begin{itemize}
\item {Grp. gram.:m.}
\end{itemize}
\begin{itemize}
\item {Utilização:Prov.}
\end{itemize}
\begin{itemize}
\item {Utilização:alg.}
\end{itemize}
\begin{itemize}
\item {Proveniência:(De \textunderscore corno\textunderscore , sob a infl. de \textunderscore corniso\textunderscore )}
\end{itemize}
Coisa dura como corno.
\section{Corniso}
\begin{itemize}
\item {Grp. gram.:m.}
\end{itemize}
\begin{itemize}
\item {Proveniência:(De \textunderscore corno\textunderscore )}
\end{itemize}
Arbusto araliáceo, espécie de abrunheiro.
\section{Cornisolo}
\begin{itemize}
\item {Grp. gram.:m.}
\end{itemize}
\begin{itemize}
\item {Utilização:Fam.}
\end{itemize}
Fruto do corniso.
Homem, trahido pela mulher, cabrão, corno.
\section{Corno}
\begin{itemize}
\item {Grp. gram.:m.}
\end{itemize}
\begin{itemize}
\item {Utilização:Pleb.}
\end{itemize}
\begin{itemize}
\item {Proveniência:(Lat. \textunderscore cornu\textunderscore )}
\end{itemize}
Cada uma das pontas sólidas da cabeça de alguns animaes.
Chavelho, chifre.
Antenna ou tentáculo, comparável ao corno.
Ponta ou objecto que tem analogia com o corno.
Substância do corno: \textunderscore pentes de corno\textunderscore .
Marido, a quem a mulher é infiel.
\section{Cornofone}
\begin{itemize}
\item {Grp. gram.:m.}
\end{itemize}
\begin{itemize}
\item {Utilização:Mús.}
\end{itemize}
Instrumento, inventado recentemente e destinado a substituir a trompa.
\section{Cornofónio}
\begin{itemize}
\item {Grp. gram.:m.}
\end{itemize}
\begin{itemize}
\item {Utilização:Mús.}
\end{itemize}
Instrumento, inventado recentemente e destinado a substituir a trompa.
\section{Cornogela}
\begin{itemize}
\item {Grp. gram.:f.}
\end{itemize}
Casta de azeitona, o mesmo que \textunderscore cornalhuda\textunderscore .
\section{Corno-godinho}
\begin{itemize}
\item {Grp. gram.:m.}
\end{itemize}
Planta, o mesmo que \textunderscore tramazeira\textunderscore .
\section{Cornophone}
\begin{itemize}
\item {Grp. gram.:m.}
\end{itemize}
\begin{itemize}
\item {Utilização:Mús.}
\end{itemize}
Instrumento, inventado recentemente e destinado a substituir a trompa.
\section{Cornozelo}
\begin{itemize}
\item {fónica:zê}
\end{itemize}
\begin{itemize}
\item {Grp. gram.:m.}
\end{itemize}
\begin{itemize}
\item {Utilização:Des.}
\end{itemize}
O mesmo que \textunderscore ferradura\textunderscore .
\section{Cornuáceas}
\begin{itemize}
\item {Grp. gram.:f. pl.}
\end{itemize}
Fam. de plantas, que tem por typo o pilriteiro.
O mesmo que \textunderscore córneas\textunderscore ^1.
\section{Cornucópia}
\begin{itemize}
\item {Grp. gram.:f.}
\end{itemize}
\begin{itemize}
\item {Grp. gram.:F. pl.}
\end{itemize}
\begin{itemize}
\item {Proveniência:(Lat. \textunderscore cornucopia\textunderscore )}
\end{itemize}
Corno mythológico da abundância.
Vaso, em fórma do corno, que se representa cheio de flôres e frutos.
Variedade de videira americana.
Gênero de plantas gramíneas.
\section{Cornuda}
\begin{itemize}
\item {Grp. gram.:f.}
\end{itemize}
\begin{itemize}
\item {Proveniência:(De \textunderscore cornudo\textunderscore )}
\end{itemize}
Peixe da costa do Algarve e dos Açores.
\section{Cornudagem}
\begin{itemize}
\item {Grp. gram.:f.}
\end{itemize}
\begin{itemize}
\item {Utilização:Pleb.}
\end{itemize}
Conjunto dos indivíduos, a quem as esposas atraiçoam.
Vida de cabrão ou corno. Cf. Camillo, \textunderscore E. Macário\textunderscore , 256.
\section{Cornudo}
\begin{itemize}
\item {Grp. gram.:adj.}
\end{itemize}
O mesmo que \textunderscore cornuto\textunderscore .
\section{Cornúpeto}
\begin{itemize}
\item {Grp. gram.:adj.}
\end{itemize}
\begin{itemize}
\item {Grp. gram.:M.}
\end{itemize}
\begin{itemize}
\item {Proveniência:(Do lat. \textunderscore cornu\textunderscore  + \textunderscore petere\textunderscore . Cp. lat. \textunderscore cornupeta\textunderscore )}
\end{itemize}
Que bate com cornos, que marra.
Toiro.
\section{Cornuto}
\begin{itemize}
\item {Grp. gram.:adj.}
\end{itemize}
\begin{itemize}
\item {Proveniência:(Lat. \textunderscore cornutus\textunderscore )}
\end{itemize}
Que tem cornos, cornífero.
\section{Côro}
\begin{itemize}
\item {Grp. gram.:m.}
\end{itemize}
\begin{itemize}
\item {Grp. gram.:Loc. adv.}
\end{itemize}
\begin{itemize}
\item {Proveniência:(Do gr. \textunderscore khoros\textunderscore , dança)}
\end{itemize}
Pessôas, que cantam em commum.
Aquillo que se conta em commum.
Espécie de palanque ou estrado alto, em que, nas igrejas, se toca ou canta, e em que fazem as suas rezas communs os cónegos, membros de collegiadas, seminaristas, etc.
Fileira de cadeiras fixas na capella-mór, nas quaes se sentam os cónegos, membros de collegiadas, beneficiados, etc.
\textunderscore Em côro\textunderscore , ao mesmo tempo, a uma voz.
\section{Córo}
\begin{itemize}
\item {Grp. gram.:m.}
\end{itemize}
\begin{itemize}
\item {Proveniência:(Lat. \textunderscore corus\textunderscore )}
\end{itemize}
Vento do Noroéste, segundo a Náutica antiga.
\section{Corôa}
\begin{itemize}
\item {Grp. gram.:f.}
\end{itemize}
\begin{itemize}
\item {Utilização:Ext.}
\end{itemize}
\begin{itemize}
\item {Proveniência:(Lat. \textunderscore corona\textunderscore )}
\end{itemize}
Ornato circular para a cabeça.
Objecto, que tem fórma de corôa ou analogia com ella.
Fecho, remate.
Tonsura circular na cabeça dos ecclesiásticos.
Cume, alto: \textunderscore a corôa do monte\textunderscore .
Poder monárchico: \textunderscore os direitos da Corôa\textunderscore .
Monarcha.
Calvície nos joêlhos do cavallo.
Parte do dente, superior aos alvéolos.
Face superior de um diamante.
Moéda de oiro, que valia 10$000 reis.
Moéda de prata, que valia 1$000.
Nome de duas constellações.
Appêndices na corolla ou na base de algumas flôres.
Limbo duradoiro do cálice de alguns frutos.
Círculo luminoso, em volta alguns astros.
Superfície plana, entre dois círculos concêntricos.
Fio de contas, por onde se rezam sete padre-nossos e sete dezenas de ave-marias.
Tufo circular de pennas, na cabeça de algumas aves.
Moéda da Dinamarca, que vale 250 reis.
Cada um dos segmentos circulares de cobre, intercalados no distribuidor do apparelho telegráphico de Baudot.
Designação vulgar da moéda de 500 reis, em prata.
Nome de várias plantas.
Cabo náutico, que encapella nos mastros e nos mastaréus de gáveas.
Banco de areia.
\section{Coroá}
\begin{itemize}
\item {Grp. gram.:m.}
\end{itemize}
Planta aromática do Brasil.
\section{Coroação}
\begin{itemize}
\item {Grp. gram.:f.}
\end{itemize}
Acto do coroar.
Esgalhos, que guarnecem a cabeça do veado.
\section{Corôa-de-rei}
\begin{itemize}
\item {Grp. gram.:f.}
\end{itemize}
Planta liliácea, de pétalas lanceolares e geralmente mosqueadas de escuro.
Planta leguminosa, (\textunderscore trifolium melilotes segetalis\textunderscore , Brot.).
\section{Coroado}
\begin{itemize}
\item {Grp. gram.:adj.}
\end{itemize}
Diz-se de uma variedade de pero.
\section{Coroados}
\begin{itemize}
\item {Grp. gram.:m. pl.}
\end{itemize}
Antiga nação de Índios do Brasil, nas nascentes do rio Embotetiú.
\section{Coroamento}
\begin{itemize}
\item {Grp. gram.:m.}
\end{itemize}
\begin{itemize}
\item {Proveniência:(De \textunderscore coroar\textunderscore )}
\end{itemize}
O mesmo que \textunderscore coroação\textunderscore .
Remate.
Adôrno da parte superior de um edifício.
\section{Coroar}
\begin{itemize}
\item {Grp. gram.:v. t.}
\end{itemize}
\begin{itemize}
\item {Proveniência:(Do lat. \textunderscore coronare\textunderscore )}
\end{itemize}
Cingir de corôa a cabeça de.
Encimar.
Rematar.
Premiar.
Elevar á dignidade real.
Preencher, satisfazer.
Rodear com um círculo.
\section{Coroás}
\begin{itemize}
\item {Grp. gram.:m. pl.}
\end{itemize}
O mesmo que \textunderscore coroados\textunderscore .
\section{Coroboca}
\begin{itemize}
\item {Grp. gram.:f.}
\end{itemize}
\begin{itemize}
\item {Utilização:Prov.}
\end{itemize}
\begin{itemize}
\item {Utilização:Minas}
\end{itemize}
Lugar deserto.
Habitação longínqua.
\section{Coroca}
\begin{itemize}
\item {Grp. gram.:m. ,  f.  e  adj.}
\end{itemize}
\begin{itemize}
\item {Utilização:Bras}
\end{itemize}
\begin{itemize}
\item {Grp. gram.:F.}
\end{itemize}
\begin{itemize}
\item {Utilização:T. do Amazonas}
\end{itemize}
Pessôa adoentada.
O mesmo que \textunderscore anum\textunderscore .
\section{Coroça}
\begin{itemize}
\item {Grp. gram.:f.}
\end{itemize}
Capa de palha.
Palhota.
Jurisdicção abusiva, sob capa de legal.
(Cp. \textunderscore croça\textunderscore ^1, que é a pronúncia usual.)
(Cast. \textunderscore coroza\textunderscore )
\section{Corocora}
\begin{itemize}
\item {Grp. gram.:f.}
\end{itemize}
Espécie de batel da costa oriental da África. Cf. Barros, \textunderscore Déc.\textunderscore  I, l. IV, c. V.
\section{Corocuturu}
\begin{itemize}
\item {Grp. gram.:m.}
\end{itemize}
\begin{itemize}
\item {Utilização:Bras}
\end{itemize}
Espécie de gavião.
\section{Corófia}
\begin{itemize}
\item {Grp. gram.:f.}
\end{itemize}
Molusco, que devora outros. Cf. Fil. Simões, \textunderscore Beiramar\textunderscore , 243.
\section{Corografia}
\begin{itemize}
\item {Grp. gram.:f.}
\end{itemize}
\begin{itemize}
\item {Proveniência:(Do gr. \textunderscore khora\textunderscore  + \textunderscore graphein\textunderscore )}
\end{itemize}
Descripção de uma região ou de uma parte importante de um território.
\section{Corográfico}
\begin{itemize}
\item {Grp. gram.:adj.}
\end{itemize}
Relativo á corografia.
\section{Corógrafo}
\begin{itemize}
\item {Grp. gram.:m.}
\end{itemize}
Aquele que escreve sôbre corografia.
\section{Coroide}
\begin{itemize}
\item {Grp. gram.:f.}
\end{itemize}
\begin{itemize}
\item {Proveniência:(Gr. \textunderscore khoroeides\textunderscore )}
\end{itemize}
Membrana da parte posterior do ôlho.
Membrana da pia-máter.
\section{Corojo}
\begin{itemize}
\item {Grp. gram.:m.}
\end{itemize}
O mesmo que \textunderscore carroá\textunderscore .
\section{Corola}
\begin{itemize}
\item {Grp. gram.:f.}
\end{itemize}
\begin{itemize}
\item {Utilização:Bot.}
\end{itemize}
\begin{itemize}
\item {Proveniência:(Lat. \textunderscore corolla\textunderscore )}
\end{itemize}
Invólucro imediato dos estames e pistilo da flôr.
Invólucro interno de um perianto duplo.
\section{Coroláceo}
\begin{itemize}
\item {Grp. gram.:adj.}
\end{itemize}
Que tem a aparência de corola.
\section{Corolado}
\begin{itemize}
\item {Grp. gram.:adj.}
\end{itemize}
Que tem corola.
\section{Corolário}
\begin{itemize}
\item {Grp. gram.:m.}
\end{itemize}
\begin{itemize}
\item {Proveniência:(Lat. \textunderscore corollarium\textunderscore )}
\end{itemize}
Consequência; proposição que se deduz daquilo que se demonstrou.
\section{Corolífero}
\begin{itemize}
\item {Grp. gram.:adj.}
\end{itemize}
\begin{itemize}
\item {Utilização:Bot.}
\end{itemize}
\begin{itemize}
\item {Proveniência:(Do lat. \textunderscore corolla\textunderscore  + \textunderscore ferre\textunderscore )}
\end{itemize}
Que sustenta a corola.
\section{Corolifloras}
\begin{itemize}
\item {Grp. gram.:f. pl.}
\end{itemize}
\begin{itemize}
\item {Proveniência:(De \textunderscore corolla\textunderscore  + \textunderscore flôr\textunderscore )}
\end{itemize}
Classe de plantas, de corola monopétala com estames.
\section{Coroliforme}
\begin{itemize}
\item {Grp. gram.:adj.}
\end{itemize}
\begin{itemize}
\item {Proveniência:(Do lat. \textunderscore corolla\textunderscore  + \textunderscore forma\textunderscore )}
\end{itemize}
Que tem fórma de corola.
\section{Corolino}
\begin{itemize}
\item {Grp. gram.:adj.}
\end{itemize}
(V.coroliforme)
\section{Corolítico}
\begin{itemize}
\item {Grp. gram.:adj.}
\end{itemize}
\begin{itemize}
\item {Utilização:Archit.}
\end{itemize}
\begin{itemize}
\item {Proveniência:(De \textunderscore corolla\textunderscore )}
\end{itemize}
Diz-se das colunas com ornatos de fôlhas ou flôres em espiral.
\section{Corolla}
\begin{itemize}
\item {Grp. gram.:f.}
\end{itemize}
\begin{itemize}
\item {Utilização:Bot.}
\end{itemize}
\begin{itemize}
\item {Proveniência:(Lat. \textunderscore corolla\textunderscore )}
\end{itemize}
Invólucro immediato dos estames e pistillo da flôr.
Invólucro interno de um periantho duplo.
\section{Corolláceo}
\begin{itemize}
\item {Grp. gram.:adj.}
\end{itemize}
Que tem a apparência de corolla.
\section{Corollado}
\begin{itemize}
\item {Grp. gram.:adj.}
\end{itemize}
Que tem corolla.
\section{Corollário}
\begin{itemize}
\item {Grp. gram.:m.}
\end{itemize}
\begin{itemize}
\item {Proveniência:(Lat. \textunderscore corollarium\textunderscore )}
\end{itemize}
Consequência; proposição que se deduz daquillo que se demonstrou.
\section{Corollífero}
\begin{itemize}
\item {Grp. gram.:adj.}
\end{itemize}
\begin{itemize}
\item {Utilização:Bot.}
\end{itemize}
\begin{itemize}
\item {Proveniência:(Do lat. \textunderscore corolla\textunderscore  + \textunderscore ferre\textunderscore )}
\end{itemize}
Que sustenta a corolla.
\section{Corollifloras}
\begin{itemize}
\item {Grp. gram.:f. pl.}
\end{itemize}
\begin{itemize}
\item {Proveniência:(De \textunderscore corolla\textunderscore  + \textunderscore flôr\textunderscore )}
\end{itemize}
Classe de plantas, de corolla monopétala com estames.
\section{Corolliforme}
\begin{itemize}
\item {Grp. gram.:adj.}
\end{itemize}
\begin{itemize}
\item {Proveniência:(Do lat. \textunderscore corolla\textunderscore  + \textunderscore forma\textunderscore )}
\end{itemize}
Que tem fórma de corolla.
\section{Corollino}
\begin{itemize}
\item {Grp. gram.:adj.}
\end{itemize}
(V.corolliforme)
\section{Corollítico}
\begin{itemize}
\item {Grp. gram.:adj.}
\end{itemize}
\begin{itemize}
\item {Utilização:Archit.}
\end{itemize}
\begin{itemize}
\item {Proveniência:(De \textunderscore corolla\textunderscore )}
\end{itemize}
Diz-se das columnas com ornatos de fôlhas ou flôres em espiral.
\section{Coróllula}
\begin{itemize}
\item {Grp. gram.:f.}
\end{itemize}
(Dem. de \textunderscore corolla\textunderscore )
\section{Corólula}
\begin{itemize}
\item {Grp. gram.:f.}
\end{itemize}
(Dem. de \textunderscore corola\textunderscore )
\section{Corombó}
\begin{itemize}
\item {Grp. gram.:adj.}
\end{itemize}
\begin{itemize}
\item {Utilização:Bras. do N}
\end{itemize}
Diz-se das reses, que têm chifres pequenos ou partidos.
\section{Corómen}
\begin{itemize}
\item {Grp. gram.:m.}
\end{itemize}
\begin{itemize}
\item {Utilização:Des.}
\end{itemize}
Vestuário de rameira.
\section{Corona}
\begin{itemize}
\item {Grp. gram.:f.}
\end{itemize}
O mesmo que \textunderscore corôa\textunderscore , em Architectura.
\section{Coronal}
\begin{itemize}
\item {Grp. gram.:adj.}
\end{itemize}
\begin{itemize}
\item {Grp. gram.:M.}
\end{itemize}
\begin{itemize}
\item {Utilização:Anat.}
\end{itemize}
\begin{itemize}
\item {Proveniência:(Lat. \textunderscore coronalis\textunderscore )}
\end{itemize}
Relativo a corôa.
Que tem fórma de corôa.
Osso correspondente á testa e á parte superior e anterior da cabeça.
\section{Coronário}
\begin{itemize}
\item {Grp. gram.:adj.}
\end{itemize}
\begin{itemize}
\item {Proveniência:(Lat. \textunderscore coronarius\textunderscore )}
\end{itemize}
Que representa a curvatura da corôa.
\section{Coronco}
\begin{itemize}
\item {Grp. gram.:m.}
\end{itemize}
Árvore da Guiné portuguesa.
\section{Coronel}
\begin{itemize}
\item {Grp. gram.:m.}
\end{itemize}
\begin{itemize}
\item {Proveniência:(It. \textunderscore colonnello\textunderscore )}
\end{itemize}
Graduação militar do commandante de um regimento.
\section{Coronel}
\begin{itemize}
\item {Grp. gram.:m.}
\end{itemize}
\begin{itemize}
\item {Utilização:Heráld.}
\end{itemize}
\begin{itemize}
\item {Proveniência:(Do lat. \textunderscore corona\textunderscore )}
\end{itemize}
Corôa aberta, que remata superiormente um escudo.
\section{Coronge}
\begin{itemize}
\item {Grp. gram.:m.}
\end{itemize}
Árvore indiana, (\textunderscore pongamia glabra\textunderscore ).
\section{Corongo}
\begin{itemize}
\item {Grp. gram.:m.}
\end{itemize}
\begin{itemize}
\item {Utilização:Bras}
\end{itemize}
Peixe serpentiforme, semelhante ao caramuru.
\section{Coronha}
\begin{itemize}
\item {Grp. gram.:f.}
\end{itemize}
\begin{itemize}
\item {Proveniência:(Do cast. \textunderscore cureña\textunderscore )}
\end{itemize}
Parte da espingarda ou de armas semelhantes, em que se encaixa ou se prende o cano.
Tudo que numa espingarda é de madeira, excepto a vareta, ainda que de madeira seja.
\section{Coronhada}
\begin{itemize}
\item {Grp. gram.:f.}
\end{itemize}
Pancada com a coronha.
\section{Coronheiro}
\begin{itemize}
\item {Grp. gram.:m.}
\end{itemize}
Aquelle que faz coronhas.
\section{Coronheiro}
\begin{itemize}
\item {Grp. gram.:m.}
\end{itemize}
Aquelle que transporta coronhos; carregador.
\section{Coronho}
\begin{itemize}
\item {Grp. gram.:m.}
\end{itemize}
\begin{itemize}
\item {Utilização:Prov.}
\end{itemize}
Feixe ou carga, que se leva á cabeça.
\section{Corónide}
\begin{itemize}
\item {Grp. gram.:f.}
\end{itemize}
\begin{itemize}
\item {Proveniência:(Lat. \textunderscore coronis\textunderscore )}
\end{itemize}
Cornija, corôa, remate, complemento.
\section{Coroniforme}
\begin{itemize}
\item {Grp. gram.:adj.}
\end{itemize}
\begin{itemize}
\item {Proveniência:(Do lat. \textunderscore corona\textunderscore  + \textunderscore forma\textunderscore )}
\end{itemize}
Que tem fórma de corôa.
\section{Coronilha}
\begin{itemize}
\item {Grp. gram.:f.}
\end{itemize}
\begin{itemize}
\item {Proveniência:(Lat. \textunderscore coronilla\textunderscore )}
\end{itemize}
Planta leguminosa.
Cabelleira curta de clérigo.
\section{Coronilina}
\begin{itemize}
\item {Grp. gram.:f.}
\end{itemize}
Medicamento, extraido da coronilha.
\section{Coronillina}
\begin{itemize}
\item {Grp. gram.:f.}
\end{itemize}
Medicamento, extrahido da coronilha.
\section{Corónium}
\begin{itemize}
\item {Grp. gram.:m.}
\end{itemize}
\begin{itemize}
\item {Utilização:Phýs.}
\end{itemize}
Corpo, descoberto na corôa solar e, recentemente, também no ar.
\section{Coronoide}
\begin{itemize}
\item {Grp. gram.:adj.}
\end{itemize}
\begin{itemize}
\item {Proveniência:(Do gr. \textunderscore korone\textunderscore  + \textunderscore eidos\textunderscore )}
\end{itemize}
Que tem a fórma de bico de gralha.
\section{Coronoídeo}
\begin{itemize}
\item {Grp. gram.:adj.}
\end{itemize}
\begin{itemize}
\item {Proveniência:(Do gr. \textunderscore korone\textunderscore  + \textunderscore eidos\textunderscore )}
\end{itemize}
Que tem a fórma de bico de gralha.
\section{Corónula}
\begin{itemize}
\item {Grp. gram.:f.}
\end{itemize}
\begin{itemize}
\item {Proveniência:(Lat. \textunderscore coronula\textunderscore )}
\end{itemize}
Mollusco parasita.
Pequena corôa.
\section{Coróphia}
\begin{itemize}
\item {Grp. gram.:f.}
\end{itemize}
Mollusco, que devora outros. Cf. Fil. Simões, \textunderscore Beiramar\textunderscore , 243.
\section{Coropião}
\begin{itemize}
\item {Grp. gram.:f.}
\end{itemize}
Ave brasileira.
\section{Coropira}
\begin{itemize}
\item {Grp. gram.:f.}
\end{itemize}
\begin{itemize}
\item {Utilização:Bras}
\end{itemize}
(Fórma divergente de \textunderscore curupira\textunderscore )
\section{Cororo}
\begin{itemize}
\item {Grp. gram.:m.}
\end{itemize}
Ave da América meridional.
\section{Corosil}
\begin{itemize}
\item {Grp. gram.:m.}
\end{itemize}
Espécie de colmo.
\section{Corovina}
\begin{itemize}
\item {Grp. gram.:f.}
\end{itemize}
Peixe brasileiro.
(O mesmo que \textunderscore corvina\textunderscore ?)
\section{Corpaço}
\begin{itemize}
\item {Grp. gram.:m.}
\end{itemize}
O mesmo que \textunderscore corpanzil\textunderscore .
\section{Corpanço}
\begin{itemize}
\item {Grp. gram.:m.}
\end{itemize}
\begin{itemize}
\item {Utilização:Ant.}
\end{itemize}
Corpo grande? corpanzil? Cf. Cortesão, \textunderscore Subs.\textunderscore 
\section{Corpanzil}
\begin{itemize}
\item {Grp. gram.:m.}
\end{itemize}
\begin{itemize}
\item {Utilização:Fam.}
\end{itemize}
\begin{itemize}
\item {Proveniência:(De \textunderscore corpo\textunderscore )}
\end{itemize}
Grande estatura.
Pessôa encorpada.
\section{Corpanzudo}
\begin{itemize}
\item {Grp. gram.:adj.}
\end{itemize}
\begin{itemize}
\item {Utilização:Chul.}
\end{itemize}
Corpulento.
\section{Corpete}
\begin{itemize}
\item {fónica:pê}
\end{itemize}
\begin{itemize}
\item {Grp. gram.:f.}
\end{itemize}
\begin{itemize}
\item {Proveniência:(De \textunderscore corpo\textunderscore )}
\end{itemize}
Peça do vestuário feminino, que se ajusta ao peito.
Justilho.
\section{Corpinheira}
\begin{itemize}
\item {Grp. gram.:f.}
\end{itemize}
\begin{itemize}
\item {Utilização:Bras}
\end{itemize}
Mulher, que faz corpinhos ou corpetes.
\section{Corpinheiro}
\begin{itemize}
\item {Grp. gram.:m.}
\end{itemize}
Fabricante de corpinhos ou corpetes para senhora.
\section{Corpinho}
\begin{itemize}
\item {Grp. gram.:m.}
\end{itemize}
(V.corpete)
\section{Corpo}
\begin{itemize}
\item {Grp. gram.:m.}
\end{itemize}
\begin{itemize}
\item {Utilização:Pop.}
\end{itemize}
\begin{itemize}
\item {Grp. gram.:Loc. adv.}
\end{itemize}
\begin{itemize}
\item {Grp. gram.:Loc. adv.}
\end{itemize}
\begin{itemize}
\item {Grp. gram.:Loc. adv.}
\end{itemize}
\begin{itemize}
\item {Proveniência:(Lat. \textunderscore corpus\textunderscore )}
\end{itemize}
Porção distinta de matéria.
Tudo que tem pêso e extensão.
Substância conformada de cada homem e de cada animal.
A parte material de um homem ou de um animal, vivo ou morto.
Tudo que, por qualidades especiaes, impressiona os nossos sentidos.
Espessura, consistencia.
Aquillo, que constitue a essência de alguma coisa.
Cadáver humano.
Parte do vestuário feminino, que se ajusta ao corpo.
Multidão.
Parte central e principal de certos objectos.
Collecção.
Vulto, importância.
Cooperação.
Regimento.
Conjunto de militares que constituem arma especial.
Contexto.
Base.
\textunderscore Corpo de delicto\textunderscore : facto material, em que se baseia a prova de um crime.
\textunderscore Dar de corpo\textunderscore , defecar.
\textunderscore De corpo bem feito\textunderscore , com o fato ajustado ao corpo, sem capa nem outro agasalho.
\textunderscore Em corpo\textunderscore , sem manto, sem capa, sem agasalho.
\textunderscore Corpo a corpo\textunderscore , em luta, de braço a braço, de corpo contra corpo.
\section{Corporação}
\begin{itemize}
\item {Grp. gram.:f.}
\end{itemize}
\begin{itemize}
\item {Proveniência:(Lat. \textunderscore corporatio\textunderscore )}
\end{itemize}
Conjunto de pessôas, sujeitas á mesma regra ou estatuto.
Indivíduos que, collectivamente, administram ou dirigem certos negócios de interesse público.
Associação.
\section{Corporal}
\begin{itemize}
\item {Grp. gram.:adj.}
\end{itemize}
\begin{itemize}
\item {Grp. gram.:M.}
\end{itemize}
\begin{itemize}
\item {Utilização:Des.}
\end{itemize}
Que tem corpo.
Relativo a corpo.
Material.
Corpo da igreja, entre o cruzeiro e a porta principal.
\section{Corporal}
\begin{itemize}
\item {Grp. gram.:m.}
\end{itemize}
Pano, em que o sacerdote colloca o cálix e a hóstia, no altar.
(B. lat. \textunderscore corporale\textunderscore )
\section{Corporalidade}
\begin{itemize}
\item {Grp. gram.:f.}
\end{itemize}
\begin{itemize}
\item {Proveniência:(Lat. \textunderscore corporalitas\textunderscore )}
\end{itemize}
Qualidade daquillo que é corpóreo.
\section{Corporalizar}
\begin{itemize}
\item {Grp. gram.:v. i.}
\end{itemize}
\begin{itemize}
\item {Proveniência:(De \textunderscore corporal\textunderscore )}
\end{itemize}
Dar corpo a.
Materializar.
Tornar palpável, patente, evidente.
\section{Corporalmente}
\begin{itemize}
\item {Grp. gram.:adv.}
\end{itemize}
\begin{itemize}
\item {Proveniência:(De \textunderscore corporal\textunderscore )}
\end{itemize}
Em pessôa; pessoalmente.
\section{Corporativo}
\begin{itemize}
\item {Grp. gram.:adj.}
\end{itemize}
\begin{itemize}
\item {Utilização:bras}
\end{itemize}
\begin{itemize}
\item {Utilização:Neol.}
\end{itemize}
Relativo a corporações: \textunderscore interesses corporativos\textunderscore .
\section{Corporatura}
\begin{itemize}
\item {Grp. gram.:f.}
\end{itemize}
\begin{itemize}
\item {Proveniência:(Lat. \textunderscore corporatura\textunderscore )}
\end{itemize}
Fórma externa de um corpo.
Estatura.
\section{Corporeidade}
\begin{itemize}
\item {Grp. gram.:f.}
\end{itemize}
Qualidade daquillo que é \textunderscore corpóreo\textunderscore .
\section{Corpóreo}
\begin{itemize}
\item {Grp. gram.:adj.}
\end{itemize}
\begin{itemize}
\item {Proveniência:(Lat. \textunderscore corporens\textunderscore )}
\end{itemize}
Relativo a corpo; corporal.
Material.
\section{Corporificação}
\begin{itemize}
\item {Grp. gram.:f.}
\end{itemize}
Acto do corporificar.
\section{Corporificar}
\begin{itemize}
\item {Grp. gram.:v. t.}
\end{itemize}
\begin{itemize}
\item {Proveniência:(Do lat. \textunderscore corpus\textunderscore  + \textunderscore facere\textunderscore )}
\end{itemize}
Attribuír corpo a (o que não o tem).
Solidificar, reunindo num corpo elementos dispersos.
\section{Corpo-santo}
\begin{itemize}
\item {Grp. gram.:m.}
\end{itemize}
\begin{itemize}
\item {Utilização:Des.}
\end{itemize}
O mesmo que \textunderscore santelmo\textunderscore . Cf. G. Viana, \textunderscore Apostilas\textunderscore .
\section{Corpulência}
\begin{itemize}
\item {Grp. gram.:f.}
\end{itemize}
\begin{itemize}
\item {Proveniência:(Lat. \textunderscore corpulentia\textunderscore )}
\end{itemize}
Qualidade daquillo ou de quem é corpulento.
\section{Corpulento}
\begin{itemize}
\item {Grp. gram.:adj.}
\end{itemize}
\begin{itemize}
\item {Proveniência:(Lat. \textunderscore corpulentus\textunderscore )}
\end{itemize}
Que tem grande corpo; encorpado.
Grosso.
Alto e grosso.
Obeso.
\section{Corpuscular}
\begin{itemize}
\item {Grp. gram.:adj.}
\end{itemize}
\begin{itemize}
\item {Proveniência:(De \textunderscore corpúsculo\textunderscore )}
\end{itemize}
Relativo a corpúsculos.
Diz-se do systema philosóphico, que explica os phenomenos pelo movimento, descanso, disposição, etc., dos corpúsculos.
\section{Corpusculista}
\begin{itemize}
\item {Grp. gram.:m.}
\end{itemize}
Sectário da philosophia corpuscular.
\section{Corpúsculo}
\begin{itemize}
\item {Grp. gram.:m.}
\end{itemize}
\begin{itemize}
\item {Proveniência:(Lat. \textunderscore corpusculum\textunderscore )}
\end{itemize}
Corpo pequenissimo.
Molécula.
Átomo.
\section{Corra}
\begin{itemize}
\item {fónica:cô}
\end{itemize}
\begin{itemize}
\item {Grp. gram.:f.}
\end{itemize}
\begin{itemize}
\item {Proveniência:(De \textunderscore correr\textunderscore )}
\end{itemize}
Corda de esparto.
Correia.
Fasquia de castínçal.
\section{Corrá}
\begin{itemize}
\item {Grp. gram.:m.}
\end{itemize}
\begin{itemize}
\item {Utilização:Bras}
\end{itemize}
Fruta encarnada, odorífera e medicinal.
\section{Corre}
\begin{itemize}
\item {fónica:cô}
\end{itemize}
\begin{itemize}
\item {Grp. gram.:f.}
\end{itemize}
\begin{itemize}
\item {Utilização:Prov.}
\end{itemize}
\begin{itemize}
\item {Utilização:trasm.}
\end{itemize}
Haste delgada e trepadora do feijoeiro hortense.
(Cp. \textunderscore corra\textunderscore )
\section{Corré}
(fem. de \textunderscore corréu\textunderscore )
\section{Correada}
\begin{itemize}
\item {Grp. gram.:f.}
\end{itemize}
Pancada com correia.
\section{Correagem}
\begin{itemize}
\item {Grp. gram.:f.}
\end{itemize}
\begin{itemize}
\item {Proveniência:(De \textunderscore correia\textunderscore )}
\end{itemize}
Conjunto de correias, especialmente as do uniforme militar.
\section{Correame}
\begin{itemize}
\item {Grp. gram.:m.}
\end{itemize}
\begin{itemize}
\item {Proveniência:(De \textunderscore correia\textunderscore )}
\end{itemize}
Conjunto de correias, especialmente as do uniforme militar.
\section{Correão}
\begin{itemize}
\item {Grp. gram.:f.}
\end{itemize}
\begin{itemize}
\item {Utilização:Prov.}
\end{itemize}
\begin{itemize}
\item {Utilização:Bras. de Minas}
\end{itemize}
\begin{itemize}
\item {Proveniência:(De \textunderscore correia\textunderscore )}
\end{itemize}
Grande correia.
Alga marinha, em fórma de correia e aproveitada para adubos de terra.
Cinta larga de coiro e fivela.
\section{Correaria}
\begin{itemize}
\item {Grp. gram.:f.}
\end{itemize}
\begin{itemize}
\item {Proveniência:(De \textunderscore correia\textunderscore )}
\end{itemize}
Lugar ou casa, onde se vendem correias e outras obras de coiro.
\section{Corre-caminho}
\begin{itemize}
\item {Grp. gram.:m.}
\end{itemize}
Avezinha do archipélago da Madeira, (\textunderscore anthus berteloti\textunderscore ).
\section{Corre-campo}
\begin{itemize}
\item {Grp. gram.:f.}
\end{itemize}
Cobra venenosa do Brasil.
\section{Correcção}
\begin{itemize}
\item {Grp. gram.:f.}
\end{itemize}
\begin{itemize}
\item {Proveniência:(Lat. \textunderscore correctio\textunderscore )}
\end{itemize}
Acto ou effeito de corrigir.
Qualidade daquillo ou de quem é correcto.
Estabelecimento público, em que se recolhem e se corrigem menores delinquentes ou vadios.
\section{Correccional}
\begin{itemize}
\item {Grp. gram.:adj.}
\end{itemize}
\begin{itemize}
\item {Grp. gram.:M.}
\end{itemize}
\begin{itemize}
\item {Proveniência:(Do lat. \textunderscore correctio\textunderscore )}
\end{itemize}
Relativo a correcção.
Diz-se do tribunal, em que são julgadas, sem assistência de jury, as causas criminaes de menor monta.
E diz-se da pena, que se applica ás contravenções e delictos de menor importância.
Jurisdicção dos tribunaes correccionaes.
\section{Correccionalmente}
\begin{itemize}
\item {Grp. gram.:adv.}
\end{itemize}
\begin{itemize}
\item {Proveniência:(De \textunderscore correccional\textunderscore )}
\end{itemize}
Segundo processo dos tribunaes correccionaes.
\section{Corre-costas}
\begin{itemize}
\item {Grp. gram.:m.}
\end{itemize}
\begin{itemize}
\item {Utilização:Bras}
\end{itemize}
Barco de cabotagem.
\section{Correctamente}
\begin{itemize}
\item {Grp. gram.:adv.}
\end{itemize}
De modo correcto.
Com correcção.
\section{Correctivo}
\begin{itemize}
\item {Grp. gram.:adj.}
\end{itemize}
\begin{itemize}
\item {Grp. gram.:M.}
\end{itemize}
\begin{itemize}
\item {Proveniência:(De \textunderscore correcto\textunderscore )}
\end{itemize}
Que corrige.
Próprio para corrigir.
Aquillo, com que se corrige.
Palavra ou phrase, que modifica a dureza de outra.
Observação; censura.
Punição.
Substância, que modifica o effeito ou o sabor de certos medicamentos.
\section{Correcto}
\begin{itemize}
\item {Grp. gram.:Adj.}
\end{itemize}
\begin{itemize}
\item {Proveniência:(Lat. \textunderscore correctus\textunderscore )}
\end{itemize}
Irreprehensivel; íntegro.
Honesto; digno.
\section{Corrector}
\begin{itemize}
\item {Grp. gram.:m.}
\end{itemize}
\begin{itemize}
\item {Proveniência:(Lat. \textunderscore corrector\textunderscore )}
\end{itemize}
Aquelle que corrige.
\section{Correctório}
\begin{itemize}
\item {Grp. gram.:m.}
\end{itemize}
\begin{itemize}
\item {Utilização:Des.}
\end{itemize}
\begin{itemize}
\item {Proveniência:(De \textunderscore correcto\textunderscore )}
\end{itemize}
Registo de correcções e penas.
Penitencial.
\section{Correctriz}
\begin{itemize}
\item {Grp. gram.:f.}
\end{itemize}
\begin{itemize}
\item {Proveniência:(De \textunderscore corrector\textunderscore )}
\end{itemize}
Superiora conventual da Ordem-terceira de San-Francisco de Paula.
\section{Corredeira}
\begin{itemize}
\item {Grp. gram.:f.}
\end{itemize}
\begin{itemize}
\item {Utilização:Bras}
\end{itemize}
\begin{itemize}
\item {Utilização:Gal}
\end{itemize}
\begin{itemize}
\item {Proveniência:(De \textunderscore correr\textunderscore )}
\end{itemize}
Parte de um rio, em que as águas, por differença de nível, correm mais velozes, difficultando ou tornando perigosa a navegação.
Rápido.
Cachoeira.
\section{Corredela}
\begin{itemize}
\item {Grp. gram.:f.}
\end{itemize}
\begin{itemize}
\item {Utilização:Pop.}
\end{itemize}
Acto de correr.
\section{Corredemptor}
\begin{itemize}
\item {Grp. gram.:m.  e  adj.}
\end{itemize}
\begin{itemize}
\item {Proveniência:(De \textunderscore com...\textunderscore  + \textunderscore redemptor\textunderscore )}
\end{itemize}
Aquelle, que redime com outrem.
\section{Corredentor}
\begin{itemize}
\item {Grp. gram.:m.  e  adj.}
\end{itemize}
\begin{itemize}
\item {Proveniência:(De \textunderscore com...\textunderscore  + \textunderscore redemptor\textunderscore )}
\end{itemize}
Aquelle, que redime com outrem.
\section{Corrediça}
\begin{itemize}
\item {Grp. gram.:f.}
\end{itemize}
\begin{itemize}
\item {Proveniência:(De \textunderscore corrediço\textunderscore )}
\end{itemize}
Encaixe dos batentes de porta, janela, tampa, etc.
Bastidor de theatro.
Cortina de correr.
Estore.
\section{Corrediço}
\begin{itemize}
\item {Grp. gram.:adj.}
\end{itemize}
O mesmo que \textunderscore corredio\textunderscore .
\section{Corredinha}
\begin{itemize}
\item {Grp. gram.:f.}
\end{itemize}
\textunderscore Andar em corredinha\textunderscore , transitar apressadamente, de um lado para o outro.
(Por \textunderscore corridinha\textunderscore  de \textunderscore corrida\textunderscore )
\section{Corredio}
\begin{itemize}
\item {Grp. gram.:adj.}
\end{itemize}
\begin{itemize}
\item {Proveniência:(De \textunderscore correr\textunderscore )}
\end{itemize}
Que corre facilmente.
Que escorrega, que resvala.
Liso: \textunderscore cabello corredio\textunderscore .
Desembaraçado, facil.
\section{Corredoira}
\begin{itemize}
\item {Grp. gram.:f.}
\end{itemize}
\begin{itemize}
\item {Utilização:Ant.}
\end{itemize}
\begin{itemize}
\item {Utilização:Prov.}
\end{itemize}
\begin{itemize}
\item {Utilização:alent.}
\end{itemize}
\begin{itemize}
\item {Utilização:Prov.}
\end{itemize}
\begin{itemize}
\item {Utilização:dur.}
\end{itemize}
\begin{itemize}
\item {Proveniência:(De \textunderscore corredoiro\textunderscore )}
\end{itemize}
Peça, que está por baixo da mó do moinho.
Rua larga e direita, própria para corridas.
Lugar, destinado nas feiras ao gado cavallar, muar e asinino.
O mesmo que \textunderscore corrida\textunderscore : \textunderscore á corredoira\textunderscore  ou \textunderscore ás corredoiras\textunderscore , de corrida.
\section{Corredoiro}
\begin{itemize}
\item {Grp. gram.:m.}
\end{itemize}
\begin{itemize}
\item {Proveniência:(De \textunderscore correr\textunderscore )}
\end{itemize}
Lugar próprio para corridas.
\section{Corredor}
\begin{itemize}
\item {Grp. gram.:adj.}
\end{itemize}
\begin{itemize}
\item {Grp. gram.:M.}
\end{itemize}
\begin{itemize}
\item {Utilização:Prov.}
\end{itemize}
\begin{itemize}
\item {Utilização:minh.}
\end{itemize}
\begin{itemize}
\item {Utilização:Bras}
\end{itemize}
\begin{itemize}
\item {Proveniência:(De \textunderscore correr\textunderscore )}
\end{itemize}
Que corre bem.
Passagem comprida, ordinariamente estreita, que estabelece communicação entre compartimentos, no interior de uma casa.
Galeria em volta de um edifício.
Passeio de um jardim.
Viela, que separa taboleiros, nas salinas.
Passagem estreita.
Utensilio de metal, com que se tiram legumes secos de sacas ou barris.
Aquelle, que corre por fado, lobis-homem.
Individuo, que cavalga em corridas.
\section{Corredora}
\begin{itemize}
\item {Grp. gram.:f.}
\end{itemize}
\begin{itemize}
\item {Proveniência:(De \textunderscore corredor\textunderscore )}
\end{itemize}
Grade pesada de madeira ou ferro, que corria verticalmente entre dois frisos, para augmentar os obstáculos na entrada de uma fortificação.
\section{Corredor-real}
\begin{itemize}
\item {Grp. gram.:m.}
\end{itemize}
Valla, que rodeia a salina e distribue a água pelas peças ou a lança na escoadeira.
\section{Corredoura}
\begin{itemize}
\item {Grp. gram.:f.}
\end{itemize}
\begin{itemize}
\item {Utilização:Ant.}
\end{itemize}
\begin{itemize}
\item {Utilização:Prov.}
\end{itemize}
\begin{itemize}
\item {Utilização:alent.}
\end{itemize}
\begin{itemize}
\item {Utilização:Prov.}
\end{itemize}
\begin{itemize}
\item {Utilização:dur.}
\end{itemize}
\begin{itemize}
\item {Proveniência:(De \textunderscore corredoiro\textunderscore )}
\end{itemize}
Peça, que está por baixo da mó do moinho.
Rua larga e direita, própria para corridas.
Lugar, destinado nas feiras ao gado cavallar, muar e asinino.
O mesmo que \textunderscore corrida\textunderscore : \textunderscore á corredoura\textunderscore  ou \textunderscore ás corredouras\textunderscore , de corrida.
\section{Corredouro}
\begin{itemize}
\item {Grp. gram.:m.}
\end{itemize}
\begin{itemize}
\item {Proveniência:(De \textunderscore correr\textunderscore )}
\end{itemize}
Lugar próprio para corridas.
\section{Corredura}
\begin{itemize}
\item {Grp. gram.:f.}
\end{itemize}
\begin{itemize}
\item {Proveniência:(De \textunderscore correr\textunderscore )}
\end{itemize}
Líquido, que adhere ás medidas por que se vende, em prejuizo do comprador.
Correia.
\section{Correeiro}
\begin{itemize}
\item {Grp. gram.:m.}
\end{itemize}
\begin{itemize}
\item {Proveniência:(De \textunderscore correia\textunderscore )}
\end{itemize}
Aquelle que faz ou vende correias ou outras obras de coiro.
\section{Correento}
\begin{itemize}
\item {Grp. gram.:adj.}
\end{itemize}
\begin{itemize}
\item {Proveniência:(De \textunderscore correia\textunderscore )}
\end{itemize}
Semelhante ao coiro.
Duro, como coiro.
\section{Corregedoiro}
\begin{itemize}
\item {Grp. gram.:adj.}
\end{itemize}
\begin{itemize}
\item {Utilização:Des.}
\end{itemize}
Que pode ou deve sêr corrigido.
\section{Corregedor}
\begin{itemize}
\item {Grp. gram.:m.}
\end{itemize}
\begin{itemize}
\item {Proveniência:(De \textunderscore correger\textunderscore )}
\end{itemize}
Antigo magistrado judicial.
\section{Corregedoria}
\begin{itemize}
\item {Grp. gram.:f.}
\end{itemize}
\begin{itemize}
\item {Proveniência:(De \textunderscore corregedor\textunderscore )}
\end{itemize}
Cargo ou jurisdicção de corregedor.
Área da sua jurisdicção.
\section{Corregedouro}
\begin{itemize}
\item {Grp. gram.:adj.}
\end{itemize}
\begin{itemize}
\item {Utilização:Des.}
\end{itemize}
Que pode ou deve sêr corrigido.
\section{Corregente}
\begin{itemize}
\item {Grp. gram.:m.  e  f.}
\end{itemize}
\begin{itemize}
\item {Proveniência:(De \textunderscore com...\textunderscore  + \textunderscore regente\textunderscore )}
\end{itemize}
Pessôa, que é regente com outrem.
\section{Correger}
\begin{itemize}
\item {Grp. gram.:v. t.}
\end{itemize}
\begin{itemize}
\item {Utilização:Ant.}
\end{itemize}
O mesmo que \textunderscore corrigir\textunderscore .
\section{Corregesco}
\begin{itemize}
\item {fónica:gês}
\end{itemize}
\begin{itemize}
\item {Grp. gram.:adj.}
\end{itemize}
\begin{itemize}
\item {Proveniência:(De \textunderscore Corrégio\textunderscore , n. p.)}
\end{itemize}
Que imita a maneira de Corrégio.
Que é da escola de Corrégio, pintor.
\section{Corregimento}
\begin{itemize}
\item {Grp. gram.:m.}
\end{itemize}
\begin{itemize}
\item {Utilização:Des.}
\end{itemize}
Correcção.
Multa.
Ornamento, alfaia:«\textunderscore e os corregimentos de suas capellas poremos em nossas mesquitas\textunderscore ». Azurara, \textunderscore Chrón. de D. João I\textunderscore , c. 69.
\section{Córrego}
\begin{itemize}
\item {Grp. gram.:m.}
\end{itemize}
\begin{itemize}
\item {Proveniência:(Do lat. \textunderscore corrugus\textunderscore )}
\end{itemize}
Regueiro, sulco aberto pelas águas correntes.
Caminho estreito entre montes ou entre muros.
Atalho fundo.
\section{Correia}
\begin{itemize}
\item {Grp. gram.:f.}
\end{itemize}
\begin{itemize}
\item {Proveniência:(Do lat. \textunderscore corrigia\textunderscore )}
\end{itemize}
Tira de coiro.
Planta diosmácea.
Espécie de jôgo popular.
\section{Correia}
\begin{itemize}
\item {Grp. gram.:f.}
\end{itemize}
\begin{itemize}
\item {Proveniência:(De \textunderscore Correia\textunderscore , n. p.)}
\end{itemize}
Variedade de pêra grande e sucosa.
\section{Correia-de-inverno}
\begin{itemize}
\item {Grp. gram.:f.}
\end{itemize}
Variedade de pêra.
\section{Correição}
\begin{itemize}
\item {Grp. gram.:f.}
\end{itemize}
\begin{itemize}
\item {Utilização:Bras}
\end{itemize}
\begin{itemize}
\item {Utilização:Prov.}
\end{itemize}
\begin{itemize}
\item {Proveniência:(Lat. \textunderscore correctio\textunderscore )}
\end{itemize}
Correcção.
Visita do corregedor aos cartórios da sua alçada.
Exame feito pelo juiz nos cartórios da sua comarca.
Pequena formiga preta.
Vistoria que, nos termos das posturas municipaes, se deve fazer ás regueiras públicas, para se verificar se cada proprietário de terreno marginal limpou a sua testada, como lhe cumpre.
\section{Correio}
\begin{itemize}
\item {Grp. gram.:m.}
\end{itemize}
\begin{itemize}
\item {Proveniência:(Do rad. de \textunderscore correr\textunderscore )}
\end{itemize}
Pessôa, expressamente encarregada de levar despachos, correspondência, etc.
Carteiro.
Repartição pública, que recebe e expede correspondência pública ou particular.
Local, onde se recebe a correspondência, que deve sêr expedida.
Mala, em que se transporta a correspondência.
Correspondência.
Aquelle, que traz noticias.
Prenúncio.
\textunderscore Combóio correio\textunderscore , combóio, que transporta a correspondência de umas para outras terras.
\textunderscore Pombo correio\textunderscore , pombo adestrado, que transporta para determinados pontos communicações escritas.
\section{Correitor}
\begin{itemize}
\item {Grp. gram.:m.}
\end{itemize}
\begin{itemize}
\item {Utilização:Ant.}
\end{itemize}
\begin{itemize}
\item {Proveniência:(Lat. \textunderscore corrector\textunderscore )}
\end{itemize}
Revisor de provas typográphicas.
\section{Correjales}
\begin{itemize}
\item {Grp. gram.:m. pl.}
\end{itemize}
\begin{itemize}
\item {Utilização:Prov.}
\end{itemize}
\begin{itemize}
\item {Utilização:alg.}
\end{itemize}
\begin{itemize}
\item {Proveniência:(De \textunderscore correr\textunderscore ?)}
\end{itemize}
Trabalhos, fadigas.
\section{Córrela}
\begin{itemize}
\item {Grp. gram.:f.}
\end{itemize}
\begin{itemize}
\item {Utilização:Prov.}
\end{itemize}
O mesmo que \textunderscore corla\textunderscore ; vómito bilioso.
(Metát. de \textunderscore cólera\textunderscore )
\section{Correlação}
\begin{itemize}
\item {Grp. gram.:f.}
\end{itemize}
\begin{itemize}
\item {Proveniência:(De \textunderscore com...\textunderscore  + \textunderscore relação\textunderscore )}
\end{itemize}
Relação mútua entre pessôas ou coisas.
\section{Correlacionar}
\begin{itemize}
\item {Grp. gram.:v. t.}
\end{itemize}
\begin{itemize}
\item {Proveniência:(De \textunderscore correlação\textunderscore )}
\end{itemize}
Estabelecer relação entre; dar correlação a.
\section{Correlatar}
\begin{itemize}
\item {Grp. gram.:v. t.}
\end{itemize}
\begin{itemize}
\item {Proveniência:(De \textunderscore com...\textunderscore  + \textunderscore relatar\textunderscore )}
\end{itemize}
Estabelecer relações entre.
\section{Correlativamente}
\begin{itemize}
\item {Grp. gram.:adv.}
\end{itemize}
De modo correlativo.
\section{Correlativo}
\begin{itemize}
\item {Grp. gram.:adj.}
\end{itemize}
\begin{itemize}
\item {Proveniência:(De \textunderscore com...\textunderscore  + \textunderscore relativo\textunderscore )}
\end{itemize}
Que constitue um dos termos da relação reciproca.
Em que há dependência mútua.
\section{Correligionário}
\begin{itemize}
\item {Grp. gram.:m.  e  adj.}
\end{itemize}
\begin{itemize}
\item {Proveniência:(De \textunderscore com...\textunderscore  + \textunderscore religião\textunderscore )}
\end{itemize}
Aquelle, que tem a mesma religião, partido ou systema, que outrem.
\section{Corrença}
\begin{itemize}
\item {Grp. gram.:f.}
\end{itemize}
\begin{itemize}
\item {Utilização:Ant.}
\end{itemize}
\begin{itemize}
\item {Proveniência:(De \textunderscore correr\textunderscore )}
\end{itemize}
Diarreia.
\section{Correntão}
\begin{itemize}
\item {Grp. gram.:m.}
\end{itemize}
\begin{itemize}
\item {Utilização:Prov.}
\end{itemize}
\begin{itemize}
\item {Utilização:alent.}
\end{itemize}
\begin{itemize}
\item {Grp. gram.:Adj.}
\end{itemize}
\begin{itemize}
\item {Utilização:Prov.}
\end{itemize}
\begin{itemize}
\item {Utilização:alent.}
\end{itemize}
\begin{itemize}
\item {Proveniência:(De \textunderscore corrente\textunderscore )}
\end{itemize}
Torrente, rio caudaloso.
Diz-se do indivíduo lhano, affável.
\section{Corrente}
\begin{itemize}
\item {Grp. gram.:adj.}
\end{itemize}
\begin{itemize}
\item {Utilização:Fig.}
\end{itemize}
\begin{itemize}
\item {Grp. gram.:F.}
\end{itemize}
\begin{itemize}
\item {Utilização:Mús.}
\end{itemize}
\begin{itemize}
\item {Grp. gram.:Adv.}
\end{itemize}
\begin{itemize}
\item {Grp. gram.:M.}
\end{itemize}
\begin{itemize}
\item {Proveniência:(Lat. \textunderscore currens\textunderscore )}
\end{itemize}
Que corre: \textunderscore águas correntes\textunderscore .
Fácil, expedito.
Fluente: \textunderscore estilo corrente\textunderscore .
Claro, evídente, sabido: \textunderscore verdades correntes\textunderscore .
Vulgar.
Curso de águas vivas.
Rio, ribeira.
Correnteza.
O ar que corre, vento: \textunderscore fecha a janela, que vem daí uma corrente\textunderscore .
Grilhão, cadeia de metal: \textunderscore a corrente do relógio\textunderscore .
Decurso.
Espécie de dança do século XVII e XVIII, em compasso ternário e andamento animado.
Espécie de caixa de madeira, com o fundo furado, onde cai a calda da cana que não chegou a coalhar-se, (nos engenhos de açúcar).
Correntemente.
\textunderscore Andar ao corrente\textunderscore , estar informado, têr conhecimento.
Têr saldas as contas.
* \textunderscore Pôr ao corrente\textunderscore , informar, esclarecer.
\section{Correntemente}
\begin{itemize}
\item {Grp. gram.:adv.}
\end{itemize}
De modo corrente.
Commummente.
Vulgarmente.
Com desembaraço: \textunderscore falar correntemente\textunderscore .
Correctamente.
\section{Correnteza}
\begin{itemize}
\item {Grp. gram.:f.}
\end{itemize}
\begin{itemize}
\item {Proveniência:(De \textunderscore corrente\textunderscore )}
\end{itemize}
Corrente.
Série.
Conjunto de edificios, dispostos em linha direita.
Desembaraço.
\section{Correntiamente}
\begin{itemize}
\item {Grp. gram.:adv.}
\end{itemize}
\begin{itemize}
\item {Proveniência:(De \textunderscore correntio\textunderscore )}
\end{itemize}
O mesmo que \textunderscore correntemente\textunderscore .
\section{Correntio}
\begin{itemize}
\item {Grp. gram.:adj.}
\end{itemize}
\begin{itemize}
\item {Proveniência:(De \textunderscore corrente\textunderscore )}
\end{itemize}
Que corre facilmente.
Que é usual; geralmente admittido: \textunderscore linguagem correntia\textunderscore .
\section{Corréo}
\begin{itemize}
\item {Grp. gram.:m.}
\end{itemize}
\begin{itemize}
\item {Proveniência:(De \textunderscore com...\textunderscore  + \textunderscore réu\textunderscore )}
\end{itemize}
Aquelle, que é réo com outrem.
\section{Correpção}
\begin{itemize}
\item {Grp. gram.:f.}
\end{itemize}
\begin{itemize}
\item {Proveniência:(Lat. \textunderscore correptio\textunderscore )}
\end{itemize}
Acto de tornar breve uma sýllaba longa, na poética antiga.
\section{Correr}
\begin{itemize}
\item {Grp. gram.:v. i.}
\end{itemize}
\begin{itemize}
\item {Grp. gram.:V. t.}
\end{itemize}
\begin{itemize}
\item {Grp. gram.:V. p.}
\end{itemize}
\begin{itemize}
\item {Proveniência:(Lat. \textunderscore currere\textunderscore )}
\end{itemize}
Andar depressa.
Sêr transportado com velocidade.
Ir rapidamente.
Apressar-se.
Escoar-se, passar: \textunderscore o tempo corre\textunderscore .
Prolongar-se.
Dizer-se, constar: \textunderscore corre que o Govêrno está em crise\textunderscore .
Têr curso, circular, como moéda ou papel de crédito: \textunderscore o pataco já não corre\textunderscore .
Percorrer: \textunderscore correr a Europa\textunderscore .
Fazer passar ligeiramente.
Fazer andar.
Estender.
Perseguir na carreira: \textunderscore o cão correu a lebre\textunderscore .
Expulsar: \textunderscore corri-o de casa\textunderscore .
Estar sujeito a: \textunderscore correr perigo\textunderscore .
Andar em busca de.
Espalhar-se, dizer-se, circular. Cf. Rui Barbosa, \textunderscore Réplica\textunderscore , 159.
\section{Correria}
\begin{itemize}
\item {Grp. gram.:f.}
\end{itemize}
\begin{itemize}
\item {Utilização:Ant.}
\end{itemize}
\begin{itemize}
\item {Proveniência:(De \textunderscore correr\textunderscore )}
\end{itemize}
Acto de correr desordenadamente.
Assalto a campo inimigo.
Obrigação de fazer serviço postal, transmittindo a correspondência, de terra para terra. Cf. G. Henriques, \textunderscore Alenquer\textunderscore .
\section{Côrres}
\begin{itemize}
\item {Grp. gram.:m. pl.}
\end{itemize}
\begin{itemize}
\item {Utilização:Prov.}
\end{itemize}
\begin{itemize}
\item {Utilização:trasm.}
\end{itemize}
Medranças dos feijoeiros ou do outras plantas trepadeiras.
(Cp. \textunderscore corra\textunderscore )
\section{Correspondência}
\begin{itemize}
\item {Grp. gram.:f.}
\end{itemize}
\begin{itemize}
\item {Proveniência:(De \textunderscore corresponder\textunderscore )}
\end{itemize}
Acto de corresponder.
Troca de cartas, bilhetes ou telegrammas.
Conjunto de cartas, bilhetes ou telegrammas, que são expedidos ou recebidos.
Carta a um periódico.
Relações entre pessôas ausentes, que se correspondem pelo correio.
Correlação.
Communicação atmosphérica entre duas aberturas ou passagens.
\section{Correspondente}
\begin{itemize}
\item {Grp. gram.:adj.}
\end{itemize}
\begin{itemize}
\item {Grp. gram.:M.}
\end{itemize}
\begin{itemize}
\item {Proveniência:(De \textunderscore corresponder\textunderscore )}
\end{itemize}
Que corresponde.
Apropriado.
Symétrico.
Diz-se do sócio, que não é effectivo, de certas corporações literárias ou scientíficas.
Aquelle, que se carteia com alguém.
Aquelle, que escreve correspondências em periódicos.
Aquelle, que fornece habitualmente dinheiro a alguém, que reside ou estuda em terra estranha, dada a autorização dos legitimos superiores do forasteiro.
Negociante, que tem relações pecuniárias e commerciaes com outro.
\section{Correspondentemente}
\begin{itemize}
\item {Grp. gram.:adv.}
\end{itemize}
De modo correspondente.
\section{Corresponder}
\begin{itemize}
\item {Grp. gram.:v. i.}
\end{itemize}
\begin{itemize}
\item {Grp. gram.:V. t.}
\end{itemize}
\begin{itemize}
\item {Grp. gram.:V. p.}
\end{itemize}
\begin{itemize}
\item {Proveniência:(De \textunderscore com...\textunderscore  + \textunderscore responder\textunderscore )}
\end{itemize}
Pertencer.
Sêr próprio, adequado, symétrico: \textunderscore êste portão não corresponde á grandeza do edifício\textunderscore .
Retribuir equivalentemente.
Retribuir:«\textunderscore ó maravilhas do amor divino! que mal as ponderamos! que peor as correspondemos!\textunderscore »\textunderscore Luz e Calor\textunderscore , 341.
Estar em correlação.
Cartear-se.
\section{Corretã}
\begin{itemize}
\item {Grp. gram.:f.}
\end{itemize}
\begin{itemize}
\item {Proveniência:(De \textunderscore correr\textunderscore )}
\end{itemize}
O mesmo que \textunderscore roldana\textunderscore .
\section{Corretagem}
\begin{itemize}
\item {Grp. gram.:f.}
\end{itemize}
\begin{itemize}
\item {Proveniência:(Do rad. de \textunderscore corretor\textunderscore )}
\end{itemize}
Salário de corretor.
Agência.
\section{Corretan}
\begin{itemize}
\item {Grp. gram.:f.}
\end{itemize}
\begin{itemize}
\item {Proveniência:(De \textunderscore correr\textunderscore )}
\end{itemize}
O mesmo que \textunderscore roldana\textunderscore .
\section{Corretor}
\begin{itemize}
\item {Grp. gram.:m.}
\end{itemize}
\begin{itemize}
\item {Utilização:Deprec.}
\end{itemize}
Agente commercial, que serve de intermediário na compra e venda de mercadorias, papéis de crédito, etc.
Inculcador, agente.
Peça do moínho de vento, onde gira a roda.
Alcoviteiro.
(Relaciona-se com \textunderscore curador\textunderscore , do lat. \textunderscore curare\textunderscore )
\section{Corréu}
\begin{itemize}
\item {Grp. gram.:m.}
\end{itemize}
\begin{itemize}
\item {Proveniência:(De \textunderscore com...\textunderscore  + \textunderscore réu\textunderscore )}
\end{itemize}
Aquelle, que é réu com outrem.
\section{Corre-vai-di-lo}
\begin{itemize}
\item {Grp. gram.:m.}
\end{itemize}
\begin{itemize}
\item {Utilização:Prov.}
\end{itemize}
\begin{itemize}
\item {Utilização:trasm.}
\end{itemize}
Homem ou mulher que só cura de mexericos.
\section{Corrião}
\begin{itemize}
\item {Grp. gram.:m.}
\end{itemize}
O mesmo que \textunderscore borrelho\textunderscore .
\section{Corrião}
\begin{itemize}
\item {Grp. gram.:m.}
\end{itemize}
(V.correão)
\section{Corrica}
\begin{itemize}
\item {Grp. gram.:f.}
\end{itemize}
\begin{itemize}
\item {Utilização:Prov.}
\end{itemize}
\begin{itemize}
\item {Proveniência:(De \textunderscore corricar\textunderscore )}
\end{itemize}
Pequena roda, rodela.
\section{Corricão}
\begin{itemize}
\item {Grp. gram.:m.}
\end{itemize}
\begin{itemize}
\item {Proveniência:(Do rad. de \textunderscore correr\textunderscore )}
\end{itemize}
Acto de levantar caça, por meio de cães.
\section{Corrição}
\begin{itemize}
\item {Grp. gram.:m.}
\end{itemize}
O mesmo que \textunderscore borrelho\textunderscore .
\section{Corricar}
\begin{itemize}
\item {Grp. gram.:v. i.}
\end{itemize}
\begin{itemize}
\item {Utilização:Prov.}
\end{itemize}
\begin{itemize}
\item {Proveniência:(De \textunderscore correr\textunderscore )}
\end{itemize}
Andar (uma roda pequena), pelo chão.
Andar ligeiramente.
Correr a passo miúdo.
\section{Corrichar}
\begin{itemize}
\item {Grp. gram.:v. i.}
\end{itemize}
\begin{itemize}
\item {Utilização:Prov.}
\end{itemize}
\begin{itemize}
\item {Proveniência:(De \textunderscore correr\textunderscore )}
\end{itemize}
Andar apressadamente, para aqui e para ali.
\section{Corricho!}
\begin{itemize}
\item {Grp. gram.:interj.}
\end{itemize}
\begin{itemize}
\item {Utilização:Prov.}
\end{itemize}
\begin{itemize}
\item {Utilização:beir.}
\end{itemize}
O mesmo que \textunderscore querruxe!\textunderscore 
\section{Corrida}
\begin{itemize}
\item {Grp. gram.:f.}
\end{itemize}
\begin{itemize}
\item {Utilização:Gír.}
\end{itemize}
\begin{itemize}
\item {Utilização:Mús.}
\end{itemize}
\begin{itemize}
\item {Utilização:Des.}
\end{itemize}
\begin{itemize}
\item {Utilização:Prov.}
\end{itemize}
\begin{itemize}
\item {Proveniência:(De \textunderscore correr\textunderscore )}
\end{itemize}
Acto de correr.
Correria.
Caminho percorrido entre dois pontos conhecidos.
Toirada.
Espectáculo ou exercício com cavallos corredores.
Affluência inopinada ás casas bancárias, para levantamento de depósitos ou troca de valores.
O mês.
O mesmo que \textunderscore volata\textunderscore .
\textunderscore Peixe de corrida\textunderscore , peixe fresco. (Colhido em a Nazaré)
\section{Corrido}
\begin{itemize}
\item {Grp. gram.:adj.}
\end{itemize}
Vexado.
Prostituido, gasto.
\section{Corrido}
\begin{itemize}
\item {Grp. gram.:m.}
\end{itemize}
\begin{itemize}
\item {Utilização:Bras}
\end{itemize}
Espécie de cascalho.
\section{Corriento}
\begin{itemize}
\item {Grp. gram.:adj.}
\end{itemize}
\begin{itemize}
\item {Proveniência:(De \textunderscore correr\textunderscore )}
\end{itemize}
Escorregadio; macio. Cf. Barros, \textunderscore Déc.\textunderscore  III, 7.
\section{Corrigenda}
\begin{itemize}
\item {Grp. gram.:f.}
\end{itemize}
\begin{itemize}
\item {Proveniência:(Lat. \textunderscore corrigenda\textunderscore )}
\end{itemize}
Erros, que se devem corrigir numa obra literária ou scientífica; erratas.
\section{Corrigibilidade}
\begin{itemize}
\item {Grp. gram.:f.}
\end{itemize}
Qualidade daquillo ou de quem é corrigível.
\section{Corrigíola}
\begin{itemize}
\item {Grp. gram.:f.}
\end{itemize}
Gênero de plantas caryophylláceas.
(B. lat. \textunderscore corrigiola\textunderscore )
\section{Corrigir}
\begin{itemize}
\item {Grp. gram.:v. t.}
\end{itemize}
\begin{itemize}
\item {Proveniência:(Lat. \textunderscore corrigere\textunderscore )}
\end{itemize}
Emendar: \textunderscore corrigir provas typográphicas\textunderscore .
Melhorar.
Censurar.
Reprimir: \textunderscore corrigir um rapaz\textunderscore .
Modificar, temperar: \textunderscore corrigir a acidez de um líquido\textunderscore .
\section{Corrigível}
\begin{itemize}
\item {Grp. gram.:adj.}
\end{itemize}
Que se póde corrigir.
\section{Corrijola}
\begin{itemize}
\item {Grp. gram.:f.}
\end{itemize}
O mesmo que \textunderscore corrigíola\textunderscore .
\section{Corrilheiro}
\begin{itemize}
\item {Grp. gram.:m.}
\end{itemize}
Frequentador ou promotor de corrilhos.
\section{Corrilho}
\begin{itemize}
\item {Grp. gram.:m.}
\end{itemize}
\begin{itemize}
\item {Proveniência:(Do lat. \textunderscore curriculum\textunderscore )}
\end{itemize}
Conciliábulo.
Reunião facciosa.
Mexerico.
\section{Corrilhó}
\begin{itemize}
\item {Grp. gram.:m.}
\end{itemize}
\begin{itemize}
\item {Utilização:Prov.}
\end{itemize}
Planta herbácea. (Colhido em Turquel)
\section{Corrilório}
\begin{itemize}
\item {Grp. gram.:m.}
\end{itemize}
\begin{itemize}
\item {Utilização:Prov.}
\end{itemize}
\begin{itemize}
\item {Utilização:beir.}
\end{itemize}
Multidão de gente, que vai correndo por muito tempo.
Tropel de gente, que vai desfilando.
(Cp. \textunderscore correr\textunderscore )
\section{Corrimaça}
\begin{itemize}
\item {Grp. gram.:f.}
\end{itemize}
\begin{itemize}
\item {Utilização:Pop.}
\end{itemize}
\begin{itemize}
\item {Proveniência:(Do rad. de \textunderscore correr\textunderscore )}
\end{itemize}
Assuada.
Perseguição com vaias.
Corrida.
\section{Corrimão}
\begin{itemize}
\item {Grp. gram.:m.}
\end{itemize}
\begin{itemize}
\item {Proveniência:(De \textunderscore correr\textunderscore  + \textunderscore mão\textunderscore )}
\end{itemize}
Peça longitudinal, ao lado de uma escada, para se firmar a mão de quem sobe ou desce.
Barrote, que, nas embarcações, sustenta os balaústres e serve de encôsto ou parapeito.
\section{Corrimboque}
\begin{itemize}
\item {Grp. gram.:m.}
\end{itemize}
\begin{itemize}
\item {Utilização:Bras}
\end{itemize}
O mesmo que \textunderscore cornimboque\textunderscore .
\section{Corrimento}
\begin{itemize}
\item {Grp. gram.:m.}
\end{itemize}
\begin{itemize}
\item {Grp. gram.:M. pl.}
\end{itemize}
\begin{itemize}
\item {Utilização:Prov.}
\end{itemize}
\begin{itemize}
\item {Utilização:alg.}
\end{itemize}
\begin{itemize}
\item {Proveniência:(De \textunderscore correr\textunderscore )}
\end{itemize}
Acto de correr.
Humor, que escorre de alguma parte do corpo.
Corrimaça; vexame.
Coisa própria do tempo.
\section{Corriol}
\begin{itemize}
\item {Grp. gram.:m.}
\end{itemize}
\begin{itemize}
\item {Utilização:Prov.}
\end{itemize}
\begin{itemize}
\item {Utilização:alent.}
\end{itemize}
Fio resistente, formado de finíssimas tiras de coiro, cortadas e tecidas em fresco.
(Cp. \textunderscore correia\textunderscore ^1)
\section{Corriola}
\begin{itemize}
\item {Grp. gram.:f.}
\end{itemize}
Planta convolvulácea.
Espécie de jôgo com uma fita dobrada.
Lôgro.
(Cp. \textunderscore corrigíola\textunderscore )
\section{Corripo}
\begin{itemize}
\item {Grp. gram.:m.}
\end{itemize}
Systema de pesca, em que o anzol, em vez de isca, é munido de um trapo branco ou chapa de metal brilhante, para attrahir o peixe.
\section{Corriqueirice}
\begin{itemize}
\item {Grp. gram.:f.}
\end{itemize}
\begin{itemize}
\item {Utilização:Fam.}
\end{itemize}
Qualidade de corriqueiro.
Coisa trivial.
\section{Corriqueiro}
\begin{itemize}
\item {Grp. gram.:adj.}
\end{itemize}
\begin{itemize}
\item {Utilização:Prov.}
\end{itemize}
\begin{itemize}
\item {Proveniência:(De \textunderscore corricar\textunderscore )}
\end{itemize}
Habitual.
Trivial.
Que leva e traz novidades. (Colhido em Lanhoso)
\section{Corritana}
\begin{itemize}
\item {Grp. gram.:f.}
\end{itemize}
(?)«\textunderscore chá de flôr de fugafina, sustância de corritana.\textunderscore »Castilho, \textunderscore Méd. á Fôrça\textunderscore , 186.
\section{Corrixo}
\begin{itemize}
\item {Grp. gram.:m.}
\end{itemize}
\begin{itemize}
\item {Utilização:Bras}
\end{itemize}
Pequeno pássaro negro, que imita o canto de todas as aves.
\section{Corro}
\begin{itemize}
\item {fónica:cô}
\end{itemize}
\begin{itemize}
\item {Grp. gram.:m.}
\end{itemize}
\begin{itemize}
\item {Utilização:Des.}
\end{itemize}
Circo para espectáculos.
Assembleia.
Roda.
Bando, multidão:«\textunderscore ali acabárão ás mãos daquelle corro de pagãos.\textunderscore »Filinto, \textunderscore D. Man.\textunderscore , II, 142.
(Cp. \textunderscore curro\textunderscore )
\section{Corrôa}
\begin{itemize}
\item {Grp. gram.:f.}
\end{itemize}
\begin{itemize}
\item {Utilização:Ant.}
\end{itemize}
O mesmo que \textunderscore corôa\textunderscore . Cf. \textunderscore Alvará\textunderscore  de Aff. V, in \textunderscore Rev. Lus.\textunderscore , XV, 121.
\section{Corroboração}
\begin{itemize}
\item {Grp. gram.:f.}
\end{itemize}
Acto de corroborar.
\section{Corroborante}
\begin{itemize}
\item {Grp. gram.:adj.}
\end{itemize}
\begin{itemize}
\item {Proveniência:(Lat. \textunderscore corroborans\textunderscore )}
\end{itemize}
Que corrobora.
\section{Corroborar}
\begin{itemize}
\item {Grp. gram.:v. t.}
\end{itemize}
\begin{itemize}
\item {Proveniência:(Lat. \textunderscore corroborare\textunderscore )}
\end{itemize}
Dar fôrça a.
Fortalecer.
Comprovar: \textunderscore os factos corroboraram a previsão\textunderscore .
\section{Corroborativo}
\begin{itemize}
\item {Grp. gram.:adj.}
\end{itemize}
Próprio para corroborar.
\section{Corrodente}
\begin{itemize}
\item {Grp. gram.:adj.}
\end{itemize}
\begin{itemize}
\item {Proveniência:(Lat. \textunderscore corrodens\textunderscore )}
\end{itemize}
Que corrói.
\section{Corroer}
\begin{itemize}
\item {Grp. gram.:v. t.}
\end{itemize}
\begin{itemize}
\item {Proveniência:(Lat. \textunderscore corrodere\textunderscore )}
\end{itemize}
Roer a pouco e pouco.
Carcomer.
Destruir: \textunderscore o tempo corroeu a inscripção\textunderscore .
\section{Corrompedor}
\begin{itemize}
\item {Grp. gram.:m.  e  adj.}
\end{itemize}
\begin{itemize}
\item {Proveniência:(De \textunderscore corromper\textunderscore )}
\end{itemize}
O que corrompe.
Corruptor.
\section{Corromper}
\begin{itemize}
\item {Grp. gram.:v. t.}
\end{itemize}
\begin{itemize}
\item {Proveniência:(Lat. \textunderscore corrumpere\textunderscore )}
\end{itemize}
Tornar podre.
Estragar.
Infectar.
Desnaturar.
Perverter: \textunderscore corromper a mocidade\textunderscore .
Peitar: \textunderscore corromper os jurados\textunderscore .
\section{Corrompimento}
\begin{itemize}
\item {Grp. gram.:m.}
\end{itemize}
(V.corrupção)
\section{Corrosão}
\begin{itemize}
\item {Grp. gram.:f.}
\end{itemize}
\begin{itemize}
\item {Proveniência:(Do rad. do lat. \textunderscore corrosus\textunderscore )}
\end{itemize}
Acto ou effeito de corroer.
\section{Corrosibilidade}
\begin{itemize}
\item {Grp. gram.:f.}
\end{itemize}
Qualidade do que é corrosível.
\section{Corrosível}
\begin{itemize}
\item {Grp. gram.:adj.}
\end{itemize}
\begin{itemize}
\item {Proveniência:(Do lat. \textunderscore corrosus\textunderscore )}
\end{itemize}
Susceptível de sêr corroído.
\section{Corrosividade}
\begin{itemize}
\item {Grp. gram.:f.}
\end{itemize}
Qualidade daquillo que é corrosivo.
\section{Corrosivo}
\begin{itemize}
\item {Grp. gram.:adj.}
\end{itemize}
\begin{itemize}
\item {Grp. gram.:M.}
\end{itemize}
\begin{itemize}
\item {Proveniência:(Lat. \textunderscore corrosivus\textunderscore )}
\end{itemize}
Que corrói.
Que destrói, que desorganiza.
Substância, que corrói.
\section{Corrução}
\begin{itemize}
\item {Grp. gram.:f.}
\end{itemize}
\begin{itemize}
\item {Utilização:Bras}
\end{itemize}
Espécie de diarreia, maculo.
(Cp. cast. ant. \textunderscore corrupción\textunderscore )
\section{Corruche!}
\begin{itemize}
\item {Grp. gram.:interj.}
\end{itemize}
(V.querruxe!)
\section{Corruda}
\begin{itemize}
\item {Grp. gram.:f.}
\end{itemize}
\begin{itemize}
\item {Proveniência:(Lat. \textunderscore corruda\textunderscore )}
\end{itemize}
O mesmo que \textunderscore espargo\textunderscore .
\section{Corrugada}
\begin{itemize}
\item {Grp. gram.:adj. f.}
\end{itemize}
\begin{itemize}
\item {Utilização:Bot.}
\end{itemize}
\begin{itemize}
\item {Proveniência:(De \textunderscore corrugar\textunderscore )}
\end{itemize}
Diz-se da prefloração irregular, em que as pétalas parecem amachucadas, como na flôr da roman, nas papoilas, etc.
\section{Corrugar}
\begin{itemize}
\item {Grp. gram.:v. t.}
\end{itemize}
\begin{itemize}
\item {Utilização:Ant.}
\end{itemize}
O mesmo que \textunderscore enrugar\textunderscore .
\section{Corrume}
\begin{itemize}
\item {Grp. gram.:m.}
\end{itemize}
\begin{itemize}
\item {Utilização:Pop.}
\end{itemize}
\begin{itemize}
\item {Proveniência:(Do rad. de \textunderscore correr\textunderscore )}
\end{itemize}
Entalhe, em que uma peça se ajusta com outra.
Rumo, caminho: \textunderscore tomou mau corrume\textunderscore .
Procedimento.
\section{Corrupção}
\begin{itemize}
\item {Grp. gram.:f.}
\end{itemize}
\begin{itemize}
\item {Proveniência:(Lat. \textunderscore corruptio\textunderscore )}
\end{itemize}
Acto ou effeito de corromper.
Devassidão.
Desmoralização.
\section{Corrupia}
\begin{itemize}
\item {Grp. gram.:f.}
\end{itemize}
\begin{itemize}
\item {Utilização:Prov.}
\end{itemize}
\begin{itemize}
\item {Utilização:trasm.}
\end{itemize}
\begin{itemize}
\item {Utilização:minh.}
\end{itemize}
Criança traquinas.
(Cp. \textunderscore corrupio\textunderscore )
\section{Corrupião}
\begin{itemize}
\item {Grp. gram.:m.}
\end{itemize}
Ave canora do Brasil, facilmente domesticável.
\section{Corrupio}
\begin{itemize}
\item {Grp. gram.:m.}
\end{itemize}
\begin{itemize}
\item {Utilização:Fam.}
\end{itemize}
\begin{itemize}
\item {Utilização:Bras. do N}
\end{itemize}
\begin{itemize}
\item {Utilização:Prov.}
\end{itemize}
\begin{itemize}
\item {Utilização:alent.}
\end{itemize}
\begin{itemize}
\item {Proveniência:(Do rad. de \textunderscore correr\textunderscore )}
\end{itemize}
Nome de vários jogos infantis.
Roda-viva, afan: \textunderscore andar num corrupio\textunderscore .
Espécie de catavento, de pennas ou papel, para crianças.
Variedade de uva.
\section{Corrupixel}
\begin{itemize}
\item {Grp. gram.:m.}
\end{itemize}
\begin{itemize}
\item {Utilização:Bras}
\end{itemize}
Vara longa, com um saco na extremidade, para apanhar frutas.
\section{Corruptamente}
\begin{itemize}
\item {Grp. gram.:adv.}
\end{itemize}
De modo corrupto.
Por meio de corrupção.
\section{Corruptela}
\begin{itemize}
\item {Grp. gram.:f.}
\end{itemize}
\begin{itemize}
\item {Proveniência:(Lat. \textunderscore corruptela\textunderscore )}
\end{itemize}
Corrupção.
Aquillo que é capaz de corromper.
Abuso.
Modo errado de falar ou de escrever uma palavra.
\section{Corruptibilidade}
\begin{itemize}
\item {Grp. gram.:f.}
\end{itemize}
\begin{itemize}
\item {Proveniência:(Lat. \textunderscore corruptibilitas\textunderscore )}
\end{itemize}
Qualidade daquillo ou de quem é corruptível.
\section{Corruptível}
\begin{itemize}
\item {Grp. gram.:adj.}
\end{itemize}
\begin{itemize}
\item {Proveniência:(Lat. \textunderscore corruptibilis\textunderscore )}
\end{itemize}
Que é susceptível de corrupção; venal.
\section{Corruptivo}
\begin{itemize}
\item {Grp. gram.:adj.}
\end{itemize}
(V.corruptível)
\section{Corrupto}
\begin{itemize}
\item {Grp. gram.:adj.}
\end{itemize}
\begin{itemize}
\item {Proveniência:(Lat. \textunderscore corruptus\textunderscore )}
\end{itemize}
Corrompido.
Desmoralizado.
Devasso.
Errado, (falando-se de línguagem ou de palavras, graphicamente consideradas).
\section{Corruptor}
\begin{itemize}
\item {Grp. gram.:m.  e  adj.}
\end{itemize}
\begin{itemize}
\item {Proveniência:(Lat. \textunderscore corruptor\textunderscore )}
\end{itemize}
O que corrompe.
\section{Corrutamente}
\begin{itemize}
\item {Grp. gram.:adv.}
\end{itemize}
De modo corruto.
Por meio de corrupção.
\section{Corrutela}
\begin{itemize}
\item {Grp. gram.:f.}
\end{itemize}
\begin{itemize}
\item {Proveniência:(Lat. \textunderscore corruptela\textunderscore )}
\end{itemize}
Corrupção.
Aquillo que é capaz de corromper.
Abuso.
Modo errado de falar ou de escrever uma palavra.
\section{Corrutibilidade}
\begin{itemize}
\item {Grp. gram.:f.}
\end{itemize}
\begin{itemize}
\item {Proveniência:(Lat. \textunderscore corruptibilitas\textunderscore )}
\end{itemize}
Qualidade daquillo ou de quem é corrutível.
\section{Corrutivel}
\begin{itemize}
\item {Grp. gram.:adj.}
\end{itemize}
\begin{itemize}
\item {Proveniência:(Lat. \textunderscore corruptibilis\textunderscore )}
\end{itemize}
Que é susceptível de corrupção; venal.
\section{Corrutivo}
\begin{itemize}
\item {Grp. gram.:adj.}
\end{itemize}
(V.corrutivel)
\section{Corruto}
\begin{itemize}
\item {Grp. gram.:adj.}
\end{itemize}
\begin{itemize}
\item {Proveniência:(Lat. \textunderscore corruptus\textunderscore )}
\end{itemize}
Corrompido.
Desmoralizado.
Devasso.
Errado, (falando-se de línguagem ou de palavras, graphicamente consideradas).
\section{Corrutor}
\begin{itemize}
\item {Grp. gram.:m.  e  adj.}
\end{itemize}
\begin{itemize}
\item {Proveniência:(Lat. \textunderscore corruptor\textunderscore )}
\end{itemize}
O que corrompe.
\section{Corsa}
\begin{itemize}
\item {fónica:côr}
\end{itemize}
\begin{itemize}
\item {Grp. gram.:f.}
\end{itemize}
Espécie de vehículo, na ilha da Madeira, puxado por gente, e em que se transportam pessôas.
(Cp. \textunderscore corso\textunderscore ^1)
\section{Corsaco}
\begin{itemize}
\item {Grp. gram.:m.}
\end{itemize}
Espécie de cão asiático.
\section{Corsão}
\begin{itemize}
\item {Grp. gram.:m.}
\end{itemize}
Corsa grande.
\section{Corsário}
\begin{itemize}
\item {Grp. gram.:m.}
\end{itemize}
\begin{itemize}
\item {Utilização:Açor}
\end{itemize}
\begin{itemize}
\item {Grp. gram.:Adj.}
\end{itemize}
Navio de corso.
Homem, que anda a corso.
Pirata.
O mesmo que \textunderscore patife\textunderscore ^1.
Relativo a corso.
\section{Corsear}
\begin{itemize}
\item {Grp. gram.:v. i.}
\end{itemize}
\begin{itemize}
\item {Proveniência:(De \textunderscore corso\textunderscore ^1)}
\end{itemize}
Andar a corso.
\section{Corselete}
\begin{itemize}
\item {fónica:lê}
\end{itemize}
\begin{itemize}
\item {Grp. gram.:m.}
\end{itemize}
\begin{itemize}
\item {Proveniência:(Fr. \textunderscore corselet\textunderscore )}
\end{itemize}
Antiga e ligeira armadura para o peito.
Corpete.
\section{Córsico}
\begin{itemize}
\item {Grp. gram.:adj.}
\end{itemize}
\begin{itemize}
\item {Grp. gram.:M.}
\end{itemize}
\begin{itemize}
\item {Proveniência:(Lat. \textunderscore corsicus\textunderscore )}
\end{itemize}
Relativo á Córsega.
Habitante da Córsega.
\section{Corsins}
\begin{itemize}
\item {Grp. gram.:m. pl.}
\end{itemize}
\begin{itemize}
\item {Proveniência:(It. corsini, provavelmente de \textunderscore Caorsa\textunderscore  n. p. de uma cidade de Piemonte)}
\end{itemize}
Mercadores italianos, que antigamente tinham em Lisbôa commércio de prata. Cf. Fern. Lopes, \textunderscore Chrón. de D. Fern.\textunderscore , c. I.
\section{Corso}
\begin{itemize}
\item {fónica:côr}
\end{itemize}
\begin{itemize}
\item {Grp. gram.:m.}
\end{itemize}
\begin{itemize}
\item {Proveniência:(Do lat. \textunderscore cursus\textunderscore )}
\end{itemize}
Excursão de navios, para perseguir embarcações mercantes de uma nação inimiga.
Pirataria.
Vida nômada de povos, que vivem do que roubam.
\section{Corso}
\begin{itemize}
\item {fónica:côroucór}
\end{itemize}
\begin{itemize}
\item {Grp. gram.:m.}
\end{itemize}
\begin{itemize}
\item {Grp. gram.:Adj.}
\end{itemize}
\begin{itemize}
\item {Proveniência:(Lat. \textunderscore corsus\textunderscore )}
\end{itemize}
Habitante da Córsega.
Relativo á Córsega.
\section{Corso}
\begin{itemize}
\item {fónica:côr}
\end{itemize}
\begin{itemize}
\item {Grp. gram.:m.}
\end{itemize}
Cardume de sardinha.
(Corr. de \textunderscore côrcho\textunderscore )
\section{Corsolete}
\begin{itemize}
\item {fónica:lê}
\end{itemize}
\begin{itemize}
\item {Grp. gram.:m.}
\end{itemize}
\begin{itemize}
\item {Utilização:Ant.}
\end{itemize}
O mesmo que \textunderscore corselete\textunderscore .
\section{Corta}
\begin{itemize}
\item {Grp. gram.:f.}
\end{itemize}
Acto de cortar: \textunderscore anda na corta dos pinheiros\textunderscore .
\section{Corta-água}
\begin{itemize}
\item {Grp. gram.:m.}
\end{itemize}
Ave aquática do norte do Brasil.
\section{Cortação}
\begin{itemize}
\item {Grp. gram.:f.}
\end{itemize}
\begin{itemize}
\item {Utilização:Prov.}
\end{itemize}
\begin{itemize}
\item {Utilização:minh.}
\end{itemize}
\begin{itemize}
\item {Proveniência:(De \textunderscore cortar\textunderscore )}
\end{itemize}
Grande mágoa; afflicção.
\section{Corta-chefe}
\begin{itemize}
\item {Grp. gram.:m.}
\end{itemize}
\begin{itemize}
\item {Utilização:Carp.}
\end{itemize}
Ferramenta para alisar curvas.
\section{Cortadeira}
\begin{itemize}
\item {Grp. gram.:f.}
\end{itemize}
\begin{itemize}
\item {Proveniência:(De \textunderscore cortar\textunderscore )}
\end{itemize}
Utensílio, com que o pasteleiro corta as massas.
Cortilha.
Nome de outros utensílios, que cortam.
\section{Cortadela}
\begin{itemize}
\item {Grp. gram.:f.}
\end{itemize}
O mesmo que \textunderscore cortadura\textunderscore .
\section{Cortadilhos}
\begin{itemize}
\item {Grp. gram.:m. pl.}
\end{itemize}
\begin{itemize}
\item {Utilização:Prov.}
\end{itemize}
\begin{itemize}
\item {Utilização:alent.}
\end{itemize}
\begin{itemize}
\item {Proveniência:(De \textunderscore cortado\textunderscore )}
\end{itemize}
Pedacinhos de chumbo, em que se dividiu um fragmento anguloso ou cylíndrico do mesmo metal, e que servem como chumbo de caça.
Bala partida em 4, 8 ou mais partes, segundo a importância da caça a que se destina.
Zagalotes.
\section{Cortado}
\begin{itemize}
\item {Grp. gram.:adj.}
\end{itemize}
Que se cortou ou se separou de um todo: \textunderscore vides cortadas\textunderscore .
\section{Cortadoiro}
\begin{itemize}
\item {Grp. gram.:m.}
\end{itemize}
\begin{itemize}
\item {Utilização:Prov.}
\end{itemize}
\begin{itemize}
\item {Utilização:alg.}
\end{itemize}
\begin{itemize}
\item {Proveniência:(De \textunderscore cortar\textunderscore )}
\end{itemize}
Depressão de terreno, entre montes.
\section{Cortador}
\begin{itemize}
\item {Grp. gram.:m.  e  adj.}
\end{itemize}
\begin{itemize}
\item {Proveniência:(De \textunderscore cortar\textunderscore )}
\end{itemize}
O que corta.
O que corta carne nos açougues.
Nome de vários instrumentos que cortam.
\section{Cortadouro}
\begin{itemize}
\item {Grp. gram.:m.}
\end{itemize}
\begin{itemize}
\item {Utilização:Prov.}
\end{itemize}
\begin{itemize}
\item {Utilização:alg.}
\end{itemize}
\begin{itemize}
\item {Proveniência:(De \textunderscore cortar\textunderscore )}
\end{itemize}
Depressão de terreno, entre montes.
\section{Cortadura}
\begin{itemize}
\item {Grp. gram.:f.}
\end{itemize}
\begin{itemize}
\item {Proveniência:(De \textunderscore cortar\textunderscore )}
\end{itemize}
Acto ou effeito de cortar.
Córte.
Sulco artificial, por onde se escôam águas.
Abertura entre montes.
\section{Corta-forragem}
\begin{itemize}
\item {Grp. gram.:f.}
\end{itemize}
Utensílio para cortar ou segar forragem.
\section{Corta-frio}
\begin{itemize}
\item {Grp. gram.:m.}
\end{itemize}
Cunha de aço, com que os ferreiros cortam uma barra de ferro frio ou a golpeiam, para melhor a partirem.
\section{Cortagem}
\begin{itemize}
\item {Grp. gram.:f.}
\end{itemize}
\begin{itemize}
\item {Proveniência:(De \textunderscore cortar\textunderscore )}
\end{itemize}
Acto de cortar carne no açougue.
\section{Corta-jaca}
\begin{itemize}
\item {Grp. gram.:f.}
\end{itemize}
\begin{itemize}
\item {Utilização:Bras}
\end{itemize}
Espécie de dança sapateada.
\section{Corta-línguas}
\begin{itemize}
\item {Grp. gram.:m.}
\end{itemize}
\begin{itemize}
\item {Utilização:T. da Bairrada}
\end{itemize}
Intérprete de línguas.
\section{Cortamão}
\begin{itemize}
\item {Grp. gram.:m.}
\end{itemize}
O mesmo que \textunderscore esquadro\textunderscore .
\section{Corta-mar}
\begin{itemize}
\item {Grp. gram.:m.}
\end{itemize}
O mesmo que \textunderscore quebra-mar\textunderscore .
Prolongamento angular dos pegões das pontes, para fortalecer a construcção.
\section{Cortamento}
\begin{itemize}
\item {Grp. gram.:m.}
\end{itemize}
Acto ou effeito de cortar.
\section{Corta-milhos}
\begin{itemize}
\item {Grp. gram.:m.}
\end{itemize}
Utensílio, o mesmo que \textunderscore corta-ferragem\textunderscore .
\section{Cortanheiro}
\begin{itemize}
\item {Grp. gram.:m.}
\end{itemize}
\begin{itemize}
\item {Utilização:Ant.}
\end{itemize}
\begin{itemize}
\item {Proveniência:(Do v. hypoth. \textunderscore cortanhar\textunderscore , freq. de \textunderscore cortar\textunderscore )}
\end{itemize}
Aquelle que, em certas festas religiosas, era encarregado de cortar e repartir entre devotos uma vaca ou bezerro. Cp. \textunderscore Alvará\textunderscore  de D. Sebast., in. \textunderscore Rev. Lus.\textunderscore , XV, 141 e 242.
\section{Cortante}
\begin{itemize}
\item {Grp. gram.:adj.}
\end{itemize}
Que corta: \textunderscore instrumento cortante\textunderscore .
\section{Corta-palha}
\begin{itemize}
\item {Grp. gram.:m.}
\end{itemize}
Serrote fixo, em que se corta palha que se dá ao gado.
\section{Corta-papel}
\begin{itemize}
\item {Grp. gram.:m.}
\end{itemize}
Utensílio de madeira ou de ôsso, ou de marfim, ou de outra substância, em fórma de faca, e próprio para cortar papel dobrado, ou separar as fôlhas de uma publicação, cortando-lhes a ligação das margens.
\section{Corta-pau}
\begin{itemize}
\item {Grp. gram.:m.}
\end{itemize}
Ave do Brasil. O mesmo que \textunderscore pêto\textunderscore ^1.
\section{Cortar}
\begin{itemize}
\item {Grp. gram.:v. t.}
\end{itemize}
\begin{itemize}
\item {Grp. gram.:V. i.}
\end{itemize}
\begin{itemize}
\item {Utilização:Marn.}
\end{itemize}
\begin{itemize}
\item {Grp. gram.:V. p.}
\end{itemize}
\begin{itemize}
\item {Utilização:Gír.}
\end{itemize}
\begin{itemize}
\item {Proveniência:(Lat. \textunderscore cortare\textunderscore )}
\end{itemize}
Dividir com instrumento de gume.
Separar de um todo, a golpes de instrumento de gume (outra parte do mesmo todo): \textunderscore cortar uma arvore\textunderscore ; \textunderscore cortar um ramo\textunderscore ; \textunderscore cortar um braço\textunderscore . (Em alguns casos, a serra substitue êsse instrumento)
Talhar (fato).
Interceptar: \textunderscore cortar a água da rega\textunderscore .
Atormentar: \textunderscore desgraças que cortam o coração\textunderscore .
Talhar ou dividir em duas ou mais partes (um baralho de cartas).
Intercalar.
Fender.
Obstruir.
\textunderscore Cortar a palavra\textunderscore , interromper, impedir que outrem continue a falar.
\textunderscore Cortar as asas a alguém\textunderscore , impedir-lhe a acção.
Dar golpe.
Fazer eliminação ou deminuição: \textunderscore cortar nas despesas\textunderscore .
Fazer caminho: \textunderscore cortou á direita\textunderscore .
Gretar.
\textunderscore Cortar direito\textunderscore , proceder rectamente.
\textunderscore Cortar na casaca\textunderscore , dizer mal, murmurar.
\textunderscore Cortar largo\textunderscore , dissipar.
Roubar alguma coisa.
\section{Corta-raízes}
\begin{itemize}
\item {Grp. gram.:m.}
\end{itemize}
\begin{itemize}
\item {Utilização:Agr.}
\end{itemize}
Utensílio, para separar das raízes os vegetaes, destinados á ração dos animaes. Cf. Baganha, \textunderscore Hyg. Pec.\textunderscore , 43.
\section{Corta-trapo}
\begin{itemize}
\item {Grp. gram.:m.}
\end{itemize}
Apparelho, com que se cortam trapos, nas fábricas de papel. Cf. \textunderscore Inquér. Indust.\textunderscore , P. II, l. 3.^o, 223.
\section{Corta-vides}
\begin{itemize}
\item {Grp. gram.:m.}
\end{itemize}
\begin{itemize}
\item {Utilização:Agr.}
\end{itemize}
Apparelho, com que se cortam vides em pequenos troços, que sirvam de adubos para as terras.
\section{Córte}
\begin{itemize}
\item {Grp. gram.:m.}
\end{itemize}
\begin{itemize}
\item {Utilização:Gír.}
\end{itemize}
\begin{itemize}
\item {Proveniência:(De \textunderscore cortar\textunderscore )}
\end{itemize}
Acto ou effeito de cortar.
Incisão.
Gume.
Plano de uma construcção.
Cada uma das faces da aduela de um arco de edifício.
Modo de talhar fato.
Peça de pano, sufficiente para um objecto de vestuário: \textunderscore um córte de calças\textunderscore .
Deminuição.
Suppressão: \textunderscore naquella Repartição, houve córte de gratificações\textunderscore .
Roubo.
\section{Córte}
\begin{itemize}
\item {Grp. gram.:f.}
\end{itemize}
\begin{itemize}
\item {Utilização:Prov.}
\end{itemize}
\begin{itemize}
\item {Utilização:Ant.}
\end{itemize}
\begin{itemize}
\item {Proveniência:(Lat. \textunderscore cors, cortis\textunderscore )}
\end{itemize}
Malhada, curral.
Lugar, em que se criam animaes domésticos.
Certa extensão de terreno lavradio.
\section{Côrte}
\begin{itemize}
\item {Grp. gram.:f.}
\end{itemize}
\begin{itemize}
\item {Utilização:Des.}
\end{itemize}
\begin{itemize}
\item {Grp. gram.:Pl.}
\end{itemize}
Residência de um Soberano.
Gente, que rodeia habitualmente o Soberano.
Povoação, em que êste reside.
O Govêrno de um país, em relação aos de outros países: \textunderscore a côrte de Madrid dirigiu uma reclamação á Itália\textunderscore .
Pessoas, que rodeiam habitualmente outra, lisonjeando-a e procurando agradar-lhe.
Galanteio: \textunderscore fazer côrte\textunderscore .
Tribunal.
Parlamento.
Edifício, onde está o parlamento: \textunderscore o Ministério apresentou-se hoje nas côrtes\textunderscore .
(Da mesma or. que \textunderscore córte\textunderscore ^2)
\section{Coribante}
\begin{itemize}
\item {Grp. gram.:m.}
\end{itemize}
\begin{itemize}
\item {Proveniência:(Lat. \textunderscore corybantes\textunderscore )}
\end{itemize}
Nome dos sacerdotes da deusa Cibele, notáveis por certas devoções violentas.
\section{Coribânticas}
\begin{itemize}
\item {Grp. gram.:f. pl.}
\end{itemize}
\begin{itemize}
\item {Proveniência:(De \textunderscore coribântico\textunderscore )}
\end{itemize}
Festas de Cibele, celebradas pelos coribantes.
\section{Coribântico}
\begin{itemize}
\item {Grp. gram.:adj.}
\end{itemize}
Relativo aos coribantes.
\section{Corício}
\begin{itemize}
\item {Grp. gram.:m.}
\end{itemize}
Indivíduo natural de Córico, cidade Cilícia:«\textunderscore conheci um corício em annos já maduros\textunderscore ». Castilho, \textunderscore Geórg.\textunderscore 
\section{Coridálida}
\begin{itemize}
\item {Grp. gram.:f.}
\end{itemize}
(V.corídalo)
\section{Coridalina}
\begin{itemize}
\item {Grp. gram.:f.}
\end{itemize}
Alcaloide, extrahido do corídalo.
\section{Corídalo}
\begin{itemize}
\item {Grp. gram.:m.}
\end{itemize}
\begin{itemize}
\item {Proveniência:(Gr. \textunderscore korudalos\textunderscore )}
\end{itemize}
Gênero de plantas fumariáceas.
\section{Corifa}
\begin{itemize}
\item {Grp. gram.:f.}
\end{itemize}
\begin{itemize}
\item {Proveniência:(Do gr. \textunderscore koruphe\textunderscore , vértice)}
\end{itemize}
Gênero de palmeiras.
\section{Corifeu}
\begin{itemize}
\item {Grp. gram.:m.}
\end{itemize}
\begin{itemize}
\item {Proveniência:(Gr. \textunderscore koryphaios\textunderscore )}
\end{itemize}
Director de córos no antigo theatro.
Pessôa, que occupa o primeiro lugar numa classe ou numa profissão; caudilho, chefe.
\section{Corilina}
\begin{itemize}
\item {Grp. gram.:f.}
\end{itemize}
\begin{itemize}
\item {Utilização:Pharm.}
\end{itemize}
\begin{itemize}
\item {Proveniência:(De \textunderscore córilo\textunderscore )}
\end{itemize}
Medicamento, derivado do mentol.
\section{Córilo}
\begin{itemize}
\item {Grp. gram.:m.}
\end{itemize}
\begin{itemize}
\item {Proveniência:(Do gr. \textunderscore korus\textunderscore )}
\end{itemize}
Gênero de plantas cupulíferas.
\section{Corimbíferas}
\begin{itemize}
\item {Grp. gram.:f. pl.}
\end{itemize}
\begin{itemize}
\item {Proveniência:(De \textunderscore corymbífero\textunderscore )}
\end{itemize}
Grande divisão de plantas, da fam. das compostas, caracterizadas por terem flôres em corymbo.
\section{Corimbífero}
\begin{itemize}
\item {Grp. gram.:adj.}
\end{itemize}
\begin{itemize}
\item {Utilização:Bot.}
\end{itemize}
\begin{itemize}
\item {Proveniência:(Do lat. \textunderscore corymbus\textunderscore  + \textunderscore ferre\textunderscore )}
\end{itemize}
Que tem flôres em corimbo.
\section{Corimbo}
\begin{itemize}
\item {Grp. gram.:m.}
\end{itemize}
\begin{itemize}
\item {Utilização:Bot.}
\end{itemize}
\begin{itemize}
\item {Proveniência:(Lat. \textunderscore corymbus\textunderscore )}
\end{itemize}
Conjunto de flôres, que, saíndo de pontos diversos da mesma haste, se elevam ao mesmo nivel.
\section{Corimboso}
\begin{itemize}
\item {Grp. gram.:adj.}
\end{itemize}
O mesmo que \textunderscore corimbífero\textunderscore .
\section{Corinocarpo}
\begin{itemize}
\item {Grp. gram.:m.}
\end{itemize}
Gênero de plantas mirsíneas.
\section{Coristo}
\begin{itemize}
\item {Grp. gram.:m.}
\end{itemize}
Gênero de crustáceos decápodes.
\section{Coriza}
\begin{itemize}
\item {Proveniência:(Gr. \textunderscore koruza\textunderscore )}
\end{itemize}
\textunderscore f.\textunderscore  (\textunderscore m.\textunderscore , segundo outros)
Humor das fossas nasaes, produzido por inflamação catarral daquelas fossas.
\section{Corteché}
\begin{itemize}
\item {Grp. gram.:m.}
\end{itemize}
Cepo estreito, com que os carpinteiros e marceneiros aperfeiçoam peças curvas; o mesmo que \textunderscore corta-chefe\textunderscore .
\section{Cortejador}
\begin{itemize}
\item {Grp. gram.:m.  e  adj.}
\end{itemize}
O que corteja.
\section{Cortejar}
\begin{itemize}
\item {Grp. gram.:v. t.}
\end{itemize}
\begin{itemize}
\item {Proveniência:(De \textunderscore côrte\textunderscore )}
\end{itemize}
Tratar com cortesia.
Cumprimentar.
Lisonjear interesseiramente.
Pretender; galantear; fazer côrte a: \textunderscore cortejar uma dama\textunderscore .
\section{Cortejo}
\begin{itemize}
\item {Grp. gram.:m.}
\end{itemize}
\begin{itemize}
\item {Utilização:Fig.}
\end{itemize}
Acto de cortejar.
Cumprimentos solennes.
Comitiva pomposa; séquito.
Homenagem.
Procissão.
Acessório: \textunderscore o vicio com o seu cortejo de misérias\textunderscore .
\section{Corteleiro}
\begin{itemize}
\item {Grp. gram.:m.}
\end{itemize}
\begin{itemize}
\item {Utilização:Bras}
\end{itemize}
Boi manso, que se recolhe sempre ao curral.
(Por \textunderscore quarteleiro\textunderscore ?)
\section{Cortelha}
\begin{itemize}
\item {fónica:tê}
\end{itemize}
\begin{itemize}
\item {Grp. gram.:f.}
\end{itemize}
\begin{itemize}
\item {Utilização:Prov.}
\end{itemize}
\begin{itemize}
\item {Utilização:beir.}
\end{itemize}
\begin{itemize}
\item {Proveniência:(De \textunderscore córte\textunderscore ^2)}
\end{itemize}
Córte, curral.
Pocilga.
Lugar cerrado, onde se recolhem as crias das cabras ou ovelhas, para que estas dêem também leite para queijos e outros usos.
\section{Cortelho}
\begin{itemize}
\item {fónica:tê}
\end{itemize}
\begin{itemize}
\item {Grp. gram.:m.}
\end{itemize}
\begin{itemize}
\item {Utilização:Prov.}
\end{itemize}
\begin{itemize}
\item {Utilização:beir.}
\end{itemize}
\begin{itemize}
\item {Proveniência:(De \textunderscore córte\textunderscore ^2)}
\end{itemize}
Córte, curral.
Pocilga.
Lugar cerrado, onde se recolhem as crias das cabras ou ovelhas, para que estas dêem também leite para queijos e outros usos.
\section{Cortês}
\begin{itemize}
\item {Grp. gram.:adj.}
\end{itemize}
Que tem cortesia: \textunderscore homem cortês\textunderscore .
Em que há cortesia: \textunderscore acto cortês\textunderscore .
(B. lat. \textunderscore curtensis\textunderscore )
\section{Cortesã}
\begin{itemize}
\item {Grp. gram.:f.}
\end{itemize}
\begin{itemize}
\item {Proveniência:(De \textunderscore cortesão\textunderscore )}
\end{itemize}
Mulher dissoluta, que vive luxuosamente.
\section{Cortesan}
\begin{itemize}
\item {Grp. gram.:f.}
\end{itemize}
\begin{itemize}
\item {Proveniência:(De \textunderscore cortesão\textunderscore )}
\end{itemize}
Mulher dissoluta, que vive luxuosamente.
\section{Cortesania}
\begin{itemize}
\item {Grp. gram.:f.}
\end{itemize}
Modos de cortesão.
\section{Cortesanice}
\begin{itemize}
\item {Grp. gram.:f.}
\end{itemize}
\begin{itemize}
\item {Proveniência:(De \textunderscore cortesão\textunderscore )}
\end{itemize}
Simulação de cortesia; urbanidade apparente.
\section{Cortesanmente}
\begin{itemize}
\item {Grp. gram.:adv.}
\end{itemize}
(V.cortêsmente)
\section{Cortesão}
\begin{itemize}
\item {Grp. gram.:adj.}
\end{itemize}
\begin{itemize}
\item {Grp. gram.:M.}
\end{itemize}
\begin{itemize}
\item {Utilização:Gír.}
\end{itemize}
\begin{itemize}
\item {Proveniência:(Do b. lat. \textunderscore cortesanus\textunderscore )}
\end{itemize}
Relativo a côrte; palaciano.
Cortês.
Homem da côrte, áulico.
Homem adulador.
Aquelle que é cortês.
Chapéu fino.
\section{Cortesia}
\begin{itemize}
\item {Grp. gram.:f.}
\end{itemize}
\begin{itemize}
\item {Proveniência:(De \textunderscore cortês\textunderscore )}
\end{itemize}
Qualidade daquillo ou de quem é cortês.
Delicadeza; polidez; urbanidade.
Homenagem.
Cumprimento, mesura: \textunderscore fez-lhe uma cortesia\textunderscore .
Modos de homem da côrte.
\section{Cortêsmente}
\begin{itemize}
\item {Grp. gram.:adv.}
\end{itemize}
De modo cortês.
Delicadamente; com polidez.
\section{Córtex}
\begin{itemize}
\item {Grp. gram.:m.}
\end{itemize}
\begin{itemize}
\item {Utilização:Bot.}
\end{itemize}
\begin{itemize}
\item {Proveniência:(Lat. \textunderscore cortex\textunderscore )}
\end{itemize}
Casca de árvore. Cf. Herculano, \textunderscore Eurico\textunderscore , 251.
\section{Corteza}
\begin{itemize}
\item {fónica:tê}
\end{itemize}
\begin{itemize}
\item {Grp. gram.:f.}
\end{itemize}
\begin{itemize}
\item {Utilização:Ant.}
\end{itemize}
\begin{itemize}
\item {Proveniência:(De \textunderscore côrto\textunderscore )}
\end{itemize}
Aquillo que se cortou ou se separou de um todo.
\section{Cortiça}
\begin{itemize}
\item {Grp. gram.:f.}
\end{itemize}
\begin{itemize}
\item {Grp. gram.:Pl.}
\end{itemize}
Casca do sobreiro e de outras árvores lenhosas.
Engaço, baganha, que sobrenada na fermentação do mosto, e á superfície do azeite que se espreme no lagar.
Peças de cortiça, com que se aprende a nadar.
Rodas de cortiça, que sustentam á tona da água uma das bordas de certas redes.
(Cp. \textunderscore cortiço\textunderscore )
\section{Cortiçada}
\begin{itemize}
\item {Grp. gram.:f.}
\end{itemize}
Série de cortiços.
\section{Cortical}
\begin{itemize}
\item {Grp. gram.:adj.}
\end{itemize}
\begin{itemize}
\item {Proveniência:(Do lat. \textunderscore cortex\textunderscore )}
\end{itemize}
Relativo á cortiça, ou á casca.
Diz-se da substância cinzenta, que reveste a substância medullar dos rins e do cérebro.
\section{Córtice}
\begin{itemize}
\item {Grp. gram.:m.}
\end{itemize}
O mesmo que \textunderscore córtex\textunderscore .
\section{Corticeira}
\begin{itemize}
\item {Grp. gram.:f.}
\end{itemize}
\begin{itemize}
\item {Proveniência:(De \textunderscore cortiça\textunderscore )}
\end{itemize}
Lugar, onde se junta cortiça, para se vender, ou para se expedir em carregamentos.
Antiga vasilha de cortiça, que levava 6 canadas.
\section{Corticeiro}
\begin{itemize}
\item {Grp. gram.:adj.}
\end{itemize}
\begin{itemize}
\item {Grp. gram.:M.}
\end{itemize}
Relativo a cortiça: \textunderscore a indústria corticeira\textunderscore .
Homem, que trabalha na tirada da cortiça, nos sobreiraes.
Negociante de cortiça.
\section{Corticento}
\begin{itemize}
\item {Grp. gram.:adj.}
\end{itemize}
Que tem o aspecto ou a natureza da cortiça.
\section{Cortíceo}
\begin{itemize}
\item {Grp. gram.:adj.}
\end{itemize}
Feito de cortiça.
\section{Corticícola}
\begin{itemize}
\item {Grp. gram.:adj.}
\end{itemize}
\begin{itemize}
\item {Proveniência:(Do lat. \textunderscore cortex\textunderscore  + \textunderscore colere\textunderscore )}
\end{itemize}
Que vive na casca das árvores.
\section{Corticífero}
\begin{itemize}
\item {Grp. gram.:adj.}
\end{itemize}
\begin{itemize}
\item {Proveniência:(Do lat. \textunderscore cortex\textunderscore  + \textunderscore ferre\textunderscore )}
\end{itemize}
Que produz cortiça.
\section{Corticiforme}
\begin{itemize}
\item {Grp. gram.:adj.}
\end{itemize}
\begin{itemize}
\item {Proveniência:(Do lat. \textunderscore cortex\textunderscore  + \textunderscore forma\textunderscore )}
\end{itemize}
Que tem a apparência de cortiça.
\section{Corticina}
\begin{itemize}
\item {Grp. gram.:f.}
\end{itemize}
\begin{itemize}
\item {Proveniência:(Do lat. \textunderscore cortex\textunderscore )}
\end{itemize}
Variedade de tanino, commum a todas as cascas lenhosas dos vegetaes.
\section{Corticite}
\begin{itemize}
\item {Grp. gram.:f.}
\end{itemize}
Substância, composta de agglomerados de cortiça, destinada a revestir pavimentos.
\section{Cortiço}
\begin{itemize}
\item {Grp. gram.:m.}
\end{itemize}
\begin{itemize}
\item {Utilização:Bras}
\end{itemize}
\begin{itemize}
\item {Utilização:Gír.}
\end{itemize}
\begin{itemize}
\item {Proveniência:(Lat. hyp. \textunderscore corticium\textunderscore )}
\end{itemize}
Caixa cylíndrica de cortiça, em que as abelhas se criam e fabricam o mel e a cera.
Agrupamento de pequenas casas ou compartimentos.
Casa de habitação.
\section{Cortiçó}
\begin{itemize}
\item {Grp. gram.:m.}
\end{itemize}
Ave gallinácea de arribação, pouco maior que uma rôla.
\section{Cortiçol}
\begin{itemize}
\item {Grp. gram.:m.}
\end{itemize}
Ave gallinácea de arribação, pouco maior que uma rôla.
\section{Cortiçola}
\begin{itemize}
\item {Grp. gram.:f.}
\end{itemize}
Ave gallinácea de arribação, pouco maior que uma rôla.
\section{Corticoso}
\begin{itemize}
\item {Grp. gram.:adj.}
\end{itemize}
\begin{itemize}
\item {Proveniência:(Lat. \textunderscore corticosus\textunderscore )}
\end{itemize}
Que tem casca muito grossa.
\section{Cortiçoso}
\begin{itemize}
\item {Grp. gram.:adj.}
\end{itemize}
Que cria cortiça.
\section{Cortil}
\begin{itemize}
\item {Grp. gram.:m.}
\end{itemize}
Córte pequena, cortelha.
\section{Cortilha}
\begin{itemize}
\item {Grp. gram.:f.}
\end{itemize}
\begin{itemize}
\item {Proveniência:(Do rad. de \textunderscore cortar\textunderscore )}
\end{itemize}
Instrumento, em fórma de roseta, com que os pasteleiros e doceiros recortam as massas; cortadeira.
\section{Cortilhar}
\begin{itemize}
\item {Grp. gram.:v. t.}
\end{itemize}
\begin{itemize}
\item {Utilização:Des.}
\end{itemize}
Cortar em bocadinhos.
\section{Cortim}
\begin{itemize}
\item {Grp. gram.:m.}
\end{itemize}
\begin{itemize}
\item {Proveniência:(Do lat. \textunderscore cortex\textunderscore ?)}
\end{itemize}
O mesmo que \textunderscore tanino\textunderscore .
\section{Cortina}
\begin{itemize}
\item {Grp. gram.:f.}
\end{itemize}
\begin{itemize}
\item {Proveniência:(Lat. \textunderscore cortina\textunderscore )}
\end{itemize}
Peça de pano, que, suspensa, resguarda, enfeita ou encobre alguma coisa.
Muro, que liga dois baluartes.
Pequeno muro, que resguarda um caminho, á beira de um precipício.
Fileira.
\section{Cortinado}
\begin{itemize}
\item {Grp. gram.:m.}
\end{itemize}
\begin{itemize}
\item {Proveniência:(De \textunderscore cortinar\textunderscore )}
\end{itemize}
Armação de cortinas; cortina.
\section{Cortinar}
\begin{itemize}
\item {Grp. gram.:v. t.}
\end{itemize}
Armar com cortina; encobrir.
\section{Cortinha}
\begin{itemize}
\item {Grp. gram.:f.}
\end{itemize}
\begin{itemize}
\item {Utilização:Prov.}
\end{itemize}
\begin{itemize}
\item {Proveniência:(De \textunderscore córte\textunderscore ^2)}
\end{itemize}
Coirela lavradia, mais comprida que larga.
Terreno, vedado por vallados, atrás da habitação, e mais extenso que o quintal ordinário.
\section{Cortinhal}
\begin{itemize}
\item {Grp. gram.:m.}
\end{itemize}
\begin{itemize}
\item {Utilização:Ant.}
\end{itemize}
Campo, dividido em cortinhas.
Cortinha, cercada de sebe ou de parede.
\section{Cortinheiro}
\begin{itemize}
\item {Grp. gram.:m.}
\end{itemize}
\begin{itemize}
\item {Utilização:Prov.}
\end{itemize}
\begin{itemize}
\item {Utilização:trasm.}
\end{itemize}
Terreno cercado, nas vizinhanças da povoação, mas não annexo ás habitações.
\section{Cortir}
\textunderscore v. t.\textunderscore  (e der.)
(V. \textunderscore curtir\textunderscore , etc.)
\section{Cortis}
\begin{itemize}
\item {Grp. gram.:m. pl.}
\end{itemize}
\begin{itemize}
\item {Utilização:Bras}
\end{itemize}
Aborígenes de Goiás.
\section{Côrto}
\begin{itemize}
\item {Grp. gram.:adj.}
\end{itemize}
O mesmo que \textunderscore cortado\textunderscore .
(\textunderscore Part. irr.\textunderscore  de \textunderscore cortar\textunderscore )
\section{Cortonomia}
\begin{itemize}
\item {Grp. gram.:f.}
\end{itemize}
\begin{itemize}
\item {Proveniência:(Do gr. \textunderscore khorton\textunderscore  + \textunderscore nomos\textunderscore )}
\end{itemize}
Arte de fazer herbários.
\section{Cortusa}
\begin{itemize}
\item {Grp. gram.:f.}
\end{itemize}
Planta utriculariácea.
\section{Coru}
\begin{itemize}
\item {Grp. gram.:m.}
\end{itemize}
Planta medicinal da Índia.
\section{Coruche}
\begin{itemize}
\item {Grp. gram.:f.}
\end{itemize}
\begin{itemize}
\item {Proveniência:(De \textunderscore Coruche\textunderscore , n. p.)}
\end{itemize}
Variedade de pêra muito apreciada.
\section{Coruchéo}
\begin{itemize}
\item {Grp. gram.:m.}
\end{itemize}
\begin{itemize}
\item {Proveniência:(Do fr. \textunderscore clocher\textunderscore )}
\end{itemize}
Parte mais elevada de uma tôrre.
Zimbório.
Tôrre ou torreão, que corôa um edifício.
\section{Coruchéu}
\begin{itemize}
\item {Grp. gram.:m.}
\end{itemize}
\begin{itemize}
\item {Proveniência:(Do fr. \textunderscore clocher\textunderscore )}
\end{itemize}
Parte mais elevada de uma tôrre.
Zimbório.
Tôrre ou torreão, que corôa um edifício.
\section{Corucho}
\begin{itemize}
\item {Grp. gram.:m.}
\end{itemize}
\begin{itemize}
\item {Utilização:Prov.}
\end{itemize}
\begin{itemize}
\item {Utilização:minh.}
\end{itemize}
Coroça com capuz, usada pelos lavradores das cercanias de Viana. Cf. O. Pratt, \textunderscore Ling. Minh.\textunderscore 
(Cp. \textunderscore coruchéu\textunderscore )
\section{Coruja}
\begin{itemize}
\item {Grp. gram.:f.}
\end{itemize}
\begin{itemize}
\item {Utilização:Fig.}
\end{itemize}
Ave nocturna de rapina.
Mulher velha e feia.
\section{Corujão}
\begin{itemize}
\item {Grp. gram.:m.}
\end{itemize}
(V. \textunderscore bufo\textunderscore ^2)
\section{Corujeira}
\begin{itemize}
\item {Grp. gram.:f.}
\end{itemize}
\begin{itemize}
\item {Proveniência:(De \textunderscore coruja\textunderscore )}
\end{itemize}
Povoação reles, em sítio penhascoso, mais próprio para criação de corujas.
\section{Corujeiro}
\begin{itemize}
\item {Grp. gram.:m.}
\end{itemize}
O mesmo que \textunderscore corujeira\textunderscore .
\section{Corujo}
\begin{itemize}
\item {Grp. gram.:m.}
\end{itemize}
\begin{itemize}
\item {Utilização:Ant.}
\end{itemize}
Macho da coruja. Cf. G. Vicente, \textunderscore Inês Pereira\textunderscore .
\section{Corumbamba}
\begin{itemize}
\item {Grp. gram.:m.}
\end{itemize}
\begin{itemize}
\item {Utilização:Bras. de Minas}
\end{itemize}
Acontecimento complicado.
\section{Corumbetaru}
\begin{itemize}
\item {Grp. gram.:m.}
\end{itemize}
\begin{itemize}
\item {Utilização:Bras}
\end{itemize}
Planta rutácea, medicinal.
\section{Corumbim}
\begin{itemize}
\item {Grp. gram.:m.}
\end{itemize}
Pastor, na Índia portuguesa. Cf. Th. Ribeiro, \textunderscore Jornadas\textunderscore , II, 100.
\section{Corumbins}
\begin{itemize}
\item {Grp. gram.:m. pl.}
\end{itemize}
Tríbo indígena do Brasil.
\section{Corumim}
\begin{itemize}
\item {Grp. gram.:m.}
\end{itemize}
\begin{itemize}
\item {Utilização:Bras}
\end{itemize}
Criado índio.
(Cp. \textunderscore corumbins\textunderscore )
\section{Corunha}
\begin{itemize}
\item {Grp. gram.:f.}
\end{itemize}
\begin{itemize}
\item {Utilização:Prov.}
\end{itemize}
\begin{itemize}
\item {Utilização:trasm.}
\end{itemize}
O mesmo que \textunderscore caroço\textunderscore .
\section{Coruscação}
\begin{itemize}
\item {Grp. gram.:f.}
\end{itemize}
\begin{itemize}
\item {Proveniência:(Lat. \textunderscore coruscatio\textunderscore )}
\end{itemize}
Acto de coruscar.
\section{Coruscante}
\begin{itemize}
\item {Grp. gram.:adj.}
\end{itemize}
\begin{itemize}
\item {Proveniência:(Lat. \textunderscore coruscans\textunderscore )}
\end{itemize}
Que corusca.
\section{Coruscar}
\begin{itemize}
\item {Grp. gram.:v. i.}
\end{itemize}
\begin{itemize}
\item {Proveniência:(Lat. \textunderscore coruscare\textunderscore )}
\end{itemize}
Reluzir.
Relampaguear; coriscar.
\section{Coruta}
\begin{itemize}
\item {Grp. gram.:f.}
\end{itemize}
O mesmo que \textunderscore coruto\textunderscore .
\section{Corutilho}
\begin{itemize}
\item {Grp. gram.:m.}
\end{itemize}
\begin{itemize}
\item {Utilização:Bot.}
\end{itemize}
\begin{itemize}
\item {Proveniência:(De \textunderscore coruto\textunderscore )}
\end{itemize}
Barba, papilho ou pragana de algumas sementes.
\section{Coruto}
\begin{itemize}
\item {Grp. gram.:m.}
\end{itemize}
\begin{itemize}
\item {Proveniência:(Do rad. de \textunderscore corôa\textunderscore ?)}
\end{itemize}
O ponto mais alto de vários objectos; pináculo, summidade; cocoruto: \textunderscore no coruto da cabeça\textunderscore .
Pennacho ou arestas do milho e de outras plantas.
\section{Córva}
Espécie de peixe, que se pesca com anzol nas costas do Algarve.
(Cp. \textunderscore corvina\textunderscore )
\section{Corvacha}
\begin{itemize}
\item {Grp. gram.:f.}
\end{itemize}
\begin{itemize}
\item {Utilização:Prov.}
\end{itemize}
\begin{itemize}
\item {Utilização:beir.}
\end{itemize}
\begin{itemize}
\item {Proveniência:(De \textunderscore corvacho\textunderscore )}
\end{itemize}
Fêmea do corvo.
\section{Corvacho}
\begin{itemize}
\item {Grp. gram.:m.}
\end{itemize}
Pequeno corvo.
\section{Corveiro}
\begin{itemize}
\item {Grp. gram.:m.}
\end{itemize}
\begin{itemize}
\item {Utilização:Prov.}
\end{itemize}
\begin{itemize}
\item {Utilização:alg.}
\end{itemize}
\begin{itemize}
\item {Utilização:alent.}
\end{itemize}
Pequeno curral, coberto de colmo, onde se prendem os chibos, até se mungirem as mães.
\section{Corvejar}
\begin{itemize}
\item {Grp. gram.:v. t.}
\end{itemize}
\begin{itemize}
\item {Grp. gram.:V. i.}
\end{itemize}
\begin{itemize}
\item {Proveniência:(De \textunderscore corvo\textunderscore )}
\end{itemize}
Remoer, repisar, (uma ideia, um assunto).
Crocitar.
\section{Corvelo}
\begin{itemize}
\item {fónica:vê}
\end{itemize}
\begin{itemize}
\item {Grp. gram.:m.  e  adj.}
\end{itemize}
O mesmo que \textunderscore coroense\textunderscore .
\section{Corvense}
\begin{itemize}
\item {Grp. gram.:adj.}
\end{itemize}
\begin{itemize}
\item {Grp. gram.:M.}
\end{itemize}
Relativo á ilha do Corvo.
Habitante da ilha do Corvo.
\section{Corveta}
\begin{itemize}
\item {fónica:vê}
\end{itemize}
\begin{itemize}
\item {Grp. gram.:f.}
\end{itemize}
\begin{itemize}
\item {Utilização:Gír.}
\end{itemize}
\begin{itemize}
\item {Proveniência:(Do lat. \textunderscore corbita\textunderscore )}
\end{itemize}
Navio de guerra com três mastros.
Cachimbo.
\section{Corvéu}
\begin{itemize}
\item {Grp. gram.:m.}
\end{itemize}
Espécie de taínha.
\section{Corvideos}
\begin{itemize}
\item {Grp. gram.:m. pl.}
\end{itemize}
\begin{itemize}
\item {Proveniência:(Do lat. \textunderscore corvus\textunderscore  + gr. \textunderscore eidos\textunderscore )}
\end{itemize}
Tríbo de pássaros conirostros, que têm por typo o corvo.
\section{Corvina}
\begin{itemize}
\item {Grp. gram.:f.}
\end{itemize}
\begin{itemize}
\item {Proveniência:(De \textunderscore córva\textunderscore ?)}
\end{itemize}
Peixe esquamodermo.
\section{Corvineiro}
\begin{itemize}
\item {Grp. gram.:m.}
\end{itemize}
\begin{itemize}
\item {Utilização:Prov.}
\end{itemize}
\begin{itemize}
\item {Utilização:alg.}
\end{itemize}
\begin{itemize}
\item {Proveniência:(De \textunderscore corvina\textunderscore )}
\end{itemize}
Pequeno roaz, que persegue as corvinas.
\section{Corvino}
\begin{itemize}
\item {Grp. gram.:adj.}
\end{itemize}
\begin{itemize}
\item {Proveniência:(Lat. \textunderscore corvinus\textunderscore )}
\end{itemize}
Relativo a corvo.
\section{Corvo}
\begin{itemize}
\item {fónica:côr}
\end{itemize}
\begin{itemize}
\item {Grp. gram.:m.}
\end{itemize}
\begin{itemize}
\item {Proveniência:(Lat. \textunderscore corvus\textunderscore )}
\end{itemize}
Ave carnívora.
Constellação austral.
Modilhão.
\section{Corvo-marinho}
\begin{itemize}
\item {Grp. gram.:m.}
\end{itemize}
Ave aquática, palmípede, (\textunderscore phalacro corax carbo\textunderscore , Leach.).
\section{Corybante}
\begin{itemize}
\item {Grp. gram.:m.}
\end{itemize}
\begin{itemize}
\item {Proveniência:(Lat. \textunderscore corybantes\textunderscore )}
\end{itemize}
Nome dos sacerdotes da deusa Cybele, notáveis por certas devoções violentas.
\section{Corybânticas}
\begin{itemize}
\item {Grp. gram.:f. pl.}
\end{itemize}
\begin{itemize}
\item {Proveniência:(De \textunderscore corybântico\textunderscore )}
\end{itemize}
Festas de Cybele, celebradas pelos corybantes.
\section{Corybântico}
\begin{itemize}
\item {Grp. gram.:adj.}
\end{itemize}
Relativo aos corybantes.
\section{Corýcio}
\begin{itemize}
\item {Grp. gram.:m.}
\end{itemize}
Indivíduo natural de Córyco, cidade Cilícia:«\textunderscore conheci um corýcio em annos já maduros\textunderscore ». Castilho, \textunderscore Geórg.\textunderscore 
\section{Corydálida}
\begin{itemize}
\item {Grp. gram.:f.}
\end{itemize}
(V.corýdalo)
\section{Corydalina}
\begin{itemize}
\item {Grp. gram.:f.}
\end{itemize}
Alcaloide, extrahido do corýdalo.
\section{Corýdalo}
\begin{itemize}
\item {Grp. gram.:m.}
\end{itemize}
\begin{itemize}
\item {Proveniência:(Gr. \textunderscore korudalos\textunderscore )}
\end{itemize}
Gênero de plantas fumariáceas.
\section{Corylina}
\begin{itemize}
\item {Grp. gram.:f.}
\end{itemize}
\begin{itemize}
\item {Utilização:Pharm.}
\end{itemize}
\begin{itemize}
\item {Proveniência:(De \textunderscore córylo\textunderscore )}
\end{itemize}
Medicamento, derivado do menthol.
\section{Córylo}
\begin{itemize}
\item {Grp. gram.:m.}
\end{itemize}
\begin{itemize}
\item {Proveniência:(Do gr. \textunderscore korus\textunderscore )}
\end{itemize}
Gênero de plantas cupulíferas.
\section{Corymbíferas}
\begin{itemize}
\item {Grp. gram.:f. pl.}
\end{itemize}
\begin{itemize}
\item {Proveniência:(De \textunderscore corymbífero\textunderscore )}
\end{itemize}
Grande divisão de plantas, da fam. das compostas, caracterizadas por terem flôres em corymbo.
\section{Corymbífero}
\begin{itemize}
\item {Grp. gram.:adj.}
\end{itemize}
\begin{itemize}
\item {Utilização:Bot.}
\end{itemize}
\begin{itemize}
\item {Proveniência:(Do lat. \textunderscore corymbus\textunderscore  + \textunderscore ferre\textunderscore )}
\end{itemize}
Que tem flôres em corymbo.
\section{Corymbo}
\begin{itemize}
\item {Grp. gram.:m.}
\end{itemize}
\begin{itemize}
\item {Utilização:Bot.}
\end{itemize}
\begin{itemize}
\item {Proveniência:(Lat. \textunderscore corymbus\textunderscore )}
\end{itemize}
Conjunto de flôres, que, saíndo de pontos diversos da mesma haste, se elevam ao mesmo nivel.
\section{Corymboso}
\begin{itemize}
\item {Grp. gram.:adj.}
\end{itemize}
O mesmo que \textunderscore corymbífero\textunderscore .
\section{Corynocarpo}
\begin{itemize}
\item {Grp. gram.:m.}
\end{itemize}
Gênero de plantas myrsíneas.
\section{Corypha}
\begin{itemize}
\item {Grp. gram.:f.}
\end{itemize}
\begin{itemize}
\item {Proveniência:(Do gr. \textunderscore koruphe\textunderscore , vértice)}
\end{itemize}
Gênero de palmeiras.
\section{Corypheu}
\begin{itemize}
\item {Grp. gram.:m.}
\end{itemize}
\begin{itemize}
\item {Proveniência:(Gr. \textunderscore koryphaios\textunderscore )}
\end{itemize}
Director de córos no antigo theatro.
Pessôa, que occupa o primeiro lugar numa classe ou numa profissão; caudilho, chefe.
\section{Corysto}
\begin{itemize}
\item {Grp. gram.:m.}
\end{itemize}
Gênero de crustáceos decápodes.
\section{Coryza}
\begin{itemize}
\item {Proveniência:(Gr. \textunderscore koruza\textunderscore )}
\end{itemize}
\textunderscore f.\textunderscore  (\textunderscore m.\textunderscore , segundo outros)
Humor das fossas nasaes, produzido por inflammação catarral daquellas fossas.
\section{Cós}
\begin{itemize}
\item {Grp. gram.:m.}
\end{itemize}
\begin{itemize}
\item {Proveniência:(Do fr. \textunderscore cors\textunderscore , por \textunderscore corps\textunderscore ?)}
\end{itemize}
Parte de vestuário, especialmente das calças, com a qual se cinge a cintura.
Tira de pano, sôbre que se ajustam os punhos e o collarinho.
\section{Cosaco}
\begin{itemize}
\item {Grp. gram.:m.}
\end{itemize}
\begin{itemize}
\item {Utilização:Ext.}
\end{itemize}
\begin{itemize}
\item {Proveniência:(Fr. \textunderscore cosaque\textunderscore )}
\end{itemize}
Cavalleiro ou guerreiro russo, pertencente a certas povoações das vizinhaças do Don.
Homem rude, feroz, meio bárbaro.
\section{Cosanza}
\begin{itemize}
\item {Grp. gram.:f.}
\end{itemize}
Árvore de Angola, (\textunderscore memecylos vogelu\textunderscore ).
\section{Coscas}
\begin{itemize}
\item {Grp. gram.:f. pl.}
\end{itemize}
\begin{itemize}
\item {Utilização:Prov.}
\end{itemize}
\begin{itemize}
\item {Utilização:trasm.}
\end{itemize}
O mesmo que \textunderscore cócegas\textunderscore .
\section{Coscinomancia}
\begin{itemize}
\item {Grp. gram.:f.}
\end{itemize}
\begin{itemize}
\item {Proveniência:(Do gr. \textunderscore koskinos\textunderscore  + \textunderscore manteia\textunderscore )}
\end{itemize}
Adivinhação, por meio de uma peneira.
\section{Coscinoscopia}
\begin{itemize}
\item {Grp. gram.:f.}
\end{itemize}
\begin{itemize}
\item {Proveniência:(Do gr. \textunderscore koskinos\textunderscore  + \textunderscore skopein\textunderscore )}
\end{itemize}
O mesmo que \textunderscore coscinomancia\textunderscore .
\section{Cosco}
\begin{itemize}
\item {fónica:côs}
\end{itemize}
\begin{itemize}
\item {Grp. gram.:m.}
\end{itemize}
\begin{itemize}
\item {Utilização:Prov.}
\end{itemize}
\begin{itemize}
\item {Utilização:minh.}
\end{itemize}
\begin{itemize}
\item {Grp. gram.:M.}
\end{itemize}
\begin{itemize}
\item {Utilização:Prov.}
\end{itemize}
\begin{itemize}
\item {Proveniência:(Do cast. \textunderscore cuesco\textunderscore )}
\end{itemize}
O mesmo que \textunderscore coscorão\textunderscore .
Casca do grão de centeio ou de trigo.
Palhiço, em que se encontram algumas espigas e grãos, na occasião da malha. (Colhido em Arganil)
\section{Coscojas}
\begin{itemize}
\item {Grp. gram.:f. pl.}
\end{itemize}
Anéis de ferro, na sella de estardiota.
(Cast. \textunderscore coscoja\textunderscore )
\section{Coscorado}
\begin{itemize}
\item {Grp. gram.:adj.}
\end{itemize}
Que tem coscóro ou crosta. Cf. Filinto, X, 129.
\section{Coscorão}
\begin{itemize}
\item {Grp. gram.:m.}
\end{itemize}
\begin{itemize}
\item {Proveniência:(De \textunderscore coscóro\textunderscore )}
\end{itemize}
Filhó de farinha e ovos.
\section{Coscorel}
\begin{itemize}
\item {Grp. gram.:m.}
\end{itemize}
(V.coscorão)
\section{Coscóro}
\begin{itemize}
\item {Grp. gram.:m.}
\end{itemize}
Endurecimento.
Crosta.
Encrespamento de um tecido que, depois de metido em líquido espêsso, se deixou secar.
\section{Cóscoro}
\begin{itemize}
\item {Grp. gram.:m.}
\end{itemize}
Casta de uva branca da região do Doiro.
\section{Coscorrão}
\begin{itemize}
\item {Grp. gram.:m.}
\end{itemize}
Carôlo, pancada com a mão.
(Cast. \textunderscore cuscurrán\textunderscore )
\section{Coscorrinho}
\begin{itemize}
\item {Grp. gram.:m.}
\end{itemize}
\begin{itemize}
\item {Utilização:Pop.}
\end{itemize}
\begin{itemize}
\item {Proveniência:(Do rad. de \textunderscore coscos\textunderscore )}
\end{itemize}
Mealheiro.
\section{Coscos}
\begin{itemize}
\item {Grp. gram.:m. pl.}
\end{itemize}
\begin{itemize}
\item {Utilização:Pop.}
\end{itemize}
Dinheiro miúdo.
Vintens.
Palhiço, o mesmo que \textunderscore cosco\textunderscore .
\section{Coscós}
\begin{itemize}
\item {Grp. gram.:m.}
\end{itemize}
\begin{itemize}
\item {Utilização:Bras}
\end{itemize}
Roseta de ferro, que se suspende do freio do cavallo.
(Cp. \textunderscore coscojas\textunderscore )
\section{Cosculheiro}
\begin{itemize}
\item {Grp. gram.:adj.}
\end{itemize}
\begin{itemize}
\item {Utilização:Prov.}
\end{itemize}
\begin{itemize}
\item {Utilização:trasm.}
\end{itemize}
O mesmo que \textunderscore coscuvilheiro\textunderscore .
\section{Coscuvilhar}
\begin{itemize}
\item {Grp. gram.:v. i.}
\end{itemize}
\begin{itemize}
\item {Utilização:Pop.}
\end{itemize}
Fazer mexericos, enredos; bisbilhotar.
\section{Coscuvilheira}
\begin{itemize}
\item {Grp. gram.:f.  e  adj.}
\end{itemize}
\begin{itemize}
\item {Proveniência:(De \textunderscore coscuvilhar\textunderscore )}
\end{itemize}
Mulher mexeriqueira; bisbilhoteira.
\section{Coscuvilheiro}
\begin{itemize}
\item {Grp. gram.:m.  e  adj.}
\end{itemize}
O que coscuvilha.
\section{Coscuvilhice}
\begin{itemize}
\item {Grp. gram.:f.}
\end{itemize}
\begin{itemize}
\item {Proveniência:(De \textunderscore coscuvilhar\textunderscore )}
\end{itemize}
Intriga, enrêdo.
Bisbilhotice.
\section{Coscuzeiro}
\begin{itemize}
\item {Grp. gram.:adj.}
\end{itemize}
\begin{itemize}
\item {Utilização:Ant.}
\end{itemize}
Dizia-se do chapéu que tinha copa alta.
\section{Co-secante}
\begin{itemize}
\item {Grp. gram.:f.  e  adj.}
\end{itemize}
\begin{itemize}
\item {Utilização:Geom.}
\end{itemize}
\begin{itemize}
\item {Proveniência:(De \textunderscore com...\textunderscore  + \textunderscore secante\textunderscore )}
\end{itemize}
Secante do complemento de um ângulo.
\section{Cosedor}
\begin{itemize}
\item {Grp. gram.:m.}
\end{itemize}
\begin{itemize}
\item {Proveniência:(De \textunderscore coser\textunderscore )}
\end{itemize}
Apparelho de encadernador, para coser livros.
\section{Cosedora}
\begin{itemize}
\item {Grp. gram.:f.}
\end{itemize}
\begin{itemize}
\item {Proveniência:(De \textunderscore coser\textunderscore )}
\end{itemize}
Mulher, que cose as seiras dos figos passados, nos armazens em que êstes se preparam para o commércio. Cf. \textunderscore Inquér. Industr.\textunderscore , p. II, l. III, 147.
\section{Cosedura}
\begin{itemize}
\item {Grp. gram.:f.}
\end{itemize}
Acto de coser.
\section{Co-seno}
\begin{itemize}
\item {Grp. gram.:m.}
\end{itemize}
\begin{itemize}
\item {Utilização:Geom.}
\end{itemize}
\begin{itemize}
\item {Proveniência:(De \textunderscore com...\textunderscore  + \textunderscore sens\textunderscore )}
\end{itemize}
Seno do complemento de um ângulo.
\section{Coser}
\begin{itemize}
\item {Grp. gram.:v. t.}
\end{itemize}
\begin{itemize}
\item {Grp. gram.:V. i.}
\end{itemize}
\begin{itemize}
\item {Grp. gram.:V. p.}
\end{itemize}
\begin{itemize}
\item {Proveniência:(Lat. \textunderscore consuere\textunderscore )}
\end{itemize}
Ligar, prender, por meio de pontos, dados com um fio ou cordel enfiado numa agulha: \textunderscore coser a roupa\textunderscore .
Costurar.
Unir-se, encostar-se: \textunderscore o desconhecido coseu-se com a parede\textunderscore .
\section{Cosjombo}
\begin{itemize}
\item {Grp. gram.:m.}
\end{itemize}
Árvore da Índia portuguesa.
\section{Cosmélia}
\begin{itemize}
\item {Grp. gram.:m.}
\end{itemize}
Gênero de plantas epacrídeas.
\section{Cosmético}
\begin{itemize}
\item {Grp. gram.:adj.}
\end{itemize}
\begin{itemize}
\item {Grp. gram.:M.}
\end{itemize}
\begin{itemize}
\item {Proveniência:(Gr. \textunderscore kosmetikos\textunderscore )}
\end{itemize}
Diz-se dos ingredientes, com que se procura conservar ou restabelecer a belleza da pelle, dos dentes ou dos cabellos.
Qualquer das substâncias próprias para êsse fim.
\section{Cósmico}
\begin{itemize}
\item {Grp. gram.:adj.}
\end{itemize}
\begin{itemize}
\item {Proveniência:(Do gr. \textunderscore kosmos\textunderscore )}
\end{itemize}
Relativo ao conjunto do universo.
\section{Cosmilra}
\begin{itemize}
\item {Grp. gram.:m.  e  f.}
\end{itemize}
\begin{itemize}
\item {Utilização:Prov.}
\end{itemize}
\begin{itemize}
\item {Utilização:alg.}
\end{itemize}
Pessôa magra e feia.
(Cp. \textunderscore mirrar\textunderscore ^2)
\section{Cosmogonia}
\begin{itemize}
\item {Grp. gram.:f.}
\end{itemize}
\begin{itemize}
\item {Proveniência:(Do gr. \textunderscore kosmos\textunderscore  + \textunderscore gonos\textunderscore )}
\end{itemize}
História hypothética da formação do mundo ou do universo.
\section{Cosmogonicamente}
\begin{itemize}
\item {Grp. gram.:adv.}
\end{itemize}
De maneira cosmogónica.
\section{Cosmogónico}
\begin{itemize}
\item {Grp. gram.:adj.}
\end{itemize}
Relativo á cosmogonia.
\section{Cosmogonista}
\begin{itemize}
\item {Grp. gram.:m.}
\end{itemize}
Aquelle quo trata de cosmogonia.
\section{Cosmografia}
\begin{itemize}
\item {Grp. gram.:f.}
\end{itemize}
\begin{itemize}
\item {Proveniência:(Do gr. \textunderscore kosmos\textunderscore  + \textunderscore graphein\textunderscore )}
\end{itemize}
Descripção astronómica do mundo.
\section{Cosmográfico}
\begin{itemize}
\item {Grp. gram.:adj.}
\end{itemize}
Relativo á Cosmografia.
\section{Cosmógrafo}
\begin{itemize}
\item {Grp. gram.:m.}
\end{itemize}
Aquele que é versado em Cosmografia, ou que trata desta ciência.
\section{Cosmographia}
\begin{itemize}
\item {Grp. gram.:f.}
\end{itemize}
\begin{itemize}
\item {Proveniência:(Do gr. \textunderscore kosmos\textunderscore  + \textunderscore graphein\textunderscore )}
\end{itemize}
Descripção astronómica do mundo.
\section{Cosmográphico}
\begin{itemize}
\item {Grp. gram.:adj.}
\end{itemize}
Relativo á Cosmographia.
\section{Cosmógrapho}
\begin{itemize}
\item {Grp. gram.:m.}
\end{itemize}
Aquelle que é versado em Cosmographia, ou que trata desta sciência.
\section{Cosmolábio}
\begin{itemize}
\item {Grp. gram.:m.}
\end{itemize}
\begin{itemize}
\item {Proveniência:(Do gr. \textunderscore kosmos\textunderscore  + \textunderscore labein\textunderscore )}
\end{itemize}
Antigo instrumento, para tomar a altura dos astros.
\section{Cosmologia}
\begin{itemize}
\item {Grp. gram.:f.}
\end{itemize}
\begin{itemize}
\item {Proveniência:(Do gr. \textunderscore kosmos\textunderscore  + \textunderscore logos\textunderscore )}
\end{itemize}
Sciência das leis geraes do mundo phýsico.
\section{Cosmologicamente}
\begin{itemize}
\item {Grp. gram.:adv.}
\end{itemize}
De maneira cosmológica.
\section{Cosmológico}
\begin{itemize}
\item {Grp. gram.:adj.}
\end{itemize}
Relativo á Cosmologia.
\section{Cosmólogo}
\begin{itemize}
\item {Grp. gram.:m.}
\end{itemize}
Aquelle que trata scientificamente de cosmologia.
\section{Cosmometria}
\begin{itemize}
\item {Grp. gram.:f.}
\end{itemize}
\begin{itemize}
\item {Proveniência:(Do gr. \textunderscore kosmos\textunderscore  + \textunderscore metron\textunderscore )}
\end{itemize}
Sciência, que trata da medida das distâncias cósmicas.
\section{Cosmométrico}
\begin{itemize}
\item {Grp. gram.:adj.}
\end{itemize}
Relativo á Cosmometria.
\section{Cosmonomia}
\begin{itemize}
\item {Grp. gram.:f.}
\end{itemize}
\begin{itemize}
\item {Proveniência:(Do gr. \textunderscore kosmos\textunderscore  + \textunderscore nomos\textunderscore )}
\end{itemize}
Conjunto das leis cósmicas.
\section{Cosmonómico}
\begin{itemize}
\item {Grp. gram.:adj.}
\end{itemize}
Relativo á cosmonomia.
\section{Cosmopolita}
\begin{itemize}
\item {Grp. gram.:m.}
\end{itemize}
\begin{itemize}
\item {Grp. gram.:Adj.}
\end{itemize}
\begin{itemize}
\item {Proveniência:(Do gr. \textunderscore kosmos\textunderscore  + \textunderscore polites\textunderscore )}
\end{itemize}
Aquelle que se considera cidadão de todo o mundo.
Aquelle que não tem residência fixa num país, e que adopta facilmente os usos das diversas nações.
Que é de todos os países; que anda por toda a parte.
\section{Cosmopolítico}
\begin{itemize}
\item {Grp. gram.:adj.}
\end{itemize}
Que anda por toda a parte, que rodeia o mundo.
\section{Cosmopolitismo}
\begin{itemize}
\item {Grp. gram.:m.}
\end{itemize}
Qualidade daquillo ou de quem é cosmopolita.
\section{Cosmorama}
\begin{itemize}
\item {Grp. gram.:m.}
\end{itemize}
\begin{itemize}
\item {Proveniência:(Do gr. \textunderscore kosmos\textunderscore  + \textunderscore orama\textunderscore )}
\end{itemize}
Conjunto de quadros, que representam regiões ou factos vários, e que são observados por apparelhos ópticos que os ampliam.
Apparelho, com que se observam êsses quadros.
Lugar, onde estão expostos.
\section{Cosmos}
\begin{itemize}
\item {Grp. gram.:m.}
\end{itemize}
\begin{itemize}
\item {Proveniência:(Lat. \textunderscore cosmus\textunderscore )}
\end{itemize}
Universo.
\section{Cosmosophia}
\begin{itemize}
\item {fónica:so}
\end{itemize}
\begin{itemize}
\item {Grp. gram.:f.}
\end{itemize}
\begin{itemize}
\item {Proveniência:(Do gr. \textunderscore kosmos\textunderscore  + \textunderscore sophia\textunderscore )}
\end{itemize}
Estudo mýstico do universo.
\section{Cosmosóphico}
\begin{itemize}
\item {fónica:so}
\end{itemize}
\begin{itemize}
\item {Grp. gram.:adj.}
\end{itemize}
Relativo á cosmosophia.
\section{Cosmossofia}
\begin{itemize}
\item {Grp. gram.:f.}
\end{itemize}
\begin{itemize}
\item {Proveniência:(Do gr. \textunderscore kosmos\textunderscore  + \textunderscore sophia\textunderscore )}
\end{itemize}
Estudo místico do universo.
\section{Cosmossófico}
\begin{itemize}
\item {Grp. gram.:adj.}
\end{itemize}
Relativo á cosmosofia.
\section{Cosmurgia}
\begin{itemize}
\item {Grp. gram.:f.}
\end{itemize}
Criação do mundo.
\section{Cosque}
\begin{itemize}
\item {Grp. gram.:m.}
\end{itemize}
\begin{itemize}
\item {Utilização:Gír.}
\end{itemize}
Casa.
(Cp. \textunderscore quiosque\textunderscore )
\section{Cosqueadura}
\begin{itemize}
\item {Grp. gram.:f.}
\end{itemize}
Acto de cosquear.
\section{Cosquear}
\begin{itemize}
\item {Grp. gram.:v. t.}
\end{itemize}
Bater, tosar, sovar.
(Cp. \textunderscore coscorrão\textunderscore )
\section{Cosquinhas}
\begin{itemize}
\item {Grp. gram.:f. pl.}
\end{itemize}
\begin{itemize}
\item {Utilização:Prov.}
\end{itemize}
\begin{itemize}
\item {Utilização:trasm.}
\end{itemize}
O mesmo que \textunderscore coscas\textunderscore .
\section{Cossa}
\begin{itemize}
\item {Grp. gram.:f.}
\end{itemize}
\begin{itemize}
\item {Utilização:Pop.}
\end{itemize}
O mesmo que \textunderscore acossa\textunderscore .
\section{Cossairo}
\begin{itemize}
\item {Grp. gram.:m.}
\end{itemize}
\begin{itemize}
\item {Utilização:Ant.}
\end{itemize}
O mesmo que \textunderscore corsário\textunderscore . Cf. \textunderscore Eufrosina\textunderscore , acto I, sc. 3.
\section{Cossas}
\begin{itemize}
\item {Grp. gram.:m. pl.}
\end{itemize}
Uma das tríbos dos Landins, na África oriental.
\section{Cosseira}
\begin{itemize}
\item {Grp. gram.:f.}
\end{itemize}
Um dos pranchões que consolidam interiormente o navio.
(Por \textunderscore coiceira\textunderscore ? Em tal caso, \textunderscore cosseira\textunderscore  é graphia errónea)
\section{Cossinete}
\begin{itemize}
\item {fónica:nê}
\end{itemize}
\begin{itemize}
\item {Grp. gram.:m.}
\end{itemize}
\begin{itemize}
\item {Utilização:Neol.}
\end{itemize}
\begin{itemize}
\item {Proveniência:(Fr. \textunderscore coussinet\textunderscore )}
\end{itemize}
Peças ôcas ou semi-cylíndricas de madeira ou metal, entre as quaes giram as extremidades de um eixo.
\section{Cóssios}
\begin{itemize}
\item {Grp. gram.:m. pl.}
\end{itemize}
\begin{itemize}
\item {Proveniência:(Do lat. \textunderscore cossus\textunderscore )}
\end{itemize}
Insectos seticórneos da ordem dos lepidópteros.
\section{Cosso}
\begin{itemize}
\item {fónica:cô}
\end{itemize}
\begin{itemize}
\item {Grp. gram.:m.}
\end{itemize}
\begin{itemize}
\item {Utilização:Prov.}
\end{itemize}
\begin{itemize}
\item {Utilização:trasm.}
\end{itemize}
\begin{itemize}
\item {Utilização:Ant.}
\end{itemize}
Lugar florido, onde pascem abelhas.
Curso, andamento, rapidez.
Acto de correr atrás, perseguindo.
Acto de bater e percorrer o mato, para caçar:«\textunderscore tomar cabras silvestres a cosso\textunderscore ». \textunderscore Ethióp. Or.\textunderscore , II, 376.
(Alter. de \textunderscore corso\textunderscore )
\section{Cossolete}
\begin{itemize}
\item {fónica:lé}
\end{itemize}
\begin{itemize}
\item {Grp. gram.:m.}
\end{itemize}
(V.corselete)
\section{Cossoleto}
\begin{itemize}
\item {fónica:lé}
\end{itemize}
\begin{itemize}
\item {Grp. gram.:m.}
\end{itemize}
(V.corselete)
\section{Cossumo}
\begin{itemize}
\item {Grp. gram.:m.}
\end{itemize}
Arvore da Índia portuguesa.
\section{Costa}
\begin{itemize}
\item {Grp. gram.:f.}
\end{itemize}
\begin{itemize}
\item {Utilização:Bras}
\end{itemize}
\begin{itemize}
\item {Grp. gram.:Loc. adv.}
\end{itemize}
\begin{itemize}
\item {Grp. gram.:Pl.}
\end{itemize}
\begin{itemize}
\item {Utilização:Fig.}
\end{itemize}
\begin{itemize}
\item {Proveniência:(Lat. \textunderscore costa\textunderscore )}
\end{itemize}
Costela.
Declive, encosta.
Praia, littoral; região á beira-mar: \textunderscore a costa de Caparica\textunderscore .
Margem do rio.
\textunderscore De costa acima\textunderscore , com difficuldade:«\textunderscore era coisa de costa acima\textunderscore ». Camillo, \textunderscore Retr. de Ricard.\textunderscore , 47. Cf. Filinto, VII, 49.
\textunderscore De costa arriba\textunderscore , o mesmo que \textunderscore de costa acima\textunderscore .
Dorso; parte posterior do tronco humano; lombo.
Ausência.
Parte posterior de vários objectos; reverso.
\section{Costa}
\begin{itemize}
\item {Grp. gram.:f.}
\end{itemize}
\begin{itemize}
\item {Proveniência:(De \textunderscore Costa\textunderscore , n. p.)}
\end{itemize}
Variedade de maçan.
\section{Costada}
\begin{itemize}
\item {Grp. gram.:f.}
\end{itemize}
\begin{itemize}
\item {Proveniência:(De \textunderscore costa\textunderscore ^1)}
\end{itemize}
Sinuosidade do curso de um rio.
\section{Costa-de-alvarelhos}
\begin{itemize}
\item {Grp. gram.:f.}
\end{itemize}
Variedade de pêra, muito sucosa e aromática.
\section{Costado}
\begin{itemize}
\item {Grp. gram.:m.}
\end{itemize}
\begin{itemize}
\item {Proveniência:(De \textunderscore costa\textunderscore ^1)}
\end{itemize}
Costas.
Pranchas, que revestem exteriormente as cavernas do navio.
Em genealogia, cada um dos quatro avós de cada indivíduo: \textunderscore nobre dos quatro costados\textunderscore .
\section{Costal}
\begin{itemize}
\item {Grp. gram.:adj.}
\end{itemize}
\begin{itemize}
\item {Grp. gram.:M.}
\end{itemize}
\begin{itemize}
\item {Proveniência:(De \textunderscore costa\textunderscore ^1)}
\end{itemize}
Relativo a costas.
Fardo, porção de mercadoria, que se póde levar ás costas.
Fios, com que se ata a meada, para que se não enrede.
\section{Costaleira}
\begin{itemize}
\item {Grp. gram.:f.}
\end{itemize}
O mesmo que \textunderscore costaneira\textunderscore .
\section{Costalgia}
\begin{itemize}
\item {Grp. gram.:f.}
\end{itemize}
\begin{itemize}
\item {Proveniência:(De \textunderscore costas\textunderscore  + gr. \textunderscore algos\textunderscore )}
\end{itemize}
Dôr violenta na região dorsal.
\section{Costálgico}
\begin{itemize}
\item {Grp. gram.:adj.}
\end{itemize}
Relativo a costalgia.
\section{Costaneira}
\begin{itemize}
\item {Grp. gram.:f.}
\end{itemize}
\begin{itemize}
\item {Utilização:Bras}
\end{itemize}
\begin{itemize}
\item {Utilização:Ant.}
\end{itemize}
\begin{itemize}
\item {Utilização:T. da Bairrada}
\end{itemize}
Primeira e última tábua de um tronco serrado em várias fôlhas.
Papel de inferior qualidade, com que se resguarda um e outro lado das resmas.
Papel grosso e ordinário.
Um dos livros em que, nas Repartições públicas, se regista a cobrança dos impostos.
Ala direita de tropas. Cf. Herculano, \textunderscore Lendas\textunderscore , 208.
Livro de papel almaço, para apontamentos, em loja de negócio.
(Cast. \textunderscore costanera\textunderscore )
\section{Costaneiro}
\begin{itemize}
\item {Grp. gram.:adj.}
\end{itemize}
\begin{itemize}
\item {Grp. gram.:M.}
\end{itemize}
Relativo a costaneira.
Lombo.
(Cast. \textunderscore costanero\textunderscore )
\section{Costeagem}
\begin{itemize}
\item {Grp. gram.:f.}
\end{itemize}
\begin{itemize}
\item {Utilização:Bras}
\end{itemize}
Acto de costear. Cf. Camillo, \textunderscore Noites de Insómn.\textunderscore , IX, 5.
O mesmo que \textunderscore cabotagem\textunderscore .
\section{Costear}
\begin{itemize}
\item {Grp. gram.:v. t.}
\end{itemize}
\begin{itemize}
\item {Utilização:Bras}
\end{itemize}
\begin{itemize}
\item {Grp. gram.:V. i.}
\end{itemize}
\begin{itemize}
\item {Proveniência:(De \textunderscore costa\textunderscore ^1)}
\end{itemize}
Rodear.
Seguir de perto.
Navegar junto á costa de: \textunderscore costear o Algarve\textunderscore .
Arrebanhar (o gado), ou habituá-lo a reunir-se em certos pontos da fazenda.
Navegar junto á costa.
\section{Costeio}
\begin{itemize}
\item {Grp. gram.:m.}
\end{itemize}
\begin{itemize}
\item {Utilização:Bras}
\end{itemize}
Acto de costear ou arrebanhar o gado.
\section{Costeira}
\begin{itemize}
\item {Grp. gram.:f.}
\end{itemize}
\begin{itemize}
\item {Utilização:Ant.}
\end{itemize}
\begin{itemize}
\item {Utilização:Prov.}
\end{itemize}
\begin{itemize}
\item {Proveniência:(De \textunderscore costeiro\textunderscore )}
\end{itemize}
Costa marítima.
O mesmo que \textunderscore encosta\textunderscore .
\section{Costeiras}
\begin{itemize}
\item {Grp. gram.:f. pl.}
\end{itemize}
\begin{itemize}
\item {Proveniência:(De \textunderscore costa\textunderscore ^1)}
\end{itemize}
Peças, que reforçam o mastro de navio, ligando-se-lhe aos lados.
\section{Costeiro}
\begin{itemize}
\item {Grp. gram.:adj.}
\end{itemize}
\begin{itemize}
\item {Proveniência:(De \textunderscore costa\textunderscore ^1)}
\end{itemize}
Relativo á costa: \textunderscore navegação costeira\textunderscore .
Que navega junto á costa, ou de pôrto a pôrto na mesma costa: \textunderscore navio costeiro\textunderscore .
\section{Costela}
\begin{itemize}
\item {Grp. gram.:f.}
\end{itemize}
\begin{itemize}
\item {Utilização:Bot.}
\end{itemize}
\begin{itemize}
\item {Proveniência:(De \textunderscore costa\textunderscore ^1)}
\end{itemize}
Osso chato, curvo e alongado, que, partindo da espinha dorsal, fórma com outros a caixa thorácica: \textunderscore na quéda, partiu as costelas\textunderscore .
Caverna (de embarcação).
Armadilha para pássaros.
Nervura média de algumas fôlhas.
\section{Costelame}
\begin{itemize}
\item {Grp. gram.:m.}
\end{itemize}
\begin{itemize}
\item {Utilização:Chul.}
\end{itemize}
As costas, as costelas: \textunderscore olha que te apalpo o costelame com um cacete\textunderscore .
\section{Costelão}
\begin{itemize}
\item {Grp. gram.:m.}
\end{itemize}
\begin{itemize}
\item {Utilização:Prov.}
\end{itemize}
\begin{itemize}
\item {Proveniência:(De \textunderscore costela\textunderscore )}
\end{itemize}
Armadilha para pássaros, formada por um arco, que tem pendente uma rêde cónica.
\section{Costeleta}
\begin{itemize}
\item {fónica:lê}
\end{itemize}
\begin{itemize}
\item {Grp. gram.:f.}
\end{itemize}
\begin{itemize}
\item {Proveniência:(De \textunderscore costela\textunderscore )}
\end{itemize}
Costela de certos animaes, separado com carne adherente.
\section{Costella}
\begin{itemize}
\item {Grp. gram.:f.}
\end{itemize}
(V.costela)
\section{Costelo}
\begin{itemize}
\item {fónica:tê}
\end{itemize}
\begin{itemize}
\item {Grp. gram.:m.}
\end{itemize}
\begin{itemize}
\item {Utilização:Prov.}
\end{itemize}
Armadilha, o mesmo que \textunderscore costilha\textunderscore , ou \textunderscore pescócia\textunderscore .
\section{Costilha}
\begin{itemize}
\item {Grp. gram.:f.}
\end{itemize}
\begin{itemize}
\item {Grp. gram.:Pl.}
\end{itemize}
\begin{itemize}
\item {Utilização:Mús.}
\end{itemize}
O mesmo que \textunderscore costela\textunderscore , armadilha.
Os lados da caixa harmónica, nos instrumentos de corda.
(Cast. \textunderscore costilla\textunderscore )
\section{Costilhar}
\begin{itemize}
\item {Grp. gram.:m.}
\end{itemize}
\begin{itemize}
\item {Utilização:Bras. do S}
\end{itemize}
Conjunto das costelas do corpo.
Parte do corpo, onde há as costelas.
(Cast. \textunderscore costillar\textunderscore )
\section{Costinha}
\begin{itemize}
\item {Grp. gram.:f.}
\end{itemize}
Casta de uva do districto de Leiria.
(Cp. \textunderscore costa\textunderscore ^2)
\section{Costinha}
\begin{itemize}
\item {Grp. gram.:f.}
\end{itemize}
\begin{itemize}
\item {Utilização:T. da Bairrada}
\end{itemize}
\begin{itemize}
\item {Proveniência:(De \textunderscore costa\textunderscore ^1)}
\end{itemize}
O mesmo que \textunderscore costela\textunderscore  do corpo do homem ou dos animaes.
\section{Costo}
\begin{itemize}
\item {fónica:côs}
\end{itemize}
\begin{itemize}
\item {Grp. gram.:m.}
\end{itemize}
\begin{itemize}
\item {Proveniência:(Lat. \textunderscore costum\textunderscore )}
\end{itemize}
Erva amomácea.
Perfume, que se extrái dessa erva.
\section{Costo...}
\begin{itemize}
\item {Grp. gram.:pref.}
\end{itemize}
\begin{itemize}
\item {Utilização:Anat.}
\end{itemize}
(Serve para designar as costas)
\section{Costo-abdominal}
\begin{itemize}
\item {Grp. gram.:adj.}
\end{itemize}
\begin{itemize}
\item {Utilização:Anat.}
\end{itemize}
Relativo ás costas e ao abdome.
\section{Costo-clavicular}
\begin{itemize}
\item {Grp. gram.:adj.}
\end{itemize}
Relativo ás costas e á clavicula.
\section{Costo-escapular}
\begin{itemize}
\item {Grp. gram.:adj.}
\end{itemize}
\begin{itemize}
\item {Utilização:Anat.}
\end{itemize}
Relativo ás costas e á omoplata.
\section{Costo-esternal}
\begin{itemize}
\item {Grp. gram.:adj.}
\end{itemize}
Relativo ás costas e ao esterno.
\section{Costo-marsupial}
\begin{itemize}
\item {Grp. gram.:adj.}
\end{itemize}
Diz-se dos músculos abdominaes da salamandra.
\section{Costo-pubiano}
\begin{itemize}
\item {Grp. gram.:adj.}
\end{itemize}
\begin{itemize}
\item {Utilização:Anat.}
\end{itemize}
Diz-se de um músculo do lado direito do baixo-ventre.
\section{Costo-thorácico}
\begin{itemize}
\item {Grp. gram.:adj.}
\end{itemize}
Relativo ás costas e ao thórax.
\section{Costo-vertebral}
\begin{itemize}
\item {Grp. gram.:adj.}
\end{itemize}
Relativo ás costelas e ás vértebras.
\section{Costra}
\begin{itemize}
\item {fónica:côs}
\end{itemize}
\begin{itemize}
\item {Grp. gram.:f.}
\end{itemize}
(V.crosta). Cf. Th. Ribeiro, \textunderscore Jornadas\textunderscore , I, 156.
\section{Costumado}
\begin{itemize}
\item {Grp. gram.:adj.}
\end{itemize}
\begin{itemize}
\item {Grp. gram.:M.}
\end{itemize}
\begin{itemize}
\item {Proveniência:(De \textunderscore costumar\textunderscore )}
\end{itemize}
Habitual; usado.
Acostumado.
Aquillo que está em costume, que é usual.
\section{Costumagem}
\begin{itemize}
\item {Grp. gram.:f.}
\end{itemize}
\begin{itemize}
\item {Utilização:Ant.}
\end{itemize}
Costume.
Direito, baseado nos costumes.
(B. lat. \textunderscore costumago\textunderscore )
\section{Costumança}
\begin{itemize}
\item {Grp. gram.:f.}
\end{itemize}
\begin{itemize}
\item {Utilização:Ant.}
\end{itemize}
O mesmo que \textunderscore costumagem\textunderscore .
\section{Costumar}
\begin{itemize}
\item {Grp. gram.:v. t.}
\end{itemize}
\begin{itemize}
\item {Grp. gram.:V. i.}
\end{itemize}
Têr por costume; têr o hábito de: \textunderscore costuma levantar-se cedo\textunderscore .
Habituar, acostumar: \textunderscore costumar os filhos á obediência\textunderscore .
Estar acostumado.
\section{Costumário}
\begin{itemize}
\item {Grp. gram.:adj.}
\end{itemize}
\begin{itemize}
\item {Proveniência:(De \textunderscore costume\textunderscore )}
\end{itemize}
O mesmo que \textunderscore consuetudinário\textunderscore .
\section{Costume}
\begin{itemize}
\item {Grp. gram.:m.}
\end{itemize}
\begin{itemize}
\item {Grp. gram.:Pl.}
\end{itemize}
\begin{itemize}
\item {Utilização:T. de Ajudá}
\end{itemize}
\begin{itemize}
\item {Proveniência:(Do lat. hyp. \textunderscore consuetumen\textunderscore )}
\end{itemize}
Uso; prática geralmente observada.
Jurisprudência, baseada no uso e não em lei escrita.
Modo vulgar.
Particularidade.
Moda.
Comportamento.
Em linguagem forense, relações de parentesco, amizade ou ódio da testemunha para quem é parte no processo.
Tributo, pago ao rei de Daomé.
\section{Costumeira}
\begin{itemize}
\item {Grp. gram.:f.}
\end{itemize}
\begin{itemize}
\item {Proveniência:(De \textunderscore costume\textunderscore )}
\end{itemize}
Usança; costume de pouca importância.
\section{Costumeiro}
\begin{itemize}
\item {Grp. gram.:adj.}
\end{itemize}
Costumário, usual. Cf. Camillo, \textunderscore Livro Negro\textunderscore , 216.
\section{Costura}
\begin{itemize}
\item {Grp. gram.:f.}
\end{itemize}
\begin{itemize}
\item {Proveniência:(Lat. hyp. \textunderscore consutura\textunderscore )}
\end{itemize}
Acto, effeito, arte, ou profissão, de coser.
Peças de estôfo ou coiro, cosidas uma á outra.
Cicatriz.
Fenda.
\section{Costurar}
\begin{itemize}
\item {Grp. gram.:v. i.}
\end{itemize}
Trabalhar em costura; coser.
\section{Costureira}
\begin{itemize}
\item {Grp. gram.:f.}
\end{itemize}
Mulher, que se emprega em trabalhos de costura.
\section{Costureiro}
\begin{itemize}
\item {Grp. gram.:m.  e  adj.}
\end{itemize}
Diz-se de um músculo comprido e estreito da região antero-interna da coxa.--Diz-se assim, porque serve para se cruzar uma perna sobre a outra, como usam geralmente costureiras e alfaiates. Cf. J. A. Serrano, \textunderscore Osteologia\textunderscore .
(Cp. \textunderscore costureira\textunderscore )
\section{Cota}
\begin{itemize}
\item {Grp. gram.:f.}
\end{itemize}
\begin{itemize}
\item {Utilização:Ant.}
\end{itemize}
\begin{itemize}
\item {Proveniência:(Do ingl. \textunderscore coat\textunderscore )}
\end{itemize}
Vestimenta, que usavam sôbre a armadura os cavalleiros antigos.
Espécie de gibão.
Corpete de dama.
\section{Cota}
\begin{itemize}
\item {Grp. gram.:f.}
\end{itemize}
\begin{itemize}
\item {Utilização:For.}
\end{itemize}
\begin{itemize}
\item {Utilização:Geod.}
\end{itemize}
\begin{itemize}
\item {Grp. gram.:Adj. f.}
\end{itemize}
\begin{itemize}
\item {Proveniência:(Lat. \textunderscore quota\textunderscore )}
\end{itemize}
Determinada porção; quinhão; prestação.
Quantia, com que cada indivíduo de um grupo contribue para certo fim.
Citação, nota, referência, á margem de um livro.
Letra, com que se classificam as peças de um processo.
Differença de nivel, entre qualquer ponto e aquelle que se toma para origem.--Há cotas \textunderscore pretas\textunderscore , quando se referem ao terreno, sendo o respectivo número indicado a tinta preta; há cotas \textunderscore vermelhas\textunderscore , quando se referem ao projecto, sendo o respectivo número escrito a tinta encarnada; e há cotas \textunderscore azues\textunderscore , quando se referem aos pontos, em que o projecto tem o nivel do terreno.
Diz-se da parte proporcional, com que cada um de vários indivíduos tem de contribuir com elles para determinado fim: \textunderscore já paguei a minha cota parte\textunderscore .
\section{Cota}
\begin{itemize}
\item {Grp. gram.:f.}
\end{itemize}
Antiga medida da Índia portuguesa.
\section{Cota}
\begin{itemize}
\item {Grp. gram.:f.}
\end{itemize}
O lado opposto ao gume de uma ferramenta.
\section{Cota}
\begin{itemize}
\item {Grp. gram.:f.}
\end{itemize}
Peixe cartilaginoso e variegado dos mares do Sul.
\section{Cotação}
\begin{itemize}
\item {Grp. gram.:f.}
\end{itemize}
\begin{itemize}
\item {Utilização:Fig.}
\end{itemize}
Acto ou effeito de cotar.
Determinação dos preços das mercadorias, títulos, acções de Bancos ou fundos públicos nas Bôlsas ou praças de commércio.
Exposição desses preços.
Conceito, aprêço: \textunderscore tem cotação fraca aquelle sujeito\textunderscore .
\section{Cotada}
\begin{itemize}
\item {Grp. gram.:f.}
\end{itemize}
\begin{itemize}
\item {Utilização:Prov.}
\end{itemize}
\begin{itemize}
\item {Utilização:trasm.}
\end{itemize}
\begin{itemize}
\item {Proveniência:(De \textunderscore cota\textunderscore ^4)}
\end{itemize}
Pancada com a cota de uma ferramenta.
\section{Cotado}
\begin{itemize}
\item {Grp. gram.:adj.}
\end{itemize}
\begin{itemize}
\item {Proveniência:(De \textunderscore cotar\textunderscore )}
\end{itemize}
Avaliado; apreciado: \textunderscore é um livro bem cotado\textunderscore .
\section{Cotador}
\begin{itemize}
\item {Grp. gram.:m.}
\end{itemize}
\begin{itemize}
\item {Proveniência:(De \textunderscore cotar\textunderscore )}
\end{itemize}
Aquelle que põe cotas.
\section{Cotamento}
\begin{itemize}
\item {Grp. gram.:m.}
\end{itemize}
Acto de cotar (autos).
\section{Cotangente}
\begin{itemize}
\item {Grp. gram.:f.}
\end{itemize}
\begin{itemize}
\item {Utilização:Geom.}
\end{itemize}
\begin{itemize}
\item {Proveniência:(De \textunderscore com...\textunderscore  + \textunderscore tangente\textunderscore )}
\end{itemize}
Tangente do complemento de um ângulo.
\section{Cotanilho}
\begin{itemize}
\item {Grp. gram.:m.}
\end{itemize}
\begin{itemize}
\item {Proveniência:(De \textunderscore cotão\textunderscore )}
\end{itemize}
Fios microscópicos, que se criam em alguns vegetaes.
\section{Cotanilhoso}
\begin{itemize}
\item {Grp. gram.:adj.}
\end{itemize}
Que tem lanugem ou cotanilho.
\section{Cotanoso}
\begin{itemize}
\item {Grp. gram.:adj.}
\end{itemize}
O mesmo que \textunderscore cotanilhoso\textunderscore .
\section{Cotão}
\begin{itemize}
\item {Grp. gram.:m.}
\end{itemize}
Lanugem de alguns frutos.
Pêlo, que se separa do pano, pelo uso ou pelo attrito.
Cisco, particulas, que se juntam ao fato, ao chão, ás paredes, aos móveis, em que não há limpeza,
(Ár. \textunderscore coton\textunderscore )
\section{Cotar}
\begin{itemize}
\item {Grp. gram.:v. t.}
\end{itemize}
\begin{itemize}
\item {Proveniência:(De \textunderscore cota\textunderscore ^2)}
\end{itemize}
Pôr cota em.
Fixar a taxa de.
Indicar o nível de.
\section{Cotarnina}
\begin{itemize}
\item {Grp. gram.:f.}
\end{itemize}
Base chímica, producto do desdobramento da narcotina.
\section{Cote}
\begin{itemize}
\item {Grp. gram.:m. Loc. adv.}
\end{itemize}
De \textunderscore cote\textunderscore , ou \textunderscore a cote\textunderscore , quotidianamente. Cf. Castilho, \textunderscore Fastos\textunderscore , II, 503.
\section{Cote}
\begin{itemize}
\item {Grp. gram.:m.}
\end{itemize}
\begin{itemize}
\item {Proveniência:(Do lat. \textunderscore cos\textunderscore , \textunderscore cotis\textunderscore )}
\end{itemize}
Pedra de afiar, de amolar.
\section{Cote}
\begin{itemize}
\item {Grp. gram.:m.}
\end{itemize}
\begin{itemize}
\item {Utilização:Náut.}
\end{itemize}
Nó falso, que se dá num cabo de embarcação.
Tortuosidade de um mastro.
\section{Çoteia}
\begin{itemize}
\item {Grp. gram.:f.}
\end{itemize}
\begin{itemize}
\item {Utilização:Prov.}
\end{itemize}
\begin{itemize}
\item {Utilização:alg.}
\end{itemize}
\begin{itemize}
\item {Grp. gram.:f.}
\end{itemize}
Eirado ou terrado, em substituição do telhado.
O mesmo que \textunderscore assoteia\textunderscore .
(Cp. \textunderscore sótão\textunderscore )
(Melhor escrita que a usual, \textunderscore soteia\textunderscore . V. \textunderscore soteia\textunderscore )
\section{Coteira}
\begin{itemize}
\item {Grp. gram.:f.}
\end{itemize}
\begin{itemize}
\item {Utilização:Ant.}
\end{itemize}
Pequena pipa.
\section{Coteiro}
\begin{itemize}
\item {Grp. gram.:m.}
\end{itemize}
\begin{itemize}
\item {Utilização:Prov.}
\end{itemize}
Montículo de terra ou de areia.
\section{Cotejador}
\begin{itemize}
\item {Grp. gram.:m.}
\end{itemize}
Aquelle que coteja.
\section{Cotejar}
\begin{itemize}
\item {Grp. gram.:v. t.}
\end{itemize}
\begin{itemize}
\item {Proveniência:(De \textunderscore cota\textunderscore ^2)}
\end{itemize}
Examinar cotas, confrontando-as.
Confrontar, comparar: \textunderscore cotejar dois autores\textunderscore .
\section{Cotejo}
\begin{itemize}
\item {Grp. gram.:m.}
\end{itemize}
Acto de cotejar.
\section{Cotete}
\begin{itemize}
\item {fónica:tê}
\end{itemize}
\begin{itemize}
\item {Grp. gram.:m.}
\end{itemize}
\begin{itemize}
\item {Proveniência:(De \textunderscore côto\textunderscore )}
\end{itemize}
Ave palmípede, cujas asas são simples cotos, e que é também conhecida por \textunderscore sotilicário\textunderscore .
\section{Coteto}
\begin{itemize}
\item {fónica:tê}
\end{itemize}
\begin{itemize}
\item {Grp. gram.:m.}
\end{itemize}
\begin{itemize}
\item {Utilização:Pop.}
\end{itemize}
\begin{itemize}
\item {Proveniência:(De \textunderscore côto\textunderscore )}
\end{itemize}
Homem muito baixo.
\section{Cothurnado}
\begin{itemize}
\item {Grp. gram.:adj.}
\end{itemize}
\begin{itemize}
\item {Proveniência:(Lat. \textunderscore cothurnatus\textunderscore )}
\end{itemize}
Que tem cothurnos.
Que tem fórma de cothurno; acothurnado.
\section{Cothurno}
\begin{itemize}
\item {Grp. gram.:m.}
\end{itemize}
\begin{itemize}
\item {Utilização:Prov.}
\end{itemize}
\begin{itemize}
\item {Utilização:minh.}
\end{itemize}
\begin{itemize}
\item {Proveniência:(Lat. \textunderscore cothurnus\textunderscore )}
\end{itemize}
Antigo borzeguim.
Meia curta, peúga.
Meia sem pé, que cobre a perna, desde o joêlho ao artelho.
\section{Cotia}
\begin{itemize}
\item {Grp. gram.:f.}
\end{itemize}
Antiga embarcação oriental.
\section{Cotia}
Casta de figueira algarvía.
\section{Cotiado}
\begin{itemize}
\item {Grp. gram.:adj.}
\end{itemize}
\begin{itemize}
\item {Proveniência:(De \textunderscore cotiar\textunderscore )}
\end{itemize}
Usado a cote.
Roçado, deteriorado pelo uso, (falando-se do fato ou peça de vestuário).
\section{Cotiar}
\begin{itemize}
\item {Grp. gram.:v. t.}
\end{itemize}
\begin{itemize}
\item {Proveniência:(De \textunderscore cotio\textunderscore )}
\end{itemize}
Usar a cote, todos os dias.
Gastar com o uso, pôr no fio, (falando-se de fato).
\section{Cotiar}
\begin{itemize}
\item {Grp. gram.:v. i.}
\end{itemize}
\begin{itemize}
\item {Utilização:Prov.}
\end{itemize}
\begin{itemize}
\item {Utilização:minh.}
\end{itemize}
Fazer escárneo; motejar. (Colhido na Póvoa de Varzim)
\section{Cotiara}
\begin{itemize}
\item {Grp. gram.:f.}
\end{itemize}
\begin{itemize}
\item {Utilização:Bras}
\end{itemize}
Nome vulgar do urutu.
\section{Cotica}
\begin{itemize}
\item {Grp. gram.:f.}
\end{itemize}
\begin{itemize}
\item {Utilização:Heráld.}
\end{itemize}
\begin{itemize}
\item {Proveniência:(Do b. lat. \textunderscore coticium\textunderscore )}
\end{itemize}
Peça estreita que atravessa o escudo nos brasões.
\section{Coticado}
\begin{itemize}
\item {Grp. gram.:adj.}
\end{itemize}
Que tem cotica.
\section{Cotícula}
\begin{itemize}
\item {Grp. gram.:f.}
\end{itemize}
\begin{itemize}
\item {Proveniência:(Lat. \textunderscore coticula\textunderscore )}
\end{itemize}
Pedra de toque do oiro e da prata.
\section{Cótila}
\begin{itemize}
\item {Grp. gram.:f.}
\end{itemize}
O mesmo que \textunderscore cótilo\textunderscore .
\section{Cotiledonário}
\begin{itemize}
\item {Grp. gram.:adj.}
\end{itemize}
Relativo aos cotilédones.
\section{Cotilédone}
\begin{itemize}
\item {Grp. gram.:m.}
\end{itemize}
\begin{itemize}
\item {Proveniência:(Do gr. \textunderscore kotuledon\textunderscore )}
\end{itemize}
Apêndice carnoso do embrião dos vegetaes, que têm aparentes os órgãos sexuaes.
Planta crassulácea.
\section{Cotiledóneas}
\begin{itemize}
\item {Grp. gram.:f. pl.}
\end{itemize}
\begin{itemize}
\item {Proveniência:(De \textunderscore cotiledóneo\textunderscore )}
\end{itemize}
Família de plantas, caracterizadas por terem um ou dois colitédones.
\section{Cotiledóneos}
\begin{itemize}
\item {Grp. gram.:adj.}
\end{itemize}
Que tem cotilédones.
\section{Cotiléforo}
\begin{itemize}
\item {Grp. gram.:adj.}
\end{itemize}
\begin{itemize}
\item {Proveniência:(Do gr. \textunderscore kotule\textunderscore  + \textunderscore phoros\textunderscore )}
\end{itemize}
Que tem cótilos.
\section{Cotilhão}
\begin{itemize}
\item {Grp. gram.:m.}
\end{itemize}
\begin{itemize}
\item {Proveniência:(Fr. \textunderscore cotillon\textunderscore )}
\end{itemize}
Espécie de dança, com que se terminam alguns bailes, e em que se intercalam diversões mimicas e facetas.
\section{Cótilo}
\begin{itemize}
\item {Grp. gram.:m.}
\end{itemize}
\begin{itemize}
\item {Utilização:Anat.}
\end{itemize}
\begin{itemize}
\item {Proveniência:(Do gr. \textunderscore kotule\textunderscore )}
\end{itemize}
Cavidade de um ôsso, na qual se articula a extremidade de outro.
\section{Cotiloide}
\begin{itemize}
\item {Grp. gram.:adj.}
\end{itemize}
\begin{itemize}
\item {Utilização:Anat.}
\end{itemize}
\begin{itemize}
\item {Proveniência:(Do gr. \textunderscore kotule\textunderscore  + \textunderscore eidos\textunderscore )}
\end{itemize}
Diz-se da cavidade óssea, em que se articula a cabeça do fêmur.
Que tem fórma de escudela.
\section{Cotiloídeo}
\begin{itemize}
\item {Grp. gram.:adj.}
\end{itemize}
O mesmo que \textunderscore cotiloide\textunderscore .
\section{Cotim}
\begin{itemize}
\item {Grp. gram.:m.}
\end{itemize}
\begin{itemize}
\item {Proveniência:(Fr. \textunderscore coutil\textunderscore ?)}
\end{itemize}
Espécie de tecido de linho ou de algodão.
\section{Cotinga}
\begin{itemize}
\item {Grp. gram.:f.}
\end{itemize}
Ave insectívora, de côres vivas.
\section{Cotio}
\begin{itemize}
\item {Grp. gram.:m.}
\end{itemize}
\begin{itemize}
\item {Grp. gram.:Loc. adv.}
\end{itemize}
\begin{itemize}
\item {Proveniência:(Do lat. \textunderscore quotidie\textunderscore )}
\end{itemize}
Uso quotidiano.
\textunderscore A cotio\textunderscore , o mesmo que \textunderscore a cote\textunderscore .
\section{Cotio}
\begin{itemize}
\item {Grp. gram.:m.}
\end{itemize}
Variedade de figo branco e grande.
\section{Cotitiribá}
\begin{itemize}
\item {Grp. gram.:m.}
\end{itemize}
\begin{itemize}
\item {Utilização:Bras}
\end{itemize}
Árvore fructífera dos sertões.
\section{Cotito}
\begin{itemize}
\item {Grp. gram.:adj.}
\end{itemize}
\begin{itemize}
\item {Utilização:ant.}
\end{itemize}
\begin{itemize}
\item {Utilização:Fam.}
\end{itemize}
\begin{itemize}
\item {Proveniência:(De \textunderscore côto\textunderscore )}
\end{itemize}
Curto e mal feito.
Mal acabado.
\section{Cotização}
\begin{itemize}
\item {Grp. gram.:f.}
\end{itemize}
Acto de cotizar.
\section{Cotizar}
\begin{itemize}
\item {Grp. gram.:v. t.}
\end{itemize}
O mesmo ou melhor que \textunderscore quotizar\textunderscore .
\section{Cotizável}
\begin{itemize}
\item {Grp. gram.:adj.}
\end{itemize}
Que se póde cotizar.
\section{Coto}
\begin{itemize}
\item {Grp. gram.:m.}
\end{itemize}
\begin{itemize}
\item {Proveniência:(T. jap.)}
\end{itemize}
Espécie de saltério, usado pelos Japoneses.
\section{Côto}
\begin{itemize}
\item {Grp. gram.:m.}
\end{itemize}
\begin{itemize}
\item {Utilização:Ant.}
\end{itemize}
\begin{itemize}
\item {Grp. gram.:Pl.}
\end{itemize}
\begin{itemize}
\item {Grp. gram.:Adj.}
\end{itemize}
\begin{itemize}
\item {Utilização:Prov.}
\end{itemize}
\begin{itemize}
\item {Utilização:beir.}
\end{itemize}
\begin{itemize}
\item {Proveniência:(Do lat. \textunderscore cubitus\textunderscore )}
\end{itemize}
Parte, que fica, de um braço, depois de amputada outra parte.
Resto de uma vela, archote, etc.
Parte das asas, em que se embebem as pennas.
Espécie de lima, com que os serradores afiam as serras, e cuja base ou vista lateral tem a figura de um losango.
Rabicho curto e grosso, usado por janotas. Cf. Filinto, V, 132.
Nós dos dedos das mãos.
Que não tem cauda, ou de cuja cauda resta uma pequena parte: \textunderscore morreu-me uma ovelha côta\textunderscore . (Colhido em Arganil)
\section{Côto}
\begin{itemize}
\item {Grp. gram.:m.}
\end{itemize}
\begin{itemize}
\item {Utilização:Prov.}
\end{itemize}
\begin{itemize}
\item {Utilização:alent.}
\end{itemize}
Série de três ou cinco partidas ao bilhar, ganhando o parceiro que primeiro ganhar duas no \textunderscore côto\textunderscore  de três, ou três no \textunderscore côto\textunderscore  de cinco.
(Cast. \textunderscore coto\textunderscore , marco, limite)
\section{Cotó}
\begin{itemize}
\item {Grp. gram.:m.}
\end{itemize}
\begin{itemize}
\item {Utilização:Des.}
\end{itemize}
\begin{itemize}
\item {Proveniência:(Fr. \textunderscore couteau\textunderscore )}
\end{itemize}
Grande faca; cutello:«\textunderscore As espadas largas degeneravam em cotós\textunderscore ». Bernárdez, \textunderscore N. Floresta\textunderscore .
\section{Cotó}
\begin{itemize}
\item {Grp. gram.:m.}
\end{itemize}
\begin{itemize}
\item {Utilização:Bras}
\end{itemize}
Homem aleijado.
Rapaz de pequena estatura.
Namorado sem ventura ou preterido.
(Relaciona-se com \textunderscore côto\textunderscore ^1?)
\section{Cotoco}
\begin{itemize}
\item {fónica:tô}
\end{itemize}
\begin{itemize}
\item {Grp. gram.:m.}
\end{itemize}
\begin{itemize}
\item {Utilização:Bras. do N}
\end{itemize}
O mesmo que \textunderscore tôco\textunderscore  ou \textunderscore côto\textunderscore  (de vela).
Pedaço de faca.
\section{Cotó-cotó}
\begin{itemize}
\item {Grp. gram.:m.}
\end{itemize}
Planta rubiácea do Brasil.
\section{Cotoína}
\begin{itemize}
\item {Grp. gram.:f.}
\end{itemize}
Medicamento contra a diarreia rebelde.
\section{Cotonária}
\begin{itemize}
\item {Grp. gram.:f.}
\end{itemize}
\begin{itemize}
\item {Proveniência:(De \textunderscore cotão\textunderscore )}
\end{itemize}
Planta, cujas fôlhas tem o aspecto e a maciez do algodão.
\section{Cotoneira}
\begin{itemize}
\item {Grp. gram.:f.}
\end{itemize}
(V.cotonária)
\section{Cotonia}
\begin{itemize}
\item {Grp. gram.:f.}
\end{itemize}
\begin{itemize}
\item {Utilização:Des.}
\end{itemize}
Pano de algodão.
(Do ár.)
\section{Cotonígero}
\begin{itemize}
\item {Grp. gram.:adj.}
\end{itemize}
\begin{itemize}
\item {Utilização:Bot.}
\end{itemize}
Revestido de lanugem ou de pêlos semelhantes a cotão.
\section{Cotonoso}
\begin{itemize}
\item {Grp. gram.:adj.}
\end{itemize}
Cotonígero.
Feito de algodão.
Que tem algodão.
\section{Cotovelada}
\begin{itemize}
\item {Grp. gram.:f.}
\end{itemize}
Pancada com o cotovelo.
\section{Cotovelão}
\begin{itemize}
\item {Grp. gram.:m.}
\end{itemize}
Pancada com o cotovelo. Cf. Arn. Gama, \textunderscore Motim\textunderscore , 959 e 376; Guilh. Braga, \textunderscore Heras\textunderscore , 247; Bern. Machado, \textunderscore Notas de Um Pai\textunderscore .
\section{Cotovelar}
\textunderscore v. t.\textunderscore  (e der.)
O mesmo que \textunderscore acotovelar\textunderscore , etc.
\section{Cotovelo}
\begin{itemize}
\item {fónica:vê}
\end{itemize}
\begin{itemize}
\item {Grp. gram.:m.}
\end{itemize}
\begin{itemize}
\item {Proveniência:(Do lat. hyp. \textunderscore cubitelus\textunderscore )}
\end{itemize}
Ângulo saliente na articulação do braço com o ante-braço.
Canto; esquina.
Nó de videira.
Parte recurva da baioneta em que a fôlha se liga ao alvado.
Curva de 90 graus e raio curto nas tubagens.
\textunderscore Falar pelos cotovelos\textunderscore , falar muito e com desembaraço.
Enseada, ou recanto de abrigo, em baía ou rio.
\section{Cotovelosa}
\begin{itemize}
\item {Grp. gram.:f.}
\end{itemize}
\begin{itemize}
\item {Proveniência:(De \textunderscore cotovêlo\textunderscore )}
\end{itemize}
O mesmo que \textunderscore pêra-de-sete-cotovelos\textunderscore .
\section{Cotovia}
\begin{itemize}
\item {Grp. gram.:f.}
\end{itemize}
\begin{itemize}
\item {Utilização:Gír.}
\end{itemize}
Pequena e cinzenta ave campestre, que canta de madrugada e é uma espécie de calhandra.
Garrafa.
\section{Cotovia-galucha}
\begin{itemize}
\item {Grp. gram.:f.}
\end{itemize}
O mesmo que \textunderscore carreirola\textunderscore .
\section{Cotra}
\begin{itemize}
\item {fónica:cô}
\end{itemize}
\begin{itemize}
\item {Grp. gram.:f.}
\end{itemize}
\begin{itemize}
\item {Utilização:Prov.}
\end{itemize}
\begin{itemize}
\item {Utilização:trasm.}
\end{itemize}
Crosta de immundície, que se fórma no fato dos sardinheiros e nas mangas da véstia de crianças ranhosas.
(Corr. de \textunderscore crosta\textunderscore )
\section{Cotrala}
\begin{itemize}
\item {Grp. gram.:f.}
\end{itemize}
\begin{itemize}
\item {Utilização:Prov.}
\end{itemize}
\begin{itemize}
\item {Utilização:Trasm.}
\end{itemize}
O mesmo que \textunderscore bélfa\textunderscore  ou \textunderscore molhelha\textunderscore .
\section{Cotreia}
\begin{itemize}
\item {Grp. gram.:f.}
\end{itemize}
\begin{itemize}
\item {Utilização:Bras. do N}
\end{itemize}
O mesmo que \textunderscore cachaça\textunderscore .
\section{Cotrim}
\begin{itemize}
\item {Grp. gram.:m.}
\end{itemize}
Antiga moéda portuguesa.
Peixe de Portugal.
\section{Cotrofe}
\begin{itemize}
\item {Grp. gram.:m.}
\end{itemize}
\begin{itemize}
\item {Utilização:Prov.}
\end{itemize}
\begin{itemize}
\item {Utilização:trasm.}
\end{itemize}
O mesmo que \textunderscore catrofa\textunderscore .
\section{Cotroso}
\begin{itemize}
\item {Grp. gram.:adj.}
\end{itemize}
\begin{itemize}
\item {Utilização:Prov.}
\end{itemize}
\begin{itemize}
\item {Utilização:trasm.}
\end{itemize}
Diz-se do indivíduo que tem cotra no fato.
\section{Cotruco}
\begin{itemize}
\item {Grp. gram.:m.}
\end{itemize}
\begin{itemize}
\item {Utilização:Bras. do N}
\end{itemize}
Vendedor ambulante de fazendas e objectos de armarinho.
\section{Cotubana}
\begin{itemize}
\item {Grp. gram.:f.}
\end{itemize}
Aforamento perpétuo, na Índia portuguesa.
\section{Cótula}
\begin{itemize}
\item {Grp. gram.:f.}
\end{itemize}
\begin{itemize}
\item {Proveniência:(Lat. \textunderscore cotula\textunderscore )}
\end{itemize}
Gênero de plantas herbáceas.
\section{Cotulo}
\textunderscore m.\textunderscore  (e der.)
O mesmo que \textunderscore cogulo\textunderscore , etc.
\section{Coturnado}
\begin{itemize}
\item {Grp. gram.:adj.}
\end{itemize}
\begin{itemize}
\item {Proveniência:(Lat. \textunderscore cothurnatus\textunderscore )}
\end{itemize}
Que tem coturnos.
Que tem fórma de coturno; acoturnado.
\section{Coturno}
\begin{itemize}
\item {Grp. gram.:m.}
\end{itemize}
\begin{itemize}
\item {Utilização:Prov.}
\end{itemize}
\begin{itemize}
\item {Utilização:minh.}
\end{itemize}
\begin{itemize}
\item {Proveniência:(Lat. \textunderscore cothurnus\textunderscore )}
\end{itemize}
Antigo borzeguim.
Meia curta, peúga.
Meia sem pé, que cobre a perna, desde o joêlho ao artelho.
\section{Co-tutor}
\begin{itemize}
\item {Grp. gram.:m.}
\end{itemize}
\begin{itemize}
\item {Proveniência:(De \textunderscore com...\textunderscore  + \textunderscore tutor\textunderscore )}
\end{itemize}
Aquelle que é tutor com outrem.
\section{Cótyla}
\begin{itemize}
\item {Grp. gram.:f.}
\end{itemize}
O mesmo que \textunderscore cótylo\textunderscore .
\section{Cotyledonário}
\begin{itemize}
\item {Grp. gram.:adj.}
\end{itemize}
Relativo aos cotylédones.
\section{Cotylédone}
\begin{itemize}
\item {Grp. gram.:m.}
\end{itemize}
\begin{itemize}
\item {Proveniência:(Do gr. \textunderscore kotuledon\textunderscore )}
\end{itemize}
Appêndice carnoso do embryão dos vegetaes, que têm apparentes os órgãos sexuaes.
Planta crassulácea.
\section{Cotyledóneas}
\begin{itemize}
\item {Grp. gram.:f. pl.}
\end{itemize}
\begin{itemize}
\item {Proveniência:(De \textunderscore cotiledóneo\textunderscore )}
\end{itemize}
Família de plantas, caracterizadas por terem um ou dois colytédones.
\section{Cotyledóneo}
\begin{itemize}
\item {Grp. gram.:adj.}
\end{itemize}
Que tem cotylédones.
\section{Cotyléforo}
\begin{itemize}
\item {Grp. gram.:adj.}
\end{itemize}
\begin{itemize}
\item {Proveniência:(Do gr. \textunderscore kotule\textunderscore  + \textunderscore phoros\textunderscore )}
\end{itemize}
Que tem cótylos.
\section{Cótylo}
\begin{itemize}
\item {Grp. gram.:m.}
\end{itemize}
\begin{itemize}
\item {Utilização:Anat.}
\end{itemize}
\begin{itemize}
\item {Proveniência:(Do gr. \textunderscore kotule\textunderscore )}
\end{itemize}
Cavidade de um ôsso, na qual se articula a extremidade de outro.
\section{Cotyloide}
\begin{itemize}
\item {Grp. gram.:adj.}
\end{itemize}
\begin{itemize}
\item {Utilização:Anat.}
\end{itemize}
\begin{itemize}
\item {Proveniência:(Do gr. \textunderscore kotule\textunderscore  + \textunderscore eidos\textunderscore )}
\end{itemize}
Diz-se da cavidade óssea, em que se articula a cabeça do fêmur.
Que tem fórma de escudela.
\section{Cotyloídio}
\begin{itemize}
\item {Grp. gram.:adj.}
\end{itemize}
O mesmo que \textunderscore cotyloide\textunderscore .
\section{Couça}
\begin{itemize}
\item {Grp. gram.:f.}
\end{itemize}
\begin{itemize}
\item {Utilização:T. de Braga}
\end{itemize}
Lagarto grande, que destrói as abelhas.
\section{Coucão}
\begin{itemize}
\item {Grp. gram.:m.}
\end{itemize}
O mesmo que \textunderscore cocão\textunderscore ^1.
\section{Coução}
\begin{itemize}
\item {Grp. gram.:m.}
\end{itemize}
\begin{itemize}
\item {Utilização:Prov.}
\end{itemize}
\begin{itemize}
\item {Proveniência:(De \textunderscore coice\textunderscore )}
\end{itemize}
O mesmo que \textunderscore couceira\textunderscore ; couceira grande.
\section{Couçar}
\begin{itemize}
\item {Grp. gram.:v. t.}
\end{itemize}
\begin{itemize}
\item {Utilização:Prov.}
\end{itemize}
\begin{itemize}
\item {Utilização:minh.}
\end{itemize}
O mesmo que \textunderscore calçar\textunderscore .
Aconchegar.
\section{Couce}
\begin{itemize}
\item {Grp. gram.:m.}
\end{itemize}
\begin{itemize}
\item {Utilização:pop.}
\end{itemize}
\begin{itemize}
\item {Utilização:Fig.}
\end{itemize}
\begin{itemize}
\item {Proveniência:(Lat. \textunderscore calx\textunderscore )}
\end{itemize}
Rètaguarda, traseira, parte posterior de alguma coisa.
Dente da rabiça.
Calcanhar.
Pancada com o calcanhar, com o pé, com a pata.
Coiceira.
Brutalidade.
Ingratidão.
\section{Coucear}
\begin{itemize}
\item {Grp. gram.:v. t.  e  i.}
\end{itemize}
Dar couces.
\section{Couceira}
\begin{itemize}
\item {Grp. gram.:f.}
\end{itemize}
\begin{itemize}
\item {Proveniência:(De \textunderscore coice\textunderscore )}
\end{itemize}
Couce da porta.
Parte da porta, em que se pregam os gonzos ou dobradiças.
Soleira da porta.
Variedade de uva preta da região do Doiro.
\section{Couceiro}
\begin{itemize}
\item {Grp. gram.:m.}
\end{itemize}
\begin{itemize}
\item {Utilização:T. da Nazaré}
\end{itemize}
\begin{itemize}
\item {Grp. gram.:Adj.}
\end{itemize}
\begin{itemize}
\item {Utilização:Bras}
\end{itemize}
\begin{itemize}
\item {Proveniência:(De \textunderscore coice\textunderscore )}
\end{itemize}
Um dos homens que levantam a rede, e que trabalha atrás dos outros.
Que costuma dar couces.
\section{Coucelo}
\begin{itemize}
\item {Grp. gram.:m.}
\end{itemize}
Planta crassulácea, o mesmo que \textunderscore conchelo\textunderscore , a que na Beira-Alta chamam \textunderscore couxilgo\textunderscore  e na Beira-Baixa \textunderscore coussilho\textunderscore .--Há desta Planta duas espécies pelo menos: uma, em fórma de pequena umbella; outra, que se desenvolve em fórma de grãos de arroz e que se chama \textunderscore arroz dos telhados\textunderscore . Cf. \textunderscore Desengano da Med.\textunderscore , 121.
(Cp. \textunderscore conchelo\textunderscore )
\section{Coucés}
\begin{itemize}
\item {Grp. gram.:m.}
\end{itemize}
Medida itinerária da antiga Índia portuguesa.
\section{Coucil}
\begin{itemize}
\item {Grp. gram.:m.}
\end{itemize}
\begin{itemize}
\item {Utilização:Prov.}
\end{itemize}
\begin{itemize}
\item {Utilização:trasm.}
\end{itemize}
\begin{itemize}
\item {Proveniência:(De \textunderscore coice\textunderscore )}
\end{itemize}
Espigão de madeira na coiceira das portas, o qual gira sôbre o lado côncavo de um fundo de garrafa ou sôbre um tacão de sapato velho.
\section{Coucilhão}
\begin{itemize}
\item {Grp. gram.:m.}
\end{itemize}
\begin{itemize}
\item {Utilização:Prov.}
\end{itemize}
\begin{itemize}
\item {Utilização:trasm.}
\end{itemize}
\begin{itemize}
\item {Proveniência:(De \textunderscore coice\textunderscore )}
\end{itemize}
Peça, em que se embebem as entriteiras do carro.
\section{Coucilho}
\begin{itemize}
\item {Grp. gram.:m.}
\end{itemize}
\begin{itemize}
\item {Utilização:Prov.}
\end{itemize}
\begin{itemize}
\item {Utilização:trasm.}
\end{itemize}
O mesmo que \textunderscore coucil\textunderscore .
\section{Couco}
\begin{itemize}
\item {Grp. gram.:m.}
\end{itemize}
\begin{itemize}
\item {Utilização:Bras}
\end{itemize}
Árvore silvestre, cuja madeira se emprega em obras de carpintaria.
\section{Couçoeira}
\begin{itemize}
\item {Grp. gram.:f.}
\end{itemize}
\begin{itemize}
\item {Utilização:Prov.}
\end{itemize}
\begin{itemize}
\item {Utilização:Bras}
\end{itemize}
\begin{itemize}
\item {Utilização:Prov.}
\end{itemize}
O mesmo que \textunderscore couceira\textunderscore .
Pessôa estúpida.
\section{Coudel}
\begin{itemize}
\item {Grp. gram.:m.}
\end{itemize}
\begin{itemize}
\item {Proveniência:(Do lat. \textunderscore capitellum\textunderscore )}
\end{itemize}
Antigo capitão de cavallaria.
\section{Coudelaria}
\begin{itemize}
\item {Grp. gram.:f.}
\end{itemize}
\begin{itemize}
\item {Proveniência:(De \textunderscore coudel\textunderscore )}
\end{itemize}
Cargo de coudel.
Casa, estabelecimento, em que se trata do aperfeiçoamento das raças cavallares, velando-se pela respectiva e conveniente procriação.
\section{Coudélico}
\begin{itemize}
\item {Grp. gram.:adj.}
\end{itemize}
\begin{itemize}
\item {Proveniência:(De \textunderscore coudel\textunderscore )}
\end{itemize}
Relativo a coudelarias.
\section{Coudilho}
\begin{itemize}
\item {Grp. gram.:m.}
\end{itemize}
\begin{itemize}
\item {Utilização:Prov.}
\end{itemize}
\begin{itemize}
\item {Utilização:trasm.}
\end{itemize}
Linhol dos sapateiros.
\section{Coudra}
\begin{itemize}
\item {Grp. gram.:f.}
\end{itemize}
\begin{itemize}
\item {Utilização:Ant.}
\end{itemize}
O mesmo que \textunderscore cócedra\textunderscore .
\section{Coumarina}
\begin{itemize}
\item {Grp. gram.:f.}
\end{itemize}
Planta leguminosa, medicinal, (\textunderscore dipterix tetraphylla\textunderscore ), também chamada \textunderscore fava tonca\textunderscore .
\section{Coumarourama}
\begin{itemize}
\item {Grp. gram.:f.}
\end{itemize}
Árvore leguminosa do Brasil.
\section{Couquilha}
\begin{itemize}
\item {Grp. gram.:f.}
\end{itemize}
\begin{itemize}
\item {Utilização:Prov.}
\end{itemize}
\begin{itemize}
\item {Utilização:minh.}
\end{itemize}
\begin{itemize}
\item {Proveniência:(De \textunderscore coucão\textunderscore )}
\end{itemize}
Peça de madeira, com que se remenda a parte superior dos coucões, quando êstes se gastam com o attrito do eixo.
\section{Couquilhada}
\begin{itemize}
\item {Grp. gram.:f.}
\end{itemize}
\begin{itemize}
\item {Utilização:Prov.}
\end{itemize}
\begin{itemize}
\item {Utilização:trasm.}
\end{itemize}
O mesmo que \textunderscore cotovia\textunderscore .
\section{Coura}
\begin{itemize}
\item {Grp. gram.:f.}
\end{itemize}
\begin{itemize}
\item {Utilização:Ant.}
\end{itemize}
Antigo gibão de couro, para guerreiros.
O mesmo que \textunderscore couraça\textunderscore .
\section{Couraça}
\begin{itemize}
\item {Grp. gram.:f.}
\end{itemize}
\begin{itemize}
\item {Utilização:Fig.}
\end{itemize}
\begin{itemize}
\item {Proveniência:(De \textunderscore coiro\textunderscore )}
\end{itemize}
Armadura para o peito.
Revestimento de navios com ferro ou outro metal.
Aquillo que serve de resguardo contra a maledicência ou contra a má sorte.
\section{Couraçado}
\begin{itemize}
\item {Grp. gram.:adj.}
\end{itemize}
\begin{itemize}
\item {Grp. gram.:M.}
\end{itemize}
Revestido de metal: \textunderscore navio couraçado\textunderscore .
Navio couraçado.
\section{Couraçar}
\begin{itemize}
\item {Grp. gram.:v. t.}
\end{itemize}
\begin{itemize}
\item {Proveniência:(De \textunderscore coiraça\textunderscore )}
\end{itemize}
Armar de couraça.
Revestir de aço ou de outro metal (navios).
Abroquelar, proteger.
Tornar impassível.
\section{Couraceiro}
\begin{itemize}
\item {Grp. gram.:m.}
\end{itemize}
Soldado, que tem couraça.
\section{Courama}
\begin{itemize}
\item {Grp. gram.:f.}
\end{itemize}
\begin{itemize}
\item {Utilização:Bras. do N}
\end{itemize}
\begin{itemize}
\item {Proveniência:(De \textunderscore coiro\textunderscore )}
\end{itemize}
Porção de couros.
Vestuário de couro, para vaqueiro.
\section{Courana}
\begin{itemize}
\item {Grp. gram.:m.}
\end{itemize}
Variedade de cestro do Brasil.
\section{Courão}
\begin{itemize}
\item {Grp. gram.:m.}
\end{itemize}
\begin{itemize}
\item {Utilização:Chul.}
\end{itemize}
\begin{itemize}
\item {Proveniência:(De \textunderscore coiro\textunderscore )}
\end{itemize}
Rameira velha.
Variedade de uva.
\section{Courato}
\begin{itemize}
\item {Grp. gram.:m.}
\end{itemize}
\begin{itemize}
\item {Utilização:Prov.}
\end{itemize}
\begin{itemize}
\item {Utilização:trasm.}
\end{itemize}
Couro de porco.
Pedaço de couro duro.
\section{Courear}
\begin{itemize}
\item {Grp. gram.:v. t.}
\end{itemize}
\begin{itemize}
\item {Utilização:Bras. do S}
\end{itemize}
Extrair o couro de (um animal).
\section{Coureiro}
\begin{itemize}
\item {Grp. gram.:m.}
\end{itemize}
Vendedor de couros; samarreiro.
\section{Courela}
\begin{itemize}
\item {Grp. gram.:f.}
\end{itemize}
\begin{itemize}
\item {Utilização:Ant.}
\end{itemize}
\begin{itemize}
\item {Utilização:Des.}
\end{itemize}
\begin{itemize}
\item {Proveniência:(Do b. lat. \textunderscore quarellus\textunderscore , contr. de \textunderscore quadrellus\textunderscore , do lat. \textunderscore quadrum\textunderscore )}
\end{itemize}
Porção de terra cultivável, longa e estreita.
Medida agrária, equivalente a 100 braças de comprimento e 10 de largura.
Casal.
\section{Coureleiro}
\begin{itemize}
\item {Grp. gram.:m.}
\end{itemize}
\begin{itemize}
\item {Proveniência:(De \textunderscore coirela\textunderscore )}
\end{itemize}
Aquelle que antigamente repartia as terras incultas, ou as conquistadas, pelos povoadores que iam arroteá-las ou habitá-las.
\section{Couro}
\begin{itemize}
\item {Grp. gram.:m.}
\end{itemize}
\begin{itemize}
\item {Utilização:Fig.}
\end{itemize}
\begin{itemize}
\item {Utilização:Prov.}
\end{itemize}
\begin{itemize}
\item {Utilização:beir.}
\end{itemize}
\begin{itemize}
\item {Proveniência:(Lat. \textunderscore corium\textunderscore )}
\end{itemize}
Pelle dura de alguns animaes.
Pelle da cabeça humana.
Pelle.
Rameira desprezível e de idade madura.
\textunderscore Estar em couro\textunderscore , estar nu.
\section{Courona}
\begin{itemize}
\item {Grp. gram.:f.}
\end{itemize}
Rameira. Cf. Macedo, \textunderscore Burrão\textunderscore , 5.
\section{Cousa}
\begin{itemize}
\item {Grp. gram.:f.}
\end{itemize}
\begin{itemize}
\item {Grp. gram.:Pl.}
\end{itemize}
\begin{itemize}
\item {Proveniência:(Do lat. \textunderscore causa\textunderscore )}
\end{itemize}
Qualquer objecto inanimado.
Aquillo que existe ou póde existir.
Realidade; facto: \textunderscore essa é que é a cousa\textunderscore .
Negócio: \textunderscore tenho uma cousa a tratar\textunderscore .
Acto.
Causa.
Espécie.
Mystério: \textunderscore aí há cousa\textunderscore .
Bens.
\section{Cousada}
\begin{itemize}
\item {Grp. gram.:f.}
\end{itemize}
\begin{itemize}
\item {Utilização:Prov.}
\end{itemize}
\begin{itemize}
\item {Utilização:burl.}
\end{itemize}
Cousa que se não quer declarar: \textunderscore é cá uma cousada\textunderscore . (Colhido em Turquel)
\section{Cousar}
\begin{itemize}
\item {Grp. gram.:v. i.}
\end{itemize}
\begin{itemize}
\item {Utilização:Prov.}
\end{itemize}
\begin{itemize}
\item {Utilização:trasm.}
\end{itemize}
Fazer alguma cousa.
\section{Couseiro}
\begin{itemize}
\item {Grp. gram.:m.}
\end{itemize}
\begin{itemize}
\item {Proveniência:(De \textunderscore coisa\textunderscore )}
\end{itemize}
Livro de notas e apontamentos, usado na Inquisição.
\section{Cousíssima}
\begin{itemize}
\item {Grp. gram.:f.}
\end{itemize}
\begin{itemize}
\item {Utilização:Fam.}
\end{itemize}
Us. na loc. \textunderscore cousíssima nenhuma\textunderscore , absolutamente nada.
\section{Couso}
\begin{itemize}
\item {Grp. gram.:m.}
\end{itemize}
\begin{itemize}
\item {Utilização:Chul.}
\end{itemize}
Qualquer sujeito; fulano: \textunderscore aquelle couso saiu-me um traste\textunderscore !
(Cp. \textunderscore coisa\textunderscore )
\section{Coussilho}
\begin{itemize}
\item {Grp. gram.:m.}
\end{itemize}
\begin{itemize}
\item {Utilização:Prov.}
\end{itemize}
\begin{itemize}
\item {Utilização:beir.}
\end{itemize}
O mesmo que \textunderscore coucelo\textunderscore .
\section{Coutada}
\begin{itemize}
\item {Grp. gram.:f.}
\end{itemize}
\begin{itemize}
\item {Proveniência:(De \textunderscore côito\textunderscore )}
\end{itemize}
Terra defesa; cerrado.
\section{Coutamento}
\begin{itemize}
\item {Grp. gram.:m.}
\end{itemize}
Acto de \textunderscore coutar\textunderscore . Cp. Herculano, \textunderscore Hist. de Port.\textunderscore , IV, 272, 274, 276 e 279.
\section{Coutar}
\begin{itemize}
\item {Grp. gram.:v. t.}
\end{itemize}
\begin{itemize}
\item {Utilização:Ant.}
\end{itemize}
\begin{itemize}
\item {Proveniência:(De \textunderscore côito\textunderscore )}
\end{itemize}
Tornar defesa (uma propriedade), prohibindo a entrada nella, ou dando-lhe certos privilégios.
O mesmo que \textunderscore acoitar\textunderscore .
\section{Coutar}
\begin{itemize}
\item {Grp. gram.:v. t.}
\end{itemize}
\begin{itemize}
\item {Utilização:Ant.}
\end{itemize}
\begin{itemize}
\item {Proveniência:(Do lat. \textunderscore captare\textunderscore )}
\end{itemize}
Aprehender, tirar:«\textunderscore ...as quaes callças e chapeo lhe coutara André Nogueira\textunderscore ». \textunderscore Alvará\textunderscore  de D. Sebast., in \textunderscore Rev. Lus.\textunderscore , XV, 217.
\section{Coutaria}
\begin{itemize}
\item {Grp. gram.:f.}
\end{itemize}
Offício de couteiro.
\section{Couteiro}
\begin{itemize}
\item {Grp. gram.:m.}
\end{itemize}
\begin{itemize}
\item {Proveniência:(De \textunderscore côito\textunderscore )}
\end{itemize}
Aquelle que guarda as coutadas, os côutos.
\section{Coutelho}
\begin{itemize}
\item {fónica:tê}
\end{itemize}
\begin{itemize}
\item {Grp. gram.:m.}
\end{itemize}
\begin{itemize}
\item {Proveniência:(De \textunderscore côito\textunderscore )}
\end{itemize}
Pomar cercado.
Quinchoso; cerrado; chouso.
\section{Co-utente}
\begin{itemize}
\item {Grp. gram.:adj.}
\end{itemize}
\begin{itemize}
\item {Utilização:Jur.}
\end{itemize}
Que usa ou frue, juntamente com outrem. Cf. Assis, \textunderscore Águas\textunderscore , 193 e 202.
\section{Coutio}
\begin{itemize}
\item {Grp. gram.:m.}
\end{itemize}
\begin{itemize}
\item {Utilização:Ant.}
\end{itemize}
O mesmo que \textunderscore couto\textunderscore ^2.
\section{Couto}
\begin{itemize}
\item {Grp. gram.:m.}
\end{itemize}
\begin{itemize}
\item {Proveniência:(Lat. \textunderscore cubitus\textunderscore . Cp. \textunderscore côto\textunderscore ^1)}
\end{itemize}
Medida antiga, talvez o mesmo que \textunderscore côvado\textunderscore .
\section{Couto}
\begin{itemize}
\item {Proveniência:(Do lat. \textunderscore cautus\textunderscore )}
\end{itemize}
\textunderscore m.\textunderscore  (e der.)
Terra coitada, defesa, privilegiada.
\section{Couval}
\begin{itemize}
\item {Grp. gram.:m.}
\end{itemize}
Terreno, em que crescem couves.
\section{Couvana}
\begin{itemize}
\item {Grp. gram.:f.}
\end{itemize}
\begin{itemize}
\item {Proveniência:(De \textunderscore couve\textunderscore )}
\end{itemize}
Planta da serra de Cintra.
\section{Couve}
\begin{itemize}
\item {Grp. gram.:f.}
\end{itemize}
\begin{itemize}
\item {Proveniência:(Do lat. \textunderscore caulis\textunderscore ?)}
\end{itemize}
Planta hortense, que abrange várias espécies.
Variedade de pêra de Penafiel.
\section{Couveflor}
\begin{itemize}
\item {Grp. gram.:f.}
\end{itemize}
Variedade de couve, (\textunderscore brassica oleracea botrytis\textunderscore ), caracterizada por seus pedúnculos floraes, que fórmam, antes de se abrirem, massa carnuda.
\section{Couveira}
\begin{itemize}
\item {Grp. gram.:f.}
\end{itemize}
\begin{itemize}
\item {Utilização:Prov.}
\end{itemize}
Pé de couve.
\section{Couveiro}
\begin{itemize}
\item {Grp. gram.:adj.}
\end{itemize}
\begin{itemize}
\item {Utilização:Pop.}
\end{itemize}
Em que se plantam couves.
Proprio para plantação de couves:«\textunderscore Fevereiro couveiro afaz a perdiz ao poleiro; Maio couveiro não é vinhateiro\textunderscore ». Prolóquios pop.
\section{Couxilgo}
\begin{itemize}
\item {Grp. gram.:m.}
\end{itemize}
\begin{itemize}
\item {Utilização:Prov.}
\end{itemize}
\begin{itemize}
\item {Utilização:beir.}
\end{itemize}
O mesmo que \textunderscore coucelo\textunderscore .
\section{Cova}
\begin{itemize}
\item {Grp. gram.:f.}
\end{itemize}
\begin{itemize}
\item {Grp. gram.:Loc.}
\end{itemize}
\begin{itemize}
\item {Utilização:fam.}
\end{itemize}
Abertura na terra.
Escavação.
Caverna.
Cavidade.
Depressão em qualquer superfície.
Alvêolo.
Sepultura.
\textunderscore Cova do ladrão\textunderscore , depressão entre o pescoço e a nuca. Cf. \textunderscore Hist. Insulana\textunderscore , II, 34.
\textunderscore Estar com os pés para a cova\textunderscore , estar em vésperas de morrer.
(B. lat. \textunderscore copha\textunderscore )
\section{Covacho}
\begin{itemize}
\item {Grp. gram.:m.}
\end{itemize}
Pequena cova.
\section{Covacova}
\begin{itemize}
\item {Grp. gram.:m.}
\end{itemize}
\begin{itemize}
\item {Utilização:Bras}
\end{itemize}
Passarinho cinzento, de cauda comprida, e cujo canto parece exprimir o seu nome. Cf. B. C. Rubim. \textunderscore Vocab. Bras.\textunderscore 
\section{Covada}
\begin{itemize}
\item {Grp. gram.:f.}
\end{itemize}
\begin{itemize}
\item {Utilização:Prov.}
\end{itemize}
\begin{itemize}
\item {Utilização:dur.}
\end{itemize}
\begin{itemize}
\item {Utilização:beir.}
\end{itemize}
\begin{itemize}
\item {Proveniência:(De \textunderscore cova\textunderscore )}
\end{itemize}
Depresão de terreno; pequena planície entre montes; valleiro.
\section{Côvado}
\begin{itemize}
\item {Grp. gram.:m.}
\end{itemize}
\begin{itemize}
\item {Utilização:Ant.}
\end{itemize}
\begin{itemize}
\item {Proveniência:(Lat. \textunderscore cubitus\textunderscore )}
\end{itemize}
Antiga medida de comprimento, equivalente a 66 centímetros.
Parte, onde a caverna de um navio começa a fazer volta para cima.
\section{Covagem}
\begin{itemize}
\item {Grp. gram.:f.}
\end{itemize}
\begin{itemize}
\item {Proveniência:(De \textunderscore cova\textunderscore )}
\end{itemize}
Acto de cavar sepultura.
Preço dêsse acto.
\section{Coval}
\begin{itemize}
\item {Grp. gram.:m.}
\end{itemize}
\begin{itemize}
\item {Utilização:Ant.}
\end{itemize}
\begin{itemize}
\item {Proveniência:(De \textunderscore cova\textunderscore )}
\end{itemize}
Divisão do terreno de um cemitério, na qual se podem abrir sepulturas.
Divisão de terra para sementeira.
\section{Covanca}
\begin{itemize}
\item {Grp. gram.:f.}
\end{itemize}
\begin{itemize}
\item {Utilização:Bras. do Rio}
\end{itemize}
\begin{itemize}
\item {Proveniência:(De \textunderscore cova\textunderscore )}
\end{itemize}
Terreno cercado de morros, com entrada natural por um lado.
\section{Covarde}
\textunderscore m.\textunderscore  (e der.)
O mesmo que \textunderscore cobarde\textunderscore , etc.
\section{Covardo}
\begin{itemize}
\item {Grp. gram.:adj.}
\end{itemize}
Outra fórma de \textunderscore cobarde\textunderscore . Cf. \textunderscore Eufrosina\textunderscore , 38 e 287.
\section{Covão}
\begin{itemize}
\item {Grp. gram.:m.}
\end{itemize}
Grande cova.
\section{Covata}
\begin{itemize}
\item {Grp. gram.:f.}
\end{itemize}
\begin{itemize}
\item {Utilização:Prov.}
\end{itemize}
O mesmo que \textunderscore covacho\textunderscore .
\section{Covato}
\begin{itemize}
\item {Grp. gram.:m.}
\end{itemize}
\begin{itemize}
\item {Utilização:Prov.}
\end{itemize}
\begin{itemize}
\item {Utilização:alent.}
\end{itemize}
\begin{itemize}
\item {Proveniência:(De \textunderscore cova\textunderscore )}
\end{itemize}
Offício de coveiro.
Lugar, em que se abrem sepulturas.
Buraco, no fundo da manta do bacêllo, em que êste se aconchega com terra para lançar raízes.
Folhagem sêca, que se queima, sotoposta a uma porção de terra, para adubo do solo.
\section{Côvedo}
\begin{itemize}
\item {Grp. gram.:m.}
\end{itemize}
\begin{itemize}
\item {Utilização:Ant.}
\end{itemize}
Cotovelo.
Joelheira de bêsta.
(Da mesma or. de \textunderscore côvado\textunderscore )
\section{Coveiro}
\begin{itemize}
\item {Grp. gram.:m.}
\end{itemize}
\begin{itemize}
\item {Proveniência:(De \textunderscore cova\textunderscore )}
\end{itemize}
Aquelle que abre covas para cadáveres.
\section{Còveiro}
\begin{itemize}
\item {Grp. gram.:m.}
\end{itemize}
\begin{itemize}
\item {Utilização:Prov.}
\end{itemize}
\begin{itemize}
\item {Utilização:alent.}
\end{itemize}
Cabana, junto á malhada, onde se guardam os cabritos, para se lhes ordenharem as mães.
(Talvez de \textunderscore cova\textunderscore )
\section{Covendedor}
\begin{itemize}
\item {Grp. gram.:m.}
\end{itemize}
\begin{itemize}
\item {Proveniência:(De \textunderscore com...\textunderscore  + \textunderscore vendedor\textunderscore )}
\end{itemize}
Aquelle que, com outrem, vende um objecto possuído em commum.
\section{Covil}
\begin{itemize}
\item {Grp. gram.:m.}
\end{itemize}
\begin{itemize}
\item {Proveniência:(Do lat. \textunderscore cubile\textunderscore )}
\end{itemize}
Cova de feras.
Lugar, onde se occulta o coêlho, a lebre, etc.
Lugar, onde se acolhem malfeitores.
Casa escura e miserável.
Alcoice.
\section{Covileiro}
\begin{itemize}
\item {Grp. gram.:adj.}
\end{itemize}
\begin{itemize}
\item {Proveniência:(De \textunderscore covil\textunderscore )}
\end{itemize}
Diz-se do caçador que, pelas pègadas da caça de pêlo, vai dar com ella.
\section{Covilhete}
\begin{itemize}
\item {fónica:lhê}
\end{itemize}
\begin{itemize}
\item {Grp. gram.:m.}
\end{itemize}
\begin{itemize}
\item {Grp. gram.:Adj.}
\end{itemize}
\begin{itemize}
\item {Proveniência:(De \textunderscore covo\textunderscore ? Por \textunderscore cubilhete\textunderscore , de \textunderscore cuba\textunderscore ?)}
\end{itemize}
Prato pequeno.
Tigelinha, pequena malga.
Espécie de copo de folha, de que se servem alguns prestidigitadores.
Diz-se do engenho de açúcar, movido por águas que vêm de pouca altura.
\section{Covinha}
\begin{itemize}
\item {Grp. gram.:f.}
\end{itemize}
\begin{itemize}
\item {Proveniência:(De \textunderscore cova\textunderscore )}
\end{itemize}
Pequena depressão natural sôbre o queixo ou na face.
Espécie de jôgo popular.
\section{Covinhado}
\begin{itemize}
\item {Grp. gram.:adj.}
\end{itemize}
Que tem covinhas:«\textunderscore covinhada esponja lhe serve de almofada.\textunderscore »Filinto, XXII, 86.
\section{Covo}
\begin{itemize}
\item {fónica:cô}
\end{itemize}
\begin{itemize}
\item {Grp. gram.:adj.}
\end{itemize}
\begin{itemize}
\item {Grp. gram.:M.}
\end{itemize}
\begin{itemize}
\item {Proveniência:(Do b. lat. \textunderscore cophus\textunderscore )}
\end{itemize}
Côncavo; fundo: \textunderscore prato covo\textunderscore .
Cêsto comprido de vimes, para pesca.
\section{Covoada}
\begin{itemize}
\item {Grp. gram.:f.}
\end{itemize}
Série de covas.
(Do \textunderscore covão\textunderscore )
\section{Covocó}
\begin{itemize}
\item {Grp. gram.:m.}
\end{itemize}
\begin{itemize}
\item {Utilização:Bras}
\end{itemize}
Caneiro ou levada, por onde se despeja a água que sái dos cubos das rodas dos engenhos de açúcar.
(Cp. \textunderscore cavouco\textunderscore )
\section{Coxa}
\begin{itemize}
\item {fónica:cô}
\end{itemize}
\begin{itemize}
\item {Grp. gram.:f.}
\end{itemize}
\begin{itemize}
\item {Utilização:Anat.}
\end{itemize}
\begin{itemize}
\item {Proveniência:(Lat. \textunderscore coxa\textunderscore )}
\end{itemize}
Parte superior dos membros locomotores, desde o joêlho ás virilhas.
\section{Coxa-de-dama}
\begin{itemize}
\item {Grp. gram.:f.}
\end{itemize}
Variedade de pêra muito sumarenta e doce.
\section{Coxa-de-dona}
\begin{itemize}
\item {Grp. gram.:f.}
\end{itemize}
Variedade de pêra, que alguns confundem com a coxa-de-dama, mas que parece sêr diversa.
\section{Coxa-de-freira}
\begin{itemize}
\item {Grp. gram.:f.}
\end{itemize}
Variedade de pêra.
\section{Coxal}
\begin{itemize}
\item {Grp. gram.:adj.}
\end{itemize}
Relativo á coxa.
\section{Coxalgia}
\begin{itemize}
\item {Grp. gram.:f.}
\end{itemize}
\begin{itemize}
\item {Proveniência:(T. hybr., do lat. \textunderscore coxa\textunderscore  + gr. \textunderscore algos\textunderscore )}
\end{itemize}
Dôr intensa na articulação superior da coxa.
\section{Coxálgico}
\begin{itemize}
\item {Grp. gram.:adj.}
\end{itemize}
Relativo a coxalgia.
\section{Coxanga}
\begin{itemize}
\item {Grp. gram.:m.}
\end{itemize}
\begin{itemize}
\item {Utilização:Prov.}
\end{itemize}
\begin{itemize}
\item {Utilização:beir.}
\end{itemize}
Designação depreciativa do indivíduo coxo.
\section{Coxé}
\begin{itemize}
\item {Grp. gram.:adj.}
\end{itemize}
\begin{itemize}
\item {Utilização:Bras. do N}
\end{itemize}
O mesmo que \textunderscore caxingó\textunderscore .
\section{Coxeadura}
\begin{itemize}
\item {Grp. gram.:f.}
\end{itemize}
Acto de coxear.
\section{Coxear}
\begin{itemize}
\item {Grp. gram.:v. i.}
\end{itemize}
\begin{itemize}
\item {Utilização:Fig.}
\end{itemize}
\begin{itemize}
\item {Proveniência:(De \textunderscore coxo\textunderscore )}
\end{itemize}
Andar, inclinando-se para um dos lados, por defeito ou doença num pé ou numa perna.
Vacillar.
Sêr imperfeito.
\section{Coxelo}
\begin{itemize}
\item {Grp. gram.:m.}
\end{itemize}
\begin{itemize}
\item {Proveniência:(Do rad. de \textunderscore coxa\textunderscore )}
\end{itemize}
Gênero de insectos coleópteros.
\section{Coxêndico}
\begin{itemize}
\item {Grp. gram.:adj.}
\end{itemize}
\begin{itemize}
\item {Utilização:Anat.}
\end{itemize}
Diz-se dos ossos dos quadrís, que, com o sacro, formam a \textunderscore bacia\textunderscore  ou \textunderscore pelve\textunderscore .
\section{Coxete}
\begin{itemize}
\item {fónica:xê}
\end{itemize}
\begin{itemize}
\item {Grp. gram.:m.}
\end{itemize}
O mesmo que \textunderscore coxote\textunderscore .
\section{Coxia}
\begin{itemize}
\item {Grp. gram.:f.}
\end{itemize}
\begin{itemize}
\item {Utilização:Prov.}
\end{itemize}
Prancha, em certas embarcações, para dar passagem da prôa á popa.
Passagem estreita, entre duas séries de bancos ou de outros objectos.
Assento móvel com dobradiças, nos theatros, quando as cadeiras ou poltronas não chegam para os espectadores da plateia.
Lugar, occupado por cada cavallo, em uma estrebaria.
\textunderscore Correr a coxia\textunderscore , andar á tuna, vadiar; andar por toda a parte.
(Cp. it. \textunderscore corsia\textunderscore )
\section{Cóxia}
\begin{itemize}
\item {Grp. gram.:f.}
\end{itemize}
Gênero de plantas primuláceas.
\section{Coxilar}
\textunderscore v. i.\textunderscore  (e der.)
(V. \textunderscore cochilar\textunderscore , etc.)
\section{Coxilgo}
\begin{itemize}
\item {Grp. gram.:m.}
\end{itemize}
\begin{itemize}
\item {Utilização:Prov.}
\end{itemize}
\begin{itemize}
\item {Utilização:beir.}
\end{itemize}
O mesmo que \textunderscore coucelo\textunderscore .
\section{Coxilha}
\begin{itemize}
\item {Grp. gram.:f.}
\end{itemize}
\begin{itemize}
\item {Utilização:Bras. do S}
\end{itemize}
Lomba prolongada, em que há pastos.
Pequeno monte, separado de outro por valles cobertos de mato.
\section{Coxim}
\begin{itemize}
\item {Grp. gram.:m.}
\end{itemize}
\begin{itemize}
\item {Proveniência:(Do fr. \textunderscore coussin\textunderscore )}
\end{itemize}
Almofada, que serve de assento.
Espécie de sofá sem costas.
Parte da sella, em que se assenta o cavalleiro.
Pequena almofada de máquina eléctrica.
Nome de vários objectos, que servem de assento a alguma coisa ou têm analogia com almofada.
\section{Coxinilho}
\begin{itemize}
\item {Grp. gram.:m.}
\end{itemize}
\begin{itemize}
\item {Utilização:Bras}
\end{itemize}
Pano de lan, que se estende sôbre a sella do cavallo.
(Cast. \textunderscore cojinilho\textunderscore )
\section{Coxípede}
\begin{itemize}
\item {Grp. gram.:adj.}
\end{itemize}
\begin{itemize}
\item {Proveniência:(Do lat. \textunderscore coxus\textunderscore  + \textunderscore pes\textunderscore )}
\end{itemize}
Aleijado dos pés ou de um pé; manco. Cf. Filinto, X, 63.
\section{Coxivorado}
\begin{itemize}
\item {Grp. gram.:m.}
\end{itemize}
Tributo, que depois se chamou foro, e que as antigas communidades indianas pagavam ao rei de Canará e depois aos dominantes moiros e maratas, para os ajudar contra os salteadores.
\section{Coxo}
\begin{itemize}
\item {fónica:cô}
\end{itemize}
\begin{itemize}
\item {Grp. gram.:adj.}
\end{itemize}
\begin{itemize}
\item {Utilização:Fig.}
\end{itemize}
\begin{itemize}
\item {Grp. gram.:M.}
\end{itemize}
Que coxeia.
Incompleto.
Diz-se de qualquer objecto, a que falta um pé ou uma perna: \textunderscore mesa coxa\textunderscore .
Aquelle que coxeia.
(B. lat. \textunderscore coxus\textunderscore )
\section{Coxo}
\begin{itemize}
\item {fónica:cô}
\end{itemize}
\begin{itemize}
\item {Grp. gram.:m.}
\end{itemize}
\begin{itemize}
\item {Utilização:Prov.}
\end{itemize}
\begin{itemize}
\item {Utilização:trasm.}
\end{itemize}
Designação genérica de qualquer animal peçonhento.
Espécie de erupção cutânea, que se attribue á passagem de animálculos venenosos pela roupa que estava no estendedoiro.
\section{Coxote}
\begin{itemize}
\item {Grp. gram.:m.}
\end{itemize}
\begin{itemize}
\item {Proveniência:(De \textunderscore coxa\textunderscore )}
\end{itemize}
Parte da armadura, que se punha nas coxas.
\section{Coxo-vertebral}
\begin{itemize}
\item {Grp. gram.:adj.}
\end{itemize}
Relativo ás coxas e ás vértebras.
\section{Cozedeira}
\begin{itemize}
\item {Grp. gram.:f.}
\end{itemize}
\begin{itemize}
\item {Utilização:Des.}
\end{itemize}
\begin{itemize}
\item {Proveniência:(De \textunderscore cozer\textunderscore )}
\end{itemize}
Tigela ou tacho, em que se faz comida.
\section{Cozedura}
\begin{itemize}
\item {Grp. gram.:f.}
\end{itemize}
Acto ou effeito de cozer.
Porção de coisas, que se coze de uma vez.
\section{Cozer}
\begin{itemize}
\item {Grp. gram.:v. t.}
\end{itemize}
\begin{itemize}
\item {Utilização:Fig.}
\end{itemize}
\begin{itemize}
\item {Proveniência:(Lat. pop. \textunderscore cocere\textunderscore )}
\end{itemize}
Preparar (alimentos) pela acção do fogo: \textunderscore cozer o pão\textunderscore .
Digerir.
Submeter á acção do lume (substâncias que estão dentro de um liquido): \textunderscore cozer a carne\textunderscore ; \textunderscore cozer batatas\textunderscore .
Aguentar, supportar.
\section{Cozido}
\begin{itemize}
\item {Grp. gram.:M.}
\end{itemize}
\begin{itemize}
\item {Proveniência:(De \textunderscore cozer\textunderscore )}
\end{itemize}
Que se cozeu: \textunderscore pão cozido\textunderscore .
Diz-se do vinho, que já deixou de fermentar, e se purificou, tornando-se bom para sêr bebido.
Parte do jantar, constituída por carne cozida acompanhada geralmente com arroz, legumes, toicinho, etc.
\section{Cozimento}
\begin{itemize}
\item {Grp. gram.:m.}
\end{itemize}
\begin{itemize}
\item {Proveniência:(De \textunderscore cozer\textunderscore )}
\end{itemize}
Acto de cozer.
Digestão.
Decocto; infusão.
Fôrro, que se fórma nos talhos das marinhas.
\section{Cozinha}
\begin{itemize}
\item {Grp. gram.:f.}
\end{itemize}
\begin{itemize}
\item {Utilização:Prov.}
\end{itemize}
\begin{itemize}
\item {Utilização:minh.}
\end{itemize}
\begin{itemize}
\item {Utilização:Gír.}
\end{itemize}
\begin{itemize}
\item {Proveniência:(Lat. pop. \textunderscore cocina\textunderscore . Cp. \textunderscore cozer\textunderscore )}
\end{itemize}
Compartimento, onde se preparam os alimentos pela acção do fogo.
Arte de preparar os alimentos.
Preparação da comida.
Fogão de cozinha. (Mais usual, \textunderscore cozinha de ferro\textunderscore )
Esquadra ou pôsto policial.
\section{Cozinhado}
\begin{itemize}
\item {Grp. gram.:m.}
\end{itemize}
\begin{itemize}
\item {Proveniência:(De \textunderscore cozinhar\textunderscore )}
\end{itemize}
Comida, preparada ao lume.
\section{Cozinhar}
\begin{itemize}
\item {Grp. gram.:v. t.  e  i.}
\end{itemize}
\begin{itemize}
\item {Utilização:Fig.}
\end{itemize}
\begin{itemize}
\item {Proveniência:(De \textunderscore cozinha\textunderscore )}
\end{itemize}
Preparar ao lume os alimentos.
Preparar, ordenar; dispor bem.
\section{Cozinheira}
\begin{itemize}
\item {Grp. gram.:f.}
\end{itemize}
\begin{itemize}
\item {Utilização:Náut.}
\end{itemize}
Mulher que cozinha.
Vela do estai da gávea, também chamada \textunderscore formosa\textunderscore .
\section{Cozinheiro}
\begin{itemize}
\item {Grp. gram.:m.}
\end{itemize}
\begin{itemize}
\item {Proveniência:(De \textunderscore cozinhar\textunderscore )}
\end{itemize}
Aquelle que cozinha.
\section{Cp.}
(abrev. de \textunderscore comparar\textunderscore  ou \textunderscore compare\textunderscore )
\section{Crabro}
\begin{itemize}
\item {Grp. gram.:m.}
\end{itemize}
\begin{itemize}
\item {Proveniência:(Lat. \textunderscore crabro\textunderscore )}
\end{itemize}
Insecto negro e amarelo, de antenas filiformes e palpos curtos.
\section{Crabunha}
\begin{itemize}
\item {Grp. gram.:f.}
\end{itemize}
\begin{itemize}
\item {Utilização:Prov.}
\end{itemize}
\begin{itemize}
\item {Utilização:minh.}
\end{itemize}
Caroço de fruta.
\section{Crac!}
\begin{itemize}
\item {Grp. gram.:interj.}
\end{itemize}
\begin{itemize}
\item {Proveniência:(T. onom.)}
\end{itemize}
Voz imitativa de um desmoronamento com ruído, ou de um objecto que estala ou se parte com estrondo.
\section{Craca}
\begin{itemize}
\item {Grp. gram.:f.}
\end{itemize}
\begin{itemize}
\item {Utilização:Ant.}
\end{itemize}
Mollusco, que vive nos rochedos e no costado dos navios.
Parte côncava das columnas estriadas.
\section{Craca}
\begin{itemize}
\item {Grp. gram.:f.}
\end{itemize}
\begin{itemize}
\item {Proveniência:(Lat. \textunderscore cracca\textunderscore )}
\end{itemize}
Planta leguminosa.
\section{Cracca}
\begin{itemize}
\item {Grp. gram.:f.}
\end{itemize}
\begin{itemize}
\item {Proveniência:(Lat. \textunderscore cracca\textunderscore )}
\end{itemize}
Planta leguminosa.
\section{Crachá}
\begin{itemize}
\item {Grp. gram.:m.}
\end{itemize}
\begin{itemize}
\item {Proveniência:(Fr. \textunderscore crachat\textunderscore )}
\end{itemize}
Insígnia honorífica, que se traz ao peito; condecoração.
\section{Cracolé}
\begin{itemize}
\item {Grp. gram.:m.}
\end{itemize}
\begin{itemize}
\item {Utilização:Prov.}
\end{itemize}
O mesmo que \textunderscore codorniz\textunderscore .
\section{Cracoviana}
\begin{itemize}
\item {Grp. gram.:f.}
\end{itemize}
\begin{itemize}
\item {Proveniência:(De \textunderscore Cracóvia\textunderscore , n. p.)}
\end{itemize}
Uma dança polaca, viva e ligeira.
\section{Crafórdia}
\begin{itemize}
\item {Grp. gram.:f.}
\end{itemize}
\begin{itemize}
\item {Proveniência:(De \textunderscore Craford\textunderscore , n. p.)}
\end{itemize}
Planta americana, leguminosa e trepadeira.
\section{Craiom}
\begin{itemize}
\item {Grp. gram.:m.}
\end{itemize}
\begin{itemize}
\item {Proveniência:(Fr. \textunderscore crayon\textunderscore )}
\end{itemize}
Substância terrosa ou metállica, para traçar linhas, desenhar, etc.
\section{Cramar}
\textunderscore v. i.\textunderscore  (e der.)
Fórma ant., e ainda us. nos Açores, por \textunderscore clamar\textunderscore , etc.
\section{Crambe}
\begin{itemize}
\item {Grp. gram.:f.}
\end{itemize}
\begin{itemize}
\item {Proveniência:(Gr. \textunderscore krambe\textunderscore )}
\end{itemize}
Planta crucífera, mais conhecida por couve-marinha.
\section{Crambo}
\begin{itemize}
\item {Grp. gram.:m.}
\end{itemize}
\begin{itemize}
\item {Proveniência:(Gr. \textunderscore krambis\textunderscore )}
\end{itemize}
Insecto lepidóptero, de fórma alongada, e que apparece no tempo quente.
\section{Cramoiço}
\begin{itemize}
\item {Grp. gram.:m.}
\end{itemize}
\begin{itemize}
\item {Utilização:Prov.}
\end{itemize}
Montão.
(Corr. de \textunderscore comoroiço\textunderscore )
\section{Cramôl}
\begin{itemize}
\item {Grp. gram.:m.}
\end{itemize}
\begin{itemize}
\item {Utilização:Prov.}
\end{itemize}
O mesmo que \textunderscore cramor\textunderscore .
\section{Cramor}
\begin{itemize}
\item {Grp. gram.:m.}
\end{itemize}
\begin{itemize}
\item {Utilização:Prov.}
\end{itemize}
\begin{itemize}
\item {Utilização:minh.}
\end{itemize}
Procissão, por votos antigos, quasi sempre de freguesia para freguesia.
O mesmo que \textunderscore clamor\textunderscore .
\section{Crampa}
\begin{itemize}
\item {Grp. gram.:f.}
\end{itemize}
\begin{itemize}
\item {Proveniência:(Al. \textunderscore krampe\textunderscore )}
\end{itemize}
Contracção espasmódica e dolorosa de certos músculos.
\section{Cramuri}
\begin{itemize}
\item {Grp. gram.:f.}
\end{itemize}
Fruto silvestre do Brasil.
\section{Crancelim}
\begin{itemize}
\item {Grp. gram.:m.}
\end{itemize}
\begin{itemize}
\item {Utilização:Heráld.}
\end{itemize}
Porção de corôa com florões, collocada em banda no meio de un escudo.
(Do. al. \textunderscore krantzlein\textunderscore , pequena corôa)
\section{Crancho}
\begin{itemize}
\item {Grp. gram.:adj.}
\end{itemize}
\begin{itemize}
\item {Utilização:Prov.}
\end{itemize}
\begin{itemize}
\item {Utilização:trasm.}
\end{itemize}
Enfatuado; empertigado; cheio de si.
\section{Crâneo}
\textunderscore m.\textunderscore  (e der.)
(V. \textunderscore crânio\textunderscore , etc.)
\section{Crangés}
\begin{itemize}
\item {Grp. gram.:m. pl.}
\end{itemize}
Tríbo de Índios Macamecrans, no Brasil.
\section{Cranguejo}
\begin{itemize}
\item {Grp. gram.:m.}
\end{itemize}
\begin{itemize}
\item {Utilização:Ant.}
\end{itemize}
O mesmo que \textunderscore caranguejo\textunderscore . Cf. \textunderscore Eufrosina\textunderscore , 183.
\section{Craniano}
\begin{itemize}
\item {Grp. gram.:adj.}
\end{itemize}
Relativo a crânio.
\section{Craniectomia}
\begin{itemize}
\item {Grp. gram.:f.}
\end{itemize}
\begin{itemize}
\item {Proveniência:(Do gr. \textunderscore kranion\textunderscore  + \textunderscore ex\textunderscore  + \textunderscore temnein\textunderscore )}
\end{itemize}
Operação cirúrgica, em que se abre o crânio, para permittir que o cérebro se desenvolva.
\section{Crânio}
\begin{itemize}
\item {Grp. gram.:m.}
\end{itemize}
\begin{itemize}
\item {Proveniência:(Lat. \textunderscore cranium\textunderscore )}
\end{itemize}
Caixa óssea, que encerra e protege o cérebro.
Caveira.
\section{Cranio-cerebral}
\begin{itemize}
\item {Grp. gram.:adj.}
\end{itemize}
Relativo ao crânio e ao cerebro.
\section{Cranio-facial}
\begin{itemize}
\item {Grp. gram.:adj.}
\end{itemize}
Que diz respeito ao crânio e á face.
\section{Craniografia}
\begin{itemize}
\item {Grp. gram.:f.}
\end{itemize}
\begin{itemize}
\item {Proveniência:(Do gr. \textunderscore kranion\textunderscore  + \textunderscore graphein\textunderscore )}
\end{itemize}
Descripção scientífica do crânio.
\section{Craniográfico}
\begin{itemize}
\item {Grp. gram.:adj.}
\end{itemize}
Relativo á craniografia.
\section{Craniógrafo}
\begin{itemize}
\item {Grp. gram.:m.}
\end{itemize}
Aquele que trata de craniografia.
Nome de dois instrumentos de craniografia.--O craniógrafo de Broca serve para medir o contôrno do perfil e a posição do ponto auricular, no crânio; e o craniógrafo de Kopernichi é para medir as curvas cranianas.
\section{Craniographia}
\begin{itemize}
\item {Grp. gram.:f.}
\end{itemize}
\begin{itemize}
\item {Proveniência:(Do gr. \textunderscore kranion\textunderscore  + \textunderscore graphein\textunderscore )}
\end{itemize}
Descripção scientífica do crânio.
\section{Craniográphico}
\begin{itemize}
\item {Grp. gram.:adj.}
\end{itemize}
Relativo á craniographia.
\section{Craniógrapho}
\begin{itemize}
\item {Grp. gram.:m.}
\end{itemize}
Aquelle que trata de craniographia.
Nome de dois instrumentos de craniographia.--O craniógrapho de Broca serve para medir o contôrno do perfil e a posição do ponto auricular, no crânio; e o craniógrapho de Kopernichi é para medir as curvas cranianas.
\section{Craniolar}
\begin{itemize}
\item {Grp. gram.:adj.}
\end{itemize}
Que tem fórma de crânio.
\section{Craniolária}
\begin{itemize}
\item {Grp. gram.:f.}
\end{itemize}
Concha craniolar.
\section{Craniologia}
\begin{itemize}
\item {Grp. gram.:f.}
\end{itemize}
\begin{itemize}
\item {Proveniência:(Do gr. \textunderscore kranion\textunderscore  + \textunderscore logos\textunderscore )}
\end{itemize}
Estudo sôbre os crânios.
Arte de conhecer a organização phýsica e as qualidades moraes ou intellectuaes de um indivíduo, pelo exame do seu crânio.
\section{Craniológico}
\begin{itemize}
\item {Grp. gram.:adj.}
\end{itemize}
Relativo a craniologia.
\section{Craniologista}
\begin{itemize}
\item {Grp. gram.:m.}
\end{itemize}
Aquelle que trata de craniologia ou é versado nella.
\section{Craniólogo}
\begin{itemize}
\item {Grp. gram.:m.}
\end{itemize}
Aquelle que trata de craniologia ou é versado nella.
\section{Craniomancia}
\begin{itemize}
\item {Grp. gram.:f.}
\end{itemize}
\begin{itemize}
\item {Proveniência:(Do gr. \textunderscore kranion\textunderscore  + \textunderscore manteia\textunderscore )}
\end{itemize}
Supposta arte de conhecer as tendências intellectuaes e moraes de um indivíduo pela observação do seu crânio.
\section{Craniometria}
\begin{itemize}
\item {Grp. gram.:f.}
\end{itemize}
Medição do crânio.
(Cp. \textunderscore craniómetro\textunderscore )
\section{Craniométrico}
\begin{itemize}
\item {Grp. gram.:adj.}
\end{itemize}
Relativo á craniometria.
\section{Craniómetro}
\begin{itemize}
\item {Grp. gram.:m.}
\end{itemize}
\begin{itemize}
\item {Proveniência:(Do gr. \textunderscore kranion\textunderscore  + \textunderscore metron\textunderscore )}
\end{itemize}
Instrumento, com que se medem os diâmetros do crânio.
\section{Cranioscopia}
\begin{itemize}
\item {Grp. gram.:f.}
\end{itemize}
\begin{itemize}
\item {Proveniência:(Do gr. \textunderscore kranion\textunderscore  + \textunderscore skopein\textunderscore )}
\end{itemize}
Arte de examinar o crânio e de apreciar, segundo êsse exame, as faculdades intellectuaes e moraes.
\section{Craniosópico}
\begin{itemize}
\item {Grp. gram.:adj.}
\end{itemize}
Relativo á cranioscopia.
\section{Craniospermo}
\begin{itemize}
\item {Grp. gram.:m.}
\end{itemize}
\begin{itemize}
\item {Proveniência:(Do gr. \textunderscore kranion\textunderscore  + \textunderscore sperma\textunderscore )}
\end{itemize}
Planta borragínea da Sibéria.
\section{Craniota}
\begin{itemize}
\item {Grp. gram.:m.  e  adj.}
\end{itemize}
Diz-se dos animaes que tem crânio.
\section{Craniotomia}
\begin{itemize}
\item {Grp. gram.:f.}
\end{itemize}
Operação por meio do craniótomo.
\section{Craniótomo}
\begin{itemize}
\item {Grp. gram.:m.}
\end{itemize}
\begin{itemize}
\item {Proveniência:(Do gr. \textunderscore kranion\textunderscore  + \textunderscore temnein\textunderscore )}
\end{itemize}
Instrumento, com que se faz a perfuração do crânio de um féto, quando se não póde realizar o parto.
\section{Cranque}
\begin{itemize}
\item {Grp. gram.:m.}
\end{itemize}
\begin{itemize}
\item {Proveniência:(Do ingl. \textunderscore crank\textunderscore )}
\end{itemize}
Eixo de máquinas, em fórma de cotovelo.
\section{Crapintina}
\begin{itemize}
\item {Grp. gram.:f.}
\end{itemize}
\begin{itemize}
\item {Utilização:Prov.}
\end{itemize}
\begin{itemize}
\item {Utilização:alg.}
\end{itemize}
O mesmo que \textunderscore carpintina\textunderscore .
\section{Crapuça}
\textunderscore f.\textunderscore  (e der.)
(Fórma popular de \textunderscore carapuça\textunderscore , etc.)
\section{Crápula}
\begin{itemize}
\item {Grp. gram.:f.}
\end{itemize}
\begin{itemize}
\item {Proveniência:(Lat. \textunderscore crapula\textunderscore )}
\end{itemize}
Modo extravagante de vida.
Desregramento.
Libertinagem; devassidão.
\section{Crapuloso}
\begin{itemize}
\item {Grp. gram.:adj.}
\end{itemize}
\begin{itemize}
\item {Proveniência:(Lat. \textunderscore crapulosus\textunderscore )}
\end{itemize}
Libertino, devasso.
Em que há crápula: \textunderscore procedimento crapuloso\textunderscore .
\section{Craque!}
\begin{itemize}
\item {Grp. gram.:interj.}
\end{itemize}
\begin{itemize}
\item {Proveniência:(T. onom.)}
\end{itemize}
Voz imitativa de um desmoronamento com ruído, ou de um objecto que estala ou se parte com estrondo.
\section{Crás}
\begin{itemize}
\item {Grp. gram.:m.}
\end{itemize}
Som imitativo da voz do corvo.
\section{Crase}
\begin{itemize}
\item {Grp. gram.:f.}
\end{itemize}
\begin{itemize}
\item {Utilização:Gram.}
\end{itemize}
\begin{itemize}
\item {Utilização:Physiol.}
\end{itemize}
\begin{itemize}
\item {Proveniência:(Gr. \textunderscore krasis\textunderscore )}
\end{itemize}
Contracção de sýllabas ou vogaes numa só.
Mistura proporcionada ou equilíbrio das partes que constituem os líquidos da economia animal.
Temperamento.
\section{Crasiografia}
\begin{itemize}
\item {Grp. gram.:f.}
\end{itemize}
\begin{itemize}
\item {Proveniência:(Do gr. \textunderscore krasis\textunderscore  + \textunderscore graphein\textunderscore )}
\end{itemize}
Descripção das diversas crases ou temperamentos.
\section{Crasiographia}
\begin{itemize}
\item {Grp. gram.:f.}
\end{itemize}
\begin{itemize}
\item {Proveniência:(Do gr. \textunderscore krasis\textunderscore  + \textunderscore graphein\textunderscore )}
\end{itemize}
Descripção das diversas crases ou temperamentos.
\section{Crasiologia}
\begin{itemize}
\item {Grp. gram.:f.}
\end{itemize}
\begin{itemize}
\item {Proveniência:(Do gr. \textunderscore krasis\textunderscore  + \textunderscore logos\textunderscore )}
\end{itemize}
Tratado das crases ou temperamentos.
\section{Crasiológico}
\begin{itemize}
\item {Grp. gram.:adj.}
\end{itemize}
Relativo á crasiologia.
\section{Craspedonte}
\begin{itemize}
\item {Grp. gram.:m.}
\end{itemize}
\begin{itemize}
\item {Proveniência:(Do gr. \textunderscore kraspedon\textunderscore )}
\end{itemize}
Insecto coleóptero.
\section{Craspedosomo}
\begin{itemize}
\item {fónica:sô}
\end{itemize}
\begin{itemize}
\item {Grp. gram.:m.}
\end{itemize}
\begin{itemize}
\item {Proveniência:(Do gr. \textunderscore kraspedon\textunderscore  + \textunderscore soma\textunderscore )}
\end{itemize}
Gênero de insectos myriápodes.
\section{Craspedossomo}
\begin{itemize}
\item {Grp. gram.:m.}
\end{itemize}
\begin{itemize}
\item {Proveniência:(Do gr. \textunderscore kraspedon\textunderscore  + \textunderscore soma\textunderscore )}
\end{itemize}
Gênero de insectos myriápodes.
\section{Crassamente}
\begin{itemize}
\item {Grp. gram.:adv.}
\end{itemize}
De modo crasso.
\section{Crassatela}
\begin{itemize}
\item {Grp. gram.:f.}
\end{itemize}
\begin{itemize}
\item {Proveniência:(Fr. \textunderscore crassatelle\textunderscore )}
\end{itemize}
Concha marinha, com duas valvas.
\section{Crassicaude}
\begin{itemize}
\item {fónica:crá}
\end{itemize}
\begin{itemize}
\item {Grp. gram.:adj.}
\end{itemize}
\begin{itemize}
\item {Utilização:Zool.}
\end{itemize}
\begin{itemize}
\item {Proveniência:(Do lat. \textunderscore crassus\textunderscore  + \textunderscore cauda\textunderscore )}
\end{itemize}
Que tem cauda grossa.
\section{Crassicaule}
\begin{itemize}
\item {fónica:crá}
\end{itemize}
\begin{itemize}
\item {Grp. gram.:adj.}
\end{itemize}
\begin{itemize}
\item {Utilização:Bot.}
\end{itemize}
\begin{itemize}
\item {Proveniência:(Do lat. \textunderscore crassus\textunderscore  + \textunderscore caulis\textunderscore )}
\end{itemize}
Que tem haste grossa.
\section{Crassície}
\begin{itemize}
\item {Grp. gram.:f.}
\end{itemize}
O mesmo que \textunderscore crassidão\textunderscore .
\section{Crassicollo}
\begin{itemize}
\item {Grp. gram.:adj.}
\end{itemize}
\begin{itemize}
\item {Proveniência:(Do lat. \textunderscore crassus\textunderscore  + \textunderscore collum\textunderscore )}
\end{itemize}
Que tem pescoço grosso.
\section{Crassicolo}
\begin{itemize}
\item {Grp. gram.:adj.}
\end{itemize}
\begin{itemize}
\item {Proveniência:(Do lat. \textunderscore crassus\textunderscore  + \textunderscore collum\textunderscore )}
\end{itemize}
Que tem pescoço grosso.
\section{Crassicórneo}
\begin{itemize}
\item {Grp. gram.:adj.}
\end{itemize}
\begin{itemize}
\item {Proveniência:(De \textunderscore crasso\textunderscore  + \textunderscore córneo\textunderscore )}
\end{itemize}
Que tem cornos ou antennas espêssas.
\section{Crassidade}
\begin{itemize}
\item {Grp. gram.:f.}
\end{itemize}
O mesmo que \textunderscore crassidão\textunderscore .
\section{Crassidão}
\begin{itemize}
\item {Grp. gram.:m.}
\end{itemize}
\begin{itemize}
\item {Proveniência:(Lat. \textunderscore crassitudo\textunderscore )}
\end{itemize}
Qualidade daquillo que é crasso.
\section{Crassifoliado}
\begin{itemize}
\item {Grp. gram.:adj.}
\end{itemize}
\begin{itemize}
\item {Utilização:Bot.}
\end{itemize}
\begin{itemize}
\item {Proveniência:(Do lat. \textunderscore crassus\textunderscore  + \textunderscore folium\textunderscore )}
\end{itemize}
Que tem fôlhas grossas.
\section{Crassilíngue}
\begin{itemize}
\item {Grp. gram.:adj.}
\end{itemize}
\begin{itemize}
\item {Grp. gram.:M.}
\end{itemize}
\begin{itemize}
\item {Proveniência:(Do lat. \textunderscore crassus\textunderscore  + \textunderscore lingua\textunderscore )}
\end{itemize}
Que tem língua grossa.
Reptil sáurio.
\section{Crassinérveo}
\begin{itemize}
\item {Grp. gram.:adj.}
\end{itemize}
\begin{itemize}
\item {Utilização:Bot.}
\end{itemize}
\begin{itemize}
\item {Proveniência:(Do lat. \textunderscore crassus\textunderscore  + \textunderscore nervus\textunderscore )}
\end{itemize}
Que tem nervuras espêssas.
\section{Crassipene}
\begin{itemize}
\item {Grp. gram.:adj.}
\end{itemize}
\begin{itemize}
\item {Utilização:Zool.}
\end{itemize}
\begin{itemize}
\item {Proveniência:(Do lat. \textunderscore crassus\textunderscore  + \textunderscore penna\textunderscore )}
\end{itemize}
Que tem pennas espêssas.
\section{Crassipenne}
\begin{itemize}
\item {Grp. gram.:adj.}
\end{itemize}
\begin{itemize}
\item {Utilização:Zool.}
\end{itemize}
\begin{itemize}
\item {Proveniência:(Do lat. \textunderscore crassus\textunderscore  + \textunderscore penna\textunderscore )}
\end{itemize}
Que tem pennas espêssas.
\section{Crassirostro}
\begin{itemize}
\item {fónica:rós}
\end{itemize}
\begin{itemize}
\item {Grp. gram.:adj.}
\end{itemize}
\begin{itemize}
\item {Utilização:Zool.}
\end{itemize}
\begin{itemize}
\item {Proveniência:(Do lat. \textunderscore crassus\textunderscore  + \textunderscore rostrum\textunderscore )}
\end{itemize}
Que tem bico grosso.
\section{Crassirrostro}
\begin{itemize}
\item {Grp. gram.:adj.}
\end{itemize}
\begin{itemize}
\item {Utilização:Zool.}
\end{itemize}
\begin{itemize}
\item {Proveniência:(Do lat. \textunderscore crassus\textunderscore  + \textunderscore rostrum\textunderscore )}
\end{itemize}
Que tem bico grosso.
\section{Crasso}
\begin{itemize}
\item {Grp. gram.:adj.}
\end{itemize}
\begin{itemize}
\item {Proveniência:(Lat. \textunderscore crassus\textunderscore )}
\end{itemize}
Espêsso.
Cerrado; denso.
Grosseiro: \textunderscore erros crassos\textunderscore .
\section{Crássula}
\begin{itemize}
\item {Grp. gram.:f.}
\end{itemize}
\begin{itemize}
\item {Utilização:Bot.}
\end{itemize}
\begin{itemize}
\item {Proveniência:(De \textunderscore crasso\textunderscore )}
\end{itemize}
Gênero de plantas gordas.
\section{Crassuláceas}
\begin{itemize}
\item {Grp. gram.:f. pl.}
\end{itemize}
Família de plantas, que têm por typo a crássula.
\section{Crasta}
\begin{itemize}
\item {Grp. gram.:f.}
\end{itemize}
\begin{itemize}
\item {Utilização:Des.}
\end{itemize}
O mesmo que \textunderscore claustro\textunderscore .
\section{Crasteiro}
\begin{itemize}
\item {Grp. gram.:adj.}
\end{itemize}
Relativo a crasta:«\textunderscore que fôra prior crasteiro de Santa-Cruz\textunderscore ». Campos Júnior, \textunderscore Camões\textunderscore .
\section{Crastejo}
\begin{itemize}
\item {Grp. gram.:m.}
\end{itemize}
Pequeno crasto; castellejo.
\section{Crastello}
\begin{itemize}
\item {Grp. gram.:m.}
\end{itemize}
O mesmo que \textunderscore castrello\textunderscore . Cf. Herculano, \textunderscore Hist. de Port.\textunderscore , IV, 88.
\section{Crastelo}
\begin{itemize}
\item {fónica:té}
\end{itemize}
\begin{itemize}
\item {Grp. gram.:m.}
\end{itemize}
O mesmo que \textunderscore castrelo\textunderscore . Cf. Herculano, \textunderscore Hist. de Port.\textunderscore , IV, 88.
\section{Crástino}
\begin{itemize}
\item {Grp. gram.:adj.}
\end{itemize}
\begin{itemize}
\item {Proveniência:(Lat. \textunderscore crastinus\textunderscore )}
\end{itemize}
Relativo ao dia de àmanhan, ao dia seguinte.
\section{Crasto}
\begin{itemize}
\item {Grp. gram.:m.}
\end{itemize}
O mesmo que \textunderscore castro\textunderscore .
(Metáth. de \textunderscore castro\textunderscore )
\section{Crategina}
\begin{itemize}
\item {Grp. gram.:f.}
\end{itemize}
\begin{itemize}
\item {Proveniência:(De \textunderscore cratego\textunderscore , ou antes de \textunderscore cratégono\textunderscore )}
\end{itemize}
Matéria crystallizável, extrahída da casca do lódão.
\section{Cratego}
\begin{itemize}
\item {Grp. gram.:m.}
\end{itemize}
\begin{itemize}
\item {Proveniência:(Lat. \textunderscore crataegum\textunderscore )}
\end{itemize}
Semente do buxo.
\section{Cratégono}
\begin{itemize}
\item {Grp. gram.:m.}
\end{itemize}
\begin{itemize}
\item {Proveniência:(Do gr. \textunderscore krataigon\textunderscore )}
\end{itemize}
Planta, que uns classificam entre as rosáceas, outros entre as escrofularíneas; espécie de lódão, segundo outros.
\section{Crateógono}
\begin{itemize}
\item {Grp. gram.:m.}
\end{itemize}
\begin{itemize}
\item {Proveniência:(Gr. \textunderscore krataiogonon\textunderscore )}
\end{itemize}
O mesmo que \textunderscore persicária\textunderscore , ou \textunderscore erva-pessegueira\textunderscore .
\section{Cratera}
\begin{itemize}
\item {Grp. gram.:f.}
\end{itemize}
\begin{itemize}
\item {Utilização:Fig.}
\end{itemize}
\begin{itemize}
\item {Proveniência:(Lat. \textunderscore cratera\textunderscore )}
\end{itemize}
Vaso antigo, sem gargalo, e cuja largura augmentava progressivamente desde o fundo até á bôca.
Abertura, por onde o vulcão expelle a lava.
Origem de desgraças.
\section{Crateriforme}
\begin{itemize}
\item {Grp. gram.:adj.}
\end{itemize}
Que tem fórma de cratera.
\section{Craticulação}
\begin{itemize}
\item {Grp. gram.:f.}
\end{itemize}
\begin{itemize}
\item {Proveniência:(Fr. \textunderscore craticulation\textunderscore )}
\end{itemize}
Processo, para copiar desenhos, dividindo-se o original em pequenos quadrados, que se repetem no papel destinado á copia.
\section{Crato}
\begin{itemize}
\item {Grp. gram.:m.}
\end{itemize}
\begin{itemize}
\item {Proveniência:(De \textunderscore Crato\textunderscore , n. p.)}
\end{itemize}
Casta de uva branca algarvia.
Casta de uva preta, também algarvia.
\section{Craúno}
\begin{itemize}
\item {Grp. gram.:adj.}
\end{itemize}
\begin{itemize}
\item {Utilização:Bras. do S}
\end{itemize}
O mesmo que \textunderscore caraúno\textunderscore .
\section{Crausta}
\begin{itemize}
\item {Grp. gram.:f.}
\end{itemize}
\begin{itemize}
\item {Utilização:Ant.}
\end{itemize}
O mesmo que \textunderscore crasta\textunderscore .
\section{Cravação}
\begin{itemize}
\item {Grp. gram.:f.}
\end{itemize}
Acto e effeito de cravar.
Ornato, feito de pregos, dispostos symetricamente.
Saliência ou relêvo, produzido no papel pela impressão typográphica, do lado opposto àquelle em que se imprime.
\section{Cravador}
\begin{itemize}
\item {Grp. gram.:m.}
\end{itemize}
Aquelle ou aquillo que crava.
Furador, instrumento de sapateiro.
\section{Cravadura}
\begin{itemize}
\item {Grp. gram.:f.}
\end{itemize}
\begin{itemize}
\item {Utilização:Veter.}
\end{itemize}
O mesmo que \textunderscore cravação\textunderscore .
Ferimento, produzido pela implantação de um cravo nos tecidos sensíveis do pé do solípede.
\section{Cravagem}
\begin{itemize}
\item {Grp. gram.:f.}
\end{itemize}
\begin{itemize}
\item {Proveniência:(De \textunderscore cravar\textunderscore )}
\end{itemize}
Doença de plantas gramíneas; fungão.
\section{Cravar}
\begin{itemize}
\item {Grp. gram.:v. t.}
\end{itemize}
\begin{itemize}
\item {Proveniência:(De \textunderscore cravo\textunderscore )}
\end{itemize}
Fazer entrar, batendo ou impellindo: \textunderscore cravar estacas\textunderscore .
Engastar: \textunderscore cravar diamantes\textunderscore .
Segurar; fixar.
\section{Cravata}
\begin{itemize}
\item {Grp. gram.:f.}
\end{itemize}
\begin{itemize}
\item {Utilização:Ant.}
\end{itemize}
\begin{itemize}
\item {Proveniência:(Fr. \textunderscore cravate\textunderscore )}
\end{itemize}
Lenço, que os homens usavam ao pescoço e que formava laço debaixo da barba. Cp. \textunderscore gravata\textunderscore .
\section{Cravatá}
\begin{itemize}
\item {Grp. gram.:m.}
\end{itemize}
Planta bromeliácea do Brasil.
\section{Craveira}
\begin{itemize}
\item {Grp. gram.:f.}
\end{itemize}
\begin{itemize}
\item {Proveniência:(De \textunderscore cravo\textunderscore )}
\end{itemize}
Bitola, com que se mede a altura dos indivíduos recenseados para o serviço militar.
Utensílio, com que o sapateiro toma a medida de um pé.
Medida.
Orifício da ferradura, em que entra o cravo.
Instrumento, com que se fazem as cabeças dos cravos e pregos.
\section{Craveiro}
\begin{itemize}
\item {Grp. gram.:adj.}
\end{itemize}
Diz-se do palmo que tem 12 pollegadas, e da braça que tem 10 palmos craveiros.
Relativo a craveira.
\section{Craveiro}
\begin{itemize}
\item {Grp. gram.:m.}
\end{itemize}
\begin{itemize}
\item {Utilização:Prov.}
\end{itemize}
Planta, que dá cravos.
Vaso, em que se cria essa planta.
Qualquer vaso de flôres.
\section{Craveiro}
\begin{itemize}
\item {Grp. gram.:m.}
\end{itemize}
\begin{itemize}
\item {Utilização:Ant.}
\end{itemize}
O mesmo que \textunderscore claviculário\textunderscore ; frade, que, nas Ordens militares, tinha a chave do convento.
(Por \textunderscore claveiro\textunderscore , do lat. \textunderscore clavis\textunderscore )
\section{Craveiro}
\begin{itemize}
\item {Grp. gram.:m.}
\end{itemize}
\begin{itemize}
\item {Proveniência:(De \textunderscore cravo\textunderscore )}
\end{itemize}
Fabricante de cravo para ferraduras.
\section{Cravejador}
\begin{itemize}
\item {Grp. gram.:m.}
\end{itemize}
\begin{itemize}
\item {Proveniência:(De \textunderscore cravejar\textunderscore )}
\end{itemize}
Aquelle que craveja.
Aquelle que faz cravos para ferradura.
\section{Cravejamento}
\begin{itemize}
\item {Grp. gram.:m.}
\end{itemize}
Acto ou effeito de cravejar.
\section{Cravejar}
\begin{itemize}
\item {Grp. gram.:v. t.}
\end{itemize}
Pregar com cravos.
Engastar: \textunderscore anel cravejado de brilhantes\textunderscore .
\section{Cravela}
\begin{itemize}
\item {Grp. gram.:f.}
\end{itemize}
(contr. de \textunderscore caravela\textunderscore )
\section{Cravelha}
\begin{itemize}
\item {fónica:vê}
\end{itemize}
\begin{itemize}
\item {Grp. gram.:f.}
\end{itemize}
\begin{itemize}
\item {Utilização:Ant.}
\end{itemize}
\begin{itemize}
\item {Proveniência:(Do lat. \textunderscore clavicula\textunderscore )}
\end{itemize}
Peça de madeira ou metal, com que se retesam as cordas de certos instrumentos músicos: \textunderscore as cravelhas da rabeca\textunderscore .
Peça, com que se obturava o ouvido dos canhões.
O mesmo que \textunderscore cravelho\textunderscore .
\section{Cravelhal}
\begin{itemize}
\item {Grp. gram.:m.}
\end{itemize}
O mesmo que \textunderscore cravelhame\textunderscore .
\section{Cravelhame}
\begin{itemize}
\item {Grp. gram.:m.}
\end{itemize}
A parte do instrumento de corda, onde estão as cravelhas.
Conjunto dessas cravelhas.
\section{Cravelho}
\begin{itemize}
\item {fónica:vê}
\end{itemize}
\begin{itemize}
\item {Grp. gram.:m.}
\end{itemize}
Peça grosseira de madeira, com que se fecham cancellas e algumas portas, postigos, etc.
(Cp. \textunderscore cravelha\textunderscore )
\section{Cravelina}
\begin{itemize}
\item {Grp. gram.:f.}
\end{itemize}
\begin{itemize}
\item {Proveniência:(Do cast. \textunderscore clavel\textunderscore )}
\end{itemize}
(V.cravina)
\section{Cravete}
\begin{itemize}
\item {fónica:vê}
\end{itemize}
\begin{itemize}
\item {Grp. gram.:m.}
\end{itemize}
\begin{itemize}
\item {Proveniência:(De \textunderscore cravo\textunderscore )}
\end{itemize}
Cada uma das pontas metállicas da fivella.
\section{Cravija}
\begin{itemize}
\item {Grp. gram.:f.}
\end{itemize}
Barra de ferro, que une a lança com os varaes do carro.
Barra, que fixa o carro no eixo deanteiro e lhe facilita o movimento para os lados.
(Cast. \textunderscore clavija\textunderscore )
\section{Cravina}
\begin{itemize}
\item {Grp. gram.:f.}
\end{itemize}
Pequeno cravo.
Nome de algumas variedades de cravo.
\section{Cravinar}
\begin{itemize}
\item {Grp. gram.:v. i.}
\end{itemize}
Tocar cravo. Cf. Filinto, XIII,158.
\section{Cravinoso}
\begin{itemize}
\item {Grp. gram.:adj.}
\end{itemize}
Que tem fórma de cravo ou de cravina.
\section{Craviorganista}
\begin{itemize}
\item {Grp. gram.:m.}
\end{itemize}
Tangedor de craviórgão.
\section{Craviórgão}
\begin{itemize}
\item {Grp. gram.:m.}
\end{itemize}
\begin{itemize}
\item {Proveniência:(De \textunderscore cravo\textunderscore  + \textunderscore órgão\textunderscore )}
\end{itemize}
Antigo instrumento músico.
\section{Cravineiro}
\begin{itemize}
\item {Grp. gram.:m.}
\end{itemize}
\begin{itemize}
\item {Utilização:Des.}
\end{itemize}
Aquelle que cravina. Cf. Filinto, XI, 105.
\section{Cravinho}
\begin{itemize}
\item {Grp. gram.:m.}
\end{itemize}
\begin{itemize}
\item {Proveniência:(De \textunderscore cravo\textunderscore )}
\end{itemize}
Pequeno cravo.
Nome de várias plantas, especialmente da que também se chama \textunderscore cravo da Índia\textunderscore  ou \textunderscore cravo de cabecinha\textunderscore , que é planta condimentosa.
Variedade de prego pequeno.
\section{Cravinho-do-mato}
\begin{itemize}
\item {Grp. gram.:m.}
\end{itemize}
(V.erva-formigueira)
\section{Cravinote}
\begin{itemize}
\item {Grp. gram.:m.}
\end{itemize}
\begin{itemize}
\item {Utilização:Bras. do N}
\end{itemize}
Cravina pequena.
\section{Cravista}
\begin{itemize}
\item {Grp. gram.:m.  e  f.}
\end{itemize}
Pessôa que toca cravo.
Official de pregaria, que fabríca cravos.
\section{Cravo}
\begin{itemize}
\item {Grp. gram.:m.}
\end{itemize}
\begin{itemize}
\item {Utilização:Gír. de gatuno lisboêta.}
\end{itemize}
\begin{itemize}
\item {Proveniência:(Do lat. \textunderscore clavus\textunderscore )}
\end{itemize}
Espécie de prego, para fixar ferradura.
Prego, com que se fixavam na cruz as mãos e os pés dos suppliciados.
Pequeno tumor ou verruga na pelle.
Tumor, junto ao casco dos cavallos.
Instrumento músico de cordas e teclado.
Flôr de craveiro.
Craveira.
Nome de várias plantas.
Carvoeiro.
\section{Cravoária}
\begin{itemize}
\item {Grp. gram.:f.}
\end{itemize}
\begin{itemize}
\item {Proveniência:(De \textunderscore cravo\textunderscore )}
\end{itemize}
O mesmo que \textunderscore cravinho\textunderscore  ou \textunderscore cravo da Índia\textunderscore .
Árvore myrtácea, que produz o cravo da Índia.
\section{Cravo-da-Índia}
\begin{itemize}
\item {Grp. gram.:m.}
\end{itemize}
Planta aromática, de applicação culinária.
\section{Cravo-de-cabecinha}
\begin{itemize}
\item {Grp. gram.:m.}
\end{itemize}
O mesmo que \textunderscore cravo-da-Índia\textunderscore .
\section{Cravo-de-defuntos}
\begin{itemize}
\item {Grp. gram.:m.}
\end{itemize}
Planta annual, (\textunderscore tagetes patula\textunderscore ), o mesmo que \textunderscore rosa-de-oiro\textunderscore .
\section{Cravo-de-poéta}
\begin{itemize}
\item {Grp. gram.:m.}
\end{itemize}
\begin{itemize}
\item {Utilização:Bras}
\end{itemize}
Planta vivaz.
\section{Cravo-do-maranhão}
\begin{itemize}
\item {Grp. gram.:m.}
\end{itemize}
O mesmo que \textunderscore pau-cravo\textunderscore .
\section{Cravo-do-monte}
\begin{itemize}
\item {Grp. gram.:m.}
\end{itemize}
\begin{itemize}
\item {Utilização:Prov.}
\end{itemize}
\begin{itemize}
\item {Utilização:minh.}
\end{itemize}
Planta venenosa, (\textunderscore anthericum planifolium\textunderscore ).
\section{Cravoila}
\begin{itemize}
\item {Grp. gram.:f.}
\end{itemize}
Planta medicinal, (\textunderscore caryophillata vulgaris\textunderscore ), também conhecida por \textunderscore erva-benta\textunderscore . Cf. \textunderscore Pharm. Port.\textunderscore 
\section{Cré}
\begin{itemize}
\item {Grp. gram.:m.}
\end{itemize}
\begin{itemize}
\item {Proveniência:(Lat. \textunderscore creta\textunderscore )}
\end{itemize}
O mesmo que \textunderscore greda\textunderscore  branca.
\section{Cré}
\begin{itemize}
\item {Grp. gram.:m.}
\end{itemize}
Sýllaba, us. na loc. fam.--\textunderscore cré com cré, lé com lé\textunderscore ,--como se disséssemos: \textunderscore cada qual com os da sua igualha\textunderscore .
(Contr. de \textunderscore crelgo com crelgo, leigo com leigo\textunderscore ?)
\section{Crear}
\textunderscore v. t.\textunderscore  (e der.)
O mesmo que \textunderscore criar\textunderscore , etc.
\section{Creatina}
\begin{itemize}
\item {Grp. gram.:f.}
\end{itemize}
\begin{itemize}
\item {Proveniência:(Do gr. \textunderscore kreas\textunderscore )}
\end{itemize}
Alcaloide animal, inodoro, insípido, crystallizável.
\section{Creatinina}
\begin{itemize}
\item {Grp. gram.:f.}
\end{itemize}
\begin{itemize}
\item {Utilização:Chím.}
\end{itemize}
\begin{itemize}
\item {Proveniência:(De \textunderscore creatina\textunderscore )}
\end{itemize}
Princípio immediato que, com a creatina, existe nos músculos, no sangue e na urina, na qual se transforma a creatina, quando esta perde dois átomos de água.
\section{Crebar}
\begin{itemize}
\item {Grp. gram.:v. t.  e  i.}
\end{itemize}
\begin{itemize}
\item {Utilização:Prov.}
\end{itemize}
\begin{itemize}
\item {Utilização:minh.}
\end{itemize}
\begin{itemize}
\item {Proveniência:(Lat. \textunderscore crepare\textunderscore )}
\end{itemize}
O mesmo que \textunderscore quebrar\textunderscore .
\section{Crebro}
\begin{itemize}
\item {Grp. gram.:adj.}
\end{itemize}
\begin{itemize}
\item {Proveniência:(Do lat. \textunderscore creber\textunderscore )}
\end{itemize}
Frequente, amiudado.
\section{Crecer}
\textunderscore v. i.\textunderscore  (e der.)
O mesmo que \textunderscore crescer\textunderscore , etc.
\section{Creche}
\begin{itemize}
\item {Grp. gram.:f.}
\end{itemize}
\begin{itemize}
\item {Proveniência:(Fr. \textunderscore crèche\textunderscore )}
\end{itemize}
Asylo diurno, para crianças pobres.--Póde substituir-se por \textunderscore presepe\textunderscore , a exemplo de A. Herculano.
\section{Crécito}
\begin{itemize}
\item {Grp. gram.:m.}
\end{itemize}
\begin{itemize}
\item {Utilização:Ant.}
\end{itemize}
\begin{itemize}
\item {Proveniência:(De \textunderscore crecer\textunderscore )}
\end{itemize}
Accrescentamento, aumento. Cf. \textunderscore Viriato Trág.\textunderscore , V, 41.
\section{Creçudo}
\begin{itemize}
\item {Grp. gram.:adj.}
\end{itemize}
\begin{itemize}
\item {Utilização:Ant.}
\end{itemize}
Crescido:«\textunderscore e o trigo era creçudo.\textunderscore »G. Vicente, \textunderscore Juiz da Beira\textunderscore .
(Part. irr. de \textunderscore crecer\textunderscore )
\section{Credença}
\begin{itemize}
\item {Grp. gram.:f.}
\end{itemize}
\begin{itemize}
\item {Utilização:Ant.}
\end{itemize}
O mesmo que \textunderscore crença\textunderscore .
\section{Credência}
\begin{itemize}
\item {Grp. gram.:f.}
\end{itemize}
\begin{itemize}
\item {Proveniência:(Do it. \textunderscore credenza\textunderscore )}
\end{itemize}
Espécie de aparador, ou pequena mesa, em que se collocam as galhetas e outros utensilios da Missa e officios divinos, junto do altar.
Nicho de madeira ou pedra, com mesa para escrever, nos corredores de alguns conventos.
Mesa, em que, nas antigas basílicas, se recebiam as offertas dos fiéis.
\section{Credencial}
\begin{itemize}
\item {Grp. gram.:adj.}
\end{itemize}
\begin{itemize}
\item {Grp. gram.:F. pl.}
\end{itemize}
\begin{itemize}
\item {Proveniência:(Do rad. do lat. \textunderscore credere\textunderscore )}
\end{itemize}
Que dá crédito ou poderes para representar um país perante o Govêrno de outro.
Carta official, que um Ministro ou Embaixador apresenta ao chefe de um Estado, para mostrar que representa o seu país.
\section{Credenciário}
\begin{itemize}
\item {Grp. gram.:m.}
\end{itemize}
\begin{itemize}
\item {Proveniência:(De \textunderscore credencia\textunderscore )}
\end{itemize}
Aquelle que trata da credência ou do altar-mór.
\section{Credibilidade}
\begin{itemize}
\item {Grp. gram.:f.}
\end{itemize}
\begin{itemize}
\item {Proveniência:(Do lat. \textunderscore credibilis\textunderscore )}
\end{itemize}
Qualidade daquillo que é crível.
\section{Creditar}
\begin{itemize}
\item {Grp. gram.:v. t.}
\end{itemize}
\begin{itemize}
\item {Proveniência:(De \textunderscore crédito\textunderscore )}
\end{itemize}
Inscrever como crèdor.
\section{Creditício}
\begin{itemize}
\item {Grp. gram.:adj.}
\end{itemize}
\begin{itemize}
\item {Utilização:Neol.}
\end{itemize}
\begin{itemize}
\item {Proveniência:(De \textunderscore crédito\textunderscore )}
\end{itemize}
Relativo ao crédito público.
\section{Crédito}
\begin{itemize}
\item {Grp. gram.:m.}
\end{itemize}
\begin{itemize}
\item {Proveniência:(Lat. \textunderscore creditus\textunderscore )}
\end{itemize}
Crença, que inspiram as bôas qualidades de alguém.
Bôa reputação.
Confiança em que alguém solverá os seus débitos.
Autoridade: \textunderscore Herculano é escritor de grande crédito\textunderscore .
Aquillo que, nas suas contas, o commerciante há de haver.
Facilidade de adquirir dinheiro por empréstimo.
Direito de receber o que se emprestou.
Quantia, a que corresponde êste direito.
Autorização para despesas, dada pelo Parlamento ao Govêrno.
\section{Creditório}
\begin{itemize}
\item {Grp. gram.:adj.}
\end{itemize}
\begin{itemize}
\item {Utilização:Jur.}
\end{itemize}
Relativo a crédito.
\section{Credível}
\begin{itemize}
\item {Grp. gram.:adj.}
\end{itemize}
\begin{itemize}
\item {Utilização:P. us.}
\end{itemize}
\begin{itemize}
\item {Proveniência:(Lat. \textunderscore credibilis\textunderscore )}
\end{itemize}
O mesmo que \textunderscore crível\textunderscore .
\section{Credo}
\begin{itemize}
\item {Grp. gram.:m.}
\end{itemize}
\begin{itemize}
\item {Proveniência:(Lat. \textunderscore credo\textunderscore , 1.^a pess. do ind. pres. de \textunderscore credere\textunderscore )}
\end{itemize}
Oração christan, que, em latim, começa pela palavra \textunderscore credo\textunderscore , (creio).
Profissão de fé.
Regra.
Programma de um partido.
Espaço de tempo, que se gasta em rezar um credo:«\textunderscore ainda não há dois credos, que eu estava á porta\textunderscore ». Campos Júnior, \textunderscore Camões\textunderscore .«\textunderscore Obra de um credo\textunderscore ». G. Resende, \textunderscore Miscell. Interj.\textunderscore 
(indicativa de espanto)
\section{Crèdor}
\begin{itemize}
\item {Grp. gram.:m.}
\end{itemize}
\begin{itemize}
\item {Proveniência:(Lat. \textunderscore creditor\textunderscore )}
\end{itemize}
Indivíduo, ou pessôa moral, a quem se deve algum dinheiro, considerados em relação á dívida e ao devedor.
Aquelle que tem direito a compensação útil, a considerações, etc.: \textunderscore o Sr. Lara é crèdor da minha estima\textunderscore .
\section{Credulamente}
\begin{itemize}
\item {Grp. gram.:adv.}
\end{itemize}
De modo crédulo.
Com credulidade.
\section{Credulidade}
\begin{itemize}
\item {Grp. gram.:f.}
\end{itemize}
\begin{itemize}
\item {Proveniência:(Lat. \textunderscore credulitas\textunderscore )}
\end{itemize}
Qualidade de quem é crédulo.
\section{Crédulo}
\begin{itemize}
\item {Grp. gram.:adj.}
\end{itemize}
\begin{itemize}
\item {Grp. gram.:M.}
\end{itemize}
\begin{itemize}
\item {Proveniência:(Lat. \textunderscore credulus\textunderscore )}
\end{itemize}
Que crê facilmente; ingênuo.
Pessôa ingênua.
\section{Crega}
\begin{itemize}
\item {Grp. gram.:f.}
\end{itemize}
\begin{itemize}
\item {Utilização:Prov.}
\end{itemize}
\begin{itemize}
\item {Utilização:minh.}
\end{itemize}
Filha de clérigo.
(Contr. de \textunderscore créliga\textunderscore , fem. de \textunderscore créligo\textunderscore )
\section{Crela}
\begin{itemize}
\item {Grp. gram.:f.}
\end{itemize}
Fórma pop. e ant. de \textunderscore querela\textunderscore :«\textunderscore entre as rumas de feitos, entre as crelas, te esquecerás das musas\textunderscore ». Filinto, V, 4.
\section{Crelgo}
\begin{itemize}
\item {Grp. gram.:m.}
\end{itemize}
O mesmo que \textunderscore clérigo\textunderscore .
\section{Créligo}
\begin{itemize}
\item {Grp. gram.:m.}
\end{itemize}
\begin{itemize}
\item {Utilização:Ant.}
\end{itemize}
O mesmo que \textunderscore clérigo\textunderscore .
\section{Cremação}
\begin{itemize}
\item {Grp. gram.:f.}
\end{itemize}
\begin{itemize}
\item {Proveniência:(Lat. \textunderscore crematio\textunderscore )}
\end{itemize}
Acto de queimar cadáveres.
\section{Cremadeiro}
\begin{itemize}
\item {Grp. gram.:m.}
\end{itemize}
\begin{itemize}
\item {Proveniência:(Do lat. \textunderscore cremare\textunderscore )}
\end{itemize}
Fogueira, em que se queimavam as viúvas na Índia.
Fogueira.
\section{Cremado}
\begin{itemize}
\item {Grp. gram.:adj.}
\end{itemize}
Que tem côr de creme. Cf. \textunderscore Inquér. Industr.\textunderscore , p. I, 88.
\section{Cremador}
\begin{itemize}
\item {Grp. gram.:m.  e  adj.}
\end{itemize}
\begin{itemize}
\item {Proveniência:(Lat. \textunderscore cremator\textunderscore )}
\end{itemize}
Aquelle que queima, que destrói.
\section{Cremalheira}
\begin{itemize}
\item {Grp. gram.:f.}
\end{itemize}
\begin{itemize}
\item {Proveniência:(Fr. \textunderscore crémaillière\textunderscore )}
\end{itemize}
O mesmo que \textunderscore gramalheira\textunderscore .
\section{Cremar}
\begin{itemize}
\item {Grp. gram.:v. t.}
\end{itemize}
\begin{itemize}
\item {Utilização:Neol.}
\end{itemize}
\begin{itemize}
\item {Proveniência:(Lat. \textunderscore cremare\textunderscore )}
\end{itemize}
Incinerar (cadáveres).
\section{Cremáster}
\begin{itemize}
\item {Grp. gram.:m.}
\end{itemize}
\begin{itemize}
\item {Utilização:Anat.}
\end{itemize}
\begin{itemize}
\item {Proveniência:(Gr. \textunderscore kremaster\textunderscore )}
\end{itemize}
Músculo do testículo.
\section{Cremastro}
\begin{itemize}
\item {Grp. gram.:m.}
\end{itemize}
\begin{itemize}
\item {Proveniência:(Gr. \textunderscore kremastra\textunderscore )}
\end{itemize}
Variedade de orchídea.
\section{Crematista}
\begin{itemize}
\item {Grp. gram.:m.}
\end{itemize}
Partidário ou defensor da incineração dos cadáveres.
(Cp. lat. \textunderscore cremare\textunderscore )
\section{Crematística}
\begin{itemize}
\item {Grp. gram.:f.}
\end{itemize}
\begin{itemize}
\item {Proveniência:(Gr. \textunderscore krematistike\textunderscore )}
\end{itemize}
Arte de produzir riqueza.
Tratado das riquezas.
\section{Crematístico}
\begin{itemize}
\item {Grp. gram.:adj.}
\end{itemize}
Relativo á crematística.
\section{Crematologia}
\begin{itemize}
\item {Grp. gram.:f.}
\end{itemize}
\begin{itemize}
\item {Proveniência:(Do gr. \textunderscore khrema\textunderscore  + \textunderscore logos\textunderscore )}
\end{itemize}
Tratado da riqueza.
\section{Crematológico}
\begin{itemize}
\item {Grp. gram.:adj.}
\end{itemize}
Relativo á crematologia.
\section{Crematonomia}
\begin{itemize}
\item {Grp. gram.:f.}
\end{itemize}
\begin{itemize}
\item {Proveniência:(Do gr. \textunderscore khrema\textunderscore  + \textunderscore nomos\textunderscore )}
\end{itemize}
Conjunto das leis naturaes, que regulam a producção e repartição da riqueza.
\section{Crematonómico}
\begin{itemize}
\item {Grp. gram.:adj.}
\end{itemize}
Relativo á chrematonomia.
\section{Crematório}
\begin{itemize}
\item {Grp. gram.:adj.}
\end{itemize}
\begin{itemize}
\item {Proveniência:(Do lat. \textunderscore crematus\textunderscore )}
\end{itemize}
Diz-se dos fornos, em que se queimam cadáveres.
\section{Creme}
\begin{itemize}
\item {Grp. gram.:m.}
\end{itemize}
\begin{itemize}
\item {Utilização:Fig.}
\end{itemize}
\begin{itemize}
\item {Proveniência:(Lat. \textunderscore cremum\textunderscore )}
\end{itemize}
Substância untuosa e amarelada, que se fórma no leite, e de que se extrai a manteiga.
\textunderscore Leite creme\textunderscore , iguaria doce, feita de leite, farinha, ovos e açúcar.
Espécie de licor espêsso.
Côr amarelada, como a do leite creme.
Aquillo que há de melhor, a nata, o escol.
\section{Cremnóbata}
\begin{itemize}
\item {Grp. gram.:m.}
\end{itemize}
\begin{itemize}
\item {Proveniência:(Gr. \textunderscore kremnobates\textunderscore )}
\end{itemize}
Acróbata, funâmbulo.
\section{Cremnometria}
\begin{itemize}
\item {Grp. gram.:f.}
\end{itemize}
\begin{itemize}
\item {Utilização:Chím.}
\end{itemize}
\begin{itemize}
\item {Proveniência:(Do gr. \textunderscore kremnao\textunderscore  + \textunderscore metron\textunderscore )}
\end{itemize}
Avaliação, da quantidade de um precipitado.
\section{Cremnométrico}
\begin{itemize}
\item {Grp. gram.:adj.}
\end{itemize}
Relativo a cremnometria.
\section{Cremnómetro}
\begin{itemize}
\item {Grp. gram.:m.}
\end{itemize}
\begin{itemize}
\item {Utilização:Bot.}
\end{itemize}
Apparelho para se avaliarem ou pesarem os resíduos dos filtros.
(Cp. \textunderscore cremnometria\textunderscore )
\section{Cremocarpo}
\begin{itemize}
\item {Grp. gram.:m.}
\end{itemize}
\begin{itemize}
\item {Proveniência:(Do gr. \textunderscore kremao\textunderscore  + \textunderscore karpos\textunderscore )}
\end{itemize}
Fruto, composto de dois ou mais achênios soldados, como o da borragem.
\section{Creadouro}
\begin{itemize}
\item {Grp. gram.:m.}
\end{itemize}
\begin{itemize}
\item {Grp. gram.:Adj.}
\end{itemize}
\begin{itemize}
\item {Proveniência:(De \textunderscore criar\textunderscore )}
\end{itemize}
Viveiro de plantas.
Susceptível de medrança, de se criar bem.
\section{Cremómetro}
\begin{itemize}
\item {Grp. gram.:m.}
\end{itemize}
\begin{itemize}
\item {Proveniência:(Do lat. \textunderscore cremum\textunderscore  + gr. \textunderscore metron\textunderscore )}
\end{itemize}
Pequeno instrumento, para determinar a proporção da substância gorda contida no leite.
\section{Cremona}
\begin{itemize}
\item {Grp. gram.:f.}
\end{itemize}
Rabeca, fabricada em Cremona.
\section{Cremor}
\begin{itemize}
\item {Grp. gram.:m.}
\end{itemize}
\begin{itemize}
\item {Proveniência:(Lat. \textunderscore cremor\textunderscore )}
\end{itemize}
Cozimento do suco de uma planta.
\textunderscore Cremor de tártaro\textunderscore , tartarato de potássio ou cal.
\section{Cremorização}
\begin{itemize}
\item {Grp. gram.:f.}
\end{itemize}
Acto de cremorizar.
\section{Cremorizar}
\begin{itemize}
\item {Grp. gram.:v. t.}
\end{itemize}
Deitar cremor em.
Misturar cremor com.
\section{Cremosa}
\begin{itemize}
\item {Grp. gram.:adj.}
\end{itemize}
Diz-se da estomatite, nos cavallos e nos bois.
\section{Crena}
\begin{itemize}
\item {Grp. gram.:f.}
\end{itemize}
\begin{itemize}
\item {Utilização:Ant.}
\end{itemize}
O mesmo que \textunderscore querena\textunderscore . Cf. \textunderscore Anat. Joc.\textunderscore , I, 21.
\section{Crenado}
\begin{itemize}
\item {Grp. gram.:adj.}
\end{itemize}
Que tem crenas.
\section{Crenas}
\begin{itemize}
\item {Grp. gram.:f. pl.}
\end{itemize}
Dentes das fôlhas vegetaes.
(Cp. fr. \textunderscore creneau\textunderscore )
\section{Crenato}
\begin{itemize}
\item {Grp. gram.:m.}
\end{itemize}
\begin{itemize}
\item {Proveniência:(Do gr. \textunderscore krene\textunderscore )}
\end{itemize}
Sal, resultante da combinação do ácido crénico com uma base.
\section{Crenátula}
\begin{itemize}
\item {Grp. gram.:f.}
\end{itemize}
\begin{itemize}
\item {Proveniência:(Do rad. de \textunderscore crenas\textunderscore )}
\end{itemize}
Concha bivalve.
\section{Crença}
\begin{itemize}
\item {Grp. gram.:f.}
\end{itemize}
Acto ou effeito de crêr.
Convicção.
Fé, especialmente a fé religiosa.
Crédito diplomático, ou qualidade de quem apresenta as suas credenciaes, em país estranho:«\textunderscore presentou ao Pontífice as cartas..., nem ellas continham mais que a crença que se pedia para os embaixadores\textunderscore ». Filinto, \textunderscore D. Man.\textunderscore , III, 57.
\section{Crenchas}
\begin{itemize}
\item {Grp. gram.:f. pl.}
\end{itemize}
\begin{itemize}
\item {Utilização:Ant.}
\end{itemize}
\begin{itemize}
\item {Proveniência:(Do b. lat. \textunderscore crinicula\textunderscore , do lat. \textunderscore crinis\textunderscore )}
\end{itemize}
Tranças de cabello.
\section{Crencho}
\begin{itemize}
\item {Grp. gram.:adj.}
\end{itemize}
\begin{itemize}
\item {Utilização:Prov.}
\end{itemize}
\begin{itemize}
\item {Utilização:trasm.}
\end{itemize}
Enfatuado; empertigado; cheio de si.
\section{Crendeirice}
\begin{itemize}
\item {Grp. gram.:f.}
\end{itemize}
Qualidade de crendeiro; crendice.
\section{Crendeiro}
\begin{itemize}
\item {Grp. gram.:m.  e  adj.}
\end{itemize}
\begin{itemize}
\item {Proveniência:(De \textunderscore crente\textunderscore )}
\end{itemize}
O que crê em absurdos ou abusões ridículas.
Simplório.
\section{Crendice}
\begin{itemize}
\item {Grp. gram.:f.}
\end{itemize}
Crença absurda ou ridícula.
(Cp. \textunderscore crendeiro\textunderscore )
\section{Crênico}
\begin{itemize}
\item {Grp. gram.:adj.}
\end{itemize}
\begin{itemize}
\item {Proveniência:(Do gr. \textunderscore krene\textunderscore )}
\end{itemize}
Diz-se de um ácido, que se encontra em águas mineraes.
\section{Crenífero}
\begin{itemize}
\item {Grp. gram.:adj.}
\end{itemize}
O mesmo que \textunderscore crenulado\textunderscore .
\section{Crenirostro}
\begin{itemize}
\item {fónica:rós}
\end{itemize}
\begin{itemize}
\item {Grp. gram.:adj.}
\end{itemize}
\begin{itemize}
\item {Utilização:Zool.}
\end{itemize}
Que tem bico crenulado.
\section{Crenirrostro}
\begin{itemize}
\item {Grp. gram.:adj.}
\end{itemize}
\begin{itemize}
\item {Utilização:Zool.}
\end{itemize}
Que tem bico crenulado.
\section{Crente}
\begin{itemize}
\item {Grp. gram.:adj.}
\end{itemize}
\begin{itemize}
\item {Grp. gram.:M.}
\end{itemize}
\begin{itemize}
\item {Proveniência:(Lat. \textunderscore credens\textunderscore )}
\end{itemize}
Que crê.
Sectário de uma religião.
\section{Crenulado}
\begin{itemize}
\item {Grp. gram.:adj.}
\end{itemize}
Que tem crênulas.
\section{Crênulas}
\begin{itemize}
\item {Grp. gram.:f. pl.}
\end{itemize}
(dem. de \textunderscore crenas\textunderscore )
\section{Creóbio}
\begin{itemize}
\item {Grp. gram.:m.}
\end{itemize}
\begin{itemize}
\item {Proveniência:(Do gr. \textunderscore kreas\textunderscore  + \textunderscore bios\textunderscore )}
\end{itemize}
Insecto coleóptero pentâmero.
\section{Creofagia}
\begin{itemize}
\item {Grp. gram.:f.}
\end{itemize}
\begin{itemize}
\item {Proveniência:(Do gr. \textunderscore kreas\textunderscore  + \textunderscore phagein\textunderscore )}
\end{itemize}
Acto ou hábito de se alimentar de carne.
\section{Creófago}
\begin{itemize}
\item {Grp. gram.:m.  e  adj.}
\end{itemize}
O mesmo que \textunderscore carnívoro\textunderscore .
\section{Creófilo}
\begin{itemize}
\item {Grp. gram.:adj.}
\end{itemize}
\begin{itemize}
\item {Utilização:Zool.}
\end{itemize}
\begin{itemize}
\item {Proveniência:(Do gr. \textunderscore kreas\textunderscore  + \textunderscore philos\textunderscore )}
\end{itemize}
Que gosta de carne, (falando-se de insectos dípteros).
\section{Creogenia}
\begin{itemize}
\item {Grp. gram.:f.}
\end{itemize}
\begin{itemize}
\item {Utilização:Physiol.}
\end{itemize}
\begin{itemize}
\item {Proveniência:(Do gr. \textunderscore kreas\textunderscore  + \textunderscore genos\textunderscore )}
\end{itemize}
Producção da carne nos corpos animados.
\section{Creogênico}
\begin{itemize}
\item {Grp. gram.:adj.}
\end{itemize}
Relativo á creogenia.
\section{Creografia}
\begin{itemize}
\item {Grp. gram.:f.}
\end{itemize}
\begin{itemize}
\item {Proveniência:(Do gr. \textunderscore kreas\textunderscore  + \textunderscore graphein\textunderscore )}
\end{itemize}
Descripção das carnes ou da parte mole dos corpos.
\section{Creographia}
\begin{itemize}
\item {Grp. gram.:f.}
\end{itemize}
\begin{itemize}
\item {Proveniência:(Do gr. \textunderscore kreas\textunderscore  + \textunderscore graphein\textunderscore )}
\end{itemize}
Descripção das carnes ou da parte molle dos corpos.
\section{Creoli}
\begin{itemize}
\item {Grp. gram.:m.}
\end{itemize}
\begin{itemize}
\item {Utilização:Bras}
\end{itemize}
Fruto do creolizeiro.
\section{Creolina}
\begin{itemize}
\item {Grp. gram.:f.}
\end{itemize}
Substância antiséptica, extrahida do alcatrão de hulha.
\section{Creolizeiro}
\begin{itemize}
\item {Grp. gram.:m.}
\end{itemize}
\begin{itemize}
\item {Proveniência:(De \textunderscore creoli\textunderscore )}
\end{itemize}
Planta brasileira.
\section{Creophagia}
\begin{itemize}
\item {Grp. gram.:f.}
\end{itemize}
\begin{itemize}
\item {Proveniência:(Do gr. \textunderscore kreas\textunderscore  + \textunderscore phagein\textunderscore )}
\end{itemize}
Acto ou hábito de se alimentar de carne.
\section{Creóphago}
\begin{itemize}
\item {Grp. gram.:m.  e  adj.}
\end{itemize}
O mesmo que \textunderscore carnívoro\textunderscore .
\section{Creóphilo}
\begin{itemize}
\item {Grp. gram.:adj.}
\end{itemize}
\begin{itemize}
\item {Utilização:Zool.}
\end{itemize}
\begin{itemize}
\item {Proveniência:(Do gr. \textunderscore kreas\textunderscore  + \textunderscore philos\textunderscore )}
\end{itemize}
Que gosta de carne, (falando-se de insectos dípteros).
\section{Creosota}
\begin{itemize}
\item {Grp. gram.:f.}
\end{itemize}
O mesmo que \textunderscore creosote\textunderscore .
\section{Creosotagem}
\begin{itemize}
\item {Grp. gram.:f.}
\end{itemize}
Acto de creosotar.
\section{Creosotal}
\begin{itemize}
\item {Grp. gram.:m.}
\end{itemize}
\begin{itemize}
\item {Proveniência:(De \textunderscore creosote\textunderscore )}
\end{itemize}
Carbonato de creosote, applicado contra a tísica.
\section{Creosotar}
\begin{itemize}
\item {Grp. gram.:v. t.}
\end{itemize}
Embeber ou infiltrar creosote em.
\section{Creosote}
\begin{itemize}
\item {Grp. gram.:m.}
\end{itemize}
\begin{itemize}
\item {Proveniência:(Do gr. \textunderscore kreias\textunderscore  + \textunderscore soto\textunderscore )}
\end{itemize}
Substância cáustica, extrahida do alcatrão e própria para conservar substâncias orgânicas.
\section{Creoulo}
\begin{itemize}
\item {Grp. gram.:m.  e  adj.}
\end{itemize}
(V.crioulo)
\section{Crepe}
\begin{itemize}
\item {Grp. gram.:m.}
\end{itemize}
\begin{itemize}
\item {Proveniência:(Fr. \textunderscore crêpe\textunderscore )}
\end{itemize}
Tecido transparente.
Fita ou tecido negro, que se usa em sinal de luto.
Luto.
\section{Crépida}
\begin{itemize}
\item {Grp. gram.:f.}
\end{itemize}
\begin{itemize}
\item {Proveniência:(Lat. \textunderscore crepida\textunderscore )}
\end{itemize}
Calçadura ferrada, que não cobria todo o pé, usada entre os antigos Romanos.
\section{Crépido}
\begin{itemize}
\item {Grp. gram.:adj.}
\end{itemize}
Crespo; encarapinhado. Cf. Rebello, \textunderscore Mocidade\textunderscore , II, 186.
(Cp. fr. \textunderscore crépus\textunderscore )
\section{Crepidópodes}
\begin{itemize}
\item {Grp. gram.:m. pl.}
\end{itemize}
\begin{itemize}
\item {Proveniência:(Do gr. \textunderscore krepis\textunderscore  + \textunderscore pous\textunderscore )}
\end{itemize}
Nome de uma ordem de molluscos.
\section{Crepins}
\begin{itemize}
\item {Grp. gram.:m. pl.}
\end{itemize}
O mesmo que \textunderscore carpins\textunderscore .
\section{Crepitação}
\begin{itemize}
\item {Grp. gram.:f.}
\end{itemize}
\begin{itemize}
\item {Proveniência:(Lat. \textunderscore crepitatio\textunderscore )}
\end{itemize}
Acto ou effeito de crepitar.
\section{Crepitáculo}
\begin{itemize}
\item {Grp. gram.:m.}
\end{itemize}
\begin{itemize}
\item {Utilização:Bot.}
\end{itemize}
\begin{itemize}
\item {Proveniência:(Lat. \textunderscore crepitaculum\textunderscore )}
\end{itemize}
Instrumento antigo; sistro.
Fruto, que se abre com ruído.
\section{Crepitante}
\begin{itemize}
\item {Grp. gram.:adj.}
\end{itemize}
\begin{itemize}
\item {Proveniência:(Lat. \textunderscore crepitans\textunderscore )}
\end{itemize}
Que crepita.
\section{Crepitar}
\begin{itemize}
\item {Grp. gram.:v. i.}
\end{itemize}
\begin{itemize}
\item {Proveniência:(Lat. \textunderscore crepitare\textunderscore )}
\end{itemize}
Estalar, como as faíscas que resaltam da madeira incendiada, ou como o sal que se deita no fogo.
\section{Crepitoso}
\begin{itemize}
\item {Grp. gram.:adj.}
\end{itemize}
O mesmo que \textunderscore crepitante\textunderscore .
\section{Crepom}
\begin{itemize}
\item {Grp. gram.:m.}
\end{itemize}
(V.crespão)
\section{Crepuscular}
\begin{itemize}
\item {Grp. gram.:adj.}
\end{itemize}
Relativo ao crepúsculo: \textunderscore á hora crepuscular\textunderscore .
\section{Crepusculários}
\begin{itemize}
\item {Grp. gram.:m. pl.}
\end{itemize}
Insectos, que apparecem á hora do crepúsculo.
\section{Crepusculino}
\begin{itemize}
\item {Grp. gram.:adj.}
\end{itemize}
O mesmo que \textunderscore crepuscular\textunderscore .
\section{Crepúsculo}
\begin{itemize}
\item {Grp. gram.:m.}
\end{itemize}
\begin{itemize}
\item {Utilização:Ext.}
\end{itemize}
\begin{itemize}
\item {Utilização:Fig.}
\end{itemize}
\begin{itemize}
\item {Grp. gram.:Adj.}
\end{itemize}
\begin{itemize}
\item {Proveniência:(Lat. \textunderscore crepusculum\textunderscore )}
\end{itemize}
Claridade froixa, que fica depois do sol-pôsto.
A primeira claridade do dia, antes do sol nado.
Decadência; occaso: \textunderscore no crepúsculo da vida\textunderscore .
O mesmo que \textunderscore crepuscular\textunderscore . Cf. Filinto, VIII, 243.
\section{Crêr}
\begin{itemize}
\item {Grp. gram.:v. t.}
\end{itemize}
\begin{itemize}
\item {Grp. gram.:V. i.}
\end{itemize}
\begin{itemize}
\item {Proveniência:(Lat. \textunderscore credere\textunderscore )}
\end{itemize}
Têr como verdadeiro; dar crédito a, acreditar: \textunderscore crêr atoardas\textunderscore .
Julgar.
Suppor, presumir: \textunderscore estás enganado, creio eu\textunderscore .
Têr fé: \textunderscore crêr em milagres\textunderscore .
Têr confiança: \textunderscore creio em ti\textunderscore .
\section{Cresamina}
\begin{itemize}
\item {Grp. gram.:f.}
\end{itemize}
\begin{itemize}
\item {Utilização:Chím.}
\end{itemize}
Substância desinfectante.
\section{Crescença}
\begin{itemize}
\item {Grp. gram.:f.}
\end{itemize}
\begin{itemize}
\item {Utilização:Pop.}
\end{itemize}
\begin{itemize}
\item {Proveniência:(De \textunderscore crescer\textunderscore )}
\end{itemize}
Acto ou effeito de crescer; crescimento.
Accrescimento.
Aquillo que excede uma medida.
\section{Crescência}
\begin{itemize}
\item {Grp. gram.:f.}
\end{itemize}
Gênero de plantas gesneriáceas.
\section{Crescendo}
\begin{itemize}
\item {Grp. gram.:m.}
\end{itemize}
\begin{itemize}
\item {Proveniência:(It. \textunderscore crescendo\textunderscore )}
\end{itemize}
Aumento progressivo de sons, em música.
Progressão; gradação.
\section{Crescente}
\begin{itemize}
\item {Grp. gram.:adj.}
\end{itemize}
\begin{itemize}
\item {Grp. gram.:F.}
\end{itemize}
\begin{itemize}
\item {Grp. gram.:M.}
\end{itemize}
\begin{itemize}
\item {Utilização:Prov.}
\end{itemize}
\begin{itemize}
\item {Utilização:dur.}
\end{itemize}
\begin{itemize}
\item {Utilização:beir.}
\end{itemize}
\begin{itemize}
\item {Proveniência:(Lat. \textunderscore crescens\textunderscore )}
\end{itemize}
Que cresce.
Enchente.
Tempo, em que a Lua cresce apparentemente, desde o novilúnio até o plenilúnio.
Aquillo que tem fórma de meia Lua.
Bandeira turca.
Aquillo que sobeja.
Arco, maior que o semicircular, na architectura árabe.
Linha, em fórma de meia Lua, na palma da mão do cavallo, doente de formigo.
Fermento, que se deita na massa de farinha, para que esta levede.
Porção de cabello postiço, usado por senhoras, para complemento do penteado.
Pequeno chinó para homem, occupando uma pequena parte descabellada da cabeça.
\section{Crescer}
\begin{itemize}
\item {Grp. gram.:v. i.}
\end{itemize}
\begin{itemize}
\item {Proveniência:(Lat. \textunderscore crescere\textunderscore )}
\end{itemize}
Aumentar: \textunderscore começam agora a crescer os dias\textunderscore .
Desenvolver-se.
Melhorar.
Medrar: \textunderscore o rapaz cresce a olhos vistos\textunderscore .
Inchar.
Sobejar: \textunderscore do jantar, nunca lhe cresce nada\textunderscore .
Avançar para alguém com modo aggressivo: \textunderscore ao ouvir-lhe a ameaça, cresceu para êlle e esbofeteou-o\textunderscore .
\section{Crescido}
\begin{itemize}
\item {Grp. gram.:adj.}
\end{itemize}
\begin{itemize}
\item {Proveniência:(De \textunderscore crescer\textunderscore )}
\end{itemize}
Aumentado.
Desenvolvido.
Importante; grande.
\section{Crescidos}
\begin{itemize}
\item {Grp. gram.:m. pl.}
\end{itemize}
\begin{itemize}
\item {Proveniência:(De \textunderscore crescido\textunderscore )}
\end{itemize}
Malhas, com que se alargam as meias em certos pontos.
Sobejos.
\section{Crescidote}
\begin{itemize}
\item {Grp. gram.:adj.}
\end{itemize}
\begin{itemize}
\item {Utilização:Fam.}
\end{itemize}
Diz-se do rapaz crescido, ou um tanto crescido.
\section{Crescimento}
\begin{itemize}
\item {Grp. gram.:m.}
\end{itemize}
\begin{itemize}
\item {Utilização:Pop.}
\end{itemize}
Acto ou effeito de crescer.
Febre intermittente, (mais us. no pl.).
\section{Créscimo}
\begin{itemize}
\item {Grp. gram.:m.}
\end{itemize}
A parte, excedente; aquillo que sobeja; resíduos.
O mesmo que \textunderscore accréscimo\textunderscore  (de febre). Cf. Júl. Dinis, \textunderscore Pupillas\textunderscore , 46.
\section{Cresílico}
\begin{itemize}
\item {Grp. gram.:adj.}
\end{itemize}
\begin{itemize}
\item {Utilização:Chím.}
\end{itemize}
\begin{itemize}
\item {Proveniência:(Do rad. de \textunderscore cresol\textunderscore )}
\end{itemize}
Diz-se de um phenol, extrahido do creosote.
\section{Cresol}
\begin{itemize}
\item {Grp. gram.:m.}
\end{itemize}
\begin{itemize}
\item {Utilização:Chím.}
\end{itemize}
Phenol homólogo de hydrato phenilo.
\section{Cresolsaponato}
\begin{itemize}
\item {Grp. gram.:m.}
\end{itemize}
\begin{itemize}
\item {Utilização:Chím.}
\end{itemize}
Substância antiséptica.
\section{Crespa}
\begin{itemize}
\item {fónica:crês}
\end{itemize}
\begin{itemize}
\item {Grp. gram.:f.}
\end{itemize}
\begin{itemize}
\item {Utilização:Ant.}
\end{itemize}
O mesmo que \textunderscore crespidão\textunderscore .
Cf. Usque, \textunderscore Tribulações\textunderscore .
\section{Crespadinha}
\begin{itemize}
\item {Grp. gram.:f.}
\end{itemize}
\begin{itemize}
\item {Utilização:Prov.}
\end{itemize}
\begin{itemize}
\item {Utilização:alent.}
\end{itemize}
Variedade de alface.
\section{Crespão}
\begin{itemize}
\item {Grp. gram.:m.}
\end{itemize}
\begin{itemize}
\item {Proveniência:(De \textunderscore crespo\textunderscore . Cf. fr. \textunderscore crépon\textunderscore )}
\end{itemize}
Variedade de tecido crespo.
\section{Crespidão}
\begin{itemize}
\item {Grp. gram.:f.}
\end{itemize}
\begin{itemize}
\item {Proveniência:(Do lat. \textunderscore crispitudo\textunderscore )}
\end{itemize}
Qualidade daquillo que é crespo.
\section{Crespina}
\begin{itemize}
\item {Grp. gram.:f.}
\end{itemize}
\begin{itemize}
\item {Utilização:Ant.}
\end{itemize}
\begin{itemize}
\item {Proveniência:(De \textunderscore crespo\textunderscore )}
\end{itemize}
O segundo estômago dos ruminantes.
Rede ou coifa de recolher o cabello.
\section{Crespir}
\begin{itemize}
\item {Grp. gram.:v. t.}
\end{itemize}
(V.encrespar)
\section{Crespo}
\begin{itemize}
\item {fónica:crês}
\end{itemize}
\begin{itemize}
\item {Grp. gram.:adj.}
\end{itemize}
\begin{itemize}
\item {Proveniência:(Do lat. \textunderscore crispus\textunderscore )}
\end{itemize}
Que tem superfície áspera, rugosa.
Riçado.
Escabroso.
Erriçado.
Encapellado.
Ameaçador.
Rude.
\section{Crespoço}
\begin{itemize}
\item {fónica:pô}
\end{itemize}
\begin{itemize}
\item {Grp. gram.:m.}
\end{itemize}
\begin{itemize}
\item {Utilização:Prov.}
\end{itemize}
O mesmo que \textunderscore pescoço\textunderscore . (Colhido em Melgaço)
\section{Cresta}
\begin{itemize}
\item {Grp. gram.:f.}
\end{itemize}
Acto ou effeito de crestar^2.
\section{Crestadeira}
\begin{itemize}
\item {Grp. gram.:f.}
\end{itemize}
\begin{itemize}
\item {Proveniência:(De \textunderscore crestar\textunderscore ^2)}
\end{itemize}
Instrumento, com que se crestam as colmeias.
Utensílio culinário, com que se dá a côr de queimado ou tostado a certas iguarias.
\section{Crestadura}
\begin{itemize}
\item {Grp. gram.:f.}
\end{itemize}
\begin{itemize}
\item {Proveniência:(De \textunderscore crestar\textunderscore ^1)}
\end{itemize}
Acto de queimar levemente á superfície.
\section{Crestamento}
\begin{itemize}
\item {Grp. gram.:m.}
\end{itemize}
Acto ou effeito de crestar^1.
\section{Crestar}
\begin{itemize}
\item {Grp. gram.:v. t.}
\end{itemize}
\begin{itemize}
\item {Proveniência:(Do lat. \textunderscore crustare\textunderscore )}
\end{itemize}
Queimar superficialmente, levemente; tostar.
Tornar sêco, por effeito do calor ou frio: \textunderscore o sol crestou as hortênsias; a geada crestou o batatal\textunderscore .
Dar côr de queimado a.
Tornar trigueiro: \textunderscore crestou-me o clima da África\textunderscore .
\section{Crestar}
\begin{itemize}
\item {Grp. gram.:v. t.}
\end{itemize}
\begin{itemize}
\item {Utilização:Fig.}
\end{itemize}
\begin{itemize}
\item {Proveniência:(Do lat. \textunderscore castrare\textunderscore )}
\end{itemize}
Tirar o mel de (colmeias), tirando parte dos favos.
Saquear.
Desfalcar.
Reduzir a quantidade de.
\section{Crestello}
\begin{itemize}
\item {Grp. gram.:m.}
\end{itemize}
(Corr. de \textunderscore castrello\textunderscore )
\section{Crestelo}
\begin{itemize}
\item {fónica:tê}
\end{itemize}
\begin{itemize}
\item {Grp. gram.:m.}
\end{itemize}
(Corr. de \textunderscore castrelo\textunderscore )
\section{Cresto}
\begin{itemize}
\item {fónica:crês}
\end{itemize}
\begin{itemize}
\item {Grp. gram.:m.}
\end{itemize}
Chibo, que foi castrado aos oito dias de idade.
(Cp. \textunderscore castrar\textunderscore )
\section{Crestomatia}
\begin{itemize}
\item {Grp. gram.:f.}
\end{itemize}
\begin{itemize}
\item {Proveniência:(Gr. \textunderscore khrestomatheia\textunderscore )}
\end{itemize}
O mesmo que \textunderscore antologia\textunderscore .
\section{Cresýlico}
\begin{itemize}
\item {Grp. gram.:adj.}
\end{itemize}
\begin{itemize}
\item {Utilização:Chím.}
\end{itemize}
\begin{itemize}
\item {Proveniência:(Do rad. de \textunderscore cresol\textunderscore )}
\end{itemize}
Diz-se de um phenol, extrahido do creosote.
\section{Creta}
\begin{itemize}
\item {Grp. gram.:f.}
\end{itemize}
\begin{itemize}
\item {Proveniência:(Lat. \textunderscore creta\textunderscore )}
\end{itemize}
O ponto de partida dos carros, nos circos romanos.
\section{Cretáceo}
\begin{itemize}
\item {Grp. gram.:adj.}
\end{itemize}
\begin{itemize}
\item {Proveniência:(Lat. \textunderscore cretaceus\textunderscore )}
\end{itemize}
Relativo a greda.
\section{Cretaico}
\begin{itemize}
\item {Grp. gram.:adj.}
\end{itemize}
\begin{itemize}
\item {Utilização:Neol.}
\end{itemize}
\begin{itemize}
\item {Proveniência:(Do lat. \textunderscore creta\textunderscore )}
\end{itemize}
Diz-se do terreno, em que o gré branco representa parte importante.
\section{Cretinação}
\begin{itemize}
\item {Grp. gram.:f.}
\end{itemize}
Estado phýsico ou moral dos cretinos.
\section{Cretinismo}
\begin{itemize}
\item {Grp. gram.:m.}
\end{itemize}
A incapacidade ou moléstia do cretino.
\section{Cretinizar}
\begin{itemize}
\item {Grp. gram.:v. t.}
\end{itemize}
Tornar cretino.
\section{Cretino}
\begin{itemize}
\item {Grp. gram.:m.}
\end{itemize}
\begin{itemize}
\item {Proveniência:(Fr. \textunderscore cretin\textunderscore )}
\end{itemize}
Aquelle que, por deformidade orgânica, tem absoluta incapacidade moral.
Pacóvio, lorpa, idiota, imbecil.
\section{Cretinoso}
\begin{itemize}
\item {Grp. gram.:adj.}
\end{itemize}
Relativo a cretino.
Próprio de cretino.
\section{Creto}
\begin{itemize}
\item {Grp. gram.:m.}
\end{itemize}
\begin{itemize}
\item {Utilização:Pop.}
\end{itemize}
O mesmo que \textunderscore crédito\textunderscore :«\textunderscore ora um homem sempre se atriga, de casar com mulher de maus cretos\textunderscore ». Camillo, \textunderscore Brasileira\textunderscore , 290.
\section{Cretone}
\begin{itemize}
\item {Grp. gram.:m.}
\end{itemize}
\begin{itemize}
\item {Proveniência:(De \textunderscore Cretonne\textunderscore , n. p. de um fabricante normando)}
\end{itemize}
Espécie de pano forte e encorpado, tecido de linho, com urdidura de cânhamo.
\section{Creúdo}
\begin{itemize}
\item {Grp. gram.:adj.}
\end{itemize}
\begin{itemize}
\item {Utilização:Ant.}
\end{itemize}
\begin{itemize}
\item {Proveniência:(De \textunderscore crêr\textunderscore )}
\end{itemize}
O mesmo que \textunderscore crido\textunderscore .
\section{Creve}
\begin{itemize}
\item {Grp. gram.:m.}
\end{itemize}
\begin{itemize}
\item {Utilização:Des.}
\end{itemize}
Tripulante, que contava os moios de sal, postos a bordo.
\section{Crevete}
\begin{itemize}
\item {Grp. gram.:f.}
\end{itemize}
\begin{itemize}
\item {Proveniência:(Fr. \textunderscore crevette\textunderscore )}
\end{itemize}
Espécie de crustáceo:«\textunderscore crevetes com môlho russo\textunderscore ». Castilho, \textunderscore Avarento\textunderscore , 182.
\section{Crevim}
\begin{itemize}
\item {Grp. gram.:m.}
\end{itemize}
Língua uralo-altaica, do grupo ugro-finlandês.
\section{Cria}
\begin{itemize}
\item {Grp. gram.:f.}
\end{itemize}
\begin{itemize}
\item {Utilização:Gír.}
\end{itemize}
\begin{itemize}
\item {Proveniência:(De \textunderscore criar\textunderscore )}
\end{itemize}
Animal recemnascido, que se está criando.
Carne de vaca.
\section{Criação}
\begin{itemize}
\item {Grp. gram.:f.}
\end{itemize}
\begin{itemize}
\item {Proveniência:(Lat. \textunderscore creatio\textunderscore )}
\end{itemize}
Acto ou effeito de criar.
Conjunto de todas as coisas criadas.
Invento.
Instituição.
Amamentação.
Educação.
Animaes domésticos, que servem para alimentação do homem: \textunderscore uma gaiola com criação\textunderscore .
Propagação da espécie.
\section{Criada}
\begin{itemize}
\item {Grp. gram.:f.}
\end{itemize}
\begin{itemize}
\item {Proveniência:(De \textunderscore criado\textunderscore )}
\end{itemize}
Mulher ou rapariga, assoldadada para trabalhos domésticos.
\section{Criadagem}
\begin{itemize}
\item {Grp. gram.:f.}
\end{itemize}
Conjunto de criados de uma casa.
Classe dos criados e criadas.
\section{Criadeira}
\begin{itemize}
\item {Grp. gram.:adj.}
\end{itemize}
\begin{itemize}
\item {Grp. gram.:F.}
\end{itemize}
\begin{itemize}
\item {Proveniência:(De \textunderscore criar\textunderscore )}
\end{itemize}
Que cria bem: \textunderscore esta ama de leite é muito criadeira\textunderscore .
Ama de leite.
Accessório da incubação, formado de uma caixa especial, adequada á vida dos pintaínhos nos seus primeiros dias.
\section{Criadilha}
\begin{itemize}
\item {Grp. gram.:f.}
\end{itemize}
\begin{itemize}
\item {Utilização:Prov.}
\end{itemize}
\begin{itemize}
\item {Utilização:beir.}
\end{itemize}
Cogumelo, o mesmo que \textunderscore trufa\textunderscore .
(Cp. cast. \textunderscore criadilla de tierra\textunderscore , túbara)
\section{Criado}
\begin{itemize}
\item {Grp. gram.:adj.}
\end{itemize}
\begin{itemize}
\item {Grp. gram.:M.}
\end{itemize}
\begin{itemize}
\item {Proveniência:(Lat. \textunderscore creatus\textunderscore )}
\end{itemize}
Que se criou.
\textunderscore Bem criado\textunderscore , bem educado; delicado.
Nédio, gordo.
\textunderscore Mal criado\textunderscore , indelicado, grosseiro.
Homem, assoldadado para serviço doméstico.
Expressão cortês de quem se põe á disposição de alguém: \textunderscore disponha dêste seu criado\textunderscore .
\section{Criadoiro}
\begin{itemize}
\item {Grp. gram.:m.}
\end{itemize}
\begin{itemize}
\item {Grp. gram.:Adj.}
\end{itemize}
\begin{itemize}
\item {Proveniência:(De \textunderscore criar\textunderscore )}
\end{itemize}
Viveiro de plantas.
Susceptível de medrança, de se criar bem.
\section{Criado-mudo}
\begin{itemize}
\item {Grp. gram.:m.}
\end{itemize}
\begin{itemize}
\item {Utilização:Bras}
\end{itemize}
Pequena mesa ou banca de cabeceira.
\section{Criador}
\begin{itemize}
\item {Grp. gram.:m.}
\end{itemize}
\begin{itemize}
\item {Grp. gram.:Adj.}
\end{itemize}
\begin{itemize}
\item {Proveniência:(Lat. \textunderscore creator\textunderscore )}
\end{itemize}
Aquelle que cria ou criou.
Deus.
Inventor.
Lavrador, que se occupa da criação de gados.
Que cria.
Fecundante: \textunderscore chuva criadora\textunderscore .
\section{Criamento}
\begin{itemize}
\item {Grp. gram.:m.}
\end{itemize}
\begin{itemize}
\item {Utilização:Ant.}
\end{itemize}
\begin{itemize}
\item {Proveniência:(De \textunderscore criar\textunderscore )}
\end{itemize}
Afago, meiguice, carinho.
\section{Criamoso}
\begin{itemize}
\item {Grp. gram.:adj.}
\end{itemize}
\begin{itemize}
\item {Utilização:Prov.}
\end{itemize}
\begin{itemize}
\item {Utilização:beir.}
\end{itemize}
\begin{itemize}
\item {Proveniência:(De \textunderscore criar\textunderscore )}
\end{itemize}
Almo, criador, propício, (falando-se do tempo).
\section{Criança}
\begin{itemize}
\item {Grp. gram.:f.}
\end{itemize}
\begin{itemize}
\item {Utilização:Prov.}
\end{itemize}
\begin{itemize}
\item {Utilização:dur.}
\end{itemize}
\begin{itemize}
\item {Utilização:Ant.}
\end{itemize}
\begin{itemize}
\item {Proveniência:(De \textunderscore criar\textunderscore )}
\end{itemize}
Sêr humano, que se começa a criar.
Menino ou menina.
A cria de um animal.
Educação; criação.
\section{Criançada}
\begin{itemize}
\item {Grp. gram.:f.}
\end{itemize}
\begin{itemize}
\item {Utilização:Fam.}
\end{itemize}
Criancice.
As crianças.
Quantidade de crianças.
\section{Criançalho}
\begin{itemize}
\item {Grp. gram.:m.}
\end{itemize}
\begin{itemize}
\item {Utilização:Fam.}
\end{itemize}
O mesmo que \textunderscore criancelho\textunderscore .
\section{Criancelho}
\begin{itemize}
\item {fónica:cê}
\end{itemize}
\begin{itemize}
\item {Grp. gram.:m.}
\end{itemize}
\begin{itemize}
\item {Utilização:Fam.}
\end{itemize}
Indivíduo muito criança ou muito acriançado. Cf. Júl. Dinis, \textunderscore Morgad.\textunderscore , 208.
\section{Criancice}
\begin{itemize}
\item {Grp. gram.:f.}
\end{itemize}
Acto, dito, modos, próprios de criança.
\section{Crianço}
\begin{itemize}
\item {Grp. gram.:m.}
\end{itemize}
\begin{itemize}
\item {Utilização:Pop.}
\end{itemize}
\begin{itemize}
\item {Utilização:Prov.}
\end{itemize}
\begin{itemize}
\item {Utilização:alent.}
\end{itemize}
\begin{itemize}
\item {Proveniência:(De \textunderscore criança\textunderscore )}
\end{itemize}
Menino.
Criançola.
A larva das abelhas.
\section{Criançola}
\begin{itemize}
\item {Grp. gram.:m.}
\end{itemize}
\begin{itemize}
\item {Proveniência:(De \textunderscore criança\textunderscore )}
\end{itemize}
Rapaz, que já não é criança, mas que o parece por seus actos ou maneiras.
\section{Criar}
\begin{itemize}
\item {Grp. gram.:v. t.}
\end{itemize}
\begin{itemize}
\item {Grp. gram.:V. i.}
\end{itemize}
\begin{itemize}
\item {Utilização:Gír.}
\end{itemize}
\begin{itemize}
\item {Utilização:Ant.}
\end{itemize}
\begin{itemize}
\item {Proveniência:(Lat. \textunderscore creare\textunderscore )}
\end{itemize}
Dar existência a.
Originar.
Inventar.
Gerar; produzir: \textunderscore criou muitos filhos\textunderscore .
Instituir, fundar: \textunderscore criar asylos\textunderscore .
Amamentar.
Educar.
Promover a procriação de: \textunderscore criar gado\textunderscore .
Cultivar.
Adquirir.
Encher-se de pus (uma ferida), resultante gde picada: \textunderscore tenho um dedo a criar\textunderscore .
Têr dinheiro.
Afagar, acarinhar.
\section{Criatura}
\begin{itemize}
\item {Grp. gram.:f.}
\end{itemize}
\begin{itemize}
\item {Utilização:Fig.}
\end{itemize}
\begin{itemize}
\item {Proveniência:(Lat. \textunderscore creatura\textunderscore )}
\end{itemize}
Effeito de criar.
Cada um dos seres criados.
Homem, indivíduo.
Pessôa, que muito deve a outrem e que lhe é inteiramente dedicada: \textunderscore o Lopes era criatura do Fontes\textunderscore .
\section{Criaturo}
\begin{itemize}
\item {Grp. gram.:m.}
\end{itemize}
\begin{itemize}
\item {Utilização:Fam.}
\end{itemize}
Rapazinho; menino.
(Cp. \textunderscore criatura\textunderscore )
\section{Cribriforme}
\begin{itemize}
\item {Grp. gram.:adj.}
\end{itemize}
\begin{itemize}
\item {Proveniência:(Do lat. \textunderscore cribrum\textunderscore  + \textunderscore forma\textunderscore )}
\end{itemize}
Que tem fórma de crivo.
\section{Crica}
\begin{itemize}
\item {Grp. gram.:f.}
\end{itemize}
\begin{itemize}
\item {Utilização:Chul.}
\end{itemize}
\begin{itemize}
\item {Proveniência:(Do gr. \textunderscore krikos\textunderscore )}
\end{itemize}
Vulva.
\section{Cricalha}
\begin{itemize}
\item {Grp. gram.:f.}
\end{itemize}
\begin{itemize}
\item {Utilização:Prov.}
\end{itemize}
\begin{itemize}
\item {Grp. gram.:M.}
\end{itemize}
Sirigaita, lambisgóia.
Crianço. (Colhido na Bairrada)
\section{Criceto}
\begin{itemize}
\item {Grp. gram.:m.}
\end{itemize}
Mammífero roedor.
\section{Crico}
\begin{itemize}
\item {Grp. gram.:m.}
\end{itemize}
\begin{itemize}
\item {Utilização:Prov.}
\end{itemize}
O mesmo que \textunderscore berbigão\textunderscore .
\section{Cricoide}
\begin{itemize}
\item {Grp. gram.:adj.}
\end{itemize}
\begin{itemize}
\item {Utilização:Anat.}
\end{itemize}
\begin{itemize}
\item {Proveniência:(Do gr. \textunderscore krikos\textunderscore  + \textunderscore eidos\textunderscore )}
\end{itemize}
Diz-se de uma cartilagem anular, no fundo da larynge.
\section{Cricóstomo}
\begin{itemize}
\item {Grp. gram.:adj.}
\end{itemize}
\begin{itemize}
\item {Proveniência:(Do gr. \textunderscore krikos\textunderscore  + \textunderscore stoma\textunderscore )}
\end{itemize}
Que tem bôca ou abertura redonda.
\section{Cricri}
\begin{itemize}
\item {Grp. gram.:m.}
\end{itemize}
\begin{itemize}
\item {Proveniência:(T. onom.)}
\end{itemize}
Instrumento, que imita o cantar do grilo.
Canto do grilo.
\section{Crido}
\begin{itemize}
\item {Grp. gram.:adj.}
\end{itemize}
\begin{itemize}
\item {Proveniência:(De \textunderscore crêr\textunderscore )}
\end{itemize}
Acreditado.
\section{Crime}
\begin{itemize}
\item {Grp. gram.:m.}
\end{itemize}
\begin{itemize}
\item {Grp. gram.:Adj.}
\end{itemize}
\begin{itemize}
\item {Utilização:Ant.}
\end{itemize}
Transgressão de um preceito legal.
Acto, que a lei declara punível: \textunderscore o assassínio é crime punível no Código Penal\textunderscore .
Acto digno de reprehensão ou de castigo.
Criminal: \textunderscore processo crime\textunderscore .
O mesmo que \textunderscore criminoso\textunderscore :«\textunderscore olhos crimes.\textunderscore »Sousa, \textunderscore Hist. de S. Dom.\textunderscore 
Que revela crime:«\textunderscore com ar e rosto crime\textunderscore ». Sousa, \textunderscore Vida do Arceb.\textunderscore , II, 70.
\section{Crimemente}
\begin{itemize}
\item {Grp. gram.:adv.}
\end{itemize}
\begin{itemize}
\item {Utilização:Ant.}
\end{itemize}
\begin{itemize}
\item {Proveniência:(De \textunderscore crime\textunderscore )}
\end{itemize}
Criminalmente.
Severamente.
\section{Crimeza}
\begin{itemize}
\item {Grp. gram.:f.}
\end{itemize}
Severidade. Cf. Sousa, \textunderscore Hist. de S. Dom.\textunderscore , p. II, 85.
\section{Criminação}
\begin{itemize}
\item {Grp. gram.:f.}
\end{itemize}
\begin{itemize}
\item {Proveniência:(Lat. \textunderscore criminatio\textunderscore )}
\end{itemize}
Acto de criminar.
\section{Criminador}
\begin{itemize}
\item {Grp. gram.:m.}
\end{itemize}
\begin{itemize}
\item {Proveniência:(Lat. \textunderscore criminator\textunderscore )}
\end{itemize}
Aquelle que crimina.
\section{Criminal}
\begin{itemize}
\item {Grp. gram.:adj.}
\end{itemize}
\begin{itemize}
\item {Grp. gram.:M.}
\end{itemize}
\begin{itemize}
\item {Proveniência:(Lat. \textunderscore criminalis\textunderscore )}
\end{itemize}
Relativo a crime: \textunderscore jurisprudência criminal\textunderscore .
Processo criminal.
Jurisdicção ou tribunal criminal.
\section{Criminalidade}
\begin{itemize}
\item {Grp. gram.:f.}
\end{itemize}
\begin{itemize}
\item {Proveniência:(De \textunderscore criminal\textunderscore )}
\end{itemize}
Qualidade de quem é criminoso.
Os crimes.
A história dos crimes: \textunderscore a criminalidade tem aumentado\textunderscore .
\section{Criminalista}
\begin{itemize}
\item {Grp. gram.:m.}
\end{itemize}
\begin{itemize}
\item {Proveniência:(De \textunderscore criminal\textunderscore )}
\end{itemize}
Jurisconsulto, que trata especialmente dos assumptos criminaes.
\section{Criminalmente}
\begin{itemize}
\item {Grp. gram.:adv.}
\end{itemize}
Segundo o processo criminal.
\section{Criminalogia}
\begin{itemize}
\item {Grp. gram.:f.}
\end{itemize}
\begin{itemize}
\item {Proveniência:(De \textunderscore criminal\textunderscore  + gr. \textunderscore logos\textunderscore )}
\end{itemize}
Sciencia, que estuda as theorias do direito criminal.
Philosophia do direito penal.
\section{Criminalogista}
\begin{itemize}
\item {Grp. gram.:m.}
\end{itemize}
Aquelle, que é versado em criminalogia.
\section{Criminaloide}
\begin{itemize}
\item {Grp. gram.:m.}
\end{itemize}
\begin{itemize}
\item {Proveniência:(De \textunderscore criminal\textunderscore  + gr. \textunderscore eidos\textunderscore , semelhante)}
\end{itemize}
Typo de uma classe de criminosos, segundo a theoria de Lombroso.
\section{Criminar}
\begin{itemize}
\item {Grp. gram.:v. t.}
\end{itemize}
\begin{itemize}
\item {Proveniência:(Lat. \textunderscore criminare\textunderscore )}
\end{itemize}
Têr como criminoso.
Imputar crime a.
Accusar.
\section{Criminável}
\begin{itemize}
\item {Grp. gram.:adj.}
\end{itemize}
Que se póde criminar, ou considerar criminoso.
\section{Criminologia}
\begin{itemize}
\item {Grp. gram.:f.}
\end{itemize}
T. us. algures, em vez de \textunderscore criminalogia\textunderscore .
\section{Criminosamente}
\begin{itemize}
\item {Grp. gram.:adj.}
\end{itemize}
De modo criminoso.
\section{Criminoso}
\begin{itemize}
\item {Grp. gram.:adj.}
\end{itemize}
\begin{itemize}
\item {Grp. gram.:M.}
\end{itemize}
\begin{itemize}
\item {Proveniência:(Lat. \textunderscore criminosus\textunderscore )}
\end{itemize}
Relativo a crime.
Que praticou crime.
Aquelle que praticou crime.
\section{Crina}
\begin{itemize}
\item {Grp. gram.:f.}
\end{itemize}
\begin{itemize}
\item {Proveniência:(Do lat. \textunderscore crinis\textunderscore )}
\end{itemize}
Pêlos no pescoço e cauda do cavallo ou de outros animaes.
\section{Crinal}
\begin{itemize}
\item {Grp. gram.:adj.}
\end{itemize}
\begin{itemize}
\item {Grp. gram.:M.}
\end{itemize}
Relativo a crina.
Crineira.
\section{Crinalvo}
\begin{itemize}
\item {Grp. gram.:adj.}
\end{itemize}
\begin{itemize}
\item {Proveniência:(De \textunderscore crina\textunderscore  + \textunderscore alvo\textunderscore )}
\end{itemize}
Que tem a crina mais clara que os outros pêlos do corpo.
\section{Crineira}
\begin{itemize}
\item {Grp. gram.:f.}
\end{itemize}
Conjunto de pêlos ou fios, que, do alto do capacete, descaem para trás.
\section{Crinicérulo}
\begin{itemize}
\item {Grp. gram.:adj.}
\end{itemize}
\begin{itemize}
\item {Utilização:Poét.}
\end{itemize}
Que tem tranças azuladas. Cf. Filinto, XVI, 170.
\section{Crinicórneo}
\begin{itemize}
\item {Grp. gram.:adj.}
\end{itemize}
\begin{itemize}
\item {Utilização:Zool.}
\end{itemize}
\begin{itemize}
\item {Proveniência:(Do lat. \textunderscore crinis\textunderscore  + \textunderscore cornu\textunderscore )}
\end{itemize}
Que tem peludas as antennas.
\section{Crinífero}
\begin{itemize}
\item {Grp. gram.:adj.}
\end{itemize}
\begin{itemize}
\item {Proveniência:(Do lat. \textunderscore crinis\textunderscore  + \textunderscore ferre\textunderscore )}
\end{itemize}
Que tem crina.
\section{Criniforme}
\begin{itemize}
\item {Grp. gram.:adj.}
\end{itemize}
\begin{itemize}
\item {Proveniência:(Do lat. \textunderscore crinis\textunderscore  + \textunderscore forma\textunderscore )}
\end{itemize}
Que tem a fórma de um cabello.
\section{Crinígero}
\begin{itemize}
\item {Grp. gram.:adj.}
\end{itemize}
O mesmo que \textunderscore crinífero\textunderscore .
\section{Crinipreto}
\begin{itemize}
\item {Grp. gram.:adj.}
\end{itemize}
\begin{itemize}
\item {Proveniência:(De \textunderscore crina\textunderscore  + \textunderscore preta\textunderscore )}
\end{itemize}
Que tem crina preta, e de outra côr os outros pêlos do corpo.
\section{Crinisparso}
\begin{itemize}
\item {Grp. gram.:adj.}
\end{itemize}
\begin{itemize}
\item {Utilização:Poét.}
\end{itemize}
\begin{itemize}
\item {Proveniência:(Do lat. \textunderscore crinis\textunderscore  + \textunderscore sparsus\textunderscore )}
\end{itemize}
Que tem cabellos soltos, desgrenhados.
\section{Crinito}
\begin{itemize}
\item {Grp. gram.:adj.}
\end{itemize}
\begin{itemize}
\item {Proveniência:(Lat. \textunderscore crinitus\textunderscore )}
\end{itemize}
Que tem crina.
\section{Crino}
\begin{itemize}
\item {Grp. gram.:m.}
\end{itemize}
\begin{itemize}
\item {Proveniência:(Do gr. \textunderscore krinon\textunderscore )}
\end{itemize}
Espécie de narciso.
\section{Crinoide}
\begin{itemize}
\item {Grp. gram.:adj.}
\end{itemize}
\begin{itemize}
\item {Grp. gram.:M. pl.}
\end{itemize}
\begin{itemize}
\item {Proveniência:(Do gr. \textunderscore krinos\textunderscore  + \textunderscore eidos\textunderscore )}
\end{itemize}
Semelhante a crino.
Família de animaes radiários.
\section{Crinolina}
\begin{itemize}
\item {Grp. gram.:f.}
\end{itemize}
\begin{itemize}
\item {Proveniência:(Fr. \textunderscore crinoline\textunderscore )}
\end{itemize}
Tecido de crina.
Tecido forte, com que se forra interiormente a fimbria do vestido.
Espécie de saia, feita de crinolina, para entufar e arquear os vestidos.
\section{Crinolinada}
\begin{itemize}
\item {Grp. gram.:adj. f.}
\end{itemize}
Diz-se da mulher com vestuário retesado por crinolina. Cf. Cortesão, \textunderscore Subs.\textunderscore 
\section{Crinomiro}
\begin{itemize}
\item {Grp. gram.:m.}
\end{itemize}
\begin{itemize}
\item {Proveniência:(Do gr. \textunderscore krinon\textunderscore  + \textunderscore muron\textunderscore )}
\end{itemize}
Medicamento antigo, feito de lírio ou de outras plantas aromáticas.
\section{Crinomyro}
\begin{itemize}
\item {Grp. gram.:m.}
\end{itemize}
\begin{itemize}
\item {Proveniência:(Do gr. \textunderscore krinon\textunderscore  + \textunderscore muron\textunderscore )}
\end{itemize}
Medicamento antigo, feito de lírio ou de outras plantas aromáticas.
\section{Crió}
\begin{itemize}
\item {Grp. gram.:f.}
\end{itemize}
\begin{itemize}
\item {Utilização:Prov.}
\end{itemize}
A fêmea do cuco. (Colhido na Bairrada)
(C. \textunderscore cruó\textunderscore )
\section{Criocéfalo}
\begin{itemize}
\item {Grp. gram.:adj.}
\end{itemize}
\begin{itemize}
\item {Proveniência:(Do gr. \textunderscore krios\textunderscore  + \textunderscore kephale\textunderscore )}
\end{itemize}
Cuja cabeça é semelhante á do carneiro.
\section{Criocéphalo}
\begin{itemize}
\item {Grp. gram.:adj.}
\end{itemize}
\begin{itemize}
\item {Proveniência:(Do gr. \textunderscore krios\textunderscore  + \textunderscore kephale\textunderscore )}
\end{itemize}
Cuja cabeça é semelhante á do carneiro.
\section{Criocérido}
\begin{itemize}
\item {Grp. gram.:adj.}
\end{itemize}
\begin{itemize}
\item {Grp. gram.:M. pl.}
\end{itemize}
\begin{itemize}
\item {Proveniência:(Do gr. \textunderscore krios\textunderscore  + \textunderscore keras\textunderscore  + \textunderscore eidos\textunderscore )}
\end{itemize}
Semelhante ao criócero.
Ordem de coleópteroa.
\section{Criócero}
\begin{itemize}
\item {Grp. gram.:m.}
\end{itemize}
\begin{itemize}
\item {Proveniência:(Do gr. \textunderscore krios\textunderscore  + \textunderscore keras\textunderscore )}
\end{itemize}
Insécto herbívoro, prejudicial ás searas.
\section{Crioilo}
\begin{itemize}
\item {Grp. gram.:m.}
\end{itemize}
\begin{itemize}
\item {Utilização:Prov.}
\end{itemize}
\begin{itemize}
\item {Utilização:minh.}
\end{itemize}
\begin{itemize}
\item {Grp. gram.:Adj.}
\end{itemize}
\begin{itemize}
\item {Utilização:T. do Ribatejo}
\end{itemize}
\begin{itemize}
\item {Grp. gram.:M.  e  adj.}
\end{itemize}
\begin{itemize}
\item {Utilização:Bras}
\end{itemize}
\begin{itemize}
\item {Grp. gram.:M.}
\end{itemize}
\begin{itemize}
\item {Utilização:Bras}
\end{itemize}
Indivíduo, nascido na América e procedente de europeus.
Dialecto dos crioulos.
Criança de collo.
Relativo a crioulo.
* Diz-se do dialecto português, falado em Cabo-Verde e noutras possessões portuguesas da África.
Diz-se das aves que, embora de arribação, se conservam em nossa terra.
Negro, nascido no Brasil.
Pessôa, animal ou vegetal, próprio de certas localidades.
\section{Criôlo}
\begin{itemize}
\item {Grp. gram.:m.}
\end{itemize}
\begin{itemize}
\item {Utilização:Prov.}
\end{itemize}
\begin{itemize}
\item {Utilização:minh.}
\end{itemize}
\begin{itemize}
\item {Grp. gram.:Adj.}
\end{itemize}
\begin{itemize}
\item {Utilização:T. do Ribatejo}
\end{itemize}
\begin{itemize}
\item {Grp. gram.:M.  e  adj.}
\end{itemize}
\begin{itemize}
\item {Utilização:Bras}
\end{itemize}
\begin{itemize}
\item {Grp. gram.:M.}
\end{itemize}
\begin{itemize}
\item {Utilização:Bras}
\end{itemize}
Indivíduo, nascido na América e procedente de europeus.
Dialecto dos criôlos.
Criança de collo.
Relativo a criôlo.
Diz-se do dialecto português, falado em Cabo-Verde e noutras possessões portuguesas da África.
Diz-se das aves que, embora de arribação, se conservam em nossa terra.
Negro, nascido no Brasil.
Pessôa, animal ou vegetal, próprio de certas localidades.
\section{Crioulada}
\begin{itemize}
\item {Grp. gram.:f.}
\end{itemize}
\begin{itemize}
\item {Proveniência:(De \textunderscore crioulo\textunderscore )}
\end{itemize}
Porção de crioulos.
\section{Crioulo}
\begin{itemize}
\item {Grp. gram.:m.}
\end{itemize}
\begin{itemize}
\item {Utilização:Prov.}
\end{itemize}
\begin{itemize}
\item {Utilização:minh.}
\end{itemize}
\begin{itemize}
\item {Grp. gram.:Adj.}
\end{itemize}
\begin{itemize}
\item {Utilização:T. do Ribatejo}
\end{itemize}
\begin{itemize}
\item {Grp. gram.:M.  e  adj.}
\end{itemize}
\begin{itemize}
\item {Utilização:Bras}
\end{itemize}
\begin{itemize}
\item {Grp. gram.:M.}
\end{itemize}
\begin{itemize}
\item {Utilização:Bras}
\end{itemize}
Indivíduo, nascido na América e procedente de europeus.
Dialecto dos crioulos.
Criança de collo.
Relativo a crioulo.
Diz-se do dialecto português, falado em Cabo-Verde e noutras possessões portuguesas da África.
Diz-se das aves que, embora de arribação, se conservam em nossa terra.
Negro, nascido no Brasil.
Pessôa, animal ou vegetal, próprio de certas localidades.
\section{Criqueiro}
\begin{itemize}
\item {Grp. gram.:adj.}
\end{itemize}
\begin{itemize}
\item {Utilização:T. de Lanhoso}
\end{itemize}
Affável.
Metediço.
\section{Criquete}
\begin{itemize}
\item {Grp. gram.:m.}
\end{itemize}
\begin{itemize}
\item {Proveniência:(Do ingl. \textunderscore cricket\textunderscore )}
\end{itemize}
Exercício gymnástico, de origem inglesa e semelhante ao jôgo da bola.
\section{Cris}
\begin{itemize}
\item {Grp. gram.:adj.}
\end{itemize}
\begin{itemize}
\item {Utilização:Ant.}
\end{itemize}
\begin{itemize}
\item {Grp. gram.:M.}
\end{itemize}
\begin{itemize}
\item {Proveniência:(De \textunderscore gris\textunderscore )}
\end{itemize}
Eclipsado.
Que põe mêdo.
Pardacento, obscuro:«\textunderscore céu cris.\textunderscore »Herculano, \textunderscore M. de Cister.\textunderscore 
Eclipse.
\section{Crisalho}
\begin{itemize}
\item {Grp. gram.:m.}
\end{itemize}
Gênero de pintura monochroma, geralmente pardacenta.
(Cp. \textunderscore grisalho\textunderscore )
\section{Crisálida}
\begin{itemize}
\item {Grp. gram.:f.}
\end{itemize}
\begin{itemize}
\item {Proveniência:(Gr. \textunderscore khrusallis\textunderscore , de \textunderscore khrusos\textunderscore , oiro)}
\end{itemize}
Fórma que os lepidópteros tomam, para passar do estado de lagarta para o de borboleta; casulo.
\section{Crisalidar}
\begin{itemize}
\item {Grp. gram.:v. i.}
\end{itemize}
\begin{itemize}
\item {Proveniência:(De \textunderscore crysállida\textunderscore )}
\end{itemize}
Converter-se (a lagarta) em crisálida ou ninfa.
\section{Crisálide}
\begin{itemize}
\item {Grp. gram.:f.}
\end{itemize}
\begin{itemize}
\item {Proveniência:(Gr. \textunderscore khrusallis\textunderscore , de \textunderscore khrusos\textunderscore , oiro)}
\end{itemize}
Fórma que os lepidópteros tomam, para passar do estado de lagarta para o de borboleta; casulo.
\section{Crisântemo}
\begin{itemize}
\item {Grp. gram.:m.}
\end{itemize}
\begin{itemize}
\item {Proveniência:(Lat. \textunderscore crysanthemum\textunderscore )}
\end{itemize}
Gênero de plantas de fôlhas alternas e flôres brancas, amarelas ou rosadas; vulgarmente, \textunderscore despedidas do verão\textunderscore .
\section{Crisanto}
\begin{itemize}
\item {Grp. gram.:m.}
\end{itemize}
\begin{itemize}
\item {Proveniência:(Do gr. \textunderscore khrusos\textunderscore  + \textunderscore anthos\textunderscore )}
\end{itemize}
O mesmo que \textunderscore crisântemo\textunderscore . Cf. M. Bernárdez, \textunderscore Floresta\textunderscore , II, 244.
\section{Crise}
\begin{itemize}
\item {Grp. gram.:f.}
\end{itemize}
\begin{itemize}
\item {Utilização:Fig.}
\end{itemize}
\begin{itemize}
\item {Utilização:Polit.}
\end{itemize}
\begin{itemize}
\item {Proveniência:(Gr. \textunderscore krisis\textunderscore  de \textunderscore krinein\textunderscore )}
\end{itemize}
Alteração no curso de uma doença.
Conjuntura perigosa.
Situação afflictiva.
Momento grave.
Situação de um Govêrno, cuja conservação encontra difficuldades muito graves: \textunderscore o Govêrno está em crise\textunderscore .
\section{Crise}
\begin{itemize}
\item {Grp. gram.:f.}
\end{itemize}
Espécie de tecido antigo:«\textunderscore calças imperiaes de crise preta.\textunderscore »\textunderscore Alvará\textunderscore  de D. Sebast., in \textunderscore Rev. Lus.\textunderscore , XV, 117.
(Relaciona-se com \textunderscore cris\textunderscore ?)
\section{Criselefantina}
\begin{itemize}
\item {Grp. gram.:f.}
\end{itemize}
\begin{itemize}
\item {Proveniência:(Do gr. \textunderscore khrusos\textunderscore  + \textunderscore elephas\textunderscore )}
\end{itemize}
Dizia-se a esculptura, em que entrava oiro e marfim.
\section{Críside}
\begin{itemize}
\item {Grp. gram.:f.}
\end{itemize}
Espécie de vespa amarela, que serve de tipo aos crisídidos.
\section{Crisídidas}
\begin{itemize}
\item {Grp. gram.:f. pl.}
\end{itemize}
\begin{itemize}
\item {Proveniência:(Do gr. \textunderscore khrusos\textunderscore  + \textunderscore eidos\textunderscore )}
\end{itemize}
Família de insectos himenópteros, que tem por tipo a críside.
\section{Crisídidos}
\begin{itemize}
\item {Grp. gram.:f. pl.}
\end{itemize}
\begin{itemize}
\item {Proveniência:(Do gr. \textunderscore khrusos\textunderscore  + \textunderscore eidos\textunderscore )}
\end{itemize}
Família de insectos himenópteros, que tem por tipo a críside.
\section{Crísio}
\begin{itemize}
\item {Grp. gram.:adj.}
\end{itemize}
\begin{itemize}
\item {Proveniência:(Lat. \textunderscore chryseus\textunderscore )}
\end{itemize}
Feito de oiro; doirado.
\section{Crísis}
\begin{itemize}
\item {Grp. gram.:f.}
\end{itemize}
\begin{itemize}
\item {Proveniência:(Do gr. \textunderscore khrusos\textunderscore , oiro)}
\end{itemize}
Designação científica da vespa doirada.
\section{Crisma}
\begin{itemize}
\item {Grp. gram.:m.}
\end{itemize}
\begin{itemize}
\item {Proveniência:(Gr. \textunderscore khrisma\textunderscore )}
\end{itemize}
Óleo perfumado, que serve na ministração de alguns sacramentos e em outras ceremónias religiosas.
Sacramento da confirmação.
\section{Crismar}
\begin{itemize}
\item {Grp. gram.:v. t.}
\end{itemize}
\begin{itemize}
\item {Utilização:Fig.}
\end{itemize}
Conferir a crisma a.
Mudar o nome a; alcunhar.
\section{Crismino}
\begin{itemize}
\item {Grp. gram.:m.}
\end{itemize}
\begin{itemize}
\item {Utilização:Prov.}
\end{itemize}
\begin{itemize}
\item {Utilização:alg.}
\end{itemize}
Espécie de pêssego suculento e grande.
\section{Crisobulo}
\begin{itemize}
\item {Grp. gram.:m.}
\end{itemize}
Diploma com sêlo doirado.
(B. gr. \textunderscore khrusobullon\textunderscore )
\section{Crisocalo}
\begin{itemize}
\item {Grp. gram.:m.}
\end{itemize}
\begin{itemize}
\item {Proveniência:(Do gr. \textunderscore khrusos\textunderscore  + \textunderscore khalos\textunderscore )}
\end{itemize}
Liga de cobre e oiro.
\section{Crisócalo}
\begin{itemize}
\item {Grp. gram.:m.}
\end{itemize}
\begin{itemize}
\item {Utilização:Fig.}
\end{itemize}
\begin{itemize}
\item {Proveniência:(Do gr. \textunderscore khrusos\textunderscore , oiro + \textunderscore kalos\textunderscore , bello)}
\end{itemize}
Aquilo que imita oiro.
Aquilo que só é bom na aparência.
\section{Crisocarpo}
\begin{itemize}
\item {Grp. gram.:adj.}
\end{itemize}
\begin{itemize}
\item {Proveniência:(Do gr. \textunderscore khrusos\textunderscore  + \textunderscore karpos\textunderscore )}
\end{itemize}
Que tem frutos côr de oiro.
\section{Crisocéfalo}
\begin{itemize}
\item {Grp. gram.:adj.}
\end{itemize}
\begin{itemize}
\item {Proveniência:(Do gr. \textunderscore khrusos\textunderscore , oiro, e \textunderscore kephale\textunderscore , cabeça)}
\end{itemize}
Que tem a cabeça ou o cimo da côr de oiro.
\section{Crisocloro}
\begin{itemize}
\item {Grp. gram.:adj.}
\end{itemize}
\begin{itemize}
\item {Proveniência:(Do gr. \textunderscore khrusos\textunderscore  + \textunderscore khloros\textunderscore )}
\end{itemize}
Auri-verde.
\section{Crisocola}
\begin{itemize}
\item {Grp. gram.:f.}
\end{itemize}
\begin{itemize}
\item {Proveniência:(Do gr. \textunderscore khrusos\textunderscore  + \textunderscore kholla\textunderscore )}
\end{itemize}
Designação antiga do bórax.
\section{Crisócoma}
\begin{itemize}
\item {Grp. gram.:adj.}
\end{itemize}
\begin{itemize}
\item {Proveniência:(Do gr. \textunderscore khrusos\textunderscore  + \textunderscore kome\textunderscore )}
\end{itemize}
Planta exótica, de flôres amarelas.
\section{Crisofânico}
\begin{itemize}
\item {Grp. gram.:adj.}
\end{itemize}
Diz-se de um ácido, extraido do ruibarbo.
\section{Crisofilo}
\begin{itemize}
\item {Grp. gram.:adj.}
\end{itemize}
\begin{itemize}
\item {Proveniência:(Do gr. \textunderscore khrusos\textunderscore  + \textunderscore phullon\textunderscore )}
\end{itemize}
Que tem fôlhas doiradas.
Árvore fructífera do Brasil.
\section{Crisoftalmo}
\begin{itemize}
\item {Grp. gram.:adj.}
\end{itemize}
\begin{itemize}
\item {Proveniência:(Do gr. \textunderscore khrusos\textunderscore  + \textunderscore ophthalmos\textunderscore )}
\end{itemize}
Que tem olhos doirados, (falando-se de certos animaes).
\section{Crisogastro}
\begin{itemize}
\item {Grp. gram.:adj.}
\end{itemize}
\begin{itemize}
\item {Proveniência:(Do gr. \textunderscore khrusos\textunderscore  + \textunderscore gaster\textunderscore )}
\end{itemize}
Que tem o ventre da côr do oiro, (falando-se de certos animaes).
\section{Crisoglifia}
\begin{itemize}
\item {Grp. gram.:f.}
\end{itemize}
Processo da gravura em relêvo sôbre cobre, que se executa por meio do oiro e agentes químicos.
\section{Crisografia}
\begin{itemize}
\item {Grp. gram.:f.}
\end{itemize}
Arte de escrever em letras de oiro.
(Cp. \textunderscore chrysógrapho\textunderscore )
\section{Crisógrafo}
\begin{itemize}
\item {Grp. gram.:m.}
\end{itemize}
\begin{itemize}
\item {Proveniência:(Do gr. \textunderscore khrusos\textunderscore  + \textunderscore graphein\textunderscore )}
\end{itemize}
Aquele que escreve em letras de oiro.
\section{Crisol}
\begin{itemize}
\item {Grp. gram.:m.}
\end{itemize}
Cadinho.
Aquillo que serve para patentear as bôas qualidades.
(Cast. \textunderscore crisuelo\textunderscore )
\section{Crisolar}
\begin{itemize}
\item {Grp. gram.:v. t.}
\end{itemize}
O mesmo que \textunderscore acrisolar\textunderscore . Cf. Filinto, I, 87.
\section{Crisólita}
\begin{itemize}
\item {Grp. gram.:f.}
\end{itemize}
\begin{itemize}
\item {Proveniência:(Gr. \textunderscore chrusolithos\textunderscore )}
\end{itemize}
Pedra preciosa, da côr do oiro.
\section{Crisólito}
\begin{itemize}
\item {Grp. gram.:m.}
\end{itemize}
O mesmo ou melhor que \textunderscore crisólita\textunderscore .
\section{Crisologia}
\begin{itemize}
\item {Grp. gram.:f.}
\end{itemize}
\begin{itemize}
\item {Utilização:P. us.}
\end{itemize}
O mesmo que \textunderscore crisonomia\textunderscore .
\section{Crisólogo}
\begin{itemize}
\item {Grp. gram.:adj.}
\end{itemize}
\begin{itemize}
\item {Proveniência:(Do gr. \textunderscore khrusos\textunderscore  + \textunderscore logos\textunderscore )}
\end{itemize}
Que tem palavras de oiro, (como se dizia de alguns Padres da Igreja).
\section{Crisómela}
\begin{itemize}
\item {Grp. gram.:f.}
\end{itemize}
\begin{itemize}
\item {Proveniência:(Do gr. \textunderscore khrusos\textunderscore  + \textunderscore melos\textunderscore )}
\end{itemize}
Insecto herbívoro, espécie de escaravelho.
\section{Crisómelo}
\begin{itemize}
\item {Grp. gram.:m.}
\end{itemize}
\begin{itemize}
\item {Proveniência:(Do gr. \textunderscore khrusos\textunderscore  + \textunderscore melos\textunderscore )}
\end{itemize}
Insecto herbívoro, espécie de escaravelho.
\section{Crisopeia}
\begin{itemize}
\item {Grp. gram.:f.}
\end{itemize}
\begin{itemize}
\item {Proveniência:(Do gr. \textunderscore khrusos\textunderscore  + \textunderscore poiein\textunderscore )}
\end{itemize}
Suposta arte de fazer oiro.
\section{Crisopraso}
\begin{itemize}
\item {Grp. gram.:m.}
\end{itemize}
\begin{itemize}
\item {Proveniência:(Do gr. \textunderscore khrusos\textunderscore  + \textunderscore prasos\textunderscore )}
\end{itemize}
Variedade de ágata.
\section{Crisóptero}
\begin{itemize}
\item {Grp. gram.:adj.}
\end{itemize}
\begin{itemize}
\item {Proveniência:(Do gr. \textunderscore khrusos\textunderscore  + \textunderscore pteron\textunderscore )}
\end{itemize}
Que tem asas doiradas.
\section{Crisorramnina}
\begin{itemize}
\item {Grp. gram.:f.}
\end{itemize}
Substância còrante de uma espécie de ramno, (\textunderscore rhamnus amygdalinus\textunderscore ).
\section{Crisóstomo}
\begin{itemize}
\item {Grp. gram.:adj.}
\end{itemize}
\begin{itemize}
\item {Utilização:Fig.}
\end{itemize}
\begin{itemize}
\item {Proveniência:(Gr. \textunderscore krusostomos\textunderscore )}
\end{itemize}
Que tem bôca doirada.
Eloquente.
\section{Crispação}
\begin{itemize}
\item {Grp. gram.:f.}
\end{itemize}
Acto ou effeito de críspar.
\section{Crispadura}
\begin{itemize}
\item {Grp. gram.:f.}
\end{itemize}
(V.crispação)
\section{Crispamento}
\begin{itemize}
\item {Grp. gram.:m.}
\end{itemize}
O mesmo que \textunderscore crispação\textunderscore .
Arrepio.
\section{Crispante}
\begin{itemize}
\item {Grp. gram.:adj.}
\end{itemize}
Que crispa.
Contrahido, franzido:«\textunderscore crispantes os beiços\textunderscore ». Camillo, \textunderscore Judeu\textunderscore , 41.
\section{Crispar}
\begin{itemize}
\item {Grp. gram.:v. t.}
\end{itemize}
\begin{itemize}
\item {Grp. gram.:V. p.}
\end{itemize}
\begin{itemize}
\item {Proveniência:(Lat. \textunderscore crispare\textunderscore )}
\end{itemize}
Enrugar; franzir.
Contrahir.
Contrahir-se espasmodicamente.
\section{Crispatura}
\begin{itemize}
\item {Grp. gram.:f.}
\end{itemize}
(V.crispação)
\section{Crispifloro}
\begin{itemize}
\item {Grp. gram.:adj.}
\end{itemize}
\begin{itemize}
\item {Utilização:Bot.}
\end{itemize}
\begin{itemize}
\item {Proveniência:(Do lat. \textunderscore crispus\textunderscore  + \textunderscore flos\textunderscore )}
\end{itemize}
Que tem as pétalas franzidas ou ondeadas nas bordas.
\section{Crispifoliado}
\begin{itemize}
\item {Grp. gram.:adj.}
\end{itemize}
\begin{itemize}
\item {Utilização:Bot.}
\end{itemize}
\begin{itemize}
\item {Proveniência:(Do lat. \textunderscore crispus\textunderscore  + \textunderscore folium\textunderscore )}
\end{itemize}
O mesmo que \textunderscore crispifloro\textunderscore .
\section{Crispina}
\begin{itemize}
\item {Grp. gram.:f.}
\end{itemize}
\begin{itemize}
\item {Utilização:Ant.}
\end{itemize}
Coifa, o mesmo que \textunderscore crespino\textunderscore .
\section{Crista}
\begin{itemize}
\item {Grp. gram.:f.}
\end{itemize}
\begin{itemize}
\item {Proveniência:(Lat. \textunderscore crista\textunderscore )}
\end{itemize}
Excrescência carnosa na cabeça dos gallos e de algumas outras gallináceas.
Excrescência na cabeça de alguns reptis.
Pennacho.
Ornato em fórma de crista.
Nome de várias plantas.
Ponto mais elevado.
Aresta (de montanha)
\section{Cristadela}
\begin{itemize}
\item {Grp. gram.:f.}
\end{itemize}
\begin{itemize}
\item {Proveniência:(De \textunderscore crista\textunderscore )}
\end{itemize}
Gênero de pólypos, o que produz o verdadeiro coral.
\section{Cristagálli}
\begin{itemize}
\item {Grp. gram.:m.}
\end{itemize}
(V.gallacrista)
\section{Cristãmente}
\begin{itemize}
\item {Grp. gram.:adv.}
\end{itemize}
De modo cristão.
\section{Cristan}
\begin{itemize}
\item {Grp. gram.:f.  e  adj.}
\end{itemize}
(fem. de \textunderscore cristão\textunderscore )
\section{Cristandia}
\begin{itemize}
\item {Grp. gram.:f.}
\end{itemize}
\begin{itemize}
\item {Utilização:Des.}
\end{itemize}
Cristandade.
Grande número de cristãos. Cf. o romance popular \textunderscore D. João da Armada\textunderscore .
\section{Cristão}
\begin{itemize}
\item {Grp. gram.:m.}
\end{itemize}
\begin{itemize}
\item {Utilização:Prov.}
\end{itemize}
\begin{itemize}
\item {Proveniência:(De \textunderscore crista?\textunderscore )}
\end{itemize}
O mesmo que \textunderscore bode\textunderscore ^1.
\section{Cristária}
\begin{itemize}
\item {Grp. gram.:f.}
\end{itemize}
\begin{itemize}
\item {Proveniência:(De \textunderscore crista\textunderscore )}
\end{itemize}
Planta malvácea da América.
\section{Cristel}
\begin{itemize}
\item {Grp. gram.:m.}
\end{itemize}
\begin{itemize}
\item {Utilização:Pop.}
\end{itemize}
O mesmo que \textunderscore clystér\textunderscore .
\section{Cristeleira}
\begin{itemize}
\item {Grp. gram.:f.}
\end{itemize}
\begin{itemize}
\item {Utilização:Des.}
\end{itemize}
\begin{itemize}
\item {Proveniência:(De \textunderscore cristel\textunderscore )}
\end{itemize}
Mulher que, por offício, andava pelas casas, applicando clysteres.
\section{Cristelizar}
\begin{itemize}
\item {Grp. gram.:v. t.}
\end{itemize}
\begin{itemize}
\item {Utilização:Ant.}
\end{itemize}
\begin{itemize}
\item {Proveniência:(De \textunderscore cristel\textunderscore )}
\end{itemize}
Applicar clyster a.
\section{Cristengo}
\begin{itemize}
\item {Grp. gram.:adj.}
\end{itemize}
\begin{itemize}
\item {Utilização:Ant.}
\end{itemize}
\begin{itemize}
\item {Proveniência:(Do lat. \textunderscore christianicus\textunderscore )}
\end{itemize}
Cristão, relativo a cristãos.
E dizia-se dos caracteres latinos.
\section{Cristianicida}
\begin{itemize}
\item {Grp. gram.:m.}
\end{itemize}
\begin{itemize}
\item {Proveniência:(Do lat. \textunderscore christianus\textunderscore  + \textunderscore caedere\textunderscore )}
\end{itemize}
Matador de cristãos.
\section{Cristianicídio}
\begin{itemize}
\item {Grp. gram.:m.}
\end{itemize}
Matança de cristãos.
(Cp. \textunderscore cristianicida\textunderscore )
\section{Cristianismo}
\begin{itemize}
\item {Grp. gram.:m.}
\end{itemize}
\begin{itemize}
\item {Proveniência:(Do lat. \textunderscore christianus\textunderscore )}
\end{itemize}
Religião de Cristo.
\section{Cristianíssimo}
\begin{itemize}
\item {Grp. gram.:adj.}
\end{itemize}
(sup. de \textunderscore cristão\textunderscore )
\section{Cristianização}
\begin{itemize}
\item {Grp. gram.:f.}
\end{itemize}
Acto de cristianizar.
\section{Cristianizador}
\begin{itemize}
\item {Grp. gram.:m.  e  adj.}
\end{itemize}
O que cristianiza. Cf. Th. Ribeiro, \textunderscore Jornadas\textunderscore , II, 71.
\section{Cristianizar}
\begin{itemize}
\item {Grp. gram.:v. t.}
\end{itemize}
\begin{itemize}
\item {Proveniência:(Do lat. \textunderscore christianus\textunderscore )}
\end{itemize}
Tornar cristão.
Incluir na disciplina ou na prátíca dos cristãos.
\section{Cristiano}
\begin{itemize}
\item {Grp. gram.:m.}
\end{itemize}
\begin{itemize}
\item {Proveniência:(De \textunderscore Christiano\textunderscore , n. p.)}
\end{itemize}
Moéda de oiro, na Dinamarca.
\section{Cristicida}
\begin{itemize}
\item {Grp. gram.:m.}
\end{itemize}
\begin{itemize}
\item {Proveniência:(Do lat. \textunderscore Christus\textunderscore , n. p. + \textunderscore caedere\textunderscore )}
\end{itemize}
Quem matou Cristo.
\section{Cristicídio}
\begin{itemize}
\item {Grp. gram.:m.}
\end{itemize}
Morte de Cristo.
(Cp. \textunderscore cristicida\textunderscore )
\section{Cristícola}
\begin{itemize}
\item {Grp. gram.:m.}
\end{itemize}
\begin{itemize}
\item {Proveniência:(Do lat. \textunderscore Christus\textunderscore , n. p. + \textunderscore colere\textunderscore )}
\end{itemize}
Aquele que adora Crísto.
\section{Cristífero}
\begin{itemize}
\item {Grp. gram.:adj.}
\end{itemize}
\begin{itemize}
\item {Proveniência:(Do lat. \textunderscore Christus\textunderscore , n. p. + \textunderscore ferre\textunderscore )}
\end{itemize}
Que sustenta ou leva uma imagem de Cristo.
\section{Cristino}
\begin{itemize}
\item {Grp. gram.:m.}
\end{itemize}
Partidário da rainha Cristina, em Espanha.
\section{Cristípara}
\begin{itemize}
\item {Grp. gram.:f.}
\end{itemize}
\begin{itemize}
\item {Proveniência:(Do lat. \textunderscore Christus\textunderscore , n. p. + \textunderscore parere\textunderscore )}
\end{itemize}
Mãe de Cristo.
\section{Cristo}
\begin{itemize}
\item {Grp. gram.:m.}
\end{itemize}
\begin{itemize}
\item {Proveniência:(Lat. \textunderscore Christus\textunderscore , n. p.)}
\end{itemize}
Imagem de Cristo crucificado: \textunderscore tinha á cabeceira um Cristo de marfim\textunderscore .
\section{Cristofania}
\begin{itemize}
\item {Grp. gram.:f.}
\end{itemize}
\begin{itemize}
\item {Proveniência:(Do gr. \textunderscore Khristos\textunderscore  + \textunderscore phainestai\textunderscore , apparecer)}
\end{itemize}
Aparição de Cristo.
\section{Cristofle}
\begin{itemize}
\item {Grp. gram.:m.}
\end{itemize}
\begin{itemize}
\item {Proveniência:(De \textunderscore Christofle\textunderscore , n. p.)}
\end{itemize}
Metal, de composição análoga á do argentão.
\section{Cristologia}
\begin{itemize}
\item {Grp. gram.:f.}
\end{itemize}
\begin{itemize}
\item {Proveniência:(Do gr. \textunderscore Khristos\textunderscore , n. p. + \textunderscore logos\textunderscore )}
\end{itemize}
Tratado á cêrca da pessôa e doutrina de Cristo.
\section{Cristológico}
\begin{itemize}
\item {Grp. gram.:adj.}
\end{itemize}
Que diz respeito á cristologia.
\section{Cristómaco}
\begin{itemize}
\item {Grp. gram.:m.}
\end{itemize}
\begin{itemize}
\item {Proveniência:(Gr. \textunderscore khristomakos\textunderscore )}
\end{itemize}
Aquele que sustenta doutrina falsa sôbre a natureza ou pessôa de Cristo.
\section{Critério}
\begin{itemize}
\item {Grp. gram.:m.}
\end{itemize}
\begin{itemize}
\item {Proveniência:(Gr. \textunderscore kriterion\textunderscore )}
\end{itemize}
Caracteres, que distinguem da verdade o êrro.
Faculdade de conhecer a verdade.
Raciocínio; modo de apreciar coisas ou pessôas.
\section{Criterioso}
\begin{itemize}
\item {Grp. gram.:adj.}
\end{itemize}
\begin{itemize}
\item {Utilização:Neol.}
\end{itemize}
\begin{itemize}
\item {Proveniência:(De \textunderscore critério\textunderscore )}
\end{itemize}
Que tem bom critério.
Ajuizado.
Em que há bom critério; que revela juízo claro e seguro.
\section{Crithmo}
\begin{itemize}
\item {Grp. gram.:m.}
\end{itemize}
\begin{itemize}
\item {Proveniência:(Do gr. \textunderscore krithe\textunderscore , cevada)}
\end{itemize}
Gênero de plantas umbellíferas.
\section{Crithomancia}
\begin{itemize}
\item {Grp. gram.:f.}
\end{itemize}
\begin{itemize}
\item {Proveniência:(Do gr. \textunderscore krithe\textunderscore  + \textunderscore manteia\textunderscore )}
\end{itemize}
Adivinhação, que se praticava, offertando bolos de cevada aos deuses.
\section{Crithóphago}
\begin{itemize}
\item {Grp. gram.:adj.}
\end{itemize}
\begin{itemize}
\item {Proveniência:(Do gr. \textunderscore krithe\textunderscore  + \textunderscore phagein\textunderscore )}
\end{itemize}
Que se alimenta de cevada.
\section{Crítica}
\begin{itemize}
\item {Grp. gram.:f.}
\end{itemize}
\begin{itemize}
\item {Proveniência:(De \textunderscore crítico\textunderscore )}
\end{itemize}
Arte de julgar as producções literárias, artísticas ou scientíficas.
Apreciação escrita de producções daquelle gênero.
Discussão dos factos históricos.
Apreciação minuciosa.
Critério.
Maledicência; apreciação desfavorável.
\section{Criticador}
\begin{itemize}
\item {Grp. gram.:m.}
\end{itemize}
\begin{itemize}
\item {Proveniência:(De \textunderscore criticar\textunderscore )}
\end{itemize}
Aquelle que tem por hábito dizer mal de alguém ou de alguma coisa.
\section{Criticante}
\begin{itemize}
\item {Grp. gram.:m.  e  adj.}
\end{itemize}
O que critíca. Cf. Filinto, III, 56.
\section{Criticar}
\begin{itemize}
\item {Grp. gram.:v. t.}
\end{itemize}
Exercer a crítica sôbre ou em: \textunderscore criticar um livro\textunderscore .
Censurar, dizer mal de: \textunderscore aquelle procedimento foi muito criticado\textunderscore .
\section{Criticaria}
\begin{itemize}
\item {Grp. gram.:f.}
\end{itemize}
\begin{itemize}
\item {Utilização:Deprec.}
\end{itemize}
\begin{itemize}
\item {Proveniência:(De \textunderscore crítica\textunderscore )}
\end{itemize}
Conjunto de críticas ou de críticos. Cf. Castilho, \textunderscore Outono\textunderscore , p. 211.
\section{Criticastro}
\begin{itemize}
\item {Grp. gram.:m.}
\end{itemize}
\begin{itemize}
\item {Utilização:Deprec.}
\end{itemize}
Crítico reles.
\section{Criticável}
\begin{itemize}
\item {Grp. gram.:adj.}
\end{itemize}
Que póde ou deve criticar-se.
\section{Criticismo}
\begin{itemize}
\item {Grp. gram.:m.}
\end{itemize}
\begin{itemize}
\item {Proveniência:(De \textunderscore crítica\textunderscore )}
\end{itemize}
Systema philosóphico, que procura determinar os limites da razão humana.
\section{Criticista}
\begin{itemize}
\item {Grp. gram.:adj.}
\end{itemize}
\begin{itemize}
\item {Grp. gram.:M.}
\end{itemize}
Relativo ao criticismo.
Sectário do criticismo.
\section{Crítico}
\begin{itemize}
\item {Grp. gram.:adj.}
\end{itemize}
\begin{itemize}
\item {Grp. gram.:M.}
\end{itemize}
\begin{itemize}
\item {Proveniência:(Gr. \textunderscore kritikos\textunderscore )}
\end{itemize}
Relativo a crítica.
Relativo a crise: \textunderscore idade crítica das mulheres\textunderscore .
Embaraçoso.
Perigoso: \textunderscore situação crítica\textunderscore .
Aquelle que critica, que faz crítica.
\section{Critiqueiro}
\begin{itemize}
\item {Grp. gram.:m.}
\end{itemize}
O mesmo que \textunderscore criticastro\textunderscore .
\section{Critiquice}
\begin{itemize}
\item {Grp. gram.:f.}
\end{itemize}
\begin{itemize}
\item {Utilização:Deprec.}
\end{itemize}
Crítica ordinária.
Mania de criticar, sem fundamento.
\section{Critmo}
\begin{itemize}
\item {Grp. gram.:m.}
\end{itemize}
\begin{itemize}
\item {Proveniência:(Do gr. \textunderscore krithe\textunderscore , cevada)}
\end{itemize}
Gênero de plantas umbellíferas.
\section{Critófago}
\begin{itemize}
\item {Grp. gram.:adj.}
\end{itemize}
\begin{itemize}
\item {Proveniência:(Do gr. \textunderscore krithe\textunderscore  + \textunderscore phagein\textunderscore )}
\end{itemize}
Que se alimenta de cevada.
\section{Critomancia}
\begin{itemize}
\item {Grp. gram.:f.}
\end{itemize}
\begin{itemize}
\item {Grp. gram.:f.}
\end{itemize}
\begin{itemize}
\item {Proveniência:(Do gr. \textunderscore krithe\textunderscore  + \textunderscore manteia\textunderscore )}
\end{itemize}
Suposta arte de adivinhar, por meio do sal com farinha de cevada. Cf. Castilho, \textunderscore Fastos\textunderscore , III, 315.
Adivinhação, que se praticava, offertando bolos de cevada aos deuses.
\section{Critónia}
\begin{itemize}
\item {Grp. gram.:f.}
\end{itemize}
\begin{itemize}
\item {Proveniência:(Do gr. \textunderscore kriton\textunderscore )}
\end{itemize}
Planta synanthérea.
\section{Criúva}
\begin{itemize}
\item {Grp. gram.:f.}
\end{itemize}
Planta guttífera do Brasil.
\section{Criva}
\begin{itemize}
\item {Grp. gram.:f.}
\end{itemize}
O mesmo que \textunderscore crivo\textunderscore , mas de orifícios mais largos. Cf. Júl. Moreira, \textunderscore Est. da L. Port.\textunderscore , I, 186.
\section{Crivação}
\begin{itemize}
\item {Grp. gram.:f.}
\end{itemize}
Acto ou effeito de crivar.
\section{Crivantes}
\begin{itemize}
\item {Grp. gram.:m. pl.}
\end{itemize}
\begin{itemize}
\item {Utilização:Gír.}
\end{itemize}
\begin{itemize}
\item {Proveniência:(De \textunderscore crivar\textunderscore )}
\end{itemize}
Dentes.
\section{Crivar}
\begin{itemize}
\item {Grp. gram.:v. t.}
\end{itemize}
\begin{itemize}
\item {Proveniência:(Lat. \textunderscore cribrare\textunderscore )}
\end{itemize}
Furar em muitos pontos.
Encher de pintas, constellar.
Passar por crivo.
\section{Criveira}
\begin{itemize}
\item {Utilização:Prov.}
\end{itemize}
\begin{itemize}
\item {Utilização:minh.}
\end{itemize}
O mesmo que \textunderscore crivo\textunderscore .
\section{Criveiro}
\begin{itemize}
\item {Grp. gram.:m.}
\end{itemize}
\begin{itemize}
\item {Utilização:Prov.}
\end{itemize}
\begin{itemize}
\item {Utilização:dur.}
\end{itemize}
Fabricante de crivos e peneiras.
\section{Crível}
\begin{itemize}
\item {Grp. gram.:adj.}
\end{itemize}
\begin{itemize}
\item {Proveniência:(Lat. \textunderscore credibilis\textunderscore )}
\end{itemize}
Que se póde crer; acreditável.
\section{Crivo}
\begin{itemize}
\item {Grp. gram.:m.}
\end{itemize}
\begin{itemize}
\item {Utilização:Ext.}
\end{itemize}
\begin{itemize}
\item {Proveniência:(Lat. \textunderscore cribrum\textunderscore )}
\end{itemize}
Conjunto de orifícios numa superfície.
Peneira de arame.
Utensílio culinário, de fôlha, com muitos orifícios, para separar de um líquido as substâncias inúteis.
Coador.
Appêndice de folha num regador, para borrifar com água.
Ralo.
Lâmina metállica, com alguns orifícios, collocada em porta de entrada, para se vêr quem está fóra, sem sêr visto quem o observa.
Bordado, que se faz com agulha de croché, tirando-se previamente dos quatro lados do pano alguns fios interpolados.
Aquillo que tem muitos buracos próximos ou que tem analogia com o crivo: \textunderscore levou tantas facadas, que parecia um crivo\textunderscore .
\section{Crixas}
\begin{itemize}
\item {Grp. gram.:m. pl.}
\end{itemize}
Tríbo indígena de Goiás.
\section{Criz}
\begin{itemize}
\item {Grp. gram.:m.}
\end{itemize}
Punhal dos Malaios.
\section{Crizada}
\begin{itemize}
\item {Grp. gram.:f.}
\end{itemize}
Golpe de criz. Cf. Barros, \textunderscore Déc.\textunderscore  II, 91.
\section{Cró}
\begin{itemize}
\item {Grp. gram.:m.}
\end{itemize}
Jôgo de cartas, em que o ganho é para o parceiro que primeiro reúne um naipe completo.
\section{Cró}
\begin{itemize}
\item {Grp. gram.:m.}
\end{itemize}
O mesmo que \textunderscore cruó\textunderscore .
\section{Crôa}
\begin{itemize}
\item {Grp. gram.:f.}
\end{itemize}
\begin{itemize}
\item {Utilização:Prov.}
\end{itemize}
Crosta de terreno, que não foi mexida recentemente por enxada ou arado.
(Contr. de \textunderscore corôa\textunderscore )
\section{Croá}
\begin{itemize}
\item {Grp. gram.:m.}
\end{itemize}
Planta têxtil do Ceará.
Corda, feita dos filamentos dessa planta.
O mesmo que \textunderscore caroá\textunderscore .
\section{Croata}
\begin{itemize}
\item {Grp. gram.:adj.}
\end{itemize}
\begin{itemize}
\item {Grp. gram.:M.}
\end{itemize}
\begin{itemize}
\item {Proveniência:(Do lat. \textunderscore Croatia\textunderscore , n. p.)}
\end{itemize}
Relativo á Croacia.
Habitante dessa região.
\section{Croatá}
\begin{itemize}
\item {Grp. gram.:m.}
\end{itemize}
Ananás silvestre.
\section{Croca}
\begin{itemize}
\item {Grp. gram.:f.}
\end{itemize}
\begin{itemize}
\item {Utilização:Prov.}
\end{itemize}
\begin{itemize}
\item {Utilização:minh.}
\end{itemize}
\begin{itemize}
\item {Utilização:Fig.}
\end{itemize}
Cylindro oco, que reveste e deixa girar livremente o eixo fixo dos carretes do vessadoiro.
Cavidade ou buraco, em madeira.
Ânus.
\section{Croca}
\begin{itemize}
\item {Grp. gram.:f.}
\end{itemize}
\begin{itemize}
\item {Utilização:T. da Bairrada}
\end{itemize}
Mulher, que não é amorável para os filhos.
Porca, que trata mal os seus leitões.
\section{Croca}
\begin{itemize}
\item {Grp. gram.:f.}
\end{itemize}
\begin{itemize}
\item {Utilização:Prov.}
\end{itemize}
\begin{itemize}
\item {Utilização:trasm.}
\end{itemize}
Castanha, assada no forno, sem que previamente seja golpeada para evitar que estoire.
\section{Croça}
\begin{itemize}
\item {Grp. gram.:f.}
\end{itemize}
\begin{itemize}
\item {Utilização:Ant.}
\end{itemize}
Capote de palha.
(Contr. de \textunderscore coroça\textunderscore )
\section{Croça}
\begin{itemize}
\item {Grp. gram.:f.}
\end{itemize}
\begin{itemize}
\item {Utilização:Anat.}
\end{itemize}
Bastão episcopal.
Parte recurva da aorta.
(Cast. ant. \textunderscore croza\textunderscore )
\section{Crocal}
\begin{itemize}
\item {Grp. gram.:m.}
\end{itemize}
\begin{itemize}
\item {Proveniência:(Do lat. \textunderscore crocus\textunderscore )}
\end{itemize}
Pedra preciosa, da côr da cereja.
\section{Cróceo}
\begin{itemize}
\item {Grp. gram.:adj.}
\end{itemize}
\begin{itemize}
\item {Proveniência:(Lat. \textunderscore croceus\textunderscore )}
\end{itemize}
Que tem côr de açafrão.
\section{Cròché}
\begin{itemize}
\item {Grp. gram.:m.}
\end{itemize}
\begin{itemize}
\item {Proveniência:(Fr. \textunderscore crochet\textunderscore )}
\end{itemize}
Renda, feita geralmente com uma só agulha especial.
\section{Crócico}
\begin{itemize}
\item {Grp. gram.:adj.}
\end{itemize}
\begin{itemize}
\item {Proveniência:(Do lat. \textunderscore crocus\textunderscore )}
\end{itemize}
Diz-se de um ácido, que se acha no producto volátil, formado pela acção do óxydo de carbóne sôbre o potássio.
\section{Crocidismo}
\begin{itemize}
\item {Grp. gram.:m.}
\end{itemize}
\begin{itemize}
\item {Proveniência:(Gr. \textunderscore krokidismos\textunderscore )}
\end{itemize}
Movimentos dos enfermos, como de quem procura apanhar fios na roupa da cama, e que é symptoma de febre atáxica.
\section{Crócino}
\begin{itemize}
\item {Grp. gram.:adj.}
\end{itemize}
O mesmo que \textunderscore cróceo\textunderscore .
\section{Crocípede}
\begin{itemize}
\item {Grp. gram.:adj.}
\end{itemize}
\begin{itemize}
\item {Utilização:Zool.}
\end{itemize}
\begin{itemize}
\item {Proveniência:(Do lat. \textunderscore crocus\textunderscore  + \textunderscore pes\textunderscore )}
\end{itemize}
Que tem os pés da côr do açafrão.
\section{Crocitante}
\begin{itemize}
\item {Grp. gram.:adj.}
\end{itemize}
Que crocita.
\section{Crocitar}
\begin{itemize}
\item {Grp. gram.:v. i.}
\end{itemize}
\begin{itemize}
\item {Proveniência:(Lat. \textunderscore crocitare\textunderscore )}
\end{itemize}
Gritar (o corvo).
Imitar a voz do corvo.
Corvejar.
\section{Crocito}
\begin{itemize}
\item {Grp. gram.:m.}
\end{itemize}
\begin{itemize}
\item {Proveniência:(De \textunderscore crocitar\textunderscore )}
\end{itemize}
A voz do corvo, do condor e de outras aves.
\section{Cróco}
\begin{itemize}
\item {Grp. gram.:m.}
\end{itemize}
\begin{itemize}
\item {Proveniência:(Lat. \textunderscore crocus\textunderscore )}
\end{itemize}
Planta, o mesmo que \textunderscore açafrão\textunderscore .
\section{Crôco}
\begin{itemize}
\item {Grp. gram.:adj.}
\end{itemize}
\begin{itemize}
\item {Utilização:Prov.}
\end{itemize}
\begin{itemize}
\item {Utilização:minh.}
\end{itemize}
\begin{itemize}
\item {Proveniência:(De \textunderscore croca\textunderscore ^1)}
\end{itemize}
Que tem cavidade.
Vazio no centro.
\section{Crocodílinos}
\begin{itemize}
\item {Grp. gram.:m. pl.}
\end{itemize}
Animaes fósseis do período secundário, parecidos ao crocodilo.
\section{Crocodilita}
\begin{itemize}
\item {Grp. gram.:f.}
\end{itemize}
Espécie de mineral da África do Sul.
\section{Crocodilo}
\begin{itemize}
\item {Grp. gram.:m.}
\end{itemize}
\begin{itemize}
\item {Proveniência:(Lat. \textunderscore crocodilus\textunderscore )}
\end{itemize}
Grande amphýbio das regiões intertropicaes.
\section{Crocoroca}
\begin{itemize}
\item {Grp. gram.:m.}
\end{itemize}
\begin{itemize}
\item {Utilização:Bras}
\end{itemize}
\begin{itemize}
\item {Proveniência:(T. onom.)}
\end{itemize}
Peixe que, ao sair da água, emitte sons que imitam o seu nome.
\section{Crocota}
\begin{itemize}
\item {Grp. gram.:f.}
\end{itemize}
\begin{itemize}
\item {Proveniência:(Lat. \textunderscore crocota\textunderscore )}
\end{itemize}
Antigo traje feminino da côr de açafrão, em uso entre os Gregos e Romanos.
\section{Crocótula}
\begin{itemize}
\item {Grp. gram.:f.}
\end{itemize}
\begin{itemize}
\item {Proveniência:(Lat. \textunderscore crocotula\textunderscore )}
\end{itemize}
Pequena e luxuosa crocota.
\section{Crocuta}
\begin{itemize}
\item {Grp. gram.:f.}
\end{itemize}
\begin{itemize}
\item {Proveniência:(Lat. \textunderscore crocuta\textunderscore )}
\end{itemize}
Espécie de hyena.
\section{Croia}
\begin{itemize}
\item {Grp. gram.:f.}
\end{itemize}
\begin{itemize}
\item {Utilização:Pop.}
\end{itemize}
Mulher desavergonhada; rameira.
(Metáth. de \textunderscore coira\textunderscore , por \textunderscore coiro\textunderscore )
\section{Cróia}
\begin{itemize}
\item {Grp. gram.:f.}
\end{itemize}
\begin{itemize}
\item {Utilização:Gír.}
\end{itemize}
Dona de casa.
\section{Croma}
\begin{itemize}
\item {Grp. gram.:f.}
\end{itemize}
\begin{itemize}
\item {Utilização:Mús.}
\end{itemize}
Escala cromática.
Melodia, que procede por semi-tons. Cf. Camillo, \textunderscore Cav. em Ruínas\textunderscore , 50.
\section{Cromado}
\begin{itemize}
\item {Grp. gram.:adj.}
\end{itemize}
Que tem cromo.
\section{Cromâmetro}
\begin{itemize}
\item {Grp. gram.:m.}
\end{itemize}
\begin{itemize}
\item {Utilização:Mús.}
\end{itemize}
\begin{itemize}
\item {Proveniência:(Do gr. \textunderscore khroma\textunderscore  + \textunderscore metron\textunderscore )}
\end{itemize}
Aparelho, hoje desusado, para exercício de quem aprende a afinar pianos, e que é uma espécie de monocórdio, com um braço em que estão marcadas todas as divisões da escala cromática.
\section{Cromática}
\begin{itemize}
\item {Grp. gram.:f.}
\end{itemize}
\begin{itemize}
\item {Proveniência:(Do gr. \textunderscore khroma\textunderscore )}
\end{itemize}
Arte de combinar as côres.
\section{Cromaticamente}
\begin{itemize}
\item {Grp. gram.:adv.}
\end{itemize}
Por semi-tons; de modo cromático.
\section{Cromático}
\begin{itemize}
\item {Grp. gram.:adj.}
\end{itemize}
\begin{itemize}
\item {Utilização:Mús.}
\end{itemize}
\begin{itemize}
\item {Proveniência:(Do gr. \textunderscore khroma\textunderscore )}
\end{itemize}
Relativo a côres, em Física.
Composto de uma série de semi-tons.
\section{Cromatina}
\begin{itemize}
\item {Grp. gram.:f.}
\end{itemize}
\begin{itemize}
\item {Utilização:Physiol.}
\end{itemize}
\begin{itemize}
\item {Proveniência:(Do gr. \textunderscore khroma\textunderscore )}
\end{itemize}
Substância, que entra na composição do núcleo celular, assim chamada pela sua afinidade com as matérias corantes.
\section{Cromatismo}
\begin{itemize}
\item {Grp. gram.:m.}
\end{itemize}
\begin{itemize}
\item {Utilização:Phýs.}
\end{itemize}
\begin{itemize}
\item {Proveniência:(Gr. \textunderscore khromatismos\textunderscore )}
\end{itemize}
Dispersão da luz.
Recomposição da luz, que atravessou corpos diáfanos.
\section{Cromato}
\begin{itemize}
\item {Grp. gram.:m.}
\end{itemize}
\begin{itemize}
\item {Proveniência:(De \textunderscore chromo\textunderscore )}
\end{itemize}
Combinação do ácido crómico com uma base.
\section{Cromatogênico}
\begin{itemize}
\item {Grp. gram.:adj.}
\end{itemize}
\begin{itemize}
\item {Proveniência:(Do gr. \textunderscore khroma\textunderscore  + \textunderscore genea\textunderscore )}
\end{itemize}
Diz-se de certos micróbios, que se revelam pela produção de côres.
\section{Crómico}
\begin{itemize}
\item {Grp. gram.:adj.}
\end{itemize}
\begin{itemize}
\item {Proveniência:(De \textunderscore chromo\textunderscore )}
\end{itemize}
Diz-se de um ácido, em que entra o chromo e o oxigênio.
Relativo a côres.
\section{Crómio}
\begin{itemize}
\item {Grp. gram.:m.}
\end{itemize}
O mesmo que \textunderscore cromo\textunderscore .
\section{Cromismo}
\begin{itemize}
\item {Grp. gram.:m.}
\end{itemize}
\begin{itemize}
\item {Utilização:Bot.}
\end{itemize}
\begin{itemize}
\item {Proveniência:(Do gr. \textunderscore khroma\textunderscore )}
\end{itemize}
Excesso anómalo de coloração.
\section{Cromite}
\begin{itemize}
\item {Grp. gram.:f.}
\end{itemize}
\begin{itemize}
\item {Utilização:Geol.}
\end{itemize}
\begin{itemize}
\item {Proveniência:(De \textunderscore chromo\textunderscore )}
\end{itemize}
Espécie de espinela.
\section{Cromo}
\begin{itemize}
\item {Grp. gram.:m.}
\end{itemize}
\begin{itemize}
\item {Proveniência:(Do gr. \textunderscore khroma\textunderscore )}
\end{itemize}
Metal cinzento, que se encontra no ferro e noutros corpos.
Desenho impresso a côres.
\section{Cromofilia}
\begin{itemize}
\item {Grp. gram.:f.}
\end{itemize}
\begin{itemize}
\item {Utilização:Neol.}
\end{itemize}
\begin{itemize}
\item {Proveniência:(De \textunderscore chromóphilo\textunderscore )}
\end{itemize}
Grande afeição ás côres vivas.
\section{Cromófilo}
\begin{itemize}
\item {Grp. gram.:adj.}
\end{itemize}
\begin{itemize}
\item {Proveniência:(Do gr. \textunderscore khroma\textunderscore  + \textunderscore philos\textunderscore )}
\end{itemize}
Que gosta de côres vivas.
\section{Cromofitose}
\begin{itemize}
\item {Grp. gram.:f.}
\end{itemize}
\begin{itemize}
\item {Utilização:Med.}
\end{itemize}
\begin{itemize}
\item {Proveniência:(Do gr. \textunderscore khroma\textunderscore  + \textunderscore phuton\textunderscore )}
\end{itemize}
Doença cutânea, conhecida vulgarmente por \textunderscore pano\textunderscore , (\textunderscore pannus hepaticus\textunderscore ), e que, sendo afecção parasitária, póde atacar o tronco e as extremidades superiores do corpo.
\section{Cromóforo}
\begin{itemize}
\item {Grp. gram.:m.}
\end{itemize}
\begin{itemize}
\item {Utilização:Zool.}
\end{itemize}
\begin{itemize}
\item {Proveniência:(Do gr. \textunderscore khroma\textunderscore  + \textunderscore phoros\textunderscore )}
\end{itemize}
Folículo colorido, que guarnece o corpo dos cafalópodes.
\section{Cromogênio}
\begin{itemize}
\item {Grp. gram.:m.}
\end{itemize}
\begin{itemize}
\item {Proveniência:(Do gr. \textunderscore khroma\textunderscore  + \textunderscore genea\textunderscore )}
\end{itemize}
Diz-se de certo micróbio, que dá coloração verde á neve, sôbre que vive.
\section{Cromógrafo}
\begin{itemize}
\item {Grp. gram.:m.}
\end{itemize}
\begin{itemize}
\item {Proveniência:(Do gr. \textunderscore khroma\textunderscore  + \textunderscore graphein\textunderscore )}
\end{itemize}
Aparelho de balística, para medir a velocidade dos projécteis e o tempo que gastam no seu percurso.
\section{Cromolitografia}
\begin{itemize}
\item {Grp. gram.:f.}
\end{itemize}
\begin{itemize}
\item {Proveniência:(De \textunderscore chromo\textunderscore  + \textunderscore lithographia\textunderscore )}
\end{itemize}
Litografia a côres.
\section{Cromolitográfico}
\begin{itemize}
\item {Grp. gram.:adj.}
\end{itemize}
Relativo á cromo-litografia.
\section{Cromoterapia}
\begin{itemize}
\item {Grp. gram.:f.}
\end{itemize}
Tratamento médico pela acção das côres.
\section{Cromotipografia}
\begin{itemize}
\item {Grp. gram.:f.}
\end{itemize}
Processo de impressão a côres.
\section{Crómula}
\begin{itemize}
\item {Grp. gram.:f.}
\end{itemize}
\begin{itemize}
\item {Proveniência:(Do gr. \textunderscore khroma\textunderscore  + \textunderscore ule\textunderscore )}
\end{itemize}
O mesmo que \textunderscore clorofila\textunderscore .
\section{Cromurgia}
\begin{itemize}
\item {Grp. gram.:f.}
\end{itemize}
\begin{itemize}
\item {Proveniência:(Do gr. \textunderscore khroma\textunderscore  + \textunderscore ergon\textunderscore )}
\end{itemize}
Parte da Química, que trata das côres e das tintas.
\section{Cromúrgico}
\begin{itemize}
\item {Grp. gram.:adj.}
\end{itemize}
Relativo á cromurgia.
\section{Cronha}
\begin{itemize}
\item {Grp. gram.:f.}
\end{itemize}
(Fórma pop. de \textunderscore coronha\textunderscore . Cf. Camillo, \textunderscore Doze Casam.\textunderscore , 221)
\section{Crónica}
\begin{itemize}
\item {Grp. gram.:f.}
\end{itemize}
\begin{itemize}
\item {Proveniência:(Lat. \textunderscore chronica\textunderscore , pl. de \textunderscore chronicum\textunderscore )}
\end{itemize}
Narração histórica, segundo a ordem dos tempos.
Noticiário dos periódicos.
Revista cientifica ou literária, que preenche periodicamente uma secção de jornal.
\section{Cronicamente}
\begin{itemize}
\item {Grp. gram.:adv.}
\end{itemize}
De modo crónico.
\section{Cronicão}
\begin{itemize}
\item {Grp. gram.:m.}
\end{itemize}
Volumosa crónica medieval. Cf. Camillo, \textunderscore Quéda\textunderscore , 9.
(B. lat. \textunderscore chronicon\textunderscore )
\section{Cronicidade}
\begin{itemize}
\item {Grp. gram.:f.}
\end{itemize}
\begin{itemize}
\item {Proveniência:(De \textunderscore chrónico\textunderscore )}
\end{itemize}
Qualidade das doenças crónicas.
\section{Crónico}
\begin{itemize}
\item {Grp. gram.:adj.}
\end{itemize}
\begin{itemize}
\item {Utilização:Fig.}
\end{itemize}
\begin{itemize}
\item {Proveniência:(Lat. \textunderscore chronicus\textunderscore )}
\end{itemize}
Que dura há muito tempo.
Inveterado: \textunderscore doenças crónicas\textunderscore .
\section{Crónicon}
\begin{itemize}
\item {Grp. gram.:m.}
\end{itemize}
(V.cronicão). Cf. Herculano, \textunderscore Hist. de Port.\textunderscore , I, 2, 182, 483.
\section{Croniqueiro}
\begin{itemize}
\item {Grp. gram.:m.}
\end{itemize}
\begin{itemize}
\item {Utilização:Fam.}
\end{itemize}
\begin{itemize}
\item {Proveniência:(De \textunderscore chrónica\textunderscore )}
\end{itemize}
Noticiarista na imprensa.
\section{Croniquizar}
\begin{itemize}
\item {Grp. gram.:v. t.}
\end{itemize}
Reduzir a uma crónica; narrar em crónica.
\section{Cronista}
\begin{itemize}
\item {Grp. gram.:m.}
\end{itemize}
\begin{itemize}
\item {Proveniência:(Do gr. \textunderscore khronos\textunderscore , tempo)}
\end{itemize}
Aquele que escreve crónicas.
\section{Cronizoico}
\begin{itemize}
\item {Grp. gram.:adj.}
\end{itemize}
\begin{itemize}
\item {Utilização:Pharm.}
\end{itemize}
Diz-se do medicamento oficinal, já preparado; oficinal.
\section{Crono}
\begin{itemize}
\item {Grp. gram.:m.}
\end{itemize}
\begin{itemize}
\item {Utilização:Geol.}
\end{itemize}
\begin{itemize}
\item {Proveniência:(Do gr. \textunderscore khronos\textunderscore )}
\end{itemize}
Lapso de tempo, correspondente a um \textunderscore andar\textunderscore , uma das subdivisões das séries em que se divide o conjunto dos terrenos sedimentares.
\section{Cronofotografia}
\begin{itemize}
\item {Grp. gram.:f.}
\end{itemize}
\begin{itemize}
\item {Proveniência:(Do gr. \textunderscore khronos\textunderscore  + \textunderscore photos\textunderscore  + \textunderscore graphein\textunderscore )}
\end{itemize}
Processo fotográfico, para analisar os movimentos de um objecto móvel, tirando fotografias instantâneas, com intervalos regularmente espaçados.
\section{Cronofotográfico}
\begin{itemize}
\item {Grp. gram.:adj.}
\end{itemize}
Relativo á cronofotografia.
\section{Cronografia}
\begin{itemize}
\item {Grp. gram.:f.}
\end{itemize}
\begin{itemize}
\item {Proveniência:(Do gr. \textunderscore kronos\textunderscore  + \textunderscore graphein\textunderscore )}
\end{itemize}
Descripção do planeta Saturno.
\textunderscore f.\textunderscore  (e der.)
O mesmo que \textunderscore cronologia\textunderscore , etc.
\section{Cronográfico}
\begin{itemize}
\item {Grp. gram.:adj.}
\end{itemize}
Relativo á cronografia.
\section{Cronographia}
\begin{itemize}
\item {Grp. gram.:f.}
\end{itemize}
\begin{itemize}
\item {Proveniência:(Do gr. \textunderscore kronos\textunderscore  + \textunderscore graphein\textunderscore )}
\end{itemize}
Descripção do planeta Saturno.
\section{Cronográphico}
\begin{itemize}
\item {Grp. gram.:adj.}
\end{itemize}
Relativo á cronographia.
\section{Cronograma}
\begin{itemize}
\item {Grp. gram.:m.}
\end{itemize}
\begin{itemize}
\item {Grp. gram.:m.}
\end{itemize}
\begin{itemize}
\item {Proveniência:(Do gr. \textunderscore khronos\textunderscore  + \textunderscore gramma\textunderscore )}
\end{itemize}
Data enigmática, formada de letras numeraes romanas, espalhadas por diferentes palavras de que fazem parte.
Inscripção, em que as letras numeraes, em cifras romanas, indicam a data de um acontecimento.
\section{Cronogramático}
\begin{itemize}
\item {Grp. gram.:adj.}
\end{itemize}
Que contém cronograma.
\section{Cronologia}
\begin{itemize}
\item {Grp. gram.:f.}
\end{itemize}
\begin{itemize}
\item {Proveniência:(Do gr. \textunderscore khronos\textunderscore  + \textunderscore logos\textunderscore )}
\end{itemize}
Tratado das divisões do tempo.
Tratado das datas hístóricas.
\section{Cronologicamente}
\begin{itemize}
\item {Grp. gram.:adv.}
\end{itemize}
\begin{itemize}
\item {Proveniência:(De \textunderscore chronológico\textunderscore )}
\end{itemize}
Segundo a ordem dos tempos.
\section{Cronológico}
\begin{itemize}
\item {Grp. gram.:adj.}
\end{itemize}
Relativo a cronologia.
\section{Cronologista}
\begin{itemize}
\item {Grp. gram.:m.}
\end{itemize}
Aquele que é versado em cronologia.
\section{Cronólogo}
\begin{itemize}
\item {Grp. gram.:m.}
\end{itemize}
O mesmo que cronologista.
\section{Cronometria}
\begin{itemize}
\item {Grp. gram.:f.}
\end{itemize}
\begin{itemize}
\item {Proveniência:(De \textunderscore chronómetro\textunderscore )}
\end{itemize}
Medida do tempo.
\section{Cronometricamente}
\begin{itemize}
\item {Grp. gram.:adv.}
\end{itemize}
De modo cronométrico; á maneira de cronómetro.
\section{Cronométrico}
\begin{itemize}
\item {Grp. gram.:adj.}
\end{itemize}
Relativo á cronometria.
\section{Cronometrista}
\begin{itemize}
\item {Grp. gram.:m.}
\end{itemize}
Aquele que fabríca cronómetros.
\section{Cronómetro}
\begin{itemize}
\item {Grp. gram.:m.}
\end{itemize}
\begin{itemize}
\item {Proveniência:(Do gr. \textunderscore khronos\textunderscore  + \textunderscore metron\textunderscore )}
\end{itemize}
Instrumento, com que se mede o tempo.
Relógio perfeito.
\section{Cronoscópio}
\begin{itemize}
\item {Grp. gram.:m.}
\end{itemize}
O mesmo que \textunderscore cronómetro\textunderscore .
\section{Croque}
\begin{itemize}
\item {Grp. gram.:m.}
\end{itemize}
\begin{itemize}
\item {Utilização:Prov.}
\end{itemize}
\begin{itemize}
\item {Utilização:beir.}
\end{itemize}
\begin{itemize}
\item {Proveniência:(Fr. \textunderscore croc\textunderscore )}
\end{itemize}
Vara, com um gancho em uma extremidade, e de que os barqueiros se servem para atracar os barcos.
Pau, com que os gandaieiros mexem e escolhem os trapos no lixo.
Carolo; pequena pancada na cabeça, com vara ou cana.
\section{Croquete}
\begin{itemize}
\item {fónica:quê}
\end{itemize}
\begin{itemize}
\item {Grp. gram.:m.}
\end{itemize}
Bolo, espécie de almôndega sem môlho. Cf. Castilho, \textunderscore Avarento\textunderscore , 182.
\section{Crosca}
\begin{itemize}
\item {fónica:crôs}
\end{itemize}
\begin{itemize}
\item {Grp. gram.:f.}
\end{itemize}
\begin{itemize}
\item {Utilização:Prov.}
\end{itemize}
\begin{itemize}
\item {Utilização:minh.}
\end{itemize}
O mesmo que \textunderscore crosta\textunderscore .
\section{Crosta}
\begin{itemize}
\item {fónica:crôs}
\end{itemize}
\begin{itemize}
\item {Grp. gram.:f.}
\end{itemize}
\begin{itemize}
\item {Proveniência:(Lat. \textunderscore crusta\textunderscore )}
\end{itemize}
Camada espessa e dura de um corpo.
Invólucro.
Casca.
Escama.
Crusta; côdea.
\section{Crosto}
\begin{itemize}
\item {fónica:crôs}
\end{itemize}
\begin{itemize}
\item {Grp. gram.:m.}
\end{itemize}
(Corr. de \textunderscore colostro\textunderscore )
\section{Crotafal}
\begin{itemize}
\item {Grp. gram.:adj.}
\end{itemize}
\begin{itemize}
\item {Utilização:Anat.}
\end{itemize}
\begin{itemize}
\item {Proveniência:(Do gr. \textunderscore krotaphos\textunderscore )}
\end{itemize}
Relativo ás fontes da cabeça.
\section{Crotáfico}
\begin{itemize}
\item {Grp. gram.:adj.}
\end{itemize}
\begin{itemize}
\item {Utilização:Anat.}
\end{itemize}
\begin{itemize}
\item {Proveniência:(Do gr. \textunderscore krotaphos\textunderscore )}
\end{itemize}
Relativo ás fontes da cabeça.
\section{Crotafito}
\begin{itemize}
\item {Grp. gram.:m.}
\end{itemize}
\begin{itemize}
\item {Utilização:Anat.}
\end{itemize}
\begin{itemize}
\item {Proveniência:(Gr. \textunderscore krotaphités\textunderscore )}
\end{itemize}
Músculo da região temporal.
\section{Crotalária}
\begin{itemize}
\item {Grp. gram.:f.}
\end{itemize}
\begin{itemize}
\item {Proveniência:(Do gr. \textunderscore krotalon\textunderscore )}
\end{itemize}
Planta papilionácea.
\section{Crotália}
\begin{itemize}
\item {Grp. gram.:f.}
\end{itemize}
\begin{itemize}
\item {Proveniência:(Lat. \textunderscore crotalia\textunderscore )}
\end{itemize}
Espécie de pérola.
\section{Crótalo}
\begin{itemize}
\item {Grp. gram.:m.}
\end{itemize}
\begin{itemize}
\item {Proveniência:(Do gr. \textunderscore krotalon\textunderscore )}
\end{itemize}
Antigo instrumento, semelhante a castanholas.
Cobra cascavel.
\section{Crotaloides}
\begin{itemize}
\item {Grp. gram.:m. pl.}
\end{itemize}
\begin{itemize}
\item {Proveniência:(De \textunderscore krotalon\textunderscore  + \textunderscore eidos\textunderscore )}
\end{itemize}
Família de serpentes, que têm por typo o crótalo, cobra.
\section{Crotaphal}
\begin{itemize}
\item {Grp. gram.:adj.}
\end{itemize}
\begin{itemize}
\item {Utilização:Anat.}
\end{itemize}
\begin{itemize}
\item {Proveniência:(Do gr. \textunderscore krotaphos\textunderscore )}
\end{itemize}
Relativo ás fontes da cabeça.
\section{Crotáphico}
\begin{itemize}
\item {Grp. gram.:adj.}
\end{itemize}
\begin{itemize}
\item {Utilização:Anat.}
\end{itemize}
\begin{itemize}
\item {Proveniência:(Do gr. \textunderscore krotaphos\textunderscore )}
\end{itemize}
Relativo ás fontes da cabeça.
\section{Crotaphito}
\begin{itemize}
\item {Grp. gram.:m.}
\end{itemize}
\begin{itemize}
\item {Utilização:Anat.}
\end{itemize}
\begin{itemize}
\item {Proveniência:(Gr. \textunderscore krotaphités\textunderscore )}
\end{itemize}
Músculo da região temporal.
\section{Crotófaga}
\begin{itemize}
\item {Grp. gram.:f.}
\end{itemize}
\begin{itemize}
\item {Proveniência:(Do gr. \textunderscore croton\textunderscore  + \textunderscore phagein\textunderscore )}
\end{itemize}
Ave trepadora da América.
\section{Crotofagíneas}
\begin{itemize}
\item {Grp. gram.:f. pl.}
\end{itemize}
Aves cuculídeas, que tem por tipo a crotófaga.
\section{Crotófago}
\begin{itemize}
\item {Grp. gram.:m.}
\end{itemize}
(V.crotófaga)
\section{Cróton}
\begin{itemize}
\item {Grp. gram.:m.}
\end{itemize}
\begin{itemize}
\item {Proveniência:(Gr. \textunderscore kroton\textunderscore )}
\end{itemize}
Planta euphorbiácea, de cujas sementes se extrai um óleo purgativo.
\section{Crotoniata}
\begin{itemize}
\item {Grp. gram.:m.}
\end{itemize}
\begin{itemize}
\item {Proveniência:(Lat. \textunderscore crotoniates\textunderscore )}
\end{itemize}
Aquelle que seguia os princípios da escola philosóphica de Crotona. Cf. Latino, \textunderscore Or. da Corôa\textunderscore , CLXXIII.
\section{Crotónico}
\begin{itemize}
\item {Grp. gram.:adj.}
\end{itemize}
Diz-se de um ácido, que se acha nas sementes do cróton.
\section{Crotonileno}
\begin{itemize}
\item {Grp. gram.:m.}
\end{itemize}
\begin{itemize}
\item {Utilização:Chím.}
\end{itemize}
Um dos carbonetos do grupo acetylênico.
\section{Crotonina}
\begin{itemize}
\item {Grp. gram.:f.}
\end{itemize}
Alcaloide muito enérgico, encontrado nas sementes do cróton.
\section{Crotonópsida}
\begin{itemize}
\item {Grp. gram.:f.}
\end{itemize}
\begin{itemize}
\item {Proveniência:(Do gr. \textunderscore kroton\textunderscore  + \textunderscore opsis\textunderscore )}
\end{itemize}
Planta, semelhante ao cróton, e natural da América do Norte.
\section{Crotóphaga}
\begin{itemize}
\item {Grp. gram.:f.}
\end{itemize}
\begin{itemize}
\item {Proveniência:(Do gr. \textunderscore croton\textunderscore  + \textunderscore phagein\textunderscore )}
\end{itemize}
Ave trepadora da América.
\section{Crotophagíneas}
\begin{itemize}
\item {Grp. gram.:f. pl.}
\end{itemize}
Aves cuculídeas, que tem por typo a crotóphaga.
\section{Crotóphago}
\begin{itemize}
\item {Grp. gram.:m.}
\end{itemize}
(V.crotóphaga)
\section{Croxa}
\begin{itemize}
\item {Grp. gram.:f.}
\end{itemize}
\begin{itemize}
\item {Utilização:Prov.}
\end{itemize}
\begin{itemize}
\item {Utilização:minh.}
\end{itemize}
Bandeira do milho.
\section{Crozóforo}
\begin{itemize}
\item {Grp. gram.:m.}
\end{itemize}
\begin{itemize}
\item {Proveniência:(Do gr. \textunderscore krossos\textunderscore  + \textunderscore phoros\textunderscore )}
\end{itemize}
Planta euphorbiácea da África.
\section{Crozóphoro}
\begin{itemize}
\item {Grp. gram.:m.}
\end{itemize}
\begin{itemize}
\item {Proveniência:(Do gr. \textunderscore krossos\textunderscore  + \textunderscore phoros\textunderscore )}
\end{itemize}
Planta euphorbiácea da África.
\section{Cru}
\begin{itemize}
\item {Grp. gram.:adj.}
\end{itemize}
\begin{itemize}
\item {Proveniência:(Lat. \textunderscore crudus\textunderscore )}
\end{itemize}
Sangrento, (des. neste sentido).
Que ainda não está cozido: \textunderscore carne crua\textunderscore .
Que ainda não teve preparação.
Que ainda não tem madureza, (no sentido moral).
Esboçado, incipiente.
Em que não há disfarce: \textunderscore verdades cruas\textunderscore .
Que resai sem transição suave.
Áspero.
Offensivo: \textunderscore palavras cruas\textunderscore .
Cruel: \textunderscore Pedro I, o Cru\textunderscore .
\section{Cruá}
\begin{itemize}
\item {Grp. gram.:f.}
\end{itemize}
Planta cucurbitácea do Brasil, espécie de abóboreira.
\section{Crubula}
\begin{itemize}
\item {Grp. gram.:f.}
\end{itemize}
Árvore de Timor.
\section{Cruciação}
\begin{itemize}
\item {Grp. gram.:f.}
\end{itemize}
\begin{itemize}
\item {Proveniência:(Lat. \textunderscore cruciatio\textunderscore )}
\end{itemize}
Acto ou effeito de cruciar.
\section{Cruciador}
\begin{itemize}
\item {Grp. gram.:m.  e  adj.}
\end{itemize}
\begin{itemize}
\item {Proveniência:(Lat. \textunderscore cruciator\textunderscore )}
\end{itemize}
O que crucia.
\section{Crucial}
\begin{itemize}
\item {Grp. gram.:adj.}
\end{itemize}
\begin{itemize}
\item {Proveniência:(Do lat. \textunderscore crux\textunderscore )}
\end{itemize}
Que tem fórma de cruz.
\section{Cruciana}
\begin{itemize}
\item {Grp. gram.:f.}
\end{itemize}
\begin{itemize}
\item {Proveniência:(Do lat. \textunderscore crux\textunderscore )}
\end{itemize}
Espécie de bambu, no Brasil.
\section{Crucianela}
\begin{itemize}
\item {Grp. gram.:f.}
\end{itemize}
\begin{itemize}
\item {Proveniência:(Do lat. \textunderscore crux\textunderscore )}
\end{itemize}
Planta rubiácea.
\section{Cruciante}
\begin{itemize}
\item {Grp. gram.:adj.}
\end{itemize}
\begin{itemize}
\item {Proveniência:(Do lat. \textunderscore crucians\textunderscore )}
\end{itemize}
Que crucia.
\section{Cruciar}
\begin{itemize}
\item {Grp. gram.:v. t.}
\end{itemize}
\begin{itemize}
\item {Proveniência:(Lat. \textunderscore cruciare\textunderscore )}
\end{itemize}
Torturar.
Affligir muito.
Mortificar.
\section{Cruciário}
\begin{itemize}
\item {Grp. gram.:adj.}
\end{itemize}
\begin{itemize}
\item {Proveniência:(Lat. \textunderscore cruciarius\textunderscore )}
\end{itemize}
O mesmo que \textunderscore cruciante\textunderscore .
\section{Cruciato}
\begin{itemize}
\item {Grp. gram.:m.}
\end{itemize}
\begin{itemize}
\item {Proveniência:(Lat. \textunderscore cruciatus\textunderscore )}
\end{itemize}
O mesmo que \textunderscore cruciação\textunderscore .
\section{Cruciferário}
\begin{itemize}
\item {Grp. gram.:m.}
\end{itemize}
\begin{itemize}
\item {Proveniência:(Do b. lat. \textunderscore crucifer\textunderscore )}
\end{itemize}
Aquelle que leva a cruz nas procissões.
\section{Crucíferas}
\begin{itemize}
\item {Grp. gram.:f. pl.}
\end{itemize}
\begin{itemize}
\item {Utilização:Bot.}
\end{itemize}
\begin{itemize}
\item {Proveniência:(De \textunderscore crucífero\textunderscore )}
\end{itemize}
Numerosa família de plantas, cujas flôres têm as pétalas em fórma de cruz.
\section{Crucífero}
\begin{itemize}
\item {Grp. gram.:adj.}
\end{itemize}
\begin{itemize}
\item {Utilização:Bot.}
\end{itemize}
\begin{itemize}
\item {Grp. gram.:M.}
\end{itemize}
\begin{itemize}
\item {Proveniência:(Do lat. \textunderscore crux\textunderscore  + \textunderscore ferre\textunderscore )}
\end{itemize}
A que é sobreposta uma cruz ou destinado a suster uma cruz.
Que tem flôres em fórma de cruz.
O mesmo que \textunderscore cruciferário\textunderscore . Cf. \textunderscore Ritual dos Cistersienses\textunderscore .
\section{Crucificação}
\begin{itemize}
\item {Grp. gram.:f.}
\end{itemize}
Acto ou effeito de crucificar.
\section{Crucificado}
\begin{itemize}
\item {Grp. gram.:m.}
\end{itemize}
Aquelle que soffreu o supplicio da cruz.
Christo: \textunderscore uma imagem do Crucificado\textunderscore .
\section{Crucificador}
\begin{itemize}
\item {Grp. gram.:m.}
\end{itemize}
Aquelle que crucifica.
\section{Crucificamento}
\begin{itemize}
\item {Grp. gram.:m.}
\end{itemize}
O mesmo que \textunderscore crucificação\textunderscore .
\section{Crucificar}
\begin{itemize}
\item {Grp. gram.:v. t.}
\end{itemize}
\begin{itemize}
\item {Utilização:Fig.}
\end{itemize}
Pregar na cruz.
Affligir muito; torturar.
(Talvez por \textunderscore crucifixar\textunderscore )
\section{Crucifixão}
\begin{itemize}
\item {Grp. gram.:f.}
\end{itemize}
(V.crucificação)
\section{Crucifixar}
\begin{itemize}
\item {Grp. gram.:v. t.}
\end{itemize}
\begin{itemize}
\item {Proveniência:(De \textunderscore crucifixo\textunderscore )}
\end{itemize}
O mesmo que \textunderscore crucificar\textunderscore . Us. por Camillo.
\section{Crucifixo}
\begin{itemize}
\item {Grp. gram.:m.}
\end{itemize}
\begin{itemize}
\item {Grp. gram.:Adj.}
\end{itemize}
\begin{itemize}
\item {Proveniência:(Lat. \textunderscore crucifixus\textunderscore )}
\end{itemize}
Esculptura ou quadro, que representa Christo na cruz.
O mesmo que \textunderscore crucificado\textunderscore .
\section{Crucifloras}
\begin{itemize}
\item {Grp. gram.:f. pl.}
\end{itemize}
\begin{itemize}
\item {Proveniência:(Do lat. \textunderscore crux\textunderscore , \textunderscore crucis\textunderscore  + \textunderscore flos\textunderscore , \textunderscore floris\textunderscore )}
\end{itemize}
Ordem de plantas, que abrange as crucíferas e outras.
\section{Cruciforme}
\begin{itemize}
\item {Grp. gram.:adj.}
\end{itemize}
\begin{itemize}
\item {Proveniência:(Do lat. \textunderscore crux\textunderscore  + \textunderscore forma\textunderscore )}
\end{itemize}
Que tem fórma de cruz.
\section{Crucigênia}
\begin{itemize}
\item {Grp. gram.:f.}
\end{itemize}
\begin{itemize}
\item {Proveniência:(Do lat. \textunderscore crux\textunderscore  + \textunderscore gignere\textunderscore )}
\end{itemize}
Alga microscópica.
\section{Crucígero}
\begin{itemize}
\item {Grp. gram.:adj.}
\end{itemize}
\begin{itemize}
\item {Proveniência:(Do lat. \textunderscore crux\textunderscore  + \textunderscore gerere\textunderscore )}
\end{itemize}
O mesmo que \textunderscore crucífero\textunderscore .
\section{Crucirostro}
\begin{itemize}
\item {fónica:rós}
\end{itemize}
\begin{itemize}
\item {Grp. gram.:adj.}
\end{itemize}
\begin{itemize}
\item {Utilização:Zool.}
\end{itemize}
\begin{itemize}
\item {Proveniência:(Do lat. \textunderscore crux\textunderscore  + \textunderscore rostrum\textunderscore )}
\end{itemize}
Que tem o bico cruzado.
\section{Crucirrostro}
\begin{itemize}
\item {Grp. gram.:adj.}
\end{itemize}
\begin{itemize}
\item {Utilização:Zool.}
\end{itemize}
\begin{itemize}
\item {Proveniência:(Do lat. \textunderscore crux\textunderscore  + \textunderscore rostrum\textunderscore )}
\end{itemize}
Que tem o bico cruzado.
\section{Crudelíssimo}
\begin{itemize}
\item {Grp. gram.:adj.}
\end{itemize}
(Sup. de \textunderscore cruel\textunderscore )
\section{Crudívoro}
\begin{itemize}
\item {Grp. gram.:adj.}
\end{itemize}
\begin{itemize}
\item {Proveniência:(Do lat. \textunderscore crudus\textunderscore  + \textunderscore vorare\textunderscore )}
\end{itemize}
Que usa alimentos crus.
\section{Crué}
\begin{itemize}
\item {Grp. gram.:m.}
\end{itemize}
O mesmo que \textunderscore cruó\textunderscore .
\section{Crueira}
\begin{itemize}
\item {Grp. gram.:f.}
\end{itemize}
\begin{itemize}
\item {Utilização:Bras}
\end{itemize}
Fragmentos de mandioca ralada, que não passam pelas malhas da peneira.
(Corr. do tupi \textunderscore curuera\textunderscore )
\section{Crueira}
\begin{itemize}
\item {Grp. gram.:f.}
\end{itemize}
\begin{itemize}
\item {Utilização:Bras}
\end{itemize}
Doença das gallinhas, manifestada por uma massa amarelada, que se lhes fórma na boca.
(Talvez corr. de \textunderscore caruara\textunderscore )
\section{Crueiro}
\begin{itemize}
\item {Grp. gram.:m.}
\end{itemize}
\begin{itemize}
\item {Utilização:Prov.}
\end{itemize}
\begin{itemize}
\item {Utilização:trasm.}
\end{itemize}
\begin{itemize}
\item {Proveniência:(De \textunderscore cru\textunderscore )}
\end{itemize}
Terra magra.
\section{Cruel}
\begin{itemize}
\item {Grp. gram.:adj.}
\end{itemize}
\begin{itemize}
\item {Proveniência:(Lat. \textunderscore crudelis\textunderscore )}
\end{itemize}
Que se compraz em fazer mal, em torturar.
Que tortura, que afflige: \textunderscore injúria cruel\textunderscore .
Que usa de tyrannia.
Severo.
Doloroso.
Sanguinolento.
Insensivel: \textunderscore homem cruel\textunderscore .
\section{Crueldade}
\begin{itemize}
\item {Grp. gram.:f.}
\end{itemize}
\begin{itemize}
\item {Proveniência:(Lat. \textunderscore crudelitas\textunderscore )}
\end{itemize}
Qualidade daquillo ou de quem é cruel.
Acto cruel.
Rigor.
\section{Cruelíssimo}
\begin{itemize}
\item {Grp. gram.:adj.}
\end{itemize}
O mesmo que \textunderscore crudelíssimo\textunderscore . Cf. Camillo, \textunderscore Retr. de Ricard.\textunderscore , 211.
\section{Cruentação}
\begin{itemize}
\item {Grp. gram.:f.}
\end{itemize}
Acto de cruentar.
\section{Cruentar}
\begin{itemize}
\item {Grp. gram.:v. t.}
\end{itemize}
\begin{itemize}
\item {Proveniência:(Lat. \textunderscore cruentare\textunderscore )}
\end{itemize}
Ensanguentar.
\section{Cruento}
\begin{itemize}
\item {Grp. gram.:adj.}
\end{itemize}
\begin{itemize}
\item {Proveniência:(Lat. \textunderscore cruentus\textunderscore )}
\end{itemize}
Ensanguentado.
Em que há sangue: \textunderscore combate cruento\textunderscore .
Banhado em sangue.
Cruel.
\section{Cruera}
\begin{itemize}
\item {Grp. gram.:f.}
\end{itemize}
\begin{itemize}
\item {Utilização:Bras}
\end{itemize}
O mesmo que \textunderscore crueira\textunderscore ^1.
\section{Crueza}
\begin{itemize}
\item {Grp. gram.:f.}
\end{itemize}
Estado daquillo que é cru.
Digestão diffícil.
Crueldade.
\section{Cruga}
\begin{itemize}
\item {Grp. gram.:f.}
\end{itemize}
Espécie de couve.
\section{Crumatá}
\begin{itemize}
\item {Grp. gram.:m.}
\end{itemize}
\begin{itemize}
\item {Utilização:Bras}
\end{itemize}
Peixe de água doce.
\section{Crumenária}
\begin{itemize}
\item {Grp. gram.:f.}
\end{itemize}
\begin{itemize}
\item {Proveniência:(Do lat. \textunderscore crumena\textunderscore )}
\end{itemize}
Planta do Brasil.
\section{Crunha}
\begin{itemize}
\item {Grp. gram.:f.}
\end{itemize}
\begin{itemize}
\item {Utilização:Prov.}
\end{itemize}
\begin{itemize}
\item {Utilização:trasm.}
\end{itemize}
O mesmo que \textunderscore carunha\textunderscore .
\section{Cruo}
\begin{itemize}
\item {Grp. gram.:adj.}
\end{itemize}
\begin{itemize}
\item {Utilização:Ant.}
\end{itemize}
O mesmo que \textunderscore cru\textunderscore .
\section{Cruó}
\begin{itemize}
\item {Grp. gram.:f.}
\end{itemize}
\begin{itemize}
\item {Utilização:Prov.}
\end{itemize}
O mesmo que \textunderscore noitibó\textunderscore .
(Colhido na Bairrada)
\section{Cruor}
\begin{itemize}
\item {Grp. gram.:m.}
\end{itemize}
\begin{itemize}
\item {Proveniência:(Lat. \textunderscore cruor\textunderscore )}
\end{itemize}
Sangue, que escorre.
Elemento còrante do sangue.
A parte do sangue, que se coagula.
\section{Cruórico}
\begin{itemize}
\item {Grp. gram.:adj.}
\end{itemize}
Diz-se do sangue que coagula, por opposição ao soro:«\textunderscore sangue cruórico, que por vezes lhe borbulha nas artérias\textunderscore ». Camillo, \textunderscore Annos de Prosa\textunderscore , 113.
\section{Crupe}
\begin{itemize}
\item {Grp. gram.:m.}
\end{itemize}
\begin{itemize}
\item {Proveniência:(Fr. \textunderscore croup\textunderscore )}
\end{itemize}
Espécie de angina, o mesmo que garrotilho.
\section{Crupe}
\begin{itemize}
\item {Grp. gram.:m.}
\end{itemize}
Peça de artilharia, fabricada nas officinas de Krupp, ou segundo o systema de Krupp.
\section{Crupina}
\begin{itemize}
\item {Grp. gram.:f.}
\end{itemize}
Espécie de centáurea.
\section{Crural}
\begin{itemize}
\item {Grp. gram.:adj.}
\end{itemize}
\begin{itemize}
\item {Proveniência:(Lat. \textunderscore cruralis\textunderscore )}
\end{itemize}
Relativo á coxa: \textunderscore inflammação crural\textunderscore .
\section{Cruscantismo}
\begin{itemize}
\item {Grp. gram.:m.}
\end{itemize}
\begin{itemize}
\item {Proveniência:(De \textunderscore Crusca\textunderscore , n. p. de uma célebre academia de Florença)}
\end{itemize}
Purismo exaggerado, (falando-se da língua italiana).
\section{Crusta}
\begin{itemize}
\item {Grp. gram.:f.}
\end{itemize}
\begin{itemize}
\item {Proveniência:(Lat. \textunderscore crusta\textunderscore )}
\end{itemize}
O mesmo que \textunderscore crosta\textunderscore .
\section{Crustáceo}
\begin{itemize}
\item {Grp. gram.:adj.}
\end{itemize}
\begin{itemize}
\item {Grp. gram.:M. pl.}
\end{itemize}
Coberto de crusta.
Relativo ou pertencente aos crustáceos.
Animaes articulados, cujo corpo é revestido de uma crusta mais ou menos calcária, como a tartaruga, a lagosta, etc.
(Cp. lat. \textunderscore crustacea\textunderscore )
\section{Crustaceologia}
\begin{itemize}
\item {Grp. gram.:f.}
\end{itemize}
\begin{itemize}
\item {Proveniência:(De \textunderscore crustáceo\textunderscore  + gr. \textunderscore logos\textunderscore )}
\end{itemize}
Tratado dos crustáceos.
\section{Crustaceológico}
\begin{itemize}
\item {Grp. gram.:adj.}
\end{itemize}
Relativo á crustaceologia.
\section{Crustaceólogo}
\begin{itemize}
\item {Grp. gram.:m.}
\end{itemize}
Aquelle que se occupa de crustaceologia.
\section{Crustacite}
\begin{itemize}
\item {Grp. gram.:m.}
\end{itemize}
\begin{itemize}
\item {Proveniência:(De \textunderscore crustáceo\textunderscore )}
\end{itemize}
Crustáceo fóssil.
\section{Crustado}
\begin{itemize}
\item {Grp. gram.:adj.}
\end{itemize}
\begin{itemize}
\item {Utilização:Ant.}
\end{itemize}
O mesmo que \textunderscore crustáceo\textunderscore .
\section{Crustoderme}
\begin{itemize}
\item {Grp. gram.:adj.}
\end{itemize}
\begin{itemize}
\item {Proveniência:(Do lat. \textunderscore crusta\textunderscore  + gr. \textunderscore derma\textunderscore )}
\end{itemize}
Que tem a pelle dura.
\section{Crústula}
\begin{itemize}
\item {Grp. gram.:f.}
\end{itemize}
O mesmo que \textunderscore crústulo\textunderscore .
\section{Crustuliforme}
\begin{itemize}
\item {Grp. gram.:adj.}
\end{itemize}
\begin{itemize}
\item {Proveniência:(Do lat. \textunderscore crustula\textunderscore  + \textunderscore forma\textunderscore )}
\end{itemize}
Que tem fórma de coscorão.
\section{Crústulo}
\begin{itemize}
\item {Grp. gram.:m.}
\end{itemize}
\begin{itemize}
\item {Proveniência:(Lat. \textunderscore crustulum\textunderscore )}
\end{itemize}
Bolo doce e rijo.
\section{Cruta}
\begin{itemize}
\item {Grp. gram.:f.}
\end{itemize}
O mesmo que \textunderscore coruta\textunderscore .
\section{Cruviana}
\begin{itemize}
\item {Grp. gram.:f.}
\end{itemize}
\begin{itemize}
\item {Utilização:Bras. do N}
\end{itemize}
Frio intenso.
\section{Cruz}
\begin{itemize}
\item {Grp. gram.:f.}
\end{itemize}
\begin{itemize}
\item {Utilização:Ext.}
\end{itemize}
\begin{itemize}
\item {Utilização:Fig.}
\end{itemize}
\begin{itemize}
\item {Utilização:Bras}
\end{itemize}
\begin{itemize}
\item {Utilização:Ant.}
\end{itemize}
\begin{itemize}
\item {Grp. gram.:Pl.}
\end{itemize}
\begin{itemize}
\item {Proveniência:(Do lat. \textunderscore crux\textunderscore )}
\end{itemize}
Instrumento de supplício, a que se prendiam, ou em que se pregavam, os criminosos.
Madeiro, em que pregaram Christo.
Disposição de dois objectos, atravessádos um sôbre o outro.
Disposição análoga á dêsses objectos.
Morte de Christo.
Christianismo: \textunderscore as victórias da cruz\textunderscore .
Supplício, tortura, afflicção: \textunderscore êste filho é a minha cruz\textunderscore .
Objecto representativo da cruz de Christo.
Sinal, feito de dois traços cruzados.
Gestos cruzados sôbre o rosto ou peito, quando alguém se persigna ou se benze, e quando benze qualquer coisa ou pessôa.
Insígnia de várias ordens.
Objecto, que tem analogia com uma cruz.
Parte superior do cachaço do toiro, onde se cruza a espinha dorsal com o prolongamento das linhas correspondentes ás mãos.
Planta medicinal, o mesmo que \textunderscore cruzeiro\textunderscore .
\textunderscore Cruz grega\textunderscore , aquella, cujos quatro ramos têm igual comprimento.
\textunderscore Cruz latina\textunderscore , a cruz que tem o ramo inferior mais comprido que os outros.
\textunderscore Cruz de santo André\textunderscore , cruz que tem fórma de X.
\textunderscore Um de cruz\textunderscore , um cruzado:«\textunderscore quem me salvou sete mil de cruz\textunderscore ». \textunderscore Peregrinação\textunderscore , LXIX.
* \textunderscore Cruz do Sul\textunderscore , constellação, o mesmo que \textunderscore cruzeiro\textunderscore . Cf. Latino, \textunderscore Humboldt\textunderscore , 159.
\textunderscore Levar a cruz ao calvário\textunderscore , levar a cabo empresa árdua.
\textunderscore Entre a cruz e a caldeirinha\textunderscore , em situação crítica.
\textunderscore Cruz de Avís\textunderscore , cruz latina, com todos os ramos terminados em flôres de lis.
\textunderscore Cruz aberta\textunderscore , duas cruzes sobrepostas, de côr differente.
Quadris.
Reverso, com cruz, de algumas moédas, em opposição a cunhos.
Constellação do cruzeiro.
\textunderscore Fazer cruzes na bôca\textunderscore , não comer.
Não perceber nada do que se ouve ou se lê.
\section{Cruza-bico}
\begin{itemize}
\item {Grp. gram.:m.}
\end{itemize}
Pássaro conirostro, (\textunderscore loxia curvirostra\textunderscore , Lin.).
\section{Cruzada}
\begin{itemize}
\item {Grp. gram.:f.}
\end{itemize}
\begin{itemize}
\item {Proveniência:(De \textunderscore cruzar\textunderscore )}
\end{itemize}
Cada uma das expedições militares, que na Idade-Média os países christãos da Europa fizeram á Palestina, para livrar dos infiéis o túmulo de Christo.
Cada uma das guerras religiosas, promovidas na Europa pelos Cathólicos contra os herejes.
Luta, empresa, para a propagação de uma ideia ou defesa de um interesse.
Acto de cruzar os fios da seda, antes de se tecerem.
O primeiro estômago dos ruminantes.
\section{Cruzado}
\begin{itemize}
\item {Grp. gram.:adj.}
\end{itemize}
\begin{itemize}
\item {Grp. gram.:M.}
\end{itemize}
\begin{itemize}
\item {Proveniência:(De \textunderscore cruzar\textunderscore )}
\end{itemize}
Disposto em cruz.
Que com outro fórma quatro ângulos.
Diz-se dos tiros convergentes.
Cada um dos expedicionários das cruzadas.
Antiga moéda de oiro portuguesa.
Antiga moéda de prata portuguesa.
Quantia de 400 réis.
\section{Criestesia}
\begin{itemize}
\item {Grp. gram.:f.}
\end{itemize}
\begin{itemize}
\item {Proveniência:(Do gr. \textunderscore kruos\textunderscore  + \textunderscore esthesis\textunderscore )}
\end{itemize}
Impressionabilidade mórbida ao frio.
\section{Crimófilo}
\begin{itemize}
\item {Grp. gram.:adj.}
\end{itemize}
\begin{itemize}
\item {Proveniência:(Do gr. \textunderscore krumos\textunderscore  + \textunderscore philos\textunderscore )}
\end{itemize}
Que se dá bem nos países frios.
\section{Criólito}
\begin{itemize}
\item {Grp. gram.:m.}
\end{itemize}
\begin{itemize}
\item {Proveniência:(Do gr. \textunderscore kruos\textunderscore  + \textunderscore lithos\textunderscore )}
\end{itemize}
Variedade de mineral branco e translúcido.
\section{Crióforo}
\begin{itemize}
\item {Grp. gram.:m.}
\end{itemize}
\begin{itemize}
\item {Proveniência:(Do gr. \textunderscore kruos\textunderscore  + \textunderscore phoros\textunderscore )}
\end{itemize}
Instrumento, para congelar a água, por efeito da evaporação.
\section{Criómetro}
\begin{itemize}
\item {Grp. gram.:m.}
\end{itemize}
\begin{itemize}
\item {Proveniência:(Do gr. \textunderscore kruos\textunderscore  + \textunderscore metron\textunderscore )}
\end{itemize}
Instrumento, para conhecer a intensidade do frio.
\section{Crioscopia}
\begin{itemize}
\item {Grp. gram.:f.}
\end{itemize}
\begin{itemize}
\item {Proveniência:(Do gr. \textunderscore kruos\textunderscore  + \textunderscore skopein\textunderscore )}
\end{itemize}
Método de exame dos líquidos que contêm substâncias dissolvidas, fundado na determinação do respectivo grau de congelação.
\section{Crioscópico}
\begin{itemize}
\item {Grp. gram.:adj.}
\end{itemize}
Relativo á crioscopia.
\section{Cripta}
\begin{itemize}
\item {Grp. gram.:f.}
\end{itemize}
\begin{itemize}
\item {Utilização:Anat.}
\end{itemize}
\begin{itemize}
\item {Proveniência:(Do gr. \textunderscore kruptos\textunderscore )}
\end{itemize}
Caverna, galeria subterrânea.
Catacumbas.
Gruta.
Pequena glândula das membranas mucosas.
\section{Criptandro}
\begin{itemize}
\item {Grp. gram.:adj.}
\end{itemize}
\begin{itemize}
\item {Proveniência:(Do gr. \textunderscore kruptos\textunderscore  + \textunderscore aner\textunderscore , \textunderscore andros\textunderscore )}
\end{itemize}
Diz-se dos vegetaes, que não têm órgãos masculinos aparentes.
\section{Criptanto}
\begin{itemize}
\item {Grp. gram.:m.}
\end{itemize}
\begin{itemize}
\item {Proveniência:(Do gr. \textunderscore kruptos\textunderscore  + \textunderscore anthos\textunderscore )}
\end{itemize}
Gênero de plantas bromeliáceas.
\section{Criptia}
\begin{itemize}
\item {Grp. gram.:f.}
\end{itemize}
\begin{itemize}
\item {Proveniência:(Gr. \textunderscore krupteia\textunderscore )}
\end{itemize}
Emboscada, para exercícios guerreiros, praticada pelos mancebos espartanos.
\section{Críptico}
\begin{itemize}
\item {Grp. gram.:adj.}
\end{itemize}
Relativo a cripta.
\section{Criptina}
\begin{itemize}
\item {Grp. gram.:f.}
\end{itemize}
\begin{itemize}
\item {Proveniência:(Do gr. \textunderscore kruptos\textunderscore )}
\end{itemize}
Gênero de plantas, cujas flôres estão ocultas.
\section{Cripto...}
\begin{itemize}
\item {Grp. gram.:pref.}
\end{itemize}
\begin{itemize}
\item {Proveniência:(Do gr. \textunderscore kruptos\textunderscore )}
\end{itemize}
(que significa oculto)
\section{Cripto}
\begin{itemize}
\item {Grp. gram.:m.}
\end{itemize}
\begin{itemize}
\item {Proveniência:(Do gr. \textunderscore kruptos\textunderscore )}
\end{itemize}
Molusco gasterópode.
Insecto himenóptero.
\section{Criptobrânquio}
\begin{itemize}
\item {Grp. gram.:adj.}
\end{itemize}
\begin{itemize}
\item {Utilização:Zool.}
\end{itemize}
\begin{itemize}
\item {Proveniência:(Do gr. \textunderscore kruptos\textunderscore  + \textunderscore brankhia\textunderscore )}
\end{itemize}
Que respira por brânquias ocultas.
\section{Criptocarpo}
\begin{itemize}
\item {Grp. gram.:adj.}
\end{itemize}
\begin{itemize}
\item {Proveniência:(Do gr. \textunderscore kruptos\textunderscore  + \textunderscore karpos\textunderscore )}
\end{itemize}
Diz-se dos vegetaes, cujos frutos estão ocultos.
\section{Criptocéfalo}
\begin{itemize}
\item {Grp. gram.:m.  e  adj.}
\end{itemize}
\begin{itemize}
\item {Utilização:Zool.}
\end{itemize}
\begin{itemize}
\item {Proveniência:(Do gr. \textunderscore kruptos\textunderscore  + \textunderscore kephale\textunderscore )}
\end{itemize}
O que tem a cabeça oculta.
\section{Criptócero}
\begin{itemize}
\item {Grp. gram.:adj.}
\end{itemize}
\begin{itemize}
\item {Utilização:Zool.}
\end{itemize}
\begin{itemize}
\item {Proveniência:(Do gr. \textunderscore kruptos\textunderscore  + \textunderscore keras\textunderscore )}
\end{itemize}
Que tem ocultas as antenas.
\section{Criptocristalino}
\begin{itemize}
\item {Grp. gram.:adj.}
\end{itemize}
\begin{itemize}
\item {Proveniência:(Do gr. \textunderscore kruptos\textunderscore  + \textunderscore krustallos\textunderscore )}
\end{itemize}
Diz-se do mineral, formado de tão pequenos indivíduos, que, para o estudar, é preciso talhar lâminas muito finas e observá-las ao microscópio.
\section{Criptoftalmia}
\begin{itemize}
\item {Grp. gram.:f.}
\end{itemize}
O mesmo que \textunderscore criptoftalmo\textunderscore .
\section{Criptoftalmo}
\begin{itemize}
\item {Grp. gram.:m.}
\end{itemize}
\begin{itemize}
\item {Proveniência:(Do gr. \textunderscore kruptos\textunderscore  + \textunderscore phthalmos\textunderscore )}
\end{itemize}
Estado patológico de quem não póde abrir naturalmente os olhos.
\section{Criptogamia}
\begin{itemize}
\item {Grp. gram.:f.}
\end{itemize}
Classe de plantas, cujos órgãos de reprodução estão ocultos.
(Cp. \textunderscore cryptógamo\textunderscore )
\section{Criptogâmico}
\begin{itemize}
\item {Grp. gram.:adj.}
\end{itemize}
\begin{itemize}
\item {Grp. gram.:F. pl.}
\end{itemize}
Relativo á criptogamia.
Plantas, que têm ocultos os órgãos de reprodução.
\section{Criptogamista}
\begin{itemize}
\item {Grp. gram.:m.}
\end{itemize}
Aquele que se dedica ao estudo das plantas criptògâmicas.
\section{Criptógamo}
\begin{itemize}
\item {Grp. gram.:adj.}
\end{itemize}
\begin{itemize}
\item {Proveniência:(Do gr. \textunderscore kruptos\textunderscore  + \textunderscore gamos\textunderscore )}
\end{itemize}
O mesmo que \textunderscore criptogâmico\textunderscore .
\section{Criptogamologia}
\begin{itemize}
\item {Grp. gram.:f.}
\end{itemize}
\begin{itemize}
\item {Proveniência:(Do gr. \textunderscore kruptos\textunderscore  + \textunderscore gamos\textunderscore  + \textunderscore logos\textunderscore )}
\end{itemize}
História das plantas criptogâmicas.
\section{Criptografia}
\begin{itemize}
\item {Grp. gram.:f.}
\end{itemize}
\begin{itemize}
\item {Proveniência:(Do gr. \textunderscore kruptos\textunderscore  + \textunderscore graphein\textunderscore )}
\end{itemize}
Escrita secreta, em cifra.
O mesmo que \textunderscore ocultismo\textunderscore .
\section{Criptográfico}
\begin{itemize}
\item {Grp. gram.:adj.}
\end{itemize}
Relativo á criptografia.
\section{Criptólito}
\begin{itemize}
\item {Grp. gram.:m.}
\end{itemize}
\begin{itemize}
\item {Proveniência:(Do gr. \textunderscore kruptos\textunderscore  + \textunderscore lithos\textunderscore )}
\end{itemize}
Gênero de crustáceos.
\section{Criptologia}
\begin{itemize}
\item {Grp. gram.:f.}
\end{itemize}
\begin{itemize}
\item {Proveniência:(Do gr. \textunderscore kruptos\textunderscore  + \textunderscore logos\textunderscore )}
\end{itemize}
Ciência oculta, ocultismo, criptografia.
\section{Criptológico}
\begin{itemize}
\item {Grp. gram.:adj.}
\end{itemize}
Relativo á criptologia.
\section{Criptoméria}
\begin{itemize}
\item {Grp. gram.:f.}
\end{itemize}
Árvore monumental, (\textunderscore cryptomeria araucarioide\textunderscore ).
\section{Criptomnesia}
\begin{itemize}
\item {Grp. gram.:f.}
\end{itemize}
Memória inconsciente, faculdade, em virtude da qual, se conservam o espirito, despercebidas, noções que depois se podem revelar.
\section{Crípton}
\begin{itemize}
\item {Grp. gram.:m.}
\end{itemize}
\begin{itemize}
\item {Proveniência:(Do gr. \textunderscore kruptos\textunderscore )}
\end{itemize}
Um dos elementos da atmosfera, recentemente descoberto.
\section{Criptónimo}
\begin{itemize}
\item {Grp. gram.:adj.}
\end{itemize}
\begin{itemize}
\item {Grp. gram.:M.}
\end{itemize}
\begin{itemize}
\item {Proveniência:(Do gr. \textunderscore kruptos\textunderscore  + \textunderscore onuma\textunderscore )}
\end{itemize}
Que ocultou o nome ou o substituiu por iniciaes ou por outro sinal.
Autor que ocultou o nome.
\section{Criptópode}
\begin{itemize}
\item {Grp. gram.:m.  e  adj.}
\end{itemize}
\begin{itemize}
\item {Utilização:Zool.}
\end{itemize}
\begin{itemize}
\item {Proveniência:(Do gr. \textunderscore kruptos\textunderscore  + \textunderscore pous\textunderscore )}
\end{itemize}
Diz-se dos animaes, que não têm pés aparentes.
\section{Criptóporo}
\begin{itemize}
\item {Grp. gram.:adj.}
\end{itemize}
\begin{itemize}
\item {Proveniência:(De gr. \textunderscore kruptos\textunderscore  + \textunderscore poros\textunderscore )}
\end{itemize}
Que tem os poros pouco aparentes ou invisíveis.
\section{Criptopórtico}
\begin{itemize}
\item {Grp. gram.:m.}
\end{itemize}
\begin{itemize}
\item {Proveniência:(Do gr. \textunderscore kruptos\textunderscore  + lat. \textunderscore porticus\textunderscore )}
\end{itemize}
Pórtico subterrâneo.
Decoração da entrada de uma gruta.
\section{Criptorquidia}
\begin{itemize}
\item {Grp. gram.:f.}
\end{itemize}
\begin{itemize}
\item {Utilização:Anat.}
\end{itemize}
Ausência dos testículos nas bolsas, em virtude da sua retenção no abdome ou no canal inguinal.
\section{Criptoscopia}
\begin{itemize}
\item {Grp. gram.:f.}
\end{itemize}
O mesmo que \textunderscore radioscopia\textunderscore . Cf. Verg. Machado, \textunderscore Raios X\textunderscore , 9.
(Cp. \textunderscore criptoscópio\textunderscore )
\section{Criptoscópio}
\begin{itemize}
\item {Grp. gram.:m.}
\end{itemize}
\begin{itemize}
\item {Proveniência:(Do gr. \textunderscore kruptos\textunderscore  + \textunderscore skopein\textunderscore )}
\end{itemize}
Instrumento, que permite vêr os objectos contidos numa caixa fechada de papelão ou de alumínio.
\section{Criptostêmono}
\begin{itemize}
\item {Grp. gram.:adj.}
\end{itemize}
\begin{itemize}
\item {Utilização:Bot.}
\end{itemize}
\begin{itemize}
\item {Proveniência:(Do gr. \textunderscore kruptos\textunderscore  + \textunderscore stemen\textunderscore )}
\end{itemize}
Que não tem os estames visiveis.
\section{Criptóstomo}
\begin{itemize}
\item {Grp. gram.:m.}
\end{itemize}
\begin{itemize}
\item {Proveniência:(Do gr. \textunderscore kruptos\textunderscore  + \textunderscore stoma\textunderscore )}
\end{itemize}
Insecto coleóptero de Caiena.
\section{Cristal}
\begin{itemize}
\item {Grp. gram.:m.}
\end{itemize}
\begin{itemize}
\item {Utilização:Fig.}
\end{itemize}
\begin{itemize}
\item {Proveniência:(Lat. \textunderscore crystallum\textunderscore )}
\end{itemize}
Quartzo hialino e incolor, a mais dura de todas as variedades de quartzo.
Vidro transparente e branco, que contém óxido de chumbo.
Sólido poliédrico, terminado por faces planas, unidas, regulares, colocadas em simetria recíproca.
\textunderscore Cristal de rocha\textunderscore , espécie de quartzo muito silicioso, quartzo hialino.
Água límpida: \textunderscore mirar-se no cristal das fontes\textunderscore .
Transparência.
\section{Cristalífero}
\begin{itemize}
\item {Grp. gram.:adj.}
\end{itemize}
\begin{itemize}
\item {Proveniência:(Do lat. \textunderscore crystallum\textunderscore  + \textunderscore ferre\textunderscore )}
\end{itemize}
Que contém cristaes.
\section{Cristalina}
\begin{itemize}
\item {Grp. gram.:f.}
\end{itemize}
\begin{itemize}
\item {Proveniência:(De \textunderscore cristalino\textunderscore )}
\end{itemize}
Solução de algodão-pólvora em álcool metílico.
\section{Cristalinidade}
\begin{itemize}
\item {Grp. gram.:f.}
\end{itemize}
Qualidade de cristalino.
\section{Cristalino}
\begin{itemize}
\item {Grp. gram.:adj.}
\end{itemize}
\begin{itemize}
\item {Utilização:Fig.}
\end{itemize}
\begin{itemize}
\item {Grp. gram.:M.}
\end{itemize}
\begin{itemize}
\item {Utilização:Anat.}
\end{itemize}
\begin{itemize}
\item {Proveniência:(Lat. \textunderscore crystallinus\textunderscore , de \textunderscore crystallum\textunderscore )}
\end{itemize}
Relativo a cristal.
Límpido como cristal: \textunderscore água cristalina\textunderscore .
Puro, sem mancha. Cf. \textunderscore Lusíadas\textunderscore , V, 47.
Corpo lenticular e transparente, na parte anterior do humor vítreo do ôlho.
\section{Cristálito}
\begin{itemize}
\item {Grp. gram.:m.}
\end{itemize}
Designação, impropriamente dada por alguns mineralogistas á \textunderscore morfostequia\textunderscore .
\section{Cristalização}
\begin{itemize}
\item {Grp. gram.:f.}
\end{itemize}
Acto ou efeito de cristalizar.
\section{Cristalizador}
\begin{itemize}
\item {Grp. gram.:m.}
\end{itemize}
\begin{itemize}
\item {Proveniência:(De \textunderscore cristalizar\textunderscore )}
\end{itemize}
Nome de cada um dos compartimentos, em que, nas marinhas, se cristaliza o sal.
\section{Cruzado-novo}
\begin{itemize}
\item {Grp. gram.:m.}
\end{itemize}
Moéda brasileira de oiro e de prata, representativa de 480 réis. Moéda portuguesa de prata, o mesmo que \textunderscore pinto\textunderscore .
\section{Cruzador}
\begin{itemize}
\item {Grp. gram.:m.}
\end{itemize}
Aquelle que cruza.
Espécie de navio de guerra.
\section{Cruzamento}
\begin{itemize}
\item {Grp. gram.:m.}
\end{itemize}
Acto ou effeito de cruzar.
\section{Cruzante}
\begin{itemize}
\item {Grp. gram.:adj.}
\end{itemize}
\begin{itemize}
\item {Proveniência:(De \textunderscore cruzar\textunderscore )}
\end{itemize}
Diz-se da raça de animaes, que, no seu cruzamento com outra, melhora esta.
\section{Cruzar}
\begin{itemize}
\item {Grp. gram.:v. t.}
\end{itemize}
\begin{itemize}
\item {Grp. gram.:V. i.}
\end{itemize}
\begin{itemize}
\item {Utilização:Ant.}
\end{itemize}
\begin{itemize}
\item {Grp. gram.:V. p.}
\end{itemize}
\begin{itemize}
\item {Proveniência:(Do lat. \textunderscore cruciare\textunderscore )}
\end{itemize}
Dispor em fórma de cruz.
Dar fórma de cruz a.
Atravessar.
Percorrer em vários sentidos: \textunderscore cruzar os mares\textunderscore .
Juntar, acasalar, (falando-se de raças diversas, ou de indivíduos de diversa procedência).
Formar cruz.
Ir através.
Entrar numa cruzada.
Dispor-se em fórma de cruz.
Encontrar-se: \textunderscore cruzaram-se os dois viandantes\textunderscore .
\section{Cruzável}
\begin{itemize}
\item {Grp. gram.:adj.}
\end{itemize}
Que se póde cruzar. Cf. Filinto, \textunderscore D. Man.\textunderscore , 278.
\section{Cruz-de-Jerusalém}
\begin{itemize}
\item {Grp. gram.:f.}
\end{itemize}
Planta dianthácea, (\textunderscore lichnis chalcedonica\textunderscore , Lin.).
\section{Cruz-de-malta}
\begin{itemize}
\item {Grp. gram.:f.}
\end{itemize}
Planta do sul do Brasil, (\textunderscore jaslievia\textunderscore ), abundante em terrenos húmidos e pantanosos.
Planta, o mesmo que \textunderscore cruz-de-jerusalém\textunderscore .
Cruz do quatro ramos iguaes, que se alargam nas extremidades.
\section{Cruz-diabo}
\begin{itemize}
\item {Grp. gram.:m.}
\end{itemize}
\begin{itemize}
\item {Utilização:Des.}
\end{itemize}
Mascarado, que percorria as ruas, levando num pau erguido umas bochechas artificiaes:«\textunderscore fazendo aos rapazes côca, em trajes de cruz-diabo\textunderscore ».
Garção.
\section{Crúzea}
\begin{itemize}
\item {Grp. gram.:f.}
\end{itemize}
\begin{itemize}
\item {Proveniência:(De \textunderscore Vera-Cruz\textunderscore , n. p.)}
\end{itemize}
Gênero de plantas rubiáceas.
\section{Cruzeirinha}
\begin{itemize}
\item {Grp. gram.:f.}
\end{itemize}
(V.cainca)
\section{Cruzeiro}
\begin{itemize}
\item {Grp. gram.:adj.}
\end{itemize}
\begin{itemize}
\item {Grp. gram.:M.}
\end{itemize}
\begin{itemize}
\item {Utilização:Bras}
\end{itemize}
\begin{itemize}
\item {Utilização:Bras}
\end{itemize}
\begin{itemize}
\item {Proveniência:(Do lat. \textunderscore cruciarius\textunderscore )}
\end{itemize}
Que tem cruz.
Marcado com cruz.
Grande cruz, erguida nos adros de algumas igrejas, em cemitérios, praças, etc.
Parte da igreja, comprehendida entre a capella-mór e a nave central.
Caixilho de tear, em que se cruzam os fios.
Parte do mar, cruzada por navios.
Navio, que anda cruzando.
Grupo austral de 4 estrêllas, que se imaginam ligadas por linhas, em fórma de cruz.
Ordem militar do Brasil.
Planta medicinal, também conhecida por \textunderscore cruz\textunderscore .
O mesmo que \textunderscore urutu\textunderscore  ou \textunderscore cotiara\textunderscore .
\section{Cruzeta}
\begin{itemize}
\item {fónica:zê}
\end{itemize}
\begin{itemize}
\item {Grp. gram.:f.}
\end{itemize}
\begin{itemize}
\item {Utilização:Náut.}
\end{itemize}
\begin{itemize}
\item {Utilização:Prov.}
\end{itemize}
\begin{itemize}
\item {Utilização:alent.}
\end{itemize}
\begin{itemize}
\item {Utilização:Constr.}
\end{itemize}
\begin{itemize}
\item {Proveniência:(De \textunderscore cruz\textunderscore )}
\end{itemize}
Pequena cruz.
Cruz, que serve de cabide.
Armação provisória de vêrgas e antennas, para supprir mastros.
Estrêlla de bronze, com quatro pontas, três das quaes se embebem na parte inferior da árvore do rodízio das azenhas, ficando a quarta de fóra e girando sôbre uma peça fixa de ferro.
Peça de madeira, em fórma de T, para nivelamentos.
\section{Cruzetado}
\begin{itemize}
\item {Grp. gram.:adj.}
\end{itemize}
Que tem fórma de cruzeta.
\section{Cruzianas}
\begin{itemize}
\item {Grp. gram.:f. pl.}
\end{itemize}
Plantas fósseis das idades primitivas da terra.
\section{Cruzilhada}
\begin{itemize}
\item {Grp. gram.:f.}
\end{itemize}
O mesmo que \textunderscore encruzilhada\textunderscore .
\section{Crúzio}
\begin{itemize}
\item {Grp. gram.:m.}
\end{itemize}
\begin{itemize}
\item {Grp. gram.:Adj.}
\end{itemize}
\begin{itemize}
\item {Proveniência:(De \textunderscore cruz\textunderscore )}
\end{itemize}
Membro da congregação religiosa de Santa-Cruz de Coimbra.
Relativo a essa congregação.
\section{Cryesthesia}
\begin{itemize}
\item {Grp. gram.:f.}
\end{itemize}
\begin{itemize}
\item {Proveniência:(Do gr. \textunderscore kruos\textunderscore  + \textunderscore esthesis\textunderscore )}
\end{itemize}
Impressionabilidade mórbida ao frio.
\section{Crymóphilo}
\begin{itemize}
\item {Grp. gram.:adj.}
\end{itemize}
\begin{itemize}
\item {Proveniência:(Do gr. \textunderscore krumos\textunderscore  + \textunderscore philos\textunderscore )}
\end{itemize}
Que se dá bem nos países frios.
\section{Cryogenina}
\begin{itemize}
\item {Grp. gram.:f.}
\end{itemize}
\begin{itemize}
\item {Proveniência:(Do gr. \textunderscore kruos\textunderscore  + \textunderscore genes\textunderscore )}
\end{itemize}
Medicamento antitérmico.
\section{Cryólitho}
\begin{itemize}
\item {Grp. gram.:m.}
\end{itemize}
\begin{itemize}
\item {Proveniência:(Do gr. \textunderscore kruos\textunderscore  + \textunderscore lithos\textunderscore )}
\end{itemize}
Variedade de mineral branco e translúcido.
\section{Cryómetro}
\begin{itemize}
\item {Grp. gram.:m.}
\end{itemize}
\begin{itemize}
\item {Proveniência:(Do gr. \textunderscore kruos\textunderscore  + \textunderscore metron\textunderscore )}
\end{itemize}
Instrumento, para conhecer a intensidade do frio.
\section{Cryóphoro}
\begin{itemize}
\item {Grp. gram.:m.}
\end{itemize}
\begin{itemize}
\item {Proveniência:(Do gr. \textunderscore kruos\textunderscore  + \textunderscore phoros\textunderscore )}
\end{itemize}
Instrumento, para congelar a água, por effeito da evaporação.
\section{Cryoscopia}
\begin{itemize}
\item {Grp. gram.:f.}
\end{itemize}
\begin{itemize}
\item {Proveniência:(Do gr. \textunderscore kruos\textunderscore  + \textunderscore skopein\textunderscore )}
\end{itemize}
Méthodo de exame dos líquidos que contêm substâncias dissolvidas, fundado na determinação do respectivo grau de congelação.
\section{Cryoscópico}
\begin{itemize}
\item {Grp. gram.:adj.}
\end{itemize}
Relativo á cryoscopia.
\section{Crypta}
\begin{itemize}
\item {Grp. gram.:f.}
\end{itemize}
\begin{itemize}
\item {Utilização:Anat.}
\end{itemize}
\begin{itemize}
\item {Proveniência:(Do gr. \textunderscore kruptos\textunderscore )}
\end{itemize}
Caverna, galeria subterrânea.
Catacumbas.
Gruta.
Pequena glândula das membranas mucosas.
\section{Cryptandro}
\begin{itemize}
\item {Grp. gram.:adj.}
\end{itemize}
\begin{itemize}
\item {Proveniência:(Do gr. \textunderscore kruptos\textunderscore  + \textunderscore aner\textunderscore , \textunderscore andros\textunderscore )}
\end{itemize}
Diz-se dos vegetaes, que não têm órgãos masculinos apparentes.
\section{Cryptantho}
\begin{itemize}
\item {Grp. gram.:m.}
\end{itemize}
\begin{itemize}
\item {Proveniência:(Do gr. \textunderscore kruptos\textunderscore  + \textunderscore anthos\textunderscore )}
\end{itemize}
Gênero de plantas bromeliáceas.
\section{Cryptia}
\begin{itemize}
\item {Grp. gram.:f.}
\end{itemize}
\begin{itemize}
\item {Proveniência:(Gr. \textunderscore krupteia\textunderscore )}
\end{itemize}
Emboscada, para exercícios guerreiros, praticada pelos mancebos espartanos.
\section{Crýptico}
\begin{itemize}
\item {Grp. gram.:adj.}
\end{itemize}
Relativo a crypta.
\section{Cryptina}
\begin{itemize}
\item {Grp. gram.:f.}
\end{itemize}
\begin{itemize}
\item {Proveniência:(Do gr. \textunderscore kruptos\textunderscore )}
\end{itemize}
Gênero de plantas, cujas flôres estão occultas.
\section{Crypto...}
\begin{itemize}
\item {Grp. gram.:pref.}
\end{itemize}
\begin{itemize}
\item {Proveniência:(Do gr. \textunderscore kruptos\textunderscore )}
\end{itemize}
(que significa occulto)
\section{Crypto}
\begin{itemize}
\item {Grp. gram.:m.}
\end{itemize}
\begin{itemize}
\item {Proveniência:(Do gr. \textunderscore kruptos\textunderscore )}
\end{itemize}
Mollusco gasterópode.
Insecto himenóptero.
\section{Cryptobranchio}
\begin{itemize}
\item {fónica:qui}
\end{itemize}
\begin{itemize}
\item {Grp. gram.:adj.}
\end{itemize}
\begin{itemize}
\item {Utilização:Zool.}
\end{itemize}
\begin{itemize}
\item {Proveniência:(Do gr. \textunderscore kruptos\textunderscore  + \textunderscore brankhia\textunderscore )}
\end{itemize}
Que respira por brânchias occultas.
\section{Cryptocarpo}
\begin{itemize}
\item {Grp. gram.:adj.}
\end{itemize}
\begin{itemize}
\item {Proveniência:(Do gr. \textunderscore kruptos\textunderscore  + \textunderscore karpos\textunderscore )}
\end{itemize}
Diz-se dos vegetaes, cujos frutos estão occultos.
\section{Cryptocéphalo}
\begin{itemize}
\item {Grp. gram.:m.  e  adj.}
\end{itemize}
\begin{itemize}
\item {Utilização:Zool.}
\end{itemize}
\begin{itemize}
\item {Proveniência:(Do gr. \textunderscore kruptos\textunderscore  + \textunderscore kephale\textunderscore )}
\end{itemize}
O que tem a cabeça occulta.
\section{Cryptócero}
\begin{itemize}
\item {Grp. gram.:adj.}
\end{itemize}
\begin{itemize}
\item {Utilização:Zool.}
\end{itemize}
\begin{itemize}
\item {Proveniência:(Do gr. \textunderscore kruptos\textunderscore  + \textunderscore keras\textunderscore )}
\end{itemize}
Que tem occultas as antennas.
\section{Cryptocrystallino}
\begin{itemize}
\item {Grp. gram.:adj.}
\end{itemize}
\begin{itemize}
\item {Proveniência:(Do gr. \textunderscore kruptos\textunderscore  + \textunderscore krustallos\textunderscore )}
\end{itemize}
Diz-se do mineral, formado de tão pequenos indivíduos, que, para o estudar, é preciso talhar lâminas muito finas e observá-las ao microscópio.
\section{Cryptogamia}
\begin{itemize}
\item {Grp. gram.:f.}
\end{itemize}
Classe de plantas, cujos órgãos de reproducção estão occultos.
(Cp. \textunderscore cryptógamo\textunderscore )
\section{Cryptogâmico}
\begin{itemize}
\item {Grp. gram.:adj.}
\end{itemize}
\begin{itemize}
\item {Grp. gram.:F. pl.}
\end{itemize}
Relativo á cryptogamia.
Plantas, que têm occultos os órgãos de reproducção.
\section{Cryptogamista}
\begin{itemize}
\item {Grp. gram.:m.}
\end{itemize}
Aquelle que se dedica ao estudo das plantas cryptògâmicas.
\section{Cryptógamo}
\begin{itemize}
\item {Grp. gram.:adj.}
\end{itemize}
\begin{itemize}
\item {Proveniência:(Do gr. \textunderscore kruptos\textunderscore  + \textunderscore gamos\textunderscore )}
\end{itemize}
O mesmo que \textunderscore cryptogâmico\textunderscore .
\section{Cryptogamologia}
\begin{itemize}
\item {Grp. gram.:f.}
\end{itemize}
\begin{itemize}
\item {Proveniência:(Do gr. \textunderscore kruptos\textunderscore  + \textunderscore gamos\textunderscore  + \textunderscore logos\textunderscore )}
\end{itemize}
História das plantas cryptogâmicas.
\section{Cryptographia}
\begin{itemize}
\item {Grp. gram.:f.}
\end{itemize}
\begin{itemize}
\item {Proveniência:(Do gr. \textunderscore kruptos\textunderscore  + \textunderscore graphein\textunderscore )}
\end{itemize}
Escrita secreta, em cifra.
O mesmo que \textunderscore occultismo\textunderscore .
\section{Cryptográphico}
\begin{itemize}
\item {Grp. gram.:adj.}
\end{itemize}
Relativo á cryptographia.
\section{Cryptólitho}
\begin{itemize}
\item {Grp. gram.:m.}
\end{itemize}
\begin{itemize}
\item {Proveniência:(Do gr. \textunderscore kruptos\textunderscore  + \textunderscore lithos\textunderscore )}
\end{itemize}
Gênero de crustáceos.
\section{Cryptologia}
\begin{itemize}
\item {Grp. gram.:f.}
\end{itemize}
\begin{itemize}
\item {Proveniência:(Do gr. \textunderscore kruptos\textunderscore  + \textunderscore logos\textunderscore )}
\end{itemize}
Sciência occulta, occultismo, cryptographia.
\section{Cryptológico}
\begin{itemize}
\item {Grp. gram.:adj.}
\end{itemize}
Relativo á cryptologia.
\section{Cryptoméria}
\begin{itemize}
\item {Grp. gram.:f.}
\end{itemize}
Árvore monumental, (\textunderscore cryptomeria araucarioide\textunderscore ).
\section{Cryptomnesia}
\begin{itemize}
\item {Grp. gram.:f.}
\end{itemize}
Memória inconsciente, faculdade, em virtude da qual, se conservam o espirito, despercebidas, noções que depois se podem revelar.
\section{Crýpton}
\begin{itemize}
\item {Grp. gram.:m.}
\end{itemize}
\begin{itemize}
\item {Proveniência:(Do gr. \textunderscore kruptos\textunderscore )}
\end{itemize}
Um dos elementos da atmosphera, recentemente descoberto.
\section{Cryptónymo}
\begin{itemize}
\item {Grp. gram.:adj.}
\end{itemize}
\begin{itemize}
\item {Grp. gram.:M.}
\end{itemize}
\begin{itemize}
\item {Proveniência:(Do gr. \textunderscore kruptos\textunderscore  + \textunderscore onuma\textunderscore )}
\end{itemize}
Que occultou o nome ou o substituiu por iniciaes ou por outro sinál.
Autor que occultou o nome.
\section{Cryptophtalmia}
\begin{itemize}
\item {Grp. gram.:f.}
\end{itemize}
O mesmo que \textunderscore cryptophthalmo\textunderscore .
\section{Cryptophthalmo}
\begin{itemize}
\item {Grp. gram.:m.}
\end{itemize}
\begin{itemize}
\item {Proveniência:(Do gr. \textunderscore kruptos\textunderscore  + \textunderscore phthalmos\textunderscore )}
\end{itemize}
Estado pathológico de quem não póde abrir naturalmente os olhos.
\section{Cryptópode}
\begin{itemize}
\item {Grp. gram.:m.  e  adj.}
\end{itemize}
\begin{itemize}
\item {Utilização:Zool.}
\end{itemize}
\begin{itemize}
\item {Proveniência:(Do gr. \textunderscore kruptos\textunderscore  + \textunderscore pous\textunderscore )}
\end{itemize}
Diz-se dos animaes, que não têm pés apparentes.
\section{Cryptóporo}
\begin{itemize}
\item {Grp. gram.:adj.}
\end{itemize}
\begin{itemize}
\item {Proveniência:(De gr. \textunderscore kruptos\textunderscore  + \textunderscore poros\textunderscore )}
\end{itemize}
Que tem os poros pouco apparentes ou invisíveis.
\section{Cryptopórtico}
\begin{itemize}
\item {Grp. gram.:m.}
\end{itemize}
\begin{itemize}
\item {Proveniência:(Do gr. \textunderscore kruptos\textunderscore  + lat. \textunderscore porticus\textunderscore )}
\end{itemize}
Pórtico subterrâneo.
Decoração da entrada de uma gruta.
\section{Cryptorchidia}
\begin{itemize}
\item {fónica:qui}
\end{itemize}
\begin{itemize}
\item {Grp. gram.:f.}
\end{itemize}
\begin{itemize}
\item {Utilização:Anat.}
\end{itemize}
Ausência dos testículos nas bolsas, em virtude da sua retenção no abdome ou no canal inguinal.
\section{Cryptoscopia}
\begin{itemize}
\item {Grp. gram.:f.}
\end{itemize}
O mesmo que \textunderscore radioscopia\textunderscore . Cf. Verg. Machado, \textunderscore Raios X\textunderscore , 9.
(Cp. \textunderscore cryptoscópio\textunderscore )
\section{Cryptoscópio}
\begin{itemize}
\item {Grp. gram.:m.}
\end{itemize}
\begin{itemize}
\item {Proveniência:(Do gr. \textunderscore kruptos\textunderscore  + \textunderscore skopein\textunderscore )}
\end{itemize}
Instrumento, que permitte vêr os objectos contidos numa caixa fechada de papelão ou de alumínio.
\section{Cryptostêmono}
\begin{itemize}
\item {Grp. gram.:adj.}
\end{itemize}
\begin{itemize}
\item {Utilização:Bot.}
\end{itemize}
\begin{itemize}
\item {Proveniência:(Do gr. \textunderscore kruptos\textunderscore  + \textunderscore stemen\textunderscore )}
\end{itemize}
Que não tem os estames visiveis.
\section{Cryptóstomo}
\begin{itemize}
\item {Grp. gram.:m.}
\end{itemize}
\begin{itemize}
\item {Proveniência:(Do gr. \textunderscore kruptos\textunderscore  + \textunderscore stoma\textunderscore )}
\end{itemize}
Insecto coleóptero de Caiena.
\section{Crystal}
\begin{itemize}
\item {Grp. gram.:m.}
\end{itemize}
\begin{itemize}
\item {Utilização:Fig.}
\end{itemize}
\begin{itemize}
\item {Proveniência:(Lat. \textunderscore crystallum\textunderscore )}
\end{itemize}
Quartzo hyalino e incolor, a mais dura de todas as variedades de quartzo.
Vidro transparente e branco, que contém óxydo de chumbo.
Sólido polyédrico, terminado por faces planas, unidas, regulares, collocadas em symetria recíproca.
\textunderscore Crystal de rocha\textunderscore , espécie de quartzo muito silicioso, quartzo hialino.
Água límpida: \textunderscore mirar-se no crystal das fontes\textunderscore .
Transparência.
\section{Crystallífero}
\begin{itemize}
\item {Grp. gram.:adj.}
\end{itemize}
\begin{itemize}
\item {Proveniência:(Do lat. \textunderscore crystallum\textunderscore  + \textunderscore ferre\textunderscore )}
\end{itemize}
Que contém crystaes.
\section{Crystallina}
\begin{itemize}
\item {Grp. gram.:f.}
\end{itemize}
\begin{itemize}
\item {Proveniência:(De \textunderscore crystallino\textunderscore )}
\end{itemize}
Solução de algodão-pólvora em álcool methílico.
\section{Crystallinidade}
\begin{itemize}
\item {Grp. gram.:f.}
\end{itemize}
Qualidade de crystallino.
\section{Crystallino}
\begin{itemize}
\item {Grp. gram.:adj.}
\end{itemize}
\begin{itemize}
\item {Utilização:Fig.}
\end{itemize}
\begin{itemize}
\item {Grp. gram.:M.}
\end{itemize}
\begin{itemize}
\item {Utilização:Anat.}
\end{itemize}
\begin{itemize}
\item {Proveniência:(Lat. \textunderscore crystallinus\textunderscore , de \textunderscore crystallum\textunderscore )}
\end{itemize}
Relativo a crystal.
Límpido como crystal: \textunderscore água crystallina\textunderscore .
Puro, sem mancha. Cf. \textunderscore Lusíadas\textunderscore , V, 47.
Corpo lenticular e transparente, na parte anterior do humor vítreo do ôlho.
\section{Crystállitho}
\begin{itemize}
\item {Grp. gram.:m.}
\end{itemize}
Designação, impropriamente dada por alguns mineralogistas á \textunderscore morphostechia\textunderscore .
\section{Crystallização}
\begin{itemize}
\item {Grp. gram.:f.}
\end{itemize}
Acto ou effeito de crystallizar.
\section{Crystallizador}
\begin{itemize}
\item {Grp. gram.:m.}
\end{itemize}
\begin{itemize}
\item {Proveniência:(De \textunderscore crystallizar\textunderscore )}
\end{itemize}
Nome de cada um dos compartimentos, em que, nas marinhas, se crystalliza o sal.
\section{Crystallizar}
\begin{itemize}
\item {Grp. gram.:v. i.}
\end{itemize}
\begin{itemize}
\item {Grp. gram.:V. i.}
\end{itemize}
\begin{itemize}
\item {Utilização:Fig.}
\end{itemize}
\begin{itemize}
\item {Proveniência:(De \textunderscore crystal\textunderscore )}
\end{itemize}
Converter em crystal.
Dar a fórma de crystal a.
Tomar a fórma de crystal.
Permanecer em determinado estado: \textunderscore aquelle sujeito crystallizou na asneira\textunderscore .
\section{Crystallizável}
\begin{itemize}
\item {Grp. gram.:adj.}
\end{itemize}
Que se póde crystallizar.
\section{Crystalloeléctrico}
\begin{itemize}
\item {Grp. gram.:adj.}
\end{itemize}
Relativo á electricidade, que o calor desenvolve em certos crystaes, como no topázio.
\section{Crystallogenia}
\begin{itemize}
\item {Grp. gram.:f.}
\end{itemize}
\begin{itemize}
\item {Proveniência:(Do gr. \textunderscore krustallos\textunderscore  + \textunderscore genos\textunderscore )}
\end{itemize}
Sciência da formação dos crystaes.
\section{Crystallographia}
\begin{itemize}
\item {Grp. gram.:f.}
\end{itemize}
\begin{itemize}
\item {Proveniência:(Do gr. \textunderscore krustallos\textunderscore  + \textunderscore graphein\textunderscore )}
\end{itemize}
Sciência, que descreve os crystaes e expõe as leis da sua formação.
\section{Crystallographicamente}
\begin{itemize}
\item {Grp. gram.:adv.}
\end{itemize}
De modo crystallographico.
\section{Crystallográphico}
\begin{itemize}
\item {Grp. gram.:adj.}
\end{itemize}
Relativo á crystallographia.
\section{Crystallographo}
\begin{itemize}
\item {Grp. gram.:m.}
\end{itemize}
Aquelle que trata da crystallographia.
\section{Crystalloide}
\begin{itemize}
\item {Grp. gram.:adj.}
\end{itemize}
\begin{itemize}
\item {Grp. gram.:M.}
\end{itemize}
\begin{itemize}
\item {Utilização:Anat.}
\end{itemize}
\begin{itemize}
\item {Proveniência:(Do gr. \textunderscore krustallos\textunderscore  + \textunderscore eidos\textunderscore )}
\end{itemize}
Semelhante ao crystal.
Membrana, que envolve o crystallino do ôlho.
\section{Crystallologia}
\begin{itemize}
\item {Grp. gram.:f.}
\end{itemize}
\begin{itemize}
\item {Proveniência:(Do gr. \textunderscore krustallos\textunderscore  + \textunderscore logos\textunderscore )}
\end{itemize}
Tratado dos crystaes.
\section{Crystallológico}
\begin{itemize}
\item {Grp. gram.:adj.}
\end{itemize}
Relativo á crystallologia.
\section{Crystallomancia}
\begin{itemize}
\item {Grp. gram.:f.}
\end{itemize}
\begin{itemize}
\item {Proveniência:(Do gr. \textunderscore krustallos\textunderscore  + \textunderscore manteia\textunderscore )}
\end{itemize}
Supposta arte de adivinhar, por meio de um pedaço de gêlo ou de crystal. Cf. Castilho, \textunderscore Fastos\textunderscore , III, 323.
\section{Crystallometria}
\begin{itemize}
\item {Grp. gram.:f.}
\end{itemize}
\begin{itemize}
\item {Proveniência:(Do gr. \textunderscore krustallos\textunderscore  + \textunderscore metron\textunderscore )}
\end{itemize}
Medida da fórma geométrica dos crystaes.
\section{Crystallométrico}
\begin{itemize}
\item {Grp. gram.:adj.}
\end{itemize}
Relativo á crystallometria.
\section{Crystallonomia}
\begin{itemize}
\item {Grp. gram.:f.}
\end{itemize}
\begin{itemize}
\item {Proveniência:(Do gr. \textunderscore krustallos\textunderscore  + \textunderscore nomos\textunderscore )}
\end{itemize}
Conhecimento das leis da crystallização.
\section{Crystallonómico}
\begin{itemize}
\item {Grp. gram.:adj.}
\end{itemize}
Relativo á crystallonomia.
\section{Crystallophilliano}
\begin{itemize}
\item {Grp. gram.:adj.}
\end{itemize}
\begin{itemize}
\item {Utilização:Geol.}
\end{itemize}
Diz-se das rochas, que são xistos de textura crystallina.
\section{Crystallotechnia}
\begin{itemize}
\item {Grp. gram.:f.}
\end{itemize}
\begin{itemize}
\item {Proveniência:(Do gr. \textunderscore krustallos\textunderscore  + \textunderscore tekhne\textunderscore )}
\end{itemize}
Arte de obter crystaes completos, com as modificações de que são susceptiveis.
\section{Crystallotéchnico}
\begin{itemize}
\item {Grp. gram.:adj.}
\end{itemize}
Relativo á crystallotechnia.
\section{Crystallotomia}
\begin{itemize}
\item {Grp. gram.:f.}
\end{itemize}
\begin{itemize}
\item {Proveniência:(Do gr. \textunderscore krustallos\textunderscore  + \textunderscore tome\textunderscore )}
\end{itemize}
Arte de cortar os crystaes.
\section{Crystallotómico}
\begin{itemize}
\item {Grp. gram.:adj.}
\end{itemize}
Relativo á crystallotomia.
\section{Csar}
\textunderscore m.\textunderscore  (e der.)
(V. \textunderscore czar\textunderscore , etc.)
\section{Ctenodonte}
\begin{itemize}
\item {Grp. gram.:adj.}
\end{itemize}
\begin{itemize}
\item {Proveniência:(Do gr. \textunderscore kteis\textunderscore , \textunderscore kteinos\textunderscore  + \textunderscore odous\textunderscore )}
\end{itemize}
Que tem dentes em fórma de pente.
\section{Ctenóforo}
\begin{itemize}
\item {Grp. gram.:m.}
\end{itemize}
\begin{itemize}
\item {Proveniência:(Do gr. \textunderscore kteis\textunderscore , \textunderscore kteinos\textunderscore  + \textunderscore phoros\textunderscore )}
\end{itemize}
Molúsco, de concha em fórma de pente, e que é um dos três tipos dos celenterados, sendo os outros dois o pólipo e a medusa.
\section{Ctenóphoro}
\begin{itemize}
\item {Grp. gram.:m.}
\end{itemize}
\begin{itemize}
\item {Proveniência:(Do gr. \textunderscore kteis\textunderscore , \textunderscore kteinos\textunderscore  + \textunderscore phoros\textunderscore )}
\end{itemize}
Mollúsco, de concha em fórma de pente, e que é um dos três typos dos celenterados, sendo os outros dois o pólypo e a medusa.
\section{Ctoniano}
\begin{itemize}
\item {Grp. gram.:adj.}
\end{itemize}
\begin{itemize}
\item {Proveniência:(Do gr. \textunderscore khthon\textunderscore , terra)}
\end{itemize}
Diz-se, em Mitologia, dos deuses que residem nas cavidades da terra.
Relativo ao culto dêsses deuses.
\section{Ctónico}
\begin{itemize}
\item {Grp. gram.:adj.}
\end{itemize}
O mesmo que \textunderscore ctoniano\textunderscore .
\section{Cobiçar}
\begin{itemize}
\item {Grp. gram.:v. t.}
\end{itemize}
Têr cobiça de.
Desejar ardentemente.
Ambicionar.
\section{Cu}
\begin{itemize}
\item {Grp. gram.:m.}
\end{itemize}
\begin{itemize}
\item {Utilização:Pleb.}
\end{itemize}
\begin{itemize}
\item {Proveniência:(Do lat. \textunderscore culus\textunderscore )}
\end{itemize}
Ânus.
Nádegas.
Extremidade da agulha, opposta ao bico.
Extremidade da bigota, opposta á cabeça.
\section{Cuaco-blanco}
\begin{itemize}
\item {Grp. gram.:m.}
\end{itemize}
Grande árvore santhomense, de fôlhas medicinaes.
\section{Cuaco-mlaguita}
\begin{itemize}
\item {Grp. gram.:m.}
\end{itemize}
Arbusto santhomense, de propriedades anaphrodisíacas.
\section{Cuácuas}
\begin{itemize}
\item {Grp. gram.:m. pl.}
\end{itemize}
O mesmo que [[hotentotes|hotentote]].
\section{Cuada}
\begin{itemize}
\item {Grp. gram.:f.}
\end{itemize}
\begin{itemize}
\item {Utilização:Pleb.}
\end{itemize}
\begin{itemize}
\item {Proveniência:(De \textunderscore cu\textunderscore )}
\end{itemize}
Pancada, que se dá com as nádegas.
Parte das calças ou das ceroilas, correspondente ás nádegas.
\section{Cuaga}
\begin{itemize}
\item {Grp. gram.:m.}
\end{itemize}
Espécie de cavallo selvagem da África austral.
\section{Cuala-mugia}
\begin{itemize}
\item {Grp. gram.:f.}
\end{itemize}
Arbusto africano, de caule flexível.
\section{Cualvo}
\begin{itemize}
\item {Grp. gram.:m.}
\end{itemize}
O mesmo que \textunderscore rabo-branco\textunderscore .
\section{Cuamatas}
\begin{itemize}
\item {Grp. gram.:m. pl.}
\end{itemize}
Indígenas do sul de Angola.
\section{Cuambu}
\begin{itemize}
\item {Grp. gram.:m.}
\end{itemize}
Planta herbácea do Brasil.
\section{Cuanhamas}
\begin{itemize}
\item {Grp. gram.:m. pl.}
\end{itemize}
Indígenas do sul de Angola.
\section{Cuaruru-guaçu}
\begin{itemize}
\item {Grp. gram.:m.}
\end{itemize}
Planta americana, applicada na coloração do vinho.
\section{Cuba}
\begin{itemize}
\item {Grp. gram.:f.}
\end{itemize}
\begin{itemize}
\item {Proveniência:(Do lat. \textunderscore cupa\textunderscore )}
\end{itemize}
Vasilha grande, de madeira.
Balceiro.
Dorna.
Tonel.
\section{Cuba}
\begin{itemize}
\item {Grp. gram.:m.}
\end{itemize}
\begin{itemize}
\item {Utilização:Bras}
\end{itemize}
Indivíduo poderoso, influente, matreiro.
\section{Cuba}
\begin{itemize}
\item {Grp. gram.:f.}
\end{itemize}
\begin{itemize}
\item {Proveniência:(De \textunderscore Cuba\textunderscore , n. p.)}
\end{itemize}
Variedade de tabaco.
\section{Cubagem}
\begin{itemize}
\item {Grp. gram.:f.}
\end{itemize}
\begin{itemize}
\item {Proveniência:(De \textunderscore cubar\textunderscore )}
\end{itemize}
Acto, effeito ou méthodo de cubar.
Quantidade de unidades cúbicas, que podem sêr contidas em determinado espaço.
\section{Cubano}
\begin{itemize}
\item {Grp. gram.:adj.}
\end{itemize}
\begin{itemize}
\item {Grp. gram.:M.}
\end{itemize}
Relativo á ilha de Cuba.
Habitante de Cuba.
\section{Cubar}
\begin{itemize}
\item {Grp. gram.:v. t.}
\end{itemize}
\begin{itemize}
\item {Proveniência:(De \textunderscore cubo\textunderscore )}
\end{itemize}
Medir cubicamente.
\section{Cubata}
\begin{itemize}
\item {Grp. gram.:f.}
\end{itemize}
Choupana, em que habitam pretos, na África.
Senzala.
\section{Cubatura}
\begin{itemize}
\item {Grp. gram.:f.}
\end{itemize}
\begin{itemize}
\item {Proveniência:(De \textunderscore cubar\textunderscore )}
\end{itemize}
Reducção de um volume a um cubo.
\section{Cúbeba}
\begin{itemize}
\item {Grp. gram.:f.}
\end{itemize}
Planta piperácea, medicinal (\textunderscore piper cubeba\textunderscore ).
\section{Cubebeira}
\begin{itemize}
\item {Grp. gram.:f.}
\end{itemize}
(V.cúbeba)
\section{Cubebena}
\begin{itemize}
\item {Grp. gram.:f.}
\end{itemize}
O mesmo que \textunderscore cubebina\textunderscore . Cf. \textunderscore Pharm. Port.\textunderscore 
\section{Cubebina}
\begin{itemize}
\item {Grp. gram.:f.}
\end{itemize}
Princípio estimulante das cúbebas.
\section{Cubeiro}
\begin{itemize}
\item {Grp. gram.:adj.}
\end{itemize}
Que esteve em cuba.
\section{Cubelo}
\begin{itemize}
\item {fónica:bê}
\end{itemize}
\begin{itemize}
\item {Grp. gram.:m.}
\end{itemize}
\begin{itemize}
\item {Utilização:Des.}
\end{itemize}
\begin{itemize}
\item {Utilização:Ant.}
\end{itemize}
\begin{itemize}
\item {Proveniência:(De \textunderscore cuba\textunderscore )}
\end{itemize}
Torreão de fortificações antigas.
Cubículo, bitesga.
Espécie de pequeno vaso para líquidos.
\section{Cubeto}
\begin{itemize}
\item {fónica:bê}
\end{itemize}
\begin{itemize}
\item {Grp. gram.:adj.}
\end{itemize}
\begin{itemize}
\item {Proveniência:(De \textunderscore Cuba\textunderscore , n. p.)}
\end{itemize}
Diz-se do toiro, que tem as hastes muito caídas e quási juntas nas pontas.
\section{Cubi}
\begin{itemize}
\item {Grp. gram.:m.}
\end{itemize}
Nome de várias espécies de avestruz.
\section{Cubiaganga}
\begin{itemize}
\item {Grp. gram.:f.}
\end{itemize}
Pássaro dentirostro da África.
\section{Cubiça}
\begin{itemize}
\item {Grp. gram.:f.}
\end{itemize}
\begin{itemize}
\item {Proveniência:(Do lat. hypoth. \textunderscore cupiditia\textunderscore , de \textunderscore cupidus\textunderscore )}
\end{itemize}
Desejo forte.
Avidez; ambição.
\section{Cubiçador}
\begin{itemize}
\item {Grp. gram.:m.}
\end{itemize}
Aquelle que cubiça.
\section{Cubicamente}
\begin{itemize}
\item {Grp. gram.:adv.}
\end{itemize}
Aos cubos, em cubos, com medida cúbica.
\section{Cubiçante}
\begin{itemize}
\item {Grp. gram.:adj.}
\end{itemize}
Que cubiça.
\section{Cubicar}
\begin{itemize}
\item {Grp. gram.:v. t.}
\end{itemize}
\begin{itemize}
\item {Proveniência:(De \textunderscore cúbico\textunderscore )}
\end{itemize}
O mesmo que \textunderscore cubar\textunderscore .
\section{Cubiçar}
\begin{itemize}
\item {Grp. gram.:v. t.}
\end{itemize}
Têr cubiça de.
Desejar ardentemente.
Ambicionar.
\section{Cubiçável}
\begin{itemize}
\item {Grp. gram.:adj.}
\end{itemize}
\begin{itemize}
\item {Proveniência:(De \textunderscore cubiçar\textunderscore )}
\end{itemize}
Susceptível ou digno de sêr cubiçado.
\section{Cúbico}
\begin{itemize}
\item {Grp. gram.:adj.}
\end{itemize}
Relativo a cubo: \textunderscore raiz cúbica\textunderscore .
Que tem fórma de cubo.
\section{Cubiçosamente}
\begin{itemize}
\item {Grp. gram.:adv.}
\end{itemize}
De modo cubiçoso.
Com cubiça.
\section{Cubiçoso}
\begin{itemize}
\item {Grp. gram.:adj.}
\end{itemize}
Que tem cubiça.
\section{Cubicular}
\begin{itemize}
\item {Grp. gram.:adj.}
\end{itemize}
Relativo a cubículo.
\section{Cubiculário}
\begin{itemize}
\item {Grp. gram.:m.}
\end{itemize}
\begin{itemize}
\item {Utilização:Des.}
\end{itemize}
\begin{itemize}
\item {Proveniência:(Lat. \textunderscore cubicularius\textunderscore )}
\end{itemize}
Criado de quarto.
\section{Cubículo}
\begin{itemize}
\item {Grp. gram.:m.}
\end{itemize}
\begin{itemize}
\item {Utilização:Fam.}
\end{itemize}
\begin{itemize}
\item {Utilização:Ant.}
\end{itemize}
\begin{itemize}
\item {Utilização:Fig.}
\end{itemize}
\begin{itemize}
\item {Proveniência:(Lat. \textunderscore cubiculum\textunderscore )}
\end{itemize}
Pequeno quarto.
Pequeno compartimento.
Câmara, quarto de cama; cella de convento.
A parte mais intima:«\textunderscore cubículos do coração\textunderscore ». \textunderscore Luz e Calor\textunderscore , 2.
\section{Cubio}
\begin{itemize}
\item {Grp. gram.:m.}
\end{itemize}
Planta sapotácea do Brasil.
Fruto dessa planta.
\section{Cubital}
\begin{itemize}
\item {Grp. gram.:adj.}
\end{itemize}
\begin{itemize}
\item {Proveniência:(Lat. \textunderscore cubitalis\textunderscore )}
\end{itemize}
Relativo ao cúbito.
\section{Cúbito}
\begin{itemize}
\item {Grp. gram.:m.}
\end{itemize}
\begin{itemize}
\item {Proveniência:(Lat. \textunderscore cubitum\textunderscore )}
\end{itemize}
O mais grosso e o mais comprido dos dois ossos que constituem o ante-braço.
\section{Cúbito-carpiano}
\begin{itemize}
\item {Grp. gram.:adj.}
\end{itemize}
\begin{itemize}
\item {Utilização:Anat.}
\end{itemize}
Diz-se de um músculo do ante-braço.
\section{Cúbito-cutâneo}
\begin{itemize}
\item {Grp. gram.:adj.}
\end{itemize}
\begin{itemize}
\item {Utilização:Anat.}
\end{itemize}
Relativo á pelle do cubito.
\section{Cúbito-digital}
\begin{itemize}
\item {Grp. gram.:adj.}
\end{itemize}
\begin{itemize}
\item {Utilização:Anat.}
\end{itemize}
Relativo ao cúbito e aos dedos.
\section{Cúbito-palmar}
\begin{itemize}
\item {Grp. gram.:adj.}
\end{itemize}
\begin{itemize}
\item {Utilização:Anat.}
\end{itemize}
Relativo ao cúbito e á palma da mão.
\section{Cúbito-phalangiano}
\begin{itemize}
\item {Grp. gram.:adj.}
\end{itemize}
\begin{itemize}
\item {Utilização:Anat.}
\end{itemize}
Diz-se do músculo, que faz mover os dedos da mão.
\section{Cúbito-radial}
\begin{itemize}
\item {Grp. gram.:adj.}
\end{itemize}
\begin{itemize}
\item {Utilização:Anat.}
\end{itemize}
Diz-se de um músculo, que se estende do cúbito ao rádio.
\section{Cubla}
\begin{itemize}
\item {Grp. gram.:f.}
\end{itemize}
Espécie de pequena pêga da África.
\section{Cubo}
\begin{itemize}
\item {Grp. gram.:m.}
\end{itemize}
\begin{itemize}
\item {Utilização:Arith.}
\end{itemize}
\begin{itemize}
\item {Utilização:Des.}
\end{itemize}
\begin{itemize}
\item {Utilização:Prov.}
\end{itemize}
\begin{itemize}
\item {Utilização:beir.}
\end{itemize}
\begin{itemize}
\item {Utilização:Prov.}
\end{itemize}
\begin{itemize}
\item {Utilização:trasm.}
\end{itemize}
\begin{itemize}
\item {Utilização:minh.}
\end{itemize}
\begin{itemize}
\item {Utilização:Prov.}
\end{itemize}
\begin{itemize}
\item {Utilização:beir.}
\end{itemize}
\begin{itemize}
\item {Proveniência:(Lat. \textunderscore cubus\textunderscore )}
\end{itemize}
Sólido, com seis faces quadradas e iguaes.
\textunderscore Cubo de um número\textunderscore , producto de três factores iguais a êsse numero: \textunderscore 27 é o cubo de 3\textunderscore .
Medida de madeira, com 1 metro de comprimento, outro de largura e outro de altura.
Cada uma das cavidades que, nas rodas hydráulicas, recebem a água que as põe em movimento.
Espécie de cale ou calha coberta, que leva água ao rodízio do moínho.
Cesto grande e fundo.
Peça, em que se encaixa a extremidade do eixo dos carros.
Pequena tôrre de fortificação.
Medida para sólidos, equivalente a alqueire e meio. (Colhido no Fundão)
Presa de água, junto de um moínho.
O mesmo que \textunderscore alcatruz\textunderscore . (Colhido em Arganil)
\section{Cubo}
\begin{itemize}
\item {Grp. gram.:m.}
\end{itemize}
Nome que, no Japão, depois da revolução de 1585, se deu ao imperador temporal, dando-se o de \textunderscore dairo\textunderscore  ao espiritual.
\section{Cubo-cúbico}
\begin{itemize}
\item {Grp. gram.:adj.}
\end{itemize}
Relativo ao cubo-cubo.
\section{Cubo-cubo}
\begin{itemize}
\item {Grp. gram.:m.}
\end{itemize}
\begin{itemize}
\item {Utilização:Mathem.}
\end{itemize}
Décima sexta potência de um número.
\section{Cuboide}
\begin{itemize}
\item {Grp. gram.:adj.}
\end{itemize}
\begin{itemize}
\item {Grp. gram.:M.}
\end{itemize}
\begin{itemize}
\item {Proveniência:(Do gr. \textunderscore kubos\textunderscore  + \textunderscore eidos\textunderscore )}
\end{itemize}
Que tem fórma de cubo.
Osso do tarso, que articula com o calcâneo.
\section{Cubre}
\begin{itemize}
\item {Grp. gram.:m.}
\end{itemize}
Planta medicinal dos Açores. Cf. \textunderscore Hist. Insulana\textunderscore , II, 86.
\section{Cubrir}
\begin{itemize}
\item {Grp. gram.:v. t.}
\end{itemize}
\begin{itemize}
\item {Utilização:Prov.}
\end{itemize}
\begin{itemize}
\item {Utilização:trasm.}
\end{itemize}
\begin{itemize}
\item {Grp. gram.:V. p.}
\end{itemize}
\begin{itemize}
\item {Proveniência:(Lat. \textunderscore cooperire\textunderscore )}
\end{itemize}
\textunderscore v. t.\textunderscore  (e der.)
(Fórma preferível a \textunderscore cobrir\textunderscore , etc.)
Tapar, occultar, com algum objecto pôsto em cima.
Resguardar, collocando-se alguém ou alguma coisa adeante ou em volta: \textunderscore cubrir a retirada\textunderscore .
Estar, alargar-se por cima de: \textunderscore o céu cobre a humanidade\textunderscore .
Proteger.
Encher.
Vestir.
Fecundar: \textunderscore o boi cubriu a vaca\textunderscore .
Disfarçar: \textunderscore cubrir os próprios defeitos\textunderscore .
Exceder.
Abafar (o som).
Pôr na cabeça: \textunderscore cubrir o chapéu\textunderscore .
Vestir-se com: \textunderscore cubrir o capote\textunderscore .
Pôr na cabeça o chapéu, barrete, etc.
Em caminhos de ferro, avisar com sinaes (o conductor de um combóio), ou pôr a salvo por meio de sinaes (um combóio).
\section{Cuca}
\begin{itemize}
\item {Grp. gram.:f.}
\end{itemize}
\begin{itemize}
\item {Utilização:Bras}
\end{itemize}
Expressão, com que se põe mêdo ás crianças; côca.
\section{Cuca}
\begin{itemize}
\item {Grp. gram.:f.}
\end{itemize}
\begin{itemize}
\item {Utilização:Bras}
\end{itemize}
\begin{itemize}
\item {Proveniência:(De \textunderscore cuco\textunderscore )}
\end{itemize}
Mulher velha e feia.
\section{Cuca}
\begin{itemize}
\item {Grp. gram.:f.}
\end{itemize}
O mesmo que \textunderscore cóca\textunderscore ^1.
\section{Cuca}
\begin{itemize}
\item {Grp. gram.:f.}
\end{itemize}
\begin{itemize}
\item {Utilização:Prov.}
\end{itemize}
\begin{itemize}
\item {Utilização:beir.}
\end{itemize}
Pedra escura e pequena, basáltica, com que os pedreiros calçam cantarias e alvenarias, e que elles apanham entre os seixos rolados dos rios.
\section{Cuca}
\begin{itemize}
\item {Grp. gram.:f.}
\end{itemize}
\begin{itemize}
\item {Utilização:Prov.}
\end{itemize}
\begin{itemize}
\item {Utilização:dur.}
\end{itemize}
Bugalho que, quando verde, tem côr avermelhada, como algumas maçans, e que por isso é conhecida também por \textunderscore maçan-de-cuco\textunderscore .
\section{Cuca}
\begin{itemize}
\item {Grp. gram.:f.}
\end{itemize}
\begin{itemize}
\item {Utilização:Bras. de Minas}
\end{itemize}
O mesmo que \textunderscore luxo\textunderscore ^1.
\section{Cuca!}
\begin{itemize}
\item {Grp. gram.:interj.}
\end{itemize}
\begin{itemize}
\item {Utilização:T. da Bairrada  e  do Alentejo}
\end{itemize}
\begin{itemize}
\item {Utilização:infant.}
\end{itemize}
Fóra! ponha-se na rua! (De \textunderscore cucar\textunderscore ^2)
\section{Cucado}
\begin{itemize}
\item {Grp. gram.:adj.}
\end{itemize}
\begin{itemize}
\item {Utilização:Prov.}
\end{itemize}
Enfezado, (falando-se de plantas).
\section{Cucar}
\textunderscore v. t.\textunderscore  (e der.)
O mesmo que \textunderscore cocar\textunderscore ^2.
\section{Cucar}
\begin{itemize}
\item {Grp. gram.:v. i.}
\end{itemize}
\begin{itemize}
\item {Utilização:Prov.}
\end{itemize}
\begin{itemize}
\item {Utilização:alent.}
\end{itemize}
\begin{itemize}
\item {Utilização:fam.}
\end{itemize}
Andar, retirar-se.
Us. nas loc. interj. \textunderscore cucar! cucar!\textunderscore 
\section{Cucar}
\begin{itemize}
\item {Grp. gram.:v. i.}
\end{itemize}
Cantar (o cuco):«\textunderscore emquanto cuca o cuco...\textunderscore »Castilho, \textunderscore Sabichonas\textunderscore , 197.
\section{Cucarne}
\begin{itemize}
\item {Grp. gram.:m.}
\end{itemize}
Jôgo de rapazes, com ganizes.
\section{Cucha}
\begin{itemize}
\item {Grp. gram.:f.}
\end{itemize}
\begin{itemize}
\item {Utilização:Bras}
\end{itemize}
Môlho de vinagre, gengibre e outros temperso.
Esparregado, temperado com vinagre, gengibre, etc.
\section{Cucharra}
\begin{itemize}
\item {Grp. gram.:f.}
\end{itemize}
Colhér de chifre.
Colhér, com que se deita polvora na peça de artilharia.
(Cast. \textunderscore cuchara\textunderscore )
\section{Cucharro}
\begin{itemize}
\item {Grp. gram.:m.}
\end{itemize}
(V.cocharro)
\section{Cuche! cuche!}
\begin{itemize}
\item {Grp. gram.:interj.}
\end{itemize}
\begin{itemize}
\item {Utilização:Prov.}
\end{itemize}
\begin{itemize}
\item {Utilização:minh.}
\end{itemize}
(Serve para chamar porcos)
\section{Cucheri}
\begin{itemize}
\item {Grp. gram.:m.}
\end{itemize}
(V.cujumari)
\section{Cuchibi}
\begin{itemize}
\item {Grp. gram.:m.}
\end{itemize}
Árvore do Bié cujo fruto, semelhante ao feijão, é comestível. Cf. Serpa Pinto, II, 250 e 251.
\section{Cuchinaras}
\begin{itemize}
\item {Grp. gram.:m. pl.}
\end{itemize}
Índios do Brasil, que constituem várias tríbos industriosas, nas margens do Amazonas.
\section{Cuchu}
\begin{itemize}
\item {Grp. gram.:m.}
\end{itemize}
Fruta de Moçambique.
\section{Cuci}
\begin{itemize}
\item {Grp. gram.:m.}
\end{itemize}
\begin{itemize}
\item {Proveniência:(Do ár. \textunderscore cou-qui\textunderscore )}
\end{itemize}
Fruto da cuciófera.
\section{Cucio}
\begin{itemize}
\item {Grp. gram.:m.}
\end{itemize}
\begin{itemize}
\item {Utilização:Ant.}
\end{itemize}
Cochino? marran?:«\textunderscore dois arrates de carneiro e tres de cucio\textunderscore ». \textunderscore Elem. para a Hist. do Munic. de Lisbôa\textunderscore .
(Por \textunderscore cocio\textunderscore =\textunderscore cochino\textunderscore ?)
\section{Cuciófera}
\begin{itemize}
\item {Grp. gram.:f.}
\end{itemize}
\begin{itemize}
\item {Proveniência:(De \textunderscore cuci\textunderscore  + lat. \textunderscore ferre\textunderscore )}
\end{itemize}
Espécie de palmeira indiana.
\section{Cuco}
\begin{itemize}
\item {Grp. gram.:m.}
\end{itemize}
\begin{itemize}
\item {Utilização:Prov.}
\end{itemize}
\begin{itemize}
\item {Proveniência:(Do lat. \textunderscore cuculus\textunderscore )}
\end{itemize}
Ave trepadora.
Planta, mais conhecida por \textunderscore campainha amarela\textunderscore .
Relógio, que imita o canto do cuco, quando dá horas.
\textunderscore Cuspo de cuco\textunderscore , ou \textunderscore linho de cuco\textunderscore , parasito vegetal, que se enrosca nas hastes do tojo.
\section{Cuco}
\begin{itemize}
\item {Grp. gram.:m.}
\end{itemize}
O mesmo que \textunderscore coque\textunderscore ^2.
\section{Cuco}
\begin{itemize}
\item {Grp. gram.:m.  e  adj.}
\end{itemize}
\begin{itemize}
\item {Utilização:Prov.}
\end{itemize}
\begin{itemize}
\item {Utilização:Ant.}
\end{itemize}
Diz-se do marido, a quem a mulher é infiel.«\textunderscore Marido cuco me levades...\textunderscore »G. Vicente, \textunderscore Inês Pereira.\textunderscore 
(Cp. fr. \textunderscore cocu\textunderscore )
\section{Cuco}
\begin{itemize}
\item {Grp. gram.:m.}
\end{itemize}
\begin{itemize}
\item {Utilização:Gír. lisb.}
\end{itemize}
O policia.
\section{Cucolecole}
\begin{itemize}
\item {Grp. gram.:m.}
\end{itemize}
Ave pernalta da África.
\section{Çuçuapara}
\begin{itemize}
\item {Grp. gram.:f.}
\end{itemize}
\begin{itemize}
\item {Utilização:Bras}
\end{itemize}
Espécie de veado.
\section{Cucúbalo}
\begin{itemize}
\item {Grp. gram.:m.}
\end{itemize}
\begin{itemize}
\item {Proveniência:(T. mal formado, do gr. \textunderscore kukos\textunderscore  + \textunderscore bole\textunderscore )}
\end{itemize}
Gênero de plantas caryophylláceas.
\section{Cucuiada}
\begin{itemize}
\item {Grp. gram.:f.}
\end{itemize}
Outra fórma de \textunderscore cuquiada\textunderscore .
(Viria do malab. \textunderscore kukkuia\textunderscore , se a fórma exacta não fôsse \textunderscore cuquiada\textunderscore , q. v.)
\section{Cucuiar}
\begin{itemize}
\item {Grp. gram.:v. i.}
\end{itemize}
\begin{itemize}
\item {Proveniência:(Do lat. \textunderscore cuculare\textunderscore )}
\end{itemize}
Cantar (o cuco); cucar.
\section{Cuculídeas}
\begin{itemize}
\item {Grp. gram.:f. pl.}
\end{itemize}
\begin{itemize}
\item {Proveniência:(Do lat. \textunderscore cuculus\textunderscore  + gr. \textunderscore eidos\textunderscore )}
\end{itemize}
Fam. de aves trepadoras, que têm por typo o cuco.
\section{Cuculídeos}
\begin{itemize}
\item {Grp. gram.:m. pl.}
\end{itemize}
\begin{itemize}
\item {Proveniência:(Do lat. \textunderscore cuculus\textunderscore  + gr. \textunderscore eidos\textunderscore )}
\end{itemize}
Fam. de aves trepadoras, que têm por typo o cuco.
\section{Cucullo}
\begin{itemize}
\item {Grp. gram.:m.}
\end{itemize}
\begin{itemize}
\item {Proveniência:(Do lat. \textunderscore cucullus\textunderscore )}
\end{itemize}
Capuz, capello.
\section{Cuculo}
\begin{itemize}
\item {Grp. gram.:m.}
\end{itemize}
\begin{itemize}
\item {Proveniência:(Do lat. \textunderscore cucullus\textunderscore )}
\end{itemize}
Capuz, capelo.
\section{Cucumela}
\begin{itemize}
\item {Grp. gram.:f.}
\end{itemize}
Espécie de caçarola antiga.
Nome que se dá á laranja branca.
\section{Cucúrbita}
\begin{itemize}
\item {Grp. gram.:f.}
\end{itemize}
\begin{itemize}
\item {Utilização:Bot.}
\end{itemize}
\begin{itemize}
\item {Proveniência:(Lat. \textunderscore cucurbita\textunderscore )}
\end{itemize}
Peça do alambique, em que se deita a substancia que se quer destillar.
Designação scientifica da abóbora.
\section{Cucurbitáceas}
\begin{itemize}
\item {Grp. gram.:f. pl.}
\end{itemize}
\begin{itemize}
\item {Proveniência:(De \textunderscore cucurbitáceo\textunderscore )}
\end{itemize}
Fam. de plantas herbáceas, que têm por typo a abóbora.
\section{Cucurbitáceo}
\begin{itemize}
\item {Grp. gram.:adj.}
\end{itemize}
\begin{itemize}
\item {Proveniência:(De \textunderscore cucúrbita\textunderscore )}
\end{itemize}
Relativo ou semelhante á abóbora.
\section{Cucurbitar}
\begin{itemize}
\item {Grp. gram.:v. i.}
\end{itemize}
\begin{itemize}
\item {Utilização:Bot.}
\end{itemize}
Nascer, em fórma de cabaça.
\section{Cucurbitina}
\begin{itemize}
\item {Grp. gram.:f.}
\end{itemize}
\begin{itemize}
\item {Proveniência:(De \textunderscore cucurbitino\textunderscore )}
\end{itemize}
Espécie de tênia, cujos anéis semelham pevides de abóbora.
\section{Cucurbitino}
\begin{itemize}
\item {Grp. gram.:adj.}
\end{itemize}
\begin{itemize}
\item {Proveniência:(Lat. \textunderscore cucurbitinus\textunderscore )}
\end{itemize}
Semelhante á abóbora.
\section{Cucuricar}
\begin{itemize}
\item {Grp. gram.:v. i.}
\end{itemize}
O mesmo que \textunderscore cucuritar\textunderscore .
\section{Cucuritar}
\begin{itemize}
\item {Grp. gram.:v. i.}
\end{itemize}
\begin{itemize}
\item {Proveniência:(T. onom.)}
\end{itemize}
Cantar, (falando-se dos gallos).
\section{Cucuru}
\begin{itemize}
\item {Grp. gram.:m.}
\end{itemize}
Planta brasileira.
\section{Cucurucu}
\begin{itemize}
\item {Grp. gram.:m.}
\end{itemize}
Serpente venenosa do Brasil.
\section{Cucurucu}
\begin{itemize}
\item {Grp. gram.:m.}
\end{itemize}
\begin{itemize}
\item {Utilização:Ant.}
\end{itemize}
Indivíduo palrador. Cf. G. Vicente, I, 258.
(Cp. \textunderscore cucuritar\textunderscore )
\section{Cuda}
\begin{itemize}
\item {Grp. gram.:m.}
\end{itemize}
\begin{itemize}
\item {Utilização:T. de Timor}
\end{itemize}
\begin{itemize}
\item {Proveniência:(Do mal. \textunderscore kuda\textunderscore )}
\end{itemize}
Cavalgadura.
\section{Cudelume}
\begin{itemize}
\item {Grp. gram.:m.}
\end{itemize}
\begin{itemize}
\item {Utilização:Bras}
\end{itemize}
O mesmo que \textunderscore vagalume\textunderscore .
\section{Cudô}
\begin{itemize}
\item {Grp. gram.:m.}
\end{itemize}
Planta indiana, cujas fôlhas produzem effeito análogo ao da quina.
(Do conc.)
\section{Cuebas}
\begin{itemize}
\item {fónica:ê}
\end{itemize}
\begin{itemize}
\item {Grp. gram.:m.}
\end{itemize}
\begin{itemize}
\item {Utilização:Bras}
\end{itemize}
O mesmo que \textunderscore cuba\textunderscore ^2.
\section{Cuecas}
\begin{itemize}
\item {Grp. gram.:f. pl.}
\end{itemize}
\begin{itemize}
\item {Proveniência:(De \textunderscore cu\textunderscore ?)}
\end{itemize}
Ceroilas curtas.
Espécie de calções brancos e largos.
Ceroilas.
\section{Cueira}
\begin{itemize}
\item {Grp. gram.:f.}
\end{itemize}
\begin{itemize}
\item {Utilização:Prov.}
\end{itemize}
\begin{itemize}
\item {Utilização:alg.}
\end{itemize}
\begin{itemize}
\item {Utilização:Prov.}
\end{itemize}
\begin{itemize}
\item {Utilização:minh.}
\end{itemize}
Orla do espaço, em que se coze o pão no forno.
Disposição das camadas de centeio na eira, por fórma que as espigas fiquem no centro do montão e os pés para fóra.
(Cp. \textunderscore cueiro\textunderscore )
\section{Cueiro}
\begin{itemize}
\item {Grp. gram.:m.}
\end{itemize}
\begin{itemize}
\item {Proveniência:(De \textunderscore cu\textunderscore )}
\end{itemize}
Pano, em que se envolve o corpo das crianças, especialmente as nádegas.
\section{Cuelva}
\begin{itemize}
\item {Grp. gram.:f.}
\end{itemize}
Nome, que, em Abrantes, se dá ao \textunderscore tanjasmo\textunderscore .
\section{Cuènê}
\begin{itemize}
\item {Grp. gram.:m.}
\end{itemize}
Arvoreta medicinal da ilha de San-Thomé.
\section{Cuera}
\begin{itemize}
\item {fónica:ê}
\end{itemize}
\begin{itemize}
\item {Grp. gram.:f.}
\end{itemize}
\begin{itemize}
\item {Utilização:Bras. do S}
\end{itemize}
O mesmo que \textunderscore unheira\textunderscore .
\section{Cuerudo}
\begin{itemize}
\item {Grp. gram.:adj.}
\end{itemize}
\begin{itemize}
\item {Utilização:Bras. do S}
\end{itemize}
Que soffre cuera.
\section{Cufaia}
\begin{itemize}
\item {Grp. gram.:f.}
\end{itemize}
\begin{itemize}
\item {Utilização:Des.}
\end{itemize}
Mulher que, em casas nobres, faz serviços ordinários.
\section{Cúfico}
\begin{itemize}
\item {Grp. gram.:adj.}
\end{itemize}
Diz-se dos caracteres arábicos, empregados em certas inscripções, como as de Córdova e Mértola.
\section{Cuguardo}
\begin{itemize}
\item {Grp. gram.:m.}
\end{itemize}
Espécie de gato bravo da América.
\section{Cuhuraquão}
\begin{itemize}
\item {Grp. gram.:m.}
\end{itemize}
O mesmo que \textunderscore pau-brasil\textunderscore .
\section{Cuí}
\begin{itemize}
\item {Grp. gram.:m.}
\end{itemize}
\begin{itemize}
\item {Utilização:Bras. do N}
\end{itemize}
\begin{itemize}
\item {Proveniência:(T. tupi)}
\end{itemize}
Escória de tabaco, em fórma de pó.
\section{Cuia}
\begin{itemize}
\item {Grp. gram.:f.}
\end{itemize}
\begin{itemize}
\item {Utilização:Bras}
\end{itemize}
\begin{itemize}
\item {Grp. gram.:Pl.}
\end{itemize}
\begin{itemize}
\item {Proveniência:(T. guar.)}
\end{itemize}
Fruto da cuieira.
Casca do fruto da cuieira.
Almofada de cabellos postiços, que faz parte de certo penteado feminino.
Vasilha, feita da fruta cuité.
Conchas da balança.
\section{Cuia}
\begin{itemize}
\item {Grp. gram.:f.}
\end{itemize}
\begin{itemize}
\item {Utilização:Prov.}
\end{itemize}
\begin{itemize}
\item {Utilização:bras. da Baía}
\end{itemize}
\begin{itemize}
\item {Utilização:minh.}
\end{itemize}
\textunderscore Levar na cuia\textunderscore , ficar vencido na luta, quer seja luta braçal, quer seja de interesses.
(Talvez euphemismo, por \textunderscore cu\textunderscore )
\section{Cuiabano}
\begin{itemize}
\item {Grp. gram.:m.}
\end{itemize}
\begin{itemize}
\item {Grp. gram.:M.}
\end{itemize}
Relativo a Cuiaba, cidade brasileira.
Aquelle que é natural de Cuiaba.
\section{Cuiabens}
\begin{itemize}
\item {Grp. gram.:m. pl.}
\end{itemize}
Olleiros dos naires, na Índia.
\section{Cuiada}
\begin{itemize}
\item {Grp. gram.:f.}
\end{itemize}
\begin{itemize}
\item {Utilização:Bras}
\end{itemize}
\begin{itemize}
\item {Proveniência:(De \textunderscore cuia\textunderscore ^1)}
\end{itemize}
Porção, que póde sêr contida numa cuia.
\section{Cuiambuca}
\begin{itemize}
\item {Grp. gram.:f.}
\end{itemize}
\begin{itemize}
\item {Utilização:Bras. do N}
\end{itemize}
\begin{itemize}
\item {Proveniência:(Do guar. \textunderscore cuia\textunderscore  + tupi \textunderscore mboca\textunderscore )}
\end{itemize}
Vaso, feito do fruto da cuieira.
\section{Cuibaba}
\begin{itemize}
\item {Grp. gram.:f.}
\end{itemize}
Árvore angolense de Cazengo.
\section{Cuica}
\begin{itemize}
\item {Grp. gram.:f.}
\end{itemize}
\begin{itemize}
\item {Utilização:Bras}
\end{itemize}
Rato amphíbio, malhado de preto e branco.
\section{Cuidação}
\begin{itemize}
\item {Grp. gram.:f.}
\end{itemize}
\begin{itemize}
\item {Utilização:Ant.}
\end{itemize}
Acto de cuidar.
Ideia, pensamento.
\section{Cuidado}
\begin{itemize}
\item {Grp. gram.:adj.}
\end{itemize}
\begin{itemize}
\item {Grp. gram.:M.}
\end{itemize}
\begin{itemize}
\item {Grp. gram.:Interj.}
\end{itemize}
\begin{itemize}
\item {Proveniência:(De \textunderscore cuidar\textunderscore )}
\end{itemize}
Imaginado.
Meditado.
Previsto.
Applicação do espírito.
Desvelo: \textunderscore tratar doentes com cuidado\textunderscore .
Vigilância, precaução: \textunderscore caminhar nas sombras com cuidado\textunderscore .
Incumbência: \textunderscore isso fica ao meu cuidado\textunderscore .
Inquietação moral: \textunderscore estou com grande cuidado\textunderscore .
Objecto de desvelos, de precauções, de inquietações.
(para recommendar vigilância ou cautela: \textunderscore cuidado, não caias!\textunderscore )
\section{Cuidador}
\begin{itemize}
\item {Grp. gram.:m.  e  adj.}
\end{itemize}
\begin{itemize}
\item {Proveniência:(De \textunderscore cuidar\textunderscore )}
\end{itemize}
O que cuida; o que é zeloso.
\section{Cuidadoso}
\begin{itemize}
\item {Grp. gram.:adj.}
\end{itemize}
\begin{itemize}
\item {Proveniência:(De \textunderscore cuidado\textunderscore )}
\end{itemize}
Que tem cuidado; diligente.
\section{Cuidança}
\begin{itemize}
\item {Grp. gram.:f.}
\end{itemize}
\begin{itemize}
\item {Utilização:Ant.}
\end{itemize}
\begin{itemize}
\item {Proveniência:(De \textunderscore cuidar\textunderscore )}
\end{itemize}
O mesmo que \textunderscore cuidado\textunderscore .
\section{Cuidar}
\begin{itemize}
\item {Grp. gram.:v. t.}
\end{itemize}
\begin{itemize}
\item {Grp. gram.:V. i.}
\end{itemize}
\begin{itemize}
\item {Proveniência:(Do lat. \textunderscore cogitare\textunderscore )}
\end{itemize}
Imaginar: \textunderscore cuidar impossíveis\textunderscore .
Meditar.
Julgar: \textunderscore cuidava eu que tinhas juízo\textunderscore .
Tratar de.
Têr cuidado em.
Applicar a attenção.
Reflectir.
Interessar-se, trabalhar: \textunderscore cuidar dos filhos\textunderscore .
\section{Cuidaru}
\begin{itemize}
\item {Grp. gram.:f.}
\end{itemize}
\begin{itemize}
\item {Utilização:Bras}
\end{itemize}
Espécie de clava, chata e esquinada, de que usam as hordas selvagens do Pará.
\section{Cuido}
\begin{itemize}
\item {Grp. gram.:m.}
\end{itemize}
\begin{itemize}
\item {Utilização:P. us.}
\end{itemize}
\begin{itemize}
\item {Proveniência:(De \textunderscore cuidar\textunderscore )}
\end{itemize}
Acto de cuidar.
Cuidado:«\textunderscore nem me lembrava por cuido nem por penso.\textunderscore »\textunderscore Eufrosina\textunderscore , 163.
\section{Cuidoso}
\begin{itemize}
\item {Grp. gram.:adj.}
\end{itemize}
(Contr. de \textunderscore cuidadoso\textunderscore )
\section{Cuieira}
\begin{itemize}
\item {Grp. gram.:f.}
\end{itemize}
\begin{itemize}
\item {Proveniência:(De \textunderscore cuia\textunderscore ^2)}
\end{itemize}
Planta bignoniácea da América, (\textunderscore crescentia cujete\textunderscore ).
\section{Cuietê}
\begin{itemize}
\item {Grp. gram.:m.}
\end{itemize}
\begin{itemize}
\item {Utilização:Bras. do S}
\end{itemize}
O mesmo ou melhor que \textunderscore cuité\textunderscore .
\section{Cuim}
\begin{itemize}
\item {Grp. gram.:m.}
\end{itemize}
\begin{itemize}
\item {Utilização:Pop.}
\end{itemize}
\begin{itemize}
\item {Proveniência:(T. onom.)}
\end{itemize}
Animal americano, da ordem dos roedores.
O grunhir do porco, quando soffre.
\section{Cuim}
\begin{itemize}
\item {Grp. gram.:m.}
\end{itemize}
\begin{itemize}
\item {Utilização:Bras}
\end{itemize}
Alimpaduras do arroz.
(Do tupi \textunderscore cuí\textunderscore )
\section{Cuincar}
\begin{itemize}
\item {Grp. gram.:v. i.}
\end{itemize}
\begin{itemize}
\item {Utilização:Prov.}
\end{itemize}
\begin{itemize}
\item {Utilização:trasm.}
\end{itemize}
\begin{itemize}
\item {Proveniência:(T. onom.)}
\end{itemize}
O mesmo que \textunderscore ladrar\textunderscore ^1.
\section{Cuinchar}
\begin{itemize}
\item {Grp. gram.:v. i.}
\end{itemize}
\begin{itemize}
\item {Utilização:Pop.}
\end{itemize}
Grunhir o porco.
(Cp. \textunderscore cuinhar\textunderscore )
\section{Cuínha}
\begin{itemize}
\item {Grp. gram.:f.}
\end{itemize}
\begin{itemize}
\item {Utilização:Prov.}
\end{itemize}
(Fórma pop. de \textunderscore cunha\textunderscore )
Ave, o mesmo que \textunderscore galeirão\textunderscore .
\section{Cuinhar}
\begin{itemize}
\item {Grp. gram.:v. i.}
\end{itemize}
\begin{itemize}
\item {Proveniência:(De \textunderscore cuim\textunderscore ^1)}
\end{itemize}
Grunhir o porco, quando o ferem.
\section{Cuini}
\begin{itemize}
\item {Grp. gram.:m.}
\end{itemize}
Tubérculo venenoso da ilha de San-Thomé.
\section{Cuipuna}
\begin{itemize}
\item {Grp. gram.:f.}
\end{itemize}
Planta myrtácea do Brasil.
\section{Cuíque}
\begin{itemize}
\item {Grp. gram.:m.}
\end{itemize}
Melodiosa ave angolense, (\textunderscore pionias Meyerü\textunderscore ).
\section{Cuiquilhada}
\begin{itemize}
\item {Grp. gram.:f.}
\end{itemize}
(V.cuquilhada)
\section{Cuita}
\begin{itemize}
\item {Grp. gram.:f.}
\end{itemize}
\begin{itemize}
\item {Utilização:Ant.}
\end{itemize}
O mesmo que \textunderscore coita\textunderscore .
\section{Cuitado}
\begin{itemize}
\item {Grp. gram.:adj.}
\end{itemize}
\begin{itemize}
\item {Utilização:Mad}
\end{itemize}
\begin{itemize}
\item {Utilização:ant.}
\end{itemize}
\begin{itemize}
\item {Proveniência:(De \textunderscore cuita\textunderscore )}
\end{itemize}
O mesmo que \textunderscore coitado\textunderscore .
\section{Cuité}
\begin{itemize}
\item {Grp. gram.:m.}
\end{itemize}
Cuieira.
Cabaço, de que se fazem cúias, no Brasil.
Pacová.
Fruta da cuitezeira.
\section{Cuitezeira}
\begin{itemize}
\item {Grp. gram.:f.}
\end{itemize}
\begin{itemize}
\item {Utilização:Bras}
\end{itemize}
\begin{itemize}
\item {Proveniência:(De \textunderscore cuité\textunderscore )}
\end{itemize}
Arvoreta bignoniácea, de cujos frutos se fazem as cúias.
\section{Cuitó}
\begin{itemize}
\item {Grp. gram.:m.}
\end{itemize}
\begin{itemize}
\item {Utilização:Ant.}
\end{itemize}
O mesmo que \textunderscore cotó\textunderscore ^1.
Espadim.
\section{Cuitó}
\begin{itemize}
\item {Grp. gram.:m.}
\end{itemize}
\begin{itemize}
\item {Utilização:ant.}
\end{itemize}
\begin{itemize}
\item {Utilização:Chul.}
\end{itemize}
O mesmo que \textunderscore penico\textunderscore .
\section{Cuiú}
\begin{itemize}
\item {Grp. gram.:m.}
\end{itemize}
\begin{itemize}
\item {Utilização:Bras}
\end{itemize}
Espécie de papagaio.
\section{Cujamarioba}
\begin{itemize}
\item {Grp. gram.:f.}
\end{itemize}
(V.fedegoso)
\section{Cujo}
\begin{itemize}
\item {Grp. gram.:pron. relat.  e  adj.}
\end{itemize}
\begin{itemize}
\item {Grp. gram.:M.}
\end{itemize}
\begin{itemize}
\item {Utilização:Ant.}
\end{itemize}
\begin{itemize}
\item {Proveniência:(Lat. \textunderscore cujus\textunderscore )}
\end{itemize}
De que.
De quem: \textunderscore homem, cujo talento aprecio\textunderscore .
Dono.
Sujeição.
Dependência:«\textunderscore bargante que não tem cujo.\textunderscore »G. Vicente, I, 257.
\section{Cujoeiro}
\begin{itemize}
\item {Grp. gram.:m.}
\end{itemize}
Árvore da Índia portuguesa.
\section{Cujuba}
\begin{itemize}
\item {Grp. gram.:f.}
\end{itemize}
\begin{itemize}
\item {Utilização:Bras. do N}
\end{itemize}
Espécie de cabaço.
\section{Cujubeira}
\begin{itemize}
\item {Grp. gram.:f.}
\end{itemize}
\begin{itemize}
\item {Utilização:Bras. do N}
\end{itemize}
Árvore, que produz a cujuba.
\section{Cujubi}
\begin{itemize}
\item {Grp. gram.:m.}
\end{itemize}
\begin{itemize}
\item {Utilização:Bras}
\end{itemize}
O mesmo que \textunderscore cujubim\textunderscore .
\section{Cujubi-bóia}
\begin{itemize}
\item {Grp. gram.:m.}
\end{itemize}
Serpente do Brasil.
\section{Cujubim}
\begin{itemize}
\item {Grp. gram.:m.}
\end{itemize}
\begin{itemize}
\item {Utilização:Bras}
\end{itemize}
Ave gallinácea do valle do Amazonas.
(Do tupi)
\section{Cujumari}
\begin{itemize}
\item {Grp. gram.:m.}
\end{itemize}
Espécie de caneleira do Brasil.
\section{Culacharim}
\begin{itemize}
\item {Grp. gram.:m.}
\end{itemize}
Colono cultivador, na Índia portuguesa.
\section{Culaga}
\begin{itemize}
\item {Grp. gram.:f.}
\end{itemize}
\begin{itemize}
\item {Utilização:Prov.}
\end{itemize}
\begin{itemize}
\item {Utilização:trasm.}
\end{itemize}
O mesmo que \textunderscore azinhaga\textunderscore .
\section{Culandro}
\begin{itemize}
\item {Grp. gram.:m.}
\end{itemize}
\begin{itemize}
\item {Utilização:Prov.}
\end{itemize}
\begin{itemize}
\item {Utilização:trasm.}
\end{itemize}
O ânus.
\section{Culapada}
\begin{itemize}
\item {Grp. gram.:f.}
\end{itemize}
\begin{itemize}
\item {Utilização:Fam.}
\end{itemize}
\begin{itemize}
\item {Utilização:Prov.}
\end{itemize}
\begin{itemize}
\item {Utilização:beir.}
\end{itemize}
\begin{itemize}
\item {Proveniência:(Do lat. \textunderscore culus\textunderscore )}
\end{itemize}
Acto de cair de nádegas.
Acto de descair, na parte posterior:«\textunderscore o navio deu uma culapada\textunderscore ». Celestino Soares, \textunderscore Quadros Navaes\textunderscore .
Movimento brusco de alguém, indicando mau modo ou má vontade. (Colhido no Fundão)
\section{Culapar}
\begin{itemize}
\item {Grp. gram.:v. i.}
\end{itemize}
Cair de nádegas; dar culapada.
(Cp. \textunderscore culapada\textunderscore )
\section{Culatra}
\begin{itemize}
\item {Grp. gram.:f.}
\end{itemize}
\begin{itemize}
\item {Utilização:Gír.}
\end{itemize}
\begin{itemize}
\item {Utilização:Gír.}
\end{itemize}
\begin{itemize}
\item {Utilização:Techn.}
\end{itemize}
\begin{itemize}
\item {Proveniência:(Do lat. \textunderscore culus\textunderscore )}
\end{itemize}
Fundo do cano, em arma de fogo.
Parte posterior de um canhão.
Nádegas.
Meretriz.
Peça de ferro ou madeira, com que se accrescenta, na parte inferior, o dente do arado, quando êste se vai gastando com o uso.
\section{Culatral}
\begin{itemize}
\item {Grp. gram.:adj.}
\end{itemize}
Relativo a culatra.
Relativo a nádegas, parecido a nádegas:«\textunderscore como as graças do Olympo desceriam a culatraes caraças.\textunderscore »Filinto, VIII, 187.
\section{Culatrar}
\begin{itemize}
\item {Grp. gram.:v. i.}
\end{itemize}
\begin{itemize}
\item {Utilização:Prov.}
\end{itemize}
Ganhar uma partida, fazendo quatro jogos.
Ganhar o quarto e último jôgo de uma partida.
(Por \textunderscore quatrar\textunderscore , de \textunderscore quatro\textunderscore , sob a infl. de \textunderscore culatra\textunderscore )
\section{Culatrona}
\begin{itemize}
\item {Grp. gram.:f.}
\end{itemize}
\begin{itemize}
\item {Utilização:Gír.}
\end{itemize}
\begin{itemize}
\item {Proveniência:(De \textunderscore culatra\textunderscore )}
\end{itemize}
Meretriz muito reles.
\section{Cúlcitra}
\begin{itemize}
\item {Grp. gram.:f.}
\end{itemize}
\begin{itemize}
\item {Utilização:Ant.}
\end{itemize}
O mesmo que \textunderscore cócedra\textunderscore .
\section{Culcornim}
\begin{itemize}
\item {Grp. gram.:m.}
\end{itemize}
Escrivão de aldeia, na Índia portuguesa.
\section{Culicídeo}
\begin{itemize}
\item {Grp. gram.:m.}
\end{itemize}
\begin{itemize}
\item {Utilização:bras}
\end{itemize}
\begin{itemize}
\item {Utilização:Neol.}
\end{itemize}
Mosquito, considerado um dos vehículos da febre amarela.
\section{Culima}
\begin{itemize}
\item {Grp. gram.:f.}
\end{itemize}
\begin{itemize}
\item {Utilização:T. de Moçambique}
\end{itemize}
Cultura de qualquer terreno.
(Do cafreal \textunderscore cu-rima\textunderscore , cultivar)
\section{Culinária}
\begin{itemize}
\item {Grp. gram.:f.}
\end{itemize}
\begin{itemize}
\item {Proveniência:(De \textunderscore culinário\textunderscore )}
\end{itemize}
Arte de cozinhar.
\section{Culinário}
\begin{itemize}
\item {Grp. gram.:adj.}
\end{itemize}
\begin{itemize}
\item {Proveniência:(Lat. \textunderscore culinarius\textunderscore )}
\end{itemize}
Relativo a cozinha: \textunderscore usos culinários\textunderscore .
\section{Culminação}
\begin{itemize}
\item {Grp. gram.:f.}
\end{itemize}
\begin{itemize}
\item {Proveniência:(De \textunderscore culminar\textunderscore )}
\end{itemize}
O ponto mais alto, que um astro attinge acima do horizonte.
\section{Culminância}
\begin{itemize}
\item {Grp. gram.:f.}
\end{itemize}
\begin{itemize}
\item {Proveniência:(De \textunderscore culminar\textunderscore )}
\end{itemize}
O ponto mais alto.
Zenith; auge.
\section{Culminante}
\begin{itemize}
\item {Grp. gram.:adj.}
\end{itemize}
\begin{itemize}
\item {Proveniência:(Lat. \textunderscore culminans\textunderscore )}
\end{itemize}
Que é o mais elevado; \textunderscore o ponto culminante da serra\textunderscore .
\section{Culminar}
\begin{itemize}
\item {Grp. gram.:v. i.}
\end{itemize}
\begin{itemize}
\item {Proveniência:(Lat. \textunderscore culminare\textunderscore )}
\end{itemize}
Chegar ao ponto mais elevado, ao ponto culminante.
\section{Culômbio}
\begin{itemize}
\item {Grp. gram.:m.}
\end{itemize}
O mesmo que \textunderscore colômbio\textunderscore .
\section{Culombó}
\begin{itemize}
\item {Grp. gram.:m.}
\end{itemize}
Árvore da Índia portuguesa, (\textunderscore mimusops indica\textunderscore ).
\section{Culoris}
\begin{itemize}
\item {Grp. gram.:m. pl.}
\end{itemize}
Indígenas da Guiana brasileira.
\section{Culpa}
\begin{itemize}
\item {Grp. gram.:f.}
\end{itemize}
\begin{itemize}
\item {Proveniência:(Lat. \textunderscore culpa\textunderscore )}
\end{itemize}
Acto reprehensível ou criminoso.
Effeito de deixar de fazer o que se deve praticar.
Consequência de se têr feito o que se não devia fazer.
Delicto, crime.
Causa de um mal: \textunderscore tens a culpa dos meus desgostos\textunderscore .
Peccado.
\section{Culpabilidade}
\begin{itemize}
\item {Grp. gram.:f.}
\end{itemize}
\begin{itemize}
\item {Proveniência:(Do lat. \textunderscore culpabilis\textunderscore )}
\end{itemize}
Estado do que é culpável.
\section{Culpado}
\begin{itemize}
\item {Grp. gram.:m.}
\end{itemize}
\begin{itemize}
\item {Proveniência:(Lat. \textunderscore culpatus\textunderscore )}
\end{itemize}
Aquelle que é culpado; o criminoso.
\section{Culpando}
\begin{itemize}
\item {Grp. gram.:adj.}
\end{itemize}
\begin{itemize}
\item {Proveniência:(Lat. \textunderscore culpandus\textunderscore )}
\end{itemize}
Que merece sêr inculpado ou incriminado:«\textunderscore instigações culpandas\textunderscore ». Filinto.
\section{Culpar}
\begin{itemize}
\item {Grp. gram.:v. t.}
\end{itemize}
\begin{itemize}
\item {Proveniência:(Lat. \textunderscore culpare\textunderscore )}
\end{itemize}
Lançar culpa sôbre.
Incriminar; accusar.
\section{Culpável}
\begin{itemize}
\item {Grp. gram.:adj.}
\end{itemize}
\begin{itemize}
\item {Proveniência:(Lat. \textunderscore culpabilis\textunderscore )}
\end{itemize}
A que se póde imputar culpa.
Digno de censura: \textunderscore procedimento culpável\textunderscore .
\section{Culposamente}
\begin{itemize}
\item {Grp. gram.:adv.}
\end{itemize}
De modo culposo.
\section{Culposo}
\begin{itemize}
\item {Grp. gram.:adj.}
\end{itemize}
\begin{itemize}
\item {Proveniência:(De \textunderscore culpa\textunderscore )}
\end{itemize}
Que praticou culpas.
Em que há culpa: \textunderscore acto culposo\textunderscore .
\section{Cultamente}
\begin{itemize}
\item {Grp. gram.:adv.}
\end{itemize}
De modo culto.
\section{Culteranismo}
\begin{itemize}
\item {Grp. gram.:m.}
\end{itemize}
Demasiado rigor no emprêgo das palavras.
Estilo affectado e subtilmente conceituoso.
Preciosismo.
(Cast. \textunderscore culteranismo\textunderscore )
\section{Culteranista}
\begin{itemize}
\item {Grp. gram.:m.}
\end{itemize}
Aquelle que pratíca o culteranismo.
\section{Culterano}
\begin{itemize}
\item {Grp. gram.:m.}
\end{itemize}
\begin{itemize}
\item {Grp. gram.:Adj.}
\end{itemize}
O mesmo que \textunderscore culteranista\textunderscore .
Relativo ao culteranismo.
(Cast. \textunderscore culterano\textunderscore )
\section{Cultiparla}
\begin{itemize}
\item {Grp. gram.:adj.}
\end{itemize}
Que fala com erudição; que fala bem. Cf. Filinto. X, 130.
\section{Cultismo}
\begin{itemize}
\item {Grp. gram.:m.}
\end{itemize}
Qualidade do que é culto, civilizado.
O mesmo que \textunderscore culteranismo\textunderscore .
\section{Cultista}
\begin{itemize}
\item {Grp. gram.:m.}
\end{itemize}
Aquelle que segue o culteranismo.
(Cp. \textunderscore cultismo\textunderscore )
\section{Cultivação}
\begin{itemize}
\item {Grp. gram.:f.}
\end{itemize}
Acto de cultivar.
Cultura.
\section{Cultivador}
\begin{itemize}
\item {Grp. gram.:m.}
\end{itemize}
\begin{itemize}
\item {Proveniência:(De \textunderscore cultivar\textunderscore )}
\end{itemize}
Aquelle que cultiva.
Instrumento de lavoira, para cortar leivas e capar a erva.
\section{Cultivar}
\begin{itemize}
\item {Grp. gram.:v. t.}
\end{itemize}
Tornar culto.
Amanhar: \textunderscore cultivar terras\textunderscore .
Dedicar-se a: \textunderscore cultivar a literatura\textunderscore .
Formar, desenvolver: \textunderscore cultivar a vocação dos filhos\textunderscore .
Conservar.
(B. lat. \textunderscore cultivare\textunderscore )
\section{Cultivável}
\begin{itemize}
\item {Grp. gram.:adj.}
\end{itemize}
Que se póde cultivar.
\section{Cultivo}
\begin{itemize}
\item {Grp. gram.:m.}
\end{itemize}
\begin{itemize}
\item {Utilização:Prov.}
\end{itemize}
\begin{itemize}
\item {Proveniência:(De \textunderscore cultivar\textunderscore )}
\end{itemize}
Cultivação.
Cultura; amanho.
Estrume.
\section{Culto}
\begin{itemize}
\item {Grp. gram.:m.}
\end{itemize}
\begin{itemize}
\item {Grp. gram.:Adj.}
\end{itemize}
\begin{itemize}
\item {Proveniência:(Lat. \textunderscore cultus\textunderscore )}
\end{itemize}
Homenagem á divindade.
Veneração; adoração.
Fórma externa da religião.
Que se cultivou: \textunderscore terrenos cultos\textunderscore .
Instruido, sabedor: \textunderscore homem culto\textunderscore .
Adeantado em civilização; civilizado: \textunderscore nações cultas\textunderscore .
\section{Cultor}
\begin{itemize}
\item {Grp. gram.:m.}
\end{itemize}
\begin{itemize}
\item {Utilização:Fig.}
\end{itemize}
\begin{itemize}
\item {Proveniência:(Lat. \textunderscore cultor\textunderscore )}
\end{itemize}
Cultivador.
Partidário.
Aquelle, que se applica a determinado estudo: \textunderscore cultor das sciências\textunderscore .
\section{Cultridentado}
\begin{itemize}
\item {Grp. gram.:adj.}
\end{itemize}
\begin{itemize}
\item {Utilização:Zool.}
\end{itemize}
\begin{itemize}
\item {Proveniência:(De \textunderscore cultro\textunderscore  + \textunderscore dente\textunderscore )}
\end{itemize}
Que tem comprimidos os dentes caninos.
\section{Cultrifoliado}
\begin{itemize}
\item {Grp. gram.:adj.}
\end{itemize}
\begin{itemize}
\item {Utilização:Bot.}
\end{itemize}
\begin{itemize}
\item {Proveniência:(Do lat. \textunderscore culter\textunderscore  + \textunderscore folium\textunderscore )}
\end{itemize}
Que tem as fôlhas em fórma de lâmina de faca.
\section{Cultriforme}
\begin{itemize}
\item {Grp. gram.:adj.}
\end{itemize}
\begin{itemize}
\item {Proveniência:(Do lat. \textunderscore culter\textunderscore  + \textunderscore forma\textunderscore )}
\end{itemize}
Que tem fórma de lâmina de faca.
\section{Cultrirostros}
\begin{itemize}
\item {fónica:rós}
\end{itemize}
\begin{itemize}
\item {Grp. gram.:m. pl.}
\end{itemize}
\begin{itemize}
\item {Proveniência:(Do lat. \textunderscore culter\textunderscore  + \textunderscore rostrum\textunderscore )}
\end{itemize}
Família de aves pernaltas, que têm o bico semelhante á lâmina de uma faca.
\section{Cultrirrostros}
\begin{itemize}
\item {Grp. gram.:m. pl.}
\end{itemize}
\begin{itemize}
\item {Proveniência:(Do lat. \textunderscore culter\textunderscore  + \textunderscore rostrum\textunderscore )}
\end{itemize}
Família de aves pernaltas, que têm o bico semelhante á lâmina de uma faca.
\section{Cultro}
\begin{itemize}
\item {Grp. gram.:m.}
\end{itemize}
\begin{itemize}
\item {Proveniência:(Lat. \textunderscore culter\textunderscore )}
\end{itemize}
Grande cutello:«\textunderscore quando o sacerdote empunhava o cultro para immolar a hostia...\textunderscore »Latino, \textunderscore Gladiador\textunderscore .
\section{Cultual}
\begin{itemize}
\item {Grp. gram.:adj.}
\end{itemize}
Relativo ao culto: \textunderscore associação cultual\textunderscore .
\section{Cultura}
\begin{itemize}
\item {Grp. gram.:f.}
\end{itemize}
\begin{itemize}
\item {Utilização:Ant.}
\end{itemize}
\begin{itemize}
\item {Proveniência:(Lat. \textunderscore cultura\textunderscore )}
\end{itemize}
Acto, modo ou effeito de cultivar: \textunderscore a cultura dos campos\textunderscore .
Estado de quem tem desenvolvimento intellectual.
Utilização industrial de certos productos naturaes.
Estudo.
Elegância, esmêro.
O mesmo que \textunderscore culteranismo\textunderscore . Cf. Jac. Freire, \textunderscore D. João de Castro\textunderscore , p. VIII.
\section{Cultural}
\begin{itemize}
\item {Grp. gram.:adj.}
\end{itemize}
Relativo a cultura.
\section{Culturanismo}
\begin{itemize}
\item {Grp. gram.:m.}
\end{itemize}
Assim designam infundadamente diccionaristas nossos, derivando-a de \textunderscore cultura\textunderscore , a escola literária, que em Espanha se chamou \textunderscore culteranismo\textunderscore .
(V. esta palavra)
\section{Culturanista}
\begin{itemize}
\item {Grp. gram.:m.}
\end{itemize}
(V.culteranista)
\section{Culturano}
\begin{itemize}
\item {Grp. gram.:adj.}
\end{itemize}
(V.culterano)
\section{Culuglis}
\begin{itemize}
\item {Grp. gram.:m. pl.}
\end{itemize}
Povo barberesco, procedente da mistura de Turcos, Moiros e Árabes.
\section{Cumacumans}
\begin{itemize}
\item {Grp. gram.:m. pl.}
\end{itemize}
Índios selvagens das margens do Apaporis, no Brasil.
\section{Çumagre}
\begin{itemize}
\item {Grp. gram.:m.}
\end{itemize}
O mesmo ou melhor que \textunderscore sumagre\textunderscore .
Gênero de plantas terebintháceas.
Pó, mais ou menos grosseiro, resultante da trituração das fôlhas, flôres, etc., dêsse gênero de plantas, e empregado em Medicina e tinturaria.
(Cp. cast. \textunderscore zumaque\textunderscore )
\section{Cumaí}
\begin{itemize}
\item {Grp. gram.:m.}
\end{itemize}
Fruto silvestre do Brasil.
\section{Cumameri}
\begin{itemize}
\item {Grp. gram.:m.}
\end{itemize}
(V.sorveira)
\section{Cumandália}
\begin{itemize}
\item {Grp. gram.:f.}
\end{itemize}
Planta americana, trepadeira e leguminosa.
\section{Cumari}
\begin{itemize}
\item {Grp. gram.:m.}
\end{itemize}
O mesmo que \textunderscore cumarim\textunderscore .
\section{Cumárico}
\begin{itemize}
\item {Grp. gram.:adj.}
\end{itemize}
Diz de um ácido, que se extrai da cumarina.
\section{Cumarim}
\begin{itemize}
\item {Grp. gram.:m.}
\end{itemize}
\begin{itemize}
\item {Utilização:Bras}
\end{itemize}
Árvore laurácea, espécie de pimenteira.
\section{Cumarina}
\begin{itemize}
\item {Grp. gram.:f.}
\end{itemize}
\begin{itemize}
\item {Utilização:Chím.}
\end{itemize}
Princípio neutro particular.
\section{Cumaru}
\begin{itemize}
\item {Grp. gram.:m.}
\end{itemize}
Árvore leguminosa do Brasil, cujo fruto se chama entre nós \textunderscore fava da Índia\textunderscore .
\section{Cumati}
\begin{itemize}
\item {Grp. gram.:m.}
\end{itemize}
Planta myrtácea do Brasil.
\section{Cumauaru}
\begin{itemize}
\item {Grp. gram.:m.}
\end{itemize}
\begin{itemize}
\item {Utilização:Bras}
\end{itemize}
Árvore silvestre do Pará, bôa para construcções civis e náuticas.
\section{Çumbaia}
\begin{itemize}
\item {Grp. gram.:f.}
\end{itemize}
(Fórma ant. de \textunderscore zumbaia\textunderscore ). Cf. Barros, \textunderscore Dec\textunderscore . II, l. V, c. 2.
\section{Cumbar}
\begin{itemize}
\item {Grp. gram.:m.}
\end{itemize}
Olleiro, na Índia portuguesa.
\textunderscore Casta dos cumbares\textunderscore , a dos olleiros.
\section{Cumbaru}
\begin{itemize}
\item {Grp. gram.:m.}
\end{itemize}
O mesmo que \textunderscore cumaru\textunderscore .
\section{Cumbé}
\begin{itemize}
\item {Grp. gram.:m.}
\end{itemize}
Dança de negros.
\section{Cumbeba}
\begin{itemize}
\item {Grp. gram.:f.}
\end{itemize}
Espécie de cacto brasileiro.
\section{Cumbeca}
\begin{itemize}
\item {Grp. gram.:f.}
\end{itemize}
\begin{itemize}
\item {Utilização:Bras}
\end{itemize}
Formosa trepadeira do Pará.
\section{Cumbicuri}
\begin{itemize}
\item {Grp. gram.:m.}
\end{itemize}
Reptil africano.
\section{Cumbio}
\begin{itemize}
\item {Grp. gram.:m.}
\end{itemize}
Árvore indiana, de fibras têxteis.
\section{Cumbo}
\begin{itemize}
\item {Grp. gram.:m.}
\end{itemize}
\begin{itemize}
\item {Utilização:Prov.}
\end{itemize}
\begin{itemize}
\item {Utilização:alent.}
\end{itemize}
Variedade de barbo, que tem a pelle do focinho áspera, como lixa.
\section{Cumbo}
\begin{itemize}
\item {Grp. gram.:m.}
\end{itemize}
Medida indiana, equivalente a trinta e dois hectolitros.
\section{Cumbo}
\begin{itemize}
\item {Grp. gram.:adj.}
\end{itemize}
O mesmo que \textunderscore combo\textunderscore . Cf. \textunderscore Elegíada\textunderscore , I, 95.
\section{Çumbos}
\begin{itemize}
\item {Grp. gram.:m. pl.}
\end{itemize}
\begin{itemize}
\item {Utilização:Ant.}
\end{itemize}
O mesmo que \textunderscore muzimbas\textunderscore .
\section{Cumbuca}
\begin{itemize}
\item {Grp. gram.:f.}
\end{itemize}
\begin{itemize}
\item {Utilização:Bras}
\end{itemize}
O mesmo que \textunderscore cuiambuca\textunderscore .
\section{Cumbuco}
\begin{itemize}
\item {Grp. gram.:adj.}
\end{itemize}
\begin{itemize}
\item {Utilização:Bras}
\end{itemize}
Diz-se do boi ou vaca, cujos chifres têm as pontas voltadas uma para a outra.
\section{Cume}
\begin{itemize}
\item {Grp. gram.:m.}
\end{itemize}
\begin{itemize}
\item {Utilização:Fig.}
\end{itemize}
\begin{itemize}
\item {Proveniência:(Do lat. \textunderscore culmen\textunderscore )}
\end{itemize}
O ponto mais alto de um monte.
Cimo; coruta.
Apogeu; auge.
\section{Cumeada}
\begin{itemize}
\item {Grp. gram.:f.}
\end{itemize}
\begin{itemize}
\item {Proveniência:(De \textunderscore cume\textunderscore )}
\end{itemize}
Série de cumes de montanhas.
Cumeeira.
\section{Cumeeira}
\begin{itemize}
\item {Grp. gram.:f.}
\end{itemize}
Cume.
A parte mais alta do telhado, da montanha.
(B. lat. \textunderscore cumenaria\textunderscore )
\section{Cumeno}
\begin{itemize}
\item {Grp. gram.:m.}
\end{itemize}
\begin{itemize}
\item {Utilização:Chím.}
\end{itemize}
Um dos carbonetos do grupo benzênico.
\section{Cumerim}
\begin{itemize}
\item {Grp. gram.:m.}
\end{itemize}
Desbaste ou córte de árvores, na Índia portuguesa.
\section{Cumiada}
\begin{itemize}
\item {Grp. gram.:f.}
\end{itemize}
(V.cumeada)
\section{Cumichá}
\begin{itemize}
\item {Grp. gram.:m.}
\end{itemize}
Planta nyctagínea do Brasil.
Outra planta, semelhante ao mangue.
\section{Cumieira}
\begin{itemize}
\item {Grp. gram.:f.}
\end{itemize}
(V.cumeeira)
\section{Cuminal}
\begin{itemize}
\item {Grp. gram.:adj.}
\end{itemize}
\begin{itemize}
\item {Proveniência:(De \textunderscore cume\textunderscore )}
\end{itemize}
O mesmo que \textunderscore culminante\textunderscore .
\section{Cuminas}
\begin{itemize}
\item {Grp. gram.:f. pl.}
\end{itemize}
\begin{itemize}
\item {Proveniência:(Do lat. \textunderscore cuminum\textunderscore )}
\end{itemize}
Tríbo de plantas, que tem por typo o cuminho.
\section{Cuminho}
\begin{itemize}
\item {Grp. gram.:m.}
\end{itemize}
\begin{itemize}
\item {Grp. gram.:Pl.}
\end{itemize}
\begin{itemize}
\item {Proveniência:(Lat. \textunderscore cuminum\textunderscore )}
\end{itemize}
Planta umbellífera.
Frutos ou sementes, que constituem especiaria.
\section{Çumo}
\begin{itemize}
\item {Grp. gram.:m.}
\end{itemize}
O mesmo ou melhor que \textunderscore sumo\textunderscore .
Líquido, extrahido de algumas substâncias vegetaes.
(Cp. cast. \textunderscore zumo\textunderscore )
\section{Cúmplice}
\begin{itemize}
\item {Grp. gram.:m.  e  adj.}
\end{itemize}
\begin{itemize}
\item {Utilização:Ext.}
\end{itemize}
\begin{itemize}
\item {Proveniência:(Lat. \textunderscore complex\textunderscore )}
\end{itemize}
O que tomou parte num delicto ou crime.
O que collabora ou toma parte com outrem nalgum facto.
\section{Cumpliciar-se}
\begin{itemize}
\item {Grp. gram.:v. p.}
\end{itemize}
Tornar-se cúmplice.
\section{Cumplicidade}
\begin{itemize}
\item {Grp. gram.:f.}
\end{itemize}
Acto ou qualidade de quem é cúmplice.
\section{Cumplimentário}
\begin{itemize}
\item {Grp. gram.:adj.}
\end{itemize}
\begin{itemize}
\item {Utilização:Jur.}
\end{itemize}
Dizia-se, em Direito commercial, do administrador de uma sociedade, ou do gerente de uma casa. Cf. Borges Carneiro, \textunderscore Dicc. Jur\textunderscore .
\section{Cumpridoiro}
\begin{itemize}
\item {Grp. gram.:adj.}
\end{itemize}
\begin{itemize}
\item {Utilização:Ant.}
\end{itemize}
\begin{itemize}
\item {Proveniência:(De \textunderscore cumprir\textunderscore )}
\end{itemize}
Que cumpre o seu dever.
Pontual.
Útil, vantajoso. Cf. \textunderscore Port. Mon. Hist., Script.\textunderscore , 282.
Necessário.
Obrigatório.
\section{Cumpridor}
\begin{itemize}
\item {Grp. gram.:adj.}
\end{itemize}
\begin{itemize}
\item {Grp. gram.:M.}
\end{itemize}
\begin{itemize}
\item {Utilização:Fig.}
\end{itemize}
\begin{itemize}
\item {Proveniência:(De \textunderscore cumprir\textunderscore )}
\end{itemize}
Que cumpre, que executa: \textunderscore cumpridor dos seus deveres\textunderscore .
Testamenteiro ou cumpridor de testamento.
\section{Cumpridouro}
\begin{itemize}
\item {Grp. gram.:adj.}
\end{itemize}
\begin{itemize}
\item {Utilização:Ant.}
\end{itemize}
\begin{itemize}
\item {Proveniência:(De \textunderscore cumprir\textunderscore )}
\end{itemize}
Que cumpre o seu dever.
Pontual.
Útil, vantajoso. Cf. \textunderscore Port. Mon. Hist., Script.\textunderscore , 282.
Necessário.
Obrigatório.
\section{Cumprimentador}
\begin{itemize}
\item {Grp. gram.:adj.}
\end{itemize}
O mesmo que \textunderscore cumprimenteiro\textunderscore .
\section{Cumprimentar}
\begin{itemize}
\item {Grp. gram.:v. t.}
\end{itemize}
Dirigir cumprimentos a.
Fazer elogios a.
\section{Cufeia}
\begin{itemize}
\item {Grp. gram.:f.}
\end{itemize}
\begin{itemize}
\item {Proveniência:(Do gr. \textunderscore kuphos\textunderscore , curvo)}
\end{itemize}
Planta litrariada.
\section{Cumprimenteiro}
\begin{itemize}
\item {Grp. gram.:adj.}
\end{itemize}
Que se demasia em cumprimentos.
\section{Cumprimento}
\begin{itemize}
\item {Grp. gram.:m.}
\end{itemize}
\begin{itemize}
\item {Grp. gram.:Pl.}
\end{itemize}
Acto ou effeito de cumprir: \textunderscore cumprimento de obrigações\textunderscore .
Execução.
Acto de baixar a cabeça ou tirar o chapéu, por cortesia.
Palavras corteses, que, de viva voz ou por escrito, se dirigem a alguém, saudando-o.
Saudação.
Expressões de civilidade, dirigidas a alguém, para significar satisfação ou pesar, por acontecimento que interessa á pessôa que se cumprimenta.
Formalidades, que se observam entre pessôas, que não têm familiaridade entre si.
Allocução ou visita de cortesia, que certos dignitários e corporações dirigem ao chefe do Estado ou a outras personagens elevadas, em dia solenne, por congratulação ou por pesar.
Modos ou palavras ceremoniosas.
\section{Cumprir}
\begin{itemize}
\item {Grp. gram.:v. t.}
\end{itemize}
\begin{itemize}
\item {Grp. gram.:V. i.}
\end{itemize}
\begin{itemize}
\item {Utilização:Ant.}
\end{itemize}
\begin{itemize}
\item {Proveniência:(Lat. \textunderscore complere\textunderscore )}
\end{itemize}
Levar a effeito, satisfazer, executar: \textunderscore cumpri as suas ordens\textunderscore .
Completar.
Convir.
Sêr preciso: \textunderscore cumpre-nos estudar\textunderscore .
Pertencer.
Impender.
Comprazer. Cf. \textunderscore Eufrosina\textunderscore , 164.
\textunderscore Cumprir com\textunderscore , dar satisfação a:«\textunderscore tendes parentes, com que é preciso cumprir\textunderscore ». \textunderscore Eufrosina\textunderscore , 350. Cf. \textunderscore Peregrinação\textunderscore , XXII.
\section{Cum-quibus}
\begin{itemize}
\item {fónica:cu-í-bus}
\end{itemize}
\begin{itemize}
\item {Grp. gram.:m. pl.}
\end{itemize}
\begin{itemize}
\item {Utilização:Fam.}
\end{itemize}
Meios de comprar.
Dinheiro.
(Loc. lat., que literalmente significa \textunderscore com que\textunderscore )
\section{Cumulação}
\begin{itemize}
\item {Grp. gram.:f.}
\end{itemize}
Acto de cumular.
\section{Cumular}
\begin{itemize}
\item {Grp. gram.:v. t.}
\end{itemize}
(V.acumular)
\section{Cumulativamente}
\begin{itemize}
\item {Grp. gram.:adv.}
\end{itemize}
De modo cumulativo.
Com acumulação.
\section{Cumulativo}
\begin{itemize}
\item {Grp. gram.:adj.}
\end{itemize}
\begin{itemize}
\item {Utilização:Jur.}
\end{itemize}
\begin{itemize}
\item {Proveniência:(Do lat. \textunderscore cumulatus\textunderscore )}
\end{itemize}
Feito por acumulação.
Que faz parte do que se acumula.
Diz-se das disposições legaes, sôbre hypótheses já prevenidas por outras disposições.
\section{Cúmulo}
\begin{itemize}
\item {Grp. gram.:m.}
\end{itemize}
\begin{itemize}
\item {Grp. gram.:Pl.}
\end{itemize}
\begin{itemize}
\item {Proveniência:(Lat. \textunderscore cumulus\textunderscore )}
\end{itemize}
Reunião de coisas sobrepostas.
Montão.
O ponto mais alto.
Auge: \textunderscore é o cúmulo do disparate\textunderscore .
Accréscimo.
Nuvens arredondadas e brancas, que se acumulam no horizonte em tempo sêco.
\section{Cumuramás}
\begin{itemize}
\item {Grp. gram.:m. pl.}
\end{itemize}
Indígenas brasileiros da região do Amazonas.
\section{Cuna}
\begin{itemize}
\item {Grp. gram.:f.}
\end{itemize}
\begin{itemize}
\item {Utilização:Ant.}
\end{itemize}
\begin{itemize}
\item {Proveniência:(Lat. \textunderscore cuna\textunderscore )}
\end{itemize}
Berço.
\section{Cunamanas}
\begin{itemize}
\item {Grp. gram.:m. pl.}
\end{itemize}
Indígenas brasileiros da região do Amazonas.
\section{Cunambi}
\begin{itemize}
\item {Grp. gram.:m.}
\end{itemize}
Árvore brasileira, cuja seiva se emprega na narcotização de peixes, para os pescar.
\section{Cunanan}
\begin{itemize}
\item {Grp. gram.:m.}
\end{itemize}
\begin{itemize}
\item {Utilização:Bras}
\end{itemize}
Espécie de cipó, que, nas florestas, fere o viandante, produzindo nelle effeito análogo ao da queimadura.
\section{Cunauaru}
\begin{itemize}
\item {Grp. gram.:m.}
\end{itemize}
\begin{itemize}
\item {Utilização:Bras. do N}
\end{itemize}
Sapo escuro, de olhos vermelhos.
\section{Cunca}
\begin{itemize}
\item {Grp. gram.:f.}
\end{itemize}
\begin{itemize}
\item {Utilização:Prov.}
\end{itemize}
\begin{itemize}
\item {Utilização:alent.}
\end{itemize}
\begin{itemize}
\item {Utilização:Prov.}
\end{itemize}
\begin{itemize}
\item {Utilização:trasm}
\end{itemize}
\begin{itemize}
\item {Utilização:Bras. do Ceará}
\end{itemize}
\begin{itemize}
\item {Utilização:Bras}
\end{itemize}
O mesmo que \textunderscore conca\textunderscore ^1.
Queijo de pequenas dimensões.
O mesmo que \textunderscore rótula\textunderscore  (do joelho).
Espécie de tubérculos sumarentos, que se desenvolvem nas raízes do imbuzeiro.
Vaso massiço de madeira, com tampa, em que se guardam e transportam comidas.
\section{Cuncharra}
\begin{itemize}
\item {Grp. gram.:f.}
\end{itemize}
\begin{itemize}
\item {Utilização:Gír.}
\end{itemize}
Colhér.
Gazúa.
(Cp. \textunderscore cucharra\textunderscore )
\section{Cunco}
\begin{itemize}
\item {Grp. gram.:m.}
\end{itemize}
\begin{itemize}
\item {Utilização:Prov.}
\end{itemize}
\begin{itemize}
\item {Utilização:trasm.}
\end{itemize}
Caçoila.
Escudela.
(Cp. \textunderscore cunca\textunderscore )
\section{Cunctatório}
\begin{itemize}
\item {Grp. gram.:adj.}
\end{itemize}
\begin{itemize}
\item {Proveniência:(Do lat. \textunderscore cunctator\textunderscore )}
\end{itemize}
Relativo a delonga, demora.
Que envolve adiamento.
Vagaroso.
Contemporizador.
\section{Cundurim}
\begin{itemize}
\item {Grp. gram.:m.}
\end{itemize}
Antigo e pequeno pêso de Malaca.
\section{Cunduru}
\begin{itemize}
\item {Grp. gram.:m.}
\end{itemize}
Árvore fructífera do Brasil.
\section{Cuneano}
\begin{itemize}
\item {Grp. gram.:adj.}
\end{itemize}
\begin{itemize}
\item {Utilização:Anat.}
\end{itemize}
\begin{itemize}
\item {Proveniência:(Do lat. \textunderscore cuneus\textunderscore )}
\end{itemize}
Que tem fórma de cunha; cuneiforme.
Relativo aos ossos cuneiformes do peito do pé.
\section{Cuneifoliado}
\begin{itemize}
\item {Grp. gram.:adj.}
\end{itemize}
\begin{itemize}
\item {Utilização:Bot.}
\end{itemize}
\begin{itemize}
\item {Proveniência:(Do lat. \textunderscore cuneus\textunderscore  + \textunderscore folium\textunderscore )}
\end{itemize}
Que tem fôlhas em fórma de cunha.
\section{Cuneiforme}
\begin{itemize}
\item {Grp. gram.:adj.}
\end{itemize}
\begin{itemize}
\item {Grp. gram.:M. pl.}
\end{itemize}
\begin{itemize}
\item {Proveniência:(Lat. \textunderscore cuneiformis\textunderscore )}
\end{itemize}
Que tem fórma de cunha.
Caracteres cuneiformes: \textunderscore os cuneiformes assýrios e persas\textunderscore .
\section{Cuneirostros}
\begin{itemize}
\item {fónica:rós}
\end{itemize}
\begin{itemize}
\item {Grp. gram.:m. pl.}
\end{itemize}
\begin{itemize}
\item {Proveniência:(Do lat. \textunderscore cuneus\textunderscore  + \textunderscore rostrum\textunderscore )}
\end{itemize}
Aves trepadoras, que têm bico em fórma de cunha.
\section{Cuneirrostros}
\begin{itemize}
\item {Grp. gram.:m. pl.}
\end{itemize}
\begin{itemize}
\item {Proveniência:(Do lat. \textunderscore cuneus\textunderscore  + \textunderscore rostrum\textunderscore )}
\end{itemize}
Aves trepadoras, que têm bico em fórma de cunha.
\section{Cúneo}
\begin{itemize}
\item {Grp. gram.:m.}
\end{itemize}
\begin{itemize}
\item {Utilização:Des.}
\end{itemize}
Escrínio, guarda-jóias. Cf. \textunderscore Viriato Trág.\textunderscore , I, 95.
\section{Cuneta}
\begin{itemize}
\item {fónica:nê}
\end{itemize}
\begin{itemize}
\item {Grp. gram.:f.}
\end{itemize}
Espécie de corvo, (\textunderscore corvus monedula\textunderscore , Lin.).
\section{Cunha}
\begin{itemize}
\item {Grp. gram.:f.}
\end{itemize}
\begin{itemize}
\item {Utilização:Fig.}
\end{itemize}
\begin{itemize}
\item {Grp. gram.:Loc. adv.}
\end{itemize}
\begin{itemize}
\item {Utilização:Náut.}
\end{itemize}
\begin{itemize}
\item {Grp. gram.:Pl.}
\end{itemize}
\begin{itemize}
\item {Proveniência:(Do lat. \textunderscore cuneus\textunderscore )}
\end{itemize}
Utensílio de ferro, em fórma de ângulo sólido, com uma aresta mais ou menos cortante, para fender madeira, pedras, etc.
Peça de madeira, de formato semelhante ao da cunha.
Instrumento, que servia para levantar a culatra da peça.
Ferro triangular, que, depois de encandescente, dá calor a certos instrumentos de engomar, hoje pouco usados.
Instrumento, com que se seguravam os mastaréus, sôbre as barras dos mastros.
Empenho, pessôa que serve de empenho.
\textunderscore Á cunha\textunderscore , apertadamente: \textunderscore a plateia estava á cunha\textunderscore .
Forçadamente.
Sem necessidade.
\textunderscore Levar os mastaréus á cunha\textunderscore , içar os mastaréus por ante-avante dos mastros, até ficarem no seu lugar.
Certas pennas da asa do falcão.
\section{Cunhada}
\begin{itemize}
\item {Grp. gram.:f.}
\end{itemize}
\begin{itemize}
\item {Proveniência:(Do lat. \textunderscore cognata\textunderscore )}
\end{itemize}
Irman de um dos cônjuges, em relação ao outro, e viceversa.
\section{Cunhadia}
\begin{itemize}
\item {Grp. gram.:f.}
\end{itemize}
\begin{itemize}
\item {Utilização:Ant.}
\end{itemize}
O mesmo que \textunderscore cunhadio\textunderscore .
\section{Cunhadio}
\begin{itemize}
\item {Grp. gram.:m.}
\end{itemize}
Parentesco entre cunhados.
\section{Cunhado}
\begin{itemize}
\item {Grp. gram.:m.}
\end{itemize}
\begin{itemize}
\item {Proveniência:(Do lat. \textunderscore cognatus\textunderscore )}
\end{itemize}
Irmão de um dos cônjuges, relativamente ao outro cônjuge e viceversa.
\section{Cunhado}
\begin{itemize}
\item {Grp. gram.:adj.}
\end{itemize}
\begin{itemize}
\item {Proveniência:(De \textunderscore cunhar\textunderscore )}
\end{itemize}
Que se cunhou.
Amoedado.
\section{Cunhador}
\begin{itemize}
\item {Grp. gram.:m.  e  adj.}
\end{itemize}
Aquelle que cunha.
\section{Cunhagem}
\begin{itemize}
\item {Grp. gram.:f.}
\end{itemize}
Acto ou effeito de cunhar (moéda).
\section{Cunhal}
\begin{itemize}
\item {Grp. gram.:m.}
\end{itemize}
\begin{itemize}
\item {Proveniência:(De \textunderscore cunha\textunderscore )}
\end{itemize}
Ângulo saliente, formado por duas paredes convergentes.
Esquina.
\section{Cunhambambe}
\begin{itemize}
\item {Grp. gram.:m.}
\end{itemize}
Árvore angolense de Caconda.
\section{Cunhan}
\begin{itemize}
\item {Grp. gram.:f.}
\end{itemize}
\begin{itemize}
\item {Utilização:Bras. do N}
\end{itemize}
\begin{itemize}
\item {Proveniência:(T. tupi)}
\end{itemize}
Menina de raça aborígene.
A mulher do caboclo.
\section{Cunhanhas}
\begin{itemize}
\item {Grp. gram.:m.}
\end{itemize}
\begin{itemize}
\item {Utilização:Chul.}
\end{itemize}
Homem muito acanhado; maricas.
(Cp. \textunderscore conana\textunderscore )
\section{Cunhar}
\begin{itemize}
\item {Grp. gram.:v. t.}
\end{itemize}
\begin{itemize}
\item {Utilização:Fig.}
\end{itemize}
\begin{itemize}
\item {Proveniência:(De \textunderscore cunho\textunderscore )}
\end{itemize}
Imprimir cunho em.
Amoedar.
Tornar saliente, notável.
Inventar.
\section{Cunheira}
\begin{itemize}
\item {Grp. gram.:f.}
\end{itemize}
\begin{itemize}
\item {Utilização:Prov.}
\end{itemize}
Cunha grande.
Fenda, aberta na pedra, para se cravarem as cunhas que a hão de partir.
\section{Cunhete}
\begin{itemize}
\item {fónica:nhê}
\end{itemize}
\begin{itemize}
\item {Grp. gram.:m.}
\end{itemize}
Caixote de madeira, que serve especialmente para guardar ou transportar pólvora.
\section{Cunho}
\begin{itemize}
\item {Grp. gram.:m.}
\end{itemize}
\begin{itemize}
\item {Utilização:Fig.}
\end{itemize}
\begin{itemize}
\item {Grp. gram.:Pl.}
\end{itemize}
\begin{itemize}
\item {Proveniência:(Do lat. \textunderscore cuneus\textunderscore )}
\end{itemize}
Ferro gravado, com que se marcam moédas, medalhas, etc.
A marca que êsse ferro deixa impressa.
Marca, sêllo, carácter.
Paus, a que se liga o linguete, em tôrno do cabrestante, nos navios.
Uma das faces de certas moédas, na qual se representavam as armas reaes.
\section{Cunho}
\begin{itemize}
\item {Grp. gram.:m.}
\end{itemize}
\begin{itemize}
\item {Utilização:Prov.}
\end{itemize}
Penedo grande e solitário, em meio de um rio.
O mesmo que \textunderscore cônho\textunderscore .
\section{Cúnia}
\begin{itemize}
\item {Grp. gram.:f.}
\end{itemize}
Gênero do plantas synanthéreas.
\section{Cunículo}
\begin{itemize}
\item {Grp. gram.:m.}
\end{itemize}
\begin{itemize}
\item {Utilização:Des.}
\end{itemize}
\begin{itemize}
\item {Proveniência:(Lat. \textunderscore cuniculum\textunderscore )}
\end{itemize}
Caminho subterrâneo.
\section{Cunicultura}
\begin{itemize}
\item {Grp. gram.:f.}
\end{itemize}
\begin{itemize}
\item {Proveniência:(Do lat. \textunderscore cuniculus\textunderscore  + \textunderscore cultura\textunderscore )}
\end{itemize}
Criação de coêlhos.
\section{Cunónia}
\begin{itemize}
\item {Grp. gram.:f.}
\end{itemize}
\begin{itemize}
\item {Proveniência:(De \textunderscore Cuno\textunderscore , n. p.)}
\end{itemize}
Gênero de plantas saxifragáceas.
\section{Cunqueiro}
\begin{itemize}
\item {Grp. gram.:m.}
\end{itemize}
\begin{itemize}
\item {Utilização:Prov.}
\end{itemize}
\begin{itemize}
\item {Utilização:trasm.}
\end{itemize}
Planta, espécie de azêdas, que nasce nos paredões das arribas e nas rampas das estradas.
(Relaciona-se com \textunderscore cunca\textunderscore ?)
\section{Cunques}
\begin{itemize}
\item {Grp. gram.:m. pl.}
\end{itemize}
\begin{itemize}
\item {Utilização:Prov.}
\end{itemize}
\begin{itemize}
\item {Utilização:trasm.}
\end{itemize}
\begin{itemize}
\item {Proveniência:(De \textunderscore com\textunderscore  + \textunderscore que\textunderscore . Cp. \textunderscore cum-quibus\textunderscore )}
\end{itemize}
Dinheiro.
\section{Cunta}
\begin{itemize}
\item {Grp. gram.:f.}
\end{itemize}
\begin{itemize}
\item {Utilização:Ant.}
\end{itemize}
O mesmo que \textunderscore conta\textunderscore .
\section{Cunuris}
\begin{itemize}
\item {Grp. gram.:m. pl.}
\end{itemize}
Índios imberbes da margem esquerda do Amazonas, e que os descobridores espanhóes confundiram com mulheres, vindo talvez daí a lenda das guerreiras amazonas.
\section{Cupá}
\begin{itemize}
\item {Grp. gram.:m.}
\end{itemize}
Planta brasileira, de raiz comestível.
\section{Cupânia}
\begin{itemize}
\item {Grp. gram.:f.}
\end{itemize}
\begin{itemize}
\item {Proveniência:(De \textunderscore Cupani\textunderscore , n. p.)}
\end{itemize}
Gênero de plantas sapindáceas, fructiferas e medicinaes, (\textunderscore cupania rapida\textunderscore , Lin.).
\section{Cupão}
\begin{itemize}
\item {Grp. gram.:m.}
\end{itemize}
\begin{itemize}
\item {Proveniência:(Fr. \textunderscore coupon\textunderscore )}
\end{itemize}
Titulo de juro, que faz parte de acção ou obrigação e que se corta na occasião do pagamento.
\section{Cupão}
\begin{itemize}
\item {Grp. gram.:m.}
\end{itemize}
Antigo e pequeno pêso de Malaca.
\section{Cupaurana}
\begin{itemize}
\item {Grp. gram.:f.}
\end{itemize}
\begin{itemize}
\item {Utilização:Bras. do N}
\end{itemize}
Planta medicinal.
\section{Cupé}
\begin{itemize}
\item {Grp. gram.:m.}
\end{itemize}
\begin{itemize}
\item {Proveniência:(Fr. \textunderscore coupé\textunderscore )}
\end{itemize}
Carruagem fechada, geralmente de dois lugares.
\section{Cupez}
\begin{itemize}
\item {Grp. gram.:m.}
\end{itemize}
\begin{itemize}
\item {Utilização:Náut.}
\end{itemize}
Encapelladura, que deita um ovem para cada bôrdo da embarcação.
\section{Cupheia}
\begin{itemize}
\item {Grp. gram.:f.}
\end{itemize}
\begin{itemize}
\item {Proveniência:(Do gr. \textunderscore kuphos\textunderscore , curvo)}
\end{itemize}
Planta lythrariada.
\section{Cupi}
\begin{itemize}
\item {Grp. gram.:m.}
\end{itemize}
\begin{itemize}
\item {Utilização:Bras}
\end{itemize}
Gênero de insectos da região dos Purus.
\section{Cupida}
\begin{itemize}
\item {Grp. gram.:f.}
\end{itemize}
\begin{itemize}
\item {Utilização:Des.}
\end{itemize}
\begin{itemize}
\item {Proveniência:(De \textunderscore Cupido\textunderscore , n. p.)}
\end{itemize}
Namorada.
\section{Cupidez}
\begin{itemize}
\item {Grp. gram.:f.}
\end{itemize}
\begin{itemize}
\item {Proveniência:(De \textunderscore cúpido\textunderscore )}
\end{itemize}
Cubiça; ambição.
\section{Cupidíneo}
\begin{itemize}
\item {Grp. gram.:adj.}
\end{itemize}
\begin{itemize}
\item {Proveniência:(Do lat. \textunderscore cupido\textunderscore )}
\end{itemize}
Relativo a Cupido, ao amor.
\section{Cupidinosamente}
\begin{itemize}
\item {Grp. gram.:adv.}
\end{itemize}
De modo cupidinoso.
\section{Cupidinoso}
\begin{itemize}
\item {Grp. gram.:adj.}
\end{itemize}
\begin{itemize}
\item {Proveniência:(Do lat. \textunderscore cupido\textunderscore , \textunderscore cupidinis\textunderscore )}
\end{itemize}
Que deseja com ardor.
Amoroso.
\section{Cupidista}
\begin{itemize}
\item {Grp. gram.:adj.}
\end{itemize}
\begin{itemize}
\item {Utilização:Chul.}
\end{itemize}
Relativo a Cupido:«\textunderscore para o mal cupidista não póde haver melhor remédio...\textunderscore »A. J. Silva, \textunderscore Guerras do Alecrim\textunderscore .
\section{Cupido}
\begin{itemize}
\item {Grp. gram.:m.}
\end{itemize}
\begin{itemize}
\item {Utilização:Fig.}
\end{itemize}
\begin{itemize}
\item {Proveniência:(Do lat. \textunderscore Cupido\textunderscore , n. p.)}
\end{itemize}
Personificação do amor.
Amor.
Homem pretensioso, que se julga bonito.
\section{Cúpido}
\begin{itemize}
\item {Grp. gram.:adj.}
\end{itemize}
\begin{itemize}
\item {Proveniência:(Lat. \textunderscore cupidus\textunderscore )}
\end{itemize}
Ávido; muito ambicioso.
\section{Cupim}
\begin{itemize}
\item {Grp. gram.:m.}
\end{itemize}
\begin{itemize}
\item {Utilização:Bras}
\end{itemize}
\begin{itemize}
\item {Utilização:Bras. do N}
\end{itemize}
\begin{itemize}
\item {Utilização:Bras. do N}
\end{itemize}
Pequena formiga, que corrói a madeira e que em Portugal é conhecida por \textunderscore formiga-branca\textunderscore .
Habitação de térmites.
Testa do boi.
Corcova do zebu.
\section{Cupineira}
\begin{itemize}
\item {Grp. gram.:f.}
\end{itemize}
\begin{itemize}
\item {Utilização:Bras}
\end{itemize}
\begin{itemize}
\item {Proveniência:(De \textunderscore cúpim\textunderscore )}
\end{itemize}
Abelha, que vive na habitação abandonada pelo cupim.
\section{Cupinhorós}
\begin{itemize}
\item {Grp. gram.:m. pl.}
\end{itemize}
\begin{itemize}
\item {Utilização:Bras}
\end{itemize}
Aborígenes do Maranhão.
\section{Cupinzeiro}
\begin{itemize}
\item {Grp. gram.:m.}
\end{itemize}
\begin{itemize}
\item {Utilização:Bras}
\end{itemize}
\begin{itemize}
\item {Proveniência:(De \textunderscore cupim\textunderscore )}
\end{itemize}
Habitação de térmites.
\section{Cupio}
\begin{itemize}
\item {Grp. gram.:m.}
\end{itemize}
Pássaro conirostro da África.
\section{Cupira}
\begin{itemize}
\item {Grp. gram.:f.}
\end{itemize}
(V.cupineira)
\section{Cupiúba}
\begin{itemize}
\item {Grp. gram.:f.}
\end{itemize}
Planta therebinthácea do Brasil.
\section{Cuprato}
\begin{itemize}
\item {Grp. gram.:m.}
\end{itemize}
\begin{itemize}
\item {Proveniência:(Do lat. \textunderscore cuprum\textunderscore )}
\end{itemize}
Sal de deutóxydo de cobre.
\section{Cupressifoliado}
\begin{itemize}
\item {Grp. gram.:adj.}
\end{itemize}
\begin{itemize}
\item {Proveniência:(Do lat. \textunderscore cupressus\textunderscore  + \textunderscore folium\textunderscore )}
\end{itemize}
Que tem fôlhas semelhantes ás do cipreste.
\section{Cupressiforme}
\begin{itemize}
\item {Grp. gram.:adj.}
\end{itemize}
\begin{itemize}
\item {Proveniência:(Do lat. \textunderscore cupressus\textunderscore  + \textunderscore forma\textunderscore )}
\end{itemize}
Semelhante ao cipreste.
\section{Cupressíneas}
\begin{itemize}
\item {Grp. gram.:f. pl.}
\end{itemize}
\begin{itemize}
\item {Utilização:Bot.}
\end{itemize}
\begin{itemize}
\item {Proveniência:(De \textunderscore cupressíneo\textunderscore )}
\end{itemize}
Família de plantas da ordem das coníferas, a qual tem por typo o cipreste.
\section{Cupressíneo}
\begin{itemize}
\item {Grp. gram.:adj.}
\end{itemize}
\begin{itemize}
\item {Proveniência:(Lat. \textunderscore cupressinus\textunderscore )}
\end{itemize}
Relativo ou semelhante ao cipreste.
Feito de cipreste.
\section{Cupressino}
\begin{itemize}
\item {Grp. gram.:adj.}
\end{itemize}
\begin{itemize}
\item {Proveniência:(Lat. \textunderscore cupressinus\textunderscore )}
\end{itemize}
Relativo ou semelhante ao cipreste.
Feito de cipreste.
\section{Cupressita}
\begin{itemize}
\item {Grp. gram.:f.}
\end{itemize}
\begin{itemize}
\item {Proveniência:(Do lat. \textunderscore cupressus\textunderscore )}
\end{itemize}
Vegetal fóssil, análogo ao cipreste.
\section{Cúprico}
\begin{itemize}
\item {Grp. gram.:adj.}
\end{itemize}
\begin{itemize}
\item {Proveniência:(Do lat. \textunderscore cuprum\textunderscore )}
\end{itemize}
Que é de cobre.
Em que há cobre.
\section{Cuprífero}
\begin{itemize}
\item {Grp. gram.:adj.}
\end{itemize}
\begin{itemize}
\item {Proveniência:(Do lat. \textunderscore cuprum\textunderscore  + \textunderscore ferre\textunderscore )}
\end{itemize}
Que contém cobre.
\section{Cuprino}
\begin{itemize}
\item {Grp. gram.:adj.}
\end{itemize}
\begin{itemize}
\item {Proveniência:(Lat. \textunderscore cuprinus\textunderscore )}
\end{itemize}
Relativo a cobre.
\section{Cupripene}
\begin{itemize}
\item {Grp. gram.:adj.}
\end{itemize}
\begin{itemize}
\item {Utilização:Zool.}
\end{itemize}
\begin{itemize}
\item {Proveniência:(Do lat. \textunderscore cuprum\textunderscore  + \textunderscore penna\textunderscore )}
\end{itemize}
Que tem asas ou elytros da côr do cobre.
\section{Cupripenne}
\begin{itemize}
\item {Grp. gram.:adj.}
\end{itemize}
\begin{itemize}
\item {Utilização:Zool.}
\end{itemize}
\begin{itemize}
\item {Proveniência:(Do lat. \textunderscore cuprum\textunderscore  + \textunderscore penna\textunderscore )}
\end{itemize}
Que tem asas ou elytros da côr do cobre.
\section{Cuprirostro}
\begin{itemize}
\item {fónica:rós}
\end{itemize}
\begin{itemize}
\item {Grp. gram.:adj.}
\end{itemize}
\begin{itemize}
\item {Utilização:Zool.}
\end{itemize}
\begin{itemize}
\item {Proveniência:(Do lat. \textunderscore cuprum\textunderscore  + \textunderscore rostrum\textunderscore )}
\end{itemize}
Que tem bico côr de cobre.
\section{Cuprirrostro}
\begin{itemize}
\item {Grp. gram.:adj.}
\end{itemize}
\begin{itemize}
\item {Utilização:Zool.}
\end{itemize}
\begin{itemize}
\item {Proveniência:(Do lat. \textunderscore cuprum\textunderscore  + \textunderscore rostrum\textunderscore )}
\end{itemize}
Que tem bico côr de cobre.
\section{Cuprite}
\begin{itemize}
\item {Grp. gram.:f.}
\end{itemize}
\begin{itemize}
\item {Proveniência:(Do lat. \textunderscore cuprum\textunderscore )}
\end{itemize}
Minério de côr vermelho-cochonilha, que se encontra com outros, quando se exploram para a extracção do cobre.
\section{Cuprofulminato}
\begin{itemize}
\item {Grp. gram.:m.}
\end{itemize}
Fulminato de cobre.
\section{Cupropotássico}
\begin{itemize}
\item {Grp. gram.:adj.}
\end{itemize}
Relativo ao cobre e á potassa.
\section{Cupróxido}
\begin{itemize}
\item {Grp. gram.:m.}
\end{itemize}
\begin{itemize}
\item {Proveniência:(Do lat. \textunderscore cuprum\textunderscore  + gr. \textunderscore oxus\textunderscore )}
\end{itemize}
Óxido de cobre.
\section{Cupróxydo}
\begin{itemize}
\item {Grp. gram.:m.}
\end{itemize}
\begin{itemize}
\item {Proveniência:(Do lat. \textunderscore cuprum\textunderscore  + gr. \textunderscore oxus\textunderscore )}
\end{itemize}
Óxydo de cobre.
\section{Cupu}
\begin{itemize}
\item {Grp. gram.:m.}
\end{itemize}
Fruto brasílico, semelhante ao cacau.
Doce, feito dêsse fruto.
\section{Cupuaú}
\begin{itemize}
\item {Grp. gram.:m.}
\end{itemize}
Árvore leguminosa do Brasil.
\section{Cupuaçu}
\begin{itemize}
\item {Grp. gram.:m.}
\end{itemize}
Árvore malvácea do Brasil.
O fruto dessa árvore.
\section{Cupuim}
\begin{itemize}
\item {Grp. gram.:m.}
\end{itemize}
Arbusto myrtáceo do Brasil, com que se envenena o peixe, para o pescar.
\section{Cúpula}
\begin{itemize}
\item {Grp. gram.:f.}
\end{itemize}
\begin{itemize}
\item {Utilização:Bot.}
\end{itemize}
\begin{itemize}
\item {Proveniência:(Lat. \textunderscore cupula\textunderscore )}
\end{itemize}
Parte côncava e superior de alguns edifícios.
Parte côncava do zimbório.
Zimbório.
Abóbada.
Aquillo que dá o aspecto de uma abóbada: \textunderscore a cúpula celeste\textunderscore .
Espécie de cálice, formado de pequenas brácteas, que envolvem a flôr e o fructo de alguns vegetaes.
\section{Cupuláceas}
\begin{itemize}
\item {Grp. gram.:f. pl.}
\end{itemize}
O mesmo ou melhor que \textunderscore copulíferas\textunderscore .
\section{Cupulado}
\begin{itemize}
\item {Grp. gram.:adj.}
\end{itemize}
Que tem cúpula.
\section{Cupulíferas}
\begin{itemize}
\item {Grp. gram.:f. pl.}
\end{itemize}
\begin{itemize}
\item {Proveniência:(Do lat. \textunderscore cupula\textunderscore  + \textunderscore ferre\textunderscore )}
\end{itemize}
Família de plantas, cujos frutos têm cúpula, como o carvalho, o castanheiro, etc.
\section{Cupuliforme}
\begin{itemize}
\item {Grp. gram.:adj.}
\end{itemize}
\begin{itemize}
\item {Proveniência:(Do lat. \textunderscore cupula\textunderscore  + \textunderscore forma\textunderscore )}
\end{itemize}
Que tem fórma de cúpula.
\section{Cupulim}
\begin{itemize}
\item {Grp. gram.:m.}
\end{itemize}
\begin{itemize}
\item {Utilização:Constr.}
\end{itemize}
\begin{itemize}
\item {Proveniência:(De \textunderscore cúpula\textunderscore )}
\end{itemize}
Lanternim, que, num terraço, resguarda a entrada de uma escada.
\section{Cupunaçu}
\begin{itemize}
\item {Grp. gram.:m.}
\end{itemize}
O mesmo que \textunderscore cupuaçu\textunderscore .
\section{Cuque}
\begin{itemize}
\item {Grp. gram.:m.}
\end{itemize}
\begin{itemize}
\item {Proveniência:(Ingl. \textunderscore cook\textunderscore )}
\end{itemize}
O mesmo que \textunderscore coque\textunderscore ^2. Cf. Castilho, \textunderscore Avarento\textunderscore , 176.
\section{Cuquenha}
\begin{itemize}
\item {Grp. gram.:f.}
\end{itemize}
\begin{itemize}
\item {Utilização:des.}
\end{itemize}
\begin{itemize}
\item {Utilização:Gír.}
\end{itemize}
Fortuna, bôa sorte.
(Cp. fr. \textunderscore cocagne\textunderscore )
\section{Cuqueiro}
\begin{itemize}
\item {Grp. gram.:m.}
\end{itemize}
\begin{itemize}
\item {Utilização:Bras. de Minas}
\end{itemize}
\begin{itemize}
\item {Proveniência:(De \textunderscore cuca\textunderscore ^6)}
\end{itemize}
Luxento; que gosta do luxo.
\section{Cuqueiro}
\begin{itemize}
\item {Grp. gram.:m.}
\end{itemize}
\begin{itemize}
\item {Utilização:Prov.}
\end{itemize}
\begin{itemize}
\item {Utilização:dur.}
\end{itemize}
\begin{itemize}
\item {Proveniência:(De \textunderscore cuca\textunderscore ^5?)}
\end{itemize}
Casinhola, na parte posterior dos barcos rabelos, onde dormem os tripulantes.
\section{Cuquiada}
\begin{itemize}
\item {Grp. gram.:f.}
\end{itemize}
\begin{itemize}
\item {Utilização:Ant.}
\end{itemize}
Vozes, com que na Índia se chamava o povo ás armas, e que eram repetidas e propagadas pelas pessôas, que as ouviam.
Vozes, com que no mar se annunciava a aproximação de terra.
Vozearia. Cp. Gonç. Viana, \textunderscore Apostilas\textunderscore , em que se discute a fórma e a origem do t.
(Cp. \textunderscore couquilhada\textunderscore )
\section{Cuquil}
\begin{itemize}
\item {Grp. gram.:m.}
\end{itemize}
Pequeno cuco de Bengala.
\section{Cuquilhada}
\begin{itemize}
\item {Grp. gram.:f.}
\end{itemize}
\begin{itemize}
\item {Utilização:Prov.}
\end{itemize}
\begin{itemize}
\item {Utilização:trasm.}
\end{itemize}
Vozearia, o mesmo que \textunderscore couquilhada\textunderscore .
\section{Cura}
\begin{itemize}
\item {Grp. gram.:f.}
\end{itemize}
\begin{itemize}
\item {Utilização:Fig.}
\end{itemize}
\begin{itemize}
\item {Grp. gram.:M.}
\end{itemize}
\begin{itemize}
\item {Proveniência:(Lat. \textunderscore cura\textunderscore )}
\end{itemize}
Acto ou effeito de curar.
Tratamento.
Restabelecimento da saúde.
Emenda: \textunderscore aquelle defeito já não tem cura\textunderscore .
Sacerdote, que pastoreia um pequeno povo.
Coadjutor de párocho.
\section{Curabi}
\begin{itemize}
\item {Grp. gram.:m.}
\end{itemize}
\begin{itemize}
\item {Utilização:Bras. do N}
\end{itemize}
Pequena seta ervada, de que usam os selvagens.
\section{Ciameia}
\begin{itemize}
\item {Grp. gram.:f.}
\end{itemize}
\begin{itemize}
\item {Proveniência:(Lat. \textunderscore cyamea\textunderscore )}
\end{itemize}
Pedra preciosa, hoje desconhecida e que parece semelhava uma fava.
\section{Cíamo}
\begin{itemize}
\item {Grp. gram.:m.}
\end{itemize}
\begin{itemize}
\item {Proveniência:(Gr. \textunderscore kuamos\textunderscore )}
\end{itemize}
Animal parasito, que vive sôbre a baleia.
O mesmo que \textunderscore colocásia\textunderscore  ou \textunderscore fava\textunderscore  do Egipto.
\section{Curunilha}
\begin{itemize}
\item {Grp. gram.:f.}
\end{itemize}
Árvore brasileira do Paraná.
\section{Curunuá}
\begin{itemize}
\item {Grp. gram.:m.}
\end{itemize}
\begin{itemize}
\item {Utilização:Bras}
\end{itemize}
Espécie de aranha das regiões do Amazonas.
\section{Curupira}
\begin{itemize}
\item {Grp. gram.:m.}
\end{itemize}
\begin{itemize}
\item {Utilização:Bras}
\end{itemize}
\begin{itemize}
\item {Proveniência:(T. tupi)}
\end{itemize}
Ente fantástico que, segundo a crendice popular, habita as matas e tem os calcanhares voltados para diante e os dedos dos pés para trás.
\section{Curupitá}
\begin{itemize}
\item {Grp. gram.:m.}
\end{itemize}
Árvore medicinal do Alto-Amazonas.
\section{Cururé}
\begin{itemize}
\item {Grp. gram.:m.}
\end{itemize}
Erva medicinal do Alto-Amazonas.
\section{Cururi}
\begin{itemize}
\item {Grp. gram.:m.}
\end{itemize}
\begin{itemize}
\item {Utilização:Bras}
\end{itemize}
Sal vegetal, extrahido de podostemáceas, para conservação do peixe.
\section{Cururu}
\begin{itemize}
\item {Grp. gram.:m.}
\end{itemize}
\begin{itemize}
\item {Utilização:Bras}
\end{itemize}
Planta trepadeira, de suco venenoso.
\section{Cururu}
\begin{itemize}
\item {Grp. gram.:m.}
\end{itemize}
\begin{itemize}
\item {Utilização:Bras}
\end{itemize}
Espécie de sapo.
Espécie de batuque.
\section{Cururu-bóia}
\begin{itemize}
\item {Grp. gram.:f.}
\end{itemize}
\begin{itemize}
\item {Utilização:Bras}
\end{itemize}
Cobra verde, de grandes dimensões.
\section{Curutzu}
\begin{itemize}
\item {Grp. gram.:m.}
\end{itemize}
Planta brasileira.
\section{Curva}
\begin{itemize}
\item {Grp. gram.:f.}
\end{itemize}
\begin{itemize}
\item {Utilização:Náut.}
\end{itemize}
\begin{itemize}
\item {Utilização:ant.}
\end{itemize}
\begin{itemize}
\item {Utilização:Veter.}
\end{itemize}
\begin{itemize}
\item {Proveniência:(De \textunderscore curvo\textunderscore )}
\end{itemize}
Linha curva ou linha sinuosa; linha, que não é recta, nem formada de rectas.
Trajecto sinuoso.
Construcção em fórma de arco; arco.
Objecto de fórma arqueada.
Curvatura.
Cada uma das duas peças, mais ou menos trabalhadas, que sustentam a parte principal de certos móveis, como as peças que sustentam a parte superior dos guarda-pratas, a gaveta de certas peniqueiras, etc.
O mesmo que \textunderscore beliche\textunderscore ?:«\textunderscore ...um caixão, que está na curva do contramestre.\textunderscore »(De um testamento de 1691)«\textunderscore Declaro que levo nesta náo na minha curva duas colchas...\textunderscore »\textunderscore Ibidem\textunderscore .
\textunderscore Curva da perna\textunderscore , parte posterior da perna, em opposição ao joêlho.
Tumor pequeno, mais ou menos duro, na curva da perna das cavalgaduras. Cf. M. Pinto, \textunderscore Comp. de Veter.\textunderscore , I, 422.
\section{Curvaça}
\begin{itemize}
\item {Grp. gram.:f.}
\end{itemize}
\begin{itemize}
\item {Utilização:Anat.}
\end{itemize}
\begin{itemize}
\item {Proveniência:(De \textunderscore curvo\textunderscore )}
\end{itemize}
Sobreosso, abaixo da curva da perna, na parte externa da extremidade superior da canela dos equídeos.
\section{Curvado}
\begin{itemize}
\item {Grp. gram.:adj.}
\end{itemize}
\begin{itemize}
\item {Utilização:Fig.}
\end{itemize}
\begin{itemize}
\item {Proveniência:(De \textunderscore curvar\textunderscore )}
\end{itemize}
Que tem fórma de arco.
Inclinado para deante.
Curvo.
Inclinado para baixo.
Opprimido; sujeito.
Paciente, resignado.
\section{Curvadura}
\begin{itemize}
\item {Grp. gram.:f.}
\end{itemize}
\begin{itemize}
\item {Utilização:Veter.}
\end{itemize}
\begin{itemize}
\item {Proveniência:(De \textunderscore curvar\textunderscore )}
\end{itemize}
Compressão, produzida pela convexidade da lâmina de um cravo, nos tecidos vivos do casco dos solípedes. Cf. Leon, \textunderscore Arte de Ferrar\textunderscore , 224.
\section{Curval}
\begin{itemize}
\item {Grp. gram.:adj.}
\end{itemize}
Relativo á curva da perna.
\section{Curvar}
\begin{itemize}
\item {Grp. gram.:v. t.}
\end{itemize}
\begin{itemize}
\item {Utilização:Fig.}
\end{itemize}
\begin{itemize}
\item {Grp. gram.:V. i.}
\end{itemize}
\begin{itemize}
\item {Grp. gram.:V. p.}
\end{itemize}
\begin{itemize}
\item {Proveniência:(Lat. \textunderscore curvare\textunderscore )}
\end{itemize}
Tornar curvo.
Arquear.
Inclinar para deante ou para baixo: \textunderscore inclinar a cabeça\textunderscore .
Dobrar em ângulo ou em arco.
Abater, sujeitar.
Tornar-se curvo.
Tornar-se curvo.
Apresentar fórma curva.
Ajoelhar; prostrar-se: \textunderscore curvar-se aos pés de alguém\textunderscore .
Resignar-se.
Abater-se.
Ir declinando.
\section{Curvativo}
\begin{itemize}
\item {Grp. gram.:adj.}
\end{itemize}
\begin{itemize}
\item {Proveniência:(De \textunderscore curvar\textunderscore )}
\end{itemize}
Diz-se das fôlhas vegetaes, que se enrolam quási imperceptivelmente.
\section{Curvatões}
\begin{itemize}
\item {Grp. gram.:m. pl.}
\end{itemize}
\begin{itemize}
\item {Utilização:Náut.}
\end{itemize}
Duas peças do mastro, acima da roman, nas quaes assentam os vaus reaes.
\section{Curvatura}
\begin{itemize}
\item {Grp. gram.:f.}
\end{itemize}
\begin{itemize}
\item {Proveniência:(De \textunderscore curvar\textunderscore )}
\end{itemize}
Estado daquillo que é curvo.
\section{Curveiro}
\begin{itemize}
\item {Grp. gram.:m.}
\end{itemize}
\begin{itemize}
\item {Utilização:T. da Figueira-da-Foz}
\end{itemize}
Remoínho de água no mar.
\section{Curvejão}
\begin{itemize}
\item {Grp. gram.:m.}
\end{itemize}
\begin{itemize}
\item {Proveniência:(De \textunderscore curvo\textunderscore )}
\end{itemize}
Jarrete do cavallo.
\section{Curvejar}
\begin{itemize}
\item {Grp. gram.:v. t.}
\end{itemize}
\begin{itemize}
\item {Utilização:Des.}
\end{itemize}
Percorrer em tôrno.
Formar curvas sôbre:«\textunderscore o bigode, curvejando os beiços...\textunderscore »\textunderscore Anat. Joc.\textunderscore 
\section{Curveta}
\begin{itemize}
\item {fónica:vê}
\end{itemize}
\begin{itemize}
\item {Grp. gram.:f.}
\end{itemize}
Movimento do cavallo, quando levanta e dobra as mãos, baixando a garupa.
(Cp. fr. courbette)
\section{Curveta}
\begin{itemize}
\item {fónica:vê}
\end{itemize}
\begin{itemize}
\item {Grp. gram.:f.}
\end{itemize}
\begin{itemize}
\item {Proveniência:(De \textunderscore curva\textunderscore )}
\end{itemize}
Pequena curva, volta de caminho ou de atalho.
\section{Curvetear}
\begin{itemize}
\item {Grp. gram.:v. i.}
\end{itemize}
\begin{itemize}
\item {Proveniência:(De \textunderscore curveta\textunderscore ^1)}
\end{itemize}
Fazer curvetas o cavallo.
\section{Curvicórneo}
\begin{itemize}
\item {Grp. gram.:adj.}
\end{itemize}
Que tem cornos curvos. Cf. Castillo, \textunderscore Metam.\textunderscore , 194.
\section{Curvidade}
\begin{itemize}
\item {Grp. gram.:f.}
\end{itemize}
O mesmo que \textunderscore curvatura\textunderscore .
\section{Curvifloro}
\begin{itemize}
\item {Grp. gram.:adj.}
\end{itemize}
\begin{itemize}
\item {Utilização:Bot.}
\end{itemize}
\begin{itemize}
\item {Proveniência:(Do lat. \textunderscore curvus\textunderscore  + \textunderscore flos\textunderscore )}
\end{itemize}
Que tem a corolla curva.
\section{Curvifoliado}
\begin{itemize}
\item {Grp. gram.:adj.}
\end{itemize}
\begin{itemize}
\item {Utilização:Bot.}
\end{itemize}
\begin{itemize}
\item {Proveniência:(De \textunderscore curvo\textunderscore  + \textunderscore foliado\textunderscore )}
\end{itemize}
Que tem fôlhas recurvadas.
\section{Curvifronte}
\begin{itemize}
\item {Grp. gram.:adj.}
\end{itemize}
\begin{itemize}
\item {Proveniência:(De \textunderscore curvo\textunderscore  + \textunderscore fronte\textunderscore )}
\end{itemize}
Que tem testa arqueada.
\section{Curvígrafo}
\begin{itemize}
\item {Grp. gram.:m.}
\end{itemize}
\begin{itemize}
\item {Proveniência:(De \textunderscore curvo\textunderscore  + gr. \textunderscore graphein\textunderscore )}
\end{itemize}
Instrumento, para traçar curvas.
\section{Curvígrapho}
\begin{itemize}
\item {Grp. gram.:m.}
\end{itemize}
\begin{itemize}
\item {Proveniência:(De \textunderscore curvo\textunderscore  + gr. \textunderscore graphein\textunderscore )}
\end{itemize}
Instrumento, para traçar curvas.
\section{Curvilhão}
\begin{itemize}
\item {Grp. gram.:m.}
\end{itemize}
O mesmo que \textunderscore curvejão\textunderscore .
\section{Curvillíneo}
\begin{itemize}
\item {Grp. gram.:adj.}
\end{itemize}
\begin{itemize}
\item {Proveniência:(Do lat. \textunderscore curvus\textunderscore  + \textunderscore linea\textunderscore )}
\end{itemize}
Formado de linhas curvas.
Que tem fórma de curva.
Que segue direcção curva.
\section{Curvípede}
\begin{itemize}
\item {Grp. gram.:adj.}
\end{itemize}
\begin{itemize}
\item {Proveniência:(Do lat. \textunderscore curvus\textunderscore  + \textunderscore pes\textunderscore )}
\end{itemize}
Que tem pernas curvas.
\section{Curvirostro}
\begin{itemize}
\item {fónica:rós}
\end{itemize}
\begin{itemize}
\item {Grp. gram.:adj.}
\end{itemize}
\begin{itemize}
\item {Utilização:Zool.}
\end{itemize}
\begin{itemize}
\item {Proveniência:(Do lat. \textunderscore curvus\textunderscore  + \textunderscore rostrum\textunderscore )}
\end{itemize}
Que tem bico curvo.
\section{Curvirrostro}
\begin{itemize}
\item {Grp. gram.:adj.}
\end{itemize}
\begin{itemize}
\item {Utilização:Zool.}
\end{itemize}
\begin{itemize}
\item {Proveniência:(Do lat. \textunderscore curvus\textunderscore  + \textunderscore rostrum\textunderscore )}
\end{itemize}
Que tem bico curvo.
\section{Curvo}
\begin{itemize}
\item {Grp. gram.:adj.}
\end{itemize}
\begin{itemize}
\item {Proveniência:(Lat. \textunderscore curvus\textunderscore )}
\end{itemize}
Que tem fórma de arco.
Que não é recto, nem formado de rectas: \textunderscore linha curva\textunderscore .
Que não é plano: \textunderscore superfície curva\textunderscore .
Curvado.
\section{Cuscó}
\begin{itemize}
\item {Grp. gram.:m.}
\end{itemize}
Arvoreta da Índia portuguesa.
\section{Cuscúcio}
\begin{itemize}
\item {Grp. gram.:m.}
\end{itemize}
\begin{itemize}
\item {Utilização:Prov.}
\end{itemize}
Cordeiro, que nasce no outono.
\section{Cuscus}
\begin{itemize}
\item {Grp. gram.:m. pl.}
\end{itemize}
\begin{itemize}
\item {Utilização:Bras. da Baía}
\end{itemize}
\begin{itemize}
\item {Grp. gram.:M. sing.}
\end{itemize}
\begin{itemize}
\item {Utilização:Bras. do N}
\end{itemize}
Grânulos de farinha, para sopa.
Massa do arroz, ou de milho, ou de mandioca, ou de tapioca, com côco ou sem elle, de que se fazem fatias, que, com manteiga, são preferidas ao pão.
Espécie de pão de farinha de arroz ou de milho, cozido ao vapor da água em ebullição.
A fórma exacta seria \textunderscore cuscuz\textunderscore , ou antes \textunderscore cuzcuz\textunderscore ; mas, como não há o plural \textunderscore cuscuzes\textunderscore  nem é admissível o plural sob a fórma de \textunderscore cuscuz\textunderscore , póde manter-se a fórma usual \textunderscore cuscus\textunderscore , para o singular e para o plural, como succedeu em \textunderscore alferes\textunderscore , \textunderscore ourives\textunderscore , \textunderscore simples\textunderscore , etc, cujas fórmas exactas seriam aliás \textunderscore alférez\textunderscore , \textunderscore ourivez\textunderscore , \textunderscore simplez\textunderscore , etc.
(Do ár.)
\section{Cuscuseira}
\begin{itemize}
\item {Grp. gram.:f.}
\end{itemize}
\begin{itemize}
\item {Utilização:Bras. da Baia}
\end{itemize}
Vaso especial, para se fazerem cuscus.
\section{Cuscuseiro}
\begin{itemize}
\item {Grp. gram.:m.}
\end{itemize}
\begin{itemize}
\item {Utilização:Bras. do N}
\end{itemize}
Fabricante ou vendedor de cuscus.
Vasilha, em que se cozem cuscus.
\section{Cuscuta}
\begin{itemize}
\item {Grp. gram.:f.}
\end{itemize}
Planta parasita e trepadeira (\textunderscore cuscuta europoea\textunderscore ).
\section{Cuscutáceas}
\begin{itemize}
\item {Grp. gram.:f. pl.}
\end{itemize}
\begin{itemize}
\item {Proveniência:(De \textunderscore cuscuta\textunderscore )}
\end{itemize}
Família de plantas, formada á custa das convulvuláceas, e que tem por typo a cuscuta.
\section{Cusina}
\begin{itemize}
\item {Grp. gram.:f.}
\end{itemize}
\begin{itemize}
\item {Utilização:Ant.}
\end{itemize}
\begin{itemize}
\item {Proveniência:(Fr. \textunderscore cousine\textunderscore )}
\end{itemize}
O mesmo que \textunderscore prima\textunderscore ^1.
\section{Cusparada}
\begin{itemize}
\item {Grp. gram.:f.}
\end{itemize}
\begin{itemize}
\item {Utilização:Bras}
\end{itemize}
Grande porção de cuspo.
\section{Cuspária}
\begin{itemize}
\item {Grp. gram.:f.}
\end{itemize}
Gênero de plantas rutáceas.
\section{Cuspidato}
\begin{itemize}
\item {Grp. gram.:adj.}
\end{itemize}
\begin{itemize}
\item {Proveniência:(Lat. \textunderscore cuspidatus\textunderscore )}
\end{itemize}
Que termina em cúspide.
\section{Cúspide}
\begin{itemize}
\item {Grp. gram.:f.}
\end{itemize}
\begin{itemize}
\item {Utilização:Bot.}
\end{itemize}
\begin{itemize}
\item {Proveniência:(Lat. \textunderscore cuspis\textunderscore )}
\end{itemize}
Extremidade aguda.
Extremidade aguda e rija das fôlhas de algumas plantas.
A parte mais elevada das plantas.
Tridente de Neptuno.
Ferrão das abelhas, do lacrau, etc.
\section{Cuspideira}
\begin{itemize}
\item {Grp. gram.:f.}
\end{itemize}
\begin{itemize}
\item {Grp. gram.:F.  e  adj.}
\end{itemize}
\begin{itemize}
\item {Proveniência:(De \textunderscore cuspir\textunderscore )}
\end{itemize}
Vaso, em que se cospe; escarrador.
Diz-se, na África, da cobra venenosa, cujo nome, geralmente conhecido, é \textunderscore naja\textunderscore .
\section{Cuspidela}
\begin{itemize}
\item {Grp. gram.:f.}
\end{itemize}
O mesmo que \textunderscore cuspidura\textunderscore .
\section{Cuspídia}
\begin{itemize}
\item {Grp. gram.:f.}
\end{itemize}
Gênero de plantas synanthereas.
\section{Cuspidiforme}
\begin{itemize}
\item {Grp. gram.:adj.}
\end{itemize}
\begin{itemize}
\item {Proveniência:(Do lat. \textunderscore cuspis\textunderscore  + \textunderscore forma\textunderscore )}
\end{itemize}
Que tem a fórma de pequena ponta.
\section{Cuspidoiro}
\begin{itemize}
\item {Grp. gram.:m.}
\end{itemize}
\begin{itemize}
\item {Utilização:P. us.}
\end{itemize}
\begin{itemize}
\item {Proveniência:(De \textunderscore cuspir\textunderscore )}
\end{itemize}
Lugar, onde se cospe.
\section{Cuspidor}
\begin{itemize}
\item {Grp. gram.:m.}
\end{itemize}
\begin{itemize}
\item {Utilização:Ant.}
\end{itemize}
\begin{itemize}
\item {Proveniência:(De \textunderscore cuspir\textunderscore )}
\end{itemize}
Aquelle que cospe muito.
Cuspideira, escarrador.
\section{Cuspidouro}
\begin{itemize}
\item {Grp. gram.:m.}
\end{itemize}
\begin{itemize}
\item {Utilização:P. us.}
\end{itemize}
\begin{itemize}
\item {Proveniência:(De \textunderscore cuspir\textunderscore )}
\end{itemize}
Lugar, onde se cospe.
\section{Cuspidura}
\begin{itemize}
\item {Grp. gram.:f.}
\end{itemize}
Acto e effeito de cuspir.
Grande porção de cuspo.
\section{Cuspilhar}
\begin{itemize}
\item {Grp. gram.:v. t.  e  i.}
\end{itemize}
O mesmo que \textunderscore cuspinhar\textunderscore .
\section{Cuspinhador}
\begin{itemize}
\item {Grp. gram.:m.}
\end{itemize}
Aquelle que cuspinha.
\section{Cuspinhadura}
\begin{itemize}
\item {Grp. gram.:f.}
\end{itemize}
Acto e effeito de cuspinhar.
\section{Cuspinhar}
\begin{itemize}
\item {Grp. gram.:v. i.}
\end{itemize}
\begin{itemize}
\item {Proveniência:(De \textunderscore cuspinho\textunderscore )}
\end{itemize}
Cuspir frequentemente, em pequena quantidade.
\section{Cuspinheira}
\begin{itemize}
\item {Grp. gram.:f.}
\end{itemize}
\begin{itemize}
\item {Proveniência:(De \textunderscore cuspinho\textunderscore )}
\end{itemize}
Grande porção de cuspo.
Acto de cuspir frequentemente.
\section{Cuspinho}
\begin{itemize}
\item {Grp. gram.:m.}
\end{itemize}
Cuspo.
Pequena porção de cuspo.
\section{Cuspir}
\begin{itemize}
\item {Grp. gram.:v. t.}
\end{itemize}
\begin{itemize}
\item {Grp. gram.:V. i.}
\end{itemize}
\begin{itemize}
\item {Proveniência:(Do lat. \textunderscore conspuere\textunderscore )}
\end{itemize}
Expellir da boca.
Vomitar, lançar.
Dirigir contra alguém.
Lançar em rosto: \textunderscore cuspir afrontas\textunderscore .
Arremessar.
Lançar de si.
Fazer resair.
Salivar, deitar cuspo para fóra da boca.
Dirigir ultraje.
\section{Cuspo}
\begin{itemize}
\item {Grp. gram.:m.}
\end{itemize}
\begin{itemize}
\item {Proveniência:(De \textunderscore cuspir\textunderscore )}
\end{itemize}
Humor, segregado por várias glândulas da boca.
Saliva.
\section{Cusso}
\begin{itemize}
\item {Grp. gram.:m.}
\end{itemize}
Planta rosácea medicinal, (\textunderscore brayera anthelmenthica\textunderscore , Kunth.).
\section{Custa}
\begin{itemize}
\item {Grp. gram.:f.}
\end{itemize}
\begin{itemize}
\item {Grp. gram.:Pl.}
\end{itemize}
\begin{itemize}
\item {Proveniência:(De \textunderscore custar\textunderscore )}
\end{itemize}
Despesa.
Custo.
Trabalho.
Despesas, feitas em processo judicial.
\section{Custagem}
\begin{itemize}
\item {Grp. gram.:f.}
\end{itemize}
\begin{itemize}
\item {Utilização:Ant.}
\end{itemize}
O mesmo que \textunderscore custo\textunderscore .
\section{Custar}
\begin{itemize}
\item {Grp. gram.:v. t.}
\end{itemize}
\begin{itemize}
\item {Grp. gram.:V. i.}
\end{itemize}
Sêr adquirido pelo preço de: \textunderscore custar 30 reis\textunderscore .
Valer, têr o valor de.
Causar a despesa de: \textunderscore esta casa custou-me muito dinheiro\textunderscore .
Causar.
Sêr obtido a trôco de.
Sêr diffícil: \textunderscore custa muito aturar-te\textunderscore .
Sêr penoso: \textunderscore custa muito perder um filho\textunderscore .
(B. lat. \textunderscore custare\textunderscore )
\section{Custeamento}
\begin{itemize}
\item {Grp. gram.:m.}
\end{itemize}
Acto ou effeito de custear.
Conjunto de despesas.
Relação dellas.
\section{Custear}
\begin{itemize}
\item {Grp. gram.:v. t.}
\end{itemize}
\begin{itemize}
\item {Proveniência:(De \textunderscore custo\textunderscore )}
\end{itemize}
Prover á despesa feita com: \textunderscore custear a lavoira\textunderscore .
\section{Custeio}
\begin{itemize}
\item {Grp. gram.:m.}
\end{itemize}
O mesmo que \textunderscore custeamento\textunderscore .
\section{Custo}
\begin{itemize}
\item {Grp. gram.:m.}
\end{itemize}
\begin{itemize}
\item {Utilização:Fig.}
\end{itemize}
\begin{itemize}
\item {Grp. gram.:Loc. adv.}
\end{itemize}
\begin{itemize}
\item {Proveniência:(De \textunderscore custar\textunderscore )}
\end{itemize}
Quantia, que se paga por uma coisa.
Valor em dinheiro.
Difficuldade.
\textunderscore A custo\textunderscore , difficilmente.
\section{Custódia}
\begin{itemize}
\item {Grp. gram.:f.}
\end{itemize}
\begin{itemize}
\item {Utilização:Fig.}
\end{itemize}
\begin{itemize}
\item {Proveniência:(Lat. \textunderscore custodia\textunderscore )}
\end{itemize}
Lugar, onde se guarda alguém ou alguma coisa com segurança.
Acto de guardar.
Detenção.
Protecção.
Residência de frade custódio.
Objecto de oiro ou prata com duas lâminas circulares de crystal, entre as quaes se colloca a hóstia consagrada, para se expor á adoração dos fiéis.
\section{Custodiar}
\begin{itemize}
\item {Grp. gram.:v. t.}
\end{itemize}
Têr em custódia.
Guardar.
\section{Custódio}
\begin{itemize}
\item {Grp. gram.:adj.}
\end{itemize}
\begin{itemize}
\item {Grp. gram.:M.}
\end{itemize}
\begin{itemize}
\item {Utilização:T. de Viana}
\end{itemize}
\begin{itemize}
\item {Proveniência:(Lat. \textunderscore custos\textunderscore )}
\end{itemize}
Que guarda, que protege: \textunderscore o meu anjo custódio\textunderscore .
Frade franciscano, que substituía o provincial na ausência dêste.
Criança, que ainda não está baptizada.
\section{Custosamente}
\begin{itemize}
\item {Grp. gram.:adv.}
\end{itemize}
De modo custoso.
Com difficuldade.
\section{Custoso}
\begin{itemize}
\item {Grp. gram.:adj.}
\end{itemize}
\begin{itemize}
\item {Proveniência:(De \textunderscore custar\textunderscore )}
\end{itemize}
Que custa muito dinheiro.
Diffícil.
Trabalhoso.
\section{Custura}
\begin{itemize}
\item {Grp. gram.:f.}
\end{itemize}
\begin{itemize}
\item {Utilização:Ant.}
\end{itemize}
\begin{itemize}
\item {Proveniência:(De \textunderscore custo\textunderscore )}
\end{itemize}
Difficuldade, embaraço. Cf. \textunderscore Peregrinação\textunderscore , XLVI.
\section{Cuta}
\begin{itemize}
\item {Grp. gram.:adj. f.}
\end{itemize}
Diz-se da língua escocesa.
\section{Cutâneo}
\begin{itemize}
\item {Grp. gram.:adj.}
\end{itemize}
Relativo á cútis.
\section{Cutarda}
\begin{itemize}
\item {Grp. gram.:f.}
\end{itemize}
Gênero de plantas, cujas flôres são monopétalas.
\section{Cute}
\begin{itemize}
\item {Grp. gram.:f.}
\end{itemize}
O mesmo ou melhor que \textunderscore cútis\textunderscore . Cf. Filinto, XV, 300.
\section{Cutela}
\begin{itemize}
\item {Grp. gram.:f.}
\end{itemize}
\begin{itemize}
\item {Proveniência:(De \textunderscore cutelo\textunderscore )}
\end{itemize}
Espécie de cutelo pequeno, sem peta, em uso entre os podadores do Doiro.
\section{Cutelaria}
\begin{itemize}
\item {Grp. gram.:f.}
\end{itemize}
O mesmo ou melhor que \textunderscore cutilaria\textunderscore .
\section{Cuteleiro}
\begin{itemize}
\item {Grp. gram.:m.}
\end{itemize}
O mesmo ou melhor que \textunderscore cutileiro\textunderscore .
\section{Cutella}
\begin{itemize}
\item {Grp. gram.:f.}
\end{itemize}
\begin{itemize}
\item {Proveniência:(De \textunderscore cutello\textunderscore )}
\end{itemize}
Espécie de cutello pequeno, sem peta, em uso entre os podadores do Doiro.
\section{Cutellaria}
\begin{itemize}
\item {Grp. gram.:f.}
\end{itemize}
O mesmo ou melhor que \textunderscore cutillaria\textunderscore .
\section{Cutelleiro}
\begin{itemize}
\item {Grp. gram.:m.}
\end{itemize}
O mesmo ou melhor que \textunderscore cutilleiro\textunderscore .
\section{Cutello}
\begin{itemize}
\item {Grp. gram.:m.}
\end{itemize}
\begin{itemize}
\item {Utilização:Fig.}
\end{itemize}
\begin{itemize}
\item {Grp. gram.:Pl.}
\end{itemize}
\begin{itemize}
\item {Proveniência:(Lat. \textunderscore cultellus\textunderscore )}
\end{itemize}
Instrumento cortante, semi-circular, com o gume na parte convexa, e que servia antigamente nas decapitações, e é hoje utensílio especial de cortadores e correeiros.
Podôa com peta.
Caramanchão.
Violência.
Pequenas velas que, num navio, servem de supplemento a outras.
Pennas da asa do falcão, tambem conhecidas por \textunderscore cunha\textunderscore .
\section{Cutelo}
\begin{itemize}
\item {Grp. gram.:m.}
\end{itemize}
\begin{itemize}
\item {Utilização:Fig.}
\end{itemize}
\begin{itemize}
\item {Grp. gram.:Pl.}
\end{itemize}
\begin{itemize}
\item {Proveniência:(Lat. \textunderscore cultellus\textunderscore )}
\end{itemize}
Instrumento cortante, semi-circular, com o gume na parte convexa, e que servia antigamente nas decapitações, e é hoje utensílio especial de cortadores e correeiros.
Podôa com peta.
Caramanchão.
Violência.
Pequenas velas que, num navio, servem de suplemento a outras.
Penas da asa do falcão, tambem conhecidas por \textunderscore cunha\textunderscore .
\section{Cutia}
\begin{itemize}
\item {Grp. gram.:f.}
\end{itemize}
\begin{itemize}
\item {Utilização:Bras}
\end{itemize}
Pequeno mammífero roedor.
(Corr. do tupi \textunderscore acuti\textunderscore . Cp. \textunderscore aguti\textunderscore )
\section{Cutia}
\begin{itemize}
\item {Grp. gram.:f.}
\end{itemize}
\begin{itemize}
\item {Utilização:Bras}
\end{itemize}
Espécie de madeira de construcção.
\section{Cutiara}
\begin{itemize}
\item {Grp. gram.:f.}
\end{itemize}
Espécie de cutia^1, pequena, vermelha e de cauda.
\section{Cutícola}
\begin{itemize}
\item {Grp. gram.:adj.}
\end{itemize}
\begin{itemize}
\item {Utilização:Hist. Nat.}
\end{itemize}
\begin{itemize}
\item {Proveniência:(Do lat. \textunderscore cutis\textunderscore  + \textunderscore colere\textunderscore )}
\end{itemize}
Parasito, que vive debaixo da pelle de outro animal.
\section{Cutícula}
\begin{itemize}
\item {Grp. gram.:f.}
\end{itemize}
\begin{itemize}
\item {Proveniência:(Lat. \textunderscore cuticula\textunderscore )}
\end{itemize}
Pellícula.
A flôr da pelle.
\section{Cuticular}
\begin{itemize}
\item {Grp. gram.:adj.}
\end{itemize}
Relativo á cutícula ou á cútis.
\section{Cutículo}
\begin{itemize}
\item {Grp. gram.:m.}
\end{itemize}
Invólucro simples ou complexo do corpo de um animal.
(Cp. \textunderscore cutícola\textunderscore )
\section{Cuticuloso}
\begin{itemize}
\item {Grp. gram.:adj.}
\end{itemize}
Que tem a fórma de cutícula ou de pequena membrana.
\section{Cutidura}
\begin{itemize}
\item {Grp. gram.:f.}
\end{itemize}
\begin{itemize}
\item {Proveniência:(De \textunderscore cútis\textunderscore )}
\end{itemize}
Saliência carnosa, no bôrdo superior do casco do cavallo.
\section{Cutilada}
\begin{itemize}
\item {Grp. gram.:f.}
\end{itemize}
Golpe de cutelo, sabre, espada, etc.
(Por \textunderscore cutelada\textunderscore , de \textunderscore cutelo\textunderscore )
\section{Cutilão}
\begin{itemize}
\item {Grp. gram.:m.}
\end{itemize}
\begin{itemize}
\item {Utilização:Des.}
\end{itemize}
Cutelo grande.
\section{Cutilaria}
\begin{itemize}
\item {Grp. gram.:f.}
\end{itemize}
Arte ou oficina de cutileiro.
\section{Cutileiro}
\begin{itemize}
\item {Grp. gram.:m.}
\end{itemize}
Aquele, que faz ou vende instrumentos cortantes, como facas, tesoiras, etc.
(Por \textunderscore cuteleiro\textunderscore , de \textunderscore cutelo\textunderscore )
\section{Cutiliquê}
\begin{itemize}
\item {Grp. gram.:m.}
\end{itemize}
O mesmo que \textunderscore quotiliquê\textunderscore . Cf. Filinto, IV, 31; V, 18 e 111; VII, 55; Camillo, \textunderscore Sebenta\textunderscore , IX, 8.
\section{Cutillada}
\begin{itemize}
\item {Grp. gram.:f.}
\end{itemize}
Golpe de cutello, sabre, espada, etc.
(Por \textunderscore cutellada\textunderscore , de \textunderscore cutello\textunderscore )
\section{Cutillão}
\begin{itemize}
\item {Grp. gram.:m.}
\end{itemize}
\begin{itemize}
\item {Utilização:Des.}
\end{itemize}
Cutello grande.
\section{Cutillaria}
\begin{itemize}
\item {Grp. gram.:f.}
\end{itemize}
Arte ou officina de cutilleiro.
\section{Cutilleiro}
\begin{itemize}
\item {Grp. gram.:m.}
\end{itemize}
Aquelle, que faz ou vende instrumentos cortantes, como facas, tesoiras, etc.
(Por \textunderscore cutelleiro\textunderscore , de \textunderscore cutello\textunderscore )
\section{Cutipiribá}
\begin{itemize}
\item {Grp. gram.:m.}
\end{itemize}
\begin{itemize}
\item {Utilização:Bras}
\end{itemize}
Árvore, espécie de guteira.
\section{Cútis}
\begin{itemize}
\item {Grp. gram.:f.}
\end{itemize}
\begin{itemize}
\item {Proveniência:(Lat. \textunderscore cutis\textunderscore , que se liga ao gr. \textunderscore kutos\textunderscore )}
\end{itemize}
Pelle das pessôas.
Epiderme; tez.
\section{Cutitiribá}
\begin{itemize}
\item {Grp. gram.:m.}
\end{itemize}
\begin{itemize}
\item {Utilização:Bras. do N}
\end{itemize}
Árvore sapotácea.
\section{Cutléria}
\begin{itemize}
\item {Grp. gram.:f.}
\end{itemize}
Gênero de plantas cryptogâmicas.
\section{Cutrelha}
\begin{itemize}
\item {fónica:trê}
\end{itemize}
\begin{itemize}
\item {Grp. gram.:f. Loc. adv.}
\end{itemize}
\begin{itemize}
\item {Utilização:Prov.}
\end{itemize}
\textunderscore De cutrelha\textunderscore , de escantilhão, aos tombos. (Colhido na Bairrada)
\section{Cutúbea}
\begin{itemize}
\item {Grp. gram.:f.}
\end{itemize}
\begin{itemize}
\item {Utilização:Bras}
\end{itemize}
Espécie de genciana.
\section{Cutuca}
\begin{itemize}
\item {Grp. gram.:m.}
\end{itemize}
\begin{itemize}
\item {Utilização:Bras}
\end{itemize}
Espécie de sellim de arções altos.
\section{Cutucão}
\begin{itemize}
\item {Grp. gram.:m.}
\end{itemize}
\begin{itemize}
\item {Utilização:Bras}
\end{itemize}
\begin{itemize}
\item {Proveniência:(De \textunderscore cutucar\textunderscore )}
\end{itemize}
Facada; cutilada.
\section{Cutucar}
\begin{itemize}
\item {Grp. gram.:v. t.}
\end{itemize}
\begin{itemize}
\item {Utilização:Bras}
\end{itemize}
(V.catocar)
\section{Cuuraquão}
\begin{itemize}
\item {Grp. gram.:m.}
\end{itemize}
O mesmo que \textunderscore pau-brasil\textunderscore .
\section{Cuva}
\begin{itemize}
\item {Grp. gram.:f.}
\end{itemize}
O mesmo que \textunderscore barbo\textunderscore .
\section{Cuviera}
\begin{itemize}
\item {Grp. gram.:f.}
\end{itemize}
Gênero de plantas rubiáceas.
\section{Cuviéria}
\begin{itemize}
\item {Grp. gram.:f.}
\end{itemize}
\begin{itemize}
\item {Proveniência:(De \textunderscore Cuvier\textunderscore , n. p.)}
\end{itemize}
Gênero de molluscos, de concha cylíndrica.
\section{Cuvilheira}
\begin{itemize}
\item {Grp. gram.:f.}
\end{itemize}
\begin{itemize}
\item {Utilização:Pop.}
\end{itemize}
\begin{itemize}
\item {Proveniência:(Do lat. \textunderscore cubicularia\textunderscore )}
\end{itemize}
Mulher, que servia particularmente pessôa nobre.
Camareira.
Alcoviteira.
\section{Cuvu}
\begin{itemize}
\item {Grp. gram.:m.}
\end{itemize}
\begin{itemize}
\item {Utilização:Bras}
\end{itemize}
O mesmo que \textunderscore juquiá\textunderscore .
\section{Cuxá}
\begin{itemize}
\item {Grp. gram.:m.}
\end{itemize}
\begin{itemize}
\item {Utilização:Bras}
\end{itemize}
Espécie de comida, feita com fôlhas de vinagreira e quiabo.
\section{Cuxiú}
\begin{itemize}
\item {Grp. gram.:m.}
\end{itemize}
Espécie de macaco das regiões do Amazonas.
\section{Cuzapada}
\begin{itemize}
\item {Grp. gram.:f.}
\end{itemize}
\begin{itemize}
\item {Utilização:Pop.}
\end{itemize}
Acto de bater com as nádegas no chão, por effeito de quéda.
(Cp. \textunderscore culapada\textunderscore )
\section{Cyameia}
\begin{itemize}
\item {Grp. gram.:f.}
\end{itemize}
\begin{itemize}
\item {Proveniência:(Lat. \textunderscore cyamea\textunderscore )}
\end{itemize}
Pedra preciosa, hoje desconhecida e que parece semelhava uma fava.
\section{Cýamo}
\begin{itemize}
\item {Grp. gram.:m.}
\end{itemize}
\begin{itemize}
\item {Proveniência:(Gr. \textunderscore kuamos\textunderscore )}
\end{itemize}
Animal parasito, que vive sôbre a baleia.
O mesmo que \textunderscore colocásia\textunderscore  ou \textunderscore fava\textunderscore  do Egypto.
\section{Ciamóide}
\begin{itemize}
\item {Grp. gram.:adj.}
\end{itemize}
\begin{itemize}
\item {Proveniência:(Do gr. \textunderscore kuamos\textunderscore  + \textunderscore eidos\textunderscore )}
\end{itemize}
Semelhante á fava.
\section{Cianado}
\begin{itemize}
\item {Grp. gram.:adj.}
\end{itemize}
\begin{itemize}
\item {Proveniência:(Do gr. \textunderscore kuanos\textunderscore )}
\end{itemize}
Que tem ácido prússico.
\section{Cianato}
\begin{itemize}
\item {Grp. gram.:m.}
\end{itemize}
\begin{itemize}
\item {Proveniência:(De \textunderscore cíano\textunderscore )}
\end{itemize}
Sal, produzido pelo ácido ciánico com uma base.
\section{Cianeicolo}
\begin{itemize}
\item {Grp. gram.:adj.}
\end{itemize}
\begin{itemize}
\item {Proveniência:(Do lat. \textunderscore cyaneus\textunderscore  + \textunderscore collum\textunderscore )}
\end{itemize}
Que tem o pescoço azul.
\section{Cianela}
\begin{itemize}
\item {Grp. gram.:f.}
\end{itemize}
\begin{itemize}
\item {Proveniência:(Do gr. \textunderscore kuanos\textunderscore , azul)}
\end{itemize}
Gênero do plantas liliáceas.
\section{Cianeto}
\begin{itemize}
\item {fónica:nê}
\end{itemize}
\begin{itemize}
\item {Grp. gram.:m.}
\end{itemize}
\begin{itemize}
\item {Proveniência:(De \textunderscore cíano\textunderscore )}
\end{itemize}
O mesmo ou melhor que \textunderscore cianureto\textunderscore . Cf. \textunderscore Inquér. Indust.\textunderscore , P. II, l. 1.^o, 245.
Combinação do cianogênio com um corpo simples.
\section{Ciânico}
\begin{itemize}
\item {Grp. gram.:adj.}
\end{itemize}
\begin{itemize}
\item {Proveniência:(De \textunderscore cíano\textunderscore )}
\end{itemize}
Diz-se de um ácido, que é o segundo grau da oxidação do cianogênio.
\section{Cianícorneo}
\begin{itemize}
\item {Grp. gram.:adj.}
\end{itemize}
\begin{itemize}
\item {Utilização:Zool.}
\end{itemize}
\begin{itemize}
\item {Proveniência:(De \textunderscore cíano\textunderscore  + \textunderscore córneo\textunderscore )}
\end{itemize}
Que tem pontas ou antenas azues.
\section{Cianido}
\begin{itemize}
\item {Grp. gram.:m.}
\end{itemize}
\begin{itemize}
\item {Proveniência:(De \textunderscore cíano\textunderscore )}
\end{itemize}
Combinação do cianogênio com um metalóide ou com um metal electro-negativo.
\section{Cianídrico}
\begin{itemize}
\item {Grp. gram.:adj.}
\end{itemize}
\begin{itemize}
\item {Proveniência:(Do gr. \textunderscore kuanos\textunderscore  + \textunderscore hudrikos\textunderscore )}
\end{itemize}
Diz-se de um ácido, resultante da combinação do hidrogênio com o cianogênio, e que antigamente se designava por \textunderscore ácido prússico\textunderscore .
\section{Cianina}
\begin{itemize}
\item {Grp. gram.:f.}
\end{itemize}
\begin{itemize}
\item {Proveniência:(Do gr. \textunderscore kuanos\textunderscore )}
\end{itemize}
Substância còrante, extraida das violetas e de outras plantas.
\section{Cianípede}
\begin{itemize}
\item {Grp. gram.:adj.}
\end{itemize}
\begin{itemize}
\item {Utilização:Zool.}
\end{itemize}
\begin{itemize}
\item {Proveniência:(Do lat. \textunderscore cyaneus\textunderscore  + \textunderscore pes\textunderscore )}
\end{itemize}
Que tem patas azues.
\section{Cianirrostro}
\begin{itemize}
\item {Grp. gram.:adj.}
\end{itemize}
\begin{itemize}
\item {Utilização:Zool.}
\end{itemize}
\begin{itemize}
\item {Proveniência:(Do lat. \textunderscore cyaneus\textunderscore  + \textunderscore rostrum\textunderscore )}
\end{itemize}
Que tem bico azul.
\section{Cianismo}
\begin{itemize}
\item {Grp. gram.:m.}
\end{itemize}
\begin{itemize}
\item {Proveniência:(Do gr. \textunderscore kuanos\textunderscore , azul)}
\end{itemize}
Intensidade do azul celeste, medida pelo cianómetro.
\section{Cianite}
\begin{itemize}
\item {Grp. gram.:f.}
\end{itemize}
\begin{itemize}
\item {Proveniência:(De \textunderscore cíano\textunderscore )}
\end{itemize}
Silicato natural de alumina, o qual contém uma pequena quantidade de ferro.
\section{Cíano}
\begin{itemize}
\item {Grp. gram.:m.}
\end{itemize}
\begin{itemize}
\item {Proveniência:(Gr. \textunderscore kuanos\textunderscore , azul)}
\end{itemize}
O mesmo que cianogênio.
\section{Cianocarpo}
\begin{itemize}
\item {Grp. gram.:adj.}
\end{itemize}
\begin{itemize}
\item {Utilização:Bot.}
\end{itemize}
\begin{itemize}
\item {Proveniência:(Do gr. \textunderscore kuanos\textunderscore  + \textunderscore karpos\textunderscore )}
\end{itemize}
Que tem frutos azulados.
\section{Cianocéfalo}
\begin{itemize}
\item {Grp. gram.:adj.}
\end{itemize}
\begin{itemize}
\item {Utilização:Zool.}
\end{itemize}
\begin{itemize}
\item {Proveniência:(Do gr. \textunderscore kuanos\textunderscore  + \textunderscore kephale\textunderscore )}
\end{itemize}
Que tem cabeça azul.
\section{Cianodermia}
\begin{itemize}
\item {Grp. gram.:f.}
\end{itemize}
\begin{itemize}
\item {Proveniência:(Do gr. \textunderscore kuanos\textunderscore  + \textunderscore derma\textunderscore )}
\end{itemize}
Coloração azul da pele.
\section{Cianodérmico}
\begin{itemize}
\item {Grp. gram.:adj.}
\end{itemize}
Relativo á cianòdermia.
\section{Cianoférrico}
\begin{itemize}
\item {Grp. gram.:adj.}
\end{itemize}
\begin{itemize}
\item {Proveniência:(De \textunderscore cianoferro\textunderscore )}
\end{itemize}
Diz-se de um ácido, que é a combinação do ácido cianidrico e do cianeto de ferro.
\section{Cianoferro}
\begin{itemize}
\item {Grp. gram.:m.}
\end{itemize}
\begin{itemize}
\item {Proveniência:(De \textunderscore cíano\textunderscore  + \textunderscore ferro\textunderscore )}
\end{itemize}
Combinação do ferro com o cianogênio.
\section{Cianogastro}
\begin{itemize}
\item {Grp. gram.:adj.}
\end{itemize}
\begin{itemize}
\item {Utilização:Zool.}
\end{itemize}
\begin{itemize}
\item {Proveniência:(Do gr. \textunderscore kuanos\textunderscore  + \textunderscore gaster\textunderscore )}
\end{itemize}
Que tem o ventre azul.
\section{Cianogênio}
\begin{itemize}
\item {Grp. gram.:m.}
\end{itemize}
\begin{itemize}
\item {Proveniência:(Do gr. \textunderscore kuanos\textunderscore  + \textunderscore genos\textunderscore )}
\end{itemize}
Gás incolor, composto de azoto e carbono, mas que entra em combinações, como se fôsse um corpo simples.
\section{Cianógino}
\begin{itemize}
\item {Grp. gram.:adj.}
\end{itemize}
\begin{itemize}
\item {Utilização:Bot.}
\end{itemize}
\begin{itemize}
\item {Proveniência:(Do gr. \textunderscore kuanos\textunderscore  + \textunderscore gune\textunderscore )}
\end{itemize}
Que tem o pistilo azul.
\section{Cianóide}
\begin{itemize}
\item {Grp. gram.:adj.}
\end{itemize}
\begin{itemize}
\item {Proveniência:(Do gr. \textunderscore kuanos\textunderscore  + \textunderscore eidos\textunderscore )}
\end{itemize}
Semelhante á centáurea.
\section{Cianometria}
\begin{itemize}
\item {Grp. gram.:f.}
\end{itemize}
\begin{itemize}
\item {Proveniência:(De \textunderscore cianómetro\textunderscore )}
\end{itemize}
Processo, para medir a intensidade do azul celeste.
\section{Cianómetro}
\begin{itemize}
\item {Grp. gram.:m.}
\end{itemize}
\begin{itemize}
\item {Proveniência:(Do gr. \textunderscore kuanos\textunderscore  + \textunderscore metron\textunderscore )}
\end{itemize}
Instrumento, para medir a intensidade do azul do ar.
\section{Cianofósforo}
\begin{itemize}
\item {Grp. gram.:m.}
\end{itemize}
\begin{itemize}
\item {Proveniência:(De \textunderscore cíano\textunderscore  + \textunderscore phósphoro\textunderscore )}
\end{itemize}
Substância explosiva, produzida pela acção do fósforo sôbre o cianeto de mercúrio.
\section{Cianoftalmo}
\begin{itemize}
\item {Grp. gram.:adj.}
\end{itemize}
\begin{itemize}
\item {Proveniência:(Do gr. \textunderscore kuanos\textunderscore  + \textunderscore ophtalmos\textunderscore )}
\end{itemize}
Que tem os olhos azues.
\section{Cianopatia}
\begin{itemize}
\item {Grp. gram.:f.}
\end{itemize}
\begin{itemize}
\item {Proveniência:(Do gr. \textunderscore kuanos\textunderscore  + \textunderscore pathos\textunderscore )}
\end{itemize}
O mesmo que \textunderscore cianose\textunderscore .
\section{Cianopático}
\begin{itemize}
\item {Grp. gram.:adj.}
\end{itemize}
Relativo á cianopatia.
\section{Cianopígio}
\begin{itemize}
\item {Grp. gram.:adj.}
\end{itemize}
\begin{itemize}
\item {Utilização:Zool.}
\end{itemize}
\begin{itemize}
\item {Proveniência:(Do gr. \textunderscore kuanos\textunderscore  + \textunderscore pugè\textunderscore )}
\end{itemize}
Que tem a rabadilha azul.
\section{Cianopirro}
\begin{itemize}
\item {Grp. gram.:adj.}
\end{itemize}
\begin{itemize}
\item {Proveniência:(Do gr. \textunderscore kuanos\textunderscore  + \textunderscore purrhos\textunderscore )}
\end{itemize}
Que é azul e roxo.
\section{Cianópode}
\begin{itemize}
\item {Grp. gram.:adj.}
\end{itemize}
O mesmo que \textunderscore cianípede\textunderscore .
\section{Cianopotássico}
\begin{itemize}
\item {Grp. gram.:adj.}
\end{itemize}
\begin{itemize}
\item {Proveniência:(De \textunderscore cíano\textunderscore  + \textunderscore potássio\textunderscore )}
\end{itemize}
Composto de cianogênio e potássio.
\section{Cianóptero}
\begin{itemize}
\item {Grp. gram.:adj.}
\end{itemize}
\begin{itemize}
\item {Utilização:Zool.}
\end{itemize}
\begin{itemize}
\item {Proveniência:(Do gr. \textunderscore kuanos\textunderscore  + \textunderscore pteron\textunderscore )}
\end{itemize}
Que tem asas ou barbatanas azues.
\section{Cianose}
\begin{itemize}
\item {Grp. gram.:f.}
\end{itemize}
\begin{itemize}
\item {Proveniência:(Gr. \textunderscore kuanosis\textunderscore )}
\end{itemize}
Coloração azul, algumas vezes escura ou lívida, da pele, em virtude de embaraço na circulação.
Espécie de cristal.
\section{Cianoso}
\begin{itemize}
\item {Grp. gram.:adj.}
\end{itemize}
(V.ciânico)
\section{Cianótico}
\begin{itemize}
\item {Grp. gram.:adj.}
\end{itemize}
Relativo á cianose.
\section{Cianurato}
\begin{itemize}
\item {Grp. gram.:m.}
\end{itemize}
\begin{itemize}
\item {Proveniência:(De \textunderscore cianúrico\textunderscore )}
\end{itemize}
Sal, produzido pela combinação do ácido cianúrico com uma base.
\section{Cianureto}
\begin{itemize}
\item {fónica:nurê}
\end{itemize}
\begin{itemize}
\item {Grp. gram.:m.}
\end{itemize}
\begin{itemize}
\item {Proveniência:(De \textunderscore cíano\textunderscore )}
\end{itemize}
Combinação do cianogênio com um corpo simples.
\section{Cianúrico}
\begin{itemize}
\item {Grp. gram.:adj.}
\end{itemize}
\begin{itemize}
\item {Proveniência:(Do gr. \textunderscore kuanos\textunderscore  + \textunderscore ouron\textunderscore )}
\end{itemize}
Diz-se de um ácido, descoberto nos productos de destilação do ácido úrico.
\section{Ciatiforme}
\begin{itemize}
\item {Grp. gram.:adj.}
\end{itemize}
\begin{itemize}
\item {Proveniência:(De \textunderscore cíato\textunderscore  + \textunderscore fórma\textunderscore )}
\end{itemize}
Que tem fórma de ciato.
\section{Cíato}
\begin{itemize}
\item {Grp. gram.:m.}
\end{itemize}
\begin{itemize}
\item {Utilização:Ant.}
\end{itemize}
\begin{itemize}
\item {Proveniência:(Lat. \textunderscore cyathus\textunderscore )}
\end{itemize}
Vaso, com asa, com que se deitava vinho nos copos dos convidados.
\section{Ciatoide}
\begin{itemize}
\item {Grp. gram.:adj.}
\end{itemize}
\begin{itemize}
\item {Proveniência:(Do gr. \textunderscore kuathos\textunderscore  + \textunderscore eidos\textunderscore )}
\end{itemize}
O mesmo que \textunderscore ciatiforme\textunderscore .
\section{Ciátula}
\begin{itemize}
\item {Grp. gram.:f.}
\end{itemize}
Gênero de plantas amarantáceas.
(Cp. \textunderscore cíato\textunderscore )
\section{Cibeia}
\begin{itemize}
\item {Grp. gram.:f.}
\end{itemize}
\begin{itemize}
\item {Proveniência:(Lat. \textunderscore cybaea\textunderscore )}
\end{itemize}
Espécie de navio grande, usado na antiguidade, para carga ou transporte.
\section{Cicadáceas}
\begin{itemize}
\item {Grp. gram.:f. pl.}
\end{itemize}
O mesmo ou melhor que \textunderscore cicádeas\textunderscore .
\section{Cicádeas}
\begin{itemize}
\item {Grp. gram.:f. pl.}
\end{itemize}
\begin{itemize}
\item {Proveniência:(Do gr. \textunderscore kukas\textunderscore  + \textunderscore eidos\textunderscore )}
\end{itemize}
Família de plantas, que têm por tipo o \textunderscore cicas\textunderscore .
\section{Cicas}
\begin{itemize}
\item {Grp. gram.:m.}
\end{itemize}
\begin{itemize}
\item {Proveniência:(Gr. \textunderscore kukas\textunderscore )}
\end{itemize}
Espécie de palmeira, com caracteres de árvore conífera.
\section{Cíclade}
\begin{itemize}
\item {Grp. gram.:f.}
\end{itemize}
\begin{itemize}
\item {Proveniência:(Lat. \textunderscore cyclas\textunderscore )}
\end{itemize}
Sáia larga e luxuosa, usada dantes por mulheres romanas.
\section{Ciclame}
\begin{itemize}
\item {Grp. gram.:m.}
\end{itemize}
\begin{itemize}
\item {Utilização:Gal}
\end{itemize}
\begin{itemize}
\item {Proveniência:(Fr. \textunderscore cyclame\textunderscore )}
\end{itemize}
(V.ciclamino)
\section{Ciclamino}
\begin{itemize}
\item {Grp. gram.:m.}
\end{itemize}
\begin{itemize}
\item {Proveniência:(Gr. \textunderscore kuklaminos\textunderscore )}
\end{itemize}
Planta primulácea, o mesmo que \textunderscore artanita\textunderscore .
\section{Ciclantera}
\begin{itemize}
\item {Grp. gram.:f.}
\end{itemize}
Gênero de plantas cucurbitáceas.
\section{Cicleta}
\begin{itemize}
\item {fónica:clê}
\end{itemize}
\begin{itemize}
\item {Grp. gram.:f.}
\end{itemize}
\begin{itemize}
\item {Proveniência:(De \textunderscore ciclo\textunderscore )}
\end{itemize}
Espécie de velocípede pouco usado.
\section{Cíclico}
\begin{itemize}
\item {Grp. gram.:adj.}
\end{itemize}
\begin{itemize}
\item {Grp. gram.:M.}
\end{itemize}
\begin{itemize}
\item {Grp. gram.:Pl.}
\end{itemize}
Relativo a ciclo.
Poéta, que punha em verso a história dos tempos heróicos da Grécia.
Família de insectos coleópteros.
\section{Ciclismo}
\begin{itemize}
\item {Grp. gram.:m.}
\end{itemize}
\begin{itemize}
\item {Proveniência:(De \textunderscore ciclo\textunderscore )}
\end{itemize}
O mesmo que \textunderscore velocipedia\textunderscore .
\section{Ciclista}
\begin{itemize}
\item {Grp. gram.:m.  e  f.}
\end{itemize}
\begin{itemize}
\item {Proveniência:(De \textunderscore ciclo\textunderscore )}
\end{itemize}
O mesmo que \textunderscore velocipedista\textunderscore .
\section{Ciclite}
\begin{itemize}
\item {Grp. gram.:f.}
\end{itemize}
\begin{itemize}
\item {Utilização:Med.}
\end{itemize}
\begin{itemize}
\item {Proveniência:(Do gr. \textunderscore kuklos\textunderscore )}
\end{itemize}
Inflamação do corpo ciliar do globo ocular.
\section{Ciclizar}
\begin{itemize}
\item {Grp. gram.:v. i.}
\end{itemize}
\begin{itemize}
\item {Utilização:Neol.}
\end{itemize}
\begin{itemize}
\item {Proveniência:(De \textunderscore ciclo\textunderscore )}
\end{itemize}
Andar em velocipede.
\section{Ciclo}
\begin{itemize}
\item {Grp. gram.:m.}
\end{itemize}
\begin{itemize}
\item {Utilização:Astron.}
\end{itemize}
\begin{itemize}
\item {Utilização:Bot.}
\end{itemize}
\begin{itemize}
\item {Proveniência:(Gr. \textunderscore kuklos\textunderscore )}
\end{itemize}
Período ou revolução de um certo número de anos, ao fim dos quaes devem repetir-se pela mesma ordem os fenómenos astronómicos.
Linha espiral entre duas fôlhas, que se correspondem exactamente sôbre um caule ou ramo.
Conjunto de poemas, em que se celebram feitos heróicos dos tempos fabulosos da Grécia.
\section{Ciclofilo}
\begin{itemize}
\item {Grp. gram.:adj.}
\end{itemize}
\begin{itemize}
\item {Utilização:Bot.}
\end{itemize}
\begin{itemize}
\item {Proveniência:(Do gr. \textunderscore kuklos\textunderscore  + \textunderscore phullon\textunderscore )}
\end{itemize}
Que tem fôlhas orbiculares.
\section{Cicloide}
\begin{itemize}
\item {Grp. gram.:f.}
\end{itemize}
\begin{itemize}
\item {Utilização:Mathem.}
\end{itemize}
\begin{itemize}
\item {Grp. gram.:M. pl.}
\end{itemize}
\begin{itemize}
\item {Proveniência:(Do gr. \textunderscore kuklos\textunderscore  + \textunderscore eidos\textunderscore )}
\end{itemize}
Linha curva, produzida pela revolução completa de um ponto, pertencente a um círculo, que gira sôbre um plano.
O mesmo que [[malacopterígios|malacopterígio]].
\section{Ciclometria}
\begin{itemize}
\item {Grp. gram.:f.}
\end{itemize}
\begin{itemize}
\item {Proveniência:(De \textunderscore ciclómetro\textunderscore )}
\end{itemize}
Arte de medir círculos ou ciclos.
\section{Ciclométrico}
\begin{itemize}
\item {Grp. gram.:adj.}
\end{itemize}
Relativo á ciclometria.
\section{Ciclómetro}
\begin{itemize}
\item {Grp. gram.:m.}
\end{itemize}
\begin{itemize}
\item {Proveniência:(Do gr. \textunderscore kuklos\textunderscore  + \textunderscore metron\textunderscore )}
\end{itemize}
Instrumento, para medir círculos ou ciclos.
\section{Ciclone}
\begin{itemize}
\item {Grp. gram.:m.}
\end{itemize}
\begin{itemize}
\item {Proveniência:(Do gr. \textunderscore kuklos\textunderscore )}
\end{itemize}
Tempestade, que redemoinha; torvelinho de vento devastador.
\section{Ciclónico}
\begin{itemize}
\item {Grp. gram.:adj.}
\end{itemize}
Relativo a ciclone.
\section{Ciclónio}
\begin{itemize}
\item {Grp. gram.:m.}
\end{itemize}
O mesmo ou melhor que \textunderscore ciclone\textunderscore .
\section{Ciclopas}
\begin{itemize}
\item {Grp. gram.:m. pl.}
\end{itemize}
O mesmo que \textunderscore ciclopes\textunderscore .
\section{Ciclopedia}
\begin{itemize}
\item {Grp. gram.:f.}
\end{itemize}
O mesmo que \textunderscore enciclopedia\textunderscore .
\section{Ciclópeo}
\begin{itemize}
\item {Grp. gram.:adj.}
\end{itemize}
Relativo aos ciclopes.
\section{Ciclopes}
\begin{itemize}
\item {Grp. gram.:m. pl.}
\end{itemize}
\begin{itemize}
\item {Utilização:Zool.}
\end{itemize}
\begin{itemize}
\item {Proveniência:(Do gr. \textunderscore kuklos\textunderscore  + \textunderscore ops\textunderscore )}
\end{itemize}
Gigantes fabulosos, com um só ôlho na testa.
Crustáceos, que vivem nas águas estagnadas.
\section{Ciclopia}
\begin{itemize}
\item {Grp. gram.:f.}
\end{itemize}
\begin{itemize}
\item {Proveniência:(De \textunderscore ciclopes\textunderscore )}
\end{itemize}
Monstruosidade, determinada pela confusão de dois olhos num.
\section{Ciclopiano}
\begin{itemize}
\item {Grp. gram.:adj.}
\end{itemize}
Que tem ciclopia.
\section{Ciclópico}
\begin{itemize}
\item {Grp. gram.:adj.}
\end{itemize}
Relativo aos ciclopes.
\section{Ciclópteros}
\begin{itemize}
\item {Grp. gram.:m. pl.}
\end{itemize}
\begin{itemize}
\item {Proveniência:(Do gr. \textunderscore kuklos\textunderscore  + \textunderscore pteron\textunderscore )}
\end{itemize}
Gênero de peixes, com barbatanas arredondadas.
\section{Ciclose}
\begin{itemize}
\item {Grp. gram.:f.}
\end{itemize}
\begin{itemize}
\item {Utilização:Bot.}
\end{itemize}
\begin{itemize}
\item {Proveniência:(Gr. \textunderscore kuklosis\textunderscore , de \textunderscore kuklein\textunderscore )}
\end{itemize}
Movimento giratório da seiva, em algumas plantas.
\section{Ciclóstomas}
\begin{itemize}
\item {Grp. gram.:m. pl.}
\end{itemize}
O mesmo que \textunderscore ciclóstomos\textunderscore .
\section{Ciclóstomos}
\begin{itemize}
\item {Grp. gram.:m. pl.}
\end{itemize}
\begin{itemize}
\item {Utilização:Zool.}
\end{itemize}
\begin{itemize}
\item {Proveniência:(Do gr. \textunderscore kuklos\textunderscore  + \textunderscore stoma\textunderscore )}
\end{itemize}
Classe de vertebrados inferiores, de bôca redonda.
Divisão da classe dos peixes, caracterizada por pelle molle, sem escamas, uma só barbatana e bôca redonda.
\section{Ciclótomo}
\begin{itemize}
\item {Grp. gram.:m.}
\end{itemize}
\begin{itemize}
\item {Proveniência:(Do gr. \textunderscore kuklos\textunderscore  + \textunderscore tome\textunderscore )}
\end{itemize}
Antigo instrumento cirúrgico, com que se fixava o globo do ôlho, na operação da cataracta.
\section{Ciclozoário}
\begin{itemize}
\item {Grp. gram.:m.}
\end{itemize}
\begin{itemize}
\item {Utilização:Zool.}
\end{itemize}
\begin{itemize}
\item {Proveniência:(Do gr. \textunderscore kuklos\textunderscore  + \textunderscore zoon\textunderscore )}
\end{itemize}
Animal, de configuração circular.
\section{Cicnoide}
\begin{itemize}
\item {Grp. gram.:adj.}
\end{itemize}
\begin{itemize}
\item {Proveniência:(Do gr. \textunderscore kuknos\textunderscore  + \textunderscore eidos\textunderscore )}
\end{itemize}
Semelhante ao cisne.
\section{Cídaro}
\begin{itemize}
\item {Grp. gram.:m.}
\end{itemize}
\begin{itemize}
\item {Proveniência:(Lat. \textunderscore cydarum\textunderscore )}
\end{itemize}
Espécie de antigo navio de transporte.
\section{Cidónia}
\begin{itemize}
\item {Grp. gram.:f.}
\end{itemize}
\begin{itemize}
\item {Proveniência:(De \textunderscore Cydon\textunderscore , n. p. de uma cidade de Creta)}
\end{itemize}
Gênero de árvores pomáceas.
\section{Ciesiologia}
\begin{itemize}
\item {Grp. gram.:f.}
\end{itemize}
\begin{itemize}
\item {Proveniência:(Do gr. \textunderscore kuesis\textunderscore  + \textunderscore logos\textunderscore )}
\end{itemize}
Teoria ou história dos fenómenos da gravidez.
\section{Cilindráceo}
\begin{itemize}
\item {Grp. gram.:adj.}
\end{itemize}
\begin{itemize}
\item {Utilização:Bot.}
\end{itemize}
Quási cilíndrico, (falando-se do certos órgãos vegetaes).
\section{Cilindragem}
\begin{itemize}
\item {Grp. gram.:f.}
\end{itemize}
\begin{itemize}
\item {Proveniência:(De \textunderscore cilindrar\textunderscore )}
\end{itemize}
Pressão de um cilindro sôbre os corpos que se lhe sotopõem.
Efeito dessa pressão.
\section{Cilindramento}
\begin{itemize}
\item {Grp. gram.:m.}
\end{itemize}
O mesmo que \textunderscore cilindragem\textunderscore .
\section{Cilindrar}
\begin{itemize}
\item {Grp. gram.:v. t.}
\end{itemize}
\begin{itemize}
\item {Proveniência:(De \textunderscore cilindro\textunderscore )}
\end{itemize}
Submeter á pressão de um cilindro: \textunderscore cilindrar o empedramento de uma estrada\textunderscore .
\section{Cilindricamente}
\begin{itemize}
\item {Grp. gram.:adv.}
\end{itemize}
\begin{itemize}
\item {Proveniência:(De \textunderscore cilindrico\textunderscore )}
\end{itemize}
Em fórma de cilindro.
\section{Cilindricidade}
\begin{itemize}
\item {Grp. gram.:f.}
\end{itemize}
Fórma daquilo que é cilíndrico.
\section{Cilíndrico}
\begin{itemize}
\item {Grp. gram.:adj.}
\end{itemize}
Que tem fórma de cilindro.
\section{Cilindricórneo}
\begin{itemize}
\item {Grp. gram.:adj.}
\end{itemize}
\begin{itemize}
\item {Utilização:Zool.}
\end{itemize}
\begin{itemize}
\item {Proveniência:(De \textunderscore cilindro\textunderscore  + \textunderscore córneo\textunderscore )}
\end{itemize}
Que tem os córnos ou as anteras cilíndricas.
\section{Cilindrifloro}
\begin{itemize}
\item {Grp. gram.:adj.}
\end{itemize}
\begin{itemize}
\item {Utilização:Bot.}
\end{itemize}
\begin{itemize}
\item {Proveniência:(De \textunderscore cilindro\textunderscore  + \textunderscore flôr\textunderscore )}
\end{itemize}
Que tem flôres cilíndricas.
\section{Cilindriforme}
\begin{itemize}
\item {Grp. gram.:adj.}
\end{itemize}
\begin{itemize}
\item {Proveniência:(De \textunderscore cilindro\textunderscore  + \textunderscore fórma\textunderscore )}
\end{itemize}
Que tem fórma de cilindro.
\section{Cilindrímetro}
\begin{itemize}
\item {Grp. gram.:m.}
\end{itemize}
\begin{itemize}
\item {Proveniência:(Do gr. \textunderscore kulindros\textunderscore  + \textunderscore metron\textunderscore )}
\end{itemize}
Instrumento, para fabricar com exactidão as rodas de relojoaria.
\section{Cilindrite}
\begin{itemize}
\item {Grp. gram.:f.}
\end{itemize}
Variedade de pedra roliça, da côr do cobre.
(Cp. \textunderscore cilindro\textunderscore )
\section{Cilindro}
\begin{itemize}
\item {Grp. gram.:m.}
\end{itemize}
\begin{itemize}
\item {Utilização:Náut.}
\end{itemize}
\begin{itemize}
\item {Proveniência:(Lat. \textunderscore cylindrus\textunderscore )}
\end{itemize}
Corpo alongado e roliço, de diâmetro igual em todo o seu comprimento.
Superfície, descrita por uma linha recta, movendo-se parallelamente a si própria, em volta de uma circumferência.
Recipiente, em que se move o êmbolo das máquinas de vapor.
Vaso de metal, em que se metem brasas, e que se immerge na água das banheiras, para a aquecer.
Peça redonda, que gira em volta de um eixo, e em que se gorne o cabo dos lemes.
\section{Cilindrocarpo}
\begin{itemize}
\item {Grp. gram.:adj.}
\end{itemize}
\begin{itemize}
\item {Utilização:Bot.}
\end{itemize}
\begin{itemize}
\item {Proveniência:(Do gr. \textunderscore kulindros\textunderscore  + \textunderscore karpos\textunderscore )}
\end{itemize}
Que tem frutos cilíndricos.
\section{Cilindrocefalia}
\begin{itemize}
\item {Grp. gram.:f.}
\end{itemize}
Estado ou qualidade de cilindrocéphalo.
\section{Cilindrocéphalo}
\begin{itemize}
\item {Grp. gram.:adj.}
\end{itemize}
\begin{itemize}
\item {Utilização:Anat.}
\end{itemize}
Que tem o crânio cilindricamente desenvolvido, na direcção antero-posterior. Cf. E. Burnay, \textunderscore Craniologia\textunderscore , 203.
\section{Cilindróide}
\begin{itemize}
\item {Grp. gram.:adj.}
\end{itemize}
\begin{itemize}
\item {Grp. gram.:M.}
\end{itemize}
\begin{itemize}
\item {Utilização:Geom.}
\end{itemize}
\begin{itemize}
\item {Proveniência:(Do gr. \textunderscore kulindros\textunderscore  + \textunderscore eidos\textunderscore )}
\end{itemize}
O mesmo que \textunderscore cilindriforme\textunderscore .
Superfície cilíndrica, com uma base diferente do círculo.
\section{Cyamóide}
\begin{itemize}
\item {Grp. gram.:adj.}
\end{itemize}
\begin{itemize}
\item {Proveniência:(Do gr. \textunderscore kuamos\textunderscore  + \textunderscore eidos\textunderscore )}
\end{itemize}
Semelhante á fava.
\section{Cyanado}
\begin{itemize}
\item {Grp. gram.:adj.}
\end{itemize}
\begin{itemize}
\item {Proveniência:(Do gr. \textunderscore kuanos\textunderscore )}
\end{itemize}
Que tem ácido prússico.
\section{Cyanato}
\begin{itemize}
\item {Grp. gram.:m.}
\end{itemize}
\begin{itemize}
\item {Proveniência:(De \textunderscore cýano\textunderscore )}
\end{itemize}
Sal, produzido pelo ácido cyánico com uma base.
\section{Cyaneicollo}
\begin{itemize}
\item {Grp. gram.:adj.}
\end{itemize}
\begin{itemize}
\item {Proveniência:(Do lat. \textunderscore cyaneus\textunderscore  + \textunderscore collum\textunderscore )}
\end{itemize}
Que tem o pescoço azul.
\section{Cyanela}
\begin{itemize}
\item {Grp. gram.:f.}
\end{itemize}
\begin{itemize}
\item {Proveniência:(Do gr. \textunderscore kuanos\textunderscore , azul)}
\end{itemize}
Gênero do plantas liliáceas.
\section{Cyaneto}
\begin{itemize}
\item {fónica:nê}
\end{itemize}
\begin{itemize}
\item {Grp. gram.:m.}
\end{itemize}
O mesmo ou melhor que \textunderscore cyanureto\textunderscore . Cf. \textunderscore Inquér. Indust.\textunderscore , P. II, l. 1.^o, 245.
\section{Cyanhýdrico}
\begin{itemize}
\item {Grp. gram.:adj.}
\end{itemize}
\begin{itemize}
\item {Proveniência:(Do gr. \textunderscore kuanos\textunderscore  + \textunderscore hudrikos\textunderscore )}
\end{itemize}
Diz-se de um ácido, resultante da combinação do hydrogênio com o cyanogênio, e que antigamente se designava por \textunderscore ácido prússico\textunderscore .
\section{Cyânico}
\begin{itemize}
\item {Grp. gram.:adj.}
\end{itemize}
\begin{itemize}
\item {Proveniência:(De \textunderscore cýano\textunderscore )}
\end{itemize}
Diz-se de um ácido, que é o segundo grau da oxydação do cyanogênio.
\section{Cyanicórneo}
\begin{itemize}
\item {Grp. gram.:adj.}
\end{itemize}
\begin{itemize}
\item {Utilização:Zool.}
\end{itemize}
\begin{itemize}
\item {Proveniência:(De \textunderscore cýano\textunderscore  + \textunderscore córneo\textunderscore )}
\end{itemize}
Que tem pontas ou antennas azues.
\section{Cyanido}
\begin{itemize}
\item {Grp. gram.:m.}
\end{itemize}
\begin{itemize}
\item {Proveniência:(De \textunderscore cýano\textunderscore )}
\end{itemize}
Combinação do cyanogênio com um metallóide ou com um metal electro-negativo.
\section{Cyanina}
\begin{itemize}
\item {Grp. gram.:f.}
\end{itemize}
\begin{itemize}
\item {Proveniência:(Do gr. \textunderscore kuanos\textunderscore )}
\end{itemize}
Substância còrante, extrahida das violetas e de outras plantas.
\section{Cyanípede}
\begin{itemize}
\item {Grp. gram.:adj.}
\end{itemize}
\begin{itemize}
\item {Utilização:Zool.}
\end{itemize}
\begin{itemize}
\item {Proveniência:(Do lat. \textunderscore cyaneus\textunderscore  + \textunderscore pes\textunderscore )}
\end{itemize}
Que tem patas azues.
\section{Cyanirostro}
\begin{itemize}
\item {fónica:rós}
\end{itemize}
\begin{itemize}
\item {Grp. gram.:adj.}
\end{itemize}
\begin{itemize}
\item {Utilização:Zool.}
\end{itemize}
\begin{itemize}
\item {Proveniência:(Do lat. \textunderscore cyaneus\textunderscore  + \textunderscore rostrum\textunderscore )}
\end{itemize}
Que tem bico azul.
\section{Cyanismo}
\begin{itemize}
\item {Grp. gram.:m.}
\end{itemize}
\begin{itemize}
\item {Proveniência:(Do gr. \textunderscore kuanos\textunderscore , azul)}
\end{itemize}
Intensidade do azul celeste, medida pelo cyanómetro.
\section{Cyanite}
\begin{itemize}
\item {Grp. gram.:f.}
\end{itemize}
\begin{itemize}
\item {Proveniência:(De \textunderscore cýano\textunderscore )}
\end{itemize}
Silicato natural de alumina, o qual contém uma pequena quantidade de ferro.
\section{Cýano}
\begin{itemize}
\item {Grp. gram.:m.}
\end{itemize}
\begin{itemize}
\item {Proveniência:(Gr. \textunderscore kuanos\textunderscore , azul)}
\end{itemize}
O mesmo que cyanogênio.
\section{Cyanocarpo}
\begin{itemize}
\item {Grp. gram.:adj.}
\end{itemize}
\begin{itemize}
\item {Utilização:Bot.}
\end{itemize}
\begin{itemize}
\item {Proveniência:(Do gr. \textunderscore kuanos\textunderscore  + \textunderscore karpos\textunderscore )}
\end{itemize}
Que tem frutos azulados.
\section{Cyanocéphalo}
\begin{itemize}
\item {Grp. gram.:adj.}
\end{itemize}
\begin{itemize}
\item {Utilização:Zool.}
\end{itemize}
\begin{itemize}
\item {Proveniência:(Do gr. \textunderscore kuanos\textunderscore  + \textunderscore kephale\textunderscore )}
\end{itemize}
Que tem cabeça azul.
\section{Cyanodermia}
\begin{itemize}
\item {Grp. gram.:f.}
\end{itemize}
\begin{itemize}
\item {Proveniência:(Do gr. \textunderscore kuanos\textunderscore  + \textunderscore derma\textunderscore )}
\end{itemize}
Coloração azul da pelle.
\section{Cyanodérmico}
\begin{itemize}
\item {Grp. gram.:adj.}
\end{itemize}
Relativo á cyanòdermia.
\section{Cyanoférrico}
\begin{itemize}
\item {Grp. gram.:adj.}
\end{itemize}
\begin{itemize}
\item {Proveniência:(De \textunderscore cyanoferro\textunderscore )}
\end{itemize}
Diz-se de um ácido, que é a combinação do ácido cyanhýdrico e do cyaneto de ferro.
\section{Cyanoferro}
\begin{itemize}
\item {Grp. gram.:m.}
\end{itemize}
\begin{itemize}
\item {Proveniência:(De \textunderscore cýano\textunderscore  + \textunderscore ferro\textunderscore )}
\end{itemize}
Combinação do ferro com o cyanogênio.
\section{Cyanogastro}
\begin{itemize}
\item {Grp. gram.:adj.}
\end{itemize}
\begin{itemize}
\item {Utilização:Zool.}
\end{itemize}
\begin{itemize}
\item {Proveniência:(Do gr. \textunderscore kuanos\textunderscore  + \textunderscore gaster\textunderscore )}
\end{itemize}
Que tem o ventre azul.
\section{Cyanogênio}
\begin{itemize}
\item {Grp. gram.:m.}
\end{itemize}
\begin{itemize}
\item {Proveniência:(Do gr. \textunderscore kuanos\textunderscore  + \textunderscore genos\textunderscore )}
\end{itemize}
Gás incolor, composto de azoto e carbono, mas que entra em combinações, como se fôsse um corpo simples.
\section{Cyanógyno}
\begin{itemize}
\item {Grp. gram.:adj.}
\end{itemize}
\begin{itemize}
\item {Utilização:Bot.}
\end{itemize}
\begin{itemize}
\item {Proveniência:(Do gr. \textunderscore kuanos\textunderscore  + \textunderscore gune\textunderscore )}
\end{itemize}
Que tem o pistillo azul.
\section{Cyanóide}
\begin{itemize}
\item {Grp. gram.:adj.}
\end{itemize}
\begin{itemize}
\item {Proveniência:(Do gr. \textunderscore kuanos\textunderscore  + \textunderscore eidos\textunderscore )}
\end{itemize}
Semelhante á centáurea.
\section{Cyanometria}
\begin{itemize}
\item {Grp. gram.:f.}
\end{itemize}
\begin{itemize}
\item {Proveniência:(De \textunderscore cyanómetro\textunderscore )}
\end{itemize}
Processo, para medir a intensidade do azul celeste.
\section{Cyanómetro}
\begin{itemize}
\item {Grp. gram.:m.}
\end{itemize}
\begin{itemize}
\item {Proveniência:(Do gr. \textunderscore kuanos\textunderscore  + \textunderscore metron\textunderscore )}
\end{itemize}
Instrumento, para medir a intensidade do azul do ar.
\section{Cyanopathia}
\begin{itemize}
\item {Grp. gram.:f.}
\end{itemize}
\begin{itemize}
\item {Proveniência:(Do gr. \textunderscore kuanos\textunderscore  + \textunderscore pathos\textunderscore )}
\end{itemize}
O mesmo que \textunderscore cyanose\textunderscore .
\section{Cyanopáthico}
\begin{itemize}
\item {Grp. gram.:adj.}
\end{itemize}
Relativo á cyanopathia.
\section{Cyanophósphoro}
\begin{itemize}
\item {Grp. gram.:m.}
\end{itemize}
\begin{itemize}
\item {Proveniência:(De \textunderscore cýano\textunderscore  + \textunderscore phósphoro\textunderscore )}
\end{itemize}
Substância explosiva, produzida pela acção do phósphoro sôbre o cyaneto de mercúrio.
\section{Cyanophthalmo}
\begin{itemize}
\item {Grp. gram.:adj.}
\end{itemize}
\begin{itemize}
\item {Proveniência:(Do gr. \textunderscore kuanos\textunderscore  + \textunderscore ophtalmos\textunderscore )}
\end{itemize}
Que tem os olhos azues.
\section{Cyanópode}
\begin{itemize}
\item {Grp. gram.:adj.}
\end{itemize}
O mesmo que \textunderscore cyanípede\textunderscore .
\section{Cyanopotássico}
\begin{itemize}
\item {Grp. gram.:adj.}
\end{itemize}
\begin{itemize}
\item {Proveniência:(De \textunderscore cýano\textunderscore  + \textunderscore potássio\textunderscore )}
\end{itemize}
Composto de cyanogênio e potássio.
\section{Cyanóptero}
\begin{itemize}
\item {Grp. gram.:adj.}
\end{itemize}
\begin{itemize}
\item {Utilização:Zool.}
\end{itemize}
\begin{itemize}
\item {Proveniência:(Do gr. \textunderscore kuanos\textunderscore  + \textunderscore pteron\textunderscore )}
\end{itemize}
Que tem asas ou barbatanas azues.
\section{Cyanopýgio}
\begin{itemize}
\item {Grp. gram.:adj.}
\end{itemize}
\begin{itemize}
\item {Utilização:Zool.}
\end{itemize}
\begin{itemize}
\item {Proveniência:(Do gr. \textunderscore kuanos\textunderscore  + \textunderscore pugè\textunderscore )}
\end{itemize}
Que tem a rabadilha azul.
\section{Cyanopyrrho}
\begin{itemize}
\item {Grp. gram.:adj.}
\end{itemize}
\begin{itemize}
\item {Proveniência:(Do gr. \textunderscore kuanos\textunderscore  + \textunderscore purrhos\textunderscore )}
\end{itemize}
Que é azul e roxo.
\section{Cyanose}
\begin{itemize}
\item {Grp. gram.:f.}
\end{itemize}
\begin{itemize}
\item {Proveniência:(Gr. \textunderscore kuanosis\textunderscore )}
\end{itemize}
Coloração azul, algumas vezes escura ou lívida, da pelle, em virtude de embaraço na circulação.
Espécie de cristal.
\section{Cyanoso}
\begin{itemize}
\item {Grp. gram.:adj.}
\end{itemize}
(V.cyânico)
\section{Cyanótico}
\begin{itemize}
\item {Grp. gram.:adj.}
\end{itemize}
Relativo á cyanose.
\section{Cyanurato}
\begin{itemize}
\item {Grp. gram.:m.}
\end{itemize}
\begin{itemize}
\item {Proveniência:(De \textunderscore cyanúrico\textunderscore )}
\end{itemize}
Sal, produzido pela combinação do ácido cyanúrico com uma base.
\section{Cyanureto}
\begin{itemize}
\item {fónica:nurê}
\end{itemize}
\begin{itemize}
\item {Grp. gram.:m.}
\end{itemize}
\begin{itemize}
\item {Proveniência:(De \textunderscore cýano\textunderscore )}
\end{itemize}
Combinação do cyanogênio com um corpo simples.
\section{Cyanúrico}
\begin{itemize}
\item {Grp. gram.:adj.}
\end{itemize}
\begin{itemize}
\item {Proveniência:(Do gr. \textunderscore kuanos\textunderscore  + \textunderscore ouron\textunderscore )}
\end{itemize}
Diz-se de um ácido, descoberto nos productos de destillação do ácido úrico.
\section{Cyathiforme}
\begin{itemize}
\item {Grp. gram.:adj.}
\end{itemize}
\begin{itemize}
\item {Proveniência:(De \textunderscore cýatho\textunderscore  + \textunderscore fórma\textunderscore )}
\end{itemize}
Que tem fórma de cyatho.
\section{Cýatho}
\begin{itemize}
\item {Grp. gram.:m.}
\end{itemize}
\begin{itemize}
\item {Utilização:Ant.}
\end{itemize}
\begin{itemize}
\item {Proveniência:(Lat. \textunderscore cyathus\textunderscore )}
\end{itemize}
Vaso, com asa, com que se deitava vinho nos copos dos convidados.
\section{Cyathoide}
\begin{itemize}
\item {Grp. gram.:adj.}
\end{itemize}
\begin{itemize}
\item {Proveniência:(Do gr. \textunderscore kuathos\textunderscore  + \textunderscore eidos\textunderscore )}
\end{itemize}
O mesmo que \textunderscore cyathiforme\textunderscore .
\section{Cyáthula}
\begin{itemize}
\item {Grp. gram.:f.}
\end{itemize}
Gênero de plantas amarantáceas.
(Cp. \textunderscore cýatho\textunderscore )
\section{Cybeia}
\begin{itemize}
\item {Grp. gram.:f.}
\end{itemize}
\begin{itemize}
\item {Proveniência:(Lat. \textunderscore cybaea\textunderscore )}
\end{itemize}
Espécie de navio grande, usado na antiguidade, para carga ou transporte.
\section{Cycadáceas}
\begin{itemize}
\item {Grp. gram.:f. pl.}
\end{itemize}
O mesmo ou melhor que \textunderscore cycádeas\textunderscore .
\section{Cycádeas}
\begin{itemize}
\item {Grp. gram.:f. pl.}
\end{itemize}
\begin{itemize}
\item {Proveniência:(Do gr. \textunderscore kukas\textunderscore  + \textunderscore eidos\textunderscore )}
\end{itemize}
Família de plantas, que têm por typo o \textunderscore cycas\textunderscore .
\section{Cycas}
\begin{itemize}
\item {Grp. gram.:m.}
\end{itemize}
\begin{itemize}
\item {Proveniência:(Gr. \textunderscore kukas\textunderscore )}
\end{itemize}
Espécie de palmeira, com caracteres de árvore conífera.
\section{Cýclade}
\begin{itemize}
\item {Grp. gram.:f.}
\end{itemize}
\begin{itemize}
\item {Proveniência:(Lat. \textunderscore cyclas\textunderscore )}
\end{itemize}
Sáia larga e luxuosa, usada dantes por mulheres romanas.
\section{Cyclame}
\begin{itemize}
\item {Grp. gram.:m.}
\end{itemize}
\begin{itemize}
\item {Utilização:Gal}
\end{itemize}
\begin{itemize}
\item {Proveniência:(Fr. \textunderscore cyclame\textunderscore )}
\end{itemize}
(V.cyclamino)
\section{Cyclamino}
\begin{itemize}
\item {Grp. gram.:m.}
\end{itemize}
\begin{itemize}
\item {Proveniência:(Gr. \textunderscore kuklaminos\textunderscore )}
\end{itemize}
Planta primulácea, o mesmo que \textunderscore arthanita\textunderscore .
\section{Cyclanthera}
\begin{itemize}
\item {Grp. gram.:f.}
\end{itemize}
Gênero de plantas cucurbitáceas.
\section{Cycleta}
\begin{itemize}
\item {fónica:clê}
\end{itemize}
\begin{itemize}
\item {Grp. gram.:f.}
\end{itemize}
\begin{itemize}
\item {Proveniência:(De \textunderscore cyclo\textunderscore )}
\end{itemize}
Espécie de velocípede pouco usado.
\section{Cýclico}
\begin{itemize}
\item {Grp. gram.:adj.}
\end{itemize}
\begin{itemize}
\item {Grp. gram.:M.}
\end{itemize}
\begin{itemize}
\item {Grp. gram.:Pl.}
\end{itemize}
Relativo a cyclo.
Poéta, que punha em verso a história dos tempos heróicos da Grécia.
Família de insectos coleópteros.
\section{Cyclismo}
\begin{itemize}
\item {Grp. gram.:m.}
\end{itemize}
\begin{itemize}
\item {Proveniência:(De \textunderscore cyclo\textunderscore )}
\end{itemize}
O mesmo que \textunderscore velocipedia\textunderscore .
\section{Cyclista}
\begin{itemize}
\item {Grp. gram.:m.  e  f.}
\end{itemize}
\begin{itemize}
\item {Proveniência:(De \textunderscore cyclo\textunderscore )}
\end{itemize}
O mesmo que \textunderscore velocipedista\textunderscore .
\section{Cyclite}
\begin{itemize}
\item {Grp. gram.:f.}
\end{itemize}
\begin{itemize}
\item {Utilização:Med.}
\end{itemize}
\begin{itemize}
\item {Proveniência:(Do gr. \textunderscore kuklos\textunderscore )}
\end{itemize}
Inflammação do corpo ciliar do globo ocular.
\section{Cyclo}
\begin{itemize}
\item {Grp. gram.:m.}
\end{itemize}
\begin{itemize}
\item {Utilização:Astron.}
\end{itemize}
\begin{itemize}
\item {Utilização:Bot.}
\end{itemize}
\begin{itemize}
\item {Proveniência:(Gr. \textunderscore kuklos\textunderscore )}
\end{itemize}
Período ou revolução de um certo número de annos, ao fim dos quaes devem repetir-se pela mesma ordem os phenómenos astronómicos.
Linha espiral entre duas fôlhas, que se correspondem exactamente sôbre um caule ou ramo.
Conjunto de poemas, em que se celebram feitos heróicos dos tempos fabulosos da Grécia.
\section{Cycloidal}
\begin{itemize}
\item {Grp. gram.:adj.}
\end{itemize}
Relativo á cycloide.
\section{Cycloide}
\begin{itemize}
\item {Grp. gram.:f.}
\end{itemize}
\begin{itemize}
\item {Utilização:Mathem.}
\end{itemize}
\begin{itemize}
\item {Grp. gram.:M. pl.}
\end{itemize}
\begin{itemize}
\item {Proveniência:(Do gr. \textunderscore kuklos\textunderscore  + \textunderscore eidos\textunderscore )}
\end{itemize}
Linha curva, produzida pela revolução completa de um ponto, pertencente a um círculo, que gira sôbre um plano.
O mesmo que [[malacopterýgios|malacopterýgio]].
\section{Cyclometria}
\begin{itemize}
\item {Grp. gram.:f.}
\end{itemize}
\begin{itemize}
\item {Proveniência:(De \textunderscore cyclómetro\textunderscore )}
\end{itemize}
Arte de medir círculos ou cyclos.
\section{Cyclométrico}
\begin{itemize}
\item {Grp. gram.:adj.}
\end{itemize}
Relativo á cyclometria.
\section{Cyclómetro}
\begin{itemize}
\item {Grp. gram.:m.}
\end{itemize}
\begin{itemize}
\item {Proveniência:(Do gr. \textunderscore kuklos\textunderscore  + \textunderscore metron\textunderscore )}
\end{itemize}
Instrumento, para medir círculos ou cyclos.
\section{Cyclone}
\begin{itemize}
\item {Grp. gram.:m.}
\end{itemize}
\begin{itemize}
\item {Proveniência:(Do gr. \textunderscore kuklos\textunderscore )}
\end{itemize}
Tempestade, que redemoinha; torvelinho de vento devastador.
\section{Cyclónico}
\begin{itemize}
\item {Grp. gram.:adj.}
\end{itemize}
Relativo a cyclone.
\section{Cyclónio}
\begin{itemize}
\item {Grp. gram.:m.}
\end{itemize}
O mesmo ou melhor que \textunderscore cyclone\textunderscore .
\section{Cyclopas}
\begin{itemize}
\item {Grp. gram.:m. pl.}
\end{itemize}
O mesmo que \textunderscore cyclopes\textunderscore .
\section{Cyclopedia}
\begin{itemize}
\item {Grp. gram.:f.}
\end{itemize}
O mesmo que \textunderscore encyclopedia\textunderscore .
\section{Cyclópeo}
\begin{itemize}
\item {Grp. gram.:adj.}
\end{itemize}
Relativo aos cyclopes.
\section{Cyclopes}
\begin{itemize}
\item {Grp. gram.:m. pl.}
\end{itemize}
\begin{itemize}
\item {Utilização:Zool.}
\end{itemize}
\begin{itemize}
\item {Proveniência:(Do gr. \textunderscore kuklos\textunderscore  + \textunderscore ops\textunderscore )}
\end{itemize}
Gigantes fabulosos, com um só ôlho na testa.
Crustáceos, que vivem nas águas estagnadas.
\section{Cyclophyllo}
\begin{itemize}
\item {Grp. gram.:adj.}
\end{itemize}
\begin{itemize}
\item {Utilização:Bot.}
\end{itemize}
\begin{itemize}
\item {Proveniência:(Do gr. \textunderscore kuklos\textunderscore  + \textunderscore phullon\textunderscore )}
\end{itemize}
Que tem fôlhas orbiculares.
\section{Cyclopia}
\begin{itemize}
\item {Grp. gram.:f.}
\end{itemize}
\begin{itemize}
\item {Proveniência:(De \textunderscore cyclopes\textunderscore )}
\end{itemize}
Monstruosidade, determinada pela confusão de dois olhos num.
\section{Cyclopiano}
\begin{itemize}
\item {Grp. gram.:adj.}
\end{itemize}
Que tem cyclopia.
\section{Cyclópico}
\begin{itemize}
\item {Grp. gram.:adj.}
\end{itemize}
Relativo aos cyclopes.
\section{Cyclópteros}
\begin{itemize}
\item {Grp. gram.:m. pl.}
\end{itemize}
\begin{itemize}
\item {Proveniência:(Do gr. \textunderscore kuklos\textunderscore  + \textunderscore pteron\textunderscore )}
\end{itemize}
Gênero de peixes, com barbatanas arredondadas.
\section{Cyclose}
\begin{itemize}
\item {Grp. gram.:f.}
\end{itemize}
\begin{itemize}
\item {Utilização:Bot.}
\end{itemize}
\begin{itemize}
\item {Proveniência:(Gr. \textunderscore kuklosis\textunderscore , de \textunderscore kuklein\textunderscore )}
\end{itemize}
Movimento giratório da seiva, em algumas plantas.
\section{Cyclóstomas}
\begin{itemize}
\item {Grp. gram.:m. pl.}
\end{itemize}
O mesmo que \textunderscore cyclóstomos\textunderscore .
\section{Cyclóstomos}
\begin{itemize}
\item {Grp. gram.:m. pl.}
\end{itemize}
\begin{itemize}
\item {Utilização:Zool.}
\end{itemize}
\begin{itemize}
\item {Proveniência:(Do gr. \textunderscore kuklos\textunderscore  + \textunderscore stoma\textunderscore )}
\end{itemize}
Classe de vertebrados inferiores, de bôca redonda.
Divisão da classe dos peixes, caracterizada por pelle molle, sem escamas, uma só barbatana e bôca redonda.
\section{Cyclótomo}
\begin{itemize}
\item {Grp. gram.:m.}
\end{itemize}
\begin{itemize}
\item {Proveniência:(Do gr. \textunderscore kuklos\textunderscore  + \textunderscore tome\textunderscore )}
\end{itemize}
Antigo instrumento cirúrgico, com que se fixava o globo do ôlho, na operação da cataracta.
\section{Cyclozoário}
\begin{itemize}
\item {Grp. gram.:m.}
\end{itemize}
\begin{itemize}
\item {Utilização:Zool.}
\end{itemize}
\begin{itemize}
\item {Proveniência:(Do gr. \textunderscore kuklos\textunderscore  + \textunderscore zoon\textunderscore )}
\end{itemize}
Animal, de configuração circular.
\section{Cycnoide}
\begin{itemize}
\item {Grp. gram.:adj.}
\end{itemize}
\begin{itemize}
\item {Proveniência:(Do gr. \textunderscore kuknos\textunderscore  + \textunderscore eidos\textunderscore )}
\end{itemize}
Semelhante ao cysne.
\section{Cýdaro}
\begin{itemize}
\item {Grp. gram.:m.}
\end{itemize}
\begin{itemize}
\item {Proveniência:(Lat. \textunderscore cydarum\textunderscore )}
\end{itemize}
Espécie de antigo navio de transporte.
\section{Cydónia}
\begin{itemize}
\item {Grp. gram.:f.}
\end{itemize}
\begin{itemize}
\item {Proveniência:(De \textunderscore Cydon\textunderscore , n. p. de uma cidade de Creta)}
\end{itemize}
Gênero de árvores pomáceas.
\section{Cyesiologia}
\begin{itemize}
\item {Grp. gram.:f.}
\end{itemize}
\begin{itemize}
\item {Proveniência:(Do gr. \textunderscore kuesis\textunderscore  + \textunderscore logos\textunderscore )}
\end{itemize}
Theoria ou história dos phenómenos da gravidez.
\section{Cylindráceo}
\begin{itemize}
\item {Grp. gram.:adj.}
\end{itemize}
\begin{itemize}
\item {Utilização:Bot.}
\end{itemize}
Quási cylíndrico, (falando-se do certos órgãos vegetaes).
\section{Cylindragem}
\begin{itemize}
\item {Grp. gram.:f.}
\end{itemize}
\begin{itemize}
\item {Proveniência:(De \textunderscore cylindrar\textunderscore )}
\end{itemize}
Pressão de um cylindro sôbre os corpos que se lhe sotopõem.
Effeito dessa pressão.
\section{Cylindramento}
\begin{itemize}
\item {Grp. gram.:m.}
\end{itemize}
O mesmo que \textunderscore cylindragem\textunderscore .
\section{Cylindrar}
\begin{itemize}
\item {Grp. gram.:v. t.}
\end{itemize}
\begin{itemize}
\item {Proveniência:(De \textunderscore cylindro\textunderscore )}
\end{itemize}
Submeter á pressão de um cylindro: \textunderscore cylindrar o empedramento de uma estrada\textunderscore .
\section{Cylindricamente}
\begin{itemize}
\item {Grp. gram.:adv.}
\end{itemize}
\begin{itemize}
\item {Proveniência:(De \textunderscore cylindrico\textunderscore )}
\end{itemize}
Em fórma de cylindro.
\section{Cylindricidade}
\begin{itemize}
\item {Grp. gram.:f.}
\end{itemize}
Fórma daquillo que é cylíndrico.
\section{Cylíndrico}
\begin{itemize}
\item {Grp. gram.:adj.}
\end{itemize}
Que tem fórma de cylindro.
\section{Cylindricórneo}
\begin{itemize}
\item {Grp. gram.:adj.}
\end{itemize}
\begin{itemize}
\item {Utilização:Zool.}
\end{itemize}
\begin{itemize}
\item {Proveniência:(De \textunderscore cylindro\textunderscore  + \textunderscore córneo\textunderscore )}
\end{itemize}
Que tem os córnos ou as anteras cylíndricas.
\section{Cylindrifloro}
\begin{itemize}
\item {Grp. gram.:adj.}
\end{itemize}
\begin{itemize}
\item {Utilização:Bot.}
\end{itemize}
\begin{itemize}
\item {Proveniência:(De \textunderscore cylindro\textunderscore  + \textunderscore flôr\textunderscore )}
\end{itemize}
Que tem flôres cylíndricas.
\section{Cylindriforme}
\begin{itemize}
\item {Grp. gram.:adj.}
\end{itemize}
\begin{itemize}
\item {Proveniência:(De \textunderscore cylindro\textunderscore  + \textunderscore fórma\textunderscore )}
\end{itemize}
Que tem fórma de cylindro.
\section{Cylindrímetro}
\begin{itemize}
\item {Grp. gram.:m.}
\end{itemize}
\begin{itemize}
\item {Proveniência:(Do gr. \textunderscore kulindros\textunderscore  + \textunderscore metron\textunderscore )}
\end{itemize}
Instrumento, para fabricar com exactidão as rodas de relojoaria.
\section{Cylindrite}
\begin{itemize}
\item {Grp. gram.:f.}
\end{itemize}
Variedade de pedra roliça, da côr do cobre.
(Cp. \textunderscore cylindro\textunderscore )
\section{Cylindro}
\begin{itemize}
\item {Grp. gram.:m.}
\end{itemize}
\begin{itemize}
\item {Utilização:Náut.}
\end{itemize}
\begin{itemize}
\item {Proveniência:(Lat. \textunderscore cylindrus\textunderscore )}
\end{itemize}
Corpo alongado e roliço, de diâmetro igual em todo o seu comprimento.
Superfície, descrita por uma linha recta, movendo-se parallelamente a si própria, em volta de uma circumferência.
Recipiente, em que se move o êmbolo das máquinas de vapor.
Vaso de metal, em que se metem brasas, e que se immerge na água das banheiras, para a aquecer.
Peça redonda, que gira em volta de um eixo, e em que se gorne o cabo dos lemes.
\section{Cylindrocarpo}
\begin{itemize}
\item {Grp. gram.:adj.}
\end{itemize}
\begin{itemize}
\item {Utilização:Bot.}
\end{itemize}
\begin{itemize}
\item {Proveniência:(Do gr. \textunderscore kulindros\textunderscore  + \textunderscore karpos\textunderscore )}
\end{itemize}
Que tem frutos cylíndricos.
\section{Cylindrocephalia}
\begin{itemize}
\item {Grp. gram.:f.}
\end{itemize}
Estado ou qualidade de cylindrocéphalo.
\section{Cylindrocéphalo}
\begin{itemize}
\item {Grp. gram.:adj.}
\end{itemize}
\begin{itemize}
\item {Utilização:Anat.}
\end{itemize}
Que tem o crânio cylindricamente desenvolvido, na direcção antero-posterior. Cf. E. Burnay, \textunderscore Craniologia\textunderscore , 203.
\section{Cylindróide}
\begin{itemize}
\item {Grp. gram.:adj.}
\end{itemize}
\begin{itemize}
\item {Grp. gram.:M.}
\end{itemize}
\begin{itemize}
\item {Utilização:Geom.}
\end{itemize}
\begin{itemize}
\item {Proveniência:(Do gr. \textunderscore kulindros\textunderscore  + \textunderscore eidos\textunderscore )}
\end{itemize}
O mesmo que \textunderscore cylindriforme\textunderscore .
Superfície cylíndrica, com uma base differente do círculo.
\section{Cifose}
\begin{itemize}
\item {Grp. gram.:f.}
\end{itemize}
\begin{itemize}
\item {Utilização:Anat.}
\end{itemize}
\begin{itemize}
\item {Grp. gram.:f.}
\end{itemize}
\begin{itemize}
\item {Utilização:Med.}
\end{itemize}
\begin{itemize}
\item {Proveniência:(Gr. \textunderscore kuphosis\textunderscore )}
\end{itemize}
Curvatura anómala da espinha dorsal para trás, isto é, formando convexidade posterior.
Curvatura anómala da columna vertebral para trás.
\section{Cifótico}
\begin{itemize}
\item {Grp. gram.:m.  e  adj.}
\end{itemize}
Doente de cifose.
\section{Cilindro-ogival}
\begin{itemize}
\item {Grp. gram.:adj.}
\end{itemize}
\begin{itemize}
\item {Proveniência:(De \textunderscore cilíndrico\textunderscore  + \textunderscore ogival\textunderscore )}
\end{itemize}
Diz-se das balas ou projécteis, usados hoje em armas de fogo.
\section{Cilindrose}
\begin{itemize}
\item {Grp. gram.:f.}
\end{itemize}
\begin{itemize}
\item {Utilização:Anat.}
\end{itemize}
\begin{itemize}
\item {Proveniência:(De \textunderscore cilindro\textunderscore )}
\end{itemize}
Espécie de sutura craniana.
\section{Cilindrossomo}
\begin{itemize}
\item {Grp. gram.:adj.}
\end{itemize}
\begin{itemize}
\item {Proveniência:(Do gr. \textunderscore kulindros\textunderscore  + \textunderscore soma\textunderscore )}
\end{itemize}
Que tem corpo cilíndrico.
\section{Cilopodia}
\begin{itemize}
\item {Grp. gram.:f.}
\end{itemize}
\begin{itemize}
\item {Proveniência:(Do gr. \textunderscore kullos\textunderscore  + \textunderscore pous\textunderscore , \textunderscore podos\textunderscore )}
\end{itemize}
O mesmo que \textunderscore cilose\textunderscore .
\section{Cilose}
\begin{itemize}
\item {Grp. gram.:f.}
\end{itemize}
\begin{itemize}
\item {Proveniência:(Do gr. \textunderscore kullos\textunderscore )}
\end{itemize}
Deformidade dos pés.
\section{Cima}
\begin{itemize}
\item {Grp. gram.:f.}
\end{itemize}
\begin{itemize}
\item {Utilização:Bot.}
\end{itemize}
\begin{itemize}
\item {Proveniência:(Gr. \textunderscore kuma\textunderscore )}
\end{itemize}
Tipo de inflorescência, caracterizada pela presença de flôres, que limitam superiormente cada eixo.
\section{Cimba}
\begin{itemize}
\item {Grp. gram.:f.}
\end{itemize}
\begin{itemize}
\item {Utilização:Des.}
\end{itemize}
\begin{itemize}
\item {Proveniência:(Do gr. \textunderscore kumbe\textunderscore )}
\end{itemize}
Batel chato, sem leme nem vela.
\section{Címbala}
\begin{itemize}
\item {Grp. gram.:f.}
\end{itemize}
\begin{itemize}
\item {Utilização:Mús.}
\end{itemize}
Registo de órgão, composto de duas ou três ordens de tubos de estanho, afinados em oitavas e quintas.
(Cp. \textunderscore címbalo\textunderscore )
\section{Cimbalária}
\begin{itemize}
\item {Grp. gram.:f.}
\end{itemize}
\begin{itemize}
\item {Utilização:Bot.}
\end{itemize}
\begin{itemize}
\item {Proveniência:(De \textunderscore címbalo\textunderscore )}
\end{itemize}
Espécie de escrofulária.
Espécie de saxífragacea.
\section{Címbalo}
\begin{itemize}
\item {Grp. gram.:m.}
\end{itemize}
\begin{itemize}
\item {Proveniência:(Do gr. \textunderscore kumbalon\textunderscore )}
\end{itemize}
Antigo instrumento musical, composto de dois meios globos de metal, que se percutiam.
Designação antiga do saltério.
\section{Cimbocefalia}
\begin{itemize}
\item {Grp. gram.:f.}
\end{itemize}
Qualidade de cimbocéfalo.
\section{Cimbocéfalo}
\begin{itemize}
\item {Grp. gram.:adj.}
\end{itemize}
\begin{itemize}
\item {Utilização:Anat.}
\end{itemize}
\begin{itemize}
\item {Proveniência:(Do gr. \textunderscore kumbe\textunderscore  + \textunderscore kephale\textunderscore )}
\end{itemize}
Que tem o crânio escavado na parte superior.
Exageradamente clinòcéfalo.
\section{Cimeira}
\begin{itemize}
\item {Grp. gram.:f.}
\end{itemize}
O mesmo que \textunderscore cima\textunderscore ^1.
\section{Cimeno}
\begin{itemize}
\item {Grp. gram.:m.}
\end{itemize}
\begin{itemize}
\item {Utilização:Chím.}
\end{itemize}
Um dos carbonetos do grupo benzênico.
\section{Cimófana}
\begin{itemize}
\item {Grp. gram.:f.}
\end{itemize}
Pedra preciosa do Brasil.
\section{Cimógrafo}
\begin{itemize}
\item {Grp. gram.:m.}
\end{itemize}
\begin{itemize}
\item {Proveniência:(Do gr. \textunderscore kuma\textunderscore  + \textunderscore graphein\textunderscore )}
\end{itemize}
Aparelho, com que se medem as pulsações do febricitante.
\section{Cimopólia}
\begin{itemize}
\item {Grp. gram.:f.}
\end{itemize}
Gênero de crustáceos, decapodes.
\section{Cina}
\begin{itemize}
\item {Grp. gram.:f.}
\end{itemize}
\begin{itemize}
\item {Proveniência:(Lat. \textunderscore cyna\textunderscore )}
\end{itemize}
Árvore de Arábia, semelhante á palmeira, ou espécie de algodoeiro.
\section{Cinacanta}
\begin{itemize}
\item {Grp. gram.:f.}
\end{itemize}
\begin{itemize}
\item {Proveniência:(Gr. \textunderscore kunakantha\textunderscore )}
\end{itemize}
Designação antiga da roseira brava.
\section{Cinamolgo}
\begin{itemize}
\item {Grp. gram.:m.}
\end{itemize}
Ave da Arábia, que faz o ninho com ramos de caneleira.
\section{Cinância}
\begin{itemize}
\item {Grp. gram.:f.}
\end{itemize}
\begin{itemize}
\item {Utilização:Des.}
\end{itemize}
\begin{itemize}
\item {Proveniência:(Do gr. \textunderscore kuon\textunderscore )}
\end{itemize}
Espécie de angina, em que os doentes deitam a língua de fora, como os cães sequiosos ou ofegantes.
\section{Cinanque}
\begin{itemize}
\item {Grp. gram.:f.}
\end{itemize}
O mesmo que \textunderscore cinância\textunderscore .
\section{Cinantropia}
\begin{itemize}
\item {Grp. gram.:f.}
\end{itemize}
\begin{itemize}
\item {Proveniência:(De \textunderscore cinantropo\textunderscore )}
\end{itemize}
Alucinação, em que o doente se julga transformado em cão.
\section{Cinantropo}
\begin{itemize}
\item {Grp. gram.:m.}
\end{itemize}
\begin{itemize}
\item {Proveniência:(Do gr. \textunderscore kuon\textunderscore  + \textunderscore anthropos\textunderscore )}
\end{itemize}
Aquele, que padece cinantropia.
\section{Cinara}
\begin{itemize}
\item {Grp. gram.:f.}
\end{itemize}
\begin{itemize}
\item {Proveniência:(Gr. \textunderscore kunara\textunderscore )}
\end{itemize}
Gênero de cardos, que da o nome ás cináreas.
\section{Cináreas}
\begin{itemize}
\item {Grp. gram.:f. pl.}
\end{itemize}
\begin{itemize}
\item {Proveniência:(De \textunderscore cinara\textunderscore )}
\end{itemize}
Grupo de plantas sinantéreas, a que pertence a alcachofra, o cardo bento, a bardana, etc.
\section{Cineces}
\begin{itemize}
\item {Grp. gram.:m. pl.}
\end{itemize}
Nome de uma tríbo africana, mencionada na \textunderscore Etiópia Or.\textunderscore , l. I, c. 1.
\section{Cinegética}
\begin{itemize}
\item {Grp. gram.:f.}
\end{itemize}
\begin{itemize}
\item {Proveniência:(De \textunderscore cinegético\textunderscore )}
\end{itemize}
Arte de caçar, com o auxílio de cães.
Arte da caça.
\section{Cinegético}
\begin{itemize}
\item {Grp. gram.:adj.}
\end{itemize}
\begin{itemize}
\item {Proveniência:(Gr. \textunderscore kunegetikos\textunderscore )}
\end{itemize}
Relativo a caça.
\section{Cinegetófilo}
\begin{itemize}
\item {Grp. gram.:m.  e  adj.}
\end{itemize}
\begin{itemize}
\item {Utilização:Neol.}
\end{itemize}
\begin{itemize}
\item {Proveniência:(Do gr. \textunderscore kunegeticos\textunderscore  + \textunderscore philos\textunderscore )}
\end{itemize}
O que gosta da caça.
\section{Cinetos}
\begin{itemize}
\item {Grp. gram.:m. pl.}
\end{itemize}
Antigos habitadores da costa de Portugal, desde o Sado ao cabo de San-Vicente.
\section{Cinicamente}
\begin{itemize}
\item {Grp. gram.:adv.}
\end{itemize}
Com cinismo.
De modo cínico.
\section{Cinico}
\begin{itemize}
\item {Grp. gram.:adj.}
\end{itemize}
\begin{itemize}
\item {Proveniência:(Lat. \textunderscore cynicus\textunderscore )}
\end{itemize}
Canino, (des. neste sentido).
Pertencente a uma filosophia, que desprezava as conveniências e fórmulas sociaes.
Impudente, desavergonhado.
\section{Cínipes}
\begin{itemize}
\item {Grp. gram.:m. pl.}
\end{itemize}
\begin{itemize}
\item {Proveniência:(Do gr. \textunderscore kuon\textunderscore  + \textunderscore ips\textunderscore )}
\end{itemize}
Pequeninos insectos, que, mordendo os vegetaes, dão origem ás galhas.
\section{Cinismo}
\begin{itemize}
\item {Grp. gram.:m.}
\end{itemize}
\begin{itemize}
\item {Proveniência:(Gr. \textunderscore kunismos\textunderscore . Cp. \textunderscore cinico\textunderscore )}
\end{itemize}
Sistema dos cínicos.
Impudência; desvergonha.
\section{Cinocefaleia}
\begin{itemize}
\item {Grp. gram.:f.}
\end{itemize}
\begin{itemize}
\item {Utilização:Bot.}
\end{itemize}
\begin{itemize}
\item {Proveniência:(Lat. \textunderscore cynocephalea\textunderscore )}
\end{itemize}
Planta, também designada por \textunderscore cabeça de cão\textunderscore , e que é contraveneno.
\section{Cinocéfalo}
\begin{itemize}
\item {Grp. gram.:m.}
\end{itemize}
\begin{itemize}
\item {Proveniência:(Gr. \textunderscore kunokephalos\textunderscore )}
\end{itemize}
Gênero de macacos, cuja cabeça é semelhante á do cão.
\section{Cinófilo}
\begin{itemize}
\item {Grp. gram.:adj.}
\end{itemize}
\begin{itemize}
\item {Proveniência:(Do gr. \textunderscore kuon\textunderscore  + \textunderscore philos\textunderscore )}
\end{itemize}
Que gosta dos cães.
\section{Cinofobia}
\begin{itemize}
\item {Grp. gram.:f.}
\end{itemize}
Estado ou qualidade de cinófobo.
\section{Cinófobo}
\begin{itemize}
\item {Grp. gram.:adj.}
\end{itemize}
\begin{itemize}
\item {Proveniência:(Do gr. \textunderscore kuon\textunderscore  + \textunderscore phobos\textunderscore )}
\end{itemize}
Que tem exagerado medo dos cães; que detesta os cães.
\section{Cinoglossa}
\begin{itemize}
\item {Grp. gram.:f.}
\end{itemize}
\begin{itemize}
\item {Utilização:Bot.}
\end{itemize}
\begin{itemize}
\item {Proveniência:(Do gr. \textunderscore kuon\textunderscore  + \textunderscore glossa\textunderscore )}
\end{itemize}
Planta, vulgarmente conhecida por \textunderscore língua-de-cão\textunderscore .
\section{Cinografia}
\begin{itemize}
\item {Grp. gram.:f.}
\end{itemize}
\begin{itemize}
\item {Proveniência:(Do gr. \textunderscore kuon\textunderscore  + \textunderscore graphein\textunderscore )}
\end{itemize}
Tratado ou história dos cães.
\section{Cinográfico}
\begin{itemize}
\item {Grp. gram.:adj.}
\end{itemize}
Relativo á cinografia.
\section{Cinomorfo}
\begin{itemize}
\item {Grp. gram.:adj.}
\end{itemize}
\begin{itemize}
\item {Proveniência:(Do gr. \textunderscore kuon\textunderscore  + \textunderscore morphe\textunderscore )}
\end{itemize}
Semelhante aos cães.
\section{Cinopiteco}
\begin{itemize}
\item {Grp. gram.:m.}
\end{itemize}
\begin{itemize}
\item {Proveniência:(Do gr. \textunderscore kuon\textunderscore  + \textunderscore pithex\textunderscore )}
\end{itemize}
Espécie de macaco.
\section{Cinopse}
\begin{itemize}
\item {Grp. gram.:f.}
\end{itemize}
\begin{itemize}
\item {Proveniência:(Do gr. \textunderscore kuon\textunderscore  + \textunderscore ops\textunderscore )}
\end{itemize}
Gênero de plantas gramíneas.
\section{Cinorexia}
\begin{itemize}
\item {fónica:csi}
\end{itemize}
\begin{itemize}
\item {Grp. gram.:f.}
\end{itemize}
\begin{itemize}
\item {Utilização:Med.}
\end{itemize}
\begin{itemize}
\item {Proveniência:(Do gr. \textunderscore kuon\textunderscore  + \textunderscore orexia\textunderscore )}
\end{itemize}
Doença, mais conhecida por \textunderscore fome canina\textunderscore .
\section{Cinosargo}
\begin{itemize}
\item {Grp. gram.:m.}
\end{itemize}
\begin{itemize}
\item {Proveniência:(Gr. \textunderscore kunosarges\textunderscore )}
\end{itemize}
Ginásio ateniense, onde Antístenes ensinava a filosofia cínica.
\section{Cinosura}
\begin{itemize}
\item {Grp. gram.:f.}
\end{itemize}
\begin{itemize}
\item {Proveniência:(De \textunderscore cinosuro\textunderscore )}
\end{itemize}
Constelação boreal, o mesmo que \textunderscore Ursa-Menor\textunderscore .
Planta gramínea, (\textunderscore cynosurus cristatus\textunderscore , Lin.).
\section{Cinosuro}
\begin{itemize}
\item {Grp. gram.:adj.}
\end{itemize}
\begin{itemize}
\item {Proveniência:(Do gr. \textunderscore kuon\textunderscore  + \textunderscore oura\textunderscore )}
\end{itemize}
Que tem cauda semelhante á do cão.
\section{Ciparisso}
\begin{itemize}
\item {Grp. gram.:m.}
\end{itemize}
\begin{itemize}
\item {Utilização:Poét.}
\end{itemize}
\begin{itemize}
\item {Proveniência:(Lat. \textunderscore cyparissus\textunderscore )}
\end{itemize}
O mesmo que \textunderscore cipreste\textunderscore . Cf. \textunderscore Lusíadas\textunderscore .
\section{Cipela}
\begin{itemize}
\item {Grp. gram.:f.}
\end{itemize}
\begin{itemize}
\item {Proveniência:(Do gr. \textunderscore kupellon\textunderscore , taça)}
\end{itemize}
Gênero de plantas irídeas.
\section{Ciperáceas}
\begin{itemize}
\item {Grp. gram.:f. pl.}
\end{itemize}
\begin{itemize}
\item {Proveniência:(De \textunderscore ciperáceo\textunderscore )}
\end{itemize}
Família de plantas, que tem por tipo a junça.
\section{Ciperáceo}
\begin{itemize}
\item {Grp. gram.:adj.}
\end{itemize}
\begin{itemize}
\item {Proveniência:(De \textunderscore cípero\textunderscore )}
\end{itemize}
Relativo ou semelhante á junça.
\section{Cípero}
\begin{itemize}
\item {Grp. gram.:m.}
\end{itemize}
\begin{itemize}
\item {Proveniência:(Gr. \textunderscore kuperos\textunderscore )}
\end{itemize}
O mesmo que \textunderscore junça\textunderscore .
\section{Cipreia}
\begin{itemize}
\item {Grp. gram.:f.}
\end{itemize}
Molusco gasterópode.
\section{Ciprestal}
\begin{itemize}
\item {Grp. gram.:m.}
\end{itemize}
Terreno, onde crescem ciprestes.
\section{Cipreste}
\begin{itemize}
\item {Grp. gram.:m.}
\end{itemize}
\begin{itemize}
\item {Utilização:Fig.}
\end{itemize}
\begin{itemize}
\item {Proveniência:(Do lat. \textunderscore cupressus\textunderscore )}
\end{itemize}
Árvore conífera.
Morte, luto, tristeza.
\section{Ciprínidas}
\begin{itemize}
\item {Grp. gram.:f. pl.}
\end{itemize}
\begin{itemize}
\item {Utilização:Zool.}
\end{itemize}
\begin{itemize}
\item {Proveniência:(Do gr. \textunderscore kuprinos\textunderscore  + \textunderscore eidos\textunderscore )}
\end{itemize}
Família de peixes, que tem por tipo a carpa.
\section{Ciprino}
\begin{itemize}
\item {Grp. gram.:adj.}
\end{itemize}
O mesmo que \textunderscore cíprio\textunderscore .
\section{Ciprino}
\begin{itemize}
\item {Grp. gram.:m.}
\end{itemize}
\begin{itemize}
\item {Proveniência:(Lat. \textunderscore cyprinum\textunderscore )}
\end{itemize}
Óleo de alfena.
\section{Ciprinoides}
\begin{itemize}
\item {Grp. gram.:m. pl.}
\end{itemize}
O mesmo que \textunderscore ciprínidas\textunderscore .
\section{Cíprio}
\begin{itemize}
\item {Grp. gram.:adj.}
\end{itemize}
\begin{itemize}
\item {Proveniência:(Lat. \textunderscore cyprius\textunderscore , de \textunderscore Cyprus\textunderscore , n.p.)}
\end{itemize}
Relativo a Cipro.
\section{Cirenaico}
\begin{itemize}
\item {Grp. gram.:m.  e  adj.}
\end{itemize}
O mesmo que \textunderscore cirenéu\textunderscore .
\section{Cirenéu}
\begin{itemize}
\item {Grp. gram.:m.}
\end{itemize}
\begin{itemize}
\item {Utilização:Fig.}
\end{itemize}
\begin{itemize}
\item {Proveniência:(Gr. \textunderscore kurenaioi\textunderscore )}
\end{itemize}
Aquele, que é natural de Cirene.
Auxiliador: \textunderscore serviu-me de cirenéu\textunderscore .
\section{Ciriologia}
\begin{itemize}
\item {Grp. gram.:f.}
\end{itemize}
\begin{itemize}
\item {Proveniência:(Do gr. \textunderscore kurios\textunderscore  + \textunderscore logos\textunderscore )}
\end{itemize}
Emprêgo exclusivo de expressões próprias.
\section{Ciriológico}
\begin{itemize}
\item {Grp. gram.:adj.}
\end{itemize}
Relativo á ciriologia.
\section{Cirne}
\begin{itemize}
\item {Grp. gram.:m.}
\end{itemize}
O mesmo que \textunderscore cisne\textunderscore . Cf. Sousa, \textunderscore Vida do Arceb.\textunderscore , III, 35.
\section{Ciropédia}
\begin{itemize}
\item {Grp. gram.:f.}
\end{itemize}
\begin{itemize}
\item {Utilização:des.}
\end{itemize}
\begin{itemize}
\item {Utilização:Ext.}
\end{itemize}
\begin{itemize}
\item {Proveniência:(Do gr. \textunderscore Kuros\textunderscore , n. p. + \textunderscore paideia\textunderscore )}
\end{itemize}
Tratado de educação. Cf. \textunderscore Dicc. Exeg.\textunderscore 
\section{Cirtopódio}
\begin{itemize}
\item {Grp. gram.:m.}
\end{itemize}
\begin{itemize}
\item {Proveniência:(Do gr. \textunderscore kurtos\textunderscore  + \textunderscore pous\textunderscore , \textunderscore podos\textunderscore )}
\end{itemize}
Gênero de orquideas.
\section{Cisne}
\begin{itemize}
\item {Grp. gram.:m.}
\end{itemize}
\begin{itemize}
\item {Utilização:Fig.}
\end{itemize}
\begin{itemize}
\item {Proveniência:(Do gr. \textunderscore kuknos\textunderscore )}
\end{itemize}
Ave palmípede e aquática do gênero do pato.
Poéta, orador ou músico célebre.
Constelação setentrional.
\section{Cisnífero}
\begin{itemize}
\item {Grp. gram.:adj.}
\end{itemize}
\begin{itemize}
\item {Utilização:Poét.}
\end{itemize}
Em que andam cisnes. Cf. \textunderscore Viriato Trág.\textunderscore , V. 91.
\section{Cistalgia}
\begin{itemize}
\item {Grp. gram.:f.}
\end{itemize}
\begin{itemize}
\item {Utilização:Med.}
\end{itemize}
\begin{itemize}
\item {Proveniência:(Do gr. \textunderscore kustos\textunderscore  + \textunderscore algos\textunderscore )}
\end{itemize}
Dôr nervosa na bexiga.
\section{Cistalgico}
\begin{itemize}
\item {Grp. gram.:adj.}
\end{itemize}
Relativo á cistalgia.
\section{Cistecercose}
\begin{itemize}
\item {Grp. gram.:f.}
\end{itemize}
Designação cientifica da doença produzida nos porcos pelo cisticerco.
\section{Cisticerco}
\begin{itemize}
\item {Grp. gram.:m.}
\end{itemize}
\begin{itemize}
\item {Proveniência:(Do gr. \textunderscore kustis\textunderscore  + \textunderscore kerkos\textunderscore )}
\end{itemize}
Animálculo dimorpo, ou estado por que passam os embriões dos vermes, como a tênia.--O cisticerco completa a sua evolução em duas fases: na primeira, (larvas), produz no porco a cistercose ou gafeira; na segunda, desenvolve-se no homem, produzindo a solitária ou tênia. As duas fórmas são solidárias entre si por contaminação mútua das espécies em que se albergam.
\section{Cístico}
\begin{itemize}
\item {Grp. gram.:adj.}
\end{itemize}
\begin{itemize}
\item {Proveniência:(Do gr. \textunderscore kustis\textunderscore )}
\end{itemize}
Relativo á bexiga.
Relativo ao cisto: \textunderscore cavidade cística\textunderscore .
\section{Cistífero}
\begin{itemize}
\item {Grp. gram.:adj.}
\end{itemize}
\begin{itemize}
\item {Utilização:Hist. Nat.}
\end{itemize}
\begin{itemize}
\item {Proveniência:(Do gr. \textunderscore kustis\textunderscore  + lat. \textunderscore ferre\textunderscore )}
\end{itemize}
Que tem uma espécie de bexiga ou bolsa.
\section{Cistina}
\begin{itemize}
\item {Grp. gram.:f.}
\end{itemize}
\begin{itemize}
\item {Proveniência:(Do gr. \textunderscore kustis\textunderscore )}
\end{itemize}
Substância, produzida pela secreção dos rins e que constitue os cálculos vesicaes e renaes, e certos depósitos urinários.
\section{Cistinoso}
\begin{itemize}
\item {Grp. gram.:adj.}
\end{itemize}
Que contém cistina.
\section{Cistinuria}
\begin{itemize}
\item {Grp. gram.:f.}
\end{itemize}
Emissão de urina cistinosa.
\section{Cistite}
\begin{itemize}
\item {Grp. gram.:f.}
\end{itemize}
\begin{itemize}
\item {Proveniência:(Do gr. \textunderscore kustis\textunderscore )}
\end{itemize}
Inflamação da bexiga.
\section{Cistítomo}
\begin{itemize}
\item {Grp. gram.:m.}
\end{itemize}
Instrumento, para abrir a cápsula do cristalino do ôlho.
\section{Cisto}
\begin{itemize}
\item {Grp. gram.:m.}
\end{itemize}
\begin{itemize}
\item {Utilização:Med.}
\end{itemize}
\begin{itemize}
\item {Proveniência:(Do gr. \textunderscore kustis\textunderscore , bexiga)}
\end{itemize}
Espécie de tumor, ou membrana em fórma de bexiga sem abertura, cheia de líquido ou de outras substâncias anormaes, e que se fórma acidentalmente.--A fórma \textunderscore kysto\textunderscore , por alguns usada, não é portuguesa.
\section{Cistocele}
\begin{itemize}
\item {Grp. gram.:f.}
\end{itemize}
\begin{itemize}
\item {Proveniência:(Do gr. \textunderscore kustis\textunderscore  + \textunderscore kele\textunderscore )}
\end{itemize}
Hérnia da bexiga.
\section{Cistóide}
\begin{itemize}
\item {Grp. gram.:adj.}
\end{itemize}
\begin{itemize}
\item {Grp. gram.:Pl.}
\end{itemize}
\begin{itemize}
\item {Proveniência:(Do gr. \textunderscore kustis\textunderscore  + \textunderscore eidos\textunderscore )}
\end{itemize}
Semelhante a uma bexiga.
Animaes fósseis, da série mesozoica, que parecem têr sido o tronco principal das classes dos echinodermes.
\section{Cistólito}
\begin{itemize}
\item {Grp. gram.:m.}
\end{itemize}
\begin{itemize}
\item {Utilização:Bot.}
\end{itemize}
\begin{itemize}
\item {Proveniência:(Do gr. \textunderscore kustis\textunderscore  + \textunderscore lithos\textunderscore )}
\end{itemize}
Corpúsculo ovoide, suspenso no interior da célula vegetal e composto de concreções de celulose da parede da mesma célula.
\section{Cistoplegia}
\begin{itemize}
\item {Grp. gram.:f.}
\end{itemize}
\begin{itemize}
\item {Proveniência:(Do gr. \textunderscore kustis\textunderscore  + \textunderscore plessein\textunderscore )}
\end{itemize}
Paralisia da bexiga.
\section{Cistoplégico}
\begin{itemize}
\item {Grp. gram.:adj.}
\end{itemize}
Relativo á cistoplegia.
Aquele que sofre cistoplegia.
\section{Cistoscopia}
\begin{itemize}
\item {Grp. gram.:f.}
\end{itemize}
\begin{itemize}
\item {Utilização:Med.}
\end{itemize}
Observação da bexiga, por meio do cistoscópio.
\section{Cistoscópio}
\begin{itemize}
\item {Grp. gram.:m.}
\end{itemize}
\begin{itemize}
\item {Proveniência:(Do gr. \textunderscore kustis\textunderscore  + \textunderscore skopein\textunderscore )}
\end{itemize}
Aparelho cirúrgico, recentemente inventado, para observações na bexiga.
\section{Cistotomia}
\begin{itemize}
\item {Grp. gram.:f.}
\end{itemize}
\begin{itemize}
\item {Proveniência:(De \textunderscore cistótomo\textunderscore )}
\end{itemize}
Operação cirúrgica, para tirar da bexiga os cálculos urinários ou outros corpos nocivos.
Litotomia.
Operação da talha.
\section{Cistótomo}
\begin{itemize}
\item {Grp. gram.:m.}
\end{itemize}
\begin{itemize}
\item {Proveniência:(Do gr. \textunderscore krustis\textunderscore  + \textunderscore tome\textunderscore )}
\end{itemize}
Instrumento, que se usa na operação da talha.
\section{Citíneas}
\begin{itemize}
\item {Grp. gram.:f. pl.}
\end{itemize}
\begin{itemize}
\item {Proveniência:(Do gr. \textunderscore kutinos\textunderscore )}
\end{itemize}
Família de plantas, que tem por tipo a roman.
\section{Citiso}
\begin{itemize}
\item {Grp. gram.:m.}
\end{itemize}
\begin{itemize}
\item {Proveniência:(Gr. \textunderscore kutisos\textunderscore )}
\end{itemize}
Espécie de trevo.
Luzerna arborescente.
\section{Citocromo}
\begin{itemize}
\item {Grp. gram.:adj.}
\end{itemize}
\begin{itemize}
\item {Proveniência:(Do gr. \textunderscore kutus\textunderscore  + \textunderscore khroma\textunderscore )}
\end{itemize}
O mesmo que \textunderscore somatocromo\textunderscore .
\section{Citode}
\begin{itemize}
\item {Grp. gram.:m.}
\end{itemize}
\begin{itemize}
\item {Proveniência:(Gr. \textunderscore kutodes\textunderscore )}
\end{itemize}
Elemento plástico dos tecidos orgânicos.
\section{Citologia}
\begin{itemize}
\item {Grp. gram.:f.}
\end{itemize}
\begin{itemize}
\item {Utilização:Physiol.}
\end{itemize}
\begin{itemize}
\item {Proveniência:(Do gr. \textunderscore kutos\textunderscore  + \textunderscore logos\textunderscore )}
\end{itemize}
Estudo da célula em geral.
\section{Citoplasma}
\begin{itemize}
\item {Grp. gram.:m.}
\end{itemize}
\begin{itemize}
\item {Utilização:Physiol.}
\end{itemize}
\begin{itemize}
\item {Proveniência:(Do gr. \textunderscore kutos\textunderscore  + \textunderscore plassein\textunderscore )}
\end{itemize}
Nome, que se dá ao protoplasma ordinário da célula, para o distinguir do nucleoplasma.
\section{Cizicena}
\begin{itemize}
\item {Grp. gram.:f.}
\end{itemize}
\begin{itemize}
\item {Proveniência:(Lat. \textunderscore cyzícenus\textunderscore )}
\end{itemize}
Sala do jantar, entre os Gregos.
\section{Cylindro-ogival}
\begin{itemize}
\item {Grp. gram.:adj.}
\end{itemize}
\begin{itemize}
\item {Proveniência:(De \textunderscore cylíndrico\textunderscore  + \textunderscore ogival\textunderscore )}
\end{itemize}
Diz-se das balas ou projécteis, usados hoje em armas de fogo.
\section{Cylindrose}
\begin{itemize}
\item {Grp. gram.:f.}
\end{itemize}
\begin{itemize}
\item {Utilização:Anat.}
\end{itemize}
\begin{itemize}
\item {Proveniência:(De \textunderscore cylindro\textunderscore )}
\end{itemize}
Espécie de sutura craniana.
\section{Cylindrosomo}
\begin{itemize}
\item {fónica:so}
\end{itemize}
\begin{itemize}
\item {Grp. gram.:adj.}
\end{itemize}
\begin{itemize}
\item {Proveniência:(Do gr. \textunderscore kulindros\textunderscore  + \textunderscore soma\textunderscore )}
\end{itemize}
Que tem corpo cylíndrico.
\section{Cyllopodia}
\begin{itemize}
\item {Proveniência:(Do gr. \textunderscore kullos\textunderscore  + \textunderscore pous\textunderscore , \textunderscore podos\textunderscore )}
\end{itemize}
O mesmo que \textunderscore cyllose\textunderscore .
\section{Cyllose}
\begin{itemize}
\item {Grp. gram.:f.}
\end{itemize}
\begin{itemize}
\item {Proveniência:(Do gr. \textunderscore kullos\textunderscore )}
\end{itemize}
Deformidade dos pés.
\section{Cyma}
\begin{itemize}
\item {Grp. gram.:f.}
\end{itemize}
\begin{itemize}
\item {Utilização:Bot.}
\end{itemize}
\begin{itemize}
\item {Proveniência:(Gr. \textunderscore kuma\textunderscore )}
\end{itemize}
Typo de inflorescência, caracterizada pela presença de flôres, que limitam superiormente cada eixo.
\section{Cymba}
\begin{itemize}
\item {Grp. gram.:f.}
\end{itemize}
\begin{itemize}
\item {Utilização:Des.}
\end{itemize}
\begin{itemize}
\item {Proveniência:(Do gr. \textunderscore kumbe\textunderscore )}
\end{itemize}
Batel chato, sem leme nem vela.
\section{Cýmbala}
\begin{itemize}
\item {Grp. gram.:f.}
\end{itemize}
\begin{itemize}
\item {Utilização:Mús.}
\end{itemize}
Registo de órgão, composto de duas ou três ordens de tubos de estanho, afinados em oitavas e quintas.
(Cp. \textunderscore cýmbalo\textunderscore )
\section{Cymbalária}
\begin{itemize}
\item {Grp. gram.:f.}
\end{itemize}
\begin{itemize}
\item {Utilização:Bot.}
\end{itemize}
\begin{itemize}
\item {Proveniência:(De \textunderscore cýmbalo\textunderscore )}
\end{itemize}
Espécie de escrofulária.
Espécie de saxífragacea.
\section{Cýmbalo}
\begin{itemize}
\item {Grp. gram.:m.}
\end{itemize}
\begin{itemize}
\item {Proveniência:(Do gr. \textunderscore kumbalon\textunderscore )}
\end{itemize}
Antigo instrumento musical, composto de dois meios globos de metal, que se percutiam.
Designação antiga do saltério.
\section{Cymbocephalia}
\begin{itemize}
\item {Grp. gram.:f.}
\end{itemize}
Qualidade de cymbocéphalo.
\section{Cymbocéphalo}
\begin{itemize}
\item {Grp. gram.:adj.}
\end{itemize}
\begin{itemize}
\item {Utilização:Anat.}
\end{itemize}
\begin{itemize}
\item {Proveniência:(Do gr. \textunderscore kumbe\textunderscore  + \textunderscore kephale\textunderscore )}
\end{itemize}
Que tem o crânio escavado na parte superior.
Exaggeradamente clinòcéphalo.
\section{Cymeira}
\begin{itemize}
\item {Grp. gram.:f.}
\end{itemize}
O mesmo que \textunderscore cyma\textunderscore .
\section{Cymeno}
\begin{itemize}
\item {Grp. gram.:m.}
\end{itemize}
\begin{itemize}
\item {Utilização:Chím.}
\end{itemize}
Um dos carbonetos do grupo benzênico.
\section{Cymógrapho}
\begin{itemize}
\item {Grp. gram.:m.}
\end{itemize}
\begin{itemize}
\item {Proveniência:(Do gr. \textunderscore kuma\textunderscore  + \textunderscore graphein\textunderscore )}
\end{itemize}
Apparelho, com que se medem as pulsações do febricitante.
\section{Cymóphana}
\begin{itemize}
\item {Grp. gram.:f.}
\end{itemize}
Pedra preciosa do Brasil.
\section{Cymopólia}
\begin{itemize}
\item {Grp. gram.:f.}
\end{itemize}
Gênero de crustáceos, decapodes.
\section{Cyna}
\begin{itemize}
\item {Grp. gram.:f.}
\end{itemize}
\begin{itemize}
\item {Proveniência:(Lat. \textunderscore cyna\textunderscore )}
\end{itemize}
Árvore de Arábia, semelhante á palmeira, ou espécie de algodoeiro.
\section{Cynacantha}
\begin{itemize}
\item {Proveniência:(Gr. \textunderscore kunakantha\textunderscore )}
\end{itemize}
Designação antiga da roseira brava.
\section{Cynamolgo}
\begin{itemize}
\item {Grp. gram.:m.}
\end{itemize}
Ave da Arábia, que faz o ninho com ramos de caneleira.
\section{Cynanche}
\begin{itemize}
\item {fónica:que}
\end{itemize}
\begin{itemize}
\item {Grp. gram.:f.}
\end{itemize}
O mesmo que \textunderscore cinância\textunderscore .
\section{Cynância}
\begin{itemize}
\item {Grp. gram.:f.}
\end{itemize}
\begin{itemize}
\item {Utilização:Des.}
\end{itemize}
\begin{itemize}
\item {Proveniência:(Do gr. \textunderscore kuon\textunderscore )}
\end{itemize}
Espécie de angina, em que os doentes deitam a língua de fora, como os cães sequiosos ou offegantes.
\section{Cynanthropia}
\begin{itemize}
\item {Grp. gram.:f.}
\end{itemize}
\begin{itemize}
\item {Proveniência:(De \textunderscore cinanthropo\textunderscore )}
\end{itemize}
Alucinação, em que o doente se julga transformado em cão.
\section{Cynanthropo}
\begin{itemize}
\item {Grp. gram.:m.}
\end{itemize}
\begin{itemize}
\item {Proveniência:(Do gr. \textunderscore kuon\textunderscore  + \textunderscore anthropos\textunderscore )}
\end{itemize}
Aquelle, que padece cynanthropia.
\section{Cynara}
\begin{itemize}
\item {Grp. gram.:f.}
\end{itemize}
\begin{itemize}
\item {Proveniência:(Gr. \textunderscore kunara\textunderscore )}
\end{itemize}
Gênero de cardos, que da o nome ás cynáreas.
\section{Cynáreas}
\begin{itemize}
\item {Grp. gram.:f. pl.}
\end{itemize}
\begin{itemize}
\item {Proveniência:(De \textunderscore cynara\textunderscore )}
\end{itemize}
Grupo de plantas synanthéreas, a que pertence a alcachofra, o cardo bento, a bardana, etc.
\section{Cyneces}
\begin{itemize}
\item {Grp. gram.:m. pl.}
\end{itemize}
Nome de uma tríbo africana, mencionada na \textunderscore Etiópia Or.\textunderscore , l. I, c. 1.
\section{Cynegética}
\begin{itemize}
\item {Grp. gram.:f.}
\end{itemize}
\begin{itemize}
\item {Proveniência:(De \textunderscore cynegético\textunderscore )}
\end{itemize}
Arte de caçar, com o auxílio de cães.
Arte da caça.
\section{Cynegético}
\begin{itemize}
\item {Grp. gram.:adj.}
\end{itemize}
\begin{itemize}
\item {Proveniência:(Gr. \textunderscore kunegetikos\textunderscore )}
\end{itemize}
Relativo a caça.
\section{Cynegetóphilo}
\begin{itemize}
\item {Grp. gram.:m.  e  adj.}
\end{itemize}
\begin{itemize}
\item {Utilização:Neol.}
\end{itemize}
\begin{itemize}
\item {Proveniência:(Do gr. \textunderscore kunegeticos\textunderscore  + \textunderscore philos\textunderscore )}
\end{itemize}
O que gosta da caça.
\section{Cynetos}
\begin{itemize}
\item {Grp. gram.:m. pl.}
\end{itemize}
Antigos habitadores da costa de Portugal, desde o Sado ao cabo de San-Vicente.
\section{Cynicamente}
\begin{itemize}
\item {Grp. gram.:adv.}
\end{itemize}
Com cynismo.
De modo cýnico.
\section{Cýnico}
\begin{itemize}
\item {Grp. gram.:adj.}
\end{itemize}
\begin{itemize}
\item {Proveniência:(Lat. \textunderscore cynicus\textunderscore )}
\end{itemize}
Canino, (des. neste sentido).
Pertencente a uma philosophia, que desprezava as conveniências e fórmulas sociaes.
Impudente, desavergonhado.
\section{Cýnipes}
\begin{itemize}
\item {Grp. gram.:m. pl.}
\end{itemize}
\begin{itemize}
\item {Proveniência:(Do gr. \textunderscore kuon\textunderscore  + \textunderscore ips\textunderscore )}
\end{itemize}
Pequeninos insectos, que, mordendo os vegetaes, dão origem ás galhas.
\section{Cynismo}
\begin{itemize}
\item {Grp. gram.:m.}
\end{itemize}
\begin{itemize}
\item {Proveniência:(Gr. \textunderscore kunismos\textunderscore . Cp. \textunderscore cýnico\textunderscore )}
\end{itemize}
Systema dos cýnicos.
Impudência; desvergonha.
\section{Cynocephaleia}
\begin{itemize}
\item {Grp. gram.:f.}
\end{itemize}
\begin{itemize}
\item {Utilização:Bot.}
\end{itemize}
\begin{itemize}
\item {Proveniência:(Lat. \textunderscore cynocephalea\textunderscore )}
\end{itemize}
Planta, também designada por \textunderscore cabeça de cão\textunderscore , e que é contraveneno.
\section{Cynocéphalo}
\begin{itemize}
\item {Grp. gram.:m.}
\end{itemize}
\begin{itemize}
\item {Proveniência:(Gr. \textunderscore kunokephalos\textunderscore )}
\end{itemize}
Gênero de macacos, cuja cabeça é semelhante á do cão.
\section{Cynoglossa}
\begin{itemize}
\item {Grp. gram.:f.}
\end{itemize}
\begin{itemize}
\item {Utilização:Bot.}
\end{itemize}
\begin{itemize}
\item {Proveniência:(Do gr. \textunderscore kuon\textunderscore  + \textunderscore glossa\textunderscore )}
\end{itemize}
Planta, vulgarmente conhecida por \textunderscore língua-de-cão\textunderscore .
\section{Cynographia}
\begin{itemize}
\item {Grp. gram.:f.}
\end{itemize}
\begin{itemize}
\item {Proveniência:(Do gr. \textunderscore kuon\textunderscore  + \textunderscore graphein\textunderscore )}
\end{itemize}
Tratado ou história dos cães.
\section{Cynográphico}
\begin{itemize}
\item {Grp. gram.:adj.}
\end{itemize}
Relativo á cynographia.
\section{Cynomorpho}
\begin{itemize}
\item {Grp. gram.:adj.}
\end{itemize}
\begin{itemize}
\item {Proveniência:(Do gr. \textunderscore kuon\textunderscore  + \textunderscore morphe\textunderscore )}
\end{itemize}
Semelhante aos cães.
\section{Cynóphilo}
\begin{itemize}
\item {Grp. gram.:adj.}
\end{itemize}
\begin{itemize}
\item {Proveniência:(Do gr. \textunderscore kuon\textunderscore  + \textunderscore philos\textunderscore )}
\end{itemize}
Que gosta dos cães.
\section{Cynophobia}
\begin{itemize}
\item {Grp. gram.:f.}
\end{itemize}
Estado ou qualidade de cynóphobo.
\section{Cynóphobo}
\begin{itemize}
\item {Grp. gram.:adj.}
\end{itemize}
\begin{itemize}
\item {Proveniência:(Do gr. \textunderscore kuon\textunderscore  + \textunderscore phobos\textunderscore )}
\end{itemize}
Que tem exaggerado medo dos cães; que detesta os cães.
\section{Cynopitheco}
\begin{itemize}
\item {Grp. gram.:m.}
\end{itemize}
\begin{itemize}
\item {Proveniência:(Do gr. \textunderscore kuon\textunderscore  + \textunderscore pithex\textunderscore )}
\end{itemize}
Espécie de macaco.
\section{Cynopse}
\begin{itemize}
\item {Grp. gram.:f.}
\end{itemize}
\begin{itemize}
\item {Proveniência:(Do gr. \textunderscore kuon\textunderscore  + \textunderscore ops\textunderscore )}
\end{itemize}
Gênero de plantas gramíneas.
\section{Cynorexia}
\begin{itemize}
\item {fónica:csi}
\end{itemize}
\begin{itemize}
\item {Grp. gram.:f.}
\end{itemize}
\begin{itemize}
\item {Utilização:Med.}
\end{itemize}
\begin{itemize}
\item {Proveniência:(Do gr. \textunderscore kuon\textunderscore  + \textunderscore orexia\textunderscore )}
\end{itemize}
Doença, mais conhecida por \textunderscore fome canina\textunderscore .
\section{Cynosargo}
\begin{itemize}
\item {Grp. gram.:m.}
\end{itemize}
\begin{itemize}
\item {Proveniência:(Gr. \textunderscore kunosarges\textunderscore )}
\end{itemize}
Gymnásio atheniense, onde Antísthenes ensinava a philosophia cýnica.
\section{Cynosura}
\begin{itemize}
\item {Grp. gram.:f.}
\end{itemize}
\begin{itemize}
\item {Proveniência:(De \textunderscore cynosuro\textunderscore )}
\end{itemize}
Constellação boreal, o mesmo que \textunderscore Ursa-Menor\textunderscore .
Planta gramínea, (\textunderscore cynosurus cristatus\textunderscore , Lin.).
\section{Cynosuro}
\begin{itemize}
\item {Grp. gram.:adj.}
\end{itemize}
\begin{itemize}
\item {Proveniência:(Do gr. \textunderscore kuon\textunderscore  + \textunderscore oura\textunderscore )}
\end{itemize}
Que tem cauda semelhante á do cão.
\section{Cyparisso}
\begin{itemize}
\item {Grp. gram.:m.}
\end{itemize}
\begin{itemize}
\item {Utilização:Poét.}
\end{itemize}
\begin{itemize}
\item {Proveniência:(Lat. \textunderscore cyparissus\textunderscore )}
\end{itemize}
O mesmo que \textunderscore cypreste\textunderscore . Cf. \textunderscore Lusíadas\textunderscore .
\section{Cypella}
\begin{itemize}
\item {Grp. gram.:f.}
\end{itemize}
\begin{itemize}
\item {Proveniência:(Do gr. \textunderscore kupellon\textunderscore , taça)}
\end{itemize}
Gênero de plantas irídeas.
\section{Cyperáceas}
\begin{itemize}
\item {Grp. gram.:f. pl.}
\end{itemize}
\begin{itemize}
\item {Proveniência:(De \textunderscore cyperáceo\textunderscore )}
\end{itemize}
Família de plantas, que tem por typo a junça.
\section{Cyperáceo}
\begin{itemize}
\item {Grp. gram.:adj.}
\end{itemize}
\begin{itemize}
\item {Proveniência:(De \textunderscore cýpero\textunderscore )}
\end{itemize}
Relativo ou semelhante á junça.
\section{Cýpero}
\begin{itemize}
\item {Grp. gram.:m.}
\end{itemize}
\begin{itemize}
\item {Proveniência:(Gr. \textunderscore kuperos\textunderscore )}
\end{itemize}
O mesmo que \textunderscore junça\textunderscore .
\section{Cyphose}
\begin{itemize}
\item {Grp. gram.:f.}
\end{itemize}
\begin{itemize}
\item {Utilização:Anat.}
\end{itemize}
\begin{itemize}
\item {Grp. gram.:f.}
\end{itemize}
\begin{itemize}
\item {Utilização:Med.}
\end{itemize}
\begin{itemize}
\item {Proveniência:(Gr. \textunderscore kuphosis\textunderscore )}
\end{itemize}
Curvatura anómala da espinha dorsal para trás, isto é, formando convexidade posterior.
Curvatura anómala da columna vertebral para trás.
\section{Cyphótico}
\begin{itemize}
\item {Grp. gram.:m.  e  adj.}
\end{itemize}
Doente de cyphose.
\section{Cypreia}
\begin{itemize}
\item {Grp. gram.:f.}
\end{itemize}
Mollusco gasterópode.
\section{Cyprestal}
\begin{itemize}
\item {Grp. gram.:m.}
\end{itemize}
Terreno, onde crescem cyprestes.
\section{Cypreste}
\begin{itemize}
\item {Grp. gram.:m.}
\end{itemize}
\begin{itemize}
\item {Utilização:Fig.}
\end{itemize}
\begin{itemize}
\item {Proveniência:(Do lat. \textunderscore cupressus\textunderscore )}
\end{itemize}
Árvore conífera.
Morte, luto, tristeza.
\section{Cyprínidas}
\begin{itemize}
\item {Grp. gram.:f. pl.}
\end{itemize}
\begin{itemize}
\item {Utilização:Zool.}
\end{itemize}
\begin{itemize}
\item {Proveniência:(Do gr. \textunderscore kuprinos\textunderscore  + \textunderscore eidos\textunderscore )}
\end{itemize}
Família de peixes, que tem por typo a carpa.
\section{Cyprino}
\begin{itemize}
\item {Grp. gram.:adj.}
\end{itemize}
O mesmo que \textunderscore cýprio\textunderscore .
\section{Cyprino}
\begin{itemize}
\item {Grp. gram.:m.}
\end{itemize}
\begin{itemize}
\item {Proveniência:(Lat. \textunderscore cyprinum\textunderscore )}
\end{itemize}
Óleo de alfena.
\section{Cyprinoides}
\begin{itemize}
\item {Grp. gram.:m. pl.}
\end{itemize}
O mesmo que \textunderscore cyprínidas\textunderscore .
\section{Cýprio}
\begin{itemize}
\item {Grp. gram.:adj.}
\end{itemize}
\begin{itemize}
\item {Proveniência:(Lat. \textunderscore cyprius\textunderscore , de \textunderscore Cyprus\textunderscore , n.p.)}
\end{itemize}
Relativo a Cypro.
\section{Cypriota}
\begin{itemize}
\item {Grp. gram.:adj.}
\end{itemize}
\begin{itemize}
\item {Grp. gram.:M.}
\end{itemize}
O mesmo que \textunderscore cýprio\textunderscore .
Habitante de Cypro.
\section{Cyra}
\begin{itemize}
\item {Grp. gram.:f.}
\end{itemize}
\begin{itemize}
\item {Utilização:Ant.}
\end{itemize}
\begin{itemize}
\item {Proveniência:(Do lat. \textunderscore xyris\textunderscore ?)}
\end{itemize}
Matagal.
Chavascal.
\section{Cyrenaico}
\begin{itemize}
\item {Grp. gram.:m.  e  adj.}
\end{itemize}
O mesmo que \textunderscore cyrenéu\textunderscore .
\section{Cyrenéu}
\begin{itemize}
\item {Grp. gram.:m.}
\end{itemize}
\begin{itemize}
\item {Utilização:Fig.}
\end{itemize}
\begin{itemize}
\item {Proveniência:(Gr. \textunderscore kurenaioi\textunderscore )}
\end{itemize}
Aquelle, que é natural de Cyrene.
Auxiliador: \textunderscore serviu-me de cyrenéu\textunderscore .
\section{Cyriologia}
\begin{itemize}
\item {Grp. gram.:f.}
\end{itemize}
\begin{itemize}
\item {Proveniência:(Do gr. \textunderscore kurios\textunderscore  + \textunderscore logos\textunderscore )}
\end{itemize}
Emprêgo exclusivo de expressões próprias.
\section{Cyriológico}
\begin{itemize}
\item {Grp. gram.:adj.}
\end{itemize}
Relativo á cyriologia.
\section{Cyrita}
\begin{itemize}
\item {Grp. gram.:m.}
\end{itemize}
\begin{itemize}
\item {Utilização:Ant.}
\end{itemize}
\begin{itemize}
\item {Proveniência:(De \textunderscore cyra\textunderscore )}
\end{itemize}
Ermita.
Aquelle que reside em charnecas e brenhas.
\section{Cyrne}
\begin{itemize}
\item {Grp. gram.:m.}
\end{itemize}
O mesmo que \textunderscore cysne\textunderscore . Cf. Sousa, \textunderscore Vida do Arceb.\textunderscore , III, 35.
\section{Cyropedia}
\begin{itemize}
\item {Grp. gram.:f.}
\end{itemize}
\begin{itemize}
\item {Utilização:des.}
\end{itemize}
\begin{itemize}
\item {Utilização:Ext.}
\end{itemize}
\begin{itemize}
\item {Proveniência:(Do gr. \textunderscore Kuros\textunderscore , n. p. + \textunderscore paideia\textunderscore )}
\end{itemize}
Tratado de educação. Cf. \textunderscore Dicc. Exeg.\textunderscore 
\section{Cyrtopódio}
\begin{itemize}
\item {Grp. gram.:m.}
\end{itemize}
\begin{itemize}
\item {Proveniência:(Do gr. \textunderscore kurtos\textunderscore  + \textunderscore pous\textunderscore , \textunderscore podos\textunderscore )}
\end{itemize}
Gênero de orchideas.
\section{Cysne}
\begin{itemize}
\item {Grp. gram.:m.}
\end{itemize}
\begin{itemize}
\item {Utilização:Fig.}
\end{itemize}
\begin{itemize}
\item {Proveniência:(Do gr. \textunderscore kuknos\textunderscore )}
\end{itemize}
Ave palmípede e aquática do gênero do pato.
Poéta, orador ou músico célebre.
Constellação setentrional.
\section{Cysnífero}
\begin{itemize}
\item {Grp. gram.:adj.}
\end{itemize}
\begin{itemize}
\item {Utilização:Poét.}
\end{itemize}
Em que andam cysnes. Cf. \textunderscore Viriato Trág.\textunderscore , V. 91.
\section{Cystalgia}
\begin{itemize}
\item {Grp. gram.:f.}
\end{itemize}
\begin{itemize}
\item {Utilização:Med.}
\end{itemize}
\begin{itemize}
\item {Proveniência:(Do gr. \textunderscore kustos\textunderscore  + \textunderscore algos\textunderscore )}
\end{itemize}
Dôr nervosa na bexiga.
\section{Cystálgico}
\begin{itemize}
\item {Grp. gram.:adj.}
\end{itemize}
Relativo á cystalgia.
\section{Cystecercose}
\begin{itemize}
\item {Grp. gram.:f.}
\end{itemize}
Designação scientifica da doença produzida nos porcos pelo cysticerco.
\section{Cysticerco}
\begin{itemize}
\item {Grp. gram.:m.}
\end{itemize}
\begin{itemize}
\item {Proveniência:(Do gr. \textunderscore kustis\textunderscore  + \textunderscore kerkos\textunderscore )}
\end{itemize}
Animálculo dimorpo, ou estado por que passam os embryões dos vermes, como a tênia.--O cysticerco completa a sua evolução em duas phases: na primeira, (larvas), produz no porco a cistercose ou gafeira; na segunda, desenvolve-se no homem, produzindo a solitária ou tênia. As duas fórmas são solidárias entre si por contaminação mútua das espécies em que se albergam.
\section{Cýstico}
\begin{itemize}
\item {Grp. gram.:adj.}
\end{itemize}
\begin{itemize}
\item {Proveniência:(Do gr. \textunderscore kustis\textunderscore )}
\end{itemize}
Relativo á bexiga.
Relativo ao cysto: \textunderscore cavidade cýstica\textunderscore .
\section{Cystífero}
\begin{itemize}
\item {Grp. gram.:adj.}
\end{itemize}
\begin{itemize}
\item {Utilização:Hist. Nat.}
\end{itemize}
\begin{itemize}
\item {Proveniência:(Do gr. \textunderscore kustis\textunderscore  + lat. \textunderscore ferre\textunderscore )}
\end{itemize}
Que tem uma espécie de bexiga ou bolsa.
\section{Cystina}
\begin{itemize}
\item {Grp. gram.:f.}
\end{itemize}
\begin{itemize}
\item {Proveniência:(Do gr. \textunderscore kustis\textunderscore )}
\end{itemize}
Substância, produzida pela secreção dos rins e que constitue os cálculos vesicaes e renaes, e certos depósitos urinários.
\section{Cystinoso}
\begin{itemize}
\item {Grp. gram.:adj.}
\end{itemize}
Que contém cystina.
\section{Cystinuria}
\begin{itemize}
\item {Grp. gram.:f.}
\end{itemize}
Emissão de urina cystinosa.
\section{Cystite}
\begin{itemize}
\item {Grp. gram.:f.}
\end{itemize}
\begin{itemize}
\item {Proveniência:(Do gr. \textunderscore kustis\textunderscore )}
\end{itemize}
Inflammação da bexiga.
\section{Cystítomo}
\begin{itemize}
\item {Grp. gram.:m.}
\end{itemize}
Instrumento, para abrir a cápsula do crystallino do ôlho.
\section{Cysto}
\begin{itemize}
\item {Grp. gram.:m.}
\end{itemize}
\begin{itemize}
\item {Utilização:Med.}
\end{itemize}
\begin{itemize}
\item {Proveniência:(Do gr. \textunderscore kustis\textunderscore , bexiga)}
\end{itemize}
Espécie de tumor, ou membrana em fórma de bexiga sem abertura, cheia de líquido ou de outras substâncias anormaes, e que se fórma accidentalmente.--A fórma \textunderscore kysto\textunderscore , por alguns usada, não é portuguesa.
\section{Cystocele}
\begin{itemize}
\item {Grp. gram.:f.}
\end{itemize}
\begin{itemize}
\item {Proveniência:(Do gr. \textunderscore kustis\textunderscore  + \textunderscore kele\textunderscore )}
\end{itemize}
Hérnia da bexiga.
\section{Cystóide}
\begin{itemize}
\item {Grp. gram.:adj.}
\end{itemize}
\begin{itemize}
\item {Grp. gram.:Pl.}
\end{itemize}
\begin{itemize}
\item {Proveniência:(Do gr. \textunderscore kustis\textunderscore  + \textunderscore eidos\textunderscore )}
\end{itemize}
Semelhante a uma bexiga.
Animaes fósseis, da série mesozoica, que parecem têr sido o tronco principal das classes dos echinodermes.
\section{Cystólitho}
\begin{itemize}
\item {Grp. gram.:m.}
\end{itemize}
\begin{itemize}
\item {Utilização:Bot.}
\end{itemize}
\begin{itemize}
\item {Proveniência:(Do gr. \textunderscore kustis\textunderscore  + \textunderscore lithos\textunderscore )}
\end{itemize}
Corpúsculo ovoide, suspenso no interior da céllula vegetal e composto de concreções de cellulose da parede da mesma céllula.
\section{Cystoplegia}
\begin{itemize}
\item {Grp. gram.:f.}
\end{itemize}
\begin{itemize}
\item {Proveniência:(Do gr. \textunderscore kustis\textunderscore  + \textunderscore plessein\textunderscore )}
\end{itemize}
Paralysia da bexiga.
\section{Cystoplégico}
\begin{itemize}
\item {Grp. gram.:adj.}
\end{itemize}
Relativo á cystoplegia.
Aquelle que soffre cystoplegia.
\section{Cystoscopia}
\begin{itemize}
\item {Grp. gram.:f.}
\end{itemize}
\begin{itemize}
\item {Utilização:Med.}
\end{itemize}
Observação da bexiga, por meio do cystoscópio.
\section{Cystoscópio}
\begin{itemize}
\item {Grp. gram.:m.}
\end{itemize}
\begin{itemize}
\item {Proveniência:(Do gr. \textunderscore kustis\textunderscore  + \textunderscore skopein\textunderscore )}
\end{itemize}
Apparelho cirúrgico, recentemente inventado, para observações na bexiga.
\section{Cystotomia}
\begin{itemize}
\item {Grp. gram.:f.}
\end{itemize}
\begin{itemize}
\item {Proveniência:(De \textunderscore cistótomo\textunderscore )}
\end{itemize}
Operação cirúrgica, para tirar da bexiga os cálculos urinários ou outros corpos nocivos.
Lithotomia.
Operação da talha.
\section{Cystótomo}
\begin{itemize}
\item {Grp. gram.:m.}
\end{itemize}
\begin{itemize}
\item {Proveniência:(Do gr. \textunderscore krustis\textunderscore  + \textunderscore tome\textunderscore )}
\end{itemize}
Instrumento, que se usa na operação da talha.
\section{Cytíneas}
\begin{itemize}
\item {Grp. gram.:f. pl.}
\end{itemize}
\begin{itemize}
\item {Proveniência:(Do gr. \textunderscore kutinos\textunderscore )}
\end{itemize}
Família de plantas, que tem por typo a roman.
\section{Cytiso}
\begin{itemize}
\item {Grp. gram.:m.}
\end{itemize}
\begin{itemize}
\item {Proveniência:(Gr. \textunderscore kutisos\textunderscore )}
\end{itemize}
Espécie de trevo.
Luzerna arborescente.
\section{Cytochromo}
\begin{itemize}
\item {Grp. gram.:adj.}
\end{itemize}
\begin{itemize}
\item {Proveniência:(Do gr. \textunderscore kutus\textunderscore  + \textunderscore khroma\textunderscore )}
\end{itemize}
O mesmo que \textunderscore somatochromo\textunderscore .
\section{Cytode}
\begin{itemize}
\item {Grp. gram.:m.}
\end{itemize}
\begin{itemize}
\item {Proveniência:(Gr. \textunderscore kutodes\textunderscore )}
\end{itemize}
Elemento plástico dos tecidos orgânicos.
\section{Cytologia}
\begin{itemize}
\item {Grp. gram.:f.}
\end{itemize}
\begin{itemize}
\item {Utilização:Physiol.}
\end{itemize}
\begin{itemize}
\item {Proveniência:(Do gr. \textunderscore kutos\textunderscore  + \textunderscore logos\textunderscore )}
\end{itemize}
Estudo da céllula em geral.
\section{Cytoplasma}
\begin{itemize}
\item {Grp. gram.:m.}
\end{itemize}
\begin{itemize}
\item {Utilização:Physiol.}
\end{itemize}
\begin{itemize}
\item {Proveniência:(Do gr. \textunderscore kutos\textunderscore  + \textunderscore plassein\textunderscore )}
\end{itemize}
Nome, que se dá ao protoplasma ordinário da céllula, para o distinguir do nucleoplasma.
\section{Cyzicena}
\begin{itemize}
\item {Grp. gram.:f.}
\end{itemize}
\begin{itemize}
\item {Proveniência:(Lat. \textunderscore cyzícenus\textunderscore )}
\end{itemize}
Sala do jantar, entre os Gregos.
\section{Czar}
\end{document}