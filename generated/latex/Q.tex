
\begin{itemize}
\item {Proveniência: }
\end{itemize}\documentclass{article}
\usepackage[portuguese]{babel}
\title{Q}
\begin{document}
Mulhér, que adivinhava.
Sacerdotisa de Apollo.
Prophetisa.
\section{Quiota}
\begin{itemize}
\item {Grp. gram.:m.  e  f.}
\end{itemize}
Habitante da ilha de Chios.
\section{Quimoso}
\begin{itemize}
\item {Grp. gram.:adj.}
\end{itemize}
Relativo ao quimo.
\section{Q}
\begin{itemize}
\item {fónica:kê}
\end{itemize}
\begin{itemize}
\item {Grp. gram.:m.}
\end{itemize}
\begin{itemize}
\item {Grp. gram.:Adj.}
\end{itemize}
\begin{itemize}
\item {Utilização:Ant.}
\end{itemize}
Décima sétima letra do alphabeto português.
Que numa série de 17 occupa o último lugar.
Como letra numeral, valia 500, e, com um til por cima, 500:000.
O \textunderscore q\textunderscore , em português, só se usa seguido de \textunderscore u\textunderscore : \textunderscore quanto\textunderscore ; \textunderscore quente\textunderscore ; \textunderscore quinto\textunderscore ; \textunderscore quota\textunderscore ; \textunderscore quutiliquê\textunderscore .
Quando o grupo \textunderscore qu...\textunderscore  é seguido de \textunderscore e\textunderscore  ou \textunderscore i\textunderscore , não sôa geralmente o \textunderscore u\textunderscore :«\textunderscore quente\textunderscore , \textunderscore quinto...\textunderscore »Excepção: \textunderscore frequente\textunderscore , \textunderscore frequência\textunderscore , etc.
Quando o mesmo grupo é seguido de \textunderscore a\textunderscore  ou \textunderscore o\textunderscore , é regra soar o \textunderscore u\textunderscore : \textunderscore quando\textunderscore , \textunderscore quanto\textunderscore , \textunderscore quatro...\textunderscore 
Exceptua-se \textunderscore quatorze\textunderscore  e seus der., que se pronunciam \textunderscore catôrze\textunderscore , etc.
\textunderscore Quota\textunderscore , \textunderscore quotidiano\textunderscore , \textunderscore quodório\textunderscore , e talvez alguns outros, começam também a perder o som do \textunderscore u\textunderscore . Cf. Vasconcélloz, \textunderscore Gram. Port.\textunderscore , 40.
\section{Quacacuja}
\begin{itemize}
\item {Grp. gram.:m.}
\end{itemize}
Peixe esquamodermo do Brasil.
\section{Quadas}
\begin{itemize}
\item {Grp. gram.:m. pl.}
\end{itemize}
O mesmo que \textunderscore quados\textunderscore . Cf. D. Ant. da Costa, \textunderscore Três Mundos\textunderscore , 232. (2.^a ed.)
\section{Quaderna}
\begin{itemize}
\item {Grp. gram.:f.}
\end{itemize}
\begin{itemize}
\item {Utilização:Heráld.}
\end{itemize}
\begin{itemize}
\item {Grp. gram.:Pl.}
\end{itemize}
\begin{itemize}
\item {Proveniência:(Do lat. \textunderscore quaternus\textunderscore )}
\end{itemize}
Objecto, composto de quatro peças em quadrado, de ordinário em fórma de crescentes.
A face do dado, que apresenta quatro pontos.
Os quatro pontos de uma face dos dados.
\section{Quadernado}
\begin{itemize}
\item {Grp. gram.:adj.}
\end{itemize}
\begin{itemize}
\item {Utilização:Bot.}
\end{itemize}
\begin{itemize}
\item {Proveniência:(De \textunderscore quaderna\textunderscore )}
\end{itemize}
Diz-se das fôlhas ou flôres, dispostas a quatro e quatro na haste das plantas.
\section{Quaderno}
\begin{itemize}
\item {Grp. gram.:m.}
\end{itemize}
(Fórma des. de \textunderscore caderno\textunderscore )
\section{Quados}
\begin{itemize}
\item {Grp. gram.:m. pl.}
\end{itemize}
\begin{itemize}
\item {Proveniência:(Lat. \textunderscore quadi\textunderscore )}
\end{itemize}
Antigo povo germânico das margens do Danúbio.
\section{Quadra}
\begin{itemize}
\item {Grp. gram.:f.}
\end{itemize}
\begin{itemize}
\item {Utilização:Bras. do S}
\end{itemize}
\begin{itemize}
\item {Utilização:Prov.}
\end{itemize}
\begin{itemize}
\item {Utilização:alent.}
\end{itemize}
\begin{itemize}
\item {Proveniência:(Lat. \textunderscore quadra\textunderscore )}
\end{itemize}
Casa ou compartimento quadrado.
Cada uma das divisões de um jardim, dispostas em fórma quadrada.
Série de quatro.
Estância de quatro versos.
Quarteto.
Cada um dos lados de um quadrado.
Carta de jogar, com quatro pontos.
Parte larga de um navio, do lado da popa.
Bandeira do mastro grande da principal nau de uma esquadra.
Occasião, época: \textunderscore na quadra da primavera\textunderscore .
Porção de terreno, equivalente a 132 metros quadrados.
Lanço de muralha.
O mesmo que \textunderscore cavallariça\textunderscore .
\section{Quadrada}
\begin{itemize}
\item {Grp. gram.:f.}
\end{itemize}
\begin{itemize}
\item {Proveniência:(De \textunderscore quadrado\textunderscore )}
\end{itemize}
Compartimento, quadra. Cf. Rebello, \textunderscore Contos\textunderscore , 76.
\section{Quadrado}
\begin{itemize}
\item {Grp. gram.:adj.}
\end{itemize}
\begin{itemize}
\item {Utilização:Arith.}
\end{itemize}
\begin{itemize}
\item {Utilização:Fam.}
\end{itemize}
\begin{itemize}
\item {Grp. gram.:M.}
\end{itemize}
\begin{itemize}
\item {Proveniência:(De \textunderscore quadrar\textunderscore ^1)}
\end{itemize}
Que tem quatro lados iguais, formando ângulos rectos: \textunderscore uma mesa quadrada\textunderscore .
Que tem quatro lados proximamente iguais: \textunderscore uma belga quadrada\textunderscore .
Diz-se da raíz de um número, a qual, elevada á segunda potência, produz êsse número.
Diz-se do indivíduo baixo, mas espadaúdo.
Completo: \textunderscore é um idiota quadrado\textunderscore .
Figura geométrica, limitada por quatro linhas iguaes, formando ângulos rectos.
Segunda potência de um número, em Arithmética.
Disposição de tropas de infantaria, dando a apparência de um quadrado.
Peça de metal de typographia, fundida segundo o corpo dos caracteres respectivos, porém á altura dos espaços, e que serve para fechar e abrir os parágraphos e estabelecer os espaços brancos. (Há-os mais altos, para a composição destinada a reproduzir-se pelo galvanismo)
\section{Quadrador}
\begin{itemize}
\item {Grp. gram.:m.  e  adj.}
\end{itemize}
O que quadra.
O que faz quadros.
\section{Quadrados-de-imposição}
\begin{itemize}
\item {Grp. gram.:m. pl.}
\end{itemize}
Material typográphico, empregado na remendagem e imposição, fundido em corpo grande e a quadratins, formando collecções.--Tomam o nome de ocos, porque effectivamente o são, para não disperdiçar material e os tornar mais leves.
\section{Quadrados-ocos}
\begin{itemize}
\item {Grp. gram.:m. pl.}
\end{itemize}
Material typográphico, empregado na remendagem e imposição, fundido em corpo grande e a quadratins, formando collecções.--Tomam o nome de ocos, porque effectivamente o são, para não disperdiçar material e os tornar mais leves.
\section{Quadradura}
\begin{itemize}
\item {Grp. gram.:f.}
\end{itemize}
O mesmo que \textunderscore quadratura\textunderscore ^1.
\section{Quadragenário}
\begin{itemize}
\item {Grp. gram.:m.  e  adj.}
\end{itemize}
\begin{itemize}
\item {Proveniência:(Lat. \textunderscore quadragenarius\textunderscore )}
\end{itemize}
O que abrange quarenta unidades.
Que tem quarenta annos de idade.
\section{Quadragésima}
\begin{itemize}
\item {Grp. gram.:f.}
\end{itemize}
\begin{itemize}
\item {Utilização:Ant.}
\end{itemize}
\begin{itemize}
\item {Proveniência:(Lat. \textunderscore quadragesima\textunderscore )}
\end{itemize}
Período de quarenta dias.
Imposto romano de um por quarenta.
O mesmo que \textunderscore quaresma\textunderscore .
\section{Quadragesimal}
\begin{itemize}
\item {Grp. gram.:adj.}
\end{itemize}
\begin{itemize}
\item {Proveniência:(Lat. \textunderscore quadragesimalis\textunderscore )}
\end{itemize}
Relativo á quadragésima ou á quaresma.
\section{Quadragésimo}
\begin{itemize}
\item {Grp. gram.:adj.}
\end{itemize}
\begin{itemize}
\item {Proveniência:(Lat. \textunderscore quadragesimus\textunderscore )}
\end{itemize}
Que occupa o último lugar numa série de quarenta.
\section{Quadrangulado}
\begin{itemize}
\item {Grp. gram.:adj.}
\end{itemize}
\begin{itemize}
\item {Proveniência:(De \textunderscore quadri...\textunderscore  + \textunderscore angular\textunderscore )}
\end{itemize}
Que tem quatro ângulos.
\section{Quadrangular}
\begin{itemize}
\item {Grp. gram.:adj.}
\end{itemize}
\begin{itemize}
\item {Proveniência:(De \textunderscore quadri...\textunderscore  + \textunderscore angular\textunderscore )}
\end{itemize}
Que tem quatro ângulos.
\section{Quadrangularmente}
\begin{itemize}
\item {Grp. gram.:adv.}
\end{itemize}
De modo quadrangular.
\section{Quadrângulo}
\begin{itemize}
\item {Grp. gram.:m.}
\end{itemize}
\begin{itemize}
\item {Proveniência:(De \textunderscore quadri...\textunderscore  + \textunderscore ângulo\textunderscore )}
\end{itemize}
Quadrilátero; aquillo que é quadrangular.
\section{Quadrantal}
\begin{itemize}
\item {Grp. gram.:adj.}
\end{itemize}
\begin{itemize}
\item {Utilização:Des.}
\end{itemize}
\begin{itemize}
\item {Grp. gram.:M.}
\end{itemize}
\begin{itemize}
\item {Proveniência:(Lat. \textunderscore quadrantal\textunderscore )}
\end{itemize}
Que é quadrado em suas quatro faces, como o dado.
Antiga medida romana para líquidos, correspondente a 48 sextários.
\section{Quadrante}
\begin{itemize}
\item {Grp. gram.:m.}
\end{itemize}
\begin{itemize}
\item {Proveniência:(Lat. \textunderscore quadrans\textunderscore )}
\end{itemize}
Quarta parte da circunferência.
Mostrador de relógio.
\section{Quadrar}
\begin{itemize}
\item {Grp. gram.:v. t.}
\end{itemize}
\begin{itemize}
\item {Utilização:Arith.}
\end{itemize}
\begin{itemize}
\item {Grp. gram.:V. i.}
\end{itemize}
\begin{itemize}
\item {Proveniência:(Lat. \textunderscore quadrare\textunderscore )}
\end{itemize}
Dar fórma quadrada a.
Elevar (um número) ao quadrado.
Convir: \textunderscore isso não me quadra\textunderscore .
Adaptar-se.
Dar vantagem.
\section{Quadrar}
\begin{itemize}
\item {Grp. gram.:v. i.}
\end{itemize}
\begin{itemize}
\item {Proveniência:(De \textunderscore quadra\textunderscore )}
\end{itemize}
Perfilar-se na frente do toiro para collocar bandarilhas.
\section{Quadrático}
\begin{itemize}
\item {Grp. gram.:adj.}
\end{itemize}
\begin{itemize}
\item {Proveniência:(Do lat. \textunderscore quadratus\textunderscore )}
\end{itemize}
Relativo ao quadrado.
\section{Quadratim}
\begin{itemize}
\item {Grp. gram.:m.}
\end{itemize}
\begin{itemize}
\item {Utilização:Typ.}
\end{itemize}
\begin{itemize}
\item {Proveniência:(Do lat. \textunderscore quadratus\textunderscore )}
\end{itemize}
Peça quadrada de metal, fundida a uma linha do corpo a que pertence, e que se emprega na composição typográphica para abrir parágraphos ou determinar medidas.
\section{Quadrato}
\begin{itemize}
\item {Grp. gram.:m.  e  adj.}
\end{itemize}
\begin{itemize}
\item {Utilização:Ant.}
\end{itemize}
O mesmo que \textunderscore quadrado\textunderscore .
\section{Quadratriz}
\begin{itemize}
\item {Grp. gram.:f.  e  adj.}
\end{itemize}
\begin{itemize}
\item {Proveniência:(Do lat. \textunderscore quadratus\textunderscore )}
\end{itemize}
Curva, que serve para a resolução aproximada do problema da quadratura do círculo e para a trisecção do ângulo.
\section{Quadratura}
\begin{itemize}
\item {Grp. gram.:f.}
\end{itemize}
\begin{itemize}
\item {Proveniência:(Lat. \textunderscore quadratura\textunderscore )}
\end{itemize}
Reducção geométrica de uma superfície limitada por uma curva a um quadrado, que lhe seja equivalente em superfície.
Aspecto de dois astros, que distam entre si 90 graus.
\section{Quadratura}
\begin{itemize}
\item {Grp. gram.:f.}
\end{itemize}
\begin{itemize}
\item {Proveniência:(De \textunderscore quadrar\textunderscore )}
\end{itemize}
Pintura de ornatos de architectura.
\section{Quadraturista}
\begin{itemize}
\item {Grp. gram.:m.  e  f.}
\end{itemize}
Pessôa que pinta quadraturas^2.
\section{Quadrela}
\begin{itemize}
\item {Grp. gram.:f.}
\end{itemize}
\begin{itemize}
\item {Utilização:Ant.}
\end{itemize}
\begin{itemize}
\item {Proveniência:(De \textunderscore quadra\textunderscore )}
\end{itemize}
Lanço de qualquer edifício ou construcção.
Muro, parede.
Quadrilha.
Quadrilha de 20 homens.
Coirela, belga.
\section{Quadrelo}
\begin{itemize}
\item {fónica:drê}
\end{itemize}
\begin{itemize}
\item {Grp. gram.:m.}
\end{itemize}
\begin{itemize}
\item {Utilização:Ant.}
\end{itemize}
\begin{itemize}
\item {Proveniência:(De \textunderscore quadra\textunderscore )}
\end{itemize}
Seta de quatro faces, que se atirava com a bésta.
\section{Quadri...}
\begin{itemize}
\item {Grp. gram.:pref.}
\end{itemize}
\begin{itemize}
\item {Proveniência:(Do lat. \textunderscore quatuor\textunderscore )}
\end{itemize}
(designativo de \textunderscore quatro\textunderscore  ou de \textunderscore quadrado\textunderscore  ou de \textunderscore quádruplo\textunderscore )
\section{Quadrialado}
\begin{itemize}
\item {Grp. gram.:adj.}
\end{itemize}
\begin{itemize}
\item {Proveniência:(De \textunderscore quadri...\textunderscore  + \textunderscore alado\textunderscore )}
\end{itemize}
Que tem quatro asas.
\section{Quadribásico}
\begin{itemize}
\item {Grp. gram.:adj.}
\end{itemize}
\begin{itemize}
\item {Utilização:Chím.}
\end{itemize}
\begin{itemize}
\item {Proveniência:(De \textunderscore quadri...\textunderscore  + \textunderscore base\textunderscore )}
\end{itemize}
Diz-se do sal, que contém quatro proporções de base para uma proporção de ácido.
\section{Quadricapsular}
\begin{itemize}
\item {Grp. gram.:adj.}
\end{itemize}
\begin{itemize}
\item {Utilização:Bot.}
\end{itemize}
\begin{itemize}
\item {Proveniência:(De \textunderscore quadri...\textunderscore  + \textunderscore capsular\textunderscore )}
\end{itemize}
Que tem quatro cápsulas.
\section{Quadricarboneto}
\begin{itemize}
\item {fónica:nê}
\end{itemize}
\begin{itemize}
\item {Grp. gram.:m.}
\end{itemize}
\begin{itemize}
\item {Utilização:Chím.}
\end{itemize}
\begin{itemize}
\item {Proveniência:(De \textunderscore quadri...\textunderscore  + \textunderscore carboneto\textunderscore )}
\end{itemize}
Carboneto, que contém, em carbone, quatro vezes mais que outra combinação do mesmo gênero.
\section{Quadricellular}
\begin{itemize}
\item {Grp. gram.:adj.}
\end{itemize}
\begin{itemize}
\item {Utilização:Bot.}
\end{itemize}
Dividido em quatro céllulas.
\section{Quadricelular}
\begin{itemize}
\item {Grp. gram.:adj.}
\end{itemize}
\begin{itemize}
\item {Utilização:Bot.}
\end{itemize}
Dividido em quatro células.
\section{Quadricípite}
\begin{itemize}
\item {Grp. gram.:adj.}
\end{itemize}
\begin{itemize}
\item {Utilização:Anat.}
\end{itemize}
Diz-se de um músculo da coxa.
\section{Quadricolor}
\begin{itemize}
\item {Grp. gram.:adj.}
\end{itemize}
\begin{itemize}
\item {Proveniência:(De \textunderscore quadri...\textunderscore  + \textunderscore color\textunderscore )}
\end{itemize}
Que tem quatro côres differentes.
\section{Quadricórneo}
\begin{itemize}
\item {Grp. gram.:adj.}
\end{itemize}
\begin{itemize}
\item {Utilização:Zool.}
\end{itemize}
\begin{itemize}
\item {Proveniência:(De \textunderscore quadri...\textunderscore  + \textunderscore córneo\textunderscore )}
\end{itemize}
Que tem quatro antennas ou cornos.
\section{Quadricotiledóneo}
\begin{itemize}
\item {Grp. gram.:adj.}
\end{itemize}
\begin{itemize}
\item {Proveniência:(De \textunderscore quadri...\textunderscore  + \textunderscore cotiledóneo\textunderscore )}
\end{itemize}
Que tem quatro cotilédones.
\section{Quadricotyledóneo}
\begin{itemize}
\item {Grp. gram.:adj.}
\end{itemize}
\begin{itemize}
\item {Proveniência:(De \textunderscore quadri...\textunderscore  + \textunderscore cotyledóneo\textunderscore )}
\end{itemize}
Que tem quatro cotylédones.
\section{Quadrícula}
\begin{itemize}
\item {Grp. gram.:f.}
\end{itemize}
Pequeno quadrado; pequena quadra.
(Fem. de \textunderscore quadrículo\textunderscore )
\section{Quadriculado}
\begin{itemize}
\item {Grp. gram.:adj.}
\end{itemize}
Pautado ou dividido em quadrículos: \textunderscore papel quadriculado\textunderscore .
\section{Quadricular}
\begin{itemize}
\item {Grp. gram.:v. t.}
\end{itemize}
Dar fórma de quadrículos a; dividir em quadrículos.
\section{Quadricular}
\begin{itemize}
\item {Grp. gram.:adj.}
\end{itemize}
O mesmo que \textunderscore quadriculado\textunderscore .
\section{Quadrículo}
\begin{itemize}
\item {Grp. gram.:m.}
\end{itemize}
Pequeno quadrado; quadradinho.
(Dem. de \textunderscore quadro\textunderscore )
\section{Quadricúspide}
\begin{itemize}
\item {Grp. gram.:adj.}
\end{itemize}
\begin{itemize}
\item {Proveniência:(De \textunderscore quadri...\textunderscore  + \textunderscore cúspide\textunderscore )}
\end{itemize}
Que tem quatro pontas agudas.
\section{Quadridentado}
\begin{itemize}
\item {Grp. gram.:adj.}
\end{itemize}
\begin{itemize}
\item {Proveniência:(De \textunderscore quadri...\textunderscore  + \textunderscore dentado\textunderscore )}
\end{itemize}
Que tem quatro dentes.
\section{Quadridente}
\begin{itemize}
\item {Grp. gram.:m.}
\end{itemize}
\begin{itemize}
\item {Proveniência:(Lat. \textunderscore quadridens\textunderscore )}
\end{itemize}
Peixe osteodermo.
\section{Quadridigitado}
\begin{itemize}
\item {Grp. gram.:adj.}
\end{itemize}
\begin{itemize}
\item {Proveniência:(De \textunderscore quadri...\textunderscore  + \textunderscore digitado\textunderscore )}
\end{itemize}
Que tem quatro dedos ou digitações.
\section{Quadrienal}
\begin{itemize}
\item {Grp. gram.:adj.}
\end{itemize}
Que sucede de quatro em quatro annos.
Relativo ao quadriênnio.
\section{Quadriênio}
\begin{itemize}
\item {Grp. gram.:m.}
\end{itemize}
\begin{itemize}
\item {Proveniência:(Lat. \textunderscore quadriennium\textunderscore )}
\end{itemize}
Espaço de quatro anos.
\section{Quadriennal}
\begin{itemize}
\item {Grp. gram.:adj.}
\end{itemize}
Que succede de quatro em quatro annos.
Relativo ao quadriênnio.
\section{Quadriênnio}
\begin{itemize}
\item {Grp. gram.:m.}
\end{itemize}
\begin{itemize}
\item {Proveniência:(Lat. \textunderscore quadriennium\textunderscore )}
\end{itemize}
Espaço de quatro annos.
\section{Quadrifendido}
\begin{itemize}
\item {Grp. gram.:adj.}
\end{itemize}
\begin{itemize}
\item {Proveniência:(Lat. \textunderscore quadrifidus\textunderscore )}
\end{itemize}
Que é fendido ou dividido em quatro partes iguaes.
Fendido em quatro partes.
Que tem quatro profundas divisões.
\section{Quadrífido}
\begin{itemize}
\item {Grp. gram.:adj.}
\end{itemize}
\begin{itemize}
\item {Proveniência:(Lat. \textunderscore quadrifidus\textunderscore )}
\end{itemize}
Que é fendido ou dividido em quatro partes iguaes.
Fendido em quatro partes.
Que tem quatro profundas divisões.
\section{Quadriflóreo}
\begin{itemize}
\item {Grp. gram.:adj.}
\end{itemize}
\begin{itemize}
\item {Utilização:Bot.}
\end{itemize}
\begin{itemize}
\item {Proveniência:(De \textunderscore quadri...\textunderscore  + \textunderscore flóreo\textunderscore )}
\end{itemize}
Que tem quatro flôres.
Cujas flôres estão dispostas a quatro e quatro.
\section{Quadrifoliado}
\begin{itemize}
\item {Grp. gram.:adj.}
\end{itemize}
\begin{itemize}
\item {Utilização:Bot.}
\end{itemize}
\begin{itemize}
\item {Proveniência:(De \textunderscore quadri...\textunderscore  + \textunderscore foliado\textunderscore )}
\end{itemize}
Que tem quatro folíolos.
\section{Quadrifólio}
\begin{itemize}
\item {Grp. gram.:adj.}
\end{itemize}
\begin{itemize}
\item {Utilização:Bot.}
\end{itemize}
\begin{itemize}
\item {Proveniência:(Do lat. \textunderscore quatuor\textunderscore  + \textunderscore folium\textunderscore )}
\end{itemize}
Que tem quatro fôlhas.
Cujas fôlhas estão dispostas a quatro e quatro.
\section{Quadriforcado}
\begin{itemize}
\item {Grp. gram.:adj.}
\end{itemize}
\begin{itemize}
\item {Proveniência:(Do lat. \textunderscore quatuor\textunderscore  + \textunderscore furca\textunderscore )}
\end{itemize}
Que tem quatro ramos.
\section{Quadriforme}
\begin{itemize}
\item {Grp. gram.:adj.}
\end{itemize}
\begin{itemize}
\item {Proveniência:(De \textunderscore quadri...\textunderscore  + \textunderscore forma\textunderscore )}
\end{itemize}
Que apresenta quatro fórmas.
Produzido pela combinação de quatro fórmas crystallinas, (falando-se de mineraes).
\section{Quadrifronte}
\begin{itemize}
\item {Grp. gram.:adj.}
\end{itemize}
\begin{itemize}
\item {Utilização:Poét.}
\end{itemize}
\begin{itemize}
\item {Proveniência:(De \textunderscore quadri...\textunderscore  + \textunderscore fronte\textunderscore )}
\end{itemize}
Que tem quatro frontes.
\section{Quadriga}
\begin{itemize}
\item {Grp. gram.:f.}
\end{itemize}
\begin{itemize}
\item {Proveniência:(Lat. \textunderscore quadriga\textunderscore )}
\end{itemize}
Quatro cavallos, que puxam um carro.
Carro, tirado por quatro cavallos.
\section{Quadrigário}
\begin{itemize}
\item {Grp. gram.:m.}
\end{itemize}
\begin{itemize}
\item {Proveniência:(Lat. \textunderscore quadrigarius\textunderscore )}
\end{itemize}
Conductor de quadriga.
\section{Quadrigêmeo}
\begin{itemize}
\item {Grp. gram.:adj.}
\end{itemize}
\begin{itemize}
\item {Utilização:Anat.}
\end{itemize}
\begin{itemize}
\item {Proveniência:(Lat. \textunderscore quadrigeminus\textunderscore )}
\end{itemize}
Diz-se dos tubérculos, cujas saliências se apresentam em número de quatro.
\section{Quadrigeminado}
\begin{itemize}
\item {Grp. gram.:adj.}
\end{itemize}
\begin{itemize}
\item {Proveniência:(Do lat. \textunderscore quatuor\textunderscore  + \textunderscore geminatus\textunderscore )}
\end{itemize}
Diz-se dos órgãos vegetaes, dispostos no mesmo nível, a quatro e quatro.
\section{Quadrigúmeo}
\begin{itemize}
\item {Grp. gram.:adj.}
\end{itemize}
\begin{itemize}
\item {Proveniência:(De \textunderscore quadri...\textunderscore  + \textunderscore gume\textunderscore )}
\end{itemize}
Que tem quatro gumes.
\section{Quadrijugado}
\begin{itemize}
\item {Grp. gram.:adj.}
\end{itemize}
\begin{itemize}
\item {Utilização:Bot.}
\end{itemize}
\begin{itemize}
\item {Proveniência:(Do lat. \textunderscore quatuor\textunderscore  + \textunderscore jugatus\textunderscore )}
\end{itemize}
Que tem quatro pares de folíolos oppostos.
\section{Quadríjugo}
\begin{itemize}
\item {Grp. gram.:adj.}
\end{itemize}
\begin{itemize}
\item {Utilização:Poét.}
\end{itemize}
\begin{itemize}
\item {Proveniência:(Lat. \textunderscore quadrijugus\textunderscore )}
\end{itemize}
Puxado por quatro cavallos.
\section{Quadril}
\begin{itemize}
\item {Grp. gram.:m.}
\end{itemize}
\begin{itemize}
\item {Proveniência:(De \textunderscore quadro\textunderscore )}
\end{itemize}
Região lateral do corpo humano, entre a cintura e a articulação superior da coxa; anca.
\section{Quadrilateral}
\begin{itemize}
\item {Grp. gram.:adj.}
\end{itemize}
\begin{itemize}
\item {Proveniência:(De \textunderscore quadri...\textunderscore  + \textunderscore lateral\textunderscore )}
\end{itemize}
Que tem quatro lados.
\section{Quadrilátero}
\begin{itemize}
\item {Grp. gram.:adj.}
\end{itemize}
\begin{itemize}
\item {Grp. gram.:M.}
\end{itemize}
\begin{itemize}
\item {Proveniência:(Lat. \textunderscore quadrilaterus\textunderscore )}
\end{itemize}
O mesmo que \textunderscore quadrilateral\textunderscore .
Figura de quatro lados.
Espaço quadrangular, fortificado.
\section{Quadrilha}
\begin{itemize}
\item {Grp. gram.:f.}
\end{itemize}
\begin{itemize}
\item {Utilização:Prov.}
\end{itemize}
\begin{itemize}
\item {Utilização:alent.}
\end{itemize}
\begin{itemize}
\item {Utilização:des.}
\end{itemize}
\begin{itemize}
\item {Utilização:Bras}
\end{itemize}
\begin{itemize}
\item {Utilização:Pop.}
\end{itemize}
Conjunto de quatro ou mais cavalleiros, que se dispunham para o jôgo das canas.
Cavalhada.
Trôço de guerreiros.
Circunscripção territorial, vigiada por quadrilheiro ou rondador.
Pequena frota, esquadrilha.
Bando de ladrões ou salteadores, subordinados a um chefe.
Matilha.
Multidão.
Contradança.
Pares, que executam uma contradança.
Peças musicaes, correspondentes á contradança.
Grupo de carrêtas, puxadas por toiros.
Manada de cavallos.
Súcia, corja.
(Cast. \textunderscore cuadrilha\textunderscore )
\section{Quadrilhado}
\begin{itemize}
\item {Grp. gram.:adj.}
\end{itemize}
\begin{itemize}
\item {Utilização:Bras}
\end{itemize}
\begin{itemize}
\item {Proveniência:(De \textunderscore quadrilhar\textunderscore )}
\end{itemize}
Diz-se do papel quadriculado. Cf. \textunderscore Jorn.-do-Comm.\textunderscore , do Rio, de 11-VIII-902.
\section{Quadrilhar}
\begin{itemize}
\item {Grp. gram.:v. i.}
\end{itemize}
\begin{itemize}
\item {Utilização:Ant.}
\end{itemize}
O mesmo que \textunderscore quadrar\textunderscore ^1:«\textunderscore e pois que vosso louvor a todo o mundo quadrilha...\textunderscore »\textunderscore Auto de S. Aleixo\textunderscore .
\section{Quadrilheiro}
\begin{itemize}
\item {Grp. gram.:m.}
\end{itemize}
\begin{itemize}
\item {Utilização:Ant.}
\end{itemize}
\begin{itemize}
\item {Utilização:Prov.}
\end{itemize}
\begin{itemize}
\item {Utilização:alent.}
\end{itemize}
\begin{itemize}
\item {Utilização:des.}
\end{itemize}
\begin{itemize}
\item {Grp. gram.:Adj.}
\end{itemize}
\begin{itemize}
\item {Proveniência:(De \textunderscore quadrilha\textunderscore )}
\end{itemize}
Aquelle que pertence a uma quadrilha de ladrões.
Membro de uma quadrilha de guerreiros ou de jogadores das canas.
Rondador, esbirro:«\textunderscore sendo quadrilheiro, acudira hũa noyte...\textunderscore »\textunderscore Alvará de D. Seb.\textunderscore , in \textunderscore Rev. Lus.\textunderscore , XV, 120.
Aquelle que distribuía os despojos de guerra.
Aquelle que conduzia as carrêtas, que formavam a quadrilha.
Próprio de quadrilheiro ou de salteador. Cf. Filinto, III, 33.
\section{Quadrilião}
\begin{itemize}
\item {Grp. gram.:m.}
\end{itemize}
O mesmo ou melhor que \textunderscore quatrilião\textunderscore .
(Cp. ingl. \textunderscore quadrillion\textunderscore )
\section{Quadrillião}
\begin{itemize}
\item {Grp. gram.:m.}
\end{itemize}
O mesmo ou melhor que \textunderscore quatrillião\textunderscore .
(Cp. ingl. \textunderscore quadrillion\textunderscore )
\section{Quadrilobado}
\begin{itemize}
\item {Grp. gram.:adj.}
\end{itemize}
\begin{itemize}
\item {Proveniência:(De \textunderscore quadri...\textunderscore  + \textunderscore lobulado\textunderscore )}
\end{itemize}
Que tem quatro lóbulos.
\section{Quadrilobulado}
\begin{itemize}
\item {Grp. gram.:adj.}
\end{itemize}
\begin{itemize}
\item {Utilização:Hist. Nat.}
\end{itemize}
\begin{itemize}
\item {Proveniência:(De \textunderscore quadri...\textunderscore  + \textunderscore lobulado\textunderscore )}
\end{itemize}
Que tem quatro lóbulos.
\section{Quadrilóbulo}
\begin{itemize}
\item {Grp. gram.:m.}
\end{itemize}
\begin{itemize}
\item {Proveniência:(De \textunderscore quadri...\textunderscore  + \textunderscore lóbulo\textunderscore )}
\end{itemize}
Ornato architectónico, formado por quatro porções ligadas de arcos ogivaes.
\section{Quadriloculado}
\begin{itemize}
\item {Grp. gram.:adj.}
\end{itemize}
\begin{itemize}
\item {Utilização:Hist. Nat.}
\end{itemize}
\begin{itemize}
\item {Proveniência:(De \textunderscore quadri...\textunderscore  + \textunderscore loculado\textunderscore )}
\end{itemize}
Que tem quatro lóculos ou cavidades.
\section{Quadrilocular}
\begin{itemize}
\item {Grp. gram.:adj.}
\end{itemize}
O mesmo que \textunderscore quadriloculado\textunderscore .
\section{Quadrilongo}
\begin{itemize}
\item {Grp. gram.:adj.}
\end{itemize}
\begin{itemize}
\item {Grp. gram.:M.}
\end{itemize}
\begin{itemize}
\item {Grp. gram.:Pl.}
\end{itemize}
\begin{itemize}
\item {Proveniência:(De \textunderscore quadri...\textunderscore  + \textunderscore longo\textunderscore )}
\end{itemize}
Que tem quatro lados parallelos dois a dois, sendo dois dos parallelos maiores que os outros dois.
Quadrilátero, com aquella propriedade.
O mesmo que \textunderscore quadrados-ocos\textunderscore .
\section{Quadrilunulado}
\begin{itemize}
\item {Grp. gram.:adj.}
\end{itemize}
\begin{itemize}
\item {Proveniência:(De \textunderscore quadri...\textunderscore  + \textunderscore lunulado\textunderscore )}
\end{itemize}
Que tem quatro malhas em fórma de crescente.
\section{Quadrímano}
\begin{itemize}
\item {Grp. gram.:adj.}
\end{itemize}
\begin{itemize}
\item {Utilização:Zool.}
\end{itemize}
\begin{itemize}
\item {Grp. gram.:M. Pl.}
\end{itemize}
\begin{itemize}
\item {Proveniência:(Do lat. \textunderscore quatuor\textunderscore  + \textunderscore manus\textunderscore )}
\end{itemize}
Que tem quatro tarsos, dilatados em fórma de mãos.
Tríbo de insectos coleópteros quadrímanos.
\section{Quadrimembre}
\begin{itemize}
\item {Grp. gram.:adj.}
\end{itemize}
Que tem quatro membros. Cf. Rui Barb., \textunderscore Réplica\textunderscore , 158.
\section{Quadrimestral}
\begin{itemize}
\item {Grp. gram.:adj.}
\end{itemize}
Relativo a quadrimestre.
Que se realiza, ou acontece, de quatro em quatro meses.
\section{Quadrimestre}
\begin{itemize}
\item {Grp. gram.:m.}
\end{itemize}
\begin{itemize}
\item {Proveniência:(Lat. \textunderscore quadrimestris\textunderscore )}
\end{itemize}
Espaço de quatro meses.
\section{Quadrimosqueado}
\begin{itemize}
\item {Grp. gram.:adj.}
\end{itemize}
\begin{itemize}
\item {Utilização:Hist. Nat.}
\end{itemize}
\begin{itemize}
\item {Proveniência:(De \textunderscore quadri...\textunderscore  + \textunderscore mosqueado\textunderscore )}
\end{itemize}
Que tem quatro manchas.
\section{Quadringentenário}
\begin{itemize}
\item {Grp. gram.:m.}
\end{itemize}
Commemoração de um facto importante, succedido quatrocentos annos antes.
(Cp. \textunderscore quadringentésimo\textunderscore )
\section{Quadringentésimo}
\begin{itemize}
\item {Grp. gram.:adj.}
\end{itemize}
\begin{itemize}
\item {Proveniência:(Lat. \textunderscore quadringentesimus\textunderscore )}
\end{itemize}
Que occupa o último lugar numa série de quatrocentos.
Diz-se de cada uma das quatrocentas partes, em que se póde dividir um todo.
\section{Quadrinómio}
\begin{itemize}
\item {Grp. gram.:m.}
\end{itemize}
\begin{itemize}
\item {Proveniência:(De \textunderscore quadri...\textunderscore  + gr. \textunderscore nomos\textunderscore )}
\end{itemize}
Expressão algébrica, formada de quatro termos.
\section{Quadripartição}
\begin{itemize}
\item {Grp. gram.:f.}
\end{itemize}
\begin{itemize}
\item {Proveniência:(De \textunderscore quadri...\textunderscore  + \textunderscore partição\textunderscore )}
\end{itemize}
Qualidade do que é quadripartido.
\section{Quadripartido}
\begin{itemize}
\item {Grp. gram.:adj.}
\end{itemize}
\begin{itemize}
\item {Proveniência:(De \textunderscore quadri...\textunderscore  + \textunderscore partido\textunderscore )}
\end{itemize}
O mesmo que \textunderscore quadrífido\textunderscore .
\section{Quadripartito}
\begin{itemize}
\item {Grp. gram.:adj.}
\end{itemize}
\begin{itemize}
\item {Proveniência:(Do lat. \textunderscore quatuor\textunderscore  + \textunderscore partitus\textunderscore )}
\end{itemize}
O mesmo que \textunderscore quadrífido\textunderscore .
\section{Quadripenado}
\begin{itemize}
\item {Grp. gram.:adj.}
\end{itemize}
\begin{itemize}
\item {Proveniência:(De \textunderscore quadri...\textunderscore  + \textunderscore penado\textunderscore )}
\end{itemize}
Que tem quatro asas ou apêndices semelhantes a asas.
\section{Quadripennado}
\begin{itemize}
\item {Grp. gram.:adj.}
\end{itemize}
\begin{itemize}
\item {Proveniência:(De \textunderscore quadri...\textunderscore  + \textunderscore pennado\textunderscore )}
\end{itemize}
Que tem quatro asas ou appêndices semelhantes a asas.
\section{Quadripétalo}
\begin{itemize}
\item {Grp. gram.:adj.}
\end{itemize}
\begin{itemize}
\item {Proveniência:(De \textunderscore quadri...\textunderscore  + \textunderscore pétala\textunderscore )}
\end{itemize}
Que tem quatro pétalas.
\section{Quadrireme}
\begin{itemize}
\item {fónica:rê}
\end{itemize}
\begin{itemize}
\item {Grp. gram.:f.}
\end{itemize}
\begin{itemize}
\item {Proveniência:(Lat. \textunderscore quadriremis\textunderscore )}
\end{itemize}
Galera de quatro ordens de remos.
\section{Quadrirreme}
\begin{itemize}
\item {Grp. gram.:f.}
\end{itemize}
\begin{itemize}
\item {Proveniência:(Lat. \textunderscore quadriremis\textunderscore )}
\end{itemize}
Galera de quatro ordens de remos.
\section{Quadrissilábico}
\begin{itemize}
\item {Grp. gram.:adj.}
\end{itemize}
O mesmo que \textunderscore quadrissílabo\textunderscore .
\section{Quadrissílabo}
\begin{itemize}
\item {Grp. gram.:adj.}
\end{itemize}
\begin{itemize}
\item {Proveniência:(Do lat. \textunderscore quatuor\textunderscore  + \textunderscore silaba\textunderscore )}
\end{itemize}
Que tem quatro sílabas.
\section{Quadrissulco}
\begin{itemize}
\item {fónica:sul}
\end{itemize}
\begin{itemize}
\item {Grp. gram.:adj.}
\end{itemize}
\begin{itemize}
\item {Utilização:Bot.}
\end{itemize}
\begin{itemize}
\item {Utilização:Zool.}
\end{itemize}
\begin{itemize}
\item {Proveniência:(De \textunderscore quadri...\textunderscore  + \textunderscore sulco\textunderscore )}
\end{itemize}
Que apresenta quatro sulcos.
Que tem o pé dividido em quatro dedos.
\section{Quadrisulco}
\begin{itemize}
\item {fónica:sul}
\end{itemize}
\begin{itemize}
\item {Grp. gram.:adj.}
\end{itemize}
\begin{itemize}
\item {Utilização:Bot.}
\end{itemize}
\begin{itemize}
\item {Utilização:Zool.}
\end{itemize}
\begin{itemize}
\item {Proveniência:(De \textunderscore quadri...\textunderscore  + \textunderscore sulco\textunderscore )}
\end{itemize}
Que apresenta quatro sulcos.
Que tem o pé dividido em quatro dedos.
\section{Quadrisyllábico}
\begin{itemize}
\item {fónica:si}
\end{itemize}
\begin{itemize}
\item {Grp. gram.:adj.}
\end{itemize}
O mesmo que \textunderscore quadrisýllabo\textunderscore .
\section{Quadrisýllabo}
\begin{itemize}
\item {fónica:si}
\end{itemize}
\begin{itemize}
\item {Grp. gram.:adj.}
\end{itemize}
\begin{itemize}
\item {Proveniência:(Do lat. \textunderscore quatuor\textunderscore  + \textunderscore syllaba\textunderscore )}
\end{itemize}
Que tem quatro sýllabas.
\section{Quadrivalve}
\begin{itemize}
\item {Grp. gram.:adj.}
\end{itemize}
\begin{itemize}
\item {Proveniência:(De \textunderscore quadri...\textunderscore  + \textunderscore valva\textunderscore )}
\end{itemize}
Que tem quatro valvas.
\section{Quadrivalvulado}
\begin{itemize}
\item {Grp. gram.:adj.}
\end{itemize}
\begin{itemize}
\item {Proveniência:(De \textunderscore quadri...\textunderscore  + \textunderscore válvula\textunderscore )}
\end{itemize}
Que tem quatro válvulas.
\section{Quadrivalvular}
\begin{itemize}
\item {Grp. gram.:adj.}
\end{itemize}
O mesmo que \textunderscore quadrivalvulado\textunderscore .
\section{Quadrívio}
\begin{itemize}
\item {Grp. gram.:m.}
\end{itemize}
\begin{itemize}
\item {Utilização:Des.}
\end{itemize}
\begin{itemize}
\item {Proveniência:(Lat. \textunderscore quadrivium\textunderscore )}
\end{itemize}
Lugar, onde se cruzam dois caminhos.
Conjunto de quatro disciplinas, que eram a Arithmética, a Geometria, a Música e a Astronomia.
\section{Quadro}
\begin{itemize}
\item {Grp. gram.:m.}
\end{itemize}
\begin{itemize}
\item {Utilização:Geom.}
\end{itemize}
\begin{itemize}
\item {Grp. gram.:Adj.}
\end{itemize}
\begin{itemize}
\item {Utilização:Ant.}
\end{itemize}
\begin{itemize}
\item {Proveniência:(Lat. \textunderscore quadrum\textunderscore )}
\end{itemize}
Aquillo que tem quatro lados.
Quadrado.
Moldura ou caixilho que contém painel ou pintura.
Tela, painel, qualquer obra de pintura emmoldurada.
Espaço quadrado.
Mappa, ou objecto análogo, em que se figuram ou descrevem lugares ou factos.
Tabella.
Resenha, synopse.
Lista; relação.
Peça quadrada, de ardósia, ou de madeira pintada de preto, para nella se escreverem cálculos, para se traçarem figuras geométricas, etc.
Panorama.
Scena.
Sub-divisão de uma peça theatral ou dos actos da mesma peça.
Número máximo e o conjunto dos funccionários de uma Repartição ou profissão.
Grupo de pessôas, que mantém certa attitude por algum tempo.
Plano, sôbre que se traça a perspectiva.
O mesmo que \textunderscore quadrado\textunderscore . Cf. B. Pereira, \textunderscore Prosodia\textunderscore , vb. \textunderscore modius agri\textunderscore .
\section{Quadrobo}
\begin{itemize}
\item {fónica:drô}
\end{itemize}
\begin{itemize}
\item {Grp. gram.:m.}
\end{itemize}
Us. por Filinto, XIII, 32, em vez de \textunderscore quádruplo\textunderscore .
(Formação arbitrária, de \textunderscore quadra\textunderscore  + \textunderscore dôbro\textunderscore )
\section{Quadro-de-calçar}
\begin{itemize}
\item {Grp. gram.:m.}
\end{itemize}
\begin{itemize}
\item {Utilização:Typ.}
\end{itemize}
Marco com um crystal por apoio e formando ponte sôbre o crystal, tendo o vão da ponte a altura da letra, e que serve para calçar sufficientemente á gravura, antes de intercalada no texto, e dêste modo evitar o apertar e desapertar da fôrma na máquina.
\section{Quadrúmano}
\begin{itemize}
\item {Grp. gram.:adj.}
\end{itemize}
\begin{itemize}
\item {Grp. gram.:M. Pl.}
\end{itemize}
\begin{itemize}
\item {Proveniência:(Do lat. \textunderscore quatuor\textunderscore  + \textunderscore manus\textunderscore )}
\end{itemize}
Que tem quatro mãos.
Ordem de mammíferos que, á semelhança dos macacos, têm o dedo pollegar separado dos demais, nos pés e nas mãos.
\section{Quadrúmviro}
\textunderscore m.\textunderscore  (e der.)
O mesmo que \textunderscore quatuórviro\textunderscore , etc.
\section{Quadrúnviro}
\textunderscore m.\textunderscore  (e der.)
O mesmo que \textunderscore quatuórviro\textunderscore , etc.
\section{Quadrupedante}
\begin{itemize}
\item {Grp. gram.:adj.}
\end{itemize}
\begin{itemize}
\item {Proveniência:(Lat. \textunderscore quadrupedans\textunderscore )}
\end{itemize}
Que anda em quatro pés.
Que monta em quadrúpedes.
Relativo a quadrúpedes.
\section{Quadrupedar}
\begin{itemize}
\item {Grp. gram.:v. i.}
\end{itemize}
\begin{itemize}
\item {Proveniência:(Lat. \textunderscore quadrupedare\textunderscore )}
\end{itemize}
Andar em quatro pés.
Fazer estrépito com os pés, como os quadrúpedes quando andam.
\section{Quadrúpede}
\begin{itemize}
\item {Grp. gram.:adj.}
\end{itemize}
\begin{itemize}
\item {Grp. gram.:M.}
\end{itemize}
\begin{itemize}
\item {Utilização:Fig.}
\end{itemize}
\begin{itemize}
\item {Proveniência:(Lat. \textunderscore quadrupes\textunderscore )}
\end{itemize}
Que tem quatro pés.
Mammífero, que anda em quatro pés.
Homem estúpido, tolo; bruto.
\section{Quadrúpeo}
\begin{itemize}
\item {Grp. gram.:m.}
\end{itemize}
\begin{itemize}
\item {Utilização:Des.}
\end{itemize}
Animal de quatro pés.
O mesmo que \textunderscore quadrúpede\textunderscore :«\textunderscore homens, vermes, quadrupeos, e elephantes...\textunderscore »Filinto, XII, 137.
\section{Quadrupleta}
\begin{itemize}
\item {fónica:plê}
\end{itemize}
\begin{itemize}
\item {Grp. gram.:f.}
\end{itemize}
\begin{itemize}
\item {Proveniência:(De \textunderscore quádruplo\textunderscore )}
\end{itemize}
Velocípede de duas rodas, para quatro pessôas.
\section{Quadruplicação}
\begin{itemize}
\item {Grp. gram.:f.}
\end{itemize}
\begin{itemize}
\item {Proveniência:(Lat. \textunderscore quadruplicatio\textunderscore )}
\end{itemize}
Acto ou effeito de quadruplicar.
\section{Quadruplicadamente}
\begin{itemize}
\item {Grp. gram.:adv.}
\end{itemize}
\begin{itemize}
\item {Proveniência:(De \textunderscore quadruplicado\textunderscore )}
\end{itemize}
Com quadruplicação.
\section{Quadruplicado}
\begin{itemize}
\item {Grp. gram.:adj.}
\end{itemize}
\begin{itemize}
\item {Proveniência:(De \textunderscore quadruplicar\textunderscore )}
\end{itemize}
Multiplicado por quatro.
Que é quatro vezes maior do que era.
\section{Quadruplicar}
\begin{itemize}
\item {Grp. gram.:v. t.}
\end{itemize}
\begin{itemize}
\item {Proveniência:(Lat. \textunderscore quadruplicare\textunderscore )}
\end{itemize}
Tornar maior quatro vezes.
Multiplicar por quatro.
\section{Quádruplo}
\begin{itemize}
\item {Grp. gram.:adj.}
\end{itemize}
\begin{itemize}
\item {Grp. gram.:M.}
\end{itemize}
\begin{itemize}
\item {Proveniência:(Lat. \textunderscore quadruplus\textunderscore )}
\end{itemize}
Que é quatro vezes maior que outro.
Aquillo que é quatro vezes maior que outro.
\section{Quaira}
\begin{itemize}
\item {Grp. gram.:f.}
\end{itemize}
Antiga medida de cereaes. Cf. Herculano, \textunderscore Hist. de Port.\textunderscore , IV, 58.
(Relaciona-se com \textunderscore alqueire\textunderscore ?)
\section{Qual}
\begin{itemize}
\item {Grp. gram.:adj.}
\end{itemize}
\begin{itemize}
\item {Proveniência:(Lat. \textunderscore qualis\textunderscore )}
\end{itemize}
(designativo de \textunderscore qualidade\textunderscore  ou \textunderscore natureza\textunderscore )
Que coisa, que pessôa: \textunderscore qual preferes\textunderscore ?
Que.
Como: \textunderscore devanear, qual poéta\textunderscore .
Semelhante.
\section{Qualidade}
\begin{itemize}
\item {Grp. gram.:f.}
\end{itemize}
\begin{itemize}
\item {Proveniência:(Lat. \textunderscore qualitas\textunderscore )}
\end{itemize}
Aquillo que caracteriza uma coisa.
Modo de sêr.
Disposição moral.
Predicado: \textunderscore tem brilhantes qualidades\textunderscore .
Nobreza: \textunderscore pessôa de qualidade\textunderscore .
Casta, espécie: \textunderscore uva de má qualidade\textunderscore .
Gravidade.
Aptidão.
\section{Qualificação}
\begin{itemize}
\item {Grp. gram.:f.}
\end{itemize}
Acto ou effeito de qualificar.
\section{Qualificadamente}
\begin{itemize}
\item {Grp. gram.:adv.}
\end{itemize}
De modo qualificado; distintamente.
\section{Qualificado}
\begin{itemize}
\item {Grp. gram.:adj.}
\end{itemize}
Que tem certas qualidades.
Distinto.
Que está em posição elevada.
Nobre.
\section{Qualificador}
\begin{itemize}
\item {Grp. gram.:m.  e  adj.}
\end{itemize}
O que qualifica.
\section{Qualificar}
\begin{itemize}
\item {Grp. gram.:v. t.}
\end{itemize}
Indicar a qualidade de.
Attribuir uma qualidade a.
Classificar.
Attribuir um título a; tornar illustre.
(B. lat. \textunderscore qualificare\textunderscore )
\section{Qualificativamente}
\begin{itemize}
\item {Grp. gram.:adv.}
\end{itemize}
De modo qualificativo.
Exprimindo qualidade.
\section{Qualificativo}
\begin{itemize}
\item {Grp. gram.:adj.}
\end{itemize}
Que qualifica; que exprime qualidade.
\section{Qualificável}
\begin{itemize}
\item {Grp. gram.:adj.}
\end{itemize}
Que se póde qualificar.
\section{Qualitativo}
\begin{itemize}
\item {Grp. gram.:adj.}
\end{itemize}
\begin{itemize}
\item {Proveniência:(Do lat. \textunderscore qualitas\textunderscore )}
\end{itemize}
O mesmo que \textunderscore qualificativo\textunderscore .
\section{Qualquer}
\begin{itemize}
\item {Grp. gram.:adj.}
\end{itemize}
\begin{itemize}
\item {Proveniência:(De \textunderscore qual\textunderscore  + \textunderscore querer\textunderscore )}
\end{itemize}
(designativo de \textunderscore coisa\textunderscore , \textunderscore lugar\textunderscore  ou \textunderscore indivíduo indeterminado\textunderscore )
Algum, alguma.
Este ou aquelle: \textunderscore qualquer póde saber\textunderscore .
\section{Quamanho}
\begin{itemize}
\item {Grp. gram.:adj.}
\end{itemize}
\begin{itemize}
\item {Proveniência:(Do lat. \textunderscore quam\textunderscore  + \textunderscore magnus\textunderscore )}
\end{itemize}
Quão grande:«\textunderscore quamanha vista que tenho, que vejo a estrella do dia!\textunderscore »\textunderscore Filodemo\textunderscore , I, 3.
\section{Quando}
\begin{itemize}
\item {Grp. gram.:adv.  e  conj.}
\end{itemize}
\begin{itemize}
\item {Grp. gram.:Loc. adv.}
\end{itemize}
\begin{itemize}
\item {Proveniência:(Lat. \textunderscore quando\textunderscore )}
\end{itemize}
No tempo em que.
Em que tempo, em que occasião: \textunderscore quando chegaste\textunderscore ?
Posto que; mas.
\textunderscore De quando em quando\textunderscore , de tempos a tempos; com intervallos.
\textunderscore De vez em quando\textunderscore , (o mesmo significado).
\textunderscore Quando em quando\textunderscore , (o mesmo). Cf. Filinto, VIII, 185.
\section{Quandu}
\begin{itemize}
\item {Grp. gram.:m.}
\end{itemize}
\begin{itemize}
\item {Utilização:Bras}
\end{itemize}
\begin{itemize}
\item {Proveniência:(T. tupi)}
\end{itemize}
Mammífero roedor, cujo corpo é coberto de espinhos e pêlos á mistura.
\section{Quanté!}
\begin{itemize}
\item {Grp. gram.:interj.}
\end{itemize}
O mesmo que \textunderscore quantés!\textunderscore :«\textunderscore canté disso nós havemos-lhe de ver fazer algũa cousa\textunderscore ». Camões, \textunderscore Seleuco\textunderscore , (pról.). Cf. \textunderscore Filodemo\textunderscore , II, 2, 3 e 5.
\section{Quantés!}
\begin{itemize}
\item {Grp. gram.:interj.}
\end{itemize}
\begin{itemize}
\item {Utilização:Prov.}
\end{itemize}
\begin{itemize}
\item {Utilização:beir.}
\end{itemize}
\begin{itemize}
\item {Utilização:Ant.}
\end{itemize}
O mesmo que \textunderscore cantés!\textunderscore , hoje mais usado. Cf. Lobo, \textunderscore Auto do Nascimento\textunderscore .
\section{Quanteu!}
\begin{itemize}
\item {Grp. gram.:interj.}
\end{itemize}
O mesmo ou melhor que \textunderscore quantés\textunderscore . Cf. \textunderscore Eufrosina\textunderscore , 109.
\section{Quantia}
\begin{itemize}
\item {Grp. gram.:f.}
\end{itemize}
\begin{itemize}
\item {Proveniência:(De \textunderscore quanto\textunderscore )}
\end{itemize}
O mesmo que \textunderscore quantidade\textunderscore :«\textunderscore ...grande quantia de gente\textunderscore ». Filinto, \textunderscore D. Man.\textunderscore , I, 139.
Porção.
Somma, especialmente somma de dinheiro.
\section{Quantidade}
\begin{itemize}
\item {Grp. gram.:f.}
\end{itemize}
\begin{itemize}
\item {Proveniência:(Lat. \textunderscore quantitas\textunderscore )}
\end{itemize}
Qualidade do que póde sêr medido ou numerado.
Aquillo que póde aumentar ou deminuir.
Grande número, multidão: \textunderscore que quantidade de gente\textunderscore !
Valor das sýllabas longas e das breves, em prosódia.
Duração relativa das notas ou sýllabas da música.
\section{Quantioso}
\begin{itemize}
\item {Grp. gram.:adj.}
\end{itemize}
Relativo a quantia.
Muito numeroso.
Valioso.
Rico, opulento.
\section{Quantissimo}
\textunderscore adj. sup.\textunderscore  de \textunderscore quanto\textunderscore ^1. Cf. B. Pato, \textunderscore Cantos e Sát.\textunderscore , 128.
\section{Quantitativamente}
\begin{itemize}
\item {Grp. gram.:adv.}
\end{itemize}
De modo quantitativo.
Relativamente a quantidade.
\section{Quantitativo}
\begin{itemize}
\item {Grp. gram.:adj.}
\end{itemize}
\begin{itemize}
\item {Proveniência:(Do lat. \textunderscore quantitas\textunderscore )}
\end{itemize}
Relativo a quantidade.
Que indica quantidade.
\section{Quanto}
\begin{itemize}
\item {Grp. gram.:adj.}
\end{itemize}
\begin{itemize}
\item {Proveniência:(Lat. \textunderscore quantus\textunderscore )}
\end{itemize}
Que número ou que quantidade de: \textunderscore quantos filhos tem\textunderscore ?
Que aumento de.
Que preço: \textunderscore quanto custou o relógio\textunderscore ?
Quão grande, quão bello.
Tudo que: \textunderscore dei-lhe quanto tinha\textunderscore .
\section{Quanto}
\begin{itemize}
\item {Grp. gram.:adv.}
\end{itemize}
\begin{itemize}
\item {Proveniência:(Lat. \textunderscore quantum\textunderscore )}
\end{itemize}
Relativamente; a respeito: \textunderscore quanto a mim duvido\textunderscore .
Até que ponto.
De que modo; da maneira que.
\section{Quão}
\begin{itemize}
\item {Grp. gram.:adv.}
\end{itemize}
\begin{itemize}
\item {Proveniência:(Lat. \textunderscore quam\textunderscore )}
\end{itemize}
Quanto; como.
\section{Quapoia}
\begin{itemize}
\item {Grp. gram.:f.}
\end{itemize}
Nome, que se dá no Brasil a uma planta trepadeira.
\section{Quáquer}
\begin{itemize}
\item {Grp. gram.:m.}
\end{itemize}
\begin{itemize}
\item {Proveniência:(Ingl. \textunderscore quaker\textunderscore )}
\end{itemize}
Membro de uma seita religiosa, que dispensa a intervenção dos padres, por bastar aos homens a luz interior que Deus lhes dispensou.
\section{Quaqueriano}
\begin{itemize}
\item {Grp. gram.:adj.}
\end{itemize}
Relativo a quáquer.
\section{Quaquerismo}
\begin{itemize}
\item {Grp. gram.:m.}
\end{itemize}
Seita dos quáqueres.
\section{Quarango}
\begin{itemize}
\item {Grp. gram.:m.}
\end{itemize}
O mesmo que \textunderscore quinaquina\textunderscore .
\section{Quardelha}
\begin{itemize}
\item {fónica:dê}
\end{itemize}
\begin{itemize}
\item {Grp. gram.:f.}
\end{itemize}
\begin{itemize}
\item {Utilização:Ant.}
\end{itemize}
O mesmo que \textunderscore cortelha\textunderscore .
(Cp. \textunderscore quadrela\textunderscore )
\section{Quarenta}
\begin{itemize}
\item {Grp. gram.:adj.}
\end{itemize}
\begin{itemize}
\item {Grp. gram.:M.}
\end{itemize}
\begin{itemize}
\item {Proveniência:(Do lat. \textunderscore quadraginta\textunderscore )}
\end{itemize}
Déz vezes quatro.
Representação dêsse número em algarismos ou conta romana.
Aquelle ou aquilo que numa série de quarenta occupa o último lugar.
\section{Quarentão}
\begin{itemize}
\item {Grp. gram.:m.  e  adj.}
\end{itemize}
\begin{itemize}
\item {Utilização:Pop.}
\end{itemize}
Aquelle que completou quarenta annos, ou que tem aproximadamente quarenta annos.
\section{Quarentena}
\begin{itemize}
\item {Grp. gram.:f.}
\end{itemize}
\begin{itemize}
\item {Proveniência:(De \textunderscore quarenta\textunderscore )}
\end{itemize}
Período de quarenta dias.
Quaresma.
Festa ou ceremónia religiosa, que se prolonga por quarenta dias.
Espaço de tempo, em que os viajantes, procedentes de países em que há doenças contagiosas ou suspeita de taes doenças, têm que se conservar incommunicáveis a bordo do respectivo navio ou num lazareto.
Porção de quarenta coisas; número de quarenta.
\section{Quarentenar}
\begin{itemize}
\item {Grp. gram.:v. i.}
\end{itemize}
Fazer quarentena, (falando-se de viajantes que chegam de países estranhos).
\section{Quarentenário}
\begin{itemize}
\item {Grp. gram.:adj.}
\end{itemize}
\begin{itemize}
\item {Grp. gram.:M.  e  adj.}
\end{itemize}
Relativo á quarentena.
O que faz quarentena, (falando-se de viajantes que chegam de países estranhos).
\section{Quarentia}
\begin{itemize}
\item {Grp. gram.:f.}
\end{itemize}
\begin{itemize}
\item {Proveniência:(De \textunderscore quarenta\textunderscore )}
\end{itemize}
Antigo tribunal veneziano, composto de 40 magistrados.
\section{Quarentona}
\begin{itemize}
\item {Grp. gram.:f.  e  adj.}
\end{itemize}
Mulhér, que completou quarenta annos ou que tem quarenta annos aproximadamente.
\section{Quaresma}
\begin{itemize}
\item {Grp. gram.:f.}
\end{itemize}
\begin{itemize}
\item {Utilização:Bras}
\end{itemize}
\begin{itemize}
\item {Grp. gram.:Pl.}
\end{itemize}
\begin{itemize}
\item {Utilização:T. de Amarante}
\end{itemize}
\begin{itemize}
\item {Proveniência:(Lat. \textunderscore quadragesima\textunderscore )}
\end{itemize}
Quarenta dias, que decorrem desde a quarta-feira de cinza até Domingo de Páschoa.
Espécie de coqueiro; fruto desta árvore.
Espécie de planta, (\textunderscore saxifraga granulata\textunderscore , Lin.).
\section{Quaresmal}
\begin{itemize}
\item {Grp. gram.:adj.}
\end{itemize}
Relativo á quaresma: \textunderscore o jejum quaresmal\textunderscore .
\section{Quaresmalmente}
\begin{itemize}
\item {Grp. gram.:adv.}
\end{itemize}
De modo quaresmal.
Como na quaresma.
\section{Quaresmar}
\begin{itemize}
\item {Grp. gram.:v. i.}
\end{itemize}
Observar os preceitos religiosos, relativos á quaresma.
\section{Quaró}
\begin{itemize}
\item {Grp. gram.:m.}
\end{itemize}
Planta malpigiácea do Brasil.
\section{Quarta}
\begin{itemize}
\item {Grp. gram.:f.}
\end{itemize}
\begin{itemize}
\item {Utilização:Bras}
\end{itemize}
\begin{itemize}
\item {Utilização:Bras. de Piauí}
\end{itemize}
\begin{itemize}
\item {Utilização:Bras. do N}
\end{itemize}
Cada uma das quatro partes iguaes, em que se póde dividir alguma coisa.
Quarta parte de um alqueire.
Pequeno cântaro, bilha.
Quarta-feira, (por abreviatura): \textunderscore saiu no Domingo e volta na quarta\textunderscore .
Intervallo musical de quatro notas.
Posição de uma junta de bois, entre a junta da ponta e a do coice, nos carros puxados por mais de duas juntas.
Medida de 72 litros, para cereaes e legumes.
Porção de qualquer coisa, equivalente a quarenta litros.
\textunderscore Jogar a quarta\textunderscore , rebolar-se, espojar-se (o burro).
(Fem. de \textunderscore quarto\textunderscore )
\section{Quarço}
\begin{itemize}
\item {Grp. gram.:m.}
\end{itemize}
\begin{itemize}
\item {Utilização:Miner.}
\end{itemize}
\begin{itemize}
\item {Proveniência:(Al. \textunderscore quarz\textunderscore )}
\end{itemize}
Sílica natural.
\section{Quartã}
\begin{itemize}
\item {Grp. gram.:adj.}
\end{itemize}
\begin{itemize}
\item {Grp. gram.:F.}
\end{itemize}
\begin{itemize}
\item {Proveniência:(Lat. \textunderscore quartana\textunderscore )}
\end{itemize}
Diz-se de uma febre intermittente, que se repete de quatro em quatro dias.
Febre quartan.
\section{Quartada}
\begin{itemize}
\item {Grp. gram.:f.}
\end{itemize}
\begin{itemize}
\item {Utilização:Des.}
\end{itemize}
Conteúdo de uma quarta ou cântaro; cantarada.
\section{Quartado}
\begin{itemize}
\item {Grp. gram.:adj.}
\end{itemize}
Dividido em quatro; formado de quatro.
\section{Quarta-feira}
\begin{itemize}
\item {Grp. gram.:f.}
\end{itemize}
Quarto dia da semana, a começar do Domingo inclusivamente.
\section{Quartaludo}
\begin{itemize}
\item {Grp. gram.:adj.}
\end{itemize}
Diz-se do cavallo, que tem defeitos nos quartos.
\section{Quartan}
\begin{itemize}
\item {Grp. gram.:adj.}
\end{itemize}
\begin{itemize}
\item {Grp. gram.:F.}
\end{itemize}
\begin{itemize}
\item {Proveniência:(Lat. \textunderscore quartana\textunderscore )}
\end{itemize}
Diz-se de uma febre intermittente, que se repete de quatro em quatro dias.
Febre quartan.
\section{Quartanaria}
\begin{itemize}
\item {Grp. gram.:f.}
\end{itemize}
\begin{itemize}
\item {Utilização:Ant.}
\end{itemize}
Dignidade ou funcções de quartanário? Cf. J. B. Castro, \textunderscore Mappa de Port.\textunderscore , II.
\section{Quartanário}
\begin{itemize}
\item {Grp. gram.:m.  e  adj.}
\end{itemize}
O que tem febre quartan.
\section{Quartanário}
\begin{itemize}
\item {Grp. gram.:m.}
\end{itemize}
\begin{itemize}
\item {Utilização:Ant.}
\end{itemize}
\begin{itemize}
\item {Proveniência:(De \textunderscore quarto\textunderscore )}
\end{itemize}
Beneficiado, que recebia a quarta parte da côngrua de um cónego.
\section{Quartanista}
\begin{itemize}
\item {Grp. gram.:m.}
\end{itemize}
\begin{itemize}
\item {Proveniência:(De \textunderscore quarto\textunderscore  + \textunderscore anno\textunderscore )}
\end{itemize}
Alumno do quarto anno de um curso escolar, especialmente em faculdades ou escolas superiores.
\section{Quartannista}
\begin{itemize}
\item {Grp. gram.:m.}
\end{itemize}
\begin{itemize}
\item {Proveniência:(De \textunderscore quarto\textunderscore  + \textunderscore anno\textunderscore )}
\end{itemize}
Alumno do quarto anno de um curso escolar, especialmente em faculdades ou escolas superiores.
\section{Quartano}
\begin{itemize}
\item {Grp. gram.:m.}
\end{itemize}
\begin{itemize}
\item {Proveniência:(De \textunderscore quarto\textunderscore )}
\end{itemize}
Antiga medida, equivalente á quarta parte de um quarteiro.
\section{Quartanos}
\begin{itemize}
\item {Grp. gram.:m. pl.}
\end{itemize}
\begin{itemize}
\item {Proveniência:(Lat. \textunderscore quartani\textunderscore )}
\end{itemize}
Os soldados da quarta legião, entre os Romanos.
\section{Quartão}
\begin{itemize}
\item {Grp. gram.:m.}
\end{itemize}
\begin{itemize}
\item {Utilização:Bras}
\end{itemize}
\begin{itemize}
\item {Utilização:Ant.}
\end{itemize}
\begin{itemize}
\item {Proveniência:(De \textunderscore quarto\textunderscore )}
\end{itemize}
Quarta de vinho, ou a quarta parte de um almude.
Cavallo pequeno mas robusto, próprio para carga, o mesmo que \textunderscore quartau\textunderscore .
O mesmo que \textunderscore painel\textunderscore .
\section{Quartapele}
\begin{itemize}
\item {Grp. gram.:f.}
\end{itemize}
\begin{itemize}
\item {Utilização:Prov.}
\end{itemize}
\begin{itemize}
\item {Utilização:beir.}
\end{itemize}
O mesmo que \textunderscore cartapé\textunderscore .
\section{Quartapisa}
\begin{itemize}
\item {Grp. gram.:f.}
\end{itemize}
\begin{itemize}
\item {Utilização:Ant.}
\end{itemize}
Barra de vestidos e de outros objectos, mas de côr differente da dêstes.
\section{Quartapisar}
\begin{itemize}
\item {Grp. gram.:v. t.}
\end{itemize}
\begin{itemize}
\item {Utilização:Ant.}
\end{itemize}
Pôr quartapisa em. Cf. \textunderscore Eufrosina\textunderscore , I, 1.
\section{Quartar}
\begin{itemize}
\item {Grp. gram.:v. i.}
\end{itemize}
\begin{itemize}
\item {Proveniência:(De \textunderscore quarta\textunderscore )}
\end{itemize}
Saír da linha, em esgrima.
\section{Quartário}
\begin{itemize}
\item {Grp. gram.:m.}
\end{itemize}
\begin{itemize}
\item {Proveniência:(Lat. \textunderscore quartarius\textunderscore )}
\end{itemize}
A quarta parte de qualquer medida, especialmente de um sextário, entre os Romanos.
\section{Quartário}
\begin{itemize}
\item {Grp. gram.:m.}
\end{itemize}
\begin{itemize}
\item {Utilização:Ant.}
\end{itemize}
\begin{itemize}
\item {Proveniência:(De \textunderscore quarta\textunderscore )}
\end{itemize}
Aquelle que cultivava terras de que pagava a quarta. Cf. S. R. Viterbo, \textunderscore Elucidário\textunderscore .
\section{Quartau}
\begin{itemize}
\item {Grp. gram.:m.}
\end{itemize}
\begin{itemize}
\item {Proveniência:(De \textunderscore quarto\textunderscore )}
\end{itemize}
Cavallo pequeno mas robusto.
Antiga e pequena peça de artilharia, de ferro.
\section{Quarteado}
\begin{itemize}
\item {Grp. gram.:adj.}
\end{itemize}
Diz-se do cavallo espadaúdo e bem proporcionado.
\section{Quarteador}
\begin{itemize}
\item {Grp. gram.:m.}
\end{itemize}
\begin{itemize}
\item {Utilização:Bras. do S}
\end{itemize}
\begin{itemize}
\item {Proveniência:(De \textunderscore quartear\textunderscore )}
\end{itemize}
Aquelle que, a cavallo, ajuda a puxar um carro, (nos bondes, por exemplo).
\section{Quartear}
\begin{itemize}
\item {Grp. gram.:v. t.}
\end{itemize}
\begin{itemize}
\item {Utilização:Taur.}
\end{itemize}
\begin{itemize}
\item {Utilização:Bras. do S}
\end{itemize}
\begin{itemize}
\item {Proveniência:(De \textunderscore quarto\textunderscore )}
\end{itemize}
Dividir em quatro partes.
Decorar com quatro côres differentes.
Fazer quarteio a.
Ajudar, a cavallo, a puxar um carro.
\section{Quarteio}
\begin{itemize}
\item {Grp. gram.:m.}
\end{itemize}
\begin{itemize}
\item {Proveniência:(De \textunderscore quartear\textunderscore )}
\end{itemize}
Quarto de volta, dado pelo toireiro, ao farpear um boi.
\section{Quarteirão}
\begin{itemize}
\item {Grp. gram.:m.}
\end{itemize}
\begin{itemize}
\item {Utilização:Prov.}
\end{itemize}
\begin{itemize}
\item {Utilização:Bras. do N}
\end{itemize}
\begin{itemize}
\item {Proveniência:(De \textunderscore quarteiro\textunderscore )}
\end{itemize}
Quarta parte de cem.
Série de casas contíguas.
Grupo de casas, que formam quadrilongo.
Trave, que parte de cada um dos quatro cantos de um tecto.
Tributo antigo sôbre cada casal.
Quarta parte de um quartilho. Cf. Camillo, \textunderscore Brasileira\textunderscore , 307.
O mesmo ou melhor que \textunderscore quarterão\textunderscore .
A quarta parte de uma garrafada.
\section{Quarteiro}
\begin{itemize}
\item {Grp. gram.:m.}
\end{itemize}
\begin{itemize}
\item {Proveniência:(Do b. lat. \textunderscore quartarius\textunderscore )}
\end{itemize}
Quarta parte de um moio.
Pensão, que antigamente se pagava em cada trimestre.
Antigo imposto, constituido pela quarta parte de um moio de cereaes.
Colono, que pagava esse imposto.
\section{Quartejar}
\textunderscore v. t.\textunderscore  (e der.)
O mesmo que \textunderscore esquartejar\textunderscore , partir em quartos ou em pedaços:«\textunderscore ...cebolas verdes quartejadas.\textunderscore »\textunderscore Bibl. do G. do Campo\textunderscore , 433.
\section{Quartel}
\begin{itemize}
\item {Grp. gram.:m.}
\end{itemize}
\begin{itemize}
\item {Utilização:Ext.}
\end{itemize}
\begin{itemize}
\item {Utilização:Heráld.}
\end{itemize}
\begin{itemize}
\item {Utilização:Prov.}
\end{itemize}
\begin{itemize}
\item {Utilização:Náut.}
\end{itemize}
\begin{itemize}
\item {Utilização:Pesc.}
\end{itemize}
\begin{itemize}
\item {Utilização:Náut.}
\end{itemize}
\begin{itemize}
\item {Utilização:Náut.}
\end{itemize}
\begin{itemize}
\item {Utilização:Fig.}
\end{itemize}
Quarta parte.
Edifício, em que se alojam tropas.
Habitação; abrigo.
Cada uma das quatro partes, em que se divide um escudo.
Cada uma das refeições diárias: \textunderscore só bebo vinho aos quartéis\textunderscore .
Accrescentamento a um mastro ou vêrga.
Uma das rêdes das armações de atum.
Tampo de escotilha, dividido em duas, três ou quatro partes.
Cada uma das partes em que se divide o tampo da escotilha.
«Espaço de tempo, período, época: \textunderscore no último quartel da vida...\textunderscore »(Cast. \textunderscore cuartel\textunderscore .)
\section{Quartela}
\begin{itemize}
\item {Grp. gram.:f.}
\end{itemize}
\begin{itemize}
\item {Proveniência:(De \textunderscore quarto\textunderscore )}
\end{itemize}
Nervuras, que ligam o casco das bêstas á primeira junta.
Peça, que sustenta outra.
Mísula.
\section{Quartelada}
\begin{itemize}
\item {Grp. gram.:f.}
\end{itemize}
\begin{itemize}
\item {Utilização:Pesc.}
\end{itemize}
\begin{itemize}
\item {Proveniência:(De \textunderscore quartel\textunderscore )}
\end{itemize}
Uma das partes componentes das rêdes que formam os apparelhos symétricos da pesca.
\section{Quarteleiro}
\begin{itemize}
\item {Grp. gram.:m.}
\end{itemize}
\begin{itemize}
\item {Proveniência:(De \textunderscore quartel\textunderscore )}
\end{itemize}
Militar, encarregado de guardar o armamento de um corpo de tropas.
\section{Quartelha}
\begin{itemize}
\item {fónica:tê}
\end{itemize}
\begin{itemize}
\item {Grp. gram.:f.}
\end{itemize}
\begin{itemize}
\item {Utilização:Prov.}
\end{itemize}
\begin{itemize}
\item {Utilização:alent.}
\end{itemize}
O mesmo que \textunderscore cortelho\textunderscore ; curral de gado.
\section{Quarteludo}
\begin{itemize}
\item {Grp. gram.:adj.}
\end{itemize}
\begin{itemize}
\item {Proveniência:(De \textunderscore quartela\textunderscore )}
\end{itemize}
Diz-se do animal, em que o osso da quartela tem mais comprimento que o normal.
\section{Quarterão}
\begin{itemize}
\item {Grp. gram.:m.}
\end{itemize}
\begin{itemize}
\item {Proveniência:(De \textunderscore quarto\textunderscore )}
\end{itemize}
Aquelle que tem por pais um indivíduo branco e outro mulato ou mestiço.
\section{Quarteto}
\begin{itemize}
\item {fónica:tê}
\end{itemize}
\begin{itemize}
\item {Grp. gram.:m.}
\end{itemize}
\begin{itemize}
\item {Utilização:Fam.}
\end{itemize}
\begin{itemize}
\item {Proveniência:(De \textunderscore quarto\textunderscore )}
\end{itemize}
Estância de quatro versos.
Peça de música, executada por quatro vozes ou quatro instrumentos.
Conjunto dêsses instrumentos ou vozes.
Reunião de quatro pessôas.
\section{Quartifalange}
\begin{itemize}
\item {Grp. gram.:f.}
\end{itemize}
\begin{itemize}
\item {Utilização:Anat.}
\end{itemize}
A quarta falange do pé.
\section{Quartifalangeta}
\begin{itemize}
\item {fónica:gê}
\end{itemize}
\begin{itemize}
\item {Grp. gram.:f.}
\end{itemize}
\begin{itemize}
\item {Utilização:Anat.}
\end{itemize}
A quarta falangeta do pé.
\section{Quartifalanginha}
\begin{itemize}
\item {Grp. gram.:f.}
\end{itemize}
\begin{itemize}
\item {Utilização:Anat.}
\end{itemize}
A quarta falanginha do pé.
\section{Quartil}
\begin{itemize}
\item {Grp. gram.:adj.}
\end{itemize}
\begin{itemize}
\item {Proveniência:(De \textunderscore quarto\textunderscore )}
\end{itemize}
Diz-se do aspecto de dois planetas, que distam reciprocamente um quarto do Zodíaco ou 90 graus.
\section{Quartilhada}
\begin{itemize}
\item {Grp. gram.:f.}
\end{itemize}
\begin{itemize}
\item {Utilização:Pop.}
\end{itemize}
Porção de líquido, contida num quartilho.
\section{Quartilhame}
\begin{itemize}
\item {Grp. gram.:m.}
\end{itemize}
Porção de quartilhos.
\section{Quartilhar}
\begin{itemize}
\item {Grp. gram.:v. t.}
\end{itemize}
\begin{itemize}
\item {Utilização:Prov.}
\end{itemize}
\begin{itemize}
\item {Utilização:trasm.}
\end{itemize}
O mesmo que \textunderscore esquartilhar\textunderscore .
\section{Quartilho}
\begin{itemize}
\item {Grp. gram.:m.}
\end{itemize}
Quarta parte de uma canada.
(Cast. \textunderscore cuartillo\textunderscore )
\section{Quartimetatársico}
\begin{itemize}
\item {Grp. gram.:adj.}
\end{itemize}
\begin{itemize}
\item {Utilização:Anat.}
\end{itemize}
Diz-se do quarto osso metatársico.
\section{Quartinha}
\begin{itemize}
\item {Grp. gram.:f.}
\end{itemize}
\begin{itemize}
\item {Utilização:Bras}
\end{itemize}
\begin{itemize}
\item {Proveniência:(De \textunderscore quarta\textunderscore )}
\end{itemize}
Espécie de bilha de barro, para conter e refrescar a água.
\section{Quartinha}
\begin{itemize}
\item {Grp. gram.:f.}
\end{itemize}
\begin{itemize}
\item {Utilização:Des.}
\end{itemize}
Cortina.
Tribuna.
(Corr. de \textunderscore cortina\textunderscore )
\section{Quartinheiro}
\begin{itemize}
\item {Grp. gram.:m.}
\end{itemize}
\begin{itemize}
\item {Utilização:Bras. da Baía}
\end{itemize}
\begin{itemize}
\item {Proveniência:(De \textunderscore quartinha\textunderscore ^1)}
\end{itemize}
Móvel, com uma série de buracos, em que se metem as quartinhas, até o bojo.
\section{Quartinho}
\begin{itemize}
\item {Grp. gram.:m.}
\end{itemize}
\begin{itemize}
\item {Proveniência:(De \textunderscore quarto\textunderscore )}
\end{itemize}
A quantia de 1$200 reis, quarta parte da antiga moéda de 4$800.
Antiga moéda de oiro, do valor de 1$200.
\section{Quartiphalange}
\begin{itemize}
\item {Grp. gram.:f.}
\end{itemize}
\begin{itemize}
\item {Utilização:Anat.}
\end{itemize}
A quarta phalange do pé.
\section{Quartiphalangeta}
\begin{itemize}
\item {fónica:gê}
\end{itemize}
\begin{itemize}
\item {Grp. gram.:f.}
\end{itemize}
\begin{itemize}
\item {Utilização:Anat.}
\end{itemize}
A quarta phalangeta do pé.
\section{Quartiphalanginha}
\begin{itemize}
\item {Grp. gram.:f.}
\end{itemize}
\begin{itemize}
\item {Utilização:Anat.}
\end{itemize}
A quarta phalanginha do pé.
\section{Quarto}
\begin{itemize}
\item {Grp. gram.:adj.}
\end{itemize}
\begin{itemize}
\item {Grp. gram.:M.}
\end{itemize}
\begin{itemize}
\item {Utilização:Bras. do Mato-Grosso}
\end{itemize}
\begin{itemize}
\item {Utilização:Bras. do N}
\end{itemize}
\begin{itemize}
\item {Utilização:T. da Bairrada}
\end{itemize}
\begin{itemize}
\item {Utilização:T. de Lisbôa}
\end{itemize}
\begin{itemize}
\item {Proveniência:(Lat. \textunderscore quartus\textunderscore )}
\end{itemize}
Que numa série de quatro occupa o último lugar.
A quarta parte.
Espaço de 15 minutos, quarta parte da hora.
Alcova, compartimento em que se dorme.
Cada um dos compartimentos de uma casa, á excepção da cozinha, casa de jantar e sala.
Indivíduo ou objecto, que está em quarto lugar.
Espaço de tempo, em que alternadamente os soldados ou marinheiros velam, em quanto outros descansam; plantão.
Espécie de pequena bala de chumbo, de fórma angular.
Cada uma das peças de tecido que, depois de cosidas, formam um casaco, um vestido, etc.
Cada uma dessas peças, correspondente ás costas e ao peito.
Fenda no casco dos bêstas.
Quantia, igual a 300 reis.
Ancas; nádegas.
O mesmo que \textunderscore quartola\textunderscore .
\textunderscore Quarto de marmelada\textunderscore , pequeno cubo de marmelada.
\section{Quartodecimanos}
\begin{itemize}
\item {Grp. gram.:m. pl.}
\end{itemize}
\begin{itemize}
\item {Proveniência:(Lat. \textunderscore quartodecimani\textunderscore )}
\end{itemize}
Nome, que se deu aos Christãos, que celebravam a Páscoa no 14.^o dia depois do equinócio da primavera.
\section{Quartola}
\begin{itemize}
\item {Grp. gram.:f.}
\end{itemize}
\begin{itemize}
\item {Proveniência:(De \textunderscore quarto\textunderscore )}
\end{itemize}
Pequena pipa, correspondente aproximadamente a um quarto de tonel.
\section{Quarto-redondo}
\begin{itemize}
\item {Grp. gram.:m.}
\end{itemize}
\begin{itemize}
\item {Utilização:Constr.}
\end{itemize}
Moldura, o mesmo que \textunderscore ovado\textunderscore .
\section{Quártzico}
\begin{itemize}
\item {Grp. gram.:adj.}
\end{itemize}
Feito de quartzo.
Em que há quartzo.
\section{Quartzífero}
\begin{itemize}
\item {Grp. gram.:adj.}
\end{itemize}
\begin{itemize}
\item {Proveniência:(De \textunderscore quartzo\textunderscore  + lat. \textunderscore ferre\textunderscore )}
\end{itemize}
Que contém quartzo.
Que abunda em quartzo.
\section{Quartzita}
\begin{itemize}
\item {Grp. gram.:f.}
\end{itemize}
(V.quartzito)
\section{Quartzite}
\begin{itemize}
\item {Grp. gram.:f.}
\end{itemize}
(V.quartzito)
\section{Quartzito}
\begin{itemize}
\item {Grp. gram.:m.}
\end{itemize}
Rocha de quartzo.
\section{Quartzo}
\begin{itemize}
\item {Grp. gram.:m.}
\end{itemize}
\begin{itemize}
\item {Utilização:Miner.}
\end{itemize}
\begin{itemize}
\item {Proveniência:(Al. \textunderscore quarz\textunderscore )}
\end{itemize}
Sílica natural.
\section{Quartzoso}
\begin{itemize}
\item {Grp. gram.:adj.}
\end{itemize}
Relativo ao quartzo; que tem a natureza do quartzo.
\section{Quarzo}
\begin{itemize}
\item {Grp. gram.:m.}
\end{itemize}
\begin{itemize}
\item {Utilização:Miner.}
\end{itemize}
\begin{itemize}
\item {Proveniência:(Al. \textunderscore quarz\textunderscore )}
\end{itemize}
Sílica natural.
\section{Quase}
\begin{itemize}
\item {Grp. gram.:adv.}
\end{itemize}
O mesmo ou melhor que \textunderscore quási\textunderscore .
\section{Quási}
\begin{itemize}
\item {Grp. gram.:adv.}
\end{itemize}
\begin{itemize}
\item {Proveniência:(Lat. \textunderscore quasi\textunderscore )}
\end{itemize}
A pouca distância; perto; proximamente: \textunderscore mora quási ao pé de mim\textunderscore .
Com pouca differença ou pouco menos: \textunderscore tem quási 40 annos\textunderscore .
Por pouco: \textunderscore esteve quási a morrer\textunderscore .
Como se:«\textunderscore ...lhe embota o gume, quasi dera em broquel diamantino\textunderscore ». Filinto, XV, 206.
\section{Quási-contrato}
\begin{itemize}
\item {Grp. gram.:m.}
\end{itemize}
Compromisso voluntário, sem fórma rigorosa de contrato.
\section{Quasi-delicto}
\begin{itemize}
\item {Grp. gram.:m.}
\end{itemize}
Damno, causado por negligência, sem intenção ruím.
\section{Quasímodo}
\begin{itemize}
\item {Grp. gram.:m.}
\end{itemize}
\begin{itemize}
\item {Utilização:Bras}
\end{itemize}
\begin{itemize}
\item {Proveniência:(Do lat. \textunderscore quasi\textunderscore  + \textunderscore modo\textunderscore )}
\end{itemize}
Domingo de Pascoéla.
O mesmo que \textunderscore mostrengo\textunderscore .
\section{Quassa}
\begin{itemize}
\item {Grp. gram.:f.}
\end{itemize}
O mesmo que \textunderscore quássia\textunderscore .
\section{Quassação}
\begin{itemize}
\item {Grp. gram.:f.}
\end{itemize}
\begin{itemize}
\item {Proveniência:(Lat. \textunderscore quassatio\textunderscore )}
\end{itemize}
Reducção de cascas ou raízes sêcas a fragmentos, para melhor se lhes extrahirem os princípios activos.
\section{Quássia}
\begin{itemize}
\item {Grp. gram.:f.}
\end{itemize}
Gênero de plantas violáceas, geralmente amargas e de qualidades medicinaes, (\textunderscore quassia amara\textunderscore , Lin.)
\section{Quassina}
\begin{itemize}
\item {Grp. gram.:f.}
\end{itemize}
Substância amarga, extrahida de uma das espécies da quássia.
\section{Quassite}
\begin{itemize}
\item {Grp. gram.:f.}
\end{itemize}
Substância amarga, extrahida de uma das espécies da quássia.
\section{Quatá}
\begin{itemize}
\item {Grp. gram.:m.}
\end{itemize}
\begin{itemize}
\item {Utilização:Bras. do N}
\end{itemize}
(V.coatá)
\section{Qua-tal}
\begin{itemize}
\item {Grp. gram.:adj.}
\end{itemize}
\begin{itemize}
\item {Utilização:Fam.}
\end{itemize}
\begin{itemize}
\item {Proveniência:(De \textunderscore qual-tal\textunderscore , inversão arbitrária de \textunderscore tal-qual\textunderscore )}
\end{itemize}
Tal qual; exactamente idêntico.
Pl. \textunderscore qua-taes\textunderscore .
\section{Quaternado}
\begin{itemize}
\item {Grp. gram.:adj.}
\end{itemize}
\begin{itemize}
\item {Utilização:Bot.}
\end{itemize}
\begin{itemize}
\item {Proveniência:(De \textunderscore quaterno\textunderscore )}
\end{itemize}
Diz-se dos órgãos vegetaes, dispostos em grupos de quatro no mesmo ponto de inserção.
\section{Quaternario}
\begin{itemize}
\item {Grp. gram.:adj.}
\end{itemize}
\begin{itemize}
\item {Utilização:Mús.}
\end{itemize}
\begin{itemize}
\item {Utilização:Geol.}
\end{itemize}
\begin{itemize}
\item {Proveniência:(Lat. \textunderscore quaternarius\textunderscore )}
\end{itemize}
Composto de quatro unidades.
Que tem quatro faces, quatro elementos, etc.
Que tem quatro tempos: \textunderscore compasso quaternário\textunderscore .
Diz-se do período geológico, immediatamente posterior ao terciário.
\section{Quaternião}
\begin{itemize}
\item {Grp. gram.:m.}
\end{itemize}
\begin{itemize}
\item {Proveniência:(Do lat. \textunderscore quaterni\textunderscore )}
\end{itemize}
Bálsamo medicamentoso, composto de quatro símplices.
\section{Quaternidade}
\begin{itemize}
\item {Grp. gram.:f.}
\end{itemize}
\begin{itemize}
\item {Proveniência:(De \textunderscore quaterno\textunderscore )}
\end{itemize}
Agrupamento de quatro pessôas ou coisas.
\section{Quaterno}
\begin{itemize}
\item {Grp. gram.:adj.}
\end{itemize}
\begin{itemize}
\item {Proveniência:(Do lat. \textunderscore quaterni\textunderscore )}
\end{itemize}
Que tem quatro coisas, modos, elementos, etc.
\section{Quati}
\begin{itemize}
\item {Grp. gram.:m.}
\end{itemize}
\begin{itemize}
\item {Utilização:Bras}
\end{itemize}
\begin{itemize}
\item {Proveniência:(T. tupi)}
\end{itemize}
Nome de duas espécies de mammiferos carnivoros; o mesmo que \textunderscore coati\textunderscore .
\section{Quati-aípe}
\begin{itemize}
\item {Grp. gram.:m.}
\end{itemize}
O mesmo que \textunderscore quati-puru\textunderscore .
\section{Quatiara}
\begin{itemize}
\item {Grp. gram.:f.}
\end{itemize}
\begin{itemize}
\item {Utilização:Bras}
\end{itemize}
Árvore silvestre, cuja madeira é amarela, raiada de preto.
\section{Quatibo}
\begin{itemize}
\item {Grp. gram.:m.}
\end{itemize}
Árvore brasileira, de cúpula rosada, na região do Amazonas.
\section{Quati-mirim}
\begin{itemize}
\item {Grp. gram.:m.}
\end{itemize}
O mesmo que \textunderscore caxinguelê\textunderscore .
\section{Quati-mundé}
\begin{itemize}
\item {Grp. gram.:m.}
\end{itemize}
O mesmo que \textunderscore caxinguelê\textunderscore .
\section{Quati-puru}
\begin{itemize}
\item {Grp. gram.:m.}
\end{itemize}
O mesmo que \textunderscore caxinguelê\textunderscore .
\section{Quatorzada}
\begin{itemize}
\item {fónica:ca}
\end{itemize}
\begin{itemize}
\item {Grp. gram.:f.}
\end{itemize}
\begin{itemize}
\item {Utilização:Fam.}
\end{itemize}
\begin{itemize}
\item {Proveniência:(De \textunderscore quatorze\textunderscore )}
\end{itemize}
Quatorze pontos, que se contam em certo jôgo.
Grande porção.
\section{Quatorze}
\begin{itemize}
\item {fónica:ca}
\end{itemize}
\begin{itemize}
\item {Grp. gram.:adj.}
\end{itemize}
\begin{itemize}
\item {Grp. gram.:M.}
\end{itemize}
\begin{itemize}
\item {Proveniência:(Lat. \textunderscore quatordecim\textunderscore )}
\end{itemize}
Diz-se do número cardinal, formado de doze e mais dois ou de duas vezes sete.
Décimo quarto: \textunderscore capítulo quatorze\textunderscore .
Aquillo ou aquelle que numa série de quatorze occupa o último lugar.
\section{Quatorzeno}
\begin{itemize}
\item {fónica:ca}
\end{itemize}
\begin{itemize}
\item {Grp. gram.:adj.}
\end{itemize}
\begin{itemize}
\item {Grp. gram.:M.}
\end{itemize}
\begin{itemize}
\item {Proveniência:(De \textunderscore quatorze\textunderscore )}
\end{itemize}
Décimo-quarto.
Último numa série de quatorze.
Antiga espécie de pano, com quatorze fios de urdidura.
\section{Quatr'alvo}
\begin{itemize}
\item {Grp. gram.:adj.}
\end{itemize}
\begin{itemize}
\item {Proveniência:(De \textunderscore quatro\textunderscore  + \textunderscore alvo\textunderscore )}
\end{itemize}
Diz-se do cavallo malhado de branco até os joêlhos.
\section{Quatrela}
\begin{itemize}
\item {Grp. gram.:f.}
\end{itemize}
\begin{itemize}
\item {Utilização:T. do Ribatejo}
\end{itemize}
\begin{itemize}
\item {Proveniência:(De \textunderscore quatro\textunderscore )}
\end{itemize}
Duas juntas de bois \textunderscore ou\textunderscore , melhor, dois cingéis, que puxam a mesma carrêta, carro, etc.
\section{Quatriduano}
\begin{itemize}
\item {Grp. gram.:adj.}
\end{itemize}
Que abrange um quatríduo.
\section{Quatríduo}
\begin{itemize}
\item {Grp. gram.:m.}
\end{itemize}
\begin{itemize}
\item {Proveniência:(Lat. \textunderscore quatriduum\textunderscore )}
\end{itemize}
Espaço de quatro dias.
\section{Quatrienal}
\begin{itemize}
\item {Grp. gram.:adj.}
\end{itemize}
Relativo a quatriênio.
\section{Quatriênio}
\begin{itemize}
\item {Grp. gram.:m.}
\end{itemize}
\begin{itemize}
\item {Utilização:Bras}
\end{itemize}
O mesmo que \textunderscore quadriênio\textunderscore .
\section{Quatrilhão}
\begin{itemize}
\item {Grp. gram.:m.}
\end{itemize}
O mesmo ou melhor que \textunderscore quatrillião\textunderscore .
\section{Quatrilião}
\begin{itemize}
\item {Grp. gram.:m.}
\end{itemize}
\begin{itemize}
\item {Proveniência:(Fr. \textunderscore quatrillion\textunderscore . Cp. \textunderscore milhão\textunderscore ^1)}
\end{itemize}
Mil triliões, segundo o systema francês; um milhão de triliões, segundo o sistema inglês.
\section{Quatrillião}
\begin{itemize}
\item {Grp. gram.:m.}
\end{itemize}
\begin{itemize}
\item {Proveniência:(Fr. \textunderscore quatrillion\textunderscore . Cp. \textunderscore milhão\textunderscore ^1)}
\end{itemize}
Mil trilliões, segundo o systema francês; um milhão de trilliões, segundo o systema inglês.
\section{Quatrim}
\begin{itemize}
\item {Grp. gram.:m.}
\end{itemize}
Pequena moéda antiga, de pouco valor.
(Cast. \textunderscore cuatrin\textunderscore )
\section{Quatrinca}
\begin{itemize}
\item {Grp. gram.:f.}
\end{itemize}
\begin{itemize}
\item {Utilização:Ant.}
\end{itemize}
Quatro cartas iguaes, no jôgo.
Série de quatro.
\section{Quatro}
\begin{itemize}
\item {Grp. gram.:adj.}
\end{itemize}
\begin{itemize}
\item {Grp. gram.:M.}
\end{itemize}
\begin{itemize}
\item {Utilização:Fam.}
\end{itemize}
\begin{itemize}
\item {Proveniência:(Do lat. \textunderscore quatuor\textunderscore )}
\end{itemize}
Diz-se de um número cardinal, formado de três e mais um.
Quarto.
O algarismo, que representa o número quatro.
Carta de jogar ou peça de dominó, que tem quatro pontos.
Aquelle ou aquillo que numa série de quatro occupa o último lugar.
\textunderscore O diabo a quatro\textunderscore , barafunda; desordem; traquinice.
\section{Quatro-cantinhos}
\begin{itemize}
\item {Grp. gram.:m. pl.}
\end{itemize}
Jôgo infantil.
\section{Quatrocentista}
\begin{itemize}
\item {Grp. gram.:adj.}
\end{itemize}
\begin{itemize}
\item {Grp. gram.:M.}
\end{itemize}
Relativo ao século que vai desde 1401 a 1500.
Escritor, que viveu nesse século. Cf. Michaëlis, \textunderscore Uma obra... do Condestável...\textunderscore , 1.
\section{Quatrocentos}
\begin{itemize}
\item {Grp. gram.:adj.}
\end{itemize}
\begin{itemize}
\item {Proveniência:(De \textunderscore quatro\textunderscore  + \textunderscore cento\textunderscore )}
\end{itemize}
Quatro vezes cem.
\section{Quatro-olhos}
\begin{itemize}
\item {Grp. gram.:m.}
\end{itemize}
Espécie de peixe do Brasil.
\section{Quatro-patacas}
\begin{itemize}
\item {Grp. gram.:f.}
\end{itemize}
Planta apocýnea do Brasil.
\section{Quatro-reis}
\begin{itemize}
\item {Grp. gram.:m.}
\end{itemize}
Espécie de jôgo de cartas.
\section{Quatro-vintens}
\begin{itemize}
\item {Grp. gram.:m.}
\end{itemize}
Antiga moéda portuguesa, de prata, do valor de 80 reis, cunhada em tempo de D. João III e de Filippe I.
\section{Quatuorvirado}
\begin{itemize}
\item {Grp. gram.:m.}
\end{itemize}
\begin{itemize}
\item {Proveniência:(Lat. \textunderscore quatuorviratus\textunderscore )}
\end{itemize}
Dignidade ou cargo de quatuórviro.
\section{Quatuórviro}
\begin{itemize}
\item {Grp. gram.:m.}
\end{itemize}
Cada um dos quatro magistrados que, entre os Romanos, tinham a seu cargo a viação pública.
Cada uma das quatro autoridades superiores nos municípios e colonias romanas. Cf. Herculano, \textunderscore Hist. de Port.\textunderscore  IV, 7, 8 e 9.
(Cp. lat. \textunderscore quatuorviri\textunderscore )
\section{Que}
\begin{itemize}
\item {Grp. gram.:pron.}
\end{itemize}
\begin{itemize}
\item {Proveniência:(Do lat. \textunderscore qui\textunderscore )}
\end{itemize}
O qual, a qual, os quaes, as quaes; êste, esta; êsse, essa; êlle, ella; aquelle, aquella.
No qual, na qual; etc.
\section{Que}
\begin{itemize}
\item {Grp. gram.:pron.}
\end{itemize}
\begin{itemize}
\item {Proveniência:(Do lat. \textunderscore quis\textunderscore )}
\end{itemize}
(\textunderscore para interrogar\textunderscore )
Qual coisa, quaes coisas: \textunderscore que lhe disseste tu\textunderscore ?
\section{Que}
\begin{itemize}
\item {Grp. gram.:adv.}
\end{itemize}
\begin{itemize}
\item {Proveniência:(Do lat. \textunderscore quam\textunderscore )}
\end{itemize}
O mesmo que \textunderscore quão\textunderscore : \textunderscore que linda\textunderscore !
\section{Que}
\begin{itemize}
\item {Grp. gram.:conj.}
\end{itemize}
\begin{itemize}
\item {Proveniência:(Do lat. \textunderscore quod\textunderscore )}
\end{itemize}
(que caracteriza e começa as orações integrantes): \textunderscore consta que fugiu\textunderscore .
\section{Que}
\begin{itemize}
\item {Grp. gram.:conj.}
\end{itemize}
\begin{itemize}
\item {Proveniência:(Do lat. \textunderscore quia\textunderscore )}
\end{itemize}
(designativa de \textunderscore causa\textunderscore )
Porque: \textunderscore não grites, que parece mal\textunderscore .
\section{Que}
\begin{itemize}
\item {Grp. gram.:prep.}
\end{itemize}
O mesmo que \textunderscore excepto\textunderscore :«\textunderscore ...sem outra sombra que a do camelo...\textunderscore »Filinto, XV, 130.«\textunderscore ...não tem de se extinguír, que no jazigo.\textunderscore »\textunderscore Idem\textunderscore , XX, 32.--Não obstante a autoridade de Filinto, parece-me haver resaibo de francesismo em tal emprêgo de \textunderscore que\textunderscore . Compare-se a syntaxe francesa: \textunderscore il n'y avait que des femmes\textunderscore .
\section{Quê}
\begin{itemize}
\item {Grp. gram.:m.}
\end{itemize}
\begin{itemize}
\item {Proveniência:(Do lat. \textunderscore quid\textunderscore )}
\end{itemize}
Alguma coisa, qualquer coisa: \textunderscore admira-se? não tem de quê\textunderscore .
Difficuldade; complicação: \textunderscore tem um quê de ingenuidade e de malícia...\textunderscore  \textunderscore Este latim é fácil, mas o latim de Horácio tem seus quês\textunderscore . Cf. Gonçalves Dias, \textunderscore Poes.\textunderscore , II, 186 e 239; Júl. Dinis, \textunderscore Morgadinha\textunderscore , 138; B. Pato, \textunderscore Ciprestes\textunderscore , 229.
\section{Queba}
\begin{itemize}
\item {fónica:cu-e}
\end{itemize}
\begin{itemize}
\item {Grp. gram.:adj.}
\end{itemize}
\begin{itemize}
\item {Utilização:Bras. de Goiás}
\end{itemize}
Antigo, velho.
\section{Quebra}
\begin{itemize}
\item {Grp. gram.:f.}
\end{itemize}
Acto ou effeito de quebrar.
Quebrada, declive.
\section{Quebra}
\begin{itemize}
\item {Grp. gram.:adj.}
\end{itemize}
\begin{itemize}
\item {Utilização:Bras. do S}
\end{itemize}
\begin{itemize}
\item {Utilização:Bras. do N}
\end{itemize}
Animal ou pessôa má, ou de má condição.
Gracejador; patusco.
\section{Quebra-bunda}
\begin{itemize}
\item {Grp. gram.:m.}
\end{itemize}
\begin{itemize}
\item {Utilização:Bras}
\end{itemize}
Epizootia, que, nas regiões paludosas, ataca os cavallos, fazendo-lhes vergar as pernas posteriores e prostrando-os.
\section{Quebra-cabeça}
\begin{itemize}
\item {Grp. gram.:m.  e  f.}
\end{itemize}
\begin{itemize}
\item {Utilização:Pop.}
\end{itemize}
Aquillo que dá cuidado, que preoccupa, que é complicado.
Adivinhação gráphica ou mecânica.
\section{Quebracho}
\begin{itemize}
\item {Grp. gram.:m.}
\end{itemize}
\begin{itemize}
\item {Utilização:Bras}
\end{itemize}
Planta apocynácea, medicinal.
\section{Quebra-costas}
\begin{itemize}
\item {Grp. gram.:m.}
\end{itemize}
\begin{itemize}
\item {Utilização:Fam.}
\end{itemize}
Rua ou caminho ingreme e escorregadio.
\section{Quebrada}
\begin{itemize}
\item {Grp. gram.:f.}
\end{itemize}
\begin{itemize}
\item {Utilização:T. de Turquel}
\end{itemize}
\begin{itemize}
\item {Utilização:Ant.}
\end{itemize}
\begin{itemize}
\item {Utilização:Ant.}
\end{itemize}
\begin{itemize}
\item {Proveniência:(De \textunderscore quebrar\textunderscore )}
\end{itemize}
Ladeira, declive.
Anfractuosidade de terreno, produzida pela água.
Desmoronamento de terra.
Coirela, pequena propriedade.
Soldada, que constava de dois pães por dia.
\section{Quebradamente}
\begin{itemize}
\item {Grp. gram.:adv.}
\end{itemize}
\begin{itemize}
\item {Proveniência:(De \textunderscore quebrado\textunderscore )}
\end{itemize}
De repente.
\section{Quebradeira}
\begin{itemize}
\item {Grp. gram.:f.}
\end{itemize}
\begin{itemize}
\item {Utilização:Bras. do N}
\end{itemize}
\begin{itemize}
\item {Proveniência:(De \textunderscore quebrar\textunderscore )}
\end{itemize}
Quebra-cabeça.
Quebreira^2.
Ruina; miséria.
\section{Quebradela}
\begin{itemize}
\item {Grp. gram.:f.}
\end{itemize}
O mesmo que \textunderscore quebra\textunderscore ^1.
\section{Quebradiço}
\begin{itemize}
\item {Grp. gram.:adj.}
\end{itemize}
\begin{itemize}
\item {Proveniência:(De \textunderscore quebrar\textunderscore )}
\end{itemize}
Que se quebra com facilidade; frágil.
\section{Quebrado}
\begin{itemize}
\item {Grp. gram.:m.}
\end{itemize}
\begin{itemize}
\item {Utilização:Gír.}
\end{itemize}
\begin{itemize}
\item {Grp. gram.:Loc.}
\end{itemize}
\begin{itemize}
\item {Utilização:pop.}
\end{itemize}
\begin{itemize}
\item {Grp. gram.:Adj.}
\end{itemize}
\begin{itemize}
\item {Utilização:Bras. do N}
\end{itemize}
\begin{itemize}
\item {Proveniência:(De \textunderscore quebrar\textunderscore )}
\end{itemize}
Quebrada.
Fracção arithmética.
Pequeno copo.
\textunderscore Tocar\textunderscore  ou \textunderscore soar a quebrado\textunderscore , diz-se de qualquer vaso de barro que, por têr alguma fenda, não sôa, ao bater-se-lhe, como sôa um vaso inteiro ou sem quebradura.
Reduzido a pedaços, fragmentado.
Partido.
Arruinado.
Fallido.
Muito pobre.
\section{Quebrador}
\begin{itemize}
\item {Grp. gram.:m.  e  adj.}
\end{itemize}
Aquillo que quebra.
\section{Quebradura}
\begin{itemize}
\item {Grp. gram.:f.}
\end{itemize}
\begin{itemize}
\item {Proveniência:(De \textunderscore quebrar\textunderscore )}
\end{itemize}
Quebra.
Hérnia.
\section{Quebra-esquinas}
\begin{itemize}
\item {Grp. gram.:m.}
\end{itemize}
\begin{itemize}
\item {Utilização:Pop.}
\end{itemize}
Homem ocioso; aquelle que tem o hábito de namorar.
\section{Quebra-facão}
\begin{itemize}
\item {Grp. gram.:m.}
\end{itemize}
Planta amarantácea do Brasil.
\section{Quebralhão}
\begin{itemize}
\item {Grp. gram.:m.  e  adj.}
\end{itemize}
\begin{itemize}
\item {Utilização:Bras. do S}
\end{itemize}
\begin{itemize}
\item {Proveniência:(De \textunderscore quebra\textunderscore ^2)}
\end{itemize}
Muito mau ou ruim, (falando-se de pessôa ou animal).
\section{Quebra-luz}
\begin{itemize}
\item {Grp. gram.:m.}
\end{itemize}
Peça, que resguarda dos olhos a luz de um candeeiro, de uma vela, etc.; pantalha; reflectidor. Cf. Camillo, \textunderscore O Bem e o Mal\textunderscore , 212; Ortigão, \textunderscore Holanda\textunderscore , 49.
\section{Quebra-machado}
\begin{itemize}
\item {Grp. gram.:m.}
\end{itemize}
Árvore de San-Thomé, espécie de pau-preto.
\section{Quebra-mar}
\begin{itemize}
\item {Grp. gram.:m.}
\end{itemize}
Muralha ou qualquer construcção, com que se oppõe resistência ao embate das ondas ou á fôrça das correntes.
\section{Quebramento}
\begin{itemize}
\item {Grp. gram.:m.}
\end{itemize}
O mesmo que \textunderscore quebra\textunderscore ^1.
Quebreira^2.
\section{Quebrança}
\begin{itemize}
\item {Grp. gram.:f.}
\end{itemize}
\begin{itemize}
\item {Proveniência:(De \textunderscore quebrar\textunderscore )}
\end{itemize}
O quebrar das ondas nos rochedos.
\section{Quebra-nozes}
\begin{itemize}
\item {Grp. gram.:m.}
\end{itemize}
Instrumento de metal, para partír nozes.
Pássaro conirostro.
\section{Quebrantador}
\begin{itemize}
\item {Grp. gram.:m.  e  adj.}
\end{itemize}
O que quebranta.
\section{Quebrantamento}
\begin{itemize}
\item {Grp. gram.:m.}
\end{itemize}
Acto ou effeito de quebrantar.
\section{Quebrantar}
\begin{itemize}
\item {Grp. gram.:v. t.}
\end{itemize}
\begin{itemize}
\item {Utilização:Fig.}
\end{itemize}
\begin{itemize}
\item {Grp. gram.:V. p.}
\end{itemize}
\begin{itemize}
\item {Proveniência:(Do b. lat. \textunderscore crepantare\textunderscore , por metáth.)}
\end{itemize}
Quebrar.
Aluír.
Amachucar.
Infringir, transgredir.
Abater, vencer.
Ultrapassar.
Abrandar, suavizar.
Afroixar; tornar-se fraco.
Perder a coragem.
\section{Quebranto}
\begin{itemize}
\item {Grp. gram.:m.}
\end{itemize}
\begin{itemize}
\item {Proveniência:(De \textunderscore quebrantar\textunderscore )}
\end{itemize}
O mesmo que \textunderscore quebrantamento\textunderscore .
Prostração, fraqueza.
Supposto resultado mórbido, que o mau olhado de certas pessôas produz noutras, segundo a superstição popular.
\section{Quebra-panela}
\begin{itemize}
\item {Grp. gram.:f.}
\end{itemize}
Planta amarantácea.
\section{Quebra-panelas}
\begin{itemize}
\item {Grp. gram.:m.}
\end{itemize}
\begin{itemize}
\item {Utilização:T. da Bairrada}
\end{itemize}
O mesmo que \textunderscore queiró\textunderscore .
\section{Quebrar}
\begin{itemize}
\item {Grp. gram.:v. t.}
\end{itemize}
\begin{itemize}
\item {Grp. gram.:V. i.}
\end{itemize}
\begin{itemize}
\item {Utilização:Bras. do N}
\end{itemize}
Reduzir a pedaços; fragmentar.
Partir.
Fazer vincos em.
Dobrar.
Interromper, pôr fim a.
Cortar.
Torcer.
Quebrantar, infringir.
Illaquear.
Tirar as fôrças a.
Inutilizar.
Abater, enfraquecer.
Fallir.
Arruinar-se.
Empobrecer.
Adquirir hérnia.
Deminuir no pêso.
Têr falta no pêso devido.
Fazer-se em pedaços.
Partir-se de encontro a alguma coisa.
Dobrar-se, formando ângulo.
Estalar.
Refranger-se.
(Metáth. de \textunderscore crebar\textunderscore , do lat. \textunderscore crepare\textunderscore )
\section{Quebreira}
\begin{itemize}
\item {Grp. gram.:f.}
\end{itemize}
\begin{itemize}
\item {Utilização:Bras}
\end{itemize}
O mesmo que \textunderscore pindaíba\textunderscore .
\section{Quebreira}
\begin{itemize}
\item {Grp. gram.:f.}
\end{itemize}
\begin{itemize}
\item {Utilização:Pop.}
\end{itemize}
\begin{itemize}
\item {Proveniência:(De \textunderscore quebrar\textunderscore )}
\end{itemize}
Fadiga, prostração; molleza de corpo, languidez.
\section{Quebro}
\begin{itemize}
\item {Grp. gram.:m.}
\end{itemize}
\begin{itemize}
\item {Utilização:Taur.}
\end{itemize}
\begin{itemize}
\item {Utilização:Mús.}
\end{itemize}
\begin{itemize}
\item {Utilização:Ant.}
\end{itemize}
\begin{itemize}
\item {Proveniência:(De \textunderscore quebrar\textunderscore )}
\end{itemize}
Inflexão da voz.
Requebro; inflexão do corpo.
Qualquer movimento, que o toireiro faz com a cintura sem mudar os pés, para evitar a marrada.
O mesmo que \textunderscore mordente\textunderscore .
\section{Quê-cê}
\begin{itemize}
\item {Grp. gram.:m.}
\end{itemize}
\begin{itemize}
\item {Utilização:Bras. do N}
\end{itemize}
O mesmo que \textunderscore caxerenguengue\textunderscore .
\section{Queche}
\begin{itemize}
\item {Grp. gram.:m.}
\end{itemize}
\begin{itemize}
\item {Proveniência:(Do ingl. \textunderscore kecht\textunderscore )}
\end{itemize}
Espécie de navio. Cf. M. de Aguiar, \textunderscore Diccion. de Marinha\textunderscore .
\section{Quechua}
\begin{itemize}
\item {Grp. gram.:m.}
\end{itemize}
O mesmo que \textunderscore quíchua\textunderscore .
\section{Quéda}
\begin{itemize}
\item {Grp. gram.:f.}
\end{itemize}
\begin{itemize}
\item {Utilização:Fig.}
\end{itemize}
\begin{itemize}
\item {Utilização:Açor}
\end{itemize}
Acto ou effeito de caír.
Decadência; ruina.
Descrédito.
Culpa, peccado.
Declíve.
Tendência, inclinação: \textunderscore têr quéda para as letras\textunderscore .
Termo.
O salto (das botas ou dos sapatos).
\textunderscore Queda de água\textunderscore , cachoeira, ou lugar, onde um rio, braço de rio ou levada se precipita, por natural disposição do terreno ou por indústria dos homens.
(Alter. de \textunderscore caida\textunderscore , de \textunderscore cair\textunderscore )
\section{Quedar}
\begin{itemize}
\item {Grp. gram.:v. i.  e  p.}
\end{itemize}
\begin{itemize}
\item {Proveniência:(Do lat. \textunderscore quietare\textunderscore )}
\end{itemize}
Estar quedo.
Estacionar.
Permanecer; conservar-se.
Parar.
\section{Quedano}
\begin{itemize}
\item {Grp. gram.:m.}
\end{itemize}
Árvore medicinal da Ilha de San-Thomé.
\section{Quedive}
\begin{itemize}
\item {Grp. gram.:m.}
\end{itemize}
\begin{itemize}
\item {Proveniência:(Fr. \textunderscore khédire\textunderscore )}
\end{itemize}
Título do vice-rei do Egypto.
\section{Quêdo}
\begin{itemize}
\item {Grp. gram.:adj.}
\end{itemize}
\begin{itemize}
\item {Proveniência:(Do lat. \textunderscore quietus\textunderscore )}
\end{itemize}
O mesmo que \textunderscore quieto\textunderscore .
Que não tem movimento.
Suspenso.
Tranquillo.
Demorado.
\section{Quefazer}
\begin{itemize}
\item {Grp. gram.:m.}
\end{itemize}
\begin{itemize}
\item {Proveniência:(De \textunderscore quê\textunderscore  + \textunderscore fazer\textunderscore )}
\end{itemize}
Aquillo que há para se fazer.
Negócios; faina. Cf. Garrett, \textunderscore Filippa de Vilh.\textunderscore , 15; A. Candido, \textunderscore Philos. Polit.\textunderscore , (introducção)
\section{Quefazeres}
\begin{itemize}
\item {Grp. gram.:m. pl.}
\end{itemize}
\begin{itemize}
\item {Proveniência:(De \textunderscore quê\textunderscore  + \textunderscore fazer\textunderscore )}
\end{itemize}
Aquillo que há para se fazer.
Negócios; faina. Cf. Garrett, \textunderscore Filippa de Vilh.\textunderscore , 15; A. Candido, \textunderscore Philos. Polit.\textunderscore , (introducção)
\section{Quefir}
\begin{itemize}
\item {Grp. gram.:m.}
\end{itemize}
\begin{itemize}
\item {Proveniência:(Do fr. \textunderscore kefir\textunderscore )}
\end{itemize}
Bebida gasosa, fermentada, fabricada especialmente pelos montanheses do Cáucaso.
\section{Quefirina}
\begin{itemize}
\item {Grp. gram.:f.}
\end{itemize}
Preparação pulverulenta, para a fabricação de quefir.
\section{Queijada}
\begin{itemize}
\item {Grp. gram.:f.}
\end{itemize}
\begin{itemize}
\item {Proveniência:(De \textunderscore queijo\textunderscore )}
\end{itemize}
Pastel chato, feito de leite, ovos, queijo, açúcar e massa de trigo.
\section{Queijadeira}
\begin{itemize}
\item {Grp. gram.:f.}
\end{itemize}
Mulhér, que fabrica ou vende queijadas.
\section{Queijadeiro}
\begin{itemize}
\item {Grp. gram.:adj.}
\end{itemize}
\begin{itemize}
\item {Grp. gram.:M.}
\end{itemize}
Relativo a queijada.
Fabricante ou vendedor de queijadas. Cf. Castilho, \textunderscore Misanthropo\textunderscore , 159.
\section{Queijadilho}
\begin{itemize}
\item {Grp. gram.:m.}
\end{itemize}
Planta primulácea.
(Provavelmente, alter. de \textunderscore cajadilho\textunderscore , de \textunderscore cajado\textunderscore )
\section{Queijadinha}
\begin{itemize}
\item {Grp. gram.:f.}
\end{itemize}
\begin{itemize}
\item {Utilização:Bras. do N}
\end{itemize}
O mesmo que \textunderscore luminária\textunderscore , doce de côco.
\section{Queijar}
\begin{itemize}
\item {Grp. gram.:v. i.}
\end{itemize}
\begin{itemize}
\item {Utilização:Des.}
\end{itemize}
\begin{itemize}
\item {Proveniência:(De \textunderscore queijo\textunderscore )}
\end{itemize}
Fazer queijos.
Tornar-se queijo: \textunderscore esta nata está quási a queijar...\textunderscore 
\section{Queijaria}
\begin{itemize}
\item {Grp. gram.:f.}
\end{itemize}
Fabricação de queijos.
Lugar, onde se fabricam queijos, queijeira.
\section{Queijeira}
\begin{itemize}
\item {Grp. gram.:f.}
\end{itemize}
Casa, em que se fabricam queijos.
Queijadeira.
Ave, o mesmo que \textunderscore tanjasno\textunderscore .
(Fem. de \textunderscore queijeiro\textunderscore )
\section{Queijeiro}
\begin{itemize}
\item {Grp. gram.:m.}
\end{itemize}
Fabricante ou vendedor de queijos.
\section{Queijinho}
\begin{itemize}
\item {Grp. gram.:m.}
\end{itemize}
Espécie de doce, especialmente o que se fabricava no convento da Visitação, de Montemór.
\section{Queijo}
\begin{itemize}
\item {Grp. gram.:m.}
\end{itemize}
\begin{itemize}
\item {Utilização:Gír.}
\end{itemize}
\begin{itemize}
\item {Grp. gram.:Loc.}
\end{itemize}
\begin{itemize}
\item {Utilização:fam.}
\end{itemize}
\begin{itemize}
\item {Proveniência:(Do lat. \textunderscore caseus\textunderscore )}
\end{itemize}
Massa, formada pelo leite de certos animaes, depois de coalhado.
Massa alimentícia, em fórma de queijo.
Negócio.
\textunderscore Têr a faca e o queijo na mão\textunderscore , têr poder amplo, dispor inteiramente de alguma coisa.
\section{Queijoso}
\begin{itemize}
\item {Grp. gram.:adj.}
\end{itemize}
O mesmo que \textunderscore caseoso\textunderscore . Cf. Baganha, \textunderscore Hyg. Pec.\textunderscore , 210.
\section{Queima}
\begin{itemize}
\item {Grp. gram.:f.}
\end{itemize}
\begin{itemize}
\item {Utilização:Prov.}
\end{itemize}
\begin{itemize}
\item {Utilização:trasm.}
\end{itemize}
\begin{itemize}
\item {Utilização:Prov.}
\end{itemize}
\begin{itemize}
\item {Utilização:beir.}
\end{itemize}
Acto ou effeito de queimar.
Curvas, que se fazem no chão, para o jôgo da raiola.
Traços análogos, para o jôgo do pião, do homem, etc.
\section{Queima}
\begin{itemize}
\item {Grp. gram.:f.}
\end{itemize}
\begin{itemize}
\item {Utilização:Prov.}
\end{itemize}
(Fórma pop. de \textunderscore cóima\textunderscore )
\section{Queimação}
\begin{itemize}
\item {Grp. gram.:f.}
\end{itemize}
\begin{itemize}
\item {Utilização:Fig.}
\end{itemize}
\begin{itemize}
\item {Proveniência:(De \textunderscore queimar\textunderscore )}
\end{itemize}
Queima.
Impertinência; enfadamento; coisa que molesta.
\section{Queimada}
\begin{itemize}
\item {Grp. gram.:f.}
\end{itemize}
\begin{itemize}
\item {Utilização:Pesc.}
\end{itemize}
Queima de mato, arvoredo, etc., para se semearem cereaes no respectivo terreno.
Lugar, onde se queimou mato.
Terra calcinada, própria para adubo.
Cardume de sardinhas.
(Fem. de \textunderscore queimado\textunderscore )
\section{Queimadeira}
\begin{itemize}
\item {Grp. gram.:f.}
\end{itemize}
Planta plumbagínea do Brasil.
Planta euphorbiácea da mesma região.
\section{Queimadeiro}
\begin{itemize}
\item {Grp. gram.:m.}
\end{itemize}
\begin{itemize}
\item {Proveniência:(De \textunderscore queimar\textunderscore )}
\end{itemize}
Lugar, onde se faziam as fogueiras para queimar os condemnados á pena de fogo. Cf. Ortigão, \textunderscore Holanda\textunderscore , 316.
\section{Queimadela}
\begin{itemize}
\item {Grp. gram.:f.}
\end{itemize}
O mesmo que \textunderscore queimadura\textunderscore .
\section{Queimado}
\begin{itemize}
\item {Grp. gram.:adj.}
\end{itemize}
\begin{itemize}
\item {Utilização:Bras}
\end{itemize}
\begin{itemize}
\item {Grp. gram.:M.}
\end{itemize}
\begin{itemize}
\item {Proveniência:(De \textunderscore queimar\textunderscore )}
\end{itemize}
Ardente; em que há muito calor.
Zangado.
Esturro, sabor ou cheiro da comida que se esturrou.
Espécie de jôgo popular.
\section{Queimado}
\begin{itemize}
\item {Grp. gram.:m.}
\end{itemize}
\begin{itemize}
\item {Utilização:Bras}
\end{itemize}
O mesmo que \textunderscore bala\textunderscore ^2 ou \textunderscore rebuçado\textunderscore .
\section{Queimadoiro}
\begin{itemize}
\item {Grp. gram.:m.}
\end{itemize}
O mesmo ou melhor que \textunderscore queimadeiro\textunderscore . Cf. Alv. Mendes, \textunderscore Discursos\textunderscore , 266.
\section{Queimador}
\begin{itemize}
\item {Grp. gram.:m.  e  adj.}
\end{itemize}
O que queima.
\section{Queimadouro}
\begin{itemize}
\item {Grp. gram.:m.}
\end{itemize}
O mesmo ou melhor que \textunderscore queimadeiro\textunderscore . Cf. Alv. Mendes, \textunderscore Discursos\textunderscore , 266.
\section{Queimadura}
\begin{itemize}
\item {Grp. gram.:f.}
\end{itemize}
\begin{itemize}
\item {Proveniência:(De \textunderscore queimar\textunderscore )}
\end{itemize}
O mesmo que \textunderscore queima\textunderscore ^1.
Ferimento ou lesão, produzida pela acção do fogo.
Alforra.
\section{Queimamento}
\begin{itemize}
\item {Grp. gram.:m.}
\end{itemize}
O mesmo que \textunderscore queima\textunderscore ^1.
\section{Queimante}
\begin{itemize}
\item {Grp. gram.:adj.}
\end{itemize}
Que queima; picante, acre.
\section{Queimão}
\begin{itemize}
\item {Grp. gram.:m.}
\end{itemize}
O mesmo que \textunderscore quimão\textunderscore .
\section{Queimão}
\begin{itemize}
\item {Grp. gram.:m.}
\end{itemize}
\begin{itemize}
\item {Utilização:Prov.}
\end{itemize}
Diz-se do pimento que queima muito, que é muito picante.
\section{Queimar}
\begin{itemize}
\item {Grp. gram.:v. t.}
\end{itemize}
\begin{itemize}
\item {Utilização:Fig.}
\end{itemize}
\begin{itemize}
\item {Grp. gram.:V. i.}
\end{itemize}
\begin{itemize}
\item {Proveniência:(Do lat. \textunderscore cremare\textunderscore )}
\end{itemize}
Destruir pelo fogo.
Destruir.
Converter em cinza.
Esbrasear, tornar ardente.
Tostar, crestar.
Aquecer muito, produzindo dôr.
Escaldar.
Afoguear.
Esbanjar, dissipar.
Estar muito quente.
Escaldar.
Causar ardor febril.
\section{Queimarço}
\begin{itemize}
\item {Grp. gram.:m.}
\end{itemize}
\begin{itemize}
\item {Utilização:Pop.}
\end{itemize}
\begin{itemize}
\item {Proveniência:(De \textunderscore queimar\textunderscore )}
\end{itemize}
Febre muito ardente; grande accréscimo de febre.
\section{Queima-roupa}
\begin{itemize}
\item {Grp. gram.:f.}
\end{itemize}
Us. apenas na loc. adv. \textunderscore á queima-roupa\textunderscore , que quer dizer \textunderscore muito de perto\textunderscore , \textunderscore cara a cara\textunderscore , \textunderscore corpo a corpo\textunderscore .
\section{Queimo}
\begin{itemize}
\item {Grp. gram.:m.}
\end{itemize}
\begin{itemize}
\item {Proveniência:(De \textunderscore queimar\textunderscore )}
\end{itemize}
Sabor picante, acre.
\section{Queimor}
\begin{itemize}
\item {Grp. gram.:m.}
\end{itemize}
O mesmo que \textunderscore queimo\textunderscore . Cf. Garrett, \textunderscore Flores sem Fruto\textunderscore , 99.
Calor enorme:«\textunderscore do estio ardentes queimores...\textunderscore »Garrett, \textunderscore Adozinda\textunderscore .
\section{Queimoso}
\begin{itemize}
\item {Grp. gram.:adj.}
\end{itemize}
\begin{itemize}
\item {Proveniência:(De \textunderscore queimar\textunderscore )}
\end{itemize}
Queimante; acre, picante.
Calmoso.
\section{Queira}
\begin{itemize}
\item {Grp. gram.:f.}
\end{itemize}
\begin{itemize}
\item {Utilização:Prov.}
\end{itemize}
\begin{itemize}
\item {Utilização:trasm.}
\end{itemize}
\begin{itemize}
\item {Proveniência:(De \textunderscore cão\textunderscore )}
\end{itemize}
(V.cãira)
O mesmo que \textunderscore matilha\textunderscore .
\section{Queiro}
\begin{itemize}
\item {Grp. gram.:adj.}
\end{itemize}
\begin{itemize}
\item {Utilização:bras}
\end{itemize}
\begin{itemize}
\item {Utilização:Ant.}
\end{itemize}
Diz-se do chamado \textunderscore dente do siso\textunderscore :«\textunderscore maceote lá o dente queiro?\textunderscore »\textunderscore Eufrosina\textunderscore , 82.
\section{Queiró}
\begin{itemize}
\item {Grp. gram.:f.}
\end{itemize}
\begin{itemize}
\item {Utilização:Prov.}
\end{itemize}
Urze do mato, (\textunderscore calluna vulgaris\textunderscore , Salisb.).
Flôr da urze do mato.
\section{Queiroga}
\begin{itemize}
\item {Grp. gram.:f.}
\end{itemize}
O mesmo que \textunderscore queiró\textunderscore .
\section{Queirós}
\begin{itemize}
\item {Grp. gram.:f.}
\end{itemize}
O mesmo que \textunderscore queiró\textunderscore .
\section{Queixa}
\begin{itemize}
\item {Grp. gram.:f.}
\end{itemize}
Acto ou effeito de se queixar.
Causa de resentimento, offensa.
Descontentamento.
Participação, feita á autoridade, sôbre offensas recebidas.
Querela.
\section{Queixa}
\begin{itemize}
\item {Grp. gram.:f.}
\end{itemize}
\begin{itemize}
\item {Utilização:Prov.}
\end{itemize}
Cada uma das duas travessas de madeira, que seguram entre si, superior e inferiormente, os dentes do pente do tear.
\section{Queixada}
\begin{itemize}
\item {Grp. gram.:f.}
\end{itemize}
\begin{itemize}
\item {Grp. gram.:M.}
\end{itemize}
\begin{itemize}
\item {Utilização:Bras}
\end{itemize}
\begin{itemize}
\item {Proveniência:(De \textunderscore queixo\textunderscore ^1)}
\end{itemize}
O mesmo que \textunderscore maxilla\textunderscore .
Javali de queixo branco.
\section{Queixagens}
\begin{itemize}
\item {Grp. gram.:f. pl.}
\end{itemize}
\begin{itemize}
\item {Utilização:Prov.}
\end{itemize}
\begin{itemize}
\item {Proveniência:(De \textunderscore queixo\textunderscore ^1)}
\end{itemize}
Guelras de peixe.
\section{Queixal}
\begin{itemize}
\item {Grp. gram.:m.  e  adj.}
\end{itemize}
\begin{itemize}
\item {Proveniência:(De \textunderscore queixo\textunderscore ^1)}
\end{itemize}
Dente molar.
\section{Queixar-se}
\begin{itemize}
\item {Grp. gram.:v. p.}
\end{itemize}
\begin{itemize}
\item {Proveniência:(Do b. lat. \textunderscore quaestiare\textunderscore )}
\end{itemize}
Manifestar dôr, lamentar-se.
Mostrar-se offendido.
Fazer denúncia do mal ou offensa que se recebeu.
Fazer censura.
\section{Queixeiro}
\begin{itemize}
\item {Grp. gram.:adj.}
\end{itemize}
\begin{itemize}
\item {Proveniência:(De \textunderscore queixo\textunderscore ^1)}
\end{itemize}
Diz-se do dente, chamado vulgarmente \textunderscore dente-do-siso\textunderscore .
\section{Queixique}
\begin{itemize}
\item {Grp. gram.:m.}
\end{itemize}
\begin{itemize}
\item {Utilização:Ant.}
\end{itemize}
Queixada?:«\textunderscore um bacorote, de queixique espantoso, trombejava.\textunderscore »Sá de Miranda.
\section{Queixo}
\begin{itemize}
\item {Grp. gram.:m.}
\end{itemize}
\begin{itemize}
\item {Proveniência:(Do lat. \textunderscore capsus\textunderscore ?)}
\end{itemize}
Maxilla dos vertebrados.
Maxilla inferior; mento.
\section{Queixo}
\begin{itemize}
\item {Grp. gram.:m.}
\end{itemize}
\begin{itemize}
\item {Utilização:Ant.}
\end{itemize}
O mesmo que \textunderscore queijo\textunderscore .
\section{Queixosa}
\begin{itemize}
\item {Grp. gram.:f.}
\end{itemize}
\begin{itemize}
\item {Proveniência:(De \textunderscore queixoso\textunderscore )}
\end{itemize}
Mulhér, que se queixa; mulhér que, offendida por alguém, fez queixa da offensa aos tribunaes.
\section{Queixosamente}
\begin{itemize}
\item {Grp. gram.:adv.}
\end{itemize}
De modo queixoso.
\section{Queixoso}
\begin{itemize}
\item {Grp. gram.:m.  e  adj.}
\end{itemize}
O que se queixa.
O que é offendido, e que faz reclamações em juízo contra o offensor.
Aquillo em que há queixa.
\section{Queixudo}
\begin{itemize}
\item {Grp. gram.:adj.}
\end{itemize}
\begin{itemize}
\item {Utilização:Pop.}
\end{itemize}
Que tem os queixos grandes, ou a maxilla inferior muito proeminente.
\section{Queixume}
\begin{itemize}
\item {Grp. gram.:m.}
\end{itemize}
Queixa, lamentação.
\section{Quejadilho}
\begin{itemize}
\item {Grp. gram.:m.}
\end{itemize}
Planta, o mesmo ou melhor que \textunderscore queijadilho\textunderscore . Cf. P. Coutinho, \textunderscore Flora\textunderscore , 466.
\section{Quejando}
\begin{itemize}
\item {Grp. gram.:adj.}
\end{itemize}
\begin{itemize}
\item {Proveniência:(De \textunderscore que\textunderscore  + ant. \textunderscore jando\textunderscore , do lat. \textunderscore genitus\textunderscore )}
\end{itemize}
Que tem a mesma natureza ou qualidade.
De que qualidade ou de que modo.
Qual.
\section{Quejendo}
\begin{itemize}
\item {Grp. gram.:adj.}
\end{itemize}
\begin{itemize}
\item {Utilização:Ant.}
\end{itemize}
O mesmo que \textunderscore quejando\textunderscore .
\section{Quele}
\begin{itemize}
\item {Grp. gram.:m.}
\end{itemize}
Ave trepadora da África.
\section{Quelha}
\begin{itemize}
\item {fónica:quêouqué}
\end{itemize}
\begin{itemize}
\item {Grp. gram.:f.}
\end{itemize}
\begin{itemize}
\item {Utilização:Prov.}
\end{itemize}
\begin{itemize}
\item {Utilização:Prov.}
\end{itemize}
\begin{itemize}
\item {Utilização:trasm.}
\end{itemize}
\begin{itemize}
\item {Proveniência:(Do lat. \textunderscore canalicula\textunderscore )}
\end{itemize}
Calha.
Rua estreita.
Cano descoberto.
Peça de madeira, que forma ângulo reentrante, por onde o grão que sai da tremonha corre para o ôlho da mó, nos moínhos de cereaes.
Socalco de terra lavradia.
O mesmo que \textunderscore quelho\textunderscore .
\section{Quelho}
\begin{itemize}
\item {fónica:quê}
\end{itemize}
\begin{itemize}
\item {Grp. gram.:m.}
\end{itemize}
\begin{itemize}
\item {Utilização:Prov.}
\end{itemize}
Caleira, por onde o grão desce da canoira e que tambem se chama \textunderscore quelha\textunderscore .
O mesmo que \textunderscore bêco\textunderscore , viela ou quelha.
\section{Quelhório}
\begin{itemize}
\item {Grp. gram.:m.}
\end{itemize}
\begin{itemize}
\item {Utilização:Prov.}
\end{itemize}
\begin{itemize}
\item {Proveniência:(De \textunderscore quelha\textunderscore )}
\end{itemize}
Socalco de terra lavradia, muito estreito e pouco productivo.
\section{Quelidro}
\begin{itemize}
\item {Grp. gram.:m.}
\end{itemize}
\begin{itemize}
\item {Proveniência:(Lat. \textunderscore chelidrus\textunderscore )}
\end{itemize}
Nome de uma serpente anfíbia e venenosa. Cf. Filinto, VI, 235.
\section{Quelma}
\begin{itemize}
\item {Grp. gram.:f.}
\end{itemize}
O mesmo que \textunderscore quelme\textunderscore .
\section{Quelme}
\begin{itemize}
\item {Grp. gram.:m.}
\end{itemize}
Peixe do Algarve e dos Açores.
\section{Quelónios}
\begin{itemize}
\item {Grp. gram.:m. pl.}
\end{itemize}
\begin{itemize}
\item {Proveniência:(Do gr. \textunderscore khelone\textunderscore )}
\end{itemize}
Ordem da classe dos reptis, que têm por tipo a tartaruga.
\section{Quelonita}
\begin{itemize}
\item {Grp. gram.:f.}
\end{itemize}
\begin{itemize}
\item {Proveniência:(Do gr. \textunderscore khelone\textunderscore )}
\end{itemize}
Tartaruga petrificada.
\section{Quelonófago}
\begin{itemize}
\item {Grp. gram.:adj.}
\end{itemize}
\begin{itemize}
\item {Proveniência:(Do gr. \textunderscore khelone\textunderscore  + \textunderscore phagein\textunderscore )}
\end{itemize}
Que se alimenta de tartarugas; que come tartarugas.
\section{Quelonografia}
\begin{itemize}
\item {Grp. gram.:f.}
\end{itemize}
\begin{itemize}
\item {Proveniência:(Do gr. \textunderscore khelone\textunderscore  + \textunderscore graphein\textunderscore )}
\end{itemize}
Descripção das tartarugas.
\section{Quelonógrafo}
\begin{itemize}
\item {Grp. gram.:m.}
\end{itemize}
Naturalista, que se ocupa especialmente das tartarugas.
(Cp. \textunderscore quelonografia\textunderscore )
\section{Quem}
\begin{itemize}
\item {Grp. gram.:pron.}
\end{itemize}
\begin{itemize}
\item {Proveniência:(Lat. \textunderscore quem\textunderscore )}
\end{itemize}
A pessôa ou as pessôas que: \textunderscore êlle foi quem me enganou\textunderscore .
A qual, as quaes; qual pessôa: \textunderscore quem vem lá\textunderscore ?
Alguém que.
O que.--Este pronome refere-se a pessôas; mas em Filinto o vejo, mais de uma vez, referido a coisas:«\textunderscore ...as naus, a quem próspero vento enfunava as velas...\textunderscore »Filinto, \textunderscore D. Man.\textunderscore , II, 52,«\textunderscore ...o mais célebre empório..., a quem concorrião mercadores.\textunderscore »\textunderscore Idem\textunderscore , \textunderscore ibidem\textunderscore , 102. E nos \textunderscore Lusiadas\textunderscore , III, 6:«\textunderscore faz a soberba Europa, a quem rodeia... o oceano.\textunderscore »
\section{Quemi}
\begin{itemize}
\item {Grp. gram.:m.}
\end{itemize}
Mammífero roedor da ilha de Cuba.
\section{Quemose}
\begin{itemize}
\item {Grp. gram.:f.}
\end{itemize}
\begin{itemize}
\item {Utilização:Med.}
\end{itemize}
\begin{itemize}
\item {Proveniência:(Gr. \textunderscore khemosis\textunderscore )}
\end{itemize}
Espécie de conjuntivite.
\section{Quem-te-pesa}
\begin{itemize}
\item {Grp. gram.:m.}
\end{itemize}
Espécie de jôgo popular.
\section{Quenga}
\begin{itemize}
\item {Grp. gram.:f.}
\end{itemize}
\begin{itemize}
\item {Utilização:Bras. da Baía}
\end{itemize}
Guisado de gallinha com quiabos.
\section{Quenga}
\begin{itemize}
\item {Grp. gram.:f.}
\end{itemize}
\begin{itemize}
\item {Utilização:Bras. do N}
\end{itemize}
Meretriz.
O mesmo que \textunderscore quengo\textunderscore .
\section{Quengo}
\begin{itemize}
\item {Grp. gram.:m.}
\end{itemize}
\begin{itemize}
\item {Utilização:Bras. do N}
\end{itemize}
\begin{itemize}
\item {Utilização:Fig.}
\end{itemize}
Espécie de vaso com cabo, feito da metade do endocarpo do côco, e que serve para tirar caldo da panela.
Cabeça.
Intelligência, talento.
\section{Quenopodiáceas}
\begin{itemize}
\item {Grp. gram.:f. pl.}
\end{itemize}
Família de plantas, que têm por tipo o \textunderscore quenopódio\textunderscore .
\section{Quenopódio}
\begin{itemize}
\item {Grp. gram.:m.}
\end{itemize}
\begin{itemize}
\item {Proveniência:(Do gr. \textunderscore khen\textunderscore  + \textunderscore pous\textunderscore , \textunderscore podos\textunderscore )}
\end{itemize}
O mesmo que \textunderscore anserina\textunderscore .
\section{Quentadura}
\begin{itemize}
\item {Grp. gram.:f.}
\end{itemize}
\begin{itemize}
\item {Utilização:Ant.}
\end{itemize}
O mesmo que \textunderscore quentura\textunderscore . Cf. \textunderscore Regim. contra a Pestenença\textunderscore , (séc. XVI).
\section{Quentar}
\textunderscore v. t.\textunderscore  (e der.)
O mesmo que \textunderscore aquentar\textunderscore , etc. Cf. Filinto, XIII, 64.
\section{Quente}
\begin{itemize}
\item {Grp. gram.:adj.}
\end{itemize}
\begin{itemize}
\item {Utilização:Fig.}
\end{itemize}
\begin{itemize}
\item {Grp. gram.:M.}
\end{itemize}
\begin{itemize}
\item {Utilização:Fam.}
\end{itemize}
\begin{itemize}
\item {Proveniência:(Do lat. \textunderscore calens\textunderscore )}
\end{itemize}
Em que há calor.
Que tem temperatura elevada: \textunderscore um dia quente\textunderscore .
Queimoso, picante.
Cálido.
Que dá a sensação de calor.
Em que há vida, enthusiasmo, muito sentimento.
Lugar quente, a cama: \textunderscore Fulano ainda está no quente\textunderscore , ainda se não levantou, ainda está na cama. Cf. Rebello, \textunderscore Mocidade\textunderscore , II, 153.
\section{Quentura}
\begin{itemize}
\item {Grp. gram.:f.}
\end{itemize}
Estado do que é quente; calor.
\section{Quepe}
\begin{itemize}
\item {Grp. gram.:m.}
\end{itemize}
\begin{itemize}
\item {Proveniência:(Do al. \textunderscore kappe\textunderscore )}
\end{itemize}
Boné, usado por militares de vários países.
\section{Quépi}
\begin{itemize}
\item {Grp. gram.:m.}
\end{itemize}
\begin{itemize}
\item {Proveniência:(Do al. \textunderscore kappe\textunderscore )}
\end{itemize}
Boné, usado por militares de vários países.
\section{Queque}
\begin{itemize}
\item {Grp. gram.:m.}
\end{itemize}
\begin{itemize}
\item {Proveniência:(Do ingl. \textunderscore cake\textunderscore )}
\end{itemize}
Variedade de bolo, feito de manteiga, açúcar e ovos, e semelhante ao pão de ló, mas mais compacto. Cf. G. Braga, \textunderscore Mal da Delf.\textunderscore , 63.
\section{Quér}
\begin{itemize}
\item {Grp. gram.:conj.}
\end{itemize}
\begin{itemize}
\item {Proveniência:(De \textunderscore querer\textunderscore )}
\end{itemize}
O mesmo que \textunderscore ou\textunderscore ^1: \textunderscore quér appareças, quér não...\textunderscore --Em muitas proposições entra como particula expletiva.
\section{Quera}
\begin{itemize}
\item {Grp. gram.:adj.}
\end{itemize}
\begin{itemize}
\item {Utilização:Bras}
\end{itemize}
Valente.
(Do tupi)
\section{Quercina}
\begin{itemize}
\item {fónica:cu-e}
\end{itemize}
\begin{itemize}
\item {Grp. gram.:f.}
\end{itemize}
\begin{itemize}
\item {Proveniência:(Do lat. \textunderscore quercus\textunderscore )}
\end{itemize}
Substância crystallizada, que se extrái do carvalho.
\section{Quercínias}
\begin{itemize}
\item {fónica:cu-er}
\end{itemize}
\begin{itemize}
\item {Grp. gram.:f. pl.}
\end{itemize}
Família de plantas, o mesmo que \textunderscore cupulíferas\textunderscore .
\section{Quercite}
\begin{itemize}
\item {fónica:cu-er}
\end{itemize}
\begin{itemize}
\item {Grp. gram.:f.}
\end{itemize}
\begin{itemize}
\item {Proveniência:(Do lat. \textunderscore quercus\textunderscore )}
\end{itemize}
Matéria saccharina, encontrada na glande do carvalho.
\section{Quercitrina}
\begin{itemize}
\item {fónica:cu-er}
\end{itemize}
\begin{itemize}
\item {Grp. gram.:f.}
\end{itemize}
Substância còrante de uma espécie de carvalho, (\textunderscore quercus tinctoríus\textunderscore ).
\section{Querco}
\begin{itemize}
\item {fónica:cu-er}
\end{itemize}
\begin{itemize}
\item {Grp. gram.:m.}
\end{itemize}
\begin{itemize}
\item {Utilização:Poét.}
\end{itemize}
\begin{itemize}
\item {Proveniência:(Lat. \textunderscore quercus\textunderscore )}
\end{itemize}
Carvalho; roble.
\section{Querela}
\begin{itemize}
\item {Grp. gram.:f.}
\end{itemize}
\begin{itemize}
\item {Utilização:Poét.}
\end{itemize}
\begin{itemize}
\item {Proveniência:(Lat. \textunderscore querela\textunderscore )}
\end{itemize}
Accusação criminal, apresentada em juízo contra alguém.
Discussão; pendência.
Queixa; canto plangente.
\section{Querelado}
\begin{itemize}
\item {Grp. gram.:m.}
\end{itemize}
\begin{itemize}
\item {Proveniência:(De \textunderscore querelar\textunderscore )}
\end{itemize}
Indivíduo, contra quem se requereu querela.
\section{Querelador}
\begin{itemize}
\item {Grp. gram.:m.  e  adj.}
\end{itemize}
O que querela.
\section{Querelante}
\begin{itemize}
\item {Grp. gram.:m.  e  adj.}
\end{itemize}
\begin{itemize}
\item {Proveniência:(Lat. \textunderscore querelans\textunderscore )}
\end{itemize}
O que querela.
\section{Querelar}
\begin{itemize}
\item {Grp. gram.:v. i.}
\end{itemize}
\begin{itemize}
\item {Grp. gram.:V. p.}
\end{itemize}
\begin{itemize}
\item {Proveniência:(Lat. \textunderscore querelari\textunderscore )}
\end{itemize}
Apresentar accusação criminal, em juízo.
Promover querela.
Queixar-se.
\section{Quereloso}
\begin{itemize}
\item {Grp. gram.:adj.}
\end{itemize}
\begin{itemize}
\item {Proveniência:(Lat. \textunderscore querelosus\textunderscore )}
\end{itemize}
Queixoso.
\section{Querena}
\begin{itemize}
\item {Grp. gram.:f.}
\end{itemize}
\begin{itemize}
\item {Utilização:Pop.}
\end{itemize}
\begin{itemize}
\item {Proveniência:(Do lat. \textunderscore carina\textunderscore )}
\end{itemize}
Parte do navio, que fica abaixo do nível da água.
Rumo, direcção.
\section{Querenar}
\begin{itemize}
\item {Grp. gram.:v. t.}
\end{itemize}
Virar de querena (um navio), para o consertar ou limpar.
Construir a querena de (um navio). Cf. Vieira, XIII, 23.
\section{Querença}
\begin{itemize}
\item {Grp. gram.:f.}
\end{itemize}
Acto ou effeito de querer.
Lugar, onde se criam falcões.
Sítio, que os animaes preferem.
\section{Querencho}
\begin{itemize}
\item {Grp. gram.:adj.}
\end{itemize}
\begin{itemize}
\item {Utilização:Prov.}
\end{itemize}
\begin{itemize}
\item {Utilização:trasm.}
\end{itemize}
Enfatuado; empertigado; cheio de si.
\section{Querência}
\begin{itemize}
\item {Grp. gram.:f.}
\end{itemize}
\begin{itemize}
\item {Utilização:Bras. do S}
\end{itemize}
Lugar, onde o gado se cria ou anda pastando, ou onde costuma parar por ter affeição a esse lugar.
Cp. \textunderscore querença\textunderscore .
(Cast. \textunderscore querencia\textunderscore )
\section{Querençoso}
\begin{itemize}
\item {Grp. gram.:adj.}
\end{itemize}
Que tem querença.
Affectuoso; benévolo.
\section{Querençudo}
\begin{itemize}
\item {Grp. gram.:adj.}
\end{itemize}
\begin{itemize}
\item {Utilização:Pop.}
\end{itemize}
O mesmo que \textunderscore querençoso\textunderscore ; caricioso, affectuoso: \textunderscore não se imagina como êste cão é querençudo para o dono\textunderscore . (Colhido no Baixo Alentejo)
\section{Querendão}
\begin{itemize}
\item {Grp. gram.:m.}
\end{itemize}
\begin{itemize}
\item {Utilização:Bras. do S}
\end{itemize}
\begin{itemize}
\item {Proveniência:(De \textunderscore querer\textunderscore )}
\end{itemize}
Namorador.
\section{Querendeiro}
\begin{itemize}
\item {Grp. gram.:adj.}
\end{itemize}
\begin{itemize}
\item {Utilização:Prov.}
\end{itemize}
\begin{itemize}
\item {Utilização:minh.}
\end{itemize}
\begin{itemize}
\item {Proveniência:(De \textunderscore querer\textunderscore )}
\end{itemize}
Insinuante; que por suas meiguices se faz querido.
\section{Quereneiro}
\begin{itemize}
\item {Grp. gram.:m.}
\end{itemize}
Fabricante de querenas.
\section{Querente}
\begin{itemize}
\item {Grp. gram.:adj.}
\end{itemize}
\begin{itemize}
\item {Proveniência:(Lat. \textunderscore quaerens\textunderscore )}
\end{itemize}
Que quere ou deseja alguma coisa.
\section{Querequexe}
\begin{itemize}
\item {Grp. gram.:m.}
\end{itemize}
\begin{itemize}
\item {Utilização:Bras}
\end{itemize}
O mesmo que \textunderscore canza\textunderscore .
\section{Querer}
\begin{itemize}
\item {Grp. gram.:v. t.}
\end{itemize}
\begin{itemize}
\item {Grp. gram.:V. i.}
\end{itemize}
\begin{itemize}
\item {Grp. gram.:M.}
\end{itemize}
\begin{itemize}
\item {Proveniência:(Lat. \textunderscore quaerere\textunderscore )}
\end{itemize}
Procurar adquirir: \textunderscore querer noiva\textunderscore .
Têr vontade de: \textunderscore querer jantar\textunderscore .
Projectar, tencionar.
Resolver-se a; dignar-se: \textunderscore queira attender-me\textunderscore .
Indicar.
Demandar.
Ambicionar: \textunderscore querer riquezas\textunderscore .
Consentir: \textunderscore não querer desattenções\textunderscore .
Pedir como preço: \textunderscore o chapeleiro queria 5:000 reis por êste chapéu\textunderscore .
Pretender.
Tender para.
Têr affecto a: \textunderscore queria muito os filhos\textunderscore .
Opinar; ordenar: \textunderscore quero que venhas\textunderscore .
Prestar-se a.
Têr amor ou affecto: \textunderscore queria-lhe muito\textunderscore .
Acto de querer.
Vontade; intenção.
Affecto.
\section{Queri}
\begin{itemize}
\item {Grp. gram.:m.}
\end{itemize}
\begin{itemize}
\item {Utilização:Bras}
\end{itemize}
Árvore silvestre.
\section{Querida}
\begin{itemize}
\item {Grp. gram.:f.}
\end{itemize}
\begin{itemize}
\item {Proveniência:(De \textunderscore querido\textunderscore )}
\end{itemize}
Mulhér, a quem outrem quere bem, ou a quem muito estima.
\section{Querido}
\begin{itemize}
\item {Grp. gram.:m.}
\end{itemize}
\begin{itemize}
\item {Proveniência:(De \textunderscore querer\textunderscore )}
\end{itemize}
Indivíduo, que é estimado ou amado por outro.
\section{Querima}
\begin{itemize}
\item {Grp. gram.:f.}
\end{itemize}
\begin{itemize}
\item {Utilização:Ant.}
\end{itemize}
O mesmo que \textunderscore querimónia\textunderscore .
\section{Querimónia}
\begin{itemize}
\item {Grp. gram.:f.}
\end{itemize}
\begin{itemize}
\item {Utilização:Ant.}
\end{itemize}
\begin{itemize}
\item {Proveniência:(Lat. \textunderscore querimonia\textunderscore )}
\end{itemize}
Queixa; querela.
\section{Quermes}
\begin{itemize}
\item {Grp. gram.:m.}
\end{itemize}
Excrescência vermelha e redonda, formada pela fêmea do pulgão sôbre as fôlhas de uma espécie de carvalho, e de que se extrai uma côr escarlate, utilizada na indústria.
Producto pharmacêutico, resultante da fusão do sulfureto do antimónio em pó e do carbonato de soda crystallizado.
(Ár. \textunderscore kermes\textunderscore )
\section{Quernite}
\begin{itemize}
\item {Grp. gram.:f.}
\end{itemize}
\begin{itemize}
\item {Proveniência:(Gr. \textunderscore khernites\textunderscore )}
\end{itemize}
Pedra branca, alabastro fino.
\section{Quero-mana}
\begin{itemize}
\item {Grp. gram.:m.}
\end{itemize}
\begin{itemize}
\item {Utilização:Bras. do S}
\end{itemize}
Bailado campestre, espécie de fandango.
\section{Quero-quero}
\begin{itemize}
\item {Grp. gram.:m.}
\end{itemize}
\begin{itemize}
\item {Utilização:Bras}
\end{itemize}
Ave, do tamanho de uma perdiz, e cujo canto imita a pronúncia daquelle nome.
\section{Quérquera}
\begin{itemize}
\item {fónica:cu-ér}
\end{itemize}
\begin{itemize}
\item {Grp. gram.:f.}
\end{itemize}
\begin{itemize}
\item {Proveniência:(Lat. \textunderscore querquera\textunderscore )}
\end{itemize}
Accesso febril, acompanhado de calefrios.
\section{Querróbia}
\begin{itemize}
\item {Grp. gram.:f.}
\end{itemize}
\begin{itemize}
\item {Utilização:Prov.}
\end{itemize}
Súcia.
Folgança ruidosa.
\section{Querruxe!}
\begin{itemize}
\item {Grp. gram.:interj.}
\end{itemize}
\begin{itemize}
\item {Utilização:Prov.}
\end{itemize}
Voz, com que se chamam os porcos.
\section{Quersoneso}
\begin{itemize}
\item {Grp. gram.:m.}
\end{itemize}
\begin{itemize}
\item {Utilização:Ant.}
\end{itemize}
\begin{itemize}
\item {Proveniência:(Do gr. \textunderscore khersos\textunderscore , terra, e \textunderscore nesos\textunderscore , ilha)}
\end{itemize}
O mesmo que \textunderscore península\textunderscore .
\section{Querúbico}
\begin{itemize}
\item {Grp. gram.:adj.}
\end{itemize}
O mesmo que \textunderscore querubínico\textunderscore .
\section{Querubim}
\begin{itemize}
\item {Grp. gram.:m.}
\end{itemize}
\begin{itemize}
\item {Proveniência:(Lat. eccl. \textunderscore cherubim\textunderscore )}
\end{itemize}
Anjo da primeira jerarchia, segundo a Teologia.
Anjo.
Pintura ou escultura de uma cabeça de criança com asas, representando um querubim.
\section{Querubínico}
\begin{itemize}
\item {Grp. gram.:adj.}
\end{itemize}
Relativo a querubim.
\section{Quérulo}
\begin{itemize}
\item {fónica:cu-é}
\end{itemize}
\begin{itemize}
\item {Grp. gram.:adj.}
\end{itemize}
\begin{itemize}
\item {Utilização:Poét.}
\end{itemize}
\begin{itemize}
\item {Proveniência:(Lat. \textunderscore querulus\textunderscore )}
\end{itemize}
Queixoso, plangente.
\section{Qués}
Fórma pop. e ant. da 2.^a pess. do indic. pres. do v. \textunderscore querer\textunderscore . Cf. G. Vicente.
\section{Quesito}
\begin{itemize}
\item {Grp. gram.:m.}
\end{itemize}
\begin{itemize}
\item {Proveniência:(Lat. \textunderscore quaesitum\textunderscore )}
\end{itemize}
Interrogação ou questão, sôbre que se pede a opinião ou juízo de alguém.
Requisito.
\section{Quesitor}
\begin{itemize}
\item {Grp. gram.:m.}
\end{itemize}
\begin{itemize}
\item {Proveniência:(Lat. \textunderscore quaesitor\textunderscore )}
\end{itemize}
Juíz que, entre os Romanos, instruía ou relatava os processos criminaes.
\section{Questa}
\begin{itemize}
\item {Grp. gram.:f.}
\end{itemize}
\begin{itemize}
\item {Utilização:Des.}
\end{itemize}
\begin{itemize}
\item {Utilização:Ant.}
\end{itemize}
\begin{itemize}
\item {Proveniência:(Lat. \textunderscore quaesta\textunderscore )}
\end{itemize}
O mesmo que \textunderscore queixa\textunderscore ^1.
Peditório; acto de pedir esmola.
\section{Questão}
\begin{itemize}
\item {Grp. gram.:f.}
\end{itemize}
\begin{itemize}
\item {Proveniência:(Do lat. \textunderscore quaestio\textunderscore )}
\end{itemize}
Pergunta.
These, assumpto.
Discussão, contenda.
\section{Questionador}
\begin{itemize}
\item {Grp. gram.:m.  e  adj.}
\end{itemize}
O que questiona.
\section{Questionar}
\begin{itemize}
\item {Grp. gram.:v. t.}
\end{itemize}
\begin{itemize}
\item {Grp. gram.:V. i.}
\end{itemize}
\begin{itemize}
\item {Proveniência:(Lat. \textunderscore quaestionare\textunderscore )}
\end{itemize}
Fazer questão sôbre.
Discutir; contestar.
Altercar.
\section{Questionário}
\begin{itemize}
\item {Grp. gram.:m.}
\end{itemize}
\begin{itemize}
\item {Proveniência:(Lat. \textunderscore quaestionarius\textunderscore )}
\end{itemize}
Compilação ou série de questões ou perguntas.
\section{Questionável}
\begin{itemize}
\item {Grp. gram.:adj.}
\end{itemize}
Que se póde questionar.
\section{Questiúncula}
\begin{itemize}
\item {Grp. gram.:f.}
\end{itemize}
\begin{itemize}
\item {Proveniência:(Lat. \textunderscore quaestiuncula\textunderscore )}
\end{itemize}
Pequena questão; discussão fútil.
\section{Questorado}
\begin{itemize}
\item {Grp. gram.:m.}
\end{itemize}
Jurisdicção de questor.
Circunscripção territorial a cargo de um questor.
\section{Questor}
\begin{itemize}
\item {Grp. gram.:m.}
\end{itemize}
\begin{itemize}
\item {Proveniência:(Lat. \textunderscore quaestor\textunderscore )}
\end{itemize}
Antigo magistrado romano, que tinha a seu cargo as finanças.
Juiz criminal entre os Romanos.
Funccionário, que superintendia com outros nas Repartições da Câmara dos Pares.
\section{Questuário}
\begin{itemize}
\item {Grp. gram.:m.  e  adj.}
\end{itemize}
\begin{itemize}
\item {Proveniência:(Lat. \textunderscore quaestuarius\textunderscore )}
\end{itemize}
O que é ambicioso, interesseiro.
\section{Questuoso}
\begin{itemize}
\item {Grp. gram.:adj.}
\end{itemize}
\begin{itemize}
\item {Proveniência:(Lat. \textunderscore quaestuosus\textunderscore )}
\end{itemize}
Que dá interesses ou vantagens.
\section{Questura}
\begin{itemize}
\item {Grp. gram.:f.}
\end{itemize}
Cargo de questor.
Commissão de três membros que superintendiam na polícia interna da Câmara dos Pares.
\section{Quetilquê}
\textunderscore m. Bras.\textunderscore  e talvez \textunderscore port. ant.\textunderscore 
O mesmo ou melhor que \textunderscore quotiliquê\textunderscore .
\section{Queto}
\begin{itemize}
\item {Grp. gram.:adj.}
\end{itemize}
\begin{itemize}
\item {Utilização:Fam.}
\end{itemize}
O mesmo que \textunderscore quieto\textunderscore .
(Cf. \textunderscore quêdo\textunderscore  e o it. \textunderscore cheto\textunderscore )
\section{Quetó}
\begin{itemize}
\item {Grp. gram.:m.}
\end{itemize}
\begin{itemize}
\item {Utilização:Prov.}
\end{itemize}
\begin{itemize}
\item {Utilização:trasm.}
\end{itemize}
O mesmo que \textunderscore bilhó\textunderscore , na accepção de criança gorducha.
\section{Quetodonte}
\begin{itemize}
\item {Grp. gram.:m.}
\end{itemize}
\begin{itemize}
\item {Proveniência:(Do gr. \textunderscore khaite\textunderscore  + \textunderscore odous\textunderscore , \textunderscore odontos\textunderscore )}
\end{itemize}
Gênero de peixes, que têm os dentes muito finos.
\section{Quetópode}
\begin{itemize}
\item {Grp. gram.:m.  e  adj.}
\end{itemize}
\begin{itemize}
\item {Proveniência:(Do gr. \textunderscore khaite\textunderscore  + \textunderscore pous\textunderscore , \textunderscore podos\textunderscore )}
\end{itemize}
Animal, que tem sedas em lugar de patas.
\section{Quetóptero}
\begin{itemize}
\item {Grp. gram.:m.}
\end{itemize}
\begin{itemize}
\item {Proveniência:(Do gr. \textunderscore khaite\textunderscore  + \textunderscore pteron\textunderscore )}
\end{itemize}
Insecto quetópode nadador, proveniente das Antilhas.
\section{Quetri}
\begin{itemize}
\item {Grp. gram.:m.}
\end{itemize}
(V.chardó)
\section{Quetulo}
\begin{itemize}
\item {Grp. gram.:m.}
\end{itemize}
\begin{itemize}
\item {Utilização:Prov.}
\end{itemize}
\begin{itemize}
\item {Utilização:minh.}
\end{itemize}
Pequena poupa, redondinha, com que nascem alguns pintaínhos.
(Cp. \textunderscore cotulo\textunderscore )
\section{Quetumbá}
\begin{itemize}
\item {Grp. gram.:m.}
\end{itemize}
Planta medicinal da ilha de San-Thomé.
\section{Quevel}
\begin{itemize}
\item {Grp. gram.:m.}
\end{itemize}
Espécie de antílope africano.
\section{Queza}
\begin{itemize}
\item {Grp. gram.:f.}
\end{itemize}
Arbusto africano, de folhas glabras, glaucas, sem estípulas, e de flôres muito miúdas em pequenos grupos axillares.
\section{Quezíla}
\begin{itemize}
\item {Grp. gram.:f.}
\end{itemize}
Repugnância; antipathia.
Inimizade ou desintelligência.
(Do quimbundo, \textunderscore kijila\textunderscore )
\section{Quezilar}
\begin{itemize}
\item {Grp. gram.:v. t.}
\end{itemize}
\begin{itemize}
\item {Grp. gram.:V. i.}
\end{itemize}
Fazer quezília a; zangar.
Importunar.
Têr quezília.
\section{Quezilento}
\begin{itemize}
\item {Grp. gram.:adj.}
\end{itemize}
Que faz quezília.
Propenso a quezilar-se.
\section{Quezília}
\begin{itemize}
\item {Grp. gram.:f.}
\end{itemize}
Repugnância; antipathia.
Inimizade ou desintelligência.
(Do quimbundo, \textunderscore kijila\textunderscore )
\section{Quiá}
\begin{itemize}
\item {Grp. gram.:m.}
\end{itemize}
\begin{itemize}
\item {Utilização:Bras}
\end{itemize}
Espécie de inambu.
\section{Quiabeiro}
\begin{itemize}
\item {Grp. gram.:m.}
\end{itemize}
Nome de várias plantas do Brasil.
Fruto dessas plantas.
Planta africana, (\textunderscore hibiscus esculentus\textunderscore ).
(Or. afr.)
\section{Quiabo}
\begin{itemize}
\item {Grp. gram.:m.}
\end{itemize}
Nome de várias plantas do Brasil.
Fruto dessas plantas.
Planta africana, (\textunderscore hibiscus esculentus\textunderscore ).
(Or. afr.)
\section{Quiáltera}
\begin{itemize}
\item {Grp. gram.:f.}
\end{itemize}
Grupo de figuras musicaes, que indica, com o respectivo algarismo, que aumenta o número das que pertencem a qualquer tempo.
(Má derivação de \textunderscore sesquiáltera\textunderscore )
\section{Quianda-muchito}
\begin{itemize}
\item {Grp. gram.:m.}
\end{itemize}
Pássaro dentirostro africano.
\section{Quiangala}
\begin{itemize}
\item {Grp. gram.:f.}
\end{itemize}
\begin{itemize}
\item {Utilização:T. da África Port}
\end{itemize}
Interrupção das chuvas nos meses de Janeiro e Fevereiro, em Cassange. Cf. Capello e Ivens, I, 342.
\section{Quianja}
\begin{itemize}
\item {Grp. gram.:f.}
\end{itemize}
Pássaro conirostro africano.
\section{Quiasma}
\begin{itemize}
\item {Grp. gram.:m.}
\end{itemize}
\begin{itemize}
\item {Utilização:Anat.}
\end{itemize}
\begin{itemize}
\item {Proveniência:(Gr. \textunderscore khiasma\textunderscore )}
\end{itemize}
Cruzamento de nervos ópticos sôbre o esfenoide.
\section{Quiassa}
\begin{itemize}
\item {Grp. gram.:f.}
\end{itemize}
Bebida alcoólica, feita de quimbombo, misturado com mel, e usada pelos Bihenos. Cf. Serpa Pinto, I, 147.
\section{Quiastolífero}
\begin{itemize}
\item {Grp. gram.:adj.}
\end{itemize}
\begin{itemize}
\item {Proveniência:(Do gr. \textunderscore khiastos\textunderscore  + lat. \textunderscore ferre\textunderscore )}
\end{itemize}
Em que há quiastólitos, (falando-se de terrenos ou piçarras).
\section{Quiastólito}
\begin{itemize}
\item {Grp. gram.:m.}
\end{itemize}
\begin{itemize}
\item {Proveniência:(Do gr. \textunderscore khiastos\textunderscore  + \textunderscore lithos\textunderscore )}
\end{itemize}
Variedade de andaluzite, que se apresenta em prismas rectangulares e quási quadrados.
\section{Quiastro}
\begin{itemize}
\item {Grp. gram.:m.}
\end{itemize}
\begin{itemize}
\item {Proveniência:(Do gr. \textunderscore khiazein\textunderscore )}
\end{itemize}
Ligadura em fórma de X, que se usava nas fracturas das pernas.
\section{Quiaz}
\begin{itemize}
\item {Grp. gram.:m.}
\end{itemize}
Antigo e pequeno pêso de Ormuz. Cf. H. Lopes de Mendonça, \textunderscore Aff. de Albuq.\textunderscore , 172.
\section{Quiba}
\begin{itemize}
\item {Grp. gram.:adj.}
\end{itemize}
\begin{itemize}
\item {Utilização:Bras}
\end{itemize}
Diz-se do animal corpulento e forte.
\section{Quibaanas}
\begin{itemize}
\item {Grp. gram.:f. pl.}
\end{itemize}
Indígenas do norte do Brasil.
\section{Quibaba}
\begin{itemize}
\item {Grp. gram.:f.}
\end{itemize}
(V.cuibaba)
\section{Quibaca}
\begin{itemize}
\item {Grp. gram.:f.}
\end{itemize}
\begin{itemize}
\item {Utilização:Bras}
\end{itemize}
O mesmo que \textunderscore tibaca\textunderscore .
\section{Quibanda}
\begin{itemize}
\item {Grp. gram.:f.}
\end{itemize}
Presente ou tributo, que as comitivas estranhas pagam, no Bié, ao soba. Cf. Serpa Pinto, I, 148.
\section{Quibandabunzi}
\begin{itemize}
\item {Grp. gram.:m.}
\end{itemize}
Ave trepadora africana.
\section{Quibandar}
\begin{itemize}
\item {Grp. gram.:v. t.}
\end{itemize}
\begin{itemize}
\item {Utilização:Bras}
\end{itemize}
Ajuntar com o quibando, para separar as alimpaduras de (arroz, café, etc).
\section{Quibando}
\begin{itemize}
\item {Grp. gram.:m.}
\end{itemize}
\begin{itemize}
\item {Utilização:Bras}
\end{itemize}
Disco de palha, tecido em zonas paralellas, e que serve para sengar.
(Talvez do quimbundo)
\section{Quibano}
\begin{itemize}
\item {Grp. gram.:m.}
\end{itemize}
\begin{itemize}
\item {Utilização:Bras. do N}
\end{itemize}
O mesmo que \textunderscore quibando\textunderscore .
\section{Quibebe}
\begin{itemize}
\item {fónica:bê}
\end{itemize}
\begin{itemize}
\item {Grp. gram.:m.}
\end{itemize}
\begin{itemize}
\item {Utilização:Bras}
\end{itemize}
Iguaria, feita de abóbora amarela.
\section{Quibôa}
\begin{itemize}
\item {Grp. gram.:f.}
\end{itemize}
Arbusto africano, talvez da fam. das gramíneas, de fôlhas glaucas, lanceoladas e alternas, e flôres miúdas, que representam um perianto amarelado.
\section{Quiboça}
\begin{itemize}
\item {Grp. gram.:f.}
\end{itemize}
Planta angolense, de fibras têxteis.
\section{Quibolo-bola}
\begin{itemize}
\item {Grp. gram.:m.}
\end{itemize}
Serpente venenosa de Angola, (\textunderscore causus rhombeatus\textunderscore ). Cf. Capello e Ivens, I, 326.
\section{Quibondo-ia-menha}
\begin{itemize}
\item {Grp. gram.:f.}
\end{itemize}
Árvore africana, da fam. das esterculláceas, (\textunderscore sterculia tragacantha\textunderscore ), que produz goma em lâminas brancas e ondeadas.
\section{Quibondo-ia-molembo}
\begin{itemize}
\item {Grp. gram.:m.}
\end{itemize}
Planta esterculiácea de Angola.
\section{Quibori}
\begin{itemize}
\item {Grp. gram.:m.}
\end{itemize}
Planta angolense, de fibras têxteis.
\section{Quibosa}
\begin{itemize}
\item {Grp. gram.:f.}
\end{itemize}
Gênero de árvores liliáceas de Angola, provavelmente o mesmo que \textunderscore quiboça\textunderscore .
\section{Quibuca}
\begin{itemize}
\item {Grp. gram.:f.}
\end{itemize}
Caravana de pretos de Angola. Cf. Capello e Ivens, I, XXXVII.
\section{Quibumbo}
\begin{itemize}
\item {Grp. gram.:m.}
\end{itemize}
\begin{itemize}
\item {Utilização:Prov.}
\end{itemize}
\begin{itemize}
\item {Utilização:Chul.}
\end{itemize}
Chapéu alto.
\section{Quiçá}
\begin{itemize}
\item {Grp. gram.:adv.}
\end{itemize}
\begin{itemize}
\item {Proveniência:(Do it. \textunderscore chi\textunderscore  + \textunderscore sa\textunderscore , de \textunderscore sapere\textunderscore , se não do lat. \textunderscore quid-sapit\textunderscore , passando \textunderscore ds\textunderscore  para \textunderscore ç\textunderscore , como o \textunderscore ds\textunderscore  de \textunderscore Gund'salvus\textunderscore  para o \textunderscore ç\textunderscore  de \textunderscore Gonçalo\textunderscore )}
\end{itemize}
O mesmo que \textunderscore talvez\textunderscore .
\section{Quiçaba}
\begin{itemize}
\item {Grp. gram.:f.}
\end{itemize}
\begin{itemize}
\item {Utilização:Bras}
\end{itemize}
Pote, talha, igaçaba.
\section{Quiçácia}
\begin{itemize}
\item {Grp. gram.:f.}
\end{itemize}
Pássaro dentirostro da África.
\section{Quicada}
\begin{itemize}
\item {Grp. gram.:f.}
\end{itemize}
\begin{itemize}
\item {Utilização:Fam.}
\end{itemize}
Pancada com quico.
\section{Quiçais}
\begin{itemize}
\item {Grp. gram.:adv.}
\end{itemize}
\begin{itemize}
\item {Utilização:Ant.}
\end{itemize}
O mesmo que \textunderscore quiçá\textunderscore . Cf. G. Vicente, \textunderscore Mofina\textunderscore ; Usque, etc.
\section{Quiçandabungi}
\begin{itemize}
\item {Grp. gram.:m.}
\end{itemize}
Pássaro dentirostro africano.
\section{Quicê}
\begin{itemize}
\item {Grp. gram.:m.}
\end{itemize}
\begin{itemize}
\item {Utilização:Bras. do N}
\end{itemize}
Ave africana. Cf. Capello e Ivens, II, 357.
\section{Quicé}
\begin{itemize}
\item {Grp. gram.:m.}
\end{itemize}
Ave africana. Cf. Capello e Ivens, II, 357.
\section{Quicé}
\begin{itemize}
\item {Grp. gram.:f.}
\end{itemize}
\begin{itemize}
\item {Utilização:Bras. do N}
\end{itemize}
Faca pequena e velha, geralmente partida ou sem ponta.
Caxerenguengue.
\section{Quicê-acica}
\begin{itemize}
\item {Grp. gram.:f.}
\end{itemize}
\begin{itemize}
\item {Utilização:Bras. do N}
\end{itemize}
O mesmo que \textunderscore quicé\textunderscore ^2.
\section{Quichaça}
\begin{itemize}
\item {Grp. gram.:f.}
\end{itemize}
\begin{itemize}
\item {Utilização:Bras}
\end{itemize}
\begin{itemize}
\item {Proveniência:(De \textunderscore cachaço\textunderscore ?)}
\end{itemize}
Teimosia.
\section{Quiché}
\begin{itemize}
\item {Grp. gram.:m.}
\end{itemize}
Língua holophrástica da América central.
\section{Quichibua}
\begin{itemize}
\item {Grp. gram.:f.}
\end{itemize}
Planta africana, da fam. das ampelídeas, muito semelhante á nossa videira.
\section{Quichiligangue}
\begin{itemize}
\item {Grp. gram.:m.}
\end{itemize}
\begin{itemize}
\item {Utilização:Bras}
\end{itemize}
Insignificância, bagatela.
\section{Quichobo}
\begin{itemize}
\item {fónica:chô}
\end{itemize}
\begin{itemize}
\item {Grp. gram.:m.}
\end{itemize}
Espécie de antílope africano. Cf. Serpa Pinto, I, 299.
\section{Quíchua}
\begin{itemize}
\item {Grp. gram.:m.}
\end{itemize}
\begin{itemize}
\item {Grp. gram.:Pl.}
\end{itemize}
Antigo idioma americano, ainda hoje falado em grande parte do Peru.
Tríbo de Índios da América, aborigenes do Peru.
\section{Quício}
\begin{itemize}
\item {Grp. gram.:m.}
\end{itemize}
O mesmo que \textunderscore gonzo\textunderscore .
(Cast. \textunderscore quicio\textunderscore )
\section{Quico}
\begin{itemize}
\item {Grp. gram.:m.}
\end{itemize}
\begin{itemize}
\item {Utilização:Fam.}
\end{itemize}
Chapéu muito pequeno e ridículo.
\section{Quiço}
\begin{itemize}
\item {Grp. gram.:m.}
\end{itemize}
O mesmo ou melhor que \textunderscore quécio\textunderscore .
(Cp. \textunderscore serviço\textunderscore  e cast. \textunderscore servicio\textunderscore  e \textunderscore quicio\textunderscore )
\section{Quicobequelababa}
\begin{itemize}
\item {Grp. gram.:f.}
\end{itemize}
Ave africana.
\section{Quicocomela}
\begin{itemize}
\item {Grp. gram.:f.}
\end{itemize}
Ave africana, (\textunderscore irrisor erythrorynchus\textunderscore ).
\section{Quiçoçoria}
\begin{itemize}
\item {Grp. gram.:f.}
\end{itemize}
Ave africana, (\textunderscore plocepasser mahali\textunderscore ).
\section{Quicole}
\begin{itemize}
\item {Grp. gram.:m.}
\end{itemize}
Arbusto africano, de caule volubilado, fôlhas inteiras, e inflorescência solitária ou em grupos de flôres hermaphroditas.
\section{Quicongo}
\begin{itemize}
\item {Grp. gram.:m.}
\end{itemize}
Árvore africana, o mesmo que \textunderscore quiseco\textunderscore .
\section{Quicóqua}
\begin{itemize}
\item {Grp. gram.:f.}
\end{itemize}
Árvore de Benguela.
\section{Quicua}
\begin{itemize}
\item {Grp. gram.:f.}
\end{itemize}
Ave palmíde africana.
\section{Quicuacula}
\begin{itemize}
\item {Grp. gram.:f.}
\end{itemize}
Pássaro dentirostro africano.
\section{Quicuambe}
\begin{itemize}
\item {Grp. gram.:m.}
\end{itemize}
Ave africana de rapina.
\section{Quicuandiata}
\begin{itemize}
\item {Grp. gram.:f.}
\end{itemize}
Pássaro dentirostro africano.
\section{Quicuanga}
\begin{itemize}
\item {Grp. gram.:f.}
\end{itemize}
Ave trepadora da África.
\section{Quicundo}
\begin{itemize}
\item {Grp. gram.:m.}
\end{itemize}
Pássaro dentirostro africano.
\section{Quicunjo}
\begin{itemize}
\item {Grp. gram.:m.}
\end{itemize}
Ave africana de rapina.
\section{Quicuta}
\begin{itemize}
\item {Grp. gram.:f.}
\end{itemize}
Planta leguminosa da África portuguesa, (\textunderscore mucuna pruriens\textunderscore , De-Candolle).
\section{Quiddidade}
\begin{itemize}
\item {fónica:cu-i}
\end{itemize}
\begin{itemize}
\item {Grp. gram.:f.}
\end{itemize}
\begin{itemize}
\item {Proveniência:(Lat. \textunderscore quidditas\textunderscore )}
\end{itemize}
A essência de uma coisa; qualidade essencial.
\section{Quiddidativo}
\begin{itemize}
\item {fónica:cu-i}
\end{itemize}
\begin{itemize}
\item {Grp. gram.:adj.}
\end{itemize}
Relativo á quiddidade.
\section{Quididade}
\begin{itemize}
\item {fónica:cu-i}
\end{itemize}
\begin{itemize}
\item {Grp. gram.:f.}
\end{itemize}
\begin{itemize}
\item {Proveniência:(Lat. \textunderscore quidditas\textunderscore )}
\end{itemize}
A essência de uma coisa; qualidade essencial.
\section{Quididativo}
\begin{itemize}
\item {fónica:cu-i}
\end{itemize}
\begin{itemize}
\item {Grp. gram.:adj.}
\end{itemize}
Relativo á quididade.
\section{Quieira}
\begin{itemize}
\item {Grp. gram.:f.}
\end{itemize}
Planta cesalpínea, de frutos comestíveis, da África central, (\textunderscore bauhinia Serpae\textunderscore ). Cf. Serpa Pinto, \textunderscore Como eu atravess. a Áfr.\textunderscore 
\section{Quiescente}
\begin{itemize}
\item {Grp. gram.:adj.}
\end{itemize}
\begin{itemize}
\item {Proveniência:(Lat. \textunderscore quiescens\textunderscore )}
\end{itemize}
Que está descansando.
\section{Quietação}
\begin{itemize}
\item {Grp. gram.:f.}
\end{itemize}
Acto ou effeito de quietar; estado de quem se acha quieto.
\section{Quietamente}
\begin{itemize}
\item {Grp. gram.:adv.}
\end{itemize}
De modo quieto.
\section{Quietar}
\begin{itemize}
\item {Grp. gram.:v. t.}
\end{itemize}
Fazer estar quieto; dar descanso a; tranquilizar.
\section{Quiete}
\begin{itemize}
\item {Grp. gram.:f.}
\end{itemize}
\begin{itemize}
\item {Utilização:Poét.}
\end{itemize}
\begin{itemize}
\item {Proveniência:(Lat. \textunderscore quies\textunderscore )}
\end{itemize}
O mesmo que \textunderscore quietação\textunderscore . Cf. \textunderscore Luz e Calor\textunderscore , 140.
\section{Quietismo}
\begin{itemize}
\item {Grp. gram.:m.}
\end{itemize}
\begin{itemize}
\item {Proveniência:(De \textunderscore quieto\textunderscore )}
\end{itemize}
Systema mýstico de alguns theólogos, baseado em que o indivíduo deve conservar-se em estado de contemplação passiva, indifferente a tudo que lhe succeda.
\section{Quietista}
\begin{itemize}
\item {Grp. gram.:m. ,  f.  e  adj.}
\end{itemize}
\begin{itemize}
\item {Proveniência:(De \textunderscore quieto\textunderscore )}
\end{itemize}
Pessôa sectária do quietismo.
\section{Quieto}
\begin{itemize}
\item {Grp. gram.:adj.}
\end{itemize}
\begin{itemize}
\item {Grp. gram.:M.}
\end{itemize}
\begin{itemize}
\item {Utilização:Bras de Minas}
\end{itemize}
\begin{itemize}
\item {Proveniência:(Lat. \textunderscore quietus\textunderscore )}
\end{itemize}
Que não trabalha nem tem cuidados.
Tranquillo.
Immóvel; plácido; sereno; pacífico.
Descanso, vida tranquilla.
\section{Quietole}
\begin{itemize}
\item {Grp. gram.:m.}
\end{itemize}
Pássaro conirostro africano.
\section{Quietude}
\begin{itemize}
\item {Grp. gram.:f.}
\end{itemize}
\begin{itemize}
\item {Proveniência:(Lat. \textunderscore quietudo\textunderscore )}
\end{itemize}
Qualidade do que é quieto.
Sossêgo; paz; tranquillidade suave.
\section{Quifacoto}
\begin{itemize}
\item {Grp. gram.:m.}
\end{itemize}
Planta africana, herbácea, de fôlhas lobadas, simples, e flôres axillares, hermaphroditas.
\section{Quifóci}
\begin{itemize}
\item {Grp. gram.:m.}
\end{itemize}
Gênero de plantas téxteis de Angola.
Talvez o mesmo que \textunderscore quifuge\textunderscore .
\section{Quifuge}
\begin{itemize}
\item {Grp. gram.:m.}
\end{itemize}
Arbusto trepador de Angola, (\textunderscore entada scandens\textunderscore , Benth.).
\section{Quifuxo}
\begin{itemize}
\item {Grp. gram.:m.}
\end{itemize}
Planta cyperácea de Angola.
Provavelmente, o mesmo que \textunderscore quifuge\textunderscore .
\section{Quigombó}
\begin{itemize}
\item {Grp. gram.:m.}
\end{itemize}
O mesmo que \textunderscore quiabo\textunderscore .
\section{Quijila}
\textunderscore f.\textunderscore  (e der.)
O mesmo que \textunderscore quezília\textunderscore , etc.
\section{Quijinga}
\begin{itemize}
\item {Grp. gram.:f.}
\end{itemize}
\begin{itemize}
\item {Utilização:T. da África Port}
\end{itemize}
O mesmo que \textunderscore quijunga\textunderscore . Cf. Capello e Ivens, II, 52.
\section{Quijunga}
\begin{itemize}
\item {Grp. gram.:m.}
\end{itemize}
\begin{itemize}
\item {Utilização:T. de Angola}
\end{itemize}
Espécie de barrete, o mesmo que \textunderscore cajinga\textunderscore .
\section{Quil}
\begin{itemize}
\item {Grp. gram.:m.}
\end{itemize}
\begin{itemize}
\item {Utilização:Ant.}
\end{itemize}
Espécie de breu da Índia.
(Do malabar \textunderscore kil\textunderscore )
\section{Quilacatembo}
\begin{itemize}
\item {Grp. gram.:m.}
\end{itemize}
Pássaro conirostro da África.
\section{Quilambalambe}
\begin{itemize}
\item {Grp. gram.:m.}
\end{itemize}
Espécie de corvo africano.
\section{Quilatação}
\begin{itemize}
\item {Grp. gram.:f.}
\end{itemize}
Acto ou effeito de quilatar.
\section{Quilatador}
\begin{itemize}
\item {Grp. gram.:m.}
\end{itemize}
O mesmo que \textunderscore aquilatador\textunderscore .
\section{Quilatar}
\begin{itemize}
\item {Grp. gram.:v. t.}
\end{itemize}
O mesmo que \textunderscore aquilatar\textunderscore .
\section{Quilate}
\begin{itemize}
\item {Grp. gram.:m.}
\end{itemize}
\begin{itemize}
\item {Utilização:Fig.}
\end{itemize}
\begin{itemize}
\item {Proveniência:(Do ár. \textunderscore quirate\textunderscore )}
\end{itemize}
A maior pureza ou perfeição do oiro e das pedras preciosas.
Pêso, equivalente á vigésima parte de uma onça.
Excellência; superioridade; perfeição; predicado.
\section{Quilateira}
\begin{itemize}
\item {Grp. gram.:f.}
\end{itemize}
\begin{itemize}
\item {Proveniência:(De \textunderscore quilate\textunderscore )}
\end{itemize}
Espécie de peneira, com que, pelo volume das pedras preciosas, se avalia o quilate dellas.
\section{Quile}
\begin{itemize}
\item {Grp. gram.:m.}
\end{itemize}
\begin{itemize}
\item {Utilização:Ant.}
\end{itemize}
Espécie de breu da Índia.
(Do malabar \textunderscore kil\textunderscore )
\section{Quilengo-lengo}
\begin{itemize}
\item {Grp. gram.:m.}
\end{itemize}
Serpente de Angola, (\textunderscore bucephalus typus\textunderscore ). Cf. Capello e Ivens, I, 326.
\section{Quilha}
\begin{itemize}
\item {Grp. gram.:f.}
\end{itemize}
\begin{itemize}
\item {Proveniência:(Do ant. méd. al. \textunderscore kiel\textunderscore )}
\end{itemize}
Peça forte de madeira, que vai da prôa á popa, na parte inferior do navio, e na qual se fixam as peças curvas, a que se pregam as tábuas do costado.
Costado do navio.
Querena.
Defeito do cavallo, que consiste em que o osso do esterno sobresai muito, em fórma de quilha.
\section{Quilha}
\begin{itemize}
\item {Grp. gram.:f.}
\end{itemize}
Peixe de Portugal.
\section{Quilhaos}
\begin{itemize}
\item {Grp. gram.:m. pl.}
\end{itemize}
Indígenas do norte do Brasil.
\section{Quilhar}
\begin{itemize}
\item {Grp. gram.:v. t.}
\end{itemize}
Pôr quilha em.
\section{Quilhar}
\begin{itemize}
\item {Grp. gram.:v. i.}
\end{itemize}
\begin{itemize}
\item {Utilização:Gír.}
\end{itemize}
\begin{itemize}
\item {Grp. gram.:V. t.}
\end{itemize}
\begin{itemize}
\item {Utilização:Gír.}
\end{itemize}
Têr cóito.
Têr cópula carnal com.
\section{Quilhar}
\begin{itemize}
\item {Grp. gram.:v. t.}
\end{itemize}
\begin{itemize}
\item {Utilização:Prov.}
\end{itemize}
\begin{itemize}
\item {Utilização:minh.}
\end{itemize}
Pregar a peça a, lograr.
(Por \textunderscore coilhar\textunderscore , contr. de \textunderscore codilhar\textunderscore ?)
\section{Quilhaus}
\begin{itemize}
\item {Grp. gram.:m. pl.}
\end{itemize}
Indígenas do norte do Brasil.
\section{Quili}
\begin{itemize}
\item {Grp. gram.:m.}
\end{itemize}
Árvore indiana, de fibras têxteis.
\section{Quilíada}
\begin{itemize}
\item {Grp. gram.:f.}
\end{itemize}
\begin{itemize}
\item {Proveniência:(Do gr. \textunderscore khilias\textunderscore )}
\end{itemize}
Um milhar.
\section{Quiliarca}
\begin{itemize}
\item {Grp. gram.:m.}
\end{itemize}
\begin{itemize}
\item {Proveniência:(Gr. \textunderscore khiliarkhos\textunderscore )}
\end{itemize}
Commandante de quiliarquia.
\section{Quiliarquia}
\begin{itemize}
\item {Grp. gram.:f.}
\end{itemize}
Formatura de 1024 homens ou duas pentacosiarquias, na phalange macedónica.
\section{Quiliare}
\begin{itemize}
\item {Grp. gram.:m.}
\end{itemize}
\begin{itemize}
\item {Proveniência:(Do gr. \textunderscore chilioi\textunderscore , e de \textunderscore are\textunderscore )}
\end{itemize}
Medida de superfície, equivalente a mil ares.
\section{Quiliasta}
\begin{itemize}
\item {Grp. gram.:m.}
\end{itemize}
\begin{itemize}
\item {Proveniência:(Do gr. \textunderscore khilioi\textunderscore , mil)}
\end{itemize}
O mesmo que \textunderscore milenário\textunderscore , sectário cristão, que afirmava que a resurreição dos santos se havia de antecipar mil annos á geral.
\section{Quilífero}
\begin{itemize}
\item {Grp. gram.:adj.}
\end{itemize}
\begin{itemize}
\item {Utilização:Physiol.}
\end{itemize}
\begin{itemize}
\item {Proveniência:(De \textunderscore chylo\textunderscore  + lat. \textunderscore ferre\textunderscore )}
\end{itemize}
Por onde passa o quilo.
\section{Quilificação}
\begin{itemize}
\item {Grp. gram.:f.}
\end{itemize}
Acto de quilificar.
\section{Quilificar}
\begin{itemize}
\item {Grp. gram.:v.}
\end{itemize}
\begin{itemize}
\item {Utilização:t. Physiol.}
\end{itemize}
\begin{itemize}
\item {Proveniência:(De \textunderscore chylo\textunderscore  + lat. \textunderscore facere\textunderscore )}
\end{itemize}
Converter em quilo.
\section{Quiliógono}
\begin{itemize}
\item {Grp. gram.:m.}
\end{itemize}
\begin{itemize}
\item {Utilização:Mathem.}
\end{itemize}
\begin{itemize}
\item {Proveniência:(Do gr. \textunderscore khilioi\textunderscore  + \textunderscore gonos\textunderscore )}
\end{itemize}
Polýgno regular de mil lados.
\section{Quilo}
\begin{itemize}
\item {Grp. gram.:m.}
\end{itemize}
(Abrev. de \textunderscore quilogramma\textunderscore , usada no commércio)
(Má derivação do gr. \textunderscore khilioi\textunderscore )
\section{Quilo}
\begin{itemize}
\item {Grp. gram.:m.}
\end{itemize}
\begin{itemize}
\item {Proveniência:(Gr. \textunderscore khulos\textunderscore )}
\end{itemize}
Parte líquida da digestão.
\section{Quilo...}
\begin{itemize}
\item {Grp. gram.:pref.}
\end{itemize}
(designativo de mil)
(Cp. \textunderscore quilo\textunderscore ^1)
\section{Quiloamba}
\begin{itemize}
\item {Grp. gram.:m.}
\end{itemize}
Pássaro conirostro da África.
\section{Quilocuenque}
\begin{itemize}
\item {Grp. gram.:m.}
\end{itemize}
Ave pernalta da África.
\section{Quilognatos}
\begin{itemize}
\item {Grp. gram.:m. pl}
\end{itemize}
\begin{itemize}
\item {Proveniência:(Do gr. \textunderscore kheilos\textunderscore  + \textunderscore gnathos\textunderscore )}
\end{itemize}
Ordem de animaes miriápodes.
\section{Quilograma}
\begin{itemize}
\item {Grp. gram.:m.}
\end{itemize}
\begin{itemize}
\item {Proveniência:(De \textunderscore quilo...\textunderscore  + \textunderscore grama\textunderscore )}
\end{itemize}
Pêso de mil gramas.
\section{Quilogramma}
\begin{itemize}
\item {Grp. gram.:m.}
\end{itemize}
\begin{itemize}
\item {Proveniência:(De \textunderscore quilo...\textunderscore  + \textunderscore gramma\textunderscore )}
\end{itemize}
Pêso de mil grammas.
\section{Quilogrâmetro}
\begin{itemize}
\item {Grp. gram.:m.}
\end{itemize}
\begin{itemize}
\item {Proveniência:(De \textunderscore quilograma\textunderscore  + gr. \textunderscore metron\textunderscore )}
\end{itemize}
Unidade, com que se avalia a fôrça das máquinas, e que corresponde á fôrça necessária para erguer um quilograma á altura de um metro, no espaço de um segundo.
\section{Quilogrâmmetro}
\begin{itemize}
\item {Grp. gram.:m.}
\end{itemize}
\begin{itemize}
\item {Proveniência:(De \textunderscore quilogramma\textunderscore  + gr. \textunderscore metron\textunderscore )}
\end{itemize}
Unidade, com que se avalia a fôrça das máquinas, e que corresponde á fôrça necessária para erguer um quilogramma á altura de um metro, no espaço de um segundo.
\section{Quilolitro}
\begin{itemize}
\item {Grp. gram.:m.}
\end{itemize}
\begin{itemize}
\item {Proveniência:(De \textunderscore quilo...\textunderscore  + \textunderscore litro\textunderscore )}
\end{itemize}
Medida de mil litros.
\section{Quilolo}
\begin{itemize}
\item {Grp. gram.:m.}
\end{itemize}
\begin{itemize}
\item {Utilização:T. da Áfr. Port}
\end{itemize}
Aquelle que vai á frente; pioneiro; deanteiro:«\textunderscore ...numerosos quilolos, vindos de muito longe, e cercando o soba, lhe dirigiam as suas súpplicas.\textunderscore »Capello e Ivens, I, 186.
(Do quimbundo)
\section{Quilologia}
\begin{itemize}
\item {Grp. gram.:f.}
\end{itemize}
\begin{itemize}
\item {Proveniência:(Do gr. \textunderscore khulos\textunderscore  + \textunderscore logos\textunderscore )}
\end{itemize}
Tratado sôbre o quilo.
\section{Quilombo}
\begin{itemize}
\item {Grp. gram.:m.}
\end{itemize}
\begin{itemize}
\item {Utilização:Bras}
\end{itemize}
Cabana no mato, onde se recolhem os negros fugitivos.
(Do quimbundo)
\section{Quilombola}
\begin{itemize}
\item {Grp. gram.:m.  e  f.}
\end{itemize}
\begin{itemize}
\item {Utilização:Bras}
\end{itemize}
Escravo ou escrava, refugiados em quilombo.
\section{Quilometragem}
\begin{itemize}
\item {Grp. gram.:f.}
\end{itemize}
Acto de quilometrar.
\section{Quilometrar}
\begin{itemize}
\item {Grp. gram.:v. t.}
\end{itemize}
Medir ou marcar por quilómetros.
\section{Quilometricamente}
\begin{itemize}
\item {Grp. gram.:adv.}
\end{itemize}
Por quilómetros.
\section{Quilométrico}
\begin{itemize}
\item {Grp. gram.:adj.}
\end{itemize}
Relativo a quilómetros.
\section{Quilómetro}
\begin{itemize}
\item {Grp. gram.:m.}
\end{itemize}
\begin{itemize}
\item {Proveniência:(De \textunderscore quilo...\textunderscore  + \textunderscore metro\textunderscore )}
\end{itemize}
Medida itinerária de mil metros.
\section{Quiloplastia}
\begin{itemize}
\item {Grp. gram.:f.}
\end{itemize}
\begin{itemize}
\item {Proveniência:(Do gr. \textunderscore kheilos\textunderscore  + \textunderscore plassein\textunderscore )}
\end{itemize}
Operação cirúrgica, com que se restaura um ou ambos os lábios.
\section{Quiloplástico}
\begin{itemize}
\item {Grp. gram.:adj.}
\end{itemize}
Relativo á \textunderscore quiloplastia\textunderscore .
\section{Quilose}
\begin{itemize}
\item {Grp. gram.:f.}
\end{itemize}
O mesmo que \textunderscore quilificação\textunderscore .
\section{Quiloso}
\begin{itemize}
\item {Grp. gram.:adj.}
\end{itemize}
Relativo ao quilo.
\section{Quiluanges}
\begin{itemize}
\item {Grp. gram.:m. pl.}
\end{itemize}
Uma das categorias, em que se divide o séquito do soba dos Jingas. Cf. Capello e Ivens, II, 52.
\section{Quiluanza}
\begin{itemize}
\item {Grp. gram.:f.}
\end{itemize}
Árvore intertropical, da fam. das leguminosas, de casca fendida e grossa, fôlhas compostas e estipuladas, e flôres miúdas, talvez unisexuaes.
\section{Quilúbio}
\begin{itemize}
\item {Grp. gram.:m.}
\end{itemize}
Ave pernalta da África.
\section{Quiluria}
\begin{itemize}
\item {Grp. gram.:f.}
\end{itemize}
\begin{itemize}
\item {Proveniência:(De \textunderscore quilo\textunderscore  + gr. \textunderscore ouron\textunderscore )}
\end{itemize}
Estado mórbido, determinado pela presença de quilo na urina.
\section{Quilúrico}
\begin{itemize}
\item {Grp. gram.:adj.}
\end{itemize}
\begin{itemize}
\item {Grp. gram.:M.}
\end{itemize}
Relativo á quiluria.
Aquele que padece quiluria.
\section{Quimalanca}
\begin{itemize}
\item {Grp. gram.:f.}
\end{itemize}
Espécie de hyena em Angola. Cf. Capello e Ivens, I, 19.
\section{Quimama}
\begin{itemize}
\item {Grp. gram.:f.}
\end{itemize}
\begin{itemize}
\item {Utilização:Bras}
\end{itemize}
Iguaria fina, de gergelim, farinha e sal.
\section{Quimanga}
\begin{itemize}
\item {Grp. gram.:f.}
\end{itemize}
\begin{itemize}
\item {Utilização:Bras. do N}
\end{itemize}
Cabaça, preparada convenientemente para se arrecadarem objectos, como em caixa ou bolsa.
\section{Quimangata}
\begin{itemize}
\item {Grp. gram.:f.}
\end{itemize}
\begin{itemize}
\item {Utilização:T. da Áfr. Port}
\end{itemize}
\textunderscore Andar de quimangata\textunderscore , andar ás costas de um indígena. (Us. em Cassange) Cf. Capello e Ivens, I, 342.
\section{Quimano}
\begin{itemize}
\item {Grp. gram.:m.}
\end{itemize}
\begin{itemize}
\item {Utilização:Bras}
\end{itemize}
Iguaria; o mesmo que \textunderscore quipoqué\textunderscore .
\section{Quimão}
\begin{itemize}
\item {Grp. gram.:m.}
\end{itemize}
\begin{itemize}
\item {Utilização:Ant.}
\end{itemize}
\begin{itemize}
\item {Proveniência:(Do jap. \textunderscore kimono\textunderscore )}
\end{itemize}
Roupão comprido, com mangas.
Vestuário feminino ao gôsto japonês.
\section{Quimbamba}
\begin{itemize}
\item {Grp. gram.:f.}
\end{itemize}
Ave africana, (\textunderscore cosmetornis vexillaris\textunderscore ).
\section{Quimbanda}
\begin{itemize}
\item {Grp. gram.:m.}
\end{itemize}
\begin{itemize}
\item {Utilização:T. de Benguela}
\end{itemize}
Adivinho ou médico indígena. Cf. Capello e Ivens, I, 23.
\section{Quimbanze}
\begin{itemize}
\item {Grp. gram.:m.}
\end{itemize}
Ave africana de rapina.
\section{Quimbar}
\begin{itemize}
\item {Grp. gram.:m.}
\end{itemize}
O mesmo que \textunderscore mambar\textunderscore . Cf. Serpa Pinto, II, 18.
\section{Quimbembe}
\begin{itemize}
\item {Grp. gram.:m.}
\end{itemize}
\begin{itemize}
\item {Utilização:Bras. do N}
\end{itemize}
Habitação pobre; cabana.
\section{Quimbembé}
\begin{itemize}
\item {Grp. gram.:m.}
\end{itemize}
\begin{itemize}
\item {Utilização:Bras}
\end{itemize}
\begin{itemize}
\item {Proveniência:(T. afr.)}
\end{itemize}
Certa bebida, preparada com milho.
\section{Quimbembeques}
\begin{itemize}
\item {Grp. gram.:m. pl.}
\end{itemize}
\begin{itemize}
\item {Utilização:Bras}
\end{itemize}
Berloques, amuletos, ou quaesquer penduricalhos, que as crianças trazem ao pescoço.
\section{Quimbete}
\begin{itemize}
\item {fónica:bê}
\end{itemize}
\begin{itemize}
\item {Grp. gram.:m.}
\end{itemize}
\begin{itemize}
\item {Utilização:Bras}
\end{itemize}
Espécie de batuque.
(Talvez t. afr.)
\section{Quimbimbe}
\begin{itemize}
\item {Grp. gram.:m.}
\end{itemize}
Ave africana, (\textunderscore fiscus Capelli\textunderscore ). Cf. Capello e Ivens, I, 326.
\section{Quimbôa}
\begin{itemize}
\item {Grp. gram.:f.}
\end{itemize}
Nome de duas plantas brasileiras, (talvez o mesmo que \textunderscore quibôa\textunderscore ).
\section{Quimbólio}
\begin{itemize}
\item {Grp. gram.:m.}
\end{itemize}
Pássaro conirostro da África.
\section{Quimbombo}
\begin{itemize}
\item {Grp. gram.:m.}
\end{itemize}
Espécie de cerveja africana, o mesmo que \textunderscore garapa\textunderscore . Cf. Serpa Pinto, I, 146.
\section{Quimbuca}
\begin{itemize}
\item {Grp. gram.:f.}
\end{itemize}
Árvore angolense, de fibras têxteis.
\section{Quimbundo}
\begin{itemize}
\item {Grp. gram.:m.}
\end{itemize}
Língua banta de Angola.
O mesmo que \textunderscore bundo\textunderscore .
\section{Quime}
\begin{itemize}
\item {Grp. gram.:m.}
\end{itemize}
Árvore medicinal da ilha de San-Thomé.--No museu colonial da \textunderscore Socied. de Geogr.\textunderscore  de Lisbôa, lê-se \textunderscore quimé\textunderscore .
\section{Quimera}
\begin{itemize}
\item {Grp. gram.:f.}
\end{itemize}
\begin{itemize}
\item {Proveniência:(Do lat. \textunderscore chimaira\textunderscore )}
\end{itemize}
Monstro fabuloso.
Fantasia; producto da imaginação.
Absurdo.
Gênero de peixes.
\section{Quimericamente}
\begin{itemize}
\item {Grp. gram.:adv.}
\end{itemize}
De modo \textunderscore quimérico\textunderscore .
\section{Quimérico}
\begin{itemize}
\item {Grp. gram.:adj.}
\end{itemize}
\begin{itemize}
\item {Proveniência:(De \textunderscore quimera\textunderscore )}
\end{itemize}
Que não existe realmente; fantástico.
Que toma a fantasia como realidade.
\section{Quimerista}
\begin{itemize}
\item {Grp. gram.:m.}
\end{itemize}
Inventor de quimeras.
\section{Quimerizar}
\begin{itemize}
\item {Grp. gram.:v. i.}
\end{itemize}
\begin{itemize}
\item {Utilização:Neol.}
\end{itemize}
\begin{itemize}
\item {Grp. gram.:V. t.}
\end{itemize}
\begin{itemize}
\item {Proveniência:(De \textunderscore quimera\textunderscore )}
\end{itemize}
Inventar quimeras.
Imaginar, supor quimericamente.
\section{Quimiatra}
\begin{itemize}
\item {Grp. gram.:m.}
\end{itemize}
\begin{itemize}
\item {Proveniência:(Do gr. \textunderscore khumia\textunderscore  + \textunderscore iatros\textunderscore )}
\end{itemize}
Médico, partidário da quimiatria.
\section{Quimiatria}
\begin{itemize}
\item {Grp. gram.:f.}
\end{itemize}
Sistema daqueles que, no fim da Idade-Média, pretendiam explicar pela Química todos os fenómenos da economia animal.
(Cp. \textunderscore quimiatra\textunderscore )
\section{Química}
\begin{itemize}
\item {Grp. gram.:f.}
\end{itemize}
\begin{itemize}
\item {Proveniência:(Gr. \textunderscore khúmia\textunderscore , de \textunderscore khumos\textunderscore , suco)}
\end{itemize}
Ciência, que estuda a natureza e propriedade dos corpos, e as leis das suas combinações e decomposições.
\section{Quimicamente}
\begin{itemize}
\item {Grp. gram.:adv.}
\end{itemize}
De modo químico.
\section{Químico}
\begin{itemize}
\item {Grp. gram.:adj.}
\end{itemize}
\begin{itemize}
\item {Grp. gram.:M.}
\end{itemize}
Relativo á \textunderscore Química\textunderscore .
Aquele que é versado em Química.
\section{Quimicoterapia}
\begin{itemize}
\item {Grp. gram.:f.}
\end{itemize}
Sistema médico, que emprega de preferência os agentes químicos.
\section{Quimificação}
\begin{itemize}
\item {Grp. gram.:f.}
\end{itemize}
Acto de quimificar.
\section{Quimificar}
\begin{itemize}
\item {Grp. gram.:v. t.}
\end{itemize}
\begin{itemize}
\item {Proveniência:(De \textunderscore quimo\textunderscore  + lat. \textunderscore facere\textunderscore )}
\end{itemize}
Converter em quimo.
\section{Quimiotaxia}
\begin{itemize}
\item {fónica:csi}
\end{itemize}
\begin{itemize}
\item {Grp. gram.:f.}
\end{itemize}
Acção atractiva ou repulsiva, exercida por certas substâncias sobre os microorganismos.
\section{Quimismo}
\begin{itemize}
\item {Grp. gram.:m.}
\end{itemize}
\begin{itemize}
\item {Proveniência:(De \textunderscore Química\textunderscore )}
\end{itemize}
Conjunto de combinações ou de composições de um organismo.
Abuso da Química.
\section{Quimista}
\begin{itemize}
\item {Grp. gram.:m.}
\end{itemize}
\begin{itemize}
\item {Proveniência:(Fr. \textunderscore chimiste\textunderscore )}
\end{itemize}
Aquele que se dedica á prática da Química.
\section{Quimitipia}
\begin{itemize}
\item {Grp. gram.:f.}
\end{itemize}
Processo de gravura química, que transforma em lâmina de alto relêvo outra, gravada em baixo relêvo, acomodando-a á impressão.
\section{Quimo}
\begin{itemize}
\item {Grp. gram.:m.}
\end{itemize}
\begin{itemize}
\item {Proveniência:(Gr. \textunderscore khumos\textunderscore , suco)}
\end{itemize}
Alimentos reduzidos a uma pasta pela digestão estomacal.
\section{Quimofila}
\begin{itemize}
\item {Grp. gram.:m.}
\end{itemize}
\begin{itemize}
\item {Proveniência:(Do gr. \textunderscore khumos\textunderscore  + \textunderscore phullon\textunderscore )}
\end{itemize}
Producto farmacêutico, contra as manifestações mórbidas da primeira dentição.
\section{Quimofobia}
\begin{itemize}
\item {Grp. gram.:f.}
\end{itemize}
Temor mórbido das tempestades.
\section{Quimonanto}
\begin{itemize}
\item {Grp. gram.:m.}
\end{itemize}
Planta calicantácea.
\section{Quimono}
\begin{itemize}
\item {Grp. gram.:m.}
\end{itemize}
(V.quimão)
\section{Quimose}
\begin{itemize}
\item {Grp. gram.:f.}
\end{itemize}
\begin{itemize}
\item {Utilização:Med.}
\end{itemize}
\begin{itemize}
\item {Proveniência:(Gr. \textunderscore khimosis\textunderscore )}
\end{itemize}
Inchação na conjuntiva.
\section{Quimosina}
\begin{itemize}
\item {Grp. gram.:f.}
\end{itemize}
Producto farmacêutico, segregado na mucosa gástrica dos mamíferos e das aves.
\section{Quimpurula}
\begin{itemize}
\item {Grp. gram.:f.}
\end{itemize}
Pássaro dentirostro da África.
\section{Quimuana-muana}
\begin{itemize}
\item {Grp. gram.:f.}
\end{itemize}
Arbusto africano herbáceo, trepador, cujas fôlhas verde-escuras são aplicadas pelos Indígenas na cura de edemas de pés e pernas.
\section{Quimuxoco}
\begin{itemize}
\item {Grp. gram.:m.}
\end{itemize}
Pássaro dentirostro da África.
\section{Quina}
\begin{itemize}
\item {Grp. gram.:f.}
\end{itemize}
\begin{itemize}
\item {Proveniência:(Lat. \textunderscore quini\textunderscore )}
\end{itemize}
Cada um dos cinco escudos, que fazem parte das armas de Portugal.
Carta de jogar, ou dado, com cinco pontos.
Pedra do dominó, com cinco pontos.
Série horizontal de cinco números, no jôgo do lôto.
\section{Quina}
\begin{itemize}
\item {Grp. gram.:f.}
\end{itemize}
\begin{itemize}
\item {Proveniência:(Do peruv. \textunderscore kinakina\textunderscore )}
\end{itemize}
Nome de várias plantas americanas, cuja casca tem propriedades antifebris.
Sulfato, extrahido dessas plantas; quinino.
\section{Quina}
\begin{itemize}
\item {Grp. gram.:f.}
\end{itemize}
O mesmo que \textunderscore esquina\textunderscore ^1.
Variedade de maçan.
\section{Quinado}
\begin{itemize}
\item {Grp. gram.:adj.}
\end{itemize}
Em que há quina^2, preparado com quina: \textunderscore vinho quinado\textunderscore .
\section{Quinado}
\begin{itemize}
\item {Grp. gram.:adj.}
\end{itemize}
\begin{itemize}
\item {Utilização:Bot.}
\end{itemize}
\begin{itemize}
\item {Proveniência:(Do lat. \textunderscore quini\textunderscore )}
\end{itemize}
Disposto em grupos de cinco, ou que fórma um grupo de cinco.
Diz-se das fôlhas, quando o pecíolo sustenta cinco foliólos.
\section{Quinal}
\begin{itemize}
\item {Grp. gram.:m.}
\end{itemize}
\begin{itemize}
\item {Utilização:Ant.}
\end{itemize}
\begin{itemize}
\item {Proveniência:(Do lat. \textunderscore quini\textunderscore )}
\end{itemize}
Medida de vinho, equivalente a 25 almudes.
Medida de vinho, equivalente a 5 almudes. Cf. S. R. Viterbo, \textunderscore Elucidário\textunderscore .
\section{Quinal}
\begin{itemize}
\item {Grp. gram.:m.}
\end{itemize}
Um dos idiomas do grupo atapasca, na América do Norte.
\section{Quinangabundo}
\begin{itemize}
\item {Grp. gram.:m.}
\end{itemize}
Pássaro dentirostro africano.
\section{Quinante}
\begin{itemize}
\item {Grp. gram.:adj.}
\end{itemize}
Que tem quinas^1 ou escudos gravados.
\section{Quinaquina}
\begin{itemize}
\item {Grp. gram.:f.}
\end{itemize}
Planta rubiácea; o mesmo que \textunderscore quina\textunderscore ^2.
\section{Quinar}
\begin{itemize}
\item {Grp. gram.:v. i.}
\end{itemize}
\begin{itemize}
\item {Proveniência:(De \textunderscore quina\textunderscore ^1)}
\end{itemize}
Ganhar no lôto, preenchendo ou cobrindo com marcas uma série de cinco números.
\section{Quinário}
\begin{itemize}
\item {Grp. gram.:adj.}
\end{itemize}
\begin{itemize}
\item {Grp. gram.:M.}
\end{itemize}
\begin{itemize}
\item {Proveniência:(Lat. \textunderscore quinarius\textunderscore )}
\end{itemize}
Que tem cinco.
Divisivel por cinco, sem deixar resto.
Relativo a cinco.
Diz-se do verso de cínco sýllabas. Cf. Th. Braga, \textunderscore Hist. da Poes. Pop.\textunderscore 
Antiga moéda romana de prata, equivalente a meio denário ou 5 asses.
\section{Quinato}
\begin{itemize}
\item {Grp. gram.:m.}
\end{itemize}
\begin{itemize}
\item {Proveniência:(De \textunderscore quina\textunderscore )}
\end{itemize}
Sal, resultante da combinação do ácido quínico com uma base.
\section{Quinau}
\begin{itemize}
\item {Grp. gram.:m.}
\end{itemize}
\begin{itemize}
\item {Utilização:T. de Turquel}
\end{itemize}
\begin{itemize}
\item {Grp. gram.:Loc.}
\end{itemize}
\begin{itemize}
\item {Utilização:Loc. de Turquel.}
\end{itemize}
Acto ou effeito de corrigir; correctivo.
Sinal, com que se marcam a um alumno os erros da lição.
O mesmo que \textunderscore tento\textunderscore ^1.
\textunderscore Dar quinau\textunderscore , corrigir com palavras; mostrar que alguêm errou.
Dar fé, têr notícia.
\section{Quincaju}
\begin{itemize}
\item {Grp. gram.:m.}
\end{itemize}
Mammífero plantígrado da América equatorial.
\section{Quincalha}
\begin{itemize}
\item {Grp. gram.:f.}
\end{itemize}
\begin{itemize}
\item {Utilização:Prov.}
\end{itemize}
\begin{itemize}
\item {Utilização:Bras}
\end{itemize}
\begin{itemize}
\item {Utilização:minh.}
\end{itemize}
\begin{itemize}
\item {Proveniência:(Fr. \textunderscore quincaille\textunderscore )}
\end{itemize}
O mesmo que \textunderscore quinquilharia\textunderscore . Cf. B. C. Rubim, \textunderscore Vocab. Bras.\textunderscore , vb. \textunderscore mascate\textunderscore .
\section{Quincálogo}
\begin{itemize}
\item {fónica:cu-in}
\end{itemize}
\begin{itemize}
\item {Grp. gram.:m.}
\end{itemize}
\begin{itemize}
\item {Proveniência:(Do lat. \textunderscore quinque\textunderscore  + gr. \textunderscore logos\textunderscore )}
\end{itemize}
Os cinco mandamentos da Igreja.
\section{Quincha}
\begin{itemize}
\item {Grp. gram.:f.}
\end{itemize}
\begin{itemize}
\item {Utilização:Bras. do S}
\end{itemize}
Tecto de palha; cobertura de palha para carros.
(Cast. \textunderscore quincha\textunderscore )
\section{Quinchar}
\begin{itemize}
\item {Grp. gram.:v. t.}
\end{itemize}
\begin{itemize}
\item {Utilização:Bras. do S}
\end{itemize}
Cobrir com quincha.
(Cast. \textunderscore quinchar\textunderscore )
\section{Quinchorro}
\begin{itemize}
\item {fónica:chô}
\end{itemize}
\begin{itemize}
\item {Grp. gram.:m.}
\end{itemize}
Pequeno quintal; cortelho. Cf. Camillo, \textunderscore Brasileira\textunderscore , 129.
(Por \textunderscore conchouso\textunderscore , do lat. \textunderscore conclausus\textunderscore ?)
\section{Quinchoso}
\begin{itemize}
\item {Grp. gram.:m.}
\end{itemize}
\begin{itemize}
\item {Utilização:Prov.}
\end{itemize}
Pequeno quintal; cortelho. Cf. Camillo, \textunderscore Brasileira\textunderscore , 129.
(Por \textunderscore conchouso\textunderscore , do lat. \textunderscore conclausus\textunderscore ?)
\section{Quincôncio}
\begin{itemize}
\item {Grp. gram.:m.}
\end{itemize}
(V.quincunce)
\section{Quincunce}
\begin{itemize}
\item {Grp. gram.:m.}
\end{itemize}
\begin{itemize}
\item {Proveniência:(Lat. \textunderscore quincunx\textunderscore )}
\end{itemize}
Plantação de árvores disposta em xadrez, sendo uma em cada canto e uma ao centro.
Grupo de cinco, formando quatro um quadrado e ficando um no centro.
\section{Quincuncial}
\begin{itemize}
\item {Grp. gram.:adj.}
\end{itemize}
\begin{itemize}
\item {Utilização:Bot.}
\end{itemize}
\begin{itemize}
\item {Proveniência:(De \textunderscore quincunce\textunderscore )}
\end{itemize}
Diz-se da perfloração, em que as peças do verticillo floral estão sobre uma espiral de duas voltas.
\section{Quincússis}
\begin{itemize}
\item {Grp. gram.:m.}
\end{itemize}
Moéda de cinco asses, entre os antigos Romanos. Cf. Castilho, \textunderscore Fastos\textunderscore , I, 387.
\section{Quinda}
\begin{itemize}
\item {Grp. gram.:f.}
\end{itemize}
\begin{itemize}
\item {Utilização:T. de Angola}
\end{itemize}
Espécie de cesto cylíndrico, sem tampa. Cf. Serpa Pinto, I, 169.
\section{Quindecágono}
\begin{itemize}
\item {fónica:cu-in}
\end{itemize}
\begin{itemize}
\item {Grp. gram.:m.}
\end{itemize}
\begin{itemize}
\item {Proveniência:(De \textunderscore quinque\textunderscore  lat. + \textunderscore decágono\textunderscore )}
\end{itemize}
Polýgono de quinze lados.
\section{Quindecemvirado}
\begin{itemize}
\item {fónica:cu-in}
\end{itemize}
\begin{itemize}
\item {Grp. gram.:m.}
\end{itemize}
\begin{itemize}
\item {Proveniência:(Lat. \textunderscore quindecimviratus\textunderscore )}
\end{itemize}
Cargo dos quindecêmviros.
\section{Quindecemviral}
\begin{itemize}
\item {fónica:cu-in}
\end{itemize}
\begin{itemize}
\item {Grp. gram.:adj.}
\end{itemize}
\begin{itemize}
\item {Proveniência:(Lat. \textunderscore quindecimviralis\textunderscore )}
\end{itemize}
Relativo aos quindecêmviros.
\section{Quindecemvirato}
\begin{itemize}
\item {fónica:cu-in}
\end{itemize}
\begin{itemize}
\item {Grp. gram.:m.}
\end{itemize}
O mesmo que \textunderscore quindecemvirado\textunderscore .
\section{Quindecêmviro}
\begin{itemize}
\item {fónica:cu-in}
\end{itemize}
\begin{itemize}
\item {Grp. gram.:m.}
\end{itemize}
\begin{itemize}
\item {Proveniência:(Lat. \textunderscore quindecimviri\textunderscore )}
\end{itemize}
Funccionário romano, encarregado principalmente da guarda dos livros sibyllinos e da celebração das festas seculares.--Talvez por infl. da fórma \textunderscore decêmviro\textunderscore , generalizou-se a fórma \textunderscore quindecêmviro\textunderscore , que não é exacta, pois que a fonte latina ordena \textunderscore quindecimviro\textunderscore , \textunderscore quindecimvirado\textunderscore , etc.
\section{Quindecenvirado}
\begin{itemize}
\item {fónica:cu-in}
\end{itemize}
\begin{itemize}
\item {Grp. gram.:m.}
\end{itemize}
\begin{itemize}
\item {Proveniência:(Lat. \textunderscore quindecimviratus\textunderscore )}
\end{itemize}
Cargo dos quindecêmviros.
\section{Quindecenviral}
\begin{itemize}
\item {fónica:cu-in}
\end{itemize}
\begin{itemize}
\item {Grp. gram.:adj.}
\end{itemize}
\begin{itemize}
\item {Proveniência:(Lat. \textunderscore quindecimviralis\textunderscore )}
\end{itemize}
Relativo aos quindecêmviros.
\section{Quindecenvirato}
\begin{itemize}
\item {fónica:cu-in}
\end{itemize}
\begin{itemize}
\item {Grp. gram.:m.}
\end{itemize}
O mesmo que \textunderscore quindecemvirado\textunderscore .
\section{Quindecênviro}
\begin{itemize}
\item {fónica:cu-in}
\end{itemize}
\begin{itemize}
\item {Grp. gram.:m.}
\end{itemize}
\begin{itemize}
\item {Proveniência:(Lat. \textunderscore quindecimviri\textunderscore )}
\end{itemize}
Funccionário romano, encarregado principalmente da guarda dos livros sibyllinos e da celebração das festas seculares.--Talvez por infl. da fórma \textunderscore decêmviro\textunderscore , generalizou-se a fórma \textunderscore quindecêmviro\textunderscore , que não é exacta, pois que a fonte latina ordena \textunderscore quindecimviro\textunderscore , \textunderscore quindecimvirado\textunderscore , etc.
\section{Quindênio}
\begin{itemize}
\item {fónica:cu-in}
\end{itemize}
\begin{itemize}
\item {Grp. gram.:m.}
\end{itemize}
\begin{itemize}
\item {Utilização:Improp.}
\end{itemize}
\begin{itemize}
\item {Proveniência:(Do lat. \textunderscore quini\textunderscore  + \textunderscore deni\textunderscore )}
\end{itemize}
Porção de quinze, quinzena.
O mesmo que \textunderscore quinquênnio\textunderscore .
Pensão, que, de quinze em quinze annos, era paga á Santa Sé por certos beneficiados ecclesiásticos, para as necessidades da Igreja.
\section{Quindim}
\begin{itemize}
\item {Grp. gram.:m.}
\end{itemize}
\begin{itemize}
\item {Utilização:Pop.}
\end{itemize}
\begin{itemize}
\item {Utilização:Bot.}
\end{itemize}
Difficuldade: \textunderscore isso tem seus quindins\textunderscore .
Meiguice.
Enfeite.
Planta leguminosa do Brasil.
\section{Quinei}
\begin{itemize}
\item {Grp. gram.:m.}
\end{itemize}
Árvore de Damão, (\textunderscore mimusops hexandra\textunderscore ).
\section{Quineira}
\begin{itemize}
\item {Grp. gram.:f.}
\end{itemize}
\begin{itemize}
\item {Proveniência:(De \textunderscore quina\textunderscore ^2)}
\end{itemize}
Árvore da quina; cinchona.
\section{Quinemetria}
\begin{itemize}
\item {Grp. gram.:f.}
\end{itemize}
\begin{itemize}
\item {Proveniência:(De \textunderscore quina\textunderscore ^2 + gr. \textunderscore metron\textunderscore )}
\end{itemize}
Avaliação da quantidade de quinino, contido na casca da quina.
\section{Quingandé}
\begin{itemize}
\item {Grp. gram.:m.}
\end{itemize}
Pássaro conirostro da África.
\section{Quingentaria}
\begin{itemize}
\item {Grp. gram.:f.}
\end{itemize}
Corpo militar de quingentários. Cf. C. Aires, \textunderscore Hist. do Exérc. Port.\textunderscore 
\section{Quingentário}
\begin{itemize}
\item {fónica:cu-in}
\end{itemize}
\begin{itemize}
\item {Grp. gram.:m.}
\end{itemize}
\begin{itemize}
\item {Proveniência:(Lat. \textunderscore quingentarius\textunderscore )}
\end{itemize}
Chefe ou capitão de 500 soldados, entre os Godos. Cf. Herculano, \textunderscore Eurico\textunderscore , c. X.
\section{Quingentésimo}
\begin{itemize}
\item {fónica:cu-in}
\end{itemize}
\begin{itemize}
\item {Grp. gram.:adj.}
\end{itemize}
\begin{itemize}
\item {Grp. gram.:M.}
\end{itemize}
\begin{itemize}
\item {Proveniência:(Lat. \textunderscore quingentesimus\textunderscore )}
\end{itemize}
Relativo a quinhentos.
Que occupa o último lugar numa série de quinhentos.
A quingentésima parte.
\section{Quingombô}
\begin{itemize}
\item {Grp. gram.:m.}
\end{itemize}
Fruto de uma das espécies de quiabo.
Planta malvácea da América e da África portuguesa, (\textunderscore hisbiscus esculentus\textunderscore , Lin.)--\textunderscore Quingombó\textunderscore  lhe chamam os diccionários; Ficalho chama-lhe \textunderscore quingombo\textunderscore ; e o botânico brasileiro Caminhoá chamou-lhe \textunderscore quingombô\textunderscore .
\section{Quingosta}
\begin{itemize}
\item {Grp. gram.:f.}
\end{itemize}
\begin{itemize}
\item {Utilização:Ant.}
\end{itemize}
(V.congosta)
\section{Quinguingu}
\begin{itemize}
\item {Grp. gram.:m.}
\end{itemize}
\begin{itemize}
\item {Utilização:Bras}
\end{itemize}
Serviço extraordinário, a que os fazendeiros obrigavam os escravos durante uma parte da noite.
(Talvez t. afr.)
\section{Quingumbe}
\begin{itemize}
\item {Grp. gram.:m.}
\end{itemize}
\begin{itemize}
\item {Proveniência:(T. lund.)}
\end{itemize}
Ave africana, pernalta, da fam. dos cultirostros.
\section{Quinguri}
\begin{itemize}
\item {Grp. gram.:m.}
\end{itemize}
\begin{itemize}
\item {Utilização:T. da África Port}
\end{itemize}
O espírito de um soba fallecido. Cf. Capello e Ivens, I, 300.
\section{Quinhame}
\begin{itemize}
\item {Grp. gram.:m.}
\end{itemize}
\begin{itemize}
\item {Utilização:Gír.}
\end{itemize}
O mesmo que \textunderscore perna\textunderscore .
(Do quimbundo \textunderscore kinama\textunderscore )
\section{Quinhames}
\begin{itemize}
\item {Grp. gram.:m.}
\end{itemize}
\begin{itemize}
\item {Utilização:Gír.}
\end{itemize}
Pé.
Sapato grosso.
(Cp. \textunderscore quinhame\textunderscore )
\section{Quinhão}
\begin{itemize}
\item {Grp. gram.:m.}
\end{itemize}
\begin{itemize}
\item {Utilização:Fig.}
\end{itemize}
\begin{itemize}
\item {Proveniência:(Lat. \textunderscore quinio\textunderscore )}
\end{itemize}
Cada uma das partes de um todo, correspondente a cada um dos indivíduos, pelos quaes se reparte êsse todo.
Porção.
Partilha.
Parte adquirida de um todo.
Reunião de trinta \textunderscore meios\textunderscore , nas salinas.
Sorte, destino.
\section{Quinhentismo}
\begin{itemize}
\item {Grp. gram.:m.}
\end{itemize}
Estilo, gôsto ou escola dos quinhentistas.
\section{Quinhentista}
\begin{itemize}
\item {Grp. gram.:adj.}
\end{itemize}
\begin{itemize}
\item {Grp. gram.:M.}
\end{itemize}
\begin{itemize}
\item {Proveniência:(De \textunderscore quinhentos\textunderscore )}
\end{itemize}
Relativo ao século, que decorre desde 1501, até 1600.
Escritor português dêsse século.
\section{Quinhentos}
\begin{itemize}
\item {Grp. gram.:adj.}
\end{itemize}
\begin{itemize}
\item {Proveniência:(Do lat. \textunderscore quingenti\textunderscore )}
\end{itemize}
Cinco vezes cem.
\section{Quinhentos-reis}
\begin{itemize}
\item {Grp. gram.:m. pl.}
\end{itemize}
Antiga moéda portuguesa de oiro, do tempo de D. Sebastião.
Moderna moéda de prata.
\section{Quinhoar}
\begin{itemize}
\item {Grp. gram.:v. t.}
\end{itemize}
O mesmo que \textunderscore aquinhoar\textunderscore ; sêr participante de; compartilhar. Cf. Castilho, \textunderscore Fastos\textunderscore , III, 298.
\section{Quinhoeiro}
\begin{itemize}
\item {Grp. gram.:m.}
\end{itemize}
Aquelle que tem ou recebe quinhão; sócio.
\section{Quíni}
\begin{itemize}
\item {Grp. gram.:m.}
\end{itemize}
\begin{itemize}
\item {Utilização:T. de Benguela}
\end{itemize}
O mesmo que \textunderscore pilão\textunderscore ^1. Cf. Capello e Ivens, I, 153.
\section{Quinica}
\begin{itemize}
\item {Grp. gram.:m.}
\end{itemize}
Uma das línguas cafreaes.
\section{Quínico}
\begin{itemize}
\item {Grp. gram.:adj.}
\end{itemize}
Diz-se de um ácido que se extrai da quinina.
\section{Quinimuras}
\begin{itemize}
\item {Grp. gram.:m. pl.}
\end{itemize}
Tríbo de Índios, que vivia perto da baía de Todos-os-Santos, no Brasil, antes da chegada dos Portugueses.
\section{Quinina}
\begin{itemize}
\item {Grp. gram.:f.}
\end{itemize}
\begin{itemize}
\item {Proveniência:(De \textunderscore quina\textunderscore ^2)}
\end{itemize}
Alcalóide vegetal, extrahido da casca da quina.
\section{Quinínico}
\begin{itemize}
\item {Grp. gram.:adj.}
\end{itemize}
Relativo á quinina.
\section{Quinino}
\begin{itemize}
\item {Grp. gram.:m.}
\end{itemize}
\begin{itemize}
\item {Proveniência:(De \textunderscore quina\textunderscore ^2)}
\end{itemize}
Sulfato de quinina, formado dêste álcali e de ácido sulfúrico.
\section{Quínio}
\begin{itemize}
\item {Grp. gram.:m.}
\end{itemize}
\begin{itemize}
\item {Proveniência:(De \textunderscore quina\textunderscore ^2)}
\end{itemize}
Quinina, antes de purificada.
\section{Quinismo}
\begin{itemize}
\item {Grp. gram.:m.}
\end{itemize}
\begin{itemize}
\item {Proveniência:(De \textunderscore quina\textunderscore ^2)}
\end{itemize}
Zumbido ou espécie de surdez, devida ao uso da quinina.
\section{Quinjuanjua}
\begin{itemize}
\item {Grp. gram.:f.}
\end{itemize}
Planta ampelídea da África Portuguesa.
\section{Quino}
\begin{itemize}
\item {Grp. gram.:m.}
\end{itemize}
O mesmo que \textunderscore lôto\textunderscore .
\section{Quinocial}
\begin{itemize}
\item {Grp. gram.:adj.}
\end{itemize}
\begin{itemize}
\item {Utilização:Ant.}
\end{itemize}
O mesmo que \textunderscore equinocial\textunderscore . Cf. \textunderscore Rot. do Mar Vermelho\textunderscore , 35.
\section{Quinólogo}
\begin{itemize}
\item {Grp. gram.:m.}
\end{itemize}
\begin{itemize}
\item {Utilização:Neol.}
\end{itemize}
\begin{itemize}
\item {Proveniência:(De \textunderscore quina\textunderscore ^2 + gr. \textunderscore logos\textunderscore )}
\end{itemize}
Chímico ou therapeuta, que se dedica ao estudo das variadas applicações da quina.
\section{Quinovato}
\begin{itemize}
\item {Grp. gram.:m.}
\end{itemize}
Sal, produzido pela combinação do ácido quinóvico com uma base.
\section{Quinóvico}
\begin{itemize}
\item {Grp. gram.:adj.}
\end{itemize}
Diz-se do ácido, extrahido de uma espécie de quina.
\section{Quinquagenário}
\begin{itemize}
\item {fónica:cu-in}
\end{itemize}
\begin{itemize}
\item {Grp. gram.:m.  e  adj.}
\end{itemize}
\begin{itemize}
\item {Proveniência:(Lat. \textunderscore quinquagenarius\textunderscore )}
\end{itemize}
Aquelle que tem cincoenta annos de idade, ou essa idade aproximadamente.
\section{Quinquagésima}
\begin{itemize}
\item {fónica:cu-in}
\end{itemize}
\begin{itemize}
\item {Grp. gram.:f.}
\end{itemize}
Espaço de cincoenta dias.
(Fem. de \textunderscore quinquagésimo\textunderscore )
\section{Quinquagésimo}
\begin{itemize}
\item {fónica:cu-in}
\end{itemize}
\begin{itemize}
\item {Grp. gram.:adj.}
\end{itemize}
\begin{itemize}
\item {Proveniência:(Lat. \textunderscore quinquagesimus\textunderscore )}
\end{itemize}
Que tem cincoenta.
Que occupa o último lugar numa série de cincoenta.
\section{Quinquátrios}
\begin{itemize}
\item {fónica:cu-in}
\end{itemize}
\begin{itemize}
\item {Grp. gram.:m. pl.}
\end{itemize}
\begin{itemize}
\item {Proveniência:(Do lat. \textunderscore quinquiatría\textunderscore )}
\end{itemize}
Antigas festas romanas que, em honra de Minerva, se celebravam no quinto dia depois dos idos. Cf. Castilho, \textunderscore Fastos\textunderscore , II, 93.
\section{Quinque...}
\begin{itemize}
\item {fónica:cu-in-cu-e}
\end{itemize}
\begin{itemize}
\item {Grp. gram.:pref.}
\end{itemize}
\begin{itemize}
\item {Proveniência:(Lat. \textunderscore quinque\textunderscore )}
\end{itemize}
(designativo de \textunderscore cinco\textunderscore )
\section{Quinqueangular}
\begin{itemize}
\item {fónica:cu-in-cu-e}
\end{itemize}
\begin{itemize}
\item {Grp. gram.:adj.}
\end{itemize}
\begin{itemize}
\item {Proveniência:(De \textunderscore quinque...\textunderscore  + \textunderscore angular\textunderscore )}
\end{itemize}
Que tem cinco ângulos.
\section{Quinquecapsular}
\begin{itemize}
\item {fónica:cu-in-cu-e}
\end{itemize}
\begin{itemize}
\item {Grp. gram.:adj.}
\end{itemize}
\begin{itemize}
\item {Utilização:Bot.}
\end{itemize}
\begin{itemize}
\item {Proveniência:(De \textunderscore quinque...\textunderscore  + \textunderscore capsular\textunderscore )}
\end{itemize}
Que tem cinco cápsulas.
\section{Quinquecellular}
\begin{itemize}
\item {fónica:cu-in-cu-e}
\end{itemize}
\begin{itemize}
\item {Grp. gram.:adj.}
\end{itemize}
\begin{itemize}
\item {Utilização:Bot.}
\end{itemize}
\begin{itemize}
\item {Proveniência:(De \textunderscore quinque...\textunderscore  + \textunderscore cellular\textunderscore )}
\end{itemize}
Que tem cinco céllulas.
\section{Quinquecelular}
\begin{itemize}
\item {fónica:cu-in-cu-e}
\end{itemize}
\begin{itemize}
\item {Grp. gram.:adj.}
\end{itemize}
\begin{itemize}
\item {Utilização:Bot.}
\end{itemize}
\begin{itemize}
\item {Proveniência:(De \textunderscore quinque...\textunderscore  + \textunderscore celular\textunderscore )}
\end{itemize}
Que tem cinco células.
\section{Quinquedentado}
\begin{itemize}
\item {fónica:cu-in-cu-e}
\end{itemize}
\begin{itemize}
\item {Grp. gram.:adj.}
\end{itemize}
\begin{itemize}
\item {Proveniência:(De \textunderscore quinque...\textunderscore  + \textunderscore dentado\textunderscore )}
\end{itemize}
Que termina em cinco dentes.
\section{Quinquefoliado}
\begin{itemize}
\item {fónica:cu-in-cu-e}
\end{itemize}
\begin{itemize}
\item {Grp. gram.:adj.}
\end{itemize}
\begin{itemize}
\item {Proveniência:(Do lat. \textunderscore quinque\textunderscore  + \textunderscore folium\textunderscore )}
\end{itemize}
Que tem cinco folíolos ou fôlhas.
\section{Quinquefólio}
\begin{itemize}
\item {fónica:cu-in-cu-e}
\end{itemize}
\begin{itemize}
\item {Grp. gram.:adj.}
\end{itemize}
O mesmo que \textunderscore quinquefoliado\textunderscore .
\section{Quinquenaes}
\begin{itemize}
\item {fónica:cu-in-cu-e}
\end{itemize}
\begin{itemize}
\item {Grp. gram.:f. pl.}
\end{itemize}
\begin{itemize}
\item {Proveniência:(Lat. \textunderscore quinquennalia\textunderscore )}
\end{itemize}
Festas, que os Romanos celebravam, de cinco em cinco anos.
\section{Quinquenal}
\begin{itemize}
\item {fónica:cu-in-cu-e}
\end{itemize}
\begin{itemize}
\item {Grp. gram.:adj.}
\end{itemize}
\begin{itemize}
\item {Grp. gram.:M.}
\end{itemize}
\begin{itemize}
\item {Proveniência:(Lat. \textunderscore quinquennalis\textunderscore )}
\end{itemize}
Que dura cinco anos.
Magistrado municipal romano, cujo cargo durava cinco anos.
\section{Quinquenalidade}
\begin{itemize}
\item {fónica:cu-in-cu-e}
\end{itemize}
\begin{itemize}
\item {Grp. gram.:f.}
\end{itemize}
\begin{itemize}
\item {Proveniência:(Lat. \textunderscore quinquennalitas\textunderscore )}
\end{itemize}
Cargo do magistrado que, nos municípios romanos, se chamava quinquenal. Cf. Herculano, \textunderscore Hist. de Port.\textunderscore , IV, 8.
\section{Quinquenalmente}
\begin{itemize}
\item {fónica:cu-in-cu-e}
\end{itemize}
\begin{itemize}
\item {Grp. gram.:adv.}
\end{itemize}
De modo quinquenal.
De cinco em cinco anos.
\section{Quinquennaes}
\begin{itemize}
\item {fónica:cu-in-cu-e}
\end{itemize}
\begin{itemize}
\item {Grp. gram.:f. pl.}
\end{itemize}
\begin{itemize}
\item {Proveniência:(Lat. \textunderscore quinquennalia\textunderscore )}
\end{itemize}
Festas, que os Romanos celebravam, de cinco em cinco annos.
\section{Quinquênio}
\begin{itemize}
\item {fónica:cu-in-cu-e}
\end{itemize}
\begin{itemize}
\item {Grp. gram.:m.}
\end{itemize}
\begin{itemize}
\item {Proveniência:(Lat. \textunderscore quinquennium\textunderscore )}
\end{itemize}
Espaço de cinco anos.
\section{Quinquennal}
\begin{itemize}
\item {fónica:cu-in-cu-e}
\end{itemize}
\begin{itemize}
\item {Grp. gram.:adj.}
\end{itemize}
\begin{itemize}
\item {Grp. gram.:M.}
\end{itemize}
\begin{itemize}
\item {Proveniência:(Lat. \textunderscore quinquennalis\textunderscore )}
\end{itemize}
Que dura cinco annos.
Magistrado municipal romano, cujo cargo durava cinco annos.
\section{Quinquennalidade}
\begin{itemize}
\item {fónica:cu-in-cu-e}
\end{itemize}
\begin{itemize}
\item {Grp. gram.:f.}
\end{itemize}
\begin{itemize}
\item {Proveniência:(Lat. \textunderscore quinquennalitas\textunderscore )}
\end{itemize}
Cargo do magistrado que, nos municípios romanos, se chamava quinquennal. Cf. Herculano, \textunderscore Hist. de Port.\textunderscore , IV, 8.
\section{Quinquennalmente}
\begin{itemize}
\item {fónica:cu-in-cu-e}
\end{itemize}
\begin{itemize}
\item {Grp. gram.:adv.}
\end{itemize}
De modo quinquennal.
De cinco em cinco annos.
\section{Quinquênnio}
\begin{itemize}
\item {fónica:cu-in-cu-e}
\end{itemize}
\begin{itemize}
\item {Grp. gram.:m.}
\end{itemize}
\begin{itemize}
\item {Proveniência:(Lat. \textunderscore quinquennium\textunderscore )}
\end{itemize}
Espaço de cinco annos.
\section{Quinquereme}
\begin{itemize}
\item {fónica:cu-in-cu-e,rê}
\end{itemize}
\begin{itemize}
\item {Grp. gram.:f.}
\end{itemize}
\begin{itemize}
\item {Proveniência:(Lat. \textunderscore quinqueremis\textunderscore )}
\end{itemize}
Embarcação com cinco ordens de remos.
\section{Quinquerreme}
\begin{itemize}
\item {fónica:cu-in-cu-e}
\end{itemize}
\begin{itemize}
\item {Grp. gram.:f.}
\end{itemize}
\begin{itemize}
\item {Proveniência:(Lat. \textunderscore quinqueremis\textunderscore )}
\end{itemize}
Embarcação com cinco ordens de remos.
\section{Quinquevalve}
\begin{itemize}
\item {fónica:cu-in-cu-e}
\end{itemize}
\begin{itemize}
\item {Grp. gram.:adj.}
\end{itemize}
\begin{itemize}
\item {Proveniência:(De \textunderscore quinque...\textunderscore  + \textunderscore valva\textunderscore )}
\end{itemize}
Que tem cinco valvas.
\section{Quinquevalvular}
\begin{itemize}
\item {fónica:cu-in-cu-e}
\end{itemize}
\begin{itemize}
\item {Grp. gram.:adj.}
\end{itemize}
\begin{itemize}
\item {Utilização:Bot.}
\end{itemize}
\begin{itemize}
\item {Proveniência:(De \textunderscore quinque...\textunderscore  + \textunderscore válvula\textunderscore )}
\end{itemize}
Que tem cinco válvulas.
\section{Quinquevirado}
\begin{itemize}
\item {fónica:cu-in-cu-e}
\end{itemize}
\begin{itemize}
\item {Grp. gram.:m.}
\end{itemize}
\begin{itemize}
\item {Proveniência:(Lat. \textunderscore quinqueviratus\textunderscore )}
\end{itemize}
Dignidade de quinquéviro.
Tribunal dos quinquéviros.
\section{Quinquevirato}
\begin{itemize}
\item {fónica:cu-in-cu-e}
\end{itemize}
\begin{itemize}
\item {Grp. gram.:m.}
\end{itemize}
O mesmo que \textunderscore quinquevirado\textunderscore .
\section{Quinquéviro}
\begin{itemize}
\item {fónica:cu-in-cu-é}
\end{itemize}
\begin{itemize}
\item {Grp. gram.:m.}
\end{itemize}
\begin{itemize}
\item {Proveniência:(Lat. \textunderscore quinquevir\textunderscore )}
\end{itemize}
Cada um dos cinco magistrados inferiores que, na republica romana, velavam pela observação dos regulamentos policiaes.
\section{Quinquíduo}
\begin{itemize}
\item {fónica:cu-in-cu-i}
\end{itemize}
\begin{itemize}
\item {Grp. gram.:m.}
\end{itemize}
\begin{itemize}
\item {Proveniência:(Do lat. \textunderscore quinque\textunderscore  + \textunderscore dies\textunderscore )}
\end{itemize}
Espaço de cinco dias:«\textunderscore ...festas de Minerva, do quinquiduo quinquátrios nomeadas\textunderscore ». Castilho, \textunderscore Fastos\textunderscore , II, 93.
\section{Quinquilharia}
\begin{itemize}
\item {Grp. gram.:f.}
\end{itemize}
\begin{itemize}
\item {Proveniência:(Do fr. \textunderscore quincaillierie\textunderscore )}
\end{itemize}
Pequenos objectos, de fórma e natureza vária, para brinquedos de criança ou para enfeites.
\section{Quinquilharias}
\begin{itemize}
\item {Grp. gram.:f. pl.}
\end{itemize}
\begin{itemize}
\item {Proveniência:(Do fr. \textunderscore quincaillierie\textunderscore )}
\end{itemize}
Pequenos objectos, de fórma e natureza vária, para brinquedos de criança ou para enfeites.
\section{Quinquilheiro}
\begin{itemize}
\item {Grp. gram.:m.}
\end{itemize}
\begin{itemize}
\item {Proveniência:(Do fr. \textunderscore quincaillier\textunderscore )}
\end{itemize}
Aquelle que vende ou fabríca quinquilharias.
\section{Quinquina}
\begin{itemize}
\item {Grp. gram.:f.}
\end{itemize}
(V.quinaquina)
\section{Quinquinados}
\begin{itemize}
\item {Grp. gram.:m. pl.}
\end{itemize}
Tríbo de índios do Brasil, em Mato-Grosso.
\section{Quinsolo}
\begin{itemize}
\item {Grp. gram.:m.}
\end{itemize}
Planta indiana.
\section{Quinta}
\begin{itemize}
\item {Grp. gram.:f.}
\end{itemize}
\begin{itemize}
\item {Utilização:Gír.}
\end{itemize}
\begin{itemize}
\item {Utilização:Prov.}
\end{itemize}
\begin{itemize}
\item {Utilização:trasm.}
\end{itemize}
\begin{itemize}
\item {Utilização:Açor}
\end{itemize}
\begin{itemize}
\item {Grp. gram.:Loc. adv. Pl.}
\end{itemize}
Propriedade rústica, com casa de habitação.
Terra de semeadura; fazenda.
Enfermaria de meretrizes.
Aggregado de casas, que pertencem a proprietários diversos, e fazem parte de uma freguesia.
Pomar de laranjeiras.
\textunderscore Estar nas suas sete quintas\textunderscore , estar como se quere, perfeitamente á vontade.
(Cast. \textunderscore quinta\textunderscore )
\section{Quinta}
\begin{itemize}
\item {Grp. gram.:f.}
\end{itemize}
\begin{itemize}
\item {Utilização:Mús.}
\end{itemize}
Conjunto de cinco cartas, no jôgo dos centos.
Intervallo de cinco notas seguidas.
(Fem. de \textunderscore quinto\textunderscore )
\section{Quintã}
\begin{itemize}
\item {Grp. gram.:adj.}
\end{itemize}
Diz-se da febre intermittente, que apparece de cinco em cinco dias.
(Cast. \textunderscore quintana\textunderscore )
\section{Quintã}
\begin{itemize}
\item {Grp. gram.:f.}
\end{itemize}
\begin{itemize}
\item {Utilização:Ant.}
\end{itemize}
\begin{itemize}
\item {Utilização:Prov.}
\end{itemize}
\begin{itemize}
\item {Utilização:Prov.}
\end{itemize}
\begin{itemize}
\item {Utilização:beir.}
\end{itemize}
O mesmo que \textunderscore quintão\textunderscore ^1.
Curral de porcos. (Colhido em Arganil)
O mesmo que \textunderscore estrumeira\textunderscore , em pátio ou em rua. (Colhido em Tondela)
\section{Quintadagila}
\begin{itemize}
\item {Grp. gram.:m.}
\end{itemize}
Reptil angolense.
\section{Quintadecimanos}
\begin{itemize}
\item {Grp. gram.:m. pl.}
\end{itemize}
\begin{itemize}
\item {Proveniência:(Lat. \textunderscore quintadecimani\textunderscore )}
\end{itemize}
Os soldados da 15.^a legião, entre os Romanos.
\section{Quintador}
\begin{itemize}
\item {Grp. gram.:m.  e  adj.}
\end{itemize}
Aquelle que quintou.
\section{Quinta-essência}
\begin{itemize}
\item {Grp. gram.:f.}
\end{itemize}
\begin{itemize}
\item {Proveniência:(De \textunderscore quinto\textunderscore  + \textunderscore essência\textunderscore )}
\end{itemize}
Extracto, levado ao último apuramento.
Requinte; o mais alto grau.
\section{Quinta-feira}
\begin{itemize}
\item {Grp. gram.:f.}
\end{itemize}
Quinto dia da semana, a contar do Domingo.
\section{Quintal}
\begin{itemize}
\item {Grp. gram.:m.}
\end{itemize}
\begin{itemize}
\item {Utilização:Prov.}
\end{itemize}
\begin{itemize}
\item {Utilização:alg.}
\end{itemize}
\begin{itemize}
\item {Proveniência:(De \textunderscore quinta\textunderscore ^1)}
\end{itemize}
Pequena quinta; pequeno terreno com jardim ou horta, junto a uma casa de habitação.
Montureira.
\section{Quintal}
\begin{itemize}
\item {Grp. gram.:m.}
\end{itemize}
\begin{itemize}
\item {Proveniência:(Do ár. \textunderscore quintar\textunderscore )}
\end{itemize}
Pêso antigo, correspondente a quatro arrobas.
\textunderscore Quintal métrico\textunderscore , cem quilogrammas.
\section{Quintalada}
\begin{itemize}
\item {Grp. gram.:f.}
\end{itemize}
\begin{itemize}
\item {Utilização:Ant.}
\end{itemize}
\begin{itemize}
\item {Proveniência:(De \textunderscore quintal\textunderscore ^2)}
\end{itemize}
Porção de pimenta, quenalguns particulares podiam carregar por sua conta, na Índia portuguesa.
Grande pêso.
Grande porção.
\section{Quintalada}
\begin{itemize}
\item {Grp. gram.:f.}
\end{itemize}
\begin{itemize}
\item {Proveniência:(De \textunderscore quintal\textunderscore ^1)}
\end{itemize}
Reunião de quintaes.
\section{Quintalão}
\begin{itemize}
\item {Grp. gram.:m.}
\end{itemize}
\begin{itemize}
\item {Utilização:Prov.}
\end{itemize}
\begin{itemize}
\item {Utilização:alg.}
\end{itemize}
\begin{itemize}
\item {Proveniência:(De \textunderscore quintal\textunderscore ^1)}
\end{itemize}
Quintal grande.
Terreno murado, em que trabalham corticeiros.
\section{Quintalejo}
\begin{itemize}
\item {Grp. gram.:m.}
\end{itemize}
Pequeno quintal^1.
\section{Quintalejo}
\begin{itemize}
\item {Grp. gram.:m.}
\end{itemize}
Duas arrobas ou meio quintal^2.
\section{Quintalório}
\begin{itemize}
\item {Grp. gram.:m.}
\end{itemize}
\begin{itemize}
\item {Utilização:Deprec.}
\end{itemize}
\begin{itemize}
\item {Proveniência:(De \textunderscore quintal\textunderscore ^2)}
\end{itemize}
Quintal grande, mas mal cuidado, ou desaproveitado.
\section{Quintan}
\begin{itemize}
\item {Grp. gram.:adj.}
\end{itemize}
Diz-se da febre intermittente, que apparece de cinco em cinco dias.
(Cast. \textunderscore quintana\textunderscore )
\section{Quintan}
\begin{itemize}
\item {Grp. gram.:f.}
\end{itemize}
\begin{itemize}
\item {Utilização:Ant.}
\end{itemize}
\begin{itemize}
\item {Utilização:Prov.}
\end{itemize}
\begin{itemize}
\item {Utilização:Prov.}
\end{itemize}
\begin{itemize}
\item {Utilização:beir.}
\end{itemize}
O mesmo que \textunderscore quintão\textunderscore ^1.
Curral de porcos. (Colhido em Arganil)
O mesmo que \textunderscore estrumeira\textunderscore , em pátio ou em rua. (Colhido em Tondela)
\section{Quintaneiro}
\begin{itemize}
\item {Grp. gram.:m.}
\end{itemize}
\begin{itemize}
\item {Utilização:Ant.}
\end{itemize}
\begin{itemize}
\item {Proveniência:(De \textunderscore quintan\textunderscore ^2)}
\end{itemize}
O mesmo que \textunderscore quinteiro\textunderscore  ou caseiro. Cf. Júl. Castilho, \textunderscore Lisb. Ant.\textunderscore 
\section{Quintanista}
\begin{itemize}
\item {Grp. gram.:m.}
\end{itemize}
\begin{itemize}
\item {Proveniência:(De \textunderscore quinto\textunderscore  + \textunderscore ano\textunderscore )}
\end{itemize}
Estudante, que frequenta o quinto anno de qualquer disciplina ou faculdade em escolas públicas.
\section{Quintannista}
\begin{itemize}
\item {Grp. gram.:m.}
\end{itemize}
\begin{itemize}
\item {Proveniência:(De \textunderscore quinto\textunderscore  + \textunderscore anno\textunderscore )}
\end{itemize}
Estudante, que frequenta o quinto anno de qualquer disciplina ou faculdade em escolas públicas.
\section{Quintano}
\begin{itemize}
\item {Grp. gram.:adj.}
\end{itemize}
\begin{itemize}
\item {Proveniência:(Lat. \textunderscore quintanus\textunderscore )}
\end{itemize}
Diz-se da febre quintan.
O mesmo que \textunderscore quinto\textunderscore , numa série.
Situado, de cinco em cinco.
\section{Quintanos}
\begin{itemize}
\item {Grp. gram.:m. pl.}
\end{itemize}
\begin{itemize}
\item {Proveniência:(Lat. \textunderscore quintani\textunderscore )}
\end{itemize}
Os soldados da quinta legião, entre os Romanos.
\section{Quintante}
\begin{itemize}
\item {Grp. gram.:m.}
\end{itemize}
\begin{itemize}
\item {Utilização:Ant.}
\end{itemize}
\begin{itemize}
\item {Proveniência:(De \textunderscore quinto\textunderscore )}
\end{itemize}
Instrumento náutico, para tomar a altura do Sol.
\section{Quintão}
\begin{itemize}
\item {Grp. gram.:m.}
\end{itemize}
\begin{itemize}
\item {Proveniência:(De \textunderscore quinta\textunderscore ^1)}
\end{itemize}
Grande quinta.
\section{Quintão}
\begin{itemize}
\item {Grp. gram.:m.}
\end{itemize}
\begin{itemize}
\item {Proveniência:(De \textunderscore quinta\textunderscore ^2)}
\end{itemize}
Antigo instrumento de cinco cordas.
\section{Quintar}
\begin{itemize}
\item {Grp. gram.:v. t.}
\end{itemize}
\begin{itemize}
\item {Grp. gram.:Loc.}
\end{itemize}
\begin{itemize}
\item {Utilização:mil.}
\end{itemize}
\begin{itemize}
\item {Proveniência:(De \textunderscore quinto\textunderscore )}
\end{itemize}
Repartir por cinco; tirar a quinta parte de um todo.
Tirar um de cinco.
\textunderscore Quintar um regimento, uma companhia\textunderscore , escolher o último homem, de cada grupo de cinco em fila, para sêr espingardeado.
\section{Quintarola}
\begin{itemize}
\item {Grp. gram.:f.}
\end{itemize}
\begin{itemize}
\item {Utilização:Fam.}
\end{itemize}
Pequena quinta. Cf. Camillo, \textunderscore Corja\textunderscore , 261 e 263.
\section{Quinta-substância}
\begin{itemize}
\item {Grp. gram.:f.}
\end{itemize}
O mesmo que \textunderscore quinta-essência\textunderscore .
\section{Quintatão}
\begin{itemize}
\item {Grp. gram.:m.}
\end{itemize}
\begin{itemize}
\item {Utilização:Mús.}
\end{itemize}
Registo de órgão, cujos tubos fazem sentir um pouco salientemente a quinta do som fundamental.
\section{Quinteira}
\begin{itemize}
\item {Grp. gram.:f.}
\end{itemize}
\begin{itemize}
\item {Utilização:Prov.}
\end{itemize}
\begin{itemize}
\item {Utilização:Prov.}
\end{itemize}
\begin{itemize}
\item {Proveniência:(De \textunderscore quinta\textunderscore ^1)}
\end{itemize}
Mulhér de quinteiro.
Mulhér, que guarda uma quinta ou que trata della.
Cortelho, quinchoso, lugar cerrado, junto de habitação, e onde os porcos fazem estrumeira.
Pátio ou recinto descoberto, geralmente estrumado, rodeado pela habitação do lavrador e por curraes, abegoarias ou outras construcções annexas.
\section{Quinteiro}
\begin{itemize}
\item {Grp. gram.:m.}
\end{itemize}
\begin{itemize}
\item {Utilização:Prov.}
\end{itemize}
\begin{itemize}
\item {Utilização:minh.}
\end{itemize}
\begin{itemize}
\item {Proveniência:(De \textunderscore quinta\textunderscore )}
\end{itemize}
Aquelle que guarda uma quinta ou trata della; caseiro; abegão.
O que vive numa quinta, para a tratar e vigiar.
Pequeno quintal ou horta murada; eido. Cf. J. Dinis, \textunderscore Pupillas\textunderscore , 6 e 86.
\section{Quintessência}
\begin{itemize}
\item {Grp. gram.:f.}
\end{itemize}
\begin{itemize}
\item {Utilização:Gal}
\end{itemize}
\begin{itemize}
\item {Proveniência:(Fr. \textunderscore quintessence\textunderscore )}
\end{itemize}
Aquillo que há de principal, de melhor, numa coisa: \textunderscore a quintessência do fanatismo\textunderscore . Cf. J. Ribeiro, \textunderscore Esthética\textunderscore .
Cp. \textunderscore quinta-essência\textunderscore .
\section{Quintessencial}
\begin{itemize}
\item {Grp. gram.:adj.}
\end{itemize}
Relativo a quintessência; principalíssimo. (Us. por C. Laet)
\section{Quinteto}
\begin{itemize}
\item {fónica:tê}
\end{itemize}
\begin{itemize}
\item {Grp. gram.:m.}
\end{itemize}
\begin{itemize}
\item {Proveniência:(De \textunderscore quinto\textunderscore )}
\end{itemize}
O mesmo que \textunderscore quintilha\textunderscore .
Composição musical para cinco instrumentos ou vozes.
Conjunto dêsses instrumentos ou vozes.
\section{Quicuala}
\begin{itemize}
\item {Grp. gram.:f.}
\end{itemize}
Espécie de corvo africano.
\section{Quicuamanga}
\begin{itemize}
\item {Grp. gram.:f.}
\end{itemize}
Espécie de corvo africano.
\section{Quintifalange}
\begin{itemize}
\item {Utilização:Anat.}
\end{itemize}
A quinta falange do pé.
\section{Quintifalangeta}
\begin{itemize}
\item {fónica:gê}
\end{itemize}
\begin{itemize}
\item {Grp. gram.:f.}
\end{itemize}
\begin{itemize}
\item {Utilização:Anat.}
\end{itemize}
A quinta falangeta do pé.
\section{Quintifalanginha}
\begin{itemize}
\item {Grp. gram.:f.}
\end{itemize}
\begin{itemize}
\item {Utilização:Anat.}
\end{itemize}
A quinta falanginha do pé.
\section{Quintil}
\begin{itemize}
\item {Grp. gram.:adj.}
\end{itemize}
\begin{itemize}
\item {Proveniência:(Lat. \textunderscore quíntilis\textunderscore )}
\end{itemize}
Diz-se do aspecto de dois planetas que distam entre si a quinta parte do Zodíaco.
\section{Quintilha}
\begin{itemize}
\item {Grp. gram.:f.}
\end{itemize}
\begin{itemize}
\item {Proveniência:(De \textunderscore quinto\textunderscore )}
\end{itemize}
Estância de cinco versos, geralmente em redondilha maior.
\section{Quintilhão}
\begin{itemize}
\item {Grp. gram.:m.}
\end{itemize}
O mesmo ou melhor que \textunderscore quintillião\textunderscore .
\section{Quintilhão}
\begin{itemize}
\item {Grp. gram.:m.}
\end{itemize}
Mil quatrilliões, segundo o systema francês; um milhão de quatrilliões, segundo o systema inglês.
\section{Quintiliano}
\begin{itemize}
\item {Grp. gram.:adj.}
\end{itemize}
\begin{itemize}
\item {Utilização:Ext.}
\end{itemize}
\begin{itemize}
\item {Proveniência:(De \textunderscore Quintiliano\textunderscore , n. p.)}
\end{itemize}
O mesmo que \textunderscore retórico\textunderscore . Cf. B. Pato, \textunderscore Cantos e Sát.\textunderscore , 221.
\section{Quintilião}
\begin{itemize}
\item {Grp. gram.:m.}
\end{itemize}
Mil quatriliões, segundo o sistema francês; um milhão de quatriliões, segundo o sistema inglês.
\section{Quintílio}
\begin{itemize}
\item {Grp. gram.:m.}
\end{itemize}
Preparado pharmacêutico de antimónio em pó.
\section{Quintillião}
\begin{itemize}
\item {Grp. gram.:m.}
\end{itemize}
Mil quatrilliões, segundo o systema francês; um milhão de quatrilliões, segundo o systema inglês.
\section{Quintimetatársico}
\begin{itemize}
\item {Grp. gram.:adj.}
\end{itemize}
\begin{itemize}
\item {Utilização:Anat.}
\end{itemize}
Diz-se do quinto osso metatársico.
\section{Quintiphalange}
\begin{itemize}
\item {Utilização:Anat.}
\end{itemize}
A quinta phalange do pé.
\section{Quintiphalangeta}
\begin{itemize}
\item {Grp. gram.:f.}
\end{itemize}
\begin{itemize}
\item {Utilização:Anat.}
\end{itemize}
A quinta phalangeta do pé.
\section{Quintiphalanginha}
\begin{itemize}
\item {Grp. gram.:f.}
\end{itemize}
\begin{itemize}
\item {Utilização:Anat.}
\end{itemize}
A quinta phalanginha do pé.
\section{Quinto}
\begin{itemize}
\item {Grp. gram.:adj.}
\end{itemize}
\begin{itemize}
\item {Grp. gram.:M.}
\end{itemize}
\begin{itemize}
\item {Utilização:T. da Bairrada}
\end{itemize}
\begin{itemize}
\item {Grp. gram.:Pl.}
\end{itemize}
\begin{itemize}
\item {Utilização:Pop.}
\end{itemize}
\begin{itemize}
\item {Proveniência:(Lat. \textunderscore quíntus\textunderscore )}
\end{itemize}
Que numa série de cinco occupa o último lugar.
Quinta parte.
Imposto, que se paga á Fazenda Nacional, e que constitue a quinta parte de valores apprehendidos.
A quinta parte de uma pipa.
Barril, que leva quatro almudes e meio, até cinco.
O inferno: \textunderscore vai para os quintos\textunderscore , vai para o demónio, foge, sume-te.
Imposto de cinco por cento, que o erário português cobrava das minas de oiro do Brasil.
\textunderscore Nau dos quintos\textunderscore , a nau que trazia para o reino êsse imposto.
\textunderscore Ir na nau dos quintos\textunderscore , ir degradado para o Brasil.
\textunderscore Andar pelos quintos\textunderscore , andar longe, ou por sítios desconhecidos.
\section{Quintoanista}
\begin{itemize}
\item {Grp. gram.:m.}
\end{itemize}
\begin{itemize}
\item {Utilização:Bras}
\end{itemize}
O mesmo que \textunderscore quintanista\textunderscore .
\section{Quintoannista}
\begin{itemize}
\item {Grp. gram.:m.}
\end{itemize}
\begin{itemize}
\item {Utilização:Bras}
\end{itemize}
O mesmo que \textunderscore quintannista\textunderscore .
\section{Quintório}
\begin{itemize}
\item {Grp. gram.:m.}
\end{itemize}
\begin{itemize}
\item {Utilização:Prov.}
\end{itemize}
\begin{itemize}
\item {Utilização:Fam.}
\end{itemize}
\begin{itemize}
\item {Proveniência:(De \textunderscore quinta\textunderscore )}
\end{itemize}
Terreno murado e cultivado.
\section{Quintupleta}
\begin{itemize}
\item {fónica:plê}
\end{itemize}
\begin{itemize}
\item {Grp. gram.:f.}
\end{itemize}
\begin{itemize}
\item {Proveniência:(De \textunderscore quíntuplo\textunderscore )}
\end{itemize}
Velocípede de duas rodas, para cinco pessôas.
\section{Quintuplicação}
\begin{itemize}
\item {Grp. gram.:f.}
\end{itemize}
Acto ou effeito de quintuplicar.
\section{Quintuplicadamente}
\begin{itemize}
\item {Grp. gram.:adv.}
\end{itemize}
Com quintuplicação.
\section{Quintuplicado}
\begin{itemize}
\item {Grp. gram.:adj.}
\end{itemize}
\begin{itemize}
\item {Proveniência:(De \textunderscore quintuplicar\textunderscore )}
\end{itemize}
Multiplicado por cinco.
Que se tornou cinco vezes maior do que era ou maior de que outro.
\section{Quintuplicador}
\begin{itemize}
\item {Grp. gram.:m.  e  adj.}
\end{itemize}
Aquelle que quintuplica.
\section{Quintuplicar}
\begin{itemize}
\item {Grp. gram.:v. t.}
\end{itemize}
\begin{itemize}
\item {Proveniência:(Lat. \textunderscore quintuplicare\textunderscore )}
\end{itemize}
Dobrar ou repetir cinco vezes; tornar cinco vezes maior.
\section{Quintuplicável}
\begin{itemize}
\item {Grp. gram.:adj.}
\end{itemize}
Que se póde quintuplicar.
\section{Quintuplinérveo}
\begin{itemize}
\item {Grp. gram.:adj.}
\end{itemize}
\begin{itemize}
\item {Utilização:Bot.}
\end{itemize}
Diz-se das fôlhas, cujas nervuras são quíntuplas.
\section{Quíntuplo}
\begin{itemize}
\item {Grp. gram.:adj.}
\end{itemize}
\begin{itemize}
\item {Grp. gram.:M.}
\end{itemize}
\begin{itemize}
\item {Proveniência:(Lat. \textunderscore quintuplus\textunderscore )}
\end{itemize}
Que é cinco vezes maior que outro.
Aquillo que é cinco vezes maior que outro objecto.
\section{Quinze}
\begin{itemize}
\item {Grp. gram.:adj.}
\end{itemize}
\begin{itemize}
\item {Grp. gram.:M.}
\end{itemize}
\begin{itemize}
\item {Proveniência:(Do lat. \textunderscore quindecim\textunderscore )}
\end{itemize}
Diz-se do número cardinal, formado de déz e mais cinco.
Décimo quinto: \textunderscore o século quinze\textunderscore .
Aquelle ou aquillo que numa série de quinze occupa o último lugar.
\section{Quinzena}
\begin{itemize}
\item {Grp. gram.:f.}
\end{itemize}
\begin{itemize}
\item {Utilização:Mús.}
\end{itemize}
\begin{itemize}
\item {Utilização:Ant.}
\end{itemize}
\begin{itemize}
\item {Utilização:Mús.}
\end{itemize}
\begin{itemize}
\item {Utilização:Bras}
\end{itemize}
\begin{itemize}
\item {Proveniência:(De \textunderscore quinze\textunderscore )}
\end{itemize}
Espaço de quinze dias.
Retribuição do trabalho de quinze dias.
Jaquetão comprido.
Intervallo de décima quinta ou dupla oitava.
Registo de órgão, cujas notas se ouvem, duas oitavas acima do diapasão.
Renda, que os lavradores pagam, de quinze em quinze dias aos senhores de engenhos.
\section{Quinzenal}
\begin{itemize}
\item {Grp. gram.:adj.}
\end{itemize}
Relativo a quinzena; que apparece ou se faz ou se publíca, de quinze em quinze dias.
\section{Quinzenalmente}
\begin{itemize}
\item {Grp. gram.:adv.}
\end{itemize}
De modo quinzenal; de quinze em quinze dias.
\section{Quinzenário}
\begin{itemize}
\item {Grp. gram.:m.}
\end{itemize}
Periódico quinzenal.
\section{Quinzol}
\begin{itemize}
\item {Grp. gram.:m.}
\end{itemize}
Arvore indiana, (\textunderscore pentaptera paniculata\textunderscore ).
\section{Quinzongo}
\begin{itemize}
\item {Grp. gram.:m.}
\end{itemize}
Arbusto africano, da fam. das leguminosas, de caule lenhoso e fôlhas alternas.
\section{Quinzunguila}
\begin{itemize}
\item {Grp. gram.:f.}
\end{itemize}
Arbusto africano, da fam. das leguminosas, de fôlhas alternas, estipuladas, e grandes flôres papilionáceas.
\section{Quioco}
\begin{itemize}
\item {Grp. gram.:m.}
\end{itemize}
\begin{itemize}
\item {Grp. gram.:Pl.}
\end{itemize}
Uma das línguas da África occidental.
Povo das margens do Zaire.
\section{Quiosque}
\begin{itemize}
\item {Grp. gram.:m.}
\end{itemize}
\begin{itemize}
\item {Utilização:Gír.}
\end{itemize}
Pequena construcção de madeira, espécie de pavilhão, situado em praças, jardins, etc., e que serve habitualmente para a venda de tabaco, jornaes, bebidas, etc.
Ânus.
(Do turco \textunderscore kosk\textunderscore , por intermédio do fr. \textunderscore kiosque\textunderscore )
\section{Quiovas}
\begin{itemize}
\item {Grp. gram.:m. pl.}
\end{itemize}
Tríbo de Índios da América do Norte.
\section{Quipele}
\begin{itemize}
\item {Grp. gram.:m.}
\end{itemize}
Pássaro dentirostro africano.
\section{Quipembe}
\begin{itemize}
\item {Grp. gram.:m.}
\end{itemize}
Ave africana, (\textunderscore mirafra africana\textunderscore ).
\section{Quipoqué}
\begin{itemize}
\item {Grp. gram.:m.}
\end{itemize}
\begin{itemize}
\item {Utilização:Bras}
\end{itemize}
Iguaria, de feijão partido e cozinhado com vários temperos.
\section{Quipos}
\begin{itemize}
\item {Grp. gram.:m. pl.}
\end{itemize}
Cordões nodosos, usados pelos Peruanos no tempo da monarchia dos Incas, e que formavam um méthodo mnemónico, fundado nas côres e ordem dos cordões, número e disposição dos nós, etc.
(Cast. \textunderscore quipos\textunderscore )
\section{Quipròquó}
\begin{itemize}
\item {fónica:cu-i}
\end{itemize}
\begin{itemize}
\item {Grp. gram.:m.}
\end{itemize}
\begin{itemize}
\item {Proveniência:(Do lat. \textunderscore qui\textunderscore  + \textunderscore pro\textunderscore  + \textunderscore quo\textunderscore , ablat. de \textunderscore qui\textunderscore )}
\end{itemize}
Acto de confundir uma coisa com outra.
Equívoco.
Facécia, resultante de um equívoco.
\section{Quipuculo}
\begin{itemize}
\item {Grp. gram.:m.}
\end{itemize}
Árvore angolense de Cazengo.
\section{Quipululo}
\begin{itemize}
\item {Grp. gram.:m.}
\end{itemize}
Árvore de Moçambique.
\section{Quipúndi}
\begin{itemize}
\item {Grp. gram.:m.}
\end{itemize}
\begin{itemize}
\item {Utilização:T. de Benguela}
\end{itemize}
O mesmo que \textunderscore travesseiro\textunderscore . Cf. Capello e Ivens, I, 153.
\section{Quipungulo}
\begin{itemize}
\item {Grp. gram.:m.}
\end{itemize}
\begin{itemize}
\item {Proveniência:(T. lund.)}
\end{itemize}
Espécie de coruja, que vive nas florestas africanas.
\section{Quiquala}
\begin{itemize}
\item {Grp. gram.:f.}
\end{itemize}
Espécie de corvo africano.
\section{Quiquamanga}
\begin{itemize}
\item {Grp. gram.:f.}
\end{itemize}
Espécie de corvo africano.
\section{Quiquanga}
\begin{itemize}
\item {Grp. gram.:f.}
\end{itemize}
\begin{itemize}
\item {Utilização:T. de Angola}
\end{itemize}
Pasta comestível de mandioca, feita no pilão. Cf. Capello e Ivens, I, 332.
\section{Quiquecuria}
\begin{itemize}
\item {Grp. gram.:f.}
\end{itemize}
Ave angolense, de canto estridente.
\section{Quiqueriqui}
\begin{itemize}
\item {Grp. gram.:m.}
\end{itemize}
Voz imitativa do canto do gallo, \textunderscore ou\textunderscore , antes, do frango.
Coisa ou pessôa insignificante.
Bagatela; càcaracá; quotiliquê.
\section{Quiqui}
\begin{itemize}
\item {Grp. gram.:m.}
\end{itemize}
Marta do Brasil.
\section{Quiquiqui}
\begin{itemize}
\item {Grp. gram.:m.}
\end{itemize}
\begin{itemize}
\item {Utilização:Bras. do N}
\end{itemize}
Indivíduo gago.
Tatibitate.
\section{Quirabi}
\begin{itemize}
\item {Grp. gram.:m.}
\end{itemize}
Árvore angolense de Caconda.
\section{Quiragra}
\begin{itemize}
\item {Grp. gram.:f.}
\end{itemize}
\begin{itemize}
\item {Utilização:Med.}
\end{itemize}
\begin{itemize}
\item {Proveniência:(Do gr. \textunderscore kheir\textunderscore  + \textunderscore agra\textunderscore )}
\end{itemize}
Doença de gota nas mãos.
\section{Quirana}
\begin{itemize}
\item {Grp. gram.:f.}
\end{itemize}
\begin{itemize}
\item {Utilização:Bras}
\end{itemize}
\begin{itemize}
\item {Proveniência:(T. tupi)}
\end{itemize}
Espécie de grânulo, que se fórma no cabello da gente que usa pomadas e lava a cabeça em água fria.
Lêndea.
\section{Quirana}
\begin{itemize}
\item {Grp. gram.:f.}
\end{itemize}
\begin{itemize}
\item {Utilização:T. da África Port}
\end{itemize}
Oito jardas de qualquer fazenda.
\section{Quiranto}
\begin{itemize}
\item {Grp. gram.:m.}
\end{itemize}
\begin{itemize}
\item {Proveniência:(Do gr. \textunderscore kheir\textunderscore , mão, e \textunderscore anthos\textunderscore , flôr)}
\end{itemize}
Gênero de plantas crucíferas.
\section{Quirat}
\begin{itemize}
\item {Grp. gram.:m.}
\end{itemize}
\begin{itemize}
\item {Utilização:Des.}
\end{itemize}
Semente de alfarroba, que foi usada por ourives e boticários, como pêso correspondente a seis grãos de trigo. Cf. Sousa, \textunderscore Vest. da Ling. Ar.\textunderscore 
(Cf. \textunderscore quilate\textunderscore )
\section{Quirate}
\begin{itemize}
\item {Grp. gram.:m.}
\end{itemize}
\begin{itemize}
\item {Utilização:Des.}
\end{itemize}
Semente de alfarroba, que foi usada por ourives e boticários, como pêso correspondente a seis grãos de trigo. Cf. Sousa, \textunderscore Vest. da Ling. Ar.\textunderscore 
(Cf. \textunderscore quilate\textunderscore )
\section{Quirato}
\begin{itemize}
\item {Grp. gram.:m.}
\end{itemize}
\begin{itemize}
\item {Utilização:Bras}
\end{itemize}
Árvore, o mesmo que \textunderscore fucamena\textunderscore .
\section{Quirera}
\begin{itemize}
\item {Grp. gram.:f.}
\end{itemize}
\begin{itemize}
\item {Utilização:Bras}
\end{itemize}
A parte mais grossa de qualquer substância pulverizada, e que não passa pelas malhas ou orifícios da peneira.
(Corr. do tupi \textunderscore curuera\textunderscore )
\section{Quirguiz}
\begin{itemize}
\item {Grp. gram.:m.}
\end{itemize}
\begin{itemize}
\item {Proveniência:(De \textunderscore Quirguiz\textunderscore , n. p.)}
\end{itemize}
Língua turco-tártara.
\section{Quiri}
\begin{itemize}
\item {Grp. gram.:m.}
\end{itemize}
\begin{itemize}
\item {Utilização:Bras. do N}
\end{itemize}
Árvore leguminosa do Brasil.
Nome de uma palmeira medicinal.
Bengala grossa; cacete.
\section{Quiriba}
\begin{itemize}
\item {Grp. gram.:f.}
\end{itemize}
\begin{itemize}
\item {Utilização:Bras. do N}
\end{itemize}
Árvore, cuja casca se emprega em curtumes.
\section{Quiriri}
\begin{itemize}
\item {Grp. gram.:m.}
\end{itemize}
\begin{itemize}
\item {Utilização:Bras}
\end{itemize}
\begin{itemize}
\item {Proveniência:(T. tupi)}
\end{itemize}
Silêncio nocturno; calada da noite.
\section{Quirita}
\begin{itemize}
\item {Grp. gram.:f.}
\end{itemize}
\begin{itemize}
\item {Proveniência:(Do gr. \textunderscore kheir\textunderscore )}
\end{itemize}
Estalactite em fórma de mão.
\section{Quiritário}
\begin{itemize}
\item {Grp. gram.:adj.}
\end{itemize}
Relativo aos quirites. Cf. C. Lobo, \textunderscore Sát. de Juv.\textunderscore , I, 16; Latino, \textunderscore Camões\textunderscore , 9.
\section{Quirites}
\begin{itemize}
\item {Grp. gram.:m. pl.}
\end{itemize}
\begin{itemize}
\item {Proveniência:(Lat. \textunderscore quirites\textunderscore )}
\end{itemize}
Título, que accresceu ao dos Romanos, quando êstes se fundiram com os Sabinos.
Cidadãos Romanos.
\section{Quirogimnasta}
\begin{itemize}
\item {Grp. gram.:m.}
\end{itemize}
O mesmo que \textunderscore quiroplasto\textunderscore .
\section{Quirografário}
\begin{itemize}
\item {Grp. gram.:adj.}
\end{itemize}
\begin{itemize}
\item {Proveniência:(Lat. \textunderscore chirographarius\textunderscore )}
\end{itemize}
Relativo a documentos particulares, não autenticados.
\section{Quirógrafo}
\begin{itemize}
\item {Grp. gram.:m.}
\end{itemize}
\begin{itemize}
\item {Proveniência:(Lat. \textunderscore chirographum\textunderscore )}
\end{itemize}
O mesmo que \textunderscore autógrafo\textunderscore .
Diploma.
Breve pontifício, não publicado.
\section{Quirologia}
\begin{itemize}
\item {Grp. gram.:f.}
\end{itemize}
O mesmo que \textunderscore dactilologia\textunderscore .
\section{Quirológico}
\begin{itemize}
\item {Grp. gram.:adj.}
\end{itemize}
Relativo á quirologia.
\section{Quiromancia}
\begin{itemize}
\item {Grp. gram.:f.}
\end{itemize}
\begin{itemize}
\item {Proveniência:(Do gr. \textunderscore kheir\textunderscore  + \textunderscore manteia\textunderscore )}
\end{itemize}
Sistema de adivinhação, pela inspecção das linhas da palma da mão.
\section{Quiromante}
\begin{itemize}
\item {Grp. gram.:m.}
\end{itemize}
Aquele que pratíca a quiromancia.
\section{Quiromântico}
\begin{itemize}
\item {Grp. gram.:adj.}
\end{itemize}
Relativo á quiromancia.
\section{Quironecto}
\begin{itemize}
\item {Grp. gram.:m.}
\end{itemize}
\begin{itemize}
\item {Proveniência:(Do gr. \textunderscore kheir\textunderscore  + \textunderscore nektes\textunderscore )}
\end{itemize}
Mamífero aquático, do gênero das sarigueias.
\section{Quironomia}
\begin{itemize}
\item {Grp. gram.:f.}
\end{itemize}
Arte de apropriar os gestos ao discurso.
(Cp. \textunderscore quirónomo\textunderscore )
\section{Quironómico}
\begin{itemize}
\item {Grp. gram.:adj.}
\end{itemize}
Relativo á quironomia.
\section{Quirónomo}
\begin{itemize}
\item {Grp. gram.:m.}
\end{itemize}
\begin{itemize}
\item {Proveniência:(Do gr. \textunderscore kheir\textunderscore  + \textunderscore nomos\textunderscore )}
\end{itemize}
Aquele que pratíca ou ensina a quironomia.
\section{Quiroplasto}
\begin{itemize}
\item {Grp. gram.:m.}
\end{itemize}
\begin{itemize}
\item {Proveniência:(Do gr. \textunderscore kheir\textunderscore  + \textunderscore plassein\textunderscore )}
\end{itemize}
Aparelho, para facilitar o estudo do piano, adaptando-se ao teclado e guiando o movimento dos dedos.
\section{Quiropótamo}
\begin{itemize}
\item {Grp. gram.:m.}
\end{itemize}
Animal fóssil, aquático.
\section{Quirópteros}
\begin{itemize}
\item {Grp. gram.:m. pl.}
\end{itemize}
\begin{itemize}
\item {Proveniência:(Do gr. \textunderscore kheir\textunderscore , mão, e \textunderscore pteron\textunderscore , asa)}
\end{itemize}
Ordem de mamíferos, que têm por tipo o morcego.
\section{Quiroscopia}
\begin{itemize}
\item {Grp. gram.:f.}
\end{itemize}
\begin{itemize}
\item {Proveniência:(Do gr. \textunderscore kheir\textunderscore  + \textunderscore skopein\textunderscore )}
\end{itemize}
O mesmo que \textunderscore quiromancia\textunderscore .
\section{Quirotecas}
\begin{itemize}
\item {Grp. gram.:f. pl.}
\end{itemize}
\begin{itemize}
\item {Proveniência:(Do gr. \textunderscore kheir\textunderscore  + \textunderscore theke\textunderscore )}
\end{itemize}
Luvas, que os Bispos ou certos Abades usavam em certas solenidades.
\section{Quirotonia}
\begin{itemize}
\item {Grp. gram.:f.}
\end{itemize}
\begin{itemize}
\item {Proveniência:(Do gr. \textunderscore kheir\textunderscore  + \textunderscore teinein\textunderscore )}
\end{itemize}
Imposição das mãos.
Entre os Gregos, acto de votar, levantando a mão.
\section{Quirurgia}
\textunderscore f.\textunderscore  (e der.)
O mesmo que \textunderscore cirurgia\textunderscore , etc.
\section{Quirúvia}
\begin{itemize}
\item {Grp. gram.:f.}
\end{itemize}
\begin{itemize}
\item {Utilização:Ant.}
\end{itemize}
Planta, o mesmo que \textunderscore bisnaga\textunderscore . Cf. \textunderscore Desengano da Med.\textunderscore , 193.
\section{Quisafu}
\begin{itemize}
\item {Grp. gram.:m.}
\end{itemize}
Arvoreta, o mesmo que \textunderscore diteque\textunderscore .
\section{Quisanana}
\begin{itemize}
\item {Grp. gram.:f.}
\end{itemize}
Erva medicinal e combustível de Angola, (\textunderscore corchorus tridens\textunderscore , Lin.).
\section{Quiseco}
\begin{itemize}
\item {Grp. gram.:m.}
\end{itemize}
Árvore africana, de propriedades medicinaes.
\section{Quiséqua}
\begin{itemize}
\item {Grp. gram.:f.}
\end{itemize}
O mesmo que \textunderscore quiseco\textunderscore .
\section{Quisole}
\begin{itemize}
\item {Grp. gram.:m.}
\end{itemize}
Árvore vulgar nas florestas africanas, pertencente á fam. das artocárpeas.
\section{Quissamas}
\begin{itemize}
\item {Grp. gram.:m. pl.}
\end{itemize}
Tríbo independente, entre o Guanza, o Longa e o mar.
\section{Quissângua}
\begin{itemize}
\item {Grp. gram.:f.}
\end{itemize}
Bebida refrigerante, usada pelos Bihenos, e feita de uma decocção da raiz do imbúndi, addicionando-se fuba fervida. Cf. Serpa Pinto, I, 147.
\section{Quissanja}
\begin{itemize}
\item {Grp. gram.:f.}
\end{itemize}
Monótono instrumento dos Negros de Benguela.
\section{Quissanje}
\begin{itemize}
\item {Grp. gram.:m.}
\end{itemize}
O mesmo que \textunderscore quissanja\textunderscore . Cf. \textunderscore Diccion. Mus.\textunderscore 
\section{Quissemo}
\begin{itemize}
\item {Grp. gram.:m.}
\end{itemize}
Animal africano. Cf. Capello e Ivens, \textunderscore De Angola\textunderscore , I, 411.
\section{Quissengo}
\begin{itemize}
\item {Grp. gram.:m.}
\end{itemize}
Ave africana.
\section{Quissocola-lôa}
\begin{itemize}
\item {Grp. gram.:m.}
\end{itemize}
Pássaro insectívoro da África Portuguesa, (\textunderscore turdus strepitans\textunderscore , Smith?).
\section{Quissonda}
\begin{itemize}
\item {Grp. gram.:m.}
\end{itemize}
Ave africana, (\textunderscore turdus lybonianus\textunderscore , Smith). Cf. Capello e Ivens, II, 357.
\section{Quissonde}
\begin{itemize}
\item {Grp. gram.:m.}
\end{itemize}
Formiga venenosa de Angola. Cf. Serpa Pinto, I, 226.
\section{Quissongo}
\begin{itemize}
\item {Grp. gram.:m.}
\end{itemize}
\begin{itemize}
\item {Utilização:T. da África Port}
\end{itemize}
Chefe ou maioral de carregadores. (Us. desde a costa occidental até Caquingue) Cf. Serpa Pinto, I, 140.
\section{Quissunge}
\begin{itemize}
\item {Grp. gram.:m.}
\end{itemize}
Festa selvagem entre os Bihenos, para a qual são sacrificadas cinco víctimas humanas, cuja carne é cozida como a de um boi, fazendo-se della ceia pública.
\section{Quisto}
\begin{itemize}
\item {Grp. gram.:adj.}
\end{itemize}
\begin{itemize}
\item {Proveniência:(Do lat. \textunderscore quaesitus\textunderscore )}
\end{itemize}
Querido; desejado; amado:«\textunderscore ...as pessoas geraes são bem quistas.\textunderscore »\textunderscore Eufrosina\textunderscore , 106.
\section{Quisuaíli}
\begin{itemize}
\item {Grp. gram.:m.}
\end{itemize}
Idioma, muito generalizado na costa oriental da África.
O mesmo que \textunderscore zanzibar\textunderscore ^1.
\section{Quita}
\begin{itemize}
\item {Grp. gram.:f.}
\end{itemize}
O mesmo que \textunderscore quitação\textunderscore .
\section{Quitaca}
\begin{itemize}
\item {Grp. gram.:f.}
\end{itemize}
Pau, com que, friccionado, os naturaes da Guiné produzem lume. Cf. Capello e Ivens, I, 156.
\section{Quitação}
\begin{itemize}
\item {Grp. gram.:f.}
\end{itemize}
Acto ou effeito de quitar.
\section{Quitador}
\begin{itemize}
\item {Grp. gram.:m.  e  adj.}
\end{itemize}
O que quita.
\section{Quitambuera}
\begin{itemize}
\item {Grp. gram.:f.  e  adj.}
\end{itemize}
\begin{itemize}
\item {Utilização:Bras. do Rio}
\end{itemize}
O mesmo que \textunderscore catimpuera\textunderscore .
\section{Quitamento}
\begin{itemize}
\item {Grp. gram.:m.}
\end{itemize}
\begin{itemize}
\item {Utilização:Ant.}
\end{itemize}
O mesmo que \textunderscore quitação\textunderscore .
Desquite, divórcio.
\section{Quitamerenda}
\begin{itemize}
\item {Grp. gram.:f.}
\end{itemize}
O mesmo que \textunderscore quitamerendas\textunderscore .
\section{Quitamerendas}
\begin{itemize}
\item {Grp. gram.:f.}
\end{itemize}
Planta liliácea, (\textunderscore merendera bulbocodium\textunderscore , Ram.), que floresce em fins de Setembro, quando a merenda já se dispensa, e que também é conhecida por \textunderscore escusa-merenda\textunderscore .
\section{Quitança}
\begin{itemize}
\item {Grp. gram.:f.}
\end{itemize}
O mesmo que \textunderscore quitação\textunderscore .
\section{Quitanda}
\begin{itemize}
\item {Grp. gram.:f.}
\end{itemize}
\begin{itemize}
\item {Utilização:Bras}
\end{itemize}
\begin{itemize}
\item {Utilização:Pop.}
\end{itemize}
\begin{itemize}
\item {Utilização:Bras. do N}
\end{itemize}
\begin{itemize}
\item {Utilização:Bras. do Rio}
\end{itemize}
\begin{itemize}
\item {Utilização:Bras. de Minas}
\end{itemize}
Lugar, onde se faz commércio.
Loja de negócio.
Tabuleiro, em que o vendedor ambulante leva as suas mercancias.
Quinquilharias, acervo de miudezas várias.
Estabelecimento, onde se vende prata.
Hortaliça, legumes.
Pastelaria caseira.
(Quimbundo \textunderscore kitanda\textunderscore , feira)
\section{Quitandar}
\begin{itemize}
\item {Grp. gram.:v. i.}
\end{itemize}
\begin{itemize}
\item {Utilização:Bras}
\end{itemize}
\begin{itemize}
\item {Proveniência:(De \textunderscore quitanda\textunderscore )}
\end{itemize}
Exercer a profissão de quitandeiro.
\section{Quitande}
\begin{itemize}
\item {Grp. gram.:m.}
\end{itemize}
\begin{itemize}
\item {Utilização:Bras}
\end{itemize}
Feijão miúdo e verde que, extrahindo-se-lhe a pellícula á unha, serve para sôpas e mais iguarias.
\section{Quitandeira}
\begin{itemize}
\item {Grp. gram.:f.}
\end{itemize}
\begin{itemize}
\item {Utilização:Bras}
\end{itemize}
\begin{itemize}
\item {Proveniência:(De \textunderscore quitandeiro\textunderscore )}
\end{itemize}
Regateira; mulhér sem educação.
\section{Quitandeiro}
\begin{itemize}
\item {Grp. gram.:m.}
\end{itemize}
\begin{itemize}
\item {Utilização:Bras}
\end{itemize}
\begin{itemize}
\item {Proveniência:(De \textunderscore quitanda\textunderscore )}
\end{itemize}
Dono de quitanda.
Revendedor de frutas, hortaliças, aves, peixe, etc.
\section{Quitanga}
\begin{itemize}
\item {Grp. gram.:f.}
\end{itemize}
\begin{itemize}
\item {Utilização:Pop.}
\end{itemize}
O mesmo que \textunderscore quitanda\textunderscore .
\section{Quitar}
\begin{itemize}
\item {Grp. gram.:v. t.}
\end{itemize}
\begin{itemize}
\item {Grp. gram.:V. i.}
\end{itemize}
Tornar quite; desobrigar.
Evitar; impedir.
Tirar.
Perder.
Deixar.
Divorciar-se de.
Sêr dispensado de fazer alguma coisa.
Não têr necessidade de praticar um acto: \textunderscore você quita de me maçar\textunderscore .
(Cast. \textunderscore quitar\textunderscore )
\section{Quitasol}
\begin{itemize}
\item {fónica:sol}
\end{itemize}
\begin{itemize}
\item {Grp. gram.:m.}
\end{itemize}
\begin{itemize}
\item {Utilização:Des.}
\end{itemize}
\begin{itemize}
\item {Proveniência:(De \textunderscore quitar\textunderscore  + \textunderscore sol\textunderscore )}
\end{itemize}
O mesmo que \textunderscore guarda-sol\textunderscore .
\section{Quitassol}
\begin{itemize}
\item {Grp. gram.:m.}
\end{itemize}
\begin{itemize}
\item {Utilização:Des.}
\end{itemize}
\begin{itemize}
\item {Proveniência:(De \textunderscore quitar\textunderscore  + \textunderscore sol\textunderscore )}
\end{itemize}
O mesmo que \textunderscore guarda-sol\textunderscore .
\section{Quite}
\begin{itemize}
\item {Grp. gram.:adj.}
\end{itemize}
\begin{itemize}
\item {Grp. gram.:M.}
\end{itemize}
\begin{itemize}
\item {Utilização:Taur.}
\end{itemize}
\begin{itemize}
\item {Proveniência:(De \textunderscore quitar\textunderscore )}
\end{itemize}
Que saldou as suas contas; livre, desembaraçado.
Acto, em que o toireiro, vendo que um peão ou cavalleiro é alcançado pelo toiro, procura desviar êste, chamando-lhe a attenção para outro ponto.
\section{Quitemente}
\begin{itemize}
\item {Grp. gram.:adv.}
\end{itemize}
\begin{itemize}
\item {Proveniência:(De \textunderscore quite\textunderscore )}
\end{itemize}
Com quitação.
Sem estôrvo, sem impedimento.
\section{Quitesse}
\begin{itemize}
\item {Grp. gram.:m.}
\end{itemize}
Arvoreta violácea de Angola.
(Cp. \textunderscore tesse\textunderscore )
\section{Quiteve}
\begin{itemize}
\item {Grp. gram.:m.}
\end{itemize}
Título de alguns chefes africanos.
\section{Quiti}
\begin{itemize}
\item {Grp. gram.:m.}
\end{itemize}
\begin{itemize}
\item {Utilização:Bras. de Piauí}
\end{itemize}
O mesmo que \textunderscore cacete\textunderscore .
\section{Quitiaquenene}
\begin{itemize}
\item {Grp. gram.:m.}
\end{itemize}
Pássaro dentirostro africano.
\section{Quito}
\begin{itemize}
\item {Grp. gram.:adj.}
\end{itemize}
\begin{itemize}
\item {Utilização:Des.}
\end{itemize}
O mesmo que \textunderscore quite\textunderscore .
\section{Quitó}
\begin{itemize}
\item {Grp. gram.:m.}
\end{itemize}
\begin{itemize}
\item {Utilização:Ant.}
\end{itemize}
Espécie de espada; o mesmo que \textunderscore cotó\textunderscore ^1. Cf. \textunderscore Anat. Joc.\textunderscore , 2.
(Alter. de \textunderscore cotó\textunderscore ^1)
\section{Quitôco}
\begin{itemize}
\item {Grp. gram.:m.}
\end{itemize}
Planta brasileira, (\textunderscore pluchea quitoc\textunderscore ).
\section{Quitólis}
\begin{itemize}
\item {Grp. gram.:m.}
\end{itemize}
Utensílio de chapeleiro. Cf. \textunderscore Inquér. Industr.\textunderscore , p. II, l. II, 37.
\section{Quituche}
\begin{itemize}
\item {Grp. gram.:m.}
\end{itemize}
\begin{itemize}
\item {Utilização:T. de Benguela}
\end{itemize}
O mesmo que \textunderscore crime\textunderscore . Cf. Capello e Ivens, I, 166.
\section{Quitué}
\begin{itemize}
\item {Grp. gram.:m.}
\end{itemize}
Planta medicinal da ilha de San-Thomé.
\section{Quitumbata}
\begin{itemize}
\item {Grp. gram.:f.}
\end{itemize}
Arbusto de Benguela.
\section{Quitundo}
\begin{itemize}
\item {Grp. gram.:m.}
\end{itemize}
Árvore anacardiácea africana, (\textunderscore anaphrenium abyssineum\textunderscore , Hochst.).
\section{Quitungo}
\begin{itemize}
\item {Grp. gram.:m.}
\end{itemize}
\begin{itemize}
\item {Utilização:Bras. do Rio}
\end{itemize}
Ave, o mesmo que \textunderscore gongá\textunderscore .
\section{Quitura}
\begin{itemize}
\item {Grp. gram.:f.}
\end{itemize}
Um moio de milho, em alguns pontos da África oriental.
\section{Quitute}
\begin{itemize}
\item {Grp. gram.:m.}
\end{itemize}
\begin{itemize}
\item {Utilização:Bras}
\end{itemize}
Iguaria delicada; paparico; acepipe. Cf. Beaurepaire-Rohan, \textunderscore Diccion. de Voc. Bras.\textunderscore ; e B. C. Rubim, \textunderscore Vocab. Bras.\textunderscore 
\section{Quituteiro}
\begin{itemize}
\item {Grp. gram.:m.}
\end{itemize}
\begin{itemize}
\item {Utilização:Bras}
\end{itemize}
Homem destro em preparar quitutes.
\section{Quiúto}
\begin{itemize}
\item {Grp. gram.:m.}
\end{itemize}
Arbusto africano, sarmentoso, da fam. das leguminosas.
\section{Quivúvi}
\begin{itemize}
\item {Grp. gram.:m.}
\end{itemize}
Espécie de aranha africana. Cf. Capello e Ivens, II, 137.
\section{Quixa}
\begin{itemize}
\item {Grp. gram.:f.}
\end{itemize}
\begin{itemize}
\item {Utilização:T. de Vouzela}
\end{itemize}
O mesmo que \textunderscore cabra\textunderscore ^1.
\section{Quixibua}
\begin{itemize}
\item {Grp. gram.:f.}
\end{itemize}
Árvore ampelídea de Angola, (\textunderscore vitis schimperiana\textunderscore , Hochst.).
\section{Quixicongo}
\begin{itemize}
\item {Grp. gram.:m.}
\end{itemize}
Língua vernácula do Congo.
\section{Quixinha}
\begin{itemize}
\item {Grp. gram.:f.}
\end{itemize}
O mesmo que \textunderscore quixa\textunderscore .
\section{Quixó}
\begin{itemize}
\item {Grp. gram.:m.}
\end{itemize}
\begin{itemize}
\item {Utilização:Bras}
\end{itemize}
Espécie de mundé.
\section{Quixotada}
\begin{itemize}
\item {Grp. gram.:f.}
\end{itemize}
\begin{itemize}
\item {Proveniência:(De \textunderscore Quixote\textunderscore , n. p.)}
\end{itemize}
Acto ou dito de fanfarrão.
Bazófia ridícula.
\section{Quixotescamente}
\begin{itemize}
\item {Grp. gram.:adv.}
\end{itemize}
De modo quixotesco.
\section{Quixotesco}
\begin{itemize}
\item {fónica:tês}
\end{itemize}
\begin{itemize}
\item {Grp. gram.:adj.}
\end{itemize}
Relativo a quixotada.
Ridiculamente pretencioso.
\section{Quixotice}
\begin{itemize}
\item {Grp. gram.:f.}
\end{itemize}
O mesmo que \textunderscore quixotada\textunderscore .
\section{Quixótico}
\begin{itemize}
\item {Grp. gram.:adj.}
\end{itemize}
O mesmo que \textunderscore quixotesco\textunderscore . Cf. Garrett, \textunderscore Retr. de Vênus\textunderscore , 194.
\section{Quiza}
\begin{itemize}
\item {Grp. gram.:f.}
\end{itemize}
\begin{itemize}
\item {Utilização:Ant.}
\end{itemize}
Espécie de túnica.
\section{Quizengo}
\begin{itemize}
\item {Grp. gram.:m.}
\end{itemize}
Planta angolense, de fibras têsteis.
\section{Quizília}
\textunderscore f.\textunderscore  (e der.)
O mesmo que \textunderscore quezília\textunderscore , etc.
\section{Quizunda}
\begin{itemize}
\item {Grp. gram.:f.}
\end{itemize}
Arbusto africano, da fam. das leguminosas, de fôlhas largas, verde-claras e flôres inodoras em espigas papilionáceas.
\section{Quizungrila}
\begin{itemize}
\item {Grp. gram.:f.}
\end{itemize}
Arbusto africano, de caule tetrágono ou estriado, flôres papilionáceas e fôlhas oppostas, trifoliadas.
\section{Quocientar}
\begin{itemize}
\item {Grp. gram.:v. t.}
\end{itemize}
\begin{itemize}
\item {Utilização:Neol.}
\end{itemize}
Achar ou determinar o quociente de. Cf. R. Jorge, \textunderscore Censo dos Tuberculosos\textunderscore , XIV.
\section{Quociente}
\begin{itemize}
\item {Grp. gram.:m.}
\end{itemize}
\begin{itemize}
\item {Utilização:Arith.}
\end{itemize}
\begin{itemize}
\item {Proveniência:(Lat. \textunderscore quociens\textunderscore )}
\end{itemize}
Número, que indica quantas vezes o divisor se contém no dividendo.
\section{Quócolo}
\begin{itemize}
\item {Grp. gram.:m.}
\end{itemize}
Pedra de Itália, facilmente vitrificável.
\section{Quodlibeto}
\begin{itemize}
\item {Grp. gram.:m.}
\end{itemize}
\begin{itemize}
\item {Utilização:Ant.}
\end{itemize}
\begin{itemize}
\item {Proveniência:(Do lat. \textunderscore quodlibet\textunderscore )}
\end{itemize}
Provas públicas, que os doutorandos da Universidade de Coimbra davam, em o nono anno dos seus estudos.
\section{Quodore}
\begin{itemize}
\item {Grp. gram.:m.}
\end{itemize}
Pequena porção de vinho; um gole de vinho.
Pequena porção de alimento.
Dejejuadoiro. Cf. Castilho, \textunderscore Fausto\textunderscore , 62 e 166.
(Da loc. lat. \textunderscore quod ore\textunderscore )
\section{Quodório}
\begin{itemize}
\item {Grp. gram.:m.}
\end{itemize}
\begin{itemize}
\item {Utilização:Fam.}
\end{itemize}
O mesmo que \textunderscore quodore\textunderscore .
\section{Quogelo}
\begin{itemize}
\item {Grp. gram.:m.}
\end{itemize}
Espécie de crocodilo da Cafraria.
\section{Quó-quó!}
\begin{itemize}
\item {Grp. gram.:interj.}
\end{itemize}
Voz imitativa da voz da gallinha. Cf. S. R. Viterbo, \textunderscore Elucidário\textunderscore .
\section{Quota}
\begin{itemize}
\item {Grp. gram.:f.}
\end{itemize}
(V. \textunderscore cota\textunderscore ^2)
\section{Quote}
\begin{itemize}
\item {Grp. gram.:m.}
\end{itemize}
(V. \textunderscore cote\textunderscore ^1)
\section{Quotidianamente}
\begin{itemize}
\item {Grp. gram.:adv.}
\end{itemize}
De modo quotidiano; diariamente; constantemente.
\section{Quotidiano}
\begin{itemize}
\item {Grp. gram.:adj.}
\end{itemize}
\begin{itemize}
\item {Proveniência:(Do lat. \textunderscore quotidie\textunderscore )}
\end{itemize}
Que succede diariamente ou que se faz todos os dias; que succede ou se pratíca habitualmente.
\section{Quotiliquê}
\begin{itemize}
\item {Grp. gram.:m.}
\end{itemize}
Coisa ou pessôa de pouca monta; bagatela; ninharia.
(Da soletração antiga de \textunderscore [~q]\textunderscore : \textunderscore ku\textunderscore  + \textunderscore til\textunderscore  = \textunderscore quê\textunderscore )
\section{Quotização}
\begin{itemize}
\item {Grp. gram.:f.}
\end{itemize}
Acto ou effeito de quotizar.
\section{Quotizar}
\begin{itemize}
\item {Grp. gram.:v. t.}
\end{itemize}
\begin{itemize}
\item {Grp. gram.:V. p.}
\end{itemize}
Distribuír por quota.
Fixar o preço de.
Contribuír com quota.
\section{Quutiliquê}
\begin{itemize}
\item {Grp. gram.:m.}
\end{itemize}
\end{document}