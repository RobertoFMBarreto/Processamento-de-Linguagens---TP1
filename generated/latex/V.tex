\documentclass{article}
\usepackage[portuguese]{babel}
\title{V}
\begin{document}
O mesmo ou melhór que \textunderscore uzífur\textunderscore .
(Cp. cast. \textunderscore uzífero\textunderscore )
\section{V}
\begin{itemize}
\item {fónica:vê}
\end{itemize}
\begin{itemize}
\item {Grp. gram.:m.}
\end{itemize}
Vigésima segunda letra do alphabeto português.
Abrev. de \textunderscore vosso\textunderscore  e \textunderscore vossa\textunderscore , em vários tratamentos.
Designação de \textunderscore 5\textunderscore  em numeração romana.
Abrev. de \textunderscore veja\textunderscore  ou \textunderscore veja-se\textunderscore .
Abrev. de \textunderscore você\textunderscore .
Em numeração antiga, com um til, valia 5:000.--Nas nossas escolas antigas, pronunciava-se \textunderscore vau\textunderscore , que é o nome desta letra no alphabeto hebraico.
\section{V. A.}
Abrev. de \textunderscore Vossa Alteza\textunderscore .
\section{Vaali}
\begin{itemize}
\item {Grp. gram.:m.}
\end{itemize}
Designação, dada pelos Persas aos Príncipes, cujos Estados são conquistados pelo Xá.
\section{Vaca}
\begin{itemize}
\item {Grp. gram.:f.}
\end{itemize}
\begin{itemize}
\item {Utilização:Fig.}
\end{itemize}
\begin{itemize}
\item {Grp. gram.:M.}
\end{itemize}
\begin{itemize}
\item {Utilização:Ant.}
\end{itemize}
\begin{itemize}
\item {Utilização:Chul.}
\end{itemize}
\begin{itemize}
\item {Proveniência:(Lat. \textunderscore vacca\textunderscore )}
\end{itemize}
Fêmea do boi.
Carne de gado vacum.
Parada ao jôgo, feita por um só parceiro, mas em nome de dois ou mais.
Origem constante de interesses ou lucros.
Correia ou corda grossa de cânhamo, que serve para fazer mover o carro do prelo typográphico.
O mesmo que \textunderscore seminarista\textunderscore .
\section{Vaca}
\begin{itemize}
\item {Grp. gram.:f.}
\end{itemize}
\begin{itemize}
\item {Utilização:Bras}
\end{itemize}
Árvore silvestre, cuja madeira serve para remos.
\section{Vacação}
\begin{itemize}
\item {Grp. gram.:f.}
\end{itemize}
\begin{itemize}
\item {Proveniência:(Do lat. \textunderscore vacatio\textunderscore )}
\end{itemize}
Acto ou effeito de vacar.
\section{Vacada}
\begin{itemize}
\item {Grp. gram.:f.}
\end{itemize}
Multidão de vacas.
Corrida de vacas.
\section{Vaca-fria}
\begin{itemize}
\item {Grp. gram.:f.}
\end{itemize}
Us. na loc. \textunderscore voltar á vaca-fria\textunderscore , voltar a assumpto já tratado, repisar assumpto de que se falou.
\section{Vaca-loira}
\begin{itemize}
\item {Grp. gram.:f.}
\end{itemize}
Gênero de insectos coleópteros, a que pertence a canthárida.
Insecto escuro, listrado de vermelho, do qual se diz que, para o matarem, basta cuspir-lhe, e que faz arrebentar o boi que o come.
\section{Vaca-marinha}
\begin{itemize}
\item {Grp. gram.:f.}
\end{itemize}
Grande mammífero amphíbio dos mares árcticos, que chega a têr sete metros de comprimento.
\section{Vacância}
\begin{itemize}
\item {Grp. gram.:f.}
\end{itemize}
\begin{itemize}
\item {Proveniência:(Lat. \textunderscore vacantia\textunderscore )}
\end{itemize}
Estado do que é vago.
Tempo, durante o qual está vago um lugar, cargo, dignidade, etc.; vagatura.
\section{Vaca-negra}
\begin{itemize}
\item {Grp. gram.:f.}
\end{itemize}
\begin{itemize}
\item {Utilização:Prov.}
\end{itemize}
O mesmo que \textunderscore vaca-loira\textunderscore .
\section{Vacante}
\begin{itemize}
\item {Grp. gram.:adj.}
\end{itemize}
\begin{itemize}
\item {Grp. gram.:F.}
\end{itemize}
\begin{itemize}
\item {Proveniência:(Lat. \textunderscore vacans\textunderscore )}
\end{itemize}
Que está vago.
O mesmo que \textunderscore vagatura\textunderscore . Cf. Latino, \textunderscore Elogios\textunderscore , 238 e 333.
\section{Vacão}
\begin{itemize}
\item {Grp. gram.:m.}
\end{itemize}
\begin{itemize}
\item {Utilização:Prov.}
\end{itemize}
\begin{itemize}
\item {Utilização:Prov.}
\end{itemize}
\begin{itemize}
\item {Utilização:T. do Fundão}
\end{itemize}
\begin{itemize}
\item {Proveniência:(De \textunderscore vaca\textunderscore )}
\end{itemize}
Campónio; rústico.
Palerma; homem estúpido.
Homem inútil, indolente, mandrião.
\section{Vacar}
\begin{itemize}
\item {Grp. gram.:v. i.}
\end{itemize}
\begin{itemize}
\item {Utilização:Ant.}
\end{itemize}
\begin{itemize}
\item {Proveniência:(Lat. \textunderscore vacare\textunderscore )}
\end{itemize}
Estar vago.
Estar em férias, não têr que fazer.
Dar-se, dedicar-se. Cf. Pant. de Aveiro, \textunderscore Itiner.\textunderscore , 90; (2.^a ed.).
\section{Vacaria}
\begin{itemize}
\item {Grp. gram.:f.}
\end{itemize}
Conjunto de vacas, vacada.
Gado vacum.
Curral de vacas.
Loja, onde se tratam e recolhem vacas, para se lhes vender o leite á vista do comprador.
(B. lat. \textunderscore vaccaria\textunderscore )
\section{Vacaril}
\begin{itemize}
\item {Grp. gram.:adj.}
\end{itemize}
\begin{itemize}
\item {Grp. gram.:M. pl.}
\end{itemize}
\begin{itemize}
\item {Utilização:Ant.}
\end{itemize}
\begin{itemize}
\item {Proveniência:(De \textunderscore vaccaria\textunderscore )}
\end{itemize}
Relativo a vaca.
Vacum, vacarino.
Coiros de bois ou de vacas.
\section{Vacarino}
\begin{itemize}
\item {Grp. gram.:adj.}
\end{itemize}
\begin{itemize}
\item {Utilização:Prov.}
\end{itemize}
\begin{itemize}
\item {Utilização:alg.}
\end{itemize}
\begin{itemize}
\item {Proveniência:(De \textunderscore vacaria\textunderscore )}
\end{itemize}
Relativo a vaca; vacum; vacaril.
\section{Vacarrona}
\begin{itemize}
\item {Grp. gram.:f.}
\end{itemize}
\begin{itemize}
\item {Utilização:Prov.}
\end{itemize}
\begin{itemize}
\item {Utilização:trasm.}
\end{itemize}
\begin{itemize}
\item {Proveniência:(De \textunderscore vaca\textunderscore )}
\end{itemize}
Mulhér vagarosa, indolente, desleixada.
\section{Vacatura}
\begin{itemize}
\item {Grp. gram.:f.}
\end{itemize}
O mesmo que \textunderscore vagatura\textunderscore .
\section{Vacca}
\textunderscore f.\textunderscore  (e der.)
(V. \textunderscore vaca\textunderscore ^1, etc.)
\section{Vacilação}
\begin{itemize}
\item {Grp. gram.:f.}
\end{itemize}
\begin{itemize}
\item {Proveniência:(Do lat. \textunderscore vacillatio\textunderscore )}
\end{itemize}
Acto ou efeito de vacilar.
Estado daquilo que vacila.
Oscilação.
Perplexidade.
\section{Vacilância}
\begin{itemize}
\item {Grp. gram.:f.}
\end{itemize}
\begin{itemize}
\item {Utilização:P. us.}
\end{itemize}
O mesmo que \textunderscore oscilação\textunderscore .
(Cp. \textunderscore vacilante\textunderscore )
\section{Vacilante}
\begin{itemize}
\item {Grp. gram.:adj.}
\end{itemize}
\begin{itemize}
\item {Proveniência:(Lat. \textunderscore vacillans\textunderscore )}
\end{itemize}
Que vacila; que oscila.
Pouco firme, perplexo.
Instável, mudável.
\section{Vacilar}
\begin{itemize}
\item {Grp. gram.:v. i.}
\end{itemize}
\begin{itemize}
\item {Utilização:Fig.}
\end{itemize}
\begin{itemize}
\item {Grp. gram.:V. t.}
\end{itemize}
\begin{itemize}
\item {Utilização:Des.}
\end{itemize}
\begin{itemize}
\item {Proveniência:(Lat. \textunderscore vacillare\textunderscore )}
\end{itemize}
Não estar firme; cambalear; oscilar.
Tremer.
Afroixar.
Estar perplexo; hesitar.
Fazer cambalear; abalar.
\section{Vacilatório}
\begin{itemize}
\item {Grp. gram.:adj.}
\end{itemize}
\begin{itemize}
\item {Proveniência:(De \textunderscore vacilar\textunderscore )}
\end{itemize}
O mesmo que \textunderscore vacilante\textunderscore .
Que produz vacilação.
\section{Vacillação}
\begin{itemize}
\item {Grp. gram.:f.}
\end{itemize}
\begin{itemize}
\item {Proveniência:(Do lat. \textunderscore vacillatio\textunderscore )}
\end{itemize}
Acto ou effeito de vacillar.
Estado daquillo que vacilla.
Oscillação.
Perplexidade.
\section{Vacillância}
\begin{itemize}
\item {Grp. gram.:f.}
\end{itemize}
\begin{itemize}
\item {Utilização:P. us.}
\end{itemize}
O mesmo que \textunderscore oscillação\textunderscore .
(Cp. \textunderscore vacillante\textunderscore )
\section{Vacillante}
\begin{itemize}
\item {Grp. gram.:adj.}
\end{itemize}
\begin{itemize}
\item {Proveniência:(Lat. \textunderscore vacillans\textunderscore )}
\end{itemize}
Que vacilla; que oscilla.
Pouco firme, perplexo.
Instável, mudável.
\section{Vacillar}
\begin{itemize}
\item {Grp. gram.:v. i.}
\end{itemize}
\begin{itemize}
\item {Utilização:Fig.}
\end{itemize}
\begin{itemize}
\item {Grp. gram.:V. t.}
\end{itemize}
\begin{itemize}
\item {Utilização:Des.}
\end{itemize}
\begin{itemize}
\item {Proveniência:(Lat. \textunderscore vacillare\textunderscore )}
\end{itemize}
Não estar firme; cambalear; oscillar.
Tremer.
Afroixar.
Estar perplexo; hesitar.
Fazer cambalear; abalar.
\section{Vacillatório}
\begin{itemize}
\item {Grp. gram.:adj.}
\end{itemize}
\begin{itemize}
\item {Proveniência:(De \textunderscore vacillar\textunderscore )}
\end{itemize}
O mesmo que \textunderscore vacillante\textunderscore .
Que produz vacillação.
\section{Vacina}
\begin{itemize}
\item {Grp. gram.:f.}
\end{itemize}
\begin{itemize}
\item {Utilização:Ext.}
\end{itemize}
\begin{itemize}
\item {Proveniência:(Lat. \textunderscore vaccina\textunderscore )}
\end{itemize}
Doença eruptiva e própria das vacas.
Humor especial, extrahido das pústulas produzidas por essa doença, e que, inoculado numa pessôa, a preserva da varíola.
Vacinação.
Inoculação de um vírus, como preservativo de qualquer doença contagiosa.
\section{Vacinação}
\begin{itemize}
\item {Grp. gram.:f.}
\end{itemize}
Acto ou effeito de vacinar.
\section{Vacinador}
\begin{itemize}
\item {Grp. gram.:m.  e  adj.}
\end{itemize}
\begin{itemize}
\item {Grp. gram.:M.}
\end{itemize}
O que vacina.
Lanceta, própria para a vacinação.
\section{Vacinal}
\begin{itemize}
\item {Grp. gram.:adj.}
\end{itemize}
O mesmo que \textunderscore vacínico\textunderscore .
\section{Vacinar}
\begin{itemize}
\item {Grp. gram.:v. t.}
\end{itemize}
Inocular a vacina em.
Inocular ou transmittir por meio do vacinador (a alguém) a substância vacínica, que se extrai de outrem, recentemente vacinado.
Inocular em (alguém) o vírus de qualquer doença contagiosa, como meio preservativo da mesma doença.
\section{Vaciniáceas}
\begin{itemize}
\item {Grp. gram.:f. pl.}
\end{itemize}
Família de plantas, que tem por typo o vacínio.
(Fem. pl. de \textunderscore vacciniácio\textunderscore )
\section{Vaciniáceo}
\begin{itemize}
\item {Grp. gram.:adj.}
\end{itemize}
Relativo ou semelhante ao vacínio.
\section{Vacínico}
\begin{itemize}
\item {Grp. gram.:adj.}
\end{itemize}
Relativo á vacina.
\section{Vacínio}
\begin{itemize}
\item {Grp. gram.:m.}
\end{itemize}
\begin{itemize}
\item {Proveniência:(Lat. \textunderscore vaccinium\textunderscore )}
\end{itemize}
Nome genérico de algumas plantas, como uma espécie de violeta arbustiva, o arando, etc.
\section{Vacinogenia}
\begin{itemize}
\item {Grp. gram.:f.}
\end{itemize}
\begin{itemize}
\item {Utilização:Neol.}
\end{itemize}
Producção de vacina.
\section{Vacinogênico}
\begin{itemize}
\item {Grp. gram.:adj.}
\end{itemize}
Relativo á vacinogenia.
Próprio para a vacinogenia: \textunderscore em Lisbôa visitámos o parque vacinogênico...\textunderscore 
\section{Vacu}
\begin{itemize}
\item {Grp. gram.:m.}
\end{itemize}
Peixe do norte do Brasil.
\section{Vacuidade}
\begin{itemize}
\item {fónica:cu-i}
\end{itemize}
\begin{itemize}
\item {Grp. gram.:f.}
\end{itemize}
\begin{itemize}
\item {Proveniência:(Do lat. \textunderscore vacuitas\textunderscore )}
\end{itemize}
Estado do que é vazio.
\section{Vacuísmo}
\begin{itemize}
\item {Grp. gram.:m.}
\end{itemize}
\begin{itemize}
\item {Proveniência:(De \textunderscore vácuo\textunderscore )}
\end{itemize}
Systema dos que admittiam o vácuo em a natureza, ou dos que criam que além dos mundos só há vácuo.
\section{Vacuista}
\begin{itemize}
\item {fónica:cu-i}
\end{itemize}
\begin{itemize}
\item {Grp. gram.:m.}
\end{itemize}
\begin{itemize}
\item {Proveniência:(De \textunderscore vácuo\textunderscore )}
\end{itemize}
Partidário do vacuísmo.
\section{Vacum}
\begin{itemize}
\item {Grp. gram.:adj.}
\end{itemize}
\begin{itemize}
\item {Grp. gram.:M.}
\end{itemize}
\begin{itemize}
\item {Proveniência:(De \textunderscore vaca\textunderscore ^1)}
\end{itemize}
Diz-se do gado, que comprehende vacas, bois e novilhos.
Gado vacum:«\textunderscore em campos sempre cheios de vacum bravo.\textunderscore »\textunderscore Viriato Trág.\textunderscore , VI, 83.
\section{Vácuo}
\begin{itemize}
\item {Grp. gram.:adj.}
\end{itemize}
\begin{itemize}
\item {Grp. gram.:M.}
\end{itemize}
\begin{itemize}
\item {Proveniência:(Lat. \textunderscore vacuus\textunderscore )}
\end{itemize}
Que não contém nada.
Apenas cheio do ar.
Vazio; despejado:«\textunderscore regiões vácuas.\textunderscore »Castilho, \textunderscore Sabichonas\textunderscore , 124.
Espaço vazio.
Espaço vazio entre os corpos celestes.
Espaço imaginário, que não é occupado por coisa alguma.
\section{Vadagaio}
\begin{itemize}
\item {Grp. gram.:m.}
\end{itemize}
\begin{itemize}
\item {Utilização:Prov.}
\end{itemize}
\begin{itemize}
\item {Utilização:alent.}
\end{itemize}
Vágado, desmaio.
(Alter. de \textunderscore vádago\textunderscore , metáth. de \textunderscore vágado\textunderscore )
\section{Vadeação}
\begin{itemize}
\item {Grp. gram.:f.}
\end{itemize}
Acto ou effeito de vadear.
\section{Vadear}
\begin{itemize}
\item {Grp. gram.:v. t.}
\end{itemize}
\begin{itemize}
\item {Proveniência:(Do lat. \textunderscore vadum\textunderscore )}
\end{itemize}
Passar a vau.
\section{Vadeável}
\begin{itemize}
\item {Grp. gram.:adj.}
\end{itemize}
Que se póde vadear.
\section{Vadeoso}
\begin{itemize}
\item {Grp. gram.:adj.}
\end{itemize}
\begin{itemize}
\item {Proveniência:(De \textunderscore vadear\textunderscore )}
\end{itemize}
O mesmo que \textunderscore vadoso\textunderscore .
\section{Vá-de-viró!}
\begin{itemize}
\item {Grp. gram.:interj.}
\end{itemize}
(Us. pelos pescadores algarvios, para fazer parar o barco, quando se está a varar)
\section{Vadiação}
\begin{itemize}
\item {Grp. gram.:f.}
\end{itemize}
Acto ou effeito de vadiar.
\section{Vadiagem}
\begin{itemize}
\item {Grp. gram.:f.}
\end{itemize}
Vadiação; vida de vadio.
\section{Vadiamente}
\begin{itemize}
\item {Grp. gram.:adv.}
\end{itemize}
Á maneira de vadio.
\section{Vadião}
\begin{itemize}
\item {Grp. gram.:m.}
\end{itemize}
\begin{itemize}
\item {Utilização:Des.}
\end{itemize}
Grande vadio.
\section{Vadiar}
\begin{itemize}
\item {fónica:vá}
\end{itemize}
\begin{itemize}
\item {Grp. gram.:v. i.}
\end{itemize}
\begin{itemize}
\item {Proveniência:(De \textunderscore vadio\textunderscore )}
\end{itemize}
Andar ociosamente de uma parte para outra.
Andar á tuna.
Têr vida de vadio.
Passar vida ociosa.
\section{Vadiaria}
\begin{itemize}
\item {Grp. gram.:f.}
\end{itemize}
O mesmo que \textunderscore vadiagem\textunderscore . Cf. \textunderscore Anat. Joc.\textunderscore , I, 320.
\section{Vadiice}
\begin{itemize}
\item {Grp. gram.:f.}
\end{itemize}
O mesmo que \textunderscore vadiagem\textunderscore .
\section{Vadiismo}
\begin{itemize}
\item {Grp. gram.:m.}
\end{itemize}
Hábito do vadiar. Cf. Macedo, \textunderscore Motim\textunderscore , I, 77.
\section{Vadio}
\begin{itemize}
\item {fónica:vá-dí-o}
\end{itemize}
\begin{itemize}
\item {Grp. gram.:m.  e  adj.}
\end{itemize}
\begin{itemize}
\item {Proveniência:(Do ár. \textunderscore baladi\textunderscore , seg. G. Viana)}
\end{itemize}
O que não tem occupação ou que não faz nada.
O que vagueia; vagabundo; tunante.
Próprio de gente ociosa.
\section{Vadiote}
\begin{itemize}
\item {Grp. gram.:m.}
\end{itemize}
\begin{itemize}
\item {Utilização:Bras}
\end{itemize}
Indivíduo, que é um tanto vadio.
\section{Vadoso}
\begin{itemize}
\item {Grp. gram.:adj.}
\end{itemize}
\begin{itemize}
\item {Proveniência:(Lat. \textunderscore vadosus\textunderscore )}
\end{itemize}
Em que há vau.
Em que há bancos de areia.
\section{Vadroíl}
\begin{itemize}
\item {Grp. gram.:m.}
\end{itemize}
Vassoira de trapos, com que se lavam as embarcações.
\section{Vafrícia}
\begin{itemize}
\item {Grp. gram.:f.}
\end{itemize}
\begin{itemize}
\item {Utilização:Des.}
\end{itemize}
\begin{itemize}
\item {Utilização:Poét.}
\end{itemize}
\begin{itemize}
\item {Proveniência:(Lat. \textunderscore vafritia\textunderscore )}
\end{itemize}
Qualidade de vafro.
Astúcia; finura.
\section{Vafro}
\begin{itemize}
\item {Grp. gram.:adj.}
\end{itemize}
\begin{itemize}
\item {Utilização:Des.}
\end{itemize}
\begin{itemize}
\item {Utilização:Poét.}
\end{itemize}
\begin{itemize}
\item {Proveniência:(Lat. \textunderscore vafer\textunderscore )}
\end{itemize}
Sagaz; astuto, manhoso; finório.
\section{Vaga}
\begin{itemize}
\item {Grp. gram.:f.}
\end{itemize}
\begin{itemize}
\item {Utilização:Fig.}
\end{itemize}
Onda grande.
Multidão, que se espalha ou invade tumultuosamente, á maneira de onda.
Grande agitação.
(Ant. alt. al. \textunderscore vâg\textunderscore )
\section{Vaga}
\begin{itemize}
\item {Grp. gram.:f.}
\end{itemize}
Acto ou effeito de vagar^1; vagatura.
Falta ou ausência.
Ócio.
Vagar, lazer. Cf. Filinto, \textunderscore D. Man.\textunderscore , III, 410.
\section{Vagabundagem}
\begin{itemize}
\item {Grp. gram.:f.}
\end{itemize}
Vida de vagabundo; os vagabundos.
\section{Vagabundear}
\begin{itemize}
\item {Grp. gram.:v. i.}
\end{itemize}
\begin{itemize}
\item {Proveniência:(De \textunderscore vagabundo\textunderscore )}
\end{itemize}
Vadiar.
Andar de terra em terra sem necessidade.
\section{Vagabundo}
\begin{itemize}
\item {Grp. gram.:adj.}
\end{itemize}
\begin{itemize}
\item {Grp. gram.:M.}
\end{itemize}
\begin{itemize}
\item {Proveniência:(Lat. \textunderscore vagabundus\textunderscore )}
\end{itemize}
Que vagabundeia; errante; nômade.
Inconstante.
Vadio.
\section{Vagação}
\begin{itemize}
\item {Grp. gram.:f.}
\end{itemize}
O mesmo que \textunderscore vaga\textunderscore ^2.
\section{Vágado}
\begin{itemize}
\item {Grp. gram.:m.}
\end{itemize}
Vertigem; delíquio; desmaio.
(Ant. alt. al. \textunderscore wagida\textunderscore )
\section{Vagalhão}
\begin{itemize}
\item {Grp. gram.:m.}
\end{itemize}
Grande vaga^1.
\section{Vagalume}
\begin{itemize}
\item {fónica:vá}
\end{itemize}
\begin{itemize}
\item {Grp. gram.:m.}
\end{itemize}
O mesmo que \textunderscore pyrilampo\textunderscore .
(Eufemismo, por \textunderscore caga-lume\textunderscore )
\section{Vagamão}
\begin{itemize}
\item {Grp. gram.:m.  e  adj.}
\end{itemize}
\begin{itemize}
\item {Utilização:Ant.}
\end{itemize}
O mesmo que \textunderscore vagamundo\textunderscore .
\section{Vagamente}
\begin{itemize}
\item {Grp. gram.:adv.}
\end{itemize}
De modo vago, de modo indeterminado.
\section{Vagamundear}
\begin{itemize}
\item {Grp. gram.:v. i.}
\end{itemize}
\begin{itemize}
\item {Proveniência:(De \textunderscore vagamundo\textunderscore )}
\end{itemize}
O mesmo que \textunderscore vagabundear\textunderscore . Cf. Camillo, \textunderscore Olho de Vidro\textunderscore , 102.
\section{Vagamundo}
\textunderscore m.\textunderscore  e \textunderscore adj.\textunderscore  (e der.)
(V. \textunderscore vagabundo\textunderscore , etc.). Cf. F. Manuel, \textunderscore Apólogos\textunderscore . Pant. de Aveiro, \textunderscore Itiner.\textunderscore , 174, (2.^a ed.)
\section{Vaganão}
\begin{itemize}
\item {Grp. gram.:adj.}
\end{itemize}
O mesmo que \textunderscore vadio\textunderscore . Cf. Arn. Gama, \textunderscore Última Dona\textunderscore , 22 e 107.
(Outra fórma de \textunderscore vaganau\textunderscore )
\section{Vaganau}
\begin{itemize}
\item {Grp. gram.:m.}
\end{itemize}
\begin{itemize}
\item {Utilização:Ant.}
\end{itemize}
\begin{itemize}
\item {Utilização:Prov.}
\end{itemize}
\begin{itemize}
\item {Utilização:trasm.}
\end{itemize}
Maganão; mariola, vadio.
Indivíduo corpulento.
(Outra fórma de \textunderscore vagamão\textunderscore ?)
\section{Vagância}
\begin{itemize}
\item {Grp. gram.:f.}
\end{itemize}
O mesmo que \textunderscore vacância\textunderscore .
\section{Vaganear}
\begin{itemize}
\item {Grp. gram.:v. i.}
\end{itemize}
Têr vida de vaganão. Cf. Arn. Gama, \textunderscore Última Dona\textunderscore , 21 e 355.
\section{Vaganice}
\begin{itemize}
\item {Grp. gram.:f.}
\end{itemize}
Acto ou vida de vaganão. Cf. Arn. Gama, \textunderscore Última Dona\textunderscore , 354.
\section{Vagante}
\begin{itemize}
\item {Grp. gram.:adj.}
\end{itemize}
\begin{itemize}
\item {Proveniência:(De \textunderscore vagar\textunderscore ^2)}
\end{itemize}
Que vagueia, que anda errante.
\section{Vagante}
\begin{itemize}
\item {Grp. gram.:adj.}
\end{itemize}
\begin{itemize}
\item {Grp. gram.:F.}
\end{itemize}
\begin{itemize}
\item {Proveniência:(Do lat. \textunderscore vacans\textunderscore )}
\end{itemize}
Que está vago.
O mesmo que \textunderscore vagatura\textunderscore .
\section{Vaganti}
\begin{itemize}
\item {Grp. gram.:m.}
\end{itemize}
Espécie de tigre indiano.
(Do concani)
\section{Vagantio}
\begin{itemize}
\item {Grp. gram.:adj.}
\end{itemize}
\begin{itemize}
\item {Utilização:Neol.}
\end{itemize}
\begin{itemize}
\item {Proveniência:(De \textunderscore vagante\textunderscore ^1)}
\end{itemize}
Inconstante, volúvel: \textunderscore corações vagantios\textunderscore .
\section{Vagão}
\begin{itemize}
\item {Grp. gram.:m.}
\end{itemize}
\begin{itemize}
\item {Proveniência:(Do fr. \textunderscore wagon\textunderscore )}
\end{itemize}
Carruagem, empregada em combóios de caminho de ferro.
\section{Vagar}
\begin{itemize}
\item {Grp. gram.:v. i.}
\end{itemize}
\begin{itemize}
\item {Grp. gram.:V. t.}
\end{itemize}
\begin{itemize}
\item {Utilização:Des.}
\end{itemize}
\begin{itemize}
\item {Grp. gram.:M.}
\end{itemize}
\begin{itemize}
\item {Grp. gram.:Loc. adv.}
\end{itemize}
\begin{itemize}
\item {Proveniência:(Do lat. \textunderscore vacare\textunderscore )}
\end{itemize}
Estar vago.
Estar vazio.
Achar-se desoccupado.
Sobejar; faltar.
Occupar-se, dedicar-se.
Abrir vagatura em.
Tornar vago.
Estado do que não tem occupação; ócio; descanso.
Demora, lentidão.
Opportunidade.
\textunderscore De vagar\textunderscore , lentamente; mansamente.
Sem barulho.
Sem pressa.
\section{Vagar}
\begin{itemize}
\item {Grp. gram.:v. i.}
\end{itemize}
\begin{itemize}
\item {Utilização:Fig.}
\end{itemize}
\begin{itemize}
\item {Grp. gram.:V. t.}
\end{itemize}
\begin{itemize}
\item {Utilização:Des.}
\end{itemize}
\begin{itemize}
\item {Proveniência:(Lat. \textunderscore vagari\textunderscore )}
\end{itemize}
Andar sem destino, ao acaso.
Vaguear.
Espalhar-se, propalar-se.
Boiar.
Percorrer ao acaso, sem destino.
\section{Vagarosa}
\begin{itemize}
\item {Grp. gram.:f.}
\end{itemize}
\begin{itemize}
\item {Utilização:Gír.}
\end{itemize}
\begin{itemize}
\item {Proveniência:(De \textunderscore vagaroso\textunderscore )}
\end{itemize}
Cárcere.
\section{Vagarosamente}
\begin{itemize}
\item {Grp. gram.:adv.}
\end{itemize}
De modo vagaroso.
Lentamente; sem ruído; mansamente.
\section{Vagaroso}
\begin{itemize}
\item {Grp. gram.:adj.}
\end{itemize}
\begin{itemize}
\item {Proveniência:(De \textunderscore vagar\textunderscore ^1)}
\end{itemize}
Em que há vagar.
Demorado, lento, pausado.
Sereno; feito sem ruído.
Que não tem pressa.
Froixo.
Indeciso.
Que não tem desembaraço.
\section{Vagatura}
\begin{itemize}
\item {Grp. gram.:f.}
\end{itemize}
\begin{itemize}
\item {Proveniência:(De \textunderscore vagar\textunderscore ^1)}
\end{itemize}
Estado daquillo que está vago ou que vagou.
Tempo, durante o qual, um lugar ou emprêgo não está preenchido; vacância.
\section{Vaga-vaga-de-obó}
\begin{itemize}
\item {Grp. gram.:f.}
\end{itemize}
Planta medicinal da ilha de San-Thomé.
\section{Vage}
\begin{itemize}
\item {Grp. gram.:f.}
\end{itemize}
Invólucro das sementes ou grãos das plantas leguminosas.
Feijão verde ou carrapato.
(Cp. b. lat. \textunderscore vaginella\textunderscore )
\section{Vagem}
\begin{itemize}
\item {Grp. gram.:f.}
\end{itemize}
Invólucro das sementes ou grãos das plantas leguminosas.
Feijão verde ou carrapato.
(Cp. b. lat. \textunderscore vaginella\textunderscore )
\section{Vagido}
\begin{itemize}
\item {Grp. gram.:m.}
\end{itemize}
\begin{itemize}
\item {Utilização:Fig.}
\end{itemize}
\begin{itemize}
\item {Proveniência:(Do lat. \textunderscore vagitus\textunderscore )}
\end{itemize}
Chôro de criança recém-nascida.
Lamento, gemido.
\section{Vagiforme}
\begin{itemize}
\item {Grp. gram.:adj.}
\end{itemize}
\begin{itemize}
\item {Utilização:Bot.}
\end{itemize}
O mesmo que \textunderscore vaginiforme\textunderscore .
\section{Vagina}
\begin{itemize}
\item {Grp. gram.:f.}
\end{itemize}
\begin{itemize}
\item {Utilização:Anat.}
\end{itemize}
\begin{itemize}
\item {Utilização:Bot.}
\end{itemize}
\begin{itemize}
\item {Proveniência:(Lat. \textunderscore vagina\textunderscore )}
\end{itemize}
Canal, que conduz á madre.
Producção membranosa, que cerca a base dos pedúnculos dos musgos.
\section{Vaginal}
\begin{itemize}
\item {Grp. gram.:adj.}
\end{itemize}
\begin{itemize}
\item {Utilização:Bot.}
\end{itemize}
Relativo á vagina.
Vaginiforme.
Que cérca ou abraça, á maneira de baínha.
\section{Vaginante}
\begin{itemize}
\item {Grp. gram.:adj.}
\end{itemize}
\begin{itemize}
\item {Utilização:Zool.}
\end{itemize}
\begin{itemize}
\item {Proveniência:(De \textunderscore vagina\textunderscore )}
\end{itemize}
Diz-se das asas superiores dos insectos coleópteros e orthópteros.
\section{Vaginária}
\begin{itemize}
\item {Grp. gram.:f.}
\end{itemize}
Gênero de plantas cyperáceas.
\section{Vaginela}
\begin{itemize}
\item {Grp. gram.:f.}
\end{itemize}
\begin{itemize}
\item {Utilização:Bot.}
\end{itemize}
Pequena baínha, que cérca cada fascículo de fôlhas, como sucede no pinheiro.
(B. lat. \textunderscore vaginella\textunderscore )
\section{Vaginella}
\begin{itemize}
\item {Grp. gram.:f.}
\end{itemize}
\begin{itemize}
\item {Utilização:Bot.}
\end{itemize}
Pequena baínha, que cérca cada fascículo de fôlhas, como succede no pinheiro.
(B. lat. \textunderscore vaginella\textunderscore )
\section{Vaginiforme}
\begin{itemize}
\item {Grp. gram.:adj.}
\end{itemize}
\begin{itemize}
\item {Utilização:Bot.}
\end{itemize}
\begin{itemize}
\item {Proveniência:(De \textunderscore vagina\textunderscore  + \textunderscore fórma\textunderscore )}
\end{itemize}
Que tem fórma de vagina.
Que tem fórma de baínha.
\section{Vaginismo}
\begin{itemize}
\item {Grp. gram.:m.}
\end{itemize}
\begin{itemize}
\item {Proveniência:(De \textunderscore vagina\textunderscore )}
\end{itemize}
Espasmo vaginal.
\section{Vaginite}
\begin{itemize}
\item {Grp. gram.:f.}
\end{itemize}
Inflammação na vagina.
\section{Vagino-labial}
\begin{itemize}
\item {Grp. gram.:adj.}
\end{itemize}
\begin{itemize}
\item {Utilização:Anat.}
\end{itemize}
Relativo á vagina e aos seus lábios.
\section{Vagino-peritoneal}
\begin{itemize}
\item {Grp. gram.:adj.}
\end{itemize}
\begin{itemize}
\item {Utilização:Anat.}
\end{itemize}
Relativo á vagina e ao seu peritoneu.
\section{Vagino-rectal}
\begin{itemize}
\item {Grp. gram.:adj.}
\end{itemize}
\begin{itemize}
\item {Utilização:Anat.}
\end{itemize}
Relativo á vagina e ao recto.
\section{Vaginoscopia}
\begin{itemize}
\item {Grp. gram.:f.}
\end{itemize}
\begin{itemize}
\item {Proveniência:(Do lat. \textunderscore vagina\textunderscore  + gr. \textunderscore skopein\textunderscore )}
\end{itemize}
Observação médica da vagina.
\section{Vagino-vesical}
\begin{itemize}
\item {Grp. gram.:adj.}
\end{itemize}
\begin{itemize}
\item {Utilização:Anat.}
\end{itemize}
Relativo á vagina e á bexiga.
\section{Vagínula}
\begin{itemize}
\item {Grp. gram.:f.}
\end{itemize}
\begin{itemize}
\item {Proveniência:(Lat. \textunderscore vaginula\textunderscore )}
\end{itemize}
Baínha pequena.
Corolla tubulosa.
\section{Vaginulado}
\begin{itemize}
\item {Grp. gram.:adj.}
\end{itemize}
Que tem vagínula.
\section{Vagir}
\begin{itemize}
\item {Grp. gram.:v. i.}
\end{itemize}
\begin{itemize}
\item {Utilização:Fig.}
\end{itemize}
\begin{itemize}
\item {Grp. gram.:M.}
\end{itemize}
\begin{itemize}
\item {Proveniência:(Lat. \textunderscore vagire\textunderscore )}
\end{itemize}
Dar vagidos (a criança).
Gemer.
Vagido.
\section{Vagneriano}
\begin{itemize}
\item {Grp. gram.:adj.}
\end{itemize}
Relativo a Wagner.
\section{Vagnerismo}
\begin{itemize}
\item {Grp. gram.:m.}
\end{itemize}
Processo ou systema de Wagner, em Música.
\section{Vago}
\begin{itemize}
\item {Grp. gram.:adj.}
\end{itemize}
\begin{itemize}
\item {Grp. gram.:M.}
\end{itemize}
\begin{itemize}
\item {Proveniência:(Lat. \textunderscore vagus\textunderscore )}
\end{itemize}
Que vagueia.
Volúvel, inconstante.
Perplexo.
Incerto.
Indistinto.
Indeterminado.
Confuso; indeciso.
Aquillo que é indefinido ou indeciso; confusão.
\section{Vago}
\begin{itemize}
\item {Grp. gram.:adj.}
\end{itemize}
\begin{itemize}
\item {Proveniência:(Do lat. \textunderscore vacuus\textunderscore )}
\end{itemize}
Que não está preenchido.
Desoccupado.
Deshabitado.
Que não pertence determinadamente a ninguém, ou para que não há herdeiro determinado, (falando-se de certos bens ou heranças).
\section{Vago}
\begin{itemize}
\item {Grp. gram.:m.}
\end{itemize}
Nome que se dá ao tigre, na Índia Portuguesa.
\section{Vagoneiro}
\begin{itemize}
\item {Grp. gram.:m.}
\end{itemize}
\begin{itemize}
\item {Utilização:Pop.}
\end{itemize}
Conductor de vagão.
\section{Vagonete}
\begin{itemize}
\item {fónica:nê}
\end{itemize}
\begin{itemize}
\item {Grp. gram.:m.}
\end{itemize}
Pequeno vagão.
\section{Vagruco}
\begin{itemize}
\item {Grp. gram.:m.}
\end{itemize}
Árvore da Índia Portuguesa.
\section{Vagueação}
\begin{itemize}
\item {Grp. gram.:f.}
\end{itemize}
Acto ou effeito de vaguear^1.
Vadiagem.
Peregrinação.
O mesmo que \textunderscore devaneio\textunderscore . Cf. \textunderscore Luz e Calor\textunderscore , 303.
\section{Vaguear}
\begin{itemize}
\item {Grp. gram.:v. i}
\end{itemize}
\begin{itemize}
\item {Grp. gram.:V. t.}
\end{itemize}
\begin{itemize}
\item {Utilização:Des.}
\end{itemize}
\begin{itemize}
\item {Proveniência:(De \textunderscore vago\textunderscore ^1)}
\end{itemize}
Andar ao acaso, de uma parte para outra.
Vagar; vagabundear.
Têr vida ociosa.
Devanear.
Mudar facilmente de posição.
Sêr inconstante.
Percorrer ao acaso.
\section{Vaguear}
\begin{itemize}
\item {Grp. gram.:v. i.}
\end{itemize}
\begin{itemize}
\item {Proveniência:(De \textunderscore vaga\textunderscore ^1)}
\end{itemize}
Andar sôbre as vagas ou á tona de água; fluctuar.
\section{Vagueira}
\begin{itemize}
\item {Grp. gram.:f.}
\end{itemize}
\begin{itemize}
\item {Utilização:Prov.}
\end{itemize}
\begin{itemize}
\item {Proveniência:(De \textunderscore vago\textunderscore ^2)}
\end{itemize}
Intervallo, falha.
\section{Vagueiro}
\begin{itemize}
\item {Grp. gram.:m.}
\end{itemize}
\begin{itemize}
\item {Proveniência:(De \textunderscore vago\textunderscore ^2)}
\end{itemize}
Pedaço de terra calva, em que não houve plantações.
\section{Vaguejar}
\begin{itemize}
\item {Grp. gram.:v. i.}
\end{itemize}
O mesmo que \textunderscore vaguear\textunderscore ^1.
\section{Vagueza}
\begin{itemize}
\item {Grp. gram.:f.}
\end{itemize}
\begin{itemize}
\item {Proveniência:(It. \textunderscore vaghezza\textunderscore )}
\end{itemize}
É, em Pintura, a ligeireza e finura da tinta, suave e docemente distribuída.
\section{Vaguidade}
\begin{itemize}
\item {fónica:gu-i}
\end{itemize}
\begin{itemize}
\item {Grp. gram.:f.}
\end{itemize}
Qualidade de vago; o mesmo que \textunderscore vacuidade\textunderscore . Cf. R. Jorge, \textunderscore El Greco\textunderscore , 26.
\section{Vaia}
\begin{itemize}
\item {Grp. gram.:f.}
\end{itemize}
Motejo; apupo; zombaria.
Matraca.
(Cast. \textunderscore vaya\textunderscore )
\section{Vaiador}
\begin{itemize}
\item {Grp. gram.:adj.}
\end{itemize}
\begin{itemize}
\item {Utilização:Neol.}
\end{itemize}
\begin{itemize}
\item {Proveniência:(De \textunderscore vaiar\textunderscore )}
\end{itemize}
Que dá vaias; que faz assuada.
\section{Vaiar}
\begin{itemize}
\item {Grp. gram.:v. i.}
\end{itemize}
Dar vaias; apupar.
Zombar.
\section{Vaidade}
\begin{itemize}
\item {Grp. gram.:f.}
\end{itemize}
\begin{itemize}
\item {Proveniência:(Do lat. \textunderscore vanitas\textunderscore )}
\end{itemize}
Qualidade do que é vão, instável ou sem duração.
Qualidade do que não tem firmeza ou solidez.
Desejo exaggerado ou injustificado de attrahir a admiração ou as homenagens dos outros.
Ostentação; presumpção.
Futilidade.
\section{Vaidosamente}
\begin{itemize}
\item {Grp. gram.:adj.}
\end{itemize}
\begin{itemize}
\item {Proveniência:(De \textunderscore vaidoso\textunderscore )}
\end{itemize}
Com vaidade.
\section{Vaidoso}
\begin{itemize}
\item {Grp. gram.:adj.}
\end{itemize}
Que tem vaidade; presumido, presumpçoso.
Jactancioso.
\section{Vaija}
\begin{itemize}
\item {Grp. gram.:f.}
\end{itemize}
\begin{itemize}
\item {Utilização:Prov.}
\end{itemize}
O mesmo que \textunderscore vagem\textunderscore .
\section{Vaillântia}
\begin{itemize}
\item {fónica:valhan}
\end{itemize}
\begin{itemize}
\item {Grp. gram.:f.}
\end{itemize}
\begin{itemize}
\item {Proveniência:(De \textunderscore Vaillant\textunderscore , n. p.)}
\end{itemize}
Gênero de plantas rubiáceas.
\section{Vai-não-vai}
\begin{itemize}
\item {Grp. gram.:loc. adv.}
\end{itemize}
\begin{itemize}
\item {Grp. gram.:M.}
\end{itemize}
Por um triz.
Momento, instante, santiâmen: \textunderscore desappareceu num vai-não-vai\textunderscore .
\section{Vai-na-villa}
\begin{itemize}
\item {Grp. gram.:m.}
\end{itemize}
\begin{itemize}
\item {Utilização:Bras}
\end{itemize}
Árvore silvestre.
\section{Vaio}
\begin{itemize}
\item {Grp. gram.:m.}
\end{itemize}
\begin{itemize}
\item {Utilização:Des.}
\end{itemize}
Cavallo baio. Cf. \textunderscore Viriato Trág.\textunderscore , XI, 69.
(Alter. de \textunderscore baio\textunderscore )
\section{Vaiqueno}
\begin{itemize}
\item {Grp. gram.:m.}
\end{itemize}
Língua, falada nos reinos da Servião, em Timor.
\section{Vai-te-a-êlle}
\begin{itemize}
\item {Grp. gram.:m.}
\end{itemize}
Jôgo de rapazes, em que uns andam em seguimento dos outros.
\section{Vaivém}
\begin{itemize}
\item {Grp. gram.:m.}
\end{itemize}
\begin{itemize}
\item {Utilização:Fig.}
\end{itemize}
\begin{itemize}
\item {Proveniência:(De \textunderscore ir\textunderscore  + \textunderscore vir\textunderscore )}
\end{itemize}
Antiga máquina de guerra, para desmoronar muralhas ou arrombar as portas das fortificações.
Aríete.
Pancada dessa máquina de guerra.
Movimento oscillatório.
Movimento de objecto que vai e vem.
Balanço, com que se impelle alguma coisa.
Alternativa.
Vicissitude.
Capricho da fortuna: \textunderscore os vaivens da sorte\textunderscore .
\section{Vaixa}
\begin{itemize}
\item {Grp. gram.:m.}
\end{itemize}
\begin{itemize}
\item {Proveniência:(Do sânscr. e conc. \textunderscore vaixia\textunderscore )}
\end{itemize}
Agricultor, homem da terceira casta indiana, segundo a organização brahmânica.
\section{Vaja}
\begin{itemize}
\item {Grp. gram.:f.}
\end{itemize}
\begin{itemize}
\item {Utilização:Prov.}
\end{itemize}
O mesmo que \textunderscore vagem\textunderscore .
\section{Val}
\begin{itemize}
\item {Grp. gram.:m.}
\end{itemize}
(V.valle)
\section{Vala-buá}
\begin{itemize}
\item {Grp. gram.:f.}
\end{itemize}
Arvoreta medicinal da ilha de San-Thomé.
\section{Valáchio}
\begin{itemize}
\item {fónica:qui}
\end{itemize}
\begin{itemize}
\item {Grp. gram.:adj.}
\end{itemize}
\begin{itemize}
\item {Grp. gram.:M.}
\end{itemize}
Relativo á Valáchia.
Habitante da Valáchia.
Um dos cinco ramos principaes das línguas novi-latinas, o mesmo que \textunderscore romeno\textunderscore .
\section{Vala-plé}
\begin{itemize}
\item {Grp. gram.:f.}
\end{itemize}
Arbusto medicinal da ilha de San-Thomé.
(Corr. de \textunderscore vara da praia\textunderscore , no dialecto santhomense)
\section{Valáquio}
\begin{itemize}
\item {Grp. gram.:adj.}
\end{itemize}
\begin{itemize}
\item {Grp. gram.:M.}
\end{itemize}
Relativo á Valáchia.
Habitante da Valáchia.
Um dos cinco ramos principaes das línguas novi-latinas, o mesmo que \textunderscore romeno\textunderscore .
\section{Valar}
\begin{itemize}
\item {Grp. gram.:v. i.}
\end{itemize}
\begin{itemize}
\item {Utilização:Pesc.}
\end{itemize}
\begin{itemize}
\item {Proveniência:(De \textunderscore valo\textunderscore )}
\end{itemize}
Espantar os peixes com varas, para irem de encontro ás rêdes de emmalhar.
\section{Valboeiro}
\begin{itemize}
\item {Grp. gram.:adj.}
\end{itemize}
Relativo a Valbom.
\textunderscore Lanchas valboeiras\textunderscore , lanchas parecidas ás dos pescadores de Valbom.
\section{Valdeiro}
\begin{itemize}
\item {Grp. gram.:adj.}
\end{itemize}
Próprio de vadio ou tunante.
Relativo a valdo.
\section{Valdense}
\begin{itemize}
\item {Grp. gram.:adj.}
\end{itemize}
\begin{itemize}
\item {Grp. gram.:M.}
\end{itemize}
Relativo ao cantão de Vaud.
Diz-se de uma espécie de terreno mesozóico.
Dialecto do cantão de Vaud.
(B. lat. \textunderscore valdensis\textunderscore )
\section{Valdense}
\begin{itemize}
\item {Grp. gram.:m.}
\end{itemize}
\begin{itemize}
\item {Proveniência:(De \textunderscore Valdo\textunderscore , n. p.)}
\end{itemize}
Membro de uma seita religiosa do século XII, seita conhecida também pelo nome de \textunderscore Pobres de Lyão\textunderscore .
\section{Valdevinos}
\begin{itemize}
\item {Grp. gram.:m.}
\end{itemize}
Vadio; estroina.
Pobretão.
Traficante.
(Corr. de \textunderscore Balduino\textunderscore , n. p.?)
\section{Valdismo}
\begin{itemize}
\item {Grp. gram.:m.}
\end{itemize}
\begin{itemize}
\item {Proveniência:(De \textunderscore Valdo\textunderscore , n. p.)}
\end{itemize}
Seita dos Valdenses.
\section{Valdo}
\begin{itemize}
\item {Grp. gram.:m.}
\end{itemize}
\begin{itemize}
\item {Utilização:Ant.}
\end{itemize}
O mesmo que \textunderscore valdevinos\textunderscore .
\section{Valdurão}
\begin{itemize}
\item {Grp. gram.:m.}
\end{itemize}
Casta de uva.
\section{Vale}
\begin{itemize}
\item {Grp. gram.:m.}
\end{itemize}
\begin{itemize}
\item {Proveniência:(De \textunderscore valer\textunderscore )}
\end{itemize}
Documento, representativo de dinheiro, e passado a favor de alguém, sem formalidades legaes.
Espécie de cheque.
Espécie de letra de câmbio, com que se transferem fundos, de uma terra para outra.
\section{Valedio}
\begin{itemize}
\item {Grp. gram.:adj.}
\end{itemize}
\begin{itemize}
\item {Proveniência:(De \textunderscore valer\textunderscore )}
\end{itemize}
Que tem valor, que póde têr curso, (falando-se de moédas).
\section{Valedoiro}
\begin{itemize}
\item {Grp. gram.:adj.}
\end{itemize}
O mesmo que \textunderscore valioso\textunderscore .
O mesmo que \textunderscore valedor\textunderscore .
\section{Valedor}
\begin{itemize}
\item {Grp. gram.:m.  e  adj.}
\end{itemize}
O que vale ou dá auxílio ou protecção a alguém.
\section{Valença}
\begin{itemize}
\item {Grp. gram.:f.}
\end{itemize}
\begin{itemize}
\item {Utilização:Ant.}
\end{itemize}
\begin{itemize}
\item {Proveniência:(De \textunderscore valer\textunderscore )}
\end{itemize}
Fortaleza, poder. Cf. S. R. Viterbo, \textunderscore Elucidário\textunderscore .
\section{Valência}
\begin{itemize}
\item {Grp. gram.:f.}
\end{itemize}
\begin{itemize}
\item {Utilização:Chím.}
\end{itemize}
\begin{itemize}
\item {Proveniência:(De \textunderscore valer\textunderscore )}
\end{itemize}
Capacidade de saturação de um corpo, na formação de um composto que se observa.
\section{Valenciana}
\begin{itemize}
\item {Grp. gram.:f.}
\end{itemize}
Renda francesa, fabricada em Valenciennes.
\section{Valenciana}
\begin{itemize}
\item {Grp. gram.:f.}
\end{itemize}
\begin{itemize}
\item {Utilização:Pesc.}
\end{itemize}
\begin{itemize}
\item {Proveniência:(De \textunderscore Valência\textunderscore , n. p.)}
\end{itemize}
Systema de armação fixa de pesca.
\section{Valencianite}
\begin{itemize}
\item {Grp. gram.:f.}
\end{itemize}
\begin{itemize}
\item {Utilização:Miner.}
\end{itemize}
Variedade de feldspatho.
\section{Valencianito}
\begin{itemize}
\item {Grp. gram.:f.}
\end{itemize}
\begin{itemize}
\item {Utilização:Miner.}
\end{itemize}
Variedade de feldspatho.
\section{Valenciano}
\begin{itemize}
\item {Grp. gram.:m.}
\end{itemize}
\begin{itemize}
\item {Proveniência:(De \textunderscore Valência\textunderscore , n. p.)}
\end{itemize}
Casta de uva trasmontana.
\section{Valencina}
\begin{itemize}
\item {Grp. gram.:f.}
\end{itemize}
\begin{itemize}
\item {Utilização:Ant.}
\end{itemize}
Pano de lan fina, fabricado em Valência:«\textunderscore ...sáia de valencina.\textunderscore »Herculano, \textunderscore Lendas\textunderscore , I, 96.--Algumas vezes se escreveu \textunderscore valancina\textunderscore . Cf. S. R. Viterbo, \textunderscore Elucidário\textunderscore .
\section{Valentaço}
\begin{itemize}
\item {Grp. gram.:m.}
\end{itemize}
\begin{itemize}
\item {Utilização:Fam.}
\end{itemize}
O mesmo que \textunderscore valentão\textunderscore .
\section{Valentão}
\begin{itemize}
\item {Grp. gram.:m.  e  adj.}
\end{itemize}
O que é muito valente; fanfarrão; gabarola.
\section{Valente}
\begin{itemize}
\item {Grp. gram.:adj.}
\end{itemize}
\begin{itemize}
\item {Grp. gram.:M.}
\end{itemize}
\begin{itemize}
\item {Utilização:Gír.}
\end{itemize}
\begin{itemize}
\item {Proveniência:(Lat. \textunderscore valens\textunderscore )}
\end{itemize}
Que tem valor.
Que não tem mêdo; intrépido.
Enérgico.
Rijo; resistente: \textunderscore pau valente\textunderscore .
Homem esforçado, homem corajoso.
Paladino, campeão.
Pequena alavanca de ferro.
\section{Valentemente}
\begin{itemize}
\item {Grp. gram.:f.}
\end{itemize}
\begin{itemize}
\item {Proveniência:(De \textunderscore valente\textunderscore )}
\end{itemize}
Com valentia.
\section{Valentia}
\begin{itemize}
\item {Grp. gram.:f.}
\end{itemize}
Qualidade do que é valente.
Acto próprio de valente.
Façanha.
Fôrça; vigor.
Qualidade daquillo que é resistente.
\section{Valentice}
\begin{itemize}
\item {Grp. gram.:f.}
\end{itemize}
\begin{itemize}
\item {Utilização:Deprec.}
\end{itemize}
O mesmo que \textunderscore valentia\textunderscore .
\section{Valentínia}
\begin{itemize}
\item {Grp. gram.:f.}
\end{itemize}
\begin{itemize}
\item {Proveniência:(De \textunderscore Valentim\textunderscore , n. p.)}
\end{itemize}
Gênero de plantas rhamnáceas.
\section{Valentinianismo}
\begin{itemize}
\item {Grp. gram.:m.}
\end{itemize}
Doutrina ou heresia dos Valentinianos.
\section{Valentinianos}
\begin{itemize}
\item {Grp. gram.:m. pl.}
\end{itemize}
\begin{itemize}
\item {Proveniência:(De \textunderscore Valentim\textunderscore , n. p.)}
\end{itemize}
Herejes, que sustentavam a existência de dois mundos, um visível e outro invisível.
\section{Valentona}
\begin{itemize}
\item {Grp. gram.:f.  e  adj.}
\end{itemize}
\begin{itemize}
\item {Grp. gram.:Loc. adv.}
\end{itemize}
\begin{itemize}
\item {Proveniência:(De \textunderscore valentão\textunderscore )}
\end{itemize}
Mulhér valente.
\textunderscore Á valentona\textunderscore , violentamente; brutalmente.
\section{Valequecér}
\begin{itemize}
\item {Grp. gram.:m.}
\end{itemize}
\begin{itemize}
\item {Utilização:Ant.}
\end{itemize}
Imposto, o mesmo que \textunderscore mandovim\textunderscore .
\section{Valer}
\begin{itemize}
\item {Grp. gram.:v. i.}
\end{itemize}
\begin{itemize}
\item {Grp. gram.:V. t.}
\end{itemize}
\begin{itemize}
\item {Grp. gram.:V. p.}
\end{itemize}
\begin{itemize}
\item {Proveniência:(Do lat. hyp. \textunderscore valuere\textunderscore )}
\end{itemize}
Têr valor.
Têr applicação ou mérito.
Exercer influência.
Ostentar importância.
Sêr de certo preço.
Dar proveito.
Dar protecção ou soccorro; acudir: \textunderscore valeu-lhe o tio\textunderscore .
Sêr igual em valor a.
Significar.
Obter, grangear.
Utilizar-se.
Têr valor ou coragem.
\section{Valerato}
\begin{itemize}
\item {Grp. gram.:m.}
\end{itemize}
O mesmo ou melhór que \textunderscore valerianato\textunderscore .
\section{Valerena}
\begin{itemize}
\item {Grp. gram.:f.}
\end{itemize}
Essência da valeriana. Cf. \textunderscore Pharmacopeia Port.\textunderscore 
\section{Valeriana}
\begin{itemize}
\item {Grp. gram.:f.}
\end{itemize}
Gênero de plantas, em que se distingue a valeriana medicinal.
(Cast. \textunderscore valeriana\textunderscore )
\section{Valerianáceas}
\begin{itemize}
\item {Grp. gram.:f. pl.}
\end{itemize}
Família de plantas, que tem por typo a valeriana.
(Fem. pl. de \textunderscore valerianáceo\textunderscore )
\section{Valerianáceo}
\begin{itemize}
\item {Grp. gram.:adj.}
\end{itemize}
Relativo ou semelhante á valeriana.
\section{Valerianato}
\begin{itemize}
\item {Grp. gram.:m.}
\end{itemize}
\begin{itemize}
\item {Proveniência:(De \textunderscore valeriana\textunderscore )}
\end{itemize}
Sal, resultante da combinação do ácido valeriânico com uma base.
\section{Valeriâneas}
\begin{itemize}
\item {Grp. gram.:f. pl.}
\end{itemize}
(V.valerianáceas)
\section{Valerianela}
\begin{itemize}
\item {Grp. gram.:f.}
\end{itemize}
\begin{itemize}
\item {Proveniência:(De \textunderscore valeriana\textunderscore )}
\end{itemize}
Gênero de plantas valerianáceas.
\section{Valeriânico}
\begin{itemize}
\item {Grp. gram.:adj.}
\end{itemize}
Diz-se de um ácido, extrahido da valeriana.
\section{Valérico}
\begin{itemize}
\item {Grp. gram.:adj.}
\end{itemize}
Diz-se de um ácido, pertencente á série dos ácidos gordos.
\section{Valerobromina}
\begin{itemize}
\item {Grp. gram.:f.}
\end{itemize}
Medicamento sedativo.
\section{Valerol}
\begin{itemize}
\item {Grp. gram.:m.}
\end{itemize}
Substância chímica (C^6H^{16}O) da valeriana. Cf. \textunderscore Pharmacopeia Port.\textunderscore 
\section{Valeroso}
\begin{itemize}
\item {Grp. gram.:adj.}
\end{itemize}
\begin{itemize}
\item {Utilização:Des.}
\end{itemize}
\begin{itemize}
\item {Proveniência:(De \textunderscore valer\textunderscore )}
\end{itemize}
O mesmo ou melhór que \textunderscore valoroso\textunderscore .
\section{Valésia}
\begin{itemize}
\item {Grp. gram.:f.}
\end{itemize}
Gênero de plantas apocýneas.
\section{Valete}
\begin{itemize}
\item {Grp. gram.:m.}
\end{itemize}
\begin{itemize}
\item {Proveniência:(Fr. \textunderscore valet\textunderscore )}
\end{itemize}
Figura das cartas de jogar, também chamada \textunderscore conde\textunderscore .
\section{Valetudinário}
\begin{itemize}
\item {Grp. gram.:adj.}
\end{itemize}
\begin{itemize}
\item {Proveniência:(Lat. \textunderscore valetudinarius\textunderscore )}
\end{itemize}
Enfermiço; que está habitualmente doente; que tem compleição fraca.
\section{Válgio}
\begin{itemize}
\item {Grp. gram.:m.}
\end{itemize}
\begin{itemize}
\item {Proveniência:(Lat. \textunderscore valgium\textunderscore )}
\end{itemize}
Instrumento rústico, com que os antigos aplanavam o terreno para a formação das eiras.
\section{Valgo}
\begin{itemize}
\item {Grp. gram.:adj.}
\end{itemize}
\begin{itemize}
\item {Utilização:Med.}
\end{itemize}
\begin{itemize}
\item {Proveniência:(Lat. \textunderscore valgus\textunderscore )}
\end{itemize}
Diz-se de um membro, ou segmento de um membro, voltado para fóra.
\section{Valhacoito}
\begin{itemize}
\item {Grp. gram.:m.}
\end{itemize}
\begin{itemize}
\item {Proveniência:(De \textunderscore valer\textunderscore  + \textunderscore coito\textunderscore )}
\end{itemize}
Abrigo; asylo.
Protecção.
\section{Valher}
\begin{itemize}
\item {Grp. gram.:v. t.  e  i.}
\end{itemize}
\begin{itemize}
\item {Utilização:Ant.}
\end{itemize}
O mesmo que \textunderscore valer\textunderscore .
\section{Váli}
\begin{itemize}
\item {Grp. gram.:m.}
\end{itemize}
\begin{itemize}
\item {Proveniência:(Do ár. \textunderscore uali\textunderscore )}
\end{itemize}
Nome, que se dava aos governadores árabes, ás vezes independentes, de territórios da Espanha.
\section{Valia}
\begin{itemize}
\item {Grp. gram.:f.}
\end{itemize}
\begin{itemize}
\item {Utilização:Prov.}
\end{itemize}
\begin{itemize}
\item {Utilização:minh.}
\end{itemize}
\begin{itemize}
\item {Proveniência:(De \textunderscore valer\textunderscore )}
\end{itemize}
Valor inherente a um objecto.
Valor intrínseco.
Valor estimativo.
Valor.
Mérito.
Preço; valimento.
\textunderscore Época das valias\textunderscore , o tempo em que o gado tem mais procura e sóbe de preço.
\section{Valiamento}
\begin{itemize}
\item {Grp. gram.:m.}
\end{itemize}
\begin{itemize}
\item {Utilização:Des.}
\end{itemize}
\begin{itemize}
\item {Proveniência:(De \textunderscore valia\textunderscore )}
\end{itemize}
O mesmo que \textunderscore avaliação\textunderscore .
\section{Validação}
\begin{itemize}
\item {Grp. gram.:f.}
\end{itemize}
Acto ou effeito de validar.
\section{Validade}
\begin{itemize}
\item {Grp. gram.:f.}
\end{itemize}
\begin{itemize}
\item {Proveniência:(Do lat. \textunderscore validitas\textunderscore )}
\end{itemize}
Qualidade do que é válido.
\section{Validamente}
\begin{itemize}
\item {Grp. gram.:adv.}
\end{itemize}
De modo válido.
Com validade.
Nos termos legaes.
\section{Validar}
\begin{itemize}
\item {Grp. gram.:v. t.}
\end{itemize}
\begin{itemize}
\item {Proveniência:(Lat. \textunderscore validare\textunderscore )}
\end{itemize}
Tornar válido.
Tornar legítimo ou legal.
\section{Validez}
\begin{itemize}
\item {Grp. gram.:f.}
\end{itemize}
Estado ou qualidade de válido.
\section{Válido}
\begin{itemize}
\item {Grp. gram.:adj.}
\end{itemize}
\begin{itemize}
\item {Proveniência:(Lat. \textunderscore validus\textunderscore )}
\end{itemize}
Que tem valor.
Que tem saúde; são; vigoroso.
Que tem effeito, legal: \textunderscore contracto válido\textunderscore .
Efficaz.
\section{Valído}
\begin{itemize}
\item {Grp. gram.:adj.}
\end{itemize}
\begin{itemize}
\item {Grp. gram.:M.}
\end{itemize}
Particularmente estimado.
Indivíduo especialmente protegido; favorito.
\section{Validol}
\begin{itemize}
\item {Grp. gram.:m.}
\end{itemize}
\begin{itemize}
\item {Utilização:Pharm.}
\end{itemize}
Valerianato de mentol.
\section{Valimento}
\begin{itemize}
\item {Grp. gram.:m.}
\end{itemize}
Acto ou effeito de valer.
Valor.
Préstimo.
Influência, importância.
Privança.
Intercessão.
\section{Valiosamente}
\begin{itemize}
\item {Grp. gram.:adv.}
\end{itemize}
De modo valioso.
Validamente.
Com muitos merecimentos.
\section{Vala}
\begin{itemize}
\item {Grp. gram.:f.}
\end{itemize}
\begin{itemize}
\item {Proveniência:(Lat. \textunderscore valla\textunderscore , pl. de \textunderscore vallum\textunderscore )}
\end{itemize}
Escavação longa e mais ou menos larga, para receber as águas que escorrem do terreno adjacente ou para as levar ao ponto onde podem sêr utilizadas.
Sepultura, em que se reúnem os cadáveres de indivíduos que não deixaram meios para se pagar uma cova separada.
\section{Valada}
\begin{itemize}
\item {Grp. gram.:f.}
\end{itemize}
Grande vale. Cf. Pant. de Aveiro, \textunderscore Itiner.\textunderscore , 189, (2.^a ed.).
\section{Valadio}
\begin{itemize}
\item {Grp. gram.:adj.}
\end{itemize}
\begin{itemize}
\item {Proveniência:(De \textunderscore valar\textunderscore )}
\end{itemize}
Diz-se do terreno, em que há valas para receberem a água.
Diz-se do telhado, feito de telhas soltas, sem cal nem argamassa.
\section{Valado}
\begin{itemize}
\item {Grp. gram.:m.}
\end{itemize}
Valla, ladeada de tapume ou sebe, para resguardo ou defesa de uma propriedade rústica.
Propriedade rústica, cercada de valado.
Elevação de terra, que limita e rodeia uma propriedade rústica.
\section{Valador}
\begin{itemize}
\item {Grp. gram.:m.  e  adj.}
\end{itemize}
\begin{itemize}
\item {Proveniência:(De \textunderscore valar\textunderscore )}
\end{itemize}
O que trabalha em valas ou valados.
\section{Valagem}
\begin{itemize}
\item {Grp. gram.:f.}
\end{itemize}
Acto de valar ou murar. Cf. \textunderscore Museu Technol.\textunderscore , 78.
\section{Valão}
\begin{itemize}
\item {Grp. gram.:m.}
\end{itemize}
\begin{itemize}
\item {Proveniência:(Fr. \textunderscore wallon\textunderscore )}
\end{itemize}
Dialecto francês, que se fala na Bélgica.
\section{Valar}
\begin{itemize}
\item {Grp. gram.:v. t.}
\end{itemize}
\begin{itemize}
\item {Utilização:Fig.}
\end{itemize}
\begin{itemize}
\item {Proveniência:(Lat. \textunderscore vallare\textunderscore )}
\end{itemize}
Fazer valas em.
Cercar de valas.
Abrir fossos em volta de.
Murar; fortificar; defender.
\section{Valar}
\begin{itemize}
\item {Grp. gram.:adj.}
\end{itemize}
\begin{itemize}
\item {Proveniência:(Lat. \textunderscore vallaris\textunderscore )}
\end{itemize}
Relativo a vala ou cêrca.
\section{Valáride}
\begin{itemize}
\item {Grp. gram.:m.}
\end{itemize}
Gênero de plantas apocíneas.
\section{Vale}
\begin{itemize}
\item {Grp. gram.:m.}
\end{itemize}
\begin{itemize}
\item {Utilização:Fig.}
\end{itemize}
\begin{itemize}
\item {Proveniência:(Lat. \textunderscore vallis\textunderscore )}
\end{itemize}
Planície entre montanhas ou na base de uma montanha.
Depressão de terreno, que se estende entre montes.
Várzea ou planície, á beira de um rio.
Talvegue.
\textunderscore Vale de lágrimas\textunderscore , o mundo, considerado como estância de sofrimentos.
\section{Valea}
\begin{itemize}
\item {Grp. gram.:f.}
\end{itemize}
Gênero de plantas tiliáceas.
\section{Valécula}
\begin{itemize}
\item {Grp. gram.:f.}
\end{itemize}
\begin{itemize}
\item {Utilização:Bot.}
\end{itemize}
\begin{itemize}
\item {Proveniência:(De \textunderscore vale\textunderscore )}
\end{itemize}
Pequena depressão, produzida pela saliência dos lados do pericarpo das umbelíferas.
\section{Valeira}
\begin{itemize}
\item {Grp. gram.:f.}
\end{itemize}
O mesmo que \textunderscore valeiro\textunderscore ^2. Cf. Júl. Castilho, \textunderscore Prim. Versos\textunderscore , 33.
\section{Valeira}
\begin{itemize}
\item {Grp. gram.:f.}
\end{itemize}
Vala pequena.
\section{Valeiro}
\begin{itemize}
\item {Grp. gram.:m.}
\end{itemize}
O mesmo que \textunderscore valeta\textunderscore .
\section{Valeiro}
\begin{itemize}
\item {Grp. gram.:m.}
\end{itemize}
\begin{itemize}
\item {Utilização:Prov.}
\end{itemize}
Vale pequeno; terreno deprimido e arborizado.
\section{Valejo}
\begin{itemize}
\item {Grp. gram.:m.}
\end{itemize}
\begin{itemize}
\item {Utilização:Des.}
\end{itemize}
Vale pequeno.
\section{Valeta}
\begin{itemize}
\item {fónica:lê}
\end{itemize}
\begin{itemize}
\item {Grp. gram.:f.}
\end{itemize}
Pequena vala para escoamento de águas, á beira de ruas ou estradas.
\section{Valigoto}
\begin{itemize}
\item {fónica:gô}
\end{itemize}
\begin{itemize}
\item {Grp. gram.:m.}
\end{itemize}
\begin{itemize}
\item {Utilização:T. de Turquel}
\end{itemize}
Pequeno vale.
\section{Valioso}
\begin{itemize}
\item {Grp. gram.:adj.}
\end{itemize}
\begin{itemize}
\item {Proveniência:(De \textunderscore valia\textunderscore )}
\end{itemize}
Que tem valia; que vale muito.
Que tem validade.
Que é importante: \textunderscore herança valiosa\textunderscore .
Que tem muitos merecimentos.
\section{Valisnéria}
\begin{itemize}
\item {Grp. gram.:f.}
\end{itemize}
\begin{itemize}
\item {Proveniência:(De \textunderscore Vallisneri\textunderscore , n. p.)}
\end{itemize}
Gênero de plantas hydrocharídeas.
\section{Valla}
\begin{itemize}
\item {Grp. gram.:f.}
\end{itemize}
\begin{itemize}
\item {Proveniência:(Lat. \textunderscore valla\textunderscore , pl. de \textunderscore vallum\textunderscore )}
\end{itemize}
Escavação longa e mais ou menos larga, para receber as águas que escorrem do terreno adjacente ou para as levar ao ponto onde podem sêr utilizadas.
Sepultura, em que se reúnem os cadáveres de indivíduos que não deixaram meios para se pagar uma cova separada.
\section{Vallada}
\begin{itemize}
\item {Grp. gram.:f.}
\end{itemize}
Grande valle. Cf. Pant. de Aveiro, \textunderscore Itiner.\textunderscore , 189, (2.^a ed.).
\section{Valladio}
\begin{itemize}
\item {Grp. gram.:adj.}
\end{itemize}
\begin{itemize}
\item {Proveniência:(De \textunderscore vallar\textunderscore )}
\end{itemize}
Diz-se do terreno, em que há vallas para receberem a água.
Diz-se do telhado, feito de telhas soltas, sem cal nem argamassa.
\section{Vallado}
\begin{itemize}
\item {Grp. gram.:m.}
\end{itemize}
Valla, ladeada de tapume ou sebe, para resguardo ou defesa de uma propriedade rústica.
Propriedade rústica, cercada de vallado.
Elevação de terra, que limita e rodeia uma propriedade rústica.
\section{Vallador}
\begin{itemize}
\item {Grp. gram.:m.  e  adj.}
\end{itemize}
\begin{itemize}
\item {Proveniência:(De \textunderscore vallar\textunderscore )}
\end{itemize}
O que trabalha em vallas ou vallados.
\section{Vallagem}
\begin{itemize}
\item {Grp. gram.:f.}
\end{itemize}
Acto de vallar ou murar. Cf. \textunderscore Museu Technol.\textunderscore , 78.
\section{Vallão}
\begin{itemize}
\item {Grp. gram.:m.}
\end{itemize}
\begin{itemize}
\item {Proveniência:(Fr. \textunderscore wallon\textunderscore )}
\end{itemize}
Dialecto francês, que se fala na Bélgica.
\section{Vallar}
\begin{itemize}
\item {Grp. gram.:v. t.}
\end{itemize}
\begin{itemize}
\item {Utilização:Fig.}
\end{itemize}
\begin{itemize}
\item {Proveniência:(Lat. \textunderscore vallare\textunderscore )}
\end{itemize}
Fazer vallas em.
Cercar de vallas.
Abrir fossos em volta de.
Murar; fortificar; defender.
\section{Vallar}
\begin{itemize}
\item {Grp. gram.:adj.}
\end{itemize}
\begin{itemize}
\item {Proveniência:(Lat. \textunderscore vallaris\textunderscore )}
\end{itemize}
Relativo a valla ou cêrca.
\section{Valláride}
\begin{itemize}
\item {Grp. gram.:m.}
\end{itemize}
Gênero de plantas apocýneas.
\section{Valle}
\begin{itemize}
\item {Grp. gram.:m.}
\end{itemize}
\begin{itemize}
\item {Utilização:Fig.}
\end{itemize}
\begin{itemize}
\item {Proveniência:(Lat. \textunderscore vallis\textunderscore )}
\end{itemize}
Planície entre montanhas ou na base de uma montanha.
Depressão de terreno, que se estende entre montes.
Várzea ou planície, á beira de um rio.
Thalvegue.
\textunderscore Valle de lágrimas\textunderscore , o mundo, considerado como estância de soffrimentos.
\section{Vállea}
\begin{itemize}
\item {Grp. gram.:f.}
\end{itemize}
Gênero de plantas tiliáceas.
\section{Vallécula}
\begin{itemize}
\item {Grp. gram.:f.}
\end{itemize}
\begin{itemize}
\item {Utilização:Bot.}
\end{itemize}
\begin{itemize}
\item {Proveniência:(De \textunderscore valle\textunderscore )}
\end{itemize}
Pequena depressão, produzida pela saliência dos lados do pericarpo das umbellíferas.
\section{Valle-de-barreiras}
\begin{itemize}
\item {Grp. gram.:m.}
\end{itemize}
Casta de uva branca, algarvia.
\section{Valleira}
\begin{itemize}
\item {Grp. gram.:f.}
\end{itemize}
O mesmo que \textunderscore valleiro\textunderscore ^2. Cf. Júl. Castilho, \textunderscore Prim. Versos\textunderscore , 33.
\section{Valleira}
\begin{itemize}
\item {Grp. gram.:f.}
\end{itemize}
Valla pequena.
\section{Valleiro}
\begin{itemize}
\item {Grp. gram.:m.}
\end{itemize}
O mesmo que \textunderscore valleta\textunderscore .
\section{Valleiro}
\begin{itemize}
\item {Grp. gram.:m.}
\end{itemize}
\begin{itemize}
\item {Utilização:Prov.}
\end{itemize}
Valle pequeno; terreno deprimido e arborizado.
\section{Vallejo}
\begin{itemize}
\item {Grp. gram.:m.}
\end{itemize}
\begin{itemize}
\item {Utilização:Des.}
\end{itemize}
Valle pequeno.
\section{Valleta}
\begin{itemize}
\item {Grp. gram.:f.}
\end{itemize}
Pequena valla para escoamento de águas, á beira de ruas ou estradas.
\section{Valligoto}
\begin{itemize}
\item {fónica:gô}
\end{itemize}
\begin{itemize}
\item {Grp. gram.:m.}
\end{itemize}
\begin{itemize}
\item {Utilização:T. de Turquel}
\end{itemize}
Pequeno valle.
\section{Vallisnéria}
\begin{itemize}
\item {Grp. gram.:f.}
\end{itemize}
\begin{itemize}
\item {Proveniência:(De \textunderscore Vallisneri\textunderscore , n. p.)}
\end{itemize}
Gênero de plantas hydrocharídeas.
\section{Vallo}
\begin{itemize}
\item {Grp. gram.:m.}
\end{itemize}
\begin{itemize}
\item {Proveniência:(Do lat. \textunderscore vallus\textunderscore )}
\end{itemize}
Parapeito, para defêsa de um campo.
Arena, liça, nas antigas justas e torneios.
Fôsso; barranco, vallado.
\section{Vallónia}
\begin{itemize}
\item {Grp. gram.:f.}
\end{itemize}
Gênero de plantas phýceas.
\section{Vallura}
\begin{itemize}
\item {Grp. gram.:f.}
\end{itemize}
\begin{itemize}
\item {Utilização:Ant.}
\end{itemize}
\begin{itemize}
\item {Proveniência:(De \textunderscore valle\textunderscore )}
\end{itemize}
Valle profundo, entre serras altíssimas.
\section{Valo}
\begin{itemize}
\item {Grp. gram.:m.}
\end{itemize}
\begin{itemize}
\item {Proveniência:(Do lat. \textunderscore vallus\textunderscore )}
\end{itemize}
Parapeito, para defêsa de um campo.
Arena, liça, nas antigas justas e torneios.
Fôsso; barranco, vallado.
\section{Valo}
\begin{itemize}
\item {Grp. gram.:m.}
\end{itemize}
\begin{itemize}
\item {Utilização:Pesc.}
\end{itemize}
Rêde de emmalhar em cêrco.
(Relaciona-se com \textunderscore vallo\textunderscore ?)
\section{Valona}
\begin{itemize}
\item {Grp. gram.:f.}
\end{itemize}
\begin{itemize}
\item {Utilização:Ant.}
\end{itemize}
\begin{itemize}
\item {Proveniência:(Do fr. \textunderscore walonne\textunderscore )}
\end{itemize}
Collarinho pendente sôbre os ombros, como o usam hoje algumas crianças.
\section{Valonado}
\begin{itemize}
\item {Grp. gram.:m.}
\end{itemize}
\begin{itemize}
\item {Utilização:Bras}
\end{itemize}
Fruto de uma variedade de carvalho, que se emprega no curtume de coiros.
\section{Valónia}
\begin{itemize}
\item {Grp. gram.:f.}
\end{itemize}
Gênero de plantas fíceas.
\section{Valor}
\begin{itemize}
\item {Grp. gram.:m.}
\end{itemize}
\begin{itemize}
\item {Grp. gram.:Pl.}
\end{itemize}
\begin{itemize}
\item {Proveniência:(Lat. \textunderscore valor\textunderscore )}
\end{itemize}
Qualidade daquelle que tem fôrça.
Valentia; coragem.
Esfôrço.
Preço.
Mérito.
Préstimo.
Papel representativo de dinheiro.
Significação precisa de um termo.
Duração de uma nota musical.
\textunderscore Valor intrínseco\textunderscore , valor real, independentemente de qualquer convenção ou arbítrio.
\textunderscore Valor extrínseco\textunderscore , valor dependente de convenção e, geralmente, superior ao intrínseco, quando os dois valores se referem ao mesmo objecto.
\textunderscore Valor estimativo\textunderscore , valor que se calcula pela estima ou aprêço em que se tem um objecto, independentemente do seu valor real.
\textunderscore Valor real\textunderscore , o valor do metal, de que se fez uma moéda, independentemente do valor que a cunhagem dá.
Grau do aproveitamento escolar de um alumno.
Termo que, junto a um número, gradua a qualificação de um exame escolar.
\section{Valorádia}
\begin{itemize}
\item {Grp. gram.:f.}
\end{itemize}
\begin{itemize}
\item {Proveniência:(De \textunderscore Valorado\textunderscore , n. p.)}
\end{itemize}
Gênero de plantas plumbagíneas.
\section{Valorização}
\begin{itemize}
\item {Grp. gram.:f.}
\end{itemize}
Acto ou effeito de valorizar.
\section{Valorizar}
\begin{itemize}
\item {Grp. gram.:v. t.}
\end{itemize}
Dar valor a.
Aumentar o valor ou o préstimo de.
\section{Valorosamente}
\begin{itemize}
\item {Grp. gram.:adv.}
\end{itemize}
De modo valoroso.
Com valor, com coragem.
\section{Valorosidade}
\begin{itemize}
\item {Grp. gram.:f.}
\end{itemize}
Qualidade do que é valoroso.
\section{Valoroso}
\begin{itemize}
\item {Grp. gram.:adj.}
\end{itemize}
\begin{itemize}
\item {Proveniência:(De \textunderscore valor\textunderscore )}
\end{itemize}
Que tem valor.
Activo, forte.
Enérgico.
Destemido.--Acham alguns artificial e contrafeita a fórma \textunderscore valoroso\textunderscore , preferindo a fórma pop. e ant. \textunderscore valeroso\textunderscore  = cast. \textunderscore valeroso\textunderscore . Cf. A. Vasconcélloz, \textunderscore Gram.\textunderscore , 257.
\section{Valoso}
\begin{itemize}
\item {Grp. gram.:adj.}
\end{itemize}
\begin{itemize}
\item {Utilização:Ant.}
\end{itemize}
\begin{itemize}
\item {Utilização:Pop.}
\end{itemize}
O mesmo que \textunderscore valioso\textunderscore .
\section{Valquíria}
\begin{itemize}
\item {Grp. gram.:f.}
\end{itemize}
\begin{itemize}
\item {Proveniência:(Do ant. al. \textunderscore walkuren\textunderscore )}
\end{itemize}
Cada uma das três nymphas ou divindades escandinavas que, pela sua formusura, incitavam os heróis em combate e serviam hydromel aos que morriam combatendo.
\section{Valsa}
\begin{itemize}
\item {Grp. gram.:f.}
\end{itemize}
\begin{itemize}
\item {Proveniência:(Fr. \textunderscore valse\textunderscore )}
\end{itemize}
Dança a três tempos moderados.
Dança a dois tempos.
Música, apropriada a essas danças.
\section{Valsador}
\begin{itemize}
\item {Grp. gram.:m.  e  adj.}
\end{itemize}
O que valsa.
\section{Valsante}
\begin{itemize}
\item {Grp. gram.:m.}
\end{itemize}
O mesmo que \textunderscore valsista\textunderscore . Cf. Filinto, VII, 279.
\section{Valsar}
\begin{itemize}
\item {Grp. gram.:v. i.}
\end{itemize}
\begin{itemize}
\item {Grp. gram.:V. t.}
\end{itemize}
Dançar valsa ou valsas.
Dançar, em andamento de valsa.
\section{Valsarina}
\begin{itemize}
\item {Grp. gram.:f.}
\end{itemize}
\begin{itemize}
\item {Utilização:P. us.}
\end{itemize}
\begin{itemize}
\item {Proveniência:(De \textunderscore valsar\textunderscore )}
\end{itemize}
Mulhér que valsa.
Dançarina. Cf. Castilho, \textunderscore Mil e Um Mist.\textunderscore , 208.
\section{Valsista}
\begin{itemize}
\item {Grp. gram.:m.   f.  e  adj.}
\end{itemize}
Pessôa, que valsa; pessôa, que valsa bem.
\section{Valuma}
\begin{itemize}
\item {Grp. gram.:f.}
\end{itemize}
\begin{itemize}
\item {Utilização:Náut.}
\end{itemize}
O mesmo que \textunderscore baluma\textunderscore .
\section{Valva}
\begin{itemize}
\item {Grp. gram.:f.}
\end{itemize}
\begin{itemize}
\item {Utilização:Bot.}
\end{itemize}
\begin{itemize}
\item {Utilização:Zool.}
\end{itemize}
\begin{itemize}
\item {Proveniência:(Lat. \textunderscore valva\textunderscore )}
\end{itemize}
Cada uma das peças de certos pericarpos.
Qualquer peça ou qualquer das peças sólidas, que revestem o corpo de um mollusco.
Concha.
\section{Valváceo}
\begin{itemize}
\item {Grp. gram.:adj.}
\end{itemize}
\begin{itemize}
\item {Utilização:Bot.}
\end{itemize}
\begin{itemize}
\item {Proveniência:(De \textunderscore valva\textunderscore )}
\end{itemize}
Diz-se do fruto que, sendo indehiscente, apresenta contudo suturas distintas.
\section{Valvar}
\begin{itemize}
\item {Grp. gram.:adj.}
\end{itemize}
\begin{itemize}
\item {Proveniência:(De \textunderscore valva\textunderscore )}
\end{itemize}
Semelhante á concha.
\section{Valverde}
\begin{itemize}
\item {fónica:vêr}
\end{itemize}
\begin{itemize}
\item {Grp. gram.:m.}
\end{itemize}
Planta ornamental, de pequenas flôres rubras.
Peça de fogo de artifício, cujas faíscas constituem proximamente uma figura pyramidal.
(Alter. de \textunderscore belverde\textunderscore . Cp. \textunderscore belverde\textunderscore )
\section{Valverde}
\begin{itemize}
\item {fónica:vêr}
\end{itemize}
\begin{itemize}
\item {Grp. gram.:m.}
\end{itemize}
\begin{itemize}
\item {Utilização:T. da Bairrada}
\end{itemize}
Balbúrdia, confusão.
Chinfrim.
\section{Valverde-ladrão}
\begin{itemize}
\item {Grp. gram.:m.}
\end{itemize}
\begin{itemize}
\item {Utilização:Prov.}
\end{itemize}
\begin{itemize}
\item {Utilização:alent.}
\end{itemize}
Espécie de dança de roda.
\section{Válvula}
\begin{itemize}
\item {Grp. gram.:f.}
\end{itemize}
\begin{itemize}
\item {Proveniência:(Lat. \textunderscore valvula\textunderscore )}
\end{itemize}
Valva pequena.
Qualquer dobra membranosa que, nos vasos sanguíneos, obsta ao refluxo dos líquidos, gradua o curso do sangue, etc.
Espécie de tampa, que fecha por si e hermeticamente um tubo.
Placa metállica que, num orifício das máquinas de vapor, evita a explosão, cedendo ao impulso do vapor que superabunda.
\section{Vã}
\begin{itemize}
\item {Grp. gram.:adj.}
\end{itemize}
(Flexão fem. de \textunderscore vão\textunderscore )
\section{Valura}
\begin{itemize}
\item {Grp. gram.:f.}
\end{itemize}
\begin{itemize}
\item {Utilização:Ant.}
\end{itemize}
\begin{itemize}
\item {Proveniência:(De \textunderscore vale\textunderscore )}
\end{itemize}
Vale profundo, entre serras altíssimas.
\section{Valvulado}
\begin{itemize}
\item {Grp. gram.:adj.}
\end{itemize}
Que tem uma válvula.
\section{Valvular}
\begin{itemize}
\item {Grp. gram.:adj.}
\end{itemize}
Que tem muitas válvulas.
\section{Valvulina}
\begin{itemize}
\item {Grp. gram.:f.}
\end{itemize}
Gênero de foraminíferos.
\section{Valvulite}
\begin{itemize}
\item {Grp. gram.:f.}
\end{itemize}
\begin{itemize}
\item {Utilização:Med.}
\end{itemize}
Inflammação das válvulas do coração.
\section{Vãmente}
\begin{itemize}
\item {Grp. gram.:adj.}
\end{itemize}
De modo vão; inutilmente, baldadamente.
\section{Vampireiro}
\begin{itemize}
\item {Grp. gram.:m.}
\end{itemize}
\begin{itemize}
\item {Utilização:Bras}
\end{itemize}
Árvore fructífera.
\section{Vampírico}
\begin{itemize}
\item {Grp. gram.:adj.}
\end{itemize}
Que tem o carácter de vampiro.
\section{Vampirismo}
\begin{itemize}
\item {Grp. gram.:m.}
\end{itemize}
\begin{itemize}
\item {Utilização:Fig.}
\end{itemize}
\begin{itemize}
\item {Proveniência:(De \textunderscore vampiro\textunderscore )}
\end{itemize}
Crença nos vampiros.
Avidez excessiva.
\section{Vampirizar}
\begin{itemize}
\item {Grp. gram.:v. t.}
\end{itemize}
\begin{itemize}
\item {Utilização:Fig.}
\end{itemize}
\begin{itemize}
\item {Proveniência:(De \textunderscore vampiro\textunderscore )}
\end{itemize}
Adquirir, explorando alguém:«\textunderscore ...as três actrizes, que mais dinheiro vampirizaram aos argentários...\textunderscore »Camillo, \textunderscore Noites de Insómn.\textunderscore , V, 5.
\section{Vampiro}
\begin{itemize}
\item {Grp. gram.:m.}
\end{itemize}
\begin{itemize}
\item {Utilização:Fig.}
\end{itemize}
\begin{itemize}
\item {Utilização:Bras}
\end{itemize}
\begin{itemize}
\item {Proveniência:(Fr. \textunderscore vampire\textunderscore )}
\end{itemize}
Entidade imaginária que, segundo a superstição popular, sai das sepulturas para sugar o sangue dos vivos.
Aquelle que enriquece á custa alheia ou por meios illícitos.
Aquelle que explora os pobres em seu proveito.
Espécie de morcego.
Fruto do vampireiro.
\section{Van}
\begin{itemize}
\item {Grp. gram.:adj.}
\end{itemize}
(Flexão fem. de \textunderscore vão\textunderscore )
\section{Vanadato}
\begin{itemize}
\item {Grp. gram.:m.}
\end{itemize}
\begin{itemize}
\item {Proveniência:(De \textunderscore vanádio\textunderscore )}
\end{itemize}
Sal, resultante da combinação do ácido vanádico com uma base.
\section{Vanádico}
\begin{itemize}
\item {Grp. gram.:adj.}
\end{itemize}
\begin{itemize}
\item {Utilização:Chím.}
\end{itemize}
Relativo ao vanádio.
Diz-se de um ácido, que é o segundo grau da sulfuração do vanádio.
Diz-se de um óxydo, que é o segundo grau da oxydação do vanádio.
Diz-se do sal, que tem por base o óxydo de vanádio.
\section{Vanádio}
\begin{itemize}
\item {Grp. gram.:m.}
\end{itemize}
\begin{itemize}
\item {Utilização:Miner.}
\end{itemize}
\begin{itemize}
\item {Proveniência:(De \textunderscore Vanadé\textunderscore , n. p.)}
\end{itemize}
Metal branco, que se encontra nas minas de ferro da Suécia, Inglaterra e México.
\section{Vanadito}
\begin{itemize}
\item {Grp. gram.:m.}
\end{itemize}
\begin{itemize}
\item {Proveniência:(De \textunderscore vanádio\textunderscore )}
\end{itemize}
Nome dos saes, em que o óxydo de vanádio representa o papel de ácido.
\section{Vanadoso}
\begin{itemize}
\item {Grp. gram.:adj.}
\end{itemize}
\begin{itemize}
\item {Utilização:Chím.}
\end{itemize}
Que contém vanádio.
\section{Vancão}
\begin{itemize}
\item {Grp. gram.:m.}
\end{itemize}
Antiga embarcação de remos, no Oriente. Cf. \textunderscore Peregrinação\textunderscore , XLIV.
\section{Vanda}
\begin{itemize}
\item {Grp. gram.:f.}
\end{itemize}
\begin{itemize}
\item {Utilização:Prov.}
\end{itemize}
\begin{itemize}
\item {Utilização:Pesc.}
\end{itemize}
\begin{itemize}
\item {Utilização:dur.}
\end{itemize}
O mesmo que \textunderscore tresmalho\textunderscore ^1, rêde.
\section{Vandalear}
\begin{itemize}
\item {Grp. gram.:v. i.}
\end{itemize}
\begin{itemize}
\item {Proveniência:(De \textunderscore vândalo\textunderscore )}
\end{itemize}
Praticar vandalismos. Cf. Filinto, III, 212.
\section{Vandálico}
\begin{itemize}
\item {Grp. gram.:adj.}
\end{itemize}
Relativo aos vândalos ou próprio dêlles.
\section{Vandalismo}
\begin{itemize}
\item {Grp. gram.:m.}
\end{itemize}
Acto próprio de vândalo.
Destruição do que é respeitável pelas suas tradições, antiguidade ou belleza.
\section{Vândalo}
\begin{itemize}
\item {Grp. gram.:m.}
\end{itemize}
\begin{itemize}
\item {Utilização:Fig.}
\end{itemize}
\begin{itemize}
\item {Grp. gram.:Adj.}
\end{itemize}
Membro de uma tríbo germânica, que dominou e devastou o Sul da Europa e o Norte da África.
Aquelle que destrói monumentos ou objectos dignos de respeito.
Inimigo das artes e das sciências.
O mesmo que \textunderscore vandálico\textunderscore .
(Cp. al. \textunderscore wandeln\textunderscore )
\section{Vandar}
\begin{itemize}
\item {Grp. gram.:v. t.}
\end{itemize}
O mesmo que \textunderscore bandar\textunderscore :«\textunderscore ...donaire, que a fronte vanda.\textunderscore »Filinto, X, 115.
\section{Vândea}
\begin{itemize}
\item {Grp. gram.:f.}
\end{itemize}
Gênero de orchídeas.
\section{Vandélia}
\begin{itemize}
\item {Grp. gram.:f.}
\end{itemize}
\begin{itemize}
\item {Proveniência:(De \textunderscore Vandelli\textunderscore , n. p.)}
\end{itemize}
Gênero de plantas escrofularíneas.
\section{Vandolear}
\begin{itemize}
\item {Grp. gram.:v. i.}
\end{itemize}
O mesmo que \textunderscore bandolear\textunderscore . Cf. Filinto, X, 13.
\section{Vanescer}
\begin{itemize}
\item {Grp. gram.:v. i.}
\end{itemize}
\begin{itemize}
\item {Proveniência:(Lat. \textunderscore vanescere\textunderscore )}
\end{itemize}
Desvanecer-se; desapparecer.
\section{Vanessa}
\begin{itemize}
\item {Grp. gram.:f.}
\end{itemize}
Espécie de borboleta diurna.
\section{Vangana}
\begin{itemize}
\item {Grp. gram.:f.}
\end{itemize}
Planta indiana.
\section{Vanglória}
\begin{itemize}
\item {Grp. gram.:f.}
\end{itemize}
\begin{itemize}
\item {Proveniência:(De \textunderscore van\textunderscore  + \textunderscore glória\textunderscore )}
\end{itemize}
Presumpção infundada.
Jactância.
Vaidade.
Ostentação; bazófia.
\section{Vangloriar}
\begin{itemize}
\item {Grp. gram.:v. t.}
\end{itemize}
\begin{itemize}
\item {Grp. gram.:V. p.}
\end{itemize}
\begin{itemize}
\item {Grp. gram.:V. i.}
\end{itemize}
Causar vanglória a.
Tornar vaidoso.
Tornar-se vaidoso.
Orgulhar-se.
Ufanar-se desmedidamente ou sem razão.
(A mesma significação). Cf. Filinto, XVIII, 175.
\section{Vangloriosamente}
\begin{itemize}
\item {Grp. gram.:adv.}
\end{itemize}
De modo vanglorioso.
Com vanglória; com jactância.
\section{Vanglorioso}
\begin{itemize}
\item {Grp. gram.:adj.}
\end{itemize}
Que tem vanglória.
Jactancioso; vaidoso.
\section{Vângor}
\begin{itemize}
\item {Grp. gram.:m.}
\end{itemize}
\begin{itemize}
\item {Utilização:T. da Índia Port}
\end{itemize}
\begin{itemize}
\item {Proveniência:(Do conc. \textunderscore vãngada\textunderscore )}
\end{itemize}
Subdivisão territorial das communidades indianas.
Agrupamento de famílias; parentela.
\section{Vanguarda}
\begin{itemize}
\item {Grp. gram.:f.}
\end{itemize}
\begin{itemize}
\item {Proveniência:(Do fr. \textunderscore avant-garde\textunderscore )}
\end{itemize}
Deanteira do exército.
Deanteira; frente.
\section{Vanguejar}
\begin{itemize}
\item {Grp. gram.:v. i.}
\end{itemize}
Escorregar; oscillar.
\section{Vanguenarau}
\begin{itemize}
\item {Grp. gram.:f.}
\end{itemize}
Espécie de prioresa das sacerdotisas de certos pagodes chineses. Cf. \textunderscore Peregrinação\textunderscore , CXXVII.
\section{Vânico}
\begin{itemize}
\item {Grp. gram.:m.}
\end{itemize}
Língua morta da Ásia.
\section{Vanidade}
\begin{itemize}
\item {Grp. gram.:f.}
\end{itemize}
\begin{itemize}
\item {Proveniência:(Do lat. \textunderscore vanitas\textunderscore )}
\end{itemize}
Vaidade; estultícia.
\section{Vanilino}
\begin{itemize}
\item {Grp. gram.:m.}
\end{itemize}
Cristaes aciculados e brilhantes do fruto da baunilha. Cf. \textunderscore Pharmacopeia Port.\textunderscore 
\section{Vanillino}
\begin{itemize}
\item {Grp. gram.:m.}
\end{itemize}
Crystaes aciculados e brilhantes do fruto da baunilha. Cf. \textunderscore Pharmacopeia Port.\textunderscore 
\section{Vaniloquência}
\begin{itemize}
\item {fónica:cu-en}
\end{itemize}
\begin{itemize}
\item {Grp. gram.:f.}
\end{itemize}
\begin{itemize}
\item {Proveniência:(Lat. \textunderscore vaniloquentia\textunderscore )}
\end{itemize}
Qualidade do que é vaníloquo.
\section{Vaniloquente}
\begin{itemize}
\item {fónica:cu-en}
\end{itemize}
\begin{itemize}
\item {Grp. gram.:adj.}
\end{itemize}
\begin{itemize}
\item {Proveniência:(Do lat. \textunderscore vaniloquens\textunderscore )}
\end{itemize}
O mesmo que \textunderscore vaníloquo\textunderscore .
\section{Vanilóquio}
\begin{itemize}
\item {Grp. gram.:m.}
\end{itemize}
\begin{itemize}
\item {Utilização:P. us.}
\end{itemize}
\begin{itemize}
\item {Proveniência:(Lat. \textunderscore vaniloquium\textunderscore )}
\end{itemize}
Palavras ôcas.
Arrazoado inútil.
\section{Vaníloquo}
\begin{itemize}
\item {Grp. gram.:adj.}
\end{itemize}
\begin{itemize}
\item {Proveniência:(Lat. \textunderscore vaniloquus\textunderscore )}
\end{itemize}
Que fala á tôa, ou diz palavras inúteis ou sem sentido.
Fanfarrão.
\section{Vaníssimo}
\begin{itemize}
\item {Grp. gram.:adj.}
\end{itemize}
\begin{itemize}
\item {Proveniência:(Do lat. \textunderscore vanus\textunderscore )}
\end{itemize}
Summamente vão; futilíssimo.
\section{Vanmente}
\begin{itemize}
\item {Grp. gram.:adj.}
\end{itemize}
De modo vão; inutilmente, baldadamente.
\section{Vantagem}
\begin{itemize}
\item {Grp. gram.:f.}
\end{itemize}
Qualidade do que está adeante ou superiormente; primazia.
Lucro, proveito.
Victória.
(Cp. fr. \textunderscore avantage\textunderscore )
\section{Vantajar}
\begin{itemize}
\item {Grp. gram.:v. t.}
\end{itemize}
Têr vantagens sôbre. Cp. Filinto, VI, 179.
\section{Vantajosamente}
\begin{itemize}
\item {Grp. gram.:adv.}
\end{itemize}
De modo vantajoso.
Com vantagem; com interesse.
\section{Vantajoso}
\begin{itemize}
\item {Grp. gram.:adj.}
\end{itemize}
Em que há vantagem.
Que dá proveito; proveitoso, útil.
\section{Vante}
\begin{itemize}
\item {Grp. gram.:f.}
\end{itemize}
\begin{itemize}
\item {Utilização:Náut.}
\end{itemize}
Deanteira (do navio)
Prôa.
Parte da coberta que fica do lado da prôa.
(Cp. \textunderscore àvante\textunderscore )
\section{Vanza}
\begin{itemize}
\item {Grp. gram.:f.}
\end{itemize}
Árvore do Congo.
\section{Vanzos}
\begin{itemize}
\item {Grp. gram.:m. pl.}
\end{itemize}
O mesmo que \textunderscore banzos\textunderscore .
\section{Vanzura}
\begin{itemize}
\item {Grp. gram.:f.}
\end{itemize}
\begin{itemize}
\item {Utilização:Prov.}
\end{itemize}
\begin{itemize}
\item {Utilização:alg.}
\end{itemize}
\begin{itemize}
\item {Proveniência:(De \textunderscore vão\textunderscore )}
\end{itemize}
O mesmo que \textunderscore vacuidade\textunderscore .
\section{Vão}
\begin{itemize}
\item {Grp. gram.:adj.}
\end{itemize}
\begin{itemize}
\item {Grp. gram.:M.}
\end{itemize}
\begin{itemize}
\item {Utilização:Bras. do N}
\end{itemize}
\begin{itemize}
\item {Grp. gram.:Loc. adv.}
\end{itemize}
\begin{itemize}
\item {Proveniência:(Do lat. \textunderscore vanus\textunderscore )}
\end{itemize}
Vazio.
Que não tem valor; fútil: \textunderscore palavras vans\textunderscore .
Que não tem effeito; \textunderscore esforços vãos\textunderscore .
Fantástico.
Destituído de talento ou de aptidões.
Vanglorioso.
Frívolo; falso.
Espaço desoccupado.
Intervallo.
Vácuo.
Abertura, formada numa parede por janela ou porta.
Tabuínhas ou cortinas, para uso de uma janela ou porta.
Região clavicular.
Ápice do pulmão.
\textunderscore Em vão\textunderscore , debalde, baldadamente, vanmente.
\section{Vapa}
\begin{itemize}
\item {Grp. gram.:f.}
\end{itemize}
\begin{itemize}
\item {Utilização:Des.}
\end{itemize}
\begin{itemize}
\item {Proveniência:(Lat. \textunderscore vappa\textunderscore )}
\end{itemize}
Vinho fraco; zurrapa.
Aguapé.
\section{Vápido}
\begin{itemize}
\item {Grp. gram.:adj.}
\end{itemize}
\begin{itemize}
\item {Utilização:Poét.}
\end{itemize}
\begin{itemize}
\item {Proveniência:(Lat. \textunderscore vapidus\textunderscore )}
\end{itemize}
O mesmo que \textunderscore insípido\textunderscore .
\section{Vapixanas}
\begin{itemize}
\item {Grp. gram.:m. pl.}
\end{itemize}
Tríbo de Índios da Guiana brasileira.
\section{Vapor}
\begin{itemize}
\item {Grp. gram.:m.}
\end{itemize}
\begin{itemize}
\item {Utilização:Fig.}
\end{itemize}
\begin{itemize}
\item {Grp. gram.:Loc. adv.}
\end{itemize}
\begin{itemize}
\item {Proveniência:(Lat. \textunderscore vapor\textunderscore )}
\end{itemize}
Fluido ou espécie de fumo, que exhalam os corpos húmidos sob a acção do calor.
Exhalação dos corpos sólidos, resultante de decomposição ou de combustão.
Fluido aeriforme, proveniente da vaporização de corpos líquidos ou sólidos e determinada pelo calor.
Navio, movido por máquina de vapor.
Perturbação, que as bebidas alcoólicas produzem no cérebro.
Modorra, entorpecimento cerebral.
\textunderscore A vapor\textunderscore , á pressa, rapidamente.
\section{Vaporação}
\begin{itemize}
\item {Grp. gram.:f.}
\end{itemize}
\begin{itemize}
\item {Proveniência:(Do lat. \textunderscore vaporatio\textunderscore )}
\end{itemize}
Acto ou effeito de vaporar.
\section{Vaporar}
\begin{itemize}
\item {Grp. gram.:v. t.}
\end{itemize}
\begin{itemize}
\item {Grp. gram.:V. i.}
\end{itemize}
\begin{itemize}
\item {Proveniência:(Lat. \textunderscore vaporare\textunderscore )}
\end{itemize}
Exhalar (vapores).
Evaporar-se.
Lançar vapores.
\section{Vaporário}
\begin{itemize}
\item {Grp. gram.:m.}
\end{itemize}
\begin{itemize}
\item {Proveniência:(Lat. \textunderscore vaporarium\textunderscore )}
\end{itemize}
Tubo ou cano, que levava para os antigos banhos romanos o calor ou vapor quente.
\section{Vaporável}
\begin{itemize}
\item {Grp. gram.:adj.}
\end{itemize}
Que se póde vaporar.
\section{Vaporífero}
\begin{itemize}
\item {Grp. gram.:adj.}
\end{itemize}
\begin{itemize}
\item {Proveniência:(Lat. \textunderscore vaporifer\textunderscore )}
\end{itemize}
Que exhala vapores.
\section{Vaporização}
\begin{itemize}
\item {Grp. gram.:f.}
\end{itemize}
Acto ou effeito de vaporizar.
\section{Vaporizador}
\begin{itemize}
\item {Grp. gram.:adj.}
\end{itemize}
\begin{itemize}
\item {Grp. gram.:M.}
\end{itemize}
Que vaporiza.
Vaso, com que se vaporiza um líquido.
\section{Vaporizar}
\begin{itemize}
\item {Grp. gram.:v. t.}
\end{itemize}
Converter em vapor.
\section{Vaporosamente}
\begin{itemize}
\item {Grp. gram.:adv.}
\end{itemize}
De modo vaporoso.
Á maneira de vapor; com transparência.
\section{Vaporoso}
\begin{itemize}
\item {Grp. gram.:adj.}
\end{itemize}
\begin{itemize}
\item {Utilização:Fig.}
\end{itemize}
\begin{itemize}
\item {Proveniência:(Lat. \textunderscore vaporosus\textunderscore )}
\end{itemize}
Em que há vapores.
Vaporífero.
Que exhala vapores.
Leve; aeriforme.
Transparente: \textunderscore traje vaporoso\textunderscore .
Muito tênue.
Muito magro.
Fantástico.
Incomprehensível.
Vaidoso.
\section{Vappa}
\begin{itemize}
\item {Grp. gram.:f.}
\end{itemize}
\begin{itemize}
\item {Utilização:Des.}
\end{itemize}
\begin{itemize}
\item {Proveniência:(Lat. \textunderscore vappa\textunderscore )}
\end{itemize}
Vinho fraco; zurrapa.
Aguapé.
\section{Vapuan}
\begin{itemize}
\item {Grp. gram.:m.}
\end{itemize}
Árvore brasileira, própria para construcções.
\section{Vapular}
\begin{itemize}
\item {Grp. gram.:v. t.}
\end{itemize}
\begin{itemize}
\item {Proveniência:(Do lat. \textunderscore vapulare\textunderscore )}
\end{itemize}
Açoitar, flagellar.
\section{Vaqueanaço}
\begin{itemize}
\item {Grp. gram.:m.}
\end{itemize}
\begin{itemize}
\item {Utilização:Bras. do S}
\end{itemize}
Bom vaqueano.
Vaqueano esforçado.
\section{Vaqueano}
\begin{itemize}
\item {Grp. gram.:m.}
\end{itemize}
\begin{itemize}
\item {Utilização:Bras}
\end{itemize}
\begin{itemize}
\item {Proveniência:(De \textunderscore vaca\textunderscore )}
\end{itemize}
Conductor, guia.
\section{Vaqueijada}
\begin{itemize}
\item {Grp. gram.:f.}
\end{itemize}
\begin{itemize}
\item {Utilização:Bras. do N}
\end{itemize}
Reunião do gado de uma fazenda, geralmente no fim do Inverno.
(Cp. \textunderscore vaca\textunderscore ^1)
\section{Vaqueijador}
\begin{itemize}
\item {Grp. gram.:m.}
\end{itemize}
\begin{itemize}
\item {Utilização:Bras. do N}
\end{itemize}
Caminho largo, por onde se conduz gado, de uma fazenda para outra.
(Cp. \textunderscore vaqueijada\textunderscore )
\section{Vaqueirama}
\begin{itemize}
\item {Grp. gram.:f.}
\end{itemize}
\begin{itemize}
\item {Utilização:Bras}
\end{itemize}
Reunião de vaqueiros.
\section{Vaqueiro}
\begin{itemize}
\item {Grp. gram.:adj.}
\end{itemize}
\begin{itemize}
\item {Grp. gram.:M.}
\end{itemize}
\begin{itemize}
\item {Utilização:Ant.}
\end{itemize}
\begin{itemize}
\item {Proveniência:(De \textunderscore vaca\textunderscore )}
\end{itemize}
Relativo a gado vacum.
Guarda ou conductor de vacas ou de gado vacum.
Traje de pastor.
Fardamento de tambores regimentaes.
\section{Vaquejar}
\begin{itemize}
\item {Grp. gram.:v. t.}
\end{itemize}
\begin{itemize}
\item {Utilização:Bras. do N}
\end{itemize}
O mesmo que \textunderscore costear\textunderscore .
O mesmo que \textunderscore perseguir\textunderscore .
\section{Vaqueta}
\begin{itemize}
\item {fónica:quê}
\end{itemize}
\begin{itemize}
\item {Grp. gram.:f.}
\end{itemize}
\begin{itemize}
\item {Proveniência:(De \textunderscore vaca\textunderscore )}
\end{itemize}
Coiro delgado para forros.
\section{Vaqueta}
\begin{itemize}
\item {fónica:quê}
\end{itemize}
\begin{itemize}
\item {Grp. gram.:f.}
\end{itemize}
Vareta de guarda-sol.
O mesmo que \textunderscore baqueta\textunderscore ^1.
\section{Vara}
\begin{itemize}
\item {Grp. gram.:f.}
\end{itemize}
\begin{itemize}
\item {Utilização:Fig.}
\end{itemize}
\begin{itemize}
\item {Proveniência:(Lat. \textunderscore vara\textunderscore )}
\end{itemize}
Ramo delgado de um arbusto ou árvore.
Cajado.
Tranca.
Viga.
Pau direito.
Báculo.
Insígnia de magistrados judiciaes e de vereadores.
Circunscripção judicial em Lisbôa e no Pôrto.
Cargo de juiz.
Jurisdicção.
Antiga medida de comprimento, equivalente a onze decimetros.
Porção de tecido, igual a essa medida, em comprimento.
Manada de porcos, para engorda.
Punição, castigo.
Vento rijo na costa de Coromandel.
Tronco de árvore, que constitue a peça principal, nas prensas antigas dos lagares.
\textunderscore Camisa de onze varas\textunderscore , a alva dos padecentes, (nos autos de fé).
Grande difficuldade ou embaraço: \textunderscore meteu-se em camisa de onze varas\textunderscore .
\textunderscore Pano de varas\textunderscore , antigo tecido de lan, de fabríco nacional, espécie de saragoça.
\section{Varação}
\begin{itemize}
\item {Grp. gram.:f.}
\end{itemize}
Acto ou effeito de varar.
Varadoiro.
\section{Varada}
\begin{itemize}
\item {Grp. gram.:f.}
\end{itemize}
\begin{itemize}
\item {Utilização:Prov.}
\end{itemize}
\begin{itemize}
\item {Utilização:minh.}
\end{itemize}
Pancada com vara.
Chibatada.
\textunderscore Ir de varada\textunderscore , ir depressa.
\section{Vara-de-canôa}
\begin{itemize}
\item {Grp. gram.:f.}
\end{itemize}
\begin{itemize}
\item {Utilização:Bras}
\end{itemize}
Espécie de mandioca.
\section{Vara-de-oiro}
\begin{itemize}
\item {Grp. gram.:f.}
\end{itemize}
Planta, da fam. das compostas, (\textunderscore solidago virga-aurea\textunderscore , Lin.).
\section{Varado}
\begin{itemize}
\item {Grp. gram.:m.}
\end{itemize}
\begin{itemize}
\item {Proveniência:(De \textunderscore vara\textunderscore )}
\end{itemize}
Uma das divisões ecclesiásticas da Índia: \textunderscore a diocese de Cochim compõe-se de quatro varados\textunderscore .
\section{Varadoiro}
\begin{itemize}
\item {Grp. gram.:m.}
\end{itemize}
\begin{itemize}
\item {Utilização:Fig.}
\end{itemize}
\begin{itemize}
\item {Proveniência:(De \textunderscore varar\textunderscore )}
\end{itemize}
Lugar, onde se fazem encalhar as embarcações, para as consertar ou para as guardar, durante o tempo em que ellas não pódem navegar.
Lugar, onde um grupo de pessôas descansa e conversa.
\section{Varador}
\begin{itemize}
\item {Grp. gram.:m.}
\end{itemize}
\begin{itemize}
\item {Proveniência:(De \textunderscore varar\textunderscore )}
\end{itemize}
Aquelle que avalia a capacidade das pipas e dos tonéis, medindo-os com a vara.
\section{Varadouro}
\begin{itemize}
\item {Grp. gram.:m.}
\end{itemize}
\begin{itemize}
\item {Utilização:Fig.}
\end{itemize}
\begin{itemize}
\item {Proveniência:(De \textunderscore varar\textunderscore )}
\end{itemize}
Lugar, onde se fazem encalhar as embarcações, para as consertar ou para as guardar, durante o tempo em que ellas não pódem navegar.
Lugar, onde um grupo de pessôas descansa e conversa.
\section{Varal}
\begin{itemize}
\item {Grp. gram.:m.}
\end{itemize}
\begin{itemize}
\item {Proveniência:(De \textunderscore vara\textunderscore )}
\end{itemize}
Cada uma das varas, que sáem dos lados de um vehículo, e entre as quaes se atrela o animal que puxa o mesmo vehículo.
Cada uma das varas análogas, nos andores, esquifes, cadeirinhas, etc., e nas quaes pegam os indivíduos, que conduzem esses objectos.
\section{Varancada}
\begin{itemize}
\item {Grp. gram.:f.}
\end{itemize}
\begin{itemize}
\item {Utilização:Des.}
\end{itemize}
\begin{itemize}
\item {Proveniência:(De \textunderscore vara\textunderscore  + \textunderscore anca\textunderscore ?)}
\end{itemize}
O mesmo que \textunderscore varada\textunderscore .
\section{Varanda}
\begin{itemize}
\item {Grp. gram.:f.}
\end{itemize}
\begin{itemize}
\item {Utilização:Ant.}
\end{itemize}
\begin{itemize}
\item {Utilização:Prov.}
\end{itemize}
\begin{itemize}
\item {Utilização:minh.}
\end{itemize}
\begin{itemize}
\item {Utilização:Bras. do Rio}
\end{itemize}
\begin{itemize}
\item {Utilização:Bras. do N}
\end{itemize}
\begin{itemize}
\item {Utilização:Bras. do N}
\end{itemize}
Terraço.
Eirado.
Balcão; sacada.
Parapeito de grade, que resguarda uma sacada.
Parapeito análogo, em janelas rasgadas até ao nível do pavimento, sem resaltar da superfície externa da parede.
Lugares para espectadores, nos theatros, por cima dos camarotes.
Roda dentada do lagar de azeite.
Tabella do bilhar.
Recinto, adjunto á casa de habitação, no qual dormem criados e hóspedes.
O primeiro dos três compartimentos do curral-de-peixe, conhecido também por \textunderscore sala\textunderscore .
Guarnição lateral das redes de dormir ou de transporte.
Sala comprida e estreita: \textunderscore varanda do jantar\textunderscore .--\textunderscore Baranda\textunderscore  é fórma pop. e talvez orthogr. preferível.
(Cp. cast. \textunderscore baranda\textunderscore  e port. \textunderscore varão\textunderscore  de ferro)
\section{Varandado}
\begin{itemize}
\item {Grp. gram.:m.}
\end{itemize}
\begin{itemize}
\item {Utilização:Bras. do N}
\end{itemize}
Espécie de alpendre, á frente da casa de campo, de que é accessório, com peitoris ou sem êlles.
\section{Varandim}
\begin{itemize}
\item {Grp. gram.:m.}
\end{itemize}
Varanda estreita.
Plataforma.
Grade baixa e elegante, usada nas janelas de peito, em construcções modernas.
\section{Varanga}
\begin{itemize}
\item {Grp. gram.:f.}
\end{itemize}
\begin{itemize}
\item {Utilização:Prov.}
\end{itemize}
\begin{itemize}
\item {Utilização:trasm.}
\end{itemize}
\begin{itemize}
\item {Proveniência:(De \textunderscore vara\textunderscore  + \textunderscore ?\textunderscore )}
\end{itemize}
Pau que, partindo horizontalmente das rodas de certos lagares, é puxado pelo boi, que faz andar as mesmas rodas.
\section{Varangada}
\begin{itemize}
\item {Grp. gram.:f.}
\end{itemize}
\begin{itemize}
\item {Utilização:Prov.}
\end{itemize}
\begin{itemize}
\item {Utilização:trasm.}
\end{itemize}
\begin{itemize}
\item {Utilização:Fig.}
\end{itemize}
Pancada, que a varanga dá, quando desanda.
Movimento convulsivo para os lados.
\section{Varano}
\begin{itemize}
\item {Grp. gram.:m.}
\end{itemize}
Gênero de reptis sáurios.
(Cp. cast. \textunderscore varano\textunderscore )
\section{Varão}
\begin{itemize}
\item {Grp. gram.:m.}
\end{itemize}
\begin{itemize}
\item {Utilização:Pop.}
\end{itemize}
\begin{itemize}
\item {Grp. gram.:Adj.}
\end{itemize}
Indivíduo do sexo masculino; homem.
Indivíduo adulto ou esforçado.
Homem respeitável.
Indivíduo casado, que domina inteiramente a mulhér. Cp. \textunderscore varela\textunderscore ^2.
Que é do sexo masculino: \textunderscore Manuel teve um filho varão...\textunderscore 
(Alter. de \textunderscore barão\textunderscore )
\section{Varão}
\begin{itemize}
\item {Grp. gram.:m.}
\end{itemize}
Vara grande de ferro ou de outro metal; tranca.
\section{Varapau}
\begin{itemize}
\item {Grp. gram.:m.}
\end{itemize}
\begin{itemize}
\item {Proveniência:(De \textunderscore vara\textunderscore  + \textunderscore pau\textunderscore )}
\end{itemize}
Pau comprido.
Bordão; cajado.
\section{Varar}
\begin{itemize}
\item {Grp. gram.:v. t.}
\end{itemize}
\begin{itemize}
\item {Grp. gram.:V. i.}
\end{itemize}
Bater com vara.
Meter no varadoiro.
Atravessar.
Expulsar.
Aterrar; encher de espanto.
Encalhar.
Passar além.
\section{Varatojano}
\begin{itemize}
\item {Grp. gram.:m.  e  adj.}
\end{itemize}
Frade do convento de Varatojo.
\section{Varatojo}
\begin{itemize}
\item {Grp. gram.:m.}
\end{itemize}
Variedade de pêra.
\section{Vardasca}
\begin{itemize}
\item {Grp. gram.:M.}
\end{itemize}
\begin{itemize}
\item {Utilização:T. do Fundão}
\end{itemize}
\textunderscore f.\textunderscore  (e der.)
O mesmo que \textunderscore verdasca\textunderscore , etc.
Homem robusto, valente.
Pimpão, farçola.
(Cp. cast. \textunderscore vardasca\textunderscore )
\section{Vareagem}
\begin{itemize}
\item {Grp. gram.:f.}
\end{itemize}
Acto ou effeito de \textunderscore varear\textunderscore .
\section{Varear}
\begin{itemize}
\item {Grp. gram.:v. t.}
\end{itemize}
\begin{itemize}
\item {Utilização:Ant.}
\end{itemize}
Medir ás varas.
Governar com vara (um barco).
\section{Varec}
\begin{itemize}
\item {Grp. gram.:m.}
\end{itemize}
\begin{itemize}
\item {Proveniência:(Fr. \textunderscore varec\textunderscore , de or. escandinava)}
\end{itemize}
Sargaço do mar, ou designação genérica das plantas marinhas da fam. das algas. Cf. F. Lapa, \textunderscore Chím. Agr.\textunderscore , 342, etc.--\textunderscore Varec\textunderscore  é fórma avêssa á índole da língua. \textunderscore Vareque\textunderscore  é fórma exacta.
\section{Varedo}
\begin{itemize}
\item {fónica:varê}
\end{itemize}
\begin{itemize}
\item {Grp. gram.:m.}
\end{itemize}
\begin{itemize}
\item {Proveniência:(De \textunderscore vara\textunderscore )}
\end{itemize}
Conjunto das vigotas de madeira ou ferro, que sustentam o ripado no telhado. Cp. \textunderscore varejo\textunderscore ^2.
\section{Vareira}
\begin{itemize}
\item {Grp. gram.:f.}
\end{itemize}
\begin{itemize}
\item {Proveniência:(De \textunderscore vareiro\textunderscore )}
\end{itemize}
Mulhér da beira-mar, entre Aveiro e o Pôrto aproximadamente; varina.
Dança e ária popular do norte de Portugal.
\section{Vareiro}
\begin{itemize}
\item {Grp. gram.:adj.}
\end{itemize}
\begin{itemize}
\item {Utilização:Taur.}
\end{itemize}
\begin{itemize}
\item {Grp. gram.:M.}
\end{itemize}
\begin{itemize}
\item {Utilização:Bras. do N}
\end{itemize}
\begin{itemize}
\item {Proveniência:(De \textunderscore vara\textunderscore )}
\end{itemize}
Relativo á beira-mar, entre Aveiro e o Pôrto proximamente: \textunderscore homem vareiro\textunderscore ; \textunderscore barco vareiro\textunderscore .
Que tem o corpo mais comprido do que é vulgar, (falando-se de toiros).
Homem vareiro.
Banco ou cavallete, em que se apoia o tronco que se há de serrar longitudinalmente.
Homem, que impulsiona a canôa com varas.
\section{Vareja}
\begin{itemize}
\item {Grp. gram.:f.}
\end{itemize}
\begin{itemize}
\item {Utilização:Fig.}
\end{itemize}
\begin{itemize}
\item {Grp. gram.:Adj.}
\end{itemize}
\begin{itemize}
\item {Proveniência:(De \textunderscore varejar\textunderscore ?)}
\end{itemize}
Lêndea da mosca varejeira.
Calúmnia.
Diz-se da mosca grande, também chamada \textunderscore varejeira\textunderscore .
\section{Varejador}
\begin{itemize}
\item {Grp. gram.:m.  e  adj.}
\end{itemize}
O que vareja; o que dá ou faz varejo.
\section{Varejadura}
\begin{itemize}
\item {Grp. gram.:f.}
\end{itemize}
Acto ou effeito de varejar.
\section{Varejamento}
\begin{itemize}
\item {Grp. gram.:m.}
\end{itemize}
O mesmo que \textunderscore varejadura\textunderscore .
\section{Varejão}
\begin{itemize}
\item {Grp. gram.:m.}
\end{itemize}
\begin{itemize}
\item {Utilização:Prov.}
\end{itemize}
\begin{itemize}
\item {Utilização:minh.}
\end{itemize}
Vara grande.
Estaca ou tutor, com que se segura uma videira ou uma arvore.
\section{Varejar}
\begin{itemize}
\item {Grp. gram.:v. t.}
\end{itemize}
\begin{itemize}
\item {Grp. gram.:V. i.}
\end{itemize}
\begin{itemize}
\item {Utilização:P. us.}
\end{itemize}
Agitar ou sacudir com vara: \textunderscore varejar um tapête\textunderscore .
Fazer cair, batendo com vara: \textunderscore varejar azeitonas\textunderscore .
Medir ás varas.
Revistar, dar varejo a: \textunderscore varejar uma casa\textunderscore .
Flagellar.
Incommodar.
Atacar.
Dar pancadas.
Disparar tiros.
\section{Varejeira}
\begin{itemize}
\item {Grp. gram.:f.}
\end{itemize}
\begin{itemize}
\item {Proveniência:(De \textunderscore vareja\textunderscore )}
\end{itemize}
Grande mosca, (\textunderscore musca carnaria\textunderscore ).
\section{Varejista}
\begin{itemize}
\item {Grp. gram.:m.}
\end{itemize}
\begin{itemize}
\item {Utilização:Bras}
\end{itemize}
\begin{itemize}
\item {Grp. gram.:Adj.}
\end{itemize}
\begin{itemize}
\item {Proveniência:(De \textunderscore varejo\textunderscore ^2)}
\end{itemize}
Industrial ou negociante, que vende a retalho ou por miúdo.
Relativo ao commércio a retalho.
\section{Varejo}
\begin{itemize}
\item {Grp. gram.:m.}
\end{itemize}
\begin{itemize}
\item {Utilização:Fig.}
\end{itemize}
\begin{itemize}
\item {Proveniência:(De \textunderscore varejar\textunderscore )}
\end{itemize}
O mesmo que \textunderscore varejadura\textunderscore .
Acto de revistar um estabelecimento industrial ou commercial, para verificar se há descaminho de direitos ou falsas declarações sôbre mercadorias expostas á venda.
Censura áspera.
\section{Varejo}
\begin{itemize}
\item {Grp. gram.:m.}
\end{itemize}
\begin{itemize}
\item {Utilização:Bras}
\end{itemize}
\begin{itemize}
\item {Proveniência:(De \textunderscore vara\textunderscore )}
\end{itemize}
Conjunto das varas, que sustentam a cobertura das choupanas e cubatas, na África.
Transacção de mercadorias ás varas ou a retalho.
\section{Varela}
\begin{itemize}
\item {Grp. gram.:f.}
\end{itemize}
\begin{itemize}
\item {Utilização:Prov.}
\end{itemize}
\begin{itemize}
\item {Utilização:minh.}
\end{itemize}
\begin{itemize}
\item {Utilização:Ant.}
\end{itemize}
\begin{itemize}
\item {Utilização:Prov.}
\end{itemize}
\begin{itemize}
\item {Utilização:minh.}
\end{itemize}
\begin{itemize}
\item {Proveniência:(De \textunderscore vara\textunderscore )}
\end{itemize}
Vara pequena, vareta.
Pauzinho, suspenso da adelha e de cuja parte anterior se suspende o adelhão.
Templo de ídolos.
Cavilha ou vareta de ferro, que prende o jugo ao tamoeiro.
\section{Varela}
\begin{itemize}
\item {Grp. gram.:m.}
\end{itemize}
\begin{itemize}
\item {Utilização:Burl.}
\end{itemize}
Homem casado que, em relação á mulher, é o meio termo entre \textunderscore varão\textunderscore  e \textunderscore varunca\textunderscore : \textunderscore se é varão, manda êlle e ella não; se é varela, ora manda êlle, ora manda ella; se é varunca, manda ella e êlle nunca\textunderscore .
\section{Varênea}
\begin{itemize}
\item {Grp. gram.:f.}
\end{itemize}
Gênero de plantas leguminosas.
\section{Varênnea}
\begin{itemize}
\item {Grp. gram.:f.}
\end{itemize}
Gênero de plantas leguminosas.
\section{Vareque}
\begin{itemize}
\item {Grp. gram.:m.}
\end{itemize}
\begin{itemize}
\item {Grp. gram.:m.}
\end{itemize}
\begin{itemize}
\item {Proveniência:(Fr. \textunderscore varec\textunderscore , de or. escandinava)}
\end{itemize}
O mesmo ou melhór que \textunderscore varec\textunderscore , que não é bôa fórma portuguesa.
Sargaço do mar, ou designação genérica das plantas marinhas da fam. das algas. Cf. F. Lapa, \textunderscore Chím. Agr.\textunderscore , 342, etc.--\textunderscore Varec\textunderscore  é fórma avessa á índole da língua. \textunderscore Vareque\textunderscore  é fórma exacta.
\section{Varestilha}
\begin{itemize}
\item {Grp. gram.:f.}
\end{itemize}
\begin{itemize}
\item {Utilização:Pesc.}
\end{itemize}
\begin{itemize}
\item {Proveniência:(De \textunderscore vara\textunderscore  + \textunderscore hastilha\textunderscore ?)}
\end{itemize}
Apparelho de anzóis.
\section{Vareta}
\begin{itemize}
\item {fónica:varê}
\end{itemize}
\begin{itemize}
\item {Grp. gram.:f.}
\end{itemize}
Vara pequena.
Vara delgada de ferro ou de pau, que termina numa das extremidades com um sacatrapo ou rosca, e que serve para limpar interiormente o cano das armas de fogo, e para calcar a carga e a bucha dessas armas.
Cada uma das pernas do compasso.
Planta irídea do Brasil.
O mesmo que \textunderscore pýrethro\textunderscore .
\section{Varga}
\begin{itemize}
\item {Grp. gram.:f.}
\end{itemize}
Planície alagadiça.
Várzea.
Armadilha para pesca, espécie de rêde.
(Cp. \textunderscore varge\textunderscore )
\section{Vargásia}
\begin{itemize}
\item {Grp. gram.:f.}
\end{itemize}
\begin{itemize}
\item {Proveniência:(De \textunderscore Vargas\textunderscore , n. p.)}
\end{itemize}
Gênero de plantas synanthéreas.
\section{Varge}
\begin{itemize}
\item {Grp. gram.:f.}
\end{itemize}
O mesmo que \textunderscore vargem\textunderscore :«\textunderscore varges fertilíssimas\textunderscore ». \textunderscore Luz e Calor\textunderscore , 552.
\section{Várgea}
\begin{itemize}
\item {Grp. gram.:f.}
\end{itemize}
O mesmo que \textunderscore vargem\textunderscore .
\section{Vargedo}
\begin{itemize}
\item {fónica:gê}
\end{itemize}
\begin{itemize}
\item {Grp. gram.:m.}
\end{itemize}
\begin{itemize}
\item {Utilização:Neol.}
\end{itemize}
Conjunto ou sequência de varges.
\section{Vargem}
\begin{itemize}
\item {Grp. gram.:f.}
\end{itemize}
O mesmo que \textunderscore várzea\textunderscore .
\section{Vargueiro}
\begin{itemize}
\item {Grp. gram.:m.}
\end{itemize}
Fabricante de rêdes, chamadas vargas.
\section{Vária}
\begin{itemize}
\item {Grp. gram.:f.}
\end{itemize}
Peixe de Portugal, do tamanho da taínha.
\section{Variabilidade}
\begin{itemize}
\item {Grp. gram.:f.}
\end{itemize}
\begin{itemize}
\item {Proveniência:(Do lat. \textunderscore variabilis\textunderscore )}
\end{itemize}
Qualidade do que é variável.
Volubilidade; inconstância.
\section{Variação}
\begin{itemize}
\item {Grp. gram.:f.}
\end{itemize}
\begin{itemize}
\item {Utilização:Gram.}
\end{itemize}
\begin{itemize}
\item {Utilização:Mús.}
\end{itemize}
\begin{itemize}
\item {Proveniência:(Do lat. \textunderscore variatio\textunderscore )}
\end{itemize}
Acto ou effeito de variar.
Mudança.
Parte variável de uma palavra.
Composição ou execução musical, em que se segue um thema, observando-lhe a melodia, mas addicionando-lhe novos ornatos.
\section{Variadamente}
\begin{itemize}
\item {Grp. gram.:adv.}
\end{itemize}
De modo variado.
De differente maneira; diversamente.
\section{Variadeira}
\begin{itemize}
\item {Grp. gram.:f.}
\end{itemize}
Máquina, guarnecida de um ou mais cylindros com pás ou varões de ferro, usada nas fábricas de lanifícios, e empregada em bater a lan e fazer saltar a terra ou a areia através de uma rêde de arame.
\section{Variado}
\begin{itemize}
\item {Grp. gram.:adj.}
\end{itemize}
\begin{itemize}
\item {Utilização:Pop.}
\end{itemize}
\begin{itemize}
\item {Proveniência:(De \textunderscore variar\textunderscore )}
\end{itemize}
Vário.
Differente; variegado.
Inconstante, leviano.
\section{Variagem}
\begin{itemize}
\item {Grp. gram.:f.}
\end{itemize}
\begin{itemize}
\item {Proveniência:(De \textunderscore variar\textunderscore )}
\end{itemize}
Antigo imposto aduaneiro.
\section{Variamente}
\begin{itemize}
\item {Grp. gram.:adv.}
\end{itemize}
De modo vário.
Diversamente.
\section{Variante}
\begin{itemize}
\item {Grp. gram.:adj.}
\end{itemize}
\begin{itemize}
\item {Grp. gram.:F.}
\end{itemize}
\begin{itemize}
\item {Proveniência:(Lat. \textunderscore varians\textunderscore )}
\end{itemize}
Que varía; differente.
Differença.
Variação.
Modificação na direcção de uma estrada.
Cada uma das diversas lições ou fórmas do mesmo texto ou vocábolo.
\section{Variar}
\begin{itemize}
\item {Grp. gram.:v. t.}
\end{itemize}
\begin{itemize}
\item {Grp. gram.:V. i.}
\end{itemize}
\begin{itemize}
\item {Proveniência:(Lat. \textunderscore variare\textunderscore )}
\end{itemize}
Tornar vário ou diverso.
Alterar; alternar.
Dar várias côres a.
Dispor variadamente.
Diversificar.
Fazer variações musicaes a respeito ou sôbre.
Fazer mudança.
Apresentar-se sob differentes fórmas ou aspectos.
Sêr inconstante.
Discrepar, discordar.
Desvairar; delirar; endoidecer.
\section{Variável}
\begin{itemize}
\item {Grp. gram.:adj.}
\end{itemize}
\begin{itemize}
\item {Utilização:Gram.}
\end{itemize}
\begin{itemize}
\item {Proveniência:(Lat. \textunderscore variabilis\textunderscore )}
\end{itemize}
Que se póde variar.
Mudável; inconstante.
Que soffre modificações na desinência, (falando-se de certas palavras).
\section{Variavelmente}
\begin{itemize}
\item {Grp. gram.:adv.}
\end{itemize}
De modo variável.
\section{Variaz}
\begin{itemize}
\item {Grp. gram.:m.}
\end{itemize}
O mesmo que \textunderscore vária\textunderscore .
\section{Varicela}
\begin{itemize}
\item {Grp. gram.:f.}
\end{itemize}
\begin{itemize}
\item {Proveniência:(Fr. \textunderscore varicelle\textunderscore )}
\end{itemize}
Doença infecciosa e contagiosa, ordinariamente benigna, caracterizada por uma erupção de pequenas bolhas, que secam ao cabo de alguns dias.
\section{Varicella}
\begin{itemize}
\item {Grp. gram.:f.}
\end{itemize}
\begin{itemize}
\item {Proveniência:(Fr. \textunderscore varicelle\textunderscore )}
\end{itemize}
Doença infecciosa e contagiosa, ordinariamente benigna, caracterizada por uma erupção de pequenas bolhas, que secam ao cabo de alguns dias.
\section{Variço}
\begin{itemize}
\item {Grp. gram.:adj.}
\end{itemize}
\begin{itemize}
\item {Utilização:Prov.}
\end{itemize}
\begin{itemize}
\item {Utilização:beir.}
\end{itemize}
\begin{itemize}
\item {Proveniência:(De \textunderscore vara\textunderscore )}
\end{itemize}
Diz-se do porco, que tem corpo comprido, em desproporção com a largura. (Colhido na Guarda)
\section{Varicocele}
\begin{itemize}
\item {Grp. gram.:m.}
\end{itemize}
\begin{itemize}
\item {Proveniência:(Do lat. \textunderscore varix\textunderscore  + gr. \textunderscore kele\textunderscore )}
\end{itemize}
Tumor, causado pela dilatação varicosa das veias do escroto.
\section{Varicoso}
\begin{itemize}
\item {Grp. gram.:adj.}
\end{itemize}
\begin{itemize}
\item {Proveniência:(Lat. \textunderscore varicosus\textunderscore )}
\end{itemize}
Que tem varizes.
Que tem disposição para varizes.
\section{Variedade}
\begin{itemize}
\item {Grp. gram.:f.}
\end{itemize}
\begin{itemize}
\item {Utilização:Hist. Nat.}
\end{itemize}
\begin{itemize}
\item {Proveniência:(Do lat. \textunderscore varietas\textunderscore )}
\end{itemize}
Qualidade do que é vário.
Variação.
Diversidade.
Multiplicidade.
Inconstância.
Sub-divisão das espécies, baseada nas ligeiras differenças entre indivíduos da mesma espécie.
\section{Variegação}
\begin{itemize}
\item {Grp. gram.:f.}
\end{itemize}
Acto ou effeito de variegar.
Qualidade ou estado do que é variegado; matiz.
\section{Variegar}
\begin{itemize}
\item {Grp. gram.:v. t.}
\end{itemize}
\begin{itemize}
\item {Proveniência:(Lat. \textunderscore variegare\textunderscore )}
\end{itemize}
Dar côres diversas a; matizar.
Diversificar.
Alternar.
\section{Varilha}
\begin{itemize}
\item {Grp. gram.:f.}
\end{itemize}
\begin{itemize}
\item {Utilização:Prov.}
\end{itemize}
Vara pequena; vareta.
\section{Varilhas}
\begin{itemize}
\item {Grp. gram.:f. pl.}
\end{itemize}
\begin{itemize}
\item {Utilização:Prov.}
\end{itemize}
\begin{itemize}
\item {Utilização:trasm.}
\end{itemize}
\begin{itemize}
\item {Proveniência:(Do cast. \textunderscore varilla\textunderscore )}
\end{itemize}
Utensílio de pau, sôbre que se movem as peneiras, deixando estas cair a farinha na sabanilha.
\section{Varina}
\begin{itemize}
\item {Grp. gram.:f.}
\end{itemize}
\begin{itemize}
\item {Utilização:T. de Lisbôa}
\end{itemize}
\begin{itemize}
\item {Proveniência:(De \textunderscore varino\textunderscore ^1)}
\end{itemize}
Vendedeira ambulante de peixe.
Mulhér da beira mar, entre Aveiro e o Pôrto aproximadamente.
\section{Varina}
\begin{itemize}
\item {Grp. gram.:f.}
\end{itemize}
\begin{itemize}
\item {Utilização:Bras}
\end{itemize}
O mesmo que \textunderscore varino\textunderscore ^2.
\section{Varinel}
\begin{itemize}
\item {Grp. gram.:m.}
\end{itemize}
\begin{itemize}
\item {Proveniência:(De \textunderscore varino\textunderscore ^2)}
\end{itemize}
Embarcação de remos, usada no século XV.
O mesmo que \textunderscore barinel\textunderscore .
\section{Varinha}
\begin{itemize}
\item {Grp. gram.:f.}
\end{itemize}
Vara delgada.
Vara mágica, usada por prestidigitadores e arlequins.
\section{Varino}
\begin{itemize}
\item {Grp. gram.:m.  e  adj.}
\end{itemize}
\begin{itemize}
\item {Grp. gram.:M.}
\end{itemize}
O mesmo que \textunderscore vareiro\textunderscore .
O mesmo que \textunderscore gabão\textunderscore ^1.
\section{Varino}
\begin{itemize}
\item {Grp. gram.:m.}
\end{itemize}
\begin{itemize}
\item {Proveniência:(De \textunderscore vara\textunderscore )}
\end{itemize}
Barco estreito e comprido.
Cp. \textunderscore vareiro\textunderscore .
\section{Vário}
\begin{itemize}
\item {Grp. gram.:adj.}
\end{itemize}
\begin{itemize}
\item {Proveniência:(Lat. \textunderscore varius\textunderscore )}
\end{itemize}
Que apresenta diversas côres ou feitios.
Matizado.
Differente: \textunderscore várias crenças\textunderscore .
Inconstante.
Muito; numeroso: \textunderscore fazer várias despesas\textunderscore .
Algum.
Um certo.
Perplexo.
Buliçoso.
Contradictório: \textunderscore sustentar opiniões várias\textunderscore .
\section{Varíola}
\begin{itemize}
\item {Grp. gram.:f.}
\end{itemize}
Doença febril, com erupção pustulosa na pelle.
Bexigas.
(B. lat. \textunderscore variola\textunderscore )
\section{Varíola-mansa}
\begin{itemize}
\item {Grp. gram.:f.}
\end{itemize}
\begin{itemize}
\item {Utilização:Bras}
\end{itemize}
Febre eruptiva, semelhante á varíola, mas distinta desta por alguns caracteres.
\section{Variolar}
\begin{itemize}
\item {Grp. gram.:adj.}
\end{itemize}
Semelhante ás manchas da varíola.
\section{Variolário}
\begin{itemize}
\item {Grp. gram.:adj.}
\end{itemize}
\begin{itemize}
\item {Utilização:Miner.}
\end{itemize}
\begin{itemize}
\item {Proveniência:(De \textunderscore varíola\textunderscore )}
\end{itemize}
Diz-se de uma variedade de rocha, que apresenta manchas redondas, de côr differente da do fundo.
\section{Variólico}
\begin{itemize}
\item {Grp. gram.:adj.}
\end{itemize}
Relativo á varíola.
\section{Varioliforme}
\begin{itemize}
\item {Grp. gram.:adj.}
\end{itemize}
\begin{itemize}
\item {Proveniência:(De \textunderscore varíola\textunderscore  + \textunderscore fórma\textunderscore )}
\end{itemize}
Que tem analogia com a varíola.
\section{Variolina}
\begin{itemize}
\item {Grp. gram.:f.}
\end{itemize}
\begin{itemize}
\item {Utilização:Miner.}
\end{itemize}
\begin{itemize}
\item {Proveniência:(De \textunderscore varíola\textunderscore )}
\end{itemize}
Base de uma espécie de variolita, a chamada variolita de Durance.
\section{Variolita}
\begin{itemize}
\item {Grp. gram.:f.}
\end{itemize}
\begin{itemize}
\item {Utilização:Miner.}
\end{itemize}
\begin{itemize}
\item {Proveniência:(De \textunderscore varíola\textunderscore )}
\end{itemize}
Rocha, formada de núcleos do feldspatho crystallino, engastados numa massa de feldspatho compacto.
\section{Variolite}
\begin{itemize}
\item {Grp. gram.:f.}
\end{itemize}
\begin{itemize}
\item {Utilização:Miner.}
\end{itemize}
\begin{itemize}
\item {Proveniência:(De \textunderscore varíola\textunderscore )}
\end{itemize}
Rocha, formada de núcleos do feldspatho crystallino, engastados numa massa de feldspatho compacto.
\section{Variolito}
\begin{itemize}
\item {Grp. gram.:m.}
\end{itemize}
O mesmo ou melhór que \textunderscore variolita\textunderscore .
\section{Variolóide}
\begin{itemize}
\item {Grp. gram.:f.}
\end{itemize}
\begin{itemize}
\item {Proveniência:(De \textunderscore varíola\textunderscore  + gr. \textunderscore eidos\textunderscore )}
\end{itemize}
Fórma benigna da varíola, caracterizada pela extensão do período da invasão, pela ausência de suppuração e pela brevidade da evolução total.
\section{Varioloso}
\begin{itemize}
\item {Grp. gram.:adj.}
\end{itemize}
\begin{itemize}
\item {Grp. gram.:M.}
\end{itemize}
O mesmo que \textunderscore variólico\textunderscore .
Que tem varíola.
Indivíduo, atacado de varíola.
\section{Varioso}
\begin{itemize}
\item {Grp. gram.:adj.}
\end{itemize}
\begin{itemize}
\item {Utilização:T. de Ceilão}
\end{itemize}
O mesmo que \textunderscore vário\textunderscore .
\section{Variospermo}
\begin{itemize}
\item {Grp. gram.:adj.}
\end{itemize}
\begin{itemize}
\item {Utilização:Bot.}
\end{itemize}
\begin{itemize}
\item {Proveniência:(Do lat. \textunderscore varius\textunderscore  + \textunderscore sperma\textunderscore )}
\end{itemize}
Que tem sementes de differente tamanho.
\section{Variz}
\begin{itemize}
\item {Grp. gram.:f.}
\end{itemize}
\begin{itemize}
\item {Proveniência:(Do lat. \textunderscore varix\textunderscore )}
\end{itemize}
Dilatação permanente de uma veia, produzida pela accumulação de sangue no interior desta.
Proeminência no bôrdo de algumas conchas univalves.
\section{Varja}
\begin{itemize}
\item {Grp. gram.:f.}
\end{itemize}
O mesmo que \textunderscore várzea\textunderscore .
\section{Varlete}
\begin{itemize}
\item {Grp. gram.:m.}
\end{itemize}
\begin{itemize}
\item {Utilização:Ant.}
\end{itemize}
\begin{itemize}
\item {Proveniência:(Fr. ant. \textunderscore varlet\textunderscore )}
\end{itemize}
Pagem; criado.
\section{Varlôas}
\begin{itemize}
\item {Grp. gram.:f. pl.}
\end{itemize}
\begin{itemize}
\item {Utilização:Náut.}
\end{itemize}
Cabos, para segurar a embarcação, quando está em querena.
\section{Varlopa}
\begin{itemize}
\item {Grp. gram.:f.}
\end{itemize}
\begin{itemize}
\item {Grp. gram.:f.}
\end{itemize}
\begin{itemize}
\item {Proveniência:(Do fr. \textunderscore varlope\textunderscore )}
\end{itemize}
O mesmo que \textunderscore garlopa\textunderscore .
Plaina grande.
\section{Varo}
\begin{itemize}
\item {Grp. gram.:adj.}
\end{itemize}
\begin{itemize}
\item {Utilização:Med.}
\end{itemize}
\begin{itemize}
\item {Proveniência:(Lat. \textunderscore varus\textunderscore )}
\end{itemize}
Diz-se de um membro, ou de um segmento de membro, votado para dentro.
\section{Varôa}
\begin{itemize}
\item {Grp. gram.:f.}
\end{itemize}
\begin{itemize}
\item {Utilização:Des.}
\end{itemize}
Mulhér forte.
Virago; heroína.
(Fem. de \textunderscore varão\textunderscore ^1)
\section{Varola}
\begin{itemize}
\item {Grp. gram.:f.}
\end{itemize}
\begin{itemize}
\item {Grp. gram.:M.}
\end{itemize}
\begin{itemize}
\item {Utilização:Prov.}
\end{itemize}
\begin{itemize}
\item {Utilização:minh.}
\end{itemize}
\begin{itemize}
\item {Proveniência:(De \textunderscore vara\textunderscore )}
\end{itemize}
O mesmo que \textunderscore vareta\textunderscore .
Indivíduo mentiroso ou gabarola.
\section{Varonia}
\begin{itemize}
\item {Grp. gram.:f.}
\end{itemize}
Qualidade de quem é varão^1.
Descendência em linha masculina.
\section{Varonil}
\begin{itemize}
\item {Grp. gram.:adj.}
\end{itemize}
Relativo a varão ou próprio de varão^1.
Valoroso; enérgico; heróico.
\section{Varonilidade}
\begin{itemize}
\item {Grp. gram.:f.}
\end{itemize}
Qualidade do que é varonil.
\section{Varonilmente}
\begin{itemize}
\item {Grp. gram.:adv.}
\end{itemize}
De modo varonil.
Esforçadamente; energicamente.
\section{Varrão}
\begin{itemize}
\item {Grp. gram.:m.}
\end{itemize}
Porco, não castrado.
(Por \textunderscore verrão\textunderscore , do lat. \textunderscore verres\textunderscore )
\section{Varrasco}
\begin{itemize}
\item {Grp. gram.:m.}
\end{itemize}
O mesmo que \textunderscore varrão\textunderscore .
(Por \textunderscore verrasco\textunderscore , do lat. \textunderscore verres\textunderscore )
\section{Varrasco-do-mar}
\begin{itemize}
\item {Grp. gram.:m.}
\end{itemize}
Espécie de escorpena.
\section{Varredeira}
\begin{itemize}
\item {Grp. gram.:f.}
\end{itemize}
\begin{itemize}
\item {Utilização:Náut.}
\end{itemize}
\begin{itemize}
\item {Proveniência:(De \textunderscore varrer\textunderscore )}
\end{itemize}
Vela quadrangular, que se iça no mastro do traquete e vai fixar-se no pau da surriola.
\section{Varredela}
\begin{itemize}
\item {Grp. gram.:f.}
\end{itemize}
Acto ou effeito de varrer.
\section{Varredoira}
\begin{itemize}
\item {Grp. gram.:f.}
\end{itemize}
\begin{itemize}
\item {Utilização:Náut.}
\end{itemize}
\begin{itemize}
\item {Utilização:Pop.}
\end{itemize}
\begin{itemize}
\item {Grp. gram.:Adj.}
\end{itemize}
\begin{itemize}
\item {Proveniência:(De \textunderscore varrer\textunderscore )}
\end{itemize}
O mesmo que \textunderscore varredeira\textunderscore .
Mortandade; grande destruição.
Diz-se de uma rêde de pescar.
\section{Varredoiro}
\begin{itemize}
\item {Grp. gram.:m.}
\end{itemize}
\begin{itemize}
\item {Proveniência:(De \textunderscore varrer\textunderscore )}
\end{itemize}
Espécie de vassoiro, com que se varre o forno do pão.
Fragüeiro.
Espécie de vassoira, que, presa entre as aivecas do arado, vai varrendo as raízes que o arado levanta.
\section{Varredor}
\begin{itemize}
\item {Grp. gram.:adj.}
\end{itemize}
\begin{itemize}
\item {Grp. gram.:M.}
\end{itemize}
\begin{itemize}
\item {Proveniência:(De \textunderscore varrer\textunderscore )}
\end{itemize}
Que varre.
Aquelle que varre.
O mesmo que \textunderscore varredeira\textunderscore .
\section{Varredoura}
\begin{itemize}
\item {Grp. gram.:f.}
\end{itemize}
\begin{itemize}
\item {Utilização:Náut.}
\end{itemize}
\begin{itemize}
\item {Utilização:Pop.}
\end{itemize}
\begin{itemize}
\item {Grp. gram.:Adj.}
\end{itemize}
\begin{itemize}
\item {Proveniência:(De \textunderscore varrer\textunderscore )}
\end{itemize}
O mesmo que \textunderscore varredeira\textunderscore .
Mortandade; grande destruição.
Diz-se de uma rêde de pescar.
\section{Varredouro}
\begin{itemize}
\item {Grp. gram.:m.}
\end{itemize}
\begin{itemize}
\item {Proveniência:(De \textunderscore varrer\textunderscore )}
\end{itemize}
Espécie de vassouro, com que se varre o forno do pão.
Fragüeiro.
Espécie de vassoura, que, presa entre as aivecas do arado, vai varrendo as raízes que o arado levanta.
\section{Varredura}
\begin{itemize}
\item {Grp. gram.:f.}
\end{itemize}
\begin{itemize}
\item {Proveniência:(De \textunderscore varrer\textunderscore )}
\end{itemize}
O mesmo que \textunderscore varredela\textunderscore .
Lixo, que se junta, varrendo.
Restos de comida, na mesa.
Restos.
\section{Varrer}
\begin{itemize}
\item {Grp. gram.:v. t.}
\end{itemize}
\begin{itemize}
\item {Utilização:Fig.}
\end{itemize}
\begin{itemize}
\item {Grp. gram.:V. i.}
\end{itemize}
\begin{itemize}
\item {Grp. gram.:V. p.}
\end{itemize}
\begin{itemize}
\item {Utilização:Pop.}
\end{itemize}
\begin{itemize}
\item {Grp. gram.:M.}
\end{itemize}
\begin{itemize}
\item {Proveniência:(Do lat. \textunderscore verrere\textunderscore )}
\end{itemize}
Limpar com vassoira, especialmente o solo ou soalho: \textunderscore varrer uma casa\textunderscore ; \textunderscore varrer uma rua, um jardim\textunderscore , etc.
Limpar.
Tirar.
Privar.
Despejar.
Passar pela superfície de; roçar.
Destruir.
Impellir, adeante de si.
Expulsar.
Extinguir; apagar: \textunderscore varrer a lembrança de um benefício\textunderscore .
Limpar lixo ou pó com a vassoira: \textunderscore tu não sabes varrer\textunderscore .
Desvanecer-se, tornar-se esquecido: \textunderscore varreu-se-me da memória o caso\textunderscore .
Varredela.
\section{Varrido}
\begin{itemize}
\item {Grp. gram.:adj.}
\end{itemize}
\begin{itemize}
\item {Utilização:Fig.}
\end{itemize}
\begin{itemize}
\item {Grp. gram.:M.}
\end{itemize}
Limpo.
Que perdeu o juízo.
Que não tem tino ou juízo.
Completo, rematado, (falando-se de um doido ou pateta).
Aquillo que se varreu.
Varredura; varredela: \textunderscore hoje cá em casa é dia de varridos\textunderscore .
\section{Varrimenta}
\begin{itemize}
\item {Grp. gram.:f.}
\end{itemize}
O mesmo que \textunderscore varredura\textunderscore . Cf. Camillo, \textunderscore Narcót.\textunderscore , I, 146.
\section{Varriscador}
\begin{itemize}
\item {Grp. gram.:m.}
\end{itemize}
\begin{itemize}
\item {Utilização:Prov.}
\end{itemize}
Vara, com que se mexem as brasas do forno. (Colhido em Arganil)
\section{Varsoviana}
\begin{itemize}
\item {Grp. gram.:f.}
\end{itemize}
\begin{itemize}
\item {Proveniência:(De \textunderscore varsoviano\textunderscore )}
\end{itemize}
Espécie de dança polaca.
\section{Varsoviano}
\begin{itemize}
\item {Grp. gram.:adj.}
\end{itemize}
\begin{itemize}
\item {Grp. gram.:M.}
\end{itemize}
Relativo a Varsóvia.
Habitante de Varsóvia.
\section{Varudo}
\begin{itemize}
\item {Grp. gram.:adj.}
\end{itemize}
\begin{itemize}
\item {Grp. gram.:Loc. adv.}
\end{itemize}
\begin{itemize}
\item {Utilização:Prov.}
\end{itemize}
\begin{itemize}
\item {Utilização:alent.}
\end{itemize}
\begin{itemize}
\item {Proveniência:(De \textunderscore vara\textunderscore )}
\end{itemize}
Diz-se de um tronco de árvore, direito e comprido, ou da árvore que tem êsse tronco.
Diz-se do boi ou vitello, de corpo comprido, direito e forte; vareiro.
\textunderscore Ao varudo\textunderscore , ao comprido.
\section{Varunca}
\begin{itemize}
\item {Grp. gram.:m.}
\end{itemize}
\begin{itemize}
\item {Utilização:Burl.}
\end{itemize}
Marido froixo, que é dominado pela mulhér:«\textunderscore certa mulher que tinha a barba tesa ao varunca e cabrão numa disputa...\textunderscore »Filinto, VIII, 169.
(Da graduação pop. de \textunderscore varão\textunderscore , \textunderscore varela\textunderscore  e \textunderscore varunca\textunderscore )
(Cp. \textunderscore varela\textunderscore ^2)
\section{Várzea}
\begin{itemize}
\item {Grp. gram.:f.}
\end{itemize}
Campina cultivada.
Chan; planície.
\section{Varzedo}
\begin{itemize}
\item {fónica:zê}
\end{itemize}
\begin{itemize}
\item {Grp. gram.:m.}
\end{itemize}
\begin{itemize}
\item {Utilização:Prov.}
\end{itemize}
Continuidade de várzeas; vargedo.
\section{Varzino}
\begin{itemize}
\item {Grp. gram.:adj.}
\end{itemize}
Relativo a várzea.
\section{Vasa}
\begin{itemize}
\item {Grp. gram.:f.}
\end{itemize}
\begin{itemize}
\item {Utilização:Fig.}
\end{itemize}
Fundo lodoso de um rio, do mar, etc.
Fundo do rio ou do mar.
Lôdo; lodaçal; terra pantanosa.
Espaço circular, em fórma de grande vaso cónico, onde trabalha a mó do moínho de azeitona.
Um dos reservatórios das marinhas, no qual se deposita uma parte das substâncias estranhas que se encontram dissolvidas na água com o sal.
Degradação moral.
(Do neerlandês \textunderscore wase\textunderscore )
\section{Vasa}
\begin{itemize}
\item {Grp. gram.:f.}
\end{itemize}
(V. \textunderscore vaza\textunderscore ^1)
\section{Vasaréu}
\begin{itemize}
\item {Grp. gram.:m.}
\end{itemize}
\begin{itemize}
\item {Utilização:T. de Avis}
\end{itemize}
Vasilha velha; caco.
\section{Vasário}
\begin{itemize}
\item {Grp. gram.:m.}
\end{itemize}
\begin{itemize}
\item {Proveniência:(Lat. \textunderscore vasarium\textunderscore )}
\end{itemize}
Dinheiro ou provisões, dadas aos magistrados romanos, que partiam para as províncias, para gastos de viagem e de installação.
\section{Vasca}
\begin{itemize}
\item {Grp. gram.:f.}
\end{itemize}
\begin{itemize}
\item {Grp. gram.:Pl.}
\end{itemize}
\begin{itemize}
\item {Proveniência:(Do lat. hypoth. \textunderscore vascare\textunderscore , do lat. \textunderscore vascus\textunderscore , curvo?)}
\end{itemize}
Grande convulsão.
Ânsia excessiva; estertor.
Náuseas.
\section{Vascão}
\begin{itemize}
\item {Grp. gram.:m.}
\end{itemize}
O mesmo que \textunderscore vasco\textunderscore .
\section{Vasco}
\begin{itemize}
\item {Grp. gram.:m.}
\end{itemize}
\begin{itemize}
\item {Proveniência:(Lat. \textunderscore vasco\textunderscore )}
\end{itemize}
Habitante da região, que comprehende a Navarra e a Biscaia; vascongado.
\section{Vascolejador}
\begin{itemize}
\item {Grp. gram.:m.  e  adj.}
\end{itemize}
O que vascoleja.
\section{Vascolejamento}
\begin{itemize}
\item {Grp. gram.:m.}
\end{itemize}
Acto ou effeito de vascolejar.
\section{Vascolejar}
\begin{itemize}
\item {Grp. gram.:v. t.}
\end{itemize}
\begin{itemize}
\item {Utilização:Fig.}
\end{itemize}
\begin{itemize}
\item {Proveniência:(Do lat. \textunderscore vasculum\textunderscore )}
\end{itemize}
Agitar (um líquido contido num vaso).
Agitar (um vaso que contém um líquido).
Perturbar.
\section{Vasconcear}
\begin{itemize}
\item {Grp. gram.:v. i.}
\end{itemize}
\begin{itemize}
\item {Utilização:Fig.}
\end{itemize}
\begin{itemize}
\item {Grp. gram.:V. t.}
\end{itemize}
\begin{itemize}
\item {Proveniência:(De \textunderscore vasconço\textunderscore )}
\end{itemize}
Falar vasconço.
Dizer algaravias ou coisas inintelligíveis.
Gracejar.
Exprimir em estilo muito subtil ou inintelligível.
\section{Vasconcélia}
\begin{itemize}
\item {Grp. gram.:f.}
\end{itemize}
Gênero de plantas papaiáceas.
(Provavelmente, de \textunderscore Vasconcellos\textunderscore , n. p.)
\section{Vasconcéllia}
\begin{itemize}
\item {Grp. gram.:f.}
\end{itemize}
Gênero de plantas papaiáceas.
(Provavelmente, de \textunderscore Vasconcellos\textunderscore , n. p.)
\section{Vasconço}
\begin{itemize}
\item {Grp. gram.:m.}
\end{itemize}
\begin{itemize}
\item {Utilização:Fig.}
\end{itemize}
\begin{itemize}
\item {Proveniência:(Do cast. \textunderscore vascuence\textunderscore )}
\end{itemize}
Idioma vernáculo dos Pirenéus, agglutinativo, sem parentesco com outro idioma conhecido, e que tem particularíssima estructura grammatical.
Linguagem inintelligivel.
\section{Vascongado}
\begin{itemize}
\item {Grp. gram.:adj.}
\end{itemize}
\begin{itemize}
\item {Grp. gram.:M.}
\end{itemize}
Relativo ás Vascongadas ou aos seus habitantes.
Aquelle que é natural das Vascongadas.
\section{Vascuence}
\begin{itemize}
\item {Grp. gram.:m.}
\end{itemize}
O mesmo que \textunderscore vasconço\textunderscore .
\section{Vascular}
\begin{itemize}
\item {Grp. gram.:adj.}
\end{itemize}
\begin{itemize}
\item {Utilização:Anat.}
\end{itemize}
\begin{itemize}
\item {Utilização:Bot.}
\end{itemize}
\begin{itemize}
\item {Proveniência:(Do lat. \textunderscore vasculum\textunderscore )}
\end{itemize}
Relativo aos vasos, especialmente aos vasos sanguíneos.
Formado de vasos.
\section{Vascularidade}
\begin{itemize}
\item {Grp. gram.:f.}
\end{itemize}
\begin{itemize}
\item {Utilização:Physiol.}
\end{itemize}
\begin{itemize}
\item {Proveniência:(De \textunderscore vascular\textunderscore )}
\end{itemize}
Existência de menor ou maiór quantidade de vasos sanguíneos ou lympháticos.
\section{Vascularização}
\begin{itemize}
\item {Grp. gram.:f.}
\end{itemize}
\begin{itemize}
\item {Utilização:Physiol.}
\end{itemize}
\begin{itemize}
\item {Proveniência:(De \textunderscore vascular\textunderscore )}
\end{itemize}
Formação de vasos num tecido que os não tinha.
Multiplicação dos vasos primitivos de um órgão ou tecido.
\section{Vasculizado}
\begin{itemize}
\item {Grp. gram.:adj.}
\end{itemize}
\begin{itemize}
\item {Utilização:Physiol.}
\end{itemize}
\begin{itemize}
\item {Proveniência:(Do lat. \textunderscore vasculum\textunderscore )}
\end{itemize}
Em que há formação de vasos sanguíneos.
\section{Vasculho}
\textunderscore m.\textunderscore  (e der.)
O mesmo que \textunderscore basculho\textunderscore , etc.
\section{Vaseiro}
\begin{itemize}
\item {Grp. gram.:m.  e  adj.}
\end{itemize}
\begin{itemize}
\item {Proveniência:(De \textunderscore vasa\textunderscore ^1?)}
\end{itemize}
Diz-se de uma espécie de pequenos veados.
\section{Vaselina}
\begin{itemize}
\item {Grp. gram.:f.}
\end{itemize}
Substância gordurosa, extrahida dos resíduos da destillação do petróleo, e applicada nas indústrias e na pharmácia.
\section{Vasento}
\begin{itemize}
\item {Grp. gram.:adj.}
\end{itemize}
Que tem vasa^1 ou lodo.
\section{Vasículo}
\begin{itemize}
\item {Grp. gram.:m.}
\end{itemize}
\begin{itemize}
\item {Utilização:P. us.}
\end{itemize}
Vaso pequeno; vasilho.
(Dem. de \textunderscore vaso\textunderscore )
\section{Vasilha}
\begin{itemize}
\item {Grp. gram.:f.}
\end{itemize}
\begin{itemize}
\item {Utilização:Ant.}
\end{itemize}
\begin{itemize}
\item {Utilização:Prov.}
\end{itemize}
\begin{itemize}
\item {Utilização:alent.}
\end{itemize}
\begin{itemize}
\item {Proveniência:(Do lat. hyp. \textunderscore vasilia\textunderscore , de \textunderscore vas\textunderscore )}
\end{itemize}
Vaso para líquidos.
Barril; pipa; tonel.
Embarcação.
Carro.
\section{Vasilhame}
\begin{itemize}
\item {Grp. gram.:m.}
\end{itemize}
Porção de vasilhas.
\section{Vasilho}
\begin{itemize}
\item {Grp. gram.:m.}
\end{itemize}
\begin{itemize}
\item {Utilização:Prov.}
\end{itemize}
Pequeno vaso; pequena vasilha de loiça. Cf. \textunderscore Bibl. da G. do Campo\textunderscore , 426.
\section{Vaso}
\begin{itemize}
\item {Grp. gram.:m.}
\end{itemize}
\begin{itemize}
\item {Utilização:Ext.}
\end{itemize}
\begin{itemize}
\item {Proveniência:(Lat. \textunderscore vasum\textunderscore )}
\end{itemize}
Qualquer objecto côncavo, próprio para conter na sua cavidade substâncias líquidas ou sólidas.
Peça análoga, de fórmas variadas, e que se enche de terra, para nesta se cultivarem plantas.
Receptáculo; tudo que póde conter objectos.
Navio.
Tubo, no organismo animal ou vegetal, próprio para a circulação dos líquidos nutrítivos.
Veia; artéria.
Vagina.
Constellação austral.
Bacio dos quartos de dormir.
\section{Vaso}
\begin{itemize}
\item {Grp. gram.:m.}
\end{itemize}
Antiga fazenda de lan preta para luto; luto.
Capuz preto de burel, com que cobriam a cabeça, cara e ombros as pessôas que estavam de luto.
\section{Vasogênio}
\begin{itemize}
\item {Grp. gram.:m.}
\end{itemize}
\begin{itemize}
\item {Utilização:Pharm.}
\end{itemize}
Excipiente de pomadas medicamentosas.
\section{Vaso-motor}
\begin{itemize}
\item {Grp. gram.:adj.}
\end{itemize}
\begin{itemize}
\item {Utilização:Physiol.}
\end{itemize}
\begin{itemize}
\item {Proveniência:(De \textunderscore vaso\textunderscore ^1 + \textunderscore motor\textunderscore )}
\end{itemize}
Que póde causar movimento nos vasos.
Diz-se dos nervos, que produzem a contracção e dilatação das fibras musculares dos vasos.
\section{Vaso-motriz}
\begin{itemize}
\item {Grp. gram.:adj.}
\end{itemize}
(Fem. de \textunderscore vaso-motor\textunderscore )
\section{Vasótribo}
\begin{itemize}
\item {Grp. gram.:m.}
\end{itemize}
\begin{itemize}
\item {Utilização:Med.}
\end{itemize}
Instrumento, com que se pratíca a vasotripsia.
\section{Vasotripsia}
\begin{itemize}
\item {Grp. gram.:f.}
\end{itemize}
\begin{itemize}
\item {Utilização:Med.}
\end{itemize}
\begin{itemize}
\item {Proveniência:(Do lat. \textunderscore vas\textunderscore  + gr. \textunderscore tripsis\textunderscore )}
\end{itemize}
Esmagamento de um vaso, por meio de um vasótribo, para se conseguir a hemóstase immediata.
\section{Vasoso}
\begin{itemize}
\item {Grp. gram.:adj.}
\end{itemize}
O mesmo que \textunderscore vasento\textunderscore .
\section{Vasqueiro}
\begin{itemize}
\item {Grp. gram.:adj.}
\end{itemize}
\begin{itemize}
\item {Utilização:Des.}
\end{itemize}
Que produz vascas ou ânsias.
\section{Vasqueiro}
\begin{itemize}
\item {Grp. gram.:adj.}
\end{itemize}
O mesmo que \textunderscore vesgo\textunderscore .
\textunderscore Dar vasqueiro\textunderscore , dar de esguelha.
(Por \textunderscore vesgueiro\textunderscore , de \textunderscore vesgo\textunderscore )
\section{Vasqueiro}
\begin{itemize}
\item {Grp. gram.:adj.}
\end{itemize}
\begin{itemize}
\item {Utilização:Bras. do N}
\end{itemize}
Raro; que difficilmente se encontra.
\section{Vasquejar}
\begin{itemize}
\item {Grp. gram.:v. i.}
\end{itemize}
Têr vascas; têr convulsões.
Contorcer-se.
Estremecer; tremular.
Agonizar.
\section{Vasquim}
\begin{itemize}
\item {Grp. gram.:m.}
\end{itemize}
\begin{itemize}
\item {Utilização:Bras}
\end{itemize}
Corpete do vestido da mulhér.
(Cp. \textunderscore vasquinha\textunderscore )
\section{Vasquinha}
\begin{itemize}
\item {Grp. gram.:f.}
\end{itemize}
\begin{itemize}
\item {Utilização:Ant.}
\end{itemize}
Saia, com muitas pregas na cintura.
Casaco curto e muito justo ao corpo. Cf. \textunderscore Eufrosina\textunderscore , 21.
(Cast. \textunderscore basquiña\textunderscore )
\section{Vassá}
\begin{itemize}
\item {Grp. gram.:m.}
\end{itemize}
Medida indiana, igual a 20 visvassis.
(Do conc.)
\section{Vassalagem}
\begin{itemize}
\item {Grp. gram.:f.}
\end{itemize}
\begin{itemize}
\item {Proveniência:(De \textunderscore vassalo\textunderscore )}
\end{itemize}
Estado ou condição de vassalo; conjunto de vassalos.
Tributo dos vassalos ao senhor feudal.
Submissão.
\section{Vassalar}
\begin{itemize}
\item {Grp. gram.:v. t.}
\end{itemize}
\begin{itemize}
\item {Utilização:P. us.}
\end{itemize}
Tributar ou prestar como vassalo.
\section{Vassalo}
\begin{itemize}
\item {Grp. gram.:m.}
\end{itemize}
\begin{itemize}
\item {Grp. gram.:Adj.}
\end{itemize}
\begin{itemize}
\item {Proveniência:(Do b. lat. \textunderscore vassalus\textunderscore )}
\end{itemize}
Aquelle que dependia de um senhor feudal; súbdito.
Que paga tributo a alguém; subordinado.
\section{Vassoira}
\begin{itemize}
\item {Grp. gram.:f.}
\end{itemize}
\begin{itemize}
\item {Proveniência:(Do lat. \textunderscore versoria\textunderscore , segundo alguns etymologistas. \textunderscore Versoria\textunderscore  vem de \textunderscore vertere\textunderscore , que não tem nada com o port. \textunderscore varrer\textunderscore ; mas o lat. \textunderscore verrere\textunderscore , varrer, tem supino \textunderscore versum\textunderscore , que daria \textunderscore versoria\textunderscore )}
\end{itemize}
Utensílio, feito de ramos, giestas, piaçaba, etc., destinado especialmente a limpar o lixo ou o pó do pavimento das casas, das ruas, etc.
Nome de diversas plantas.
\section{Vassoirada}
\begin{itemize}
\item {Grp. gram.:f.}
\end{itemize}
Varredela.
Pancada com a vassoira.
Aquillo que se varre com um só movimento de vassoira.
\section{Vassoirar}
\begin{itemize}
\item {Grp. gram.:v. t.}
\end{itemize}
\begin{itemize}
\item {Grp. gram.:V. i.}
\end{itemize}
Varrer com vassoira.
Limpar lixo com vassoira.
\section{Vassoireiro}
\begin{itemize}
\item {Grp. gram.:m.}
\end{itemize}
Fabricante ou vendedor de vassoiras.
Árvore leguminosa do Brasil.
\section{Vassoirinha}
\begin{itemize}
\item {Grp. gram.:f.}
\end{itemize}
\begin{itemize}
\item {Utilização:Bras}
\end{itemize}
Espécie de jôgo de crianças.
Planta malvácea, medicinal.
\section{Vassoirinha-do-brejo}
\begin{itemize}
\item {Grp. gram.:f.}
\end{itemize}
\begin{itemize}
\item {Utilização:Bras}
\end{itemize}
O mesmo que \textunderscore ervão\textunderscore  ou \textunderscore pataqueira\textunderscore . Cf. \textunderscore País\textunderscore , do Rio, de 9-I-901.
\section{Vassoura}
\begin{itemize}
\item {Grp. gram.:f.}
\end{itemize}
\begin{itemize}
\item {Proveniência:(Do lat. \textunderscore versoria\textunderscore , segundo alguns etymologistas. \textunderscore Versoria\textunderscore  vem de \textunderscore vertere\textunderscore , que não tem nada com o port. \textunderscore varrer\textunderscore ; mas o lat. \textunderscore verrere\textunderscore , varrer, tem supino \textunderscore versum\textunderscore , que daria \textunderscore versoria\textunderscore )}
\end{itemize}
Utensílio, feito de ramos, giestas, piaçaba, etc., destinado especialmente a limpar o lixo ou o pó do pavimento das casas, das ruas, etc.
Nome de diversas plantas.
\section{Vassourada}
\begin{itemize}
\item {Grp. gram.:f.}
\end{itemize}
Varredela.
Pancada com a vassoura.
Aquillo que se varre com um só movimento de vassoura.
\section{Vassourar}
\begin{itemize}
\item {Grp. gram.:v. t.}
\end{itemize}
\begin{itemize}
\item {Grp. gram.:V. i.}
\end{itemize}
Varrer com vassoura.
Limpar lixo com vassoura.
\section{Vassoureiro}
\begin{itemize}
\item {Grp. gram.:m.}
\end{itemize}
Fabricante ou vendedor de vassouras.
Árvore leguminosa do Brasil.
\section{Vassourinha}
\begin{itemize}
\item {Grp. gram.:f.}
\end{itemize}
\begin{itemize}
\item {Utilização:Bras}
\end{itemize}
Espécie de jôgo de crianças.
Planta malvácea, medicinal.
\section{Váo}
\begin{itemize}
\item {Grp. gram.:m.}
\end{itemize}
\begin{itemize}
\item {Utilização:Fig.}
\end{itemize}
\begin{itemize}
\item {Grp. gram.:Pl.}
\end{itemize}
\begin{itemize}
\item {Utilização:Náut.}
\end{itemize}
\begin{itemize}
\item {Proveniência:(Do lat. \textunderscore vadum\textunderscore )}
\end{itemize}
Lugar no rio ou no mar, em que a água é tão baixa que se póde transitar a pé.
Baixio; parcel.
Ensejo; commodidade.
Paus, que se cruzam nas gáveas.
Madeiro, em que se assenta a coberta dos navios.
\section{Vassoiro}
\begin{itemize}
\item {Grp. gram.:m.}
\end{itemize}
\begin{itemize}
\item {Proveniência:(De \textunderscore vassoira\textunderscore )}
\end{itemize}
Varredoiro para fornos.
\section{Vassouro}
\begin{itemize}
\item {Grp. gram.:m.}
\end{itemize}
\begin{itemize}
\item {Proveniência:(De \textunderscore vassoura\textunderscore )}
\end{itemize}
Varredouro para fornos.
\section{Vasta}
\begin{itemize}
\item {Grp. gram.:f.}
\end{itemize}
Espécie de rêde de pescar.
\section{Vastar}
\begin{itemize}
\item {Proveniência:(Lat. \textunderscore vastare\textunderscore )}
\end{itemize}
\textunderscore v. t.\textunderscore  (e der.)
O mesmo que \textunderscore devastar\textunderscore , etc.
\section{Vasteza}
\begin{itemize}
\item {fónica:tê}
\end{itemize}
\begin{itemize}
\item {Grp. gram.:f.}
\end{itemize}
O mesmo que \textunderscore vastidão\textunderscore .
\section{Vastidão}
\begin{itemize}
\item {Grp. gram.:f.}
\end{itemize}
\begin{itemize}
\item {Proveniência:(Do lat. \textunderscore vastitudo\textunderscore )}
\end{itemize}
Qualidade do que é vasto.
Grande extensão.
Amplidão; dimensão grande.
Grande desenvolvimento.
Grande importância.
\section{Vasto}
\begin{itemize}
\item {Grp. gram.:adj.}
\end{itemize}
\begin{itemize}
\item {Utilização:Fig.}
\end{itemize}
\begin{itemize}
\item {Proveniência:(Do lat. \textunderscore vastus\textunderscore )}
\end{itemize}
Que tem grande extensão; amplo; muito dilatado.
Grande; importante: \textunderscore vastas riquezas\textunderscore .
Que abrange muitos conhecimentos: \textunderscore vasto saber\textunderscore .
\section{Vatapá}
\begin{itemize}
\item {Grp. gram.:m.}
\end{itemize}
\begin{itemize}
\item {Utilização:Bras}
\end{itemize}
Papas de mandioca com azeite, pimenta e carne ou peixe.
(Or. afr.)
\section{Vate}
\begin{itemize}
\item {Grp. gram.:m.}
\end{itemize}
\begin{itemize}
\item {Proveniência:(Lat. \textunderscore vates\textunderscore )}
\end{itemize}
Aquelle que faz vaticínios; propheta.
Poéta.
\section{Vatel}
\begin{itemize}
\item {Grp. gram.:m.}
\end{itemize}
\begin{itemize}
\item {Utilização:Fig.}
\end{itemize}
\begin{itemize}
\item {Proveniência:(De \textunderscore Vatel\textunderscore , n. p.)}
\end{itemize}
O mesmo que \textunderscore cozinheiro\textunderscore . Cf. B. Pato, \textunderscore Ciprestes\textunderscore , 49 e 275.
\section{Vaticanismo}
\begin{itemize}
\item {Grp. gram.:m.}
\end{itemize}
\begin{itemize}
\item {Proveniência:(De \textunderscore vaticano\textunderscore )}
\end{itemize}
Systema ou partido dos que pugnam pelos interesses moraes e materiaes do Pontífice romano.
O mesmo que \textunderscore papismo\textunderscore .
\section{Vaticanista}
\begin{itemize}
\item {Grp. gram.:m.  e  adj.}
\end{itemize}
Partidário do vaticanismo.
\section{Vaticano}
\begin{itemize}
\item {Grp. gram.:m.}
\end{itemize}
\begin{itemize}
\item {Utilização:Ext.}
\end{itemize}
Palácio do Pontífice, na collina, chamada Vaticano.
Govêrno pontíficio.
Cúria romana.
\section{Vaticinação}
\begin{itemize}
\item {Grp. gram.:f.}
\end{itemize}
\begin{itemize}
\item {Proveniência:(Do lat. \textunderscore vaticinatio\textunderscore )}
\end{itemize}
O mesmo que \textunderscore vaticínio\textunderscore .
\section{Vaticinador}
\begin{itemize}
\item {Grp. gram.:m.  e  adj.}
\end{itemize}
\begin{itemize}
\item {Proveniência:(Do lat. \textunderscore vaticinator\textunderscore )}
\end{itemize}
O que vaticina.
\section{Vaticinante}
\begin{itemize}
\item {Grp. gram.:adj.}
\end{itemize}
\begin{itemize}
\item {Proveniência:(Lat. \textunderscore vaticinans\textunderscore )}
\end{itemize}
Que vaticina.
\section{Vaticinar}
\begin{itemize}
\item {Grp. gram.:v. t.}
\end{itemize}
\begin{itemize}
\item {Proveniência:(Lat. \textunderscore vaticinari\textunderscore )}
\end{itemize}
Proferir como vate.
Predizer, adivinhar, prophetizar.
\section{Vaticínio}
\begin{itemize}
\item {Grp. gram.:m.}
\end{itemize}
\begin{itemize}
\item {Proveniência:(Lat. \textunderscore vaticinium\textunderscore )}
\end{itemize}
Acto ou effeito de vaticinar.
Predicção; prophecia.
\section{Vático}
\begin{itemize}
\item {Grp. gram.:adj.}
\end{itemize}
\begin{itemize}
\item {Utilização:Des.}
\end{itemize}
Relativo a vate; poético:«\textunderscore explosão vática.\textunderscore »Macedo, \textunderscore Motim\textunderscore , I, 76.
\section{Vatídico}
\begin{itemize}
\item {Grp. gram.:adj.}
\end{itemize}
\begin{itemize}
\item {Proveniência:(Do lat. \textunderscore vates\textunderscore  + \textunderscore dicere\textunderscore )}
\end{itemize}
Que faz vaticínios; que é oráculo:«\textunderscore ...o delphim, sýmbolo do deus vatídico\textunderscore  (Apollo)»Castilho, \textunderscore Fastos\textunderscore , II, 609.
\section{Vatinga}
\begin{itemize}
\item {Grp. gram.:f.}
\end{itemize}
Árvore brasileira, própria para construcções.
\section{Vátio}
\begin{itemize}
\item {Grp. gram.:m.}
\end{itemize}
\begin{itemize}
\item {Utilização:Phýs.}
\end{itemize}
\begin{itemize}
\item {Proveniência:(De \textunderscore Walt\textunderscore , n. p.)}
\end{itemize}
Quantidade de trabalho eléctrico, correspondente a um júlio por segundo.
\section{Vatricoso}
\begin{itemize}
\item {Grp. gram.:adj.}
\end{itemize}
\begin{itemize}
\item {Utilização:Des.}
\end{itemize}
\begin{itemize}
\item {Proveniência:(Lat. \textunderscore vatricosus\textunderscore )}
\end{itemize}
Que tem os pés tortos ou defeituosos.
\section{Vátuas}
\begin{itemize}
\item {Grp. gram.:m. pl.}
\end{itemize}
Numerosa tríbo da África austro-oriental, também conhecida pelo nome de \textunderscore Landins\textunderscore .
\section{Vau}
\begin{itemize}
\item {Grp. gram.:m.}
\end{itemize}
\begin{itemize}
\item {Utilização:Fig.}
\end{itemize}
\begin{itemize}
\item {Grp. gram.:Pl.}
\end{itemize}
\begin{itemize}
\item {Utilização:Náut.}
\end{itemize}
\begin{itemize}
\item {Proveniência:(Do lat. \textunderscore vadum\textunderscore )}
\end{itemize}
Lugar no rio ou no mar, em que a água é tão baixa que se póde transitar a pé.
Baixio; parcel.
Ensejo; commodidade.
Paus, que se cruzam nas gáveas.
Madeiro, em que se assenta a coberta dos navios.
\section{Vauló}
\begin{itemize}
\item {Grp. gram.:m.}
\end{itemize}
Árvore da Índia Portuguesa.
\section{Vavavá}
\begin{itemize}
\item {Grp. gram.:m.}
\end{itemize}
\begin{itemize}
\item {Utilização:Bras}
\end{itemize}
\begin{itemize}
\item {Proveniência:(T. onom.)}
\end{itemize}
Tumulto, motim.
\section{Vaza}
\begin{itemize}
\item {Grp. gram.:f.}
\end{itemize}
\begin{itemize}
\item {Utilização:T. de jôgo}
\end{itemize}
Conjunto das cartas, que os parceiros jogam de cada vez ou de cada lance e que são recolhidas pelo que ganha.
(Cast. \textunderscore baza\textunderscore )
\section{Vaza}
\begin{itemize}
\item {Grp. gram.:f.}
\end{itemize}
\begin{itemize}
\item {Proveniência:(De \textunderscore vazar\textunderscore )}
\end{itemize}
Lavor ou feitio vazado ou escavado.
\section{Vàzabarris}
\begin{itemize}
\item {Grp. gram.:m.}
\end{itemize}
\begin{itemize}
\item {Utilização:Fig.}
\end{itemize}
\begin{itemize}
\item {Utilização:Pop.}
\end{itemize}
\begin{itemize}
\item {Proveniência:(De \textunderscore vazar\textunderscore  + \textunderscore barril\textunderscore )}
\end{itemize}
Enseada ou costa, em que se dão muitos naufrágios.
Lugar, onde há riquezas escondidas.
Pantana, ruína: \textunderscore aquella empresa deu em vàzabarris\textunderscore .
\section{Vazadeira}
\begin{itemize}
\item {Grp. gram.:f.}
\end{itemize}
\begin{itemize}
\item {Utilização:Prov.}
\end{itemize}
\begin{itemize}
\item {Proveniência:(De \textunderscore vazar\textunderscore )}
\end{itemize}
Vaso para leite.
\section{Vazadoiro}
\begin{itemize}
\item {Grp. gram.:m.}
\end{itemize}
\begin{itemize}
\item {Proveniência:(De \textunderscore vazar\textunderscore )}
\end{itemize}
Lugar, onde se fazem despejos de immundícies.
Lugar, onde se vaza qualquer líquido. Cf. C. Lobo, \textunderscore Sát. de Jur.\textunderscore , I, 56.
\section{Vazador}
\begin{itemize}
\item {Grp. gram.:m.  e  adj.}
\end{itemize}
\begin{itemize}
\item {Grp. gram.:M.}
\end{itemize}
\begin{itemize}
\item {Proveniência:(De \textunderscore vazar\textunderscore )}
\end{itemize}
O que vaza.
Instrumento de correeiro e de outros artífices, próprio para abrir ilhós.
\section{Vazadouro}
\begin{itemize}
\item {Grp. gram.:m.}
\end{itemize}
\begin{itemize}
\item {Proveniência:(De \textunderscore vazar\textunderscore )}
\end{itemize}
Lugar, onde se fazem despejos de immundícies.
Lugar, onde se vaza qualquer líquido. Cf. C. Lobo, \textunderscore Sát. de Jur.\textunderscore , I, 56.
\section{Vazadura}
\begin{itemize}
\item {Grp. gram.:f.}
\end{itemize}
Acto ou effeito de vazar.
\section{Vazamento}
\begin{itemize}
\item {Grp. gram.:m.}
\end{itemize}
Acto ou effeito de vazar.
\section{Vazante}
\begin{itemize}
\item {Grp. gram.:adj.}
\end{itemize}
\begin{itemize}
\item {Grp. gram.:F.}
\end{itemize}
\begin{itemize}
\item {Utilização:Bras. do N}
\end{itemize}
Que vaza.
Refluxo; vazão.
Horta, que se cultiva nos leitos das torrentes, durante a estação sêca.
\section{Vazão}
\begin{itemize}
\item {Grp. gram.:f.}
\end{itemize}
\begin{itemize}
\item {Utilização:Fig.}
\end{itemize}
O mesmo que \textunderscore vazamento\textunderscore .
Quantidade de fluido, fornecido por qualquer corrente líquida ou gasosa, na unidade de tempo.
Extracção; venda.
Solução, resolução.
\section{Vazar}
\begin{itemize}
\item {Grp. gram.:v. t.}
\end{itemize}
\begin{itemize}
\item {Grp. gram.:V. i.}
\end{itemize}
\begin{itemize}
\item {Grp. gram.:V. p.}
\end{itemize}
\begin{itemize}
\item {Utilização:Des.}
\end{itemize}
\begin{itemize}
\item {Proveniência:(De \textunderscore vaziar\textunderscore , de \textunderscore vazio\textunderscore )}
\end{itemize}
Tornar vazio.
Entornar, despejar.
Verter.
Enterrar.
Furar.
Arrancar: \textunderscore vazar um ôlho\textunderscore .
Cavar, tornar ôco.
Esgotar-se a pouco e pouco.
Entornar-se.
Saír.
Sêr transparente.
Esvaziar-se.
Entornar-se.
Despejar-se.
Dizer quanto sabe, pôr tudo em pratos limpos, revelar quanto ouviu.
\section{Vazeiro}
\begin{itemize}
\item {Grp. gram.:m.}
\end{itemize}
\begin{itemize}
\item {Utilização:Prov.}
\end{itemize}
\begin{itemize}
\item {Utilização:trasm.}
\end{itemize}
\begin{itemize}
\item {Proveniência:(De \textunderscore vazar\textunderscore ?)}
\end{itemize}
Cacete.
Fueiro.
\section{Vazia}
\begin{itemize}
\item {Grp. gram.:f.}
\end{itemize}
\begin{itemize}
\item {Utilização:Pop.}
\end{itemize}
\begin{itemize}
\item {Utilização:Carp.}
\end{itemize}
\begin{itemize}
\item {Utilização:T. de açougue}
\end{itemize}
\begin{itemize}
\item {Proveniência:(De \textunderscore vazio\textunderscore )}
\end{itemize}
Ilharga, quadril.
Peça de madeira, com ranhura numa das faces estreitas, para servir de bitola ao replainado das almofadas.
Parte da perna deanteira do boi, junto á barriga, abaixo da pá.
\section{Vaziador}
\begin{itemize}
\item {Grp. gram.:adj.}
\end{itemize}
Que vazia.
Que estraba muito, (falando-se de animaes).
\section{Vaziamente}
\begin{itemize}
\item {Grp. gram.:m.}
\end{itemize}
Acto ou effeito de vaziar.
\section{Vaziar}
\begin{itemize}
\item {Grp. gram.:v. t.}
\end{itemize}
\begin{itemize}
\item {Grp. gram.:V. i.}
\end{itemize}
\begin{itemize}
\item {Proveniência:(De \textunderscore vazio\textunderscore )}
\end{itemize}
O mesmo que \textunderscore esvaziar\textunderscore .
Estrabar muito, (falando se de animaes)
\section{Vazio}
\begin{itemize}
\item {Grp. gram.:adj.}
\end{itemize}
\begin{itemize}
\item {Grp. gram.:M.}
\end{itemize}
\begin{itemize}
\item {Grp. gram.:Pl.}
\end{itemize}
\begin{itemize}
\item {Proveniência:(Do lat. \textunderscore vacivus\textunderscore , de \textunderscore vacare\textunderscore )}
\end{itemize}
Que não contém nada ou que só contém ar.
Despovoado.
Desoccupado.
Despejado: \textunderscore garrafa vazia\textunderscore .
Fútil.
Destituído: \textunderscore palavras, vazias de sentido\textunderscore .
O mesmo que \textunderscore vácuo\textunderscore .
Ilhargas da cavalgadura.
\section{Vb.}
Abrev. de \textunderscore verbo\textunderscore .
\section{V. E.}
O mesmo que [[v. ex.^a]].
\section{Veação}
\begin{itemize}
\item {Grp. gram.:f.}
\end{itemize}
\begin{itemize}
\item {Proveniência:(Do lat. \textunderscore venatio\textunderscore )}
\end{itemize}
Caça de animaes bravos; montaria.
Iguaria, feita da carne de animaes, mortos na caça.
\section{Veada}
\begin{itemize}
\item {Grp. gram.:f.}
\end{itemize}
O mesmo que cerva.
(Cp. \textunderscore veado\textunderscore ^1)
\section{Veadeira}
\begin{itemize}
\item {Grp. gram.:f.}
\end{itemize}
(?):«\textunderscore ...malefícios como ligamentos, obras de veadeira, carântulas...\textunderscore »Camillo, \textunderscore Anáthema\textunderscore , 147.
\section{Veadeiro}
\begin{itemize}
\item {Grp. gram.:m.}
\end{itemize}
\begin{itemize}
\item {Utilização:Bras}
\end{itemize}
\begin{itemize}
\item {Proveniência:(De \textunderscore veado\textunderscore )}
\end{itemize}
Cão, adestrado na caça dos veados.
Caçador de veados:«\textunderscore Parabens, ó veadeiro, da tua feliz caçada\textunderscore ». Araújo Porto-Alegre.
\section{Veadinho}
\begin{itemize}
\item {Grp. gram.:m.}
\end{itemize}
\begin{itemize}
\item {Utilização:Bras}
\end{itemize}
\begin{itemize}
\item {Proveniência:(De \textunderscore veado\textunderscore ^2)}
\end{itemize}
Espécie de mandioca de talo vermelho e de boas raízes.
\section{Veado}
\begin{itemize}
\item {Grp. gram.:m.}
\end{itemize}
\begin{itemize}
\item {Proveniência:(Do lat. \textunderscore venatus\textunderscore )}
\end{itemize}
Quadrúpede ruminante, de pontas redondas e ramosas.
\section{Veado}
\begin{itemize}
\item {Grp. gram.:m.}
\end{itemize}
\begin{itemize}
\item {Utilização:Bras}
\end{itemize}
Espécie de mandioca, de talo vermelho e raiz curta e grossa.
\section{Veado-canela}
\begin{itemize}
\item {Grp. gram.:m.}
\end{itemize}
\begin{itemize}
\item {Utilização:Bras}
\end{itemize}
Espécie de mandioca, de talo branco e raízes compridas e apparentes.
\section{Veador}
\begin{itemize}
\item {Grp. gram.:m.}
\end{itemize}
\begin{itemize}
\item {Utilização:Ant.}
\end{itemize}
\begin{itemize}
\item {Proveniência:(Do lat. \textunderscore venator\textunderscore )}
\end{itemize}
Aquelle que caça nos montes; monteiro.
\section{Veairo}
\begin{itemize}
\item {Grp. gram.:m.}
\end{itemize}
\begin{itemize}
\item {Utilização:Ant.}
\end{itemize}
O mesmo que \textunderscore loucura\textunderscore ? Cf. G. Vicente.
\section{Vearia}
\begin{itemize}
\item {Grp. gram.:f.}
\end{itemize}
\begin{itemize}
\item {Proveniência:(De um hyp. \textunderscore vear\textunderscore , do lat. \textunderscore venari\textunderscore . Cp. fr. \textunderscore venerie\textunderscore )}
\end{itemize}
Casa, em que se guarda a veação.
\section{Vectação}
\begin{itemize}
\item {Grp. gram.:f.}
\end{itemize}
\begin{itemize}
\item {Utilização:P. us.}
\end{itemize}
\begin{itemize}
\item {Proveniência:(Do lat. \textunderscore vectatio\textunderscore )}
\end{itemize}
Acto de sêr transportado em carro, a cavallo, etc.
\section{Vectões}
\begin{itemize}
\item {Grp. gram.:m. pl.}
\end{itemize}
\begin{itemize}
\item {Proveniência:(Lat. \textunderscore vectones\textunderscore )}
\end{itemize}
O mesmo que \textunderscore vettões\textunderscore .
\section{Vector}
\begin{itemize}
\item {Grp. gram.:adj.}
\end{itemize}
\begin{itemize}
\item {Utilização:Astron.}
\end{itemize}
\begin{itemize}
\item {Utilização:Geom.}
\end{itemize}
\begin{itemize}
\item {Proveniência:(Lat. \textunderscore vector\textunderscore )}
\end{itemize}
Diz-se do raio, que é a distância variável, do centro do Sol ao centro de um planeta.
Diz-se do raio, representado por uma recta, que parte do foco de uma figura geométrica para qualquer ponto da curva da mesma figura.
\section{Veda}
\begin{itemize}
\item {Grp. gram.:m.}
\end{itemize}
Nome de cada um dos quatro mais antigos livros sagrados dos Índios.
(Sânscr. \textunderscore veda\textunderscore )
\section{Veda}
\begin{itemize}
\item {Grp. gram.:f.}
\end{itemize}
Acto de vedar; prohibição:«\textunderscore principia hoje neste concelho o período da veda da caça.\textunderscore »\textunderscore Elvense\textunderscore , de 15-III-900.
\section{Vedação}
\begin{itemize}
\item {Grp. gram.:f.}
\end{itemize}
\begin{itemize}
\item {Proveniência:(Do lat. \textunderscore vedatio\textunderscore )}
\end{itemize}
Acto ou effeito de vedar; aquillo que veda; tapume.
\section{Vedado}
\begin{itemize}
\item {Grp. gram.:adj.}
\end{itemize}
\begin{itemize}
\item {Grp. gram.:M.}
\end{itemize}
\begin{itemize}
\item {Utilização:Ant.}
\end{itemize}
Prohibido.
Que tem tapume; murado: \textunderscore horta vedada\textunderscore .
Terreno vedado, coito.
\section{Vedador}
\begin{itemize}
\item {Grp. gram.:m.  e  adj.}
\end{itemize}
O que veda.
\section{Vedalhas}
\begin{itemize}
\item {Grp. gram.:f. pl.}
\end{itemize}
\begin{itemize}
\item {Utilização:Prov.}
\end{itemize}
\begin{itemize}
\item {Utilização:Prov.}
\end{itemize}
\begin{itemize}
\item {Utilização:trasm.}
\end{itemize}
Presente, que a noiva recebe do padrinho, no dia do noivado.
Presente, que se leva ás parturientes.
(Corr. de \textunderscore vitualhas\textunderscore )
\section{Vedália}
\begin{itemize}
\item {Grp. gram.:f.}
\end{itemize}
Espécie de cochinilha, semelhante á joaninha, mas mais pequena, e inimiga da icérya, de que se sustenta.
\section{Vedanta}
\begin{itemize}
\item {Grp. gram.:m.}
\end{itemize}
Parte theológica dos Vedas.
Sacerdote gentio, que na Índia explica os Vedas ás pessôas da sua casta.
(Sânscr. \textunderscore vedanta\textunderscore )
\section{Vedantino}
\begin{itemize}
\item {Grp. gram.:adj.}
\end{itemize}
Relativo á doutrina do vedanta.
\section{Vedantismo}
\begin{itemize}
\item {Grp. gram.:m.}
\end{itemize}
O systema metaphýsico do vedanta; o vedanta.
\section{Vedar}
\begin{itemize}
\item {Grp. gram.:v. t.}
\end{itemize}
\begin{itemize}
\item {Grp. gram.:V. i.}
\end{itemize}
\begin{itemize}
\item {Proveniência:(Do lat. \textunderscore vetare\textunderscore )}
\end{itemize}
Prohibir; impedir; embaraçar: \textunderscore vedar a caça\textunderscore .
Estancar: \textunderscore vedar o sangue\textunderscore .
Tapar; rodear com tapume.
Impedir que corra: \textunderscore vedar um regato\textunderscore .
Deixar de correr.
Estancar-se.
\section{Vedável}
\begin{itemize}
\item {Grp. gram.:adj.}
\end{itemize}
Que se póde vedar.
\section{Vedeta}
\begin{itemize}
\item {fónica:dê}
\end{itemize}
\begin{itemize}
\item {Grp. gram.:f.}
\end{itemize}
\begin{itemize}
\item {Proveniência:(It. \textunderscore vedetta\textunderscore )}
\end{itemize}
Guarita de sentinela em sítio alto.
Guarda avançada.
Cavalleiro, pôsto de sentinella, e que rapidamente vem dar aviso do que descobriu.
\section{Védico}
\begin{itemize}
\item {Grp. gram.:adj.}
\end{itemize}
Relativo aos Vedas.
\section{Vedo}
\begin{itemize}
\item {Grp. gram.:m.}
\end{itemize}
Árvore indiana, também chamada \textunderscore árvore de gralha\textunderscore , (\textunderscore ficus religiosa\textunderscore , Lin.).
\section{Vedóia}
\begin{itemize}
\item {Grp. gram.:m.}
\end{itemize}
\begin{itemize}
\item {Utilização:Bras. do N}
\end{itemize}
Caloteiro.
Traficante.
Indivíduo ladino, finório.
\section{Vedonho}
\begin{itemize}
\item {Grp. gram.:m.}
\end{itemize}
\begin{itemize}
\item {Utilização:Prov.}
\end{itemize}
\begin{itemize}
\item {Utilização:alg.}
\end{itemize}
O mesmo que \textunderscore vidonho\textunderscore .
\section{Vedor}
\begin{itemize}
\item {Grp. gram.:m.  e  adj.}
\end{itemize}
Aquelle que vê.
Inspector.
Fiscal.
Intendente.
Pesquisador de nascentes de água.
\section{Vedoria}
\begin{itemize}
\item {Grp. gram.:f.}
\end{itemize}
Funcções de vedor.
Repartição, dirigida pelo vedor.
\section{Vedro}
\begin{itemize}
\item {Grp. gram.:adj.}
\end{itemize}
\begin{itemize}
\item {Utilização:Ant.}
\end{itemize}
\begin{itemize}
\item {Proveniência:(Do lat. \textunderscore veter\textunderscore )}
\end{itemize}
O mesmo que \textunderscore velho\textunderscore .
Antigo:«\textunderscore ...cavalleiros vedros e novos.\textunderscore »\textunderscore Nobil. D. Pedro\textunderscore .
\section{Vedro}
\begin{itemize}
\item {Grp. gram.:m.}
\end{itemize}
\begin{itemize}
\item {Utilização:Des.}
\end{itemize}
Vallado.
Tapume, em volta dos campos de lavoíra.
(Relacionam-no geralmente com \textunderscore vedro\textunderscore ^1, não sei porquê)
\section{Veeiro}
\begin{itemize}
\item {Grp. gram.:m.}
\end{itemize}
\begin{itemize}
\item {Utilização:Ant.}
\end{itemize}
\begin{itemize}
\item {Proveniência:(Do lat. \textunderscore velarius\textunderscore )}
\end{itemize}
Pelliça delicada e fina.
\section{Veemência}
\begin{itemize}
\item {Grp. gram.:f.}
\end{itemize}
\begin{itemize}
\item {Proveniência:(Lat. \textunderscore vehementia\textunderscore )}
\end{itemize}
Qualidade do que é veemente.
Impulso rápido, na alma ou nas paixões.
Impetuosidade.
Grande energia.
Intensidade.
Instância.
Rigor.
Vivacidade.
Eloquência comovente.
\section{Veemente}
\begin{itemize}
\item {Grp. gram.:adj.}
\end{itemize}
\begin{itemize}
\item {Proveniência:(Lat. \textunderscore vehemens\textunderscore )}
\end{itemize}
Impetuoso; arrojado.
Animado.
Enérgico, irritável.
Enthusiástico.
Fervoroso.
\section{Veementemente}
\begin{itemize}
\item {Grp. gram.:adv.}
\end{itemize}
De modo veemente.
\section{Vega}
\begin{itemize}
\item {Grp. gram.:f.}
\end{itemize}
\begin{itemize}
\item {Utilização:Ant.}
\end{itemize}
O mesmo que \textunderscore vez\textunderscore . Cf. \textunderscore Port. Mon. Hist.\textunderscore , \textunderscore Script.\textunderscore , 240.
\section{Vegada}
\begin{itemize}
\item {Grp. gram.:f.}
\end{itemize}
\begin{itemize}
\item {Utilização:Ant.}
\end{itemize}
O mesmo que \textunderscore vez\textunderscore . Cf. \textunderscore Port. Mon. Hist.\textunderscore , \textunderscore Script.\textunderscore , 240.
\section{Vegetabilidade}
\begin{itemize}
\item {Grp. gram.:f.}
\end{itemize}
\begin{itemize}
\item {Proveniência:(Do lat. \textunderscore vegetabilis\textunderscore )}
\end{itemize}
Qualidade do que é vegetável.
\section{Vegetação}
\begin{itemize}
\item {Grp. gram.:f.}
\end{itemize}
\begin{itemize}
\item {Utilização:Med.}
\end{itemize}
\begin{itemize}
\item {Proveniência:(Do lat. \textunderscore vegetatio\textunderscore )}
\end{itemize}
Acto ou effeito de vegetar.
Fôrça vegetativa.
Os vegetaes.
Produto chímico que, na sua crystallização, dá o aspecto de planta.
Excrescência mórbida, de tecido mais ou menos esponjoso.
\section{Vegetal}
\begin{itemize}
\item {Grp. gram.:adj.}
\end{itemize}
\begin{itemize}
\item {Grp. gram.:M.}
\end{itemize}
\begin{itemize}
\item {Proveniência:(De \textunderscore vegetar\textunderscore )}
\end{itemize}
Relativo ás plantas: \textunderscore o reino vegetal\textunderscore .
Proveniente de plantas.
Planta.
Corpo orgânico, que vegeta.
\section{Vegetaliano}
\begin{itemize}
\item {Grp. gram.:adj.}
\end{itemize}
\begin{itemize}
\item {Utilização:Neol.}
\end{itemize}
\begin{itemize}
\item {Proveniência:(De \textunderscore vegetal\textunderscore )}
\end{itemize}
Relativo aos vegetaes ou ao vegetalismo.
\section{Vegetalidade}
\begin{itemize}
\item {Grp. gram.:f.}
\end{itemize}
Qualidade do que é vegetal.
Conjunto de vegetaes.
\section{Vegetalina}
\begin{itemize}
\item {Grp. gram.:f.}
\end{itemize}
\begin{itemize}
\item {Proveniência:(De \textunderscore vegetalino\textunderscore )}
\end{itemize}
Antídoto contra o veneno ophídico.
\section{Vegetalino}
\begin{itemize}
\item {Grp. gram.:adj.}
\end{itemize}
\begin{itemize}
\item {Proveniência:(De \textunderscore vegetal\textunderscore )}
\end{itemize}
Relativo a vegetaes.
\section{Vegetalismo}
\begin{itemize}
\item {Grp. gram.:m.}
\end{itemize}
\begin{itemize}
\item {Utilização:Neol.}
\end{itemize}
\begin{itemize}
\item {Proveniência:(De \textunderscore vegetal\textunderscore )}
\end{itemize}
Systema dos vegetalistas.
Estilo architectónico, em que predominam as fórmas e ornatos vegetaes.
\section{Vegetalista}
\begin{itemize}
\item {Grp. gram.:m.  e  adj.}
\end{itemize}
Partidário da alimentação exclusivamente vegetal.
\section{Vegetalizar}
\begin{itemize}
\item {Grp. gram.:v. t.}
\end{itemize}
Dar fórma de vegetal a.
\section{Vegetante}
\begin{itemize}
\item {Grp. gram.:adj.}
\end{itemize}
\begin{itemize}
\item {Proveniência:(Lat. \textunderscore vegetans\textunderscore )}
\end{itemize}
Que vegeta.
\section{Vegetar}
\begin{itemize}
\item {Grp. gram.:v. i.}
\end{itemize}
\begin{itemize}
\item {Utilização:Fig.}
\end{itemize}
\begin{itemize}
\item {Grp. gram.:V. t.}
\end{itemize}
\begin{itemize}
\item {Utilização:P. us.}
\end{itemize}
\begin{itemize}
\item {Proveniência:(Lat. \textunderscore vegetare\textunderscore )}
\end{itemize}
Viver e desenvolver-se, (falando-se de plantas).
Viver na inércia ou inactividade.
Não sentir commoções.
Desenvolver, nutrir.
\section{Vegetariano}
\begin{itemize}
\item {Grp. gram.:adj.}
\end{itemize}
(Palavra mal formada, em vez de \textunderscore vegetalino\textunderscore  ou \textunderscore vegetaliano\textunderscore ).
\section{Vegetário}
\begin{itemize}
\item {Grp. gram.:m.}
\end{itemize}
\begin{itemize}
\item {Proveniência:(De \textunderscore vegetar\textunderscore )}
\end{itemize}
Aquelle que se alimenta só de vegetaes:«\textunderscore tanto banquete para cem vegetários.\textunderscore »C. Lobo, \textunderscore Sát. de Juv.\textunderscore , I, 140.
\section{Vegetarismo}
\begin{itemize}
\item {Grp. gram.:m.}
\end{itemize}
Systema alimentício dos vegetaristas.
\section{Vegetarista}
\begin{itemize}
\item {Grp. gram.:m.}
\end{itemize}
Fórma convencional, para designar o \textunderscore vegetalista\textunderscore  que aceita alguns alimentos de origem animal, como o queijo, o leite, os ovos.
\section{Vegetativo}
\begin{itemize}
\item {Grp. gram.:adj.}
\end{itemize}
Que faz vegetar.
Relativo a vegetaes e animaes: \textunderscore a vida vegetativa\textunderscore .
\section{Vegetável}
\begin{itemize}
\item {Grp. gram.:adj.}
\end{itemize}
\begin{itemize}
\item {Proveniência:(Do lat. \textunderscore vegetabilis\textunderscore )}
\end{itemize}
Que póde vegetar.
\section{Vegete}
\begin{itemize}
\item {fónica:gê}
\end{itemize}
\begin{itemize}
\item {Grp. gram.:m.}
\end{itemize}
\begin{itemize}
\item {Utilização:Gír.}
\end{itemize}
Homem velho e ridículo ou mal vestido.
Amante velho.
(Cast. \textunderscore vegete\textunderscore )
\section{Vegetívoro}
\begin{itemize}
\item {Grp. gram.:adj.}
\end{itemize}
\begin{itemize}
\item {Utilização:bras}
\end{itemize}
\begin{itemize}
\item {Utilização:Neol.}
\end{itemize}
\begin{itemize}
\item {Proveniência:(De \textunderscore vegetal\textunderscore  + lat. \textunderscore vorare\textunderscore )}
\end{itemize}
Que se alimenta de vegetaes.
\section{Vegeto}
\begin{itemize}
\item {Grp. gram.:adj.}
\end{itemize}
\begin{itemize}
\item {Proveniência:(Lat. \textunderscore vegetus\textunderscore )}
\end{itemize}
O mesmo que \textunderscore vegetativo\textunderscore .
Robusto.
\section{Vegeto-animal}
\begin{itemize}
\item {Grp. gram.:adj.}
\end{itemize}
Que participa da natureza dos animaes e dos vegetaes.
\section{Vegeto-mineral}
\begin{itemize}
\item {Grp. gram.:adj.}
\end{itemize}
Que participa da natureza dos mineraes e dos vegetaes.
\section{Vehemência}
\begin{itemize}
\item {Grp. gram.:f.}
\end{itemize}
\begin{itemize}
\item {Proveniência:(Lat. \textunderscore vehementia\textunderscore )}
\end{itemize}
Qualidade do que é vehemente.
Impulso rápido, na alma ou nas paixões.
Impetuosidade.
Grande energia.
Intensidade.
Instância.
Rigor.
Vivacidade.
Eloquência commovente.
\section{Vehemente}
\begin{itemize}
\item {Grp. gram.:adj.}
\end{itemize}
\begin{itemize}
\item {Proveniência:(Lat. \textunderscore vehemens\textunderscore )}
\end{itemize}
Impetuoso; arrojado.
Animado.
Enérgico, irritável.
Enthusiástico.
Fervoroso.
\section{Vehementemente}
\begin{itemize}
\item {Grp. gram.:adv.}
\end{itemize}
De modo vehemente.
\section{Vehiculação}
\begin{itemize}
\item {Grp. gram.:f.}
\end{itemize}
\begin{itemize}
\item {Utilização:bras}
\end{itemize}
\begin{itemize}
\item {Utilização:Neol.}
\end{itemize}
Viação, por meio de vehículos.
\section{Vehicular}
\begin{itemize}
\item {Grp. gram.:adj.}
\end{itemize}
Próprio de vehículo; relativo a vehículo.
\section{Vehicular}
\begin{itemize}
\item {Grp. gram.:v. t.}
\end{itemize}
\begin{itemize}
\item {Utilização:Neol.}
\end{itemize}
\begin{itemize}
\item {Utilização:Ext.}
\end{itemize}
Transportar em vehículo.
Transportar; introduzir, importar: \textunderscore a peste foi vehiculada pelos emigrantes\textunderscore . Cf. R. Jorge, \textunderscore Epidemia do Pôrto\textunderscore , 37.
\section{Vehículo}
\begin{itemize}
\item {Grp. gram.:m.}
\end{itemize}
\begin{itemize}
\item {Proveniência:(Lat. \textunderscore vehiculum\textunderscore )}
\end{itemize}
Qualquer meio de transporte.
Carro.
Tudo que transmitte ou conduz.
Aquillo que auxilia ou promove.
Mênstruo.
Excipiente líquido.
\section{Veiculação}
\begin{itemize}
\item {fónica:ve-í}
\end{itemize}
\begin{itemize}
\item {Grp. gram.:f.}
\end{itemize}
\begin{itemize}
\item {Utilização:bras}
\end{itemize}
\begin{itemize}
\item {Utilização:Neol.}
\end{itemize}
Viação, por meio de veículos.
\section{Veicular}
\begin{itemize}
\item {fónica:ve-í}
\end{itemize}
\begin{itemize}
\item {Grp. gram.:adj.}
\end{itemize}
Próprio de veículo; relativo a veículo.
\section{Veicular}
\begin{itemize}
\item {fónica:ve-í}
\end{itemize}
\begin{itemize}
\item {Grp. gram.:v. t.}
\end{itemize}
\begin{itemize}
\item {Utilização:Neol.}
\end{itemize}
\begin{itemize}
\item {Utilização:Ext.}
\end{itemize}
Transportar em veículo.
Transportar; introduzir, importar: \textunderscore a peste foi veiculada pelos emigrantes\textunderscore . Cf. R. Jorge, \textunderscore Epidemia do Pôrto\textunderscore , 37.
\section{Veículo}
\begin{itemize}
\item {Grp. gram.:m.}
\end{itemize}
\begin{itemize}
\item {Proveniência:(Lat. \textunderscore vehiculum\textunderscore )}
\end{itemize}
Qualquer meio de transporte.
Carro.
Tudo que transmite ou conduz.
Aquilo que auxilia ou promove.
Mênstruo.
Excipiente líquido.
\section{Vei}
\begin{itemize}
\item {Grp. gram.:m.}
\end{itemize}
Língua da África setentrional.
\section{Veia}
\begin{itemize}
\item {Grp. gram.:f.}
\end{itemize}
\begin{itemize}
\item {Utilização:Anat.}
\end{itemize}
\begin{itemize}
\item {Utilização:Fig.}
\end{itemize}
\begin{itemize}
\item {Utilização:Bot.}
\end{itemize}
\begin{itemize}
\item {Proveniência:(Do lat. \textunderscore vena\textunderscore )}
\end{itemize}
Canal tênue, que leva ao coração o sangue distribuído pelas artérias em todas as partes do corpo.
Qualquer dos vasos sanguíneos.
Tendência, vocação: \textunderscore tem veia para a poesia\textunderscore .
Qualidade.
Âmago.
Meio de communicação.
Veio (de água).
Cada uma das nervuras secundárias das fôlhas dos vegetaes.
* \textunderscore Veia da arca\textunderscore , ou \textunderscore veia real\textunderscore , o mesmo que \textunderscore salvatella\textunderscore .
\section{Veiar}
\begin{itemize}
\item {Grp. gram.:v. t.}
\end{itemize}
Formar veios em. Cf. C. Neto, \textunderscore Baladilhas\textunderscore , 197.
\section{Veiga}
\begin{itemize}
\item {Grp. gram.:f.}
\end{itemize}
\begin{itemize}
\item {Utilização:Prov.}
\end{itemize}
\begin{itemize}
\item {Utilização:minh.}
\end{itemize}
\begin{itemize}
\item {Proveniência:(Do cast. \textunderscore vega\textunderscore )}
\end{itemize}
Várzea; planície cultivada e fértil.
Terra de cultura de centeio ou de milho serôdio.
\section{Veio}
\begin{itemize}
\item {Grp. gram.:m.}
\end{itemize}
\begin{itemize}
\item {Utilização:Bras. do N}
\end{itemize}
\begin{itemize}
\item {Utilização:Fig.}
\end{itemize}
\begin{itemize}
\item {Proveniência:(De \textunderscore veia\textunderscore )}
\end{itemize}
Faixa comprida e mais ou menos estreita de terra ou de rocha, a qual se distingue, pela côr ou pela natureza, da terra ou da rocha que a ladeia; filão.
Ribeiro, regato.
Pequena porção de água corrente.
Eixo de ferro.
Manivela.
Ponto capital; fundamento; essência.
\section{Veirado}
\begin{itemize}
\item {Grp. gram.:adj.}
\end{itemize}
Que tem veiros.
\section{Veiro}
\begin{itemize}
\item {Grp. gram.:m.}
\end{itemize}
\begin{itemize}
\item {Utilização:Heráld.}
\end{itemize}
\begin{itemize}
\item {Grp. gram.:Pl.}
\end{itemize}
\begin{itemize}
\item {Proveniência:(Do fr. \textunderscore vair\textunderscore )}
\end{itemize}
Cada um dos metaes dos brasões, composto de pequenas peças azues e prateadas, iguaes e dispostas de maneira, que a ponta das peças azues é opposta á das peças prateadas, e a base de umas á base das outras.
Pelles delicadas e preciosas, taes como arminho, zebelinas, etc., que se importavam da Hungria e de outras nações. Cf. \textunderscore Orden. Aff.\textunderscore , liv. V, tit. 43.
\section{Veiza}
\begin{itemize}
\item {Grp. gram.:f.}
\end{itemize}
\begin{itemize}
\item {Utilização:Ant.}
\end{itemize}
O mesmo que \textunderscore hortaliça\textunderscore .
\section{Veja}
\begin{itemize}
\item {Grp. gram.:f.}
\end{itemize}
Peixe dos Açores.
\section{Vela}
\begin{itemize}
\item {Grp. gram.:f.}
\end{itemize}
\begin{itemize}
\item {Proveniência:(De \textunderscore velar\textunderscore ^1)}
\end{itemize}
Acto de velar^2.
Veladura.
Sentinella: \textunderscore estar de vela\textunderscore .
Pessôa, que vigia.
Peça cylíndrica, de substância gorda e combustível, que tem ao centro, em todo o seu comprimento, um pavio.
\textunderscore Vela mística\textunderscore , preparado, com que se communica fogo a certas peças de artilharia.
\textunderscore Vela Maria\textunderscore , a vela mais alta do candelabro triangular, que se usa nos officios da Semana Santa.
\textunderscore Vela eléctrica\textunderscore , conjunto dos carvões que, nos apparelhos de illuminação, produzem a luz eléctrica.
\textunderscore Estar de vela\textunderscore , estar acordado; estar velando ou vigiando.
\section{Vela}
\begin{itemize}
\item {Grp. gram.:f.}
\end{itemize}
\begin{itemize}
\item {Utilização:Fig.}
\end{itemize}
\begin{itemize}
\item {Grp. gram.:Loc. adv.}
\end{itemize}
\begin{itemize}
\item {Utilização:Pop.}
\end{itemize}
\begin{itemize}
\item {Proveniência:(Lat. \textunderscore vela\textunderscore )}
\end{itemize}
Pano de vário feitio, que se prende ao mastro da embarcação, e que, sob a acção do vento, faz vogar a mesma embarcação ou lhe facilita o movimento.
Peça de pano, que se prende aos braços dos moínhos de vento, para imprimir movimento á mó.
Navio: \textunderscore lá vêm as velas do Gama\textunderscore .
\textunderscore Á vela\textunderscore , com as velas desfraldadas: \textunderscore navegar á vela\textunderscore .
Em camisa; de corpo descoberto.
\textunderscore Fazer-se de vela\textunderscore  ou \textunderscore fazer-se á vela\textunderscore , começar a navegar, saír de um pôrto.
\section{Velacho}
\begin{itemize}
\item {Grp. gram.:m.}
\end{itemize}
\begin{itemize}
\item {Utilização:Bras. do N}
\end{itemize}
\begin{itemize}
\item {Proveniência:(De \textunderscore vela\textunderscore  ^2)}
\end{itemize}
Vela dos mastros da prôa.
Appellido; alcunha.
\section{Velado}
\begin{itemize}
\item {Grp. gram.:m.}
\end{itemize}
\begin{itemize}
\item {Utilização:Bras}
\end{itemize}
O mesmo que \textunderscore avelado\textunderscore , (falando-se do côco).
\section{Velador}
\begin{itemize}
\item {Grp. gram.:m}
\end{itemize}
\begin{itemize}
\item {Grp. gram.:Adj.}
\end{itemize}
\begin{itemize}
\item {Proveniência:(De \textunderscore velar\textunderscore  ^1)}
\end{itemize}
Aquelle que vela.
Utensílio, formado por uma haste de madeira com peanha, na extremidade superior do qual se colloca uma candeia, candeeiro ou vela.
Que vela, que faz velar; que vigia.
\section{Veladura}
\begin{itemize}
\item {Grp. gram.:f.}
\end{itemize}
\begin{itemize}
\item {Proveniência:(De \textunderscore velar\textunderscore ^2)}
\end{itemize}
Acto de velar.
Ligeira mão de tinta, applicada numa pintura, deixando transparecer a tinta que está por baixo.
Velatura.
\section{Velame}
\begin{itemize}
\item {Grp. gram.:m.}
\end{itemize}
\begin{itemize}
\item {Proveniência:(Lat. \textunderscore velamen\textunderscore )}
\end{itemize}
Porção de velas náuticas, ou o conjunto das velas de um navio.
Disfarce; cobertura.
\section{Velame}
\begin{itemize}
\item {Grp. gram.:m.}
\end{itemize}
Erva medicinal do Brasil.
\section{Velame-do-campo}
\begin{itemize}
\item {Grp. gram.:m.}
\end{itemize}
\begin{itemize}
\item {Utilização:Bras}
\end{itemize}
Planta euphorbiácea medicinal, (\textunderscore croton campestris\textunderscore ).
\section{Velame-do-mato}
\begin{itemize}
\item {Grp. gram.:m.}
\end{itemize}
\begin{itemize}
\item {Utilização:Bras. de S. Paulo}
\end{itemize}
Planta, o mesmo que \textunderscore bolsa-do-pastor\textunderscore .
\section{Velâmen}
\begin{itemize}
\item {Grp. gram.:m.}
\end{itemize}
O mesmo que \textunderscore velame\textunderscore ^1.
\section{Velamento}
\begin{itemize}
\item {Grp. gram.:m.}
\end{itemize}
\begin{itemize}
\item {Proveniência:(Lat. \textunderscore velamentum\textunderscore )}
\end{itemize}
Acto ou effeito de velar^2.
Velame.
\section{Velame-verdadeiro}
\begin{itemize}
\item {Grp. gram.:m.}
\end{itemize}
\begin{itemize}
\item {Utilização:Bras}
\end{itemize}
O mesmo que \textunderscore velame-do-campo\textunderscore .
\section{Velaminar}
\begin{itemize}
\item {Grp. gram.:adj.}
\end{itemize}
\begin{itemize}
\item {Utilização:Bot.}
\end{itemize}
\begin{itemize}
\item {Proveniência:(Do lat. \textunderscore velamen\textunderscore )}
\end{itemize}
Diz-se de certos órgãos vegetaes, que se desenvolvem como uma vela.
\section{Velar}
\begin{itemize}
\item {Grp. gram.:v. t.}
\end{itemize}
\begin{itemize}
\item {Utilização:Fig.}
\end{itemize}
\begin{itemize}
\item {Grp. gram.:V. i.}
\end{itemize}
\begin{itemize}
\item {Proveniência:(Do lat. \textunderscore vigilare\textunderscore )}
\end{itemize}
O mesmo que \textunderscore vigiar\textunderscore .
Passar sem dormir: \textunderscore velei duas noites\textunderscore .
Proteger.
Dispensar cuidados a.
Passar a noite sem dormir.
Conservar-se acceso, (falando-se de candeeiro, castiçal, etc.).
Interessar-se; têr vigilância: \textunderscore eu velo por êlle\textunderscore .
\section{Velar}
\begin{itemize}
\item {Grp. gram.:v. t.}
\end{itemize}
\begin{itemize}
\item {Utilização:Fig.}
\end{itemize}
\begin{itemize}
\item {Proveniência:(Lat. \textunderscore velare\textunderscore )}
\end{itemize}
Encobrir com véu.
Encobrir.
Esconder.
Tornar escuro.
Pôr velatura em.
Tornar sombrio; anuvear.
\section{Velário}
\begin{itemize}
\item {Grp. gram.:m.}
\end{itemize}
\begin{itemize}
\item {Proveniência:(Lat. \textunderscore velarium\textunderscore )}
\end{itemize}
Tôldo, com que, entre os antigos, se cobriam os circos e theatros, por causa da chuva. Cf. G. Crespo, \textunderscore Nocturnos\textunderscore , 87.
\section{Velatura}
\begin{itemize}
\item {Grp. gram.:f.}
\end{itemize}
\begin{itemize}
\item {Proveniência:(Lat. \textunderscore velatura\textunderscore )}
\end{itemize}
Acto de cobrir uma pintura com uma ligeira mão de tinta, de fórma que transpareça a tinta que está por baixo. Cp. \textunderscore veladura\textunderscore .
\section{Vele}
\begin{itemize}
\item {Grp. gram.:m.}
\end{itemize}
\begin{itemize}
\item {Utilização:Chapel.}
\end{itemize}
Dá-se êste nome ao pêlo de coêlho, antes de receber a acção do mercúrio e água forte.
(Provavelmente, corr. de \textunderscore vello\textunderscore )
\section{Vélea}
\begin{itemize}
\item {Grp. gram.:f.}
\end{itemize}
Gênero de plantas umbellíferas.
Gênero de plantas crucíferas.
\section{Velear}
\begin{itemize}
\item {Grp. gram.:v. t.}
\end{itemize}
Prover de velas^2 (o navio).
\section{Veleira}
\begin{itemize}
\item {Grp. gram.:f.}
\end{itemize}
Criada de freiras, para serviço fóra dos conventos.
(Fem. de \textunderscore veleiro\textunderscore ^1)
\section{Veleiro}
\begin{itemize}
\item {Grp. gram.:m.}
\end{itemize}
\begin{itemize}
\item {Proveniência:(De \textunderscore vela\textunderscore  ^1)}
\end{itemize}
Criado de frades, para serviço fóra dos conventos.
\section{Veleiro}
\begin{itemize}
\item {Grp. gram.:adj.}
\end{itemize}
\begin{itemize}
\item {Utilização:Ext.}
\end{itemize}
\begin{itemize}
\item {Grp. gram.:M.}
\end{itemize}
\begin{itemize}
\item {Proveniência:(De \textunderscore vela\textunderscore ^2)}
\end{itemize}
Que anda bem á vela: \textunderscore barco veleiro\textunderscore .
Que anda ou que se move facilmente; ligeiro: \textunderscore rapariga veleira\textunderscore .
Aquelle que faz velas de navio.
\section{Velejar}
\begin{itemize}
\item {Grp. gram.:v. i.}
\end{itemize}
\begin{itemize}
\item {Proveniência:(De \textunderscore vela\textunderscore ^2)}
\end{itemize}
Navegar á vela; navegar.
\section{Velenho}
\begin{itemize}
\item {Grp. gram.:m.}
\end{itemize}
O mesmo que \textunderscore meimendro\textunderscore .
\section{Veleta}
\begin{itemize}
\item {fónica:lê}
\end{itemize}
\begin{itemize}
\item {Grp. gram.:f.}
\end{itemize}
\begin{itemize}
\item {Utilização:Fig.}
\end{itemize}
\begin{itemize}
\item {Proveniência:(De \textunderscore vela\textunderscore ^2)}
\end{itemize}
O mesmo que \textunderscore catavento\textunderscore .
Pessôa volúvel, inconstante.
\section{Velga}
\begin{itemize}
\item {Grp. gram.:f.}
\end{itemize}
\begin{itemize}
\item {Utilização:Prov.}
\end{itemize}
O mesmo que \textunderscore belga\textunderscore ^1.
\section{Velha}
\begin{itemize}
\item {Grp. gram.:f.}
\end{itemize}
\begin{itemize}
\item {Utilização:fam.}
\end{itemize}
\begin{itemize}
\item {Utilização:Fig.}
\end{itemize}
\begin{itemize}
\item {Utilização:Pop.}
\end{itemize}
\begin{itemize}
\item {Grp. gram.:Loc.}
\end{itemize}
\begin{itemize}
\item {Utilização:fam.}
\end{itemize}
\begin{itemize}
\item {Proveniência:(Do lat. \textunderscore vetula\textunderscore )}
\end{itemize}
Mulhér avançada em idade.
O mesmo que \textunderscore morte\textunderscore .
\textunderscore Arco da velha\textunderscore , arco íris.
\textunderscore Coisas do arco da velha\textunderscore , coisas extraordinárias, espaventosas.
\section{Velhaca}
\begin{itemize}
\item {Grp. gram.:f.}
\end{itemize}
\begin{itemize}
\item {Proveniência:(De \textunderscore velhaco\textunderscore )}
\end{itemize}
Mulhér fraudulenta, traiçoeira.
Mulhér brejeira, maliciosa.
\section{Velhacada}
\begin{itemize}
\item {Grp. gram.:f}
\end{itemize}
Acto de velhaco.
Reunião de velhacos.
\section{Velhacamente}
\begin{itemize}
\item {Grp. gram.:adv.}
\end{itemize}
De modo velhaco.
Com velhacaria.
\section{Velhacão}
\begin{itemize}
\item {Grp. gram.:m.}
\end{itemize}
Grande velhaco.
\section{Velhacar}
\begin{itemize}
\item {Grp. gram.:v. i.}
\end{itemize}
\begin{itemize}
\item {Proveniência:(De \textunderscore velhaco\textunderscore )}
\end{itemize}
Praticar velhacarias. Cf. Castilho, \textunderscore Felic. pela Agr.\textunderscore 
\section{Velhacaria}
\begin{itemize}
\item {Grp. gram.:f.}
\end{itemize}
Velhacada.
Qualidade do que é velhaco.
Manha de velhaco.
\section{Vèlhaças}
\begin{itemize}
\item {Grp. gram.:m.}
\end{itemize}
\begin{itemize}
\item {Utilização:Fam.}
\end{itemize}
Homem muito velho.
\section{Velhacaz}
\begin{itemize}
\item {Grp. gram.:m.}
\end{itemize}
O mesmo que \textunderscore velhacão\textunderscore .
\section{Velhaco}
\begin{itemize}
\item {Grp. gram.:m.}
\end{itemize}
\begin{itemize}
\item {Grp. gram.:Adj.}
\end{itemize}
\begin{itemize}
\item {Proveniência:(Do cast. \textunderscore bellaco\textunderscore ?)}
\end{itemize}
Aquelle que engana de propósito ou por má índole.
Indivíduo traiçoeiro, fraudulento.
Patife.
Devasso.
Brejeiro.
Que é velhaco; próprio de velhacos.
Diz-se de uma variedade de feijão minhoto.
\section{Vèlhada}
\begin{itemize}
\item {Grp. gram.:f.}
\end{itemize}
Acto ou dito próprio de velho.
Reunião de velhos; os velhos.
\section{Velhancão}
\begin{itemize}
\item {Grp. gram.:m.  e  adj.}
\end{itemize}
(Corr. de \textunderscore velhacão\textunderscore )
\section{Velhancaria}
\begin{itemize}
\item {Grp. gram.:f.}
\end{itemize}
(Corr. de \textunderscore velhacaría\textunderscore )
\section{Velhão}
\begin{itemize}
\item {Grp. gram.:adj.}
\end{itemize}
\begin{itemize}
\item {Utilização:P. us.}
\end{itemize}
Muito velho.
\section{Velhaqueadoiro}
\begin{itemize}
\item {Grp. gram.:m.}
\end{itemize}
\begin{itemize}
\item {Utilização:Bras}
\end{itemize}
\begin{itemize}
\item {Proveniência:(De \textunderscore velhaquear\textunderscore ^2)}
\end{itemize}
Virilha do cavallo.
\section{Velhaqueadouro}
\begin{itemize}
\item {Grp. gram.:m.}
\end{itemize}
\begin{itemize}
\item {Utilização:Bras}
\end{itemize}
\begin{itemize}
\item {Proveniência:(De \textunderscore velhaquear\textunderscore ^2)}
\end{itemize}
Virilha do cavallo.
\section{Velhaquear}
\begin{itemize}
\item {Grp. gram.:v. i.}
\end{itemize}
\begin{itemize}
\item {Grp. gram.:V. t.}
\end{itemize}
Proceder como velhaco.
Burlar, enganar.
\section{Velhaquear}
\begin{itemize}
\item {Grp. gram.:v. i.}
\end{itemize}
\begin{itemize}
\item {Utilização:Bras}
\end{itemize}
Fazer dar corcovos ao cavallo.
\section{Velhaquesco}
\begin{itemize}
\item {fónica:quês}
\end{itemize}
\begin{itemize}
\item {Grp. gram.:adj.}
\end{itemize}
Relativo a velhaco; próprio de velhaco.
\section{Velhaqueta}
\begin{itemize}
\item {fónica:quê}
\end{itemize}
\begin{itemize}
\item {Grp. gram.:f.}
\end{itemize}
(Flex. fem. de \textunderscore velhaquete\textunderscore )
\section{Velhaquete}
\begin{itemize}
\item {fónica:quê}
\end{itemize}
\begin{itemize}
\item {Grp. gram.:m.  e  adj.}
\end{itemize}
\begin{itemize}
\item {Proveniência:(De \textunderscore velhaco\textunderscore )}
\end{itemize}
Indivíduo sonso, mas um tanto velhaco.
\section{Velhaquez}
\begin{itemize}
\item {Grp. gram.:f.}
\end{itemize}
Qualidade de velhaco. Cf. Júl. Diniz, \textunderscore Morgadinha\textunderscore , 140.
\section{Velharaco}
\begin{itemize}
\item {Grp. gram.:m.}
\end{itemize}
\begin{itemize}
\item {Utilização:T. da Bairrada}
\end{itemize}
Espécie de filhó.
\section{Velharia}
\begin{itemize}
\item {Grp. gram.:f.}
\end{itemize}
\begin{itemize}
\item {Proveniência:(De \textunderscore velho\textunderscore )}
\end{itemize}
Acto, dito ou tudo aquillo que é próprio de pessôa idosa.
Traste ou objecto antigo.
Costume antiquado.
Termo ou locução antiga.
\section{Vèlhentado}
\begin{itemize}
\item {Grp. gram.:adj.}
\end{itemize}
(V.avelhentado)
\section{Vèlhez}
\begin{itemize}
\item {Grp. gram.:f.}
\end{itemize}
\begin{itemize}
\item {Utilização:P. us.}
\end{itemize}
O mesmo que \textunderscore velhice\textunderscore . Cf. Filinto, VI, 210.
\section{Velhice}
\begin{itemize}
\item {Grp. gram.:f.}
\end{itemize}
Condição ou estado de velho.
Idade avançada.
Período que, na vida do indivíduo, succede á idade madura.
As pessôas velhas.
Rabugice, própria de velho.
\section{Vèlhice}
\begin{itemize}
\item {Grp. gram.:f.}
\end{itemize}
Condição ou estado de velho.
Idade avançada.
Período que, na vida do indivíduo, succede á idade madura.
As pessôas velhas.
Rabugice, própria de velho.
\section{Velho}
\begin{itemize}
\item {Grp. gram.:adj.}
\end{itemize}
\begin{itemize}
\item {Utilização:Prov.}
\end{itemize}
\begin{itemize}
\item {Utilização:Agr.}
\end{itemize}
\begin{itemize}
\item {Utilização:minh.}
\end{itemize}
\begin{itemize}
\item {Grp. gram.:M.}
\end{itemize}
\begin{itemize}
\item {Utilização:Bras}
\end{itemize}
\begin{itemize}
\item {Utilização:Prov.}
\end{itemize}
\begin{itemize}
\item {Utilização:Ant.}
\end{itemize}
\begin{itemize}
\item {Proveniência:(Do lat. \textunderscore vetulus\textunderscore )}
\end{itemize}
Que tem muita idade: \textunderscore homem velho\textunderscore .
Antigo: \textunderscore velhas tradições\textunderscore .
Que existe há muito tempo.
Avèlhentado.
Muito usado; gasto pelo uso: \textunderscore chapéu velho\textunderscore .
Que há muito possue certa qualidade, ou que exerce há muito uma profissão: \textunderscore o velho professor\textunderscore .
Desusado; antiquado.
\textunderscore Deixar\textunderscore  ou \textunderscore ficar de velho\textunderscore , deixar ou ficar de poisio.
Homem velho.
Nome de um peixe, que parece gemer quando o apanham. Cf. \textunderscore Jorn.-do-Comm.\textunderscore , do Rio, de 24-X-901.
A parte velha das varas das videiras.
Dança e ária popular do norte de Portugal.
\section{Velhori}
\begin{itemize}
\item {Grp. gram.:adj.}
\end{itemize}
Diz-se do cavallo de côr acinzentada.
(Cast. \textunderscore vellori\textunderscore )
\section{Vèlhorro}
\begin{itemize}
\item {fónica:lhô}
\end{itemize}
\begin{itemize}
\item {Grp. gram.:m.}
\end{itemize}
O mesmo que \textunderscore vèlhusco\textunderscore . Cf. Filinto, IV, 231.
\section{Vèlhota}
\begin{itemize}
\item {Grp. gram.:f.}
\end{itemize}
(Flex. fem. de \textunderscore vèlhote\textunderscore )
\section{Vèlhote}
\begin{itemize}
\item {Grp. gram.:m.  e  adj.}
\end{itemize}
\begin{itemize}
\item {Utilização:Fam.}
\end{itemize}
Homem velho, mas bem disposto.
Velho folgazão.
\section{Vèlhusca}
\begin{itemize}
\item {Grp. gram.:f.}
\end{itemize}
(Flexão fem. de \textunderscore vèlhusco\textunderscore )
\section{Vèlhusco}
\begin{itemize}
\item {Grp. gram.:m.  e  adj.}
\end{itemize}
\begin{itemize}
\item {Utilização:Fam.}
\end{itemize}
\begin{itemize}
\item {Proveniência:(De \textunderscore velho\textunderscore )}
\end{itemize}
Velho; vèlhote.
\section{Vèlhustro}
\begin{itemize}
\item {Grp. gram.:m.}
\end{itemize}
O mesmo que \textunderscore vèlhusco\textunderscore .
\section{Velido}
\begin{itemize}
\item {Grp. gram.:adj.}
\end{itemize}
\begin{itemize}
\item {Utilização:Ant.}
\end{itemize}
O mesmo que \textunderscore bello\textunderscore ^1? Cf. \textunderscore Cancion. da Vaticana\textunderscore .
(Por \textunderscore bellido\textunderscore , de \textunderscore bello\textunderscore ?)
\section{Velífero}
\begin{itemize}
\item {Grp. gram.:adj.}
\end{itemize}
\begin{itemize}
\item {Utilização:Poét.}
\end{itemize}
\begin{itemize}
\item {Grp. gram.:M.}
\end{itemize}
\begin{itemize}
\item {Proveniência:(Lat. \textunderscore velifer\textunderscore )}
\end{itemize}
Que tem velas, (falando-se de navios).
Peixe das profundidades do Oceano Índico, o qual iça as barbatanas á semelhança de velas de navio.
\section{Velilho}
\begin{itemize}
\item {Grp. gram.:m.}
\end{itemize}
Véu transparente.
Espécie de gaza.
(Cast. \textunderscore velillo\textunderscore )
\section{Velinha}
\begin{itemize}
\item {Grp. gram.:f.}
\end{itemize}
\begin{itemize}
\item {Proveniência:(De \textunderscore vela\textunderscore ^1)}
\end{itemize}
Pequena vela.
Substância sólida e medicamentosa, em fórma de vela delgada, e que se introduz no canal da urethra, como meio therapêutico.
\section{Velino}
\begin{itemize}
\item {Grp. gram.:m.  e  adj.}
\end{itemize}
\begin{itemize}
\item {Proveniência:(Fr. \textunderscore velin\textunderscore )}
\end{itemize}
Diz-se de uma qualidade de papel, branco e consistente, semelhante ao pergaminho fino.
\section{Velite}
\begin{itemize}
\item {Grp. gram.:m.}
\end{itemize}
\begin{itemize}
\item {Proveniência:(Lat. \textunderscore velites\textunderscore )}
\end{itemize}
Entre os Romanos, soldado de infantaria ligeira.
\section{Veleia}
\begin{itemize}
\item {Grp. gram.:f.}
\end{itemize}
Gênero de plantas goodeniáceas.
\section{Veleidade}
\begin{itemize}
\item {Grp. gram.:f.}
\end{itemize}
Vontade imperfeita, sem resultado.
Capricho; leviandade.
Utopia.
Volubilidade.
(Cp. it. \textunderscore velleita\textunderscore )
\section{Velicação}
\begin{itemize}
\item {Grp. gram.:f.}
\end{itemize}
\begin{itemize}
\item {Proveniência:(Do lat. \textunderscore vellicatio\textunderscore )}
\end{itemize}
Acto ou efeito de velicar.
\section{Velicar}
\begin{itemize}
\item {Grp. gram.:v. t.}
\end{itemize}
\begin{itemize}
\item {Proveniência:(Lat. \textunderscore vellicare\textunderscore )}
\end{itemize}
O mesmo que \textunderscore beliscar\textunderscore .
\section{Velicativo}
\begin{itemize}
\item {Grp. gram.:adj.}
\end{itemize}
Que velica; pungente.
\section{Veliqueiro}
\begin{itemize}
\item {Grp. gram.:adj.}
\end{itemize}
\begin{itemize}
\item {Utilização:Prov.}
\end{itemize}
\begin{itemize}
\item {Utilização:trasm.}
\end{itemize}
\begin{itemize}
\item {Proveniência:(De \textunderscore velicar\textunderscore )}
\end{itemize}
Que apenas toca no comer; debiqueiro.
\section{Velisca}
\begin{itemize}
\item {Grp. gram.:f.}
\end{itemize}
\begin{itemize}
\item {Utilização:Prov.}
\end{itemize}
\begin{itemize}
\item {Utilização:trasm.}
\end{itemize}
\begin{itemize}
\item {Proveniência:(De \textunderscore veliscar\textunderscore )}
\end{itemize}
Incisão com a unha.
\section{Veliscar}
\begin{itemize}
\item {Grp. gram.:v. t.}
\end{itemize}
O mesmo que \textunderscore beliscar\textunderscore . Cf. Camillo. \textunderscore Sc. da Foz\textunderscore , 121.
\section{Velívago}
\begin{itemize}
\item {Grp. gram.:adj.}
\end{itemize}
\begin{itemize}
\item {Utilização:Poét.}
\end{itemize}
\begin{itemize}
\item {Proveniência:(Do lat. \textunderscore velum\textunderscore  + \textunderscore vagari\textunderscore )}
\end{itemize}
Que veleja.
Que é movido por vela.
\section{Velívolo}
\begin{itemize}
\item {Grp. gram.:adj.}
\end{itemize}
\begin{itemize}
\item {Utilização:Poét.}
\end{itemize}
\begin{itemize}
\item {Proveniência:(Lat. \textunderscore velivolus\textunderscore )}
\end{itemize}
Que veleja rapidamente.
\section{Velleia}
\begin{itemize}
\item {Grp. gram.:f.}
\end{itemize}
Gênero de plantas goodeniáceas.
\section{Velleidade}
\begin{itemize}
\item {Grp. gram.:f.}
\end{itemize}
Vontade imperfeita, sem resultado.
Capricho; leviandade.
Utopia.
Volubilidade.
(Cp. it. \textunderscore velleita\textunderscore )
\section{Vellicação}
\begin{itemize}
\item {Grp. gram.:f.}
\end{itemize}
\begin{itemize}
\item {Proveniência:(Do lat. \textunderscore vellicatio\textunderscore )}
\end{itemize}
Acto ou effeito de vellicar.
\section{Vellicar}
\begin{itemize}
\item {Grp. gram.:v. t.}
\end{itemize}
\begin{itemize}
\item {Proveniência:(Lat. \textunderscore vellicare\textunderscore )}
\end{itemize}
O mesmo que \textunderscore beliscar\textunderscore .
\section{Vellicativo}
\begin{itemize}
\item {Grp. gram.:adj.}
\end{itemize}
Que vellica; pungente.
\section{Velliqueiro}
\begin{itemize}
\item {Grp. gram.:adj.}
\end{itemize}
\begin{itemize}
\item {Utilização:Prov.}
\end{itemize}
\begin{itemize}
\item {Utilização:trasm.}
\end{itemize}
\begin{itemize}
\item {Proveniência:(De \textunderscore vellicar\textunderscore )}
\end{itemize}
Que apenas toca no comer; debiqueiro.
\section{Vellisca}
\begin{itemize}
\item {Grp. gram.:f.}
\end{itemize}
\begin{itemize}
\item {Utilização:Prov.}
\end{itemize}
\begin{itemize}
\item {Utilização:trasm.}
\end{itemize}
\begin{itemize}
\item {Proveniência:(De \textunderscore velliscar\textunderscore )}
\end{itemize}
Incisão com a unha.
\section{Velliscar}
\begin{itemize}
\item {Grp. gram.:v. t.}
\end{itemize}
O mesmo que \textunderscore beliscar\textunderscore . Cf. Camillo. \textunderscore Sc. da Foz\textunderscore , 121.
\section{Vello}
\begin{itemize}
\item {Grp. gram.:m.}
\end{itemize}
\begin{itemize}
\item {Utilização:Prov.}
\end{itemize}
\begin{itemize}
\item {Utilização:alent.}
\end{itemize}
\begin{itemize}
\item {Proveniência:(Lat. \textunderscore vellus\textunderscore )}
\end{itemize}
Lan de carneiro ou de ovelha.
Lan de cordeiro.
Lan cardada.
Pelle de uma rês com a respectiva lan.
Lan de cada carneiro: \textunderscore o velo produziu três quilos\textunderscore .
\section{Vellocino}
\begin{itemize}
\item {Grp. gram.:m.}
\end{itemize}
\begin{itemize}
\item {Utilização:Ext.}
\end{itemize}
Pelle de carneiro ou de ovelha com lan.
Carneiro mythológico de vello de oiro.
(Cast. \textunderscore vellocino\textunderscore )
\section{Velloso}
\begin{itemize}
\item {Grp. gram.:adj.}
\end{itemize}
Que tem vello.
Que tem pêlo ou cabello comprido.
Felpudo; cabelludo.
Lanoso.
\section{Velludilho}
\begin{itemize}
\item {Grp. gram.:m.}
\end{itemize}
Velludo de algodão.
Planta amarantácea.
(Cast. \textunderscore velludillo\textunderscore )
\section{Velludíneo}
\begin{itemize}
\item {Grp. gram.:adj.}
\end{itemize}
\begin{itemize}
\item {Proveniência:(De \textunderscore velludo\textunderscore )}
\end{itemize}
O mesmo que \textunderscore avelludado\textunderscore .
\section{Velludo}
\begin{itemize}
\item {Grp. gram.:adj.}
\end{itemize}
\begin{itemize}
\item {Grp. gram.:M.}
\end{itemize}
\begin{itemize}
\item {Utilização:Ext.}
\end{itemize}
\begin{itemize}
\item {Proveniência:(Do lat. \textunderscore vellutus\textunderscore )}
\end{itemize}
O mesmo que \textunderscore velloso\textunderscore .
Tecido de algodão ou seda, que de um lado é mais ou menos velloso e macio.
Objecto macio, superfície macia.
Planta, o mesmo que \textunderscore velludilho\textunderscore .
Bredo.
Árvore medicinal da Guiné, fruto avermelhado e ácido.
\section{Velludoso}
\begin{itemize}
\item {Grp. gram.:adj.}
\end{itemize}
Semelhante ao velludo; macio como o velludo:«\textunderscore froixéis de folhagem velludosa\textunderscore ». Camillo, \textunderscore Mar. da Fonte\textunderscore , 6.
\section{Velo}
\begin{itemize}
\item {Grp. gram.:m.}
\end{itemize}
\begin{itemize}
\item {Utilização:Prov.}
\end{itemize}
\begin{itemize}
\item {Utilização:alent.}
\end{itemize}
\begin{itemize}
\item {Proveniência:(Lat. \textunderscore vellus\textunderscore )}
\end{itemize}
Lan de carneiro ou de ovelha.
Lan de cordeiro.
Lan cardada.
Pelle de uma rês com a respectiva lan.
Lan de cada carneiro: \textunderscore o velo produziu três quilos\textunderscore .
\section{Veloce}
\begin{itemize}
\item {Grp. gram.:adj.}
\end{itemize}
\begin{itemize}
\item {Utilização:Des.}
\end{itemize}
O mesmo que \textunderscore veloz\textunderscore :«\textunderscore as embarcaçães... mui veloces, estreitas e compridas...\textunderscore »\textunderscore Lusíadas\textunderscore , I, 46.
\section{Velocidade}
\begin{itemize}
\item {Grp. gram.:f.}
\end{itemize}
\begin{itemize}
\item {Proveniência:(Do lat. \textunderscore velocitas\textunderscore )}
\end{itemize}
Qualidade do que é veloz.
Movimento ligeiro.
Relação entre um espaço percorrido e a unidade do tempo.
\section{Velocífero}
\begin{itemize}
\item {Grp. gram.:m.}
\end{itemize}
\begin{itemize}
\item {Proveniência:(Do lat. \textunderscore velox\textunderscore  + \textunderscore ferre\textunderscore )}
\end{itemize}
O mesmo que \textunderscore celerifero\textunderscore .
\section{Velocígrafo}
\begin{itemize}
\item {Grp. gram.:m.}
\end{itemize}
\begin{itemize}
\item {Proveniência:(Do lat. \textunderscore velox\textunderscore  + gr. \textunderscore graphein\textunderscore )}
\end{itemize}
Espécie de copiógrafo.
\section{Velocígrapho}
\begin{itemize}
\item {Grp. gram.:m.}
\end{itemize}
\begin{itemize}
\item {Proveniência:(Do lat. \textunderscore velox\textunderscore  + gr. \textunderscore graphein\textunderscore )}
\end{itemize}
Espécie de copiógrapho.
\section{Velocino}
\begin{itemize}
\item {Grp. gram.:m.}
\end{itemize}
\begin{itemize}
\item {Utilização:Ext.}
\end{itemize}
Pele de carneiro ou de ovelha com lan.
Carneiro mitológico de velo de oiro.
(Cast. \textunderscore vellocino\textunderscore )
\section{Velocípede}
\begin{itemize}
\item {Grp. gram.:adj.}
\end{itemize}
\begin{itemize}
\item {Grp. gram.:M.}
\end{itemize}
\begin{itemize}
\item {Proveniência:(Do lat. \textunderscore velox\textunderscore  + \textunderscore pes\textunderscore , \textunderscore pedis\textunderscore )}
\end{itemize}
Que anda rapidamente ou que tem pés velozes.
Apparelho, com duas, três ou quatro rodas, e em que montam ou se assentam uma ou mais pessôas, que o impellem com os pés.
\section{Velocipedia}
\begin{itemize}
\item {Grp. gram.:f.}
\end{itemize}
Arte de andar em velocípede.
\section{Velocipédico}
\begin{itemize}
\item {Grp. gram.:adj.}
\end{itemize}
Relativo á velocipedia.
\section{Velocipedismo}
\begin{itemize}
\item {Grp. gram.:m.}
\end{itemize}
O mesmo que \textunderscore velocipedia\textunderscore .
\section{Velocipedista}
\begin{itemize}
\item {Grp. gram.:m.  e  f.}
\end{itemize}
Pessôa, que anda em velocípede.
\section{Velocíssimo}
\begin{itemize}
\item {Grp. gram.:adj.}
\end{itemize}
\begin{itemize}
\item {Proveniência:(Lat. \textunderscore velocissimus\textunderscore )}
\end{itemize}
Muito veloz.
\section{Velódromo}
\begin{itemize}
\item {Grp. gram.:m.}
\end{itemize}
\begin{itemize}
\item {Utilização:Neol.}
\end{itemize}
\begin{itemize}
\item {Proveniência:(Do lat. \textunderscore velox\textunderscore  + gr. \textunderscore dromos\textunderscore )}
\end{itemize}
Terreno em que se fazem corridas de velocípedes.
\section{Veloêmos}
\begin{itemize}
\item {Grp. gram.:m.}
\end{itemize}
\begin{itemize}
\item {Utilização:Ant.}
\end{itemize}
Remissa; adiamento. Cf. \textunderscore Eufrosina\textunderscore , 260.
(Por \textunderscore vê-lo-emos\textunderscore )
\section{Velório}
\begin{itemize}
\item {Grp. gram.:m.}
\end{itemize}
Variedade de uvas muito miúdas e sem préstimo.
(Aphér. de \textunderscore avelório\textunderscore )
\section{Velórios}
\begin{itemize}
\item {Grp. gram.:m. pl.}
\end{itemize}
O mesmo que \textunderscore avelórios\textunderscore .
\section{Veloso}
\begin{itemize}
\item {Grp. gram.:adj.}
\end{itemize}
Que tem velo.
Que tem pêlo ou cabelo comprido.
Felpudo; cabeludo.
Lanoso.
\section{Veloz}
\begin{itemize}
\item {Grp. gram.:adj.}
\end{itemize}
\begin{itemize}
\item {Proveniência:(Lat. \textunderscore velox\textunderscore )}
\end{itemize}
Que anda ou corre com rapidez; rápido, ligeiro.
\section{Velozmente}
\begin{itemize}
\item {Grp. gram.:adv.}
\end{itemize}
De modo veloz; rapidamente.
\section{Veludilho}
\begin{itemize}
\item {Grp. gram.:m.}
\end{itemize}
Velludo de algodão.
Planta amarantácea.
(Cast. \textunderscore velludillo\textunderscore )
\section{Veludíneo}
\begin{itemize}
\item {Grp. gram.:adj.}
\end{itemize}
\begin{itemize}
\item {Proveniência:(De \textunderscore veludo\textunderscore )}
\end{itemize}
O mesmo que \textunderscore aveludado\textunderscore .
\section{Veludo}
\begin{itemize}
\item {Grp. gram.:adj.}
\end{itemize}
\begin{itemize}
\item {Grp. gram.:M.}
\end{itemize}
\begin{itemize}
\item {Utilização:Ext.}
\end{itemize}
\begin{itemize}
\item {Proveniência:(Do lat. \textunderscore vellutus\textunderscore )}
\end{itemize}
O mesmo que \textunderscore veloso\textunderscore .
Tecido de algodão ou seda, que de um lado é mais ou menos veloso e macio.
Objecto macio, superfície macia.
Planta, o mesmo que \textunderscore veludilho\textunderscore .
Bredo.
Árvore medicinal da Guiné, fruto avermelhado e ácido.
\section{Veludoso}
\begin{itemize}
\item {Grp. gram.:adj.}
\end{itemize}
Semelhante ao veludo; macio como o veludo:«\textunderscore froixéis de folhagem veludosa\textunderscore ». Camillo, \textunderscore Mar. da Fonte\textunderscore , 6.
\section{Velutina}
\begin{itemize}
\item {Grp. gram.:f.}
\end{itemize}
\begin{itemize}
\item {Proveniência:(Fr. \textunderscore veloutine\textunderscore )}
\end{itemize}
Espécie de tecido de sêda, especialmente o que se usava no séc. XVIII.
Pó de arroz, preparado com bismutho.
\section{V. Em.^a}
Abrev. de \textunderscore Vossa Eminência\textunderscore .
\section{Venablo}
\begin{itemize}
\item {Grp. gram.:m.}
\end{itemize}
O mesmo que \textunderscore venábulo\textunderscore . Cf. Filinto, \textunderscore D. Man.\textunderscore , II, 331.
\section{Venábulo}
\begin{itemize}
\item {Grp. gram.:m.}
\end{itemize}
\begin{itemize}
\item {Utilização:Fig.}
\end{itemize}
\begin{itemize}
\item {Proveniência:(Lat. \textunderscore venabulum\textunderscore )}
\end{itemize}
Espécie de lança, para caça de feras; zarguncho.
Meio de defesa; expediente.
\section{Venado}
\begin{itemize}
\item {Grp. gram.:adj.}
\end{itemize}
\begin{itemize}
\item {Proveniência:(Do lat. \textunderscore vena\textunderscore )}
\end{itemize}
Que tem veias.
\section{Venado}
\begin{itemize}
\item {Grp. gram.:m.}
\end{itemize}
\begin{itemize}
\item {Utilização:Prov.}
\end{itemize}
\begin{itemize}
\item {Utilização:minh.}
\end{itemize}
\textunderscore Tirar-lhe o venado\textunderscore , diz-se de uma criada que foi substituír outra, servindo-se de manhas ou intrigas.
\section{Venador}
\begin{itemize}
\item {Grp. gram.:m.}
\end{itemize}
\begin{itemize}
\item {Proveniência:(Do lat. \textunderscore venator\textunderscore )}
\end{itemize}
O mesmo que \textunderscore caçador\textunderscore .
Aquelle que exercia as funcções de vigilante, nos jogos dos circos romanos. Cf. A. Costa. \textunderscore Três Mundos\textunderscore , 161 e 164.
\section{Venal}
\begin{itemize}
\item {Grp. gram.:adj.}
\end{itemize}
\begin{itemize}
\item {Utilização:Fig.}
\end{itemize}
\begin{itemize}
\item {Proveniência:(Lat. \textunderscore venalis\textunderscore )}
\end{itemize}
Que se póde vender.
Exposto á venda.
Relativo á venda.
Que procede por interesse illícito; que se deixa peitar.
\section{Venal}
\begin{itemize}
\item {Grp. gram.:adj.}
\end{itemize}
O mesmo que \textunderscore venoso\textunderscore .
\section{Venalidade}
\begin{itemize}
\item {Grp. gram.:f.}
\end{itemize}
\begin{itemize}
\item {Proveniência:(Do lat. \textunderscore venalitas\textunderscore )}
\end{itemize}
Qualidade do que é venal^1.
\section{Venalmente}
\begin{itemize}
\item {Grp. gram.:adv.}
\end{itemize}
De modo venal^1.
\section{Venário}
\begin{itemize}
\item {Grp. gram.:m.}
\end{itemize}
\begin{itemize}
\item {Utilização:Ant.}
\end{itemize}
\begin{itemize}
\item {Proveniência:(Do lat. \textunderscore venare\textunderscore )}
\end{itemize}
Aquelle que habita no campo ou na aldeia.
\section{Venatória}
\begin{itemize}
\item {Grp. gram.:f.}
\end{itemize}
\begin{itemize}
\item {Utilização:Poét.}
\end{itemize}
\begin{itemize}
\item {Proveniência:(Lat. \textunderscore venatoria\textunderscore )}
\end{itemize}
Composição poética, cujas personagens são caçadores.
\section{Venatório}
\begin{itemize}
\item {Grp. gram.:adj.}
\end{itemize}
\begin{itemize}
\item {Proveniência:(Lat. \textunderscore venatorius\textunderscore )}
\end{itemize}
Relativo á caça.
\section{Venatura}
\begin{itemize}
\item {Grp. gram.:f.}
\end{itemize}
\begin{itemize}
\item {Utilização:Ant.}
\end{itemize}
\begin{itemize}
\item {Proveniência:(Lat. \textunderscore venatura\textunderscore )}
\end{itemize}
O mesmo que \textunderscore caçada\textunderscore .
\section{Vencedor}
\begin{itemize}
\item {Grp. gram.:m.  e  adj.}
\end{itemize}
O que vence ou venceu.
Homem victorioso.
\section{Vencelho}
\begin{itemize}
\item {Grp. gram.:m.}
\end{itemize}
O mesmo que \textunderscore vincilho\textunderscore .
\section{Vencer}
\begin{itemize}
\item {Grp. gram.:v. t.}
\end{itemize}
\begin{itemize}
\item {Grp. gram.:V. p.}
\end{itemize}
\begin{itemize}
\item {Proveniência:(Lat. \textunderscore vincere\textunderscore )}
\end{itemize}
Conseguir victória sôbre.
Triumphar de.
Obter vantagem sôbre.
Têr bom êxito á cêrca de: \textunderscore vencer uma demanda\textunderscore .
Lucrar, auferir.
Têr de ordenado: \textunderscore vence um conto por anno\textunderscore .
Exceder.
Têr primazia sôbre.
Conter; dominar: \textunderscore vencer tentações\textunderscore .
Subjugar.
Convencer.
Andar, percorrer.
Terminar; executar.
Conter-se, refrear os vícios próprios.
Reprimir-se.
Chegar ao fim do tempo, em que se deve fazer um pagamento: \textunderscore vencer-se uma letra\textunderscore .
\section{Vencida}
\begin{itemize}
\item {Grp. gram.:f.}
\end{itemize}
O mesmo que \textunderscore vencimento\textunderscore .
\section{Vencido}
Aquelle que foi vencido.
\section{Vencimento}
\begin{itemize}
\item {Grp. gram.:m.}
\end{itemize}
Acto ou effeito de vencer.
Triumpho.
Acto de expirar o prazo para o pagamento de uma letra de câmbio, ou para o cumprimento de qualquer encargo.
\section{Vencível}
\begin{itemize}
\item {Grp. gram.:adj.}
\end{itemize}
\begin{itemize}
\item {Proveniência:(Do lat. \textunderscore vincibilis\textunderscore )}
\end{itemize}
Que se póde vencer.
\section{Venda}
\begin{itemize}
\item {Grp. gram.:f.}
\end{itemize}
\begin{itemize}
\item {Utilização:Ant.}
\end{itemize}
Acto ou effeito de vender.
Loja em que se vende; taberna.
Laudêmio, que se pagava, de uma fazenda foreira, quando esta se vendia:«\textunderscore e se algum caseiro quijer vender, que nós ajamos a venda.\textunderscore »\textunderscore Doc. de 1356\textunderscore , de San-Tiago de Coímbra.
\section{Venda}
\begin{itemize}
\item {Grp. gram.:f.}
\end{itemize}
\begin{itemize}
\item {Proveniência:(Do ant. alt. al. \textunderscore binda\textunderscore )}
\end{itemize}
Faixa com que se cobrem os olhos.
\section{Vendagem}
\begin{itemize}
\item {Grp. gram.:f.}
\end{itemize}
\begin{itemize}
\item {Proveniência:(De \textunderscore venda\textunderscore ^1)}
\end{itemize}
Venda.
Percentagem do preço da venda, em favor do que vende por conta alheia.
\section{Vendar}
\begin{itemize}
\item {Grp. gram.:v. t.}
\end{itemize}
\begin{itemize}
\item {Utilização:Fig.}
\end{itemize}
\begin{itemize}
\item {Proveniência:(De \textunderscore venda\textunderscore ^2)}
\end{itemize}
Cobrir com venda.
Tapar os olhos de.
Obscurecer, cegar.
\section{Vendaval}
\begin{itemize}
\item {Grp. gram.:m.}
\end{itemize}
\begin{itemize}
\item {Utilização:Ant.}
\end{itemize}
\begin{itemize}
\item {Utilização:Ant.}
\end{itemize}
\begin{itemize}
\item {Proveniência:(Do fr. \textunderscore vent d'aval\textunderscore )}
\end{itemize}
Vento tempestuoso; temporal.
Vento do Sul.
O lado do Sul: \textunderscore a herdade confina do vendaval com a estrada pública\textunderscore .
\section{Vendável}
\begin{itemize}
\item {Grp. gram.:adj.}
\end{itemize}
(V.vendível)
\section{Vendedeira}
\begin{itemize}
\item {Grp. gram.:f.}
\end{itemize}
\begin{itemize}
\item {Proveniência:(De \textunderscore vender\textunderscore )}
\end{itemize}
Mulhér, que vende em público, nas ruas ou nas praças.
Mulhér, que tem loja para venda.
\section{Vendedoiro}
\begin{itemize}
\item {Grp. gram.:m.}
\end{itemize}
\begin{itemize}
\item {Proveniência:(De \textunderscore vender\textunderscore )}
\end{itemize}
Lugar público, onde se vende alguma coisa.
\section{Vendedor}
\begin{itemize}
\item {Grp. gram.:m.  e  adj.}
\end{itemize}
\begin{itemize}
\item {Proveniência:(Do lat. \textunderscore venditor\textunderscore )}
\end{itemize}
O que vende.
\section{Vendedora}
\begin{itemize}
\item {Grp. gram.:f.}
\end{itemize}
O mesmo que \textunderscore vendedeira\textunderscore .
\section{Vendedouro}
\begin{itemize}
\item {Grp. gram.:m.}
\end{itemize}
\begin{itemize}
\item {Proveniência:(De \textunderscore vender\textunderscore )}
\end{itemize}
Lugar público, onde se vende alguma coisa.
\section{Vendeiro}
\begin{itemize}
\item {Grp. gram.:m.}
\end{itemize}
\begin{itemize}
\item {Proveniência:(De \textunderscore venda\textunderscore ^1)}
\end{itemize}
O mesmo que \textunderscore taberneiro\textunderscore .
\section{Vender}
\begin{itemize}
\item {Grp. gram.:v. t.}
\end{itemize}
\begin{itemize}
\item {Utilização:Fig.}
\end{itemize}
\begin{itemize}
\item {Grp. gram.:V. p.}
\end{itemize}
\begin{itemize}
\item {Proveniência:(Lat. \textunderscore vendere\textunderscore )}
\end{itemize}
Alienar ou ceder por certo preço.
Não conceder gratuitamente.
Deixar-se peitar para ceder ou fazer.
Denunciar.
Trahir, interessadamente.
Ceder a sua liberdade por certo preço.
Praticar por interesse actos indignos.
Deixar-se peitar.
\section{Vendição}
\begin{itemize}
\item {Grp. gram.:f.}
\end{itemize}
\begin{itemize}
\item {Utilização:P. us.}
\end{itemize}
\begin{itemize}
\item {Proveniência:(Do lat. \textunderscore venditio\textunderscore )}
\end{itemize}
O mesmo que \textunderscore venda\textunderscore ^1.
\section{Vêndico}
\begin{itemize}
\item {Grp. gram.:m.}
\end{itemize}
Um dos idiomas esclavónicos.
\section{Vendido}
\begin{itemize}
\item {Grp. gram.:adj.}
\end{itemize}
\begin{itemize}
\item {Utilização:Fig.}
\end{itemize}
\begin{itemize}
\item {Proveniência:(De \textunderscore vender\textunderscore )}
\end{itemize}
Que se vendeu; que foi comprado; adquirido por vendição.
Peitado, subornado.
Contrafeito, violentado; contrariado.
Espantado.
\section{Vendilhão}
\begin{itemize}
\item {Grp. gram.:m.}
\end{itemize}
\begin{itemize}
\item {Utilização:Fig.}
\end{itemize}
\begin{itemize}
\item {Proveniência:(De \textunderscore vender\textunderscore )}
\end{itemize}
Vendedor ambulante.
Indivíduo, que vende pelas ruas ou nas praças.
Aquelle que trafica publicamente em coisas de ordem moral.
\section{Vendimento}
\begin{itemize}
\item {Grp. gram.:m.}
\end{itemize}
\begin{itemize}
\item {Utilização:Ant.}
\end{itemize}
O mesmo que \textunderscore vendição\textunderscore .
\section{Vendível}
\begin{itemize}
\item {Grp. gram.:adj.}
\end{itemize}
\begin{itemize}
\item {Proveniência:(Do lat. \textunderscore vendibilis\textunderscore )}
\end{itemize}
Que se póde vender.
Próprio para venda.
Que se vende facilmente ou pôde têr bôa venda.
\section{Vendudo}
\begin{itemize}
\item {Grp. gram.:adj.}
\end{itemize}
\begin{itemize}
\item {Utilização:Ant.}
\end{itemize}
O mesmo que \textunderscore vendido\textunderscore .
(Ant. part. de \textunderscore vender\textunderscore )
\section{Venefício}
\begin{itemize}
\item {Grp. gram.:m.}
\end{itemize}
\begin{itemize}
\item {Proveniência:(Lat. \textunderscore veneficium\textunderscore )}
\end{itemize}
Acto de preparar veneno, para fim criminoso.
Crime de envenenar alguém.
\section{Venéfico}
\begin{itemize}
\item {Grp. gram.:adj.}
\end{itemize}
\begin{itemize}
\item {Proveniência:(Lat. \textunderscore veneficus\textunderscore )}
\end{itemize}
Relativo a venefício; maléfico.
Venenoso.
\section{Venenífero}
\begin{itemize}
\item {Grp. gram.:adj.}
\end{itemize}
\begin{itemize}
\item {Proveniência:(Lat. \textunderscore venenifer\textunderscore )}
\end{itemize}
Que produz veneno.
Venenoso.
\section{Veneníparo}
\begin{itemize}
\item {Grp. gram.:adj.}
\end{itemize}
\begin{itemize}
\item {Proveniência:(Do lat. \textunderscore venenum\textunderscore  + \textunderscore p[-a]rere\textunderscore )}
\end{itemize}
Que segrega veneno.
\section{Veneno}
\begin{itemize}
\item {Grp. gram.:m.}
\end{itemize}
\begin{itemize}
\item {Utilização:Fig.}
\end{itemize}
\begin{itemize}
\item {Utilização:Veter.}
\end{itemize}
\begin{itemize}
\item {Proveniência:(Lat. \textunderscore venenum\textunderscore )}
\end{itemize}
Substância, geralmente líquida, que perturba ou destrói as funcções vitaes.
Peçonha; vírus.
Aquillo que corrompe moralmente, podendo comparar-se ao veneno material.
Malignidade.
Má intenção.
Interpretação maliciosa.
Pessôa de má índole.
Doença dos animaes, variedade de carbúnculo. Cf. M. Pinto, \textunderscore Comp. de Veter.\textunderscore , I, 451.
\section{Venenosamente}
\begin{itemize}
\item {Grp. gram.:adv.}
\end{itemize}
De modo venenoso.
\section{Venenosidade}
\begin{itemize}
\item {Grp. gram.:f.}
\end{itemize}
Qualidade do que é venenoso.
\section{Venenoso}
\begin{itemize}
\item {Grp. gram.:adj.}
\end{itemize}
\begin{itemize}
\item {Utilização:Fig.}
\end{itemize}
\begin{itemize}
\item {Proveniência:(Lat. \textunderscore venenosus\textunderscore )}
\end{itemize}
Que contém veneno.
Que é da natureza do veneno.
Que opéra como veneno.
Em que há veneno moral.
Malévolo; nocivo: \textunderscore allusões venenosas\textunderscore .
\section{Venera}
\begin{itemize}
\item {Grp. gram.:f.}
\end{itemize}
\begin{itemize}
\item {Utilização:Ext.}
\end{itemize}
\begin{itemize}
\item {Proveniência:(Do lat. \textunderscore veneriae\textunderscore )}
\end{itemize}
Vieira ou concha de romeiro.
Insígnia dos condecorados com qualquer grau de uma Ordem militar.
Condecoração.
\section{Venerabilidade}
\begin{itemize}
\item {Grp. gram.:f.}
\end{itemize}
\begin{itemize}
\item {Proveniência:(Do lat. \textunderscore venerabilitas\textunderscore )}
\end{itemize}
Qualidade do que é venerável.
\section{Venerabundo}
\begin{itemize}
\item {Grp. gram.:adj.}
\end{itemize}
\begin{itemize}
\item {Proveniência:(Lat. \textunderscore venerabundus\textunderscore )}
\end{itemize}
Que venera; reverente.
\section{Veneração}
\begin{itemize}
\item {Grp. gram.:f.}
\end{itemize}
\begin{itemize}
\item {Proveniência:(Do lat. \textunderscore veneratio\textunderscore )}
\end{itemize}
Acto ou effeito de venerar.
Reverência, culto; preito.
\section{Veneradamente}
\begin{itemize}
\item {Grp. gram.:adv.}
\end{itemize}
\begin{itemize}
\item {Proveniência:(De \textunderscore venerado\textunderscore )}
\end{itemize}
Com veneração.
\section{Venerado}
\begin{itemize}
\item {Grp. gram.:adj.}
\end{itemize}
\begin{itemize}
\item {Proveniência:(De \textunderscore venerar\textunderscore )}
\end{itemize}
Que é objecto de veneração.
Muito respeitado: \textunderscore o venerado mestre\textunderscore .
\section{Venerador}
\begin{itemize}
\item {Grp. gram.:m.  e  adj.}
\end{itemize}
\begin{itemize}
\item {Proveniência:(Lat. \textunderscore venerator\textunderscore )}
\end{itemize}
O que venera.
\section{Veneraes}
\begin{itemize}
\item {Grp. gram.:f. pl.}
\end{itemize}
\begin{itemize}
\item {Proveniência:(Lat. \textunderscore veneralia\textunderscore )}
\end{itemize}
Antigas festas romanas que, em honra de Vênus, se celebravam durante os três dias immediatos ás calendas de Abril.
\section{Venerais}
\begin{itemize}
\item {Grp. gram.:f. pl.}
\end{itemize}
\begin{itemize}
\item {Proveniência:(Lat. \textunderscore veneralia\textunderscore )}
\end{itemize}
Antigas festas romanas que, em honra de Vênus, se celebravam durante os três dias immediatos ás calendas de Abril.
\section{Veneralato}
\begin{itemize}
\item {Grp. gram.:m.}
\end{itemize}
Cargo de venerável, na Maçonaria.
(Por \textunderscore veneravelato\textunderscore , de \textunderscore venerável\textunderscore )
\section{Venerando}
\begin{itemize}
\item {Grp. gram.:adj.}
\end{itemize}
\begin{itemize}
\item {Proveniência:(Lat. \textunderscore venerandus\textunderscore )}
\end{itemize}
O mesmo que \textunderscore venerável\textunderscore .
\section{Venerário}
\begin{itemize}
\item {Grp. gram.:adj.}
\end{itemize}
\begin{itemize}
\item {Proveniência:(Lat. \textunderscore venerarius\textunderscore )}
\end{itemize}
Relativo a Vênus.
Relativo aos prazeres sensuaes.
\section{Venerar}
\begin{itemize}
\item {Grp. gram.:v. t.}
\end{itemize}
\begin{itemize}
\item {Utilização:Prov.}
\end{itemize}
\begin{itemize}
\item {Utilização:minh.}
\end{itemize}
\begin{itemize}
\item {Proveniência:(Lat. \textunderscore venerari\textunderscore )}
\end{itemize}
Tributar grande respeito a.
Tratar com respeito e affeição.
Reverenciar; acatar.
Sustentar, dar alimentos a. (Colhido em Paredes de Coira)
\section{Venerável}
\begin{itemize}
\item {Grp. gram.:adj.}
\end{itemize}
\begin{itemize}
\item {Grp. gram.:M.}
\end{itemize}
\begin{itemize}
\item {Proveniência:(Do lat. \textunderscore venerabilis\textunderscore )}
\end{itemize}
Que se deve venerar; respeitável.
Aquelle que preside a uma loja maçónica.
\section{Veneravelmente}
\begin{itemize}
\item {Grp. gram.:adv.}
\end{itemize}
De modo venerável.
\section{Venéreo}
\begin{itemize}
\item {Grp. gram.:adj.}
\end{itemize}
\begin{itemize}
\item {Proveniência:(Lat. \textunderscore venereus\textunderscore )}
\end{itemize}
Relativo a Vênus.
Relativo á aproximação dos sexos.
Sensual.
Erótico.
Adquirido em relações sexuaes, (falando-se de doenças).
\section{Vênero}
\begin{itemize}
\item {Grp. gram.:adj.}
\end{itemize}
\begin{itemize}
\item {Utilização:Poét.}
\end{itemize}
\begin{itemize}
\item {Proveniência:(Do lat. \textunderscore Venus\textunderscore , \textunderscore Veneris\textunderscore , n. p.)}
\end{itemize}
Relativo a Vênus.
\section{Venericárdia}
\begin{itemize}
\item {Grp. gram.:f.}
\end{itemize}
Gênero de molluscos, cuja concha tem o vértice arqueado e apresenta lados salientes.
\section{Veneroso}
\begin{itemize}
\item {Grp. gram.:adj.}
\end{itemize}
\begin{itemize}
\item {Utilização:Ant.}
\end{itemize}
O mesmo que \textunderscore venerando\textunderscore . Cf. Azurara.
\section{Veneta}
\begin{itemize}
\item {fónica:nê}
\end{itemize}
\begin{itemize}
\item {Grp. gram.:f.}
\end{itemize}
\begin{itemize}
\item {Utilização:Ext.}
\end{itemize}
\begin{itemize}
\item {Grp. gram.:Loc.}
\end{itemize}
\begin{itemize}
\item {Utilização:fam.}
\end{itemize}
Accesso de loucura.
Impulso repentino; capricho; tineta.
\textunderscore Dar na veneta\textunderscore , vir á ideia.
(E affim do fr. \textunderscore venette\textunderscore ?)
\section{Vênetos}
\begin{itemize}
\item {Grp. gram.:m. pl.}
\end{itemize}
\begin{itemize}
\item {Proveniência:(Lat. \textunderscore Veneti\textunderscore )}
\end{itemize}
Antigo povo da Itália, procedente da Paphlagónia.
\section{Veneziana}
\begin{itemize}
\item {Grp. gram.:f.}
\end{itemize}
\begin{itemize}
\item {Proveniência:(De \textunderscore veneziano\textunderscore )}
\end{itemize}
Appêndice das janelas, também, chamado \textunderscore persianas\textunderscore , \textunderscore tabuinhas\textunderscore , etc.
\section{Veneziano}
\begin{itemize}
\item {Grp. gram.:adj.}
\end{itemize}
\begin{itemize}
\item {Grp. gram.:M.}
\end{itemize}
\begin{itemize}
\item {Proveniência:(Do lat. \textunderscore venetianus\textunderscore )}
\end{itemize}
Relativo a Veneza.
Diz-se de uma variedade de pêssego.
Habitante de Veneza.
Um dos dialectos italianos.
Moéda de oiro, veneziana, que corria na Índia Portuguesa e valia 5 rupias.
\section{Vênia}
\begin{itemize}
\item {Grp. gram.:f.}
\end{itemize}
\begin{itemize}
\item {Proveniência:(Lat. \textunderscore venia\textunderscore )}
\end{itemize}
Licença, permissão.
Desculpa.
Absolvição.
Sinal de cortesia.
Mesura; cumprimento.
\section{Veniaga}
\begin{itemize}
\item {Grp. gram.:f.}
\end{itemize}
\begin{itemize}
\item {Utilização:Fig.}
\end{itemize}
\begin{itemize}
\item {Proveniência:(De \textunderscore veniagar\textunderscore , se não veio do malaio \textunderscore beniaga\textunderscore , lugar de commércio, como affirmou Barros, nas \textunderscore Décadas\textunderscore )}
\end{itemize}
Mercadoria.
Commércio.
Tranquibérnia; procedimento de agiota.
\section{Veniagar}
\begin{itemize}
\item {Grp. gram.:v. t.}
\end{itemize}
\begin{itemize}
\item {Proveniência:(Do lat. \textunderscore venum\textunderscore  + \textunderscore agere\textunderscore . Cp. entretanto \textunderscore veniaga\textunderscore )}
\end{itemize}
O mesmo que \textunderscore traficar\textunderscore .
\section{Venial}
\begin{itemize}
\item {Grp. gram.:adj.}
\end{itemize}
\begin{itemize}
\item {Proveniência:(Lat. \textunderscore venialis\textunderscore )}
\end{itemize}
Perdoável; digno de vênia.
Diz-se das faltas ou peccados leves.
\section{Venialidade}
\begin{itemize}
\item {Grp. gram.:f.}
\end{itemize}
Qualidade do que é venial.
\section{Venialmente}
\begin{itemize}
\item {Grp. gram.:adv.}
\end{itemize}
De modo venial.
\section{Venida}
\begin{itemize}
\item {Grp. gram.:f.}
\end{itemize}
\begin{itemize}
\item {Utilização:Des.}
\end{itemize}
\begin{itemize}
\item {Proveniência:(Do lat. \textunderscore venire\textunderscore )}
\end{itemize}
Vingança.
Investida repentina do inimigo.
Golpe de espada para ferir, no jôgo da esgrima: \textunderscore idas e venidas\textunderscore .
Diligência.
\section{Venífluo}
\begin{itemize}
\item {Grp. gram.:adj.}
\end{itemize}
\begin{itemize}
\item {Utilização:Poét.}
\end{itemize}
\begin{itemize}
\item {Proveniência:(Do lat. \textunderscore vena\textunderscore  + \textunderscore fluere\textunderscore )}
\end{itemize}
Que corre pelas veias.
\section{Veníssimos}
\begin{itemize}
\item {Grp. gram.:m. pl.}
\end{itemize}
\begin{itemize}
\item {Utilização:T. de Turquel}
\end{itemize}
Tempos passados: \textunderscore há que veníssimos isso lá vai\textunderscore .
\section{Venómero}
\begin{itemize}
\item {Grp. gram.:m.}
\end{itemize}
\begin{itemize}
\item {Utilização:Anat.}
\end{itemize}
\begin{itemize}
\item {Proveniência:(Do lat. \textunderscore vena\textunderscore  + gr. \textunderscore meros\textunderscore )}
\end{itemize}
Parte venosa do metâmero.
\section{Venoso}
\begin{itemize}
\item {Grp. gram.:adj.}
\end{itemize}
\begin{itemize}
\item {Proveniência:(Lat. \textunderscore venosus\textunderscore )}
\end{itemize}
Que tem veias.
Relativo a veias; venífluo.
\section{Venta}
\begin{itemize}
\item {Grp. gram.:f.}
\end{itemize}
\begin{itemize}
\item {Grp. gram.:Pl.}
\end{itemize}
\begin{itemize}
\item {Utilização:Fig.}
\end{itemize}
\begin{itemize}
\item {Utilização:Cyn.}
\end{itemize}
\begin{itemize}
\item {Proveniência:(De \textunderscore ventan\textunderscore , como \textunderscore campa\textunderscore  de \textunderscore campan\textunderscore . Cf. G. Viana, \textunderscore Apostilas\textunderscore )}
\end{itemize}
Cada uma das fossas nasaes.
Nariz.
Presença: \textunderscore disse obscenidades, mesmo nas ventas da autoridade\textunderscore .
Olfacto.
\section{Ventã}
\begin{itemize}
\item {Grp. gram.:f.}
\end{itemize}
\begin{itemize}
\item {Utilização:Ant.}
\end{itemize}
Vesícula do ruivo, cheia de ar.
Janela.
(Cast. \textunderscore ventana\textunderscore )
\section{Ventage}
\begin{itemize}
\item {Grp. gram.:f.}
\end{itemize}
O mesmo que \textunderscore vantagem\textunderscore . Cf. B. Pereira, \textunderscore Prosodia\textunderscore , vb. \textunderscore excedo\textunderscore .
\section{Ventagem}
\begin{itemize}
\item {Grp. gram.:f.}
\end{itemize}
\begin{itemize}
\item {Utilização:ant.}
\end{itemize}
\begin{itemize}
\item {Utilização:Pop.}
\end{itemize}
O mesmo que \textunderscore vantagem\textunderscore . Cf. B. Pereira, \textunderscore Prosodia\textunderscore , vb. \textunderscore excedo\textunderscore .
\section{Ventan}
\begin{itemize}
\item {Grp. gram.:f.}
\end{itemize}
\begin{itemize}
\item {Utilização:Ant.}
\end{itemize}
Vesícula do ruivo, cheia de ar.
Janela.
(Cast. \textunderscore ventana\textunderscore )
\section{Ventana}
\begin{itemize}
\item {Grp. gram.:f.}
\end{itemize}
\begin{itemize}
\item {Utilização:Ant.}
\end{itemize}
Janela; ventanilha.
O mesmo que \textunderscore leque\textunderscore ^1. Cf. Camillo, \textunderscore Cav. em Ruínas\textunderscore , 80.
O mesmo que \textunderscore sineira\textunderscore . Cf. Castilho, \textunderscore Escavações\textunderscore , 261.
(Cast. \textunderscore ventana\textunderscore )
\section{Ventanear}
\begin{itemize}
\item {Grp. gram.:v. t.}
\end{itemize}
\begin{itemize}
\item {Utilização:Fig.}
\end{itemize}
\begin{itemize}
\item {Proveniência:(De \textunderscore ventana\textunderscore )}
\end{itemize}
O mesmo que \textunderscore ventilar\textunderscore .
Sacudir; agitar.
\section{Ventaneira}
\begin{itemize}
\item {Grp. gram.:f.}
\end{itemize}
\begin{itemize}
\item {Proveniência:(De \textunderscore ventana\textunderscore )}
\end{itemize}
Ventania.
Válvula, por onde entra o ar no folle.
\section{Ventaneiro}
\begin{itemize}
\item {Grp. gram.:m.}
\end{itemize}
\begin{itemize}
\item {Utilização:T. do Fundão}
\end{itemize}
Estroina, valdevinos.
(Cp. cast. \textunderscore ventanera\textunderscore )
\section{Ventania}
\begin{itemize}
\item {Grp. gram.:f.}
\end{itemize}
\begin{itemize}
\item {Proveniência:(De \textunderscore ventana\textunderscore )}
\end{itemize}
Vento forte e contínuo.
\section{Ventanilha}
\begin{itemize}
\item {Grp. gram.:f.}
\end{itemize}
Cada uma das aberturas do bilhar, por onde entra a bola.
(Cast. \textunderscore ventanilla\textunderscore )
\section{Ventapopa}
\begin{itemize}
\item {Grp. gram.:adv.}
\end{itemize}
\begin{itemize}
\item {Utilização:Fig.}
\end{itemize}
De vento ou com vento em popa.
Prosperamente.
(Contr. de \textunderscore vento\textunderscore  + \textunderscore a\textunderscore  + \textunderscore popa\textunderscore )
\section{Ventar}
\begin{itemize}
\item {Grp. gram.:v. i.}
\end{itemize}
\begin{itemize}
\item {Utilização:Fam.}
\end{itemize}
\begin{itemize}
\item {Utilização:Fig.}
\end{itemize}
\begin{itemize}
\item {Proveniência:(De \textunderscore vento\textunderscore )}
\end{itemize}
Soprar o vento.
Soltar ventosidades.
Surgir ou mostrar-se de repente.
Sêr próspero, favorável.
\section{Ventarola}
\begin{itemize}
\item {Grp. gram.:f.}
\end{itemize}
\begin{itemize}
\item {Proveniência:(It. \textunderscore ventarola\textunderscore , com a significação de ventoínha)}
\end{itemize}
Espécie de leque, com um só cabo, e sem varetas.
\section{Vente}
\begin{itemize}
\item {Grp. gram.:adj.}
\end{itemize}
\begin{itemize}
\item {Utilização:Ant.}
\end{itemize}
Que vê, que está vendo.
O mesmo que [[vendo|vêr]], gerúndio ou part. imperfeito do v. [[vêr]]. Cf. \textunderscore Aulegrafia\textunderscore , 5.
\section{Venteada}
\begin{itemize}
\item {Grp. gram.:adj. f.}
\end{itemize}
\begin{itemize}
\item {Utilização:Prov.}
\end{itemize}
\begin{itemize}
\item {Utilização:trasm.}
\end{itemize}
\begin{itemize}
\item {Proveniência:(De \textunderscore vento\textunderscore )}
\end{itemize}
Diz-se da pedra, que tem fendas, embora quási imperceptíveis.
Diz-se da rapariga leviana.
\section{Ventiela}
\begin{itemize}
\item {Grp. gram.:f.}
\end{itemize}
\begin{itemize}
\item {Utilização:Prov.}
\end{itemize}
\begin{itemize}
\item {Utilização:minh.}
\end{itemize}
Ventoínha, catavento.
\section{Ventígeno}
\begin{itemize}
\item {Grp. gram.:adj.}
\end{itemize}
\begin{itemize}
\item {Utilização:Poét.}
\end{itemize}
\begin{itemize}
\item {Proveniência:(Lat. \textunderscore ventigenus\textunderscore )}
\end{itemize}
Que produz vento.
Produzido pelo vento.
\section{Ventilabro}
\begin{itemize}
\item {Grp. gram.:m.}
\end{itemize}
\begin{itemize}
\item {Proveniência:(Lat. \textunderscore ventilabrum\textunderscore )}
\end{itemize}
Espécie de joeira, com que se limpa o trigo.
\section{Ventilação}
\begin{itemize}
\item {Grp. gram.:f.}
\end{itemize}
\begin{itemize}
\item {Proveniência:(Do lat. \textunderscore ventilatio\textunderscore )}
\end{itemize}
Acto ou effeito de ventilar.
\section{Ventilador}
\begin{itemize}
\item {Grp. gram.:m.}
\end{itemize}
\begin{itemize}
\item {Proveniência:(Do lat. \textunderscore ventilator\textunderscore )}
\end{itemize}
Apparelho para ventilar.
\section{Ventilante}
\begin{itemize}
\item {Grp. gram.:adj.}
\end{itemize}
\begin{itemize}
\item {Proveniência:(Lat. \textunderscore ventilans\textunderscore )}
\end{itemize}
Que ventila.
\section{Ventilar}
\begin{itemize}
\item {Grp. gram.:v. t.}
\end{itemize}
\begin{itemize}
\item {Utilização:Fig.}
\end{itemize}
\begin{itemize}
\item {Proveniência:(Lat. \textunderscore ventilare\textunderscore )}
\end{itemize}
Introduzir vento em.
Expor ao vento.
Estabelecer corrente de ar em.
Renovar o ar em; arejar: \textunderscore ventilar uma casa\textunderscore .
Limpar com joeira ou pá (o trigo ou outros cereaes).
Debater, discutir.
\section{Ventilativo}
\begin{itemize}
\item {Grp. gram.:adj.}
\end{itemize}
Próprio para ventilar; que ventila.
\section{Vento}
\begin{itemize}
\item {Grp. gram.:m.}
\end{itemize}
\begin{itemize}
\item {Utilização:Fig.}
\end{itemize}
\begin{itemize}
\item {Grp. gram.:Loc.}
\end{itemize}
\begin{itemize}
\item {Utilização:fig.}
\end{itemize}
\begin{itemize}
\item {Utilização:Ant.}
\end{itemize}
\begin{itemize}
\item {Utilização:Jur.}
\end{itemize}
\begin{itemize}
\item {Grp. gram.:Loc. adv.}
\end{itemize}
\begin{itemize}
\item {Grp. gram.:Pl.}
\end{itemize}
\begin{itemize}
\item {Utilização:Prov.}
\end{itemize}
\begin{itemize}
\item {Utilização:trasm.}
\end{itemize}
\begin{itemize}
\item {Proveniência:(Lat. \textunderscore ventus\textunderscore )}
\end{itemize}
Corrente de ar, mais ou menos rápida, resultante das mudanças no pêso específico da atmosphera.
Ar.
Atmosphera.
Espécie de bolha, que apparece nas obras fundidas e proveniente de uma porção de ar que entrou no metal ao solidificar-se.
Influência.
Flatulência; ventosidade.
Faro.
Coisa rápida.
Vaidade, coisa van.
\textunderscore Pé de vento\textunderscore , redemoínho, furacão.
\textunderscore De vento em popa\textunderscore , prosperamente.
\textunderscore Cama de vento\textunderscore , antigo utensílio de borracha, que, enchendo-se de ar, como os chamados \textunderscore cintos-de-salvação\textunderscore , ou como certos assentos de goma elástica, usados em viagem, servia de leito, a bordo:«\textunderscore deixo uma cama de vento...\textunderscore »(De um testamento de 1693)
\textunderscore Seguir no mesmo vento\textunderscore , ir na mesma direcção.
\textunderscore Gado de vento\textunderscore , animaes sem dono ou cujo dono é desconhecido.
\textunderscore Águas do vento\textunderscore , águas pluviaes, não apropriadas.
\textunderscore Com vento fresco\textunderscore , sem ceremónia; sem dizer nada: \textunderscore foi-lhe aos lombos com vento fresco\textunderscore .
Fendas de uma pedra.
\textunderscore Beber os ventos por alguém\textunderscore , gostar muito; estar disposto a tudo para servir alguêm.
\section{Ventoínha}
\begin{itemize}
\item {Grp. gram.:f.}
\end{itemize}
\begin{itemize}
\item {Utilização:Fig.}
\end{itemize}
\begin{itemize}
\item {Proveniência:(De \textunderscore vento\textunderscore )}
\end{itemize}
O mesmo que \textunderscore catavento\textunderscore .
Chincra; abibe.
Pessôa inconstante.
\section{Ventoirinho}
\begin{itemize}
\item {Grp. gram.:m.}
\end{itemize}
\begin{itemize}
\item {Utilização:Prov.}
\end{itemize}
\begin{itemize}
\item {Utilização:trasm.}
\end{itemize}
\begin{itemize}
\item {Proveniência:(De \textunderscore vento\textunderscore )}
\end{itemize}
Pouco juízo, cabeça leve.
\section{Ventoninho}
\begin{itemize}
\item {Grp. gram.:m.}
\end{itemize}
Ave, o mesmo que \textunderscore abibe\textunderscore .
(Cp. \textunderscore abitoninha\textunderscore )
\section{Ventor}
\begin{itemize}
\item {Grp. gram.:m.}
\end{itemize}
\begin{itemize}
\item {Utilização:Cyn.}
\end{itemize}
\begin{itemize}
\item {Proveniência:(De \textunderscore vento\textunderscore )}
\end{itemize}
Cão, que tem bom faro.
\section{Ventosa}
\begin{itemize}
\item {Grp. gram.:f.}
\end{itemize}
\begin{itemize}
\item {Proveniência:(Lat. \textunderscore ventosa\textunderscore )}
\end{itemize}
Espécie de vaso, que, applicando-se na pelle, produz effeito revulsivo e local, rarefazendo-se-lhe interiormente o ar.
Nome de alguns órgãos, com que certos animaes aquáticos rarefazem o ar, sugando os corpos a que adherem.
\section{Ventosidade}
\begin{itemize}
\item {Grp. gram.:f.}
\end{itemize}
\begin{itemize}
\item {Proveniência:(Do lat. \textunderscore ventositas\textunderscore )}
\end{itemize}
Accumulação de gases, no estômago ou nos intestinos.
Saída dêsses gases, mais ou menos ruidosa.
\section{Ventoso}
\begin{itemize}
\item {Grp. gram.:adj.}
\end{itemize}
\begin{itemize}
\item {Utilização:Fig.}
\end{itemize}
\begin{itemize}
\item {Grp. gram.:M.}
\end{itemize}
\begin{itemize}
\item {Proveniência:(Lat. \textunderscore ventosus\textunderscore )}
\end{itemize}
Cheio de vento.
Exposto ao vento: \textunderscore lugar ventoso\textunderscore .
Produzido por ventosidades.
Fútil; vão.
Arrogante.
Sexto mês do calendário da primeira républica francesa.
\section{Ventral}
\begin{itemize}
\item {Grp. gram.:adj.}
\end{itemize}
\begin{itemize}
\item {Proveniência:(Lat. \textunderscore ventralis\textunderscore )}
\end{itemize}
Relativo ao ventre.
Que está sôbre o abdome de certos animaes.
\section{Ventrapi}
\begin{itemize}
\item {Grp. gram.:m.}
\end{itemize}
Antigo religioso armênio. Cf. Pant. de Aveiro, \textunderscore Itiner.\textunderscore , 98, v.^o. (2.^a ed.).
\section{Ventre}
\begin{itemize}
\item {Grp. gram.:m.}
\end{itemize}
\begin{itemize}
\item {Utilização:Fig.}
\end{itemize}
\begin{itemize}
\item {Grp. gram.:Loc.}
\end{itemize}
\begin{itemize}
\item {Utilização:fam.}
\end{itemize}
\begin{itemize}
\item {Grp. gram.:Loc.}
\end{itemize}
\begin{itemize}
\item {Utilização:Prov.}
\end{itemize}
\begin{itemize}
\item {Proveniência:(Do lat. \textunderscore venter\textunderscore )}
\end{itemize}
Cavidade do corpo, que contém o estômago e os intestinos.
As vísceras, consideradas sob o ponto de vista das funcções digestivas: \textunderscore incômmodos de ventre\textunderscore .
Barriga.
Proeminência exterior do abdome: \textunderscore que grande ventre êlle tem\textunderscore !
Útero: \textunderscore antes êlle morresse no ventre da mãe\textunderscore !
Bojo ou a parte mais larga de um vaso.
Concavidade.
Parte média e mais volumosa de alguns músculos.
Parte interior, âmago: \textunderscore o juiz pôs se a estudar o ventre dos autos\textunderscore .
\textunderscore Tirar o ventre de misérias\textunderscore , dar em cheio, têr grandes vantagens, após a adversidade.
\textunderscore Dar de ventre\textunderscore , dejectar, defecar.
\section{Ventrecha}
\begin{itemize}
\item {fónica:trê}
\end{itemize}
\begin{itemize}
\item {Grp. gram.:f.}
\end{itemize}
\begin{itemize}
\item {Utilização:Ant.}
\end{itemize}
\begin{itemize}
\item {Proveniência:(Do lat. hyp. \textunderscore ventricula\textunderscore )}
\end{itemize}
Posta de peixe, immediata á cabeça.
Compartimento.
Divisão interior. Cf. Soropita, \textunderscore Poes. e Pros.\textunderscore , 20.
\section{Ventricular}
\begin{itemize}
\item {Grp. gram.:adj.}
\end{itemize}
Relativo aos ventrículos.
\section{Ventrículo}
\begin{itemize}
\item {Grp. gram.:m.}
\end{itemize}
\begin{itemize}
\item {Utilização:Anat.}
\end{itemize}
\begin{itemize}
\item {Utilização:Anat.}
\end{itemize}
\begin{itemize}
\item {Utilização:Ant.}
\end{itemize}
\begin{itemize}
\item {Utilização:Zool.}
\end{itemize}
\begin{itemize}
\item {Proveniência:(Lat. \textunderscore ventriculus\textunderscore )}
\end{itemize}
Designação de certas cavidades de certos órgãos.
Cada uma das duas cavidades inferiores do coração.
Cada uma das cinco cavidades no âmago do cérebro.
O estômago.
A cavidade única do coração de certos animaes.
\section{Ventril}
\begin{itemize}
\item {Grp. gram.:m.}
\end{itemize}
\begin{itemize}
\item {Utilização:Des.}
\end{itemize}
Nome, que se deu a uma peça de madeira destinada a equilibrar o movimento da vara que preme o bagaço nos lagares de azeite.
(Cast. \textunderscore ventril\textunderscore )
\section{Ventrilavado}
\begin{itemize}
\item {Grp. gram.:adj.}
\end{itemize}
\begin{itemize}
\item {Proveniência:(De \textunderscore ventre\textunderscore  + \textunderscore lavado\textunderscore )}
\end{itemize}
Que tem o ventre esbranquiçado, (falando-se do cavallo).
\section{Ventriloquia}
\begin{itemize}
\item {Grp. gram.:f}
\end{itemize}
Qualidade do que é ventríloquo.
\section{Ventriloquista}
\begin{itemize}
\item {Grp. gram.:m.}
\end{itemize}
Aquelle que pratica a ventriloquia.
\section{Ventríloquo}
\begin{itemize}
\item {Grp. gram.:m.  e  adj.}
\end{itemize}
\begin{itemize}
\item {Proveniência:(Lat. \textunderscore ventriloquus\textunderscore )}
\end{itemize}
Indivíduo, que modifica a voz natural, abafando-a á saída da larynge, por fórma que essa voz parece vir de longe, suppondo os antigos que ella se formava no ventre.
\section{Ventripotente}
\begin{itemize}
\item {Grp. gram.:adj.}
\end{itemize}
\begin{itemize}
\item {Proveniência:(Do lat. \textunderscore venter\textunderscore  + \textunderscore potens\textunderscore , \textunderscore potentis\textunderscore )}
\end{itemize}
Que tem estômago forte; que é gastrónomo.
\section{Ventrisca}
\begin{itemize}
\item {Grp. gram.:f.}
\end{itemize}
O mesmo que \textunderscore ventrecha\textunderscore .
\section{Ventrudo}
\begin{itemize}
\item {Grp. gram.:adj.}
\end{itemize}
Que tem grande ventre.
Barrigudo; obeso.
\section{Ventrulho}
\begin{itemize}
\item {Grp. gram.:m.}
\end{itemize}
\begin{itemize}
\item {Utilização:T. de Resende}
\end{itemize}
O mesmo que \textunderscore ventre\textunderscore .
\section{Ventura}
\begin{itemize}
\item {Grp. gram.:f.}
\end{itemize}
\begin{itemize}
\item {Grp. gram.:Loc. adv.}
\end{itemize}
\begin{itemize}
\item {Grp. gram.:M.}
\end{itemize}
\begin{itemize}
\item {Proveniência:(Lat. \textunderscore ventura\textunderscore )}
\end{itemize}
Fortuna próspera.
Destino.
Acaso.
Risco, perigo.
\textunderscore Á ventura\textunderscore , á tôa, ao acaso.
\textunderscore Por ventura\textunderscore , o mesmo que \textunderscore porventura\textunderscore .
\textunderscore Sem ventura\textunderscore , indivíduo desgraçado: \textunderscore consolai o sem ventura...\textunderscore 
\section{Ventureiro}
\begin{itemize}
\item {Grp. gram.:adj.}
\end{itemize}
\begin{itemize}
\item {Utilização:Pop.}
\end{itemize}
\begin{itemize}
\item {Grp. gram.:M.}
\end{itemize}
\begin{itemize}
\item {Proveniência:(De \textunderscore ventura\textunderscore )}
\end{itemize}
Casual, incerto, arriscado: \textunderscore as sementeiras fóra de tempo são muito ventureiras\textunderscore .
Soldado voluntário:«\textunderscore o esquadrão de ventureiros ia na vanguarda.\textunderscore »\textunderscore Jorn. de África\textunderscore , V.
\section{Venturina}
\begin{itemize}
\item {Grp. gram.:f.}
\end{itemize}
O mesmo que \textunderscore aventurina\textunderscore .
\section{Venturo}
\begin{itemize}
\item {Grp. gram.:adj.}
\end{itemize}
\begin{itemize}
\item {Proveniência:(Lat. \textunderscore venturus\textunderscore )}
\end{itemize}
Que há de vir; futuro; porvindoiro.
\section{Venturosamente}
\begin{itemize}
\item {Grp. gram.:adv.}
\end{itemize}
De modo venturoso; felizmente.
\section{Venturoso}
\begin{itemize}
\item {Grp. gram.:adj.}
\end{itemize}
Que tem ventura; ditoso.
Em que há ventura.
Arriscado.
\section{Ventusa}
\begin{itemize}
\item {Grp. gram.:f.}
\end{itemize}
\begin{itemize}
\item {Utilização:Neol.}
\end{itemize}
\begin{itemize}
\item {Proveniência:(Fr. \textunderscore ventouse\textunderscore )}
\end{itemize}
Apparelho especial, para fazer sair o vento das canalizações subterrâneas de água. Cf. \textunderscore Século\textunderscore , de 7-IX-98.
\section{Vênula}
\begin{itemize}
\item {Grp. gram.:f.}
\end{itemize}
\begin{itemize}
\item {Proveniência:(Lat. \textunderscore venula\textunderscore )}
\end{itemize}
Pequena veia, veiazinha.
\section{Vênus}
\begin{itemize}
\item {Grp. gram.:f.}
\end{itemize}
\begin{itemize}
\item {Utilização:Ext.}
\end{itemize}
\begin{itemize}
\item {Utilização:Chím.}
\end{itemize}
\begin{itemize}
\item {Utilização:Ant.}
\end{itemize}
\begin{itemize}
\item {Utilização:Anat.}
\end{itemize}
\begin{itemize}
\item {Utilização:Fam.}
\end{itemize}
\begin{itemize}
\item {Proveniência:(De \textunderscore Vênus\textunderscore , n. p.)}
\end{itemize}
Mulhér muito formosa.
Um dos sete planetas principaes.
Gênero de conchas bivalves.
O mesmo que \textunderscore cobre\textunderscore .
\textunderscore Monte de Vênus\textunderscore , proeminência púbica da mulhér.
\textunderscore Camisa de Vênus\textunderscore , envoltório preservativo de doenças venéreas.
\section{Venusino}
\begin{itemize}
\item {Grp. gram.:adj.}
\end{itemize}
\begin{itemize}
\item {Grp. gram.:M.}
\end{itemize}
\begin{itemize}
\item {Proveniência:(Lat. \textunderscore venusinus\textunderscore )}
\end{itemize}
Relativo a Venúsia, cidade italiana e pátria de Horácio.
Relativo a Horácio ou ao seu talento poético:«\textunderscore o plectro venusino...\textunderscore »Bocage.
O poeta Horácio.
\section{Venúsio}
\begin{itemize}
\item {Grp. gram.:m.}
\end{itemize}
\begin{itemize}
\item {Utilização:Miner.}
\end{itemize}
Espécie de cobre.
\section{Venustidade}
\begin{itemize}
\item {Grp. gram.:f.}
\end{itemize}
Qualidade de venusto. Cf. Filinto, XI, 295.
\section{Venusto}
\begin{itemize}
\item {Grp. gram.:adj.}
\end{itemize}
\begin{itemize}
\item {Proveniência:(Lat. \textunderscore venustus\textunderscore )}
\end{itemize}
Muito formoso ou muito gracioso.
\section{Veo}
\begin{itemize}
\item {Grp. gram.:m.}
\end{itemize}
\begin{itemize}
\item {Utilização:Ant.}
\end{itemize}
O mesmo que \textunderscore vello\textunderscore . Cf. \textunderscore Lusíadas\textunderscore , III, 72; IV, 83.
\section{Vépris}
\begin{itemize}
\item {Grp. gram.:m.}
\end{itemize}
Gênero de plantas xanthoxýleas.
\section{Vepsa}
\begin{itemize}
\item {Grp. gram.:m.}
\end{itemize}
Língua uralo-altaica do grupo ugro-finlandês.
\section{Vequiá!}
\begin{itemize}
\item {Grp. gram.:interj.}
\end{itemize}
\begin{itemize}
\item {Utilização:T. do Fundão}
\end{itemize}
(Us. para chamar porcos)
\section{Vêr}
\begin{itemize}
\item {Grp. gram.:v. t.}
\end{itemize}
\begin{itemize}
\item {Utilização:Fam.}
\end{itemize}
\begin{itemize}
\item {Grp. gram.:V. p.}
\end{itemize}
\begin{itemize}
\item {Grp. gram.:M.}
\end{itemize}
\begin{itemize}
\item {Proveniência:(Do lat. \textunderscore videre\textunderscore )}
\end{itemize}
Conhecer ou perceber pelo sentido da vista.
Contemplar; observar.
Sêr testemunha de.
Examinar.
Enxergar.
Advertir.
Idear; imaginar.
Calcular.
Recordar.
Ponderar; deduzir.
Antever.
Apreciar.
Visitar: \textunderscore foi vêr os parentes\textunderscore .
\textunderscore Vêr por um óculo\textunderscore , não conseguir.
\textunderscore Ficar a vêr navios\textunderscore , ficar logrado; não obter o que pretendia.
Encontrar-se, têr entrevista:«\textunderscore foi El-Rei ao lugar da fonte do Arcebispo, onde se viu com El-Rei D. Anrique.\textunderscore »R. Pina, \textunderscore Aff.\textunderscore  V, CLVII.
\textunderscore Vêr-se e desejar-se\textunderscore , estar muito afflicto, muito embaraçado.
Conceito, opinião: \textunderscore isso a meu vêr, é asneira\textunderscore .
\section{Veracidade}
\begin{itemize}
\item {Grp. gram.:f.}
\end{itemize}
\begin{itemize}
\item {Proveniência:(Do lat. \textunderscore veracitas\textunderscore )}
\end{itemize}
Qualidade do que é veraz; verdade.
Amor á verdade.
\section{Vera-effígie}
\begin{itemize}
\item {Grp. gram.:f.}
\end{itemize}
\begin{itemize}
\item {Proveniência:(De \textunderscore vero\textunderscore  + \textunderscore effigie\textunderscore )}
\end{itemize}
Retrato fiel; cópia perfeita.
\section{Veramente}
\begin{itemize}
\item {Grp. gram.:adv.}
\end{itemize}
De modo vero.
Exactamente; realmente.
\section{Veranear}
\begin{itemize}
\item {Grp. gram.:v. i.}
\end{itemize}
Passar o Verão algures.
\section{Veranico}
\begin{itemize}
\item {Grp. gram.:m.}
\end{itemize}
\begin{itemize}
\item {Proveniência:(De \textunderscore verão\textunderscore )}
\end{itemize}
O mesmo que \textunderscore verãozinho\textunderscore :«\textunderscore entrou S. Martinho com seu veranico.\textunderscore »Vieira.
\section{Veranito}
\begin{itemize}
\item {Grp. gram.:m.}
\end{itemize}
O mesmo que \textunderscore veranico\textunderscore .
\section{Verão}
\begin{itemize}
\item {Grp. gram.:m.}
\end{itemize}
\begin{itemize}
\item {Utilização:Ant.}
\end{itemize}
Quadra do anno, immediata á Primavera; estio.
Tempo quente.
O mesmo que \textunderscore Primavera\textunderscore .
(Cast. \textunderscore verano\textunderscore )
\section{Verãozinho}
\begin{itemize}
\item {Grp. gram.:m.}
\end{itemize}
Pequeno Verão.
Tempo quente, de pouca duração.
\section{Veras}
\begin{itemize}
\item {Grp. gram.:f. pl.}
\end{itemize}
\begin{itemize}
\item {Grp. gram.:Loc. adv.}
\end{itemize}
\begin{itemize}
\item {Grp. gram.:Loc. adv.}
\end{itemize}
Coisas verdadeiras; realidade.
\textunderscore De veras\textunderscore , o mesmo que \textunderscore devéras\textunderscore .
\textunderscore Com todas as veras\textunderscore , com toda a verdade, muito realmente; de todo o coração; cordealissimamente.
(Fem. pl. de \textunderscore vero\textunderscore )
\section{Verascópio}
\begin{itemize}
\item {Grp. gram.:m.}
\end{itemize}
\begin{itemize}
\item {Proveniência:(Do lat. \textunderscore verus\textunderscore  + gr. \textunderscore skopein\textunderscore )}
\end{itemize}
Moderno apparelho photográphico, que tem o aspecto de um binóculo de theatro e as vantagens do estereoscópio.
\section{Verátrico}
\begin{itemize}
\item {Grp. gram.:adj.}
\end{itemize}
Relativo ao veratro.
\section{Veratrina}
\begin{itemize}
\item {Grp. gram.:f.}
\end{itemize}
Alcalóide, que se encontra no veratro.
\section{Veratrinato}
\begin{itemize}
\item {Grp. gram.:m.}
\end{itemize}
\begin{itemize}
\item {Proveniência:(De \textunderscore veratrina\textunderscore )}
\end{itemize}
Sal, resultante da combinação do ácido verátrico com uma base.
\section{Veratrino}
\begin{itemize}
\item {Grp. gram.:m.}
\end{itemize}
\begin{itemize}
\item {Utilização:Chím.}
\end{itemize}
Substância resinosa, que acompanha a veratrina.
\section{Veratro}
\begin{itemize}
\item {Grp. gram.:m.}
\end{itemize}
\begin{itemize}
\item {Proveniência:(Lat. \textunderscore veratrum\textunderscore )}
\end{itemize}
Gênero de plantas colchicáceas, a que pertence uma planta medicinal, o helléboro branco, (\textunderscore veratrum album\textunderscore , Lin.).
\section{Veraz}
\begin{itemize}
\item {Grp. gram.:adj.}
\end{itemize}
\begin{itemize}
\item {Proveniência:(Lat. \textunderscore verax\textunderscore )}
\end{itemize}
Que diz a verdade; em que há verdade; veridico.
\section{Verba}
\begin{itemize}
\item {Grp. gram.:f.}
\end{itemize}
\begin{itemize}
\item {Utilização:Ext.}
\end{itemize}
\begin{itemize}
\item {Proveniência:(Lat. \textunderscore verba\textunderscore )}
\end{itemize}
Cada uma das cláusulas ou artigos de um documento ou escritura.
Parcella.
Commentário; nota.
Quantia.
\section{Verbal}
\begin{itemize}
\item {Grp. gram.:adj.}
\end{itemize}
\begin{itemize}
\item {Proveniência:(Lat. \textunderscore verbalis\textunderscore )}
\end{itemize}
Relativo ao verbo; oral.
Expresso ou significado de viva voz: \textunderscore confissão verbal\textunderscore .
\section{Verbalizar}
\begin{itemize}
\item {Grp. gram.:v.}
\end{itemize}
\begin{itemize}
\item {Utilização:t. Gram.}
\end{itemize}
Tornar verbal.
\section{Verbalmente}
\begin{itemize}
\item {Grp. gram.:adv.}
\end{itemize}
De modo verbal.
\section{Verbascáceas}
\begin{itemize}
\item {Grp. gram.:f. pl.}
\end{itemize}
Família de plantas, separada das escrofularíneas e que tem por typo o verbasco.
\section{Verbasco}
\begin{itemize}
\item {Grp. gram.:m.}
\end{itemize}
\begin{itemize}
\item {Proveniência:(Lat. \textunderscore verbascum\textunderscore )}
\end{itemize}
Gênero de plantas, entre as quaes se distingue o verbasco branco, (\textunderscore verbascum thapsus\textunderscore , Lin.), de flôres e fôlhas medicinaes.
\section{Verbejar}
\begin{itemize}
\item {Grp. gram.:v. i.}
\end{itemize}
\begin{itemize}
\item {Utilização:Ant.}
\end{itemize}
\begin{itemize}
\item {Proveniência:(De \textunderscore verbo\textunderscore )}
\end{itemize}
Palavrear.
Pronunciar ditados, dizer anexins.
\section{Verbena}
\begin{itemize}
\item {Grp. gram.:f.}
\end{itemize}
\begin{itemize}
\item {Proveniência:(Lat. \textunderscore verbena\textunderscore )}
\end{itemize}
Gênero de plantas, em que se distingue a verbena odorífera ou lucialima, e a verbena officinal.
\section{Verbenáceas}
\begin{itemize}
\item {Grp. gram.:f. pl.}
\end{itemize}
\begin{itemize}
\item {Proveniência:(De \textunderscore verbenáceo\textunderscore )}
\end{itemize}
Família de plantas, que tem por typo a verbena.
\section{Verbenáceo}
\begin{itemize}
\item {Grp. gram.:adj.}
\end{itemize}
Relativo ou semelhante a verbena.
\section{Verberação}
\begin{itemize}
\item {Grp. gram.:f.}
\end{itemize}
\begin{itemize}
\item {Proveniência:(Do lat. \textunderscore verberatio\textunderscore )}
\end{itemize}
Acto ou effeito de verberar.
\section{Verberador}
\begin{itemize}
\item {Grp. gram.:adj.}
\end{itemize}
O mesmo que \textunderscore verberante\textunderscore .
\section{Verberante}
\begin{itemize}
\item {Grp. gram.:adj.}
\end{itemize}
\begin{itemize}
\item {Proveniência:(Lat. \textunderscore verberans\textunderscore )}
\end{itemize}
Que verbera.
\section{Verberão}
\begin{itemize}
\item {Grp. gram.:m.}
\end{itemize}
O mesmo que \textunderscore urgebão\textunderscore .
\section{Verberar}
\begin{itemize}
\item {Grp. gram.:v. t.}
\end{itemize}
\begin{itemize}
\item {Utilização:Fig.}
\end{itemize}
\begin{itemize}
\item {Grp. gram.:V. i.}
\end{itemize}
\begin{itemize}
\item {Proveniência:(Lat. \textunderscore verberare\textunderscore )}
\end{itemize}
Açoitar; flagellar.
Censurar.
Reprovar energicamente.
O mesmo que \textunderscore reverberar\textunderscore .
\section{Verberativo}
\begin{itemize}
\item {Grp. gram.:adj.}
\end{itemize}
Próprio para verberar ou flagellar.
\section{Verbesina}
\begin{itemize}
\item {Grp. gram.:f.}
\end{itemize}
\begin{itemize}
\item {Proveniência:(De \textunderscore verbena\textunderscore )}
\end{itemize}
Gênero de plantas americanas, da fam. das synanthéreas.
\section{Verbesina-da-índia}
\begin{itemize}
\item {Grp. gram.:f.}
\end{itemize}
Planta, (\textunderscore guisotia abyssinica\textunderscore , Cass).
\section{Verbete}
\begin{itemize}
\item {fónica:bê}
\end{itemize}
\begin{itemize}
\item {Grp. gram.:m.}
\end{itemize}
Apontamento, nota.
Pequeno papel, em que se faz um apontamento ou nota.
(Cp. cast. \textunderscore bervete\textunderscore )
\section{Verbiagem}
\begin{itemize}
\item {Grp. gram.:f.}
\end{itemize}
\begin{itemize}
\item {Utilização:Bras}
\end{itemize}
\begin{itemize}
\item {Proveniência:(De \textunderscore verbo\textunderscore )}
\end{itemize}
Palanfrório; verborrheia.
\section{Verbo}
\begin{itemize}
\item {Grp. gram.:m.}
\end{itemize}
\begin{itemize}
\item {Utilização:Gram.}
\end{itemize}
\begin{itemize}
\item {Utilização:Ant.}
\end{itemize}
\begin{itemize}
\item {Utilização:Fam.}
\end{itemize}
\begin{itemize}
\item {Proveniência:(Lat. \textunderscore verbum\textunderscore )}
\end{itemize}
Palavra.
Tom de voz.
A segunda pessôa da Trindade christan.
A sabedoria eterna.
Palavra, que exprime existência, estado, qualidade ou acção de uma pessôa ou coisa.
Palavra, que indíca a existência de um attributo no sujeito.
O mesmo que \textunderscore expressão\textunderscore .
Rifão, ditado. Cf. Herculano, \textunderscore Quest. Públ.\textunderscore , II, 81.
\textunderscore Verbo de encher\textunderscore , palavra expletiva.
\section{Verborreia}
\begin{itemize}
\item {Grp. gram.:f.}
\end{itemize}
\begin{itemize}
\item {Utilização:Deprec.}
\end{itemize}
\begin{itemize}
\item {Proveniência:(Do lat. \textunderscore verbum\textunderscore  + gr. \textunderscore rhein\textunderscore )}
\end{itemize}
Qualidade de quem fala ou discute com grande fluência e abundância de palavras, mas com poucas ideias; logorrheia.
\section{Verbosamente}
\begin{itemize}
\item {Grp. gram.:adv.}
\end{itemize}
De modo verboso.
Com muitas palavras.
\section{Verbosidade}
\begin{itemize}
\item {Grp. gram.:f.}
\end{itemize}
\begin{itemize}
\item {Proveniência:(Do lat. \textunderscore verbositas\textunderscore )}
\end{itemize}
Qualidade do que é verboso.
\section{Verboso}
\begin{itemize}
\item {Grp. gram.:adj.}
\end{itemize}
\begin{itemize}
\item {Proveniência:(Lat. \textunderscore verbosus\textunderscore )}
\end{itemize}
Que fala muito, loquaz; palavroso.
Que fala com facilidade.
\section{Verdacho}
\begin{itemize}
\item {Grp. gram.:adj.}
\end{itemize}
\begin{itemize}
\item {Grp. gram.:M.}
\end{itemize}
O mesmo que \textunderscore esverdeado\textunderscore .
Tinta, de côr tirante a verde, ou da côr da cana verde.
\section{Verdade}
\begin{itemize}
\item {Grp. gram.:f.}
\end{itemize}
\begin{itemize}
\item {Proveniência:(Do lat. \textunderscore veritas\textunderscore )}
\end{itemize}
Qualidade, pela qual as coisas se apresentam taes quaes são.
Realidade; coisa verdadeira.
Sinceridade.
Opinião conforme á realidade: \textunderscore dizer a verdade\textunderscore .
Princípio exacto.
Representação fiel de alguma coisa, que exista em a natureza.
Carácter.
\section{Verdadeiramente}
\begin{itemize}
\item {Grp. gram.:adv.}
\end{itemize}
De modo verdadeiro.
Com verdade.
Na verdade; realmente.
\section{Verdadeiro}
\begin{itemize}
\item {Grp. gram.:adj.}
\end{itemize}
\begin{itemize}
\item {Grp. gram.:M.}
\end{itemize}
\begin{itemize}
\item {Proveniência:(De \textunderscore verdade\textunderscore )}
\end{itemize}
Em que há verdade: \textunderscore palavras verdadeiras\textunderscore .
Que fala verdade: \textunderscore é homem verdadeiro\textunderscore .
Conforme á verdade; exacto: \textunderscore história verdadeira\textunderscore .
Authêntico, genuíno: \textunderscore verdadeiro vinho do Pôrto\textunderscore .
Sincero.
A verdade; o dever.
\section{Verdadíssima}
\begin{itemize}
\item {Grp. gram.:f.}
\end{itemize}
\begin{itemize}
\item {Utilização:Fam.}
\end{itemize}
\begin{itemize}
\item {Proveniência:(De \textunderscore verdade\textunderscore )}
\end{itemize}
Us. na loc. \textunderscore verdade verdadíssima\textunderscore  = verdade inteira, absoluta.
\section{Verdaduras}
\begin{itemize}
\item {Grp. gram.:f. pl.}
\end{itemize}
\begin{itemize}
\item {Utilização:Ant.}
\end{itemize}
O mesmo que \textunderscore esverdados\textunderscore .
\section{Verdal}
\begin{itemize}
\item {Grp. gram.:adj.}
\end{itemize}
Diz-se de uma variedade de milho amarelado, de caule delgado e alto. Cf. \textunderscore Port. au point de vue agr.\textunderscore , 578.
\section{Verdasca}
\begin{itemize}
\item {Grp. gram.:f.}
\end{itemize}
\begin{itemize}
\item {Proveniência:(De \textunderscore verde\textunderscore ? Corr. de \textunderscore vergasta\textunderscore ? Entretanto, cp. cast. \textunderscore vardasca\textunderscore , que presuppõe o rad. de \textunderscore vara\textunderscore )}
\end{itemize}
Pequena vara, muito flexível.
\section{Verdascada}
\begin{itemize}
\item {Grp. gram.:f.}
\end{itemize}
Pancada com verdasca.
\section{Verdascar}
\begin{itemize}
\item {Grp. gram.:v. t.}
\end{itemize}
Dar verdascadas em.
\section{Verdasco}
\begin{itemize}
\item {Grp. gram.:m.  e  adj.}
\end{itemize}
\begin{itemize}
\item {Proveniência:(De \textunderscore verde\textunderscore )}
\end{itemize}
Diz-se de uma espécie de vinho verde, muito ácido.
Diz-se do vinho entre maduro e verde.
\section{Verde}
\begin{itemize}
\item {fónica:vêr}
\end{itemize}
\begin{itemize}
\item {Grp. gram.:adj.}
\end{itemize}
\begin{itemize}
\item {Utilização:Fig.}
\end{itemize}
\begin{itemize}
\item {Grp. gram.:M.}
\end{itemize}
\begin{itemize}
\item {Utilização:T. de Turquel}
\end{itemize}
\begin{itemize}
\item {Utilização:Prov.}
\end{itemize}
\begin{itemize}
\item {Utilização:trasm.}
\end{itemize}
\begin{itemize}
\item {Utilização:Prov.}
\end{itemize}
\begin{itemize}
\item {Utilização:alent.}
\end{itemize}
\begin{itemize}
\item {Utilização:Prov.}
\end{itemize}
\begin{itemize}
\item {Utilização:Bras. do N}
\end{itemize}
\begin{itemize}
\item {Utilização:Gír.}
\end{itemize}
\begin{itemize}
\item {Utilização:Prov.}
\end{itemize}
\begin{itemize}
\item {Utilização:minh.}
\end{itemize}
\begin{itemize}
\item {Proveniência:(Do lat. \textunderscore viridis\textunderscore )}
\end{itemize}
Que é da côr ordinária da erva e das fôlhas das árvores.
Que aínda tem seiva, (falando-se de plantas).
Que aínda não amadureceu: \textunderscore fruta verde\textunderscore .
Que não está sêco.
Fresco, (falando-se da carne).
Tenro; delicado.
Fraco.
Inexperiente.
Relativo aos primeiros tempos da existência: \textunderscore em annos verdes\textunderscore .
Diz-se do vinho, feito de uvas pouco saccharinas ou pouco maduras.
A côr verde.
Erva de pasto para animaes.
\textunderscore Caldo verde\textunderscore , caldo de fôlhas de nabos ou nabiças, ou de couves tenras muito migadas, temperado com azeite.
Intervallo de dois regos.
O mesmo que \textunderscore sangue\textunderscore .
Iguaria de sangue de porco.
Vinho verde.
A estação das chuvas.
O frio.
\textunderscore Chouriça de verde\textunderscore , chouriço de sangue, com gordura e cebola picada.
\section{Verdeal}
\begin{itemize}
\item {Grp. gram.:adj.}
\end{itemize}
\begin{itemize}
\item {Grp. gram.:M.}
\end{itemize}
\begin{itemize}
\item {Utilização:Prov.}
\end{itemize}
\begin{itemize}
\item {Utilização:minh.}
\end{itemize}
Que tem côr tirante a verde.
Diz-se de uma variedade de trigo, de oliveira, de uva, de azeitona e de pêra.
Nome vulgar dos archeiros da Universidade, que antigamente trajavam de verde.
Planta gramínea.
\section{Verdeal-branca}
\begin{itemize}
\item {Grp. gram.:f.}
\end{itemize}
Casta de uva branca do Doiro e da Beira-Alta.
\section{Verdeal-tinto}
\begin{itemize}
\item {Grp. gram.:m.}
\end{itemize}
Casta de uva preta.
\section{Verde-alvo}
\begin{itemize}
\item {Grp. gram.:adj.}
\end{itemize}
O mesmo que \textunderscore verde-claro\textunderscore . Cf. Garrett, \textunderscore D. Branca\textunderscore , XV.
\section{Verdear}
\begin{itemize}
\item {Grp. gram.:v. i.}
\end{itemize}
\begin{itemize}
\item {Utilização:Bras. do S}
\end{itemize}
O mesmo que \textunderscore verdejar\textunderscore .
Dar ração de capim verde ao cavallo.
\section{Verde-bexiga}
\begin{itemize}
\item {Grp. gram.:m.}
\end{itemize}
Tinta verde-escura, cujo principal ingrediente é o fel de vaca, e que se usa na Pintura, especialmente para representar illuminações.
\section{Verde-bronze}
\begin{itemize}
\item {Grp. gram.:adj.}
\end{itemize}
Que participa da côr verde e da do bronze. Cf. Castilho, \textunderscore Fastos\textunderscore , II, 373 e 374.
\section{Verdecer}
\begin{itemize}
\item {Grp. gram.:v. i.}
\end{itemize}
\begin{itemize}
\item {Proveniência:(Lat. \textunderscore viridiscere\textunderscore )}
\end{itemize}
Tomar côr verde, tornar-se verde.
\section{Verde-claro}
\begin{itemize}
\item {Grp. gram.:adj.}
\end{itemize}
Entre verde e branco.
\section{Verde-cré}
\begin{itemize}
\item {Grp. gram.:m.}
\end{itemize}
Verde sôbre oiro.
\section{Verde-escuro}
\begin{itemize}
\item {Grp. gram.:adj.}
\end{itemize}
Entre verde e preto.
\section{Verde-gaio}
\begin{itemize}
\item {Grp. gram.:adj.}
\end{itemize}
\begin{itemize}
\item {Grp. gram.:M.}
\end{itemize}
O mesmo que \textunderscore verde-claro\textunderscore .
Música e dança popular.
(Segundo Bluteau, do fr. \textunderscore verdgai\textunderscore , verde alegre)
\section{Verdegar}
\begin{itemize}
\item {Grp. gram.:v. i.}
\end{itemize}
\begin{itemize}
\item {Utilização:Pop.}
\end{itemize}
\begin{itemize}
\item {Proveniência:(Do lat. hyp. \textunderscore viridicare\textunderscore )}
\end{itemize}
O mesmo que \textunderscore verdejar\textunderscore : \textunderscore em Janeiro, se vires verdegar, põe-te a chorar\textunderscore . (Prolóquio popular)
\section{Verdegulho}
\begin{itemize}
\item {Grp. gram.:m.}
\end{itemize}
Estoque de toireiro, mais comprido e estreito do que é vulgar.
(Cp. \textunderscore verduguilho\textunderscore )
\section{Verdeia}
\begin{itemize}
\item {Grp. gram.:f.}
\end{itemize}
Vinho branco, de côr esverdeada.
(Cast. \textunderscore verdea\textunderscore )
\section{Verdeio}
\begin{itemize}
\item {Grp. gram.:m.}
\end{itemize}
\begin{itemize}
\item {Utilização:Bras. do S}
\end{itemize}
\begin{itemize}
\item {Proveniência:(De \textunderscore verdear\textunderscore )}
\end{itemize}
Forragem verde para o cavallo.
Acto de dar forragem verde para o cavallo.
\section{Verdejante}
\begin{itemize}
\item {Grp. gram.:adj.}
\end{itemize}
Que verdeja.
\section{Verdejar}
\begin{itemize}
\item {Grp. gram.:v. i.}
\end{itemize}
Apresentar-se verde.
Sêr verde.
Verdecer.
\section{Verdelha}
\begin{itemize}
\item {fónica:dê}
\end{itemize}
\begin{itemize}
\item {Grp. gram.:f.}
\end{itemize}
\begin{itemize}
\item {Proveniência:(De \textunderscore verde\textunderscore )}
\end{itemize}
Pássaro conirostro, (\textunderscore emberiza citrinella\textunderscore ).
Verdelhão.
\section{Verdelhão}
\begin{itemize}
\item {Grp. gram.:m.}
\end{itemize}
\begin{itemize}
\item {Proveniência:(De \textunderscore verdelha\textunderscore )}
\end{itemize}
Pássaro conirostro, (\textunderscore loxia chloris\textunderscore ).
Peixe de Portugal.
\section{Verdelho}
\begin{itemize}
\item {fónica:dê}
\end{itemize}
\begin{itemize}
\item {Grp. gram.:m.  e  adj.}
\end{itemize}
\begin{itemize}
\item {Proveniência:(De \textunderscore verde\textunderscore )}
\end{itemize}
Variedade de uva branca, o mesmo que \textunderscore gouveío\textunderscore .
\section{Verdelho-feijão}
\begin{itemize}
\item {Grp. gram.:m.}
\end{itemize}
Casta de uva minhota.
\section{Verdelho-tinto}
\begin{itemize}
\item {Grp. gram.:m.}
\end{itemize}
Casta de uva preta do Minho.
\section{Verde-limo}
\begin{itemize}
\item {Grp. gram.:m.}
\end{itemize}
\begin{itemize}
\item {Utilização:Gír.}
\end{itemize}
A prisão do Limoeiro, em Lisbôa.
\section{Verde-mar}
\begin{itemize}
\item {Grp. gram.:m.  e  adj.}
\end{itemize}
O mesmo que \textunderscore verde-claro\textunderscore .
\section{Verde-mau}
\begin{itemize}
\item {Grp. gram.:m.}
\end{itemize}
Peixe de Portugal.
\section{Verde-montanha}
\begin{itemize}
\item {Grp. gram.:adj.}
\end{itemize}
\begin{itemize}
\item {Grp. gram.:M.}
\end{itemize}
O mesmo que \textunderscore verde-escuro\textunderscore .
A côr verde, com uns tons levemente azulados.
Espécie de tinta, de origem mineral, de que se servem os pintores, especialmente para representar coloração semelhante á dos montes, vistos de longe.
\section{Verde-negro}
\begin{itemize}
\item {Grp. gram.:adj.}
\end{itemize}
O mesmo que \textunderscore verde-escuro\textunderscore .
\section{Verdengório}
\begin{itemize}
\item {Grp. gram.:m.}
\end{itemize}
\begin{itemize}
\item {Utilização:Prov.}
\end{itemize}
\begin{itemize}
\item {Utilização:beir.}
\end{itemize}
\begin{itemize}
\item {Proveniência:(Do rad. de \textunderscore verde\textunderscore )}
\end{itemize}
Campo, coberto de verdura.
\section{Verdepeso}
\begin{itemize}
\item {fónica:pê}
\end{itemize}
\begin{itemize}
\item {Grp. gram.:m.}
\end{itemize}
\begin{itemize}
\item {Utilização:Des.}
\end{itemize}
O mesmo que \textunderscore ver-o-pêso\textunderscore .
\section{Verderol}
\begin{itemize}
\item {Grp. gram.:m.}
\end{itemize}
\begin{itemize}
\item {Proveniência:(T. cast.)}
\end{itemize}
Ave, o mesmo que \textunderscore verdelhão\textunderscore .
\section{Verdeselha}
\begin{itemize}
\item {fónica:zê}
\end{itemize}
\begin{itemize}
\item {Grp. gram.:f.}
\end{itemize}
\begin{itemize}
\item {Proveniência:(De \textunderscore verde\textunderscore )}
\end{itemize}
O mesmo que \textunderscore corriola\textunderscore .
\section{Verdete}
\begin{itemize}
\item {fónica:dê}
\end{itemize}
\begin{itemize}
\item {Grp. gram.:m.}
\end{itemize}
\begin{itemize}
\item {Proveniência:(De \textunderscore verde\textunderscore )}
\end{itemize}
Nome do acetato de cobre, sal venenoso.
Tinta de azebre.
Casta de uva.
\section{Verde-virgo}
\begin{itemize}
\item {Grp. gram.:m.}
\end{itemize}
\begin{itemize}
\item {Utilização:Bras}
\end{itemize}
Planta medicinal.
\section{Verdial}
\begin{itemize}
\item {Grp. gram.:m.  e  adj.}
\end{itemize}
(V.verdeal)
\section{Vérdico}
\begin{itemize}
\item {Grp. gram.:adj.}
\end{itemize}
\begin{itemize}
\item {Proveniência:(De \textunderscore verde\textunderscore )}
\end{itemize}
Diz-se de um ácido, descoberto em grande número de plantas.
\section{Verdilhão}
\begin{itemize}
\item {Grp. gram.:m.}
\end{itemize}
(V. \textunderscore verdelhão\textunderscore , ave)
\section{Verdilhote}
\begin{itemize}
\item {Grp. gram.:m.}
\end{itemize}
(V. \textunderscore verdelhão\textunderscore , ave)
\section{Verdiseco}
\begin{itemize}
\item {fónica:sê}
\end{itemize}
\begin{itemize}
\item {Grp. gram.:adj.}
\end{itemize}
\begin{itemize}
\item {Utilização:P. us.}
\end{itemize}
\begin{itemize}
\item {Proveniência:(De \textunderscore verde\textunderscore  + \textunderscore sêco\textunderscore )}
\end{itemize}
Meio sêco, quási sêco.
\section{Verdisseco}
\begin{itemize}
\item {Grp. gram.:adj.}
\end{itemize}
\begin{itemize}
\item {Utilização:P. us.}
\end{itemize}
\begin{itemize}
\item {Proveniência:(De \textunderscore verde\textunderscore  + \textunderscore sêco\textunderscore )}
\end{itemize}
Meio sêco, quási sêco.
\section{Verdizela}
\begin{itemize}
\item {Grp. gram.:f.}
\end{itemize}
\begin{itemize}
\item {Utilização:Ext.}
\end{itemize}
\begin{itemize}
\item {Grp. gram.:M.}
\end{itemize}
\begin{itemize}
\item {Utilização:Prov.}
\end{itemize}
\begin{itemize}
\item {Proveniência:(De \textunderscore verde\textunderscore )}
\end{itemize}
Vara flexível, com que se arma a boíz.
Um dos paus da armadilha, chamada loisa.
Ave, o mesmo que \textunderscore abibe\textunderscore .
Rapaz alto e magro ou fraco.
O mesmo que \textunderscore verdeselha\textunderscore .
\section{Verdizella}
\begin{itemize}
\item {Grp. gram.:f.}
\end{itemize}
\begin{itemize}
\item {Utilização:Ext.}
\end{itemize}
\begin{itemize}
\item {Grp. gram.:M.}
\end{itemize}
\begin{itemize}
\item {Utilização:Prov.}
\end{itemize}
\begin{itemize}
\item {Proveniência:(De \textunderscore verde\textunderscore )}
\end{itemize}
Vara flexível, com que se arma a boíz.
Um dos paus da armadilha, chamada loisa.
Ave, o mesmo que \textunderscore abibe\textunderscore .
Rapaz alto e magro ou fraco.
O mesmo que \textunderscore verdeselha\textunderscore .
\section{Verdizello}
\begin{itemize}
\item {Grp. gram.:m.}
\end{itemize}
\begin{itemize}
\item {Proveniência:(De \textunderscore verde\textunderscore )}
\end{itemize}
Verdelhão.
Variedade de alvéola.
\section{Verdizellos}
\begin{itemize}
\item {Grp. gram.:m. pl.}
\end{itemize}
\begin{itemize}
\item {Utilização:Ant.}
\end{itemize}
Castiçaes de pau, em que se enrolavam os pavios que ardiam nos offícios divinos. Cf. S. R. Viterbo, \textunderscore Elucidário\textunderscore .
(Cp. \textunderscore verdizello\textunderscore )
\section{Verdizelo}
\begin{itemize}
\item {Grp. gram.:m.}
\end{itemize}
\begin{itemize}
\item {Proveniência:(De \textunderscore verde\textunderscore )}
\end{itemize}
Verdelhão.
Variedade de alvéola.
\section{Verdizelos}
\begin{itemize}
\item {Grp. gram.:m. pl.}
\end{itemize}
\begin{itemize}
\item {Utilização:Ant.}
\end{itemize}
Castiçaes de pau, em que se enrolavam os pavios que ardiam nos ofícios divinos. Cf. S. R. Viterbo, \textunderscore Elucidário\textunderscore .
(Cp. \textunderscore verdizello\textunderscore )
\section{Verdoengo}
\begin{itemize}
\item {Grp. gram.:adj.}
\end{itemize}
\begin{itemize}
\item {Proveniência:(De \textunderscore verde\textunderscore )}
\end{itemize}
Esverdeado.
Que não está bem maduro: \textunderscore fruta verdoenga\textunderscore .
\section{Verdor}
\begin{itemize}
\item {Grp. gram.:m.}
\end{itemize}
\begin{itemize}
\item {Utilização:Fig.}
\end{itemize}
\begin{itemize}
\item {Proveniência:(De \textunderscore verde\textunderscore )}
\end{itemize}
Propriedade do que é verde.
Côr verde dos vegetaes.
Verdura.
Inexperiência.
Viço.
Vigor.
\section{Verdoso}
\begin{itemize}
\item {Grp. gram.:adj.}
\end{itemize}
\begin{itemize}
\item {Utilização:Neol.}
\end{itemize}
\begin{itemize}
\item {Proveniência:(De \textunderscore verde\textunderscore )}
\end{itemize}
Esverdeado; verdejante.
\section{Verdugada}
\begin{itemize}
\item {Grp. gram.:f.}
\end{itemize}
\begin{itemize}
\item {Utilização:Ant.}
\end{itemize}
\begin{itemize}
\item {Proveniência:(De \textunderscore verde\textunderscore )}
\end{itemize}
Círculo de varinhas ou barbatanas, com que se dava roda ao vestido das mulheres. Cf. Pant. de Aveiro. \textunderscore Itiner.\textunderscore , 8, (2.^a ed.); \textunderscore Fenix Renasc.\textunderscore , IV. 20.
\section{Verdugal}
\begin{itemize}
\item {Grp. gram.:m.}
\end{itemize}
\begin{itemize}
\item {Utilização:Prov.}
\end{itemize}
\begin{itemize}
\item {Utilização:alent.}
\end{itemize}
\begin{itemize}
\item {Proveniência:(Do rad. de \textunderscore verde\textunderscore )}
\end{itemize}
Mato delgado, basto e muito verde.
\section{Verdugo}
\begin{itemize}
\item {Grp. gram.:m.}
\end{itemize}
\begin{itemize}
\item {Utilização:Des.}
\end{itemize}
\begin{itemize}
\item {Utilização:Náut.}
\end{itemize}
\begin{itemize}
\item {Utilização:T. do Fundão}
\end{itemize}
\begin{itemize}
\item {Utilização:Bras. do S}
\end{itemize}
\begin{itemize}
\item {Utilização:Prov.}
\end{itemize}
\begin{itemize}
\item {Utilização:minh.}
\end{itemize}
Aquelle que dá maus tratos.
Carrasco.
Espada sem gume, delgada e comprida.
Pequena navalha.
Parte saliente da chapa de trilho, nas rodas dos vagões, do lado interior da via, para impedir descarrilamentos.
Friso saliente, ao longo da borda do navio.
Qualquer objecto de grandes dimensões.
Qualquer cobra muito grande.
Depressão digital em fabrico de olaria.
(Cast. \textunderscore verdugo\textunderscore )
\section{Verduguilho}
\begin{itemize}
\item {Grp. gram.:m.}
\end{itemize}
\begin{itemize}
\item {Utilização:Prov.}
\end{itemize}
\begin{itemize}
\item {Utilização:trasm.}
\end{itemize}
Seitoira de fôlha estreita e bom córte.
(Cast. \textunderscore verduguillo\textunderscore )
\section{Verdum}
\begin{itemize}
\item {Grp. gram.:m.}
\end{itemize}
\begin{itemize}
\item {Utilização:Prov.}
\end{itemize}
\begin{itemize}
\item {Utilização:alent.}
\end{itemize}
O mesmo que \textunderscore verdete\textunderscore .
\section{Verdura}
\begin{itemize}
\item {Grp. gram.:f.}
\end{itemize}
\begin{itemize}
\item {Utilização:Fig.}
\end{itemize}
\begin{itemize}
\item {Grp. gram.:Pl.}
\end{itemize}
\begin{itemize}
\item {Proveniência:(De \textunderscore verde\textunderscore )}
\end{itemize}
O mesmo que \textunderscore verdor\textunderscore .
O verde das plantas.
Os vegetaes; hortaliça.
Mocidade.
Inexperiência.
Actos próprios da mocidade.
\section{Verdurengo}
\begin{itemize}
\item {Grp. gram.:adj.}
\end{itemize}
\begin{itemize}
\item {Utilização:Prov.}
\end{itemize}
\begin{itemize}
\item {Utilização:minh.}
\end{itemize}
\begin{itemize}
\item {Proveniência:(De \textunderscore verdura\textunderscore )}
\end{itemize}
O mesmo que \textunderscore verdoengo\textunderscore .
\section{Veréa}
\begin{itemize}
\item {Grp. gram.:f.}
\end{itemize}
(Outra fórma de \textunderscore vereda\textunderscore )
\section{Vereação}
\begin{itemize}
\item {Grp. gram.:f.}
\end{itemize}
Acto ou effeito de verear.
Cargo de vereador; os vereadores.
Tempo, que dura o cargo dos vereadores.
\section{Vereador}
\begin{itemize}
\item {Grp. gram.:m.}
\end{itemize}
Aquelle que vereia; membro da Câmara Municipal; camarista.
\section{Vereamento}
\begin{itemize}
\item {Grp. gram.:m.}
\end{itemize}
\begin{itemize}
\item {Proveniência:(De \textunderscore verear\textunderscore )}
\end{itemize}
O mesmo que \textunderscore vereação\textunderscore .
Jurisdicção de vereadores.
\section{Verear}
\begin{itemize}
\item {Grp. gram.:v. t.}
\end{itemize}
\begin{itemize}
\item {Grp. gram.:V. i.}
\end{itemize}
\begin{itemize}
\item {Proveniência:(De \textunderscore vereia\textunderscore )}
\end{itemize}
Administrar como vereador.
Exercer funcções de vereador.
\section{Verecúndia}
\begin{itemize}
\item {Grp. gram.:f.}
\end{itemize}
\begin{itemize}
\item {Proveniência:(Lat. \textunderscore verecundia\textunderscore )}
\end{itemize}
(Fórma erudita ou artificial de \textunderscore vergonha\textunderscore )
\section{Verecundo}
\begin{itemize}
\item {Grp. gram.:adj.}
\end{itemize}
\begin{itemize}
\item {Proveniência:(Lat. \textunderscore verecundus\textunderscore )}
\end{itemize}
Vergonhoso.
\section{Vereda}
\begin{itemize}
\item {fónica:verê}
\end{itemize}
\begin{itemize}
\item {Grp. gram.:f.}
\end{itemize}
Caminho estreito; senda.
Rumo; direcção.
(B. lat. \textunderscore vereda\textunderscore )
\section{Verede}
\begin{itemize}
\item {Grp. gram.:m.}
\end{itemize}
\begin{itemize}
\item {Utilização:Ant.}
\end{itemize}
O mesmo que \textunderscore pomar\textunderscore .
(Por \textunderscore varede\textunderscore , de \textunderscore vara\textunderscore ? Cp. \textunderscore varedo\textunderscore )
\section{Veredicto}
\begin{itemize}
\item {Grp. gram.:m.}
\end{itemize}
\begin{itemize}
\item {Proveniência:(Lat. \textunderscore veredictum\textunderscore )}
\end{itemize}
Resolução de um júry, á cêrca de uma causa cível ou criminal.
Opinião autorizada.
Ratificação.
\section{Vereia}
\begin{itemize}
\item {Grp. gram.:f.}
\end{itemize}
(Outra fórma de \textunderscore vereda\textunderscore )
\section{Verendo}
\begin{itemize}
\item {Grp. gram.:adj.}
\end{itemize}
\begin{itemize}
\item {Utilização:Des.}
\end{itemize}
\begin{itemize}
\item {Proveniência:(Lat. \textunderscore verendus\textunderscore )}
\end{itemize}
Venerável.
\section{Vereno}
\begin{itemize}
\item {Grp. gram.:adj.}
\end{itemize}
\begin{itemize}
\item {Utilização:T. de Turquel}
\end{itemize}
Defeituosamente brando; que não attingiu a natural consistência: \textunderscore esta cana não presta; é verena\textunderscore .
\section{Vêrga}
\begin{itemize}
\item {Grp. gram.:f.}
\end{itemize}
\begin{itemize}
\item {Utilização:Gír.}
\end{itemize}
\begin{itemize}
\item {Utilização:Chul.}
\end{itemize}
\begin{itemize}
\item {Proveniência:(Do lat. \textunderscore virga\textunderscore )}
\end{itemize}
Vara flexível e delgada.
Cada uma das fasquias flexíveis, com que se fabricam cestos.
Barra delgada de metal.
Pau, que se atravessa no mastro e a que se prende a vela do navio.
Peça de madeira ou de pedra, que se colloca transversalmente sôbre ombreiras de porta ou janela.
Padieira.
Torça.
Parte antero-superior da entrada de uma chaminé.
O mesmo que \textunderscore fato\textunderscore : \textunderscore êste gajo trás hoje vêrga nova\textunderscore .
O mesmo que \textunderscore pênis\textunderscore .
\section{Vêrga-áurea}
\begin{itemize}
\item {Grp. gram.:f.}
\end{itemize}
Planta violácea, vulgar na serra de Sintra.
\section{Vergada}
\begin{itemize}
\item {Grp. gram.:f.}
\end{itemize}
\begin{itemize}
\item {Utilização:T. do Fundão}
\end{itemize}
\begin{itemize}
\item {Proveniência:(De \textunderscore vêrga\textunderscore )}
\end{itemize}
Nó, muito rijo, na madeira.
\section{Vergal}
\begin{itemize}
\item {Grp. gram.:m.}
\end{itemize}
\begin{itemize}
\item {Proveniência:(De \textunderscore vêrga\textunderscore )}
\end{itemize}
Correia, que prende as bêstas ao carro.
\section{Vergalhada}
\begin{itemize}
\item {Grp. gram.:f.}
\end{itemize}
Pancada com vergalho.
Chibatada.
Patifaria, maroteira.
\section{Vergalhão}
\begin{itemize}
\item {Grp. gram.:m.}
\end{itemize}
\begin{itemize}
\item {Proveniência:(De \textunderscore vergalho\textunderscore )}
\end{itemize}
Vergalho grande.
Barra de ferro, quadrada.
\section{Vergalhar}
\begin{itemize}
\item {Grp. gram.:v. t.}
\end{itemize}
\begin{itemize}
\item {Proveniência:(De \textunderscore vergalho\textunderscore )}
\end{itemize}
O mesmo que \textunderscore azorragar\textunderscore . Cf. Camillo, \textunderscore Myst. de Lisb.\textunderscore , I, 55.
\section{Vergalho}
\begin{itemize}
\item {Grp. gram.:m.}
\end{itemize}
\begin{itemize}
\item {Utilização:Pop.}
\end{itemize}
\begin{itemize}
\item {Utilização:Prov.}
\end{itemize}
\begin{itemize}
\item {Utilização:chul.}
\end{itemize}
Membro genital dos bois ou dos cavallos, depois de cortado e sêco.
Azorrague, feito dêsse órgão.
Azorrague.
Velhaco; patife.
Homem de rija têmpera, homem teso.
(Cast. \textunderscore vergajo\textunderscore )
\section{Vergame}
\begin{itemize}
\item {Grp. gram.:m.}
\end{itemize}
\begin{itemize}
\item {Utilização:Prov.}
\end{itemize}
\begin{itemize}
\item {Utilização:trasm.}
\end{itemize}
Conjunto das vêrgas de uma embarcação.
Varas de castanheiro para cestos.
\section{Vergamota}
\begin{itemize}
\item {Grp. gram.:f.}
\end{itemize}
(V.bergamota)
\section{Vergão}
\begin{itemize}
\item {Grp. gram.:m.}
\end{itemize}
Grande vêrga.
Vinco na pelle, produzido por pancada ou por outra causa.
\section{Vergar}
\begin{itemize}
\item {Grp. gram.:v. t.}
\end{itemize}
\begin{itemize}
\item {Utilização:Fig.}
\end{itemize}
\begin{itemize}
\item {Grp. gram.:V. i.}
\end{itemize}
\begin{itemize}
\item {Utilização:Fig.}
\end{itemize}
\begin{itemize}
\item {Proveniência:(De \textunderscore vêrga\textunderscore )}
\end{itemize}
Curvar, como se curva uma vêrga.
Humilhar, abater.
Dominar.
Commover.
Curvar-se como uma vêrga.
Torcer-se.
Ceder ao pêso de alguma coisa.
Humilhar-se.
Ceder á influência de outrem.
\section{Vergasta}
\begin{itemize}
\item {Grp. gram.:f.}
\end{itemize}
\begin{itemize}
\item {Utilização:Fig.}
\end{itemize}
Pequena vêrga; chibata; verdasca.
Açoite, flagello.
\section{Vergastada}
\begin{itemize}
\item {Grp. gram.:f.}
\end{itemize}
Pancada com vergasta.
\section{Vergastador}
\begin{itemize}
\item {Grp. gram.:adj.}
\end{itemize}
Que vergasta.
\section{Vergastar}
\begin{itemize}
\item {Grp. gram.:v. t.}
\end{itemize}
\begin{itemize}
\item {Utilização:Ext.}
\end{itemize}
Bater com vergasta em.
Açoitar; fustigar.
\section{Vergasteiro}
\begin{itemize}
\item {Grp. gram.:m.}
\end{itemize}
\begin{itemize}
\item {Utilização:Prov.}
\end{itemize}
\begin{itemize}
\item {Proveniência:(De \textunderscore vergasta\textunderscore )}
\end{itemize}
Ramo de giesta ou de outro arbusto, para vergastar.
\section{Vergel}
\begin{itemize}
\item {Grp. gram.:m.}
\end{itemize}
\begin{itemize}
\item {Proveniência:(Do lat. \textunderscore viridiarium\textunderscore )}
\end{itemize}
Jardim; pomar.
\section{Vergília}
\begin{itemize}
\item {Grp. gram.:f.}
\end{itemize}
\begin{itemize}
\item {Proveniência:(De \textunderscore Vergílio\textunderscore , n. p.)}
\end{itemize}
Gênero de plantas leguminosas.
\section{Vergiliano}
\begin{itemize}
\item {Grp. gram.:adj.}
\end{itemize}
\begin{itemize}
\item {Proveniência:(Lat. \textunderscore vergilianus\textunderscore )}
\end{itemize}
Relativo ao poéta Vergílio.
Semelhante ao estilo de Vergílio ou ao gênero das suas composições.
\section{Vergilista}
\begin{itemize}
\item {Grp. gram.:m.}
\end{itemize}
\begin{itemize}
\item {Utilização:Des.}
\end{itemize}
Homem versado nas obras de Vergílio.
\section{Vergonça}
\begin{itemize}
\item {Grp. gram.:f.}
\end{itemize}
\begin{itemize}
\item {Utilização:Ant.}
\end{itemize}
\begin{itemize}
\item {Proveniência:(Do cast. \textunderscore vergonza\textunderscore )}
\end{itemize}
O mesmo que \textunderscore vergonha\textunderscore .
\section{Vergonçante}
\begin{itemize}
\item {Grp. gram.:adj.}
\end{itemize}
\begin{itemize}
\item {Utilização:Ant.}
\end{itemize}
Que tem vergonça; envergonhado:«\textunderscore ...pobre vergonçante...\textunderscore »\textunderscore Anat. Joc.\textunderscore  I, 337.
\section{Vergonçoso}
\begin{itemize}
\item {Grp. gram.:adj.}
\end{itemize}
\begin{itemize}
\item {Utilização:Ant.}
\end{itemize}
O mesmo que \textunderscore vergonhoso\textunderscore .
\section{Vergonha}
\begin{itemize}
\item {Grp. gram.:f.}
\end{itemize}
\begin{itemize}
\item {Grp. gram.:Pl.}
\end{itemize}
\begin{itemize}
\item {Proveniência:(Do lat. \textunderscore verecundia\textunderscore )}
\end{itemize}
Pudor; pejo.
Receio da deshonra.
Desgôsto, produzido pela ideia da deshonra.
Deshonra.
Rubor que o pejo produz nas faces.
Timidez, acanhamento.
Órgãos sexuaes do corpo humano:«\textunderscore o despirão nu e lhe puserão em suas vergonhas huas nespras.\textunderscore »\textunderscore Alvará\textunderscore  de D. Sebast., in \textunderscore Rev. Lus.\textunderscore , XV, 140.
\section{Vergonhaça}
\begin{itemize}
\item {Grp. gram.:f.}
\end{itemize}
Grande vergonha ou vexame.
\section{Vergonhaço}
\begin{itemize}
\item {Grp. gram.:m.}
\end{itemize}
\begin{itemize}
\item {Utilização:Prov.}
\end{itemize}
\begin{itemize}
\item {Utilização:trasm.}
\end{itemize}
O mesmo que \textunderscore vergonhaça\textunderscore .
\section{Vergonhar-se}
\begin{itemize}
\item {Grp. gram.:v. p.}
\end{itemize}
O mesmo que [[envergonhar-se|envergonhar]]. Cf. Usque, 38 v.^o; Alex. Lobo, III, 51.
\section{Vergonhosa}
\begin{itemize}
\item {Grp. gram.:f.}
\end{itemize}
O mesmo que \textunderscore sensitiva\textunderscore .
\section{Vergonhosamente}
\begin{itemize}
\item {Grp. gram.:adv.}
\end{itemize}
De modo vergonhoso.
Com deshonra.
Com infâmia.
Cobardemente.
\section{Vergonhoso}
\begin{itemize}
\item {Grp. gram.:adj.}
\end{itemize}
Que tem vergonha; tímido.
Que produz deshonra; deshonroso.
Indigno, infame.
Obsceno: \textunderscore expressões vergonhosas\textunderscore .
\section{Vergôntea}
\begin{itemize}
\item {Grp. gram.:f.}
\end{itemize}
\begin{itemize}
\item {Utilização:Fig.}
\end{itemize}
Ramo de árvore, rebento.
Pimpolho.
Haste.
Prole.
(Cp. \textunderscore virgulta\textunderscore )
\section{Vergonteado}
\begin{itemize}
\item {Grp. gram.:adj.}
\end{itemize}
Semelhante a uma vergôntea.
\section{Vergontear}
\begin{itemize}
\item {Grp. gram.:v. i.}
\end{itemize}
Lançar vergôntea.
\section{Vergueiro}
\begin{itemize}
\item {Grp. gram.:m.}
\end{itemize}
\begin{itemize}
\item {Utilização:Náut.}
\end{itemize}
\begin{itemize}
\item {Grp. gram.:Adj.}
\end{itemize}
\begin{itemize}
\item {Utilização:Prov.}
\end{itemize}
\begin{itemize}
\item {Utilização:trasm.}
\end{itemize}
\begin{itemize}
\item {Proveniência:(Do b. lat. \textunderscore vergarius\textunderscore )}
\end{itemize}
O mesmo que \textunderscore vergasta\textunderscore .
Cabo de pau, em certos utensílios de ferreiro.
Cadeia de ferro, que prende o leme.
Cabo grosso, que passa pelos olhaes das falcas.
Diz-se do homem ou do animal pouco próprio para o trabalho e que vérga com qualquer pêso.
\section{Vergueta}
\begin{itemize}
\item {fónica:guê}
\end{itemize}
\begin{itemize}
\item {Grp. gram.:f.}
\end{itemize}
\begin{itemize}
\item {Utilização:Heráld.}
\end{itemize}
\begin{itemize}
\item {Proveniência:(De \textunderscore vêrga\textunderscore )}
\end{itemize}
Pala estreita nos escudos.
\section{Verguio}
\begin{itemize}
\item {Grp. gram.:adj.}
\end{itemize}
\begin{itemize}
\item {Utilização:Prov.}
\end{itemize}
Que se vérga; flexível: \textunderscore pau verguio\textunderscore .
\section{Veridicamente}
\begin{itemize}
\item {Grp. gram.:adv.}
\end{itemize}
De modo verídico.
Falando verdade.
\section{Veridicidade}
\begin{itemize}
\item {Grp. gram.:f.}
\end{itemize}
Qualidade do que é verídico.
O mesmo que \textunderscore veracidade\textunderscore .
\section{Verídico}
\begin{itemize}
\item {Grp. gram.:adj.}
\end{itemize}
\begin{itemize}
\item {Proveniência:(Lat. \textunderscore veridicus\textunderscore )}
\end{itemize}
Que diz a verdade.
Em que há verdade; exacto.
\section{Verificação}
\begin{itemize}
\item {Grp. gram.:f.}
\end{itemize}
Acto ou effeito de verificar.
Cumprimento, realização.
\section{Verificador}
\begin{itemize}
\item {Grp. gram.:adj.}
\end{itemize}
\begin{itemize}
\item {Grp. gram.:M.}
\end{itemize}
Que verifica.
Empregado aduaneiro, que verifica a applicação dos respectivos impostos ás fazendas ou mercadorias apresentadas a despacho.
\section{Verificar}
\begin{itemize}
\item {Grp. gram.:v. t.}
\end{itemize}
\begin{itemize}
\item {Grp. gram.:V. p.}
\end{itemize}
\begin{itemize}
\item {Proveniência:(Lat. \textunderscore verificare\textunderscore )}
\end{itemize}
Provar a verdade de.
Investigar a verdade de.
Achar que é exacto.
Averiguar.
Effectuar-se; realizar-se.
\section{Verificativo}
\begin{itemize}
\item {Grp. gram.:adj.}
\end{itemize}
Próprio para verificar.
\section{Verificável}
\begin{itemize}
\item {Grp. gram.:adj.}
\end{itemize}
Que se póde verificar.
\section{Verilha}
\begin{itemize}
\item {Grp. gram.:f.}
\end{itemize}
O mesmo ou melhór que \textunderscore virilha\textunderscore .
(Cp. cast. \textunderscore verija\textunderscore )
\section{Verisímil}
\begin{itemize}
\item {fónica:si}
\end{itemize}
\begin{itemize}
\item {Proveniência:(Lat. \textunderscore verisimilis\textunderscore )}
\end{itemize}
\textunderscore adj.\textunderscore  (e der.)
O mesmo que \textunderscore verosimil\textunderscore , etc.
\section{Verisímile}
\begin{itemize}
\item {fónica:si}
\end{itemize}
\begin{itemize}
\item {Grp. gram.:adj.}
\end{itemize}
O mesmo que \textunderscore verisímil\textunderscore . Cf. \textunderscore Eufrosina\textunderscore , 98.
\section{Verisimilmente}
\begin{itemize}
\item {fónica:si}
\end{itemize}
\begin{itemize}
\item {Grp. gram.:adv.}
\end{itemize}
O mesmo que \textunderscore verosimilmente\textunderscore .
\section{Verissímil}
\begin{itemize}
\item {Proveniência:(Lat. \textunderscore verisimilis\textunderscore )}
\end{itemize}
\textunderscore adj.\textunderscore  (e der.)
O mesmo que \textunderscore verossimil\textunderscore , etc.
\section{Verissímile}
\begin{itemize}
\item {Grp. gram.:adj.}
\end{itemize}
O mesmo que \textunderscore verissímil\textunderscore . Cf. \textunderscore Eufrosina\textunderscore , 98.
\section{Verissimilmente}
\begin{itemize}
\item {Grp. gram.:adv.}
\end{itemize}
O mesmo que \textunderscore verossimilmente\textunderscore .
\section{Veríssimo}
\begin{itemize}
\item {Grp. gram.:adj.}
\end{itemize}
\begin{itemize}
\item {Proveniência:(De \textunderscore vero\textunderscore )}
\end{itemize}
Muito verdadeiro; exactíssimo.
\section{Verjel}
\begin{itemize}
\item {Grp. gram.:m.}
\end{itemize}
O mesmo ou melhór que \textunderscore vergel\textunderscore .
\section{Verlete}
\begin{itemize}
\item {fónica:lê}
\end{itemize}
\begin{itemize}
\item {Grp. gram.:m.}
\end{itemize}
\begin{itemize}
\item {Utilização:Ant.}
\end{itemize}
Espécie de bandeja de ferro, em que se serviam iguarias nos mosteiros.
(Provavelmente por \textunderscore varlete\textunderscore , do fr. \textunderscore varlet\textunderscore , pagem)
\section{Verme}
\begin{itemize}
\item {Grp. gram.:m.}
\end{itemize}
\begin{itemize}
\item {Utilização:Fig.}
\end{itemize}
\begin{itemize}
\item {Proveniência:(Lat. \textunderscore vermis\textunderscore )}
\end{itemize}
Minhoca ou lombriga terrestre.
Qualquer animal, semelhante á minhoca.
Gusano.
Helmintho.
Animálculo intestinal.
Cada um dos suppostos animálculos, que corroem os cadáveres nas sepulturas.
Larva.
Aquillo que mina ou corrói lentamente.
\section{Vermejoilo}
\begin{itemize}
\item {Grp. gram.:adj.}
\end{itemize}
O mesmo que \textunderscore vermelejoilo\textunderscore .
\section{Vermelejoilo}
\begin{itemize}
\item {Grp. gram.:adj.}
\end{itemize}
\begin{itemize}
\item {Utilização:Agr.}
\end{itemize}
Diz-se de uma variedade de trigo rijo.
\section{Vermelhaço}
\begin{itemize}
\item {Grp. gram.:adj.}
\end{itemize}
O mesmo que \textunderscore avermelhado\textunderscore .
\section{Vermelhal}
\begin{itemize}
\item {Grp. gram.:adj. f.}
\end{itemize}
Diz-se de uma casta de azeitona, também chamada \textunderscore cordovesa\textunderscore  ou \textunderscore cordovil\textunderscore .
\section{Vermelhão}
\begin{itemize}
\item {Grp. gram.:m.}
\end{itemize}
\begin{itemize}
\item {Proveniência:(De \textunderscore vermelho\textunderscore )}
\end{itemize}
Substância tinctória, o mesmo que \textunderscore mínio\textunderscore .
Qualquer ingrediente, com que se torna còrado o rosto.
Rubor da cara, vermelhidão.
\section{Vermelhar}
\begin{itemize}
\item {Grp. gram.:v. t.}
\end{itemize}
\begin{itemize}
\item {Grp. gram.:V. i.}
\end{itemize}
\begin{itemize}
\item {Proveniência:(De \textunderscore vermelho\textunderscore )}
\end{itemize}
O mesmo que \textunderscore avermelhar\textunderscore .
Têr côr vermelha.
Apresentar côr vermelha.
\section{Vermelhear}
\begin{itemize}
\item {Grp. gram.:v. i.}
\end{itemize}
(V.vermelhar)
\section{Vermelhejar}
\begin{itemize}
\item {Grp. gram.:v. i.}
\end{itemize}
(V.vermelhar)
\section{Vermelhidão}
\begin{itemize}
\item {Grp. gram.:f.}
\end{itemize}
Qualidade do que é vermelho.
Rubor.
\section{Vermelhinha}
\begin{itemize}
\item {Grp. gram.:f.}
\end{itemize}
Jôgo de cartas.
\section{Vermelhinha-de-galho}
\begin{itemize}
\item {Grp. gram.:f.}
\end{itemize}
\begin{itemize}
\item {Utilização:Bras}
\end{itemize}
Espécie de mandioca.
\section{Vermelho}
\begin{itemize}
\item {fónica:mê}
\end{itemize}
\begin{itemize}
\item {Grp. gram.:adj.}
\end{itemize}
\begin{itemize}
\item {Utilização:pop.}
\end{itemize}
\begin{itemize}
\item {Utilização:Fig.}
\end{itemize}
\begin{itemize}
\item {Grp. gram.:M.}
\end{itemize}
\begin{itemize}
\item {Utilização:Bras}
\end{itemize}
\begin{itemize}
\item {Proveniência:(Do lat. \textunderscore vermiculus\textunderscore )}
\end{itemize}
Muito encarnado; rubro.
Revolucionário.
A côr vermelha.
Verniz, feito de resina, sangue de drago e álcool.
Peixe marítimo.
Árvore da ilha de San-Thomé.
Variedade de trigo rijo.
\section{Vermelhusco}
\begin{itemize}
\item {Grp. gram.:adj.}
\end{itemize}
Um tanto vermelho.
\section{Vermem}
\begin{itemize}
\item {Grp. gram.:m.}
\end{itemize}
\begin{itemize}
\item {Utilização:Ant.}
\end{itemize}
O mesmo que \textunderscore verme\textunderscore .
(Infl. de \textunderscore vérmina\textunderscore )
\section{Vermicida}
\begin{itemize}
\item {Grp. gram.:m.  e  adj.}
\end{itemize}
\begin{itemize}
\item {Proveniência:(Do lat. \textunderscore vermis\textunderscore  + \textunderscore caedere\textunderscore )}
\end{itemize}
Aquillo que serve para matar ou destruir os vermes; anthelmínthico.
\section{Vermiculado}
\begin{itemize}
\item {Grp. gram.:adj.}
\end{itemize}
\begin{itemize}
\item {Utilização:Bot.}
\end{itemize}
\begin{itemize}
\item {Proveniência:(Lat. \textunderscore vermiculatus\textunderscore )}
\end{itemize}
Que tem ornatos em fórma de vermes.
Diz-se dos órgãos vegetaes que apresentam saliências em fórma de vermes.
\section{Vermicular}
\begin{itemize}
\item {Grp. gram.:adj.}
\end{itemize}
\begin{itemize}
\item {Proveniência:(De \textunderscore vermículo\textunderscore )}
\end{itemize}
Relativo ou semelhante a vermes.
\section{Vermiculária}
\begin{itemize}
\item {Grp. gram.:f.}
\end{itemize}
Planta crassulácea, o mesmo que \textunderscore sanguinária\textunderscore , (\textunderscore sedum acre\textunderscore , Lin.).
\section{Vermículo}
\begin{itemize}
\item {Grp. gram.:m.}
\end{itemize}
\begin{itemize}
\item {Proveniência:(Lat. \textunderscore vermiculus\textunderscore )}
\end{itemize}
Pequeno verme.
\section{Vermiculoso}
\begin{itemize}
\item {Grp. gram.:adj.}
\end{itemize}
\begin{itemize}
\item {Proveniência:(Lat. \textunderscore vermiculosus\textunderscore )}
\end{itemize}
O mesmo que \textunderscore vermiculado\textunderscore .
\section{Vermiculura}
\begin{itemize}
\item {Grp. gram.:f.}
\end{itemize}
\begin{itemize}
\item {Proveniência:(De \textunderscore vermículo\textunderscore )}
\end{itemize}
Ornato architectónico, que imita o sulco que os vermes deixam quando se arrastam.
\section{Vermiforme}
\begin{itemize}
\item {Grp. gram.:adj.}
\end{itemize}
\begin{itemize}
\item {Proveniência:(Do lat. \textunderscore vermis\textunderscore  + \textunderscore forma\textunderscore )}
\end{itemize}
Semelhante ao verme.
\section{Vermífugo}
\begin{itemize}
\item {Grp. gram.:m.  e  adj.}
\end{itemize}
\begin{itemize}
\item {Proveniência:(Do lat. \textunderscore vermis\textunderscore  + \textunderscore fugere\textunderscore )}
\end{itemize}
Que afugenta ou destrói os vermes.
O mesmo que \textunderscore vermicida\textunderscore .
\section{Vérmina}
\begin{itemize}
\item {Grp. gram.:f.}
\end{itemize}
\begin{itemize}
\item {Utilização:P. us.}
\end{itemize}
\begin{itemize}
\item {Proveniência:(Lat. \textunderscore vermina\textunderscore )}
\end{itemize}
O mesmo que \textunderscore verminose\textunderscore .
\section{Verminação}
\begin{itemize}
\item {Grp. gram.:f.}
\end{itemize}
\begin{itemize}
\item {Proveniência:(Do lat. \textunderscore verminatio\textunderscore )}
\end{itemize}
Producção de vermes nos intestinos.
\section{Verminado}
\begin{itemize}
\item {Grp. gram.:adj.}
\end{itemize}
\begin{itemize}
\item {Utilização:Fig.}
\end{itemize}
\begin{itemize}
\item {Proveniência:(Do lat. \textunderscore verminatus\textunderscore )}
\end{itemize}
Em que há vermes.
Corroído por vermes.
Amofinado, consumido.
\section{Vermineira}
\begin{itemize}
\item {Grp. gram.:f.}
\end{itemize}
\begin{itemize}
\item {Proveniência:(Do lat. \textunderscore verminare\textunderscore )}
\end{itemize}
Lugar onde, por meio da fermentação de matérias orgânicas, se produzem vermes, destinados á alimentação de gallinhas e de outras aves.
\section{Verminose}
\begin{itemize}
\item {Grp. gram.:f.}
\end{itemize}
\begin{itemize}
\item {Proveniência:(Do lat. \textunderscore vermina\textunderscore )}
\end{itemize}
Doença, produzida pela abundância de vermes nos intestinos.
\section{Verminoso}
\begin{itemize}
\item {Grp. gram.:adj.}
\end{itemize}
\begin{itemize}
\item {Proveniência:(Lat. \textunderscore verminosus\textunderscore )}
\end{itemize}
Verminado.
Produzido pelos vermes.
\section{Vermívoro}
\begin{itemize}
\item {Grp. gram.:adj.}
\end{itemize}
\begin{itemize}
\item {Proveniência:(Do lat. \textunderscore vermis\textunderscore  + \textunderscore vorare\textunderscore )}
\end{itemize}
Que come vermes.
\section{Vermizela}
\begin{itemize}
\item {Grp. gram.:f.}
\end{itemize}
Verme da terra, nocivo ás raízes de certas plantas. Cf. \textunderscore Bibl. da G. do Campo\textunderscore , 304.
\section{Vermizella}
\begin{itemize}
\item {Grp. gram.:f.}
\end{itemize}
Verme da terra, nocivo ás raízes de certas plantas. Cf. \textunderscore Bibl. da G. do Campo\textunderscore , 304.
\section{Vermutho}
\begin{itemize}
\item {Grp. gram.:m.}
\end{itemize}
\begin{itemize}
\item {Proveniência:(Do al. \textunderscore wermuth\textunderscore )}
\end{itemize}
Vinho branco, em que se infunde absintho, e que serve para excitar o appetíte.
\section{Vermuto}
\begin{itemize}
\item {Grp. gram.:m.}
\end{itemize}
\begin{itemize}
\item {Proveniência:(Do al. \textunderscore wermuth\textunderscore )}
\end{itemize}
Vinho branco, em que se infunde absintho, e que serve para excitar o appetíte.
\section{Vernação}
\begin{itemize}
\item {Grp. gram.:f.}
\end{itemize}
\begin{itemize}
\item {Utilização:Bot.}
\end{itemize}
\begin{itemize}
\item {Proveniência:(Do lat. \textunderscore vernatio\textunderscore )}
\end{itemize}
Modo como as fôlhas dos vegetaes estão dispostas nos gomos, dobrando-se ou enrolando-se.
Tempo, em que se formam as fôlhas dos vegetaes.
\section{Vernaculamente}
\begin{itemize}
\item {Grp. gram.:adv.}
\end{itemize}
De modo vernáculo.
Em linguagem castiça e pura.
\section{Vernaculidade}
\begin{itemize}
\item {Grp. gram.:f.}
\end{itemize}
Qualidade do que é vernáculo.
\section{Vernaculista}
\begin{itemize}
\item {Grp. gram.:adj.}
\end{itemize}
\begin{itemize}
\item {Proveniência:(De \textunderscore vernáculo\textunderscore )}
\end{itemize}
Que escreve ou fala vernaculamente.
\section{Vernáculo}
\begin{itemize}
\item {Grp. gram.:adj.}
\end{itemize}
\begin{itemize}
\item {Utilização:Fig.}
\end{itemize}
\begin{itemize}
\item {Proveniência:(Lat. \textunderscore vernaculus\textunderscore )}
\end{itemize}
Nacional.
Próprio da região em que está.
Genuíno, correcto e puro, sem mescla de estrangeirismos, (falando-se da linguagem).
Que mantém correcção e pureza no falar e no escrever, (falando-se de alguém).
\section{Vernal}
\begin{itemize}
\item {Grp. gram.:adj.}
\end{itemize}
\begin{itemize}
\item {Proveniência:(Lat. \textunderscore vernalis\textunderscore )}
\end{itemize}
Relativo á Primavera.
Diz-se dos vegetaes, que desabrocham na Primavera.
\section{Vernante}
\begin{itemize}
\item {Grp. gram.:adj.}
\end{itemize}
\begin{itemize}
\item {Proveniência:(Lat. \textunderscore vernans\textunderscore )}
\end{itemize}
Que rebenta ou floresce na Primavera.
\section{Vernes}
\begin{itemize}
\item {Grp. gram.:m. pl.}
\end{itemize}
\begin{itemize}
\item {Utilização:Veter.}
\end{itemize}
Inchação, entre a pelle dos animaes e o tecido subjacente.
\section{Verniz}
\begin{itemize}
\item {Grp. gram.:m.}
\end{itemize}
\begin{itemize}
\item {Utilização:Fig.}
\end{itemize}
\begin{itemize}
\item {Utilização:Pop.}
\end{itemize}
\begin{itemize}
\item {Proveniência:(Do b. lat. \textunderscore vernicium\textunderscore )}
\end{itemize}
Solução de resina ou goma resinosa em álcool, para polir a superfície de certos objectos, ou para os preservar da acção do ar e da humidade.
Cabedal de polimento: \textunderscore botas de verniz\textunderscore .
Delicadeza; elegância.
Embriaguez.
\section{Vernizar}
\begin{itemize}
\item {Grp. gram.:v. t.}
\end{itemize}
(V.envernizar). Cf. Filinto, \textunderscore D. Man.\textunderscore , I, 326.
\section{Vernizeira}
\begin{itemize}
\item {Grp. gram.:f.}
\end{itemize}
\begin{itemize}
\item {Utilização:Prov.}
\end{itemize}
\begin{itemize}
\item {Utilização:alg.}
\end{itemize}
\begin{itemize}
\item {Proveniência:(De \textunderscore verniz\textunderscore )}
\end{itemize}
O mesmo que \textunderscore sujidade\textunderscore .
\section{Verno}
\begin{itemize}
\item {Grp. gram.:adj.}
\end{itemize}
\begin{itemize}
\item {Proveniência:(Lat. \textunderscore vernus\textunderscore )}
\end{itemize}
O mesmo que \textunderscore vernal\textunderscore .
\section{Vernónia}
\begin{itemize}
\item {Grp. gram.:f.}
\end{itemize}
\begin{itemize}
\item {Proveniência:(De \textunderscore Vernon\textunderscore , n. p.)}
\end{itemize}
Gênero de plantas synanthércas.
\section{Vero}
\begin{itemize}
\item {Grp. gram.:adj.}
\end{itemize}
\begin{itemize}
\item {Proveniência:(Lat. \textunderscore verus\textunderscore )}
\end{itemize}
Verdadeiro; real.
\section{Veronal}
\begin{itemize}
\item {Grp. gram.:m.}
\end{itemize}
Medicamento hypnótico.
\section{Veronense}
\begin{itemize}
\item {Grp. gram.:adj.}
\end{itemize}
O mesmo que \textunderscore veronês\textunderscore .
\section{Veronês}
\begin{itemize}
\item {Grp. gram.:adj.}
\end{itemize}
\begin{itemize}
\item {Grp. gram.:M.}
\end{itemize}
Relativo a Verona.
Habitante de Verona.
\section{Verónica}
\begin{itemize}
\item {Grp. gram.:f.}
\end{itemize}
\begin{itemize}
\item {Utilização:Ext.}
\end{itemize}
\begin{itemize}
\item {Utilização:Pop.}
\end{itemize}
\begin{itemize}
\item {Utilização:Bot.}
\end{itemize}
\begin{itemize}
\item {Utilização:Taur.}
\end{itemize}
Imagem do rosto de Christo, gravada em metal.
A mesma imagem, pintada, ou estampada com traços vermelhos num pano branco.
Antiga moéda de oiro. Cf. Filinto, VIII, 278.
Rosto.
Mulhér que, nas procissões do entêrro de Christo, leva o santo sudário.
Gênero de plantas, em que se distingue a verónica officinal.
Uma das sortes de capinha.
(B. lat. \textunderscore veronica\textunderscore )
\section{Veronicáceas}
\begin{itemize}
\item {Grp. gram.:f. pl.}
\end{itemize}
Família de plantas, que tem por typo a verónica.
\section{Ver-o-pêso}
\begin{itemize}
\item {Grp. gram.:m.}
\end{itemize}
Designação antiga da casa fiscal, onde se examinam os gêneros de consumo, que se hão de expor á venda.
(Contr. de \textunderscore haver o pêso\textunderscore )
\section{Verorola}
\begin{itemize}
\item {Grp. gram.:f.}
\end{itemize}
(V.ucuuba)
\section{Verosemelhança}
\begin{itemize}
\item {fónica:se}
\end{itemize}
\begin{itemize}
\item {Grp. gram.:f.}
\end{itemize}
Qualidade de verosemelhante.
\section{Verosemelhante}
\begin{itemize}
\item {fónica:se}
\end{itemize}
\begin{itemize}
\item {Grp. gram.:adj.}
\end{itemize}
O mesmo que \textunderscore verosimil\textunderscore .
\section{Verosimil}
\begin{itemize}
\item {fónica:si}
\end{itemize}
\begin{itemize}
\item {Grp. gram.:adj.}
\end{itemize}
\begin{itemize}
\item {Proveniência:(Do lat. \textunderscore verus\textunderscore  + \textunderscore similis\textunderscore )}
\end{itemize}
Semelhante á verdade.
Que tem a apparência de verdade.
Que não repugna á verdade; provável.
\section{Verosimilidade}
\begin{itemize}
\item {fónica:si}
\end{itemize}
\begin{itemize}
\item {Grp. gram.:f.}
\end{itemize}
O mesmo que \textunderscore verosemelhança\textunderscore .
(Cp. lat. \textunderscore verisimilitas\textunderscore )
\section{Verosimilitude}
\begin{itemize}
\item {fónica:si}
\end{itemize}
\begin{itemize}
\item {Grp. gram.:f.}
\end{itemize}
O mesmo que \textunderscore verosemelhança\textunderscore .
(Cp. lat. \textunderscore verisimilitudo\textunderscore )
\section{Verosimilmente}
\begin{itemize}
\item {fónica:si}
\end{itemize}
\begin{itemize}
\item {Grp. gram.:adv.}
\end{itemize}
De modo verosimil.
\section{Verossemelhança}
\begin{itemize}
\item {Grp. gram.:f.}
\end{itemize}
Qualidade de verossemelhante.
\section{Verossemelhante}
\begin{itemize}
\item {Grp. gram.:adj.}
\end{itemize}
O mesmo que \textunderscore verossimil\textunderscore .
\section{Verossimil}
\begin{itemize}
\item {Grp. gram.:adj.}
\end{itemize}
\begin{itemize}
\item {Proveniência:(Do lat. \textunderscore verus\textunderscore  + \textunderscore similis\textunderscore )}
\end{itemize}
Semelhante á verdade.
Que tem a apparência de verdade.
Que não repugna á verdade; provável.
\section{Verossimilidade}
\begin{itemize}
\item {Grp. gram.:f.}
\end{itemize}
O mesmo que \textunderscore verossemelhança\textunderscore .
(Cp. lat. \textunderscore verisimilitas\textunderscore )
\section{Verossimilitude}
\begin{itemize}
\item {Grp. gram.:f.}
\end{itemize}
O mesmo que \textunderscore verossemelhança\textunderscore .
(Cp. lat. \textunderscore verisimilitudo\textunderscore )
\section{Verossimilmente}
\begin{itemize}
\item {Grp. gram.:adv.}
\end{itemize}
De modo verossimil.
\section{Verrina}
\begin{itemize}
\item {Grp. gram.:f.}
\end{itemize}
\begin{itemize}
\item {Utilização:Ext.}
\end{itemize}
\begin{itemize}
\item {Proveniência:(Lat. \textunderscore verrina\textunderscore )}
\end{itemize}
Cada um dos discursos, feitos por Cícero contra Verres.
Censura violenta, de ordinário escrita, ou feita em discurso público.
Crítica apaixonada e violenta.
\section{Verrinar}
\begin{itemize}
\item {Grp. gram.:v. i.}
\end{itemize}
Fazer verrina ou crítica apaixonada.
\section{Verrinário}
\begin{itemize}
\item {Grp. gram.:adj.}
\end{itemize}
Relativo a verrina.
\section{Verrineiro}
\begin{itemize}
\item {Grp. gram.:m.  e  adj.}
\end{itemize}
O que faz verrinas.
\section{Verrucal}
\begin{itemize}
\item {Grp. gram.:adj.}
\end{itemize}
\begin{itemize}
\item {Proveniência:(Do lat. \textunderscore verruca\textunderscore )}
\end{itemize}
Relativo á verruga.
\section{Verrucária}
\begin{itemize}
\item {Grp. gram.:f.}
\end{itemize}
\begin{itemize}
\item {Proveniência:(Lat. \textunderscore verrucaria\textunderscore )}
\end{itemize}
Gênero de líchens.
Girasol.
\section{Verrucifero}
\begin{itemize}
\item {Grp. gram.:adj.}
\end{itemize}
\begin{itemize}
\item {Proveniência:(Do lat. \textunderscore verruca\textunderscore  + \textunderscore fere\textunderscore )}
\end{itemize}
Que tem verrugas.
\section{Verruciforme}
\begin{itemize}
\item {Grp. gram.:adj.}
\end{itemize}
\begin{itemize}
\item {Proveniência:(Do lat. \textunderscore verruca\textunderscore  + \textunderscore forma\textunderscore )}
\end{itemize}
Que tem fórma de verruga.
\section{Verrucoso}
\begin{itemize}
\item {Grp. gram.:adj.}
\end{itemize}
O mesmo que \textunderscore verrucal\textunderscore .
\section{Verruculária}
\begin{itemize}
\item {Grp. gram.:f.}
\end{itemize}
\begin{itemize}
\item {Proveniência:(Do lat. \textunderscore verrucula\textunderscore )}
\end{itemize}
Gênero de plantas malpigiáceas.
\section{Verruga}
\begin{itemize}
\item {Grp. gram.:f.}
\end{itemize}
\begin{itemize}
\item {Proveniência:(Do lat. \textunderscore verruca\textunderscore )}
\end{itemize}
Pequena saliência consistente, na pelle; pequena protuberância rugosa.
\section{Verrugoso}
\begin{itemize}
\item {Grp. gram.:adj.}
\end{itemize}
\begin{itemize}
\item {Proveniência:(Do lat. \textunderscore verrucosus\textunderscore )}
\end{itemize}
Que tem verrugas.
\section{Verruguento}
\begin{itemize}
\item {Grp. gram.:adj.}
\end{itemize}
O mesmo que \textunderscore verrugoso\textunderscore .
\section{Verruma}
\begin{itemize}
\item {Grp. gram.:f.}
\end{itemize}
\begin{itemize}
\item {Proveniência:(Do ár. \textunderscore barrima\textunderscore ?)}
\end{itemize}
Instrumento, cuja extremidade inferior é lavrada em espiral e termina em ponta, o qual serve para fazer furos na madeira.
Broca.
\section{Verrumão}
\begin{itemize}
\item {Grp. gram.:m.}
\end{itemize}
\begin{itemize}
\item {Utilização:Açor}
\end{itemize}
\begin{itemize}
\item {Proveniência:(De \textunderscore verruma\textunderscore )}
\end{itemize}
Grande verruma.
Um dos insectos, que corroem a madeira.
Operário reles, sarrafaçal.
\section{Verrumar}
\begin{itemize}
\item {Grp. gram.:v. t.}
\end{itemize}
\begin{itemize}
\item {Grp. gram.:V. i.}
\end{itemize}
\begin{itemize}
\item {Utilização:Pop.}
\end{itemize}
Furar com verruma; furar.
Espicaçar.
Affligir, torturar:«\textunderscore as dores verrumam-me as pernas\textunderscore ». Camillo.
Fazer furos com verruma ou outro instrumento análogo.
Parafusar, meditar.
\section{Verrusga}
\begin{itemize}
\item {Grp. gram.:f.}
\end{itemize}
\begin{itemize}
\item {Utilização:Prov.}
\end{itemize}
\begin{itemize}
\item {Utilização:alg.}
\end{itemize}
O mesmo que \textunderscore ruga\textunderscore .
\section{Versa}
\begin{itemize}
\item {fónica:vê}
\end{itemize}
\begin{itemize}
\item {Grp. gram.:f.}
\end{itemize}
(V.berça)
\section{Versa}
\begin{itemize}
\item {fónica:vê}
\end{itemize}
\begin{itemize}
\item {Grp. gram.:f.}
\end{itemize}
\begin{itemize}
\item {Utilização:Gal}
\end{itemize}
\begin{itemize}
\item {Proveniência:(Fr. \textunderscore verse\textunderscore )}
\end{itemize}
Estado das searas, acamadas pela chuva ou por outra causa. Cf. Filinto, IX, 25.
\section{Versado}
\begin{itemize}
\item {Grp. gram.:adj.}
\end{itemize}
\begin{itemize}
\item {Proveniência:(De \textunderscore versar\textunderscore ^1)}
\end{itemize}
Perito, prático; experimentado.
\section{Versal}
\begin{itemize}
\item {Grp. gram.:f.  e  adj.}
\end{itemize}
\begin{itemize}
\item {Proveniência:(De \textunderscore verso\textunderscore ^1)}
\end{itemize}
Diz-se da letra maiúscula.
\section{Versalete}
\begin{itemize}
\item {fónica:lê}
\end{itemize}
\begin{itemize}
\item {Grp. gram.:f.}
\end{itemize}
\begin{itemize}
\item {Proveniência:(De \textunderscore versal\textunderscore )}
\end{itemize}
Letra versal de pequeno corpo.
\section{Versalhada}
\begin{itemize}
\item {Grp. gram.:f.}
\end{itemize}
\begin{itemize}
\item {Utilização:Deprec.}
\end{itemize}
Conjunto de versos.
Versos mal feitos ou insípidos.
\section{Versalhês}
\begin{itemize}
\item {Grp. gram.:m.}
\end{itemize}
Indivíduo, natural de Versalhes. Cf. Ortigão, \textunderscore Praias\textunderscore , 98.
\section{Versão}
\begin{itemize}
\item {Grp. gram.:f.}
\end{itemize}
\begin{itemize}
\item {Proveniência:(Do lat. \textunderscore versio\textunderscore )}
\end{itemize}
Acto ou effeito de verter ou de voltar.
Traducção literal de um texto.
Traducção.
Explicação.
Cada uma das differentes explicações do mesmo ponto.
Variante.
Boato.
Revolução de um astro.
Operação cirúrgica, para mudar a posição do féto no útero.
\section{Versar}
\begin{itemize}
\item {Grp. gram.:v. t.}
\end{itemize}
\begin{itemize}
\item {Grp. gram.:V. i.}
\end{itemize}
\begin{itemize}
\item {Proveniência:(Lat. \textunderscore versare\textunderscore )}
\end{itemize}
Volver; manejar: \textunderscore versar livros\textunderscore .
Praticar; estudar: \textunderscore versar Chímica\textunderscore .
Ponderar.
Deitar de um vaso para outro.
Dizer respeito.
Incidir; consistir: \textunderscore a difficuldade versa nisso\textunderscore .
Conviver.
\section{Versar}
\begin{itemize}
\item {Grp. gram.:v. i.}
\end{itemize}
\begin{itemize}
\item {Grp. gram.:V. t.}
\end{itemize}
O mesmo que \textunderscore versejar\textunderscore .
Pôr em verso.
\section{Versaria}
\begin{itemize}
\item {Grp. gram.:f.}
\end{itemize}
O mesmo que \textunderscore versalhada\textunderscore .
\section{Versátil}
\begin{itemize}
\item {Grp. gram.:adj.}
\end{itemize}
\begin{itemize}
\item {Proveniência:(Lat. \textunderscore versatilis\textunderscore )}
\end{itemize}
Volúvel, inconstante; vário.
\section{Versatilidade}
\begin{itemize}
\item {Grp. gram.:f.}
\end{itemize}
Qualidade ou estado do que é versátil.
\section{Verseira}
\begin{itemize}
\item {Grp. gram.:f.}
\end{itemize}
\begin{itemize}
\item {Utilização:Ant.}
\end{itemize}
(V.berceira)
\section{Verseiro}
\begin{itemize}
\item {Grp. gram.:adj.}
\end{itemize}
\begin{itemize}
\item {Utilização:Prov.}
\end{itemize}
\begin{itemize}
\item {Utilização:beir.}
\end{itemize}
\begin{itemize}
\item {Proveniência:(Do lat. \textunderscore versus\textunderscore )}
\end{itemize}
Diz-se que está \textunderscore verseiro ao sol\textunderscore  o alpende ou coberto, cuja frente ou abertura não recebe o sol.
Diz-se do terreno, em que não dá o sol.
\section{Versejador}
\begin{itemize}
\item {Grp. gram.:m.  e  adj.}
\end{itemize}
O que verseja.
\section{Versejadura}
\begin{itemize}
\item {Grp. gram.:f.}
\end{itemize}
Acto ou effeito de versejar.
\section{Versejal}
\begin{itemize}
\item {Grp. gram.:adj.}
\end{itemize}
\begin{itemize}
\item {Utilização:Burl.}
\end{itemize}
Relativo a verso:«\textunderscore abre cada era a versejal boceta.\textunderscore »Filinto, VIII, 12.
(Cp. \textunderscore versejar\textunderscore )
\section{Versejar}
\begin{itemize}
\item {Grp. gram.:v. i.}
\end{itemize}
\begin{itemize}
\item {Utilização:Deprec.}
\end{itemize}
\begin{itemize}
\item {Grp. gram.:V. t.}
\end{itemize}
Fazer versos.
Sêr poetastro, fazer maus versos.
Pôr em verso^1.
\section{Versejote}
\begin{itemize}
\item {Grp. gram.:m.}
\end{itemize}
Figo preto, redondo, encarnado por dentro.
(Corr. de \textunderscore berjaçote\textunderscore )
\section{Verseto}
\begin{itemize}
\item {fónica:sê}
\end{itemize}
\begin{itemize}
\item {Grp. gram.:m.}
\end{itemize}
\begin{itemize}
\item {Proveniência:(De \textunderscore verso\textunderscore ^1)}
\end{itemize}
Trecho bíblico de duas ou três linhas, formando sentido completo.
Palavras bíblicas seguidas ás vezes de um responso, que se reza ou se canta nos Offícios da Igreja.
Trecho musical, correspondente a um verseto.
Sinal typográphico, para marcar o princípio de cada verseto.
\section{Versicolor}
\begin{itemize}
\item {Grp. gram.:adj.}
\end{itemize}
\begin{itemize}
\item {Proveniência:(Lat. \textunderscore versicolor\textunderscore )}
\end{itemize}
Que é de várias côres; variegado; matizado; furta-côr.
\section{Versículo}
\begin{itemize}
\item {Grp. gram.:m.}
\end{itemize}
\begin{itemize}
\item {Proveniência:(Lat. \textunderscore versiculus\textunderscore )}
\end{itemize}
Divisão de artigo ou parágrapho.
O mesmo e mais usado que \textunderscore verseto\textunderscore .
\section{Versífero}
\begin{itemize}
\item {Grp. gram.:adj.}
\end{itemize}
\begin{itemize}
\item {Proveniência:(Do lat. \textunderscore versus\textunderscore  + \textunderscore ferre\textunderscore )}
\end{itemize}
Que tem versos; que faz versos.
\section{Versificação}
\begin{itemize}
\item {Grp. gram.:f.}
\end{itemize}
\begin{itemize}
\item {Proveniência:(Lat. \textunderscore versificatio\textunderscore )}
\end{itemize}
Acto ou effeito de versificar.
Arte de versificar; metrificação; modo de versificar.
\section{Versificador}
\begin{itemize}
\item {Grp. gram.:m.  e  adj.}
\end{itemize}
\begin{itemize}
\item {Proveniência:(Do lat. \textunderscore versificator\textunderscore )}
\end{itemize}
O que versifica.
\section{Versificar}
\begin{itemize}
\item {Grp. gram.:v. i.  e  t.}
\end{itemize}
\begin{itemize}
\item {Proveniência:(Lat. \textunderscore versificare\textunderscore )}
\end{itemize}
O mesmo que \textunderscore versejar\textunderscore .
\section{Versífico}
\begin{itemize}
\item {Grp. gram.:adj.}
\end{itemize}
\begin{itemize}
\item {Proveniência:(Lat. \textunderscore versificus\textunderscore )}
\end{itemize}
Relativo a versos ou á versificação.
\section{Versista}
\begin{itemize}
\item {Grp. gram.:m. ,  f.  e  adj.}
\end{itemize}
\begin{itemize}
\item {Proveniência:(De \textunderscore verso\textunderscore ^1)}
\end{itemize}
Pessôa, que verseja.
\section{Verso}
\begin{itemize}
\item {Grp. gram.:m.}
\end{itemize}
\begin{itemize}
\item {Proveniência:(Lat. \textunderscore versus\textunderscore )}
\end{itemize}
Conjunto de palavras, medidas e cadenciadas segundo certas regras fixas.
Gênero poético.
Poesia; versificação.
\section{Verso}
\begin{itemize}
\item {Grp. gram.:m.}
\end{itemize}
\begin{itemize}
\item {Utilização:Ext.}
\end{itemize}
\begin{itemize}
\item {Proveniência:(Lat. \textunderscore versus\textunderscore )}
\end{itemize}
Página opposta á da frente.
Face inferior das fôlhas dos vegetaes.
Lado posterior, face opposta á da frente.
\section{Versória}
\begin{itemize}
\item {Grp. gram.:f.}
\end{itemize}
\begin{itemize}
\item {Utilização:Náut.}
\end{itemize}
\begin{itemize}
\item {Utilização:Ant.}
\end{itemize}
\begin{itemize}
\item {Proveniência:(Lat. \textunderscore versoria\textunderscore )}
\end{itemize}
Corda ou cabo, para fazer voltar a vela.
\section{Versúcia}
\begin{itemize}
\item {Grp. gram.:f.}
\end{itemize}
\begin{itemize}
\item {Utilização:Des.}
\end{itemize}
\begin{itemize}
\item {Proveniência:(Lat. \textunderscore versutia\textunderscore )}
\end{itemize}
Manha; solércia.
\section{Versudo}
\begin{itemize}
\item {Grp. gram.:adj.}
\end{itemize}
(V.berçudo)
\section{Versuto}
\begin{itemize}
\item {Grp. gram.:adj.}
\end{itemize}
\begin{itemize}
\item {Proveniência:(Lat. \textunderscore versutus\textunderscore )}
\end{itemize}
Que tem versúcia.
\section{Vértebra}
\begin{itemize}
\item {Grp. gram.:f.}
\end{itemize}
\begin{itemize}
\item {Proveniência:(Lat. \textunderscore vertebra\textunderscore )}
\end{itemize}
Cada um dos vinte e quatro ossos, que constituem a espinha dorsal do homem.
Osso análogo, nos outros animaes.
\section{Vertebrado}
\begin{itemize}
\item {Grp. gram.:adj.}
\end{itemize}
\begin{itemize}
\item {Grp. gram.:M. Pl.}
\end{itemize}
\begin{itemize}
\item {Proveniência:(Lat. \textunderscore vertebratus\textunderscore )}
\end{itemize}
Que tem vértebras.
Grande divisão do reino animal, que comprehende todos os animaes, de cuja estructura faz parte um esqueleto ósseo ou cartilaginoso, composto de peças ligadas entre si, e móveis umas sôbre as outras.
\section{Vertebral}
\begin{itemize}
\item {Grp. gram.:adj.}
\end{itemize}
Relativo ás vértebras.
Composto de vértebras.
\section{Vértebro-ilíaco}
\begin{itemize}
\item {Grp. gram.:adj.}
\end{itemize}
\begin{itemize}
\item {Utilização:Anat.}
\end{itemize}
Relativo ás vértebras e ao osso ilíaco.
\section{Vertebroso}
\begin{itemize}
\item {Grp. gram.:adj.}
\end{itemize}
O mesmo que \textunderscore vertebral\textunderscore .
\section{Vertedoiro}
\begin{itemize}
\item {Grp. gram.:m.}
\end{itemize}
\begin{itemize}
\item {Proveniência:(De \textunderscore verter\textunderscore )}
\end{itemize}
Espécie de escudella, com que se despeja a água para fóra das embarcações.
\section{Vertedor}
\begin{itemize}
\item {Grp. gram.:adj.}
\end{itemize}
\begin{itemize}
\item {Grp. gram.:M.}
\end{itemize}
\begin{itemize}
\item {Proveniência:(De \textunderscore verter\textunderscore )}
\end{itemize}
Que verte.
Vaso para deitar ou despejar água.
\section{Vertedouro}
\begin{itemize}
\item {Grp. gram.:m.}
\end{itemize}
\begin{itemize}
\item {Proveniência:(De \textunderscore verter\textunderscore )}
\end{itemize}
Espécie de escudella, com que se despeja a água para fóra das embarcações.
\section{Vertedura}
\begin{itemize}
\item {Grp. gram.:f.}
\end{itemize}
Acto ou effeito de verter.
Porção de líquido, que trasborda do vaso em que se deita.
\section{Vertente}
\begin{itemize}
\item {Grp. gram.:adj.}
\end{itemize}
\begin{itemize}
\item {Grp. gram.:F.}
\end{itemize}
\begin{itemize}
\item {Grp. gram.:Pl.}
\end{itemize}
\begin{itemize}
\item {Utilização:Prov.}
\end{itemize}
\begin{itemize}
\item {Proveniência:(Lat. \textunderscore vertens\textunderscore )}
\end{itemize}
Que verte.
Declive de montanha, por onde derivam as águas pluviaes.
Cada uma das superfícies de um telhado.
Verteduras.
\section{Verter}
\begin{itemize}
\item {Grp. gram.:v. t.}
\end{itemize}
\begin{itemize}
\item {Utilização:Fig.}
\end{itemize}
\begin{itemize}
\item {Grp. gram.:V. i.}
\end{itemize}
\begin{itemize}
\item {Proveniência:(Lat. \textunderscore vertere\textunderscore )}
\end{itemize}
Fazer trasbordar.
Entornar: \textunderscore verter um copo de água\textunderscore .
Jorrar.
Diffundir; espalhar.
Traduzir litteralmente.
Traduzir.
Brotar; manar.
Desaguar.
Trasbordar; resumar.
\section{Vertical}
\begin{itemize}
\item {Grp. gram.:adj.}
\end{itemize}
\begin{itemize}
\item {Grp. gram.:F.}
\end{itemize}
\begin{itemize}
\item {Proveniência:(Lat. \textunderscore verticalis\textunderscore )}
\end{itemize}
Perpendicular ao plano do horizonte ou á superfície das águas tranquillas.
Situado por cima da cabeça; aprumado.
Linha vertical.
\section{Verticalidade}
\begin{itemize}
\item {Grp. gram.:f.}
\end{itemize}
Qualidade ou estado do que é vertical.
\section{Verticalização}
\begin{itemize}
\item {Grp. gram.:f.}
\end{itemize}
Acto ou effeito de verticalizar.
\section{Verticalizar}
\begin{itemize}
\item {Grp. gram.:v. t.}
\end{itemize}
Tornar vertical.
\section{Verticalmente}
\begin{itemize}
\item {Grp. gram.:adv.}
\end{itemize}
De modo vertical; a prumo.
\section{Vértice}
\begin{itemize}
\item {Grp. gram.:m.}
\end{itemize}
\begin{itemize}
\item {Proveniência:(Lat. \textunderscore vertex\textunderscore )}
\end{itemize}
O ponto mais elevado da abóbada craniana.
Cimo; cume; culminância.
Ponto, onde se juntam as duas linhas que formam um ângulo.
No triângulo, o vértice do ângulo opposto á base.
Ponto, em que se reúnem todos os lados de uma pyrâmide.
\section{Verticelado}
\begin{itemize}
\item {Grp. gram.:adj.}
\end{itemize}
(V.verticilado)
\section{Verticellado}
\begin{itemize}
\item {Grp. gram.:adj.}
\end{itemize}
(V.verticillado)
\section{Verticello}
\begin{itemize}
\item {Grp. gram.:m.}
\end{itemize}
(V.verticillo)
\section{Verticelo}
\begin{itemize}
\item {Grp. gram.:m.}
\end{itemize}
(V.verticilo)
\section{Verticidade}
\begin{itemize}
\item {Grp. gram.:f.}
\end{itemize}
\begin{itemize}
\item {Proveniência:(Do lat. \textunderscore vertex\textunderscore )}
\end{itemize}
Tendência de uma coisa para se dirigir mais para um lado do que para outro.
\section{Verticilado}
\begin{itemize}
\item {Grp. gram.:adj.}
\end{itemize}
\begin{itemize}
\item {Proveniência:(De \textunderscore verticilo\textunderscore )}
\end{itemize}
Disposto em verticilo, (falando-se de órgãos vegetaes).
Constituído por órgãos vegetaes.
\section{Verticilifloro}
\begin{itemize}
\item {Grp. gram.:adj.}
\end{itemize}
\begin{itemize}
\item {Utilização:Bot.}
\end{itemize}
\begin{itemize}
\item {Proveniência:(De \textunderscore verticilo\textunderscore  + \textunderscore flôr\textunderscore )}
\end{itemize}
Diz-se das espigas, compostas de verticilos.
\section{Verticillado}
\begin{itemize}
\item {Grp. gram.:adj.}
\end{itemize}
\begin{itemize}
\item {Proveniência:(De \textunderscore verticillo\textunderscore )}
\end{itemize}
Disposto em verticillo, (falando-se de órgãos vegetaes).
Constituído por órgãos vegetaes.
\section{Verticillifloro}
\begin{itemize}
\item {Grp. gram.:adj.}
\end{itemize}
\begin{itemize}
\item {Utilização:Bot.}
\end{itemize}
\begin{itemize}
\item {Proveniência:(De \textunderscore verticillo\textunderscore  + \textunderscore flôr\textunderscore )}
\end{itemize}
Diz-se das espigas, compostas de verticillos.
\section{Verticillo}
\begin{itemize}
\item {Grp. gram.:m.}
\end{itemize}
\begin{itemize}
\item {Proveniência:(Lat. \textunderscore verticillus\textunderscore )}
\end{itemize}
Conjunto das partes das flôres ou dos órgãos foliáceos, dispostos em volta de um eixo commum e no mesmo plano horizontal.
\section{Verticilo}
\begin{itemize}
\item {Grp. gram.:m.}
\end{itemize}
\begin{itemize}
\item {Proveniência:(Lat. \textunderscore verticillus\textunderscore )}
\end{itemize}
Conjunto das partes das flôres ou dos órgãos foliáceos, dispostos em volta de um eixo comum e no mesmo plano horizontal.
\section{Vertigem}
\begin{itemize}
\item {Grp. gram.:f.}
\end{itemize}
\begin{itemize}
\item {Utilização:Fig.}
\end{itemize}
\begin{itemize}
\item {Proveniência:(Lat. \textunderscore vertigo\textunderscore )}
\end{itemize}
Estado mórbido, em que ao indivíduo parece que todos os objectos giram em volta dêlle e que êlle proprio gira.
Tontura.
Desmaio.
Desvario.
Tentação súbita.
\section{Vertiginosamente}
\begin{itemize}
\item {Grp. gram.:adj.}
\end{itemize}
De modo vertiginoso.
Rapidamente.
\section{Vertiginoso}
\begin{itemize}
\item {Grp. gram.:adj.}
\end{itemize}
\begin{itemize}
\item {Utilização:Fig.}
\end{itemize}
\begin{itemize}
\item {Proveniência:(Lat. \textunderscore vertiginosus\textunderscore )}
\end{itemize}
Que tem vertigens.
Que produz vertigens.
Que gira rapidamente; rápido.
Que perturba a razão ou a serenidade do espirito.
\section{Vertígio}
\begin{itemize}
\item {Grp. gram.:m.}
\end{itemize}
\begin{itemize}
\item {Utilização:Des.}
\end{itemize}
O mesmo que \textunderscore vertigem\textunderscore .
\section{Veruto}
\begin{itemize}
\item {Grp. gram.:m.}
\end{itemize}
\begin{itemize}
\item {Proveniência:(Lat. \textunderscore verutum\textunderscore )}
\end{itemize}
Lança ou dardo, entre os antigos.
\section{Verzéa}
\begin{itemize}
\item {Grp. gram.:f.}
\end{itemize}
\begin{itemize}
\item {Utilização:Ant.}
\end{itemize}
O mesmo que \textunderscore vergel\textunderscore ? Cf. \textunderscore Peregrinação\textunderscore , XVI.
\section{Vesânia}
\begin{itemize}
\item {Grp. gram.:f.}
\end{itemize}
\begin{itemize}
\item {Proveniência:(Lat. \textunderscore vesania\textunderscore )}
\end{itemize}
Nome genérico das differentes espécies de alienação mental.
\section{Vesânico}
\begin{itemize}
\item {Grp. gram.:adj.}
\end{itemize}
Relativo á vesânia.
\section{Vesano}
\begin{itemize}
\item {Grp. gram.:adj.}
\end{itemize}
\begin{itemize}
\item {Proveniência:(Lat. \textunderscore vesanus\textunderscore )}
\end{itemize}
Demente.
Delirante; insensato.
\section{Vesco}
\begin{itemize}
\item {Grp. gram.:adj.}
\end{itemize}
\begin{itemize}
\item {Proveniência:(Lat. \textunderscore vescus\textunderscore )}
\end{itemize}
O mesmo que \textunderscore comestível\textunderscore .
\section{Vesgo}
\begin{itemize}
\item {fónica:vês}
\end{itemize}
\begin{itemize}
\item {Grp. gram.:adj.}
\end{itemize}
\begin{itemize}
\item {Grp. gram.:M.}
\end{itemize}
\begin{itemize}
\item {Proveniência:(Do lat. hyp. \textunderscore versicus\textunderscore , de \textunderscore versus\textunderscore )}
\end{itemize}
Que tem o defeito do estrabismo; zarolho.
Indivíduo vesgo.
\section{Vesguear}
\begin{itemize}
\item {Grp. gram.:v. i.}
\end{itemize}
\begin{itemize}
\item {Utilização:Fig.}
\end{itemize}
Sêr vesgo.
Olhar de soslaio.
Vêr mal.
\section{Vesgueiro}
\begin{itemize}
\item {Grp. gram.:adj.}
\end{itemize}
O mesmo que \textunderscore vesgo\textunderscore .
\section{Vesicação}
\begin{itemize}
\item {Grp. gram.:f.}
\end{itemize}
\begin{itemize}
\item {Proveniência:(De \textunderscore vesicar\textunderscore )}
\end{itemize}
Acto de produzir vesículas por meio de uma substância irritante.
\section{Vesical}
\begin{itemize}
\item {Grp. gram.:adj.}
\end{itemize}
\begin{itemize}
\item {Proveniência:(Do lat. \textunderscore vesica\textunderscore )}
\end{itemize}
Relativo á bexiga.
\section{Vesicante}
\begin{itemize}
\item {Grp. gram.:adj.}
\end{itemize}
\begin{itemize}
\item {Grp. gram.:M.}
\end{itemize}
\begin{itemize}
\item {Grp. gram.:Pl.}
\end{itemize}
\begin{itemize}
\item {Proveniência:(Lat. \textunderscore vesicans\textunderscore )}
\end{itemize}
Que produz vesículas.
Substância, que produz vesículas.
Família de insectos coleópteros.
\section{Vesicar}
\begin{itemize}
\item {Grp. gram.:v. t.}
\end{itemize}
\begin{itemize}
\item {Proveniência:(Lat. \textunderscore vesicare\textunderscore )}
\end{itemize}
Produzir vesículas em.
\section{Vesicária}
\begin{itemize}
\item {Grp. gram.:f.}
\end{itemize}
\begin{itemize}
\item {Proveniência:(Lat. \textunderscore vesicaria\textunderscore )}
\end{itemize}
Gênero de plantas crucíferas.
\section{Vesicatório}
\begin{itemize}
\item {Grp. gram.:m.  e  adj.}
\end{itemize}
O mesmo que \textunderscore vesicante\textunderscore .
\section{Vesico-rectal}
\begin{itemize}
\item {Grp. gram.:adj.}
\end{itemize}
\begin{itemize}
\item {Utilização:Anat.}
\end{itemize}
Relativo á bexiga e ao recto.
\section{Vesico-uterino}
\begin{itemize}
\item {Grp. gram.:adj.}
\end{itemize}
\begin{itemize}
\item {Utilização:Anat.}
\end{itemize}
Relativo á bexiga e ao útero.
\section{Vesico-vaginal}
\begin{itemize}
\item {Grp. gram.:adj.}
\end{itemize}
\begin{itemize}
\item {Utilização:Anat.}
\end{itemize}
Relativo á bexiga e á vagina.
\section{Vesícula}
\begin{itemize}
\item {Grp. gram.:f.}
\end{itemize}
\begin{itemize}
\item {Proveniência:(Lat. \textunderscore vesicula\textunderscore )}
\end{itemize}
Pequena bexiga ou cavidade.
Bolha.
Pequeno saco, cheio de ar, que se encontra nos peixes, e que os torna mais ou menos ligeiros, segundo querem subir ou descer na água.
\section{Vesicular}
\begin{itemize}
\item {Grp. gram.:adj.}
\end{itemize}
Semelhante a uma vesícula.
Formado por vesículas.
\section{Vesiculoso}
\begin{itemize}
\item {Grp. gram.:adj.}
\end{itemize}
\begin{itemize}
\item {Proveniência:(Lat. \textunderscore vesiculosus\textunderscore )}
\end{itemize}
Vesicular; que tem vesículas.
\section{Vespa}
\begin{itemize}
\item {fónica:vês}
\end{itemize}
\begin{itemize}
\item {Grp. gram.:f.}
\end{itemize}
\begin{itemize}
\item {Utilização:Fig.}
\end{itemize}
\begin{itemize}
\item {Proveniência:(Lat. \textunderscore vespa\textunderscore )}
\end{itemize}
Gênero de insectos hymenópteros, semelhantes ás abelhas, voláteis e munidos de ferrão como ellas.
Pessôa intratável e mordaz.
\section{Vespão}
\begin{itemize}
\item {Grp. gram.:m.}
\end{itemize}
Grande vespa.
\section{Vespeiro}
\begin{itemize}
\item {Grp. gram.:m.}
\end{itemize}
\begin{itemize}
\item {Utilização:Fig.}
\end{itemize}
Reunião de vespas.
Toca, habitada por vespas.
Lugar, onde ellas se ajuntam.
Lugar, onde imprevistamente se deparam insidias ou perigos.
\section{Vésper}
\begin{itemize}
\item {Grp. gram.:m.}
\end{itemize}
O mesmo que \textunderscore véspero\textunderscore .
\section{Véspera}
\begin{itemize}
\item {Grp. gram.:f.}
\end{itemize}
\begin{itemize}
\item {Grp. gram.:F. Pl.}
\end{itemize}
\begin{itemize}
\item {Proveniência:(Do lat. \textunderscore vesper\textunderscore )}
\end{itemize}
A tarde.
O dia, que precede immediatamente aquelle de que se trata.
Época ou tempo, que antecede certos acontecimentos.
Uma das horas canónicas, que se reza de tarde.
\section{Vesperal}
\begin{itemize}
\item {Grp. gram.:adj.}
\end{itemize}
\begin{itemize}
\item {Grp. gram.:M.}
\end{itemize}
\begin{itemize}
\item {Proveniência:(Lat. \textunderscore vesperalis\textunderscore )}
\end{itemize}
Relativo á tarde.
Livro, que contém as rezas litúrgicas, chamadas vésperas.
\section{Vespérias}
\begin{itemize}
\item {Grp. gram.:f. pl.}
\end{itemize}
\begin{itemize}
\item {Proveniência:(De \textunderscore véspera\textunderscore )}
\end{itemize}
Exame, que o doutorando da Universidade fazia na véspera do doutoramento.
\section{Véspero}
\begin{itemize}
\item {Grp. gram.:m.}
\end{itemize}
\begin{itemize}
\item {Utilização:Fig.}
\end{itemize}
\begin{itemize}
\item {Proveniência:(Do lat. \textunderscore vesper\textunderscore )}
\end{itemize}
O planeta Vênus, quando se avista de tarde.
Estrêlla da tarde.
O Occidente.
\section{Vespertino}
\begin{itemize}
\item {Grp. gram.:adj.}
\end{itemize}
\begin{itemize}
\item {Proveniência:(Lat. \textunderscore vespertinus\textunderscore )}
\end{itemize}
O mesmo que \textunderscore vesperal\textunderscore .
\section{Vespianos}
\begin{itemize}
\item {Grp. gram.:m. pl.}
\end{itemize}
Tríbo de insectos hymenópteros que têm por typo a vespa.
\section{Vespícia}
\begin{itemize}
\item {Grp. gram.:f.}
\end{itemize}
Tecido antigo de Cambaia.
\section{Vespilão}
\begin{itemize}
\item {Grp. gram.:m.}
\end{itemize}
\begin{itemize}
\item {Proveniência:(Lat. \textunderscore vespillo\textunderscore )}
\end{itemize}
Aquele que, entre os Romanos, enterrava de noite os cadáveres dos pobres.
\section{Vespilheira}
\begin{itemize}
\item {Grp. gram.:f.}
\end{itemize}
\begin{itemize}
\item {Utilização:Prov.}
\end{itemize}
\begin{itemize}
\item {Utilização:minh.}
\end{itemize}
Mulhér mexeriqueira, linguareira, intriguista. (Colhido em Barcelos)
(Cp. \textunderscore vespa\textunderscore )
\section{Vespillão}
\begin{itemize}
\item {Grp. gram.:m.}
\end{itemize}
\begin{itemize}
\item {Proveniência:(Lat. \textunderscore vespillo\textunderscore )}
\end{itemize}
Aquelle que, entre os Romanos, enterrava de noite os cadáveres dos pobres.
\section{Véspora}
\begin{itemize}
\item {Grp. gram.:f.}
\end{itemize}
\begin{itemize}
\item {Utilização:ant.}
\end{itemize}
\begin{itemize}
\item {Utilização:Pop.}
\end{itemize}
O mesmo que \textunderscore véspera\textunderscore . Cf. R. Pina, \textunderscore Aff. V\textunderscore , CXXXI.
\section{Vessada}
\begin{itemize}
\item {Grp. gram.:f.}
\end{itemize}
\begin{itemize}
\item {Utilização:Prov.}
\end{itemize}
\begin{itemize}
\item {Utilização:minh.}
\end{itemize}
\begin{itemize}
\item {Proveniência:(De \textunderscore vessar\textunderscore )}
\end{itemize}
Terra fértil e regadia.
Terra, que se lavra num dia, com uma junta de bois; geira.
Vessadela.
\section{Vessada}
\begin{itemize}
\item {Grp. gram.:f.}
\end{itemize}
\begin{itemize}
\item {Utilização:Ant.}
\end{itemize}
Correia, o mesmo que \textunderscore avessada\textunderscore . Cf. Mestre Geraldo.
\section{Vessadela}
\begin{itemize}
\item {Grp. gram.:F.}
\end{itemize}
Acto de vessar.
Terreno, que se lavra num dia.
\section{Vessadoiro}
\begin{itemize}
\item {Grp. gram.:m.}
\end{itemize}
\begin{itemize}
\item {Utilização:Prov.}
\end{itemize}
\begin{itemize}
\item {Utilização:beir.}
\end{itemize}
\begin{itemize}
\item {Utilização:minh.}
\end{itemize}
\begin{itemize}
\item {Grp. gram.:Adj.}
\end{itemize}
\begin{itemize}
\item {Utilização:Prov.}
\end{itemize}
\begin{itemize}
\item {Proveniência:(De \textunderscore vessar\textunderscore )}
\end{itemize}
O mesmo que \textunderscore vessadela\textunderscore .
Direito de vessar uma terra.

Arado, que se emprega na lavoira de terras em que se semeia milho grosso, desde fins de Abril a fins de Maio.
Diz-se de arado, que se emprega nas lavoiras de milho grosso.
\section{Vessadouro}
\begin{itemize}
\item {Grp. gram.:m.}
\end{itemize}
\begin{itemize}
\item {Utilização:Prov.}
\end{itemize}
\begin{itemize}
\item {Utilização:beir.}
\end{itemize}
\begin{itemize}
\item {Utilização:minh.}
\end{itemize}
\begin{itemize}
\item {Grp. gram.:Adj.}
\end{itemize}
\begin{itemize}
\item {Utilização:Prov.}
\end{itemize}
\begin{itemize}
\item {Proveniência:(De \textunderscore vessar\textunderscore )}
\end{itemize}
O mesmo que \textunderscore vessadela\textunderscore .
Direito de vessar uma terra.

Arado, que se emprega na lavoira de terras em que se semeia milho grosso, desde fins de Abril a fins de Maio.
Diz-se de arado, que se emprega nas lavoiras de milho grosso.
\section{Vessar}
\begin{itemize}
\item {Grp. gram.:v. t.}
\end{itemize}
\begin{itemize}
\item {Proveniência:(Do lat. \textunderscore versare\textunderscore )}
\end{itemize}
Lavrar profundamente.
Lavrar para sementeiras.
\section{Vessas}
\begin{itemize}
\item {Grp. gram.:f. pl.}
\end{itemize}
\begin{itemize}
\item {Grp. gram.:Loc.}
\end{itemize}
\begin{itemize}
\item {Utilização:fam.}
\end{itemize}
O mesmo que \textunderscore avessas\textunderscore .
\textunderscore Ás vessas\textunderscore , o mesmo que \textunderscore ás avessas\textunderscore . Cf. \textunderscore Filodemo\textunderscore , act. II, sc. IV.
\section{Vestaes}
\begin{itemize}
\item {Grp. gram.:f. pl.}
\end{itemize}
\begin{itemize}
\item {Proveniência:(Lat. \textunderscore vestalia\textunderscore )}
\end{itemize}
Antigas festas romanas, em honra da deusa Vesta.
\section{Vestais}
\begin{itemize}
\item {Grp. gram.:f. pl.}
\end{itemize}
\begin{itemize}
\item {Proveniência:(Lat. \textunderscore vestalia\textunderscore )}
\end{itemize}
Antigas festas romanas, em honra da deusa Vesta.
\section{Vestal}
\begin{itemize}
\item {Grp. gram.:f.}
\end{itemize}
\begin{itemize}
\item {Utilização:Fig.}
\end{itemize}
\begin{itemize}
\item {Grp. gram.:Adj.}
\end{itemize}
\begin{itemize}
\item {Utilização:Des.}
\end{itemize}
\begin{itemize}
\item {Proveniência:(Lat. \textunderscore vestalis\textunderscore )}
\end{itemize}
Sacerdotisa da deusa Vesta.
Mulhér muito honesta.
Mulhér casta ou virgem.
Relativo ou semelhante ás sacerdotisas de Vesta; virginal.
\section{Vestálias}
\begin{itemize}
\item {Grp. gram.:f. pl.}
\end{itemize}
\begin{itemize}
\item {Proveniência:(Lat. \textunderscore vestalia\textunderscore )}
\end{itemize}
Antigas festas pagans, em honra de Vesta, o mesmo que \textunderscore vestaes\textunderscore .
\section{Vestalidade}
\begin{itemize}
\item {Grp. gram.:f.}
\end{itemize}
Qualidade de vestal.
\section{Vestalino}
\begin{itemize}
\item {Grp. gram.:adj.}
\end{itemize}
\begin{itemize}
\item {Utilização:Neol.}
\end{itemize}
\begin{itemize}
\item {Proveniência:(De \textunderscore vestal\textunderscore )}
\end{itemize}
Puro como as sacerdotisas de Vesta.
Immaculado.
\section{Veste}
\begin{itemize}
\item {Grp. gram.:f.}
\end{itemize}
\begin{itemize}
\item {Proveniência:(Lat. \textunderscore vestis\textunderscore )}
\end{itemize}
Vestuário; véstia.
Vestidura sacerdotal.
\section{Véstia}
\begin{itemize}
\item {Grp. gram.:f.}
\end{itemize}
\begin{itemize}
\item {Utilização:Bras. do N}
\end{itemize}
\begin{itemize}
\item {Proveniência:(De \textunderscore veste\textunderscore )}
\end{itemize}
Espécie de casaco curto, que se differença da jaqueta em que esta acompanha a fórma da cintura, e a véstia desce direita, não se ajustando á cintura.
Casaco de coiro, usado por vaqueiros.
\section{Véstia}
\begin{itemize}
\item {Grp. gram.:f.}
\end{itemize}
Gênero de plantas solanáceas.
\section{Vestiairo}
\begin{itemize}
\item {Grp. gram.:m.}
\end{itemize}
\begin{itemize}
\item {Utilização:Ant.}
\end{itemize}
O mesmo que \textunderscore vestiário\textunderscore .
\section{Vestiaria}
\begin{itemize}
\item {Grp. gram.:f.}
\end{itemize}
\begin{itemize}
\item {Proveniência:(De \textunderscore veste\textunderscore )}
\end{itemize}
Guarda-roupa de uma corporação.
O mesmo que \textunderscore indumentaria\textunderscore .
\section{Vestiário}
\begin{itemize}
\item {Grp. gram.:m.}
\end{itemize}
\begin{itemize}
\item {Proveniência:(Lat. \textunderscore vestiarius\textunderscore )}
\end{itemize}
Aquelle que tem a seu cargo o guarda-roupa de uma corporação.
Aquelle que inspeccionava as vestiarias.
O mesmo que \textunderscore vestiaria\textunderscore .
Compartimento, annexo aos tribunaes, onde os magistrados guardam as suas vestes profissionaes.
\section{Vestibular}
\begin{itemize}
\item {Grp. gram.:adj.}
\end{itemize}
Relativo ao vestíbulo.
\section{Vestíbulo}
\begin{itemize}
\item {Grp. gram.:m.}
\end{itemize}
\begin{itemize}
\item {Proveniência:(Lat. \textunderscore vestibulum\textunderscore )}
\end{itemize}
Espaço entre a via pública e a entrada de um edifício; átrio.
Entrada de um edifício.
Porta principal.
Espaço entre a porta e a principal escadaria interior.
Pátio.
Uma das cavidades do ouvido interior.
\section{Vestideira}
\begin{itemize}
\item {Grp. gram.:f.}
\end{itemize}
Um dos apparelhos das fábricas de fiação. Cf. \textunderscore Inquér. Industr.\textunderscore , p. II, l. II, 122.
\section{Vestido}
\begin{itemize}
\item {Grp. gram.:m.}
\end{itemize}
\begin{itemize}
\item {Proveniência:(De \textunderscore vestir\textunderscore )}
\end{itemize}
Objecto de vestuário.
Veste, própria de senhoras ou de meninas, que cobre todo o corpo, ou composta de saia e casaco da mesma fazenda, sem separação entre o casaco ou corpete e a saia.
\section{Vestidura}
\begin{itemize}
\item {Grp. gram.:f.}
\end{itemize}
\begin{itemize}
\item {Proveniência:(Do lat. \textunderscore vestitura\textunderscore )}
\end{itemize}
Tudo que se póde vestir.
Fato.
Ceremónia monástica, em que se toma o hábito religioso.
\section{Vestígio}
\begin{itemize}
\item {Grp. gram.:m.}
\end{itemize}
\begin{itemize}
\item {Utilização:Fig.}
\end{itemize}
\begin{itemize}
\item {Proveniência:(Lat. \textunderscore vestigium\textunderscore )}
\end{itemize}
Sinal, que o homem ou o animal faz com os pés no sítio por onde passa.
Rasto.
Pègada.
Indício.
Restos.
\section{Vestimenta}
\begin{itemize}
\item {Grp. gram.:f.}
\end{itemize}
\begin{itemize}
\item {Grp. gram.:Pl.}
\end{itemize}
\begin{itemize}
\item {Proveniência:(Lat. \textunderscore vestimenta\textunderscore )}
\end{itemize}
O mesmo que \textunderscore vestidura\textunderscore .
Vestes sacerdotaes em actos solennes.
\section{Vestimenteiro}
\begin{itemize}
\item {Grp. gram.:m.}
\end{itemize}
Aquelle que faz vestimentas.
\section{Vestir}
\begin{itemize}
\item {Grp. gram.:v. t.}
\end{itemize}
\begin{itemize}
\item {Grp. gram.:V. i.}
\end{itemize}
\begin{itemize}
\item {Proveniência:(Lat. \textunderscore vestire\textunderscore )}
\end{itemize}
Cobrir com veste.
Pôr sôbre si (qualquer peça de vestuário): \textunderscore vestir um casaco\textunderscore .
Dar vestuário a: \textunderscore vestir os pobres\textunderscore .
Calçar (luvas):«\textunderscore vestia luvas gemma de ovo.\textunderscore »Camillo, \textunderscore Noites de Lam.\textunderscore , 234.«\textunderscore ...vestindo as luvas, em ar de retirar-se.\textunderscore »\textunderscore Idem\textunderscore , \textunderscore Onde está a Felic.\textunderscore , 258.
Fazer as despesas de vestuário a favor de.
Cobrir, revestir, resguardar.
Alcatifar, ferrar.
Adornar.
Tingir-se de.
Tingir.
Munir.
Disfarçar.
Pôr veste, trajar.
\section{Véo}
\begin{itemize}
\item {Grp. gram.:m.}
\end{itemize}
\begin{itemize}
\item {Utilização:Fig.}
\end{itemize}
\begin{itemize}
\item {Proveniência:(Do lat. \textunderscore velum\textunderscore )}
\end{itemize}
Tecido, com que se cobre qualquer coisa.
Tecido transparente, com que as senhoras cobrem o rosto.
Mantilha de freira.
Aquillo que é comparável a um véo.
Aquillo que serve para encobrir alguma coisa: \textunderscore já levantei o véo dêsse mystério\textunderscore .
Pretexto.
Trevas: \textunderscore o véo da noite\textunderscore .
Amargura.
\section{Vesto}
\begin{itemize}
\item {Grp. gram.:m.  e  adj.}
\end{itemize}
\begin{itemize}
\item {Utilização:Ant.}
\end{itemize}
\begin{itemize}
\item {Proveniência:(De \textunderscore vestir\textunderscore )}
\end{itemize}
O mesmo que \textunderscore vestido\textunderscore .
\section{Vestoria}
\begin{itemize}
\item {Grp. gram.:f.}
\end{itemize}
\begin{itemize}
\item {Utilização:Pop.}
\end{itemize}
O mesmo que \textunderscore vistoria\textunderscore .
\section{Vestuário}
\begin{itemize}
\item {Grp. gram.:m.}
\end{itemize}
\begin{itemize}
\item {Proveniência:(Do lat. \textunderscore vestiarius\textunderscore )}
\end{itemize}
Conjunto das peças de roupa, que se vestem; traje; fato.
\section{Vesugo}
\begin{itemize}
\item {Grp. gram.:m.}
\end{itemize}
O mesmo que \textunderscore besugo\textunderscore :«\textunderscore grande comão de vesugos.\textunderscore »Resende, \textunderscore Cancion. Geral.\textunderscore 
\section{Vesuviana}
\begin{itemize}
\item {Grp. gram.:f.}
\end{itemize}
\begin{itemize}
\item {Proveniência:(De \textunderscore Vesúvio\textunderscore , n. p.)}
\end{itemize}
Espécie de pedra preciosa.
\section{Vesuviano}
\begin{itemize}
\item {Grp. gram.:adj.}
\end{itemize}
Relativo ao Vesúvio. Cf. Camillo, \textunderscore Volcões\textunderscore , 10.
\section{Vesúvio}
\begin{itemize}
\item {Grp. gram.:m.}
\end{itemize}
\begin{itemize}
\item {Utilização:Fig.}
\end{itemize}
\begin{itemize}
\item {Proveniência:(De \textunderscore Vesúvio\textunderscore , n. p.)}
\end{itemize}
Cataclismo.
Irrupção (de paixões). Cf. Camillo, \textunderscore Myst. de Lisb.\textunderscore , I, 19; II, 172 e 179.
\section{Veteranice}
\begin{itemize}
\item {Grp. gram.:f.}
\end{itemize}
Qualidade do que é veterano.
\section{Veterano}
\begin{itemize}
\item {Grp. gram.:adj.}
\end{itemize}
\begin{itemize}
\item {Utilização:Fig.}
\end{itemize}
\begin{itemize}
\item {Grp. gram.:M.}
\end{itemize}
\begin{itemize}
\item {Utilização:Escol.}
\end{itemize}
\begin{itemize}
\item {Utilização:Fig.}
\end{itemize}
\begin{itemize}
\item {Proveniência:(Lat. \textunderscore veteranus\textunderscore )}
\end{itemize}
Que envelheceu no serviço militar.
Que envelheceu em qualquer serviço.
Soldado antigo ou reformado.
Estudante, que frequenta algum dos últimos annos de qualquer faculdade ou escola superior.
Pessôa, que envelheceu numa profissão ou offício.
\section{Veterinária}
\begin{itemize}
\item {Grp. gram.:f.}
\end{itemize}
\begin{itemize}
\item {Proveniência:(De \textunderscore veterinário\textunderscore )}
\end{itemize}
Conhecimento de Anatomia e das doenças dos animaes irracionaes.
\section{Veterinário}
\begin{itemize}
\item {Grp. gram.:adj.}
\end{itemize}
\begin{itemize}
\item {Grp. gram.:M.}
\end{itemize}
\begin{itemize}
\item {Proveniência:(Lat. \textunderscore veterinarius\textunderscore )}
\end{itemize}
Relativo á Veterinária.
Relativo aos animaes irracionaes.
Aquelle que sabe Veterinária.
Médico veterinário.
\section{Vetiver}
\begin{itemize}
\item {Grp. gram.:m.}
\end{itemize}
Planta gramínea e muito aromática, da Índia, (\textunderscore andropogon squarrosus\textunderscore , Lin.).
\section{Veto}
\begin{itemize}
\item {Grp. gram.:m.}
\end{itemize}
\begin{itemize}
\item {Proveniência:(Lat. \textunderscore veto\textunderscore , indic. de \textunderscore vetare\textunderscore )}
\end{itemize}
Prohibição.
Suspensão.
Opposição.
Direito, que o chefe de Estado tem, de recusar a sua sancção a uma lei votada pelas câmaras legislativas.
\section{Vetões}
\begin{itemize}
\item {Grp. gram.:m.}
\end{itemize}
\begin{itemize}
\item {Proveniência:(Lat. \textunderscore vettones\textunderscore )}
\end{itemize}
Povo antigo da Lusitânia.
\section{Vettões}
\begin{itemize}
\item {Grp. gram.:m.}
\end{itemize}
\begin{itemize}
\item {Proveniência:(Lat. \textunderscore vettones\textunderscore )}
\end{itemize}
Povo antigo da Lusitânia.
\section{Vetustade}
\begin{itemize}
\item {Grp. gram.:f.}
\end{itemize}
\begin{itemize}
\item {Proveniência:(Lat. \textunderscore vetustas\textunderscore )}
\end{itemize}
O mesmo que \textunderscore vetustez\textunderscore .
\section{Vetustez}
\begin{itemize}
\item {Grp. gram.:f.}
\end{itemize}
Qualidade de vetusto. Cf. Júl. Dinís, \textunderscore Morgadinha\textunderscore , 47.
\section{Vetusto}
\begin{itemize}
\item {Grp. gram.:adj.}
\end{itemize}
\begin{itemize}
\item {Proveniência:(Lat. \textunderscore vetustus\textunderscore )}
\end{itemize}
Muito velho.
Antigo.
Deteriorado pelo tempo.
Respeitável pela sua idade.
\section{Véu}
\begin{itemize}
\item {Grp. gram.:m.}
\end{itemize}
\begin{itemize}
\item {Utilização:Fig.}
\end{itemize}
\begin{itemize}
\item {Proveniência:(Do lat. \textunderscore velum\textunderscore )}
\end{itemize}
Tecido, com que se cobre qualquer coisa.
Tecido transparente, com que as senhoras cobrem o rosto.
Mantilha de freira.
Aquillo que é comparável a um véu.
Aquillo que serve para encobrir alguma coisa: \textunderscore já levantei o véu dêsse mystério\textunderscore .
Pretexto.
Trevas: \textunderscore o véu da noite\textunderscore .
Amargura.
\section{Veuzinho}
\begin{itemize}
\item {Grp. gram.:m.}
\end{itemize}
\begin{itemize}
\item {Proveniência:(De \textunderscore véu\textunderscore )}
\end{itemize}
Variedade de uva branca.
\section{Veuzinho-verdeal}
\begin{itemize}
\item {Grp. gram.:m.}
\end{itemize}
Casta de uva, na região do Doiro.
A mesma que \textunderscore veuzinho\textunderscore ?
\section{V. Ex.^a}
(Abrev. de \textunderscore Vossa Excellência\textunderscore )
\section{Vexação}
\begin{itemize}
\item {Grp. gram.:f.}
\end{itemize}
\begin{itemize}
\item {Proveniência:(Do lat. \textunderscore vexatio\textunderscore )}
\end{itemize}
Acto ou effeito de vexar.
\section{Vexador}
\begin{itemize}
\item {Grp. gram.:m.  e  adj.}
\end{itemize}
\begin{itemize}
\item {Proveniência:(Do lat. \textunderscore vexator\textunderscore )}
\end{itemize}
O que vexa.
\section{Vexame}
\begin{itemize}
\item {Grp. gram.:m.}
\end{itemize}
\begin{itemize}
\item {Proveniência:(Lat. \textunderscore vexamen\textunderscore )}
\end{itemize}
O mesmo que \textunderscore vexação\textunderscore .
Aquillo que produz vexação; vergonha; afronta.
\section{Vexante}
\begin{itemize}
\item {Grp. gram.:adj.}
\end{itemize}
\begin{itemize}
\item {Proveniência:(Lat. \textunderscore vexans\textunderscore )}
\end{itemize}
Que vexa.
\section{Vexar}
\begin{itemize}
\item {Grp. gram.:v. t.}
\end{itemize}
\begin{itemize}
\item {Proveniência:(Lat. \textunderscore vexare\textunderscore )}
\end{itemize}
Maltratar; molestar; afrontar.
Humilhar; envergonhar.
\section{Vexativo}
\begin{itemize}
\item {Grp. gram.:adj.}
\end{itemize}
\begin{itemize}
\item {Proveniência:(Lat. \textunderscore vexativus\textunderscore )}
\end{itemize}
O mesmo que \textunderscore vexatório\textunderscore .
\section{Vexatório}
\begin{itemize}
\item {Grp. gram.:adj.}
\end{itemize}
\begin{itemize}
\item {Proveniência:(De \textunderscore vexar\textunderscore )}
\end{itemize}
O mesmo que \textunderscore vexante\textunderscore .
\section{Vexiga}
\begin{itemize}
\item {Grp. gram.:f.}
\end{itemize}
(V.bexiga)
\section{Vexilar}
\begin{itemize}
\item {fónica:csi}
\end{itemize}
\begin{itemize}
\item {Grp. gram.:adj.}
\end{itemize}
\begin{itemize}
\item {Utilização:Bot.}
\end{itemize}
\begin{itemize}
\item {Proveniência:(Do lat. \textunderscore vexillum\textunderscore )}
\end{itemize}
Diz-se da prefloração, própria das corolas das papilionáceas.
\section{Vexilária}
\begin{itemize}
\item {fónica:csi}
\end{itemize}
\begin{itemize}
\item {Grp. gram.:f.}
\end{itemize}
\begin{itemize}
\item {Proveniência:(Lat. \textunderscore vexillaria\textunderscore )}
\end{itemize}
Gênero de plantas leguminosas.
\section{Vexilário}
\begin{itemize}
\item {fónica:csi}
\end{itemize}
\begin{itemize}
\item {Grp. gram.:m.}
\end{itemize}
\begin{itemize}
\item {Utilização:Ant.}
\end{itemize}
\begin{itemize}
\item {Proveniência:(Lat. \textunderscore vexillarius\textunderscore )}
\end{itemize}
O mesmo que \textunderscore porta-bandeira\textunderscore .
\section{Vexillar}
\begin{itemize}
\item {fónica:csi}
\end{itemize}
\begin{itemize}
\item {Grp. gram.:adj.}
\end{itemize}
\begin{itemize}
\item {Utilização:Bot.}
\end{itemize}
\begin{itemize}
\item {Proveniência:(Do lat. \textunderscore vexillum\textunderscore )}
\end{itemize}
Diz-se da prefloração, própria das corollas das papilionáceas.
\section{Vexillária}
\begin{itemize}
\item {fónica:csi}
\end{itemize}
\begin{itemize}
\item {Grp. gram.:f.}
\end{itemize}
\begin{itemize}
\item {Proveniência:(Lat. \textunderscore vexillaria\textunderscore )}
\end{itemize}
Gênero de plantas leguminosas.
\section{Vexillário}
\begin{itemize}
\item {fónica:csi}
\end{itemize}
\begin{itemize}
\item {Grp. gram.:m.}
\end{itemize}
\begin{itemize}
\item {Utilização:Ant.}
\end{itemize}
\begin{itemize}
\item {Proveniência:(Lat. \textunderscore vexillarius\textunderscore )}
\end{itemize}
O mesmo que \textunderscore porta-bandeira\textunderscore .
\section{Vexillo}
\begin{itemize}
\item {fónica:csi}
\end{itemize}
\begin{itemize}
\item {Grp. gram.:m.}
\end{itemize}
\begin{itemize}
\item {Utilização:Ant.}
\end{itemize}
\begin{itemize}
\item {Proveniência:(Lat. \textunderscore vexillum\textunderscore )}
\end{itemize}
Estandarte, bandeira.
\section{Vexilo}
\begin{itemize}
\item {fónica:csi}
\end{itemize}
\begin{itemize}
\item {Grp. gram.:m.}
\end{itemize}
\begin{itemize}
\item {Utilização:Ant.}
\end{itemize}
\begin{itemize}
\item {Proveniência:(Lat. \textunderscore vexillum\textunderscore )}
\end{itemize}
Estandarte, bandeira.
\section{Vez}
\begin{itemize}
\item {Grp. gram.:f.}
\end{itemize}
\begin{itemize}
\item {Proveniência:(Lat. \textunderscore vicis\textunderscore )}
\end{itemize}
Termo, com que se indica um facto na sua unidade ou na sua repetição: \textunderscore enganou-se uma vez\textunderscore .
Ensejo, occasião: \textunderscore para outra vez, falaremos\textunderscore .
Turno; alternativa.
Reciprocidade.
Quinhão.
\section{Vezada}
\begin{itemize}
\item {Grp. gram.:f.}
\end{itemize}
\begin{itemize}
\item {Utilização:Prov.}
\end{itemize}
\begin{itemize}
\item {Utilização:beir.}
\end{itemize}
\begin{itemize}
\item {Proveniência:(De \textunderscore vez\textunderscore )}
\end{itemize}
Cada uma das vezes, em que se pratíca ou succede qualquer coisa; vez.
\section{Vezar}
\begin{itemize}
\item {Grp. gram.:v. t.  e  p.}
\end{itemize}
O mesmo que \textunderscore avezar\textunderscore ^1.
\section{Vezeira}
\begin{itemize}
\item {Grp. gram.:f.}
\end{itemize}
\begin{itemize}
\item {Utilização:Prov.}
\end{itemize}
\begin{itemize}
\item {Utilização:trasm.}
\end{itemize}
\begin{itemize}
\item {Utilização:minh.}
\end{itemize}
Rebanho, que se reveza com outros em certas pastagens.
(Cast. \textunderscore vezera\textunderscore )
\section{Vezeiro}
\begin{itemize}
\item {Grp. gram.:m.}
\end{itemize}
\begin{itemize}
\item {Utilização:T. do Gerez}
\end{itemize}
O mesmo que \textunderscore vezeira\textunderscore .
Proprietário de vezeira.
\section{Vezeiro}
\begin{itemize}
\item {Grp. gram.:adj.}
\end{itemize}
Que tem vêzo; acostumado; reincidente.
\section{Vezer}
\begin{itemize}
\item {Grp. gram.:v. t.}
\end{itemize}
\begin{itemize}
\item {Utilização:Gír.}
\end{itemize}
O mesmo que \textunderscore vêr\textunderscore .
\section{Vezinho}
\textunderscore m.\textunderscore  e \textunderscore adj.\textunderscore  (e der.)
O mesmo ou melhór que \textunderscore vizinho\textunderscore , etc. Cf. \textunderscore Eufrosina\textunderscore , 108.
\section{Vezino}
\begin{itemize}
\item {Grp. gram.:m.  e  adj.}
\end{itemize}
\begin{itemize}
\item {Utilização:Ant.}
\end{itemize}
O mesmo que \textunderscore vizinho\textunderscore . Cf. Usque, 35.
\section{Vêzo}
\begin{itemize}
\item {Grp. gram.:m.}
\end{itemize}
\begin{itemize}
\item {Utilização:Pop.}
\end{itemize}
\begin{itemize}
\item {Proveniência:(Do lat. \textunderscore vitium\textunderscore )}
\end{itemize}
Costume censurável ou vicioso.
Qualquer hábito ou costume.
\section{Via}
\begin{itemize}
\item {Grp. gram.:f.}
\end{itemize}
\begin{itemize}
\item {Proveniência:(Lat. \textunderscore via\textunderscore )}
\end{itemize}
Lugar, por onde se vai ou se é levado.
Caminho.
Direcção.
Espaço entre os carris, no caminho de ferro.
Qualquer canal do organismo humano.
Rumo.
Meio.
Modo.
Exemplar de uma letra ou documento commercial.
\textunderscore Via férrea\textunderscore , caminho de ferro.
\section{Viação}
\begin{itemize}
\item {Grp. gram.:f.}
\end{itemize}
\begin{itemize}
\item {Proveniência:(De \textunderscore via\textunderscore )}
\end{itemize}
Modo ou meio de andar ou transportar de um lugar para outro, por caminho ou ruas.
Conjunto de estradas ou caminhos.
Serviço de vehículos que fazem carreira entre vários pontos, para uso público.
\section{Viabilidade}
\begin{itemize}
\item {Grp. gram.:f.}
\end{itemize}
\begin{itemize}
\item {Proveniência:(Do lat. \textunderscore viabilis\textunderscore )}
\end{itemize}
Qualidade do que é viável.
\section{Viadalhas}
\begin{itemize}
\item {Grp. gram.:f. pl.}
\end{itemize}
\begin{itemize}
\item {Utilização:Ant.}
\end{itemize}
O mesmo que \textunderscore vitualhas\textunderscore :«\textunderscore leuarão hum alqueyre de pão... e algum binho, que... chamão viadalhas\textunderscore ». \textunderscore Alvará\textunderscore  de D. Sebast., in \textunderscore Rev. Lus.\textunderscore , XV, 132.
(Cp. \textunderscore viático\textunderscore )
\section{Viado}
\begin{itemize}
\item {Grp. gram.:m.}
\end{itemize}
Antigo pano listrado.
(Por \textunderscore veado\textunderscore , de \textunderscore veio\textunderscore )
\section{Viador}
\begin{itemize}
\item {Grp. gram.:m.}
\end{itemize}
\begin{itemize}
\item {Utilização:Des.}
\end{itemize}
\begin{itemize}
\item {Proveniência:(De \textunderscore via\textunderscore )}
\end{itemize}
Aquelle que viaja.
Passageiro.
Antigo empregado superior da casa real, ao serviço da raínha.
Antigo camarista da raínha.
O homem vivente, aquelle que vai seguindo a viagem da vida. Cf. Vieira.
\section{Viadora}
\begin{itemize}
\item {Grp. gram.:f.}
\end{itemize}
\begin{itemize}
\item {Utilização:Bras}
\end{itemize}
\begin{itemize}
\item {Proveniência:(De \textunderscore viador\textunderscore )}
\end{itemize}
O mesmo que \textunderscore viatura\textunderscore . Cf. \textunderscore País\textunderscore , do Rio, de 11-I-901.
\section{Viaducto}
\begin{itemize}
\item {Grp. gram.:m.}
\end{itemize}
\begin{itemize}
\item {Proveniência:(Do lat. \textunderscore via\textunderscore  + \textunderscore ductus\textunderscore )}
\end{itemize}
Ponte, que liga as duas vertentes que formam um valle ou qualquer depressão de terreno, e destinada a fazer parte de um caminho de ferro, estrada, etc.
\section{Viaduto}
\begin{itemize}
\item {Grp. gram.:m.}
\end{itemize}
\begin{itemize}
\item {Proveniência:(Do lat. \textunderscore via\textunderscore  + \textunderscore ductus\textunderscore )}
\end{itemize}
Ponte, que liga as duas vertentes que formam um vale ou qualquer depressão de terreno, e destinada a fazer parte de um caminho de ferro, estrada, etc.
\section{Viage}
\begin{itemize}
\item {Grp. gram.:f.}
\end{itemize}
(V.viagem)
\section{Viageiro}
\begin{itemize}
\item {Grp. gram.:adj.}
\end{itemize}
\begin{itemize}
\item {Grp. gram.:M.}
\end{itemize}
\begin{itemize}
\item {Proveniência:(De \textunderscore viagem\textunderscore )}
\end{itemize}
Relativo a viagem.
Aquelle que viaja.
\section{Viagem}
\begin{itemize}
\item {Grp. gram.:f.}
\end{itemize}
\begin{itemize}
\item {Proveniência:(Do lat. \textunderscore viaticum\textunderscore )}
\end{itemize}
Acto de andar, para chegar de um ponto a outro, mais ou menos distante; jornada longa; navegação.
\section{Viagíssimo}
\begin{itemize}
\item {Grp. gram.:adj.}
\end{itemize}
Que viajou muito:«\textunderscore ...companheiro do Pinto viagíssimo\textunderscore ». Filinto, XI, 134.
(Extravagância filintiana, de \textunderscore viagem\textunderscore )
\section{Viajador}
\begin{itemize}
\item {Grp. gram.:m.  e  adj.}
\end{itemize}
(V.viajante)
\section{Viajante}
\begin{itemize}
\item {Grp. gram.:adj.}
\end{itemize}
\begin{itemize}
\item {Grp. gram.:M.  e  f.}
\end{itemize}
Que viaja.
Pessôa, que viaja.
\section{Viajar}
\begin{itemize}
\item {Grp. gram.:v. i.}
\end{itemize}
\begin{itemize}
\item {Grp. gram.:V. t.}
\end{itemize}
\begin{itemize}
\item {Proveniência:(De \textunderscore viage\textunderscore )}
\end{itemize}
Fazer viagem.
Percorrer.
\section{Viajata}
\begin{itemize}
\item {Grp. gram.:f.}
\end{itemize}
\begin{itemize}
\item {Utilização:Fam.}
\end{itemize}
\begin{itemize}
\item {Proveniência:(De \textunderscore viage\textunderscore )}
\end{itemize}
Viagem de recreio.
\section{Viajor}
\begin{itemize}
\item {Grp. gram.:m.}
\end{itemize}
O mesmo que \textunderscore viageiro\textunderscore .
\section{Vianda}
\begin{itemize}
\item {Grp. gram.:f.}
\end{itemize}
\begin{itemize}
\item {Utilização:Prov.}
\end{itemize}
\begin{itemize}
\item {Utilização:Prov.}
\end{itemize}
\begin{itemize}
\item {Proveniência:(Fr. \textunderscore viande\textunderscore )}
\end{itemize}
Qualquer espécie de alimento.
Qualquer carne, que serve de alimento.
Carne de animaes terrestres.
A parte do caldo, que não é líquida.
Cozido para os porcos; lavaduras, restos de comida.
\section{Viandante}
\begin{itemize}
\item {Grp. gram.:m. ,  f.  e  adj.}
\end{itemize}
Pessôa, que vianda ou viaja.
\section{Viandar}
\begin{itemize}
\item {Grp. gram.:v. i.}
\end{itemize}
\begin{itemize}
\item {Proveniência:(De \textunderscore via\textunderscore  + \textunderscore andar\textunderscore )}
\end{itemize}
Viajar; peregrinar.
\section{Viandeiro}
\begin{itemize}
\item {Grp. gram.:m.  e  adj.}
\end{itemize}
O que gosta de vianda; glotão.
\section{Vianense}
\begin{itemize}
\item {Grp. gram.:m.  e  adj.}
\end{itemize}
O mesmo que \textunderscore vianês\textunderscore .
\section{Vianês}
\begin{itemize}
\item {Grp. gram.:adj.}
\end{itemize}
\begin{itemize}
\item {Grp. gram.:M.}
\end{itemize}
Relativo a Viana.
Habitante de Viana.
\section{Vianesa}
\begin{itemize}
\item {Grp. gram.:f.  e  adj.}
\end{itemize}
\begin{itemize}
\item {Proveniência:(De \textunderscore vianês\textunderscore )}
\end{itemize}
Diz-se de uma espécie de uva preta.
\section{Viário}
\begin{itemize}
\item {Grp. gram.:m.}
\end{itemize}
\begin{itemize}
\item {Utilização:Neol.}
\end{itemize}
\begin{itemize}
\item {Proveniência:(De \textunderscore via\textunderscore )}
\end{itemize}
Leito da via-férrea.
Espaço, occupado por ella.
\section{Viaticar}
\begin{itemize}
\item {Grp. gram.:v. t.}
\end{itemize}
Dar viático a.
\section{Viático}
\begin{itemize}
\item {Grp. gram.:m.}
\end{itemize}
\begin{itemize}
\item {Utilização:T. de Lisbôa}
\end{itemize}
\begin{itemize}
\item {Proveniência:(Lat. \textunderscore viaticum\textunderscore )}
\end{itemize}
Provisão de dinheiro ou de mantimentos para viagem.
Farnel.
Sacramento da communhão, ministrado aos enfermos em sua residência.--É abusivo êste significado, porque a communhão aos enfermos não tem ritualmente a fórma exclusiva do Sagrado Viático: \textunderscore accipe viaticum\textunderscore .
\section{Viatório}
\begin{itemize}
\item {Grp. gram.:adj.}
\end{itemize}
\begin{itemize}
\item {Utilização:Ant.}
\end{itemize}
\begin{itemize}
\item {Proveniência:(Lat. \textunderscore viatorius\textunderscore )}
\end{itemize}
Relativo a caminho, a via:«\textunderscore ...as linhas viatórias que mostram os caminhos do mar.\textunderscore »Fern. Oliv., \textunderscore Livro da Fabr. das Naus\textunderscore .
\section{Viatura}
\begin{itemize}
\item {Grp. gram.:f.}
\end{itemize}
Designação genérica de qualquer vehículo.
Meio de transporte.
(Do mesmo rad. que \textunderscore viático\textunderscore , sob a infl. do fr. \textunderscore voiture\textunderscore )
\section{Viável}
\begin{itemize}
\item {Grp. gram.:adj.}
\end{itemize}
\begin{itemize}
\item {Proveniência:(Do lat. \textunderscore viare\textunderscore , caminhar)}
\end{itemize}
Que póde sêr percorrido.
Que não offerece obstáculo.
\section{Viável}
\begin{itemize}
\item {Grp. gram.:adj.}
\end{itemize}
\begin{itemize}
\item {Utilização:Gal}
\end{itemize}
\begin{itemize}
\item {Proveniência:(Fr. \textunderscore viable\textunderscore )}
\end{itemize}
(V.vivedoiro)
\section{Viba}
\begin{itemize}
\item {Grp. gram.:f.}
\end{itemize}
Cana de açúcar.
\section{Vibal}
\begin{itemize}
\item {Grp. gram.:m.}
\end{itemize}
\begin{itemize}
\item {Utilização:Prov.}
\end{itemize}
\begin{itemize}
\item {Utilização:trasm.}
\end{itemize}
O mesmo que \textunderscore alvanhal\textunderscore .
\section{Víbice}
\begin{itemize}
\item {Grp. gram.:m.}
\end{itemize}
\begin{itemize}
\item {Proveniência:(Do lat. \textunderscore vibex\textunderscore , \textunderscore vibicis\textunderscore )}
\end{itemize}
O mesmo que \textunderscore vergastada\textunderscore .
\section{Víbora}
\begin{itemize}
\item {Grp. gram.:f.}
\end{itemize}
\begin{itemize}
\item {Utilização:Açor}
\end{itemize}
\begin{itemize}
\item {Utilização:Fig.}
\end{itemize}
\begin{itemize}
\item {Proveniência:(Do lat. \textunderscore vipera\textunderscore )}
\end{itemize}
Gênero de reptís ophídios.
Peixe, espécie de moreia, cuja mordedura é venenosa.
Pessôa de má índole ou de mau gênio.
\section{Víbora}
\begin{itemize}
\item {Grp. gram.:m.}
\end{itemize}
\begin{itemize}
\item {Utilização:Prov.}
\end{itemize}
Ave, o mesmo que \textunderscore abibe\textunderscore .
\section{Vibordo}
\begin{itemize}
\item {Grp. gram.:m.}
\end{itemize}
\begin{itemize}
\item {Utilização:Náut.}
\end{itemize}
\begin{itemize}
\item {Proveniência:(Do ingl. \textunderscore waist\textunderscore , cinta, e \textunderscore board\textunderscore , prancha)}
\end{itemize}
Prancha grossa, que serve de parapeito a um navio; amurada.
\section{Vibração}
\begin{itemize}
\item {Grp. gram.:f.}
\end{itemize}
\begin{itemize}
\item {Proveniência:(Do lat. \textunderscore vibratio\textunderscore )}
\end{itemize}
Acto ou effeito de vibrar.
Balanço, oscillação.
Movimento especial de corda ou fio tenso, fixo nas extremidades e desviado da posição recta por qualquer impulso, como nas cordas de viola tangida.
Estado do ar ou da atmosphera, quando se lhes communica movimento análogo.
Qualidade de uma voz ou de som, que communica análogo movimento ás camadas atmosphéricas.
\section{Vibrante}
\begin{itemize}
\item {Grp. gram.:adj.}
\end{itemize}
\begin{itemize}
\item {Proveniência:(Lat. \textunderscore vibrans\textunderscore )}
\end{itemize}
Que vibra.
\section{Vibrar}
\begin{itemize}
\item {Grp. gram.:v. t.}
\end{itemize}
\begin{itemize}
\item {Grp. gram.:V. i.}
\end{itemize}
\begin{itemize}
\item {Proveniência:(Lat. \textunderscore vibrare\textunderscore )}
\end{itemize}
Fazer oscillar ou tremular.
Agitar.
Brandir.
Fazer soar, tanger: \textunderscore vibrar as cordas do bandolim\textunderscore .
Arremessar, atirar.
Abalar.
Commover.
Infundir.
Espargir.
Sentir ternura, estremecer.
Commover-se.
Produzir sons.
Têr som claro e distinto.
\section{Vibrátil}
\begin{itemize}
\item {Grp. gram.:adj.}
\end{itemize}
O mesmo que \textunderscore vibrante\textunderscore .
\section{Vibratilidade}
\begin{itemize}
\item {Grp. gram.:f.}
\end{itemize}
Qualidade do que é vibrátil.
\section{Vibratoriamente}
\begin{itemize}
\item {Grp. gram.:adv.}
\end{itemize}
De modo vibratório; com vibração.
\section{Vibratório}
\begin{itemize}
\item {Grp. gram.:adj.}
\end{itemize}
\begin{itemize}
\item {Proveniência:(De \textunderscore vibrar\textunderscore )}
\end{itemize}
O mesmo que \textunderscore vibrante\textunderscore .
Que produz vibração ou é acompanhado por ella.
\section{Vibrião}
\begin{itemize}
\item {Grp. gram.:m.}
\end{itemize}
\begin{itemize}
\item {Proveniência:(Fr. \textunderscore vibrion\textunderscore )}
\end{itemize}
Gênero de infusórios, de figura linear e de movimento vibratório ou onduloso, (\textunderscore vibrio lineola\textunderscore , Lin.)
Espécie de bacillo; bacillo curvo.
\section{Vibrioniano}
\begin{itemize}
\item {Grp. gram.:adj.}
\end{itemize}
Relativo a vibrião:«\textunderscore carácter vibrioniano.\textunderscore »R. Jorge, \textunderscore Epidemia de Lisbôa\textunderscore , 8.
\section{Vibriónidos}
\begin{itemize}
\item {Grp. gram.:m. pl.}
\end{itemize}
\begin{itemize}
\item {Proveniência:(De \textunderscore vibrião\textunderscore )}
\end{itemize}
Família de infusórios que se apresentam sob o aspecto de simples linhas ou filamentos.
\section{Vibriónios}
\begin{itemize}
\item {Grp. gram.:m. pl.}
\end{itemize}
O mesmo ou melhór que \textunderscore vibriónidos\textunderscore .
\section{Vibrissas}
\begin{itemize}
\item {Grp. gram.:f. pl.}
\end{itemize}
\begin{itemize}
\item {Proveniência:(Lat. \textunderscore vibrissae\textunderscore )}
\end{itemize}
Pêlos, que se desenvolvem nas fossas nasaes.
\section{Vibroscópio}
\begin{itemize}
\item {Grp. gram.:m.}
\end{itemize}
\begin{itemize}
\item {Proveniência:(De \textunderscore vibrar\textunderscore  + gr. \textunderscore skopein\textunderscore )}
\end{itemize}
Instrumento, para estudar as vibrações dos corpos sonoros.
\section{Vibúrneas}
\begin{itemize}
\item {Grp. gram.:f. pl.}
\end{itemize}
\begin{itemize}
\item {Utilização:Bot.}
\end{itemize}
\begin{itemize}
\item {Proveniência:(De \textunderscore viburno\textunderscore )}
\end{itemize}
O mesmo que \textunderscore sambucáceas\textunderscore .
\section{Viburno}
\begin{itemize}
\item {Grp. gram.:m.}
\end{itemize}
\begin{itemize}
\item {Proveniência:(Lat. \textunderscore viburnum\textunderscore )}
\end{itemize}
Espécie de vime.
\section{Vicácia}
\begin{itemize}
\item {Grp. gram.:f.}
\end{itemize}
\begin{itemize}
\item {Proveniência:(De \textunderscore Vicat\textunderscore , n. p.)}
\end{itemize}
Gênero de plantas umbellíferas.
\section{Viçar}
\begin{itemize}
\item {Grp. gram.:v. t.  e  i.}
\end{itemize}
\begin{itemize}
\item {Grp. gram.:V. i.}
\end{itemize}
\begin{itemize}
\item {Utilização:Fig.}
\end{itemize}
\begin{itemize}
\item {Utilização:Bras. do N}
\end{itemize}
\begin{itemize}
\item {Utilização:Bras. do N}
\end{itemize}
O mesmo que \textunderscore vicejar\textunderscore .
Desenvolver-se, aumentar, alastrar-se: \textunderscore nos povos decadentes, viçam os escândalos, o egoísmo...\textunderscore 
Têr cio (o gado).
Comer terra.
\section{Vicarial}
\begin{itemize}
\item {Grp. gram.:adj.}
\end{itemize}
\begin{itemize}
\item {Proveniência:(Do lat. \textunderscore vicarius\textunderscore )}
\end{itemize}
Relativo ao vigário ou ao vicariato.
\section{Vicariante}
\begin{itemize}
\item {Grp. gram.:adj.}
\end{itemize}
\begin{itemize}
\item {Utilização:Med.}
\end{itemize}
\begin{itemize}
\item {Proveniência:(Fr. \textunderscore vicariant\textunderscore )}
\end{itemize}
Diz-se de um órgão, cuja actividade suppre, mais ou menos, a falta de actividade em outro órgão.
\section{Vicariato}
\begin{itemize}
\item {Grp. gram.:m.}
\end{itemize}
\begin{itemize}
\item {Utilização:Ext.}
\end{itemize}
\begin{itemize}
\item {Proveniência:(Do lat. \textunderscore vicarius\textunderscore )}
\end{itemize}
Cargo de vigário.
Exercício dêsse cargo.
Tempo que elle dura.
Residência do vigário.
Território, comprehendido na jurisdicção de um vigário.
Substituição no exercício de quaesquer funcções.
\section{Vicário}
\begin{itemize}
\item {Grp. gram.:adj.}
\end{itemize}
\begin{itemize}
\item {Utilização:Des.}
\end{itemize}
\begin{itemize}
\item {Utilização:Gram.}
\end{itemize}
\begin{itemize}
\item {Grp. gram.:M.}
\end{itemize}
Que faz as vezes de outrem.
Diz-se do verbo, que se emprega, para evitar a repetição de outro: \textunderscore desejo«ir»lá, mas não o«faço»\textunderscore .
O mesmo que \textunderscore vigário\textunderscore .
\section{Vice...}
\begin{itemize}
\item {Grp. gram.:pref.}
\end{itemize}
\begin{itemize}
\item {Proveniência:(Do lat. \textunderscore vicis\textunderscore )}
\end{itemize}
(designativo de \textunderscore substituição\textunderscore , \textunderscore inferioridade\textunderscore , ou \textunderscore categoria\textunderscore  immediatamente inferior a outra)
\section{Vice-almirantado}
\begin{itemize}
\item {Grp. gram.:m.}
\end{itemize}
Cargo ou dignidade de Vice-Almirante.
\section{Vice-Almirante}
\begin{itemize}
\item {Grp. gram.:m.}
\end{itemize}
Official de marinha, immediatamente inferior ao Almirante.
\section{Vice-bailio}
\begin{itemize}
\item {Grp. gram.:m.}
\end{itemize}
Aquelle que substituía o bailio.
\section{Vice-Chanceller}
\begin{itemize}
\item {Grp. gram.:m.}
\end{itemize}
Aquelle que substitue o Chanceller, na falta ou impedimento dêste.
Cardeal presidente da cúria romana, para o despacho de bullas e breves apostólicos.
\section{Vice-Cônsul}
\begin{itemize}
\item {Grp. gram.:m.}
\end{itemize}
Pessôa, que exerce as funcções de Cônsul, na falta ou impedimento dêste.
\section{Vice-consulado}
\begin{itemize}
\item {Grp. gram.:m.}
\end{itemize}
Cargo de Vice-Cônsul.
Repartição ou casa, onde o Vice-Cônsul exerce as suas funcções.
Território, comprehendido na jurisdicção do Vice-Cônsul.
\section{Vice-Deus}
\begin{itemize}
\item {Grp. gram.:m.}
\end{itemize}
\begin{itemize}
\item {Utilização:Des.}
\end{itemize}
Aquelle que faz as vezes de Deus, (falando-se de alguns santos).
Título respeitoso, que também se tem dado aos Pontífices e aos Monarchas.
\section{Vice-dómino}
\begin{itemize}
\item {Grp. gram.:m.}
\end{itemize}
\begin{itemize}
\item {Utilização:Des.}
\end{itemize}
\begin{itemize}
\item {Proveniência:(Do lat. \textunderscore vicis\textunderscore  + \textunderscore dominus\textunderscore )}
\end{itemize}
Aquelle que representa o senhor ou faz as suas vezes. Cf. Herculano, \textunderscore Hist. de Port.\textunderscore , IV, 52.
\section{Vice-gerente}
\begin{itemize}
\item {Grp. gram.:m.}
\end{itemize}
Aquelle que substitue o gerente:«\textunderscore ...e nos mostramos imagens e vice-gerentes do Criador.\textunderscore »Castilho.
\section{Vice-Governador}
\begin{itemize}
\item {Grp. gram.:m.}
\end{itemize}
Aquelle que faz as vezes de Governador.
\section{Vicejante}
\begin{itemize}
\item {Grp. gram.:adj.}
\end{itemize}
Que viceja.
\section{Vicejar}
\begin{itemize}
\item {Grp. gram.:v. i.}
\end{itemize}
\begin{itemize}
\item {Grp. gram.:V. t.}
\end{itemize}
Têr viço.
Vegetar opulentamente.
Ostentar-se brilhante ou exuberante.
Garrir.
Fazer brotar exuberantemente.
Promover o viço a.
\section{Vicejo}
\begin{itemize}
\item {Grp. gram.:m.}
\end{itemize}
Acto ou effeito de vicejar.
\section{Vice-legação}
\begin{itemize}
\item {Grp. gram.:f.}
\end{itemize}
Cargo de vice-legado.
Edifício, onde funcciona o vice-legado.
\section{Vice-legado}
\begin{itemize}
\item {Grp. gram.:m.}
\end{itemize}
Aquelle que faz as vezes de legado.
\section{Vice-mordómo}
\begin{itemize}
\item {Grp. gram.:m.}
\end{itemize}
Aquelle que faz as vezes de mordómo.
\section{Vice-morte}
\begin{itemize}
\item {Grp. gram.:f.}
\end{itemize}
Estado, semelhante ao da morte.
\section{Vicenal}
\begin{itemize}
\item {Grp. gram.:adj.}
\end{itemize}
\begin{itemize}
\item {Proveniência:(Lat. \textunderscore vicennalis\textunderscore )}
\end{itemize}
Relativo ao vicênio.
\section{Vicênio}
\begin{itemize}
\item {Grp. gram.:m.}
\end{itemize}
\begin{itemize}
\item {Proveniência:(Lat. \textunderscore vicennium\textunderscore )}
\end{itemize}
Espaço de vinte anos.
\section{Vicennal}
\begin{itemize}
\item {Grp. gram.:adj.}
\end{itemize}
\begin{itemize}
\item {Proveniência:(Lat. \textunderscore vicennalis\textunderscore )}
\end{itemize}
Relativo ao vicênnio.
\section{Vicênnio}
\begin{itemize}
\item {Grp. gram.:m.}
\end{itemize}
\begin{itemize}
\item {Proveniência:(Lat. \textunderscore vicennium\textunderscore )}
\end{itemize}
Espaço de vinte annos.
\section{Vicente}
\begin{itemize}
\item {Grp. gram.:m.}
\end{itemize}
\begin{itemize}
\item {Utilização:Gír.}
\end{itemize}
\begin{itemize}
\item {Utilização:Prov.}
\end{itemize}
\begin{itemize}
\item {Utilização:trasm.}
\end{itemize}
\begin{itemize}
\item {Proveniência:(De \textunderscore Vicente\textunderscore , n. p.)}
\end{itemize}
Corvo.
O mesmo que \textunderscore gato\textunderscore .
\section{Vicente}
\begin{itemize}
\item {Grp. gram.:m.}
\end{itemize}
Moéda de oiro, no tempo de D. Sebastião.
\section{Vicentes}
\begin{itemize}
\item {Grp. gram.:m. pl.}
\end{itemize}
\begin{itemize}
\item {Utilização:Prov.}
\end{itemize}
\begin{itemize}
\item {Utilização:trasm.}
\end{itemize}
Tamancos, sócos. (Colhido em Sabrosa)
\section{Vicentino}
\begin{itemize}
\item {Grp. gram.:adj.}
\end{itemize}
Relativo a S. Vicente de Fóra, em Lisbôa. Cf. Camillo, \textunderscore Perfil do Marquês\textunderscore , 181.
Relativo a Gil Vicente: \textunderscore os autos vicentinos\textunderscore .
\section{Vice-pai}
\begin{itemize}
\item {Grp. gram.:m.}
\end{itemize}
Aquelle que faz as vezes de pai:«\textunderscore ...um varão, que as famílias respeitem, como vice-pai de todas elas.\textunderscore »Castilho, \textunderscore Avarento\textunderscore , 75.
\section{Vice-presidência}
\begin{itemize}
\item {Grp. gram.:f.}
\end{itemize}
Cargo ou dignidade de vice-presidente.
\section{Vice-presidencial}
\begin{itemize}
\item {Grp. gram.:adj.}
\end{itemize}
Relativo á vice-presidência ou ao vice-presidente.
\section{Vice-presidente}
\begin{itemize}
\item {Grp. gram.:m.}
\end{itemize}
Aquelle que faz as vezes de presidente ou o substitue nos seus impedimentos.
\section{Vice-Providência}
\begin{itemize}
\item {Grp. gram.:f.}
\end{itemize}
Aquelle ou aquillo, que representa a Providência. Cf. Castilho, \textunderscore Sabichonas\textunderscore , 73.
\section{Vice-província}
\begin{itemize}
\item {Grp. gram.:f.}
\end{itemize}
Conjunto de casas religiosas ou conventos, que ainda não constituem província, mas como tal são considerados.
\section{Vice-provincial}
\begin{itemize}
\item {Grp. gram.:m.}
\end{itemize}
Aquelle que faz as vezes do provincial.
\section{Vice-questor}
\begin{itemize}
\item {Grp. gram.:m.}
\end{itemize}
\begin{itemize}
\item {Proveniência:(Lat. \textunderscore vicequaestor\textunderscore )}
\end{itemize}
Substituto ou adjunto do questor.
\section{Vice-questura}
\begin{itemize}
\item {Grp. gram.:f.}
\end{itemize}
\begin{itemize}
\item {Proveniência:(Lat. \textunderscore vicequaestura\textunderscore )}
\end{itemize}
Cargo ou dignidade vice-questor.
\section{Vice-Raínha}
\begin{itemize}
\item {Grp. gram.:f.}
\end{itemize}
Esposa ou viúva do Vice-Rei.
Mulhér, que exerce as funcções de Vice-Rei.
\section{Vice-real}
\begin{itemize}
\item {Grp. gram.:adj.}
\end{itemize}
Relativo ao Vice-Rei.
\section{Vice-Rei}
\begin{itemize}
\item {Grp. gram.:m.}
\end{itemize}
\begin{itemize}
\item {Proveniência:(De \textunderscore vice...\textunderscore  + \textunderscore rei\textunderscore )}
\end{itemize}
Aquelle que governa um Estado subordinado a um reino.
O que exerce poderes quase iguaes aos do Rei.
\section{Vice-Reina}
(V.Vice-Raínha)
\section{Vice-reinado}
\begin{itemize}
\item {Grp. gram.:m.}
\end{itemize}
Cargo do Vice-Rei.
Tempo que dura esse cargo.
Território governado por um Vice-Rei.
\section{Vice-reinar}
\begin{itemize}
\item {Grp. gram.:v. i.}
\end{itemize}
Exercer funcções de Vice-Rei. Cf. Filinto, \textunderscore D. Man.\textunderscore , II, 163.
\section{Vice-Reitor}
\begin{itemize}
\item {Grp. gram.:m.}
\end{itemize}
Aquelle que faz as vezes de Reitor.
Funcionário, de categoria immediatamente inferior á de Reitor, e que, juntamente com êste exerce suas funcções.
\section{Vice-reitorado}
\begin{itemize}
\item {Grp. gram.:m.}
\end{itemize}
Cargo de Vice-Reitor.
Tempo que dura êsse cargo.
Lugar onde funcciona o Vice-Reitor.
\section{Vice-reitoria}
\begin{itemize}
\item {Grp. gram.:f.}
\end{itemize}
O mesmo que \textunderscore vice-reitorado\textunderscore .
\section{Vicésimo}
\begin{itemize}
\item {Grp. gram.:adj.}
\end{itemize}
\begin{itemize}
\item {Proveniência:(Lat. \textunderscore vicesimus\textunderscore )}
\end{itemize}
(V.vigésimo)
\section{Vice-versa}
\begin{itemize}
\item {Grp. gram.:loc. adv.}
\end{itemize}
\begin{itemize}
\item {Proveniência:(Do lat. \textunderscore vicis\textunderscore  + \textunderscore versus\textunderscore )}
\end{itemize}
Mutuamente.
Em sentido inverso; invertendo os termos.
\section{Viceversa}
\begin{itemize}
\item {Grp. gram.:loc. adv.}
\end{itemize}
\begin{itemize}
\item {Proveniência:(Do lat. \textunderscore vicis\textunderscore  + \textunderscore versus\textunderscore )}
\end{itemize}
Mutuamente.
Em sentido inverso; invertendo os termos.
\section{Vícia}
\begin{itemize}
\item {Grp. gram.:f.}
\end{itemize}
Planta leguminosa do Brasil.
\section{Viciação}
\begin{itemize}
\item {Grp. gram.:f.}
\end{itemize}
\begin{itemize}
\item {Proveniência:(Do lat. \textunderscore vitiatio\textunderscore )}
\end{itemize}
Acto ou effeito de viciar.
\section{Viciador}
\begin{itemize}
\item {Grp. gram.:m.  e  adj.}
\end{itemize}
\begin{itemize}
\item {Proveniência:(Do lat. \textunderscore vitiator\textunderscore )}
\end{itemize}
O que vicia.
\section{Viciamento}
\begin{itemize}
\item {Grp. gram.:m.}
\end{itemize}
O mesmo que \textunderscore viciação\textunderscore .
\section{Viciar}
\begin{itemize}
\item {Grp. gram.:v. t.}
\end{itemize}
\begin{itemize}
\item {Proveniência:(Lat. \textunderscore vitiare\textunderscore )}
\end{itemize}
Communicar vício a.
Corromper; deteriorar; adulterar; falsificar.
Fazer que não tenha effeito, annullar.
\section{Vicilino}
\begin{itemize}
\item {Grp. gram.:m.}
\end{itemize}
\begin{itemize}
\item {Proveniência:(Do lat. \textunderscore vicilinus\textunderscore , vigilante?)}
\end{itemize}
O mesmo que \textunderscore colibri\textunderscore .
\section{Vicinal}
\begin{itemize}
\item {Grp. gram.:adj.}
\end{itemize}
\begin{itemize}
\item {Proveniência:(Lat. \textunderscore vicinalis\textunderscore )}
\end{itemize}
Vizinho.
Diz-se especialmente do caminho ou estrada, que liga aldeias ou freguesias, dentro de um concelho.
\section{Vicinalidade}
\begin{itemize}
\item {Grp. gram.:f.}
\end{itemize}
Qualidade do que é vicinal.
\section{Vício}
\begin{itemize}
\item {Grp. gram.:m.}
\end{itemize}
\begin{itemize}
\item {Proveniência:(Lat. \textunderscore vitium\textunderscore )}
\end{itemize}
Grande defeito ou imperfeição.
Tendência habitual para certo mal.
Hábito de proceder mal.
Costume censurável ou condemnável.
Libertinagem.
Desmoralização.
Pessôas viciosas.
Deformidade phýsica ou constituição orgânica, defeituosa.
Viciação.
Hábito que, sem sêr absolutamente condemnável, prejudica de alguma fórma quem o tem; costumeira.
\section{Viciosamente}
\begin{itemize}
\item {Grp. gram.:adv.}
\end{itemize}
De modo vicioso.
\section{Viciosidade}
\begin{itemize}
\item {Grp. gram.:f.}
\end{itemize}
\begin{itemize}
\item {Proveniência:(Do lat. \textunderscore viciositas\textunderscore )}
\end{itemize}
Qualidade do que é vicioso.
\section{Vicioso}
\begin{itemize}
\item {Grp. gram.:adj.}
\end{itemize}
\begin{itemize}
\item {Proveniência:(Lat. \textunderscore vitiosus\textunderscore )}
\end{itemize}
Que tem vício ou vícios.
Em que há vícios.
Desmoralizado; corrompido.
Que tem defeito grave.
Que é opposto a certos preceitos ou regras: \textunderscore linguagem viciosa\textunderscore .
\section{Vicissitude}
\begin{itemize}
\item {Grp. gram.:f.}
\end{itemize}
\begin{itemize}
\item {Proveniência:(Lat. \textunderscore vicissitudo\textunderscore )}
\end{itemize}
Mudança ou diversidade de coisas que se succedem.
Alternativa; alteração.
Instabilidade das coisas.
Eventualidade; revés.
\section{Vicissitudinário}
\begin{itemize}
\item {Grp. gram.:adj.}
\end{itemize}
Em que há vicissitudes.
Sujeito a vicissitudes.
\section{Viço}
\begin{itemize}
\item {Grp. gram.:m.}
\end{itemize}
\begin{itemize}
\item {Utilização:Ant.}
\end{itemize}
Vigor de vegetação, numa planta ou em plantas.
Verdor.
Exuberância de vida.
Vigor.
Frescura.
Mimo.
Carinho excessivo.
Bravura de animal, em consequência de descanso.
O mesmo que \textunderscore vício\textunderscore . Cf. Pant. de Aveiro, \textunderscore Itiner.\textunderscore , 15, (2.^a ed.); Frei Fortun., \textunderscore Inéd.\textunderscore , 316.
\section{Vico}
\begin{itemize}
\item {Grp. gram.:m.}
\end{itemize}
\begin{itemize}
\item {Proveniência:(Lat. \textunderscore vicus\textunderscore )}
\end{itemize}
Bairro de uma cidade.
Aldeia.
Prédio rústico, entre os Romanos. Cf. Herculano, \textunderscore Hist. de Port.\textunderscore , IV, 94.
\section{Vícoa}
\begin{itemize}
\item {Grp. gram.:f.}
\end{itemize}
Gênero de plantas synanthéreas.
\section{Viçor}
\begin{itemize}
\item {Grp. gram.:m.}
\end{itemize}
\begin{itemize}
\item {Utilização:P. us.}
\end{itemize}
O mesmo que \textunderscore viço\textunderscore .
Acto de viçar. Cf. Camillo, \textunderscore Noite de Insómn.\textunderscore , II, 20; IX, 66.
\section{Viçosamente}
\begin{itemize}
\item {Grp. gram.:adv.}
\end{itemize}
De modo viçoso; com viço.
\section{Viçoso}
\begin{itemize}
\item {Grp. gram.:adj.}
\end{itemize}
\begin{itemize}
\item {Utilização:Fig.}
\end{itemize}
\begin{itemize}
\item {Utilização:Bras. do N}
\end{itemize}
Que tem viço.
Tenro.
Inexperiente: \textunderscore idade viçosa\textunderscore .
Que tem o vício de comer terra.
\section{Victima}
\begin{itemize}
\item {Grp. gram.:f.}
\end{itemize}
\begin{itemize}
\item {Proveniência:(Lat. \textunderscore victima\textunderscore )}
\end{itemize}
Criatura viva, immolada em holocausto a uma divindade.
Pessôa, sacrificada aos interesses ou paixões de outrem.
Pessôa, que foi ferida ou assassinada casualmente ou com intuitos criminosos ou ainda em legítima defesa.
Pessôa, que succumbe a uma desgraça.
Pessôa, que soffre um infortúnio.
Tudo que soffre qualquer damno.
\section{Victimador}
\begin{itemize}
\item {Grp. gram.:m.}
\end{itemize}
\begin{itemize}
\item {Proveniência:(De \textunderscore victimar\textunderscore )}
\end{itemize}
O mesmo que \textunderscore sacrificador\textunderscore .
\section{Victimar}
\begin{itemize}
\item {Grp. gram.:v. t.}
\end{itemize}
\begin{itemize}
\item {Proveniência:(Lat. \textunderscore victimare\textunderscore )}
\end{itemize}
Tornar victima; sacrificar.
Damnificar.
\section{Victimário}
\begin{itemize}
\item {Grp. gram.:m.}
\end{itemize}
\begin{itemize}
\item {Grp. gram.:Adj.}
\end{itemize}
\begin{itemize}
\item {Proveniência:(Lat. \textunderscore victimarius\textunderscore )}
\end{itemize}
Ministro dos sacrifícios, entre os antigos.
Sacerdote, que immolava as víctimas.
Sacrificador.
Relativo a víctima.
\section{Víctor-feição}
\begin{itemize}
\item {Grp. gram.:m.}
\end{itemize}
Bom humor:«\textunderscore o sr. doutor leva isto de víctor-feição; mas eu agouro desgraças.\textunderscore »Camillo, \textunderscore Caveira\textunderscore , 280.
\section{Victòrhuguesco}
\begin{itemize}
\item {Grp. gram.:adj.}
\end{itemize}
\begin{itemize}
\item {Utilização:Neol.}
\end{itemize}
Relativo a Víctor Hugo.
Semelhante ao estílo de Víctor Hugo.
\section{Victória}
\begin{itemize}
\item {Grp. gram.:f.}
\end{itemize}
\begin{itemize}
\item {Utilização:Fig.}
\end{itemize}
\begin{itemize}
\item {Utilização:Pop.}
\end{itemize}
\begin{itemize}
\item {Proveniência:(Lat. \textunderscore victoria\textunderscore )}
\end{itemize}
Acto ou effeito de vencer o inimigo numa batalha.
Triúmpho.
Espécie de carruagem moderna. Cf. Camillo, \textunderscore Mulhér Fatal\textunderscore , 152.
Vantagem; bom êxito.
Soberano, moéda inglesa.
\section{Victoriar}
\begin{itemize}
\item {Grp. gram.:v. t.}
\end{itemize}
\begin{itemize}
\item {Proveniência:(De \textunderscore victória\textunderscore )}
\end{itemize}
Applaudir estrepitosamente; acclamar; saudar com enthusiasmo.
\section{Victória-régia}
\begin{itemize}
\item {Grp. gram.:f.}
\end{itemize}
Planta nympheácea da América.
\section{Victória-regina}
\begin{itemize}
\item {Grp. gram.:f.}
\end{itemize}
O mesmo que \textunderscore victória-régia\textunderscore .
\section{Victoriosamente}
\begin{itemize}
\item {Grp. gram.:adv.}
\end{itemize}
De modo victorioso; triumphantemente.
\section{Victorioso}
\begin{itemize}
\item {Grp. gram.:adj.}
\end{itemize}
\begin{itemize}
\item {Proveniência:(Lat. \textunderscore victoriosus\textunderscore )}
\end{itemize}
Que conseguiu victória; que triumphou.
\section{Víctor-sério!}
\begin{itemize}
\item {Grp. gram.:interj.}
\end{itemize}
Menos isso! mudemos de conversa! Cf. Castilho, \textunderscore Collóq. Ald.\textunderscore , 124; Garção, II, 177.
\section{Victrice}
\begin{itemize}
\item {Grp. gram.:f.  e  adj.}
\end{itemize}
\begin{itemize}
\item {Utilização:Poét.}
\end{itemize}
\begin{itemize}
\item {Proveniência:(Lat. \textunderscore victrix\textunderscore )}
\end{itemize}
Aquella que obteve victória; vencedora.
\section{Vicuíba}
\begin{itemize}
\item {Grp. gram.:f.}
\end{itemize}
(V.bicuíba)
\section{Vicunha}
\begin{itemize}
\item {Grp. gram.:f.}
\end{itemize}
Quadrúpede peruano, que produz lan finíssima.
Tecido, feito dessa lan.
(Cast. \textunderscore vicuña\textunderscore )
\section{Vida}
\begin{itemize}
\item {Grp. gram.:f.}
\end{itemize}
\begin{itemize}
\item {Grp. gram.:Loc. adv.}
\end{itemize}
\begin{itemize}
\item {Proveniência:(Do lat. \textunderscore vita\textunderscore )}
\end{itemize}
Estado de actividade, commum ás plantas e aos animaes.
Existência humana.
Tempo, que decorre entre o nascimento e a morte.
Modo de viver: \textunderscore levar bôa vida\textunderscore .
A existência além da morte.
Princípio de existência e de fôrça.
Aquillo que nas composições literária e artísticas é como a vida num corpo.
Expressão viva, animada: \textunderscore aquella pequena não tem vida\textunderscore .
Agitação.
Vitalidade.
Parte determinada de uma existência.
Subsistência.
Sustentáculo.
Origem.
Antigo tributo, também conhecido por \textunderscore parada\textunderscore .
\textunderscore Vida airada\textunderscore , viver de estróina, vagabundagem: \textunderscore gado da vida\textunderscore , gado de alfeire.
\textunderscore Má vida\textunderscore , prostituição.
\textunderscore Mulhér da vida\textunderscore , meretriz.
\textunderscore Á bôa vida\textunderscore , na ociosidade; sem trabalhar; á tuna.
\section{Vidal}
\begin{itemize}
\item {Grp. gram.:adj.}
\end{itemize}
\begin{itemize}
\item {Utilização:Ant.}
\end{itemize}
\begin{itemize}
\item {Proveniência:(De \textunderscore vida\textunderscore )}
\end{itemize}
O mesmo que \textunderscore vital\textunderscore ^1.
\section{Vidama}
\begin{itemize}
\item {Grp. gram.:m.}
\end{itemize}
\begin{itemize}
\item {Proveniência:(Fr. \textunderscore vidame\textunderscore )}
\end{itemize}
Indivíduo, que governava temporalmente terras de um bispado, ou que as possuía como feudo hereditário.
\section{Vidamia}
\begin{itemize}
\item {Grp. gram.:f.}
\end{itemize}
Dignidade de vidama.
\section{Vidar}
\begin{itemize}
\item {Grp. gram.:v. t.}
\end{itemize}
\begin{itemize}
\item {Proveniência:(De \textunderscore vide\textunderscore )}
\end{itemize}
Plantar vides ou vinha em; plantar (vinha).
\section{Vidar}
\begin{itemize}
\item {Grp. gram.:m.}
\end{itemize}
\begin{itemize}
\item {Proveniência:(Do fr. \textunderscore vider\textunderscore )}
\end{itemize}
Instrumento, com que se formavam os dentes dos pentes.
\section{Vide}
\begin{itemize}
\item {Grp. gram.:f.}
\end{itemize}
\begin{itemize}
\item {Proveniência:(Do lat. \textunderscore vitis\textunderscore )}
\end{itemize}
Braço ou vara de videira.
Bacêllo; videira.
Cordão umbilical.
\section{Videar}
\begin{itemize}
\item {Grp. gram.:v. t.}
\end{itemize}
O mesmo que \textunderscore vidar\textunderscore ^1.
\section{Videira}
\begin{itemize}
\item {Grp. gram.:f.}
\end{itemize}
\begin{itemize}
\item {Proveniência:(De \textunderscore vide\textunderscore )}
\end{itemize}
Arbusto sarmentoso, da fam. das ampelídeas.
Cepa.
\section{Videirinho}
\begin{itemize}
\item {Grp. gram.:m.  e  adj.}
\end{itemize}
O mesmo que \textunderscore videiro\textunderscore .
O que, para chegar aos seus fins, não olha aos meios nem hesita em commeter baixezas.
\section{Videiro}
\begin{itemize}
\item {Grp. gram.:m.  e  adj.}
\end{itemize}
\begin{itemize}
\item {Utilização:Pop.}
\end{itemize}
\begin{itemize}
\item {Proveniência:(De \textunderscore vida\textunderscore )}
\end{itemize}
O que trata cuidadosamente da sua vida ou dos seus interesses; fura-vidas.
\section{Vidência}
\begin{itemize}
\item {Grp. gram.:f.}
\end{itemize}
Qualidade de quem é vidente.
\section{Vidente}
\begin{itemize}
\item {Grp. gram.:m.  e  adj.}
\end{itemize}
\begin{itemize}
\item {Grp. gram.:M.  e  f.}
\end{itemize}
\begin{itemize}
\item {Proveniência:(Lat. \textunderscore videns\textunderscore )}
\end{itemize}
Pessôa, que vê o que não existe ou o que está para existir.
Pessôa, que prophetiza.
Pessôa perspicaz.
Pessôa, que faz uso da vista, (ao contrário dos cegos): \textunderscore naquelle asilo, havia oito cegos e quatro videntes...\textunderscore 
\section{Videntro}
\begin{itemize}
\item {Grp. gram.:adj.}
\end{itemize}
(Metáth. de \textunderscore vidrento\textunderscore ). Cf. \textunderscore Aulegrafia\textunderscore , XVI.
\section{Vidiano}
\begin{itemize}
\item {Grp. gram.:adj.}
\end{itemize}
\begin{itemize}
\item {Utilização:Anat.}
\end{itemize}
\begin{itemize}
\item {Proveniência:(De \textunderscore Vidius\textunderscore , n. p. lat. de \textunderscore Guidi\textunderscore , anatómico italiano)}
\end{itemize}
Diz-se de um canal, que atravessa a base das apóphyses pterigoídeas do esphenóide.
Diz-se de certos órgãos, relacionados com êsse canal.
\section{Vido}
\begin{itemize}
\item {Grp. gram.:m.}
\end{itemize}
\begin{itemize}
\item {Utilização:P. us.}
\end{itemize}
\begin{itemize}
\item {Proveniência:(Do lat. \textunderscore betulum\textunderscore )}
\end{itemize}
O mesmo que \textunderscore vidoeiro\textunderscore .
\section{Vidoeiro}
\begin{itemize}
\item {Grp. gram.:m.}
\end{itemize}
\begin{itemize}
\item {Proveniência:(Do lat. hyp. \textunderscore betularius\textunderscore )}
\end{itemize}
O mesmo que \textunderscore bétula\textunderscore .
\section{Vidoiro}
\begin{itemize}
\item {Grp. gram.:m.}
\end{itemize}
\begin{itemize}
\item {Utilização:Prov.}
\end{itemize}
\begin{itemize}
\item {Utilização:minh.}
\end{itemize}
\begin{itemize}
\item {Proveniência:(De \textunderscore vide\textunderscore )}
\end{itemize}
Viveiro de vides.
\section{Vidonha}
\begin{itemize}
\item {Grp. gram.:f.}
\end{itemize}
\begin{itemize}
\item {Utilização:Prov.}
\end{itemize}
\begin{itemize}
\item {Utilização:alent.}
\end{itemize}
\begin{itemize}
\item {Utilização:Ext.}
\end{itemize}
\begin{itemize}
\item {Proveniência:(De \textunderscore vide\textunderscore )}
\end{itemize}
Designação genérica das diversas castas e qualidades de uvas ou videiras.
Uva.
Qualquer espécie de oliveira.
\section{Vidonho}
\begin{itemize}
\item {Grp. gram.:m.}
\end{itemize}
\begin{itemize}
\item {Utilização:Prov.}
\end{itemize}
\begin{itemize}
\item {Utilização:alg.}
\end{itemize}
\begin{itemize}
\item {Utilização:Prov.}
\end{itemize}
\begin{itemize}
\item {Utilização:alg.}
\end{itemize}
\begin{itemize}
\item {Utilização:deprec.}
\end{itemize}
\begin{itemize}
\item {Utilização:fig.}
\end{itemize}
Vide cortada, mas que traz consigo um pedaço da cepa.
Variedade de uva.
Natureza do indivíduo.
(Cp. cast. \textunderscore veduño\textunderscore )
\section{Vidouro}
\begin{itemize}
\item {Grp. gram.:m.}
\end{itemize}
\begin{itemize}
\item {Utilização:Prov.}
\end{itemize}
\begin{itemize}
\item {Utilização:minh.}
\end{itemize}
\begin{itemize}
\item {Proveniência:(De \textunderscore vide\textunderscore )}
\end{itemize}
Viveiro de vides.
\section{Vidraça}
\begin{itemize}
\item {Grp. gram.:f.}
\end{itemize}
\begin{itemize}
\item {Utilização:Prov.}
\end{itemize}
\begin{itemize}
\item {Proveniência:(De \textunderscore vidro\textunderscore )}
\end{itemize}
Lâmina de vidro.
Caixilho ou caixilhos com vidro para janela ou porta.
Lâmina de gêlo, caramelo. (Colhido na Bairrada)
\section{Vidraçaria}
\begin{itemize}
\item {Grp. gram.:f.}
\end{itemize}
Conjunto de vidraças.
Estabelecimento, onde se vendem vidros.
\section{Vidraceiro}
\begin{itemize}
\item {Grp. gram.:m.}
\end{itemize}
\begin{itemize}
\item {Proveniência:(De \textunderscore vidraça\textunderscore )}
\end{itemize}
Aquelle que trabalha em fábrica de vidros.
Aquelle que vende vidros.
Aquelle que os colloca em caixilhos.
\section{Vidracento}
\begin{itemize}
\item {Grp. gram.:adj.}
\end{itemize}
Que tem aspecto de vidraça:«\textunderscore gêlo vidracento\textunderscore ». Camillo, \textunderscore Myst. de Lisb.\textunderscore , I, 137.
\section{Vidraço}
\begin{itemize}
\item {Grp. gram.:m.}
\end{itemize}
\begin{itemize}
\item {Proveniência:(De \textunderscore vidro\textunderscore )}
\end{itemize}
Pedra branca, semelhante ao vidro.
\section{Vidrado}
\begin{itemize}
\item {Grp. gram.:adj.}
\end{itemize}
\begin{itemize}
\item {Proveniência:(De \textunderscore vidrar\textunderscore )}
\end{itemize}
Revestido de substância vitrificável: \textunderscore loiça vidrada\textunderscore .
Embaciado, sem brilho: \textunderscore olhos vidrados\textunderscore .
\section{Vidrador}
\begin{itemize}
\item {Grp. gram.:m.}
\end{itemize}
\begin{itemize}
\item {Proveniência:(De \textunderscore vidrar\textunderscore )}
\end{itemize}
Operário, que reveste artefactos de substância vitrificável.
\section{Vidragem}
\begin{itemize}
\item {Grp. gram.:f.}
\end{itemize}
Acto ou operação de vidrar.
\section{Vidral}
\begin{itemize}
\item {Grp. gram.:m.}
\end{itemize}
O mesmo ou melhór que \textunderscore vitral\textunderscore .
\section{Vidrão}
\begin{itemize}
\item {Grp. gram.:m.}
\end{itemize}
Peixe labróide, de lombo azul e ventre amarelo doirado.
\section{Vidrar}
\begin{itemize}
\item {Grp. gram.:v. t.}
\end{itemize}
\begin{itemize}
\item {Utilização:Fig.}
\end{itemize}
\begin{itemize}
\item {Grp. gram.:V. p.}
\end{itemize}
\begin{itemize}
\item {Proveniência:(De \textunderscore vidro\textunderscore )}
\end{itemize}
Cobrir ou revestir de substância vitrificável.
Embaciar, fazer perder o brilho a.
Embaciar-se, perder o brilho, (falando-se especialmente dos olhos).
\section{Vidraria}
\begin{itemize}
\item {Grp. gram.:f.}
\end{itemize}
Fábrica de vidros.
Estabelecimento, onde se vendem vidros.
Commércio de vidros.
Arte de fabricar vidros.
Porção de vidros.
\section{Vidrecome}
\begin{itemize}
\item {Grp. gram.:m.}
\end{itemize}
Copo grande, cuja fabricação remonta ao século XVI, e de que os Alemães se servem nos banquetes de mais ceremónia.
(Cast. \textunderscore vidrecome\textunderscore )
\section{Vidreiro}
\begin{itemize}
\item {Grp. gram.:m.}
\end{itemize}
\begin{itemize}
\item {Grp. gram.:Adj.}
\end{itemize}
\begin{itemize}
\item {Proveniência:(Do lat. \textunderscore vitrarius\textunderscore )}
\end{itemize}
Aquelle que trabalha em vidro.
Relativo á indústria dos vidreiros: \textunderscore Companhia vidreira\textunderscore ; \textunderscore operários vidreiros\textunderscore .
\section{Vidrento}
\begin{itemize}
\item {Grp. gram.:adj.}
\end{itemize}
\begin{itemize}
\item {Utilização:Fig.}
\end{itemize}
\begin{itemize}
\item {Proveniência:(De \textunderscore vidro\textunderscore )}
\end{itemize}
Semelhante ao vidro.
Vidrado.
Quebradiço.
Agastadiço.
\section{Vidrilho}
\begin{itemize}
\item {Grp. gram.:m.}
\end{itemize}
\begin{itemize}
\item {Grp. gram.:Pl.}
\end{itemize}
\begin{itemize}
\item {Proveniência:(De \textunderscore vidro\textunderscore )}
\end{itemize}
Cada um dos pequenos tubos de vidro ou de uma massa análoga, que, enfiados á maneira de contas, servem para ornatos e bordados.
O mesmo que \textunderscore avelórios\textunderscore .
\section{Vidrino}
\begin{itemize}
\item {Grp. gram.:adj.}
\end{itemize}
Feito de vidro; vidrento.
\section{Vidro}
\begin{itemize}
\item {Grp. gram.:m.}
\end{itemize}
\begin{itemize}
\item {Utilização:Fig.}
\end{itemize}
\begin{itemize}
\item {Proveniência:(Do lat. \textunderscore vitrum\textunderscore )}
\end{itemize}
Corpo sólido, transparente, duro e frágil, obtido pela fusão da areia com potassa ou soda.
Qualquer objecto, feito dessa substância.
Frasco; garrafa pequena.
Lâmina de vidro, com que se resguarda um desenho ou uma estampa, ou com que se preenche um caixilho de porta ou de janela.
Coisa quebradiça.
Pessôa muito susceptível ou melindrosa.
\section{Vidroso}
\begin{itemize}
\item {Grp. gram.:adj.}
\end{itemize}
\begin{itemize}
\item {Utilização:Prov.}
\end{itemize}
\begin{itemize}
\item {Utilização:trasm.}
\end{itemize}
\begin{itemize}
\item {Proveniência:(Do lat. \textunderscore vitrosus\textunderscore )}
\end{itemize}
O mesmo que \textunderscore vidrento\textunderscore .
Muito delicado ou melindroso, (falando-se de pessôas).
\section{Vidual}
\begin{itemize}
\item {Grp. gram.:adj.}
\end{itemize}
\begin{itemize}
\item {Proveniência:(Lat. \textunderscore vidualis\textunderscore )}
\end{itemize}
Relativo a viuvez ou a pessôa viúva.
\section{Vieira}
\begin{itemize}
\item {Grp. gram.:f.}
\end{itemize}
\begin{itemize}
\item {Utilização:Heráld.}
\end{itemize}
\begin{itemize}
\item {Utilização:Bras}
\end{itemize}
\begin{itemize}
\item {Proveniência:(Do lat. \textunderscore viaria\textunderscore )}
\end{itemize}
Mollusco acéphalo.
Concha dêsses molluscos.
Ornato, em fórma de concha.
Árvore silvestre.
\section{Vieirense}
\begin{itemize}
\item {Grp. gram.:adj.}
\end{itemize}
Relativo ao escritor António Vieira ou ao seu estílo.
\section{Vieiro}
\begin{itemize}
\item {Grp. gram.:m.}
\end{itemize}
Veio de metal, filão.
Antigo imposto, que se pagava á Corôa e consistia na têrça parte do oiro, prata ou cobre que se extrahia das minas em Portugal.
Linha, por onde uma pedra se fende, quando se lhe bate.
(Por \textunderscore veeiro\textunderscore , do lat. \textunderscore venarius\textunderscore )
\section{Viela}
\begin{itemize}
\item {Grp. gram.:f.}
\end{itemize}
\begin{itemize}
\item {Utilização:Prov.}
\end{itemize}
\begin{itemize}
\item {Utilização:alent.}
\end{itemize}
\begin{itemize}
\item {Proveniência:(Do fr. \textunderscore bielle\textunderscore )}
\end{itemize}
Cada um dos ferros com argolas, no rodízio dos moínhos.
Argola, que prende a cabeça do arado á garganta.
\section{Viela}
\begin{itemize}
\item {Grp. gram.:f.}
\end{itemize}
\begin{itemize}
\item {Proveniência:(De \textunderscore via\textunderscore )}
\end{itemize}
Rua estreita; quelha.
\section{Vielo}
\begin{itemize}
\item {Grp. gram.:m.}
\end{itemize}
Planta herbácea, alimentícia, da África, (\textunderscore voandzeia subterranea\textunderscore , Thouars).
\section{Vienense}
\begin{itemize}
\item {Grp. gram.:adj.}
\end{itemize}
\begin{itemize}
\item {Grp. gram.:M.}
\end{itemize}
Relativo a Viena.
Habitante de Viena.
\section{Viennense}
\begin{itemize}
\item {Grp. gram.:adj.}
\end{itemize}
\begin{itemize}
\item {Grp. gram.:M.}
\end{itemize}
Relativo a Vienna.
Habitante de Vienna.
\section{Vierina}
\begin{itemize}
\item {Grp. gram.:f.}
\end{itemize}
\begin{itemize}
\item {Utilização:Bras}
\end{itemize}
Planta cinchonácea medicinal.
\section{Viés}
\begin{itemize}
\item {Grp. gram.:m.}
\end{itemize}
\begin{itemize}
\item {Proveniência:(Do fr. \textunderscore biais\textunderscore )}
\end{itemize}
Direcção oblíqua.
Tira estreita de pano, cortada obliquamente ou no sentido diagonal da peça.
\section{Viéz}
\begin{itemize}
\item {Grp. gram.:m.}
\end{itemize}
\begin{itemize}
\item {Proveniência:(Do fr. \textunderscore biais\textunderscore )}
\end{itemize}
Direcção oblíqua.
Tira estreita de pano, cortada obliquamente ou no sentido diagonal da peça.
\section{Viga}
\begin{itemize}
\item {Grp. gram.:f.}
\end{itemize}
Madeiro grosso, preparado para construcções; trave.
(Cast. \textunderscore viga\textunderscore )
\section{Vigada}
\begin{itemize}
\item {Grp. gram.:f.}
\end{itemize}
\begin{itemize}
\item {Utilização:Ant.}
\end{itemize}
\begin{itemize}
\item {Proveniência:(Do lat. \textunderscore viscis\textunderscore )}
\end{itemize}
O mesmo ou melhór que \textunderscore vegada\textunderscore . Cf. Frei Fortun., \textunderscore Inéd.\textunderscore , 316.
\section{Vigairada}
\begin{itemize}
\item {Grp. gram.:f.}
\end{itemize}
\begin{itemize}
\item {Utilização:Prov.}
\end{itemize}
\begin{itemize}
\item {Utilização:minh.}
\end{itemize}
\begin{itemize}
\item {Proveniência:(De \textunderscore vigairo\textunderscore )}
\end{itemize}
Jornada ligeira.
Visita apressada.
Corrida, de um lado para outro.
\textunderscore Andar de vigairada\textunderscore , andar ligeiro; demorar-se pouco em qualquer parte.
\section{Vigairaria}
\begin{itemize}
\item {Grp. gram.:f.}
\end{itemize}
\begin{itemize}
\item {Proveniência:(De \textunderscore vigairo\textunderscore )}
\end{itemize}
Cargo ou dignidade de vigário.
\section{Vigairo}
\begin{itemize}
\item {Grp. gram.:m.}
\end{itemize}
(Fórma pop. e ant. de \textunderscore vigário\textunderscore )
\section{Vigamento}
\begin{itemize}
\item {Grp. gram.:m.}
\end{itemize}
\begin{itemize}
\item {Proveniência:(De \textunderscore vigar\textunderscore )}
\end{itemize}
Conjunto das vigas, que fazem parte de uma construcção; travejamento.
\section{Vigar}
\begin{itemize}
\item {Grp. gram.:v. t.}
\end{itemize}
Pôr vigas em; pôr sôbre vigas.
\section{Vigàraria}
\begin{itemize}
\item {Grp. gram.:f.}
\end{itemize}
(V.vigairaria):«\textunderscore ...eu pastoreava uma vigararia...\textunderscore »Camillo, \textunderscore Voltareis, ó Christo\textunderscore ?, 9.
\section{Vigária}
\begin{itemize}
\item {Grp. gram.:f.}
\end{itemize}
\begin{itemize}
\item {Proveniência:(Do lat. \textunderscore vicaria\textunderscore )}
\end{itemize}
Freira, que fazia as vezes de superiora.
\section{Vigário}
\begin{itemize}
\item {Grp. gram.:m.}
\end{itemize}
\begin{itemize}
\item {Utilização:T. da Grândola}
\end{itemize}
\begin{itemize}
\item {Proveniência:(Do lat. \textunderscore vicarius\textunderscore )}
\end{itemize}
Aquelle que faz as vezes de outro.
Cabreiro da serra.
Padre, que faz as vezes do Prelado.
Titulo do párocho em algumas freguesias.
\textunderscore Vigário da vara\textunderscore , delegado do Prelado diocesano, nalgumas povoações de uma diocese.
\textunderscore Conto do vigário\textunderscore , passagem de moéda falsa; burla da venda de objectos de metal ordinário, como se fôssem de metal precioso.
\section{Vigarista}
\begin{itemize}
\item {Grp. gram.:m.}
\end{itemize}
\begin{itemize}
\item {Proveniência:(De \textunderscore vigário\textunderscore )}
\end{itemize}
Burlão, que transmitte com interesse objectos de metal ordinário, ou outros objectos, como se tivessem grande valor.
Passador de moéda falsa.
\section{Vigência}
\begin{itemize}
\item {Grp. gram.:f.}
\end{itemize}
Qualidade daquillo que é vigente: \textunderscore na vigência da Carta Constitucional...\textunderscore 
\section{Vigenel}
\begin{itemize}
\item {Grp. gram.:m.}
\end{itemize}
Espécie de jôgo popular.
\section{Vigente}
\begin{itemize}
\item {Grp. gram.:adj.}
\end{itemize}
\begin{itemize}
\item {Proveniência:(Lat. \textunderscore vigens\textunderscore )}
\end{itemize}
Que vige; que está em vigor.
\section{Viger}
\begin{itemize}
\item {Grp. gram.:v. i.}
\end{itemize}
\begin{itemize}
\item {Proveniência:(Lat. \textunderscore vigere\textunderscore )}
\end{itemize}
Têr vigor.
Estar em vigor, estar em execução.
\section{Vigésimo}
\begin{itemize}
\item {Grp. gram.:adj.}
\end{itemize}
\begin{itemize}
\item {Grp. gram.:M.}
\end{itemize}
\begin{itemize}
\item {Proveniência:(Lat. \textunderscore vigesimus\textunderscore )}
\end{itemize}
Que numa série de vinte occupa o último lugar.
Cada uma das vinte partes em que se divide um todo.
\section{Vigia}
\begin{itemize}
\item {Grp. gram.:f.}
\end{itemize}
\begin{itemize}
\item {Utilização:Náut.}
\end{itemize}
\begin{itemize}
\item {Grp. gram.:Pl.}
\end{itemize}
\begin{itemize}
\item {Grp. gram.:M.}
\end{itemize}
Acto ou effeito de vigiar.
Estado de quem vigia.
Sentinella.
Vedeta.
Guarita.
Orifício por onde se espreita.
Espécie de janela ou fresta, por onde entra a luz nos camarotes das grandes embarcações.
O mesmo que [[parcéis|parcel]].
Aquelle que vigia.
Guarda; sentinella.
\section{Vigiador}
\begin{itemize}
\item {Grp. gram.:m.  e  adj.}
\end{itemize}
O que vigia.
\section{Vigiante}
\begin{itemize}
\item {Grp. gram.:adj.}
\end{itemize}
O mesmo que \textunderscore vigilante\textunderscore .
\section{Vigiar}
\begin{itemize}
\item {Grp. gram.:v. i.}
\end{itemize}
\begin{itemize}
\item {Grp. gram.:V. t.}
\end{itemize}
\begin{itemize}
\item {Utilização:Bras. de Minas}
\end{itemize}
\begin{itemize}
\item {Proveniência:(Do lat. \textunderscore vigilare\textunderscore )}
\end{itemize}
Estar acordado.
Estar de sentinella.
Tomar cuidado.
Estar attento.
Observar attentamente; espreitar: \textunderscore vigiar os passos de alguém\textunderscore .
Estar attento a; velar por.
O mesmo que \textunderscore apromptar\textunderscore : \textunderscore Maria, vigia o café\textunderscore .
\section{Vigieiro}
\begin{itemize}
\item {Grp. gram.:m.}
\end{itemize}
\begin{itemize}
\item {Utilização:Des.}
\end{itemize}
\begin{itemize}
\item {Proveniência:(De \textunderscore vigia\textunderscore )}
\end{itemize}
Guarda campestre.
\section{Vígil}
\begin{itemize}
\item {Grp. gram.:adj.}
\end{itemize}
\begin{itemize}
\item {Proveniência:(Lat. \textunderscore vigil\textunderscore )}
\end{itemize}
Acordado; que está velando; que vigia.
\section{Vigilador}
\begin{itemize}
\item {Grp. gram.:m.  e  adj.}
\end{itemize}
\begin{itemize}
\item {Proveniência:(De \textunderscore vigilar\textunderscore )}
\end{itemize}
O que faz vigília, o que vela. Cf. Júl. Dinis, \textunderscore Serões\textunderscore , 113.
\section{Vigilância}
\begin{itemize}
\item {Grp. gram.:f.}
\end{itemize}
\begin{itemize}
\item {Proveniência:(Lat. \textunderscore vigilantia\textunderscore )}
\end{itemize}
Acto ou effeito de vigilar.
Precaução.
Diligência.
\section{Vigilante}
\begin{itemize}
\item {Grp. gram.:adj.}
\end{itemize}
\begin{itemize}
\item {Grp. gram.:M.}
\end{itemize}
\begin{itemize}
\item {Proveniência:(Lat. \textunderscore vigilans\textunderscore )}
\end{itemize}
Que vigila.
Diligente.
Cauteloso.
Attento.
Indivíduo vigilante.
Aquelle que, nas lojas maçónicas, desempenha as funcções de vigia ou sentinella.
\section{Vigilantemente}
\begin{itemize}
\item {Grp. gram.:adv.}
\end{itemize}
De modo vigilante.
\section{Vigilar}
\begin{itemize}
\item {Grp. gram.:v. t.  e  i.}
\end{itemize}
O mesmo que \textunderscore vigiar\textunderscore .
\section{Vigilenga}
\begin{itemize}
\item {Grp. gram.:f.}
\end{itemize}
\begin{itemize}
\item {Utilização:Bras. do N}
\end{itemize}
\begin{itemize}
\item {Proveniência:(De \textunderscore Vigia\textunderscore , n. p.)}
\end{itemize}
Espécie de embarcação, armada como hiate.
\section{Vigília}
\begin{itemize}
\item {Grp. gram.:f.}
\end{itemize}
\begin{itemize}
\item {Utilização:Prov.}
\end{itemize}
\begin{itemize}
\item {Utilização:alg.}
\end{itemize}
\begin{itemize}
\item {Utilização:Ant.}
\end{itemize}
\begin{itemize}
\item {Proveniência:(Lat. \textunderscore vigilia\textunderscore )}
\end{itemize}
Insómnia; privação do somno durante a noite.
Lucubração.
Desvelo, cuidado.
Véspera de festa.
Arraial, festa campestre.
Quarto da noite.
Offício de defuntos.
\section{Vigilinga}
\begin{itemize}
\item {Grp. gram.:f.}
\end{itemize}
(V.vigilenga)
\section{Vigintivirado}
\begin{itemize}
\item {Grp. gram.:m.}
\end{itemize}
\begin{itemize}
\item {Proveniência:(Lat. \textunderscore vigintiviratus\textunderscore )}
\end{itemize}
Cargo ou dignidade de vigintíviro.
\section{Vigintivirato}
\begin{itemize}
\item {Grp. gram.:m.}
\end{itemize}
\begin{itemize}
\item {Proveniência:(Lat. \textunderscore vigintiviratus\textunderscore )}
\end{itemize}
Cargo ou dignidade de vigintíviro.
\section{Vigintíviro}
\begin{itemize}
\item {Grp. gram.:m.}
\end{itemize}
\begin{itemize}
\item {Proveniência:(Lat. \textunderscore vigintivir\textunderscore )}
\end{itemize}
Cada um dos vinte magistrados romanos, metade dos quaes eram adjuntos do pretor, occupando-se a outra metade na cunhagem da moéda, polícia e limpeza das ruas e execução dos criminosos.
Era também o nome dos vinte membros da commissão, que César nomeou para a divisão das terras da Campânia.
\section{Vigna}
\begin{itemize}
\item {Grp. gram.:f.}
\end{itemize}
\begin{itemize}
\item {Proveniência:(De \textunderscore Vigna\textunderscore , n. p.)}
\end{itemize}
Gênero de plantas leguminosas.
\section{Vigonho}
\begin{itemize}
\item {Grp. gram.:m.}
\end{itemize}
O mesmo que \textunderscore vicunha\textunderscore .
\section{Vigor}
\begin{itemize}
\item {Grp. gram.:m.}
\end{itemize}
\begin{itemize}
\item {Proveniência:(Lat. \textunderscore vigor\textunderscore )}
\end{itemize}
Fôrça, robustez.
Actividade.
Efficácia.
Valor.
\section{Vigorante}
\begin{itemize}
\item {Grp. gram.:adj.}
\end{itemize}
Que vigora.
\section{Vigorar}
\begin{itemize}
\item {Grp. gram.:v. t.}
\end{itemize}
\begin{itemize}
\item {Grp. gram.:V. i.}
\end{itemize}
Dar vigor a.
Fortalecer.
Tornar mais enérgico.
Adquirir fôrça ou robustez.
Têr vigor.
Estar em vigor ou não estar abrogado ou prescrito.
\section{Vigorite}
\begin{itemize}
\item {Grp. gram.:f.}
\end{itemize}
\begin{itemize}
\item {Proveniência:(De \textunderscore vigor\textunderscore )}
\end{itemize}
Pólvora muito explosiva.
\section{Vigorizar}
\begin{itemize}
\item {Grp. gram.:v. t.}
\end{itemize}
O mesmo que \textunderscore vigorar\textunderscore .
\section{Vigorosamente}
\begin{itemize}
\item {Grp. gram.:adv.}
\end{itemize}
De modo vigoroso.
Com vigor.
Violentamente.
\section{Vigoroso}
\begin{itemize}
\item {Grp. gram.:adj.}
\end{itemize}
\begin{itemize}
\item {Proveniência:(Lat. \textunderscore vigorosus\textunderscore )}
\end{itemize}
Que tem vigor.
Forte; robusto.
Enérgico.
Que tem expressão viva.
Bem accentuado: \textunderscore traços vigorosos\textunderscore .
\section{Vigota}
\begin{itemize}
\item {Grp. gram.:f.}
\end{itemize}
Pequena viga; sarrafo.
\section{Vigote}
\begin{itemize}
\item {Grp. gram.:m.}
\end{itemize}
O mesmo que \textunderscore vigota\textunderscore .
\section{Vil}
\begin{itemize}
\item {Grp. gram.:adj.}
\end{itemize}
\begin{itemize}
\item {Utilização:T. do Fundão}
\end{itemize}
\begin{itemize}
\item {Grp. gram.:M.  e  f.}
\end{itemize}
\begin{itemize}
\item {Proveniência:(Do lat. \textunderscore vilis\textunderscore )}
\end{itemize}
Que se compra por baixo preço.
Que é de pouco valor: \textunderscore um trapo vil\textunderscore .
Ordinário, reles.
Mesquinho, miserável.
Desprezível; infame: \textunderscore homem vil\textunderscore .
Travesso, inquieto, muito vivo.
Pessôa desprezível.
\section{Vila}
\begin{itemize}
\item {Grp. gram.:f.}
\end{itemize}
\begin{itemize}
\item {Utilização:Ext.}
\end{itemize}
\begin{itemize}
\item {Utilização:Prov.}
\end{itemize}
\begin{itemize}
\item {Utilização:alg.}
\end{itemize}
\begin{itemize}
\item {Proveniência:(Lat. \textunderscore villa\textunderscore )}
\end{itemize}
Povoação, de categoria inferior á de cidade e superior á de aldeia.
Casa de campo ou habitação de recreio, nos arrabaldes das cidades italianas.
Casa de campo, de construcção elegante ou mais ou menos caprichosa.
Quinta, com casa de habitação.
Casa de habitação com jardim, dentro da cidade.
Fiada ou camada de amêijoas a assar.
\section{Vilã}
\begin{itemize}
\item {Grp. gram.:f.}
\end{itemize}
(Flexão, fem. de \textunderscore vilão\textunderscore )
\section{Vila-diogo}
\begin{itemize}
\item {Grp. gram.:f.}
\end{itemize}
(só us. na loc. \textunderscore dar ás de vila-diogo\textunderscore , fugir). Cf. G. Viana, \textunderscore Apostilas\textunderscore .
\section{Vilado}
\begin{itemize}
\item {Grp. gram.:adj.}
\end{itemize}
\begin{itemize}
\item {Utilização:Ant.}
\end{itemize}
\begin{itemize}
\item {Proveniência:(De \textunderscore vil\textunderscore )}
\end{itemize}
O mesmo que [[envilecido|envilecer]].
\section{Vilafrancada}
\begin{itemize}
\item {Grp. gram.:f.}
\end{itemize}
Movimento político, que partiu de Vila-Franca-de-Xira, no séc. XIX.
\section{Vilagem}
\begin{itemize}
\item {Grp. gram.:f.}
\end{itemize}
\begin{itemize}
\item {Utilização:Ant.}
\end{itemize}
O mesmo que \textunderscore vila\textunderscore .
\section{Vilãmente}
\begin{itemize}
\item {Grp. gram.:adv.}
\end{itemize}
De modo vilão.
Grosseiramente.
\section{Vilan}
\begin{itemize}
\item {Grp. gram.:f.}
\end{itemize}
(Flexão, fem. de \textunderscore villão\textunderscore )
\section{Vilanaço}
\begin{itemize}
\item {Grp. gram.:m.  e  adj.}
\end{itemize}
O mesmo que \textunderscore vilanaz\textunderscore .
\section{Vilanagem}
\begin{itemize}
\item {Grp. gram.:f.}
\end{itemize}
\begin{itemize}
\item {Proveniência:(Do lat. \textunderscore villanus\textunderscore )}
\end{itemize}
Vilania.
Ajuntamento de vilãos.
\section{Vilanaz}
\begin{itemize}
\item {Grp. gram.:m.  e  adj.}
\end{itemize}
\begin{itemize}
\item {Proveniência:(Do lat. \textunderscore villanus\textunderscore )}
\end{itemize}
O que tem, como preponderante, a qualidade de vilão.
\section{Vilancete}
\begin{itemize}
\item {fónica:cê}
\end{itemize}
\begin{itemize}
\item {Grp. gram.:m.}
\end{itemize}
Composição poética, geralmente curta e de carácter campesino.
(Cast. \textunderscore villancete\textunderscore )
\section{Vilancico}
\begin{itemize}
\item {Grp. gram.:m.}
\end{itemize}
Pequena composição poética, que se cantava em festividades religiosas.
(Cast. \textunderscore villancico\textunderscore )
\section{Vilanesco}
\begin{itemize}
\item {fónica:nês}
\end{itemize}
\begin{itemize}
\item {Grp. gram.:adj.}
\end{itemize}
\begin{itemize}
\item {Proveniência:(Do lat. \textunderscore villanus\textunderscore )}
\end{itemize}
Relativo a vilão, ou próprio dêle.
Rude:«\textunderscore suas frautas e vilanescos instrumentos...\textunderscore »Usque.
\section{Vilania}
\begin{itemize}
\item {Grp. gram.:f.}
\end{itemize}
\begin{itemize}
\item {Proveniência:(Do lat. \textunderscore villanus\textunderscore )}
\end{itemize}
Qualidade do que é vilão; vileza.
Vilanagem.
Mesquinhez ou avareza.
\section{Vilanmente}
\begin{itemize}
\item {Grp. gram.:adv.}
\end{itemize}
De modo vilão.
Grosseiramente.
\section{Vilanzete}
\begin{itemize}
\item {fónica:zê}
\end{itemize}
\begin{itemize}
\item {Grp. gram.:m.}
\end{itemize}
\begin{itemize}
\item {Proveniência:(De \textunderscore vilão\textunderscore )}
\end{itemize}
Vilanaz muito ordinário, desprezível:«\textunderscore o vilanzete é calvo.\textunderscore »Chiado, \textunderscore Autos\textunderscore , 39.
\section{Vilão}
\begin{itemize}
\item {Grp. gram.:adj.}
\end{itemize}
\begin{itemize}
\item {Utilização:Fig.}
\end{itemize}
\begin{itemize}
\item {Grp. gram.:M.}
\end{itemize}
\begin{itemize}
\item {Utilização:Açor}
\end{itemize}
\begin{itemize}
\item {Utilização:Prov.}
\end{itemize}
\begin{itemize}
\item {Utilização:trasm.}
\end{itemize}
\begin{itemize}
\item {Proveniência:(Do lat. \textunderscore villanus\textunderscore )}
\end{itemize}
Que habita numa vila.
Rústico.
Plebeu, grosseiro.
Desprezível, abjecto.
Avaro; sórdido.
Habitante de vila.
Camponês.
Indivíduo plebeu.
Avarento.
Homem miserável e desprezível.
Antiga dança popular.
Sujeito mascarado, que entra nas moiriscas, satirizando a gente da terra.
Chouriço, feito da mistura de várias espécies de carne, com sopa triga e pingo de porco.
Pl. \textunderscore Vilãos\textunderscore , mas usa-se \textunderscore vilões\textunderscore .
\section{Vilar}
\begin{itemize}
\item {Grp. gram.:m.}
\end{itemize}
\begin{itemize}
\item {Utilização:Des.}
\end{itemize}
\begin{itemize}
\item {Proveniência:(Lat. \textunderscore villaris\textunderscore )}
\end{itemize}
Pequena aldeia; logarejo.
\section{Vilarinho}
\begin{itemize}
\item {Grp. gram.:m.}
\end{itemize}
Pequeno vilar.
\section{Vilaverde}
\begin{itemize}
\item {Grp. gram.:f.}
\end{itemize}
Variedade de pêra portuguesa.
\section{Vileco}
\begin{itemize}
\item {Grp. gram.:m.}
\end{itemize}
\begin{itemize}
\item {Proveniência:(De \textunderscore vil\textunderscore )}
\end{itemize}
O mesmo que \textunderscore velhaco\textunderscore .
\section{Vilegiatura}
\begin{itemize}
\item {Grp. gram.:f.}
\end{itemize}
\begin{itemize}
\item {Utilização:Neol.}
\end{itemize}
\begin{itemize}
\item {Proveniência:(It. \textunderscore villegiatura\textunderscore )}
\end{itemize}
Temporada, que pessôas da cidade passam no campo ou em digressão de recreio, na estação calmosa.
Digressão recreativa, fóra das grandes povoações ou por estações balneares.
\section{Vilegiaturista}
\begin{itemize}
\item {Grp. gram.:m.}
\end{itemize}
\begin{itemize}
\item {Utilização:Neol.}
\end{itemize}
Aquelle que anda em vilegiatura.
\section{Vilela}
\begin{itemize}
\item {Grp. gram.:f.}
\end{itemize}
Pequena vila.
\section{Vilescer}
\begin{itemize}
\item {Grp. gram.:v. t.}
\end{itemize}
Tornar vil, envilecer. Cf. Camillo, \textunderscore Cav. em Ruínas\textunderscore , 48.
\section{Vileta}
\begin{itemize}
\item {fónica:lê}
\end{itemize}
\begin{itemize}
\item {Grp. gram.:f.}
\end{itemize}
Pequena vila.
\section{Vileu}
\begin{itemize}
\item {Grp. gram.:m.}
\end{itemize}
Espécie de masmorra, na China. Cf. \textunderscore Peregrinação\textunderscore , CXL.
\section{Vileza}
\begin{itemize}
\item {Grp. gram.:f.}
\end{itemize}
Qualidade daquelle ou daquillo que é vil.
Acto vil.
\section{Vilhancete}
\begin{itemize}
\item {fónica:cê}
\end{itemize}
\begin{itemize}
\item {Grp. gram.:m.}
\end{itemize}
O mesmo que \textunderscore villancete\textunderscore .
\section{Vilhancico}
\begin{itemize}
\item {Grp. gram.:m.}
\end{itemize}
O mesmo que \textunderscore villancico\textunderscore .
\section{Vilhanesca}
\begin{itemize}
\item {Grp. gram.:f.}
\end{itemize}
\begin{itemize}
\item {Utilização:Ant.}
\end{itemize}
Poesia pastoril.
(Cast. \textunderscore villanesca\textunderscore )
\section{Viliastro}
\begin{itemize}
\item {Grp. gram.:m.}
\end{itemize}
\begin{itemize}
\item {Utilização:Ant.}
\end{itemize}
Pequeno vilar; aldeola.
(T., mal derivado de \textunderscore vila\textunderscore )
\section{Vilico}
\begin{itemize}
\item {Grp. gram.:m.}
\end{itemize}
\begin{itemize}
\item {Utilização:Ant.}
\end{itemize}
\begin{itemize}
\item {Proveniência:(De \textunderscore vila\textunderscore )}
\end{itemize}
Espécie de regedor de pequena localidade, onde arrecadava os impostos geraes e administrava justiça. Cf. Herculano, \textunderscore Hist. de Port.\textunderscore , IV, 86 e 474; \textunderscore Bobo\textunderscore , 105 e 183.
\section{Vilificar}
\begin{itemize}
\item {Grp. gram.:v. t.}
\end{itemize}
\begin{itemize}
\item {Proveniência:(Do lat. \textunderscore vilis\textunderscore  + \textunderscore facere\textunderscore )}
\end{itemize}
Tornar vil.
\section{Vilipendiador}
\begin{itemize}
\item {Grp. gram.:m.  e  adj.}
\end{itemize}
O que vilipendia.
\section{Vilipendiar}
\begin{itemize}
\item {Grp. gram.:v. t.}
\end{itemize}
Tratar com vilipêndio.
Considerar como vil; desprezar.
\section{Vilipêndio}
\begin{itemize}
\item {Grp. gram.:m.}
\end{itemize}
\begin{itemize}
\item {Proveniência:(Do lat. \textunderscore vilipendere\textunderscore )}
\end{itemize}
Grande desprêzo.
Acto de aviltar.
\section{Vilipendiosamente}
\begin{itemize}
\item {Grp. gram.:adv.}
\end{itemize}
De modo vilipendioso; com vilipêndio.
\section{Vilipendioso}
\begin{itemize}
\item {Grp. gram.:adj.}
\end{itemize}
Em que há vilipêndio.
\section{Villa}
\begin{itemize}
\item {Grp. gram.:f.}
\end{itemize}
\begin{itemize}
\item {Utilização:Ext.}
\end{itemize}
\begin{itemize}
\item {Utilização:Prov.}
\end{itemize}
\begin{itemize}
\item {Utilização:alg.}
\end{itemize}
\begin{itemize}
\item {Proveniência:(Lat. \textunderscore villa\textunderscore )}
\end{itemize}
Povoação, de categoria inferior á de cidade e superior á de aldeia.
Casa de campo ou habitação de recreio, nos arrabaldes das cidades italianas.
Casa de campo, de construcção elegante ou mais ou menos caprichosa.
Quinta, com casa de habitação.
Casa de habitação com jardim, dentro da cidade.
Fiada ou camada de amêijoas a assar.
\section{Villa-diogo}
\begin{itemize}
\item {Grp. gram.:f.}
\end{itemize}
(só us. na loc. \textunderscore dar ás de villa-diogo\textunderscore , fugir). Cf. G. Viana, \textunderscore Apostilas\textunderscore .
\section{Villafrancada}
\begin{itemize}
\item {Grp. gram.:f.}
\end{itemize}
Movimento político, que partiu de Villa-Franca-de-Xira, no séc. XIX.
\section{Villagem}
\begin{itemize}
\item {Grp. gram.:f.}
\end{itemize}
\begin{itemize}
\item {Utilização:Ant.}
\end{itemize}
O mesmo que \textunderscore villa\textunderscore .
\section{Villan}
\begin{itemize}
\item {Grp. gram.:f.}
\end{itemize}
(Flexão, fem. de \textunderscore villão\textunderscore )
\section{Villanaço}
\begin{itemize}
\item {Grp. gram.:m.  e  adj.}
\end{itemize}
O mesmo que \textunderscore villanaz\textunderscore .
\section{Villanagem}
\begin{itemize}
\item {Grp. gram.:f.}
\end{itemize}
\begin{itemize}
\item {Proveniência:(Do lat. \textunderscore villanus\textunderscore )}
\end{itemize}
Villania.
Ajuntamento de villãos.
\section{Villanaz}
\begin{itemize}
\item {Grp. gram.:m.  e  adj.}
\end{itemize}
\begin{itemize}
\item {Proveniência:(Do lat. \textunderscore villanus\textunderscore )}
\end{itemize}
O que tem, como preponderante, a qualidade de villão.
\section{Villancete}
\begin{itemize}
\item {fónica:cê}
\end{itemize}
\begin{itemize}
\item {Grp. gram.:m.}
\end{itemize}
Composição poética, geralmente curta e de carácter campesino.
(Cast. \textunderscore villancete\textunderscore )
\section{Villancico}
\begin{itemize}
\item {Grp. gram.:m.}
\end{itemize}
Pequena composição poética, que se cantava em festividades religiosas.
(Cast. \textunderscore villancico\textunderscore )
\section{Villanesco}
\begin{itemize}
\item {fónica:nês}
\end{itemize}
\begin{itemize}
\item {Grp. gram.:adj.}
\end{itemize}
\begin{itemize}
\item {Proveniência:(Do lat. \textunderscore villanus\textunderscore )}
\end{itemize}
Relativo a villão, ou próprio dêlle.
Rude:«\textunderscore suas frautas e villanescos instrumentos...\textunderscore »Usque.
\section{Villania}
\begin{itemize}
\item {Grp. gram.:f.}
\end{itemize}
\begin{itemize}
\item {Proveniência:(Do lat. \textunderscore villanus\textunderscore )}
\end{itemize}
Qualidade do que é villão; vileza.
Villanagem.
Mesquinhez ou avareza.
\section{Villanmente}
\begin{itemize}
\item {Grp. gram.:adv.}
\end{itemize}
De modo villão.
Grosseiramente.
\section{Villanzete}
\begin{itemize}
\item {fónica:zê}
\end{itemize}
\begin{itemize}
\item {Grp. gram.:m.}
\end{itemize}
\begin{itemize}
\item {Proveniência:(De \textunderscore villão\textunderscore )}
\end{itemize}
Villanaz muito ordinário, desprezível:«\textunderscore o villanzete é calvo.\textunderscore »Chiado, \textunderscore Autos\textunderscore , 39.
\section{Villão}
\begin{itemize}
\item {Grp. gram.:adj.}
\end{itemize}
\begin{itemize}
\item {Utilização:Fig.}
\end{itemize}
\begin{itemize}
\item {Grp. gram.:M.}
\end{itemize}
\begin{itemize}
\item {Utilização:Açor}
\end{itemize}
\begin{itemize}
\item {Utilização:Prov.}
\end{itemize}
\begin{itemize}
\item {Utilização:trasm.}
\end{itemize}
\begin{itemize}
\item {Proveniência:(Do lat. \textunderscore villanus\textunderscore )}
\end{itemize}
Que habita numa villa.
Rústico.
Plebeu, grosseiro.
Desprezível, abjecto.
Avaro; sórdido.
Habitante de villa.
Camponês.
Indivíduo plebeu.
Avarento.
Homem miserável e desprezível.
Antiga dança popular.
Sujeito mascarado, que entra nas moiriscas, satirizando a gente da terra.
Chouriço, feito da mistura de várias espécies de carne, com sopa triga e pingo de porco.
Pl. \textunderscore Villãos\textunderscore , mas usa-se \textunderscore villões\textunderscore .
\section{Villar}
\begin{itemize}
\item {Grp. gram.:m.}
\end{itemize}
\begin{itemize}
\item {Utilização:Des.}
\end{itemize}
\begin{itemize}
\item {Proveniência:(Lat. \textunderscore villaris\textunderscore )}
\end{itemize}
Pequena aldeia; logarejo.
\section{Villarinho}
\begin{itemize}
\item {Grp. gram.:m.}
\end{itemize}
Pequeno villar.
\section{Villaverde}
\begin{itemize}
\item {Grp. gram.:f.}
\end{itemize}
Variedade de pêra portuguesa.
\section{Villegiatura}
\begin{itemize}
\item {Grp. gram.:f.}
\end{itemize}
\begin{itemize}
\item {Utilização:Neol.}
\end{itemize}
\begin{itemize}
\item {Proveniência:(It. \textunderscore villegiatura\textunderscore )}
\end{itemize}
Temporada, que pessôas da cidade passam no campo ou em digressão de recreio, na estação calmosa.
Digressão recreativa, fóra das grandes povoações ou por estações balneares.
\section{Villegiaturista}
\begin{itemize}
\item {Grp. gram.:m.}
\end{itemize}
\begin{itemize}
\item {Utilização:Neol.}
\end{itemize}
Aquelle que anda em villegiatura.
\section{Villela}
\begin{itemize}
\item {Grp. gram.:f.}
\end{itemize}
Pequena villa.
\section{Villeta}
\begin{itemize}
\item {Grp. gram.:f.}
\end{itemize}
Pequena villa.
\section{Villiastro}
\begin{itemize}
\item {Grp. gram.:m.}
\end{itemize}
\begin{itemize}
\item {Utilização:Ant.}
\end{itemize}
Pequeno villar; aldeola.
(T., mal derivado de \textunderscore villa\textunderscore )
\section{Villico}
\begin{itemize}
\item {Grp. gram.:m.}
\end{itemize}
\begin{itemize}
\item {Utilização:Ant.}
\end{itemize}
\begin{itemize}
\item {Proveniência:(De \textunderscore villa\textunderscore )}
\end{itemize}
Espécie de regedor de pequena localidade, onde arrecadava os impostos geraes e administrava justiça. Cf. Herculano, \textunderscore Hist. de Port.\textunderscore , IV, 86 e 474; \textunderscore Bobo\textunderscore , 105 e 183.
\section{Villôa}
\begin{itemize}
\item {Grp. gram.:f.}
\end{itemize}
O mesmo que \textunderscore villan\textunderscore .
\section{Villória}
\begin{itemize}
\item {Grp. gram.:f.}
\end{itemize}
\begin{itemize}
\item {Utilização:Deprec.}
\end{itemize}
Villa pequena e pouco importante.
\section{Villório}
\begin{itemize}
\item {Grp. gram.:m.}
\end{itemize}
O mesmo que \textunderscore villória\textunderscore .
\section{Villosidade}
\begin{itemize}
\item {Grp. gram.:f.}
\end{itemize}
Qualidade do que é villoso.
Lanugem vegetal.
Ajuntamento de saliências filiformes nas mucosas.
\section{Villoso}
\begin{itemize}
\item {Grp. gram.:adj.}
\end{itemize}
\begin{itemize}
\item {Proveniência:(Lat. \textunderscore villosus\textunderscore )}
\end{itemize}
Cheio de pêlos.
Cabelludo.
Hirsuto.
\section{Villota}
\begin{itemize}
\item {Grp. gram.:f.}
\end{itemize}
O mesmo que \textunderscore villeta\textunderscore .
\section{Víllula}
\begin{itemize}
\item {Grp. gram.:f.}
\end{itemize}
\begin{itemize}
\item {Utilização:Ant.}
\end{itemize}
\begin{itemize}
\item {Proveniência:(De \textunderscore villa\textunderscore )}
\end{itemize}
Pequeno prédio, pequena herdade.
Villarinho.
\section{Vilmente}
\begin{itemize}
\item {Grp. gram.:adv.}
\end{itemize}
De modo vil.
\section{Viló}
\begin{itemize}
\item {Grp. gram.:m.}
\end{itemize}
Pequena foice, com que os Índios ceifam o arroz.
(Do marata \textunderscore vilo\textunderscore )
\section{Vilôa}
\begin{itemize}
\item {Grp. gram.:f.}
\end{itemize}
O mesmo que \textunderscore vilan\textunderscore .
\section{Vilória}
\begin{itemize}
\item {Grp. gram.:f.}
\end{itemize}
\begin{itemize}
\item {Utilização:Deprec.}
\end{itemize}
Vila pequena e pouco importante.
\section{Vilório}
\begin{itemize}
\item {Grp. gram.:m.}
\end{itemize}
O mesmo que \textunderscore vilória\textunderscore .
\section{Vilosidade}
\begin{itemize}
\item {Grp. gram.:f.}
\end{itemize}
Qualidade do que é viloso.
Lanugem vegetal.
Ajuntamento de saliências filiformes nas mucosas.
\section{Viloso}
\begin{itemize}
\item {Grp. gram.:adj.}
\end{itemize}
\begin{itemize}
\item {Proveniência:(Lat. \textunderscore villosus\textunderscore )}
\end{itemize}
Cheio de pêlos.
Cabeludo.
Hirsuto.
\section{Vilota}
\begin{itemize}
\item {Grp. gram.:f.}
\end{itemize}
O mesmo que \textunderscore vileta\textunderscore .
\section{Vilta}
\begin{itemize}
\item {Grp. gram.:f.}
\end{itemize}
\begin{itemize}
\item {Utilização:Ant.}
\end{itemize}
\begin{itemize}
\item {Proveniência:(De \textunderscore viltar\textunderscore )}
\end{itemize}
Injúria; aviltamento; vitupério.
\section{Viltança}
\begin{itemize}
\item {Grp. gram.:f.}
\end{itemize}
\begin{itemize}
\item {Proveniência:(De \textunderscore viltar\textunderscore )}
\end{itemize}
O mesmo que \textunderscore aviltamento\textunderscore .
\section{Viltar}
\begin{itemize}
\item {Grp. gram.:v. t.}
\end{itemize}
O mesmo que \textunderscore aviltar\textunderscore . Cf. \textunderscore Port. Mon. Hist.\textunderscore , \textunderscore Script.\textunderscore , 241.
\section{Vílula}
\begin{itemize}
\item {Grp. gram.:f.}
\end{itemize}
\begin{itemize}
\item {Utilização:Ant.}
\end{itemize}
\begin{itemize}
\item {Proveniência:(De \textunderscore vila\textunderscore )}
\end{itemize}
Pequeno prédio, pequena herdade.
Vilarinho.
\section{Vima}
\begin{itemize}
\item {Grp. gram.:f.}
\end{itemize}
\begin{itemize}
\item {Utilização:Prov.}
\end{itemize}
\begin{itemize}
\item {Utilização:trasm.}
\end{itemize}
\begin{itemize}
\item {Proveniência:(De \textunderscore vimar\textunderscore )}
\end{itemize}
Segunda cava ou lavra.
\section{Vima}
\begin{itemize}
\item {Grp. gram.:f.}
\end{itemize}
\begin{itemize}
\item {Utilização:Prov.}
\end{itemize}
\begin{itemize}
\item {Utilização:alent.}
\end{itemize}
Mèzinha, feita de pão torrado e embebido em vinho, ou de carne de gallinha cozida, picada e também embebida em vinho, de que se faz um emplastro para os pulsos ou para o estômago, especialmente contra a debilidade. Cf. Rev. \textunderscore Tradição\textunderscore , III, 177.
\section{Vima}
\begin{itemize}
\item {Grp. gram.:f.}
\end{itemize}
Planta salicínea, (\textunderscore vima salix\textunderscore , Lin.).
\section{Vimar}
\begin{itemize}
\item {Grp. gram.:v. t.}
\end{itemize}
\begin{itemize}
\item {Utilização:Prov.}
\end{itemize}
\begin{itemize}
\item {Utilização:trasm.}
\end{itemize}
Dar a segunda cava ou lavra em; redar.
\section{Vimaranense}
\begin{itemize}
\item {Grp. gram.:adj.}
\end{itemize}
\begin{itemize}
\item {Grp. gram.:M.}
\end{itemize}
\begin{itemize}
\item {Proveniência:(Do b. lat. \textunderscore Vimaranum\textunderscore , n. p.)}
\end{itemize}
Relativo a Guimarães.
Habitante de Guimarães.
\section{Vimba}
\begin{itemize}
\item {Grp. gram.:f.}
\end{itemize}
Árvore de Cabinda, cuja madeira se emprega em cabos de ferramentas e trabalhos de tôrno.
\section{Vime}
\begin{itemize}
\item {Grp. gram.:m.}
\end{itemize}
\begin{itemize}
\item {Proveniência:(Lat. \textunderscore vimen\textunderscore )}
\end{itemize}
Vara tenra e flexível de vimeiro.
Qualquer vara flexível, que serve para atar mólhos ou quaesquer objectos, e também para a fabricação de cestos, etc.
Vimeiro.
\section{Vímea}
\begin{itemize}
\item {Grp. gram.:f.}
\end{itemize}
\begin{itemize}
\item {Utilização:Prov.}
\end{itemize}
\begin{itemize}
\item {Utilização:trasm.}
\end{itemize}
\begin{itemize}
\item {Proveniência:(De \textunderscore vime\textunderscore )}
\end{itemize}
Varas de vimeiro para atar as vides.
\section{Vimeiro}
\begin{itemize}
\item {Grp. gram.:m.}
\end{itemize}
\begin{itemize}
\item {Proveniência:(De \textunderscore vime\textunderscore )}
\end{itemize}
Gênero de plantas salicíneas.
\section{Vimial}
\begin{itemize}
\item {Grp. gram.:m.}
\end{itemize}
\begin{itemize}
\item {Proveniência:(Do lat. \textunderscore viminalis\textunderscore )}
\end{itemize}
O mesmo que \textunderscore vimieiro\textunderscore :«\textunderscore ...monte, ao qual os vimiaes deram o nome.\textunderscore »Cf. Lobo, \textunderscore Sát. de Juv.\textunderscore , I, 148.
\section{Vimieiro}
\begin{itemize}
\item {Grp. gram.:m.}
\end{itemize}
\begin{itemize}
\item {Proveniência:(Do lat. \textunderscore viminarius\textunderscore )}
\end{itemize}
Terreno, onde crescem vimes.
\section{Vímina}
\begin{itemize}
\item {Grp. gram.:f.}
\end{itemize}
\begin{itemize}
\item {Proveniência:(Lat. \textunderscore vimina\textunderscore , pl. de \textunderscore vimen\textunderscore )}
\end{itemize}
Atilho de vimes; vímea.
\section{Viminária}
\begin{itemize}
\item {Grp. gram.:f.}
\end{itemize}
\begin{itemize}
\item {Proveniência:(Do lat. \textunderscore vimen\textunderscore )}
\end{itemize}
Gênero de plantas leguminosas.
\section{Vimíneo}
\begin{itemize}
\item {Grp. gram.:adj.}
\end{itemize}
\begin{itemize}
\item {Proveniência:(Lat. \textunderscore vimineus\textunderscore )}
\end{itemize}
Feito de vime.
\section{Viminoso}
\begin{itemize}
\item {Grp. gram.:adj.}
\end{itemize}
O mesmo que \textunderscore vimíneo\textunderscore :«\textunderscore ...de ramilhetes o viminoso cesto.\textunderscore »Castilho, \textunderscore Fastos\textunderscore , II, 153.
\section{Vimoso}
\begin{itemize}
\item {Grp. gram.:adj.}
\end{itemize}
O mesmo que \textunderscore vimíneo\textunderscore .
\section{Vim-vim}
\begin{itemize}
\item {Grp. gram.:m.}
\end{itemize}
\begin{itemize}
\item {Utilização:Bras}
\end{itemize}
Espécie de canário azul e amarelo, cuja voz imita o seu nome.
\section{Vina}
\begin{itemize}
\item {Grp. gram.:f.}
\end{itemize}
Espécie de palmeira.
\section{Vináceo}
\begin{itemize}
\item {Grp. gram.:adj.}
\end{itemize}
\begin{itemize}
\item {Proveniência:(Lat. \textunderscore vinaceus\textunderscore )}
\end{itemize}
O mesmo que \textunderscore víneo\textunderscore .
\section{Vinaes}
\begin{itemize}
\item {Grp. gram.:f. pl.}
\end{itemize}
\begin{itemize}
\item {Proveniência:(Lat. \textunderscore vinalia\textunderscore )}
\end{itemize}
Antigas festas, em honra de Júpiter. Cf. Castilho, \textunderscore Fastos\textunderscore , II, 203.
\section{Vinagem}
\begin{itemize}
\item {Grp. gram.:f.}
\end{itemize}
\begin{itemize}
\item {Proveniência:(Do lat. \textunderscore vinum\textunderscore )}
\end{itemize}
Fabríco do vinho. Cf. Lapa \textunderscore Proc. de Vin.\textunderscore , 22.
\section{Vinagrada}
\begin{itemize}
\item {Grp. gram.:f.}
\end{itemize}
\begin{itemize}
\item {Utilização:Prov.}
\end{itemize}
\begin{itemize}
\item {Utilização:alg.}
\end{itemize}
\begin{itemize}
\item {Utilização:Prov.}
\end{itemize}
\begin{itemize}
\item {Proveniência:(De \textunderscore vinagre\textunderscore )}
\end{itemize}
O mesmo que \textunderscore caspacho\textunderscore .
Refrêsco, feito de vinagre, água e açúcar.
\section{Vinagrar}
\begin{itemize}
\item {Grp. gram.:v. t.}
\end{itemize}
O mesmo que \textunderscore avinagrar\textunderscore .
\section{Vinagre}
\begin{itemize}
\item {Grp. gram.:m.}
\end{itemize}
\begin{itemize}
\item {Utilização:Fig.}
\end{itemize}
\begin{itemize}
\item {Utilização:Bras. de Pernambuco}
\end{itemize}
\begin{itemize}
\item {Proveniência:(Do lat. \textunderscore vinum\textunderscore  + \textunderscore acre\textunderscore )}
\end{itemize}
Líquido, resultante da fermentação ácida do vinho.
Ácido acético.
Coisa azêda.
Pessôa de gênio ou modos desabridos.
O mesmo que \textunderscore usurário\textunderscore .
\section{Vinagre}
\begin{itemize}
\item {Grp. gram.:adj.}
\end{itemize}
Diz-se do toiro, que tem o pêlo castanho claro, tirante a rubro.
(Cp. \textunderscore vinagre\textunderscore ^1)
\section{Vinagreira}
\begin{itemize}
\item {Grp. gram.:f.}
\end{itemize}
\begin{itemize}
\item {Utilização:Prov.}
\end{itemize}
\begin{itemize}
\item {Utilização:alent.}
\end{itemize}
\begin{itemize}
\item {Proveniência:(De \textunderscore vinagre\textunderscore )}
\end{itemize}
Vasilha, em que se guarda ou prepara o vinagre.
Planta, também conhecida por \textunderscore azêda\textunderscore .
Bôlsa molle, espécie de alforreca, que os temporaes arrojam á praia e que, pisada, expelle um líquido avermelhado.
O mesmo que \textunderscore caspacho\textunderscore .
\section{Vinagreiro}
\begin{itemize}
\item {Grp. gram.:m.}
\end{itemize}
\begin{itemize}
\item {Utilização:T. do Pôrto}
\end{itemize}
\begin{itemize}
\item {Proveniência:(De \textunderscore vinagre\textunderscore )}
\end{itemize}
Aquelle que fabríca ou vende vinagre.
Borrachão, bebedolas.
\section{Vinagrento}
\begin{itemize}
\item {Grp. gram.:m.}
\end{itemize}
Que sabe a vinagre.
\section{Vinagreta}
\begin{itemize}
\item {fónica:grê}
\end{itemize}
\begin{itemize}
\item {Grp. gram.:f.}
\end{itemize}
\begin{itemize}
\item {Utilização:Fam.}
\end{itemize}
\begin{itemize}
\item {Proveniência:(De \textunderscore vinagre\textunderscore )}
\end{itemize}
Vinho ordinário e um tanto azêdo.
\section{Vinagrinho}
\begin{itemize}
\item {Grp. gram.:m.}
\end{itemize}
Espécie de rapé.
\section{Vinais}
\begin{itemize}
\item {Grp. gram.:f. pl.}
\end{itemize}
\begin{itemize}
\item {Proveniência:(Lat. \textunderscore vinalia\textunderscore )}
\end{itemize}
Antigas festas, em honra de Júpiter. Cf. Castilho, \textunderscore Fastos\textunderscore , II, 203.
\section{Vinário}
\begin{itemize}
\item {Grp. gram.:adj.}
\end{itemize}
\begin{itemize}
\item {Proveniência:(Lat. \textunderscore vinarius\textunderscore )}
\end{itemize}
Relativo ao vinho.
Próprio para conter vinho.
\section{Vinca}
\begin{itemize}
\item {Grp. gram.:f.}
\end{itemize}
\begin{itemize}
\item {Utilização:Prov.}
\end{itemize}
\begin{itemize}
\item {Utilização:minh.}
\end{itemize}
Camada.
Fieira.
\section{Vinca}
\begin{itemize}
\item {Grp. gram.:f.}
\end{itemize}
Planta, o mesmo que \textunderscore pervinca\textunderscore .
\section{Vincada}
\begin{itemize}
\item {Grp. gram.:f.}
\end{itemize}
O mesmo que \textunderscore vinco\textunderscore ^1.
\section{Vinca-pervinca}
\begin{itemize}
\item {Grp. gram.:f.}
\end{itemize}
O mesmo que \textunderscore pervinca\textunderscore ^1.
\section{Vincar}
\begin{itemize}
\item {Grp. gram.:v. t.}
\end{itemize}
Fazer vincos em.
Fazer dobras em; enrugar.
\section{Vincelha}
\begin{itemize}
\item {Grp. gram.:f.}
\end{itemize}
\begin{itemize}
\item {Utilização:Prov.}
\end{itemize}
\begin{itemize}
\item {Utilização:beir.}
\end{itemize}
O mesmo que \textunderscore vincelho\textunderscore .
\section{Vincelho}
\begin{itemize}
\item {Grp. gram.:m.}
\end{itemize}
\begin{itemize}
\item {Proveniência:(Do lat. hyp. \textunderscore vinciculum\textunderscore , por \textunderscore vinculum\textunderscore )}
\end{itemize}
O mesmo que \textunderscore vincilho\textunderscore .
\section{Vincendo}
\begin{itemize}
\item {Grp. gram.:adj.}
\end{itemize}
\begin{itemize}
\item {Proveniência:(Lat. \textunderscore vincendus\textunderscore )}
\end{itemize}
Que se há de vencer, (falando-se de dívidas): \textunderscore prestações vincendas\textunderscore .
\section{Vincetóxico}
\begin{itemize}
\item {fónica:csi}
\end{itemize}
\begin{itemize}
\item {Grp. gram.:m.}
\end{itemize}
\begin{itemize}
\item {Proveniência:(Do lat. \textunderscore vincere\textunderscore  + \textunderscore toxicum\textunderscore )}
\end{itemize}
Planta apocýnea.
\section{Vincilho}
\begin{itemize}
\item {Grp. gram.:m.}
\end{itemize}
\begin{itemize}
\item {Proveniência:(Do b. lat. \textunderscore vincilium\textunderscore )}
\end{itemize}
Vime, vêrga ou corda de palha, para atar feixes, empar videiras, etc.
\section{Vincituro}
\begin{itemize}
\item {Grp. gram.:adj.}
\end{itemize}
\begin{itemize}
\item {Utilização:Neol.}
\end{itemize}
\begin{itemize}
\item {Proveniência:(Lat. \textunderscore vinciturus\textunderscore )}
\end{itemize}
Que há de vencer.
\section{Vinco}
\begin{itemize}
\item {Grp. gram.:m.}
\end{itemize}
Aresta ou sinal, deixado por uma dobra.
Sulco ou vestígio, deixado por pancada, pela passagem de uma roda, por um cordão que se apertou em volta de um corpo, por unhada, etc.
Vergão.
Pedaço de metal, que se prende á tromba do porco, para que êste não fósse na terra.
Primeira camada, immediata á côdea inferior da brôa, quando esta sai mal cozida do forno.
(Cp. \textunderscore vínculo\textunderscore )
\section{Vinco}
\begin{itemize}
\item {Grp. gram.:m.}
\end{itemize}
\begin{itemize}
\item {Utilização:Prov.}
\end{itemize}
\begin{itemize}
\item {Utilização:extrem.}
\end{itemize}
Quantidade de azeitona, que a moenda leva de cada vez.
\section{Vinculado}
\begin{itemize}
\item {Grp. gram.:adj.}
\end{itemize}
Instituido por vínculo; que tem natureza de vínculo; vincular.
\section{Vinculador}
\begin{itemize}
\item {Grp. gram.:m.  e  adj.}
\end{itemize}
O que vincula.
\section{Vincular}
\begin{itemize}
\item {Grp. gram.:adj.}
\end{itemize}
Relativo a vínculo.
\section{Vincular}
\begin{itemize}
\item {Grp. gram.:v. t.}
\end{itemize}
\begin{itemize}
\item {Utilização:Fig.}
\end{itemize}
\begin{itemize}
\item {Proveniência:(Lat. \textunderscore vinculare\textunderscore )}
\end{itemize}
Ligar; apertar; prender.
Ligar moralmente.
Converter em prazo inalienável ou em morgado.
Firmar a posse de.
Sujeitar, obrigar.
\section{Vinculativo}
\begin{itemize}
\item {Grp. gram.:adj.}
\end{itemize}
\begin{itemize}
\item {Proveniência:(De \textunderscore vincular\textunderscore )}
\end{itemize}
Que vincula.
\section{Vinculatório}
\begin{itemize}
\item {Grp. gram.:adj.}
\end{itemize}
\begin{itemize}
\item {Proveniência:(Lat. \textunderscore vinculatorius\textunderscore )}
\end{itemize}
O mesmo que \textunderscore vinculativo\textunderscore .
\section{Vinculável}
\begin{itemize}
\item {Grp. gram.:adj.}
\end{itemize}
Que se póde vincular.
\section{Vínculo}
\begin{itemize}
\item {Grp. gram.:m.}
\end{itemize}
\begin{itemize}
\item {Utilização:Fig.}
\end{itemize}
\begin{itemize}
\item {Proveniência:(Lat. \textunderscore vinculum\textunderscore )}
\end{itemize}
Tudo que ata, liga ou aperta.
Nó; liame.
Ligação moral.
Conjunto de certos bens inalienáveis, que se transmittiam indivisivelmente.
Morgado.
\section{Vinda}
\begin{itemize}
\item {Grp. gram.:f.}
\end{itemize}
\begin{itemize}
\item {Proveniência:(De \textunderscore vindo\textunderscore )}
\end{itemize}
Acto ou effeito de vir.
Chegada.
Regresso.
\section{Vinda-cáa}
\begin{itemize}
\item {Grp. gram.:f.}
\end{itemize}
Planta amomácea.
\section{Vindemiaes}
\begin{itemize}
\item {Grp. gram.:f. pl.}
\end{itemize}
\begin{itemize}
\item {Proveniência:(Lat. \textunderscore vindemialia\textunderscore )}
\end{itemize}
Festa das vindimas, na antiga Roma.
\section{Vindemiais}
\begin{itemize}
\item {Grp. gram.:f. pl.}
\end{itemize}
\begin{itemize}
\item {Proveniência:(Lat. \textunderscore vindemialia\textunderscore )}
\end{itemize}
Festa das vindimas, na antiga Roma.
\section{Vindemiário}
\begin{itemize}
\item {Grp. gram.:m.}
\end{itemize}
\begin{itemize}
\item {Proveniência:(Do fr. \textunderscore vendemiaire\textunderscore )}
\end{itemize}
Primeiro mês do calendário da primeira República Francesa, o qual começava a 22 de Setembro.
\section{Vindicá}
\begin{itemize}
\item {Grp. gram.:m.}
\end{itemize}
\begin{itemize}
\item {Utilização:Bras}
\end{itemize}
Planta aromática, provavelmente o mesmo que \textunderscore vinda-cáa\textunderscore .
\section{Vindicação}
\begin{itemize}
\item {Grp. gram.:f.}
\end{itemize}
\begin{itemize}
\item {Utilização:Jur.}
\end{itemize}
\begin{itemize}
\item {Proveniência:(Do lat. \textunderscore vindicatio\textunderscore )}
\end{itemize}
Acto ou effeito de vindicar.
Reclamação.
Acto de exigir judicialmente que a alguém se reconheça o estado civil que lhe pertence.
\section{Vindicador}
\begin{itemize}
\item {Grp. gram.:m.  e  adj.}
\end{itemize}
\begin{itemize}
\item {Proveniência:(Do lat. \textunderscore vindicator\textunderscore )}
\end{itemize}
O que vindica.
\section{Vindicar}
\begin{itemize}
\item {Grp. gram.:v. t.}
\end{itemize}
\begin{itemize}
\item {Proveniência:(Lat. \textunderscore vindicare\textunderscore )}
\end{itemize}
Exigir em nome da lei.
Pretender a legalização de.
Recuperar.
Reivindicar.
\section{Vindicativo}
\begin{itemize}
\item {Grp. gram.:adj.}
\end{itemize}
Próprio para vindicar.
Que defende; que vinga.
\section{Víndice}
\begin{itemize}
\item {Grp. gram.:adj.}
\end{itemize}
\begin{itemize}
\item {Grp. gram.:M.  e  f.}
\end{itemize}
\begin{itemize}
\item {Proveniência:(Do lat. \textunderscore vindex\textunderscore )}
\end{itemize}
O mesmo que \textunderscore vingador\textunderscore :«\textunderscore ...a víndice visão de tão mau feito.\textunderscore »Sousa Monteiro, \textunderscore Falstaff\textunderscore , (pról.).
Pessôa, que vinga.
\section{Vindícia}
\begin{itemize}
\item {Grp. gram.:f.}
\end{itemize}
\begin{itemize}
\item {Proveniência:(Lat. \textunderscore vindicia\textunderscore )}
\end{itemize}
Acto ou effeito de reivindicar.
\section{Vindiço}
\begin{itemize}
\item {Grp. gram.:adj.}
\end{itemize}
\begin{itemize}
\item {Proveniência:(Do lat. hyp. \textunderscore venititius\textunderscore )}
\end{itemize}
Adventício.
Que veio de fóra:«\textunderscore ...enxertos de outra parte com as vindiças da Sýria...\textunderscore »Castilho, \textunderscore Geórgicas\textunderscore , 79.
\section{Vindicta}
\begin{itemize}
\item {Grp. gram.:f.}
\end{itemize}
\begin{itemize}
\item {Proveniência:(Lat. \textunderscore vindicta\textunderscore )}
\end{itemize}
Punição legal; castigo.
Vingança.
\section{Vindima}
\begin{itemize}
\item {Grp. gram.:f}
\end{itemize}
\begin{itemize}
\item {Utilização:Fig.}
\end{itemize}
\begin{itemize}
\item {Utilização:Pop.}
\end{itemize}
\begin{itemize}
\item {Proveniência:(Do lat. \textunderscore vindemia\textunderscore )}
\end{itemize}
Colheita ou apanha de uvas.
Uvas vindimadas.
Tempo em que se vindima.
Colheita.
Grangeio; acquisição.
Cesto vindimo.
\section{Vindimadeira}
\begin{itemize}
\item {Grp. gram.:f.}
\end{itemize}
Mulhér, que vindima.
(Cp. \textunderscore vindimadeiro\textunderscore )
\section{Vindimadeiro}
\begin{itemize}
\item {Grp. gram.:m.  e  adj.}
\end{itemize}
O mesmo que \textunderscore vindimador\textunderscore .
\section{Vindimado}
\begin{itemize}
\item {Grp. gram.:adj.}
\end{itemize}
\begin{itemize}
\item {Utilização:Fig.}
\end{itemize}
\begin{itemize}
\item {Utilização:Pleb.}
\end{itemize}
\begin{itemize}
\item {Proveniência:(De \textunderscore vindimar\textunderscore )}
\end{itemize}
Em que já se colheram as uvas: \textunderscore vinha vindimada\textunderscore .
Acabado, extincto.
Assassinado.
\section{Vindimador}
\begin{itemize}
\item {Grp. gram.:m.  e  adj.}
\end{itemize}
\begin{itemize}
\item {Proveniência:(Do lat. \textunderscore vindimiator\textunderscore )}
\end{itemize}
O que vindima.
\section{Vindimadura}
\begin{itemize}
\item {Grp. gram.:f.}
\end{itemize}
(V.vindima)
\section{Vindimal}
\begin{itemize}
\item {Grp. gram.:adj.}
\end{itemize}
\begin{itemize}
\item {Proveniência:(Lat. \textunderscore vindemialis\textunderscore )}
\end{itemize}
Relativo a vindima.
\section{Vindimar}
\begin{itemize}
\item {Grp. gram.:v. t.}
\end{itemize}
\begin{itemize}
\item {Utilização:Fig.}
\end{itemize}
\begin{itemize}
\item {Utilização:Pleb.}
\end{itemize}
\begin{itemize}
\item {Proveniência:(Do lat. \textunderscore vindemiare\textunderscore )}
\end{itemize}
Fazer a vindima de; colher as uvas de: \textunderscore vindimar a parreira\textunderscore .
Colher.
Destruír; dar cabo de.
Matar, assassinar.
\section{Vindimeiro}
\begin{itemize}
\item {Grp. gram.:m.}
\end{itemize}
\begin{itemize}
\item {Utilização:Prov.}
\end{itemize}
\begin{itemize}
\item {Utilização:Prov.}
\end{itemize}
\begin{itemize}
\item {Utilização:trasm.}
\end{itemize}
O mesmo que \textunderscore vindimador\textunderscore .
O mesmo que \textunderscore vindimo\textunderscore , (falando-se de cestos).
\section{Vindimo}
\begin{itemize}
\item {Grp. gram.:adj.}
\end{itemize}
\begin{itemize}
\item {Proveniência:(De \textunderscore vindimar\textunderscore )}
\end{itemize}
Vindimal.
Próprio para a vindima, (falando-se de cestos).
Serôdio.
\section{Vindita}
\begin{itemize}
\item {Grp. gram.:f.}
\end{itemize}
Espécie de ádem das costas da América.
\section{Vindo}
\begin{itemize}
\item {Grp. gram.:adj.}
\end{itemize}
\begin{itemize}
\item {Proveniência:(De \textunderscore vir\textunderscore )}
\end{itemize}
Que veio, que chegou: \textunderscore seja bem vindo\textunderscore .
Procedente, proveniente.
\section{Vindoiro}
\begin{itemize}
\item {Grp. gram.:adj.}
\end{itemize}
\begin{itemize}
\item {Grp. gram.:M. Pl.}
\end{itemize}
\begin{itemize}
\item {Proveniência:(Do lat. \textunderscore venturus\textunderscore )}
\end{itemize}
Que há de vir ou acontecer.
Futuro.
Os homens futuros; a posteridade.
\section{Vindouro}
\begin{itemize}
\item {Grp. gram.:adj.}
\end{itemize}
\begin{itemize}
\item {Grp. gram.:M. Pl.}
\end{itemize}
\begin{itemize}
\item {Proveniência:(Do lat. \textunderscore venturus\textunderscore )}
\end{itemize}
Que há de vir ou acontecer.
Futuro.
Os homens futuros; a posteridade.
\section{Vínea}
\begin{itemize}
\item {Grp. gram.:f.}
\end{itemize}
\begin{itemize}
\item {Proveniência:(Do lat. \textunderscore vinea\textunderscore , vinha, pela sua semelhança a uma latada)}
\end{itemize}
Antiga máquina de guerra, formada de paus á maneira de grade, e revestida de coiros, para que a não perfurassem os dardos e as pedras. Cf. Du-Cange, vb. \textunderscore vinea\textunderscore ; Herculano, \textunderscore Bobo\textunderscore , 17.
\section{Víneo}
\begin{itemize}
\item {Grp. gram.:adj.}
\end{itemize}
\begin{itemize}
\item {Utilização:Poét.}
\end{itemize}
\begin{itemize}
\item {Proveniência:(Lat. \textunderscore vineus\textunderscore )}
\end{itemize}
Feito de vinho.
Misturado com vinho.
Que tem a natureza ou a côr do vinho.
\section{Vinga}
\begin{itemize}
\item {Grp. gram.:f.}
\end{itemize}
\begin{itemize}
\item {Utilização:Prov.}
\end{itemize}
\begin{itemize}
\item {Proveniência:(De \textunderscore vingar\textunderscore )}
\end{itemize}
Haste nova, rebento.
\section{Vingador}
\begin{itemize}
\item {Grp. gram.:m.  e  adj.}
\end{itemize}
O que vinga ou serve para vingar: \textunderscore espada vingadora\textunderscore .
\section{Vingança}
\begin{itemize}
\item {Grp. gram.:f.}
\end{itemize}
Acto ou effeito de vingar.
Desforra; castigo.
\section{Vingar}
\begin{itemize}
\item {Grp. gram.:v. t.}
\end{itemize}
\begin{itemize}
\item {Grp. gram.:V. i.}
\end{itemize}
\begin{itemize}
\item {Grp. gram.:V. p.}
\end{itemize}
\begin{itemize}
\item {Proveniência:(Do lat. \textunderscore vindicare\textunderscore )}
\end{itemize}
Infligir punição a alguém, para satisfação pessoal de (uma pessôa offendida): \textunderscore vingar um innocente\textunderscore .
Tirar desforra de: \textunderscore vingar uma afronta\textunderscore .
Causar a punição de.
Desforrar, desafrontar.
Promover a reparação de.
Galardoar.
Defender.
Libertar, salvar.
Attingir; galgar; ultrapassar; subir: \textunderscore vingar o vêrtice da collina\textunderscore .
Chegar ao cabo de.
Conseguir.
Têr bom êxito: \textunderscore aquella tentativa vingou\textunderscore .
Resultar; realizar-se.
Desenvolver-se: \textunderscore as sementeiras vingaram\textunderscore .
Passar além.
Desforrar-se.
Castigar alguém, de quem recebeu offensa.
Contentar-se com certas compensações de coisas ou factos nocivos ou desagradáveis.
\section{Vingativamente}
\begin{itemize}
\item {Grp. gram.:adv.}
\end{itemize}
De modo vingativo; por vingança.
\section{Vingativo}
\begin{itemize}
\item {Grp. gram.:adj.}
\end{itemize}
\begin{itemize}
\item {Proveniência:(De \textunderscore vingar\textunderscore )}
\end{itemize}
Em que há vingança.
Que se vinga.
Que se compraz em se vingar; que gosta da vingança.
\section{Vinha}
\begin{itemize}
\item {Grp. gram.:f.}
\end{itemize}
\begin{itemize}
\item {Utilização:Fig.}
\end{itemize}
\begin{itemize}
\item {Proveniência:(Do lat. \textunderscore vinea\textunderscore )}
\end{itemize}
Terreno, onde crescem videiras.
Aquillo que dá proveito.
Pechincha.
\textunderscore Vinha do Senhor\textunderscore , prática da religião.
\textunderscore Vinha de alhos\textunderscore , môlho do conserva, em que entram alhos, vinagre e outras especiarias.
\section{Vinhaça}
\begin{itemize}
\item {Grp. gram.:f.}
\end{itemize}
Grande porção de vinho.
Vinho reles.
Embriaguez.
(Cp. \textunderscore vinháceo\textunderscore )
\section{Vinháceo}
\begin{itemize}
\item {Grp. gram.:adj.}
\end{itemize}
\begin{itemize}
\item {Proveniência:(Do lat. \textunderscore vinaceus\textunderscore )}
\end{itemize}
Vinário; semelhante ao vinho.
\section{Vinhaço}
\begin{itemize}
\item {Grp. gram.:m.}
\end{itemize}
Bagaço de uvas.
Resíduos da pisa de uvas, nos quaes se contém ainda vinho.
(Cp. \textunderscore vinháceo\textunderscore )
\section{Vinhadeiro}
\begin{itemize}
\item {Grp. gram.:m.}
\end{itemize}
O mesmo que \textunderscore vinheiro\textunderscore .
\section{Vinhado}
\begin{itemize}
\item {Grp. gram.:m.}
\end{itemize}
\begin{itemize}
\item {Utilização:Bras}
\end{itemize}
Passarinho, que se domestica em gaiola.
\section{Vinhaga}
\begin{itemize}
\item {Grp. gram.:f.}
\end{itemize}
\begin{itemize}
\item {Utilização:Ant.}
\end{itemize}
O mesmo que \textunderscore vinhago\textunderscore .
\section{Vinhagem}
\begin{itemize}
\item {Grp. gram.:f.}
\end{itemize}
O mesmo que \textunderscore vinhago\textunderscore .
\section{Vinhago}
\begin{itemize}
\item {Grp. gram.:m.}
\end{itemize}
\begin{itemize}
\item {Proveniência:(Do lat. \textunderscore vinago\textunderscore )}
\end{itemize}
O mesmo que \textunderscore vinhedo\textunderscore .
\section{Vinhal}
\begin{itemize}
\item {Grp. gram.:m.}
\end{itemize}
\begin{itemize}
\item {Proveniência:(Do lat. \textunderscore vinealis\textunderscore )}
\end{itemize}
Terrenos, onde há vinhas.
Casta de uva.
\section{Vinhão}
\begin{itemize}
\item {Grp. gram.:m.}
\end{itemize}
Bom vinho.
Variedade de vinho encorpado e de bôa côr, com que se lotam vinhos de qualidade inferior.
Casta de uva. Cf. \textunderscore Techn. Rur.\textunderscore , 90.
\section{Vinhão-de-tinta}
\begin{itemize}
\item {Grp. gram.:m.}
\end{itemize}
Casta de uva preta.
\section{Vinhão-molle}
\begin{itemize}
\item {Grp. gram.:m.}
\end{itemize}
Casta de uva minhota, variedade do vinhão-de-tinta.
\section{Vinhão-tinto}
\begin{itemize}
\item {Grp. gram.:m.}
\end{itemize}
Casta de uva minhota, o mesmo que \textunderscore vinhão-de-tinta\textunderscore .
\section{Vinhar}
\begin{itemize}
\item {Grp. gram.:m.}
\end{itemize}
O mesmo que \textunderscore vinhal\textunderscore .
\section{Vinhataria}
\begin{itemize}
\item {Grp. gram.:f.}
\end{itemize}
Cultura de vinhas.
Fabricação de vinho.
(Cp. \textunderscore vinhateiro\textunderscore )
\section{Vinhateira}
\begin{itemize}
\item {Grp. gram.:f.}
\end{itemize}
\begin{itemize}
\item {Utilização:Náut.}
\end{itemize}
Pedaço de cabo, que tem numa extremidade uma alça e na outra um nó que se engasga na mesma alça.--Amarra-se nos primeiros ovens do mastro grande e de traquete, e serve para se meterem as amuras e escotas da vela grande e traquete. Cf. M. C. Campos, \textunderscore Voc. Mar.\textunderscore 
\section{Vinhateiro}
\begin{itemize}
\item {Grp. gram.:adj.}
\end{itemize}
\begin{itemize}
\item {Grp. gram.:M.}
\end{itemize}
\begin{itemize}
\item {Proveniência:(De \textunderscore vinha\textunderscore  ou \textunderscore vinho\textunderscore )}
\end{itemize}
Relativo á cultura das vinhas.
Que cultiva vinhas.
Cultivador de vinhas.
Fabricante de vinho.
\section{Vinhático}
\begin{itemize}
\item {Grp. gram.:m.}
\end{itemize}
\begin{itemize}
\item {Proveniência:(Do lat. \textunderscore vineaticus\textunderscore )}
\end{itemize}
Árvore leguminosa do Brasil e dos Açores.
Madeira dessa árvore.
\section{Vinhato}
\begin{itemize}
\item {Grp. gram.:m.}
\end{itemize}
\begin{itemize}
\item {Utilização:Prov.}
\end{itemize}
O mesmo que \textunderscore vinhático\textunderscore .
\section{Vinhedo}
\begin{itemize}
\item {fónica:nhê}
\end{itemize}
\begin{itemize}
\item {Grp. gram.:m.}
\end{itemize}
\begin{itemize}
\item {Proveniência:(De \textunderscore vinha\textunderscore )}
\end{itemize}
Vinha; vinhal.
\section{Vinheiro}
\begin{itemize}
\item {Grp. gram.:m.}
\end{itemize}
\begin{itemize}
\item {Proveniência:(Do lat. \textunderscore vinearius\textunderscore )}
\end{itemize}
Aquelle que cultiva vinhas; guarda de vinhas.
\section{Vinheta}
\begin{itemize}
\item {fónica:nhê}
\end{itemize}
\begin{itemize}
\item {Grp. gram.:f.}
\end{itemize}
\begin{itemize}
\item {Proveniência:(Fr. \textunderscore vignette\textunderscore )}
\end{itemize}
Pequena estampa de um livro, para explicação de texto ou para ornato.
\section{Vinhete}
\begin{itemize}
\item {fónica:nhê}
\end{itemize}
\begin{itemize}
\item {Grp. gram.:m.}
\end{itemize}
Vinho fraco.
\section{Vinhetista}
\begin{itemize}
\item {Grp. gram.:m.}
\end{itemize}
Aquelle que desenha ou grava vinhetas.
\section{Vinho}
\begin{itemize}
\item {Grp. gram.:m.}
\end{itemize}
\begin{itemize}
\item {Utilização:Fig.}
\end{itemize}
\begin{itemize}
\item {Proveniência:(Do lat. \textunderscore vinum\textunderscore )}
\end{itemize}
Líquido alcoólico, resultante da fermentação do sumo das uvas ou de outros vegetaes: \textunderscore vinho de maçans\textunderscore .
\textunderscore Vinho de maçans\textunderscore , sidra.
\textunderscore Vinho de cheiro\textunderscore , vinho aromático, fabricado nos Açores com uva isabel.
\textunderscore Vinho de arbusto\textunderscore , vinho fabricado com uvas das uveiras.
\textunderscore Vinho de enforcado\textunderscore , vinho clarete, o mesmo que \textunderscore vinho de arbusto\textunderscore .
\textunderscore Vinho abafado\textunderscore , geropiga forte.
\textunderscore Vinho surdo\textunderscore , nome, que dão na Madeira ao vinho abafado.
\textunderscore Vinho fino\textunderscore , vinho generoso, vinho velho e alcoólico.
\textunderscore Vinho verde\textunderscore , vinho de sabor ácido, menos alcoólico que o vinho commum, e fabricado no Minho e em parte da Beira com uvas especiaes, ás vezes colhidas antes da maturação.
Bebedeira.
\section{Vinhoca}
\begin{itemize}
\item {Grp. gram.:f.}
\end{itemize}
Mau vinho, vinhaça.
\section{Vinhocel}
\begin{itemize}
\item {Grp. gram.:m.}
\end{itemize}
Casta de uva.
\section{Vinhocelo}
\begin{itemize}
\item {Grp. gram.:m.}
\end{itemize}
Casta de uva.
\section{Vinhogo}
\begin{itemize}
\item {fónica:nhô}
\end{itemize}
\begin{itemize}
\item {Grp. gram.:m.}
\end{itemize}
\begin{itemize}
\item {Utilização:Ant.}
\end{itemize}
Lugar, onde há muitas vinhas.
\section{Vinho-judeu}
\begin{itemize}
\item {Grp. gram.:m.}
\end{itemize}
Bebida alcoólica, que se fabríca em Dio com arroz e certas ervas. Cf. \textunderscore Século\textunderscore , de 20-XI-98.
\section{Vinhote}
\begin{itemize}
\item {Grp. gram.:m.}
\end{itemize}
\begin{itemize}
\item {Utilização:Pop.}
\end{itemize}
\begin{itemize}
\item {Proveniência:(De \textunderscore vinho\textunderscore )}
\end{itemize}
O mesmo que \textunderscore vinhete\textunderscore .
Homem, que se embriaga muitas vezes.
\section{Vinhozel}
\begin{itemize}
\item {Grp. gram.:m.}
\end{itemize}
\begin{itemize}
\item {Utilização:Prov.}
\end{itemize}
\begin{itemize}
\item {Utilização:beir.}
\end{itemize}
O mesmo que \textunderscore vinhocel\textunderscore .
\section{Vínico}
\begin{itemize}
\item {Grp. gram.:adj.}
\end{itemize}
\begin{itemize}
\item {Proveniência:(Do lat. \textunderscore vinum\textunderscore )}
\end{itemize}
O mesmo que \textunderscore vinário\textunderscore .
Procedente do vinho.
\section{Vinícola}
\begin{itemize}
\item {Grp. gram.:adj.}
\end{itemize}
\begin{itemize}
\item {Proveniência:(Do lat. \textunderscore vinum\textunderscore  + \textunderscore colere\textunderscore )}
\end{itemize}
Relativo á vinicultura.
\section{Vinicultor}
\begin{itemize}
\item {Grp. gram.:m.}
\end{itemize}
Aquelle que se occupa de vinicultura.
\section{Vinicultura}
\begin{itemize}
\item {Grp. gram.:f.}
\end{itemize}
\begin{itemize}
\item {Proveniência:(Do lat. \textunderscore vinum\textunderscore  + \textunderscore cultura\textunderscore )}
\end{itemize}
Fabrico de vinho.
O mesmo que \textunderscore viticultura\textunderscore .
\section{Viníferas}
\begin{itemize}
\item {Grp. gram.:f. pl.}
\end{itemize}
O mesmo que \textunderscore ampelídeas\textunderscore .
(Fem. pl. de \textunderscore vinífero\textunderscore )
\section{Vinífero}
\begin{itemize}
\item {Grp. gram.:adj.}
\end{itemize}
\begin{itemize}
\item {Proveniência:(Lat. \textunderscore vinifer\textunderscore )}
\end{itemize}
Que produz vinho.
\section{Vinificação}
\begin{itemize}
\item {Grp. gram.:f.}
\end{itemize}
\begin{itemize}
\item {Proveniência:(De \textunderscore vinificar\textunderscore )}
\end{itemize}
Fabrico de vinhos.
Processo de tratar os vinhos.
\section{Vinificador}
\begin{itemize}
\item {Grp. gram.:m.}
\end{itemize}
\begin{itemize}
\item {Proveniência:(De \textunderscore vinificar\textunderscore )}
\end{itemize}
Apparelho, para se fabricar vinho.
\section{Vinificar}
\begin{itemize}
\item {Grp. gram.:v. t.}
\end{itemize}
\begin{itemize}
\item {Proveniência:(Do lat. \textunderscore vinum\textunderscore  + \textunderscore facere\textunderscore )}
\end{itemize}
Reduzir a vinho (uvas). Cf. \textunderscore Techn. Rur.\textunderscore , 90.
\section{Vino-colorímetro}
\begin{itemize}
\item {Grp. gram.:m.}
\end{itemize}
Apparelho, para a comparação e classificação da côr do vinho.
\section{Vinolência}
\begin{itemize}
\item {Grp. gram.:f.}
\end{itemize}
\begin{itemize}
\item {Proveniência:(Lat. \textunderscore vinolentia\textunderscore )}
\end{itemize}
Qualidade do que é vinolento.
\section{Vinolento}
\begin{itemize}
\item {Grp. gram.:adj.}
\end{itemize}
\begin{itemize}
\item {Proveniência:(Lat. \textunderscore vinolentus\textunderscore )}
\end{itemize}
Que bebe muito vinho; ébrio.
\section{Vinórica}
\begin{itemize}
\item {Grp. gram.:f.}
\end{itemize}
\begin{itemize}
\item {Utilização:Ant.}
\end{itemize}
\begin{itemize}
\item {Utilização:Gír.}
\end{itemize}
O mesmo que \textunderscore cara\textunderscore ^1.
(Metáth. de \textunderscore verónica\textunderscore )
\section{Vinosidade}
\begin{itemize}
\item {Grp. gram.:f.}
\end{itemize}
\begin{itemize}
\item {Proveniência:(Do lat. \textunderscore vinositas\textunderscore )}
\end{itemize}
Qualidade ou carácter do que é vinoso.
\section{Vinoso}
\begin{itemize}
\item {Grp. gram.:adj.}
\end{itemize}
\begin{itemize}
\item {Proveniência:(Lat. \textunderscore vinosus\textunderscore )}
\end{itemize}
Que produz vinho.
Semelhante ao vinho na côr ou no sabor.
Que tem qualidades análogas á do vinho.
\section{Vintaneiro}
\begin{itemize}
\item {Grp. gram.:adj.}
\end{itemize}
\begin{itemize}
\item {Utilização:Des.}
\end{itemize}
\begin{itemize}
\item {Proveniência:(De \textunderscore vinte\textunderscore  + \textunderscore anos\textunderscore )}
\end{itemize}
Que tem vinte anos.
Que só produz de vinte em vinte anos, (falando-se de certos terrenos).
\section{Vintanneiro}
\begin{itemize}
\item {Grp. gram.:adj.}
\end{itemize}
\begin{itemize}
\item {Utilização:Des.}
\end{itemize}
\begin{itemize}
\item {Proveniência:(De \textunderscore vinte\textunderscore  + \textunderscore annos\textunderscore )}
\end{itemize}
Que tem vinte annos.
Que só produz de vinte em vinte annos, (falando-se de certos terrenos).
\section{Vintavo}
\begin{itemize}
\item {Grp. gram.:m.}
\end{itemize}
A vigésima parte.
(Cast. \textunderscore veintavo\textunderscore )
\section{Vinte}
\begin{itemize}
\item {Grp. gram.:adj.}
\end{itemize}
\begin{itemize}
\item {Grp. gram.:M.}
\end{itemize}
\begin{itemize}
\item {Utilização:Fam.}
\end{itemize}
\begin{itemize}
\item {Proveniência:(Do lat. \textunderscore viginti\textunderscore )}
\end{itemize}
Diz-se do número, formado de déz e mais déz.
Vigésimo.
Aquelle ou aquillo que numa série de vinte occupa o último lugar.
Pau, que vale vinte pontos, no jôgo da bóla.
\textunderscore Dar no vinte\textunderscore , adivinhar, perceber; ganhar; acertar.
\section{Vinte}
\begin{itemize}
\item {Grp. gram.:adj.}
\end{itemize}
\begin{itemize}
\item {Utilização:Ant.}
\end{itemize}
\begin{itemize}
\item {Proveniência:(De \textunderscore vir\textunderscore )}
\end{itemize}
Que vem; que chega.
O mesmo que \textunderscore vindo\textunderscore , (gerúndio de \textunderscore vir\textunderscore ).
O mesmo que \textunderscore vindoiro\textunderscore .
\section{Vintedezena}
\begin{itemize}
\item {Grp. gram.:f.}
\end{itemize}
\begin{itemize}
\item {Utilização:Mús.}
\end{itemize}
\begin{itemize}
\item {Proveniência:(De \textunderscore vintedozeno\textunderscore )}
\end{itemize}
Registo de órgão, que resôa três oitavas acima do diapasão.
\section{Vintedozeno}
\begin{itemize}
\item {Grp. gram.:m.  e  adj.}
\end{itemize}
\begin{itemize}
\item {Proveniência:(De \textunderscore vinte\textunderscore  + \textunderscore doze\textunderscore )}
\end{itemize}
Dizia-se do pano, que tem 3200 fios de urdidura.
\section{Vinte-e-ocheno}
\begin{itemize}
\item {Grp. gram.:adj.}
\end{itemize}
Dizia-se do pano, que tem 2800 fios de urdidura.
(Cast. \textunderscore vinteocheno\textunderscore )
\section{Vinte-e-quatreno}
\begin{itemize}
\item {Grp. gram.:adj.}
\end{itemize}
Dizia do pano, que tem 2400 fios de urdidura.
(Cast. \textunderscore veintequatreno\textunderscore )
\section{Vinte-e-quatro}
\begin{itemize}
\item {Grp. gram.:m.}
\end{itemize}
Antigo agrupamento dos representantes dos offícios mecânicos em Lisbôa, que tinham voto na administração do município.
\section{Vinte-e-quatro-horas}
\begin{itemize}
\item {Grp. gram.:f.}
\end{itemize}
Planta trepadeira da ilha de San-Thomé.
\section{Vinte-e-um}
\begin{itemize}
\item {Grp. gram.:m.}
\end{itemize}
Jôgo de cartas, em que se distribuem duas a cada parceiro, e em que o ganho é para aquelle que reúne um número de pontos igual a vinte e um, ou próximo dêste, mas nunca excedendo-o.
\section{Vintém}
\begin{itemize}
\item {Grp. gram.:m.}
\end{itemize}
\begin{itemize}
\item {Utilização:Fam.}
\end{itemize}
\begin{itemize}
\item {Utilização:Pop.}
\end{itemize}
\begin{itemize}
\item {Proveniência:(De \textunderscore vinteno\textunderscore )}
\end{itemize}
Moéda de cobre, que valia 20 reis e equivale a 2 centavos da actual moéda portuguesa.
Dinheiro: \textunderscore não tenho um vintém.\textunderscore 
\textunderscore Três vinténs\textunderscore , o mesmo que \textunderscore virgindade\textunderscore .
\section{Vintém}
\begin{itemize}
\item {Grp. gram.:m.}
\end{itemize}
Peixe de Portugal.
\section{Vintena}
\begin{itemize}
\item {Grp. gram.:f.}
\end{itemize}
\begin{itemize}
\item {Utilização:Ant.}
\end{itemize}
\begin{itemize}
\item {Proveniência:(De \textunderscore vinteno\textunderscore )}
\end{itemize}
Série de um grupo de vinte.
A vigésima parte.
Reunião de vinte fogos numa povoação.
Tributo antigo, equivalente á vigésima parte do rendimento.
\section{Vintenário}
\begin{itemize}
\item {Grp. gram.:m.  e  adj.}
\end{itemize}
O mesmo que \textunderscore vinteneiro\textunderscore .
O mesmo que \textunderscore vintaneiro\textunderscore .
\section{Vinteneira}
\begin{itemize}
\item {Grp. gram.:f.}
\end{itemize}
\begin{itemize}
\item {Utilização:Ant.}
\end{itemize}
\begin{itemize}
\item {Proveniência:(De \textunderscore vinteno\textunderscore )}
\end{itemize}
Recenseamento dos mancebos capazes de pegar em armas e servir a bordo, no qual se tomava, de cada vinte, um, á proporção que eram precisos.
\section{Vinteneiro}
\begin{itemize}
\item {Grp. gram.:m.}
\end{itemize}
\begin{itemize}
\item {Utilização:Ant.}
\end{itemize}
\begin{itemize}
\item {Grp. gram.:Adj.}
\end{itemize}
\begin{itemize}
\item {Proveniência:(De \textunderscore vinteno\textunderscore )}
\end{itemize}
Commandante de vinte homens.
Magistrado popular das vintenas ou grupos de vinte fogos.
O mesmo que \textunderscore vintanneiro\textunderscore .
\section{Vinteno}
\begin{itemize}
\item {Grp. gram.:adj.}
\end{itemize}
\begin{itemize}
\item {Proveniência:(De \textunderscore vinte\textunderscore )}
\end{itemize}
Vigésimo; vintanneiro.
Diz-se do pano que tem 2:000 fios de urdidura.
\section{Vinte-ocheno}
\begin{itemize}
\item {Grp. gram.:adj.}
\end{itemize}
\begin{itemize}
\item {Utilização:Ant.}
\end{itemize}
\begin{itemize}
\item {Proveniência:(Do cast. \textunderscore veinteocheno\textunderscore )}
\end{itemize}
Dizia-se do pano, que tem dois mil e oito centos fios de urdidura. Cf. Soropita, \textunderscore Poes. e Pros.\textunderscore , 60.
O mesmo que \textunderscore vinte-e-ocheno\textunderscore .
\section{Vintequatreno}
\begin{itemize}
\item {Grp. gram.:adj.}
\end{itemize}
\begin{itemize}
\item {Proveniência:(De \textunderscore vinte\textunderscore  + \textunderscore quatreno\textunderscore )}
\end{itemize}
O mesmo que \textunderscore vinte-e-quatreno\textunderscore . Cf. Soropita, \textunderscore Poes. e Pros.\textunderscore , 60.
\section{Vintequatria}
\begin{itemize}
\item {Grp. gram.:f.}
\end{itemize}
Agrupamento ou direitos dos vinte-e-quatro.
\section{Vinteseiseno}
\begin{itemize}
\item {fónica:sei}
\end{itemize}
\begin{itemize}
\item {Grp. gram.:adj.}
\end{itemize}
\begin{itemize}
\item {Proveniência:(De \textunderscore vinte\textunderscore  + \textunderscore seis\textunderscore )}
\end{itemize}
Dizia-se do pano que tem dois mil e seiscentos fios de urdidura.
\section{Vintesseiseno}
\begin{itemize}
\item {Grp. gram.:adj.}
\end{itemize}
\begin{itemize}
\item {Proveniência:(De \textunderscore vinte\textunderscore  + \textunderscore seis\textunderscore )}
\end{itemize}
Dizia-se do pano que tem dois mil e seiscentos fios de urdidura.
\section{Vintilho}
\begin{itemize}
\item {Grp. gram.:m.}
\end{itemize}
\begin{itemize}
\item {Utilização:Prov.}
\end{itemize}
\begin{itemize}
\item {Utilização:minh.}
\end{itemize}
Fita vermelha que, no primeiro de Março, as mulheres põem no ombro, para que as não tisne o sol daquelle mês.
(Relaciona-se com \textunderscore bentinho\textunderscore ?)
\section{Vintista}
\begin{itemize}
\item {Grp. gram.:m.}
\end{itemize}
Partidário da revolução de 1820.
\section{Vĩo}
\begin{itemize}
\item {Grp. gram.:m.}
\end{itemize}
\begin{itemize}
\item {Utilização:Ant.}
\end{itemize}
O mesmo que \textunderscore vinho\textunderscore .
\section{Viola}
\begin{itemize}
\item {Grp. gram.:f.}
\end{itemize}
\begin{itemize}
\item {Utilização:Ant.}
\end{itemize}
\begin{itemize}
\item {Proveniência:(Do b. lat. \textunderscore vitula\textunderscore )}
\end{itemize}
Instrumento músico de cordas, de sons mais baixos que os da guitarra, e de caixa em fórma de 8.
Peixe do Algarve.
\textunderscore Viola de arco\textunderscore , o mesmo que \textunderscore rabeca\textunderscore . Cf. Pant. de Aveiro, \textunderscore Itiner.\textunderscore , 25 v.^o e 28 v.^o, (2.^a ed.).
\section{Viola}
\begin{itemize}
\item {Grp. gram.:f.}
\end{itemize}
\begin{itemize}
\item {Proveniência:(Lat. \textunderscore viola\textunderscore )}
\end{itemize}
O mesmo que \textunderscore violeta\textunderscore ^2.
\section{Víola}
\begin{itemize}
\item {Grp. gram.:f.}
\end{itemize}
\begin{itemize}
\item {Proveniência:(Lat. \textunderscore viola\textunderscore )}
\end{itemize}
(que é prosódia authêntica, mas desusada)
O mesmo que \textunderscore violeta\textunderscore ^2.
\section{Viola}
\begin{itemize}
\item {Grp. gram.:f.}
\end{itemize}
Peixe de Portugal.
\section{Violabilidade}
\begin{itemize}
\item {Grp. gram.:f.}
\end{itemize}
Qualidade de violável. Cf. Camillo, \textunderscore Narcót.\textunderscore , I, 37.
\section{Violação}
\begin{itemize}
\item {Grp. gram.:f.}
\end{itemize}
\begin{itemize}
\item {Proveniência:(Do lat. \textunderscore violatio\textunderscore )}
\end{itemize}
Acto ou effeito de violar.
Estupro.
\section{Violáceas}
\begin{itemize}
\item {Grp. gram.:f. pl.}
\end{itemize}
Família de plantas, que tem por typo a violeta.
(Fem. pl. de \textunderscore violáceo\textunderscore )
\section{Violáceo}
\begin{itemize}
\item {Grp. gram.:adj.}
\end{itemize}
\begin{itemize}
\item {Proveniência:(Lat. \textunderscore violaceus\textunderscore )}
\end{itemize}
O mesmo que \textunderscore violete\textunderscore .
Relativo ou semelhante á violeta.
\section{Viola-de-amor}
\begin{itemize}
\item {Grp. gram.:f.}
\end{itemize}
Instrumento de sete cordas, espécie de violeta^2.
\section{Viola-de-arco}
\begin{itemize}
\item {Grp. gram.:f.}
\end{itemize}
\begin{itemize}
\item {Utilização:Ant.}
\end{itemize}
O mesmo que \textunderscore rabeca\textunderscore .
\section{Violador}
\begin{itemize}
\item {Grp. gram.:m.  e  adj.}
\end{itemize}
\begin{itemize}
\item {Proveniência:(Do lat. \textunderscore violator\textunderscore )}
\end{itemize}
O que viola ou violou.
\section{Viola-francesa}
\begin{itemize}
\item {Grp. gram.:f.}
\end{itemize}
O mesmo que \textunderscore violão\textunderscore .
\section{Violal}
\begin{itemize}
\item {Grp. gram.:f.}
\end{itemize}
\begin{itemize}
\item {Proveniência:(De \textunderscore viola\textunderscore ^2)}
\end{itemize}
Terreno, onde crescem violetas.
\section{Violana}
\begin{itemize}
\item {Grp. gram.:f.}
\end{itemize}
\begin{itemize}
\item {Proveniência:(Do lat. \textunderscore viola\textunderscore )}
\end{itemize}
Substância mineral roxa.
\section{Violão}
\begin{itemize}
\item {Grp. gram.:m.}
\end{itemize}
\begin{itemize}
\item {Proveniência:(De \textunderscore viola\textunderscore ^1)}
\end{itemize}
Viola grande, com seis cordas, três das quaes são de tripa, e bordões as outras três.
\section{Violar}
\begin{itemize}
\item {Grp. gram.:v. t.}
\end{itemize}
\begin{itemize}
\item {Proveniência:(Lat. \textunderscore violare\textunderscore )}
\end{itemize}
Offender violentamente.
Transgredir: \textunderscore violar as leis\textunderscore .
Forçar.
Polluir.
Attentar contra o pudor de.
Profanar.
Devassar ou divulgar abusivamente: \textunderscore violar segredos\textunderscore .
\section{Violar}
\begin{itemize}
\item {Grp. gram.:m.}
\end{itemize}
Espécie de jôgo popular.
\section{Violáreas}
\begin{itemize}
\item {Grp. gram.:f. pl.}
\end{itemize}
O mesmo que \textunderscore violáceas\textunderscore .
\section{Violaríneas}
\begin{itemize}
\item {Grp. gram.:f. pl.}
\end{itemize}
O mesmo que \textunderscore violáceas\textunderscore .
\section{Violável}
\begin{itemize}
\item {Grp. gram.:adj.}
\end{itemize}
\begin{itemize}
\item {Proveniência:(Do lat. \textunderscore violabilis\textunderscore )}
\end{itemize}
Que se póde violar.
\section{Vióleas}
\begin{itemize}
\item {Grp. gram.:f. pl.}
\end{itemize}
\begin{itemize}
\item {Utilização:Bot.}
\end{itemize}
\begin{itemize}
\item {Proveniência:(Do lat. \textunderscore viola\textunderscore )}
\end{itemize}
Tríbo de violáceas.
\section{Violeiro}
\begin{itemize}
\item {Grp. gram.:m.}
\end{itemize}
Fabricante ou vendedor de violas^1.
\section{Violência}
\begin{itemize}
\item {Grp. gram.:f.}
\end{itemize}
\begin{itemize}
\item {Proveniência:(Lat. \textunderscore violentia\textunderscore )}
\end{itemize}
Qualidade do que é violento.
Acto violento.
Acto de violentar.
\section{Violentador}
\begin{itemize}
\item {Grp. gram.:m.  e  adj.}
\end{itemize}
O que violenta.
\section{Violentamente}
\begin{itemize}
\item {Grp. gram.:adv.}
\end{itemize}
De modo violento.
\section{Violentar}
\begin{itemize}
\item {Grp. gram.:v. t.}
\end{itemize}
\begin{itemize}
\item {Proveniência:(De \textunderscore violento\textunderscore )}
\end{itemize}
Exercer violência sôbre.
Forçar; constranger.
Violar.
Arrombar: \textunderscore violentar uma porta\textunderscore .
Inverter, alterar.
\section{Violento}
\begin{itemize}
\item {Grp. gram.:adj.}
\end{itemize}
\begin{itemize}
\item {Proveniência:(Lat. \textunderscore violentus\textunderscore )}
\end{itemize}
Que procede com ímpeto.
Que se exerce com fôrça: \textunderscore intimação violenta\textunderscore .
Tumultuoso.
Intenso.
Irascível.
Em que há emprêgo de fôrça brutal.
Opposto ao direito ou á justiça.
\section{Violeta}
\begin{itemize}
\item {fónica:lê}
\end{itemize}
\begin{itemize}
\item {Grp. gram.:f.}
\end{itemize}
\begin{itemize}
\item {Proveniência:(De \textunderscore viola\textunderscore ^2)}
\end{itemize}
Planta aromática, (\textunderscore viola odorata\textunderscore ).
A flôr dessa planta.
Variedade de figueira algarvia.
\section{Violeta}
\begin{itemize}
\item {fónica:lê}
\end{itemize}
\begin{itemize}
\item {Grp. gram.:f.}
\end{itemize}
\begin{itemize}
\item {Proveniência:(De \textunderscore viola\textunderscore ^1)}
\end{itemize}
Espécie de rabeca, maior que a rabeca commum.
\textunderscore Violeta brava\textunderscore , o mesmo que \textunderscore benefe\textunderscore , (\textunderscore viola canina\textunderscore ).
\section{Violeta-do-pará}
\begin{itemize}
\item {Grp. gram.:f.}
\end{itemize}
O mesmo que \textunderscore rasteirinha\textunderscore .
\section{Violeta-tricolor}
\begin{itemize}
\item {Grp. gram.:f.}
\end{itemize}
Planta, o mesmo que \textunderscore amor-perfeito\textunderscore .
\section{Violete}
\begin{itemize}
\item {Grp. gram.:adj.}
\end{itemize}
\begin{itemize}
\item {Utilização:Gal}
\end{itemize}
(V.roxo)
\section{Viólico}
\begin{itemize}
\item {Grp. gram.:adj.}
\end{itemize}
\begin{itemize}
\item {Proveniência:(De \textunderscore viola\textunderscore ^2)}
\end{itemize}
Diz-se de um ácido crystallino, extrahido das pétalas sêcas da violeta.
\section{Violina}
\begin{itemize}
\item {Grp. gram.:f.}
\end{itemize}
\begin{itemize}
\item {Utilização:Chím.}
\end{itemize}
Base ou substância, que existe na violeta.
\section{Violinha}
\begin{itemize}
\item {Grp. gram.:f.}
\end{itemize}
\begin{itemize}
\item {Utilização:Des.}
\end{itemize}
O mesmo que \textunderscore violino\textunderscore .
\section{Violinista}
\begin{itemize}
\item {Grp. gram.:m.  e  f.}
\end{itemize}
\begin{itemize}
\item {Proveniência:(De \textunderscore violino\textunderscore )}
\end{itemize}
O mesmo que \textunderscore rabequista\textunderscore .
\section{Violino}
\begin{itemize}
\item {Grp. gram.:m.}
\end{itemize}
\begin{itemize}
\item {Proveniência:(T. it.)}
\end{itemize}
O mesmo que \textunderscore rabeca\textunderscore .
Tocador dêsse instrumento.
\section{Violoncelista}
\begin{itemize}
\item {Grp. gram.:m.  e  f.}
\end{itemize}
Pessôa, que toca violoncelo.
\section{Violoncellista}
\begin{itemize}
\item {Grp. gram.:m.  e  f.}
\end{itemize}
Pessôa, que toca violoncello.
\section{Violoncello}
\begin{itemize}
\item {Grp. gram.:m.}
\end{itemize}
\begin{itemize}
\item {Proveniência:(It. \textunderscore violoncello\textunderscore )}
\end{itemize}
Instrumento, do feitio da rabeca, mas maior que a violeta e menor que o rabecão.
Tocador dêsse instrumento.
Registo nos órgãos e harmónios.
\section{Violoncelo}
\begin{itemize}
\item {Grp. gram.:m.}
\end{itemize}
\begin{itemize}
\item {Proveniência:(It. \textunderscore violoncello\textunderscore )}
\end{itemize}
Instrumento, do feitio da rabeca, mas maior que a violeta e menor que o rabecão.
Tocador dêsse instrumento.
Registo nos órgãos e harmónios.
\section{Viomal}
\begin{itemize}
\item {Grp. gram.:m.}
\end{itemize}
Planta, da fam. das compostas, o mesmo que \textunderscore làvapé\textunderscore , (\textunderscore centaurea sempervirens\textunderscore , Lin.).
\section{Vioneira}
\begin{itemize}
\item {Grp. gram.:f.}
\end{itemize}
\begin{itemize}
\item {Utilização:Pesc.}
\end{itemize}
Nome de um cabo, que se usa nas baleeiras, e ao qual se prendem arpões, distanciados oito braças entre si. Cf. \textunderscore Jorn.-do-Comm.\textunderscore , do Rio, de 27-VI-900.
\section{Vipéreo}
\begin{itemize}
\item {Grp. gram.:adj.}
\end{itemize}
\begin{itemize}
\item {Proveniência:(Lat. \textunderscore vipereus\textunderscore )}
\end{itemize}
O mesmo que \textunderscore viperino\textunderscore .
\section{Viperina}
\begin{itemize}
\item {Grp. gram.:f.}
\end{itemize}
Planta, o mesmo que \textunderscore soagem\textunderscore .
\section{Viperino}
\begin{itemize}
\item {Grp. gram.:adj.}
\end{itemize}
\begin{itemize}
\item {Utilização:Fig.}
\end{itemize}
\begin{itemize}
\item {Proveniência:(Lat. \textunderscore viperinus\textunderscore )}
\end{itemize}
Relativo ou semelhante á víbora.
Que tem a natureza da víbora.
Venenoso.
Mordaz.
Perverso; maléfico.
\section{Vípero}
\begin{itemize}
\item {Grp. gram.:adj.}
\end{itemize}
\begin{itemize}
\item {Utilização:Des.}
\end{itemize}
O mesmo que \textunderscore viperino\textunderscore .
\section{Vir}
\begin{itemize}
\item {Grp. gram.:v. i.}
\end{itemize}
\begin{itemize}
\item {Proveniência:(Do lat. \textunderscore venire\textunderscore )}
\end{itemize}
Transportar-se de um lugar para aquelle em que estamos, ou em que está a pessôa, a quem falamos.
Andar para cá.
Chegar.
Sêr trazido: \textunderscore veio uma carta do Reinaldo\textunderscore .
Regressar.
Succeder, realizar-se.
Surgir, apparecer: \textunderscore quando veio o sol\textunderscore .
Descer.
Transmittir-se através dos tempos.
Intervir.
Concordar:«\textunderscore vieram nisto.\textunderscore »Camillo, \textunderscore Enjeitada\textunderscore , 37.
Adaptar-se.
Convir.
Dimanar: \textunderscore aquella tendência já lhe vinha dos pais\textunderscore .
Proceder.
Descender.
Estar para acontecer ou para chegar: \textunderscore ninguém sabe o que há de vir\textunderscore .
Referir-se.
Allegar-se.
Estar occupado por pessôa ou coisa, que se move ou anda.--É muitas vezes expressão de realce ou de refôrço e usa-se também como auxiliar: \textunderscore vão fazer-se eleições\textunderscore .
\section{Vira}
\begin{itemize}
\item {Grp. gram.:f.}
\end{itemize}
\begin{itemize}
\item {Utilização:Ant.}
\end{itemize}
\begin{itemize}
\item {Proveniência:(Do lat. \textunderscore viriae\textunderscore ?)}
\end{itemize}
Tira de coiro, que se cose entre as solas do calçado, junto á borda destas.
Tira de coiro, com que os bèsteiros revestiam a palma da mão, para armarem as béstas.
\section{Vira}
\begin{itemize}
\item {Grp. gram.:f.}
\end{itemize}
\begin{itemize}
\item {Utilização:Ant.}
\end{itemize}
Espécie de seta muito aguda.
\section{Vira}
\begin{itemize}
\item {Grp. gram.:m.}
\end{itemize}
\begin{itemize}
\item {Proveniência:(De \textunderscore virar\textunderscore )}
\end{itemize}
Dança e música popular.
\section{Virá}
\begin{itemize}
\item {Grp. gram.:m.}
\end{itemize}
Espécie de pequeno veado do Brasil.
\section{Vira-bostas}
\begin{itemize}
\item {Grp. gram.:m.}
\end{itemize}
\begin{itemize}
\item {Utilização:Bras}
\end{itemize}
Ave azul-escura, do tamanho de uma pomba e muito nociva aos milharaes.
\section{Viração}
\begin{itemize}
\item {Grp. gram.:f.}
\end{itemize}
\begin{itemize}
\item {Proveniência:(De \textunderscore virar\textunderscore )}
\end{itemize}
Vento brando e fresco; brisa; aragem.
\section{Viràccento}
\begin{itemize}
\item {Grp. gram.:m.}
\end{itemize}
\begin{itemize}
\item {Proveniência:(De \textunderscore virar\textunderscore  + \textunderscore accento\textunderscore )}
\end{itemize}
O mesmo que \textunderscore apóstropho\textunderscore .
\section{Viràcento}
\begin{itemize}
\item {Grp. gram.:m.}
\end{itemize}
\begin{itemize}
\item {Proveniência:(De \textunderscore virar\textunderscore  + \textunderscore accento\textunderscore )}
\end{itemize}
O mesmo que \textunderscore apóstropho\textunderscore .
\section{Viracu}
\begin{itemize}
\item {Grp. gram.:m.}
\end{itemize}
\begin{itemize}
\item {Utilização:Prov.}
\end{itemize}
Cambalhota, que os rapazes dão, pondo no chão a cabeça e voltando as pernas para o outro lado.
\section{Viradela}
\begin{itemize}
\item {Grp. gram.:f.}
\end{itemize}
Acto de virar.
\section{Viradinho}
\begin{itemize}
\item {Grp. gram.:m.}
\end{itemize}
\begin{itemize}
\item {Utilização:Bras}
\end{itemize}
\begin{itemize}
\item {Proveniência:(De \textunderscore virado\textunderscore )}
\end{itemize}
Iguaria, feita de feijão, torresmos, farinha e ovos.
\section{Virado}
\begin{itemize}
\item {Grp. gram.:m.}
\end{itemize}
\begin{itemize}
\item {Utilização:Bras}
\end{itemize}
\begin{itemize}
\item {Proveniência:(De \textunderscore virar\textunderscore )}
\end{itemize}
O mesmo que \textunderscore viradinho\textunderscore .
\section{Virador}
\begin{itemize}
\item {Grp. gram.:m.}
\end{itemize}
\begin{itemize}
\item {Proveniência:(De \textunderscore virar\textunderscore )}
\end{itemize}
Cabo náutico, próprio para reboques.
Cabo, em que se prende o pêso que se move com cabrestante.
Utensílio, com que os encadernadores doiram as capas dos livros.
\section{Viragem}
\begin{itemize}
\item {Grp. gram.:f.}
\end{itemize}
\begin{itemize}
\item {Utilização:Phot.}
\end{itemize}
Mudança na direcção dos automóveis. Cf. Benevides, \textunderscore Automóveis\textunderscore .
Primeiro banho das provas photográphicas.
\section{Virago}
\begin{itemize}
\item {Grp. gram.:f.}
\end{itemize}
\begin{itemize}
\item {Proveniência:(Lat. \textunderscore virago\textunderscore )}
\end{itemize}
Mulhér robusta ou de maneiras varonis.
\section{Viramento}
\begin{itemize}
\item {Grp. gram.:m.}
\end{itemize}
Acto ou effeito de virar.
\section{Virão}
\begin{itemize}
\item {Grp. gram.:m.}
\end{itemize}
\begin{itemize}
\item {Utilização:Marn.}
\end{itemize}
Buraco, por onde a água sai das caldeiras para os corredores. Cf. \textunderscore Museu Technol.\textunderscore , 105.
\section{Vira-pedras}
\begin{itemize}
\item {Grp. gram.:m.}
\end{itemize}
Ave madeirense, (\textunderscore strepsilas interpres\textunderscore ).
\section{Virar}
\begin{itemize}
\item {Grp. gram.:v. t.}
\end{itemize}
\begin{itemize}
\item {Grp. gram.:V. i.}
\end{itemize}
Mudar de um lado para outro a direcção ou posição de: \textunderscore virar uma pedra\textunderscore .
Desvirar, pôr do avesso: \textunderscore virar um casaco\textunderscore .
Voltar para um lado.
Voltar para trás ou para o lado: \textunderscore virar a cara\textunderscore .
Voltar para cima.
Voltar para a frente o lado posterior de: \textunderscore virar as costas\textunderscore .
Dirigir.
Despejar.
Converter; transformar.
Mudar de direcção.
Voltar-se.
Insurgir-se.
Estar voltado.
Defrontar.
Mudar de opinião ou de systema.
(B. lat. \textunderscore virare\textunderscore )
\section{Vira-teimão}
\begin{itemize}
\item {Grp. gram.:m.}
\end{itemize}
\begin{itemize}
\item {Utilização:Bras. de Minas}
\end{itemize}
Azáfama; confusão.
\section{Viravolta}
\begin{itemize}
\item {Grp. gram.:f.}
\end{itemize}
\begin{itemize}
\item {Utilização:Fig.}
\end{itemize}
\begin{itemize}
\item {Proveniência:(De \textunderscore virar\textunderscore  + \textunderscore voltar\textunderscore )}
\end{itemize}
Volta completa.
Cambalhota.
Vicissitude.
\section{Viravoltar}
\begin{itemize}
\item {Grp. gram.:v. i.}
\end{itemize}
\begin{itemize}
\item {Utilização:bras}
\end{itemize}
\begin{itemize}
\item {Utilização:Neol.}
\end{itemize}
Dar viravoltas.
\section{Virente}
\begin{itemize}
\item {Grp. gram.:adj.}
\end{itemize}
\begin{itemize}
\item {Utilização:Fig.}
\end{itemize}
\begin{itemize}
\item {Proveniência:(Lat. \textunderscore virens\textunderscore )}
\end{itemize}
Que verdeja; verde.
Próspero; magnífico; florescente.
\section{Virga}
\begin{itemize}
\item {Grp. gram.:f.}
\end{itemize}
O mesmo que \textunderscore vêrga\textunderscore .
\section{Virga-férrea}
\begin{itemize}
\item {Grp. gram.:f.}
\end{itemize}
\begin{itemize}
\item {Proveniência:(Do lat. \textunderscore virga\textunderscore  + \textunderscore ferreus\textunderscore )}
\end{itemize}
Grande violência.
Severidade extrema.
Emprêgo da fôrça.
\section{Virgáurea}
\begin{itemize}
\item {Grp. gram.:f.}
\end{itemize}
Planta, o mesmo que \textunderscore vara-de-oiro\textunderscore .
\section{Virgem}
\begin{itemize}
\item {Grp. gram.:f.}
\end{itemize}
\begin{itemize}
\item {Utilização:Restrict.}
\end{itemize}
\begin{itemize}
\item {Utilização:Ext.}
\end{itemize}
\begin{itemize}
\item {Grp. gram.:M.}
\end{itemize}
\begin{itemize}
\item {Grp. gram.:Adj.}
\end{itemize}
\begin{itemize}
\item {Grp. gram.:F. pl.}
\end{itemize}
\begin{itemize}
\item {Proveniência:(Do lat. \textunderscore virgo\textunderscore )}
\end{itemize}
Mulhér ou menina, isenta de relações carnaes com homem.
Donzella.
A mãe de Christo.
Retrato da mãe de Christo.
Rapariga.
Aquelle que é virgem:«\textunderscore vós\textunderscore  (Jesus) \textunderscore sois advogado das virgens, porque fostes puríssimo virgem\textunderscore ». \textunderscore Luz e Calor\textunderscore , 604.
Puro; casto.
Intacto.
Innocente.
Isento.
Ingênuo.
Sincero.
Que ainda não serviu: \textunderscore espingarda virgem\textunderscore .
Diz-se da primeira cortiça, tirada a um sobreiro. Cf. \textunderscore Port. au point de vue agr.\textunderscore , 649.
Grossas traves de madeira que, enterradas no chão, sustentam os dormentes nos engenhos de açúcar.
\section{Virgéu}
\begin{itemize}
\item {Grp. gram.:m.}
\end{itemize}
\begin{itemize}
\item {Utilização:Ant.}
\end{itemize}
O mesmo que \textunderscore vergel\textunderscore .
\section{Virgília}
\begin{itemize}
\item {Grp. gram.:f.}
\end{itemize}
(V. \textunderscore vergília\textunderscore .)
\section{Virginal}
\begin{itemize}
\item {Grp. gram.:adj.}
\end{itemize}
\begin{itemize}
\item {Proveniência:(Lat. \textunderscore virginalis\textunderscore )}
\end{itemize}
Relativo a virgem; virgem.
\section{Virginalizar}
\begin{itemize}
\item {Grp. gram.:v. t.}
\end{itemize}
Tornar virginal. Cf. Camillo, \textunderscore Narcót.\textunderscore , I, 204.
\section{Virginalmente}
\begin{itemize}
\item {Grp. gram.:adv.}
\end{itemize}
De modo virginal.
\section{Virgindade}
\begin{itemize}
\item {Grp. gram.:f.}
\end{itemize}
\begin{itemize}
\item {Proveniência:(Do lat. \textunderscore virginitas\textunderscore )}
\end{itemize}
Estado ou qualidade de pessôa virgem.
\section{Virgíneo}
\begin{itemize}
\item {Grp. gram.:adj.}
\end{itemize}
\begin{itemize}
\item {Proveniência:(Lat. \textunderscore virgineus\textunderscore )}
\end{itemize}
O mesmo que \textunderscore virginal\textunderscore .
\section{Virgínia}
\begin{itemize}
\item {Grp. gram.:f.}
\end{itemize}
Variedade de tabaco.
\section{Virginismo}
\begin{itemize}
\item {Grp. gram.:m.}
\end{itemize}
O preceito da virgindade:«\textunderscore encarnação do virginismo\textunderscore ». Camillo, \textunderscore Perfil do Marquês\textunderscore , 248.
\section{Virginizar}
\begin{itemize}
\item {Grp. gram.:v. t.}
\end{itemize}
\begin{itemize}
\item {Utilização:Neol.}
\end{itemize}
Dar o carácter de virgem a; purificar. Cf. Camillo, \textunderscore Cancion. Al.\textunderscore , I, 7.
\section{Virgo}
\begin{itemize}
\item {Grp. gram.:m.}
\end{itemize}
\begin{itemize}
\item {Utilização:Chul.}
\end{itemize}
\begin{itemize}
\item {Proveniência:(Lat. \textunderscore virgo\textunderscore )}
\end{itemize}
Um dos signos de zodíaco.
Virgindade da mulhér.
\section{Virgueiro}
\begin{itemize}
\item {Grp. gram.:adj.}
\end{itemize}
\begin{itemize}
\item {Utilização:Chul.}
\end{itemize}
Que ainda tem o virgo.
\section{Vírgula}
\begin{itemize}
\item {Grp. gram.:f.}
\end{itemize}
\begin{itemize}
\item {Proveniência:(Lat. \textunderscore virgula\textunderscore )}
\end{itemize}
Sinal orthográphico, que indica a menor de todas as pausas.
\section{Virgulação}
\begin{itemize}
\item {Grp. gram.:f.}
\end{itemize}
Acto de virgular.
\section{Virgular}
\begin{itemize}
\item {Grp. gram.:v. t.}
\end{itemize}
\begin{itemize}
\item {Grp. gram.:V. i.}
\end{itemize}
Pôr vírgulas em.
Pontuar.
Pôr vírgulas no lugar próprio.
\section{Virgulária}
\begin{itemize}
\item {Grp. gram.:f.}
\end{itemize}
\begin{itemize}
\item {Proveniência:(Do lat. \textunderscore virgula\textunderscore )}
\end{itemize}
Gênero de pólypos dos mares da Noruega.
\section{Virgulosa}
\begin{itemize}
\item {Grp. gram.:f.  e  adj.}
\end{itemize}
\begin{itemize}
\item {Proveniência:(Do lat. \textunderscore vírgula\textunderscore )}
\end{itemize}
Diz-se de uma casta de pêras sumarentas.
\section{Virgulta}
\begin{itemize}
\item {Grp. gram.:f.}
\end{itemize}
\begin{itemize}
\item {Utilização:Poét.}
\end{itemize}
\begin{itemize}
\item {Proveniência:(Lat. \textunderscore virgulta\textunderscore )}
\end{itemize}
Varinha flexível.
\section{Viricultura}
\begin{itemize}
\item {Grp. gram.:f.}
\end{itemize}
\begin{itemize}
\item {Utilização:Neol.}
\end{itemize}
\begin{itemize}
\item {Proveniência:(Do lat. \textunderscore vir\textunderscore  + \textunderscore cultura\textunderscore )}
\end{itemize}
Sciência, que trata das questões da população, do malthusianismo, da prostituição, etc.
\section{Viridante}
\begin{itemize}
\item {Grp. gram.:adj.}
\end{itemize}
\begin{itemize}
\item {Proveniência:(Lat. \textunderscore viridans\textunderscore )}
\end{itemize}
O mesmo que \textunderscore viridente\textunderscore .
\section{Viridário}
\begin{itemize}
\item {Grp. gram.:m.}
\end{itemize}
\begin{itemize}
\item {Utilização:P. us.}
\end{itemize}
\begin{itemize}
\item {Proveniência:(Lat. \textunderscore viridarium\textunderscore )}
\end{itemize}
O mesmo que \textunderscore jardim\textunderscore .
\section{Víride}
\begin{itemize}
\item {Grp. gram.:adj.}
\end{itemize}
\begin{itemize}
\item {Utilização:Poét.}
\end{itemize}
O mesmo que \textunderscore verde\textunderscore . Cf. Th. Ribeiro, \textunderscore Jornadas\textunderscore , I, 401.
\section{Viridente}
\begin{itemize}
\item {Grp. gram.:adj.}
\end{itemize}
\begin{itemize}
\item {Proveniência:(Do lat. \textunderscore viridis\textunderscore )}
\end{itemize}
O mesmo que \textunderscore virente\textunderscore .
\section{Viril}
\begin{itemize}
\item {Grp. gram.:adj.}
\end{itemize}
\begin{itemize}
\item {Proveniência:(Lat. \textunderscore virilis\textunderscore )}
\end{itemize}
Relativo ao homem.
Próprio de homem; varonil.
Esforçado; enérgico.
\section{Viril}
\begin{itemize}
\item {Grp. gram.:m.}
\end{itemize}
Espécie de âmbula ou redoma de vidro, em que se guardam relíquias ou objecto valiosos.
(Por \textunderscore vidril\textunderscore , de \textunderscore vidro\textunderscore )
\section{Virilha}
\begin{itemize}
\item {Grp. gram.:f.}
\end{itemize}
\begin{itemize}
\item {Proveniência:(Do lat. \textunderscore virilia\textunderscore )}
\end{itemize}
Ponto de juncção da coxa com o ventre.
\section{Viriliana}
\begin{itemize}
\item {Grp. gram.:f.}
\end{itemize}
Planta da serra de Sintra.
\section{Virilidade}
\begin{itemize}
\item {Grp. gram.:f.}
\end{itemize}
\begin{itemize}
\item {Proveniência:(Do lat. \textunderscore virilitas\textunderscore )}
\end{itemize}
Qualidade do que é viril.
Idade do homem, entre a adolescência e a velhice.
\section{Virilmente}
\begin{itemize}
\item {Grp. gram.:adv.}
\end{itemize}
De modo viril; esforçadamente; com coragem.
\section{Viripotente}
\begin{itemize}
\item {Grp. gram.:adj.}
\end{itemize}
\begin{itemize}
\item {Proveniência:(Lat. \textunderscore viripotens\textunderscore )}
\end{itemize}
O mesmo que \textunderscore núbil\textunderscore .
Que póde casar, (falando-se de indivíduos do sexo feminino).
\section{Viripotente}
\begin{itemize}
\item {Grp. gram.:adj.}
\end{itemize}
\begin{itemize}
\item {Proveniência:(Lat. \textunderscore viripotens\textunderscore )}
\end{itemize}
Robusto; varonil.
Que tem muita fôrça.
\section{Viró}
\begin{itemize}
\item {Grp. gram.:m.}
\end{itemize}
Gênero de árvores, da ilha de San-Thomé.
\section{Virola}
\begin{itemize}
\item {Grp. gram.:f.}
\end{itemize}
\begin{itemize}
\item {Proveniência:(Do lat. \textunderscore viriola\textunderscore )}
\end{itemize}
Arco de metal, para apertar ou reforçar um objecto, e algumas vezes para ornato.
\section{Viroso}
\begin{itemize}
\item {Grp. gram.:adj.}
\end{itemize}
\begin{itemize}
\item {Proveniência:(Lat. \textunderscore virosus\textunderscore )}
\end{itemize}
Que tem vírus ou veneno.
Venenoso.
Nocivo.
Nauseabundo.
Que tem mau cheiro.
\section{Virotaço}
\begin{itemize}
\item {Grp. gram.:m.}
\end{itemize}
O mesmo que \textunderscore virotão\textunderscore . Cf. R. Jorge, \textunderscore Mal do Bicho\textunderscore , 9.
\section{Virotada}
\begin{itemize}
\item {Grp. gram.:f.}
\end{itemize}
Ferimento, feito com virote.
\section{Virotão}
\begin{itemize}
\item {Grp. gram.:m.}
\end{itemize}
Grande virote.
\section{Virote}
\begin{itemize}
\item {Grp. gram.:m.}
\end{itemize}
\begin{itemize}
\item {Utilização:Náut.}
\end{itemize}
\begin{itemize}
\item {Utilização:Pop.}
\end{itemize}
\begin{itemize}
\item {Utilização:T. da Bairrada}
\end{itemize}
\begin{itemize}
\item {Utilização:Bras}
\end{itemize}
\begin{itemize}
\item {Proveniência:(De \textunderscore vira\textunderscore ^2)}
\end{itemize}
Seta curta.
Cada uma das peças, que constituem o remate do navio.
Travéssa de ferro, nos copos das antigas espadas.
Haste quadrada, que era a peça principal da balestilha, instrumento náutico.
Pessôa remexida e pouco circunspecta.
Pessôa de elevada estatura.
Espécie de loireiro.
\section{Virtal}
\begin{itemize}
\item {Grp. gram.:m.}
\end{itemize}
\begin{itemize}
\item {Proveniência:(De \textunderscore virte\textunderscore )}
\end{itemize}
Aquelle que pagava avença, (na Índia portuguesa).
Avençal.
\section{Virte}
\begin{itemize}
\item {Grp. gram.:m.}
\end{itemize}
Relação dos virtaes, na Índia portuguesa.
\section{Virtual}
\begin{itemize}
\item {Grp. gram.:adj.}
\end{itemize}
\begin{itemize}
\item {Proveniência:(It. \textunderscore virtuale\textunderscore )}
\end{itemize}
Que existe como faculdade, mas sem exercício ou effeito actual.
Possível; susceptível de se realizar.
Potencial.
Diz-se do fóco de um espelho, determinado pelo encontro dos prolongamentos geométricos dos raios luminosos.
\section{Virtualidade}
\begin{itemize}
\item {Grp. gram.:f.}
\end{itemize}
Qualidade do que é virtual.
\section{Virtualmente}
\begin{itemize}
\item {Grp. gram.:adv.}
\end{itemize}
De modo virtual.
\section{Virtude}
\begin{itemize}
\item {Grp. gram.:f.}
\end{itemize}
\begin{itemize}
\item {Grp. gram.:Pl.}
\end{itemize}
\begin{itemize}
\item {Proveniência:(Do lat. \textunderscore virtus\textunderscore )}
\end{itemize}
Fôrça moral.
Disposição firme e habitual para a prática do bem.
Bôa qualidade moral.
Acto virtuoso.
Castidade.
Modo austero de vida.
Propriedade.
Qualidade própria para a producção de certos effeitos: \textunderscore o íman tem a virtude de attrahir\textunderscore .
Efficácia.
Validade.
Motivo: \textunderscore foi condemnado em virtude de perseguições\textunderscore .
Um dos córos ou categorias dos anjos, segundo a Theologia.
\section{Virtuosamente}
\begin{itemize}
\item {Grp. gram.:adv.}
\end{itemize}
De modo virtuoso.
\section{Virtuosidade}
\begin{itemize}
\item {Grp. gram.:f.}
\end{itemize}
\begin{itemize}
\item {Utilização:Mús.}
\end{itemize}
Qualidade de virtuoso^2.
\section{Virtuoso}
\begin{itemize}
\item {Grp. gram.:adj.}
\end{itemize}
\begin{itemize}
\item {Utilização:Ant.}
\end{itemize}
\begin{itemize}
\item {Proveniência:(Lat. \textunderscore virtuosus\textunderscore )}
\end{itemize}
Que tem virtude ou virtudes.
Efficaz.
Vigoroso, esforçado.
\section{Virtuoso}
\begin{itemize}
\item {Grp. gram.:m.}
\end{itemize}
Músico de grande talento.
Amador de música. Cf. \textunderscore Hyssope\textunderscore , 103; Camillo, \textunderscore Noite de Insómn.\textunderscore , IX, 34.
\section{Virulência}
\begin{itemize}
\item {Grp. gram.:f.}
\end{itemize}
\begin{itemize}
\item {Proveniência:(Lat. \textunderscore virulentia\textunderscore )}
\end{itemize}
Qualidade ou estado do que é virulento.
\section{Virulentamente}
\begin{itemize}
\item {Grp. gram.:adv.}
\end{itemize}
De modo virulento; com virulência.
\section{Virulento}
\begin{itemize}
\item {Grp. gram.:adj.}
\end{itemize}
\begin{itemize}
\item {Utilização:Fig.}
\end{itemize}
\begin{itemize}
\item {Proveniência:(Lat. \textunderscore virulentus\textunderscore )}
\end{itemize}
Que tem vírus ou veneno.
Que é da natureza do vírus.
Causado por um vírus.
O mesmo que \textunderscore rancoroso\textunderscore .
\section{Vírus}
\begin{itemize}
\item {Grp. gram.:m.}
\end{itemize}
\begin{itemize}
\item {Proveniência:(Lat. \textunderscore virus\textunderscore )}
\end{itemize}
Princípio mórbido, inherente a certas doenças contagiosas: \textunderscore o vírus da diphteria\textunderscore .
Veneno de reptis.
\section{Visagem}
\begin{itemize}
\item {Grp. gram.:f.}
\end{itemize}
\begin{itemize}
\item {Utilização:Ant.}
\end{itemize}
\begin{itemize}
\item {Utilização:Bras. do N}
\end{itemize}
\begin{itemize}
\item {Proveniência:(Fr. \textunderscore visage\textunderscore )}
\end{itemize}
Trejeitos da cara.
Careta.
Cara.
Viseira.
Fantasma; apparição sobrenatural.
\section{Viságia}
\begin{itemize}
\item {Grp. gram.:f.}
\end{itemize}
(V.bisagra)
\section{Visagra}
\begin{itemize}
\item {Grp. gram.:f.}
\end{itemize}
(V.bisagra)
\section{Visão}
\begin{itemize}
\item {Grp. gram.:m.}
\end{itemize}
\begin{itemize}
\item {Proveniência:(Lat. \textunderscore visio\textunderscore )}
\end{itemize}
Acto ou effeito de vêr.
Aspecto.
Imagem van, que se julga vêr em sonhos, ou por mêdo, por loucura, por superstição, etc.
Fantasia.
Anhelo.
Chímera.
\section{Visar}
\begin{itemize}
\item {Grp. gram.:v. t.}
\end{itemize}
\begin{itemize}
\item {Grp. gram.:V. i.}
\end{itemize}
Dirigir o olhar para.
Apontar uma arma de fogo contra.
Pôr o sinal de visto em: \textunderscore visar um diploma\textunderscore .
Têr em mira um fim.
Tender; dispor-se.
(Cp. lat. \textunderscore visere\textunderscore  )
\section{Visato-chôco}
\begin{itemize}
\item {Grp. gram.:m.}
\end{itemize}
Planta da serra de Sintra.
\section{Víscera}
\begin{itemize}
\item {Grp. gram.:f.}
\end{itemize}
\begin{itemize}
\item {Grp. gram.:Pl.}
\end{itemize}
\begin{itemize}
\item {Utilização:Fig.}
\end{itemize}
\begin{itemize}
\item {Proveniência:(Lat. \textunderscore viscera\textunderscore )}
\end{itemize}
Designação genérica de qualquer órgão, alojado em uma das três cavidades, a craniana, a thorácica e a abdominal.
Entranhas; intestinos.
A parte mais íntima de qualquer coisa.
\section{Visceral}
\begin{itemize}
\item {Grp. gram.:adj.}
\end{itemize}
\begin{itemize}
\item {Proveniência:(Lat. \textunderscore visceralis\textunderscore )}
\end{itemize}
Relativo ás vísceras.
\section{Visceralmente}
\begin{itemize}
\item {Grp. gram.:adv.}
\end{itemize}
De modo visceral.
\section{Visceroso}
\begin{itemize}
\item {Grp. gram.:adj.}
\end{itemize}
O mesmo que \textunderscore visceral\textunderscore .
\section{Viscidez}
\begin{itemize}
\item {Grp. gram.:f.}
\end{itemize}
\begin{itemize}
\item {Proveniência:(De \textunderscore víscido\textunderscore )}
\end{itemize}
O mesmo que \textunderscore viscosidade\textunderscore .
\section{Víscido}
\begin{itemize}
\item {Grp. gram.:adj.}
\end{itemize}
\begin{itemize}
\item {Proveniência:(Lat. \textunderscore viscidus\textunderscore )}
\end{itemize}
O mesmo que \textunderscore viscoso\textunderscore .
\section{Viscina}
\begin{itemize}
\item {Grp. gram.:f.}
\end{itemize}
\begin{itemize}
\item {Utilização:Chím.}
\end{itemize}
Substância essencial do viscívoro.
\section{Viscívoro}
\begin{itemize}
\item {Grp. gram.:adj.}
\end{itemize}
\begin{itemize}
\item {Utilização:Zool.}
\end{itemize}
\begin{itemize}
\item {Proveniência:(Do lat. \textunderscore viscum\textunderscore  + \textunderscore vorare\textunderscore )}
\end{itemize}
Que come os frutos do visco.
\section{Visco}
\begin{itemize}
\item {Grp. gram.:m.}
\end{itemize}
\begin{itemize}
\item {Utilização:Fig.}
\end{itemize}
\begin{itemize}
\item {Proveniência:(Lat. \textunderscore viscum\textunderscore )}
\end{itemize}
Planta parasita, da fam. das lorantháceas.
Agárico.
Suco glutinoso, com que se envolvem pequenas varas, para apanhar pássaros.
Isca, engôdo, chamariz.
\section{Viscondado}
\begin{itemize}
\item {Grp. gram.:m.}
\end{itemize}
\begin{itemize}
\item {Utilização:P. us.}
\end{itemize}
Título ou dignidade de Visconde ou Viscondessa.
Terras ou bens de Visconde ou de Viscondessa.
\section{Visconde}
\begin{itemize}
\item {Grp. gram.:m.}
\end{itemize}
\begin{itemize}
\item {Proveniência:(De \textunderscore vice...\textunderscore  + \textunderscore conde\textunderscore . Cp. \textunderscore vizconde\textunderscore )}
\end{itemize}
Título nobiliárchico, superior ao de Barão e inferior ao de Conde.
Funccionário, que fazia as vezes de Conde, no govêrno do respectivo condado.
Senhor feudal de um território que tinha o título de viscondado.
\section{Viscondessa}
\begin{itemize}
\item {fónica:dê}
\end{itemize}
\begin{itemize}
\item {Grp. gram.:f.}
\end{itemize}
\begin{itemize}
\item {Proveniência:(De \textunderscore vice...\textunderscore  + \textunderscore condessa\textunderscore )}
\end{itemize}
Mulhér ou viúva de Visconde.
Mulhér, que tem o título de viscondado.
\section{Viscondesso}
\begin{itemize}
\item {fónica:dê}
\end{itemize}
\begin{itemize}
\item {Grp. gram.:m.}
\end{itemize}
Designação depreciativa do homem que casou com Viscondessa.
\section{Viscosidade}
\begin{itemize}
\item {Grp. gram.:f.}
\end{itemize}
Qualidade do que é viscoso.
Propriedade, pela qual as partículas de uma substância adherem reciprocamente.
Coisa viscosa.
\section{Viscoso}
\begin{itemize}
\item {Grp. gram.:adj.}
\end{itemize}
\begin{itemize}
\item {Proveniência:(Lat. \textunderscore viscosus\textunderscore )}
\end{itemize}
Que tem visco.
Pegajoso como o visco.
Cujas muléculas adherem umas ás outras.
\section{Viseense}
\begin{itemize}
\item {Grp. gram.:m.  e  adj.}
\end{itemize}
O mesmo ou melhór que \textunderscore visiense\textunderscore .
\section{Viseira}
\begin{itemize}
\item {Grp. gram.:f.}
\end{itemize}
\begin{itemize}
\item {Utilização:Fig.}
\end{itemize}
\begin{itemize}
\item {Utilização:Pop.}
\end{itemize}
\begin{itemize}
\item {Proveniência:(Fr. \textunderscore visiere\textunderscore )}
\end{itemize}
Parte anterior do capacete, que resguarda e defende o rosto.
Pala de boné.
Aquillo que resguarda.
Disfarce.
Aspecto; carranca.
\section{Viseira}
\begin{itemize}
\item {Grp. gram.:f.}
\end{itemize}
\begin{itemize}
\item {Utilização:Prov.}
\end{itemize}
\begin{itemize}
\item {Utilização:minh.}
\end{itemize}
(V.vezeira)
\section{Visgo}
\begin{itemize}
\item {Grp. gram.:m.}
\end{itemize}
O mesmo que \textunderscore visco\textunderscore .
\section{Visgueiro}
\begin{itemize}
\item {Grp. gram.:m.}
\end{itemize}
\begin{itemize}
\item {Proveniência:(Do lat. \textunderscore viscarius\textunderscore )}
\end{itemize}
Árvore leguminosa do Brasil.
\section{Visguento}
\begin{itemize}
\item {Grp. gram.:adj.}
\end{itemize}
O mesmo que \textunderscore viscoso\textunderscore .
\section{Visíbil}
\begin{itemize}
\item {Grp. gram.:adj.}
\end{itemize}
\begin{itemize}
\item {Utilização:Des.}
\end{itemize}
\begin{itemize}
\item {Proveniência:(Lat. \textunderscore visibilis\textunderscore )}
\end{itemize}
O mesmo que \textunderscore visível\textunderscore :«\textunderscore a cujo império obedece o visibil e invisibil.\textunderscore »\textunderscore Lusíadas\textunderscore , I, 65.
\section{Visibilidade}
\begin{itemize}
\item {Grp. gram.:f.}
\end{itemize}
\begin{itemize}
\item {Proveniência:(Do lat. \textunderscore visibilitas\textunderscore )}
\end{itemize}
Qualidade do que é visível.
\section{Visiense}
\begin{itemize}
\item {Grp. gram.:adj.}
\end{itemize}
\begin{itemize}
\item {Grp. gram.:M.}
\end{itemize}
Relativo a Viseu.
Habitante de Viseu.
(Por \textunderscore viseense\textunderscore , de \textunderscore Viseu\textunderscore )
\section{Visigodos}
\begin{itemize}
\item {Grp. gram.:m.}
\end{itemize}
\begin{itemize}
\item {Proveniência:(Lat. \textunderscore Lisigothai\textunderscore )}
\end{itemize}
Godos do Occidente.
\section{Visigótico}
\begin{itemize}
\item {Grp. gram.:adj.}
\end{itemize}
Relativo aos Visigodos.
\section{Visiómetro}
\begin{itemize}
\item {Grp. gram.:m.}
\end{itemize}
\begin{itemize}
\item {Proveniência:(Do lat. \textunderscore visio\textunderscore  + gr. \textunderscore metron\textunderscore )}
\end{itemize}
Instrumento, para indicar o grau da fôrça visual num indivíduo e os vidros ou lunetas que a êste convém.
\section{Visionação}
\begin{itemize}
\item {Grp. gram.:f.}
\end{itemize}
Acto ou effeito de visionar.
\section{Visionar}
\begin{itemize}
\item {Grp. gram.:v. t.}
\end{itemize}
\begin{itemize}
\item {Utilização:Neol.}
\end{itemize}
\begin{itemize}
\item {Grp. gram.:V. i.}
\end{itemize}
Entrever, como em visão.
Têr visões, fantasias.
\section{Visionário}
\begin{itemize}
\item {Grp. gram.:adj.}
\end{itemize}
\begin{itemize}
\item {Grp. gram.:M.}
\end{itemize}
\begin{itemize}
\item {Proveniência:(Do lat. \textunderscore visio\textunderscore , \textunderscore visionis\textunderscore )}
\end{itemize}
Relativo ás visões.
Que tem ideias extravagantes.
Excêntrico.
Aquelle que tem visões ou julga vêr fantasmas.
Devaneador; utopista.
\section{Visionice}
\begin{itemize}
\item {Grp. gram.:f.}
\end{itemize}
\begin{itemize}
\item {Proveniência:(Do lat. \textunderscore visio\textunderscore , \textunderscore visionis\textunderscore )}
\end{itemize}
Visualidade; fantasia:«\textunderscore se vai nessa esteira, perde-se nas restingas da visionice.\textunderscore »Camillo, \textunderscore Caveira\textunderscore , 267.
\section{Visita}
\begin{itemize}
\item {Grp. gram.:f.}
\end{itemize}
\begin{itemize}
\item {Utilização:Fam.}
\end{itemize}
\begin{itemize}
\item {Grp. gram.:Pl.}
\end{itemize}
\begin{itemize}
\item {Proveniência:(De \textunderscore visitar\textunderscore )}
\end{itemize}
O mesmo que \textunderscore visitação\textunderscore .
Acto de ir vêr alguém por dever, cortesia ou affeição.
Pessôa, que visita.
Inspecção: \textunderscore visita sanitária\textunderscore .
Tributo antigo, que consistia num mimo ou presente de comestíveis, e que o emphyteuta pagava ao senhorio.
Catamênio, mênstruo.
Cumprimentos, saudações, lembranças: \textunderscore olha, dá lá muitas visitas a teu irmão\textunderscore .
\section{Visitação}
\begin{itemize}
\item {Grp. gram.:s.}
\end{itemize}
\begin{itemize}
\item {Grp. gram.:Pl.}
\end{itemize}
\begin{itemize}
\item {Utilização:Des.}
\end{itemize}
\begin{itemize}
\item {Utilização:Ant.}
\end{itemize}
\begin{itemize}
\item {Proveniência:(Lat. \textunderscore visitatio\textunderscore )}
\end{itemize}
Acto ou effeito de visitar.
Visita.
Cumprimentos, lembranças, visitas:«\textunderscore ...não a tinha inda visto, mandando-me ella mil visitações e mimos.\textunderscore »\textunderscore Eufrosina\textunderscore , 31.
Espécie de tributo ou foro, pago ao Rei ou ao senhorio.
\section{Visitador}
\begin{itemize}
\item {Grp. gram.:m.  e  adj.}
\end{itemize}
\begin{itemize}
\item {Proveniência:(Do lat. \textunderscore visitator\textunderscore )}
\end{itemize}
O que visita.
\section{Visitandina}
\begin{itemize}
\item {Grp. gram.:f.}
\end{itemize}
Educanda conventual, que póde ser visitada? Cf. Filinto, X, 112.
\section{Visitante}
\begin{itemize}
\item {Grp. gram.:m. ,  f.  e  adj.}
\end{itemize}
\begin{itemize}
\item {Proveniência:(Lat. \textunderscore visitans\textunderscore )}
\end{itemize}
Pessôa, que visita.
\section{Visitar}
\begin{itemize}
\item {Grp. gram.:v. t.}
\end{itemize}
\begin{itemize}
\item {Proveniência:(Lat. \textunderscore visitare\textunderscore )}
\end{itemize}
Ir vêr (alguém) em sua casa.
Ir vêr por caridade, devoção, cortesia ou dever: \textunderscore visitar os encarcerados\textunderscore .
Ir vêr por interesse ou curiosidade (regiões, monumentos, etc.): \textunderscore visitar os museus\textunderscore .
Inspeccionar, examinar minuciosamente.
Revelar Deus a sua cólera ou sua graça a.
\section{Visiteiro}
\begin{itemize}
\item {Grp. gram.:m.}
\end{itemize}
\begin{itemize}
\item {Utilização:Deprec.}
\end{itemize}
Aquelle que faz visitas:«\textunderscore visiteiro importuno\textunderscore ». Filinto, VIII, 97.
\section{Visiva}
\begin{itemize}
\item {Grp. gram.:f.}
\end{itemize}
\begin{itemize}
\item {Proveniência:(De \textunderscore visivo\textunderscore )}
\end{itemize}
Órgão da vista; vista.
\section{Visível}
\begin{itemize}
\item {Grp. gram.:adj.}
\end{itemize}
\begin{itemize}
\item {Proveniência:(Do lat. \textunderscore visibilis\textunderscore )}
\end{itemize}
Que se póde ver.
Claro.
Manifesto.
Apparente.
Perceptível.
Accessível ou que póde receber visita: \textunderscore o Ministro hoje não está visível\textunderscore .
\section{Visivelmente}
\begin{itemize}
\item {Grp. gram.:adv.}
\end{itemize}
De modo visível.
\section{Visivo}
\begin{itemize}
\item {Grp. gram.:adj.}
\end{itemize}
\begin{itemize}
\item {Proveniência:(Do lat. \textunderscore visus\textunderscore )}
\end{itemize}
Visual; visto; visível. Cf. \textunderscore Luz e Calor\textunderscore , 32.
\section{Vislumbrar}
\begin{itemize}
\item {Grp. gram.:v. t.}
\end{itemize}
\begin{itemize}
\item {Grp. gram.:V. i.}
\end{itemize}
\begin{itemize}
\item {Proveniência:(De \textunderscore vislumbre\textunderscore )}
\end{itemize}
Alumiar froixamente.
Entrever, lobrigar.
Conhecer imperfeitamente.
Conjecturar.
Lançar luz froixa.
Entremostrar-se.
Começar a surgir ou a apparecer.
\section{Vislumbre}
\begin{itemize}
\item {Grp. gram.:m.}
\end{itemize}
Luz froixa, pequeno clarão.
Reflexo.
Apparência vaga.
Ideia indistinta.
Conjectura.
Parecença.
Vestígio.
(Cast. \textunderscore vislumbre\textunderscore )
\section{Viso}
\begin{itemize}
\item {Grp. gram.:m.}
\end{itemize}
\begin{itemize}
\item {Utilização:Ant.}
\end{itemize}
\begin{itemize}
\item {Proveniência:(Lat. \textunderscore visus\textunderscore )}
\end{itemize}
Aspecto.
Physionomia.
Indício, vislumbre: \textunderscore isso não tem visos de probabilidade\textunderscore .
Pequena porção.
Cume de oiteiro; oiteiro.
O órgão da vista.
\section{Viso...}
\begin{itemize}
\item {Grp. gram.:pref.}
\end{itemize}
O mesmo que \textunderscore vice...\textunderscore 
\section{Visonha}
\begin{itemize}
\item {Grp. gram.:f.}
\end{itemize}
\begin{itemize}
\item {Proveniência:(De \textunderscore visão\textunderscore )}
\end{itemize}
Visão medonha, fantasma.
\section{Viso-rei}
\begin{itemize}
\item {Grp. gram.:m.}
\end{itemize}
O mesmo que \textunderscore Vice-Rei\textunderscore .
\section{Viso-reinado}
\begin{itemize}
\item {Grp. gram.:m.}
\end{itemize}
Govêrno de um Viso-Rei.
Tempo, durante o qual um Viso-Rei exerce ou exerceu o seu govêrno.
\section{Viso-reinar}
\begin{itemize}
\item {Grp. gram.:v. i.}
\end{itemize}
Exercer as funcções de Viso-Rei. Cf. Filinto, \textunderscore D. Man.\textunderscore , I, 323; Camillo, \textunderscore Doze Casam.\textunderscore , 177.
\section{Visório}
\begin{itemize}
\item {Grp. gram.:adj.}
\end{itemize}
O mesmo que \textunderscore visual\textunderscore .
\section{Víspar-se}
\begin{itemize}
\item {Grp. gram.:v. p.}
\end{itemize}
\begin{itemize}
\item {Utilização:Pop.}
\end{itemize}
Safar-se, esgueirar-se.
Desapparecer.
Ir-se embora rapidamente.
(Cp. \textunderscore víspere!\textunderscore )
\section{Víspere!}
\begin{itemize}
\item {Grp. gram.:interj.}
\end{itemize}
(designativa de \textunderscore repulsão\textunderscore , \textunderscore intimação para sair\textunderscore , \textunderscore retirada\textunderscore )
\textunderscore Fazer víspere\textunderscore , sumir-se, desapparecer.
\section{Vispora}
\begin{itemize}
\item {Grp. gram.:f.}
\end{itemize}
\begin{itemize}
\item {Utilização:Bras}
\end{itemize}
O mesmo que \textunderscore quino\textunderscore  ou \textunderscore loto\textunderscore .
\section{Visporar}
\begin{itemize}
\item {Grp. gram.:v. i.}
\end{itemize}
\begin{itemize}
\item {Utilização:Bras}
\end{itemize}
\begin{itemize}
\item {Proveniência:(De \textunderscore vispora\textunderscore )}
\end{itemize}
O mesmo que \textunderscore quinar\textunderscore .
\section{Visqueira}
\begin{itemize}
\item {Grp. gram.:f.}
\end{itemize}
O mesmo que \textunderscore visgueiro\textunderscore .
\section{Vista}
\begin{itemize}
\item {Grp. gram.:f.}
\end{itemize}
\begin{itemize}
\item {Utilização:Marcen.}
\end{itemize}
\begin{itemize}
\item {Utilização:Pop.}
\end{itemize}
\begin{itemize}
\item {Grp. gram.:Loc. adv.}
\end{itemize}
\begin{itemize}
\item {Utilização:Jur.}
\end{itemize}
\begin{itemize}
\item {Utilização:Jur.}
\end{itemize}
\begin{itemize}
\item {Grp. gram.:Pl.}
\end{itemize}
\begin{itemize}
\item {Utilização:Ant.}
\end{itemize}
\begin{itemize}
\item {Proveniência:(De \textunderscore visto\textunderscore )}
\end{itemize}
Acto ou effeito de vêr.
O órgão visual.
O sentido de vêr.
Os olhos.
Aquillo que se vê; panorama.
Estampa.
Aspecto.
Quadro.
Desígnio, plano.
Abertura por onde se póde estender a vista.
Scenário theatral.
Modo de julgar ou apreciar um assumpto.
Cavaca, que se accende á entrada do forno, para o illuminar interiormente, a fim de que se disponham convenientemente os pães que vão cozer-se.
Tira de fazenda, que se cose geralmente nas bordas de um vestuário e que, pela sua côr differente, resai da fazenda do mesmo vestuário.
Parte do capacete, em que há duas fendas correspondentes aos olhos.
Peça de madeira delgada, com que se revestem peças de móveis.
Coisa, que se gasta ou desapparece facilmente.
\textunderscore Á vista\textunderscore , na presença, deante.
\textunderscore Vista curta\textunderscore , difficuldade ou impossibilidade de vêr longe.
\textunderscore Dar vista\textunderscore , curar a cegueira.
\textunderscore Dar vista de um processo\textunderscore , entregar o processo a quem tem de o examinar e lançar nelle um despacho, resposta, reflexões jurídicas, etc.
\textunderscore Dar na vista\textunderscore  ou \textunderscore nas vistas\textunderscore , sêr notado, tornar-se evidente, sêr escandaloso.
\textunderscore Ir com vista\textunderscore , sêr entregue (um processo) a magistrado ou ás partes, para nelle se lançar despacho, allegações, etc.
Planos, intuitos.
Decoração theatral.
O mesmo que \textunderscore entrevista\textunderscore . Cf. R. Pina, \textunderscore Aff. V\textunderscore , c. LCIV.
\section{Visto}
\begin{itemize}
\item {Grp. gram.:adj.}
\end{itemize}
\begin{itemize}
\item {Grp. gram.:M.}
\end{itemize}
\begin{itemize}
\item {Proveniência:(De \textunderscore vêr\textunderscore )}
\end{itemize}
Acolhido, acceito.
Reputado: \textunderscore homem bem visto\textunderscore .
Sabido.
Sabedor.
Declaração, feita por uma autoridade ou funccionário num documento, para lhe dar validade.
\section{Vistor}
\begin{itemize}
\item {Grp. gram.:m.}
\end{itemize}
\begin{itemize}
\item {Utilização:Ant.}
\end{itemize}
\begin{itemize}
\item {Proveniência:(De \textunderscore vista\textunderscore )}
\end{itemize}
Aquelle que faz vistorias.
\section{Vistoria}
\begin{itemize}
\item {Grp. gram.:f.}
\end{itemize}
\begin{itemize}
\item {Utilização:Ext.}
\end{itemize}
\begin{itemize}
\item {Proveniência:(De \textunderscore vistor\textunderscore )}
\end{itemize}
Inspecção judicial a um prédio ou lugar, á cêrca do qual há litígio.
Inspecção; revista.
\section{Vistoriar}
\begin{itemize}
\item {Grp. gram.:v. t.}
\end{itemize}
Fazer vistoria a. Cf. Camillo, \textunderscore Perfil do Marquês\textunderscore , 294.
\section{Vistorizar}
\begin{itemize}
\item {Grp. gram.:v. t.}
\end{itemize}
Fazer vistoria a.
\section{Vistosamente}
\begin{itemize}
\item {Grp. gram.:adv.}
\end{itemize}
De modo vistoso; ostentosamente; com apparato.
\section{Vistoso}
\begin{itemize}
\item {Grp. gram.:adj.}
\end{itemize}
Que dá na vista.
Que attrai a attenção.
Ostentoso; apparatoso.
Admirável; agradável á vista.
\section{Visual}
\begin{itemize}
\item {Grp. gram.:adj.}
\end{itemize}
\begin{itemize}
\item {Proveniência:(Lat. \textunderscore visualis\textunderscore )}
\end{itemize}
Relativo á vista ou á visão.
\section{Visualidade}
\begin{itemize}
\item {Grp. gram.:f.}
\end{itemize}
\begin{itemize}
\item {Proveniência:(Do lat. \textunderscore visualitas\textunderscore )}
\end{itemize}
Vista.
Miragem.
Aspecto cambiante. Cf. Herculano, \textunderscore Cister\textunderscore , II, 252; \textunderscore Bobo\textunderscore , 320; Camillo, \textunderscore Doze Casam.\textunderscore , 50; Castilho, \textunderscore Montalverne\textunderscore .
\section{Visualmente}
\begin{itemize}
\item {Grp. gram.:adv.}
\end{itemize}
Por meio da vista; de modo visual.
\section{Vitáceas}
\begin{itemize}
\item {Grp. gram.:f. pl.}
\end{itemize}
\begin{itemize}
\item {Proveniência:(Do lat. \textunderscore vitis\textunderscore )}
\end{itemize}
O mesmo que \textunderscore ampelídeas\textunderscore .
\section{Vitadínia}
\begin{itemize}
\item {Grp. gram.:f.}
\end{itemize}
\begin{itemize}
\item {Proveniência:(De \textunderscore Vittadini\textunderscore , n. p.)}
\end{itemize}
Gênero de plantas sinantéreas.
\section{Vitadínia-das-floristas}
\begin{itemize}
\item {Grp. gram.:f.}
\end{itemize}
Planta trepadeira, (\textunderscore erigeron macronatus\textunderscore , De-Cand.).
\section{Vital}
\begin{itemize}
\item {Grp. gram.:adj.}
\end{itemize}
\begin{itemize}
\item {Proveniência:(Lat. \textunderscore vitalis\textunderscore )}
\end{itemize}
Relativo á vida.
Próprio para conservar a vida.
Fortificante.
Essencial.
Que tem importância capital.
\section{Vital}
\begin{itemize}
\item {Grp. gram.:m.}
\end{itemize}
\begin{itemize}
\item {Proveniência:(De \textunderscore Vital\textunderscore , n. p.)}
\end{itemize}
Casta de uva. Cf. \textunderscore Rev. Agron.\textunderscore , I, 18.
\section{Vitalício}
\begin{itemize}
\item {Grp. gram.:adj.}
\end{itemize}
\begin{itemize}
\item {Proveniência:(De \textunderscore vitalitas\textunderscore )}
\end{itemize}
Vital.
Que dura toda a vida ou destinado a durar toda a vida: \textunderscore emprêgo vitalício\textunderscore .
\section{Vitalidade}
\begin{itemize}
\item {Grp. gram.:f.}
\end{itemize}
\begin{itemize}
\item {Proveniência:(Do lat. \textunderscore vitalitas\textunderscore )}
\end{itemize}
Qualidade do que é vital.
Conjunto das funcções orgânicas.
\section{Vitalismo}
\begin{itemize}
\item {Grp. gram.:m.}
\end{itemize}
\begin{itemize}
\item {Proveniência:(De \textunderscore vital\textunderscore )}
\end{itemize}
Systema dos médicos vitalistas.
Conjunto das funcções orgânicas; vitalidade:«\textunderscore morta já ella estava na mais viva faculdade do vitalismo--a memória\textunderscore ». Camillo, \textunderscore Caveira\textunderscore , I, XX.
\section{Vitalista}
\begin{itemize}
\item {Grp. gram.:adj.}
\end{itemize}
\begin{itemize}
\item {Grp. gram.:M.}
\end{itemize}
\begin{itemize}
\item {Proveniência:(De \textunderscore vital\textunderscore )}
\end{itemize}
Relativo ao vitalismo.
Médico, que explica os phenómenos physiológicos e pathológicos pela influência do princípio vital.
\section{Vitalização}
\begin{itemize}
\item {Grp. gram.:f.}
\end{itemize}
Acto ou effeito de vitalizar.
\section{Vitalizador}
\begin{itemize}
\item {Grp. gram.:adj.}
\end{itemize}
Que vitaliza. Cf. B. Moreno, \textunderscore Com. do Campo\textunderscore , II, 160.
\section{Vitalizar}
\begin{itemize}
\item {Grp. gram.:v. t.}
\end{itemize}
\begin{itemize}
\item {Utilização:Neol.}
\end{itemize}
\begin{itemize}
\item {Proveniência:(De \textunderscore vital\textunderscore )}
\end{itemize}
Restituir á vida; dar vida nova a.
\section{Vitalmente}
\begin{itemize}
\item {Grp. gram.:adv.}
\end{itemize}
De modo vital.
\section{Vitando}
\begin{itemize}
\item {Grp. gram.:adj.}
\end{itemize}
\begin{itemize}
\item {Proveniência:(Lat. \textunderscore vitandus\textunderscore )}
\end{itemize}
Que se deve evitar; abominável.
\section{Vitaró}
\begin{itemize}
\item {Grp. gram.:m.}
\end{itemize}
Antigo estribilho popular:«\textunderscore ...os rapazes... correndo estrugião a rua com o seu vitaró, vitaró...\textunderscore »\textunderscore Anat. Joc.\textunderscore , I, 279.
\section{Vitascópio}
\begin{itemize}
\item {Grp. gram.:m.}
\end{itemize}
\begin{itemize}
\item {Proveniência:(Do lat. \textunderscore vita\textunderscore  + gr. \textunderscore skopein\textunderscore )}
\end{itemize}
Um dos muitos neologismos, propostos para designar o cinematógrapho.
\section{Vitatório}
\begin{itemize}
\item {Grp. gram.:adj.}
\end{itemize}
\begin{itemize}
\item {Utilização:Ant.}
\end{itemize}
\begin{itemize}
\item {Proveniência:(Do lat. \textunderscore vitare\textunderscore )}
\end{itemize}
Próprio para evitar.
Dizia-se do pregão, que se soltava, antes de sêr executado um condemnado. Cf. G. Vicente, \textunderscore Barca do Inferno\textunderscore .
\section{Vitável}
\begin{itemize}
\item {Grp. gram.:adj.}
\end{itemize}
\begin{itemize}
\item {Proveniência:(Do lat. \textunderscore vitabilis\textunderscore )}
\end{itemize}
Que se deve evitar. Cf. Castilho, \textunderscore Fastos\textunderscore , II, 238.
\section{Vitela}
\begin{itemize}
\item {Grp. gram.:f.}
\end{itemize}
\begin{itemize}
\item {Proveniência:(De \textunderscore vitello\textunderscore )}
\end{itemize}
Novilha, que tem menos de um anno.
Carne de novilha ou de novilho.
Pelle dêstes animaes, preparada para fabricação de calçado e outros usos.
\section{Vítele}
\begin{itemize}
\item {Grp. gram.:m.}
\end{itemize}
\begin{itemize}
\item {Proveniência:(Do malab. \textunderscore vettila\textunderscore )}
\end{itemize}
O mesmo que \textunderscore betle\textunderscore .
\section{Vitelífero}
\begin{itemize}
\item {Grp. gram.:adj.}
\end{itemize}
\begin{itemize}
\item {Proveniência:(Do lat. \textunderscore vitellum\textunderscore  + \textunderscore ferre\textunderscore )}
\end{itemize}
Que tem gema de ovo.
\section{Vitelina}
\begin{itemize}
\item {Grp. gram.:f.}
\end{itemize}
\begin{itemize}
\item {Proveniência:(De \textunderscore vitelino\textunderscore )}
\end{itemize}
Substância orgânica azotada, contida na gema do ovo.
Membrana, que envolve a gema do ovo das aves.
\section{Vitelino}
\begin{itemize}
\item {Grp. gram.:adj.}
\end{itemize}
\begin{itemize}
\item {Utilização:Zool.}
\end{itemize}
\begin{itemize}
\item {Proveniência:(Lat. \textunderscore vitelinus\textunderscore )}
\end{itemize}
Relativo á gema do ovo.
Amarelo, como a gema do ovo.
\textunderscore Saco vitelino\textunderscore , espécie de bolsa, recheada do gema, que os seres ovíparos trazem consigo ao nascer, e com que êlles ocorrem ás primeiras necessidades da sua alimentação. Cf. P. Moraes, \textunderscore Zool. Elem.\textunderscore , 484.
\section{Vitella}
\begin{itemize}
\item {Grp. gram.:f.}
\end{itemize}
\begin{itemize}
\item {Proveniência:(De \textunderscore vitello\textunderscore )}
\end{itemize}
Novilha, que tem menos de um anno.
Carne de novilha ou de novilho.
Pelle dêstes animaes, preparada para fabricação de calçado e outros usos.
\section{Vitellífero}
\begin{itemize}
\item {Grp. gram.:adj.}
\end{itemize}
\begin{itemize}
\item {Proveniência:(Do lat. \textunderscore vitellum\textunderscore  + \textunderscore ferre\textunderscore )}
\end{itemize}
Que tem gemma de ovo.
\section{Vitellina}
\begin{itemize}
\item {Grp. gram.:f.}
\end{itemize}
\begin{itemize}
\item {Proveniência:(De \textunderscore vitellino\textunderscore )}
\end{itemize}
Substância orgânica azotada, contida na gemma do ovo.
Membrana, que envolve a gemma do ovo das aves.
\section{Vitellino}
\begin{itemize}
\item {Grp. gram.:adj.}
\end{itemize}
\begin{itemize}
\item {Utilização:Zool.}
\end{itemize}
\begin{itemize}
\item {Proveniência:(Lat. \textunderscore vitelinus\textunderscore )}
\end{itemize}
Relativo á gemma do ovo.
Amarelo, como a gemma do ovo.
\textunderscore Saco vitellino\textunderscore , espécie de bolsa, recheada do gemma, que os seres ovíparos trazem consigo ao nascer, e com que êlles occorrem ás primeiras necessidades da sua alimentação. Cf. P. Moraes, \textunderscore Zool. Elem.\textunderscore , 484.
\section{Vitello}
\begin{itemize}
\item {Grp. gram.:m.}
\end{itemize}
\begin{itemize}
\item {Utilização:Physiol.}
\end{itemize}
\begin{itemize}
\item {Proveniência:(Lat. \textunderscore vitellus\textunderscore )}
\end{itemize}
Novilho, que tem menos de um anno.
Parte essencial do óvulo dos animaes.
\section{Vitelo}
\begin{itemize}
\item {Grp. gram.:m.}
\end{itemize}
\begin{itemize}
\item {Utilização:Physiol.}
\end{itemize}
\begin{itemize}
\item {Proveniência:(Lat. \textunderscore vitellus\textunderscore )}
\end{itemize}
Novilho, que tem menos de um ano.
Parte essencial do óvulo dos animaes.
\section{Vitícola}
\begin{itemize}
\item {Grp. gram.:adj.}
\end{itemize}
\begin{itemize}
\item {Grp. gram.:M.}
\end{itemize}
\begin{itemize}
\item {Proveniência:(Lat. \textunderscore viticola\textunderscore )}
\end{itemize}
Relativo á viticultura.
O mesmo que \textunderscore viticultor\textunderscore .
\section{Viticomado}
\begin{itemize}
\item {Grp. gram.:adj.}
\end{itemize}
\begin{itemize}
\item {Utilização:Poét.}
\end{itemize}
\begin{itemize}
\item {Proveniência:(Do lat. \textunderscore vitis\textunderscore  + \textunderscore comatus\textunderscore )}
\end{itemize}
Coroado de parras.
\section{Viticultor}
\begin{itemize}
\item {Grp. gram.:m.  e  adj.}
\end{itemize}
\begin{itemize}
\item {Proveniência:(Lat. \textunderscore viticultor\textunderscore )}
\end{itemize}
Cultivador de vinhas.
\section{Viticultura}
\begin{itemize}
\item {Grp. gram.:f.}
\end{itemize}
\begin{itemize}
\item {Proveniência:(Do lat. \textunderscore vitis\textunderscore  + \textunderscore cultura\textunderscore )}
\end{itemize}
Cultura das vinhas.
\section{Vitífero}
\begin{itemize}
\item {Grp. gram.:adj.}
\end{itemize}
\begin{itemize}
\item {Proveniência:(Lat. \textunderscore vitifer\textunderscore )}
\end{itemize}
Coberto de videiras.
Que produz vinhas ou videiras.
Próprio para a cultura das vinhas.
\section{Vitiligem}
\begin{itemize}
\item {Grp. gram.:m.}
\end{itemize}
\begin{itemize}
\item {Proveniência:(Do lat. \textunderscore vitiligo\textunderscore )}
\end{itemize}
Doença cutânea, caracterizada por placas esbranquiçadas, rodeadas de uma zona, em que a pelle é mais pigmentada que normalmente.
\section{Vitima}
\begin{itemize}
\item {Grp. gram.:f.}
\end{itemize}
\begin{itemize}
\item {Proveniência:(Lat. \textunderscore victima\textunderscore )}
\end{itemize}
Criatura viva, imolada em holocausto a uma divindade.
Pessôa, sacrificada aos interesses ou paixões de outrem.
Pessôa, que foi ferida ou assassinada casualmente ou com intuitos criminosos ou ainda em legítima defesa.
Pessôa, que sucumbe a uma desgraça.
Pessôa, que sofre um infortúnio.
Tudo que sofre qualquer damno.
\section{Vitimador}
\begin{itemize}
\item {Grp. gram.:m.}
\end{itemize}
\begin{itemize}
\item {Proveniência:(De \textunderscore vitimar\textunderscore )}
\end{itemize}
O mesmo que \textunderscore sacrificador\textunderscore .
\section{Vitimar}
\begin{itemize}
\item {Grp. gram.:v. t.}
\end{itemize}
\begin{itemize}
\item {Proveniência:(Lat. \textunderscore victimare\textunderscore )}
\end{itemize}
Tornar vitima; sacrificar.
Danificar.
\section{Vitimário}
\begin{itemize}
\item {Grp. gram.:m.}
\end{itemize}
\begin{itemize}
\item {Grp. gram.:Adj.}
\end{itemize}
\begin{itemize}
\item {Proveniência:(Lat. \textunderscore victimarius\textunderscore )}
\end{itemize}
Ministro dos sacrifícios, entre os antigos.
Sacerdote, que imolava as víctimas.
Sacrificador.
Relativo a vítima.
\section{Vitinga}
\begin{itemize}
\item {Grp. gram.:f.}
\end{itemize}
\begin{itemize}
\item {Utilização:Bras}
\end{itemize}
Espécie de farinha.
\section{Vitivinicultor}
\begin{itemize}
\item {Grp. gram.:m.}
\end{itemize}
\begin{itemize}
\item {Utilização:Neol.}
\end{itemize}
\begin{itemize}
\item {Proveniência:(Do lat. \textunderscore vitis\textunderscore  + \textunderscore vinum\textunderscore  + \textunderscore cultor\textunderscore )}
\end{itemize}
Aquelle que cultiva vinhas e fabríca vinho.
\section{Vito!}
\begin{itemize}
\item {Grp. gram.:interj.}
\end{itemize}
\begin{itemize}
\item {Utilização:Prov.}
\end{itemize}
\begin{itemize}
\item {Utilização:trasm.}
\end{itemize}
Viva!
\section{Vitòrhuguesco}
\begin{itemize}
\item {Grp. gram.:adj.}
\end{itemize}
\begin{itemize}
\item {Utilização:Neol.}
\end{itemize}
Relativo a Víctor Hugo.
Semelhante ao estílo de Víctor Hugo.
\section{Vitória}
\begin{itemize}
\item {Grp. gram.:f.}
\end{itemize}
\begin{itemize}
\item {Utilização:Fig.}
\end{itemize}
\begin{itemize}
\item {Utilização:Pop.}
\end{itemize}
\begin{itemize}
\item {Proveniência:(Lat. \textunderscore victoria\textunderscore )}
\end{itemize}
Acto ou efeito de vencer o inimigo numa batalha.
Triúnfo.
Espécie de carruagem moderna. Cf. Camillo, \textunderscore Mulhér Fatal\textunderscore , 152.
Vantagem; bom êxito.
Soberano, moéda inglesa.
\section{Vitoriar}
\begin{itemize}
\item {Grp. gram.:v. t.}
\end{itemize}
\begin{itemize}
\item {Proveniência:(De \textunderscore vitória\textunderscore )}
\end{itemize}
Aplaudir estrepitosamente; aclamar; saudar com entusiasmo.
\section{Vitoriosamente}
\begin{itemize}
\item {Grp. gram.:adv.}
\end{itemize}
De modo vitorioso; triunfantemente.
\section{Vitorioso}
\begin{itemize}
\item {Grp. gram.:adj.}
\end{itemize}
\begin{itemize}
\item {Proveniência:(Lat. \textunderscore victoriosus\textunderscore )}
\end{itemize}
Que conseguiu vitória; que triunfou.
\section{Vito-sério!}
\begin{itemize}
\item {Grp. gram.:interj.}
\end{itemize}
O mesmo que \textunderscore víctor-sério!\textunderscore 
\section{Vitral}
\begin{itemize}
\item {Grp. gram.:m.}
\end{itemize}
\begin{itemize}
\item {Utilização:Neol.}
\end{itemize}
\begin{itemize}
\item {Proveniência:(Do fr. \textunderscore vitrail\textunderscore )}
\end{itemize}
Vidraça de côres ou com pinturas sôbre o vidro.
\section{Vitre}
\begin{itemize}
\item {Grp. gram.:m.}
\end{itemize}
Espécie de lona para toldos e velas de botes.
(Cast. \textunderscore vitre\textunderscore )
\section{Vítreo}
\begin{itemize}
\item {Grp. gram.:adj.}
\end{itemize}
\begin{itemize}
\item {Proveniência:(Lat. \textunderscore vitreus\textunderscore )}
\end{itemize}
Relativo a vidro.
Feito de vidro.
Que tem a natureza ou aspecto do vidro.
Límpido, transparente.
\section{Vitrescibilidade}
\begin{itemize}
\item {Grp. gram.:f.}
\end{itemize}
Qualidade do que é vitrescível.
\section{Vitrescível}
\begin{itemize}
\item {Grp. gram.:adj.}
\end{itemize}
\begin{itemize}
\item {Proveniência:(Do lat. \textunderscore vitrum\textunderscore )}
\end{itemize}
Que se póde transformar em vidro; vitrificável.
\section{Vitrificação}
\begin{itemize}
\item {Grp. gram.:f.}
\end{itemize}
Acto ou effeito de vitrificar.
\section{Vitrificar}
\begin{itemize}
\item {Grp. gram.:v. t.}
\end{itemize}
\begin{itemize}
\item {Grp. gram.:V. i.  e  p.}
\end{itemize}
\begin{itemize}
\item {Proveniência:(Do lat. \textunderscore vitrum\textunderscore  + \textunderscore facere\textunderscore )}
\end{itemize}
Converter em vidro.
Dar o aspecto de vidro a.
Converter-se em vidro.
Tomar o aspecto de vidro.
\section{Vitrificável}
\begin{itemize}
\item {Grp. gram.:adj.}
\end{itemize}
Que se póde vitrificar.
\section{Vitrina}
\begin{itemize}
\item {Grp. gram.:f.}
\end{itemize}
\begin{itemize}
\item {Utilização:Neol.}
\end{itemize}
Vidraça, por dentro da qual se expõem fazendas ou outros objectos, destinados á venda.
Espécie de caixa com tampa envidraçada, ou armário com vidraça móvel, em que se resguardam objectos expostos á venda.
(Cast. \textunderscore vitrina\textunderscore )
\section{Vitríola}
\begin{itemize}
\item {Grp. gram.:f.}
\end{itemize}
Utensílio de ferro, com que os fabricantes de botões de casquinha tiram os vestígios dos cunhos.
\section{Vitriolado}
\begin{itemize}
\item {Grp. gram.:adj.}
\end{itemize}
\begin{itemize}
\item {Proveniência:(De \textunderscore vitriolar\textunderscore )}
\end{itemize}
Que tem vitríolo.
Deformado pela acção do vitríolo: \textunderscore tinha a cara vitriolada\textunderscore .
\section{Vitriolar}
\begin{itemize}
\item {Grp. gram.:v. t.}
\end{itemize}
Vitriolizar.
Atacar alguém, arremessando-lhe vitríolo.
\section{Vitriólico}
\begin{itemize}
\item {Grp. gram.:adj.}
\end{itemize}
Que é da natureza do vitríolo; sulfúrico.
\section{Vitriolização}
\begin{itemize}
\item {Grp. gram.:f.}
\end{itemize}
Acto ou effeito de vitriolizar.
\section{Vitriolizar}
\begin{itemize}
\item {Grp. gram.:v. t.}
\end{itemize}
Transformar em vitríolo.
\section{Vitríolo}
\begin{itemize}
\item {Grp. gram.:m.}
\end{itemize}
Designação vulgar de diversos sulfatos.
Ácido sulfúrico.
(B. lat. \textunderscore vitriolum\textunderscore )
\section{Vitro}
\begin{itemize}
\item {Grp. gram.:m.}
\end{itemize}
\begin{itemize}
\item {Utilização:Ant.}
\end{itemize}
O mesmo que \textunderscore applauso\textunderscore .
\section{Vitro-metállico}
\begin{itemize}
\item {Grp. gram.:adj.}
\end{itemize}
\begin{itemize}
\item {Proveniência:(Do lat. \textunderscore vitrum\textunderscore  + \textunderscore metallum\textunderscore )}
\end{itemize}
Feito de vidro e metal.
\section{Vitroporfírico}
\begin{itemize}
\item {Grp. gram.:adj.}
\end{itemize}
\begin{itemize}
\item {Utilização:Miner.}
\end{itemize}
\begin{itemize}
\item {Proveniência:(De \textunderscore vitropórfiro\textunderscore )}
\end{itemize}
Diz-se da rocha vitrosa, em cuja massa se notam elementos cristalinos.
\section{Vitropórfiro}
\begin{itemize}
\item {Grp. gram.:m.}
\end{itemize}
\begin{itemize}
\item {Proveniência:(T. hyb., do lat. \textunderscore vitrum\textunderscore  + gr. \textunderscore porphura\textunderscore )}
\end{itemize}
Espécie de pórfiro, em cuja massa se exibem cristaes grandes.
\section{Vitroporphýrico}
\begin{itemize}
\item {Grp. gram.:adj.}
\end{itemize}
\begin{itemize}
\item {Utilização:Miner.}
\end{itemize}
\begin{itemize}
\item {Proveniência:(De \textunderscore vitropórphyro\textunderscore )}
\end{itemize}
Diz-se da rocha vitrosa, em cuja massa se notam elementos crystalinos.
\section{Vitropórphyro}
\begin{itemize}
\item {Grp. gram.:m.}
\end{itemize}
\begin{itemize}
\item {Proveniência:(T. hyb., do lat. \textunderscore vitrum\textunderscore  + gr. \textunderscore porphura\textunderscore )}
\end{itemize}
Espécie de pórphyro, em cuja massa se exhibem crystaes grandes.
\section{Vittadínia}
\begin{itemize}
\item {Grp. gram.:f.}
\end{itemize}
\begin{itemize}
\item {Proveniência:(De \textunderscore Vittadini\textunderscore , n. p.)}
\end{itemize}
Gênero de plantas synanthéreas.
\section{Vitualha}
\begin{itemize}
\item {Grp. gram.:f.}
\end{itemize}
(V.vitualhas)
\section{Vitualhar}
\begin{itemize}
\item {Grp. gram.:v. t.}
\end{itemize}
Prover de vitualhas.
\section{Vitualhas}
\begin{itemize}
\item {Grp. gram.:f. pl.}
\end{itemize}
\begin{itemize}
\item {Proveniência:(Do lat. \textunderscore victualia\textunderscore )}
\end{itemize}
Provisões de mantimentos.
Mantimentos.
\section{Vítulo}
\begin{itemize}
\item {Grp. gram.:m.}
\end{itemize}
\begin{itemize}
\item {Proveniência:(Lat. \textunderscore vitulus\textunderscore )}
\end{itemize}
Vitello.
Phoca.
\section{Vituperação}
\begin{itemize}
\item {Grp. gram.:f.}
\end{itemize}
\begin{itemize}
\item {Proveniência:(Do lat. \textunderscore vituperatio\textunderscore )}
\end{itemize}
O mesmo que \textunderscore vitupério\textunderscore .
\section{Vituperador}
\begin{itemize}
\item {Grp. gram.:m.  e  adj.}
\end{itemize}
\begin{itemize}
\item {Proveniência:(Do lat. \textunderscore vituperator\textunderscore )}
\end{itemize}
O que vitupera.
\section{Vituperar}
\begin{itemize}
\item {Grp. gram.:v. t.}
\end{itemize}
\begin{itemize}
\item {Proveniência:(Lat. \textunderscore vituperare\textunderscore )}
\end{itemize}
Reprehender duramente.
Injuriar, afrontar.
Diffamar.
Menoscabar; aviltar.
\section{Vituperável}
\begin{itemize}
\item {Grp. gram.:adj.}
\end{itemize}
\begin{itemize}
\item {Proveniência:(Do lat. \textunderscore vituperabilis\textunderscore )}
\end{itemize}
Que merece vitupério.
\section{Vituperavelmente}
\begin{itemize}
\item {Grp. gram.:adv.}
\end{itemize}
De modo vituperável.
\section{Vitupério}
\begin{itemize}
\item {Grp. gram.:m.}
\end{itemize}
\begin{itemize}
\item {Proveniência:(Lat. \textunderscore vituperium\textunderscore )}
\end{itemize}
Acto ou effeito de vituperar.
Acto vergonhoso, infame ou criminoso.
\section{Vituperiosamente}
\begin{itemize}
\item {Grp. gram.:adv.}
\end{itemize}
De modo vituperioso.
Com vitupério.
\section{Vituperioso}
\begin{itemize}
\item {Grp. gram.:adj.}
\end{itemize}
Em que há vitupério.
Vergonhoso.
\section{Vituperosamente}
\begin{itemize}
\item {Grp. gram.:adv.}
\end{itemize}
De modo vituperoso.
Com vitupério.
\section{Vituperoso}
\begin{itemize}
\item {Grp. gram.:adj.}
\end{itemize}
(V.vituperioso)
\section{Viúva}
\begin{itemize}
\item {Grp. gram.:f.}
\end{itemize}
\begin{itemize}
\item {Utilização:Gír.}
\end{itemize}
\begin{itemize}
\item {Proveniência:(Do cast. \textunderscore viuda\textunderscore )}
\end{itemize}
Mulhér, a quem morreu o marido e que ainda não casou de novo.
Pássaro conirostro, procedente da África.
Nome de algumas plantas brasileiras.
Peixe dos Açores.
O mesmo que \textunderscore corda\textunderscore .
\section{Viuvada}
\begin{itemize}
\item {fónica:vi-u}
\end{itemize}
\begin{itemize}
\item {Grp. gram.:f.}
\end{itemize}
\begin{itemize}
\item {Utilização:Ant.}
\end{itemize}
\begin{itemize}
\item {Proveniência:(De \textunderscore viúva\textunderscore )}
\end{itemize}
Vida desregrada, (por allusão aos desmandos de viúva nova e desatinada)
\section{Viúva-moça}
\begin{itemize}
\item {Grp. gram.:f.}
\end{itemize}
Pássaro, o mesmo que \textunderscore viúva\textunderscore . Cf. Filinto, XII, 251.
\section{Viuvar}
\begin{itemize}
\item {fónica:vi-u}
\end{itemize}
\begin{itemize}
\item {Grp. gram.:v. i.}
\end{itemize}
\begin{itemize}
\item {Proveniência:(Do b. lat. \textunderscore viduvare\textunderscore )}
\end{itemize}
O mesmo que \textunderscore enviuvar\textunderscore .
\section{Viúvas}
\begin{itemize}
\item {Grp. gram.:f. pl.}
\end{itemize}
Planta cucurbitácea, (\textunderscore trachelium coeruleum\textunderscore , Lin.).
\section{Viuvez}
\begin{itemize}
\item {fónica:vi-u}
\end{itemize}
\begin{itemize}
\item {Grp. gram.:f.}
\end{itemize}
\begin{itemize}
\item {Utilização:Fig.}
\end{itemize}
\begin{itemize}
\item {Proveniência:(De \textunderscore viúvo\textunderscore  ou \textunderscore viúva\textunderscore )}
\end{itemize}
Estado de quem é viúvo.
Solidão.
Privação; desconsôlo por desamparo.
\section{Viuveza}
\begin{itemize}
\item {fónica:vi-u}
\end{itemize}
\begin{itemize}
\item {Grp. gram.:f.}
\end{itemize}
(V.viuvez)
\section{Viuvidade}
\begin{itemize}
\item {fónica:vi-u}
\end{itemize}
\begin{itemize}
\item {Grp. gram.:f.}
\end{itemize}
\begin{itemize}
\item {Utilização:Ant.}
\end{itemize}
O mesmo que \textunderscore viuvez\textunderscore .
\section{Viuvinha}
\begin{itemize}
\item {fónica:vi-u}
\end{itemize}
\begin{itemize}
\item {Grp. gram.:f.}
\end{itemize}
\begin{itemize}
\item {Utilização:Prov.}
\end{itemize}
\begin{itemize}
\item {Utilização:alent.}
\end{itemize}
Espécie de jôgo popular.
Pássaro, o mesmo que \textunderscore viúva\textunderscore .
Dança de roda.
\section{Viúvo}
\begin{itemize}
\item {Grp. gram.:m.}
\end{itemize}
\begin{itemize}
\item {Grp. gram.:Adj.}
\end{itemize}
\begin{itemize}
\item {Utilização:Fig.}
\end{itemize}
\begin{itemize}
\item {Proveniência:(Do cast. \textunderscore viudus\textunderscore )}
\end{itemize}
Homem, a quem morreu a espôsa e que ainda não casou de novo.
Que é viúvo.
Privado; desamparado.
\section{Viva!}
\begin{itemize}
\item {Grp. gram.:interj.}
\end{itemize}
\begin{itemize}
\item {Grp. gram.:M.}
\end{itemize}
\begin{itemize}
\item {Proveniência:(De \textunderscore viver\textunderscore )}
\end{itemize}
(designativa de \textunderscore applauso\textunderscore  e \textunderscore enthusiasmo\textunderscore )
Exclamação de applauso ou felicitação, que envolve o desejo de que viva e prospere a pessôa ou coisa a que se dirige.
\section{Viva-artética}
\begin{itemize}
\item {Grp. gram.:f.}
\end{itemize}
Planta da serra de Sintra.
\section{Vivacidade}
\begin{itemize}
\item {Grp. gram.:f.}
\end{itemize}
\begin{itemize}
\item {Proveniência:(Do lat. \textunderscore vivacitas\textunderscore )}
\end{itemize}
Qualidade do que é vivaz.
Actividade.
Finura.
Modo expressivo de falar ou gesticular.
Brilho; brilhantismo.
\section{Viva-el-amor}
\begin{itemize}
\item {Grp. gram.:m.}
\end{itemize}
Espécie de jôgo de cartas, também chamado \textunderscore cró\textunderscore .
(Loc. cast.)
\section{Vivamente}
\begin{itemize}
\item {Grp. gram.:adv.}
\end{itemize}
De modo vivo; com vivacidade; rapidamente; energicamente.
\section{Vivandeira}
\begin{itemize}
\item {Grp. gram.:f.}
\end{itemize}
\begin{itemize}
\item {Proveniência:(De \textunderscore vivandeiro\textunderscore )}
\end{itemize}
Mulhér, que vende ou leva mantimentos, acompanhando tropas em marcha.
\section{Vivandeiro}
\begin{itemize}
\item {Grp. gram.:m.}
\end{itemize}
\begin{itemize}
\item {Proveniência:(Do fr. \textunderscore vivandier\textunderscore )}
\end{itemize}
Aquelle que vende mantimentos nas feiras ou ás tropas que acompanha.
\section{Vivar}
\begin{itemize}
\item {Grp. gram.:v. i.}
\end{itemize}
Dar vivas. Cf. Filinto, \textunderscore D. Man.\textunderscore , II, 300.
\section{Vivaz}
\begin{itemize}
\item {Grp. gram.:adj.}
\end{itemize}
\begin{itemize}
\item {Utilização:Bot.}
\end{itemize}
\begin{itemize}
\item {Proveniência:(Lat. \textunderscore vivax\textunderscore )}
\end{itemize}
Vivedoiro.
Activo.
Ardente.
Vigoroso.
Prompto.
Diz-se das plantas herbáceas, que duram muitos annos, embora as hastes se renovem annualmente.
\section{Vivazmente}
\begin{itemize}
\item {Grp. gram.:adv.}
\end{itemize}
De modo vivaz; com vivacidade.
\section{Vivedoiro}
\begin{itemize}
\item {Grp. gram.:adj.}
\end{itemize}
\begin{itemize}
\item {Proveniência:(De \textunderscore viver\textunderscore )}
\end{itemize}
Que póde viver.
Que vive muito; que póde viver muito; duradoiro.
\section{Vivedor}
\begin{itemize}
\item {Grp. gram.:adj.}
\end{itemize}
\begin{itemize}
\item {Proveniência:(De \textunderscore viver\textunderscore )}
\end{itemize}
Vivedoiro.
Solícito, agenciador.
\section{Vivedouro}
\begin{itemize}
\item {Grp. gram.:adj.}
\end{itemize}
\begin{itemize}
\item {Proveniência:(De \textunderscore viver\textunderscore )}
\end{itemize}
Que póde viver.
Que vive muito; que póde viver muito; duradoiro.
\section{Viveirista}
\begin{itemize}
\item {Grp. gram.:m.}
\end{itemize}
\begin{itemize}
\item {Proveniência:(De \textunderscore viveiro\textunderscore )}
\end{itemize}
Aquelle que se occupa habitualmente de viveiros de plantas.
Aquelle que possue viveiros de plantas para commércio.
\section{Viveiro}
\begin{itemize}
\item {Grp. gram.:m.}
\end{itemize}
\begin{itemize}
\item {Utilização:Prov.}
\end{itemize}
\begin{itemize}
\item {Utilização:minh.}
\end{itemize}
\begin{itemize}
\item {Utilização:Ext.}
\end{itemize}
\begin{itemize}
\item {Proveniência:(Do lat. \textunderscore vivarius\textunderscore )}
\end{itemize}
Lugar, em que se conservam e se reproduzem animaes vivos.
Gado, criação.
Escavação natural ou artificial em que se criam peixes.
Canteiro ou recinto apropriado, onde se semeiam vegetaes que hão de ser transplantados.
Espécie de caixa com água, para guarda ou transporte de peixes vivos.
Aquário.
Lugar, onde se criam ou educam pessôas de certa classe.
O maior dos tanques das marinhas, nos quaes se recebe a água salgada.
Grande porção; enxame.
\section{Vivenda}
\begin{itemize}
\item {Grp. gram.:f.}
\end{itemize}
Lugar, onde se vive.
Morada, habitação.
Modo de vida.
Subsistência; passadío.
Comportamento.
(B. lat. \textunderscore vivenda\textunderscore )
\section{Vivente}
\begin{itemize}
\item {Grp. gram.:m.  e  adj.}
\end{itemize}
\begin{itemize}
\item {Proveniência:(Lat. \textunderscore vivens\textunderscore )}
\end{itemize}
O que vive.
Criatura viva.
O homem.
\section{Viver}
\begin{itemize}
\item {Grp. gram.:v. i.}
\end{itemize}
\begin{itemize}
\item {Grp. gram.:V. t.}
\end{itemize}
\begin{itemize}
\item {Grp. gram.:V. p.}
\end{itemize}
\begin{itemize}
\item {Grp. gram.:M.}
\end{itemize}
\begin{itemize}
\item {Proveniência:(Lat. \textunderscore vivere\textunderscore )}
\end{itemize}
Têr vida.
Estar em condições de vida.
Existir.
Consagrar a vida.
Dedicar-se.
Cohabitar.
Residir.
Nutrir-se.
Alimentar-se.
Comportar-se.
Estar em relações habituaes, têr contacto habitual.
Durar.
Passar aos vindoiros.
Passar (a vida): \textunderscore viver vida de privações\textunderscore .
Existir, ir vivendo.
A vida.
\section{Víveres}
\begin{itemize}
\item {Grp. gram.:m. pl.}
\end{itemize}
(\textunderscore Gal.\textunderscore  disparatado, na significação de gêneros alimentícios, mantimentos)
(Escrita errada, do fr. \textunderscore vivres\textunderscore )
\section{Viverrídeo}
\begin{itemize}
\item {Grp. gram.:adj.}
\end{itemize}
\begin{itemize}
\item {Grp. gram.:M. Pl.}
\end{itemize}
\begin{itemize}
\item {Proveniência:(Do lat. \textunderscore viverra\textunderscore  + gr. \textunderscore eidos\textunderscore )}
\end{itemize}
Relativo ou semelhante ao furão.
Família de animaes, que tem por typo o furão.
\section{Viveza}
\begin{itemize}
\item {Grp. gram.:f.}
\end{itemize}
O mesmo que \textunderscore vivacidade\textunderscore .
\section{Vivido}
\begin{itemize}
\item {Grp. gram.:adj.}
\end{itemize}
\begin{itemize}
\item {Utilização:Ant.}
\end{itemize}
\begin{itemize}
\item {Utilização:Chul.}
\end{itemize}
\begin{itemize}
\item {Proveniência:(De \textunderscore viver\textunderscore )}
\end{itemize}
Vivo. Cf. Lobo, \textunderscore Auto do Nascimento\textunderscore .
\section{Vívido}
\begin{itemize}
\item {Grp. gram.:adj.}
\end{itemize}
\begin{itemize}
\item {Proveniência:(Lat. \textunderscore vividus\textunderscore )}
\end{itemize}
Que tem vivacidade.
Ardente.
Luminoso.
Brilhante; expressivo.
\section{Vivificação}
\begin{itemize}
\item {Grp. gram.:f.}
\end{itemize}
\begin{itemize}
\item {Proveniência:(Do lat. \textunderscore vivificatio\textunderscore )}
\end{itemize}
Acto ou effeito de vivificar.
\section{Vivificador}
\begin{itemize}
\item {Grp. gram.:m.  e  adj.}
\end{itemize}
\begin{itemize}
\item {Proveniência:(Do lat. \textunderscore vivificator\textunderscore )}
\end{itemize}
O que vivifica.
\section{Vivificante}
\begin{itemize}
\item {Grp. gram.:adj.}
\end{itemize}
\begin{itemize}
\item {Proveniência:(Lat. \textunderscore vivificans\textunderscore )}
\end{itemize}
Que vivifica.
\section{Vivificar}
\begin{itemize}
\item {Grp. gram.:v. t.}
\end{itemize}
\begin{itemize}
\item {Proveniência:(Lat. \textunderscore vivificare\textunderscore )}
\end{itemize}
Dar vida ou existência a.
Conservar a existência de.
Tornar vívido.
Animar; fecundar.
\section{Vivificativo}
\begin{itemize}
\item {Grp. gram.:adj.}
\end{itemize}
Que vivifica.
\section{Vivífico}
\begin{itemize}
\item {Grp. gram.:adj.}
\end{itemize}
\begin{itemize}
\item {Proveniência:(Lat. \textunderscore vivificus\textunderscore )}
\end{itemize}
O mesmo que \textunderscore vivificante\textunderscore .
\section{Viviparação}
\begin{itemize}
\item {Grp. gram.:m.}
\end{itemize}
\begin{itemize}
\item {Utilização:Neol.}
\end{itemize}
Qualidade de vivíparo.
\section{Vivíparo}
\begin{itemize}
\item {Grp. gram.:adj.}
\end{itemize}
\begin{itemize}
\item {Utilização:Bot.}
\end{itemize}
\begin{itemize}
\item {Grp. gram.:M.}
\end{itemize}
\begin{itemize}
\item {Proveniência:(Lat. \textunderscore vivíparus\textunderscore )}
\end{itemize}
Que pare filhos vivos, ou não incluidos em ôvo.
Diz-se das plantas, cujos grãos são substituidos por bolbos, ou cujos grãos germinam no pericarpo.
Animal vivíparo, mammífero.
\section{Vivisecção}
\begin{itemize}
\item {fónica:sé}
\end{itemize}
\begin{itemize}
\item {Grp. gram.:f.}
\end{itemize}
\begin{itemize}
\item {Proveniência:(Do lat. \textunderscore vivus\textunderscore  + \textunderscore sectio\textunderscore )}
\end{itemize}
Dissecção ou operação cirúrgica, feita em animaes vivos.
\section{Viviseccionista}
\begin{itemize}
\item {fónica:sé}
\end{itemize}
\begin{itemize}
\item {Grp. gram.:m.}
\end{itemize}
Aquelle que pratíca a vivisecção, como processo cirúrgico ou como exploração scientífica.
\section{Vivissecção}
\begin{itemize}
\item {Grp. gram.:f.}
\end{itemize}
\begin{itemize}
\item {Proveniência:(Do lat. \textunderscore vivus\textunderscore  + \textunderscore sectio\textunderscore )}
\end{itemize}
Dissecção ou operação cirúrgica, feita em animaes vivos.
\section{Vivisseccionista}
\begin{itemize}
\item {Grp. gram.:m.}
\end{itemize}
Aquelle que pratíca a vivisecção, como processo cirúrgico ou como exploração scientífica.
\section{Viviu}
\begin{itemize}
\item {Grp. gram.:m.}
\end{itemize}
Nome de um passarinho brasileiro.
\section{Vivo}
\begin{itemize}
\item {Grp. gram.:adj.}
\end{itemize}
\begin{itemize}
\item {Grp. gram.:M.}
\end{itemize}
\begin{itemize}
\item {Grp. gram.:Pl.}
\end{itemize}
\begin{itemize}
\item {Utilização:Prov.}
\end{itemize}
\begin{itemize}
\item {Utilização:beir.}
\end{itemize}
\begin{itemize}
\item {Proveniência:(Lat. \textunderscore vivus\textunderscore )}
\end{itemize}
Que vive.
Animado: \textunderscore estilo vivo\textunderscore .
Activo.
Intenso: \textunderscore claridade viva\textunderscore .
Penetrante: \textunderscore dores vivas\textunderscore .
Persistente.
Fervoroso: \textunderscore crenças vivas\textunderscore .
Diligente.
Ardente.
Efficaz.
Rápido.
Prompto.
Persuasivo.
Criatura viva.
O sêr, que é dotado de vida: \textunderscore tratar dos vivos e enterrar os mortos\textunderscore .
Parte viva ou extremamente sensível do organismo animal.
Âmago.
Auge.
Debrum ou guarnição em peças de vestuário.
Vívula.
Designação genérica dos animaes domésticos, (bois, cavallos, ovelhas, porcos, etc.): \textunderscore dize ao criado que vá tratar do vivo\textunderscore . (Colhido na Guarda)
\section{Vivório}
\begin{itemize}
\item {Grp. gram.:m.}
\end{itemize}
\begin{itemize}
\item {Utilização:Deprec.}
\end{itemize}
\begin{itemize}
\item {Proveniência:(De \textunderscore viva\textunderscore )}
\end{itemize}
Muitos vivas.
Enthusiasmo ruidoso.
\section{Vivo-to-dou}
\begin{itemize}
\item {Grp. gram.:m.}
\end{itemize}
\begin{itemize}
\item {Utilização:T. da Bairrada}
\end{itemize}
O mesmo que \textunderscore dou-te-lo-vivo\textunderscore .
\section{Vívula}
\begin{itemize}
\item {Grp. gram.:f.}
\end{itemize}
\begin{itemize}
\item {Proveniência:(De \textunderscore vivo\textunderscore )}
\end{itemize}
Inflammação da pelle e tendões, na parte anterior da quartella da cavalgadura.
\section{Vixnutismo}
\begin{itemize}
\item {Grp. gram.:m.}
\end{itemize}
\begin{itemize}
\item {Proveniência:(De \textunderscore Víxnu\textunderscore , uma das pessôas da trimurti)}
\end{itemize}
Seita indiana.
\section{Vixnutista}
\begin{itemize}
\item {Grp. gram.:m.}
\end{itemize}
Sectário do vixnutismo.
\section{Vizconde}
\textunderscore m.\textunderscore  (e der.)
O mesmo ou melhór que \textunderscore visconde\textunderscore , etc.
(Cast. \textunderscore vizconde\textunderscore )
\section{Vizeiro}
\begin{itemize}
\item {Grp. gram.:m.}
\end{itemize}
\begin{itemize}
\item {Utilização:Ant.}
\end{itemize}
Procurador, solicitador.
(Talvez por \textunderscore vezeiro\textunderscore , que faz as vezes de)
\section{Vizindade}
\begin{itemize}
\item {Grp. gram.:f.}
\end{itemize}
\begin{itemize}
\item {Utilização:Ant.}
\end{itemize}
\begin{itemize}
\item {Proveniência:(Do cast. \textunderscore vecindad\textunderscore )}
\end{itemize}
O mesmo que \textunderscore vizinhança\textunderscore . Cf. Frei Fortun., \textunderscore Inéd.\textunderscore , 316.
\section{Vizindário}
\begin{itemize}
\item {Grp. gram.:m.}
\end{itemize}
\begin{itemize}
\item {Utilização:Bras}
\end{itemize}
O mesmo que \textunderscore vizinhança\textunderscore .
(Cast. \textunderscore vecindario\textunderscore )
\section{Vizinhal}
\begin{itemize}
\item {Grp. gram.:adj.}
\end{itemize}
Relativo a vizinho. Cf. Filinto, XII, 84.
O mesmo que \textunderscore vicinal\textunderscore .
\section{Vizinhança}
\begin{itemize}
\item {Grp. gram.:f.}
\end{itemize}
\begin{itemize}
\item {Utilização:Fig.}
\end{itemize}
Qualidade do que é vizinho.
Pessôas ou famílias vizinhas.
Arrabaldes.
Cercanias; proximidades.
Semelhança; analogia.
\section{Vizinhar}
\begin{itemize}
\item {Grp. gram.:v. i.}
\end{itemize}
\begin{itemize}
\item {Grp. gram.:V. t.}
\end{itemize}
\begin{itemize}
\item {Utilização:Ant.}
\end{itemize}
\begin{itemize}
\item {Grp. gram.:V. p.}
\end{itemize}
\begin{itemize}
\item {Proveniência:(Do lat. \textunderscore vicinari\textunderscore )}
\end{itemize}
Sêr vizinho; confinar.
Aproximar-se:«\textunderscore Angélica vizinhou da enferma\textunderscore ». Camillo, \textunderscore Bruxa\textunderscore .
Sêr vizinho de.
Estar contíguo a.
Aproximar-se.
\section{Vizinho}
\begin{itemize}
\item {Grp. gram.:adj.}
\end{itemize}
\begin{itemize}
\item {Grp. gram.:M.}
\end{itemize}
\begin{itemize}
\item {Proveniência:(Do lat. \textunderscore vicinus\textunderscore )}
\end{itemize}
Que está perto.
Que mora próximo.
Limítrophe; confinante: \textunderscore a vizinha Espanha\textunderscore .
Semelhante; análogo; parecido.
Não afastado, (falando-se de parentes).
Cada um dos habitantes de uma terra: \textunderscore o meu vizinho é coxo\textunderscore .
Família: \textunderscore esta casa tem só dois vizinhos\textunderscore .
Casa habitada: \textunderscore aldeia de cem vizinhos\textunderscore .
\section{Vizir}
\begin{itemize}
\item {Grp. gram.:m.}
\end{itemize}
\begin{itemize}
\item {Proveniência:(Do ár. \textunderscore uazir\textunderscore )}
\end{itemize}
Cada um dos principaes officiaes do conselho do Imperador da Turquia.
Ministro do Imperador da Turquia.
\section{Vizirado}
\begin{itemize}
\item {Grp. gram.:m.}
\end{itemize}
Cargo de vizir.
O tempo que dura êsse cargo.
\section{Vizirato}
\begin{itemize}
\item {Grp. gram.:m.}
\end{itemize}
O mesmo que \textunderscore vizirado\textunderscore .
\section{Vizo...}
\begin{itemize}
\item {Grp. gram.:pref.}
\end{itemize}
O mesmo que \textunderscore vice...\textunderscore 
\section{Vizófagos}
\begin{itemize}
\item {Grp. gram.:m. pl.}
\end{itemize}
Aqueles que se alimentam de... (?):«\textunderscore Junto a estes, vivem os vizófagos... barbaros...\textunderscore »\textunderscore Ethiópia Or.\textunderscore , liv. I, cap. I.
(Êrro tipográfico?)
\section{Vizóphagos}
\begin{itemize}
\item {Grp. gram.:m. pl.}
\end{itemize}
Aquelles que se alimentam de... (?):«\textunderscore Junto a estes, vivem os vizóphagos... barbaros...\textunderscore »\textunderscore Ethiópia Or.\textunderscore , liv. I, cap. I.
(Êrro typográphico?)
\section{Vizo-Rei}
\begin{itemize}
\item {Grp. gram.:m.}
\end{itemize}
O mesmo que \textunderscore Vice-Rei\textunderscore .
\section{V.^o}
Abrev. de \textunderscore verso\textunderscore  ou segunda página da fôlha de um livro antigo, onde as fôlhas só são numeradas de um lado.
Abrev. de \textunderscore verso\textunderscore  ou segunda página de autos forenses, etc.
\section{Voacanga}
\begin{itemize}
\item {Grp. gram.:f.}
\end{itemize}
Gênero de plantas apocýneas.
\section{Voaço}
\begin{itemize}
\item {Grp. gram.:m.}
\end{itemize}
\begin{itemize}
\item {Proveniência:(De \textunderscore voaçar\textunderscore , por \textunderscore esvoaçar\textunderscore )}
\end{itemize}
Fios tênues e numerosos, que esvoaçam no ar dentro das fábricas de fiação e de tecidos.
\section{Voadeiras}
\begin{itemize}
\item {Grp. gram.:f. pl.}
\end{itemize}
O mesmo que \textunderscore voadoiros\textunderscore .
\section{Voadoiros}
\begin{itemize}
\item {Grp. gram.:m. pl.}
\end{itemize}
\begin{itemize}
\item {Utilização:Fig.}
\end{itemize}
\begin{itemize}
\item {Proveniência:(De \textunderscore voar\textunderscore )}
\end{itemize}
As pennas mais compridas das asas, na extremidade dos respectivos cotos.
Guias; voadeiras.
Meios de proceder, de abrir carreira:«\textunderscore D. Andreza, escandalizada, cortava-lhe os voadoiros.\textunderscore »Camillo, \textunderscore Brasileira\textunderscore , 86.
\section{Voador}
\begin{itemize}
\item {Grp. gram.:m.  e  adj.}
\end{itemize}
\begin{itemize}
\item {Utilização:Fig.}
\end{itemize}
\begin{itemize}
\item {Grp. gram.:M.}
\end{itemize}
\begin{itemize}
\item {Utilização:Bras}
\end{itemize}
\begin{itemize}
\item {Proveniência:(Do lat. \textunderscore volator\textunderscore )}
\end{itemize}
O que vôa.
Veloz.
Muito rápido.
Acrobata, que salta de um trapézio para outro mais ou menos distante.
E diz-se de um peixe de Portugal.
Moéda falsa de cobre.
\section{Voadouros}
\begin{itemize}
\item {Grp. gram.:m. pl.}
\end{itemize}
\begin{itemize}
\item {Utilização:Fig.}
\end{itemize}
\begin{itemize}
\item {Proveniência:(De \textunderscore voar\textunderscore )}
\end{itemize}
As pennas mais compridas das asas, na extremidade dos respectivos cotos.
Guias; voadeiras.
Meios de proceder, de abrir carreira:«\textunderscore D. Andreza, escandalizada, cortava-lhe os voadoiros.\textunderscore »Camillo, \textunderscore Brasileira\textunderscore , 86.
\section{Voadura}
\begin{itemize}
\item {Grp. gram.:f.}
\end{itemize}
\begin{itemize}
\item {Proveniência:(Do lat. \textunderscore volatura\textunderscore )}
\end{itemize}
Acto ou effeito de voar.
Vôo.
\section{Voagem}
\begin{itemize}
\item {Grp. gram.:f.}
\end{itemize}
\begin{itemize}
\item {Proveniência:(De \textunderscore voar\textunderscore )}
\end{itemize}
Alimpadura ou rabeiras dos cereaes debulhados nas eiras.
\section{Voamento}
\begin{itemize}
\item {Grp. gram.:m.}
\end{itemize}
\begin{itemize}
\item {Utilização:Constr.}
\end{itemize}
O mesmo que \textunderscore saliência\textunderscore .
\section{Voante}
\begin{itemize}
\item {Grp. gram.:adj.}
\end{itemize}
\begin{itemize}
\item {Proveniência:(Do lat. \textunderscore volans\textunderscore )}
\end{itemize}
Que vôa.
Volante.
Rápido; transitório.
\section{Voar}
\begin{itemize}
\item {Grp. gram.:v. i.}
\end{itemize}
\begin{itemize}
\item {Utilização:Fig.}
\end{itemize}
\begin{itemize}
\item {Grp. gram.:V. p.}
\end{itemize}
\begin{itemize}
\item {Proveniência:(Do lat. \textunderscore volare\textunderscore )}
\end{itemize}
Suster-se ou mover-se no ar por meio de asas.
Ir pelo ar com grande rapidez, á maneira de ave.
Correr velozmente.
Desapparecer de súbito ou com rapidez.
Propalar-se rapidamente.
Consumir-se.
Decorrer constantemente, rapidamente, (falando-se do tempo).
Soffrer mudança continuamente.
Explodir.
Desapparecer no ar.
Elevar-se em pensamento.
Têr concepções sublimes.
O mesmo significado. Cf. Vieira, \textunderscore Obras Inéd.\textunderscore , II, 148.
\section{Voaria}
\begin{itemize}
\item {Grp. gram.:f.}
\end{itemize}
\begin{itemize}
\item {Utilização:Des.}
\end{itemize}
\begin{itemize}
\item {Proveniência:(De \textunderscore voar\textunderscore )}
\end{itemize}
Conjunto de aves; volataria.
\section{Voato}
\begin{itemize}
\item {Grp. gram.:m.}
\end{itemize}
(V.boato). Cf. Filinto, XXII, 133 e 146.
\section{Voaz}
\begin{itemize}
\item {Grp. gram.:adj.}
\end{itemize}
Semelhante a vôo:«\textunderscore voaz arranco.\textunderscore »Filinto, X, 112.
\section{Voborde}
\begin{itemize}
\item {Grp. gram.:m.}
\end{itemize}
Amurada de navio.
(Alter. de \textunderscore bombordo\textunderscore ?)
\section{Vocabro}
\begin{itemize}
\item {Grp. gram.:m.}
\end{itemize}
\begin{itemize}
\item {Utilização:Ant.}
\end{itemize}
O mesmo que \textunderscore vocábulo\textunderscore .
Nome; appellido.
\section{Vocabular}
\begin{itemize}
\item {Grp. gram.:m.}
\end{itemize}
\begin{itemize}
\item {Utilização:Gram.}
\end{itemize}
Relativo a vocábulo: \textunderscore notações vocabulares\textunderscore .
\section{Vocabulário}
\begin{itemize}
\item {Grp. gram.:m.}
\end{itemize}
\begin{itemize}
\item {Utilização:Ext.}
\end{itemize}
\begin{itemize}
\item {Proveniência:(Lat. \textunderscore vocabularium\textunderscore )}
\end{itemize}
Lista de vocábulos, acompanhados de explicações succintas, e dispostos geralmente por ordem alphabética.
Diccionário.
Conjunto de termos ou vocábulos, pertencentes a uma arte ou sciência.
\section{Vocabularista}
\begin{itemize}
\item {Grp. gram.:m.  e  f.}
\end{itemize}
Pessôa, que fez um vocabulário.
\section{Vocabulista}
\begin{itemize}
\item {Grp. gram.:m.  e  f.}
\end{itemize}
\begin{itemize}
\item {Proveniência:(De \textunderscore vocábulo\textunderscore )}
\end{itemize}
O mesmo que \textunderscore vocabularista\textunderscore .
\section{Vocábulo}
\begin{itemize}
\item {Grp. gram.:m.}
\end{itemize}
Palavra que faz parte de uma língua.
Termo, dicção.
\textunderscore Jôgo de vocábulo\textunderscore , trocadilho. Cf. Vieira, \textunderscore in\textunderscore  Camillo, \textunderscore Caveira\textunderscore , 460.
\section{Vocação}
\begin{itemize}
\item {Grp. gram.:f.}
\end{itemize}
\begin{itemize}
\item {Utilização:Ext.}
\end{itemize}
\begin{itemize}
\item {Proveniência:(Lat. \textunderscore vocatio\textunderscore )}
\end{itemize}
Acto de chamar.
Escolha.
Predestinação.
Tendência ou inclinação para um estado, profissão, etc.
Talento.
\section{Vocal}
\begin{itemize}
\item {Grp. gram.:adj.}
\end{itemize}
\begin{itemize}
\item {Proveniência:(Lat. \textunderscore vocalis\textunderscore )}
\end{itemize}
Relativo á voz.
Que serve para a producção da voz.
Que se exprime por meio da voz.
\section{Vocálico}
\begin{itemize}
\item {Grp. gram.:adj.}
\end{itemize}
\begin{itemize}
\item {Proveniência:(De \textunderscore vocal\textunderscore )}
\end{itemize}
Relativo ás letras vogaes.
\section{Vocalismo}
\begin{itemize}
\item {Grp. gram.:m.}
\end{itemize}
\begin{itemize}
\item {Utilização:Gram.}
\end{itemize}
\begin{itemize}
\item {Proveniência:(De \textunderscore vocal\textunderscore )}
\end{itemize}
Theoria, á cêrca das vogaes.
\section{Vocalização}
\begin{itemize}
\item {Grp. gram.:f.}
\end{itemize}
Acto ou effeito de vocalizar.
\section{Vocalizador}
\begin{itemize}
\item {Grp. gram.:m.  e  adj.}
\end{itemize}
O que vocaliza.
\section{Vocalizar}
\begin{itemize}
\item {Grp. gram.:v. t.}
\end{itemize}
\begin{itemize}
\item {Utilização:Gram.}
\end{itemize}
\begin{itemize}
\item {Proveniência:(De \textunderscore vocal\textunderscore )}
\end{itemize}
Cantar, sem articular palavras nem nomear notas, modelando a voz sôbre uma vogal, que é geralmente o \textunderscore a\textunderscore  ou o \textunderscore e\textunderscore .
Transformar, (consoantes) em vogaes.
\section{Vocalizo}
\begin{itemize}
\item {Grp. gram.:m.}
\end{itemize}
\begin{itemize}
\item {Proveniência:(De \textunderscore vocalizar\textunderscore )}
\end{itemize}
Exercício de canto sôbre uma vogal.
\section{Vocalmente}
\begin{itemize}
\item {Grp. gram.:adv.}
\end{itemize}
De modo vocal.
\section{Vocativo}
\begin{itemize}
\item {Grp. gram.:m.}
\end{itemize}
\begin{itemize}
\item {Proveniência:(Lat. \textunderscore vocativus\textunderscore )}
\end{itemize}
Caso grammatical, que se emprega para chamar alguém, nas línguas que têm casos.
Nome que, nas línguas que não têm casos, corresponde a um verbo na segunda pessôa, sem sêr o sujeito delle.
\section{Você}
\begin{itemize}
\item {fónica:vó}
\end{itemize}
Dicção pronominal, que designa tratamento, dirigido a pessôa de inferior condição, ou usado familiarmente entre pessôas que se estimam.
(Contr. de \textunderscore vossemecê\textunderscore )
\section{Vocência}
\begin{itemize}
\item {fónica:vó}
\end{itemize}
(contr. de \textunderscore Vossa Excellência\textunderscore )
\section{Vociferação}
\begin{itemize}
\item {Grp. gram.:f.}
\end{itemize}
\begin{itemize}
\item {Proveniência:(Do lat. \textunderscore vociferatio\textunderscore )}
\end{itemize}
Acto ou effeito de vociferar.
\section{Vociferador}
\begin{itemize}
\item {Grp. gram.:m.  e  adj.}
\end{itemize}
\begin{itemize}
\item {Proveniência:(Do lat. \textunderscore vociferator\textunderscore )}
\end{itemize}
O que vocifera.
\section{Vociferante}
\begin{itemize}
\item {Grp. gram.:adj.}
\end{itemize}
\begin{itemize}
\item {Proveniência:(Lat. \textunderscore vociferans\textunderscore )}
\end{itemize}
Que vocifera.
\section{Vociferar}
\begin{itemize}
\item {Grp. gram.:v. t.}
\end{itemize}
\begin{itemize}
\item {Grp. gram.:V. i.}
\end{itemize}
\begin{itemize}
\item {Proveniência:(Lat. \textunderscore vociferari\textunderscore )}
\end{itemize}
Pronunciar em voz alta ou clamorosa.
Clamar.
Falar com cólera; berrar.
\section{Vocificação}
\begin{itemize}
\item {Grp. gram.:f.}
\end{itemize}
\begin{itemize}
\item {Utilização:P. us.}
\end{itemize}
\begin{itemize}
\item {Proveniência:(Do lat. \textunderscore vox\textunderscore  + \textunderscore facere\textunderscore )}
\end{itemize}
O mesmo que \textunderscore phonação\textunderscore .
\section{Vôda}
\begin{itemize}
\item {Grp. gram.:f.}
\end{itemize}
(V.boda)
\section{Voejar}
\begin{itemize}
\item {Grp. gram.:v. i.}
\end{itemize}
\begin{itemize}
\item {Proveniência:(De \textunderscore vôo\textunderscore )}
\end{itemize}
O mesmo que \textunderscore esvoaçar\textunderscore .
\section{Voejo}
\begin{itemize}
\item {Grp. gram.:m.}
\end{itemize}
Pó, que se levanta da farinha, quando esta se agita.
Acto de voejar, adejo.
\section{Voga}
\begin{itemize}
\item {Grp. gram.:f.}
\end{itemize}
\begin{itemize}
\item {Grp. gram.:M.}
\end{itemize}
Acto de vogar.
Movimento de remos.
Divulgação.
Reputação.
Popularidade.
Uso actual, moda.
Remeiro da guiga, que vai atrás dos outros.
\section{Voga-avante}
\begin{itemize}
\item {Grp. gram.:m.}
\end{itemize}
O mesmo que \textunderscore remador\textunderscore .
\section{Vogado}
\begin{itemize}
\item {Grp. gram.:m.}
\end{itemize}
\begin{itemize}
\item {Utilização:Ant.}
\end{itemize}
O mesmo que \textunderscore advogado\textunderscore . Cf. \textunderscore Port. Mon. Hist.\textunderscore , \textunderscore Script.\textunderscore , 331.
\section{Vogal}
\begin{itemize}
\item {Grp. gram.:adj.}
\end{itemize}
\begin{itemize}
\item {Utilização:Gram.}
\end{itemize}
\begin{itemize}
\item {Grp. gram.:F.}
\end{itemize}
\begin{itemize}
\item {Grp. gram.:M.}
\end{itemize}
\begin{itemize}
\item {Proveniência:(Do lat. \textunderscore vocalis\textunderscore )}
\end{itemize}
Diz-se do som e da letra, que representa um som simples, independente de articulação.
Letra vogal.
Pessôa, que tem voto numa assembleia.
Membro de uma corporação, junta, etc.
\section{Vogante}
\begin{itemize}
\item {Grp. gram.:adj.}
\end{itemize}
Que voga.
\section{Vogar}
\begin{itemize}
\item {Grp. gram.:v. i.}
\end{itemize}
\begin{itemize}
\item {Utilização:Fig.}
\end{itemize}
\begin{itemize}
\item {Grp. gram.:V. t.}
\end{itemize}
\begin{itemize}
\item {Proveniência:(Do ant. alt. al. \textunderscore vagon\textunderscore )}
\end{itemize}
Ir sôbre a água, impellido com o auxílio de remos.
Remar.
Fluctuar.
Deslizar.
Correr, divulgar-se.
Circular.
Estar em uso.
Percorrer, navegando.
Fazer mover com os remos.
\section{Vogar}
\begin{itemize}
\item {Grp. gram.:v. t.  e  i.}
\end{itemize}
\begin{itemize}
\item {Utilização:Ant.}
\end{itemize}
\begin{itemize}
\item {Utilização:Prov.}
\end{itemize}
\begin{itemize}
\item {Utilização:trasm.}
\end{itemize}
O mesmo que \textunderscore advogar\textunderscore .
Importar, valer: \textunderscore que voga isso\textunderscore !
\section{Vogaria}
\begin{itemize}
\item {Grp. gram.:f.}
\end{itemize}
\begin{itemize}
\item {Utilização:Ant.}
\end{itemize}
\begin{itemize}
\item {Proveniência:(De \textunderscore vogar\textunderscore ^2)}
\end{itemize}
O mesmo que \textunderscore advocacia\textunderscore .
\section{Vogélia}
\begin{itemize}
\item {Grp. gram.:f.}
\end{itemize}
\begin{itemize}
\item {Proveniência:(De \textunderscore Vogeli\textunderscore , n. p.)}
\end{itemize}
Gênero de plantas plumbagíneas.
\section{Vogue}
\begin{itemize}
\item {Grp. gram.:m.}
\end{itemize}
Pequena embarcação indiana.
\section{Vogueiro}
\begin{itemize}
\item {Grp. gram.:m.}
\end{itemize}
\begin{itemize}
\item {Utilização:Prov.}
\end{itemize}
\begin{itemize}
\item {Utilização:trasm.}
\end{itemize}
O mesmo que \textunderscore argueiro\textunderscore .
\section{Vogul}
\begin{itemize}
\item {Grp. gram.:m.}
\end{itemize}
Língua uralo-altaica, vernácula na Rússia.
\section{Voivoda}
\begin{itemize}
\item {Grp. gram.:m.}
\end{itemize}
\begin{itemize}
\item {Proveniência:(Do pol. \textunderscore wojewoda\textunderscore )}
\end{itemize}
Designação antiga dos Príncipes soberanos da Moldávia, da Valáchia e de outros países.
Cobrador de impostos, na Turquia.
\section{Voivodia}
\begin{itemize}
\item {Grp. gram.:f.}
\end{itemize}
Govêrno de um vaivoda.
País, governado por um vaivoda.
\section{Volandeira}
\begin{itemize}
\item {Grp. gram.:f.}
\end{itemize}
\begin{itemize}
\item {Utilização:Bras}
\end{itemize}
O mesmo que \textunderscore bolandeira\textunderscore .
\section{Volanta}
\begin{itemize}
\item {Grp. gram.:f.}
\end{itemize}
\begin{itemize}
\item {Utilização:Prov.}
\end{itemize}
\begin{itemize}
\item {Utilização:minh.}
\end{itemize}
\begin{itemize}
\item {Proveniência:(De \textunderscore volante\textunderscore )}
\end{itemize}
Pesca no alto mar. (Colhido em Viana)
\section{Volante}
\begin{itemize}
\item {Grp. gram.:adj.}
\end{itemize}
\begin{itemize}
\item {Grp. gram.:M.}
\end{itemize}
\begin{itemize}
\item {Utilização:Pesc.}
\end{itemize}
\begin{itemize}
\item {Utilização:Prov.}
\end{itemize}
\begin{itemize}
\item {Utilização:minh.}
\end{itemize}
\begin{itemize}
\item {Proveniência:(Lat. \textunderscore volans\textunderscore )}
\end{itemize}
Que vôa ou póde voar.
Que fluctua.
Que se póde mudar facilmente.
Móvel.
Errante, que não tem domicílio certo.
Volúvel.
Passageiro, transitório, que desapparece rapidamente.
Tecido ligeiro e transparente, próprio para véus de senhora e outros enfeites.
Pequena péla, feita de substância leve e que tem pennas espetadas em volta, própria para se deitar ao ar com a raqueta.
Jôgo, em que os parceiros impellem com a raqueta êsse objecto, de uns para outros.
Seta.
Peça, que regula o movimento de um maquinismo.
Rêde de um só pano, para emmalhar pescadas.
Correia contínua, na roda das máquinas.
Lacaio; servo. Cf. Garrett, \textunderscore Helena\textunderscore , 26 e 28.
Instante, momento: \textunderscore fugiu num volante\textunderscore .
\section{Volanteira}
\begin{itemize}
\item {Grp. gram.:f.}
\end{itemize}
\begin{itemize}
\item {Utilização:Pesc.}
\end{itemize}
\begin{itemize}
\item {Proveniência:(De \textunderscore volante\textunderscore )}
\end{itemize}
Rêde, que se manobra successivamente de uma para outra posição.
\section{Volantim}
\begin{itemize}
\item {Grp. gram.:m.}
\end{itemize}
(V.volatim)
\section{Volantina}
\begin{itemize}
\item {Grp. gram.:f.}
\end{itemize}
\begin{itemize}
\item {Utilização:Prov.}
\end{itemize}
\begin{itemize}
\item {Utilização:minh.}
\end{itemize}
Momento, instante: \textunderscore desappareceu numa volantina\textunderscore .
(Cp. \textunderscore volante\textunderscore )
\section{Volapuque}
\begin{itemize}
\item {Grp. gram.:m.}
\end{itemize}
\begin{itemize}
\item {Proveniência:(Do ingl. \textunderscore world\textunderscore  + \textunderscore speak\textunderscore )}
\end{itemize}
Língua artificial, inventada em 1879.
\section{Volapuquista}
\begin{itemize}
\item {Grp. gram.:m.}
\end{itemize}
Aquelle que conhece o volapuque ou que faz delle a apologia.
\section{Volata}
\begin{itemize}
\item {Grp. gram.:f.}
\end{itemize}
\begin{itemize}
\item {Proveniência:(It. \textunderscore volata\textunderscore )}
\end{itemize}
Série de tons, executados rapidamente.
Progressão das notas de uma oitava, executadas velozmente.
\section{Volataria}
\begin{itemize}
\item {Grp. gram.:f.}
\end{itemize}
\begin{itemize}
\item {Proveniência:(Do lat. \textunderscore volatus\textunderscore )}
\end{itemize}
Arte de caçar por meio de falcões ou outras aves.
Altanaria.
Aves caçadas.
\section{Volatear}
\begin{itemize}
\item {Grp. gram.:v. i.}
\end{itemize}
\begin{itemize}
\item {Proveniência:(Do lat. \textunderscore volatus\textunderscore )}
\end{itemize}
O mesmo que \textunderscore esvoaçar\textunderscore .
\section{Volátil}
\begin{itemize}
\item {Grp. gram.:adj.}
\end{itemize}
\begin{itemize}
\item {Utilização:Fig.}
\end{itemize}
\begin{itemize}
\item {Grp. gram.:M.}
\end{itemize}
\begin{itemize}
\item {Grp. gram.:Pl.}
\end{itemize}
\begin{itemize}
\item {Proveniência:(Lat. \textunderscore volatilis\textunderscore )}
\end{itemize}
Que tem a faculdade de voar.
Voador.
Relativo a aves.
Volúvel, inconstante.
Que se póde reduzir a gás ou a vapor.
Animal que vôa; ave.
Voláteis.
\section{Volatilidade}
\begin{itemize}
\item {Grp. gram.:f.}
\end{itemize}
Qualidade do que é volátil.
\section{Volatilização}
\begin{itemize}
\item {Grp. gram.:f.}
\end{itemize}
Acto ou effeito de volatilizar.
\section{Volatilizante}
\begin{itemize}
\item {Grp. gram.:adj.}
\end{itemize}
Que volatiliza.
\section{Volatilizar}
\begin{itemize}
\item {Grp. gram.:v. t.}
\end{itemize}
\begin{itemize}
\item {Grp. gram.:V. i.  e  p.}
\end{itemize}
\begin{itemize}
\item {Proveniência:(De \textunderscore volátil\textunderscore )}
\end{itemize}
Reduzir a gás ou a vapor; vaporizar.
Reduzir-se a gás ou a vapor.
\section{Volatilizável}
\begin{itemize}
\item {Grp. gram.:adj.}
\end{itemize}
Que se póde volatilizar. Cf. \textunderscore Museu Technol.\textunderscore , 41 e 42.
\section{Volatim}
\begin{itemize}
\item {Grp. gram.:m.}
\end{itemize}
Andarilho; funâmbulo.
(Cast. \textunderscore volatin\textunderscore )
\section{Volatina}
\begin{itemize}
\item {Grp. gram.:f.}
\end{itemize}
\begin{itemize}
\item {Proveniência:(De \textunderscore volata\textunderscore )}
\end{itemize}
Trecho musical, simples e rápido.
\section{Volatíssimo}
\begin{itemize}
\item {Grp. gram.:adj.}
\end{itemize}
Muito volátil; muito subtil:«\textunderscore volatissima era nelle aquella de nós mais subtil parte\textunderscore ». Filinto, XVIII, 118.
(Má derivação de \textunderscore volátil\textunderscore )
\section{Volatório}
\begin{itemize}
\item {Grp. gram.:adj.}
\end{itemize}
\begin{itemize}
\item {Proveniência:(Do lat. \textunderscore volare\textunderscore )}
\end{itemize}
Que serve para voar: \textunderscore são volatórios os membros anteriores dos morcegos\textunderscore .
\section{Volcão}
\textunderscore m.\textunderscore  (e der.)
(V. \textunderscore vulcão\textunderscore , etc.)
\section{Volcar}
\begin{itemize}
\item {Grp. gram.:v. t.}
\end{itemize}
Tombar. Cf. Latino, \textunderscore Humboldt\textunderscore , 351.
(Cast. \textunderscore volcar\textunderscore )
\section{Volentina}
\begin{itemize}
\item {Grp. gram.:f.}
\end{itemize}
Antigo tecido de lan.
\section{Volfrâmio}
\begin{itemize}
\item {Grp. gram.:m.}
\end{itemize}
(V.tungstênio)
\section{Volfrão}
\begin{itemize}
\item {Grp. gram.:m.}
\end{itemize}
O mesmo que \textunderscore volfrâmio\textunderscore .
\section{Volição}
\begin{itemize}
\item {Grp. gram.:f.}
\end{itemize}
Acto, pelo qual se determina a vontade.
(Cp. cast. \textunderscore volición\textunderscore )
\section{Volitante}
\begin{itemize}
\item {Grp. gram.:adj.}
\end{itemize}
\begin{itemize}
\item {Grp. gram.:M. pl.}
\end{itemize}
\begin{itemize}
\item {Utilização:Zool.}
\end{itemize}
\begin{itemize}
\item {Proveniência:(De \textunderscore volitar\textunderscore )}
\end{itemize}
Que volita.
Ordem da classe dos mammíferos.
O mesmo que \textunderscore chirópteros\textunderscore .
\section{Volitar}
\begin{itemize}
\item {Grp. gram.:v. i.}
\end{itemize}
\begin{itemize}
\item {Proveniência:(Lat. \textunderscore volitare\textunderscore )}
\end{itemize}
O mesmo que \textunderscore esvoaçar\textunderscore .
\section{Volitivo}
\begin{itemize}
\item {Grp. gram.:adj.}
\end{itemize}
Relativo á volição ou á vontade.
\section{Volível}
\begin{itemize}
\item {Grp. gram.:adj.}
\end{itemize}
\begin{itemize}
\item {Utilização:P. us.}
\end{itemize}
\begin{itemize}
\item {Proveniência:(Do rad. do lat. \textunderscore volo\textunderscore )}
\end{itemize}
Que se póde querer; que póde depender da vontade.
\section{Volo}
\begin{itemize}
\item {Grp. gram.:m.}
\end{itemize}
Um dos lances do jôgo do solo.
\section{Volofo}
\begin{itemize}
\item {Grp. gram.:m.}
\end{itemize}
Língua africana do Senegal.
\section{Volovan}
\begin{itemize}
\item {Grp. gram.:m.}
\end{itemize}
\begin{itemize}
\item {Proveniência:(Fr. \textunderscore vol-au-vent\textunderscore )}
\end{itemize}
Peça de pastelaria, que contém peixe ou carne, e cujas bordas são de pastel folhado. Cf. Castilho, \textunderscore Avarento\textunderscore , 183.
\section{Volta}
\begin{itemize}
\item {Grp. gram.:f.}
\end{itemize}
\begin{itemize}
\item {Utilização:Bras. do N}
\end{itemize}
\begin{itemize}
\item {Utilização:Ant.}
\end{itemize}
\begin{itemize}
\item {Utilização:Náut.}
\end{itemize}
Acto ou effeito de voltar.
Mudança.
Réplica.
Revés.
Repercussão.
Substituição.
Giro; circuito.
Tira branca, na parte superior do cabeção dos padres, dos estudantes e lentes da universidade, e dos alumnos dos seminários.
Recado ou serviço leve de criado ou criada: \textunderscore tomou um criado para voltas\textunderscore .
Utensílio de tanoeiro, para raspar o lado côncavo das aduelas.
Solução de uma difficuldade; interpretação: \textunderscore não lhe sei dar volta\textunderscore .
Peça de tecido branco, pendente do pescoço, no uniforme de certos funccionários.
Espécie de collar, usado por mulhéres.
Curva de uma rua, estrada, etc.
Sinuosidade.
Qualquer curva.
Glosa poética, em que o glosador escolhe e distribue a seu talante as palavras do mote.
Acto de se turvar o vinho.
Compensação, que se offerece ao dono de um objecto, quando se pretende havê-lo por outro.
Desordem, briga.
\textunderscore Na volta do mar\textunderscore , ao sabor das ondas, das marés ou correntes, sem vento, e sem se poder seguir carreira.
(B. lat. \textunderscore volta\textunderscore )
\section{Volta-cara}
\begin{itemize}
\item {Grp. gram.:f.}
\end{itemize}
Acto de voltar o rosto.
\section{Voltagem}
\begin{itemize}
\item {Grp. gram.:f.}
\end{itemize}
\begin{itemize}
\item {Utilização:Phýs.}
\end{itemize}
\begin{itemize}
\item {Proveniência:(De \textunderscore vóltio\textunderscore )}
\end{itemize}
Conjunto dos vóltios, que funccionam num apparelho eléctrico.
\section{Voltaico}
\begin{itemize}
\item {Grp. gram.:adj.}
\end{itemize}
\begin{itemize}
\item {Proveniência:(De \textunderscore Volta\textunderscore , n. p.)}
\end{itemize}
Diz-se da pilha eléctrica e dos seus effeitos.
\section{Voltaireano}
\begin{itemize}
\item {fónica:té}
\end{itemize}
\begin{itemize}
\item {Grp. gram.:adj.}
\end{itemize}
\begin{itemize}
\item {Grp. gram.:M.}
\end{itemize}
Relativo a Voltaire.
Sectário das doutrinas de Voltaire.
\section{Voltairismo}
\begin{itemize}
\item {fónica:té}
\end{itemize}
\begin{itemize}
\item {Grp. gram.:m.}
\end{itemize}
Doutrina ou estilo de Voltaire. Cf. Camillo, \textunderscore Ratazzi\textunderscore , 8.
\section{Voltaísmo}
\begin{itemize}
\item {Grp. gram.:m.}
\end{itemize}
\begin{itemize}
\item {Utilização:Phýs.}
\end{itemize}
Electricidade, desenvolvida pelo contacto de substâncias heterogêneas.
(Cp. \textunderscore voltaico\textunderscore )
\section{Voltaíta}
\begin{itemize}
\item {Grp. gram.:f.}
\end{itemize}
\begin{itemize}
\item {Utilização:Miner.}
\end{itemize}
Variedade de sulfureto de ferro, que se encontra em Nápoles.
\section{Voltaíte}
\begin{itemize}
\item {Grp. gram.:f.}
\end{itemize}
\begin{itemize}
\item {Utilização:Miner.}
\end{itemize}
Variedade de sulfureto de ferro, que se encontra em Nápoles.
\section{Voltaíto}
\begin{itemize}
\item {Grp. gram.:m.}
\end{itemize}
O mesmo ou melhór que \textunderscore voltaíta\textunderscore .
\section{Voltâmetro}
\begin{itemize}
\item {Grp. gram.:m.}
\end{itemize}
\begin{itemize}
\item {Proveniência:(De \textunderscore Volta\textunderscore , n. p. + gr. \textunderscore metron\textunderscore )}
\end{itemize}
(V.voltímetro)
\section{Volta-no-meio}
\begin{itemize}
\item {Grp. gram.:m.}
\end{itemize}
\begin{itemize}
\item {Utilização:Bras}
\end{itemize}
Dança popular, usada nas roças.
\section{Voltar}
\begin{itemize}
\item {Grp. gram.:v. i.}
\end{itemize}
\begin{itemize}
\item {Grp. gram.:V. t.}
\end{itemize}
\begin{itemize}
\item {Grp. gram.:V. p.}
\end{itemize}
\begin{itemize}
\item {Proveniência:(Do lat. \textunderscore volutare\textunderscore )}
\end{itemize}
Ir ao ponto donde partiu.
Tornar a ir ou a vir.
Girar.
Tratar novamente de um assumpto.
Mudar de direcção.
Mudar a posição ou a direcção de.
Virar.
Pôr do avêsso.
Remexer.
Inclinar, dirigir.
Passar além de.
Dar em trôco ou em recompensa: \textunderscore trocámos os relógios, e êlle voltou-me déz mil reis\textunderscore .
Transformar.
Replicar.
Fazer mudar de opinião.
Turvar-se (o vinho) na mãe ou nas fezes. Cf. \textunderscore Gaz. das Ald.\textunderscore , VII, 40.
\section{Voltarete}
\begin{itemize}
\item {fónica:tarê}
\end{itemize}
\begin{itemize}
\item {Grp. gram.:m.}
\end{itemize}
\begin{itemize}
\item {Proveniência:(Do cast. \textunderscore voltareta\textunderscore )}
\end{itemize}
Jôgo de cartas, em que entram três parceiros, distribuindo-se nove cartas a cada um.
\section{Voltaretista}
\begin{itemize}
\item {Grp. gram.:m.}
\end{itemize}
Jogador de voltarete.
\section{Voltário}
\begin{itemize}
\item {Grp. gram.:adj.}
\end{itemize}
\begin{itemize}
\item {Proveniência:(De \textunderscore volta\textunderscore )}
\end{itemize}
Volúvel; inconstante:«\textunderscore diffamado pelas voltárias multidões...\textunderscore »Latino, \textunderscore Hist. Pol. e Mil.\textunderscore , I, 157.
\section{Voltazinha}
\begin{itemize}
\item {fónica:vól}
\end{itemize}
\begin{itemize}
\item {Grp. gram.:f.}
\end{itemize}
\begin{itemize}
\item {Utilização:Ant.}
\end{itemize}
Requebro na voz ou no canto. Cf. Pant. de Aveiro, \textunderscore Itiner.\textunderscore , 101 v.^o, (2.^a ed.).
\section{Volte}
\begin{itemize}
\item {Grp. gram.:m.}
\end{itemize}
\begin{itemize}
\item {Proveniência:(De \textunderscore voltar\textunderscore )}
\end{itemize}
Acto de voltar a primeira das cartas que estão na mesa, tomando como trunfo o naipe que ella indica, no jôgo do voltarete.
\section{Volteada}
\begin{itemize}
\item {Grp. gram.:f.}
\end{itemize}
\begin{itemize}
\item {Utilização:Bras. do S}
\end{itemize}
\begin{itemize}
\item {Proveniência:(De \textunderscore voltear\textunderscore )}
\end{itemize}
Acto de apanhar o gado amontado.
\section{Volteador}
\begin{itemize}
\item {Grp. gram.:m.  e  adj.}
\end{itemize}
O que volteia.
\section{Volteadura}
\begin{itemize}
\item {Grp. gram.:f.}
\end{itemize}
Acto ou effeito de voltear.
\section{Volteante}
\begin{itemize}
\item {Grp. gram.:adj.}
\end{itemize}
Que volteia.
\section{Voltear}
\begin{itemize}
\item {Grp. gram.:v. t.}
\end{itemize}
\begin{itemize}
\item {Proveniência:(De \textunderscore volta\textunderscore )}
\end{itemize}
Andar á volta de.
Fazer girar.
Fazer dar muitas voltas.
Remexer; dar voltas; rodopiar; girar.
Passar.
Esvoaçar.
\section{Volteio}
\begin{itemize}
\item {Grp. gram.:m.}
\end{itemize}
\begin{itemize}
\item {Proveniência:(De \textunderscore voltear\textunderscore )}
\end{itemize}
O mesmo que \textunderscore volteadura\textunderscore .
Exercícios de funâmbulo.
\section{Volteiro}
\begin{itemize}
\item {Grp. gram.:adj.}
\end{itemize}
\begin{itemize}
\item {Utilização:Ant.}
\end{itemize}
\begin{itemize}
\item {Proveniência:(De \textunderscore volta\textunderscore )}
\end{itemize}
Que dá voltas.
Volante.
Instável.
Brigão; desordeiro.
\section{Voltejar}
\textunderscore v. t.\textunderscore  e \textunderscore i.\textunderscore  (e der.)
O mesmo que \textunderscore voltear\textunderscore , etc.
\section{Voltiagem}
\begin{itemize}
\item {Grp. gram.:f.}
\end{itemize}
\begin{itemize}
\item {Utilização:Phýs.}
\end{itemize}
Producção de vóltios.
\section{Voltímetro}
\begin{itemize}
\item {Grp. gram.:m.}
\end{itemize}
\begin{itemize}
\item {Utilização:Phýs.}
\end{itemize}
\begin{itemize}
\item {Proveniência:(De \textunderscore vóltio\textunderscore  + gr. \textunderscore metron\textunderscore )}
\end{itemize}
Apparelho, para medição das potências eléctricas.
\section{Vóltio}
\begin{itemize}
\item {Grp. gram.:m.}
\end{itemize}
\begin{itemize}
\item {Utilização:Phýs.}
\end{itemize}
\begin{itemize}
\item {Proveniência:(De \textunderscore Volta\textunderscore , n. p.)}
\end{itemize}
Quantidade de fôrça electro-motriz, a qual, applicada a um conductor cuja resistência seja um óhmio, produz uma corrente de um ampério. Cp. \textunderscore ampério\textunderscore  e \textunderscore óhmio\textunderscore .
\section{Voltiómetro}
\begin{itemize}
\item {Grp. gram.:m.}
\end{itemize}
O mesmo que \textunderscore voltímetro\textunderscore .
\section{Voltívolo}
\begin{itemize}
\item {Grp. gram.:adj.}
\end{itemize}
\begin{itemize}
\item {Utilização:Fig.}
\end{itemize}
\begin{itemize}
\item {Proveniência:(De \textunderscore voltar\textunderscore  + lat. \textunderscore volare\textunderscore )}
\end{itemize}
Que dá muitas voltas.
Volante.
Volúvel.
\section{Volubilado}
\begin{itemize}
\item {Grp. gram.:adj.}
\end{itemize}
\begin{itemize}
\item {Utilização:Bot.}
\end{itemize}
\begin{itemize}
\item {Proveniência:(Do lat. \textunderscore volubilis\textunderscore )}
\end{itemize}
Diz-se do caule de certas plantas que, não podendo suster-se por si próprias, têm a propriedade de se enroscar nos corpos vizinhos.
\section{Volubilidade}
\begin{itemize}
\item {Grp. gram.:f.}
\end{itemize}
\begin{itemize}
\item {Proveniência:(Do lat. \textunderscore volubilitas\textunderscore )}
\end{itemize}
Qualidade do que é volúvel.
\section{Volumaço}
\begin{itemize}
\item {Grp. gram.:m.}
\end{itemize}
Grande volume.
\section{Volumão}
\begin{itemize}
\item {Grp. gram.:m.}
\end{itemize}
Grande volume.
\section{Volumar}
\begin{itemize}
\item {Grp. gram.:v. t.}
\end{itemize}
(V.avolumar)
\section{Volumar}
\begin{itemize}
\item {Grp. gram.:adj.}
\end{itemize}
\begin{itemize}
\item {Utilização:Neol.}
\end{itemize}
Relativo a volumes, em Geometria.
\section{Volume}
\begin{itemize}
\item {Grp. gram.:m.}
\end{itemize}
\begin{itemize}
\item {Utilização:Ant.}
\end{itemize}
\begin{itemize}
\item {Proveniência:(Lat. \textunderscore volumen\textunderscore )}
\end{itemize}
Livro, encadernado ou brochado, impresso ou manuscrito.
Livro.
Tomo.
Pacote.
Rôlo.
Extensão.
Corpulência.
Tamanho.
Desenvolvimento.
Intensidade (de som ou voz).
Espaço, occupado por um corpo.
Livro, que consistia em fôlhas, que se enrolavam em volta de um cylindro de madeira, osso ou marfim.
\section{Volumenómetro}
\begin{itemize}
\item {Grp. gram.:m.}
\end{itemize}
\begin{itemize}
\item {Proveniência:(Do lat. \textunderscore volumen\textunderscore  + gr. \textunderscore metron\textunderscore )}
\end{itemize}
Instrumento de Phýsica, para determinar a densidade dos corpos, sem mergulhar na água.
\section{Volumétrico}
\begin{itemize}
\item {Grp. gram.:adj.}
\end{itemize}
\begin{itemize}
\item {Proveniência:(De \textunderscore volúmetro\textunderscore )}
\end{itemize}
Relativo á determinação dos volumes.
\section{Volúmetro}
\begin{itemize}
\item {Grp. gram.:m.}
\end{itemize}
\begin{itemize}
\item {Proveniência:(De \textunderscore volume\textunderscore  + gr. \textunderscore metron\textunderscore )}
\end{itemize}
Nome de certos areómetros, que fazem conhecer a densidade dos líquidos pelos volumes deslocados.
\section{Voluminador}
\begin{itemize}
\item {Grp. gram.:m.}
\end{itemize}
\begin{itemize}
\item {Proveniência:(Do lat. hyp. \textunderscore voluminare\textunderscore )}
\end{itemize}
Aquelle que reúne fôlhas em volume; encadernador. Cf. Castilho, \textunderscore Fastos\textunderscore , I, 323.
\section{Voluminoso}
\begin{itemize}
\item {Grp. gram.:adj.}
\end{itemize}
\begin{itemize}
\item {Proveniência:(Lat. \textunderscore voluminosus\textunderscore )}
\end{itemize}
(V.volumoso)
\section{Volumoso}
\begin{itemize}
\item {Grp. gram.:adj.}
\end{itemize}
Que tem grande volume.
Que occupa grande espaço.
Que tem grandes dimensões em todos os sentidos.
Intenso, forte, (falando-se da voz ou do som).
Que comprehende muitos volumes: \textunderscore obra volumosa\textunderscore .
\section{Voluntariado}
\begin{itemize}
\item {Grp. gram.:m.}
\end{itemize}
Qualidade de voluntário no exército.
Classe dos voluntários.
\section{Voluntariamente}
\begin{itemize}
\item {Grp. gram.:adv.}
\end{itemize}
De modo voluntário.
De moto próprio; espontaneamente.
\section{Voluntariedade}
\begin{itemize}
\item {Grp. gram.:f.}
\end{itemize}
Qualidade do que é voluntário.
Espontaneidade.
Capricho, arbítrio, teima.
\section{Voluntário}
\begin{itemize}
\item {Grp. gram.:adj.}
\end{itemize}
\begin{itemize}
\item {Grp. gram.:M.}
\end{itemize}
\begin{itemize}
\item {Proveniência:(Lat. \textunderscore voluntarius\textunderscore )}
\end{itemize}
Que procede espontaneamente.
Que deriva da vontade própria: \textunderscore faltas voluntárias\textunderscore .
Em que não há coacção.
Espontâneo: instinctivo.
Voluntarioso.
Aquelle que se alista no exército espontaneamente.
Estudante, admittido a frequentar uma aula em condições differentes das dos alumnos ordinários.
\section{Voluntariosamente}
\begin{itemize}
\item {Grp. gram.:adv.}
\end{itemize}
De modo voluntarioso.
\section{Voluntariosidade}
\begin{itemize}
\item {Grp. gram.:f.}
\end{itemize}
Qualidade de voluntarioso.
\section{Voluntarioso}
\begin{itemize}
\item {Grp. gram.:adj.}
\end{itemize}
\begin{itemize}
\item {Proveniência:(De \textunderscore voluntário\textunderscore )}
\end{itemize}
Que se dirige só pela sua vontade.
Caprichoso; teimoso.
\section{Volúpia}
\begin{itemize}
\item {Grp. gram.:f.}
\end{itemize}
\begin{itemize}
\item {Proveniência:(De \textunderscore Volúpia\textunderscore , n. p.)}
\end{itemize}
O mesmo que \textunderscore voluptuosidade\textunderscore .
\section{Voluptade}
\begin{itemize}
\item {Grp. gram.:f.}
\end{itemize}
\begin{itemize}
\item {Proveniência:(Lat. \textunderscore voluptas\textunderscore )}
\end{itemize}
Termo, proposto por Bluteau, em substituição de \textunderscore voluptuosidade\textunderscore .
\section{Voluptário}
\begin{itemize}
\item {Grp. gram.:adj.}
\end{itemize}
\begin{itemize}
\item {Utilização:Ant.}
\end{itemize}
O mesmo que \textunderscore voluptuário\textunderscore .
\section{Voluptuariamente}
\begin{itemize}
\item {Grp. gram.:adv.}
\end{itemize}
\begin{itemize}
\item {Proveniência:(De \textunderscore voluptuário\textunderscore )}
\end{itemize}
O mesmo que \textunderscore voluptuosamente\textunderscore . Cf. Camillo, \textunderscore Ôlho de Vidro\textunderscore , 69.
\section{Voluptuário}
\begin{itemize}
\item {Grp. gram.:adj.}
\end{itemize}
\begin{itemize}
\item {Proveniência:(Lat. \textunderscore voluptuarius\textunderscore )}
\end{itemize}
Relativo a volúpia.
Propenso á volúpia.
Que gosta de se divertir.
Relativo a divertimentos ou a despesas supérfluas.
Relativo a gozos moraes ou materiaes.
\section{Voluptuosamente}
\begin{itemize}
\item {Grp. gram.:adv.}
\end{itemize}
De modo voluptuoso.
\section{Voluptuosidade}
\begin{itemize}
\item {Grp. gram.:f.}
\end{itemize}
Qualidade do que é voluptuoso.
Prazer sensual.
Prazer moral.
Satisfação íntima.--É gallicismo para alguns puristas. Bluteau preferia \textunderscore voluptade\textunderscore .
\section{Voluptuoso}
\begin{itemize}
\item {Grp. gram.:adj.}
\end{itemize}
\begin{itemize}
\item {Proveniência:(Lat. \textunderscore voluptuosus\textunderscore )}
\end{itemize}
Em que há prazer ou volúpia.
Sensual.
Delicioso.
Dado á libertinagem.
Que procura divertimentos ou deleites.
\section{Volúsia}
\begin{itemize}
\item {Grp. gram.:f.}
\end{itemize}
Gênero de insectos dípteros.
\section{Voluta}
\begin{itemize}
\item {Grp. gram.:f.}
\end{itemize}
\begin{itemize}
\item {Utilização:Mús.}
\end{itemize}
\begin{itemize}
\item {Proveniência:(Lat. \textunderscore voluta\textunderscore )}
\end{itemize}
Ornato de um capitel de columna, em fórma de espiral.
Concha univalve.
Parte superior da cabeça dos instrumentos de arco, entalhada em fórma de espiral.
\section{Volutabro}
\begin{itemize}
\item {Grp. gram.:m.}
\end{itemize}
\begin{itemize}
\item {Utilização:Fig.}
\end{itemize}
\begin{itemize}
\item {Proveniência:(Lat. \textunderscore volutabrum\textunderscore )}
\end{itemize}
Lodaçal; esterqueira.
O mesmo que \textunderscore torpeza\textunderscore .
\section{Volutear}
\begin{itemize}
\item {Grp. gram.:v. i.}
\end{itemize}
\begin{itemize}
\item {Grp. gram.:M.}
\end{itemize}
\begin{itemize}
\item {Proveniência:(Do lat. \textunderscore volutus\textunderscore )}
\end{itemize}
O mesmo que \textunderscore voltear\textunderscore .
Giro.
\section{Volutela}
\begin{itemize}
\item {Grp. gram.:f.}
\end{itemize}
\begin{itemize}
\item {Proveniência:(De \textunderscore voluta\textunderscore )}
\end{itemize}
Gênero de molluscos gasterópodes.
\section{Volutita}
\begin{itemize}
\item {Grp. gram.:f.}
\end{itemize}
\begin{itemize}
\item {Proveniência:(De \textunderscore voluta\textunderscore )}
\end{itemize}
Voluta ou concha univalve em estado fóssil.
\section{Volutite}
\begin{itemize}
\item {Grp. gram.:f.}
\end{itemize}
\begin{itemize}
\item {Proveniência:(De \textunderscore voluta\textunderscore )}
\end{itemize}
Voluta ou concha univalve em estado fóssil.
\section{Volúvel}
\begin{itemize}
\item {Grp. gram.:adj.}
\end{itemize}
\begin{itemize}
\item {Proveniência:(Lat. \textunderscore volubilis\textunderscore )}
\end{itemize}
Que gira.
Instável.
Inconstante.
Que se enrola em tôrno dos corpos vizinhos, (falando-se de vegetaes).
\section{Volva}
\begin{itemize}
\item {fónica:vôl}
\end{itemize}
\begin{itemize}
\item {Grp. gram.:f.}
\end{itemize}
\begin{itemize}
\item {Proveniência:(Lat. \textunderscore volva\textunderscore )}
\end{itemize}
Membrana, que envolve os cogumelos, no primeiro período do seu desenvolvimento.
\section{Volváceo}
\begin{itemize}
\item {Grp. gram.:adj.}
\end{itemize}
Que tem fórma de volva ou bôlsa.
\section{Volvado}
\begin{itemize}
\item {Grp. gram.:adj.}
\end{itemize}
Que tem volva.
\section{Volver}
\begin{itemize}
\item {Grp. gram.:v. t.}
\end{itemize}
\begin{itemize}
\item {Grp. gram.:V. i.}
\end{itemize}
\begin{itemize}
\item {Grp. gram.:M.}
\end{itemize}
\begin{itemize}
\item {Proveniência:(Lat. \textunderscore volvere\textunderscore )}
\end{itemize}
Voltar: \textunderscore volver os olhos ao passado\textunderscore .
Transportar.
Fazer rolar.
Remexer.
Meditar.
Tornar; voltar.
Revolver-se.
Revirar-se.
Decorrer.
Transformar-se.
Acto ou effeito de volver.
\section{Volvido}
\begin{itemize}
\item {Grp. gram.:adj.}
\end{itemize}
\begin{itemize}
\item {Proveniência:(De \textunderscore volver\textunderscore )}
\end{itemize}
Passado; decorrido.
\section{Volvo}
\begin{itemize}
\item {fónica:vôl}
\end{itemize}
\begin{itemize}
\item {Grp. gram.:m.}
\end{itemize}
\begin{itemize}
\item {Proveniência:(De \textunderscore volver\textunderscore . Os diccion. port. citam o lat. \textunderscore volvus\textunderscore , que me parece não existir, senão como nome de uma planta)}
\end{itemize}
Cólica violenta, em que os intestinos têm movimento opposto ao da contracção que impelle até á saída as substâncias digeridas.
\section{Volvoce}
\begin{itemize}
\item {Grp. gram.:m.}
\end{itemize}
\begin{itemize}
\item {Proveniência:(Fr. \textunderscore volvoce\textunderscore )}
\end{itemize}
Animálculo das águas estagnadas, espheroidal, sem bôca nem intestinos, e que tem na sua primeira phase uma celha vibrátil como os esporos das plantas. Cf. Caminhoá, \textunderscore Bot. Ger.\textunderscore 
\section{Vólvolo}
\begin{itemize}
\item {Grp. gram.:m.}
\end{itemize}
O mesmo que \textunderscore volvo\textunderscore .
Volta ou rosca (de serpente):«\textunderscore ...não saberemos descrever todos os vólvulos de serpente, que alli a tinham amarrado.\textunderscore »Camillo, \textunderscore Filha do Regicida\textunderscore , 37.
(Cast. \textunderscore volvulo\textunderscore , de \textunderscore volvo\textunderscore . Os diccion. citam o lat. \textunderscore volvulus\textunderscore , que me parece não existir, senão como dem. hyp. O b. lat. tinha \textunderscore volvolus\textunderscore , nome de uma rêde)
\section{Vomecê}
(Contr. de \textunderscore vocemecê\textunderscore )
\section{Vómer}
\begin{itemize}
\item {Grp. gram.:m.}
\end{itemize}
\begin{itemize}
\item {Utilização:Anat.}
\end{itemize}
\begin{itemize}
\item {Proveniência:(Lat. \textunderscore vomer\textunderscore )}
\end{itemize}
Pequeno osso, que constitue a parte posterior da parede que divide as fossas nasaes.
\section{Vomeriano}
\begin{itemize}
\item {Grp. gram.:adj.}
\end{itemize}
Relativo ao vómer.
\section{Vómica}
\begin{itemize}
\item {Grp. gram.:f.}
\end{itemize}
\begin{itemize}
\item {Proveniência:(Lat. \textunderscore vomica\textunderscore )}
\end{itemize}
Depósito purulento e fétido no parênchyma pulmonar, susceptível de se expellir pelos brônchios.
\section{Vomição}
\begin{itemize}
\item {Grp. gram.:f.}
\end{itemize}
\begin{itemize}
\item {Proveniência:(Do lat. \textunderscore vomitio\textunderscore )}
\end{itemize}
O mesmo que \textunderscore vómito\textunderscore .
\section{Vomil}
\begin{itemize}
\item {Grp. gram.:m.}
\end{itemize}
\begin{itemize}
\item {Utilização:Ant.}
\end{itemize}
O mesmo ou melhór que \textunderscore gomil\textunderscore .
(Relaciona-se talvez com o lat. \textunderscore vomere\textunderscore , vomitar)
\section{Vomitado}
\begin{itemize}
\item {Grp. gram.:m.}
\end{itemize}
\begin{itemize}
\item {Proveniência:(De \textunderscore vomitar\textunderscore )}
\end{itemize}
Aquillo que se vomitou.
\section{Vomitador}
\begin{itemize}
\item {Grp. gram.:m.  e  adj.}
\end{itemize}
O que vomita.
\section{Vomitar}
\begin{itemize}
\item {Grp. gram.:v. t.}
\end{itemize}
\begin{itemize}
\item {Utilização:Fig.}
\end{itemize}
\begin{itemize}
\item {Utilização:Pop.}
\end{itemize}
\begin{itemize}
\item {Proveniência:(Lat. \textunderscore vomitare\textunderscore )}
\end{itemize}
Expellir pela bôca (substâncias que o estômago continha).
Lançar pela bôca.
Manchar com substancias expellidas pela bôca.
Proferir com intenção de injuriar.
Pronunciar (coisas vergonhosas ou irreverentes).
Expellir impetuosamente.
Jorrar.
Espalhar, despejar.
Causar.
Contar (o que era um segrêdo).
Desembuchar; dizer.
\section{Vomitivo}
\begin{itemize}
\item {Grp. gram.:adj.}
\end{itemize}
\begin{itemize}
\item {Grp. gram.:M.}
\end{itemize}
\begin{itemize}
\item {Proveniência:(Lat. \textunderscore vomitivus\textunderscore )}
\end{itemize}
Que produz vómito.
O mesmo que \textunderscore vomitório\textunderscore .
\section{Vómito}
\begin{itemize}
\item {Grp. gram.:m.}
\end{itemize}
\begin{itemize}
\item {Proveniência:(Lat. \textunderscore vomitus\textunderscore )}
\end{itemize}
Acto ou effeito de vomitar.
O vomitado.
\section{Vómito-negro}
\begin{itemize}
\item {Grp. gram.:m.}
\end{itemize}
Doença epidêmica, o mesmo que febre amarela.
\section{Vomitório}
\begin{itemize}
\item {Grp. gram.:adj.}
\end{itemize}
\begin{itemize}
\item {Grp. gram.:M.}
\end{itemize}
\begin{itemize}
\item {Proveniência:(Lat. \textunderscore vomitorius\textunderscore )}
\end{itemize}
Que faz vomitar.
Vomitivo.
Substância medicamentosa, destinada a provocar o vómito.
Entre os antigos Romanos, compartimento de uma casa, onde os convivas, depois de encher o estômago, iam vomitar, voltando para a mesa a encher novamente o estômago.
Portas, que davam entrada para os degraus dos theatros e amphitheatros romanos, e pelas quaes a multidão era, por assim dizer, vomitada.
\section{Vonda}
\begin{itemize}
\item {Grp. gram.:adv.}
\end{itemize}
\begin{itemize}
\item {Utilização:Ant.}
\end{itemize}
\begin{itemize}
\item {Grp. gram.:Interj.}
\end{itemize}
Muito.
Basta! bonda! Cf. G. Vicente.
(Cp. \textunderscore bonda!\textunderscore )
\section{Vontade}
\begin{itemize}
\item {Grp. gram.:f.}
\end{itemize}
\begin{itemize}
\item {Grp. gram.:Pl.}
\end{itemize}
\begin{itemize}
\item {Utilização:Ant.}
\end{itemize}
\begin{itemize}
\item {Proveniência:(Do lat. \textunderscore voluntas\textunderscore )}
\end{itemize}
Potência ou faculdade interior, em virtude da qual o homem e ainda os animaes se determinam a fazer ou não fazer alguma coisa.
Desejo.
Desígnio; resolução.
Talante; capricho.
Espontaneidade.
Prazer.
Appetite.
Desvelo.
Necessidade phýsica ou moral: \textunderscore vontade de comer\textunderscore .
Tendência, disposição de espírito.
Appetites; caprichos: \textunderscore a menina quere que lhe façam as vontades\textunderscore .
Móveis ou alfaias de uma casa ou quinta.
\section{Vonvoleiro}
\begin{itemize}
\item {Grp. gram.:m.}
\end{itemize}
Planta indiana, de flôres semelhantes a estrêllas e muito aromáticas. Cf. Th. Ribeiro, \textunderscore Jornadas\textunderscore , II, 107.
\section{Vôo}
\begin{itemize}
\item {Grp. gram.:m.}
\end{itemize}
\begin{itemize}
\item {Utilização:Fig.}
\end{itemize}
\begin{itemize}
\item {Proveniência:(De \textunderscore voar\textunderscore )}
\end{itemize}
Modo e meio do locomoção, próprio dos animaes que têm asas ou órgãos aliformes.
Extensão, que uma ave percorre de uma vez, voando.
Movimento rápido de qualquer objecto pelo ar: \textunderscore o vôo da seta\textunderscore .
Marcha rápida.
Elevação do pensamento ou do talento.
Arroubamento, extase.
\section{Voracidade}
\begin{itemize}
\item {Grp. gram.:f.}
\end{itemize}
\begin{itemize}
\item {Proveniência:(Do lat. \textunderscore voracitas\textunderscore )}
\end{itemize}
Qualidade do que é voraz.
\section{Voragem}
\begin{itemize}
\item {Grp. gram.:f.}
\end{itemize}
\begin{itemize}
\item {Utilização:Fig.}
\end{itemize}
\begin{itemize}
\item {Proveniência:(Lat. \textunderscore vorago\textunderscore )}
\end{itemize}
Aquillo que sorve ou devora.
Sorvedoiro.
Redemoínho no mar.
Qualquer abysmo.
Tudo que consome ou subverte.
\section{Voraginoso}
\begin{itemize}
\item {Grp. gram.:adj.}
\end{itemize}
\begin{itemize}
\item {Proveniência:(Lat. \textunderscore voraginosus\textunderscore )}
\end{itemize}
Em que há voragem.
Que tem a fórma ou a natureza de voragem.
Que consome ou subverte como a voragem.
\section{Voraz}
\begin{itemize}
\item {Grp. gram.:adj.}
\end{itemize}
\begin{itemize}
\item {Utilização:Fig.}
\end{itemize}
\begin{itemize}
\item {Proveniência:(Lat. \textunderscore vorax\textunderscore )}
\end{itemize}
Que devora.
Que come com avidez.
Que se não farta.
Que consome ou subverte com violência.
Destruidor.
Muito ávido, ambicioso.
\section{Vorazmente}
\begin{itemize}
\item {Grp. gram.:adv.}
\end{itemize}
De modo voraz; com voracidade.
\section{Vórmio}
\begin{itemize}
\item {Grp. gram.:adj.}
\end{itemize}
\begin{itemize}
\item {Utilização:Anat.}
\end{itemize}
\begin{itemize}
\item {Proveniência:(De \textunderscore Wormius\textunderscore , n. p.)}
\end{itemize}
Diz-se dos pequenos ossos, variáveis quanto ao número e á fórma, e collocados ordinariamente nos ângulos das suturas cranianas.
\section{Vórtice}
\begin{itemize}
\item {Grp. gram.:m.}
\end{itemize}
\begin{itemize}
\item {Proveniência:(Lat. \textunderscore vortex\textunderscore )}
\end{itemize}
Turbilhão.
Redemoínho.
Furacão.
\section{Vorticela}
\begin{itemize}
\item {Grp. gram.:f.}
\end{itemize}
Gênero de infusórios.
(Dem. de \textunderscore vórtice\textunderscore )
\section{Vorticoso}
\begin{itemize}
\item {Grp. gram.:adj.}
\end{itemize}
\begin{itemize}
\item {Proveniência:(De \textunderscore vórtice \textunderscore )}
\end{itemize}
Que fórma redemoínho.
Que se move em turbilhão.
\section{Vortilhão}
\begin{itemize}
\item {Grp. gram.:m.}
\end{itemize}
\begin{itemize}
\item {Utilização:T. do Pôrto  e  bras}
\end{itemize}
Grande vórtice.
Reunião de muitas águas.
(Cp. \textunderscore vórtice\textunderscore )
\section{Vos}
\begin{itemize}
\item {fónica:vus}
\end{itemize}
(Flexão do pron. \textunderscore vós\textunderscore )
\section{Vós}
\begin{itemize}
\item {Grp. gram.:pron.}
\end{itemize}
\begin{itemize}
\item {Proveniência:(Lat. \textunderscore vos\textunderscore )}
\end{itemize}
(indicativo de várias pessôas a que se fala. Antigamente, também se dirigia vulgarmente a uma só pessôa, a quem se queria tratar com cortesia ou carinho. Hoje, raramente se emprega nêste segundo caso)
\section{Vosco}
\begin{itemize}
\item {fónica:vôs}
\end{itemize}
\begin{itemize}
\item {Grp. gram.:pron.}
\end{itemize}
\begin{itemize}
\item {Utilização:Ant.}
\end{itemize}
O mesmo que \textunderscore convosco\textunderscore .
\section{Vòssemecê}
\begin{itemize}
\item {Grp. gram.:f.}
\end{itemize}
Tratamento, que de ordinário se dirige a pessôas de mediana condição.
(Contr. de \textunderscore Vossa-Mercê\textunderscore )
\section{Vosso}
\begin{itemize}
\item {Grp. gram.:pron. adj.}
\end{itemize}
Pertencente a vós.
Relativo a vós.
(Cp. it. \textunderscore vostro\textunderscore )
\section{Votação}
\begin{itemize}
\item {Grp. gram.:f.}
\end{itemize}
Acto ou effeito de votar.
Conjunto dos votos de uma assembleia eleitoral.
\section{Votamares}
(?):«\textunderscore ...cuidastes.... que vos entrasse com mantenhauos Deos votamares.\textunderscore »\textunderscore Eufrosina\textunderscore , (no prólogo).
\section{Votante}
\begin{itemize}
\item {Grp. gram.:m. ,  f.  e  adj.}
\end{itemize}
Pessôa, que vota.
\section{Votar}
\begin{itemize}
\item {Grp. gram.:v. t.}
\end{itemize}
\begin{itemize}
\item {Grp. gram.:V. i.}
\end{itemize}
\begin{itemize}
\item {Proveniência:(De \textunderscore voto\textunderscore )}
\end{itemize}
Approvar por meio de voto: \textunderscore votar um projecto de lei\textunderscore .
Eleger por meio de voto.
Prometer por meio de voto ou solennemente.
Dedicar: \textunderscore votar affecto a alguém\textunderscore .
Conferir.
Conceder.
Consagrar.
Sacrificar.
Dar ou emittir voto.
Manifestar por voto o que sente ou pensa.
Jurar. Cf. \textunderscore Filodemo\textunderscore , act. I, sc. VII.
\section{Vôte!}
\begin{itemize}
\item {Grp. gram.:interj.}
\end{itemize}
\begin{itemize}
\item {Utilização:Bras}
\end{itemize}
O mesmo que \textunderscore tíbi!\textunderscore 
\section{Vossência}
\begin{itemize}
\item {fónica:vó}
\end{itemize}
(contr. de \textunderscore Vossa Excellência\textunderscore )
\section{Votiáco}
\begin{itemize}
\item {Grp. gram.:m.}
\end{itemize}
Língua uralo-altaica, vernácula na Rússia.
\section{Votivo}
\begin{itemize}
\item {Grp. gram.:adj.}
\end{itemize}
\begin{itemize}
\item {Proveniência:(Lat. \textunderscore votivus\textunderscore )}
\end{itemize}
Relativo ao voto.
Offerecido em cumprimento de voto.
\section{Voto}
\begin{itemize}
\item {Grp. gram.:m.}
\end{itemize}
\begin{itemize}
\item {Proveniência:(Lat. \textunderscore votum\textunderscore )}
\end{itemize}
Promessa solenne, com que nos obrigamos para com a Divindade.
Promessa solenne.
Juramento.
Offerenda, que se faz em cumprimento de promessa anterior ou em testemunho de gratidão por um benefício recebido.
Súpplica á Divindade.
Desejo íntimo, ardente: \textunderscore os meus votos são pela tua felicidade\textunderscore .
Modo de manifestar a vontade ou opinião de um indivíduo num acto eleitoral, numa assembleia consultiva ou deliberativa.
Decisão.
Suffrágio.
Cada lista que, num acto eleitoral, manifesta a opinião de cada eleitor: \textunderscore teve oito votos para deputado\textunderscore .
\section{Voto}
\begin{itemize}
\item {Grp. gram.:m.}
\end{itemize}
Língua uralo-altaica, do ramo ugro-finlandês.
\section{Votona}
\begin{itemize}
\item {Grp. gram.:f.}
\end{itemize}
Pensão hereditária, na Índia Portuguesa.
\section{Vouvés}
\begin{itemize}
\item {Grp. gram.:m. pl.}
\end{itemize}
Tríbo de Índios sertanejos do Brasil, que tem vivido junto da serra Araripe.
\section{Vovente}
\begin{itemize}
\item {Grp. gram.:m. ,  f.  e  adj.}
\end{itemize}
\begin{itemize}
\item {Proveniência:(Lat. \textunderscore vovens\textunderscore )}
\end{itemize}
Pessôa, que faz votos ou promessas.
\section{Vô-vô}
\begin{itemize}
\item {Grp. gram.:m.}
\end{itemize}
\begin{itemize}
\item {Utilização:Bras}
\end{itemize}
\begin{itemize}
\item {Utilização:Infant.}
\end{itemize}
O mesmo que \textunderscore avô\textunderscore .
\section{Vó-vó}
\begin{itemize}
\item {Grp. gram.:f.}
\end{itemize}
\begin{itemize}
\item {Utilização:Bras}
\end{itemize}
\begin{itemize}
\item {Utilização:Infant.}
\end{itemize}
O mesmo que \textunderscore avó\textunderscore .
\section{Voz}
\begin{itemize}
\item {Grp. gram.:f.}
\end{itemize}
\begin{itemize}
\item {Utilização:Ant.}
\end{itemize}
\begin{itemize}
\item {Proveniência:(Lat. \textunderscore vox\textunderscore )}
\end{itemize}
Producção de um som na larynge, especialmente na larynge humana.
Som da larynge, subordinado ás regras do canto.
Faculdade de falar.
Grito.
Queixa.
Parte vocal de uma composição musical.
Ordem, em voz alta: \textunderscore á voz de marchar\textunderscore .
Boato.
Palavra; phrase.
Rumor, ruído.
Som, representado na escrita por uma vogal.
Modificação, que se dá nos verbos de algumas línguas, para indicar se o sujeito pratíca a acção ou é objecto della.
Faculdade de falar em seu nome ou em nome de outrem.
Suggestão íntima: \textunderscore a voz da consciência\textunderscore .
O mesmo que \textunderscore voto\textunderscore ^1.
\section{Vozaria}
\begin{itemize}
\item {Grp. gram.:f.}
\end{itemize}
O mesmo que \textunderscore vozearia\textunderscore . Cf. \textunderscore Peregrinação\textunderscore , XXXII, etc.
\section{Vozeada}
\begin{itemize}
\item {Grp. gram.:f.}
\end{itemize}
\begin{itemize}
\item {Proveniência:(De \textunderscore vozear\textunderscore )}
\end{itemize}
O mesmo que \textunderscore vozearia\textunderscore :«\textunderscore ...a guerrilha, cuja vozeada se approximava.\textunderscore »Camillo, \textunderscore Brasileira\textunderscore , 54.
\section{Vozeador}
\begin{itemize}
\item {Grp. gram.:m.  e  adj.}
\end{itemize}
O que vozeia.
\section{Vozeamento}
\begin{itemize}
\item {Grp. gram.:m.}
\end{itemize}
O mesmo que \textunderscore vozearia\textunderscore .
\section{Vozear}
\begin{itemize}
\item {Grp. gram.:v. i.}
\end{itemize}
\begin{itemize}
\item {Grp. gram.:V. t.}
\end{itemize}
\begin{itemize}
\item {Grp. gram.:M.}
\end{itemize}
\begin{itemize}
\item {Proveniência:(De \textunderscore voz\textunderscore )}
\end{itemize}
Falar em voz alta.
Clamar, gritar.
Proferir em voz alta.
Clamor; grito.
\section{Vozearia}
\begin{itemize}
\item {Grp. gram.:f.}
\end{itemize}
Acto de vozear.
Clamor de muitas vozes reunidas.
\section{Vozeio}
\begin{itemize}
\item {Grp. gram.:m.}
\end{itemize}
Acto ou effeito de vozear.
\section{Vozeirada}
\begin{itemize}
\item {Grp. gram.:f.}
\end{itemize}
\begin{itemize}
\item {Utilização:Prov.}
\end{itemize}
\begin{itemize}
\item {Utilização:alg.}
\end{itemize}
Tolice.
(Cp. \textunderscore vozeirão\textunderscore )
\section{Vozeirão}
\begin{itemize}
\item {Grp. gram.:m.}
\end{itemize}
\begin{itemize}
\item {Proveniência:(De \textunderscore vozeiro\textunderscore )}
\end{itemize}
Voz muito forte.
Pessôa, que tem voz muito grossa.
\section{Vozeirar}
\begin{itemize}
\item {Grp. gram.:v. i.}
\end{itemize}
\begin{itemize}
\item {Proveniência:(De \textunderscore vozeiro\textunderscore )}
\end{itemize}
Têr ou soltar voz forte.
\section{Vozeiro}
\begin{itemize}
\item {Grp. gram.:m.  e  adj.}
\end{itemize}
\begin{itemize}
\item {Utilização:Ant.}
\end{itemize}
\begin{itemize}
\item {Proveniência:(Do b. lat. \textunderscore vociarius\textunderscore )}
\end{itemize}
O que fala muito.
Palrador.
Vozeirão.
Procurador, advogado, aquelle que tem voz ou voto em nome de outrem.
\section{Vrancelhas}
\begin{itemize}
\item {fónica:cê}
\end{itemize}
\begin{itemize}
\item {Grp. gram.:f. pl.}
\end{itemize}
Variedade de uva tinta do Minho.
\section{Vu}
\begin{itemize}
\item {Grp. gram.:m.}
\end{itemize}
\begin{itemize}
\item {Utilização:Bras}
\end{itemize}
O mesmo que \textunderscore puíta\textunderscore .
\section{Vuarame}
\begin{itemize}
\item {Grp. gram.:m.}
\end{itemize}
Nome de dois arbustos esterculiáceos do Brasil.
\section{Vuba}
\begin{itemize}
\item {Grp. gram.:f.}
\end{itemize}
Nome de duas plantas gramíneas do Brasil.
\section{Vulcanaes}
\begin{itemize}
\item {Grp. gram.:f. pl.}
\end{itemize}
\begin{itemize}
\item {Proveniência:(Lat. \textunderscore vulcanalia\textunderscore )}
\end{itemize}
Festas annuaes que se celebravam em Roma, a 23 de Agosto, em honra de Vulcano.
\section{Vulcanais}
\begin{itemize}
\item {Grp. gram.:f. pl.}
\end{itemize}
\begin{itemize}
\item {Proveniência:(Lat. \textunderscore vulcanalia\textunderscore )}
\end{itemize}
Festas annuaes que se celebravam em Roma, a 23 de Agosto, em honra de Vulcano.
\section{Vulcâneo}
\begin{itemize}
\item {Grp. gram.:adj.}
\end{itemize}
Relativo a Vulcano. Cf. \textunderscore Lusíadas\textunderscore , IX, 35.
\section{Vulcaniano}
\begin{itemize}
\item {Grp. gram.:adj.}
\end{itemize}
Relativo ao vulcanismo. Cf. Latino, \textunderscore Camões\textunderscore , 92.
\section{Vulcanicidade}
\begin{itemize}
\item {Grp. gram.:f.}
\end{itemize}
\begin{itemize}
\item {Proveniência:(De \textunderscore vulcânico\textunderscore )}
\end{itemize}
Incandescência do centro da Terra.
Estado daquillo que tem origem vulcânica.
Acção dos vulcões.
\section{Vulcânico}
\begin{itemize}
\item {Grp. gram.:adj.}
\end{itemize}
\begin{itemize}
\item {Utilização:Fig.}
\end{itemize}
Relativo a vulcão.
Constituído por lavas.
Impetuoso, ardente.
\section{Vulcanismo}
\begin{itemize}
\item {Grp. gram.:m.}
\end{itemize}
\begin{itemize}
\item {Utilização:Fig.}
\end{itemize}
\begin{itemize}
\item {Proveniência:(De \textunderscore vulcão\textunderscore )}
\end{itemize}
Acção dos vulcões.
Hypóthese scientífica, que attribue a formação da crosta da Terra á acção do fogo.
Irrupção desastrosa.
\section{Vulcanista}
\begin{itemize}
\item {Grp. gram.:m. ,  f.  e  adj.}
\end{itemize}
\begin{itemize}
\item {Proveniência:(De \textunderscore vulcão\textunderscore )}
\end{itemize}
Pessôa, que é sectária do vulcanismo.
\section{Vulcanite}
\begin{itemize}
\item {Grp. gram.:f.}
\end{itemize}
Substância, composta de borracha vulcanizada, enxôfre e sílica, e inaccessível á acção dos ácidos e dos dissolventes.
(Cp. \textunderscore vulcanizar\textunderscore )
\section{Vulcanização}
\begin{itemize}
\item {Grp. gram.:f.}
\end{itemize}
\begin{itemize}
\item {Proveniência:(De \textunderscore vulcanizar\textunderscore )}
\end{itemize}
Combinação de uma pequena porção de enxôfre com a borracha.
\section{Vulcanizar}
\begin{itemize}
\item {Grp. gram.:v. t.}
\end{itemize}
\begin{itemize}
\item {Utilização:Fig.}
\end{itemize}
\begin{itemize}
\item {Proveniência:(De \textunderscore vulcão\textunderscore )}
\end{itemize}
Calcinar.
Sujeitar á vulcanização (a borracha).
Tornar ardente, enthusiasmar, exaltar.
\section{Vulcanologia}
\begin{itemize}
\item {Grp. gram.:f.}
\end{itemize}
\begin{itemize}
\item {Proveniência:(De \textunderscore vulcão\textunderscore  + gr. \textunderscore logos\textunderscore )}
\end{itemize}
Parte da Geologia, que trata dos vulcões. Cf. Latino, \textunderscore Humboldt\textunderscore , 364.
\section{Vulcão}
\begin{itemize}
\item {Grp. gram.:m.}
\end{itemize}
\begin{itemize}
\item {Utilização:Fig.}
\end{itemize}
Montanha, ou abertura numa montanha, donde sáem turbilhões de fogo e substâncias em fusão.
Imaginação fogosa.
Perigo imminente contra a ordem social: \textunderscore a Rússia está sôbre um vulcão\textunderscore .
(Cast. \textunderscore volcan\textunderscore )
\section{Vulgacho}
\begin{itemize}
\item {Grp. gram.:m.}
\end{itemize}
\begin{itemize}
\item {Proveniência:(De \textunderscore vulgo\textunderscore )}
\end{itemize}
A camada inferior da sociedade; vulgo.
Arraia miúda.
\section{Vulgar}
\begin{itemize}
\item {Grp. gram.:v. t.}
\end{itemize}
\begin{itemize}
\item {Proveniência:(Lat. \textunderscore vulgare\textunderscore )}
\end{itemize}
Tornar conhecido do vulgo.
Divulgar.
Tornar público.
\section{Vulgar}
\begin{itemize}
\item {Grp. gram.:adj.}
\end{itemize}
\begin{itemize}
\item {Grp. gram.:M.}
\end{itemize}
\begin{itemize}
\item {Proveniência:(Lat. \textunderscore vulgaris\textunderscore )}
\end{itemize}
Relativo ao vulgo.
Notório.
Commum; trivial: \textunderscore expressões vulgares\textunderscore .
Usado.
Reles.
Aquillo que é vulgar.
Língua vernácula.
\section{Vulgaridade}
\begin{itemize}
\item {Grp. gram.:f.}
\end{itemize}
\begin{itemize}
\item {Proveniência:(Do lat. \textunderscore vulgaritas\textunderscore )}
\end{itemize}
Qualidade do que é vulgar.
Coisa ou pessôa vulgar.
\section{Vulgarismo}
\begin{itemize}
\item {Grp. gram.:m.}
\end{itemize}
\begin{itemize}
\item {Proveniência:(De \textunderscore vulgar\textunderscore ^2)}
\end{itemize}
O falar ou o pensar, próprio do vulgo; vulgaridade.
\section{Vulgarização}
\begin{itemize}
\item {Grp. gram.:f.}
\end{itemize}
Acto ou effeito de vulgarizar.
\section{Vulgarizador}
\begin{itemize}
\item {Grp. gram.:m.  e  adj.}
\end{itemize}
O que vulgariza.
\section{Vulgarizar}
\begin{itemize}
\item {Grp. gram.:v. i.}
\end{itemize}
\begin{itemize}
\item {Proveniência:(De \textunderscore vulgar\textunderscore ^2)}
\end{itemize}
Tornar vulgar ou notório.
Divulgar; propagar: \textunderscore vulgarizar princípios elevados\textunderscore .
\section{Vulgarmente}
\begin{itemize}
\item {Grp. gram.:adv.}
\end{itemize}
De modo vulgar.
Commummente; em geral.
\section{Vulgata}
\begin{itemize}
\item {Grp. gram.:f.}
\end{itemize}
\begin{itemize}
\item {Utilização:P. us.}
\end{itemize}
\begin{itemize}
\item {Proveniência:(Lat. \textunderscore vulgata\textunderscore )}
\end{itemize}
Versão latina da \textunderscore Bíblia\textunderscore , feita no século IV e attribuída a San-Jerónymo.
O vulgo, o populacho.
\section{Vulgívago}
\begin{itemize}
\item {Grp. gram.:adj.}
\end{itemize}
\begin{itemize}
\item {Proveniência:(Lat. \textunderscore vulgivagus\textunderscore )}
\end{itemize}
Que se avilta, que se abandalha, que se prostitue.
\section{Vulgo}
\begin{itemize}
\item {Grp. gram.:m.}
\end{itemize}
\begin{itemize}
\item {Proveniência:(Lat. \textunderscore vulgus\textunderscore )}
\end{itemize}
O povo; a plebe.
O commum dos homens.
\section{Vulgò}
\begin{itemize}
\item {Grp. gram.:adv.}
\end{itemize}
\begin{itemize}
\item {Proveniência:(T. lat.)}
\end{itemize}
O mesmo que \textunderscore vulgarmente\textunderscore .
\section{Vulgocracia}
\begin{itemize}
\item {Grp. gram.:f.}
\end{itemize}
\begin{itemize}
\item {Proveniência:(Do lat. \textunderscore vulgus\textunderscore  + gr. \textunderscore kratos\textunderscore )}
\end{itemize}
Predomínio da classe popular; democracia.
\section{Vulnerabilidade}
\begin{itemize}
\item {Grp. gram.:f.}
\end{itemize}
Qualidade de vulnerável.
\section{Vulneração}
\begin{itemize}
\item {Grp. gram.:f.}
\end{itemize}
\begin{itemize}
\item {Proveniência:(Do lat. \textunderscore vulneratio\textunderscore )}
\end{itemize}
Acto ou effeito de vulnerar.
\section{Vulneral}
\begin{itemize}
\item {Grp. gram.:adj.}
\end{itemize}
O mesmo que \textunderscore vulnerário\textunderscore .
\section{Vulnerante}
\begin{itemize}
\item {Grp. gram.:adj.}
\end{itemize}
\begin{itemize}
\item {Proveniência:(Lat. \textunderscore vulnerans\textunderscore )}
\end{itemize}
Que vulnera.
\section{Vulnerar}
\begin{itemize}
\item {Grp. gram.:v. t.}
\end{itemize}
\begin{itemize}
\item {Utilização:Fig.}
\end{itemize}
\begin{itemize}
\item {Proveniência:(Lat. \textunderscore vulnerare\textunderscore )}
\end{itemize}
Ferir.
Melindrar, offender.
\section{Vulnerária}
\begin{itemize}
\item {Grp. gram.:f.}
\end{itemize}
\begin{itemize}
\item {Proveniência:(De \textunderscore vulnerario\textunderscore )}
\end{itemize}
Planta leguminosa, de flôres amarelas, applicável contra as feridas recentes.
\section{Vulnerário}
\begin{itemize}
\item {Grp. gram.:adj.}
\end{itemize}
\begin{itemize}
\item {Proveniência:(Lat. \textunderscore vulnerarius\textunderscore )}
\end{itemize}
Próprio para curar feridas.
\section{Vulnerativo}
\begin{itemize}
\item {Grp. gram.:adj.}
\end{itemize}
O mesmo que \textunderscore vulnerante\textunderscore .
\section{Vulnerável}
\begin{itemize}
\item {Grp. gram.:adj.}
\end{itemize}
\begin{itemize}
\item {Proveniência:(Do lat. \textunderscore vulnerabilis\textunderscore )}
\end{itemize}
Que se póde vulnerar.
Diz-se do lado fraco de um assumpto ou questão, e do ponto por onde alguém póde sêr ferido ou atacado.
\section{Vulnífico}
\begin{itemize}
\item {Grp. gram.:adj.}
\end{itemize}
\begin{itemize}
\item {Proveniência:(Lat. \textunderscore vulnificus\textunderscore )}
\end{itemize}
Que fere ou póde ferir.
\section{Vulpina}
\begin{itemize}
\item {Grp. gram.:f.}
\end{itemize}
\begin{itemize}
\item {Utilização:Chím.}
\end{itemize}
Substância còrante, extrahida de uma espécie de líchen, (\textunderscore líchen valpinus\textunderscore , Lin.).
\section{Vulpinita}
\begin{itemize}
\item {Grp. gram.:f.}
\end{itemize}
\begin{itemize}
\item {Proveniência:(De \textunderscore Vulpino\textunderscore , n. p.)}
\end{itemize}
Espécie de mármore.
\section{Vulpinite}
\begin{itemize}
\item {Grp. gram.:f.}
\end{itemize}
\begin{itemize}
\item {Proveniência:(De \textunderscore Vulpino\textunderscore , n. p.)}
\end{itemize}
Espécie de mármore.
\section{Vulpinito}
\begin{itemize}
\item {Grp. gram.:m.}
\end{itemize}
O mesmo ou melhór que \textunderscore vulpinita\textunderscore .
\section{Vulpino}
\begin{itemize}
\item {Grp. gram.:adj.}
\end{itemize}
\begin{itemize}
\item {Utilização:Fig.}
\end{itemize}
\begin{itemize}
\item {Proveniência:(Lat. \textunderscore vulpinus\textunderscore )}
\end{itemize}
Relativo á raposa ou próprio della.
Astuto, manhoso; traiçoeiro.
\section{Vulto}
\begin{itemize}
\item {Grp. gram.:m.}
\end{itemize}
\begin{itemize}
\item {Utilização:Fig.}
\end{itemize}
\begin{itemize}
\item {Proveniência:(Lat. \textunderscore vultus\textunderscore )}
\end{itemize}
Rosto, aspecto.
Corpo, figura.
Figura indistinta: \textunderscore avistei um vulto...\textunderscore 
Imagem.
Volume; tamanho.
Importância.
Pessôa importante: \textunderscore os principaes vultos políticos da villa\textunderscore .
Ponderação.
\section{Vultoso}
\begin{itemize}
\item {Grp. gram.:adj.}
\end{itemize}
\begin{itemize}
\item {Proveniência:(De \textunderscore vulto\textunderscore )}
\end{itemize}
O mesmo que \textunderscore volumoso\textunderscore .
\section{Vultuosidade}
\begin{itemize}
\item {Grp. gram.:f.}
\end{itemize}
Qualidade do que é vultuoso.
\section{Vultuoso}
\begin{itemize}
\item {Grp. gram.:adj.}
\end{itemize}
\begin{itemize}
\item {Utilização:Med.}
\end{itemize}
\begin{itemize}
\item {Proveniência:(Lat. \textunderscore vultuosus\textunderscore )}
\end{itemize}
Diz-se do rosto, quando as faces e os lábios estão vermelhos e inchados, os olhos salientes e mais ou menos injectados.
\section{Vulturinamente}
\begin{itemize}
\item {Grp. gram.:adv.}
\end{itemize}
De modo vulturino.
\section{Vulturino}
\begin{itemize}
\item {Grp. gram.:adj.}
\end{itemize}
\begin{itemize}
\item {Proveniência:(Lat. \textunderscore vulturinus\textunderscore )}
\end{itemize}
Relativo ao abutre ou próprio delle.
\section{Vulturnaes}
\begin{itemize}
\item {Grp. gram.:f. pl.}
\end{itemize}
\begin{itemize}
\item {Proveniência:(Lat. \textunderscore vulturnalia\textunderscore )}
\end{itemize}
Antigas festas romanas, em honra do deus Vulturno.
\section{Vulturnais}
\begin{itemize}
\item {Grp. gram.:f. pl.}
\end{itemize}
\begin{itemize}
\item {Proveniência:(Lat. \textunderscore vulturnalia\textunderscore )}
\end{itemize}
Antigas festas romanas, em honra do deus Vulturno.
\section{Vulturno}
\begin{itemize}
\item {Grp. gram.:m.}
\end{itemize}
\begin{itemize}
\item {Proveniência:(Lat. \textunderscore vulturnus\textunderscore )}
\end{itemize}
Vento de suèste. Cf. J. Castilho, \textunderscore Grin. Ovid.\textunderscore 
\section{Vulva}
\begin{itemize}
\item {Grp. gram.:f.}
\end{itemize}
\begin{itemize}
\item {Utilização:Anat.}
\end{itemize}
\begin{itemize}
\item {Proveniência:(Lat. \textunderscore vulva\textunderscore )}
\end{itemize}
Parte exterior do apparelho genital da mulher.
\section{Vulvar}
\begin{itemize}
\item {Grp. gram.:adj.}
\end{itemize}
Relativo á vulva.
\section{Vulvária}
\begin{itemize}
\item {Grp. gram.:f.}
\end{itemize}
\begin{itemize}
\item {Proveniência:(De \textunderscore vulvário\textunderscore )}
\end{itemize}
Espécie de anserina fétida.
\section{Vulvário}
\begin{itemize}
\item {Grp. gram.:adj.}
\end{itemize}
O mesmo que \textunderscore vulvar\textunderscore .
\section{Vulvite}
\begin{itemize}
\item {Grp. gram.:f.}
\end{itemize}
Inflammação da vulva.
\section{Vulvo-uterino}
\begin{itemize}
\item {Grp. gram.:adj.}
\end{itemize}
\begin{itemize}
\item {Proveniência:(De \textunderscore vulva\textunderscore  + \textunderscore uterino\textunderscore )}
\end{itemize}
Relativo á vulva e ao útero conjuntamente.
\section{Vum-vum}
\begin{itemize}
\item {Grp. gram.:m.}
\end{itemize}
Árvore medicinal da ilha de San-Thomé.
\section{Vunda}
\begin{itemize}
\item {Grp. gram.:m.}
\end{itemize}
Personagem, que é uma espécie de Duque, na senzala do soba dos Jingas. Cf. Capello e Ivens, II, 38.
\section{Vunge}
\begin{itemize}
\item {Grp. gram.:m.}
\end{itemize}
\begin{itemize}
\item {Utilização:Bras}
\end{itemize}
Homem esperto, atilado.
\section{Vunzar}
\begin{itemize}
\item {Grp. gram.:v. t.}
\end{itemize}
\begin{itemize}
\item {Utilização:Bras. da Baía}
\end{itemize}
Remexer (gaveta ou caixa ou mala).
\section{Vurubana}
\begin{itemize}
\item {Grp. gram.:m.}
\end{itemize}
Peixe da América meridional, semelhante á truta.
\section{Vurmo}
\begin{itemize}
\item {Grp. gram.:m.}
\end{itemize}
\begin{itemize}
\item {Proveniência:(Do lat. \textunderscore vulnus\textunderscore )}
\end{itemize}
O pus das chagas.
\section{Vurmoso}
\begin{itemize}
\item {Grp. gram.:adj.}
\end{itemize}
\end{document}