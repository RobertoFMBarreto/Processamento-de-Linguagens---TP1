
\begin{itemize}
\item {Proveniência: }
\end{itemize}\documentclass{article}
\usepackage[portuguese]{babel}
\title{X}
\begin{document}
Gênero de plantas apocýneas.
\section{X}
\begin{itemize}
\item {fónica:xis}
\end{itemize}
\begin{itemize}
\item {Grp. gram.:m.}
\end{itemize}
\begin{itemize}
\item {Grp. gram.:Loc.}
\end{itemize}
\begin{itemize}
\item {Utilização:fam.}
\end{itemize}
\begin{itemize}
\item {Utilização:Mathem.}
\end{itemize}
\begin{itemize}
\item {Utilização:Ext.}
\end{itemize}
\begin{itemize}
\item {Proveniência:(Lat. \textunderscore x\textunderscore )}
\end{itemize}
Vigésima terceira letra do alphabeto português.
Designação de \textunderscore déz\textunderscore , em numeração romana.
\textunderscore Uma de X\textunderscore , quantia mínima:«\textunderscore póde dar os bens ao outro filho, que eu não lhe quero uma de X.\textunderscore »Camillo, \textunderscore Brasileira\textunderscore , 29.
Valor desconhecido numa quantidade ou equação.
Aquillo que se desconhece.
\section{X}
\begin{itemize}
\item {Grp. gram.:adj. pl.}
\end{itemize}
Diz-se dos raios, em que se baseia o processo da photographia através dos corpos opacos.
\section{Xá}
\begin{itemize}
\item {Grp. gram.:m.}
\end{itemize}
Título do soberano da Pérsia. Cf. \textunderscore Ethiópia Or.\textunderscore , II, 27.
(Do persa \textunderscore xah\textunderscore )
\section{Xàbândar}
\begin{itemize}
\item {Grp. gram.:m.}
\end{itemize}
\begin{itemize}
\item {Utilização:Ant.}
\end{itemize}
Patrão de pôrto, entre os Índios.
(Pers. \textunderscore xah bánder\textunderscore , de \textunderscore xah\textunderscore , rei, e \textunderscore bánder\textunderscore , pôrto)
\section{Xàbandaria}
\begin{itemize}
\item {Grp. gram.:f.}
\end{itemize}
\begin{itemize}
\item {Utilização:Ant.}
\end{itemize}
Cargo xàbândar.
E talvez ribeira das naus, entre os Índios. Cf. Barros, \textunderscore Dec.\textunderscore  I, liv. IX, fol. 8.^o v.^o; Castanheda, \textunderscore Descobr.\textunderscore , liv. V, 307; liv. III, 77; Filinto, \textunderscore D. Man.\textunderscore , II, 278.
\section{Xàbânder}
\begin{itemize}
\item {Grp. gram.:m.}
\end{itemize}
\begin{itemize}
\item {Utilização:Ant.}
\end{itemize}
Patrão de pôrto, entre os Índios.
(Pers. \textunderscore xah bánder\textunderscore , de \textunderscore xah\textunderscore , rei, e \textunderscore bánder\textunderscore , pôrto)
\section{Xaboco}
\begin{itemize}
\item {fónica:bô}
\end{itemize}
\begin{itemize}
\item {Grp. gram.:m.}
\end{itemize}
\begin{itemize}
\item {Utilização:T. do Ribatejo}
\end{itemize}
\begin{itemize}
\item {Proveniência:(Do ár. \textunderscore xaboke\textunderscore )}
\end{itemize}
Lagôa ou grande poço de água.
\section{Xabrega}
\begin{itemize}
\item {Grp. gram.:f.}
\end{itemize}
\begin{itemize}
\item {Utilização:Prov.}
\end{itemize}
\begin{itemize}
\item {Utilização:alg.}
\end{itemize}
Estrago, damno, especialmente nas vinhas, campos, etc.
\section{Xabregano}
\begin{itemize}
\item {Grp. gram.:m.}
\end{itemize}
\begin{itemize}
\item {Utilização:Ext.}
\end{itemize}
Frade franciscano do convento de Xabregas.
Franciscano. Cf. Filinto, VI, 27; X, 134.
\section{Xácara}
\begin{itemize}
\item {Grp. gram.:f.}
\end{itemize}
\begin{itemize}
\item {Proveniência:(Do ár. \textunderscore zacara\textunderscore )}
\end{itemize}
Narrativa popular, em verso.
\section{Xacoco}
\begin{itemize}
\item {fónica:cô}
\end{itemize}
\begin{itemize}
\item {Grp. gram.:m.  e  adj.}
\end{itemize}
\begin{itemize}
\item {Utilização:Fam.}
\end{itemize}
O mesmo que \textunderscore enxacoco\textunderscore .
O que é desemxabido, desengraçado, ordinário.
\section{Xacoina}
\begin{itemize}
\item {Grp. gram.:f.}
\end{itemize}
\begin{itemize}
\item {Utilização:Ant.}
\end{itemize}
O mesmo que \textunderscore chacona\textunderscore :«\textunderscore ...na xacoina por cima tem mui differente sentido este conceito.\textunderscore »\textunderscore Anat. Joc.\textunderscore , II, 433.
\section{Xácoma}
\begin{itemize}
\item {Grp. gram.:f.}
\end{itemize}
\begin{itemize}
\item {Utilização:Ant.}
\end{itemize}
O mesmo que \textunderscore xáquema\textunderscore .
\section{Xacoto}
\begin{itemize}
\item {fónica:cô}
\end{itemize}
\begin{itemize}
\item {Grp. gram.:m.}
\end{itemize}
\begin{itemize}
\item {Utilização:Prov.}
\end{itemize}
\begin{itemize}
\item {Utilização:trasm.}
\end{itemize}
Pau pequeno.
(Cp. \textunderscore jangoto\textunderscore )
\section{Xadrez}
\begin{itemize}
\item {Grp. gram.:m.}
\end{itemize}
\begin{itemize}
\item {Utilização:Bras}
\end{itemize}
\begin{itemize}
\item {Proveniência:(Do ár. \textunderscore ax-xitrenj\textunderscore )}
\end{itemize}
Jôgo, sôbre um tabuleiro com 64 casas, em que se fazem mover differentes peças ou figuras.
Tabuleiro dêsse jôgo.
Gênero de tecidos, cujas côres são dispostas em quadrados alternados, como as casas do tabuleiro, em que se joga o xadrez.
Embutidos de pedra ou madeira, que dão o aspecto daquelles quadrados.
Engradamento de madeira, que serve de sobrado, a bordo.
Insecto lepidóptero.
Mosaico.
Pôsto policial.
\section{Xadrezar}
\begin{itemize}
\item {Grp. gram.:v. t.}
\end{itemize}
Dispor em fórma de xadrez; enxadrezar.
\section{Xadrezista}
\begin{itemize}
\item {Grp. gram.:m.}
\end{itemize}
O mesmo que \textunderscore enxadrezista\textunderscore .
\section{Xaes}
\begin{itemize}
\item {Grp. gram.:m.}
\end{itemize}
Antiga moéda de prata, na Pérsia. Cf. Tenreiro, \textunderscore Itiner.\textunderscore , c. XV, 368.
(Cp. \textunderscore xal\textunderscore )
\section{Xafarraz}
\begin{itemize}
\item {Grp. gram.:m.}
\end{itemize}
Espécie de jôgo popular.
\section{Xágara}
\begin{itemize}
\item {Grp. gram.:f.}
\end{itemize}
Espécie de açúcar, o mesmo que \textunderscore jágara\textunderscore .
\section{Xaguão}
\begin{itemize}
\item {Grp. gram.:m.}
\end{itemize}
(Fórma preferível a \textunderscore saguão\textunderscore )
\section{Xaimão}
\begin{itemize}
\item {Grp. gram.:m.}
\end{itemize}
\begin{itemize}
\item {Utilização:Pesc.}
\end{itemize}
Cabo, preso por uma extremidade na forcada da sardinheira, fixando a outra na fateixa de tanchar a rêde.
\section{Xaimel}
\begin{itemize}
\item {Grp. gram.:m.}
\end{itemize}
O mesmo que \textunderscore enxaimel\textunderscore .
\section{Xairel}
\begin{itemize}
\item {Grp. gram.:m.}
\end{itemize}
\begin{itemize}
\item {Utilização:Prov.}
\end{itemize}
\begin{itemize}
\item {Utilização:trasm.}
\end{itemize}
\begin{itemize}
\item {Grp. gram.:Adj.}
\end{itemize}
\begin{itemize}
\item {Utilização:Prov.}
\end{itemize}
\begin{itemize}
\item {Utilização:trasm.}
\end{itemize}
Cobertura da cavalgadura, sôbre que se põe o sellim ou a albarda.
Chale ordinário; vestido reles.
O mesmo que \textunderscore xairelado\textunderscore .
Fraco, adoentado.
(Cp. cast. \textunderscore girel\textunderscore , do ár. \textunderscore jilel\textunderscore )
\section{Xairelado}
\begin{itemize}
\item {Grp. gram.:adj.}
\end{itemize}
\begin{itemize}
\item {Proveniência:(De \textunderscore xairel\textunderscore )}
\end{itemize}
Diz-se do cavallo, que tem mancha branca no selladoiro.
\section{Xaja}
\begin{itemize}
\item {Grp. gram.:f.}
\end{itemize}
Planta tinctorial da Índia.
\section{Xal}
\begin{itemize}
\item {Grp. gram.:m.}
\end{itemize}
O mesmo que \textunderscore xaes\textunderscore .
\section{Xalapa}
\begin{itemize}
\item {Grp. gram.:f.}
\end{itemize}
O mesmo que \textunderscore jalapa\textunderscore ^1.
\section{Xalavar}
\begin{itemize}
\item {Grp. gram.:m.}
\end{itemize}
\begin{itemize}
\item {Utilização:Pesc.}
\end{itemize}
Rêde, de fórma quási cónica, em que se deita peixe, para medir ou determinar a porção pescada.
\section{Xale}
\begin{itemize}
\item {Grp. gram.:m.}
\end{itemize}
O mesmo ou melhór que \textunderscore chale\textunderscore .
\section{Xales}
\begin{itemize}
\item {Grp. gram.:m.}
\end{itemize}
\begin{itemize}
\item {Utilização:Prov.}
\end{itemize}
(V. \textunderscore chale\textunderscore ^1)
\section{Xalmas}
\begin{itemize}
\item {Grp. gram.:f. pl.}
\end{itemize}
Engradamento, que se faz num carro ou num barco, para segurar a palha que se transporta.
O mesmo que \textunderscore xelma\textunderscore .
(Cp. cast. \textunderscore jalma\textunderscore )
\section{Xamacocos}
\begin{itemize}
\item {Grp. gram.:m. pl.}
\end{itemize}
\begin{itemize}
\item {Utilização:Bras}
\end{itemize}
Tríbo de aborígenes de Mato-Grosso.
\section{Xamas}
\begin{itemize}
\item {Grp. gram.:m. pl.}
\end{itemize}
Indígenas do Norte do Brasil.
\section{Xamata}
\begin{itemize}
\item {Grp. gram.:f.}
\end{itemize}
Espécie de manto oriental.
\section{Xàmáta}
\begin{itemize}
\item {Grp. gram.:m.}
\end{itemize}
O mesmo que \textunderscore xàmáte\textunderscore . Cf. Mad. Feijó, \textunderscore Orthogr.\textunderscore 
\section{Xàmáte}
\begin{itemize}
\item {Grp. gram.:m.}
\end{itemize}
Posição, em que uma das figuras do xadrez, chamada cavallo, não póde mover-se sem sêr comida por outra figura ou peça do adversário.
(Contr. de \textunderscore xaque\textunderscore  + \textunderscore mate\textunderscore )
\section{Xangó}
\begin{itemize}
\item {Grp. gram.:m.}
\end{itemize}
\begin{itemize}
\item {Utilização:Bras}
\end{itemize}
Pequeno peixe marítimo.
\section{Xanteína}
\begin{itemize}
\item {Grp. gram.:f.}
\end{itemize}
\begin{itemize}
\item {Proveniência:(Do gr. \textunderscore xanthos\textunderscore )}
\end{itemize}
Matéria còrante, que se extrai da dáhlia amarela.
\section{Xantelasma}
\begin{itemize}
\item {Grp. gram.:m.}
\end{itemize}
\begin{itemize}
\item {Utilização:Med.}
\end{itemize}
\begin{itemize}
\item {Proveniência:(Do gr. \textunderscore xanthos\textunderscore  + \textunderscore elasma\textunderscore )}
\end{itemize}
Formação de pequenas placas amareladas na pelle.
\section{Xanteloma}
\begin{itemize}
\item {Grp. gram.:m.}
\end{itemize}
(V.xantoma)
\section{Xantena}
\begin{itemize}
\item {Grp. gram.:f.}
\end{itemize}
Variedade de pedra preciosa.
(Cp. \textunderscore xanto\textunderscore )
\section{Xantheína}
\begin{itemize}
\item {Grp. gram.:f.}
\end{itemize}
\begin{itemize}
\item {Proveniência:(Do gr. \textunderscore xanthos\textunderscore )}
\end{itemize}
Matéria còrante, que se extrai da dáhlia amarela.
\section{Xanthelasma}
\begin{itemize}
\item {Grp. gram.:m.}
\end{itemize}
\begin{itemize}
\item {Utilização:Med.}
\end{itemize}
\begin{itemize}
\item {Proveniência:(Do gr. \textunderscore xanthos\textunderscore  + \textunderscore elasma\textunderscore )}
\end{itemize}
Formação de pequenas placas amareladas na pelle.
\section{Xantheloma}
\begin{itemize}
\item {Grp. gram.:m.}
\end{itemize}
(V.xanthoma)
\section{Xanthena}
\begin{itemize}
\item {Grp. gram.:f.}
\end{itemize}
Variedade de pedra preciosa.
(Cp. \textunderscore xantho\textunderscore )
\section{Xânthico}
\begin{itemize}
\item {Grp. gram.:adj.}
\end{itemize}
\begin{itemize}
\item {Utilização:Chím.}
\end{itemize}
\begin{itemize}
\item {Proveniência:(Do gr. \textunderscore xantos\textunderscore , amarelo)}
\end{itemize}
Relativo á côr amarela.
Diz-se de um ácido; e diz-se de um óxydo que é a xanthina azotada.
\section{Xanthina}
\begin{itemize}
\item {Grp. gram.:f.}
\end{itemize}
\begin{itemize}
\item {Proveniência:(Do gr. \textunderscore xanthos\textunderscore )}
\end{itemize}
Substância còrante, extrahida da granza.
\section{Xânthio}
\begin{itemize}
\item {Grp. gram.:m.}
\end{itemize}
\begin{itemize}
\item {Proveniência:(Do gr. \textunderscore xanthion\textunderscore )}
\end{itemize}
Planta aquática.
\section{Xantho}
\begin{itemize}
\item {Grp. gram.:m.}
\end{itemize}
\begin{itemize}
\item {Proveniência:(Lat. \textunderscore xanthos\textunderscore )}
\end{itemize}
Pedra preciosa amarela, mencionada pelos antigos, mas desconhecida hoje.
Gênero de crustáceos.
Gênero de plantas, que segregam um suco amarelado.
\section{Xanthochromia}
\begin{itemize}
\item {Grp. gram.:f.}
\end{itemize}
\begin{itemize}
\item {Utilização:Med.}
\end{itemize}
\begin{itemize}
\item {Proveniência:(Do gr. \textunderscore xanthos\textunderscore  + \textunderscore khroma\textunderscore )}
\end{itemize}
Côr amarelada da pelle, devida, não á ictericia, mas á generalização do xanthelasma.
\section{Xanthócomo}
\begin{itemize}
\item {Grp. gram.:m.}
\end{itemize}
\begin{itemize}
\item {Proveniência:(Do gr. \textunderscore xanthos\textunderscore  + \textunderscore kome\textunderscore )}
\end{itemize}
Gênero de plantas americanas.
\section{Xanthogênico}
\begin{itemize}
\item {Grp. gram.:adj.}
\end{itemize}
\begin{itemize}
\item {Proveniência:(Do gr. \textunderscore xanthos\textunderscore  + \textunderscore genos\textunderscore )}
\end{itemize}
Diz-se do micróbio da febre amarela.
\section{Xanthoma}
\begin{itemize}
\item {Grp. gram.:m.}
\end{itemize}
\begin{itemize}
\item {Utilização:Med.}
\end{itemize}
\begin{itemize}
\item {Proveniência:(Do gr. \textunderscore xanthos\textunderscore , amarelo)}
\end{itemize}
Formação de pequenas placas amareladas na pelle.--É preferível \textunderscore xantheplasma\textunderscore .
\section{Xanthophylla}
\begin{itemize}
\item {Grp. gram.:f.}
\end{itemize}
\begin{itemize}
\item {Utilização:Chím.}
\end{itemize}
\begin{itemize}
\item {Proveniência:(Do gr. \textunderscore xanthos\textunderscore  + \textunderscore phullon\textunderscore )}
\end{itemize}
Substância còrante, que se desenvolve, quando as fôlhas caem, no outono.
\section{Xanthopicrito}
\begin{itemize}
\item {Grp. gram.:m.}
\end{itemize}
\begin{itemize}
\item {Utilização:Chím.}
\end{itemize}
Substância amarela e amarga, que se extrai de algumas xanthoxýleas.
\section{Xanthoproteico}
\begin{itemize}
\item {Grp. gram.:adj.}
\end{itemize}
\begin{itemize}
\item {Utilização:Chím.}
\end{itemize}
\begin{itemize}
\item {Proveniência:(Do gr. \textunderscore xanthos\textunderscore  + \textunderscore protos\textunderscore )}
\end{itemize}
Diz-se de um ácido amarelo, que resulta da decomposição de substâncias orgânicas azotadas, pela acção do ácido nítrico.
\section{Xanthopsia}
\begin{itemize}
\item {Grp. gram.:f.}
\end{itemize}
\begin{itemize}
\item {Proveniência:(Do gr. \textunderscore xanthos\textunderscore  + \textunderscore ops\textunderscore )}
\end{itemize}
Doença visual dos que vêem tudo amarelo.
\section{Xanthóptero}
\begin{itemize}
\item {Grp. gram.:adj.}
\end{itemize}
\begin{itemize}
\item {Grp. gram.:M.}
\end{itemize}
\begin{itemize}
\item {Proveniência:(Do gr. \textunderscore xanthos\textunderscore  + \textunderscore pteron\textunderscore )}
\end{itemize}
Que tem asas amarelas.
Gênero de insectos lepidópteros nocturnos da Rússia meridional.
\section{Xanthorrheia}
\begin{itemize}
\item {Grp. gram.:f.}
\end{itemize}
\begin{itemize}
\item {Proveniência:(Do gr. \textunderscore xanthos\textunderscore  + \textunderscore rhein\textunderscore )}
\end{itemize}
Gênero de plantas monocotyledóneas da Nova-Hollanda.
\section{Xanthorrhizo}
\begin{itemize}
\item {Grp. gram.:adj.}
\end{itemize}
\begin{itemize}
\item {Utilização:Bot.}
\end{itemize}
\begin{itemize}
\item {Proveniência:(Do gr. \textunderscore xanthos\textunderscore  + \textunderscore rhiza\textunderscore )}
\end{itemize}
Que tem raízes amarelas.
\section{Xanthose}
\begin{itemize}
\item {Grp. gram.:f.}
\end{itemize}
\begin{itemize}
\item {Utilização:Chím.}
\end{itemize}
\begin{itemize}
\item {Proveniência:(Do gr. \textunderscore xanthos\textunderscore )}
\end{itemize}
Substância amarela, que se encontra nas manchas irregulares do caranguejo.
\section{Xanthospermo}
\begin{itemize}
\item {Grp. gram.:adj.}
\end{itemize}
\begin{itemize}
\item {Utilização:Bot.}
\end{itemize}
\begin{itemize}
\item {Proveniência:(Do gr. \textunderscore xanthos\textunderscore  + \textunderscore sperma\textunderscore )}
\end{itemize}
Que tem sementes amarelas.
\section{Xanthoxýleas}
\begin{itemize}
\item {fónica:csi}
\end{itemize}
\begin{itemize}
\item {Grp. gram.:f. pl.}
\end{itemize}
\begin{itemize}
\item {Proveniência:(De \textunderscore xanthóxylo\textunderscore )}
\end{itemize}
Tríbo de plantas rutáceas, que outros consideram família.
\section{Xanthóxylo}
\begin{itemize}
\item {fónica:csi}
\end{itemize}
\begin{itemize}
\item {Grp. gram.:adj.}
\end{itemize}
\begin{itemize}
\item {Utilização:Bot.}
\end{itemize}
\begin{itemize}
\item {Grp. gram.:M.}
\end{itemize}
\begin{itemize}
\item {Proveniência:(Do gr. \textunderscore xanthos\textunderscore  + \textunderscore xulon\textunderscore )}
\end{itemize}
Cuja madeira é amarela.
Arbusto, (\textunderscore xanthophylum fraxineum\textunderscore , Wild.), que é o typo das xanthoxýleas.
\section{Xântico}
\begin{itemize}
\item {Grp. gram.:adj.}
\end{itemize}
\begin{itemize}
\item {Utilização:Chím.}
\end{itemize}
\begin{itemize}
\item {Proveniência:(Do gr. \textunderscore xantos\textunderscore , amarelo)}
\end{itemize}
Relativo á côr amarela.
Diz-se de um ácido; e diz-se de um óxido que é a xantina azotada.
\section{Xantina}
\begin{itemize}
\item {Grp. gram.:f.}
\end{itemize}
\begin{itemize}
\item {Proveniência:(Do gr. \textunderscore xanthos\textunderscore )}
\end{itemize}
Substância còrante, extraida da granza.
\section{Xântio}
\begin{itemize}
\item {Grp. gram.:m.}
\end{itemize}
\begin{itemize}
\item {Proveniência:(Do gr. \textunderscore xanthion\textunderscore )}
\end{itemize}
Planta aquática.
\section{Xanto}
\begin{itemize}
\item {Grp. gram.:m.}
\end{itemize}
\begin{itemize}
\item {Proveniência:(Lat. \textunderscore xanthos\textunderscore )}
\end{itemize}
Pedra preciosa amarela, mencionada pelos antigos, mas desconhecida hoje.
Gênero de crustáceos.
Gênero de plantas, que segregam um suco amarelado.
\section{Xantocromia}
\begin{itemize}
\item {Grp. gram.:f.}
\end{itemize}
\begin{itemize}
\item {Utilização:Med.}
\end{itemize}
\begin{itemize}
\item {Proveniência:(Do gr. \textunderscore xanthos\textunderscore  + \textunderscore khroma\textunderscore )}
\end{itemize}
Côr amarelada da pele, devida, não á ictericia, mas á generalização do xantelasma.
\section{Xantócomo}
\begin{itemize}
\item {Grp. gram.:m.}
\end{itemize}
\begin{itemize}
\item {Proveniência:(Do gr. \textunderscore xanthos\textunderscore  + \textunderscore kome\textunderscore )}
\end{itemize}
Gênero de plantas americanas.
\section{Xantofila}
\begin{itemize}
\item {Grp. gram.:f.}
\end{itemize}
\begin{itemize}
\item {Utilização:Chím.}
\end{itemize}
\begin{itemize}
\item {Proveniência:(Do gr. \textunderscore xanthos\textunderscore  + \textunderscore phullon\textunderscore )}
\end{itemize}
Substância còrante, que se desenvolve, quando as fôlhas caem, no outono.
\section{Xantogênico}
\begin{itemize}
\item {Grp. gram.:adj.}
\end{itemize}
\begin{itemize}
\item {Proveniência:(Do gr. \textunderscore xanthos\textunderscore  + \textunderscore genos\textunderscore )}
\end{itemize}
Diz-se do micróbio da febre amarela.
\section{Xantoma}
\begin{itemize}
\item {Grp. gram.:m.}
\end{itemize}
\begin{itemize}
\item {Utilização:Med.}
\end{itemize}
\begin{itemize}
\item {Proveniência:(Do gr. \textunderscore xanthos\textunderscore , amarelo)}
\end{itemize}
Formação de pequenas placas amareladas na pelle.--É preferível \textunderscore xanteplasma\textunderscore .
\section{Xantopicrito}
\begin{itemize}
\item {Grp. gram.:m.}
\end{itemize}
\begin{itemize}
\item {Utilização:Chím.}
\end{itemize}
Substância amarela e amarga, que se extrai de algumas xantoxíleas.
\section{Xantoproteico}
\begin{itemize}
\item {Grp. gram.:adj.}
\end{itemize}
\begin{itemize}
\item {Utilização:Chím.}
\end{itemize}
\begin{itemize}
\item {Proveniência:(Do gr. \textunderscore xanthos\textunderscore  + \textunderscore protos\textunderscore )}
\end{itemize}
Diz-se de um ácido amarelo, que resulta da decomposição de substâncias orgânicas azotadas, pela acção do ácido nítrico.
\section{Xantopsia}
\begin{itemize}
\item {Grp. gram.:f.}
\end{itemize}
\begin{itemize}
\item {Proveniência:(Do gr. \textunderscore xanthos\textunderscore  + \textunderscore ops\textunderscore )}
\end{itemize}
Doença visual dos que vêem tudo amarelo.
\section{Xantóptero}
\begin{itemize}
\item {Grp. gram.:adj.}
\end{itemize}
\begin{itemize}
\item {Grp. gram.:M.}
\end{itemize}
\begin{itemize}
\item {Proveniência:(Do gr. \textunderscore xanthos\textunderscore  + \textunderscore pteron\textunderscore )}
\end{itemize}
Que tem asas amarelas.
Gênero de insectos lepidópteros nocturnos da Rússia meridional.
\section{Xantorreia}
\begin{itemize}
\item {Grp. gram.:f.}
\end{itemize}
\begin{itemize}
\item {Proveniência:(Do gr. \textunderscore xanthos\textunderscore  + \textunderscore rhein\textunderscore )}
\end{itemize}
Gênero de plantas monocotiledóneas da Nova-Holanda.
\section{Xantorrizo}
\begin{itemize}
\item {Grp. gram.:adj.}
\end{itemize}
\begin{itemize}
\item {Utilização:Bot.}
\end{itemize}
\begin{itemize}
\item {Proveniência:(Do gr. \textunderscore xanthos\textunderscore  + \textunderscore rhiza\textunderscore )}
\end{itemize}
Que tem raízes amarelas.
\section{Xantose}
\begin{itemize}
\item {Grp. gram.:f.}
\end{itemize}
\begin{itemize}
\item {Utilização:Chím.}
\end{itemize}
\begin{itemize}
\item {Proveniência:(Do gr. \textunderscore xanthos\textunderscore )}
\end{itemize}
Substância amarela, que se encontra nas manchas irregulares do caranguejo.
\section{Xantospermo}
\begin{itemize}
\item {Grp. gram.:adj.}
\end{itemize}
\begin{itemize}
\item {Utilização:Bot.}
\end{itemize}
\begin{itemize}
\item {Proveniência:(Do gr. \textunderscore xanthos\textunderscore  + \textunderscore sperma\textunderscore )}
\end{itemize}
Que tem sementes amarelas.
\section{Xantoxíleas}
\begin{itemize}
\item {fónica:csi}
\end{itemize}
\begin{itemize}
\item {Grp. gram.:f. pl.}
\end{itemize}
\begin{itemize}
\item {Proveniência:(De \textunderscore xantóxilo\textunderscore )}
\end{itemize}
Tríbo de plantas rutáceas, que outros consideram família.
\section{Xantóxilo}
\begin{itemize}
\item {fónica:csi}
\end{itemize}
\begin{itemize}
\item {Grp. gram.:adj.}
\end{itemize}
\begin{itemize}
\item {Utilização:Bot.}
\end{itemize}
\begin{itemize}
\item {Grp. gram.:M.}
\end{itemize}
\begin{itemize}
\item {Proveniência:(Do gr. \textunderscore xanthos\textunderscore  + \textunderscore xulon\textunderscore )}
\end{itemize}
Cuja madeira é amarela.
Arbusto, (\textunderscore xanthophylum fraxineum\textunderscore , Wild.), que é o tipo das xantoxíleas.
\section{Xaperus}
\begin{itemize}
\item {Grp. gram.:m. pl.}
\end{itemize}
Indígenas do Norte do Brasil.
\section{Xaque}
\begin{itemize}
\item {Grp. gram.:m.}
\end{itemize}
O mesmo ou melhór que \textunderscore xeque\textunderscore , no xadrez. Cf. R. Lobo, \textunderscore Côrte na Aldeia\textunderscore , I, 9.
\section{Xaquear}
\begin{itemize}
\item {Grp. gram.:v. t.}
\end{itemize}
Dar xaque a. Cf. \textunderscore Eufrosina\textunderscore , 267.
\section{Xaqueca}
\begin{itemize}
\item {fónica:quê}
\end{itemize}
\begin{itemize}
\item {Grp. gram.:f.}
\end{itemize}
\begin{itemize}
\item {Utilização:Ant.}
\end{itemize}
O mesmo que \textunderscore enxaqueca\textunderscore .
\section{Xáquema}
\begin{itemize}
\item {Grp. gram.:f.}
\end{itemize}
\begin{itemize}
\item {Proveniência:(Do ár. \textunderscore xaquima\textunderscore )}
\end{itemize}
Tecido grosso, próprio para silhas.
\section{Xaque-mate}
\begin{itemize}
\item {Grp. gram.:m.}
\end{itemize}
O mesmo ou melhór que \textunderscore xeque-mate\textunderscore .
\section{Xaquetar}
\begin{itemize}
\item {Grp. gram.:v. t.}
\end{itemize}
Salpicar.
Entremear: \textunderscore vinhas xaquetadas de oliveiras\textunderscore .
\section{Xáquima}
\begin{itemize}
\item {Grp. gram.:f.}
\end{itemize}
\begin{itemize}
\item {Utilização:Ant.}
\end{itemize}
Cabeçada.
Corda, com que se prende uma bêsta.
(Cp. \textunderscore xáquema\textunderscore )
\section{Xara}
\begin{itemize}
\item {Grp. gram.:f.}
\end{itemize}
\begin{itemize}
\item {Proveniência:(Do ár. \textunderscore xara\textunderscore )}
\end{itemize}
Seta, feita de pau tostado.
O mesmo que \textunderscore estêva\textunderscore ^2.
\section{Xara}
\begin{itemize}
\item {Grp. gram.:f.}
\end{itemize}
Casta de uva beirôa.
\section{Xará}
\begin{itemize}
\item {Grp. gram.:m. ,  f.  e  adj.}
\end{itemize}
\begin{itemize}
\item {Utilização:Bras}
\end{itemize}
\begin{itemize}
\item {Grp. gram.:M.}
\end{itemize}
Pessôa, que tem o mesmo nome que outra.
Homónymo; tucaio.
Bailado campestre.
\section{Xarafim}
\begin{itemize}
\item {Grp. gram.:m.}
\end{itemize}
O mesmo que \textunderscore xerafim\textunderscore .
\section{Xarafo}
\begin{itemize}
\item {Grp. gram.:m.}
\end{itemize}
\begin{itemize}
\item {Utilização:Ant.}
\end{itemize}
Cambiador, na Índia Portuguesa.
\section{Xarapim}
\begin{itemize}
\item {Grp. gram.:m.  e  f.}
\end{itemize}
\begin{itemize}
\item {Utilização:Bras}
\end{itemize}
O mesmo que \textunderscore xará\textunderscore ?
\section{Xaraque}
\begin{itemize}
\item {Grp. gram.:m.}
\end{itemize}
\begin{itemize}
\item {Utilização:Ant.}
\end{itemize}
Praça grande.
(Do ár.)
\section{Xarau}
\begin{itemize}
\item {Grp. gram.:m.}
\end{itemize}
\begin{itemize}
\item {Utilização:Ant.}
\end{itemize}
Sura ou vinho de palmeira, cozido duas ou três vezes.
\section{Xarda}
\begin{itemize}
\item {Grp. gram.:f.}
\end{itemize}
Peixe de Portugal.
(Corr. de \textunderscore sarda\textunderscore ?)
\section{Xarel}
\begin{itemize}
\item {Grp. gram.:m.}
\end{itemize}
O mesmo que \textunderscore xairel\textunderscore .
\section{Xarém}
\begin{itemize}
\item {Grp. gram.:m.}
\end{itemize}
\begin{itemize}
\item {Utilização:Prov.}
\end{itemize}
\begin{itemize}
\item {Utilização:alg.}
\end{itemize}
O mesmo que \textunderscore xerém\textunderscore .
\section{Xareta}
\begin{itemize}
\item {fónica:xarê}
\end{itemize}
\begin{itemize}
\item {Grp. gram.:f.}
\end{itemize}
\begin{itemize}
\item {Proveniência:(Do ár. \textunderscore xarita\textunderscore )}
\end{itemize}
Rêde, com que se impede a abordagem de um navio.
Rêde de pescar.
\section{Xaréu}
\begin{itemize}
\item {Grp. gram.:m.}
\end{itemize}
Peixe grande, mas ordinário, do Brasil.
\section{Xaréu}
\begin{itemize}
\item {Grp. gram.:m.}
\end{itemize}
\begin{itemize}
\item {Utilização:Prov.}
\end{itemize}
\begin{itemize}
\item {Utilização:trasm.}
\end{itemize}
Frio intenso.
\section{Xaréu}
\begin{itemize}
\item {Grp. gram.:m.}
\end{itemize}
\begin{itemize}
\item {Utilização:Bras. do N}
\end{itemize}
Capa de coiro, com que os vaqueiros cobrem as ancas do cavallo.
(Cp. \textunderscore xairel\textunderscore )
\section{Xarifa}
\begin{itemize}
\item {Grp. gram.:f.}
\end{itemize}
\begin{itemize}
\item {Utilização:Gír.}
\end{itemize}
Partes pudendas da mulhér.
\section{Xarife}
\begin{itemize}
\item {Grp. gram.:m.}
\end{itemize}
\begin{itemize}
\item {Proveniência:(T. ár.)}
\end{itemize}
Título, usado por príncipes moiros, descendentes de Mafoma.
Título dos Muçulmanos, que visitaram três vezes o templo de Meca, podendo por isso usar turbante verde.
\section{Xarimbote}
\begin{itemize}
\item {Grp. gram.:m.}
\end{itemize}
\begin{itemize}
\item {Utilização:Prov.}
\end{itemize}
\begin{itemize}
\item {Utilização:alent.}
\end{itemize}
Jôgo popular em que um tição vai passando de mão em mão, acompanhado por um estribilho, perdendo a pessôa, em cujas mãos êlle se apagou.
(Cp. \textunderscore chirimbote\textunderscore , que é outra fórma da mesma palavra)
\section{Xaroco}
\begin{itemize}
\item {fónica:xarô}
\end{itemize}
\begin{itemize}
\item {Grp. gram.:m.}
\end{itemize}
\begin{itemize}
\item {Utilização:Prov.}
\end{itemize}
\begin{itemize}
\item {Utilização:alent.}
\end{itemize}
\begin{itemize}
\item {Proveniência:(Do it. \textunderscore scirocco\textunderscore )}
\end{itemize}
Vento quente do Suéste, sôbre o Mediterrâneo.
Vento frio, que no inverno sopra do Levante, e a que também se chama vento espanhol.
\section{Xaropada}
\begin{itemize}
\item {Grp. gram.:f.}
\end{itemize}
\begin{itemize}
\item {Utilização:Pop.}
\end{itemize}
Porção de xarope.
Xarope, que se póde tomar de uma vez.
Qualquer medicamento contra a tosse.
\section{Xaropar}
\begin{itemize}
\item {Grp. gram.:v. t.}
\end{itemize}
Tratar com xarope; dar tisanas a.
\section{Xarope}
\begin{itemize}
\item {Grp. gram.:m.}
\end{itemize}
\begin{itemize}
\item {Utilização:Fam.}
\end{itemize}
\begin{itemize}
\item {Proveniência:(Do fr. \textunderscore sirop\textunderscore )}
\end{itemize}
Medicamento líquido e viscoso, resultante da mistura de certos líquidos com a porção de açúcar necessária para os saturar.
Lambedor.
Tisana.
Remédio caseiro.
\section{Xaroposo}
\begin{itemize}
\item {Grp. gram.:adj.}
\end{itemize}
Que tem a consistência do xarope.
\section{Xarque}
\textunderscore m.\textunderscore  (e der.)
(V. \textunderscore charque\textunderscore , etc.)
\section{Xarrasca}
\begin{itemize}
\item {Grp. gram.:f.}
\end{itemize}
\begin{itemize}
\item {Utilização:Pesc.}
\end{itemize}
Apparelho especial de linha e anzol, para a pesca de peixes que têm beiços carnudos.
\section{Xarroco}
\begin{itemize}
\item {fónica:rô}
\end{itemize}
\begin{itemize}
\item {Grp. gram.:m.}
\end{itemize}
\begin{itemize}
\item {Utilização:Prov.}
\end{itemize}
\begin{itemize}
\item {Utilização:alent.}
\end{itemize}
Peixe, da classe dos thorácicos.
Peixe pércida do Mediterrâneo.
Dedeira, com que os ceifeiros resguardam dos golpes da foice o dedo pollegar da mão esquerda.
\section{Xarta}
\begin{itemize}
\item {Grp. gram.:f.}
\end{itemize}
\begin{itemize}
\item {Utilização:Ant.}
\end{itemize}
O mesmo que \textunderscore enxárcia\textunderscore ?:«\textunderscore ...com levarmos xarta tomada e brandaes...\textunderscore »\textunderscore Hist. Trág. Marit.\textunderscore 
(Cp. \textunderscore sarta\textunderscore )
\section{Xastre}
\begin{itemize}
\item {Grp. gram.:m.}
\end{itemize}
\begin{itemize}
\item {Utilização:Ant.}
\end{itemize}
O mesmo que \textunderscore alfaiate\textunderscore . Cf. Garrett, \textunderscore Romanceiro\textunderscore , II, 173 e 196.
(Cast. \textunderscore sastre\textunderscore )
\section{Xátria}
\begin{itemize}
\item {Grp. gram.:m.}
\end{itemize}
Membro da segunda das castas, em que se dividem os sectários do Brahmanismo, (a classe dos guerreiros, a que pertencem os Rajas).
\section{Xautér}
\begin{itemize}
\item {Grp. gram.:m.}
\end{itemize}
Muçulmano, que guia os viandantes nos desertos da Arábia. Cf. Godinho, \textunderscore Viagem da Índia\textunderscore , liv. I, c. LXIV, 116.
(Do ár.)
\section{Xaveco}
\begin{itemize}
\item {Grp. gram.:m.}
\end{itemize}
\begin{itemize}
\item {Utilização:Fam.}
\end{itemize}
\begin{itemize}
\item {Grp. gram.:m.}
\end{itemize}
\begin{itemize}
\item {Proveniência:(Do ár. \textunderscore xabeca\textunderscore )}
\end{itemize}
Pequena embarcação.
Barco pequeno e mal construído ou velho; embarcação ordinária.
O mesmo ou melhór que \textunderscore chaveco\textunderscore .
\section{Xávega}
\begin{itemize}
\item {Grp. gram.:f.}
\end{itemize}
\begin{itemize}
\item {Grp. gram.:f.}
\end{itemize}
Rede para a pesca de peixe miúdo.
Barco, em que os pescadores levam essa rede.
(Da mesma or. que \textunderscore chaveco\textunderscore )
O mesmo ou melhór que \textunderscore chávega\textunderscore .
(Cast. \textunderscore jábega\textunderscore )
\section{Xeco}
\begin{itemize}
\item {Grp. gram.:m.}
\end{itemize}
\begin{itemize}
\item {Utilização:Ant.}
\end{itemize}
(?):«\textunderscore ...chamo pelos vizinhos, e ella nega dar-me em xeco.\textunderscore »G. Vicente, I, 169.
\section{Xelim}
\begin{itemize}
\item {Grp. gram.:m.}
\end{itemize}
\begin{itemize}
\item {Proveniência:(Do ingl. \textunderscore shilling\textunderscore )}
\end{itemize}
Moéda inglesa de prata, correspondente a 225 reis.
\section{Xelma}
\begin{itemize}
\item {Grp. gram.:f.}
\end{itemize}
\begin{itemize}
\item {Proveniência:(Do ár. \textunderscore sollam\textunderscore )}
\end{itemize}
Espécie de sebe, com que se ladeia o tabuleiro de um carro, para amparar a carrada.
\section{Xelro}
\begin{itemize}
\item {Grp. gram.:m.}
\end{itemize}
\begin{itemize}
\item {Utilização:Gír.}
\end{itemize}
Prisão.
\section{Xenim}
\begin{itemize}
\item {Grp. gram.:m.}
\end{itemize}
Espécie de capitão, entre os antigos povos da Indo-China. Cf. \textunderscore Conq. do Pegu\textunderscore , VIII.
\section{Xém-xém}
\begin{itemize}
\item {Grp. gram.:m.}
\end{itemize}
\begin{itemize}
\item {Utilização:Bras}
\end{itemize}
Moéda falsa de cobre que há meio século tinha curso no Brasil.
Ave canora do Brasil.
\section{Xenagia}
\begin{itemize}
\item {Grp. gram.:f.}
\end{itemize}
Divisão militar grega, o mesmo que \textunderscore syntagma\textunderscore .
\section{Xendi}
\begin{itemize}
\item {Grp. gram.:m.}
\end{itemize}
\begin{itemize}
\item {Utilização:Ant.}
\end{itemize}
Trança solta de cabello.
(Conc. \textunderscore xendi\textunderscore )
\section{Xenelásia}
\begin{itemize}
\item {Grp. gram.:f.}
\end{itemize}
\begin{itemize}
\item {Proveniência:(Gr. \textunderscore xenelasia\textunderscore )}
\end{itemize}
Interdicção, que se faz aos estrangeiros, de entrar num país ou numa cidade.
\section{Xênia}
\begin{itemize}
\item {Grp. gram.:f.}
\end{itemize}
\begin{itemize}
\item {Proveniência:(Do gr. \textunderscore xenos\textunderscore )}
\end{itemize}
Presente que, entre os Gregos antigos, se dava aos hóspedes, depois da refeição.
Presente que, em certas épocas do anno, se mandava aos amigos.
\section{Xenodonte}
\begin{itemize}
\item {Grp. gram.:m.}
\end{itemize}
\begin{itemize}
\item {Proveniência:(Do gr. \textunderscore xenos\textunderscore  + \textunderscore odous\textunderscore , \textunderscore odontos\textunderscore )}
\end{itemize}
Grande serpente venenosa.
\section{Xenofobia}
\begin{itemize}
\item {Grp. gram.:f.}
\end{itemize}
O mesmo que \textunderscore xenofobismo\textunderscore .
\section{Xenofobismo}
\begin{itemize}
\item {Grp. gram.:m.}
\end{itemize}
Aversão ás pessôas e coisas estrangeiras.
(Cp. \textunderscore xenófobo\textunderscore )
\section{Xenófobo}
\begin{itemize}
\item {Grp. gram.:m.}
\end{itemize}
\begin{itemize}
\item {Proveniência:(Do gr. \textunderscore xenos\textunderscore  + \textunderscore phobein\textunderscore )}
\end{itemize}
Aquele que tem xenofobia.
\section{Xenomania}
\begin{itemize}
\item {Grp. gram.:f.}
\end{itemize}
\begin{itemize}
\item {Proveniência:(Do gr. \textunderscore xenos\textunderscore  + \textunderscore mania\textunderscore )}
\end{itemize}
Paixão por tudo que é estrangeiro.
\section{Xenophobia}
\begin{itemize}
\item {Grp. gram.:f.}
\end{itemize}
O mesmo que \textunderscore xenophobismo\textunderscore .
\section{Xenophobismo}
\begin{itemize}
\item {Grp. gram.:m.}
\end{itemize}
Aversão ás pessôas e coisas estrangeiras.
(Cp. \textunderscore xenóphobo\textunderscore )
\section{Xenóphobo}
\begin{itemize}
\item {Grp. gram.:m.}
\end{itemize}
\begin{itemize}
\item {Proveniência:(Do gr. \textunderscore xenos\textunderscore  + \textunderscore phobein\textunderscore )}
\end{itemize}
Aquelle que tem xenophobia.
\section{Xeque}
\begin{itemize}
\item {Grp. gram.:m.}
\end{itemize}
\begin{itemize}
\item {Proveniência:(Do ár. \textunderscore xeik\textunderscore )}
\end{itemize}
Chefe de tríbo africana.
\section{Xeque}
\begin{itemize}
\item {Grp. gram.:m.}
\end{itemize}
Incidente ao jôgo de xadrez, que consiste em atacar-se o rei ou fazer-se recuar a raínha, sob pena de se perder a peça.
Successo parlamentar, que envolve perigo para o Ministério.
Perigo; contratempo.
(Do pers., por intermédio do fr. \textunderscore échec\textunderscore )
\section{Xeque-mate}
\begin{itemize}
\item {Grp. gram.:m.}
\end{itemize}
\begin{itemize}
\item {Proveniência:(Fr. \textunderscore échec et mat\textunderscore , do pers. \textunderscore xah\textunderscore  + ár. \textunderscore mat\textunderscore )}
\end{itemize}
O mesmo que \textunderscore xàmáte\textunderscore .
\section{Xerafim}
\begin{itemize}
\item {Grp. gram.:m.}
\end{itemize}
\begin{itemize}
\item {Proveniência:(Do ár. \textunderscore xarifi\textunderscore )}
\end{itemize}
Moéda de prata, da Índia Portuguesa.
\section{Xerântemo}
\begin{itemize}
\item {Grp. gram.:m.}
\end{itemize}
\begin{itemize}
\item {Proveniência:(Do gr. \textunderscore xeros\textunderscore , sêco e \textunderscore anthema\textunderscore , florescência)}
\end{itemize}
Gênero de plantas, da fam. das compostas.
\section{Xerânthemo}
\begin{itemize}
\item {Grp. gram.:m.}
\end{itemize}
\begin{itemize}
\item {Proveniência:(Do gr. \textunderscore xeros\textunderscore , sêco e \textunderscore anthema\textunderscore , florescência)}
\end{itemize}
Gênero de plantas, da fam. das compostas.
\section{Xerasia}
\begin{itemize}
\item {Grp. gram.:f.}
\end{itemize}
\begin{itemize}
\item {Proveniência:(Gr. \textunderscore xerasia\textunderscore )}
\end{itemize}
Doença, que impede o crescimento dos cabellos e das sobrancelhas.
\section{Xerém}
\begin{itemize}
\item {Grp. gram.:m.}
\end{itemize}
\begin{itemize}
\item {Utilização:Prov.}
\end{itemize}
\begin{itemize}
\item {Utilização:alg.}
\end{itemize}
Farinha de milho para papas.
Papas de milho.
Nome de um baile de roda.
\section{Xerentes}
\begin{itemize}
\item {Grp. gram.:m. pl.}
\end{itemize}
Índios, que vivem ao norte de Goiás, no Brasil.
\section{Xerez}
\begin{itemize}
\item {Grp. gram.:m.}
\end{itemize}
\begin{itemize}
\item {Proveniência:(De \textunderscore Jerez\textunderscore , n. p.)}
\end{itemize}
Espécie de uva tinta.
Vinho muito estimado da Andaluzia.
\section{Xerga}
\begin{itemize}
\item {fónica:xêr}
\end{itemize}
\begin{itemize}
\item {Grp. gram.:m.}
\end{itemize}
\begin{itemize}
\item {Utilização:Prov.}
\end{itemize}
\begin{itemize}
\item {Utilização:trasm.}
\end{itemize}
\begin{itemize}
\item {Utilização:Bras}
\end{itemize}
\begin{itemize}
\item {Utilização:Prov.}
\end{itemize}
\begin{itemize}
\item {Utilização:trasm.}
\end{itemize}
\begin{itemize}
\item {Utilização:Prov.}
\end{itemize}
\begin{itemize}
\item {Utilização:trasm.}
\end{itemize}
Tecido grosseiro, espécie de burel.
Espécie de almofada, que se põe por baixo da albarda das bêstas.
O mesmo que \textunderscore enxêrga\textunderscore .
Manta grosseira, em que se transporta a palha trilhada das eiras, e que se estende debaixo das oliveiras quando se vareja a azeitona.
(Cast. \textunderscore jerga\textunderscore )
\section{Xergão}
\begin{itemize}
\item {Grp. gram.:m.}
\end{itemize}
\begin{itemize}
\item {Utilização:Prov.}
\end{itemize}
\begin{itemize}
\item {Utilização:bras}
\end{itemize}
O mesmo que \textunderscore enxergão\textunderscore .
(Cast. \textunderscore jergon\textunderscore )
\section{Xeri}
\begin{itemize}
\item {Grp. gram.:m.}
\end{itemize}
\begin{itemize}
\item {Utilização:Ant.}
\end{itemize}
Moéda de prata, em Dio.
\section{Xerife}
\begin{itemize}
\item {Grp. gram.:m.}
\end{itemize}
O mesmo que \textunderscore xarifa\textunderscore .
\section{Xerimbabo}
\begin{itemize}
\item {Grp. gram.:m.}
\end{itemize}
\begin{itemize}
\item {Utilização:Bras}
\end{itemize}
\begin{itemize}
\item {Proveniência:(T. tupi)}
\end{itemize}
Qualquer animal doméstico.
Insecto.
\section{Xeringosa}
\begin{itemize}
\item {Grp. gram.:f.}
\end{itemize}
Nome, que se dá á cólica na Índia Portuguesa.
\section{Xeringueira}
\begin{itemize}
\item {Grp. gram.:f.}
\end{itemize}
O mesmo que \textunderscore seringueira\textunderscore .
\section{Xerodermia}
\begin{itemize}
\item {Grp. gram.:f.}
\end{itemize}
\begin{itemize}
\item {Utilização:Med.}
\end{itemize}
\begin{itemize}
\item {Proveniência:(Do gr. \textunderscore xeros\textunderscore  + \textunderscore derma\textunderscore )}
\end{itemize}
O mesmo que \textunderscore ichthyose\textunderscore .
\section{Xerofagia}
\begin{itemize}
\item {Grp. gram.:f.}
\end{itemize}
\begin{itemize}
\item {Proveniência:(Lat. \textunderscore xerophagia\textunderscore )}
\end{itemize}
Dieta, em que se não póde beber.
Abstinência quaresmal dos primitivos Christãos.
\section{Xerófago}
\begin{itemize}
\item {Grp. gram.:m.}
\end{itemize}
\begin{itemize}
\item {Proveniência:(Do gr. \textunderscore xeros\textunderscore  + \textunderscore phagein\textunderscore )}
\end{itemize}
Aquele que observa a xerofagia.
Aquele que, entre os primitivos Cristãos, só se alimentava de pão e fruta sêca.
\section{Xerófilo}
\begin{itemize}
\item {Grp. gram.:adj.}
\end{itemize}
\begin{itemize}
\item {Utilização:Bot.}
\end{itemize}
\begin{itemize}
\item {Proveniência:(Do gr. \textunderscore xeros\textunderscore  + \textunderscore philos\textunderscore )}
\end{itemize}
Diz-se das plantas, que se dão bem nos terrenos secos.
\section{Xerofitismo}
\begin{itemize}
\item {Grp. gram.:m.}
\end{itemize}
\begin{itemize}
\item {Utilização:Bot.}
\end{itemize}
Estado ou qualidade da vegetação xerófita.
\section{Xerofórmio}
\begin{itemize}
\item {Grp. gram.:m.}
\end{itemize}
Medicamento, que tem as mesmas applicações que o iodofórmio.
\section{Xeroftalmia}
\begin{itemize}
\item {Grp. gram.:f.}
\end{itemize}
\begin{itemize}
\item {Proveniência:(Gr. \textunderscore xerophthalmia\textunderscore )}
\end{itemize}
Ofthalmia sêca, isto é, sem secreção lacrimal.
\section{Xerografia}
\begin{itemize}
\item {Grp. gram.:f.}
\end{itemize}
\begin{itemize}
\item {Proveniência:(Do gr. \textunderscore xeros\textunderscore  + \textunderscore graphein\textunderscore )}
\end{itemize}
Parte da Geografia, que se occupa da parte sêca do globo.
\section{Xerográfico}
\begin{itemize}
\item {Grp. gram.:adj.}
\end{itemize}
Relativo á xerografia.
\section{Xerographia}
\begin{itemize}
\item {Grp. gram.:f.}
\end{itemize}
\begin{itemize}
\item {Proveniência:(Do gr. \textunderscore xeros\textunderscore  + \textunderscore graphein\textunderscore )}
\end{itemize}
Parte da Geographia, que se occupa da parte sêca do globo.
\section{Xerográphico}
\begin{itemize}
\item {Grp. gram.:adj.}
\end{itemize}
Relativo á xerographia.
\section{Xerophagia}
\begin{itemize}
\item {Grp. gram.:f.}
\end{itemize}
\begin{itemize}
\item {Proveniência:(Lat. \textunderscore xerophagia\textunderscore )}
\end{itemize}
Dieta, em que se não póde beber.
Abstinência quaresmal dos primitivos Christãos.
\section{Xeróphago}
\begin{itemize}
\item {Grp. gram.:m.}
\end{itemize}
\begin{itemize}
\item {Proveniência:(Do gr. \textunderscore xeros\textunderscore  + \textunderscore phagein\textunderscore )}
\end{itemize}
Aquelle que observa a xerophagia.
Aquelle que, entre os primitivos Christãos, só se alimentava de pão e fruta sêca.
\section{Xeróphilo}
\begin{itemize}
\item {Grp. gram.:adj.}
\end{itemize}
\begin{itemize}
\item {Utilização:Bot.}
\end{itemize}
\begin{itemize}
\item {Proveniência:(Do gr. \textunderscore xeros\textunderscore  + \textunderscore philos\textunderscore )}
\end{itemize}
Diz-se das plantas, que se dão bem nos terrenos secos.
\section{Xeróphito}
\begin{itemize}
\item {Grp. gram.:adj.}
\end{itemize}
\begin{itemize}
\item {Utilização:Bot.}
\end{itemize}
\begin{itemize}
\item {Proveniência:(Do gr. \textunderscore xeros\textunderscore  + \textunderscore phuton\textunderscore )}
\end{itemize}
Diz-se da vegetação, própria de lugares secos.
\section{Xerophthalmia}
\begin{itemize}
\item {Grp. gram.:f.}
\end{itemize}
\begin{itemize}
\item {Proveniência:(Gr. \textunderscore xerophthalmia\textunderscore )}
\end{itemize}
Ophthalmia sêca, isto é, sem secreção lacrimal.
\section{Xerophytismo}
\begin{itemize}
\item {Grp. gram.:m.}
\end{itemize}
\begin{itemize}
\item {Utilização:Bot.}
\end{itemize}
Estado ou qualidade da vegetação xeróphyta.
\section{Xerose}
\begin{itemize}
\item {Grp. gram.:f.}
\end{itemize}
O mesmo que \textunderscore xerophthalmia\textunderscore .
\section{Xerque}
\begin{itemize}
\item {Grp. gram.:adj.}
\end{itemize}
\begin{itemize}
\item {Utilização:Ant.}
\end{itemize}
Feito á maneira oriental.
\section{Xerume}
\begin{itemize}
\item {Grp. gram.:m.}
\end{itemize}
Erva, de que se servem os indígenas de Manica, para fumar.
\section{Xerva}
\begin{itemize}
\item {Grp. gram.:f.}
\end{itemize}
Variedade de linho.
\section{Xeta}
\begin{itemize}
\item {fónica:xê}
\end{itemize}
\begin{itemize}
\item {Grp. gram.:f.}
\end{itemize}
\begin{itemize}
\item {Utilização:Bras}
\end{itemize}
Beijo, atirado de longe com os dedos, por galanteio.
\section{Xetá!}
\begin{itemize}
\item {Grp. gram.:interj.}
\end{itemize}
\begin{itemize}
\item {Utilização:Prov.}
\end{itemize}
\begin{itemize}
\item {Utilização:trasm.}
\end{itemize}
O mesmo que \textunderscore chó\textunderscore !
\section{Xéu}
\begin{itemize}
\item {Grp. gram.:m.}
\end{itemize}
\begin{itemize}
\item {Utilização:Prov.}
\end{itemize}
\begin{itemize}
\item {Utilização:trasm.}
\end{itemize}
Pessôa ordinária, desprezível.
\section{Xeura}
\begin{itemize}
\item {Grp. gram.:f.}
\end{itemize}
\begin{itemize}
\item {Utilização:Náut.}
\end{itemize}
Escantilhão.
Inclinação da face dos madeiros de um navio, para nella se assentar o costado.
\section{Xèxé}
\begin{itemize}
\item {Grp. gram.:m.}
\end{itemize}
\begin{itemize}
\item {Utilização:Ant.}
\end{itemize}
\begin{itemize}
\item {Utilização:Pop.}
\end{itemize}
Mascarado, que, no Carnaval, representa um velho ridículo, vestido de casaca de sêda, calção e meia, e armado de uma grande faca de pau.
Pateta, idiota.
\section{Xexéu}
\begin{itemize}
\item {Grp. gram.:m.}
\end{itemize}
Ave brasileira.
\section{Xiba}
\begin{itemize}
\item {Grp. gram.:m.}
\end{itemize}
\begin{itemize}
\item {Utilização:Bras. do Rio}
\end{itemize}
Espécie de batuque.
\section{Xibé}
\begin{itemize}
\item {Grp. gram.:m.}
\end{itemize}
\begin{itemize}
\item {Utilização:Bras}
\end{itemize}
Bebida, o mesmo que \textunderscore jacuba\textunderscore .
Forma preferível a \textunderscore chibé\textunderscore .
(V. \textunderscore chibé\textunderscore ^1)
\section{Xicá}
\begin{itemize}
\item {Grp. gram.:m.}
\end{itemize}
\begin{itemize}
\item {Utilização:Bras}
\end{itemize}
Árvore silvestre.
\section{Xicaca}
\begin{itemize}
\item {Grp. gram.:f.}
\end{itemize}
\begin{itemize}
\item {Utilização:Bras}
\end{itemize}
Pequeno cesto com tampa.
\section{Xicaca}
\begin{itemize}
\item {Grp. gram.:f.}
\end{itemize}
\begin{itemize}
\item {Utilização:T. de Angola}
\end{itemize}
Imposto aduaneiro; alcavala.
\section{Xifarote}
\begin{itemize}
\item {Grp. gram.:m.}
\end{itemize}
O mesmo ou melhór que \textunderscore chifarote\textunderscore .
\section{Xífia}
\begin{itemize}
\item {Grp. gram.:f.}
\end{itemize}
\begin{itemize}
\item {Proveniência:(Gr. \textunderscore xiphias\textunderscore )}
\end{itemize}
Grande peixe-espada do Atlântico e do Mediterrâneo.
\section{Xífias}
\begin{itemize}
\item {Grp. gram.:m.}
\end{itemize}
\begin{itemize}
\item {Proveniência:(Gr. \textunderscore xiphias\textunderscore )}
\end{itemize}
Grande peixe-espada do Atlântico e do Mediterrâneo.
\section{Xífio}
\begin{itemize}
\item {Grp. gram.:m.}
\end{itemize}
Planta medicinal emoliente.
(Cp. gr. \textunderscore xíphos\textunderscore )
\section{Xifisternal}
\begin{itemize}
\item {Grp. gram.:adj.}
\end{itemize}
Relativo ao xifisterno.
\section{Xifisterno}
\begin{itemize}
\item {Grp. gram.:m.}
\end{itemize}
\begin{itemize}
\item {Proveniência:(Do gr. \textunderscore xiphos\textunderscore , espada e port. \textunderscore esterno\textunderscore )}
\end{itemize}
O mesmo que \textunderscore metasterno\textunderscore .
\section{Xifódimo}
\begin{itemize}
\item {Grp. gram.:adj.}
\end{itemize}
\begin{itemize}
\item {Utilização:Terat.}
\end{itemize}
Diz-se do monstro, composto de dois corpos distintos na parte superior, mas cujo tronco se confunde na parte inferior.
\section{Xifóide}
\begin{itemize}
\item {Grp. gram.:adj.}
\end{itemize}
O mesmo que \textunderscore xifoídeo\textunderscore .
\section{Xifoídeo}
\begin{itemize}
\item {Grp. gram.:adj.}
\end{itemize}
\begin{itemize}
\item {Utilização:Anat.}
\end{itemize}
\begin{itemize}
\item {Proveniência:(Do gr. \textunderscore xiphoides\textunderscore )}
\end{itemize}
Diz-se do apêndice alongado e cartilaginoso, que termina inferiormente o esterno.
\section{Xifoidiano}
\begin{itemize}
\item {Grp. gram.:adj.}
\end{itemize}
\begin{itemize}
\item {Utilização:Anat.}
\end{itemize}
Relativo ao apêndice xifóide.
\section{Xifu}
\begin{itemize}
\item {Grp. gram.:m.}
\end{itemize}
Espécie de meirinho ou esbirro chinês.
\section{Xiitas}
\begin{itemize}
\item {Grp. gram.:m. pl.}
\end{itemize}
Seita religiosa, que adaptou ao Islamismo as antigas crenças da Pérsia.
\section{Xilindró}
\begin{itemize}
\item {Grp. gram.:m.}
\end{itemize}
\begin{itemize}
\item {Utilização:Chul.}
\end{itemize}
Outra fórma de \textunderscore chilindró\textunderscore .
\section{Xilréu}
\begin{itemize}
\item {Grp. gram.:m.}
\end{itemize}
\begin{itemize}
\item {Utilização:Pesc.}
\end{itemize}
Pequena rêde triangular, usada no Algarve, para a pesca de lulas e chocos.
\section{Ximarrão}
\begin{itemize}
\item {Grp. gram.:m.}
\end{itemize}
\begin{itemize}
\item {Utilização:Bras}
\end{itemize}
Diz-se do mate sem açúcar.
\section{Ximbaúva}
\begin{itemize}
\item {Grp. gram.:f.}
\end{itemize}
Espécie de acácia.
\section{Ximbeque}
\begin{itemize}
\item {Grp. gram.:m.}
\end{itemize}
Habitação entre os Mussorongos.
\section{Ximbuás}
\begin{itemize}
\item {Grp. gram.:m. pl.}
\end{itemize}
Nação de Índios do Brasil, em Mato-Grosso.
\section{Ximburu}
\begin{itemize}
\item {Grp. gram.:m.}
\end{itemize}
\begin{itemize}
\item {Utilização:Bras}
\end{itemize}
Peixe fluvial.
\section{Ximenézia}
\begin{itemize}
\item {Grp. gram.:f.}
\end{itemize}
\begin{itemize}
\item {Proveniência:(De \textunderscore Ximenez\textunderscore , n. p.)}
\end{itemize}
Gênero de plantas mexicanas.
\section{Xinga}
\begin{itemize}
\item {Grp. gram.:f.}
\end{itemize}
Trombeta de guerra, na Índia Portuguesa.
\section{Xingamento}
\begin{itemize}
\item {Grp. gram.:m.}
\end{itemize}
Acto de xingar.
\section{Xingar}
\begin{itemize}
\item {Grp. gram.:v. t.}
\end{itemize}
\begin{itemize}
\item {Utilização:Bras}
\end{itemize}
\begin{itemize}
\item {Grp. gram.:V. i.}
\end{itemize}
Insultar com palavras.
Zombar.
(Do quimb.)
\section{Xinxarel}
\begin{itemize}
\item {Grp. gram.:m.}
\end{itemize}
Peça de madeira, que se prega no tecto, entre dois barrotes, para fixar a suspensão de um candeeiro.
Peça de madeira, que se colloca diagonalmente sôbre o vigamento, para assentamento do parquete.
\section{Xinxilha}
\begin{itemize}
\item {Grp. gram.:m.}
\end{itemize}
\begin{itemize}
\item {Utilização:Ant.}
\end{itemize}
\begin{itemize}
\item {Utilização:Pop.}
\end{itemize}
Homem desprezível, um joão-ninguém.
\section{Xíphia}
\begin{itemize}
\item {Grp. gram.:f.}
\end{itemize}
\begin{itemize}
\item {Proveniência:(Gr. \textunderscore xiphias\textunderscore )}
\end{itemize}
Grande peixe-espada do Atlântico e do Mediterrâneo.
\section{Xíphias}
\begin{itemize}
\item {Grp. gram.:m.}
\end{itemize}
\begin{itemize}
\item {Proveniência:(Gr. \textunderscore xiphias\textunderscore )}
\end{itemize}
Grande peixe-espada do Atlântico e do Mediterrâneo.
\section{Xíphio}
\begin{itemize}
\item {Grp. gram.:m.}
\end{itemize}
Planta medicinal emolliente.
(Cp. gr. \textunderscore xíphos\textunderscore )
\section{Xiphisternal}
\begin{itemize}
\item {Grp. gram.:adj.}
\end{itemize}
Relativo ao xiphisterno.
\section{Xiphisterno}
\begin{itemize}
\item {Grp. gram.:m.}
\end{itemize}
\begin{itemize}
\item {Proveniência:(Do gr. \textunderscore xiphos\textunderscore , espada e port. \textunderscore esterno\textunderscore )}
\end{itemize}
O mesmo que \textunderscore metasterno\textunderscore .
\section{Xiphódymo}
\begin{itemize}
\item {Grp. gram.:adj.}
\end{itemize}
\begin{itemize}
\item {Utilização:Terat.}
\end{itemize}
Diz-se do monstro, composto de dois corpos distintos na parte superior, mas cujo tronco se confunde na parte inferior.
\section{Xiphóide}
\begin{itemize}
\item {Grp. gram.:adj.}
\end{itemize}
O mesmo que \textunderscore xiphoídeo\textunderscore .
\section{Xiphoídeo}
\begin{itemize}
\item {Grp. gram.:adj.}
\end{itemize}
\begin{itemize}
\item {Utilização:Anat.}
\end{itemize}
\begin{itemize}
\item {Proveniência:(Do gr. \textunderscore xiphoides\textunderscore )}
\end{itemize}
Diz-se do appêndice alongado e cartilaginoso, que termina inferiormente o esterno.
\section{Xiphoidiano}
\begin{itemize}
\item {Grp. gram.:adj.}
\end{itemize}
\begin{itemize}
\item {Utilização:Anat.}
\end{itemize}
Relativo ao appêndice xiphóide.
\section{Xifofilo}
\begin{itemize}
\item {Grp. gram.:adj.}
\end{itemize}
\begin{itemize}
\item {Utilização:Bot.}
\end{itemize}
\begin{itemize}
\item {Proveniência:(Do gr. \textunderscore xiphos\textunderscore  + \textunderscore phullon\textunderscore )}
\end{itemize}
Que tem fôlhas ensiformes.
\section{Xifopagia}
\begin{itemize}
\item {Grp. gram.:f.}
\end{itemize}
Qualidade de xifópago.
\section{Xifópago}
\begin{itemize}
\item {Grp. gram.:adj.}
\end{itemize}
\begin{itemize}
\item {Utilização:Terat.}
\end{itemize}
\begin{itemize}
\item {Proveniência:(Do gr. \textunderscore xiphos\textunderscore  + \textunderscore pagein\textunderscore )}
\end{itemize}
Diz-se do monstro, resultante da união de dois indivíduos, desde a parte inferior do esterno até ao umbigo.--Estavam neste caso os celebrados irmãos siameses.
\section{Xilarmónica}
\begin{itemize}
\item {Grp. gram.:f.}
\end{itemize}
O mesmo que \textunderscore xilarmónico\textunderscore .
\section{Xilarmónico}
Instrumento músico, semelhante ao eufónio.
Espécie de marimbas, com lâminas de madeira.
\section{Xilena}
\begin{itemize}
\item {Grp. gram.:f.}
\end{itemize}
\begin{itemize}
\item {Utilização:Chím.}
\end{itemize}
\begin{itemize}
\item {Proveniência:(Do gr. \textunderscore xulon\textunderscore )}
\end{itemize}
Gênero de insectos lepidópteros nocturnos.
Variedade de carboneto de hidrogênio líquido.
\section{Xileno}
\begin{itemize}
\item {Grp. gram.:m.}
\end{itemize}
\begin{itemize}
\item {Utilização:Chím.}
\end{itemize}
\begin{itemize}
\item {Proveniência:(Do gr. \textunderscore xulon\textunderscore )}
\end{itemize}
Um dos carbonetos do grupo benzênico.
\section{Xilharmónica}
\begin{itemize}
\item {fónica:lar}
\end{itemize}
\begin{itemize}
\item {Grp. gram.:f.}
\end{itemize}
O mesmo que \textunderscore xilharmónico\textunderscore .
\section{Xilharmónico}
\begin{itemize}
\item {fónica:lar}
\end{itemize}
Instrumento músico, semelhante ao euphónio.
Espécie de marimbas, com lâminas de madeira.
\section{Xilite}
\begin{itemize}
\item {Grp. gram.:f.}
\end{itemize}
\begin{itemize}
\item {Proveniência:(Do gr. \textunderscore xulon\textunderscore )}
\end{itemize}
Producto da destilação do espírito de madeira.
\section{Xilo...}
\begin{itemize}
\item {Grp. gram.:pref.}
\end{itemize}
\begin{itemize}
\item {Proveniência:(Do gr. \textunderscore xulon\textunderscore )}
\end{itemize}
(designativo de \textunderscore madeira\textunderscore )
\section{Xilo}
\begin{itemize}
\item {Grp. gram.:m.}
\end{itemize}
\begin{itemize}
\item {Proveniência:(Do gr. \textunderscore xulon\textunderscore )}
\end{itemize}
A planta do algodão; algodoeiro.
\section{Xilobálsamo}
\begin{itemize}
\item {Grp. gram.:m.}
\end{itemize}
\begin{itemize}
\item {Proveniência:(Lat. \textunderscore xylobalsamum\textunderscore )}
\end{itemize}
O mesmo que \textunderscore balsameiro\textunderscore .
\section{Xilocarpo}
\begin{itemize}
\item {Grp. gram.:adj.}
\end{itemize}
\begin{itemize}
\item {Utilização:Bot.}
\end{itemize}
\begin{itemize}
\item {Proveniência:(Do gr. \textunderscore xulon\textunderscore  + \textunderscore karpos\textunderscore )}
\end{itemize}
Diz-se das árvores de frutos duros ou lenhosos.
\section{Xilócopo}
\begin{itemize}
\item {Grp. gram.:adj.}
\end{itemize}
\begin{itemize}
\item {Utilização:Zool.}
\end{itemize}
\begin{itemize}
\item {Proveniência:(Gr. \textunderscore xulokopos\textunderscore )}
\end{itemize}
Que córta a madeira, que pica paus, que fura os troncos: \textunderscore há abelhas xilócopas\textunderscore .
\section{Xilódia}
\begin{itemize}
\item {Grp. gram.:f.}
\end{itemize}
\begin{itemize}
\item {Proveniência:(Do gr. \textunderscore xulodes\textunderscore )}
\end{itemize}
Gênero de frutos lenhosos, como a avelan.
\section{Xilogênio}
\begin{itemize}
\item {Grp. gram.:m.}
\end{itemize}
\begin{itemize}
\item {Utilização:Chím.}
\end{itemize}
\begin{itemize}
\item {Proveniência:(Do gr. \textunderscore xulon\textunderscore  + \textunderscore genos\textunderscore )}
\end{itemize}
Substância vegetal, que se encontra especialmente na parede primitiva das céllulas das plantas.
\section{Xiloglifia}
\begin{itemize}
\item {Grp. gram.:f.}
\end{itemize}
Arte de esculpir em madeira; arte de gravar caracteres em madeira.
(Cp. \textunderscore xilóglifo\textunderscore )
\section{Xilóglífico}
\begin{itemize}
\item {Grp. gram.:adj.}
\end{itemize}
Relativo á xiloglifia.
\section{Xilóglifo}
\begin{itemize}
\item {Grp. gram.:m.}
\end{itemize}
\begin{itemize}
\item {Proveniência:(Do gr. \textunderscore xulon\textunderscore  + \textunderscore gluphos\textunderscore )}
\end{itemize}
Aquele que exerce a xiloglifia.
Aquele que faz letras com ornatos, para livros, ou grandes letras, para cartazes.
\section{Xilofagia}
\begin{itemize}
\item {Grp. gram.:f.}
\end{itemize}
Acto de roer a madeira.
Propriedade de xilófago.
\section{Xilografia}
\begin{itemize}
\item {Grp. gram.:f.}
\end{itemize}
\begin{itemize}
\item {Proveniência:(De \textunderscore xilógrafo\textunderscore )}
\end{itemize}
Arte de gravar em madeira.
\section{Xilográfico}
\begin{itemize}
\item {Grp. gram.:adj.}
\end{itemize}
Relativo á xilografia.
\section{Xilógrafo}
\begin{itemize}
\item {Grp. gram.:m.}
\end{itemize}
\begin{itemize}
\item {Proveniência:(Do gr. \textunderscore xulon\textunderscore  + \textunderscore graphein\textunderscore )}
\end{itemize}
Aquele que grava em madeira.
\section{Xilóide}
\begin{itemize}
\item {Grp. gram.:adj.}
\end{itemize}
\begin{itemize}
\item {Proveniência:(Gr. \textunderscore xuloides\textunderscore )}
\end{itemize}
Relativo a madeira.
Proveniente de um corpo lenhoso.
\section{Xiloídico}
\begin{itemize}
\item {Grp. gram.:adj.}
\end{itemize}
Diz-se de um suposto tipo de arquitectura em madeira, donde teria procedido a arquitectura gótica.
(Cp. \textunderscore xilóide\textunderscore )
\section{Xiloidina}
\begin{itemize}
\item {Grp. gram.:f.}
\end{itemize}
\begin{itemize}
\item {Utilização:Chím.}
\end{itemize}
\begin{itemize}
\item {Proveniência:(De \textunderscore xilóide\textunderscore )}
\end{itemize}
Matéria muito combustível, que se obtém pela acção do ácido nítrico sôbre os princípios neutros vegetaes.
\section{Xilólatra}
\begin{itemize}
\item {Grp. gram.:m.}
\end{itemize}
\begin{itemize}
\item {Proveniência:(Do gr. \textunderscore xulolatres\textunderscore )}
\end{itemize}
Aquele que adora os ídolos de madeira.
\section{Xilolatria}
\begin{itemize}
\item {Grp. gram.:f.}
\end{itemize}
Qualidade de xilólatra.
Adoração dos ídolos de madeira.
\section{Xilólito}
\begin{itemize}
\item {Grp. gram.:f.}
\end{itemize}
\begin{itemize}
\item {Proveniência:(Do gr. \textunderscore xulon\textunderscore  + \textunderscore lithos\textunderscore )}
\end{itemize}
Madeira fóssil.
\section{Xilologia}
\begin{itemize}
\item {Grp. gram.:f.}
\end{itemize}
\begin{itemize}
\item {Proveniência:(Do gr. \textunderscore xulon\textunderscore  + \textunderscore logos\textunderscore )}
\end{itemize}
Tratado ou história das madeiras.
\section{Xilológico}
\begin{itemize}
\item {Grp. gram.:adj.}
\end{itemize}
Relativo á xilologia.
\section{Xilólogo}
\begin{itemize}
\item {Grp. gram.:m.}
\end{itemize}
Aquele que é perito em xilologia.
\section{Xiloma}
\begin{itemize}
\item {Grp. gram.:m.}
\end{itemize}
\begin{itemize}
\item {Proveniência:(Do gr. \textunderscore xulon\textunderscore  + \textunderscore omos\textunderscore )}
\end{itemize}
Gênero de cogumelos lenhosos.
\section{Xilomancia}
\begin{itemize}
\item {Grp. gram.:f.}
\end{itemize}
\begin{itemize}
\item {Proveniência:(Do gr. \textunderscore xulon\textunderscore  + \textunderscore manteia\textunderscore )}
\end{itemize}
Suposta arte de adivinhar, por meio da disposição dos pauzinhos secos, que se encontram pelo caminho.
\section{Xilómice}
\begin{itemize}
\item {Grp. gram.:adj.}
\end{itemize}
\begin{itemize}
\item {Utilização:Bot.}
\end{itemize}
\begin{itemize}
\item {Proveniência:(Do gr. \textunderscore xulon\textunderscore  + \textunderscore mukes\textunderscore )}
\end{itemize}
Diz-se dos cogumelos, que crescem sôbre a madeira ou nas árvores.
\section{Xílon}
\begin{itemize}
\item {Grp. gram.:m.}
\end{itemize}
\begin{itemize}
\item {Utilização:Chím.}
\end{itemize}
\begin{itemize}
\item {Proveniência:(Gr. \textunderscore xulon\textunderscore )}
\end{itemize}
Celulose da madeira ou dos invólucros dos frutos duros.
\section{Xiphopagia}
\begin{itemize}
\item {Grp. gram.:f.}
\end{itemize}
Qualidade de xiphópago.
\section{Xiphtópago}
\begin{itemize}
\item {Grp. gram.:adj.}
\end{itemize}
\begin{itemize}
\item {Utilização:Terat.}
\end{itemize}
\begin{itemize}
\item {Proveniência:(Do gr. \textunderscore xiphos\textunderscore  + \textunderscore pagein\textunderscore )}
\end{itemize}
Diz-se do monstro, resultante da união de dois indivíduos, desde a parte inferior do esterno até ao umbigo.--Estavam neste caso os celebrados irmãos siameses.
\section{Xiphophyllo}
\begin{itemize}
\item {Grp. gram.:adj.}
\end{itemize}
\begin{itemize}
\item {Utilização:Bot.}
\end{itemize}
\begin{itemize}
\item {Proveniência:(Do gr. \textunderscore xiphos\textunderscore  + \textunderscore phullon\textunderscore )}
\end{itemize}
Que tem fôlhas ensiformes.
\section{Xipo}
\begin{itemize}
\item {Grp. gram.:m.}
\end{itemize}
Cinto de coiro, de que os Lundeses suspendem os panos, com que cobrem as partes genitaes. Cf. H. Carvalho, \textunderscore Exped. ao Muat.\textunderscore 
\section{Xique-xique}
\begin{itemize}
\item {Grp. gram.:m.}
\end{itemize}
Planta leguminosa do Brasil.
\section{Xiquir}
\begin{itemize}
\item {Grp. gram.:adv.}
\end{itemize}
\begin{itemize}
\item {Utilização:Ant.}
\end{itemize}
\begin{itemize}
\item {Utilização:Pop.}
\end{itemize}
O mesmo que \textunderscore sequer\textunderscore .
\section{Xira}
\begin{itemize}
\item {Grp. gram.:f.}
\end{itemize}
\begin{itemize}
\item {Utilização:Ant.}
\end{itemize}
O mesmo que \textunderscore xara\textunderscore ^1, seta.
\section{Xira}
\begin{itemize}
\item {Grp. gram.:f.}
\end{itemize}
\begin{itemize}
\item {Utilização:Ant.}
\end{itemize}
O mesmo que \textunderscore cira\textunderscore .
\section{Xira}
\begin{itemize}
\item {Grp. gram.:f.}
\end{itemize}
\begin{itemize}
\item {Utilização:Ant.}
\end{itemize}
Alimentação; pasto.
(Cp. \textunderscore bona-chira\textunderscore )
\section{Xiraz}
\begin{itemize}
\item {Grp. gram.:m.}
\end{itemize}
\begin{itemize}
\item {Utilização:Ant.}
\end{itemize}
Leite coalhado.
(Do ár.)
\section{Xiricaá}
\begin{itemize}
\item {Grp. gram.:m.}
\end{itemize}
\begin{itemize}
\item {Utilização:Bras}
\end{itemize}
Planta medicinal, das regiões do Amazonas.
\section{Xiriquanos}
\begin{itemize}
\item {Grp. gram.:m. pl.}
\end{itemize}
\begin{itemize}
\item {Utilização:Bras}
\end{itemize}
Tríbo de aborigenes de Mato-Grosso.
\section{Xiriubeira}
\begin{itemize}
\item {Grp. gram.:f.}
\end{itemize}
Planta convolvulácea do Brasil.
\section{Xiró}
\begin{itemize}
\item {Grp. gram.:m.}
\end{itemize}
\begin{itemize}
\item {Utilização:Bras}
\end{itemize}
Caldo de arroz.
\section{Xis}
\begin{itemize}
\item {Grp. gram.:m.}
\end{itemize}
\begin{itemize}
\item {Utilização:Pop.}
\end{itemize}
Moéda de déz reis: \textunderscore não tenho uma de xis\textunderscore .
(Da letra \textunderscore X\textunderscore )
\section{Xisgaraviz}
\begin{itemize}
\item {Grp. gram.:m.}
\end{itemize}
\begin{itemize}
\item {Utilização:Ant.}
\end{itemize}
Aquelle ou aquillo que se intromete, que remexe:«\textunderscore a dúvida de um amante vem a ser o ciúme..., o xisgaraviz das memórias, o contágio das suspeitas...\textunderscore »\textunderscore Anat. Joc.\textunderscore , I, 203.
\section{Xisto}
\begin{itemize}
\item {Grp. gram.:m.}
\end{itemize}
\begin{itemize}
\item {Proveniência:(Do gr. \textunderscore skhistos\textunderscore )}
\end{itemize}
Mineral friável, formado especialmente de sílica e argilla.
\section{Xistocarpo}
\begin{itemize}
\item {Grp. gram.:adj.}
\end{itemize}
\begin{itemize}
\item {Proveniência:(Do gr. \textunderscore skhistos\textunderscore  + \textunderscore karpos\textunderscore )}
\end{itemize}
Diz-se dos frutos, que se abrem, fendendo-se.
\section{Xistóide}
\begin{itemize}
\item {Grp. gram.:adj.}
\end{itemize}
\begin{itemize}
\item {Utilização:Zool.}
\end{itemize}
\begin{itemize}
\item {Proveniência:(Do gr. \textunderscore skhistos\textunderscore  + \textunderscore eidos\textunderscore )}
\end{itemize}
Diz-se das rochas, em que apenas há vestígios ou apparências de textura xistosa.
\section{Xistosidade}
\begin{itemize}
\item {Grp. gram.:f.}
\end{itemize}
\begin{itemize}
\item {Proveniência:(De \textunderscore xistoso\textunderscore )}
\end{itemize}
Caracter especial das rochas xistosas.
\section{Xistoso}
\begin{itemize}
\item {Grp. gram.:adj.}
\end{itemize}
Que é da natureza do xisto.
\section{Xixan}
\begin{itemize}
\item {Grp. gram.:M.}
\end{itemize}
\begin{itemize}
\item {Utilização:Bras. do N}
\end{itemize}
Árvore fructífera.
\section{Xixi}
\begin{itemize}
\item {Grp. gram.:m.}
\end{itemize}
Planta leguminosa do Brasil.
\section{Xixica}
\begin{itemize}
\item {Grp. gram.:f.}
\end{itemize}
\begin{itemize}
\item {Utilização:Bras}
\end{itemize}
Gorgeta; propina.
\section{Xixilado}
\begin{itemize}
\item {Grp. gram.:adj.}
\end{itemize}
\begin{itemize}
\item {Utilização:Bras. da Baía}
\end{itemize}
Descarado; sem vergonha. Cf. \textunderscore Jorn.-do-Comm.\textunderscore , do Rio, de 25-VII-902.
\section{Xixuão}
\begin{itemize}
\item {Grp. gram.:m.}
\end{itemize}
\begin{itemize}
\item {Utilização:Ant.}
\end{itemize}
Espécie de pano moirisco. Cf. L. Sousa, \textunderscore Ann. de D. João III\textunderscore .
\section{Xizinho}
\begin{itemize}
\item {Grp. gram.:m.}
\end{itemize}
\begin{itemize}
\item {Utilização:T. de Lanhoso}
\end{itemize}
\begin{itemize}
\item {Proveniência:(De \textunderscore xis\textunderscore ?)}
\end{itemize}
Pouca coisa; bocadinho.
\section{Xiziro}
\begin{itemize}
\item {Grp. gram.:m.}
\end{itemize}
\begin{itemize}
\item {Utilização:Ant.}
\end{itemize}
Substância encarnada, que servia especialmente para com ella se escreverem os títulos dos livros e capítulos.
(Por \textunderscore sirizo\textunderscore . Cp. b. lat. \textunderscore siricum\textunderscore )
\section{Xó!}
\begin{itemize}
\item {Grp. gram.:interj.}
\end{itemize}
(Fórma exacta, em vez de \textunderscore chó\textunderscore )
(Cp. cast. \textunderscore so\textunderscore )
\section{Xô!}
\begin{itemize}
\item {Grp. gram.:interj.}
\end{itemize}
(que serve para enxotar gallinhas ou outras aves); o mesmo que \textunderscore xote\textunderscore .
\section{Xocar}
\begin{itemize}
\item {Grp. gram.:v. i.}
\end{itemize}
\begin{itemize}
\item {Utilização:Prov.}
\end{itemize}
\begin{itemize}
\item {Utilização:trasm.}
\end{itemize}
\begin{itemize}
\item {Proveniência:(De \textunderscore xô\textunderscore )}
\end{itemize}
Enxotar gallinhas ou outras aves.
\section{Xofrango}
\begin{itemize}
\item {Grp. gram.:m.}
\end{itemize}
\begin{itemize}
\item {Proveniência:(Do lat. \textunderscore ossifragus\textunderscore ?)}
\end{itemize}
Espécie de águia pequena.
\section{Xolra}
\begin{itemize}
\item {Grp. gram.:f.}
\end{itemize}
\begin{itemize}
\item {Utilização:Pesc.}
\end{itemize}
Antiga rêde, usada na pesca costeira do Algarve.
\section{Xomanas}
\begin{itemize}
\item {Grp. gram.:m. pl.}
\end{itemize}
Índios selvagens das margens do Japurá, no Brasil.
\section{Xona}
\begin{itemize}
\item {Grp. gram.:f.}
\end{itemize}
\begin{itemize}
\item {Utilização:Gír.}
\end{itemize}
O mesmo que \textunderscore noite\textunderscore .
\section{Xòninhas}
\begin{itemize}
\item {Grp. gram.:m.}
\end{itemize}
\begin{itemize}
\item {Utilização:Prov.}
\end{itemize}
\begin{itemize}
\item {Utilização:minh.}
\end{itemize}
Indivíduo acanhado, mollangueirão.
\section{Xorca}
\begin{itemize}
\item {Grp. gram.:f.}
\end{itemize}
O mesmo que \textunderscore axorca\textunderscore .
\section{Xorcas}
\begin{itemize}
\item {Grp. gram.:f. pl.}
\end{itemize}
\begin{itemize}
\item {Utilização:Prov.}
\end{itemize}
\begin{itemize}
\item {Proveniência:(T. onom., do som que as xorcas produzem quando se anda com ellas?)}
\end{itemize}
Sapatos grandes, acalcanhados e mal feitos.
\section{Xordo}
\begin{itemize}
\item {fónica:xôr}
\end{itemize}
\begin{itemize}
\item {Grp. gram.:adj.}
\end{itemize}
\begin{itemize}
\item {Utilização:Prov.}
\end{itemize}
\begin{itemize}
\item {Utilização:trasm.}
\end{itemize}
Manco e surdo.
(Cp. \textunderscore surdo\textunderscore )
\section{Xotar}
\begin{itemize}
\item {Grp. gram.:v. t.}
\end{itemize}
\begin{itemize}
\item {Utilização:Prov.}
\end{itemize}
\begin{itemize}
\item {Utilização:trasm.}
\end{itemize}
O mesmo que \textunderscore enxotar\textunderscore .
\section{Xote}
\begin{itemize}
\item {Grp. gram.:interj.}
\end{itemize}
O mesmo ou melhór que \textunderscore chote\textunderscore .
\section{Xote}
\begin{itemize}
\item {Grp. gram.:m.}
\end{itemize}
\begin{itemize}
\item {Utilização:T. de Leiria}
\end{itemize}
O mesmo que \textunderscore caixote\textunderscore .
\section{Xote!}
\begin{itemize}
\item {Grp. gram.:interj.}
\end{itemize}
\begin{itemize}
\item {Proveniência:(T. onom.)}
\end{itemize}
(para afugentar aves)
\section{Xô-Xô!}
\begin{itemize}
\item {Grp. gram.:interj.}
\end{itemize}
Voz, com que se enxotam animaes, mormente as aves; o mesmo que \textunderscore xô\textunderscore .
\section{X. P. T. O.}
\begin{itemize}
\item {Proveniência:(De uma abrev. medieval da palavra \textunderscore Christo\textunderscore )}
\end{itemize}
Expressão burlesca, para designar qualidade excellente, magnífica.
\section{Xrámana}
\begin{itemize}
\item {Grp. gram.:m.}
\end{itemize}
Asceta mendicante, entre os Índios.
Asceta budista.
(Do sânscr.)
\section{Xrámane}
\begin{itemize}
\item {Grp. gram.:m.}
\end{itemize}
Asceta mendicante, entre os Índios.
Asceta budista.
(Do sânscr.)
\section{Xuarézia}
\begin{itemize}
\item {Grp. gram.:f.}
\end{itemize}
\begin{itemize}
\item {Proveniência:(De \textunderscore Xuarez\textunderscore , n. p.)}
\end{itemize}
Gênero de plantas escrofularíneas.
\section{Xucro}
\begin{itemize}
\item {Grp. gram.:adj.}
\end{itemize}
\begin{itemize}
\item {Utilização:Bras}
\end{itemize}
Bravo, ou ainda não domesticado.
\section{Xudairo}
\begin{itemize}
\item {Grp. gram.:m.}
\end{itemize}
\begin{itemize}
\item {Utilização:Prov.}
\end{itemize}
\begin{itemize}
\item {Utilização:trasm.}
\end{itemize}
Mulhér desprezível, rota, esfrangalhada.
(Cp. \textunderscore sudário\textunderscore )
\section{Xupé}
\begin{itemize}
\item {Grp. gram.:m.}
\end{itemize}
\begin{itemize}
\item {Utilização:Bras}
\end{itemize}
Espécie de abelha negra.
\section{Xurdir}
\begin{itemize}
\item {Grp. gram.:v. i.}
\end{itemize}
\begin{itemize}
\item {Utilização:Prov.}
\end{itemize}
\begin{itemize}
\item {Utilização:trasm.}
\end{itemize}
Lutar pela vida, moirejar.
\section{Xurreira}
\begin{itemize}
\item {Grp. gram.:f.}
\end{itemize}
Buraco ou lugar, por onde entra o enxurro.
(Por \textunderscore enxurreira\textunderscore , de \textunderscore enxurro\textunderscore )
\section{Xurumbambos}
\begin{itemize}
\item {Grp. gram.:m. pl.}
\end{itemize}
\begin{itemize}
\item {Utilização:Bras}
\end{itemize}
Cacaréus.
\section{Xuxiapom}
\begin{itemize}
\item {Grp. gram.:m.}
\end{itemize}
Residência do Soberano, entre os antigos Chineses. Cf. \textunderscore Peregrinação\textunderscore , CXXI.
\section{Xuxo}
\begin{itemize}
\item {Grp. gram.:m.}
\end{itemize}
Peixe de Portugal.
\section{Xylena}
\begin{itemize}
\item {Grp. gram.:f.}
\end{itemize}
\begin{itemize}
\item {Utilização:Chím.}
\end{itemize}
\begin{itemize}
\item {Proveniência:(Do gr. \textunderscore xulon\textunderscore )}
\end{itemize}
Gênero de insectos lepidópteros nocturnos.
Variedade de carboneto de hydrogênio líquido.
\section{Xyleno}
\begin{itemize}
\item {Grp. gram.:m.}
\end{itemize}
\begin{itemize}
\item {Utilização:Chím.}
\end{itemize}
\begin{itemize}
\item {Proveniência:(Do gr. \textunderscore xulon\textunderscore )}
\end{itemize}
Um dos carbonetos do grupo benzênico.
\section{Xylite}
\begin{itemize}
\item {Grp. gram.:f.}
\end{itemize}
\begin{itemize}
\item {Proveniência:(Do gr. \textunderscore xulon\textunderscore )}
\end{itemize}
Producto da destillação do espírito de madeira.
\section{Xylo...}
\begin{itemize}
\item {Grp. gram.:pref.}
\end{itemize}
\begin{itemize}
\item {Proveniência:(Do gr. \textunderscore xulon\textunderscore )}
\end{itemize}
(designativo de \textunderscore madeira\textunderscore )
\section{Xylo}
\begin{itemize}
\item {Grp. gram.:m.}
\end{itemize}
\begin{itemize}
\item {Proveniência:(Do gr. \textunderscore xulon\textunderscore )}
\end{itemize}
A planta do algodão; algodoeiro.
\section{Xylobálsamo}
\begin{itemize}
\item {Grp. gram.:m.}
\end{itemize}
\begin{itemize}
\item {Proveniência:(Lat. \textunderscore xylobalsamum\textunderscore )}
\end{itemize}
O mesmo que \textunderscore balsameiro\textunderscore .
\section{Xylocarpo}
\begin{itemize}
\item {Grp. gram.:adj.}
\end{itemize}
\begin{itemize}
\item {Utilização:Bot.}
\end{itemize}
\begin{itemize}
\item {Proveniência:(Do gr. \textunderscore xulon\textunderscore  + \textunderscore karpos\textunderscore )}
\end{itemize}
Diz-se das árvores de frutos duros ou lenhosos.
\section{Xylócopo}
\begin{itemize}
\item {Grp. gram.:adj.}
\end{itemize}
\begin{itemize}
\item {Utilização:Zool.}
\end{itemize}
\begin{itemize}
\item {Proveniência:(Gr. \textunderscore xulokopos\textunderscore )}
\end{itemize}
Que córta a madeira, que pica paus, que fura os troncos: \textunderscore há abelhas xylócopas\textunderscore .
\section{Xylódia}
\begin{itemize}
\item {Grp. gram.:f.}
\end{itemize}
\begin{itemize}
\item {Proveniência:(Do gr. \textunderscore xulodes\textunderscore )}
\end{itemize}
Gênero de frutos lenhosos, como a avelan.
\section{Xylogênio}
\begin{itemize}
\item {Grp. gram.:m.}
\end{itemize}
\begin{itemize}
\item {Utilização:Chím.}
\end{itemize}
\begin{itemize}
\item {Proveniência:(Do gr. \textunderscore xulon\textunderscore  + \textunderscore genos\textunderscore )}
\end{itemize}
Substância vegetal, que se encontra especialmente na parede primitiva das céllulas das plantas.
\section{Xyloglyphia}
\begin{itemize}
\item {Grp. gram.:f.}
\end{itemize}
Arte de esculpir em madeira; arte de gravar caracteres em madeira.
(Cp. \textunderscore xylóglypho\textunderscore )
\section{Xyloglýphico}
\begin{itemize}
\item {Grp. gram.:adj.}
\end{itemize}
Relativo á xyloglyphia.
\section{Xylóglypho}
\begin{itemize}
\item {Grp. gram.:m.}
\end{itemize}
\begin{itemize}
\item {Proveniência:(Do gr. \textunderscore xulon\textunderscore  + \textunderscore gluphos\textunderscore )}
\end{itemize}
Aquelle que exerce a xyloglyphia.
Aquelle que faz letras com ornatos, para livros, ou grandes letras, para cartazes.
\section{Xylographia}
\begin{itemize}
\item {Grp. gram.:f.}
\end{itemize}
\begin{itemize}
\item {Proveniência:(De \textunderscore xylógrapho\textunderscore )}
\end{itemize}
Arte de gravar em madeira.
\section{Xylográphico}
\begin{itemize}
\item {Grp. gram.:adj.}
\end{itemize}
Relativo á xylographia.
\section{Xylógrapho}
\begin{itemize}
\item {Grp. gram.:m.}
\end{itemize}
\begin{itemize}
\item {Proveniência:(Do gr. \textunderscore xulon\textunderscore  + \textunderscore graphein\textunderscore )}
\end{itemize}
Aquelle que grava em madeira.
\section{Xylóide}
\begin{itemize}
\item {Grp. gram.:adj.}
\end{itemize}
\begin{itemize}
\item {Proveniência:(Gr. \textunderscore xuloides\textunderscore )}
\end{itemize}
Relativo a madeira.
Proveniente de um corpo lenhoso.
\section{Xyloídico}
\begin{itemize}
\item {Grp. gram.:adj.}
\end{itemize}
Diz-se de um supposto typo de architectura em madeira, donde teria procedido a architectura gótica.
(Cp. \textunderscore xylóide\textunderscore )
\section{Xyloidina}
\begin{itemize}
\item {Grp. gram.:f.}
\end{itemize}
\begin{itemize}
\item {Utilização:Chím.}
\end{itemize}
\begin{itemize}
\item {Proveniência:(De \textunderscore xylóide\textunderscore )}
\end{itemize}
Matéria muito combustível, que se obtém pela acção do ácido nítrico sôbre os princípios neutros vegetaes.
\section{Xylólatra}
\begin{itemize}
\item {Grp. gram.:m.}
\end{itemize}
\begin{itemize}
\item {Proveniência:(Do gr. \textunderscore xulolatres\textunderscore )}
\end{itemize}
Aquelle que adora os ídolos de madeira.
\section{Xylolatria}
\begin{itemize}
\item {Grp. gram.:f.}
\end{itemize}
Qualidade de xylólatra.
Adoração dos ídolos de madeira.
\section{Xylólitho}
\begin{itemize}
\item {Grp. gram.:f.}
\end{itemize}
\begin{itemize}
\item {Proveniência:(Do gr. \textunderscore xulon\textunderscore  + \textunderscore lithos\textunderscore )}
\end{itemize}
Madeira fóssil.
\section{Xylologia}
\begin{itemize}
\item {Grp. gram.:f.}
\end{itemize}
\begin{itemize}
\item {Proveniência:(Do gr. \textunderscore xulon\textunderscore  + \textunderscore logos\textunderscore )}
\end{itemize}
Tratado ou história das madeiras.
\section{Xylológico}
\begin{itemize}
\item {Grp. gram.:adj.}
\end{itemize}
Relativo á xylologia.
\section{Xylólogo}
\begin{itemize}
\item {Grp. gram.:m.}
\end{itemize}
Aquelle que é perito em xylologia.
\section{Xyloma}
\begin{itemize}
\item {Grp. gram.:m.}
\end{itemize}
\begin{itemize}
\item {Proveniência:(Do gr. \textunderscore xulon\textunderscore  + \textunderscore omos\textunderscore )}
\end{itemize}
Gênero de cogumelos lenhosos.
\section{Xylomancia}
\begin{itemize}
\item {Grp. gram.:f.}
\end{itemize}
\begin{itemize}
\item {Proveniência:(Do gr. \textunderscore xulon\textunderscore  + \textunderscore manteia\textunderscore )}
\end{itemize}
Supposta arte de adivinhar, por meio da disposição dos pauzinhos secos, que se encontram pelo caminho.
\section{Xylómice}
\begin{itemize}
\item {Grp. gram.:adj.}
\end{itemize}
\begin{itemize}
\item {Utilização:Bot.}
\end{itemize}
\begin{itemize}
\item {Proveniência:(Do gr. \textunderscore xulon\textunderscore  + \textunderscore mukes\textunderscore )}
\end{itemize}
Diz-se dos cogumelos, que crescem sôbre a madeira ou nas árvores.
\section{Xýlon}
\begin{itemize}
\item {Grp. gram.:m.}
\end{itemize}
\begin{itemize}
\item {Utilização:Chím.}
\end{itemize}
\begin{itemize}
\item {Proveniência:(Gr. \textunderscore xulon\textunderscore )}
\end{itemize}
Cellulose da madeira ou dos invólucros dos frutos duros.
\section{Xylophagia}
\begin{itemize}
\item {Grp. gram.:f.}
\end{itemize}
Acto de roer a madeira.
Propriedade de xylóphago.
\section{Xilofágico}
\begin{itemize}
\item {Grp. gram.:adj.}
\end{itemize}
Relativo á xilofagia.
\section{Xilófago}
\begin{itemize}
\item {Grp. gram.:m.}
\end{itemize}
\begin{itemize}
\item {Proveniência:(Do gr. \textunderscore xulon\textunderscore  + \textunderscore phagein\textunderscore )}
\end{itemize}
Insecto, que rói a madeira.
\section{Xilófilo}
\begin{itemize}
\item {Grp. gram.:m.}
\end{itemize}
\begin{itemize}
\item {Proveniência:(Do gr. \textunderscore xulon\textunderscore  + \textunderscore philos\textunderscore )}
\end{itemize}
Insecto escarabídeo.
\section{Xilofónio}
\begin{itemize}
\item {Grp. gram.:m.}
\end{itemize}
Instrumento, o mesmo que \textunderscore xilharmónico\textunderscore .
\section{Xilofonista}
\begin{itemize}
\item {Grp. gram.:m.}
\end{itemize}
Tocador de xilofónio.
\section{Xilofória}
\begin{itemize}
\item {Grp. gram.:f.}
\end{itemize}
\begin{itemize}
\item {Proveniência:(Do gr. \textunderscore xulon\textunderscore  + \textunderscore phoros\textunderscore )}
\end{itemize}
Festa, em que os Hebreus levavam lenha para o fogo sagrado.
\section{Xilóforo}
\begin{itemize}
\item {Grp. gram.:m.}
\end{itemize}
\begin{itemize}
\item {Proveniência:(Do gr. \textunderscore xulon\textunderscore  + \textunderscore phoros\textunderscore )}
\end{itemize}
Cada um dos sacerdotes judeus, que acendiam e mantinham o fogo sagrado.
\section{Xilópia}
\begin{itemize}
\item {Grp. gram.:f.}
\end{itemize}
\begin{itemize}
\item {Proveniência:(Do gr. \textunderscore xulon\textunderscore  + \textunderscore ops\textunderscore )}
\end{itemize}
Planta anonácea dos trópicos.
\section{Xilórgão}
\begin{itemize}
\item {Grp. gram.:m.}
\end{itemize}
Instrumento, o mesmo que \textunderscore xilharmónico\textunderscore .
\section{Xilossistro}
\begin{itemize}
\item {Grp. gram.:m.}
\end{itemize}
Instrumento, o mesmo que \textunderscore xilharmónico\textunderscore .
\section{Xilótomo}
\begin{itemize}
\item {Grp. gram.:m.}
\end{itemize}
\begin{itemize}
\item {Utilização:Zool.}
\end{itemize}
\begin{itemize}
\item {Proveniência:(Gr. \textunderscore xulotomos\textunderscore )}
\end{itemize}
O mesmo que \textunderscore xilócopo\textunderscore .
\section{Xiridáceas}
\begin{itemize}
\item {Grp. gram.:f. pl.}
\end{itemize}
Família de plantas, que tem por tipo o xíris.
(Fem. pl. de \textunderscore xyridáceo\textunderscore )
\section{Xiridáceo}
\begin{itemize}
\item {Grp. gram.:adj.}
\end{itemize}
Relativo ou semelhante ao xíris.
\section{Xíris}
\begin{itemize}
\item {Grp. gram.:m.}
\end{itemize}
\begin{itemize}
\item {Proveniência:(Gr. \textunderscore xuris\textunderscore )}
\end{itemize}
Gênero de plantas, cujas espécies crescem principalmente na América.
\section{Xisto}
\begin{itemize}
\item {Grp. gram.:m.}
\end{itemize}
\begin{itemize}
\item {Proveniência:(Lat. \textunderscore xystus\textunderscore )}
\end{itemize}
Pórtico coberto, entre os Gregos.
Lugar ou galeria descoberta, para passeio, entre os Romanos.
\section{Xisto}
\begin{itemize}
\item {Grp. gram.:m.}
\end{itemize}
\begin{itemize}
\item {Proveniência:(Do gr. \textunderscore xustos\textunderscore )}
\end{itemize}
Gênero de insectos dípteros.
\section{Xistrópodes}
\begin{itemize}
\item {Grp. gram.:m. pl.}
\end{itemize}
\begin{itemize}
\item {Utilização:Zool.}
\end{itemize}
\begin{itemize}
\item {Proveniência:(Do gr. \textunderscore xustra\textunderscore  + \textunderscore pous\textunderscore , \textunderscore podos\textunderscore )}
\end{itemize}
Divisão da classe das aves, que compreende as galináceas e as columbinas.
\section{Xylophágico}
\begin{itemize}
\item {Grp. gram.:adj.}
\end{itemize}
Relativo á xylophagia.
\section{Xylóphago}
\begin{itemize}
\item {Grp. gram.:m.}
\end{itemize}
\begin{itemize}
\item {Proveniência:(Do gr. \textunderscore xulon\textunderscore  + \textunderscore phagein\textunderscore )}
\end{itemize}
Insecto, que rói a madeira.
\section{Xylóphilo}
\begin{itemize}
\item {Grp. gram.:m.}
\end{itemize}
\begin{itemize}
\item {Proveniência:(Do gr. \textunderscore xulon\textunderscore  + \textunderscore philos\textunderscore )}
\end{itemize}
Insecto escarabídeo.
\section{Xylophónio}
\begin{itemize}
\item {Grp. gram.:m.}
\end{itemize}
Instrumento, o mesmo que \textunderscore xylharmónico\textunderscore .
\section{Xylophonista}
\begin{itemize}
\item {Grp. gram.:m.}
\end{itemize}
Tocador de xylophónio.
\section{Xylophória}
\begin{itemize}
\item {Grp. gram.:f.}
\end{itemize}
\begin{itemize}
\item {Proveniência:(Do gr. \textunderscore xulon\textunderscore  + \textunderscore phoros\textunderscore )}
\end{itemize}
Festa, em que os Hebreus levavam lenha para o fogo sagrado.
\section{Xylóphoro}
\begin{itemize}
\item {Grp. gram.:m.}
\end{itemize}
\begin{itemize}
\item {Proveniência:(Do gr. \textunderscore xulon\textunderscore  + \textunderscore phoros\textunderscore )}
\end{itemize}
Cada um dos sacerdotes judeus, que acendiam e mantinham o fogo sagrado.
\section{Xylópia}
\begin{itemize}
\item {Grp. gram.:f.}
\end{itemize}
\begin{itemize}
\item {Proveniência:(Do gr. \textunderscore xulon\textunderscore  + \textunderscore ops\textunderscore )}
\end{itemize}
Planta anonácea dos trópicos.
\section{Xylórgão}
\begin{itemize}
\item {Grp. gram.:m.}
\end{itemize}
Instrumento, o mesmo que \textunderscore xylharmónico\textunderscore .
\section{Xylosistro}
\begin{itemize}
\item {fónica:sis}
\end{itemize}
\begin{itemize}
\item {Grp. gram.:m.}
\end{itemize}
Instrumento, o mesmo que \textunderscore xylharmónico\textunderscore .
\section{Xylótomo}
\begin{itemize}
\item {Grp. gram.:m.}
\end{itemize}
\begin{itemize}
\item {Utilização:Zool.}
\end{itemize}
\begin{itemize}
\item {Proveniência:(Gr. \textunderscore xulotomos\textunderscore )}
\end{itemize}
O mesmo que \textunderscore xylócopo\textunderscore .
\section{Xyridáceas}
\begin{itemize}
\item {Grp. gram.:f. pl.}
\end{itemize}
Família de plantas, que tem por typo o xýris.
(Fem. pl. de \textunderscore xyridáceo\textunderscore )
\section{Xyridáceo}
\begin{itemize}
\item {Grp. gram.:adj.}
\end{itemize}
Relativo ou semelhante ao xýris.
\section{Xýris}
\begin{itemize}
\item {Grp. gram.:m.}
\end{itemize}
\begin{itemize}
\item {Proveniência:(Gr. \textunderscore xuris\textunderscore )}
\end{itemize}
Gênero de plantas, cujas espécies crescem principalmente na América.
\section{Xysto}
\begin{itemize}
\item {Grp. gram.:m.}
\end{itemize}
\begin{itemize}
\item {Proveniência:(Lat. \textunderscore xystus\textunderscore )}
\end{itemize}
Pórtico coberto, entre os Gregos.
Lugar ou galeria descoberta, para passeio, entre os Romanos.
\section{Xysto}
\begin{itemize}
\item {Grp. gram.:m.}
\end{itemize}
\begin{itemize}
\item {Proveniência:(Do gr. \textunderscore xustos\textunderscore )}
\end{itemize}
Gênero de insectos dípteros.
\section{Xystrópodes}
\begin{itemize}
\item {Grp. gram.:m. pl.}
\end{itemize}
\begin{itemize}
\item {Utilização:Zool.}
\end{itemize}
\begin{itemize}
\item {Proveniência:(Do gr. \textunderscore xustra\textunderscore  + \textunderscore pous\textunderscore , \textunderscore podos\textunderscore )}
\end{itemize}
\end{document}