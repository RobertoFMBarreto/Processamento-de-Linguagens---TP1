\documentclass{article}
\usepackage[portuguese]{babel}
\title{J}
\begin{document}
Língua uralo altaica do ramo samoiedo.
\section{J}
\begin{itemize}
\item {fónica:jota}
\end{itemize}
\begin{itemize}
\item {Grp. gram.:m.}
\end{itemize}
\begin{itemize}
\item {Grp. gram.:Adj.}
\end{itemize}
Décima letra do alphabeto português.
Que numa série occupa o décimo lugar.--Tem a mesma or. que o \textunderscore i\textunderscore , e com êlle se confundiu entre os Latinos e na antiga escrita portuguesa, embora se destinguisse, na pronúncia, em letra vogal e letra consoante.
\section{Já}
\begin{itemize}
\item {Grp. gram.:adv.}
\end{itemize}
\begin{itemize}
\item {Grp. gram.:Loc. conjunct.}
\end{itemize}
\begin{itemize}
\item {Proveniência:(Lat. \textunderscore jam\textunderscore )}
\end{itemize}
Neste momento; promptamente, sem demora: \textunderscore tira-te já daqui\textunderscore .
Agora: \textunderscore já é tempo de teres juízo\textunderscore .
Anteriormente: \textunderscore já o encontrei uma vez\textunderscore .
Desde então.
Nesse tempo.
Noutro tempo.
Até.
Daqui a pouco.
Ora: \textunderscore já ensinando, já praticando\textunderscore .
\textunderscore Já que\textunderscore , visto que.--Emprega-se também como expletiva.
\section{Jaarabôa}
\begin{itemize}
\item {Grp. gram.:m.}
\end{itemize}
\begin{itemize}
\item {Utilização:Bras}
\end{itemize}
Espécie de feijão.
\section{Jabá}
\begin{itemize}
\item {Grp. gram.:m.}
\end{itemize}
\begin{itemize}
\item {Utilização:Bras}
\end{itemize}
O mesmo que \textunderscore charque\textunderscore .
\section{Jabão}
\begin{itemize}
\item {Grp. gram.:m.}
\end{itemize}
\begin{itemize}
\item {Utilização:T. de Setúbal}
\end{itemize}
O mesmo que \textunderscore urgebão\textunderscore .
\section{Jabebireta}
\begin{itemize}
\item {fónica:birê}
\end{itemize}
\begin{itemize}
\item {Grp. gram.:f.}
\end{itemize}
\begin{itemize}
\item {Utilização:Bras}
\end{itemize}
Peixe, espécie de arraia.
\section{Jábeca}
\begin{itemize}
\item {Grp. gram.:f.}
\end{itemize}
Espécie de flauta, com que os Moiros imitam o órgão.
(Do ár.)
\section{Jabiru}
\begin{itemize}
\item {Grp. gram.:m.}
\end{itemize}
Ave pernalta do Brasil.
\section{Jaborândi}
\begin{itemize}
\item {Grp. gram.:m.}
\end{itemize}
Nome de várias plantas intertropicaes, rutáceas e piperáceas; especialmente, arbusto rutáceo, cujas fôlhas sêcas e pulverizadas constituem um medicamento sudorifico.
Medicamento, constituído pelos pós do jaborândi.
\section{Jaborandina}
\begin{itemize}
\item {Grp. gram.:f.}
\end{itemize}
Alcaloide, extrahido das fôlhas do jaborândi.
\section{Jaborina}
\begin{itemize}
\item {Grp. gram.:f.}
\end{itemize}
Base isómera da pilorcapina, extrahida das fôlhas do jaborândi.
\section{Jaboru}
\begin{itemize}
\item {Grp. gram.:m.}
\end{itemize}
Ave aquática do Brasil.
\section{Jabotá}
\begin{itemize}
\item {Grp. gram.:m.}
\end{itemize}
Planta cucurbitácea do Brasil.
\section{Jabotapitá}
\begin{itemize}
\item {Grp. gram.:f.}
\end{itemize}
(V.batiputá)
\section{Jaburu}
\begin{itemize}
\item {Grp. gram.:m.}
\end{itemize}
(V.jabiru)
\section{Jabuti}
\begin{itemize}
\item {Grp. gram.:m.}
\end{itemize}
\begin{itemize}
\item {Utilização:Bras}
\end{itemize}
\begin{itemize}
\item {Proveniência:(T. tupi)}
\end{itemize}
Espécie de tartaruga.
\section{Jabuticaba}
\begin{itemize}
\item {Grp. gram.:f.}
\end{itemize}
Fruto de jabuticabeira.
Jabuticabeira.
(Do tupi)
\section{Jabuticabeira}
\begin{itemize}
\item {Grp. gram.:f.}
\end{itemize}
\begin{itemize}
\item {Utilização:Bras}
\end{itemize}
\begin{itemize}
\item {Proveniência:(De \textunderscore jabuticaba\textunderscore )}
\end{itemize}
Árvore myrtácea.
\section{Jabutim}
\begin{itemize}
\item {Grp. gram.:m.}
\end{itemize}
O mesmo que \textunderscore jabuti\textunderscore . Cf. Gonç. Dias, \textunderscore Poesias\textunderscore , 312.
\section{Jabutimata}
\begin{itemize}
\item {Grp. gram.:f.}
\end{itemize}
\begin{itemize}
\item {Utilização:Bras}
\end{itemize}
Planta leguminosa.
\section{Jabutipé}
\begin{itemize}
\item {Grp. gram.:m.}
\end{itemize}
Árvore brasileira, própria para construcções.
\section{Jaca}
\begin{itemize}
\item {Grp. gram.:m.}
\end{itemize}
Régulo de régulos, ou chefe superiôr de várias tríbos africanas.
\section{Jaca}
\begin{itemize}
\item {Grp. gram.:m.}
\end{itemize}
Fruto de jaqueira.
Jaqueira.
(Do malaialim \textunderscore chakka\textunderscore )
\section{Jaca}
\begin{itemize}
\item {Grp. gram.:f.}
\end{itemize}
(?):«\textunderscore Mas já tudo, queres que te conte, sabe que me deteve a jaca leve.\textunderscore »D. Bernárdez, \textunderscore Lima\textunderscore , 99.
\section{Jacá}
\begin{itemize}
\item {Grp. gram.:m.}
\end{itemize}
\begin{itemize}
\item {Utilização:Bras}
\end{itemize}
Espécie de cesto, em que os animaes transportam carne, peixe, queijo, etc.
(Contr. do tupi \textunderscore aiacá\textunderscore )
\section{Jaça}
\begin{itemize}
\item {Grp. gram.:f.}
\end{itemize}
Substância heterogênea, que se vê nas pedras preciosas.
Mancha:«\textunderscore diamante tão sem jaça.\textunderscore »Bernárdez, \textunderscore Luz e Calor\textunderscore , 309.
\section{Jaça}
\begin{itemize}
\item {Grp. gram.:f.}
\end{itemize}
\begin{itemize}
\item {Utilização:Chul.}
\end{itemize}
\begin{itemize}
\item {Proveniência:(Do rad. do lat. \textunderscore jacere\textunderscore )}
\end{itemize}
Calaboiço; cama.
\section{Jacacal}
\begin{itemize}
\item {Grp. gram.:m.}
\end{itemize}
Ave do Brasil.
\section{Jacaiol}
\begin{itemize}
\item {Grp. gram.:m.}
\end{itemize}
Ave do Brasil.
\section{Jacama}
\begin{itemize}
\item {Grp. gram.:f.}
\end{itemize}
\begin{itemize}
\item {Utilização:Bras. do Maranhão}
\end{itemize}
Variedade de araticu.
\section{Jacamáici}
\begin{itemize}
\item {Grp. gram.:m.}
\end{itemize}
Ave trepadora do Brasil.
\section{Jacamar}
\begin{itemize}
\item {Grp. gram.:m.}
\end{itemize}
Ave trepadora do Brasil.
\section{Jacami}
\begin{itemize}
\item {Grp. gram.:m.}
\end{itemize}
\begin{itemize}
\item {Proveniência:(T. tupi)}
\end{itemize}
Nome de várias espécies de aves ribeirinhas.
\section{Jacamim}
\begin{itemize}
\item {Grp. gram.:m.}
\end{itemize}
\begin{itemize}
\item {Utilização:Bras. do N}
\end{itemize}
\begin{itemize}
\item {Proveniência:(T. tupi)}
\end{itemize}
Nome de várias espécies de aves ribeirinhas.
\section{Jacamincá}
\begin{itemize}
\item {Grp. gram.:f.}
\end{itemize}
Planta herbácia do Brasil.
\section{Jacana}
\begin{itemize}
\item {Grp. gram.:f.}
\end{itemize}
Gênero de aves americanas e asiáticas, semelhantes á gallinhola.
(Cp. \textunderscore jaçanha\textunderscore )
\section{Jaçanan}
\begin{itemize}
\item {Grp. gram.:m.}
\end{itemize}
\begin{itemize}
\item {Utilização:Bras}
\end{itemize}
\begin{itemize}
\item {Grp. gram.:f.}
\end{itemize}
\begin{itemize}
\item {Utilização:Bras}
\end{itemize}
Ave de peito avermelhado.
O mesmo que \textunderscore jaçanha\textunderscore ?
Pequena ave ribeirinha.
\section{Jaçanha}
\begin{itemize}
\item {Grp. gram.:f.}
\end{itemize}
Ave pernalta do Brasil.--Vejo êste nome em diccionários; em escritores brasileiros vejo \textunderscore jacana\textunderscore  e \textunderscore jaçanan\textunderscore . Serão aves distintas? Haverá êrro typográphico ou equívoco de alguns dos alludidos escritores?
\section{Jacapa}
\begin{itemize}
\item {Grp. gram.:f.}
\end{itemize}
Pássaro brasileiro escuro, de bico branco.
Cp. \textunderscore bico-de-prata\textunderscore .
\section{Jacapâni}
\begin{itemize}
\item {Grp. gram.:m.}
\end{itemize}
Ave do Brasil.
\section{Jacapau}
\begin{itemize}
\item {Grp. gram.:m.}
\end{itemize}
\begin{itemize}
\item {Utilização:Bras}
\end{itemize}
Espécie de cotovia.
\section{Jaçapé}
\begin{itemize}
\item {Grp. gram.:m.}
\end{itemize}
Planta herbácea do Brasil, espécie de capim.
\section{Jacapucaio}
\begin{itemize}
\item {Grp. gram.:m.}
\end{itemize}
(V.sapucaia)
\section{Jacará}
\begin{itemize}
\item {Grp. gram.:m.}
\end{itemize}
Quadrúpede do Malabar.
\section{Jacarácia}
\begin{itemize}
\item {Grp. gram.:f.}
\end{itemize}
Planta espinhosa do Brasil.
\section{Jacarandá}
\begin{itemize}
\item {Grp. gram.:f.}
\end{itemize}
\begin{itemize}
\item {Utilização:Bras}
\end{itemize}
Nome de várias plantas bignoniáceas e leguminosas.
\section{Jacarandá-caroba}
\begin{itemize}
\item {Grp. gram.:m.}
\end{itemize}
O mesmo que \textunderscore caroba\textunderscore .
\section{Jacarandana}
\begin{itemize}
\item {Grp. gram.:f.}
\end{itemize}
Árvore silvestre da América.
\section{Jacaratinga}
\begin{itemize}
\item {Grp. gram.:f.}
\end{itemize}
Planta myrtácea e silvestre do Brasil.
Fruto dessa planta.
\section{Jacaré}
\begin{itemize}
\item {Grp. gram.:m.}
\end{itemize}
\begin{itemize}
\item {Utilização:Prov.}
\end{itemize}
\begin{itemize}
\item {Utilização:minh.}
\end{itemize}
\begin{itemize}
\item {Utilização:Bras}
\end{itemize}
\begin{itemize}
\item {Proveniência:(T. tupi)}
\end{itemize}
Espécie de crocodilo; caimão.
Bisbórria; pateta.
Variedade de pimenta roxa.
\section{Jacaré-aru}
\begin{itemize}
\item {Grp. gram.:m.}
\end{itemize}
\begin{itemize}
\item {Utilização:Bras}
\end{itemize}
O mesmo que \textunderscore caferana\textunderscore .
\section{Jacaré-cacau}
\begin{itemize}
\item {Grp. gram.:f.}
\end{itemize}
Fruto silvestre do Brasil.
\section{Jacaré-copaíba}
\begin{itemize}
\item {Grp. gram.:f.}
\end{itemize}
Árvore clusiácea do Alto Amazonas, (\textunderscore calaphilum brásiliensis\textunderscore ).
\section{Jacaré-de-óculos}
\begin{itemize}
\item {Grp. gram.:m.}
\end{itemize}
Espécie de jacaré inoffensivo, (\textunderscore aligator sclerops\textunderscore ), que tem uma listra entre os olhos, o que dá o aspecto de óculos.
\section{Jacarei-ataúna}
\begin{itemize}
\item {Grp. gram.:f.}
\end{itemize}
\begin{itemize}
\item {Utilização:Bras}
\end{itemize}
Planta trepadeira, (\textunderscore gonania apidenculata\textunderscore ).
\section{Jacaretafás}
\begin{itemize}
\item {Grp. gram.:m. pl.}
\end{itemize}
\begin{itemize}
\item {Utilização:Bras}
\end{itemize}
Uma das tríbos aborígenes do Pará.
\section{Jacaréu}
\begin{itemize}
\item {Grp. gram.:m.}
\end{itemize}
\begin{itemize}
\item {Utilização:Des.}
\end{itemize}
O mesmo que \textunderscore jacaré\textunderscore .
Casta de sardinha, que, em a Nazaré, se séca ao sol, depois de salpicada de sal.
\section{Jacaré-uva}
\begin{itemize}
\item {Grp. gram.:m.}
\end{itemize}
O mesmo que \textunderscore lantim\textunderscore .
\section{Jacarini}
\begin{itemize}
\item {Grp. gram.:m.}
\end{itemize}
\begin{itemize}
\item {Utilização:Bras}
\end{itemize}
Espécie de pardal.
\section{Jacatá}
\begin{itemize}
\item {Grp. gram.:m.  e  adj.}
\end{itemize}
\begin{itemize}
\item {Utilização:Ant.}
\end{itemize}
O mesmo que \textunderscore japonês\textunderscore .
\section{Jacatirão}
\begin{itemize}
\item {Grp. gram.:m.}
\end{itemize}
Árvore melastomácea da América.
\section{Jacatupé}
\begin{itemize}
\item {Grp. gram.:m.}
\end{itemize}
\begin{itemize}
\item {Utilização:Bras}
\end{itemize}
Planta leguminosa, trepadeira e de raiz comestível.
(Talvez do tupi)
\section{Jácea}
\begin{itemize}
\item {Grp. gram.:f.}
\end{itemize}
Planta, da fam. das compostas, parecida á centáurea.
\section{Jacente}
\begin{itemize}
\item {Grp. gram.:adj.}
\end{itemize}
\begin{itemize}
\item {Grp. gram.:M.}
\end{itemize}
\begin{itemize}
\item {Grp. gram.:Pl.}
\end{itemize}
\begin{itemize}
\item {Proveniência:(Lat. \textunderscore jacens\textunderscore )}
\end{itemize}
Que jaz.
Diz-se da herança, constituída por bens, para que não há herdeiros e que portanto passa para o Estado.
Viga, que se assenta na construcção das pontes, no sentido longitudinal destas, e sobre a qual se fixam as travéssas do tabuleiro.
Cp. \textunderscore jazente\textunderscore .
Recifes.
\section{Jacer}
\begin{itemize}
\item {Grp. gram.:v. i.}
\end{itemize}
\begin{itemize}
\item {Utilização:Ant.}
\end{itemize}
O mesmo que \textunderscore jazer\textunderscore . Cf. \textunderscore Eufrosína\textunderscore , 141; \textunderscore Viriato Trág.\textunderscore , XII, 78.
\section{Jacerino}
\textunderscore adj.\textunderscore  (e der.)
O mesmo que \textunderscore jazerino\textunderscore , etc.
\section{Jaci}
\begin{itemize}
\item {Grp. gram.:m.}
\end{itemize}
\begin{itemize}
\item {Utilização:Bras}
\end{itemize}
Espécie de palmeira.
\section{Jaciaba}
\begin{itemize}
\item {Grp. gram.:f.}
\end{itemize}
O mesmo que \textunderscore uauíra\textunderscore .
\section{Jaciná}
\begin{itemize}
\item {Grp. gram.:m.}
\end{itemize}
\begin{itemize}
\item {Utilização:Bras}
\end{itemize}
Espécie de borboleta.
\section{Jacintho}
\begin{itemize}
\item {Grp. gram.:m.}
\end{itemize}
\begin{itemize}
\item {Proveniência:(Do lat. \textunderscore hyacinthus\textunderscore )}
\end{itemize}
Gênero de plantas liliáceas, de que há várias espécies.
Pedra fina e variegada.
\section{Jacintho}
\begin{itemize}
\item {Grp. gram.:adj.}
\end{itemize}
Relativo ao jacintho; que tem a côr do jacintho.
\section{Jacinto}
\begin{itemize}
\item {Grp. gram.:m.}
\end{itemize}
\begin{itemize}
\item {Proveniência:(Do lat. \textunderscore hyacinthus\textunderscore )}
\end{itemize}
Gênero de plantas liliáceas, de que há várias espécies.
Pedra fina e variegada.
\section{Jacinto}
\begin{itemize}
\item {Grp. gram.:adj.}
\end{itemize}
Relativo ao jacinto; que tem a côr do jacinto.
\section{Jacitara}
\begin{itemize}
\item {Grp. gram.:f.}
\end{itemize}
O mesmo que \textunderscore titara\textunderscore .
Nome de várias espécies de palmeiras.
\section{Jacitata}
\begin{itemize}
\item {Grp. gram.:f.}
\end{itemize}
(V.jacitara)
\section{Jacksónia}
\begin{itemize}
\item {Grp. gram.:f.}
\end{itemize}
\begin{itemize}
\item {Proveniência:(De \textunderscore Jackson\textunderscore , n. p.)}
\end{itemize}
Gênero de plantas leguminosas.
\section{Jaco}
\begin{itemize}
\item {Grp. gram.:m.}
\end{itemize}
O mesmo que \textunderscore jaque\textunderscore .
\section{Jacob}
\begin{itemize}
\item {Grp. gram.:m.}
\end{itemize}
\begin{itemize}
\item {Proveniência:(De \textunderscore Jacob\textunderscore , n. p.)}
\end{itemize}
Nome, com que algumas vezes se designa Israel ou o povo hebreu.
\section{Jacobéa}
\begin{itemize}
\item {Grp. gram.:f.}
\end{itemize}
\begin{itemize}
\item {Proveniência:(Do lat. \textunderscore Jacobus\textunderscore , n. p.)}
\end{itemize}
Planta synanthéra, espécie de cardo.
\section{Jacobéa}
\begin{itemize}
\item {Grp. gram.:f.}
\end{itemize}
Seita dos jacobeus.(V.jacobeu)
\section{Jacobeia}
\begin{itemize}
\item {Grp. gram.:f.}
\end{itemize}
\begin{itemize}
\item {Proveniência:(Do lat. \textunderscore Jacobus\textunderscore , n. p.)}
\end{itemize}
Planta sinantéra, espécie de cardo.
\section{Jacobeia}
\begin{itemize}
\item {Grp. gram.:f.}
\end{itemize}
Seita dos jacobeus.(V.jacobeu)
\section{Jacobeu}
\begin{itemize}
\item {Grp. gram.:m.}
\end{itemize}
Membro de uma seita de fanáticos, política e religiosa, que principiou em tempos de D. João V. Cf. Latino, \textunderscore Hist. Pol. e Mil\textunderscore , I, 100.
\section{Jacobice}
\begin{itemize}
\item {Grp. gram.:f.}
\end{itemize}
Acção ou dito, próprio de jacobeu. Cf. Filinto, XIX, 10.
\section{Jacobínia}
\begin{itemize}
\item {Grp. gram.:f.}
\end{itemize}
Planta acanthácea.
\section{Jacobinismo}
\begin{itemize}
\item {Grp. gram.:m.}
\end{itemize}
\begin{itemize}
\item {Utilização:Ext.}
\end{itemize}
\begin{itemize}
\item {Proveniência:(De \textunderscore jacobino\textunderscore )}
\end{itemize}
Doutrina dos jacobinos.
Radicalismo, opiniões exaltadas ou revolucionárias.
\section{Jacobino}
\begin{itemize}
\item {Grp. gram.:m.}
\end{itemize}
\begin{itemize}
\item {Utilização:Ext.}
\end{itemize}
\begin{itemize}
\item {Proveniência:(Fr. \textunderscore jacobin\textunderscore )}
\end{itemize}
Membro de uma associação revolucionária, fundada em 1789, em Paris.
Partidário exaltado da democracia.
\section{Jacobitas}
\begin{itemize}
\item {Grp. gram.:m. pl.}
\end{itemize}
\begin{itemize}
\item {Proveniência:(De \textunderscore Jacob\textunderscore , n. p.)}
\end{itemize}
Seita religiosa do Oriente, que teve por chefe, no século VI, o bispo de Edessa, Jacob.
\section{Jacoba}
\begin{itemize}
\item {Grp. gram.:m.}
\end{itemize}
Antiga moéda inglesa. Cf. D. F. Manuel, \textunderscore Apólogos\textunderscore .
\section{Jacquemôncia}
\begin{itemize}
\item {Grp. gram.:f.}
\end{itemize}
\begin{itemize}
\item {Proveniência:(De \textunderscore Jacquemont\textunderscore , n. p.)}
\end{itemize}
Planta convolvulácea.
\section{Jacqueria}
\begin{itemize}
\item {Grp. gram.:f.}
\end{itemize}
\begin{itemize}
\item {Proveniência:(Fr. \textunderscore jacquerie\textunderscore )}
\end{itemize}
Revolta dos camponeses contra os nobres, em França, meiado o século XIV.
\section{Jacquínia}
\begin{itemize}
\item {Grp. gram.:f.}
\end{itemize}
\begin{itemize}
\item {Proveniência:(De \textunderscore Jacquin\textunderscore , n. p.)}
\end{itemize}
Gênero de plantas myrsíneas.
\section{Jacra}
\begin{itemize}
\item {Grp. gram.:f.}
\end{itemize}
O mesmo que \textunderscore jágara\textunderscore .
\section{Jacre}
\begin{itemize}
\item {Grp. gram.:m.}
\end{itemize}
O mesmo que \textunderscore jágara\textunderscore .
\section{Jactação}
\begin{itemize}
\item {Grp. gram.:f.}
\end{itemize}
\begin{itemize}
\item {Utilização:Ant.}
\end{itemize}
\begin{itemize}
\item {Proveniência:(Lat. \textunderscore jactatio\textunderscore )}
\end{itemize}
Desordenada agitação de membros, produzida por perturbação nervosa.
\section{Jactância}
\begin{itemize}
\item {Grp. gram.:f.}
\end{itemize}
\begin{itemize}
\item {Proveniência:(Lat. \textunderscore jactantia\textunderscore )}
\end{itemize}
Ostentação; vaidade; amôr próprio; arrogância.
\section{Jactanciar-se}
\begin{itemize}
\item {Grp. gram.:v. p.}
\end{itemize}
\begin{itemize}
\item {Proveniência:(De \textunderscore jactância\textunderscore )}
\end{itemize}
O mesmo que \textunderscore jactar-se\textunderscore . Cf. Cortesão, \textunderscore Subsídios\textunderscore .
\section{Jactanciosamente}
\begin{itemize}
\item {Grp. gram.:adv.}
\end{itemize}
De modo jactancioso; com vaidade.
\section{Jactancioso}
\begin{itemize}
\item {Grp. gram.:adj.}
\end{itemize}
Que tem jactância.
Vaidoso.
Orgulhoso; arrogante.
\section{Jactante}
\begin{itemize}
\item {Grp. gram.:adj.}
\end{itemize}
\begin{itemize}
\item {Utilização:Ant.}
\end{itemize}
\begin{itemize}
\item {Proveniência:(Lat. \textunderscore jactans\textunderscore )}
\end{itemize}
Que padece jactação.
O mesmo que \textunderscore jactancioso\textunderscore .
\section{Jactar-se}
\begin{itemize}
\item {Grp. gram.:v. p.}
\end{itemize}
\begin{itemize}
\item {Proveniência:(Lat. \textunderscore jactare\textunderscore )}
\end{itemize}
Têr jactância; ufanar-se; gloriar-se; vangloriar-se.
\section{Jacto}
\begin{itemize}
\item {Grp. gram.:m.}
\end{itemize}
\begin{itemize}
\item {Grp. gram.:Loc. adv.}
\end{itemize}
\begin{itemize}
\item {Proveniência:(Lat. \textunderscore jactus\textunderscore )}
\end{itemize}
Acto de arremessar; impulso; aquillo que se arremessa de uma vez.
Evacuação fecal.
Saída impetuosa do um líquido.
\textunderscore De um jacto\textunderscore , de uma só vez.
\section{Jactura}
\begin{itemize}
\item {Grp. gram.:f.}
\end{itemize}
\begin{itemize}
\item {Utilização:Des.}
\end{itemize}
\begin{itemize}
\item {Proveniência:(Lat. \textunderscore jactura\textunderscore )}
\end{itemize}
Perda; damno.
\section{Jacu}
\begin{itemize}
\item {Grp. gram.:m.}
\end{itemize}
Áve gallinácea, avermelhada, do Brasil.
\section{Jacua-acanga}
\begin{itemize}
\item {Grp. gram.:f.}
\end{itemize}
Planta borragínea do Brasil.
\section{Jacuba}
\begin{itemize}
\item {Grp. gram.:f.}
\end{itemize}
\begin{itemize}
\item {Utilização:Bras}
\end{itemize}
Bebida, preparada com água, farinha e açúcar.
(Do tupi \textunderscore jecuacuba\textunderscore ?)
\section{Jacu-guaçu}
\begin{itemize}
\item {Grp. gram.:m.}
\end{itemize}
Formosa gallinácea americana, de barbela rubra e cauda em leque, espécie de jacu.
\section{Jacuí}
\begin{itemize}
\item {Grp. gram.:m.}
\end{itemize}
Espécie de pequeno jacu.
\section{Jaculação}
\begin{itemize}
\item {Grp. gram.:f.}
\end{itemize}
\begin{itemize}
\item {Proveniência:(Lat. \textunderscore jaculatio\textunderscore )}
\end{itemize}
Tiro de artilharia.
Tiro.
O espaço vencido pelo tiro. Cf. Leoni, \textunderscore Diccion. de Artilh.\textunderscore , inédito.
\section{Jacular}
\textunderscore v. t.\textunderscore  (e der.)
(V. \textunderscore ejacular\textunderscore , etc.)
\section{Jaculatória}
\begin{itemize}
\item {Grp. gram.:f.}
\end{itemize}
\begin{itemize}
\item {Proveniência:(De \textunderscore jaculatório\textunderscore )}
\end{itemize}
Curta oração, que se diz nas novenas e noutras rezas.
\section{Jaculatório}
\begin{itemize}
\item {Grp. gram.:adj.}
\end{itemize}
\begin{itemize}
\item {Grp. gram.:M.}
\end{itemize}
\begin{itemize}
\item {Utilização:Fig.}
\end{itemize}
\begin{itemize}
\item {Proveniência:(Lat. \textunderscore jaculatorius\textunderscore )}
\end{itemize}
Que expede jactos; próprio para arremessar.
O mesmo que \textunderscore jaculatória\textunderscore .
\section{Jáculo}
\begin{itemize}
\item {Grp. gram.:m.}
\end{itemize}
\begin{itemize}
\item {Utilização:Des.}
\end{itemize}
\begin{itemize}
\item {Proveniência:(Lat. \textunderscore jaculum\textunderscore )}
\end{itemize}
Mammífero roedor.
Arremêsso, tiro.
\section{Jacuma}
\begin{itemize}
\item {Grp. gram.:f.}
\end{itemize}
\begin{itemize}
\item {Utilização:Bras}
\end{itemize}
Pá, usada em canôas, em que serve de leme.--Vejo escrito \textunderscore jacuma\textunderscore , mas provavelmente deveria escrever-se \textunderscore jacumá\textunderscore  = \textunderscore jacuman\textunderscore .(V.jacuman)
\section{Jacumaíba}
\begin{itemize}
\item {Grp. gram.:m.}
\end{itemize}
\begin{itemize}
\item {Utilização:Bras}
\end{itemize}
Piloto de canôa, em navegações arriscadas.
\section{Jacuman}
\begin{itemize}
\item {Grp. gram.:m.}
\end{itemize}
\begin{itemize}
\item {Utilização:Bras. do N}
\end{itemize}
Pequeno remo, que serve de leme nas canôas.
Popa de canôa.
Piloto de canôa.
\section{Jacumaúba}
\begin{itemize}
\item {Grp. gram.:f.}
\end{itemize}
\begin{itemize}
\item {Utilização:Bras}
\end{itemize}
O mesmo que \textunderscore jacumaíba\textunderscore .
\section{Jacunas}
\begin{itemize}
\item {Grp. gram.:m. pl.}
\end{itemize}
Índios selvagens das margens do Apaporis, no Brasil.
\section{Jacundá}
\begin{itemize}
\item {Grp. gram.:m.}
\end{itemize}
Peixe do norte do Brasil.
\section{Jacupemba}
\begin{itemize}
\item {Grp. gram.:f.}
\end{itemize}
Ave gallinácea do Brasil.
\section{Jacupéua}
\begin{itemize}
\item {Grp. gram.:f.}
\end{itemize}
\begin{itemize}
\item {Utilização:Bras}
\end{itemize}
Pequeno jacu.
\section{Jacuru}
\begin{itemize}
\item {Grp. gram.:m.}
\end{itemize}
Espécie de cobra do Brasil.
\section{Jacuruaru}
\begin{itemize}
\item {Grp. gram.:m.}
\end{itemize}
Planta rutácea do Brasil.
\section{Jacurutu}
\begin{itemize}
\item {Grp. gram.:m.}
\end{itemize}
\begin{itemize}
\item {Utilização:Bras}
\end{itemize}
Espécie de coruja.
\section{Jacutinga}
\begin{itemize}
\item {Grp. gram.:m.}
\end{itemize}
\begin{itemize}
\item {Utilização:Bras}
\end{itemize}
Ave gallinácea do Brasil, negra, com pennacho branco.
\section{Jacutinga}
\begin{itemize}
\item {Grp. gram.:f.}
\end{itemize}
\begin{itemize}
\item {Utilização:Bras}
\end{itemize}
Xisto ferruginoso e manganífero decomposto.
\section{Jade}
\begin{itemize}
\item {Grp. gram.:m.}
\end{itemize}
\begin{itemize}
\item {Proveniência:(Fr. \textunderscore jade\textunderscore )}
\end{itemize}
Pedra dura, que risca o vidro e até o quartzo, e que é um silicato de alumina e de cal.
\section{Jadeíta}
\begin{itemize}
\item {Grp. gram.:f.}
\end{itemize}
\begin{itemize}
\item {Utilização:Miner.}
\end{itemize}
Silicato de alumina e sódio, com cal, magnésia e protóxydo de ferro.
\section{Jadeíte}
\begin{itemize}
\item {Grp. gram.:f.}
\end{itemize}
\begin{itemize}
\item {Utilização:Miner.}
\end{itemize}
Silicato de alumina e sódio, com cal, magnésia e protóxydo de ferro.
\section{Jaez}
\begin{itemize}
\item {Grp. gram.:m.}
\end{itemize}
\begin{itemize}
\item {Utilização:Fig.}
\end{itemize}
\begin{itemize}
\item {Proveniência:(Do ár. \textunderscore jahez\textunderscore )}
\end{itemize}
Apparelho de cavalgadura, ou de animaes que pódem apparelhar-se como cavallos.
Qualidade, gênero; índole: \textunderscore e outros patifes de igual jaez\textunderscore .
\section{Jaezar}
\begin{itemize}
\item {Grp. gram.:v. t.}
\end{itemize}
(V.ajaezar)
\section{Jaga}
\begin{itemize}
\item {Grp. gram.:m.}
\end{itemize}
Chefe electivo dos Bângalas de Cassange.
(Provavelmente, de \textunderscore jagas\textunderscore )
\section{Jagado}
\begin{itemize}
\item {Grp. gram.:m.}
\end{itemize}
Território, governado por um jaga.
Govêrno de um jaga.
Jurisdicção de um jaga.
\section{Jagaque}
\begin{itemize}
\item {Grp. gram.:m.}
\end{itemize}
Espécie de peixe das costas do Brasil.
\section{Jágara}
\begin{itemize}
\item {Grp. gram.:f.}
\end{itemize}
Nome, que se dava na Índia portuguesa, e ainda se dá na África oriental, ao açúcar feito de coqueiro e de cana.
\section{Jagas}
\begin{itemize}
\item {Grp. gram.:m. pl.}
\end{itemize}
Antigo povo africano, que conquistou o Congo e foi rechaçado e dispersado pelos Portugueses, no século XVI.
\section{Jagodes}
\begin{itemize}
\item {Grp. gram.:m.}
\end{itemize}
\begin{itemize}
\item {Utilização:Pop.}
\end{itemize}
Homém ordinário, sem crédito.
Pandilha; troca-tintas.
(Talvez alter, de \textunderscore zégodes\textunderscore . Cp. \textunderscore zé-godes\textunderscore )
\section{Jagoirana}
\begin{itemize}
\item {Grp. gram.:f.}
\end{itemize}
Árvore leguminosa do Brasil.
\section{Jagomeiro}
\begin{itemize}
\item {Grp. gram.:m.}
\end{itemize}
Árvore indiana, o mesmo que \textunderscore jangomas\textunderscore . Cf. Garcia Orta, \textunderscore Coll.\textunderscore , XXVIII.
\section{Jagonça}
\begin{itemize}
\item {Grp. gram.:f.}
\end{itemize}
\begin{itemize}
\item {Utilização:Ant.}
\end{itemize}
Pedra preciosa.
\section{Jagra}
\begin{itemize}
\item {Grp. gram.:f.}
\end{itemize}
O mesmo que \textunderscore jágara\textunderscore .
\section{Jagre}
\begin{itemize}
\item {Grp. gram.:m.}
\end{itemize}
O mesmo que \textunderscore jágara\textunderscore .
\section{Jaguacati-guaçu}
\begin{itemize}
\item {Grp. gram.:m.}
\end{itemize}
Ave do Brasil, espécie de pica-peixe.
\section{Jaguané}
\begin{itemize}
\item {Grp. gram.:m.}
\end{itemize}
Cão bravio e pequeno do Brasil.
\section{Jaguané}
\begin{itemize}
\item {Grp. gram.:adj.}
\end{itemize}
\begin{itemize}
\item {Utilização:Bras. do S}
\end{itemize}
Diz-se do boi ou vaca, que tem branco o fio do lombo, preto ou vermelho o lado das costellas e de ordinário branca a barriga.
\section{Jaguapeba}
\begin{itemize}
\item {Grp. gram.:m.}
\end{itemize}
\begin{itemize}
\item {Utilização:Bras}
\end{itemize}
Espécie de pequeno cão de pernas curtas.
(Do tupi \textunderscore jagua\textunderscore  + \textunderscore peba\textunderscore )
\section{Jaguar}
\begin{itemize}
\item {Grp. gram.:m.}
\end{itemize}
Ferocíssimo quadrúpede felino, de pelle mosqueada como a do leopardo e da panthera.
Onça.
(Do tupi)
\section{Jaguara}
\begin{itemize}
\item {Grp. gram.:m.}
\end{itemize}
\begin{itemize}
\item {Utilização:Bras}
\end{itemize}
Cão.
Designação de outros mammiferos.
Designação, que alguns naturalistas deram ao jaguar.
\section{Jaguaratirica}
\begin{itemize}
\item {Grp. gram.:m.}
\end{itemize}
Espécie de cão bravio do Brasil.
\section{Jaguaré}
\begin{itemize}
\item {Grp. gram.:m.}
\end{itemize}
\begin{itemize}
\item {Utilização:Bras}
\end{itemize}
Espécie de forragem.
\section{Jaguarete}
\begin{itemize}
\item {fónica:guarê}
\end{itemize}
\begin{itemize}
\item {Grp. gram.:m.}
\end{itemize}
Pequeno jaguar.
\section{Jaguareté}
\begin{itemize}
\item {Grp. gram.:m.}
\end{itemize}
Ave do Brasil.
\section{Jaguaruanas}
\begin{itemize}
\item {Grp. gram.:m. pl.}
\end{itemize}
\begin{itemize}
\item {Utilização:Bras}
\end{itemize}
Tríbo de aborígenes do Ceará.
\section{Jagudi}
\begin{itemize}
\item {Grp. gram.:m.}
\end{itemize}
Espécie de falcão da África Occidental.
\section{Jague-jaga-mamona}
\begin{itemize}
\item {Grp. gram.:f.}
\end{itemize}
Árvore da Guiné, de fôlhas medicinaes.
\section{Jagunço}
\begin{itemize}
\item {Grp. gram.:m.}
\end{itemize}
\begin{itemize}
\item {Utilização:Bras}
\end{itemize}
Valentão; guarda-costas; capanga.
\section{Jagunda}
\begin{itemize}
\item {Grp. gram.:f.}
\end{itemize}
Antiga e pequena medida portuguesa.
\section{Jahicós}
\begin{itemize}
\item {Grp. gram.:m. pl.}
\end{itemize}
Extinta tríbo de Índios brasileiros, em Piauí.
\section{Jaicós}
\begin{itemize}
\item {fónica:ja-i}
\end{itemize}
\begin{itemize}
\item {Grp. gram.:m. pl.}
\end{itemize}
Extinta tribo de Índios brasileiros, em Piauí.
\section{Jaja}
\begin{itemize}
\item {Grp. gram.:f.}
\end{itemize}
\begin{itemize}
\item {Utilização:Infant.}
\end{itemize}
Fato de crianças.
(Colhido na Beira-Baixa)
\section{Jaja}
\begin{itemize}
\item {Grp. gram.:f.}
\end{itemize}
\begin{itemize}
\item {Utilização:Mad}
\end{itemize}
Mossa, amolgadura.
\section{Jalão}
\begin{itemize}
\item {Grp. gram.:m.}
\end{itemize}
Haste de madeira, de dois a três metros e com ponteira de ferro, para alinhamentos e agrimensura.
\section{Jalapa}
\begin{itemize}
\item {Grp. gram.:f.}
\end{itemize}
\begin{itemize}
\item {Proveniência:(De \textunderscore Xalapa\textunderscore , n. p. de uma cidade mexicana)}
\end{itemize}
Nome de várias plantas convolvuáceas, cuja raiz é purgativa.
\section{Jalapa}
\begin{itemize}
\item {Grp. gram.:f.}
\end{itemize}
\begin{itemize}
\item {Utilização:T. da Bairrada}
\end{itemize}
O mesmo que \textunderscore zurrapa\textunderscore .
\section{Jalapão}
\begin{itemize}
\item {Grp. gram.:m.}
\end{itemize}
\begin{itemize}
\item {Utilização:Bras}
\end{itemize}
\begin{itemize}
\item {Proveniência:(De \textunderscore jalapa\textunderscore )}
\end{itemize}
Planta e raiz medicinaes, o mesmo que \textunderscore tiú\textunderscore .
\section{Jalapeiro}
\begin{itemize}
\item {Grp. gram.:m.}
\end{itemize}
Medicamento purgativo de jalapa. Cf. Macedo, \textunderscore Burros\textunderscore , 240.
\section{Jalápico}
\begin{itemize}
\item {Grp. gram.:adj.}
\end{itemize}
\begin{itemize}
\item {Proveniência:(De \textunderscore jalapa\textunderscore )}
\end{itemize}
Diz-se de um ácido, resultante da hydratação da jalapina pelos álcalis.
\section{Jalapina}
\begin{itemize}
\item {Grp. gram.:f.}
\end{itemize}
Substância de certa resina que se extrai da raiz e da haste da jalapa.
\section{Jalapinha}
\begin{itemize}
\item {Grp. gram.:f.}
\end{itemize}
Espécie de jalapa.
\section{Jalapinol}
\begin{itemize}
\item {Grp. gram.:m.}
\end{itemize}
Producto chímico, extrahido da jalapina.
\section{Jalde}
\begin{itemize}
\item {Grp. gram.:adj.}
\end{itemize}
O mesmo que \textunderscore jalne\textunderscore .
\section{Jaldeta}
\begin{itemize}
\item {fónica:dê}
\end{itemize}
\begin{itemize}
\item {Grp. gram.:f.}
\end{itemize}
\begin{itemize}
\item {Utilização:Ant.}
\end{itemize}
\begin{itemize}
\item {Proveniência:(De \textunderscore jalde\textunderscore ?)}
\end{itemize}
Espécie de jôgo prohibido.
\section{Jaldete}
\begin{itemize}
\item {fónica:dê}
\end{itemize}
\begin{itemize}
\item {Grp. gram.:m.}
\end{itemize}
\begin{itemize}
\item {Utilização:Ant.}
\end{itemize}
\begin{itemize}
\item {Proveniência:(De \textunderscore jalde\textunderscore ?)}
\end{itemize}
Espécie de jôgo prohibido.
\section{Jaldinino}
\begin{itemize}
\item {Grp. gram.:adj.}
\end{itemize}
\begin{itemize}
\item {Proveniência:(Do rad. de \textunderscore jalde\textunderscore )}
\end{itemize}
Que tem côr jalde.
\section{Jaléa}
\begin{itemize}
\item {Grp. gram.:f.}
\end{itemize}
Embarcação asiática.
\section{Jaleca}
\begin{itemize}
\item {Grp. gram.:f.}
\end{itemize}
O mesmo que \textunderscore jaqueta\textunderscore .
(Cp. \textunderscore jaleco\textunderscore )
\section{Jaleco}
\begin{itemize}
\item {Grp. gram.:m.}
\end{itemize}
Casaco curto, semelhante á jaqueta.
Fardeta.
(Ár. \textunderscore ielec\textunderscore , do turco)
\section{Jaleia}
\begin{itemize}
\item {Grp. gram.:f.}
\end{itemize}
Embarcação asiática.
\section{Jalne}
\begin{itemize}
\item {Grp. gram.:adj.}
\end{itemize}
\begin{itemize}
\item {Proveniência:(Fr. ant. \textunderscore jalne\textunderscore  = mod. \textunderscore jaune\textunderscore )}
\end{itemize}
Amarelo da côr do oiro.
\section{Jalofo}
\begin{itemize}
\item {Grp. gram.:m.}
\end{itemize}
\begin{itemize}
\item {Utilização:Fig.}
\end{itemize}
\begin{itemize}
\item {Grp. gram.:Pl.}
\end{itemize}
\begin{itemize}
\item {Proveniência:(De \textunderscore Jalof\textunderscore , n. p. afr.)}
\end{itemize}
Indivíduo de uma tríbo gentílica da África occidental.
Homem rude, grosseiro.
Tríbo gentílica da África occidental.
\section{Jalusia}
\begin{itemize}
\item {Grp. gram.:f.}
\end{itemize}
O mesmo que \textunderscore gelosia\textunderscore . Cf. Castilho, \textunderscore Escavações Poét.\textunderscore , 90.
\section{Jaluto}
\begin{itemize}
\item {Grp. gram.:m.}
\end{itemize}
Espécie de peixe da costa occidental da África.
\section{Jamacaí}
\begin{itemize}
\item {Grp. gram.:m.}
\end{itemize}
Pássaro brasileiro, que destrói as lagartas, como o japu.
\section{Jamacaru}
\begin{itemize}
\item {Grp. gram.:m.}
\end{itemize}
O mesmo que \textunderscore cumbeba\textunderscore .
\section{Jamaicano}
\begin{itemize}
\item {Grp. gram.:adj.}
\end{itemize}
\begin{itemize}
\item {Grp. gram.:M.}
\end{itemize}
Relativo á Jamaica.
Habitante da Jamaica.
\section{Jàmais}
\begin{itemize}
\item {Grp. gram.:adv.}
\end{itemize}
\begin{itemize}
\item {Utilização:Pop.}
\end{itemize}
\begin{itemize}
\item {Proveniência:(De \textunderscore já\textunderscore  + \textunderscore mais\textunderscore )}
\end{itemize}
Nunca; em nenhum tempo.
Principalmente.
\section{Jamanta}
\begin{itemize}
\item {Grp. gram.:f.}
\end{itemize}
Nome, dado pelos pescadores á raia grande.
\section{Jamanta}
\begin{itemize}
\item {Grp. gram.:m.}
\end{itemize}
\begin{itemize}
\item {Utilização:Bras}
\end{itemize}
Homemzarrão desajeitado.
Calçado caseiro.
\section{Jamaracaú}
\begin{itemize}
\item {Grp. gram.:m.}
\end{itemize}
\begin{itemize}
\item {Utilização:Bras}
\end{itemize}
Espécie de mandacaru.
\section{Jamaris}
\begin{itemize}
\item {Grp. gram.:m. pl.}
\end{itemize}
Índios do Amazonas, nas margens do Jamari.
\section{Jamaru}
\begin{itemize}
\item {Grp. gram.:m.}
\end{itemize}
\begin{itemize}
\item {Utilização:Bras. do N}
\end{itemize}
Grande planta cucurbitácea, que se prepara para servir de vasilha de água.
\section{Jamaxi}
\begin{itemize}
\item {Grp. gram.:m.}
\end{itemize}
\begin{itemize}
\item {Utilização:Bras}
\end{itemize}
Espécie de paneiro, feito de timbó, e em que os seringueiros transportam suas mercadorias.
\section{Jamba}
\begin{itemize}
\item {Grp. gram.:f.}
\end{itemize}
\begin{itemize}
\item {Utilização:Prov.}
\end{itemize}
\begin{itemize}
\item {Utilização:trasm.}
\end{itemize}
Empenho.
\section{Jambalueiro}
\begin{itemize}
\item {Grp. gram.:m.}
\end{itemize}
Árvore de Moçambique.--Vejo a palavra no museu da \textunderscore Sociedade de Geographia\textunderscore , mas supponho-a corruptela de \textunderscore jamboleiro\textunderscore .
\section{Jambatuto}
\begin{itemize}
\item {Grp. gram.:m.}
\end{itemize}
Ave africana, de olhos encarnados e voz vibrante.
\section{Jambé}
\begin{itemize}
\item {Grp. gram.:m.}
\end{itemize}
\begin{itemize}
\item {Utilização:Bras}
\end{itemize}
Iguaria, que se faz com o fruto do caruru.
\section{Jambeiro}
\begin{itemize}
\item {Grp. gram.:m.}
\end{itemize}
\begin{itemize}
\item {Proveniência:(De \textunderscore jambo\textunderscore ^2)}
\end{itemize}
Árvore myrtácea da Índia e do Brasil.
\section{Jambelo}
\begin{itemize}
\item {Grp. gram.:m.}
\end{itemize}
\begin{itemize}
\item {Utilização:Prov.}
\end{itemize}
\begin{itemize}
\item {Utilização:trasm.}
\end{itemize}
Presunto pequeno.
(Cp. fr. \textunderscore jambe\textunderscore )
\section{Jâmbico}
\begin{itemize}
\item {Grp. gram.:adj.}
\end{itemize}
\begin{itemize}
\item {Proveniência:(Lat. \textunderscore iambicus\textunderscore )}
\end{itemize}
Relativo ao jambo^1.
\section{Jambo}
\begin{itemize}
\item {Grp. gram.:m.}
\end{itemize}
\begin{itemize}
\item {Proveniência:(Lat. \textunderscore iambus\textunderscore )}
\end{itemize}
Pé de verso, composto de duas sýllabas, a primeira breve e a segunda longa.
Vérso jâmbico.
\section{Jambo}
\begin{itemize}
\item {Grp. gram.:m.}
\end{itemize}
\begin{itemize}
\item {Proveniência:(Do mal. \textunderscore djanibu\textunderscore )}
\end{itemize}
Fruto do jambeiro.
O mesmo que \textunderscore jambeiro\textunderscore .
\section{Jambó}
\begin{itemize}
\item {Grp. gram.:m.}
\end{itemize}
Árvore da Índia e da África, (\textunderscore inga xilocarpa\textunderscore ).
O mesmo que \textunderscore jambol\textunderscore ?
\section{Jambôa}
\begin{itemize}
\item {Grp. gram.:f.}
\end{itemize}
\begin{itemize}
\item {Utilização:T. de Macau}
\end{itemize}
O mesmo que \textunderscore toranja\textunderscore .
\section{Jamboeiro}
\begin{itemize}
\item {Grp. gram.:m.}
\end{itemize}
O mesmo que \textunderscore jambeiro\textunderscore .
\section{Jambol}
\begin{itemize}
\item {Grp. gram.:m.}
\end{itemize}
\begin{itemize}
\item {Proveniência:(Do conc. \textunderscore jambul\textunderscore )}
\end{itemize}
Árvore indiana, de madeira avermelhada e flexível, provavelmente o mesmo que \textunderscore jamboleiro\textunderscore . Cf. Lopes Mendes, \textunderscore Índia Port.\textunderscore 
\section{Jambolano}
\begin{itemize}
\item {Grp. gram.:m.}
\end{itemize}
O mesmo que \textunderscore jambolão\textunderscore . Cf. \textunderscore Jorn. do Comm.\textunderscore , do Rio, de 4-XII-87.
\section{Jambolão}
\begin{itemize}
\item {Grp. gram.:m.}
\end{itemize}
Árvore fructífera do Brasil e da Índia portuguesa, (\textunderscore eugenia jambolanda\textunderscore , ou \textunderscore cizigium jambolanum\textunderscore ). Cf. Garcia Orta, \textunderscore Coll.\textunderscore , XXVIII.
\section{Jamboleiro}
\begin{itemize}
\item {Grp. gram.:m.}
\end{itemize}
O mesmo que \textunderscore jambolão\textunderscore . Cf. Garcia Orta, \textunderscore Coll.\textunderscore , XXVIII.
\section{Jambosa}
\begin{itemize}
\item {Grp. gram.:f.}
\end{itemize}
Gênero de plantas myrtáceas.
\section{Jambu}
\begin{itemize}
\item {Grp. gram.:m.}
\end{itemize}
Planta alimentícia do Brasil e da Índia, (\textunderscore spilantes oleracea\textunderscore , Lin.).--Chama-se também \textunderscore agrião do Pará\textunderscore .
\section{Jambuaçu}
\begin{itemize}
\item {Grp. gram.:m.}
\end{itemize}
Espécie de jambu.
\section{Jambul}
\begin{itemize}
\item {Grp. gram.:m.}
\end{itemize}
O mesmo ou melhor que \textunderscore jambol\textunderscore .
\section{Jamburana}
\begin{itemize}
\item {Grp. gram.:f.}
\end{itemize}
Espécie de jambu.
\section{Jamegão}
\begin{itemize}
\item {Grp. gram.:m.}
\end{itemize}
\begin{itemize}
\item {Utilização:Bras. do N}
\end{itemize}
Assinatura; firma.
Rubrica.
(Provavelmente, da soletração da sýllaba \textunderscore gam\textunderscore : \textunderscore gê\textunderscore , \textunderscore a\textunderscore , \textunderscore m\textunderscore  = \textunderscore gão\textunderscore )
\section{Jamelão}
\begin{itemize}
\item {Grp. gram.:m.}
\end{itemize}
Planta ornamental do Brasil. Cf. \textunderscore Jorn. do Comm.\textunderscore , do Rio, de 23-V-902.
\section{Jamésia}
\begin{itemize}
\item {Grp. gram.:f.}
\end{itemize}
\begin{itemize}
\item {Proveniência:(De \textunderscore James\textunderscore , n. p.)}
\end{itemize}
Planta saxifragácea.
\section{Jâmi}
\begin{itemize}
\item {Grp. gram.:m.}
\end{itemize}
\begin{itemize}
\item {Utilização:Ant.}
\end{itemize}
O mesmo que \textunderscore aljama\textunderscore .
\section{Jampal}
\begin{itemize}
\item {Grp. gram.:m.}
\end{itemize}
Casta de uva extremenha. Cf. F. Lapa, \textunderscore Proc. de Vin.\textunderscore , 63.
\section{Jampaulo}
\begin{itemize}
\item {Grp. gram.:m.}
\end{itemize}
Casta de uva extremenha. Cf. F. Lapa, \textunderscore Proc. de Vin.\textunderscore , 63.
\section{Jamsónia}
\begin{itemize}
\item {Grp. gram.:f.}
\end{itemize}
\begin{itemize}
\item {Proveniência:(De \textunderscore Jamson\textunderscore , n. p.)}
\end{itemize}
Planta polypodiácea.
\section{Janaca}
\begin{itemize}
\item {Grp. gram.:m.}
\end{itemize}
Quadrúpede africano.
\section{Janal}
\begin{itemize}
\item {Grp. gram.:adj.}
\end{itemize}
\begin{itemize}
\item {Proveniência:(Lat. \textunderscore janalis\textunderscore )}
\end{itemize}
Relativo a Jano.
\section{Janambá}
\begin{itemize}
\item {Grp. gram.:f.}
\end{itemize}
Árvore silvestre do Brasil.
\section{Janapucá}
\begin{itemize}
\item {Grp. gram.:m.}
\end{itemize}
(V. \textunderscore puçá\textunderscore ^1)
\section{Janari}
\begin{itemize}
\item {Grp. gram.:m.}
\end{itemize}
Árvore brasileira, da região do Amazonas.
\section{Janaúba}
\begin{itemize}
\item {Grp. gram.:f.}
\end{itemize}
Árvore silvestre do Brasil.
\section{Janaui}
\begin{itemize}
\item {Grp. gram.:m.}
\end{itemize}
\begin{itemize}
\item {Utilização:Bras. do Amazonas}
\end{itemize}
Pequeno cão selvagem, destro e perigoso.
\section{Janda}
\begin{itemize}
\item {Grp. gram.:f.}
\end{itemize}
Áve palmípede do Brasil.
\section{Jan-da-cruz}
\begin{itemize}
\item {Grp. gram.:m.}
\end{itemize}
\begin{itemize}
\item {Utilização:Gír.}
\end{itemize}
O mesmo que \textunderscore dinheiro\textunderscore .
\section{Jandaia}
\begin{itemize}
\item {Grp. gram.:f.}
\end{itemize}
Áve do Brasil.
\section{Jandaíra}
\begin{itemize}
\item {Grp. gram.:f.}
\end{itemize}
Espécie de abelha brasileira.
\section{Jandia}
\begin{itemize}
\item {Grp. gram.:f.}
\end{itemize}
Peixe do norte do Brasil.
\section{Jandiparana}
\begin{itemize}
\item {Grp. gram.:f.}
\end{itemize}
(V.japaranduba)
\section{Jandiroba}
\begin{itemize}
\item {Grp. gram.:f.}
\end{itemize}
Planta cucurbitácea e trepadeira da América do Sul.
\section{Jando}
\begin{itemize}
\item {Grp. gram.:m.}
\end{itemize}
\begin{itemize}
\item {Utilização:Ant.}
\end{itemize}
O mesmo que \textunderscore gente\textunderscore .
\section{Janeanes}
\begin{itemize}
\item {Grp. gram.:f.  e  adj.}
\end{itemize}
\begin{itemize}
\item {Grp. gram.:M.}
\end{itemize}
\begin{itemize}
\item {Utilização:Ant.}
\end{itemize}
\begin{itemize}
\item {Proveniência:(De \textunderscore João\textunderscore  + \textunderscore Eanes\textunderscore , n. p.)}
\end{itemize}
Diz-se de uma espécie de uva.
Jagodes; joão-ninguém.
\section{Janeiradas}
\begin{itemize}
\item {Grp. gram.:f. pl.}
\end{itemize}
\begin{itemize}
\item {Utilização:Prov.}
\end{itemize}
\begin{itemize}
\item {Utilização:alent.}
\end{itemize}
Excursões de carácter religioso, feitas por indivíduos que percorrem várias localidades, vestidos de opas pardas, com violas e pandeiros, tocando, cantando e pedindo para as almas.
(Cp. \textunderscore janeiras\textunderscore )
\section{Janeiras}
\begin{itemize}
\item {Grp. gram.:f. pl.}
\end{itemize}
Cantigas populares do princípio do anno.
Bôas-festas, presentes de anno-bom.
Nome popular de algumas plantas, cujas flôres se abrem em Janeiro.
\section{Janeireiro}
\begin{itemize}
\item {Grp. gram.:m.}
\end{itemize}
\begin{itemize}
\item {Grp. gram.:Adj.}
\end{itemize}
Cantador de janeiras.
Aquelle que dá bôas-festas ou presentes de anno-bom.
Relativo a Janeiro: \textunderscore luar janeireiro\textunderscore .
\section{Janeirento}
\begin{itemize}
\item {Grp. gram.:adj.}
\end{itemize}
Diz-se do gato, que em janeiro anda com cio.
\section{Janeirinho}
\begin{itemize}
\item {Grp. gram.:adj.}
\end{itemize}
Relativo a Janeiro.
Que nasceu em Janeiro, ou que só apparece ou se cria em Janeiro.
\section{Janeiro}
\begin{itemize}
\item {Grp. gram.:m.}
\end{itemize}
\begin{itemize}
\item {Utilização:Pop.}
\end{itemize}
\begin{itemize}
\item {Grp. gram.:Adj.}
\end{itemize}
\begin{itemize}
\item {Grp. gram.:M. pl.}
\end{itemize}
\begin{itemize}
\item {Proveniência:(Do lat. \textunderscore januarius\textunderscore )}
\end{itemize}
Primeiro mês do anno, segundo a chronologia moderna e entre os antigos Romanos.
O cio dos gatos.
O mesmo que \textunderscore durázio\textunderscore .
Annos de idade: \textunderscore já conta sessenta janeiros\textunderscore .
\section{Janela}
\begin{itemize}
\item {Grp. gram.:f.}
\end{itemize}
\begin{itemize}
\item {Utilização:Fam.}
\end{itemize}
\begin{itemize}
\item {Grp. gram.:Pl.}
\end{itemize}
\begin{itemize}
\item {Utilização:Pop.}
\end{itemize}
Abertura nas paredes dos edifícios, para deixar entrar nelles o ar e a luz.
A porta, com que se fecha essa abertura.
Rasgão, buraco.
Olhos.
(Por \textunderscore januela\textunderscore , dem. hypoth., do lat. \textunderscore janua\textunderscore , porta)
\section{Janelar}
\begin{itemize}
\item {Grp. gram.:v. i.}
\end{itemize}
\begin{itemize}
\item {Utilização:Fam.}
\end{itemize}
Estar á janela, habitualmente. Cf. Eça, \textunderscore P. Basílio\textunderscore , 94.
\section{Janeleira}
\begin{itemize}
\item {Grp. gram.:f.}
\end{itemize}
\begin{itemize}
\item {Proveniência:(De \textunderscore janeleiro\textunderscore )}
\end{itemize}
Mulher namoradeira ou que gosta muito de estar á janela.
\section{Janeleiro}
\begin{itemize}
\item {Grp. gram.:m.  e  adj.}
\end{itemize}
Aquelle que gosta muito de estar á janela.
\section{Janelo}
\begin{itemize}
\item {fónica:nê}
\end{itemize}
\begin{itemize}
\item {Grp. gram.:m.}
\end{itemize}
\begin{itemize}
\item {Utilização:Prov.}
\end{itemize}
Pequena janela; postigo.
\section{Janga}
\begin{itemize}
\item {Grp. gram.:f.}
\end{itemize}
\begin{itemize}
\item {Proveniência:(T. de or. pracrítica?)}
\end{itemize}
Antiga e pequena embarcação de remos. Cf. \textunderscore Peregrinação\textunderscore , CIV.
\section{Jangada}
\begin{itemize}
\item {Grp. gram.:f.}
\end{itemize}
\begin{itemize}
\item {Utilização:Bras}
\end{itemize}
Armação, feita com madeiras de um navio, para recolher náufragos e quaesquer objectos, em occasião de naufrágio.
Série do embarcações chatas, ligadas umas ás outras.
Ligeira construcção, em fórma de grade, que serve para transportes por mar ou rio.
Caranguejola.
Árvore silvestre, de pêso insignificante e que por isso convém para a construcção de jangadas.
(Do tâmil \textunderscore xangadam\textunderscore )
\section{Jangadeira}
\begin{itemize}
\item {Grp. gram.:f.}
\end{itemize}
Árvore tiliácea do Brasil.
\section{Jangadeiro}
\begin{itemize}
\item {Grp. gram.:m.}
\end{itemize}
\begin{itemize}
\item {Utilização:Bras}
\end{itemize}
Dono de jangada.
Aquelle que dirige uma jangada.
Barco de pesca, espécie de jangada, com um mastro e uma vela.
\section{Jangalamaste}
\begin{itemize}
\item {Grp. gram.:m.}
\end{itemize}
\begin{itemize}
\item {Utilização:Bras}
\end{itemize}
Brincadeira, também chamada \textunderscore arreburrinho\textunderscore  ou \textunderscore gangorra\textunderscore .
\section{Jangaz}
\begin{itemize}
\item {Grp. gram.:m.}
\end{itemize}
\begin{itemize}
\item {Utilização:Chul.}
\end{itemize}
Trangalhadanças; homem desajeitado.
\section{Jangomas}
\begin{itemize}
\item {Grp. gram.:m.}
\end{itemize}
Árvore fructífera da Índia portuguesa, (\textunderscore flacourtia cataphracta\textunderscore , Roxn.). Cf. Garcia Orta, \textunderscore Coll.\textunderscore , XXVIII.
\section{Jangoto}
\begin{itemize}
\item {fónica:gô}
\end{itemize}
\begin{itemize}
\item {Grp. gram.:m.}
\end{itemize}
\begin{itemize}
\item {Utilização:Prov.}
\end{itemize}
\begin{itemize}
\item {Utilização:beir.}
\end{itemize}
(Corr. de \textunderscore jingoto\textunderscore )
\section{Jangué}
\begin{itemize}
\item {Grp. gram.:m.}
\end{itemize}
\begin{itemize}
\item {Utilização:Prov.}
\end{itemize}
\begin{itemize}
\item {Utilização:trasm.}
\end{itemize}
Indivíduo reles.
(Abrev. de \textunderscore joão-ninguém\textunderscore )
\section{Janguista}
\begin{itemize}
\item {Grp. gram.:adj. f.}
\end{itemize}
\begin{itemize}
\item {Utilização:Prov.}
\end{itemize}
\begin{itemize}
\item {Utilização:trasm.}
\end{itemize}
Mulher ou rapariga vestida com esmêro, ajanotada.
\section{Janíçaro}
\begin{itemize}
\item {Grp. gram.:m.}
\end{itemize}
\begin{itemize}
\item {Utilização:Fig.}
\end{itemize}
\begin{itemize}
\item {Utilização:Gír.}
\end{itemize}
\begin{itemize}
\item {Grp. gram.:Pl.}
\end{itemize}
\begin{itemize}
\item {Utilização:Ext.}
\end{itemize}
\begin{itemize}
\item {Proveniência:(Do turc. \textunderscore ieni-cheri\textunderscore )}
\end{itemize}
Soldado turco, quefaz parte da guarda do sultão.
Guarda-costas ou satélite de uma autoridade despótica.
Tunante; vádio.
Tropas, que agridem violentamente o povo.
\section{Janicefalia}
\begin{itemize}
\item {Grp. gram.:f.}
\end{itemize}
Conformação de janicéfalo.
\section{Janicéfalo}
\begin{itemize}
\item {Grp. gram.:m.}
\end{itemize}
\begin{itemize}
\item {Proveniência:(De \textunderscore Jano\textunderscore , n. p. + gr. \textunderscore kephale\textunderscore )}
\end{itemize}
Monstro do duas cabeças, com as faces em sentido oposto.
\section{Janicephalia}
\begin{itemize}
\item {Grp. gram.:f.}
\end{itemize}
Conformação de janicéphalo.
\section{Janicéphalo}
\begin{itemize}
\item {Grp. gram.:m.}
\end{itemize}
\begin{itemize}
\item {Proveniência:(De \textunderscore Jano\textunderscore , n. p. + gr. \textunderscore kephale\textunderscore )}
\end{itemize}
Monstro do duas cabeças, com as faces em sentido opposto.
\section{Janícipe}
\begin{itemize}
\item {Grp. gram.:m.}
\end{itemize}
\begin{itemize}
\item {Proveniência:(Do lat. \textunderscore Janus\textunderscore  + \textunderscore caput\textunderscore )}
\end{itemize}
O mesmo que \textunderscore janicéphalo\textunderscore .
\section{Janipaba}
\begin{itemize}
\item {Grp. gram.:m.}
\end{itemize}
O mesmo que \textunderscore jenipapo\textunderscore .
\section{Janipapo}
\begin{itemize}
\item {Grp. gram.:m.}
\end{itemize}
O mesmo que \textunderscore jenipapo\textunderscore .
\section{Janiparandiba}
\begin{itemize}
\item {Grp. gram.:f.}
\end{itemize}
O mesmo que \textunderscore japaranduba\textunderscore .
\section{Janiparanduba}
\begin{itemize}
\item {Grp. gram.:f.}
\end{itemize}
O mesmo que \textunderscore japaranduba\textunderscore .
\section{Janistroques}
\begin{itemize}
\item {Grp. gram.:m.}
\end{itemize}
\begin{itemize}
\item {Utilização:Pleb.}
\end{itemize}
Jagodes; joão-ninguém.
\section{Janitor}
\begin{itemize}
\item {Grp. gram.:m.}
\end{itemize}
\begin{itemize}
\item {Utilização:Des.}
\end{itemize}
\begin{itemize}
\item {Proveniência:(Lat. \textunderscore janitor\textunderscore )}
\end{itemize}
O mesmo que \textunderscore porteiro\textunderscore .
\section{Janízaro}
\begin{itemize}
\item {Grp. gram.:m.}
\end{itemize}
\begin{itemize}
\item {Utilização:Fig.}
\end{itemize}
\begin{itemize}
\item {Utilização:Gír.}
\end{itemize}
\begin{itemize}
\item {Grp. gram.:Pl.}
\end{itemize}
\begin{itemize}
\item {Utilização:Ext.}
\end{itemize}
\begin{itemize}
\item {Proveniência:(Do turc. \textunderscore ieni-cheri\textunderscore )}
\end{itemize}
Soldado turco, que faz parte da guarda do sultão.
Guarda-costas ou satéllite de uma autoridade despótica.
Tunante; vádio.
Tropas, que aggridem violentamente o povo.
\section{Jafético}
\begin{itemize}
\item {Grp. gram.:adj.}
\end{itemize}
\begin{itemize}
\item {Proveniência:(De \textunderscore Japhet\textunderscore , n. p. bíblico)}
\end{itemize}
Diz-se de uma raça que, na mais alta antiguidade, povoou os planaltos da Ásia ocidental, e que também se chamou \textunderscore ariana\textunderscore  e \textunderscore indo-europeia\textunderscore .
\section{Jafetita}
\begin{itemize}
\item {Grp. gram.:adj.}
\end{itemize}
O mesmo que \textunderscore jafético\textunderscore .
\section{Janja}
\begin{itemize}
\item {Grp. gram.:f.}
\end{itemize}
Nome de algumas aves de Benguela.
\section{Janjangufai}
\begin{itemize}
\item {Grp. gram.:m.}
\end{itemize}
Planta da Guiné, de fôlhas purgativas.
\section{Jan-mijão}
\begin{itemize}
\item {Grp. gram.:m.}
\end{itemize}
\begin{itemize}
\item {Utilização:des.}
\end{itemize}
\begin{itemize}
\item {Utilização:Pop.}
\end{itemize}
Indivíduo, que urina muito.
\section{Jan-ninguém}
\begin{itemize}
\item {Grp. gram.:m.}
\end{itemize}
O mesmo que \textunderscore joão-ninguém\textunderscore .
\section{Janota}
\begin{itemize}
\item {Grp. gram.:adj.}
\end{itemize}
\begin{itemize}
\item {Grp. gram.:M.}
\end{itemize}
\begin{itemize}
\item {Utilização:Bras}
\end{itemize}
\begin{itemize}
\item {Proveniência:(Do fr. \textunderscore jeannot\textunderscore ? Cp. cast. \textunderscore janota\textunderscore , pateta)}
\end{itemize}
Vestido com esmero; elegante.
Peralta, peralvilho, casquilho.
Corpete do vestido.
\section{Janotada}
\begin{itemize}
\item {Grp. gram.:f.}
\end{itemize}
O mesmo que \textunderscore janotice\textunderscore . Reunião de janotas.
\section{Janotar}
\begin{itemize}
\item {Grp. gram.:v. i.}
\end{itemize}
\begin{itemize}
\item {Proveniência:(De \textunderscore janota\textunderscore )}
\end{itemize}
Sêr janota.
Vestir-se com demasiado apuro; mostrar-se casquilho.
\section{Janotaria}
\begin{itemize}
\item {Grp. gram.:f.}
\end{itemize}
O mesmo que \textunderscore janotada\textunderscore .
\section{Janotice}
\begin{itemize}
\item {Grp. gram.:f.}
\end{itemize}
Qualidade ou acto de janota.
\section{Janotistmo}
\begin{itemize}
\item {Grp. gram.:m.}
\end{itemize}
\begin{itemize}
\item {Proveniência:(De \textunderscore janota\textunderscore )}
\end{itemize}
Janotaria; grande luxo ou apuro no trajar.
\section{Jansónia}
\begin{itemize}
\item {Grp. gram.:f.}
\end{itemize}
\begin{itemize}
\item {Proveniência:(De \textunderscore Janson\textunderscore , n. p.)}
\end{itemize}
Planta leguminosa.
\section{Janta}
\begin{itemize}
\item {Grp. gram.:f.}
\end{itemize}
\begin{itemize}
\item {Utilização:Pop.}
\end{itemize}
Acto do jantar: \textunderscore vamos á janta\textunderscore .
\section{Jantado}
\begin{itemize}
\item {Grp. gram.:adj.}
\end{itemize}
\begin{itemize}
\item {Proveniência:(De \textunderscore jantar\textunderscore )}
\end{itemize}
Que jantou: \textunderscore depois de barbeado e jantado, saiu\textunderscore .
\section{Jantar}
\begin{itemize}
\item {Grp. gram.:v. t.}
\end{itemize}
\begin{itemize}
\item {Grp. gram.:V. i.}
\end{itemize}
\begin{itemize}
\item {Grp. gram.:M.}
\end{itemize}
\begin{itemize}
\item {Proveniência:(Do lat. \textunderscore jentare\textunderscore )}
\end{itemize}
Comer, por occasião da principal refeição do dia: \textunderscore jantar um leitão\textunderscore .
Tomar a principal refeição do dia: \textunderscore ainda não jantei\textunderscore .
Principal refeição diária, algumas vezes a última do dia, mas ordinariamente a que se toma entre o almôço e a ceia.
Espécie de antigo tributo emphytêutico.
\section{Jantarão}
\begin{itemize}
\item {Grp. gram.:m.}
\end{itemize}
\begin{itemize}
\item {Utilização:Fam.}
\end{itemize}
Jantar abundante; grande jantar.
\section{Jantarela}
\begin{itemize}
\item {Grp. gram.:f.}
\end{itemize}
\begin{itemize}
\item {Utilização:Prov.}
\end{itemize}
\begin{itemize}
\item {Utilização:beir.}
\end{itemize}
Jantar modesto, jantar frugal.
\section{Jantareta}
\begin{itemize}
\item {fónica:tarê}
\end{itemize}
\begin{itemize}
\item {Grp. gram.:f.}
\end{itemize}
\begin{itemize}
\item {Utilização:Prov.}
\end{itemize}
\begin{itemize}
\item {Utilização:beir.}
\end{itemize}
O mesmo que \textunderscore jantarela\textunderscore .
\section{Janual}
\begin{itemize}
\item {Grp. gram.:m.}
\end{itemize}
\begin{itemize}
\item {Proveniência:(Lat. \textunderscore janualis\textunderscore )}
\end{itemize}
Bolo que se offertava a Jano, entre os antigos Romanos.
\section{Januadim}
\begin{itemize}
\item {Grp. gram.:m.}
\end{itemize}
Moéda antiga da Índia Portuguesa, correspondente a 6 reis antigos.
\section{Januário}
\begin{itemize}
\item {Grp. gram.:m.}
\end{itemize}
Passarinho de rabo comprido, em Angola.
\section{Janufo}
\begin{itemize}
\item {Grp. gram.:m.}
\end{itemize}
\begin{itemize}
\item {Utilização:T. de Lanhoso}
\end{itemize}
Cigarro ordinario, o mesmo que \textunderscore chanato\textunderscore .
\section{Janumás}
\begin{itemize}
\item {Grp. gram.:m. pl.}
\end{itemize}
Indígenas brasileiros da região do Amazonas.
\section{Janundás}
\begin{itemize}
\item {Grp. gram.:m. pl.}
\end{itemize}
Indígenas brasileiros das margens do Japurá.
\section{Janúsia}
\begin{itemize}
\item {Grp. gram.:f.}
\end{itemize}
\begin{itemize}
\item {Proveniência:(Do lat. \textunderscore Janus\textunderscore , n. p.)}
\end{itemize}
Gênero de arbustos brasileiros.
\section{Jan-vaz}
\begin{itemize}
\item {Grp. gram.:m.}
\end{itemize}
\begin{itemize}
\item {Utilização:Ant.}
\end{itemize}
O mesmo que \textunderscore joão-ninguém\textunderscore . Cf. \textunderscore Eufrosina\textunderscore , 217.
\section{Jaó}
\begin{itemize}
\item {Grp. gram.:m.}
\end{itemize}
Ave brasileira, semelhante ao zabelê.
\section{Jao}
\begin{itemize}
\item {Grp. gram.:m.}
\end{itemize}
Antiga medida itinerária da Malásia, correspondente a quatro léguas e meia. Cf. \textunderscore Peregrinação\textunderscore , XLI e XCV.
\section{Japá}
\begin{itemize}
\item {Grp. gram.:m.}
\end{itemize}
\begin{itemize}
\item {Utilização:Bras. do N}
\end{itemize}
\begin{itemize}
\item {Proveniência:(Do guar. \textunderscore yapá\textunderscore )}
\end{itemize}
Esteira, tecida do fôlhas de palmeira.
\section{Japacani}
\begin{itemize}
\item {Grp. gram.:m.}
\end{itemize}
Pequena ave brasileira.
\section{Japana}
\begin{itemize}
\item {Grp. gram.:f.}
\end{itemize}
Planta, da fam. das compostas, (\textunderscore eupatorium ayapana\textunderscore ).
O mesmo que \textunderscore aiapaina\textunderscore .
\section{Japão}
\begin{itemize}
\item {Grp. gram.:adj.}
\end{itemize}
\begin{itemize}
\item {Grp. gram.:M.}
\end{itemize}
Relativo ao Japão ou aos Japoneses:«\textunderscore ...a nação japoa\textunderscore ». \textunderscore Peregrinação\textunderscore , CCXII.
Habitante do Japão.
Nome de uma espécie de papel.
\section{Japão}
\begin{itemize}
\item {Grp. gram.:m.}
\end{itemize}
\begin{itemize}
\item {Utilização:Bras. do N}
\end{itemize}
Japi grande.
\section{Japaranduba}
\begin{itemize}
\item {Grp. gram.:f.}
\end{itemize}
Arbusto myrtáceo da América do Sul.
\section{Japecanga}
\begin{itemize}
\item {Grp. gram.:f.}
\end{itemize}
Planta asparagínea do Brasil.
\section{Japhético}
\begin{itemize}
\item {Grp. gram.:adj.}
\end{itemize}
\begin{itemize}
\item {Proveniência:(De \textunderscore Japhet\textunderscore , n. p. bíblico)}
\end{itemize}
Diz-se de uma raça que, na mais alta antiguidade, povoou os planaltos da Ásia occidental, e que também se chamou \textunderscore ariana\textunderscore  e \textunderscore indo-europeia\textunderscore .
\section{Japhetita}
\begin{itemize}
\item {Grp. gram.:adj.}
\end{itemize}
O mesmo que \textunderscore japhético\textunderscore .
\section{Japi}
\begin{itemize}
\item {Grp. gram.:m.}
\end{itemize}
\begin{itemize}
\item {Utilização:Bras. do N}
\end{itemize}
O mesmo que \textunderscore japim\textunderscore .
\section{Japiaçu}
\begin{itemize}
\item {Grp. gram.:m.}
\end{itemize}
Ave brasileira, espécie de japim.
\section{Japical}
\begin{itemize}
\item {Grp. gram.:f.}
\end{itemize}
\begin{itemize}
\item {Utilização:Bras}
\end{itemize}
Preparação de certas fôlhas vegetaes, com que se atordoam os peixes, para os pescar.
\section{Japicangar}
\begin{itemize}
\item {Grp. gram.:m.}
\end{itemize}
\begin{itemize}
\item {Utilização:Bras}
\end{itemize}
Salsaparrilha; o mesmo que \textunderscore japecanga\textunderscore .
\section{Japim}
\begin{itemize}
\item {Grp. gram.:m.}
\end{itemize}
\begin{itemize}
\item {Utilização:Bras}
\end{itemize}
Ave canora, espécie de pomba, que imita o canto de outras aves.
\section{Japinabeiro}
\begin{itemize}
\item {Grp. gram.:m.}
\end{itemize}
Árvore fructifera do Brasil.
\section{Japi-uaçá}
\begin{itemize}
\item {Grp. gram.:m.}
\end{itemize}
(V.japiaçu)
\section{Jápix}
\begin{itemize}
\item {Grp. gram.:m.}
\end{itemize}
Vento de Noroéste.--Má escrita de Filinto, X, 262. A fórma exacta seria \textunderscore iápige\textunderscore , do lat. \textunderscore iapix\textunderscore , \textunderscore iapigis\textunderscore . E não é vento de \textunderscore oesnoroéste\textunderscore , que o mesmo Filinto indica.
\section{Japoarandiba}
\begin{itemize}
\item {Grp. gram.:f.}
\end{itemize}
(V.japaranduba)
\section{Japona}
\begin{itemize}
\item {Grp. gram.:f.}
\end{itemize}
\begin{itemize}
\item {Utilização:Pop.}
\end{itemize}
\begin{itemize}
\item {Proveniência:(Do it. \textunderscore gippone\textunderscore )}
\end{itemize}
O mesmo que \textunderscore jaquetão\textunderscore .
\section{Japoneira}
\begin{itemize}
\item {Grp. gram.:f.}
\end{itemize}
\begin{itemize}
\item {Utilização:Prov.}
\end{itemize}
Árvore ou arbusto, que produz a flôr chamada camélia.
\section{Japonense}
\begin{itemize}
\item {Grp. gram.:adj.}
\end{itemize}
O mesmo que \textunderscore japonês\textunderscore .
\section{Japonês}
\begin{itemize}
\item {Grp. gram.:adj.}
\end{itemize}
\begin{itemize}
\item {Grp. gram.:M.}
\end{itemize}
\begin{itemize}
\item {Utilização:T. de Lisbôa}
\end{itemize}
\begin{itemize}
\item {Proveniência:(Do chin. \textunderscore ji\textunderscore  + \textunderscore pen\textunderscore )}
\end{itemize}
Relativo ao Japão.
Habitante do Japão ou indivíduo natural do Japão.
Língua falada pelos Japoneses.
Provinciano, que vem a Lisbôa, fazendo parte de commissões, destinadas a representar aos poderes públicos sôbre interesses locaes.
\section{Japonesismo}
\begin{itemize}
\item {Grp. gram.:m.}
\end{itemize}
Modos ou uso de japonês.
Affeição aos Japoneses ou ás coisas japonesas.
\section{Japónia}
\begin{itemize}
\item {Grp. gram.:f.}
\end{itemize}
\begin{itemize}
\item {Utilização:T. de Turquel}
\end{itemize}
Fruto, conhecido por néspera-do-Japão.
\section{Japonicamente}
\begin{itemize}
\item {Grp. gram.:adv.}
\end{itemize}
\begin{itemize}
\item {Proveniência:(De \textunderscore japónico\textunderscore )}
\end{itemize}
Á maneira dos Japoneses.
\section{Japónico}
\begin{itemize}
\item {Grp. gram.:adj.}
\end{itemize}
Relativo ao Japão ou aos Japoneses.
\section{Japonim}
\begin{itemize}
\item {Grp. gram.:m.}
\end{itemize}
\begin{itemize}
\item {Proveniência:(De \textunderscore japona\textunderscore )}
\end{itemize}
Espécie do japona, usada antigamente.
\section{Japonizar}
\begin{itemize}
\item {Grp. gram.:v. t.}
\end{itemize}
\begin{itemize}
\item {Proveniência:(De \textunderscore Japão\textunderscore , n. p.)}
\end{itemize}
Dar feição ou hábitos de japonês a.
Dar nova cozedura a (loiça de porcelana), para a tornar semelhante á porcelana do Japão.
\section{Japu}
\begin{itemize}
\item {Grp. gram.:m.}
\end{itemize}
Ave brasileira, (\textunderscore ostinops cristatus\textunderscore ), o mesmo que \textunderscore japim\textunderscore .
\section{Japuanga}
\begin{itemize}
\item {Grp. gram.:f.}
\end{itemize}
\begin{itemize}
\item {Utilização:Bras}
\end{itemize}
Espécie de cipó medicinal.
\section{Japubá}
\begin{itemize}
\item {Grp. gram.:m.}
\end{itemize}
Pássaro brasileiro, nocivo aos frutos e destruidor de insectos.
\section{Japué}
\begin{itemize}
\item {Grp. gram.:m.}
\end{itemize}
Pequena ave do Brasil.
\section{Japujaba}
\begin{itemize}
\item {Grp. gram.:m.}
\end{itemize}
O mesmo que \textunderscore japu\textunderscore .
\section{Japurás}
\begin{itemize}
\item {Grp. gram.:m. pl.}
\end{itemize}
Indígenas brasileiros da região do Amazonas.
\section{Jaque}
\begin{itemize}
\item {Grp. gram.:m.}
\end{itemize}
\begin{itemize}
\item {Utilização:Ant.}
\end{itemize}
\begin{itemize}
\item {Utilização:Náut.}
\end{itemize}
\begin{itemize}
\item {Proveniência:(De \textunderscore Jacoh\textunderscore , n. p., segundo Körting)}
\end{itemize}
Espécie de saio militar.
Pequena bandeira branca, orlada de azul, com as armas da nação ao centro, e que se iça no gurupés, para pedir soccorro.
\section{Jaqué}
\begin{itemize}
\item {Grp. gram.:m.}
\end{itemize}
\begin{itemize}
\item {Utilização:Prov.}
\end{itemize}
Casaco curto de mulher.
Collete de homem. Cf. Camillo, \textunderscore Volcões\textunderscore , 103; \textunderscore Narcót.\textunderscore , I, 201.
\section{Jaqueira}
\begin{itemize}
\item {Grp. gram.:f.}
\end{itemize}
Árvore urticácea, que produz a jaca^2.
\section{Jaqueiral}
\begin{itemize}
\item {Grp. gram.:m.}
\end{itemize}
Lugar, onde crescem jaqueiras.
\section{Jaquejaque}
\begin{itemize}
\item {Grp. gram.:m.}
\end{itemize}
\begin{itemize}
\item {Utilização:Bras}
\end{itemize}
Espécie de mamoneiro.
\section{Jaqueta}
\begin{itemize}
\item {fónica:quê}
\end{itemize}
\begin{itemize}
\item {Grp. gram.:f.}
\end{itemize}
\begin{itemize}
\item {Proveniência:(Fr. \textunderscore jaquette\textunderscore )}
\end{itemize}
Casaco curto, sem abas, que se ajusta á cintura; véstia.
\section{Jaquetão}
\begin{itemize}
\item {Grp. gram.:m.}
\end{itemize}
Jaqueta larga, ordinariamente de pano grosso e que chega um pouco abaixo da cintura.
\section{Jaquete}
\begin{itemize}
\item {fónica:quê}
\end{itemize}
\begin{itemize}
\item {Grp. gram.:m.}
\end{itemize}
\begin{itemize}
\item {Utilização:Ant.}
\end{itemize}
Casaco curto ou véstia de soldado:«\textunderscore ...voltavam os jaquetes, o de dentro por de fóra, por não serem conhecidos...\textunderscore »Fern. Lopes, \textunderscore Chrón. de D. João I\textunderscore , p. 2.^a, c. XLV.
(Cp. \textunderscore jaqueta\textunderscore )
\section{Jaquiranabóia}
\begin{itemize}
\item {Grp. gram.:f.}
\end{itemize}
\begin{itemize}
\item {Utilização:Bras}
\end{itemize}
Espécie de cigarra, de cabeça grande, e cuja picada é fatal aos homens e ás plantas.
\section{Jará-açu}
\begin{itemize}
\item {Grp. gram.:m.}
\end{itemize}
Espécie de palmeira, (\textunderscore leopoldina major\textunderscore ).
\section{Jaracatiá}
\begin{itemize}
\item {Grp. gram.:m.}
\end{itemize}
Árvore fructífera dos sertões do Brasil.
Variedade de cacto medicinal.
\section{Jaraguá}
\begin{itemize}
\item {Grp. gram.:m.}
\end{itemize}
\begin{itemize}
\item {Utilização:Bras}
\end{itemize}
Espécie de forragem.
\section{Jaraiuba}
\begin{itemize}
\item {Grp. gram.:f.}
\end{itemize}
Espécie de palmeira.
\section{Jaramacaru}
\begin{itemize}
\item {Grp. gram.:m.}
\end{itemize}
(V.cumbeba)
\section{Jaramataia}
\begin{itemize}
\item {Grp. gram.:f.}
\end{itemize}
Árvore leguminosa do Brasil.
\section{Jarapé}
\begin{itemize}
\item {Grp. gram.:m.}
\end{itemize}
(V.juçapé)
\section{Jaraqui}
\begin{itemize}
\item {Grp. gram.:m.}
\end{itemize}
Peixe do Amazonas.
\section{Jararaca}
\begin{itemize}
\item {Grp. gram.:f.}
\end{itemize}
Espécie de serpentária do Brasil.
Árvore silvestre do Brasil.
Planta aroídea do Brasil.
Cobra venenosa da América do Sul, (\textunderscore lachesis tanceolatus\textunderscore ).
\section{Jararaca-uaçu}
\begin{itemize}
\item {Grp. gram.:f.}
\end{itemize}
O mesmo que \textunderscore jararacuçu\textunderscore .
\section{Jararacuçu}
\begin{itemize}
\item {Grp. gram.:m.}
\end{itemize}
\begin{itemize}
\item {Utilização:Bras}
\end{itemize}
Cobra venenosa, comprida e verde-negra, o mesmo que \textunderscore jararaca\textunderscore .
\section{Jaraticaca}
\begin{itemize}
\item {Grp. gram.:f.}
\end{itemize}
(V.manacá)
\section{Jarauá}
\begin{itemize}
\item {Grp. gram.:m.}
\end{itemize}
Planta brasileira, de fibras têxteis.
\section{Jarava}
\begin{itemize}
\item {Grp. gram.:f.}
\end{itemize}
Gênero de plantas gramíneas.
\section{Jarda}
\begin{itemize}
\item {Grp. gram.:f.}
\end{itemize}
\begin{itemize}
\item {Proveniência:(Do ingl. \textunderscore yard\textunderscore )}
\end{itemize}
Medida inglesa, de comprimento equivalente a 91 centímetros.
\section{Jarda}
\begin{itemize}
\item {Grp. gram.:f.}
\end{itemize}
O mesmo que \textunderscore jardia\textunderscore .
\section{Jardar}
\begin{itemize}
\item {Grp. gram.:v. t.}
\end{itemize}
\begin{itemize}
\item {Utilização:Pop.}
\end{itemize}
Fazer á tôa, sem ordem, sem utilidade: \textunderscore que andas tu a jardar\textunderscore ?
(Cp. \textunderscore jardinar\textunderscore )
\section{Jardia}
\begin{itemize}
\item {Grp. gram.:f.}
\end{itemize}
\begin{itemize}
\item {Utilização:Prov.}
\end{itemize}
\begin{itemize}
\item {Utilização:alent.}
\end{itemize}
\begin{itemize}
\item {Proveniência:(De \textunderscore jardo\textunderscore ^2)}
\end{itemize}
Charneca de rosmano, alecrim, camarinha, jóina, etc. Cf. B. Pato, \textunderscore Livro do Monte\textunderscore , 116.
\section{Jardim}
\begin{itemize}
\item {Grp. gram.:m.}
\end{itemize}
\begin{itemize}
\item {Utilização:Fig.}
\end{itemize}
\begin{itemize}
\item {Utilização:T. de Alcobaça}
\end{itemize}
\begin{itemize}
\item {Proveniência:(Fr. \textunderscore jardin\textunderscore )}
\end{itemize}
Terreno, ordinariamente gradeado ou murado e plantado de vegetaes úteis ou recreativos.
Corredor da popa, numa embarcação.
País fértil e de cultura variada:«\textunderscore meu Portugal..., jardim da Europa\textunderscore . »Th. Ribeiro, \textunderscore D. Jaime\textunderscore .
O mesmo que \textunderscore laranjal\textunderscore .
\section{Jardinação}
\begin{itemize}
\item {Grp. gram.:f.}
\end{itemize}
Acto de jardinar. Cf. Castilho \textunderscore Fastos\textunderscore , III, 550.
\section{Jardinagem}
\begin{itemize}
\item {Grp. gram.:f.}
\end{itemize}
\begin{itemize}
\item {Proveniência:(De \textunderscore jardinar\textunderscore )}
\end{itemize}
Cultura de jardins.
\section{Jardinar}
\begin{itemize}
\item {Grp. gram.:v. t.}
\end{itemize}
\begin{itemize}
\item {Utilização:Pop.}
\end{itemize}
Trabalhar em jardim recreativamente.
Passear, divagar.
\section{Jardineira}
\begin{itemize}
\item {Grp. gram.:f.}
\end{itemize}
\begin{itemize}
\item {Utilização:Fig.}
\end{itemize}
\begin{itemize}
\item {Proveniência:(De \textunderscore jardim\textunderscore )}
\end{itemize}
Mesa, em que se collocam flôres e outros objectos de adôrno, ordinariamente em meio de uma sala.
Mulher de jardineiro ou que trata do jardim.
Mulher muito garrida.
Maneira de preparar certas iguarias, rodeando-as de legumes variados.
\section{Jardineiro}
\begin{itemize}
\item {Grp. gram.:m.}
\end{itemize}
Indivíduo, que trata de jardins ou sabe cultivá-los.
\section{Jardineto}
\begin{itemize}
\item {fónica:nê}
\end{itemize}
\begin{itemize}
\item {Grp. gram.:m.}
\end{itemize}
Jardim pequeno. Cf. Eça, \textunderscore P. Basílio\textunderscore , 582.
\section{Jardinista}
\begin{itemize}
\item {Grp. gram.:m.}
\end{itemize}
Aquelle que gosta muito de jardins.
\section{Jardo}
\begin{itemize}
\item {Grp. gram.:adj.}
\end{itemize}
(Corr. de \textunderscore jalde\textunderscore )
\section{Jardo}
\begin{itemize}
\item {Grp. gram.:adj.}
\end{itemize}
\begin{itemize}
\item {Utilização:T. da Bairrada}
\end{itemize}
\begin{itemize}
\item {Grp. gram.:M.}
\end{itemize}
\begin{itemize}
\item {Utilização:ant.}
\end{itemize}
Diz-se de uma espécie de mato mollar.
Espécie de tecido de lan, de côr cinzenta.
\section{Jareré}
\begin{itemize}
\item {Grp. gram.:m.}
\end{itemize}
\begin{itemize}
\item {Utilização:Bras}
\end{itemize}
Rêde de pescar.
Planta brasileira, de semente oleosa e medicinal.
\section{Jargão}
\begin{itemize}
\item {Grp. gram.:m.}
\end{itemize}
\begin{itemize}
\item {Utilização:Gal}
\end{itemize}
\begin{itemize}
\item {Proveniência:(Fr. \textunderscore jargon\textunderscore )}
\end{itemize}
Calão; gíria.
Linguagem estropiada. Cf. Macedo, \textunderscore Burros\textunderscore , 330; Camillo, \textunderscore Doze Casam\textunderscore ., 180.
\section{Jarivá}
\begin{itemize}
\item {Grp. gram.:f.}
\end{itemize}
Palmeira silvestre do Brasil.
(Do tupi)
\section{Jarmeleiro}
\begin{itemize}
\item {Grp. gram.:adj.}
\end{itemize}
O mesmo que \textunderscore jarmelista\textunderscore .
\section{Jarmelista}
\begin{itemize}
\item {Grp. gram.:adj.}
\end{itemize}
Diz-se da vaca, que foi criada em terras de Jarmelo, no districto da Guarda.
\section{Jaro}
\begin{itemize}
\item {Grp. gram.:m.}
\end{itemize}
\begin{itemize}
\item {Utilização:Des.}
\end{itemize}
O mesmo que \textunderscore jarro\textunderscore ^2, planta. Cf. B. Pereira, \textunderscore Prosódia\textunderscore , vb. \textunderscore iaron\textunderscore .
\section{Jaroba}
\begin{itemize}
\item {Grp. gram.:f.}
\end{itemize}
Planta solânea trepadeira.
\section{Jaronda}
\begin{itemize}
\item {Grp. gram.:f.}
\end{itemize}
\begin{itemize}
\item {Utilização:Prov.}
\end{itemize}
\begin{itemize}
\item {Utilização:alent.}
\end{itemize}
O mesmo que \textunderscore gironda\textunderscore ^1.
\section{Jarra}
\begin{itemize}
\item {Grp. gram.:m.}
\end{itemize}
\begin{itemize}
\item {Utilização:Des.}
\end{itemize}
\begin{itemize}
\item {Grp. gram.:M.  e  adj.}
\end{itemize}
\begin{itemize}
\item {Utilização:Des.}
\end{itemize}
Velho ridículo.
Homem, dado a bebidas alcoólicas.
(Cp. \textunderscore jarro\textunderscore ^1)
\section{Jarra}
\begin{itemize}
\item {Grp. gram.:f.}
\end{itemize}
\begin{itemize}
\item {Utilização:Náut.}
\end{itemize}
\begin{itemize}
\item {Proveniência:(De \textunderscore jarro\textunderscore )}
\end{itemize}
Vaso para ornato ou para conter flôres.
Depósito de água, para ração diária da marinhagem.
\section{Jarrafa}
\begin{itemize}
\item {Grp. gram.:f.}
\end{itemize}
Sável das costas de África.
\section{Jarrão}
\begin{itemize}
\item {Grp. gram.:m.}
\end{itemize}
\begin{itemize}
\item {Proveniência:(De \textunderscore jarra\textunderscore ^2)}
\end{itemize}
Jarra grande.
\section{Jarrear}
\begin{itemize}
\item {Grp. gram.:v. t.}
\end{itemize}
\begin{itemize}
\item {Utilização:T. de Albergaria}
\end{itemize}
\begin{itemize}
\item {Grp. gram.:V. i.}
\end{itemize}
\begin{itemize}
\item {Proveniência:(De \textunderscore jarro\textunderscore ^1)}
\end{itemize}
Beber em grande quantidade: \textunderscore jarrear água\textunderscore .
Beber vinho: \textunderscore anda sempre nas tabernas a jarrear\textunderscore .
\section{Jarreiro}
\begin{itemize}
\item {Grp. gram.:m.}
\end{itemize}
Planta, mais conhecida por \textunderscore jarro\textunderscore . Cf. \textunderscore Bibl. da G. do Campo\textunderscore , 335.
\section{Jarrêta}
\begin{itemize}
\item {Grp. gram.:m., f.  e  adj.}
\end{itemize}
\begin{itemize}
\item {Utilização:Des.}
\end{itemize}
\begin{itemize}
\item {Proveniência:(De \textunderscore jarra\textunderscore ^1)}
\end{itemize}
Pessôa, que traja mal ou á antiga.
Indivíduo velho e ridículo.
Beberrão.
\section{Jarretar}
\begin{itemize}
\item {Grp. gram.:v. t.}
\end{itemize}
\begin{itemize}
\item {Utilização:Ext.}
\end{itemize}
\begin{itemize}
\item {Utilização:Fig.}
\end{itemize}
Cortar os tendões dos jarretes a.
Amputar.
Tornar inhábil; inutilizar.
\section{Jarrête}
\begin{itemize}
\item {Grp. gram.:m.}
\end{itemize}
\begin{itemize}
\item {Proveniência:(Fr. \textunderscore jarret\textunderscore )}
\end{itemize}
Curvejão.
Tendão da perna dos quadrúpedes.
Região posterior do joelho.
\section{Jarreteira}
\begin{itemize}
\item {Grp. gram.:f.}
\end{itemize}
\begin{itemize}
\item {Utilização:Ant.}
\end{itemize}
\begin{itemize}
\item {Proveniência:(De \textunderscore jarrête\textunderscore )}
\end{itemize}
Liga, para atar meias na perna.
Ordem de cavallaria na Inglaterra.
\section{Jarretice}
\begin{itemize}
\item {Grp. gram.:f.}
\end{itemize}
\begin{itemize}
\item {Utilização:ant.}
\end{itemize}
\begin{itemize}
\item {Utilização:Pop.}
\end{itemize}
\begin{itemize}
\item {Proveniência:(De \textunderscore jarrêta\textunderscore )}
\end{itemize}
Hábito de beberrão.
Coisa irrisória, própria de jarrêta.
\section{Jarrilho}
\begin{itemize}
\item {Grp. gram.:m.}
\end{itemize}
\begin{itemize}
\item {Utilização:Ant.}
\end{itemize}
\begin{itemize}
\item {Proveniência:(De \textunderscore jarro\textunderscore ^2)}
\end{itemize}
O mesmo que \textunderscore salsa-parrilha\textunderscore .
\section{Jarrinha}
\begin{itemize}
\item {Grp. gram.:f.}
\end{itemize}
Planta, o mesmo que \textunderscore mil-homens\textunderscore .
\section{Jarro}
\begin{itemize}
\item {Grp. gram.:m.}
\end{itemize}
\begin{itemize}
\item {Utilização:Açor}
\end{itemize}
\begin{itemize}
\item {Proveniência:(Do ár. \textunderscore jarra\textunderscore )}
\end{itemize}
Vaso bojudo e alto, com bico e asa, próprio para conter água e que se applica geralmente para deitar água ás mãos ou na bacia em que se lavam as mãos.
O mesmo que \textunderscore bilha\textunderscore .
\section{Jarro}
\begin{itemize}
\item {Grp. gram.:m.}
\end{itemize}
\begin{itemize}
\item {Proveniência:(Do lat. \textunderscore arum\textunderscore , por intermédio do cast. \textunderscore yaro\textunderscore , ou do lat. \textunderscore iarus\textunderscore . Cf. B. Pereira, \textunderscore Prosódia\textunderscore , vb. \textunderscore iaron\textunderscore )}
\end{itemize}
Planta, que é o typo das aráceas.
\section{Jarundadela}
\begin{itemize}
\item {Grp. gram.:f.}
\end{itemize}
\begin{itemize}
\item {Utilização:Prov.}
\end{itemize}
\begin{itemize}
\item {Utilização:trasm.}
\end{itemize}
Pancada com jarundo.
\section{Jarundar}
\begin{itemize}
\item {Grp. gram.:v. t.}
\end{itemize}
\begin{itemize}
\item {Utilização:Prov.}
\end{itemize}
\begin{itemize}
\item {Utilização:trasm.}
\end{itemize}
Bater com jarundo; sovar.
\section{Jarundo}
\begin{itemize}
\item {Grp. gram.:m.}
\end{itemize}
\begin{itemize}
\item {Utilização:Prov.}
\end{itemize}
\begin{itemize}
\item {Utilização:trasm.}
\end{itemize}
Grande cacete.
Fueiro.
\section{Jasmim}
\begin{itemize}
\item {Grp. gram.:m.}
\end{itemize}
Gênero typo das plantas jasmináceas.
Flôr de jasmim.
Essência aromática de jasmim.
(Do persa \textunderscore iãsemin\textunderscore )
\section{Jasmim-da-terra}
\begin{itemize}
\item {Grp. gram.:m.}
\end{itemize}
Árvore, procedente da Pérsia, (\textunderscore melia azederach\textunderscore , Lin.) e vulgarizada na África.
\section{Jasmim-manga}
\begin{itemize}
\item {Grp. gram.:m.}
\end{itemize}
Planta brasileira, (\textunderscore plumeria drastica\textunderscore ).
\section{Jasmináceas}
\begin{itemize}
\item {Grp. gram.:f. pl.}
\end{itemize}
\begin{itemize}
\item {Proveniência:(De \textunderscore jasmináceo\textunderscore )}
\end{itemize}
Família de plantas dicotyledóneas.
\section{Jasmináceo}
\begin{itemize}
\item {Grp. gram.:adj.}
\end{itemize}
Relativo a jasmim.
\section{Jasmíneas}
\begin{itemize}
\item {Grp. gram.:f. pl.}
\end{itemize}
O mesmo que \textunderscore jasmináceas\textunderscore .
\section{Jasmineiro}
\begin{itemize}
\item {Grp. gram.:m.}
\end{itemize}
O mesmo que \textunderscore jasmim\textunderscore , planta.
\section{Jasmíneo}
\begin{itemize}
\item {Grp. gram.:adj.}
\end{itemize}
O mesmo que \textunderscore jasmináceo\textunderscore .
\section{Jasónia}
\begin{itemize}
\item {Grp. gram.:f.}
\end{itemize}
\begin{itemize}
\item {Proveniência:(Do lat. \textunderscore Jason\textunderscore , n. p.)}
\end{itemize}
Planta, da fam. das compostas.
\section{Jaspe}
\begin{itemize}
\item {Grp. gram.:m.}
\end{itemize}
\begin{itemize}
\item {Proveniência:(Do lat. \textunderscore iaspis\textunderscore )}
\end{itemize}
Variedade de quartzo, pedra fina e opaca, da natureza da ágata.
\section{Jaspeador}
\begin{itemize}
\item {Grp. gram.:m.}
\end{itemize}
\begin{itemize}
\item {Proveniência:(De \textunderscore jaspear\textunderscore )}
\end{itemize}
Aquelle que trabalha em jaspe.
\section{Jaspeadura}
\begin{itemize}
\item {Grp. gram.:f.}
\end{itemize}
Acto ou effeito de jaspear.
\section{Jaspear}
\begin{itemize}
\item {Grp. gram.:v. t.}
\end{itemize}
Dar apparência de jaspe a.
\section{Jaspe-negro}
\begin{itemize}
\item {Grp. gram.:m.}
\end{itemize}
Variedade negra de jaspe, usada no contraste ou ensaios de objectos de oiro, também conhecida por \textunderscore pedra-de-toque\textunderscore .
\section{Jáspeo}
\begin{itemize}
\item {Grp. gram.:adj.}
\end{itemize}
Que tem a côr do jaspe.
\section{Jaspe-sanguíneo}
\begin{itemize}
\item {Grp. gram.:m.}
\end{itemize}
O mesmo que \textunderscore heliotrópio\textunderscore .
\section{Jáspico}
\begin{itemize}
\item {Grp. gram.:adj.}
\end{itemize}
Relativo a jaspe.
Feito de jaspe.
\section{Jaspoide}
\begin{itemize}
\item {Grp. gram.:adj.}
\end{itemize}
\begin{itemize}
\item {Utilização:P. us.}
\end{itemize}
\begin{itemize}
\item {Proveniência:(De \textunderscore jaspe\textunderscore  + gr. \textunderscore eidos\textunderscore )}
\end{itemize}
Que tem a apparência de jaspe.
\section{Jassanan}
\begin{itemize}
\item {Grp. gram.:f.}
\end{itemize}
\begin{itemize}
\item {Utilização:Bras}
\end{itemize}
Pequena ave ribeirinha.
\section{Játaca}
\begin{itemize}
\item {Grp. gram.:f.}
\end{itemize}
Cada uma das antigas histórias orientaes, relativas a Buda.
\section{Jataí}
\begin{itemize}
\item {Utilização:Bras}
\end{itemize}
Espécie de abelha, cujo mel é muito apreciado.
\section{Jataí}
\begin{itemize}
\item {Grp. gram.:m.}
\end{itemize}
\begin{itemize}
\item {Utilização:Bras}
\end{itemize}
Nome de várias plantas leguminosas da América.
\section{Jataíba}
\begin{itemize}
\item {Grp. gram.:f.}
\end{itemize}
O mesmo que \textunderscore jataúba\textunderscore .
\section{Jataúba}
\begin{itemize}
\item {Grp. gram.:f.}
\end{itemize}
Variedade de palmeira.
\section{Jatemar}
\begin{itemize}
\item {Grp. gram.:m.}
\end{itemize}
Árvore asiática, própria para construcções.
\section{Jati}
\begin{itemize}
\item {Grp. gram.:f.}
\end{itemize}
Espécie de abelha, no Brasil.
O mesmo que \textunderscore jataí\textunderscore ^1?
\section{Jatibá}
\begin{itemize}
\item {Grp. gram.:m.}
\end{itemize}
(V.jatobá)
\section{Jatium}
\begin{itemize}
\item {Grp. gram.:m.}
\end{itemize}
\begin{itemize}
\item {Utilização:Bras}
\end{itemize}
Espécie de môsca.
\section{Jatobá}
\begin{itemize}
\item {Grp. gram.:m.}
\end{itemize}
O mesmo que \textunderscore jataí\textunderscore ^2.
\section{Jatu}
\begin{itemize}
\item {Grp. gram.:m.}
\end{itemize}
Planta trepadeira, medicinal, da Guiné.
\section{Jau}
\begin{itemize}
\item {Grp. gram.:m.}
\end{itemize}
Habitante de Java; javanês.
Moderno idioma javanês.
Antiga medida itinerária da Índia. Cf. Couto, \textunderscore Déc.\textunderscore , IV, l. III, c. I; \textunderscore Peregrinação\textunderscore , XLVIII.
\section{Jaú}
\begin{itemize}
\item {Grp. gram.:m.}
\end{itemize}
\begin{itemize}
\item {Utilização:Bras}
\end{itemize}
Peixe de água doce.
\section{Jauaperis}
\begin{itemize}
\item {Grp. gram.:m. pl.}
\end{itemize}
Tríbo selvagem da região do Amazonas.
\section{Jauara-icica}
\begin{itemize}
\item {Grp. gram.:f.}
\end{itemize}
\begin{itemize}
\item {Utilização:Bras. do N}
\end{itemize}
\begin{itemize}
\item {Proveniência:(T. tupi)}
\end{itemize}
Espécie de resina escura, que se emprega como betume.
Planta medicinal, de que se extrai aquella resina.
\section{Jauaratacéuua}
\begin{itemize}
\item {Grp. gram.:f.}
\end{itemize}
\begin{itemize}
\item {Utilização:Bras}
\end{itemize}
Planta medicinal do Amazonas.
\section{Jauari}
\begin{itemize}
\item {Grp. gram.:m.}
\end{itemize}
\begin{itemize}
\item {Utilização:Bras}
\end{itemize}
Espécie de palmeira, de fibras têxteis, (\textunderscore astrocarium janory\textunderscore ).
\section{Jaula}
\begin{itemize}
\item {Grp. gram.:f.}
\end{itemize}
Clausura de animaes ferozes.
(Cast. \textunderscore jaula\textunderscore , gaiola)
\section{Jauna}
\begin{itemize}
\item {Grp. gram.:f.}
\end{itemize}
Planta solânea do Pará.
\section{Jaúnas}
\begin{itemize}
\item {Grp. gram.:m. pl.}
\end{itemize}
Índios selvagens das margens do Apaporis, no Brasil.
\section{Jaupati}
\begin{itemize}
\item {Grp. gram.:m.}
\end{itemize}
Planta brasileira, de fibras têxteis.
\section{Javaés}
\begin{itemize}
\item {Grp. gram.:m. pl.}
\end{itemize}
Antiga tríbo de Índios do Brasil, que se fundiu com a dos Carajás.
\section{Javali}
\begin{itemize}
\item {Grp. gram.:m.}
\end{itemize}
\begin{itemize}
\item {Proveniência:(Do ár. \textunderscore jabali\textunderscore , montês)}
\end{itemize}
Porco bravo ou porco montês.
\section{Javalina}
\begin{itemize}
\item {Grp. gram.:f.}
\end{itemize}
A fêmea do javali.
(Cp. cast. \textunderscore jabalina\textunderscore )
\section{Javalino}
\begin{itemize}
\item {Grp. gram.:adj.}
\end{itemize}
Relativo a javali.
Próprio de javali. Cf. Filinto, XII, 91.
\section{Javanês}
\begin{itemize}
\item {Grp. gram.:adj.}
\end{itemize}
\begin{itemize}
\item {Grp. gram.:M.}
\end{itemize}
\begin{itemize}
\item {Proveniência:(Do rad. do jav. \textunderscore javona\textunderscore )}
\end{itemize}
Relativo a Java.
Habitante de Java.
Língua de Java; jau.
\section{Javardo}
\begin{itemize}
\item {Grp. gram.:m.}
\end{itemize}
\begin{itemize}
\item {Utilização:Fig.}
\end{itemize}
\begin{itemize}
\item {Grp. gram.:Adj.}
\end{itemize}
O mesmo que \textunderscore javali\textunderscore .
Homem grosseiro; brutamontes.
Diz-se de uma variedade de trigo rijo.
\section{Javari}
\begin{itemize}
\item {Grp. gram.:m.}
\end{itemize}
Espécie de palmeira do Brasil.
\section{Javevó}
\begin{itemize}
\item {Grp. gram.:adj.}
\end{itemize}
\begin{itemize}
\item {Utilização:Bras}
\end{itemize}
Que tem aspecto desagradável; mal encarado; mal trajado.
Que tem gordura balôfa.
\section{Javradeira}
\begin{itemize}
\item {Grp. gram.:f.}
\end{itemize}
Instrumento para javrar.
\section{Javradoira}
\begin{itemize}
\item {Grp. gram.:f.}
\end{itemize}
O mesmo que \textunderscore javradeira\textunderscore .
\section{Javradoura}
\begin{itemize}
\item {Grp. gram.:f.}
\end{itemize}
O mesmo que \textunderscore javradeira\textunderscore .
\section{Javrar}
\begin{itemize}
\item {Grp. gram.:v. t.}
\end{itemize}
Fazer javres em.
\section{Javre}
\begin{itemize}
\item {Grp. gram.:m.}
\end{itemize}
\begin{itemize}
\item {Proveniência:(Fr. \textunderscore jable\textunderscore )}
\end{itemize}
Encaixe, nas extremidades das aduelas, para se embutirem os tampos.
\section{Jaza}
\begin{itemize}
\item {Grp. gram.:f.}
\end{itemize}
\begin{itemize}
\item {Utilização:Prov.}
\end{itemize}
\begin{itemize}
\item {Utilização:minh.}
\end{itemize}
\begin{itemize}
\item {Proveniência:(De \textunderscore jazer\textunderscore )}
\end{itemize}
O mesmo que \textunderscore trave\textunderscore .
\section{Jazeda}
\begin{itemize}
\item {fónica:zê}
\end{itemize}
\begin{itemize}
\item {Grp. gram.:f.}
\end{itemize}
\begin{itemize}
\item {Utilização:Ant.}
\end{itemize}
\begin{itemize}
\item {Proveniência:(De \textunderscore jazer\textunderscore )}
\end{itemize}
Ancoragem de navios em bôa enseada.
Jazida.
\section{Jazêncio}
\begin{itemize}
\item {Grp. gram.:adj.}
\end{itemize}
\begin{itemize}
\item {Utilização:Ant.}
\end{itemize}
Próprio, adequado.
(Cp. \textunderscore jazente\textunderscore )
\section{Jazente}
\begin{itemize}
\item {Grp. gram.:adj.}
\end{itemize}
\begin{itemize}
\item {Utilização:Des.}
\end{itemize}
\begin{itemize}
\item {Grp. gram.:M.}
\end{itemize}
\begin{itemize}
\item {Utilização:Prov.}
\end{itemize}
\begin{itemize}
\item {Utilização:trasm.}
\end{itemize}
O mesmo que \textunderscore jacente\textunderscore .
Caibro forte, dormente.
\section{Jazentio}
\begin{itemize}
\item {Grp. gram.:adj.}
\end{itemize}
\begin{itemize}
\item {Utilização:Ant.}
\end{itemize}
\begin{itemize}
\item {Grp. gram.:M.}
\end{itemize}
\begin{itemize}
\item {Utilização:Ant.}
\end{itemize}
\begin{itemize}
\item {Proveniência:(De \textunderscore jazente\textunderscore )}
\end{itemize}
Que está junto; que jaz ao pé.
Lugar, onde alguma coisa jaz ou está; sítio. Cf. Cenáculo, \textunderscore Pastoral\textunderscore , 68.
\section{Jazer}
\begin{itemize}
\item {Grp. gram.:v. i.}
\end{itemize}
\begin{itemize}
\item {Utilização:Ant.}
\end{itemize}
\begin{itemize}
\item {Grp. gram.:V. p.}
\end{itemize}
\begin{itemize}
\item {Grp. gram.:M.}
\end{itemize}
\begin{itemize}
\item {Proveniência:(Lat. \textunderscore jacere\textunderscore )}
\end{itemize}
Estar deitado.
Estar morto.
Estar sepultado: \textunderscore aqui jaz um valente\textunderscore .
Estar quieto, immóvel.
Prostrar-se.
Persistir.
Estar vaga (a herança), sêr jacente.
(A mesma significação). Cf. Castilho, \textunderscore Fausto. Geórg.\textunderscore , etc.
Jazida.
\section{Jazerão}
\begin{itemize}
\item {Grp. gram.:m.}
\end{itemize}
\begin{itemize}
\item {Utilização:Ant.}
\end{itemize}
\begin{itemize}
\item {Proveniência:(Fr. \textunderscore jaseran\textunderscore )}
\end{itemize}
O mesmo que \textunderscore jazerina\textunderscore .
\section{Jazerina}
\begin{itemize}
\item {Grp. gram.:f.}
\end{itemize}
\begin{itemize}
\item {Utilização:Ant.}
\end{itemize}
\begin{itemize}
\item {Proveniência:(De \textunderscore jazerino\textunderscore )}
\end{itemize}
Cota de malha, muito miúda.
\section{Jazerino}
\begin{itemize}
\item {Grp. gram.:adj.}
\end{itemize}
\begin{itemize}
\item {Utilização:Ant.}
\end{itemize}
Relativo a jazerina.
Feito de malha de ferro.
(Cast. \textunderscore jacerino\textunderscore , argelino)
\section{Jazida}
\begin{itemize}
\item {Grp. gram.:f.}
\end{itemize}
\begin{itemize}
\item {Utilização:Fig.}
\end{itemize}
Lugar em que alguém jaz.
Acto ou posição de jazer.
Serenidade, quietação.
\section{Jazido}
\begin{itemize}
\item {Grp. gram.:m.}
\end{itemize}
\begin{itemize}
\item {Utilização:Prov.}
\end{itemize}
\begin{itemize}
\item {Utilização:alg.}
\end{itemize}
\begin{itemize}
\item {Proveniência:(De \textunderscore jazer\textunderscore )}
\end{itemize}
Jazida; jazigo.
\section{Jazigo}
\begin{itemize}
\item {Grp. gram.:m.}
\end{itemize}
\begin{itemize}
\item {Utilização:Fig.}
\end{itemize}
\begin{itemize}
\item {Proveniência:(De \textunderscore jazer\textunderscore )}
\end{itemize}
Sepultura; túmulo.
Mina.
Terreno, em que abundam metaes ou pedras preciosas.
Abrigo; depósito.
\section{Jecoral}
\begin{itemize}
\item {Grp. gram.:adj.}
\end{itemize}
\begin{itemize}
\item {Proveniência:(Lat. \textunderscore jecoralis\textunderscore )}
\end{itemize}
Relativo ao fígado.
\section{Jecorário}
\begin{itemize}
\item {Grp. gram.:adj.}
\end{itemize}
O mesmo que \textunderscore jecoral\textunderscore .
\section{Jecuíba}
\begin{itemize}
\item {Grp. gram.:f.}
\end{itemize}
Árvore brasileira.
\section{Jecuiriti}
\begin{itemize}
\item {Grp. gram.:m.}
\end{itemize}
Planta leguminosa, (\textunderscore abrus precatorius\textunderscore ), das regiões intertropicaes.
\section{Jegra}
\begin{itemize}
\item {Grp. gram.:f.}
\end{itemize}
\begin{itemize}
\item {Utilização:Bras. da Baía}
\end{itemize}
Mula, fêmea do jegre.
\section{Jegre}
\begin{itemize}
\item {Grp. gram.:m.}
\end{itemize}
O mesmo que \textunderscore jegue\textunderscore .
\section{Jegue}
\begin{itemize}
\item {Grp. gram.:m.}
\end{itemize}
\begin{itemize}
\item {Utilização:Bras. da Baía}
\end{itemize}
Mulo.
Jumento.
\section{Jeguiri}
\begin{itemize}
\item {Grp. gram.:m.}
\end{itemize}
O mesmo que \textunderscore jequiri\textunderscore .
\section{Jehovah}
\begin{itemize}
\item {Grp. gram.:m.}
\end{itemize}
\begin{itemize}
\item {Proveniência:(T. hebr.)}
\end{itemize}
Deus, em linguagem bíblica.
\section{Jehóvico}
\begin{itemize}
\item {Grp. gram.:adj.}
\end{itemize}
Relativo a Jehovah ou ao jehovaísmo. Cf. Oliveira Martins, \textunderscore Myth. Relig.\textunderscore 
\section{Jehovismo}
\begin{itemize}
\item {Grp. gram.:m.}
\end{itemize}
\begin{itemize}
\item {Proveniência:(De \textunderscore Jehovah\textunderscore )}
\end{itemize}
O mesmo que \textunderscore judaísmo\textunderscore .
\section{Jehovista}
\begin{itemize}
\item {Grp. gram.:adj.}
\end{itemize}
Diz-se de alguns textos do \textunderscore Pentateuco\textunderscore , nos quaes se dá a Deus o nome de \textunderscore Jehovah\textunderscore , e que alguns criticos distinguem dos textos elohistas, quanto á época e quanto á origem.
\section{Jeira}
\begin{itemize}
\item {Grp. gram.:f.}
\end{itemize}
(Fórma exacta, em vez da usual \textunderscore geira\textunderscore )
\section{Jeitar}
\begin{itemize}
\item {Grp. gram.:v. t.}
\end{itemize}
\begin{itemize}
\item {Utilização:Ant.}
\end{itemize}
\begin{itemize}
\item {Proveniência:(Do lat. \textunderscore jactare\textunderscore )}
\end{itemize}
O mesmo que \textunderscore arremessar\textunderscore . Cf. G. Vicente.
\section{Jeitar-se}
\begin{itemize}
\item {Grp. gram.:v. p.}
\end{itemize}
\begin{itemize}
\item {Utilização:Ant.}
\end{itemize}
\begin{itemize}
\item {Proveniência:(De \textunderscore jeito\textunderscore )}
\end{itemize}
Estabelecer-se; fazer domicílio.
\section{Jeiteira}
\begin{itemize}
\item {Grp. gram.:f.}
\end{itemize}
\begin{itemize}
\item {Utilização:Prov.}
\end{itemize}
\begin{itemize}
\item {Proveniência:(De \textunderscore jeito\textunderscore )}
\end{itemize}
Jeito, habilidade.
\section{Jeito}
\begin{itemize}
\item {Grp. gram.:m.}
\end{itemize}
\begin{itemize}
\item {Proveniência:(Do lat. \textunderscore jactus\textunderscore )}
\end{itemize}
Disposição, propensão, aptidão: \textunderscore têr jeito para a pintura\textunderscore .
Hábito: \textunderscore não tenho o jeito de andar sem bengala\textunderscore .
Ligeiro movimento; gesto.
Torcedura: \textunderscore quando fala, dá um jeito á bôca\textunderscore .
Defeito.
Modo: \textunderscore jeito esquisito de falar\textunderscore .
Arranjo, conveniência: \textunderscore aquelle lucro fez-lhe jeito\textunderscore .
\section{Jeitosamente}
\begin{itemize}
\item {Grp. gram.:adv.}
\end{itemize}
De modo jeitoso; com jeito; com habilidade.
\section{Jeitoso}
\begin{itemize}
\item {Grp. gram.:adj.}
\end{itemize}
Que tem jeito ou aptidão.
Que tem applicação útil: \textunderscore esta vara é jeitosa para sacudir tapetes\textunderscore .
Que tem bôa apparência ou gentileza.
\section{Jejuadeiro}
\begin{itemize}
\item {Grp. gram.:m.  e  adj.}
\end{itemize}
\begin{itemize}
\item {Proveniência:(Do lat. \textunderscore jejunator\textunderscore )}
\end{itemize}
Aquelle que jejua.
\section{Jejuador}
\begin{itemize}
\item {Grp. gram.:m.  e  adj.}
\end{itemize}
\begin{itemize}
\item {Proveniência:(Do lat. \textunderscore jejunator\textunderscore )}
\end{itemize}
Aquelle que jejua.
\section{Jejuar}
\begin{itemize}
\item {Grp. gram.:v. i.}
\end{itemize}
\begin{itemize}
\item {Utilização:Fig.}
\end{itemize}
\begin{itemize}
\item {Utilização:Fam.}
\end{itemize}
\begin{itemize}
\item {Proveniência:(Do lat. \textunderscore jejunare\textunderscore )}
\end{itemize}
Praticar o jejum.
Abster-se de alguma coisa.
Não saber alguma coisa.
\section{Jejum}
\begin{itemize}
\item {Grp. gram.:m.}
\end{itemize}
\begin{itemize}
\item {Utilização:Fig.}
\end{itemize}
\begin{itemize}
\item {Utilização:Fam.}
\end{itemize}
\begin{itemize}
\item {Proveniência:(Do lat. \textunderscore jejunium\textunderscore )}
\end{itemize}
Reducção ou abstinência de alimento, como penitência ou em virtude de preceito ecclesiástico.
Abstenção ou reducção da quantidade ordinária dos alimentos.
Estado de quem não come dêsde o dia anterior: \textunderscore ainda não almocei; estou em jejum\textunderscore .
Privação de alguma coisa.
Ignorância de determinado assumpto: \textunderscore a êste respeito, estás em jejum\textunderscore .
\section{Jejuno}
\begin{itemize}
\item {Grp. gram.:adj.}
\end{itemize}
\begin{itemize}
\item {Grp. gram.:M.}
\end{itemize}
\begin{itemize}
\item {Utilização:Des.}
\end{itemize}
\begin{itemize}
\item {Proveniência:(Lat. \textunderscore jejunus\textunderscore )}
\end{itemize}
Que está em jejum.
Parte do intestino delgado, entre o duodeno e o ílio.
O mesmo que \textunderscore jejum\textunderscore . Cf. Frei Fortun., \textunderscore Inéd.\textunderscore , I, 308.
\section{Jembé}
\begin{itemize}
\item {Grp. gram.:m.}
\end{itemize}
\begin{itemize}
\item {Utilização:Bras}
\end{itemize}
Espécie de esparregado, com lombo salgado de porco.
\section{Jenequéu}
\begin{itemize}
\item {Grp. gram.:m.}
\end{itemize}
(V.ágave)
\section{Jenipapada}
\begin{itemize}
\item {Grp. gram.:f.}
\end{itemize}
\begin{itemize}
\item {Utilização:Bras}
\end{itemize}
Dôce, feito de jenipapo.
\section{Jenipapeiro}
\begin{itemize}
\item {Grp. gram.:m.}
\end{itemize}
\begin{itemize}
\item {Utilização:Bras}
\end{itemize}
\begin{itemize}
\item {Proveniência:(De \textunderscore jenipapo\textunderscore )}
\end{itemize}
Árvore rubiácea da América.
\section{Jenipapo}
\begin{itemize}
\item {Grp. gram.:m.}
\end{itemize}
\begin{itemize}
\item {Utilização:Bras}
\end{itemize}
Fruto do jenipapeiro.
O jenipapeiro.
(Do tupi)
\section{Jenissei}
\begin{itemize}
\item {Grp. gram.:m.}
\end{itemize}
Língua uralo altaica do ramo samoiedo.
\section{Jenisseu}
\begin{itemize}
\item {Grp. gram.:m.}
\end{itemize}
Língua uralo altaica do ramo samoiedo.
\section{Jenolim}
\begin{itemize}
\item {Grp. gram.:m.}
\end{itemize}
O mesmo que \textunderscore massicote\textunderscore .
Côr amarelada.
(Parece relacionar-se com o fr. \textunderscore jaune\textunderscore )
\section{Jeová}
\begin{itemize}
\item {Grp. gram.:m.}
\end{itemize}
\begin{itemize}
\item {Grp. gram.:m.}
\end{itemize}
\begin{itemize}
\item {Proveniência:(T. hebr.)}
\end{itemize}
Deus, em linguagem bíblica.
(V.Jehovah). Cf. Macedo, \textunderscore Oriente\textunderscore , II, 58.
\section{Jeóvico}
\begin{itemize}
\item {Grp. gram.:adj.}
\end{itemize}
Relativo a Jeova ou ao jeovaísmo. Cf. Oliveira Martins, \textunderscore Myth. Relig.\textunderscore 
\section{Jeovismo}
\begin{itemize}
\item {Grp. gram.:m.}
\end{itemize}
\begin{itemize}
\item {Proveniência:(De \textunderscore Jeova\textunderscore )}
\end{itemize}
O mesmo que \textunderscore judaísmo\textunderscore .
\section{Jeovista}
\begin{itemize}
\item {Grp. gram.:adj.}
\end{itemize}
Diz-se de alguns textos do \textunderscore Pentateuco\textunderscore , nos quaes se dá a Deus o nome de \textunderscore Jeova\textunderscore , e que alguns criticos distinguem dos textos eloistas, quanto á época e quanto á origem.
\section{Jequi}
\begin{itemize}
\item {Grp. gram.:m.}
\end{itemize}
\begin{itemize}
\item {Utilização:Bras. do N}
\end{itemize}
Espécie de nassa.
\section{Jequiá}
\begin{itemize}
\item {Grp. gram.:m.}
\end{itemize}
O mesmo que \textunderscore jequi\textunderscore . Cf. \textunderscore Jorn. do Comm.\textunderscore , do Rio, de 22-III-902.
\section{Jequiri}
\begin{itemize}
\item {Grp. gram.:m.}
\end{itemize}
\begin{itemize}
\item {Utilização:Bras}
\end{itemize}
Planta venenosa, que dobra as fôlhas quando se lhes toca.
\section{Jequirioba}
\begin{itemize}
\item {Grp. gram.:f.}
\end{itemize}
Planta solânea, (\textunderscore solanum jequirioba\textunderscore ).
\section{Jequitibá}
\begin{itemize}
\item {Grp. gram.:m.}
\end{itemize}
Grande árvore da América.
\section{Jequitiranabóia}
\begin{itemize}
\item {Grp. gram.:f.}
\end{itemize}
\begin{itemize}
\item {Utilização:Bras}
\end{itemize}
Borboleta venenosa dos sertões.
\section{Jequitivá}
\begin{itemize}
\item {Grp. gram.:m.}
\end{itemize}
Grande árvore da América.
\section{Jerarca}
\begin{itemize}
\item {Grp. gram.:m.}
\end{itemize}
\begin{itemize}
\item {Proveniência:(Do lat. \textunderscore hierarcha\textunderscore )}
\end{itemize}
Autoridade superior, em matéria eclesiástica.
\section{Jerarcha}
\begin{itemize}
\item {fónica:ca}
\end{itemize}
\begin{itemize}
\item {Grp. gram.:m.}
\end{itemize}
\begin{itemize}
\item {Proveniência:(Do lat. \textunderscore hierarcha\textunderscore )}
\end{itemize}
Autoridade superior, em matéria ecclesiástica.
\section{Jerarchia}
\begin{itemize}
\item {fónica:qui}
\end{itemize}
\begin{itemize}
\item {Grp. gram.:f.}
\end{itemize}
\begin{itemize}
\item {Utilização:Eccles.}
\end{itemize}
\begin{itemize}
\item {Proveniência:(Do gr. \textunderscore hieres\textunderscore  + \textunderscore arkhe\textunderscore )}
\end{itemize}
Dignidade do chefe dos antigos sacerdotes gregos.
Ordenada distribuição dos poderes, com subordinação successiva de uns a outros.
Classe.
Ordem e subordinação dos differentes coros de anjos.--Há três jerarquias de anjos: 1.^a, seraphins, cherubins e thronos; 2.^a, dominações, potestades e principados; 3.^a, virtudes, arcanjos e anjos.
\section{Jerarchicamente}
\begin{itemize}
\item {fónica:qui}
\end{itemize}
\begin{itemize}
\item {Grp. gram.:adv.}
\end{itemize}
De modo jerárchico; segundo a jerarchia.
\section{Jerárchico}
\begin{itemize}
\item {fónica:qui}
\end{itemize}
\begin{itemize}
\item {Grp. gram.:adj.}
\end{itemize}
Relativo a jerarchia.
\section{Jerarquia}
\begin{itemize}
\item {Grp. gram.:f.}
\end{itemize}
\begin{itemize}
\item {Utilização:Eccles.}
\end{itemize}
\begin{itemize}
\item {Proveniência:(Do gr. \textunderscore hieres\textunderscore  + \textunderscore arkhe\textunderscore )}
\end{itemize}
Dignidade do chefe dos antigos sacerdotes gregos.
Ordenada distribuição dos poderes, com subordinação sucessiva de uns a outros.
Classe.
Ordem e subordinação dos diferentes coros de anjos.--Há três jerarquias de anjos: 1.^a, serafins, querubins e tronos; 2.^a, dominações, potestades e principados; 3.^a, virtudes, arcanjos e anjos.
\section{Jerarquicamente}
\begin{itemize}
\item {Grp. gram.:adv.}
\end{itemize}
De modo jerárquico; segundo a jerarquia.
\section{Jerárquico}
\begin{itemize}
\item {Grp. gram.:adj.}
\end{itemize}
Relativo a jerarquia.
\section{Jerdónia}
\begin{itemize}
\item {Grp. gram.:f.}
\end{itemize}
Planta gesnerácea.
\section{Jerebita}
\begin{itemize}
\item {Grp. gram.:f.}
\end{itemize}
\begin{itemize}
\item {Utilização:Gír.}
\end{itemize}
O mesmo que \textunderscore mandureba\textunderscore .
Aguardente.
\section{Jeremiada}
\begin{itemize}
\item {Grp. gram.:f.}
\end{itemize}
\begin{itemize}
\item {Utilização:Gal}
\end{itemize}
\begin{itemize}
\item {Proveniência:(Fr. \textunderscore jerémiade\textunderscore )}
\end{itemize}
Lamentação importuna e inútil.
\section{Jeremiar}
\begin{itemize}
\item {Grp. gram.:v. t.  e  i.}
\end{itemize}
\begin{itemize}
\item {Proveniência:(De \textunderscore Jeremias\textunderscore , n. p.)}
\end{itemize}
Lastimar.
Choramingar; fazer lamúria.
\section{Jerepemonga}
\begin{itemize}
\item {Grp. gram.:f.}
\end{itemize}
\begin{itemize}
\item {Utilização:Bras}
\end{itemize}
Serpente aquática.
\section{Jereré}
\begin{itemize}
\item {Grp. gram.:m.}
\end{itemize}
\begin{itemize}
\item {Utilização:Bras. do N}
\end{itemize}
O mesmo que \textunderscore jareré\textunderscore .
\section{Jeribá}
\begin{itemize}
\item {Grp. gram.:m.}
\end{itemize}
O mesmo que \textunderscore jarivá\textunderscore .
\section{Jeridina}
\begin{itemize}
\item {Grp. gram.:f.}
\end{itemize}
Gênero de crustáceos isópodes.
\section{Jerimu}
\begin{itemize}
\item {Grp. gram.:m.}
\end{itemize}
O mesmo que \textunderscore jirimu\textunderscore .
\section{Jerivá}
\begin{itemize}
\item {Grp. gram.:m.}
\end{itemize}
\begin{itemize}
\item {Utilização:Bras. do S}
\end{itemize}
O mesmo que \textunderscore jarivá\textunderscore .
\section{Jero}
\begin{itemize}
\item {Grp. gram.:m.}
\end{itemize}
\begin{itemize}
\item {Utilização:Prov.}
\end{itemize}
O mesmo que \textunderscore jarro\textunderscore ^2, planta. (Colhido em Turquel e em Cintra)
(Cp. \textunderscore jaro\textunderscore )
\section{Jeroglífica}
\begin{itemize}
\item {Grp. gram.:f.}
\end{itemize}
\begin{itemize}
\item {Proveniência:(De \textunderscore jeroglífico\textunderscore )}
\end{itemize}
Sistema de escritura, em que se empregam jeroglifos.
\section{Jeroglificamente}
\begin{itemize}
\item {Grp. gram.:adv.}
\end{itemize}
\begin{itemize}
\item {Proveniência:(De \textunderscore jeroglífico\textunderscore )}
\end{itemize}
Por meio do jeroglifos.
\section{Jeroglífico}
\begin{itemize}
\item {Grp. gram.:adj.}
\end{itemize}
\begin{itemize}
\item {Utilização:Fig.}
\end{itemize}
\begin{itemize}
\item {Proveniência:(Gr. \textunderscore hierogluphikos\textunderscore )}
\end{itemize}
Relativo a jeroglifos.
Misterioso; que dificilmente tem explicação.
\section{Jeroglifo}
\begin{itemize}
\item {Grp. gram.:m.}
\end{itemize}
\begin{itemize}
\item {Utilização:Fig.}
\end{itemize}
\begin{itemize}
\item {Proveniência:(Do gr. \textunderscore hieros\textunderscore  + \textunderscore gluphos\textunderscore )}
\end{itemize}
Espécie de caracteres ou letras, usadas pelos antigos Egípcios, e que imitavam objectos.
Coisa obscura; aquilo que se não compreende bem.
\section{Jeroglýphica}
\begin{itemize}
\item {Grp. gram.:f.}
\end{itemize}
\begin{itemize}
\item {Proveniência:(De \textunderscore jeroglýphico\textunderscore )}
\end{itemize}
Systema de escritura, em que se empregam jeroglyphos.
\section{Jeroglyphicamente}
\begin{itemize}
\item {Grp. gram.:adv.}
\end{itemize}
\begin{itemize}
\item {Proveniência:(De \textunderscore jeroglýphico\textunderscore )}
\end{itemize}
Por meio do jeroglyphos.
\section{Jeroglýphico}
\begin{itemize}
\item {Grp. gram.:adj.}
\end{itemize}
\begin{itemize}
\item {Utilização:Fig.}
\end{itemize}
\begin{itemize}
\item {Proveniência:(Gr. \textunderscore hierogluphikos\textunderscore )}
\end{itemize}
Relativo a jeroglyphos.
Mysterioso; que difficilmente tem explicação.
\section{Jeroglypho}
\begin{itemize}
\item {Grp. gram.:m.}
\end{itemize}
\begin{itemize}
\item {Utilização:Fig.}
\end{itemize}
\begin{itemize}
\item {Proveniência:(Do gr. \textunderscore hieros\textunderscore  + \textunderscore gluphos\textunderscore )}
\end{itemize}
Espécie de caracteres ou letras, usadas pelos antigos Egýpcios, e que imitavam objectos.
Coisa obscura; aquillo que se não comprehende bem.
\section{Jeronimita}
\begin{itemize}
\item {Grp. gram.:m.}
\end{itemize}
\begin{itemize}
\item {Proveniência:(De \textunderscore Jerónimo\textunderscore , n. p.)}
\end{itemize}
Frade da antiga Ordem espanhola de San-Jerónimo.
\section{Jerónimos}
\begin{itemize}
\item {Grp. gram.:m. pl.}
\end{itemize}
Congregação religiosa, que tinha San-Jerónimo por padroeiro.--Em Portugal, chamaram-se \textunderscore jerónimos\textunderscore  os frades que em Espanha se diziam \textunderscore jeronimitas\textunderscore .
\section{Jeronymita}
\begin{itemize}
\item {Grp. gram.:m.}
\end{itemize}
\begin{itemize}
\item {Proveniência:(De \textunderscore Jerónymo\textunderscore , n. p.)}
\end{itemize}
Frade da antiga Ordem espanhola de San-Jerónymo.
\section{Jerónymos}
\begin{itemize}
\item {Grp. gram.:m. pl.}
\end{itemize}
Congregação religiosa, que tinha San-Jerónymo por padroeiro.--Em Portugal, chamaram-se \textunderscore jerónymos\textunderscore  os frades que em Espanha se diziam \textunderscore jeronymitas\textunderscore .
\section{Jeropari}
\begin{itemize}
\item {Grp. gram.:m.}
\end{itemize}
\begin{itemize}
\item {Utilização:Bras}
\end{itemize}
\begin{itemize}
\item {Utilização:pop.}
\end{itemize}
O diabo.
\section{Jeropia}
\begin{itemize}
\item {Grp. gram.:f.}
\end{itemize}
\begin{itemize}
\item {Utilização:Prov.}
\end{itemize}
O mesmo que \textunderscore jeropiga\textunderscore .
\section{Jeropiga}
\begin{itemize}
\item {Grp. gram.:f.}
\end{itemize}
(Fórma preferível a \textunderscore geropiga\textunderscore )
\section{Jerosolymitano}
\begin{itemize}
\item {Grp. gram.:adj.}
\end{itemize}
O mesmo quo \textunderscore hierosolymitano\textunderscore . Cf. Herculano, \textunderscore Hist. de Port.\textunderscore , II, 14.
\section{Jerra}
\begin{itemize}
\item {Grp. gram.:f.}
\end{itemize}
\begin{itemize}
\item {Utilização:Prov.}
\end{itemize}
\begin{itemize}
\item {Utilização:minh.}
\end{itemize}
O mesmo que \textunderscore almotolia\textunderscore .
\section{Jerusano}
\begin{itemize}
\item {Grp. gram.:m.  e  adj.}
\end{itemize}
Casta de uva preta de Azeitão.
\section{Jesu!}
\begin{itemize}
\item {Grp. gram.:interj.}
\end{itemize}
O mesmo que \textunderscore Jesus!\textunderscore 
\section{Jesuatos}
\begin{itemize}
\item {Grp. gram.:m. pl.}
\end{itemize}
\begin{itemize}
\item {Proveniência:(De \textunderscore Jesu\textunderscore , n. p.)}
\end{itemize}
Ordem religiosa, italiana, da regra de Santo Agostinho.
\section{Jesuíta}
\begin{itemize}
\item {Grp. gram.:m.}
\end{itemize}
\begin{itemize}
\item {Proveniência:(De \textunderscore Jesu\textunderscore , n. p.)}
\end{itemize}
Membro da Ordem religiosa, chamada Companhia de Jesus.
\section{Jesuitada}
\begin{itemize}
\item {fónica:zu-i}
\end{itemize}
\begin{itemize}
\item {Grp. gram.:f.}
\end{itemize}
\begin{itemize}
\item {Utilização:Deprec.}
\end{itemize}
Os jesuítas.
\section{Jesuiticamente}
\begin{itemize}
\item {fónica:zu-i}
\end{itemize}
\begin{itemize}
\item {Grp. gram.:adv.}
\end{itemize}
De modo jesuítico; á maneira dos jesuítas.
\section{Jesuitice}
\begin{itemize}
\item {fónica:zu-i}
\end{itemize}
\begin{itemize}
\item {Grp. gram.:f.}
\end{itemize}
Acto ou modos de jesuíta.
\section{Jesuítico}
\begin{itemize}
\item {Grp. gram.:adj.}
\end{itemize}
\begin{itemize}
\item {Utilização:Deprec.}
\end{itemize}
Relativo aos jesuítas.
Próprio de jesuítas.
Fanático; astucioso; hyprócrita.
\section{Jesuitismo}
\begin{itemize}
\item {fónica:zu-i}
\end{itemize}
\begin{itemize}
\item {Grp. gram.:m.}
\end{itemize}
\begin{itemize}
\item {Proveniência:(De \textunderscore jesuíta\textunderscore )}
\end{itemize}
Systema ou carácter dos jesuítas.
\section{Jesuitofobia}
\begin{itemize}
\item {Grp. gram.:f.}
\end{itemize}
\begin{itemize}
\item {Utilização:Neol.}
\end{itemize}
Aversão aos jesuítas. Cf. Camillo, \textunderscore Perfil\textunderscore , 229.
\section{Jesuitophobia}
\begin{itemize}
\item {Grp. gram.:f.}
\end{itemize}
\begin{itemize}
\item {Utilização:Neol.}
\end{itemize}
Aversão aos jesuítas. Cf. Camillo, \textunderscore Perfil\textunderscore , 229.
\section{Jesus!}
\begin{itemize}
\item {Grp. gram.:interj.}
\end{itemize}
\begin{itemize}
\item {Proveniência:(De \textunderscore Jesus\textunderscore , n. p.)}
\end{itemize}
(designativa de \textunderscore admiração\textunderscore , \textunderscore mêdo\textunderscore , etc.)
\section{Jetaicica}
\begin{itemize}
\item {fónica:ta-i}
\end{itemize}
\begin{itemize}
\item {Grp. gram.:f.}
\end{itemize}
\begin{itemize}
\item {Utilização:Bras}
\end{itemize}
Goma copal.
\section{Jetaiuva}
\begin{itemize}
\item {fónica:i-u}
\end{itemize}
\begin{itemize}
\item {Grp. gram.:f.}
\end{itemize}
O mesmo que \textunderscore jataí\textunderscore ^2.
\section{Jetica}
\begin{itemize}
\item {Grp. gram.:f.}
\end{itemize}
\begin{itemize}
\item {Utilização:Bras}
\end{itemize}
Batata doce.
\section{Jeticucu}
\begin{itemize}
\item {Grp. gram.:m.}
\end{itemize}
Planta convolvulácea do Brasil.
\section{Jetiranumbóia}
\begin{itemize}
\item {Grp. gram.:f.}
\end{itemize}
O mesmo que \textunderscore jequitiranabóia\textunderscore . Cf. \textunderscore Jorn. do Comm.\textunderscore , do Rio, de 13-IV-901.
\section{Jetuca}
\begin{itemize}
\item {Grp. gram.:f.}
\end{itemize}
O mesmo quo \textunderscore jetica\textunderscore .
\section{Jevura}
\begin{itemize}
\item {Grp. gram.:m.  e  adj.}
\end{itemize}
\begin{itemize}
\item {Utilização:Bras}
\end{itemize}
Diz-se do feijão que se planta em Janeiro ou Março.
\section{Jibóia}
\begin{itemize}
\item {Grp. gram.:f.}
\end{itemize}
O mesmo ou melhor que \textunderscore gibóia\textunderscore .
\section{Jibungo}
\begin{itemize}
\item {Grp. gram.:m.}
\end{itemize}
\begin{itemize}
\item {Utilização:T. dos negros do Brasil}
\end{itemize}
O mesmo que \textunderscore dinheiro\textunderscore .
(Cp. \textunderscore jimbo\textunderscore ^1)
\section{Jiçara}
\begin{itemize}
\item {Grp. gram.:f.}
\end{itemize}
Coqueiro americano, o mesmo que \textunderscore açahizeiro\textunderscore .
\section{Jifingo}
\begin{itemize}
\item {Grp. gram.:m.}
\end{itemize}
Planta leguminosa da África tropical, (\textunderscore abrus precatorius\textunderscore , Lin.).
\section{Jiga}
\begin{itemize}
\item {Grp. gram.:f.}
\end{itemize}
\begin{itemize}
\item {Proveniência:(Ingl. \textunderscore jig\textunderscore )}
\end{itemize}
Antiga dança popular, muito viva.
\section{Jiga-joga}
\begin{itemize}
\item {Grp. gram.:f.}
\end{itemize}
\begin{itemize}
\item {Utilização:Fig.}
\end{itemize}
Antigo jôgo de cartas: Jôgo da cabra-cega.
Ludíbrio.
Coisa pouco estável.
\section{Jilaba}
\begin{itemize}
\item {Grp. gram.:f.}
\end{itemize}
\begin{itemize}
\item {Proveniência:(Do ár. \textunderscore jalib\textunderscore )}
\end{itemize}
Espécie de vestuário moirisco.
\section{Jilhevo}
\begin{itemize}
\item {Grp. gram.:m.}
\end{itemize}
O mesmo que \textunderscore inguefo\textunderscore .
\section{Jiló}
\begin{itemize}
\item {Grp. gram.:m.}
\end{itemize}
\begin{itemize}
\item {Utilização:Bras}
\end{itemize}
Fruto de jiloeiro.
(Or. afr.)
\section{Jiloeiro}
\begin{itemize}
\item {Grp. gram.:m.}
\end{itemize}
\begin{itemize}
\item {Utilização:Bras}
\end{itemize}
\begin{itemize}
\item {Proveniência:(De \textunderscore jiló\textunderscore )}
\end{itemize}
Planta hortense, da fam. das solâneas.
\section{Jimbelê}
\begin{itemize}
\item {Grp. gram.:m.}
\end{itemize}
\begin{itemize}
\item {Utilização:Bras}
\end{itemize}
Iguaria, o mesmo que \textunderscore canjica\textunderscore .
\section{Jimbo}
\begin{itemize}
\item {Grp. gram.:m.}
\end{itemize}
\begin{itemize}
\item {Utilização:Bras}
\end{itemize}
\begin{itemize}
\item {Utilização:Gír.}
\end{itemize}
\begin{itemize}
\item {Proveniência:(T. quimb.)}
\end{itemize}
Dinheiro.
\section{Jimbo}
\begin{itemize}
\item {Grp. gram.:m.}
\end{itemize}
O mesmo que \textunderscore zimbro\textunderscore ^1.
\section{Jimbolamento}
\begin{itemize}
\item {Grp. gram.:m.}
\end{itemize}
\begin{itemize}
\item {Utilização:T. de Angola}
\end{itemize}
\begin{itemize}
\item {Proveniência:(De \textunderscore jimbolo\textunderscore )}
\end{itemize}
Apresentação de estrangeiros a um soba.
Recepção. Cf. Capello e Ivens, I, 163.
\section{Jimbolo}
\begin{itemize}
\item {Grp. gram.:m.}
\end{itemize}
\begin{itemize}
\item {Utilização:T. de Angola}
\end{itemize}
Espécie de pão, feito só de farinha e água, salvo quando se lhe adicionam ovos.
\section{Jimbongo}
\begin{itemize}
\item {Grp. gram.:m.}
\end{itemize}
O mesmo que \textunderscore jibungo\textunderscore .
\section{Jinete}
\begin{itemize}
\item {fónica:nê}
\end{itemize}
\textunderscore m.\textunderscore  (e der.)
(Forma preferível á usual \textunderscore ginete\textunderscore , etc.)
\section{Jingas}
\begin{itemize}
\item {Grp. gram.:m. pl.}
\end{itemize}
Tríbo de raça conguesa.
\section{Jingilo}
\begin{itemize}
\item {Grp. gram.:m.}
\end{itemize}
Planta hortícola africana, (\textunderscore solanum edule\textunderscore ).
\section{Jingo}
\begin{itemize}
\item {Grp. gram.:m.}
\end{itemize}
\begin{itemize}
\item {Utilização:T. da América do N}
\end{itemize}
Patriota exaggerado e ridículo, que preconiza a guerra contra tudo que é estrangeiro, e que corresponde ao \textunderscore chauvin\textunderscore  fr.
(Teve or. como expressão anódina, numa canção patriótica dos Estados-Unidos)
\section{Jingoísta}
\begin{itemize}
\item {Grp. gram.:m.}
\end{itemize}
Partidário do jingoismo.
\section{Jingoísmo}
\begin{itemize}
\item {Grp. gram.:m.}
\end{itemize}
O partido dos jingos.
\section{Jingoto}
\begin{itemize}
\item {fónica:gô}
\end{itemize}
\begin{itemize}
\item {Grp. gram.:m.}
\end{itemize}
Pau delgado e flexível; vergasta.
(Relaciona-se talvez com \textunderscore gingar\textunderscore )
\section{Jinja}
\textunderscore f.\textunderscore  (e der.)
(Fórma exacta, em vez da usual \textunderscore ginja\textunderscore , etc.).
(Cp. cast. \textunderscore jinja\textunderscore )
\section{Jisonge}
\begin{itemize}
\item {Grp. gram.:m.}
\end{itemize}
Arbusto leguminoso das regiões tropicaes, (\textunderscore cajanus indicus\textunderscore , Spreng.).
\section{Jipepe}
\begin{itemize}
\item {Grp. gram.:m.}
\end{itemize}
Grande árvore da África occidental, (\textunderscore monodora myristica\textunderscore , Dun.).
\section{Jiqui}
\begin{itemize}
\item {Grp. gram.:m.}
\end{itemize}
\begin{itemize}
\item {Utilização:Bras}
\end{itemize}
Espécie de nassa, feita de varas finas e flexíveis.
(Do tupi)
\section{Jiquitaia}
\begin{itemize}
\item {Grp. gram.:f.}
\end{itemize}
\begin{itemize}
\item {Utilização:Bras}
\end{itemize}
Espécie de pimenta moída, que, lançada em caldo, vinagre ou sumo de limão, serve de tempêro á mesa.
(Do tupi \textunderscore juquitaia\textunderscore )
\section{Jiquitibá}
\begin{itemize}
\item {Grp. gram.:m.}
\end{itemize}
\begin{itemize}
\item {Utilização:Bras}
\end{itemize}
Árvore silvestre, que se eleva a grande altura e cuja madeira se emprega em construcções.
\section{Jirafa}
\begin{itemize}
\item {Grp. gram.:f.}
\end{itemize}
(Fórma exacta, em vez da usual \textunderscore girafa\textunderscore )
(Cp. cast. \textunderscore jirafa\textunderscore )
\section{Jirau}
\begin{itemize}
\item {Grp. gram.:m.}
\end{itemize}
\begin{itemize}
\item {Utilização:Bras}
\end{itemize}
O mesmo quo \textunderscore girau\textunderscore .
\section{Jirimu}
\begin{itemize}
\item {Grp. gram.:m.}
\end{itemize}
\begin{itemize}
\item {Utilização:Bras}
\end{itemize}
Espécie do abóbora amarela, (\textunderscore cucurbita pepo\textunderscore , Roxb.). Cp. a variante \textunderscore girimu\textunderscore .
(Do tupi)
\section{Jirimum}
\begin{itemize}
\item {Grp. gram.:m.}
\end{itemize}
\begin{itemize}
\item {Utilização:Bras}
\end{itemize}
O mesmo que \textunderscore jirimu\textunderscore .
\section{Jito}
\begin{itemize}
\item {Grp. gram.:m.}
\end{itemize}
\begin{itemize}
\item {Utilização:Bras}
\end{itemize}
Planta meliácea, purgativa e vermífuga.
\section{Joalharia}
\begin{itemize}
\item {Grp. gram.:f.}
\end{itemize}
Offício, arte ou estabelecimento de joalheiro.
(Cp. \textunderscore joalheiro\textunderscore )
\section{Joalheiro}
\begin{itemize}
\item {Grp. gram.:m.}
\end{itemize}
\begin{itemize}
\item {Proveniência:(Do fr. \textunderscore joaillier\textunderscore )}
\end{itemize}
Fabricante ou negociante de jóias.
Aquelle que engasta pedras preciosas.
\section{Joana}
\begin{itemize}
\item {Grp. gram.:f.}
\end{itemize}
\begin{itemize}
\item {Utilização:Pop.}
\end{itemize}
Variedade de pêra muito sucosa e pouco açucarada.
Burra, jumenta.
\section{Joanes}
\begin{itemize}
\item {Grp. gram.:m. pl.}
\end{itemize}
Aborígenes do Pará.
\section{Joanete}
\begin{itemize}
\item {fónica:nê}
\end{itemize}
\begin{itemize}
\item {Grp. gram.:m.}
\end{itemize}
\begin{itemize}
\item {Utilização:Náut.}
\end{itemize}
\begin{itemize}
\item {Utilização:Anat.}
\end{itemize}
Vela superior á gávea, e na direcção desta.
Saliência na articulação da phalange do primeiro osso do metatarso com a phalange correspondente do dedo grande do pé.
\section{Joanga}
\begin{itemize}
\item {Grp. gram.:f.}
\end{itemize}
\begin{itemize}
\item {Utilização:Ant.}
\end{itemize}
O mesmo que \textunderscore janga\textunderscore .
\section{Joangá}
\begin{itemize}
\item {Grp. gram.:f.}
\end{itemize}
\begin{itemize}
\item {Utilização:Ant.}
\end{itemize}
O mesmo que \textunderscore janga\textunderscore .
\section{Joaninha}
\begin{itemize}
\item {Grp. gram.:f.}
\end{itemize}
\begin{itemize}
\item {Utilização:Prov.}
\end{itemize}
\begin{itemize}
\item {Utilização:dur.}
\end{itemize}
Designação popular da coccinela.
Pequena angoreta.
\section{Joanino}
\begin{itemize}
\item {Grp. gram.:adj.}
\end{itemize}
\begin{itemize}
\item {Proveniência:(De \textunderscore Joane\textunderscore  = \textunderscore João\textunderscore , n. p.)}
\end{itemize}
Relativo ao rei D. João I, de Portugal.
Relativo ao tempo de D. João III: \textunderscore architectura joanina\textunderscore .
\section{João-branco}
\begin{itemize}
\item {Grp. gram.:m.}
\end{itemize}
\begin{itemize}
\item {Proveniência:(Fr. \textunderscore Jean-le blanc\textunderscore )}
\end{itemize}
Grande ave de rapina, dos Alpes e Pyrenéus.
\section{João-congo}
\begin{itemize}
\item {Grp. gram.:m.}
\end{itemize}
\begin{itemize}
\item {Utilização:Bras}
\end{itemize}
O mesmo que \textunderscore guache\textunderscore ^1.
\section{João-da-cadeneta}
\begin{itemize}
\item {Grp. gram.:m.}
\end{itemize}
Espécie de jôgo popular.
\section{João-de-barros}
\begin{itemize}
\item {Grp. gram.:m.}
\end{itemize}
Ave amarela do Brasil, que é do tamanho de um melro, e faz o ninho com barro, do que lhe proveio o nome.
\section{João-de-puçá}
\begin{itemize}
\item {Grp. gram.:m.}
\end{itemize}
Fruto do um arbusto silvestre do Maranhão.
\section{João-domingos}
\begin{itemize}
\item {Grp. gram.:m.}
\end{itemize}
Casta de uva preta de Azeitão.
\section{João-fernandes}
\begin{itemize}
\item {Grp. gram.:m.}
\end{itemize}
\begin{itemize}
\item {Utilização:Pop.}
\end{itemize}
\begin{itemize}
\item {Utilização:Bras}
\end{itemize}
Homem insignificante; joão-ninguém.

Bailado popular, espécie de fandango.
\section{João-ferreira}
\begin{itemize}
\item {Grp. gram.:m.}
\end{itemize}
Espécie de verdelho.
\section{João-galamarte}
\begin{itemize}
\item {Grp. gram.:m.}
\end{itemize}
\begin{itemize}
\item {Utilização:Bras. do N}
\end{itemize}
O mesmo que \textunderscore gangorra\textunderscore ^1.
\section{João-gomes}
\begin{itemize}
\item {Grp. gram.:m.}
\end{itemize}
(V.maria-gomes)
Arvoreta medicinal da ilha de San-Thomé, talvez o mesmo que \textunderscore jangomas\textunderscore .
\section{João-grande}
\begin{itemize}
\item {Grp. gram.:m.}
\end{itemize}
\begin{itemize}
\item {Utilização:Bras}
\end{itemize}
O mesmo que \textunderscore gaivota\textunderscore .
\section{João-mendes}
\begin{itemize}
\item {Grp. gram.:f.}
\end{itemize}
Espécie de videira, de fruto negro.
O fruto della.
\section{João-ninguém}
\begin{itemize}
\item {Grp. gram.:m.}
\end{itemize}
\begin{itemize}
\item {Utilização:Pop.}
\end{itemize}
Homúnculo; homem insignificante; joão-fernandes.
\section{João-noivo}
\begin{itemize}
\item {Grp. gram.:m.}
\end{itemize}
Casta de uva branca serôdia do Cartaxo.
\section{João-paulo}
\begin{itemize}
\item {Grp. gram.:m.}
\end{itemize}
Videira do Brasil.
(Cp. \textunderscore jampaulo\textunderscore )
\section{João-pestana}
\begin{itemize}
\item {Grp. gram.:m.}
\end{itemize}
\begin{itemize}
\item {Utilização:Pop.}
\end{itemize}
Somno: \textunderscore quando me chegou o joão-pestana...\textunderscore 
\section{João-santarém}
\begin{itemize}
\item {Grp. gram.:m.}
\end{itemize}
\begin{itemize}
\item {Utilização:T. de Torres Vedras}
\end{itemize}
Variedade de uva, o mesmo que \textunderscore trincadeira\textunderscore .
\section{João-tolo}
\begin{itemize}
\item {Grp. gram.:m.}
\end{itemize}
Ave variegada do Brasil.
\section{Joapitanga}
\begin{itemize}
\item {Grp. gram.:f.}
\end{itemize}
Planta brasileira.
\section{Job}
\begin{itemize}
\item {Grp. gram.:m.}
\end{itemize}
\begin{itemize}
\item {Utilização:Ant.}
\end{itemize}
Cada uma das travessas que limitavam os bancos dos remadores.
\section{Jocó}
\begin{itemize}
\item {Grp. gram.:m.}
\end{itemize}
Gênero de mammíferos anthropomorphos, o mesmo que \textunderscore chimpanzé\textunderscore .
\section{Jocos}
\begin{itemize}
\item {Grp. gram.:m. pl.}
\end{itemize}
\begin{itemize}
\item {Proveniência:(Lat. \textunderscore jocus\textunderscore )}
\end{itemize}
Personificação poética do prazer, da alegria, dos folguedos. Cf. Filinto, IX, 189.
\section{Jocosamente}
\begin{itemize}
\item {Grp. gram.:adv.}
\end{itemize}
De modo jocoso.
\section{Jocosério}
\begin{itemize}
\item {fónica:sé}
\end{itemize}
\begin{itemize}
\item {Grp. gram.:adj.}
\end{itemize}
\begin{itemize}
\item {Proveniência:(Do lat. \textunderscore jocus\textunderscore  + \textunderscore serius\textunderscore )}
\end{itemize}
Um tanto sério e um tanto jocoso, simultaneamente.
\section{Jocosidade}
\begin{itemize}
\item {Grp. gram.:f.}
\end{itemize}
Qualidade daquelle ou daquillo que é jocoso.
Acto ou dito jocoso.
\section{Jocoso}
\begin{itemize}
\item {Grp. gram.:adj.}
\end{itemize}
\begin{itemize}
\item {Proveniência:(Lat. \textunderscore jocosus\textunderscore )}
\end{itemize}
Alegre; gracioso; que provoca o riso.
\section{Jocossério}
\begin{itemize}
\item {Grp. gram.:adj.}
\end{itemize}
\begin{itemize}
\item {Proveniência:(Do lat. \textunderscore jocus\textunderscore  + \textunderscore serius\textunderscore )}
\end{itemize}
Um tanto sério e um tanto jocoso, simultaneamente.
\section{Jocotupé}
\begin{itemize}
\item {Grp. gram.:m.}
\end{itemize}
\begin{itemize}
\item {Utilização:Bras}
\end{itemize}
Planta, de raíz farinácea, alimentar e doce.
Iguaria, feita dessa raíz.
\section{Jocuístle}
\begin{itemize}
\item {Grp. gram.:m.}
\end{itemize}
Fruto de uma árvore anonácea do México.
\section{Joeira}
\begin{itemize}
\item {Grp. gram.:f.}
\end{itemize}
\begin{itemize}
\item {Utilização:Fig.}
\end{itemize}
Peneira, para separar do trigo o joio; crivo.
Acto de joeirar.
Acto de separar do bom aquillo que é mau ou inútil.
(Por \textunderscore joieira\textunderscore , de \textunderscore joio\textunderscore )
\section{Joeiramento}
\begin{itemize}
\item {Grp. gram.:m.}
\end{itemize}
Acto de joeirar.
\section{Joeirar}
\begin{itemize}
\item {Grp. gram.:v. t.}
\end{itemize}
\begin{itemize}
\item {Utilização:Fig.}
\end{itemize}
\begin{itemize}
\item {Proveniência:(De \textunderscore joeira\textunderscore )}
\end{itemize}
Passar pela joeira ou pelo crivo.
Escolher, separando o bom do mau.
Examinar attentamente.
\section{Joeireiro}
\begin{itemize}
\item {Grp. gram.:m.}
\end{itemize}
Aquelle que joeira.
Aquelle que faz joeiras; peneireiro.
Joeira.
\section{Joeiro}
\begin{itemize}
\item {Grp. gram.:m.}
\end{itemize}
Acto de joeirar. Cf. \textunderscore Tech. Rur.\textunderscore , 360.
\section{Joelhada}
\begin{itemize}
\item {fónica:jo-e}
\end{itemize}
\begin{itemize}
\item {Grp. gram.:f.}
\end{itemize}
Pancada com joêlhos.
\section{Joelhar}
\begin{itemize}
\item {fónica:jo-e}
\end{itemize}
\begin{itemize}
\item {Grp. gram.:v. i.}
\end{itemize}
\begin{itemize}
\item {Utilização:Prov.}
\end{itemize}
\begin{itemize}
\item {Utilização:alent.}
\end{itemize}
O mesmo que \textunderscore ajoelhar\textunderscore .
\section{Joelheira}
\begin{itemize}
\item {fónica:jo-e}
\end{itemize}
\begin{itemize}
\item {Grp. gram.:f.}
\end{itemize}
Parte da armadura, correspondente ao joêlho.
Parte da bota, que cobre o joêlho.
Peça de coiro, com que se resguardam os joêlhos das bêstas.
Peça de madeira, em que se assentam os joêlhos para se fazerem certos serviços, como esfregar casas, etc.
Peça do malha de lan, com que se revestem os joêlhos, por baixo das calças e das ceroilas, para os resguardar do frio.
Deformidade nas calças, no lugar correspondente aos joêlhos.
Vestígio de ferimentos nos joêlhos das bêstas.
\section{Joelheiro}
\begin{itemize}
\item {fónica:jo-e}
\end{itemize}
\begin{itemize}
\item {Grp. gram.:adj.}
\end{itemize}
Que chega até o joêlho.
\section{Joêlho}
\begin{itemize}
\item {Grp. gram.:m.}
\end{itemize}
\begin{itemize}
\item {Proveniência:(Lat. \textunderscore geniculum\textunderscore )}
\end{itemize}
Parte anterior da articulação da coxa com a tíbia.
Apparelho, que liga os instrumentos topográphicos aos tripés.
\section{Joêta}
\begin{itemize}
\item {Grp. gram.:f.}
\end{itemize}
\begin{itemize}
\item {Utilização:Ant.}
\end{itemize}
Pequena jóia.
Enfeite precioso. Cf. \textunderscore Cancion. da Vaticana\textunderscore , 1188.
(Por \textunderscore joieta\textunderscore , de \textunderscore jóia\textunderscore )
\section{Jôga}
\begin{itemize}
\item {Grp. gram.:f.}
\end{itemize}
\begin{itemize}
\item {Utilização:Prov.}
\end{itemize}
\begin{itemize}
\item {Utilização:trasm.}
\end{itemize}
Pedra redonda e lisa, boleada pelas correntes dos rios.
Pedra, que os rapazes atiram.
(Cp. \textunderscore jógo\textunderscore )
\section{Jogada}
\begin{itemize}
\item {Grp. gram.:f.}
\end{itemize}
Lance de jôgo.
\section{Jògada}
\begin{itemize}
\item {Grp. gram.:f.}
\end{itemize}
\begin{itemize}
\item {Utilização:Prov.}
\end{itemize}
\begin{itemize}
\item {Utilização:trasm.}
\end{itemize}
Pancada com jôga.
\section{Jogador}
\begin{itemize}
\item {Grp. gram.:m.  e  adj.}
\end{itemize}
Aquelle que joga, especialmente aquelle que joga por costume.
Aquelle que sabe jogar.
Aquelle que tem o vício do jôgo.
\section{Jogar}
\begin{itemize}
\item {Grp. gram.:v. t.}
\end{itemize}
\begin{itemize}
\item {Utilização:Fig.}
\end{itemize}
\begin{itemize}
\item {Grp. gram.:V. i.}
\end{itemize}
\begin{itemize}
\item {Proveniência:(Do lat. \textunderscore jocari\textunderscore )}
\end{itemize}
Entregar-se ao jôgo de.
Tomar parte no jôgo de: \textunderscore jogar a bisca\textunderscore .
Aventurar.
Sujeitar á sorte.
Sêr destro em: \textunderscore jogar armas\textunderscore .
Arremessar: \textunderscore jogar uma pedra\textunderscore .
Dizer ou fazer por brincadeira: \textunderscore jogar uma chalaça\textunderscore .
\textunderscore Jogar as últimas\textunderscore , empregar os últimos esforços.
Entregar-se a um jôgo.
Têr o hábito do jôgo.
Harmonizar-se, condizer: \textunderscore o teu casaco não joga com as calças\textunderscore .
Agitar-se.
Mover-se, oscillando.
Fazer tiro ou arremêsso.
Funccionar.
\section{Jogata}
\begin{itemize}
\item {Grp. gram.:f.}
\end{itemize}
Partida do jôgo.
\section{Jogatar}
\begin{itemize}
\item {Grp. gram.:v. i.}
\end{itemize}
\begin{itemize}
\item {Utilização:Ant.}
\end{itemize}
Zombar.
(Por \textunderscore joguetar\textunderscore , de \textunderscore joguete\textunderscore )
\section{Jogatina}
\begin{itemize}
\item {Grp. gram.:f.}
\end{itemize}
\begin{itemize}
\item {Utilização:Pop.}
\end{itemize}
\begin{itemize}
\item {Proveniência:(De \textunderscore jogata\textunderscore )}
\end{itemize}
Hábito ou vício do jôgo.
Jogata.
\section{Jógo}
\begin{itemize}
\item {Grp. gram.:m.}
\end{itemize}
\begin{itemize}
\item {Utilização:Prov.}
\end{itemize}
\begin{itemize}
\item {Utilização:trasm.}
\end{itemize}
Pedra roliça dos rios.
O mesmo que \textunderscore gôgo\textunderscore .
(Colhido em Valpaços)
\section{Jôgo}
\begin{itemize}
\item {Grp. gram.:m.}
\end{itemize}
\begin{itemize}
\item {Proveniência:(Do lat. \textunderscore jocus\textunderscore )}
\end{itemize}
Divertimento; recreio; brincadeira.
Passatempo, sujeito a certas regras, e em que geralmente se arrisca dinheiro.
Lugar, em que se realiza êsse passatempo.
Objectos, que servem para êsse passatempo.
Brincadeira infantil, em que se procura revelar habilidade ou destreza: \textunderscore jôgo de prendas\textunderscore .
Modo de jogar.
Preceitos que regulam o bom jogador.
Especulação gananciosa sôbre fundos públicos.
Vício de jogar.
Cada uma das partidas, em que se póde dividir o jôgo.
Parte do vehículo, em que se fixa o rodeiro.
Conjunto de peças ou coisas emparelhadas, que formam um todo.
Gracejo; motejo.
Trocadilho ou combinação artificiosa de (palavras).
Ludíbrio.
Manobra.
Contrato aleatório.
Corridas ou lutas ao desafio, entre os antigos.
Andadura do cavallo, quanto ás mãos.
Funcções mecânicas.
Disfarce: \textunderscore deixou descobrir o jôgo\textunderscore .
Bilhetes e cautelas de lotaria: \textunderscore o cauteleiro vendeu todo o jôgo\textunderscore .
Nome de vários apparelhos de Phýsica.
\textunderscore Jôgo de contas\textunderscore , operação de contabilidade, pela qual os descontos de um pagamento official figuram ficticiamente, como receita e como despesa, nos livros da respectiva Repartição.
\textunderscore Jôgo de águas\textunderscore , artifício, com que, dando-se a certos conductos de água posição ou direcção determinada, se combinam os respectivos jactos, para que produzam aspecto caprichoso ou fantástico.
\section{Jogó}
\begin{itemize}
\item {Grp. gram.:m.}
\end{itemize}
Guisado, usado pelos indígenas de San-Thomé.
\section{Jogral}
\begin{itemize}
\item {Grp. gram.:m.}
\end{itemize}
\begin{itemize}
\item {Utilização:Ant.}
\end{itemize}
\begin{itemize}
\item {Proveniência:(Do lat. \textunderscore jocularis\textunderscore )}
\end{itemize}
Truão; bobo; farsista.
Músico, que tocava, por salário, em festas populares.
\section{Jogralesca}
\begin{itemize}
\item {fónica:lês}
\end{itemize}
\begin{itemize}
\item {Grp. gram.:f.}
\end{itemize}
Cantiga de jogral. Cf. Herculano, \textunderscore Cister\textunderscore , II, 282.
\section{Jogralesco}
\begin{itemize}
\item {fónica:lês}
\end{itemize}
\begin{itemize}
\item {Grp. gram.:adj.}
\end{itemize}
Relativo a jogral.
Próprio de jogral.
\section{Jogralidade}
\begin{itemize}
\item {Grp. gram.:f.}
\end{itemize}
Qualidade de jogral; acto ou dito próprio do jogral.
\section{Jogrão}
\begin{itemize}
\item {Grp. gram.:m.}
\end{itemize}
\begin{itemize}
\item {Utilização:Ant.}
\end{itemize}
O mesmo que \textunderscore jogral\textunderscore .
\section{Jogue}
\begin{itemize}
\item {Grp. gram.:m.}
\end{itemize}
Peregrino penitente na Índia.
Designação de uma categoria de daroêses. Cf. Filinto, \textunderscore D. Man.\textunderscore , I, 175.
\section{Joguete}
\begin{itemize}
\item {fónica:guê}
\end{itemize}
\begin{itemize}
\item {Grp. gram.:m.}
\end{itemize}
\begin{itemize}
\item {Proveniência:(De jôgo)}
\end{itemize}
Brinco; ludíbrio.
Aquelle ou aquillo que serve de brinco ou é objecto de ludíbrio.
\section{Joguetear}
\begin{itemize}
\item {Grp. gram.:v. i.}
\end{itemize}
Gracejar; dizer joguetes.
Esgrimir.
\section{Jogueto}
\begin{itemize}
\item {fónica:guê}
\end{itemize}
\begin{itemize}
\item {Grp. gram.:m.}
\end{itemize}
\begin{itemize}
\item {Utilização:Ant.}
\end{itemize}
O mesmo que \textunderscore joguete\textunderscore . Cf. Sim. Mach., f. 79.
\section{Jóia}
\begin{itemize}
\item {Grp. gram.:f.}
\end{itemize}
\begin{itemize}
\item {Utilização:Fig.}
\end{itemize}
Artefacto, de substância preciosa, pedra ou metal.
Pessôa ou coisa, que é tida em grande valor.
Propina ou direito, que se paga, pela entrada ou admissão numa associação ou grêmio.
(Cp. it. \textunderscore gioia\textunderscore )
\section{Joíba}
\begin{itemize}
\item {Grp. gram.:f.}
\end{itemize}
\begin{itemize}
\item {Utilização:Bras}
\end{itemize}
Árvore silvestre, cuja madeira se emprega em construcções.
\section{Joiça}
\begin{itemize}
\item {Grp. gram.:f.}
\end{itemize}
\begin{itemize}
\item {Utilização:Prov.}
\end{itemize}
\begin{itemize}
\item {Utilização:alent.}
\end{itemize}
\begin{itemize}
\item {Utilização:minh.}
\end{itemize}
O mesmo que \textunderscore bosta\textunderscore .
\section{Jóina}
\begin{itemize}
\item {Grp. gram.:f.}
\end{itemize}
\begin{itemize}
\item {Utilização:Prov.}
\end{itemize}
\begin{itemize}
\item {Utilização:alent.}
\end{itemize}
Erva medicinal, leguminosa.
Lenha miúda, o mesmo que \textunderscore chamiço\textunderscore .
\section{Joio}
\begin{itemize}
\item {Grp. gram.:m.}
\end{itemize}
\begin{itemize}
\item {Utilização:Fig.}
\end{itemize}
\begin{itemize}
\item {Proveniência:(Do lat. \textunderscore lolium\textunderscore )}
\end{itemize}
Planta gramínea, que nasce habitualmente entre o trigo e o damnifica: \textunderscore não há trigo sem joio\textunderscore .
Semente dessa planta.
Coisa ruím, que nasce ou apparece entre as bôas e as damnifica.
\section{Joiosa}
\begin{itemize}
\item {Grp. gram.:f.}
\end{itemize}
\begin{itemize}
\item {Proveniência:(Do fr. \textunderscore joyeux\textunderscore ?)}
\end{itemize}
Nome, que se deu a uma espada do Cid, e que outros attribuíram á de Carlos-Magno, á de Roldão e á de Reinaldo.
\section{Jôla}
\begin{itemize}
\item {Grp. gram.:f.}
\end{itemize}
\begin{itemize}
\item {Utilização:Prov.}
\end{itemize}
\begin{itemize}
\item {Utilização:minh.}
\end{itemize}
\begin{itemize}
\item {Utilização:Gír.}
\end{itemize}
O mesmo que \textunderscore jôrra\textunderscore , vinho.
\section{Jolda}
\begin{itemize}
\item {Grp. gram.:f.}
\end{itemize}
\begin{itemize}
\item {Utilização:Prov.}
\end{itemize}
\begin{itemize}
\item {Utilização:Prov.}
\end{itemize}
\begin{itemize}
\item {Utilização:alent.}
\end{itemize}
O mesmo que \textunderscore choldra\textunderscore .
Súcia.
Pândega.
Bando de animaes.
Qualquer ajuntamento de pessôas; rancho.
\section{Joldeiro}
\begin{itemize}
\item {Grp. gram.:adj.}
\end{itemize}
\begin{itemize}
\item {Utilização:Prov.}
\end{itemize}
\begin{itemize}
\item {Utilização:trasm.}
\end{itemize}
Amigo da jolda; que gosta de andar em jolda.
\section{Joldra}
\begin{itemize}
\item {Grp. gram.:f.}
\end{itemize}
O mesmo que \textunderscore choldra\textunderscore .
\section{Joliz}
\begin{itemize}
\item {Grp. gram.:adj.}
\end{itemize}
\begin{itemize}
\item {Utilização:Ant.}
\end{itemize}
\begin{itemize}
\item {Proveniência:(Fr. \textunderscore joli\textunderscore )}
\end{itemize}
Alegre.
Aprazível. Cf. \textunderscore Roteiro do Mar-Vermelho\textunderscore .
\section{Jomo}
\begin{itemize}
\item {Grp. gram.:m.}
\end{itemize}
Medida itenerária da Pérsia.
\section{Jonadático}
\begin{itemize}
\item {Grp. gram.:adj.}
\end{itemize}
Diz-se da linguagem artificial, como o calão. Cf. J. Ribeiro, \textunderscore Frases Feitas\textunderscore , I, 174.
\section{Jondapuçá}
\begin{itemize}
\item {Grp. gram.:m.}
\end{itemize}
\begin{itemize}
\item {Utilização:Bras}
\end{itemize}
Arvore fructífera dos sertões.
\section{Jongar}
\begin{itemize}
\item {Grp. gram.:v. i.}
\end{itemize}
\begin{itemize}
\item {Utilização:Bras}
\end{itemize}
Dançar o jongo.
\section{Jongo}
\begin{itemize}
\item {Grp. gram.:m.}
\end{itemize}
\begin{itemize}
\item {Utilização:Bras}
\end{itemize}
Dança, usada pelos Negros, nas fazendas.
\section{Jónico}
\begin{itemize}
\item {Grp. gram.:adj.}
\end{itemize}
\begin{itemize}
\item {Grp. gram.:M.}
\end{itemize}
\begin{itemize}
\item {Proveniência:(Lat. \textunderscore ionicus\textunderscore )}
\end{itemize}
Relativo á antiga Jónia.
Diz-se da terceira das cinco ordens de architectura.
Verso grego ou latino, composto de pés jónios.
\section{Jónio}
\begin{itemize}
\item {Grp. gram.:adj.}
\end{itemize}
\begin{itemize}
\item {Grp. gram.:M.}
\end{itemize}
\begin{itemize}
\item {Grp. gram.:Pl.}
\end{itemize}
\begin{itemize}
\item {Proveniência:(Lat. \textunderscore ionius\textunderscore )}
\end{itemize}
Relativo á antiga Jónia.
O dialecto dos Jónios.
Na prosódia antiga, pé de verso, composto de duas breves e duas longas ou vice-versa.
Povos gregos, que habitavam a Jónia.
\section{Jono}
\begin{itemize}
\item {Grp. gram.:m.}
\end{itemize}
Espécie de terreno foreiro, entre os gancares, na Índia portuguesa.
\section{Jonoeiro}
\begin{itemize}
\item {Grp. gram.:m.}
\end{itemize}
Possuidor de jono.
\section{Jordânico}
\begin{itemize}
\item {Grp. gram.:adj.}
\end{itemize}
Relativo ao rio Jordão. Cf. Filinto, XVI, 60, 193 e 283.
\section{Jorgelim}
\begin{itemize}
\item {Grp. gram.:m.}
\end{itemize}
(V.gergelim)
\section{Jorna}
\begin{itemize}
\item {Grp. gram.:f.}
\end{itemize}
\begin{itemize}
\item {Utilização:Pop.}
\end{itemize}
\begin{itemize}
\item {Utilização:ant.}
\end{itemize}
\begin{itemize}
\item {Utilização:Gír.}
\end{itemize}
\begin{itemize}
\item {Proveniência:(Do lat. \textunderscore diurna\textunderscore )}
\end{itemize}
Jornal; salário.
Vagar.
\section{Jornada}
\begin{itemize}
\item {Grp. gram.:f.}
\end{itemize}
\begin{itemize}
\item {Proveniência:(Do lat. hyp. \textunderscore diurnata\textunderscore )}
\end{itemize}
Marcha de um dia.
Viagem por terra.
Acção militar, expedição: \textunderscore em 1580, a triste jornada da África...\textunderscore 
\section{Jornadear}
\begin{itemize}
\item {Grp. gram.:v. i.}
\end{itemize}
Fazer jornada.
\section{Jornal}
\begin{itemize}
\item {Grp. gram.:m.}
\end{itemize}
\begin{itemize}
\item {Utilização:Ext.}
\end{itemize}
\begin{itemize}
\item {Proveniência:(Do fr. \textunderscore journal\textunderscore )}
\end{itemize}
Salário.
Retribuição de um dia de trabalho.
Gazeta diária.
Periódico.
\section{Jornalada}
\begin{itemize}
\item {Grp. gram.:f.}
\end{itemize}
\begin{itemize}
\item {Utilização:Deprec.}
\end{itemize}
\begin{itemize}
\item {Proveniência:(De \textunderscore jornal\textunderscore )}
\end{itemize}
Porção de jornaes.
\section{Jornaleco}
\begin{itemize}
\item {Grp. gram.:m.}
\end{itemize}
\begin{itemize}
\item {Utilização:Deprec.}
\end{itemize}
Jornal pouco importante ou mal redigido.
\section{Jornaleiro}
\begin{itemize}
\item {Grp. gram.:m.}
\end{itemize}
\begin{itemize}
\item {Utilização:Deprec.}
\end{itemize}
\begin{itemize}
\item {Proveniência:(De \textunderscore jornal\textunderscore )}
\end{itemize}
Trabalhador a quem se paga jornal.
O mesmo que \textunderscore jornalista\textunderscore .
\section{Jornalismo}
\begin{itemize}
\item {Grp. gram.:m.}
\end{itemize}
\begin{itemize}
\item {Proveniência:(De \textunderscore jornal\textunderscore )}
\end{itemize}
Imprensa periódica.
Funcções de jornalista.
\section{Jornalista}
\begin{itemize}
\item {Grp. gram.:m.}
\end{itemize}
\begin{itemize}
\item {Proveniência:(De \textunderscore jornal\textunderscore )}
\end{itemize}
Aquelle que por hábito ou profissão escreve em jornal ou jornaes.
Aquelle que escreve com mais ou menos assiduidade na imprensa periódica.
Director de periódico.
\section{Jornalístico}
\begin{itemize}
\item {Grp. gram.:adj.}
\end{itemize}
Relativo a jornalistas ou a jornaes.
\section{Jorne}
\begin{itemize}
\item {Grp. gram.:f.}
\end{itemize}
\begin{itemize}
\item {Utilização:Ant.}
\end{itemize}
Vestuário encanudado, que se usava sôbre a cota de malha, espécie de camisola, em que se bordavam as armas da família. Cf. Herculano, \textunderscore Cister\textunderscore , 165.
Capa, o mesmo que \textunderscore palhoça\textunderscore .
\section{Joropa}
\begin{itemize}
\item {Grp. gram.:f.}
\end{itemize}
Variedade de palmeira americana.
\section{Jôrra}
\begin{itemize}
\item {Grp. gram.:f.}
\end{itemize}
\begin{itemize}
\item {Utilização:Prov.}
\end{itemize}
\begin{itemize}
\item {Utilização:minh.}
\end{itemize}
Breu para vasilhas de barro.
Escumalha; escória.
Vinho.
(Cast. \textunderscore sorra\textunderscore )
\section{Jorramento}
\begin{itemize}
\item {Grp. gram.:m.}
\end{itemize}
\begin{itemize}
\item {Proveniência:(De \textunderscore jorrar\textunderscore ^2)}
\end{itemize}
O mesmo que \textunderscore jôrro\textunderscore .
Inclinação de um muro, formando bojo.
\section{Jorrão}
\begin{itemize}
\item {Grp. gram.:m.}
\end{itemize}
O mesmo que \textunderscore zorra\textunderscore ^1.
Utensílio, semelhante a um leito de carro, para aplanar a terra. Cf. Leoni, \textunderscore Diccion. de Artilh.\textunderscore , inédito.
(Por \textunderscore zorrão\textunderscore , de \textunderscore zorra\textunderscore )
\section{Jorrar}
\begin{itemize}
\item {Grp. gram.:v. t.}
\end{itemize}
Besuntar com jôrra.
\section{Jorrar}
\begin{itemize}
\item {Grp. gram.:v. i.}
\end{itemize}
\begin{itemize}
\item {Grp. gram.:V. t.}
\end{itemize}
Saír com ímpeto, em jôrro.
Formar bojo.
Fazer saír com ímpeto.
\section{Jorreiro}
\begin{itemize}
\item {Grp. gram.:m.}
\end{itemize}
\begin{itemize}
\item {Utilização:Pop.}
\end{itemize}
O mesmo que \textunderscore jorrieiro\textunderscore .
O mesmo que \textunderscore chorreiro\textunderscore .
\section{Jorrieiro}
\begin{itemize}
\item {Grp. gram.:m.}
\end{itemize}
\begin{itemize}
\item {Utilização:Prov.}
\end{itemize}
\begin{itemize}
\item {Proveniência:(Do rad. de \textunderscore jorrar\textunderscore )}
\end{itemize}
Grande porção de água entornada.
\section{Jôrro}
\begin{itemize}
\item {Grp. gram.:m.}
\end{itemize}
Grande jacto; saída impetuosa de um líquido.
Fluência.
Alambor.
\section{Josefino}
\begin{itemize}
\item {Grp. gram.:adj.}
\end{itemize}
\begin{itemize}
\item {Grp. gram.:M.}
\end{itemize}
\begin{itemize}
\item {Proveniência:(Do b. lat. \textunderscore Josephus\textunderscore , n. p.)}
\end{itemize}
Relativo a José, n. p.
Relativo ao tempo de D. José I: \textunderscore legislação josephina\textunderscore  ou \textunderscore pombalina\textunderscore .
Partidário de José Bonaparte, rei de Espanha, nomeado por Napoleão I.
\section{Josephino}
\begin{itemize}
\item {Grp. gram.:adj.}
\end{itemize}
\begin{itemize}
\item {Grp. gram.:M.}
\end{itemize}
\begin{itemize}
\item {Proveniência:(Do b. lat. \textunderscore Josephus\textunderscore , n. p.)}
\end{itemize}
Relativo a José, n. p.
Relativo ao tempo de D. José I: \textunderscore legislação josephina\textunderscore  ou \textunderscore pombalina\textunderscore .
Partidário de José Bonaparte, rei de Espanha, nomeado por Napoleão I.
\section{Josèzinho}
\begin{itemize}
\item {Grp. gram.:m.}
\end{itemize}
\begin{itemize}
\item {Utilização:ant.}
\end{itemize}
\begin{itemize}
\item {Utilização:Pop.}
\end{itemize}
Capote de pouca roda, sem mangas e com cabeção.
\section{Jota}
\begin{itemize}
\item {Grp. gram.:m.}
\end{itemize}
\begin{itemize}
\item {Utilização:Ant.}
\end{itemize}
\begin{itemize}
\item {Utilização:Prov.}
\end{itemize}
\begin{itemize}
\item {Utilização:trasm.}
\end{itemize}
Nome da letra \textunderscore j\textunderscore .
Nada, coisa nenhuma:«\textunderscore ...se nam fosse necessidade, de vergonha nam vos pediria jota.\textunderscore »\textunderscore Eufrosina\textunderscore , acto I, sc. 3.
Pouca coisa; bocado; gota.
\section{Jouça}
\begin{itemize}
\item {Grp. gram.:f.}
\end{itemize}
\begin{itemize}
\item {Utilização:Prov.}
\end{itemize}
\begin{itemize}
\item {Utilização:alent.}
\end{itemize}
\begin{itemize}
\item {Utilização:minh.}
\end{itemize}
O mesmo que \textunderscore bosta\textunderscore .
\section{Jovem}
\begin{itemize}
\item {Grp. gram.:m.  e  adj.}
\end{itemize}
O mesmo ou melhor que \textunderscore jóven\textunderscore .
\section{Jóven}
\begin{itemize}
\item {Grp. gram.:adj.}
\end{itemize}
\begin{itemize}
\item {Grp. gram.:M.  e  f.}
\end{itemize}
\begin{itemize}
\item {Proveniência:(Do lat. \textunderscore juvenis\textunderscore )}
\end{itemize}
Juvenil; que é moço.
E diz-se do animal de tenra idade.
Pessôa que está na juventude.
\section{Jovial}
\begin{itemize}
\item {Grp. gram.:adj.}
\end{itemize}
\begin{itemize}
\item {Proveniência:(Lat. \textunderscore jovialis\textunderscore )}
\end{itemize}
Faceto.
Alegre; gracioso; chistoso: \textunderscore ditos joviaes\textunderscore .
\section{Jovialidade}
\begin{itemize}
\item {Grp. gram.:f.}
\end{itemize}
Qualidade de jovial; dito alegre; facécia.
\section{Jovializante}
\begin{itemize}
\item {Grp. gram.:adj.}
\end{itemize}
Que jovializa.
\section{Jovializar}
\begin{itemize}
\item {Grp. gram.:v. t.}
\end{itemize}
\begin{itemize}
\item {Grp. gram.:V. i.}
\end{itemize}
Tornar jovial.
Mostrar-se jovial; conversar alegremente.
\section{Jovialmente}
\begin{itemize}
\item {Grp. gram.:adv.}
\end{itemize}
De modo jovial.
\section{Juá}
\begin{itemize}
\item {Grp. gram.:f.}
\end{itemize}
\begin{itemize}
\item {Utilização:Bras}
\end{itemize}
Planta solânea, (\textunderscore solanum paniculatum\textunderscore ).
Fruta do juazeiro, o mesmo que \textunderscore juaz\textunderscore .
\section{Juamis}
\begin{itemize}
\item {Grp. gram.:m. pl.}
\end{itemize}
Indios selvagens das margens do Japurá, no Brasil.
\section{Juan-de-las-vinhas}
\begin{itemize}
\item {Grp. gram.:m.}
\end{itemize}
\begin{itemize}
\item {Utilização:Prov.}
\end{itemize}
\begin{itemize}
\item {Utilização:alent.}
\end{itemize}
O mesmo que \textunderscore espantalho\textunderscore .
Homem reles, inútil.
\section{Juá-poca}
\begin{itemize}
\item {Grp. gram.:m.}
\end{itemize}
\begin{itemize}
\item {Utilização:Bras}
\end{itemize}
O mesmo que \textunderscore camapu\textunderscore .
\section{Juaris}
\begin{itemize}
\item {Grp. gram.:m. pl.}
\end{itemize}
Tríbo extincta do Alto Amazonas.
\section{Juaz}
\begin{itemize}
\item {Grp. gram.:m.}
\end{itemize}
Fruto do juazeiro.
\section{Juazeiro}
\begin{itemize}
\item {Grp. gram.:m.}
\end{itemize}
\begin{itemize}
\item {Utilização:Bras}
\end{itemize}
\begin{itemize}
\item {Proveniência:(De \textunderscore juaz\textunderscore )}
\end{itemize}
Árvore rhamnácea.
\section{Juba}
\begin{itemize}
\item {Grp. gram.:f.}
\end{itemize}
\begin{itemize}
\item {Proveniência:(Lat. \textunderscore juba\textunderscore )}
\end{itemize}
Crina de leão.
\section{Jubado}
\begin{itemize}
\item {Grp. gram.:adj.}
\end{itemize}
\begin{itemize}
\item {Proveniência:(Lat. \textunderscore jubatus\textunderscore )}
\end{itemize}
Que tem juba.
\section{Jubai}
\begin{itemize}
\item {Grp. gram.:m.}
\end{itemize}
O mesmo que \textunderscore tamarinheiro\textunderscore .
\section{Jubão}
\begin{itemize}
\item {Grp. gram.:m.}
\end{itemize}
\begin{itemize}
\item {Utilização:Ant.}
\end{itemize}
O mesmo que \textunderscore gibão\textunderscore ^1. Cf. \textunderscore Peregrinação\textunderscore , XCI.
(Cp. \textunderscore aljuba\textunderscore )
\section{Jubéa}
\begin{itemize}
\item {Grp. gram.:f.}
\end{itemize}
\begin{itemize}
\item {Proveniência:(De \textunderscore Juba\textunderscore , n. p.)}
\end{itemize}
Palmeira do Chile.
\section{Jubeba}
\begin{itemize}
\item {Grp. gram.:f.}
\end{itemize}
(V.jurubeba)
\section{Jubeia}
\begin{itemize}
\item {Grp. gram.:f.}
\end{itemize}
\begin{itemize}
\item {Proveniência:(De \textunderscore Juba\textunderscore , n. p.)}
\end{itemize}
Palmeira do Chile.
\section{Jubetaria}
\begin{itemize}
\item {Grp. gram.:f.}
\end{itemize}
\begin{itemize}
\item {Utilização:Ant.}
\end{itemize}
Arruamento de algibebes.
(Cp. \textunderscore jubeteiro\textunderscore )
\section{Jubeteiro}
\begin{itemize}
\item {Grp. gram.:m.}
\end{itemize}
\begin{itemize}
\item {Utilização:Ant.}
\end{itemize}
O mesmo que \textunderscore algibebe\textunderscore .
(Cp. \textunderscore aljubeteiro\textunderscore )
\section{Jubilação}
\begin{itemize}
\item {Grp. gram.:f.}
\end{itemize}
\begin{itemize}
\item {Proveniência:(Lat. \textunderscore jubilatio\textunderscore )}
\end{itemize}
Acto de jubilar.
Estado de quem jubila.
Aposentação de um professor.
\section{Jubilante}
\begin{itemize}
\item {Grp. gram.:adj.}
\end{itemize}
\begin{itemize}
\item {Proveniência:(Lat. \textunderscore jubilans\textunderscore )}
\end{itemize}
Que jubila.
\section{Jubilar}
\begin{itemize}
\item {Grp. gram.:v. t.}
\end{itemize}
\begin{itemize}
\item {Grp. gram.:V. i.}
\end{itemize}
\begin{itemize}
\item {Grp. gram.:V. p.}
\end{itemize}
\begin{itemize}
\item {Proveniência:(Lat. \textunderscore jubilare\textunderscore )}
\end{itemize}
Encher de júbilo.
Encher-se de júbilo.
Aposentar-se, tendo sido professor.
\section{Jubilar}
\begin{itemize}
\item {Grp. gram.:adj.}
\end{itemize}
Relativo a jubileu ou a um anniversário solenne: \textunderscore commemoração jubilar\textunderscore .
\section{Jubileu}
\begin{itemize}
\item {Grp. gram.:m.}
\end{itemize}
\begin{itemize}
\item {Utilização:Ant.}
\end{itemize}
\begin{itemize}
\item {Utilização:Pop.}
\end{itemize}
\begin{itemize}
\item {Proveniência:(Lat. \textunderscore jubilaeus\textunderscore )}
\end{itemize}
Indulgência plenária, concedida pelo Pontífice em certas solennidades.
Solennidade, em que se recebe essa indulgência.
Remissão de servidão, dividas e culpas, entre os Hebreus, de cincoenta em cincoenta annos.
Grande espaço de tempo.
Anniversário solenne.
\section{Júbilo}
\begin{itemize}
\item {Grp. gram.:m.}
\end{itemize}
\begin{itemize}
\item {Proveniência:(Lat. \textunderscore jubilum\textunderscore )}
\end{itemize}
Alegria ruidosa.
Grande contentamento.
\section{Jubiloso}
\begin{itemize}
\item {Grp. gram.:adj.}
\end{itemize}
Que tem júbilo.
Em que há júbilo ou grande alegria.
Muito contente.
\section{Juboso}
\begin{itemize}
\item {Grp. gram.:adj.}
\end{itemize}
Que tem juba. Cf. Castilho, \textunderscore Fastos\textunderscore , II, 129.
\section{Juburu}
\begin{itemize}
\item {Grp. gram.:m.}
\end{itemize}
\begin{itemize}
\item {Utilização:Bras}
\end{itemize}
Ave ribeirinha da região do Purus.
\section{Jucá}
\begin{itemize}
\item {Grp. gram.:m.}
\end{itemize}
Árvore sapotácea, (\textunderscore lacuna gigantea\textunderscore ).
\section{Juçapé}
\begin{itemize}
\item {Grp. gram.:m.}
\end{itemize}
(V.sapé)
\section{Juçara}
\begin{itemize}
\item {Grp. gram.:f.}
\end{itemize}
\begin{itemize}
\item {Utilização:Bras}
\end{itemize}
O mesmo que \textunderscore jiçara\textunderscore .
\section{Juciri}
\begin{itemize}
\item {Grp. gram.:m.}
\end{itemize}
\begin{itemize}
\item {Utilização:Bras}
\end{itemize}
Planta solânea, de fruto comestível.
\section{Jucu}
\begin{itemize}
\item {Grp. gram.:m.}
\end{itemize}
\begin{itemize}
\item {Utilização:Bras}
\end{itemize}
Espécie de canela.
Ave do Brasil.
\section{Jucubaúba}
\begin{itemize}
\item {Grp. gram.:m.}
\end{itemize}
\begin{itemize}
\item {Utilização:Bras}
\end{itemize}
Homem, que vai ao leme, nas canôas.
\section{Jucunda}
\begin{itemize}
\item {Grp. gram.:f.}
\end{itemize}
\begin{itemize}
\item {Proveniência:(Do lat. \textunderscore jucundus\textunderscore )}
\end{itemize}
Gênero de arbustos do Brasil.
\section{Jucundamente}
\begin{itemize}
\item {Grp. gram.:adv.}
\end{itemize}
De modo jucundo; alegremente.
\section{Jucundidade}
\begin{itemize}
\item {Grp. gram.:f.}
\end{itemize}
\begin{itemize}
\item {Proveniência:(Lat. \textunderscore jucunditas\textunderscore )}
\end{itemize}
Qualidade de jucundo.
\section{Jucundo}
\begin{itemize}
\item {Grp. gram.:adj.}
\end{itemize}
\begin{itemize}
\item {Proveniência:(Lat. \textunderscore jucundus\textunderscore )}
\end{itemize}
Agradável; suave: \textunderscore dia jocundo\textunderscore .
Alegre.
\section{Jucuri}
\begin{itemize}
\item {Grp. gram.:m.}
\end{itemize}
Árvore brasileira, cujas fibras servem para cordas e tecidos.
\section{Judaica}
\begin{itemize}
\item {Grp. gram.:f.}
\end{itemize}
O mesmo que \textunderscore judiaga\textunderscore .
\section{Judaico}
\begin{itemize}
\item {Grp. gram.:adj.}
\end{itemize}
\begin{itemize}
\item {Proveniência:(Lat. \textunderscore judaicus\textunderscore )}
\end{itemize}
Relativo a Judeus: \textunderscore raça judaica\textunderscore .
\section{Judaísmo}
\begin{itemize}
\item {Grp. gram.:m.}
\end{itemize}
\begin{itemize}
\item {Proveniência:(Lat. \textunderscore judaismus\textunderscore )}
\end{itemize}
Religião judaica.
Os Judeus.
\section{Judaizante}
\begin{itemize}
\item {fónica:da-i}
\end{itemize}
\begin{itemize}
\item {Grp. gram.:m. ,  f.  e  adj.}
\end{itemize}
\begin{itemize}
\item {Proveniência:(Lat. \textunderscore judaizans\textunderscore )}
\end{itemize}
Pessôa, que judaíza.
\section{Judaizar}
\begin{itemize}
\item {fónica:da-i}
\end{itemize}
\begin{itemize}
\item {Grp. gram.:v. i.}
\end{itemize}
\begin{itemize}
\item {Proveniência:(Lat. \textunderscore judaizare\textunderscore )}
\end{itemize}
Observar todos ou alguns dos ritos e leis dos Judeus.
\section{Judaria}
\begin{itemize}
\item {Grp. gram.:f.}
\end{itemize}
\begin{itemize}
\item {Utilização:Ant.}
\end{itemize}
O mesmo que \textunderscore judiaria\textunderscore .
\section{Judas}
\begin{itemize}
\item {Grp. gram.:m.}
\end{itemize}
\begin{itemize}
\item {Utilização:Fig.}
\end{itemize}
\begin{itemize}
\item {Proveniência:(De \textunderscore Judas\textunderscore , n. p.)}
\end{itemize}
O mesmo que \textunderscore traidor\textunderscore .
Boneco ou estafermo que, nalgumas localidades, se queima sábbado da Alleluia.
\section{Judenga}
\begin{itemize}
\item {Grp. gram.:f.}
\end{itemize}
\begin{itemize}
\item {Proveniência:(De \textunderscore judengo\textunderscore )}
\end{itemize}
Tributo de trinta dinheiros, que os Judeus pagavam por cabeça, em memória e pena de terem vendido Jesus por igual preço.
\section{Judengo}
\begin{itemize}
\item {Grp. gram.:adj.}
\end{itemize}
Relativo a Judeus.
\section{Juderega}
\begin{itemize}
\item {fónica:derê}
\end{itemize}
\begin{itemize}
\item {Grp. gram.:f.}
\end{itemize}
\begin{itemize}
\item {Proveniência:(Do rad. de \textunderscore judeu\textunderscore )}
\end{itemize}
Tributo antigo, que pagavam os Judeus tolerados.
\section{Judeu}
\begin{itemize}
\item {Grp. gram.:adj.}
\end{itemize}
\begin{itemize}
\item {Grp. gram.:M.}
\end{itemize}
\begin{itemize}
\item {Utilização:Pop.}
\end{itemize}
\begin{itemize}
\item {Utilização:ant.}
\end{itemize}
\begin{itemize}
\item {Utilização:Pop.}
\end{itemize}
\begin{itemize}
\item {Proveniência:(Lat. \textunderscore judaeus\textunderscore )}
\end{itemize}
Relativo á Judeia ou aos Judeus.
Aquelle que era natural da Judeia.
Aquelle que segue a religião dos Judeus.
Homem de má índole.
Peixe de Portugal.
O mesmo que \textunderscore phariseu\textunderscore ^2.
\section{Judeu}
\begin{itemize}
\item {Grp. gram.:m.}
\end{itemize}
\begin{itemize}
\item {Utilização:Bras. de Minas}
\end{itemize}
\begin{itemize}
\item {Utilização:T. de Alcanena}
\end{itemize}
Feixe de capim, com pedras dentro, para a formação dos tapumes, em trabalhos de mineração.
Cada uma das duas partes de um enxergão bipartido.
\section{Judia}
\begin{itemize}
\item {Grp. gram.:f.}
\end{itemize}
\begin{itemize}
\item {Utilização:Fig.}
\end{itemize}
Mulher de raça judaica.
Espécie de capa moirisca, á imitação da dos vendedores de tâmaras, mas mais curta e com mais enfeites. (Usou-a Garrett)
Nome de um peixe.
Mulher ou rapariga de má índole, muito travêssa ou escarninha.
(Fem. do cast. \textunderscore judio\textunderscore )
\section{Judiação}
\begin{itemize}
\item {Grp. gram.:f.}
\end{itemize}
\begin{itemize}
\item {Utilização:Bras. da Baía}
\end{itemize}
Acto de judiar; escárneo, motejo.
\section{Judiaga}
\begin{itemize}
\item {Grp. gram.:f.}
\end{itemize}
Variedade de azeitona.
\section{Judiamente}
\begin{itemize}
\item {Grp. gram.:adv.}
\end{itemize}
\begin{itemize}
\item {Proveniência:(De \textunderscore judia\textunderscore )}
\end{itemize}
Á maneira dos Judeus.
\section{Judiar}
\begin{itemize}
\item {Grp. gram.:v. i.}
\end{itemize}
\begin{itemize}
\item {Utilização:Fig.}
\end{itemize}
\begin{itemize}
\item {Proveniência:(De \textunderscore judia\textunderscore , ou por \textunderscore judear\textunderscore , de \textunderscore judeu\textunderscore )}
\end{itemize}
O mesmo que \textunderscore judaizar\textunderscore .
Fazer judiarias.
Apoquentar alguém.
Escarnecer.
\section{Judiaria}
\begin{itemize}
\item {Grp. gram.:f.}
\end{itemize}
\begin{itemize}
\item {Utilização:Fig.}
\end{itemize}
\begin{itemize}
\item {Proveniência:(De \textunderscore judiar\textunderscore )}
\end{itemize}
Grande porção de Judeus.
Arruamento ou bairro de Judeus.
Pirraça.
Chacota.
Maus tratos.
\section{Judicativo}
\begin{itemize}
\item {Grp. gram.:adj.}
\end{itemize}
\begin{itemize}
\item {Proveniência:(Lat. \textunderscore judicativus\textunderscore )}
\end{itemize}
Que tem a faculdade de julgar; que sentenceia.
\section{Judicatório}
\begin{itemize}
\item {Grp. gram.:adj.}
\end{itemize}
\begin{itemize}
\item {Proveniência:(Lat. \textunderscore judicatorius\textunderscore )}
\end{itemize}
Próprio para julgar.
Relativo a julgamento.
\section{Judicatura}
\begin{itemize}
\item {Grp. gram.:f.}
\end{itemize}
\begin{itemize}
\item {Proveniência:(Lat. \textunderscore judicatura\textunderscore )}
\end{itemize}
Cargo ou dignidade de juiz.
Tribunal.
Poder de julgar.
\section{Judicial}
\begin{itemize}
\item {Grp. gram.:adj.}
\end{itemize}
\begin{itemize}
\item {Proveniência:(Lat. \textunderscore judicialis\textunderscore )}
\end{itemize}
Relativo aos tribunaes ou á justiça: \textunderscore reforma judicial\textunderscore .
\section{Judicialmente}
\begin{itemize}
\item {Grp. gram.:adv.}
\end{itemize}
De modo judicial; com intervenção da justiça ou dos tribunaes.
\section{Judiciar}
\begin{itemize}
\item {Grp. gram.:v. i.}
\end{itemize}
\begin{itemize}
\item {Proveniência:(Do lat. \textunderscore judicium\textunderscore )}
\end{itemize}
Tomar decisões judiciaes.
\section{Judiciário}
\begin{itemize}
\item {Grp. gram.:adj.}
\end{itemize}
\begin{itemize}
\item {Proveniência:(Lat. \textunderscore judiciarius\textunderscore )}
\end{itemize}
O mesmo que \textunderscore judicial\textunderscore : \textunderscore organização judiciária\textunderscore .
\section{Judiciosamente}
\begin{itemize}
\item {Grp. gram.:adv.}
\end{itemize}
De modo judicioso; sensatamente.
\section{Judicioso}
\begin{itemize}
\item {Grp. gram.:adj.}
\end{itemize}
\begin{itemize}
\item {Proveniência:(Do lat. \textunderscore judicium\textunderscore )}
\end{itemize}
Que tem bom juízo; que julga acertadamente: \textunderscore homem judícioso\textunderscore .
Acertado; feito com sensatez: \textunderscore um acto judicioso\textunderscore .
Sentencioso.
\section{Judio}
\begin{itemize}
\item {Grp. gram.:m.}
\end{itemize}
\begin{itemize}
\item {Utilização:Pop.}
\end{itemize}
\begin{itemize}
\item {Grp. gram.:Adj.}
\end{itemize}
O mesmo que \textunderscore judeu\textunderscore ^1.
Travêsso, maléfico.
Relativo aos Judeus, judaico. Cf. Garrett, \textunderscore Romanceiro\textunderscore , I, 186.
\section{Juerana}
\begin{itemize}
\item {Grp. gram.:f.}
\end{itemize}
(V.jagoirana)
\section{Juga}
\begin{itemize}
\item {Grp. gram.:f.}
\end{itemize}
Cabeço; lugar alto.
(Cp. \textunderscore jugo\textunderscore )
\section{Jugada}
\begin{itemize}
\item {Grp. gram.:f.}
\end{itemize}
\begin{itemize}
\item {Utilização:Prov.}
\end{itemize}
\begin{itemize}
\item {Utilização:trasm.}
\end{itemize}
\begin{itemize}
\item {Proveniência:(Lat. \textunderscore jugata\textunderscore )}
\end{itemize}
Terreno, que uma junta de bois póde lavrar num dia.
Geira.
Antigo tributo, que recaía em terras lavradias.
Junta de bois.
\section{Jugadar}
\begin{itemize}
\item {Grp. gram.:v. t.}
\end{itemize}
\begin{itemize}
\item {Utilização:Ant.}
\end{itemize}
Medir (o pão da jugada, tributo).
\section{Jugadeiro}
\begin{itemize}
\item {Grp. gram.:adj.}
\end{itemize}
\begin{itemize}
\item {Grp. gram.:M.}
\end{itemize}
\begin{itemize}
\item {Utilização:Prov.}
\end{itemize}
\begin{itemize}
\item {Utilização:trasm.}
\end{itemize}
Relativo a jugada.
Cultivador ou proprietário de jugada. Cf. Herculano, \textunderscore Hist. de Port.\textunderscore , III, 89 e 338.
Pequeno lavrador, que sustenta uma junta de bois.
\section{Jugador}
\begin{itemize}
\item {Grp. gram.:m.}
\end{itemize}
\begin{itemize}
\item {Proveniência:(De \textunderscore jugar\textunderscore )}
\end{itemize}
Instrumento, com que se abatem os carneiros, no matadoiro.
\section{Jugal}
\begin{itemize}
\item {Grp. gram.:adj.}
\end{itemize}
\begin{itemize}
\item {Proveniência:(Lat. \textunderscore jugalis\textunderscore )}
\end{itemize}
Relativo a marido e mulher; matrimonial.
\section{Jugar}
\begin{itemize}
\item {Grp. gram.:v. t.}
\end{itemize}
\begin{itemize}
\item {Proveniência:(De \textunderscore jugo\textunderscore )}
\end{itemize}
Abater (reses), ferindo-as na secção da medulla espinhal.
\section{Jugaria}
\begin{itemize}
\item {Grp. gram.:f.}
\end{itemize}
\begin{itemize}
\item {Proveniência:(De \textunderscore jugo\textunderscore ^1)}
\end{itemize}
O mesmo que \textunderscore jugada\textunderscore , tributo.
\section{Juglândeas}
\begin{itemize}
\item {Grp. gram.:f. pl.}
\end{itemize}
\begin{itemize}
\item {Proveniência:(Do lat. \textunderscore juglans\textunderscore , por \textunderscore jovis\textunderscore  + \textunderscore glans\textunderscore )}
\end{itemize}
Família de árvores dicotyledóneas, que têm por typo a nogueira.
\section{Juglandiáceas}
\begin{itemize}
\item {Grp. gram.:f. pl.}
\end{itemize}
O mesmo que \textunderscore juglândeas\textunderscore .
\section{Juglandina}
\begin{itemize}
\item {Grp. gram.:f.}
\end{itemize}
\begin{itemize}
\item {Proveniência:(Do lat. \textunderscore juglans\textunderscore )}
\end{itemize}
Princípio amargo da casca verde da noz.
\section{Juglandíneas}
\begin{itemize}
\item {Grp. gram.:f. pl.}
\end{itemize}
O mesmo que \textunderscore juglândeas\textunderscore .
\section{Jugo}
\begin{itemize}
\item {Grp. gram.:m.}
\end{itemize}
\begin{itemize}
\item {Utilização:T. de Amarante}
\end{itemize}
\begin{itemize}
\item {Utilização:Fig.}
\end{itemize}
\begin{itemize}
\item {Proveniência:(Lat. \textunderscore jugum\textunderscore )}
\end{itemize}
Canga de bois.
Junta de bois.
Espécie de fôrca, por baixo da qual os Romanos faziam passar os inimigos vencidos.
Parte anterior do pescoço sôbre o peito, (desus. neste sentido).
Acto ou processo de jugar.
Peça de madeira, que, nalguns sobrados, se assenta sôbre as vigas, para nella se apoiarem as vigotas.
Submissão; oppressão; autoridade.
\section{Jugueiro}
\begin{itemize}
\item {Grp. gram.:m.}
\end{itemize}
O mesmo que \textunderscore jugadeiro\textunderscore .
\section{Júgula}
\begin{itemize}
\item {Grp. gram.:f.}
\end{itemize}
\begin{itemize}
\item {Proveniência:(Lat. \textunderscore jugula\textunderscore )}
\end{itemize}
Constellação de três estrelas, em volta do Órion.
\section{Jugulação}
\begin{itemize}
\item {Grp. gram.:f.}
\end{itemize}
Acto de jugular.
\section{Jugulante}
\begin{itemize}
\item {Grp. gram.:adj.}
\end{itemize}
\begin{itemize}
\item {Proveniência:(Lat. \textunderscore jugulans\textunderscore )}
\end{itemize}
Diz-se do méthodo curativo, destinado a fazer abortar a doença.
\section{Jugular}
\begin{itemize}
\item {Grp. gram.:v. t.}
\end{itemize}
\begin{itemize}
\item {Proveniência:(Lat. \textunderscore jugulare\textunderscore )}
\end{itemize}
Debellar, extinguir (uma revolta, uma epidemia).
Decapitar; assassinar.
\section{Jugular}
\begin{itemize}
\item {Grp. gram.:adj.}
\end{itemize}
\begin{itemize}
\item {Grp. gram.:M.}
\end{itemize}
\begin{itemize}
\item {Utilização:Anat.}
\end{itemize}
\begin{itemize}
\item {Proveniência:(Lat. \textunderscore jugularis\textunderscore )}
\end{itemize}
Relativo á garganta.
A região da garganta. Cf. Castilho, \textunderscore Fastos\textunderscore , II, 234.
\section{Juguleiras}
\begin{itemize}
\item {Grp. gram.:f. pl.}
\end{itemize}
\begin{itemize}
\item {Proveniência:(Do lat. \textunderscore jugulum\textunderscore )}
\end{itemize}
Depressões longitudinaes na garganta das bêstas.
\section{Juigar}
\begin{itemize}
\item {fónica:ju-i}
\end{itemize}
\begin{itemize}
\item {Grp. gram.:v. t.  e  i.}
\end{itemize}
\begin{itemize}
\item {Utilização:Ant.}
\end{itemize}
\begin{itemize}
\item {Proveniência:(Do lat. \textunderscore judicare\textunderscore )}
\end{itemize}
O mesmo que \textunderscore julgar\textunderscore . Cf. Frei Fortun., \textunderscore Inéditos\textunderscore , 309.
\section{Juiz}
\begin{itemize}
\item {Grp. gram.:m.}
\end{itemize}
\begin{itemize}
\item {Utilização:Fig.}
\end{itemize}
\begin{itemize}
\item {Utilização:Prov.}
\end{itemize}
\begin{itemize}
\item {Proveniência:(Do lat. \textunderscore judex\textunderscore )}
\end{itemize}
Aquelle que tem competência legal para tomar conhecimento das causas dos litigantes e julgá-las por sentença.
Aquelle que julga; julgador.
Árbitro.
Membro do júry.
Membro do poder judicial.
Aquelle que em certos jogos faz cumprir a lei e regras estabelecidas a respeito dêsses jogos.
Aquelle que tem a seu cargo dirigir certas festividades de igreja.
\textunderscore Juiz da fome\textunderscore , pessôa, que passa muita fome ou anda muito magra.
\section{Juíza}
\begin{itemize}
\item {Grp. gram.:f.}
\end{itemize}
\begin{itemize}
\item {Utilização:des.}
\end{itemize}
\begin{itemize}
\item {Utilização:Pop.}
\end{itemize}
Mulher que julga.
Mulher que dirige certas festividades de igreja.
Mulher do juiz.
(Fem. de \textunderscore juiz\textunderscore )
\section{Juiz-do-rio}
\begin{itemize}
\item {Grp. gram.:m.}
\end{itemize}
\begin{itemize}
\item {Utilização:Prov.}
\end{itemize}
O mesmo que \textunderscore pica-peixe\textunderscore , ave.
\section{Juízo}
\begin{itemize}
\item {Grp. gram.:m.}
\end{itemize}
\begin{itemize}
\item {Proveniência:(Lat. \textunderscore judicium\textunderscore )}
\end{itemize}
Acto de julgar.
Tribunal, em que se administra justiça e se discutem e sentenceiam litígios.
Fôro, jurisdicção.
Opinião: \textunderscore enganou-se no seu juizo\textunderscore .
Bom senso: \textunderscore pessôa de juizo\textunderscore .
Apreciação: \textunderscore formar juizo\textunderscore .
Processo intellectual, com que se affirma a relação das ideias; prognóstico.
\section{Jujá}
\begin{itemize}
\item {Grp. gram.:m.}
\end{itemize}
\begin{itemize}
\item {Proveniência:(T. onom.?)}
\end{itemize}
Pequeno pássaro, também chamado \textunderscore rabirruiva\textunderscore  e \textunderscore pisco-ferreiro\textunderscore .
\section{Jujuba}
\begin{itemize}
\item {Grp. gram.:f.}
\end{itemize}
\begin{itemize}
\item {Proveniência:(Do gr. \textunderscore zizuphon\textunderscore )}
\end{itemize}
Arbusto, da fam. das rhamnáceas.
Fruto dessa planta, semelhante a um feijão côr de castanha, e applicado contra indigestões.
\section{Jujubeira}
\begin{itemize}
\item {Grp. gram.:f.}
\end{itemize}
O mesmo que \textunderscore jujuba\textunderscore , arbusto.
\section{Jula}
\begin{itemize}
\item {Grp. gram.:f.}
\end{itemize}
Peixe acanthopterýgio, (\textunderscore labrus julis\textunderscore ).
\section{Julata}
\begin{itemize}
\item {Grp. gram.:f.}
\end{itemize}
\begin{itemize}
\item {Utilização:Bras}
\end{itemize}
Espécie de lençol.
Espécie de tanga, usada pelos Índios de Mato-Grosso.
\section{Julavento}
\begin{itemize}
\item {Grp. gram.:m.}
\end{itemize}
\begin{itemize}
\item {Proveniência:(Do it. \textunderscore jiu\textunderscore  + \textunderscore al\textunderscore  + \textunderscore vento\textunderscore )}
\end{itemize}
O mesmo que \textunderscore sotavento\textunderscore .
\section{Julepe}
\begin{itemize}
\item {Grp. gram.:m.}
\end{itemize}
O mesmo que \textunderscore julepo\textunderscore .
\section{Julepo}
\begin{itemize}
\item {Grp. gram.:m.}
\end{itemize}
\begin{itemize}
\item {Proveniência:(Do ár. \textunderscore gulab\textunderscore )}
\end{itemize}
Bebida calmante, que tem por base algum xarope.
\section{Julgado}
\begin{itemize}
\item {Grp. gram.:m.}
\end{itemize}
\begin{itemize}
\item {Utilização:Fig.}
\end{itemize}
\begin{itemize}
\item {Proveniência:(De \textunderscore julgar\textunderscore )}
\end{itemize}
Coisa julgada.
Divisão territorial, em que se exerce a jurisdicção de um juíz ordinário ou de um juiz municipal.
\textunderscore Passar em julgado\textunderscore , diz-se do pleito ou acção judicial, sôbre que já houve julgamento, sem que delle appellassem ou recorressem os litigantes.
Diz-se, em geral, de qualquer assumpto ou questão, sôbre que já não há dúvidas.
\section{Julgador}
\begin{itemize}
\item {Grp. gram.:m.  e  adj.}
\end{itemize}
Aquelle que julga; juíz.
Apreciador: \textunderscore mau julgador dos próprios actos\textunderscore .
\section{Julgajul}
\begin{itemize}
\item {Grp. gram.:m.}
\end{itemize}
\begin{itemize}
\item {Utilização:ant.}
\end{itemize}
\begin{itemize}
\item {Utilização:Pop.}
\end{itemize}
\begin{itemize}
\item {Proveniência:(Do lat. \textunderscore judicat\textunderscore  + \textunderscore jure\textunderscore )}
\end{itemize}
O mesmo que \textunderscore juiz\textunderscore .
\section{Julgamento}
\begin{itemize}
\item {Grp. gram.:m.}
\end{itemize}
Acto ou effeito de julgar.
Exame; apreciação.
Sentença judicial.
\section{Julgar}
\begin{itemize}
\item {Grp. gram.:v. t.}
\end{itemize}
\begin{itemize}
\item {Grp. gram.:V. i.}
\end{itemize}
\begin{itemize}
\item {Proveniência:(Do lat. \textunderscore judicare\textunderscore )}
\end{itemize}
Sentencear.
Resolver como juíz ou como árbitro.
Avaliar.
Conjecturar; entender: \textunderscore julgo que tens razão\textunderscore .
Formar conceito sôbre alguma coisa.
Lavrar ou pronunciar sentença.
\section{Julgo}
\begin{itemize}
\item {Grp. gram.:m.}
\end{itemize}
\begin{itemize}
\item {Utilização:Ant.}
\end{itemize}
\begin{itemize}
\item {Proveniência:(De \textunderscore julgar\textunderscore )}
\end{itemize}
O mesmo que \textunderscore julgamento\textunderscore .
\section{Julho}
\begin{itemize}
\item {Grp. gram.:m.}
\end{itemize}
\begin{itemize}
\item {Proveniência:(Do lat. \textunderscore Julius\textunderscore , n. p.)}
\end{itemize}
Sétimo mês do anno, segundo o calendário moderno, e entre os antigos Romanos.
\section{Juliana}
\begin{itemize}
\item {Grp. gram.:f.}
\end{itemize}
\begin{itemize}
\item {Grp. gram.:F.  e  adj.}
\end{itemize}
\begin{itemize}
\item {Utilização:Pop.}
\end{itemize}
Peixe gádida, (\textunderscore molua elongata\textunderscore ).
Sopa preparada com várias espécies de legumes, cortados miúdamente.
Planta, o mesmo que \textunderscore hésperis\textunderscore .
O mesmo que \textunderscore água-pé\textunderscore ^1.
\section{Juliânia}
\begin{itemize}
\item {Grp. gram.:f.}
\end{itemize}
\begin{itemize}
\item {Proveniência:(De \textunderscore Juliano\textunderscore , n. p.)}
\end{itemize}
Gênero de arbustos do México.
\section{Juliano}
\begin{itemize}
\item {Grp. gram.:adj.}
\end{itemize}
\begin{itemize}
\item {Proveniência:(Lat. \textunderscore julianus\textunderscore )}
\end{itemize}
Relativo á reforma chronológica, mandada fazer por Júlio César.
Diz-se do anno commum de 365 dias ou do bissexto de 366.
E diz-se do período de 7980 annos, resultante da multiplicação do cyclo solar, do cyclo lunar e do cyclo de indicção.
\section{Júlio}
\begin{itemize}
\item {Grp. gram.:m.}
\end{itemize}
\begin{itemize}
\item {Utilização:Phýs.}
\end{itemize}
\begin{itemize}
\item {Proveniência:(De \textunderscore Joule\textunderscore , n. p.)}
\end{itemize}
Unidade da medida do trabalho eléctrico, a qual é equivalente a um vóltio por um colômbio.
\section{Júlio}
\begin{itemize}
\item {Grp. gram.:adj.}
\end{itemize}
Relativo a Júlio César. Cf. Castilho, \textunderscore Geòrgicas\textunderscore , 87.
\section{Jumas}
\begin{itemize}
\item {Grp. gram.:m. pl.}
\end{itemize}
Indígenas brasileiros da região do Amazonas.
\section{Jumbeba}
\begin{itemize}
\item {Grp. gram.:f.}
\end{itemize}
Figueira da África setentrional.
\section{Jumenta}
\begin{itemize}
\item {Grp. gram.:f.}
\end{itemize}
A fêmea do jumento.
\section{Jumentada}
\begin{itemize}
\item {Grp. gram.:f.}
\end{itemize}
\begin{itemize}
\item {Proveniência:(De \textunderscore jumento\textunderscore )}
\end{itemize}
Asneira; parvoíce.
\section{Jumental}
\begin{itemize}
\item {Grp. gram.:adj.}
\end{itemize}
Relativo a jumento; asinino.
\section{Jumentão}
\begin{itemize}
\item {Grp. gram.:m.}
\end{itemize}
Jumento grande. Cf. Macedo, \textunderscore Burros\textunderscore , 146.
\section{Jumentice}
\begin{itemize}
\item {Grp. gram.:f.}
\end{itemize}
O mesmo que \textunderscore jumentada\textunderscore . Cf. Macedo, \textunderscore Burros\textunderscore , 51.
\section{Jumentico}
\begin{itemize}
\item {Grp. gram.:m.}
\end{itemize}
Jumento pequeno.
\section{Jumentil}
\begin{itemize}
\item {Grp. gram.:adj.}
\end{itemize}
O mesmo que \textunderscore jumental\textunderscore .
\section{Jumento}
\begin{itemize}
\item {Grp. gram.:m.}
\end{itemize}
\begin{itemize}
\item {Proveniência:(Lat. \textunderscore jumentum\textunderscore )}
\end{itemize}
O mesmo que \textunderscore burro\textunderscore ^1.
\section{Jumusjungil}
\begin{itemize}
\item {Grp. gram.:m.}
\end{itemize}
Planta trepadeira da Guiné, applicável contra o rheumatismo.
\section{Junça}
\begin{itemize}
\item {Grp. gram.:f.}
\end{itemize}
\begin{itemize}
\item {Proveniência:(Do b. lat. \textunderscore juntia\textunderscore )}
\end{itemize}
Planta cyperácea, (\textunderscore cyperus esculentus\textunderscore ).
\section{Juncáceas}
\begin{itemize}
\item {Grp. gram.:f. pl.}
\end{itemize}
Família de plantas, que tem por typo o junco.
\section{Juncada}
\begin{itemize}
\item {Grp. gram.:f.}
\end{itemize}
\begin{itemize}
\item {Proveniência:(De \textunderscore junco\textunderscore )}
\end{itemize}
Grande porção de juncos.
Pancada com junco.
Porção de juncos, fôlhas ou flôres, que se espalham nas ruas ou nas igrejas, em occasião de festa.
\section{Junça-de-cheiro}
\begin{itemize}
\item {Grp. gram.:f.}
\end{itemize}
\begin{itemize}
\item {Utilização:T. da Bairrada}
\end{itemize}
Planta, o mesmo que \textunderscore albafor\textunderscore .
\section{Juncagíneas}
\begin{itemize}
\item {Grp. gram.:f. pl.}
\end{itemize}
O mesmo que \textunderscore juncáceas\textunderscore .
\section{Juncal}
\begin{itemize}
\item {Grp. gram.:m.}
\end{itemize}
Terreno, em que crescem juncos.
\section{Junçal}
\begin{itemize}
\item {Grp. gram.:m.}
\end{itemize}
Terreno, onde cresce junça.
\section{Juncão}
\begin{itemize}
\item {Grp. gram.:m.}
\end{itemize}
\begin{itemize}
\item {Utilização:Ant.}
\end{itemize}
Grande junco, embarcação asiática.
\section{Junção}
\begin{itemize}
\item {Grp. gram.:f.}
\end{itemize}
\begin{itemize}
\item {Proveniência:(Lat. \textunderscore junctio\textunderscore )}
\end{itemize}
Acto ou effeito de juntar.
Ponto, em que duas ou mais coisas coincidem ou se juntam; confluência.
\section{Juncar}
\begin{itemize}
\item {Grp. gram.:v. t.}
\end{itemize}
\begin{itemize}
\item {Utilização:Ext.}
\end{itemize}
\begin{itemize}
\item {Utilização:Fig.}
\end{itemize}
Cobrir com juncos: \textunderscore juncar as ruas\textunderscore .
Cobrir de fôlhas ou flôres.
Espalhar; alastrar.
\section{Juncção}
\begin{itemize}
\item {Grp. gram.:f.}
\end{itemize}
\begin{itemize}
\item {Proveniência:(Lat. \textunderscore junctio\textunderscore )}
\end{itemize}
Acto ou effeito de juntar.
Ponto, em que duas ou mais coisas coincidem ou se juntam; confluência.
\section{Junceira}
\begin{itemize}
\item {Grp. gram.:f.}
\end{itemize}
\begin{itemize}
\item {Utilização:Pop.}
\end{itemize}
O mesmo que \textunderscore junça\textunderscore .
\section{Junco}
\begin{itemize}
\item {Grp. gram.:m.}
\end{itemize}
\begin{itemize}
\item {Proveniência:(Lat. \textunderscore juncus\textunderscore )}
\end{itemize}
Gênero de plantas delgadas, lisas e flexíveis, que crescem em terrenos húmidos e dentro de água.
Chibata.
Bengala de junco.
\section{Junco}
\begin{itemize}
\item {Grp. gram.:m.}
\end{itemize}
\begin{itemize}
\item {Proveniência:(Do chin. \textunderscore jonk\textunderscore )}
\end{itemize}
Pequena embarcação oriental.
\section{Juncoso}
\begin{itemize}
\item {Grp. gram.:adj.}
\end{itemize}
\begin{itemize}
\item {Proveniência:(Lat. \textunderscore juncosus\textunderscore )}
\end{itemize}
Em que há muitos juncos.
\section{Junçoso}
\begin{itemize}
\item {Grp. gram.:adj.}
\end{itemize}
Diz-se do terreno, onde crescem junças.
\section{Jundaí}
\begin{itemize}
\item {Grp. gram.:m.}
\end{itemize}
\begin{itemize}
\item {Utilização:Bras}
\end{itemize}
Espécie de aranha.
\section{Jundiá}
\begin{itemize}
\item {Grp. gram.:m.}
\end{itemize}
\begin{itemize}
\item {Utilização:Bras}
\end{itemize}
Planta labiada.
Peixe de água doce.
(Do tupi)
\section{Jundiaíba}
\begin{itemize}
\item {Grp. gram.:f.}
\end{itemize}
\begin{itemize}
\item {Utilização:Bras}
\end{itemize}
Árvore silvestre.
\section{Jungir}
\begin{itemize}
\item {Grp. gram.:v. t.}
\end{itemize}
\begin{itemize}
\item {Proveniência:(Lat. \textunderscore jungere\textunderscore )}
\end{itemize}
Ligar por meio da canga; emparelhar.
Submeter.
Atar; unir.
\section{Jungo}
\begin{itemize}
\item {Grp. gram.:m.}
\end{itemize}
Ave trepadora da África occidental.
\section{Jungo}
\begin{itemize}
\item {Grp. gram.:m.}
\end{itemize}
\begin{itemize}
\item {Utilização:Prov.}
\end{itemize}
\begin{itemize}
\item {Utilização:trasm.}
\end{itemize}
O mesmo que \textunderscore jugo\textunderscore .
\section{Junguer}
\begin{itemize}
\item {Grp. gram.:v. t.}
\end{itemize}
\begin{itemize}
\item {Utilização:Prov.}
\end{itemize}
\begin{itemize}
\item {Utilização:trasm.}
\end{itemize}
O mesmo que \textunderscore junguir\textunderscore .
\section{Junguir}
\begin{itemize}
\item {Grp. gram.:v. t.}
\end{itemize}
\begin{itemize}
\item {Utilização:Pop.}
\end{itemize}
O mesmo que \textunderscore jungir\textunderscore .
(Por \textunderscore juguir\textunderscore , de \textunderscore jugo\textunderscore )
\section{Junhães}
\begin{itemize}
\item {Grp. gram.:f.}
\end{itemize}
Antiga variedade de pêra portuguesa, hoje desconhecida.
\section{Junho}
\begin{itemize}
\item {Grp. gram.:m.}
\end{itemize}
\begin{itemize}
\item {Proveniência:(Lat. \textunderscore junius\textunderscore )}
\end{itemize}
Sexto mês do anno.
\section{Junior}
\begin{itemize}
\item {fónica:júniòr}
\end{itemize}
\begin{itemize}
\item {Grp. gram.:adj.}
\end{itemize}
\begin{itemize}
\item {Grp. gram.:M.}
\end{itemize}
\begin{itemize}
\item {Proveniência:(T. lat.)}
\end{itemize}
Mais moço.
Velocipedista novato, que nunca entrou em jogos de corridas, ou que nesses jogos nunca foi premiado, ou que tem tido prêmios insignificantes, chamando-se, neste último caso, \textunderscore junior forte\textunderscore , e \textunderscore junior fraco\textunderscore  nos dois primeiros.
\section{Junipena}
\begin{itemize}
\item {Grp. gram.:f.}
\end{itemize}
\begin{itemize}
\item {Proveniência:(Do lat. \textunderscore juniperus\textunderscore )}
\end{itemize}
Princípio diurético do zimbro.--Dir-se-ia melhor \textunderscore juniperina\textunderscore .
\section{Juniperáceas}
\begin{itemize}
\item {Grp. gram.:f. pl.}
\end{itemize}
\begin{itemize}
\item {Proveniência:(De \textunderscore juniperáceo\textunderscore )}
\end{itemize}
Família de plantas, separadas das coníferas, que têm por typo o junípero.
\section{Juniperáceo}
\begin{itemize}
\item {Grp. gram.:adj.}
\end{itemize}
Relativo ou semelhante ao junípero.
\section{Juniperíneas}
\begin{itemize}
\item {Grp. gram.:f.}
\end{itemize}
O mesmo que \textunderscore juniperáceas\textunderscore .
\section{Junípero}
\begin{itemize}
\item {Grp. gram.:m.}
\end{itemize}
\begin{itemize}
\item {Proveniência:(Lat. \textunderscore juniperus\textunderscore )}
\end{itemize}
O mesmo que \textunderscore zimbro\textunderscore ^1.
\section{Junonal}
\begin{itemize}
\item {Grp. gram.:adj.}
\end{itemize}
\begin{itemize}
\item {Proveniência:(Lat. \textunderscore junonalis\textunderscore )}
\end{itemize}
Relativo a Juno. Cf. Castilho, \textunderscore Fastos\textunderscore , III, 99.
\section{Junónias}
\begin{itemize}
\item {Grp. gram.:f.}
\end{itemize}
\begin{itemize}
\item {Proveniência:(Do lat. \textunderscore junonius\textunderscore )}
\end{itemize}
Antigas festas em honra de Juno, deusa dos casamentos e partos.
\section{Junqueira}
\begin{itemize}
\item {Grp. gram.:f.}
\end{itemize}
\begin{itemize}
\item {Grp. gram.:M.}
\end{itemize}
\begin{itemize}
\item {Utilização:Bras}
\end{itemize}
O mesmo que \textunderscore juncal\textunderscore .
Planta convolvulácea do Brasil.
Espécie de boi, em Goiás.
\section{Junqueiro}
\begin{itemize}
\item {Grp. gram.:m.}
\end{itemize}
Raça de bois, o mesmo que \textunderscore franqueiro\textunderscore .
\section{Junquilho}
\begin{itemize}
\item {Grp. gram.:m.}
\end{itemize}
Planta amaryllídea, (\textunderscore narcisus jonquilla\textunderscore ).
Flôr desta planta.
(Cast. \textunderscore junquillo\textunderscore )
\section{Junta}
\begin{itemize}
\item {Grp. gram.:f.}
\end{itemize}
Nome de várias plantas brasileiras.
\section{Junta}
\begin{itemize}
\item {Grp. gram.:f.}
\end{itemize}
\begin{itemize}
\item {Utilização:Geol.}
\end{itemize}
\begin{itemize}
\item {Proveniência:(De \textunderscore junto\textunderscore )}
\end{itemize}
Ligação de ossos, que se articulam.
Articulação.
Ponto ou superfície, em que adherem entre si dois objectos.
Grupo de pessôas, assembleia.
Conferência entre facultativos.
Commissão.
Corporação administrativa ou consultiva.
Dois bois, que emparelham sob a mesma canga.
O mesmo que \textunderscore diáclase\textunderscore .
\section{Juntada}
\begin{itemize}
\item {Grp. gram.:f.}
\end{itemize}
\begin{itemize}
\item {Utilização:For.}
\end{itemize}
\begin{itemize}
\item {Proveniência:(De \textunderscore juntar\textunderscore )}
\end{itemize}
Termo de juncção, num processo forense.
\section{Juntadamente}
\begin{itemize}
\item {Grp. gram.:adv.}
\end{itemize}
\begin{itemize}
\item {Utilização:Des.}
\end{itemize}
O mesmo que \textunderscore juntamente\textunderscore .
\section{Juntadeira}
\begin{itemize}
\item {Grp. gram.:f.}
\end{itemize}
O mesmo que \textunderscore ajuntadeira\textunderscore .
\section{Juntamente}
\begin{itemize}
\item {Grp. gram.:adv.}
\end{itemize}
\begin{itemize}
\item {Proveniência:(De \textunderscore junto\textunderscore )}
\end{itemize}
Ligadamente; simultaneamente.
Também.
De companhia.
\section{Juntamento}
\begin{itemize}
\item {Grp. gram.:m.}
\end{itemize}
\begin{itemize}
\item {Utilização:Des.}
\end{itemize}
O mesmo que \textunderscore ajuntamento\textunderscore . Cf. \textunderscore Auto de S. António\textunderscore .
\section{Juntar}
\begin{itemize}
\item {Grp. gram.:v. t.}
\end{itemize}
\begin{itemize}
\item {Utilização:Carp.}
\end{itemize}
O mesmo que \textunderscore ajuntar\textunderscore .
Alisar com junteira os lados de (tábuas), para que estas se adaptem perfeitamente.
Coser, ligando, as peças superiores do calçado.
\section{Junteira}
\begin{itemize}
\item {Grp. gram.:f.}
\end{itemize}
\begin{itemize}
\item {Proveniência:(De \textunderscore junta\textunderscore ^2)}
\end{itemize}
Espécie de plaina, com que se abrem os encaixes ou juntas das tábuas.
Planta, da fam. das commeliáceas.
Ao lado; perto.
\section{Junteiro}
\begin{itemize}
\item {Grp. gram.:m.}
\end{itemize}
\begin{itemize}
\item {Utilização:Polit.}
\end{itemize}
Partidário da Junta do Pôrto, (1846).
\section{Junto}
\begin{itemize}
\item {Grp. gram.:adv.}
\end{itemize}
\begin{itemize}
\item {Proveniência:(Lat. \textunderscore junctus\textunderscore )}
\end{itemize}
Juntamente.
Ao pé; ao lado; perto: \textunderscore passou junto de mim\textunderscore .
\section{Juntoira}
\begin{itemize}
\item {Grp. gram.:f.}
\end{itemize}
Pedra, que vai de uma á outra face da parede.
Pedra, que resái de uma parede, para se embeber noutra parede contígua.
O mesmo que \textunderscore junteira\textunderscore .
(Cp. \textunderscore juntoiro\textunderscore )
\section{Juntoiro}
\begin{itemize}
\item {Grp. gram.:m.}
\end{itemize}
\begin{itemize}
\item {Proveniência:(De \textunderscore junto\textunderscore )}
\end{itemize}
O mesmo que \textunderscore juntoira\textunderscore .
\section{Juntoura}
\begin{itemize}
\item {Grp. gram.:f.}
\end{itemize}
Pedra, que vai de uma á outra face da parede.
Pedra, que resái de uma parede, para se embeber noutra parede contígua.
O mesmo que \textunderscore junteira\textunderscore .
(Cp. \textunderscore juntoiro\textunderscore )
\section{Juntouro}
\begin{itemize}
\item {Grp. gram.:m.}
\end{itemize}
\begin{itemize}
\item {Proveniência:(De \textunderscore junto\textunderscore )}
\end{itemize}
O mesmo que \textunderscore juntoira\textunderscore .
\section{Juntura}
\begin{itemize}
\item {Grp. gram.:f.}
\end{itemize}
\begin{itemize}
\item {Proveniência:(Lat. \textunderscore junctura\textunderscore )}
\end{itemize}
Junta, articulação; ligação.
\section{Juó}
\begin{itemize}
\item {Grp. gram.:m.}
\end{itemize}
\begin{itemize}
\item {Utilização:Bras}
\end{itemize}
Ave noctívaga, cujo canto parece exprimir o seu nome.
\section{Jupati}
\begin{itemize}
\item {Grp. gram.:m.}
\end{itemize}
(V. \textunderscore jataí\textunderscore ^1)
\section{Jupeba}
\begin{itemize}
\item {Grp. gram.:f.}
\end{itemize}
(V.jurubeba)
\section{Jupiá}
\begin{itemize}
\item {Grp. gram.:m.}
\end{itemize}
\begin{itemize}
\item {Utilização:Bras}
\end{itemize}
Redemoínho de água num rio.
Espécie de voragem.
\section{Jupiede}
\begin{itemize}
\item {Grp. gram.:m.}
\end{itemize}
Planta resedácea da Índia oriental.
\section{Júpiter}
\begin{itemize}
\item {Grp. gram.:m.}
\end{itemize}
\begin{itemize}
\item {Proveniência:(Lat. \textunderscore Jupiter\textunderscore , n. p.)}
\end{itemize}
Grande planeta, muito brilhante, entre Marte e Saturno.
\section{Jupiteriano}
\begin{itemize}
\item {Grp. gram.:adj.}
\end{itemize}
\begin{itemize}
\item {Utilização:Neol.}
\end{itemize}
\begin{itemize}
\item {Proveniência:(De \textunderscore Júpiter\textunderscore , n. p. myth.)}
\end{itemize}
Muito altivo; imperioso; que tem carácter dominador.
\section{Jupuas}
\begin{itemize}
\item {Grp. gram.:m. pl.}
\end{itemize}
Índios selvagens das margens do Apaporis, no Brasil.
\section{Juqueira-açu}
\begin{itemize}
\item {Grp. gram.:m.}
\end{itemize}
Árvore leguminosa do Pará.
\section{Juquiá}
\begin{itemize}
\item {Grp. gram.:m.}
\end{itemize}
\begin{itemize}
\item {Utilização:Bras}
\end{itemize}
Espécie de nassa, aberta nas duas extremidades.
\section{Juquiri}
\begin{itemize}
\item {Grp. gram.:m.}
\end{itemize}
Arbusto leguminoso do Brasil.
\section{Juquirionano}
\begin{itemize}
\item {Grp. gram.:m.}
\end{itemize}
(V.bonduque)
\section{Juquis}
\begin{itemize}
\item {Grp. gram.:m. pl.}
\end{itemize}
Indígenas brasileiros da região do Amazonas.
\section{Jura}
\begin{itemize}
\item {Grp. gram.:f.}
\end{itemize}
Acto de jurar: \textunderscore fazer uma jura\textunderscore .
Praga: \textunderscore dizer juras\textunderscore .
\section{Juradia}
\begin{itemize}
\item {Grp. gram.:f.}
\end{itemize}
\begin{itemize}
\item {Utilização:Ant.}
\end{itemize}
Offício do jurado.
\section{Jurado}
\begin{itemize}
\item {Grp. gram.:adj.}
\end{itemize}
\begin{itemize}
\item {Grp. gram.:M.}
\end{itemize}
\begin{itemize}
\item {Utilização:Ant.}
\end{itemize}
\begin{itemize}
\item {Utilização:Prov.}
\end{itemize}
\begin{itemize}
\item {Utilização:alent.}
\end{itemize}
Que prestou juramento.
Declarado; inconciliável: \textunderscore inimigo jurado\textunderscore .
Membro de um júry judicial.
Louvado, perito.
Espécie de espátula, com que os macobios mexem as migas, e em que assentam as multas que, segundo o seu regulamento, são impostas a companheiros delinquentes.
\section{Jurador}
\begin{itemize}
\item {Grp. gram.:m.  e  adj.}
\end{itemize}
\begin{itemize}
\item {Proveniência:(Lat. \textunderscore jurator\textunderscore )}
\end{itemize}
O que jura.
\section{Juraico}
\begin{itemize}
\item {Grp. gram.:adj.}
\end{itemize}
O mesmo que \textunderscore jurássico\textunderscore .
\section{Juramentar}
\begin{itemize}
\item {Grp. gram.:v. t.}
\end{itemize}
O mesmo que \textunderscore ajuramentar\textunderscore .
\section{Juramenteiro}
\begin{itemize}
\item {Grp. gram.:adj.}
\end{itemize}
Que faz juramento amiúde. Cf. Castilho, \textunderscore Fastos\textunderscore , III, 471.
\section{Juramento}
\begin{itemize}
\item {Grp. gram.:m.}
\end{itemize}
\begin{itemize}
\item {Proveniência:(Lat. \textunderscore juramentum\textunderscore )}
\end{itemize}
Acto de jurar.
\textunderscore Juramento de alma\textunderscore , o juramento decisório, de que fala o \textunderscore Código Civíl\textunderscore , art. 2522.
\textunderscore Juramento de calúmnía\textunderscore , o que era feito pelo participante ou queixoso de um crime, asseverando que não calumniava. Cf. \textunderscore Noviss. Ref. Jud.\textunderscore , art. 874.
\section{Jurami!}
\begin{itemize}
\item {Grp. gram.:interj.}
\end{itemize}
\begin{itemize}
\item {Utilização:Ant.}
\end{itemize}
(Contr. da loc. \textunderscore juro a mim\textunderscore ). Cf. \textunderscore Eufrosina\textunderscore , 86; Arn. Gama, \textunderscore Bailio\textunderscore .
\section{Jurana}
\begin{itemize}
\item {Grp. gram.:f.}
\end{itemize}
\begin{itemize}
\item {Utilização:Bras}
\end{itemize}
Espécie de maçaranduba.
\section{Jurão}
\begin{itemize}
\item {Grp. gram.:m.}
\end{itemize}
\begin{itemize}
\item {Utilização:Bras}
\end{itemize}
Casa erguida em estacarias, para resistir ás enchentes.
(Cp. \textunderscore jurau\textunderscore )
\section{Jurar}
\begin{itemize}
\item {Grp. gram.:v. t.}
\end{itemize}
\begin{itemize}
\item {Grp. gram.:V. i.}
\end{itemize}
\begin{itemize}
\item {Proveniência:(Lat. \textunderscore jurare\textunderscore )}
\end{itemize}
Declarar solennemente.
Assegurar; afiançar.
Affirmar ou prometer, invocando-se o nome de Deus ou de uma coisa que se reputa sagrada ou venerável.
Protestar.
Invocar.
Prestar juramento.
Praguejar.
\section{Jurará}
\begin{itemize}
\item {Grp. gram.:m.}
\end{itemize}
\begin{itemize}
\item {Utilização:Bras}
\end{itemize}
Espécie de cágado.
\section{Jurássico}
\begin{itemize}
\item {Grp. gram.:adj.}
\end{itemize}
\begin{itemize}
\item {Utilização:Geol.}
\end{itemize}
\begin{itemize}
\item {Proveniência:(Do lat. \textunderscore Jurassus\textunderscore , n. p. do monte Jura)}
\end{itemize}
Diz-se de um dos terrenos da série secundária ou mesozoica.
\section{Juratório}
\begin{itemize}
\item {Grp. gram.:adj.}
\end{itemize}
\begin{itemize}
\item {Proveniência:(De \textunderscore jurar\textunderscore )}
\end{itemize}
Relativo a juramento.
\section{Jurau}
\begin{itemize}
\item {Grp. gram.:m.}
\end{itemize}
\begin{itemize}
\item {Utilização:Bras}
\end{itemize}
O mesmo que \textunderscore girau\textunderscore .
\section{Jurdição}
\begin{itemize}
\item {Grp. gram.:f.}
\end{itemize}
\begin{itemize}
\item {Utilização:Pop.}
\end{itemize}
O mesmo que \textunderscore jurisdicção\textunderscore . Cf. \textunderscore Eufrosina\textunderscore , 165.
\section{Jurema}
\begin{itemize}
\item {Grp. gram.:f.}
\end{itemize}
Árvore leguminosa do Brasil.
\section{Jurepeba}
\begin{itemize}
\item {Grp. gram.:f.}
\end{itemize}
O mesmo que \textunderscore juá\textunderscore .
\section{Juribeba}
\begin{itemize}
\item {Grp. gram.:f.}
\end{itemize}
(V.jurubeba)
\section{Juridicamente}
\begin{itemize}
\item {Grp. gram.:adv.}
\end{itemize}
De modo jurídico.
Em harmonia com as prescripções do Direito.
\section{Juridicidade}
\begin{itemize}
\item {Grp. gram.:f.}
\end{itemize}
\begin{itemize}
\item {Utilização:Bras}
\end{itemize}
\begin{itemize}
\item {Utilização:Neol.}
\end{itemize}
Qualidade de jurídico. Cf. \textunderscore Jornal do Comm.\textunderscore , do Rio, de 2-X-904.
\section{Jurídico}
\begin{itemize}
\item {Grp. gram.:adj.}
\end{itemize}
\begin{itemize}
\item {Proveniência:(Lat. \textunderscore juridicus\textunderscore )}
\end{itemize}
Relativo ao Direito.
Conforme aos princípios de Direito.
\section{Jurimáguas}
\begin{itemize}
\item {Grp. gram.:m. pl.}
\end{itemize}
Tríbo de selvagens americanos, no Peru.
\section{Jurimanás}
\begin{itemize}
\item {Grp. gram.:m. pl.}
\end{itemize}
Aguerrida tríbo de Índios, no Alto Amazonas.
\section{Jurínea}
\begin{itemize}
\item {Grp. gram.:f.}
\end{itemize}
Gênero de plantas synanthéreas.
\section{Jurinite}
\begin{itemize}
\item {Grp. gram.:f.}
\end{itemize}
\begin{itemize}
\item {Utilização:Miner.}
\end{itemize}
Óxydo de titano.
\section{Juris}
\begin{itemize}
\item {Grp. gram.:m. pl.}
\end{itemize}
Índios selvagens das margens do Apaporis, no Brasil.
Nome de outra tríbo, nas margens do Japurá.
\section{Jurisconsulto}
\begin{itemize}
\item {Grp. gram.:m.}
\end{itemize}
\begin{itemize}
\item {Proveniência:(Lat. \textunderscore jurisconsultus\textunderscore )}
\end{itemize}
Aquelle que é versado em leis; advogado.
\section{Jurisdição}
\begin{itemize}
\item {Grp. gram.:f.}
\end{itemize}
\begin{itemize}
\item {Proveniência:(Lat. \textunderscore jurisdictio\textunderscore )}
\end{itemize}
Faculdade de applicar as leis e de conhecer e punir as infrações delas.
Competência; alçada.
Influência; poder.
\section{Jurisdicção}
\begin{itemize}
\item {Grp. gram.:f.}
\end{itemize}
\begin{itemize}
\item {Proveniência:(Lat. \textunderscore jurisdictio\textunderscore )}
\end{itemize}
Faculdade de applicar as leis e de conhecer e punir as infracções dellas.
Competência; alçada.
Influência; poder.
\section{Jurisdiccional}
\begin{itemize}
\item {Grp. gram.:adj.}
\end{itemize}
\begin{itemize}
\item {Proveniência:(Do lat. \textunderscore jurisdictio\textunderscore )}
\end{itemize}
Relativo á jurisdicção.
\section{Jurisdicional}
\begin{itemize}
\item {Grp. gram.:adj.}
\end{itemize}
\begin{itemize}
\item {Proveniência:(Do lat. \textunderscore jurisdictio\textunderscore )}
\end{itemize}
Relativo á jurisdicção.
\section{Jurisperícia}
\begin{itemize}
\item {Grp. gram.:f.}
\end{itemize}
\begin{itemize}
\item {Proveniência:(Lat. \textunderscore jurisperitia\textunderscore )}
\end{itemize}
Qualidade de jurisperito.
\section{Jurisperito}
\begin{itemize}
\item {Grp. gram.:m.}
\end{itemize}
\begin{itemize}
\item {Proveniência:(Lat. \textunderscore jurisperitus\textunderscore )}
\end{itemize}
O mesmo que \textunderscore jurisconsulto\textunderscore .
\section{Jurisprudência}
\begin{itemize}
\item {Grp. gram.:f.}
\end{itemize}
\begin{itemize}
\item {Proveniência:(Lat. \textunderscore jurisprudentia\textunderscore )}
\end{itemize}
Sciência da legislação e do Direito.
\section{Júri}
\begin{itemize}
\item {Grp. gram.:m.}
\end{itemize}
\begin{itemize}
\item {Proveniência:(Ingl. \textunderscore jury\textunderscore )}
\end{itemize}
Cidadãos, convocados em nome da lei, para julgamento de uma causa criminal ou cível.
Comissão, encarregada de julgar do mérito de alguém ou de alguma coisa.
\section{Jurista}
\begin{itemize}
\item {Grp. gram.:m.}
\end{itemize}
Aquelle que empresta dinheiro a juro.
Aquelle que possue títulos da dívida pública e recebe os respectivos juros.
\section{Jurista}
\begin{itemize}
\item {Grp. gram.:m.}
\end{itemize}
\begin{itemize}
\item {Proveniência:(Do lat. \textunderscore jus\textunderscore , \textunderscore juris\textunderscore )}
\end{itemize}
O mesmo que \textunderscore jurisconsulto\textunderscore .
\section{Juriti}
\begin{itemize}
\item {Grp. gram.:m.}
\end{itemize}
\begin{itemize}
\item {Utilização:Bras}
\end{itemize}
Espécie de rôla ou variedade de pomba.
\section{Juro}
\begin{itemize}
\item {Grp. gram.:m.}
\end{itemize}
\begin{itemize}
\item {Utilização:Ant.}
\end{itemize}
\begin{itemize}
\item {Utilização:Fig.}
\end{itemize}
\begin{itemize}
\item {Grp. gram.:Loc. adv.}
\end{itemize}
\begin{itemize}
\item {Utilização:fam.}
\end{itemize}
\begin{itemize}
\item {Proveniência:(Do lat. \textunderscore jus\textunderscore , \textunderscore juris\textunderscore )}
\end{itemize}
Lucro de dinheiro emprestado.
Jus: \textunderscore fidalgo de juro e herdade\textunderscore .
Recompensa.
\textunderscore A razão de juros\textunderscore , sem juizo, tresloucadamente.
\textunderscore De juro e herdade\textunderscore , diz-se dos brasões e títulos nobiliárchicos, que vieram por herança e direito antigo.
\section{Juru}
\begin{itemize}
\item {Grp. gram.:m.}
\end{itemize}
\begin{itemize}
\item {Utilização:Bras}
\end{itemize}
Espécie de papagaio.
\section{Jurubaca}
\begin{itemize}
\item {Grp. gram.:m.}
\end{itemize}
Antigo intérprete chinês. Cf. \textunderscore Peregrinação\textunderscore , CXL.
\section{Jurubeba}
\begin{itemize}
\item {Grp. gram.:f.}
\end{itemize}
Planta solânea da América equatorial.
(Talvez do tupi)
\section{Jurucuá}
\begin{itemize}
\item {Grp. gram.:f.}
\end{itemize}
Tartaruga do Brasil.
\section{Jurujuba}
\begin{itemize}
\item {Grp. gram.:f.}
\end{itemize}
\begin{itemize}
\item {Utilização:Bras}
\end{itemize}
O mesmo que \textunderscore verbena\textunderscore .
\section{Jurumbeba}
\begin{itemize}
\item {Grp. gram.:f.}
\end{itemize}
\begin{itemize}
\item {Utilização:Bras. do Rio}
\end{itemize}
Espécie de cacto.
(Alter. do tupi \textunderscore ururumbeba\textunderscore )
\section{Jurumu}
\begin{itemize}
\item {Grp. gram.:m.}
\end{itemize}
O mesmo que \textunderscore girimu\textunderscore .
\section{Jurumum}
\begin{itemize}
\item {Grp. gram.:m.}
\end{itemize}
\begin{itemize}
\item {Utilização:Bras}
\end{itemize}
O mesmo que \textunderscore girimu\textunderscore .
\section{Juruna}
\begin{itemize}
\item {Grp. gram.:m.}
\end{itemize}
\begin{itemize}
\item {Utilização:Bras}
\end{itemize}
Espécie de macaco do Amazonas.
\section{Jurupango}
\begin{itemize}
\item {Grp. gram.:m.}
\end{itemize}
\begin{itemize}
\item {Utilização:Ant.}
\end{itemize}
Pequena embarcação asiática. Cf. \textunderscore Peregrinação\textunderscore , XIV.
\section{Jurupari}
\begin{itemize}
\item {Grp. gram.:m.}
\end{itemize}
\begin{itemize}
\item {Utilização:Bras}
\end{itemize}
Espécie de macaco.
O mesmo que \textunderscore jeropari\textunderscore .
\section{Jurupari-bóia}
\begin{itemize}
\item {Grp. gram.:m.}
\end{itemize}
Serpente do Brasil.
\section{Jurupema}
\begin{itemize}
\item {Grp. gram.:f.}
\end{itemize}
\begin{itemize}
\item {Utilização:Bras}
\end{itemize}
O mesmo que \textunderscore urupema\textunderscore .
\section{Jurupencu}
\begin{itemize}
\item {Grp. gram.:m.}
\end{itemize}
\begin{itemize}
\item {Utilização:Bras}
\end{itemize}
Peixe de água doce.
\section{Jurupetinga}
\begin{itemize}
\item {Grp. gram.:f.}
\end{itemize}
Espécie de jurubeba.
\section{Jurupoca}
\begin{itemize}
\item {Grp. gram.:f.}
\end{itemize}
\begin{itemize}
\item {Utilização:Bras}
\end{itemize}
Peixe de água doce.
\section{Jururu}
\begin{itemize}
\item {Grp. gram.:adj.}
\end{itemize}
\begin{itemize}
\item {Utilização:Bras}
\end{itemize}
Tristonho; melancólico.
(Do tupi)
\section{Jurutanhi}
\begin{itemize}
\item {Grp. gram.:m.}
\end{itemize}
Ave nocturna do Brasil.
\section{Juruté}
\begin{itemize}
\item {Grp. gram.:m.}
\end{itemize}
\begin{itemize}
\item {Utilização:Bras}
\end{itemize}
Planta fructífera de San-Paulo.
\section{Juruti}
\begin{itemize}
\item {Grp. gram.:m.}
\end{itemize}
\begin{itemize}
\item {Utilização:Bras}
\end{itemize}
Ave gallinácea, o mesmo que \textunderscore juriti\textunderscore .
\section{Juruunas}
\begin{itemize}
\item {Grp. gram.:m. pl.}
\end{itemize}
\begin{itemize}
\item {Utilização:Bras}
\end{itemize}
Uma das tríbos aborígenes do Pará.
\section{Juruvá}
\begin{itemize}
\item {Grp. gram.:m.}
\end{itemize}
\begin{itemize}
\item {Utilização:Bras}
\end{itemize}
O mesmo que \textunderscore jerivá\textunderscore .
\section{Júry}
\begin{itemize}
\item {Grp. gram.:m.}
\end{itemize}
\begin{itemize}
\item {Proveniência:(Ingl. \textunderscore jury\textunderscore )}
\end{itemize}
Cidadãos, convocados em nome da lei, para julgamento de uma causa criminal ou cível.
Commissão, encarregada de julgar do mérito de alguém ou de alguma coisa.
\section{Jus}
\begin{itemize}
\item {Grp. gram.:m.}
\end{itemize}
\begin{itemize}
\item {Proveniência:(Lat. \textunderscore jus\textunderscore )}
\end{itemize}
O mesmo que \textunderscore direito\textunderscore ^2:«\textunderscore o triste jus da nossa idade\textunderscore ». Th. Ribeiro, \textunderscore D. Jaime\textunderscore . Cf. Herculano, \textunderscore Quest. Públ.\textunderscore , II, 132 e 135.
\section{Jusã}
\begin{itemize}
\item {Grp. gram.:adj.}
\end{itemize}
(fem. de \textunderscore jusão\textunderscore )
\section{Jusan}
\begin{itemize}
\item {Grp. gram.:adj.}
\end{itemize}
(fem. de \textunderscore jusão\textunderscore )
\section{Jusano}
\begin{itemize}
\item {Grp. gram.:adj.}
\end{itemize}
O mesmo que \textunderscore jusão\textunderscore .
\section{Jusante}
\begin{itemize}
\item {Grp. gram.:f.}
\end{itemize}
\begin{itemize}
\item {Grp. gram.:Loc. adv.}
\end{itemize}
\begin{itemize}
\item {Proveniência:(De \textunderscore juso\textunderscore )}
\end{itemize}
Baixa-mar, refluxo da maré.
\textunderscore A jusante\textunderscore , para o lado de baixo.
Para o lado, onde vasa a maré.
\section{Jusão}
\begin{itemize}
\item {Grp. gram.:adj.}
\end{itemize}
\begin{itemize}
\item {Utilização:Ant.}
\end{itemize}
\begin{itemize}
\item {Proveniência:(De \textunderscore juso\textunderscore )}
\end{itemize}
Que está abaixo.
Dizia-se especialmente de algumas terras, divididas em duas partes que se distinguiam pela sua posição: \textunderscore Caria jusan\textunderscore  e \textunderscore Caria susan\textunderscore ; \textunderscore villa susan\textunderscore  e \textunderscore villa jusan\textunderscore .
\section{Juso}
\begin{itemize}
\item {Grp. gram.:m.}
\end{itemize}
\begin{itemize}
\item {Utilização:Ant.}
\end{itemize}
\begin{itemize}
\item {Grp. gram.:Adv.}
\end{itemize}
A parte inferior.
Abaixo, debaixo.
(B. lat. \textunderscore jusum\textunderscore )
\section{Jusquina}
\begin{itemize}
\item {Grp. gram.:adj. f.}
\end{itemize}
Dizia-se de certa música, no estilo do compositor francês Jusquin:«\textunderscore ...músicas mais jusquinas, mais suaves\textunderscore ». A. Prestes, \textunderscore Auto da Ave-Maria\textunderscore .
\section{Jussão}
\begin{itemize}
\item {Grp. gram.:adj.}
\end{itemize}
\begin{itemize}
\item {Utilização:Ant.}
\end{itemize}
O mesmo que \textunderscore jusano\textunderscore .
\section{Jussará}
\begin{itemize}
\item {Grp. gram.:f.}
\end{itemize}
Espécie de palmeira do Brasil, (\textunderscore euterpe linicaules\textunderscore ).
(Supponho que os diccionaristas trocaram \textunderscore jussara\textunderscore  por \textunderscore jussará\textunderscore )
\section{Jussieua}
\begin{itemize}
\item {Grp. gram.:f.}
\end{itemize}
\begin{itemize}
\item {Proveniência:(De \textunderscore Jussieu\textunderscore , n. p.)}
\end{itemize}
Gênero de plantas onothéreas.
\section{Justa}
\begin{itemize}
\item {Grp. gram.:f.}
\end{itemize}
\begin{itemize}
\item {Utilização:Ant.}
\end{itemize}
\begin{itemize}
\item {Utilização:ant.}
\end{itemize}
\begin{itemize}
\item {Utilização:Gír.}
\end{itemize}
\begin{itemize}
\item {Utilização:Gír. do Pôrto.}
\end{itemize}
\begin{itemize}
\item {Proveniência:(De \textunderscore justo\textunderscore )}
\end{itemize}
Vaso ou taça, em que se lançava o vinho para cada conviva.
Casaca.
Camisa.
\section{Justa}
\begin{itemize}
\item {Grp. gram.:f.}
\end{itemize}
\begin{itemize}
\item {Utilização:Ext.}
\end{itemize}
\begin{itemize}
\item {Proveniência:(Lat. \textunderscore justa\textunderscore )}
\end{itemize}
Combate entre dois homens armados de lança.
Combate entre dois homens, duello.
Luta; questão.
\section{Justador}
\begin{itemize}
\item {Grp. gram.:m.  e  adj.}
\end{itemize}
\begin{itemize}
\item {Proveniência:(De \textunderscore justar\textunderscore ^1)}
\end{itemize}
O que entra em justas; competidor.
\section{Justamente}
\begin{itemize}
\item {Grp. gram.:adv.}
\end{itemize}
De modo justo; precisamente, com exactidão.
\section{Justar}
\begin{itemize}
\item {Grp. gram.:v. i.}
\end{itemize}
\begin{itemize}
\item {Proveniência:(De \textunderscore justa\textunderscore ^2)}
\end{itemize}
Entrar em justa, lutar.
Fazer competência a alguém.
\section{Justar}
\begin{itemize}
\item {Grp. gram.:v. t.}
\end{itemize}
\begin{itemize}
\item {Utilização:Pop.}
\end{itemize}
O mesmo que \textunderscore ajustar\textunderscore .
\section{Justedade}
\begin{itemize}
\item {Grp. gram.:f.}
\end{itemize}
\begin{itemize}
\item {Utilização:Ant.}
\end{itemize}
\begin{itemize}
\item {Proveniência:(De \textunderscore justo\textunderscore )}
\end{itemize}
Integridade, rectidão. Cf. Usque, 29.
\section{Justeza}
\begin{itemize}
\item {Grp. gram.:f.}
\end{itemize}
\begin{itemize}
\item {Proveniência:(Do lat. \textunderscore justitia\textunderscore )}
\end{itemize}
Qualidade daquillo que é justo.
Exactidão.
Propriedade, conveniência.
\section{Justiça}
\begin{itemize}
\item {Grp. gram.:f.}
\end{itemize}
\begin{itemize}
\item {Proveniência:(Lat. \textunderscore justitia\textunderscore )}
\end{itemize}
Conformidade com o Direito.
Vontade permanente de dar a cada um o que é seu.
Acto de dar ou attribuir a cada qual o que por direito lhe pertence: \textunderscore fazer justiça a alguém\textunderscore .
Faculdade de premiar ou punir, segundo o Direito.
Direito escrito.
Alçada.
Magistratura; conjunto dos magistrados e pessôas, que servem junto delles: \textunderscore a respeitabilidade da justiça\textunderscore .
Innocência primitiva, antes do peccado do primeiro homem.
\section{Justiçado}
\begin{itemize}
\item {Grp. gram.:m.}
\end{itemize}
\begin{itemize}
\item {Proveniência:(De \textunderscore justiçar\textunderscore )}
\end{itemize}
Indivíduo suppliciado.
\section{Justiçadoiro}
\begin{itemize}
\item {Grp. gram.:adj.}
\end{itemize}
Que merece ser justiçado.
\section{Justiçadouro}
\begin{itemize}
\item {Grp. gram.:adj.}
\end{itemize}
Que merece ser justiçado.
\section{Justiçar}
\begin{itemize}
\item {Grp. gram.:v. t.}
\end{itemize}
\begin{itemize}
\item {Utilização:Des.}
\end{itemize}
\begin{itemize}
\item {Proveniência:(De \textunderscore justiça\textunderscore )}
\end{itemize}
Atormentar em nome da lei; punir com a morte.
Tratar com severidade; applicar justiça severa a:«\textunderscore ...foi para todos muy justiçoso e para si sobre todos justiçado\textunderscore ». Rui de Pina, \textunderscore Chrón. de D. Dinis\textunderscore .
\section{Justiceiro}
\begin{itemize}
\item {Grp. gram.:adj.}
\end{itemize}
\begin{itemize}
\item {Grp. gram.:M.}
\end{itemize}
\begin{itemize}
\item {Utilização:Prov.}
\end{itemize}
\begin{itemize}
\item {Utilização:trasm.}
\end{itemize}
\begin{itemize}
\item {Proveniência:(De \textunderscore justiça\textunderscore )}
\end{itemize}
Que executa severamente as leis.
Severo; implacável.
Litigante, demandista.
\section{Justícia}
\begin{itemize}
\item {Grp. gram.:f.}
\end{itemize}
\begin{itemize}
\item {Utilização:Ant.}
\end{itemize}
O mesmo que \textunderscore justiça\textunderscore . Cf. Usque, 42, v.^o
\section{Justiçoso}
\begin{itemize}
\item {Grp. gram.:adj.}
\end{itemize}
O mesmo que \textunderscore justiceiro\textunderscore .
\section{Justidade}
\begin{itemize}
\item {Grp. gram.:f.}
\end{itemize}
\begin{itemize}
\item {Utilização:Des.}
\end{itemize}
O mesmo que \textunderscore justedade\textunderscore . Cf. Filinto, XVIII, 118; XIX, 250.
\section{Justificação}
\begin{itemize}
\item {Grp. gram.:f.}
\end{itemize}
\begin{itemize}
\item {Proveniência:(Lat. \textunderscore justificatio\textunderscore )}
\end{itemize}
Acto ou effeito de justificar.
Aquillo que serve para justificar.
Processo para justificar.
\section{Justificadamente}
\begin{itemize}
\item {Grp. gram.:adv.}
\end{itemize}
De modo justificado; com razão.
\section{Justificador}
\begin{itemize}
\item {Grp. gram.:m.  e  adj.}
\end{itemize}
\begin{itemize}
\item {Proveniência:(Lat. \textunderscore justificator\textunderscore )}
\end{itemize}
O que justifica.
\section{Justificante}
\begin{itemize}
\item {Grp. gram.:adj.}
\end{itemize}
\begin{itemize}
\item {Grp. gram.:M.}
\end{itemize}
\begin{itemize}
\item {Proveniência:(Lat. \textunderscore justificans\textunderscore )}
\end{itemize}
Que justifica.
Aquelle que requere uma justificação em juízo.
\section{Justificar}
\begin{itemize}
\item {Grp. gram.:v. t.}
\end{itemize}
\begin{itemize}
\item {Utilização:Typ.}
\end{itemize}
\begin{itemize}
\item {Proveniência:(Lat. \textunderscore justificare\textunderscore )}
\end{itemize}
Provar a justiça ou a innocência de.
Provar em juízo.
Provar.
Desculpar.
Restituír ao estado de innocência.
Tornar tão comprida como outra (uma linha).
\section{Justificativo}
\begin{itemize}
\item {Grp. gram.:adj.}
\end{itemize}
Próprio para justificar.
\section{Justificatório}
\begin{itemize}
\item {Grp. gram.:adj.}
\end{itemize}
Que serve para justificar. Cf. Júlio Dinís, \textunderscore Serões\textunderscore , 163.
\section{Justificável}
\begin{itemize}
\item {Grp. gram.:adj.}
\end{itemize}
Que se póde justificar.
\section{Justilho}
\begin{itemize}
\item {Grp. gram.:m.}
\end{itemize}
\begin{itemize}
\item {Utilização:T. de Miranda}
\end{itemize}
Corpete; espartilho.
Collete de homem.
(Cp. cast. \textunderscore justillo\textunderscore )
\section{Justinianeu}
\begin{itemize}
\item {Grp. gram.:adj.}
\end{itemize}
Relativo ao imperador Justiniano:«\textunderscore o Direito justinianeu\textunderscore ».
\section{Justiniano}
\begin{itemize}
\item {Grp. gram.:adj.}
\end{itemize}
O mesmo que \textunderscore justinianeu\textunderscore . Cf. \textunderscore Parnaso Lusit.\textunderscore , V, 14.
\section{Justo}
\begin{itemize}
\item {Grp. gram.:adj.}
\end{itemize}
\begin{itemize}
\item {Grp. gram.:Loc. adv.}
\end{itemize}
\begin{itemize}
\item {Grp. gram.:M.}
\end{itemize}
\begin{itemize}
\item {Utilização:Gír.}
\end{itemize}
\begin{itemize}
\item {Proveniência:(Lat. \textunderscore justus\textunderscore )}
\end{itemize}
Conforme ao Direito: \textunderscore sentença justa\textunderscore .
Equitativo.
Recto; imparcial.
Razoável.
Exacto.
Fundado: \textunderscore motivos justos\textunderscore .
Legítimo.
Apertado: \textunderscore um casaco justo\textunderscore .
Ajustado: \textunderscore ficou justo o contrato\textunderscore .
\textunderscore Á justa\textunderscore , exactamente.
Homem justo, virtuoso.
Moéda de oiro, correspondente a 800 reis, do tempo de D. João II.
Collete, justilho.
\section{Justura}
\begin{itemize}
\item {Grp. gram.:f.}
\end{itemize}
\begin{itemize}
\item {Proveniência:(De \textunderscore justo\textunderscore )}
\end{itemize}
Acto de justar^2 ou ajustar.
Fórma, que o ferrador dá á ferradura, dobrando-a na parte anterior, de baixo para cima.
\section{Juta}
\begin{itemize}
\item {Grp. gram.:f.}
\end{itemize}
Planta liliácea, de fibras têxteis, (\textunderscore corchoris capsularis\textunderscore ).
\section{Jutaí}
\begin{itemize}
\item {Grp. gram.:m.}
\end{itemize}
O mesmo que \textunderscore tamarinheiro\textunderscore .
\section{Jutaí-cica}
\begin{itemize}
\item {Grp. gram.:m.}
\end{itemize}
\begin{itemize}
\item {Utilização:Bras}
\end{itemize}
Resina do jutaí, com que se dá lustro á loiça de barro.
\section{Jutaúba}
\begin{itemize}
\item {Grp. gram.:f.}
\end{itemize}
\begin{itemize}
\item {Utilização:Bras}
\end{itemize}
Árvore, de bôa madeira para construcções.
\section{Juticupiúba}
\begin{itemize}
\item {Grp. gram.:f.}
\end{itemize}
\begin{itemize}
\item {Utilização:Bras}
\end{itemize}
Árvore silvestre.
\section{Jutuarana}
\begin{itemize}
\item {Grp. gram.:f.}
\end{itemize}
\begin{itemize}
\item {Utilização:Bras}
\end{itemize}
Peixe do Amazonas.
\section{Jutua-uba}
\begin{itemize}
\item {Grp. gram.:f.}
\end{itemize}
Árvore meliácea, (\textunderscore guarea pendula\textunderscore ).
\section{Juvassivo}
\begin{itemize}
\item {Grp. gram.:adj.}
\end{itemize}
\begin{itemize}
\item {Utilização:Ant.}
\end{itemize}
\begin{itemize}
\item {Proveniência:(Do lat. \textunderscore juvare\textunderscore )}
\end{itemize}
Dizia-se das armas, próprias para defesa.
O mesmo que \textunderscore defensivo\textunderscore .
\section{Juvenaes}
\begin{itemize}
\item {Grp. gram.:m. pl.}
\end{itemize}
\begin{itemize}
\item {Proveniência:(Lat. \textunderscore juvenales\textunderscore )}
\end{itemize}
Jogos romanos, instituídos por Nero.
\section{Juvenais}
\begin{itemize}
\item {Grp. gram.:m. pl.}
\end{itemize}
\begin{itemize}
\item {Proveniência:(Lat. \textunderscore juvenales\textunderscore )}
\end{itemize}
Jogos romanos, instituídos por Nero.
\section{Juvenal}
\begin{itemize}
\item {Grp. gram.:adj.}
\end{itemize}
\begin{itemize}
\item {Utilização:Des.}
\end{itemize}
\begin{itemize}
\item {Proveniência:(Lat. \textunderscore juvenalis\textunderscore )}
\end{itemize}
O mesmo que \textunderscore juvenil\textunderscore .
\section{Juvenalesco}
\begin{itemize}
\item {fónica:lês}
\end{itemize}
\begin{itemize}
\item {Grp. gram.:adj.}
\end{itemize}
Relativo a Juvenal, ou ao seu estílo.
\section{Juvenca}
\begin{itemize}
\item {Grp. gram.:f.}
\end{itemize}
Novilha, bezerra.--T. vulgar em Melgaço.
(Cp. \textunderscore juvenco\textunderscore )
\section{Juvenco}
\begin{itemize}
\item {Grp. gram.:m.}
\end{itemize}
\begin{itemize}
\item {Proveniência:(Lat. \textunderscore juvencus\textunderscore )}
\end{itemize}
O mesmo que \textunderscore novilho\textunderscore .
\section{Juvenil}
\begin{itemize}
\item {Grp. gram.:adj.}
\end{itemize}
\begin{itemize}
\item {Proveniência:(Lat. \textunderscore juvenilis\textunderscore )}
\end{itemize}
Relativo á juventude: \textunderscore distracções juvenis\textunderscore .
Moço.
\section{Juvenilidade}
\begin{itemize}
\item {Grp. gram.:f.}
\end{itemize}
\begin{itemize}
\item {Proveniência:(Lat. \textunderscore juvenilitas\textunderscore )}
\end{itemize}
Qualidade de juvenil; idade juvenil.
\section{Juvenilmente}
\begin{itemize}
\item {Grp. gram.:adv.}
\end{itemize}
De modo juvenil.
\section{Juveníssimo}
\begin{itemize}
\item {Grp. gram.:adj.}
\end{itemize}
\begin{itemize}
\item {Proveniência:(Do lat. \textunderscore juvenis\textunderscore )}
\end{itemize}
Muito jóvem. Cf. Castilho, \textunderscore Escav. Poét.\textunderscore , 16.
\section{Juventa}
\begin{itemize}
\item {Grp. gram.:f.}
\end{itemize}
\begin{itemize}
\item {Proveniência:(Lat. \textunderscore juventa\textunderscore )}
\end{itemize}
O mesmo que \textunderscore juventude\textunderscore . Cf. Filinto, IV, 109.
\section{Juventude}
\begin{itemize}
\item {Grp. gram.:f.}
\end{itemize}
\begin{itemize}
\item {Proveniência:(Lat. \textunderscore juventus\textunderscore )}
\end{itemize}
Mocidade; adolescência.
Gente moça: \textunderscore a juventude folga e espera\textunderscore .
\section{Júvia}
\begin{itemize}
\item {Grp. gram.:f.}
\end{itemize}
Árvore myrtácea do Brasil.
\section{Juxtafluvial}
\begin{itemize}
\item {Grp. gram.:adj.}
\end{itemize}
\begin{itemize}
\item {Proveniência:(Do lat. \textunderscore juxta\textunderscore  + \textunderscore fluvialis\textunderscore )}
\end{itemize}
Que está nas margens de um rio; marginal.
\section{Juxtalinear}
\begin{itemize}
\item {Grp. gram.:adj.}
\end{itemize}
\begin{itemize}
\item {Proveniência:(Do lat. \textunderscore juxta\textunderscore  + \textunderscore linea\textunderscore )}
\end{itemize}
Traduzido linha a linha.
\section{Juxtapor}
\begin{itemize}
\item {Grp. gram.:v. t.}
\end{itemize}
\begin{itemize}
\item {Proveniência:(Do lat. \textunderscore juxta\textunderscore  + \textunderscore ponere\textunderscore )}
\end{itemize}
Pôr junto.
\section{Juxtaposição}
\begin{itemize}
\item {Grp. gram.:f.}
\end{itemize}
\end{document}