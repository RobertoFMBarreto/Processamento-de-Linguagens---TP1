\documentclass{article}
\usepackage[portuguese]{babel}
\title{H}
\begin{document}
O mesmo quo \textunderscore hierosolimitano\textunderscore . Cf. Herculano, \textunderscore Hist. de Port.\textunderscore , II, 14.
\section{Harpagão}
\begin{itemize}
\item {Grp. gram.:m.}
\end{itemize}
\begin{itemize}
\item {Utilização:Ext.}
\end{itemize}
\begin{itemize}
\item {Proveniência:(De \textunderscore Harpagão\textunderscore , n. p. de uma personagem de Molière)}
\end{itemize}
Indivíduo, muito avarento.
\section{Hebefrenia}
\begin{itemize}
\item {Grp. gram.:f.}
\end{itemize}
\begin{itemize}
\item {Utilização:Med.}
\end{itemize}
\begin{itemize}
\item {Proveniência:(Do gr. \textunderscore hebe\textunderscore  + \textunderscore phren\textunderscore )}
\end{itemize}
Conjunto de perturbações intellectuaes, que apparecem ás vezes na puberdade.
\section{Hebephrenia}
\begin{itemize}
\item {Grp. gram.:f.}
\end{itemize}
\begin{itemize}
\item {Utilização:Med.}
\end{itemize}
\begin{itemize}
\item {Proveniência:(Do gr. \textunderscore hebe\textunderscore  + \textunderscore phren\textunderscore )}
\end{itemize}
Conjunto de perturbações intellectuaes, que apparecem ás vezes na puberdade.
\section{Hecticidade}
\begin{itemize}
\item {Grp. gram.:f.}
\end{itemize}
\begin{itemize}
\item {Utilização:Med.}
\end{itemize}
Estado de héctico; magreza e fraqueza, causadas pela febre da tísica.
\section{Heliofobia}
\begin{itemize}
\item {Grp. gram.:f.}
\end{itemize}
\begin{itemize}
\item {Utilização:Med.}
\end{itemize}
\begin{itemize}
\item {Proveniência:(Do gr. \textunderscore helios\textunderscore  + \textunderscore phobos\textunderscore )}
\end{itemize}
Mêdo mórbido da luz.
\section{Heliophobia}
\begin{itemize}
\item {Grp. gram.:f.}
\end{itemize}
\begin{itemize}
\item {Utilização:Med.}
\end{itemize}
\begin{itemize}
\item {Proveniência:(Do gr. \textunderscore helios\textunderscore  + \textunderscore phobos\textunderscore )}
\end{itemize}
Mêdo mórbido da luz.
\section{Helófito}
\begin{itemize}
\item {Grp. gram.:adj.}
\end{itemize}
\begin{itemize}
\item {Utilização:Bot.}
\end{itemize}
\begin{itemize}
\item {Proveniência:(Do gr. \textunderscore helos\textunderscore  + \textunderscore phuton\textunderscore )}
\end{itemize}
Diz-se da planta, que habita terrenos encharcados ou inundados.
\section{Helóphyto}
\begin{itemize}
\item {Grp. gram.:adj.}
\end{itemize}
\begin{itemize}
\item {Utilização:Bot.}
\end{itemize}
\begin{itemize}
\item {Proveniência:(Do gr. \textunderscore helos\textunderscore  + \textunderscore phuton\textunderscore )}
\end{itemize}
Diz-se da planta, que habita terrenos encharcados ou inundados.
\section{Hematímetro}
\begin{itemize}
\item {Grp. gram.:m.}
\end{itemize}
\begin{itemize}
\item {Utilização:Med.}
\end{itemize}
\begin{itemize}
\item {Proveniência:(Do gr. \textunderscore haima\textunderscore  + \textunderscore metron\textunderscore )}
\end{itemize}
Apparelho, para contar os glóbulos do sangue.
\section{Hemiataxia}
\begin{itemize}
\item {fónica:csi}
\end{itemize}
\begin{itemize}
\item {Grp. gram.:f.}
\end{itemize}
\begin{itemize}
\item {Utilização:Med.}
\end{itemize}
\begin{itemize}
\item {Proveniência:(De \textunderscore hemi...\textunderscore  + \textunderscore ataxia\textunderscore )}
\end{itemize}
Ataxia em metade do corpo.
\section{Hemiatrofia}
\begin{itemize}
\item {Grp. gram.:f.}
\end{itemize}
\begin{itemize}
\item {Utilização:Med.}
\end{itemize}
\begin{itemize}
\item {Proveniência:(De \textunderscore hemi...\textunderscore  + \textunderscore atrophia\textunderscore )}
\end{itemize}
Atrofia unilateral.
\section{Hemiatrophia}
\begin{itemize}
\item {Grp. gram.:f.}
\end{itemize}
\begin{itemize}
\item {Utilização:Med.}
\end{itemize}
\begin{itemize}
\item {Proveniência:(De \textunderscore hemi...\textunderscore  + \textunderscore atrophia\textunderscore )}
\end{itemize}
Atrophia unilateral.
\section{Hemodiagnóstico}
\begin{itemize}
\item {Grp. gram.:m.}
\end{itemize}
\begin{itemize}
\item {Utilização:Med.}
\end{itemize}
\begin{itemize}
\item {Proveniência:(De \textunderscore hemo...\textunderscore  + \textunderscore diagnóstico\textunderscore )}
\end{itemize}
Diagnóstico, baseado no exame do sangue.
\section{Heterosexual}
\begin{itemize}
\item {fónica:se,csu}
\end{itemize}
\begin{itemize}
\item {Grp. gram.:adj.}
\end{itemize}
\begin{itemize}
\item {Proveniência:(De \textunderscore hetero...\textunderscore  + \textunderscore assexual\textunderscore )}
\end{itemize}
Relativo aos dois differentes sexos: \textunderscore relações heterosexuaes\textunderscore . Cf. J. Saavedra &amp; A. Barradas, \textunderscore Diccion.\textunderscore , vb. \textunderscore assexualidade\textunderscore .
\section{Heterossexual}
\begin{itemize}
\item {fónica:csu}
\end{itemize}
\begin{itemize}
\item {Grp. gram.:adj.}
\end{itemize}
\begin{itemize}
\item {Proveniência:(De \textunderscore hetero...\textunderscore  + \textunderscore assexual\textunderscore )}
\end{itemize}
Relativo aos dois differentes sexos: \textunderscore relações heterossexuaes\textunderscore . Cf. J. Saavedra &amp; A. Barradas, \textunderscore Diccion.\textunderscore , vb. \textunderscore assexualidade\textunderscore .
\section{Homenageado}
\begin{itemize}
\item {Grp. gram.:m.  e  adj.}
\end{itemize}
\begin{itemize}
\item {Utilização:Neol.}
\end{itemize}
O que recebe ou recebeu homenagens.
\section{Hidroaeroplano}
\begin{itemize}
\item {fónica:a-e}
\end{itemize}
\begin{itemize}
\item {Grp. gram.:m.}
\end{itemize}
\begin{itemize}
\item {Proveniência:(De \textunderscore hidro...\textunderscore  + \textunderscore aeroplano\textunderscore )}
\end{itemize}
Aeroplano, que póde fluctuar sôbre a água.
\section{Hienóide}
\begin{itemize}
\item {Grp. gram.:m.}
\end{itemize}
\begin{itemize}
\item {Proveniência:(De \textunderscore hiena\textunderscore )}
\end{itemize}
Variedade de cão, também chamado \textunderscore cão hiena\textunderscore .
\section{Hiperazoturia}
\begin{itemize}
\item {Grp. gram.:f.}
\end{itemize}
\begin{itemize}
\item {Utilização:Med.}
\end{itemize}
\begin{itemize}
\item {Proveniência:(De \textunderscore hiper...\textunderscore  + \textunderscore azoturia\textunderscore )}
\end{itemize}
Aumento da ureia, eliminada pela urina.
\section{Hipercolia}
\begin{itemize}
\item {Grp. gram.:f.}
\end{itemize}
\begin{itemize}
\item {Utilização:Med.}
\end{itemize}
\begin{itemize}
\item {Proveniência:(Do gr. \textunderscore huper\textunderscore  + \textunderscore khole\textunderscore )}
\end{itemize}
Excessiva secreção biliar.
\section{Hipersplênia}
\begin{itemize}
\item {Grp. gram.:f.}
\end{itemize}
\begin{itemize}
\item {Utilização:Med.}
\end{itemize}
\begin{itemize}
\item {Proveniência:(Do gr. \textunderscore huper\textunderscore  + \textunderscore splen\textunderscore )}
\end{itemize}
Aumento do volume do baço.
\section{Hipocolia}
\begin{itemize}
\item {Grp. gram.:f.}
\end{itemize}
\begin{itemize}
\item {Utilização:Med.}
\end{itemize}
\begin{itemize}
\item {Proveniência:(Do gr. \textunderscore hupo\textunderscore  + \textunderscore khole\textunderscore )}
\end{itemize}
Deminuição da secreção biliar.
\section{Hipoosmia}
\begin{itemize}
\item {Grp. gram.:f.}
\end{itemize}
\begin{itemize}
\item {Utilização:Med.}
\end{itemize}
\begin{itemize}
\item {Proveniência:(Do gr. \textunderscore hupo\textunderscore  + \textunderscore osme\textunderscore )}
\end{itemize}
Deminuição do olfato.
\section{Hipotermia}
\begin{itemize}
\item {Grp. gram.:f.}
\end{itemize}
\begin{itemize}
\item {Utilização:Med.}
\end{itemize}
\begin{itemize}
\item {Proveniência:(Do gr. \textunderscore hupo\textunderscore  + \textunderscore therme\textunderscore )}
\end{itemize}
Abaixamento anormal da temperatura do corpo ou de uma parte do corpo.
\section{Histeróforo}
\begin{itemize}
\item {Grp. gram.:m.}
\end{itemize}
\begin{itemize}
\item {Utilização:Med.}
\end{itemize}
\begin{itemize}
\item {Proveniência:(Do gr. \textunderscore hustera\textunderscore  + \textunderscore phoros\textunderscore )}
\end{itemize}
Pessário, cuja haste é segura por uma cinta.
\section{Histeropexia}
\begin{itemize}
\item {fónica:csi}
\end{itemize}
\begin{itemize}
\item {Grp. gram.:f.}
\end{itemize}
\begin{itemize}
\item {Utilização:Med.}
\end{itemize}
\begin{itemize}
\item {Proveniência:(Do gr. \textunderscore hustera\textunderscore  + \textunderscore pexis\textunderscore )}
\end{itemize}
Operação, que consiste em fixar o útero á parede abdominal anterior.
\section{Hydroaeroplano}
\begin{itemize}
\item {fónica:a-e}
\end{itemize}
\begin{itemize}
\item {Grp. gram.:m.}
\end{itemize}
\begin{itemize}
\item {Proveniência:(De \textunderscore hydro...\textunderscore  + \textunderscore aeroplano\textunderscore )}
\end{itemize}
Aeroplano, que póde fluctuar sôbre a água.
\section{Hyenóide}
\begin{itemize}
\item {Grp. gram.:m.}
\end{itemize}
\begin{itemize}
\item {Proveniência:(De \textunderscore hyena\textunderscore )}
\end{itemize}
Variedade de cão, também chamado \textunderscore cão hyena\textunderscore .
\section{Hyperazoturia}
\begin{itemize}
\item {Grp. gram.:f.}
\end{itemize}
\begin{itemize}
\item {Utilização:Med.}
\end{itemize}
\begin{itemize}
\item {Proveniência:(De \textunderscore hyper...\textunderscore  + \textunderscore azoturia\textunderscore )}
\end{itemize}
Aumento da ureia, eliminada pela urina.
\section{Hypercholia}
\begin{itemize}
\item {fónica:co}
\end{itemize}
\begin{itemize}
\item {Grp. gram.:f.}
\end{itemize}
\begin{itemize}
\item {Utilização:Med.}
\end{itemize}
\begin{itemize}
\item {Proveniência:(Do gr. \textunderscore huper\textunderscore  + \textunderscore khole\textunderscore )}
\end{itemize}
Excessiva secreção biliar.
\section{Hypersplenia}
\begin{itemize}
\item {Grp. gram.:f.}
\end{itemize}
\begin{itemize}
\item {Utilização:Med.}
\end{itemize}
\begin{itemize}
\item {Proveniência:(Do gr. \textunderscore huper\textunderscore  + \textunderscore splen\textunderscore )}
\end{itemize}
Aumento do volume do baço.
\section{Hypocholia}
\begin{itemize}
\item {fónica:co}
\end{itemize}
\begin{itemize}
\item {Grp. gram.:f.}
\end{itemize}
\begin{itemize}
\item {Utilização:Med.}
\end{itemize}
\begin{itemize}
\item {Proveniência:(Do gr. \textunderscore hupo\textunderscore  + \textunderscore khole\textunderscore )}
\end{itemize}
Deminuição da secreção biliar.
\section{Hypoosmia}
\begin{itemize}
\item {Grp. gram.:f.}
\end{itemize}
\begin{itemize}
\item {Utilização:Med.}
\end{itemize}
\begin{itemize}
\item {Proveniência:(Do gr. \textunderscore hupo\textunderscore  + \textunderscore osme\textunderscore )}
\end{itemize}
Deminuição do olfato.
\section{Hypothermia}
\begin{itemize}
\item {Grp. gram.:f.}
\end{itemize}
\begin{itemize}
\item {Utilização:Med.}
\end{itemize}
\begin{itemize}
\item {Proveniência:(Do gr. \textunderscore hupo\textunderscore  + \textunderscore therme\textunderscore )}
\end{itemize}
Abaixamento anormal da temperatura do corpo ou de uma parte do corpo.
\section{Hysteropexia}
\begin{itemize}
\item {fónica:csi}
\end{itemize}
\begin{itemize}
\item {Grp. gram.:f.}
\end{itemize}
\begin{itemize}
\item {Utilização:Med.}
\end{itemize}
\begin{itemize}
\item {Proveniência:(Do gr. \textunderscore hustera\textunderscore  + \textunderscore pexis\textunderscore )}
\end{itemize}
Operação, que consiste em fixar o útero á parede abdominal anterior.
\section{Hysteróphoro}
\begin{itemize}
\item {Grp. gram.:m.}
\end{itemize}
\begin{itemize}
\item {Utilização:Med.}
\end{itemize}
\begin{itemize}
\item {Proveniência:(Do gr. \textunderscore hustera\textunderscore  + \textunderscore phoros\textunderscore )}
\end{itemize}
Pessário, cuja haste é segura por uma cinta.
\section{H}
\begin{itemize}
\item {fónica:agá}
\end{itemize}
\begin{itemize}
\item {Grp. gram.:m.}
\end{itemize}
\begin{itemize}
\item {Utilização:Chím.}
\end{itemize}
\begin{itemize}
\item {Grp. gram.:Adj.}
\end{itemize}
Oitava letra do alphabeto português.
Abrev. de \textunderscore hydrogênio\textunderscore .
Que numa série occupa o oitavo lugar.
\section{Habena}
\begin{itemize}
\item {Grp. gram.:f.}
\end{itemize}
\begin{itemize}
\item {Utilização:Poét.}
\end{itemize}
\begin{itemize}
\item {Proveniência:(Lat. \textunderscore habena\textunderscore )}
\end{itemize}
Chicote; rédea.
\section{Habés}
\begin{itemize}
\item {Grp. gram.:m. pl.}
\end{itemize}
Tríbo guerreira das vizinhanças do Níger, na região do Mossi.
\section{Hábil}
\begin{itemize}
\item {Grp. gram.:adj.}
\end{itemize}
\begin{itemize}
\item {Proveniência:(Lat. \textunderscore habilis\textunderscore )}
\end{itemize}
Que tem aptidão para alguma coisa.
Que executa ou póde executar uma coisa com perfeição.
Que tem capacidade legal para certos hábitos.
Intelligente.
Que revela engenho ou destreza.
\section{Hábile}
\begin{itemize}
\item {Grp. gram.:adj.}
\end{itemize}
O mesmo que \textunderscore hábil\textunderscore . Cf. \textunderscore Eufrosina\textunderscore , 336.
\section{Habilhamento}
\begin{itemize}
\item {Grp. gram.:m.}
\end{itemize}
\begin{itemize}
\item {Utilização:Gal}
\end{itemize}
\begin{itemize}
\item {Utilização:ant.}
\end{itemize}
\begin{itemize}
\item {Proveniência:(Do fr. \textunderscore habillement\textunderscore )}
\end{itemize}
Acto de se enfeitar alguém.
Adôrno, atavio.
\section{Habilidade}
\begin{itemize}
\item {Grp. gram.:f.}
\end{itemize}
\begin{itemize}
\item {Grp. gram.:Pl.}
\end{itemize}
\begin{itemize}
\item {Proveniência:(Lat. \textunderscore habilitas\textunderscore )}
\end{itemize}
Qualidade de quem é hábil.
Exercícios gymnásticos.
Sortes de prestidigitação.
Entretenimentos, que produzem admiração ou surpresa em quem os presenceia.
\section{Habilidosamente}
\begin{itemize}
\item {Grp. gram.:adv.}
\end{itemize}
De modo habilidoso.
Com habilidade.
Habilmente.
\section{Habilidoso}
\begin{itemize}
\item {Grp. gram.:adj.}
\end{itemize}
\begin{itemize}
\item {Proveniência:(De \textunderscore habilidade\textunderscore )}
\end{itemize}
Que tem ou revela habilidade.
Que faz habilidades.
Destro; hábil.
\section{Habilitação}
\begin{itemize}
\item {Grp. gram.:f.}
\end{itemize}
\begin{itemize}
\item {Grp. gram.:Pl.}
\end{itemize}
Acto ou effeito de habilitar.
Aptidão.
Conjunto de conhecimentos.
Acção judicial ou formalidades jurídicas, para adquirir um direito ou para demonstrar certa capacidade legal.
Conjunto de provas ou documentos, para requerer ou provar alguma coisa.
\section{Habilitador}
\begin{itemize}
\item {Grp. gram.:m.  e  adj.}
\end{itemize}
O que habilita.
\section{Habilitanço}
\begin{itemize}
\item {Grp. gram.:m.}
\end{itemize}
\begin{itemize}
\item {Proveniência:(De \textunderscore habilitar\textunderscore )}
\end{itemize}
Termo de jogadores, para designar a quantia que um parceiro empresta a outro, no jôgo de asar.
\section{Habilitando}
\begin{itemize}
\item {Grp. gram.:m.  e  adj.}
\end{itemize}
\begin{itemize}
\item {Proveniência:(Lat. \textunderscore habilitandus\textunderscore )}
\end{itemize}
O que trata de habilitar-se.
\section{Habilitante}
\begin{itemize}
\item {Grp. gram.:adj.}
\end{itemize}
\begin{itemize}
\item {Proveniência:(Lat. \textunderscore habilitans\textunderscore )}
\end{itemize}
Aquelle que requere habilitação judicial.
\section{Habilitar}
\begin{itemize}
\item {Grp. gram.:v. t.}
\end{itemize}
\begin{itemize}
\item {Grp. gram.:V. p.}
\end{itemize}
\begin{itemize}
\item {Utilização:Fam.}
\end{itemize}
\begin{itemize}
\item {Proveniência:(Lat. \textunderscore habilitare\textunderscore )}
\end{itemize}
Tornar hábil.
Tornar apto.
Preparar para alguma coisa.
Requerer habilitação judicial.
Justificar judicialmente uma habilitação requerida.
Tornar-se apto ou capaz para certo fim: \textunderscore habilitar-se para professor\textunderscore .
Jogar (na lotaria).
\section{Habilítzia}
\begin{itemize}
\item {Grp. gram.:f.}
\end{itemize}
\begin{itemize}
\item {Proveniência:(De \textunderscore Habilitz\textunderscore , n. p.)}
\end{itemize}
Trepadeira vivaz.
\section{Habilmente}
\begin{itemize}
\item {Grp. gram.:adv.}
\end{itemize}
\begin{itemize}
\item {Proveniência:(De \textunderscore hábil\textunderscore )}
\end{itemize}
Com habilidade.
\section{Habitabilidade}
\begin{itemize}
\item {Grp. gram.:f.}
\end{itemize}
Qualidade de habitável.
\section{Habitação}
\begin{itemize}
\item {Grp. gram.:f.}
\end{itemize}
\begin{itemize}
\item {Proveniência:(Lat. \textunderscore habitatio\textunderscore )}
\end{itemize}
Lugar, em que se habita.
Morada.
Residência.
\section{Habitacional}
\begin{itemize}
\item {Grp. gram.:adj.}
\end{itemize}
Relativo a habitação:«\textunderscore ...hygiene habitacional.\textunderscore »\textunderscore Diár. de Not.\textunderscore , de 28-VI-910.
\section{Habitáculo}
\begin{itemize}
\item {Grp. gram.:m.}
\end{itemize}
\begin{itemize}
\item {Proveniência:(Lat. \textunderscore habitaculum\textunderscore )}
\end{itemize}
Pequena habitação.
\section{Habitador}
\begin{itemize}
\item {Grp. gram.:m.  e  adj.}
\end{itemize}
\begin{itemize}
\item {Proveniência:(Lat. \textunderscore habitator\textunderscore )}
\end{itemize}
O mesmo que \textunderscore habitante\textunderscore .
\section{Habitante}
\begin{itemize}
\item {Grp. gram.:m. ,  f.  e  adj.}
\end{itemize}
\begin{itemize}
\item {Proveniência:(Lat. \textunderscore habitans\textunderscore )}
\end{itemize}
Pessôa, que habita.
Residente.
Morador; povoador.
\section{Habitar}
\begin{itemize}
\item {Grp. gram.:v. t.}
\end{itemize}
\begin{itemize}
\item {Utilização:Des.}
\end{itemize}
\begin{itemize}
\item {Grp. gram.:V. i.}
\end{itemize}
\begin{itemize}
\item {Proveniência:(Lat. \textunderscore habitare\textunderscore )}
\end{itemize}
Residir ou viver em.
Povoar; occupar como morada.
Frequentar.
Residir; viver.
\section{Habitável}
\begin{itemize}
\item {Grp. gram.:adj.}
\end{itemize}
\begin{itemize}
\item {Proveniência:(Lat. \textunderscore habitabilis\textunderscore )}
\end{itemize}
Próprio para se habitar.
Que póde servir para habitação.
\section{Hábito}
\begin{itemize}
\item {Grp. gram.:m.}
\end{itemize}
\begin{itemize}
\item {Proveniência:(Lat. \textunderscore habitus\textunderscore )}
\end{itemize}
Costume.
Uso.
Vestuário.
Roupagem de frade ou freira.
Aspecto, apparência.
Insígnia de Ordem militar ou religiosa.
\section{Habitual}
\begin{itemize}
\item {Grp. gram.:adj.}
\end{itemize}
\begin{itemize}
\item {Proveniência:(Lat. \textunderscore habitualis\textunderscore )}
\end{itemize}
Que succede ou se faz por hábito.
Usual; vulgar; frequente.
\section{Habitualismo}
\begin{itemize}
\item {Grp. gram.:m.}
\end{itemize}
Qualidade de habitual. Cf. \textunderscore Diccion. Contemp.\textunderscore , p. XI.
\section{Habitualmente}
\begin{itemize}
\item {Grp. gram.:adv.}
\end{itemize}
De modo habitual.
\section{Habituar}
\begin{itemize}
\item {Grp. gram.:v. t.}
\end{itemize}
\begin{itemize}
\item {Proveniência:(Lat. \textunderscore habituare\textunderscore )}
\end{itemize}
Fazer tomar costume a.
Acostumar.
Exercitar.
Avezar.
\section{Habitude}
\begin{itemize}
\item {Grp. gram.:f.}
\end{itemize}
\begin{itemize}
\item {Utilização:Ant.}
\end{itemize}
\begin{itemize}
\item {Proveniência:(Lat. \textunderscore habitudo\textunderscore )}
\end{itemize}
O mesmo que \textunderscore hábito\textunderscore , costume.
\section{Haca}
\begin{itemize}
\item {Grp. gram.:f.}
\end{itemize}
Planta espinhosa de Angola.
\section{Haca}
\begin{itemize}
\item {Grp. gram.:f.}
\end{itemize}
Antigo tributo das communidades indianas.
\section{Hacanéa}
\begin{itemize}
\item {Grp. gram.:f.}
\end{itemize}
\begin{itemize}
\item {Proveniência:(Do holl. ant. \textunderscore hakkenei\textunderscore )}
\end{itemize}
Cavalgadura bem proporcionada, mansa, e de mediana grandeza.
\section{Hacaneia}
\begin{itemize}
\item {Grp. gram.:f.}
\end{itemize}
\begin{itemize}
\item {Proveniência:(Do holl. ant. \textunderscore hakkenei\textunderscore )}
\end{itemize}
Cavalgadura bem proporcionada, mansa, e de mediana grandeza.
\section{Hacer}
\begin{itemize}
\item {Grp. gram.:m.}
\end{itemize}
Oração, que os Moiros fazem a Deus, antes do nascer do sol. Cf. Barros, \textunderscore Dec.\textunderscore  II, l. X, c. 6.
\section{Háckia}
\begin{itemize}
\item {Grp. gram.:f.}
\end{itemize}
\begin{itemize}
\item {Proveniência:(De \textunderscore Hack\textunderscore , n. p.)}
\end{itemize}
Formosa árvore da Guiana inglesa, de bellas flôres amarelas, (\textunderscore siderodendron triflorum\textunderscore ).
\section{Hacos}
\begin{itemize}
\item {Grp. gram.:m. pl.}
\end{itemize}
Tríbo independente, entre o Cuanza e o Angango.
\section{Hacpólique}
\begin{itemize}
\item {Grp. gram.:m.}
\end{itemize}
Nome que, em Timor, se dá á tanga usada pelos indígenas.
\section{Hacube}
\begin{itemize}
\item {Grp. gram.:m.}
\end{itemize}
Alcachofra da Índia.
\section{Ha-de-haver}
\begin{itemize}
\item {Grp. gram.:m.}
\end{itemize}
\begin{itemize}
\item {Proveniência:(De \textunderscore haver\textunderscore )}
\end{itemize}
Crédito ou receita de um estabelecimento commercial, indicado no livro chamado \textunderscore Razão\textunderscore .
\section{Haglura}
\begin{itemize}
\item {Grp. gram.:f.}
\end{itemize}
\begin{itemize}
\item {Proveniência:(Fr. \textunderscore haglure\textunderscore )}
\end{itemize}
Mancha nas pennas das aves.
Mancha na parte superior das pernas das aves.
\section{Hahnemanniano}
\begin{itemize}
\item {Grp. gram.:adj.}
\end{itemize}
\begin{itemize}
\item {Grp. gram.:M.}
\end{itemize}
Relativo a Hahnemann, fundador da Medicina homeopáthica.
Homeopatha, que segue estrictamente o systema de Hahnemann.
\section{Hahnemânnico}
\begin{itemize}
\item {Grp. gram.:adj.}
\end{itemize}
O mesmo que \textunderscore hahnemanniano\textunderscore . Cf. Camillo, \textunderscore Cav. em Ruínas\textunderscore , 249.
\section{Hahnemannismo}
\begin{itemize}
\item {Grp. gram.:m.}
\end{itemize}
\begin{itemize}
\item {Proveniência:(De \textunderscore Hahnemann\textunderscore , n. p.)}
\end{itemize}
Systema médico de Hahnemann; homeopathia.
\section{Hai-cá}
\begin{itemize}
\item {Grp. gram.:m.}
\end{itemize}
Árvore de Timor.
\section{Haimoré}
\begin{itemize}
\item {Grp. gram.:m.}
\end{itemize}
Peixe da Guiana inglesa, (\textunderscore crythimus\textunderscore ).
\section{Haissuaque}
\begin{itemize}
\item {Grp. gram.:m.}
\end{itemize}
Instrumento aguçado, de madeira, com que os timores, em vez de enxada e arado, revolvem e amanham a terra.
\section{Haitiano}
\begin{itemize}
\item {Grp. gram.:adj.}
\end{itemize}
\begin{itemize}
\item {Grp. gram.:M.}
\end{itemize}
Relativo ao Haiti.
Aquelle que é natural do Haiti.
\section{Hajib}
\begin{itemize}
\item {Grp. gram.:m.}
\end{itemize}
Designação do primeiro ministro, nas côrtes dos califas da Espanha. Cf. Herculano, \textunderscore Hist. de Port.\textunderscore , I, 100, 3.^a ed.
\section{Hajul}
\begin{itemize}
\item {Grp. gram.:m.}
\end{itemize}
Peixe da China.
\section{Halalávia}
\begin{itemize}
\item {Grp. gram.:f.}
\end{itemize}
Espécie de periquito de Madagáscar.
\section{Halênia}
\begin{itemize}
\item {Grp. gram.:f.}
\end{itemize}
Planta gencianácea da Sibéria.
\section{Halésia}
\begin{itemize}
\item {Grp. gram.:f.}
\end{itemize}
Arvoreta medicinal da Asia e da América.
\section{Halial}
\begin{itemize}
\item {Grp. gram.:adj.}
\end{itemize}
\begin{itemize}
\item {Proveniência:(Do lat. \textunderscore hallus\textunderscore )}
\end{itemize}
Relativo ao dedo polegar.
\section{Halichelidónios}
\begin{itemize}
\item {fónica:que}
\end{itemize}
\begin{itemize}
\item {Grp. gram.:m. pl.}
\end{itemize}
\begin{itemize}
\item {Proveniência:(Do gr. \textunderscore hals\textunderscore , \textunderscore halos\textunderscore , mar, e \textunderscore khelidon\textunderscore , andorinha)}
\end{itemize}
Família de aves, que comprehende a andorinha do mar.
\section{Halicolimbos}
\begin{itemize}
\item {Grp. gram.:m. pl.}
\end{itemize}
\begin{itemize}
\item {Proveniência:(Do gr. \textunderscore hals\textunderscore , \textunderscore halos\textunderscore  + \textunderscore columban\textunderscore )}
\end{itemize}
Família de aves mergulhadoras.
\section{Halicolymbos}
\begin{itemize}
\item {Grp. gram.:m. pl.}
\end{itemize}
\begin{itemize}
\item {Proveniência:(Do gr. \textunderscore hals\textunderscore , \textunderscore halos\textunderscore  + \textunderscore columban\textunderscore )}
\end{itemize}
Família de aves mergulhadoras.
\section{Halicoráceos}
\begin{itemize}
\item {Grp. gram.:m. pl.}
\end{itemize}
\begin{itemize}
\item {Proveniência:(Do gr. \textunderscore hals\textunderscore , \textunderscore halos\textunderscore , mar, e \textunderscore korax\textunderscore , corvo)}
\end{itemize}
Aves marítimas, a que pertence o corvo marinho.
\section{Halieto}
\begin{itemize}
\item {Grp. gram.:m.}
\end{itemize}
\begin{itemize}
\item {Proveniência:(Do gr. \textunderscore hals\textunderscore , \textunderscore halos\textunderscore , mar, e \textunderscore aietos\textunderscore , águia)}
\end{itemize}
Gênero de aves de rapina.
\section{Haliêutica}
\begin{itemize}
\item {Grp. gram.:f.}
\end{itemize}
\begin{itemize}
\item {Proveniência:(De \textunderscore haliêutico\textunderscore )}
\end{itemize}
Arte da pesca.
\section{Haliêutico}
\begin{itemize}
\item {Grp. gram.:adj.}
\end{itemize}
\begin{itemize}
\item {Proveniência:(Gr. \textunderscore halieutikos\textunderscore )}
\end{itemize}
Relativo á pesca.
\section{Halígona}
\begin{itemize}
\item {Grp. gram.:f.}
\end{itemize}
\begin{itemize}
\item {Proveniência:(Do gr. \textunderscore hals\textunderscore  + \textunderscore gone\textunderscore )}
\end{itemize}
Gênero de plantas.
\section{Halimeda}
\begin{itemize}
\item {Grp. gram.:f.}
\end{itemize}
\begin{itemize}
\item {Proveniência:(Do lat. \textunderscore haliums\textunderscore )}
\end{itemize}
Gênero de crustáceos decápodes, cujas espécies vivem nos mares do Japão.
Gênero de polypeiros flexíveis.
\section{Halimênia}
\begin{itemize}
\item {Grp. gram.:f.}
\end{itemize}
Gênero de algas.
\section{Halimetria}
\begin{itemize}
\item {Grp. gram.:f.}
\end{itemize}
(V.halometria)
\section{Hálimo}
\begin{itemize}
\item {Grp. gram.:m.}
\end{itemize}
\begin{itemize}
\item {Proveniência:(Lat. \textunderscore halimus\textunderscore )}
\end{itemize}
Gênero de plantas chenopódeas.
\section{Haliote}
\begin{itemize}
\item {Grp. gram.:m.  ou  f.}
\end{itemize}
\begin{itemize}
\item {Proveniência:(Do gr. \textunderscore hals\textunderscore  + \textunderscore ous\textunderscore , \textunderscore otos\textunderscore )}
\end{itemize}
Gênero de molluscos gasterópodes marinhos, de concha univalve.
\section{Haliplos}
\begin{itemize}
\item {Grp. gram.:m. pl.}
\end{itemize}
\begin{itemize}
\item {Proveniência:(Do gr. \textunderscore hals\textunderscore , \textunderscore halos\textunderscore  + \textunderscore pleo\textunderscore )}
\end{itemize}
Gênero de ínsectos coleópteros.
\section{Haliptenos}
\begin{itemize}
\item {Grp. gram.:m. pl.}
\end{itemize}
\begin{itemize}
\item {Proveniência:(Do gr. \textunderscore hals\textunderscore , \textunderscore halos\textunderscore  + \textunderscore ptenos\textunderscore )}
\end{itemize}
Família de aves marítimas.
\section{Haliquelidónios}
\begin{itemize}
\item {Grp. gram.:m. pl.}
\end{itemize}
\begin{itemize}
\item {Proveniência:(Do gr. \textunderscore hals\textunderscore , \textunderscore halos\textunderscore , mar, e \textunderscore khelidon\textunderscore , andorinha)}
\end{itemize}
Família de aves, que compreende a andorinha do mar.
\section{Halisáurio}
\begin{itemize}
\item {fónica:sau}
\end{itemize}
\begin{itemize}
\item {Grp. gram.:m.}
\end{itemize}
\begin{itemize}
\item {Proveniência:(Do gr. \textunderscore hals\textunderscore , \textunderscore halos\textunderscore , e \textunderscore sáurio\textunderscore )}
\end{itemize}
Sáurio, que vive no mar.
\section{Halispôngia}
\begin{itemize}
\item {Grp. gram.:f.}
\end{itemize}
\begin{itemize}
\item {Proveniência:(Do gr. \textunderscore hals\textunderscore  + \textunderscore spongos\textunderscore )}
\end{itemize}
Gênero de espongiários.
\section{Halissáurio}
\begin{itemize}
\item {Grp. gram.:m.}
\end{itemize}
\begin{itemize}
\item {Proveniência:(Do gr. \textunderscore hals\textunderscore , \textunderscore halos\textunderscore , e \textunderscore sáurio\textunderscore )}
\end{itemize}
Sáurio, que vive no mar.
\section{Hálito}
\begin{itemize}
\item {Grp. gram.:m.}
\end{itemize}
\begin{itemize}
\item {Proveniência:(Lat. \textunderscore halitus\textunderscore )}
\end{itemize}
Ar que sái dos pulmões, depois de aspirado.
Cheiro; exhalação.
Viração.
\section{Hallial}
\begin{itemize}
\item {Grp. gram.:adj.}
\end{itemize}
\begin{itemize}
\item {Proveniência:(Do lat. \textunderscore hallus\textunderscore )}
\end{itemize}
Relativo ao dedo pollegar.
\section{Hallo}
\begin{itemize}
\item {Grp. gram.:m.}
\end{itemize}
\begin{itemize}
\item {Utilização:Med.}
\end{itemize}
\begin{itemize}
\item {Utilização:Fig.}
\end{itemize}
\begin{itemize}
\item {Proveniência:(Gr. \textunderscore halos\textunderscore , eira, redondel)}
\end{itemize}
Espécie de corôa dupla e luminosa, que se apresenta em volta do Sol e de alguns planetas, em certas condições atmosphéricas.
Círculo avermelhado em volta do mamillo.
O mesmo que \textunderscore auréola\textunderscore .
Glória, prestígio.
\section{Halloysite}
\begin{itemize}
\item {Grp. gram.:f.}
\end{itemize}
Substância mineral, espécie de silicato aluminoso hydratado.
\section{Halo...}
\begin{itemize}
\item {Proveniência:(Do gr. \textunderscore hals\textunderscore , \textunderscore halos\textunderscore )}
\end{itemize}
Elemento, que entra na composição do nome de várias plantas, com a significação de \textunderscore sal\textunderscore  ou \textunderscore relativo a sal\textunderscore .
\section{Halochímica}
\begin{itemize}
\item {fónica:qui}
\end{itemize}
\begin{itemize}
\item {Grp. gram.:f.}
\end{itemize}
\begin{itemize}
\item {Proveniência:(Do gr. \textunderscore hals\textunderscore , \textunderscore halos\textunderscore  e \textunderscore Chímica\textunderscore )}
\end{itemize}
Parte da Chímica, que se occupa dos saes ou da história dos saes.
\section{Halófilo}
\begin{itemize}
\item {Grp. gram.:adj.}
\end{itemize}
\begin{itemize}
\item {Utilização:Bot.}
\end{itemize}
\begin{itemize}
\item {Proveniência:(Do gr. \textunderscore hals\textunderscore , \textunderscore halos\textunderscore  + \textunderscore philos\textunderscore )}
\end{itemize}
Que cresce em terrenos salgados.
\section{Halogêneo}
\begin{itemize}
\item {Grp. gram.:adj.}
\end{itemize}
\begin{itemize}
\item {Proveniência:(Do gr. \textunderscore hals\textunderscore  + \textunderscore genea\textunderscore )}
\end{itemize}
Diz-se dos corpos electrò-negativos, que produzem saes, combinando-se com os metaes electro-positivos.
\section{Halogênico}
\begin{itemize}
\item {Grp. gram.:adj.}
\end{itemize}
\begin{itemize}
\item {Proveniência:(De \textunderscore halogêneo\textunderscore )}
\end{itemize}
Diz-se do resíduo, que se obteria, privando do seu hydrogênio básico os ácidos oxygenados.
\section{Halógeno}
\begin{itemize}
\item {Grp. gram.:adj.}
\end{itemize}
O mesmo que \textunderscore halogêneo\textunderscore .
\section{Halografia}
\begin{itemize}
\item {Grp. gram.:f.}
\end{itemize}
\begin{itemize}
\item {Proveniência:(Do gr. \textunderscore hals\textunderscore  + \textunderscore graphein\textunderscore )}
\end{itemize}
Tratado dos saes.
\section{Halógrafo}
\begin{itemize}
\item {Grp. gram.:m.}
\end{itemize}
Aquele que é versado em halografia.
\section{Halographia}
\begin{itemize}
\item {Grp. gram.:f.}
\end{itemize}
\begin{itemize}
\item {Proveniência:(Do gr. \textunderscore hals\textunderscore  + \textunderscore graphein\textunderscore )}
\end{itemize}
Tratado dos saes.
\section{Halógrapho}
\begin{itemize}
\item {Grp. gram.:m.}
\end{itemize}
Aquelle que é versado em halographia.
\section{Haloide}
\begin{itemize}
\item {Grp. gram.:adj.}
\end{itemize}
\begin{itemize}
\item {Proveniência:(Do gr. \textunderscore hals\textunderscore  + \textunderscore eidos\textunderscore )}
\end{itemize}
Semelhante ao sal marinho.
Diz-se dos saes, que resultam da combinação de um corpo halogêneo com um metal.
\section{Haloisite}
\begin{itemize}
\item {fónica:lo-i}
\end{itemize}
\begin{itemize}
\item {Grp. gram.:f.}
\end{itemize}
Substância mineral, espécie de silicato aluminoso hydratado.
\section{Halologia}
\begin{itemize}
\item {Grp. gram.:f.}
\end{itemize}
\begin{itemize}
\item {Proveniência:(Do gr. \textunderscore hals\textunderscore  + \textunderscore logos\textunderscore )}
\end{itemize}
O mesmo que \textunderscore halographia\textunderscore .
\section{Halomancia}
\begin{itemize}
\item {Grp. gram.:f.}
\end{itemize}
\begin{itemize}
\item {Proveniência:(Do gr. \textunderscore hals\textunderscore , \textunderscore halos\textunderscore  + \textunderscore manteia\textunderscore )}
\end{itemize}
Supposta arte de adivinhar, por meio do sal.
\section{Halomântico}
\begin{itemize}
\item {Grp. gram.:m.}
\end{itemize}
Aquelle que pratïca a halomancia.
\section{Halometria}
\begin{itemize}
\item {Grp. gram.:f.}
\end{itemize}
\begin{itemize}
\item {Proveniência:(De \textunderscore halómetro\textunderscore )}
\end{itemize}
Processo, para avaliar a qualidade das soluções salinas, empregadas no commércio.
\section{Halométrico}
\begin{itemize}
\item {Grp. gram.:adj.}
\end{itemize}
Relativo á halometria.
\section{Halómetro}
\begin{itemize}
\item {Grp. gram.:m.}
\end{itemize}
\begin{itemize}
\item {Proveniência:(Do gr. \textunderscore hals\textunderscore , \textunderscore halos\textunderscore  + \textunderscore metron\textunderscore )}
\end{itemize}
Instrumento, para determinar o álcool e o extracto sêco, na cerveja e noutras bebidas alcoólicas.
\section{Halóphilo}
\begin{itemize}
\item {Grp. gram.:adj.}
\end{itemize}
\begin{itemize}
\item {Utilização:Bot.}
\end{itemize}
\begin{itemize}
\item {Proveniência:(Do gr. \textunderscore hals\textunderscore , \textunderscore halos\textunderscore  + \textunderscore philos\textunderscore )}
\end{itemize}
Que cresce em terrenos salgados.
\section{Haloplanta}
\begin{itemize}
\item {Grp. gram.:m.}
\end{itemize}
\begin{itemize}
\item {Utilização:P. us.}
\end{itemize}
\begin{itemize}
\item {Proveniência:(Gr. \textunderscore haloplantes\textunderscore )}
\end{itemize}
Impostor, embusteiro.
\section{Haloquímica}
\begin{itemize}
\item {Grp. gram.:f.}
\end{itemize}
\begin{itemize}
\item {Proveniência:(Do gr. \textunderscore hals\textunderscore , \textunderscore halos\textunderscore  e \textunderscore Química\textunderscore )}
\end{itemize}
Parte da Química, que se ocupa dos saes ou da história dos saes.
\section{Halorágeas}
\begin{itemize}
\item {Grp. gram.:f. pl.}
\end{itemize}
\begin{itemize}
\item {Proveniência:(Do gr. \textunderscore hals\textunderscore  + \textunderscore ragion\textunderscore )}
\end{itemize}
Familia de plantas polypétalas, geralmente aquáticas.
\section{Halòtechnia}
\begin{itemize}
\item {Grp. gram.:f.}
\end{itemize}
\begin{itemize}
\item {Proveniência:(Do gr. \textunderscore hals\textunderscore  + \textunderscore tekhnè\textunderscore )}
\end{itemize}
Parte da Chímica, que trata da preparação dos saes.
\section{Halotéchnico}
\begin{itemize}
\item {Grp. gram.:adj.}
\end{itemize}
Relativo a halotechnia.
\section{Halotecnia}
\begin{itemize}
\item {Grp. gram.:f.}
\end{itemize}
\begin{itemize}
\item {Proveniência:(Do gr. \textunderscore hals\textunderscore  + \textunderscore tekhnè\textunderscore )}
\end{itemize}
Parte da Química, que trata da preparação dos saes.
\section{Halotecnico}
\begin{itemize}
\item {Grp. gram.:adj.}
\end{itemize}
Relativo a halotecnia.
\section{Haltera}
\begin{itemize}
\item {Grp. gram.:f.}
\end{itemize}
\begin{itemize}
\item {Utilização:Bras}
\end{itemize}
O mesmo que \textunderscore haltere\textunderscore .
\section{Haltere}
\begin{itemize}
\item {Grp. gram.:m.}
\end{itemize}
\begin{itemize}
\item {Proveniência:(Lat. \textunderscore haltere\textunderscore )}
\end{itemize}
Instrumento de gymnástica, formado de duas espheras de ferro, reunidas por uma pequena haste do mesmo metal, que a mão abraça facilmente.
\section{Halurgia}
\begin{itemize}
\item {Grp. gram.:f.}
\end{itemize}
\begin{itemize}
\item {Proveniência:(Do gr. \textunderscore hals\textunderscore  + \textunderscore ergon\textunderscore )}
\end{itemize}
Arte de preparar saes.
\section{Hamadria}
\begin{itemize}
\item {Grp. gram.:f.}
\end{itemize}
\begin{itemize}
\item {Utilização:Myth.}
\end{itemize}
Nympha dos bosques. Cf. Filinto, II, 55.
(Cp. \textunderscore hamadrýada\textunderscore )
\section{Hamadrýada}
\begin{itemize}
\item {Grp. gram.:f.}
\end{itemize}
\begin{itemize}
\item {Proveniência:(Do gr. \textunderscore hamadruas\textunderscore )}
\end{itemize}
Planta ranunculácea.
\section{Hamamélida}
\begin{itemize}
\item {Grp. gram.:f.}
\end{itemize}
Gênero de plantas dicotyledónias da América do Norte.
\section{Hamamelídeas}
\begin{itemize}
\item {Grp. gram.:f. pl.}
\end{itemize}
Ordem de plantas, que têm por typo a hamamélida.
\section{Hambalita}
\begin{itemize}
\item {Grp. gram.:m.}
\end{itemize}
Um dos quatro ritos orthodoxos dos Muçulmanos. Cf. Benalcanfor, \textunderscore Cartas de Viagem\textunderscore , XXX.
\section{Hamburguês}
\begin{itemize}
\item {Grp. gram.:adj.}
\end{itemize}
\begin{itemize}
\item {Grp. gram.:M.}
\end{itemize}
Relativo a Hamburgo.
Habitante de Hamburgo.
\section{Hamélia}
\begin{itemize}
\item {Grp. gram.:f.}
\end{itemize}
\begin{itemize}
\item {Proveniência:(De Du-\textunderscore Hamel\textunderscore , n. p.)}
\end{itemize}
Gênero de arbustos americanos, da fam. das rubiáceas.
\section{Hameque}
\begin{itemize}
\item {Grp. gram.:m.}
\end{itemize}
\begin{itemize}
\item {Utilização:Des.}
\end{itemize}
\begin{itemize}
\item {Proveniência:(Do ár. \textunderscore habique\textunderscore )}
\end{itemize}
Electuário de coloquíntida.
\section{Hamígero}
\begin{itemize}
\item {Grp. gram.:adj.}
\end{itemize}
\begin{itemize}
\item {Utilização:Bot.}
\end{itemize}
\begin{itemize}
\item {Proveniência:(Do lat. \textunderscore hamus\textunderscore  + \textunderscore gerere\textunderscore )}
\end{itemize}
Que tem pêlos recurvados em fórma de anzol.
\section{Hamiglossos}
\begin{itemize}
\item {Grp. gram.:m. pl.}
\end{itemize}
\begin{itemize}
\item {Proveniência:(Do lat. \textunderscore hamus\textunderscore  + gr. \textunderscore glossa\textunderscore )}
\end{itemize}
Família de molluscos gasterópodes.
\section{Hamita}
\begin{itemize}
\item {Grp. gram.:f.}
\end{itemize}
\begin{itemize}
\item {Proveniência:(Do lat. \textunderscore hamus\textunderscore )}
\end{itemize}
Mollusco cephalópode fóssil.
\section{Hamítico}
\begin{itemize}
\item {Grp. gram.:adj.}
\end{itemize}
Diz-se de várias línguas da África setentrional.
\section{Hangar}
\begin{itemize}
\item {Grp. gram.:m.}
\end{itemize}
\begin{itemize}
\item {Proveniência:(Fr. \textunderscore hangar\textunderscore , do germ.)}
\end{itemize}
Tecto, suspenso por pilares.
Abrigo ou armazém aberto, para mercadorias; trapiche.
\section{Hango}
\begin{itemize}
\item {Grp. gram.:m.}
\end{itemize}
Ave gallinácea da África.
\section{Hanifismo}
\begin{itemize}
\item {Grp. gram.:m.}
\end{itemize}
Doutrinas dos hanifitas.
\section{Hanifita}
\begin{itemize}
\item {Grp. gram.:m.}
\end{itemize}
\begin{itemize}
\item {Proveniência:(De \textunderscore Hanifah\textunderscore , n. p.)}
\end{itemize}
Membro de uma das quatro seitas, consideradas orthodoxas, da religião muçulmana.
\section{Hanoveriano}
\begin{itemize}
\item {Grp. gram.:m.  e  adj.}
\end{itemize}
\begin{itemize}
\item {Grp. gram.:M.}
\end{itemize}
\begin{itemize}
\item {Proveniência:(De \textunderscore Hanóver\textunderscore  = \textunderscore Hanovre\textunderscore )}
\end{itemize}
O mesmo que \textunderscore hanovriano\textunderscore .
Cavallo hanoveriano:«\textunderscore ...caleche puxada por hanoverianos\textunderscore ». Camillo, \textunderscore Mulher Fatal\textunderscore , 25.
\section{Hanovriano}
\begin{itemize}
\item {Grp. gram.:adj.}
\end{itemize}
\begin{itemize}
\item {Grp. gram.:M.}
\end{itemize}
Relativo ao Hanovre.
Habitante do Hanovre.
\section{Hansa}
\begin{itemize}
\item {Grp. gram.:f.}
\end{itemize}
\begin{itemize}
\item {Proveniência:(Do ant. alt. al. \textunderscore hansa\textunderscore , companhia)}
\end{itemize}
Liga, entre várias cidades do Norte, na Idade-Média, para fins commerciaes.
Corporação de artes e offícios em França, na Idade-Média.
\section{Hanseático}
\begin{itemize}
\item {Grp. gram.:adj.}
\end{itemize}
Relativo á hansa; pertencente á hansa.
\section{Hansmanita}
\begin{itemize}
\item {Grp. gram.:f.}
\end{itemize}
Espécie de rocha, constituída por sesquióxydo de manganês.
\section{Hansmanite}
\begin{itemize}
\item {Grp. gram.:f.}
\end{itemize}
Espécie de rocha, constituída por sesquióxydo de manganês.
\section{Haplantho}
\begin{itemize}
\item {Grp. gram.:m.}
\end{itemize}
Gênero de plantas, vulgares na Índia portuguesa.
\section{Haplanto}
\begin{itemize}
\item {Grp. gram.:m.}
\end{itemize}
Gênero de plantas, vulgares na Índia portuguesa.
\section{Haplologia}
\begin{itemize}
\item {Grp. gram.:f.}
\end{itemize}
\begin{itemize}
\item {Utilização:Gram.}
\end{itemize}
\begin{itemize}
\item {Proveniência:(Do gr. \textunderscore haplos\textunderscore , simples, e \textunderscore logos\textunderscore , tratado)}
\end{itemize}
Contracção ou reducção dos elementos similares de um vocábulo: \textunderscore bondoso\textunderscore , por \textunderscore bondadoso\textunderscore ; \textunderscore semínima\textunderscore , por \textunderscore semimínima\textunderscore ; \textunderscore sericultura\textunderscore , por \textunderscore sericicultura\textunderscore , etc.
\section{Haplológico}
\begin{itemize}
\item {Grp. gram.:adj.}
\end{itemize}
Relativo á haplologia.
Em que há haplologia:«\textunderscore formicida é fórma haplológica de formicicida\textunderscore ».
\section{Haplopétalo}
\begin{itemize}
\item {Grp. gram.:adj.}
\end{itemize}
\begin{itemize}
\item {Utilização:Bot.}
\end{itemize}
\begin{itemize}
\item {Proveniência:(Do gr. \textunderscore haplos\textunderscore  + \textunderscore petalon\textunderscore )}
\end{itemize}
Diz-se das plantas, cuja corolla é formada de uma só pétala.
\section{Haplostela}
\begin{itemize}
\item {Grp. gram.:f.}
\end{itemize}
Gênero de orquídeas.
\section{Haplostella}
\begin{itemize}
\item {Grp. gram.:f.}
\end{itemize}
Gênero de orchídeas.
\section{Haplotomia}
\begin{itemize}
\item {Grp. gram.:f.}
\end{itemize}
\begin{itemize}
\item {Utilização:Cir.}
\end{itemize}
\begin{itemize}
\item {Proveniência:(Do gr. \textunderscore haplos\textunderscore  + \textunderscore tome\textunderscore )}
\end{itemize}
Simples incisão.
\section{Haptogêneo}
\begin{itemize}
\item {Grp. gram.:adj.}
\end{itemize}
\begin{itemize}
\item {Utilização:Chím.}
\end{itemize}
\begin{itemize}
\item {Proveniência:(Do gr. \textunderscore haptein\textunderscore  + \textunderscore genea\textunderscore )}
\end{itemize}
Diz-se da membrana ou pellícula, que se fórma em volta de um glóbulo de albumina, pôsto em contacto com uma gordura líquida.
\section{Haragano}
\begin{itemize}
\item {Grp. gram.:adj.}
\end{itemize}
\begin{itemize}
\item {Utilização:Bras. do S}
\end{itemize}
Diz-se do cavallo, que difficilmente se deixa agarrar.
(Cast. \textunderscore haragán\textunderscore , mandrião, vadio)
\section{Harari}
\begin{itemize}
\item {Grp. gram.:m.}
\end{itemize}
Língua africana, o mesmo que \textunderscore tigré\textunderscore .
\section{Harém}
\begin{itemize}
\item {Grp. gram.:m.}
\end{itemize}
\begin{itemize}
\item {Utilização:Fig.}
\end{itemize}
\begin{itemize}
\item {Proveniência:(Fr. \textunderscore harem\textunderscore , do ár. \textunderscore karam\textunderscore )}
\end{itemize}
Lugar, em que se guardam as concubinas de um sultão.
Serralho.
Conjunto das odaliscas de um harém.
Parte da casa de um muçulmano, destinada á habitação das mulheres.
Lupanar.
\section{Harenque}
\begin{itemize}
\item {Grp. gram.:m.}
\end{itemize}
\begin{itemize}
\item {Utilização:Fam.}
\end{itemize}
\begin{itemize}
\item {Proveniência:(Do ant. alt. al. \textunderscore harinc\textunderscore )}
\end{itemize}
Tríbo de peixes marinhos, (\textunderscore clupea karangus\textunderscore ).
Pessôa tisnada e magra.
\section{Harfango}
\begin{itemize}
\item {Grp. gram.:m.}
\end{itemize}
Ave nocturna de rapina.
(Do sueco \textunderscore hurfang\textunderscore )
\section{Haridim}
\begin{itemize}
\item {Grp. gram.:m.}
\end{itemize}
Um dos noventa graus do rito maçónico egýpcio de Misraim.
\section{Haríolo}
\begin{itemize}
\item {Grp. gram.:m.}
\end{itemize}
\begin{itemize}
\item {Proveniência:(Lat. \textunderscore hariolus\textunderscore )}
\end{itemize}
O mesmo que \textunderscore adivinho\textunderscore .
\section{Harlina}
\begin{itemize}
\item {Grp. gram.:f.}
\end{itemize}
\begin{itemize}
\item {Utilização:Chím.}
\end{itemize}
Substância crystallizável, que, com a hartite, se encontra no carvão mineral.
\section{Harlo}
\begin{itemize}
\item {Grp. gram.:m.}
\end{itemize}
Lontra marinha, espécie de castor.
Ave palmípede, das regiões setentrionaes, abutre da Islândia.
\section{Harmala}
\begin{itemize}
\item {Grp. gram.:f.}
\end{itemize}
\begin{itemize}
\item {Proveniência:(Do ár. \textunderscore harmal\textunderscore )}
\end{itemize}
Espécie de arruda silvestre.
\section{Harmalina}
\begin{itemize}
\item {Grp. gram.:f.}
\end{itemize}
Substância, que se encontra nas sementes da harmala.
\section{Harmatão}
\begin{itemize}
\item {Grp. gram.:f.}
\end{itemize}
\begin{itemize}
\item {Proveniência:(T. afr.)}
\end{itemize}
Vento muito quente, que sopra no Senegal em Dezembro, Janeiro e Fevereiro.
\section{Harmófano}
\begin{itemize}
\item {Grp. gram.:adj.}
\end{itemize}
\begin{itemize}
\item {Proveniência:(Do gr. \textunderscore harmos\textunderscore  + \textunderscore phane\textunderscore )}
\end{itemize}
Diz-se do mineral que apresenta indícios de ligações naturaes.
\section{Harmonia}
\begin{itemize}
\item {Grp. gram.:f.}
\end{itemize}
\begin{itemize}
\item {Proveniência:(Lat. \textunderscore harmonia\textunderscore )}
\end{itemize}
Disposição entre as partes de um todo, concorrendo todas para o mesmo fim.
Concórdia.
Conjunto das qualidades, que tornam a phrase ou o discurso agradável ao ouvido.
Sons consonantes ou successão de sons agradáveis ao ouvido.
Tudo que é agradável ao ouvido.
Arte de combinar os sons ou de formar os acordes.
União por engrenagem.
Acôrdo.
Paz e amizade (entre duas ou mais pessoas).
\section{Harmoníaco}
\begin{itemize}
\item {Grp. gram.:adj.}
\end{itemize}
\begin{itemize}
\item {Utilização:Des.}
\end{itemize}
\begin{itemize}
\item {Proveniência:(Lat. \textunderscore harmoniacus\textunderscore )}
\end{itemize}
O mesmo que \textunderscore harmónico\textunderscore .
\section{Harmónica}
\begin{itemize}
\item {Grp. gram.:f.}
\end{itemize}
\begin{itemize}
\item {Proveniência:(De \textunderscore harmónico\textunderscore )}
\end{itemize}
Instrumento músico com teclas de vidro.
Marimba.
Pequeno instrumento, de folles, espécie de órgão portátil.
Registo, extremamente suave, nos órgãos alemães.
\textunderscore Harmónica chímica\textunderscore , apparelho, composto de um frasco, que produz hydrogênio, o qual, ardendo na extremidade de um tubo, faz vibrar outro tubo que envolve aquelle.
\section{Harmónicamente}
\begin{itemize}
\item {Grp. gram.:adv.}
\end{itemize}
De modo harmónico; com harmonia.
\section{Harmónico}
\begin{itemize}
\item {Grp. gram.:adj.}
\end{itemize}
\begin{itemize}
\item {Proveniência:(Lat. \textunderscore harmonicus\textunderscore )}
\end{itemize}
Relativo á harmonia.
Que tem harmonia.
Coherente.
Regular; proporcionado.
\section{Harmonicorde}
\begin{itemize}
\item {Grp. gram.:m.}
\end{itemize}
\begin{itemize}
\item {Proveniência:(Do gr. \textunderscore harmonia\textunderscore  + \textunderscore khorde\textunderscore )}
\end{itemize}
Piano de cauda, acompanhado de um mecanismo, que se põe em movimento com o pé.
\section{Harmónio}
\begin{itemize}
\item {Grp. gram.:m.}
\end{itemize}
\begin{itemize}
\item {Proveniência:(Fr. \textunderscore harmonium\textunderscore )}
\end{itemize}
Pequeno órgão de sala.
\section{Harmoniosamente}
\begin{itemize}
\item {Grp. gram.:adv.}
\end{itemize}
De modo harmonioso.
\section{Harmonioso}
\begin{itemize}
\item {Grp. gram.:adj.}
\end{itemize}
Que tem harmonia.
Que tem sons agradáveis ao ouvido.
\section{Harmonista}
\begin{itemize}
\item {Grp. gram.:m.}
\end{itemize}
\begin{itemize}
\item {Proveniência:(De \textunderscore harmonia\textunderscore )}
\end{itemize}
Músico, que conhece as regras da harmonia.
Pintor, que comprehende a harmonia das côres.
\section{Harmonística}
\begin{itemize}
\item {Grp. gram.:f.}
\end{itemize}
Processo de conciliar as differentes passagens do \textunderscore Novo Testamento\textunderscore , que parecem contradictórias.
(Cp. \textunderscore harmonizar\textunderscore )
\section{Harmónium}
\begin{itemize}
\item {Grp. gram.:m.}
\end{itemize}
(V.harmónio)
\section{Harmonização}
\begin{itemize}
\item {Grp. gram.:f.}
\end{itemize}
Acto ou effeito de harmonizar.
\section{Harmonizar}
\begin{itemize}
\item {Grp. gram.:v. t.}
\end{itemize}
\begin{itemize}
\item {Proveniência:(De \textunderscore harmonia\textunderscore )}
\end{itemize}
Tornar harmónico.
Conciliar; pôr em harmonia.
Dividir em partes harmónicas.
\section{Harmonómetro}
\begin{itemize}
\item {Grp. gram.:m.}
\end{itemize}
Instrumento, que mede as relações harmónicas dos sons.
(Por \textunderscore harmoniómetro\textunderscore , do gr. \textunderscore harmonia\textunderscore  + \textunderscore metron\textunderscore )
\section{Harmóphano}
\begin{itemize}
\item {Grp. gram.:adj.}
\end{itemize}
\begin{itemize}
\item {Proveniência:(Do gr. \textunderscore harmos\textunderscore  + \textunderscore phane\textunderscore )}
\end{itemize}
Diz-se do mineral que apresenta indícios de ligações naturaes.
\section{Harmosta}
\begin{itemize}
\item {Grp. gram.:m.}
\end{itemize}
\begin{itemize}
\item {Proveniência:(Gr. \textunderscore harmostes\textunderscore )}
\end{itemize}
Designação de cada um dos governadores que Esparta impunha aos povos vencidos.
\section{Harmótoma}
\begin{itemize}
\item {Grp. gram.:f.}
\end{itemize}
O mesmo ou melhor que \textunderscore harmótomo\textunderscore .
\section{Harmótomo}
\begin{itemize}
\item {Grp. gram.:m.}
\end{itemize}
\begin{itemize}
\item {Proveniência:(Do gr. \textunderscore harmos\textunderscore  + \textunderscore tome\textunderscore )}
\end{itemize}
Mineral alvacento, cujos crystaes estão divididos por junturas.
\section{Harpa}
\begin{itemize}
\item {Grp. gram.:f.}
\end{itemize}
\begin{itemize}
\item {Utilização:Ext.}
\end{itemize}
Instrumento triangular, de cordas desiguaes, que se tocam com os dedos.
A poesia religiosa.
A poesia em geral.
Mollusco gasterópode.
(B. lat. \textunderscore harpa\textunderscore )
\section{Harpa-eólia}
\begin{itemize}
\item {Grp. gram.:f.}
\end{itemize}
Instrumento antigo, ainda usado no século XVIII, composto de uma caixa, armada de cordas metállicas e que se collocava nas árvores ou em bosques de recreio, para surprehender agradavelmente o transeunte com os sons que o ar produzia naquelle instrumento.
\section{Hárpaga}
\begin{itemize}
\item {Grp. gram.:f.}
\end{itemize}
\begin{itemize}
\item {Proveniência:(Lat. \textunderscore harpaga\textunderscore )}
\end{itemize}
Antiga máquina de guerra, espécie de catapulta.
\section{Harpálio}
\begin{itemize}
\item {Grp. gram.:m.}
\end{itemize}
\begin{itemize}
\item {Proveniência:(De \textunderscore Hárpalo\textunderscore , n. p.)}
\end{itemize}
Planta de jardins, originária da América do Norte.
\section{Harpaneta}
\begin{itemize}
\item {fónica:nê}
\end{itemize}
\begin{itemize}
\item {Grp. gram.:f.}
\end{itemize}
Antigo instrumento de cordas.
\section{Harpar}
\begin{itemize}
\item {Grp. gram.:v. t.}
\end{itemize}
O mesmo que \textunderscore harpear\textunderscore .
\section{Harpe}
\begin{itemize}
\item {Grp. gram.:f.}
\end{itemize}
\begin{itemize}
\item {Utilização:Ant.}
\end{itemize}
\begin{itemize}
\item {Proveniência:(Lat. \textunderscore harpe\textunderscore )}
\end{itemize}
Espada curva.
A espada de Mercúrio.
\section{Harpear}
\begin{itemize}
\item {Grp. gram.:v. t.}
\end{itemize}
\begin{itemize}
\item {Grp. gram.:V. i.}
\end{itemize}
Tocar na harpa: \textunderscore harpear uma cavatina\textunderscore .
Tocar harpa.
\section{Harpeiro}
\begin{itemize}
\item {Grp. gram.:m.  e  adj.}
\end{itemize}
\begin{itemize}
\item {Utilização:Ant.}
\end{itemize}
O que toca harpa; harpista.
\section{Harpia}
\begin{itemize}
\item {Grp. gram.:f.}
\end{itemize}
\begin{itemize}
\item {Utilização:Fig.}
\end{itemize}
\begin{itemize}
\item {Proveniência:(Gr. \textunderscore harpuia\textunderscore )}
\end{itemize}
Monstro fabuloso, com rosto de mulher, corpo de abutre, asas, etc.
Espécie de falcão da América.
Pessôa, que vive de extorsões.
\section{Harpista}
\begin{itemize}
\item {Grp. gram.:m.  e  f.}
\end{itemize}
Pessôa, que toca harpa ou que ensina a tocar harpa.
\section{Hartite}
\begin{itemize}
\item {Grp. gram.:f.}
\end{itemize}
Substância, que se encontra no carvão mineral.
\section{Harto}
\begin{itemize}
\item {Grp. gram.:adj.}
\end{itemize}
\begin{itemize}
\item {Grp. gram.:Adv.}
\end{itemize}
Forte.
Cheio.
Muito; assaz.
(Cast. \textunderscore harto\textunderscore )
\section{Harveyização}
\begin{itemize}
\item {Grp. gram.:f.}
\end{itemize}
\begin{itemize}
\item {Proveniência:(De \textunderscore Harvey\textunderscore , n. p.)}
\end{itemize}
Processo de composição de chapas, que resistem á perfuração das balas.
\section{Hasta}
\begin{itemize}
\item {Grp. gram.:f.}
\end{itemize}
\begin{itemize}
\item {Proveniência:(Lat. \textunderscore hasta\textunderscore )}
\end{itemize}
Lança.
O mesmo que \textunderscore leilão\textunderscore .
\section{Hastado}
\begin{itemize}
\item {Grp. gram.:m.}
\end{itemize}
O mesmo que \textunderscore hastato\textunderscore .
\section{Hastapura}
\begin{itemize}
\item {Grp. gram.:f.}
\end{itemize}
\begin{itemize}
\item {Proveniência:(De \textunderscore hasta\textunderscore  + \textunderscore puro\textunderscore )}
\end{itemize}
Lança sem ferro, que se dava como prêmio aos mancebos, que se distinguiam no primeiro combate.
\section{Hastaria}
\begin{itemize}
\item {Grp. gram.:f.}
\end{itemize}
\begin{itemize}
\item {Utilização:Des.}
\end{itemize}
\begin{itemize}
\item {Proveniência:(De \textunderscore hasta\textunderscore )}
\end{itemize}
Lugar, em que se encostavam lanças.
\section{Hastário}
\begin{itemize}
\item {Grp. gram.:adj.}
\end{itemize}
\begin{itemize}
\item {Proveniência:(Lat. \textunderscore hastarius\textunderscore )}
\end{itemize}
O mesmo que \textunderscore hastato\textunderscore .
\section{Hastato}
\begin{itemize}
\item {Grp. gram.:adj.}
\end{itemize}
\begin{itemize}
\item {Grp. gram.:M.}
\end{itemize}
\begin{itemize}
\item {Proveniência:(Lat. \textunderscore hastatus\textunderscore )}
\end{itemize}
Armado de hasta.
Armado de pontas; córneo.
Soldado romano, armado de hasta ou de lança.
\section{Haste}
\begin{itemize}
\item {Grp. gram.:f.}
\end{itemize}
\begin{itemize}
\item {Proveniência:(Do lat. \textunderscore hasta\textunderscore )}
\end{itemize}
Pau ou ferro direito, delgado e comprido, em que se encrava ou apoia qualquer coisa.
Pau de bandeira.
Tronco, caule.
Vergôntea.
Pedúnculo.
Chifre, corno.
\section{Hástea}
\begin{itemize}
\item {Grp. gram.:f.}
\end{itemize}
(V.haste). Cf. \textunderscore Luz e Calor\textunderscore , 509.
\section{Hasteal}
\begin{itemize}
\item {Grp. gram.:m.}
\end{itemize}
\begin{itemize}
\item {Proveniência:(De \textunderscore haste\textunderscore )}
\end{itemize}
Complexo das hastes ou ramos, que partem do veio mineral.
\section{Hastear}
\begin{itemize}
\item {Grp. gram.:v. t.}
\end{itemize}
\begin{itemize}
\item {Grp. gram.:V. p.}
\end{itemize}
\begin{itemize}
\item {Proveniência:(De \textunderscore haste\textunderscore )}
\end{itemize}
Elevar ou prender ao cimo de uma haste.
Desfraldar; içar: \textunderscore hastear uma bandeira\textunderscore .
Içar-se.
Erguer-se alto; desfraldar-se.
\section{Hástia}
\begin{itemize}
\item {Grp. gram.:f.}
\end{itemize}
O mesmo que \textunderscore haste\textunderscore . Cf. Filinto, \textunderscore D. Man.\textunderscore , II, 129.
\section{Hastibranco}
\begin{itemize}
\item {Grp. gram.:adj.}
\end{itemize}
\begin{itemize}
\item {Proveniência:(De \textunderscore haste\textunderscore  + \textunderscore branco\textunderscore )}
\end{itemize}
Diz-se do toiro, que tem as hastes brancas com ponta negra.
\section{Hastifino}
\begin{itemize}
\item {Grp. gram.:adj.}
\end{itemize}
\begin{itemize}
\item {Proveniência:(De \textunderscore haste\textunderscore  + \textunderscore fino\textunderscore )}
\end{itemize}
Que tem hastes delgadas, (falando-se do toiro).
\section{Hastifoliado}
\begin{itemize}
\item {Grp. gram.:adj.}
\end{itemize}
\begin{itemize}
\item {Utilização:Bot.}
\end{itemize}
\begin{itemize}
\item {Proveniência:(Do lat. \textunderscore hasta\textunderscore  + \textunderscore folium\textunderscore )}
\end{itemize}
Que tem fôlhas lanceoladas ou em fórma de ferro de lança.
\section{Hastiforme}
\begin{itemize}
\item {Grp. gram.:adj.}
\end{itemize}
\begin{itemize}
\item {Proveniência:(Do lat. \textunderscore hasta\textunderscore  + \textunderscore forma\textunderscore )}
\end{itemize}
Que tem fórma de lança.
\section{Hastil}
\begin{itemize}
\item {Grp. gram.:m.}
\end{itemize}
\begin{itemize}
\item {Proveniência:(De \textunderscore haste\textunderscore )}
\end{itemize}
Haste.
Cabo de lança.
Pequena haste.
Vergôntea.
Pedúnculo.
\section{Hastilha}
\begin{itemize}
\item {Grp. gram.:f.}
\end{itemize}
Pequena haste.
\section{Hastilhaço}
\begin{itemize}
\item {Grp. gram.:m.}
\end{itemize}
\begin{itemize}
\item {Proveniência:(De \textunderscore hastilha\textunderscore )}
\end{itemize}
O mesmo que \textunderscore estilhaço\textunderscore .
\section{Hastilheira}
\begin{itemize}
\item {Grp. gram.:f.}
\end{itemize}
\begin{itemize}
\item {Proveniência:(De \textunderscore hasta\textunderscore )}
\end{itemize}
Lugar ou peça, a que se encostavam as lanças; hastaria.
\section{Hastim}
\begin{itemize}
\item {Grp. gram.:m.}
\end{itemize}
\begin{itemize}
\item {Utilização:Prov.}
\end{itemize}
\begin{itemize}
\item {Proveniência:(De \textunderscore haste\textunderscore )}
\end{itemize}
Antiga medida agrária.
Coirela.
Porção de terreno, comprida e estreita.
\section{Hastiverde}
\begin{itemize}
\item {Grp. gram.:adj.}
\end{itemize}
\begin{itemize}
\item {Proveniência:(De \textunderscore haste\textunderscore  + \textunderscore verde\textunderscore )}
\end{itemize}
Que tem hastes esverdeadas, (falando-se do toiro).
\section{Hastre}
\begin{itemize}
\item {Grp. gram.:f.}
\end{itemize}
\begin{itemize}
\item {Utilização:Prov.}
\end{itemize}
\begin{itemize}
\item {Utilização:trasm.}
\end{itemize}
O mesmo que \textunderscore haste\textunderscore .
\section{Hauarunas}
\begin{itemize}
\item {Grp. gram.:m. pl.}
\end{itemize}
Tríbo selvagem do Alto Amazonas.
\section{Haúça}
\begin{itemize}
\item {Grp. gram.:m.}
\end{itemize}
A língua commercial de uma grande parte do Sudão.
\section{Haurir}
\begin{itemize}
\item {Grp. gram.:v. t.}
\end{itemize}
\begin{itemize}
\item {Proveniência:(Lat. \textunderscore haurire\textunderscore )}
\end{itemize}
Esgotar.
Sorver, aspirar: \textunderscore haurir perfumes\textunderscore .
\section{Haurível}
\begin{itemize}
\item {Grp. gram.:adj.}
\end{itemize}
Que se póde haurir.
\section{Haúsa}
\begin{itemize}
\item {Grp. gram.:m.}
\end{itemize}
A língua commercial de uma grande parte do Sudão.
\section{Haussmannita}
\begin{itemize}
\item {Grp. gram.:f.}
\end{itemize}
\begin{itemize}
\item {Proveniência:(De \textunderscore Haussmann\textunderscore , n. p.)}
\end{itemize}
Um dos óxydos do manganés.
\section{Haustello}
\begin{itemize}
\item {Grp. gram.:m.}
\end{itemize}
\begin{itemize}
\item {Proveniência:(Lat. \textunderscore haustellum\textunderscore )}
\end{itemize}
O sugadoiro de certos insectos.
\section{Haustelo}
\begin{itemize}
\item {Grp. gram.:m.}
\end{itemize}
\begin{itemize}
\item {Proveniência:(Lat. \textunderscore haustellum\textunderscore )}
\end{itemize}
O sugadoiro de certos insectos.
\section{Hausto}
\begin{itemize}
\item {Grp. gram.:m.}
\end{itemize}
\begin{itemize}
\item {Proveniência:(Lat. \textunderscore haustus\textunderscore )}
\end{itemize}
Acto de haurir.
Sôrvo; gole, trago.
Medicamento, que se bebe.
\section{Havaneiro}
\begin{itemize}
\item {Grp. gram.:m.}
\end{itemize}
Official das fábricas de tabacos, encarregado dos productos que imitam os cigarros e charutos da Havana.
\section{Havanera}
\begin{itemize}
\item {Grp. gram.:f.}
\end{itemize}
Canto popular da Havana.
\section{Havanês}
\begin{itemize}
\item {Grp. gram.:adj.}
\end{itemize}
\begin{itemize}
\item {Grp. gram.:M.}
\end{itemize}
Relativo á Havana.
Indivíduo natural da Havana.
\section{Havano}
\begin{itemize}
\item {Grp. gram.:m.}
\end{itemize}
Charuto, fabricado em Havana, ou que imita os alli fabricados.
\section{Havedoiro}
\begin{itemize}
\item {Grp. gram.:adj.}
\end{itemize}
\begin{itemize}
\item {Utilização:Ant.}
\end{itemize}
\begin{itemize}
\item {Proveniência:(De \textunderscore haver\textunderscore )}
\end{itemize}
Que deve sêr considerado ou feito em termos hábeis, discretamente, convenientemente.
\section{Havedouro}
\begin{itemize}
\item {Grp. gram.:adj.}
\end{itemize}
\begin{itemize}
\item {Utilização:Ant.}
\end{itemize}
\begin{itemize}
\item {Proveniência:(De \textunderscore haver\textunderscore )}
\end{itemize}
Que deve sêr considerado ou feito em termos hábeis, discretamente, convenientemente.
\section{Haver}
\begin{itemize}
\item {Grp. gram.:v. t.}
\end{itemize}
\begin{itemize}
\item {Grp. gram.:V. impess.}
\end{itemize}
\begin{itemize}
\item {Grp. gram.:M.}
\end{itemize}
\begin{itemize}
\item {Grp. gram.:Pl.}
\end{itemize}
\begin{itemize}
\item {Proveniência:(Lat. \textunderscore habere\textunderscore )}
\end{itemize}
Têr: \textunderscore aquelle rei houve dez filhos bastardos\textunderscore .
Estar na posse de.
Conseguir.
Receber: \textunderscore a paga que êlle há de haver\textunderscore .
Julgar: \textunderscore hei por bem ordenar...\textunderscore 
Conceber.
Existir: \textunderscore há homens, que são feras\textunderscore .
Acontecer: \textunderscore houve ontem desordens\textunderscore .
O mesmo que \textunderscore ha-de-haver\textunderscore , em escrituração ou linguagem commercial.
Bens, fazenda.
Bagagens.
O que alguém tem ou possue; fazenda; propriedades.
Riqueza: \textunderscore homem de muitos haveres\textunderscore .
\section{Hawaiano}
\begin{itemize}
\item {Grp. gram.:adj.}
\end{itemize}
\begin{itemize}
\item {Grp. gram.:M.}
\end{itemize}
Relativo ás ilhas de Hawai.
Habitante do archipélago de Hawai.
\section{Haxixe}
\begin{itemize}
\item {Grp. gram.:m.}
\end{itemize}
Fôlhas de cânhamo índico, que se secam, para se fumarem ou se mascarem.
Poção narcótica, feita com aquelle vegetal e que produz uma somnolência, acompanhada de visões deliciosas e de estranhas illusões.
(Ár. \textunderscore haxixe\textunderscore , erva sêca)
\section{Hazazel}
\begin{itemize}
\item {Grp. gram.:m.}
\end{itemize}
\begin{itemize}
\item {Proveniência:(T. hebr.)}
\end{itemize}
Nome, com que se designava o bode expiatório, entre os Hebreus.
\section{Heá}
\begin{itemize}
\item {Grp. gram.:m.}
\end{itemize}
Espécie de macaco do Amazonas.
\section{Heantognose}
\begin{itemize}
\item {Grp. gram.:f.}
\end{itemize}
\begin{itemize}
\item {Proveniência:(Do gr. \textunderscore heantos\textunderscore  + \textunderscore gnosis\textunderscore )}
\end{itemize}
Conhecimento de si próprio.
\section{Hebdômada}
\begin{itemize}
\item {Grp. gram.:f.}
\end{itemize}
\begin{itemize}
\item {Proveniência:(Lat. \textunderscore hebdomada\textunderscore )}
\end{itemize}
Semana.
Espaço de sete dias, semanas ou annos.
\section{Hebdomadariamente}
\begin{itemize}
\item {Grp. gram.:adv.}
\end{itemize}
De modo hebdomadário; semanalmente.
\section{Hebdomadário}
\begin{itemize}
\item {Grp. gram.:adj.}
\end{itemize}
\begin{itemize}
\item {Grp. gram.:M.}
\end{itemize}
\begin{itemize}
\item {Utilização:Ant.}
\end{itemize}
\begin{itemize}
\item {Proveniência:(Lat. \textunderscore hebdomadarius\textunderscore )}
\end{itemize}
Relativo á semana.
Publicação semanal.
Nome, que se dava em algumas cathedraes ao beneficiado, de ordem immediatamente inferior á dos cónegos.
\section{Hebdomático}
\begin{itemize}
\item {Grp. gram.:adj.}
\end{itemize}
\begin{itemize}
\item {Proveniência:(Lat. \textunderscore hebdomaticus\textunderscore )}
\end{itemize}
Relativo a sete.
\section{Hebeclínio}
\begin{itemize}
\item {Grp. gram.:m.}
\end{itemize}
\begin{itemize}
\item {Proveniência:(Do gr. \textunderscore hebe\textunderscore  + \textunderscore kline\textunderscore )}
\end{itemize}
Género de plantas compostas.
\section{Hebetação}
\begin{itemize}
\item {Grp. gram.:f.}
\end{itemize}
\begin{itemize}
\item {Proveniência:(Lat. \textunderscore sebetatio\textunderscore )}
\end{itemize}
Acto ou effeito de hebetar.
\section{Hebetante}
\begin{itemize}
\item {Grp. gram.:adj.}
\end{itemize}
\begin{itemize}
\item {Proveniência:(Lat. \textunderscore hebetans\textunderscore )}
\end{itemize}
Que hebeta.
\section{Hebetar}
\begin{itemize}
\item {Grp. gram.:v. t.}
\end{itemize}
\begin{itemize}
\item {Proveniência:(Lat. \textunderscore hebetare\textunderscore )}
\end{itemize}
Tornar bronco, obtuso, embotado.
\section{Hebetismo}
\begin{itemize}
\item {Grp. gram.:m.}
\end{itemize}
\begin{itemize}
\item {Proveniência:(Do lat. \textunderscore hebes\textunderscore , \textunderscore hebetis\textunderscore )}
\end{itemize}
Qualidade de hebetado ou de estúpido; imbecilidade.
\section{Hebraico}
\begin{itemize}
\item {Grp. gram.:m.}
\end{itemize}
\begin{itemize}
\item {Grp. gram.:Adj.}
\end{itemize}
\begin{itemize}
\item {Proveniência:(Gr. \textunderscore hebraikos\textunderscore )}
\end{itemize}
Idioma dos hebreus.
Hebreu.
Espécie de insecto.
Relativo aos Hebreus.
\section{Hebraísmo}
\begin{itemize}
\item {Grp. gram.:m.}
\end{itemize}
Locução propria da língua hebraica.
(Cp. \textunderscore hebraizar\textunderscore )
\section{Hebraísta}
\begin{itemize}
\item {Grp. gram.:m.}
\end{itemize}
Aquelle que se dedica ao estudo do hebreu.
(Cp. \textunderscore hebraizar\textunderscore )
\section{Hebraizante}
\begin{itemize}
\item {fónica:bra-i}
\end{itemize}
\begin{itemize}
\item {Grp. gram.:adj.}
\end{itemize}
Que hebraíza.
\section{Hebraizar}
\begin{itemize}
\item {fónica:bra-i}
\end{itemize}
\begin{itemize}
\item {Grp. gram.:v. i.}
\end{itemize}
\begin{itemize}
\item {Proveniência:(Gr. \textunderscore hebraizein\textunderscore )}
\end{itemize}
Conhecer o hebraico.
Seguir as doutrinas ou praticar a religião dos Hebreus; judaizar.
\section{Hebreu}
\begin{itemize}
\item {Grp. gram.:m.}
\end{itemize}
\begin{itemize}
\item {Grp. gram.:Pl.}
\end{itemize}
\begin{itemize}
\item {Grp. gram.:Adj.}
\end{itemize}
\begin{itemize}
\item {Proveniência:(Lat. \textunderscore hebraeus\textunderscore )}
\end{itemize}
Língua hebraica.
Indivíduo de raça hebraica.
Nome primitivo do povo judaico.
Relativo ou pertencente aos Hebreus.
\section{Hecateia}
\begin{itemize}
\item {Grp. gram.:f.}
\end{itemize}
\begin{itemize}
\item {Proveniência:(Gr. \textunderscore hekateia\textunderscore )}
\end{itemize}
Fantasma giganteu, que apparecia, segundo se acreditava, durante as hecatésias.
Árvore euphorbiácea de Madagáscar.
\section{Hecatésia}
\begin{itemize}
\item {Grp. gram.:f.}
\end{itemize}
\begin{itemize}
\item {Grp. gram.:F. pl.}
\end{itemize}
Gênero de insectos crepusculares da Austrália.
Festas em honra da Hécate, que se celebravam em vários pontos da Grécia.
(Gr. \textunderscore hekatesial\textunderscore ).
\section{Hecatomba}
\begin{itemize}
\item {Grp. gram.:f.}
\end{itemize}
O mesmo que \textunderscore hecatombe\textunderscore :«\textunderscore ...o Senhor não quer hecatombas...\textunderscore »Camillo, \textunderscore Suicida\textunderscore , 105.
\section{Hecatombaio}
\begin{itemize}
\item {Grp. gram.:m.}
\end{itemize}
\begin{itemize}
\item {Proveniência:(Gr. \textunderscore hekatombaios\textunderscore )}
\end{itemize}
Primeiro mês do anno áttico, no qual se sacrificavam cem bois, em honra de Júpiter.
\section{Hecatombe}
\begin{itemize}
\item {Grp. gram.:m.}
\end{itemize}
\begin{itemize}
\item {Utilização:Ext.}
\end{itemize}
\begin{itemize}
\item {Utilização:Fig.}
\end{itemize}
\begin{itemize}
\item {Proveniência:(Do gr. \textunderscore hecatombe\textunderscore )}
\end{itemize}
Antigo sacrifício de cem bois.
Sacrifício de muitas víctimas.
Matança humana.
\section{Hecatômpedo}
\begin{itemize}
\item {Grp. gram.:m.}
\end{itemize}
\begin{itemize}
\item {Proveniência:(Lat. \textunderscore hecatompedum\textunderscore )}
\end{itemize}
Templo de cem pés de extensão, como o de Mínerva em Athenas.
\section{Hecatonstilo}
\begin{itemize}
\item {Grp. gram.:m.}
\end{itemize}
\begin{itemize}
\item {Proveniência:(Do gr. \textunderscore hecaton\textunderscore  + \textunderscore stulos\textunderscore )}
\end{itemize}
Pórtico ou edifício de cem columnas.
\section{Hecatonstylo}
\begin{itemize}
\item {Grp. gram.:m.}
\end{itemize}
\begin{itemize}
\item {Proveniência:(Do gr. \textunderscore hecaton\textunderscore  + \textunderscore stulos\textunderscore )}
\end{itemize}
Pórtico ou edifício de cem columnas.
\section{Hecatontarca}
\begin{itemize}
\item {Grp. gram.:m.}
\end{itemize}
Chefe de uma hecatontarchia.
\section{Hecatontarcha}
\begin{itemize}
\item {fónica:ca}
\end{itemize}
\begin{itemize}
\item {Grp. gram.:m.}
\end{itemize}
Chefe de uma hecatontarchia.
\section{Hecatontarchia}
\begin{itemize}
\item {fónica:qui}
\end{itemize}
\begin{itemize}
\item {Grp. gram.:f.}
\end{itemize}
\begin{itemize}
\item {Proveniência:(Do gr. \textunderscore hekatontas\textunderscore  + \textunderscore arkhein\textunderscore )}
\end{itemize}
Subdivisão do antigo exército grego, composta de 128 soldados de infantaria ligeira.
\section{Hecatontarquia}
\begin{itemize}
\item {Grp. gram.:f.}
\end{itemize}
\begin{itemize}
\item {Proveniência:(Do gr. \textunderscore hekatontas\textunderscore  + \textunderscore arkhein\textunderscore )}
\end{itemize}
Subdivisão do antigo exército grego, composta de 128 soldados de infantaria ligeira.
\section{Hechor}
\begin{itemize}
\item {Grp. gram.:adj.}
\end{itemize}
\begin{itemize}
\item {Utilização:Bras}
\end{itemize}
Diz-se do burro, que vai na frente de uma manada de éguas.
(Cast. ant. \textunderscore hechor\textunderscore )
\section{Hectare}
\begin{itemize}
\item {Grp. gram.:m.}
\end{itemize}
\begin{itemize}
\item {Proveniência:(De \textunderscore hecto...\textunderscore  + \textunderscore are\textunderscore )}
\end{itemize}
Medida agrária, equivalente a cem ares.
\section{Héctica}
\begin{itemize}
\item {Grp. gram.:f.}
\end{itemize}
\begin{itemize}
\item {Proveniência:(De \textunderscore héctico\textunderscore )}
\end{itemize}
Consumpção progressiva do organismo; tísica.
\section{Héctico}
\begin{itemize}
\item {Grp. gram.:adj.}
\end{itemize}
\begin{itemize}
\item {Grp. gram.:M.}
\end{itemize}
\begin{itemize}
\item {Proveniência:(Gr. \textunderscore hektikos\textunderscore )}
\end{itemize}
Que tem héctica.
Relativo a héctica.
Aquelle que soffre héctica.
\section{Héctigo}
\begin{itemize}
\item {Grp. gram.:m.  e  adj.}
\end{itemize}
\begin{itemize}
\item {Utilização:Prov.}
\end{itemize}
\begin{itemize}
\item {Utilização:alent.}
\end{itemize}
\begin{itemize}
\item {Utilização:Ant.}
\end{itemize}
O mesmo que \textunderscore héctico\textunderscore .
\section{Hectiguidade}
\begin{itemize}
\item {Grp. gram.:f.}
\end{itemize}
\begin{itemize}
\item {Utilização:Ant.}
\end{itemize}
Estado de quem é héctigo; tísica.
\section{Hecto...}
\begin{itemize}
\item {Grp. gram.:pref.}
\end{itemize}
(designativo de \textunderscore cem\textunderscore )
(Contr. incorrecta do gr. \textunderscore hekaton\textunderscore , cem)
\section{Hectoedria}
\begin{itemize}
\item {fónica:to-e}
\end{itemize}
\begin{itemize}
\item {Grp. gram.:f.}
\end{itemize}
Qualidade de hectoédrico.
\section{Hectoédrico}
\begin{itemize}
\item {Grp. gram.:adj.}
\end{itemize}
\begin{itemize}
\item {Proveniência:(Do gr. \textunderscore hex\textunderscore  + \textunderscore edra\textunderscore )}
\end{itemize}
Diz-se, em Mineralogia, dos crystaes que têm seis faces.
\section{Hectograma}
\begin{itemize}
\item {Grp. gram.:m.}
\end{itemize}
\begin{itemize}
\item {Proveniência:(De \textunderscore hecto...\textunderscore  + \textunderscore grama\textunderscore )}
\end{itemize}
Pêso de cem gramas.
\section{Hectogramma}
\begin{itemize}
\item {Grp. gram.:m.}
\end{itemize}
\begin{itemize}
\item {Proveniência:(De \textunderscore hecto...\textunderscore  + \textunderscore gramma\textunderscore )}
\end{itemize}
Pêso de cem grammas.
\section{Hectolitro}
\begin{itemize}
\item {Grp. gram.:m.}
\end{itemize}
\begin{itemize}
\item {Proveniência:(De \textunderscore hecto...\textunderscore  + \textunderscore litro\textunderscore )}
\end{itemize}
Medida de cem litros.
\section{Hectométrico}
\begin{itemize}
\item {Grp. gram.:adj.}
\end{itemize}
Relativo ao hectómetro.
\section{Hectómetro}
\begin{itemize}
\item {Grp. gram.:m.}
\end{itemize}
\begin{itemize}
\item {Proveniência:(De \textunderscore hecto...\textunderscore  + \textunderscore metro\textunderscore )}
\end{itemize}
Medida de cem metros.
\section{Hectórea}
\begin{itemize}
\item {Grp. gram.:f.}
\end{itemize}
\begin{itemize}
\item {Proveniência:(De \textunderscore Hector\textunderscore , n. p.)}
\end{itemize}
Gênero de plantas labiadas.
\section{Hectostere}
\begin{itemize}
\item {Grp. gram.:m.}
\end{itemize}
\begin{itemize}
\item {Proveniência:(De \textunderscore hecto...\textunderscore  + \textunderscore estere\textunderscore )}
\end{itemize}
Medida de cem esteres.
\section{Hecuste}
\begin{itemize}
\item {Grp. gram.:m.}
\end{itemize}
\begin{itemize}
\item {Utilização:Ant.}
\end{itemize}
Planta aromática, o mesmo que \textunderscore pucho\textunderscore .
\section{Hédera}
\begin{itemize}
\item {Grp. gram.:f.}
\end{itemize}
\begin{itemize}
\item {Utilização:Prov.}
\end{itemize}
\begin{itemize}
\item {Utilização:trasm.}
\end{itemize}
\begin{itemize}
\item {Proveniência:(Lat. \textunderscore hedera\textunderscore )}
\end{itemize}
O mesmo que \textunderscore hera\textunderscore .
\section{Hederáceas}
\begin{itemize}
\item {Grp. gram.:f. pl.}
\end{itemize}
\begin{itemize}
\item {Proveniência:(De \textunderscore hederáceo\textunderscore )}
\end{itemize}
Família de plantas, que têm por typo a hera.
\section{Hederáceo}
\begin{itemize}
\item {Grp. gram.:adj.}
\end{itemize}
\begin{itemize}
\item {Proveniência:(Lat. \textunderscore hederaceus\textunderscore )}
\end{itemize}
Relativo ou semelhante á hera.
\section{Hederiforme}
\begin{itemize}
\item {Grp. gram.:adj.}
\end{itemize}
\begin{itemize}
\item {Proveniência:(Do lat. \textunderscore hedera\textunderscore  + \textunderscore forma\textunderscore )}
\end{itemize}
Que tem fórma de hera.
\section{Hederígero}
\begin{itemize}
\item {Grp. gram.:adj.}
\end{itemize}
\begin{itemize}
\item {Proveniência:(Lat. \textunderscore hederiger\textunderscore )}
\end{itemize}
Que tem heras.
Ornado de heras.
\section{Hederina}
\begin{itemize}
\item {Grp. gram.:f.}
\end{itemize}
\begin{itemize}
\item {Proveniência:(Do lat. \textunderscore hedera\textunderscore )}
\end{itemize}
Suco, que os troncos das heras velhas destillam.
\section{Hederoso}
\begin{itemize}
\item {Grp. gram.:adj.}
\end{itemize}
\begin{itemize}
\item {Proveniência:(Lat. \textunderscore hederosus\textunderscore )}
\end{itemize}
Abundante de heras.
\section{Hediondamente}
\begin{itemize}
\item {Grp. gram.:adv.}
\end{itemize}
De modo hediondo.
\section{Hediondez}
\begin{itemize}
\item {Grp. gram.:f.}
\end{itemize}
\begin{itemize}
\item {Utilização:Fig.}
\end{itemize}
Qualidade daquillo que é hediondo.
Procedimento hediondo.
\section{Hediondeza}
\begin{itemize}
\item {Grp. gram.:f.}
\end{itemize}
(V.hediondez)
\section{Hediondo}
\begin{itemize}
\item {Grp. gram.:adj.}
\end{itemize}
\begin{itemize}
\item {Proveniência:(Lat. \textunderscore foetibundus\textunderscore )}
\end{itemize}
Depravado.
Vicioso.
Sórdido; repugnante; nojento.
\section{Hediótida}
\begin{itemize}
\item {Grp. gram.:f.}
\end{itemize}
\begin{itemize}
\item {Proveniência:(Do gr. \textunderscore hedus\textunderscore  + \textunderscore ous\textunderscore , \textunderscore otos\textunderscore )}
\end{itemize}
Gênero de plantas rubiáceas.
\section{Hedónico}
\begin{itemize}
\item {Grp. gram.:adj.}
\end{itemize}
Relativo ao hedonismo.
\section{Hedonismo}
\begin{itemize}
\item {Grp. gram.:m.}
\end{itemize}
\begin{itemize}
\item {Proveniência:(Do gr. \textunderscore hedon\textunderscore , prazer)}
\end{itemize}
Antigo systema philosóphico, que cifrava no prazer toda a ventura.
\section{Hedra}
\begin{itemize}
\item {Grp. gram.:f.}
\end{itemize}
\begin{itemize}
\item {Utilização:Prov.}
\end{itemize}
\begin{itemize}
\item {Utilização:trasm.}
\end{itemize}
\begin{itemize}
\item {Proveniência:(Lat. \textunderscore hedera\textunderscore )}
\end{itemize}
O mesmo que \textunderscore hera\textunderscore .
\section{Hedu}
\begin{itemize}
\item {Grp. gram.:m.}
\end{itemize}
Grande árvore intertropical, de madeira branca, (\textunderscore nauclea cordifolia\textunderscore ).
\section{Hedwígia}
\begin{itemize}
\item {Grp. gram.:f.}
\end{itemize}
\begin{itemize}
\item {Proveniência:(De \textunderscore Hedwig\textunderscore , n. p.)}
\end{itemize}
Arvore resinosa, terebinthácea.
\section{Hedyótida}
\begin{itemize}
\item {Grp. gram.:f.}
\end{itemize}
\begin{itemize}
\item {Proveniência:(Do gr. \textunderscore hedus\textunderscore  + \textunderscore ous\textunderscore , \textunderscore otos\textunderscore )}
\end{itemize}
Gênero de plantas rubiáceas.
\section{Hegelianismo}
\begin{itemize}
\item {Grp. gram.:m.}
\end{itemize}
\begin{itemize}
\item {Proveniência:(De \textunderscore hegeliano\textunderscore )}
\end{itemize}
Doutrinas philosóphicas de Hegel.
\section{Hegeliano}
\begin{itemize}
\item {Grp. gram.:adj.}
\end{itemize}
\begin{itemize}
\item {Proveniência:(De \textunderscore Hegel\textunderscore , n. p.)}
\end{itemize}
Relativo a Hegel ou á doutrina dêste philósopho. Cf. Herculano, \textunderscore Quest. Públ.\textunderscore , II, 115.
\section{Hegelismo}
\begin{itemize}
\item {Grp. gram.:m.}
\end{itemize}
(V.hegelianismo)
\section{Hegemonia}
\begin{itemize}
\item {Grp. gram.:f.}
\end{itemize}
\begin{itemize}
\item {Proveniência:(Do gr. \textunderscore hegemonia\textunderscore )}
\end{itemize}
Preponderância de uma cidade ou povo entre outros povos ou cidades.
\section{Hegemónico}
\begin{itemize}
\item {Grp. gram.:adj.}
\end{itemize}
Relativo a hegemonia.
\section{Hégira}
\begin{itemize}
\item {Grp. gram.:f.}
\end{itemize}
\begin{itemize}
\item {Proveniência:(Do ár. \textunderscore hijra\textunderscore , fugida)}
\end{itemize}
Era muçulmana, que tem por ponto de partida a fuga de Mahomet em 622 da nossa era.
\section{Hegúmeno}
\begin{itemize}
\item {Grp. gram.:m.}
\end{itemize}
\begin{itemize}
\item {Proveniência:(Gr. \textunderscore hegumenos\textunderscore )}
\end{itemize}
Espécie de abbade ou chefe, nos mosteiros gregos.
\section{Heim!}
\begin{itemize}
\item {Grp. gram.:interj.}
\end{itemize}
(designativa de pergunta ou de admiração)(V.hem!)
\section{Hein!}
\begin{itemize}
\item {Grp. gram.:interj.}
\end{itemize}
(designativa de pergunta ou de admiração)(V.hem!)
\section{Heínsia}
\begin{itemize}
\item {Grp. gram.:f.}
\end{itemize}
Gênero de plantas rubiáceas.
\section{Heire}
\begin{itemize}
\item {Grp. gram.:adv.}
\end{itemize}
\begin{itemize}
\item {Utilização:Ant.}
\end{itemize}
\begin{itemize}
\item {Proveniência:(Do lat. \textunderscore heri\textunderscore )}
\end{itemize}
O mesmo que \textunderscore ontem\textunderscore .
\section{Helche}
\begin{itemize}
\item {Grp. gram.:m.}
\end{itemize}
\begin{itemize}
\item {Utilização:Ant.}
\end{itemize}
O mesmo que \textunderscore elche\textunderscore :«\textunderscore ...irmehey fazer helche\textunderscore ». \textunderscore Aulegrafia\textunderscore , 107.
\section{Helciário}
\begin{itemize}
\item {Grp. gram.:m.}
\end{itemize}
\begin{itemize}
\item {Utilização:Ant.}
\end{itemize}
\begin{itemize}
\item {Proveniência:(Do lat. \textunderscore helcium\textunderscore )}
\end{itemize}
Aquelle que puxava a embarcação á sirga.
\section{Helcologia}
\begin{itemize}
\item {Grp. gram.:f.}
\end{itemize}
\begin{itemize}
\item {Proveniência:(Do gr. \textunderscore helkos\textunderscore  + \textunderscore logos\textunderscore )}
\end{itemize}
Tratado á cêrca de úlceras.
\section{Helcose}
\begin{itemize}
\item {Grp. gram.:f.}
\end{itemize}
\begin{itemize}
\item {Proveniência:(Do gr. \textunderscore helkos\textunderscore )}
\end{itemize}
O mesmo que \textunderscore ulceração\textunderscore .
\section{Helênia}
\begin{itemize}
\item {Grp. gram.:f.}
\end{itemize}
O mesmo que \textunderscore helenina\textunderscore .
\section{Helenina}
\begin{itemize}
\item {Grp. gram.:f.}
\end{itemize}
Nome, que se deu á inulina.
(Corr. de \textunderscore inulina\textunderscore )
\section{Helênio}
\begin{itemize}
\item {Grp. gram.:m.}
\end{itemize}
\begin{itemize}
\item {Proveniência:(De \textunderscore Helena\textunderscore , n. p.)}
\end{itemize}
Gênero de plantas americanas, da fam. das compostas.
\section{Helépole}
\begin{itemize}
\item {Grp. gram.:f.}
\end{itemize}
\begin{itemize}
\item {Proveniência:(Lat. \textunderscore helepolis\textunderscore )}
\end{itemize}
Antiga máquina de guerra, espécie de catapulta.
\section{Hélia}
\begin{itemize}
\item {Grp. gram.:f.}
\end{itemize}
Gênero de plantas gencianáceas.
\section{Helíacas}
\begin{itemize}
\item {Grp. gram.:f. pl.}
\end{itemize}
\begin{itemize}
\item {Proveniência:(De \textunderscore helíaco\textunderscore )}
\end{itemize}
Antigas festas dos Gregos, em honra do Sol.
\section{Helíaco}
\begin{itemize}
\item {Grp. gram.:adj.}
\end{itemize}
\begin{itemize}
\item {Proveniência:(Lat. \textunderscore heliacus\textunderscore )}
\end{itemize}
Diz-se do nascimento ou occaso de um astro, quando incide com o nascimento ou occaso do Sol.
\section{Heliânteas}
\begin{itemize}
\item {Grp. gram.:f. pl.}
\end{itemize}
\begin{itemize}
\item {Proveniência:(De \textunderscore heliânteo\textunderscore )}
\end{itemize}
Tríbo de plantas compostas, que têm por tipo o helianto.
\section{Heliântemo}
\begin{itemize}
\item {Grp. gram.:m.}
\end{itemize}
\begin{itemize}
\item {Proveniência:(De \textunderscore helianto\textunderscore )}
\end{itemize}
Planta cistínea, espécie de esteva.
\section{Heliânteo}
\begin{itemize}
\item {Grp. gram.:adj.}
\end{itemize}
Relativo ou semelhante ao helianto.
\section{Heliântheas}
\begin{itemize}
\item {Grp. gram.:f. pl.}
\end{itemize}
\begin{itemize}
\item {Proveniência:(De \textunderscore heliântheo\textunderscore )}
\end{itemize}
Tríbo de plantas compostas, que têm por typo o heliantho.
\section{Heliânthemo}
\begin{itemize}
\item {Grp. gram.:m.}
\end{itemize}
\begin{itemize}
\item {Proveniência:(De \textunderscore heliantho\textunderscore )}
\end{itemize}
Planta cystínea, espécie de esteva.
\section{Heliântheo}
\begin{itemize}
\item {Grp. gram.:adj.}
\end{itemize}
Relativo ou semelhante ao heliantho.
\section{Heliantho}
\begin{itemize}
\item {Grp. gram.:m.}
\end{itemize}
\begin{itemize}
\item {Proveniência:(Gr. \textunderscore helianthes\textunderscore )}
\end{itemize}
Nome scientífico do girasol.
\section{Helianto}
\begin{itemize}
\item {Grp. gram.:m.}
\end{itemize}
\begin{itemize}
\item {Proveniência:(Gr. \textunderscore helianthes\textunderscore )}
\end{itemize}
Nome científico do girasol.
\section{Hélice}
\begin{itemize}
\item {Grp. gram.:m.  ou  f.}
\end{itemize}
\begin{itemize}
\item {Proveniência:(Lat. \textunderscore helix\textunderscore )}
\end{itemize}
Linha, traçada em fórma de rosca á volta de um cylindro.
Propulsor, que, collocado na parte póstero-inferior do navio, substitue o antigo systema de rodas.
Pequenas volutas no capitel corýnthio.
Gênero de molluscos gasterópodes.
Qualquer objecto em fórma de caracol; espiral.
\section{Heliceiro}
\begin{itemize}
\item {Grp. gram.:m.}
\end{itemize}
\begin{itemize}
\item {Proveniência:(De \textunderscore hélice\textunderscore )}
\end{itemize}
Mollusco, que adhere ás hélices dos navios.
\section{Helícia}
\begin{itemize}
\item {Grp. gram.:f.}
\end{itemize}
Gênero de plantas proteáceas.
\section{Helicídios}
\begin{itemize}
\item {Grp. gram.:m. pl.}
\end{itemize}
\begin{itemize}
\item {Proveniência:(Do gr. \textunderscore helix\textunderscore  + \textunderscore eidos\textunderscore )}
\end{itemize}
Família de molluscos, que têm por typo o caracol commum.
\section{Helicina}
\begin{itemize}
\item {Grp. gram.:f.}
\end{itemize}
\begin{itemize}
\item {Proveniência:(Do gr. \textunderscore helix\textunderscore )}
\end{itemize}
Mucilagem, procedente dos caracoes.
\section{Helicita}
\begin{itemize}
\item {Grp. gram.:f.}
\end{itemize}
\begin{itemize}
\item {Proveniência:(Do gr. \textunderscore helix\textunderscore )}
\end{itemize}
Concha fóssil, turbinada em rosca.
\section{Helicógrafo}
\begin{itemize}
\item {Grp. gram.:m.}
\end{itemize}
\begin{itemize}
\item {Proveniência:(Do gr. \textunderscore helix\textunderscore  + \textunderscore graphein\textunderscore )}
\end{itemize}
Instrumento, para traçar hélices.
\section{Helicógrapho}
\begin{itemize}
\item {Grp. gram.:m.}
\end{itemize}
\begin{itemize}
\item {Proveniência:(Do gr. \textunderscore helix\textunderscore  + \textunderscore graphein\textunderscore )}
\end{itemize}
Instrumento, para traçar hélices.
\section{Helicoidal}
\begin{itemize}
\item {Grp. gram.:adj.}
\end{itemize}
O mesmo que \textunderscore helicoide\textunderscore .
\section{Helicoide}
\begin{itemize}
\item {Grp. gram.:adj.}
\end{itemize}
\begin{itemize}
\item {Grp. gram.:M.  ou  f.}
\end{itemize}
\begin{itemize}
\item {Utilização:Mathem.}
\end{itemize}
\begin{itemize}
\item {Proveniência:(Do gr. \textunderscore helix\textunderscore  + \textunderscore eidos\textunderscore )}
\end{itemize}
Semelhante á hélice.
Superfície, gerada por uma recta horizontal, apoiada constantemente sôbre uma hélice e sôbre o eixo vertical do cylindro recto, em que está traçada essa curva.
\section{Helicómetro}
\begin{itemize}
\item {Grp. gram.:m.}
\end{itemize}
\begin{itemize}
\item {Proveniência:(Do gr. \textunderscore helix\textunderscore  + \textunderscore metron\textunderscore )}
\end{itemize}
Apparelho, para medir a fôrça dos hélices, nos barcos de vapor.
\section{Helicónia}
\begin{itemize}
\item {Grp. gram.:f.}
\end{itemize}
\begin{itemize}
\item {Proveniência:(De \textunderscore Helicon\textunderscore , n. p.)}
\end{itemize}
Gênero de insectos lepidópteros diurnos.
Gênero de plantas musáceas.
\section{Heliconiano}
\begin{itemize}
\item {Grp. gram.:adj.}
\end{itemize}
Relativo ao monte Hélicon.
\section{Helicóptero}
\begin{itemize}
\item {Grp. gram.:m.}
\end{itemize}
\begin{itemize}
\item {Proveniência:(Do gr. \textunderscore helix\textunderscore  + \textunderscore pteron\textunderscore )}
\end{itemize}
Apparelho aerostático, que se eleva na atmosphera por meio de hélices, os quaes se movem em tôrno de um eixo vertical.
\section{Helicóstegos}
\begin{itemize}
\item {Grp. gram.:m. pl.}
\end{itemize}
\begin{itemize}
\item {Proveniência:(Do gr. \textunderscore helix\textunderscore  + \textunderscore stegein\textunderscore )}
\end{itemize}
Família de molluscos, cuja concha se compõe de lóculos reunidos sôbre um ou dois eixos distintos, mas formando uma espiral regular.
\section{Helicotrema}
\begin{itemize}
\item {Grp. gram.:m.}
\end{itemize}
\begin{itemize}
\item {Utilização:Anat.}
\end{itemize}
\begin{itemize}
\item {Proveniência:(Do gr. \textunderscore helix\textunderscore  + \textunderscore trema\textunderscore )}
\end{itemize}
Pequena abertura no cimo do caracol do ouvido interno.
\section{Helícula}
\begin{itemize}
\item {Grp. gram.:f.}
\end{itemize}
Pequena hélice.
Haste espiral de algumas plantas.
(Dem. de \textunderscore hélice\textunderscore )
\section{Hélio}
\begin{itemize}
\item {Grp. gram.:m.}
\end{itemize}
\begin{itemize}
\item {Proveniência:(Do gr. \textunderscore helios\textunderscore )}
\end{itemize}
Metalloide gasoso, descoberto em 1895.
\section{Helio...}
\begin{itemize}
\item {Grp. gram.:pref.}
\end{itemize}
\begin{itemize}
\item {Proveniência:(Do gr. \textunderscore helios\textunderscore )}
\end{itemize}
(designativo de \textunderscore Sol\textunderscore )
\section{Heliocêntrico}
\begin{itemize}
\item {Grp. gram.:adj.}
\end{itemize}
\begin{itemize}
\item {Proveniência:(De \textunderscore helio...\textunderscore  + \textunderscore centro\textunderscore )}
\end{itemize}
Relativo a Sol, como centro, (falando-se da latitude e longitude dos planetas).
\section{Heliochromia}
\begin{itemize}
\item {Grp. gram.:f.}
\end{itemize}
\begin{itemize}
\item {Proveniência:(Do gr. \textunderscore helios\textunderscore  + \textunderscore khroma\textunderscore )}
\end{itemize}
Reproducção das côres, com o auxílio do sol, sôbre uma camada de chloreto de prata, sustentada por uma placa metállica.
Reproducção das côres pela photographia.
\section{Heliochrómico}
\begin{itemize}
\item {Grp. gram.:adj.}
\end{itemize}
Relativo á heliochromia.
\section{Heliochryso}
\begin{itemize}
\item {Grp. gram.:m.}
\end{itemize}
\begin{itemize}
\item {Proveniência:(Lat. \textunderscore heliochrysum\textunderscore )}
\end{itemize}
Planta medicinal.
\section{Heliocometa}
\begin{itemize}
\item {fónica:mê}
\end{itemize}
\begin{itemize}
\item {Grp. gram.:f.}
\end{itemize}
\begin{itemize}
\item {Proveniência:(De \textunderscore helio...\textunderscore  + \textunderscore cometa\textunderscore )}
\end{itemize}
Phenómeno, apresentado ás vezes pelo Sol poente e que consiste numa faixa luminosa, semelhante á cauda de um cometa.
\section{Heliocriso}
\begin{itemize}
\item {Grp. gram.:m.}
\end{itemize}
\begin{itemize}
\item {Proveniência:(Lat. \textunderscore heliochrysum\textunderscore )}
\end{itemize}
Planta medicinal.
\section{Heliocromia}
\begin{itemize}
\item {Grp. gram.:f.}
\end{itemize}
\begin{itemize}
\item {Proveniência:(Do gr. \textunderscore helios\textunderscore  + \textunderscore khroma\textunderscore )}
\end{itemize}
Reprodução das côres, com o auxílio do sol, sôbre uma camada de cloreto de prata, sustentada por uma placa metálica.
Reprodução das côres pela fotografia.
\section{Heliocrómico}
\begin{itemize}
\item {Grp. gram.:adj.}
\end{itemize}
Relativo á heliocromia.
\section{Heliodinâmico}
\begin{itemize}
\item {Grp. gram.:m.}
\end{itemize}
\begin{itemize}
\item {Proveniência:(Do gr. \textunderscore helios\textunderscore  + \textunderscore dunamis\textunderscore )}
\end{itemize}
Aparelho, para fazer evaporar a água, utilizando-se o calor do sol.
\section{Heliodynâmico}
\begin{itemize}
\item {Grp. gram.:m.}
\end{itemize}
\begin{itemize}
\item {Proveniência:(Do gr. \textunderscore helios\textunderscore  + \textunderscore dunamis\textunderscore )}
\end{itemize}
Apparelho, para fazer evaporar a água, utilizando-se o calor do sol.
\section{Heliófila}
\begin{itemize}
\item {Grp. gram.:f.}
\end{itemize}
\begin{itemize}
\item {Proveniência:(Do gr. \textunderscore helios\textunderscore  + \textunderscore philos\textunderscore )}
\end{itemize}
Planta crucífera.
\section{Heliofíleas}
\begin{itemize}
\item {Grp. gram.:f. pl.}
\end{itemize}
\begin{itemize}
\item {Proveniência:(Do gr. \textunderscore helios\textunderscore  + \textunderscore phullon\textunderscore )}
\end{itemize}
Tríbo de plantas crucíferas.
\section{Heliófugo}
\begin{itemize}
\item {Grp. gram.:adj.}
\end{itemize}
\begin{itemize}
\item {Proveniência:(Do gr. \textunderscore helios\textunderscore  + lat. \textunderscore fugere\textunderscore )}
\end{itemize}
\begin{itemize}
\item {Grp. gram.:adj.}
\end{itemize}
\begin{itemize}
\item {Utilização:Bot.}
\end{itemize}
\begin{itemize}
\item {Proveniência:(Do gr. \textunderscore helios\textunderscore  + \textunderscore phuge\textunderscore )}
\end{itemize}
Que foge do sol.
Que evita o sol; que se desvia da acção do sol.
\section{Heliografia}
\begin{itemize}
\item {Grp. gram.:f.}
\end{itemize}
\begin{itemize}
\item {Proveniência:(De \textunderscore heliógrafo\textunderscore )}
\end{itemize}
Descripção do Sol.
Espécie de fotografia, em que se reproduzem desenhos, por meio da acção dos raios solares.
\section{Heliográfico}
\begin{itemize}
\item {Grp. gram.:adj.}
\end{itemize}
Relativo á heliografia.
\section{Heliógrafo}
\begin{itemize}
\item {Grp. gram.:m.}
\end{itemize}
\begin{itemize}
\item {Proveniência:(Do gr. \textunderscore helios\textunderscore  + \textunderscore graphein\textunderscore )}
\end{itemize}
Instrumento, para tirar fotografias do Sol.
\section{Heliographia}
\begin{itemize}
\item {Grp. gram.:f.}
\end{itemize}
\begin{itemize}
\item {Proveniência:(De \textunderscore heliógrapho\textunderscore )}
\end{itemize}
Descripção do Sol.
Espécie de photographia, em que se reproduzem desenhos, por meio da acção dos raios solares.
\section{Heliográphico}
\begin{itemize}
\item {Grp. gram.:adj.}
\end{itemize}
Relativo á heliographia.
\section{Heliógrapho}
\begin{itemize}
\item {Grp. gram.:m.}
\end{itemize}
\begin{itemize}
\item {Proveniência:(Do gr. \textunderscore helios\textunderscore  + \textunderscore graphein\textunderscore )}
\end{itemize}
Instrumento, para tirar photographias do Sol.
\section{Heliogravura}
\begin{itemize}
\item {Grp. gram.:f.}
\end{itemize}
\begin{itemize}
\item {Proveniência:(De \textunderscore helio...\textunderscore  + \textunderscore gravura\textunderscore )}
\end{itemize}
Gravura heliográphica.
\section{Heliomagnetómetro}
\begin{itemize}
\item {Grp. gram.:m.}
\end{itemize}
\begin{itemize}
\item {Proveniência:(De \textunderscore helio...\textunderscore  + \textunderscore magnetómetro\textunderscore )}
\end{itemize}
Instrumento de Phýsica, para conhecer a declinação da agulha magnética e determinar pelo Sol a hora do dia.
\section{Heliómano}
\begin{itemize}
\item {Grp. gram.:m.}
\end{itemize}
\begin{itemize}
\item {Proveniência:(Do gr. \textunderscore helios\textunderscore  + \textunderscore mania\textunderscore )}
\end{itemize}
Gênero de insectos coleópteros tetrâmeros, da fam. dos longicórneos.
\section{Heliométrico}
\begin{itemize}
\item {Grp. gram.:adj.}
\end{itemize}
Relativo ao heliómetro.
\section{Heliómetro}
\begin{itemize}
\item {Grp. gram.:m.}
\end{itemize}
\begin{itemize}
\item {Proveniência:(Do gr. \textunderscore helios\textunderscore  + \textunderscore metron\textunderscore )}
\end{itemize}
Apparelho, para medir o diâmetro apparente dos astros e a distância apparente dêstes entre si.
\section{Helióphila}
\begin{itemize}
\item {Grp. gram.:f.}
\end{itemize}
\begin{itemize}
\item {Proveniência:(Do gr. \textunderscore helios\textunderscore  + \textunderscore philos\textunderscore )}
\end{itemize}
Planta crucífera.
\section{Helióphugo}
\begin{itemize}
\item {Grp. gram.:adj.}
\end{itemize}
\begin{itemize}
\item {Utilização:Bot.}
\end{itemize}
\begin{itemize}
\item {Proveniência:(Do gr. \textunderscore helios\textunderscore  + \textunderscore phuge\textunderscore )}
\end{itemize}
Que evita o sol; que se desvia da acção do sol.
\section{Heliophýlleas}
\begin{itemize}
\item {Grp. gram.:f. pl.}
\end{itemize}
\begin{itemize}
\item {Proveniência:(Do gr. \textunderscore helios\textunderscore  + \textunderscore phullon\textunderscore )}
\end{itemize}
Tríbo de plantas crucíferas.
\section{Heliopolita}
\begin{itemize}
\item {Grp. gram.:adj.}
\end{itemize}
Relativo a Heliópolis.
Diz-se das dynastias egýpcias, que tiveram a sua séde em Heliópolis, n. gr. de uma cidade egýpcia.
\section{Heliopse}
\begin{itemize}
\item {Grp. gram.:f.}
\end{itemize}
\begin{itemize}
\item {Proveniência:(Do gr. \textunderscore helios\textunderscore  + \textunderscore opsis\textunderscore )}
\end{itemize}
Gênero de plantas heliântheas.
\section{Helebóreas}
\begin{itemize}
\item {Grp. gram.:f. pl.}
\end{itemize}
\begin{itemize}
\item {Proveniência:(De \textunderscore heléboro\textunderscore )}
\end{itemize}
Tríbo de plantas ranunculáceas.
\section{Heleborinha}
\begin{itemize}
\item {Grp. gram.:f.}
\end{itemize}
\begin{itemize}
\item {Proveniência:(De \textunderscore heléboro\textunderscore )}
\end{itemize}
Espécie de orquídea, (\textunderscore epidendrum elongatum\textunderscore ).
\section{Heleborismo}
\begin{itemize}
\item {Grp. gram.:m.}
\end{itemize}
Antigo sistema de tratar e prevenir doenças por meio do heléboro.
\section{Heleborizar}
\begin{itemize}
\item {Grp. gram.:v. t.}
\end{itemize}
\begin{itemize}
\item {Utilização:Des.}
\end{itemize}
Purgar com heléboro.
\section{Heléboro}
\begin{itemize}
\item {Grp. gram.:m.}
\end{itemize}
\begin{itemize}
\item {Proveniência:(Lat. \textunderscore helleborus\textunderscore )}
\end{itemize}
Planta medicinal, ranunculácea, (\textunderscore helleborus niger\textunderscore ).
Nome, que se dá a outras plantas, como a \textunderscore erva-dos-bèsteiros\textunderscore .
\section{Helênia}
\begin{itemize}
\item {Grp. gram.:f.}
\end{itemize}
Gênero de plantas amomáceas.
O mesmo que \textunderscore helênio\textunderscore ?
\section{Helenicamente}
\begin{itemize}
\item {Grp. gram.:adv.}
\end{itemize}
\begin{itemize}
\item {Proveniência:(De \textunderscore helênico\textunderscore )}
\end{itemize}
Á maneira dos Gregos.
\section{Helênico}
\begin{itemize}
\item {Grp. gram.:adj.}
\end{itemize}
\begin{itemize}
\item {Grp. gram.:M.}
\end{itemize}
\begin{itemize}
\item {Proveniência:(Gr. \textunderscore hellenikos\textunderscore )}
\end{itemize}
Relativo á Hélada ou á Grécia antiga.
O grego antigo.
\section{Helenismo}
\begin{itemize}
\item {Grp. gram.:m.}
\end{itemize}
\begin{itemize}
\item {Proveniência:(Gr. \textunderscore hellenismos\textunderscore )}
\end{itemize}
Locução própria da língua grega.
Conjunto das ideias e costumes da Grécia.
\section{Helenista}
\begin{itemize}
\item {Grp. gram.:m.}
\end{itemize}
\begin{itemize}
\item {Proveniência:(Gr. \textunderscore hellenistes\textunderscore )}
\end{itemize}
Aquele que é versado na língua e antiguidades gregas.
\section{Helenização}
\begin{itemize}
\item {Grp. gram.:f.}
\end{itemize}
Acto ou efeito de helenizar.
\section{Helenizar}
\begin{itemize}
\item {Grp. gram.:v. t.}
\end{itemize}
\begin{itemize}
\item {Grp. gram.:V. i.}
\end{itemize}
\begin{itemize}
\item {Proveniência:(Gr. \textunderscore hellenizein\textunderscore )}
\end{itemize}
Tornar conforme ao carácter grego.
Dedicar-se ao estudo do grego.
\section{Helenos}
\begin{itemize}
\item {Grp. gram.:m. pl.}
\end{itemize}
\begin{itemize}
\item {Proveniência:(Gr. \textunderscore hellenes\textunderscore )}
\end{itemize}
Povos que, substituindo a dominação dos Pelasgos, povoaram a Grécia.
Gregos.
\section{Helespontíaco}
\begin{itemize}
\item {Grp. gram.:adj.}
\end{itemize}
O mesmo que \textunderscore helespôntico\textunderscore .
\section{Helespôntico}
\begin{itemize}
\item {Grp. gram.:adj.}
\end{itemize}
Relativo ao Helesponto.
\section{Helioscopia}
\begin{itemize}
\item {Grp. gram.:f.}
\end{itemize}
Observação do Sol pelo helioscópio.
\section{Helioscópico}
\begin{itemize}
\item {Grp. gram.:adj.}
\end{itemize}
Relativo á helioscopia.
\section{Helioscópio}
\begin{itemize}
\item {Grp. gram.:m.}
\end{itemize}
\begin{itemize}
\item {Proveniência:(Do gr. \textunderscore helios\textunderscore  + \textunderscore skopein\textunderscore )}
\end{itemize}
Instrumento, armado de um vidro de côr, para se observar o Sol.
Instrumento, com que se dirige a imagem do Sol para uma câmara escura.
\section{Heliose}
\begin{itemize}
\item {Grp. gram.:f.}
\end{itemize}
\begin{itemize}
\item {Proveniência:(Gr. \textunderscore heliosis\textunderscore )}
\end{itemize}
Doença, produzida pela acção do sol; insolação.
\section{Heliostática}
\begin{itemize}
\item {Grp. gram.:f.}
\end{itemize}
\begin{itemize}
\item {Proveniência:(De \textunderscore heliostático\textunderscore )}
\end{itemize}
Doutrina sôbre o movimento dos planetas, referido á posição do Sol no centro do systema planetário.
\section{Heliostático}
\begin{itemize}
\item {Grp. gram.:adj.}
\end{itemize}
Relativo ao helióstato.
\section{Helióstato}
\begin{itemize}
\item {Grp. gram.:m.}
\end{itemize}
\begin{itemize}
\item {Proveniência:(Do gr. \textunderscore helios\textunderscore  + \textunderscore statos\textunderscore )}
\end{itemize}
Apparelho, que conserva numa direcção constante, apesar do movimento do Sol, um raio solar introduzido numa câmara escura.
\section{Helioterapia}
\begin{itemize}
\item {Grp. gram.:f.}
\end{itemize}
\begin{itemize}
\item {Proveniência:(De \textunderscore helio...\textunderscore  + \textunderscore therapia\textunderscore )}
\end{itemize}
Aplicação terapêutica do sol.
\section{Heliotermómetro}
\begin{itemize}
\item {Grp. gram.:m.}
\end{itemize}
\begin{itemize}
\item {Proveniência:(De \textunderscore hélio...\textunderscore  + \textunderscore thermómetro\textunderscore )}
\end{itemize}
Aparelho, para medir a intensidade do calor solar.
\section{Heliotherapia}
\begin{itemize}
\item {Grp. gram.:f.}
\end{itemize}
\begin{itemize}
\item {Proveniência:(De \textunderscore helio...\textunderscore  + \textunderscore therapia\textunderscore )}
\end{itemize}
Applicação terapêutica do sol.
\section{Heliothermómetro}
\begin{itemize}
\item {Grp. gram.:m.}
\end{itemize}
\begin{itemize}
\item {Proveniência:(De \textunderscore hélio...\textunderscore  + \textunderscore thermómetro\textunderscore )}
\end{itemize}
Apparelho, para medir a intensidade do calor solar.
\section{Heliotropia}
\begin{itemize}
\item {Grp. gram.:f.}
\end{itemize}
\begin{itemize}
\item {Proveniência:(Do rad. de \textunderscore heliotrópio\textunderscore )}
\end{itemize}
Qualidade de heliotrópico, (falando-se das plantas).
\section{Heliotrópia}
\begin{itemize}
\item {Grp. gram.:f.}
\end{itemize}
Planta, o mesmo que \textunderscore heliotrópio\textunderscore .
\section{Heliotrópico}
\begin{itemize}
\item {Grp. gram.:adj.}
\end{itemize}
\begin{itemize}
\item {Proveniência:(De \textunderscore heliotropia\textunderscore )}
\end{itemize}
Diz-se das plantas, cujas flôres, fôlhas ou hastes, se voltam para o Sol, quando êste se acha acima do horizonte.
\section{Heliotrópio}
\begin{itemize}
\item {Grp. gram.:m.}
\end{itemize}
\begin{itemize}
\item {Proveniência:(Do gr. \textunderscore helios\textunderscore  + \textunderscore trope\textunderscore )}
\end{itemize}
Gênero de plantas borragíneas; girasol.
Nome de várias plantas, que se voltam para o Sol, em-quanto êste se acha sôbre o horizonte.
Apparelho, para concentrar num ponto distante os raios solares.
Pedra preciosa, esverdeada e com estrias vermelhas.
\section{Heliotropismo}
\begin{itemize}
\item {Grp. gram.:m.}
\end{itemize}
O mesmo que \textunderscore heliotropia\textunderscore .
\section{Hélix}
\begin{itemize}
\item {Grp. gram.:m.}
\end{itemize}
\begin{itemize}
\item {Utilização:Anat.}
\end{itemize}
\begin{itemize}
\item {Proveniência:(Do gr. \textunderscore helix\textunderscore )}
\end{itemize}
Rebordo exterior do pavilhão da orelha.
\section{Hellebóreas}
\begin{itemize}
\item {Grp. gram.:f. pl.}
\end{itemize}
\begin{itemize}
\item {Proveniência:(De \textunderscore helléboro\textunderscore )}
\end{itemize}
Tríbo de plantas ranunculáceas.
\section{Helleborinha}
\begin{itemize}
\item {Grp. gram.:f.}
\end{itemize}
\begin{itemize}
\item {Proveniência:(De \textunderscore helléboro\textunderscore )}
\end{itemize}
Espécie de orchídea, (\textunderscore epidendrum elongatum\textunderscore ).
\section{Helleborismo}
\begin{itemize}
\item {Grp. gram.:m.}
\end{itemize}
Antigo systema de tratar e prevenir doenças por meio do helléboro.
\section{Helleborizar}
\begin{itemize}
\item {Grp. gram.:v. t.}
\end{itemize}
\begin{itemize}
\item {Utilização:Des.}
\end{itemize}
Purgar com helléboro.
\section{Helléboro}
\begin{itemize}
\item {Grp. gram.:m.}
\end{itemize}
\begin{itemize}
\item {Proveniência:(Lat. \textunderscore helleborus\textunderscore )}
\end{itemize}
Planta medicinal, ranunculácea, (\textunderscore helleborus niger\textunderscore ).
Nome, que se dá a outras plantas, como a \textunderscore erva-dos-bèsteiros\textunderscore .
\section{Hellênia}
\begin{itemize}
\item {Grp. gram.:f.}
\end{itemize}
Gênero de plantas amomáceas.
O mesmo que \textunderscore helênio\textunderscore ?
\section{Hellenicamente}
\begin{itemize}
\item {Grp. gram.:adv.}
\end{itemize}
\begin{itemize}
\item {Proveniência:(De \textunderscore hellênico\textunderscore )}
\end{itemize}
Á maneira dos Gregos.
\section{Hellênico}
\begin{itemize}
\item {Grp. gram.:adj.}
\end{itemize}
\begin{itemize}
\item {Grp. gram.:M.}
\end{itemize}
\begin{itemize}
\item {Proveniência:(Gr. \textunderscore hellenikos\textunderscore )}
\end{itemize}
Relativo á Héllada ou á Grécia antiga.
O grego antigo.
\section{Hellenismo}
\begin{itemize}
\item {Grp. gram.:m.}
\end{itemize}
\begin{itemize}
\item {Proveniência:(Gr. \textunderscore hellenismos\textunderscore )}
\end{itemize}
Locução própria da língua grega.
Conjunto das ideias e costumes da Grécia.
\section{Hellenista}
\begin{itemize}
\item {Grp. gram.:m.}
\end{itemize}
\begin{itemize}
\item {Proveniência:(Gr. \textunderscore hellenistes\textunderscore )}
\end{itemize}
Aquelle que é versado na língua e antiguidades gregas.
\section{Hellenização}
\begin{itemize}
\item {Grp. gram.:f.}
\end{itemize}
Acto ou effeito de hellenizar.
\section{Hellenizar}
\begin{itemize}
\item {Grp. gram.:v. t.}
\end{itemize}
\begin{itemize}
\item {Grp. gram.:V. i.}
\end{itemize}
\begin{itemize}
\item {Proveniência:(Gr. \textunderscore hellenizein\textunderscore )}
\end{itemize}
Tornar conforme ao carácter grego.
Dedicar-se ao estudo do grego.
\section{Hellenos}
\begin{itemize}
\item {Grp. gram.:m. pl.}
\end{itemize}
\begin{itemize}
\item {Proveniência:(Gr. \textunderscore hellenes\textunderscore )}
\end{itemize}
Povos que, substituindo a dominação dos Pelasgos, povoaram a Grécia.
Gregos.
\section{Helléria}
\begin{itemize}
\item {Grp. gram.:f.}
\end{itemize}
\begin{itemize}
\item {Proveniência:(De \textunderscore Heller\textunderscore , n. p.)}
\end{itemize}
Gênero de arbustos brasileiros.
\section{Hellespontíaco}
\begin{itemize}
\item {Grp. gram.:adj.}
\end{itemize}
O mesmo que \textunderscore hellespôntico\textunderscore .
\section{Hellespôntico}
\begin{itemize}
\item {Grp. gram.:adj.}
\end{itemize}
Relativo ao Hellesponto.
\section{Helmintagogo}
\begin{itemize}
\item {Grp. gram.:adj.}
\end{itemize}
\begin{itemize}
\item {Proveniência:(Do gr. \textunderscore helmins\textunderscore , \textunderscore helminthos\textunderscore  + \textunderscore agogos\textunderscore )}
\end{itemize}
O mesmo que \textunderscore vermífugo\textunderscore .
\section{Helminthagogo}
\begin{itemize}
\item {Grp. gram.:adj.}
\end{itemize}
\begin{itemize}
\item {Proveniência:(Do gr. \textunderscore helmins\textunderscore , \textunderscore helminthos\textunderscore  + \textunderscore agogos\textunderscore )}
\end{itemize}
O mesmo que \textunderscore vermífugo\textunderscore .
\section{Helmínthia}
\begin{itemize}
\item {Grp. gram.:f.}
\end{itemize}
Gênero de plantas compostas, cujo fruto tem a apparência de um helmintho.
\section{Helminthíase}
\begin{itemize}
\item {Grp. gram.:f.}
\end{itemize}
\begin{itemize}
\item {Proveniência:(Lat. \textunderscore helminthiasis\textunderscore )}
\end{itemize}
Doença, produzida pela presença de entozoários.
\section{Helmínthico}
\begin{itemize}
\item {Grp. gram.:adj.}
\end{itemize}
Relativo aos helminthos.
\section{Helmintho}
\begin{itemize}
\item {Grp. gram.:m.}
\end{itemize}
\begin{itemize}
\item {Proveniência:(Gr. \textunderscore helmins\textunderscore , \textunderscore helminthos\textunderscore )}
\end{itemize}
Entozoário, ou verme intestinal.
\section{Helminthoide}
\begin{itemize}
\item {Grp. gram.:adj.}
\end{itemize}
\begin{itemize}
\item {Grp. gram.:M. pl.}
\end{itemize}
\begin{itemize}
\item {Proveniência:(Do gr. \textunderscore helmins\textunderscore  + \textunderscore eidos\textunderscore )}
\end{itemize}
Semelhante a helminthos.
Ordem de peixes, que respiram como os vermes.
\section{Helminthólitho}
\begin{itemize}
\item {Grp. gram.:m.}
\end{itemize}
\begin{itemize}
\item {Proveniência:(Do gr. \textunderscore helmins\textunderscore  + \textunderscore lithos\textunderscore )}
\end{itemize}
Verme fóssil.
\section{Helminthologia}
\begin{itemize}
\item {Grp. gram.:f.}
\end{itemize}
\begin{itemize}
\item {Proveniência:(Do gr. \textunderscore helmins\textunderscore  + \textunderscore logos\textunderscore )}
\end{itemize}
Tratado dos vermes intestinaes.
\section{Helminthológico}
\begin{itemize}
\item {Grp. gram.:adj.}
\end{itemize}
Relativo a helminthologia.
\section{Helminthologista}
\begin{itemize}
\item {Grp. gram.:m.}
\end{itemize}
Naturalista que trata de helminthologia.
\section{Helmíntia}
\begin{itemize}
\item {Grp. gram.:f.}
\end{itemize}
Gênero de plantas compostas, cujo fruto tem a aparência de um helminto.
\section{Helmintíase}
\begin{itemize}
\item {Grp. gram.:f.}
\end{itemize}
\begin{itemize}
\item {Proveniência:(Lat. \textunderscore helminthiasis\textunderscore )}
\end{itemize}
Doença, produzida pela presença de entozoários.
\section{Helmíntico}
\begin{itemize}
\item {Grp. gram.:adj.}
\end{itemize}
Relativo aos helmintos.
\section{Helminto}
\begin{itemize}
\item {Grp. gram.:m.}
\end{itemize}
\begin{itemize}
\item {Proveniência:(Gr. \textunderscore helmins\textunderscore , \textunderscore helminthos\textunderscore )}
\end{itemize}
Entozoário, ou verme intestinal.
\section{Helmintoide}
\begin{itemize}
\item {Grp. gram.:adj.}
\end{itemize}
\begin{itemize}
\item {Grp. gram.:M. pl.}
\end{itemize}
\begin{itemize}
\item {Proveniência:(Do gr. \textunderscore helmins\textunderscore  + \textunderscore eidos\textunderscore )}
\end{itemize}
Semelhante a helmintos.
Ordem de peixes, que respiram como os vermes.
\section{Helmintólito}
\begin{itemize}
\item {Grp. gram.:m.}
\end{itemize}
\begin{itemize}
\item {Proveniência:(Do gr. \textunderscore helmins\textunderscore  + \textunderscore lithos\textunderscore )}
\end{itemize}
Verme fóssil.
\section{Helmintologia}
\begin{itemize}
\item {Grp. gram.:f.}
\end{itemize}
\begin{itemize}
\item {Proveniência:(Do gr. \textunderscore helmins\textunderscore  + \textunderscore logos\textunderscore )}
\end{itemize}
Tratado dos vermes intestinaes.
\section{Helmintológico}
\begin{itemize}
\item {Grp. gram.:adj.}
\end{itemize}
Relativo a helmintologia.
\section{Helmintologista}
\begin{itemize}
\item {Grp. gram.:m.}
\end{itemize}
Naturalista que trata de helmintologia.
\section{Helmitol}
\begin{itemize}
\item {Grp. gram.:m.}
\end{itemize}
\begin{itemize}
\item {Utilização:Pharm.}
\end{itemize}
Medicamento desinfectante, applicado nas cistites.
\section{Helócero}
\begin{itemize}
\item {Grp. gram.:adj.}
\end{itemize}
\begin{itemize}
\item {Utilização:Zool.}
\end{itemize}
\begin{itemize}
\item {Grp. gram.:M. pl.}
\end{itemize}
\begin{itemize}
\item {Proveniência:(Do gr. \textunderscore helos\textunderscore  + \textunderscore keras\textunderscore )}
\end{itemize}
Que tem as antennas em fórma de prego.
Família de insectos coleópteros.
\section{Heloides}
\begin{itemize}
\item {Grp. gram.:m.}
\end{itemize}
Tumor, formado em cicatrizes antigas.
\section{Helope}
\begin{itemize}
\item {Grp. gram.:m.}
\end{itemize}
\begin{itemize}
\item {Proveniência:(Do gr. \textunderscore helos\textunderscore  + \textunderscore ops\textunderscore )}
\end{itemize}
Gênero de insectos.
\section{Helopianos}
\begin{itemize}
\item {Grp. gram.:m. pl.}
\end{itemize}
Família de insectos coleópteros, que têm por typo o helope.
\section{Helopitecos}
\begin{itemize}
\item {Grp. gram.:m. pl.}
\end{itemize}
\begin{itemize}
\item {Proveniência:(Do gr. \textunderscore helein\textunderscore , agarrar, e \textunderscore pithkos\textunderscore , macaco)}
\end{itemize}
Família de macacos, que podem apreender com a cauda.
\section{Helopithecos}
\begin{itemize}
\item {Grp. gram.:m. pl.}
\end{itemize}
\begin{itemize}
\item {Proveniência:(Do gr. \textunderscore helein\textunderscore , agarrar, e \textunderscore pithkos\textunderscore , macaco)}
\end{itemize}
Família de macacos, que podem apprehender com a cauda.
\section{Helosciádio}
\begin{itemize}
\item {Grp. gram.:m.}
\end{itemize}
\begin{itemize}
\item {Proveniência:(Do gr. \textunderscore helos\textunderscore , pantano, e \textunderscore skiadeion\textunderscore , umbella)}
\end{itemize}
Gênero de plantas umbellíferas.
\section{Helota}
\begin{itemize}
\item {Grp. gram.:m.}
\end{itemize}
O mesmo que \textunderscore ilota\textunderscore . Cf. Herculano, \textunderscore Quest. Públ.\textunderscore , I, 145; Latino, \textunderscore Hist. Pol.\textunderscore , I, 16.
\section{Helvécio}
\begin{itemize}
\item {Grp. gram.:adj.}
\end{itemize}
\begin{itemize}
\item {Grp. gram.:M. pl.}
\end{itemize}
\begin{itemize}
\item {Proveniência:(Lat. \textunderscore helvetius\textunderscore )}
\end{itemize}
Relativo á Helvécia.
Relativo á Suíça.
Povo gállio, que habitava na Helvécia; suíços.
\section{Helvético}
\begin{itemize}
\item {Grp. gram.:adj.}
\end{itemize}
\begin{itemize}
\item {Proveniência:(De \textunderscore Helvetos\textunderscore , n. p.)}
\end{itemize}
Relativo aos Helvetas, ou aos suíços.
\section{Helvetismo}
\begin{itemize}
\item {Grp. gram.:m.}
\end{itemize}
Locução privativa da Suíça francesa.
(Cp. \textunderscore helvético\textunderscore )
\section{Helvetos}
\begin{itemize}
\item {Grp. gram.:m. pl.}
\end{itemize}
O mesmo ou melhor que \textunderscore helvécios\textunderscore .
\section{Helvidiano}
\begin{itemize}
\item {Grp. gram.:m.}
\end{itemize}
\begin{itemize}
\item {Proveniência:(De \textunderscore Helvidio\textunderscore , n. p.)}
\end{itemize}
Membro de uma antiga seita christan, que sustentava que Maria tivera filhos de San-José.
\section{Hem!}
\begin{itemize}
\item {Proveniência:(Lat. \textunderscore hem\textunderscore )}
\end{itemize}
\textunderscore interj. interrogativa\textunderscore :«\textunderscore vais lá, hem?\textunderscore »Rebello, \textunderscore Mocidade\textunderscore , I, 103. Cf. Garrett, \textunderscore Filippa\textunderscore , 62, 76 e 147.
Cp. \textunderscore hein!\textunderscore 
\section{Hema...}
\begin{itemize}
\item {Grp. gram.:pref.}
\end{itemize}
\begin{itemize}
\item {Proveniência:(Do gr. \textunderscore haima\textunderscore )}
\end{itemize}
(designativo de \textunderscore sangue\textunderscore )
\section{Hemachroína}
\begin{itemize}
\item {Grp. gram.:f.}
\end{itemize}
O mesmo que \textunderscore hematosina\textunderscore .
\section{Hemacroína}
\begin{itemize}
\item {Grp. gram.:f.}
\end{itemize}
O mesmo que \textunderscore hematosina\textunderscore .
\section{Hemadinâmica}
\begin{itemize}
\item {Grp. gram.:f.}
\end{itemize}
\begin{itemize}
\item {Proveniência:(Do gr. \textunderscore haima\textunderscore  + \textunderscore dunamos\textunderscore )}
\end{itemize}
Teoria mecânica da circulação do sangue.--Seria preferível \textunderscore hemodinâmica\textunderscore .
\section{Hemadromómetro}
\begin{itemize}
\item {Grp. gram.:m.}
\end{itemize}
\begin{itemize}
\item {Proveniência:(Do gr. \textunderscore haima\textunderscore  + \textunderscore dromos\textunderscore  + \textunderscore metron\textunderscore )}
\end{itemize}
Instrumento, para avaliar a rapidez do sangue nos troncos arteriaes.
\section{Hemadynâmica}
\begin{itemize}
\item {Grp. gram.:f.}
\end{itemize}
\begin{itemize}
\item {Proveniência:(Do gr. \textunderscore haima\textunderscore  + \textunderscore dunamos\textunderscore )}
\end{itemize}
Theoria mecânica da circulação do sangue.--Seria preferível \textunderscore hemodynâmica\textunderscore .
\section{Hemafeína}
\begin{itemize}
\item {Grp. gram.:f.}
\end{itemize}
Supposto pigmento biliar, que se verificou modernamente sêr a \textunderscore bilifuscina\textunderscore .
\section{Hemagogo}
\begin{itemize}
\item {Grp. gram.:adj.}
\end{itemize}
\begin{itemize}
\item {Proveniência:(Do gr. \textunderscore haima\textunderscore  + \textunderscore agogos\textunderscore )}
\end{itemize}
O mesmo que \textunderscore emmenagogo\textunderscore .
Que provoca fluxo hemorrhoidal.
\section{Hemal}
\begin{itemize}
\item {Grp. gram.:adj.}
\end{itemize}
\begin{itemize}
\item {Utilização:Anat.}
\end{itemize}
\begin{itemize}
\item {Proveniência:(Do gr. \textunderscore haima\textunderscore )}
\end{itemize}
Diz-se do arco chondrocostal, que é elemento do metâmero.
\section{Hemalopia}
\begin{itemize}
\item {Grp. gram.:f.}
\end{itemize}
\begin{itemize}
\item {Proveniência:(Do gr. \textunderscore haimalops\textunderscore )}
\end{itemize}
Derramamento de sangue no globo ocular.
\section{Hemantho}
\begin{itemize}
\item {Grp. gram.:m.}
\end{itemize}
\begin{itemize}
\item {Proveniência:(Do gr. \textunderscore haima\textunderscore  + \textunderscore anthos\textunderscore )}
\end{itemize}
Gênero de plantas amaryllídeas.
\section{Hemanto}
\begin{itemize}
\item {Grp. gram.:m.}
\end{itemize}
\begin{itemize}
\item {Proveniência:(Do gr. \textunderscore haima\textunderscore  + \textunderscore anthos\textunderscore )}
\end{itemize}
Gênero de plantas amarilídeas.
\section{Hemapheína}
\begin{itemize}
\item {Grp. gram.:f.}
\end{itemize}
Supposto pigmento biliar, que se verificou modernamente sêr a \textunderscore bilifuscina\textunderscore .
\section{Hemastática}
\begin{itemize}
\item {Grp. gram.:f.}
\end{itemize}
\begin{itemize}
\item {Proveniência:(De \textunderscore hema...\textunderscore  + \textunderscore estática\textunderscore )}
\end{itemize}
Doutrina das leis do equilibrio do sangue nos respectivos vasos.--Seria preferivel \textunderscore hemostática\textunderscore .
\section{Hemataporia}
\begin{itemize}
\item {Grp. gram.:f.}
\end{itemize}
\begin{itemize}
\item {Proveniência:(Do gr. \textunderscore haima\textunderscore  + \textunderscore aporia\textunderscore )}
\end{itemize}
Cachexia, que tem por causa a má qualidade do sangue.
\section{Hemateína}
\begin{itemize}
\item {Grp. gram.:f.}
\end{itemize}
Substância, que se obtém pela acção do ammoníaco sôbre a hematina.
\section{Hematemese}
\begin{itemize}
\item {Grp. gram.:f.}
\end{itemize}
\begin{itemize}
\item {Proveniência:(Do gr. \textunderscore haima\textunderscore , \textunderscore haimetos\textunderscore , sangue, e \textunderscore emesis\textunderscore , vómito)}
\end{itemize}
Vómito de sangue, derivado da membrana mucosa do estômago.
\section{Hematia}
\begin{itemize}
\item {Grp. gram.:f.}
\end{itemize}
\begin{itemize}
\item {Proveniência:(Do gr. \textunderscore haima\textunderscore )}
\end{itemize}
Os glóbulos vermelhos do sangue.
\section{Hemático}
\begin{itemize}
\item {Grp. gram.:adj.}
\end{itemize}
\begin{itemize}
\item {Utilização:Ant.}
\end{itemize}
\begin{itemize}
\item {Proveniência:(Do gr. \textunderscore haima\textunderscore , \textunderscore haimatos\textunderscore )}
\end{itemize}
Diz-se do animal que tem sangue, por opposição a \textunderscore anemático\textunderscore , segundo a classificação zoológica de Aristóteles.
\section{Hematidrose}
\begin{itemize}
\item {Grp. gram.:f.}
\end{itemize}
\begin{itemize}
\item {Utilização:Med.}
\end{itemize}
\begin{itemize}
\item {Proveniência:(Do gr. \textunderscore haima\textunderscore  + \textunderscore idrosis\textunderscore )}
\end{itemize}
Suor de sangue.
\section{Hematina}
\begin{itemize}
\item {Grp. gram.:f.}
\end{itemize}
O mesmo que \textunderscore hematosina\textunderscore .
\section{Hematita}
\begin{itemize}
\item {Grp. gram.:f.}
\end{itemize}
\begin{itemize}
\item {Proveniência:(Do gr. \textunderscore haima\textunderscore , \textunderscore haimatos\textunderscore )}
\end{itemize}
Peróxydo de ferro, de que há duas espécies: a hematita parda e a vermelha. Cf. Castilho, \textunderscore Fastos\textunderscore , II, 355.
\section{Hematite}
\begin{itemize}
\item {Grp. gram.:f.}
\end{itemize}
\begin{itemize}
\item {Proveniência:(Do gr. \textunderscore haima\textunderscore , \textunderscore haimatos\textunderscore )}
\end{itemize}
Peróxydo de ferro, de que há duas espécies: a hematita parda e a vermelha. Cf. Castilho, \textunderscore Fastos\textunderscore , II, 355.
\section{Hemato...}
\begin{itemize}
\item {Grp. gram.:pref.}
\end{itemize}
O mesmo que \textunderscore hema...\textunderscore 
\section{Hematóbio}
\begin{itemize}
\item {Grp. gram.:adj.}
\end{itemize}
\begin{itemize}
\item {Proveniência:(Do gr. \textunderscore haima\textunderscore , \textunderscore haimatos\textunderscore  + \textunderscore bios\textunderscore )}
\end{itemize}
Que vive no sangue.
\section{Hematocarpo}
\begin{itemize}
\item {Grp. gram.:adj.}
\end{itemize}
\begin{itemize}
\item {Utilização:Bot.}
\end{itemize}
\begin{itemize}
\item {Proveniência:(Do gr. \textunderscore haima\textunderscore  + \textunderscore karpos\textunderscore )}
\end{itemize}
Que dá frutos raiados de sangue.
\section{Hematocéfalo}
\begin{itemize}
\item {Grp. gram.:m.}
\end{itemize}
\begin{itemize}
\item {Proveniência:(Do gr. \textunderscore haima\textunderscore  + \textunderscore kephale\textunderscore )}
\end{itemize}
Tumor sanguíneo no cérebro.
\section{Hematocele}
\begin{itemize}
\item {Grp. gram.:m.}
\end{itemize}
\begin{itemize}
\item {Proveniência:(Do gr. \textunderscore haima\textunderscore  + \textunderscore kele\textunderscore )}
\end{itemize}
Tumor sanguíneo.
\section{Hematocéphalo}
\begin{itemize}
\item {Grp. gram.:m.}
\end{itemize}
\begin{itemize}
\item {Proveniência:(Do gr. \textunderscore haima\textunderscore  + \textunderscore kephale\textunderscore )}
\end{itemize}
Tumor sanguíneo no cérebro.
\section{Hematócero}
\begin{itemize}
\item {Grp. gram.:m.}
\end{itemize}
\begin{itemize}
\item {Proveniência:(Do gr. \textunderscore haima\textunderscore , \textunderscore haimatos\textunderscore  + \textunderscore keras\textunderscore )}
\end{itemize}
Gênero de insectos hemípteros.
\section{Hematodo}
\begin{itemize}
\item {Grp. gram.:adj.}
\end{itemize}
\begin{itemize}
\item {Proveniência:(Gr. \textunderscore haimatodes\textunderscore )}
\end{itemize}
Que é da natureza do sangue; hematoide.
\section{Hematófago}
\begin{itemize}
\item {Grp. gram.:adj.}
\end{itemize}
\begin{itemize}
\item {Proveniência:(Do gr. \textunderscore haima\textunderscore  + \textunderscore phagein\textunderscore )}
\end{itemize}
Que se alimenta de sangue.--A pulga e o persevejo são hematófagos.
\section{Hematofilo}
\begin{itemize}
\item {Grp. gram.:adj.}
\end{itemize}
\begin{itemize}
\item {Utilização:Bot.}
\end{itemize}
\begin{itemize}
\item {Proveniência:(Do gr. \textunderscore haima\textunderscore  + \textunderscore phullon\textunderscore )}
\end{itemize}
Que tem fôlhas vermelhas como sangue.
\section{Hematofobia}
\begin{itemize}
\item {Grp. gram.:f.}
\end{itemize}
Horror ao sangue.
(Cp. \textunderscore hematófobo\textunderscore )
\section{Hematófobo}
\begin{itemize}
\item {Grp. gram.:m.}
\end{itemize}
\begin{itemize}
\item {Proveniência:(Do gr. \textunderscore haima\textunderscore  + \textunderscore phobos\textunderscore )}
\end{itemize}
Aquele que não póde vêr sangue, que tem horror ao sangue.
\section{Hematografia}
\begin{itemize}
\item {Grp. gram.:f.}
\end{itemize}
\begin{itemize}
\item {Proveniência:(Do gr. \textunderscore haima\textunderscore  + \textunderscore graphein\textunderscore )}
\end{itemize}
Tratado á cêrca do sangue.
\section{Hematógrafo}
\begin{itemize}
\item {Grp. gram.:m.}
\end{itemize}
Aquele que é versado em hematografia ou trata dela.
\section{Hematographia}
\begin{itemize}
\item {Grp. gram.:f.}
\end{itemize}
\begin{itemize}
\item {Proveniência:(Do gr. \textunderscore haima\textunderscore  + \textunderscore graphein\textunderscore )}
\end{itemize}
Tratado á cêrca do sangue.
\section{Hematógrapho}
\begin{itemize}
\item {Grp. gram.:m.}
\end{itemize}
Aquelle que é versado em hematographia ou trata della.
\section{Hematoide}
\begin{itemize}
\item {Grp. gram.:adj.}
\end{itemize}
\begin{itemize}
\item {Proveniência:(Do gr. \textunderscore haima\textunderscore  + \textunderscore eidos\textunderscore )}
\end{itemize}
Semelhante ao sangue.
\section{Hematoidina}
\begin{itemize}
\item {Grp. gram.:f.}
\end{itemize}
\begin{itemize}
\item {Proveniência:(De \textunderscore hematoide\textunderscore )}
\end{itemize}
Uma das matérias còrantes da bílis.
\section{Hematologia}
\begin{itemize}
\item {Grp. gram.:f.}
\end{itemize}
\begin{itemize}
\item {Proveniência:(Do gr. \textunderscore haima\textunderscore  + \textunderscore logos\textunderscore )}
\end{itemize}
Tratado theórico á cêrca do sangue.
\section{Hematológico}
\begin{itemize}
\item {Grp. gram.:adj.}
\end{itemize}
Relativo á hematologia.
\section{Hematoma}
\begin{itemize}
\item {Grp. gram.:m.}
\end{itemize}
\begin{itemize}
\item {Proveniência:(Do gr. \textunderscore haima\textunderscore )}
\end{itemize}
Tumor sanguíneo, resultante de contusão, da ruptura de varizes, etc.
\section{Hematômphalo}
\begin{itemize}
\item {Grp. gram.:m.}
\end{itemize}
\begin{itemize}
\item {Proveniência:(Do gr. \textunderscore haima\textunderscore  + \textunderscore omphalos\textunderscore )}
\end{itemize}
Hérnia umbilical, cujo saco contém serosidade e sangue derramado, e apresenta na superfície veias varicosas.
\section{Hematóphago}
\begin{itemize}
\item {Grp. gram.:adj.}
\end{itemize}
\begin{itemize}
\item {Proveniência:(Do gr. \textunderscore haima\textunderscore  + \textunderscore phagein\textunderscore )}
\end{itemize}
Que se alimenta de sangue.--A pulga e o persevejo são hematóphagos.
\section{Hematophobia}
\begin{itemize}
\item {Grp. gram.:f.}
\end{itemize}
Horror ao sangue.
(Cp. \textunderscore hematóphobo\textunderscore )
\section{Hematóphobo}
\begin{itemize}
\item {Grp. gram.:m.}
\end{itemize}
\begin{itemize}
\item {Proveniência:(Do gr. \textunderscore haima\textunderscore  + \textunderscore phobos\textunderscore )}
\end{itemize}
Aquelle que não póde vêr sangue, que tem horror ao sangue.
\section{Hematophyllo}
\begin{itemize}
\item {Grp. gram.:adj.}
\end{itemize}
\begin{itemize}
\item {Utilização:Bot.}
\end{itemize}
\begin{itemize}
\item {Proveniência:(Do gr. \textunderscore haima\textunderscore  + \textunderscore phullon\textunderscore )}
\end{itemize}
Que tem fôlhas vermelhas como sangue.
\section{Hemátopo}
\begin{itemize}
\item {Grp. gram.:m.}
\end{itemize}
\begin{itemize}
\item {Proveniência:(Do gr. \textunderscore haima\textunderscore  + \textunderscore pous\textunderscore )}
\end{itemize}
Ave asiática, que tem os pés vermelhos como sangue.
\section{Hematopoése}
\begin{itemize}
\item {Grp. gram.:f.}
\end{itemize}
\begin{itemize}
\item {Proveniência:(Do gr. \textunderscore haima\textunderscore , \textunderscore haimatos\textunderscore  + \textunderscore poiein\textunderscore )}
\end{itemize}
Formação dos glóbulos vermelhos do sangue.
\section{Hematopoético}
\begin{itemize}
\item {Grp. gram.:adj.}
\end{itemize}
\begin{itemize}
\item {Proveniência:(De \textunderscore hematopoése\textunderscore )}
\end{itemize}
Que fórma sangue.
\section{Hematosar-se}
\begin{itemize}
\item {Grp. gram.:v. p.}
\end{itemize}
\begin{itemize}
\item {Proveniência:(De \textunderscore hematose\textunderscore )}
\end{itemize}
Converter-se de venoso em arterial (o sangue).
\section{Hematoscópio}
\begin{itemize}
\item {Grp. gram.:m.}
\end{itemize}
\begin{itemize}
\item {Proveniência:(Do gr. \textunderscore aima\textunderscore , \textunderscore aimatos\textunderscore  + \textunderscore skopein\textunderscore )}
\end{itemize}
Espectroscópio, para examinar o sangue, que circula debaixo dos tegumentos.
\section{Hematose}
\begin{itemize}
\item {Grp. gram.:f.}
\end{itemize}
\begin{itemize}
\item {Proveniência:(Gr. \textunderscore haimatosis\textunderscore )}
\end{itemize}
Conversão do sangue venoso em arterial, no pulmão, ao contacto do ar aspirado.
Sanguificação.
\section{Hematosina}
\begin{itemize}
\item {Grp. gram.:f.}
\end{itemize}
\begin{itemize}
\item {Proveniência:(De \textunderscore hematose\textunderscore )}
\end{itemize}
Matéria còrante do sangue.
\section{Hematoxilina}
\begin{itemize}
\item {Grp. gram.:f.}
\end{itemize}
\begin{itemize}
\item {Proveniência:(De \textunderscore hematóxilo\textunderscore )}
\end{itemize}
Princípio còrante do pau campeche.
\section{Hematóxilo}
\begin{itemize}
\item {Grp. gram.:m.}
\end{itemize}
\begin{itemize}
\item {Proveniência:(Do gr. \textunderscore haima\textunderscore  + \textunderscore xulon\textunderscore )}
\end{itemize}
Grande árvore leguminosa das regiões tropicaes, mais conhecida por \textunderscore campeche\textunderscore .
\section{Hematoxylina}
\begin{itemize}
\item {Grp. gram.:f.}
\end{itemize}
\begin{itemize}
\item {Proveniência:(De \textunderscore hematóxylo\textunderscore )}
\end{itemize}
Princípio còrante do pau campeche.
\section{Hematóxylo}
\begin{itemize}
\item {Grp. gram.:m.}
\end{itemize}
\begin{itemize}
\item {Proveniência:(Do gr. \textunderscore haima\textunderscore  + \textunderscore xulon\textunderscore )}
\end{itemize}
Grande árvore leguminosa das regiões tropicaes, mais conhecida por \textunderscore campeche\textunderscore .
\section{Hematozoários}
\begin{itemize}
\item {Grp. gram.:m. pl.}
\end{itemize}
\begin{itemize}
\item {Proveniência:(Do gr. \textunderscore haima\textunderscore  + \textunderscore zoarion\textunderscore )}
\end{itemize}
Animaes, que vivem no sangue.
\section{Hematropina}
\begin{itemize}
\item {Grp. gram.:f.}
\end{itemize}
Espécie de collýrio.
\section{Hematuria}
\begin{itemize}
\item {Grp. gram.:f.}
\end{itemize}
\begin{itemize}
\item {Proveniência:(Do gr. \textunderscore haima\textunderscore  + \textunderscore ouron\textunderscore )}
\end{itemize}
Fluxo de sangue pela uretra.
\section{Hematúrico}
\begin{itemize}
\item {Grp. gram.:adj.}
\end{itemize}
\begin{itemize}
\item {Grp. gram.:M.}
\end{itemize}
Relativo á hematuria.
Aquelle que soffre hematuria.
\section{Hemeralopia}
\begin{itemize}
\item {Grp. gram.:f.}
\end{itemize}
\begin{itemize}
\item {Proveniência:(T. mal formado, do gr. \textunderscore hemera\textunderscore  + \textunderscore ops\textunderscore )}
\end{itemize}
Cegueira nocturna, ou inaptidão para perceber a escassa luz que haja á noite ou ao crepúsculo.
\section{Hemeralópico}
\begin{itemize}
\item {Grp. gram.:adj.}
\end{itemize}
\begin{itemize}
\item {Grp. gram.:M.}
\end{itemize}
Relativo á hemeralopia.
Aquelle que soffre hemeralopia.
\section{Hemeróbidos}
\begin{itemize}
\item {Grp. gram.:m. pl.}
\end{itemize}
\begin{itemize}
\item {Proveniência:(Do gr. \textunderscore hemera\textunderscore  + \textunderscore bios\textunderscore  + \textunderscore eidos\textunderscore )}
\end{itemize}
Família de insectos, que têm por typo o hemeróbio.
\section{Hemeróbio}
\begin{itemize}
\item {Grp. gram.:m.}
\end{itemize}
\begin{itemize}
\item {Proveniência:(Do gr. \textunderscore hemera\textunderscore  + \textunderscore bíos\textunderscore )}
\end{itemize}
Gênero de insectos neurópteros, que só vivem um dia.
\section{Hemerocale}
\begin{itemize}
\item {Grp. gram.:f.}
\end{itemize}
\begin{itemize}
\item {Proveniência:(Do gr. \textunderscore hemera\textunderscore , dia, e \textunderscore kallis\textunderscore , belleza)}
\end{itemize}
Formosa planta liliácea.
\section{Hemerocalídeos}
\begin{itemize}
\item {Grp. gram.:f.}
\end{itemize}
\begin{itemize}
\item {Proveniência:(De \textunderscore hemerocalle\textunderscore )}
\end{itemize}
Família de plantas, estabelecida por Brown, á custa das liliáceas.
\section{Hemerocalle}
\begin{itemize}
\item {Grp. gram.:f.}
\end{itemize}
\begin{itemize}
\item {Proveniência:(Do gr. \textunderscore hemera\textunderscore , dia, e \textunderscore kallis\textunderscore , belleza)}
\end{itemize}
Formosa planta liliácea.
\section{Hemerocallídeos}
\begin{itemize}
\item {Grp. gram.:f.}
\end{itemize}
\begin{itemize}
\item {Proveniência:(De \textunderscore hemerocalle\textunderscore )}
\end{itemize}
Família de plantas, estabelecida por Brown, á custa das liliáceas.
\section{Hemeródromo}
\begin{itemize}
\item {Grp. gram.:m.}
\end{itemize}
\begin{itemize}
\item {Proveniência:(Gr. \textunderscore hemerodromos\textunderscore )}
\end{itemize}
Correio ou postilhão que, entre os Gregos, levava a correspondência official com a maior rapidez.
\section{Hemerologia}
\begin{itemize}
\item {Grp. gram.:f.}
\end{itemize}
\begin{itemize}
\item {Proveniência:(Do gr. \textunderscore hemera\textunderscore , dia, e \textunderscore logos\textunderscore , tratado)}
\end{itemize}
Arte de compor calendários.
\section{Hemerológio}
\begin{itemize}
\item {Grp. gram.:m.}
\end{itemize}
Tratado á cêrca da concordância dos calendários.
(Cp. \textunderscore hemerologia\textunderscore )
\section{Hemeropata}
\begin{itemize}
\item {Grp. gram.:m.  e  f.}
\end{itemize}
\begin{itemize}
\item {Proveniência:(Do gr. \textunderscore hemera\textunderscore , dia, e \textunderscore pathos\textunderscore , doença)}
\end{itemize}
Pessôa, que sofre hemeropatia.
\section{Hemeropatha}
\begin{itemize}
\item {Grp. gram.:m.  e  f.}
\end{itemize}
\begin{itemize}
\item {Proveniência:(Do gr. \textunderscore hemera\textunderscore , dia, e \textunderscore pathos\textunderscore , doença)}
\end{itemize}
Pessôa, que soffre hemeropathia.
\section{Hemeropathia}
\begin{itemize}
\item {Grp. gram.:f.}
\end{itemize}
\begin{itemize}
\item {Proveniência:(De \textunderscore hemeropatha\textunderscore )}
\end{itemize}
Doença, que só se manifesta durante o dia.
\section{Hemeropatia}
\begin{itemize}
\item {Grp. gram.:f.}
\end{itemize}
\begin{itemize}
\item {Proveniência:(De \textunderscore hemeropata\textunderscore )}
\end{itemize}
Doença, que só se manifesta durante o dia.
\section{Hematônfalo}
\begin{itemize}
\item {Grp. gram.:m.}
\end{itemize}
\begin{itemize}
\item {Proveniência:(Do gr. \textunderscore haima\textunderscore  + \textunderscore omphalos\textunderscore )}
\end{itemize}
Hérnia umbilical, cujo saco contém serosidade e sangue derramado, e apresenta na superfície veias varicosas.
\section{Hemi...}
\begin{itemize}
\item {Grp. gram.:pref.}
\end{itemize}
\begin{itemize}
\item {Proveniência:(Do gr. \textunderscore hemi\textunderscore )}
\end{itemize}
(designativo de \textunderscore metade\textunderscore )
\section{Hemiacefalia}
\begin{itemize}
\item {Grp. gram.:f.}
\end{itemize}
\begin{itemize}
\item {Proveniência:(De \textunderscore hemi...\textunderscore  + \textunderscore acéfalo\textunderscore )}
\end{itemize}
Monstruosidade, em que a cabeça é representada por um tumor informe, com alguns apêndices.
\section{Hemiacephalia}
\begin{itemize}
\item {Grp. gram.:f.}
\end{itemize}
\begin{itemize}
\item {Proveniência:(De \textunderscore hemi...\textunderscore  + \textunderscore acéphalo\textunderscore )}
\end{itemize}
Monstruosidade, em que a cabeça é representada por um tumor informe, com alguns appêndices.
\section{Hemialgia}
\begin{itemize}
\item {Grp. gram.:f.}
\end{itemize}
\begin{itemize}
\item {Proveniência:(Do gr. \textunderscore hemi\textunderscore  + \textunderscore algos\textunderscore )}
\end{itemize}
O mesmo que \textunderscore hemicrânia\textunderscore .
\section{Hemianestesia}
\begin{itemize}
\item {Grp. gram.:f.}
\end{itemize}
\begin{itemize}
\item {Proveniência:(De \textunderscore hemi...\textunderscore  + \textunderscore anestesia\textunderscore )}
\end{itemize}
Anestesia incompleta.
\section{Hemianesthesia}
\begin{itemize}
\item {Grp. gram.:f.}
\end{itemize}
\begin{itemize}
\item {Proveniência:(De \textunderscore hemi...\textunderscore  + \textunderscore anesthesia\textunderscore )}
\end{itemize}
Anesthesia incompleta.
\section{Hemicarpo}
\begin{itemize}
\item {Grp. gram.:m.}
\end{itemize}
\begin{itemize}
\item {Utilização:Bot.}
\end{itemize}
\begin{itemize}
\item {Proveniência:(Do gr. \textunderscore hemi\textunderscore  + \textunderscore karpos\textunderscore )}
\end{itemize}
Metade de um fruto que naturalmente se divide em dois.
\section{Hemichoreia}
\begin{itemize}
\item {fónica:co}
\end{itemize}
\begin{itemize}
\item {Grp. gram.:f.}
\end{itemize}
\begin{itemize}
\item {Utilização:Med.}
\end{itemize}
\begin{itemize}
\item {Proveniência:(De \textunderscore hemi...\textunderscore  + \textunderscore choreia\textunderscore )}
\end{itemize}
Choreia, em metade do corpo.
\section{Hemicíclico}
\begin{itemize}
\item {Grp. gram.:adj}
\end{itemize}
Relativo a hemiciclo; semicircular.
\section{Hemiciclo}
\begin{itemize}
\item {Grp. gram.:m.}
\end{itemize}
\begin{itemize}
\item {Proveniência:(Gr. \textunderscore hemikuklos\textunderscore )}
\end{itemize}
Espaço semi-circular.
\section{Hemicilíndrico}
\begin{itemize}
\item {Grp. gram.:adj.}
\end{itemize}
\begin{itemize}
\item {Proveniência:(De \textunderscore hemi...\textunderscore  + \textunderscore cilíndrico\textunderscore )}
\end{itemize}
Semelhante a metade de um cilindro.
\section{Hemicilindro}
\begin{itemize}
\item {Grp. gram.:m.}
\end{itemize}
\begin{itemize}
\item {Proveniência:(De \textunderscore hemi...\textunderscore  + \textunderscore cilindro\textunderscore )}
\end{itemize}
Metade de um cilindro.
\section{Hemicoreia}
\begin{itemize}
\item {Grp. gram.:f.}
\end{itemize}
\begin{itemize}
\item {Utilização:Med.}
\end{itemize}
\begin{itemize}
\item {Proveniência:(De \textunderscore hemi...\textunderscore  + \textunderscore coreia\textunderscore )}
\end{itemize}
Choreia, em metade do corpo.
\section{Hemicrânia}
\begin{itemize}
\item {Grp. gram.:f.}
\end{itemize}
\begin{itemize}
\item {Proveniência:(De \textunderscore hemi...\textunderscore  + \textunderscore crânio\textunderscore )}
\end{itemize}
Dôr, que ataca parte da cabeça, ordinariamente a fronte e uma das regiões temporaes.
Enxaqueca.
\section{Hemicrânico}
\begin{itemize}
\item {Grp. gram.:adj.}
\end{itemize}
Relativo á hemicrânia.
\section{Hemicraniectomia}
\begin{itemize}
\item {Grp. gram.:f.}
\end{itemize}
\begin{itemize}
\item {Proveniência:(De \textunderscore hemi...\textunderscore  + \textunderscore craniectomia\textunderscore )}
\end{itemize}
Operação cirúrgica, em que se abre metade do crânio, para tratamento do cérebro.
\section{Hemicýclico}
\begin{itemize}
\item {Grp. gram.:adj}
\end{itemize}
Relativo a hemicyclo; semicircular.
\section{Hemicyclo}
\begin{itemize}
\item {Grp. gram.:m.}
\end{itemize}
\begin{itemize}
\item {Proveniência:(Gr. \textunderscore hemikuklos\textunderscore )}
\end{itemize}
Espaço semi-circular.
\section{Hemicylíndrico}
\begin{itemize}
\item {Grp. gram.:adj.}
\end{itemize}
\begin{itemize}
\item {Proveniência:(De \textunderscore hemi...\textunderscore  + \textunderscore cylíndrico\textunderscore )}
\end{itemize}
Semelhante a metade de um cylindro.
\section{Hemicylindro}
\begin{itemize}
\item {Grp. gram.:m.}
\end{itemize}
\begin{itemize}
\item {Proveniência:(De \textunderscore hemi...\textunderscore  + \textunderscore cylindro\textunderscore )}
\end{itemize}
Metade de um cylindro.
\section{Hemidesmo}
\begin{itemize}
\item {Grp. gram.:m.}
\end{itemize}
Designação scientífica da salsaparrilha indiana.
\section{Hemiedria}
\begin{itemize}
\item {Grp. gram.:f.}
\end{itemize}
\begin{itemize}
\item {Proveniência:(De \textunderscore hemiedro\textunderscore )}
\end{itemize}
Qualidade, que têm certos metaes, de não apresentar modificações, senão em metade das arestas ou dos ângulos semelhantes.
\section{Hemiédrico}
\begin{itemize}
\item {Grp. gram.:adj.}
\end{itemize}
\begin{itemize}
\item {Proveniência:(De \textunderscore hemiedria\textunderscore )}
\end{itemize}
Que tem a qualidade da hemiedria, (falando-se de um crystal).
\section{Hemiedro}
\begin{itemize}
\item {Grp. gram.:m.}
\end{itemize}
\begin{itemize}
\item {Proveniência:(Do gr. \textunderscore hemi\textunderscore  + \textunderscore edra\textunderscore )}
\end{itemize}
Cristal, que tem o carácter da hemiedria ou que não possue senão metade das suas faces.
\section{Hemifacial}
\begin{itemize}
\item {Grp. gram.:adj.}
\end{itemize}
\begin{itemize}
\item {Proveniência:(De \textunderscore hemi...\textunderscore  + \textunderscore facial\textunderscore )}
\end{itemize}
Relativo a metade de face.
\section{Hemigamia}
\begin{itemize}
\item {Grp. gram.:f.}
\end{itemize}
\begin{itemize}
\item {Utilização:Bot.}
\end{itemize}
\begin{itemize}
\item {Proveniência:(Do gr. \textunderscore hemi\textunderscore  + \textunderscore gamos\textunderscore )}
\end{itemize}
Carácter das plantas gramíneas, em que a mesma gluma encerra ao mesmo tempo flôres masculinas, femininas e neutras.
\section{Hemigâmico}
\begin{itemize}
\item {Grp. gram.:adj.}
\end{itemize}
Que tem o carácter da hemigamia.
\section{Hemigiro}
\begin{itemize}
\item {Grp. gram.:m.}
\end{itemize}
Nome, dado por Desvaux ao fruto das proteáceas, quási sempre lenhoso, deiscente de um lado.
\section{Hemigirosa}
\begin{itemize}
\item {Grp. gram.:f.}
\end{itemize}
\begin{itemize}
\item {Proveniência:(De \textunderscore hemigiro\textunderscore )}
\end{itemize}
Árvore indiana, própria para construcções.
\section{Hemigoniário}
\begin{itemize}
\item {Grp. gram.:adj.}
\end{itemize}
\begin{itemize}
\item {Utilização:Bot.}
\end{itemize}
\begin{itemize}
\item {Proveniência:(Do gr. \textunderscore hemi\textunderscore  + \textunderscore gonos\textunderscore )}
\end{itemize}
Diz-se da flôr, em que uma só parte dos órgãos masculinos e femininos se tranforma em pétalas.
\section{Hemigyro}
\begin{itemize}
\item {Grp. gram.:m.}
\end{itemize}
Nome, dado por Desvaux ao fruto das proteáceas, quási sempre lenhoso, dehiscente de um lado.
\section{Hemigyrosa}
\begin{itemize}
\item {Grp. gram.:f.}
\end{itemize}
\begin{itemize}
\item {Proveniência:(De \textunderscore hemigyro\textunderscore )}
\end{itemize}
Árvore indiana, própria para construcções.
\section{Hemilabial}
\begin{itemize}
\item {Grp. gram.:adj.}
\end{itemize}
\begin{itemize}
\item {Proveniência:(De \textunderscore hemi...\textunderscore  + \textunderscore labial\textunderscore )}
\end{itemize}
Relativo a metade dos lábios.
\section{Hemimelia}
\begin{itemize}
\item {Grp. gram.:f.}
\end{itemize}
Qualidade de hemímelo.
\section{Hemímelo}
\begin{itemize}
\item {Grp. gram.:adj.}
\end{itemize}
\begin{itemize}
\item {Proveniência:(Do gr. \textunderscore hemi\textunderscore  + \textunderscore melos\textunderscore )}
\end{itemize}
Diz-se dos monstros, cujos membros thorácicos e abdominaes terminam em côto, sem dedos ou com dedos incompletos.
\section{Hemimeróptero}
\begin{itemize}
\item {Grp. gram.:adj.}
\end{itemize}
\begin{itemize}
\item {Proveniência:(Do gr. \textunderscore hemi\textunderscore  + \textunderscore meros\textunderscore  + \textunderscore pteron\textunderscore )}
\end{itemize}
Diz-se dos insectos que só têm meios elytros.
\section{Hêmina}
\begin{itemize}
\item {Grp. gram.:f.}
\end{itemize}
O mesmo que \textunderscore êmina\textunderscore .
\section{Hemióbolo}
\begin{itemize}
\item {Grp. gram.:m.}
\end{itemize}
\begin{itemize}
\item {Proveniência:(Do gr. \textunderscore hemiobolion\textunderscore )}
\end{itemize}
Metade do óbolo, entre os antigos Gregos.
\section{Hemioctaédro}
\begin{itemize}
\item {Grp. gram.:m.}
\end{itemize}
\begin{itemize}
\item {Proveniência:(De \textunderscore hemi...\textunderscore  + \textunderscore octaédro\textunderscore )}
\end{itemize}
O mesmo que \textunderscore tètraédro\textunderscore .
\section{Hemiólia}
\begin{itemize}
\item {Grp. gram.:f.}
\end{itemize}
\begin{itemize}
\item {Utilização:Mús.}
\end{itemize}
\begin{itemize}
\item {Utilização:ant.}
\end{itemize}
O mesmo que \textunderscore sèsquiáltera\textunderscore .
\section{Hemíona}
\begin{itemize}
\item {Grp. gram.:f.}
\end{itemize}
\begin{itemize}
\item {Proveniência:(Gr. \textunderscore hemionos\textunderscore )}
\end{itemize}
Espécie de cavallo selvagem, (\textunderscore equus emionus\textunderscore ).
\section{Hemíono}
\begin{itemize}
\item {Grp. gram.:m.}
\end{itemize}
O mesmo ou melhor que \textunderscore hemíona\textunderscore .
\section{Hemiopia}
\begin{itemize}
\item {Grp. gram.:f.}
\end{itemize}
\begin{itemize}
\item {Proveniência:(Do gr. \textunderscore hemi\textunderscore  + \textunderscore ops\textunderscore )}
\end{itemize}
Enfermidade, que só deixa vêr uma parte dos objectos.
\section{Hemiopsia}
\begin{itemize}
\item {Grp. gram.:f.}
\end{itemize}
O mesmo que \textunderscore hemiopia\textunderscore .
\section{Hemiorganizado}
\begin{itemize}
\item {Grp. gram.:adj.}
\end{itemize}
\begin{itemize}
\item {Utilização:Physiol.}
\end{itemize}
\begin{itemize}
\item {Proveniência:(De \textunderscore hemi...\textunderscore  + \textunderscore organizado\textunderscore )}
\end{itemize}
Diz-se das albuminas, das fibrinas, e de outras substâncias, que occupam o meio termo entre o princípio immediato e o tecido organizado.
\section{Hemiplegia}
\begin{itemize}
\item {Grp. gram.:f.}
\end{itemize}
\begin{itemize}
\item {Proveniência:(Lat. \textunderscore hemiplegia\textunderscore )}
\end{itemize}
Paralysia de um dos lados do corpo.
\section{Hemiplégico}
\begin{itemize}
\item {Grp. gram.:adj.}
\end{itemize}
\begin{itemize}
\item {Grp. gram.:M.}
\end{itemize}
\begin{itemize}
\item {Proveniência:(De \textunderscore hemiplegia\textunderscore )}
\end{itemize}
Que tem hemiplegia.
Indivíduo paralýtico de um lado.
\section{Hemipirâmide}
\begin{itemize}
\item {Grp. gram.:f.}
\end{itemize}
\begin{itemize}
\item {Utilização:Miner.}
\end{itemize}
\begin{itemize}
\item {Proveniência:(De \textunderscore hemi...\textunderscore  + \textunderscore pirâmide\textunderscore )}
\end{itemize}
Fórma holoédrica dos metaes, que representam metade de uma pirâmide.
\section{Hemiplexia}
\begin{itemize}
\item {fónica:csi}
\end{itemize}
\begin{itemize}
\item {Grp. gram.:f.}
\end{itemize}
O mesmo que \textunderscore hemiplegia\textunderscore .
\section{Hemipo}
\begin{itemize}
\item {Grp. gram.:m.}
\end{itemize}
\begin{itemize}
\item {Proveniência:(Do gr. \textunderscore hemi\textunderscore  + \textunderscore hippos\textunderscore )}
\end{itemize}
Equideo da Síria.
\section{Hemippo}
\begin{itemize}
\item {Grp. gram.:m.}
\end{itemize}
\begin{itemize}
\item {Proveniência:(Do gr. \textunderscore hemi\textunderscore  + \textunderscore hippos\textunderscore )}
\end{itemize}
Equideo da Sýria.
\section{Hemiprismático}
\begin{itemize}
\item {Grp. gram.:adj.}
\end{itemize}
\begin{itemize}
\item {Utilização:Miner.}
\end{itemize}
\begin{itemize}
\item {Proveniência:(De \textunderscore hemi...\textunderscore  + \textunderscore prismático\textunderscore )}
\end{itemize}
Diz-se dos crystaes que só deixam vêr metade das suas faces.
\section{Hemíptero}
\begin{itemize}
\item {Grp. gram.:adj.}
\end{itemize}
\begin{itemize}
\item {Utilização:Zool.}
\end{itemize}
\begin{itemize}
\item {Grp. gram.:M. pl.}
\end{itemize}
\begin{itemize}
\item {Proveniência:(Do gr. \textunderscore hemi\textunderscore  + \textunderscore pteron\textunderscore )}
\end{itemize}
Que tem asas ou barbatanas curtas.
Gênero de insectos, que têm a bôca em fórma de bico apropriado á sucção, e as asas anteriores membranosas na ponta e duras na base, quando não são duras em toda a extensão.
\section{Hemipyrâmide}
\begin{itemize}
\item {Grp. gram.:f.}
\end{itemize}
\begin{itemize}
\item {Utilização:Miner.}
\end{itemize}
\begin{itemize}
\item {Proveniência:(De \textunderscore hemi...\textunderscore  + \textunderscore pyrâmide\textunderscore )}
\end{itemize}
Fórma holoédrica dos metaes, que representam metade de uma pyrâmide.
\section{Hemisalamandras}
\begin{itemize}
\item {fónica:sa}
\end{itemize}
\begin{itemize}
\item {Grp. gram.:f. pl.}
\end{itemize}
\begin{itemize}
\item {Proveniência:(De \textunderscore hemi...\textunderscore  + \textunderscore salamandra\textunderscore )}
\end{itemize}
Tríbo de batrácios, semelhantes á salamandra.
\section{Hemisférico}
\begin{itemize}
\item {Grp. gram.:adj.}
\end{itemize}
Que têm fórma de Hemisfério.
\section{Hemisfério}
\begin{itemize}
\item {Grp. gram.:m.}
\end{itemize}
\begin{itemize}
\item {Utilização:Des.}
\end{itemize}
\begin{itemize}
\item {Proveniência:(Lat. \textunderscore hemisphaerium\textunderscore )}
\end{itemize}
Metade de uma esfera.
Cada uma das duas metades da Terra, separadas imaginariamente pelo círculo do Equador.
Metade do globo celeste.
Cerebelo.
\section{Hemisferoédrico}
\begin{itemize}
\item {Grp. gram.:adj.}
\end{itemize}
\begin{itemize}
\item {Utilização:Miner.}
\end{itemize}
\begin{itemize}
\item {Proveniência:(Do gr. \textunderscore hemisphairion\textunderscore  + \textunderscore edra\textunderscore )}
\end{itemize}
Diz-se dos cristaes, que têm a aparência de um hemisferoide.
\section{Hemisferoidal}
\begin{itemize}
\item {Grp. gram.:adj.}
\end{itemize}
Semelhante a um hemisferoide.
\section{Hemisferoide}
\begin{itemize}
\item {Grp. gram.:m.}
\end{itemize}
\begin{itemize}
\item {Grp. gram.:Adj.}
\end{itemize}
\begin{itemize}
\item {Proveniência:(De \textunderscore hemi...\textunderscore  + \textunderscore esferoide\textunderscore )}
\end{itemize}
Metade de um esferoide.
Semelhante a metade de um esferoide.
\section{Hemispério}
\begin{itemize}
\item {Grp. gram.:m.}
\end{itemize}
\begin{itemize}
\item {Utilização:Ant.}
\end{itemize}
O mesmo que \textunderscore hemisphério\textunderscore . Cf. Usque, etc.
\section{Hemisphérico}
\begin{itemize}
\item {Grp. gram.:adj.}
\end{itemize}
Que têm fórma de hemisphério.
\section{Hemisphério}
\begin{itemize}
\item {Grp. gram.:m.}
\end{itemize}
\begin{itemize}
\item {Utilização:Des.}
\end{itemize}
\begin{itemize}
\item {Proveniência:(Lat. \textunderscore hemisphaerium\textunderscore )}
\end{itemize}
Metade de uma esphera.
Cada uma das duas metades da Terra, separadas imaginariamente pelo círculo do Equador.
Metade do globo celeste.
Cerebello.
\section{Hemispheroédrico}
\begin{itemize}
\item {Grp. gram.:adj.}
\end{itemize}
\begin{itemize}
\item {Utilização:Miner.}
\end{itemize}
\begin{itemize}
\item {Proveniência:(Do gr. \textunderscore hemisphairion\textunderscore  + \textunderscore edra\textunderscore )}
\end{itemize}
Diz-se dos crystaes, que têm a apparência de um hemispheroide.
\section{Hemispheroidal}
\begin{itemize}
\item {Grp. gram.:adj.}
\end{itemize}
Semelhante a um hemispheroide.
\section{Hemispheroide}
\begin{itemize}
\item {Grp. gram.:m.}
\end{itemize}
\begin{itemize}
\item {Grp. gram.:Adj.}
\end{itemize}
\begin{itemize}
\item {Proveniência:(De \textunderscore hemi...\textunderscore  + \textunderscore espheroide\textunderscore )}
\end{itemize}
Metade de um espheroide.
Semelhante a metade de um espheroide.
\section{Hemissalamandras}
\begin{itemize}
\item {Grp. gram.:f. pl.}
\end{itemize}
\begin{itemize}
\item {Proveniência:(De \textunderscore hemi...\textunderscore  + \textunderscore salamandra\textunderscore )}
\end{itemize}
Tríbo de batrácios, semelhantes á salamandra.
\section{Hemissingínico}
\begin{itemize}
\item {Grp. gram.:adj.}
\end{itemize}
\begin{itemize}
\item {Utilização:Bot.}
\end{itemize}
\begin{itemize}
\item {Proveniência:(Do gr. \textunderscore hemi\textunderscore  + \textunderscore sun\textunderscore  + \textunderscore gune\textunderscore )}
\end{itemize}
Diz-se do cálice, que está meio aderente ao ovário.
\section{Hemistíchio}
\begin{itemize}
\item {fónica:qui}
\end{itemize}
\begin{itemize}
\item {Grp. gram.:m.}
\end{itemize}
\begin{itemize}
\item {Proveniência:(Gr. \textunderscore hemistikhion\textunderscore )}
\end{itemize}
Metade de um verso alexandrino.
\section{Hemistíquio}
\begin{itemize}
\item {Grp. gram.:m.}
\end{itemize}
\begin{itemize}
\item {Proveniência:(Gr. \textunderscore hemistikhion\textunderscore )}
\end{itemize}
Metade de um verso alexandrino.
\section{Hemisyngýnico}
\begin{itemize}
\item {fónica:sin}
\end{itemize}
\begin{itemize}
\item {Grp. gram.:adj.}
\end{itemize}
\begin{itemize}
\item {Utilização:Bot.}
\end{itemize}
\begin{itemize}
\item {Proveniência:(Do gr. \textunderscore hemi\textunderscore  + \textunderscore sun\textunderscore  + \textunderscore gune\textunderscore )}
\end{itemize}
Diz-se do cálice, que está meio adherente ao ovário.
\section{Hemiteria}
\begin{itemize}
\item {Grp. gram.:f.}
\end{itemize}
\begin{itemize}
\item {Utilização:Anat.}
\end{itemize}
\begin{itemize}
\item {Proveniência:(Do gr. \textunderscore hemi\textunderscore  + \textunderscore teras\textunderscore )}
\end{itemize}
Anomalia orgânica, de pequena gravidade.
\section{Hemithrena}
\begin{itemize}
\item {Grp. gram.:f.}
\end{itemize}
Rocha, espécie de amphíbolo.
\section{Hemítomo}
\begin{itemize}
\item {Grp. gram.:adj.}
\end{itemize}
\begin{itemize}
\item {Utilização:Miner.}
\end{itemize}
\begin{itemize}
\item {Proveniência:(Do gr. \textunderscore hemi\textunderscore  + \textunderscore tome\textunderscore )}
\end{itemize}
Diz-se dos crystaes, compostos de duas partes distintas, quando as faces de uma encontram o eixo da outra no meio da sua altura.
\section{Hemitono}
\begin{itemize}
\item {Grp. gram.:m.}
\end{itemize}
\begin{itemize}
\item {Utilização:Des.}
\end{itemize}
O mesmo que \textunderscore semitom\textunderscore .
\section{Hemitrena}
\begin{itemize}
\item {Grp. gram.:f.}
\end{itemize}
Rocha, espécie de anfíbolo.
\section{Hemitriteia}
\begin{itemize}
\item {Grp. gram.:f.  e  adj. f.}
\end{itemize}
\begin{itemize}
\item {Utilização:Med.}
\end{itemize}
\begin{itemize}
\item {Proveniência:(Do gr. \textunderscore hemi\textunderscore  + \textunderscore tritaios\textunderscore )}
\end{itemize}
Diz-se de uma febre intermittente, que consiste em um accesso cada dia, sendo um mais forte, de dois em dois dias.
\section{Hemitritia}
\begin{itemize}
\item {Grp. gram.:f.  e  adj. f.}
\end{itemize}
O mesmo que \textunderscore hemitriteia\textunderscore .
\section{Hemitrítica}
\begin{itemize}
\item {Grp. gram.:f.  e  adj. f.}
\end{itemize}
O mesmo que \textunderscore hemitriteia\textunderscore .
\section{Hemitropia}
\begin{itemize}
\item {Grp. gram.:f.}
\end{itemize}
\begin{itemize}
\item {Utilização:Miner.}
\end{itemize}
\begin{itemize}
\item {Proveniência:(De \textunderscore hemítropo\textunderscore )}
\end{itemize}
Crystallização, que torna hemítropos os crystaes.
\section{Hemítropo}
\begin{itemize}
\item {Grp. gram.:adj.}
\end{itemize}
\begin{itemize}
\item {Utilização:Miner.}
\end{itemize}
\begin{itemize}
\item {Proveniência:(Do gr. \textunderscore hemi\textunderscore  + \textunderscore trope\textunderscore )}
\end{itemize}
Diz-se de um crystal, em que uma das duas faces oppostas parece têr feito, sôbre a outra, metade de uma rotação.
\section{Hemo...}
\begin{itemize}
\item {Grp. gram.:pref.}
\end{itemize}
O mesmo ou melhor que \textunderscore hema...\textunderscore 
\section{Hemochroína}
\begin{itemize}
\item {Grp. gram.:f.}
\end{itemize}
(V.hematosina)
\section{Hemocroína}
\begin{itemize}
\item {Grp. gram.:f.}
\end{itemize}
(V.hematosina)
\section{Hemódia}
\begin{itemize}
\item {Grp. gram.:f.}
\end{itemize}
\begin{itemize}
\item {Utilização:Med.}
\end{itemize}
\begin{itemize}
\item {Proveniência:(Gr. \textunderscore haimodia\textunderscore )}
\end{itemize}
Embotamento dos dentes, acompanhado de rangido e sabor ácido.
\section{Hemodinamómetro}
\begin{itemize}
\item {Grp. gram.:m.}
\end{itemize}
\begin{itemize}
\item {Proveniência:(De \textunderscore hemo...\textunderscore  + \textunderscore dinamómetro\textunderscore )}
\end{itemize}
Instrumento manométrico, para medir a pressão ou a fôrça, com que o sangue circula nos vasos do organismo.
\section{Hemodoráceas}
\begin{itemize}
\item {Grp. gram.:f. pl.}
\end{itemize}
\begin{itemize}
\item {Proveniência:(Do gr. \textunderscore haimodoron\textunderscore , n. de uma planta)}
\end{itemize}
Família de plantas monocotyledóneas da Austrália.
\section{Hemodynamómetro}
\begin{itemize}
\item {Grp. gram.:m.}
\end{itemize}
\begin{itemize}
\item {Proveniência:(De \textunderscore hemo...\textunderscore  + \textunderscore dynamómetro\textunderscore )}
\end{itemize}
Instrumento manométrico, para medir a pressão ou a fôrça, com que o sangue circula nos vasos do organismo.
\section{Hemofilia}
\begin{itemize}
\item {Grp. gram.:f.}
\end{itemize}
\begin{itemize}
\item {Proveniência:(Do gr. \textunderscore haima\textunderscore  + \textunderscore philos\textunderscore )}
\end{itemize}
Alteração no sangue, em virtude da qual é difícil estancar o que sái pelo naríz ou por qualquer ferimento.
\section{Hemófobo}
\textunderscore m.\textunderscore  (e der.)
O mesmo que \textunderscore hematófobo\textunderscore , etc.
\section{Hemoftalmia}
\begin{itemize}
\item {Grp. gram.:f.}
\end{itemize}
(Termo, indevidamente usado pelos dicionaristas, em vez de \textunderscore hemoftalmo\textunderscore )
\section{Hemoftalmo}
\begin{itemize}
\item {Grp. gram.:m.}
\end{itemize}
\begin{itemize}
\item {Proveniência:(Do gr. \textunderscore haima\textunderscore  + \textunderscore ophthalmos\textunderscore )}
\end{itemize}
Derramamento de sangue no ôlho.
\section{Hemoglobina}
\begin{itemize}
\item {Grp. gram.:f.}
\end{itemize}
\begin{itemize}
\item {Proveniência:(De \textunderscore hemo...\textunderscore  + \textunderscore globo\textunderscore )}
\end{itemize}
Substância, que constitue os 9/10 do pêso dos princípios fixos dos glóbulos sanguíneos.
\section{Hemoglobinemia}
\begin{itemize}
\item {Grp. gram.:f.}
\end{itemize}
\begin{itemize}
\item {Proveniência:(De \textunderscore hemoglobina\textunderscore  + gr. \textunderscore haima\textunderscore )}
\end{itemize}
Infecção microbiana do boi e de outros animaes, que se manifesta por urina sangrenta.
\section{Hemoglobinuria}
\begin{itemize}
\item {Grp. gram.:f.}
\end{itemize}
\begin{itemize}
\item {Proveniência:(De \textunderscore hemoglobina\textunderscore  + gr. \textunderscore ouron\textunderscore )}
\end{itemize}
Urina vermelho-escura, não acompanhada de glóbulos sanguíneos, que occorre com várias doenças, como a escarlatina, a diphteria, a febre typhoide, etc. Cf. Verg. Machado, \textunderscore Urosemiologia\textunderscore .
\section{Hemómetro}
\begin{itemize}
\item {Grp. gram.:m.}
\end{itemize}
\begin{itemize}
\item {Proveniência:(Do gr. \textunderscore haima\textunderscore  + \textunderscore metron\textunderscore )}
\end{itemize}
Apparelho, para medir sangue.
\section{Hemopathia}
\begin{itemize}
\item {Grp. gram.:f.}
\end{itemize}
\begin{itemize}
\item {Proveniência:(Do gr. \textunderscore haima\textunderscore  + \textunderscore pathos\textunderscore )}
\end{itemize}
Doença do sangue, em geral.
\section{Hemopathologia}
\begin{itemize}
\item {Grp. gram.:f.}
\end{itemize}
\begin{itemize}
\item {Proveniência:(Do gr. \textunderscore haima\textunderscore  + \textunderscore pathos\textunderscore  + \textunderscore logos\textunderscore )}
\end{itemize}
Estudo das doenças do sangue.
\section{Hemopatia}
\begin{itemize}
\item {Grp. gram.:f.}
\end{itemize}
\begin{itemize}
\item {Proveniência:(Do gr. \textunderscore haima\textunderscore  + \textunderscore pathos\textunderscore )}
\end{itemize}
Doença do sangue, em geral.
\section{Hemopatologia}
\begin{itemize}
\item {Grp. gram.:f.}
\end{itemize}
\begin{itemize}
\item {Proveniência:(Do gr. \textunderscore haima\textunderscore  + \textunderscore pathos\textunderscore  + \textunderscore logos\textunderscore )}
\end{itemize}
Estudo das doenças do sangue.
\section{Hemophilia}
\begin{itemize}
\item {Grp. gram.:f.}
\end{itemize}
\begin{itemize}
\item {Proveniência:(Do gr. \textunderscore haima\textunderscore  + \textunderscore philos\textunderscore )}
\end{itemize}
Alteração no sangue, em virtude da qual é diffícil estancar o que sái pelo naríz ou por qualquer ferimento.
\section{Hemóphobo}
\textunderscore m.\textunderscore  (e der.)
O mesmo que \textunderscore hematóphobo\textunderscore , etc.
\section{Hemophthalmia}
\begin{itemize}
\item {Grp. gram.:f.}
\end{itemize}
(Termo, indevidamente usado pelos diccionaristas, em vez de \textunderscore hemophthalmo\textunderscore )
\section{Hemophthalmo}
\begin{itemize}
\item {Grp. gram.:m.}
\end{itemize}
\begin{itemize}
\item {Proveniência:(Do gr. \textunderscore haima\textunderscore  + \textunderscore ophthalmos\textunderscore )}
\end{itemize}
Derramamento de sangue no ôlho.
\section{Hemoplania}
\begin{itemize}
\item {Grp. gram.:f.}
\end{itemize}
\begin{itemize}
\item {Proveniência:(Do gr. \textunderscore haima\textunderscore  + \textunderscore plane\textunderscore )}
\end{itemize}
Gênero de doenças, que comprehende as hemorrhagias supplementares.
\section{Hemoplástico}
\begin{itemize}
\item {Grp. gram.:adj.}
\end{itemize}
\begin{itemize}
\item {Proveniência:(De \textunderscore hemo...\textunderscore  + \textunderscore plástico\textunderscore )}
\end{itemize}
Diz-se dos alimentos, que concorrem rapidamente para a producção do sangue.
\section{Hemopoése}
\begin{itemize}
\item {Grp. gram.:f.}
\end{itemize}
O mesmo que \textunderscore hematopoése\textunderscore .
\section{Hemopoético}
\begin{itemize}
\item {Grp. gram.:adj.}
\end{itemize}
Relativo á hemopoese.
\section{Hemóptico}
\begin{itemize}
\item {Grp. gram.:adj.}
\end{itemize}
\begin{itemize}
\item {Proveniência:(Gr. \textunderscore haimoptikos\textunderscore )}
\end{itemize}
Relativo á hemoptise.
Que está atacado de hemoptise.
\section{Hemoptise}
\begin{itemize}
\item {Grp. gram.:f.}
\end{itemize}
\begin{itemize}
\item {Proveniência:(Lat. \textunderscore haemoptysis\textunderscore )}
\end{itemize}
Hemorragia da membrana mucosa do pulmão.
Expectoração de sangue.
\section{Hemoptoico}
\begin{itemize}
\item {Grp. gram.:m.}
\end{itemize}
\begin{itemize}
\item {Proveniência:(Lat. \textunderscore haemoptoicus\textunderscore )}
\end{itemize}
Aquelle que escarra sangue.
\section{Hemóptyco}
\begin{itemize}
\item {Grp. gram.:adj.}
\end{itemize}
\begin{itemize}
\item {Proveniência:(Gr. \textunderscore haimoptikos\textunderscore )}
\end{itemize}
Relativo á hemoptyse.
Que está atacado de hemoptyse.
\section{Hemoptyse}
\begin{itemize}
\item {Grp. gram.:f.}
\end{itemize}
\begin{itemize}
\item {Proveniência:(Lat. \textunderscore haemoptysis\textunderscore )}
\end{itemize}
Hemorrhagia da membrana mucosa do pulmão.
Expectoração de sangue.
\section{Hemorragia}
\begin{itemize}
\item {Grp. gram.:f.}
\end{itemize}
\begin{itemize}
\item {Proveniência:(Lat. \textunderscore haemorrhagia\textunderscore )}
\end{itemize}
Derramamento de sangue para fóra dos vasos que o devem conter.
\section{Hemorragíaco}
\begin{itemize}
\item {Grp. gram.:adj.}
\end{itemize}
Relativo á hemorragia.
Que padece hemorragia.
\section{Hemorrágico}
\begin{itemize}
\item {Grp. gram.:adj.}
\end{itemize}
Relativo á hemorragia.
Que padece hemorragia.
\section{Hemorrhagia}
\begin{itemize}
\item {Grp. gram.:f.}
\end{itemize}
\begin{itemize}
\item {Proveniência:(Lat. \textunderscore haemorrhagia\textunderscore )}
\end{itemize}
Derramamento de sangue para fóra dos vasos que o devem conter.
\section{Hemorrhagíaco}
\begin{itemize}
\item {Grp. gram.:adj.}
\end{itemize}
Relativo á hemorrhagia.
Que padece hemorrhagia.
\section{Hemorrhágico}
\begin{itemize}
\item {Grp. gram.:adj.}
\end{itemize}
Relativo á hemorrhagia.
Que padece hemorrhagia.
\section{Hemorrhoidal}
\begin{itemize}
\item {Grp. gram.:adj.}
\end{itemize}
\begin{itemize}
\item {Grp. gram.:M.}
\end{itemize}
\begin{itemize}
\item {Utilização:Pop.}
\end{itemize}
Relativo ás hemorrhoidas.
Manifestações mórbidas das hemorrhoidas: \textunderscore soffre muito de hemorrhoidal\textunderscore .
\section{Hemorrhoidaria}
\begin{itemize}
\item {Grp. gram.:f.}
\end{itemize}
\begin{itemize}
\item {Utilização:Fam.}
\end{itemize}
Ataque de hemorrhoidas. Cf. Camillo, \textunderscore Narcót.\textunderscore , I, 266.
\section{Hemorrhoidário}
\begin{itemize}
\item {Grp. gram.:adj.}
\end{itemize}
\begin{itemize}
\item {Grp. gram.:M.}
\end{itemize}
Relativo a hemorrhoidas.
Aquelle que padece hemorrhoidas.
\section{Hemorrhoidas}
\begin{itemize}
\item {Grp. gram.:f. pl.}
\end{itemize}
\begin{itemize}
\item {Proveniência:(Lat. \textunderscore haemorrhoida\textunderscore )}
\end{itemize}
Doença, que consiste principalmente em tumores nas veias do ânus.
\section{Hemorrhoidoso}
\begin{itemize}
\item {Grp. gram.:m. e adj.}
\end{itemize}
O mesmo que \textunderscore hemorrhoidário\textunderscore .
\section{Hemorrhoíssa}
\begin{itemize}
\item {Grp. gram.:f.}
\end{itemize}
\begin{itemize}
\item {Proveniência:(Lat. \textunderscore haemorrhoissa\textunderscore )}
\end{itemize}
Mulher, que tinha fluxo de sangue e que se curou ao tocar a túnica de Christo, segundo o \textunderscore Evangelho\textunderscore .
\section{Hemorroidal}
\begin{itemize}
\item {Grp. gram.:adj.}
\end{itemize}
\begin{itemize}
\item {Grp. gram.:M.}
\end{itemize}
\begin{itemize}
\item {Utilização:Pop.}
\end{itemize}
Relativo ás hemorroidas.
Manifestações mórbidas das hemorroidas: \textunderscore sofre muito de hemorroidal\textunderscore .
\section{Hemorroidaria}
\begin{itemize}
\item {Grp. gram.:f.}
\end{itemize}
\begin{itemize}
\item {Utilização:Fam.}
\end{itemize}
Ataque de hemorroidas. Cf. Camillo, \textunderscore Narcót.\textunderscore , I, 266.
\section{Hemorroidário}
\begin{itemize}
\item {Grp. gram.:adj.}
\end{itemize}
\begin{itemize}
\item {Grp. gram.:M.}
\end{itemize}
Relativo a hemorroidas.
Aquele que padece hemorroidas.
\section{Hemorroidas}
\begin{itemize}
\item {Grp. gram.:f. pl.}
\end{itemize}
\begin{itemize}
\item {Proveniência:(Lat. \textunderscore haemorrhoida\textunderscore )}
\end{itemize}
Doença, que consiste principalmente em tumores nas veias do ânus.
\section{Hemorroidoso}
\begin{itemize}
\item {Grp. gram.:m. e adj.}
\end{itemize}
O mesmo que \textunderscore hemorroidário\textunderscore .
\section{Hemospasia}
\begin{itemize}
\item {Grp. gram.:f.}
\end{itemize}
\begin{itemize}
\item {Proveniência:(Do gr. \textunderscore haima\textunderscore  + \textunderscore spasis\textunderscore )}
\end{itemize}
Meio therapêutico, com que, formando-se vácuo na superfície do corpo, se faz aí affluir o sangue.
\section{Hemospásico}
\begin{itemize}
\item {Grp. gram.:adj.}
\end{itemize}
Relativo á hemospasia.
\section{Hemóstase}
\begin{itemize}
\item {Grp. gram.:f.}
\end{itemize}
\begin{itemize}
\item {Proveniência:(Gr. \textunderscore haimostasis\textunderscore )}
\end{itemize}
Estagnação de sangue pela plethora.
Operação cirúrgica, para sustar um derramamento sanguíneo.
\section{Hemostasia}
\begin{itemize}
\item {Grp. gram.:f.}
\end{itemize}
(V.hemóstase)
\section{Hemostático}
\begin{itemize}
\item {Grp. gram.:adj.}
\end{itemize}
\begin{itemize}
\item {Grp. gram.:M.}
\end{itemize}
\begin{itemize}
\item {Proveniência:(Gr. \textunderscore haimostatikos\textunderscore )}
\end{itemize}
Relativo á hemóstase.
Medicamento contra as hemorrhagias.
\section{Hemotermes}
\begin{itemize}
\item {Grp. gram.:m. pl.}
\end{itemize}
\begin{itemize}
\item {Utilização:Zool.}
\end{itemize}
\begin{itemize}
\item {Proveniência:(Do gr. \textunderscore haima\textunderscore  + \textunderscore terme\textunderscore )}
\end{itemize}
Animaes de sangue quente.
\section{Hemotexia}
\begin{itemize}
\item {fónica:csi}
\end{itemize}
\begin{itemize}
\item {Grp. gram.:f.}
\end{itemize}
\begin{itemize}
\item {Proveniência:(Do gr. \textunderscore haima\textunderscore  + \textunderscore texis\textunderscore )}
\end{itemize}
Dissolução do sangue.
\section{Hemothermes}
\begin{itemize}
\item {Grp. gram.:m. pl.}
\end{itemize}
\begin{itemize}
\item {Utilização:Zool.}
\end{itemize}
\begin{itemize}
\item {Proveniência:(Do gr. \textunderscore haima\textunderscore  + \textunderscore terme\textunderscore )}
\end{itemize}
Animaes de sangue quente.
\section{Hemothórax}
\begin{itemize}
\item {Grp. gram.:m.}
\end{itemize}
\begin{itemize}
\item {Proveniência:(De \textunderscore hemo...\textunderscore  + \textunderscore thórax\textunderscore )}
\end{itemize}
Derramamento do sangue no thórax.
\section{Hendecafilo}
\begin{itemize}
\item {Grp. gram.:adj.}
\end{itemize}
\begin{itemize}
\item {Utilização:Bot.}
\end{itemize}
\begin{itemize}
\item {Proveniência:(Do gr. \textunderscore hendeka\textunderscore  + \textunderscore phullon\textunderscore )}
\end{itemize}
Cujas fôlhas são compostas de onze folíolos.
\section{Hendecágino}
\begin{itemize}
\item {Grp. gram.:adj.}
\end{itemize}
\begin{itemize}
\item {Utilização:Bot.}
\end{itemize}
\begin{itemize}
\item {Proveniência:(Do gr. \textunderscore hendeka\textunderscore  + \textunderscore gune\textunderscore )}
\end{itemize}
Que tem onze pistilos.
\section{Hendecagonal}
\begin{itemize}
\item {Grp. gram.:adj.}
\end{itemize}
\begin{itemize}
\item {Proveniência:(De \textunderscore hendecágono\textunderscore )}
\end{itemize}
Que tem onze ângulos.
\section{Hendecágono}
\begin{itemize}
\item {Grp. gram.:m.}
\end{itemize}
\begin{itemize}
\item {Grp. gram.:Adj.}
\end{itemize}
\begin{itemize}
\item {Proveniência:(Do gr. \textunderscore hendeka\textunderscore  + \textunderscore gonos\textunderscore )}
\end{itemize}
Polygono de onze lados.
Que tem onze ângulos e onze lados.
\section{Hendecágyno}
\begin{itemize}
\item {Grp. gram.:adj.}
\end{itemize}
\begin{itemize}
\item {Utilização:Bot.}
\end{itemize}
\begin{itemize}
\item {Proveniência:(Do gr. \textunderscore hendeka\textunderscore  + \textunderscore gune\textunderscore )}
\end{itemize}
Que tem onze pistillos.
\section{Hendecandria}
\begin{itemize}
\item {Grp. gram.:f.}
\end{itemize}
\begin{itemize}
\item {Utilização:Bot.}
\end{itemize}
\begin{itemize}
\item {Proveniência:(De \textunderscore hendecandro\textunderscore )}
\end{itemize}
Classe de vegetaes, a qual, segundo o systema de Linneu, comprehende as plantas de onze estames.
\section{Hendecandro}
\begin{itemize}
\item {Grp. gram.:adj.}
\end{itemize}
\begin{itemize}
\item {Utilização:Bot.}
\end{itemize}
\begin{itemize}
\item {Proveniência:(Do gr. \textunderscore hendeka\textunderscore  + \textunderscore aner\textunderscore , \textunderscore andros\textunderscore )}
\end{itemize}
Que tem onze estames.
\section{Hendecaphyllo}
\begin{itemize}
\item {Grp. gram.:adj.}
\end{itemize}
\begin{itemize}
\item {Utilização:Bot.}
\end{itemize}
\begin{itemize}
\item {Proveniência:(Do gr. \textunderscore hendeka\textunderscore  + \textunderscore phullon\textunderscore )}
\end{itemize}
Cujas fôlhas são compostas de onze folíolos.
\section{Hendecassilábico}
\begin{itemize}
\item {Grp. gram.:adj.}
\end{itemize}
\begin{itemize}
\item {Proveniência:(De \textunderscore hendecassílabo\textunderscore )}
\end{itemize}
Que tem onze sílabas.
\section{Hendecassílabo}
\begin{itemize}
\item {Grp. gram.:adj.}
\end{itemize}
\begin{itemize}
\item {Grp. gram.:M.}
\end{itemize}
\begin{itemize}
\item {Proveniência:(Do gr. \textunderscore hendeka\textunderscore  + \textunderscore sullabe\textunderscore )}
\end{itemize}
Que tem onze sílabas.
Verso de onze sílabas.--Alguns, á imitação dos Italianos, chamam hendecassílabos os versos dos \textunderscore Lusíadas\textunderscore , os versos heroicos, porque, quando \textunderscore inteiros\textunderscore  ou \textunderscore graves\textunderscore , contam neles uma sílaba, além do último acento tónico. Castilho porém vulgarizou sensatamente o sistema francês de se contarem as sílabas do verso até ao último acento tónico. E, assim, segundo êste sistema, os versos dos \textunderscore Lusíadas\textunderscore , ou os \textunderscore heroicos\textunderscore , são decassílabos.
\section{Hendecasyllábico}
\begin{itemize}
\item {fónica:si}
\end{itemize}
\begin{itemize}
\item {Grp. gram.:adj.}
\end{itemize}
\begin{itemize}
\item {Proveniência:(De \textunderscore hendecassýllabo\textunderscore )}
\end{itemize}
Que tem onze sýllabas.
\section{Hendecassýllabo}
\begin{itemize}
\item {fónica:si}
\end{itemize}
\begin{itemize}
\item {Grp. gram.:adj.}
\end{itemize}
\begin{itemize}
\item {Grp. gram.:M.}
\end{itemize}
\begin{itemize}
\item {Proveniência:(Do gr. \textunderscore hendeka\textunderscore  + \textunderscore sullabe\textunderscore )}
\end{itemize}
Que tem onze sýllabas.
Verso de onze sýllabas.--Alguns, á imitação dos Italianos, chamam hendecassýllabos os versos dos \textunderscore Lusíadas\textunderscore , os versos heroicos, porque, quando \textunderscore inteiros\textunderscore  ou \textunderscore graves\textunderscore , contam nelles uma sýllaba, além do último accento tónico. Castilho porém vulgarizou sensatamente o systema francês de se contarem as sýllabas do verso até ao último accento tónico. E, assim, segundo êste systema, os versos dos \textunderscore Lusíadas\textunderscore , ou os \textunderscore heroicos\textunderscore , são decassýllabos.
\section{Hena}
\begin{itemize}
\item {Grp. gram.:f.}
\end{itemize}
\begin{itemize}
\item {Proveniência:(Fr. \textunderscore hennè\textunderscore , ingl. \textunderscore henna\textunderscore )}
\end{itemize}
Planta da Índia portuguesa, cultivada nos jardins, (\textunderscore lawsonia alba\textunderscore , Lamk.).
\section{Henequém}
\begin{itemize}
\item {Grp. gram.:m.}
\end{itemize}
Planta de fibras têxteis, explorada no commércio mexicano. Cf. \textunderscore Jorn. do Comm.\textunderscore , do Rio, de 21-VII-901.
\section{Henna}
\begin{itemize}
\item {Grp. gram.:f.}
\end{itemize}
\begin{itemize}
\item {Proveniência:(Fr. \textunderscore hennè\textunderscore , ingl. \textunderscore henna\textunderscore )}
\end{itemize}
Planta da Índia portuguesa, cultivada nos jardins, (\textunderscore lawsonia alba\textunderscore , Lamk.).
\section{Henótico}
\begin{itemize}
\item {Grp. gram.:m.}
\end{itemize}
\begin{itemize}
\item {Proveniência:(Gr. \textunderscore henotikos\textunderscore )}
\end{itemize}
Decreto, com que o Imperador Zenão, (482), procurou estabelecer a unidade da crença religiosa no Império Romano do Oriente.
\section{Henriquenho}
\begin{itemize}
\item {Grp. gram.:adj.}
\end{itemize}
\begin{itemize}
\item {Utilização:Ant.}
\end{itemize}
\begin{itemize}
\item {Proveniência:(De \textunderscore Henrique\textunderscore , n. p.)}
\end{itemize}
Relativo a Henrique, especialmente ao infante D. Henrique, iniciador dos descobrimentos portugueses.
\section{Henriquino}
\begin{itemize}
\item {Grp. gram.:adj.}
\end{itemize}
\begin{itemize}
\item {Proveniência:(De \textunderscore Henrique\textunderscore , n. p.)}
\end{itemize}
Relativo a Henrique, especialmente ao infante D. Henrique, iniciador dos descobrimentos portugueses.
\section{Hépar}
\begin{itemize}
\item {Grp. gram.:m.}
\end{itemize}
\begin{itemize}
\item {Proveniência:(Lat. \textunderscore hepar\textunderscore , fígado)}
\end{itemize}
Nome, que os chímicos antigos davam aos sulfuretos.
Nome de um peixe, espécie de lagosta.
\section{Hepatal}
\begin{itemize}
\item {Grp. gram.:adj.}
\end{itemize}
\begin{itemize}
\item {Proveniência:(Do lat. \textunderscore hepar\textunderscore , \textunderscore hepatis\textunderscore )}
\end{itemize}
Relativo ao fígado.
\section{Hepatalgia}
\begin{itemize}
\item {Grp. gram.:f.}
\end{itemize}
\begin{itemize}
\item {Proveniência:(Do gr. \textunderscore hepar\textunderscore , \textunderscore hepatos\textunderscore  + \textunderscore algos\textunderscore )}
\end{itemize}
Dôr neurálgica do fígado.
\section{Hepatálgico}
\begin{itemize}
\item {Grp. gram.:adj.}
\end{itemize}
Relativo á hepatalgia.
\section{Hepática}
\begin{itemize}
\item {Grp. gram.:f.}
\end{itemize}
\begin{itemize}
\item {Grp. gram.:Pl.}
\end{itemize}
\begin{itemize}
\item {Proveniência:(De \textunderscore hepático\textunderscore )}
\end{itemize}
Planta medicinal, que foi recommendada contra doenças do fígado.
Família de plantas acotyledóneas, que contém pequenas espécies herbáceas, trepadeiras e parasitas.
\section{Hepático}
\begin{itemize}
\item {Grp. gram.:adj.}
\end{itemize}
\begin{itemize}
\item {Proveniência:(Lat. \textunderscore hepaticus\textunderscore )}
\end{itemize}
Relativo ao fígado.
Que tem côr de fígado, (falando-se de plantas).
\section{Hepatite}
\begin{itemize}
\item {Grp. gram.:f.}
\end{itemize}
\begin{itemize}
\item {Proveniência:(Lat. \textunderscore hepatitis\textunderscore )}
\end{itemize}
Inflammação do fígado.
Pedra preciosa, da côr do fígado.
\section{Hepatização}
\begin{itemize}
\item {Grp. gram.:f.}
\end{itemize}
\begin{itemize}
\item {Proveniência:(De \textunderscore hepatizar\textunderscore )}
\end{itemize}
Passagem de um tecido orgânico a um estado, em que apresenta o aspecto de fígado.
\section{Hepatizar-se}
\begin{itemize}
\item {Grp. gram.:v. p.}
\end{itemize}
\begin{itemize}
\item {Proveniência:(Do gr. \textunderscore hepar\textunderscore , \textunderscore hepatos\textunderscore )}
\end{itemize}
Tomar o aspecto de fígado, (falando-se de um tecido orgânico):«\textunderscore olha que ás vezes o pulmão hepatiza-se\textunderscore ». Camillo, \textunderscore Mulher Fatal\textunderscore , 109.
\section{Hépato}
\begin{itemize}
\item {Grp. gram.:m.}
\end{itemize}
\begin{itemize}
\item {Proveniência:(Do gr. \textunderscore hepar\textunderscore , \textunderscore hepatos\textunderscore )}
\end{itemize}
Grande peixe marítimo, da côr do fígado humano; o mesmo que \textunderscore hépar\textunderscore ?
\section{Hepato...}
\begin{itemize}
\item {Proveniência:(Do gr. \textunderscore hepar\textunderscore , \textunderscore hepatos\textunderscore )}
\end{itemize}
Elemento, que entra na composição de várias palavras, com a significação de \textunderscore fígado\textunderscore  ou de \textunderscore relativo ao fígado\textunderscore .
\section{Hepatocele}
\begin{itemize}
\item {Grp. gram.:m.}
\end{itemize}
\begin{itemize}
\item {Proveniência:(Do gr. \textunderscore hepar\textunderscore  + \textunderscore kele\textunderscore )}
\end{itemize}
Hérnia do fígado.
\section{Hepatocístico}
\begin{itemize}
\item {Grp. gram.:adj.}
\end{itemize}
\begin{itemize}
\item {Proveniência:(Do gr. \textunderscore hepar\textunderscore  + \textunderscore kustis\textunderscore )}
\end{itemize}
Relativo ao fígado e á vesícula do fel.
\section{Hepatocýstico}
\begin{itemize}
\item {Grp. gram.:adj.}
\end{itemize}
\begin{itemize}
\item {Proveniência:(Do gr. \textunderscore hepar\textunderscore  + \textunderscore kustis\textunderscore )}
\end{itemize}
Relativo ao fígado e á vesícula do fel.
\section{Hepatogástrico}
\begin{itemize}
\item {Grp. gram.:adj.}
\end{itemize}
\begin{itemize}
\item {Proveniência:(De \textunderscore hepato...\textunderscore  + \textunderscore gástrico\textunderscore )}
\end{itemize}
Relativo ao fígado e ao estômago.
\section{Hepatogastrite}
\begin{itemize}
\item {Grp. gram.:f.}
\end{itemize}
Inflammação do fígado e do estômago.
\section{Hepatografia}
\begin{itemize}
\item {Grp. gram.:f.}
\end{itemize}
\begin{itemize}
\item {Proveniência:(Do gr. \textunderscore hepar\textunderscore , \textunderscore hepatos\textunderscore  + \textunderscore graphein\textunderscore )}
\end{itemize}
Descripção científica do fígado.
\section{Hepatographia}
\begin{itemize}
\item {Grp. gram.:f.}
\end{itemize}
\begin{itemize}
\item {Proveniência:(Do gr. \textunderscore hepar\textunderscore , \textunderscore hepatos\textunderscore  + \textunderscore graphein\textunderscore )}
\end{itemize}
Descripção scientífica do fígado.
\section{Hepato-intestinal}
\begin{itemize}
\item {Grp. gram.:adj.}
\end{itemize}
Relativo ao fígado e aos intestinos.
\section{Hepatologia}
\begin{itemize}
\item {Grp. gram.:f.}
\end{itemize}
\begin{itemize}
\item {Proveniência:(Do gr. \textunderscore hepar\textunderscore  + \textunderscore logos\textunderscore )}
\end{itemize}
Tratado á cêrca do fígado.
\section{Hepatopse}
\begin{itemize}
\item {Grp. gram.:f.}
\end{itemize}
\begin{itemize}
\item {Utilização:Med.}
\end{itemize}
\begin{itemize}
\item {Proveniência:(Do gr. \textunderscore hepar\textunderscore , \textunderscore hepatos\textunderscore  + \textunderscore ptosis\textunderscore )}
\end{itemize}
Quéda ou prolapso do fígado.
Mobilidade anómala do fígado.
\section{Hepatorreia}
\begin{itemize}
\item {Grp. gram.:f.}
\end{itemize}
\begin{itemize}
\item {Proveniência:(Do gr. \textunderscore hepar\textunderscore , \textunderscore hepatos\textunderscore  + \textunderscore rhein\textunderscore )}
\end{itemize}
Abundante dejeção de matérias, formadas principalmente de bílis.
\section{Hepatorrheia}
\begin{itemize}
\item {Grp. gram.:f.}
\end{itemize}
\begin{itemize}
\item {Proveniência:(Do gr. \textunderscore hepar\textunderscore , \textunderscore hepatos\textunderscore  + \textunderscore rhein\textunderscore )}
\end{itemize}
Abundante dejecção de matérias, formadas principalmente de bílis.
\section{Hepatoscopia}
\begin{itemize}
\item {Grp. gram.:f.}
\end{itemize}
\begin{itemize}
\item {Proveniência:(Do gr. \textunderscore hepar\textunderscore , \textunderscore hepatos\textunderscore  + \textunderscore skopein\textunderscore )}
\end{itemize}
Supposta arte de adivinhar, exercida pelos antigos, por meio da inspecção do fígado das víctimas.
\section{Hepatotomia}
\begin{itemize}
\item {Grp. gram.:f.}
\end{itemize}
\begin{itemize}
\item {Proveniência:(Do gr. \textunderscore hepar\textunderscore , \textunderscore hepatos\textunderscore  + \textunderscore tome\textunderscore )}
\end{itemize}
Dissecção do fígado.
\section{Hepe!}
\begin{itemize}
\item {Grp. gram.:interj.}
\end{itemize}
\begin{itemize}
\item {Utilização:Bras}
\end{itemize}
Emprega-se para estimular os animaes na marcha.
\section{Hepiálidos}
\begin{itemize}
\item {Grp. gram.:m. pl.}
\end{itemize}
Tríbo de insectos, que têm por typo o hepíalo.
\section{Hepíalo}
\begin{itemize}
\item {Grp. gram.:m.}
\end{itemize}
\begin{itemize}
\item {Proveniência:(Gr. \textunderscore hepialos\textunderscore )}
\end{itemize}
Gênero de insectos lepidópteros nocturnos.
\section{Hepta...}
\begin{itemize}
\item {Grp. gram.:pref.}
\end{itemize}
\begin{itemize}
\item {Proveniência:(Gr. \textunderscore hepta\textunderscore )}
\end{itemize}
(designativo de \textunderscore sete\textunderscore )
\section{Heptacórdio}
\begin{itemize}
\item {Grp. gram.:adj.}
\end{itemize}
\begin{itemize}
\item {Grp. gram.:M.}
\end{itemize}
\begin{itemize}
\item {Proveniência:(De \textunderscore hepta...\textunderscore  + \textunderscore corda\textunderscore )}
\end{itemize}
Que tem sete cordas.
Cíthara de sete cordas.
Systema de sons, composto de sete notas.
\section{Heptadáctilo}
\begin{itemize}
\item {Grp. gram.:adj.}
\end{itemize}
\begin{itemize}
\item {Proveniência:(Do gr. \textunderscore hepta\textunderscore  + \textunderscore daktulos\textunderscore )}
\end{itemize}
Que tem sete dedos.
\section{Heptadáctylo}
\begin{itemize}
\item {Grp. gram.:adj.}
\end{itemize}
\begin{itemize}
\item {Proveniência:(Do gr. \textunderscore hepta\textunderscore  + \textunderscore daktulos\textunderscore )}
\end{itemize}
Que tem sete dedos.
\section{Heptaédrico}
\begin{itemize}
\item {Grp. gram.:adj.}
\end{itemize}
Relativo ao heptaédro.
\section{Heptaédro}
\begin{itemize}
\item {Grp. gram.:m.}
\end{itemize}
\begin{itemize}
\item {Proveniência:(Do gr. \textunderscore hepta\textunderscore  + \textunderscore edra\textunderscore )}
\end{itemize}
Sólido de sete faces.
\section{Heptafilo}
\begin{itemize}
\item {Grp. gram.:adj.}
\end{itemize}
\begin{itemize}
\item {Utilização:Bot.}
\end{itemize}
\begin{itemize}
\item {Proveniência:(Do gr. \textunderscore hepta\textunderscore  + \textunderscore phullon\textunderscore )}
\end{itemize}
Diz-se das fôlhas que são formadas de sete folíolos.
\section{Heptafónico}
\begin{itemize}
\item {Grp. gram.:adj.}
\end{itemize}
\begin{itemize}
\item {Proveniência:(Do gr. \textunderscore hepta\textunderscore  + \textunderscore phone\textunderscore )}
\end{itemize}
Diz-se do eco, que repete um som sete vezes.
\section{Heptaginia}
\begin{itemize}
\item {Grp. gram.:f.}
\end{itemize}
\begin{itemize}
\item {Proveniência:(De \textunderscore heptágino\textunderscore )}
\end{itemize}
Ordem de plantas, estabelecida por Linneu, para as plantas que têm sete pistilos.
\section{Heptágino}
\begin{itemize}
\item {Grp. gram.:adj.}
\end{itemize}
\begin{itemize}
\item {Utilização:Bot.}
\end{itemize}
\begin{itemize}
\item {Proveniência:(Do gr. \textunderscore hepta\textunderscore  + \textunderscore gune\textunderscore )}
\end{itemize}
Que têm sete pistilos.
\section{Heptagonal}
\begin{itemize}
\item {Grp. gram.:adj.}
\end{itemize}
Relativo ao heptágono.
\section{Heptágono}
\begin{itemize}
\item {Grp. gram.:m.}
\end{itemize}
\begin{itemize}
\item {Grp. gram.:Adj.}
\end{itemize}
\begin{itemize}
\item {Proveniência:(Do gr. \textunderscore hepta\textunderscore  + \textunderscore gonos\textunderscore )}
\end{itemize}
Polýgono de sete lados.
Fortificação de sete bastiões.
Que tem sete ângulos e sete lados.
\section{Heptagynia}
\begin{itemize}
\item {Grp. gram.:f.}
\end{itemize}
\begin{itemize}
\item {Proveniência:(De \textunderscore heptágyno\textunderscore )}
\end{itemize}
Ordem de plantas, estabelecida por Linneu, para as plantas que têm sete pistillos.
\section{Heptágyno}
\begin{itemize}
\item {Grp. gram.:adj.}
\end{itemize}
\begin{itemize}
\item {Utilização:Bot.}
\end{itemize}
\begin{itemize}
\item {Proveniência:(Do gr. \textunderscore hepta\textunderscore  + \textunderscore gune\textunderscore )}
\end{itemize}
Que têm sete pistillos.
\section{Heptâmeron}
\begin{itemize}
\item {Grp. gram.:m.}
\end{itemize}
\begin{itemize}
\item {Proveniência:(T. mal formado, em vez de \textunderscore heptaémeron\textunderscore , do gr. \textunderscore hepta\textunderscore  + \textunderscore emera\textunderscore )}
\end{itemize}
Obra literária, dividida em sete partes.
\section{Heptâmetro}
\begin{itemize}
\item {Grp. gram.:m.  e  adj.}
\end{itemize}
\begin{itemize}
\item {Proveniência:(Do gr. \textunderscore hepta\textunderscore  + \textunderscore metron\textunderscore )}
\end{itemize}
Diz-se de um verso grego ou latino que tem sete pés.
\section{Heptaminas}
\begin{itemize}
\item {Grp. gram.:f. pl.}
\end{itemize}
\begin{itemize}
\item {Utilização:Chím.}
\end{itemize}
\begin{itemize}
\item {Proveniência:(De \textunderscore hepta...\textunderscore  + \textunderscore amina\textunderscore )}
\end{itemize}
Aminas, formadas por sete moléculas de ammoníaco.
\section{Heptandria}
\begin{itemize}
\item {Grp. gram.:f.}
\end{itemize}
Qualidade de heptandro.
Conjunto dos vegetaes heptandros, que constituem uma classe no systema de Linneu.
\section{Heptandro}
\begin{itemize}
\item {Grp. gram.:adj.}
\end{itemize}
\begin{itemize}
\item {Utilização:Bot.}
\end{itemize}
\begin{itemize}
\item {Proveniência:(Do gr. \textunderscore hepta\textunderscore  + \textunderscore aner\textunderscore , \textunderscore andros\textunderscore )}
\end{itemize}
Que tem sete estames, livres entre si.
\section{Heptânemo}
\begin{itemize}
\item {Grp. gram.:adj.}
\end{itemize}
\begin{itemize}
\item {Utilização:Zool.}
\end{itemize}
\begin{itemize}
\item {Proveniência:(Do gr. \textunderscore hepta\textunderscore  + \textunderscore nema\textunderscore )}
\end{itemize}
Que tem sete tentáculos.
\section{Heptano}
\begin{itemize}
\item {Grp. gram.:m.}
\end{itemize}
\begin{itemize}
\item {Utilização:Chím.}
\end{itemize}
Variedade de carboneto do grupo formênico.
\section{Heptanterado}
\begin{itemize}
\item {Grp. gram.:adj.}
\end{itemize}
\begin{itemize}
\item {Utilização:Bot.}
\end{itemize}
\begin{itemize}
\item {Proveniência:(De \textunderscore hepta...\textunderscore  + \textunderscore anthera\textunderscore )}
\end{itemize}
Que tem sete antheras.
\section{Heptantero}
\begin{itemize}
\item {Grp. gram.:adj.}
\end{itemize}
O mesmo que \textunderscore heptanterado\textunderscore .
\section{Heptantherado}
\begin{itemize}
\item {Grp. gram.:adj.}
\end{itemize}
\begin{itemize}
\item {Utilização:Bot.}
\end{itemize}
\begin{itemize}
\item {Proveniência:(De \textunderscore hepta...\textunderscore  + \textunderscore anthera\textunderscore )}
\end{itemize}
Que tem sete antheras.
\section{Heptanthero}
\begin{itemize}
\item {Grp. gram.:adj.}
\end{itemize}
O mesmo que \textunderscore heptantherado\textunderscore .
\section{Heptapétalo}
\begin{itemize}
\item {Grp. gram.:adj.}
\end{itemize}
\begin{itemize}
\item {Utilização:Bot.}
\end{itemize}
\begin{itemize}
\item {Proveniência:(De \textunderscore hepta...\textunderscore  + \textunderscore petala\textunderscore )}
\end{itemize}
Cuja corolla se compõe de sete pétalas.
\section{Heptaphónico}
\begin{itemize}
\item {Grp. gram.:adj.}
\end{itemize}
\begin{itemize}
\item {Proveniência:(Do gr. \textunderscore hepta\textunderscore  + \textunderscore phone\textunderscore )}
\end{itemize}
Diz-se do echo, que repete um som sete vezes.
\section{Heptaphyllo}
\begin{itemize}
\item {Grp. gram.:adj.}
\end{itemize}
\begin{itemize}
\item {Utilização:Bot.}
\end{itemize}
\begin{itemize}
\item {Proveniência:(Do gr. \textunderscore hepta\textunderscore  + \textunderscore phullon\textunderscore )}
\end{itemize}
Diz-se das fôlhas que são formadas de sete folíolos.
\section{Heptarca}
\begin{itemize}
\item {Grp. gram.:f.}
\end{itemize}
\begin{itemize}
\item {Proveniência:(Do gr. \textunderscore hepta\textunderscore  + \textunderscore arkhe\textunderscore )}
\end{itemize}
Cada um dos membros de uma heptarquia.
\section{Heptarcha}
\begin{itemize}
\item {fónica:ca}
\end{itemize}
\begin{itemize}
\item {Grp. gram.:f.}
\end{itemize}
\begin{itemize}
\item {Proveniência:(Do gr. \textunderscore hepta\textunderscore  + \textunderscore arkhe\textunderscore )}
\end{itemize}
Cada um dos membros de uma heptarchia.
\section{Heptarchia}
\begin{itemize}
\item {fónica:qui}
\end{itemize}
\begin{itemize}
\item {Grp. gram.:f.}
\end{itemize}
Conjunto dos sete reinos, fundado pelos Ânglos e Saxões na Bretanha.
Govêrno, formado de sete indivíduos.
(Cp. \textunderscore heptarcha\textunderscore )
\section{Heptárchico}
\begin{itemize}
\item {fónica:qui}
\end{itemize}
\begin{itemize}
\item {Grp. gram.:adj.}
\end{itemize}
Relativo a heptarchia.
\section{Heptarquia}
\begin{itemize}
\item {Grp. gram.:f.}
\end{itemize}
Conjunto dos sete reinos, fundado pelos Ânglos e Saxões na Bretanha.
Govêrno, formado de sete indivíduos.
(Cp. \textunderscore heptarca\textunderscore )
\section{Heptárquico}
\begin{itemize}
\item {Grp. gram.:adj.}
\end{itemize}
Relativo a heptarquia.
\section{Heptasépalo}
\begin{itemize}
\item {fónica:se}
\end{itemize}
\begin{itemize}
\item {Grp. gram.:adj.}
\end{itemize}
\begin{itemize}
\item {Utilização:Bot.}
\end{itemize}
\begin{itemize}
\item {Proveniência:(De \textunderscore hepta...\textunderscore  + \textunderscore sépala\textunderscore )}
\end{itemize}
Formado de sete sépalas.
\section{Heptassépalo}
\begin{itemize}
\item {Grp. gram.:adj.}
\end{itemize}
\begin{itemize}
\item {Utilização:Bot.}
\end{itemize}
\begin{itemize}
\item {Proveniência:(De \textunderscore hepta...\textunderscore  + \textunderscore sépala\textunderscore )}
\end{itemize}
Formado de sete sépalas.
\section{Heptassílabo}
\begin{itemize}
\item {Grp. gram.:m.}
\end{itemize}
\begin{itemize}
\item {Grp. gram.:Adj.}
\end{itemize}
\begin{itemize}
\item {Proveniência:(De \textunderscore hepta\textunderscore  + \textunderscore sílaba\textunderscore )}
\end{itemize}
Verso, que tem sete sílabas.
Palavra de sete sílabas.
Diz-se do verso de sete sílabas.
\section{Heptástico}
\begin{itemize}
\item {Grp. gram.:m.}
\end{itemize}
Estrophe de sete versos.
\section{Heptasýllabo}
\begin{itemize}
\item {fónica:si}
\end{itemize}
\begin{itemize}
\item {Grp. gram.:m.}
\end{itemize}
\begin{itemize}
\item {Grp. gram.:Adj.}
\end{itemize}
\begin{itemize}
\item {Proveniência:(De \textunderscore hepta\textunderscore  + \textunderscore sýllaba\textunderscore )}
\end{itemize}
Verso, que tem sete sýllabas.
Palavra de sete sýllabas.
Diz-se do verso de sete sýllabas.
\section{Heptateuco}
\begin{itemize}
\item {Grp. gram.:m.}
\end{itemize}
\begin{itemize}
\item {Proveniência:(Do gr. \textunderscore hepta\textunderscore  + \textunderscore teukos\textunderscore )}
\end{itemize}
Obra, dividida em sete livros.
Os sete primeiros livros do \textunderscore Antigo Testamento\textunderscore , incluíndo o \textunderscore Pentateuco\textunderscore , o livro de \textunderscore Josué\textunderscore  e o livro dos \textunderscore Juízes\textunderscore .
\section{Heptátomo}
\begin{itemize}
\item {Grp. gram.:adj.}
\end{itemize}
\begin{itemize}
\item {Utilização:Zool.}
\end{itemize}
\begin{itemize}
\item {Proveniência:(Do gr. \textunderscore hepta\textunderscore  + \textunderscore tome\textunderscore )}
\end{itemize}
Que tem sete articulações.
\section{Heptemímero}
\begin{itemize}
\item {Grp. gram.:adj.}
\end{itemize}
\begin{itemize}
\item {Proveniência:(Do gr. \textunderscore hepta\textunderscore  + \textunderscore hemi\textunderscore  + \textunderscore meros\textunderscore )}
\end{itemize}
Que tem metade de sete partes, (falando-se dos versos gregos de sete sílabas, como os de Anacreonte, ou das cesuras de três pés e meio).
\section{Hepthemímero}
\begin{itemize}
\item {Grp. gram.:adj.}
\end{itemize}
\begin{itemize}
\item {Proveniência:(Do gr. \textunderscore hepta\textunderscore  + \textunderscore hemi\textunderscore  + \textunderscore meros\textunderscore )}
\end{itemize}
Que tem metade de sete partes, (falando-se dos versos gregos de sete sýllabas, como os de Anacreonte, ou das cesuras de três pés e meio).
\section{Heptílico}
\begin{itemize}
\item {Grp. gram.:adj.}
\end{itemize}
Diz-se de um dos álcooes dos vinhos, cuja fórmula química é C^{7}H^{16}O.
\section{Heptýlico}
\begin{itemize}
\item {Grp. gram.:adj.}
\end{itemize}
Diz-se de um dos álcooes dos vinhos, cuja fórmula chímica é C^{7}H^{16}O.
\section{Hera}
\begin{itemize}
\item {Grp. gram.:f.}
\end{itemize}
\begin{itemize}
\item {Proveniência:(Do lat. \textunderscore hedera\textunderscore )}
\end{itemize}
Nome de várias plantas trepadeiras, da fam. das araliáceas.
\section{Heracleias}
\begin{itemize}
\item {Grp. gram.:f. pl.}
\end{itemize}
\begin{itemize}
\item {Proveniência:(Do lat. \textunderscore heracleus\textunderscore )}
\end{itemize}
Antigas festas gregas em honra de Hércules.
\section{Heráclias}
\begin{itemize}
\item {Grp. gram.:f. pl.}
\end{itemize}
\begin{itemize}
\item {Proveniência:(Do lat. \textunderscore heraclius\textunderscore )}
\end{itemize}
O mesmo que \textunderscore heracleias\textunderscore .
\section{Heráclidas}
\begin{itemize}
\item {Grp. gram.:m. pl.}
\end{itemize}
\begin{itemize}
\item {Proveniência:(Do gr. \textunderscore Herakles\textunderscore , n. p.)}
\end{itemize}
Descendentes de Hércules ou Héracles.
Nome de várias dynastias gregas.
\section{Heraclíteo}
\begin{itemize}
\item {Grp. gram.:adj.}
\end{itemize}
Relativo a Heráclito. Cf. Latino, \textunderscore Or. da Corôa\textunderscore , LXI.
\section{Heradeira}
\begin{itemize}
\item {Grp. gram.:f.}
\end{itemize}
\begin{itemize}
\item {Utilização:Prov.}
\end{itemize}
\begin{itemize}
\item {Utilização:trasm.}
\end{itemize}
O mesmo que \textunderscore hera\textunderscore .
(Por \textunderscore hedereira\textunderscore , do lat. \textunderscore hedera\textunderscore )
\section{Hera-do-verão}
\begin{itemize}
\item {Grp. gram.:f.}
\end{itemize}
\begin{itemize}
\item {Utilização:Bras}
\end{itemize}
Trepadeira vivaz, (\textunderscore mikanea scandens\textunderscore ).
\section{Heráldica}
\begin{itemize}
\item {Grp. gram.:f.}
\end{itemize}
\begin{itemize}
\item {Proveniência:(De \textunderscore heráldico\textunderscore )}
\end{itemize}
Arte ou sciência dos brasões.
Conjunto dos emblemas de brasão.
\section{Heráldico}
\begin{itemize}
\item {Grp. gram.:adj.}
\end{itemize}
\begin{itemize}
\item {Grp. gram.:M.}
\end{itemize}
\begin{itemize}
\item {Proveniência:(De \textunderscore heraldo\textunderscore )}
\end{itemize}
Relativo a brasões.
Aquelle que é versado em heráldica.
\section{Heraldo}
\begin{itemize}
\item {Grp. gram.:m.}
\end{itemize}
\begin{itemize}
\item {Utilização:Ant.}
\end{itemize}
\begin{itemize}
\item {Proveniência:(Lat. \textunderscore heraldus\textunderscore )}
\end{itemize}
O mesmo que \textunderscore arauto\textunderscore .
\section{Herança}
\begin{itemize}
\item {Grp. gram.:f.}
\end{itemize}
Aquillo que se herda.
Aquillo que se transmitte com o sangue.
Hereditariedade.
(Corr. de \textunderscore herdança\textunderscore )
\section{Hera-terrestre}
\begin{itemize}
\item {Grp. gram.:f.}
\end{itemize}
\begin{itemize}
\item {Utilização:Bras}
\end{itemize}
Planta labiada medicinal, (\textunderscore glechoma hederacea\textunderscore , Lin.; \textunderscore nepeta glechoma\textunderscore , Benth.).
\section{Herbáceo}
\begin{itemize}
\item {Grp. gram.:adj.}
\end{itemize}
\begin{itemize}
\item {Proveniência:(Lat. \textunderscore herbaceus\textunderscore )}
\end{itemize}
Relativo a erva.
Diz-se das plantas, cujos ramos e haste não produzem madeira, e perecem, depois de alguns meses de vegetação.
\section{Herbanário}
\begin{itemize}
\item {Grp. gram.:m.}
\end{itemize}
O mesmo que \textunderscore ervanário\textunderscore . Cf. Filinto, XII, 186.
\section{Herbário}
\begin{itemize}
\item {Grp. gram.:m.}
\end{itemize}
\begin{itemize}
\item {Proveniência:(Lat. \textunderscore herbarium\textunderscore )}
\end{itemize}
Collecção de plantas, para exposição ou estudo.
\section{Herbartiano}
\begin{itemize}
\item {Grp. gram.:adj.}
\end{itemize}
\begin{itemize}
\item {Grp. gram.:M.}
\end{itemize}
Relativo ao pedagogo Herbart.
Sectário de Herbart. Cf. Ad. Coelho, in \textunderscore Boletim da Dir. Ger. da Instr. Públ.\textunderscore , I.
\section{Herbático}
\begin{itemize}
\item {Grp. gram.:adj.}
\end{itemize}
\begin{itemize}
\item {Proveniência:(Lat. \textunderscore herbaticus\textunderscore )}
\end{itemize}
Relativo a erva.
\section{Herbífero}
\begin{itemize}
\item {Grp. gram.:adj.}
\end{itemize}
\begin{itemize}
\item {Proveniência:(Lat. \textunderscore herbifer\textunderscore )}
\end{itemize}
Que produz erva.
\section{Herbiforme}
\begin{itemize}
\item {Grp. gram.:adj.}
\end{itemize}
\begin{itemize}
\item {Proveniência:(Do lat. \textunderscore herba\textunderscore  + \textunderscore forma\textunderscore )}
\end{itemize}
Que tem apparência de erva sêca.
\section{Herbívoro}
\begin{itemize}
\item {Grp. gram.:adj.}
\end{itemize}
\begin{itemize}
\item {Grp. gram.:M.}
\end{itemize}
\begin{itemize}
\item {Proveniência:(Lat. \textunderscore herbivorus\textunderscore )}
\end{itemize}
Que se alimenta de vegetaes.
Animal que se alimenta de vegetaes.
\section{Herbolária}
\begin{itemize}
\item {Grp. gram.:f.}
\end{itemize}
\begin{itemize}
\item {Proveniência:(De \textunderscore herbolário\textunderscore )}
\end{itemize}
Mulher, que fazia feitiços ou preparava venenos, com vegetaes.
\section{Herbolário}
\begin{itemize}
\item {Grp. gram.:m.  e  adj.}
\end{itemize}
O que faz collecção de plantas.
Aquelle que é conhecedor de plantas medicinaes; ervanário.
(Cp. lat. \textunderscore herbula\textunderscore )
\section{Herbóreo}
\begin{itemize}
\item {Grp. gram.:adj.}
\end{itemize}
O mesmo que \textunderscore herbático\textunderscore .
\section{Herborista}
\begin{itemize}
\item {Grp. gram.:m.}
\end{itemize}
\begin{itemize}
\item {Proveniência:(Do rad. do lat. \textunderscore herba\textunderscore )}
\end{itemize}
Aquelle que herboriza.
\section{Herborização}
\begin{itemize}
\item {Grp. gram.:f.}
\end{itemize}
Acto ou effeito de herborizar.
\section{Herborizador}
\begin{itemize}
\item {Grp. gram.:m.  e  adj.}
\end{itemize}
O que herboriza.
\section{Herborizante}
\begin{itemize}
\item {Grp. gram.:adj.}
\end{itemize}
Que herboriza. Cf. Garrett, \textunderscore Helena\textunderscore , 97.
\section{Herborizar}
\begin{itemize}
\item {Grp. gram.:v. i.}
\end{itemize}
\begin{itemize}
\item {Proveniência:(Do lat. \textunderscore herba\textunderscore . T. mal formado, por confusão do sentido e da fórma de \textunderscore herba\textunderscore  e \textunderscore arbor\textunderscore )}
\end{itemize}
Colleccionar plantas, para estudo ou para applicações medicinaes.
\section{Herboso}
\begin{itemize}
\item {Grp. gram.:adj.}
\end{itemize}
O mesmo que \textunderscore ervoso\textunderscore .
\section{Herciniano}
\begin{itemize}
\item {Grp. gram.:adj.}
\end{itemize}
Relativo á Floresta Hercínia (na Alemanha).
Diz-se de uma época geológica.
\section{Hercotectónica}
\begin{itemize}
\item {Grp. gram.:f.}
\end{itemize}
\begin{itemize}
\item {Proveniência:(Do gr. \textunderscore herkos\textunderscore  + \textunderscore tektonike\textunderscore )}
\end{itemize}
Arte de fortificar praças.
\section{Herculano}
\begin{itemize}
\item {Grp. gram.:adj.}
\end{itemize}
Relativo a Hércules. Cf. \textunderscore Lusíadas\textunderscore , IX, 21.
\section{Hercúleo}
\begin{itemize}
\item {Grp. gram.:adj.}
\end{itemize}
\begin{itemize}
\item {Utilização:Fig.}
\end{itemize}
\begin{itemize}
\item {Proveniência:(De \textunderscore Hércules\textunderscore , n. p.)}
\end{itemize}
Valente; robusto; que tem fôrça extraordinária.
\section{Hércules}
\begin{itemize}
\item {Grp. gram.:m.}
\end{itemize}
\begin{itemize}
\item {Utilização:Fig.}
\end{itemize}
\begin{itemize}
\item {Proveniência:(De \textunderscore Hércules\textunderscore , n. p.)}
\end{itemize}
Homem de fôrça extraordinária.
Constellação boreal.
\section{Hercyniano}
\begin{itemize}
\item {Grp. gram.:adj.}
\end{itemize}
Relativo á Floresta Hercýnia (na Alemanha).
Diz-se de uma época geológica.
\section{Herdade}
\begin{itemize}
\item {Grp. gram.:f.}
\end{itemize}
\begin{itemize}
\item {Proveniência:(Do lat. \textunderscore hereditas\textunderscore )}
\end{itemize}
Grande propriedade rústica, composta geralmente de montados, terra de semeadura e casa de habitação.
Herança.
\section{Herdadola}
\begin{itemize}
\item {Grp. gram.:f.}
\end{itemize}
Pequena herdade.
\section{Herdador}
\begin{itemize}
\item {Grp. gram.:m.}
\end{itemize}
O mesmo que \textunderscore herdeiro\textunderscore . Cf. Herculano, \textunderscore Bobo\textunderscore , 308; \textunderscore Hist. de Port.\textunderscore , II, 344 e 496; III, 283, 287 e 318.
\section{Herdamento}
\begin{itemize}
\item {Grp. gram.:m.}
\end{itemize}
\begin{itemize}
\item {Utilização:P. us.}
\end{itemize}
\begin{itemize}
\item {Utilização:Ant.}
\end{itemize}
\begin{itemize}
\item {Proveniência:(De \textunderscore herdar\textunderscore )}
\end{itemize}
O mesmo que \textunderscore herança\textunderscore .
Propriedade rústica ou urbana.
\section{Herdança}
\begin{itemize}
\item {Grp. gram.:f.}
\end{itemize}
\begin{itemize}
\item {Utilização:Prov.}
\end{itemize}
\begin{itemize}
\item {Proveniência:(De \textunderscore herdar\textunderscore )}
\end{itemize}
O mesmo que \textunderscore herança\textunderscore .
\section{Herdar}
\begin{itemize}
\item {Grp. gram.:v. t.}
\end{itemize}
\begin{itemize}
\item {Proveniência:(Lat. \textunderscore hereditare\textunderscore )}
\end{itemize}
Receber por herança.
Têr direito a receber por herança.
Adquirir por parentesco ou hereditariedade, (falando-se de virtudes ou vícios).
Legar.
\section{Herdeiro}
\begin{itemize}
\item {Grp. gram.:m.}
\end{itemize}
\begin{itemize}
\item {Utilização:Prov.}
\end{itemize}
\begin{itemize}
\item {Utilização:minh.}
\end{itemize}
\begin{itemize}
\item {Proveniência:(De \textunderscore herdar\textunderscore )}
\end{itemize}
Aquelle que herda; legatário.
Successor.
Consorte, sócio.
\section{Herdo}
\begin{itemize}
\item {Grp. gram.:m.}
\end{itemize}
\begin{itemize}
\item {Utilização:T. de Pare -de-Coira}
\end{itemize}
\begin{itemize}
\item {Utilização:des.}
\end{itemize}
\begin{itemize}
\item {Proveniência:(De \textunderscore herdar\textunderscore )}
\end{itemize}
O mesmo que \textunderscore herança\textunderscore .
\section{Heredar}
\textunderscore v. t. Ant.\textunderscore  (e der.)
O mesmo que \textunderscore herdar\textunderscore , etc. Cf. Usque, (\textunderscore passim\textunderscore ).
\section{Hereditariamente}
\begin{itemize}
\item {Grp. gram.:adv.}
\end{itemize}
De modo hereditário.
\section{Hereditariedade}
\begin{itemize}
\item {Grp. gram.:f.}
\end{itemize}
Qualidade daquillo que é hereditário.
Successão; transmissão das qualidades phýsicas ou moraes de alguém aos seus descendentes.
\section{Hereditário}
\begin{itemize}
\item {Grp. gram.:adj.}
\end{itemize}
\begin{itemize}
\item {Proveniência:(Lat. \textunderscore hereditarius\textunderscore )}
\end{itemize}
Que se transmitte por successão, de pais a filhos, ou de ascendentes a descendentes.
\section{Herege}
\begin{itemize}
\item {Grp. gram.:m.  e  adj.}
\end{itemize}
\begin{itemize}
\item {Utilização:Ext.}
\end{itemize}
\begin{itemize}
\item {Utilização:Pop.}
\end{itemize}
\begin{itemize}
\item {Proveniência:(Do lat. \textunderscore haereticus\textunderscore )}
\end{itemize}
O que professa doutrina contrária aos dogmas da Igreja.
Aquelle que tem ideias contrárias ás geralmente recebidas.
Homem ímpio, que não pratica o culto externo.
\section{Heregia}
\begin{itemize}
\item {Grp. gram.:f.}
\end{itemize}
\begin{itemize}
\item {Utilização:Pop.}
\end{itemize}
\begin{itemize}
\item {Proveniência:(De \textunderscore hereje\textunderscore )}
\end{itemize}
O mesmo que \textunderscore heresia\textunderscore .
\section{Hereja}
\begin{itemize}
\item {Grp. gram.:f.}
\end{itemize}
\begin{itemize}
\item {Utilização:Des.}
\end{itemize}
\begin{itemize}
\item {Proveniência:(De \textunderscore hereje\textunderscore )}
\end{itemize}
Mulher ímpia; mulher, que não crê nos dogmas da Igreja.
\section{Hereró}
\begin{itemize}
\item {Grp. gram.:m.}
\end{itemize}
\begin{itemize}
\item {Grp. gram.:Pl.}
\end{itemize}
Língua banta da África occidental.
Tríbo africana austro-occidental.
\section{Heresia}
\begin{itemize}
\item {Grp. gram.:f.}
\end{itemize}
\begin{itemize}
\item {Utilização:Fam.}
\end{itemize}
\begin{itemize}
\item {Proveniência:(Do lat. \textunderscore haeresís\textunderscore )}
\end{itemize}
Doutrina, opposta aos dogmas da Igreja.
Contra-senso.
Acto ou palavra offensíva da religião.
\section{Heresiarca}
\begin{itemize}
\item {Grp. gram.:m.  e  f.}
\end{itemize}

\section{Heresiarcha}
\begin{itemize}
\item {fónica:ca}
\end{itemize}
\begin{itemize}
\item {Grp. gram.:m.  e  f.}
\end{itemize}
\begin{itemize}
\item {Proveniência:(Lat. \textunderscore haeresiarcha\textunderscore )}
\end{itemize}
Pessôa, que fundou uma seita herética.
\section{Herètical}
\begin{itemize}
\item {Grp. gram.:adj.}
\end{itemize}
\begin{itemize}
\item {Utilização:Des.}
\end{itemize}
Próprio do herético.
\section{Herèticamente}
\begin{itemize}
\item {Grp. gram.:adv.}
\end{itemize}
De modo herético; como herege.
\section{Herèticidade}
\begin{itemize}
\item {Grp. gram.:f.}
\end{itemize}
\begin{itemize}
\item {Utilização:P. us.}
\end{itemize}
Qualidade ou estado de herético.
\section{Herético}
\begin{itemize}
\item {Grp. gram.:adj.}
\end{itemize}
\begin{itemize}
\item {Grp. gram.:M.}
\end{itemize}
\begin{itemize}
\item {Proveniência:(Lat. \textunderscore haereticus\textunderscore )}
\end{itemize}
Relativo a heresia.
O mesmo que \textunderscore herege\textunderscore .
\section{Heréu}
\begin{itemize}
\item {Grp. gram.:m.}
\end{itemize}
\begin{itemize}
\item {Utilização:Ant.}
\end{itemize}
O mesmo que \textunderscore herege\textunderscore ?:«\textunderscore Bem abasta estorvares os hereos dos altos ceos\textunderscore ». G. Vicente, I, 199.
(Cp. \textunderscore herege\textunderscore )
\section{Heréu}
\begin{itemize}
\item {Grp. gram.:m.}
\end{itemize}
\begin{itemize}
\item {Utilização:Ant.}
\end{itemize}
\begin{itemize}
\item {Proveniência:(Do lat. \textunderscore haeres\textunderscore )}
\end{itemize}
O mesmo que \textunderscore herdeiro\textunderscore .
\section{Herífuga}
\begin{itemize}
\item {Grp. gram.:m.}
\end{itemize}
\begin{itemize}
\item {Utilização:Des.}
\end{itemize}
\begin{itemize}
\item {Proveniência:(Lat. \textunderscore herifuga\textunderscore )}
\end{itemize}
Escravo fugitivo.
\section{Heril}
\begin{itemize}
\item {Grp. gram.:adj.}
\end{itemize}
\begin{itemize}
\item {Utilização:Des.}
\end{itemize}
\begin{itemize}
\item {Proveniência:(Lat. \textunderscore heritis\textunderscore )}
\end{itemize}
Próprio do senhor, relativamente ao escravo.
\section{Herilmente}
\begin{itemize}
\item {Grp. gram.:adv.}
\end{itemize}
De modo heril.
\section{Herma}
\begin{itemize}
\item {Grp. gram.:f.}
\end{itemize}
(V.hermes)
\section{Hermafrodita}
\begin{itemize}
\item {Grp. gram.:m.}
\end{itemize}
\begin{itemize}
\item {Utilização:Ext.}
\end{itemize}
\begin{itemize}
\item {Grp. gram.:Adj.}
\end{itemize}
\begin{itemize}
\item {Proveniência:(Do gr. \textunderscore Hermaphroditos\textunderscore , n. p. myth)}
\end{itemize}
Monstruosidade humana, que reúne em si alguns caracteres dos dois sexos.
Diz-se dos animaes e das plantas, que têm os dois sexos.
\section{Hermafroditismo}
\begin{itemize}
\item {Grp. gram.:m.}
\end{itemize}
Qualidade de hermafrodita.
Reunião dos dois sexos no mesmo indivíduo.
\section{Hermafrodito}
\begin{itemize}
\item {Grp. gram.:m.  e  adj.}
\end{itemize}
O mesmo ou melhor que \textunderscore hermafrodita\textunderscore :«\textunderscore espírito hermafrodito\textunderscore ». Castilho, \textunderscore Sabichonas\textunderscore , (na Advertência).
\section{Hermânia}
\begin{itemize}
\item {Grp. gram.:f.}
\end{itemize}
\begin{itemize}
\item {Proveniência:(De \textunderscore Hermann\textunderscore , n. p.)}
\end{itemize}
Gênero de arbustos do Cabo da Bôa-Esperança.
\section{Hermânnia}
\begin{itemize}
\item {Grp. gram.:f.}
\end{itemize}
\begin{itemize}
\item {Proveniência:(De \textunderscore Hermann\textunderscore , n. p.)}
\end{itemize}
Gênero de arbustos do Cabo da Bôa-Esperança.
\section{Hermaphrodita}
\begin{itemize}
\item {Grp. gram.:m.}
\end{itemize}
\begin{itemize}
\item {Utilização:Ext.}
\end{itemize}
\begin{itemize}
\item {Grp. gram.:Adj.}
\end{itemize}
\begin{itemize}
\item {Proveniência:(Do gr. \textunderscore Hermaphroditos\textunderscore , n. p. myth)}
\end{itemize}
Monstruosidade humana, que reúne em si alguns caracteres dos dois sexos.
Diz-se dos animaes e das plantas, que têm os dois sexos.
\section{Hermaphroditismo}
\begin{itemize}
\item {Grp. gram.:m.}
\end{itemize}
Qualidade de hermaphrodita.
Reunião dos dois sexos no mesmo indivíduo.
\section{Hermaphrodito}
\begin{itemize}
\item {Grp. gram.:m.  e  adj.}
\end{itemize}
O mesmo ou melhor que \textunderscore hermaphrodita\textunderscore :«\textunderscore espírito hermaphrodito\textunderscore ». Castilho, \textunderscore Sabichonas\textunderscore , (na Advertência).
\section{Hermeneuta}
\begin{itemize}
\item {Grp. gram.:m.}
\end{itemize}
Aquelle que é perito em hermenêutica.
\section{Hermenêutica}
\begin{itemize}
\item {Grp. gram.:f.}
\end{itemize}
\begin{itemize}
\item {Proveniência:(De \textunderscore hermenêutico\textunderscore )}
\end{itemize}
Interpretação do sentido das palavras.
Arte de interpretar leis.
Interpretação dos textos sagrados.
\section{Hermenêutico}
\begin{itemize}
\item {Grp. gram.:adj.}
\end{itemize}
\begin{itemize}
\item {Proveniência:(Lat. \textunderscore hermeneuticus\textunderscore )}
\end{itemize}
Relativo á hermenêutica.
\section{Hermes}
\begin{itemize}
\item {Grp. gram.:m.}
\end{itemize}
\begin{itemize}
\item {Proveniência:(Do gr. \textunderscore Hermes\textunderscore , n. p. myth.)}
\end{itemize}
Estátua de Mercúrio.
Cabeça ou busto de uma divindade, sôbre um pedestal ou pyrâmide.
Chama-se assim, em esculptura, a um escabello, que tem representada uma cabeça de Mercúrio.
Uma das manchas da lua.
\section{Hermeta}
\begin{itemize}
\item {Grp. gram.:f.}
\end{itemize}
Columna, que tem um hermes sobreposto.
\section{Hermete}
\begin{itemize}
\item {Grp. gram.:m.}
\end{itemize}
O mesmo que \textunderscore hermeta\textunderscore .
\section{Hermeticamente}
\begin{itemize}
\item {Grp. gram.:adv.}
\end{itemize}
De modo hermético.
\section{Hermético}
\begin{itemize}
\item {Grp. gram.:adj.}
\end{itemize}
\begin{itemize}
\item {Proveniência:(Do rad. de \textunderscore hermes\textunderscore )}
\end{itemize}
Que é encimado por um hermes.
Fechado completamente, de fórma que não deixe penetrar o ar, (falando-se de vaso, janelas, etc.).
Relativo á sciencia da transformação dos metaes ou á alchimia.
Relativo a certa Medicina, cujos processos se diziam encontrados nos livros de Hermes.
\section{Hérmia}
\begin{itemize}
\item {Grp. gram.:f.}
\end{itemize}
Fruto indiano, do tamanho da pimenta.
\section{Hermiano}
\begin{itemize}
\item {Grp. gram.:m.}
\end{itemize}
\begin{itemize}
\item {Proveniência:(De \textunderscore Hérmias\textunderscore , n. p.)}
\end{itemize}
Membro de uma seita religiosa do século II, a qual sustentava que Deus é corpóreo.
\section{Hermínio}
\begin{itemize}
\item {Grp. gram.:adj.}
\end{itemize}
\begin{itemize}
\item {Utilização:Des.}
\end{itemize}
Áspero.
Intratável.
Bravio.
(Parece relacionar-se com o b. lat. \textunderscore herminius\textunderscore , se não com o lat. \textunderscore Herminius\textunderscore , n. p.)
\section{Hermodáctilo}
\begin{itemize}
\item {Grp. gram.:m.}
\end{itemize}
\begin{itemize}
\item {Proveniência:(Gr. \textunderscore hermadahtulos\textunderscore )}
\end{itemize}
Bolbo ou tubérculo vegetal, trazido do Levante para o comércio europeu.
\section{Hermodáctylo}
\begin{itemize}
\item {Grp. gram.:m.}
\end{itemize}
\begin{itemize}
\item {Proveniência:(Gr. \textunderscore hermadahtulos\textunderscore )}
\end{itemize}
Bolbo ou tubérculo vegetal, trazido do Levante para o commércio europeu.
\section{Hermofenil}
\begin{itemize}
\item {Grp. gram.:m.}
\end{itemize}
Medicamento desinfectante.
\section{Hermogeniano}
\begin{itemize}
\item {Grp. gram.:m.}
\end{itemize}
\begin{itemize}
\item {Proveniência:(De \textunderscore Hermógenes\textunderscore , n. p.)}
\end{itemize}
Membro de uma seita religiosa do século III, a qual rejeitava a Trindade.
\section{Hermografia}
\begin{itemize}
\item {Grp. gram.:f.}
\end{itemize}
\begin{itemize}
\item {Proveniência:(Do gr. \textunderscore Hermes\textunderscore , n. p. + \textunderscore graphein\textunderscore )}
\end{itemize}
Descripção do planeta Mercúrio.
\section{Hermographia}
\begin{itemize}
\item {Grp. gram.:f.}
\end{itemize}
\begin{itemize}
\item {Proveniência:(Do gr. \textunderscore Hermes\textunderscore , n. p. + \textunderscore graphein\textunderscore )}
\end{itemize}
Descripção do planeta Mercúrio.
\section{Hermophenyl}
\begin{itemize}
\item {Grp. gram.:m.}
\end{itemize}
Medicamento desinfectante.
\section{Hernândia}
\begin{itemize}
\item {Grp. gram.:f.}
\end{itemize}
\begin{itemize}
\item {Proveniência:(De \textunderscore Hernando\textunderscore , n. p.)}
\end{itemize}
Gênero de plantas lauráceas da América.
\section{Hérnia}
\begin{itemize}
\item {Grp. gram.:f.}
\end{itemize}
\begin{itemize}
\item {Utilização:Pop.}
\end{itemize}
\begin{itemize}
\item {Proveniência:(Lat. \textunderscore hernia\textunderscore )}
\end{itemize}
Tumor, produzido pela saída ou deslocação de uma víscera.
Quebradura.
\section{Hernial}
\begin{itemize}
\item {Grp. gram.:adj.}
\end{itemize}
Relativo á hérnia.
\section{Herniária}
\begin{itemize}
\item {Grp. gram.:f.}
\end{itemize}
O mesmo que \textunderscore erva-turca\textunderscore .
\section{Herniário}
\begin{itemize}
\item {Grp. gram.:adj.}
\end{itemize}
\begin{itemize}
\item {Grp. gram.:M. pl.}
\end{itemize}
\begin{itemize}
\item {Proveniência:(Lat. \textunderscore hernici\textunderscore )}
\end{itemize}
O mesmo que \textunderscore hernial\textunderscore .
Antigos habitantes de Lácio.
\section{Hérnico}
\begin{itemize}
\item {Grp. gram.:adj.}
\end{itemize}
\begin{itemize}
\item {Grp. gram.:M. pl.}
\end{itemize}
\begin{itemize}
\item {Proveniência:(Lat. \textunderscore hernici\textunderscore )}
\end{itemize}
O mesmo que \textunderscore hernial\textunderscore .
Antigos habitantes de Lácio.
\section{Herníola}
\begin{itemize}
\item {Grp. gram.:f.}
\end{itemize}
Pequena planta, (\textunderscore herniaria glabra\textunderscore , Lin.), que se applicava em cataplasmas contra a hérnia.
\section{Hernioso}
\begin{itemize}
\item {Grp. gram.:m.  e  adj.}
\end{itemize}
O que padece hérnia.
\section{Herniotomia}
\begin{itemize}
\item {Grp. gram.:f.}
\end{itemize}
\begin{itemize}
\item {Proveniência:(Do lat. \textunderscore hernia\textunderscore  + gr. \textunderscore tome\textunderscore )}
\end{itemize}
Estrangulação cirúrgiga da hérnia.
\section{Hernutismo}
\begin{itemize}
\item {Grp. gram.:m.}
\end{itemize}
Doutrina dos hernutos; vida austera dos hernutos.
\section{Hernuto}
\begin{itemize}
\item {Grp. gram.:m.}
\end{itemize}
\begin{itemize}
\item {Proveniência:(Do al. \textunderscore Herrenhut\textunderscore , n. p.)}
\end{itemize}
Membro de uma seita religiosa, também conhecida por \textunderscore irmãos morávios\textunderscore , a qual proclama uma espécie de communismo e se distingue pela pureza dos seus costumes.
\section{Herodes}
\begin{itemize}
\item {Grp. gram.:m.}
\end{itemize}
\begin{itemize}
\item {Utilização:Fig.}
\end{itemize}
\begin{itemize}
\item {Proveniência:(De \textunderscore Herodes\textunderscore , n. p.)}
\end{itemize}
Homem feroz, tyranno.
Aquelle que é cruel ou muito severo para com crianças.
\section{Herodianos}
\begin{itemize}
\item {Grp. gram.:m. pl.}
\end{itemize}
\begin{itemize}
\item {Proveniência:(De \textunderscore Herodes\textunderscore , n. p.)}
\end{itemize}
Aquelles que, entre os Judeus, faziam profissão de honrar a memória do rei Herodes, que reedificara o templo de Jerusalém.
\section{Heróe}
\begin{itemize}
\item {Grp. gram.:m.}
\end{itemize}
\begin{itemize}
\item {Utilização:Deprec.}
\end{itemize}
\begin{itemize}
\item {Proveniência:(Do lat. \textunderscore heros\textunderscore )}
\end{itemize}
Homem extraordinário, pelas suas qualidades guerreiras, triumphos, valor ou magnanimidade.
Protagonista ou principal personagem de uma obra literária.
Homem notável por seus desmandos ou irregularidade de proceder.
\section{Herofone}
\begin{itemize}
\item {Grp. gram.:m.}
\end{itemize}
\begin{itemize}
\item {Utilização:Mús.}
\end{itemize}
Moderno instrumento de manivela, semelhante ao aríston.
\section{Herofónio}
\begin{itemize}
\item {Grp. gram.:m.}
\end{itemize}
\begin{itemize}
\item {Utilização:Mús.}
\end{itemize}
Moderno instrumento de manivela, semelhante ao aríston.
\section{Herói}
\begin{itemize}
\item {Grp. gram.:m.}
\end{itemize}
\begin{itemize}
\item {Utilização:Deprec.}
\end{itemize}
\begin{itemize}
\item {Proveniência:(Do lat. \textunderscore heros\textunderscore )}
\end{itemize}
Homem extraordinário, pelas suas qualidades guerreiras, triumphos, valor ou magnanimidade.
Protagonista ou principal personagem de uma obra literária.
Homem notável por seus desmandos ou irregularidade de proceder.
\section{Heróica}
\begin{itemize}
\item {Grp. gram.:f.}
\end{itemize}
\begin{itemize}
\item {Proveniência:(De \textunderscore heroico\textunderscore )}
\end{itemize}
O mesmo que \textunderscore heroína\textunderscore ^1. Cf. Filinto, XIX, 203.
\section{Heroicamente}
\begin{itemize}
\item {Grp. gram.:adv.}
\end{itemize}
De modo heroico.
\section{Heroicidade}
\begin{itemize}
\item {Grp. gram.:f.}
\end{itemize}
O mesmo ou melhor que \textunderscore heroísmo\textunderscore .
\section{Heróico}
\begin{itemize}
\item {Grp. gram.:adj.}
\end{itemize}
\begin{itemize}
\item {Proveniência:(Lat. \textunderscore heroicus\textunderscore )}
\end{itemize}
Próprio de um herói.
Enérgico.
E diz-se do estilo ou gênero literário, em que se celebram façanhas de heróis.
\textunderscore Verso heroico\textunderscore , verso de déz sýllabas, o mais usual em poemas heroicos portugueses.
\section{Heroicómico}
\begin{itemize}
\item {Grp. gram.:adj.}
\end{itemize}
\begin{itemize}
\item {Proveniência:(De \textunderscore heróico\textunderscore  + \textunderscore cómico\textunderscore )}
\end{itemize}
Que participa juntamente da feição heróica e cómica.
\section{Heróide}
\begin{itemize}
\item {Grp. gram.:f.}
\end{itemize}
\begin{itemize}
\item {Proveniência:(Do gr. \textunderscore herois\textunderscore , \textunderscore heroidos\textunderscore , mulher de herói, heroína)}
\end{itemize}
Epístola amorosa em verso, sob o nome de um herói ou de personagem notável.
\section{Heroificar}
\begin{itemize}
\item {Grp. gram.:v. t.}
\end{itemize}
\begin{itemize}
\item {Proveniência:(Do lat. \textunderscore heros\textunderscore  + \textunderscore facere\textunderscore )}
\end{itemize}
Qualificar de herói; incluir em o número dos heróis.
\section{Heroína}
\begin{itemize}
\item {Grp. gram.:f.}
\end{itemize}
\begin{itemize}
\item {Proveniência:(Lat. \textunderscore heroina\textunderscore )}
\end{itemize}
Mulher de valor, belleza ou talento extraordinários.
Mulher, que figura, como principal personagem, numa obra literária.
\section{Heroína}
\begin{itemize}
\item {Grp. gram.:f.}
\end{itemize}
\begin{itemize}
\item {Utilização:Pharm.}
\end{itemize}
Medicamento, que é um succedâneo da morphina.
\section{Heroísmo}
\begin{itemize}
\item {Grp. gram.:m.}
\end{itemize}
\begin{itemize}
\item {Proveniência:(De \textunderscore herói\textunderscore )}
\end{itemize}
Qualidade daquelle que é herói ou daquillo que é heróico.
Magnanimidade.
Acto heróico.
\section{Heróon}
\begin{itemize}
\item {Grp. gram.:m.}
\end{itemize}
\begin{itemize}
\item {Proveniência:(Gr. \textunderscore heroon\textunderscore )}
\end{itemize}
Designação antiga de qualquer monumento, elevado á memória de um herói ou de uma heroína.
\section{Herophone}
\begin{itemize}
\item {Grp. gram.:m.}
\end{itemize}
\begin{itemize}
\item {Utilização:Mús.}
\end{itemize}
Moderno instrumento de manivela, semelhante ao aríston.
\section{Herpes}
\begin{itemize}
\item {Grp. gram.:m. pl.}
\end{itemize}
\begin{itemize}
\item {Utilização:Fig.}
\end{itemize}
\begin{itemize}
\item {Proveniência:(Lat. \textunderscore herpes\textunderscore )}
\end{itemize}
Affecção vesiculosa da pelle.
Contágio.
\section{Herpético}
\begin{itemize}
\item {Grp. gram.:adj.}
\end{itemize}
Que tem a natureza de herpes.
Que padece herpes.
\section{Herpetismo}
\begin{itemize}
\item {Grp. gram.:m.}
\end{itemize}
Estado mórbido do organismo, caracterizado por herpes.
\section{Herpetografia}
\begin{itemize}
\item {Grp. gram.:f.}
\end{itemize}
\begin{itemize}
\item {Proveniência:(Do gr. \textunderscore herpeton\textunderscore  + \textunderscore graphein\textunderscore )}
\end{itemize}
Tratado dos reptis.
\section{Herpetographia}
\begin{itemize}
\item {Grp. gram.:f.}
\end{itemize}
\begin{itemize}
\item {Proveniência:(Do gr. \textunderscore herpeton\textunderscore  + \textunderscore graphein\textunderscore )}
\end{itemize}
Tratado dos reptis.
\section{Herpetologia}
\begin{itemize}
\item {Grp. gram.:f.}
\end{itemize}
\begin{itemize}
\item {Proveniência:(Do gr. \textunderscore herpes\textunderscore , \textunderscore herpetos\textunderscore  + \textunderscore logos\textunderscore )}
\end{itemize}
Tratado á cêrca dos herpes.
\section{Herpetologia}
\begin{itemize}
\item {Grp. gram.:f.}
\end{itemize}
\begin{itemize}
\item {Proveniência:(Do gr. \textunderscore herpeton\textunderscore  + \textunderscore logos\textunderscore )}
\end{itemize}
O mesmo que \textunderscore herpetographia\textunderscore .
\section{Herpetólogo}
\begin{itemize}
\item {Grp. gram.:m.}
\end{itemize}
Aquelle que é perito em herpetologia.
\section{Herschell}
\begin{itemize}
\item {Grp. gram.:m.}
\end{itemize}
Nome, que se deu primeiro ao planeta Urano, descoberto por Herschell em 1881.
\section{Hertziano}
\begin{itemize}
\item {Grp. gram.:adj.}
\end{itemize}
\begin{itemize}
\item {Proveniência:(De \textunderscore Hertz\textunderscore , n. p.)}
\end{itemize}
Diz-se de uma variedade de telégrapho. Cf. \textunderscore Jorn. do Comm.\textunderscore , do Rio, de 13-VI-901.
\section{Hertzógrafo}
\begin{itemize}
\item {Grp. gram.:m.}
\end{itemize}
\begin{itemize}
\item {Proveniência:(De \textunderscore Hertz\textunderscore , n. p.)}
\end{itemize}
Aparelho da telegrafia sem fios.
\section{Hertzógrapho}
\begin{itemize}
\item {Grp. gram.:m.}
\end{itemize}
\begin{itemize}
\item {Proveniência:(De \textunderscore Hertz\textunderscore , n. p.)}
\end{itemize}
Apparelho da telegraphia sem fios.
\section{Hérulos}
\begin{itemize}
\item {Grp. gram.:m. pl.}
\end{itemize}
\begin{itemize}
\item {Proveniência:(Lat. \textunderscore heruli\textunderscore )}
\end{itemize}
Povo germânico que, invadindo a Itália, determinou a quéda do Império Romano do Occidente.
\section{Herva}
\textunderscore f.\textunderscore  (e der.)
(V. \textunderscore erva\textunderscore , etc.)
\section{Hervoeira}
\begin{itemize}
\item {Grp. gram.:f.}
\end{itemize}
\begin{itemize}
\item {Utilização:Ant.}
\end{itemize}
O mesmo que \textunderscore prostituta\textunderscore .
\section{Hesiódico}
\begin{itemize}
\item {Grp. gram.:adj.}
\end{itemize}
Relativo ao poéta Hesíodo. Cf. Latino, \textunderscore Or. da Corôa\textunderscore , LIII.
\section{Hesitação}
\begin{itemize}
\item {Grp. gram.:f.}
\end{itemize}
\begin{itemize}
\item {Proveniência:(Lat. \textunderscore haesitatio\textunderscore )}
\end{itemize}
Acção de hesitar.
Estado de quem hesita; indecisão.
\section{Hesitante}
\begin{itemize}
\item {Grp. gram.:adj.}
\end{itemize}
\begin{itemize}
\item {Proveniência:(Lat. \textunderscore haesitans\textunderscore )}
\end{itemize}
Que hesita.
\section{Hesitar}
\begin{itemize}
\item {Grp. gram.:v. i.}
\end{itemize}
\begin{itemize}
\item {Proveniência:(Lat. \textunderscore haesitare\textunderscore )}
\end{itemize}
Estar indeciso, perplexo.
Não tomar resolução.
Duvidar; titubear.
\section{Hespanhol}
\textunderscore m.\textunderscore  e \textunderscore adj.\textunderscore  (e der.)
(V. \textunderscore espanhol\textunderscore , etc.)
\section{Hésper}
\begin{itemize}
\item {Grp. gram.:m.}
\end{itemize}
O mesmo que \textunderscore vésper\textunderscore .
\section{Hespéria}
\begin{itemize}
\item {Grp. gram.:f.}
\end{itemize}
\begin{itemize}
\item {Proveniência:(Do lat. \textunderscore Hesperia\textunderscore , n. p.)}
\end{itemize}
Insecto lepidóptero diurno.
\section{Hespéridas}
\begin{itemize}
\item {Grp. gram.:f. pl.}
\end{itemize}
Tríbo de insectos lepidópteros, que têm por typo a hespéria.
\section{Hesperídeas}
\begin{itemize}
\item {Grp. gram.:f. pl.}
\end{itemize}
\begin{itemize}
\item {Proveniência:(De \textunderscore hesperideo\textunderscore )}
\end{itemize}
Ordem de plantas dicotyledóneas, a que pertence a laranjeira.
\section{Hesperídeo}
\begin{itemize}
\item {Grp. gram.:adj.}
\end{itemize}
\begin{itemize}
\item {Utilização:Bot.}
\end{itemize}
\begin{itemize}
\item {Proveniência:(Do rad. de \textunderscore Hespérides\textunderscore , n. p.)}
\end{itemize}
Diz-se dos frutos carnosos, que têm por typo a laranja.
\section{Hespérides}
\begin{itemize}
\item {Grp. gram.:m. pl.}
\end{itemize}
O mesmo que [[pléiades|pléiade]].
\section{Hesperidina}
\begin{itemize}
\item {Grp. gram.:f.}
\end{itemize}
\begin{itemize}
\item {Utilização:Bot.}
\end{itemize}
\begin{itemize}
\item {Proveniência:(De \textunderscore hesperídeo\textunderscore )}
\end{itemize}
Princípio, descoberto na parte branca que envolve o fruto esperídeo.
\section{Hespério}
\begin{itemize}
\item {Grp. gram.:adj.}
\end{itemize}
\begin{itemize}
\item {Utilização:Poét.}
\end{itemize}
\begin{itemize}
\item {Proveniência:(Lat. \textunderscore hesperius\textunderscore )}
\end{itemize}
O mesmo que \textunderscore occidental\textunderscore .
\section{Hespérios}
\begin{itemize}
\item {Grp. gram.:m. pl.}
\end{itemize}
O mesmo que \textunderscore hespéridas\textunderscore .
\section{Hésperis}
\begin{itemize}
\item {Grp. gram.:f.}
\end{itemize}
\begin{itemize}
\item {Utilização:Bot.}
\end{itemize}
\begin{itemize}
\item {Proveniência:(Lat. \textunderscore Hesperis\textunderscore , n. p.)}
\end{itemize}
O mesmo que \textunderscore juliana\textunderscore , planta crucífera ornamental.
\section{Héspero}
\begin{itemize}
\item {Grp. gram.:m.}
\end{itemize}
\begin{itemize}
\item {Proveniência:(Lat. \textunderscore hesperos\textunderscore )}
\end{itemize}
O mesmo que \textunderscore véspero\textunderscore .
\section{Hessocênico}
\begin{itemize}
\item {Grp. gram.:adj.}
\end{itemize}
\begin{itemize}
\item {Utilização:Geol.}
\end{itemize}
\begin{itemize}
\item {Proveniência:(Do gr. \textunderscore hesson\textunderscore  + \textunderscore kainos\textunderscore )}
\end{itemize}
Diz-se do terreno, a que os antigos autores chamaram terciário. Cf. G. Guimarães, \textunderscore Geol.\textunderscore , 203.
\section{Hestér}
\begin{itemize}
\item {Grp. gram.:m.}
\end{itemize}
Madeira escura das Antilhas.
\section{Hesterno}
\begin{itemize}
\item {Grp. gram.:adj.}
\end{itemize}
\begin{itemize}
\item {Utilização:Poét.}
\end{itemize}
\begin{itemize}
\item {Proveniência:(Lat. \textunderscore hesternus\textunderscore )}
\end{itemize}
Relativo ao dia de ontem.
\section{Hetaira}
\begin{itemize}
\item {Grp. gram.:f.}
\end{itemize}
(Fórma usual, mas inexacta, em vez de \textunderscore hetera\textunderscore . Cf. Latino, \textunderscore Or. da Corôa\textunderscore , CCXVIII e CCXXX.)
\section{Hétego}
\begin{itemize}
\item {Grp. gram.:adj.}
\end{itemize}
\begin{itemize}
\item {Utilização:Ant.}
\end{itemize}
O mesmo que \textunderscore héctico\textunderscore :«\textunderscore Qu'eu quando casei com ella, dizião-me: hetega é.\textunderscore »G. Vicente, I, 167.
\section{Hetera}
\begin{itemize}
\item {Grp. gram.:f.}
\end{itemize}
Mulher dissoluta, cortesan, na antiguidade grega.
Hoje, prostituta, elegante e distinta, no seu gênero.
\section{Heteria}
\begin{itemize}
\item {Grp. gram.:f.}
\end{itemize}
\begin{itemize}
\item {Proveniência:(Lat. \textunderscore hetaeria\textunderscore )}
\end{itemize}
Collégio ou sociedade de heteras.
\section{Heteriarca}
\begin{itemize}
\item {Grp. gram.:m.}
\end{itemize}
\begin{itemize}
\item {Proveniência:(Do gr. \textunderscore hetairia\textunderscore  + \textunderscore arkhein\textunderscore )}
\end{itemize}
Oficial que, no império grego, comandava as tropas aliadas.
\section{Heteriarcha}
\begin{itemize}
\item {fónica:ca}
\end{itemize}
\begin{itemize}
\item {Grp. gram.:m.}
\end{itemize}
\begin{itemize}
\item {Proveniência:(Do gr. \textunderscore hetairia\textunderscore  + \textunderscore arkhein\textunderscore )}
\end{itemize}
Official que, no império grego, commandava as tropas alliadas.
\section{Hetérice}
\begin{itemize}
\item {Grp. gram.:f.}
\end{itemize}
\begin{itemize}
\item {Proveniência:(Lat. \textunderscore hetaerice\textunderscore )}
\end{itemize}
Companhia de soldados de cavallaria, na antiga Macedónia.
\section{Heterismo}
\begin{itemize}
\item {Grp. gram.:m.}
\end{itemize}
\begin{itemize}
\item {Utilização:P. us.}
\end{itemize}
\begin{itemize}
\item {Proveniência:(De \textunderscore hetera\textunderscore )}
\end{itemize}
Amor livre, nas mulheres. Cf. Oliv. Martins, \textunderscore Quadro das Instit. Primit.\textunderscore , 10.
\section{Heterista}
\begin{itemize}
\item {Grp. gram.:adj.}
\end{itemize}
Relativo ás heteras.
Sensual.
\section{Hetero...}
\begin{itemize}
\item {Grp. gram.:pref.}
\end{itemize}
\begin{itemize}
\item {Proveniência:(Gr. \textunderscore heteros\textunderscore )}
\end{itemize}
(designativo de \textunderscore differente\textunderscore , \textunderscore irregular\textunderscore , \textunderscore outro\textunderscore , \textunderscore anómalo\textunderscore , etc.)
\section{Heterobrânchio}
\begin{itemize}
\item {fónica:qui}
\end{itemize}
\begin{itemize}
\item {Grp. gram.:adj.}
\end{itemize}
\begin{itemize}
\item {Utilização:Ichthyol.}
\end{itemize}
\begin{itemize}
\item {Grp. gram.:M. pl.}
\end{itemize}
\begin{itemize}
\item {Proveniência:(De \textunderscore hetero...\textunderscore  + \textunderscore brânchias\textunderscore )}
\end{itemize}
Cujas brânchias variam.
Peixes, cujas brânchias são acompanhadas de appêndices ramificados.
\section{Heterobrânquio}
\begin{itemize}
\item {Grp. gram.:adj.}
\end{itemize}
\begin{itemize}
\item {Utilização:Ichthyol.}
\end{itemize}
\begin{itemize}
\item {Grp. gram.:M. pl.}
\end{itemize}
\begin{itemize}
\item {Proveniência:(De \textunderscore hetero...\textunderscore  + \textunderscore brânquias\textunderscore )}
\end{itemize}
Cujas brânquias variam.
Peixes, cujas brânquias são acompanhadas de apêndices ramificados.
\section{Heterocarpo}
\begin{itemize}
\item {Grp. gram.:adj.}
\end{itemize}
\begin{itemize}
\item {Utilização:Bot.}
\end{itemize}
\begin{itemize}
\item {Proveniência:(Do gr. \textunderscore heteros\textunderscore  + \textunderscore karpos\textunderscore )}
\end{itemize}
Que produz flôres ou frutos de natureza diversa.
\section{Heteróceros}
\begin{itemize}
\item {Grp. gram.:m.}
\end{itemize}
\begin{itemize}
\item {Utilização:Bot.}
\end{itemize}
\begin{itemize}
\item {Proveniência:(Do gr. \textunderscore heteros\textunderscore  + \textunderscore keras\textunderscore )}
\end{itemize}
Gênero ou divisão de lepidópteros, que comprehende, segundo alguns zoólogos, os lepidópteros que não são rhopalóceros.
\section{Heteróclito}
\begin{itemize}
\item {Grp. gram.:adj.}
\end{itemize}
\begin{itemize}
\item {Utilização:Ext.}
\end{itemize}
\begin{itemize}
\item {Proveniência:(Lat. \textunderscore heteroclitus\textunderscore )}
\end{itemize}
Que se desvia dos princípios da analogia grammatical ou das regras da arte.
Extravagante; excêntrico.
\section{Heterocrânia}
\begin{itemize}
\item {Grp. gram.:f.}
\end{itemize}
\begin{itemize}
\item {Utilização:Des.}
\end{itemize}
\begin{itemize}
\item {Proveniência:(Lat. \textunderscore heterocrania\textunderscore )}
\end{itemize}
O mesmo que \textunderscore hemicrânia\textunderscore .
\section{Heterodáctilos}
\begin{itemize}
\item {Grp. gram.:m. pl.}
\end{itemize}
\begin{itemize}
\item {Proveniência:(Do gr. \textunderscore heteros\textunderscore  + \textunderscore daktulos\textunderscore )}
\end{itemize}
Família de aves trepadoras, que têm o dedo externo reversível.
\section{Heterodáctylos}
\begin{itemize}
\item {Grp. gram.:m. pl.}
\end{itemize}
\begin{itemize}
\item {Proveniência:(Do gr. \textunderscore heteros\textunderscore  + \textunderscore daktulos\textunderscore )}
\end{itemize}
Família de aves trepadoras, que têm o dedo externo reversível.
\section{Heterodermes}
\begin{itemize}
\item {Grp. gram.:m. pl.}
\end{itemize}
\begin{itemize}
\item {Proveniência:(Do gr. \textunderscore heteros\textunderscore  + \textunderscore derma\textunderscore )}
\end{itemize}
Família de reptis, que têm escamas de fórmas diversas.
\section{Heterodinâmico}
\begin{itemize}
\item {Grp. gram.:adj.}
\end{itemize}
\begin{itemize}
\item {Proveniência:(Do gr. \textunderscore heteros\textunderscore  + \textunderscore dunamis\textunderscore )}
\end{itemize}
Que tem fôrça desigual.
\section{Heterodonte}
\begin{itemize}
\item {Grp. gram.:m.}
\end{itemize}
\begin{itemize}
\item {Proveniência:(Do gr. \textunderscore heteros\textunderscore  + \textunderscore odous\textunderscore , \textunderscore odontos\textunderscore )}
\end{itemize}
Gênero de reptis ophídios.
\section{Heterodoxia}
\begin{itemize}
\item {fónica:csi}
\end{itemize}
\begin{itemize}
\item {Grp. gram.:f.}
\end{itemize}
Qualidade de heterodoxo.
Opposição aos sentimentos orthodoxos.
\section{Heterodoxo}
\begin{itemize}
\item {fónica:cso}
\end{itemize}
\begin{itemize}
\item {Grp. gram.:adj.}
\end{itemize}
\begin{itemize}
\item {Proveniência:(Lat. \textunderscore heterodoxus\textunderscore )}
\end{itemize}
Que não é orthodoxo.
Opposto aos princípios de uma religião.
Herético.
Diz-se dos botânicos, que não tomaram a frutificação como base das suas classificações.
\section{Heteródromo}
\begin{itemize}
\item {Grp. gram.:m.}
\end{itemize}
\begin{itemize}
\item {Proveniência:(Do gr. \textunderscore heteros\textunderscore  + \textunderscore dromos\textunderscore )}
\end{itemize}
(V.interfixo)
\section{Heterodynâmico}
\begin{itemize}
\item {Grp. gram.:adj.}
\end{itemize}
\begin{itemize}
\item {Proveniência:(Do gr. \textunderscore heteros\textunderscore  + \textunderscore dunamis\textunderscore )}
\end{itemize}
Que tem fôrça desigual.
\section{Heterogamia}
\begin{itemize}
\item {Grp. gram.:f.}
\end{itemize}
\begin{itemize}
\item {Utilização:Bot.}
\end{itemize}
\begin{itemize}
\item {Proveniência:(Do gr. \textunderscore heteros\textunderscore  + \textunderscore gamos\textunderscore )}
\end{itemize}
Estado das plantas heterogâmicas.
\section{Heterogâmico}
\begin{itemize}
\item {Grp. gram.:adj.}
\end{itemize}
\begin{itemize}
\item {Proveniência:(De \textunderscore heterogamia\textunderscore )}
\end{itemize}
Que tem flôres de duas espécies.
\section{Heterógamo}
\begin{itemize}
\item {Grp. gram.:adj.}
\end{itemize}
O mesmo que \textunderscore heterogâmico\textunderscore .
\section{Heterogeneidade}
\begin{itemize}
\item {Grp. gram.:f.}
\end{itemize}
Qualidade daquillo que é heterogêneo.
\section{Heterogêneo}
\begin{itemize}
\item {Grp. gram.:adj.}
\end{itemize}
\begin{itemize}
\item {Grp. gram.:M. pl.}
\end{itemize}
\begin{itemize}
\item {Proveniência:(Do gr. \textunderscore heteros\textunderscore  + \textunderscore genos\textunderscore )}
\end{itemize}
Que tem natureza differente de outra coisa.
Ordem de zoóphytos.
\section{Heterogenesía}
\begin{itemize}
\item {Grp. gram.:f.}
\end{itemize}
\begin{itemize}
\item {Proveniência:(Do gr. \textunderscore heteros\textunderscore  + \textunderscore genesis\textunderscore )}
\end{itemize}
Ausência de fecundação, com ou sem aproximação sexual.
O mesmo que \textunderscore heterogenia\textunderscore .
\section{Heterogenia}
\begin{itemize}
\item {Grp. gram.:f.}
\end{itemize}
\begin{itemize}
\item {Proveniência:(Do gr. \textunderscore heteros\textunderscore  + \textunderscore genea\textunderscore )}
\end{itemize}
Geração espontânea, ou producção de seres vivos em substâncias orgânicas ou inorgânicas, sem germes nem óvulos.
\section{Heterofilia}
\begin{itemize}
\item {Grp. gram.:f.}
\end{itemize}
\begin{itemize}
\item {Proveniência:(De \textunderscore heterofilo\textunderscore )}
\end{itemize}
Estado de uma planta heterofila.
\section{Heterofilo}
\begin{itemize}
\item {Grp. gram.:adj.}
\end{itemize}
\begin{itemize}
\item {Proveniência:(Do gr. \textunderscore heteros\textunderscore  + \textunderscore phullon\textunderscore )}
\end{itemize}
Diz-se das plantas, cujas fôlhas têm fórma e grandeza diferentes.
\section{Heteroftalmia}
\begin{itemize}
\item {Grp. gram.:f.}
\end{itemize}
\begin{itemize}
\item {Proveniência:(Do gr. \textunderscore heteros\textunderscore  + \textunderscore ophthalmos\textunderscore )}
\end{itemize}
Diferença entre ambos os olhos.
\section{Heteróginos}
\begin{itemize}
\item {Grp. gram.:m. pl.}
\end{itemize}
\begin{itemize}
\item {Utilização:Zool.}
\end{itemize}
\begin{itemize}
\item {Proveniência:(Do gr. \textunderscore heteros\textunderscore  + \textunderscore gune\textunderscore )}
\end{itemize}
Animaes, cuja espécie se compõe de machos, fêmeas e neutros.
\section{Heterógono}
\begin{itemize}
\item {Grp. gram.:adj.}
\end{itemize}
\begin{itemize}
\item {Utilização:Geom.}
\end{itemize}
\begin{itemize}
\item {Proveniência:(Do gr. \textunderscore heteros\textunderscore  + \textunderscore gonos\textunderscore )}
\end{itemize}
Que tem ângulos differentes.
\section{Heterógynos}
\begin{itemize}
\item {Grp. gram.:m. pl.}
\end{itemize}
\begin{itemize}
\item {Utilização:Zool.}
\end{itemize}
\begin{itemize}
\item {Proveniência:(Do gr. \textunderscore heteros\textunderscore  + \textunderscore gune\textunderscore )}
\end{itemize}
Animaes, cuja espécie se compõe de machos, fêmeas e neutros.
\section{Heteroide}
\begin{itemize}
\item {Grp. gram.:adj.}
\end{itemize}
\begin{itemize}
\item {Utilização:Bot.}
\end{itemize}
\begin{itemize}
\item {Proveniência:(Do gr. \textunderscore heteros\textunderscore  + \textunderscore eidos\textunderscore )}
\end{itemize}
Diz-se das partes vegetaes que, pertencendo á mesma planta, são todavia diversas na fórma.
\section{Hetero-infecção}
\begin{itemize}
\item {Grp. gram.:f.}
\end{itemize}
Infecção, produzida num indivíduo por vírus ou miasma trazido por outro.
\section{Heterologia}
\begin{itemize}
\item {Grp. gram.:f.}
\end{itemize}
Carácter daquillo que é heterólogo.
\section{Heterólogo}
\begin{itemize}
\item {Grp. gram.:adj.}
\end{itemize}
\begin{itemize}
\item {Utilização:Physiol.}
\end{itemize}
\begin{itemize}
\item {Proveniência:(Do gr. \textunderscore heteros\textunderscore  + \textunderscore logos\textunderscore )}
\end{itemize}
Diz-se dos tecidos mórbidos, ou dos tecidos que não têm analogia com os tecidos do corpo.
\section{Heteromachia}
\begin{itemize}
\item {fónica:qui}
\end{itemize}
\begin{itemize}
\item {Grp. gram.:f.}
\end{itemize}
\begin{itemize}
\item {Proveniência:(Do gr. \textunderscore heteros\textunderscore  + \textunderscore makhe\textunderscore )}
\end{itemize}
Luta de homem com homem.
\section{Heteromaquia}
\begin{itemize}
\item {Grp. gram.:f.}
\end{itemize}
\begin{itemize}
\item {Proveniência:(Do gr. \textunderscore heteros\textunderscore  + \textunderscore makhe\textunderscore )}
\end{itemize}
Luta de homem com homem.
\section{Heterómeros}
\begin{itemize}
\item {Grp. gram.:m. pl.}
\end{itemize}
\begin{itemize}
\item {Proveniência:(Do gr. \textunderscore heteros\textunderscore  + \textunderscore meros\textunderscore )}
\end{itemize}
Secção de insectos coleópteros, que comprehende aquelles, cujos tarsos não têm o mesmo número de artículos em todas as patas.
\section{Heteromorfia}
\begin{itemize}
\item {Grp. gram.:f.}
\end{itemize}
Sistema que atribue ás doenças elementos mórbidos distintos.
Qualidade daquilo que é heterómorfo.
\section{Heteromorfismo}
\begin{itemize}
\item {Grp. gram.:m.}
\end{itemize}
O mesmo que \textunderscore heteromorfia\textunderscore .
\section{Heteromorfo}
\begin{itemize}
\item {Grp. gram.:adj.}
\end{itemize}
\begin{itemize}
\item {Grp. gram.:M. pl.}
\end{itemize}
\begin{itemize}
\item {Proveniência:(Do gr. \textunderscore heteros\textunderscore  + \textunderscore morphe\textunderscore )}
\end{itemize}
Que tem fórma diferente nas suas diversas partes.
O mesmo que \textunderscore espongiários\textunderscore .
\section{Heteromorphia}
\begin{itemize}
\item {Grp. gram.:f.}
\end{itemize}
Systema que attribue ás doenças elementos mórbidos distintos.
Qualidade daquillo que é heterómorpho.
\section{Heteromorphismo}
\begin{itemize}
\item {Grp. gram.:m.}
\end{itemize}
O mesmo que \textunderscore heteromorphia\textunderscore .
\section{Heteromorpho}
\begin{itemize}
\item {Grp. gram.:adj.}
\end{itemize}
\begin{itemize}
\item {Grp. gram.:M. pl.}
\end{itemize}
\begin{itemize}
\item {Proveniência:(Do gr. \textunderscore heteros\textunderscore  + \textunderscore morphe\textunderscore )}
\end{itemize}
Que tem fórma differente nas suas diversas partes.
O mesmo que \textunderscore espongiários\textunderscore .
\section{Heteronomia}
\begin{itemize}
\item {Grp. gram.:f.}
\end{itemize}
\begin{itemize}
\item {Proveniência:(De \textunderscore heterónomo\textunderscore )}
\end{itemize}
Leis da natureza, cuja violência se exerce nas nossas paixões e necessidades, oppondo-se á autonomia.
Desvio das leis normaes.
\section{Heterónomo}
\begin{itemize}
\item {Grp. gram.:adj.}
\end{itemize}
\begin{itemize}
\item {Utilização:Miner.}
\end{itemize}
\begin{itemize}
\item {Proveniência:(Do gr. \textunderscore heteros\textunderscore  + \textunderscore nomos\textunderscore )}
\end{itemize}
Diz-se dos crystaes, cuja formação se desvia das leis conhecidas.
\section{Heteropagia}
\begin{itemize}
\item {Grp. gram.:f.}
\end{itemize}
Estado de heterópago.
\section{Heterópago}
\begin{itemize}
\item {Grp. gram.:adj.}
\end{itemize}
\begin{itemize}
\item {Proveniência:(Do gr. \textunderscore heteros\textunderscore  + \textunderscore pageis\textunderscore )}
\end{itemize}
Diz-se dos monstros duplos, em que o indivíduo accessório, muito imperfeito, mas de cabeça distinta, tem o corpo implantado na parte anterior do indivíduo principal.
\section{Heteropathia}
\begin{itemize}
\item {Grp. gram.:f.}
\end{itemize}
\begin{itemize}
\item {Proveniência:(Do gr. \textunderscore heteros\textunderscore  + \textunderscore pathos\textunderscore )}
\end{itemize}
O mesmo que \textunderscore allopathia\textunderscore .
\section{Heteropáthico}
\begin{itemize}
\item {Grp. gram.:adj.}
\end{itemize}
\begin{itemize}
\item {Utilização:Med.}
\end{itemize}
\begin{itemize}
\item {Proveniência:(Do gr. \textunderscore heteros\textunderscore  + \textunderscore pathos\textunderscore )}
\end{itemize}
Diz-se do tratamento, que produz alterações, não contrárias nem semelhantes ás das doenças.
\section{Heteropatia}
\begin{itemize}
\item {Grp. gram.:f.}
\end{itemize}
\begin{itemize}
\item {Proveniência:(Do gr. \textunderscore heteros\textunderscore  + \textunderscore pathos\textunderscore )}
\end{itemize}
O mesmo que \textunderscore alopatia\textunderscore .
\section{Heteropático}
\begin{itemize}
\item {Grp. gram.:adj.}
\end{itemize}
\begin{itemize}
\item {Utilização:Med.}
\end{itemize}
\begin{itemize}
\item {Proveniência:(Do gr. \textunderscore heteros\textunderscore  + \textunderscore pathos\textunderscore )}
\end{itemize}
Diz-se do tratamento, que produz alterações, não contrárias nem semelhantes ás das doenças.
\section{Heteropétalo}
\begin{itemize}
\item {Grp. gram.:adj.}
\end{itemize}
\begin{itemize}
\item {Proveniência:(De \textunderscore hetero...\textunderscore  + \textunderscore pétala\textunderscore )}
\end{itemize}
Que tem pétalas differentes entre si.
\section{Heterophthalmia}
\begin{itemize}
\item {Grp. gram.:f.}
\end{itemize}
\begin{itemize}
\item {Proveniência:(Do gr. \textunderscore heteros\textunderscore  + \textunderscore ophthalmos\textunderscore )}
\end{itemize}
Differença entre ambos os olhos.
\section{Heterophyllia}
\begin{itemize}
\item {Grp. gram.:f.}
\end{itemize}
\begin{itemize}
\item {Proveniência:(De \textunderscore heterophyllo\textunderscore )}
\end{itemize}
Estado de uma planta heterophylla.
\section{Heterophyllo}
\begin{itemize}
\item {Grp. gram.:adj.}
\end{itemize}
\begin{itemize}
\item {Proveniência:(Do gr. \textunderscore heteros\textunderscore  + \textunderscore phullon\textunderscore )}
\end{itemize}
Diz-se das plantas, cujas fôlhas têm fórma e grandeza differentes.
\section{Heteroplasia}
\begin{itemize}
\item {Grp. gram.:f.}
\end{itemize}
\begin{itemize}
\item {Proveniência:(Do gr. \textunderscore heteros\textunderscore  + \textunderscore plasis\textunderscore )}
\end{itemize}
Formação pathológica de productos estranhos á economia animal, como o tubérculo.
\section{Heteroplásico}
\begin{itemize}
\item {Grp. gram.:adj.}
\end{itemize}
Relativo á heteroplasia.
\section{Heteroplasma}
\begin{itemize}
\item {Grp. gram.:m.}
\end{itemize}
\begin{itemize}
\item {Proveniência:(De \textunderscore hetero...\textunderscore  + \textunderscore plasma\textunderscore )}
\end{itemize}
Substância, que constitue um producto mórbido, estranho á economia animal.
\section{Heteroplástico}
\begin{itemize}
\item {Grp. gram.:adj.}
\end{itemize}
\begin{itemize}
\item {Proveniência:(De \textunderscore hetero...\textunderscore  + \textunderscore plástico\textunderscore )}
\end{itemize}
Relativo á heteroplasia.
Diz-se dos medicamentos, que alteram o estado dos sólidos e dos líquidos.
\section{Heterópodes}
\begin{itemize}
\item {Grp. gram.:m. pl.}
\end{itemize}
\begin{itemize}
\item {Proveniência:(Do gr. \textunderscore heteros\textunderscore  + \textunderscore pous\textunderscore , \textunderscore podos\textunderscore )}
\end{itemize}
Molluscos de pés desiguaes.
\section{Heteróporo}
\begin{itemize}
\item {Grp. gram.:adj.}
\end{itemize}
\begin{itemize}
\item {Proveniência:(De \textunderscore hetero...\textunderscore  + \textunderscore poro\textunderscore )}
\end{itemize}
Diz-se dos polypeiros, em que as aberturas das céllulas são dirigidas em todos os sentidos.
\section{Heterópteros}
\begin{itemize}
\item {Grp. gram.:m. pl.}
\end{itemize}
\begin{itemize}
\item {Proveniência:(Do gr. \textunderscore heteros\textunderscore  + \textunderscore pteron\textunderscore )}
\end{itemize}
Divisão da ordem dos hemipteros.
\section{Heterorexia}
\begin{itemize}
\item {fónica:re,csi}
\end{itemize}
\begin{itemize}
\item {Grp. gram.:f.}
\end{itemize}
\begin{itemize}
\item {Utilização:Med.}
\end{itemize}
\begin{itemize}
\item {Proveniência:(Do gr. \textunderscore heteros\textunderscore  + \textunderscore rexis\textunderscore )}
\end{itemize}
Perversão ou depravação do apetite.
\section{Heterorrexia}
\begin{itemize}
\item {fónica:csi}
\end{itemize}
\begin{itemize}
\item {Grp. gram.:f.}
\end{itemize}
\begin{itemize}
\item {Utilização:Med.}
\end{itemize}
\begin{itemize}
\item {Proveniência:(Do gr. \textunderscore heteros\textunderscore  + \textunderscore rexis\textunderscore )}
\end{itemize}
Perversão ou depravação do appetite.
\section{Heteróscios}
\begin{itemize}
\item {Grp. gram.:m. pl.}
\end{itemize}
\begin{itemize}
\item {Utilização:Geogr.}
\end{itemize}
\begin{itemize}
\item {Proveniência:(Do gr. \textunderscore heteros\textunderscore  + \textunderscore skia\textunderscore )}
\end{itemize}
Povos, que habitam as zonas temperadas, e cuja sombra, em relação ao Sol, está na direcção do polo mais próximo.
\section{Heterotaxia}
\begin{itemize}
\item {fónica:csi}
\end{itemize}
\begin{itemize}
\item {Grp. gram.:f.}
\end{itemize}
\begin{itemize}
\item {Proveniência:(Do gr. \textunderscore heteros\textunderscore  + \textunderscore taxis\textunderscore )}
\end{itemize}
Anomalia teratológica, não apparente nem prejudicial ás funcções regulares.
\section{Heterotechnia}
\begin{itemize}
\item {Grp. gram.:f.}
\end{itemize}
\begin{itemize}
\item {Proveniência:(Do gr. \textunderscore heteros\textunderscore  + \textunderscore tekhne\textunderscore )}
\end{itemize}
Divergência entre práticas ou processos. Cf. Castilho, \textunderscore Fastos\textunderscore , I, 202.
\section{Heterotecnia}
\begin{itemize}
\item {Grp. gram.:f.}
\end{itemize}
\begin{itemize}
\item {Proveniência:(Do gr. \textunderscore heteros\textunderscore  + \textunderscore tekhne\textunderscore )}
\end{itemize}
Divergência entre práticas ou processos. Cf. Castilho, \textunderscore Fastos\textunderscore , I, 202.
\section{Heterotérmico}
\begin{itemize}
\item {Grp. gram.:adj.}
\end{itemize}
\begin{itemize}
\item {Proveniência:(Do gr. \textunderscore heteros\textunderscore  + \textunderscore therme\textunderscore )}
\end{itemize}
Que tem temperatura diferente.
\section{Heterotético}
\begin{itemize}
\item {Grp. gram.:adj.}
\end{itemize}
\begin{itemize}
\item {Proveniência:(Do gr. \textunderscore heteros\textunderscore  + \textunderscore thetikos\textunderscore )}
\end{itemize}
Diz-se da filosofia transcendente ou da ciência das coisas absolutas.
\section{Heterothérmico}
\begin{itemize}
\item {Grp. gram.:adj.}
\end{itemize}
\begin{itemize}
\item {Proveniência:(Do gr. \textunderscore heteros\textunderscore  + \textunderscore therme\textunderscore )}
\end{itemize}
Que tem temperatura differente.
\section{Heterothético}
\begin{itemize}
\item {Grp. gram.:adj.}
\end{itemize}
\begin{itemize}
\item {Proveniência:(Do gr. \textunderscore heteros\textunderscore  + \textunderscore thetikos\textunderscore )}
\end{itemize}
Diz-se da philosophia transcendente ou da sciência das coisas absolutas.
\section{Heterotipia}
\begin{itemize}
\item {Grp. gram.:f.}
\end{itemize}
Carácter dos órgãos heterótipos.
\section{Heterotípico}
\begin{itemize}
\item {Grp. gram.:adj.}
\end{itemize}
Relativo á heterotipia.
\section{Heterótipo}
\begin{itemize}
\item {Grp. gram.:adj.}
\end{itemize}
\begin{itemize}
\item {Proveniência:(Do gr. \textunderscore heteros\textunderscore  + \textunderscore tupos\textunderscore )}
\end{itemize}
Que é diverso ou tem tipo diferente.
Diz-se dos monstros duplos, em que o indivíduo secundário está suspenso na parte anterior do principal.
\section{Heterótomo}
\begin{itemize}
\item {Grp. gram.:adj.}
\end{itemize}
\begin{itemize}
\item {Utilização:Bot.}
\end{itemize}
\begin{itemize}
\item {Proveniência:(Do gr. \textunderscore heteros\textunderscore  + \textunderscore tome\textunderscore )}
\end{itemize}
Cujas divisões ou secções não têm fórma igual.
\section{Heterotopia}
\begin{itemize}
\item {Grp. gram.:f.}
\end{itemize}
\begin{itemize}
\item {Utilização:Anat.}
\end{itemize}
\begin{itemize}
\item {Proveniência:(Do gr. \textunderscore heteros\textunderscore  + \textunderscore topos\textunderscore )}
\end{itemize}
\textunderscore Heterotopia plâstica\textunderscore , formação de tecidos simples ou compostos, em lugares do corpo, onde normalmente se não encontram.
\section{Heterótropo}
\begin{itemize}
\item {Grp. gram.:adj.}
\end{itemize}
\begin{itemize}
\item {Utilização:Bot.}
\end{itemize}
\begin{itemize}
\item {Proveniência:(Do gr. \textunderscore heteros\textunderscore  + \textunderscore trope\textunderscore )}
\end{itemize}
Diz-se do embryão, cuja radícula está desviada do hilo, mas não opposta.
\section{Heterotypia}
\begin{itemize}
\item {Grp. gram.:f.}
\end{itemize}
Carácter dos órgãos heterótypos.
\section{Heterotýpico}
\begin{itemize}
\item {Grp. gram.:adj.}
\end{itemize}
Relativo á heterotypia.
\section{Heterótypo}
\begin{itemize}
\item {Grp. gram.:adj.}
\end{itemize}
\begin{itemize}
\item {Proveniência:(Do gr. \textunderscore heteros\textunderscore  + \textunderscore tupos\textunderscore )}
\end{itemize}
Que é diverso ou tem typo differente.
Diz-se dos monstros duplos, em que o indivíduo secundário está suspenso na parte anterior do principal.
\section{Heterozoário}
\begin{itemize}
\item {Grp. gram.:m.}
\end{itemize}
\begin{itemize}
\item {Proveniência:(Do gr. \textunderscore heteros\textunderscore  + \textunderscore zoon\textunderscore )}
\end{itemize}
(V.espongiários)
\section{Heteus}
\begin{itemize}
\item {Grp. gram.:m. pl.}
\end{itemize}
\begin{itemize}
\item {Proveniência:(Lat. \textunderscore Hethaei\textunderscore )}
\end{itemize}
Antigo povo de Canaan, estabelecido primitivamente ao Sul da Palestina e depois ao Norte, nas vizinhanças de Betel.
\section{Hetheus}
\begin{itemize}
\item {Grp. gram.:m. pl.}
\end{itemize}
\begin{itemize}
\item {Proveniência:(Lat. \textunderscore Hethaei\textunderscore )}
\end{itemize}
Antigo povo de Canaan, estabelecido primitivamente ao Sul da Palestina e depois ao Norte, nas vizinhanças de Bethel.
\section{Hétman}
\begin{itemize}
\item {Grp. gram.:m.}
\end{itemize}
\begin{itemize}
\item {Proveniência:(T. russo)}
\end{itemize}
Dignitário, entre os Cosacos.
\section{Hetol}
\begin{itemize}
\item {Grp. gram.:m.}
\end{itemize}
\begin{itemize}
\item {Utilização:Pharm.}
\end{itemize}
Cinamato de soda, applicado contra a tuberculose.
\section{Heu}
\begin{itemize}
\item {Grp. gram.:m.}
\end{itemize}
\begin{itemize}
\item {Utilização:P. us.}
\end{itemize}
\begin{itemize}
\item {Proveniência:(Lat. \textunderscore heu\textunderscore )}
\end{itemize}
Lamentação; canto fúnebre.
\section{Heureca!}
\begin{itemize}
\item {Grp. gram.:interj.}
\end{itemize}
Já achei!
(Expressão attribuída a Archimedes, ao descobrir o pêso específico dos corpos)
\section{Heurística}
\begin{itemize}
\item {Grp. gram.:f.}
\end{itemize}
\begin{itemize}
\item {Proveniência:(Do gr. \textunderscore heuristiké\textunderscore )}
\end{itemize}
Arte de inventar, de fazer descobrimentos.
\section{Heurístico}
\begin{itemize}
\item {Grp. gram.:adj.}
\end{itemize}
Relativo á heurística.
\section{Hévea}
\begin{itemize}
\item {Grp. gram.:f.}
\end{itemize}
Árvore americana, que produz o catechu.
\section{Heveena}
\begin{itemize}
\item {Grp. gram.:f.}
\end{itemize}
\begin{itemize}
\item {Proveniência:(De \textunderscore hévea\textunderscore )}
\end{itemize}
Substância, descoberta nos productos da destillação do catechu.
\section{Hexa...}
\begin{itemize}
\item {fónica:csa}
\end{itemize}
\begin{itemize}
\item {Grp. gram.:pref.}
\end{itemize}
\begin{itemize}
\item {Proveniência:(Do gr. \textunderscore hex\textunderscore )}
\end{itemize}
(designativo de \textunderscore seis\textunderscore )
\section{Hexacantho}
\begin{itemize}
\item {Grp. gram.:adj.}
\end{itemize}
\begin{itemize}
\item {Utilização:Zool.}
\end{itemize}
\begin{itemize}
\item {Proveniência:(Do gr. \textunderscore hex\textunderscore  + \textunderscore akantha\textunderscore )}
\end{itemize}
Que tem seis espinhos ou aguilhões.
\section{Hexacanto}
\begin{itemize}
\item {Grp. gram.:adj.}
\end{itemize}
\begin{itemize}
\item {Utilização:Zool.}
\end{itemize}
\begin{itemize}
\item {Proveniência:(Do gr. \textunderscore hex\textunderscore  + \textunderscore akantha\textunderscore )}
\end{itemize}
Que tem seis espinhos ou aguilhões.
\section{Hexaciclo}
\begin{itemize}
\item {fónica:csa}
\end{itemize}
\begin{itemize}
\item {Grp. gram.:adj.}
\end{itemize}
\begin{itemize}
\item {Proveniência:(Do gr. \textunderscore hex\textunderscore  + \textunderscore kuklos\textunderscore )}
\end{itemize}
Que tem seis rodas.
\section{Hexacorália}
\begin{itemize}
\item {fónica:csa}
\end{itemize}
\begin{itemize}
\item {Grp. gram.:f.}
\end{itemize}
\begin{itemize}
\item {Proveniência:(De \textunderscore hexa...\textunderscore  + \textunderscore coral\textunderscore )}
\end{itemize}
Uma das três ordens dos coraliários, que compreende as anêmonas do mar, o coral preto do Mediterrâneo e os madreporários.
\section{Hexacorállia}
\begin{itemize}
\item {fónica:csa}
\end{itemize}
\begin{itemize}
\item {Grp. gram.:f.}
\end{itemize}
\begin{itemize}
\item {Proveniência:(De \textunderscore hexa...\textunderscore  + \textunderscore coral\textunderscore )}
\end{itemize}
Uma das três ordens dos coralliários, que comprehende as anêmonas do mar, o coral preto do Mediterrâneo e os madreporários.
\section{Hexacorde}
\begin{itemize}
\item {fónica:csa}
\end{itemize}
\begin{itemize}
\item {Grp. gram.:m.}
\end{itemize}
\begin{itemize}
\item {Proveniência:(Lat. \textunderscore hexachordos\textunderscore )}
\end{itemize}
Escala de seis notas, no cantochão.
Instrumento de seis cordas.
\section{Hexacórdio}
\begin{itemize}
\item {fónica:csa}
\end{itemize}
\begin{itemize}
\item {Grp. gram.:m.}
\end{itemize}
O mesmo que \textunderscore hexacorde\textunderscore .
\section{Hexacyclo}
\begin{itemize}
\item {fónica:csa}
\end{itemize}
\begin{itemize}
\item {Grp. gram.:adj.}
\end{itemize}
\begin{itemize}
\item {Proveniência:(Do gr. \textunderscore hex\textunderscore  + \textunderscore kuklos\textunderscore )}
\end{itemize}
Que tem seis rodas.
\section{Hexadáctilo}
\begin{itemize}
\item {fónica:csa}
\end{itemize}
\begin{itemize}
\item {Grp. gram.:adj.}
\end{itemize}
\begin{itemize}
\item {Proveniência:(Do gr. \textunderscore hex\textunderscore  + \textunderscore daktulos\textunderscore )}
\end{itemize}
Que tem seis dedos.
\section{Hexadáctylo}
\begin{itemize}
\item {fónica:csa}
\end{itemize}
\begin{itemize}
\item {Grp. gram.:adj.}
\end{itemize}
\begin{itemize}
\item {Proveniência:(Do gr. \textunderscore hex\textunderscore  + \textunderscore daktulos\textunderscore )}
\end{itemize}
Que tem seis dedos.
\section{Hexaédrico}
\begin{itemize}
\item {fónica:csa}
\end{itemize}
\begin{itemize}
\item {Grp. gram.:adj.}
\end{itemize}
Relativo ao hexaédro.
\section{Hexaédro}
\begin{itemize}
\item {fónica:csa}
\end{itemize}
\begin{itemize}
\item {Grp. gram.:adj.}
\end{itemize}
\begin{itemize}
\item {Utilização:Geom.}
\end{itemize}
\begin{itemize}
\item {Grp. gram.:M.}
\end{itemize}
\begin{itemize}
\item {Proveniência:(Do gr. \textunderscore hex\textunderscore  + \textunderscore edra\textunderscore )}
\end{itemize}
Que tem seis faces.
Corpo regular de seis faces, cada uma das quaes é um quadrado.
\section{Hexafilo}
\begin{itemize}
\item {fónica:csa}
\end{itemize}
\begin{itemize}
\item {Grp. gram.:adj.}
\end{itemize}
\begin{itemize}
\item {Proveniência:(Do gr. \textunderscore hex\textunderscore  + \textunderscore phullon\textunderscore )}
\end{itemize}
Que tem seis fôlhas ou folíolos.
\section{Hexáforo}
\begin{itemize}
\item {fónica:csa}
\end{itemize}
\begin{itemize}
\item {Grp. gram.:m.}
\end{itemize}
\begin{itemize}
\item {Proveniência:(Lat. \textunderscore hexaphorum\textunderscore )}
\end{itemize}
Liteira, conduzida por seis escravos, entre os antigos Gregos e Romanos.
\section{Hexáforo}
\begin{itemize}
\item {fónica:csá}
\end{itemize}
\begin{itemize}
\item {Grp. gram.:m.}
\end{itemize}
\begin{itemize}
\item {Proveniência:(Lat. \textunderscore hexaphoros\textunderscore )}
\end{itemize}
Cada um dos seis escravos que conduziam uma liteira.
\section{Hexagínia}
\begin{itemize}
\item {fónica:csa}
\end{itemize}
\begin{itemize}
\item {Grp. gram.:f.}
\end{itemize}
\begin{itemize}
\item {Utilização:Bot.}
\end{itemize}
\begin{itemize}
\item {Proveniência:(De \textunderscore hexágino\textunderscore )}
\end{itemize}
Classe de plantas com seis pistilos.
\section{Hexágino}
\begin{itemize}
\item {fónica:csa}
\end{itemize}
\begin{itemize}
\item {Grp. gram.:adj.}
\end{itemize}
\begin{itemize}
\item {Utilização:Bot.}
\end{itemize}
\begin{itemize}
\item {Proveniência:(Do gr. \textunderscore hex\textunderscore  + \textunderscore gune\textunderscore )}
\end{itemize}
Que tem seis pistilos.
\section{Hexagonal}
\begin{itemize}
\item {fónica:csa}
\end{itemize}
\begin{itemize}
\item {Grp. gram.:adj.}
\end{itemize}
Relativo ao hexágono; que tem por base um hexágono.
\section{Hexágono}
\begin{itemize}
\item {fónica:csa}
\end{itemize}
\begin{itemize}
\item {Grp. gram.:adj.}
\end{itemize}
\begin{itemize}
\item {Utilização:Geom.}
\end{itemize}
\begin{itemize}
\item {Grp. gram.:F.}
\end{itemize}
\begin{itemize}
\item {Proveniência:(Do lat. \textunderscore hexagonum\textunderscore )}
\end{itemize}
Que tem seis ângulos e seis lados.
Figura geométrica, que tem seis ângulos e seis lados.
\section{Hexagrama}
\begin{itemize}
\item {fónica:csa}
\end{itemize}
\begin{itemize}
\item {Grp. gram.:m.}
\end{itemize}
\begin{itemize}
\item {Proveniência:(Do gr. \textunderscore hex\textunderscore  + \textunderscore gramma\textunderscore )}
\end{itemize}
Reunião de seis letras ou caracteres.
\section{Hexagramma}
\begin{itemize}
\item {fónica:csa}
\end{itemize}
\begin{itemize}
\item {Grp. gram.:m.}
\end{itemize}
\begin{itemize}
\item {Proveniência:(Do gr. \textunderscore hex\textunderscore  + \textunderscore gramma\textunderscore )}
\end{itemize}
Reunião de seis letras ou caracteres.
\section{Hexagynia}
\begin{itemize}
\item {fónica:csa}
\end{itemize}
\begin{itemize}
\item {Grp. gram.:f.}
\end{itemize}
\begin{itemize}
\item {Utilização:Bot.}
\end{itemize}
\begin{itemize}
\item {Proveniência:(De \textunderscore hexágyno\textunderscore )}
\end{itemize}
Classe de plantas com seis pistillos.
\section{Hexágyno}
\begin{itemize}
\item {fónica:csa}
\end{itemize}
\begin{itemize}
\item {Grp. gram.:adj.}
\end{itemize}
\begin{itemize}
\item {Utilização:Bot.}
\end{itemize}
\begin{itemize}
\item {Proveniência:(Do gr. \textunderscore hex\textunderscore  + \textunderscore gune\textunderscore )}
\end{itemize}
Que tem seis pistillos.
\section{Hexalépido}
\begin{itemize}
\item {fónica:csa}
\end{itemize}
\begin{itemize}
\item {Grp. gram.:adj.}
\end{itemize}
\begin{itemize}
\item {Utilização:Bot.}
\end{itemize}
\begin{itemize}
\item {Proveniência:(Do gr. \textunderscore hexa\textunderscore  + \textunderscore lepis\textunderscore )}
\end{itemize}
Diz-se do invólucro de certas synanthéreas, quando é formado de seis escamas.
\section{Hexâmetro}
\begin{itemize}
\item {fónica:csa}
\end{itemize}
\begin{itemize}
\item {Grp. gram.:m.  e  adj.}
\end{itemize}
\begin{itemize}
\item {Proveniência:(Lat. \textunderscore hexametrus\textunderscore )}
\end{itemize}
Diz-se do verso grego e latino que tem seis pés.
\section{Hexaminas}
\begin{itemize}
\item {fónica:csa}
\end{itemize}
\begin{itemize}
\item {Grp. gram.:f. pl.}
\end{itemize}
\begin{itemize}
\item {Utilização:Chím.}
\end{itemize}
\begin{itemize}
\item {Proveniência:(De \textunderscore hexa...\textunderscore  + \textunderscore amina\textunderscore )}
\end{itemize}
Aminas, formadas de seis moléculas de ammoníaco.
\section{Hexandria}
\begin{itemize}
\item {fónica:csan}
\end{itemize}
\begin{itemize}
\item {Grp. gram.:f.}
\end{itemize}
\begin{itemize}
\item {Proveniência:(De \textunderscore hexandro\textunderscore )}
\end{itemize}
Classe de plantas, cuja flôr tem seis estames.
\section{Hexándrico}
\begin{itemize}
\item {fónica:csan}
\end{itemize}
\begin{itemize}
\item {Grp. gram.:adj.}
\end{itemize}
Relativo á hexandria.
\section{Hexandro}
\begin{itemize}
\item {fónica:csan}
\end{itemize}
\begin{itemize}
\item {Grp. gram.:adj.}
\end{itemize}
\begin{itemize}
\item {Utilização:Bot.}
\end{itemize}
\begin{itemize}
\item {Proveniência:(Do gr. \textunderscore hex\textunderscore  + \textunderscore aner\textunderscore , \textunderscore andros\textunderscore )}
\end{itemize}
Que tem seis estames, livres entre si.
\section{Hexano}
\begin{itemize}
\item {fónica:csa}
\end{itemize}
\begin{itemize}
\item {Grp. gram.:m.}
\end{itemize}
\begin{itemize}
\item {Utilização:Chím.}
\end{itemize}
Variedade de carboneto do grupo formênico.
\section{Hexantéreo}
\begin{itemize}
\item {fónica:csan}
\end{itemize}
\begin{itemize}
\item {Grp. gram.:adj.}
\end{itemize}
\begin{itemize}
\item {Utilização:Bot.}
\end{itemize}
\begin{itemize}
\item {Proveniência:(De \textunderscore hexa...\textunderscore  + \textunderscore anthera\textunderscore )}
\end{itemize}
Que tem seis estames soldados.
\section{Hexanthéreo}
\begin{itemize}
\item {fónica:csan}
\end{itemize}
\begin{itemize}
\item {Grp. gram.:adj.}
\end{itemize}
\begin{itemize}
\item {Utilização:Bot.}
\end{itemize}
\begin{itemize}
\item {Proveniência:(De \textunderscore hexa...\textunderscore  + \textunderscore anthera\textunderscore )}
\end{itemize}
Que tem seis estames soldados.
\section{Hexaoctaédro}
\begin{itemize}
\item {fónica:csa-o}
\end{itemize}
\begin{itemize}
\item {Grp. gram.:m.}
\end{itemize}
\begin{itemize}
\item {Proveniência:(De \textunderscore hexa...\textunderscore  + \textunderscore octaédo\textunderscore )}
\end{itemize}
Polyedro, limitado por 48 triângulos escalenos, iguaes entre si, formando três espécies de ângulos sólidos.
\section{Hexapétalo}
\begin{itemize}
\item {fónica:csa}
\end{itemize}
\begin{itemize}
\item {Grp. gram.:adj.}
\end{itemize}
\begin{itemize}
\item {Proveniência:(De \textunderscore hexa...\textunderscore  + \textunderscore pétala\textunderscore )}
\end{itemize}
Que tem seis pétalas.
\section{Hexáphoro}
\begin{itemize}
\item {fónica:csa}
\end{itemize}
\begin{itemize}
\item {Grp. gram.:m.}
\end{itemize}
\begin{itemize}
\item {Proveniência:(Lat. \textunderscore hexaphorum\textunderscore )}
\end{itemize}
Liteira, conduzida por seis escravos, entre os antigos Gregos e Romanos.
\section{Hexáphoro}
\begin{itemize}
\item {fónica:csá}
\end{itemize}
\begin{itemize}
\item {Grp. gram.:m.}
\end{itemize}
\begin{itemize}
\item {Proveniência:(Lat. \textunderscore hexaphoros\textunderscore )}
\end{itemize}
Cada um dos seis escravos que conduziam uma liteira.
\section{Hexaphyllo}
\begin{itemize}
\item {fónica:csa}
\end{itemize}
\begin{itemize}
\item {Grp. gram.:adj.}
\end{itemize}
\begin{itemize}
\item {Proveniência:(Do gr. \textunderscore hex\textunderscore  + \textunderscore phullon\textunderscore )}
\end{itemize}
Que tem seis fôlhas ou folíolos.
\section{Hexápode}
\begin{itemize}
\item {fónica:csá}
\end{itemize}
\begin{itemize}
\item {Grp. gram.:adj.}
\end{itemize}
\begin{itemize}
\item {Grp. gram.:M. pl.}
\end{itemize}
\begin{itemize}
\item {Proveniência:(Do gr. \textunderscore hex\textunderscore  + \textunderscore pous\textunderscore )}
\end{itemize}
Que tem seis pés.
Insectos ápteros que têm seis pés.
\section{Hexápole}
\begin{itemize}
\item {fónica:csá}
\end{itemize}
\begin{itemize}
\item {Grp. gram.:f.}
\end{itemize}
\begin{itemize}
\item {Proveniência:(Do gr. \textunderscore hex\textunderscore  + \textunderscore polis\textunderscore )}
\end{itemize}
Confederação, composta de seis cidades dóricas, na antiga Grécia.
\section{Hexáptero}
\begin{itemize}
\item {fónica:csá}
\end{itemize}
\begin{itemize}
\item {Grp. gram.:adj.}
\end{itemize}
\begin{itemize}
\item {Utilização:Zool.}
\end{itemize}
\begin{itemize}
\item {Proveniência:(Do gr. \textunderscore hex\textunderscore  + \textunderscore pteron\textunderscore )}
\end{itemize}
Que tem seis asas.
\section{Hexaptoto}
\begin{itemize}
\item {fónica:csá}
\end{itemize}
\begin{itemize}
\item {Grp. gram.:m.  e  adj.}
\end{itemize}
\begin{itemize}
\item {Utilização:Gram.}
\end{itemize}
\begin{itemize}
\item {Proveniência:(Lat. \textunderscore hexaptotos\textunderscore )}
\end{itemize}
Diz-se do nome latino, que tem terminação differente em todos os seis casos, como \textunderscore unus\textunderscore , \textunderscore unius\textunderscore , \textunderscore uni\textunderscore , \textunderscore unum\textunderscore , \textunderscore une\textunderscore , \textunderscore uno\textunderscore .
\section{Hexasíllabo}
\begin{itemize}
\item {fónica:csa}
\end{itemize}
\begin{itemize}
\item {Grp. gram.:adj.}
\end{itemize}
\begin{itemize}
\item {Grp. gram.:M.}
\end{itemize}
\begin{itemize}
\item {Proveniência:(De \textunderscore hexa...\textunderscore  + \textunderscore síllaba\textunderscore )}
\end{itemize}
Que tem seis síllabas.
Verso de seis síllabas; palavra de seis síllabas.
\section{Hexastilo}
\begin{itemize}
\item {fónica:csas}
\end{itemize}
\begin{itemize}
\item {Grp. gram.:m.}
\end{itemize}
\begin{itemize}
\item {Proveniência:(Lat. \textunderscore hexastylos\textunderscore )}
\end{itemize}
Pórtico com seis colunas.
\section{Hexasépalo}
\begin{itemize}
\item {fónica:csá,se}
\end{itemize}
\begin{itemize}
\item {Grp. gram.:adj.}
\end{itemize}
\begin{itemize}
\item {Utilização:Bot.}
\end{itemize}
\begin{itemize}
\item {Proveniência:(De \textunderscore hexa...\textunderscore  + \textunderscore sépala\textunderscore )}
\end{itemize}
Formado de seis sépalas.
\section{Hexaspermo}
\begin{itemize}
\item {fónica:csas}
\end{itemize}
\begin{itemize}
\item {Grp. gram.:adj.}
\end{itemize}
\begin{itemize}
\item {Utilização:Bot.}
\end{itemize}
\begin{itemize}
\item {Proveniência:(Do gr. \textunderscore hex\textunderscore  + \textunderscore sperma\textunderscore )}
\end{itemize}
Que tem seis sementes.
\section{Hexassépalo}
\begin{itemize}
\item {fónica:csá}
\end{itemize}
\begin{itemize}
\item {Grp. gram.:adj.}
\end{itemize}
\begin{itemize}
\item {Utilização:Bot.}
\end{itemize}
\begin{itemize}
\item {Proveniência:(De \textunderscore hexa...\textunderscore  + \textunderscore sépala\textunderscore )}
\end{itemize}
Formado de seis sépalas.
\section{Hexastêmone}
\begin{itemize}
\item {fónica:csas}
\end{itemize}
\begin{itemize}
\item {Grp. gram.:adj.}
\end{itemize}
\begin{itemize}
\item {Utilização:Bot.}
\end{itemize}
\begin{itemize}
\item {Proveniência:(Do gr. \textunderscore hex\textunderscore  + \textunderscore stemon\textunderscore )}
\end{itemize}
Que tem seis estames livres.
\section{Hexástico}
\begin{itemize}
\item {fónica:csás}
\end{itemize}
\begin{itemize}
\item {Grp. gram.:adj.}
\end{itemize}
\begin{itemize}
\item {Grp. gram.:M.}
\end{itemize}
\begin{itemize}
\item {Proveniência:(Lat. \textunderscore hexastichus\textunderscore )}
\end{itemize}
Composto de seis versos.
Composição de seis versos.
\section{Hexastylo}
\begin{itemize}
\item {fónica:csas}
\end{itemize}
\begin{itemize}
\item {Grp. gram.:m.}
\end{itemize}
\begin{itemize}
\item {Proveniência:(Lat. \textunderscore hexastylos\textunderscore )}
\end{itemize}
Pórtico com seis columnas.
\section{Hexasýllabo}
\begin{itemize}
\item {fónica:si}
\end{itemize}
\begin{itemize}
\item {Grp. gram.:adj.}
\end{itemize}
\begin{itemize}
\item {Grp. gram.:M.}
\end{itemize}
\begin{itemize}
\item {Proveniência:(De \textunderscore hexa...\textunderscore  + \textunderscore sýllaba\textunderscore )}
\end{itemize}
Que tem seis sýllabas.
Verso de seis sýllabas; palavra de seis sýllabas.
\section{Hexátomo}
\begin{itemize}
\item {fónica:sá}
\end{itemize}
\begin{itemize}
\item {Grp. gram.:m.}
\end{itemize}
\begin{itemize}
\item {Proveniência:(Do gr. \textunderscore hex\textunderscore  + \textunderscore tomo\textunderscore )}
\end{itemize}
Gênero de insectos dípteros.
\section{Hexecontálitho}
\begin{itemize}
\item {fónica:cse}
\end{itemize}
\begin{itemize}
\item {Grp. gram.:m.}
\end{itemize}
\begin{itemize}
\item {Proveniência:(Lat. \textunderscore hexecontalithos\textunderscore )}
\end{itemize}
Pedra preciosa, hoje desconhecida, e da qual se diz que tinha sessenta côres.
\section{Hexecontálito}
\begin{itemize}
\item {fónica:cse}
\end{itemize}
\begin{itemize}
\item {Grp. gram.:m.}
\end{itemize}
\begin{itemize}
\item {Proveniência:(Lat. \textunderscore hexecontalithos\textunderscore )}
\end{itemize}
Pedra preciosa, hoje desconhecida, e da qual se diz que tinha sessenta côres.
\section{Hexere}
\begin{itemize}
\item {fónica:csê}
\end{itemize}
\begin{itemize}
\item {Grp. gram.:f.}
\end{itemize}
\begin{itemize}
\item {Proveniência:(Lat. \textunderscore hexeris\textunderscore )}
\end{itemize}
Galera grega, com seis ordens de remos.
\section{Hexileno}
\begin{itemize}
\item {fónica:csi}
\end{itemize}
\begin{itemize}
\item {Grp. gram.:m.}
\end{itemize}
\begin{itemize}
\item {Utilização:Chím.}
\end{itemize}
Variedade de carboneto do grupo etilênico.
\section{Hexílico}
\begin{itemize}
\item {fónica:csi}
\end{itemize}
\begin{itemize}
\item {Grp. gram.:adj.}
\end{itemize}
Diz-se de um dos álcooes dos vinhos, cuja fórmula química é C^{6}H^{14}O.
\section{Hexodonte}
\begin{itemize}
\item {fónica:cso}
\end{itemize}
\begin{itemize}
\item {Grp. gram.:m.}
\end{itemize}
\begin{itemize}
\item {Proveniência:(Do gr. \textunderscore hex\textunderscore  + \textunderscore odous\textunderscore , \textunderscore odontos\textunderscore )}
\end{itemize}
Gênero de insectos coleópteros de Madagáscar.
\section{Hexoileno}
\begin{itemize}
\item {fónica:cso-i}
\end{itemize}
\begin{itemize}
\item {Grp. gram.:m.}
\end{itemize}
\begin{itemize}
\item {Utilização:Chím.}
\end{itemize}
Um dos carbonetos do grupo acetilênico.
\section{Hexoyleno}
\begin{itemize}
\item {fónica:cso-i}
\end{itemize}
\begin{itemize}
\item {Grp. gram.:m.}
\end{itemize}
\begin{itemize}
\item {Utilização:Chím.}
\end{itemize}
Um dos carbonetos do grupo acetylênico.
\section{Hexyleno}
\begin{itemize}
\item {fónica:csi}
\end{itemize}
\begin{itemize}
\item {Grp. gram.:m.}
\end{itemize}
\begin{itemize}
\item {Utilização:Chím.}
\end{itemize}
Variedade de carboneto do grupo ethylênico.
\section{Hexýlico}
\begin{itemize}
\item {fónica:csi}
\end{itemize}
\begin{itemize}
\item {Grp. gram.:adj.}
\end{itemize}
Diz-se de um dos álcooes dos vinhos, cuja fórmula chímica é C^{6}H^{14}O.
\section{Hi}
\begin{itemize}
\item {Grp. gram.:adv.}
\end{itemize}
(V.aí)
\section{Hi,-hi,-hi}
Voz imitativa de chôro ou riso.
\section{Hiacinthino}
\begin{itemize}
\item {Grp. gram.:adj.}
\end{itemize}
Relativo ao hiacintho.
\section{Hiacintino}
\begin{itemize}
\item {Grp. gram.:adj.}
\end{itemize}
Relativo ao hiacinto.
\section{Hiacintho}
\begin{itemize}
\item {Grp. gram.:m.}
\end{itemize}
O mesmo que \textunderscore jacintho\textunderscore ^1.
\section{Hiacinto}
\begin{itemize}
\item {Grp. gram.:m.}
\end{itemize}
O mesmo que \textunderscore jacinto\textunderscore ^1.
\section{Hiante}
\begin{itemize}
\item {Grp. gram.:adj.}
\end{itemize}
\begin{itemize}
\item {Utilização:Poét.}
\end{itemize}
\begin{itemize}
\item {Utilização:Fig.}
\end{itemize}
\begin{itemize}
\item {Proveniência:(Lat. \textunderscore hians\textunderscore )}
\end{itemize}
Que tem a bôca aberta.
Que tem grande abertura ou fenda.
Faminto.
\section{Hiapuá}
\begin{itemize}
\item {Grp. gram.:m.}
\end{itemize}
\begin{itemize}
\item {Utilização:Bras. do N}
\end{itemize}
Espécie de mandioca silvestre.
\section{Hiate}
\begin{itemize}
\item {Grp. gram.:m.}
\end{itemize}
Embarcação costeira, que apparelha com dois latinos e duas velas de prôa.
(É t. de formação arbitrária, em que só o uso explica o \textunderscore h\textunderscore  inicial, e que procedeu directamente do ingl. \textunderscore yacht\textunderscore , cuja etym. é o hol. \textunderscore iaecht\textunderscore  ou \textunderscore jachten\textunderscore )
\section{Hiato}
\begin{itemize}
\item {Grp. gram.:m.}
\end{itemize}
\begin{itemize}
\item {Utilização:Fig.}
\end{itemize}
\begin{itemize}
\item {Proveniência:(Lat. \textunderscore hiatus\textunderscore )}
\end{itemize}
Encontro de duas vogaes no fim de uma palavra e princípio de outra, ou no meio de uma palavra, se as vogaes não formam ditongo: \textunderscore á alma\textunderscore ; \textunderscore é echo\textunderscore .
Orifício ou fenda no corpo humano.
Fenda na terra.
Espaço entre dois lábios da corolla.
Lacuna.
\section{Hiava}
\begin{itemize}
\item {Grp. gram.:f.}
\end{itemize}
Gênero de árvores da Guiana inglesa.
\section{Hibernação}
\begin{itemize}
\item {Grp. gram.:f.}
\end{itemize}
\begin{itemize}
\item {Proveniência:(Lat. \textunderscore hibernatio\textunderscore )}
\end{itemize}
Entorpecimento ou somno lethárgico de certos animaes, durante o inverno.
\section{Hibernáculo}
\begin{itemize}
\item {Grp. gram.:m.}
\end{itemize}
\begin{itemize}
\item {Proveniência:(Lat. \textunderscore hibernaculum\textunderscore )}
\end{itemize}
Parte de um vegetal, que lhe resguarda os gomos, protegendo-os contra o frio.
Arraial de inverno, entre os antigos Romanos. Cf. Herculano, \textunderscore Eurico\textunderscore , 241.
\section{Hibernal}
\begin{itemize}
\item {Grp. gram.:adj.}
\end{itemize}
\begin{itemize}
\item {Proveniência:(Lat. \textunderscore híbernalis\textunderscore )}
\end{itemize}
Relativo ao inverno.
\section{Hibernante}
\begin{itemize}
\item {Grp. gram.:adj.}
\end{itemize}
\begin{itemize}
\item {Proveniência:(Lat. \textunderscore hibernans\textunderscore )}
\end{itemize}
Que hiberna.
\section{Hibernar}
\begin{itemize}
\item {Grp. gram.:v. i.}
\end{itemize}
\begin{itemize}
\item {Proveniência:(Lat. \textunderscore hibernare\textunderscore )}
\end{itemize}
Estar em hibernação.
\section{Hibérnia}
\begin{itemize}
\item {Grp. gram.:f.}
\end{itemize}
\begin{itemize}
\item {Proveniência:(Do lat. \textunderscore hibernus\textunderscore )}
\end{itemize}
Gênero de insectos lepidópteros nocturnos.
\section{Hibérnico}
\begin{itemize}
\item {Grp. gram.:adj.}
\end{itemize}
\begin{itemize}
\item {Grp. gram.:M.}
\end{itemize}
\begin{itemize}
\item {Proveniência:(Lat. \textunderscore hibernicus\textunderscore )}
\end{itemize}
Relativo á Hibérnia, hoje Irlanda.
Irlandês.
Antiga língua da Irlanda.
\section{Hibérnio}
\begin{itemize}
\item {Grp. gram.:m.}
\end{itemize}
O mesmo que \textunderscore hibérnico\textunderscore , \textunderscore m.\textunderscore 
\section{Hibérnios}
\begin{itemize}
\item {Grp. gram.:m. pl.}
\end{itemize}
\begin{itemize}
\item {Proveniência:(De \textunderscore Hibérnia\textunderscore , n. p.)}
\end{itemize}
Antigos habitadores da Irlanda.
\section{Hiberno}
\begin{itemize}
\item {Grp. gram.:adj.}
\end{itemize}
\begin{itemize}
\item {Proveniência:(Lat. \textunderscore hibernus\textunderscore )}
\end{itemize}
O mesmo que \textunderscore hibernal\textunderscore .
\section{Hibísceas}
\begin{itemize}
\item {Grp. gram.:f. pl.}
\end{itemize}
Tríbo de plantas, que tem por typo o hibisco.
\section{Hibisco}
\begin{itemize}
\item {Grp. gram.:m.}
\end{itemize}
\begin{itemize}
\item {Proveniência:(Lat. \textunderscore hibiscus\textunderscore )}
\end{itemize}
Gênero de plantas malváceas, a que pertence a rosa-da-china.
\section{Hicungo-miapia}
\begin{itemize}
\item {Grp. gram.:m.}
\end{itemize}
Pássaro fissirostro da África occidental.
\section{Hidrótico}
\begin{itemize}
\item {Grp. gram.:adj.}
\end{itemize}
\begin{itemize}
\item {Proveniência:(Gr. \textunderscore hidrotikos\textunderscore )}
\end{itemize}
Que provoca o suor; sudorífico.
\section{Hiemação}
\begin{itemize}
\item {Grp. gram.:f.}
\end{itemize}
\begin{itemize}
\item {Proveniência:(Lat. \textunderscore híematio\textunderscore )}
\end{itemize}
Acto de hibernar.
Propriedade que certas plantas têm, de crescer no inverno.
\section{Hiemal}
\begin{itemize}
\item {Grp. gram.:adj.}
\end{itemize}
\begin{itemize}
\item {Proveniência:(Lat. \textunderscore hiemalis\textunderscore )}
\end{itemize}
O mesmo que \textunderscore hibernal\textunderscore .
\section{Hierácio}
\begin{itemize}
\item {Grp. gram.:m.}
\end{itemize}
\begin{itemize}
\item {Proveniência:(Lat. \textunderscore hieracium\textunderscore )}
\end{itemize}
Espécie de collýrio antigo, feito com o suco de leituga.
\section{Hieracite}
\begin{itemize}
\item {Grp. gram.:f.}
\end{itemize}
\begin{itemize}
\item {Proveniência:(Gr. \textunderscore hierakitis\textunderscore )}
\end{itemize}
Pedra preciosa, hoje desconhecida, da qual se dizia que o seu contacto era medicinal contra as hemorroidas.
\section{Hieranose}
\begin{itemize}
\item {Grp. gram.:f.}
\end{itemize}
\begin{itemize}
\item {Proveniência:(Do gr. \textunderscore hieros\textunderscore  + \textunderscore nosos\textunderscore )}
\end{itemize}
(V.epilepsia)
\section{Hierarchia}
\begin{itemize}
\item {fónica:qui}
\end{itemize}
\begin{itemize}
\item {Utilização:Poét.}
\end{itemize}
\textunderscore f.\textunderscore  (e der.)
O mesmo que \textunderscore jerarchia\textunderscore , etc.
Formosura e pureza angélica:«\textunderscore vós, minha hierarchia...\textunderscore »Camões, ode III.
\section{Hierarquia}
\begin{itemize}
\item {Utilização:Poét.}
\end{itemize}
\textunderscore f.\textunderscore  (e der.)
O mesmo que \textunderscore jerarquia\textunderscore , etc.
Formosura e pureza angélica:«\textunderscore vós, minha hierarquia...\textunderscore »Camões, ode III.
\section{Hierática}
\begin{itemize}
\item {Grp. gram.:f.}
\end{itemize}
\begin{itemize}
\item {Proveniência:(De \textunderscore hierático\textunderscore )}
\end{itemize}
Espécie de papel finíssimo, que só se empregava na escrita dos livros sagrados.
\section{Hierático}
\begin{itemize}
\item {Grp. gram.:adj.}
\end{itemize}
\begin{itemize}
\item {Proveniência:(Lat. \textunderscore hieraticus\textunderscore )}
\end{itemize}
Relativo ás coisas sagradas.
Religioso.
\section{Hiérnios}
\begin{itemize}
\item {Grp. gram.:m. pl.}
\end{itemize}
O mesmo que \textunderscore hibérnios\textunderscore .
\section{Hiero...}
\begin{itemize}
\item {Proveniência:(Gr. \textunderscore hieros\textunderscore )}
\end{itemize}
Elemento, que entra na formação de várias palavras, com a significação de \textunderscore sagrado\textunderscore .
\section{Hierodrama}
\begin{itemize}
\item {Grp. gram.:m.}
\end{itemize}
\begin{itemize}
\item {Proveniência:(Do gr. \textunderscore hieros\textunderscore  + \textunderscore drama\textunderscore )}
\end{itemize}
Representação scênica dos feitos de um deus, nos templos pagãos.
\section{Hierodula}
\begin{itemize}
\item {Grp. gram.:f.}
\end{itemize}
\begin{itemize}
\item {Proveniência:(De \textunderscore hierodulo\textunderscore )}
\end{itemize}
Meretriz, que se comprava, para sêr offerecida a Vênus em certas festas, e que, a trôco dos seus encantos, obtinha dinheiro para as despesas dessas festas.
Mulher adstrita ao serviço de um templo grego.
\section{Hierodulo}
\begin{itemize}
\item {Grp. gram.:m.}
\end{itemize}
\begin{itemize}
\item {Proveniência:(Lat. \textunderscore hierodulos\textunderscore )}
\end{itemize}
Escravo, adstrito ao serviço de um templo, na Grécia antiga.
\section{Hierofanta}
\begin{itemize}
\item {Grp. gram.:m.}
\end{itemize}
O mesmo que \textunderscore hierofante\textunderscore .
\section{Hierofante}
\begin{itemize}
\item {Grp. gram.:m.}
\end{itemize}
\begin{itemize}
\item {Utilização:Fig.}
\end{itemize}
\begin{itemize}
\item {Proveniência:(Lat. \textunderscore hierophantes\textunderscore )}
\end{itemize}
Sacerdote, que presidia aos mistérios de Elêusis, na Grécia.
Indivíduo, que se inculca conhecedor de ciências ou de mistérios.
\section{Hierofântide}
\begin{itemize}
\item {Grp. gram.:f.}
\end{itemize}
\begin{itemize}
\item {Proveniência:(Gr. \textunderscore hierophantís\textunderscore )}
\end{itemize}
Sacerdotisa de Ceres, em Atenas, subordinada ao hierofante.
\section{Hieroglifo}
\textunderscore m.\textunderscore  (e der.)
O mesmo que \textunderscore geroglifo\textunderscore , etc.
\section{Hieroglypho}
\textunderscore m.\textunderscore  (e der.)
O mesmo que \textunderscore geroglypho\textunderscore , etc.
\section{Hierografia}
\begin{itemize}
\item {Grp. gram.:f.}
\end{itemize}
\begin{itemize}
\item {Proveniência:(Do gr. \textunderscore hieros\textunderscore  + \textunderscore graphein\textunderscore )}
\end{itemize}
Descripção das coisas sagradas.
\section{Hierográfico}
\begin{itemize}
\item {Grp. gram.:adj.}
\end{itemize}
Relativo á hierografia.
\section{Hierograma}
\begin{itemize}
\item {Grp. gram.:f.}
\end{itemize}
\begin{itemize}
\item {Proveniência:(Do gr. \textunderscore hieros\textunderscore  + \textunderscore gramma\textunderscore )}
\end{itemize}
Grafia hierática.
\section{Hierogramático}
\begin{itemize}
\item {Grp. gram.:adj.}
\end{itemize}
\begin{itemize}
\item {Proveniência:(De \textunderscore hierograma\textunderscore )}
\end{itemize}
Relativo ás escritas sagradas dos Egípcios; hierático.
\section{Hierogramma}
\begin{itemize}
\item {Grp. gram.:f.}
\end{itemize}
\begin{itemize}
\item {Proveniência:(Do gr. \textunderscore hieros\textunderscore  + \textunderscore gramma\textunderscore )}
\end{itemize}
Graphia hierática.
\section{Hierogrammático}
\begin{itemize}
\item {Grp. gram.:adj.}
\end{itemize}
\begin{itemize}
\item {Proveniência:(De \textunderscore hierogramma\textunderscore )}
\end{itemize}
Relativo ás escritas sagradas dos Egýpcios; hierático.
\section{Hierographia}
\begin{itemize}
\item {Grp. gram.:f.}
\end{itemize}
\begin{itemize}
\item {Proveniência:(Do gr. \textunderscore hieros\textunderscore  + \textunderscore graphein\textunderscore )}
\end{itemize}
Descripção das coisas sagradas.
\section{Hierográphico}
\begin{itemize}
\item {Grp. gram.:adj.}
\end{itemize}
Relativo á hierographia.
\section{Hierologia}
\begin{itemize}
\item {Grp. gram.:f.}
\end{itemize}
\begin{itemize}
\item {Proveniência:(Do gr. \textunderscore hieros\textunderscore  + \textunderscore logos\textunderscore )}
\end{itemize}
Estudo das diversas religiões.
\section{Hierológico}
\begin{itemize}
\item {Grp. gram.:adj.}
\end{itemize}
Relativo á hierologia.
\section{Hierónica}
\begin{itemize}
\item {Grp. gram.:m.}
\end{itemize}
\begin{itemize}
\item {Proveniência:(Lat. \textunderscore hieronica\textunderscore )}
\end{itemize}
O mesmo ou melhor que \textunderscore hierónico\textunderscore .
\section{Hierónico}
\begin{itemize}
\item {Grp. gram.:m.}
\end{itemize}
\begin{itemize}
\item {Proveniência:(Gr. \textunderscore hieronikes\textunderscore )}
\end{itemize}
Indivíduo, que saía vencedor de algum dos jogos sagrados da Grécia, olýmpicos, ísthmicos, etc.
\section{Hieronímico}
\begin{itemize}
\item {Grp. gram.:adj.}
\end{itemize}
\begin{itemize}
\item {Proveniência:(Do gr. \textunderscore Hieronumos\textunderscore , n. p.)}
\end{itemize}
Relativo a San-Jerónimo.
\section{Hieronimitas}
\begin{itemize}
\item {Grp. gram.:m. pl.}
\end{itemize}
O mesmo que \textunderscore jerónimos\textunderscore .
(Cp. \textunderscore hieronímico\textunderscore )
\section{Hieronýmico}
\begin{itemize}
\item {Grp. gram.:adj.}
\end{itemize}
\begin{itemize}
\item {Proveniência:(Do gr. \textunderscore Hieronumos\textunderscore , n. p.)}
\end{itemize}
Relativo a San-Jerónymo.
\section{Hieronymitas}
\begin{itemize}
\item {Grp. gram.:m. pl.}
\end{itemize}
O mesmo que \textunderscore jerónymos\textunderscore .
(Cp. \textunderscore hieronýmico\textunderscore )
\section{Hieropeu}
\begin{itemize}
\item {Grp. gram.:m.}
\end{itemize}
\begin{itemize}
\item {Proveniência:(Gr. \textunderscore hieropoios\textunderscore )}
\end{itemize}
Funccionário, que, na antiga Athenas, fiscalizava os sacrifícios públicos.
\section{Hierophanta}
\begin{itemize}
\item {Grp. gram.:m.}
\end{itemize}
O mesmo que \textunderscore hierophante\textunderscore .
\section{Hierophante}
\begin{itemize}
\item {Grp. gram.:m.}
\end{itemize}
\begin{itemize}
\item {Utilização:Fig.}
\end{itemize}
\begin{itemize}
\item {Proveniência:(Lat. \textunderscore hierophantes\textunderscore )}
\end{itemize}
Sacerdote, que presidia aos mystérios de Elêusis, na Grécia.
Indivíduo, que se inculca conhecedor de sciências ou de mystérios.
\section{Hierophântide}
\begin{itemize}
\item {Grp. gram.:f.}
\end{itemize}
\begin{itemize}
\item {Proveniência:(Gr. \textunderscore hierophantís\textunderscore )}
\end{itemize}
Sacerdotisa de Ceres, em Athenas, subordinada ao hierophante.
\section{Hieroscopia}
\begin{itemize}
\item {Grp. gram.:f.}
\end{itemize}
\begin{itemize}
\item {Proveniência:(Do gr. \textunderscore hieros\textunderscore  + \textunderscore skopein\textunderscore )}
\end{itemize}
Antigo systema de adivinhação por meio da inspecção das entranhas das víctimas.
\section{Hierosolimitano}
\begin{itemize}
\item {Grp. gram.:adj.}
\end{itemize}
\begin{itemize}
\item {Grp. gram.:M.}
\end{itemize}
\begin{itemize}
\item {Proveniência:(Lat. \textunderscore hierosolymitanus\textunderscore )}
\end{itemize}
Relativo a Jerusalém.
Aquele que é natural de Jerusalém.
\section{Hierosolymitano}
\begin{itemize}
\item {Grp. gram.:adj.}
\end{itemize}
\begin{itemize}
\item {Grp. gram.:M.}
\end{itemize}
\begin{itemize}
\item {Proveniência:(Lat. \textunderscore hierosolymitanus\textunderscore )}
\end{itemize}
Relativo a Jerusalém.
Aquelle que é natural de Jerusalém.
\section{Hílare}
\begin{itemize}
\item {Grp. gram.:adj.}
\end{itemize}
\begin{itemize}
\item {Utilização:Poét.}
\end{itemize}
\begin{itemize}
\item {Proveniência:(Lat. \textunderscore hilaris\textunderscore )}
\end{itemize}
Contente; risonho; folgazão.
\section{Hilária}
\begin{itemize}
\item {Grp. gram.:f.}
\end{itemize}
\begin{itemize}
\item {Proveniência:(De \textunderscore Saint-Hílaire\textunderscore , n. p.)}
\end{itemize}
Planta vivaz, da fam. das gramíneas.
\section{Hilariante}
\begin{itemize}
\item {Grp. gram.:adj.}
\end{itemize}
\begin{itemize}
\item {Proveniência:(De \textunderscore hílare\textunderscore )}
\end{itemize}
Que produz alegria.
Que tem alegria.
\textunderscore Gás hilariante\textunderscore , o mesmo que \textunderscore protóxydo de azoto\textunderscore .
\section{Hilárias}
\begin{itemize}
\item {Grp. gram.:f. pl.}
\end{itemize}
\begin{itemize}
\item {Proveniência:(Lat. \textunderscore hilaria\textunderscore )}
\end{itemize}
Festas em honra de Cybele, entre os antigos Gregos e Romanos, no equinóccio da primavera.
\section{Hilariedade}
\begin{itemize}
\item {Grp. gram.:f.}
\end{itemize}
\begin{itemize}
\item {Proveniência:(Lat. \textunderscore hilaritas\textunderscore )}
\end{itemize}
Alegria.
Folguedo.
Vontade de rir.
\section{Hilário}
\begin{itemize}
\item {Grp. gram.:adj.}
\end{itemize}
Relativo ao hilo.
\section{Hilarizar}
\begin{itemize}
\item {Grp. gram.:v. t.}
\end{itemize}
\begin{itemize}
\item {Utilização:P. us.}
\end{itemize}
\begin{itemize}
\item {Proveniência:(Lat. \textunderscore hilarissare\textunderscore )}
\end{itemize}
Tornar hílare.
Dar alegria a; alegrar.
\section{Hilaródia}
\begin{itemize}
\item {Grp. gram.:f.}
\end{itemize}
Versos, cantados pelos hilarodos.
\section{Hilarodos}
\begin{itemize}
\item {Grp. gram.:m. pl.}
\end{itemize}
\begin{itemize}
\item {Proveniência:(Lat. \textunderscore hilarodos\textunderscore )}
\end{itemize}
Os cantores de poesias lascivas, entre os antigos Gregos e Romanos.
\section{Hilo}
\begin{itemize}
\item {Grp. gram.:m.}
\end{itemize}
\begin{itemize}
\item {Proveniência:(Lat. \textunderscore hilum\textunderscore )}
\end{itemize}
Cicatriz exterior da semente, no ponto em que esta adheria á placenta.
Ponto, geralmente deprimido, em que uma víscera parenchymatosa recebe os seus vasos.
\section{Hilófero}
\begin{itemize}
\item {Grp. gram.:m.}
\end{itemize}
\begin{itemize}
\item {Proveniência:(Do lat. \textunderscore hilum\textunderscore  + \textunderscore ferre\textunderscore )}
\end{itemize}
O mesmo que \textunderscore endopleura\textunderscore .
\section{Him}
Voz onom., que significa o rincho da mula.
\section{Himalaico}
\begin{itemize}
\item {Grp. gram.:adj.}
\end{itemize}
Relativo ao monte Himalaia. Cf. D. Lopes, \textunderscore Chrón. dos Reis de Bisnaga\textunderscore , IX.
\section{Himanthophyllo}
\begin{itemize}
\item {Grp. gram.:m.}
\end{itemize}
Planta de jardim.
\section{Himantofilo}
\begin{itemize}
\item {Grp. gram.:m.}
\end{itemize}
Planta de jardim.
\section{Himba}
\begin{itemize}
\item {Grp. gram.:f.}
\end{itemize}
Pássaro dentirostro da África occidental.
\section{Himiárico}
\begin{itemize}
\item {Grp. gram.:m.}
\end{itemize}
\begin{itemize}
\item {Proveniência:(De \textunderscore Himíar\textunderscore , n. p.)}
\end{itemize}
Lingua da Abyssínia; o abexim.
\section{Himiarítico}
\begin{itemize}
\item {Grp. gram.:m.}
\end{itemize}
\begin{itemize}
\item {Proveniência:(De \textunderscore Himíar\textunderscore , n. p.)}
\end{itemize}
Lingua da Abyssínia; o abexim.
\section{Hindu}
\begin{itemize}
\item {Grp. gram.:m.  e  adj.}
\end{itemize}
(V.indu)
\section{Hindustano}
\begin{itemize}
\item {Grp. gram.:m.}
\end{itemize}
(V.indostano)
\section{Hinidor}
\begin{itemize}
\item {Grp. gram.:adj.}
\end{itemize}
\begin{itemize}
\item {Proveniência:(Do lat. \textunderscore hinnire\textunderscore )}
\end{itemize}
Que rincha. Cf. Filinto, I, 232.
\section{Hinnidor}
\begin{itemize}
\item {Grp. gram.:adj.}
\end{itemize}
\begin{itemize}
\item {Proveniência:(Do lat. \textunderscore hinnire\textunderscore )}
\end{itemize}
Que rincha. Cf. Filinto, I, 232.
\section{Hió}
\begin{itemize}
\item {Grp. gram.:m.}
\end{itemize}
Árvore indiana, de fibras têxteis.
\section{Hipacaça}
\begin{itemize}
\item {Grp. gram.:m.}
\end{itemize}
Corpulento ruminante de Angola.
\section{Hipantropia}
\begin{itemize}
\item {Grp. gram.:f.}
\end{itemize}
\begin{itemize}
\item {Proveniência:(Do gr. \textunderscore hippos\textunderscore  + \textunderscore anthropos\textunderscore )}
\end{itemize}
Doença mental dos indivíduos, que se julgam transformados em cavalos.
\section{Hiparca}
\begin{itemize}
\item {Grp. gram.:m.}
\end{itemize}
\begin{itemize}
\item {Proveniência:(Do gr. \textunderscore hippos\textunderscore  + \textunderscore arkhein\textunderscore )}
\end{itemize}
Designação dos generaes de cavalaria, entre os antigos Gregos.
\section{Hiparco}
\begin{itemize}
\item {Grp. gram.:m.}
\end{itemize}
O mesmo que \textunderscore hiparca\textunderscore . Cf. Latino, \textunderscore Or. da Corôa\textunderscore , XL.
\section{Hipélafo}
\begin{itemize}
\item {Grp. gram.:m.}
\end{itemize}
\begin{itemize}
\item {Proveniência:(Gr. \textunderscore hippelaphos\textunderscore )}
\end{itemize}
Nome de um veado de Samatra e Java.
Nome, dado por Cuvier a outro veado da India.
\section{Hipiatra}
\begin{itemize}
\item {Grp. gram.:m.}
\end{itemize}
\begin{itemize}
\item {Proveniência:(Do gr. \textunderscore hippos\textunderscore  + \textunderscore iatros\textunderscore )}
\end{itemize}
O mesmo que \textunderscore veterinário\textunderscore .
\section{Hipiatria}
\begin{itemize}
\item {Grp. gram.:f.}
\end{itemize}
\begin{itemize}
\item {Utilização:Ext.}
\end{itemize}
\begin{itemize}
\item {Proveniência:(De \textunderscore hipiátrico\textunderscore )}
\end{itemize}
Medicina veterinária, que trata especialmente dos cavalos.
Aquilo que diz respeito a cavalos.
\section{Hipiátrica}
\begin{itemize}
\item {Grp. gram.:f.}
\end{itemize}
\begin{itemize}
\item {Utilização:Ext.}
\end{itemize}
\begin{itemize}
\item {Proveniência:(De \textunderscore hipiátrico\textunderscore )}
\end{itemize}
Medicina veterinária, que trata especialmente dos cavalos.
Aquilo que diz respeito a cavalos.
\section{Hipiátrico}
\begin{itemize}
\item {Grp. gram.:adj.}
\end{itemize}
\begin{itemize}
\item {Proveniência:(Gr. \textunderscore hippiatrikos\textunderscore )}
\end{itemize}
Relativo á hipiátrica.
\section{Hípico}
\begin{itemize}
\item {Grp. gram.:adj.}
\end{itemize}
\begin{itemize}
\item {Proveniência:(Gr. \textunderscore hippikos\textunderscore )}
\end{itemize}
Relativo a cavalos.
\section{Hipiscafo}
\begin{itemize}
\item {Grp. gram.:m.}
\end{itemize}
\begin{itemize}
\item {Utilização:Ant.}
\end{itemize}
Embarcação, destinada especialmente ao transporte de cavalos.
\section{Hipismo}
\begin{itemize}
\item {Grp. gram.:m.}
\end{itemize}
\begin{itemize}
\item {Utilização:Neol.}
\end{itemize}
\begin{itemize}
\item {Proveniência:(Do gr. \textunderscore hippos\textunderscore )}
\end{itemize}
O desporte das corridas de cavalos.
\section{Hipnala}
\begin{itemize}
\item {Grp. gram.:f.}
\end{itemize}
Serpente da Ásia, (\textunderscore boa hipnale\textunderscore , Lin.).
\section{Hipo...}
\begin{itemize}
\item {Grp. gram.:pref.}
\end{itemize}
\begin{itemize}
\item {Proveniência:(Do gr. \textunderscore hippos\textunderscore )}
\end{itemize}
(designativo de \textunderscore cavallo\textunderscore )
\section{Hipoboscos}
\begin{itemize}
\item {Grp. gram.:m. pl.}
\end{itemize}
\begin{itemize}
\item {Proveniência:(Do gr. \textunderscore hippos\textunderscore  + \textunderscore bosko\textunderscore )}
\end{itemize}
Gênero de insectos dípteros.
\section{Hipocampo}
\begin{itemize}
\item {Grp. gram.:m.}
\end{itemize}
\begin{itemize}
\item {Proveniência:(Lat. \textunderscore hippocampus\textunderscore )}
\end{itemize}
Gênero de peixes marítimos, a que também se dá o nome de \textunderscore cavalo-marinho\textunderscore , por causa da fórma da cabeça e do arqueado do corpo.
Nome de eminências, nos ventrículos do cérebro.
\section{Hipocastâneas}
\begin{itemize}
\item {Grp. gram.:f. pl.}
\end{itemize}
\begin{itemize}
\item {Proveniência:(Do gr. \textunderscore hippos\textunderscore  + \textunderscore kastana\textunderscore )}
\end{itemize}
Família de plantas, que têm por tipo o castanheiro da Índia.
\section{Hipocentauro}
\begin{itemize}
\item {Grp. gram.:m.}
\end{itemize}
\begin{itemize}
\item {Proveniência:(Lat. \textunderscore hippocentaurus\textunderscore )}
\end{itemize}
O mesmo que \textunderscore centauro\textunderscore .
\section{Hipocratismo}
\begin{itemize}
\item {Grp. gram.:m.}
\end{itemize}
A doutrina médica de Hipócrates. Cf. Latino, \textunderscore Or. da Corôa\textunderscore , CLXXXV.
\section{Hipocrático}
\begin{itemize}
\item {Grp. gram.:adj.}
\end{itemize}
Relativo a Hipócrates ou á sua doutrina.
\section{Hipocraz}
\begin{itemize}
\item {Grp. gram.:m.}
\end{itemize}
\begin{itemize}
\item {Proveniência:(Do rad. de \textunderscore hipocrático\textunderscore )}
\end{itemize}
Infusão de canela, açúcar, etc., em vinho.
\section{Hipodromia}
\begin{itemize}
\item {Grp. gram.:f.}
\end{itemize}
\begin{itemize}
\item {Proveniência:(De \textunderscore hipódromo\textunderscore )}
\end{itemize}
Arte de dirigir corridas de cavalos.
Arte de correr a cavalo, em campo.
\section{Hipódromo}
\begin{itemize}
\item {Grp. gram.:m.}
\end{itemize}
\begin{itemize}
\item {Proveniência:(Do lat. \textunderscore hippodromus\textunderscore )}
\end{itemize}
Terreno, em que se fazem corridas de cavalos.
\section{Hipofagia}
\begin{itemize}
\item {Grp. gram.:f.}
\end{itemize}
\begin{itemize}
\item {Proveniência:(De \textunderscore hipófago\textunderscore )}
\end{itemize}
Acto ou hábito de se alimentar com carne de cavalo.
\section{Hipófago}
\begin{itemize}
\item {Grp. gram.:m.  e  adj.}
\end{itemize}
\begin{itemize}
\item {Proveniência:(Do gr. \textunderscore hippos\textunderscore  + \textunderscore phagein\textunderscore )}
\end{itemize}
O que se alimenta de carne de cavalo.
\section{Hipógrifo}
\begin{itemize}
\item {Grp. gram.:m.}
\end{itemize}
\begin{itemize}
\item {Proveniência:(De \textunderscore hipo...\textunderscore  + \textunderscore grifo\textunderscore ^1)}
\end{itemize}
Animal fabuloso, meio cavalo e meio grifo.
\section{Hipólito}
\begin{itemize}
\item {Grp. gram.:m.}
\end{itemize}
\begin{itemize}
\item {Proveniência:(Do gr. \textunderscore hippos\textunderscore  + \textunderscore lithos\textunderscore )}
\end{itemize}
Pedra amarelada, que se encontra nos intestinos e na bexiga do cavalo.
\section{Hipologia}
\begin{itemize}
\item {Grp. gram.:f.}
\end{itemize}
\begin{itemize}
\item {Proveniência:(De \textunderscore hipólogo\textunderscore )}
\end{itemize}
Tratado ou estudo, á cêrca da raça cavalar.
\section{Hipólogo}
\begin{itemize}
\item {Grp. gram.:m.}
\end{itemize}
\begin{itemize}
\item {Proveniência:(Do gr. \textunderscore hippos\textunderscore  + \textunderscore logos\textunderscore )}
\end{itemize}
Aquele que se ocupa de hipòlogia.
\section{Hipómanes}
\begin{itemize}
\item {Grp. gram.:m.}
\end{itemize}
\begin{itemize}
\item {Proveniência:(Lat. \textunderscore hippomanes\textunderscore )}
\end{itemize}
Feitiço, com que as bruxas pretendiam curar males de amor.
Veneno, extraido das éguas que sofrem cio.
\section{Hipomania}
\begin{itemize}
\item {Grp. gram.:f.}
\end{itemize}
\begin{itemize}
\item {Proveniência:(De \textunderscore hippó...\textunderscore  + \textunderscore mania\textunderscore )}
\end{itemize}
Gôsto apaixonado por cavalos.
Espécie de frenesi, que ataca algumas vezes os cavalos.
\section{Hipomaníaco}
\begin{itemize}
\item {Grp. gram.:m.  e  adj.}
\end{itemize}
O que tem hipomania.
\section{Hipomânico}
\begin{itemize}
\item {Grp. gram.:adj.}
\end{itemize}
Relativo ao hipómanes.
\section{Hipómetro}
\begin{itemize}
\item {Grp. gram.:m.}
\end{itemize}
\begin{itemize}
\item {Proveniência:(Do gr. \textunderscore hippos\textunderscore  + \textunderscore metron\textunderscore )}
\end{itemize}
Instrumento de veterinária, para medir a altura dos cavalos.
\section{Hiponacto}
\begin{itemize}
\item {Grp. gram.:adj.}
\end{itemize}
\begin{itemize}
\item {Proveniência:(De \textunderscore Hipponax\textunderscore , n. p.)}
\end{itemize}
Diz-se de uma espécie de verso jâmbico trímetro, cujo último pé, em vez de jambo, é espondeu.
\section{Hipopatologia}
\begin{itemize}
\item {Grp. gram.:f.}
\end{itemize}
\begin{itemize}
\item {Proveniência:(De \textunderscore hipo...\textunderscore  + \textunderscore patologia\textunderscore )}
\end{itemize}
Patologia do cavalo.
\section{Hipopatológico}
\begin{itemize}
\item {Grp. gram.:adj.}
\end{itemize}
Relativo á hipopatologia.
\section{Hipópode}
\begin{itemize}
\item {Grp. gram.:adj.}
\end{itemize}
\begin{itemize}
\item {Utilização:ant.}
\end{itemize}
\begin{itemize}
\item {Utilização:Poét.}
\end{itemize}
\begin{itemize}
\item {Proveniência:(Do gr. \textunderscore hippos\textunderscore  + \textunderscore pous\textunderscore )}
\end{itemize}
Que tem pés de cavalo.
\section{Hipópotamo}
\begin{itemize}
\item {Grp. gram.:m.}
\end{itemize}
\begin{itemize}
\item {Utilização:Fig.}
\end{itemize}
\begin{itemize}
\item {Proveniência:(Do gr. \textunderscore hippos\textunderscore  + \textunderscore potamos\textunderscore )}
\end{itemize}
Gênero de mamíferos paquidermes, que habitam as margens dos rios africanos, e aos quaes se dá vulgarmente, como ao hipocampo, o nome de \textunderscore cavalo-marinho\textunderscore .
Indivíduo corpulento.
Brutamontes.
\section{Hipotomia}
\begin{itemize}
\item {Grp. gram.:f.}
\end{itemize}
\begin{itemize}
\item {Proveniência:(Do gr. \textunderscore hippos\textunderscore  + \textunderscore tome\textunderscore )}
\end{itemize}
Anatomia do cavalo.
\section{Hipotómico}
\begin{itemize}
\item {Grp. gram.:adj.}
\end{itemize}
Relativo á hipotomia.
\section{Hippanthropia}
\begin{itemize}
\item {Grp. gram.:f.}
\end{itemize}
\begin{itemize}
\item {Proveniência:(Do gr. \textunderscore hippos\textunderscore  + \textunderscore anthropos\textunderscore )}
\end{itemize}
Doença mental dos indivíduos, que se julgam transformados em cavallos.
\section{Hipparcha}
\begin{itemize}
\item {fónica:ca}
\end{itemize}
\begin{itemize}
\item {Grp. gram.:m.}
\end{itemize}
\begin{itemize}
\item {Proveniência:(Do gr. \textunderscore hippos\textunderscore  + \textunderscore arkhein\textunderscore )}
\end{itemize}
Designação dos generaes de cavallaria, entre os antigos Gregos.
\section{Hipparcho}
\begin{itemize}
\item {fónica:co}
\end{itemize}
\begin{itemize}
\item {Grp. gram.:m.}
\end{itemize}
O mesmo que \textunderscore hipparcha\textunderscore . Cf. Latino, \textunderscore Or. da Corôa\textunderscore , XL.
\section{Hippélapho}
\begin{itemize}
\item {Grp. gram.:m.}
\end{itemize}
\begin{itemize}
\item {Proveniência:(Gr. \textunderscore hippelaphos\textunderscore )}
\end{itemize}
Nome de um veado de Samatra e Java.
Nome, dado por Cuvier a outro veado da India.
\section{Hippiatra}
\begin{itemize}
\item {Grp. gram.:m.}
\end{itemize}
\begin{itemize}
\item {Proveniência:(Do gr. \textunderscore hippos\textunderscore  + \textunderscore iatros\textunderscore )}
\end{itemize}
O mesmo que \textunderscore veterinário\textunderscore .
\section{Hippiatria}
\begin{itemize}
\item {Grp. gram.:f.}
\end{itemize}
\begin{itemize}
\item {Utilização:Ext.}
\end{itemize}
\begin{itemize}
\item {Proveniência:(De \textunderscore hippiátrico\textunderscore )}
\end{itemize}
Medicina veterinária, que trata especialmente dos cavallos.
Aquillo que diz respeito a cavallos.
\section{Hippiátrica}
\begin{itemize}
\item {Grp. gram.:f.}
\end{itemize}
\begin{itemize}
\item {Utilização:Ext.}
\end{itemize}
\begin{itemize}
\item {Proveniência:(De \textunderscore hippiátrico\textunderscore )}
\end{itemize}
Medicina veterinária, que trata especialmente dos cavallos.
Aquillo que diz respeito a cavallos.
\section{Hippiátrico}
\begin{itemize}
\item {Grp. gram.:adj.}
\end{itemize}
\begin{itemize}
\item {Proveniência:(Gr. \textunderscore hippiatrikos\textunderscore )}
\end{itemize}
Relativo á hippiátrica.
\section{Híppico}
\begin{itemize}
\item {Grp. gram.:adj.}
\end{itemize}
\begin{itemize}
\item {Proveniência:(Gr. \textunderscore hippikos\textunderscore )}
\end{itemize}
Relativo a cavallos.
\section{Hippiscapho}
\begin{itemize}
\item {Grp. gram.:m.}
\end{itemize}
\begin{itemize}
\item {Utilização:Ant.}
\end{itemize}
Embarcação, destinada especialmente ao transporte de cavallos.
\section{Hippismo}
\begin{itemize}
\item {Grp. gram.:m.}
\end{itemize}
\begin{itemize}
\item {Utilização:Neol.}
\end{itemize}
\begin{itemize}
\item {Proveniência:(Do gr. \textunderscore hippos\textunderscore )}
\end{itemize}
O desporte das corridas de cavallos.
\section{Hippo...}
\begin{itemize}
\item {Grp. gram.:pref.}
\end{itemize}
\begin{itemize}
\item {Proveniência:(Do gr. \textunderscore hippos\textunderscore )}
\end{itemize}
(designativo de \textunderscore cavallo\textunderscore )
\section{Hippoboscos}
\begin{itemize}
\item {Grp. gram.:m. pl.}
\end{itemize}
\begin{itemize}
\item {Proveniência:(Do gr. \textunderscore hippos\textunderscore  + \textunderscore bosko\textunderscore )}
\end{itemize}
Gênero de insectos dípteros.
\section{Hippocampo}
\begin{itemize}
\item {Grp. gram.:m.}
\end{itemize}
\begin{itemize}
\item {Proveniência:(Lat. \textunderscore hippocampus\textunderscore )}
\end{itemize}
Gênero de peixes marítimos, a que também se dá o nome de \textunderscore cavallo-marinho\textunderscore , por causa da fórma da cabeça e do arqueado do corpo.
Nome de eminências, nos ventrículos do cérebro.
\section{Hippocastâneas}
\begin{itemize}
\item {Grp. gram.:f. pl.}
\end{itemize}
\begin{itemize}
\item {Proveniência:(Do gr. \textunderscore hippos\textunderscore  + \textunderscore kastana\textunderscore )}
\end{itemize}
Família de plantas, que têm por typo o castanheiro da Índia.
\section{Hippocentauro}
\begin{itemize}
\item {Grp. gram.:m.}
\end{itemize}
\begin{itemize}
\item {Proveniência:(Lat. \textunderscore hippocentaurus\textunderscore )}
\end{itemize}
O mesmo que \textunderscore centauro\textunderscore .
\section{Hippocratismo}
\begin{itemize}
\item {Grp. gram.:m.}
\end{itemize}
A doutrina médica de Hippócrates. Cf. Latino, \textunderscore Or. da Corôa\textunderscore , CLXXXV.
\section{Hippocrático}
\begin{itemize}
\item {Grp. gram.:adj.}
\end{itemize}
Relativo a Hippócrates ou á sua doutrina.
\section{Hippocraz}
\begin{itemize}
\item {Grp. gram.:m.}
\end{itemize}
\begin{itemize}
\item {Proveniência:(Do rad. de \textunderscore hippocrático\textunderscore )}
\end{itemize}
Infusão de canela, açúcar, etc., em vinho.
\section{Hippodromia}
\begin{itemize}
\item {Grp. gram.:f.}
\end{itemize}
\begin{itemize}
\item {Proveniência:(De \textunderscore hippódromo\textunderscore )}
\end{itemize}
Arte de dirigir corridas de cavallos.
Arte de correr a cavallo, em campo.
\section{Hippódromo}
\begin{itemize}
\item {Grp. gram.:m.}
\end{itemize}
\begin{itemize}
\item {Proveniência:(Do lat. \textunderscore hippodromus\textunderscore )}
\end{itemize}
Terreno, em que se fazem corridas de cavallos.
\section{Hippógrypho}
\begin{itemize}
\item {Grp. gram.:m.}
\end{itemize}
\begin{itemize}
\item {Proveniência:(De \textunderscore hippo...\textunderscore  + \textunderscore grypho\textunderscore ^1)}
\end{itemize}
Animal fabuloso, meio cavallo e meio grypho.
\section{Hippólitho}
\begin{itemize}
\item {Grp. gram.:m.}
\end{itemize}
\begin{itemize}
\item {Proveniência:(Do gr. \textunderscore hippos\textunderscore  + \textunderscore lithos\textunderscore )}
\end{itemize}
Pedra amarelada, que se encontra nos intestinos e na bexiga do cavallo.
\section{Hippologia}
\begin{itemize}
\item {Grp. gram.:f.}
\end{itemize}
\begin{itemize}
\item {Proveniência:(De \textunderscore hippólogo\textunderscore )}
\end{itemize}
Tratado ou estudo, á cêrca da raça cavallar.
\section{Hippólogo}
\begin{itemize}
\item {Grp. gram.:m.}
\end{itemize}
\begin{itemize}
\item {Proveniência:(Do gr. \textunderscore hippos\textunderscore  + \textunderscore logos\textunderscore )}
\end{itemize}
Aquelle que se occupa de hippòlogia.
\section{Hippómanes}
\begin{itemize}
\item {Grp. gram.:m.}
\end{itemize}
\begin{itemize}
\item {Proveniência:(Lat. \textunderscore hippomanes\textunderscore )}
\end{itemize}
Feitiço, com que as bruxas pretendiam curar males de amor.
Veneno, extrahido das éguas que soffrem cio.
\section{Hippomania}
\begin{itemize}
\item {Grp. gram.:f.}
\end{itemize}
\begin{itemize}
\item {Proveniência:(De \textunderscore hippó...\textunderscore  + \textunderscore mania\textunderscore )}
\end{itemize}
Gôsto apaixonado por cavallos.
Espécie de frenesi, que ataca algumas vezes os cavallos.
\section{Hippomaníaco}
\begin{itemize}
\item {Grp. gram.:m.  e  adj.}
\end{itemize}
O que tem hippomania.
\section{Hippomânico}
\begin{itemize}
\item {Grp. gram.:adj.}
\end{itemize}
Relativo ao hippómanes.
\section{Hippómetro}
\begin{itemize}
\item {Grp. gram.:m.}
\end{itemize}
\begin{itemize}
\item {Proveniência:(Do gr. \textunderscore hippos\textunderscore  + \textunderscore metron\textunderscore )}
\end{itemize}
Instrumento de veterinária, para medir a altura dos cavallos.
\section{Hipponacto}
\begin{itemize}
\item {Grp. gram.:adj.}
\end{itemize}
\begin{itemize}
\item {Proveniência:(De \textunderscore Hipponax\textunderscore , n. p.)}
\end{itemize}
Diz-se de uma espécie de verso jâmbico trímetro, cujo último pé, em vez de jambo, é espondeu.
\section{Hippopathologia}
\begin{itemize}
\item {Grp. gram.:f.}
\end{itemize}
\begin{itemize}
\item {Proveniência:(De \textunderscore hippo...\textunderscore  + \textunderscore pathologia\textunderscore )}
\end{itemize}
Pathologia do cavallo.
\section{Hippopathológico}
\begin{itemize}
\item {Grp. gram.:adj.}
\end{itemize}
Relativo á hippopathologia.
\section{Hippophagia}
\begin{itemize}
\item {Grp. gram.:f.}
\end{itemize}
\begin{itemize}
\item {Proveniência:(De \textunderscore hippóphago\textunderscore )}
\end{itemize}
Acto ou hábito de se alimentar com carne de cavallo.
\section{Hippóphago}
\begin{itemize}
\item {Grp. gram.:m.  e  adj.}
\end{itemize}
\begin{itemize}
\item {Proveniência:(Do gr. \textunderscore hippos\textunderscore  + \textunderscore phagein\textunderscore )}
\end{itemize}
O que se alimenta de carne de cavallo.
\section{Hippópode}
\begin{itemize}
\item {Grp. gram.:adj.}
\end{itemize}
\begin{itemize}
\item {Utilização:ant.}
\end{itemize}
\begin{itemize}
\item {Utilização:Poét.}
\end{itemize}
\begin{itemize}
\item {Proveniência:(Do gr. \textunderscore hippos\textunderscore  + \textunderscore pous\textunderscore )}
\end{itemize}
Que tem pés de cavallo.
\section{Hippópotamo}
\begin{itemize}
\item {Grp. gram.:m.}
\end{itemize}
\begin{itemize}
\item {Utilização:Fig.}
\end{itemize}
\begin{itemize}
\item {Proveniência:(Do gr. \textunderscore hippos\textunderscore  + \textunderscore potamos\textunderscore )}
\end{itemize}
Gênero de mammíferos pachidermes, que habitam as margens dos rios africanos, e aos quaes se dá vulgarmente, como ao hippocampo, o nome de \textunderscore cavallo-marinho\textunderscore .
Indivíduo corpulento.
Brutamontes.
\section{Hippotomia}
\begin{itemize}
\item {Grp. gram.:f.}
\end{itemize}
\begin{itemize}
\item {Proveniência:(Do gr. \textunderscore hippos\textunderscore  + \textunderscore tome\textunderscore )}
\end{itemize}
Anatomia do cavallo.
\section{Hippotómico}
\begin{itemize}
\item {Grp. gram.:adj.}
\end{itemize}
Relativo á hippotomia.
\section{Hippurato}
\begin{itemize}
\item {Grp. gram.:m.}
\end{itemize}
\begin{itemize}
\item {Proveniência:(Al. \textunderscore hippurat\textunderscore )}
\end{itemize}
Sal, que, na urina dos herbívoros, tem o lugar que a ureia occupa na urina dos carnívoros.
\section{Hippuria}
\begin{itemize}
\item {Grp. gram.:f.}
\end{itemize}
Presença accidental do ácido hippúrico na urina humana.
(Cp. \textunderscore hippúrico\textunderscore )
\section{Hippúrico}
\begin{itemize}
\item {Grp. gram.:adj.}
\end{itemize}
\begin{itemize}
\item {Proveniência:(Do gr. \textunderscore hippos\textunderscore  + \textunderscore ouron\textunderscore )}
\end{itemize}
Diz-se de um ácido de muitos saes, que são peculiares á urina dos mammíferos herbívoros e até do próprio homem.
\section{Hippurina}
\begin{itemize}
\item {Grp. gram.:f.}
\end{itemize}
\begin{itemize}
\item {Proveniência:(Do gr. \textunderscore hippos\textunderscore  + \textunderscore oura\textunderscore )}
\end{itemize}
Gênero de algas marinhas.
\section{Hippurita}
\begin{itemize}
\item {Grp. gram.:f.}
\end{itemize}
\begin{itemize}
\item {Proveniência:(Do gr. \textunderscore hippos\textunderscore  + \textunderscore oura\textunderscore )}
\end{itemize}
Gênero fóssil de molluscos acéphalos.
\section{Hipsometria}
\begin{itemize}
\item {Grp. gram.:f.}
\end{itemize}
\begin{itemize}
\item {Proveniência:(De \textunderscore hipsómetro\textunderscore )}
\end{itemize}
Arte de medir a altura de um lugar, por meio de observações barométricas.
\section{Hipsométrico}
\begin{itemize}
\item {Grp. gram.:adj.}
\end{itemize}
Relativo á hipsometria.
Medido com hipsómetro.
\section{Hipsómetro}
\begin{itemize}
\item {Grp. gram.:m.}
\end{itemize}
\begin{itemize}
\item {Proveniência:(Do gr. \textunderscore hupsos\textunderscore  + \textunderscore metron\textunderscore )}
\end{itemize}
Instrumento, que faz conhecer a altura de um lugar.
\section{Hipurato}
\begin{itemize}
\item {Grp. gram.:m.}
\end{itemize}
\begin{itemize}
\item {Proveniência:(Al. \textunderscore hippurat\textunderscore )}
\end{itemize}
Sal, que, na urina dos herbívoros, tem o lugar que a ureia occupa na urina dos carnívoros.
\section{Hipuria}
\begin{itemize}
\item {Grp. gram.:f.}
\end{itemize}
Presença acidental do ácido hipúrico na urina humana.
(Cp. \textunderscore hipúrico\textunderscore )
\section{Hipúrico}
\begin{itemize}
\item {Grp. gram.:adj.}
\end{itemize}
\begin{itemize}
\item {Proveniência:(Do gr. \textunderscore hippos\textunderscore  + \textunderscore ouron\textunderscore )}
\end{itemize}
Diz-se de um ácido de muitos saes, que são peculiares á urina dos mamíferos herbívoros e até do próprio homem.
\section{Hipurina}
\begin{itemize}
\item {Grp. gram.:f.}
\end{itemize}
\begin{itemize}
\item {Proveniência:(Do gr. \textunderscore hippos\textunderscore  + \textunderscore oura\textunderscore )}
\end{itemize}
Gênero de algas marinhas.
\section{Hipurita}
\begin{itemize}
\item {Grp. gram.:f.}
\end{itemize}
\begin{itemize}
\item {Proveniência:(Do gr. \textunderscore hippos\textunderscore  + \textunderscore oura\textunderscore )}
\end{itemize}
Gênero fóssil de moluscos acéfalos.
\section{Hirara}
\begin{itemize}
\item {Grp. gram.:f.}
\end{itemize}
\begin{itemize}
\item {Utilização:Bras}
\end{itemize}
Quadrúpede, semelhante ao macaco.
\section{Hirarana}
\begin{itemize}
\item {Grp. gram.:m.}
\end{itemize}
Árvore, de que os Índios da América extráhem veneno para ervar as frechas.
\section{Hircâneo}
\begin{itemize}
\item {Grp. gram.:adj.}
\end{itemize}
Relativo á Hircânia, região da Ásia.
\section{Hircina}
\begin{itemize}
\item {Grp. gram.:f.}
\end{itemize}
\begin{itemize}
\item {Proveniência:(Do lat. \textunderscore hircus\textunderscore )}
\end{itemize}
Substância, que se extrai da gordura do bode e do carneiro.
\section{Hircino}
\begin{itemize}
\item {Grp. gram.:adj.}
\end{itemize}
\begin{itemize}
\item {Proveniência:(Lat. \textunderscore hircinus\textunderscore )}
\end{itemize}
Relativo ao bode.
\section{Hircismo}
\begin{itemize}
\item {Grp. gram.:m.}
\end{itemize}
\begin{itemize}
\item {Proveniência:(Do lat. \textunderscore hircus\textunderscore )}
\end{itemize}
Cheiro desagradável, que sái das axillas de certas pessôas, fazendo lembrar o cheiro do bode.
\section{Hirco}
\begin{itemize}
\item {Grp. gram.:m.}
\end{itemize}
\begin{itemize}
\item {Utilização:Poét.}
\end{itemize}
\begin{itemize}
\item {Proveniência:(Lat. \textunderscore hircus\textunderscore )}
\end{itemize}
O mesmo que \textunderscore bode\textunderscore ^1.
\section{Hircoso}
\begin{itemize}
\item {Grp. gram.:adj.}
\end{itemize}
\begin{itemize}
\item {Proveniência:(Do lat. \textunderscore hircus\textunderscore )}
\end{itemize}
Diz-se de certas plantas, que exhalam cheiro desagradável, parecido ao bodum.
\section{Hirculação}
\begin{itemize}
\item {Grp. gram.:f.}
\end{itemize}
\begin{itemize}
\item {Proveniência:(Do rad. do lat. \textunderscore hircus\textunderscore )}
\end{itemize}
Doenças das vinhas, devida á demasiada fortaleza do estrume.
\section{Hírculo}
\begin{itemize}
\item {Grp. gram.:m.}
\end{itemize}
\begin{itemize}
\item {Proveniência:(Lat. \textunderscore hirculus\textunderscore )}
\end{itemize}
Nome antigo de uma planta, hoje desconhecida, talvez espécie de nardo.
\section{Hirirá}
\begin{itemize}
\item {Grp. gram.:m.}
\end{itemize}
Espécie de macaco do Amazonas.
(Cp. \textunderscore hirara\textunderscore )
\section{Hirsuto}
\begin{itemize}
\item {Grp. gram.:adj.}
\end{itemize}
\begin{itemize}
\item {Proveniência:(Lat. \textunderscore hirsutus\textunderscore )}
\end{itemize}
Que tem pêlos longos, duros e bastos.
Cerdoso; erriçado; espêsso, emmaranhado.
\section{Hirtar-se}
\begin{itemize}
\item {Grp. gram.:v. p.}
\end{itemize}
Tornar-se hirto; eriçar-se. Cf. Camillo, \textunderscore Cancion. Al.\textunderscore , 424.
\section{Hirteza}
\begin{itemize}
\item {Grp. gram.:f.}
\end{itemize}
Estado do que é hirto.
\section{Hirto}
\begin{itemize}
\item {Grp. gram.:adj.}
\end{itemize}
\begin{itemize}
\item {Proveniência:(Lat. \textunderscore hirtus\textunderscore )}
\end{itemize}
Inteiriçado; erecto; retesado.
Immóvel.
Crespo; hirsuto.
\section{Hirudinicultura}
\begin{itemize}
\item {Grp. gram.:f.}
\end{itemize}
\begin{itemize}
\item {Proveniência:(Do lat. \textunderscore hirudo\textunderscore , \textunderscore hirudinis\textunderscore  + \textunderscore cultura\textunderscore )}
\end{itemize}
Arte de criar e multiplicar as sanguesugas.
Processo, para a fecundação artificial das sanguesugas.
\section{Hirundino}
\begin{itemize}
\item {Grp. gram.:adj.}
\end{itemize}
Relativo a andorinha; próprio de andorinha.
(Por \textunderscore hirundinino\textunderscore , lat. \textunderscore hirundininus\textunderscore )
\section{Hispalense}
\begin{itemize}
\item {Grp. gram.:adj.}
\end{itemize}
\begin{itemize}
\item {Proveniência:(Lat. \textunderscore hispalensis\textunderscore )}
\end{itemize}
Relativo a Sevilha:«\textunderscore ...onde o chão hispalense o Bétis lava.\textunderscore »\textunderscore Caramuru\textunderscore , VI, 24.
\section{Hispanhol}
\textunderscore m.\textunderscore  (e der.)
(V. \textunderscore espanhol\textunderscore , etc.)
\section{Hispânico}
\begin{itemize}
\item {Grp. gram.:adj.}
\end{itemize}
\begin{itemize}
\item {Proveniência:(Lat. \textunderscore hispanicus\textunderscore )}
\end{itemize}
Relativo á Espanha.
\section{Hispaniense}
\begin{itemize}
\item {Grp. gram.:adj.}
\end{itemize}
\begin{itemize}
\item {Proveniência:(Lat. \textunderscore hispaniensis\textunderscore )}
\end{itemize}
O mesmo que \textunderscore hispânico\textunderscore .
\section{Hispanismo}
\begin{itemize}
\item {Grp. gram.:m.}
\end{itemize}
\begin{itemize}
\item {Proveniência:(Do lat. \textunderscore hispanus\textunderscore )}
\end{itemize}
O mesmo que \textunderscore espanholismo\textunderscore .
\section{Hispanista}
\begin{itemize}
\item {Grp. gram.:m.}
\end{itemize}
Aquelle que é perito nas línguas ou literaturas hispânicas. Cf. Viana, \textunderscore Apostilas\textunderscore , vb. \textunderscore morilho\textunderscore .
\section{Hispanizado}
\begin{itemize}
\item {Grp. gram.:adj.}
\end{itemize}
\begin{itemize}
\item {Proveniência:(De \textunderscore hispanizar\textunderscore )}
\end{itemize}
Que tem carácter ou feição hispânica.
\section{Hispanizante}
\begin{itemize}
\item {Grp. gram.:m.  e  adj.}
\end{itemize}
O que hispaniza.
\section{Hispanizar}
\begin{itemize}
\item {Grp. gram.:v. t.}
\end{itemize}
Dar carácter hispânico a.
Dar feição ou fórma hispânica a (termo ou locução). Cf. C. Michaëlis, \textunderscore in\textunderscore  Rev. \textunderscore Tradição\textunderscore , I, 165.
\section{Hispano}
\begin{itemize}
\item {Grp. gram.:adj.}
\end{itemize}
O mesmo que \textunderscore hispânico\textunderscore .
\section{Hispano...}
Elemento, que entra na composição de várias palavras, com a significação de espanhol ou relativo á Espanha.
\section{Hispano-americano}
\begin{itemize}
\item {Grp. gram.:adj.}
\end{itemize}
Relativo á Espanha e á América.
\section{Hispano-árabe}
\begin{itemize}
\item {Grp. gram.:adj.}
\end{itemize}
Relativo aos Árabes da Espanha.
\section{Hispano-godo}
\begin{itemize}
\item {Grp. gram.:adj.}
\end{itemize}
Relativo aos Godos da península hispânica.
\section{Hispano-luso}
\begin{itemize}
\item {Grp. gram.:adj.}
\end{itemize}
Relativo a Espanhóes e Portugueses.
\section{Hispano-marroquino}
\begin{itemize}
\item {Grp. gram.:adj.}
\end{itemize}
Relativo á Espanha e a Marrocos.
\section{Hispano-romano}
\begin{itemize}
\item {Grp. gram.:m.}
\end{itemize}
Um dos cinco ramos principaes das línguas novi-latinas, comprehendendo o espanhol, o português, o gallego e o asturiano.
\section{Hispano-russo}
\begin{itemize}
\item {Grp. gram.:adj.}
\end{itemize}
Relativo á Espanha e á Rússia.
\section{Hispano-turco}
\begin{itemize}
\item {Grp. gram.:adj.}
\end{itemize}
Relativo á Espanha e á Turquia.
\section{Hispas}
\begin{itemize}
\item {Grp. gram.:f. pl.}
\end{itemize}
\begin{itemize}
\item {Proveniência:(Do rad. do lat. \textunderscore hispidus\textunderscore )}
\end{itemize}
Gênero de insectos coleópteros tetrâmeros.
\section{Hispidar-se}
\begin{itemize}
\item {Grp. gram.:v. p.}
\end{itemize}
Tornar-se híspido, eriçar-se.
\section{Hispidez}
\begin{itemize}
\item {Grp. gram.:f.}
\end{itemize}
Estado do que é híspido.
\section{Híspido}
\begin{itemize}
\item {Grp. gram.:adj.}
\end{itemize}
\begin{itemize}
\item {Proveniência:(Lat. \textunderscore hispidus\textunderscore )}
\end{itemize}
Eriçado de pêlos; hirsuto.
\section{Hissom}
\begin{itemize}
\item {Grp. gram.:adj.}
\end{itemize}
Diz-se de uma espécie de chá verde, muito estimada, e cujas fôlhas são estreitas, longas, carnudas e voltadas em espiral.
\section{Histerelhos}
\begin{itemize}
\item {fónica:terê}
\end{itemize}
\begin{itemize}
\item {Grp. gram.:m. pl.}
\end{itemize}
\begin{itemize}
\item {Proveniência:(Do rad. do lat. \textunderscore hister\textunderscore )}
\end{itemize}
Gênero de insectos coleópteros, da fam. dos clavicórneos.
\section{Histo...}
\begin{itemize}
\item {Grp. gram.:pref.}
\end{itemize}
\begin{itemize}
\item {Proveniência:(Do gr. \textunderscore histos\textunderscore , tecido)}
\end{itemize}
(designativo de tecidos orgânicos)
\section{Histochímica}
\begin{itemize}
\item {fónica:qui}
\end{itemize}
\begin{itemize}
\item {Grp. gram.:f.}
\end{itemize}
\begin{itemize}
\item {Proveniência:(De \textunderscore histo...\textunderscore  + \textunderscore chímica\textunderscore )}
\end{itemize}
Estudo chímico dos princípios immediatos dos tecidos orgânicos.
\section{Histofisiologia}
\begin{itemize}
\item {Grp. gram.:f.}
\end{itemize}
\begin{itemize}
\item {Proveniência:(De \textunderscore histo...\textunderscore  + \textunderscore fisiologia\textunderscore )}
\end{itemize}
Fisiologia dos tecidos orgânicos.
\section{Histofisiológico}
\begin{itemize}
\item {Grp. gram.:adj.}
\end{itemize}
Relativo á histofisiologia.
\section{Histogêneo}
\begin{itemize}
\item {Grp. gram.:adj.}
\end{itemize}
\begin{itemize}
\item {Proveniência:(Do gr. \textunderscore histos\textunderscore  + \textunderscore genos\textunderscore )}
\end{itemize}
Que gera tecidos orgânicos, (falando-se de tecidos animaes).
\section{Histogenia}
\begin{itemize}
\item {Grp. gram.:f.}
\end{itemize}
\begin{itemize}
\item {Proveniência:(De \textunderscore histogêneo\textunderscore )}
\end{itemize}
Formação de tecidos orgânicos.
\section{Histogênico}
\begin{itemize}
\item {Grp. gram.:adj.}
\end{itemize}
Relativo á histogenia.
\section{Histogenol}
\begin{itemize}
\item {Grp. gram.:m.}
\end{itemize}
\begin{itemize}
\item {Utilização:Pharm.}
\end{itemize}
Medicamento contra a tuberculose pulmonar.
\section{Histografia}
\begin{itemize}
\item {Grp. gram.:f.}
\end{itemize}
\begin{itemize}
\item {Proveniência:(De \textunderscore histógrafo\textunderscore )}
\end{itemize}
Descripção dos tecidos orgânicos.
\section{Histográfico}
\begin{itemize}
\item {Grp. gram.:adj.}
\end{itemize}
Relativo á histografia.
\section{Histógrafo}
\begin{itemize}
\item {Grp. gram.:m.}
\end{itemize}
\begin{itemize}
\item {Proveniência:(Do gr. \textunderscore histos\textunderscore  + \textunderscore graphein\textunderscore )}
\end{itemize}
Aquele que se occupa de histografia.
\section{Histographia}
\begin{itemize}
\item {Grp. gram.:f.}
\end{itemize}
\begin{itemize}
\item {Proveniência:(De \textunderscore histógrapho\textunderscore )}
\end{itemize}
Descripção dos tecidos orgânicos.
\section{Histográphico}
\begin{itemize}
\item {Grp. gram.:adj.}
\end{itemize}
Relativo á histographia.
\section{Histógrapho}
\begin{itemize}
\item {Grp. gram.:m.}
\end{itemize}
\begin{itemize}
\item {Proveniência:(Do gr. \textunderscore histos\textunderscore  + \textunderscore graphein\textunderscore )}
\end{itemize}
Aquelle que se occupa de histographia.
\section{Histolise}
\begin{itemize}
\item {Grp. gram.:f.}
\end{itemize}
\begin{itemize}
\item {Utilização:Zool.}
\end{itemize}
\begin{itemize}
\item {Proveniência:(Do gr. \textunderscore histos\textunderscore  + \textunderscore lusis\textunderscore )}
\end{itemize}
Fenómeno da reabsorpção dos tecidos, para sua renovação, como se observa nos briozoários.
\section{Histologia}
\begin{itemize}
\item {Grp. gram.:f.}
\end{itemize}
\begin{itemize}
\item {Proveniência:(Do gr. \textunderscore histos\textunderscore  + \textunderscore logos\textunderscore )}
\end{itemize}
Exposição scientífica das leis que presidem á formação e disposição dos tecidos orgânicos.
\section{Histológico}
\begin{itemize}
\item {Grp. gram.:adj.}
\end{itemize}
Relativo á histologia.
\section{Histolyse}
\begin{itemize}
\item {Grp. gram.:f.}
\end{itemize}
\begin{itemize}
\item {Utilização:Zool.}
\end{itemize}
\begin{itemize}
\item {Proveniência:(Do gr. \textunderscore histos\textunderscore  + \textunderscore lusis\textunderscore )}
\end{itemize}
Phenómeno da reabsorpção dos tecidos, para sua renovação, como se observa nos bryozoários.
\section{Histoneurologia}
\begin{itemize}
\item {Grp. gram.:f.}
\end{itemize}
\begin{itemize}
\item {Proveniência:(Do gr. \textunderscore histos\textunderscore  + \textunderscore neuron\textunderscore  + \textunderscore logos\textunderscore )}
\end{itemize}
Estudo ou exposição scientífica das leis que regulam a disposição e as funcções do systema nervoso.
\section{Histoneurológico}
\begin{itemize}
\item {Grp. gram.:adj.}
\end{itemize}
Relativo á histoneurologia.
\section{Histoneurologista}
\begin{itemize}
\item {Grp. gram.:m.}
\end{itemize}
Aquelle que se dedica á histoneurologia.
\section{Histonomia}
\begin{itemize}
\item {Grp. gram.:f.}
\end{itemize}
\begin{itemize}
\item {Proveniência:(Do gr. \textunderscore histos\textunderscore  + \textunderscore nomos\textunderscore )}
\end{itemize}
Conjunto das leis, que a histologia expõe.
\section{Histonómico}
\begin{itemize}
\item {Grp. gram.:adj.}
\end{itemize}
Relativo á histonomia.
\section{Histophysiologia}
\begin{itemize}
\item {Grp. gram.:f.}
\end{itemize}
\begin{itemize}
\item {Proveniência:(De \textunderscore histo...\textunderscore  + \textunderscore physiologia\textunderscore )}
\end{itemize}
Physiologia dos tecidos orgânicos.
\section{Histophysiológico}
\begin{itemize}
\item {Grp. gram.:adj.}
\end{itemize}
Relativo á histophysiologia.
\section{Histoquímica}
\begin{itemize}
\item {Grp. gram.:f.}
\end{itemize}
\begin{itemize}
\item {Proveniência:(De \textunderscore histo...\textunderscore  + \textunderscore química\textunderscore )}
\end{itemize}
Estudo químico dos princípios imediatos dos tecidos orgânicos.
\section{História}
\begin{itemize}
\item {Grp. gram.:f.}
\end{itemize}
\begin{itemize}
\item {Utilização:Fam.}
\end{itemize}
\begin{itemize}
\item {Proveniência:(Lat. \textunderscore historia\textunderscore )}
\end{itemize}
Narração de factos sociaes.
Série de acontecimentos sociaes, políticos, económicos, intellectuaes, etc., que mais ou menos influíram na existência dos povos: \textunderscore a história moderna\textunderscore .
Estudo ou conjunto de phenómenos naturaes ou scientíficos: \textunderscore história natural\textunderscore .
Estudo das origens e progressos de uma arte ou sciência: \textunderscore história da Astronomia\textunderscore .
Biographia de uma personagem célebre: \textunderscore história de Carlos Magno\textunderscore .
Narração, narrativa.
Conto: \textunderscore um livro de histórias\textunderscore .
Qualquer coisa ou negócio.
Qualquer objecto que se não póde ou não se quer nomear.
Fábula; patranha: \textunderscore isso é história\textunderscore .--Tem ainda outras accepções indeterminadas, no estilo familiar, e também se emprega interjectivamente.
\section{Historiado}
\begin{itemize}
\item {Grp. gram.:adj.}
\end{itemize}
\begin{itemize}
\item {Utilização:Fam.}
\end{itemize}
\begin{itemize}
\item {Proveniência:(De \textunderscore historiar\textunderscore )}
\end{itemize}
Cheio ou adornado de episódios: \textunderscore uma carta muito historiada\textunderscore .
\section{Historiador}
\begin{itemize}
\item {Grp. gram.:m.  e  adj.}
\end{itemize}
\begin{itemize}
\item {Utilização:Ext.}
\end{itemize}
\begin{itemize}
\item {Proveniência:(De \textunderscore historiar\textunderscore )}
\end{itemize}
O que escreve sôbre história.
O que escreve uma história ou histórias.
Aquelle que narra um acontecimento.
\section{Historial}
\begin{itemize}
\item {Grp. gram.:adj.}
\end{itemize}
\begin{itemize}
\item {Utilização:Ant.}
\end{itemize}
Relativo á história ou aos historiadores. Cf. Rui de Pina, \textunderscore Chrón. de D. João\textunderscore , II, (pról.).
\section{Historiar}
\begin{itemize}
\item {Grp. gram.:v. t.}
\end{itemize}
\begin{itemize}
\item {Utilização:Fam.}
\end{itemize}
\begin{itemize}
\item {Utilização:Archit.}
\end{itemize}
Fazer a história de: \textunderscore historiar uma revolução\textunderscore .
Contar: \textunderscore historiar um naufrágio\textunderscore .
Enfeitar; embrincar.
Alindar com pequenos ornatos.
\section{Historicamente}
\begin{itemize}
\item {Grp. gram.:adv.}
\end{itemize}
De modo histórico.
\section{Histórico}
\begin{itemize}
\item {Grp. gram.:adj.}
\end{itemize}
\begin{itemize}
\item {Grp. gram.:M.}
\end{itemize}
\begin{itemize}
\item {Utilização:Des.}
\end{itemize}
\begin{itemize}
\item {Proveniência:(Lat. \textunderscore historicus\textunderscore )}
\end{itemize}
Relativo á história.
Real, que não é producto de imaginação.
Digno de figurar na história.
Que recorda algum acontecimento notável.
Relativo ao tempo posterior á formação das sociedades ou á época, a que respeitam os mais antigos documentos e monumentos conhecidos.
O mesmo que \textunderscore historiador\textunderscore ; historiógrapho. Cf. G. Barreiros, \textunderscore Corografia\textunderscore , fl. 177 v.^o, (1.^a ed.).
\section{Historieta}
\begin{itemize}
\item {fónica:ê}
\end{itemize}
\begin{itemize}
\item {Grp. gram.:f.}
\end{itemize}
\begin{itemize}
\item {Proveniência:(De \textunderscore história\textunderscore )}
\end{itemize}
Narração de um facto pouco importante.
Conto; anecdota.
\section{Historiografia}
\begin{itemize}
\item {Grp. gram.:f.}
\end{itemize}
\begin{itemize}
\item {Proveniência:(De \textunderscore historiógrafo\textunderscore )}
\end{itemize}
Arte de escrever história.
Estudo histórico e crítico á cêrca de historiadores.
\section{Historiógrafo}
\begin{itemize}
\item {Grp. gram.:m.}
\end{itemize}
\begin{itemize}
\item {Proveniência:(Do gr. \textunderscore historia\textunderscore  + \textunderscore graphein\textunderscore )}
\end{itemize}
Aquele que escreve a história de uma época; cronista; historiador.
\section{Historiographia}
\begin{itemize}
\item {Grp. gram.:f.}
\end{itemize}
\begin{itemize}
\item {Proveniência:(De \textunderscore historiógrapho\textunderscore )}
\end{itemize}
Arte de escrever história.
Estudo histórico e crítico á cêrca de historiadores.
\section{Historiógrapho}
\begin{itemize}
\item {Grp. gram.:m.}
\end{itemize}
\begin{itemize}
\item {Proveniência:(Do gr. \textunderscore historia\textunderscore  + \textunderscore graphein\textunderscore )}
\end{itemize}
Aquelle que escreve a história de uma época; chronista; historiador.
\section{Historíola}
\begin{itemize}
\item {Grp. gram.:f.}
\end{itemize}
\begin{itemize}
\item {Proveniência:(De \textunderscore história\textunderscore )}
\end{itemize}
O mesmo que \textunderscore historieta\textunderscore .
\section{Historiúncula}
\begin{itemize}
\item {Grp. gram.:f.}
\end{itemize}
Historieta. Cf. Castilho, \textunderscore Collóq. Ald.\textunderscore , 325.
\section{Historizar}
\begin{itemize}
\item {Grp. gram.:v. t.}
\end{itemize}
O mesmo que \textunderscore historiar\textunderscore .
\section{Histotripsia}
\begin{itemize}
\item {Grp. gram.:f.}
\end{itemize}
\begin{itemize}
\item {Proveniência:(Do gr. \textunderscore histos\textunderscore  + \textunderscore tripsis\textunderscore )}
\end{itemize}
Operação cirúrgica, que consiste no esmagamento de tecidos.
\section{Histotriptor}
\begin{itemize}
\item {Grp. gram.:m.}
\end{itemize}
Instrumento, para se praticar a histotripsia.
\section{Histotromia}
\begin{itemize}
\item {Grp. gram.:f.}
\end{itemize}
\begin{itemize}
\item {Proveniência:(Do gr. \textunderscore histos\textunderscore  + \textunderscore tromos\textunderscore )}
\end{itemize}
Contracção fibrillar, que se observa nos músculos e especialmente nas pálpebras.
\section{Histrião}
\begin{itemize}
\item {Grp. gram.:m.}
\end{itemize}
\begin{itemize}
\item {Utilização:Fig.}
\end{itemize}
\begin{itemize}
\item {Proveniência:(Lat. \textunderscore histrio\textunderscore )}
\end{itemize}
Palhaço; bobo.
Homem abjecto pelo seu procedimento.
\section{Histriónico}
\begin{itemize}
\item {Grp. gram.:adj.}
\end{itemize}
\begin{itemize}
\item {Proveniência:(Lat. \textunderscore histrionicus\textunderscore )}
\end{itemize}
Relativo a histrião, próprio de histrião.
\section{Hiulco}
\begin{itemize}
\item {Grp. gram.:adj.}
\end{itemize}
\begin{itemize}
\item {Utilização:Poét.}
\end{itemize}
Hiante; fendido.
(Lát. \textunderscore hiulcus\textunderscore )
\section{Hobo}
\begin{itemize}
\item {Grp. gram.:m.}
\end{itemize}
Ameixoeira da China.
\section{Hoco}
\begin{itemize}
\item {Grp. gram.:m.}
\end{itemize}
(V.mutum)
\section{Hodiernamente}
\begin{itemize}
\item {Grp. gram.:adv.}
\end{itemize}
\begin{itemize}
\item {Proveniência:(De \textunderscore hodierno\textunderscore )}
\end{itemize}
No tempo de agora, actualmente.
Recentemente.
\section{Hodierno}
\begin{itemize}
\item {Grp. gram.:adj.}
\end{itemize}
\begin{itemize}
\item {Proveniência:(Lat. \textunderscore hodiernus\textunderscore )}
\end{itemize}
Relativo ao dia de hoje; recente.
\section{Hodometria}
\begin{itemize}
\item {Grp. gram.:f.}
\end{itemize}
\begin{itemize}
\item {Proveniência:(De \textunderscore hodómetro\textunderscore )}
\end{itemize}
Arte de medir as distâncias percorridas.
\section{Hodométrico}
\begin{itemize}
\item {Grp. gram.:adj.}
\end{itemize}
Relativo á hodometria.
\section{Hodómetro}
\begin{itemize}
\item {Grp. gram.:m.}
\end{itemize}
\begin{itemize}
\item {Proveniência:(Do gr. \textunderscore hodos\textunderscore  + \textunderscore metron\textunderscore )}
\end{itemize}
Instrumento, com que se medem distâncias percorridas.
Instrumento, para contar o número das voltas de uma manivela.
\section{Hogano}
\begin{itemize}
\item {Grp. gram.:adv.}
\end{itemize}
\begin{itemize}
\item {Utilização:Ant.}
\end{itemize}
\begin{itemize}
\item {Proveniência:(Do lat. \textunderscore hoc\textunderscore  + \textunderscore anno\textunderscore )}
\end{itemize}
Neste anno, neste tempo.
\section{Hohóbia}
\begin{itemize}
\item {Grp. gram.:f.}
\end{itemize}
Pássaro conirostro, (\textunderscore coracius noevia\textunderscore ).
\section{Hohombe}
\begin{itemize}
\item {Grp. gram.:m.}
\end{itemize}
Espécie de águia, (\textunderscore helotarsus ecaudatus\textunderscore ).
\section{Hoje}
\begin{itemize}
\item {Grp. gram.:adv.}
\end{itemize}
\begin{itemize}
\item {Grp. gram.:M.}
\end{itemize}
\begin{itemize}
\item {Proveniência:(Do lat. \textunderscore hodie\textunderscore )}
\end{itemize}
No dia actual, no dia em que estamos: \textunderscore hoje, vou ao theatro\textunderscore .
Actualmente; no tempo corrente: \textunderscore hoje, há pouco patriotismo\textunderscore .
O dia ou a época, em que estamos.
\section{Holanda}
\begin{itemize}
\item {Grp. gram.:f.}
\end{itemize}
\begin{itemize}
\item {Proveniência:(De \textunderscore Holanda\textunderscore , n. p.)}
\end{itemize}
Tecido de linho, muito fino e fechado, que se fabríca na Hollanda.
Genebra da Hollanda.
Espécie de papel.
\section{Holandês}
\begin{itemize}
\item {Grp. gram.:adj.}
\end{itemize}
\begin{itemize}
\item {Grp. gram.:M.}
\end{itemize}
Relativo á Holanda.
Habitante da Holanda.
Dialecto neerlandês, falado na Holanda.
\section{Holandilha}
\begin{itemize}
\item {Grp. gram.:f.}
\end{itemize}
\begin{itemize}
\item {Proveniência:(De \textunderscore holanda\textunderscore )}
\end{itemize}
Tecido grosso de linho, usado principalmente em entretelas.
\section{Holandilheiro}
\begin{itemize}
\item {Grp. gram.:m.}
\end{itemize}
Fabricante ou vendedor de holandilha. Cf. Arn. Gama, \textunderscore Motim\textunderscore , 410.
\section{Holandizar}
\begin{itemize}
\item {Grp. gram.:v. t.}
\end{itemize}
Dar a feição ou os costumes holandeses a.
\section{Holão}
\begin{itemize}
\item {Grp. gram.:m.}
\end{itemize}
\begin{itemize}
\item {Utilização:Ant.}
\end{itemize}
Espécie de tecido.
(Cp. \textunderscore holanda\textunderscore )
\section{Holerca}
\begin{itemize}
\item {Grp. gram.:f.}
\end{itemize}
Bebida alcoólica, fabricada de frutas e cevada e usada na Transylvânia. Cf. \textunderscore Techn. Rur.\textunderscore , 444.
\section{Holicismo}
\begin{itemize}
\item {Grp. gram.:m.}
\end{itemize}
\begin{itemize}
\item {Proveniência:(Do gr. \textunderscore holikes\textunderscore )}
\end{itemize}
Expressão commum a vários dialectos ou a várias línguas.
\section{Hollanda}
\begin{itemize}
\item {Grp. gram.:f.}
\end{itemize}
\begin{itemize}
\item {Proveniência:(De \textunderscore Hollanda\textunderscore , n. p.)}
\end{itemize}
Tecido de linho, muito fino e fechado, que se fabríca na Hollanda.
Genebra da Hollanda.
Espécie de papel.
\section{Hollandês}
\begin{itemize}
\item {Grp. gram.:adj.}
\end{itemize}
\begin{itemize}
\item {Grp. gram.:M.}
\end{itemize}
Relativo á Hollanda.
Habitante da Hollanda.
Dialecto neerlandês, falado na Hollanda.
\section{Hollandilha}
\begin{itemize}
\item {Grp. gram.:f.}
\end{itemize}
\begin{itemize}
\item {Proveniência:(De \textunderscore hollanda\textunderscore )}
\end{itemize}
Tecido grosso de linho, usado principalmente em entretelas.
\section{Hollandilheiro}
\begin{itemize}
\item {Grp. gram.:m.}
\end{itemize}
Fabricante ou vendedor de hollandilha. Cf. Arn. Gama, \textunderscore Motim\textunderscore , 410.
\section{Hollandizar}
\begin{itemize}
\item {Grp. gram.:v. t.}
\end{itemize}
Dar a feição ou os costumes hollandeses a.
\section{Hollão}
\begin{itemize}
\item {Grp. gram.:m.}
\end{itemize}
\begin{itemize}
\item {Utilização:Ant.}
\end{itemize}
Espécie de tecido.
(Cp. \textunderscore hollanda\textunderscore )
\section{Holo...}
\begin{itemize}
\item {Grp. gram.:pref.}
\end{itemize}
\begin{itemize}
\item {Proveniência:(Gr. \textunderscore holos\textunderscore )}
\end{itemize}
(designativo de \textunderscore inteiro\textunderscore )
\section{Holoblástico}
\begin{itemize}
\item {Grp. gram.:adj.}
\end{itemize}
\begin{itemize}
\item {Utilização:Zool.}
\end{itemize}
\begin{itemize}
\item {Proveniência:(Do gr. \textunderscore holos\textunderscore  + \textunderscore blastos\textunderscore )}
\end{itemize}
Diz-se dos ovos de segmentação total.
\section{Holobrânchio}
\begin{itemize}
\item {fónica:qui}
\end{itemize}
\begin{itemize}
\item {Grp. gram.:adj.}
\end{itemize}
\begin{itemize}
\item {Utilização:Ichthyol.}
\end{itemize}
\begin{itemize}
\item {Proveniência:(De \textunderscore holo...\textunderscore  + \textunderscore brânchias\textunderscore )}
\end{itemize}
Que tem brânchias completas.
\section{Holobrânquio}
\begin{itemize}
\item {Grp. gram.:adj.}
\end{itemize}
\begin{itemize}
\item {Utilização:Ichthyol.}
\end{itemize}
\begin{itemize}
\item {Proveniência:(De \textunderscore holo...\textunderscore  + \textunderscore brânquias\textunderscore )}
\end{itemize}
Que tem brânquias completas.
\section{Holocarpo}
\begin{itemize}
\item {Grp. gram.:adj.}
\end{itemize}
\begin{itemize}
\item {Proveniência:(Do gr. \textunderscore holos\textunderscore  + \textunderscore karpos\textunderscore )}
\end{itemize}
Diz-se das plantas, cujos frutos se não abrem.
\section{Holocausto}
\begin{itemize}
\item {Grp. gram.:m.}
\end{itemize}
\begin{itemize}
\item {Utilização:Fig.}
\end{itemize}
\begin{itemize}
\item {Proveniência:(Lat. \textunderscore holocaustum\textunderscore )}
\end{itemize}
Sacrificio, em que se queimavam as víctimas, entre os Judeus.
Víctima sacrificada.
Sacrificio.
Abstracção da vontade propria, para satisfazer a de outrem.
\section{Holocristalino}
\begin{itemize}
\item {Grp. gram.:adj.}
\end{itemize}
\begin{itemize}
\item {Utilização:Miner.}
\end{itemize}
\begin{itemize}
\item {Proveniência:(Do gr. \textunderscore holos\textunderscore  + \textunderscore krustallos\textunderscore )}
\end{itemize}
Diz-se de uma das texturas de rochas, que representam a transição do estado vítreo primitivo para o estado crystallino mais perfeito.
\section{Holocrystallino}
\begin{itemize}
\item {Grp. gram.:adj.}
\end{itemize}
\begin{itemize}
\item {Utilização:Miner.}
\end{itemize}
\begin{itemize}
\item {Proveniência:(Do gr. \textunderscore holos\textunderscore  + \textunderscore krustallos\textunderscore )}
\end{itemize}
Diz-se de uma das texturas de rochas, que representam a transição do estado vítreo primitivo para o estado crystallino mais perfeito.
\section{Holoedria}
\begin{itemize}
\item {fónica:lo-e}
\end{itemize}
\begin{itemize}
\item {Grp. gram.:f.}
\end{itemize}
\begin{itemize}
\item {Proveniência:(De \textunderscore holoédro\textunderscore )}
\end{itemize}
Estado de um crystal holoédrico.
\section{Holoédrico}
\begin{itemize}
\item {Grp. gram.:adj.}
\end{itemize}
Que tem o carácter de holoédro.
\section{Holoédro}
\begin{itemize}
\item {Grp. gram.:m.}
\end{itemize}
\begin{itemize}
\item {Utilização:Miner.}
\end{itemize}
\begin{itemize}
\item {Proveniência:(Do gr. \textunderscore holos\textunderscore  + \textunderscore edra\textunderscore )}
\end{itemize}
Crystal, que tem todas as suas faces, por opposição a hemiedro.
\section{Holofote}
\begin{itemize}
\item {Grp. gram.:m.}
\end{itemize}
\begin{itemize}
\item {Proveniência:(Do gr. \textunderscore holos\textunderscore  + \textunderscore phos\textunderscore , \textunderscore photos\textunderscore )}
\end{itemize}
Espécie de lanterna, cuja luz, projectada por uma lente, ilumina os objectos a distância.
Foco eléctrico.
\section{Holofrástico}
\begin{itemize}
\item {Grp. gram.:adj.}
\end{itemize}
\begin{itemize}
\item {Utilização:Philol.}
\end{itemize}
\begin{itemize}
\item {Proveniência:(Do gr. \textunderscore holos\textunderscore  + \textunderscore phrazein\textunderscore )}
\end{itemize}
Diz-se do grupo, que abrange quási todos os idiomas americanos, em que os principaes elementos de uma oração se encorporam num só vocábulo.
\section{Hologastros}
\begin{itemize}
\item {Grp. gram.:m. pl.}
\end{itemize}
\begin{itemize}
\item {Utilização:Zool.}
\end{itemize}
\begin{itemize}
\item {Proveniência:(Do gr. \textunderscore holos\textunderscore  + \textunderscore gaster\textunderscore )}
\end{itemize}
Secção dos arachnídeos.
\section{Hológrafo}
\begin{itemize}
\item {Grp. gram.:adj.}
\end{itemize}
\begin{itemize}
\item {Utilização:Jur.}
\end{itemize}
\begin{itemize}
\item {Utilização:ant.}
\end{itemize}
\begin{itemize}
\item {Proveniência:(Do gr. \textunderscore holos\textunderscore  + \textunderscore graphein\textunderscore )}
\end{itemize}
Dizia-se do testamento, todo escrito pela mão do testador.
\section{Hológrapho}
\begin{itemize}
\item {Grp. gram.:adj.}
\end{itemize}
\begin{itemize}
\item {Utilização:Jur.}
\end{itemize}
\begin{itemize}
\item {Utilização:ant.}
\end{itemize}
\begin{itemize}
\item {Proveniência:(Do gr. \textunderscore holos\textunderscore  + \textunderscore graphein\textunderscore )}
\end{itemize}
Dizia-se do testamento, todo escrito pela mão do testador.
\section{Holométrico}
\begin{itemize}
\item {Grp. gram.:adj.}
\end{itemize}
Relativo ao holómetro.
\section{Holómetro}
\begin{itemize}
\item {Grp. gram.:m.}
\end{itemize}
\begin{itemize}
\item {Proveniência:(Do gr. \textunderscore holos\textunderscore  + \textunderscore metron\textunderscore )}
\end{itemize}
Instrumento, para medir a altura angular de um ponto acima do horizonte.
\section{Holopetalar}
\begin{itemize}
\item {Grp. gram.:adj.}
\end{itemize}
\begin{itemize}
\item {Utilização:Bot.}
\end{itemize}
\begin{itemize}
\item {Proveniência:(De \textunderscore holo...\textunderscore  + \textunderscore pétala\textunderscore )}
\end{itemize}
Diz-se da flôr, cujas partes se converteram todas em pétalas.
\section{Holophote}
\begin{itemize}
\item {Grp. gram.:m.}
\end{itemize}
\begin{itemize}
\item {Proveniência:(Do gr. \textunderscore holos\textunderscore  + \textunderscore phos\textunderscore , \textunderscore photos\textunderscore )}
\end{itemize}
Espécie de lanterna, cuja luz, projectada por uma lente, illumina os objectos a distância.
Foco eléctrico.
\section{Holophrástico}
\begin{itemize}
\item {Grp. gram.:adj.}
\end{itemize}
\begin{itemize}
\item {Utilização:Philol.}
\end{itemize}
\begin{itemize}
\item {Proveniência:(Do gr. \textunderscore holos\textunderscore  + \textunderscore phrazein\textunderscore )}
\end{itemize}
Diz-se do grupo, que abrange quási todos os idiomas americanos, em que os principaes elementos de uma oração se encorporam num só vocábulo.
\section{Holostério}
\begin{itemize}
\item {Grp. gram.:m.}
\end{itemize}
\begin{itemize}
\item {Utilização:Phýs.}
\end{itemize}
Espécie de barómetro, muito sensível e hoje desusado.
\section{Holothúria}
\begin{itemize}
\item {Grp. gram.:f.}
\end{itemize}
\begin{itemize}
\item {Proveniência:(Do gr. \textunderscore holos\textunderscore  + \textunderscore thouros\textunderscore )}
\end{itemize}
Gênero de echinodermes, de tegumento coriáceo e corpo cylíndrico.
\section{Holothúrido}
\begin{itemize}
\item {Grp. gram.:adj.}
\end{itemize}
\begin{itemize}
\item {Grp. gram.:M. pl.}
\end{itemize}
Semelhante á holothúria.
Classe de animaes radiários, que têm por typo a holothúria.
\section{Holotomia}
\begin{itemize}
\item {Grp. gram.:f.}
\end{itemize}
\begin{itemize}
\item {Utilização:Cir.}
\end{itemize}
\begin{itemize}
\item {Proveniência:(Do gr. \textunderscore holos\textunderscore  + \textunderscore tome\textunderscore )}
\end{itemize}
Incisão ou ablação completa de uma parte do corpo.
\section{Holotónico}
\begin{itemize}
\item {Grp. gram.:adj.}
\end{itemize}
\begin{itemize}
\item {Utilização:Med.}
\end{itemize}
\begin{itemize}
\item {Proveniência:(Do gr. \textunderscore holos\textunderscore  + \textunderscore tonos\textunderscore )}
\end{itemize}
Que ataca todas as partes do corpo.
\section{Holotúria}
\begin{itemize}
\item {Grp. gram.:f.}
\end{itemize}
\begin{itemize}
\item {Proveniência:(Do gr. \textunderscore holos\textunderscore  + \textunderscore thouros\textunderscore )}
\end{itemize}
Gênero de equinodermes, de tegumento coriáceo e corpo cilíndrico.
\section{Holotúrido}
\begin{itemize}
\item {Grp. gram.:adj.}
\end{itemize}
\begin{itemize}
\item {Grp. gram.:M. pl.}
\end{itemize}
Semelhante á holothúria.
Classe de animaes radiários, que têm por typo a holothúria.
\section{Homalíneas}
\begin{itemize}
\item {Grp. gram.:f. pl.}
\end{itemize}
Família de plantas, estabelecida por Brown, á custa das rosáceas e das rhamnáceas.
\section{Homalocéfalo}
\begin{itemize}
\item {Grp. gram.:m.}
\end{itemize}
\begin{itemize}
\item {Proveniência:(Do gr. \textunderscore homalos\textunderscore  + \textunderscore kephale\textunderscore )}
\end{itemize}
Reptil sáurio.
\section{Homalocéphalo}
\begin{itemize}
\item {Grp. gram.:m.}
\end{itemize}
\begin{itemize}
\item {Proveniência:(Do gr. \textunderscore homalos\textunderscore  + \textunderscore kephale\textunderscore )}
\end{itemize}
Reptil sáurio.
\section{Homalográfico}
\begin{itemize}
\item {Grp. gram.:adj.}
\end{itemize}
\begin{itemize}
\item {Utilização:Geogr.}
\end{itemize}
\begin{itemize}
\item {Proveniência:(Do gr. \textunderscore homalos\textunderscore  + \textunderscore graphein\textunderscore )}
\end{itemize}
Diz-se da projecção da esfera, em que os paralelos são rectilíneos.
\section{Homalográphico}
\begin{itemize}
\item {Grp. gram.:adj.}
\end{itemize}
\begin{itemize}
\item {Utilização:Geogr.}
\end{itemize}
\begin{itemize}
\item {Proveniência:(Do gr. \textunderscore homalos\textunderscore  + \textunderscore graphein\textunderscore )}
\end{itemize}
Diz-se da projecção da esphera, em que os parallelos são rectilíneos.
\section{Hombo}
\begin{itemize}
\item {Grp. gram.:m.}
\end{itemize}
Ave africana, (\textunderscore biconia episcopus\textunderscore ).
\section{Hombridade}
\begin{itemize}
\item {Grp. gram.:f.}
\end{itemize}
\begin{itemize}
\item {Utilização:Fig.}
\end{itemize}
\begin{itemize}
\item {Utilização:Ext.}
\end{itemize}
\begin{itemize}
\item {Proveniência:(Do cast. \textunderscore hombre\textunderscore , homem)}
\end{itemize}
Aspecto varonil.
O mesmo que \textunderscore corporatura\textunderscore . Cf. Garrett, \textunderscore Fábulas\textunderscore , 60.
Nobreza de carácter.
Altivez louvável.
Magnanimidade.
Desejo de hombrear com alguém.
\section{Hombrigolulo}
\begin{itemize}
\item {Grp. gram.:m.}
\end{itemize}
Árvore de Angola.
\section{Hombro}
\textunderscore m.\textunderscore  (e der.)
(V. \textunderscore ombro\textunderscore , etc.)
\section{Home}
\begin{itemize}
\item {Grp. gram.:m.}
\end{itemize}
\begin{itemize}
\item {Utilização:ant.}
\end{itemize}
\begin{itemize}
\item {Utilização:Pop.}
\end{itemize}
O mesmo que \textunderscore homem\textunderscore . Cf. Camillo, \textunderscore Brasileira\textunderscore , 290.
\section{Homem}
\begin{itemize}
\item {Grp. gram.:m.}
\end{itemize}
\begin{itemize}
\item {Utilização:Fam.}
\end{itemize}
\begin{itemize}
\item {Utilização:Pop.}
\end{itemize}
\begin{itemize}
\item {Proveniência:(Do lat. \textunderscore homo\textunderscore )}
\end{itemize}
Animal racional e mammífero, que pela sua intelligência, pelo dom da palavra e pela história, se distingue dos outros seres organizados, occupando entre elles o primeiro lugar.
Indivíduo da espécie humana: \textunderscore o homem é mortal\textunderscore .
A humanidade: \textunderscore o homem nasceu para a sociedade\textunderscore .
Pessôa do sexo masculino: \textunderscore duas mulheres e três homens\textunderscore .
Pessôa adulta do sexo masculino: \textunderscore vão ali dois rapazes e um homem\textunderscore .
Marido: \textunderscore a Quitéria nunca sai sem o seu homem\textunderscore .
Aquelle que tem capacidade para certos fins.
Aquelle que procede com madureza, que tem experiência do mundo.
Espécie de jôgo de rapazes.
\section{Homem-bom}
\begin{itemize}
\item {Grp. gram.:m.}
\end{itemize}
\begin{itemize}
\item {Utilização:Ant.}
\end{itemize}
Indivíduo da classe dos herdadores, entre as classes não nobres.
Indivíduo mais respeitável das classes nobres. Cf. Herculano, \textunderscore Hist. de Port.\textunderscore , III, 320.
\section{Homemzarrão}
\begin{itemize}
\item {Grp. gram.:m.}
\end{itemize}
Homem muito encorpado.
Homem distinto.
\section{Homemzinho}
\begin{itemize}
\item {Grp. gram.:m.}
\end{itemize}
\begin{itemize}
\item {Utilização:Fig.}
\end{itemize}
\begin{itemize}
\item {Proveniência:(De \textunderscore homem\textunderscore )}
\end{itemize}
Homem de pequena estatura.
Rapaz que vai entrando na adolescência.
Homem insignificante, sem importância.
\section{Homenage}
\begin{itemize}
\item {Grp. gram.:f.}
\end{itemize}
\begin{itemize}
\item {Utilização:Pop.}
\end{itemize}
O mesmo que \textunderscore homenagem\textunderscore . Cf. Filinto, I, 361.
\section{Homenagear}
\begin{itemize}
\item {Grp. gram.:v. t.}
\end{itemize}
\begin{itemize}
\item {Utilização:Neol.}
\end{itemize}
Prestar homenagem a.
\section{Homenagem}
\begin{itemize}
\item {Grp. gram.:f.}
\end{itemize}
\begin{itemize}
\item {Proveniência:(Do b. lat. \textunderscore hominaticus\textunderscore )}
\end{itemize}
Promessa da fidelidade, que o vassallo fazia ao senhor feudal.
Protesto de veneração e respeito.
Preito: \textunderscore receba as minhas homenagens\textunderscore .
\section{Homeopata}
\begin{itemize}
\item {Grp. gram.:m.}
\end{itemize}
Partidário da homeopatia.
\section{Homeopatha}
\begin{itemize}
\item {Grp. gram.:m.}
\end{itemize}
Partidário da homeopathia.
\section{Homeopathia}
\begin{itemize}
\item {Grp. gram.:f.}
\end{itemize}
\begin{itemize}
\item {Proveniência:(Do gr. \textunderscore homoios\textunderscore  + \textunderscore pathos\textunderscore )}
\end{itemize}
Systema de tratar doenças por meio de agentes, que podem determinar affecção análoga á que se procura debellar.
\section{Homeopathicamente}
\begin{itemize}
\item {Grp. gram.:adj.}
\end{itemize}
Por processos homeopáthicos.
\section{Homeopáthico}
\begin{itemize}
\item {Grp. gram.:adj.}
\end{itemize}
\begin{itemize}
\item {Utilização:Fam.}
\end{itemize}
Relativo á homeopathia.
Feito aos poucos, em pequenas porções, como por gotas: \textunderscore doses homeopáthicas\textunderscore ; \textunderscore alimentação homeopáthica\textunderscore .
\section{Homeopatia}
\begin{itemize}
\item {Grp. gram.:f.}
\end{itemize}
\begin{itemize}
\item {Proveniência:(Do gr. \textunderscore homoios\textunderscore  + \textunderscore pathos\textunderscore )}
\end{itemize}
Sistema de tratar doenças por meio de agentes, que podem determinar afecção análoga á que se procura debelar.
\section{Homeopaticamente}
\begin{itemize}
\item {Grp. gram.:adj.}
\end{itemize}
Por processos homeopáticos.
\section{Homeopático}
\begin{itemize}
\item {Grp. gram.:adj.}
\end{itemize}
\begin{itemize}
\item {Utilização:Fam.}
\end{itemize}
Relativo á homeopatia.
Feito aos poucos, em pequenas porções, como por gotas: \textunderscore doses homeopáticas\textunderscore ; \textunderscore alimentação homeopática\textunderscore .
\section{Homeotropia}
\begin{itemize}
\item {Grp. gram.:f.}
\end{itemize}
Qualidade de homeótropo.
\section{Homeótropo}
\begin{itemize}
\item {Grp. gram.:adj.}
\end{itemize}
\begin{itemize}
\item {Utilização:Philol.}
\end{itemize}
\begin{itemize}
\item {Proveniência:(Do gr. \textunderscore homoios\textunderscore  + \textunderscore trepein\textunderscore )}
\end{itemize}
Diz-se da fórma que, em virtude de leis phonéticas, resulta de dois ou mais étimos differentes, como \textunderscore pena\textunderscore , do lat. \textunderscore poena\textunderscore , e \textunderscore penna\textunderscore .
\section{Homérico}
\begin{itemize}
\item {Grp. gram.:adj.}
\end{itemize}
\begin{itemize}
\item {Utilização:Fig.}
\end{itemize}
\begin{itemize}
\item {Proveniência:(De \textunderscore Homero\textunderscore , n. p.)}
\end{itemize}
Relativo a Homero ou ás suas obras ou ao seu estilo.
Grande; épico: \textunderscore campanhas homéricas\textunderscore .
Retumbante.
\section{Homérida}
\begin{itemize}
\item {Grp. gram.:m.}
\end{itemize}
\begin{itemize}
\item {Proveniência:(Do rad. de \textunderscore Homero\textunderscore , n. p.)}
\end{itemize}
Membro de uma escola de rapsodos, que se diffundiu pela Grécia.
Qualquer imitador de Homero.
Aquelle que em público recitava versos de Homero.
\section{Homério}
\begin{itemize}
\item {Grp. gram.:adj.}
\end{itemize}
O mesmo que \textunderscore homérico\textunderscore . Cf. Filinto, XVI, 137.
\section{Homicida}
\begin{itemize}
\item {Grp. gram.:m. ,  f.  e  adj.}
\end{itemize}
\begin{itemize}
\item {Proveniência:(Lat. \textunderscore homicida\textunderscore )}
\end{itemize}
Pessôa, que pratíca homicídio.
Que produz morte de outra pessôa ou pessôas.
\section{Homicidiar}
\begin{itemize}
\item {Grp. gram.:v. t.}
\end{itemize}
Praticar homicídio contra. Cf. Filinto, XVII, 114.
\section{Homicídio}
\begin{itemize}
\item {Grp. gram.:m.}
\end{itemize}
\begin{itemize}
\item {Proveniência:(Lat. \textunderscore homicidium\textunderscore )}
\end{itemize}
Morte, causada por uma pessôa ou pessôas a outra ou outras.
Tributo antigo, pago pelos povos, que tinham o direito de não entregar á justiça o homicida que entre elles se refugiava.
\section{Homília}
\begin{itemize}
\item {Grp. gram.:f.}
\end{itemize}
\begin{itemize}
\item {Proveniência:(Lat. \textunderscore homilia\textunderscore )}
\end{itemize}
Prática sôbre coisas de religião; catechese.
\section{Homiliar}
\begin{itemize}
\item {Grp. gram.:v. i.}
\end{itemize}
Fazer homílias.
\section{Homiliário}
\begin{itemize}
\item {Grp. gram.:m.}
\end{itemize}
Livro, que contém homílias.
\section{Homiliasta}
\begin{itemize}
\item {Grp. gram.:m.}
\end{itemize}
Aquelle que faz homília.
\section{Hominal}
\begin{itemize}
\item {Grp. gram.:adj.}
\end{itemize}
\begin{itemize}
\item {Proveniência:(Do lat. \textunderscore homo\textunderscore , \textunderscore hominis\textunderscore )}
\end{itemize}
Relativo ao homem.
O mesmo que \textunderscore hominiano\textunderscore .
\section{Hominalidade}
\begin{itemize}
\item {Grp. gram.:f.}
\end{itemize}
\begin{itemize}
\item {Utilização:Neol.}
\end{itemize}
\begin{itemize}
\item {Proveniência:(De \textunderscore hominal\textunderscore )}
\end{itemize}
Carácter hominal.
Essência hominal.
Acção ou fôrça privativa da natureza humana.
\section{Hòminho}
\begin{itemize}
\item {Grp. gram.:m.}
\end{itemize}
\begin{itemize}
\item {Utilização:Fam.}
\end{itemize}
\begin{itemize}
\item {Proveniência:(De \textunderscore home\textunderscore )}
\end{itemize}
Indivíduo, de fraca figura.
Homem sem importância.
\section{Hominiano}
\begin{itemize}
\item {Grp. gram.:adj.}
\end{itemize}
O mesmo que \textunderscore homínido\textunderscore .
\section{Hominícola}
\begin{itemize}
\item {Grp. gram.:m.}
\end{itemize}
\begin{itemize}
\item {Proveniência:(Do lat. \textunderscore homo\textunderscore  + \textunderscore colere\textunderscore )}
\end{itemize}
Aquelle que adora um homem.
\section{Homínido}
\begin{itemize}
\item {Grp. gram.:adj.}
\end{itemize}
\begin{itemize}
\item {Grp. gram.:M. pl.}
\end{itemize}
\begin{itemize}
\item {Proveniência:(Do lat. \textunderscore homo\textunderscore  + gr. \textunderscore eidos\textunderscore )}
\end{itemize}
Semelhante ao homem, (falando-se de mammíferos).
Família de mammíferos primatas, que tem por typo o homem.
\section{Homiziação}
\begin{itemize}
\item {Grp. gram.:f.}
\end{itemize}
O mesmo que \textunderscore homizio\textunderscore . Cf. Júl. Dinis, \textunderscore Fidalgos\textunderscore , II, 121.
\section{Homiziado}
\begin{itemize}
\item {Grp. gram.:m.}
\end{itemize}
\begin{itemize}
\item {Proveniência:(De \textunderscore homiziar\textunderscore )}
\end{itemize}
Aquelle que anda fugido á justiça.
\section{Homizião}
\begin{itemize}
\item {Grp. gram.:m.}
\end{itemize}
\begin{itemize}
\item {Utilização:Ant.}
\end{itemize}
Inimigo, adversário.
(Cp. \textunderscore homizio\textunderscore )
\section{Homiziar}
\begin{itemize}
\item {Grp. gram.:v. t.}
\end{itemize}
\begin{itemize}
\item {Proveniência:(De \textunderscore homizio\textunderscore )}
\end{itemize}
Indispor.
Intrigar, (p. us. nestes sentidos).
Dar guarida a.
Acceitar.
Esconder á acção da justiça.
Esconder; encobrir.
\section{Homizieiro}
\begin{itemize}
\item {Grp. gram.:m.}
\end{itemize}
\begin{itemize}
\item {Utilização:Ant.}
\end{itemize}
\begin{itemize}
\item {Proveniência:(De \textunderscore homizío\textunderscore )}
\end{itemize}
O mesmo que \textunderscore homicida\textunderscore . Cf. Herculano, \textunderscore Hist. de Port.\textunderscore , IV, 292 e 394.
\section{Homizio}
\begin{itemize}
\item {Grp. gram.:m.}
\end{itemize}
\begin{itemize}
\item {Utilização:Ant.}
\end{itemize}
\begin{itemize}
\item {Proveniência:(Do lat. \textunderscore homicidium\textunderscore )}
\end{itemize}
Acto ou effeito de homiziar.
Esconderijo; valhacoito.
Homicídio, morte de homem.
Coima ou tributo antigo.
Malefício ou crime que, segundo as nossas leis antigas, era punido com a morte, destêrro, perdição de bens, etc. Cf. \textunderscore Elucidário\textunderscore .
\section{Homo...}
\begin{itemize}
\item {Grp. gram.:pref.}
\end{itemize}
\begin{itemize}
\item {Proveniência:(Do gr. \textunderscore homos\textunderscore )}
\end{itemize}
(designativo de igual, semelhante)
\section{Homoblásteo}
\begin{itemize}
\item {Grp. gram.:adj.}
\end{itemize}
\begin{itemize}
\item {Utilização:Bot.}
\end{itemize}
\begin{itemize}
\item {Proveniência:(De \textunderscore homo...\textunderscore  + \textunderscore blasto\textunderscore )}
\end{itemize}
Que tem a radícula voltada para o hilo.
\section{Homocatalecto}
\begin{itemize}
\item {Grp. gram.:m.}
\end{itemize}
\begin{itemize}
\item {Proveniência:(Do gr. \textunderscore homos\textunderscore  + \textunderscore katalegein\textunderscore )}
\end{itemize}
Designação commum dos homoptotos e dos homoteleutos.
\section{Homocatalexia}
\begin{itemize}
\item {fónica:csi}
\end{itemize}
\begin{itemize}
\item {Grp. gram.:f.}
\end{itemize}
\begin{itemize}
\item {Utilização:Gram.}
\end{itemize}
O mesmo que \textunderscore consonância\textunderscore .
(Cp. \textunderscore homocatalecto\textunderscore )
\section{Homocentricamente}
\begin{itemize}
\item {Grp. gram.:adv.}
\end{itemize}
De modo homocêntrico.
\section{Homocêntrico}
\begin{itemize}
\item {Grp. gram.:adj.}
\end{itemize}
O mesmo que \textunderscore concêntrico\textunderscore .
\section{Homocentro}
\begin{itemize}
\item {Grp. gram.:m.}
\end{itemize}
\begin{itemize}
\item {Proveniência:(De \textunderscore homo...\textunderscore  + \textunderscore centro\textunderscore )}
\end{itemize}
Centro commum de muitos circulos.
\section{Homochrónico}
\begin{itemize}
\item {Grp. gram.:adj.}
\end{itemize}
O mesmo que \textunderscore homóchrono\textunderscore .
\section{Homóchrono}
\begin{itemize}
\item {Grp. gram.:adj.}
\end{itemize}
\begin{itemize}
\item {Proveniência:(Do gr. \textunderscore homos\textunderscore  + \textunderscore khronos\textunderscore )}
\end{itemize}
O mesmo que \textunderscore sýnchrono\textunderscore .
\section{Homocrónico}
\begin{itemize}
\item {Grp. gram.:adj.}
\end{itemize}
O mesmo que \textunderscore homócrono\textunderscore .
\section{Homócrono}
\begin{itemize}
\item {Grp. gram.:adj.}
\end{itemize}
\begin{itemize}
\item {Proveniência:(Do gr. \textunderscore homos\textunderscore  + \textunderscore khronos\textunderscore )}
\end{itemize}
O mesmo que \textunderscore síncrono\textunderscore .
\section{Homodermes}
\begin{itemize}
\item {Grp. gram.:m. pl.}
\end{itemize}
\begin{itemize}
\item {Proveniência:(Do gr. \textunderscore homo\textunderscore  + \textunderscore derma\textunderscore )}
\end{itemize}
Família de reptis, cuja pelle é toda coberta de escamas iguaes.
\section{Homodinâmica}
\begin{itemize}
\item {Grp. gram.:f.}
\end{itemize}
\begin{itemize}
\item {Proveniência:(Do gr. \textunderscore homos\textunderscore  + \textunderscore dunamis\textunderscore )}
\end{itemize}
Relação homológica entre órgãos ímpares, seriados axialmente.
\section{Homodinâmico}
\begin{itemize}
\item {Grp. gram.:adj.}
\end{itemize}
Relativo á homodinamia.
\section{Homódromo}
\begin{itemize}
\item {Grp. gram.:adj.}
\end{itemize}
\begin{itemize}
\item {Proveniência:(Do gr. \textunderscore homos\textunderscore  + \textunderscore dromos\textunderscore )}
\end{itemize}
Dizia-se da alavanca, em que a resistência e a potência estavam do mesmo lado, relativamente ao ponto de apoio.
\section{Homodynâmica}
\begin{itemize}
\item {Grp. gram.:f.}
\end{itemize}
\begin{itemize}
\item {Proveniência:(Do gr. \textunderscore homos\textunderscore  + \textunderscore dunamis\textunderscore )}
\end{itemize}
Relação homológica entre órgãos ímpares, seriados axialmente.
\section{Homodynâmico}
\begin{itemize}
\item {Grp. gram.:adj.}
\end{itemize}
Relativo á homodynamia.
\section{Homófago}
\begin{itemize}
\item {Grp. gram.:adj.}
\end{itemize}
\begin{itemize}
\item {Proveniência:(Do gr. \textunderscore homos\textunderscore  + \textunderscore phagein\textunderscore )}
\end{itemize}
Que se alimenta de carne crua.
\section{Homofilo}
\begin{itemize}
\item {Grp. gram.:adj.}
\end{itemize}
\begin{itemize}
\item {Utilização:Bot.}
\end{itemize}
\begin{itemize}
\item {Proveniência:(Do gr. \textunderscore homos\textunderscore  + \textunderscore phullon\textunderscore )}
\end{itemize}
Cujas fôlhas ou folíolos são semelhantes.
\section{Homofonia}
\begin{itemize}
\item {Grp. gram.:f.}
\end{itemize}
\begin{itemize}
\item {Proveniência:(Do gr. \textunderscore homo\textunderscore  + \textunderscore phone\textunderscore )}
\end{itemize}
Semelhança de sons ou de pronúncia.
\section{Homofónico}
\begin{itemize}
\item {Grp. gram.:adj.}
\end{itemize}
\begin{itemize}
\item {Proveniência:(De \textunderscore homofonia\textunderscore )}
\end{itemize}
Que tem o mesmo som ou que se pronuncía da mesma fórma; homónimo.
\section{Homofonismo}
\begin{itemize}
\item {Grp. gram.:m.}
\end{itemize}
O mesmo que \textunderscore homofonia\textunderscore .
\section{Homofonógrafo}
\begin{itemize}
\item {Grp. gram.:adj.}
\end{itemize}
\begin{itemize}
\item {Utilização:Gram.}
\end{itemize}
\begin{itemize}
\item {Proveniência:(Do gr. \textunderscore homos\textunderscore  + \textunderscore phone\textunderscore  + \textunderscore grapheín\textunderscore )}
\end{itemize}
Diz-se das palavras, que se escrevem e se pronunciam da mesma fórma, tendo sentido e origens diferentes, como \textunderscore peça\textunderscore , substantivo, e \textunderscore peça\textunderscore , do verbo \textunderscore pedir\textunderscore .
\section{Homofonologia}
\begin{itemize}
\item {Grp. gram.:f.}
\end{itemize}
\begin{itemize}
\item {Proveniência:(Do gr. \textunderscore homos\textunderscore  + \textunderscore phone\textunderscore  + \textunderscore logos\textunderscore )}
\end{itemize}
Estudo das palavras homofonicas.
\section{Homofonológico}
\begin{itemize}
\item {Grp. gram.:adj.}
\end{itemize}
Relativo á homofonologia. Cf. Pedro Leal, \textunderscore Diccion. Homophon.\textunderscore 
\section{Homogamia}
\begin{itemize}
\item {Grp. gram.:f.}
\end{itemize}
\begin{itemize}
\item {Proveniência:(De \textunderscore homógamo\textunderscore )}
\end{itemize}
Estado de uma planta homógama.
\section{Homógamo}
\begin{itemize}
\item {Grp. gram.:adj.}
\end{itemize}
\begin{itemize}
\item {Proveniência:(Do gr. \textunderscore homos\textunderscore  + \textunderscore gamos\textunderscore )}
\end{itemize}
Diz-se das plantas, cujas flôres são do mesmo sexo.
\section{Homogeneamente}
\begin{itemize}
\item {Grp. gram.:adv.}
\end{itemize}
De modo homogêneo.
\section{Homogeneidade}
\begin{itemize}
\item {Grp. gram.:f.}
\end{itemize}
Qualidade daquillo que é homogêneo.
\section{Homogeneizar}
\begin{itemize}
\item {Grp. gram.:v. t.}
\end{itemize}
Tornar homogêneo.
\section{Homogêneo}
\begin{itemize}
\item {Grp. gram.:adj.}
\end{itemize}
\begin{itemize}
\item {Proveniência:(Do gr. \textunderscore homos\textunderscore  + \textunderscore genes\textunderscore )}
\end{itemize}
Que tem a mesma natureza, ou que é do mesmo gênero que outro objecto.
Idêntico.
\section{Homogenesia}
\begin{itemize}
\item {Grp. gram.:f.}
\end{itemize}
Affinidade sexual.
Homogenia. Cf. E. Burnay, \textunderscore Craniologia\textunderscore , 64.
\section{Homogenia}
\begin{itemize}
\item {Grp. gram.:f.}
\end{itemize}
\begin{itemize}
\item {Proveniência:(Gr. \textunderscore homogeneia\textunderscore )}
\end{itemize}
Modo de geração de um sêr, produzido por sêres da mesma espécie.
\section{Homografia}
\begin{itemize}
\item {Grp. gram.:f.}
\end{itemize}
\begin{itemize}
\item {Utilização:Geom.}
\end{itemize}
\begin{itemize}
\item {Proveniência:(De \textunderscore homôgrafo\textunderscore )}
\end{itemize}
Dependência recíproca de duas linhas.
\section{Homograficamente}
\begin{itemize}
\item {Grp. gram.:adj.}
\end{itemize}
De modo homográfico.
\section{Homográfico}
\begin{itemize}
\item {Grp. gram.:adj.}
\end{itemize}
Relativo á homografia.
\section{Homógrafo}
\begin{itemize}
\item {Grp. gram.:adj.}
\end{itemize}
\begin{itemize}
\item {Utilização:Gram.}
\end{itemize}
\begin{itemize}
\item {Proveniência:(Do gr. \textunderscore homos\textunderscore  + \textunderscore graphein\textunderscore )}
\end{itemize}
Diz-se das palavras, que se escrevem da mesma fórma e têm sentido diferente: \textunderscore o pequeno, quando nada, não vê nada\textunderscore .
\section{Homographia}
\begin{itemize}
\item {Grp. gram.:f.}
\end{itemize}
\begin{itemize}
\item {Utilização:Geom.}
\end{itemize}
\begin{itemize}
\item {Proveniência:(De \textunderscore homôgrapho\textunderscore )}
\end{itemize}
Dependência recíproca de duas linhas.
\section{Homographicamente}
\begin{itemize}
\item {Grp. gram.:adj.}
\end{itemize}
De modo homográphico.
\section{Homográphico}
\begin{itemize}
\item {Grp. gram.:adj.}
\end{itemize}
Relativo á homographia.
\section{Homógrapho}
\begin{itemize}
\item {Grp. gram.:adj.}
\end{itemize}
\begin{itemize}
\item {Utilização:Gram.}
\end{itemize}
\begin{itemize}
\item {Proveniência:(Do gr. \textunderscore homos\textunderscore  + \textunderscore graphein\textunderscore )}
\end{itemize}
Diz-se das palavras, que se escrevem da mesma fórma e têm sentido differente: \textunderscore o pequeno, quando nada, não vê nada\textunderscore .
\section{Homoide}
\begin{itemize}
\item {Grp. gram.:adj.}
\end{itemize}
\begin{itemize}
\item {Proveniência:(Do gr. \textunderscore homos\textunderscore  + \textunderscore eidos\textunderscore )}
\end{itemize}
Diz-se das partes das plantas, que têm a mesma fórma que os seus invólucros.
Diz-se do mestiço, procedente de duas raças da mesma espécie.
\section{Hómolo}
\begin{itemize}
\item {Grp. gram.:m.}
\end{itemize}
Gênero de crustáceos decápodes.
\section{Homologação}
\begin{itemize}
\item {Grp. gram.:f.}
\end{itemize}
Acto ou effeito de homologar.
\section{Homologar}
\begin{itemize}
\item {Grp. gram.:v. t.}
\end{itemize}
\begin{itemize}
\item {Proveniência:(De \textunderscore homólogo\textunderscore )}
\end{itemize}
Confirmar por sentença ou por autoridade judicial; conformar-se com: \textunderscore o Ministro homologou aquella decisão do Conselho Superior\textunderscore .
\section{Homologia}
\begin{itemize}
\item {Grp. gram.:f.}
\end{itemize}
\begin{itemize}
\item {Proveniência:(De \textunderscore homólogo\textunderscore )}
\end{itemize}
Repetição das mesmas palavras, conceitos, figuras, etc., no mesmo discurso.
\section{Homológico}
\begin{itemize}
\item {Grp. gram.:adj.}
\end{itemize}
Relativo a homologia.
\section{Homólogo}
\begin{itemize}
\item {Grp. gram.:adj.}
\end{itemize}
\begin{itemize}
\item {Utilização:Geom.}
\end{itemize}
\begin{itemize}
\item {Utilização:Hist. Nat.}
\end{itemize}
\begin{itemize}
\item {Proveniência:(Gr. \textunderscore homologos\textunderscore )}
\end{itemize}
Diz-se dos lados, que se correspondem e são oppostos a ângulos iguaes, em figuras semelhantes.
Diz-se das substâncias orgânicas, que desempenham funcções idênticas.
\section{Homomeria}
\begin{itemize}
\item {Grp. gram.:f.}
\end{itemize}
\begin{itemize}
\item {Proveniência:(Do gr. \textunderscore homos\textunderscore  + \textunderscore meros\textunderscore )}
\end{itemize}
Homogeneidade dos elementos, a que alguns sábios attribuíram a formação do mundo.
\section{Homómero}
\begin{itemize}
\item {Grp. gram.:adj.}
\end{itemize}
\begin{itemize}
\item {Proveniência:(Do gr. \textunderscore homos\textunderscore  + \textunderscore meros\textunderscore )}
\end{itemize}
Cujas partes são todas semelhantes.
\section{Homomerologia}
\begin{itemize}
\item {Grp. gram.:f.}
\end{itemize}
\begin{itemize}
\item {Proveniência:(Do gr. \textunderscore homos\textunderscore  + \textunderscore meros\textunderscore  + \textunderscore logos\textunderscore )}
\end{itemize}
Tratado dos systemas orgânicos.
\section{Homométrico}
\begin{itemize}
\item {Grp. gram.:adj.}
\end{itemize}
\begin{itemize}
\item {Proveniência:(Do gr. \textunderscore homos\textunderscore  + \textunderscore metron\textunderscore )}
\end{itemize}
Diz-se das composições poéticas, cuja medida é igual á de outras.
\section{Homomorfismo}
\begin{itemize}
\item {Grp. gram.:m.}
\end{itemize}
Qualidade daquilo que é homomorfo.
\section{Homomorfo}
\begin{itemize}
\item {Grp. gram.:adj.}
\end{itemize}
\begin{itemize}
\item {Proveniência:(Do gr. \textunderscore homos\textunderscore  + \textunderscore morphe\textunderscore )}
\end{itemize}
Que tem a mesma fórma.
\section{Homomorphismo}
\begin{itemize}
\item {Grp. gram.:m.}
\end{itemize}
Qualidade daquillo que é homomorpho.
\section{Homomorpho}
\begin{itemize}
\item {Grp. gram.:adj.}
\end{itemize}
\begin{itemize}
\item {Proveniência:(Do gr. \textunderscore homos\textunderscore  + \textunderscore morphe\textunderscore )}
\end{itemize}
Que tem a mesma fórma.
\section{Homonímia}
\begin{itemize}
\item {Grp. gram.:f.}
\end{itemize}
Qualidade daquilo que é homónimo.
\section{Homonímico}
\begin{itemize}
\item {Grp. gram.:adj.}
\end{itemize}
Em que há homonímia. Cf. Camillo, \textunderscore Noites de Insómn.\textunderscore , III, 58.
\section{Homónimo}
\begin{itemize}
\item {Grp. gram.:adj.}
\end{itemize}
\begin{itemize}
\item {Grp. gram.:M.}
\end{itemize}
\begin{itemize}
\item {Proveniência:(Do gr. \textunderscore homos\textunderscore  + \textunderscore onuma\textunderscore )}
\end{itemize}
Que tem o mesmo nome; que se pronuncía da mesma fórma, embora a ortografia e a origem sejam diferentes: \textunderscore «cinto»e«sinto»são vocábulos homónimos\textunderscore .
Aquele que tem o mesmo nome que outrem.
\section{Homonomia}
\begin{itemize}
\item {Grp. gram.:f.}
\end{itemize}
\begin{itemize}
\item {Utilização:Anat.}
\end{itemize}
\begin{itemize}
\item {Proveniência:(Do gr. \textunderscore homos\textunderscore  + \textunderscore nomos\textunderscore )}
\end{itemize}
Relação homológica entre órgãos pares, seriados transversalmente, mas só comparados unilateralmente.
\section{Homonómico}
\begin{itemize}
\item {Grp. gram.:adj.}
\end{itemize}
Relativo á homonomia.
\section{Homónomo}
\begin{itemize}
\item {Grp. gram.:adj.}
\end{itemize}
\begin{itemize}
\item {Utilização:Miner.}
\end{itemize}
\begin{itemize}
\item {Proveniência:(Do gr. \textunderscore homos\textunderscore  + \textunderscore nomos\textunderscore )}
\end{itemize}
Diz-se dos crystaes que, em todos os seus pontos, obedecem á mesma lei.
\section{Homonýmia}
\begin{itemize}
\item {Grp. gram.:f.}
\end{itemize}
Qualidade daquillo que é homónymo.
\section{Homonýmico}
\begin{itemize}
\item {Grp. gram.:adj.}
\end{itemize}
Em que há homonýmia. Cf. Camillo, \textunderscore Noites de Insómn.\textunderscore , III, 58.
\section{Homónymo}
\begin{itemize}
\item {Grp. gram.:adj.}
\end{itemize}
\begin{itemize}
\item {Grp. gram.:M.}
\end{itemize}
\begin{itemize}
\item {Proveniência:(Do gr. \textunderscore homos\textunderscore  + \textunderscore onuma\textunderscore )}
\end{itemize}
Que tem o mesmo nome; que se pronuncía da mesma fórma, embora a orthographia e a origem sejam differentes: \textunderscore «cinto»e«sinto»são vocábulos homónymos\textunderscore .
Aquelle que tem o mesmo nome que outrem.
\section{Homopétalo}
\begin{itemize}
\item {Grp. gram.:adj.}
\end{itemize}
\begin{itemize}
\item {Proveniência:(De \textunderscore homo...\textunderscore  + \textunderscore pétela\textunderscore )}
\end{itemize}
Que tem pétalas semelhantes.
\section{Homóphago}
\begin{itemize}
\item {Grp. gram.:adj.}
\end{itemize}
\begin{itemize}
\item {Proveniência:(Do gr. \textunderscore homos\textunderscore  + \textunderscore phagein\textunderscore )}
\end{itemize}
Que se alimenta de carne crua.
\section{Homophonia}
\begin{itemize}
\item {Grp. gram.:f.}
\end{itemize}
\begin{itemize}
\item {Proveniência:(Do gr. \textunderscore homo\textunderscore  + \textunderscore phone\textunderscore )}
\end{itemize}
Semelhança de sons ou de pronúncia.
\section{Homophónico}
\begin{itemize}
\item {Grp. gram.:adj.}
\end{itemize}
\begin{itemize}
\item {Proveniência:(De \textunderscore homophonia\textunderscore )}
\end{itemize}
Que tem o mesmo som ou que se pronuncía da mesma fórma; homónymo.
\section{Homophonismo}
\begin{itemize}
\item {Grp. gram.:m.}
\end{itemize}
O mesmo que \textunderscore homophonia\textunderscore .
\section{Homophonógrapho}
\begin{itemize}
\item {Grp. gram.:adj.}
\end{itemize}
\begin{itemize}
\item {Utilização:Gram.}
\end{itemize}
\begin{itemize}
\item {Proveniência:(Do gr. \textunderscore homos\textunderscore  + \textunderscore phone\textunderscore  + \textunderscore grapheín\textunderscore )}
\end{itemize}
Diz-se das palavras, que se escrevem e se pronunciam da mesma fórma, tendo sentido e origens differentes, como \textunderscore peça\textunderscore , substantivo, e \textunderscore peça\textunderscore , do verbo \textunderscore pedir\textunderscore .
\section{Homophonologia}
\begin{itemize}
\item {Grp. gram.:f.}
\end{itemize}
\begin{itemize}
\item {Proveniência:(Do gr. \textunderscore homos\textunderscore  + \textunderscore phone\textunderscore  + \textunderscore logos\textunderscore )}
\end{itemize}
Estudo das palavras homophonicas.
\section{Homophonológico}
\begin{itemize}
\item {Grp. gram.:adj.}
\end{itemize}
Relativo á homophonologia. Cf. Pedro Leal, \textunderscore Diccion. Homophon.\textunderscore 
\section{Homophyllo}
\begin{itemize}
\item {Grp. gram.:adj.}
\end{itemize}
\begin{itemize}
\item {Utilização:Bot.}
\end{itemize}
\begin{itemize}
\item {Proveniência:(Do gr. \textunderscore homos\textunderscore  + \textunderscore phullon\textunderscore )}
\end{itemize}
Cujas fôlhas ou folíolos são semelhantes.
\section{Homoplasia}
\begin{itemize}
\item {Grp. gram.:f.}
\end{itemize}
\begin{itemize}
\item {Utilização:Physiol.}
\end{itemize}
\begin{itemize}
\item {Proveniência:(Do gr. \textunderscore homos\textunderscore  + \textunderscore plassein\textunderscore )}
\end{itemize}
Formação de tecidos mórbidos, semelhantes aos normaes.
\section{Homoplástico}
\begin{itemize}
\item {Grp. gram.:adj.}
\end{itemize}
Relativo á homoplasia.
\section{Homópodes}
\begin{itemize}
\item {Grp. gram.:m. pl.}
\end{itemize}
\begin{itemize}
\item {Proveniência:(Do gr. \textunderscore homos\textunderscore  + \textunderscore pous\textunderscore , \textunderscore podos\textunderscore )}
\end{itemize}
Ordem de crustáceos.
\section{Homópteros}
\begin{itemize}
\item {Grp. gram.:m. pl.}
\end{itemize}
\begin{itemize}
\item {Proveniência:(Do gr. \textunderscore homos\textunderscore  + \textunderscore pteron\textunderscore )}
\end{itemize}
Tríbo de hemípteros.
\section{Homoptoto}
\begin{itemize}
\item {Grp. gram.:m.}
\end{itemize}
\begin{itemize}
\item {Grp. gram.:Adj.}
\end{itemize}
\begin{itemize}
\item {Utilização:Gram.}
\end{itemize}
O mesmo ou melhor que \textunderscore homoptoton\textunderscore .
Diz-se dos vocábulos, que têm o mesmo prefixo.
\section{Homoptóton}
\begin{itemize}
\item {Grp. gram.:m.}
\end{itemize}
\begin{itemize}
\item {Utilização:Gram.}
\end{itemize}
\begin{itemize}
\item {Proveniência:(Do gr. \textunderscore homos\textunderscore  + \textunderscore ptosis\textunderscore )}
\end{itemize}
Emprêgo successivo de verbos nos mesmos tempos e pessôas, ou de nomes nos mesmos casos.
\section{Homorgânico}
\begin{itemize}
\item {Grp. gram.:adj.}
\end{itemize}
\begin{itemize}
\item {Utilização:Gram.}
\end{itemize}
\begin{itemize}
\item {Utilização:Anat.}
\end{itemize}
\begin{itemize}
\item {Proveniência:(De \textunderscore homo...\textunderscore  + \textunderscore orgânico\textunderscore )}
\end{itemize}
Diz-se das letras, cuja pronúncia depende do mesmo órgão.
Diz-se daquillo que é semelhante em organização a outro objecto.
\section{Homose}
\begin{itemize}
\item {Grp. gram.:f.}
\end{itemize}
\begin{itemize}
\item {Proveniência:(Do gr. \textunderscore homos\textunderscore )}
\end{itemize}
Comparação de um objecto com outro.
Assimilação e cocção de suco nutritivo.
\section{Homosexual}
\begin{itemize}
\item {fónica:se}
\end{itemize}
\begin{itemize}
\item {Grp. gram.:adj.}
\end{itemize}
\begin{itemize}
\item {Proveniência:(De \textunderscore homo...\textunderscore  + \textunderscore sexual\textunderscore )}
\end{itemize}
Relativo a actos sensuaes entre indivíduos do mesmo sexo.
Que pratíca êsses actos.
\section{Homosexualismo}
\begin{itemize}
\item {fónica:se}
\end{itemize}
\begin{itemize}
\item {Grp. gram.:m.}
\end{itemize}
\begin{itemize}
\item {Proveniência:(De \textunderscore homosexual\textunderscore )}
\end{itemize}
Prática de actos sensuaes entre indivíduos do mesmo sexo.
\section{Homossexual}
\begin{itemize}
\item {Grp. gram.:adj.}
\end{itemize}
\begin{itemize}
\item {Proveniência:(De \textunderscore homo...\textunderscore  + \textunderscore sexual\textunderscore )}
\end{itemize}
Relativo a actos sensuaes entre indivíduos do mesmo sexo.
Que pratíca êsses actos.
\section{Homossexualismo}
\begin{itemize}
\item {Grp. gram.:m.}
\end{itemize}
\begin{itemize}
\item {Proveniência:(De \textunderscore homossexual\textunderscore )}
\end{itemize}
Prática de actos sensuaes entre indivíduos do mesmo sexo.
\section{Homotecia}
\begin{itemize}
\item {Grp. gram.:f.}
\end{itemize}
O mesmo que \textunderscore homotetia\textunderscore .
\section{Homoteleuto}
\begin{itemize}
\item {Grp. gram.:m.}
\end{itemize}
O mesmo ou melhor que \textunderscore homotelêuton\textunderscore .
\section{Homotelêuton}
\begin{itemize}
\item {Grp. gram.:m.}
\end{itemize}
\begin{itemize}
\item {Utilização:Gram.}
\end{itemize}
\begin{itemize}
\item {Proveniência:(Do gr. \textunderscore homos\textunderscore  + \textunderscore teleute\textunderscore )}
\end{itemize}
Desinência semelhante de palavras successivas.
\section{Homotermal}
\begin{itemize}
\item {Grp. gram.:adj.}
\end{itemize}
\begin{itemize}
\item {Proveniência:(De \textunderscore homo...\textunderscore  + \textunderscore termal\textunderscore )}
\end{itemize}
Que tem a mesma temperatura.
\section{Homotérmico}
\begin{itemize}
\item {Grp. gram.:adj.}
\end{itemize}
O mesmo que \textunderscore homotermal\textunderscore .
\section{Homotesia}
\begin{itemize}
\item {Grp. gram.:f.}
\end{itemize}
\begin{itemize}
\item {Utilização:Geom.}
\end{itemize}
\begin{itemize}
\item {Proveniência:(Do gr. \textunderscore homos\textunderscore  + \textunderscore thesis\textunderscore )}
\end{itemize}
Estado de figuras geométricas semelhantes, e semelhantemente colocadas.
O mesmo que \textunderscore homotetia\textunderscore .
\section{Homotetia}
\begin{itemize}
\item {Grp. gram.:f.}
\end{itemize}
\begin{itemize}
\item {Utilização:Geom.}
\end{itemize}
\begin{itemize}
\item {Proveniência:(Do gr. \textunderscore homos\textunderscore  + \textunderscore tithenai\textunderscore )}
\end{itemize}
Relação entre duas séries de pontos.
\section{Homotético}
\begin{itemize}
\item {Grp. gram.:adj.}
\end{itemize}
Relativo á homotetia ou á homotesia.
\section{Homothecia}
\begin{itemize}
\item {Grp. gram.:f.}
\end{itemize}
O mesmo que \textunderscore homothetia\textunderscore .
\section{Homothermal}
\begin{itemize}
\item {Grp. gram.:adj.}
\end{itemize}
\begin{itemize}
\item {Proveniência:(De \textunderscore homo...\textunderscore  + \textunderscore thermal\textunderscore )}
\end{itemize}
Que tem a mesma temperatura.
\section{Homothérmico}
\begin{itemize}
\item {Grp. gram.:adj.}
\end{itemize}
O mesmo que \textunderscore homothermal\textunderscore .
\section{Homothesia}
\begin{itemize}
\item {Grp. gram.:f.}
\end{itemize}
\begin{itemize}
\item {Utilização:Geom.}
\end{itemize}
\begin{itemize}
\item {Proveniência:(Do gr. \textunderscore homos\textunderscore  + \textunderscore thesis\textunderscore )}
\end{itemize}
Estado de figuras geométricas semelhantes, e semelhantemente collocadas.
O mesmo que \textunderscore homothetia\textunderscore .
\section{Homothetia}
\begin{itemize}
\item {Grp. gram.:f.}
\end{itemize}
\begin{itemize}
\item {Utilização:Geom.}
\end{itemize}
\begin{itemize}
\item {Proveniência:(Do gr. \textunderscore homos\textunderscore  + \textunderscore tithenai\textunderscore )}
\end{itemize}
Relação entre duas séries de pontos.
\section{Homothético}
\begin{itemize}
\item {Grp. gram.:adj.}
\end{itemize}
Relativo á homothetia ou á homothesia.
\section{Homotipia}
\begin{itemize}
\item {Grp. gram.:f.}
\end{itemize}
\begin{itemize}
\item {Proveniência:(De \textunderscore homòtipo\textunderscore )}
\end{itemize}
Carácter dos órgãos homótipos.
Comparação dos órgãos análogos, no mesmo indivíduo.
\section{Homotipicamente}
\begin{itemize}
\item {Grp. gram.:adv.}
\end{itemize}
De modo homotípico.
\section{Homotípico}
\begin{itemize}
\item {Grp. gram.:adj.}
\end{itemize}
Relativo á homotipia.
\section{Homótipo}
\begin{itemize}
\item {Grp. gram.:adj.}
\end{itemize}
\begin{itemize}
\item {Proveniência:(De \textunderscore homo...\textunderscore  + \textunderscore tipo\textunderscore )}
\end{itemize}
Que é análogo ou tem o mesmo tipo, (falando-se dos órgãos do mesmo indivíduo).
\section{Homotomia}
\begin{itemize}
\item {Grp. gram.:f.}
\end{itemize}
\begin{itemize}
\item {Proveniência:(Do gr. \textunderscore homos\textunderscore  + \textunderscore tome\textunderscore )}
\end{itemize}
Escarificação do palato e das amýgdalas.
\section{Homotómico}
\begin{itemize}
\item {Grp. gram.:adj.}
\end{itemize}
Relativo á homotomia.
\section{Homótono}
\begin{itemize}
\item {Grp. gram.:adj.}
\end{itemize}
\begin{itemize}
\item {Proveniência:(De \textunderscore homo...\textunderscore  + lat. \textunderscore tonus\textunderscore )}
\end{itemize}
Que tem o mesmo tom; uniforme.
\section{Homotópico}
\begin{itemize}
\item {Grp. gram.:adj.}
\end{itemize}
\begin{itemize}
\item {Utilização:Bot.}
\end{itemize}
\begin{itemize}
\item {Proveniência:(De \textunderscore homo...\textunderscore  + \textunderscore tópico\textunderscore )}
\end{itemize}
Que se dá ou que vegeta nas mesmas regiões em que outro se dá ou vegeta.
\section{Homótropo}
\begin{itemize}
\item {Grp. gram.:adj.}
\end{itemize}
\begin{itemize}
\item {Utilização:Bot.}
\end{itemize}
\begin{itemize}
\item {Proveniência:(Do gr. \textunderscore homos\textunderscore  + \textunderscore trope\textunderscore )}
\end{itemize}
Diz-se das partes do vegetal, que tomam a mesma direcção.
\section{Homotypia}
\begin{itemize}
\item {Grp. gram.:f.}
\end{itemize}
\begin{itemize}
\item {Proveniência:(De \textunderscore homòtypo\textunderscore )}
\end{itemize}
Carácter dos órgãos homótypos.
Comparação dos órgãos análogos, no mesmo indivíduo.
\section{Homotypicamente}
\begin{itemize}
\item {Grp. gram.:adv.}
\end{itemize}
De modo homotýpico.
\section{Homotýpico}
\begin{itemize}
\item {Grp. gram.:adj.}
\end{itemize}
Relativo á homotypia.
\section{Homótypo}
\begin{itemize}
\item {Grp. gram.:adj.}
\end{itemize}
\begin{itemize}
\item {Proveniência:(De \textunderscore homo...\textunderscore  + \textunderscore typo\textunderscore )}
\end{itemize}
Que é análogo ou tem o mesmo typo, (falando-se dos órgãos do mesmo indivíduo).
\section{Homovalve}
\begin{itemize}
\item {Grp. gram.:adj.}
\end{itemize}
\begin{itemize}
\item {Utilização:Bot.}
\end{itemize}
\begin{itemize}
\item {Proveniência:(De \textunderscore homo...\textunderscore  + \textunderscore valva\textunderscore )}
\end{itemize}
Diz-se do fruto, cujas válvulas são semelhantes.
\section{Homovalvo}
\begin{itemize}
\item {Grp. gram.:adj.}
\end{itemize}
\begin{itemize}
\item {Utilização:Bot.}
\end{itemize}
\begin{itemize}
\item {Proveniência:(De \textunderscore homo...\textunderscore  + \textunderscore valva\textunderscore )}
\end{itemize}
Diz-se do fruto, cujas válvulas são semelhantes.
\section{Homum}
\begin{itemize}
\item {Grp. gram.:m.}
\end{itemize}
\begin{itemize}
\item {Utilização:Prov.}
\end{itemize}
\begin{itemize}
\item {Utilização:alg.}
\end{itemize}
\begin{itemize}
\item {Proveniência:(De \textunderscore homem\textunderscore )}
\end{itemize}
Muitos homens: \textunderscore passou por entre o homum que enchia o adro\textunderscore .
\section{Homúnculo}
\begin{itemize}
\item {Grp. gram.:m.}
\end{itemize}
\begin{itemize}
\item {Proveniência:(Lat. \textunderscore homunculus\textunderscore )}
\end{itemize}
Pequeno homem.
Homemzinho.
Indivíduo insignificante; bisbórria.
\section{Hondurenhismo}
\begin{itemize}
\item {Grp. gram.:m.}
\end{itemize}
\begin{itemize}
\item {Proveniência:(De \textunderscore hondurenho\textunderscore )}
\end{itemize}
Expressão privativa de Honduras.
\section{Hondurenho}
\begin{itemize}
\item {Grp. gram.:adj.}
\end{itemize}
\begin{itemize}
\item {Grp. gram.:M.}
\end{itemize}
Relativo a Honduras.
Aquelle que é natural de Honduras.
\section{Honestador}
\begin{itemize}
\item {Grp. gram.:m.  e  adj.}
\end{itemize}
O que honesta.
\section{Honestamente}
\begin{itemize}
\item {Grp. gram.:adv.}
\end{itemize}
De modo honesto.
\section{Honestar}
\begin{itemize}
\item {Grp. gram.:v. t.}
\end{itemize}
Tornar honesto; honrar.
Adornar; tornar bello.
\section{Honestidade}
\begin{itemize}
\item {Grp. gram.:f.}
\end{itemize}
Qualidade daquelle ou daquillo que é honesto.
Honradez; decoro; probidade.
\section{Honestizar}
\begin{itemize}
\item {Grp. gram.:v. t.}
\end{itemize}
\begin{itemize}
\item {Utilização:Neol.}
\end{itemize}
Tornar honesto; nobilitar.
\section{Honesto}
\begin{itemize}
\item {Grp. gram.:adj.}
\end{itemize}
\begin{itemize}
\item {Proveniência:(Lat. \textunderscore honestus\textunderscore )}
\end{itemize}
Decoroso.
Honrado; virtuoso; casto.
Conveniente.
Attencioso; agradável.
\section{Honor}
\begin{itemize}
\item {Grp. gram.:m.}
\end{itemize}
\begin{itemize}
\item {Utilização:Ant.}
\end{itemize}
\begin{itemize}
\item {Proveniência:(Lat. \textunderscore honor\textunderscore )}
\end{itemize}
O mesmo que \textunderscore honra\textunderscore :«\textunderscore perdi meu honor.\textunderscore »\textunderscore Eufrosina\textunderscore , 111.
\textunderscore Dama de honor\textunderscore , dama que fazia parte da côrte da raínha.
\section{Honorabilidade}
\begin{itemize}
\item {Grp. gram.:f.}
\end{itemize}
\begin{itemize}
\item {Utilização:Neol.}
\end{itemize}
\begin{itemize}
\item {Proveniência:(Do lat. \textunderscore honorabilis\textunderscore )}
\end{itemize}
Qualidade daquelle ou daquillo que é digno de receber honras.
Benemerência. Cf. \textunderscore Projecto do Cod. Civ.\textunderscore  do Brasil, art. 223.
\section{Honorar}
\begin{itemize}
\item {Grp. gram.:v. t.}
\end{itemize}
\begin{itemize}
\item {Utilização:Des.}
\end{itemize}
\begin{itemize}
\item {Proveniência:(De \textunderscore honor\textunderscore )}
\end{itemize}
O mesmo que \textunderscore honrar\textunderscore . Cf. Rui Barb., \textunderscore Réplica\textunderscore , II, 157.
\section{Honorariamente}
\begin{itemize}
\item {Grp. gram.:adv.}
\end{itemize}
De modo honorário; honorificamente.
\section{Honorário}
\begin{itemize}
\item {Grp. gram.:adj.}
\end{itemize}
\begin{itemize}
\item {Proveniência:(Lat. \textunderscore honorarius\textunderscore )}
\end{itemize}
O mesmo que \textunderscore honorífico\textunderscore .
Que dá honras, sem proveito material: \textunderscore títulos honorários\textunderscore .
Que tem honras, sem proventos, de um cargo.
\section{Honorários}
\begin{itemize}
\item {Grp. gram.:m. pl.}
\end{itemize}
\begin{itemize}
\item {Proveniência:(Lat. \textunderscore honorarium\textunderscore )}
\end{itemize}
Retribuição aos que exercem uma profissão liberal, como os advogados, os médicos, etc.
Dinheiro, que o magistrado municipal, entre os Romanos, devia pagar, em reconhecimento da honra que recebia com a sua nomeação.
\section{Honorificamente}
\begin{itemize}
\item {Grp. gram.:adv.}
\end{itemize}
De modo honorífico.
\section{Honorificar}
\begin{itemize}
\item {Grp. gram.:v. t.}
\end{itemize}
\begin{itemize}
\item {Proveniência:(Lat. \textunderscore honorificare\textunderscore )}
\end{itemize}
Dar honras ou mercês a.
Honrar.
\section{Honorificência}
\begin{itemize}
\item {Grp. gram.:f.}
\end{itemize}
\begin{itemize}
\item {Proveniência:(Lat. \textunderscore honorificentia\textunderscore )}
\end{itemize}
Aquillo que constitue honra ou distincção.
\section{Honorífico}
\begin{itemize}
\item {Grp. gram.:adj.}
\end{itemize}
\begin{itemize}
\item {Proveniência:(Lat. \textunderscore honorificus\textunderscore )}
\end{itemize}
Que dá honra; honroso.
Que dá honras sem proveitos materiaes: \textunderscore mercês honoríficas\textunderscore .
\section{Honoveleno}
\begin{itemize}
\item {Grp. gram.:m.}
\end{itemize}
Árvore da Índia portuguesa.
\section{Honra}
\begin{itemize}
\item {Grp. gram.:f.}
\end{itemize}
\begin{itemize}
\item {Utilização:Ant.}
\end{itemize}
\begin{itemize}
\item {Grp. gram.:Pl.}
\end{itemize}
\begin{itemize}
\item {Proveniência:(De \textunderscore honrar\textunderscore )}
\end{itemize}
Consideração ou homenagem á virtude, ao talento, ás bôas qualidades humanas.
Pundonor.
Sentimento, que leva o homem a procurar merecer e manter a consideração pública.
Bôa fama.
Glória.
Favor, distincção: \textunderscore conceder honras\textunderscore .
Castidade; virgindade.
Terra privilegiada, pertencente a fidalgos ou cavalleiros.
Cada um dos cinco trunfos maiores, no jôgo da imperial. Cf. \textunderscore Man. dos Jogos\textunderscore , 244.
Título honorífico.
Honraria.
\textunderscore Honras de Miranda\textunderscore , espécie de capote, usado naquella cidade e nas suas vizinhanças.
\section{Honradamente}
\begin{itemize}
\item {Grp. gram.:adv.}
\end{itemize}
De modo honrado.
\section{Honradez}
\begin{itemize}
\item {Grp. gram.:f.}
\end{itemize}
Qualidade de honrado.
Integridade de carácter.
O mesmo que \textunderscore honra\textunderscore .
\section{Honrado}
\begin{itemize}
\item {Grp. gram.:adj.}
\end{itemize}
\begin{itemize}
\item {Proveniência:(De \textunderscore honrar\textunderscore )}
\end{itemize}
Que tem honra.
Casto.
\section{Honradoiro}
\begin{itemize}
\item {Grp. gram.:adj.}
\end{itemize}
\begin{itemize}
\item {Utilização:Ant.}
\end{itemize}
\begin{itemize}
\item {Proveniência:(De \textunderscore honrar\textunderscore )}
\end{itemize}
Que faz honra, que ennobrece ou glorifica.
\section{Honrador}
\begin{itemize}
\item {Grp. gram.:m.  e  adj.}
\end{itemize}
Aquelle ou aquillo que honra.
\section{Honradouro}
\begin{itemize}
\item {Grp. gram.:adj.}
\end{itemize}
\begin{itemize}
\item {Utilização:Ant.}
\end{itemize}
\begin{itemize}
\item {Proveniência:(De \textunderscore honrar\textunderscore )}
\end{itemize}
Que faz honra, que ennobrece ou glorifica.
\section{Honrar}
\begin{itemize}
\item {Grp. gram.:v. t.}
\end{itemize}
\begin{itemize}
\item {Proveniência:(Do lat. \textunderscore honorare\textunderscore )}
\end{itemize}
Conferir honras a.
Respeitar: \textunderscore honrar pai e mãe\textunderscore .
Ennobrecer.
Distinguir.
Glorificar.
Dar isenções ou privilégios a.
\textunderscore Honrar uma firma\textunderscore , diz-se, no commércio, de alguém que acceita ou paga uma letra que outro não acceitou ou não pagou.
\section{Honraria}
\begin{itemize}
\item {Grp. gram.:f.}
\end{itemize}
\begin{itemize}
\item {Proveniência:(De \textunderscore honrar\textunderscore )}
\end{itemize}
Importância de um cargo.
Concessão de mercês honoríficas.
Distincção.
Graça ou mercê, que nobilita.
\section{Honricas}
\begin{itemize}
\item {Grp. gram.:f. pl.}
\end{itemize}
\begin{itemize}
\item {Utilização:Prov.}
\end{itemize}
Capote, o mesmo que \textunderscore honras de Miranda\textunderscore .
\section{Honrilha}
\begin{itemize}
\item {Grp. gram.:f.}
\end{itemize}
\begin{itemize}
\item {Utilização:Des.}
\end{itemize}
Pequena honra, supposta honra.
Gloríola. Cf. Cortesão, \textunderscore Subs.\textunderscore 
\section{Honrosamente}
\begin{itemize}
\item {Grp. gram.:adv.}
\end{itemize}
De modo honroso; com lustre, com distincção: \textunderscore proceder honrosamente\textunderscore .
\section{Honroso}
\begin{itemize}
\item {Grp. gram.:adj.}
\end{itemize}
\begin{itemize}
\item {Utilização:Heráld.}
\end{itemize}
Que dá honras; que ennobrece.
Que torna respeitado.
Diz-se das faces ou figuras heráldicas de primeira ordem.
\section{Hoombe}
\begin{itemize}
\item {Grp. gram.:m.}
\end{itemize}
Espécie de águia, (\textunderscore helotarsus ecaudatus\textunderscore ).
\section{Hopa}
\begin{itemize}
\item {Grp. gram.:f.}
\end{itemize}
O mesmo ou melhor que \textunderscore opa\textunderscore .
(Cp. cast. \textunderscore hopa\textunderscore )
\section{Hoplita}
\begin{itemize}
\item {Grp. gram.:m.}
\end{itemize}
\begin{itemize}
\item {Proveniência:(Do gr. \textunderscore hoplites\textunderscore )}
\end{itemize}
Soldado, de armadura pesada, na infantaria dos antigos Gregos.
\section{Hoplómacho}
\begin{itemize}
\item {fónica:co}
\end{itemize}
\begin{itemize}
\item {Grp. gram.:m.}
\end{itemize}
\begin{itemize}
\item {Proveniência:(Gr. \textunderscore hoplomakhos\textunderscore )}
\end{itemize}
O gladiador antigo, quando completamente armado.
\section{Hoplómaco}
\begin{itemize}
\item {Grp. gram.:m.}
\end{itemize}
\begin{itemize}
\item {Proveniência:(Gr. \textunderscore hoplomakhos\textunderscore )}
\end{itemize}
O gladiador antigo, quando completamente armado.
\section{Hora}
\begin{itemize}
\item {Grp. gram.:f.}
\end{itemize}
\begin{itemize}
\item {Grp. gram.:Adv.}
\end{itemize}
\begin{itemize}
\item {Grp. gram.:Loc.}
\end{itemize}
\begin{itemize}
\item {Utilização:ant.}
\end{itemize}
\begin{itemize}
\item {Grp. gram.:Pl.}
\end{itemize}
\begin{itemize}
\item {Proveniência:(Lat. \textunderscore hora\textunderscore )}
\end{itemize}
Vigésima quarta parte do dia, ou do tempo que a terra gasta para fazer uma rotação completa sôbre si mesma.
Occasião ou tempo em que se faz ou se deve fazer alguma coisa: \textunderscore a hora do almôço\textunderscore .
Sinal designativo de cada uma das doze partes de um mostrador de relógio.
Opportunidade, ensejo: \textunderscore ainda não chegou a hora do nosso ajuste de contas\textunderscore .
O mesmo que \textunderscore ora\textunderscore ^1. Cf. \textunderscore Eufrosina\textunderscore , (pról.).
\textunderscore Hora de prima\textunderscore , nove horas da manhan.
\textunderscore Fóra de horas\textunderscore , a deshoras.
\textunderscore A tempo e horas\textunderscore , opportunamente.
Livro de orações, que se rezam em certas horas do dia.
\section{Horaciano}
\begin{itemize}
\item {Grp. gram.:adj.}
\end{itemize}
Relativo ao poéta Horácio, ás suas obras ou ao seu estilo.
\section{Horal}
\begin{itemize}
\item {Grp. gram.:adj.}
\end{itemize}
\begin{itemize}
\item {Proveniência:(Lat. \textunderscore horalis\textunderscore )}
\end{itemize}
Relativo a hora ou horas: \textunderscore tarefa horal\textunderscore .
\section{Horar}
\begin{itemize}
\item {Grp. gram.:v. i.}
\end{itemize}
\begin{itemize}
\item {Utilização:Gír.}
\end{itemize}
Fazer horas, matar o tempo.
\section{Horário}
\begin{itemize}
\item {Grp. gram.:adj.}
\end{itemize}
\begin{itemize}
\item {Grp. gram.:M.}
\end{itemize}
\begin{itemize}
\item {Proveniência:(Lat. \textunderscore horarius\textunderscore )}
\end{itemize}
Relativo a hora ou a horas.
Tabella, indicativa das horas, em que se fazem ou se devem fazer certos serviços: \textunderscore horário de caminhos de ferro\textunderscore .
\section{Horda}
\begin{itemize}
\item {Grp. gram.:f.}
\end{itemize}
\begin{itemize}
\item {Proveniência:(Do fr. \textunderscore horde\textunderscore )}
\end{itemize}
Tríbo nômade.
Guerrilha; bando indisciplinado.
\section{Hordeáceas}
\begin{itemize}
\item {Grp. gram.:f. pl.}
\end{itemize}
\begin{itemize}
\item {Proveniência:(De \textunderscore hordáceo\textunderscore )}
\end{itemize}
Tríbo de plantas gramíneas, que tem por typo a cevada.
\section{Hordeáceo}
\begin{itemize}
\item {Grp. gram.:adj.}
\end{itemize}
\begin{itemize}
\item {Proveniência:(Lat. \textunderscore hordeaceus\textunderscore )}
\end{itemize}
Semelhante a grãos ou espigas de cevada.
\section{Hordéolo}
\begin{itemize}
\item {Grp. gram.:m.}
\end{itemize}
\begin{itemize}
\item {Proveniência:(Lat. \textunderscore hordeolus\textunderscore )}
\end{itemize}
O mesmo que \textunderscore terçol\textunderscore .
\section{Horispício}
\begin{itemize}
\item {Grp. gram.:m.}
\end{itemize}
\begin{itemize}
\item {Proveniência:(Do lat. \textunderscore hora\textunderscore  + \textunderscore spicere\textunderscore )}
\end{itemize}
O mesmo que \textunderscore horóscopo\textunderscore ^1. Cf. Castilho, \textunderscore Fastos\textunderscore , III, 226.
\section{Horizontal}
\begin{itemize}
\item {Grp. gram.:adj.}
\end{itemize}
\begin{itemize}
\item {Grp. gram.:F.}
\end{itemize}
Que é parallelo ao horizonte, (em sentido astronómico).
Relativo ao horizonte.
Deitado ao comprido.
Nivelado.
Linha parallela ao horizonte.
Meretriz fina.
\section{Horizontalidade}
\begin{itemize}
\item {Grp. gram.:f.}
\end{itemize}
Qualidade daquillo que é horizontal.
\section{Horizontalmente}
\begin{itemize}
\item {Grp. gram.:adv.}
\end{itemize}
De modo horizontal.
\section{Horizontar}
\begin{itemize}
\item {Grp. gram.:v. t.}
\end{itemize}
Occupar o horizonte de; confinar com:«\textunderscore ...nações que horizontavam os Lusitanos.\textunderscore »\textunderscore Viriato Trág.\textunderscore , III, 100.
\section{Horizonte}
\begin{itemize}
\item {Grp. gram.:m.}
\end{itemize}
\begin{itemize}
\item {Utilização:Fig.}
\end{itemize}
\begin{itemize}
\item {Proveniência:(Do lat. \textunderscore horizon\textunderscore , \textunderscore horizontis\textunderscore )}
\end{itemize}
Linha circular, de que é centro o observador, e em que o céu e a Terra parecem juntar-se.
Parte da superfície da Terra, que a nossa vista abrange.
Em Astronomia, plano tangente á Terra, no ponto em que está o observador, ou plano parallelo a êste e passando pelo centro da terra.
Extensão.
Espaço: \textunderscore ergueu-se um balão no horizonte\textunderscore .
Perspectiva: \textunderscore que bello horizonte\textunderscore !
Futuro.
Linha, que termína o céu de um quadro.
\textunderscore Horizonte artificial\textunderscore , apparelho simplicíssimo, de que se servem os observadores de alturas de astros.
\section{Hormino}
\begin{itemize}
\item {Grp. gram.:m.}
\end{itemize}
Gênero de plantas labiadas.
\section{Hornaveque}
\begin{itemize}
\item {Grp. gram.:m.}
\end{itemize}
\begin{itemize}
\item {Proveniência:(Do al. \textunderscore hornwerk\textunderscore )}
\end{itemize}
Obra cornuta, em architectura.
\section{Horneblenda}
\begin{itemize}
\item {Grp. gram.:f.}
\end{itemize}
\begin{itemize}
\item {Proveniência:(Do al. \textunderscore hornbelende\textunderscore )}
\end{itemize}
Silicato, de fractura esquirolosa.
\section{Horografia}
\begin{itemize}
\item {Grp. gram.:f.}
\end{itemize}
\begin{itemize}
\item {Proveniência:(Do gr. \textunderscore hora\textunderscore  + \textunderscore graphein\textunderscore )}
\end{itemize}
Arte de fazer quadrantes.
\section{Horographia}
\begin{itemize}
\item {Grp. gram.:f.}
\end{itemize}
\begin{itemize}
\item {Proveniência:(Do gr. \textunderscore hora\textunderscore  + \textunderscore graphein\textunderscore )}
\end{itemize}
Arte de fazer quadrantes.
\section{Horologial}
\begin{itemize}
\item {Grp. gram.:adj.}
\end{itemize}
\begin{itemize}
\item {Proveniência:(Do lat. \textunderscore horologium\textunderscore )}
\end{itemize}
Relativo a relógios.
\section{Horóptero}
\begin{itemize}
\item {Grp. gram.:m.}
\end{itemize}
\begin{itemize}
\item {Proveniência:(Do gr. \textunderscore horos\textunderscore  + \textunderscore opter\textunderscore )}
\end{itemize}
Linha recta, paralella á linha que une os centros dos olhos, e que passa pelo ponto, em que coincidem os eixos ópticos.
\section{Horoscopar}
\begin{itemize}
\item {Grp. gram.:v. i.}
\end{itemize}
\begin{itemize}
\item {Utilização:Des.}
\end{itemize}
\begin{itemize}
\item {Proveniência:(Lat. \textunderscore horoscopare\textunderscore )}
\end{itemize}
Tirar o horóscopo.
\section{Horoscopia}
\begin{itemize}
\item {Grp. gram.:f.}
\end{itemize}
Acto de horoscopizar.
(Cp. \textunderscore horoscópio\textunderscore )
\section{Horoscópio}
\begin{itemize}
\item {Grp. gram.:m.}
\end{itemize}
O mesmo que \textunderscore horóscopo\textunderscore ^1.
\section{Horoscopista}
\begin{itemize}
\item {Grp. gram.:m.}
\end{itemize}
Aquelle que tira horóscopo, que horoscopiza. Cf. Filinto, XIII, 75.
\section{Horoscopizar}
\begin{itemize}
\item {Grp. gram.:v. i.}
\end{itemize}
\begin{itemize}
\item {Proveniência:(De \textunderscore horoscópio\textunderscore )}
\end{itemize}
O mesmo que \textunderscore horoscopar\textunderscore .
\section{Horóscopo}
\begin{itemize}
\item {Grp. gram.:m.}
\end{itemize}
\begin{itemize}
\item {Proveniência:(Lat. \textunderscore horoscopus\textunderscore )}
\end{itemize}
Aquillo que se prediz, por simples conjecturas, á cêrca de uma pessôa ou coisa.
Prognóstico, que os astrólogos pretendiam tirar da situação de certos astros, quando alguém nascia.
Ponto da eclíptica, que está acima do horizonte, quando nasce uma criança.
\section{Horóscopo}
\begin{itemize}
\item {Grp. gram.:m.}
\end{itemize}
\begin{itemize}
\item {Proveniência:(Gr. \textunderscore horoskopos\textunderscore . Cp. \textunderscore horóscopo\textunderscore ^1)}
\end{itemize}
Sacerdote egýpcio que, nas ceremónias do culto, levava na mão um relógio e uma palma, sýmbolos da Astronomia.
\section{Horra}
\begin{itemize}
\item {Grp. gram.:m.}
\end{itemize}
Madeira de uma planta aquática, na Ásia.
\section{Horrendamente}
\begin{itemize}
\item {Grp. gram.:adv.}
\end{itemize}
De modo horrendo; horrorosamente.
\section{Horrendo}
\begin{itemize}
\item {Grp. gram.:adj.}
\end{itemize}
\begin{itemize}
\item {Proveniência:(Lat. \textunderscore horrendus\textunderscore )}
\end{itemize}
Que horroriza; que faz mêdo: \textunderscore noite horrenda\textunderscore .
Muito feio: \textunderscore cara horrenda\textunderscore .
\section{Horrente}
\begin{itemize}
\item {Grp. gram.:adj.}
\end{itemize}
\begin{itemize}
\item {Utilização:Poét.}
\end{itemize}
\begin{itemize}
\item {Proveniência:(Lat. \textunderscore horrens\textunderscore )}
\end{itemize}
Que causa mêdo ou horror; que tem mêdo.
\section{Horribilidade}
\begin{itemize}
\item {Grp. gram.:f.}
\end{itemize}
\begin{itemize}
\item {Proveniência:(Do lat. \textunderscore horribilis\textunderscore )}
\end{itemize}
Qualidade do que é horrível.
\section{Hórrido}
\begin{itemize}
\item {Grp. gram.:adj.}
\end{itemize}
\begin{itemize}
\item {Proveniência:(Lat. \textunderscore horridus\textunderscore )}
\end{itemize}
O mesmo que \textunderscore horrendo\textunderscore .
\section{Horrífero}
\begin{itemize}
\item {Grp. gram.:adj.}
\end{itemize}
\begin{itemize}
\item {Proveniência:(Lat. \textunderscore horrifer\textunderscore )}
\end{itemize}
O mesmo que \textunderscore horrífico\textunderscore .
\section{Horrificamente}
\begin{itemize}
\item {Grp. gram.:adv.}
\end{itemize}
De modo horrífico; horrendamente.
\section{Horrífico}
\begin{itemize}
\item {Grp. gram.:adj.}
\end{itemize}
\begin{itemize}
\item {Proveniência:(Lat. \textunderscore horrificus\textunderscore )}
\end{itemize}
O mesmo que \textunderscore horrendo\textunderscore .
\section{Horripilação}
\begin{itemize}
\item {Grp. gram.:f.}
\end{itemize}
\begin{itemize}
\item {Proveniência:(Lat. \textunderscore horripilatio\textunderscore )}
\end{itemize}
Acto ou effeito de arrepiar-se.
Calefrio, que antecede a febre.
\section{Horripilante}
\begin{itemize}
\item {Grp. gram.:adj.}
\end{itemize}
\begin{itemize}
\item {Proveniência:(Lat. \textunderscore horripilans\textunderscore )}
\end{itemize}
Que horripila; que causa horror, que assusta.
\section{Horripilar}
\begin{itemize}
\item {Grp. gram.:v. t.}
\end{itemize}
\begin{itemize}
\item {Proveniência:(Lat. \textunderscore horripilare\textunderscore )}
\end{itemize}
Causar arrepios a.
Horrorizar.
\section{Horrípilo}
\begin{itemize}
\item {Grp. gram.:adj.}
\end{itemize}
\begin{itemize}
\item {Utilização:bras}
\end{itemize}
\begin{itemize}
\item {Utilização:Neol.}
\end{itemize}
O mesmo que \textunderscore horripilante\textunderscore . Cf. Rev. \textunderscore Renascença\textunderscore , do Rio, n.^o 10.
\section{Horrisonante}
\begin{itemize}
\item {fónica:so}
\end{itemize}
\begin{itemize}
\item {Grp. gram.:adj.}
\end{itemize}
O mesmo que \textunderscore horrísono\textunderscore . Cf. Castilho, \textunderscore Fastos\textunderscore , II, 27.
\section{Horrísono}
\begin{itemize}
\item {fónica:so}
\end{itemize}
\begin{itemize}
\item {Grp. gram.:adj.}
\end{itemize}
\begin{itemize}
\item {Proveniência:(Lat. \textunderscore horrisonus\textunderscore )}
\end{itemize}
Que causa um som aterrador.
\section{Horrissonante}
\begin{itemize}
\item {Grp. gram.:adj.}
\end{itemize}
O mesmo que \textunderscore horríssono\textunderscore . Cf. Castilho, \textunderscore Fastos\textunderscore , II, 27.
\section{Horríssono}
\begin{itemize}
\item {Grp. gram.:adj.}
\end{itemize}
\begin{itemize}
\item {Proveniência:(Lat. \textunderscore horrisonus\textunderscore )}
\end{itemize}
Que causa um som aterrador.
\section{Horritroante}
\begin{itemize}
\item {Grp. gram.:adj.}
\end{itemize}
Que produz estrondo horroroso. Cf. Rui Barb., \textunderscore Réplica\textunderscore , II, 157.
\section{Horrível}
\begin{itemize}
\item {Grp. gram.:adj.}
\end{itemize}
\begin{itemize}
\item {Proveniência:(Do lat. \textunderscore horribilis\textunderscore )}
\end{itemize}
Que causa horror.
Péssimo.
Feíssimo.
\section{Horrivelmente}
\begin{itemize}
\item {Grp. gram.:adv.}
\end{itemize}
De modo horrível.
\section{Horror}
\begin{itemize}
\item {Grp. gram.:m.}
\end{itemize}
\begin{itemize}
\item {Proveniência:(Lat. \textunderscore horror\textunderscore )}
\end{itemize}
Sensação phýsica, que faz arrepiar os cabellos e a pelle.
Estremecimento ou agitação, causada por coisa horrorosa.
Repulsão, repugnância, causada por coisa contrária á natureza, á moral ou aos sentimentos humanitários.
Aversão.
Aquillo que causa horror.
Susto, pavor.
\section{Horrorífico}
\begin{itemize}
\item {Grp. gram.:adj.}
\end{itemize}
O mesmo que \textunderscore horrífico\textunderscore . Cf. Filinto, \textunderscore D. Man.\textunderscore 
\section{Horrorizar}
\begin{itemize}
\item {Grp. gram.:v. t.}
\end{itemize}
Causar horror a.
Horripilar.
\section{Horrorosamente}
\begin{itemize}
\item {Grp. gram.:adv.}
\end{itemize}
De modo horroroso; horrivelmente.
\section{Horroroso}
\begin{itemize}
\item {Grp. gram.:adj.}
\end{itemize}
Que causa horror; horrível.
\section{Horsa}
\begin{itemize}
\item {Grp. gram.:f.}
\end{itemize}
\begin{itemize}
\item {Utilização:Pop.}
\end{itemize}
\begin{itemize}
\item {Proveniência:(Ingl. \textunderscore horse\textunderscore )}
\end{itemize}
Cavallo inglês ou égua inglesa, de tamanho mais que ordinário.
Cavallo grande, de pouca fôrça.
\section{Horta}
\begin{itemize}
\item {Grp. gram.:f.}
\end{itemize}
\begin{itemize}
\item {Grp. gram.:Pl.}
\end{itemize}
\begin{itemize}
\item {Utilização:T. de Lisbôa}
\end{itemize}
Terreno, plantado de legumes ou hortaliças.
Taberna no campo, fóra de Lisbôa.
(Cp. lat. \textunderscore hortus\textunderscore )
\section{Hortaliça}
\begin{itemize}
\item {Grp. gram.:f.}
\end{itemize}
Nome genérico das plantas leguminosas e comestíveis, que geralmente se cultivam nas hortas.
(B. lat. \textunderscore hortalitia\textunderscore )
\section{Hortaliceira}
\begin{itemize}
\item {Grp. gram.:f.}
\end{itemize}
Mulher, que vende hortaliça.
\section{Hortaliceiro}
\begin{itemize}
\item {Grp. gram.:m.}
\end{itemize}
Vendedor de hortaliça.
\section{Hortar}
\begin{itemize}
\item {Grp. gram.:v. t.}
\end{itemize}
Converter em horta; preparar (um terreno), para produzir hortaliça.
\section{Hortativo}
\begin{itemize}
\item {Grp. gram.:adj.}
\end{itemize}
\begin{itemize}
\item {Proveniência:(Lat. \textunderscore hortativus\textunderscore )}
\end{itemize}
Que exhorta.
\section{Hortejar}
\begin{itemize}
\item {Grp. gram.:v. t.}
\end{itemize}
\begin{itemize}
\item {Utilização:Prov.}
\end{itemize}
\begin{itemize}
\item {Utilização:alg.}
\end{itemize}
O mesmo que \textunderscore hortar\textunderscore .
\section{Hortejo}
\begin{itemize}
\item {Grp. gram.:m.}
\end{itemize}
\begin{itemize}
\item {Utilização:Prov.}
\end{itemize}
\begin{itemize}
\item {Proveniência:(De \textunderscore hôrto\textunderscore )}
\end{itemize}
Horta pequena.
\section{Hortelã}
\begin{itemize}
\item {Grp. gram.:f.}
\end{itemize}
\begin{itemize}
\item {Proveniência:(Do lat. \textunderscore hortulana\textunderscore )}
\end{itemize}
Gênero de plantas, de que é principal espécie a \textunderscore hortelã das cozinhas\textunderscore  ou \textunderscore hortelan verde\textunderscore , (\textunderscore menta viridis\textunderscore , Lin.).
\section{Hortelan}
\begin{itemize}
\item {Grp. gram.:f.}
\end{itemize}
\begin{itemize}
\item {Proveniência:(Do lat. \textunderscore hortulana\textunderscore )}
\end{itemize}
Gênero de plantas, de que é principal espécie a \textunderscore hortelan das cozinhas\textunderscore  ou \textunderscore hortelan verde\textunderscore , (\textunderscore menta viridis\textunderscore , Lin.).
\section{Hortelan-brava}
\begin{itemize}
\item {Grp. gram.:f.}
\end{itemize}
Planta labiada, (\textunderscore pyenanthenum protiferum\textunderscore ).
\section{Hortelan-de-boi}
\begin{itemize}
\item {Grp. gram.:f.}
\end{itemize}
\begin{itemize}
\item {Utilização:Bras}
\end{itemize}
Planta labiada, (\textunderscore pyenanthenum protiferum\textunderscore ).
\section{Hortelan-de-cheiro}
\begin{itemize}
\item {Grp. gram.:f.}
\end{itemize}
Planta labiada, (\textunderscore menta crispa\textunderscore ).
\section{Hortelan-do-brasil}
\begin{itemize}
\item {Grp. gram.:f.}
\end{itemize}
O mesmo que \textunderscore paracuri\textunderscore .
\section{Hortelan-doce}
\begin{itemize}
\item {Grp. gram.:f.}
\end{itemize}
Planta hortense, (\textunderscore mentha arvensis\textunderscore , Lin.), muito semelhante á hortelan verde e, como esta, applicada em temperos.
\section{Hortelan-do-maranhão}
\begin{itemize}
\item {Grp. gram.:f.}
\end{itemize}
O mesmo que \textunderscore segurelha\textunderscore ^2.
\section{Hortelan-do-mato}
\begin{itemize}
\item {Grp. gram.:f.}
\end{itemize}
Planta labiada, (\textunderscore peltodon radicans\textunderscore ).
\section{Hortelan-francesa}
\begin{itemize}
\item {Grp. gram.:f.}
\end{itemize}
Planta, da fam. das compostas, (\textunderscore pyrethrum balsamita\textunderscore ).
\section{Hortelan-pimenta}
\begin{itemize}
\item {Grp. gram.:f.}
\end{itemize}
Planta labiada, (\textunderscore mentha piperita\textunderscore , Lin.).
\section{Hortelan-silvestre}
\begin{itemize}
\item {Grp. gram.:f.}
\end{itemize}
Planta medicinal, (\textunderscore mentha silveris\textunderscore , Lin.).
\section{Hortelan-verde}
\begin{itemize}
\item {Grp. gram.:f.}
\end{itemize}
(V.hortelan)
\section{Hortelão}
\begin{itemize}
\item {Grp. gram.:m.}
\end{itemize}
\begin{itemize}
\item {Proveniência:(Do lat. \textunderscore hortulanus\textunderscore )}
\end{itemize}
Aquelle que trata de uma horta ou de hortas.
\section{Hortelôa}
\begin{itemize}
\item {Grp. gram.:f.}
\end{itemize}
\begin{itemize}
\item {Proveniência:(De \textunderscore hortelão\textunderscore )}
\end{itemize}
Mulher, que trata de horta ou de hortas.
Mulher de hortelão.
\section{Hortense}
\begin{itemize}
\item {Grp. gram.:adj.}
\end{itemize}
\begin{itemize}
\item {Grp. gram.:F.}
\end{itemize}
\begin{itemize}
\item {Proveniência:(Lat. \textunderscore hortensis\textunderscore )}
\end{itemize}
Relativo a horta; que é produzido em horta.
Planta rosácea, (\textunderscore poterium sanguisorba\textunderscore ).
\section{Hortênsia}
\begin{itemize}
\item {Grp. gram.:f.}
\end{itemize}
\begin{itemize}
\item {Proveniência:(Lat. \textunderscore hortensia\textunderscore )}
\end{itemize}
Planta saxifragácea, também conhecida por \textunderscore hydranja\textunderscore  ou \textunderscore novelos\textunderscore .
\section{Hortícola}
\begin{itemize}
\item {Grp. gram.:adj.}
\end{itemize}
\begin{itemize}
\item {Proveniência:(Do lat. \textunderscore hortus\textunderscore  + \textunderscore colere\textunderscore )}
\end{itemize}
Relativo a hortas: \textunderscore trabalhos hortícolas\textunderscore .
\section{Horticultor}
\begin{itemize}
\item {Grp. gram.:m.}
\end{itemize}
\begin{itemize}
\item {Proveniência:(Do lat. \textunderscore hortus\textunderscore  + \textunderscore cultor\textunderscore )}
\end{itemize}
Aquelle que trata de hortas.
Jardineiro.
Aquelle que é versado em coisas de horticultura.
\section{Horticultura}
\begin{itemize}
\item {Grp. gram.:f.}
\end{itemize}
\begin{itemize}
\item {Proveniência:(Do lat. \textunderscore hortus\textunderscore  + \textunderscore cultura\textunderscore )}
\end{itemize}
Arte de cultivar hortas e jardins.
\section{Hôrto}
\begin{itemize}
\item {Grp. gram.:m.}
\end{itemize}
\begin{itemize}
\item {Grp. gram.:Pl.}
\end{itemize}
\begin{itemize}
\item {Utilização:Prov.}
\end{itemize}
\begin{itemize}
\item {Utilização:minh.}
\end{itemize}
\begin{itemize}
\item {Proveniência:(Lat. \textunderscore hortus\textunderscore )}
\end{itemize}
Pequena horta.
Pequena porção de terreno, em que se cultivam plantas de jardim; jardim.
Lugar de tormento, (por allusão ao hôrto das oliveiras, em que Jesus soffreu).
Couves, hortaliça.
\section{Hortulana}
\begin{itemize}
\item {Grp. gram.:f.}
\end{itemize}
\begin{itemize}
\item {Proveniência:(Do lat. \textunderscore hortulanus\textunderscore )}
\end{itemize}
Pássaro conirostro, de arribação.
\section{Hosana}
\begin{itemize}
\item {Grp. gram.:m.}
\end{itemize}
\begin{itemize}
\item {Utilização:Fig.}
\end{itemize}
\begin{itemize}
\item {Grp. gram.:Interj.}
\end{itemize}
\begin{itemize}
\item {Proveniência:(Do lat. \textunderscore hosanna\textunderscore )}
\end{itemize}
Hino eclesiástico, que se canta no domingo de Ramos.
Louvor; saudação.
Salvè! avè!
\section{Hosanna}
\begin{itemize}
\item {Grp. gram.:m.}
\end{itemize}
\begin{itemize}
\item {Utilização:Fig.}
\end{itemize}
\begin{itemize}
\item {Grp. gram.:Interj.}
\end{itemize}
\begin{itemize}
\item {Proveniência:(Do lat. \textunderscore hosanna\textunderscore )}
\end{itemize}
Hymno ecclesiástico, que se canta no domingo de Ramos.
Louvor; saudação.
Salvè! avè!
\section{Hosco}
\begin{itemize}
\item {fónica:ôs}
\end{itemize}
\begin{itemize}
\item {Grp. gram.:adj.}
\end{itemize}
\begin{itemize}
\item {Utilização:Bras}
\end{itemize}
\begin{itemize}
\item {Proveniência:(T. cast. Cp. \textunderscore fusco\textunderscore )}
\end{itemize}
Que tem côr escura, (falando-se de gado vacum).
\section{Hóspeda}
\begin{itemize}
\item {Grp. gram.:f.}
\end{itemize}
\begin{itemize}
\item {Proveniência:(Do lat. \textunderscore hospita\textunderscore )}
\end{itemize}
Mulher, que vive temporariamente em casa de outrem.
Mulher, que dá poisada.
Hospedeira; estalajadeira.
\section{Hospedádego}
\begin{itemize}
\item {Grp. gram.:m.}
\end{itemize}
\begin{itemize}
\item {Proveniência:(Do b. lat. \textunderscore hospitaticum\textunderscore )}
\end{itemize}
O mesmo que \textunderscore hospedagem\textunderscore .
\section{Hospedádigo}
\begin{itemize}
\item {Grp. gram.:m.}
\end{itemize}
\begin{itemize}
\item {Utilização:Ant.}
\end{itemize}
\begin{itemize}
\item {Proveniência:(Do b. lat. \textunderscore hospitaticum\textunderscore )}
\end{itemize}
O mesmo que \textunderscore hospedagem\textunderscore .
\section{Hospedador}
\begin{itemize}
\item {Grp. gram.:m.  e  adj.}
\end{itemize}
\begin{itemize}
\item {Proveniência:(Lat. \textunderscore hospitator\textunderscore )}
\end{itemize}
O que hospeda.
\section{Hospedagem}
\begin{itemize}
\item {Grp. gram.:f.}
\end{itemize}
Acto ou effeito de \textunderscore hospedar\textunderscore .
Hospedaria.
\section{Hospedal}
\begin{itemize}
\item {Grp. gram.:adj.}
\end{itemize}
\begin{itemize}
\item {Proveniência:(Do lat. \textunderscore hospitalis\textunderscore )}
\end{itemize}
Relativo a hospedagem.
Hospedeiro.
\section{Hospedamento}
\begin{itemize}
\item {Grp. gram.:m.}
\end{itemize}
(V.hospedagem)
\section{Hospedar}
\begin{itemize}
\item {Grp. gram.:v. t.}
\end{itemize}
\begin{itemize}
\item {Grp. gram.:V. p.}
\end{itemize}
\begin{itemize}
\item {Proveniência:(Do lat. \textunderscore hospitare\textunderscore )}
\end{itemize}
Receber por hóspede; dar hospedagem a.
Tornar-se hóspede; alojar-se; tomar aposento.
\section{Hospedaria}
\begin{itemize}
\item {Grp. gram.:f.}
\end{itemize}
\begin{itemize}
\item {Proveniência:(De \textunderscore hospedar\textunderscore )}
\end{itemize}
Casa, em que se admittem hóspedes, mediante retribuição.
Albergaria; estalagem.
\section{Hospedável}
\begin{itemize}
\item {Grp. gram.:adj.}
\end{itemize}
Que póde hospedar ou sêr hospedado.
\section{Hospedavelmente}
\begin{itemize}
\item {Grp. gram.:adv.}
\end{itemize}
\begin{itemize}
\item {Proveniência:(De \textunderscore hospedável\textunderscore )}
\end{itemize}
Com hospitalidade.
\section{Hóspede}
\begin{itemize}
\item {Grp. gram.:m.}
\end{itemize}
\begin{itemize}
\item {Grp. gram.:Adj.}
\end{itemize}
\begin{itemize}
\item {Utilização:Fig.}
\end{itemize}
\begin{itemize}
\item {Proveniência:(Lat. \textunderscore hospes\textunderscore )}
\end{itemize}
Indivíduo, que vive temporariamente em casa alheia.
Aquelle que recebe alguém em sua casa, dando-lhe cama e mesa, com retribuição ou sem ella.
Indivíduo estranho, peregrino.
Estranho; alheio.
Ignorante de alguma coisa: \textunderscore estás hóspede na grammática\textunderscore .
\section{Hospedeira}
\begin{itemize}
\item {Grp. gram.:f.}
\end{itemize}
\begin{itemize}
\item {Proveniência:(De \textunderscore hospedeiro\textunderscore )}
\end{itemize}
Mulher, que hospeda, que tem hospedaria, que dá poisada.
\section{Hospedeiro}
\begin{itemize}
\item {Grp. gram.:adj.}
\end{itemize}
\begin{itemize}
\item {Utilização:Ext.}
\end{itemize}
\begin{itemize}
\item {Grp. gram.:M.}
\end{itemize}
Relativo a hóspede.
Que hospeda.
Obsequiador; benévolo; lhano.
Aquelle que tem hospedaria ou dá hospedagem.
\section{Hospedoso}
\begin{itemize}
\item {Grp. gram.:adj.}
\end{itemize}
\begin{itemize}
\item {Proveniência:(De \textunderscore hospedar\textunderscore )}
\end{itemize}
Hospitaleiro, que abriga:«\textunderscore hospedosos férteis coqueiros...\textunderscore »Filinto, VIII, 219.
\section{Hospício}
\begin{itemize}
\item {Grp. gram.:m.}
\end{itemize}
\begin{itemize}
\item {Proveniência:(Lat. \textunderscore hospitium\textunderscore )}
\end{itemize}
Casa, em que se hospedam e tratam pessôas pobres, sem retribuição.
Lugar ou casa, em que se recolhem e tratam animaes abandonados.
Estabelecimento público, em que se recolhem loucos ou doentes, mediante retribuição ou sem ella.
\section{Hospitação}
\begin{itemize}
\item {Grp. gram.:f.}
\end{itemize}
\begin{itemize}
\item {Utilização:Ant.}
\end{itemize}
\begin{itemize}
\item {Proveniência:(Do lat. \textunderscore hospitari\textunderscore )}
\end{itemize}
Obrigação de dar hospedagem a fidalgos, ministros e outras elevadas personagens.
\section{Hospital}
\begin{itemize}
\item {Grp. gram.:m.}
\end{itemize}
\begin{itemize}
\item {Grp. gram.:Adj.}
\end{itemize}
\begin{itemize}
\item {Utilização:Des.}
\end{itemize}
\begin{itemize}
\item {Proveniência:(Lat. \textunderscore hospitalis\textunderscore )}
\end{itemize}
Edifício, para nelle se recolherem e tratarem doentes.
Caridoso; benévolo.
\section{Hospitalar}
\begin{itemize}
\item {Grp. gram.:adj.}
\end{itemize}
Relativo a hospital ou a hospício.
\section{Hospitalariamente}
\begin{itemize}
\item {Grp. gram.:adv.}
\end{itemize}
\begin{itemize}
\item {Proveniência:(De \textunderscore hospitalário\textunderscore )}
\end{itemize}
Hospedavelmente.
\section{Hospitalário}
\begin{itemize}
\item {Grp. gram.:adj.}
\end{itemize}
\begin{itemize}
\item {Grp. gram.:M.}
\end{itemize}
\begin{itemize}
\item {Proveniência:(De \textunderscore hospital\textunderscore )}
\end{itemize}
O mesmo que \textunderscore hospitalar\textunderscore .
Cavalleiro da Ordem do Hospital ou de Malta.
\section{Hospitaleira}
\begin{itemize}
\item {Grp. gram.:f.}
\end{itemize}
\begin{itemize}
\item {Proveniência:(De \textunderscore hospitaleiro\textunderscore )}
\end{itemize}
Mulher religiosa ou caritativa, que trata de enfermos sem retribuição e por obediência aos regulamentos da sua communidade.
\section{Hospitaleiro}
\begin{itemize}
\item {Grp. gram.:m.}
\end{itemize}
\begin{itemize}
\item {Utilização:Des.}
\end{itemize}
\begin{itemize}
\item {Grp. gram.:Adj.}
\end{itemize}
\begin{itemize}
\item {Proveniência:(Do b. lat. \textunderscore hospitalarius\textunderscore )}
\end{itemize}
Aquelle que dá hospedagem por caridade ou bondade.
Frade lóio. Cf. Carvalho, \textunderscore Chorographia Port.\textunderscore , II, 10.
Que franqueia hospedagem.
\section{Hospitalidade}
\begin{itemize}
\item {Grp. gram.:f.}
\end{itemize}
\begin{itemize}
\item {Proveniência:(Lat. \textunderscore hospitalitas\textunderscore )}
\end{itemize}
Acto de hospedar.
Qualidade de hospitaleiro.
Acolhimento affectuoso.
\section{Hospitalização}
\begin{itemize}
\item {Grp. gram.:f.}
\end{itemize}
Acto ou effeito de hospitalizar.
\section{Hospitalizar}
\begin{itemize}
\item {Grp. gram.:v. t.}
\end{itemize}
Converter em hospital.
Meter em hospital.
\section{Hospodar}
\begin{itemize}
\item {Grp. gram.:m.}
\end{itemize}
Antigo titulo dos soberanos da Moldávia e Valáchia.
(Do bohêmio \textunderscore hospodar\textunderscore , dono da casa, soberano)
\section{Hospodarato}
\begin{itemize}
\item {Grp. gram.:m.}
\end{itemize}
Dignidade de hospodar.
\section{Hostal}
\begin{itemize}
\item {Grp. gram.:m.}
\end{itemize}
\begin{itemize}
\item {Utilização:Ant.}
\end{itemize}
O mesmo que \textunderscore hostau\textunderscore .
(Cp. cast. \textunderscore hostal\textunderscore , hospedaria)
\section{Hostalagem}
\begin{itemize}
\item {Grp. gram.:m.}
\end{itemize}
\begin{itemize}
\item {Utilização:Ant.}
\end{itemize}
\begin{itemize}
\item {Proveniência:(De \textunderscore hostal\textunderscore )}
\end{itemize}
O mesmo que \textunderscore estalagem\textunderscore .
\section{Hostau}
\begin{itemize}
\item {Grp. gram.:m.}
\end{itemize}
\begin{itemize}
\item {Utilização:Ant.}
\end{itemize}
O mesmo que \textunderscore estau\textunderscore .
\section{Hoste}
\begin{itemize}
\item {Grp. gram.:f.}
\end{itemize}
\begin{itemize}
\item {Utilização:Fig.}
\end{itemize}
\begin{itemize}
\item {Proveniência:(Do lat. \textunderscore hostis\textunderscore )}
\end{itemize}
Tropa; exército.
Bando.
Chusma; multidão.
\section{Hosteia}
\begin{itemize}
\item {Grp. gram.:f.}
\end{itemize}
Gênero de plantas saxifragáceas.
\section{Hostes}
\begin{itemize}
\item {Grp. gram.:m. pl.}
\end{itemize}
\begin{itemize}
\item {Utilização:Ant.}
\end{itemize}
\begin{itemize}
\item {Proveniência:(Lat. \textunderscore hostes\textunderscore )}
\end{itemize}
Inimigos.
\section{Hóstia}
\begin{itemize}
\item {Grp. gram.:f.}
\end{itemize}
\begin{itemize}
\item {Proveniência:(Lat. \textunderscore hostia\textunderscore )}
\end{itemize}
Qualquer víctima, que os Hebreus sacrificavam e offereciam a Deus.
Lâmina circular de massa de trigo, sem fermento, consagrada e offerecida a Deus pelo sacerdote na Missa.
Partícula análoga, que se emprega na administração do sacramento da Eucharistia.
Pasta delgada de massa de trigo, de que se extrahem as alludidas partículas, e a que se dão outras applicações, como em envoltórios de certos medicamentos, para êstes se deglutirem facilmente, etc.
A presença de Christo na Eucharistia.
\section{Hostiário}
\begin{itemize}
\item {Grp. gram.:m.}
\end{itemize}
Caixa para hóstias, ainda não consagradas.
(B. lat. \textunderscore hostiaria\textunderscore )
\section{Hostil}
\begin{itemize}
\item {Grp. gram.:adj.}
\end{itemize}
\begin{itemize}
\item {Proveniência:(Lat. \textunderscore hostilis\textunderscore )}
\end{itemize}
Adverso.
Adversário.
Inimigo.
Provocante.
\section{Hostilidade}
\begin{itemize}
\item {Grp. gram.:f.}
\end{itemize}
\begin{itemize}
\item {Proveniência:(Lat. \textunderscore hostilitas\textunderscore )}
\end{itemize}
Acto ou effeito de hostilizar.
Qualidade daquelle ou daquillo que é hostil.
\section{Hostilizar}
\begin{itemize}
\item {Grp. gram.:v. t.}
\end{itemize}
\begin{itemize}
\item {Proveniência:(De \textunderscore hostil\textunderscore )}
\end{itemize}
Oppor-se a; guerrear.
Causar prejuízo a.
\section{Hostilmente}
\begin{itemize}
\item {Grp. gram.:adv.}
\end{itemize}
De modo hostil; com hostilidade; com animadversão.
\section{Hoteia}
\begin{itemize}
\item {Grp. gram.:f.}
\end{itemize}
\begin{itemize}
\item {Proveniência:(De \textunderscore Hotei\textunderscore , n. p. de um bot. jap.)}
\end{itemize}
Gênero de plantas saxífragáceas.
\section{Hotel}
\begin{itemize}
\item {Grp. gram.:m.}
\end{itemize}
\begin{itemize}
\item {Utilização:Neol.}
\end{itemize}
\begin{itemize}
\item {Proveniência:(Fr. \textunderscore hotel\textunderscore )}
\end{itemize}
O mesmo que \textunderscore hospedaria\textunderscore , especialmente hospedaria asseada ou luxuosa.
\section{Hoteleiro}
\begin{itemize}
\item {Grp. gram.:m.}
\end{itemize}
\begin{itemize}
\item {Utilização:Neol.}
\end{itemize}
Dono de hotel.
\section{Hotentote}
\begin{itemize}
\item {Grp. gram.:adj.}
\end{itemize}
\begin{itemize}
\item {Grp. gram.:M.}
\end{itemize}
\begin{itemize}
\item {Grp. gram.:Pl.}
\end{itemize}
Relativo aos Hotentotes ou ao seu país.
Lingua dos Hotentotes.
Numeroso povo da África meridional.
\section{Hotentotismo}
\begin{itemize}
\item {Grp. gram.:m.}
\end{itemize}
\begin{itemize}
\item {Proveniência:(De \textunderscore hotentote\textunderscore )}
\end{itemize}
Pronúncia viciosa, semelhante á dos Hotentotes, e que consiste em articulações confusas.
\section{Hou-de-lá!}
\begin{itemize}
\item {Grp. gram.:interj.}
\end{itemize}
O mesmo que \textunderscore houlá!\textunderscore 
\section{Houlá!}
\begin{itemize}
\item {Grp. gram.:interj.}
\end{itemize}
\begin{itemize}
\item {Utilização:Ant.}
\end{itemize}
O mesmo que \textunderscore olá\textunderscore 
\section{Houllétia}
\begin{itemize}
\item {Grp. gram.:f.}
\end{itemize}
\begin{itemize}
\item {Proveniência:(De \textunderscore Houllet\textunderscore , n. p.)}
\end{itemize}
Orchídea brasileira.
\section{Hova}
\begin{itemize}
\item {Grp. gram.:m.}
\end{itemize}
\begin{itemize}
\item {Grp. gram.:Pl.}
\end{itemize}
Língua dos Hovas.
Tríbo guerreira, de origem malaia, que submeteu Madagáscar, onde predomina.
\section{Hóvea}
\begin{itemize}
\item {Grp. gram.:f.}
\end{itemize}
\begin{itemize}
\item {Proveniência:(De \textunderscore Hove\textunderscore , n. p.)}
\end{itemize}
Planta leguminosa da Austrália, espécie de lódão.
\section{Hu}
\begin{itemize}
\item {Grp. gram.:adv.}
\end{itemize}
\begin{itemize}
\item {Utilização:Ant.}
\end{itemize}
\begin{itemize}
\item {Proveniência:(Do fr. \textunderscore où\textunderscore )}
\end{itemize}
Onde.
\section{Huambizas}
\begin{itemize}
\item {Grp. gram.:m. pl.}
\end{itemize}
Tríbo feroz da região do Alto-Amazonas.
\section{Hucha}
\begin{itemize}
\item {Grp. gram.:f.}
\end{itemize}
\begin{itemize}
\item {Proveniência:(Do b. lat. \textunderscore hutica\textunderscore )}
\end{itemize}
Caixa ou casa, em que se guardam gêneros alimenticios.
\section{Huchão}
\begin{itemize}
\item {Grp. gram.:m.}
\end{itemize}
\begin{itemize}
\item {Proveniência:(De \textunderscore hucha\textunderscore )}
\end{itemize}
Aquelle que tem a seu cargo a hucharia.
\section{Hucharia}
\begin{itemize}
\item {Grp. gram.:f.}
\end{itemize}
\begin{itemize}
\item {Utilização:Prov.}
\end{itemize}
\begin{itemize}
\item {Proveniência:(De \textunderscore hucha\textunderscore )}
\end{itemize}
Deposito de gêneros alimentícios.
Abundância, fartura: \textunderscore é muito rico, tem em casa uma ucharia de tudo que é preciso\textunderscore .
\section{Huchote}
\begin{itemize}
\item {Grp. gram.:m.}
\end{itemize}
\begin{itemize}
\item {Utilização:Des.}
\end{itemize}
Pequena hucha.
\section{Hudsónia}
\begin{itemize}
\item {Grp. gram.:f.}
\end{itemize}
\begin{itemize}
\item {Proveniência:(De \textunderscore Hudson\textunderscore , n. p.)}
\end{itemize}
Gênero de arbustos cystíneos da América boreal.
\section{Huguenotes}
\begin{itemize}
\item {Grp. gram.:m. pl.}
\end{itemize}
\begin{itemize}
\item {Proveniência:(Fr. \textunderscore huguenots\textunderscore )}
\end{itemize}
Designação depreciativa, que os Cathólicos deram em França aos Protestantes, especialmente aos Calvinistas, e que êstes adoptaram.
\section{Huido}
\begin{itemize}
\item {Grp. gram.:m.}
\end{itemize}
Árvore angolense de Caconda.
\section{Hulha}
\begin{itemize}
\item {Grp. gram.:f.}
\end{itemize}
\begin{itemize}
\item {Proveniência:(Do fr. \textunderscore houille\textunderscore )}
\end{itemize}
Carvão de pedra.
\section{Hulheira}
\begin{itemize}
\item {Grp. gram.:f.}
\end{itemize}
Mina de hulha.
\section{Hulheiro}
\begin{itemize}
\item {Grp. gram.:adj.}
\end{itemize}
Relativo a hulha.
\section{Hulhífero}
\begin{itemize}
\item {Grp. gram.:adj.}
\end{itemize}
\begin{itemize}
\item {Proveniência:(De \textunderscore hulha\textunderscore  + lat. \textunderscore ferre\textunderscore )}
\end{itemize}
Que tem hulha ou produz hulha.
\section{Hum!}
\begin{itemize}
\item {Grp. gram.:interj.}
\end{itemize}
(designativa de hesitação, dúvida, impaciência, etc.)
\section{Hum}
\begin{itemize}
\item {Grp. gram.:adv.}
\end{itemize}
\begin{itemize}
\item {Utilização:Ant.}
\end{itemize}
O mesmo que \textunderscore hu\textunderscore .
\section{Humanal}
\begin{itemize}
\item {Grp. gram.:adj.}
\end{itemize}
O mesmo que \textunderscore humano\textunderscore .
\section{Humanamente}
\begin{itemize}
\item {Grp. gram.:adv.}
\end{itemize}
De modo humano, segundo a natureza humana.
Compassivamente.
\section{Humanar}
\begin{itemize}
\item {Grp. gram.:v. t.}
\end{itemize}
Tornar humano; dar a condição de homem a.
Tornar benévolo, affável.
\section{Humanas}
\begin{itemize}
\item {Grp. gram.:m. pl.}
\end{itemize}
Tríbo extinta do Alto-Amazonas.
\section{Humanidade}
\begin{itemize}
\item {Grp. gram.:f.}
\end{itemize}
\begin{itemize}
\item {Grp. gram.:Pl.}
\end{itemize}
\begin{itemize}
\item {Proveniência:(Lat. \textunderscore humanitas\textunderscore )}
\end{itemize}
O conjunto dos homens.
Natureza humana.
O gênero humano.
Clemência; sentímento de benevolência do homem para homem: \textunderscore tratar alguém com humanidade\textunderscore .
Benevolência.
Estudo das bellas-letras.
\section{Humanismo}
\begin{itemize}
\item {Grp. gram.:m.}
\end{itemize}
\begin{itemize}
\item {Proveniência:(De \textunderscore humano\textunderscore )}
\end{itemize}
Deificação da humanidade.
Influência do estudo de humanidades ou bellas-letras. Cf. C. Micaëlis, \textunderscore Inf. D. Maria\textunderscore , 5.
\section{Humanista}
\begin{itemize}
\item {Grp. gram.:m.}
\end{itemize}
\begin{itemize}
\item {Proveniência:(De \textunderscore humano\textunderscore )}
\end{itemize}
Aquelle que é versado em humanidades.
\section{Humanístico}
\begin{itemize}
\item {Grp. gram.:adj.}
\end{itemize}
Relativo aos humanistas.
\section{Humanitário}
\begin{itemize}
\item {Grp. gram.:adj.}
\end{itemize}
\begin{itemize}
\item {Grp. gram.:M.}
\end{itemize}
\begin{itemize}
\item {Proveniência:(Do lat. \textunderscore humanitas\textunderscore )}
\end{itemize}
Relativo á humanidade.
Que interessa a toda a humanidade.
Que tem bons sentimentos para com o gênero humano.
Conducente ao bem geral da humanidade, ou ao bem de um ou mais indivíduos: \textunderscore actos humanitários\textunderscore .
Homem, que deseja e procura o bem da humanidade, considerada collectivamente.
Philantropo.
\section{Humanitarismo}
\begin{itemize}
\item {Grp. gram.:m.}
\end{itemize}
\begin{itemize}
\item {Proveniência:(De \textunderscore humanitário\textunderscore )}
\end{itemize}
Amor á humanidade.
Philantropia.
\section{Humanização}
\begin{itemize}
\item {Grp. gram.:f.}
\end{itemize}
Acto ou effeito de humanizar.
\section{Humanizar}
\begin{itemize}
\item {Grp. gram.:v. t.}
\end{itemize}
O mesmo que \textunderscore humanar\textunderscore .
\section{Humano}
\begin{itemize}
\item {Grp. gram.:adj.}
\end{itemize}
\begin{itemize}
\item {Grp. gram.:M. pl.}
\end{itemize}
\begin{itemize}
\item {Proveniência:(Lat. \textunderscore humanus\textunderscore )}
\end{itemize}
Relativo ao homem; próprio do homem.
Humanitário; bondoso.
Os homens.
\section{Humará}
\begin{itemize}
\item {Grp. gram.:m.}
\end{itemize}
Ave nocturna da região do Amazonas.
\section{Humboldtite}
\begin{itemize}
\item {Grp. gram.:f.}
\end{itemize}
\begin{itemize}
\item {Proveniência:(De \textunderscore Humboldt\textunderscore , n. p.)}
\end{itemize}
Oxalato de ferro mineral.
\section{Humbral}
\begin{itemize}
\item {Grp. gram.:m.}
\end{itemize}
\begin{itemize}
\item {Proveniência:(Do lat. \textunderscore humerale\textunderscore )}
\end{itemize}
O mesmo que \textunderscore ombreira\textunderscore .
\section{Humectação}
\begin{itemize}
\item {Grp. gram.:f.}
\end{itemize}
\begin{itemize}
\item {Proveniência:(Lat. \textunderscore humectatio\textunderscore )}
\end{itemize}
Acto ou effeito de humectar.
\section{Humectante}
\begin{itemize}
\item {Grp. gram.:adj.}
\end{itemize}
\begin{itemize}
\item {Proveniência:(Lat. \textunderscore humectans\textunderscore )}
\end{itemize}
Que humecta.
\section{Humectar}
\begin{itemize}
\item {Grp. gram.:v. t.}
\end{itemize}
\begin{itemize}
\item {Proveniência:(Lat. \textunderscore humectare\textunderscore )}
\end{itemize}
Humedecer; molhar.
Diluír.
\section{Humectativo}
\begin{itemize}
\item {Grp. gram.:adj.}
\end{itemize}
\begin{itemize}
\item {Proveniência:(Lat. \textunderscore humectativus\textunderscore )}
\end{itemize}
O mesmo que \textunderscore humectante\textunderscore .
\section{Humedecer}
\begin{itemize}
\item {Grp. gram.:v. t.}
\end{itemize}
Tornar húmido; molhar ligeiramente.
(Por \textunderscore humidecer\textunderscore , de \textunderscore húmido\textunderscore )
\section{Humedecimento}
\begin{itemize}
\item {Grp. gram.:m.}
\end{itemize}
Acto ou effeito de humedecer.
\section{Humente}
\begin{itemize}
\item {Grp. gram.:adj.}
\end{itemize}
\begin{itemize}
\item {Utilização:Poét.}
\end{itemize}
\begin{itemize}
\item {Proveniência:(Lat. \textunderscore humens\textunderscore )}
\end{itemize}
Húmido; humectante.
\section{Híadas}
\begin{itemize}
\item {Grp. gram.:f. pl.}
\end{itemize}
\begin{itemize}
\item {Proveniência:(Gr. \textunderscore huades\textunderscore )}
\end{itemize}
Constelação de sete estrelas.
\section{Hial}
\begin{itemize}
\item {Grp. gram.:adj.}
\end{itemize}
Relativo ao osso hioide.
(Cp. \textunderscore hioide\textunderscore )
\section{Híala}
\begin{itemize}
\item {Grp. gram.:f.}
\end{itemize}
\begin{itemize}
\item {Proveniência:(Do gr. \textunderscore hualos\textunderscore , vidro)}
\end{itemize}
Espécie de molusco de barbatanas amarelas e conchas transparentes.
\section{Hialino}
\begin{itemize}
\item {Grp. gram.:adj.}
\end{itemize}
\begin{itemize}
\item {Proveniência:(Do gr. \textunderscore hualos\textunderscore )}
\end{itemize}
Relativo ao vidro.
Que tem a aparência ou a transparência do vidro.
\section{Hialite}
\begin{itemize}
\item {Grp. gram.:m.}
\end{itemize}
\begin{itemize}
\item {Proveniência:(Do gr. \textunderscore hualos\textunderscore , vidro)}
\end{itemize}
Variedade de quartzo, semelhante ao vidro.
Inflamação do humor vítreo do ôlho.
\section{Hiálito}
\begin{itemize}
\item {Grp. gram.:m.}
\end{itemize}
Vidro opaco, geralmente negro, empregado em objectos de ornato ou luxo, em vasos para conter líquidos em ebulição, etc.
(Por \textunderscore hialólito\textunderscore , do gr. \textunderscore hualos\textunderscore  + \textunderscore lithos\textunderscore )
\section{Hialografia}
\begin{itemize}
\item {Grp. gram.:f.}
\end{itemize}
Pintura, feita com o hialógrafo.
\section{Hialógrafo}
\begin{itemize}
\item {Grp. gram.:m.}
\end{itemize}
\begin{itemize}
\item {Proveniência:(Do gr. \textunderscore hualos\textunderscore  + \textunderscore graphein\textunderscore )}
\end{itemize}
Instrumento, para desenhar a perspectiva e tirar provas de um desenho.
\section{Hialoide}
\begin{itemize}
\item {Grp. gram.:f.}
\end{itemize}
\begin{itemize}
\item {Grp. gram.:Adj.}
\end{itemize}
\begin{itemize}
\item {Proveniência:(Do gr. \textunderscore hualos\textunderscore  + \textunderscore eidos\textunderscore )}
\end{itemize}
Membrana translúcida, que contém o humor vítreo do ôlho.
Que tem a aparência do vidro.
\section{Hialoídeo}
\begin{itemize}
\item {Grp. gram.:adj.}
\end{itemize}
Relativo á hialoide.
\section{Hialoplasma}
\begin{itemize}
\item {Grp. gram.:m.}
\end{itemize}
\begin{itemize}
\item {Proveniência:(Do gr. \textunderscore hualos\textunderscore  + \textunderscore plasma\textunderscore )}
\end{itemize}
Plasma hialino.
\section{Hialossomo}
\begin{itemize}
\item {Grp. gram.:adj.}
\end{itemize}
\begin{itemize}
\item {Utilização:Zool.}
\end{itemize}
\begin{itemize}
\item {Proveniência:(Do gr. \textunderscore hualos\textunderscore  + \textunderscore soma\textunderscore )}
\end{itemize}
Que tem corpo translúcido como o vidro.
\section{Hialotecnia}
\begin{itemize}
\item {Grp. gram.:f.}
\end{itemize}
\begin{itemize}
\item {Proveniência:(Do gr. \textunderscore hualos\textunderscore  + \textunderscore tekhne\textunderscore )}
\end{itemize}
Arte de trabalhar em vidro.
\section{Hialotécnico}
\begin{itemize}
\item {Grp. gram.:adj.}
\end{itemize}
Relativo á hialotecnia.
\section{Hialurgia}
\begin{itemize}
\item {Grp. gram.:f.}
\end{itemize}
\begin{itemize}
\item {Proveniência:(Do gr. \textunderscore hualos\textunderscore  + \textunderscore ergon\textunderscore )}
\end{itemize}
Arte de fabricar vidro.
\section{Hialúrgico}
\begin{itemize}
\item {Grp. gram.:adj.}
\end{itemize}
Relativo á hialurgia.
\section{Hibode}
\begin{itemize}
\item {Grp. gram.:m.}
\end{itemize}
\begin{itemize}
\item {Proveniência:(Do gr. \textunderscore hubos\textunderscore  + \textunderscore odous\textunderscore )}
\end{itemize}
Gênero de peixes plagióstomos.
\section{Hiboma}
\begin{itemize}
\item {Grp. gram.:m.}
\end{itemize}
\begin{itemize}
\item {Proveniência:(Gr. \textunderscore huboma\textunderscore )}
\end{itemize}
Gênero de insectos coleópteros pentâmeros.
\section{Hibridação}
\begin{itemize}
\item {Grp. gram.:f.}
\end{itemize}
Produção de plantas ou de animaes híbridos.
\section{Hibridade}
\begin{itemize}
\item {Grp. gram.:f.}
\end{itemize}
O mesmo que \textunderscore hibridez\textunderscore .
\section{Hibridez}
\begin{itemize}
\item {Grp. gram.:f.}
\end{itemize}
Qualidade de híbrido.
\section{Hibridismo}
\begin{itemize}
\item {Grp. gram.:m.}
\end{itemize}
O mesmo que \textunderscore hibridez\textunderscore .
\section{Híbrido}
\begin{itemize}
\item {Grp. gram.:adj.}
\end{itemize}
\begin{itemize}
\item {Utilização:Philol.}
\end{itemize}
\begin{itemize}
\item {Proveniência:(Lat. \textunderscore hybrida\textunderscore )}
\end{itemize}
Que provém de espécies diferentes.
Que se afasta das leis naturaes.
Composto de elementos provenientes de diferentes línguas, (falando-se de um vocábulo).
\section{Hibrísticas}
\begin{itemize}
\item {Grp. gram.:f. pl.}
\end{itemize}
Festas, que se celebravam em Argos, em honra de uma heroína que libertou a cidade do cêrco dos Lacedemónios, e nas quaes os homens se vestiam como mulheres e as mulheres como homens. Cf. Castilho, \textunderscore Fastos\textunderscore , I, 542.
(Cp. \textunderscore híbrido\textunderscore )
\section{Hidático}
\begin{itemize}
\item {Grp. gram.:adj.}
\end{itemize}
Relativo aos hidátides.
\section{Hidátides}
\begin{itemize}
\item {Grp. gram.:m. pl.}
\end{itemize}
\begin{itemize}
\item {Proveniência:(Do gr. \textunderscore hudatis\textunderscore )}
\end{itemize}
Espécie de parasitos, caracterizados por vesículas livres, de todos os lados.
Cistos, que contêm um líquido aquoso e transparente.
\section{Hidatídico}
\begin{itemize}
\item {Grp. gram.:adj.}
\end{itemize}
O mesmo que \textunderscore hidático\textunderscore .
\section{Hidatidina}
\begin{itemize}
\item {Grp. gram.:f.}
\end{itemize}
Substância, encontrada nos hidátides.
\section{Hidatidocele}
\begin{itemize}
\item {Grp. gram.:m.}
\end{itemize}
\begin{itemize}
\item {Proveniência:(Do gr. \textunderscore hudatis\textunderscore  + \textunderscore kele\textunderscore )}
\end{itemize}
Tumor, que contém hidátides.
\section{Hidatiforme}
\begin{itemize}
\item {Grp. gram.:adj.}
\end{itemize}
\begin{itemize}
\item {Proveniência:(De \textunderscore hidátides\textunderscore  + \textunderscore fórma\textunderscore )}
\end{itemize}
Que tem a transparência dos hidatides.
\section{Hidatígero}
\begin{itemize}
\item {Grp. gram.:adj.}
\end{itemize}
O mesmo que \textunderscore cisticerco\textunderscore .
\section{Hidatismo}
\begin{itemize}
\item {Grp. gram.:m.}
\end{itemize}
\begin{itemize}
\item {Utilização:Med.}
\end{itemize}
\begin{itemize}
\item {Proveniência:(Do gr. \textunderscore hudor\textunderscore , \textunderscore hudatos\textunderscore , água)}
\end{itemize}
Ruído, causado pela fluctuação de um líquido numa cavidade.
\section{Hidatoide}
\begin{itemize}
\item {Grp. gram.:adj.}
\end{itemize}
\begin{itemize}
\item {Utilização:Anat.}
\end{itemize}
\begin{itemize}
\item {Proveniência:(Do gr. \textunderscore hudor\textunderscore , \textunderscore hudatos\textunderscore  + \textunderscore eidos\textunderscore )}
\end{itemize}
Diz-se da membrana do humor aquoso.
\section{Hidatologia}
\begin{itemize}
\item {Grp. gram.:f.}
\end{itemize}
\begin{itemize}
\item {Proveniência:(Do gr. \textunderscore hudor\textunderscore , \textunderscore hudatos\textunderscore  + \textunderscore logos\textunderscore )}
\end{itemize}
O mesmo que \textunderscore hidrologia\textunderscore .
\section{Hidatoscopia}
\begin{itemize}
\item {Grp. gram.:f.}
\end{itemize}
\begin{itemize}
\item {Proveniência:(Do gr. \textunderscore hudor\textunderscore , \textunderscore hudatos\textunderscore  + \textunderscore skopein\textunderscore )}
\end{itemize}
Suposta arte de adivinhar por meio da água.
\section{Hidátulo}
\begin{itemize}
\item {Grp. gram.:adj.}
\end{itemize}
O mesmo que \textunderscore cisticerco\textunderscore .
\section{Hidra}
\begin{itemize}
\item {Grp. gram.:f.}
\end{itemize}
\begin{itemize}
\item {Utilização:Fig.}
\end{itemize}
\begin{itemize}
\item {Proveniência:(Gr. \textunderscore hudra\textunderscore )}
\end{itemize}
Serpente fabulosa.
Constelação austral.
Pólipo de água doce.
Cobra de água doce.
Espécie de esqualo.
Coisa ou facto, que envolve perigo público ou que ameaça a ordem social.
\section{Hidrácido}
\begin{itemize}
\item {Grp. gram.:m.}
\end{itemize}
\begin{itemize}
\item {Utilização:Chím.}
\end{itemize}
Ácido, resultante da combinação de um corpo simples, ou composto, com o hidrogênico.
(Contr. \textunderscore hidrogênio\textunderscore  + \textunderscore ácido\textunderscore )
\section{Hidragogo}
\begin{itemize}
\item {Grp. gram.:adj.}
\end{itemize}
\begin{itemize}
\item {Utilização:Med.}
\end{itemize}
\begin{itemize}
\item {Grp. gram.:M.}
\end{itemize}
\begin{itemize}
\item {Proveniência:(Do gr. \textunderscore hudor\textunderscore  + \textunderscore agogos\textunderscore )}
\end{itemize}
Que serve para evacuar a serosidade.
Medicamento hidragogo.
\section{Hidrálcool}
\begin{itemize}
\item {Grp. gram.:m.}
\end{itemize}
\begin{itemize}
\item {Proveniência:(De \textunderscore hidro...\textunderscore  + \textunderscore álcool\textunderscore )}
\end{itemize}
Álcool de 22 graus centesimaes, metade de cujo volume é água.
Aguardente.
\section{Hidrângea}
\begin{itemize}
\item {Grp. gram.:f.}
\end{itemize}
O mesmo que \textunderscore hidranja\textunderscore .
\section{Hidranja}
\begin{itemize}
\item {Grp. gram.:f.}
\end{itemize}
\begin{itemize}
\item {Proveniência:(Do gr. \textunderscore hudor\textunderscore  + \textunderscore angos\textunderscore )}
\end{itemize}
Gênero de arbustos, o mesmo que \textunderscore hortênsia\textunderscore .
\section{Hidrante}
\begin{itemize}
\item {Grp. gram.:m.}
\end{itemize}
\begin{itemize}
\item {Utilização:Bras. de Pernambuco}
\end{itemize}
\begin{itemize}
\item {Proveniência:(De \textunderscore hidro...\textunderscore )}
\end{itemize}
Válvula ou torneira, a que se liga a mangueira que conduz a água para extincção de um incêndio.
\section{Hidrargiria}
\begin{itemize}
\item {Grp. gram.:f.}
\end{itemize}
\begin{itemize}
\item {Utilização:Med.}
\end{itemize}
\begin{itemize}
\item {Proveniência:(De \textunderscore hdrargírio\textunderscore )}
\end{itemize}
Erupção cutânea, caracterizada por pequenas vesículas e resultante da aplicação de medicamentos mercuriaes.
\section{Hidrargírico}
\begin{itemize}
\item {Grp. gram.:adj.}
\end{itemize}
\begin{itemize}
\item {Proveniência:(De \textunderscore hidrargírio\textunderscore )}
\end{itemize}
Relativo ao hidrargirio.
Feito de mercúrio; em cuja composição entra o mercúrio.
\section{Hidrargíridos}
\begin{itemize}
\item {Grp. gram.:m. pl.}
\end{itemize}
\begin{itemize}
\item {Utilização:Chím.}
\end{itemize}
Família de corpos, que tem por tipo o hidrargírio.
\section{Hidrargírio}
\begin{itemize}
\item {Grp. gram.:m.}
\end{itemize}
\begin{itemize}
\item {Utilização:Chím.}
\end{itemize}
\begin{itemize}
\item {Proveniência:(Do gr. \textunderscore hudor\textunderscore  + \textunderscore argurion\textunderscore )}
\end{itemize}
Designação antiga do mercúrio.
\section{Hidrargirismo}
\begin{itemize}
\item {Grp. gram.:m.}
\end{itemize}
\begin{itemize}
\item {Proveniência:(De \textunderscore hidrargírio\textunderscore )}
\end{itemize}
Estado mórbido, produzido pelo uso excessivo do mercúrio.
\section{Hidrargirose}
\begin{itemize}
\item {Grp. gram.:f.}
\end{itemize}
\begin{itemize}
\item {Utilização:Med.}
\end{itemize}
\begin{itemize}
\item {Proveniência:(De \textunderscore hidrargírio\textunderscore )}
\end{itemize}
Fricção mercurial.
\section{Hidrartro}
\begin{itemize}
\item {Grp. gram.:m.}
\end{itemize}
O mesmo que \textunderscore hidrartrose\textunderscore .
\section{Hidrartrose}
\begin{itemize}
\item {Grp. gram.:f.}
\end{itemize}
\begin{itemize}
\item {Proveniência:(Do gr. \textunderscore hudor\textunderscore  + \textunderscore arthron\textunderscore )}
\end{itemize}
Tumor, em volta de uma articulação, estorvando-lhe os movimentos.
Hidropisia articular.
\section{Humeral}
\begin{itemize}
\item {Grp. gram.:adj.}
\end{itemize}
Relativo ao húmero.
\section{Humerário}
\begin{itemize}
\item {Grp. gram.:adj.}
\end{itemize}
Relativo ao húmero.
\section{Húmero}
\begin{itemize}
\item {Grp. gram.:m.}
\end{itemize}
\begin{itemize}
\item {Proveniência:(Lat. \textunderscore humerus\textunderscore , ou \textunderscore umerus\textunderscore )}
\end{itemize}
Parte do braço, comprehendida entre o cotovelo e a espádua.
\section{Humidade}
\begin{itemize}
\item {Grp. gram.:f.}
\end{itemize}
Qualidade daquillo que é húmido.
Abundância do humor no organismo animal.
Relento da noite.
\section{Humidífobo}
\begin{itemize}
\item {Grp. gram.:adj.}
\end{itemize}
\begin{itemize}
\item {Proveniência:(T. hybr., do lat. \textunderscore humidus\textunderscore  + gr. \textunderscore phobos\textunderscore )}
\end{itemize}
Diz-se de certas plantas, que se não dão bem nos terrenos húmidos.
\section{Humidíphobo}
\begin{itemize}
\item {Grp. gram.:adj.}
\end{itemize}
\begin{itemize}
\item {Proveniência:(T. hybr., do lat. \textunderscore humidus\textunderscore  + gr. \textunderscore phobos\textunderscore )}
\end{itemize}
Diz-se de certas plantas, que se não dão bem nos terrenos húmidos.
\section{Húmido}
\begin{itemize}
\item {Grp. gram.:adj.}
\end{itemize}
\begin{itemize}
\item {Proveniência:(Lat. \textunderscore humidus\textunderscore , ou \textunderscore umidus\textunderscore )}
\end{itemize}
Levemente molhado; que tem a natureza da água.
Aquoso; impregnando de vapores aquosos.
\section{Humificação}
\begin{itemize}
\item {Grp. gram.:f.}
\end{itemize}
\begin{itemize}
\item {Proveniência:(Do lat. \textunderscore humus\textunderscore  + \textunderscore facere\textunderscore )}
\end{itemize}
Transformação em humo.
\section{Húmil}
\begin{itemize}
\item {Grp. gram.:adj.}
\end{itemize}
(V.húmile)
\section{Humildação}
\begin{itemize}
\item {Grp. gram.:f.}
\end{itemize}
\begin{itemize}
\item {Proveniência:(De \textunderscore humildar\textunderscore )}
\end{itemize}
O mesmo que \textunderscore humilhação\textunderscore .
\section{Humildade}
\begin{itemize}
\item {Grp. gram.:f.}
\end{itemize}
\begin{itemize}
\item {Proveniência:(Lat. \textunderscore humilitas\textunderscore )}
\end{itemize}
Qualidade daquelle ou daquillo que é humilde.
Demonstração de respeito, de submissão.
Inferioridade; pobreza.
\section{Humildar}
\begin{itemize}
\item {Grp. gram.:v. t.}
\end{itemize}
Tornar humilde; submeter, sujeitar.
\section{Humilde}
\begin{itemize}
\item {Grp. gram.:adj.}
\end{itemize}
\begin{itemize}
\item {Grp. gram.:M.}
\end{itemize}
\begin{itemize}
\item {Proveniência:(Do lat. \textunderscore humilis\textunderscore )}
\end{itemize}
Baixo, rasteiro: \textunderscore a humilde violeta\textunderscore .
Modesto; submisso.
Obscuro.
Medíocre.
Que tem o sentimento do pouco que vale.
Mísero.
Indivíduo humilde.
\section{Humildemente}
\begin{itemize}
\item {Grp. gram.:adv.}
\end{itemize}
De modo humilde; com humildade; submissamente.
\section{Humildeza}
\begin{itemize}
\item {fónica:dê}
\end{itemize}
\begin{itemize}
\item {Grp. gram.:f.}
\end{itemize}
\begin{itemize}
\item {Utilização:Prov.}
\end{itemize}
\begin{itemize}
\item {Utilização:alent.}
\end{itemize}
O mesmo que \textunderscore humildade\textunderscore .
\section{Humildíssimo}
\begin{itemize}
\item {Grp. gram.:adj.}
\end{itemize}
\begin{itemize}
\item {Proveniência:(De \textunderscore humilde\textunderscore )}
\end{itemize}
O mesmo que \textunderscore humíllimo\textunderscore :«\textunderscore ...implorava humildíssimo a vênia...\textunderscore »Filinto, \textunderscore D. Man.\textunderscore , II, 127.
\section{Humildosamente}
\begin{itemize}
\item {Grp. gram.:adv.}
\end{itemize}
O mesmo que \textunderscore humildemente\textunderscore .
\section{Humildoso}
\begin{itemize}
\item {Grp. gram.:adj.}
\end{itemize}
O mesmo que \textunderscore humilde\textunderscore .
\section{Húmile}
\begin{itemize}
\item {Grp. gram.:adj.}
\end{itemize}
\begin{itemize}
\item {Utilização:Poét.}
\end{itemize}
\begin{itemize}
\item {Proveniência:(Lat. \textunderscore humilis\textunderscore )}
\end{itemize}
O mesmo que \textunderscore humilde\textunderscore .
\section{Humilhação}
\begin{itemize}
\item {Grp. gram.:f.}
\end{itemize}
\begin{itemize}
\item {Proveniência:(Do lat. \textunderscore humiliatio\textunderscore )}
\end{itemize}
Acto ou effeito de humilhar.
\section{Humilhadamente}
\begin{itemize}
\item {Grp. gram.:adv.}
\end{itemize}
\begin{itemize}
\item {Proveniência:(De \textunderscore humilhar\textunderscore )}
\end{itemize}
Com humilhação.
\section{Humilhante}
\begin{itemize}
\item {Grp. gram.:adj.}
\end{itemize}
\begin{itemize}
\item {Proveniência:(Do lat. \textunderscore humilians\textunderscore )}
\end{itemize}
Que humilha; que desdoira; vexatorio.
\section{Humilhar}
\begin{itemize}
\item {Grp. gram.:v. t.}
\end{itemize}
\begin{itemize}
\item {Grp. gram.:V. i.}
\end{itemize}
\begin{itemize}
\item {Proveniência:(Do lat. \textunderscore humiliare\textunderscore )}
\end{itemize}
Humildar; abater.
Vexar; tratar desdenhosamente, com soberba.
Diz-se do toiro, quando abaixa a cabeça para marrar.
\section{Humilhoso}
\begin{itemize}
\item {Grp. gram.:adj.}
\end{itemize}
\begin{itemize}
\item {Utilização:P. us.}
\end{itemize}
O mesmo que \textunderscore humilhante\textunderscore . Cf. Júlio Ribeiro, \textunderscore Carne\textunderscore .
\section{Humiliação}
\begin{itemize}
\item {Grp. gram.:f.}
\end{itemize}
\begin{itemize}
\item {Proveniência:(Lat. \textunderscore humilitatio\textunderscore )}
\end{itemize}
O mesmo que \textunderscore humilhação\textunderscore . Cf. \textunderscore Luz e Calor\textunderscore .
\section{Humiliante}
\begin{itemize}
\item {Grp. gram.:adj.}
\end{itemize}
\begin{itemize}
\item {Proveniência:(Lat. \textunderscore humilians\textunderscore )}
\end{itemize}
O mesmo que \textunderscore humilhante\textunderscore .
\section{Humíllimo}
\begin{itemize}
\item {Grp. gram.:adj.}
\end{itemize}
\begin{itemize}
\item {Proveniência:(De \textunderscore húmie\textunderscore )}
\end{itemize}
Muito humilde.
\section{Humilmente}
\begin{itemize}
\item {Grp. gram.:adv.}
\end{itemize}
\begin{itemize}
\item {Proveniência:(De \textunderscore húmil\textunderscore )}
\end{itemize}
O mesmo que \textunderscore humildemente\textunderscore . Cf. Filinto, XIII, 285.
\section{Humo}
\begin{itemize}
\item {Grp. gram.:m.}
\end{itemize}
\begin{itemize}
\item {Proveniência:(Lat. \textunderscore humus\textunderscore )}
\end{itemize}
Terra vegetal, que fornece a nutrição das plantas.
\section{Humor}
\begin{itemize}
\item {Grp. gram.:m.}
\end{itemize}
\begin{itemize}
\item {Utilização:Fig.}
\end{itemize}
\begin{itemize}
\item {Utilização:Neol.}
\end{itemize}
\begin{itemize}
\item {Proveniência:(Lat. \textunderscore humor\textunderscore )}
\end{itemize}
Qualquer fluido, contido num corpo organizado.
Producto mórbido e líquido de um corpo orgânico.
Humidade.
Disposição de ânimo: \textunderscore bom humor\textunderscore ; \textunderscore mau humor\textunderscore .
Bôa disposição de espírito; veia comica.
Ironía delicada e alegre, graça, espírito: \textunderscore conversar com humor\textunderscore .
\section{Humorado}
\begin{itemize}
\item {Grp. gram.:adj.}
\end{itemize}
Que tem humores; que está bem ou mal disposto de ânimo.
\section{Humoral}
\begin{itemize}
\item {Grp. gram.:adj.}
\end{itemize}
Relativo a humor.
\section{Humorismo}
\begin{itemize}
\item {Grp. gram.:m.}
\end{itemize}
\begin{itemize}
\item {Proveniência:(De \textunderscore humor\textunderscore )}
\end{itemize}
Systema dos que attribuem todas as doenças á alteração dos humores.
Qualidade de humorísta ou dos escritos humorísticos.
\section{Humorista}
\begin{itemize}
\item {Grp. gram.:m.}
\end{itemize}
\begin{itemize}
\item {Proveniência:(De \textunderscore humor\textunderscore )}
\end{itemize}
Sectário do humorismo.
Aquelle que escreve humoristicamente; em que há feição humorística.
\section{Humoristicamente}
\begin{itemize}
\item {Grp. gram.:adv.}
\end{itemize}
De modo humorístico.
\section{Humorístico}
\begin{itemize}
\item {Grp. gram.:adj.}
\end{itemize}
\begin{itemize}
\item {Proveniência:(De \textunderscore humorista\textunderscore )}
\end{itemize}
Relativo a humor; em que há estilo espirituoso e irónico. Cf. Latino, \textunderscore Humboldt\textunderscore , 54, 253 e 347.
\section{Humoroso}
\begin{itemize}
\item {Grp. gram.:adj.}
\end{itemize}
\begin{itemize}
\item {Grp. gram.:M. pl.}
\end{itemize}
\begin{itemize}
\item {Proveniência:(Lat. \textunderscore humorosus\textunderscore )}
\end{itemize}
O mesmo que \textunderscore humorado\textunderscore .
Que tem humor ou humidade.
Membros de uma academia romana.
\section{Humoso}
\begin{itemize}
\item {Grp. gram.:adj.}
\end{itemize}
Que tem humo: \textunderscore terreno humoso\textunderscore .
\section{Humui}
\begin{itemize}
\item {Grp. gram.:m.}
\end{itemize}
Árvore angolense.
\section{Humuláceas}
\begin{itemize}
\item {Grp. gram.:f. pl.}
\end{itemize}
\begin{itemize}
\item {Proveniência:(De \textunderscore humuláceo\textunderscore )}
\end{itemize}
Família de plantas, que têm por typo o húmulo.
\section{Humuláceo}
\begin{itemize}
\item {Grp. gram.:adj.}
\end{itemize}
Relativo ou semelhante ao húmulo.
\section{Húmulo}
\begin{itemize}
\item {Grp. gram.:m.}
\end{itemize}
Nome scientífico do lúpulo.
\section{Húmus}
\begin{itemize}
\item {Grp. gram.:m.}
\end{itemize}
O mesmo que \textunderscore humo\textunderscore .
\section{Húngaro}
\begin{itemize}
\item {Grp. gram.:adj.}
\end{itemize}
\begin{itemize}
\item {Grp. gram.:M.}
\end{itemize}
\begin{itemize}
\item {Proveniência:(Lat. \textunderscore Hungari\textunderscore )}
\end{itemize}
Relativo á Hungria.
Aquelle que é natural da Hungria.
Língua húngara.
\section{Hungo}
\begin{itemize}
\item {Grp. gram.:m.}
\end{itemize}
Árvore de Angola.
\section{Hungumbei}
\begin{itemize}
\item {Grp. gram.:m.}
\end{itemize}
Árvore angolense.
\section{Hunos}
\begin{itemize}
\item {Grp. gram.:m. pl.}
\end{itemize}
\begin{itemize}
\item {Proveniência:(Lat. \textunderscore Huni\textunderscore )}
\end{itemize}
Povo guerreiro, procedente talvez da Scýthia asiática, e que na Idade-Média assolou várias regiões da Ásia e da Europa.
\section{Hunteriano}
\begin{itemize}
\item {Grp. gram.:adj.}
\end{itemize}
\begin{itemize}
\item {Utilização:Med.}
\end{itemize}
Diz-se da doença ou do mal, mais conhecido por sýphilis.
\section{Hura}
\begin{itemize}
\item {Grp. gram.:f.}
\end{itemize}
Gênero de plantas euphorbiáceas.
\section{Hurdício}
\begin{itemize}
\item {Grp. gram.:m.}
\end{itemize}
\begin{itemize}
\item {Utilização:Ant.}
\end{itemize}
Grade de madeira, com que se resguardavam algumas muralhas, para não serem muito damnificadas por aríetes e projécteis.
(B. lat. \textunderscore hurdicium\textunderscore )
\section{Huri}
\begin{itemize}
\item {Grp. gram.:f.}
\end{itemize}
\begin{itemize}
\item {Utilização:Fig.}
\end{itemize}
\begin{itemize}
\item {Proveniência:(Do fr. \textunderscore houri\textunderscore )}
\end{itemize}
Cada uma das mulheres extremamente bellas que, segundo o \textunderscore Alcorão\textunderscore , devem desposar no céu os crentes muçulmanos.
Mulher bella.
\section{Hurídeas}
\begin{itemize}
\item {Grp. gram.:f. pl.}
\end{itemize}
\begin{itemize}
\item {Utilização:Bot.}
\end{itemize}
Gênero de euphorbiáceas, que tem por typo a hura.
\section{Huroniano}
\begin{itemize}
\item {Grp. gram.:adj.}
\end{itemize}
\begin{itemize}
\item {Utilização:Geol.}
\end{itemize}
\begin{itemize}
\item {Proveniência:(De \textunderscore Huron\textunderscore , n. p. de um lago)}
\end{itemize}
Diz-se do terreno que constitue uma das secções da série paleozoica.
\section{Hurra!}
\begin{itemize}
\item {Grp. gram.:interj.}
\end{itemize}
\begin{itemize}
\item {Grp. gram.:M.}
\end{itemize}
\begin{itemize}
\item {Proveniência:(Do fr. \textunderscore hourra\textunderscore )}
\end{itemize}
(com que se acompanham brindes nos banquetes)
Grito de guerra, com que as tropas russas, especialmente os cosacos, arremetem contra o inimigo.
Grito de alegria, com que os marinheiros ingleses saúdam os seus commandantes ou pessôas notáveis que visitam o navio.
Grito de marinheiros, para fazerem fôrça ao mesmo tempo, ao laborar com os cabos.
\section{Husa}
\begin{itemize}
\item {Grp. gram.:f.}
\end{itemize}
Planta malvácea angolense, (\textunderscore hibiscus sabdariffa\textunderscore , Lin.).
\section{Hussardo}
\begin{itemize}
\item {Grp. gram.:m.}
\end{itemize}
\begin{itemize}
\item {Proveniência:(Do fr. \textunderscore hussard\textunderscore )}
\end{itemize}
Cavalleiro húngaro.
Gentil-homem polaco, armado e seguido de outros homens de armas, na Idade-Média.
Soldado de cavallaria ligeira, na França e na Alemanha, cujo uniforme se assemelha ao da cavallaria húngara.
\section{Hussitas}
\begin{itemize}
\item {Grp. gram.:m. pl.}
\end{itemize}
\begin{itemize}
\item {Proveniência:(De \textunderscore Huss\textunderscore , n. p.)}
\end{itemize}
Herejes, que sustentavam serem indifferentes as bôas obras para a salvação eterna.
\section{Huvejança}
\begin{itemize}
\item {Grp. gram.:f.}
\end{itemize}
Peixe de Portugal.
\section{Huzvareche}
\begin{itemize}
\item {Grp. gram.:m.}
\end{itemize}
Dialecto morto, pertencente ao grupo irânico das línguas indo-europeias.
\section{Hýadas}
\begin{itemize}
\item {Grp. gram.:f. pl.}
\end{itemize}
\begin{itemize}
\item {Proveniência:(Gr. \textunderscore huades\textunderscore )}
\end{itemize}
Constellação de sete estrellas.
\section{Hyal}
\begin{itemize}
\item {Grp. gram.:adj.}
\end{itemize}
Relativo ao osso hyoide.
(Cp. \textunderscore hyoide\textunderscore )
\section{...hyal}
\begin{itemize}
\item {Grp. gram.:suf.}
\end{itemize}
\begin{itemize}
\item {Utilização:Anat.}
\end{itemize}
Designa uma fórma semelhante á da letra grega úpsilo.
\section{Hýala}
\begin{itemize}
\item {Grp. gram.:f.}
\end{itemize}
\begin{itemize}
\item {Proveniência:(Do gr. \textunderscore hualos\textunderscore , vidro)}
\end{itemize}
Espécie de mollusco de barbatanas amarelas e conchas transparentes.
\section{Hyalino}
\begin{itemize}
\item {Grp. gram.:adj.}
\end{itemize}
\begin{itemize}
\item {Proveniência:(Do gr. \textunderscore hualos\textunderscore )}
\end{itemize}
Relativo ao vidro.
Que tem a apparência ou a transparência do vidro.
\section{Hyalite}
\begin{itemize}
\item {Grp. gram.:m.}
\end{itemize}
\begin{itemize}
\item {Proveniência:(Do gr. \textunderscore hualos\textunderscore , vidro)}
\end{itemize}
Variedade de quartzo, semelhante ao vidro.
Inflammação do humor vítreo do ôlho.
\section{Hyálitho}
\begin{itemize}
\item {Grp. gram.:m.}
\end{itemize}
Vidro opaco, geralmente negro, empregado em objectos de ornato ou luxo, em vasos para conter líquidos em ebullição, etc.
(Por \textunderscore hyalólitho\textunderscore , do gr. \textunderscore hualos\textunderscore  + \textunderscore lithos\textunderscore )
\section{Hyalographia}
\begin{itemize}
\item {Grp. gram.:f.}
\end{itemize}
Pintura, feita com o hyalógrapho.
\section{Hyalógrapho}
\begin{itemize}
\item {Grp. gram.:m.}
\end{itemize}
\begin{itemize}
\item {Proveniência:(Do gr. \textunderscore hualos\textunderscore  + \textunderscore graphein\textunderscore )}
\end{itemize}
Instrumento, para desenhar a perspectiva e tirar provas de um desenho.
\section{Hyaloide}
\begin{itemize}
\item {Grp. gram.:f.}
\end{itemize}
\begin{itemize}
\item {Grp. gram.:Adj.}
\end{itemize}
\begin{itemize}
\item {Proveniência:(Do gr. \textunderscore hualos\textunderscore  + \textunderscore eidos\textunderscore )}
\end{itemize}
Membrana translúcida, que contém o humor vítreo do ôlho.
Que tem a apparência do vidro.
\section{Hyaloídeo}
\begin{itemize}
\item {Grp. gram.:adj.}
\end{itemize}
Relativo á hyaloide.
\section{Hyaloplasma}
\begin{itemize}
\item {Grp. gram.:m.}
\end{itemize}
\begin{itemize}
\item {Proveniência:(Do gr. \textunderscore hualos\textunderscore  + \textunderscore plasma\textunderscore )}
\end{itemize}
Plasma hyalino.
\section{Hyalosomo}
\begin{itemize}
\item {fónica:sô}
\end{itemize}
\begin{itemize}
\item {Grp. gram.:adj.}
\end{itemize}
\begin{itemize}
\item {Utilização:Zool.}
\end{itemize}
\begin{itemize}
\item {Proveniência:(Do gr. \textunderscore hualos\textunderscore  + \textunderscore soma\textunderscore )}
\end{itemize}
Que tem corpo translúcido como o vidro.
\section{Hyalotechnia}
\begin{itemize}
\item {Grp. gram.:f.}
\end{itemize}
\begin{itemize}
\item {Proveniência:(Do gr. \textunderscore hualos\textunderscore  + \textunderscore tekhne\textunderscore )}
\end{itemize}
Arte de trabalhar em vidro.
\section{Hyalotéchnico}
\begin{itemize}
\item {Grp. gram.:adj.}
\end{itemize}
Relativo á hyalotechnia.
\section{Hyalurgia}
\begin{itemize}
\item {Grp. gram.:f.}
\end{itemize}
\begin{itemize}
\item {Proveniência:(Do gr. \textunderscore hualos\textunderscore  + \textunderscore ergon\textunderscore )}
\end{itemize}
Arte de fabricar vidro.
\section{Hyalúrgico}
\begin{itemize}
\item {Grp. gram.:adj.}
\end{itemize}
Relativo á hyalurgia.
\section{Hybode}
\begin{itemize}
\item {Grp. gram.:m.}
\end{itemize}
\begin{itemize}
\item {Proveniência:(Do gr. \textunderscore hubos\textunderscore  + \textunderscore odous\textunderscore )}
\end{itemize}
Gênero de peixes plagióstomos.
\section{Hyboma}
\begin{itemize}
\item {Grp. gram.:m.}
\end{itemize}
\begin{itemize}
\item {Proveniência:(Gr. \textunderscore huboma\textunderscore )}
\end{itemize}
Gênero de insectos coleópteros pentâmeros.
\section{Hybridação}
\begin{itemize}
\item {Grp. gram.:f.}
\end{itemize}
Producção de plantas ou de animaes hýbridos.
\section{Hybridade}
\begin{itemize}
\item {Grp. gram.:f.}
\end{itemize}
O mesmo que \textunderscore hybridez\textunderscore .
\section{Hybridez}
\begin{itemize}
\item {Grp. gram.:f.}
\end{itemize}
Qualidade de hýbrido.
\section{Hybridismo}
\begin{itemize}
\item {Grp. gram.:m.}
\end{itemize}
O mesmo que \textunderscore hybridez\textunderscore .
\section{Hýbrido}
\begin{itemize}
\item {Grp. gram.:adj.}
\end{itemize}
\begin{itemize}
\item {Utilização:Philol.}
\end{itemize}
\begin{itemize}
\item {Proveniência:(Lat. \textunderscore hybrida\textunderscore )}
\end{itemize}
Que provém de espécies differentes.
Que se afasta das leis naturaes.
Composto de elementos provenientes de differentes línguas, (falando-se de um vocábulo).
\section{Hybrísticas}
\begin{itemize}
\item {Grp. gram.:f. pl.}
\end{itemize}
Festas, que se celebravam em Argos, em honra de uma heroína que libertou a cidade do cêrco dos Lacedemónios, e nas quaes os homens se vestiam como mulheres e as mulheres como homens. Cf. Castilho, \textunderscore Fastos\textunderscore , I, 542.
(Cp. \textunderscore hýbrido\textunderscore )
\section{Hýdático}
\begin{itemize}
\item {Grp. gram.:adj.}
\end{itemize}
Relativo aos hydátides.
\section{Hydátides}
\begin{itemize}
\item {Grp. gram.:m. pl.}
\end{itemize}
\begin{itemize}
\item {Proveniência:(Do gr. \textunderscore hudatis\textunderscore )}
\end{itemize}
Espécie de parasitos, caracterizados por vesículas livres, de todos os lados.
Cystos, que contêm um líquido aquoso e transparente.
\section{Hydatídico}
\begin{itemize}
\item {Grp. gram.:adj.}
\end{itemize}
O mesmo que \textunderscore hidático\textunderscore .
\section{Hydatidina}
\begin{itemize}
\item {Grp. gram.:f.}
\end{itemize}
Substância, encontrada nos hydátides.
\section{Hydatidocele}
\begin{itemize}
\item {Grp. gram.:m.}
\end{itemize}
\begin{itemize}
\item {Proveniência:(Do gr. \textunderscore hudatis\textunderscore  + \textunderscore kele\textunderscore )}
\end{itemize}
Tumor, que contém hydátides.
\section{Hydatiforme}
\begin{itemize}
\item {Grp. gram.:adj.}
\end{itemize}
\begin{itemize}
\item {Proveniência:(De \textunderscore hydátides\textunderscore  + \textunderscore fórma\textunderscore )}
\end{itemize}
Que tem a transparência dos hydatides.
\section{Hydatígero}
\begin{itemize}
\item {Grp. gram.:adj.}
\end{itemize}
O mesmo que \textunderscore cysticerco\textunderscore .
\section{Hydatismo}
\begin{itemize}
\item {Grp. gram.:m.}
\end{itemize}
\begin{itemize}
\item {Utilização:Med.}
\end{itemize}
\begin{itemize}
\item {Proveniência:(Do gr. \textunderscore hudor\textunderscore , \textunderscore hudatos\textunderscore , água)}
\end{itemize}
Ruído, causado pela fluctuação de um líquido numa cavidade.
\section{Hydatoide}
\begin{itemize}
\item {Grp. gram.:adj.}
\end{itemize}
\begin{itemize}
\item {Utilização:Anat.}
\end{itemize}
\begin{itemize}
\item {Proveniência:(Do gr. \textunderscore hudor\textunderscore , \textunderscore hudatos\textunderscore  + \textunderscore eidos\textunderscore )}
\end{itemize}
Diz-se da membrana do humor aquoso.
\section{Hydatologia}
\begin{itemize}
\item {Grp. gram.:f.}
\end{itemize}
\begin{itemize}
\item {Proveniência:(Do gr. \textunderscore hudor\textunderscore , \textunderscore hudatos\textunderscore  + \textunderscore logos\textunderscore )}
\end{itemize}
O mesmo que \textunderscore hydrologia\textunderscore .
\section{Hydatoscopia}
\begin{itemize}
\item {Grp. gram.:f.}
\end{itemize}
\begin{itemize}
\item {Proveniência:(Do gr. \textunderscore hudor\textunderscore , \textunderscore hudatos\textunderscore  + \textunderscore skopein\textunderscore )}
\end{itemize}
Supposta arte de adivinhar por meio da água.
\section{Hydátulo}
\begin{itemize}
\item {Grp. gram.:adj.}
\end{itemize}
O mesmo que \textunderscore cysticerco\textunderscore .
\section{Hydra}
\begin{itemize}
\item {Grp. gram.:f.}
\end{itemize}
\begin{itemize}
\item {Utilização:Fig.}
\end{itemize}
\begin{itemize}
\item {Proveniência:(Gr. \textunderscore hudra\textunderscore )}
\end{itemize}
Serpente fabulosa.
Constellação austral.
Pólypo de água doce.
Cobra de água doce.
Espécie de esqualo.
Coisa ou facto, que envolve perigo público ou que ameaça a ordem social.
\section{Hydrácido}
\begin{itemize}
\item {Grp. gram.:m.}
\end{itemize}
\begin{itemize}
\item {Utilização:Chím.}
\end{itemize}
Ácido, resultante da combinação de um corpo simples, ou composto, com o hydrogênico.
(Contr. \textunderscore hydrogênio\textunderscore  + \textunderscore ácido\textunderscore )
\section{Hydragogo}
\begin{itemize}
\item {Grp. gram.:adj.}
\end{itemize}
\begin{itemize}
\item {Utilização:Med.}
\end{itemize}
\begin{itemize}
\item {Grp. gram.:M.}
\end{itemize}
\begin{itemize}
\item {Proveniência:(Do gr. \textunderscore hudor\textunderscore  + \textunderscore agogos\textunderscore )}
\end{itemize}
Que serve para evacuar a serosidade.
Medicamento hydragogo.
\section{Hydrálcool}
\begin{itemize}
\item {Grp. gram.:m.}
\end{itemize}
\begin{itemize}
\item {Proveniência:(De \textunderscore hydro...\textunderscore  + \textunderscore álcool\textunderscore )}
\end{itemize}
Álcool de 22 graus centesimaes, metade de cujo volume é água.
Aguardente.
\section{Hydrângea}
\begin{itemize}
\item {Grp. gram.:f.}
\end{itemize}
O mesmo que \textunderscore hydranja\textunderscore .
\section{Hydranja}
\begin{itemize}
\item {Grp. gram.:f.}
\end{itemize}
\begin{itemize}
\item {Proveniência:(Do gr. \textunderscore hudor\textunderscore  + \textunderscore angos\textunderscore )}
\end{itemize}
Gênero de arbustos, o mesmo que \textunderscore hortênsia\textunderscore .
\section{Hydrante}
\begin{itemize}
\item {Grp. gram.:m.}
\end{itemize}
\begin{itemize}
\item {Utilização:Bras. de Pernambuco}
\end{itemize}
\begin{itemize}
\item {Proveniência:(De \textunderscore hydro...\textunderscore )}
\end{itemize}
Válvula ou torneira, a que se liga a mangueira que conduz a água para extincção de um incêndio.
\section{Hydrargyria}
\begin{itemize}
\item {Grp. gram.:f.}
\end{itemize}
\begin{itemize}
\item {Utilização:Med.}
\end{itemize}
\begin{itemize}
\item {Proveniência:(De \textunderscore hydrargýrio\textunderscore )}
\end{itemize}
Erupção cutânea, caracterizada por pequenas vesículas e resultante da applicação de medicamentos mercuriaes.
\section{Hydrargýrico}
\begin{itemize}
\item {Grp. gram.:adj.}
\end{itemize}
\begin{itemize}
\item {Proveniência:(De \textunderscore hydrargýrio\textunderscore )}
\end{itemize}
Relativo ao hydrargyrio.
Feito de mercúrio; em cuja composição entra o mercúrio.
\section{Hydrargýridos}
\begin{itemize}
\item {Grp. gram.:m. pl.}
\end{itemize}
\begin{itemize}
\item {Utilização:Chím.}
\end{itemize}
Família de corpos, que tem por typo o hydrargýrio.
\section{Hydrargýrio}
\begin{itemize}
\item {Grp. gram.:m.}
\end{itemize}
\begin{itemize}
\item {Utilização:Chím.}
\end{itemize}
\begin{itemize}
\item {Proveniência:(Do gr. \textunderscore hudor\textunderscore  + \textunderscore argurion\textunderscore )}
\end{itemize}
Designação antiga do mercúrio.
\section{Hydrargyrismo}
\begin{itemize}
\item {Grp. gram.:m.}
\end{itemize}
\begin{itemize}
\item {Proveniência:(De \textunderscore hydrargýrio\textunderscore )}
\end{itemize}
Estado mórbido, produzido pelo uso excessivo do mercúrio.
\section{Hydrargyrose}
\begin{itemize}
\item {Grp. gram.:f.}
\end{itemize}
\begin{itemize}
\item {Utilização:Med.}
\end{itemize}
\begin{itemize}
\item {Proveniência:(De \textunderscore hydrargýrio\textunderscore )}
\end{itemize}
Fricção mercurial.
\section{Hydrarthro}
\begin{itemize}
\item {Grp. gram.:m.}
\end{itemize}
O mesmo que \textunderscore hydrarthrose\textunderscore .
\section{Hydrarthrose}
\begin{itemize}
\item {Grp. gram.:f.}
\end{itemize}
\begin{itemize}
\item {Proveniência:(Do gr. \textunderscore hudor\textunderscore  + \textunderscore arthron\textunderscore )}
\end{itemize}
Tumor, em volta de uma articulação, estorvando-lhe os movimentos.
Hydropisia articular.
\section{Hidrastina}
\begin{itemize}
\item {Grp. gram.:f.}
\end{itemize}
Alcaloide da ranunculácea \textunderscore hydrastis canadensis\textunderscore .
\section{Hidrastinina}
\begin{itemize}
\item {Grp. gram.:f.}
\end{itemize}
Medicamento anti-febril.
\section{Hidratação}
\begin{itemize}
\item {Grp. gram.:f.}
\end{itemize}
Acto de hidratar. Cf. \textunderscore Techn. Rur.\textunderscore , 113 e 347.
\section{Hidratar}
\begin{itemize}
\item {Grp. gram.:v. t.}
\end{itemize}
Dar o carácter de hidrato a.
\section{Hidratável}
\begin{itemize}
\item {Grp. gram.:adj.}
\end{itemize}
Que se póde hidratar.
\section{Hidrático}
\begin{itemize}
\item {Grp. gram.:adj.}
\end{itemize}
Que tem alguns caracteres de hidrato.
\section{Hidrato}
\begin{itemize}
\item {Grp. gram.:m.}
\end{itemize}
\begin{itemize}
\item {Proveniência:(Do gr. \textunderscore hudor\textunderscore )}
\end{itemize}
Combinação de um óxido metálico com a água, desempenhando esta o papel de ácido.
\section{Hidráulica}
\begin{itemize}
\item {Grp. gram.:f.}
\end{itemize}
\begin{itemize}
\item {Proveniência:(De \textunderscore hidráulico\textunderscore )}
\end{itemize}
Ciência ou arte, que tem por objecto a direcção e emprêgo das águas.
\section{Hidraulicidade}
\begin{itemize}
\item {Grp. gram.:f.}
\end{itemize}
\begin{itemize}
\item {Proveniência:(De \textunderscore hidráulico\textunderscore )}
\end{itemize}
Qualidade das argamassas hidráulicas.
\section{Hidráulico}
\begin{itemize}
\item {Grp. gram.:adj.}
\end{itemize}
\begin{itemize}
\item {Grp. gram.:M.}
\end{itemize}
\begin{itemize}
\item {Proveniência:(Do gr. \textunderscore hudor\textunderscore  + \textunderscore aulos\textunderscore )}
\end{itemize}
Relativo ao movimento das águas nos canos ou canaes e, em geral, a qualquer movimento dos líquidos.
Relativo á hidráulica.
Que endurece na água.
Engenheiro de obras hidráulicas.
Aquele que é versado em hidráulica.
\section{Hidraulo}
\begin{itemize}
\item {Grp. gram.:m.}
\end{itemize}
\begin{itemize}
\item {Proveniência:(Gr. \textunderscore hudraulos\textunderscore )}
\end{itemize}
Instrumento músico, usado pelos Gregos e que funcionava com o auxilio de água.--(O instrumento é hoje desconhecido, mas são talvez reflexo dele os \textunderscore rouxinóis de barro\textunderscore , que os rapazes de Lisbôa enchem de água e tocam, nas noites de Santo António, San-João e San-Pedro)
\section{Hidremia}
\begin{itemize}
\item {Grp. gram.:f.}
\end{itemize}
\begin{itemize}
\item {Utilização:Med.}
\end{itemize}
\begin{itemize}
\item {Proveniência:(Do gr. \textunderscore hudor\textunderscore  + \textunderscore haima\textunderscore )}
\end{itemize}
Predomíneo do plasma sanguíneo sôbre os glóbulos de sangue.
\section{Hidrião}
\begin{itemize}
\item {Grp. gram.:m.}
\end{itemize}
\begin{itemize}
\item {Proveniência:(Gr. \textunderscore hudrion\textunderscore )}
\end{itemize}
Vaso de água lustral, entre os Gregos do Egipto.
Espécie de bilha.
\section{Hidriatria}
\begin{itemize}
\item {Grp. gram.:f.}
\end{itemize}
\begin{itemize}
\item {Utilização:Med.}
\end{itemize}
\begin{itemize}
\item {Proveniência:(Do gr. \textunderscore hudor\textunderscore  + \textunderscore iatreia\textunderscore )}
\end{itemize}
Parte da Terapêutica, que se occupa do emprêgo das águas salgadas, doces e mineraes.
\section{Hidro}
\begin{itemize}
\item {Grp. gram.:m.}
\end{itemize}
\begin{itemize}
\item {Proveniência:(Lat. \textunderscore hydrus\textunderscore )}
\end{itemize}
Cobra de água, hidra.
\section{Hidro...}
\begin{itemize}
\item {Grp. gram.:pref.}
\end{itemize}
\begin{itemize}
\item {Proveniência:(Do gr. \textunderscore hudor\textunderscore )}
\end{itemize}
(designativo de \textunderscore água\textunderscore )
\section{Hidroaéreo}
\begin{itemize}
\item {Grp. gram.:adj.}
\end{itemize}
\begin{itemize}
\item {Utilização:Med.}
\end{itemize}
\begin{itemize}
\item {Proveniência:(De \textunderscore hidro...\textunderscore  + \textunderscore aéreo\textunderscore )}
\end{itemize}
Diz-se do ruído, que denuncia ar e líquidos dentro de uma cavidade orgânica.
\section{Hidroário}
\begin{itemize}
\item {Grp. gram.:m.}
\end{itemize}
\begin{itemize}
\item {Utilização:Med.}
\end{itemize}
\begin{itemize}
\item {Proveniência:(Do gr. \textunderscore hudor\textunderscore  + \textunderscore oarion\textunderscore )}
\end{itemize}
Hidropisia do ovário.
\section{Hidrobatrácios}
\begin{itemize}
\item {Grp. gram.:m. pl.}
\end{itemize}
\begin{itemize}
\item {Proveniência:(De \textunderscore hidro...\textunderscore  + \textunderscore batrácio\textunderscore )}
\end{itemize}
Família de reptis, que vivem habitualmente na água ou em lugares húmidos.
\section{Hidróbio}
\begin{itemize}
\item {Grp. gram.:adj.}
\end{itemize}
\begin{itemize}
\item {Proveniência:(Do gr. \textunderscore hudor\textunderscore  + \textunderscore bios\textunderscore )}
\end{itemize}
Que vive na água.
\section{Hidrobrânquio}
\begin{itemize}
\item {Grp. gram.:adj.}
\end{itemize}
\begin{itemize}
\item {Proveniência:(De \textunderscore hidro...\textunderscore  + \textunderscore brânquias\textunderscore )}
\end{itemize}
Cujas brânquias são próprias para respirar a água.
\section{Hidrobromato}
\begin{itemize}
\item {Grp. gram.:m.}
\end{itemize}
\begin{itemize}
\item {Utilização:Chím.}
\end{itemize}
Sal, produzido pela combinação do ácido hidrobrómico com as bases.
(Cp. \textunderscore hidrobrómico\textunderscore )
\section{Hidrobrómico}
\begin{itemize}
\item {Grp. gram.:adj.}
\end{itemize}
\begin{itemize}
\item {Proveniência:(De \textunderscore hidrogênico\textunderscore  e \textunderscore bromo\textunderscore )}
\end{itemize}
Diz-se de um ácido, resultante da combinação do hidrogênico com o bromo.
\section{Hidrocarbonato}
\begin{itemize}
\item {Grp. gram.:m.}
\end{itemize}
\begin{itemize}
\item {Proveniência:(De \textunderscore hidro...\textunderscore  + \textunderscore carbonato\textunderscore )}
\end{itemize}
Carbonato, que contém água no estado de combinação química.
\section{Hidrocarídeas}
\begin{itemize}
\item {Grp. gram.:f. pl.}
\end{itemize}
Ordem das plantas, que têm por tipo o hidrocáris.
\section{Hidrocáris}
\begin{itemize}
\item {Grp. gram.:m.}
\end{itemize}
\begin{itemize}
\item {Proveniência:(Do gr. \textunderscore hudor\textunderscore  + \textunderscore kharis\textunderscore )}
\end{itemize}
Gênero de plantas aquáticas, perennes e levemente medicinaes.
\section{Hidrocefalia}
\begin{itemize}
\item {Grp. gram.:f.}
\end{itemize}
\begin{itemize}
\item {Proveniência:(Do gr. \textunderscore hudor\textunderscore  + \textunderscore kephale\textunderscore )}
\end{itemize}
Hidropisia cerebral, chamada vulgarmente \textunderscore cabeça de água\textunderscore .
\section{Hidrocéfalo}
\begin{itemize}
\item {Grp. gram.:m.}
\end{itemize}
\begin{itemize}
\item {Grp. gram.:Adj.}
\end{itemize}
O mesmo que \textunderscore hidrocefalia\textunderscore .
Aquele que padece hidrocefalia.
Que sofre hidrocefalia.
\section{Hidrocele}
\begin{itemize}
\item {Grp. gram.:m.}
\end{itemize}
\begin{itemize}
\item {Proveniência:(Lat. \textunderscore hydrocele\textunderscore )}
\end{itemize}
Tumor, formado por um acervo de serosidade.
\section{Hidrocélico}
\begin{itemize}
\item {Grp. gram.:adj.}
\end{itemize}
\begin{itemize}
\item {Grp. gram.:M.}
\end{itemize}
\begin{itemize}
\item {Proveniência:(Lat. \textunderscore hydrocelicus\textunderscore )}
\end{itemize}
Relativo ao hidrocele.
Aquele que padece hidrocele.
\section{Hidrocianato}
\begin{itemize}
\item {Grp. gram.:m.}
\end{itemize}
\begin{itemize}
\item {Utilização:Chím.}
\end{itemize}
\begin{itemize}
\item {Proveniência:(De \textunderscore hidro...\textunderscore  + \textunderscore cianato\textunderscore )}
\end{itemize}
Sal, produzido pela combinação do ácido cianhídrico com as bases.
\section{Hidrociânico}
\begin{itemize}
\item {Grp. gram.:adj.}
\end{itemize}
O mesmo que \textunderscore cianídrico\textunderscore .
\section{Hidrocisto}
\begin{itemize}
\item {Grp. gram.:m.}
\end{itemize}
\begin{itemize}
\item {Proveniência:(De \textunderscore hidro...\textunderscore  + \textunderscore cisto\textunderscore )}
\end{itemize}
Cisto seroso.
\section{Hidrocoriza}
\begin{itemize}
\item {Grp. gram.:f.}
\end{itemize}
Percevejo da água.
\section{Hidrocótila}
\begin{itemize}
\item {Grp. gram.:f.}
\end{itemize}
\begin{itemize}
\item {Proveniência:(Do gr. \textunderscore hudor\textunderscore  + \textunderscore kotule\textunderscore )}
\end{itemize}
Gênero de plantas umbelíferas.
\section{Hidro-dinâmica}
\begin{itemize}
\item {Grp. gram.:f.}
\end{itemize}
\begin{itemize}
\item {Proveniência:(De \textunderscore hidro-dinâmico\textunderscore )}
\end{itemize}
Parte da Hidráulica, que trata do movimento, equilíbrio e pêso dos líquidos.
\section{Hidro-dinâmico}
\begin{itemize}
\item {Grp. gram.:adj.}
\end{itemize}
\begin{itemize}
\item {Proveniência:(De \textunderscore hidro...\textunderscore  + \textunderscore dinâmico\textunderscore )}
\end{itemize}
Relativo ás leis do movimento dos líquidos.
\section{Hidro-dinasta}
\begin{itemize}
\item {Grp. gram.:m.}
\end{itemize}
\begin{itemize}
\item {Proveniência:(Do gr. \textunderscore hudor\textunderscore  + \textunderscore dunastes\textunderscore )}
\end{itemize}
Espécie de cobra.
\section{Hidro-electrico}
\begin{itemize}
\item {Grp. gram.:adj.}
\end{itemize}
Diz-se da corrente eléctrica, obtida com pilhas em água ou noutro líquido.
\section{Hidro-extractor}
\begin{itemize}
\item {Grp. gram.:m.}
\end{itemize}
Máquina das fabricas de lanifícios, para extrair dos estofos a água, por meio da fôrça centrífuga.
\section{Hidrófero}
\begin{itemize}
\item {Grp. gram.:m.}
\end{itemize}
\begin{itemize}
\item {Proveniência:(T. hybr. do gr. \textunderscore hudor\textunderscore  + lat. \textunderscore ferre\textunderscore )}
\end{itemize}
Aparelho que pulveriza as águas mineraes e as difunde sôbre o banhista.
\section{Hidroferrocianato}
\begin{itemize}
\item {Grp. gram.:m.}
\end{itemize}
Medicamento anti-febril.
\section{Hidrófugo}
\begin{itemize}
\item {Grp. gram.:adj.}
\end{itemize}
\begin{itemize}
\item {Proveniência:(T. hybr., do gr. \textunderscore hudor\textunderscore  + \textunderscore fugere\textunderscore )}
\end{itemize}
Diz-se de certas substâncias, especialmente de certos vernizes, que preservam da humidade as paredes, obstando á formação do salitre e á deterioração das pinturas em pedra e gêsso.
\section{Hidrogenação}
\begin{itemize}
\item {Grp. gram.:f.}
\end{itemize}
Acto ou efeito de hidrogenar.
\section{Hidrogenar}
\begin{itemize}
\item {Grp. gram.:v. t.}
\end{itemize}
Combinar com o hidrogênio.
\section{Hidrogenía}
\begin{itemize}
\item {Grp. gram.:f.}
\end{itemize}
\begin{itemize}
\item {Proveniência:(Do gr. \textunderscore hudor\textunderscore  + \textunderscore genes\textunderscore )}
\end{itemize}
Teoria sôbre a formação das massas de água, difundidas sôbre o nosso globo.
\section{Hidrogênio}
\begin{itemize}
\item {Grp. gram.:m.}
\end{itemize}
\begin{itemize}
\item {Proveniência:(Do gr. \textunderscore hudor\textunderscore  + \textunderscore genes\textunderscore )}
\end{itemize}
Corpo simples, gasoso, incolor, cuja combinação com o oxigênio produz a água.
Gás líquido, destinado á iluminação e composto de álcool e essência de terebentina.
\section{Hidrogeologia}
\begin{itemize}
\item {Grp. gram.:f.}
\end{itemize}
\begin{itemize}
\item {Proveniência:(De \textunderscore hidro...\textunderscore  + \textunderscore geologia\textunderscore )}
\end{itemize}
Estado das águas espalhadas á superfície da Terra.
\section{Hidrognomonia}
\begin{itemize}
\item {Grp. gram.:f.}
\end{itemize}
\begin{itemize}
\item {Proveniência:(Do gr. \textunderscore hudor\textunderscore  + \textunderscore gnomon\textunderscore )}
\end{itemize}
Arte de descobrir as nascentes da água.
\section{Hidrognosia}
\begin{itemize}
\item {Grp. gram.:f.}
\end{itemize}
\begin{itemize}
\item {Proveniência:(Do gr. \textunderscore hudor\textunderscore  + \textunderscore gnosis\textunderscore )}
\end{itemize}
O mesmo que \textunderscore hidrogeologia\textunderscore .
\section{Hidrógono}
\begin{itemize}
\item {Grp. gram.:adj.}
\end{itemize}
\begin{itemize}
\item {Utilização:Geol.}
\end{itemize}
\begin{itemize}
\item {Proveniência:(Do gr. \textunderscore hudor\textunderscore  + \textunderscore gonos\textunderscore )}
\end{itemize}
Diz-se das rochas formadas no seio das águas ou por intervenção da água.
\section{Hidrografia}
\begin{itemize}
\item {Grp. gram.:f.}
\end{itemize}
\begin{itemize}
\item {Proveniência:(De \textunderscore hidrógrafo\textunderscore )}
\end{itemize}
Descripção da parte líquida do globo terrestre.
Ciência, que ensina a conhecer os mares.
\section{Hidrográfico}
\begin{itemize}
\item {Grp. gram.:adj.}
\end{itemize}
Relativo á hidrografia.
\section{Hidrógrafo}
\begin{itemize}
\item {Grp. gram.:m.}
\end{itemize}
\begin{itemize}
\item {Proveniência:(Do gr. \textunderscore hudor\textunderscore  + \textunderscore graphein\textunderscore )}
\end{itemize}
Aquelle que trata de hidrografia.
\section{Hidroides}
\begin{itemize}
\item {Grp. gram.:m. pl.}
\end{itemize}
\begin{itemize}
\item {Proveniência:(Do gr. \textunderscore hudra\textunderscore  + \textunderscore eidos\textunderscore )}
\end{itemize}
Animaes aquáticos, semelhantes ao pòlipo hidra.
\section{Hidrol}
\begin{itemize}
\item {Grp. gram.:m.}
\end{itemize}
\begin{itemize}
\item {Proveniência:(De \textunderscore hidro...\textunderscore , tomando \textunderscore álcool\textunderscore  por modêlo)}
\end{itemize}
Palavra, que foi proposta para designar genericamente ás aguas mineraes.
\section{Hidrolato}
\begin{itemize}
\item {Grp. gram.:m.}
\end{itemize}
\begin{itemize}
\item {Proveniência:(Do rad. do gr. \textunderscore hudor\textunderscore )}
\end{itemize}
Líquido incolor, obtido pela destilação da água com plantas ou outras substâncias aromáticas.
\section{Hidrólatra}
\begin{itemize}
\item {Grp. gram.:m.}
\end{itemize}
\begin{itemize}
\item {Proveniência:(Do gr. \textunderscore hudor\textunderscore  + \textunderscore latreia\textunderscore )}
\end{itemize}
Aquelle que adora a água.
\section{Hidrolatria}
\begin{itemize}
\item {Grp. gram.:f.}
\end{itemize}
\begin{itemize}
\item {Proveniência:(Do gr. \textunderscore hudor\textunderscore  + \textunderscore latreia\textunderscore )}
\end{itemize}
Culto da água.
\section{Hidroleáceas}
\begin{itemize}
\item {Grp. gram.:f. pl.}
\end{itemize}
Família de plantas annuaes, próprias da América tropical, formada por Brown, á custa de alguns gêneros das convolvuláceas de Jussieu.
\section{Hidróleas}
\begin{itemize}
\item {Grp. gram.:f. pl.}
\end{itemize}
(V.hidroleáceas)
\section{Hidrologia}
\begin{itemize}
\item {Grp. gram.:f.}
\end{itemize}
\begin{itemize}
\item {Proveniência:(De \textunderscore hidrólogo\textunderscore )}
\end{itemize}
Parte da Historia Natural, que trata das águas, e das suas propriedades e espécies.
\section{Hidrológico}
\begin{itemize}
\item {Grp. gram.:adj.}
\end{itemize}
Relativo á hidrologia.
\section{Hidrólogo}
\begin{itemize}
\item {Grp. gram.:m.}
\end{itemize}
\begin{itemize}
\item {Proveniência:(Do gr. \textunderscore hudor\textunderscore  + \textunderscore logos\textunderscore )}
\end{itemize}
Aquele que ensina ou sabe hidrologia.
\section{Hidromancia}
\begin{itemize}
\item {Grp. gram.:f.}
\end{itemize}
\begin{itemize}
\item {Proveniência:(Lat. \textunderscore hydromantia\textunderscore )}
\end{itemize}
Arte de adivinhar por meio da água.
\section{Hidromania}
\begin{itemize}
\item {Grp. gram.:f.}
\end{itemize}
\begin{itemize}
\item {Proveniência:(Do gr. \textunderscore hudor\textunderscore  + \textunderscore mania\textunderscore )}
\end{itemize}
Sede excessiva.
Delírio, em que o doente mostra tendência para se afogar.
\section{Hidromântico}
\begin{itemize}
\item {Grp. gram.:adj.}
\end{itemize}
Aquele que pratíca a hidromancia.
\section{Hidromecânico}
\begin{itemize}
\item {Grp. gram.:adj.}
\end{itemize}
\begin{itemize}
\item {Proveniência:(De \textunderscore hidro...\textunderscore  + \textunderscore mecanico\textunderscore )}
\end{itemize}
Em que se emprega água como fôrça motriz.
\section{Hidromedicina}
\begin{itemize}
\item {Grp. gram.:f.}
\end{itemize}
O mesmo que \textunderscore hidroterapia\textunderscore .
\section{Hidromedicinal}
\begin{itemize}
\item {Grp. gram.:adj.}
\end{itemize}
Relativo a hidromedicina: \textunderscore estancias hidromedicinaes\textunderscore .
\section{Hidromel}
\begin{itemize}
\item {Grp. gram.:m.}
\end{itemize}
\begin{itemize}
\item {Proveniência:(Lat. \textunderscore hydromeli\textunderscore )}
\end{itemize}
Mistura de água e mel.
\section{Hidrometeóro}
\begin{itemize}
\item {Grp. gram.:m.}
\end{itemize}
\begin{itemize}
\item {Proveniência:(De \textunderscore hidro...\textunderscore  + \textunderscore meteóro\textunderscore )}
\end{itemize}
Meteóro, produzido pela água, em estado de vapor, de líquido ou de gêlo.
Meteóro aquoso.
\section{Hidrometria}
\begin{itemize}
\item {Grp. gram.:f.}
\end{itemize}
\begin{itemize}
\item {Proveniência:(De \textunderscore hidrómetro\textunderscore ^1)}
\end{itemize}
Ciência, que ensina a medir a velocidade e fôrça dos líquidos, especialmente da água.
\section{Hidrométrico}
\begin{itemize}
\item {Grp. gram.:adj.}
\end{itemize}
Relativo á hidrometria. Cf. \textunderscore Techn. Rur.\textunderscore , 51.
\section{Hidrómetro}
\begin{itemize}
\item {Grp. gram.:m.}
\end{itemize}
\begin{itemize}
\item {Proveniência:(Do gr. \textunderscore hudor\textunderscore  + \textunderscore metron\textunderscore )}
\end{itemize}
Instrumento, para as aplicações da hidrometria.
\section{Hidrófana}
\begin{itemize}
\item {Grp. gram.:f.}
\end{itemize}
\begin{itemize}
\item {Proveniência:(De \textunderscore hidrófano\textunderscore )}
\end{itemize}
Pedra silicosa, que é translúcida na água.
\section{Hidrófano}
\begin{itemize}
\item {Grp. gram.:adj.}
\end{itemize}
\begin{itemize}
\item {Proveniência:(Do gr. \textunderscore hudor\textunderscore  + \textunderscore phainein\textunderscore )}
\end{itemize}
Que é translúcido na água.
\section{Hidrófido}
\begin{itemize}
\item {Grp. gram.:m.}
\end{itemize}
\begin{itemize}
\item {Proveniência:(Do gr. \textunderscore hudor\textunderscore  + \textunderscore ophis\textunderscore )}
\end{itemize}
Serpente, que vive na água.
\section{Hidrofíleas}
\begin{itemize}
\item {Grp. gram.:f. pl.}
\end{itemize}
\begin{itemize}
\item {Proveniência:(Do gr. \textunderscore hudor\textunderscore  + \textunderscore phullon\textunderscore )}
\end{itemize}
Tríbo de plantas borragíneas.
\section{Hidrófilo}
\begin{itemize}
\item {Grp. gram.:m.}
\end{itemize}
\begin{itemize}
\item {Grp. gram.:Adj.}
\end{itemize}
\begin{itemize}
\item {Utilização:Pharm.}
\end{itemize}
\begin{itemize}
\item {Proveniência:(Do gr. \textunderscore hudor\textunderscore  + \textunderscore philos\textunderscore )}
\end{itemize}
Gênero de insectos coleópteros pentâmeros.
Diz-se do algodão simples ou desinfectado.
\section{Hidrofisocele}
\begin{itemize}
\item {Grp. gram.:m.}
\end{itemize}
\begin{itemize}
\item {Proveniência:(Do gr. \textunderscore hudor\textunderscore  + \textunderscore phusa\textunderscore  + \textunderscore kele\textunderscore )}
\end{itemize}
Hérnia, que contém água e gás.
\section{Hidrofobia}
\begin{itemize}
\item {Grp. gram.:f.}
\end{itemize}
\begin{itemize}
\item {Proveniência:(De \textunderscore hidrófobo\textunderscore )}
\end{itemize}
Horror aos líquidos.
Enfermidade, caracterizada pelo horror aos líquidos.
Abusivamente, o mesmo que \textunderscore raiva\textunderscore , doença.
\section{Hidrofóbico}
\begin{itemize}
\item {Grp. gram.:adj.}
\end{itemize}
Relativo á hidrofobia.
\section{Hidrófobo}
\begin{itemize}
\item {Grp. gram.:m.  e  adj.}
\end{itemize}
\begin{itemize}
\item {Utilização:Ext.}
\end{itemize}
\begin{itemize}
\item {Proveniência:(Lat. \textunderscore hydrophobus\textunderscore )}
\end{itemize}
O que tem horror aos líquidos.
Aquele que é atacado de raiva, doença.
\section{Hidrofórias}
\begin{itemize}
\item {Grp. gram.:f. pl.}
\end{itemize}
\begin{itemize}
\item {Proveniência:(Do gr. \textunderscore hudor\textunderscore  + \textunderscore phorein\textunderscore )}
\end{itemize}
Festas gregas em honra de Apolo e em memória dos que tinham perecido no dilúvio de Deucalião.
\section{Hidróforo}
\begin{itemize}
\item {Grp. gram.:adj.}
\end{itemize}
\begin{itemize}
\item {Proveniência:(Do gr. \textunderscore hudor\textunderscore  + \textunderscore phoros\textunderscore )}
\end{itemize}
Que conduz água ou serosidade nos corpos organizados.
\section{Hidrofosfato}
\begin{itemize}
\item {Grp. gram.:m.}
\end{itemize}
\begin{itemize}
\item {Proveniência:(De \textunderscore hidro...\textunderscore  + \textunderscore fosfato\textunderscore )}
\end{itemize}
Fosfato, combinado com água.
\section{Hidrofráctico}
\begin{itemize}
\item {Grp. gram.:adj.}
\end{itemize}
\begin{itemize}
\item {Proveniência:(Do gr. \textunderscore hudor\textunderscore  + \textunderscore phraktikos\textunderscore )}
\end{itemize}
Impermeável á água.
\section{Hidroftalmia}
\begin{itemize}
\item {Grp. gram.:f.}
\end{itemize}
É termo, usado indevidamente pelos diccionários, em vez de \textunderscore hidroftalmo\textunderscore .
\section{Hidroftalmo}
\begin{itemize}
\item {Grp. gram.:m.}
\end{itemize}
\begin{itemize}
\item {Proveniência:(Do gr. \textunderscore hudor\textunderscore  + \textunderscore ophthalmos\textunderscore )}
\end{itemize}
Dilatação congênita do globo ocular.
\section{Hidrómetro}
\begin{itemize}
\item {Grp. gram.:m.}
\end{itemize}
\begin{itemize}
\item {Proveniência:(Do gr. \textunderscore hudor\textunderscore  + \textunderscore metra\textunderscore )}
\end{itemize}
Hidropisia do útero.
\section{Hidrómetros}
\begin{itemize}
\item {Grp. gram.:m. pl.}
\end{itemize}
\begin{itemize}
\item {Proveniência:(Do gr. \textunderscore hudor\textunderscore  + \textunderscore metron\textunderscore )}
\end{itemize}
Gênero de insectos hemípteros, que habíta as águas da Europa.
\section{Hidromineral}
\begin{itemize}
\item {Grp. gram.:adj.}
\end{itemize}
\begin{itemize}
\item {Proveniência:(De \textunderscore hidro...\textunderscore  + \textunderscore mineral\textunderscore )}
\end{itemize}
Relativo a água mineral.
\section{Hidrómia}
\begin{itemize}
\item {Grp. gram.:f.}
\end{itemize}
\begin{itemize}
\item {Proveniência:(Do gr. \textunderscore hudor\textunderscore  + \textunderscore muia\textunderscore )}
\end{itemize}
Gênero de insectos dípteros, cujas larvas vivem na água.
\section{Hidronefrose}
\begin{itemize}
\item {Grp. gram.:f.}
\end{itemize}
\begin{itemize}
\item {Utilização:Med.}
\end{itemize}
\begin{itemize}
\item {Proveniência:(Do gr. \textunderscore hudor\textunderscore  + \textunderscore nephros\textunderscore )}
\end{itemize}
Hidropisia dos rins.
\section{Hidrônfalo}
\begin{itemize}
\item {Grp. gram.:m.}
\end{itemize}
\begin{itemize}
\item {Proveniência:(Do gr. \textunderscore hudor\textunderscore  + \textunderscore omphalos\textunderscore )}
\end{itemize}
Humor aquoso umbilical.
\section{Hidrooforia}
\begin{itemize}
\item {Grp. gram.:f.}
\end{itemize}
\begin{itemize}
\item {Proveniência:(Do gr. \textunderscore hudor\textunderscore  + \textunderscore oon\textunderscore  + \textunderscore phoros\textunderscore )}
\end{itemize}
O mesmo que \textunderscore hidroário\textunderscore .
\section{Hidropata}
\begin{itemize}
\item {Grp. gram.:m.}
\end{itemize}
\begin{itemize}
\item {Proveniência:(Do gr. \textunderscore hudor\textunderscore  + \textunderscore pathos\textunderscore )}
\end{itemize}
Aquele que trata de doentes pela hidropatia.
\section{Hidropatia}
\begin{itemize}
\item {Grp. gram.:f.}
\end{itemize}
\begin{itemize}
\item {Proveniência:(De \textunderscore hidropata\textunderscore )}
\end{itemize}
Tratamento de certas moléstias por meio da água.
\section{Hidropedese}
\begin{itemize}
\item {Grp. gram.:f.}
\end{itemize}
\begin{itemize}
\item {Utilização:Med.}
\end{itemize}
\begin{itemize}
\item {Proveniência:(Do gr. \textunderscore hudor\textunderscore  + \textunderscore pedesis\textunderscore )}
\end{itemize}
Suor excessivo.
\section{Hidropericárdio}
\begin{itemize}
\item {Grp. gram.:m.}
\end{itemize}
\begin{itemize}
\item {Proveniência:(De \textunderscore hidro...\textunderscore  + \textunderscore pericárdio\textunderscore )}
\end{itemize}
Engorgitamento seroso do invólucro do coração.
\section{Hydrastina}
\begin{itemize}
\item {Grp. gram.:f.}
\end{itemize}
Alcaloide da ranunculácea \textunderscore hydrastis canadensis\textunderscore .
\section{Hydrastinina}
\begin{itemize}
\item {Grp. gram.:f.}
\end{itemize}
Medicamento anti-febril.
\section{Hydratação}
\begin{itemize}
\item {Grp. gram.:f.}
\end{itemize}
Acto de hydratar. Cf. \textunderscore Techn. Rur.\textunderscore , 113 e 347.
\section{Hydratar}
\begin{itemize}
\item {Grp. gram.:v. t.}
\end{itemize}
Dar o carácter de hydrato a.
\section{Hydratável}
\begin{itemize}
\item {Grp. gram.:adj.}
\end{itemize}
Que se póde hydratar.
\section{Hydrático}
\begin{itemize}
\item {Grp. gram.:adj.}
\end{itemize}
Que tem alguns caracteres de hydrato.
\section{Hydrato}
\begin{itemize}
\item {Grp. gram.:m.}
\end{itemize}
\begin{itemize}
\item {Proveniência:(Do gr. \textunderscore hudor\textunderscore )}
\end{itemize}
Combinação de um óxido metállico com a água, desempenhando esta o papel de ácido.
\section{Hydráulica}
\begin{itemize}
\item {Grp. gram.:f.}
\end{itemize}
\begin{itemize}
\item {Proveniência:(De \textunderscore hydráulico\textunderscore )}
\end{itemize}
Sciência ou arte, que tem por objecto a direcção e emprêgo das águas.
\section{Hydraulicidade}
\begin{itemize}
\item {Grp. gram.:f.}
\end{itemize}
\begin{itemize}
\item {Proveniência:(De \textunderscore hidráulico\textunderscore )}
\end{itemize}
Qualidade das argamassas hydráulicas.
\section{Hydráulico}
\begin{itemize}
\item {Grp. gram.:adj.}
\end{itemize}
\begin{itemize}
\item {Grp. gram.:M.}
\end{itemize}
\begin{itemize}
\item {Proveniência:(Do gr. \textunderscore hudor\textunderscore  + \textunderscore aulos\textunderscore )}
\end{itemize}
Relativo ao movimento das águas nos canos ou canaes e, em geral, a qualquer movimento dos líquidos.
Relativo á hydráulica.
Que endurece na água.
Engenheiro de obras hydráulicas.
Aquelle que é versado em hydráulica.
\section{Hydraulo}
\begin{itemize}
\item {Grp. gram.:m.}
\end{itemize}
\begin{itemize}
\item {Proveniência:(Gr. \textunderscore hudraulos\textunderscore )}
\end{itemize}
Instrumento músico, usado pelos Gregos e que funccionava com o auxilio de água.--(O instrumento é hoje desconhecido, mas são talvez reflexo delle os \textunderscore rouxinóis de barro\textunderscore , que os rapazes de Lisbôa enchem de água e tocam, nas noites de Santo António, San-João e San-Pedro)
\section{Hydremia}
\begin{itemize}
\item {Grp. gram.:f.}
\end{itemize}
\begin{itemize}
\item {Utilização:Med.}
\end{itemize}
\begin{itemize}
\item {Proveniência:(Do gr. \textunderscore hudor\textunderscore  + \textunderscore haima\textunderscore )}
\end{itemize}
Predomíneo do plasma sanguíneo sôbre os glóbulos de sangue.
\section{Hydrião}
\begin{itemize}
\item {Grp. gram.:m.}
\end{itemize}
\begin{itemize}
\item {Proveniência:(Gr. \textunderscore hudrion\textunderscore )}
\end{itemize}
Vaso de água lustral, entre os Gregos do Egypto.
Espécie de bilha.
\section{Hydriatria}
\begin{itemize}
\item {Grp. gram.:f.}
\end{itemize}
\begin{itemize}
\item {Utilização:Med.}
\end{itemize}
\begin{itemize}
\item {Proveniência:(Do gr. \textunderscore hudor\textunderscore  + \textunderscore iatreia\textunderscore )}
\end{itemize}
Parte da Therapêutica, que se occupa do emprêgo das águas salgadas, doces e mineraes.
\section{Hydro}
\begin{itemize}
\item {Grp. gram.:m.}
\end{itemize}
\begin{itemize}
\item {Proveniência:(Lat. \textunderscore hydrus\textunderscore )}
\end{itemize}
Cobra de água, hydra.
\section{Hydro...}
\begin{itemize}
\item {Grp. gram.:pref.}
\end{itemize}
\begin{itemize}
\item {Proveniência:(Do gr. \textunderscore hudor\textunderscore )}
\end{itemize}
(designativo de \textunderscore água\textunderscore )
\section{Hydroaéreo}
\begin{itemize}
\item {Grp. gram.:adj.}
\end{itemize}
\begin{itemize}
\item {Utilização:Med.}
\end{itemize}
\begin{itemize}
\item {Proveniência:(De \textunderscore hydro...\textunderscore  + \textunderscore aéreo\textunderscore )}
\end{itemize}
Diz-se do ruído, que denuncia ar e líquidos dentro de uma cavidade orgânica.
\section{Hydroário}
\begin{itemize}
\item {Grp. gram.:m.}
\end{itemize}
\begin{itemize}
\item {Utilização:Med.}
\end{itemize}
\begin{itemize}
\item {Proveniência:(Do gr. \textunderscore hudor\textunderscore  + \textunderscore oarion\textunderscore )}
\end{itemize}
Hydropisia do ovário.
\section{Hydrobatrácios}
\begin{itemize}
\item {Grp. gram.:m. pl.}
\end{itemize}
\begin{itemize}
\item {Proveniência:(De \textunderscore hydro...\textunderscore  + \textunderscore batrácio\textunderscore )}
\end{itemize}
Família de reptis, que vivem habitualmente na água ou em lugares húmidos.
\section{Hydróbio}
\begin{itemize}
\item {Grp. gram.:adj.}
\end{itemize}
\begin{itemize}
\item {Proveniência:(Do gr. \textunderscore hudor\textunderscore  + \textunderscore bios\textunderscore )}
\end{itemize}
Que vive na água.
\section{Hydrobrânchio}
\begin{itemize}
\item {fónica:qui}
\end{itemize}
\begin{itemize}
\item {Grp. gram.:adj.}
\end{itemize}
\begin{itemize}
\item {Proveniência:(De \textunderscore hydro...\textunderscore  + \textunderscore brânchias\textunderscore )}
\end{itemize}
Cujas brânchias são próprias para respirar a água.
\section{Hydrobromato}
\begin{itemize}
\item {Grp. gram.:m.}
\end{itemize}
\begin{itemize}
\item {Utilização:Chím.}
\end{itemize}
Sal, produzido pela combinação do ácido hydrobrómico com as bases.
(Cp. \textunderscore hydrobrómico\textunderscore )
\section{Hydrobrómico}
\begin{itemize}
\item {Grp. gram.:adj.}
\end{itemize}
\begin{itemize}
\item {Proveniência:(De \textunderscore hydrogênico\textunderscore  e \textunderscore bromo\textunderscore )}
\end{itemize}
Diz-se de um ácido, resultante da combinação do hydrogênico com o bromo.
\section{Hydrocarbonato}
\begin{itemize}
\item {Grp. gram.:m.}
\end{itemize}
\begin{itemize}
\item {Proveniência:(De \textunderscore hydro...\textunderscore  + \textunderscore carbonato\textunderscore )}
\end{itemize}
Carbonato, que contém água no estado de combinação chímica.
\section{Hydrocele}
\begin{itemize}
\item {Grp. gram.:m.}
\end{itemize}
\begin{itemize}
\item {Proveniência:(Lat. \textunderscore hydrocele\textunderscore )}
\end{itemize}
Tumor, formado por um acervo de serosidade.
\section{Hydrocélico}
\begin{itemize}
\item {Grp. gram.:adj.}
\end{itemize}
\begin{itemize}
\item {Grp. gram.:M.}
\end{itemize}
\begin{itemize}
\item {Proveniência:(Lat. \textunderscore hydrocelicus\textunderscore )}
\end{itemize}
Relativo ao hydrocele.
Aquelle que padece hydrocele.
\section{Hydrocephalia}
\begin{itemize}
\item {Grp. gram.:f.}
\end{itemize}
\begin{itemize}
\item {Proveniência:(Do gr. \textunderscore hudor\textunderscore  + \textunderscore kephale\textunderscore )}
\end{itemize}
Hydropisia cerebral, chamada vulgarmente \textunderscore cabeça de água\textunderscore .
\section{Hydrocéphalo}
\begin{itemize}
\item {Grp. gram.:m.}
\end{itemize}
\begin{itemize}
\item {Grp. gram.:Adj.}
\end{itemize}
O mesmo que \textunderscore hydrocephalia\textunderscore .
Aquelle que padece hydrocephalia.
Que soffre hydrocephalia.
\section{Hydrocharídeas}
\begin{itemize}
\item {fónica:ca}
\end{itemize}
\begin{itemize}
\item {Grp. gram.:f. pl.}
\end{itemize}
Ordem das plantas, que têm por typo o hydrocháris.
\section{Hydrocháris}
\begin{itemize}
\item {fónica:cá}
\end{itemize}
\begin{itemize}
\item {Grp. gram.:m.}
\end{itemize}
\begin{itemize}
\item {Proveniência:(Do gr. \textunderscore hudor\textunderscore  + \textunderscore kharis\textunderscore )}
\end{itemize}
Gênero de plantas aquáticas, perennes e levemente medicinaes.
\section{Hydrocisto}
\begin{itemize}
\item {Grp. gram.:m.}
\end{itemize}
\begin{itemize}
\item {Proveniência:(De \textunderscore hydro...\textunderscore  + \textunderscore cisto\textunderscore )}
\end{itemize}
Cysto seroso.
\section{Hydrocoriza}
\begin{itemize}
\item {Grp. gram.:f.}
\end{itemize}
Percevejo da água.
\section{Hydrocótyla}
\begin{itemize}
\item {Grp. gram.:f.}
\end{itemize}
\begin{itemize}
\item {Proveniência:(Do gr. \textunderscore hudor\textunderscore  + \textunderscore kotule\textunderscore )}
\end{itemize}
Gênero de plantas umbellíferas.
\section{Hydrocyanato}
\begin{itemize}
\item {Grp. gram.:m.}
\end{itemize}
\begin{itemize}
\item {Utilização:Chím.}
\end{itemize}
\begin{itemize}
\item {Proveniência:(De \textunderscore hydro...\textunderscore  + \textunderscore cyanato\textunderscore )}
\end{itemize}
Sal, produzido pela combinação do ácido cyanhýdrico com as bases.
\section{Hydrocyânico}
\begin{itemize}
\item {Grp. gram.:adj.}
\end{itemize}
O mesmo que \textunderscore cyanhýdrico\textunderscore .
\section{Hydro-dynâmica}
\begin{itemize}
\item {Grp. gram.:f.}
\end{itemize}
\begin{itemize}
\item {Proveniência:(De \textunderscore hydro-dynâmico\textunderscore )}
\end{itemize}
Parte da Hydráulica, que trata do movimento, equilíbrio e pêso dos líquidos.
\section{Hydro-dynâmico}
\begin{itemize}
\item {Grp. gram.:adj.}
\end{itemize}
\begin{itemize}
\item {Proveniência:(De \textunderscore hydro...\textunderscore  + \textunderscore dynâmico\textunderscore )}
\end{itemize}
Relativo ás leis do movimento dos líquidos.
\section{Hydro-dynasta}
\begin{itemize}
\item {Grp. gram.:m.}
\end{itemize}
\begin{itemize}
\item {Proveniência:(Do gr. \textunderscore hudor\textunderscore  + \textunderscore dunastes\textunderscore )}
\end{itemize}
Espécie de cobra.
\section{Hydro-eléctrico}
\begin{itemize}
\item {Grp. gram.:adj.}
\end{itemize}
Diz-se da corrente eléctrica, obtida com pilhas em água ou noutro líquido.
\section{Hydro-extractor}
\begin{itemize}
\item {Grp. gram.:m.}
\end{itemize}
Máquina das fabricas de lanifícios, para extrahir dos estofos a água, por meio da fôrça centrífuga.
\section{Hydrófero}
\begin{itemize}
\item {Grp. gram.:m.}
\end{itemize}
\begin{itemize}
\item {Proveniência:(T. hybr. do gr. \textunderscore hudor\textunderscore  + lat. \textunderscore ferre\textunderscore )}
\end{itemize}
Apparelho que pulveriza as águas mineraes e as diffunde sôbre o banhista.
\section{Hydroferrocyanato}
\begin{itemize}
\item {Grp. gram.:m.}
\end{itemize}
Medicamento anti-febril.
\section{Hydrófugo}
\begin{itemize}
\item {Grp. gram.:adj.}
\end{itemize}
\begin{itemize}
\item {Proveniência:(T. hybr., do gr. \textunderscore hudor\textunderscore  + \textunderscore fugere\textunderscore )}
\end{itemize}
Diz-se de certas substâncias, especialmente de certos vernizes, que preservam da humidade as paredes, obstando á formação do salitre e á deterioração das pinturas em pedra e gêsso.
\section{Hydrogenação}
\begin{itemize}
\item {Grp. gram.:f.}
\end{itemize}
Acto ou effeito de hydrogenar.
\section{Hydrogenar}
\begin{itemize}
\item {Grp. gram.:v. t.}
\end{itemize}
Combinar com o hydrogênio.
\section{Hydrogenía}
\begin{itemize}
\item {Grp. gram.:f.}
\end{itemize}
\begin{itemize}
\item {Proveniência:(Do gr. \textunderscore hudor\textunderscore  + \textunderscore genes\textunderscore )}
\end{itemize}
Theoria sôbre a formação das massas de água, diffundidas sôbre o nosso globo.
\section{Hydrogênio}
\begin{itemize}
\item {Grp. gram.:m.}
\end{itemize}
\begin{itemize}
\item {Proveniência:(Do gr. \textunderscore hudor\textunderscore  + \textunderscore genes\textunderscore )}
\end{itemize}
Corpo simples, gasoso, incolor, cuja combinação com o oxygênio produz a água.
Gás líquido, destinado á illuminação e composto de álcool e essência de terebenthina.
\section{Hydrogeologia}
\begin{itemize}
\item {Grp. gram.:f.}
\end{itemize}
\begin{itemize}
\item {Proveniência:(De \textunderscore hydro...\textunderscore  + \textunderscore geologia\textunderscore )}
\end{itemize}
Estado das águas espalhadas á superfície da Terra.
\section{Hydrognomonia}
\begin{itemize}
\item {Grp. gram.:f.}
\end{itemize}
\begin{itemize}
\item {Proveniência:(Do gr. \textunderscore hudor\textunderscore  + \textunderscore gnomon\textunderscore )}
\end{itemize}
Arte de descobrir as nascentes da água.
\section{Hydrognosia}
\begin{itemize}
\item {Grp. gram.:f.}
\end{itemize}
\begin{itemize}
\item {Proveniência:(Do gr. \textunderscore hudor\textunderscore  + \textunderscore gnosis\textunderscore )}
\end{itemize}
O mesmo que \textunderscore hydrogeologia\textunderscore .
\section{Hydrógono}
\begin{itemize}
\item {Grp. gram.:adj.}
\end{itemize}
\begin{itemize}
\item {Utilização:Geol.}
\end{itemize}
\begin{itemize}
\item {Proveniência:(Do gr. \textunderscore hudor\textunderscore  + \textunderscore gonos\textunderscore )}
\end{itemize}
Diz-se das rochas formadas no seio das águas ou por intervenção da água.
\section{Hydrographia}
\begin{itemize}
\item {Grp. gram.:f.}
\end{itemize}
\begin{itemize}
\item {Proveniência:(De \textunderscore hydrógrapho\textunderscore )}
\end{itemize}
Descripção da parte líquida do globo terrestre.
Sciência, que ensina a conhecer os mares.
\section{Hydrográphico}
\begin{itemize}
\item {Grp. gram.:adj.}
\end{itemize}
Relativo á hydrographia.
\section{Hydrógrapho}
\begin{itemize}
\item {Grp. gram.:m.}
\end{itemize}
\begin{itemize}
\item {Proveniência:(Do gr. \textunderscore hudor\textunderscore  + \textunderscore graphein\textunderscore )}
\end{itemize}
Aquelle que trata de hydrographia.
\section{Hydroides}
\begin{itemize}
\item {Grp. gram.:m. pl.}
\end{itemize}
\begin{itemize}
\item {Proveniência:(Do gr. \textunderscore hudra\textunderscore  + \textunderscore eidos\textunderscore )}
\end{itemize}
Animaes aquáticos, semelhantes ao pòlypo hydra.
\section{Hydrol}
\begin{itemize}
\item {Grp. gram.:m.}
\end{itemize}
\begin{itemize}
\item {Proveniência:(De \textunderscore hydro...\textunderscore , tomando \textunderscore álcool\textunderscore  por modêlo)}
\end{itemize}
Palavra, que foi proposta para designar genericamente ás aguas mineraes.
\section{Hydrolato}
\begin{itemize}
\item {Grp. gram.:m.}
\end{itemize}
\begin{itemize}
\item {Proveniência:(Do rad. do gr. \textunderscore hudor\textunderscore )}
\end{itemize}
Líquido incolor, obtido pela destillação da água com plantas ou outras substâncias aromáticas.
\section{Hydrólatra}
\begin{itemize}
\item {Grp. gram.:m.}
\end{itemize}
\begin{itemize}
\item {Proveniência:(Do gr. \textunderscore hudor\textunderscore  + \textunderscore latreia\textunderscore )}
\end{itemize}
Aquelle que adora a água.
\section{Hydrolatria}
\begin{itemize}
\item {Grp. gram.:f.}
\end{itemize}
\begin{itemize}
\item {Proveniência:(Do gr. \textunderscore hudor\textunderscore  + \textunderscore latreia\textunderscore )}
\end{itemize}
Culto da água.
\section{Hydroleáceas}
\begin{itemize}
\item {Grp. gram.:f. pl.}
\end{itemize}
Família de plantas annuaes, próprias da América tropical, formada por Brown, á custa de alguns gêneros das convolvuláceas de Jussieu.
\section{Hydróleas}
\begin{itemize}
\item {Grp. gram.:f. pl.}
\end{itemize}
(V.hydroleáceas)
\section{Hydrologia}
\begin{itemize}
\item {Grp. gram.:f.}
\end{itemize}
\begin{itemize}
\item {Proveniência:(De \textunderscore hydrólogo\textunderscore )}
\end{itemize}
Parte da Historia Natural, que trata das águas, e das suas propriedades e espécies.
\section{Hydrológico}
\begin{itemize}
\item {Grp. gram.:adj.}
\end{itemize}
Relativo á hydrologia.
\section{Hydrólogo}
\begin{itemize}
\item {Grp. gram.:m.}
\end{itemize}
\begin{itemize}
\item {Proveniência:(Do gr. \textunderscore hudor\textunderscore  + \textunderscore logos\textunderscore )}
\end{itemize}
Aquelle que ensina ou sabe hydrologia.
\section{Hydromancia}
\begin{itemize}
\item {Grp. gram.:f.}
\end{itemize}
\begin{itemize}
\item {Proveniência:(Lat. \textunderscore hydromantia\textunderscore )}
\end{itemize}
Arte de adivinhar por meio da água.
\section{Hydromania}
\begin{itemize}
\item {Grp. gram.:f.}
\end{itemize}
\begin{itemize}
\item {Proveniência:(Do gr. \textunderscore hudor\textunderscore  + \textunderscore mania\textunderscore )}
\end{itemize}
Sede excessiva.
Delírio, em que o doente mostra tendência para se afogar.
\section{Hydromântico}
\begin{itemize}
\item {Grp. gram.:adj.}
\end{itemize}
Aquelle que pratíca a hydromancia.
\section{Hydromecânico}
\begin{itemize}
\item {Grp. gram.:adj.}
\end{itemize}
\begin{itemize}
\item {Proveniência:(De \textunderscore hydro...\textunderscore  + \textunderscore mechanico\textunderscore )}
\end{itemize}
Em que se emprega água como fôrça motriz.
\section{Hydromedicina}
\begin{itemize}
\item {Grp. gram.:f.}
\end{itemize}
O mesmo que \textunderscore hydrotherapia\textunderscore .
\section{Hydromedicinal}
\begin{itemize}
\item {Grp. gram.:adj.}
\end{itemize}
Relativo a hidromedicina: \textunderscore estancias hydromedicinaes\textunderscore .
\section{Hydromedusa}
\begin{itemize}
\item {Grp. gram.:f.}
\end{itemize}
Animal, cuja configuração participa da da hydra e da medusa.
\section{Hydromel}
\begin{itemize}
\item {Grp. gram.:m.}
\end{itemize}
\begin{itemize}
\item {Proveniência:(Lat. \textunderscore hydromeli\textunderscore )}
\end{itemize}
Mistura de água e mel.
\section{Hydrometeóro}
\begin{itemize}
\item {Grp. gram.:m.}
\end{itemize}
\begin{itemize}
\item {Proveniência:(De \textunderscore hydro...\textunderscore  + \textunderscore meteóro\textunderscore )}
\end{itemize}
Meteóro, produzido pela água, em estado de vapor, de líquido ou de gêlo.
Meteóro aquoso.
\section{Hydrometria}
\begin{itemize}
\item {Grp. gram.:f.}
\end{itemize}
\begin{itemize}
\item {Proveniência:(De \textunderscore hydrómetro\textunderscore ^1)}
\end{itemize}
Sciência, que ensina a medir a velocidade e fôrça dos líquidos, especialmente da água.
\section{Hydrométrico}
\begin{itemize}
\item {Grp. gram.:adj.}
\end{itemize}
Relativo á hydrometria. Cf. \textunderscore Techn. Rur.\textunderscore , 51.
\section{Hydrómetro}
\begin{itemize}
\item {Grp. gram.:m.}
\end{itemize}
\begin{itemize}
\item {Proveniência:(Do gr. \textunderscore hudor\textunderscore  + \textunderscore metron\textunderscore )}
\end{itemize}
Instrumento, para as applicações da hydrometria.
\section{Hydrómetro}
\begin{itemize}
\item {Grp. gram.:m.}
\end{itemize}
\begin{itemize}
\item {Proveniência:(Do gr. \textunderscore hudor\textunderscore  + \textunderscore metra\textunderscore )}
\end{itemize}
Hydropisia do útero.
\section{Hydrómetros}
\begin{itemize}
\item {Grp. gram.:m. pl.}
\end{itemize}
\begin{itemize}
\item {Proveniência:(Do gr. \textunderscore hudor\textunderscore  + \textunderscore metron\textunderscore )}
\end{itemize}
Gênero de insectos hemípteros, que habíta as águas da Europa.
\section{Hydromineral}
\begin{itemize}
\item {Grp. gram.:adj.}
\end{itemize}
\begin{itemize}
\item {Proveniência:(De \textunderscore hydro...\textunderscore  + \textunderscore mineral\textunderscore )}
\end{itemize}
Relativo a água mineral.
\section{Hydrômphalo}
\begin{itemize}
\item {Grp. gram.:m.}
\end{itemize}
\begin{itemize}
\item {Proveniência:(Do gr. \textunderscore hudor\textunderscore  + \textunderscore omphalos\textunderscore )}
\end{itemize}
Humor aquoso umbilical.
\section{Hydrómya}
\begin{itemize}
\item {Grp. gram.:f.}
\end{itemize}
\begin{itemize}
\item {Proveniência:(Do gr. \textunderscore hudor\textunderscore  + \textunderscore muia\textunderscore )}
\end{itemize}
Gênero de insectos dípteros, cujas larvas vivem na água.
\section{Hydronephrose}
\begin{itemize}
\item {Grp. gram.:f.}
\end{itemize}
\begin{itemize}
\item {Utilização:Med.}
\end{itemize}
\begin{itemize}
\item {Proveniência:(Do gr. \textunderscore hudor\textunderscore  + \textunderscore nephros\textunderscore )}
\end{itemize}
Hydropisia dos rins.
\section{Hydroophoria}
\begin{itemize}
\item {Grp. gram.:f.}
\end{itemize}
\begin{itemize}
\item {Proveniência:(Do gr. \textunderscore hudor\textunderscore  + \textunderscore oon\textunderscore  + \textunderscore phoros\textunderscore )}
\end{itemize}
O mesmo que \textunderscore hydroário\textunderscore .
\section{Hydropatha}
\begin{itemize}
\item {Grp. gram.:m.}
\end{itemize}
\begin{itemize}
\item {Proveniência:(Do gr. \textunderscore hudor\textunderscore  + \textunderscore pathos\textunderscore )}
\end{itemize}
Aquelle que trata de doentes pela hydropathia.
\section{Hydropathia}
\begin{itemize}
\item {Grp. gram.:f.}
\end{itemize}
\begin{itemize}
\item {Proveniência:(De \textunderscore hydropatha\textunderscore )}
\end{itemize}
Tratamento de certas moléstias por meio da água.
\section{Hydropedese}
\begin{itemize}
\item {Grp. gram.:f.}
\end{itemize}
\begin{itemize}
\item {Utilização:Med.}
\end{itemize}
\begin{itemize}
\item {Proveniência:(Do gr. \textunderscore hudor\textunderscore  + \textunderscore pedesis\textunderscore )}
\end{itemize}
Suor excessivo.
\section{Hydropericárdio}
\begin{itemize}
\item {Grp. gram.:m.}
\end{itemize}
\begin{itemize}
\item {Proveniência:(De \textunderscore hydro...\textunderscore  + \textunderscore pericárdio\textunderscore )}
\end{itemize}
Engorgitamento seroso do invólucro do coração.
\section{Hydróphana}
\begin{itemize}
\item {Grp. gram.:f.}
\end{itemize}
\begin{itemize}
\item {Proveniência:(De \textunderscore hydróphano\textunderscore )}
\end{itemize}
Pedra silicosa, que é translúcida na água.
\section{Hydróphano}
\begin{itemize}
\item {Grp. gram.:adj.}
\end{itemize}
\begin{itemize}
\item {Proveniência:(Do gr. \textunderscore hudor\textunderscore  + \textunderscore phainein\textunderscore )}
\end{itemize}
Que é translúcido na água.
\section{Hydróphido}
\begin{itemize}
\item {Grp. gram.:m.}
\end{itemize}
\begin{itemize}
\item {Proveniência:(Do gr. \textunderscore hudor\textunderscore  + \textunderscore ophis\textunderscore )}
\end{itemize}
Serpente, que vive na água.
\section{Hydróphilo}
\begin{itemize}
\item {Grp. gram.:m.}
\end{itemize}
\begin{itemize}
\item {Grp. gram.:Adj.}
\end{itemize}
\begin{itemize}
\item {Utilização:Pharm.}
\end{itemize}
\begin{itemize}
\item {Proveniência:(Do gr. \textunderscore hudor\textunderscore  + \textunderscore philos\textunderscore )}
\end{itemize}
Gênero de insectos coleópteros pentâmeros.
Diz-se do algodão simples ou desinfectado.
\section{Hydrophobia}
\begin{itemize}
\item {Grp. gram.:f.}
\end{itemize}
\begin{itemize}
\item {Proveniência:(De \textunderscore hydróphobo\textunderscore )}
\end{itemize}
Horror aos líquidos.
Enfermidade, caracterizada pelo horror aos líquidos.
Abusivamente, o mesmo que \textunderscore raiva\textunderscore , doença.
\section{Hydrophóbico}
\begin{itemize}
\item {Grp. gram.:adj.}
\end{itemize}
Relativo á hidrophobia.
\section{Hydróphobo}
\begin{itemize}
\item {Grp. gram.:m.  e  adj.}
\end{itemize}
\begin{itemize}
\item {Utilização:Ext.}
\end{itemize}
\begin{itemize}
\item {Proveniência:(Lat. \textunderscore hydrophobus\textunderscore )}
\end{itemize}
O que tem horror aos líquidos.
Aquelle que é atacado de raiva, doença.
\section{Hydrophórias}
\begin{itemize}
\item {Grp. gram.:f. pl.}
\end{itemize}
\begin{itemize}
\item {Proveniência:(Do gr. \textunderscore hudor\textunderscore  + \textunderscore phorein\textunderscore )}
\end{itemize}
Festas gregas em honra de Apollo e em memória dos que tinham perecido no dilúvio de Deucalião.
\section{Hydróphoro}
\begin{itemize}
\item {Grp. gram.:adj.}
\end{itemize}
\begin{itemize}
\item {Proveniência:(Do gr. \textunderscore hudor\textunderscore  + \textunderscore phoros\textunderscore )}
\end{itemize}
Que conduz água ou serosidade nos corpos organizados.
\section{Hydrophosphato}
\begin{itemize}
\item {Grp. gram.:m.}
\end{itemize}
\begin{itemize}
\item {Proveniência:(De \textunderscore hydro...\textunderscore  + \textunderscore phosphato\textunderscore )}
\end{itemize}
Phosphato, combinado com água.
\section{Hydrophráctico}
\begin{itemize}
\item {Grp. gram.:adj.}
\end{itemize}
\begin{itemize}
\item {Proveniência:(Do gr. \textunderscore hudor\textunderscore  + \textunderscore phraktikos\textunderscore )}
\end{itemize}
Impermeável á água.
\section{Hydrophthalmia}
\begin{itemize}
\item {Grp. gram.:f.}
\end{itemize}
É termo, usado indevidamente pelos diccionários, em vez de \textunderscore hydrophthalmo\textunderscore .
\section{Hydrophthalmo}
\begin{itemize}
\item {Grp. gram.:m.}
\end{itemize}
\begin{itemize}
\item {Proveniência:(Do gr. \textunderscore hudor\textunderscore  + \textunderscore ophthalmos\textunderscore )}
\end{itemize}
Dilatação congênita do globo ocular.
\section{Hydrophýlleas}
\begin{itemize}
\item {Grp. gram.:f. pl.}
\end{itemize}
\begin{itemize}
\item {Proveniência:(Do gr. \textunderscore hudor\textunderscore  + \textunderscore phullon\textunderscore )}
\end{itemize}
Tríbo de plantas borragíneas.
\section{Hydrophysocele}
\begin{itemize}
\item {Grp. gram.:m.}
\end{itemize}
\begin{itemize}
\item {Proveniência:(Do gr. \textunderscore hudor\textunderscore  + \textunderscore phusa\textunderscore  + \textunderscore kele\textunderscore )}
\end{itemize}
Hérnia, que contém água e gás.
\section{Hidrófitas}
\begin{itemize}
\item {Grp. gram.:f. pl.}
\end{itemize}
\begin{itemize}
\item {Proveniência:(De \textunderscore hidrófito\textunderscore )}
\end{itemize}
Classe de plantas, que vegetam na água ou em terreno alagadiço.
\section{Hidrófito}
\begin{itemize}
\item {Grp. gram.:m.}
\end{itemize}
\begin{itemize}
\item {Grp. gram.:Adj.}
\end{itemize}
\begin{itemize}
\item {Proveniência:(Do gr. \textunderscore hudor\textunderscore  + \textunderscore phuton\textunderscore )}
\end{itemize}
Qualquer planta, que vive na água.
Que vive na água, (falando-se de plantas).
\section{Hidrofitografia}
\begin{itemize}
\item {Grp. gram.:f.}
\end{itemize}
\begin{itemize}
\item {Proveniência:(Do gr. \textunderscore hudor\textunderscore  + \textunderscore phuton\textunderscore  + \textunderscore graphein\textunderscore )}
\end{itemize}
Descripção científica das hidrófitas.
\section{Hidrofitologia}
\begin{itemize}
\item {Grp. gram.:f.}
\end{itemize}
\begin{itemize}
\item {Proveniência:(Do gr. \textunderscore hudor\textunderscore  + \textunderscore phuton\textunderscore  + \textunderscore logos\textunderscore )}
\end{itemize}
Parte da Botânica, que trata das hidrófitas.
\section{Hidrofitológico}
\begin{itemize}
\item {Grp. gram.:adj.}
\end{itemize}
Relativo á hidrofitologia.
\section{Hidrópico}
\begin{itemize}
\item {Grp. gram.:m.  e  adj.}
\end{itemize}
\begin{itemize}
\item {Proveniência:(Lat. \textunderscore hydropicus\textunderscore )}
\end{itemize}
O que tem hidropisia.
\section{Hidropirética}
\begin{itemize}
\item {Grp. gram.:adj. f.}
\end{itemize}
\begin{itemize}
\item {Proveniência:(Do gr. \textunderscore hudor\textunderscore  + \textunderscore pur\textunderscore )}
\end{itemize}
Diz-se de uma febre maligna, acompanhada de suores.
\section{Hidropírico}
\begin{itemize}
\item {Grp. gram.:adj.}
\end{itemize}
\begin{itemize}
\item {Proveniência:(Do gr. \textunderscore hudor\textunderscore  + \textunderscore pur\textunderscore )}
\end{itemize}
Diz-se dos vulcões, que lançam fogo e água.
\section{Hidropisia}
\begin{itemize}
\item {Grp. gram.:f.}
\end{itemize}
\begin{itemize}
\item {Proveniência:(Do lat. \textunderscore hydropisis\textunderscore )}
\end{itemize}
Acumulação de serosidade no tecido celular ou numa cavidade do corpo.
\section{Hidropisina}
\begin{itemize}
\item {Grp. gram.:f.}
\end{itemize}
\begin{itemize}
\item {Proveniência:(De \textunderscore hidropisia\textunderscore )}
\end{itemize}
Substância orgânica, que se encontra na serosidade normal de certas membranas.
\section{Hidropneumática}
\begin{itemize}
\item {Grp. gram.:adj. f.}
\end{itemize}
\begin{itemize}
\item {Proveniência:(De \textunderscore hidro...\textunderscore  + \textunderscore pneumático\textunderscore )}
\end{itemize}
Diz-se de uma máquina, inventada há poucos anos em Lisbôa, para rarefazer o vácuo em determinado espaço.
\section{Hidrópota}
\begin{itemize}
\item {Grp. gram.:m.}
\end{itemize}
\begin{itemize}
\item {Proveniência:(Do gr. \textunderscore hudor\textunderscore  + \textunderscore potes\textunderscore )}
\end{itemize}
Aquele que não bebe senão água.
\section{Hidropúlvis}
\begin{itemize}
\item {Grp. gram.:m.}
\end{itemize}
\begin{itemize}
\item {Proveniência:(De \textunderscore hidro...\textunderscore  + lat. \textunderscore pulvis\textunderscore , pó)}
\end{itemize}
Aparelho, que espalha a água sôbre as plantas em gotas finas e intensas.
\section{Hidroquinone}
\begin{itemize}
\item {Grp. gram.:m.}
\end{itemize}
\begin{itemize}
\item {Utilização:Phot.}
\end{itemize}
Producto químico, empregado como revelador enérgico, cuja fórmula é 3C^{6}H^{6}O^{2}CO^{2}.
\section{Hidrorragia}
\begin{itemize}
\item {Grp. gram.:f.}
\end{itemize}
\begin{itemize}
\item {Utilização:Med.}
\end{itemize}
\begin{itemize}
\item {Proveniência:(Do gr. \textunderscore hudor\textunderscore  + \textunderscore rhagnumi\textunderscore )}
\end{itemize}
Abundante derramamento de águas, como o que precede o parto.
\section{Hidrorreia}
\begin{itemize}
\item {Grp. gram.:f.}
\end{itemize}
\begin{itemize}
\item {Utilização:Med.}
\end{itemize}
\begin{itemize}
\item {Proveniência:(Do gr. \textunderscore hudor\textunderscore  + \textunderscore rhein\textunderscore )}
\end{itemize}
Derramamento lento e crónico de um líquido aquoso.
\section{Hidroscopia}
\begin{itemize}
\item {Grp. gram.:f.}
\end{itemize}
Arte de procurar fontes ou águas subterrâneas.
Hidromancia.
(Cp. \textunderscore hidróscopo\textunderscore )
\section{Hidróscopo}
\begin{itemize}
\item {Grp. gram.:m.}
\end{itemize}
\begin{itemize}
\item {Proveniência:(Gr. \textunderscore hudroskopos\textunderscore )}
\end{itemize}
Aquele que pratíca a hidroscopia.
\section{Hidrosfera}
\begin{itemize}
\item {Grp. gram.:f.}
\end{itemize}
\begin{itemize}
\item {Proveniência:(Do gr. \textunderscore hudros\textunderscore  + \textunderscore sphaira\textunderscore )}
\end{itemize}
A parte líquida da superfície do globo terrestre.
\section{Hidrosférico}
\begin{itemize}
\item {Grp. gram.:adj.}
\end{itemize}
Relativo á hidrosfera.
\section{Hidrossácaro}
\begin{itemize}
\item {Grp. gram.:m.}
\end{itemize}
\begin{itemize}
\item {Utilização:Pharm.}
\end{itemize}
\begin{itemize}
\item {Proveniência:(Do gr. \textunderscore hudor\textunderscore  + \textunderscore sakkharon\textunderscore )}
\end{itemize}
Água açucarada.
\section{Hidrossilicato}
\begin{itemize}
\item {Grp. gram.:m.}
\end{itemize}
\begin{itemize}
\item {Proveniência:(De \textunderscore hidro...\textunderscore  + \textunderscore silicato\textunderscore )}
\end{itemize}
Silicato, que contém água em combinação.
\section{Hidrossilicoso}
\begin{itemize}
\item {Grp. gram.:adj.}
\end{itemize}
\begin{itemize}
\item {Proveniência:(De \textunderscore hidro...\textunderscore  + \textunderscore silicoso\textunderscore )}
\end{itemize}
Que contém água e sílica.
\section{Hidrostática}
\begin{itemize}
\item {Grp. gram.:f.}
\end{itemize}
\begin{itemize}
\item {Proveniência:(De \textunderscore hidrostático\textunderscore )}
\end{itemize}
Parte da Mecânica, que trata do equilíbrio dos líquidos, e da pressão que eles exercem.
\section{Hidrostático}
\begin{itemize}
\item {Grp. gram.:adj.}
\end{itemize}
\begin{itemize}
\item {Proveniência:(De \textunderscore hidro...\textunderscore  + \textunderscore estático\textunderscore )}
\end{itemize}
Relativo á hidrostática.
\section{Hidróstato}
\begin{itemize}
\item {Grp. gram.:m.}
\end{itemize}
\begin{itemize}
\item {Proveniência:(Do rad. de \textunderscore hidrostático\textunderscore )}
\end{itemize}
Instrumento de metal, fluctuante, para pesar corpos.
\section{Hidrotecnia}
\begin{itemize}
\item {Grp. gram.:f.}
\end{itemize}
\begin{itemize}
\item {Proveniência:(Do gr. \textunderscore hudor\textunderscore  + \textunderscore tekhne\textunderscore )}
\end{itemize}
Parte da Mecânica, que trata da distribuição e condução das águas.
\section{Hidrotécnico}
\begin{itemize}
\item {Grp. gram.:adj.}
\end{itemize}
Relativo á hidrotecnia.
\section{Hidroterapeuta}
\begin{itemize}
\item {Grp. gram.:m.}
\end{itemize}
\begin{itemize}
\item {Proveniência:(De \textunderscore hidro...\textunderscore  + \textunderscore terapeuta\textunderscore )}
\end{itemize}
Aquele que exerce a hidroterapêutica.
\section{Hidroterapêutica}
\begin{itemize}
\item {Grp. gram.:f.}
\end{itemize}
O mesmo que \textunderscore hidroterapia\textunderscore .
\section{Hidroterapia}
\begin{itemize}
\item {Grp. gram.:f.}
\end{itemize}
\begin{itemize}
\item {Proveniência:(Do gr. \textunderscore hudor\textunderscore  + \textunderscore therapeia\textunderscore )}
\end{itemize}
Tratamento de doenças, por meio de água fria, em aplicações exteriores.
\section{Hidroterápico}
\begin{itemize}
\item {Grp. gram.:adj.}
\end{itemize}
Relativo á hidroterapia.
\section{Hidrotermal}
\begin{itemize}
\item {Grp. gram.:adj.}
\end{itemize}
O mesmo que \textunderscore hidrotérmico\textunderscore .
\section{Hidrotérmico}
\begin{itemize}
\item {Grp. gram.:adj.}
\end{itemize}
\begin{itemize}
\item {Proveniência:(De \textunderscore hidro...\textunderscore  + \textunderscore térmico\textunderscore )}
\end{itemize}
Relativo á água e ao calor.
\section{Hidrótica}
\begin{itemize}
\item {Grp. gram.:m.}
\end{itemize}
\begin{itemize}
\item {Utilização:Des.}
\end{itemize}
\begin{itemize}
\item {Proveniência:(De \textunderscore hidrótico\textunderscore ^1)}
\end{itemize}
Parte da Medicina, que se ocupa dos suores ou da transpiração. Cf. \textunderscore Diccion. Exegét.\textunderscore 
\section{Hidrótico}
\begin{itemize}
\item {Grp. gram.:adj.}
\end{itemize}
O mesmo que \textunderscore hidragogo\textunderscore .
\section{Hidrotimetria}
\begin{itemize}
\item {Grp. gram.:f.}
\end{itemize}
Processo de aplicar o hidrotímetro.
\section{Hidrotimétrico}
\begin{itemize}
\item {Grp. gram.:adj.}
\end{itemize}
Relativo á hidrotimetria.
\section{Hidrotímetro}
\begin{itemize}
\item {Grp. gram.:m.}
\end{itemize}
Instrumento, para avaliar a existência da água das nascentes e rios e a proporção das matérias que se depositam, sob a influência de uma ebulição prolongada.
(Talvez do gr. \textunderscore hudrotes\textunderscore  + \textunderscore metron\textunderscore )
\section{Hidrotipia}
\begin{itemize}
\item {Grp. gram.:f.}
\end{itemize}
\begin{itemize}
\item {Proveniência:(Do gr. \textunderscore hudor\textunderscore  + \textunderscore tupos\textunderscore )}
\end{itemize}
Processo fotográfico para obter côres.
\section{Hidrotomia}
\begin{itemize}
\item {Grp. gram.:m.}
\end{itemize}
\begin{itemize}
\item {Utilização:Anat.}
\end{itemize}
\begin{itemize}
\item {Proveniência:(Do gr. \textunderscore hudor\textunderscore  + \textunderscore tome\textunderscore )}
\end{itemize}
Processo de inocular água nas artérias, fazendo-as transudar por pressão, para desviar fibras, separar certos órgãos, etc.
\section{Hidrotórax}
\begin{itemize}
\item {Grp. gram.:m.}
\end{itemize}
\begin{itemize}
\item {Proveniência:(De \textunderscore hidro...\textunderscore  + \textunderscore torax\textunderscore )}
\end{itemize}
Doença, caracterizada por opressão do peito, produzida por pleurisia ou por simples pontada.
\section{Hidróxido}
\begin{itemize}
\item {Grp. gram.:m.}
\end{itemize}
\begin{itemize}
\item {Utilização:Chím.}
\end{itemize}
\begin{itemize}
\item {Proveniência:(De \textunderscore hidro...\textunderscore  + \textunderscore óxido\textunderscore )}
\end{itemize}
Combinação da água com um óxido metálico.
\section{Hidrozôa}
\begin{itemize}
\item {Grp. gram.:f.}
\end{itemize}
\begin{itemize}
\item {Proveniência:(Do gr. \textunderscore hudor\textunderscore  + \textunderscore zoon\textunderscore )}
\end{itemize}
Classe de animaes fósseis da série paleozoica.
\section{Hidruria}
\begin{itemize}
\item {Grp. gram.:f.}
\end{itemize}
\begin{itemize}
\item {Proveniência:(Do gr. \textunderscore hudor\textunderscore  + \textunderscore ouron\textunderscore )}
\end{itemize}
Excesso de água nas urinas humanas.
\section{Hidrúrico}
\begin{itemize}
\item {Grp. gram.:adj.}
\end{itemize}
\begin{itemize}
\item {Grp. gram.:M.}
\end{itemize}
Relativo á hidruria.
Aquele que padece hidruria.
\section{Hiena}
\begin{itemize}
\item {Grp. gram.:f.}
\end{itemize}
\begin{itemize}
\item {Proveniência:(Do gr. \textunderscore huaina\textunderscore )}
\end{itemize}
Gênero de mamíferos carnívoros e digitígrados, muito vorazes e parecidos aos cães.
\section{Higiama}
\begin{itemize}
\item {Grp. gram.:f.}
\end{itemize}
\begin{itemize}
\item {Utilização:Pharm.}
\end{itemize}
Alimento, feito de leite, cereaes e açúcar.
\section{Higidamente}
\begin{itemize}
\item {Grp. gram.:adv.}
\end{itemize}
\begin{itemize}
\item {Proveniência:(De \textunderscore hígido\textunderscore )}
\end{itemize}
Relativo á saúde.
\section{Hígido}
\begin{itemize}
\item {Grp. gram.:adj.}
\end{itemize}
Relativo á saúde; salutar.
(Palavra mal formada, do gr. \textunderscore hugies\textunderscore )
\section{Higiene}
\begin{itemize}
\item {Grp. gram.:f.}
\end{itemize}
\begin{itemize}
\item {Utilização:Fig.}
\end{itemize}
\begin{itemize}
\item {Proveniência:(Do gr. \textunderscore hugiainein\textunderscore )}
\end{itemize}
Parte da Medicina, que trata dos meios de conservar a saúde.
Limpeza.
Regime elementar.
Regime.
\textunderscore Higiene moral\textunderscore , aplicação da Fisiologia á Moral e á educação.
\section{Higienicamente}
\begin{itemize}
\item {Grp. gram.:adv.}
\end{itemize}
De modo higiênico.
Segundo as leis da higiene.
\section{Higiênico}
\begin{itemize}
\item {Grp. gram.:adj.}
\end{itemize}
Relativo á higiene.
Conforme aos preceitos da higiene.
Saudável.
Propício ou favorável á saúde.
\section{Higienista}
\begin{itemize}
\item {Grp. gram.:m.}
\end{itemize}
Indivíduo períto em higiene.
Professor de higiene.
\section{Higiologia}
\begin{itemize}
\item {Grp. gram.:f.}
\end{itemize}
\begin{itemize}
\item {Proveniência:(Do gr. \textunderscore hugies\textunderscore  + \textunderscore logos\textunderscore )}
\end{itemize}
História da saúde ou dos actos normaes da economia animal.
\section{Higioterapia}
\begin{itemize}
\item {Grp. gram.:f.}
\end{itemize}
\begin{itemize}
\item {Utilização:Med.}
\end{itemize}
Aplicação dos meios higiênicos á cura das doenças.
\section{Higra}
\begin{itemize}
\item {Grp. gram.:f.}
\end{itemize}
\begin{itemize}
\item {Proveniência:(Lat. \textunderscore hygra\textunderscore )}
\end{itemize}
Espécie de colírio antigo.
\section{Higro...}
\begin{itemize}
\item {Grp. gram.:pref.}
\end{itemize}
\begin{itemize}
\item {Proveniência:(Do gr. \textunderscore hugros\textunderscore )}
\end{itemize}
(designativo de \textunderscore humidade\textunderscore )
\section{Higróbio}
\begin{itemize}
\item {Grp. gram.:adj.}
\end{itemize}
\begin{itemize}
\item {Proveniência:(Do gr. \textunderscore hugros\textunderscore  + \textunderscore bios\textunderscore )}
\end{itemize}
Que vive na água; hidróbio.
\section{Higrocolírio}
\begin{itemize}
\item {Grp. gram.:m.}
\end{itemize}
\begin{itemize}
\item {Proveniência:(De \textunderscore higro...\textunderscore  + \textunderscore colírio\textunderscore )}
\end{itemize}
Colírio liquido.
\section{Higrofobia}
\begin{itemize}
\item {Grp. gram.:f.}
\end{itemize}
\begin{itemize}
\item {Proveniência:(Do gr. \textunderscore hugros\textunderscore  + \textunderscore phobein\textunderscore )}
\end{itemize}
O mesmo que \textunderscore hidrofobia\textunderscore .
\section{Higroftálmico}
\begin{itemize}
\item {Grp. gram.:adj.}
\end{itemize}
\begin{itemize}
\item {Utilização:Med.}
\end{itemize}
\begin{itemize}
\item {Proveniência:(Do gr. \textunderscore hugros\textunderscore  + \textunderscore ophthalmos\textunderscore )}
\end{itemize}
Que serve para humedecer o ôlho.
\section{Higrologia}
\begin{itemize}
\item {Grp. gram.:f.}
\end{itemize}
\begin{itemize}
\item {Proveniência:(Do gr. \textunderscore hugros\textunderscore  + \textunderscore logos\textunderscore )}
\end{itemize}
História da água.
Tratado dos fluidos ou humores do corpo humano.
\section{Higrológico}
\begin{itemize}
\item {Grp. gram.:adj.}
\end{itemize}
Relativo á higrologia.
\section{Higroma}
\begin{itemize}
\item {Grp. gram.:m.}
\end{itemize}
\begin{itemize}
\item {Proveniência:(Do gr. \textunderscore hugros\textunderscore )}
\end{itemize}
Hidropisia nas cápsulas mucosas subcutâneas.
\section{Higrometria}
\begin{itemize}
\item {Grp. gram.:f.}
\end{itemize}
\begin{itemize}
\item {Proveniência:(De \textunderscore higrômetro\textunderscore )}
\end{itemize}
Parte da Física, que determina a quantidade de água em vapor, contida na atmosfera.
\section{Higrométrico}
\begin{itemize}
\item {Grp. gram.:adj.}
\end{itemize}
Relativo á higrometria.
\section{Higrómetro}
\begin{itemize}
\item {Grp. gram.:m.}
\end{itemize}
\begin{itemize}
\item {Proveniência:(Do gr. \textunderscore hugros\textunderscore  + \textunderscore metron\textunderscore )}
\end{itemize}
Instrumento, com que se determina o grau da humidade atmosférica.
\section{Higroscopicidade}
\begin{itemize}
\item {Grp. gram.:f.}
\end{itemize}
\begin{itemize}
\item {Utilização:Bot.}
\end{itemize}
\begin{itemize}
\item {Proveniência:(De \textunderscore higroscópico\textunderscore )}
\end{itemize}
Fôrça, pela qual um tecido, vivo ou morto, tende a absorver ou exalar a humidade, em ordem a pôr-se em equilíbrio com o meio ambiente.
\section{Higroscópico}
\begin{itemize}
\item {Grp. gram.:adj.}
\end{itemize}
Relativo ao higroscópio.
\section{Higroscópio}
\begin{itemize}
\item {Grp. gram.:m.}
\end{itemize}
\begin{itemize}
\item {Proveniência:(Do gr. \textunderscore hugros\textunderscore  + \textunderscore skopein\textunderscore )}
\end{itemize}
O mesmo que \textunderscore higrómetro\textunderscore .
\section{Hilárquico}
\begin{itemize}
\item {Grp. gram.:adj.}
\end{itemize}
\begin{itemize}
\item {Proveniência:(Do gr. \textunderscore hule\textunderscore  + \textunderscore arkhein\textunderscore )}
\end{itemize}
Diz-se do espírito universal que, segundo alguns filósofos, rege a matéria prima.
\section{Hileias}
\begin{itemize}
\item {Grp. gram.:f. pl.}
\end{itemize}
\begin{itemize}
\item {Proveniência:(Do gr. \textunderscore hulaios\textunderscore )}
\end{itemize}
Insectos himenópteros.
\section{Hilesino}
\begin{itemize}
\item {Grp. gram.:m.}
\end{itemize}
\begin{itemize}
\item {Proveniência:(Do gr. \textunderscore hule\textunderscore , madeira)}
\end{itemize}
Insecto, nocivo aos arvoredos e descoberto recentemente nas matas de Portugal.
\section{Hilóbio}
\begin{itemize}
\item {Grp. gram.:m.}
\end{itemize}
\begin{itemize}
\item {Proveniência:(Do gr. \textunderscore hule\textunderscore  + \textunderscore bios\textunderscore )}
\end{itemize}
Gênero de insectos coleópteros tetrâmeros.
\section{Hilogenia}
\begin{itemize}
\item {Grp. gram.:f.}
\end{itemize}
\begin{itemize}
\item {Proveniência:(Do gr. \textunderscore hule\textunderscore  + \textunderscore genos\textunderscore )}
\end{itemize}
Formação da matéria.
\section{Hilótomos}
\begin{itemize}
\item {Grp. gram.:m. pl.}
\end{itemize}
\begin{itemize}
\item {Proveniência:(Do gr. \textunderscore hule\textunderscore  + \textunderscore tome\textunderscore )}
\end{itemize}
Insectos himenópteros, que na madeira fazem entalhes onde põem os ovos.
\section{Hilozoico}
\begin{itemize}
\item {Grp. gram.:adj.}
\end{itemize}
\begin{itemize}
\item {Grp. gram.:M.}
\end{itemize}
Relativo ao hilozoísmo.
Sectário do hilozoísmo.
\section{Hilozoísmo}
\begin{itemize}
\item {Grp. gram.:m.}
\end{itemize}
\begin{itemize}
\item {Proveniência:(Do gr. \textunderscore hule\textunderscore  + \textunderscore zoon\textunderscore )}
\end{itemize}
Sistema filosófico dos que sustentam que a matéria tem existência necessária e é dotada de vida.
\section{Hilozoísta}
\begin{itemize}
\item {Grp. gram.:m.}
\end{itemize}
Sectário do hilozoísmo.
\section{Hímen}
\begin{itemize}
\item {Grp. gram.:m.}
\end{itemize}
\begin{itemize}
\item {Utilização:Bot.}
\end{itemize}
\begin{itemize}
\item {Proveniência:(Lat. \textunderscore hymen\textunderscore )}
\end{itemize}
Membrana, que fecha em parte o orifício da vagina.
Membrana, que envolve o botão da corola.
Himeneu.
\section{Himeneu}
\begin{itemize}
\item {Grp. gram.:m.}
\end{itemize}
\begin{itemize}
\item {Utilização:Fig.}
\end{itemize}
\begin{itemize}
\item {Proveniência:(Do lat. \textunderscore hymenaeus\textunderscore )}
\end{itemize}
Casamento; festa nupcial.
\section{Himênio}
\begin{itemize}
\item {Grp. gram.:m.}
\end{itemize}
\begin{itemize}
\item {Proveniência:(De \textunderscore hímen\textunderscore )}
\end{itemize}
Camada membranosa e superficial que, nos cogumelos, sustenta os órgãos da frutificação.
Camada esporífera dos basidiomicetos.
\section{Himenocarpo}
\begin{itemize}
\item {Grp. gram.:adj.}
\end{itemize}
\begin{itemize}
\item {Utilização:Bot.}
\end{itemize}
\begin{itemize}
\item {Proveniência:(Do gr. \textunderscore humen\textunderscore  + \textunderscore karpos\textunderscore )}
\end{itemize}
Que tem fruto membranoso.
\section{Himenofíleas}
\begin{itemize}
\item {Grp. gram.:f. pl.}
\end{itemize}
\begin{itemize}
\item {Proveniência:(Do gr. \textunderscore humen\textunderscore  + \textunderscore phullon\textunderscore )}
\end{itemize}
Tríbo de musgos.
\section{Himenóforo}
\begin{itemize}
\item {Grp. gram.:m.}
\end{itemize}
\begin{itemize}
\item {Proveniência:(Do gr. \textunderscore humen\textunderscore  + \textunderscore phoros\textunderscore )}
\end{itemize}
Parte do cogumelo, em que assenta o himênio.
\section{Himenografia}
\begin{itemize}
\item {Grp. gram.:f.}
\end{itemize}
\begin{itemize}
\item {Proveniência:(Do gr. \textunderscore humen\textunderscore  + \textunderscore graphein\textunderscore )}
\end{itemize}
Descripção de membranas.
\section{Himenográfico}
\begin{itemize}
\item {Grp. gram.:adj.}
\end{itemize}
Relativo á himenografia.
\section{Himenolitro}
\begin{itemize}
\item {Grp. gram.:adj.}
\end{itemize}
\begin{itemize}
\item {Utilização:Zool.}
\end{itemize}
\begin{itemize}
\item {Proveniência:(De \textunderscore hímen\textunderscore  + \textunderscore elitro\textunderscore )}
\end{itemize}
Que tem elitros membranosos.
\section{Himenologia}
\begin{itemize}
\item {Grp. gram.:f.}
\end{itemize}
\begin{itemize}
\item {Proveniência:(Do gr. \textunderscore humen\textunderscore  + \textunderscore logos\textunderscore )}
\end{itemize}
Tratado das membranas.
\section{Himenomicetos}
\begin{itemize}
\item {Grp. gram.:m. pl.}
\end{itemize}
\begin{itemize}
\item {Proveniência:(Do gr. \textunderscore humen\textunderscore  + \textunderscore mukes\textunderscore )}
\end{itemize}
Ordem de cogumelos, que têm um himênio.
\section{Himenópode}
\begin{itemize}
\item {Grp. gram.:adj.}
\end{itemize}
\begin{itemize}
\item {Proveniência:(Do gr. \textunderscore humen\textunderscore  + \textunderscore pous\textunderscore )}
\end{itemize}
Diz-se das aves, que têm os dedos meio ligados por membrana.
\section{Himenóptero}
\begin{itemize}
\item {Grp. gram.:adj.}
\end{itemize}
\begin{itemize}
\item {Grp. gram.:M. pl.}
\end{itemize}
\begin{itemize}
\item {Proveniência:(Do gr. \textunderscore humen\textunderscore  + \textunderscore pteron\textunderscore )}
\end{itemize}
Que tem quatro asas membranosas e nuas, como as abelhas, certas formigas, etc.
Ordem de insectos himenópteros.
\section{Himenopterologia}
\begin{itemize}
\item {Grp. gram.:f.}
\end{itemize}
\begin{itemize}
\item {Proveniência:(Do gr. \textunderscore humen\textunderscore  + \textunderscore pteron\textunderscore  + \textunderscore logos\textunderscore )}
\end{itemize}
Parte da Entomologia, que trata dos himenópteros.
\section{Himenorrizo}
\begin{itemize}
\item {Grp. gram.:adj.}
\end{itemize}
\begin{itemize}
\item {Utilização:Bot.}
\end{itemize}
\begin{itemize}
\item {Proveniência:(Do gr. \textunderscore humen\textunderscore  + \textunderscore rhiza\textunderscore )}
\end{itemize}
Que tem raízes membranosas.
\section{Himenotomia}
\begin{itemize}
\item {Grp. gram.:f.}
\end{itemize}
\begin{itemize}
\item {Proveniência:(Do gr. \textunderscore humen\textunderscore  + \textunderscore tome\textunderscore )}
\end{itemize}
Dissecção das membranas.
Incisão do hímen.
\section{Hinário}
\begin{itemize}
\item {Grp. gram.:m.}
\end{itemize}
\begin{itemize}
\item {Proveniência:(Do gr. \textunderscore hymnarium\textunderscore )}
\end{itemize}
Colecção de hinos.
Livro de hinos religiosos.
\section{Hínico}
\begin{itemize}
\item {Grp. gram.:adj.}
\end{itemize}
Relativo a hino.
Que é do gênero do hino.
\section{Hinista}
\begin{itemize}
\item {Grp. gram.:m.}
\end{itemize}
\begin{itemize}
\item {Proveniência:(Lat. \textunderscore hymnista\textunderscore )}
\end{itemize}
Cantor ou compositor de hinos.
\section{Hino}
\begin{itemize}
\item {Grp. gram.:m.}
\end{itemize}
\begin{itemize}
\item {Utilização:Ext.}
\end{itemize}
\begin{itemize}
\item {Proveniência:(Gr. \textunderscore humnos\textunderscore )}
\end{itemize}
Canção religiosa.
Canto em louvor dos heróis.
Composição poética, acompanhada de música, em louvor de um monarcha, de uma nação, de um partido, de uma personagem célebre.
Canção, canto.
\section{Hinodo}
\begin{itemize}
\item {Grp. gram.:m.}
\end{itemize}
\begin{itemize}
\item {Proveniência:(Gr. \textunderscore humnodos\textunderscore )}
\end{itemize}
Aquele que, entre os Gregos, cantava hinos nas solenidades religiosas.
\section{Hydróphytas}
\begin{itemize}
\item {Grp. gram.:f. pl.}
\end{itemize}
\begin{itemize}
\item {Proveniência:(De \textunderscore hydróphyto\textunderscore )}
\end{itemize}
Classe de plantas, que vegetam na água ou em terreno alagadiço.
\section{Hydróphyto}
\begin{itemize}
\item {Grp. gram.:m.}
\end{itemize}
\begin{itemize}
\item {Grp. gram.:Adj.}
\end{itemize}
\begin{itemize}
\item {Proveniência:(Do gr. \textunderscore hudor\textunderscore  + \textunderscore phuton\textunderscore )}
\end{itemize}
Qualquer planta, que vive na água.
Que vive na água, (falando-se de plantas).
\section{Hydrophytographia}
\begin{itemize}
\item {Grp. gram.:f.}
\end{itemize}
\begin{itemize}
\item {Proveniência:(Do gr. \textunderscore hudor\textunderscore  + \textunderscore phuton\textunderscore  + \textunderscore graphein\textunderscore )}
\end{itemize}
Descripção scientífica das hydróphytas.
\section{Hydrophytologia}
\begin{itemize}
\item {Grp. gram.:f.}
\end{itemize}
\begin{itemize}
\item {Proveniência:(Do gr. \textunderscore hudor\textunderscore  + \textunderscore phuton\textunderscore  + \textunderscore logos\textunderscore )}
\end{itemize}
Parte da Botânica, que trata das hydróphytas.
\section{Hydrophytológico}
\begin{itemize}
\item {Grp. gram.:adj.}
\end{itemize}
Relativo á hydrophytologia.
\section{Hydrópico}
\begin{itemize}
\item {Grp. gram.:m.  e  adj.}
\end{itemize}
\begin{itemize}
\item {Proveniência:(Lat. \textunderscore hydropicus\textunderscore )}
\end{itemize}
O que tem hydropisia.
\section{Hydropisia}
\begin{itemize}
\item {Grp. gram.:f.}
\end{itemize}
\begin{itemize}
\item {Proveniência:(Do lat. \textunderscore hydropisis\textunderscore )}
\end{itemize}
Acumulação de serosidade no tecido cellular ou numa cavidade do corpo.
\section{Hydropisina}
\begin{itemize}
\item {Grp. gram.:f.}
\end{itemize}
\begin{itemize}
\item {Proveniência:(De \textunderscore hydropisia\textunderscore )}
\end{itemize}
Substância orgânica, que se encontra na serosidade normal de certas membranas.
\section{Hydropneumática}
\begin{itemize}
\item {Grp. gram.:adj. f.}
\end{itemize}
\begin{itemize}
\item {Proveniência:(De \textunderscore hydro...\textunderscore  + \textunderscore pneumático\textunderscore )}
\end{itemize}
Diz-se de uma máquina, inventada há poucos annos em Lisbôa, para rarefazer o vácuo em determinado espaço.
\section{Hydrópota}
\begin{itemize}
\item {Grp. gram.:m.}
\end{itemize}
\begin{itemize}
\item {Proveniência:(Do gr. \textunderscore hudor\textunderscore  + \textunderscore potes\textunderscore )}
\end{itemize}
Aquelle que não bebe senão água.
\section{Hydropúlvis}
\begin{itemize}
\item {Grp. gram.:m.}
\end{itemize}
\begin{itemize}
\item {Proveniência:(De \textunderscore hydro...\textunderscore  + lat. \textunderscore pulvis\textunderscore , pó)}
\end{itemize}
Apparelho, que espalha a água sôbre as plantas em gotas finas e intensas.
\section{Hydropyrética}
\begin{itemize}
\item {Grp. gram.:adj. f.}
\end{itemize}
\begin{itemize}
\item {Proveniência:(Do gr. \textunderscore hudor\textunderscore  + \textunderscore pur\textunderscore )}
\end{itemize}
Diz-se de uma febre maligna, acompanhada de suores.
\section{Hydropýrico}
\begin{itemize}
\item {Grp. gram.:adj.}
\end{itemize}
\begin{itemize}
\item {Proveniência:(Do gr. \textunderscore hudor\textunderscore  + \textunderscore pur\textunderscore )}
\end{itemize}
Diz-se dos vulcões, que lançam fogo e água.
\section{Hydroquinone}
\begin{itemize}
\item {Grp. gram.:m.}
\end{itemize}
\begin{itemize}
\item {Utilização:Phot.}
\end{itemize}
Producto chímico, empregado como revelador enérgico, cuja fórmula é 3C^{6}H^{6}O^{2}CO^{2}.
\section{Hydrorrhagia}
\begin{itemize}
\item {Grp. gram.:f.}
\end{itemize}
\begin{itemize}
\item {Utilização:Med.}
\end{itemize}
\begin{itemize}
\item {Proveniência:(Do gr. \textunderscore hudor\textunderscore  + \textunderscore rhagnumi\textunderscore )}
\end{itemize}
Abundante derramamento de águas, como o que precede o parto.
\section{Hydrorrhéa}
\begin{itemize}
\item {Grp. gram.:f.}
\end{itemize}
\begin{itemize}
\item {Utilização:Med.}
\end{itemize}
\begin{itemize}
\item {Proveniência:(Do gr. \textunderscore hudor\textunderscore  + \textunderscore rhein\textunderscore )}
\end{itemize}
Derramamento lento e chrónico de um líquido aquoso.
\section{Hydrosáccharo}
\begin{itemize}
\item {fónica:sá}
\end{itemize}
\begin{itemize}
\item {Grp. gram.:m.}
\end{itemize}
\begin{itemize}
\item {Utilização:Pharm.}
\end{itemize}
\begin{itemize}
\item {Proveniência:(Do gr. \textunderscore hudor\textunderscore  + \textunderscore sakkharon\textunderscore )}
\end{itemize}
Água açucarada.
\section{Hydroscopia}
\begin{itemize}
\item {Grp. gram.:f.}
\end{itemize}
Arte de procurar fontes ou águas subterrâneas.
Hydromancia.
(Cp. \textunderscore hydróscopo\textunderscore )
\section{Hydróscopo}
\begin{itemize}
\item {Grp. gram.:m.}
\end{itemize}
\begin{itemize}
\item {Proveniência:(Gr. \textunderscore hudroskopos\textunderscore )}
\end{itemize}
Aquelle que pratíca a hydroscopia.
\section{Hydrosilicato}
\begin{itemize}
\item {fónica:si}
\end{itemize}
\begin{itemize}
\item {Grp. gram.:m.}
\end{itemize}
\begin{itemize}
\item {Proveniência:(De \textunderscore hydro...\textunderscore  + \textunderscore silicato\textunderscore )}
\end{itemize}
Silicato, que contém água em combinação.
\section{Hydrosilicoso}
\begin{itemize}
\item {fónica:si}
\end{itemize}
\begin{itemize}
\item {Grp. gram.:adj.}
\end{itemize}
\begin{itemize}
\item {Proveniência:(De \textunderscore hydro...\textunderscore  + \textunderscore silicoso\textunderscore )}
\end{itemize}
Que contém água e sílica.
\section{Hydrosphera}
\begin{itemize}
\item {Grp. gram.:f.}
\end{itemize}
\begin{itemize}
\item {Proveniência:(Do gr. \textunderscore hudros\textunderscore  + \textunderscore sphaira\textunderscore )}
\end{itemize}
A parte líquida da superfície do globo terrestre.
\section{Hydrosphérico}
\begin{itemize}
\item {Grp. gram.:adj.}
\end{itemize}
Relativo á hydrosphera.
\section{Hydrostática}
\begin{itemize}
\item {Grp. gram.:f.}
\end{itemize}
\begin{itemize}
\item {Proveniência:(De \textunderscore hydrostático\textunderscore )}
\end{itemize}
Parte da Mecânica, que trata do equilíbrio dos líquidos, e da pressão que elles exercem.
\section{Hydrostático}
\begin{itemize}
\item {Grp. gram.:adj.}
\end{itemize}
\begin{itemize}
\item {Proveniência:(De \textunderscore hydro...\textunderscore  + \textunderscore estático\textunderscore )}
\end{itemize}
Relativo á hydrostática.
\section{Hydróstato}
\begin{itemize}
\item {Grp. gram.:m.}
\end{itemize}
\begin{itemize}
\item {Proveniência:(Do rad. de \textunderscore hydrostático\textunderscore )}
\end{itemize}
Instrumento de metal, fluctuante, para pesar corpos.
\section{Hydrotechnia}
\begin{itemize}
\item {Grp. gram.:f.}
\end{itemize}
\begin{itemize}
\item {Proveniência:(Do gr. \textunderscore hudor\textunderscore  + \textunderscore tekhne\textunderscore )}
\end{itemize}
Parte da Mecânica, que trata da distribuição e conducção das águas.
\section{Hydrotéchnico}
\begin{itemize}
\item {Grp. gram.:adj.}
\end{itemize}
Relativo á hydrotechnia.
\section{Hydrotherapeuta}
\begin{itemize}
\item {Grp. gram.:m.}
\end{itemize}
\begin{itemize}
\item {Proveniência:(De \textunderscore hydro...\textunderscore  + \textunderscore therapeuta\textunderscore )}
\end{itemize}
Aquelle que exerce a hydrotherapêutica.
\section{Hydrotherapêutica}
\begin{itemize}
\item {Grp. gram.:f.}
\end{itemize}
O mesmo que \textunderscore hydrotherapia\textunderscore .
\section{Hydrotherapia}
\begin{itemize}
\item {Grp. gram.:f.}
\end{itemize}
\begin{itemize}
\item {Proveniência:(Do gr. \textunderscore hudor\textunderscore  + \textunderscore therapeia\textunderscore )}
\end{itemize}
Tratamento de doenças, por meio de água fria, em applicações exteriores.
\section{Hydrotherápico}
\begin{itemize}
\item {Grp. gram.:adj.}
\end{itemize}
Relativo á hydrotherapia.
\section{Hydrothermal}
\begin{itemize}
\item {Grp. gram.:adj.}
\end{itemize}
O mesmo que \textunderscore hydrothérmico\textunderscore .
\section{Hydrothérmico}
\begin{itemize}
\item {Grp. gram.:adj.}
\end{itemize}
\begin{itemize}
\item {Proveniência:(De \textunderscore hydro...\textunderscore  + \textunderscore thérmico\textunderscore )}
\end{itemize}
Relativo á água e ao calor.
\section{Hydrothórax}
\begin{itemize}
\item {Grp. gram.:m.}
\end{itemize}
\begin{itemize}
\item {Proveniência:(De \textunderscore hydro...\textunderscore  + \textunderscore thorax\textunderscore )}
\end{itemize}
Doença, caracterizada por oppressão do peito, produzida por pleurisia ou por simples pontada.
\section{Hydrótica}
\begin{itemize}
\item {Grp. gram.:m.}
\end{itemize}
\begin{itemize}
\item {Utilização:Des.}
\end{itemize}
\begin{itemize}
\item {Proveniência:(De \textunderscore hydrótico\textunderscore )}
\end{itemize}
Parte da Medicina, que se occupa dos suores ou da transpiração. Cf. \textunderscore Diccion. Exegét.\textunderscore 
\section{Hydrótico}
\begin{itemize}
\item {Grp. gram.:adj.}
\end{itemize}
O mesmo que \textunderscore hydragogo\textunderscore .
\section{Hydrotimetria}
\begin{itemize}
\item {Grp. gram.:f.}
\end{itemize}
Processo de applicar o hydrotímetro.
\section{Hydrotimétrico}
\begin{itemize}
\item {Grp. gram.:adj.}
\end{itemize}
Relativo á hydrotimetria.
\section{Hydrotímetro}
\begin{itemize}
\item {Grp. gram.:m.}
\end{itemize}
Instrumento, para avaliar a existência da água das nascentes e rios e a proporção das matérias que se depositam, sob a influência de uma ebullição prolongada.
(Talvez do gr. \textunderscore hudrotes\textunderscore  + \textunderscore metron\textunderscore )
\section{Hydrotomia}
\begin{itemize}
\item {Grp. gram.:m.}
\end{itemize}
\begin{itemize}
\item {Utilização:Anat.}
\end{itemize}
\begin{itemize}
\item {Proveniência:(Do gr. \textunderscore hudor\textunderscore  + \textunderscore tome\textunderscore )}
\end{itemize}
Processo de inocular água nas artérias, fazendo-as transudar por pressão, para desviar fibras, separar certos órgãos, etc.
\section{Hydrotypia}
\begin{itemize}
\item {Grp. gram.:f.}
\end{itemize}
\begin{itemize}
\item {Proveniência:(Do gr. \textunderscore hudor\textunderscore  + \textunderscore tupos\textunderscore )}
\end{itemize}
Processo photográphico para obter côres.
\section{Hydróxydo}
\begin{itemize}
\item {Grp. gram.:m.}
\end{itemize}
\begin{itemize}
\item {Utilização:Chím.}
\end{itemize}
\begin{itemize}
\item {Proveniência:(De \textunderscore hydro...\textunderscore  + \textunderscore óxido\textunderscore )}
\end{itemize}
Combinação da água com um óxydo metállico.
\section{Hydrozôa}
\begin{itemize}
\item {Grp. gram.:f.}
\end{itemize}
\begin{itemize}
\item {Proveniência:(Do gr. \textunderscore hudor\textunderscore  + \textunderscore zoon\textunderscore )}
\end{itemize}
Classe de animaes fósseis da série paleozoica.
\section{Hydruria}
\begin{itemize}
\item {Grp. gram.:f.}
\end{itemize}
\begin{itemize}
\item {Proveniência:(Do gr. \textunderscore hudor\textunderscore  + \textunderscore ouron\textunderscore )}
\end{itemize}
Excesso de água nas urinas humanas.
\section{Hydrúrico}
\begin{itemize}
\item {Grp. gram.:adj.}
\end{itemize}
\begin{itemize}
\item {Grp. gram.:M.}
\end{itemize}
Relativo á hydruria.
Aquelle que padece hydruria.
\section{Hyemal}
\begin{itemize}
\item {Grp. gram.:adj.}
\end{itemize}
\begin{itemize}
\item {Utilização:Des.}
\end{itemize}
\begin{itemize}
\item {Proveniência:(Lat. \textunderscore hyemalis\textunderscore )}
\end{itemize}
O mesmo que \textunderscore hibernal\textunderscore .
\section{Hyena}
\begin{itemize}
\item {Grp. gram.:f.}
\end{itemize}
\begin{itemize}
\item {Proveniência:(Do gr. \textunderscore huaina\textunderscore )}
\end{itemize}
Gênero de mammíferos carnívoros e digitígrados, muito vorazes e parecidos aos cães.
\section{Hygiama}
\begin{itemize}
\item {Grp. gram.:f.}
\end{itemize}
\begin{itemize}
\item {Utilização:Pharm.}
\end{itemize}
Alimento, feito de leite, cereaes e açúcar.
\section{Hygidamente}
\begin{itemize}
\item {Grp. gram.:adv.}
\end{itemize}
\begin{itemize}
\item {Proveniência:(De \textunderscore hýgido\textunderscore )}
\end{itemize}
Relativo á saúde.
\section{Hýgido}
\begin{itemize}
\item {Grp. gram.:adj.}
\end{itemize}
Relativo á saúde; salutar.
(Palavra mal formada, do gr. \textunderscore hugies\textunderscore )
\section{Hygiene}
\begin{itemize}
\item {Grp. gram.:f.}
\end{itemize}
\begin{itemize}
\item {Utilização:Fig.}
\end{itemize}
\begin{itemize}
\item {Proveniência:(Do gr. \textunderscore hugiainein\textunderscore )}
\end{itemize}
Parte da Medicina, que trata dos meios de conservar a saúde.
Limpeza.
Regime elementar.
Regime.
\textunderscore Hygiene moral\textunderscore , applicação da Physiologia á Moral e á educação.
\section{Hygienicamente}
\begin{itemize}
\item {Grp. gram.:adv.}
\end{itemize}
De modo hygiênico.
Segundo as leis da hygiene.
\section{Hygiênico}
\begin{itemize}
\item {Grp. gram.:adj.}
\end{itemize}
Relativo á hygiene.
Conforme aos preceitos da hygiene.
Saudável.
Propício ou favorável á saúde.
\section{Hygienista}
\begin{itemize}
\item {Grp. gram.:m.}
\end{itemize}
Indivíduo períto em hygiene.
Professor de hygiene.
\section{Hygiologia}
\begin{itemize}
\item {Grp. gram.:f.}
\end{itemize}
\begin{itemize}
\item {Proveniência:(Do gr. \textunderscore hugies\textunderscore  + \textunderscore logos\textunderscore )}
\end{itemize}
História da saúde ou dos actos normaes da economia animal.
\section{Hygiotherapia}
\begin{itemize}
\item {Grp. gram.:f.}
\end{itemize}
\begin{itemize}
\item {Utilização:Med.}
\end{itemize}
Applicação dos meios hygiênicos á cura das doenças.
\section{Hygra}
\begin{itemize}
\item {Grp. gram.:f.}
\end{itemize}
\begin{itemize}
\item {Proveniência:(Lat. \textunderscore hygra\textunderscore )}
\end{itemize}
Espécie de collýrio antigo.
\section{Hygro...}
\begin{itemize}
\item {Grp. gram.:pref.}
\end{itemize}
\begin{itemize}
\item {Proveniência:(Do gr. \textunderscore hugros\textunderscore )}
\end{itemize}
(designativo de \textunderscore humidade\textunderscore )
\section{Hygróbio}
\begin{itemize}
\item {Grp. gram.:adj.}
\end{itemize}
\begin{itemize}
\item {Proveniência:(Do gr. \textunderscore hugros\textunderscore  + \textunderscore bios\textunderscore )}
\end{itemize}
Que vive na água; hydróbio.
\section{Hygrocollýrio}
\begin{itemize}
\item {Grp. gram.:m.}
\end{itemize}
\begin{itemize}
\item {Proveniência:(De \textunderscore hygro...\textunderscore  + \textunderscore collýrio\textunderscore )}
\end{itemize}
Collýrio liquido.
\section{Hygrologia}
\begin{itemize}
\item {Grp. gram.:f.}
\end{itemize}
\begin{itemize}
\item {Proveniência:(Do gr. \textunderscore hugros\textunderscore  + \textunderscore logos\textunderscore )}
\end{itemize}
História da água.
Tratado dos fluidos ou humores do corpo humano.
\section{Hygrológico}
\begin{itemize}
\item {Grp. gram.:adj.}
\end{itemize}
Relativo á hygrologia.
\section{Hygroma}
\begin{itemize}
\item {Grp. gram.:m.}
\end{itemize}
\begin{itemize}
\item {Proveniência:(Do gr. \textunderscore hugros\textunderscore )}
\end{itemize}
Hydropisia nas cápsulas mucosas subcutâneas.
\section{Hygrometria}
\begin{itemize}
\item {Grp. gram.:f.}
\end{itemize}
\begin{itemize}
\item {Proveniência:(De \textunderscore hygrômetro\textunderscore )}
\end{itemize}
Parte da Phýsica, que determina a quantidade de água em vapor, contida na atmosphera.
\section{Hygrométrico}
\begin{itemize}
\item {Grp. gram.:adj.}
\end{itemize}
Relativo á hygrometria.
\section{Hygrómetro}
\begin{itemize}
\item {Grp. gram.:m.}
\end{itemize}
\begin{itemize}
\item {Proveniência:(Do gr. \textunderscore hugros\textunderscore  + \textunderscore metron\textunderscore )}
\end{itemize}
Instrumento, com que se determina o grau da humidade atmosphérica.
\section{Hygrophobia}
\begin{itemize}
\item {Grp. gram.:f.}
\end{itemize}
\begin{itemize}
\item {Proveniência:(Do gr. \textunderscore hugros\textunderscore  + \textunderscore phobein\textunderscore )}
\end{itemize}
O mesmo que \textunderscore hydrophobia\textunderscore .
\section{Hygrophthálmico}
\begin{itemize}
\item {Grp. gram.:adj.}
\end{itemize}
\begin{itemize}
\item {Utilização:Med.}
\end{itemize}
\begin{itemize}
\item {Proveniência:(Do gr. \textunderscore hugros\textunderscore  + \textunderscore ophthalmos\textunderscore )}
\end{itemize}
Que serve para humedecer o ôlho.
\section{Hygroscopicidade}
\begin{itemize}
\item {Grp. gram.:f.}
\end{itemize}
\begin{itemize}
\item {Utilização:Bot.}
\end{itemize}
\begin{itemize}
\item {Proveniência:(De \textunderscore hygroscópico\textunderscore )}
\end{itemize}
Fôrça, pela qual um tecido, vivo ou morto, tende a absorver ou exhalar a humidade, em ordem a pôr-se em equilíbrio com o meio ambiente.
\section{Hygroscópico}
\begin{itemize}
\item {Grp. gram.:adj.}
\end{itemize}
Relativo ao hygroscópio.
\section{Hygroscópio}
\begin{itemize}
\item {Grp. gram.:m.}
\end{itemize}
\begin{itemize}
\item {Proveniência:(Do gr. \textunderscore hugros\textunderscore  + \textunderscore skopein\textunderscore )}
\end{itemize}
O mesmo que \textunderscore hygrómetro\textunderscore .
\section{Hylarchico}
\begin{itemize}
\item {fónica:qui}
\end{itemize}
\begin{itemize}
\item {Grp. gram.:adj.}
\end{itemize}
\begin{itemize}
\item {Proveniência:(Do gr. \textunderscore hule\textunderscore  + \textunderscore arkhein\textunderscore )}
\end{itemize}
Diz-se do espírito universal que, segundo alguns philósophos, rege a matéria prima.
\section{Hyleias}
\begin{itemize}
\item {Grp. gram.:f. pl.}
\end{itemize}
\begin{itemize}
\item {Proveniência:(Do gr. \textunderscore hulaios\textunderscore )}
\end{itemize}
Insectos hymenópteros.
\section{Hylesino}
\begin{itemize}
\item {Grp. gram.:m.}
\end{itemize}
\begin{itemize}
\item {Proveniência:(Do gr. \textunderscore hule\textunderscore , madeira)}
\end{itemize}
Insecto, nocivo aos arvoredos e descoberto recentemente nas matas de Portugal.
\section{Hylóbio}
\begin{itemize}
\item {Grp. gram.:m.}
\end{itemize}
\begin{itemize}
\item {Proveniência:(Do gr. \textunderscore hule\textunderscore  + \textunderscore bios\textunderscore )}
\end{itemize}
Gênero de insectos coleópteros tetrâmeros.
\section{Hylogenia}
\begin{itemize}
\item {Grp. gram.:f.}
\end{itemize}
\begin{itemize}
\item {Proveniência:(Do gr. \textunderscore hule\textunderscore  + \textunderscore genos\textunderscore )}
\end{itemize}
Formação da matéria.
\section{Hylótomos}
\begin{itemize}
\item {Grp. gram.:m. pl.}
\end{itemize}
\begin{itemize}
\item {Proveniência:(Do gr. \textunderscore hule\textunderscore  + \textunderscore tome\textunderscore )}
\end{itemize}
Insectos hymenópteros, que na madeira fazem entalhes onde põem os ovos.
\section{Hylozoico}
\begin{itemize}
\item {Grp. gram.:adj.}
\end{itemize}
\begin{itemize}
\item {Grp. gram.:M.}
\end{itemize}
Relativo ao hylozoísmo.
Sectário do hylozoísmo.
\section{Hylozoísmo}
\begin{itemize}
\item {Grp. gram.:m.}
\end{itemize}
\begin{itemize}
\item {Proveniência:(Do gr. \textunderscore hule\textunderscore  + \textunderscore zoon\textunderscore )}
\end{itemize}
Systema philosóphico dos que sustentam que a matéria tem existência necessária e é dotada de vida.
\section{Hylozoísta}
\begin{itemize}
\item {Grp. gram.:m.}
\end{itemize}
Sectário do hylozoísmo.
\section{Hýmen}
\begin{itemize}
\item {Grp. gram.:m.}
\end{itemize}
\begin{itemize}
\item {Utilização:Bot.}
\end{itemize}
\begin{itemize}
\item {Proveniência:(Lat. \textunderscore hymen\textunderscore )}
\end{itemize}
Membrana, que fecha em parte o orifício da vagina.
Membrana, que envolve o botão da corolla.
Hymeneu.
\section{Hymeneu}
\begin{itemize}
\item {Grp. gram.:m.}
\end{itemize}
\begin{itemize}
\item {Utilização:Fig.}
\end{itemize}
\begin{itemize}
\item {Proveniência:(Do lat. \textunderscore hymenaeus\textunderscore )}
\end{itemize}
Casamento; festa nupcial.
\section{Hymênio}
\begin{itemize}
\item {Grp. gram.:m.}
\end{itemize}
\begin{itemize}
\item {Proveniência:(De \textunderscore hýmen\textunderscore )}
\end{itemize}
Camada membranosa e superficial que, nos cogumelos, sustenta os órgãos da frutificação.
Camada esporífera dos basidiomycetos.
\section{Hymenocarpo}
\begin{itemize}
\item {Grp. gram.:adj.}
\end{itemize}
\begin{itemize}
\item {Utilização:Bot.}
\end{itemize}
\begin{itemize}
\item {Proveniência:(Do gr. \textunderscore humen\textunderscore  + \textunderscore karpos\textunderscore )}
\end{itemize}
Que tem fruto membranoso.
\section{Hymenographia}
\begin{itemize}
\item {Grp. gram.:f.}
\end{itemize}
\begin{itemize}
\item {Proveniência:(Do gr. \textunderscore humen\textunderscore  + \textunderscore graphein\textunderscore )}
\end{itemize}
Descripção de membranas.
\section{Hymenográphico}
\begin{itemize}
\item {Grp. gram.:adj.}
\end{itemize}
Relativo á hymenographia.
\section{Hymenologia}
\begin{itemize}
\item {Grp. gram.:f.}
\end{itemize}
\begin{itemize}
\item {Proveniência:(Do gr. \textunderscore humen\textunderscore  + \textunderscore logos\textunderscore )}
\end{itemize}
Tratado das membranas.
\section{Hymenolytro}
\begin{itemize}
\item {Grp. gram.:adj.}
\end{itemize}
\begin{itemize}
\item {Utilização:Zool.}
\end{itemize}
\begin{itemize}
\item {Proveniência:(De \textunderscore hýmen\textunderscore  + \textunderscore elytro\textunderscore )}
\end{itemize}
Que tem elytros membranosos.
\section{Hymenomycetos}
\begin{itemize}
\item {Grp. gram.:m. pl.}
\end{itemize}
\begin{itemize}
\item {Proveniência:(Do gr. \textunderscore humen\textunderscore  + \textunderscore mukes\textunderscore )}
\end{itemize}
Ordem de cogumelos, que têm um hymênio.
\section{Hymenóphoro}
\begin{itemize}
\item {Grp. gram.:m.}
\end{itemize}
\begin{itemize}
\item {Proveniência:(Do gr. \textunderscore humen\textunderscore  + \textunderscore phoros\textunderscore )}
\end{itemize}
Parte do cogumelo, em que assenta o hymênio.
\section{Hymenophýlleas}
\begin{itemize}
\item {Grp. gram.:f. pl.}
\end{itemize}
\begin{itemize}
\item {Proveniência:(Do gr. \textunderscore humen\textunderscore  + \textunderscore phullon\textunderscore )}
\end{itemize}
Tríbo de musgos.
\section{Hymenópode}
\begin{itemize}
\item {Grp. gram.:adj.}
\end{itemize}
\begin{itemize}
\item {Proveniência:(Do gr. \textunderscore humen\textunderscore  + \textunderscore pous\textunderscore )}
\end{itemize}
Diz-se das aves, que têm os dedos meio ligados por membrana.
\section{Hymenóptero}
\begin{itemize}
\item {Grp. gram.:adj.}
\end{itemize}
\begin{itemize}
\item {Grp. gram.:M. pl.}
\end{itemize}
\begin{itemize}
\item {Proveniência:(Do gr. \textunderscore humen\textunderscore  + \textunderscore pteron\textunderscore )}
\end{itemize}
Que tem quatro asas membranosas e nuas, como as abelhas, certas formigas, etc.
Ordem de insectos hymenópteros.
\section{Hymenopterologia}
\begin{itemize}
\item {Grp. gram.:f.}
\end{itemize}
\begin{itemize}
\item {Proveniência:(Do gr. \textunderscore humen\textunderscore  + \textunderscore pteron\textunderscore  + \textunderscore logos\textunderscore )}
\end{itemize}
Parte da Entomologia, que trata dos hymenópteros.
\section{Hymenorrhizo}
\begin{itemize}
\item {Grp. gram.:adj.}
\end{itemize}
\begin{itemize}
\item {Utilização:Bot.}
\end{itemize}
\begin{itemize}
\item {Proveniência:(Do gr. \textunderscore humen\textunderscore  + \textunderscore rhiza\textunderscore )}
\end{itemize}
Que tem raízes membranosas.
\section{Hymenotomia}
\begin{itemize}
\item {Grp. gram.:f.}
\end{itemize}
\begin{itemize}
\item {Proveniência:(Do gr. \textunderscore humen\textunderscore  + \textunderscore tome\textunderscore )}
\end{itemize}
Dissecção das membranas.
Incisão do hýmen.
\section{Hymnário}
\begin{itemize}
\item {Grp. gram.:m.}
\end{itemize}
\begin{itemize}
\item {Proveniência:(Do gr. \textunderscore hymnarium\textunderscore )}
\end{itemize}
Collecção de hymnos.
Livro de hymnos religiosos.
\section{Hymnico}
\begin{itemize}
\item {Grp. gram.:adj.}
\end{itemize}
Relativo a hymno.
Que é do gênero do hymno.
\section{Hymnista}
\begin{itemize}
\item {Grp. gram.:m.}
\end{itemize}
\begin{itemize}
\item {Proveniência:(Lat. \textunderscore hymnista\textunderscore )}
\end{itemize}
Cantor ou compositor de hymnos.
\section{Hymno}
\begin{itemize}
\item {Grp. gram.:m.}
\end{itemize}
\begin{itemize}
\item {Utilização:Ext.}
\end{itemize}
\begin{itemize}
\item {Proveniência:(Gr. \textunderscore humnos\textunderscore )}
\end{itemize}
Canção religiosa.
Canto em louvor dos heróis.
Composição poética, acompanhada de música, em louvor de um monarcha, de uma nação, de um partido, de uma personagem célebre.
Canção, canto.
\section{Hymnodo}
\begin{itemize}
\item {Grp. gram.:m.}
\end{itemize}
\begin{itemize}
\item {Proveniência:(Gr. \textunderscore humnodos\textunderscore )}
\end{itemize}
Aquelle que, entre os Gregos, cantava hymnos nas solennidades religiosas.
\section{Hinografia}
\begin{itemize}
\item {Grp. gram.:f.}
\end{itemize}
\begin{itemize}
\item {Proveniência:(De \textunderscore hinógrafo\textunderscore )}
\end{itemize}
Tratado bibliográfico dos hinos.
\section{Hinógrafo}
\begin{itemize}
\item {Grp. gram.:m.  e  adj.}
\end{itemize}
\begin{itemize}
\item {Proveniência:(Do gr. \textunderscore humnos\textunderscore  + \textunderscore graphein\textunderscore )}
\end{itemize}
O que compõe hinos.
\section{Hinologia}
\begin{itemize}
\item {Grp. gram.:f.}
\end{itemize}
\begin{itemize}
\item {Proveniência:(Do gr. \textunderscore humnos\textunderscore  + \textunderscore logos\textunderscore )}
\end{itemize}
Arte de compor hinos.
Acto de recitar ou cantar hinos.
\section{Hinologista}
\begin{itemize}
\item {Grp. gram.:m.}
\end{itemize}
Apologista entusiástico; panegirista. Cf. Camillo, \textunderscore Serões\textunderscore , IV, 29.
(Cp. \textunderscore hinologia\textunderscore )
\section{Hinólogo}
\begin{itemize}
\item {Grp. gram.:m.}
\end{itemize}
\begin{itemize}
\item {Proveniência:(Lat. \textunderscore hymnologus\textunderscore )}
\end{itemize}
O mesmo que \textunderscore hinista\textunderscore .
\section{Hioide}
\begin{itemize}
\item {Grp. gram.:m.}
\end{itemize}
Pequeno osso, entre a laringe e a base da língua.
(Contr. de \textunderscore upsiloide\textunderscore , de \textunderscore upsilon\textunderscore , n. gr. da letra \textunderscore y\textunderscore  + \textunderscore eidos\textunderscore )
\section{Hioídeo}
\begin{itemize}
\item {Grp. gram.:adj.}
\end{itemize}
Relativo ao hioide.
\section{Hioscíama}
\begin{itemize}
\item {Grp. gram.:f.}
\end{itemize}
\begin{itemize}
\item {Proveniência:(Do gr. \textunderscore huos\textunderscore  + \textunderscore kuamos\textunderscore )}
\end{itemize}
Nome científico do meimendro.
\section{Hiosciamina}
\begin{itemize}
\item {Grp. gram.:f.}
\end{itemize}
\begin{itemize}
\item {Proveniência:(De \textunderscore hiosciama\textunderscore )}
\end{itemize}
Alcaloide, descoberto recentemente na alface.
Alcaloide, extraido do meimendro.
\section{Hioscíamo}
\begin{itemize}
\item {Grp. gram.:m.}
\end{itemize}
\begin{itemize}
\item {Proveniência:(Lat. \textunderscore hyoscyamus\textunderscore )}
\end{itemize}
O mesmo ou melhor que \textunderscore hioscíama\textunderscore .
\section{Hiosternal}
\begin{itemize}
\item {Grp. gram.:m.}
\end{itemize}
\begin{itemize}
\item {Utilização:Anat.}
\end{itemize}
\begin{itemize}
\item {Proveniência:(De \textunderscore hioide\textunderscore  e \textunderscore esternal\textunderscore )}
\end{itemize}
Terceira peça do esterno.
\section{Hipálage}
\begin{itemize}
\item {Grp. gram.:f.}
\end{itemize}
\begin{itemize}
\item {Proveniência:(Lat. \textunderscore hypallage\textunderscore )}
\end{itemize}
Figura de retórica, com que parece atribuír-se ás palavras de uma frase o que pertence a outras palavras da mesma frase.
\section{Hipanto}
\begin{itemize}
\item {Grp. gram.:m.}
\end{itemize}
\begin{itemize}
\item {Utilização:Bot.}
\end{itemize}
\begin{itemize}
\item {Proveniência:(Do gr. \textunderscore hupo\textunderscore  + \textunderscore anthos\textunderscore )}
\end{itemize}
Parte inferior do cálice.
Inflorescência, própria da figueira.
\section{Hipantódio}
\begin{itemize}
\item {Grp. gram.:m.}
\end{itemize}
\begin{itemize}
\item {Proveniência:(Lat. \textunderscore hypanthodium\textunderscore )}
\end{itemize}
O mesmo que \textunderscore sícono\textunderscore .
\section{Hípata}
\begin{itemize}
\item {Grp. gram.:f.}
\end{itemize}
\begin{itemize}
\item {Utilização:Ant.}
\end{itemize}
\begin{itemize}
\item {Proveniência:(Lat. \textunderscore hypate\textunderscore )}
\end{itemize}
A corda mais grave da lira e de outros instrumentos.
\section{Hipemia}
\begin{itemize}
\item {Grp. gram.:f.}
\end{itemize}
\begin{itemize}
\item {Utilização:Med.}
\end{itemize}
\begin{itemize}
\item {Proveniência:(Do gr. \textunderscore hupo\textunderscore  + \textunderscore haima\textunderscore )}
\end{itemize}
Anemia, localizada numa parte do organismo.
\section{Hiper...}
\begin{itemize}
\item {Grp. gram.:pref.}
\end{itemize}
\begin{itemize}
\item {Proveniência:(Gr. \textunderscore huper\textunderscore )}
\end{itemize}
(designativo de \textunderscore muito\textunderscore ; \textunderscore em alto gráu\textunderscore ; \textunderscore além\textunderscore )
\section{Hiperacidez}
\begin{itemize}
\item {Grp. gram.:f.}
\end{itemize}
Estado ou qualidade daquilo que é hiperácido.
\section{Hiperácido}
\begin{itemize}
\item {Grp. gram.:adj.}
\end{itemize}
\begin{itemize}
\item {Proveniência:(De \textunderscore hiper...\textunderscore  + \textunderscore ácido\textunderscore )}
\end{itemize}
Excessivamente ácido, (falando-se da urina ou de outros humores orgânicos).
\section{Hiperacusia}
\begin{itemize}
\item {Grp. gram.:f.}
\end{itemize}
\begin{itemize}
\item {Utilização:Med.}
\end{itemize}
\begin{itemize}
\item {Proveniência:(Do gr. \textunderscore huper\textunderscore  + \textunderscore acusis\textunderscore )}
\end{itemize}
Excitação auditiva.
Percepção dolorosa e confusa de certos sons, mormente dos elevados e agudos.
\section{Hiperacúsico}
\begin{itemize}
\item {Grp. gram.:adj.}
\end{itemize}
Relativo a hiperacusia.
\section{Hiperalbuminose}
\begin{itemize}
\item {Grp. gram.:f.}
\end{itemize}
\begin{itemize}
\item {Proveniência:(De \textunderscore hiper...\textunderscore  + \textunderscore albumina\textunderscore )}
\end{itemize}
Excesso de albumina no sangue.
\section{Hiperalgesia}
\begin{itemize}
\item {Grp. gram.:f.}
\end{itemize}
\begin{itemize}
\item {Utilização:Med.}
\end{itemize}
\begin{itemize}
\item {Proveniência:(Do gr. \textunderscore huper\textunderscore  + \textunderscore algesis\textunderscore )}
\end{itemize}
Exagêro da sensibilidade á dôr.
\section{Hipérbato}
\begin{itemize}
\item {Grp. gram.:m.}
\end{itemize}
O mesmo ou melhor que \textunderscore hipérbaton\textunderscore .
\section{Hipérbaton}
\begin{itemize}
\item {Grp. gram.:m.}
\end{itemize}
\begin{itemize}
\item {Utilização:Gram.}
\end{itemize}
\begin{itemize}
\item {Proveniência:(Lat. \textunderscore hyperbaton\textunderscore )}
\end{itemize}
Transposição ou inversão da ordem natural das palavras ou das proposições.
\section{Hiperbibasmo}
\begin{itemize}
\item {Grp. gram.:m.}
\end{itemize}
Deslocação do accento tónico de uma palavra, tornando-se paroxítona a que era proparoxítona e vice-versa. Cf. J. V. Boscoli, \textunderscore Gram.\textunderscore  Exemplo: \textunderscore míope\textunderscore , por \textunderscore miópe\textunderscore ; \textunderscore patêna\textunderscore , por \textunderscore pátena\textunderscore .
\section{Hipérbole}
\begin{itemize}
\item {Grp. gram.:f.}
\end{itemize}
\begin{itemize}
\item {Proveniência:(Lat. \textunderscore hyperbole\textunderscore )}
\end{itemize}
Figura de retórica, que exagera ou deminue excessivamente a verdade das coisas, para que produzam maior impressão.
Curva geométrica, em que cada um dos pontos mantém igual distância de dois pontos fixos, chamados focos.--Alguns clássicos fazem masculino o termo. Cf. Filinto, XXII, 101 e 104.
\section{Hiperbolicamente}
\begin{itemize}
\item {Grp. gram.:adv.}
\end{itemize}
De modo hiperbólico; exageradamente.
\section{Hiperbólico}
\begin{itemize}
\item {Grp. gram.:adj.}
\end{itemize}
\begin{itemize}
\item {Utilização:Fig.}
\end{itemize}
\begin{itemize}
\item {Proveniência:(Lat. \textunderscore hyperbolicus\textunderscore )}
\end{itemize}
Relativo a hipérbole.
Exagerado.
\section{Hiperboliforme}
\begin{itemize}
\item {Grp. gram.:adj.}
\end{itemize}
\begin{itemize}
\item {Utilização:Geom.}
\end{itemize}
\begin{itemize}
\item {Proveniência:(De \textunderscore hipérbole\textunderscore  + \textunderscore fórma\textunderscore )}
\end{itemize}
Que tem proximamente a fórma da hipérbole.
\section{Hiperbolismo}
\begin{itemize}
\item {Grp. gram.:m.}
\end{itemize}
\begin{itemize}
\item {Utilização:Neol.}
\end{itemize}
\begin{itemize}
\item {Proveniência:(De \textunderscore hipérbole\textunderscore )}
\end{itemize}
Emprego abusivo da hipérbole, em linguagem.
\section{Hiperbolizar}
\begin{itemize}
\item {Grp. gram.:v. i.}
\end{itemize}
Empregar hipérboles.
\section{Hiperboloide}
\begin{itemize}
\item {Grp. gram.:m.}
\end{itemize}
\begin{itemize}
\item {Proveniência:(Do gr. \textunderscore huperbole\textunderscore  + \textunderscore eidos\textunderscore )}
\end{itemize}
Sólido geométrico, produzido pela revolução de uma hipérbole.
Hiperbólico.
\section{Hiperbóreo}
\begin{itemize}
\item {Grp. gram.:adj.}
\end{itemize}
\begin{itemize}
\item {Grp. gram.:M. Pl.}
\end{itemize}
\begin{itemize}
\item {Proveniência:(Lat. \textunderscore hyperboreus\textunderscore )}
\end{itemize}
Setentrional.
Relativo ao Norte.
Que cresce em lugares muito frios, (falando-se de plantas).
Povos setentrionaes da América do Norte, como os Groenlandeses, etc.
\section{Hipercataléctico}
\begin{itemize}
\item {Grp. gram.:adj.}
\end{itemize}
\begin{itemize}
\item {Proveniência:(Lat. \textunderscore hypercatalecticus\textunderscore )}
\end{itemize}
Que tem uma sílaba de mais, (falando-se de versos gregos e latinos).
\section{Hipercatalecto}
\begin{itemize}
\item {Grp. gram.:m.}
\end{itemize}
\begin{itemize}
\item {Proveniência:(Lat. \textunderscore hypercatalectus\textunderscore )}
\end{itemize}
Verso grego ou latino, com uma sílaba de mais.
\section{Hipercatarse}
\begin{itemize}
\item {Grp. gram.:f.}
\end{itemize}
\begin{itemize}
\item {Utilização:Med.}
\end{itemize}
\begin{itemize}
\item {Utilização:Ant.}
\end{itemize}
\begin{itemize}
\item {Proveniência:(Do gr. \textunderscore huper\textunderscore  + \textunderscore cathairo\textunderscore )}
\end{itemize}
Purgação excessiva.
\section{Hiperceratose}
\begin{itemize}
\item {Grp. gram.:f.}
\end{itemize}
\begin{itemize}
\item {Utilização:Med.}
\end{itemize}
\begin{itemize}
\item {Proveniência:(Do gr. \textunderscore huper\textunderscore  + \textunderscore keras\textunderscore )}
\end{itemize}
Hipertrofia da córnea.
\section{Hipercerebração}
\begin{itemize}
\item {Grp. gram.:f.}
\end{itemize}
\begin{itemize}
\item {Proveniência:(De \textunderscore hiper...\textunderscore  + \textunderscore cérebro\textunderscore )}
\end{itemize}
Excessivo trabalho intelectual.
\section{Hipercinesia}
\begin{itemize}
\item {Grp. gram.:f.}
\end{itemize}
\begin{itemize}
\item {Proveniência:(Do gr. \textunderscore huper\textunderscore  + \textunderscore kinesis\textunderscore )}
\end{itemize}
Excitação da motilidade; exagêro de movimentos.
\section{Hiperclorato}
\begin{itemize}
\item {Grp. gram.:m.}
\end{itemize}
\begin{itemize}
\item {Proveniência:(De \textunderscore hiper...\textunderscore  + \textunderscore clorato\textunderscore )}
\end{itemize}
Sal, resultante da combinação do ácido hiperclórico com uma base.
\section{Hiperclórico}
\begin{itemize}
\item {Grp. gram.:adj.}
\end{itemize}
\begin{itemize}
\item {Proveniência:(De \textunderscore hiper...\textunderscore  + \textunderscore clórico\textunderscore )}
\end{itemize}
Diz-se de um dos oxácidos do cloro.
\section{Hipercloridria}
\begin{itemize}
\item {Grp. gram.:f.}
\end{itemize}
\begin{itemize}
\item {Proveniência:(De \textunderscore hiper\textunderscore  + \textunderscore clorídrico\textunderscore )}
\end{itemize}
Excesso de ácido clorídrico no suco gástrico.
\section{Hipercrinia}
\begin{itemize}
\item {Grp. gram.:f.}
\end{itemize}
\begin{itemize}
\item {Utilização:Med.}
\end{itemize}
\begin{itemize}
\item {Proveniência:(Do gr. \textunderscore huper\textunderscore  + \textunderscore krinein\textunderscore )}
\end{itemize}
Secreção excessiva.
\section{Hipercrise}
\begin{itemize}
\item {Grp. gram.:f.}
\end{itemize}
\begin{itemize}
\item {Proveniência:(De \textunderscore hiper...\textunderscore  + \textunderscore crise\textunderscore )}
\end{itemize}
Crise patológica, fóra do commum.
\section{Hipercrítico}
\begin{itemize}
\item {Grp. gram.:m.}
\end{itemize}
\begin{itemize}
\item {Grp. gram.:Adj.}
\end{itemize}
\begin{itemize}
\item {Proveniência:(De \textunderscore hiper\textunderscore  + \textunderscore crítico\textunderscore )}
\end{itemize}
Censor exagerado, crítico que nada perdôa:«\textunderscore demasiou-se, dizem os hipercríticos, e disse-o eu já.\textunderscore »Castilho.
Que critica com exagêro; relativo á crítica exagerada. Cf. Herculano, \textunderscore Hist. de Port.\textunderscore , III, 161.
\section{Hipercroma}
\begin{itemize}
\item {Grp. gram.:m.}
\end{itemize}
\begin{itemize}
\item {Utilização:Med.}
\end{itemize}
\begin{itemize}
\item {Proveniência:(Do gr. \textunderscore huper\textunderscore  + \textunderscore khroma\textunderscore )}
\end{itemize}
Excrescência carnosa, junto da carúncula, no grande ângulo do ôlho.
\section{Hipercromia}
\begin{itemize}
\item {Grp. gram.:f.}
\end{itemize}
\begin{itemize}
\item {Proveniência:(Do gr. \textunderscore huper\textunderscore  + \textunderscore khroma\textunderscore )}
\end{itemize}
Exagêro da pigmentação da pele.
\section{Hiperdiacrise}
\begin{itemize}
\item {Grp. gram.:f.}
\end{itemize}
\begin{itemize}
\item {Proveniência:(Do gr. \textunderscore huper\textunderscore  + \textunderscore diakrisis\textunderscore )}
\end{itemize}
O mesmo que \textunderscore hipercrinia\textunderscore .
\section{Hiperdramático}
\begin{itemize}
\item {Grp. gram.:adj.}
\end{itemize}
\begin{itemize}
\item {Proveniência:(De \textunderscore hiper...\textunderscore  + \textunderscore dramático\textunderscore )}
\end{itemize}
Excessivamente dramático; em que se exageram os meios dramáticos.
\section{Hiperdulia}
\begin{itemize}
\item {Grp. gram.:f.}
\end{itemize}
\begin{itemize}
\item {Proveniência:(De \textunderscore hiper...\textunderscore  + \textunderscore dulia\textunderscore )}
\end{itemize}
Culto, que se presta especialmente á Virgem Maria.
\section{Hiperelíptico}
\begin{itemize}
\item {Grp. gram.:adj.}
\end{itemize}
\begin{itemize}
\item {Utilização:Mathem.}
\end{itemize}
\begin{itemize}
\item {Proveniência:(De \textunderscore hiper...\textunderscore  + \textunderscore elíptico\textunderscore )}
\end{itemize}
Diz-se da função, formada por uma integral, cuja diferencial contém, sob uma raiz do segundo grau, um polinómio de grau superior ao quarto.
\section{Hiperemia}
\begin{itemize}
\item {Grp. gram.:f.}
\end{itemize}
\begin{itemize}
\item {Proveniência:(Do gr. \textunderscore huper\textunderscore  + \textunderscore haima\textunderscore )}
\end{itemize}
Superabundância de sangue em qualquer parte do corpo.
\section{Hiperemiar}
\begin{itemize}
\item {Grp. gram.:v. t.}
\end{itemize}
Causar hiperemia em.
\section{Hiperenterose}
\begin{itemize}
\item {Grp. gram.:f.}
\end{itemize}
\begin{itemize}
\item {Proveniência:(De \textunderscore hiper...\textunderscore  + \textunderscore enterose\textunderscore )}
\end{itemize}
Hipertrofia dos intestinos.
\section{Hiperestesia}
\begin{itemize}
\item {Grp. gram.:f.}
\end{itemize}
\begin{itemize}
\item {Proveniência:(Do gr. \textunderscore huper\textunderscore  + \textunderscore aisthesis\textunderscore )}
\end{itemize}
Sensibilidade excessiva e dolorosa.
\section{Hiperestesiado}
\begin{itemize}
\item {Grp. gram.:adj.}
\end{itemize}
\begin{itemize}
\item {Utilização:Neol.}
\end{itemize}
Que tem hiperestesia.
\section{Hiperestésico}
\begin{itemize}
\item {Grp. gram.:adj.}
\end{itemize}
\begin{itemize}
\item {Proveniência:(De \textunderscore hiperestesia\textunderscore )}
\end{itemize}
Diz-se do medicamento, que exaggera a sensibilidade geral ou especial; estimulante.
\section{Hiperfísico}
\begin{itemize}
\item {Grp. gram.:adj.}
\end{itemize}
\begin{itemize}
\item {Proveniência:(De \textunderscore huper...\textunderscore  + \textunderscore fisico\textunderscore )}
\end{itemize}
Superior á natureza; sobrenatural.
\section{Hipergenesia}
\begin{itemize}
\item {Grp. gram.:f.}
\end{itemize}
\begin{itemize}
\item {Proveniência:(Do gr. \textunderscore huper\textunderscore  + \textunderscore genesis\textunderscore )}
\end{itemize}
Desenvolvimento anormal de um elemento anatómico no seio de um tecido, ou de um tecido no seio de um órgão.
\section{Hipergenético}
\begin{itemize}
\item {Grp. gram.:adj.}
\end{itemize}
Relativo á hipergenesia.
\section{Hiperglicemia}
\begin{itemize}
\item {Grp. gram.:f.}
\end{itemize}
\begin{itemize}
\item {Proveniência:(Do gr. \textunderscore huper\textunderscore  + \textunderscore glukos\textunderscore  + \textunderscore kaima\textunderscore )}
\end{itemize}
Superabundância de glicose na urina.
\section{Hiperglobulia}
\begin{itemize}
\item {Grp. gram.:f.}
\end{itemize}
\begin{itemize}
\item {Proveniência:(De \textunderscore hiper...\textunderscore  + \textunderscore glóbulo\textunderscore )}
\end{itemize}
Aumento do número de hematias do sangue.
\section{Hipericáceas}
\begin{itemize}
\item {Grp. gram.:f. pl.}
\end{itemize}
O mesmo ou melhor que \textunderscore hipericineas\textunderscore .
\section{Hipericão}
\begin{itemize}
\item {Grp. gram.:m.}
\end{itemize}
\begin{itemize}
\item {Proveniência:(Gr. \textunderscore huperikon\textunderscore )}
\end{itemize}
Gênero de plantas lenhosas ou herbáceas.
\section{Hipericíneas}
\begin{itemize}
\item {Grp. gram.:f. pl.}
\end{itemize}
Família de plantas dicotiledóneas, que têm por tipo o hipericão.
\section{Hiperidrose}
\begin{itemize}
\item {Grp. gram.:f.}
\end{itemize}
\begin{itemize}
\item {Utilização:Med.}
\end{itemize}
\begin{itemize}
\item {Proveniência:(Do gr. \textunderscore huper\textunderscore  + \textunderscore hidros\textunderscore )}
\end{itemize}
Excessiva secreção de suor.
\section{Hiperinose}
\begin{itemize}
\item {Grp. gram.:f.}
\end{itemize}
\begin{itemize}
\item {Utilização:Med.}
\end{itemize}
\begin{itemize}
\item {Proveniência:(Do gr. \textunderscore huper\textunderscore  + \textunderscore inos\textunderscore )}
\end{itemize}
Excesso de fibrina.
\section{Hiperintelectualidade}
\begin{itemize}
\item {Grp. gram.:f.}
\end{itemize}
\begin{itemize}
\item {Proveniência:(De \textunderscore hiper...\textunderscore  + \textunderscore intelectualidade\textunderscore )}
\end{itemize}
Vastidão de qualidades intelectuaes.
\section{Hiperiodato}
\begin{itemize}
\item {Grp. gram.:m.}
\end{itemize}
Designação genérica dos saes de acido hiperiódico.
\section{Hiperiódico}
\begin{itemize}
\item {Grp. gram.:adj.}
\end{itemize}
\begin{itemize}
\item {Proveniência:(De \textunderscore hiper...\textunderscore  + \textunderscore iódico\textunderscore )}
\end{itemize}
Diz-se do ácido, que contém mais oxigênio que ácido iódico.
\section{Hiperleucocitose}
\begin{itemize}
\item {Grp. gram.:f.}
\end{itemize}
\begin{itemize}
\item {Utilização:Med.}
\end{itemize}
\begin{itemize}
\item {Proveniência:(De \textunderscore hiper...\textunderscore  + \textunderscore leucòcito\textunderscore )}
\end{itemize}
Excesso de leucócitos no sangue.
\section{Hiperlinfia}
\begin{itemize}
\item {Grp. gram.:f.}
\end{itemize}
\begin{itemize}
\item {Utilização:Med.}
\end{itemize}
\begin{itemize}
\item {Proveniência:(De \textunderscore hiper...\textunderscore  + \textunderscore linfa\textunderscore )}
\end{itemize}
Excesso de linfa.
\section{Hipermastia}
\begin{itemize}
\item {Grp. gram.:f.}
\end{itemize}
\begin{itemize}
\item {Utilização:Med.}
\end{itemize}
\begin{itemize}
\item {Proveniência:(Do gr. \textunderscore huper\textunderscore  + \textunderscore mastos\textunderscore )}
\end{itemize}
Hipertrofia da mama.
\section{Hipermetria}
\begin{itemize}
\item {Grp. gram.:f.}
\end{itemize}
\begin{itemize}
\item {Proveniência:(Do gr. \textunderscore huper\textunderscore  + \textunderscore metron\textunderscore )}
\end{itemize}
Separação de uma palavra composta, ficando parte no fim de um verso e outra parte no princípio do seguinte.
\section{Hipérmetro}
\begin{itemize}
\item {Grp. gram.:m.}
\end{itemize}
\begin{itemize}
\item {Proveniência:(Lat. \textunderscore hypermetrus\textunderscore )}
\end{itemize}
Verso hexâmetro, que termina por uma sílaba que sai além da medida do verso.
\section{Hipermetrope}
\begin{itemize}
\item {Grp. gram.:adj.}
\end{itemize}
Que tem hipermetropia.
\section{Hipermetropia}
\begin{itemize}
\item {Grp. gram.:f.}
\end{itemize}
\begin{itemize}
\item {Proveniência:(Do gr. \textunderscore huper\textunderscore  + \textunderscore metron\textunderscore  + \textunderscore ops\textunderscore )}
\end{itemize}
O mesmo que \textunderscore hiperopia\textunderscore .
\section{Hipermiopia}
\begin{itemize}
\item {Grp. gram.:f.}
\end{itemize}
\begin{itemize}
\item {Proveniência:(De \textunderscore huper...\textunderscore  + \textunderscore miopia\textunderscore )}
\end{itemize}
Miopia muito pronunciada.
\section{Hipermisticismo}
\begin{itemize}
\item {Grp. gram.:m.}
\end{itemize}
Misticismo exagerado ou elevado ao mais alto grau.
\section{Hipermístico}
\begin{itemize}
\item {Grp. gram.:adj.}
\end{itemize}
Excessivamente místico.
\section{Hipermnesia}
\begin{itemize}
\item {Grp. gram.:f.}
\end{itemize}
\begin{itemize}
\item {Proveniência:(Do gr. \textunderscore huper\textunderscore  + \textunderscore mnesis\textunderscore )}
\end{itemize}
Exaltação da memória.
\section{Hipérope}
\begin{itemize}
\item {Grp. gram.:adj.}
\end{itemize}
Que tem o defeito da hiperopia.
\section{Hiperopia}
\begin{itemize}
\item {Grp. gram.:f.}
\end{itemize}
\begin{itemize}
\item {Proveniência:(Do gr. \textunderscore huper\textunderscore  + \textunderscore ops\textunderscore )}
\end{itemize}
Defeito da vista, que póde sêr corrigido por vidros convexos.
\section{Hiperosmia}
\begin{itemize}
\item {Grp. gram.:f.}
\end{itemize}
\begin{itemize}
\item {Proveniência:(Do gr. \textunderscore huper\textunderscore  + \textunderscore osma\textunderscore )}
\end{itemize}
Excitação do olfato.
\section{Hiperostose}
\begin{itemize}
\item {Grp. gram.:f.}
\end{itemize}
\begin{itemize}
\item {Utilização:Med.}
\end{itemize}
\begin{itemize}
\item {Proveniência:(Do gr. \textunderscore huper\textunderscore  + \textunderscore osteon\textunderscore )}
\end{itemize}
Desenvolvimento anormal de certas partes ósseas do corpo.
\section{Hiperóxido}
\begin{itemize}
\item {Grp. gram.:m.}
\end{itemize}
\begin{itemize}
\item {Proveniência:(De \textunderscore huper...\textunderscore  + \textunderscore óxido\textunderscore )}
\end{itemize}
Óxido, que contém excessivo oxigênio.
\section{Hiperplasia}
\begin{itemize}
\item {Grp. gram.:f.}
\end{itemize}
\begin{itemize}
\item {Utilização:Physiol.}
\end{itemize}
\begin{itemize}
\item {Proveniência:(Do gr. \textunderscore huper\textunderscore  + \textunderscore plassein\textunderscore )}
\end{itemize}
Proliferação celular anormal, que origina inflamação ou tumores.
\section{Hiperplástico}
\begin{itemize}
\item {Grp. gram.:adj.}
\end{itemize}
\begin{itemize}
\item {Proveniência:(Do gr. \textunderscore huper\textunderscore  + \textunderscore plassein\textunderscore )}
\end{itemize}
Que tem plasticidade excessiva.
\section{Hipersalino}
\begin{itemize}
\item {Grp. gram.:adj.}
\end{itemize}
Muitissimo salino.
\section{Hipersarcose}
\begin{itemize}
\item {Grp. gram.:f.}
\end{itemize}
\begin{itemize}
\item {Utilização:Anat.}
\end{itemize}
\begin{itemize}
\item {Proveniência:(Lat. \textunderscore hypersarcosis\textunderscore )}
\end{itemize}
Excrescência de carne.
\section{Hipersecreção}
\begin{itemize}
\item {Grp. gram.:f.}
\end{itemize}
\begin{itemize}
\item {Utilização:Med.}
\end{itemize}
\begin{itemize}
\item {Proveniência:(De \textunderscore hiper...\textunderscore  + \textunderscore secreção\textunderscore )}
\end{itemize}
Excesso de secreção.
\section{Hiperstenia}
\begin{itemize}
\item {Grp. gram.:f.}
\end{itemize}
\begin{itemize}
\item {Utilização:Med.}
\end{itemize}
\begin{itemize}
\item {Proveniência:(Do gr. \textunderscore huper\textunderscore  + \textunderscore sthenos\textunderscore )}
\end{itemize}
Funcionamento exagerado de órgãos, aparelhos ou sistemas.
\section{Hiperstênico}
\begin{itemize}
\item {Grp. gram.:adj.}
\end{itemize}
\begin{itemize}
\item {Proveniência:(Do gr. \textunderscore huper\textunderscore  + \textunderscore sthenos\textunderscore )}
\end{itemize}
Diz-se do medicamento, que exagera a actividade do sistema nervoso.
\section{Hiperstílico}
\begin{itemize}
\item {Grp. gram.:adj.}
\end{itemize}
\begin{itemize}
\item {Utilização:Bot.}
\end{itemize}
\begin{itemize}
\item {Proveniência:(Do gr. \textunderscore huper\textunderscore  + \textunderscore stulos\textunderscore )}
\end{itemize}
Que se insere por cima do estilete.
\section{Hiperstómico}
\begin{itemize}
\item {Grp. gram.:adj.}
\end{itemize}
\begin{itemize}
\item {Utilização:Bot.}
\end{itemize}
\begin{itemize}
\item {Proveniência:(Do gr. \textunderscore huper\textunderscore  + \textunderscore stoma\textunderscore )}
\end{itemize}
Que se insere por cima do orifício do cálice.
\section{Hipertermia}
\begin{itemize}
\item {Grp. gram.:f.}
\end{itemize}
\begin{itemize}
\item {Utilização:Med.}
\end{itemize}
\begin{itemize}
\item {Proveniência:(Do gr. \textunderscore huper\textunderscore  + \textunderscore therme\textunderscore )}
\end{itemize}
Excessiva elevação de temperatura.
\section{Hipértese}
\begin{itemize}
\item {Grp. gram.:f.}
\end{itemize}
\begin{itemize}
\item {Utilização:Gram.}
\end{itemize}
\begin{itemize}
\item {Proveniência:(Lat. \textunderscore hyperthesis\textunderscore )}
\end{itemize}
Transposição de consoantes entre sílabas diversas, como em \textunderscore desvariado\textunderscore  e \textunderscore desvairado\textunderscore .
\section{Hipertiro}
\begin{itemize}
\item {Grp. gram.:m.}
\end{itemize}
\begin{itemize}
\item {Proveniência:(Gr. \textunderscore huperthuron\textunderscore )}
\end{itemize}
Friso ou cornija de uma porta.
\section{Hipertonia}
\begin{itemize}
\item {Grp. gram.:f.}
\end{itemize}
\begin{itemize}
\item {Proveniência:(De \textunderscore hiper...\textunderscore  + \textunderscore tom\textunderscore )}
\end{itemize}
Aumento de tensão no ôlho ou em qualquer outro órgão.
\section{Hipertónico}
\begin{itemize}
\item {Grp. gram.:adj.}
\end{itemize}
Relativo á hipertonia.
\section{Hipertricose}
\begin{itemize}
\item {Grp. gram.:f.}
\end{itemize}
\begin{itemize}
\item {Utilização:Med.}
\end{itemize}
\begin{itemize}
\item {Proveniência:(Do gr. \textunderscore huper\textunderscore  + \textunderscore trikhos\textunderscore )}
\end{itemize}
Produção exagerada de pêlos.
\section{Hymnographia}
\begin{itemize}
\item {Grp. gram.:f.}
\end{itemize}
\begin{itemize}
\item {Proveniência:(De \textunderscore hymnógrapho\textunderscore )}
\end{itemize}
Tratado bibliográphico dos hymnos.
\section{Hymnógrapho}
\begin{itemize}
\item {Grp. gram.:m.  e  adj.}
\end{itemize}
\begin{itemize}
\item {Proveniência:(Do gr. \textunderscore humnos\textunderscore  + \textunderscore graphein\textunderscore )}
\end{itemize}
O que compõe hymnos.
\section{Hymnologia}
\begin{itemize}
\item {Grp. gram.:f.}
\end{itemize}
\begin{itemize}
\item {Proveniência:(Do gr. \textunderscore humnos\textunderscore  + \textunderscore logos\textunderscore )}
\end{itemize}
Arte de compor hymnos.
Acto de recitar ou cantar hymnos.
\section{Hymnologista}
\begin{itemize}
\item {Grp. gram.:m.}
\end{itemize}
Apologista enthusiástico; panegyrista. Cf. Camillo, \textunderscore Serões\textunderscore , IV, 29.
(Cp. \textunderscore hymnologia\textunderscore )
\section{Hymnólogo}
\begin{itemize}
\item {Grp. gram.:m.}
\end{itemize}
\begin{itemize}
\item {Proveniência:(Lat. \textunderscore hymnologus\textunderscore )}
\end{itemize}
O mesmo que \textunderscore hymnista\textunderscore .
\section{Hyoglosso}
\begin{itemize}
\item {Grp. gram.:m.}
\end{itemize}
\begin{itemize}
\item {Utilização:Ant.}
\end{itemize}
Músculo par, que se liga ao osso hyoide e á língua.
\section{Hyoide}
\begin{itemize}
\item {Grp. gram.:m.}
\end{itemize}
Pequeno osso, entre a larynge e a base da língua.
(Contr. de \textunderscore upsiloide\textunderscore , de \textunderscore upsilon\textunderscore , n. gr. da letra \textunderscore y\textunderscore  + \textunderscore eidos\textunderscore )
\section{Hyoídeo}
\begin{itemize}
\item {Grp. gram.:adj.}
\end{itemize}
Relativo ao hyoide.
\section{Hyo-pharýngeo}
\begin{itemize}
\item {Grp. gram.:adj.}
\end{itemize}
Relativo ao osso hyoide e á pharynge.
\section{Hyoscýama}
\begin{itemize}
\item {Grp. gram.:f.}
\end{itemize}
\begin{itemize}
\item {Proveniência:(Do gr. \textunderscore huos\textunderscore  + \textunderscore kuamos\textunderscore )}
\end{itemize}
Nome scientífico do meimendro.
\section{Hyoscyamina}
\begin{itemize}
\item {Grp. gram.:f.}
\end{itemize}
\begin{itemize}
\item {Proveniência:(De \textunderscore hyosciama\textunderscore )}
\end{itemize}
Alcaloide, descoberto recentemente na alface.
Alcaloide, extrahido do meimendro.
\section{Hyoscýamo}
\begin{itemize}
\item {Grp. gram.:m.}
\end{itemize}
\begin{itemize}
\item {Proveniência:(Lat. \textunderscore hyoscyamus\textunderscore )}
\end{itemize}
O mesmo ou melhor que \textunderscore hyoscýama\textunderscore .
\section{Hyosternal}
\begin{itemize}
\item {Grp. gram.:m.}
\end{itemize}
\begin{itemize}
\item {Utilização:Anat.}
\end{itemize}
\begin{itemize}
\item {Proveniência:(De \textunderscore hyoide\textunderscore  e \textunderscore esternal\textunderscore )}
\end{itemize}
Terceira peça do esterno.
\section{Hypállage}
\begin{itemize}
\item {Grp. gram.:f.}
\end{itemize}
\begin{itemize}
\item {Proveniência:(Lat. \textunderscore hypallage\textunderscore )}
\end{itemize}
Figura de rhetórica, com que parece attribuír-se ás palavras de uma phrase o que pertence a outras palavras da mesma phrase.
\section{Hypantho}
\begin{itemize}
\item {Grp. gram.:m.}
\end{itemize}
\begin{itemize}
\item {Utilização:Bot.}
\end{itemize}
\begin{itemize}
\item {Proveniência:(Do gr. \textunderscore hupo\textunderscore  + \textunderscore anthos\textunderscore )}
\end{itemize}
Parte inferior do cálice.
Inflorescência, própria da figueira.
\section{Hypanthódio}
\begin{itemize}
\item {Grp. gram.:m.}
\end{itemize}
\begin{itemize}
\item {Proveniência:(Lat. \textunderscore hypanthodium\textunderscore )}
\end{itemize}
O mesmo que \textunderscore sýcono\textunderscore .
\section{Hýpata}
\begin{itemize}
\item {Grp. gram.:f.}
\end{itemize}
\begin{itemize}
\item {Utilização:Ant.}
\end{itemize}
\begin{itemize}
\item {Proveniência:(Lat. \textunderscore hypate\textunderscore )}
\end{itemize}
A corda mais grave da lyra e de outros instrumentos.
\section{Hypemia}
\begin{itemize}
\item {Grp. gram.:f.}
\end{itemize}
\begin{itemize}
\item {Utilização:Med.}
\end{itemize}
\begin{itemize}
\item {Proveniência:(Do gr. \textunderscore hupo\textunderscore  + \textunderscore haima\textunderscore )}
\end{itemize}
Anemia, localizada numa parte do organismo.
\section{Hyper...}
\begin{itemize}
\item {Grp. gram.:pref.}
\end{itemize}
\begin{itemize}
\item {Proveniência:(Gr. \textunderscore huper\textunderscore )}
\end{itemize}
(designativo de \textunderscore muito\textunderscore ; \textunderscore em alto gráu\textunderscore ; \textunderscore além\textunderscore )
\section{Hyperacidez}
\begin{itemize}
\item {Grp. gram.:f.}
\end{itemize}
Estado ou qualidade daquillo que é hyperácido.
\section{Hyperácido}
\begin{itemize}
\item {Grp. gram.:adj.}
\end{itemize}
\begin{itemize}
\item {Proveniência:(De \textunderscore hyper...\textunderscore  + \textunderscore ácido\textunderscore )}
\end{itemize}
Excessivamente ácido, (falando-se da urina ou de outros humores orgânicos).
\section{Hyperacusia}
\begin{itemize}
\item {Grp. gram.:f.}
\end{itemize}
\begin{itemize}
\item {Utilização:Med.}
\end{itemize}
\begin{itemize}
\item {Proveniência:(Do gr. \textunderscore huper\textunderscore  + \textunderscore acusis\textunderscore )}
\end{itemize}
Excitação auditiva.
Percepção dolorosa e confusa de certos sons, mormente dos elevados e agudos.
\section{Hyperacúsico}
\begin{itemize}
\item {Grp. gram.:adj.}
\end{itemize}
Relativo a hyperacusia.
\section{Hyperalbuminose}
\begin{itemize}
\item {Grp. gram.:f.}
\end{itemize}
\begin{itemize}
\item {Proveniência:(De \textunderscore hyper...\textunderscore  + \textunderscore albumina\textunderscore )}
\end{itemize}
Excesso de albumina no sangue.
\section{Hyperalgesia}
\begin{itemize}
\item {Grp. gram.:f.}
\end{itemize}
\begin{itemize}
\item {Utilização:Med.}
\end{itemize}
\begin{itemize}
\item {Proveniência:(Do gr. \textunderscore huper\textunderscore  + \textunderscore algesis\textunderscore )}
\end{itemize}
Exaggêro da sensibilidade á dôr.
\section{Hypérbato}
\begin{itemize}
\item {Grp. gram.:m.}
\end{itemize}
O mesmo ou melhor que \textunderscore hypérbaton\textunderscore .
\section{Hypérbaton}
\begin{itemize}
\item {Grp. gram.:m.}
\end{itemize}
\begin{itemize}
\item {Utilização:Gram.}
\end{itemize}
\begin{itemize}
\item {Proveniência:(Lat. \textunderscore hyperbaton\textunderscore )}
\end{itemize}
Transposição ou inversão da ordem natural das palavras ou das proposições.
\section{Hyperbibasmo}
\begin{itemize}
\item {Grp. gram.:m.}
\end{itemize}
Deslocação do accento tónico de uma palavra, tornando-se paroxýtona a que era proparoxýtona e vice-versa. Cf. J. V. Boscoli, \textunderscore Gram.\textunderscore  Exemplo: \textunderscore mýope\textunderscore , por \textunderscore myópe\textunderscore ; \textunderscore patêna\textunderscore , por \textunderscore pátena\textunderscore .
\section{Hypérbole}
\begin{itemize}
\item {Grp. gram.:f.}
\end{itemize}
\begin{itemize}
\item {Proveniência:(Lat. \textunderscore hyperbole\textunderscore )}
\end{itemize}
Figura de rhetórica, que exaggera ou deminue excessivamente a verdade das coisas, para que produzam maior impressão.
Curva geométrica, em que cada um dos pontos mantém igual distância de dois pontos fixos, chamados focos.--Alguns clássicos fazem masculino o termo. Cf. Filinto, XXII, 101 e 104.
\section{Hyperbolicamente}
\begin{itemize}
\item {Grp. gram.:adv.}
\end{itemize}
De modo hyperbólico; exaggeradamente.
\section{Hyperbólico}
\begin{itemize}
\item {Grp. gram.:adj.}
\end{itemize}
\begin{itemize}
\item {Utilização:Fig.}
\end{itemize}
\begin{itemize}
\item {Proveniência:(Lat. \textunderscore hyperbolicus\textunderscore )}
\end{itemize}
Relativo a hypérbole.
Exaggerado.
\section{Hyperboliforme}
\begin{itemize}
\item {Grp. gram.:adj.}
\end{itemize}
\begin{itemize}
\item {Utilização:Geom.}
\end{itemize}
\begin{itemize}
\item {Proveniência:(De \textunderscore hypérbole\textunderscore  + \textunderscore fórma\textunderscore )}
\end{itemize}
Que tem proximamente a fórma da hypérbole.
\section{Hyperbolismo}
\begin{itemize}
\item {Grp. gram.:m.}
\end{itemize}
\begin{itemize}
\item {Utilização:Neol.}
\end{itemize}
\begin{itemize}
\item {Proveniência:(De \textunderscore hypérbole\textunderscore )}
\end{itemize}
Emprego abusivo da hypérbole, em linguagem.
\section{Hyperbolizar}
\begin{itemize}
\item {Grp. gram.:v. i.}
\end{itemize}
Empregar hypérboles.
\section{Hyperboloide}
\begin{itemize}
\item {Grp. gram.:m.}
\end{itemize}
\begin{itemize}
\item {Proveniência:(Do gr. \textunderscore huperbole\textunderscore  + \textunderscore eidos\textunderscore )}
\end{itemize}
Sólido geométrico, produzido pela revolução de uma hypérbole.
Hyperbólico.
\section{Hyperbóreo}
\begin{itemize}
\item {Grp. gram.:adj.}
\end{itemize}
\begin{itemize}
\item {Grp. gram.:M. Pl.}
\end{itemize}
\begin{itemize}
\item {Proveniência:(Lat. \textunderscore hyperboreus\textunderscore )}
\end{itemize}
Setentrional.
Relativo ao Norte.
Que cresce em lugares muito frios, (falando-se de plantas).
Povos setentrionaes da América do Norte, como os Groenlandeses, etc.
\section{Hypercataléctico}
\begin{itemize}
\item {Grp. gram.:adj.}
\end{itemize}
\begin{itemize}
\item {Proveniência:(Lat. \textunderscore hypercatalecticus\textunderscore )}
\end{itemize}
Que tem uma sýllaba de mais, (falando-se de versos gregos e latinos).
\section{Hypercatalecto}
\begin{itemize}
\item {Grp. gram.:m.}
\end{itemize}
\begin{itemize}
\item {Proveniência:(Lat. \textunderscore hypercatalectus\textunderscore )}
\end{itemize}
Verso grego ou latino, com uma sýllaba de mais.
\section{Hypercatharse}
\begin{itemize}
\item {Grp. gram.:f.}
\end{itemize}
\begin{itemize}
\item {Utilização:Med.}
\end{itemize}
\begin{itemize}
\item {Utilização:Ant.}
\end{itemize}
\begin{itemize}
\item {Proveniência:(Do gr. \textunderscore huper\textunderscore  + \textunderscore cathairo\textunderscore )}
\end{itemize}
Purgação excessiva.
\section{Hyperceratose}
\begin{itemize}
\item {Grp. gram.:f.}
\end{itemize}
\begin{itemize}
\item {Utilização:Med.}
\end{itemize}
\begin{itemize}
\item {Proveniência:(Do gr. \textunderscore huper\textunderscore  + \textunderscore keras\textunderscore )}
\end{itemize}
Hypertrophia da córnea.
\section{Hypercerebração}
\begin{itemize}
\item {Grp. gram.:f.}
\end{itemize}
\begin{itemize}
\item {Proveniência:(De \textunderscore hyper...\textunderscore  + \textunderscore cérebro\textunderscore )}
\end{itemize}
Excessivo trabalho intellectual.
\section{Hyperchlorato}
\begin{itemize}
\item {Grp. gram.:m.}
\end{itemize}
\begin{itemize}
\item {Proveniência:(De \textunderscore hyper...\textunderscore  + \textunderscore chlorato\textunderscore )}
\end{itemize}
Sal, resultante da combinação do ácido hyperchlórico com uma base.
\section{Hyperchlorhydria}
\begin{itemize}
\item {Grp. gram.:f.}
\end{itemize}
\begin{itemize}
\item {Proveniência:(De \textunderscore hyper\textunderscore  + \textunderscore chlorýdrico\textunderscore )}
\end{itemize}
Excesso de ácido chlorýdrico no suco gástrico.
\section{Hyperchlórico}
\begin{itemize}
\item {Grp. gram.:adj.}
\end{itemize}
\begin{itemize}
\item {Proveniência:(De \textunderscore hyper...\textunderscore  + \textunderscore chlórico\textunderscore )}
\end{itemize}
Diz-se de um dos oxácidos do chloro.
\section{Hyperchroma}
\begin{itemize}
\item {Grp. gram.:m.}
\end{itemize}
\begin{itemize}
\item {Utilização:Med.}
\end{itemize}
\begin{itemize}
\item {Proveniência:(Do gr. \textunderscore huper\textunderscore  + \textunderscore khroma\textunderscore )}
\end{itemize}
Excrescência carnosa, junto da carúncula, no grande ângulo do ôlho.
\section{Hyperchromia}
\begin{itemize}
\item {Grp. gram.:f.}
\end{itemize}
\begin{itemize}
\item {Proveniência:(Do gr. \textunderscore huper\textunderscore  + \textunderscore khroma\textunderscore )}
\end{itemize}
Exaggêro da pigmentação da pelle.
\section{Hypercinesia}
\begin{itemize}
\item {Grp. gram.:f.}
\end{itemize}
\begin{itemize}
\item {Proveniência:(Do gr. \textunderscore huper\textunderscore  + \textunderscore kinesis\textunderscore )}
\end{itemize}
Excitação da motilidade; exaggêro de movimentos.
\section{Hypercrinia}
\begin{itemize}
\item {Grp. gram.:f.}
\end{itemize}
\begin{itemize}
\item {Utilização:Med.}
\end{itemize}
\begin{itemize}
\item {Proveniência:(Do gr. \textunderscore huper\textunderscore  + \textunderscore krinein\textunderscore )}
\end{itemize}
Secreção excessiva.
\section{Hypercrise}
\begin{itemize}
\item {Grp. gram.:f.}
\end{itemize}
\begin{itemize}
\item {Proveniência:(De \textunderscore hyper...\textunderscore  + \textunderscore crise\textunderscore )}
\end{itemize}
Crise pathológica, fóra do commum.
\section{Hypercrítico}
\begin{itemize}
\item {Grp. gram.:m.}
\end{itemize}
\begin{itemize}
\item {Grp. gram.:Adj.}
\end{itemize}
\begin{itemize}
\item {Proveniência:(De \textunderscore hyper\textunderscore  + \textunderscore crítico\textunderscore )}
\end{itemize}
Censor exaggerado, crítico que nada perdôa:«\textunderscore demasiou-se, dizem os hypercríticos, e disse-o eu já.\textunderscore »Castilho.
Que critica com exaggêro; relativo á crítica exaggerada. Cf. Herculano, \textunderscore Hist. de Port.\textunderscore , III, 161.
\section{Hyperdiacrise}
\begin{itemize}
\item {Grp. gram.:f.}
\end{itemize}
\begin{itemize}
\item {Proveniência:(Do gr. \textunderscore huper\textunderscore  + \textunderscore diakrisis\textunderscore )}
\end{itemize}
O mesmo que \textunderscore hypercrinia\textunderscore .
\section{Hyperdramático}
\begin{itemize}
\item {Grp. gram.:adj.}
\end{itemize}
\begin{itemize}
\item {Proveniência:(De \textunderscore hyper...\textunderscore  + \textunderscore dramático\textunderscore )}
\end{itemize}
Excessivamente dramático; em que se exaggeram os meios dramáticos.
\section{Hyperdulia}
\begin{itemize}
\item {Grp. gram.:f.}
\end{itemize}
\begin{itemize}
\item {Proveniência:(De \textunderscore hyper...\textunderscore  + \textunderscore dulia\textunderscore )}
\end{itemize}
Culto, que se presta especialmente á Virgem Maria.
\section{Hyperellíptico}
\begin{itemize}
\item {Grp. gram.:adj.}
\end{itemize}
\begin{itemize}
\item {Utilização:Mathem.}
\end{itemize}
\begin{itemize}
\item {Proveniência:(De \textunderscore hyper...\textunderscore  + \textunderscore ellíptico\textunderscore )}
\end{itemize}
Diz-se da funcção, formada por uma integral, cuja differencial contém, sob uma raiz do segundo grau, um polynómio de grau superior ao quarto.
\section{Hyperemia}
\begin{itemize}
\item {Grp. gram.:f.}
\end{itemize}
\begin{itemize}
\item {Proveniência:(Do gr. \textunderscore huper\textunderscore  + \textunderscore haima\textunderscore )}
\end{itemize}
Superabundância de sangue em qualquer parte do corpo.
\section{Hyperemiar}
\begin{itemize}
\item {Grp. gram.:v. t.}
\end{itemize}
Causar hyperhemia em.
\section{Hyperenterose}
\begin{itemize}
\item {Grp. gram.:f.}
\end{itemize}
\begin{itemize}
\item {Proveniência:(De \textunderscore hyper...\textunderscore  + \textunderscore enterose\textunderscore )}
\end{itemize}
Hypertrophia dos intestinos.
\section{Hyperesthesia}
\begin{itemize}
\item {Grp. gram.:f.}
\end{itemize}
\begin{itemize}
\item {Proveniência:(Do gr. \textunderscore huper\textunderscore  + \textunderscore aisthesis\textunderscore )}
\end{itemize}
Sensibilidade excessiva e dolorosa.
\section{Hyperesthesiado}
\begin{itemize}
\item {Grp. gram.:adj.}
\end{itemize}
\begin{itemize}
\item {Utilização:Neol.}
\end{itemize}
Que tem hyperesthesia.
\section{Hyperesthésico}
\begin{itemize}
\item {Grp. gram.:adj.}
\end{itemize}
\begin{itemize}
\item {Proveniência:(De \textunderscore hyperesthesia\textunderscore )}
\end{itemize}
Diz-se do medicamento, que exaggera a sensibilidade geral ou especial; estimulante.
\section{Hypergenesia}
\begin{itemize}
\item {Grp. gram.:f.}
\end{itemize}
\begin{itemize}
\item {Proveniência:(Do gr. \textunderscore huper\textunderscore  + \textunderscore genesis\textunderscore )}
\end{itemize}
Desenvolvimento anormal de um elemento anatómico no seio de um tecido, ou de um tecido no seio de um órgão.
\section{Hypergenético}
\begin{itemize}
\item {Grp. gram.:adj.}
\end{itemize}
Relativo á hypergenesia.
\section{Hyperglobulia}
\begin{itemize}
\item {Grp. gram.:f.}
\end{itemize}
\begin{itemize}
\item {Proveniência:(De \textunderscore hyper...\textunderscore  + \textunderscore glóbulo\textunderscore )}
\end{itemize}
Aumento do número de hematias do sangue.
\section{Hyperglycemia}
\begin{itemize}
\item {Grp. gram.:f.}
\end{itemize}
\begin{itemize}
\item {Proveniência:(Do gr. \textunderscore huper\textunderscore  + \textunderscore glukos\textunderscore  + \textunderscore kaima\textunderscore )}
\end{itemize}
Superabundância de glycose na urina.
\section{Hypericáceas}
\begin{itemize}
\item {Grp. gram.:f. pl.}
\end{itemize}
O mesmo ou melhor que \textunderscore hypericineas\textunderscore .
\section{Hypericão}
\begin{itemize}
\item {Grp. gram.:m.}
\end{itemize}
\begin{itemize}
\item {Proveniência:(Gr. \textunderscore huperikon\textunderscore )}
\end{itemize}
Gênero de plantas lenhosas ou herbáceas.
\section{Hypericíneas}
\begin{itemize}
\item {Grp. gram.:f. pl.}
\end{itemize}
Família de plantas dicotyledóneas, que têm por typo o hypericão.
\section{Hyperidrose}
\begin{itemize}
\item {Grp. gram.:f.}
\end{itemize}
\begin{itemize}
\item {Utilização:Med.}
\end{itemize}
\begin{itemize}
\item {Proveniência:(Do gr. \textunderscore huper\textunderscore  + \textunderscore hidros\textunderscore )}
\end{itemize}
Excessiva secreção de suor.
\section{Hyperinose}
\begin{itemize}
\item {Grp. gram.:f.}
\end{itemize}
\begin{itemize}
\item {Utilização:Med.}
\end{itemize}
\begin{itemize}
\item {Proveniência:(Do gr. \textunderscore huper\textunderscore  + \textunderscore inos\textunderscore )}
\end{itemize}
Excesso de fibrina.
\section{Hyperintellectualidade}
\begin{itemize}
\item {Grp. gram.:f.}
\end{itemize}
\begin{itemize}
\item {Proveniência:(De \textunderscore hyper...\textunderscore  + \textunderscore intellectualidade\textunderscore )}
\end{itemize}
Vastidão de qualidades intellectuaes.
\section{Hyperiodato}
\begin{itemize}
\item {Grp. gram.:m.}
\end{itemize}
Designação genérica dos saes de acido hyperiódico.
\section{Hyperiódico}
\begin{itemize}
\item {Grp. gram.:adj.}
\end{itemize}
\begin{itemize}
\item {Proveniência:(De \textunderscore hyper...\textunderscore  + \textunderscore iódico\textunderscore )}
\end{itemize}
Diz-se do ácido, que contém mais oxygênio que ácido iódico.
\section{Hyperleucocytose}
\begin{itemize}
\item {Grp. gram.:f.}
\end{itemize}
\begin{itemize}
\item {Utilização:Med.}
\end{itemize}
\begin{itemize}
\item {Proveniência:(De \textunderscore hyper...\textunderscore  + \textunderscore leucòcyto\textunderscore )}
\end{itemize}
Excesso de leucócytos no sangue.
\section{Hyperlymphia}
\begin{itemize}
\item {Grp. gram.:f.}
\end{itemize}
\begin{itemize}
\item {Utilização:Med.}
\end{itemize}
\begin{itemize}
\item {Proveniência:(De \textunderscore hyper...\textunderscore  + \textunderscore lympha\textunderscore )}
\end{itemize}
Excesso de lympha.
\section{Hypermastia}
\begin{itemize}
\item {Grp. gram.:f.}
\end{itemize}
\begin{itemize}
\item {Utilização:Med.}
\end{itemize}
\begin{itemize}
\item {Proveniência:(Do gr. \textunderscore huper\textunderscore  + \textunderscore mastos\textunderscore )}
\end{itemize}
Hypertrophia da mama.
\section{Hypermetria}
\begin{itemize}
\item {Grp. gram.:f.}
\end{itemize}
\begin{itemize}
\item {Proveniência:(Do gr. \textunderscore huper\textunderscore  + \textunderscore metron\textunderscore )}
\end{itemize}
Separação de uma palavra composta, ficando parte no fim de um verso e outra parte no princípio do seguinte.
\section{Hypérmetro}
\begin{itemize}
\item {Grp. gram.:m.}
\end{itemize}
\begin{itemize}
\item {Proveniência:(Lat. \textunderscore hypermetrus\textunderscore )}
\end{itemize}
Verso hexâmetro, que termina por uma sýllaba que sai além da medida do verso.
\section{Hypermetrope}
\begin{itemize}
\item {Grp. gram.:adj.}
\end{itemize}
Que tem hypermetropia.
\section{Hypermetropia}
\begin{itemize}
\item {Grp. gram.:f.}
\end{itemize}
\begin{itemize}
\item {Proveniência:(Do gr. \textunderscore huper\textunderscore  + \textunderscore metron\textunderscore  + \textunderscore ops\textunderscore )}
\end{itemize}
O mesmo que \textunderscore hyperopia\textunderscore .
\section{Hypermnesia}
\begin{itemize}
\item {Grp. gram.:f.}
\end{itemize}
\begin{itemize}
\item {Proveniência:(Do gr. \textunderscore huper\textunderscore  + \textunderscore mnesis\textunderscore )}
\end{itemize}
Exaltação da memória.
\section{Hypermyopia}
\begin{itemize}
\item {Grp. gram.:f.}
\end{itemize}
\begin{itemize}
\item {Proveniência:(De \textunderscore huper...\textunderscore  + \textunderscore myopia\textunderscore )}
\end{itemize}
Myopia muito pronunciada.
\section{Hypermysticismo}
\begin{itemize}
\item {Grp. gram.:m.}
\end{itemize}
Mysticismo exaggerado ou elevado ao mais alto grau.
\section{Hypermýstico}
\begin{itemize}
\item {Grp. gram.:adj.}
\end{itemize}
Excessivamente mýstico.
\section{Hypérope}
\begin{itemize}
\item {Grp. gram.:adj.}
\end{itemize}
Que tem o defeito da hyperopia.
\section{Hyperopia}
\begin{itemize}
\item {Grp. gram.:f.}
\end{itemize}
\begin{itemize}
\item {Proveniência:(Do gr. \textunderscore huper\textunderscore  + \textunderscore ops\textunderscore )}
\end{itemize}
Defeito da vista, que póde sêr corrigido por vidros convexos.
\section{Hyperosmia}
\begin{itemize}
\item {Grp. gram.:f.}
\end{itemize}
\begin{itemize}
\item {Proveniência:(Do gr. \textunderscore huper\textunderscore  + \textunderscore osma\textunderscore )}
\end{itemize}
Excitação do olfato.
\section{Hyperostose}
\begin{itemize}
\item {Grp. gram.:f.}
\end{itemize}
\begin{itemize}
\item {Utilização:Med.}
\end{itemize}
\begin{itemize}
\item {Proveniência:(Do gr. \textunderscore huper\textunderscore  + \textunderscore osteon\textunderscore )}
\end{itemize}
Desenvolvimento anormal de certas partes ósseas do corpo.
\section{Hyperóxydo}
\begin{itemize}
\item {Grp. gram.:m.}
\end{itemize}
\begin{itemize}
\item {Proveniência:(De \textunderscore huper...\textunderscore  + \textunderscore óxydo\textunderscore )}
\end{itemize}
Óxydo, que contém excessivo oxygênio.
\section{Hyperphýsico}
\begin{itemize}
\item {Grp. gram.:adj.}
\end{itemize}
\begin{itemize}
\item {Proveniência:(De \textunderscore huper...\textunderscore  + \textunderscore phýsico\textunderscore )}
\end{itemize}
Superior á natureza; sobrenatural.
\section{Hyperplasia}
\begin{itemize}
\item {Grp. gram.:f.}
\end{itemize}
\begin{itemize}
\item {Utilização:Physiol.}
\end{itemize}
\begin{itemize}
\item {Proveniência:(Do gr. \textunderscore huper\textunderscore  + \textunderscore plassein\textunderscore )}
\end{itemize}
Proliferação cellular anormal, que origina inflammação ou tumores.
\section{Hyperplástico}
\begin{itemize}
\item {Grp. gram.:adj.}
\end{itemize}
\begin{itemize}
\item {Proveniência:(Do gr. \textunderscore huper\textunderscore  + \textunderscore plassein\textunderscore )}
\end{itemize}
Que tem plasticidade excessiva.
\section{Hypersalino}
\begin{itemize}
\item {Grp. gram.:adj.}
\end{itemize}
Muitissimo salino.
\section{Hypersarcose}
\begin{itemize}
\item {Grp. gram.:f.}
\end{itemize}
\begin{itemize}
\item {Utilização:Anat.}
\end{itemize}
\begin{itemize}
\item {Proveniência:(Lat. \textunderscore hypersarcosis\textunderscore )}
\end{itemize}
Excrescência de carne.
\section{Hypersecreção}
\begin{itemize}
\item {Grp. gram.:f.}
\end{itemize}
\begin{itemize}
\item {Utilização:Med.}
\end{itemize}
\begin{itemize}
\item {Proveniência:(De \textunderscore hyper...\textunderscore  + \textunderscore secreção\textunderscore )}
\end{itemize}
Excesso de secreção.
\section{Hypersthenia}
\begin{itemize}
\item {Grp. gram.:f.}
\end{itemize}
\begin{itemize}
\item {Utilização:Med.}
\end{itemize}
\begin{itemize}
\item {Proveniência:(Do gr. \textunderscore huper\textunderscore  + \textunderscore sthenos\textunderscore )}
\end{itemize}
Funccionamento exaggerado de órgãos, apparelhos ou systemas.
\section{Hypersthênico}
\begin{itemize}
\item {Grp. gram.:adj.}
\end{itemize}
\begin{itemize}
\item {Proveniência:(Do gr. \textunderscore huper\textunderscore  + \textunderscore sthenos\textunderscore )}
\end{itemize}
Diz-se do medicamento, que exaggera a actividade do systema nervoso.
\section{Hyperstílico}
\begin{itemize}
\item {Grp. gram.:adj.}
\end{itemize}
\begin{itemize}
\item {Utilização:Bot.}
\end{itemize}
\begin{itemize}
\item {Proveniência:(Do gr. \textunderscore huper\textunderscore  + \textunderscore stulos\textunderscore )}
\end{itemize}
Que se insere por cima do estilete.
\section{Hyperstómico}
\begin{itemize}
\item {Grp. gram.:adj.}
\end{itemize}
\begin{itemize}
\item {Utilização:Bot.}
\end{itemize}
\begin{itemize}
\item {Proveniência:(Do gr. \textunderscore huper\textunderscore  + \textunderscore stoma\textunderscore )}
\end{itemize}
Que se insere por cima do orifício do cálice.
\section{Hyperthermia}
\begin{itemize}
\item {Grp. gram.:f.}
\end{itemize}
\begin{itemize}
\item {Utilização:Med.}
\end{itemize}
\begin{itemize}
\item {Proveniência:(Do gr. \textunderscore huper\textunderscore  + \textunderscore therme\textunderscore )}
\end{itemize}
Excessiva elevação de temperatura.
\section{Hypérthese}
\begin{itemize}
\item {Grp. gram.:f.}
\end{itemize}
\begin{itemize}
\item {Utilização:Gram.}
\end{itemize}
\begin{itemize}
\item {Proveniência:(Lat. \textunderscore hyperthesis\textunderscore )}
\end{itemize}
Transposição de consoantes entre sýllabas diversas, como em \textunderscore desvariado\textunderscore  e \textunderscore desvairado\textunderscore .
\section{Hyperthyro}
\begin{itemize}
\item {Grp. gram.:m.}
\end{itemize}
\begin{itemize}
\item {Proveniência:(Gr. \textunderscore huperthuron\textunderscore )}
\end{itemize}
Friso ou cornija de uma porta.
\section{Hypertonia}
\begin{itemize}
\item {Grp. gram.:f.}
\end{itemize}
\begin{itemize}
\item {Proveniência:(De \textunderscore hyper...\textunderscore  + \textunderscore tom\textunderscore )}
\end{itemize}
Aumento de tensão no ôlho ou em qualquer outro órgão.
\section{Hypertónico}
\begin{itemize}
\item {Grp. gram.:adj.}
\end{itemize}
Relativo á hypertonia.
\section{Hypertrichose}
\begin{itemize}
\item {fónica:co}
\end{itemize}
\begin{itemize}
\item {Grp. gram.:f.}
\end{itemize}
\begin{itemize}
\item {Utilização:Med.}
\end{itemize}
\begin{itemize}
\item {Proveniência:(Do gr. \textunderscore huper\textunderscore  + \textunderscore trikhos\textunderscore )}
\end{itemize}
Producção exaggerada de pêlos.
\section{Hifema}
\begin{itemize}
\item {Grp. gram.:m.}
\end{itemize}
\begin{itemize}
\item {Utilização:Med.}
\end{itemize}
\begin{itemize}
\item {Proveniência:(Do gr. \textunderscore hupo...\textunderscore  + \textunderscore haima\textunderscore )}
\end{itemize}
Sangue, na câmara anterior do ôlho.
\section{Hifemia}
\begin{itemize}
\item {Grp. gram.:f.}
\end{itemize}
\begin{itemize}
\item {Utilização:Med.}
\end{itemize}
\begin{itemize}
\item {Proveniência:(Do gr. \textunderscore hupo...\textunderscore  + \textunderscore haima\textunderscore )}
\end{itemize}
Deminuição do sangue.
\section{Hífen}
\begin{itemize}
\item {Grp. gram.:m.}
\end{itemize}
\begin{itemize}
\item {Proveniência:(Lat. \textunderscore hyphen\textunderscore )}
\end{itemize}
Traço de união.
Sinal, com que se ligam palavras.
\section{Hifósporos}
\begin{itemize}
\item {Grp. gram.:adj. Pl.}
\end{itemize}
\begin{itemize}
\item {Proveniência:(Do gr. \textunderscore huphe\textunderscore  + \textunderscore sporos\textunderscore )}
\end{itemize}
Diz-se dos líchens, que têm fórma de filamentos.
\section{Hipertrofia}
\begin{itemize}
\item {Grp. gram.:f.}
\end{itemize}
\begin{itemize}
\item {Proveniência:(Do gr. \textunderscore huper\textunderscore  + \textunderscore trophe\textunderscore )}
\end{itemize}
Desenvolvimento excessivo de um órgão ou de uma parte de um órgão, sem alteração real do seu tecido.
\section{Hipertrofiado}
\begin{itemize}
\item {Grp. gram.:adj.}
\end{itemize}
Que tem hipertrofia.
\section{Hipertrofiar}
\begin{itemize}
\item {Grp. gram.:v. t.}
\end{itemize}
Produzir hipertrofia em.
\section{Hipetro}
\begin{itemize}
\item {Grp. gram.:m.}
\end{itemize}
\begin{itemize}
\item {Utilização:Ant.}
\end{itemize}
\begin{itemize}
\item {Proveniência:(Do gr. \textunderscore hupaithron\textunderscore )}
\end{itemize}
Templo descoberto.
\section{Hipiemia}
\begin{itemize}
\item {Grp. gram.:f.}
\end{itemize}
\begin{itemize}
\item {Proveniência:(Do gr. \textunderscore hupos\textunderscore  + \textunderscore haima\textunderscore )}
\end{itemize}
Hemorragia anal.
\section{Hipnagógico}
\begin{itemize}
\item {Grp. gram.:adj.}
\end{itemize}
\begin{itemize}
\item {Proveniência:(Do gr. \textunderscore hupnos\textunderscore  + \textunderscore agogos\textunderscore )}
\end{itemize}
Que produz sono.
Diz-se das alucinações e visões, que se têm, ao cair no sono.
\section{Hipnal}
\begin{itemize}
\item {Grp. gram.:m.}
\end{itemize}
\begin{itemize}
\item {Proveniência:(Lat. \textunderscore hypnale\textunderscore )}
\end{itemize}
Designação antiga de uma serpente venenosa, da qual se dizia que matava produzindo sono.
Medicamento hipnótico e antineurálgico.
\section{Hipniatria}
\begin{itemize}
\item {Grp. gram.:f.}
\end{itemize}
Tratamento de doenças, segundo as indicações do hipniatro.
\section{Hipniatro}
\begin{itemize}
\item {Grp. gram.:m.}
\end{itemize}
\begin{itemize}
\item {Proveniência:(Do gr. \textunderscore hupnos\textunderscore  + \textunderscore iatros\textunderscore )}
\end{itemize}
Sonâmbulo, que, durante o sono magnético, indica remédios para as suas doenças ou para as alheias.
\section{Hipnóbata}
\begin{itemize}
\item {Grp. gram.:m.}
\end{itemize}
\begin{itemize}
\item {Utilização:P. us.}
\end{itemize}
\begin{itemize}
\item {Proveniência:(Do gr. \textunderscore hupnos\textunderscore  + \textunderscore bainein\textunderscore )}
\end{itemize}
O mesmo que \textunderscore sonâmbulo\textunderscore .
\section{Hipnoblepsia}
\begin{itemize}
\item {Grp. gram.:f.}
\end{itemize}
\begin{itemize}
\item {Proveniência:(Do gr. \textunderscore hupnos\textunderscore  + \textunderscore blepein\textunderscore )}
\end{itemize}
Sonâmbulismo lúcido.
\section{Hipnococo}
\begin{itemize}
\item {Grp. gram.:m.}
\end{itemize}
\begin{itemize}
\item {Proveniência:(Do gr. \textunderscore hupnos\textunderscore  + \textunderscore kokkos\textunderscore )}
\end{itemize}
Bacilo da hipnose, transição entre o estreplococo e o diplococo.
\section{Hipnofobia}
\begin{itemize}
\item {Grp. gram.:f.}
\end{itemize}
Mêdo de dormir.
Terror, que advém durante o somno.
(Cp. \textunderscore hipnófobo\textunderscore )
\section{Hipnófobo}
\begin{itemize}
\item {Grp. gram.:m.}
\end{itemize}
\begin{itemize}
\item {Proveniência:(Do gr. \textunderscore hupnos\textunderscore  + \textunderscore phobein\textunderscore )}
\end{itemize}
Aquele que tem hipnofobia.
\section{Hipnofono}
\begin{itemize}
\item {Grp. gram.:m.}
\end{itemize}
\begin{itemize}
\item {Proveniência:(Do gr. \textunderscore hupnos\textunderscore  + \textunderscore phone\textunderscore )}
\end{itemize}
Aquele que fala durante o somno magnético.
\section{Hipnofrenose}
\begin{itemize}
\item {Grp. gram.:f.}
\end{itemize}
\begin{itemize}
\item {Proveniência:(Do gr. \textunderscore hupnos\textunderscore  + \textunderscore phren\textunderscore )}
\end{itemize}
O mesmo que \textunderscore sonambulismo\textunderscore .
\section{Hipnógeno}
\begin{itemize}
\item {Grp. gram.:adj.}
\end{itemize}
\begin{itemize}
\item {Proveniência:(Do gr. \textunderscore hupnos\textunderscore  + \textunderscore genos\textunderscore )}
\end{itemize}
Que produz sono.
\section{Hipnografia}
\begin{itemize}
\item {Grp. gram.:f.}
\end{itemize}
\begin{itemize}
\item {Proveniência:(Do gr. \textunderscore hupnos\textunderscore  + \textunderscore graphein\textunderscore )}
\end{itemize}
Descripção do sono.
\section{Hipnologia}
\begin{itemize}
\item {Grp. gram.:f.}
\end{itemize}
\begin{itemize}
\item {Proveniência:(Do gr. \textunderscore hupnos\textunderscore  + \textunderscore logos\textunderscore )}
\end{itemize}
Tratado á cêrca do sono.
\section{Hipnológico}
\begin{itemize}
\item {Grp. gram.:adj.}
\end{itemize}
Relativo á hipnologia.
\section{Hipnólogo}
\begin{itemize}
\item {Grp. gram.:m.}
\end{itemize}
Tratadista de hipnologia.
\section{Hipnome}
\begin{itemize}
\item {Grp. gram.:m.}
\end{itemize}
O mesmo que \textunderscore cloral\textunderscore .
\section{Hipnopatia}
\begin{itemize}
\item {Grp. gram.:f.}
\end{itemize}
\begin{itemize}
\item {Proveniência:(Do gr. \textunderscore hupnos\textunderscore  + \textunderscore pathos\textunderscore )}
\end{itemize}
O mesmo que \textunderscore hipnosia\textunderscore .
\section{Hipnose}
\begin{itemize}
\item {Grp. gram.:f.}
\end{itemize}
\begin{itemize}
\item {Proveniência:(Do gr. \textunderscore hupnos\textunderscore )}
\end{itemize}
Estado particular, caracterizado pelo sono nervoso e pela sugestão.
\section{Hipnosia}
\begin{itemize}
\item {Grp. gram.:f.}
\end{itemize}
Doença do sono.
O mesmo que \textunderscore hipnose\textunderscore .
\section{Hipnótico}
\begin{itemize}
\item {Grp. gram.:adj.}
\end{itemize}
\begin{itemize}
\item {Grp. gram.:M.}
\end{itemize}
\begin{itemize}
\item {Proveniência:(Lat. \textunderscore hypnoticus\textunderscore )}
\end{itemize}
Relativo á hipnose; que produz sono.
Aquilo que produz sono; narcótico.
\section{Hipnotismo}
\begin{itemize}
\item {Grp. gram.:m.}
\end{itemize}
\begin{itemize}
\item {Proveniência:(Do rad. de \textunderscore hipnótico\textunderscore )}
\end{itemize}
Processo, para produzir somno, fazendo fixar a vista num objecto brilhante, a pouca distância dos olhos, ou empregando outros meios.
Sono, provocado por êsse processo.
\section{Hipnotista}
\begin{itemize}
\item {Grp. gram.:m.}
\end{itemize}
Aquele que pratíca o hipnotismo.
\section{Hipnotização}
\begin{itemize}
\item {Grp. gram.:f.}
\end{itemize}
Acto ou efeito de hipnotizar.
\section{Hipnotizador}
\begin{itemize}
\item {Grp. gram.:m.}
\end{itemize}
Aquele que hipnotiza.
\section{Hipnotizar}
\begin{itemize}
\item {Grp. gram.:v. t.}
\end{itemize}
\begin{itemize}
\item {Proveniência:(De \textunderscore hipnótico\textunderscore )}
\end{itemize}
Produzir o hipnotismo em.
\section{Hipnotoxina}
\begin{itemize}
\item {fónica:csi}
\end{itemize}
\begin{itemize}
\item {Grp. gram.:f.}
\end{itemize}
Veneno, extraido dos tentáculos da fisália. Cf. \textunderscore Jorn. do Comm.\textunderscore , do Rio de 2-V-902.
\section{Hipo...}
\begin{itemize}
\item {Grp. gram.:pref.}
\end{itemize}
\begin{itemize}
\item {Proveniência:(Do gr. \textunderscore hupo\textunderscore )}
\end{itemize}
(designativo de \textunderscore deminuição\textunderscore ; \textunderscore grau inferior\textunderscore ; \textunderscore debaixo\textunderscore )
\section{Hipoacusia}
\begin{itemize}
\item {Grp. gram.:f.}
\end{itemize}
\begin{itemize}
\item {Utilização:Med.}
\end{itemize}
Deminuição do sentido auditivo.
\section{Hipoalgesia}
\begin{itemize}
\item {Grp. gram.:f.}
\end{itemize}
\begin{itemize}
\item {Proveniência:(Do gr. \textunderscore hupo\textunderscore  + \textunderscore algesis\textunderscore )}
\end{itemize}
Deminuição da sensibilidade á dôr.
\section{Hipoazótico}
\begin{itemize}
\item {Grp. gram.:adj.}
\end{itemize}
\begin{itemize}
\item {Proveniência:(De \textunderscore hipo...\textunderscore  + \textunderscore azotato\textunderscore )}
\end{itemize}
Diz-se de um ácido, que se obtém pela destilação do azotato de chumbo sêco.
\section{Hipoazoturia}
\begin{itemize}
\item {Grp. gram.:f.}
\end{itemize}
\begin{itemize}
\item {Proveniência:(De \textunderscore hipo...\textunderscore  + \textunderscore azoturia\textunderscore )}
\end{itemize}
Deminuição do azote, eliminado pela urina.
\section{Hipoblasto}
\begin{itemize}
\item {Grp. gram.:m.}
\end{itemize}
\begin{itemize}
\item {Proveniência:(Do gr. \textunderscore hupo\textunderscore  + \textunderscore blastos\textunderscore )}
\end{itemize}
Folheto interior do blastoderme.
\section{Hipobrânquio}
\begin{itemize}
\item {Grp. gram.:adj.}
\end{itemize}
\begin{itemize}
\item {Proveniência:(De \textunderscore hipo...\textunderscore  + \textunderscore brânquias\textunderscore )}
\end{itemize}
Que tem as brânquias por baixo do corpo.
\section{Hipocardo}
\begin{itemize}
\item {Grp. gram.:m.}
\end{itemize}
Carda aperfeiçoada, inventada em 1855 por um fabricante de tecidos em Colmar.
\section{Hipocarístico}
\begin{itemize}
\item {Grp. gram.:adj.}
\end{itemize}
\begin{itemize}
\item {Utilização:Gram.}
\end{itemize}
Diz-se dos vocábulos familiares ou infantis, sobretudo daqueles em que há duplicação de sílaba: \textunderscore papá\textunderscore , \textunderscore Lulu\textunderscore , \textunderscore Lili\textunderscore , etc. Cf. Júl. Ribeiro, \textunderscore Diccion. Gram.\textunderscore 
\section{Hipocarpo}
\begin{itemize}
\item {Grp. gram.:m.}
\end{itemize}
\begin{itemize}
\item {Utilização:Bot.}
\end{itemize}
\begin{itemize}
\item {Proveniência:(Do gr. \textunderscore hupo\textunderscore  + \textunderscore karpos\textunderscore )}
\end{itemize}
Parte da planta, em que assenta o fruto.
\section{Hipocausto}
\begin{itemize}
\item {Grp. gram.:m.}
\end{itemize}
\begin{itemize}
\item {Utilização:Ant.}
\end{itemize}
\begin{itemize}
\item {Proveniência:(Gr. \textunderscore hupokauston\textunderscore )}
\end{itemize}
Forno subterrâneo, nas antigas termas.
\section{Hipocinético}
\begin{itemize}
\item {Grp. gram.:adj.}
\end{itemize}
\begin{itemize}
\item {Proveniência:(Do gr. \textunderscore hupo\textunderscore  + \textunderscore kinetikos\textunderscore )}
\end{itemize}
Diz-se dos medicamentos, que modificam a acção nervosa.
\section{Hipoclórico}
\begin{itemize}
\item {Grp. gram.:adj.}
\end{itemize}
\begin{itemize}
\item {Proveniência:(De \textunderscore hipo...\textunderscore  + \textunderscore cloro\textunderscore )}
\end{itemize}
Diz-se de um ácido, que se obtém, decompondo o cloreto de potassa pelo ácido sulfúrico.
\section{Hipocloridria}
\begin{itemize}
\item {Grp. gram.:f.}
\end{itemize}
\begin{itemize}
\item {Utilização:Med.}
\end{itemize}
\begin{itemize}
\item {Proveniência:(De \textunderscore hipo...\textunderscore  + \textunderscore clorídrico\textunderscore )}
\end{itemize}
Deminuição da acidez do suco gástrico.
\section{Hipoclorito}
\begin{itemize}
\item {Grp. gram.:m.}
\end{itemize}
\begin{itemize}
\item {Utilização:Chím.}
\end{itemize}
Nome de um sal.
\section{Hipocloroso}
\begin{itemize}
\item {Grp. gram.:adj.}
\end{itemize}
\begin{itemize}
\item {Proveniência:(De \textunderscore hipo...\textunderscore  + \textunderscore cloro\textunderscore )}
\end{itemize}
Diz-se de um ácido, que é um dos oxácidos do cloro.
\section{Hipocicloide}
\begin{itemize}
\item {Grp. gram.:f.}
\end{itemize}
\begin{itemize}
\item {Utilização:Mathem.}
\end{itemize}
\begin{itemize}
\item {Proveniência:(De \textunderscore hipo...\textunderscore  + \textunderscore ciclo\textunderscore )}
\end{itemize}
Espécie de curva transcendente.
\section{Hipociste}
\begin{itemize}
\item {Grp. gram.:f.}
\end{itemize}
Planta parasita, (\textunderscore cytinus hypocistis\textunderscore , Lin.).
\section{Hipocofose}
\begin{itemize}
\item {Grp. gram.:f.}
\end{itemize}
\begin{itemize}
\item {Proveniência:(Do gr. \textunderscore hupo\textunderscore  + \textunderscore kophosis\textunderscore )}
\end{itemize}
Surdez completa.
\section{Hipocondria}
\begin{itemize}
\item {Grp. gram.:f.}
\end{itemize}
\begin{itemize}
\item {Proveniência:(De \textunderscore hipocôndrio\textunderscore )}
\end{itemize}
Doença nervosa, que faz crer na existência de várias enfermidades, produzindo habitual tristeza.
Melancolia; misantropia.
\section{Hipocondríaco}
\begin{itemize}
\item {Grp. gram.:m.}
\end{itemize}
\begin{itemize}
\item {Grp. gram.:Adj.}
\end{itemize}
Que tem hipocondria.
Relativo a hipocondria.
Que sofre hipocondria.
\section{Hipocôndrio}
\begin{itemize}
\item {Grp. gram.:m.}
\end{itemize}
\begin{itemize}
\item {Proveniência:(Gr. \textunderscore hupokhondrion\textunderscore )}
\end{itemize}
Cada uma das partes lateraes do abdome, debaixo das falsas costelas.
\section{Hipocorístico}
\begin{itemize}
\item {Grp. gram.:adj.}
\end{itemize}
\begin{itemize}
\item {Utilização:Gram.}
\end{itemize}
O mesmo que \textunderscore deminutivo\textunderscore .
\section{Hipocorolado}
\begin{itemize}
\item {Grp. gram.:adj.}
\end{itemize}
\begin{itemize}
\item {Utilização:Bot.}
\end{itemize}
\begin{itemize}
\item {Proveniência:(De \textunderscore hipo...\textunderscore  + \textunderscore corola\textunderscore )}
\end{itemize}
Que tem a corola inserta no ovário.
\section{Hipocorolia}
\begin{itemize}
\item {Grp. gram.:f.}
\end{itemize}
Classe de plantas, que compreende as dicotiledóneas monopétalas de corola hipógina.
(Cp. \textunderscore hipocorolado\textunderscore )
\section{Hipocraniano}
\begin{itemize}
\item {Grp. gram.:adj.}
\end{itemize}
\begin{itemize}
\item {Proveniência:(De \textunderscore hipo...\textunderscore  + \textunderscore craniano\textunderscore )}
\end{itemize}
Que está debaixo do crânio.
\section{Hipocrisia}
\begin{itemize}
\item {Grp. gram.:f.}
\end{itemize}
\begin{itemize}
\item {Proveniência:(Do lat. \textunderscore hypocrisis\textunderscore )}
\end{itemize}
Manifestação de qualidades ou sentimentos bons, que realmente se não têm.
Impostura; fingimento.
\section{Hipócrita}
\begin{itemize}
\item {Grp. gram.:adj.}
\end{itemize}
\begin{itemize}
\item {Grp. gram.:M. ,  f.}
\end{itemize}
\begin{itemize}
\item {Proveniência:(Lat. \textunderscore hypocrita\textunderscore )}
\end{itemize}
Em que há hipocrisia: \textunderscore palavras hipócritas\textunderscore .
Pessôa, que tem o defeito da hipocrisia.
\section{Hipocritamente}
\begin{itemize}
\item {Grp. gram.:adv.}
\end{itemize}
\begin{itemize}
\item {Proveniência:(De \textunderscore hipócrita\textunderscore )}
\end{itemize}
Com hipocrisia.
\section{Hipodáctilo}
\begin{itemize}
\item {Grp. gram.:m.}
\end{itemize}
\begin{itemize}
\item {Utilização:Zool.}
\end{itemize}
\begin{itemize}
\item {Proveniência:(Do gr. \textunderscore hupo\textunderscore  + \textunderscore daktulos\textunderscore )}
\end{itemize}
A parte inferior dos dedos das aves.
\section{Hipodermatomia}
\begin{itemize}
\item {Grp. gram.:f.}
\end{itemize}
\begin{itemize}
\item {Proveniência:(Do gr. \textunderscore hupo\textunderscore  + \textunderscore derma\textunderscore  + \textunderscore tome\textunderscore )}
\end{itemize}
Incisão cirúrgica, subcutânea.
\section{Hipoderme}
\begin{itemize}
\item {Grp. gram.:m.  ou  f.}
\end{itemize}
\begin{itemize}
\item {Grp. gram.:Adj.}
\end{itemize}
\begin{itemize}
\item {Proveniência:(Do gr. \textunderscore hupo\textunderscore  + \textunderscore derma\textunderscore )}
\end{itemize}
Pele, que reveste os elitros dos coleópteros.
Que vive debaixo da pele.
\section{Hipodérmico}
\begin{itemize}
\item {Grp. gram.:adj.}
\end{itemize}
\begin{itemize}
\item {Utilização:Bot.}
\end{itemize}
\begin{itemize}
\item {Proveniência:(Do gr. \textunderscore hupo\textunderscore  + \textunderscore derma\textunderscore )}
\end{itemize}
Que cresce sob a epiderme dos vegetaes.
\section{Hipodermoclise}
\begin{itemize}
\item {Grp. gram.:f.}
\end{itemize}
\begin{itemize}
\item {Utilização:Med.}
\end{itemize}
Injecção subcutânea de água.
\section{Hipoestesia}
\begin{itemize}
\item {Grp. gram.:f.}
\end{itemize}
\begin{itemize}
\item {Utilização:Med.}
\end{itemize}
\begin{itemize}
\item {Proveniência:(De \textunderscore hipo...\textunderscore  + \textunderscore estesia\textunderscore )}
\end{itemize}
Deminuição da sensibilidade em geral.
\section{Hipogástrico}
\begin{itemize}
\item {Grp. gram.:adj.}
\end{itemize}
Relativo ao hipogástrio.
\section{Hipogástrio}
\begin{itemize}
\item {Grp. gram.:m.}
\end{itemize}
\begin{itemize}
\item {Proveniência:(Gr. \textunderscore hupogastrion\textunderscore )}
\end{itemize}
Parte inferior do ventre.
\section{Hipogeu}
\begin{itemize}
\item {Grp. gram.:m.}
\end{itemize}
\begin{itemize}
\item {Utilização:Bot.}
\end{itemize}
\begin{itemize}
\item {Proveniência:(Lat. \textunderscore hypogaeos\textunderscore )}
\end{itemize}
Escavação subterrânea, em que os antigos depositavam os seus mortos.
Caule subterrâneo.
\section{Hipoglobolia}
\begin{itemize}
\item {Grp. gram.:f.}
\end{itemize}
\begin{itemize}
\item {Proveniência:(De \textunderscore hipo...\textunderscore  + \textunderscore glóbulo\textunderscore )}
\end{itemize}
Deminuição da quantidade dos glóbulos vermelhos do sangue.
\section{Hipoglossa}
\begin{itemize}
\item {Grp. gram.:f.}
\end{itemize}
\begin{itemize}
\item {Proveniência:(Lat. \textunderscore hypoglossa\textunderscore )}
\end{itemize}
Espécie de espargo.
\section{Hipoglosso}
\begin{itemize}
\item {Grp. gram.:adj.}
\end{itemize}
\begin{itemize}
\item {Grp. gram.:M.}
\end{itemize}
\begin{itemize}
\item {Proveniência:(Do gr. \textunderscore hupo\textunderscore  + \textunderscore glossa\textunderscore )}
\end{itemize}
Que está debaixo da língua.
Nervo, que preside aos movimentos da língua e da faringe.
\section{Hipoema}
\begin{itemize}
\item {Grp. gram.:m.}
\end{itemize}
\begin{itemize}
\item {Utilização:Med.}
\end{itemize}
\begin{itemize}
\item {Proveniência:(Do gr. \textunderscore hupo\textunderscore  + \textunderscore haima\textunderscore )}
\end{itemize}
Derramamento de sangue na câmara do ôlho.
\section{Hipoemia}
\begin{itemize}
\item {Grp. gram.:f.}
\end{itemize}
\begin{itemize}
\item {Utilização:Med.}
\end{itemize}
\begin{itemize}
\item {Proveniência:(Do gr. \textunderscore hupo\textunderscore  + \textunderscore haima\textunderscore )}
\end{itemize}
Deminuição na quantidade dos elementos do sangue.
\section{Hipofaringe}
\begin{itemize}
\item {Grp. gram.:f.}
\end{itemize}
\begin{itemize}
\item {Proveniência:(De \textunderscore hipo...\textunderscore  + \textunderscore faringe\textunderscore )}
\end{itemize}
Apêndice á faringe de certos himenópteros.
\section{Hipófase}
\begin{itemize}
\item {Grp. gram.:f.}
\end{itemize}
\begin{itemize}
\item {Utilização:Med.}
\end{itemize}
\begin{itemize}
\item {Proveniência:(Do gr. \textunderscore huppo\textunderscore  + \textunderscore phainein\textunderscore )}
\end{itemize}
Estado dos olhos, quando, quasi fechados, apenas deixam vêr parte da esclerótica.
\section{Hipofila}
\begin{itemize}
\item {Grp. gram.:adj. f.}
\end{itemize}
\begin{itemize}
\item {Utilização:Bot.}
\end{itemize}
\begin{itemize}
\item {Proveniência:(Do gr. \textunderscore hupo\textunderscore  + \textunderscore phullon\textunderscore )}
\end{itemize}
Diz-se da inflorescência anómala, em que as flôres nascem por baixo da bráctea.
\section{Hipófise}
\begin{itemize}
\item {Grp. gram.:f.}
\end{itemize}
\begin{itemize}
\item {Proveniência:(Do gr. \textunderscore hupo\textunderscore  + \textunderscore phusis\textunderscore )}
\end{itemize}
A glândula pituitária.
\section{Hipófora}
\begin{itemize}
\item {Grp. gram.:f.}
\end{itemize}
\begin{itemize}
\item {Utilização:Med.}
\end{itemize}
\begin{itemize}
\item {Proveniência:(Do gr. \textunderscore hupo\textunderscore  + \textunderscore phorein\textunderscore )}
\end{itemize}
Chaga profunda e fistulosa.
\section{Hipofosfato}
\begin{itemize}
\item {Grp. gram.:m.}
\end{itemize}
\begin{itemize}
\item {Proveniência:(De \textunderscore hipo...\textunderscore  + \textunderscore fosfato\textunderscore )}
\end{itemize}
Sal, produzido pela combinação do ácido hipofosfórico com uma base.
\section{Hipofosfaturia}
\begin{itemize}
\item {Grp. gram.:f.}
\end{itemize}
\begin{itemize}
\item {Utilização:Med.}
\end{itemize}
\begin{itemize}
\item {Proveniência:(De \textunderscore hipo...\textunderscore  + \textunderscore phosphato\textunderscore  + gr. \textunderscore ouron\textunderscore )}
\end{itemize}
Deminuição na quantidade dos fosfatos, eliminados pela urina.
\section{Hipofosfórico}
\begin{itemize}
\item {Grp. gram.:adj.}
\end{itemize}
\begin{itemize}
\item {Proveniência:(De \textunderscore hipo...\textunderscore  + \textunderscore fosforo\textunderscore )}
\end{itemize}
Diz-se de um dos oxácidos do fósforo.
\section{Hipofosforoso}
\begin{itemize}
\item {Grp. gram.:adj.}
\end{itemize}
\begin{itemize}
\item {Proveniência:(De \textunderscore hipo...\textunderscore  + \textunderscore fosforoso\textunderscore )}
\end{itemize}
Diz-se do primeiro oxácido do fósforo.
\section{Hipoftalmia}
\begin{itemize}
\item {Grp. gram.:f.}
\end{itemize}
\begin{itemize}
\item {Proveniência:(De \textunderscore hipo...\textunderscore  + \textunderscore oftalmia\textunderscore )}
\end{itemize}
Inflamação do ôlho, por baixo da pálpebra inferior.
Inflamação da pálpebra inferior.
\section{Hipoginia}
\begin{itemize}
\item {Grp. gram.:f.}
\end{itemize}
Estado ou qualidade de hipogínio.
\section{Hipogínio}
\begin{itemize}
\item {Grp. gram.:adj.}
\end{itemize}
\begin{itemize}
\item {Proveniência:(Do gr. \textunderscore hupo\textunderscore  + \textunderscore gune\textunderscore )}
\end{itemize}
Inserto abaixo do ovário ou ao nível dele, (falando-se de órgãos vegetaes).
\section{Hipógino}
\begin{itemize}
\item {Grp. gram.:adj.}
\end{itemize}
O mesmo ou melhor que \textunderscore hipogínio\textunderscore .
\section{Hipognatia}
\begin{itemize}
\item {Grp. gram.:f.}
\end{itemize}
Estado de hipognato.
\section{Hipognato}
\begin{itemize}
\item {Grp. gram.:m.}
\end{itemize}
\begin{itemize}
\item {Proveniência:(Do gr. \textunderscore hupo\textunderscore  + \textunderscore gnathos\textunderscore )}
\end{itemize}
Monstro, com uma cabeça acessória, muito imperfeita, ligada á maxila inferior da cabeça principal.
\section{Hipolinfia}
\begin{itemize}
\item {Grp. gram.:f.}
\end{itemize}
\begin{itemize}
\item {Utilização:Med.}
\end{itemize}
\begin{itemize}
\item {Proveniência:(De \textunderscore hipo...\textunderscore  + \textunderscore linfa\textunderscore )}
\end{itemize}
Deminuição de linfa.
\section{Hipomóclion}
\begin{itemize}
\item {Grp. gram.:m.}
\end{itemize}
Nome, dado em Mecânica ao ponto que serve de apoio á alavanca. Cf. Leoni, \textunderscore Diccion. de Artilh.\textunderscore , (inédito).
\section{Hipopetalia}
\begin{itemize}
\item {Grp. gram.:f.}
\end{itemize}
Estado das plantas hipopétalas.
\section{Hipopétalo}
\begin{itemize}
\item {Grp. gram.:adj.}
\end{itemize}
\begin{itemize}
\item {Utilização:Bot.}
\end{itemize}
\begin{itemize}
\item {Proveniência:(De \textunderscore hipo...\textunderscore  + \textunderscore pétala\textunderscore )}
\end{itemize}
Que tem as pétalas insertas no ovário.
\section{Hipópio}
\begin{itemize}
\item {Grp. gram.:m.}
\end{itemize}
O mesmo ou melhor que \textunderscore hipópion\textunderscore .
\section{Hipópion}
\begin{itemize}
\item {Grp. gram.:m.}
\end{itemize}
\begin{itemize}
\item {Proveniência:(Lat. \textunderscore hypopium\textunderscore )}
\end{itemize}
Derramamento de pus ou de matéria puriforme, nas câmaras do ôlho.
Alteração na transparência do humor vítreo do ôlho.
\section{Hipoplasia}
\begin{itemize}
\item {Grp. gram.:f.}
\end{itemize}
\begin{itemize}
\item {Utilização:Physiol.}
\end{itemize}
\begin{itemize}
\item {Proveniência:(Do gr. \textunderscore hupo\textunderscore  + \textunderscore plassein\textunderscore )}
\end{itemize}
Deminuição da actividade formadora dos tecidos.
\section{Hipopódio}
\begin{itemize}
\item {Grp. gram.:m.}
\end{itemize}
\begin{itemize}
\item {Proveniência:(Do gr. \textunderscore hupos\textunderscore  + \textunderscore pous\textunderscore )}
\end{itemize}
Estrado ou tarima nos banhos antigos.
\section{Hypertrophia}
\begin{itemize}
\item {Grp. gram.:f.}
\end{itemize}
\begin{itemize}
\item {Proveniência:(Do gr. \textunderscore huper\textunderscore  + \textunderscore trophe\textunderscore )}
\end{itemize}
Desenvolvimento excessivo de um órgão ou de uma parte de um órgão, sem alteração real do seu tecido.
\section{Hypertrophiado}
\begin{itemize}
\item {Grp. gram.:adj.}
\end{itemize}
Que tem hipertrophia.
\section{Hypertrophiar}
\begin{itemize}
\item {Grp. gram.:v. t.}
\end{itemize}
Produzir hypertrophia em.
\section{Hypethro}
\begin{itemize}
\item {Grp. gram.:m.}
\end{itemize}
\begin{itemize}
\item {Utilização:Ant.}
\end{itemize}
\begin{itemize}
\item {Proveniência:(Do gr. \textunderscore hupaithron\textunderscore )}
\end{itemize}
Templo descoberto.
\section{Hyphema}
\begin{itemize}
\item {Grp. gram.:m.}
\end{itemize}
\begin{itemize}
\item {Utilização:Med.}
\end{itemize}
\begin{itemize}
\item {Proveniência:(Do gr. \textunderscore hupo...\textunderscore  + \textunderscore haima\textunderscore )}
\end{itemize}
Sangue, na câmara anterior do ôlho.
\section{Hyphemia}
\begin{itemize}
\item {Grp. gram.:f.}
\end{itemize}
\begin{itemize}
\item {Utilização:Med.}
\end{itemize}
\begin{itemize}
\item {Proveniência:(Do gr. \textunderscore hupo...\textunderscore  + \textunderscore haima\textunderscore )}
\end{itemize}
Deminuição do sangue.
\section{Hýphen}
\begin{itemize}
\item {Grp. gram.:m.}
\end{itemize}
\begin{itemize}
\item {Proveniência:(Lat. \textunderscore hyphen\textunderscore )}
\end{itemize}
Traço de união.
Sinal, com que se ligam palavras.
\section{Hyphérese}
\begin{itemize}
\item {Grp. gram.:f.}
\end{itemize}
\begin{itemize}
\item {Utilização:Gram.}
\end{itemize}
Metaplasmo, em que se extrai de um vocábulo um ou mais phonemas. Cf. M. Maciel, \textunderscore Gram.\textunderscore 
\section{Hyphósporos}
\begin{itemize}
\item {Grp. gram.:adj. Pl.}
\end{itemize}
\begin{itemize}
\item {Proveniência:(Do gr. \textunderscore huphe\textunderscore  + \textunderscore sporos\textunderscore )}
\end{itemize}
Diz-se dos líchens, que têm fórma de filamentos.
\section{Hypiemia}
\begin{itemize}
\item {Grp. gram.:f.}
\end{itemize}
\begin{itemize}
\item {Proveniência:(Do gr. \textunderscore hupos\textunderscore  + \textunderscore haima\textunderscore )}
\end{itemize}
Hemorragia anal.
\section{Hypnagógico}
\begin{itemize}
\item {Grp. gram.:adj.}
\end{itemize}
\begin{itemize}
\item {Proveniência:(Do gr. \textunderscore hupnos\textunderscore  + \textunderscore agogos\textunderscore )}
\end{itemize}
Que produz somno.
Diz-se das alucinações e visões, que se têm, ao cair no somno.
\section{Hypnal}
\begin{itemize}
\item {Grp. gram.:m.}
\end{itemize}
\begin{itemize}
\item {Proveniência:(Lat. \textunderscore hypnale\textunderscore )}
\end{itemize}
Designação antiga de uma serpente venenosa, da qual se dizia que matava produzindo somno.
Medicamento hypnótico e antineurálgico.
\section{Hypniatria}
\begin{itemize}
\item {Grp. gram.:f.}
\end{itemize}
Tratamento de doenças, segundo as indicações do hypniatro.
\section{Hypniatro}
\begin{itemize}
\item {Grp. gram.:m.}
\end{itemize}
\begin{itemize}
\item {Proveniência:(Do gr. \textunderscore hupnos\textunderscore  + \textunderscore iatros\textunderscore )}
\end{itemize}
Somnâmbulo, que, durante o somno magnético, indica remédios para as suas doenças ou para as alheias.
\section{Hypnóbata}
\begin{itemize}
\item {Grp. gram.:m.}
\end{itemize}
\begin{itemize}
\item {Utilização:P. us.}
\end{itemize}
\begin{itemize}
\item {Proveniência:(Do gr. \textunderscore hupnos\textunderscore  + \textunderscore bainein\textunderscore )}
\end{itemize}
O mesmo que \textunderscore somnâmbulo\textunderscore .
\section{Hypnoblepsia}
\begin{itemize}
\item {Grp. gram.:f.}
\end{itemize}
\begin{itemize}
\item {Proveniência:(Do gr. \textunderscore hupnos\textunderscore  + \textunderscore blepein\textunderscore )}
\end{itemize}
Somnâmbulismo lúcido.
\section{Hypnococco}
\begin{itemize}
\item {Grp. gram.:m.}
\end{itemize}
\begin{itemize}
\item {Proveniência:(Do gr. \textunderscore hupnos\textunderscore  + \textunderscore kokkos\textunderscore )}
\end{itemize}
Bacillo da hypnose, transição entre o estreplococco e o diplococco.
\section{Hypnógeno}
\begin{itemize}
\item {Grp. gram.:adj.}
\end{itemize}
\begin{itemize}
\item {Proveniência:(Do gr. \textunderscore hupnos\textunderscore  + \textunderscore genos\textunderscore )}
\end{itemize}
Que produz somno.
\section{Hypnographia}
\begin{itemize}
\item {Grp. gram.:f.}
\end{itemize}
\begin{itemize}
\item {Proveniência:(Do gr. \textunderscore hupnos\textunderscore  + \textunderscore graphein\textunderscore )}
\end{itemize}
Descripção do somno.
\section{Hypnologia}
\begin{itemize}
\item {Grp. gram.:f.}
\end{itemize}
\begin{itemize}
\item {Proveniência:(Do gr. \textunderscore hupnos\textunderscore  + \textunderscore logos\textunderscore )}
\end{itemize}
Tratado á cêrca do somno.
\section{Hypnológico}
\begin{itemize}
\item {Grp. gram.:adj.}
\end{itemize}
Relativo á hypnologia.
\section{Hypnólogo}
\begin{itemize}
\item {Grp. gram.:m.}
\end{itemize}
Tratadista de hypnologia.
\section{Hypnome}
\begin{itemize}
\item {Grp. gram.:m.}
\end{itemize}
O mesmo que \textunderscore chloral\textunderscore .
\section{Hypnopathia}
\begin{itemize}
\item {Grp. gram.:f.}
\end{itemize}
\begin{itemize}
\item {Proveniência:(Do gr. \textunderscore hupnos\textunderscore  + \textunderscore pathos\textunderscore )}
\end{itemize}
O mesmo que \textunderscore hypnosia\textunderscore .
\section{Hypnophobia}
\begin{itemize}
\item {Grp. gram.:f.}
\end{itemize}
Mêdo de dormir.
Terror, que advém durante o somno.
(Cp. \textunderscore hypnóphobo\textunderscore )
\section{Hypnóphobo}
\begin{itemize}
\item {Grp. gram.:m.}
\end{itemize}
\begin{itemize}
\item {Proveniência:(Do gr. \textunderscore hupnos\textunderscore  + \textunderscore phobein\textunderscore )}
\end{itemize}
Aquelle que tem hypnophobia.
\section{Hypnophono}
\begin{itemize}
\item {Grp. gram.:m.}
\end{itemize}
\begin{itemize}
\item {Proveniência:(Do gr. \textunderscore hupnos\textunderscore  + \textunderscore phone\textunderscore )}
\end{itemize}
Aquelle que fala durante o somno magnético.
\section{Hypnophrenose}
\begin{itemize}
\item {Grp. gram.:f.}
\end{itemize}
\begin{itemize}
\item {Proveniência:(Do gr. \textunderscore hupnos\textunderscore  + \textunderscore phren\textunderscore )}
\end{itemize}
O mesmo que \textunderscore somnambulismo\textunderscore .
\section{Hypnose}
\begin{itemize}
\item {Grp. gram.:f.}
\end{itemize}
\begin{itemize}
\item {Proveniência:(Do gr. \textunderscore hupnos\textunderscore )}
\end{itemize}
Estado particular, caracterizado pelo somno nervoso e pela suggestão.
\section{Hypnosia}
\begin{itemize}
\item {Grp. gram.:f.}
\end{itemize}
Doença do somno.
O mesmo que \textunderscore hypnose\textunderscore .
\section{Hypnótico}
\begin{itemize}
\item {Grp. gram.:adj.}
\end{itemize}
\begin{itemize}
\item {Grp. gram.:M.}
\end{itemize}
\begin{itemize}
\item {Proveniência:(Lat. \textunderscore hypnoticus\textunderscore )}
\end{itemize}
Relativo á hypnose; que produz somno.
Aquillo que produz somno; narcótico.
\section{Hypnotismo}
\begin{itemize}
\item {Grp. gram.:m.}
\end{itemize}
\begin{itemize}
\item {Proveniência:(Do rad. de \textunderscore hypnótico\textunderscore )}
\end{itemize}
Processo, para produzir somno, fazendo fixar a vista num objecto brilhante, a pouca distância dos olhos, ou empregando outros meios.
Somno, provocado por êsse processo.
\section{Hypnotista}
\begin{itemize}
\item {Grp. gram.:m.}
\end{itemize}
Aquelle que pratíca o hypnotismo.
\section{Hypnotização}
\begin{itemize}
\item {Grp. gram.:f.}
\end{itemize}
Acto ou effeito de hypnotizar.
\section{Hypnotizador}
\begin{itemize}
\item {Grp. gram.:m.}
\end{itemize}
Aquelle que hypnotiza.
\section{Hypnotizar}
\begin{itemize}
\item {Grp. gram.:v. t.}
\end{itemize}
\begin{itemize}
\item {Proveniência:(De \textunderscore hypnótico\textunderscore )}
\end{itemize}
Produzir o hypnotismo em.
\section{Hypnotoxina}
\begin{itemize}
\item {fónica:csi}
\end{itemize}
\begin{itemize}
\item {Grp. gram.:f.}
\end{itemize}
Veneno, extrahido dos tentáculos da physália. Cf. \textunderscore Jorn. do Comm.\textunderscore , do Rio de 2-V-902.
\section{Hypo...}
\begin{itemize}
\item {Grp. gram.:pref.}
\end{itemize}
\begin{itemize}
\item {Proveniência:(Do gr. \textunderscore hupo\textunderscore )}
\end{itemize}
(designativo de \textunderscore deminuição\textunderscore ; \textunderscore grau inferior\textunderscore ; \textunderscore debaixo\textunderscore )
\section{Hypoacusia}
\begin{itemize}
\item {Grp. gram.:f.}
\end{itemize}
\begin{itemize}
\item {Utilização:Med.}
\end{itemize}
Deminuição do sentido auditivo.
\section{Hypoalgesia}
\begin{itemize}
\item {Grp. gram.:f.}
\end{itemize}
\begin{itemize}
\item {Proveniência:(Do gr. \textunderscore hupo\textunderscore  + \textunderscore algesis\textunderscore )}
\end{itemize}
Deminuição da sensibilidade á dôr.
\section{Hypoazótico}
\begin{itemize}
\item {Grp. gram.:adj.}
\end{itemize}
\begin{itemize}
\item {Proveniência:(De \textunderscore hypo...\textunderscore  + \textunderscore azotato\textunderscore )}
\end{itemize}
Diz-se de um ácido, que se obtém pela destillação do azotato de chumbo sêco.
\section{Hypoazoturia}
\begin{itemize}
\item {Grp. gram.:f.}
\end{itemize}
\begin{itemize}
\item {Proveniência:(De \textunderscore hypo...\textunderscore  + \textunderscore azoturia\textunderscore )}
\end{itemize}
Deminuição do azote, eliminado pela urina.
\section{Hypoblasto}
\begin{itemize}
\item {Grp. gram.:m.}
\end{itemize}
\begin{itemize}
\item {Proveniência:(Do gr. \textunderscore hupo\textunderscore  + \textunderscore blastos\textunderscore )}
\end{itemize}
Folheto interior do blastoderme.
\section{Hypobrânchio}
\begin{itemize}
\item {fónica:qui}
\end{itemize}
\begin{itemize}
\item {Grp. gram.:adj.}
\end{itemize}
\begin{itemize}
\item {Proveniência:(De \textunderscore hypo...\textunderscore  + \textunderscore brânchias\textunderscore )}
\end{itemize}
Que tem as brânchias por baixo do corpo.
\section{Hypocardo}
\begin{itemize}
\item {Grp. gram.:m.}
\end{itemize}
Carda aperfeiçoada, inventada em 1855 por um fabricante de tecidos em Colmar.
\section{Hypocarístico}
\begin{itemize}
\item {Grp. gram.:adj.}
\end{itemize}
\begin{itemize}
\item {Utilização:Gram.}
\end{itemize}
Diz-se dos vocábulos familiares ou infantis, sobretudo daquelles em que há duplicação de sýllaba: \textunderscore papá\textunderscore , \textunderscore Lulu\textunderscore , \textunderscore Lili\textunderscore , etc. Cf. Júl. Ribeiro, \textunderscore Diccion. Gram.\textunderscore 
\section{Hypocarpo}
\begin{itemize}
\item {Grp. gram.:m.}
\end{itemize}
\begin{itemize}
\item {Utilização:Bot.}
\end{itemize}
\begin{itemize}
\item {Proveniência:(Do gr. \textunderscore hupo\textunderscore  + \textunderscore karpos\textunderscore )}
\end{itemize}
Parte da planta, em que assenta o fruto.
\section{Hypocausto}
\begin{itemize}
\item {Grp. gram.:m.}
\end{itemize}
\begin{itemize}
\item {Utilização:Ant.}
\end{itemize}
\begin{itemize}
\item {Proveniência:(Gr. \textunderscore hupokauston\textunderscore )}
\end{itemize}
Forno subterrâneo, nas antigas thermas.
\section{Hypochlorhydria}
\begin{itemize}
\item {Grp. gram.:f.}
\end{itemize}
\begin{itemize}
\item {Utilização:Med.}
\end{itemize}
\begin{itemize}
\item {Proveniência:(De \textunderscore hypo...\textunderscore  + \textunderscore chlorhýdrico\textunderscore )}
\end{itemize}
Deminuição da acidez do suco gástrico.
\section{Hypochlórico}
\begin{itemize}
\item {Grp. gram.:adj.}
\end{itemize}
\begin{itemize}
\item {Proveniência:(De \textunderscore hypo...\textunderscore  + \textunderscore chloro\textunderscore )}
\end{itemize}
Diz-se de um ácido, que se obtém, decompondo o chloreto de potassa pelo ácido sulfúrico.
\section{Hypochlorito}
\begin{itemize}
\item {Grp. gram.:m.}
\end{itemize}
\begin{itemize}
\item {Utilização:Chím.}
\end{itemize}
Nome de um sal.
\section{Hypochloroso}
\begin{itemize}
\item {Grp. gram.:adj.}
\end{itemize}
\begin{itemize}
\item {Proveniência:(De \textunderscore hypo...\textunderscore  + \textunderscore chloro\textunderscore )}
\end{itemize}
Diz-se de um ácido, que é um dos oxácidos do chloro.
\section{Hypocinético}
\begin{itemize}
\item {Grp. gram.:adj.}
\end{itemize}
\begin{itemize}
\item {Proveniência:(Do gr. \textunderscore hupo\textunderscore  + \textunderscore kinetikos\textunderscore )}
\end{itemize}
Diz-se dos medicamentos, que modificam a acção nervosa.
\section{Hypociste}
\begin{itemize}
\item {Grp. gram.:f.}
\end{itemize}
Planta parasita, (\textunderscore cytinus hypocistis\textunderscore , Lin.)
\section{Hypocondria}
\begin{itemize}
\item {Grp. gram.:f.}
\end{itemize}
\begin{itemize}
\item {Proveniência:(De \textunderscore hypocôndrio\textunderscore )}
\end{itemize}
Doença nervosa, que faz crer na existência de várias enfermidades, produzindo habitual tristeza.
Melancolia; misanthropia.
\section{Hypocondríaco}
\begin{itemize}
\item {Grp. gram.:m.}
\end{itemize}
\begin{itemize}
\item {Grp. gram.:Adj.}
\end{itemize}
Que tem hypocondria.
Relativo a hypocondria.
Que soffre hypocondria.
\section{Hypocôndrio}
\begin{itemize}
\item {Grp. gram.:m.}
\end{itemize}
\begin{itemize}
\item {Proveniência:(Gr. \textunderscore hupokhondrion\textunderscore )}
\end{itemize}
Cada uma das partes lateraes do abdome, debaixo das falsas costellas.
\section{Hypocophose}
\begin{itemize}
\item {Grp. gram.:f.}
\end{itemize}
\begin{itemize}
\item {Proveniência:(Do gr. \textunderscore hupo\textunderscore  + \textunderscore kophosis\textunderscore )}
\end{itemize}
Surdez completa.
\section{Hypocorístico}
\begin{itemize}
\item {Grp. gram.:adj.}
\end{itemize}
\begin{itemize}
\item {Utilização:Gram.}
\end{itemize}
O mesmo que \textunderscore deminutivo\textunderscore .
\section{Hypocorollado}
\begin{itemize}
\item {Grp. gram.:adj.}
\end{itemize}
\begin{itemize}
\item {Utilização:Bot.}
\end{itemize}
\begin{itemize}
\item {Proveniência:(De \textunderscore hypo...\textunderscore  + \textunderscore corolla\textunderscore )}
\end{itemize}
Que tem a corolla inserta no ovário.
\section{Hypocorollia}
\begin{itemize}
\item {Grp. gram.:f.}
\end{itemize}
Classe de plantas, que comprehende as dicotyledóneas monopétalas de corolla hypógyna.
(Cp. \textunderscore hypocorollado\textunderscore )
\section{Hypocraniano}
\begin{itemize}
\item {Grp. gram.:adj.}
\end{itemize}
\begin{itemize}
\item {Proveniência:(De \textunderscore hypo...\textunderscore  + \textunderscore craniano\textunderscore )}
\end{itemize}
Que está debaixo do crânio.
\section{Hypocrisia}
\begin{itemize}
\item {Grp. gram.:f.}
\end{itemize}
\begin{itemize}
\item {Proveniência:(Do lat. \textunderscore hypocrisis\textunderscore )}
\end{itemize}
Manifestação de qualidades ou sentimentos bons, que realmente se não têm.
Impostura; fingimento.
\section{Hypócrita}
\begin{itemize}
\item {Grp. gram.:adj.}
\end{itemize}
\begin{itemize}
\item {Grp. gram.:M. ,  f.}
\end{itemize}
\begin{itemize}
\item {Proveniência:(Lat. \textunderscore hypocrita\textunderscore )}
\end{itemize}
Em que há hypocrisia: \textunderscore palavras hypócritas\textunderscore .
Pessôa, que tem o defeito da hypocrisia.
\section{Hypocritamente}
\begin{itemize}
\item {Grp. gram.:adv.}
\end{itemize}
\begin{itemize}
\item {Proveniência:(De \textunderscore hypócrita\textunderscore )}
\end{itemize}
Com hypocrisia.
\section{Hypocycloide}
\begin{itemize}
\item {Grp. gram.:f.}
\end{itemize}
\begin{itemize}
\item {Utilização:Mathem.}
\end{itemize}
\begin{itemize}
\item {Proveniência:(De \textunderscore hypo...\textunderscore  + \textunderscore cyclo\textunderscore )}
\end{itemize}
Espécie de curva transcendente.
\section{Hypodáctylo}
\begin{itemize}
\item {Grp. gram.:m.}
\end{itemize}
\begin{itemize}
\item {Utilização:Zool.}
\end{itemize}
\begin{itemize}
\item {Proveniência:(Do gr. \textunderscore hupo\textunderscore  + \textunderscore daktulos\textunderscore )}
\end{itemize}
A parte inferior dos dedos das aves.
\section{Hypodermatomia}
\begin{itemize}
\item {Grp. gram.:f.}
\end{itemize}
\begin{itemize}
\item {Proveniência:(Do gr. \textunderscore hupo\textunderscore  + \textunderscore derma\textunderscore  + \textunderscore tome\textunderscore )}
\end{itemize}
Incisão cirúrgica, subcutânea.
\section{Hypoderme}
\begin{itemize}
\item {Grp. gram.:m.  ou  f.}
\end{itemize}
\begin{itemize}
\item {Grp. gram.:Adj.}
\end{itemize}
\begin{itemize}
\item {Proveniência:(Do gr. \textunderscore hupo\textunderscore  + \textunderscore derma\textunderscore )}
\end{itemize}
Pelle, que reveste os elytros dos coleópteros.
Que vive debaixo da pelle.
\section{Hypodérmico}
\begin{itemize}
\item {Grp. gram.:adj.}
\end{itemize}
\begin{itemize}
\item {Utilização:Bot.}
\end{itemize}
\begin{itemize}
\item {Proveniência:(Do gr. \textunderscore hupo\textunderscore  + \textunderscore derma\textunderscore )}
\end{itemize}
Que cresce sob a epiderme dos vegetaes.
\section{Hypodermoclyse}
\begin{itemize}
\item {Grp. gram.:f.}
\end{itemize}
\begin{itemize}
\item {Utilização:Med.}
\end{itemize}
Injecção subcutânea de água.
\section{Hypoesthesia}
\begin{itemize}
\item {Grp. gram.:f.}
\end{itemize}
\begin{itemize}
\item {Utilização:Med.}
\end{itemize}
\begin{itemize}
\item {Proveniência:(De \textunderscore hypo...\textunderscore  + \textunderscore esthesia\textunderscore )}
\end{itemize}
Deminuição da sensibilidade em geral.
\section{Hypogástrico}
\begin{itemize}
\item {Grp. gram.:adj.}
\end{itemize}
Relativo ao hypogástrio.
\section{Hypogástrio}
\begin{itemize}
\item {Grp. gram.:m.}
\end{itemize}
\begin{itemize}
\item {Proveniência:(Gr. \textunderscore hupogastrion\textunderscore )}
\end{itemize}
Parte inferior do ventre.
\section{Hypogeu}
\begin{itemize}
\item {Grp. gram.:m.}
\end{itemize}
\begin{itemize}
\item {Utilização:Bot.}
\end{itemize}
\begin{itemize}
\item {Proveniência:(Lat. \textunderscore hypogaeos\textunderscore )}
\end{itemize}
Escavação subterrânea, em que os antigos depositavam os seus mortos.
Caule subterrâneo.
\section{Hypoglobolia}
\begin{itemize}
\item {Grp. gram.:f.}
\end{itemize}
\begin{itemize}
\item {Proveniência:(De \textunderscore hypo...\textunderscore  + \textunderscore glóbulo\textunderscore )}
\end{itemize}
Deminuição da quantidade dos glóbulos vermelhos do sangue.
\section{Hypoglossa}
\begin{itemize}
\item {Grp. gram.:f.}
\end{itemize}
\begin{itemize}
\item {Proveniência:(Lat. \textunderscore hypoglossa\textunderscore )}
\end{itemize}
Espécie de espargo.
\section{Hypoglosso}
\begin{itemize}
\item {Grp. gram.:adj.}
\end{itemize}
\begin{itemize}
\item {Grp. gram.:M.}
\end{itemize}
\begin{itemize}
\item {Proveniência:(Do gr. \textunderscore hupo\textunderscore  + \textunderscore glossa\textunderscore )}
\end{itemize}
Que está debaixo da língua.
Nervo, que preside aos movimentos da língua e da pharynge.
\section{Hypognathia}
\begin{itemize}
\item {Grp. gram.:f.}
\end{itemize}
Estado de hypognatho.
\section{Hypognatho}
\begin{itemize}
\item {Grp. gram.:m.}
\end{itemize}
\begin{itemize}
\item {Proveniência:(Do gr. \textunderscore hupo\textunderscore  + \textunderscore gnathos\textunderscore )}
\end{itemize}
Monstro, com uma cabeça accessória, muito imperfeita, ligada á maxilla inferior da cabeça principal.
\section{Hypogynia}
\begin{itemize}
\item {Grp. gram.:f.}
\end{itemize}
Estado ou qualidade de hypogýnio.
\section{Hypogýnio}
\begin{itemize}
\item {Grp. gram.:adj.}
\end{itemize}
\begin{itemize}
\item {Proveniência:(Do gr. \textunderscore hupo\textunderscore  + \textunderscore gune\textunderscore )}
\end{itemize}
Inserto abaixo do ovário ou ao nível delle, (falando-se de órgãos vegetaes).
\section{Hypógyno}
\begin{itemize}
\item {Grp. gram.:adj.}
\end{itemize}
O mesmo ou melhor que \textunderscore hypogýnio\textunderscore .
\section{Hypohema}
\begin{itemize}
\item {Grp. gram.:m.}
\end{itemize}
\begin{itemize}
\item {Utilização:Med.}
\end{itemize}
\begin{itemize}
\item {Proveniência:(Do gr. \textunderscore hupo\textunderscore  + \textunderscore haima\textunderscore )}
\end{itemize}
Derramamento de sangue na câmara do ôlho.
\section{Hypohemia}
\begin{itemize}
\item {Grp. gram.:f.}
\end{itemize}
\begin{itemize}
\item {Utilização:Med.}
\end{itemize}
\begin{itemize}
\item {Proveniência:(Do gr. \textunderscore hupo\textunderscore  + \textunderscore haima\textunderscore )}
\end{itemize}
Deminuição na quantidade dos elementos do sangue.
\section{Hypolymphia}
\begin{itemize}
\item {Grp. gram.:f.}
\end{itemize}
\begin{itemize}
\item {Utilização:Med.}
\end{itemize}
\begin{itemize}
\item {Proveniência:(De \textunderscore hypo...\textunderscore  + \textunderscore lympha\textunderscore )}
\end{itemize}
Deminuição de lympha.
\section{Hypomóchlion}
\begin{itemize}
\item {Grp. gram.:m.}
\end{itemize}
Nome, dado em Mecânica ao ponto que serve de apoio á alavanca. Cf. Leoni, \textunderscore Diccion. de Artilh.\textunderscore , (inédito).
\section{Hypopetalia}
\begin{itemize}
\item {Grp. gram.:f.}
\end{itemize}
Estado das plantas hypopétalas.
\section{Hypopétalo}
\begin{itemize}
\item {Grp. gram.:adj.}
\end{itemize}
\begin{itemize}
\item {Utilização:Bot.}
\end{itemize}
\begin{itemize}
\item {Proveniência:(De \textunderscore hipo...\textunderscore  + \textunderscore pétala\textunderscore )}
\end{itemize}
Que tem as pétalas insertas no ovário.
\section{Hypopharynge}
\begin{itemize}
\item {Grp. gram.:f.}
\end{itemize}
\begin{itemize}
\item {Proveniência:(De \textunderscore hipo...\textunderscore  + \textunderscore pharynge\textunderscore )}
\end{itemize}
Appêndice á pharynge de certos hymenópteros.
\section{Hypóphase}
\begin{itemize}
\item {Grp. gram.:f.}
\end{itemize}
\begin{itemize}
\item {Utilização:Med.}
\end{itemize}
\begin{itemize}
\item {Proveniência:(Do gr. \textunderscore huppo\textunderscore  + \textunderscore phainein\textunderscore )}
\end{itemize}
Estado dos olhos, quando, quasi fechados, apenas deixam vêr parte da esclerótica.
\section{Hypóphora}
\begin{itemize}
\item {Grp. gram.:f.}
\end{itemize}
\begin{itemize}
\item {Utilização:Med.}
\end{itemize}
\begin{itemize}
\item {Proveniência:(Do gr. \textunderscore hupo\textunderscore  + \textunderscore phorein\textunderscore )}
\end{itemize}
Chaga profunda e fistulosa.
\section{Hypophosphato}
\begin{itemize}
\item {Grp. gram.:m.}
\end{itemize}
\begin{itemize}
\item {Proveniência:(De \textunderscore hypo...\textunderscore  + \textunderscore phosphato\textunderscore )}
\end{itemize}
Sal, produzido pela combinação do ácido hypophosphórico com uma base.
\section{Hypophosphaturia}
\begin{itemize}
\item {Grp. gram.:f.}
\end{itemize}
\begin{itemize}
\item {Utilização:Med.}
\end{itemize}
\begin{itemize}
\item {Proveniência:(De \textunderscore hypo...\textunderscore  + \textunderscore phosphato\textunderscore  + gr. \textunderscore ouron\textunderscore )}
\end{itemize}
Deminuição na quantidade dos phosphatos, eliminados pela urina.
\section{Hypophosphórico}
\begin{itemize}
\item {Grp. gram.:adj.}
\end{itemize}
\begin{itemize}
\item {Proveniência:(De \textunderscore hipo...\textunderscore  + \textunderscore phosphoro\textunderscore )}
\end{itemize}
Diz-se de um dos oxácidos do phósphoro.
\section{Hypophosphoroso}
\begin{itemize}
\item {Grp. gram.:adj.}
\end{itemize}
\begin{itemize}
\item {Proveniência:(De \textunderscore hipo...\textunderscore  + \textunderscore phosphoroso\textunderscore )}
\end{itemize}
Diz-se do primeiro oxácido do phósphoro.
\section{Hypophtalmia}
\begin{itemize}
\item {Grp. gram.:f.}
\end{itemize}
\begin{itemize}
\item {Proveniência:(De \textunderscore hypo...\textunderscore  + \textunderscore ophtalmia\textunderscore )}
\end{itemize}
Inflammação do ôlho, por baixo da pálpebra inferior.
Inflammação da pálpebra inferior.
\section{Hypophylla}
\begin{itemize}
\item {Grp. gram.:adj. f.}
\end{itemize}
\begin{itemize}
\item {Utilização:Bot.}
\end{itemize}
\begin{itemize}
\item {Proveniência:(Do gr. \textunderscore hupo\textunderscore  + \textunderscore phullon\textunderscore )}
\end{itemize}
Diz-se da inflorescência anómala, em que as flôres nascem por baixo da bráctea.
\section{Hypóphyse}
\begin{itemize}
\item {Grp. gram.:f.}
\end{itemize}
\begin{itemize}
\item {Proveniência:(Do gr. \textunderscore hupo\textunderscore  + \textunderscore phusis\textunderscore )}
\end{itemize}
A glândula pituitária.
\section{Hypópio}
\begin{itemize}
\item {Grp. gram.:m.}
\end{itemize}
O mesmo ou melhor que \textunderscore hypópion\textunderscore .
\section{Hypópion}
\begin{itemize}
\item {Grp. gram.:m.}
\end{itemize}
\begin{itemize}
\item {Proveniência:(Lat. \textunderscore hypopium\textunderscore )}
\end{itemize}
Derramamento de pus ou de matéria puriforme, nas câmaras do ôlho.
Alteração na transparência do humor vítreo do ôlho.
\section{Hypoplasia}
\begin{itemize}
\item {Grp. gram.:f.}
\end{itemize}
\begin{itemize}
\item {Utilização:Physiol.}
\end{itemize}
\begin{itemize}
\item {Proveniência:(Do gr. \textunderscore hupo\textunderscore  + \textunderscore plassein\textunderscore )}
\end{itemize}
Deminuição da actividade formadora dos tecidos.
\section{Hypopódio}
\begin{itemize}
\item {Grp. gram.:m.}
\end{itemize}
\begin{itemize}
\item {Proveniência:(Do gr. \textunderscore hupos\textunderscore  + \textunderscore pous\textunderscore )}
\end{itemize}
Estrado ou tarima nos banhos antigos.
\section{Hipopígio}
\begin{itemize}
\item {Grp. gram.:m.}
\end{itemize}
\begin{itemize}
\item {Proveniência:(Do gr. \textunderscore hupo\textunderscore  + \textunderscore puge\textunderscore )}
\end{itemize}
Último segmento neutral do abdome dos insectos.
\section{Hiporritmo}
\begin{itemize}
\item {Grp. gram.:m.}
\end{itemize}
\begin{itemize}
\item {Proveniência:(Do gr. \textunderscore hupo\textunderscore  + \textunderscore rhuthmos\textunderscore )}
\end{itemize}
Hexâmetro sem cesura.
\section{Hiposcênio}
\begin{itemize}
\item {Grp. gram.:m.}
\end{itemize}
\begin{itemize}
\item {Proveniência:(De \textunderscore hipo...\textunderscore  + \textunderscore cena\textunderscore )}
\end{itemize}
A parte inferior da cena, nos teatros gregos.
Parede ou taipa, que suporta a frente do palco.
Lugar, ocupado pelos músicos, junto dessa taipa.
\section{Hiposfagma}
\begin{itemize}
\item {Grp. gram.:m.}
\end{itemize}
\begin{itemize}
\item {Proveniência:(Gr. \textunderscore huposphagma\textunderscore )}
\end{itemize}
Equimose no ôlho.
\section{Hiposmia}
\begin{itemize}
\item {Grp. gram.:f.}
\end{itemize}
\begin{itemize}
\item {Utilização:Med.}
\end{itemize}
\begin{itemize}
\item {Proveniência:(Do gr. \textunderscore hupo\textunderscore  + \textunderscore osme\textunderscore )}
\end{itemize}
Enfraquecimento do olfato.
\section{Hipospadia}
\begin{itemize}
\item {Grp. gram.:f.}
\end{itemize}
\begin{itemize}
\item {Proveniência:(Do gr. \textunderscore hupo\textunderscore  + \textunderscore spao\textunderscore )}
\end{itemize}
Deformidade dos órgãos genitaes do homem, que consiste em que a uretra se abre por baixo do pênis, em qualquer ponto.
\section{Hipospado}
\begin{itemize}
\item {Grp. gram.:m.}
\end{itemize}
Aquele que tem a hipospadia.
\section{Hipospatismo}
\begin{itemize}
\item {Grp. gram.:m.}
\end{itemize}
\begin{itemize}
\item {Proveniência:(Gr. \textunderscore hupospathismos\textunderscore )}
\end{itemize}
Desusada operação cirúrgica, em que, para os casos de oftalmia crónica, se faziam três incisões na fronte, passando-se uma espátula entre as carnes e o pericrânio, para êste ficar descoberto em certa extensão.
\section{Hipossistolia}
\begin{itemize}
\item {Grp. gram.:f.}
\end{itemize}
\begin{itemize}
\item {Utilização:Med.}
\end{itemize}
\begin{itemize}
\item {Proveniência:(Do gr. \textunderscore hupo\textunderscore  + \textunderscore sustole\textunderscore )}
\end{itemize}
Fraqueza cardiaca.
\section{Hipossulfato}
\begin{itemize}
\item {Grp. gram.:m.}
\end{itemize}
\begin{itemize}
\item {Proveniência:(De \textunderscore hipo...\textunderscore  + \textunderscore sulfato\textunderscore )}
\end{itemize}
Sal, produzido pela combinação do ácido hiposulfúrico com uma base.
\section{Hipossulfito}
\begin{itemize}
\item {Grp. gram.:m.}
\end{itemize}
\begin{itemize}
\item {Utilização:Chím.}
\end{itemize}
\begin{itemize}
\item {Proveniência:(De \textunderscore hipo\textunderscore  + \textunderscore sulfito\textunderscore )}
\end{itemize}
Sal, resultante da combinação do ácido hiposulfuroso com uma base.
\section{Hipossulfúrico}
\begin{itemize}
\item {Grp. gram.:adj.}
\end{itemize}
\begin{itemize}
\item {Proveniência:(De \textunderscore hipo...\textunderscore  + \textunderscore sulfúrico\textunderscore )}
\end{itemize}
Diz-se do terceiro dos oxácidos de enxôfre.
\section{Hipossulfuroso}
\begin{itemize}
\item {Grp. gram.:adj.}
\end{itemize}
\begin{itemize}
\item {Proveniência:(De \textunderscore hipo...\textunderscore  + \textunderscore sulfuroso\textunderscore )}
\end{itemize}
Diz-se do primeiro dos oxácidos do enxôfre.
\section{Hipostaminado}
\begin{itemize}
\item {Grp. gram.:adj.}
\end{itemize}
\begin{itemize}
\item {Utilização:Bot.}
\end{itemize}
\begin{itemize}
\item {Proveniência:(De \textunderscore hipo...\textunderscore  + \textunderscore estame\textunderscore )}
\end{itemize}
Que tem os estames insertos no ovário.
\section{Hipostaminia}
\begin{itemize}
\item {Grp. gram.:f.}
\end{itemize}
Estado de uma planta, que tem estames hipóginos.
(Cp. \textunderscore hipostaminado\textunderscore )
\section{Hipóstase}
\begin{itemize}
\item {Grp. gram.:f.}
\end{itemize}
\begin{itemize}
\item {Proveniência:(Lat. \textunderscore hypostasis\textunderscore )}
\end{itemize}
União do Verbo com a natureza divina, como substância única, (em Teologia).
Sedimento da urina.
Sarro.
\section{Hipostaticamente}
\begin{itemize}
\item {Grp. gram.:adv.}
\end{itemize}
De modo hipostático.
\section{Hipostático}
\begin{itemize}
\item {Grp. gram.:adj.}
\end{itemize}
\begin{itemize}
\item {Proveniência:(Gr. \textunderscore hupostatikos\textunderscore )}
\end{itemize}
Relativo á hipóstase: \textunderscore união hipostática\textunderscore .
\section{Hiposternal}
\begin{itemize}
\item {Grp. gram.:m.}
\end{itemize}
\begin{itemize}
\item {Proveniência:(De \textunderscore hipo...\textunderscore  + \textunderscore esterno\textunderscore )}
\end{itemize}
Peça do esterno das tartarugas.
\section{Hipostenia}
\begin{itemize}
\item {Grp. gram.:f.}
\end{itemize}
\begin{itemize}
\item {Utilização:Med.}
\end{itemize}
\begin{itemize}
\item {Proveniência:(Do gr. \textunderscore hupo\textunderscore  + \textunderscore sthenos\textunderscore )}
\end{itemize}
Deminuição de fôrças.
\section{Hipostênico}
\begin{itemize}
\item {Grp. gram.:adj.}
\end{itemize}
Relativo á hipostenia.
\section{Hipostilo}
\begin{itemize}
\item {Grp. gram.:adj.}
\end{itemize}
\begin{itemize}
\item {Proveniência:(Do gr. \textunderscore hupo\textunderscore  + \textunderscore stulos\textunderscore )}
\end{itemize}
Dizia-se das salas ou compartimentos, cujo tecto é sustentado por colunas.
\section{Hipóstoma}
\begin{itemize}
\item {Grp. gram.:m.}
\end{itemize}
\begin{itemize}
\item {Proveniência:(Do gr. \textunderscore hupo\textunderscore  + \textunderscore stoma\textunderscore )}
\end{itemize}
Parte da cabeça dos insectos.
\section{Hipóstroma}
\begin{itemize}
\item {Grp. gram.:m.}
\end{itemize}
\begin{itemize}
\item {Utilização:Bot.}
\end{itemize}
\begin{itemize}
\item {Proveniência:(Do gr. \textunderscore hupo\textunderscore  + \textunderscore stroma\textunderscore )}
\end{itemize}
Base, em que se apoiam os pedúnculos que sustêm os corpúsculos reproductores de certas plantas criptogâmicas.
\section{Hipotalássico}
\begin{itemize}
\item {Grp. gram.:adj.}
\end{itemize}
\begin{itemize}
\item {Proveniência:(Do gr. \textunderscore hupo\textunderscore  + \textunderscore thalassa\textunderscore )}
\end{itemize}
Que se realiza debaixo da água do mar; submarino.
\section{Hipotalástico}
\begin{itemize}
\item {Grp. gram.:adj.}
\end{itemize}
(V.hipotalássico)
\section{Hipoteca}
\begin{itemize}
\item {Grp. gram.:f.}
\end{itemize}
\begin{itemize}
\item {Proveniência:(Lat. \textunderscore hypotheca\textunderscore )}
\end{itemize}
Sujeição de bens imóveis ao pagamento de uma dívida.
Direito ou privilégio, que certos credores têm, de ser pagos pelo valor de certos bens imóveis do devedor, de preferência a outros credores, contanto que os seus créditos estejam devidamente registados.
\section{Hipotecar}
\begin{itemize}
\item {Grp. gram.:v. t.}
\end{itemize}
Dar por hipoteca; onerar com hipoteca.
\section{Hipotecariamente}
\begin{itemize}
\item {Grp. gram.:adv.}
\end{itemize}
De modo hipotecário.
\section{Hipotecário}
\begin{itemize}
\item {Grp. gram.:adj.}
\end{itemize}
\begin{itemize}
\item {Proveniência:(Lat. \textunderscore hypothecarius\textunderscore )}
\end{itemize}
Relativo a hipoteca.
\section{Hipotenar}
\begin{itemize}
\item {Grp. gram.:m.}
\end{itemize}
\begin{itemize}
\item {Proveniência:(Do gr. \textunderscore hupo\textunderscore  + \textunderscore thenar\textunderscore )}
\end{itemize}
Saliência da palma da mão, na direcção do dedo minimo.
\section{Hipotensão}
\begin{itemize}
\item {Grp. gram.:f.}
\end{itemize}
\begin{itemize}
\item {Utilização:Med.}
\end{itemize}
\begin{itemize}
\item {Proveniência:(De \textunderscore hipo...\textunderscore  + \textunderscore tensão\textunderscore )}
\end{itemize}
Deficiência de fluido nervoso, ou falta de ubiquidade de tensão no sistema nervoso. Cf. Sousa Martins, \textunderscore Nosographia\textunderscore .
\section{Hipotenusa}
\begin{itemize}
\item {Grp. gram.:f.}
\end{itemize}
\begin{itemize}
\item {Proveniência:(Lat. \textunderscore hypotenusa\textunderscore )}
\end{itemize}
Lado oposto ao ângulo recto, num triângulo rectângulo.
\section{Hipotermal}
\begin{itemize}
\item {Grp. gram.:adj.}
\end{itemize}
\begin{itemize}
\item {Proveniência:(De \textunderscore hipo...\textunderscore  + \textunderscore termal\textunderscore )}
\end{itemize}
Menos que medianamente termal.
\section{Hipótese}
\begin{itemize}
\item {Grp. gram.:f.}
\end{itemize}
\begin{itemize}
\item {Proveniência:(Lat. \textunderscore hypothesis\textunderscore )}
\end{itemize}
Suposição de coisas possíveis ou impossíveis, da qual se tira uma conclusão.
Teoria não demonstrada, mas provável; suposição.
\section{Hipoteticamente}
\begin{itemize}
\item {Grp. gram.:adv.}
\end{itemize}
De modo hipotetico; por conjectura.
\section{Hipotético}
\begin{itemize}
\item {Grp. gram.:adj.}
\end{itemize}
\begin{itemize}
\item {Proveniência:(Lat. \textunderscore hypotheticus\textunderscore )}
\end{itemize}
Relativo a hipótese: \textunderscore etimologia hipotética\textunderscore .
\section{Hipotipose}
\begin{itemize}
\item {Grp. gram.:f.}
\end{itemize}
\begin{itemize}
\item {Proveniência:(Lat. \textunderscore hypotyposis\textunderscore )}
\end{itemize}
Descripção viva e animada de uma acção ou de um objecto. Cf. Latino, \textunderscore Humboldt\textunderscore , 488.
\section{Hipotonia}
\begin{itemize}
\item {Grp. gram.:f.}
\end{itemize}
\begin{itemize}
\item {Proveniência:(De \textunderscore hipo...\textunderscore  e \textunderscore tom\textunderscore )}
\end{itemize}
Deminuição de tensão no ôlho ou em qualquer órgão.
\section{Hipotónico}
\begin{itemize}
\item {Grp. gram.:adj.}
\end{itemize}
Relativo á hipotonia.
\section{Hipotraquélio}
\begin{itemize}
\item {Grp. gram.:m.}
\end{itemize}
\begin{itemize}
\item {Proveniência:(Lat. \textunderscore hypotrachelium\textunderscore )}
\end{itemize}
Parte superior do fuste da coluna, onde começa o capitel.
\section{Hipotrofia}
\begin{itemize}
\item {Grp. gram.:f.}
\end{itemize}
\begin{itemize}
\item {Proveniência:(Do gr. \textunderscore hupo\textunderscore  + \textunderscore trophe\textunderscore )}
\end{itemize}
Nutrição deficiente.
\section{Hipoxantina}
\begin{itemize}
\item {Grp. gram.:f.}
\end{itemize}
\begin{itemize}
\item {Proveniência:(De \textunderscore hipo...\textunderscore  + \textunderscore xantina\textunderscore )}
\end{itemize}
Substância, extraida do baço.
\section{Hipoxídeas}
\begin{itemize}
\item {fónica:csi}
\end{itemize}
\begin{itemize}
\item {Grp. gram.:f. pl.}
\end{itemize}
Ordem de plantas, que tem por tipo a hipóxis.
\section{Hipóxido}
\begin{itemize}
\item {fónica:csi}
\end{itemize}
\begin{itemize}
\item {Grp. gram.:m.}
\end{itemize}
\begin{itemize}
\item {Proveniência:(De \textunderscore hipo...\textunderscore  + \textunderscore òxido\textunderscore )}
\end{itemize}
Óxido da mais baixa graduação.
\section{Hipoxíleas}
\begin{itemize}
\item {fónica:csi}
\end{itemize}
\begin{itemize}
\item {Grp. gram.:f. pl.}
\end{itemize}
\begin{itemize}
\item {Proveniência:(Do gr. \textunderscore hupo\textunderscore  + \textunderscore xule\textunderscore )}
\end{itemize}
Família de plantas acotiledóneas.
\section{Hipóxis}
\begin{itemize}
\item {fónica:csis}
\end{itemize}
\begin{itemize}
\item {Grp. gram.:f.}
\end{itemize}
\begin{itemize}
\item {Proveniência:(Do gr. \textunderscore hupo\textunderscore  + \textunderscore oxus\textunderscore )}
\end{itemize}
Gênero de plantas vivazes, de raíz tuberosa ou fibrosa.
\section{Hipozoico}
\begin{itemize}
\item {Grp. gram.:adj.}
\end{itemize}
\begin{itemize}
\item {Utilização:Geol.}
\end{itemize}
\begin{itemize}
\item {Proveniência:(Do gr. \textunderscore hupo\textunderscore  + \textunderscore zoon\textunderscore )}
\end{itemize}
Diz-se do terreno inferior àqueles em que se acham vestígios de corpos organizados.
\section{Hipsocefalia}
\begin{itemize}
\item {Grp. gram.:f.}
\end{itemize}
Qualidade ou estado de hipsocéfalo.
\section{Hipsocéfalo}
\begin{itemize}
\item {Grp. gram.:adj.}
\end{itemize}
\begin{itemize}
\item {Proveniência:(Do gr. \textunderscore hupsos\textunderscore  + \textunderscore kephale\textunderscore )}
\end{itemize}
Que tem cabeça alta.
\section{Hipsografia}
\begin{itemize}
\item {Grp. gram.:f.}
\end{itemize}
\begin{itemize}
\item {Proveniência:(Do gr. \textunderscore hupsos\textunderscore  + \textunderscore graphein\textunderscore )}
\end{itemize}
Descripção dos lugares elevados.
\section{Hissopada}
\begin{itemize}
\item {Grp. gram.:f.}
\end{itemize}
Acto de hissopar:«\textunderscore de repente prega-lhe a hissopada\textunderscore ». Garrett.
\section{Hissopar}
\begin{itemize}
\item {Grp. gram.:v. t.}
\end{itemize}
Aspergir água benta com o hissope.
\section{Hissope}
\begin{itemize}
\item {Grp. gram.:m.}
\end{itemize}
Instrumento de madeira ou metal, com que se asperge água benta.
(Cp. \textunderscore hissopo\textunderscore )
\section{Hissopina}
\begin{itemize}
\item {Grp. gram.:f.}
\end{itemize}
\begin{itemize}
\item {Utilização:Chím.}
\end{itemize}
Substância extraida do hissopo.
\section{Hissopo}
\begin{itemize}
\item {fónica:sô}
\end{itemize}
\begin{itemize}
\item {Grp. gram.:m.}
\end{itemize}
\begin{itemize}
\item {Proveniência:(Lat. \textunderscore hyssopus\textunderscore )}
\end{itemize}
Planta medicinal da fam. das labiadas.
\section{Histeralgia}
\begin{itemize}
\item {Grp. gram.:f.}
\end{itemize}
\begin{itemize}
\item {Proveniência:(Do gr. \textunderscore hustera\textunderscore  + \textunderscore algos\textunderscore )}
\end{itemize}
Dôr aguda no útero.
\section{Histerandria}
\begin{itemize}
\item {Grp. gram.:f.}
\end{itemize}
\begin{itemize}
\item {Proveniência:(Do gr. \textunderscore hustera\textunderscore  + \textunderscore aner\textunderscore )}
\end{itemize}
Classe de plantas, que têm mais de vinte estames insertos num ovário inferior.
\section{Histerândrico}
\begin{itemize}
\item {Grp. gram.:adj.}
\end{itemize}
Relativo á histerandria.
\section{Histeranto}
\begin{itemize}
\item {Grp. gram.:adj.}
\end{itemize}
\begin{itemize}
\item {Utilização:Bot.}
\end{itemize}
\begin{itemize}
\item {Proveniência:(Do gr. \textunderscore hustera\textunderscore  + \textunderscore anthos\textunderscore )}
\end{itemize}
Diz-se das plantas, cujas flôres apparecem depois das fôlhas.
\section{Histerectomia}
\begin{itemize}
\item {Grp. gram.:f.}
\end{itemize}
\begin{itemize}
\item {Utilização:Med.}
\end{itemize}
\begin{itemize}
\item {Utilização:Cir.}
\end{itemize}
\begin{itemize}
\item {Proveniência:(Do gr. \textunderscore hustera\textunderscore  + \textunderscore ektome\textunderscore )}
\end{itemize}
Ablação do útero ou de parte dele.
\section{Histeria}
\begin{itemize}
\item {Grp. gram.:f.}
\end{itemize}
\begin{itemize}
\item {Proveniência:(Do gr. \textunderscore hustera\textunderscore )}
\end{itemize}
Doença nervosa, caracterizada principalmente por convulsões e pela sensação de uma bola que subisse do útero á garganta.
Índole caprichosa ou desequilibrada.
\section{Histérica}
\begin{itemize}
\item {Grp. gram.:f.}
\end{itemize}
\begin{itemize}
\item {Utilização:Fig.}
\end{itemize}
\begin{itemize}
\item {Proveniência:(Lat. \textunderscore hystérica\textunderscore )}
\end{itemize}
Mulher, que padece histerismo.
Mulher caprichosa ou desequilibrada.
\section{Histericismo}
\begin{itemize}
\item {Grp. gram.:m.}
\end{itemize}
\begin{itemize}
\item {Proveniência:(De \textunderscore histérico\textunderscore )}
\end{itemize}
O mesmo que \textunderscore histerismo\textunderscore .
\section{Histérico}
\begin{itemize}
\item {Grp. gram.:adj.}
\end{itemize}
\begin{itemize}
\item {Grp. gram.:M.}
\end{itemize}
Relativo á histeria.
Que tem histeria.
Aquele que sofre histeria.
\section{Histerismo}
\begin{itemize}
\item {Grp. gram.:m.}
\end{itemize}
Estado de quem padece histeria.
O mesmo que \textunderscore histeria\textunderscore .
\section{Histerocele}
\begin{itemize}
\item {Grp. gram.:m.}
\end{itemize}
\begin{itemize}
\item {Proveniência:(Do gr. \textunderscore hustera\textunderscore  + \textunderscore kele\textunderscore )}
\end{itemize}
Hêrnia do útero.
\section{Histerófisa}
\begin{itemize}
\item {Grp. gram.:f.}
\end{itemize}
\begin{itemize}
\item {Utilização:Med.}
\end{itemize}
\begin{itemize}
\item {Proveniência:(Do gr. \textunderscore hustera\textunderscore  + \textunderscore phusa\textunderscore )}
\end{itemize}
Distensão do útero, produzida por gases.
\section{Histerografia}
\begin{itemize}
\item {Grp. gram.:f.}
\end{itemize}
\begin{itemize}
\item {Proveniência:(Do gr. \textunderscore hustera\textunderscore  + \textunderscore graphein\textunderscore )}
\end{itemize}
Descripção do útero.
\section{Histerólito}
\begin{itemize}
\item {Grp. gram.:m.}
\end{itemize}
\begin{itemize}
\item {Proveniência:(Do gr. \textunderscore hustera\textunderscore  + \textunderscore lithos\textunderscore )}
\end{itemize}
Concreção calcária, formada nas paredes do útero.
\section{Histerologia}
\begin{itemize}
\item {Grp. gram.:f.}
\end{itemize}
\begin{itemize}
\item {Proveniência:(Lat. \textunderscore histerologia\textunderscore )}
\end{itemize}
Defeito do escritor ou do orador, que se refere primeiro àquilo que devia têr lugar depois.
\section{Histerólogo}
\begin{itemize}
\item {Grp. gram.:m.}
\end{itemize}
\begin{itemize}
\item {Utilização:Des.}
\end{itemize}
Aquele que fala confusamente, sem nexo.
(Cp. \textunderscore histerologia\textunderscore )
\section{Histeroloxia}
\begin{itemize}
\item {fónica:csi}
\end{itemize}
\begin{itemize}
\item {Grp. gram.:f.}
\end{itemize}
\begin{itemize}
\item {Utilização:Med.}
\end{itemize}
\begin{itemize}
\item {Proveniência:(Do gr. \textunderscore hustera\textunderscore  + \textunderscore oxos\textunderscore )}
\end{itemize}
Obliquidade do útero; desvio, a que êste órgão está sujeito durante a gravidez.
\section{Histeromalacia}
\begin{itemize}
\item {Grp. gram.:f.}
\end{itemize}
\begin{itemize}
\item {Utilização:Med.}
\end{itemize}
\begin{itemize}
\item {Proveniência:(Do gr. \textunderscore hustera\textunderscore  + \textunderscore malakia\textunderscore )}
\end{itemize}
Amolecimento dos tecidos do útero, o que póde contribuir para que ele se rompa na ocasião do parto.
\section{Histeromania}
\begin{itemize}
\item {Grp. gram.:f.}
\end{itemize}
\begin{itemize}
\item {Proveniência:(Do gr. \textunderscore hustera\textunderscore  + \textunderscore mania\textunderscore )}
\end{itemize}
Furor uterino.
Ninfomania.
\section{Histerómetro}
\begin{itemize}
\item {Grp. gram.:m.}
\end{itemize}
\begin{itemize}
\item {Proveniência:(Do gr. \textunderscore hustera\textunderscore  + \textunderscore metron\textunderscore )}
\end{itemize}
Sonda uterina, ou instrumento para sondar o útero e trazê-lo á direcção normal, em caso de desvio.
\section{Histeroptose}
\begin{itemize}
\item {Grp. gram.:f.}
\end{itemize}
\begin{itemize}
\item {Proveniência:(Do gr. \textunderscore hustera\textunderscore  + \textunderscore ptosis\textunderscore )}
\end{itemize}
Quéda ou reviramento do útero.
\section{Histerorreia}
\begin{itemize}
\item {Grp. gram.:f.}
\end{itemize}
\begin{itemize}
\item {Proveniência:(Do gr. \textunderscore hustera\textunderscore  + \textunderscore rhein\textunderscore )}
\end{itemize}
(V.leucorreia)
\section{Histeroscópio}
\begin{itemize}
\item {Grp. gram.:m.}
\end{itemize}
\begin{itemize}
\item {Proveniência:(Do gr. \textunderscore hustera\textunderscore  + \textunderscore skopein\textunderscore )}
\end{itemize}
O mesmo que \textunderscore espéculo\textunderscore .
\section{Histerostomátomo}
\begin{itemize}
\item {Grp. gram.:m.}
\end{itemize}
\begin{itemize}
\item {Proveniência:(Do gr. \textunderscore hustera\textunderscore  + \textunderscore stoma\textunderscore  + \textunderscore tome\textunderscore )}
\end{itemize}
Instrumento para fender o colo do útero, quando êste dificulta o parto.
\section{Histerotocotomia}
\begin{itemize}
\item {Grp. gram.:f.}
\end{itemize}
\begin{itemize}
\item {Utilização:Des.}
\end{itemize}
\begin{itemize}
\item {Proveniência:(Do gr. \textunderscore hustera\textunderscore  + \textunderscore tokos\textunderscore  + \textunderscore tome\textunderscore )}
\end{itemize}
A operação cesariana.
\section{Histerotomia}
\begin{itemize}
\item {Grp. gram.:f.}
\end{itemize}
\begin{itemize}
\item {Proveniência:(De \textunderscore histerótomo\textunderscore )}
\end{itemize}
Dissecção do útero.
\section{Histerótomo}
\begin{itemize}
\item {Grp. gram.:m.}
\end{itemize}
\begin{itemize}
\item {Proveniência:(Do gr. \textunderscore hustera\textunderscore  + \textunderscore tome\textunderscore )}
\end{itemize}
Instrumento, com que se pratíca a histerotomia.
\section{Histrícios}
\begin{itemize}
\item {Grp. gram.:m. pl.}
\end{itemize}
\begin{itemize}
\item {Proveniência:(Do lat. \textunderscore hystrix\textunderscore )}
\end{itemize}
Família de mamíferos roedores, que tem por tipo o porco-espinho.
\section{Hypopýgio}
\begin{itemize}
\item {Grp. gram.:m.}
\end{itemize}
\begin{itemize}
\item {Proveniência:(Do gr. \textunderscore hupo\textunderscore  + \textunderscore puge\textunderscore )}
\end{itemize}
Último segmento neutral do abdome dos insectos.
\section{Hyporrhytmo}
\begin{itemize}
\item {Grp. gram.:m.}
\end{itemize}
\begin{itemize}
\item {Proveniência:(Do gr. \textunderscore hupo\textunderscore  + \textunderscore rhuthmos\textunderscore )}
\end{itemize}
Hexâmetro sem cesura.
\section{Hyposcênio}
\begin{itemize}
\item {Grp. gram.:m.}
\end{itemize}
\begin{itemize}
\item {Proveniência:(De \textunderscore hipo...\textunderscore  + \textunderscore scena\textunderscore )}
\end{itemize}
A parte inferior da scena, nos theatros gregos.
Parede ou taipa, que supporta a frente do palco.
Lugar, occupado pelos músicos, junto dessa taipa.
\section{Hyposmia}
\begin{itemize}
\item {Grp. gram.:f.}
\end{itemize}
\begin{itemize}
\item {Utilização:Med.}
\end{itemize}
\begin{itemize}
\item {Proveniência:(Do gr. \textunderscore hupo\textunderscore  + \textunderscore osme\textunderscore )}
\end{itemize}
Enfraquecimento do olfato.
\section{Hypospadia}
\begin{itemize}
\item {Grp. gram.:f.}
\end{itemize}
\begin{itemize}
\item {Proveniência:(Do gr. \textunderscore hupo\textunderscore  + \textunderscore spao\textunderscore )}
\end{itemize}
Deformidade dos órgãos genitaes do homem, que consiste em que a uretra se abre por baixo do pênis, em qualquer ponto.
\section{Hypospado}
\begin{itemize}
\item {Grp. gram.:m.}
\end{itemize}
Aquelle que tem a hipospadia.
\section{Hypospathismo}
\begin{itemize}
\item {Grp. gram.:m.}
\end{itemize}
\begin{itemize}
\item {Proveniência:(Gr. \textunderscore hupospathismos\textunderscore )}
\end{itemize}
Desusada operação cirúrgica, em que, para os casos de ophtalmia chrónica, se faziam três incisões na fronte, passando-se uma espátula entre as carnes e o pericrânio, para êste ficar descoberto em certa extensão.
\section{Hyposphagma}
\begin{itemize}
\item {Grp. gram.:m.}
\end{itemize}
\begin{itemize}
\item {Proveniência:(Gr. \textunderscore huposphagma\textunderscore )}
\end{itemize}
Ecchymose no ôlho.
\section{Hypostaminado}
\begin{itemize}
\item {Grp. gram.:adj.}
\end{itemize}
\begin{itemize}
\item {Utilização:Bot.}
\end{itemize}
\begin{itemize}
\item {Proveniência:(De \textunderscore hypo...\textunderscore  + \textunderscore estame\textunderscore )}
\end{itemize}
Que tem os estames insertos no ovário.
\section{Hypostaminia}
\begin{itemize}
\item {Grp. gram.:f.}
\end{itemize}
Estado de uma planta, que tem estames hypógynos.
(Cp. \textunderscore hypostaminado\textunderscore )
\section{Hypóstase}
\begin{itemize}
\item {Grp. gram.:f.}
\end{itemize}
\begin{itemize}
\item {Proveniência:(Lat. \textunderscore hypostasis\textunderscore )}
\end{itemize}
União do Verbo com a natureza divina, como substância única, (em Theologia).
Sedimento da urina.
Sarro.
\section{Hypostaticamente}
\begin{itemize}
\item {Grp. gram.:adv.}
\end{itemize}
De modo hypostático.
\section{Hypostático}
\begin{itemize}
\item {Grp. gram.:adj.}
\end{itemize}
\begin{itemize}
\item {Proveniência:(Gr. \textunderscore hupostatikos\textunderscore )}
\end{itemize}
Relativo á hypóstase: \textunderscore união hypostática\textunderscore .
\section{Hyposternal}
\begin{itemize}
\item {Grp. gram.:m.}
\end{itemize}
\begin{itemize}
\item {Proveniência:(De \textunderscore hypo...\textunderscore  + \textunderscore esterno\textunderscore )}
\end{itemize}
Peça do esterno das tartarugas.
\section{Hyposthenia}
\begin{itemize}
\item {Grp. gram.:f.}
\end{itemize}
\begin{itemize}
\item {Utilização:Med.}
\end{itemize}
\begin{itemize}
\item {Proveniência:(Do gr. \textunderscore hupo\textunderscore  + \textunderscore sthenos\textunderscore )}
\end{itemize}
Deminuição de fôrças.
\section{Hyposthênico}
\begin{itemize}
\item {Grp. gram.:adj.}
\end{itemize}
Relativo á hyposthenia.
\section{Hypóstoma}
\begin{itemize}
\item {Grp. gram.:m.}
\end{itemize}
\begin{itemize}
\item {Proveniência:(Do gr. \textunderscore hupo\textunderscore  + \textunderscore stoma\textunderscore )}
\end{itemize}
Parte da cabeça dos insectos.
\section{Hypóstroma}
\begin{itemize}
\item {Grp. gram.:m.}
\end{itemize}
\begin{itemize}
\item {Utilização:Bot.}
\end{itemize}
\begin{itemize}
\item {Proveniência:(Do gr. \textunderscore hupo\textunderscore  + \textunderscore stroma\textunderscore )}
\end{itemize}
Base, em que se apoiam os pedúnculos que sustêm os corpúsculos reproductores de certas plantas cryptogâmicas.
\section{Hypostylo}
\begin{itemize}
\item {Grp. gram.:adj.}
\end{itemize}
\begin{itemize}
\item {Proveniência:(Do gr. \textunderscore hupo\textunderscore  + \textunderscore stulos\textunderscore )}
\end{itemize}
Dizia-se das salas ou compartimentos, cujo tecto é sustentado por columnas.
\section{Hyposulfato}
\begin{itemize}
\item {fónica:sul}
\end{itemize}
\begin{itemize}
\item {Grp. gram.:m.}
\end{itemize}
\begin{itemize}
\item {Proveniência:(De \textunderscore hypo...\textunderscore  + \textunderscore sulfato\textunderscore )}
\end{itemize}
Sal, produzido pela combinação do ácido hyposulfúrico com uma base.
\section{Hyposulfito}
\begin{itemize}
\item {fónica:sul}
\end{itemize}
\begin{itemize}
\item {Grp. gram.:m.}
\end{itemize}
\begin{itemize}
\item {Utilização:Chím.}
\end{itemize}
\begin{itemize}
\item {Proveniência:(De \textunderscore hypo\textunderscore  + \textunderscore sulfito\textunderscore )}
\end{itemize}
Sal, resultante da combinação do ácido hyposulfuroso com uma base.
\section{Hyposulfúrico}
\begin{itemize}
\item {fónica:sul}
\end{itemize}
\begin{itemize}
\item {Grp. gram.:adj.}
\end{itemize}
\begin{itemize}
\item {Proveniência:(De \textunderscore hypo...\textunderscore  + \textunderscore sulfúrico\textunderscore )}
\end{itemize}
Diz-se do terceiro dos oxácidos de enxôfre.
\section{Hyposulfuroso}
\begin{itemize}
\item {fónica:sul}
\end{itemize}
\begin{itemize}
\item {Grp. gram.:adj.}
\end{itemize}
\begin{itemize}
\item {Proveniência:(De \textunderscore hypo...\textunderscore  + \textunderscore sulfuroso\textunderscore )}
\end{itemize}
Diz-se do primeiro dos oxácidos do enxôfre.
\section{Hyposystolia}
\begin{itemize}
\item {fónica:sis}
\end{itemize}
\begin{itemize}
\item {Grp. gram.:f.}
\end{itemize}
\begin{itemize}
\item {Utilização:Med.}
\end{itemize}
\begin{itemize}
\item {Proveniência:(Do gr. \textunderscore hupo\textunderscore  + \textunderscore sustole\textunderscore )}
\end{itemize}
Fraqueza cardiaca.
\section{Hypotensão}
\begin{itemize}
\item {Grp. gram.:f.}
\end{itemize}
\begin{itemize}
\item {Utilização:Med.}
\end{itemize}
\begin{itemize}
\item {Proveniência:(De \textunderscore hypo...\textunderscore  + \textunderscore tensão\textunderscore )}
\end{itemize}
Deficiência de fluido nervoso, ou falta de ubiquidade de tensão no systema nervoso. Cf. Sousa Martins, \textunderscore Nosographia\textunderscore .
\section{Hypotenusa}
\begin{itemize}
\item {Grp. gram.:f.}
\end{itemize}
\begin{itemize}
\item {Proveniência:(Lat. \textunderscore hypotenusa\textunderscore )}
\end{itemize}
Lado opposto ao ângulo recto, num triângulo rectângulo.
\section{Hypothalássico}
\begin{itemize}
\item {Grp. gram.:adj.}
\end{itemize}
\begin{itemize}
\item {Proveniência:(Do gr. \textunderscore hupo\textunderscore  + \textunderscore thalassa\textunderscore )}
\end{itemize}
Que se realiza debaixo da água do mar; submarino.
\section{Hypothalástico}
\begin{itemize}
\item {Grp. gram.:adj.}
\end{itemize}
(V.hypothalássico)
\section{Hypotheca}
\begin{itemize}
\item {Grp. gram.:f.}
\end{itemize}
\begin{itemize}
\item {Proveniência:(Lat. \textunderscore hypotheca\textunderscore )}
\end{itemize}
Sujeição de bens immóveis ao pagamento de uma dívida.
Direito ou privilégio, que certos credores têm, de ser pagos pelo valor de certos bens immóveis do devedor, de preferência a outros credores, contanto que os seus créditos estejam devidamente registados.
\section{Hypothecar}
\begin{itemize}
\item {Grp. gram.:v. t.}
\end{itemize}
Dar por hypotheca; onerar com hypotheca.
\section{Hypothecariamente}
\begin{itemize}
\item {Grp. gram.:adv.}
\end{itemize}
De modo hypothecário.
\section{Hypothecário}
\begin{itemize}
\item {Grp. gram.:adj.}
\end{itemize}
\begin{itemize}
\item {Proveniência:(Lat. \textunderscore hypothecarius\textunderscore )}
\end{itemize}
Relativo a hypotheca.
\section{Hypothenar}
\begin{itemize}
\item {Grp. gram.:m.}
\end{itemize}
\begin{itemize}
\item {Proveniência:(Do gr. \textunderscore hupo\textunderscore  + \textunderscore thenar\textunderscore )}
\end{itemize}
Saliência da palma da mão, na direcção do dedo minimo.
\section{Hypothermal}
\begin{itemize}
\item {Grp. gram.:adj.}
\end{itemize}
\begin{itemize}
\item {Proveniência:(De \textunderscore hypo...\textunderscore  + \textunderscore thermal\textunderscore )}
\end{itemize}
Menos que medianamente thermal.
\section{Hypóthese}
\begin{itemize}
\item {Grp. gram.:f.}
\end{itemize}
\begin{itemize}
\item {Proveniência:(Lat. \textunderscore hypothesis\textunderscore )}
\end{itemize}
Supposição de coisas possíveis ou impossíveis, da qual se tira uma conclusão.
Theoria não demonstrada, mas provável; supposição.
\section{Hypotheticamente}
\begin{itemize}
\item {Grp. gram.:adv.}
\end{itemize}
De modo hypothetico; por conjectura.
\section{Hypothético}
\begin{itemize}
\item {Grp. gram.:adj.}
\end{itemize}
\begin{itemize}
\item {Proveniência:(Lat. \textunderscore hypotheticus\textunderscore )}
\end{itemize}
Relativo a hypóthese: \textunderscore etymologia hypothética\textunderscore .
\section{Hypotonia}
\begin{itemize}
\item {Grp. gram.:f.}
\end{itemize}
\begin{itemize}
\item {Proveniência:(De \textunderscore hypo...\textunderscore  e \textunderscore tom\textunderscore )}
\end{itemize}
Deminuição de tensão no ôlho ou em qualquer órgão.
\section{Hypotónico}
\begin{itemize}
\item {Grp. gram.:adj.}
\end{itemize}
Relativo á hypotonia.
\section{Hypotrachélio}
\begin{itemize}
\item {fónica:que}
\end{itemize}
\begin{itemize}
\item {Grp. gram.:m.}
\end{itemize}
\begin{itemize}
\item {Proveniência:(Lat. \textunderscore hypotrachelium\textunderscore )}
\end{itemize}
Parte superior do fuste da columna, onde começa o capitel.
\section{Hypotrophia}
\begin{itemize}
\item {Grp. gram.:f.}
\end{itemize}
\begin{itemize}
\item {Proveniência:(Do gr. \textunderscore hupo\textunderscore  + \textunderscore trophe\textunderscore )}
\end{itemize}
Nutrição deficiente.
\section{Hypotypose}
\begin{itemize}
\item {Grp. gram.:f.}
\end{itemize}
\begin{itemize}
\item {Proveniência:(Lat. \textunderscore hypotyposis\textunderscore )}
\end{itemize}
Descripção viva e animada de uma acção ou de um objecto. Cf. Latino, \textunderscore Humboldt\textunderscore , 488.
\section{Hypoxanthina}
\begin{itemize}
\item {Grp. gram.:f.}
\end{itemize}
\begin{itemize}
\item {Proveniência:(De \textunderscore hypo...\textunderscore  + \textunderscore xanthina\textunderscore )}
\end{itemize}
Substância, extrahida do baço.
\section{Hypoxýdeas}
\begin{itemize}
\item {fónica:csi}
\end{itemize}
\begin{itemize}
\item {Grp. gram.:f. pl.}
\end{itemize}
Ordem de plantas, que tem por typo a hypóxis.
\section{Hypóxydo}
\begin{itemize}
\item {fónica:csi}
\end{itemize}
\begin{itemize}
\item {Grp. gram.:m.}
\end{itemize}
\begin{itemize}
\item {Proveniência:(De \textunderscore hypo...\textunderscore  + \textunderscore òxydo\textunderscore )}
\end{itemize}
Óxydo da mais baixa graduação.
\section{Hypoxýleas}
\begin{itemize}
\item {fónica:csi}
\end{itemize}
\begin{itemize}
\item {Grp. gram.:f. pl.}
\end{itemize}
\begin{itemize}
\item {Proveniência:(Do gr. \textunderscore hupo\textunderscore  + \textunderscore xule\textunderscore )}
\end{itemize}
Família de plantas acotyledóneas.
\section{Hypóxys}
\begin{itemize}
\item {fónica:csis}
\end{itemize}
\begin{itemize}
\item {Grp. gram.:f.}
\end{itemize}
\begin{itemize}
\item {Proveniência:(Do gr. \textunderscore hupo\textunderscore  + \textunderscore oxus\textunderscore )}
\end{itemize}
Gênero de plantas vivazes, de raíz tuberosa ou fibrosa.
\section{Hypozoico}
\begin{itemize}
\item {Grp. gram.:adj.}
\end{itemize}
\begin{itemize}
\item {Utilização:Geol.}
\end{itemize}
\begin{itemize}
\item {Proveniência:(Do gr. \textunderscore hupo\textunderscore  + \textunderscore zoon\textunderscore )}
\end{itemize}
Diz-se do terreno inferior àquelles em que se acham vestígios de corpos organizados.
\section{Hypsocephalia}
\begin{itemize}
\item {Grp. gram.:f.}
\end{itemize}
Qualidade ou estado de hypsocéphalo.
\section{Hypsocéphalo}
\begin{itemize}
\item {Grp. gram.:adj.}
\end{itemize}
\begin{itemize}
\item {Proveniência:(Do gr. \textunderscore hupsos\textunderscore  + \textunderscore kephale\textunderscore )}
\end{itemize}
Que tem cabeça alta.
\section{Hypsographia}
\begin{itemize}
\item {Grp. gram.:f.}
\end{itemize}
\begin{itemize}
\item {Proveniência:(Do gr. \textunderscore hupsos\textunderscore  + \textunderscore graphein\textunderscore )}
\end{itemize}
Descripção dos lugares elevados.
\section{Hypsometria}
\begin{itemize}
\item {Grp. gram.:f.}
\end{itemize}
Arte de medir a altura de um lugar por nivelamentos, ou por observações barométricas, ou por operações geodésicas.
(Cp. \textunderscore hypsómetro\textunderscore )
\section{Hypsométrico}
\begin{itemize}
\item {Grp. gram.:adj.}
\end{itemize}
Relativo a hypsometria.
\section{Hypsómetro}
\begin{itemize}
\item {Grp. gram.:m.}
\end{itemize}
\begin{itemize}
\item {Proveniência:(Do gr. \textunderscore hupsos\textunderscore  + \textunderscore metron\textunderscore )}
\end{itemize}
Instrumento de Phýsica, para medir a altura de um lugar, segundo a temperatura, a que a água póde começar alli em ebullição.
\section{Hyssopada}
\begin{itemize}
\item {Grp. gram.:f.}
\end{itemize}
Acto de hyssopar:«\textunderscore de repente prega-lhe a hyssopada\textunderscore ». Garrett.
\section{Hyssopar}
\begin{itemize}
\item {Grp. gram.:v. t.}
\end{itemize}
Aspergir água benta com o hyssope.
\section{Hyssope}
\begin{itemize}
\item {Grp. gram.:m.}
\end{itemize}
Instrumento de madeira ou metal, com que se asperge água benta.
(Cp. \textunderscore hyssopo\textunderscore )
\section{Hyssopina}
\begin{itemize}
\item {Grp. gram.:f.}
\end{itemize}
\begin{itemize}
\item {Utilização:Chím.}
\end{itemize}
Substância extrahida do hyssopo.
\section{Hyssopo}
\begin{itemize}
\item {fónica:sô}
\end{itemize}
\begin{itemize}
\item {Grp. gram.:m.}
\end{itemize}
\begin{itemize}
\item {Proveniência:(Lat. \textunderscore hyssopus\textunderscore )}
\end{itemize}
Planta medicinal da fam. das labiadas.
\section{Hysteralgia}
\begin{itemize}
\item {Grp. gram.:f.}
\end{itemize}
\begin{itemize}
\item {Proveniência:(Do gr. \textunderscore hustera\textunderscore  + \textunderscore algos\textunderscore )}
\end{itemize}
Dôr aguda no útero.
\section{Hysterandria}
\begin{itemize}
\item {Grp. gram.:f.}
\end{itemize}
\begin{itemize}
\item {Proveniência:(Do gr. \textunderscore hustera\textunderscore  + \textunderscore aner\textunderscore )}
\end{itemize}
Classe de plantas, que têm mais de vinte estames insertos num ovário inferior.
\section{Hysterândrico}
\begin{itemize}
\item {Grp. gram.:adj.}
\end{itemize}
Relativo á histerandria.
\section{Hysterantho}
\begin{itemize}
\item {Grp. gram.:adj.}
\end{itemize}
\begin{itemize}
\item {Utilização:Bot.}
\end{itemize}
\begin{itemize}
\item {Proveniência:(Do gr. \textunderscore hustera\textunderscore  + \textunderscore anthos\textunderscore )}
\end{itemize}
Diz-se das plantas, cujas flôres apparecem depois das fôlhas.
\section{Hysterectomia}
\begin{itemize}
\item {Grp. gram.:f.}
\end{itemize}
\begin{itemize}
\item {Utilização:Med.}
\end{itemize}
\begin{itemize}
\item {Utilização:Cir.}
\end{itemize}
\begin{itemize}
\item {Proveniência:(Do gr. \textunderscore hustera\textunderscore  + \textunderscore ektome\textunderscore )}
\end{itemize}
Ablação do útero ou de parte delle.
\section{Hysteria}
\begin{itemize}
\item {Grp. gram.:f.}
\end{itemize}
\begin{itemize}
\item {Proveniência:(Do gr. \textunderscore hustera\textunderscore )}
\end{itemize}
Doença nervosa, caracterizada principalmente por convulsões e pela sensação de uma bola que subisse do útero á garganta.
Índole caprichosa ou desequilibrada.
\section{Hystérica}
\begin{itemize}
\item {Grp. gram.:f.}
\end{itemize}
\begin{itemize}
\item {Utilização:Fig.}
\end{itemize}
\begin{itemize}
\item {Proveniência:(Lat. \textunderscore hystérica\textunderscore )}
\end{itemize}
Mulher, que padece hysterismo.
Mulher caprichosa ou desequilibrada.
\section{Hystericismo}
\begin{itemize}
\item {Grp. gram.:m.}
\end{itemize}
\begin{itemize}
\item {Proveniência:(De \textunderscore hystérico\textunderscore )}
\end{itemize}
O mesmo que \textunderscore hysterismo\textunderscore .
\section{Hystérico}
\begin{itemize}
\item {Grp. gram.:adj.}
\end{itemize}
\begin{itemize}
\item {Grp. gram.:M.}
\end{itemize}
Relativo á hysteria.
Que tem hysteria.
Aquelle que soffre hysteria.
\section{Hysterismo}
\begin{itemize}
\item {Grp. gram.:m.}
\end{itemize}
Estado de quem padece hysteria.
O mesmo que \textunderscore hysteria\textunderscore .
\section{Hystero-catalepsia}
\begin{itemize}
\item {Grp. gram.:f.}
\end{itemize}
Ataque de hysterismo, complicado com symptomas de catalepsia.
\section{Hysterocele}
\begin{itemize}
\item {Grp. gram.:m.}
\end{itemize}
\begin{itemize}
\item {Proveniência:(Do gr. \textunderscore hustera\textunderscore  + \textunderscore kele\textunderscore )}
\end{itemize}
Hêrnia do útero.
\section{Hystero-epilepsia}
\begin{itemize}
\item {Grp. gram.:f.}
\end{itemize}
Histeria, complicada de accessos epileptiformes.
\section{Hysterographia}
\begin{itemize}
\item {Grp. gram.:f.}
\end{itemize}
\begin{itemize}
\item {Proveniência:(Do gr. \textunderscore hustera\textunderscore  + \textunderscore graphein\textunderscore )}
\end{itemize}
Descripção do útero.
\section{Hysterólitho}
\begin{itemize}
\item {Grp. gram.:m.}
\end{itemize}
\begin{itemize}
\item {Proveniência:(Do gr. \textunderscore hustera\textunderscore  + \textunderscore lithos\textunderscore )}
\end{itemize}
Concreção calcária, formada nas paredes do útero.
\section{Hysterologia}
\begin{itemize}
\item {Grp. gram.:f.}
\end{itemize}
\begin{itemize}
\item {Proveniência:(Lat. \textunderscore histerologia\textunderscore )}
\end{itemize}
Defeito do escritor ou do orador, que se refere primeiro àquillo que devia têr lugar depois.
\section{Hysterólogo}
\begin{itemize}
\item {Grp. gram.:m.}
\end{itemize}
\begin{itemize}
\item {Utilização:Des.}
\end{itemize}
Aquelle que fala confusamente, sem nexo.
(Cp. \textunderscore hysterologia\textunderscore )
\section{Hysteroloxia}
\begin{itemize}
\item {fónica:csi}
\end{itemize}
\begin{itemize}
\item {Grp. gram.:f.}
\end{itemize}
\begin{itemize}
\item {Utilização:Med.}
\end{itemize}
\begin{itemize}
\item {Proveniência:(Do gr. \textunderscore hustera\textunderscore  + \textunderscore oxos\textunderscore )}
\end{itemize}
Obliquidade do útero; desvio, a que êste órgão está sujeito durante a gravidez.
\section{Hysteromalacia}
\begin{itemize}
\item {Grp. gram.:f.}
\end{itemize}
\begin{itemize}
\item {Utilização:Med.}
\end{itemize}
\begin{itemize}
\item {Proveniência:(Do gr. \textunderscore hustera\textunderscore  + \textunderscore malakia\textunderscore )}
\end{itemize}
Amollecimento dos tecidos do útero, o que póde contribuir para que elle se rompa na occasião do parto.
\section{Hysteromania}
\begin{itemize}
\item {Grp. gram.:f.}
\end{itemize}
\begin{itemize}
\item {Proveniência:(Do gr. \textunderscore hustera\textunderscore  + \textunderscore mania\textunderscore )}
\end{itemize}
Furor uterino.
Nymphomania.
\section{Hysterómetro}
\begin{itemize}
\item {Grp. gram.:m.}
\end{itemize}
\begin{itemize}
\item {Proveniência:(Do gr. \textunderscore hustera\textunderscore  + \textunderscore metron\textunderscore )}
\end{itemize}
Sonda uterina, ou instrumento para sondar o útero e trazê-lo á direcção normal, em caso de desvio.
\section{Hysteróphysa}
\begin{itemize}
\item {Grp. gram.:f.}
\end{itemize}
\begin{itemize}
\item {Utilização:Med.}
\end{itemize}
\begin{itemize}
\item {Proveniência:(Do gr. \textunderscore hustera\textunderscore  + \textunderscore phusa\textunderscore )}
\end{itemize}
Distensão do útero, produzida por gases.
\section{Hysteroptose}
\begin{itemize}
\item {Grp. gram.:f.}
\end{itemize}
\begin{itemize}
\item {Proveniência:(Do gr. \textunderscore hustera\textunderscore  + \textunderscore ptosis\textunderscore )}
\end{itemize}
Quéda ou reviramento do útero.
\section{Hysterorreia}
\begin{itemize}
\item {Grp. gram.:f.}
\end{itemize}
\begin{itemize}
\item {Proveniência:(Do gr. \textunderscore hustera\textunderscore  + \textunderscore rhein\textunderscore )}
\end{itemize}
(V.leucorreia)
\section{Hysterorrheia}
\begin{itemize}
\item {Grp. gram.:f.}
\end{itemize}
\begin{itemize}
\item {Proveniência:(Do gr. \textunderscore hustera\textunderscore  + \textunderscore rhein\textunderscore )}
\end{itemize}
(V.leucorreia)
\section{Hysteroscópio}
\begin{itemize}
\item {Grp. gram.:m.}
\end{itemize}
\begin{itemize}
\item {Proveniência:(Do gr. \textunderscore hustera\textunderscore  + \textunderscore skopein\textunderscore )}
\end{itemize}
O mesmo que \textunderscore espéculo\textunderscore .
\section{Hysterostomátomo}
\begin{itemize}
\item {Grp. gram.:m.}
\end{itemize}
\begin{itemize}
\item {Proveniência:(Do gr. \textunderscore hustera\textunderscore  + \textunderscore stoma\textunderscore  + \textunderscore tome\textunderscore )}
\end{itemize}
Instrumento para fender o collo do útero, quando êste difficulta o parto.
\section{Hysterotocotomia}
\begin{itemize}
\item {Grp. gram.:f.}
\end{itemize}
\begin{itemize}
\item {Utilização:Des.}
\end{itemize}
\begin{itemize}
\item {Proveniência:(Do gr. \textunderscore hustera\textunderscore  + \textunderscore tokos\textunderscore  + \textunderscore tome\textunderscore )}
\end{itemize}
A operação cesariana.
\section{Hysterotomia}
\begin{itemize}
\item {Grp. gram.:f.}
\end{itemize}
\begin{itemize}
\item {Proveniência:(De \textunderscore hysterótomo\textunderscore )}
\end{itemize}
Dissecção do útero.
\section{Hysterótomo}
\begin{itemize}
\item {Grp. gram.:m.}
\end{itemize}
\begin{itemize}
\item {Proveniência:(Do gr. \textunderscore hustera\textunderscore  + \textunderscore tome\textunderscore )}
\end{itemize}
Instrumento, com que se pratíca a hysterotomia.
\section{Hystrícios}
\begin{itemize}
\item {Grp. gram.:m. pl.}
\end{itemize}
\begin{itemize}
\item {Proveniência:(Do lat. \textunderscore hystrix\textunderscore )}
\end{itemize}
\end{document}